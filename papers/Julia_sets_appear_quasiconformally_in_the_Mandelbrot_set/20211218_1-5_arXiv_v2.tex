\section{Introduction}

Let $P_c(z) := z^2 +c \ (c \in \C)$ and recall that its 
{\it filled Julia set} $K_c$ is defined by
$$
  K_c := \{ z \in \C \ | \ \{ P_c^n(z) \}_{n=0}^\infty \ \text{is bounded} \}
$$
and its {\it Julia set} $J_c$ is the boundary of $K_c$, that is, 
$J_c := \partial K_c$. It is known that $J_c$ is connected if and
only if the critical orbit $\{ P_c^n(0) \}_{n=0}^\infty$ is bounded
and if $J_c$ is disconnected, it is a Cantor set. The connectedness locus
of the quadratic family $\{ P_c \}_{c \in \mathbb C}$ is the famous 
{\it Mandelbrot set} and we denote it by $M$:
$$
M := \{ c \in \C \ | \ J_c \  \text{is connected} \} 
= \{ c \in \C \ | \ \{ P_c^n(0) \}_{n=0}^\infty  \  \text{is bounded} \}.
$$
A parameter $c$ is called a {\it Misiurewicz parameter} if the 
critical point $0$ is strictly preperiodic, that is,  
$$
  P_c^k(P_c^l(0)) = P_c^l(0) \quad \text{and} \quad 
  P_c^k(P_c^{l-1}(0)) \ne P_c^{l-1}(0)
$$
for some $k, \ l \in \N = \{ 1,2,3,\cdots \}$. A parameter $c$ is called a 
{\it parabolic parameter}
if $P_c$ has a parabolic periodic point. Here, a periodic point $z_0$ with 
period $m$ is called {\it parabolic} if $P_c^m(z_0) = z_0$ and its multiplier
$(P_c^m)'(z_0)$ is a root of unity. For the basic knowledge of complex dynamics,
we refer to \cite{Beardon 1991} and \cite{Milnor 2006}.


Douady et al. (\cite{Douady 2000}) proved the following: At a small neighborhood
of the cusp point $c_0 \ne 1/4$ in $M$, which is in a \lq\lq primitive
small Mandelbrot set", there is a sequence $\{ M_n \}_{n \in \N}$ of
small quasiconformal copies of $M$ tending to $c_0$. Moreover each $M_n$ is
encaged in a nested sequence of sets which are homeomorphic to the preimage
of $J_{1/4 + \eta}$ (for $\eta > 0$ small) by $z \mapsto z^{2^m}$ for 
$m \geq 0$ and accumulate on $M_n$. 

In this paper, firstly we generalize part of their results (Theorem A).
Actually this kind of phenomena can be observed not only in a small 
neighborhood of the cusp of a \lq\lq primitive small Mandelbrot set", that 
is, the point corresponding to a parabolic parameter $1/4 \in \partial M$, 
but also in every neighborhood of a point corresponding to any Misiurewicz 
or parabolic parameters $c_0$ in a small Mandelbrot set. (For example, 
$c_0=1/4 \in \partial M$ can be replaced by a Misiurewicz parameter 
$c_0=i\in \partial M$ or a parabolic parameter $c_0=-3/4 \in \partial M$ etc.) 
More precisely, we show the following: Take any small Mandelbrot set 
$M_{s_0}$ (Figure 1-(1))) and zoom in the neighborhood of 
$c_1 = s_0 \perp c_0 \in \partial M_{s_0}$ corresponding to $c_0 \in \pt M$ 
which is a Misiurewicz or a parabolic parameter (Figure 1-(2) to (6))). 
(Note that $c_1$ itself is also a Misiurewicz or a parabolic parameter.) 
Then we can find a subset $J' \subset \partial M$ which looks very similar to
$J_{c_0}$ (Figure \ref{figures of a primitive-Misiurewicz case}--(6)). 
Zoom in further, then this $J'$ turns out to be similar to $J_{c_0+\eta}$
rather than $J_{c_0}$, 
where $|\eta|$ is very small and $c_0+\eta \notin M$, 
because $J'$ looks disconnected 
(Figure \ref{figures of a primitive-Misiurewicz case}--(8), (9)). 
Furthermore, as we further zoom in the middle part of $J'$, we can see a 
nested structure which is very similar to the iterated preimages of 
$J_{c_0+\eta}$ by $z \mapsto z^2$ (we call these a {\it decoration})
(Figure \ref{figures of a primitive-Misiurewicz case}--(10), (12), (14))
and finally another smaller Mandelbrot set $M_{s_1}$ appears
(Figure \ref{figures of a primitive-Misiurewicz case}--(15)). 



Secondly we show the following result for filled Julia sets (Theorem B): 
Take a parameter $s_1 \perp c \ (c \in M)$ from the above smaller Mandelbrot
set $M_{s_1}$ and look at the filled Julia set $K_{s_1 \perp c}$ and its 
zooms around the neighborhood of $0 \in K_{s_1 \perp c}$. Then we can observe
a very similar nested structure to what we saw as zooming in the middle
part of the set $J' \subset \partial M$ 
(see Figure \ref{nested structure for a filled Julia set}). 



Thirdly we show that some of the smaller Mandelbrot sets $M_{s_1}$ and their 
decorations are images of  certain model sets by quasiconformal maps whose 
dilatations are arbitrarily close to $1$ (Theorem C). This answers the first
part of the \lq\lq Final remarks" in \cite[p.35]{Douady 2000}.

Finally we show that all 
the parameters belonging to the decorations are semihyperbolic and also the 
set of semihyperbolic but non-Misiurewicz and non-hyperbolic parameters are 
dense in the boundary
of the Mandelbrot set (Corollary D). This together with Theorem C leads to a 
direct and intuitive explanation for the fact that the Hausdorff dimension 
of $\partial M$ is equal to 2, which is a famous result by 
Shishikura (\cite{Shishikura 1998}).




According to Wolf Jung, a structure in the Mandelbrot set which resembles a
whole Julia set in appearance was observed in
computer experiments decades ago by Robert Munafo and Jonathan Leavitt. 
He also claims that he described a general explanation in his
website (\cite{Jung 2015}). We believe some other people  have already 
observed these phenomena so far. For example, we note that Morosawa, 
Nishimura, Taniguchi and Ueda observed this kind of \lq\lq similarity" 
in their book (\cite[p.19]{MTU 1995}, \cite[p.26]{MNTU 2000}). Further, 
earlier than this observation, Peitgen observed a kind of local 
similarity between Mandelbrot set and a Julia set by computer experiment 
(\cite[Figure 4.23]{Peitgen-Saupe 1988}). 










%\if0 %%%%%%%%%%%%%%%%%%%%%%%%%%%%%%%%%%%%
%Figure 1.
\begin{figure}[htbp]
\hskip -40mm
{\small (1)}
\hskip 43mm
{\small (2)}
\hskip 43mm
{\small (3)}

\includegraphics[scale=0.19, bb = 0 0 640 480]{fig_M-4dendrite-001.jpg} \hskip 5mm
\includegraphics[scale=0.19, bb = 0 0 640 480]{fig_M-4dendrite-002.jpg} \hskip 5mm
\includegraphics[scale=0.19, bb = 0 0 640 480]{fig_M-4dendrite-003.jpg} \hskip 5mm

\hskip -40mm
{\small (4)}
\hskip 43mm
{\small (5)}
\hskip 43mm
{\small (6)}


\includegraphics[scale=0.19, bb = 0 0 640 480]{fig_M-4dendrite-004.jpg} \hskip 5mm
\includegraphics[scale=0.19, bb = 0 0 640 480]{fig_M-4dendrite-005.jpg} \hskip 5mm
\includegraphics[scale=0.19, bb = 0 0 640 480]{fig_M-4dendrite-006.jpg} \hskip 5mm


\hskip -40mm
{\small (7)}
\hskip 43mm
{\small (8)}
\hskip 43mm
{\small (9)}

\includegraphics[scale=0.19, bb = 0 0 640 480]{fig_M-4dendrite-007.jpg} \hskip 5mm
\includegraphics[scale=0.19, bb = 0 0 640 480]{fig_M-4dendrite-008.jpg} \hskip 5mm
\includegraphics[scale=0.19, bb = 0 0 640 480]{fig_M-4dendrite-009.jpg} \hskip 5mm


\hskip -38mm
{\small (10)}
\hskip 41mm
{\small (11)}
\hskip 41mm
{\small (12)}

\includegraphics[scale=0.19, bb = 0 0 640 480]{fig_M-4dendrite-010.jpg} \hskip 5mm
\includegraphics[scale=0.19, bb = 0 0 640 480]{fig_M-4dendrite-011.jpg} \hskip 5mm
\includegraphics[scale=0.19, bb = 0 0 640 480]{fig_M-4dendrite-012.jpg} \hskip 5mm



\hskip -38mm
{\small (13)}
\hskip 41mm
{\small (14)}
\hskip 41mm
{\small (15)}

\includegraphics[scale=0.19, bb = 0 0 640 480]{fig_M-4dendrite-013.jpg} \hskip 5mm
\includegraphics[scale=0.19, bb = 0 0 640 480]{fig_M-4dendrite-014.jpg} \hskip 5mm
\includegraphics[scale=0.19, bb = 0 0 640 480]{fig_M-4dendrite-015.jpg} \hskip 5mm

\caption{\small Zooms around a Misiurewicz point 
$c_1 = s_0 \perp c_0$ 
in a primitive small Mandelbrot set $M_{s_0}$, where $c_0$ is a 
Misiurewicz parameter satisfying $P_{c_0}(P_{c_0}^4(0)) = P_{c_0}^4(0)$.
After a sequence of nested structures, another smaller Mandelbrot
set $M_{s_1}$ appears in (15). Here, 
$s_0 \approx 0.3591071125276155 + 0.6423830938166145i$, \
$c_0 \approx -0.1010963638456221 + 0.9562865108091415i$, \
$c_1 \approx 0.3626697754647427 + 0.6450273437137847i$ and
$s_1 \approx 0.3626684938191616 + 0.6450238859863952i$. 
The widths of the figures (1) and (15) are about $10^{-1.5}$
and $10^{-11.9}$, respectively.}
\label{figures of a primitive-Misiurewicz case}
\end{figure}
%\fi %%%%%%%%%%%%%%%%%%%%%%%%%%%%%%%%%%%%









There are different kinds of known results so far which show that
some parts of the Mandelbrot set are similar to some (part of) Julia sets. 
The first famous result for this kind of phenomena is the one by Tan Lei 
(\cite{Tan Lei 1990}). She showed that as we zoom in the neighborhood of
any Misiurewicz parameter $c \in \pt M$, it looks like very much the same as
the magnification of $J_c$ in the neighborhood of $c \in J_c$. Later this
result was generalized to the case where $c$ is a semihyperbolic parameter
by Rivera-Letelier (\cite{Rivera-Letelier 2001}) and its alternative proof
is given by the first author (\cite{Kawahira 2014}). 
On the other hand, some connected Julia sets of quadratic polynomial can
appear quasiconformally in a certain parameter space of a family of cubic 
polynomials. Buff and Henriksen showed that the bifurcation locus of the family 
$\{ f_b(z) = \lambda z + bz^2 + z^3 \}_{b \in \C}$, where  $\lambda \in \C$ with
$|\lambda|=1$ contains quasiconformal copies of $J(\lambda z + z^2)$ 
(\cite{Buff-Henriksen 2001}. See also \cite{Cornell-Rojas-Yampolsky 2017} 
for (non-)computability of the bifurcation locus of such a family for some 
$\lambda$.).



%\if0 %%%%%%%%%%%%%%%%%%%%%%%%%%%%%%%%%%%%%%
%Figure 2.
\begin{figure}[htbp] \small
\hskip -40mm
(1)
\hskip 43mm
(2)
\hskip 43mm
(3)

\includegraphics[scale=0.19, bb = 0 0 640 480]{fig_J-4dendrite-001.jpg} \hskip 5mm
\includegraphics[scale=0.19, bb = 0 0 640 480]{fig_J-4dendrite-002.jpg} \hskip 5mm
\includegraphics[scale=0.19, bb = 0 0 640 480]{fig_J-4dendrite-003.jpg} 


\hskip -40mm
(4)
\hskip 43mm
(5)
\hskip 43mm
(6)

\includegraphics[scale=0.19, bb = 0 0 640 480]{fig_J-4dendrite-004.jpg} \hskip 5mm
\includegraphics[scale=0.19, bb = 0 0 640 480]{fig_J-4dendrite-005.jpg} \hskip 5mm
\includegraphics[scale=0.19, bb = 0 0 640 480]{fig_J-4dendrite-006.jpg}

\hskip -40mm
(7)
\hskip 43mm
(8)
\hskip 43mm
(9)

\includegraphics[scale=0.19, bb = 0 0 640 480]{fig_J-4dendrite-007.jpg} \hskip 5mm
\includegraphics[scale=0.19, bb = 0 0 640 480]{fig_J-4dendrite-008.jpg} \hskip 5mm
\includegraphics[scale=0.19, bb = 0 0 640 480]{fig_J-4dendrite-009.jpg}

\hskip -38mm
(10)
\hskip 41mm
(11)
\hskip 41mm
(12)

\includegraphics[scale=0.19, bb = 0 0 640 480]{fig_J-4dendrite-010.jpg} \hskip 5mm
\includegraphics[scale=0.19, bb = 0 0 640 480]{fig_J-4dendrite-011.jpg} \hskip 5mm
\includegraphics[scale=0.19, bb = 0 0 640 480]{fig_J-4dendrite-012.jpg} 
\caption{\small Zooms around the critical point $0$ in $K_{s_1 \perp c}$
for $s_1 \perp c$ in $M_{s_1}$, 
%($s_1 \perp c \in M_{s_1}$), 
which is the smaller Mandelbrot set in 
Figure \ref{figures of a primitive-Misiurewicz case}--(15) and $c \in M$
is the parameter for the Douady rabbit. 
$s_1 \approx 0.3626684938191616+0.6450238859863952i$, 
$c \approx -0.12256+0.74486i$ and 
$s_1 \perp c \approx 0.3626684938192285 + 0.6450238859865394i$.
}
\label{nested structure for a filled Julia set}
\end{figure}
%\fi %%%%%%%%%%%%%%%%%%%%%%%%%%%%%%%%%%%%%%%%






The organization of this paper is as follows: 
In section 2, we construct models for the nested structures mentioned 
above, define the small Mandelbrot set, and show the precise statements 
of the main results (Theorems A, B, C and Corollary D). In section 3 we
recall the definitions and basic facts on quadratic-like maps
and Mandelbrot-like families. We prove Theorem A for the Misiurewicz case in 
section 4 and for the parabolic case in section 5. We prove Theorem B in 
section 6.
In section 7 we establish a general formulation
of quadratic-like families that generate \lq\lq fine" copies of the Mandelbrot 
set and we prove Theorem C based on this formulation in section 8. We prove 
Corollary D in section 9
and finally we end this paper with some concluding remarks in section 10. 




\noin
{\bf Acknowledgment: } 
We thank Arnaud Ch\'eritat for informing us a work by Wolf Jung. Also we thank 
Wolf Jung for the information of the web pages (\cite{Jung 2015}).










\section{The Model Sets and the Statements of the Results}

\noin
{\bf Notation.} 
We use the following notation for disks and annuli:
\begin{eqnarray*}
& &  D(R) := \{ z \in \C \ | \ |z| < R \}, \quad
  D(\alpha, R) := \{ z \in \C \ | \ |z-\alpha| < R \}, \\
& & A(r, R) := \{ z \in \C \ | \ r < |z| < R \} \quad (0 < r < R).
\end{eqnarray*}
We mostly follow Douady's notations in \cite{Douady 2000} in the 
following. 

\noin
{\bf Models.}
Let $c' \notin M$. Then $J_{c'}$ is a Cantor set which does not contain $0$. 
Now take two positive numbers $\rho'$ and $\rho$ such that 
$$
J_{c'} 
\subset 
A(\rho', \rho) \quad (\rho' < \rho). 
$$
We define 
the {\it rescaled Julia set}
$\Gamma_0(c')=\Gamma_0(c')_{\rho',\rho}$
by
$$
\Gamma_0(c') 
:= 
J_{c'} \times \frac{\rho}{(\rho')^2}
=
\braces{ \frac{\rho}{(\rho')^2} \,z \ \big| \ z \in J_{c'} }
$$ 
such that 
$\Gamma_0(c')$ 
is contained in the annulus 
$A(R, R^2)$ with $R:=\rho/\rho'$. 
(In \cite{Douady 2000}, Douady used the radii of the form 
$\rho'=R^{-1/2}$ and $\rho=R^{1/2}$ for some $R>1$ 
such that $\Gamma(c')=J_{c'} \times R^{3/2}$
is contained in $A(R, R^2)$. 
In this paper, however, we need more flexibility
when we are concerned with the dilatation.)

Let $\Gamma_m(c') \ (m \in \N)$ be the inverse image of $\Gamma_0(c')$ by 
$z \mapsto z^{2^m}$. Then $\Gamma_m(c') \ (m=0, \ 1, \ 2, \ \cdots)$ 
are mutually disjoint, because we have
$$
\Gamma_0(c') \subset A(R, R^2), \
\Gamma_1(c') \subset A(R^{1/2}, R), \
\Gamma_2(c') \subset A(R^{1/4}, R^{1/2}), \cdots.
$$


For another parameter $c \in M$, let 
$\Phi_c : \C \smallsetminus K_c \to \C \smallsetminus \overline{\D}$ be the
B\"ottcher coordinate (i.e., $\Phi_c$ is a conformal isomorphism with 
$\Phi_c(P_c(z)) = (\Phi_c(z))^2$). 
Let $\Phi_M : \C \smallsetminus M \to \C \smallsetminus \overline{\D}$ 
be the conformal isomorphism with 
$\Phi_M(c)/c \to 1 \ \text{as} \ |c| \to \infty$. (It is known that
$\Phi_M(c) := \Phi_c(c)$. See \cite{DH Orsay}.) Now define the {\it model sets} 
$\mathcal{M}(c')$ and $\mathcal{K}_c(c')$
as follows (see Figure \ref{figure of models_MK}):
$$
  \mathcal{M}(c') 
:= M \cup \Phi_M^{-1}\Big( \bigcup_{m=0}^\infty \Gamma_m(c') \Big), \quad
  \mathcal{K}_c(c') 
:= K_c \cup \Phi_c^{-1}\Big( \bigcup_{m=0}^\infty \Gamma_m(c') \Big).
$$
We especially call 
${\mathcal M}(c')$ a {\it decorated Mandelbrot set},
${\mathcal M}(c') \smallsetminus M 
= \Phi_M^{-1}\Big( \bigcup_{m=0}^\infty \Gamma_m(c') \Big)$ its 
{\it decoration} and $M \subset {\mathcal M}(c')$
the {\it main Mandelbrot set} of ${\mathcal M}(c')$. 
Also we call
${\mathcal K}_c(c')$ a {\it decorated filled Julia set} and
${\mathcal K}_c(c') \smallsetminus K_c
= \Phi_c^{-1}\Big( \bigcup_{m=0}^\infty \Gamma_m(c') \Big)$ its 
{\it decoration}. We will apply the same terminologies to the images of 
${\mathcal M}(c')$ or
${\mathcal K}_c(c')$ by 
quasiconformal maps.
Note that the sets 
$\Gamma_m(c')~(m \ge 0)$,
$\cM(c')$ and 
$\cK_{c}(c')$
depend on the choice of $\rho'$ and $\rho$.
When we want to
emphasize the dependence,
we denote them by 
$\Gamma_m(c')_{\rho', \rho}$, 
$\cM(c')_{\rho', \rho}$ and
$\cK_{c}(c')_{\rho', \rho}$, 
respectively. 



%Figure 3.
\begin{figure}[htbp]
%\includegraphics[scale=0.9]{fig_models_MK.eps} 
%\includegraphics[scale=0.9]{fig_models_MK.png} 

%\includegraphics[scale=0.96]{fig_models_MK.eps} 
%\includegraphics[scale=0.56, bb= 0 0 1604 1158]{fig_models_MK.png}
%\includegraphics[width=0.99\textwidth, bb= 0 0 535 386]{fig_models_MK.png}
\includegraphics[width=0.95\textwidth]{fig_models_MK.png}
%\includegraphics[width=1.99\textwidth, bb= 0 0 1604 1158]{fig_models_MK.png}
%\includegraphics[bb= 0 0 1604 1158]{fig_models_MK.png}
\caption{\small 
The first row depicts the decorated Mandelbrot set
$\mathcal{M}(c')$ for $c'= -0.10 + 0.97i$
(close to the Misiurewicz parameter $c_0 \approx -0.1011+0.9563i$,
the landing point of the external ray of angle $11/56$)
and $R=220$.
The second row depicts the set $\bigcup_{m \ge 0} \Gamma_m(c')$.
The third row depicts the decorated filled Julia set
$\mathcal{K}_c(c')$ for $c \approx -0.123+0.745$ (the rabbit).
}
\label{figure of models_MK}
\end{figure}







\noin
{\bf Small Mandelbrot sets.} 
When we zoom in the boundary of $M$, a lot of \lq\lq small Mandelbrot sets" 
appear and it is known that these sets are obtained as follows: (This is 
the result by Douady and Hubbard and its proof can be found in 
\cite[Th\'eor\`eme 1 du Modulation]{Haissinsky 2000}. See also 
\cite{Milnor 2000}.) 
Let $s_0 \ne 0$ be a {\it superattracting parameter}, that is, 
$P_{s_0}(z) = z^2 + s_0$ has a superattracting periodic point, and denote
its period by $p \geq 2$. Then there exists a unique small Mandelbrot set 
$M_{s_0}$ containing $s_0$ and a canonical homeomorphism 
$\chi : M_{s_0} \to M$ with $\chi(s_0) = 0$. Following Douady and Hubbard
we use the notation $s_0 \perp M$ and $s_0 \perp c_0$  to 
denote $M_{s_0} = \chi^{-1}(M)$ and $\chi^{-1}(c_0) \ (c_0 \in M)$, 
respectively. The set $M_{s_0}$ is called the {\it small Mandelbrot set 
with center $s_0$} (see Figure \ref{primitive and satellite small M-set}). 
If $c_1 := s_0 \perp c_0 \ (c_0 \in M)$, then $c_1$ is a parameter in 
$M_{s_0}$ which corresponds to $c_0 \in M$ and it is known that $P_{c_1}$ 
is renormalizable with period $p$ and $P_{c_1}^p$ is hybrid equivalent 
(see section 3) to $P_{c_0}$. 
We say $M_{s_0}$ is {\it primitive} if $K_{s_0 \perp (1/4)}$ has a parabolic
periodic point with a single petal. Otherwise we say $M_{s_0}$ is 
{\it satellite}, in which case $K_{s_0 \perp (1/4)}$ has a parabolic
periodic point with more than one petal. 
It is known that $M_{s_0}$ is primitive if the hyperbolic component
containing $s_0$ has a cusp on the boundary curve, and that $M_{s_0}$
is satellite if the hyperbolic component containing $s_0$ is attached
to another hyperbolic component at $s_0 \perp (1/4)$
(see Figure \ref{primitive and satellite small M-set}).



%Figure 4.
\begin{figure}[htbp]
%\includegraphics[width=0.95\textwidth, bb=0 0 2026 588]{fig_tuning.png}
\includegraphics[width=0.95\textwidth]{fig_tuning.png}
%\includegraphics[scale=0.9]{fig_tuning.eps}
%\includegraphics[scale=0.9]{fig_tuning.png}
%\includegraphics[scale=0.95]{fig_tuning.eps}
\caption{\small
The original Mandelbrot set (left), a ``satellite" small Mandelbrot
set (middle),
and a ``primitive" small Mandelbrot set (right).
The stars indicate the central superattracting parameters.
}
\label{primitive and satellite small M-set}
\end{figure}



%Figure 5.
\fboxsep=0pt
\fboxrule=1pt
\begin{figure}[htbp]
\begin{center}
(i)\\[.5em]
\fbox{\includegraphics[width=.18\textwidth, bb = 0 0 1000 1002]{dM01.png}}
\fbox{\includegraphics[width=.18\textwidth, bb = 0 0 1002 1000]{dM02.png}}
\fbox{\includegraphics[width=.18\textwidth, bb = 0 0 1002 998]{dM03.png}}
\fbox{\includegraphics[width=.18\textwidth, bb = 0 0 1002 1002]{dM04.png}}
\fbox{\includegraphics[width=.18\textwidth, bb = 0 0 1002 1002]{dM05.png}}
\\[.5em]
(ii) \\[.5em]%Satellite case with center 
%$c=-0.10134021126195294+0.8393619749604935 i$:
\fbox{\includegraphics[width=.18\textwidth, bb = 0 0 500 500]{sb01.png}}
\fbox{\includegraphics[width=.18\textwidth, bb = 0 0 500 500]{sb02.png}}
\fbox{\includegraphics[width=.18\textwidth, bb = 0 0 500 500]{sb03.png}}
\fbox{\includegraphics[width=.18\textwidth, bb = 0 0 500 500]{sb04.png}}
\fbox{\includegraphics[width=.18\textwidth, bb = 0 0 500 500]{sb05.png}}
\\[.5em]
\fbox{\includegraphics[width=.18\textwidth, bb = 0 0 500 500]{sb06.png}}
\fbox{\includegraphics[width=.18\textwidth, bb = 0 0 500 500]{sb07.png}}
\fbox{\includegraphics[width=.18\textwidth, bb = 0 0 500 500]{sb08.png}}
\fbox{\includegraphics[width=.18\textwidth, bb = 0 0 500 500]{sb09.png}}
\fbox{\includegraphics[width=.18\textwidth, bb = 0 0 500 500]{sb10.png}}
\\[.5em]
(iii) \\[.5em]
%Primitive case with center $c=-0.160658867 +1.0371058141073i$:\\[.3em]
\fbox{\includegraphics[width=.18\textwidth, bb = 0 0 500 500]{pb01.png}}
\fbox{\includegraphics[width=.18\textwidth, bb = 0 0 500 500]{pb02.png}}
\fbox{\includegraphics[width=.18\textwidth, bb = 0 0 500 500]{pb03.png}}
\fbox{\includegraphics[width=.18\textwidth, bb = 0 0 500 500]{pb04.png}}
\fbox{\includegraphics[width=.18\textwidth, bb = 0 0 500 500]{pb05.png}}
\\[.5em]
\fbox{\includegraphics[width=.18\textwidth, bb = 0 0 500 500]{pb06.png}}
\fbox{\includegraphics[width=.18\textwidth, bb = 0 0 500 500]{pb07.png}}
\fbox{\includegraphics[width=.18\textwidth, bb = 0 0 500 500]{pb08.png}}
\fbox{\includegraphics[width=.18\textwidth, bb = 0 0 500 500]{pb09.png}}
\fbox{\includegraphics[width=.18\textwidth, bb = 0 0 500 500]{pb10.png}}
\end{center}
\caption{\small
(i): The decorated Mandelbrot set $\mathcal{M}(c')$ for $c'=-0.77+0.18 i$
(close to the parabolic parameter $c_0=-0.75$. 
(ii) and (iii): Embedded quasiconformal copies of $\mathcal{M}(c')$ 
above near the satellite/primitive small Mandelbrot sets 
in Figure \ref{primitive and satellite small M-set}.
}
\label{fig_satellite_primitive}
\end{figure}










\begin{defn}
Let $X$ and $Y$ be non-empty compact sets in $\C$. We say {\it $X$ 
appears ($K$-)quasiconformally in $Y$} or {\it $Y$ contains a 
($K$-)quasiconformal copy of $X$} if there is a ($K$-)quasiconformal map 
$\chi$ on a neighborhood of $X$ such that
$\chi(X) \subset Y$ and $\chi(\partial X) \subset \partial Y$. Note that
the condition $\chi(\partial X) \subset \partial Y$ is to exclude the case
$\chi(X) \subset \text{int}(Y)$.
\end{defn}


Now our results are as follows:


\begin{thmA*}[\bf Julia sets appear quasiconformally in $M$]
Let $M_{s_0}$ be any small Mandelbrot set, where $s_0 \ne 0$ is a
superattracting parameter and $c_0 \in \partial M$ 
any Misiurewicz or parabolic parameter. Then
for every small $\vep > 0$ and $\vep' > 0$, there exists an $\eta \in \C$ 
with $|\eta| < \vep$ and
$c_0+\eta \notin M$ such that
${\mathcal M}(c_0+\eta)$ appears quasiconformally in $M$ in the neighborhood
$D(s_0 \perp c_0, \vep')$ of $s_0 \perp c_0$. In particular, the Cantor
Julia set $J_{c_0+\eta}$ appears quasiconformally in $M$.
\end{thmA*}


\noin
Theorem A shows the following: Take any small Mandelbrot set $M_{s_0}$
and zoom in any small neighborhood of $s_0 \perp c_0 \in M_{s_0}$, then we 
can find a quasiconformal image of ${\mathcal M}(c_0+\eta)$. That is, as we 
zoom in, first we observe a quasiconformal image of $J_{c_0+\eta}$, which 
corresponds to the $\Phi_M^{-1}$ image of the rescaled Cantor Julia set 
$\Gamma_0(c_0+\eta)$ in ${\mathcal M}(c_0+\eta)$ and its iterated preimages
(decoration) by $z \mapsto z^2$ and finally the main Mandelbrot set of the
quasiconformal image of ${\mathcal M}(c_0+\eta)$, say $M_{s_1}$, appears.


Figure \ref{figures of a primitive-Misiurewicz case} shows zooms around
a Misiurewicz parameter $c_1 = s_0 \perp c_0$ in a primitive small 
Mandelbrot set $M_{s_0}$ 
(Figure \ref{figures of a primitive-Misiurewicz case}--(1)). 
After a sequence of nested structures, a smaller 
\lq\lq small Mandelbrot set" $M_{s_1}$ appears 
(Figure \ref{figures of a primitive-Misiurewicz case}--(15)). 
Here $M_{s_0}$ is the relatively big \lq\lq small 
Mandelbrot set" which is located in the upper right part of $M$. The map 
$P_{s_0}$ has a superattracting periodic point of period 4 and $c_0$ is the 
Misiurewicz parameter which satisfies $P_{c_0}(P_{c_0}^4(0)) = P_{c_0}^4(0)$
and corresponds to the \lq\lq junction of three roads" as shown in
Figure \ref{primitive and satellite small M-set} in the middle.





Since Misiurewicz or parabolic parameters are dense in $\partial M$, 
we can reformulate Theorem A as follows: 
{\it 
Let $M_{s_0}$ be any small Mandelbrot set, where $s_0 \ne 0$ is a
superattracting parameter and $c_0 \in \partial M$ 
any parameter. Then
for every small $\vep > 0$ and $\vep' > 0$, there exists an $\eta \in \C$
with $|\eta| < \vep$ and $c_0+\eta \notin  M$ such that
${\mathcal M}(c_0+\eta)$ appears quasiconformally in $M$ in the 
neighborhood $D(s_0 \perp c_0, \vep')$. In particular, the Cantor
Julia set $J_{c_0+\eta}$ appears quasiconformally in $M$. 
}


Next we show that the same decoration of $\cM(c_0+\eta)$ in Theorem A
appears quasiconformally also in some filled Julia sets.



\begin{thmB*}[\bf Decoration in filled Julia sets]
Let $M_{s_1}$ denote the main Mandelbrot set of the quasiconformal image of
${\mathcal M}(c_0+\eta)$ in Theorem A. Then for every 
$c \in M$, ${\mathcal K}_{c}(c_0+\eta)$ appears 
quasiconformally in $K_{s_1 \perp c}$, where $s_1 \perp c \in M_{s_1}$.
\end{thmB*} 




\noin
Theorem B shows the following: Choose any parameter from the main Mandelbrot
set $M_{s_1}$ in the quasiconformal image of ${\mathcal M}(c_0+\eta)$,
that is, choose any $c \in M$ and consider $s_1 \perp c$ $\in M_{s_1}$
and zoom in the neighborhood of $0 \in K_{s_1 \perp c}$. Then we can find
a quasiconformal image of ${\mathcal K}_{c}(c_0+\eta)$, whose decoration is
conformally the same as that of ${\mathcal M}(c_0+\eta)$.





Next we show that there are smaller Mandelbrot sets and their 
decorations which are images of model sets by quasiconformal maps
whose dilatations are arbitrarily close to 1.




\begin{thmC*}[\bf Almost conformal copies]
Let $c_0$ be any Misiurewicz or parabolic parameter
and $B$ any small closed disk whose interior intersects with $\partial M$.
Then for any small $\vep > 0$ and $\kappa>0$,
there exist an $\eta \in \C$ with $|\eta| < \vep$ and two positive numbers
$\rho$ and $\rho'$ with $\rho' < \rho$ 
such that $c_0 +\eta \notin M$ and $\cM(c_0+\eta)_{\rho', \rho}$ 
appears $(1+\kappa)$-quasiconformally 
in $B \cap M$.
In particular, $B \cap \partial M$ contains a $(1+\kappa)$-quasiconformal
copy of the Cantor Julia set $J_{c_0+\eta}$.
\end{thmC*}





\begin{defn}[\bf Semihyperbolicity]
A quadratic polynomial $P_c(z) = z^2 + c$ (or the parameter $c$) is 
called {\it semihyperbolic} if  
\begin{itemize}
\item[(1)]  the critical point $0$ is non-recurrent, that is, 
$0 \notin \omega(0)$, where $\omega(0)$ is the $\omega$-limit set of 
the critical point $0$ and

\item[(2)] $P_c$ has no parabolic periodic points.
\end{itemize}
\end{defn}




\noin
It is easy to see that if $P_c$ is hyperbolic then it is semihyperbolic.
If $P_c$ is semihyperbolic, then it is known that it has no Siegel
disks and Cremer points 
(\cite{Mane 1993}, \cite{Carleson-Jones-Yoccoz 1994}). Also it is not 
difficult to see that $J_c$ is measure 0 from the result by Lyubich 
(\cite{Lyubich 1991}) and Shishikura. (This also follows from 
\cite[p.2, Theorem 1.1]{Carleson-Jones-Yoccoz 1994}.) Thus the 
semihyperbolic dynamics is relatively understandable. A typical 
semihyperbolic but non-hyperbolic parameter $c$ is a Misiurewicz 
parameter. But there seems less concrete examples of semihyperbolic
parameter $c$ which is neither hyperbolic nor Misiurewicz. Next 
corollary shows that we can at least \lq\lq see" such a parameter 
everywhere in $\partial M$.




\begin{corD*}[\bf Abundance of semihyperbolicity]
For every parameter $c$ belonging to the quasiconformal image
of the decoration of ${\mathcal M}(c_0+\eta)$ in Theorem A, 
$P_c$ is semihyperbolic. Also the set of semihyperbolic parameters which 
are not Misiurewicz and non-hyperbolic is dense in $\partial M$. 
\end{corD*}


\noindent
Corollary D together with Theorem C explains the following famous result
by Shishikura:


\begin{thm*}[Shishikura, 1998]
Let
$$
  SH := \{ c \in \partial M \ | \ P_c \text{ is semihyperbolic} \},
$$
then the Hausdorff dimension of $SH$ is 2. In particular, the Hausdorff
dimension of the boundary of $M$ is 2.
\end{thm*}

\noin
{\bf Explanation.} \ 
Since there exist quadratic Cantor Julia sets with Hausdorff dimension
arbitrarily close to 2 and we can find such parameters in every 
neighborhood of a point in $\partial M$ 
(\cite[p.231, {\it proof of Theorem B} and p.232, {\it Remark} 1.1 (iii)]{Shishikura 1998}), we can find an 
$\eta$ such that $\dim_H(J_{c_0+\eta})$ is
arbitrarily close to 2. Then by Theorem C and Corollary D it follows that
we can find a subset of $\partial M$ with Hausdorff dimension
arbitrarily close to 2 and consisting of semihyperbolic parameters 
as a quasiconformal image of the decoration of ${\mathcal M}(c_0+\eta)$.
This implies that $\dim_H(SH) = 2$. 
\QED




%\if0 %%%%%%%%%%%%%%%%%%%%
%Figure 6.
\begin{figure}[htbp]
\hskip -40mm
{\small (1)}
\hskip 43mm
{\small (2)}
\hskip 43mm
{\small (3)}

\includegraphics[scale=0.19, bb = 0 0 640 480]{fig_M-4dendrite-015.jpg} \hskip 5mm
\includegraphics[scale=0.19, bb = 0 0 640 480]{fig-M-4per-3Mis-1para-0002N.jpg} \hskip 5mm
\includegraphics[scale=0.19, bb = 0 0 640 480]{fig-M-4per-3Mis-1para-0003N.jpg} \hskip 5mm

\hskip -40mm
{\small (4)}
\hskip 43mm
{\small (5)}
\hskip 43mm
{\small (6)}

\includegraphics[scale=0.19, bb = 0 0 640 480]{fig-M-4per-3Mis-1para-0004N.jpg} \hskip 5mm
\includegraphics[scale=0.19, bb = 0 0 640 480]{fig-M-4per-3Mis-1para-0005N.jpg} \hskip 5mm
\includegraphics[scale=0.19, bb = 0 0 640 480]{fig-M-4per-3Mis-1para-0006N.jpg} \hskip 5mm


\hskip -40mm
{\small (7)}
\hskip 43mm
{\small (8)}
\hskip 43mm
{\small (9)}

\includegraphics[scale=0.19, bb = 0 0 640 480]{fig-M-4per-3Mis-1para-0007N.jpg} \hskip 5mm
\includegraphics[scale=0.19, bb = 0 0 640 480]{fig-M-4per-3Mis-1para-0008N.jpg} \hskip 5mm
\includegraphics[scale=0.19, bb = 0 0 640 480]{fig-M-4per-3Mis-1para-0009N.jpg} \hskip 5mm


\hskip -38mm
{\small (10)}
\hskip 41mm
{\small (11)}
\hskip 41mm
{\small (12)}

\includegraphics[scale=0.19, bb = 0 0 640 480]{fig-M-4per-3Mis-1para-0010N.jpg} \hskip 5mm
\includegraphics[scale=0.19, bb = 0 0 640 480]{fig-M-4per-3Mis-1para-0011N.jpg} \hskip 5mm
\includegraphics[scale=0.19, bb = 0 0 640 480]{fig-M-4per-3Mis-1para-0012N.jpg} \hskip 5mm



\hskip -38mm
{\small (13)}
\hskip 41mm
{\small (14)}
\hskip 41mm
{\small (15)}

\includegraphics[scale=0.19, bb = 0 0 640 480]{fig-M-4per-3Mis-1para-0013N.jpg} \hskip 5mm
\includegraphics[scale=0.19, bb = 0 0 640 480]{fig-M-4per-3Mis-1para-0014N.jpg} \hskip 5mm
\includegraphics[scale=0.19, bb = 0 0 640 480]{fig-M-4per-3Mis-1para-0015N.jpg} \hskip 5mm

\caption{\small Zooms around a parabolic point $s_1 \perp c_1$ 
in a primitive small Mandelbrot set $M_{s_1}$. After a sequence of 
complicated nested structures, another smaller Mandelbrot set 
$M_{s_2}$ appears ((15)).}
\label{figures of a complicated structure}
\end{figure}
%\fi %%%%%%%%%%%%%%%%%%%%%%%%%%%%%%%












\begin{rem*}
(1) \ 
A similar result to Theorem B still holds even when $c \in \C \smallsetminus M$
is sufficiently close to $M$. Actually when $M_{s_1}$ is primitive (resp. satellite),
the homeomorphism $\chi : M_{s_1} \to M$ can be extended to a homeomorphism between
some neighborhoods of $M_{s_1}$ and $M$ 
(resp. $M_{s_1} \smallsetminus \{ s_1 \perp (1/4) \}$ and 
$M \smallsetminus \{ 1/4 \}$)
and so $s_1 \perp c = \chi^{-1}(c)$ can be still defined for such a 
$c \in \C \smallsetminus M$. Then in this case, by modifying the definition 
of ${\mathcal K}_{c}(c_0+\eta)$ for this $c \in \C \smallsetminus M$ we can 
prove that a \lq\lq ${\mathcal K}_{c}(c_0+\eta)$" appears quasiconformally
in $K_{s_1 \perp c}$, where $s_1 \perp c \in \C \smallsetminus M_{s_1}$ is a
point which is sufficiently close to $M_{s_1}$. We omit the details.


\noin
(2) \ 
Corollary D and the above Theorem (Shishikura, 1998) means that relatively
understandable dynamics is abundant in $\partial M$ (provided that Lebesgue
measure of $\partial M$ is 0). 
In \cite[p.225, THEOREM A]{Shishikura 1998}, Shishikura actually proved
$\dim_H(SH) = 2$, which immediately implies $\dim_H(\partial M) = 2$.
A new point of our \lq\lq explanation" is that we constructed a decoration 
in $M$ which contains a quasiconformal image of a whole Cantor Julia set
and consists of semihyperbolic parameters. 
So now we can say that \lq\lq$\dim_H(\partial M) = 2$ holds, because
we can see a lot of almost conformal images of Cantor quadratic Julia sets
whose Hausdorff dimension are arbitrarily close to 2".

\noin
(3) \ 
Take a small Mandelbrot set $M_{s_1}$ (e.g. 
Figure \ref{figures of a primitive-Misiurewicz case}--(15) 
= Figure \ref{figures of a complicated structure}--(1)) 
and another Misiurewicz or parabolic parameter $c_*$ (e.g. $c_* = 1/4$ in
Figure 6) and zoom in the neighborhood of $s_1 \perp c_*$. Then we
see much more complicated structure than we expected as follows:
According to Theorem A, by replacing $s_0$ with $s_1$ and $c_0$ with $c_*$, 
it says that ${\mathcal M}(c_* + \eta)$
appears quasiconformally in $D(s_1 \perp c_*, \vep')$. This means that as 
we zoom in, we first see a quasiconformal image of $J_{c_*+\eta_*}$, say 
$\wt{J}_{c_*+\eta_*}$ (e.g. \lq\lq broken cauliflower", when $c_* = 1/4$). 
But in reality as we zoom in, what we first see is a $\wt{J}_{c_0+\eta_0}$ 
(e.g. \lq\lq broken dendrite". 
See Figure \ref{figures of a complicated structure}--(5)).
This seems to contradict with Theorem A, but actually it does not. As 
we zoom in further in the middle part of $\wt{J}_{c_0+\eta_0}$,  we
see iterated preimages of $\wt{J}_{c_0+\eta_0}$ by $z \mapsto z^2$ 
(Figure \ref{figures of a complicated structure}--(6), (7)) 
and then 
$\wt{J}_{c_*+\eta_*}$ appears 
(Figure \ref{figures of a complicated structure}--(8)). After that we 
see again iterated preimages of $\wt{J}_{c_0+\eta_0}$ by $z \mapsto z^2$ 
(Figure \ref{figures of a complicated structure}--(9), (10)) 
and then a once iterated preimage of $\wt{J}_{c_*+\eta_*}$ appears
(Figure \ref{figures of a complicated structure}--(11)). This complicated 
structure continues and finally, we see a smaller Mandelbrot set,
say $M_{s_2}$ (Figure \ref{figures of a complicated structure}--(15)). 
We can explain this complicated phenomena as follows: What we see
in the series of magnifications above is a quasiconformal image of 
$\cM (\cK_{c_*+\eta_*}(c_0+\eta_0))$, where
\begin{eqnarray*}
\cM (\cK_{c_*+\eta_*}(c_0+\eta_0))
& := &
M \cup 
\Phi_M^{-1}\Big( 
       \bigcup_{m=0}^\infty \Gamma_m(\cK_{c_*+\eta_*}(c_0+\eta_0)) \Big), \\
\Gamma_m(\cK_{c_*+\eta_*}(c_0+\eta_0)) 
& := &
\text{the inverse image of } \Gamma_0(\cK_{c_*+\eta_*}(c_0+\eta_0)) \ \text{by} \
z \mapsto z^{2^m}. 
\end{eqnarray*}
Here $\cM (\cK_{c_*+\eta_*}(c_0+\eta_0))$ is obtained just by replacing
$J_{c'}$ with $\cK_{c_*+\eta_*}(c_0+\eta_0)$ in the definition of 
$\cM(c')$. Although $c_*+\eta_* \notin M$, 
$\cK_{c_*+\eta_*}(c_0+\eta_0)$ can be defined in the similar manner. 
See the Remark (1) above. So what we first 
see as we zoom in the neighborhood of $s_1 \perp c_*$ is a quasiconformal image of
$\cK_{c_*+\eta_*}(c_0+\eta_0)$, whose outer most part is 
$\wt{J}_{c_0+\eta_0}$ ($=$ broken dendrite) and inner most part is 
$\wt{J}_{c_*+\eta_*}$ ($=$ broken cauliflower). As we zoom in further, 
we see quasiconformal image of the preimage of $\cK_{c_*+\eta_*}(c_0+\eta_0)$ by 
$z \mapsto z^2$, whose inner most part is a once iterated preimage of 
$\wt{J}_{c_0+\eta_0}$. After we see successive preimages of 
$\cK_{c_*+\eta_*}(c_0+\eta_0)$ by $z \mapsto z^2$, a much more smaller Mandelbrot
set $M_{s_2}$ finally appears. Since $\cK_{c_*+\eta_*}(c_0+\eta_0)$
itself has a nested structure, the total picture has this very 
complicated structure. The proof is completely the same as for the 
Theorem A. 
\end{rem*}








\section{The quadratic-like maps and the Mandelbrot-like family}

In this section, we briefly recall the definitions of the quadratic-like map and
the Mandelbrot-like family and explain the key Proposition \ref{D-BDS Proposition} 
which is crucial for the proof of Theorem A.


A map $h : U' \to U$ is called {\it a polynomial-like map} if
$U', \ U \subset \C$ are topological disks with $U' \Subset U$ (which means 
$\overline{U'} \subset U$) 
and $h$ is holomorphic and proper map of degree $d$ with respect to $z$. 
It is called a {\it quadratic-like map} when $d=2$.
The {\it filled Julia set} $K(h)$ and the {\it Julia set} $J(h)$ of a 
polynomial-like map $h$ are defined by
\begin{eqnarray*}
  K(h) 
& :=  &
\{ z \in U' \ | \ h^n(z) \text{ are defined for every } n \in \N \}
= \bigcap_{n=0}^\infty h^{-n}(U'), \\
  J(h)
& := &
\partial K(h).
\end{eqnarray*}
The famous straightening theorem by Douady and Hubbard 
(\cite[p.296, THEOREM 1]{Douady-Hubbard 1985}) says that
every polynomial-like map $h : U' \to U$ of degree $d$ is quasiconformally 
conjugate to a polynomial $P$ of degree $d$. More precisely $h$ is 
{\it hybrid equivalent} to $P$, that is, there exists a quasiconformal map
$\phi$ sending a neighborhood of $K(h)$ to a neighborhood of $K(P)$ such
that $\phi \circ h = P \circ \phi$ and $\overline{\partial} \phi = 0 \ \text{a.e.}$
on $K(h)$.
Also if $K(h)$ is connected, then $P$ is unique up to conjugacy by an affine map.


A family of holomorphic maps $\boldsymbol{h} = \{ h_\lambda \}_{\lambda \in W}$ 
is called a {\it Mandelbrot-like family} if the 
following (1)--(8) hold: 


\begin{itemize}
\setlength{\itemsep}{5pt}
\item[(1)]
$W \subset \C$ is a Jordan domain with $C^1$ boundary $\partial W$.

\item[(2)]
There exists a family of maps 
$\Theta = \{ \Theta_\lambda \}_{\lambda \in W}$ such that
for every $\lambda \in W$, $\Theta_\lambda : \overline{A(R, R^2)} \to \C$ is a 
quasiconformal embedding and that $\Theta_\lambda(Z)$ is holomorphic in 
$\lambda$ for every $Z \in \overline{A(R, R^2)}$.


%Figure 7.
\begin{figure}[htbp]
\hskip 1cm
%\includegraphics[scale=0.6]{fig_tubing.eps}
\includegraphics[width=.65\textwidth]{fig_tubing.png}
\caption{\small Tubing 
$\Theta = \{ \Theta_\lambda \}_{\lambda \in W}$.}
\end{figure}


\item[(3)]
Define $C_\lambda := \Theta_\lambda(\partial D(R^2)), \
C'_\lambda := \Theta_\lambda(\partial D(R))$
and let 
$U_\lambda \ (\text{resp.} \ U'_\lambda)$ be the Jordan domain bounded by 
$C_\lambda \ (\text{resp.} \ C'_\lambda)$. Then
$h_\lambda : U'_\lambda \to U_\lambda$ is a quadratic-like map with
a critical point $\omega_\lambda$. Also let
$$
  {\mathcal U} := \{ (\lambda, z) \ | \ \lambda \in W, \ z \in U_\lambda \}, \quad
  {\mathcal U'} := \{ (\lambda, z) \ | \ \lambda \in W, \ z \in U'_\lambda \}
$$
then
$\boldsymbol{h} : {\mathcal U} \to {\mathcal U'}, \ 
(\lambda, z) \mapsto (\lambda, h_\lambda(z))$ is analytic and proper.


\item[(4)]
$\Theta_\lambda(Z^2) = h_\lambda(\Theta_\lambda(Z))$ for $Z \in \partial D(R)$.


\vskip 2mm

\noin
The family of maps 
$\Theta = \{ \Theta_\lambda \}_{\lambda \in W}$ satisfying the above conditions
(1)--(4) is called a {\it tubing}. 

\vskip 2mm


\item[(5)]
$\boldsymbol{h}$ extends continuously to a map 
$\overline{{\mathcal U}'} \to \overline{\mathcal U}$ and 
$\Theta_\lambda : (\lambda ,z) \mapsto (\lambda, \Theta_\lambda(Z))$ extends 
continuously to a map 
$\overline{W} \times \overline{A(R, R^2)} \to \overline{\mathcal U}$
such that $\Theta_\lambda$ is injective on $A(R, R^2)$ for $\lambda \in \partial W$.


\item[(6)]
The map $\lambda \mapsto \omega_\lambda$ extends continuously to $\overline{W}$.

\item[(7)]
$h_\lambda(\omega_\lambda) \in C_\lambda$ for $\lambda \in \partial W$.

\item[(8)]
{\it The one turn condition}: 
When $\lambda$ ranges over $\partial W$ making one turn, then the vector
$h_\lambda(\omega_\lambda) - \omega_\lambda$ makes one turn around 0.
\end{itemize}








Now let $M_{\boldsymbol{h}}$ be the {\it connectedness locus} of the family
$\boldsymbol{h} = \{ h_\lambda \}_{\lambda \in W}$:
$$
M_{\boldsymbol{h}} := \{ \lambda \in W \ | \ K(h_\lambda)  \text{ is connected} \}
= \{ \lambda \in W \ | \ \omega_\lambda \in K(h_\lambda) \}.
$$
Douady and Hubbard (\cite[Chapter IV]{Douady-Hubbard 1985}) showed that there
exists a homeomorphism 
$$
  \chi : M_{\boldsymbol{h}} \to M.
$$ 
This is just a correspondence by the Straightening Theorem, that is, for every
$\lambda \in M_{\boldsymbol{h}}$ there exist a unique
$c = \chi(\lambda) \in M$ such that $h_\lambda$ is hybrid equivalent	
to $P_c(z) = z^2 + c$. Furthermore they
showed that this
$\chi$ can be extended to a homeomorphism $\chi_\Theta : W \to W_M$ by using
$\Theta = \{ \Theta_\lambda \}_{\lambda \in W}$, where
$$
  W_M := \{ c \in \C \ | \ {\mathcal G}_M(c) < 2 \log R \}, 
  \quad {{\mathcal G}_M := }  \text{ the Green function of } M
$$
is a neighborhood of $M$. Also Lyubich showed that $\chi_\Theta$ is 
quasiconformal on any $W'$ with $W' \Subset W$ 
(\cite[p.366, THEOREM 5.5 (The QC Theorem)]{Lyubich 1999}).


Then Douady et al. showed the following:



\begin{prop}\cite[p.29, PROPOSITION 3]{Douady 2000}
For any $\Gamma \subset A(R, R^2)$, 
let $\Gamma_m$ be the preimage of $\Gamma$ by $z \mapsto z^{2^m}$. Then 
$$
  \chi_\Theta^{-1}(\Phi_M^{-1}(\Gamma_m)) 
=\{ \lambda \in W \ | \ h_\lambda^{m+1}(\omega_\lambda) \in \Theta_\lambda(\Gamma) \}
$$
and therefore
$$
  M_{\boldsymbol h}
\cup \{ \lambda \ | \ 
   h_\lambda^k(\omega_\lambda) \in \Theta_\lambda(\Gamma) \ 
   \text{for some} \ k \in \N \}
=
\chi_\Theta^{-1}
\bigg( M \cup \Big( \bigcup_{m=0}^\infty \Phi_M^{-1}(\Gamma_m) \Big) \bigg).
$$
\label{D-BDS Proposition}
\end{prop}


\noin
In what follows, we shall apply this proposition to the rescaled Julia set 
$\Gamma = \Gamma_0(c') (:= J_{c'} \times (\rho/(\rho')^2))$ contained
in $A(R, R^2)$, where $c' \notin M$, $J_{c'} \subset A(\rho', \rho)$ 
and $R := \rho/\rho'$. Then we have
$$
\chi_\Theta^{-1}
\bigg( M \cup \Big( \bigcup_{m=0}^\infty \Phi_M^{-1}(\Gamma_m) \Big) \bigg) 
= 
\chi_\Theta^{-1}
\bigg( M \cup \Phi_M^{-1} \Big( \bigcup_{m=0}^\infty \Gamma_m(c') \Big) \bigg) 
= 
\chi_\Theta^{-1}({\mathcal M}(c')).
$$





\section{Proof of Theorem A for the Misiurewicz case}

Let $M_{s_0}$ be any small Mandelbrot set, where $s_0 \ne 0$ is a 
superattracting parameter (i.e., the critical point $0$ is a periodic 
point of period $p \geq 2$ for $P_{s_0}$).
By the tuning theorem by Douady and Hubbard 
\cite[p.42, Th\'eor\`eme 1 du Modulation]{Haissinsky 2000},
there exists a simply connected domain $\Lambda=\Lambda_{s_0}$ in 
the parameter plane  with the following properties:
\begin{itemize}
\item
If $M_{s_0}$ is a primitive small Mandelbrot set, then 
$M_{s_0}\subset \Lambda$.
\item
If $M_{s_0}$ is a satellite small Mandelbrot set,   
then $M_{s_0}\sminus\{s_0 \perp (1/4)\}\subset \Lambda$.
\item
For any $c \in \Lambda$, $P_c$ is renormalizable with period $p$. 
More precisely, there exist two Jordan domains $\widetilde{U}_c'$ and $\widetilde{U}_c$ with piecewise analytic boundaries such that
$$
  f_c := P_c^p|_{\widetilde{U}_c'} : \widetilde{U}_c' \to \widetilde{U}_c
$$
is a quadratic-like map with a critical point $0 \in \widetilde{U}_c'$. 
In particular, the boundaries of $\widetilde{U}_c'$ and 
$\widetilde{U}_c$ move holomorphically with respect to $c$ over $\Lambda$.
\end{itemize}
In this section let $c_0 \in \partial M$ be any Misiurewicz parameter and $c_1 := s_0 \perp c_0 \in M_{s_0}$. 
The proof of Theorem A for the Misiurewicz case 
breaks into four steps, (M1) to (M4).


\vskip 2mm

\noin
{\bf Step (M1): Definitions of $U_c$, $U_c'$ and $V_c$.} 

\indent
In this step, we shall first construct a 
family $\{f_c:U_c' \to U_c\}_{c\, \in \,S}$ 
of quadratic-like maps over a neighborhood $S$ of $c_1$
by slightly shrinking the original domains of $f_c$,
together with a family of isomorphisms 
$\{g_c:V_c \to U_c\}_{c\, \in \,S}$ 
with a specific property.



We start with the dynamics of $P_{c_1}$ for the parameter $c_1=s_0 \perp c_0$.
Since the quadratic-like map $f_{c_1}= P_{c_1}^p|_{\widetilde{U}_{c_1}'}$ is hybrid equivalent to Misiurewicz $P_{c_0}$,
the ``small" Julia set $J(f_{c_1})$ of $f_{c_1}$ 
is a connected subset of the ``global" Julia set $J_{c_1}$ of $P_{c_1}$.
Note that the parameter $c_1$ is also Misiurewicz and thus $J_{c_1}$ itself is connected. 
Moreover,
$$
  q_{c_1} := f_{c_1}^l(0) \quad \text{for some} \ l \in \N
$$
is a repelling periodic point of period $k$ with 
multiplier 
$$
  \mu_{c_1} := (f_{c_1}^k)'(q_{c_1}).
$$
Let $\phi_{c_1}: \Omega_{c_1} \to \C$ 
be a linearizing coordinate of $q_{c_1}$
defined on a neighborhood $\Omega_{c_1}$ of $q_{c_1}$
such that $\phi_{c_1}(q_{c_1}) = 0$ and 
$\phi_{c_1}(f_{c_1}^k(z)) = \mu_{c_1} \,\phi_{c_1}(z)$.




\begin{lem}
There exist Jordan domains $U_{c_1}$, $U_{c_1}'$ and $V_{c_1}$ with $C^1$
boundaries and integers $N, \ j \in \N$ which satisfy 
the following:
\begin{itemize}
\item[\rm (1)]
$0 \in U_{c_1}' \subset \widetilde{U}_{c_1}'$ and 
$f_{c_1}: U_{c_1}' \to U_{c_1}$ is a 
quadratic-like map.

\item[\rm (2)]
$g_{c_1} := P_{c_1}^N|_{V_{c_1}} : V_{c_1} \to U_{c_1}$ is an isomorphism and
$\overline{f_{c_1}^j(V_{c_1})} 
\subset U_{c_1} \smallsetminus \overline{U_{c_1}'}$.

\item[\rm (3)]
$\overline{V_{c_1}} \subset \Omega_{c_1}$. Also we can take $V_{c_1}$ arbitrarily
close to $q_{c_1} \in \Omega_{c_1}$.
\end{itemize}
\label{def of U_{c_1} etc}
\end{lem}





%Figure 8.
\begin{figure}[htbp]
\begin{center}
%\includegraphics[width=.60\textwidth]{fig_UV.eps}
\includegraphics[width=.60\textwidth]{fig_UV.png}
\caption{\small
The Jordan domains $U_c,\,U_c',\, V_c$, and $V_c'$. 
}\label{UV}
\end{center}
\end{figure}


\paragraph{\bf Proof.}
By slightly shrinking $\widetilde{U}_{c_1}$ and $\widetilde{U}_{c_1}'$,
we can take Jordan domains $U_{c_1}$ and $U_{c_1}'$ 
with $C^1$ boundaries which are neighborhoods of $J(f_{c_1})$ 
and $f_{c_1} : U_{c_1}' \to U_{c_1}$ is a quadratic-like map. 

For $j \in \N$ let
$$
  A_j := f_{c_1}^{-j}(U_{c_1} \smallsetminus \overline{U_{c_1}'}).
$$
Since $f_{c_1}$ is conjugate to $z^2$ on 
$U_{c_1}' \smallsetminus J(f_{c_1})$,
the annulus $A_j$ is uniformly close to $J(f_{c_1})$ and  
thus $A_j \cap \Omega_{c_1} \ne \emptyset$ for every sufficiently large $j$.
Also since the ``global" Julia set $J_{c_1}$ of $P_{c_1}$ is a connected set containing the ``small" Julia set $J(f_{c_1})$, 
the annulus $U_{c_1} \smallsetminus \overline{U_{c_1}'}$
intersects with $J_{c_1}$ 
and so does $A_j$.
In particular, for every sufficiently large $j$, 
$A_j \cap \Omega_{c_1}$ contains 
a point $z_0 \in J_{c_1}$ 
arbitrarily close to the repelling periodic point 
$q_{c_1} \in J(f_{c_1})$.
Let $B$ be any closed disk in $\Omega_{c_1} \cap A_j$ 
centered at this $z_0$ with an arbitrarily small radius. 

Note that the postcritical set of 
the dynamics by $P_{c_1}$ in $\mathbb C$ 
is contained in 
$U_{c_1}' \cup P_{c_1}(U_{c_1}') \cup \cdots \cup P_{c_1}^{p-1}(U_{c_1}')$.
There are two disjoint connected components 
$X:=P_{c_1}^{p-1}(U_{c_1}')$
and 
$-X:=\{-x \in \mathbb{C} ~:~ x \in X\}$
of $P_{c_1}^{-1}(U_{c_1})$, 
where $-X$ does not intersect with neither 
the postcritical set of $P_{c_1}$ nor the critical point $0$.
Hence for any $n \in \N$ and any connected component $V$ of
$P_{c_1}^{-n}(-X)$,
$P_{c_1}^{n+1}:V \to U_{c_1}$ is an isomorphism.

Since the inverse images of $-X$ in the dynamics of $P_{c_1}$
accumulate on any point in the Julia set $J_{c_1}$ of $P_{c_1}$ 
(by Montel's theorem), 
the shrinking lemma (\cite[p.86]{Lyubich-Minsky 1997} or 
\cite[Lem.2.9]{Cui-Tan 2018})
implies that we can find a component $V_{c_1}$ of 
$P_{c_1}^{-N+1}(-X)$ contained in the closed disk $B$ 
for some $N \in \N$. 
This gives a desired isomorphism $g_{c_1}:=P_{c_1}^N:V_{c_1} \to U_{c_1}$.

\hfill
\QED ~{\small (Lemma \ref{def of U_{c_1} etc})}
\\


\begin{rem*}
This proof indicates that there are infinitely many different choices of $V_{c_1}$ and one can choose $V_{c_1}$ with arbitrarily small diameter. 
Indeed, each choice of $V_{c_1}$ will give a different \lq\lq decorated small Mandelbrot set".
\end{rem*}

For $ \vep'>0$ given in the statement of Theorem A,
we take a small neighborhood $S$ of $c_1$ contained in 
$D(c_1, \vep') \cap \Lambda$ and 
let $U_c := U_{c_1}$ for each $c \in S$.
By taking a smaller $U_{c_1}$ 
in the previous lemma
and a smaller $S$ if necessary,
we may assume that 
$U_c \,(\equiv U_{c_1})$ is contained in 
$\widetilde{U}_{c}$ 
for each $c \in S$ and 
$U_c':=f_c^{-1}(U_c) \subset \widetilde{U}_{c}'$ 
gives a quadratic-like map $f_c:U_c' \to U_c$
that is a restriction of the original $f_c: \widetilde{U}_{c}' \to \widetilde{U}_{c}$.
Again by taking a smaller $S$ if necessary, 
we may assume in addition that 
for $c \in S$ there exists a component $V_c$
of $P_c^{-N}(U_c)$ which is close to $V_{c_1}$ such that
$$
  \overline{f_c^j(V_c)} \subset U_c \smallsetminus \overline{U_c'}
$$
and $g_c : V_c \to U_c \,(\equiv U_{c_1})$ is an isomorphism. 
See 
Figure \ref{UV}. Let $b_c := g_c^{-1}(0) \in V_c$ and call it
a {\it pre-critical point}. 

\noindent
{\bf Linearizing coordinates.}
Since the fixed point $q_{c_1}$ of $f_{c_1}^k$ is repelling, 
there exists a repelling fixed point $q_c$ of $f_c^k$
that depends holomorphically on $c$ near $c_1$
by the implicit function theorem.\footnote{
While $q_{c_1} = f_{c_1}^l(0)$, 
we have $q_c \ne f_c^l(0)$ for any $c \ne c_1$ 
in a sufficiently small neighborhood of $c_1$. 
Otherwise $q_c = f_c^l(0)$ for any $c$ close to $c_1$ by the identity
theorem.}
To take advantage of the linearizing coordinates
in the next step, 
let us replace the neighborhood $S$ of $c_1$
by an even smaller one such that for each $c \in S$ there exists a unique linearizing coordinate 
$\phi_c : \Omega_c \to \C$ satisfying the following conditions (\cite[\S 8]{Milnor 2006}):
\begin{itemize}
\item
The domain $\Omega_c$ is a neighborhood of $q_c$
and $\phi_c(q_c) = 0$.
\item
Let $\mu(c) := (f_c^k)'(q_c)$. Then
$
  \phi_c(f_c^k(z)) = \mu(c) \cdot \phi_c(z)
$
if both $z$ and $f_c^{k}(z)$ are contained in $\Omega_c$.
\item
(Holomorphic dependence) 
Every compact set $E$ in $\Omega_{c_1}$ 
is contained in $\Omega_c$ for $c \in S$ sufficiently close to $c_1$,
and $\phi_c(z)$ depends holomorphically on $c$ near $c_1$ for each $z \in \Omega_{c_1}$.
\item
$\overline{V_c} \subset \Omega_c$ for any $c \in S$. 
\item
(Normalization) 
$\phi_c(b_c)=1$, where $b_c=g_c^{-1}(0) \in V_c$ is the pre-critical point.
\end{itemize}



\vskip 2mm


\noin
{\bf Step (M2): Construction of the Mandelbrot-like family 
$\boldsymbol{G} = \boldsymbol{G}_n$.} \ 

We shall construct a Mandelbrot-like family 
$$
\boldsymbol{G} = \boldsymbol{G}_n 
:= 
\{ G_c=G_{c, n} : V_c'=V'_{c,n} \to U_c 
   \}_{c \in W=W_n} 
$$
such that $V_c'=V'_{c,n} \subset U_c'$ and $W = W_n \subset S$ for every
sufficiently large $n \in \N$. Note that $G_c$ and $V_c'$ depend on $n$ 
but $U_c'$ and $U_c$ do not. 
Define
$$
  a_c := f_c^l(0)
$$
then as we mentioned above, note that $a_{c_1} = q_{c_1}$ and $a_c \ne q_c$
when $c \ne c_1$ is close to $c_1$.  Hence for such a parameter $c$,  
$a_c$ is repelled from $q_c$ by the dynamics of $f_c^k$. 
By taking a sufficiently small $S$, we may assume that 
$a_c = f_c^l(0) \in \Omega_c$
for each $c \in S$. Now we define
$$
W = W_n 
:= 
\{ c \in S \ | \ f_c^{ki}(a_c) \in \Omega_c \ \text{ for } \ i=1, \dots, n-1
\ \text{ and } \ 
   f_c^{kn}(a_c) (= f_c^{l+kn}(0)) \in V_c \}, 
$$
that is, we consider the parameter $c$ such that the orbit of $a_c$ by $f_c^k$
hits $V_c$.



\begin{lem}
By shrinking $U_c \equiv U_{c_1}$ slightly, the set $W=W_n$ is a non-empty
Jordan domain with $C^1$ boundary for every 
sufficiently large $n$. Moreover there exists an $s_n \in W_n$ such that 
$f_{s_n}^{kn}(a_{s_n}) = b_{s_n}$, which implies
$g_{s_n} \circ f_{s_n}^{l+kn}(0) = P_{s_n}^{(l+kn)p+N}(0) = 0$ and hence
$P_{s_n}$ has a superattracting periodic point. 
\label{W_n is not empty}
\end{lem}


\paragraph{\bf Proof.}
We work with the original $U_c \equiv U_{c_1}$ for the moment and then
shrink (i.e., change the definition of) $U_c \equiv U_{c_1}$ slightly
later to get the result.
In order to do this, 
we observe the dynamics near $q_c$ through the linearizing coordinate 
$\phi_c : \Omega_c \to \C$ of $q_c$. Let
$$
  \tau(c) := \phi_c(a_c), \quad \wt{V}_c := \phi_c(V_c)
$$
then $c \in W_n$ if and only if 
$$
  \mu(c)^n \tau(c) \in \wt{V}_c.
$$
Next recall that
$$
  g_c : V_c \to U_c \equiv U_{c_1}
$$
is an isomorphism by Lemma~\ref{def of U_{c_1} etc} (2) and let
$$
  u : U_c \equiv U_{c_1} \to \D, \quad u(0) = 0
$$
be a Riemann map of $U_c \equiv U_{c_1}$. Then
$$
  u \circ g_c : V_c \to \D, \quad u \circ g_c(b_c)=0
$$
is a Riemann map of $V_c$ and hence
$$
  u \circ g_c \circ \phi_c^{-1} : \wt{V}_c \to \D, \quad 
  u \circ g_c \circ \phi_c^{-1}(1) = 0
$$
is a Riemann map of $\wt{V}_c$. Take the inverse of this map and define
$$
  v(c, \zeta) := \phi_c \circ (u \circ g_c)^{-1}(\zeta), \quad 
  \zeta \in \D \ \text{ with } \ v(c, 0)=1.
$$
Now we solve the equation with respect to the variable $c$
\begin{equation}
  \mu(c)^n \tau(c) = v(c,\zeta)
\label{eqn for v(c,zeta)}
\end{equation}
for each fixed $\zeta \in \D$. 
Since $\mu(c)$ and $\tau(c)$ depend holomorphically on $c$, 
there exist $\alpha \in \N, \ M_0 \ne 0$ and $K_0 \ne 0$ such that
\begin{eqnarray*}
  \mu(c) 
& = &
(f_c^k)'(q_c) = \mu_{c_1} + M_0(c-c_1)^\alpha + O((c-c_1)^{\alpha+1}),  \\
\tau(c) 
& = &
K_0(c-c_1) + O((c-c_1)^2), \quad (c \to c_1).
\end{eqnarray*}
The fact that $K_0 \ne 0$ for the expansion of $\tau(c)$ follows from the 
result by Douady and Hubbard
(\cite[p.333, Lemma 1]{Douady-Hubbard 1985}. See also 
\cite[p.609, Lemma 5.4]{Tan Lei 1990}). 
Now we have
\begin{eqnarray*}
\mu(c)^n \tau(c)
& = &
\big( \mu_{c_1} + M_0(c-c_1)^\alpha + O((c-c_1)^{\alpha+1}) \big)^n 
\cdot \big( K_0(c-c_1) + O((c-c_1)^2) \big) \\
& = & 
\bigg\{ 
  \mu_{c_1}\Big(1 + \frac{M_0}{\mu_{c_1}}(c-c_1)^\alpha + O((c-c_1)^{\alpha+1}) \Big)
\bigg\}^n
\cdot K_0(c-c_1)(1+ O(c-c_1)) \\
& = & 
\mu_{c_1}^nK_0(c-c_1) 
\bigg( 1 + \frac{nM_0}{\mu_{c_1}}(c-c_1)^\alpha + O((c-c_1)^{\alpha+1}) \bigg)
\big( 1+ O(c-c_1) \big) \\
& = & 
\mu_{c_1}^nK_0(c-c_1) + h(c), 
\end{eqnarray*}
where 
$h(c)=O(n\mu_{c_1}^n(c-c_1)^2)$ when $\alpha=1$ and
$h(c)=O(\mu_{c_1}^n(c-c_1)^2)$ when $\alpha \geq 2$. So 
$$
  h(c)=O(n\mu_{c_1}^n(c-c_1)^2)
$$
is true in any case. 
Then the equation (\ref{eqn for v(c,zeta)}) can be rewritten as
\begin{equation}
F(c,\zeta) + G(c,\zeta) = 0,
\label{eqn F+G=0}
\end{equation}
where
$$
  F(c,\zeta) := \mu_{c_1}^nK_0(c-c_1) - v(c_1,\zeta), \quad
  G(c,\zeta) := h(c) - \big( v(c,\zeta)- v(c_1,\zeta) \big).
$$
The equation $F(c,\zeta)=0$ has a unique solution
$$
  c = c_n(\zeta) := c_1 + \frac{v(c_1,\zeta)}{\mu_{c_1}^nK_0}.
$$
Let
$$
  r_n(\zeta) := \bigg| \frac{v(c_1,\zeta)}{\mu_{c_1}^nK_0} \bigg| 
= O(\mu_{c_1}^{-n})
$$
and $\beta := 1/2$. Consider (\ref{eqn F+G=0}) in the disk 
$D(c_n(\zeta), r_n(\zeta)^{1+\beta})$. Since it is easy to see that
$$
  |F(c,\zeta)| = O(r_n(\zeta)^\beta) = O(\mu_{c_1}^{-\beta n}), \quad 
  |G(c,\zeta)| = O(n\mu_{c_1}^{-n})
$$
on the boundary $C := \partial D(c_n(\zeta), r_n(\zeta)^{1+\beta})$ of this disk, 
we have $|F(c,\zeta)| > |G(c,\zeta)|$ on $C$ for sufficiently large $n$. 
By Rouch\'e's theorem (\ref{eqn F+G=0}) has a unique solution 
$c = \check{c}_n(\zeta)$ in $D(c_n(\zeta), r_n(\zeta)^{1+\beta})$, so it 
satisfies
$$
  \check{c}_n(\zeta) 
= c_1 + \frac{v(c_1,\zeta)}{\mu_{c_1}^nK_0}
\big( 1 + O(\mu_{c_1}^{-\beta n}) \big).
$$
By using this solution, we can write
$$
  W_n = \{ \check{c}_n(\zeta) \in \C \ | \ \zeta \in \D \}.
$$

\vskip 2mm

\paragraph{\bf Claim.} \ 
{\it 
{\rm (1)} The map $\check{c}_n : \D \to W_n$ is holomorphic.

\noindent
{\rm (2)} For every $r \in (0,1)$, $\check{c}_n$ is univalent on 
$\overline{\D(r)}$ for every sufficiently large $n$.
}

\vskip 2mm


\paragraph{\bf Proof.}
(1) By the argument principle, for each $\zeta \in \D$ we have
$$
  \check{c}_n(\zeta)
= \frac{1}{2 \pi i}
  \int_C  H(c,\zeta) c \cdot dc, 
$$
where
$$
H(c,\zeta) 
:= 
\frac{\frac{\partial}{\partial c} \big(F(c,\zeta) + G(c,\zeta)\big)}
{F(c,\zeta)+G(c,\zeta)}, \quad
C = \{ z \ | \ |c-c_n(\zeta)|=r_n(t)^{1+\beta} \}
$$
Hence if $|\Delta \zeta| \ll 1$ and $c_n(\zeta+\Delta \zeta) \in \text{int}(C)$,
we have
$$
  \check{c}_n(\zeta+\Delta\zeta)
= \frac{1}{2 \pi i}
  \int_C  H(c,\zeta+\Delta\zeta) c \cdot dc.
$$
Then it follows that $H$ is holomorphic with respect to $\zeta$ and hence
$\check{c}_n(\zeta)$ is holomorphic in a neighborhood of $\zeta$. Thus
$\check{c}_n : \D \to W_n$ is holomorphic.


\noindent
(2) Let $v(\zeta) := v(c_1, \zeta)$ and
$$
  v_n(\zeta) := \mu_{c_1}^nK_0(\check{c}_n(\zeta)-c_1)
= v(c_1,\zeta)\big( 1 + O(\mu_{c_1}^{-\beta n}) \big)
= v(\zeta)\big( 1 + O(\mu_{c_1}^{-\beta n}) \big).
$$
Then for every $r \in (0,1)$, we have
$$
  v_n(\zeta) \to v(c_1,\zeta) = v(\zeta) \quad (n \to \infty) 
$$
uniformly on $\overline{\D(r)}$. In order to show the assertion, it is 
enough to show that $v_n$ is injective on $\overline{\D(r)}$ for every
sufficiently large $n$. Suppose that
$$
  v_n(\zeta_n) = v_n(\zeta_n')
$$
for some $\zeta_n, \ \zeta_n' \in \overline{\D(r)}, \ \zeta_n \ne \zeta_n'$,
where $n$ ranges over a subsequence $\{ n_k \}_{k=1}^\infty$. By taking a
further subsequence, we may assume that
$$
  \zeta_n \to \hat{\zeta}, \quad \zeta_n' \to \hat{\zeta}' \quad \text{ for }
  \ n=n_k, \ k \to \infty.
$$

\noindent
(a) When $\hat{\zeta} = \hat{\zeta}'$ : 
Let $A_0 := v'(\hat{\zeta}) \ne 0$, then there exists a $\delta > 0$
such that
$$
  |v'(\zeta)-A_0| \leq \frac{|A_0|}{4} 
\quad \text{on} \ \D(\hat{\zeta}, \delta).
$$
Since $v_n \to v$ uniformly, we have
$$
  |v_n'(\zeta)-A_0| \leq \frac{|A_0|}{2} 
\quad \text{on} \ \D(\hat{\zeta}, \delta) \ \text{for} \ n \gg 0.
$$
Hence for $\zeta, \ \zeta' \in \D(\hat{\zeta}, \delta)$ we have
\begin{eqnarray*}
\big| \{ v_n(\zeta) - v_n(\zeta') \} - A_0(\zeta-\zeta') \big|
& = &
\Bigg| \int_\zeta^{\zeta'} (v_n'(\zeta) - A_0) d\zeta \Bigg| \\
& \leq &
\frac{|A_0|}{2} |\zeta-\zeta'|
\end{eqnarray*}
It follows that
$$
   \frac{|A_0|}{2} |\zeta-\zeta'|
\leq |v_n(\zeta) - v_n(\zeta')| 
\leq 
\frac 32 |A_0| |\zeta-\zeta'|.
$$
In particular, $v_n$ is injective on $\D(\hat{\zeta}, \delta)$ for
$n \gg 0$. However, $\zeta_n, \ \zeta_n' \in \D(\hat{\zeta}, \delta)$ 
for $n \gg 0$ and this is a contradiction.


\noindent
(b) When $\hat{\zeta} \ne \hat{\zeta}'$ : 
We have
$$
  |v(\hat{\zeta}) - v(\hat{\zeta}')| 
\leq |v(\hat{\zeta}) - v(\zeta_n)| 
+|v(\zeta_n) - v(\zeta_n')| 
+|v(\zeta_n') - v(\hat{\zeta}')|.
$$
As $n=n_k \to \infty$, the first and the third terms of the right hand side 
of this inequality tend to $0$ by the continuity of $v$. Also by
using $v_n(\zeta_n) = v_n(\zeta_n')$,  we have
\begin{eqnarray*}
|v(\zeta_n) - v(\zeta_n')| 
& \leq &
|v(\zeta_n) - v_n(\zeta_n)| + |v_n(\zeta_n') - v(\zeta_n')| \\
& \leq &
2 \sup_{\zeta \in \overline{\D(r)}} |v(\zeta) - v_n(\zeta)| \to 0 \quad
(n=n_k \to \infty), 
\end{eqnarray*}
since $v_n \to v$ uniformly on $\overline{\D(r)}$. This implies 
$v(\hat{\zeta}) = v(\hat{\zeta}')$, but this contradicts the univalence of
$v$.
\QED (Claim)


\medskip

\noindent
By shrinking $U_c \equiv U_{c_1}$ slightly and using the Riemann map $u$ 
of the original $U_c \equiv U_{c_1}$, the boundary of the new 
$U_c \equiv U_{c_1}$ is parametrized as $u^{-1}(\gamma(t))$, where 
$\gamma(t) = re^{2\pi it} \subset \D \ (t \in [0,1])$ and $r \in (0,1)$ 
is close to $1$. Then 
$\partial \wt{V}_c$ is parameterized as $v(c, \gamma(t))$ and hence 
$\partial W_n$ (for the new $W_n$) is parameterized as 
$\check{c}_n(\gamma(t))$ by using the solution $\check{c}_n(\zeta)$ for 
the equation (\ref{eqn for v(c,zeta)}). Clearly this is a $C^1$ Jordan curve
and $W_n$ is the image of $\D(r)$ by $\check{c}_n(\zeta)$. This shows that 
$W_n$ is a non-empty Jordan domain with $C^1$ boundary. 
In particular, let $s_n := \check{c}_n(0)$ then this satisfies
$\mu(s_n)^n\tau(s_n) = 1$. This means that
$f_{s_n}^{kn}(a_{s_n}) = b_{s_n}$,
which implies $g_{s_n} \circ f_{s_n}^{l+kn}(0) = P_{s_n}^{(l+kn)p+N}(0)= 0$. 
Hence $P_{s_n}$ has a superattracting periodic point. 
This completes the proof of Lemma \ref{W_n is not empty}.
\QED ~{\small (Lemma \ref{W_n is not empty})}
\\

We call 
$s_n \in W_n$ the {\it center} of $W_n$. Now let $L = L_n := l+kn$ and
$V_c'=V'_{c,n}$ be the Jordan domain bounded by the component of $f_c^{-L}(V_c)$ 
containing $0$ and define
$$
  G_c = G_{c, n} := g_c \circ f_c^L : V_c' \to U_c
  \ \  \text{and} \ \ 
 \boldsymbol{G} = \boldsymbol{G}_n 
 := 
\{ G_c \}_{c \in W_n}, 
$$
where $W_n = \{ c \in S \ | \ f_c^L(0) \in V_c \}$.
See Figure \ref{UV}. 


\medskip

\noin
{\bf Step (M3): Proof for $\boldsymbol{G} = \boldsymbol{G}_n$ 
being a Mandelbrot-like family.} 


The map
$f_c^L : V_c' \to V_c$ is a branched covering of degree 2 and
$g_c : V_c \to U_c$ is a holomorphic isomorphism. Hence 
$G_c := g_c \circ f_c^L : V_c' \to U_c $ is a quadratic-like map.


Next we construct a tubing 
$\Theta = \Theta_n = \{ \Theta_c \}_{c \in W_n}$ for $\boldsymbol{G}_n$
as follows: For $s_n \in W_n$,  since $f_{s_n}^L(0) \in V_{s_n}$ and 
$f_{s_n}^j(V_{s_n}) \subset U_{s_n} \smallsetminus \overline{U_{s_n}'}$, 
from Lemma \ref{def of U_{c_1} etc}, we have
$f_{s_n}^{L+j}(0) \notin U_{s_n}'$. It follows that $J(f_{s_n})$ is a
Cantor set which is quasiconformally homeomorphic to a quadratic Cantor
Julia set $J_{c_0+\eta_n}$ for some $\eta=\eta_n$ with 
$c_0+\eta_n \notin M$. By continuity of the straightening of $f_c$ for 
$c \in \Lambda$, we have $|\eta_n| < \vep$ for sufficiently large $n$.
We denote the homeomorphism from $J(f_{s_n})$ to $J_{c_0+\eta_n}$ by 
$\Psi_{s_n} : J(f_{s_n}) \to J_{c_0+\eta_n}$, which
is induced by the straightening theorem.
Take an $R > 1$ and let $\rho' := R^{-1/2}$ and $\rho := R^{1/2}$ such
that $J_{c_0+\eta_n} \subset A(\rho', \rho)$. 
Define the rescaled Julia set
$$
  \Gamma := \Gamma_0(c_0+\eta_n) 
= \Gamma_0(c_0+\eta_n)_{\rho', \rho}
:= J_{c_0+\eta_n} \times R^{3/2}
\subset A(R, R^2) 
$$
and a homeomorphism
$$
\Theta_n^0 : \overline{A(R, R^2)} \to 
              \overline{U}_{s_n} \smallsetminus V_{s_n}'
$$ 
for $s_n$ appropriately so that 
\begin{itemize}
\setlength{\itemsep}{0.5mm} %
\item
$\Theta_n^0$ is quasiconformal,

\item
$\Theta_n^0(Z^2) = G_{s_n}(\Theta_n^0(Z))$ for $|Z| = R$, 

\item
$\Theta_n^0(Z) = \Psi_{s_n}^{-1}(R^{- 3/2}Z)$ 
for $Z \in \Gamma_0(c_0+\eta_n)$,

\item
$\Theta_n^0(\Gamma_0(c_0+\eta_n)) = J(f_{s_n})$. 
\end{itemize}

\noin 
Then the Julia set
$J(f_{c}) \subset U_c' \smallsetminus \overline{V_c'}$ is a Cantor set
for every $c \in W_n$ for the same reason for $J(f_{s_n})$ and this, 
as well as $\partial U_c$ and $\partial V_c'$ undergo holomorphic motion 
(see \cite[p.229]{Shishikura 1998}).
By S{\l}odkowski's theorem (\cite{Slodkowski 1991}) there exists a 
holomorphic motion $\iota_c$ on $\C$
which induces these motions. Finally 
define $\Theta_c := \iota_c \circ \Theta_n^0$ , then 
$\Theta = \Theta_n := \{ \Theta_c \}_{c \in W_n}$ is a
tubing for $\boldsymbol{G}_n$. 



Now we have to check that
$\boldsymbol{G}_n$ with $\Theta_n$
satisfies the conditions (1)--(8) for a Mandelbrot-like family.
The condition (1) is already shown in Lemma~\ref{W_n is not empty}. 
It is easy to check the conditions (2)--(7). 
Finally the one turn condition (8) is proved as follows: 
Note that $\check{c}_n(\gamma(t))$ satisfies
$$
\mu (\check{c}_n(\gamma(t)))^n 
\tau(\check{c}_n(\gamma(t)))
= v(\check{c}_n(\gamma(t)),\gamma(t)).
$$
When $c$ ranges over $\partial W_n$ making one turn, the variable $t$ for
both sides varies from $t=0$ to $t=1$. 
Since $v(\check{c}_n(\gamma(t)),\gamma(t))$ 
is very close to $v(c_1, \gamma(t))$, which is a parameterization of
$\partial \wt{V}_{c_1}$ for sufficiently large $n$, 
$v(\check{c}_n(\gamma(t)),\gamma(t))$ and hence
$\mu(\check{c}_n(\gamma(t)))^n \tau(\check{c}_n(\gamma(t)))$ 
makes one turn in a very thin tubular neighborhood of 
$\partial \wt{V}_{c_1}$ as $t$ moves from 0 to 1. 
This implies that $f_c^{kn}(a_c) = f_c^{l+kn}(0) = f_c^L(0)$ 
makes one turn in a very thin tubular neighborhood of $\partial V_{c_1}$. 
Hence $G_c(0)-0 = G(0) = g_c \circ f_c^{l+kn}(0) = g_c \circ f_c^L(0)$ makes
one turn in a very thin tubular neighborhood of $\partial U_{c_1}$. 
In particular this shows that $G_c(0)-0$ makes one turn around $0 \in U_{c_1}$.



\vskip 2mm


\noin
{\bf Step (M4): End of the proof of Theorem A for the Misiurewicz case.}

For every $\vep > 0$ and $\vep' > 0$, take a sufficiently large $n \in \N$ such
that $c_0 + \eta = c_0 + \eta_n \in D(c_0, \vep) \smallsetminus M$. We conclude that
the model ${\mathcal M}(c_0+\eta)$ appears quasiconformally in $M$ in the neighborhood
$D(c_1, \vep') = D(s_0 \perp c_0, \vep')$ of $c_1 = s_0 \perp c_0$ by applying 
Proposition \ref{D-BDS Proposition} to the Mandelbrot-like family 
$\boldsymbol{G}=\boldsymbol{G_n}$ with $\Theta = \Theta_n$. Indeed from
Proposition \ref{D-BDS Proposition}, the set
$$
{\mathcal N}
:=
M_{\boldsymbol G}
\cup 
\{ c \ | \ 
     G_c^k(0) \in \Theta_c(\Gamma_0(c_0+\eta)) \quad \text{for some} \ k \in \N \}
$$
is the image of ${\mathcal M}(c_0+\eta)$ by the quasiconformal map 
$\chi_{\Theta}^{-1} = \chi_{\Theta_n}^{-1}$, where $M_{\boldsymbol G}$ is the
connectedness locus of ${\boldsymbol G}$. On the other hand, for 
$c \in M_{\boldsymbol G}$, the orbit of the critical point $0$ by 
$G_c = g_c \circ f_c^{l+kn} = P_c^{Lp+N}$ is bounded, which
implies that the orbit of $0$ by $P_c$ is also bounded and hence $c \in M$.
If $G_c^k(0) \in \Theta_c(\Gamma_0(c_0+\eta))$ for some $k \in \N$, then
$c \in M$ as well. So the set $\mathcal N$ is a subset of $M$.
In particular, since a conformal image $\Phi_M^{-1}(J_{c_0+\eta} \times R^{3/2})$
of $J_{c_0+\eta}$ is a subset of ${\mathcal M}(c_0+\eta)$, we conclude that 
$J_{c_0+\eta}$ appears 
quasiconformally in $M$. This completes the proof of Theorem A for the 
Misiurewicz case.
\QED 
\medskip

\begin{rem*}
In \cite{Douady 2000}, there is no proof for $\partial W_n$ being a $C^1$ 
Jordan curve and also the proof for the one turn condition (8) is intuitive. 
\end{rem*}









\section{Proof of Theorem A for the parabolic case}
As in the previous section, 
let $M_{s_0}$ be the small Mandelbrot set with center $s_0 \neq 0$
such that $0$ is a periodic point of period $p \ge 2$,
and let $\Lambda=\Lambda_{s_0}$ be the simply connected domain 
where the family 
$
\{
f_c := P_c^p|_{\widetilde{U}_c'} : \widetilde{U}_c' \to \widetilde{U}_c
\}_{c \, \in \, \Lambda}
$
of quadratic-like maps is defined.
In this section let $c_0 \in \partial M$ be any parabolic parameter 
and $c_1:=s_0 \perp c_0 \in M_{s_0}$.

A simple way to show Theorem A for the parabolic case is the following:
since the Misiurewicz parameters are dense in the boundary of the Mandelbrot set,
we can find a Misiurewicz parameter $c_0'$ that is arbitrarily close to the parabolic parameter $c_0$. By continuity of the tuning map 
$c \mapsto s_0 \perp c$, we may apply Theorem A for the Misiurewicz case.

The aim of this section is to present a direct proof that is 
independent of the Misiurewicz case,
based on Douady's original proof for the cauliflower. 
(Hence by the same logic the parabolic case implies the Misiurewicz case.) 
It breaks into four steps (P1) to (P4),
that are parallel with (M1) to (M4) of the Misiurewicz case.


\medskip


\noin
{\bf Step (P1): Definitions of $U_c$, $U_c'$, and $V_c$.}

\indent
For a technical reason, 
{\it 
we first assume that 
the parabolic parameter $c_1=s_0 \perp c_0 \in M_{s_0}$ 
belongs to $\Lambda$}.
Note that this assumption only excludes the case 
where $M_{s_0}$ is a satellite small Mandelbrot set and $c_0=1/4$. This case will be discussed at the end of Step (P1).

Under this assumption, we shall construct a 
family $\{f_c:U_c' \to U_c\}_{c\, \in \,S}$ 
of quadratic-like maps and a family of isomorphisms 
$\{g_c:V_c \to U_c\}_{c\, \in \,S}$ 
as in Step (M1) of the Misiurewicz case;
however, 
we take a ``sector" $S$ attached to $c_1 \in \partial M_{s_0}$
rather than a neighborhood of $c_1$.


\paragraph{\bf A pair of petals and the Fatou coordinates.}
We start with the dynamics of $P_{c_1}$
including that of $f_{c_1}= P_{c_1}^p|_{\widetilde{U}_{c_1}'}$.
Let $\Delta$ be the Fatou component in $K(f_{c_1})$
containing $0$. 
The boundary $\partial \Delta$ contains
a unique parabolic periodic point $q_{c_1}$ of $f_{c_1}$ 
(resp. $P_{c_1}$)
of period $k$ 
(resp. $kp$).
The multiplier $(f_{c_1}^k)'(q_{c_1})$ is of the form
$$
\mu_{c_1}:=e^{2 \pi i \nu'/\nu},
$$
where $\nu'$ and $\nu$ are coprime integers.
Since $P_{c_1}$ has only one critical point,
$q_{c_1}$ has $\nu$-petals.
That is, by choosing an appropriate local coordinate $w=\psi_{c_1}(z)$ 
near $q_{c_1}$ with $\psi_{c_1}(q_{c_1})=0$,
we have 
$$
\psi_{c_1} \circ f_{c_1}^{k\nu} \circ \psi_{c_1}^{-1}(w)=w(1 + w^{\nu}+O(w^{2\nu})).
$$
See \cite[Proof of Theorem 6.5.7]{Beardon 1991} 
and \cite[Appendix A.2]{Kawahira 2009}.
The set of $w$'s with $\arg w^\nu=0$ (resp. $\arg w^\nu =\pi$) 
determines {\it the repelling} 
(resp. {\it attracting})  {\it directions} of this parabolic point. 
Note that the Fatou component $\Delta$ is invariant under 
$f_{c_1}^{k\nu}$,
and it contains a unique attracting direction.
In particular, the sequence $f_{c_1}^{k\nu m}(0)~(m \in \N)$ 
converges to $q_{c_1}$ within $\Delta$ 
tangentially to the attracting direction.

Set 
\begin{align*}
\Omega_{c_1}^+&:=
\skakko{
z=\psi_{c_1}^{-1}(w) 
\in \C \st -\frac{2\pi}{3\nu} \le \arg w \le \frac{2\pi}{3\nu},~0 < |w|<r
},\\
\Omega_{c_1}^-&:=
\skakko{
z=\psi_{c_1}^{-1}(w) \in \C \st 
-\frac{5\pi}{3\nu} \le \arg w \le -\frac{\pi}{3\nu}
,~0 < |w|<r}
\end{align*}
for some sufficiently small $r>0$ such that 
$\Omega_{c_1}^+$ and $\Omega_{c_1}^-$ are a pair of repelling and attracting petals 
with $\Omega_{c_1}^+ \cap \Omega_{c_1}^- \neq \emptyset$.
(See Figure \ref{fig_petals}.)
By multiplying a $\nu$-th root of unity to the local coordinate $w=\psi_{c_1}(z)$ if necessary, 
we may assume that the attracting petal $\Omega_{c_1}^-$ is contained in $\Delta$.

%Figure 9.
\begin{figure}[htbp]
\begin{center}
%\includegraphics[width=.7\textwidth]{fig_petals.eps}
\includegraphics[width=.7\textwidth]{fig_petals.png}
\end{center}
\caption{\small We choose a pair of repelling and attracting petals.
Their intersection has two components when $\nu=1$.}
\label{fig_petals}
\end{figure}



For the coordinate $w=\psi_{c_1}(z)$, 
we consider an additional coordinate change 
$w \mapsto W=-1/(\nu w^\nu)$. In this $W$-coordinate,
the action of $f_{c_1}^{k\nu}$ on each petal is 
$$
W \mapsto W+1+ O(W^{-1}).
$$
By taking a smaller $r$ if necessary,
there exist 
conformal mappings 
$\phi_{c_1}^ + :\Omega_{c_1}^ +  \to \C$
and 
$\phi_{c_1}^-:\Omega_{c_1}^- \to \C$
such that $\phi_{c_1}^\pm (f_{c_1}^{k\nu}(z)) =\phi_{c_1}^\pm(z)+1$
which are unique up to adding constants.
(We will normalize them later.) 
We call $\phi_{c_1}^\pm$ the {\it Fatou coordinates}.


Now we are ready to state the parabolic counterpart of Lemma 4.1:

\vskip 2mm

\paragraph{\bf Lemma 4.1'.}
{\it
There exist Jordan domains $U_{c_1}$, $U_{c_1}'$ and $V_{c_1}$ with 
$C^1$ boundaries and 
integers $N, \ j \in \N$ which satisfy 
the following:
\begin{itemize}
\item[\rm (1)]
$0 \in U_{c_1}' \subset \widetilde{U}_{c_1}'$ and 
$f_{c_1}: U_{c_1}' \to U_{c_1}$ is a 
quadratic-like map.

\item[\rm (2)]
$g_{c_1} := P_{c_1}^N|_{V_{c_1}} : V_{c_1} \to U_{c_1}$ is an isomorphism
and $\overline{f_{c_1}^j(V_{c_1})} 
\subset U_{c_1} \smallsetminus \overline{U_{c_1}'}$.

\item[\rm (3)]
$\overline{V_{c_1}} \subset \Omega_{c_1}^+$. Also we can take $V_{c_1}$ arbitrarily
close to $q_{c_1} \in \partial \Omega_{c_1}^+$.
\end{itemize}
\label{def of U_{c_1} etc for parabolic case}
}

\vskip 2mm



\noin
The proof is the same as that of Lemma \ref{def of U_{c_1} etc}. 
See Figure \ref{fig_V_para}.


%Figure 10.
\begin{figure}[htbp]
\begin{center}
%\includegraphics[width=.31\textwidth]{fig_V_para.eps}
\includegraphics[width=.31\textwidth]{fig_V_para.png}
\end{center}
\caption{\small We choose $V_{c_1}$ in the repelling petal $\Omega_{c_1}^+$.}
\label{fig_V_para}
\end{figure}




For $ \vep'>0$ given in the statement of Theorem A,
we take a small neighborhood $S'$ of $c_1$ 
contained in $D(c_1, \vep') \cap \Lambda$ and 
let $U_c := U_{c_1}$ for each $c \in S'$.
As in Step (M1) of the Misiurewicz case,
by taking a smaller $U_{c_1}$ in the previous lemma
and a smaller $S'$ if necessary,
we obtain a quadratic-like map $f_c:U_c' \to U_c$
with $U_c':=f_c^{-1}(U_c) \subset \widetilde{U}_{c}'$,
a component $V_c$ of $P_c^{-N}(U_c)$ 
 close to $V_{c_1}$ satisfying 
 $\overline{f_c^j(V_c)} \subset U_c \smallsetminus \overline{U_c'}$,
and an isomorphism $g_c : V_c \to U_c \,(\equiv U_{c_1})$ 
for each $c \in S'$.
See 
Figure \ref{UV} again. 
We also define the {\it pre-critical point} $b_c$
by $b_c:=g_c^{-1}(0) \in V_c$.


\paragraph{\bf Fatou coordinates.}
Although the families $\{f_c:U_c' \to U_c\}_{c \, \in \, S'}$
and 
$\{g_c:V_c \to U_c\}_{c \, \in \, S'}$ 
are defined over a neighborhood $S'$ of $c_1$,
we shall restrict them to 
a ``sector" $S \subset S'$ attached to $c_1$
to take advantage of (perturbed) {\it Fatou coordinates} 
and {\it lifted phase} in the next step. 


When $\nu=1$ (equivalently, $\mu_{c_1}=1$), 
the parabolic fixed point $q_{c_1}$ of $f_{c_1}^k$ 
splits into two distinct fixed points of 
$f_c^k$ for each $c \neq c_1$. 
To describe the bifurcation, 
it is convenient to use a parameter $u$ that satisfies $c=c_1+u^2$. 
It is known that there are two holomorphic functions 
$q_+(u)$ and $q_-(u)$ defined near $0$ such that 
$f_{c_1+u^2}^k(q_\pm(u))=q_\pm(u)$; 
$q_{c_1}=q_+(0)=q_-(0)$; and their multipliers satisfy
$$
\mu_\pm(u) := (f_{c_1+u^2}^k)'(q_\pm(u))=1 \pm A_0 u + O(u^2)
$$
for some $A_0 \neq 0$. 
(See \cite[Expos\'e XI]{DH Orsay}, \cite[Theorem 1.1 (c)]{Tan Lei 2000},
 or the primitive case of \cite[Lemma 4.2]{Milnor 2000}.)
Note that the maps $u \mapsto \mu_\pm(u)$ are univalent near $u=0$
and hence locally invertible. 

For $r > 0$ we define 
a sector $S_\mu(r) \subset \C$ attached to $1$ by
$$
S_\mu(r):=\braces{\mu \in \C \st 
0<|\mu-1|<r
~~~
\text{and}~~~
\abs{\arg (\mu -1)-\frac{\pi}{2}}<\frac{\pi}{8}
}.
$$
We choose a sufficiently small $r_0>0$ such that 
the set
$$
S:=\braces{c=c_1+u^2  \st \mu_+(u) \in S_\mu(r_0)}
$$
is contained in $S' \subset \Lambda$ 
and that the correspondence between $\mu=\mu_+(u) \in S_\mu(r_0)$ 
and $c=c_1+u^2 \in S$ is one-to-one.
See Figure \ref{fig_sectors} (left).
We may regard the parameter $u$ that 
mediates this one-to-one correspondence 
as a holomorphic branch of $\sqrt{c-c_1}$ over $S$.
We may also regard 
\begin{align}
q_c&:=q_+(u)=q_+(\sqrt{c-c_1})\qquad \text{and} \notag\\
\mu_c&:=\mu_+(u)=1+A_0 \sqrt{c-c_1} + O(c-c_1) \label{eq_A_0}
\end{align}
as a fixed point of $f_c^k$ and its multiplier that depend
holomorphically on $c \in S$.

When $\nu\ge 2$ (equivalently, $\mu_{c_1} \neq 1$),
the parabolic fixed point $q_{c_1}$ of $f_{c_1}^k$ 
splits into one fixed point $q_c$ and 
a cycle of period $\nu$ of $f_c^k$ for each $c \neq c_1$. 
By the implicit function theorem,
$q_c$ and its multiplier $\mu_c:=(f_c^k)'(q_c)$ depend holomorphically on $c$ near $c_1$,
and it is known that 
\begin{equation}\label{eq_B_0}
\mu_c=\mu_{c_1} \, \paren{1+B_0 (c-c_1)+ O((c-c_1)^2)}
\end{equation}
for some constant $B_0 \neq 0$.
(See \cite[Expos\'e XI]{DH Orsay}, \cite[Theorem A.1 (c)]{Tan Lei 2000},
or the satellite case of \cite[Lemma 4.2]{Milnor 2000}.)
Note that the map $c \mapsto \mu_c$ is univalent near $c_1$
and hence locally invertible. 

We choose a sufficiently small $r_0>0$ such that the set
$$
S:=\braces{c \in \C \st \frac{\mu_c}{\mu_{c_1}} \in S_\mu(r_0)}
$$
is contained in $S' \subset \Lambda$ 
and that 
the correspondence between $\mu=\mu_c \in  \mu_{c_1} \times S_\mu(r_0)$ 
and $c \in S$ is one-to-one.
See Figure \ref{fig_sectors} (right).
We call $S$ a {\it sector} attached to $c_1 \in \partial M_{s_0}$.

%Figure 11.
\begin{figure}[htbp]
\begin{center}
%\includegraphics[width=.75\textwidth]{fig_sectors.eps}
\includegraphics[width=.75\textwidth]{fig_sectors.png}
\end{center}
\caption{\small The sector $S$ for $\nu=1$ (left) and $\nu \ge 2$ (right).}
\label{fig_sectors}
\end{figure}



By taking a sufficiently small $r_0$, 
we may assume that for each $c \in S$
there exists a holomorphic local coordinate $w=\psi_c(z)$ 
near $q_{c}$ with $\psi_c(q_{c})=0$ such that
$$
\psi_c \circ f_{c}^{k\nu} \circ \psi_c^{-1}(w)
=\mu_c^\nu w\,(1 + w^{\nu}+O(w^{2\nu})),
$$
where $\mu_c^\nu \to 1$ and $\psi_c \to \psi$ 
uniformly as $c \in S$ tends to $c_1$.
See \cite[Appendix A.2]{Kawahira 2009}. 
By a further $\nu$-fold coordinate change $W=-\mu_c^{\nu^2}/(\nu w^\nu)$,
the action of $f_c^{k \nu} $ is 
$$
W \mapsto \mu_c^{-\nu^2} \,W +1+O(W^{-1}),
$$ 
where $W=\infty$ is a fixed point 
with multiplier $\mu_c^{\nu^2}$ 
that corresponds to the fixed point $q_c$ of $f_c^{k\nu}$.
There is another fixed point of the form $W=1/(1-\mu_c^{-\nu^2})+O(1)$
with multiplier close to $\mu_c^{-\nu^2}$
on each branch of the $\nu$-fold coordinate.
Note that  
\begin{align*}
\mu_c^{\pm \nu^2}&=
1 \pm A_0\sqrt{c-c_1}+O(c-c_1)
\qquad \text{or}\\
\mu_c^{\pm \nu^2}&=1 \pm \nu^2 B_0(c-c_1)+O((c-c_1)^2)
\end{align*}
according to $\nu=1$ or $\nu \ge 2$ by (\ref{eq_A_0}) and (\ref{eq_B_0}).

%Figure 12.
\begin{figure}[htbp]
\begin{center}
%\includegraphics[width=.44\textwidth]{fig_implosion.eps}
\includegraphics[width=.44\textwidth]{fig_implosion.png}
\end{center}
\caption{\small A typical behavior of the critical orbit 
by $f_c^{k \nu }$ near $q_c$ for $\nu =3$.}
\label{fig_implosion}
\end{figure}

It is known that for each $c$ in $S \cup \{c_1\}$ 
(by taking a smaller $r_0$ if necessary),
there exist unique (perturbed) {\it Fatou coordinates} 
$\phi_c^ + :\Omega_c^ +  \to \C$ 
and
$\phi_c^-:\Omega_c^- \to \C$
satisfying the following conditions
(\cite{Lavaurs 1989}, \cite{DSZ 1997} and \cite[Proposition A.2.1]{Shishikura 1998}):

\begin{itemize}
\item
For any $c \in S$,
both $\partial \Omega_c^+$ and $\partial \Omega_c^-$ contain two fixed points $q_c$ and $q_c'$ 
of $f_c^{k\nu}$ that converge to $q_{c_1}$ as $c \in S$ tends to  $c_1$.
\item 
For any $c \in S \cup \{c_1\}$, 
$\phi_c^\pm$ is a conformal map from a domain $\Omega_c^\pm$ onto the image in $\C$ that satisfies
$\phi_c^\pm(f_c^{k\nu}(z))=\phi_c^\pm(z)+1$ 
if both $z$ and $f_c^{k\nu}(z)$ are contained in $\Omega_c^\pm$.
\item
(Holomorphic dependence) 
Every compact set $E$ in $\Omega_{c_1}^\pm$ 
is contained in $\Omega_c^\pm$ for $c \in S$ sufficiently close to $c_1$,
and $\phi_c^\pm(z)$ depends holomorphically on $c \in S$ near $c_1$ for each $z \in \Omega_{c_1}^\pm$.
\item
$\overline{V_c} \subset \Omega_c^+$ for any $c \in S \cup \{c_1\}$. 
\item
(Normalization) 
\begin{itemize}
\item
There exists an $m \in \N$
such that $f_{c}^{k\nu m}(0) \in \Omega_c^-$ 
for any $c \in S \cup \{c_1\}$,
and $\phi_c^-$ is normalized such that $\phi_c^-(f_{c}^{k\nu m}(0))=m$.
\item
$\phi_c^+(b_c)=0$, where $b_c=g_c^{-1}(0) \in V_c$ is the pre-critical point.
\end{itemize}
\end{itemize}
We can arrange the domains $\Omega_c^\pm$
such that $\Omega_c^+=\Omega_c^-=:\Omega_c^\ast$ for each $c \in S$
(Figure \ref{fig_implosion}).
Hence for each $z \in \Omega_c^\ast$, 
$$
\tau(c):=\phi_c^+(z)-\phi_c^-(z) \in \C
$$
is defined and independent of $z$. 
The function $\tau:S \to \C$
is called the {\it lifted phase}.
Note that the value $\tau(c)$ is determined 
by the unique normalized Fatou coordinates
associated with the analytic germ $f_c^{k\nu}$,
and it does not depend on the choice of the parametrization
\footnote{
In \cite{Douady 2000}, the lifted phase 
for $\nu=1$ is described 
in terms of the normalized germ
$f_\mu (z)=z+z^2+\mu+\cdots~(\mu \to 0)$.
In this case the multiplier for two fixed points of $f_\mu$ are 
$1 \pm 2 \sqrt{\mu}i\,(1+O(\mu))$ and $\tau=-\pi/\sqrt{\mu}+O(1)$
as $\mu \to 0$. 
In \cite{DSZ 1997}, they use $\al=(\nu/2\pi i)\log (\mu_c/\mu_{c_1}-1)$
(so that $\mu_c=\exp(2 \pi i (\nu'+\al)/\nu)$)
to parametrize the germs. 
Any parameterizations are analytically equivalent and they determine
the same value $\tau$.
}.
It is known that if $\nu=1$, then
$$
\tau(c)= -\frac{2\pi i}{A_0\sqrt{c-c_1}}+O(1)
$$
as $c \in S$ tends to $c_1$, where $A_0$ is given in (\ref{eq_A_0}).
Similarly if $\nu \ge 2$, we obtain
$$
\tau(c)= -\frac{2 \pi i}{\nu^2 B_0(c-c_1)}+O(1)
$$
as $c \in S$ tends to $c_1$,
where $B_0$ is given in (\ref{eq_B_0}). 
In both cases it can be also shown that $\tau(c)$ is univalent on $S$ if 
$S$ is sufficiently small.



\paragraph{\bf Satellite roots.}
Now we deal with the remaining case 
where $M_{s_0}$ is a satellite small Mandelbrot set 
with renormalization period $p$ and $c_0 =1/4$. 
Let $\Lambda$ be the simply connected domain associated with $M_{s_0}$. 
Since $M_{s_0}$ is satellite, 
the quadratic-like family 
$
\{ 
f_c = P_c^p|_{\widetilde{U}_c'} : \widetilde{U}_c' \to \widetilde{U}_c
\}_{c \, \in \,\Lambda}
$
excludes the parameter $c_1 =s_0 \perp (1/4) \notin \Lambda$.
However, by slightly modifying the notion of quadratic-like map 
at this parameter,
one can establish a version of Lemma 4.1' as follows.


Let $q_{c_1}$ be the fixed point of $P_{c_1}^{p}$ with multiplier $1$, 
whose petal number is $\nu \ge 2$. 
(Then the period of $q_{c_1}$ in the dynamics of $P_{c_1}$ 
is $p'=p/\nu$.)
Let $\Omega_{c_1}^+$ be the repelling petal attached to $q_{c_1}$ as in Figure \ref{fig_petals} (right). 
Then we have:

\paragraph{\bf Lemma 4.1' for the Satellite Roots} 
{\it 
There exist Jordan domains $U_{c_1}$, $U_{c_1}'$ and $V_{c_1}$ with 
$C^1$ boundaries and 
integers $N, \ j \in \N$ which satisfy 
the following:
\begin{itemize}
\item[\rm (1)]
$0 \in {U}_{c_1}' \subset {U}_{c_1}$,
$\partial {U}_{c_1}' \cap \partial {U}_{c_1}=\skakko{q_{c_1}}$,
and $f_{c_1}:{U}_{c_1}' \to {U}_{c_1}$
is a proper branched covering of degree two.
\item[\rm (2)]
$g_{c_1} := P_{c_1}^N|_{V_{c_1}} : V_{c_1} \to U_{c_1}$ is an isomorphism
and $\overline{f_{c_1}^j(V_{c_1})} 
\subset U_{c_1} \smallsetminus \overline{U_{c_1}'}$.
\item[\rm (3)]
$\overline{V_{c_1}} \subset \Omega_{c_1}^+$. Also we can take $V_{c_1}$ arbitrarily
close to $q_{c_1} \in \partial \Omega_{c_1}^+$.
\end{itemize}
}
The proof is analogous to those of Lemmas 4.1 and 4.1'. 
However, to obtain (1) we need the idea of \cite[\S 3]{Haissinsky 2000} 
which is originally used in the construction of $\Lambda$.

For each $c \in \Lambda$ there exists a repelling fixed point $q_c$ of $P_{c}^{p}$ that depends holomorphically on $c$ and 
$q_c \to q_{c_1}$ as $c \in \Lambda$ tends to $c_1$.
We define a Jordan domain $U_c$ by adding 
a small disk centered at $q_c$ to $U_{c_1}$.
(We slightly modify $U_c$ such that 
$\partial U_c$ is a $C^1$ Jordan curve 
that moves holomorphically with respect to $c$.)
Then for any $c \in \Lambda$ sufficiently close to $c_1$,
we have a quadratic-like map $f_c:U_c' \to U_c$
and an isomorphism $g_c:V_c \to U_c$
where $U_c'$ is a connected component of $P_c^{-p}(U_c)$ 
with $\overline{U_c'} \subset U_c$
and $V_c$ is a connected component of $P_c^{-N}(U_c)$
that is close to $V_{c_1}$ and
$\overline{f_{c}^j(V_{c})} \subset U_{c} \smallsetminus \overline{U_{c}'}$. 
Since $q_{c_1}$ is parabolic with $\nu \ge 2$ petals,
we take the sector $S$ attached to $c_1$ as in Figure \ref{fig_sectors} (right).
By taking $S$ with sufficiently small radius, we obtain
the families $\{f_c:U_c' \to U_c\}_{c \,\in \,S \,\cap \,\Lambda}$ 
and $\{g_c:V_c \to U_c\}_{c \,\in \,S \,\cap\, \Lambda}$
over the set $S \cap \Lambda$ together with Fatou coordinates 
and lifted phase. 

\paragraph{\bf Remark.}
In the satellite root case we have $S \not\subset \Lambda$,
but $S \subset \Lambda$ for the other cases. 
Hence we regard $\{f_c\}$ and $\{g_c\}$ as families defined over $S \cap \Lambda$ for all cases.





\medskip






\noin
{\bf Step (P2): Construction of the Mandelbrot-like family 
$\boldsymbol{G} = \boldsymbol{G}_n$.} \ 

We shall construct a Mandelbrot-like family 
$$
\boldsymbol{G} = \boldsymbol{G}_n 
:= 
\{ G_c = G_{c,n} : V_c'=V'_{c,n} \to U_c 
   \}_{c \,\in\, W_n} 
$$
such that $V_c'=V'_{c,n} \subset U_c'$ and $W = W_n \subset S \cap \Lambda$ for 
every sufficiently large $n \in \N$. 
Recall that 
there exists an $m \in \N$
such that $f_{c}^{k\nu m}(0) \in \Omega_c^* ( = \Omega_c^-)$ 
for any $c \in S \cup \{c_1\}$.
Now for each $n \ge m$, define
$$
W = W_n 
:= 
\{ c \in S \ | \ f_c^{k\nu i}(0) \in \Omega_c^* \ \text{ for } \ i=m, \dots, n-1
\ \text{ and } \ 
  f_c^{k\nu n}(0) \in V_c \},
$$
that is, we consider the parameter $c$ such that the orbit of $0$
by $f_c^{k\nu}$ hits $V_c$.

\medskip



\noin
{\bf Lemma \ref{W_n is not empty}'.} \
{\it
By shrinking $U_c \equiv U_{c_1}$ slightly, the set $W=W_n$ is a non-empty
Jordan domain with $C^1$ boundary for every 
sufficiently large $n$. Moreover there 
exists an $s_n \in W_n$ such that $f_{s_n}^{k\nu n}(0) = b_{s_n}$,
which implies $g_{s_n} \circ f_{s_n}^{k\nu n}(0) = P_{s_n}^{k\nu np+N}(0) = 0$ 
and hence $P_{s_n}$ has a superattracting periodic point. 
\label{W_n is not empty for parabolic case}
}




\medskip


\paragraph{\bf Proof.}
The proof is parallel to that of Lemma \ref{W_n is not empty}. 
We observe the dynamics of $f_c^{k\nu}$ near $q_c$ through
the perturbed Fatou coordinate 
$\phi_c^+ : \Omega_c^+=\Omega_c^\ast \to \C$ of $q_c$. 
Let
$$
  \wt{V}_c := \phi_c^+(V_c),
$$
then $c \in W_n$ if and only if $\phi_c^+(f_c^{k\nu n}(0)) \in \wt{V}_c$.
By the normalization of $\phi_c^+$, we have
$$
  \tau(c) = \phi_c^+(f_c^{k\nu n}(0)) - \phi_c^-(f_c^{k\nu n}(0))
= \phi_c^+(f_c^{k\nu n}(0)) - n.
$$
Hence it follows that $c \in W_n$ if and only if 
$$
  \tau(c) + n \in \wt{V}_c.
$$
Next take a Riemann map
$$
  u : U_c \equiv U_{c_1} \to \D, \quad u(0) = 0
$$
and define
$$
  v(c, \zeta) := \phi_c^+ \circ (u \circ g_c)^{-1}(\zeta), \quad 
  \zeta \in \D \ \text{ with } \ v(c, 0)=0, 
$$
which is the inverse of a Riemann map $u \circ g_c \circ (\phi_c^+)^{-1}$
of $\wt{V}_c$. 
Now we solve the equation with respect to the variable $c$
\begin{equation}
  \tau(c)+n = v(c,\zeta)
\label{eqn for v(c,zeta) for parabolic case}
\end{equation}
for each fixed $\zeta \in \D$. 



\noin
{\bf Case 1 : $\nu = 1$. }
Since
$$
\tau(c)
=
-\frac{2\pi i}{A_0\sqrt{c - c_1}} + h(c), \quad h(c)=O(1) \quad (c \to c_1),
$$
the equation (\ref{eqn for v(c,zeta) for parabolic case}) 
can be rewritten as
\begin{equation}
F(c,\zeta) + G(c,\zeta) = 0,
\label{eqn F+G=0 for parabolic case}
\end{equation}
where
$$
  F(c,\zeta) :=  -\frac{2\pi i}{A_0\sqrt{c - c_1}} + n - v(c_1,\zeta), \quad
  G(c,\zeta) := h(c) - \big( v(c,\zeta)- v(c_1,\zeta) \big).
$$
The equation $F(c,\zeta)=0$ has a unique solution
$$
  c = c_n(\zeta) := c_1 -\frac{4\pi^2}{A_0^2(n - v(c_1,\zeta))^2}.
$$
Let
$$
  r_n(\zeta) := \bigg| -\frac{4\pi^2}{A_0^2(n - v(c_1,\zeta))^2} \bigg| 
= O(n^{-2})
$$
and take any $\beta$ with $0 < \beta < 1/2$. Consider 
(\ref{eqn F+G=0 for parabolic case}) in the disk 
$D(c_n(\zeta), r_n(\zeta)^{1+\beta})$. Since it is easy to see that
$$
  |F(c,\zeta)| = O(r_n(\zeta)^{\beta-1/2}) = O(n^{1-2\beta}), \quad 
  |G(c,\zeta)| = O(1)
$$
on the boundary $C$ of this disk, 
we have $|F(c,\zeta)| > |G(c,\zeta)|$ on $C$ for sufficiently large $n$. 
By Rouch\'e's theorem (\ref{eqn F+G=0}) has
a unique solution $c = \check{c}_n(\zeta)$ in $D(c_n(\zeta), r_n(\zeta)^{1+\beta})$.
By using this solution, we can write
$$
  W_n = \{ \check{c}_n(\zeta) \in \C \ | \ \zeta \in \D \}.
$$

\medskip

\paragraph{\bf Claim'.} \ 
{\it 
{\rm (1)} $\check{c}_n : \D \to W_n$ is holomorphic.

\noindent
{\rm (2)} For every $r \in (0,1)$, $\check{c}_n$ is univalent on $\D(r)$ for every
sufficiently large $n$.
}

\medskip


\paragraph{\bf Proof.}
(1) The proof is the same as in Claim (1) in Step (M2). 


\noindent
(2) Since
$$
  \tau(\check{c}_n(\zeta)) + n = v(\check{c}_n(\zeta), \zeta)
$$
and $v(\check{c}_n(\zeta), \zeta) \to v(c_1, \zeta) \ (n \to \infty)$
uniformly on $\overline{\D(r)}$, we can show that $v(\check{c}_n(\zeta), \zeta)$
is univalent on $\overline{\D(r)}$ for sufficiently large $n$ by the same method
as in the proof of the Claim (2) in Step (M2). Then
$\tau(\check{c}_n(\zeta)) + n$ is univalent and it follows that 
$\check{c}_n(\zeta)$ is also univalent, since $\tau(c)$ is univalent.
\QED (Claim'.)




\noin
{\bf Case 2 : $\nu \geq 2$. }
The argument is completely parallel to the Case 1. In this case
the functions $\tau(c), \ F(c,\zeta), \ G(c,\zeta), \ c_n(\zeta)$ and $r_n(\zeta)$
are replaced with
\begin{eqnarray*}
& & \tau(c)
= 
-\frac{2 \pi i}{\nu^2 B_0(c-c_1)} + h(c), \quad h(c) = O(1) \quad (c \to c_1), \\
& & F(c,\zeta) 
= 
-\frac{2 \pi i}{\nu^2 B_0(c-c_1)} + n - v(c_1,\zeta), \quad
G(c,\zeta)
=  h(c) - \big( v(c,\zeta)- v(c_1,\zeta) \big), \\
& & c_n(\zeta) 
=
c_1 + \frac{2 \pi i}{\nu^2 B_0(n-v(c_1,\zeta)} \quad \text{and} \quad
r_n(\zeta) = \bigg| \frac{2\pi i}{\nu^2 B_0(n - v(c_1,\zeta))} \bigg| = O(n^{-1}).
\end{eqnarray*}
Then we have the estimates
$$
  |F(c,\zeta)| = O(r_n(\zeta)^{-1+\beta}) = O(n^{1-\beta}), \quad
  |G(c,\zeta)| = O(1)
$$
on $\partial D(c_n(\zeta), r_n(\zeta)^{1+\beta})$ for $0 < \beta < 1/2$.
The rest of the argument is the same as in Case 1 and hence 
the same conclusion follows also in Case 2. 




Now we can show that $W_n$ is a non-empty Jordan domain with $C^1$ boundary
by the same method as in Step (M2). 
In particular, let $s_n := \check{c}_n(0)$ then this satisfies
$\tau(s_n) + n = 0$. This means that
$f_{s_n}^{k\nu n}(0) = b_{s_n}$,
which implies $g_{s_n} \circ f_{s_n}^{k\nu n}(0) = P_{s_n}^{k\nu np + N}(0)= 0$. 
Hence $P_{s_n}$ has a superattracting periodic point. 
This completes the proof.
\QED ~{\small (Lemma \ref{W_n is not empty}')}
\\



We call $s_n \in W_n$ the {\it center} of $W_n$. Now let 
$L = L_n  := k\nu n$ and
$V_c'=V'_{c,n}$ be the component of $f_c^{-L}(V_c)$ containing 
$0$ and define
$$
  G_c = G_{c,n} := g_c \circ f_c^L : V_c' \to U_c
  \quad \text{and} \quad
 \boldsymbol{G} = \boldsymbol{G}_n 
 := 
\{ G_c : V_c' \to U_c \}_{c \,\in\, W_n},
$$
where $W_n=\{c \in S \st f_c^L(0) \in V_c\}$.



\medskip




\noin
{\bf Step (P3): Proof for $\boldsymbol{G} = \boldsymbol{G}_n$ 
being a Mandelbrot-like family.} 

\noin
{\bf Step (P4): End of the proof of Theorem A for Parabolic case.}


These parts are completely the same as in the Misiurewicz case
and hence this completes the proof of Theorem A for the Parabolic case.
\QED
\\



\begin{rem*}
Theorem A is a kind of generalization of the Douady's result but
the statements of the results of ours and his are not quite parallel.
Actually Douady considered not only the case of the quadratic family
but also more general situation and proved a theorem 
(\cite{Douady 2000}, p.23, THEOREM 2) and then showed the theorem for 
the Mandelbrot set (\cite{Douady 2000}, p.22, THEOREM 1) by using it. 
Douady's result also shows that a sequence of quasiconformal images of 
${\mathcal M}(1/4 + \vep)$ appears in 
$D(s_0 \perp (1/4), \vep')$. It is possible to state our result like 
Douady's. But in order to do this, it is necessary to assume several 
conditions which are almost obvious for the quadratic family case and 
this would make the argument more complicated. So we just concentrated 
on the case of the quadratic family. We avoided stating our result like 
\lq\lq a sequence of quasiconformal images of 
${\mathcal M}(c_0 + \eta)$ appears" for the same reason.

In what follows, we summarize the general situation under which a result similar
to THEOREM 2 in \cite{Douady 2000} (that is, Theorem A' below) hold and this 
implies our Theorem A. These are the  essential assumptions for more
general and abstract settings, which leads to the general result Theorem A'.

\medskip

\noin
$\bullet$ \
$\{ f_c : U_c' \to U_c \}_{c \,\in\, \Lambda}$ is an analytic family of
quadratic-like maps with a critical point $\omega_c$, where $\Lambda \subset \C$
is an open set. The parameter $c_1 \in \Lambda$ is either Misiurewicz or 
parabolic.

\noin
$\bullet$ \
$\{ g_c : V_c \to U_c \}_{c \,\in\, \Lambda}$ is an analytic family of analytic
isomorphism, where $V_c$ satisfies 
$\overline{f_c^j(V_c)} \subset U_c \smallsetminus \overline{U_c'}$ 
for some $j \in \N$.

\noin
$\bullet$ \
The open sets $U_c, \ U_c'$ and $V_c$ are Jordan domains with $C^1$ boundary
and move by a holomorphic motion. Let $z_c(t)$ be a parametrization
of $\partial V_c$. Then $z_c(t)$ is holomorphic in $c$ and $C^1$ in $t$ and
$\frac{\partial^2}{\partial c \partial t}z_c(t)$ exists and continuous.


\noin
$\bullet$ \
(1) \ When $c_1$ is Misiurewicz, for some $l \in \N$, 
$f_{c_1}^l(\omega_{c_1})$ is a repelling periodic point of period $k$ 
and we let $a_c := f_c^l(\omega_c)$.
Let $q_c$ be the repelling periodic point persisting when $c$ is perturbed 
from $c_1$. Then assume that $a_c \ne q_c$ for $c (\ne c_1)$ which is 
sufficiently close to $c_1$.

(2) \ When $c_1$ is parabolic, $f_{c_1}$ has a parabolic periodic point 
$q_{c_1}$ of period $k$ with multiplier $\lambda_1$. Then assume that 
$q_{c_1}$ bifurcates such that $f_c$ has an appropriate normal 
form as in {\bf Step (P1)} for $c$ which is sufficiently close to $c_1$. 
(Douady gives a sufficient condition for this condition when $k=1$ and 
$\lambda_1 = 1$ in \cite[p.23]{Douady 2000}.)

\medskip


\noin
Note that there exists a $c_0$ such that $f_{c_1}$ is hybrid equivalent to
$P_{c_0}$. Now define the map $F_c : U_c' \cup V_c \to U_c$ so that
$F_c := f_c$ on $U_c'$ and $F_c := g_c$ on $V_c$. Also define
\begin{eqnarray*}
& & K(F_c) := \{ z \ | \ F_c^n(z) \ \text{is defined for all} \ n \ 
              \text{and} \ F_c^n(z) \in U_c' \cup V_c \}, \\
& & M_F := \{ c \in \Lambda \ | \ \omega_c \in K(F_c) \}.
\end{eqnarray*}

\noin
Under the above assumptions, we can show the following theorem which
implies our Theorem A:

\begin{thmA'*}
For every small $\vep > 0$ and $\vep' > 0$, there exists an 
$\eta \in \C$ with $|\eta| < \vep$ and
$c_0+\eta \notin M$ such that the
decorated Mandelbrot set ${\mathcal M}(c_0+\eta)$ appears quasiconformally
in $M_F$. 
\end{thmA'*}

\end{rem*}






