\section{RELATED WORK}
\paragraph{Inter-domain GPs} The idea of the inter-domain Gaussian processes has been discussed
in \citep{lazaro2009inter, van2020framework}.
However, their primary interests are 
using inter-domain transformations to define inducing variables for specifying GP approximations, whereas our work explores the usage of SVGP framework to perform scalable modeling and inference
with inter-domain observations.
%HIP-GP addresses the latter problem under the SVGP framework.


% kernel interpolation methods
\paragraph{Scalable Inducing Point Methods}
% variational GP methods
We note several recent approaches to scaling
the number of inducing points in GP approximations.
\citet{shi2020sparse} takes an orthogonal strategy to ours by approximating GP with inducing points in two independent directions, whereas HIP-GP requires inducing points to densely cover the input space. However, while improved over standard SVGP, their method still remains a cubic complexity.
\citet{izmailov2018scalable} introduces
the tensor train decomposition into the variational approximation.
%They use a factorized approach that does not have to explicitly represent all
%$M$ or ($M \times M$) variational parameters.
Alternatively, \citet{evans2018scalable} directly approximate the kernel with a
finite number of eigenfunctions evaluated on a dense grid of inducing points.
Both methods rely on \emph{separable} covariance kernels
to utilize the Kronecker product structure. This limits the
class of usable kernels.
The Matérn kernel, for example, is not separable across dimensions.
To fill that gap, we instead focus on the class of \emph{stationary kernels}.  

Another line of inducing point work is based on \emph{sparse kernel interpolations}.
KISS-GP uses a local kernel interpolation of inducing points
to reduce both the space and time complexity to $O(N+M^2)$ \citep{wilson2015kernel}.
SV-DKL also uses local kernel interpolation, and
exploits separable covariance structures and deep learning techniques
to address the problem of multi-output classification \citep{wilson2016stochastic}.
But these kernel interpolation methods are not applicable to
inter-domain observations under transformations.
More specifically, for standard (non-inter-domain) problems,  kernel interpolation methods approximate the $N \times N$ covariance matrix $\mathbf{K}_{\mathbf{N,N}}$ with $\mathbf{W} \mathbf{K}_{\mathbf{u, u}}\mathbf{W}^T$, where $\mathbf{W}$ is a sparse interpolation weight matrix. 
 However, for problems with integral observations, we must compute the integrated kernel $\mathbf{K}^{**}_{\mathbf{N,N}} = [\int \int Cov(\rho(\mathbf{x}_i), \rho(\mathbf{x}_j)) d\mathbf{x}_i d \mathbf{x}_j]_{i,j=1}^N$.
  Approximating this integral with local interpolation is not straightforward, and computing every integrated cross-covariance term is costly.
  Alternatively, HIP-GP decouples observations and inducing points into different domains
  through the \textit{inter-domain prior} (Equation~\ref{eq:inter-domain-GP}).
  This decoupled prior enables mini-batch processing of $k_{n,n}^{**}$, eliminates the need to compute cross-covariance terms  $k_{{n_i}, {n_j}}^{**}$,
  while still maintaining structure exploitation of $\mathbf{K}_{\mathbf{u,u}}$. 
  
 \paragraph{Fast Whitening Strategy} As is mentioned before, the classical whitening strategy is the Cholesky decomposition that has $O(M^2)$ space and $O(M^3)$ time complexity. 
 \citet{pleiss2020fast} provides a more general purpose method for fast matrix roots and is in particular applicable to whitening GP. Their method is an MVM-based  approach that leverages the contour integral quadrature and  requires $O(M \log M+QM)$ time for $Q$ quadrature points. Our whitening strategy specifically targets gridded inducing points and achieves more complexity savings ($O(M\log M)$ time). 
  
%The integral transformation, for example,
%operates on the whole latent function,
%making a direct local interpolation impossible.
%We note that the SV-DKL also considers a mean-field variational approximation.
%Here we extend it to the more general block-independent approximation.
 
