\section{SCALING $M$: HIP-GP for INTER-DOMAIN PROBLEMS}
\label{sec:hip-gps}
We first formulate the SVGP framework for inter-domain observations, and identify
its computational bottlenecks in Section~\ref{sec:method-interdomain-svgp}.
We then address these bottlenecks by the HIP-GP algorithm using the techniques developed in
Section~\ref{sec:computational} - \ref{sec:variational-families}.
In Section~\ref{sec:method-summary}, we summarize our methods and discuss
optimization procedures for HIP-GP.
\subsection{Inter-domain SVGP Formulation}
\label{sec:method-interdomain-svgp}
We show that the inter-domain observations can be easily incorporated into the SVGP framework.
We place a set of inducing points $\bu = \rho(\bar{\bx})$ in the latent domain
at input locations $\bar{\bx} = \left( \bar{\bx}_1, \cdots, \bar{\bx}_M \right)$.
Connections to the observations are made through the
inter-domain covariance,
while the observations are characterized by
the transformed-domain covariance.
Formally, we have the inter-domain GP prior:
\begin{align}
  \label{eq:inter-domain-GP}
\begin{pmatrix}
\rho_n^* \\
\bu
\end{pmatrix}
\sim \mathcal{N} \left( 0,
\begin{pmatrix}
 k^{**}_{n,n} & \bk^*_{n, \bu} \\
 \bk^*_{\bu, n} & \bK_{\bu, \bu} \\
\end{pmatrix}
\right),
\end{align}
where the inter-domain covariance and the transformed-domain covariance are defined as
\begin{align}
\label{eq:domain-kernel}
\bk^*_{\bu, n} &\triangleq Cov \left(\rho(\bar{\bx}), \rho^*(\bx_n)\right)
 = Cov \left(\bu, \rho^*_n \right)  \, , \\
\label{eq:domain-kernel2}
k^{**}_{n,n} &\triangleq Cov \left(\rho^*(\bx_n), \rho^*(\bx_n) \right)
 = Cov \left( \rho^*_n, \rho^*_n  \right) \, , 
\end{align}
and the latent domain covariance is
\begin{align}
  \bK_{\bu, \bu} &\triangleq Cov \left( \rho (\bar{\bx}), \rho (\bar{\bx}) \right)
    = Cov \left( \bu, \bu \right).
\end{align}
This form of the prior suggests formulating the inter-domain SVGP objective
as follows
\begin{align}
\mathcal{L}(\blambda) \label{eq:inter-domain-svgp} &= \sum_{n=1}^N \mathbb{E}_{q_{\blambda}(\bu)}\left[
      \mathbb{E}_{p(\rho_n^* \given \bu)} \left[ \ln p(y_n \given \rho_n^*) \right] \right] \\
& \qquad  - KL(q_{\blambda}(\bu)\,||\,p(\bu)) \nonumber\,,
\end{align}
where
\begin{align*}
  p(\rho_n^* | \bu) &= N(\rho_n^* | \bk_{n,\bu}^* \bK_{\bu, \bu}^{-1} \bu,
  k_{n,n}^{**} - \bk_{n,\bu}^* \bK_{\bu, \bu}^{-1} \bk_{\bu, n}^*)\,.
\end{align*}
Note that this framework can be extended to observations in multiple domains
by including them with their corresponding inter-domain and transformed-domain covariances.
Under this formulation, we avoid computing the $N \times N$
transformed-domain covariance matrix $\bK_{N,N}^{**}$ that appears in the exact GP objective.
Instead, only $N$ terms of variance $k_{n,n}^{**}$ need to be evaluated. 
Importantly, the disentanglement of observed and latent domains
enables us to exploit structure of  $\bK_{\bu, \bu}$ for efficient computations.
Such exploitation would be difficult without a variational approximation,
especially in the case of mixed observations from multiple domains.

\paragraph{Whitened Parameterization}
Whitened parameterizations are used to improve
inference in models with correlated priors because they offer a
better-conditioned posterior
\citep{murray2010slice, hensman2015mcmc}.
Here we will show an additional computational benefit
in the variational setting --- the whitened posterior allows us to
avoid computing $\ln |\bK_{\bu,\bu}|$ which appears in the KL term
in Equation~\ref{eq:inter-domain-svgp}.
%a fact that has not been observed in the literature.
To define the whitened parameterization, we describe the GP prior over
$\bu$ as a deterministic function of standard normal parameters $\bepsilon$: 
\begin{align}
  \bepsilon \sim \mathcal{N}(0, I) \,, \quad
  \bu = \bR \bepsilon.
\end{align}
To preserve the covariance structure in the prior distribution of $(\rho^*_n, \bu)$ (Equation~\ref{eq:inter-domain-GP}),
the transformation $\bR$ and
the whitened correlation $\bk_n \triangleq Cov(\bepsilon, \rho^*_n)$ need to satisfy the following two equalities:  
\begin{equation}
  \begin{aligned}
  \label{eq:whitened-criteria}
  \bK_{\bu, \bu} &=  Cov( \bR \bepsilon, \bR \bepsilon) =   \bR \bR^\top \, , \\
  \quad  \bk^{*}_{\bu, n} &= Cov(\bR \bepsilon, \rho^*_n) = \bR \bk_n \, . 
\end{aligned}
\end{equation}
The classical whitening strategy in GP inference is to use the Cholesky decomposition: $\bK_{\bu, \bu} = \bL \bL^\top$ where  $\bL$ is a lower triangular matrix. In this case, $\bR = \bL$ 
and $\bk_n = \bL^{-1} \bk^*_{\bu, n}$.

\iffalse
\begin{align}
  \bepsilon \sim \mathcal{N}(0, I) \,, \quad
  \bu = \bK_{\bu,\bu}^{1/2} \bepsilon \sim \mathcal{N}(0, \bK^{}_{\bu,\bu}) \,.
   \label{eq:matrix-sqrt}
\end{align}
\fi
Now we can target the variational posterior over
the whitened parameters
$\bepsilon$: 
${q_{\blambda}(\bepsilon) = \mathcal{N}(\bepsilon \given \bm, \bS)}$.
The resulting \emph{whitened variational objective} is
\begin{align}
\label{eq:whitened-elbo}
\mathcal{L}(\blambda)
  %&= \mathbb{E}_{q_{\blambda}(\epsilon) p(\brho \given \epsilon)}
  %    \left[ \ln p(\by \given \brho) \right] -
  %    KL(q_{\blambda}(\epsilon) \,||\, p(\epsilon)) \\
  &= \sum_{n} \underbrace{
  	    \mathbb{E}_{q_{\blambda}(\bepsilon) p(\rho_n^* \given \bepsilon)} \left[
	    	\ln p(y_n \given \rho_n^*)
	    \right]
	  }_{\circled{$a'_n$}} -
      \underbrace{
	  	KL(q_{\blambda}(\bepsilon) \,||\, p(\bepsilon))
	  }_{\circled{b$'$}} \,
\end{align}
where
\begin{align}
\circled{$a'_n$}
&= -\frac{1}{2}\ln \sigma_n^2 - \frac{1}{2\sigma_n^2}
    \Big( y_n^2 + k^{**}_{n,n} - \bk_n^\intercal \bk^{}_n  \nonumber \\
&\quad + \bk_n^\intercal
     \left( \bS + \bm \bm^\intercal \right) \bk_n - 2 y^{}_n \bk_n^\intercal \bm
    \Big) \,\,,\\
\circled{b$'$}
%KL(q_{\blambda}(\epsilon) \,||\, p(\epsilon))
  &= \frac{1}{2}\left( \text{tr}(\bS) + \bm^\intercal \bm - \ln |\bS| - M \right) .
\label{eq:kl-term}
\end{align}
\iffalse
 and we define the correlation term for clarity
 \begin{align}
    \bk_n \triangleq Cov(\bepsilon, \be_n)  = \bK_{\bu,\bu}^{-1/2}\bK^*_{\bu,n}  \,\,.
    \label{eq:k-vector}
 \end{align}
 \fi
%Becuase $p(\bepsilon)$ is a standard normal we have simplified term \circled{$b'$},
%by avoiding computing $\ln |\bK_{\bu,\bu}|$.
\paragraph{Computational Bottlenecks}
The whitened objective above still factorizes over data points. However, there remain two computational bottlenecks. First, the correlation term $\bk_n$ in \circled{$a_n'$}
depends on the choice of the whitening strategy.
The common Cholesky strategy requires $O(M^3)$ computation and $O(M^2)$ storage
which is infeasible for large $M$.
We address this bottleneck in Section~\ref{sec:computational}.
The second bottleneck lies in the variational covariance $\bS$ which is an $M \times M$ matrix,
requiring $O(M^2)$ to store and $O(M^3)$ to compute the $\ln |\bS|$ in \circled{$b'$}.
We will address this problem by a structured
variational approximation in Section~\ref{sec:variational-families}.


%\subsection{Whitened parameterization}
\label{sec:whitened}

Whitened (or non-centered) parameterizations are used to improve
inference in models with correlated priors because they offer a
better-conditioned posterior
\citep{murray2010slice, hensman2015mcmc}.
Here we obtain an additional important computational benefit of it
in the variational setting --- the whitened posterior allows us to
avoid computing $\ln |\bK_{\bu,\bu}|$, a fact that has not been observed
in the literature.
To define the whitened parameterization, we describe the GP prior over
$\bu$ as a deterministic function of standard normal parameters $\bepsilon$
\begin{align}
  \bepsilon \sim \mathcal{N}(0, I) \,, \quad
  \bu = \bK_{\bu,\bu}^{1/2} \bepsilon \sim \mathcal{N}(0, \bK^{}_{\bu,\bu}) \,.
   \label{eq:matrix-sqrt}
\end{align}
We target the posterior over
the \emph{whitened parameters}
$\bepsilon$ with variational approximation
${q_{\blambda}(\bepsilon) = \mathcal{N}(\bepsilon \given \bm, \bS)}$.
The resulting whitened variational objective is
\begin{align*}
\mathcal{L}(\blambda)
  %&= \mathbb{E}_{q_{\blambda}(\epsilon) p(\brho \given \epsilon)}
  %    \left[ \ln p(\by \given \brho) \right] -
  %    KL(q_{\blambda}(\epsilon) \,||\, p(\epsilon)) \\
  &= \sum_{n} \underbrace{
  	    \mathbb{E}_{q_{\blambda}(\bepsilon) p(\brho \given \bepsilon)} \left[
	    	\ln p(y_n \given \rho_n)
	    \right]
	  }_{\circled{$a'_n$}} -
      \underbrace{
	  	KL(q_{\blambda}(\bepsilon) \,||\, p(\bepsilon))
	  }_{\circled{b$'$}} \,
\end{align*}
where
\begin{align}
\circled{$a'_n$}
&= -\frac{1}{2}\ln \sigma_n^2 - \frac{1}{2\sigma_n^2}
    \Big( y_n^2 + \bK^{**}_{n,n} - \bk_n^\intercal \bk^{}_n  \nonumber \\
&\quad + \bk_n^\intercal
     \left( \bS + \bm \bm^\intercal \right) \bk_n - 2 y^{}_n \bk_n^\intercal \bm
    \Big) \,\,,\\
\circled{b$'$}
%KL(q_{\blambda}(\epsilon) \,||\, p(\epsilon))
  &= \frac{1}{2}\left( \text{tr}(\bS) + \bm^\intercal \bm - \ln |\bS| - M \right) \,
\label{eq:kl-term}
\end{align}
 and we define the correlation term for clarity
 \begin{align}
    \bk_n \triangleq Cov(\bepsilon, \be_n)  = \bK_{\bu,\bu}^{-1/2}\bK^*_{\bu,n}  \,\,.
    \label{eq:k-vector}
 \end{align}

Becuase $p(\bepsilon)$ is a standard normal we have simplified term \circled{$b'$},
by avoiding computing $\ln |\bK_{\bu,\bu}|$.
The remaining computational bottlenecks are
the correlation term $\bk_n$ in \circled{$a_n'$} and the $\ln |\bS|$ in \circled{$b'$}
which we address in the following.
Specifically, we note that the efficient computation of $\bk_n$
depends on the specific choice of whitening strategy,
which we will discuss in Section~\ref{sec:computational}.

\subsection{Computational Accelerations}
\label{sec:computational}
We now turn to the first bottleneck --- how to design an efficient whitening strategy
to compute the term $\bk_n$. %for
%low-dimensional dense spatial-(temporal) problems.
To do so,
we rely on judicious placement of inducing points and assume a stationary
covariance function, a general and commonly used class.
We describe three key ingredients below.
%Together they present an
%efficient algorithm for computing $\bk_n$.

%The following section gives background on the method of conjugate gradients
%(CG) and preconditioned conjugate gradients (PCG) to iteratively and
%efficiently solve big linear systems with matrix multiplies.
%Section~\ref{sec:toeplitz-structure} describes efficient matrix multiplies
%we use by exploiting hierarchical Toeplitz structure in $\bK_{\bu,\bu}$.
%Section~\ref{sec:toeplitz-preconditioner} describes a novel preconditioner
%for hierarchical Toeplitz matrices that dramatically speeds up the iterative
%solver compared to standard conjugate gradients.


%Above, we established that we can compute matrix multiplies quickly, $K
%\bv$ for hierarchical Toeplitz matrices.  However, for use within the SVGP
%framework, we need to compute $K^{-1}\bv$.  This can be done efficiently
%with a preconditioned conjugate gradient (PCG) method.  The PCG method is
%an iterative linear system solver that only uses matrix multiplies.  The
%solution is exact after $M$ steps (for a solution of size $M$), but in
%practice it can be made much more efficient by stopping early at a near
%optimal solution (e.g.~after 10 or 20 steps).

\iffalse
A symmetric Toeplitz matrix is a symmetric matrix that has \emph{constant
diagonals}. For a stationary kernel, the covariance matrix evaluated
at evenly-spaced grid points forms a symmetric Toeplitz matrix.
One-dimensional Toeplitz matrices have been used for GP \citep{cunningham2008fast},
and the multi-dimensional extension, which we term as the \emph{hierarchical Toeplitz structure},
is explored in \citet{wilson2015thoughts}.
\fi

\paragraph{Hierarchical Toeplitz Structure}
\label{sec:toeplitz-structure}
Consider a $D$-dimensional grid of evenly spaced points
of size $M \triangleq M_1 \times \cdots \times M_D$, characterized by
one-dimensional grids of size $M_i$ along dimension $i, i=1:D$,
where $D$ is the input dimension.
Under a stationary kernel, we construct a covariance matrix
for this set of points in $x$-major
order (i.e.~\texttt{C}-order).
%The covariance matrix $\bK$ over this ordering is \emph{block Toeplitz
%with Toeplitz blocks} with constant block diagonals.
Such a matrix will have \emph{hierarchical Toeplitz structure},
which means the diagonals of the matrix are constant.
Because of this data redundancy, a hierarchical Toeplitz matrix
is characterized by its first row.
Now we place the inducing points along
a fixed, equally-spaced grid, resulting in a $M \times M$ hierarchical
Toeplitz Gram matrix $\bK^{}_{\bu,\bu}$.
The efficient manipulation of $\bK_{\bu, \bu}$ is through its circulant embedding:
\begin{align}
    \bC =
    \begin{pmatrix}
    \bK_{\bu, \bu} & \tilde{\bK} \\
    \tilde{\bK}^\top & \bK_{\bu, \bu}
    \end{pmatrix}
\end{align}
where $\tilde{\bK}$ is the appropriate reversal of $\bK_{\bu, \bu}$ to make $\bC$ circulant.
$\bC$ admits a convenient diagonalization
\begin{align}
  \bC = \bF^\top \bD \bF
      = \bF^\top \text{diag} \left(\bF \bc \right) \bF \,\,,
\end{align}
where $\bF$ is the fast Fourier transform matrix, $\bD$ is a
diagonal matrix of $\bC$'s eigenvalues, and $\bc$ is the first row of $\bC$.
%that represents the circulant matrix.
This diagonalization enables fast MVMs with $\bC$ and
hence the embedded $\bK_{\bu, \bu}$ via the FFT algorithm
in $O(M \ln M)$ time, further making it sufficient for use
within CG to efficiently solve a linear system.

The fast solves afforded by Toeplitz structure have been previously
utilized for exact GP inference \citep{cunningham2008fast, wilson2015thoughts}.
Here, we extend the applicability of Toeplitz structure
to the variational inter-domain case
by introducing a fast whitening procedure and an effective preconditioner for CG.


\paragraph{Fast Whitening Strategy}
\label{sec:toeplitz-whitening}
%To avoid the cubic cost of Cholesky decomposition,
%we develop a new whitening strategy
%utilizing the hierahical Toeplitz structure.
Similar to the Cholesky decomposition, we aim to
find a whitened matrix $\bR$ that serves as a root of $\bK_{\bu, \bu}$,
i.e. $\bR \bR^T = \bK_{\bu, \bu}$.
Directly solving $\bK_{\bu, \bu}^{1/2}$ is not trivial. Alternatively, we access the root from the circulant embedding of $\bK_{\bu, \bu}$.  We consider the square root of the circulant matrix
\begin{equation}\label{eq:csquared-fft}
  \bC^{1/2} = \bF^\intercal \bD^{1/2} \bF,
\end{equation}
and its block representation
\begin{equation}
  \bC^{1/2} = \begin{pmatrix}
  \bA & \bB \\
  \bB^\top & \bD \\
\end{pmatrix}.
\end{equation}
We make a key observation that the first row-block $(\bA, \bB)$ can be viewed as a
``rectangular root" of $\bK_{\bu, \bu}$. That is, we define a
 \emph{non-square} whitening matrix $\bR$ and the correlation vector $\bk_n$ as follows 
\begin{align}\label{eq:csquared}
\bR &\triangleq \begin{pmatrix}
\bA & \bB
\end{pmatrix} \,, \quad \bk_n \triangleq \bR^T \bK_{\bu, \bu}^{-1} \bk^*_{\bu, n}.
\end{align}
One can verify such $\bR$ and $\bk_n$ satisfy
Equation~\ref{eq:whitened-criteria}, thus offering a valid whitening strategy.
We note that since $\bR$ is non-square and $\bu = \bR \bepsilon$,
this strategy doubles the number of variational parameters
in each dimension of the whitened space.

Now we address how to efficiently compute $\bk_n$  defined in Equation~\ref{eq:csquared}. We first compute the intermediate quantity $\bk_n'= \bK_{\bu, \bu}^{-1} \bk^*_{\bu, n}$ via CG in $O(M \ln M)$ time. We then compute $\bk_n = \bR^\top \bk_n'$. Note that $\bR^{T}$ is embedded in the matrix $\bC^{1/2}$ which also admits the FFT diagonalization (Equation \ref{eq:csquared-fft}). Hence,  MVM with $\bR^\top$ can be also done in $O(M \ln M)$ time.  

Lastly, we show how to make CG's computation of $\bK_{\bu, \bu}^{-1} \bk^*_{\bu, n}$ faster with a well-structured preconditioner. 

\paragraph{Efficient Preconditioner}
\label{sec:toeplitz-preconditioner}
The ideal preconditioner $\bP$ is a matrix that whitens the matrix to be
inverted --- the ideal $\bP$ is $\bK_{\bu, \bu}^{-1}$.  However, we cannot
efficiently compute $\bK_{\bu, \bu}^{-1}$. But due to the convenient
diagonalization of the circulant embedding matrix $\bC$, we can
efficiently compute the inverse of $\bC$: 
\begin{align}
    \bC^{-1} = \bF^\top \bD^{-1} \bF \, .
\end{align}
%Naively, one might think this is a direct way to compute $\bK_{\bu, \bu}^{-1} \bp$,
%obviating PCG in the first place.
Note that the upper left block of
$\bC^{-1}$ \emph{does not} correspond to $\bK_{\bu, \bu}^{-1}$
% so directly computing $\bK^{-1} \bp'$ via FFTs and inverse FFTs is unavailable.
as we explicitly write out
\begin{align}
    \bC^{-1} =
    \begin{pmatrix}
    \left( \bK_{\bu, \bu} - \tilde{\bK} \bK_{\bu, \bu}^{-1} \tilde{\bK}^{\intercal}\right)^{-1} & ... \quad\\
    ... & ... \quad
    \end{pmatrix} \,\,.
\end{align}

However, when the number of inducing points are large enough,
$\bK_{\bu,\bu}$ approaches a banded matrix,
and so $\tilde{\bK}$ is increasingly sparse.
Therefore, the upper left block of $\bC^{-1}$ would be close to $\bK_{\bu, \bu}^{-1}$, suggesting that it can serve as an effective \emph{preconditioner} within PCG,
and therefore an effective strategy for solving a linear system with the 
kernel matrix. % \citep{cutajar2016preconditioning}.
We note that this banded property is often exploited in developing
effective preconditioners \citep{chan1996conjugate, saad2003iterative}.
To justify this intuition, we anlayze the PCG convergence speed under various settings of 
kernel functions and inducing point densities in appendix.
We compare the performance of PCG and CG in systems of varying size
in Section~\ref{sec:experiments-preconditioner}.
%depicted in Figure~\ref{fig:preconditioner} and summarized
We find that PCG
converges faster than CG across all systems, 
taking only a fraction of the number
of iterations that standard CG requires to converge.
This speedup is crucial --- PCG is a subroutine we use to compute the gradient
term corresponding to each observation $n$.

\paragraph{Summary of Fast Computation for $\bk_n$}
To summarize, we exploit additional computational benefits of
the hierarchical Toeplitz matrix through its circulant embedding matrix, which enables fast matrix square-root and matrix inverse. We further utilize these fast operations to design novel whitening and preconditioning strategies. 
Thus, the whitened correlation term $\bk_n = \bR^T \bK_{\bu, \bu}^{-1} \bk^*_{\bu, n}$
can be efficiently processed as follows:
\begin{enumerate}
  \item embed $\bK_{\bu, \bu}$ into a larger circulant matrix $\bC$;
  \item solve $\bK_{\bu, \bu} \bk'_n = \bk^*_{\bu, n}$ for the  intermediate term $\bk'_n$ with PCG,
  where we utilize the FFT diagonalization of $\bC$ and $\bC^{-1}$;
  \item compute $\bk_n = \bR^\top \bk'_n$ , where we utilize the FFT diagonalization of $\bC^{1/2}$.
\end{enumerate}
%Note that the circulant embedding takes linear time, and only the row representation of $\bC$ is stored.
The space and time complexity of this procedure are $O(M)$ and $O(M \ln M)$.
This offers a speed-up over the Cholesky decomposition which has $O(M^2)$ space and $O(M^3)$ time complexity, respectively.
In Section~\ref{sec:experiments-whitened}, we examine this acceleration by comparing the time of computing $\bk_n$ using Cholesky and using HIP-GP, as the system size $M$ varying from $10^3$ to $10^6$. HIP-GP's strategy outperforms Cholesky for small values of $M$, and scales to larger $M$ where Cholesky is no longer feasible. We present HIP-GP's algorithmic details in appendix. 


\subsection{Structured Variational Approximation}
\label{sec:variational-families}
Finally, we turn to the second bottleneck: how to represent
and manipulate variational parameters of mean $\bm$ and covariance $\bS$.
%Inducing point methods were introduced to alleviate the cubic scaling of
%exact Gaussian process inference.  We can view this approximation as
%creating a low-rank covariance over all observations
%\citep{quinonero2005unifying}.  In that same vein, we introduce
%hierarchical structure into the inducing points in order to scale to larger
%number of inducing points.
We propose the \emph{block independent} variational family
%two variational families: \emph{independent} (i.e.~mean field)
\iffalse
\begin{align}
  q(\bu) &= \prod_{m}^M \mathcal{N}(\bu_{m} \given \bm_{m}, \bs^2_m) \,,
\end{align}
\fi
%and \emph{block independent} (i.e.~structured)
\begin{align}
  q(\bu) &= \prod_{b}^B \mathcal{N}(\bu_{b} \given \bm_{b}, \bS_{b}) \,,
  \label{eq:block-cov}
\end{align}
where $\bu_{b}$ denotes a subset of inducing points of size $M_b < M$
and $\bS_b$ is the $M_b \times M_b$ variational covariance for that
subset. Note that when $M_b=1$, it reduces to the $\emph{mean-field}$ variational family,
and when $M_b = M$, it is the \emph{full-rank} variational family.
%In both cases, manipulating the variational covariance $\bS$ is computationally
%feasible.
%For diagonal $\bS$, calculating the inverse and log-determinant are
%both linear in $M$.
Calculations of the inverse and log-determinant of block independent $\bS$ scale $O(B M_b^3)$
--- we must choose $M_b$ to be small enough to be practical.


We note that independence in the posterior is a more reasonable approximation
constraint \emph{in the whitened parameterization} than the original space.
The original GP prior, $p(\bu)$, is designed to have high correlation, and therefore
data are unlikely to decorrelate inducing point values.
In the whitened space, on the other hand, the prior is already uncorrelated.
Hence the whitened posterior
is not spatially correlated as much as the original posterior.
%To see this, the correlation structure in the original posterior:
%$\tilde{\bS} = \bK_{\bu, \bu}^{1/2} \bS \bK_{\bu, \bu}^{1/2}$
%is preserved through the transform of $\bK_{\bu,\bu}^{1/2}$
%even if $\bS$ is mean-field. % as is also observed in \citet{wilson2016stochastic}.
This is in addition to the benefits of optimizing in the whitened space
due to better conditioning.

\paragraph{Constructing Blocks}
The block independent approximation of Equation~\ref{eq:block-cov}
requires assigning inducing points to $B$
blocks.  Intuitively, blocks should include nearby points, and so we focus on
blocks of points that tile the space.
% In two-dimensions we
%divide each one-dimensional grid into $\tilde{B}$ segments, resulting in a
%grid of $B = \tilde{B}^2$ blocks.
%We note that the \emph{Toeplitz ordering} and \emph{block ordering} may not be
%the same.  This complicates implementation, as we represent the block diagonal
%matrix $\bS$ as a $B \times M_b$ (blocks by block-size) array.
To reconcile the Toeplitz ordering and the block orderings (they may not be the same),
we simply have to permute any $M$-length vector
(e.g.~$\bk^*_{\bu,n}$ or $\bm$) before multiplication with $\bS$ and then undo
the permutation after multiplication.  Fortunately, this permutation is linear
in $M$.

\subsection{Method Summary}
\label{sec:method-summary}
%Sections~\ref{sec:whitened}, \ref{sec:computational}, and
%\ref{sec:variational-families} address the three computational bottlenecks
%described in Section~\ref{sec:bottlenecks}.
The modeling difficulty of inter-domain GP problems arises from the numerical intractability
of computing the full transformed-domain covariance
$\bK^{**}_{N,N}$ of size $N \times N$. We avoid this difficulty by decoupling the
 the observations and the inducing points into different domains under the SVGP framework.
%Hence, the costly transformed-domain covariance is only required for $N$ single data points.
Moreover, we leverage the kernel structure of the Gram matrix
$\bK^{}_{\bu,\bu}$ in the latent domain for efficient computations.

The computational
difficulty stems from the computations with the kernel matrix $\bK_{\bu,\bu}$ and the variational covariance $\bS$.  We avoid having to
compute $\ln|\bK^{}_{\bu,\bu}|$ by using a \emph{whitened parameterization};
we develop a \emph{fast whitening strategy} to compute the whitened correlation term
 $\bk_n = \bR^\top \bK_{\bu, \bu}^{-1} \bk^{*}_{\bu,n}$
by exploiting  the \emph{hierarchical Toeplitz structure} with a \emph{novel
preconditioner};
and finally we explore a \emph{structred representation} for $\bS$.

\paragraph{Optimization} We perform natural gradient descent on
variational parameters using closed-form gradient updates.
For gradient-based learning of kernel hyperparameters, automatically differentiating
through the CG procedure is not numerically stable.
Fortunately, we can efficiently compute the analytical gradient of CG solves 
utilizing the hierarchical Toeplitz structure,
without increasing the computational complexity.
See appendix for more details on gradient derivations.
