\section{DISCUSSION}
% summarize
We formulate a general SVGP framework for inter-domain GP problems.
Upon this framework, we further scale the standard SVGP inference
by developing the HIP-GP algorithm,
with three technical innovations (i) a fast whitened parameterization,
(ii) a novel preconditioner for fast linear system solves with
hierarchical Toeplitz structure, and (iii)
a structured variational approximation.
The core idea of HIP-GP lies in the structured exploitations of the kernel matrix and the variational posterior.
%Therefore, it can be applied to various setting as a drop-in replacement
%for the SVGP but with better scalability
Therefore, it can be potentially extended to various settings, e.g.
the case where a GP is a part
of a bigger probabilistic model,
and the non-Gaussian likelihoods thanks to recent advance
in non-conjugate GP inference \citep{salimbeni2018natural}.

% future works
%We note that the computational acceleration
%of the proposed CG preconditioner
%relies on the kernel matrix's structure.
%which further relies on the relationship between the density of inducing grids
%and the smootheness of the kernel.
%This affects PCG's early-stopping point,
%yet making the number of inducing points
%another hyper-parameter to tune.
Future works involve more in-depth analysis of
such CG-based approximate GP methods.
On the applied side, we will to apply HIP-GP to
the Gaia dataset \citep{gaia2018gaia}
which consists of nearly 2 billion stellar observations. 
