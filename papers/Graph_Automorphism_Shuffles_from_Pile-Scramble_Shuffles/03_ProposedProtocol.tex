\section{Graph shuffle protocols}\label{s:graph}

In this section, we construct a card-based protocol called  the graph shuffle protocol for a directed graph.
First, we introduce a graph shuffle in Subsection \ref{ss:graph}. Second, we construct the graph shuffle protocol, which is a shuffle protocol for any graph shuffle in Subsection \ref{ss:graphprotocol}. We note that our protocol requires PSSs only. 


\subsection{Graph shuffle}\label{ss:graph}
First, we recall some fundamentals from graph theory; for example, see \cite{CLZ}.

A directed graph is a quadruple $G=(V_G,E_G,s_G,t_G)$ consisting of two sets $V_G$, $E_G$ and two maps $s_G,t_G:E_G\to V_G$.
Each element of $V_G$ (resp. $E_G$) is called a vertex (resp. an \md{edge}). 
\md{Note that there might be two or more edges from $a$ to $b$ for some $a, b \in V_G$, that is, $G$ admits multiple edges.} 
For an \md{edge} $e\in E_G$, we call $s_G(e)$ (resp. $t_G(e)$) the source (resp. the target) of $e$. We will commonly write $a\xrightarrow{e}b$ or $e:a\to b$ to indicate that an \md{edge} $e$ has the source $a$ and the target $b$, and identify $e$ with a pair $(s_G(e),t_G(e))$.
A directed graph $G$ is finite if two sets $V_G$ and $E_G$ are finite sets. 
In \md{this paper}, a graph means a finite directed graph with $n$ vertices and $m$ \md{edges}.

Let $G$ be a graph.
For a vertex  $v\in V_G$, we define the following three functions:
\[ \f{in}(v)=|\{e\in E_G\mid v=t_G(e)\}|,\quad \f{out}(v)=|\{e\in E_G \mid v=s_G(e)\}|, \quad\text{and}\quad \f{deg}(v)=\f{in}(v)+\f{out}(v).\]
%and
%\[  \f{deg}(v)=\f{in}(v)+\f{out}(v).\]
The number $\f{deg}(v)$ is called \textit{the degree} of $v$.
We set $\f{Deg}_G = \{\f{deg}(v) \mid v \in V_G\}$.

For graphs $G$ and $G'$, a pair $f=(f_0,f_1):G\to G'$ consisting of maps $f_0:V_G\to V_{G'}$ and $f_1:E_G\to E_{G'}$ is a morphism of graphs if $(f_0\times f_0)\circ(s_G\times t_G)=(s_{G'}\times t_{G'})\circ f_1$ holds. In addition, if $f_0$ and $f_1$ are bijective, $f$ is called  \textit{an isomorphism} of graphs. In this case, we say that $G$ and $G'$ are isomorphic as graphs. 
In other words, two graphs $G$ and $G'$ are isomorphic as graphs when $x\xrightarrow{e}y$ in $G$ exists if and only if $f_0(x)\xrightarrow{f_1(e)}f_0(y)$ exists in $G'$.
We denote by $\f{Iso}(G,G')$ the set of all isomorphisms from $G$ to $G'$, and $\f{Iso}_0(G,G')$ the set of all $f_0$ such that $(f_0, *) \in \f{Iso}(G,G')$.
For a graph $G$, an isomorphism from $G$ to itself is called \textit{an automorphism}.
We denote by $\mathsf{Aut}(G)$ the set of all automorphisms of $G$, and $\mathsf{Aut}_0(G)$ the set of all $f_0$ such that \md{there exists $(f_0, f_1) \in \mathsf{Aut}(G)$}. \md{Then it is obvious that $\mathsf{Aut}(G)$ is a group by the composition of maps.
Furthermore, the group structure of $\mathsf{Aut}(G)$ induces the group structure on $\mathsf{Aut}_0(G)$.}
Note that, if $G$ has no multiple \md{edges}, then an automorphism $f = (f_0, f_1)\in \mathsf{Aut}(G)$ is determined by $f_0$. 
In the case that $G$ is an undirected graph, one can transform $G$ into the following directed graph $\overset{\to}{G}$:
\[ \text{ $V_{\overset{\to}{G}}=V_G$,\quad $E_{\overset{\to}{G}}=\{i\to j, \ j\to i\mid (i,j) \in E_G\}$}. \]

\begin{definition}\label{def:graphshuffle}
Let $G$ be a graph. 
% \md{and $\F$ a uniform distribution on $\aut_0(G)$}. 
The uniform closed shuffle $\shuffle_{(\aut_0(G), \F)}$ is called  \textit{the graph shuffle} for $G$ over $n$ cards. (Recall that $G$ has $n$ vertices.)
\end{definition}

\subsection{Graph shuffle protocols}\label{ss:graphprotocol}

In this subsection, we construct a graph shuffle protocol, which is a shuffle protocol of the graph shuffle for a graph $G = (V_G,E_G,s_G,t_G)$. 
We set $V_G=\{1,2,\ldots, n\}$. 
Let $D_{\f{inp}}=\{x_1,x_2,\ldots, x_n\}$ be any deck, and $U_{\f{inp}}$ any input set from $D_{\f{inp}}$. 
%Assume that all symbols of $D_{\f{inp}}$ are distinct and $\f{front}(\f{x})=(?,?,\ldots, ?)$ for all $\f{x} \in U_{\f{inp}}$. 
%We note that our protocol works for any deck $D_{\f{inp}}$ and any input set $U_{\f{inp}}$. 
%In particular, it works for $D_{\f{inp}}$ such that all symbols of $D_{\f{inp}}$ are distinct and $\f{front}(\f{x})=(?,?,\ldots, ?)$ for all $\f{x} \in U_{\f{inp}}$ as in Remark \ref{rem:shuffleprotocol}. 
We set a card-sequence $\f{h}$ of helping cards as follows:
\[
\f{h} = \crd{$\stext{1}$} \, \crd{$\stext{2}$} \, \crd{$\stext{3}$} ~\cdots ~\crd{$\stext{n}$}~
\overbrace{\crd{1} \, \cdots \, \crd{1}}^{\f{deg}(1)} ~ \overbrace{\crd{2} \, \cdots \, \crd{2}}^{\f{deg}(2)} ~ \overbrace{\crd{3} \, \cdots \, \crd{3}}^{\f{deg}(3)} ~ \cdots \cdots~ \overbrace{\crd{$n$} \, \cdots \, \crd{$n$}}^{\f{deg}(n)}~.
\]
Thus the deck of helping cards is $D_{\f{help}} = \{\stext{1}, \stext{2}, \ldots, \stext{n}, 1^{\f{deg}(1)}, 2^{\f{deg}(2)}, \ldots, n^{\f{deg}(n)}\}$, where the superscript denotes the number of the symbol in the deck $D_{\f{help}}$. 
The deck $D$ is the union of $D_{\f{inp}}$ and $D_{\f{help}}$ as multisets and it consists of $2(n+m)$ symbols. 

For an input card-sequence $\f{x}\in U_{\f{inp}}$, our protocol proceeds as follows:

\begin{enumerate}
\item[(1)] Place the cards as follows.
\[
\underbrace{\back \, \back \, \cdots \, \back}_{\f{x}} ~ 
\underbrace{\crd{$\stext{1}$} \, \crd{$\stext{2}$} \, \crd{$\stext{3}$} ~\cdots ~\crd{$\stext{n}$}~
\crd{1} \, \cdots \, \crd{1} ~ \crd{2} \, \cdots \, \crd{2} ~ \crd{3} \, \cdots \, \crd{3} ~ \cdots \cdots~ \crd{$n$} \, \cdots \, \crd{$n$}}_{\f{h}}~. 
\]
\item[(2)] For each $i$, we define $\f{pile}[i]$ by
\[
\f{pile}[i]= \biggl( \,\dfrac{?}{\stext{i}}, \overbrace{\dfrac{?}{i}, \ldots, \dfrac{?}{i}}^{\f{deg}(i)} \,\biggr)  = \underset{\stext{i}}{\back}\, \overbrace{\underset{i}{\back} \, \cdots \, \underset{i}{\back}}^{\f{deg}(i)}.
\]
Arrange the card-sequence as $(\f{x}, \f{pile}[1], \f{pile}[2], \f{pile}[3], \ldots, \f{pile}[n])$, that is:
\[
\underbrace{\back \, \back \, \cdots \, \back}_{\f{x}} ~ \underbrace{\back \, \back \, \cdots \, \back}_{\f{pile}[1]}~ 
\underbrace{\back \, \back \, \cdots \, \back}_{\f{pile}[2]}~ 
\underbrace{\back \, \back \, \cdots \, \back}_{\f{pile}[3]}~\cdots\cdots~ 
\underbrace{\back \, \back \, \cdots \, \back}_{\f{pile}[n]}.
\]

\item[(3)]  For each $d \in \f{Deg}_G$, we set $V_G^{(d)} = \{v_1^{(d)}, v_2^{(d)}, \ldots, v_{\ell_d}^{(d)}\}$ for all vertices with degree $d$, and apply $\f{PSS}_{(\ell_d, d+1)}$ to the card-sequence  $(\f{pile}[v_1^{(d)}], \f{pile}[v_2^{(d)}], \ldots, \f{pile}[v_{\ell_d}^{(d)}])$.
%By the definition of PSS, there is a permutation $\sigma_d\in \mathfrak{S}_{2(n+m)}$ such that $\sigma_d$ realizes $\f{PSS}_{(\ell_d, d+1)}$.
%Now, we set $\sigma := \prod_{d \in \f{Deg}_G} \sigma_d\in\mathfrak{S}_{2(n+m)}$. Notice that $\sigma$ is determined independently of the order of the multiplication of $\sigma_i$'s by the choice of $\sigma_i$.
Then we obtain a card-sequence
\[
\underbrace{\back \, \back \, \cdots \, \back}_{\f{x}} ~ \underbrace{\back \, \back \, \cdots \, \back}_{\f{pile}[\alpha_1]}~ \underbrace{\back \, \back \, \cdots \, \back}_{\f{pile}[\alpha_2]}~ \underbrace{\back \, \back \, \cdots \, \back}_{\f{pile}[\alpha_3]}~\cdots\cdots~ \underbrace{\back \, \back \, \cdots \, \back}_{\f{pile}[\alpha_n]}~.
\]
Let $\sigma \in \mathfrak{S}_n$ be the chosen permutation such that $\alpha_i = \sigma^{-1}(i)$. 
%where $\alpha_i = \sigma^{-1}(i)$. We note that the first card of $\f{pile}[\alpha_i]$ is $\dfrac{?}{\stext{\alpha_i}}$ for any $i=1,2,\ldots, n$. 

\item[(4)] For each $i\in V_G$ and $j \ra k \in E_G$, we set $\f{vertex}[i]= \left(\dfrac{?}{\stext{\alpha_i}}, \f{x}_i\right)$ and $\f{\md{edge}}[j \ra k]= \left(\dfrac{?}{\alpha_j}, \dfrac{?}{\alpha_k}\right)$, respectively.
Arrange the card-sequence\footnote{\md{Note that this rearrangement is possible without looking under the cards since the subscripts of $\alpha_i$ are public information.}} as follows:
\[
\underbrace{\back\,\back}_{\f{vertex}[1]}\,\underbrace{\back\,\back}_{\f{vertex}[2]}\,\cdots\,\underbrace{\back\,\back}_{\f{vertex}[n]}~~
\underbrace{\back\,\back}_{\f{\md{edge}}[e_1]}\,\underbrace{\back\,\back}_{\f{\md{edge}}[e_2]}\,\cdots\,\underbrace{\back\,\back}_{\f{\md{edge}}[e_m]}~,
\]
where $E_G = \{e_1, e_2, \ldots, e_m\}$. 
%$\f{vertex}[i]$ and $\f{\md{edge}}[j \ra k]$ can be }

\item[(5)] Apply $\f{PSS}_{(m+n,2)}$ to the card-sequence as follows: 
\[
%\bigg|\,\back\,\back\,\bigg|\,\back\,\back\,\bigg|\,\cdots\bigg|\,\back\,\back\,\bigg|\,\back\,\back\,\bigg|
\bigg|\,\underbrace{\back\,\back}_{\f{vertex}[1]}\,\bigg|\,\underbrace{\back\,\back}_{\f{vertex}[2]}\,\bigg|\,\cdots\bigg|\,\underbrace{\back\,\back}_{\f{\md{edge}}[e_m]}\,\bigg|
~\ra~
\back\,\back~\back\,\back~\cdots~\back\,\back~\back\,\back\,.
\]
%\md{カード列$\f{z}$に$\f{PSS}_{(m+n,2)}$を施す. このときある$\rho\in \mathfrak{S}_{n+m}$によって, $\f{z}$の上段は$\rho(\f{z}^u)$, 下段は$\rho(\f{z}^b)$に並び替えられている.}
\item[(6)] For each pile, turn over the left card, and if it is a black-card, turn over the right card. 
Then sort $n+m$ piles\footnote{\md{It is not essential the order of pairs of helping cards.}} so that the left card is in ascending order via $\preccurlyeq$ as follows:
%Here, suppose that we obtain the following card-sequence from $D'$ by sorting.
\[
\crd{$\stext{1}$}\,\back~\crd{$\stext{2}$}\,\back~\crd{$\stext{3}$}\,\back~\cdots~\crd{$\stext{n}$}\,\back~\crd{$i_1$}\,\crd{$j_1$}~\crd{$i_2$}\,\crd{$j_2$}~\crd{$i_3$}\,\crd{$j_3$}~\cdots~\crd{$i_m$}\,\crd{$j_m$}\,,
\]
where $i_1\preccurlyeq i_2 \preccurlyeq i_3 \preccurlyeq \cdots \preccurlyeq i_m$. 

\item[(7)] We define a graph $G'$ by $V_{G'} = V_G$ and $E_{G'} = \{i_1\ra j_1, i_2\ra j_2, i_3\ra j_3, \ldots, i_m\ra j_m\}$.
%Note that the graph $G'$ has become a relabelled graph of $G$, and $\sigma$ gives an isomorphism $\hat{\sigma}$ from $G'$ to $G$ such that $\hat{\sigma}_0=\sigma$.

\item[(8)] Take an isomorphism $\psi:G\to G'$, and set $\beta_i := \psi^{-1}_0(i)$. 
Let $\f{y}_i$ be the right next card of $\crd{$\stext{\beta_i}$}\,$ and $\f{y} =(\f{y}_1, \f{y}_2, \ldots, \f{y}_n)$. 
Arrange the card-sequence as follows:
%s so that the card to the right of red-card \crd{$\stext{\beta_i}$}\ is the $i$-th position from the left. 
%All other cards are placed in ascending order. 
%The output card-sequence of the input $\f{x}$ is $\f{y}$, which is the card-sequence from left to $n$-th:
\[
\underbrace{\back \, \back \, \cdots \, \back}_{\f{y}} ~ 
\underbrace{\crd{$\stext{1}$} \, \crd{$\stext{2}$} \, \crd{$\stext{3}$} ~\cdots ~\crd{$\stext{n}$}~
\crd{1} \, \cdots \, \crd{1} ~ \crd{2} \, \cdots \, \crd{2} ~ \crd{3} \, \cdots \, \crd{3} ~ \cdots \cdots~ \crd{$n$} \, \cdots \, \crd{$n$}}_{\f{h}}~. 
%\underset{\f{y}_1}{\back} \, \underset{\f{y}_2}{\back} \, \cdots \, \underset{\f{y}_n}{\back} ~ 
%\crd{$\stext{1}$} \, \crd{$\stext{2}$} \, \crd{$\stext{3}$} ~\cdots ~\crd{$\stext{n}$}~
%\crd{1} \, \cdots \, \crd{1} ~ \crd{2} \, \cdots \, \crd{2} ~ \crd{3} \, \cdots \, \crd{3} ~ \cdots \cdots~ \crd{$n$} \, \cdots \, \crd{$n$}~. 
\]
The output card-sequence for the input $\f{x}$ is $\f{y}$. 
\end{enumerate}

\begin{remark}
Regarding the number of cards, the number of cards in the proposed protocol is $2n + 2m$, of which $n + 2m$ are helping cards. 
As for the number of shuffles, it is \md{$|\f{Deg}_G|+1$}, and all of them are PSSs. 
We remark that the PSSs in Step (3) can be executed in parallel. 
\end{remark}

\begin{remark}\label{rem:iso}
In \md{Step (8)}, given two isomorphic graphs $G$ and $G'$, we need to solve the problem of finding one specific isomorphism between them. 
However, no polynomial-time algorithm for this problem has been found so far in general.
On the other hand, there exist polynomial-time algorithms to find isomorphisms for some \md{specific} graph classes. In addition, for small \md{specific} examples, an isomorphism can be computed by using a mathematical library for graph computation (e.g., Nauty \cite{Nauty}).  
\end{remark}

\subsection{Proof of correctness}

Let $\f{x} = (\f{x}_1, \f{x}_2, \ldots, \f{x}_n)$ be an input sequence and $\f{y} = (\f{y}_1, \f{y}_2, \ldots, \f{y}_n)$ a random variable of an output sequence of the protocol when $\f{x}$ is given as input. 
Fix a graph $G'$, which is defined in Step (7) following the opened result in Step (6). 
Let $\sigma \in \mathfrak{S}_n$ be a random variable of the permutation chosen by the PSSs in Step (3) such that $\alpha_i = \sigma^{-1}(i)$ for all $i \in V_G$. 
Since an \md{edge} $i \ra j \in E_G$ of $G$ corresponds to an \md{edge} $\alpha_i \ra \alpha_j = \sigma^{-1}(i) \ra \sigma^{-1}(j) \in E_{G'}$ of $G'$, there is an isomorphism $\phi = (\phi_0, \phi_1) \in \f{Iso}(G, G')$ such that $\phi_0 = \sigma^{-1}$. 
%the permutation $\sigma^{-1}$ induces an isomorphism from $G$ to $G'$, i.e., $\sigma^{-1} \in \f{Iso}(G, G')$. 
From the property of the PSSs, the permutation $\phi_0$ is a uniform random variable on $\f{Iso}_0(G, G')$. 
Let $\psi = (\psi_0, \psi_1) \in \f{Iso}(G', G)$ be a random variable of the isomorphism chosen in Step (9). 

We first claim that $\f{y} = \psi_0 \circ \phi_0 (\f{x})$. 
This is shown by observing a sequence of red cards $\crd{$\stext{1}$} \, \crd{$\stext{2}$} \, \crd{$\stext{3}$} \, \cdots \,\crd{$\stext{n}$}\,$. 
Hereafter, for the sake of clarity, we do not distinguish face-up $\dfrac{\stext{i}}{?}$ and face-down $\dfrac{?}{\stext{i}}$ and use ``$\stext{i}$" to denote the red card $i$. 
In Step (1), the sequence of red cards omitting other cards is $(\stext{1}, \stext{2}, \ldots, \stext{n})$. 
In Steps (3), (6), and (8), it is arranged as follows:
\[
\underset{\text{Step (1)}}{(\stext{1}, \stext{2}, \ldots, \stext{n})}
\xra{\phi_0^{-1}}
\underset{\text{Step (3)}}{(\stext{\alpha_1}, \stext{\alpha_2}, \ldots, \stext{\alpha_n})}
\xra{\phi_0}
\underset{\text{Step (6)}}{(\stext{1}, \stext{2}, \ldots, \stext{n})}
\xra{\psi_0}
\underset{\text{Step (8)}}{(\stext{\beta_1}, \stext{\beta_2}, \ldots, \stext{\beta_n})}.
\]
Since the input sequence $\f{x}$ is arranged as $((\stext{\alpha_1}, \f{x}_1), (\stext{\alpha_2}, \f{x}_2), \ldots, (\stext{\alpha_n}, \f{x}_n))$ in Step (3), the permutation $\psi_0 \circ \phi_0$ is applied to $\f{x}$. 
Thus, it holds $\f{y} = \psi_0 \circ \phi_0 (\f{x})$. 
We note that $\psi \circ \phi_0$ is an automorphism of $G$.  

%Now we prove the security of our protocol. 
%In order to prove the security, it is sufficient to show that a distribution of the opened symbols in Step (6) is independent of the distribution of $\psi\sigma^{-1} \in \aut(G)$ since cards are opened in Step (6) only. 

It remains to prove that the distribution of $\psi_0\circ\phi_0 \in \aut_0(G)$ is uniformly random. 
We note that given the graph $G'$, the distributions of $\phi_0$ and $\psi_0$ are independent. 
This is because the choice of $\psi_0$ depends on the opened symbols in Step (6) only, and they are independent of $\phi_0$ due to the PSS in Step (5). 
Thus, we can change the order of choice without harming the distributions of $\phi_0, \psi_0$: first, $\psi_0$ is chosen, and then $\phi_0$ is chosen. 
Since the distribution of $\phi_0 \in \f{Iso}_0(G, G')$ is uniformly random, it is sufficient to show that the function
\[ 
\begin{array}{cccc}
\Phi:  & \f{Iso}_0(G, G') &  \longrightarrow &  \aut_0(G) \\
           & \phi_0                                                    & \longmapsto & \psi_0 \circ\phi_0
 \end{array}
 \]
is bijective. 

We first prove that $\Phi$ is injective. 
Suppose that $\Phi(\phi'_0) = \Phi(\phi''_0)$ for some $\phi'_0, \phi''_0 \in \f{Iso}_0(G, G')$, that is,  $\psi_0 \circ\phi'_0 = \psi_0 \circ\phi''_0$. 
Since $\psi_0$ is a bijection, $\phi'_0 = \phi''_0$ holds.
Thus $\Phi$ is injective. 
We next prove that $\Phi$ is surjective. 
For any $\tau\in\aut_0(G)$, we have 
\[ \tau = \psi_0\circ\psi_0^{-1}\circ \tau = \Phi (\psi_0^{-1}\circ \tau). \] 
It yields that $\Phi$ is surjective. 
Therefore, $\Phi$ is bijective. 

This shows that the distribution of $\psi \circ \sigma^{-1}$ is uniformly random, and hence our protocol is correct. 

%\begin{remark}
%If $G$ has multiple \md{edges}, then there are many ways to extend the $\sigma^{-1}$ given above to an isomorphism  from $G$ to $G'$.
%However, the proof of correctness is independent of the way $\sigma^{-1}$ is extended.
%\end{remark}

\subsection{Proof of security}

%Let $\f{x} = (\f{x}_1, \f{x}_2, \ldots, \f{x}_n)$ and $\f{y} = (\f{y}_1, \f{y}_2, \ldots, \f{y}_n)$ be an input and output sequence in the protocol. 
%We use the same notations as in the proof of the correctness. 
In the proof of the correctness, we have already claimed that the distribution of the opened symbols in Step (6) is independent of $\sigma$ due to the PSS in Step (5). 
Since cards are opened in Step (6) only, this shows a distribution of the permutation $\psi\circ \sigma^{-1} \in \aut(G)$ is independent of the distribution of the visible sequence-trace of our protocol. 
Therefore, our protocol is secure. 

\subsection{Example of our protocol for a graph}
%\begin{example}
Let $G$ be a directed graph with $5$ vertices as follows:
\[ G=  \begin{xy}
                     (0,8)*[o]+{1}="1",(0,-8)*[o]+{2}="2",(12,0)*[o]+{3}="3",(24,8)*[o]+{4}="4",(24,-8)*[o]+{5\md{.}}="5",
                     \ar @<1mm>"1";"2"^{e_1}
                     \ar @<1mm>"2";"1"^{e_3}
                     \ar "1";"3"^{e_2}
                     \ar "2";"3"_{e_4}
                     \ar "3";"4"^{e_5}
                     \ar "3";"5"_{e_6}
            \end{xy} \]
We perform our graph shuffle protocol for $G$. 
Let $D_{\f{inp}}$ be an arbitrary deck with $D_{\f{inp}} = \{x_1, x_2, x_3, x_4, x_5\}$. 
The card-sequence $\f{h}$ of helping cards is defined as follows:
%In this case, helping cards are the following 17 cards. 
\[
\f{h} = \crd{$\stext{1}$} \, \crd{$\stext{2}$} \, \crd{$\stext{3}$} \, \crd{$\stext{4}$} \, \crd{$\stext{5}$}~
\crd{1} \, \crd{1} \, \crd{1}  ~ \crd{2} \, \crd{2} \, \crd{2} ~ \crd{3} \, \crd{3} \, \crd{3} \, \crd{3} ~  \crd{4} ~ \crd{5}\,\md{.}
\]
Set $D_{\f{help}} = \{\stext{1}, \stext{2}, \stext{3}, \stext{4}, \stext{5},1, 1, 1, 2, 2, 2, 3, 3, 3, 3, 4, 5\}$ and $D=D_{\f{inp}}\cup D_{\f{help}}$. 
% = \{x_1, x_2, x_3, x_4, x_5, \md{1}, \md{2}, \md{3}, \md{4}, \md{5},1, 1, 1, 2, 2, 2, 3, 3, 3, 3, 4, 5\}$.
For an input card-sequence $\f{x} = (\f{x}_1, \f{x}_2, \f{x}_3, \f{x}_4, \f{x}_5) \in U$, the graph shuffle protocol proceeds as follows:

\begin{enumerate}
\item[(1)] Place the cards such as:
\[
\underbrace{\underset{\f{x}_1}{\back} \, \underset{\f{x}_2}{\back} \, \underset{\f{x}_3}{\back} \, \underset{\f{x}_4}{\back} \, \underset{\f{x}_5}{\back}}_{\f{x}} ~ 
\underbrace{\crd{$\stext{1}$} \, \crd{$\stext{2}$} \, \crd{$\stext{3}$} \, \crd{$\stext{4}$} \, \crd{$\stext{5}$}~
\crd{1} \, \crd{1} \, \crd{1}  ~ \crd{2} \, \crd{2} \, \crd{2} ~ \crd{3} \, \crd{3} \, \crd{3} \, \crd{3} ~  \crd{4} ~ \crd{5}}_{\f{h}}~. 
\]
\item[(2)] Arrange the card-sequence as follows:
\[
\underbrace{\underset{\f{x}_1}{\back} \, \underset{\f{x}_2}{\back} \, \underset{\f{x}_3}{\back} \, \underset{\f{x}_4}{\back} \, \underset{\f{x}_5}{\back}}_{\f{x}} ~ \underbrace{\underset{\stext{1}}{\back} \, \underset{1}{\back} \, \underset{1}{\back} \, \underset{1}{\back}}_{\f{pile}[1]}~ 
\underbrace{\underset{\stext{2}}{\back} \, \underset{2}{\back} \, \underset{2}{\back} \, \underset{2}{\back}}_{\f{pile}[2]}~ 
\underbrace{\underset{\stext{3}}{\back} \, \underset{3}{\back} \, \underset{3}{\back} \, \underset{3}{\back} \, \underset{3}{\back}}_{\f{pile}[3]}~ 
\underbrace{\underset{\stext{4}}{\back} \, \underset{4}{\back}}_{\f{pile}[4]}~ 
\underbrace{\underset{\stext{5}}{\back} \, \underset{5}{\back}}_{\f{pile}[5]}~.
\]

\item[(3)] Perform $\f{PSS}_{(2, 4)}$ and $\f{PSS}_{(2, 2)}$ as follows:
\begin{align*}
&\bigg|~
\underset{\stext{1}}{\back} \, \underset{1}{\back} \, \underset{1}{\back} \, \underset{1}{\back}
~\bigg|~
\underset{\stext{2}}{\back} \, \underset{2}{\back} \, \underset{2}{\back} \, \underset{2}{\back}
~\bigg|
~~\ra~~
\underset{\stext{\alpha_1}}{\back} \,\underset{\alpha_1}{\back} \, \underset{\alpha_1}{\back} \,\underset{\alpha_1}{\back}~~
\underset{\stext{\alpha_2}}{\back} \, \underset{\alpha_2}{\back} \, \underset{\alpha_2}{\back} \, \underset{\alpha_2}{\back}~,\\
%
&\bigg|~
\underset{\stext{4}}{\back} \, \underset{4}{\back}
~\bigg|~
\underset{\stext{5}}{\back} \, \underset{5}{\back}
~\bigg|
~~\ra~~
\underset{\stext{\alpha_4}}{\back} \, \underset{\alpha_4}{\back}~~
\underset{\stext{\alpha_5}}{\back} \, \underset{\alpha_5}{\back}~.
\end{align*}
By setting $\alpha_3 = 3$, we have the following card-sequence:
\[
\underbrace{\underset{\f{x}_1}{\back} \, \underset{\f{x}_2}{\back} \, \underset{\f{x}_3}{\back} \, \underset{\f{x}_4}{\back} \, \underset{\f{x}_5}{\back}}_{\f{x}} ~ \underbrace{\underset{\stext{\alpha_1}}{\back} \, \underset{\alpha_1}{\back} \, \underset{\alpha_1}{\back} \, \underset{\alpha_1}{\back}}_{\f{pile}[\alpha_1]}~ 
\underbrace{\underset{\stext{\alpha_2}}{\back} \, \underset{\alpha_2}{\back} \, \underset{\alpha_2}{\back} \, \underset{\alpha_2}{\back}}_{\f{pile}[\alpha_2]}~ 
\underbrace{\underset{\stext{\alpha_3}}{\back} \, \underset{\alpha_3}{\back} \, \underset{\alpha_3}{\back} \, \underset{\alpha_3}{\back} \, \underset{\alpha_3}{\back}}_{\f{pile}[\alpha_3]}~ 
\underbrace{\underset{\stext{\alpha_4}}{\back} \, \underset{\alpha_4}{\back}}_{\f{pile}[\alpha_4]}~ 
\underbrace{\underset{\stext{\alpha_5}}{\back} \, \underset{\alpha_5}{\back}}_{\f{pile}[\alpha_5]}~.
\]
%For each $d \in \f{Deg}_G$, we set $V_G^{(d)} = \{v_1^{(d)}, v_2^{(d)}, \ldots, v_{\ell_d}^{(d)}\}$ for all vertices with degree $d$.
%Apply $\f{PSS}_{(\ell_d, d+1)}$ to a card-sequence  $(\f{pile}[v_1^{(d)}], \f{pile}[v_2^{(d)}], \ldots, \f{pile}[v_{\ell_d}^{(d)}])$.
%Then we obtain a card-sequence
%\[
%\underbrace{\back \, \back \, \cdots \, \back}_{\f{x}} ~ \underbrace{\back \, \back \, \cdots \, \back}_{\f{pile}[\alpha_1]}~ \underbrace{\back \, \back \, \cdots \, \back}_{\f{pile}[\alpha_2]}~ \underbrace{\back \, \back \, \cdots \, \back}_{\f{pile}[\alpha_3]}~\cdots\cdots~ \underbrace{\back \, \back \, \cdots \, \back}_{\f{pile}[\alpha_n]}~.
%\]
%Let $\sigma \in \mathfrak{S}_n$ be the chosen permutation such that $\alpha_i = \sigma^{-1}(i)$. 

\item[(4)] Arrange the card-sequence as follows:
\[
\underbrace{\underset{\f{x}_1}{\back}\,\underset{\stext{\alpha_1}}{\back}}_{\f{vertex}[1]}\,
\underbrace{\underset{\f{x}_2}{\back}\,\underset{\stext{\alpha_2}}{\back}}_{\f{vertex}[2]}\,
\underbrace{\underset{\f{x}_3}{\back}\,\underset{\stext{\alpha_3}}{\back}}_{\f{vertex}[3]}\,
\underbrace{\underset{\f{x}_4}{\back}\,\underset{\stext{\alpha_4}}{\back}}_{\f{vertex}[4]}\,
\underbrace{\underset{\f{x}_5}{\back}\,\underset{\stext{\alpha_5}}{\back}}_{\f{vertex}[5]}~~
\underbrace{\underset{\alpha_1}{\back}\,\underset{\alpha_2}{\back}}_{\f{\md{edge}}[1\ra 2]}\,
\underbrace{\underset{\alpha_1}{\back}\,\underset{\alpha_3}{\back}}_{\f{\md{edge}}[1\ra 3]}\,
\underbrace{\underset{\alpha_2}{\back}\,\underset{\alpha_1}{\back}}_{\f{\md{edge}}[2\ra 1]}\,
\underbrace{\underset{\alpha_2}{\back}\,\underset{\alpha_3}{\back}}_{\f{\md{edge}}[2\ra 3]}\,
\underbrace{\underset{\alpha_3}{\back}\,\underset{\alpha_4}{\back}}_{\f{\md{edge}}[3\ra 4]}\,
\underbrace{\underset{\alpha_3}{\back}\,\underset{\alpha_5}{\back}}_{\f{\md{edge}}[3\ra 5]}\,.
\]

\item[(5)] Apply $\f{PSS}_{(11,2)}$ to the card-sequence as follows: 
\[
%\bigg|\,\back\,\back\,\bigg|\,\back\,\back\,\bigg|\,\cdots\bigg|\,\back\,\back\,\bigg|\,\back\,\back\,\bigg|
\bigg|\underbrace{\underset{\f{x}_1}{\back}\,\underset{\stext{\alpha_1}}{\back}}_{\f{vertex}[1]}
\bigg|\underbrace{\underset{\f{x}_2}{\back}\,\underset{\stext{\alpha_2}}{\back}}_{\f{vertex}[2]}
\bigg|\underbrace{\underset{\f{x}_3}{\back}\,\underset{\stext{\alpha_3}}{\back}}_{\f{vertex}[3]}
\bigg|\underbrace{\underset{\f{x}_4}{\back}\,\underset{\stext{\alpha_4}}{\back}}_{\f{vertex}[4]}
\bigg|\underbrace{\underset{\f{x}_5}{\back}\,\underset{\stext{\alpha_5}}{\back}}_{\f{vertex}[5]}
\bigg|\underbrace{\underset{\alpha_1}{\back}\,\underset{\alpha_2}{\back}}_{\f{\md{edge}}[1\ra 2]}
\bigg|\underbrace{\underset{\alpha_1}{\back}\,\underset{\alpha_3}{\back}}_{\f{\md{edge}}[1\ra 3]}
\bigg|\underbrace{\underset{\alpha_2}{\back}\,\underset{\alpha_1}{\back}}_{\f{\md{edge}}[2\ra 1]}
\bigg|\underbrace{\underset{\alpha_2}{\back}\,\underset{\alpha_3}{\back}}_{\f{\md{edge}}[2\ra 3]}
\bigg|\underbrace{\underset{\alpha_3}{\back}\,\underset{\alpha_4}{\back}}_{\f{\md{edge}}[3\ra 4]}
\bigg|\underbrace{\underset{\alpha_3}{\back}\,\underset{\alpha_5}{\back}}_{\f{\md{edge}}[3\ra 5]}\bigg|\,.
\]

\item[(6)] For each pile, turn over the left card, and if it is a black-card, turn over the right card. 
The following card-sequence is an example outcome:
\[
\crd{$\stext{5}$}\,\back~
\crd{1}\,\crd{3}~
\crd{2}\,\crd{3}~
\crd{$\stext{4}$}\,\back~
\crd{$\stext{2}$}\,\back~
\crd{2}\,\crd{1}~
\crd{$\stext{1}$}\,\back~
\crd{3}\,\crd{5}~
\crd{3}\,\crd{4}~
\crd{1}\,\crd{2}~
\crd{$\stext{3}$}\,\back\,.
\]
Sort $11$ piles so that the left card is in ascending order via $\preccurlyeq$ as follows:
\[
\crd{$\stext{1}$}\,\underset{\f{y'}_1}{\back}~\crd{$\stext{2}$}\,\underset{\f{y'}_2}{\back}~\crd{$\stext{3}$}\,\underset{\f{y'}_3}{\back}~\crd{$\stext{4}$}\,\underset{\f{y'}_4}{\back}~\crd{$\stext{5}$}\,\underset{\f{y'}_5}{\back}~
\crd{1}\,\crd{3}~\crd{1}\,\crd{2}~\crd{2}\,\crd{3}~\crd{2}\,\crd{1}~\crd{3}\,\crd{5}~\crd{3}\,\crd{4}\,.~
\]

\item[(7)] Define a graph $G'$ by $V_{G'} = \{1, 2, 3, 4, 5\}$ and $E_{G'} = \{1\ra3, 1\ra2, 2\ra3, 2\ra1, 3\ra4, 3\ra5\}$;
\[ G'=  \begin{xy}
                     (0,8)*[o]+{1}="1",(0,-8)*[o]+{2}="2",(12,0)*[o]+{3}="3",(24,8)*[o]+{4}="4",(24,-8)*[o]+{5.}="5",
                     \ar @<1mm>"1";"2"^{}
                     \ar @<1mm>"2";"1"^{}
                     \ar "1";"3"^{}
                     \ar "2";"3"_{}
                     \ar "3";"4"^{}
                     \ar "3";"5"_{}
            \end{xy}\]

\item[(8)] Take an isomorphism $\psi: G\to G'$ defined by 
\[
1\longmapsto 2,\quad 2\longmapsto 1,\quad 3\longmapsto 3 ,\quad 4\longmapsto 4,\quad 5\longmapsto 5.
\]
Arrange the above card-sequence as follows:
\[
\underset{\f{y'}_2}{\back} \, \underset{\f{y'}_1}{\back} \, \underset{\f{y'}_3}{\back} \, \underset{\f{y'}_4}{\back} \, \underset{\f{y'}_5}{\back}~ 
\underbrace{\crd{$\stext{1}$} \, \crd{$\stext{2}$} \, \crd{$\stext{3}$} \, \crd{$\stext{4}$} \, \crd{$\stext{5}$}~
\crd{1} \, \crd{1} \, \crd{1}  ~ \crd{2} \, \crd{2} \, \crd{2} ~ \crd{3} \, \crd{3} \, \crd{3} \, \crd{3} ~  \crd{4} ~ \crd{5}}_{\f{h}}~. 
\]
The output card-sequence for the input $\f{x}$ is $(\f{y'}_2, \f{y'}_1, \f{y'}_3, \f{y'}_4, \f{y'}_5)$. 
\end{enumerate}

\subsection{Implication of our protocol}\label{ss:implication}

\md{In this subsection, we consider several interesting graph shuffles.}
% and give the numbers of cards and PSSs required in our protocol. }

\md{
%We remark that many standard shuffles are contained in the class of graph shuffles. 
%As a trivial case, a PSS for $n$ cards is equivalent to a graph shuffle for a graph with $n$ vertices and no edges. 
We first observe that a RC for $n$ cards are graph shuffles for the directed $n$-cycle graph $\overset{\ra}{C_n}$ (see Section \ref{4-1}). 
%a RC for $n$ cards is equivalent to a graph shuffle for the \md{directed} $n$-cycle graph $\overset{\ra}{C_n}$ (see Section \ref{4-1}). 
Since it holds $2n + 2m = 4n$ and $|\f{Deg}_G|+1 = 2$, a RC can be done by $4n$ cards and two PSSs. 
In Section \ref{4-1}, the number of cards is improved to $3n$.
We remark that our graph shuffle protocol works even for a sequence of piles each having equivalent number of face-down cards. 
Thus a pile-shifting shuffle (i.e., a pile-version of RC) can be done by the same number of helping cards. 
In particular, for a pile-shifting shuffle for $n$ piles of $m$ cards, it can be done by $nm + 3n$ cards and two PSSs. 
We note that PSSs and RBCs are graph shuffles for graphs with no edges in this sense.}

\md{A graph shuffle for the undirected $n$-cycle graph $C_n$ is equivalent to the \emph{dihedral shuffle}, which is introduced by Niemi and Renvall \cite{Niemi98}. 
Since it holds $2n + 2m = 5n$ and $|\f{Deg}_G|+1 = 2$, our result implies that a RC can be done by $5n$ cards and two PSSs. 
In Section \ref{sec:dihedral}, the number of cards is improved to $3n$, although the number of PSSs is increased to three.}

\md{For a cyclic group $\Pi = \langle (1\;2)(3\;4\;5\;6) \rangle$, a uniform closed shuffle $(\shuffle, \Pi, \F)$ is a graph shuffle for $G$ where $V_G = \{1, 2, 3, 4, 5, 6\}$ and $E_G = E_1 \cup E_2 \cup E_3$ with $E_1 = \{1\to 2, 2 \to 1\}$, $E_2 = \{3\to 4, 4\to 5, 5\to 6, 6\to 4\}$, and $E_3 = \{1 \to 3, 1\to 5, 2\to 4, 2 \to 6\}$. 
Since it holds $\mathsf{Aut}_0(G) =  \langle (1\;2)(3\;4\;5\;6) \rangle$, we can conclude that a graph shuffle for $G$ is equivalent to a uniform closed shuffle $(\shuffle, \Pi, \F)$. 
Since it holds $2n + 2m = 32$ and $|\f{Deg}_G|+1 = 3$, our result implies that it can be done by $32$ cards and three PSSs. 
By generalizing this idea, for any cyclic group $\Pi = \langle \pi \rangle$, a uniform closed shuffle $(\shuffle, \Pi, \F)$ is a graph shuffle for some graph. 
}
%We can observe that $G$ has two cycles of length $2$ defined by $E_1$ and length $4$ defined by $E_2$. 
%These cycles are corresponding to two cyclic permutations $(1\;2)$ and $(3\;4\;5\;6)$, respectively. 
%The subset $E_1$ defines a cycle of length $2$ and the subset $E_2$ defines a cycle of length $4$. 
%We note that the uniform closed shuffle is not equivalent to two RCs since two cyclic permutations $(1\;2)$ and $(3\;4\;5\;6)$ must be ``synchronized". 
%An interesting nontrivial 
%For any cyclic group $\Pi \subset \mathfrak{S}_n$, a uniform closed shuffle $(\shuffle, \Pi)$ is a graph shuffle for some graph. 
%We as follows. 
%An interesting observation is that every uniform closed shuffle $(\shuffle, \Pi, \F)$ for a cyclic group $\Pi$ is contained in the class of graph shuffles. 
%Firstly, consider a simple case of $\Pi = \langle (1\;2)(3\;4\;5\;6) \rangle$. 
% is generated by a permutation $(1\;2)(3\;4\;5\;6)$. 
%(Note that when cycle lengths are coprime, e.g., $\Pi = \langle (1\;2)(3\;4\;5) \rangle$, the uniform closed shuffle can be done by applying two shuffles for $(1\;2)$ and $(3\;4\;5)$ independently.)