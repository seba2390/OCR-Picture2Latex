\documentclass[tikz]{amsart}
\usepackage[dvips]{graphicx}
\usepackage{amsmath,amssymb,amsthm,bm,fancybox,fullpage,dashbox}
\usepackage{bm}
\usepackage[all]{xy}
\usepackage{cite}
\usepackage{verbatim}
\usepackage{enumerate}
\usepackage{url}
\usepackage{color}
\usepackage{tikz-cd, mathdots}
\newcommand{\pnt}[1]{\stackrel{#1}{\bullet}}
\usepackage{setspace}


\theoremstyle{definition}
\newtheorem*{maintheorem}{Theorem}
\newtheorem*{ques}{Question}
\newtheorem*{theo}{Theorem}
\newtheorem{theorem}{Theorem}[section]
\newtheorem{corollary}[theorem]{Corollary}
\newtheorem{lemma}[theorem]{Lemma}
\newtheorem{proposition}[theorem]{Proposition}
\newtheorem{definition}[theorem]{Definition}
\newtheorem{remark}[theorem]{Remark}
\newtheorem{example}[theorem]{Example}
\newtheorem{question}[theorem]{Question}
\newtheorem*{acknowledgement}{{\bf Acknowledgement}}

\theoremstyle{remark}
\newtheorem{assumption}[theorem]{Assumption}

% General symbols
\newcommand{\bin}{\{0,1\}}
\newcommand{\N}{\mathbb{N}}
\newcommand{\Z}{\mathbb{Z}}
\newcommand{\ra}{\rightarrow}
\newcommand{\la}{\leftarrow}
\newcommand{\xra}[1]{\xrightarrow{#1}}
\newcommand{\lra}{\longrightarrow}
\newcommand{\lla}{\longleftarrow}
\newcommand{\Ra}{\Rightarrow}
\newcommand{\La}{\Leftarrow}
\newcommand{\ol}{\overline}
\newcommand{\wt}{\widetilde}
\def\qed{\hfill $\Box$}
\newcommand{\degree}{\mathrm{deg}}
\newcommand{\outdegree}{\mathrm{out}}
\newcommand{\indegree}{\mathrm{in}}
\newcommand{\aut}{\mathsf{Aut}}
%\newcommand{\stext}[1]{\hat{#1}}
%\newcommand{\stext}[1]{#1^{*}}
%\newcommand{\stext}[1]{\textcolor{red}{#1}}
\newcommand{\stext}[1]{\overline{#1}}
\newcommand{\PSS}{\mathsf{PSS}}
\newcommand{\turn}{\mathsf{turn}}
\newcommand{\shuffle}{\mathsf{shuffle}}
\newcommand{\F}{\mathcal{F}}
\newcommand{\id}{\mathsf{id}}
\newcommand{\PT}{\mathcal{P}}
\newcommand{\act}{\mathsf{act}}

% Cards
\newlength{\ralen}
\setlength{\ralen}{-0.8ex}
\setlength{\unitlength}{1truept}
\newcommand{\blk}{\raisebox{\ralen}{\framebox(11,13){{\small $\clubsuit$}}}}
\newcommand{\red}{\raisebox{\ralen}{\framebox(11,13){{\small $\heartsuit$}}}}
\newcommand{\dia}{\raisebox{\ralen}{\framebox(11,13){{\small $\diamondsuit$}}}}
\newcommand{\spa}{\raisebox{\ralen}{\framebox(11,13){{\small $\spadesuit$}}}}
\newcommand{\back}{\raisebox{\ralen}{\framebox(11,13){{\large \textbf{?}}}}}
\newcommand{\redback}{\raisebox{\ralen}{\framebox(11,13){{\large \textbf{\textcolor{red}{?}}}}}}
\newcommand{\crd}[1]{\raisebox{\ralen}{\framebox(11,13){#1}}}
\newcommand{\heart}{\heartsuit}
\newcommand{\club}{\clubsuit}

% Trump
\newcommand{\ace}{\raisebox{\ralen}{\framebox(11,13){A}}}
\newcommand{\two}{\raisebox{\ralen}{\framebox(11,13){2}}}
\newcommand{\thr}{\raisebox{\ralen}{\framebox(11,13){3}}}
\newcommand{\fou}{\raisebox{\ralen}{\framebox(11,13){4}}}
\newcommand{\fiv}{\raisebox{\ralen}{\framebox(11,13){5}}}
\newcommand{\six}{\raisebox{\ralen}{\framebox(11,13){6}}}
\newcommand{\sev}{\raisebox{\ralen}{\framebox(11,13){7}}}
\newcommand{\eig}{\raisebox{\ralen}{\framebox(11,13){8}}}
\newcommand{\nin}{\raisebox{\ralen}{\framebox(11,13){9}}}
\newcommand{\ten}{\raisebox{\ralen}{\framebox(11,13){10}}}
\newcommand{\jac}{\raisebox{\ralen}{\framebox(11,13){J}}}
\newcommand{\que}{\raisebox{\ralen}{\framebox(11,13){Q}}}
\newcommand{\kin}{\raisebox{\ralen}{\framebox(11,13){K}}}
\newcommand{\joe}{\raisebox{\ralen}{\framebox(11,13){Jo}}}

\newcommand{\yes}{\raisebox{\ralen}{\framebox(11,13){$\mathsf{Y}$}}}
\newcommand{\no}{\raisebox{\ralen}{\framebox(11,13){$\mathsf{N}$}}}

\newcommand{\alice}{\raisebox{\ralen}{\framebox(11,13){$\mathsf{A}$}}}
\newcommand{\bob}{\raisebox{\ralen}{\framebox(11,13){$\mathsf{B}$}}}
\newcommand{\carol}{\raisebox{\ralen}{\framebox(11,13){$\mathsf{C}$}}}

% Encoding
\newcommand{\encA}{{\sf E}}
\newcommand{\encB}{{\sf B}}

% Permutation
\newcommand{\idcard}{{\bf e}}

% Modification
\newcommand{\md}[1]{\textcolor{black}{#1}}
\newcommand{\f}[1]{\mathsf{#1}}
\definecolor{Green}{rgb}{0.1, 0.5, 0.1}

\pagestyle{plain}

\begin{document}
\title{Graph Automorphism Shuffles from Pile-Scramble Shuffles}

\author[K. Miyamoto]{Kengo Miyamoto}
\author[K. Shinagawa]{Kazumasa Shinagawa}
%\address[K. Miyamoto, K. Shinagawa]{Ibaraki University, 4-12-1 Nakanarusawa, Hitachi, Ibaraki, Japan.}
\address[K. Miyamoto]{Ibaraki University, 4-12-1 Nakanarusawa, Hitachi, Ibaraki, 316-8511, Japan.}
\email{kengo.miyamoto.uz63@vc.ibaraki.ac.jp}
\address[K. Shinagawa]{Ibaraki University, 4-12-1 Nakanarusawa, Hitachi, Ibaraki, 316-8511, Japan; National Institute of Advanced Industrial Science and Technology (AIST), Tokyo Waterfront Bio-IT Research Building 2-4-7 Aomi, Koto-ku, Tokyo, 135-0064, Japan.}
%\address[K. Shinagawa]{National Institute of Advanced Industrial Science and Technology (AIST), Tokyo Waterfront Bio-IT Research Building 2-4-7 Aomi, Koto-ku, Tokyo, 135-0064, Japan.}
\email{shinagawakazumasa@gmail.com}
%\author{Kengo Miyamoto and Kazumasa Shinagawa}

\maketitle
\begin{abstract}
A pile-scramble shuffle is one of the most effective shuffles in card-based cryptography. 
Indeed, many card-based protocols are constructed from pile-scramble shuffles. 
This article aims to study the power of pile-scramble shuffles. 
In particular, for any directed graph $G$, we introduce a new protocol called ``a graph shuffle protocol for $G$'', and show that it \md{can be implemented} by using pile-scramble shuffles only. 
Our proposed protocol requires $2(n+m)$ cards, where $n$ and $m$ are the numbers of vertices and arrows of $G$, respectively. 
The number of pile-scramble shuffles is $k+1$, where $1 \leq k \leq n$ is the number of distinct degrees of vertices of $G$. 
As an application, a random cut for $n$ cards, which is also an important shuffle, can be realized by \md{$3n$} cards and \md{two} pile-scramble shuffles. 
\end{abstract}
\keywords{Secure computation; Card-based cryptography; Pile-scramble shuffles; Graph automorphisms}


%%%% Introduction %%%%%
\section{Introduction}\label{sec:introduction}

%\textit{Expand on the purpose of the paper and contextualise. Most crucially, provide and discuss reference to recent work in terms of motivating the need for the HuCI. Clarify we include arts. Introduce GLAM as a whole sector.}

%\vspace{0.5cm}
%Introduce terminology:
%\begin{itemize}
    %\item HuCI is the Arts\&Humanities citation corpus. HuCI will exist as a virtual corpus, materialized by several co-existing repositories (e.g. OpenCitations, Wikidata) providing access to their citation data via SPARQL and other endpoints. HuCI is, in this sense, fully distributed and existing by means of an infrastructure proposed in what follows.
    %\item Scholar Index is the application layer which will allow to distribute the creation and curation of citation data via a digital library application embedding the necessary machine learning components, and to centrally expose them via a citation index.
%\end{itemize}

Citation  indexes  are  by  now  part  of  the  research  infrastructure  in  use  by  most  scientists:  a  necessary  tool  in  order  to  cope  with  the  increasing  amounts  of  scientific  literature  being  published. However, existing commercial  citation  indexes  are  designed  for  the  sciences  and  have  uneven  coverage  and  unsatisfactory  characteristics  for humanities\footnote{Throughout this paper we use the term \textit{humanities} as a shorthand for Arts \& Humanities (A\&H). To a degree, the Social Sciences are also concerned.} scholars. This situation has both discouraged the usage of citation indexes and hindered bibliometric studies of humanities disciplines.  

The creation of a citation index for the humanities may well appear as a daunting task due to several characteristics of this field, such as its fragmentation into several sub-disciplines, the common practice of publishing research in languages other than English, as well as the amount of scholarship from past centuries that is still waiting to be digitised.

Notwithstanding these challenges, we argue that the creation of such an index can be highly beneficial to humanities scholars for, at least, the following reasons. Firstly, humanities scholars have long been relying on information seeking behaviours that leverage citations and reference lists for the discovery of relevant publications -- a strategy that citation indexes are designed to support and facilitate. Secondly, a comprehensive citation index for the humanities will be a valuable source of data for researchers willing to conduct bibliometric studies of the humanities. Lastly, capturing the wealth of references to primary and secondary sources contained in humanities literature will allow to create links between archives, galleries, libraries and museums where digitized copies of these sources can increasingly be found.

Before continuing with this paper, we introduce key terminology related to citation indexing that will be used throughout this paper, adopting the definitions from \cite{peroni_opencitations_2018}. These are: bibliographic entity, bibliographic resource and bibliographic citation.
%We provide herein some definitions so as to avoid ambiguities when these terms will be mentioned in the rest of the paper. 
A \textbf{bibliographic entity} is any entity which can be part of the bibliographic metadata of a bibliographic artifact: it can be a person, an article, an identifier for a particular entity (e.g., a DOI), a particular role held by a person (e.g., being an author) in the context of defining another entity (e.g., a journal article), and so forth. A \textbf{bibliographic resource} is a kind of bibliographic entity that can cite or be cited by other bibliographic resources (e.g., a journal article), or that contains other  resources (e.g., a journal). A \textbf{bibliographic citation} is another kind of bibliographic entity: a conceptual directional link from a citing bibliographic resource to a cited bibliographic resource.
The citation data defining a particular citation must include the representation of the conceptual directional link of the citation and the basic metadata of the involved bibliographic resources, that is to say sufficient information to create or retrieve textual bibliographic references for each of the bibliographic resources. Following \cite{peroni_open_2018}, we say that a bibliographic citation is an open citation when the citation data needed to define it are compliant with the following principles: structured, separate, open, identifiable, available.

The remaining of this paper is organised as follows. In Section \ref{sec:related-work} we discuss previous work on analysing the behaviour of humanities scholars in relation to information retrieval. We also present the main limitations of existing citation indexes, seen from the perspective of the humanities, and outline the main obstacle that citation indexing has faced in this area. In Section \ref{sec:citation-index-AH-needs} we argue for the need of a Humanities Citation Index (HuCI from now onwards) and in Section \ref{sec:citation-index-characteristics} we present what we believe are the essential characteristics that such an index should have. We then propose a possible implementation of HuCI, based on a federated and distributed research infrastructure (Section \ref{sec:research-infrastructure}). We conclude with some considerations on how HuCI relates to recent efforts to create open infrastructures for research.

%%%% Preliminaries %%%%%


\section{Preliminaries}

In this section, we collect some fundamentals in card-based cryptography; see \cite{MizukiIJISEC14} for example.


\subsection{Cards}\label{ss:card}
Throughout \md{this paper}, we deal with physical \textit{cards} with the symbol ``?"  on the backs. 
\md{We use two collections of cards: \textit{black-cards} $\crd{1}\,\crd{2}\,\crd{3}\,\cdots$ and \textit{red-cards} $\crd{$\stext{1}$}\,\crd{$\stext{2}$}\,\crd{$\stext{3}$}\,\cdots$ as follows: }
%A card with a natural number $\stext{i}$ (resp. $i$) on the front is called \textit{a red-card} (resp. \textit{a black-card}). 
\[ \begin{array}{ll}
\text{front:} & \overbrace{\crd{1}~\crd{2}~\crd{3}~\crd{4}~\crd{5}~\crd{6}~\cdots}^{\text{black-cards}}~\overbrace{\crd{$\stext{1}$}~\crd{$\stext{2}$}~\crd{$\stext{3}$}~\crd{$\stext{4}$}~\crd{$\stext{5}$}~\crd{$\stext{6}$}~\cdots}^{\text{red-cards}}\, \\
\text{back:} & \back~\back~\back~\back~\back~\back~\cdots~\back~\back~\back~\back~\back~\back~\cdots.
\end{array}
\]
We distinguish between the natural number $\stext{i}$ (written in red) and the natural number $i$ (written in black). 
We denote by $\mathbb{N}^{\f{red}}$ the set of all natural numbers written in red, i.e., $\mathbb{N}^{\f{red}} = \{\stext{1}, \stext{2}, \stext{3}, \ldots\}$.
The set $\mathbb{N}^{\f{red}}$ is a totally ordered set by using the natural order on $\mathbb{N}$.
We define a totally order $\preccurlyeq$ on $\mathbb{N}\cup\mathbb{N}^{\f{red}}$ by
$\alpha \preccurlyeq \beta$ if and only if 
\begin{itemize}
\item $x, y \in \mathbb{N}$ and $x\leq y$, where $\alpha = x$ and $\beta = y$, 
\item $\stext{x}, \stext{y}\in\mathbb{N}^{\f{red}}$ and $x\leq y$, where $\alpha = \stext{x}$ and $\beta = \stext{y}$, or
\item $\stext{x}\in \mathbb{N}^{\f{red}}$ and $y\in\mathbb{N}$, where $\alpha = \stext{x}$ and $\beta = y$.
\end{itemize}
\md{\textit{A deck} $D$} is a non-empty multiset such that $\{?\}\cap D=\varnothing$. 
Let $D$ be a deck.  
An expression $\dfrac{x}{?}$ $\left(\text{resp. } \dfrac{?}{x}\right)$ with $x\in D$ is said to be \textit{a face-up card} (resp.  \textit{a face-down card}) of $D$.
\textit{A lying card} $y$ of $D$ is the face-up card $y=\dfrac{x}{?}$ of $D$ or the face-down card $y=\dfrac{?}{x}$ of $D$, and in this case, we set $\f{atom}(y)=x$.
\textit{A card-sequence} from $D$ is a list of lying cards of $D$, say $(x_1,\ldots,x_n)$, such that $\{\f{atom}(x_i)\mid i=1,2,\ldots,n\}=D$ as multisets.
For a card-sequence $\f{x}$, we write $\f{x}_i$ for the $i$-th term. 
A face-up card $\dfrac{x}{?}$ is represented by \crd{$x$}~, and a face-down card $\dfrac{?}{x}$ is represented by $\back$~.
Given a card $x$ with the expression $\dfrac{y}{z}$, we write $\f{front}(x)=y$, $\f{back}(x)=z$, and $\f{swap}(x)=\dfrac{z}{y}$.
For a card-sequence $\f{x}=(\f{x}_1,\ldots,\f{x}_n)$ and a subset $T\md{\subseteq} \{1,2,\ldots, n\}$, we define an operator $\turn_T(-)$ by
\[ 
\f{turn}_T(\f{x})=(y_1,\ldots, y_n), \quad y_i=
\left\{\begin{array}{ll}
\f{swap}(\f{x}_i) & \text{if $i\in T$,}\\
\f{x}_i  &\text{if $i\notin T$.}
\end{array}\right.
\]
The card-sequence $\f{front}(\f{x}) = (\f{front}(\f{x}_1),\ldots, \f{front}(\f{x}_n))$  is called \textit{the visible sequence} of $\f{x}$.
Let $(\mathcal{T}, \mathcal{G})$ be a pair of a collection of subsets of $\{1,2,\ldots, n\}$ (i.e., $\mathcal{T} \md{\subseteq} 2^{\{1,2,\ldots, n\}}$) and a probability distribution on $\mathcal{T}$.
Now, we also define an operation $\f{rflip}_{(\mathcal{T},\mathcal{G})}(-)$ associated with the pair $(\mathcal{T}, \mathcal{G})$ by
\[ \f{rflip}_{(\mathcal{T},\mathcal{G})}(\f{x})= \f{turn}_T(\f{x}), \]
where $T$ is chosen from $\mathcal{T}$ depending on the probability distribution $\mathcal{G}$.
\md{Note that if $\mathcal{T}=\{T\}$ with a subset $T\md{\subseteq} \{1,2,\ldots, n\}$,} then $ \f{rflip}_{(\mathcal{T},\mathcal{G})}(-)=\f{turn}_T(-)$.




\subsection{Shuffles}\label{ss:shuffle}
For a natural number $n\in\mathbb{N}$, we denote by  $\mathfrak{S}_n$ the symmetric group of degree $n$, that is, the group whose elements are all bijective maps from $\{1,2,\ldots, n\}$ to itself, and whose group multiplication is the composition of functions. An element of the symmetric group is called \textit{a permutation}.

Given a card-sequence  $\mathsf{x}=(\f{x}_1,\ldots,\f{x}_n)$ and $\sigma\in\mathfrak{S}_n$, we have a card-sequence $\sigma(\f{x})$ in the natural way:
\[ \sigma(\mathsf{x})=(\f{x}_{\sigma^{-1}(1)},\ldots, \f{x}_{\sigma^{-1}(n)}).\]

Now, we recall an operation on a card-sequence which is called a ``shuffle".
Roughly speaking, a shuffle is a probabilistic reordering operation on a card-sequence. 
Let $(\Pi,\F)$ be a pair of a subset of $\mathfrak{S}_n$ and a probability distribution on $\Pi$. 
For a card-sequence $\f{x}=(\f{x}_1,\ldots,\f{x}_n)$, an operation $\shuffle_{(\Pi,\F)}(-)$ associated with the pair $(\Pi,\F)$ is defined by
\[ \shuffle_{(\Pi,\F)}(\f{x})= \sigma(\f{x}).  \]
\md{Here, $\sigma$ is chosen according to the probability distribution $\mathcal{F}$ on $\Pi$.}
Note that when we apply a shuffle to a card-sequence, no one knows which permutation was actually chosen.
We also note that if $\Pi=\{\sigma\}$ for some $\sigma\in\mathfrak{S}_{n}$, then $ \f{shuffle}_{(\Pi,\mathcal{F})}(-)=\sigma(-)$. 

\begin{definition}\label{def:ucshuffle}
A shuffle $\shuffle_{(\Pi,\F)}$ is said to be \textit{uniform closed} if $\Pi$ is closed under the multiplication of the symmetric group, and $\F$ is the uniform distribution on $\Pi$. 
\end{definition}

All shuffles dealt with \md{this paper} are uniform closed shuffles. 

\begin{example}
\begin{enumerate}
\item[(1)] \md{
For a sequence of $\ell$ cards, suppose that a subsequence of the sequence is divided into $n$ piles of $m$ cards. (It holds $\ell \geq nm$.) 
\textit{A pile-scramble shuffle} (PSS for short) is a uniform closed shuffle that completely randomly permutes $n$ piles.}
%Let $\ell, n, m, d$ be positive integers such that $\ell \geq nm + d - 1$. 
%For a sequence of $\ell$ cards, \textit{a pile-scramble shuffle} for $n$ piles of $m$ cards starting from the $d$-th card is a uniform closed shuffle defined as follows: 
%Suppose that $\ell$ cards are divided into $n$ piles of $m$ cards where the $i$-th pile ($1 \leq i \leq n$) consists of the $j$-th cards for $(i-1)m + 1 \leq j \leq im$. 
%A uniform closed shuffle $\shuffle_{(\Pi,\F)}$ for $\ell$ cards is called \textit{a pile-scramble shuffle} (PSS for short) if there exists a natural number $n \leq \ell$ such that $\Pi$ is isomorphic to $\mathfrak{S}_n$. 
The following shuffle is an example of a PSS:
\[\mathsf{PSS}_{(3,2)}:
\left(~\crd{1}~\crd{2}~,~\crd{3}~\crd{4}~,~\crd{5}~\crd{6}~ \right) \overset{\sigma}{\longmapsto}
\begin{cases}
\left(~\crd{1}~\crd{2}~,~\crd{3}~\crd{4}~,~\crd{5}~\crd{6}~ \right)& \text{if $\sigma=\mathsf{id}$,}\\
\left(~\crd{1}~\crd{2}~,~\crd{5}~\crd{6}~,~\crd{3}~\crd{4}~ \right)& \text{if $\sigma=\mathsf{(2~3)}$,}\\
\left(~\crd{3}~\crd{4}~,~\crd{5}~\crd{6}~,~\crd{1}~\crd{2}~ \right)&\text{if $\sigma=\mathsf{(1~3~2)}$,}\\
\left(~\crd{3}~\crd{4}~,~\crd{1}~\crd{2}~,~\crd{5}~\crd{6}~ \right)& \text{if $\sigma=\mathsf{(1~2)}$,}\\
\left(~\crd{5}~\crd{6}~,~\crd{1}~\crd{2}~,~\crd{3}~\crd{4}~ \right) &\text{if $\sigma=\mathsf{(1~2~3)}$,}\\
\left(~\crd{5}~\crd{6}~,~\crd{3}~\crd{4}~,~\crd{1}~\crd{2}~ \right) & \text{if $\sigma=\mathsf{(1~3)}$.}\\
\end{cases}
\]
%In the above example, since this rearranges $3$ piles with $2$ cards, one can take $\Pi \subset \mathfrak{S}_6$ which is isomorphic to $\mathfrak{S}_3$. 
% and the rearrangement permutation is chosen from $\mathfrak{S}_3$. 
%Although it is a rearrangement of $6$ cards, we regard $\mathfrak{S}_3$ to ; for example, a permutation $\sigma = (2~3) \in \mathfrak{S}_3$ denotes the second pile, and the third pile is swapped. 
We use $\mathsf{PSS}_{(n,m)}$ to denote a PSS for $n$ piles each having $m$ cards. 
\md{We remark that $\mathsf{PSS}_{(n,m)}$ can be easily implemented by putting each pile into each physical envelope
 and then permute them.}
%Since it consists of $3$ piles each having $2$ cards, this pile-scramble shuffle is denoted by $\mathsf{PSS}_{(3,2)}$. 
%each $\sigma\in \mathfrak{S}_3$ is chosen with probability $\dfrac{1}{6}$.
%In this study, PSS plays an important role.

\item[(2)] Let $\pi_k\in\mathfrak{S}_n$ be the permutation
\[\pi_k=\begin{pmatrix}
1 & 2 & \cdots &  k & k+1 & \cdots & n \\
n-k+1 & n-k+2 & \cdots & n & 1 & \cdots & n-k
\end{pmatrix},\]
and set $\Pi=\{\pi_k\mid k=1,2,\ldots,n\}$. \md{This} uniform closed shuffle $\shuffle_{(\Pi,\F)}$ is called \textit{a \md{random} cut} (RC for short). 
\end{enumerate}
\end{example}



\subsection{Procotols}\label{ss:protocol}
Mizuki and Shizuya \cite{MizukiIJISEC14} define the formal definition of a card-based protocol via an abstract machine.
In this section, we recall the definition of a card-based protocol and introduce a shuffle protocol, which is a particular card-based protocol realizing a shuffle. 

\subsubsection{Card-based protocols}\label{sss:reduction}

To put it briefly, a ``protocol" is a Turing machine that chooses one of the following operations to be applied to a card-sequence $\f{x}$: turning $(\f{x}\mapsto \f{rflip}_{(\mathcal{T},\mathcal{G})}(\f{x}))$ or shuffling $(\f{x}\mapsto \shuffle_{(\Pi,\F)}(\f{x}))$.

For a deck $D$, the set of all card-sequences from $D$ will be denoted by $\f{Seq}^D$.
Then \textit{the visible sequence set} $\f{Vis}^D$ is defined as the set of all sequences $\f{front(x)}$ for $\f{x}\in \f{Seq}^D$.
We also define the sets of the actions:
\begin{align*}
& \f{turn}^n=\{\f{turn}_T(-)\mid T\md{\subseteq} \{1,2,\ldots, n\}\}, \\
& \f{perm}^n=\{\sigma(-)\mid \sigma\in\mathfrak{S}_n\}, \\
& \f{SP}^n=\{\shuffle_{(\Pi,\mathcal{F})}(-)\mid \text{$\mathcal{F}$ is  a probability distribution on $\Pi\in 2^{\mathfrak{S}_n}\}, \text{ and}$}\\
& \f{TP}^n=\{\f{rflip}_{(\mathcal{T},\mathcal{G})}(-)\mid \text{$\mathcal{G}$ is a probability distribution on $\mathcal{T}\md{\subseteq} 2^{\{1,2,\ldots, n\}}\}$}.
\end{align*}
%Here, we use the symbol $\f{P}(S)$ to denote the power set of a set $S$. 

A protocol is a Markov chain, that  is, a stochastic model describing a sequence of possible actions in which the probability of each action depends only on the state attained in the previous event.
Let $Q$ be a finite set with two distinguished states, which are called \textit{an initial state} $q_0$ and \textit{a final state} $q_{\rm f}$.

\begin{definition}
\textit{A card-based protocol} is a quadruple $\mathcal{P}=(D,U,Q,\f{A})$, where $U\md{\subseteq} \f{Seq}^D$ is an input set and $\f{A}$ is a partial action function
\[ 
\begin{array}{cccc}
 \f{A}:  & (Q\setminus\{q_{\rm f}\})\times \f{Vis}^D &  \longrightarrow &  Q\times (\f{turn}^n\cup\f{perm}^n\cup \f{SP}^n\cup \f{TP}^n), \\
           & (q,\f{y})                                                    & \longmapsto & (q',\f{act}_{q,\f{y}})
 \end{array} \]
which depends only on the current state and visible sequence, specifying the next state and an operation on the card-sequence from $(\f{turn}^n\cup\f{perm}^n\cup \f{SP}^n\cup \f{TP}^n)$, such that $\f{A}(q_0,\f{front(x)})$ is defined if $\f{x}\in U$.
For a state $q\in Q\setminus\{q_{\rm f}\}$ and a visible sequence $\f{y}=\f{front(x)}\in \f{Vis}^D$ such that $\f{A}(q,\f{y})=(q',\f{act}_{q,\f{y}})$, we obtain the next state $(q' ,\f{front}(\act_{q,\f{y}}(\f{x})))$.
\md{By the above process, if we have $(q_{\rm f},\mathsf{X}) \in Q\times\f{Vis}^D$ for some $\mathsf{X}\in \f{Vis}^D$,}
the protocol $\mathcal{P}$ terminates. 
\end{definition}

Let $\mathcal{P}=(D,U,Q,\f{A})$ be a card-based protocol.
%\md{Recall that $Q$ is a finite set of states.}
For an execution of $\mathcal{P}$ with an input card-sequence $\f{x}^{(0)}\in U$, we obtain a sequence of results of actions as follows:
\[ (q_0,\f{x}^{(0)})\longmapsto (q_1,\f{x}^{(1)})\longmapsto (q_2,\f{x}^{(2)})\longmapsto(\md{q_3},\f{x}^{(3)})\longmapsto \cdots,\]
where $\f{x}^{(i)}=\act_{q_{i-1},\f{front}(\f{x}^{(i-1)})}(\md{\f{x}^{(i-1)}})$ for $i\geq 1$. \md{Here, $q_i$ ($i = 0, 1, 2, \ldots$) are not necessarily distinct}. 
If \md{the action function value} $\f{A}(q_i,\f{x}^{(i)})$ is undefined for some $i \in \N$, we say that ``$\mathcal{P}$ aborts at Step $i$ in the execution". 
Note that even for the same input card-sequence $\f{x}^{(0)}$, the obtained chains may be different for each execution.
If the protocol $\mathcal{P}$ terminates for an input card-sequence $\f{x}^{(0)}$, then we have a chain of results as follows:
\[ (q_0,\f{x}^{(0)})\longmapsto (q_1,\f{x}^{(1)})\longmapsto (q_2,\f{x}^{(2)})\longmapsto(q_2,\f{x}^{(3)})\longmapsto \cdots \longmapsto (q_{\rm f},\f{x}^{(\ell)}).\]
In this case, $\f{x}^{(0)}$ is called \textit{an initial sequence}, $\f{x}^{(\ell)}$ is called \textit{a final sequence}, and the sequence 
\[ (\f{y}^{(0)},\f{y}^{(1)},\ldots, \f{y}^{(\ell)}), \]
where $\f{y}^{(i)}=\f{front}(\f{x}^{(i)})$, is called \textit{a visible sequence-trace} of $\mathcal{P}$. 
We denote by $\f{Fin}(\mathcal{P})$ the set of  all \md{final} sequences, which is obtained by $\mathcal{P}$.

%For a protocol $\mathcal{P}=(D,U,Q,\f{A})$, let $\f{x}$ be a probability distribution on the set of input sequences $U$. 
%Let $\f{v}$ be a random variable for the visible sequence-trace of $\mathcal{P}$.
%Then $\mathcal{P}$ is \textit{secure} if $\f{x}$ and $\f{v}$ are stochastically independent for any $\f{x}$ and $\f{v}$.

\begin{example}
Let us consider the following. Take the deck $D=\{1,2,3,4\}$, and hence use as follows:
\[ \begin{array}{ccccc}
\text{front:} & \crd{1}~& \crd{2}~ & \crd{3}~ & \crd{4}~\;\\
\text{back:} & \back~ & \back~   &  \back~  & \back~.
\end{array}
\]
Now, we give a card-based protocol $\mathcal{P}=\left(D, \left\{\left(\dfrac{1}{?},\dfrac{2}{?},\dfrac{3}{?},\dfrac{4}{?}\right)\right\}, \{q_0,q_1,q_2,q_{\rm f}\}, \f{A}\right)$ such that
\begin{align*}
&\f{A}\left(q_0,1234\right) = (q_1, \f{turn}_{\{1,2,3,4\}}(-)), \\
&\f{A}\left(q_1,????\right) = (q_2,(1\;3)(-)), \\
&\f{A}\left(q_2,????\right) = (q_{\rm f}, \turn_{\{3\}}(-)).
\end{align*}
In this case, the card-sequence $\crd{1}~\crd{2}~\crd{3}~\crd{4}$ is changed by the protocol $\mathcal{P}$ as follows:
\[ \crd{1}~\crd{2}~\crd{3}~\crd{4}~ \longrightarrow \overset{1}{\back}~\overset{2}{\back}~\overset{3}{\back}~\overset{4}{\back}~ \longrightarrow\overset{3}{\back}~\overset{2}{\back}~\overset{1}{\back}~\overset{4}{\back}~ \longrightarrow \overset{3}{\back}~\overset{2}{\back}~\crd{1}~\overset{4}{\back}~.\]
Thus, the final sequence is $ \back~\back~\crd{1}~\back~$.
\end{example}

%\md{プロトコルの正当性とは, プロトコルが必ず正しい計算を行う性質である. (後に回す)}

\subsubsection{Shuffle protocols}\label{sss:reduction2}

A shuffle protocol\footnote{\md{Koch and Walzer \cite{KochFUN21} considered a similar notion and proposed a protocol for any uniform closed shuffles. The main difference of their model and our model is that their model allows a randomness generation in the head (see Section \ref{relatedworks} in Introduction). }} is a card-based protocol realizing a shuffle operation. 
%We newly introduce the concept of a shuffle protocol, which brings the same effect as a shuffle operation. 
It takes a card-sequence $\f{x} = (\f{x}_1, \f{x}_2, \ldots, \f{x}_n)$ such that $\f{back}(\f{x}) = (?, ?, \ldots, ?)$ as input and outputs $\f{y} = (\f{y}_1, \f{y}_2, \ldots, \f{y}_n)$ such that $\f{y} = \sigma(\f{x})$ for a permutation $\sigma$ is chosen from $\mathfrak{S}_n$ depending on some probability distribution:
\[
\underbrace{\back \, \back \, \cdots \, \back}_{\f{x}} ~ \underbrace{\crd{$h_1$} \, \crd{$h_2$} \, \cdots \, \crd{$h_k$}}_{\f{h}} ~ 
\lra~
\underbrace{\back \, \back \, \cdots \, \back}_{\f{y}} ~ \underbrace{\crd{$h_1$} \, \crd{$h_2$} \, \cdots \, \crd{$h_k$}}_{\f{h}}~,
\]
where $\f{h}$ is a card-sequence of helping cards. 
%\md{Here, the asterisk symbol ``$*$" denotes some natural number in $\mathbb{N}\cup\mathbb{N}^{\f{red}}$ and $\crd{$*$}$ means some face-up card.} 
\md{Informally speaking, the correctness requires $\mathsf{y}=\mathsf{shuffle}_{(\Pi,\mathcal{F})}(\mathsf{x})$ and the security requires that no one learns nothing about the chosen permutation $\sigma \in \Pi$.}

\begin{definition}
Let $D_{\f{inp}}, D_{\f{help}}$ be decks, $U_{\f{inp}}$ an input set from $D_{\f{inp}}$, and $\f{h} \in \f{Seq}^{D_{\f{help}}}$ a card-sequence from $D_{\f{help}}$. 
We define an input set $U$ from $D = D_{\f{inp}} \cup D_{\f{help}}$ by $U = \{(\f{x}, \f{h}) \mid \f{x} \in U_{\f{inp}}\}$. 
A card-based protocol $\mathcal{P}=(D, U,Q,\f{A})$ is said to be a \textit{shuffle protocol} if the following conditions are satisfied:
\begin{enumerate}
\item[(a)] $\mathcal{P}$ always terminates within a fixed number of steps, i.e., it is a finite-runtime protocol;
\item[(b)] for any input sequence $(\f{x}, \f{h}) \in U$ \md{and for any final sequence $\f{y} \in \f{Fin}(\mathcal{P})$ of the form $\f{y} = (\f{x'}, \f{h})$, there exists a permutation $\sigma \in \mathfrak{S}_{|D_{\f{inp}}|}$ such that $\f{x'} = \sigma(\f{x})$;}
\item[(c)] for any input sequence $(\f{x}, \f{h})\in U$, \md{any card contained in $\f{x}$ has not been turned} at any step of a protocol execution.
\end{enumerate}
\md{%Let $\shuffle_{(\Pi,\F)}$ be a shuffle. 
We say that $\mathcal{P}$ \md{implements} a shuffle $\shuffle_{(\Pi,\F)}$ if every permutation $\sigma$ in (b) belongs to $\Pi$ and it is chosen according to the distribution $\F$.}
%Let $\f{y}_{\f{x}}$ be a random variable of the first $n$ cards of the final sequence when $(\f{x}, \f{h}) \in U$ is given as an input card-sequence. 
%Let $\shuffle_{(\Pi,\F)}$ be a shuffle. 
%% and $\f{y'}_{\f{x}}$ be a random variable of the card-sequence which is the result of $\shuffle_{(\Pi,\F)}$ with $\f{x} \in U_{\f{inp}}$. 
%We say that $\mathcal{P}$ \md{implements} $\shuffle_{(\Pi,\F)}$ if for any $\f{x} \in U_{\f{inp}}$, $\f{y}_{\f{x}}$ and $\shuffle_{(\Pi,\F)}(\f{x})$ are the same. 
%Let $\shuffle_{(\Pi,\F)}$ be a shuffle and $\mathcal{F}_{\f{x}}$ be a probability distribution of the resultant card-sequence when $\shuffle_{(\Pi,\F)}$ is applied to a card-sequence $\f{x} \in U$. 
%Let $\mathcal{F'}_{\f{x}}$ be a probability distribution of the final sequence of $\mathcal{P}$ when $\f{x} \in U$ is an initial sequence. 
%We say that $\mathcal{P}$ realizes $\shuffle_{(\Pi,\F)}$ if for any input sequence $\f{x} \in U$, $\mathcal{F}_{\f{x}}$ and $\mathcal{F'}_{\f{x}}$ are the same. 
We say that $\mathcal{P}$ is secure if for any $\f{x} \in U_{\f{inp}}$, \md{a random variable of $\sigma$} is stochastically independent of the random variable of the visible sequence-trace of $\mathcal{P}$. 
% when $(\f{x}, \f{h}) \in U$ is an initial card-sequence. 
\end{definition}


%%%% Our Protocol %%%%
\section{Graph shuffle protocols}\label{s:graph}

In this section, we construct a card-based protocol called  the graph shuffle protocol for a directed graph.
First, we introduce a graph shuffle in Subsection \ref{ss:graph}. Second, we construct the graph shuffle protocol, which is a shuffle protocol for any graph shuffle in Subsection \ref{ss:graphprotocol}. We note that our protocol requires PSSs only. 


\subsection{Graph shuffle}\label{ss:graph}
First, we recall some fundamentals from graph theory; for example, see \cite{CLZ}.

A directed graph is a quadruple $G=(V_G,E_G,s_G,t_G)$ consisting of two sets $V_G$, $E_G$ and two maps $s_G,t_G:E_G\to V_G$.
Each element of $V_G$ (resp. $E_G$) is called a vertex (resp. an \md{edge}). 
\md{Note that there might be two or more edges from $a$ to $b$ for some $a, b \in V_G$, that is, $G$ admits multiple edges.} 
For an \md{edge} $e\in E_G$, we call $s_G(e)$ (resp. $t_G(e)$) the source (resp. the target) of $e$. We will commonly write $a\xrightarrow{e}b$ or $e:a\to b$ to indicate that an \md{edge} $e$ has the source $a$ and the target $b$, and identify $e$ with a pair $(s_G(e),t_G(e))$.
A directed graph $G$ is finite if two sets $V_G$ and $E_G$ are finite sets. 
In \md{this paper}, a graph means a finite directed graph with $n$ vertices and $m$ \md{edges}.

Let $G$ be a graph.
For a vertex  $v\in V_G$, we define the following three functions:
\[ \f{in}(v)=|\{e\in E_G\mid v=t_G(e)\}|,\quad \f{out}(v)=|\{e\in E_G \mid v=s_G(e)\}|, \quad\text{and}\quad \f{deg}(v)=\f{in}(v)+\f{out}(v).\]
%and
%\[  \f{deg}(v)=\f{in}(v)+\f{out}(v).\]
The number $\f{deg}(v)$ is called \textit{the degree} of $v$.
We set $\f{Deg}_G = \{\f{deg}(v) \mid v \in V_G\}$.

For graphs $G$ and $G'$, a pair $f=(f_0,f_1):G\to G'$ consisting of maps $f_0:V_G\to V_{G'}$ and $f_1:E_G\to E_{G'}$ is a morphism of graphs if $(f_0\times f_0)\circ(s_G\times t_G)=(s_{G'}\times t_{G'})\circ f_1$ holds. In addition, if $f_0$ and $f_1$ are bijective, $f$ is called  \textit{an isomorphism} of graphs. In this case, we say that $G$ and $G'$ are isomorphic as graphs. 
In other words, two graphs $G$ and $G'$ are isomorphic as graphs when $x\xrightarrow{e}y$ in $G$ exists if and only if $f_0(x)\xrightarrow{f_1(e)}f_0(y)$ exists in $G'$.
We denote by $\f{Iso}(G,G')$ the set of all isomorphisms from $G$ to $G'$, and $\f{Iso}_0(G,G')$ the set of all $f_0$ such that $(f_0, *) \in \f{Iso}(G,G')$.
For a graph $G$, an isomorphism from $G$ to itself is called \textit{an automorphism}.
We denote by $\mathsf{Aut}(G)$ the set of all automorphisms of $G$, and $\mathsf{Aut}_0(G)$ the set of all $f_0$ such that \md{there exists $(f_0, f_1) \in \mathsf{Aut}(G)$}. \md{Then it is obvious that $\mathsf{Aut}(G)$ is a group by the composition of maps.
Furthermore, the group structure of $\mathsf{Aut}(G)$ induces the group structure on $\mathsf{Aut}_0(G)$.}
Note that, if $G$ has no multiple \md{edges}, then an automorphism $f = (f_0, f_1)\in \mathsf{Aut}(G)$ is determined by $f_0$. 
In the case that $G$ is an undirected graph, one can transform $G$ into the following directed graph $\overset{\to}{G}$:
\[ \text{ $V_{\overset{\to}{G}}=V_G$,\quad $E_{\overset{\to}{G}}=\{i\to j, \ j\to i\mid (i,j) \in E_G\}$}. \]

\begin{definition}\label{def:graphshuffle}
Let $G$ be a graph. 
% \md{and $\F$ a uniform distribution on $\aut_0(G)$}. 
The uniform closed shuffle $\shuffle_{(\aut_0(G), \F)}$ is called  \textit{the graph shuffle} for $G$ over $n$ cards. (Recall that $G$ has $n$ vertices.)
\end{definition}

\subsection{Graph shuffle protocols}\label{ss:graphprotocol}

In this subsection, we construct a graph shuffle protocol, which is a shuffle protocol of the graph shuffle for a graph $G = (V_G,E_G,s_G,t_G)$. 
We set $V_G=\{1,2,\ldots, n\}$. 
Let $D_{\f{inp}}=\{x_1,x_2,\ldots, x_n\}$ be any deck, and $U_{\f{inp}}$ any input set from $D_{\f{inp}}$. 
%Assume that all symbols of $D_{\f{inp}}$ are distinct and $\f{front}(\f{x})=(?,?,\ldots, ?)$ for all $\f{x} \in U_{\f{inp}}$. 
%We note that our protocol works for any deck $D_{\f{inp}}$ and any input set $U_{\f{inp}}$. 
%In particular, it works for $D_{\f{inp}}$ such that all symbols of $D_{\f{inp}}$ are distinct and $\f{front}(\f{x})=(?,?,\ldots, ?)$ for all $\f{x} \in U_{\f{inp}}$ as in Remark \ref{rem:shuffleprotocol}. 
We set a card-sequence $\f{h}$ of helping cards as follows:
\[
\f{h} = \crd{$\stext{1}$} \, \crd{$\stext{2}$} \, \crd{$\stext{3}$} ~\cdots ~\crd{$\stext{n}$}~
\overbrace{\crd{1} \, \cdots \, \crd{1}}^{\f{deg}(1)} ~ \overbrace{\crd{2} \, \cdots \, \crd{2}}^{\f{deg}(2)} ~ \overbrace{\crd{3} \, \cdots \, \crd{3}}^{\f{deg}(3)} ~ \cdots \cdots~ \overbrace{\crd{$n$} \, \cdots \, \crd{$n$}}^{\f{deg}(n)}~.
\]
Thus the deck of helping cards is $D_{\f{help}} = \{\stext{1}, \stext{2}, \ldots, \stext{n}, 1^{\f{deg}(1)}, 2^{\f{deg}(2)}, \ldots, n^{\f{deg}(n)}\}$, where the superscript denotes the number of the symbol in the deck $D_{\f{help}}$. 
The deck $D$ is the union of $D_{\f{inp}}$ and $D_{\f{help}}$ as multisets and it consists of $2(n+m)$ symbols. 

For an input card-sequence $\f{x}\in U_{\f{inp}}$, our protocol proceeds as follows:

\begin{enumerate}
\item[(1)] Place the cards as follows.
\[
\underbrace{\back \, \back \, \cdots \, \back}_{\f{x}} ~ 
\underbrace{\crd{$\stext{1}$} \, \crd{$\stext{2}$} \, \crd{$\stext{3}$} ~\cdots ~\crd{$\stext{n}$}~
\crd{1} \, \cdots \, \crd{1} ~ \crd{2} \, \cdots \, \crd{2} ~ \crd{3} \, \cdots \, \crd{3} ~ \cdots \cdots~ \crd{$n$} \, \cdots \, \crd{$n$}}_{\f{h}}~. 
\]
\item[(2)] For each $i$, we define $\f{pile}[i]$ by
\[
\f{pile}[i]= \biggl( \,\dfrac{?}{\stext{i}}, \overbrace{\dfrac{?}{i}, \ldots, \dfrac{?}{i}}^{\f{deg}(i)} \,\biggr)  = \underset{\stext{i}}{\back}\, \overbrace{\underset{i}{\back} \, \cdots \, \underset{i}{\back}}^{\f{deg}(i)}.
\]
Arrange the card-sequence as $(\f{x}, \f{pile}[1], \f{pile}[2], \f{pile}[3], \ldots, \f{pile}[n])$, that is:
\[
\underbrace{\back \, \back \, \cdots \, \back}_{\f{x}} ~ \underbrace{\back \, \back \, \cdots \, \back}_{\f{pile}[1]}~ 
\underbrace{\back \, \back \, \cdots \, \back}_{\f{pile}[2]}~ 
\underbrace{\back \, \back \, \cdots \, \back}_{\f{pile}[3]}~\cdots\cdots~ 
\underbrace{\back \, \back \, \cdots \, \back}_{\f{pile}[n]}.
\]

\item[(3)]  For each $d \in \f{Deg}_G$, we set $V_G^{(d)} = \{v_1^{(d)}, v_2^{(d)}, \ldots, v_{\ell_d}^{(d)}\}$ for all vertices with degree $d$, and apply $\f{PSS}_{(\ell_d, d+1)}$ to the card-sequence  $(\f{pile}[v_1^{(d)}], \f{pile}[v_2^{(d)}], \ldots, \f{pile}[v_{\ell_d}^{(d)}])$.
%By the definition of PSS, there is a permutation $\sigma_d\in \mathfrak{S}_{2(n+m)}$ such that $\sigma_d$ realizes $\f{PSS}_{(\ell_d, d+1)}$.
%Now, we set $\sigma := \prod_{d \in \f{Deg}_G} \sigma_d\in\mathfrak{S}_{2(n+m)}$. Notice that $\sigma$ is determined independently of the order of the multiplication of $\sigma_i$'s by the choice of $\sigma_i$.
Then we obtain a card-sequence
\[
\underbrace{\back \, \back \, \cdots \, \back}_{\f{x}} ~ \underbrace{\back \, \back \, \cdots \, \back}_{\f{pile}[\alpha_1]}~ \underbrace{\back \, \back \, \cdots \, \back}_{\f{pile}[\alpha_2]}~ \underbrace{\back \, \back \, \cdots \, \back}_{\f{pile}[\alpha_3]}~\cdots\cdots~ \underbrace{\back \, \back \, \cdots \, \back}_{\f{pile}[\alpha_n]}~.
\]
Let $\sigma \in \mathfrak{S}_n$ be the chosen permutation such that $\alpha_i = \sigma^{-1}(i)$. 
%where $\alpha_i = \sigma^{-1}(i)$. We note that the first card of $\f{pile}[\alpha_i]$ is $\dfrac{?}{\stext{\alpha_i}}$ for any $i=1,2,\ldots, n$. 

\item[(4)] For each $i\in V_G$ and $j \ra k \in E_G$, we set $\f{vertex}[i]= \left(\dfrac{?}{\stext{\alpha_i}}, \f{x}_i\right)$ and $\f{\md{edge}}[j \ra k]= \left(\dfrac{?}{\alpha_j}, \dfrac{?}{\alpha_k}\right)$, respectively.
Arrange the card-sequence\footnote{\md{Note that this rearrangement is possible without looking under the cards since the subscripts of $\alpha_i$ are public information.}} as follows:
\[
\underbrace{\back\,\back}_{\f{vertex}[1]}\,\underbrace{\back\,\back}_{\f{vertex}[2]}\,\cdots\,\underbrace{\back\,\back}_{\f{vertex}[n]}~~
\underbrace{\back\,\back}_{\f{\md{edge}}[e_1]}\,\underbrace{\back\,\back}_{\f{\md{edge}}[e_2]}\,\cdots\,\underbrace{\back\,\back}_{\f{\md{edge}}[e_m]}~,
\]
where $E_G = \{e_1, e_2, \ldots, e_m\}$. 
%$\f{vertex}[i]$ and $\f{\md{edge}}[j \ra k]$ can be }

\item[(5)] Apply $\f{PSS}_{(m+n,2)}$ to the card-sequence as follows: 
\[
%\bigg|\,\back\,\back\,\bigg|\,\back\,\back\,\bigg|\,\cdots\bigg|\,\back\,\back\,\bigg|\,\back\,\back\,\bigg|
\bigg|\,\underbrace{\back\,\back}_{\f{vertex}[1]}\,\bigg|\,\underbrace{\back\,\back}_{\f{vertex}[2]}\,\bigg|\,\cdots\bigg|\,\underbrace{\back\,\back}_{\f{\md{edge}}[e_m]}\,\bigg|
~\ra~
\back\,\back~\back\,\back~\cdots~\back\,\back~\back\,\back\,.
\]
%\md{カード列$\f{z}$に$\f{PSS}_{(m+n,2)}$を施す. このときある$\rho\in \mathfrak{S}_{n+m}$によって, $\f{z}$の上段は$\rho(\f{z}^u)$, 下段は$\rho(\f{z}^b)$に並び替えられている.}
\item[(6)] For each pile, turn over the left card, and if it is a black-card, turn over the right card. 
Then sort $n+m$ piles\footnote{\md{It is not essential the order of pairs of helping cards.}} so that the left card is in ascending order via $\preccurlyeq$ as follows:
%Here, suppose that we obtain the following card-sequence from $D'$ by sorting.
\[
\crd{$\stext{1}$}\,\back~\crd{$\stext{2}$}\,\back~\crd{$\stext{3}$}\,\back~\cdots~\crd{$\stext{n}$}\,\back~\crd{$i_1$}\,\crd{$j_1$}~\crd{$i_2$}\,\crd{$j_2$}~\crd{$i_3$}\,\crd{$j_3$}~\cdots~\crd{$i_m$}\,\crd{$j_m$}\,,
\]
where $i_1\preccurlyeq i_2 \preccurlyeq i_3 \preccurlyeq \cdots \preccurlyeq i_m$. 

\item[(7)] We define a graph $G'$ by $V_{G'} = V_G$ and $E_{G'} = \{i_1\ra j_1, i_2\ra j_2, i_3\ra j_3, \ldots, i_m\ra j_m\}$.
%Note that the graph $G'$ has become a relabelled graph of $G$, and $\sigma$ gives an isomorphism $\hat{\sigma}$ from $G'$ to $G$ such that $\hat{\sigma}_0=\sigma$.

\item[(8)] Take an isomorphism $\psi:G\to G'$, and set $\beta_i := \psi^{-1}_0(i)$. 
Let $\f{y}_i$ be the right next card of $\crd{$\stext{\beta_i}$}\,$ and $\f{y} =(\f{y}_1, \f{y}_2, \ldots, \f{y}_n)$. 
Arrange the card-sequence as follows:
%s so that the card to the right of red-card \crd{$\stext{\beta_i}$}\ is the $i$-th position from the left. 
%All other cards are placed in ascending order. 
%The output card-sequence of the input $\f{x}$ is $\f{y}$, which is the card-sequence from left to $n$-th:
\[
\underbrace{\back \, \back \, \cdots \, \back}_{\f{y}} ~ 
\underbrace{\crd{$\stext{1}$} \, \crd{$\stext{2}$} \, \crd{$\stext{3}$} ~\cdots ~\crd{$\stext{n}$}~
\crd{1} \, \cdots \, \crd{1} ~ \crd{2} \, \cdots \, \crd{2} ~ \crd{3} \, \cdots \, \crd{3} ~ \cdots \cdots~ \crd{$n$} \, \cdots \, \crd{$n$}}_{\f{h}}~. 
%\underset{\f{y}_1}{\back} \, \underset{\f{y}_2}{\back} \, \cdots \, \underset{\f{y}_n}{\back} ~ 
%\crd{$\stext{1}$} \, \crd{$\stext{2}$} \, \crd{$\stext{3}$} ~\cdots ~\crd{$\stext{n}$}~
%\crd{1} \, \cdots \, \crd{1} ~ \crd{2} \, \cdots \, \crd{2} ~ \crd{3} \, \cdots \, \crd{3} ~ \cdots \cdots~ \crd{$n$} \, \cdots \, \crd{$n$}~. 
\]
The output card-sequence for the input $\f{x}$ is $\f{y}$. 
\end{enumerate}

\begin{remark}
Regarding the number of cards, the number of cards in the proposed protocol is $2n + 2m$, of which $n + 2m$ are helping cards. 
As for the number of shuffles, it is \md{$|\f{Deg}_G|+1$}, and all of them are PSSs. 
We remark that the PSSs in Step (3) can be executed in parallel. 
\end{remark}

\begin{remark}\label{rem:iso}
In \md{Step (8)}, given two isomorphic graphs $G$ and $G'$, we need to solve the problem of finding one specific isomorphism between them. 
However, no polynomial-time algorithm for this problem has been found so far in general.
On the other hand, there exist polynomial-time algorithms to find isomorphisms for some \md{specific} graph classes. In addition, for small \md{specific} examples, an isomorphism can be computed by using a mathematical library for graph computation (e.g., Nauty \cite{Nauty}).  
\end{remark}

\subsection{Proof of correctness}

Let $\f{x} = (\f{x}_1, \f{x}_2, \ldots, \f{x}_n)$ be an input sequence and $\f{y} = (\f{y}_1, \f{y}_2, \ldots, \f{y}_n)$ a random variable of an output sequence of the protocol when $\f{x}$ is given as input. 
Fix a graph $G'$, which is defined in Step (7) following the opened result in Step (6). 
Let $\sigma \in \mathfrak{S}_n$ be a random variable of the permutation chosen by the PSSs in Step (3) such that $\alpha_i = \sigma^{-1}(i)$ for all $i \in V_G$. 
Since an \md{edge} $i \ra j \in E_G$ of $G$ corresponds to an \md{edge} $\alpha_i \ra \alpha_j = \sigma^{-1}(i) \ra \sigma^{-1}(j) \in E_{G'}$ of $G'$, there is an isomorphism $\phi = (\phi_0, \phi_1) \in \f{Iso}(G, G')$ such that $\phi_0 = \sigma^{-1}$. 
%the permutation $\sigma^{-1}$ induces an isomorphism from $G$ to $G'$, i.e., $\sigma^{-1} \in \f{Iso}(G, G')$. 
From the property of the PSSs, the permutation $\phi_0$ is a uniform random variable on $\f{Iso}_0(G, G')$. 
Let $\psi = (\psi_0, \psi_1) \in \f{Iso}(G', G)$ be a random variable of the isomorphism chosen in Step (9). 

We first claim that $\f{y} = \psi_0 \circ \phi_0 (\f{x})$. 
This is shown by observing a sequence of red cards $\crd{$\stext{1}$} \, \crd{$\stext{2}$} \, \crd{$\stext{3}$} \, \cdots \,\crd{$\stext{n}$}\,$. 
Hereafter, for the sake of clarity, we do not distinguish face-up $\dfrac{\stext{i}}{?}$ and face-down $\dfrac{?}{\stext{i}}$ and use ``$\stext{i}$" to denote the red card $i$. 
In Step (1), the sequence of red cards omitting other cards is $(\stext{1}, \stext{2}, \ldots, \stext{n})$. 
In Steps (3), (6), and (8), it is arranged as follows:
\[
\underset{\text{Step (1)}}{(\stext{1}, \stext{2}, \ldots, \stext{n})}
\xra{\phi_0^{-1}}
\underset{\text{Step (3)}}{(\stext{\alpha_1}, \stext{\alpha_2}, \ldots, \stext{\alpha_n})}
\xra{\phi_0}
\underset{\text{Step (6)}}{(\stext{1}, \stext{2}, \ldots, \stext{n})}
\xra{\psi_0}
\underset{\text{Step (8)}}{(\stext{\beta_1}, \stext{\beta_2}, \ldots, \stext{\beta_n})}.
\]
Since the input sequence $\f{x}$ is arranged as $((\stext{\alpha_1}, \f{x}_1), (\stext{\alpha_2}, \f{x}_2), \ldots, (\stext{\alpha_n}, \f{x}_n))$ in Step (3), the permutation $\psi_0 \circ \phi_0$ is applied to $\f{x}$. 
Thus, it holds $\f{y} = \psi_0 \circ \phi_0 (\f{x})$. 
We note that $\psi \circ \phi_0$ is an automorphism of $G$.  

%Now we prove the security of our protocol. 
%In order to prove the security, it is sufficient to show that a distribution of the opened symbols in Step (6) is independent of the distribution of $\psi\sigma^{-1} \in \aut(G)$ since cards are opened in Step (6) only. 

It remains to prove that the distribution of $\psi_0\circ\phi_0 \in \aut_0(G)$ is uniformly random. 
We note that given the graph $G'$, the distributions of $\phi_0$ and $\psi_0$ are independent. 
This is because the choice of $\psi_0$ depends on the opened symbols in Step (6) only, and they are independent of $\phi_0$ due to the PSS in Step (5). 
Thus, we can change the order of choice without harming the distributions of $\phi_0, \psi_0$: first, $\psi_0$ is chosen, and then $\phi_0$ is chosen. 
Since the distribution of $\phi_0 \in \f{Iso}_0(G, G')$ is uniformly random, it is sufficient to show that the function
\[ 
\begin{array}{cccc}
\Phi:  & \f{Iso}_0(G, G') &  \longrightarrow &  \aut_0(G) \\
           & \phi_0                                                    & \longmapsto & \psi_0 \circ\phi_0
 \end{array}
 \]
is bijective. 

We first prove that $\Phi$ is injective. 
Suppose that $\Phi(\phi'_0) = \Phi(\phi''_0)$ for some $\phi'_0, \phi''_0 \in \f{Iso}_0(G, G')$, that is,  $\psi_0 \circ\phi'_0 = \psi_0 \circ\phi''_0$. 
Since $\psi_0$ is a bijection, $\phi'_0 = \phi''_0$ holds.
Thus $\Phi$ is injective. 
We next prove that $\Phi$ is surjective. 
For any $\tau\in\aut_0(G)$, we have 
\[ \tau = \psi_0\circ\psi_0^{-1}\circ \tau = \Phi (\psi_0^{-1}\circ \tau). \] 
It yields that $\Phi$ is surjective. 
Therefore, $\Phi$ is bijective. 

This shows that the distribution of $\psi \circ \sigma^{-1}$ is uniformly random, and hence our protocol is correct. 

%\begin{remark}
%If $G$ has multiple \md{edges}, then there are many ways to extend the $\sigma^{-1}$ given above to an isomorphism  from $G$ to $G'$.
%However, the proof of correctness is independent of the way $\sigma^{-1}$ is extended.
%\end{remark}

\subsection{Proof of security}

%Let $\f{x} = (\f{x}_1, \f{x}_2, \ldots, \f{x}_n)$ and $\f{y} = (\f{y}_1, \f{y}_2, \ldots, \f{y}_n)$ be an input and output sequence in the protocol. 
%We use the same notations as in the proof of the correctness. 
In the proof of the correctness, we have already claimed that the distribution of the opened symbols in Step (6) is independent of $\sigma$ due to the PSS in Step (5). 
Since cards are opened in Step (6) only, this shows a distribution of the permutation $\psi\circ \sigma^{-1} \in \aut(G)$ is independent of the distribution of the visible sequence-trace of our protocol. 
Therefore, our protocol is secure. 

\subsection{Example of our protocol for a graph}
%\begin{example}
Let $G$ be a directed graph with $5$ vertices as follows:
\[ G=  \begin{xy}
                     (0,8)*[o]+{1}="1",(0,-8)*[o]+{2}="2",(12,0)*[o]+{3}="3",(24,8)*[o]+{4}="4",(24,-8)*[o]+{5\md{.}}="5",
                     \ar @<1mm>"1";"2"^{e_1}
                     \ar @<1mm>"2";"1"^{e_3}
                     \ar "1";"3"^{e_2}
                     \ar "2";"3"_{e_4}
                     \ar "3";"4"^{e_5}
                     \ar "3";"5"_{e_6}
            \end{xy} \]
We perform our graph shuffle protocol for $G$. 
Let $D_{\f{inp}}$ be an arbitrary deck with $D_{\f{inp}} = \{x_1, x_2, x_3, x_4, x_5\}$. 
The card-sequence $\f{h}$ of helping cards is defined as follows:
%In this case, helping cards are the following 17 cards. 
\[
\f{h} = \crd{$\stext{1}$} \, \crd{$\stext{2}$} \, \crd{$\stext{3}$} \, \crd{$\stext{4}$} \, \crd{$\stext{5}$}~
\crd{1} \, \crd{1} \, \crd{1}  ~ \crd{2} \, \crd{2} \, \crd{2} ~ \crd{3} \, \crd{3} \, \crd{3} \, \crd{3} ~  \crd{4} ~ \crd{5}\,\md{.}
\]
Set $D_{\f{help}} = \{\stext{1}, \stext{2}, \stext{3}, \stext{4}, \stext{5},1, 1, 1, 2, 2, 2, 3, 3, 3, 3, 4, 5\}$ and $D=D_{\f{inp}}\cup D_{\f{help}}$. 
% = \{x_1, x_2, x_3, x_4, x_5, \md{1}, \md{2}, \md{3}, \md{4}, \md{5},1, 1, 1, 2, 2, 2, 3, 3, 3, 3, 4, 5\}$.
For an input card-sequence $\f{x} = (\f{x}_1, \f{x}_2, \f{x}_3, \f{x}_4, \f{x}_5) \in U$, the graph shuffle protocol proceeds as follows:

\begin{enumerate}
\item[(1)] Place the cards such as:
\[
\underbrace{\underset{\f{x}_1}{\back} \, \underset{\f{x}_2}{\back} \, \underset{\f{x}_3}{\back} \, \underset{\f{x}_4}{\back} \, \underset{\f{x}_5}{\back}}_{\f{x}} ~ 
\underbrace{\crd{$\stext{1}$} \, \crd{$\stext{2}$} \, \crd{$\stext{3}$} \, \crd{$\stext{4}$} \, \crd{$\stext{5}$}~
\crd{1} \, \crd{1} \, \crd{1}  ~ \crd{2} \, \crd{2} \, \crd{2} ~ \crd{3} \, \crd{3} \, \crd{3} \, \crd{3} ~  \crd{4} ~ \crd{5}}_{\f{h}}~. 
\]
\item[(2)] Arrange the card-sequence as follows:
\[
\underbrace{\underset{\f{x}_1}{\back} \, \underset{\f{x}_2}{\back} \, \underset{\f{x}_3}{\back} \, \underset{\f{x}_4}{\back} \, \underset{\f{x}_5}{\back}}_{\f{x}} ~ \underbrace{\underset{\stext{1}}{\back} \, \underset{1}{\back} \, \underset{1}{\back} \, \underset{1}{\back}}_{\f{pile}[1]}~ 
\underbrace{\underset{\stext{2}}{\back} \, \underset{2}{\back} \, \underset{2}{\back} \, \underset{2}{\back}}_{\f{pile}[2]}~ 
\underbrace{\underset{\stext{3}}{\back} \, \underset{3}{\back} \, \underset{3}{\back} \, \underset{3}{\back} \, \underset{3}{\back}}_{\f{pile}[3]}~ 
\underbrace{\underset{\stext{4}}{\back} \, \underset{4}{\back}}_{\f{pile}[4]}~ 
\underbrace{\underset{\stext{5}}{\back} \, \underset{5}{\back}}_{\f{pile}[5]}~.
\]

\item[(3)] Perform $\f{PSS}_{(2, 4)}$ and $\f{PSS}_{(2, 2)}$ as follows:
\begin{align*}
&\bigg|~
\underset{\stext{1}}{\back} \, \underset{1}{\back} \, \underset{1}{\back} \, \underset{1}{\back}
~\bigg|~
\underset{\stext{2}}{\back} \, \underset{2}{\back} \, \underset{2}{\back} \, \underset{2}{\back}
~\bigg|
~~\ra~~
\underset{\stext{\alpha_1}}{\back} \,\underset{\alpha_1}{\back} \, \underset{\alpha_1}{\back} \,\underset{\alpha_1}{\back}~~
\underset{\stext{\alpha_2}}{\back} \, \underset{\alpha_2}{\back} \, \underset{\alpha_2}{\back} \, \underset{\alpha_2}{\back}~,\\
%
&\bigg|~
\underset{\stext{4}}{\back} \, \underset{4}{\back}
~\bigg|~
\underset{\stext{5}}{\back} \, \underset{5}{\back}
~\bigg|
~~\ra~~
\underset{\stext{\alpha_4}}{\back} \, \underset{\alpha_4}{\back}~~
\underset{\stext{\alpha_5}}{\back} \, \underset{\alpha_5}{\back}~.
\end{align*}
By setting $\alpha_3 = 3$, we have the following card-sequence:
\[
\underbrace{\underset{\f{x}_1}{\back} \, \underset{\f{x}_2}{\back} \, \underset{\f{x}_3}{\back} \, \underset{\f{x}_4}{\back} \, \underset{\f{x}_5}{\back}}_{\f{x}} ~ \underbrace{\underset{\stext{\alpha_1}}{\back} \, \underset{\alpha_1}{\back} \, \underset{\alpha_1}{\back} \, \underset{\alpha_1}{\back}}_{\f{pile}[\alpha_1]}~ 
\underbrace{\underset{\stext{\alpha_2}}{\back} \, \underset{\alpha_2}{\back} \, \underset{\alpha_2}{\back} \, \underset{\alpha_2}{\back}}_{\f{pile}[\alpha_2]}~ 
\underbrace{\underset{\stext{\alpha_3}}{\back} \, \underset{\alpha_3}{\back} \, \underset{\alpha_3}{\back} \, \underset{\alpha_3}{\back} \, \underset{\alpha_3}{\back}}_{\f{pile}[\alpha_3]}~ 
\underbrace{\underset{\stext{\alpha_4}}{\back} \, \underset{\alpha_4}{\back}}_{\f{pile}[\alpha_4]}~ 
\underbrace{\underset{\stext{\alpha_5}}{\back} \, \underset{\alpha_5}{\back}}_{\f{pile}[\alpha_5]}~.
\]
%For each $d \in \f{Deg}_G$, we set $V_G^{(d)} = \{v_1^{(d)}, v_2^{(d)}, \ldots, v_{\ell_d}^{(d)}\}$ for all vertices with degree $d$.
%Apply $\f{PSS}_{(\ell_d, d+1)}$ to a card-sequence  $(\f{pile}[v_1^{(d)}], \f{pile}[v_2^{(d)}], \ldots, \f{pile}[v_{\ell_d}^{(d)}])$.
%Then we obtain a card-sequence
%\[
%\underbrace{\back \, \back \, \cdots \, \back}_{\f{x}} ~ \underbrace{\back \, \back \, \cdots \, \back}_{\f{pile}[\alpha_1]}~ \underbrace{\back \, \back \, \cdots \, \back}_{\f{pile}[\alpha_2]}~ \underbrace{\back \, \back \, \cdots \, \back}_{\f{pile}[\alpha_3]}~\cdots\cdots~ \underbrace{\back \, \back \, \cdots \, \back}_{\f{pile}[\alpha_n]}~.
%\]
%Let $\sigma \in \mathfrak{S}_n$ be the chosen permutation such that $\alpha_i = \sigma^{-1}(i)$. 

\item[(4)] Arrange the card-sequence as follows:
\[
\underbrace{\underset{\f{x}_1}{\back}\,\underset{\stext{\alpha_1}}{\back}}_{\f{vertex}[1]}\,
\underbrace{\underset{\f{x}_2}{\back}\,\underset{\stext{\alpha_2}}{\back}}_{\f{vertex}[2]}\,
\underbrace{\underset{\f{x}_3}{\back}\,\underset{\stext{\alpha_3}}{\back}}_{\f{vertex}[3]}\,
\underbrace{\underset{\f{x}_4}{\back}\,\underset{\stext{\alpha_4}}{\back}}_{\f{vertex}[4]}\,
\underbrace{\underset{\f{x}_5}{\back}\,\underset{\stext{\alpha_5}}{\back}}_{\f{vertex}[5]}~~
\underbrace{\underset{\alpha_1}{\back}\,\underset{\alpha_2}{\back}}_{\f{\md{edge}}[1\ra 2]}\,
\underbrace{\underset{\alpha_1}{\back}\,\underset{\alpha_3}{\back}}_{\f{\md{edge}}[1\ra 3]}\,
\underbrace{\underset{\alpha_2}{\back}\,\underset{\alpha_1}{\back}}_{\f{\md{edge}}[2\ra 1]}\,
\underbrace{\underset{\alpha_2}{\back}\,\underset{\alpha_3}{\back}}_{\f{\md{edge}}[2\ra 3]}\,
\underbrace{\underset{\alpha_3}{\back}\,\underset{\alpha_4}{\back}}_{\f{\md{edge}}[3\ra 4]}\,
\underbrace{\underset{\alpha_3}{\back}\,\underset{\alpha_5}{\back}}_{\f{\md{edge}}[3\ra 5]}\,.
\]

\item[(5)] Apply $\f{PSS}_{(11,2)}$ to the card-sequence as follows: 
\[
%\bigg|\,\back\,\back\,\bigg|\,\back\,\back\,\bigg|\,\cdots\bigg|\,\back\,\back\,\bigg|\,\back\,\back\,\bigg|
\bigg|\underbrace{\underset{\f{x}_1}{\back}\,\underset{\stext{\alpha_1}}{\back}}_{\f{vertex}[1]}
\bigg|\underbrace{\underset{\f{x}_2}{\back}\,\underset{\stext{\alpha_2}}{\back}}_{\f{vertex}[2]}
\bigg|\underbrace{\underset{\f{x}_3}{\back}\,\underset{\stext{\alpha_3}}{\back}}_{\f{vertex}[3]}
\bigg|\underbrace{\underset{\f{x}_4}{\back}\,\underset{\stext{\alpha_4}}{\back}}_{\f{vertex}[4]}
\bigg|\underbrace{\underset{\f{x}_5}{\back}\,\underset{\stext{\alpha_5}}{\back}}_{\f{vertex}[5]}
\bigg|\underbrace{\underset{\alpha_1}{\back}\,\underset{\alpha_2}{\back}}_{\f{\md{edge}}[1\ra 2]}
\bigg|\underbrace{\underset{\alpha_1}{\back}\,\underset{\alpha_3}{\back}}_{\f{\md{edge}}[1\ra 3]}
\bigg|\underbrace{\underset{\alpha_2}{\back}\,\underset{\alpha_1}{\back}}_{\f{\md{edge}}[2\ra 1]}
\bigg|\underbrace{\underset{\alpha_2}{\back}\,\underset{\alpha_3}{\back}}_{\f{\md{edge}}[2\ra 3]}
\bigg|\underbrace{\underset{\alpha_3}{\back}\,\underset{\alpha_4}{\back}}_{\f{\md{edge}}[3\ra 4]}
\bigg|\underbrace{\underset{\alpha_3}{\back}\,\underset{\alpha_5}{\back}}_{\f{\md{edge}}[3\ra 5]}\bigg|\,.
\]

\item[(6)] For each pile, turn over the left card, and if it is a black-card, turn over the right card. 
The following card-sequence is an example outcome:
\[
\crd{$\stext{5}$}\,\back~
\crd{1}\,\crd{3}~
\crd{2}\,\crd{3}~
\crd{$\stext{4}$}\,\back~
\crd{$\stext{2}$}\,\back~
\crd{2}\,\crd{1}~
\crd{$\stext{1}$}\,\back~
\crd{3}\,\crd{5}~
\crd{3}\,\crd{4}~
\crd{1}\,\crd{2}~
\crd{$\stext{3}$}\,\back\,.
\]
Sort $11$ piles so that the left card is in ascending order via $\preccurlyeq$ as follows:
\[
\crd{$\stext{1}$}\,\underset{\f{y'}_1}{\back}~\crd{$\stext{2}$}\,\underset{\f{y'}_2}{\back}~\crd{$\stext{3}$}\,\underset{\f{y'}_3}{\back}~\crd{$\stext{4}$}\,\underset{\f{y'}_4}{\back}~\crd{$\stext{5}$}\,\underset{\f{y'}_5}{\back}~
\crd{1}\,\crd{3}~\crd{1}\,\crd{2}~\crd{2}\,\crd{3}~\crd{2}\,\crd{1}~\crd{3}\,\crd{5}~\crd{3}\,\crd{4}\,.~
\]

\item[(7)] Define a graph $G'$ by $V_{G'} = \{1, 2, 3, 4, 5\}$ and $E_{G'} = \{1\ra3, 1\ra2, 2\ra3, 2\ra1, 3\ra4, 3\ra5\}$;
\[ G'=  \begin{xy}
                     (0,8)*[o]+{1}="1",(0,-8)*[o]+{2}="2",(12,0)*[o]+{3}="3",(24,8)*[o]+{4}="4",(24,-8)*[o]+{5.}="5",
                     \ar @<1mm>"1";"2"^{}
                     \ar @<1mm>"2";"1"^{}
                     \ar "1";"3"^{}
                     \ar "2";"3"_{}
                     \ar "3";"4"^{}
                     \ar "3";"5"_{}
            \end{xy}\]

\item[(8)] Take an isomorphism $\psi: G\to G'$ defined by 
\[
1\longmapsto 2,\quad 2\longmapsto 1,\quad 3\longmapsto 3 ,\quad 4\longmapsto 4,\quad 5\longmapsto 5.
\]
Arrange the above card-sequence as follows:
\[
\underset{\f{y'}_2}{\back} \, \underset{\f{y'}_1}{\back} \, \underset{\f{y'}_3}{\back} \, \underset{\f{y'}_4}{\back} \, \underset{\f{y'}_5}{\back}~ 
\underbrace{\crd{$\stext{1}$} \, \crd{$\stext{2}$} \, \crd{$\stext{3}$} \, \crd{$\stext{4}$} \, \crd{$\stext{5}$}~
\crd{1} \, \crd{1} \, \crd{1}  ~ \crd{2} \, \crd{2} \, \crd{2} ~ \crd{3} \, \crd{3} \, \crd{3} \, \crd{3} ~  \crd{4} ~ \crd{5}}_{\f{h}}~. 
\]
The output card-sequence for the input $\f{x}$ is $(\f{y'}_2, \f{y'}_1, \f{y'}_3, \f{y'}_4, \f{y'}_5)$. 
\end{enumerate}

\subsection{Implication of our protocol}\label{ss:implication}

\md{In this subsection, we consider several interesting graph shuffles.}
% and give the numbers of cards and PSSs required in our protocol. }

\md{
%We remark that many standard shuffles are contained in the class of graph shuffles. 
%As a trivial case, a PSS for $n$ cards is equivalent to a graph shuffle for a graph with $n$ vertices and no edges. 
We first observe that a RC for $n$ cards are graph shuffles for the directed $n$-cycle graph $\overset{\ra}{C_n}$ (see Section \ref{4-1}). 
%a RC for $n$ cards is equivalent to a graph shuffle for the \md{directed} $n$-cycle graph $\overset{\ra}{C_n}$ (see Section \ref{4-1}). 
Since it holds $2n + 2m = 4n$ and $|\f{Deg}_G|+1 = 2$, a RC can be done by $4n$ cards and two PSSs. 
In Section \ref{4-1}, the number of cards is improved to $3n$.
We remark that our graph shuffle protocol works even for a sequence of piles each having equivalent number of face-down cards. 
Thus a pile-shifting shuffle (i.e., a pile-version of RC) can be done by the same number of helping cards. 
In particular, for a pile-shifting shuffle for $n$ piles of $m$ cards, it can be done by $nm + 3n$ cards and two PSSs. 
We note that PSSs and RBCs are graph shuffles for graphs with no edges in this sense.}

\md{A graph shuffle for the undirected $n$-cycle graph $C_n$ is equivalent to the \emph{dihedral shuffle}, which is introduced by Niemi and Renvall \cite{Niemi98}. 
Since it holds $2n + 2m = 5n$ and $|\f{Deg}_G|+1 = 2$, our result implies that a RC can be done by $5n$ cards and two PSSs. 
In Section \ref{sec:dihedral}, the number of cards is improved to $3n$, although the number of PSSs is increased to three.}

\md{For a cyclic group $\Pi = \langle (1\;2)(3\;4\;5\;6) \rangle$, a uniform closed shuffle $(\shuffle, \Pi, \F)$ is a graph shuffle for $G$ where $V_G = \{1, 2, 3, 4, 5, 6\}$ and $E_G = E_1 \cup E_2 \cup E_3$ with $E_1 = \{1\to 2, 2 \to 1\}$, $E_2 = \{3\to 4, 4\to 5, 5\to 6, 6\to 4\}$, and $E_3 = \{1 \to 3, 1\to 5, 2\to 4, 2 \to 6\}$. 
Since it holds $\mathsf{Aut}_0(G) =  \langle (1\;2)(3\;4\;5\;6) \rangle$, we can conclude that a graph shuffle for $G$ is equivalent to a uniform closed shuffle $(\shuffle, \Pi, \F)$. 
Since it holds $2n + 2m = 32$ and $|\f{Deg}_G|+1 = 3$, our result implies that it can be done by $32$ cards and three PSSs. 
By generalizing this idea, for any cyclic group $\Pi = \langle \pi \rangle$, a uniform closed shuffle $(\shuffle, \Pi, \F)$ is a graph shuffle for some graph. 
}
%We can observe that $G$ has two cycles of length $2$ defined by $E_1$ and length $4$ defined by $E_2$. 
%These cycles are corresponding to two cyclic permutations $(1\;2)$ and $(3\;4\;5\;6)$, respectively. 
%The subset $E_1$ defines a cycle of length $2$ and the subset $E_2$ defines a cycle of length $4$. 
%We note that the uniform closed shuffle is not equivalent to two RCs since two cyclic permutations $(1\;2)$ and $(3\;4\;5\;6)$ must be ``synchronized". 
%An interesting nontrivial 
%For any cyclic group $\Pi \subset \mathfrak{S}_n$, a uniform closed shuffle $(\shuffle, \Pi)$ is a graph shuffle for some graph. 
%We as follows. 
%An interesting observation is that every uniform closed shuffle $(\shuffle, \Pi, \F)$ for a cyclic group $\Pi$ is contained in the class of graph shuffles. 
%Firstly, consider a simple case of $\Pi = \langle (1\;2)(3\;4\;5\;6) \rangle$. 
% is generated by a permutation $(1\;2)(3\;4\;5\;6)$. 
%(Note that when cycle lengths are coprime, e.g., $\Pi = \langle (1\;2)(3\;4\;5) \rangle$, the uniform closed shuffle can be done by applying two shuffles for $(1\;2)$ and $(3\;4\;5)$ independently.)

%%%% Application %%%%
\section{Efficiency improvements for graph shuffles for cycles}\label{s:app}

In this section, we \md{implement} efficient graph shuffle protocols for some \md{specific} graph classes. 
%\md{Since cyclic graphs have the same degree, we can remove the red cards. In particular, for an undirected $n$-cyclic graph, even though it has $2n$ edges, the number of cards corresponding to edges is reduced to $n$ pairs of cards using the symmetry of the graph.}
In particular, we improve the number of cards in our protocol.  

\subsection{The $n$-cycle graph} \label{4-1}
First, we consider the $n$-cycle graph $\overset{\ra}{C_n}$:
\[ \overset{\ra}{C_n}= \begin{xy}
                     (0,-2)*[o]+{1}="1",(15,-2)*[o]+{2}="2",(30,-2)*[o]+{\cdots}="3",(45,-2)*[o]+{n-1}="4",(60,-2)*[o]+{n.}="n",
                     \ar "1";"2"^{}
                     \ar "2";"3"^{}
                     \ar "3";"4"_{}
                     \ar "4";"n"_{}
                     \ar @(lu,ur)"n";"1"_{}
            \end{xy}\]
The graph shuffle for  $\overset{\ra}{C_n}$ is equivalent to a RC of $n$ cards since the automorphism group $\aut(\overset{\ra}{C_n})$ is isomorphic to the cyclic group of degree $n$.
If we apply our graph shuffle protocol for $\overset{\ra}{C_n}$ proposed in Section 3, we need $4n$ cards.
In this subsection, we propose a graph shuffle protocol for $\overset{\ra}{C_n}$ with $3n$ cards only. 

\md{Before describing the improved protocol, we shortly mention how to improve the number of cards. The idea\footnote{We remark that this idea works for every graphs such that all vertices have the same degree.} is to remove the red cards by making a pile of $(\f{x}_i, \alpha_i, \alpha_{i+1})$ instead of a pile of $(\f{x}_i, \stext{\alpha_i})$ and a pile of $(\alpha_i, \alpha_{i+1})$ in the previous protocol. Since all vertices of $\overset{\ra}{C_n}$ have the same degree, all piles of $(\f{x}_i, \alpha_i, \alpha_{i+1})$ have the same number of cards and thus the final randomization (corresponding to Step (5) in the previous protocol) can be done by a single PSS.}
%since all vertices of $\overset{\ra}{C_n}$ have the same degree, we can remove the red cards $\crd{$\stext{1}$}\, \crd{$\stext{2}$} \, \cdots \, \crd{$\stext{n}$}$ by associating each card $\f{x}_i$ with each edge. }

%We set $V_G=\{1,2,\ldots, n\}$ and $D_{\f{inp}}=\{x_1,x_2,\ldots, x_n\}$. 
%Assume\footnote{By Remark \ref{rem:shuffleprotocol}, our graph shuffle protocol works for any deck $D_{\f{inp}}$ and any input set $U_{\f{inp}}$.} that all symbols of $D_{\f{inp}}$ are distinct and $\f{front}(\f{x})=(?,?,\ldots, ?)$ for all $\f{x} \in U_{\f{inp}}$. 
%We set a card-sequence $\f{h}$ of helping cards as follows:
%\[
%\f{h} = \crd{$\stext{1}$} \, \crd{$\stext{2}$} \, \crd{$\stext{3}$} ~\cdots ~\crd{$\stext{n}$}~
%\overbrace{\crd{1} \, \cdots \, \crd{1}}^{\f{deg}(1)} ~ \overbrace{\crd{2} \, \cdots \, \crd{2}}^{\f{deg}(2)} ~ \overbrace{\crd{3} \, \cdots \, \crd{3}}^{\f{deg}(3)} ~ \cdots \cdots~ \overbrace{\crd{$n$} \, \cdots \, \crd{$n$}}^{\f{deg}(n)}~.
%\]
%Thus the deck of helping cards is $D_{\f{help}} = \{\stext{1}, \stext{2}, \ldots, \stext{n}, 1^{\f{deg}(1)}, 2^{\f{deg}(2)}, \ldots, n^{\f{deg}(n)}\}$, where the superscript denotes the number of the symbol in the deck $D_{\f{help}}$. 
%The deck $D$ is the union of $D_{\f{inp}}$ and $D_{\f{help}}$ as multisets and it consists of $2n+2m$ symbols. 

Let $D_{\f{inp}}=\{x_1,x_2,\ldots, x_n\}$ be an arbitrary deck and $D_{\f{help}} = \{1, 1, 2, 2, 3, 3, \ldots, n, n\}$ a deck of the symbols of $2n$ helping cards. 
The sequence of helping cards $\f{h}$ is defined as follows:
\[ 
\f{h} = \crd{1}\,\crd{1}~\crd{2}\,\crd{2}~\crd{3}\,\crd{3}~\cdots\crd{$n$}\,\crd{$n$}~. \]
For $i=1,2,\ldots ,n$, we set $\f{pile}[i]=\left(\dfrac{?}{i}, \dfrac{?}{i}\right)$.

\begin{enumerate}
\item[(1)] Place the $3n$ cards as follows:
\[
\underbrace{\back \, \back \, \back \, \cdots \, \back}_{\f{x}} ~ 
\underbrace{\crd{1}\,\crd{1}~\crd{2}\,\crd{2}~\crd{3}\,\crd{3}~\cdots\crd{$n$}\,\crd{$n$}}_{\f{h}}~. 
\]

\item[(2)] Arrange the card-sequence as follows:
\[
\underbrace{\back \, \back \, \back \, \cdots \, \back}_{\f{x}} ~ 
\underbrace{\underset{1}{\back} \, \underset{1}{\back}}_{\f{pile}[1]}~ 
\underbrace{\underset{2}{\back} \, \underset{2}{\back}}_{\f{pile}[2]}~ 
\underbrace{\underset{3}{\back} \, \underset{3}{\back}}_{\f{pile}[3]}~ 
~\cdots~ 
\underbrace{\underset{n}{\back} \, \underset{n}{\back}}_{\f{pile}[n]}~ .
\]
%\[
%\underbrace{\back \, \back \, \cdots \, \back}_{\f{x}} ~ 
%\underbrace{\crd{1}\,\crd{1}~\crd{2}\,\crd{2}~\crd{3}\,\crd{3}~\cdots\crd{$n$}\,\crd{$n$}}_{\f{h}}~. 
%\]
Apply $\PSS_{(n,2)}$ to $(\f{pile}[1], \f{pile}[2] ,\ldots,\f{pile}[n])$ and then we obtain the card-sequence as follows:
\[
\underbrace{\back \, \back \, \back \, \cdots \, \back}_{\f{x}} ~ 
\underset{\alpha_1}{\back} \, \underset{\alpha_1}{\back}
~
\underset{\alpha_2}{\back} \, \underset{\alpha_2}{\back}
~
\underset{\alpha_3}{\back} \, \underset{\alpha_3}{\back}
~
\cdots
~
\underset{\alpha_n}{\back} \, \underset{\alpha_n}{\back}~,
\]
where $\{\alpha_1,\alpha_2,\ldots,\alpha_n\}=\{1,2,\ldots, n\}$.

\item[(3)] Arrange the card-sequence as follows:
\[ 
\underset{\f{x}_1}{\back} \,
\underset{\alpha_1}{\back} \, \underset{\alpha_2}{\back} ~
\underset{\f{x}_2}{\back} \,
\underset{\alpha_2}{\back} \, \underset{\alpha_3}{\back} ~
\underset{\f{x}_3}{\back} \,
\underset{\alpha_3}{\back} \, \underset{\alpha_4}{\back} ~
\cdots ~
\underset{\f{x}_n}{\back} \,
\underset{\alpha_n}{\back} \, \underset{\alpha_1}{\back}\,.
\]

\item[(4)] Apply $\PSS_{(n,3)}$ to the card-sequence as follows:
\[
\begin{tabular}{|c|c|c|c|c|}
$\back\, \back\, \back$ &
$\back\, \back\, \back$ &
$\back\, \back\, \back$ &
$\cdots$ &
$\back\, \back\, \back$
\end{tabular}\,.
\]

\item[(5)] For all piles, turn over the second and third cards. Let $a_i, b_i \in \{1,2,\ldots, n\}$ be the opened symbols of the second and third cards, respectively, in the $i$-th pile as follows:
%Then we suppose that the following card-sequence is obtained.
\[
\back\,\crd{$a_1$}\,\crd{$b_1$}~~\back\,\crd{$a_2$}\,\crd{$b_2$}~~\back\,\crd{$a_3$}\,\crd{$b_3$}~~\cdots~~\back\,\crd{$a_n$}\,\crd{$b_n$}~\md{.}
\]

\item[(6)] Arrange $n$ piles so that $(c_1, d_1) = (a_1, b_1)$, $d_i = c_{i+1}$, $(1 \leq i \leq n-1)$, and $d_n = c_1$ as follows:
\[
\underset{\f{y}_1}{\back} \,\crd{$c_1$}\,\crd{$d_1$}~~\underset{\f{y}_2}{\back} \,\crd{$c_2$}\,\crd{$d_2$}~~\underset{\f{y}_3}{\back} \,\crd{$c_3$}\,\crd{$d_3$}~~\cdots~~\underset{\f{y}_n}{\back} \,\crd{$c_n$}\,\crd{$d_n$}~.
\]
After that, we arrange the card-sequence as follows:
\[
\underset{\f{y}_1}{\back} \, \underset{\f{y}_2}{\back} \, \underset{\f{y}_3}{\back} \, \cdots \, \underset{\f{y}_n}{\back} ~ 
\underbrace{\crd{1}\,\crd{1}~\crd{2}\,\crd{2}~\crd{3}\,\crd{3}~\cdots\crd{$n$}\,\crd{$n$}}_{\f{h}}~. 
\]
Then the output card-sequence is $\mathsf{y}=(\f{y}_1,\f{y}_2,\ldots, \f{y}_n)$.
\end{enumerate}


\md{We show the correctness of the protocol. Let $\mathsf{x}=(\mathsf{x}_1,\ldots, \mathsf{x}_n)$ be an input sequence. Assume that the protocol outputs the sequence $\mathsf{y}=(\mathsf{y}_1,\ldots, \mathsf{y}_n)$ when $\mathsf{x}$ is given as input.
First, we see that $\mathsf{y}=\sigma (\mathsf{x})$ for some $\sigma$ in the cyclic group of degree $n$. 
For $i=1,\ldots, n$, we set $P_i=(\mathsf{x}_i,\alpha_i,\alpha_{i+1})$, where $\alpha_{n+1}=\alpha_1$, in Step (3) and put $P=(P_1,P_2,\ldots, P_n)$. 
Let $Q=(Q_1,\ldots, Q_n)=(P_{\sigma^{-1}(1)},\ldots, P_{\sigma^{-1}(n)})$ for some $\sigma\in\mathfrak{S}_n$. Then, $Q$ is obtained by $\sigma$ in the cyclic group of degree $n$ if and only if the third entry of $Q_i$ and the second entry of $Q_{i+1}$ are same for any $1\leq i\leq n-1$. It follows that the components of obtained sequence in Step (6) are sorted in a cyclic fashion of $P$. Therefore, $\mathsf{y}$ is equal to $\sigma (\mathsf{x})$ for some $\sigma$ in the cyclic group of degree $n$. 
Note that each element $\sigma$ of the cyclic group is determined by $\sigma^{-1}(1)$, and it is determined by $d_1$.  
For each $k\in\{1,2,\ldots, n\}$, the probability that $k=d_1$ is $\dfrac{1}{n}$ since $d_1$ is dependent on the PSS in Step (4) only. Thus the distribution of $\sigma$ is uniformly random, and hence the protocol is correct.}

\md{We show the security of the protocol. Assume that $\sigma\in\mathfrak{S}_n$ and $\tau\in \mathfrak{S}_n$ are chosen in Steps (2) and (4), respectively. Then the first card in the $i$-th pile in Step (5) is $\mathsf{x}_{\tau^{-1}(i)}$. On the other hand, the second and third cards in the $i$-th pile in Step (5) are $a_i=\tau^{-1}\sigma^{-1}(i)$ and $b_i=\tau^{-1}\sigma^{-1}(i+1)$. Here, we consider $n+1$ as $1$. 
This implies that these opened symbols $a_1,\ldots, a_n$ and $b_1,\ldots, b_n$ do not allow us to guess the first card of any pile since $\sigma$ is chosen uniformly at random in Step (4). Therefore, the protocol is secure.}






\subsection{The undirected $n$-cycle}\label{sec:dihedral}
Next, we consider the undirected $n$-cycle graph $C_n$:
\[ C_n= \begin{xy}
                     (0,-2)*[o]+{1}="1",(15,-2)*[o]+{2}="2",(30,-2)*[o]+{\cdots}="3",(45,-2)*[o]+{n-1}="4",(60,-2)*[o]+{n.}="n",
                     \ar @{-} "1";"2"^{}
                     \ar @{-} "2";"3"^{}
                     \ar @{-} "3";"4"_{}
                     \ar @{-} "4";"n"_{}
                     \ar @{-} @(lu,ur)"n";"1"_{}
            \end{xy}\]
\md{Recall that we regard undirected edge as two directed edges with opposite directions (see the paragraph just before Definition \ref{def:graphshuffle}).} 
The automorphism group $\aut(C_n)$ is isomorphic to the dihedral group of degree $n$. 
\md{For example, the graph shuffle for $C_n$ when $n = 4$ is given as follows: 
\[
\crd{1}~\crd{2}~\crd{3}~\crd{4}~ \longmapsto
\begin{cases}
~\crd{1}~\crd{2}~\crd{3}~\crd{4}~\\
~\crd{2}~\crd{3}~\crd{4}~\crd{1}~\\
~\crd{3}~\crd{4}~\crd{1}~\crd{2}~\\
~\crd{4}~\crd{1}~\crd{2}~\crd{3}~\\
~\crd{4}~\crd{3}~\crd{2}~\crd{1}~\\
~\crd{3}~\crd{2}~\crd{1}~\crd{4}~\\
~\crd{2}~\crd{1}~\crd{4}~\crd{3}~\\
~\crd{1}~\crd{4}~\crd{3}~\crd{2}~,
\end{cases}
\]
where each sequence is obtained with probability $1/8$.}
If we apply the graph shuffle for $C_n$, we need $6n$ cards.
%To streamline our protocol for $C_n$, we give a procedure to realize it with $2n$ auxiliary cards. 
In this subsection, we propose a graph shuffle protocol for $C_n$ with $3n$ cards only. 

\md{
%Now we shortly mention how to improve the number of cards. 
For an undirected $n$-cyclic graph, even though it has $2n$ edges, the number of cards corresponding to edges is reduced to $n$ pairs of cards using the symmetry of the graph.
This improvement is done by the pile-scramble shuffle in Step (3) in the below protocol. }
%Besides the same idea in Subsection \ref\label{4-1}, 

Let $D_{\f{inp}}=\{x_1,x_2,\ldots, x_n\}$ be an arbitrary deck and $D_{\f{help}} = \{1, 1, 2, 2, 3, 3, \ldots, n, n\}$ a deck of the symbols of $2n$ helping cards. 
The sequence of helping cards $\f{h}$ is defined as follows:
\[ 
\f{h} = \crd{1}\,\crd{1}~\crd{2}\,\crd{2}~\crd{3}\,\crd{3}~\cdots\crd{$n$}\,\crd{$n$}~. \]
For $i=1,2,\ldots ,n$, we set $\f{pile}[i]=\left(\dfrac{?}{i}, \dfrac{?}{i}\right)$.

\begin{enumerate}
\item[(1)] Place the $3n$ cards as follows:
\[
\underbrace{\back \, \back \, \back \, \cdots \, \back}_{\f{x}} ~ 
\underbrace{\crd{1}\,\crd{1}~\crd{2}\,\crd{2}~\crd{3}\,\crd{3}~\cdots\crd{$n$}\,\crd{$n$}}_{\f{h}}~. 
\]

\item[(2)] Arrange the card-sequence as follows:
\[
\underbrace{\back \, \back \, \back \, \cdots \, \back}_{\f{x}} ~ 
\underbrace{\underset{1}{\back} \, \underset{1}{\back}}_{\f{pile}[1]}~ 
\underbrace{\underset{2}{\back} \, \underset{2}{\back}}_{\f{pile}[2]}~ 
\underbrace{\underset{3}{\back} \, \underset{3}{\back}}_{\f{pile}[3]}~ 
~\cdots~ 
\underbrace{\underset{n}{\back} \, \underset{n}{\back}}_{\f{pile}[n]}~ .
\]
%\[
%\underbrace{\back \, \back \, \cdots \, \back}_{\f{x}} ~ 
%\underbrace{\crd{1}\,\crd{1}~\crd{2}\,\crd{2}~\crd{3}\,\crd{3}~\cdots\crd{$n$}\,\crd{$n$}}_{\f{h}}~. 
%\]
Apply $\PSS_{(n,2)}$ to $(\f{pile}[1], \f{pile}[2] ,\ldots,\f{pile}[n])$ and then we obtain the card-sequence as follows:
\[
\underbrace{\back \, \back \, \back \, \cdots \, \back}_{\f{x}} ~ 
\underset{\alpha_1}{\back} \, \underset{\alpha_1}{\back}
~
\underset{\alpha_2}{\back} \, \underset{\alpha_2}{\back}
~
\underset{\alpha_3}{\back} \, \underset{\alpha_3}{\back}
~
\cdots
~
\underset{\alpha_n}{\back} \, \underset{\alpha_n}{\back}~,
\]
where $\{\alpha_1,\alpha_2,\ldots,\alpha_n\}=\{1,2,\ldots, n\}$.

\item[(3)] Arrange the card-sequence as follows:
\[ 
\underbrace{\back \, \back \, \back \, \cdots \, \back}_{\f{x}} ~
\underset{\alpha_1}{\back} \, \underset{\alpha_2}{\back} \, \underset{\alpha_3}{\back} \, \cdots \, \underset{\alpha_{n-1}}{\back} \, \underset{\alpha_n}{\back} ~~
\underset{\alpha_2}{\back} \, \underset{\alpha_3}{\back} \, \underset{\alpha_4}{\back} \, \cdots \, \underset{\alpha_n}{\back} \, \underset{\alpha_1}{\back}\,.
\]
Apply $\PSS_{(2,n)}$ to the rightmost card-sequence of $2n$ cards as follows:
\[
\bigg|~
\underset{\alpha_1}{\back} \, \underset{\alpha_2}{\back} \, \underset{\alpha_3}{\back} \, \cdots \, \underset{\alpha_{n-1}}{\back} \, \underset{\alpha_n}{\back}
~\bigg|~
\underset{\alpha_2}{\back} \, \underset{\alpha_3}{\back} \, \underset{\alpha_4}{\back} \, \cdots \, \underset{\alpha_n}{\back} \, \underset{\alpha_1}{\back}
~\bigg|.\]
Then we obtain the following card-sequence:
\[\underbrace{\back\, \back\,  \cdots\, \back}_{\f{x}} ~~\underset{\beta_1}{\back} \, \underset{\beta_2}{\back} \, \underset{\beta_3}{\back} \, \cdots \, \underset{\beta_n}{\back}
~~
\underset{\gamma_1}{\back} \, \underset{\gamma_2}{\back} \, \underset{\gamma_3}{\back} \, \cdots \, \underset{\gamma_n}{\back}~,
\]
where $\{(\alpha_1,\alpha_2,\ldots,\alpha_n), (\alpha_2,\ldots,\alpha_n,\alpha_1)\}=\{(\beta_1,\beta_2,\ldots, \beta_n), (\gamma_1,\gamma_2,\ldots,\gamma_n)\}$.

\item[(4)] Arrange the card-sequence as follows:
\[ 
\underset{\f{x}_1}{\back} \,
\underset{\beta_1}{\back} \, \underset{\gamma_1}{\back} ~
\underset{\f{x}_2}{\back} \,
\underset{\beta_2}{\back} \, \underset{\gamma_2}{\back} ~
\underset{\f{x}_3}{\back} \,
\underset{\beta_3}{\back} \, \underset{\gamma_3}{\back} ~
\cdots ~
\underset{\f{x}_n}{\back} \,
\underset{\beta_n}{\back} \, \underset{\gamma_n}{\back}\,.
\]

\item[(5)] Apply $\PSS_{(n,3)}$ to the card-sequence as follows:
\[
\begin{tabular}{|c|c|c|c|c|}
$\back\, \back\, \back$ &
$\back\, \back\, \back$ &
$\back\, \back\, \back$ &
$\cdots$ &
$\back\, \back\, \back$
\end{tabular}\,.
\]

\item[(6)] For all piles, turn over the second and third cards. Let $a_i, b_i \in \{1,2,\ldots, n\}$ be the opened symbols of the second and third cards, respectively, in the $i$-th pile as follows:
%Then we suppose that the following card-sequence is obtained.
\[
\back\,\crd{$a_1$}\,\crd{$b_1$}~~\back\,\crd{$a_2$}\,\crd{$b_2$}~~\back\,\crd{$a_3$}\,\crd{$b_3$}~~\cdots~~\back\,\crd{$a_n$}\,\crd{$b_n$}~\md{.}
\]

\item[(7)] Arrange $n$ piles so that $(c_1, d_1) = (a_1, b_1)$, $d_i = c_{i+1}$, $(1 \leq i \leq n-1)$, and $d_n = c_1$ as follows:
\[
\underset{\f{y}_1}{\back} \,\crd{$c_1$}\,\crd{$d_1$}~~\underset{\f{y}_2}{\back} \,\crd{$c_2$}\,\crd{$d_2$}~~\underset{\f{y}_3}{\back} \,\crd{$c_3$}\,\crd{$d_3$}~~\cdots~~\underset{\f{y}_n}{\back} \,\crd{$c_n$}\,\crd{$d_n$}~.
\]
Then arrange the card-sequence as follows:
\[
\underset{\f{y}_1}{\back} \, \underset{\f{y}_2}{\back} \, \underset{\f{y}_3}{\back} \, \cdots \, \underset{\f{y}_n}{\back} ~ 
\underbrace{\crd{1}\,\crd{1}~\crd{2}\,\crd{2}~\crd{3}\,\crd{3}~\cdots\crd{$n$}\,\crd{$n$}}_{\f{h}}~. 
\]
Then the output card-sequence is $(\f{y}_1,\f{y}_2,\ldots, \f{y}_n)$.
\end{enumerate}


\md{We first show the correctness of the protocol. 
%In the last of this subsection, we show that the correctness and the security.
Let $\mathsf{x}=(\mathsf{x}_1,\ldots, \mathsf{x}_n)$ be an input sequence. 
Assume that the protocol outputs the sequence $\mathsf{y}=(\mathsf{y}_1,\ldots, \mathsf{y}_n)$ when $\mathsf{x}$ is given as input. 
%We first show that $\mathsf{y}$ is equal to $\sigma(\f{x})$ for some $\sigma \in \aut_0(C_n)$. 
%Assume that $\mathsf{x}$ outputs the sequence $\mathsf{y}=(\mathsf{y}_1,\ldots, \mathsf{y}_n)$ by applying the above protocol. 
Observe that if we apply a graph shuffle for $C_n$ to $\mathsf{x}$, the output sequence is one of the following sequences}
\[ \md{(\mathsf{x}_k,\mathsf{x}_{k+1},\ldots, \mathsf{x}_n, \mathsf{x}_1,\mathsf{x}_2\ldots, \mathsf{x}_{k-1}),\quad  (\mathsf{x}_k,\mathsf{x}_{k-1},\ldots, \mathsf{x}_1, \mathsf{x}_n,\mathsf{x}_{n-1},\ldots, \mathsf{x}_{k+1})}\]
\md{for some $k\in \{1,2,\ldots, n \}$. 
We denote by $\mathsf{Cyc}(k)$ and $\mathsf{Rev}(k)$ the former sequence and the latter sequence, respectively. 
To show the correctness of the protocol, we see that $\mathsf{y}$ is one of $\mathsf{Cyc}(k)$ and $\mathsf{Rev}(k)$ for some $k=1,\ldots, n$.
For $i=1,\ldots, n$, we set $P_i=(\mathsf{x}_i, \beta_i, \gamma_i)$ and put $P=(P_1,\ldots, P_n)$. 
Suppose that $(\alpha_1,\ldots, \alpha_n)$ is equal to $(\beta_1,\ldots, \beta_n)$ in Step (3). 
In this case, it holds $\gamma_i=\beta_{i+1}$ for any $i\in\{1,2,\ldots,n\}$, where $\beta_{n+1}=\beta_1$. 
%, hold. 
It follows from the above equations and the argument in the proof of the correctness of the protocol in Subsection \ref{4-1} that $\mathsf{y}=\mathsf{Cyc}(k)$ for some $k$. Similarly, if $(\alpha_1,\ldots, \alpha_n)=(\gamma_1,\ldots, \gamma_n)$ in Step (3), the equations $\gamma_i=\beta_{n-i+1}$ hold for any $i\in\{1,2,\ldots,n\}$. This implies that $\mathsf{y}=\mathsf{Rev}(k)$ for some $k$.}

\md{Next, we show that the distribution of $\mathsf{y}$ is uniform. Assume that $n=2$. We note that $\mathsf{Cyc}(1)=\mathsf{Rev}(1)$ and $\mathsf{Cyc}(2)=\mathsf{Rev}(2)$. Then the candidates appearing as a result of Step (4) are: }
\[ \md{\underset{\f{x}_1}{\back} \,
\underset{1}{\back} \, \underset{2}{\back} ~
\underset{\f{x}_2}{\back} \,
\underset{2}{\back} \, \underset{1}{\back} ~}, \quad 
\md{\underset{\f{x}_1}{\back} \,
\underset{2}{\back} \, \underset{1}{\back} ~
\underset{\f{x}_2}{\back} \,
\underset{1}{\back} \, \underset{2}{\back} ~}\ ,\] 
\md{and these each have a probability of $\dfrac{1}{2}$. Thus, the probabilities that $\mathsf{y}=(\mathsf{x}_1,\mathsf{x}_2)$ and $\mathsf{y}=(\mathsf{x}_2,\mathsf{x}_1)$ are same. Now, we assume that $n\geq 3$. In this case, for any $k=1,\ldots, n$, all sequences $\mathsf{Cyc}(k)$ and $\mathsf{Rev}(k)$ are distinct. In order to get $\mathsf{y}=\mathsf{Cyc}(k)$,  it requires that $(\alpha_1,\ldots, \alpha_n)=(\beta_1,\ldots, \beta_n)$ in Step (3) and $\sigma^{-1}(1)=k$, where $\sigma$ is the chosen permutation in Step (5). 
Hence, the probability that $\mathsf{y}=\mathsf{Cyc}(k)$ is $\dfrac{1}{2k}$. Similarly, the probability that $\mathsf{y}=\mathsf{Rev}(k)$ is also $\dfrac{1}{2k}$. This shows that the protocol is correct.}

\md{We show the correctness of the protocol. 
%Lastly, we show that the security of the protocol.
Assume that $\sigma\in\mathfrak{S}_n$ and $\tau\in \mathfrak{S}_n$ are chosen in Step (2) and Step (5), respectively. Then the first card in the $i$-th pile in Step (6) is $\mathsf{x}_{\tau^{-1}(i)}$. On the other hand, the second and third cards in the $i$-th pile in Step (5) are depending on the result of Step (3), and they are determined as follows. 
If $(\alpha_1,\ldots, \alpha_n)=(\beta_1,\ldots, \beta_n)$, then $a_i=\tau^{-1}\sigma^{-1}(i)$ and $b_i=\tau^{-1}\sigma^{-1}(i+1)$, 
otherwise, $a_i=\tau^{-1}\sigma^{-1}(i+1)$ and $b_i=\tau^{-1}\sigma^{-1}(i)$. Here, we consider $n+1$ as $1$. 
In either case,  these open symbols $a_1,\ldots, a_n$ and $b_1,\ldots, b_n$ do not allow us to guess the first card of any pile since $\sigma$ is chosen uniformly at random in Step (5). Therefore, the protocol is secure.}


%Let $D$ be an arbitrary deck with $n$ cards, say $x_1,x_2,\ldots ,x_n$, and $D''$ the set of $2n$ cards such that their fronts are 
%\[ \crd{1}\,\crd{1}~\crd{2}\,\crd{2}~\crd{3}\,\crd{3}~\cdots\crd{$n$}\,\crd{$n$}~. \]
%Similarly, we set $\f{pile}[i]=\left(\dfrac{?}{i}, \dfrac{?}{i}\right)$.
%
%\begin{enumerate}
%\item[(1)] For an input card-sequence $\f{x}$, we take $(\f{x}, \f{pile}[1], \f{pile}[2] ,\ldots,\f{pile}[n])$~.
%Applying $\PSS_{(n,2)}$ to the card-sequence $(\f{pile}[1], \f{pile}[2], ,\ldots,\f{pile}[n])$, we obtain the following card sequence 
%\[
%\underbrace{\back\, \back\,  \cdots\, \back}_{\f{x}}
%\underset{\phi_1}{\back} \, \underset{\phi_1}{\back}
%~
%\underset{\phi_2}{\back} \, \underset{\phi_2}{\back}
%~
%\underset{\phi_3}{\back} \, \underset{\phi_3}{\back}
%~
%\cdots
%~
%\underset{\phi_n}{\back} \, \underset{\phi_n}{\back}~,
%\]
%where $\{\phi_1,\phi_2,\ldots,\phi_n\}=\{1,2,\ldots, n\}$.
%
%\item[(2)] Arrange the card-sequence as follows:
%\[\f{z}=\underbrace{\back\, \back\,  \cdots\, \back}_{\f{x}} \underset{\phi_1}{\back} \, \underset{\phi_2}{\back} \, \underset{\phi_3}{\back} \, \cdots \, \underset{\phi_{n-1}}{\back} \, \underset{\phi_n}{\back}\,
%\underset{\phi_2}{\back} \, \underset{\phi_3}{\back} \, \underset{\phi_4}{\back} \, \cdots \, \underset{\phi_n}{\back} \, \underset{\phi_1}{\back}~. \] 
%
%\item[(3)] Apply $\PSS_{(2,n)}$ to the rightmost card-sequence of $2n$ cards as follows:
%%Choose between $\f{id}$ and $\sigma$ with a probability of $\dfrac{1}{2}$, where 
%%\[ \sigma=\left(\begin{array}{cccccccccccccccc} 
%%1 & \cdots & n & n+1 &n+2 & \cdots &2n  & 2n+1 & 2n+2 & \cdots & 3n\\
%%1 & \cdots & n &  2n+1 & 2n+2  & \cdots & 3n  & n+1 &n+2 & \cdots & 2n
%%\end{array}\right).  \]
%%Then we apply the chosen substitution to the card-sequence $\f{z}$. 
%%In other words, we perform the PSS
%\[
%\bigg|~
%\underset{\phi_1}{\back} \, \underset{\phi_2}{\back} \, \underset{\phi_3}{\back} \, \cdots \, \underset{\phi_{n-1}}{\back} \, \underset{\phi_n}{\back}
%~\bigg|~
%\underset{\phi_2}{\back} \, \underset{\phi_3}{\back} \, \underset{\phi_4}{\back} \, \cdots \, \underset{\phi_n}{\back} \, \underset{\phi_1}{\back}
%~\bigg|.\]
%Suppose that we obtain the following card-sequence by the operation described above:
%\[\underbrace{\back\, \back\,  \cdots\, \back}_{\f{x}} ~~\underset{\psi_1}{\back} \, \underset{\psi_2}{\back} \, \underset{\psi_3}{\back} \, \cdots \, \underset{\psi_n}{\back}
%~~
%\underset{\chi_1}{\back} \, \underset{\chi_2}{\back} \, \underset{\chi_3}{\back} \, \cdots \, \underset{\chi_n}{\back}~,
%\]
%where $\{(\phi_1,\phi_2,\ldots,\phi_n), (\phi_2,\ldots,\phi_n,\phi_1)\}=\{(\psi_1,\psi_2,\ldots, \psi_n), (\chi_1,\chi_2,\ldots,\chi_n)\}$.
%
%
%\item[(4)] Arrange the card-sequence as follows:
%\[ 
%\f{z}' = 
%\overset{\f{x}_1}{\back} \,
%\underset{\psi_1}{\back} \, \underset{\chi_1}{\back} ~
%\overset{\f{x}_2}{\back} \,
%\underset{\psi_2}{\back} \, \underset{\chi_2}{\back} ~
%\overset{\f{x}_3}{\back} \,
%\underset{\psi_3}{\back} \, \underset{\chi_3}{\back} ~
%\cdots ~
%\overset{\f{x}_n}{\back} \,
%\underset{\psi_n}{\back} \, \underset{\chi_n}{\back}~.
%\]
%
%\item[(5)] Apply $\PSS_{(n,3)}$ to the card-sequence $\f{z}'$.
%\[
%\begin{tabular}{|c|c|c|c|c|}
%$\back\, \back\, \back$ &
%$\back\, \back\, \back$ &
%$\back\, \back\, \back$ &
%$\cdots$ &
%$\back\, \back\, \back$
%\end{tabular}
%\]
%
%\item[(6)] For all piles, turn over the second and third cards. Let $a_i, b_i \in \{1,2,\ldots, n\}$ be the opened symbols of the second and third cards, respectively, in the $i$-th pile as follows:
%%Turn over all the second and third cards of each pile.  Then we suppose that the following card-sequence is obtained.
%\[
%\back\,\crd{$a_1$}\,\crd{$b_1$}~~\back\,\crd{$a_2$}\,\crd{$b_2$}~~\back\,\crd{$a_3$}\,\crd{$b_3$}~~\cdots~~\back\,\crd{$a_n$}\,\crd{$b_n$}
%\]
%
%\item[(7)] Arrange $n$ piles so that $(c_1, d_1) = (a_1, b_1)$, $d_i = c_{i+1}$, $(1 \leq i \leq n-1)$, and $d_n = c_1$ as follows:
%%Sort the above sequence of piles of cards so that $(c_1, d_1) = (a_1, b_1)$, $d_i = c_{i+1}$, $(1 \leq i \leq n-1)$, and $d_n = c_1$ are satisfied:
%\[
%\overset{\f{y}_1}{\back} \,\crd{$c_1$}\,\crd{$d_1$}~~\overset{\f{y}_2}{\back} \,\crd{$c_2$}\,\crd{$d_2$}~~\overset{\f{y}_3}{\back} \,\crd{$c_3$}\,\crd{$d_3$}~~\cdots~~\overset{\f{y}_n}{\back} \,\crd{$c_n$}\,\crd{$d_n$}~.
%\]
%Then the output card-sequence is $\f{y}=(\f{y}_1,\f{y}_2,\ldots, \f{y}_n)$.
%\end{enumerate}


%%%% Application %%%%
\section{Conclusions and Future Works}

\md{In this paper, we show that any graph shuffle can be done by PSSs. 
In particular, we need $2(n+m)$ cards and $|\f{Deg}_G|+1$ PSSs, where $n$ and $m$ are the numbers of vertices and arrows of $G$, respectively. 
We left as open problems (1) to remove the computation of an isomorphism between two isomorphic graphs in a graph shuffle protocol keeping everything efficient and (2) to find another interesting applications for our graph shuffle protocol. 
We hope that this research direction (i.e., constructing a nontrivial shuffle from the standard shuffles such as RCs, RBCs, and PSSs) will attract the interest of researchers on card-based cryptography and new shuffle protocols will be proposed in future work. 
}

%\section{準備}
%
%\subsection{カード}\label{ss:card}
%
%本稿では以下のカードを用いる。
%\[
%\crd{1}~\crd{2}~\crd{3}~\crd{4}~\crd{5}~\crd{6}~\cdots~\crd{$\stext{1}$}~\crd{$\stext{2}$}~\crd{$\stext{3}$}~\crd{$\stext{4}$}~\crd{$\stext{5}$}~\crd{$\stext{6}$}~\cdots
%\]
%$\crd{1}\,\crd{2}\,\crd{3}\,\crd{4}\,\crd{5}\,\crd{6}\,\cdots$を黒カードと呼び、$\crd{$\stext{1}$}\,\crd{$\stext{2}$}\,\crd{$\stext{3}$}\,\crd{$\stext{4}$}\,\crd{$\stext{5}$}\,\crd{$\stext{6}$}\,\cdots$を赤カードと呼ぶ。
%全てのカードの裏面は以下のように区別がつかないものとする。
%\[
%\back~\back~\back~\back~\back~\back~\cdots~\back~\back~\back~\back~\back~\back~\cdots
%\]
%
%\subsection{シャッフル}\label{ss:shuffle}
%
%シャッフルとは、カード列の確率的な並べ替え操作である。
%$n$枚のカード列に対するシャッフルは、置換の集合$\Pi \subset \mathfrak{S}_n$(ここで$\mathfrak{S}_n$は$n$次対称群)及び$\Pi$上の確率分布$\F$によってパラメトライズされ、$(\shuffle, \Pi, \F)$と表記される。
%シャッフル$(\shuffle, \Pi, \F)$は、確率分布$\F$によって選ばれた置換$\pi \in \Pi$に従って、$i$番目のカードを$\pi(i)$番目に移動する並べ替え操作を表す\footnote{シャッフル操作を適用したときに、実際にどの置換が選ばれたかは誰も知らない。}。
%$\Pi$が置換の合成について閉じており、$\F$が一様分布であるようなシャッフルは一様閉シャッフルと呼ばれており、性質の良いシャッフルのクラスとして知られている。
%本稿で登場するシャッフルは、全て一様閉シャッフルである。
%特に本研究では、一様閉シャッフルの中でも最も単純なシャッフルであるパイルスクランブルシャッフルが重要な役割を果たす。
%%この性質(選ばれた置換の秘匿性)を実際に実現するシャッフルの実装方法は非自明であり、安全なシャッフルの実装方法を提案する研究も知られているが、本研究ではここに深く
%%シャッフルの実装方法には、手操作で実装する方法や、補助的な道具を用いる方法など、さまざまな方法がある。
%%あるシャッフル$X$を実装方法$Y$によって実装した際に、どのプレイヤーに対しても選ばれた置換が秘匿されているとき、$Y$は$X$の安全な実装であるといい、$X$は安全に実装可能であると言う。
%%\[
%%\overset{1}{\back}\,\overset{2}{\back}\,\overset{3}{\back}\,\overset{4}{\back}\,\overset{5}{\back}\,\overset{6}{\back}\,\cdots\,\overset{n}{\back}\, \xrightarrow{\pi \la \F} \overset{\pi^{-1}(1)}{\back}\,\overset{\pi^{-1}(2)}{\back}\,\overset{\pi^{-1}(3)}{\back}\,\overset{\pi^{-1}(4)}{\back}\,\overset{\pi^{-1}(5)}{\back}\,\overset{\pi^{-1}(6)}{\back}\,\cdots\,\overset{\pi^{-1}(n)}{\back}
%%\]
%
%%\subsubsection{シャッフルの定義}\label{sss:shuffle}
%%
%%シャッフルとは、カード列の確率的な並べ替え操作である。
%%$n$枚のカード列に対するシャッフルは、置換の集合$\Pi \subset S_n$(ここで$S_n$は$n$次対称群)及び$\Pi$上の確率分布$\F$で定義され、$(\shuffle, \Pi, \F)$と表記する。
%%このシャッフルは、確率分布$\F$に従って置換$\pi \in \Pi$が選ばれ、$\pi$に従ってカード列を並び替えるシャッフルを意味する。
%%シャッフル$(\shuffle, \Pi, \F)$が一様シャッフルであるとは、$\F$が$\Pi$上の一様分布(各$\pi \in \Pi$が確率$1/|\Pi|$で選ばれるような分布)であることである。
%%一様シャッフルのときは確率分布$\F$を省略し、$(\shuffle, \Pi)$と書くことにする。
%%シャッフル$(\shuffle, \Pi, \F)$が閉シャッフルであるとは、$\Pi$が置換の合成について閉じていること(任意の$\pi, \pi' \in \Pi$について$\pi \pi' \in \Pi$ )である。
%%シャッフルが一様閉シャッフルであるとは、一様シャッフルかつ閉シャッフルであることである。
%%
%%%カードベース暗号プロトコルにおいて特に重要と考えられているシャッフルは、ランダムカット(以降RCと表記)とパイルスクランブルシャッフル(以降PSSと表記)である\footnote{もう一つ特筆すべきシャッフルとして、Mizuki--Soneの提案したランダム二等分割カットがあるが、これはパイルスクランブルシャッフルの特別な場合と考えられるため、本稿ではランダム二等分割カットを特別扱いしないこととする。}。
%%%RCはランダム巡回置換を施すシャッフルであり、PSSはランダム置換を施すシャッフルである。
%
%\subsubsection{パイルスクランブルシャッフル}\label{sss:pss}
%
%%カード列のカード枚数を$\ell = mk$とする。
%($k$個の$m$枚束に対する)パイルスクランブルシャッフルとは、$\ell = km$枚のカード列を各$m$枚の$k$個の束に分割し、$k$個の束に対して$k!$通りの一様ランダムな並び替えを施すシャッフルである。
%以降、パイルスクランブルシャッフルをPSSと略記する。
%
%以下の例は$3$個の$2$枚束に対するPSSである。
%\[
%\overset{1}{\back}\,\overset{2}{\back}~~\overset{3}{\back}\,\overset{4}{\back}~~\overset{5}{\back}\,\overset{6}{\back}\, \rightarrow
%\begin{cases}
%\overset{1}{\back}\,\overset{2}{\back}~~\overset{3}{\back}\,\overset{4}{\back}~~\overset{5}{\back}\,\overset{6}{\back} & \text{w.p. $1/6$}\\
%\overset{1}{\back}\,\overset{2}{\back}~~\overset{5}{\back}\,\overset{6}{\back}~~\overset{3}{\back}\,\overset{4}{\back} & \text{w.p. $1/6$}\\
%\overset{3}{\back}\,\overset{4}{\back}~~\overset{5}{\back}\,\overset{6}{\back}~~\overset{1}{\back}\,\overset{2}{\back} & \text{w.p. $1/6$}\\
%\overset{3}{\back}\,\overset{4}{\back}~~\overset{1}{\back}\,\overset{2}{\back}~~\overset{5}{\back}\,\overset{6}{\back} & \text{w.p. $1/6$}\\
%\overset{5}{\back}\,\overset{6}{\back}~~\overset{1}{\back}\,\overset{2}{\back}~~\overset{3}{\back}\,\overset{4}{\back} & \text{w.p. $1/6$}\\
%\overset{5}{\back}\,\overset{6}{\back}~~\overset{3}{\back}\,\overset{4}{\back}~~\overset{1}{\back}\,\overset{2}{\back} & \text{w.p. $1/6$}\\
%\end{cases}
%\]
%右側の$3! = 6$通りのカード列は、それぞれ$1/6$の確率で選ばれる。
%
%各束を$|\cdot |$で囲うことにより、PSSを以下のように表記する。
%\[
%\begin{tabular}{|c|c|c|}
%\back\,\back & \back\,\back & \back\,\back \\
%\end{tabular}
%\]
%以下のような二段組の表記も、同様のPSSを表す。
%例えば、以下は$5$個の$2$枚束に対するPSSを表す。
%\[
%\begin{tabular}{|c|c|c|c|c|}
%\back & \back & \back & \back & \back\\
%\back & \back & \back & \back & \back
%\end{tabular}
%\]
%
%%\subsubsection{ランダムカット}\label{sss:rc}
%
%\subsection{プロトコル}\label{ss:protocol}
%
%カードベース暗号プロトコルの標準的な定義はMizuki and Shizuyaで与えられており、本稿もそれに従う。
%本節では、プロトコルの簡単な説明をする。
%また、シャッフルプロトコルという新しい概念を説明する。
%
%\subsubsection{プロトコル}\label{sss:reduction}
%
%プロトコルとは、カード列に対して次に適用する操作を選ぶチューリングマシンである。
%カード列に対して適用する操作は、並べ替え操作、ターン操作(カードをめくったり伏せたりする操作)、シャッフル操作の三種類である。
%次に適用する操作は、visible sequenceとチューリングマシンの内部状態のみによって定まる。
%ここで、visible sequenceについて具体例を一つ与える。
%プロトコルが$\back\,\back\,\back\,\back$に対してシャッフルを行い、$1$枚目をめくり$\red$が出たとすると、カード列は以下のように変遷する。
%\[
%\back\,\back\,\back\,\back ~\xrightarrow{\shuffle}~
%\back\,\back\,\back\,\back ~\xrightarrow{\turn}~
%\red\,\back\,\back\,\back
%\]
%このときのvisible sequenceは、シンボルの列$(\textbf{?}\textbf{?}\textbf{?}\textbf{?}, \textbf{?}\textbf{?}\textbf{?}\textbf{?}, \heart\textbf{?}\textbf{?}\textbf{?})$のことである。
%
%プロトコルの正当性とは、プロトコルが必ず正しい計算を行う性質である。
%プロトコルの安全性とは、visible sequenceの確率分布と入力の確率分布が独立である性質である。
%
%\subsubsection{シャッフルプロトコル}\label{sss:reduction}
%
%シャッフル$(\shuffle, \Pi, \F)$を$n$枚のカード列に対するシャッフルであるとする。
%$n$枚のカード列$\back\,\back\,\cdots\,\back$を入力カード列$S_0$とし、$m \geq 0$枚の補助カード列$S_1$を用いて、連結したカード列$S_0||S_1$から出発し、最終的に$S'_0||S'_1$を出力するプロトコルを考える。
%\[
%\underbrace{\back\,\back\,\cdots\,\back}_{S_0}~\underbrace{\blk\,\red\,\cdots\,\crd{1}~\crd{2}~\crd{3}\,\cdots}_{S_1}
%~\lra~\cdots~\lra~
%\underbrace{\back\,\back\,\cdots\,\back}_{S'_0}~\underbrace{\blk\,\red\,\cdots\,\crd{1}~\crd{2}~\crd{3}\,\cdots}_{S'_1}
%\]
%このプロトコルが以下の条件を全て満たすとき、これはシャッフル$(\shuffle, \Pi, \F)$に対するシャッフルプロトコルであるという。
%\begin{itemize}
%\item 出力前半部$S'_0$に含まれるカードは、入力カード列$S_0$に含まれるいずれかのカードである。
%\item 出力後半部$S'_1$に含まれるカードは、補助カード列$S_1$に含まれるいずれかのカードである。
%\item プロトコル中に入力カード列$S_0$に含まれるどのカードもめくらない。
%\item $S'_0$の確率分布は、入力カード列$S_0$に対してシャッフル$(\shuffle, \Pi, \F)$を適用した確率分布と同一である。
%\end{itemize}
%
%シャッフルプロトコルが安全であるとは、visible sequenceの確率分布と$S'_0$の確率分布が独立であることである。
%
%あるシャッフルの集合$X$が存在して、シャッフル$(\shuffle, \Pi, \F)$に対するシャッフルプロトコルの中で用いるシャッフルは全て$X$に含まれるとき、シャッフル$(\shuffle, \Pi, \F)$は$X$によって実現可能であるという。
%特に本稿では$X$としてPSSの集合を考える。
%
%%シャッフルプロトコルとは、あるシャッフルと同等の効果を持つプロトコルである。
%%例えば、一様閉シャッフル$(\shuffle, \{\id, (1\;2), (3\;4), (1\;2)(3\;4)\})$は、「二つの一様閉シャッフル$(\shuffle, \{\id, (1\;2)\})$と$(\shuffle, \{\id, (3\;4)\})$を連続して適用する」というプロトコルと同等である(ここで、一様シャッフルだから確率分布を省略している)。
%%あるカード列に対して、シャッフル$s$を適用した際の出力のカード列の確率分布と、あるプロトコル$\Pi$を実行した際の出力のカード列の確率分布が等しいとき、$\Pi$は$s$のシャッフルプロトコルであるという。
%%さらに、あるシャッフルの族$S$が存在して、$\Pi$の実行中に用いるシャッフルは全て$S$に含まれるとき、シャッフル$s$は$S$によって実現可能であるという。
%%特に本稿では$S$としてPSSの族を考える。
%
%%$S = \{(\shuffle, \Pi_i, \F_i)\}_{i \in I}$をシャッフルの族とする。
%%あるシャッフル$s = (\shuffle, \Pi, \F)$が$S$から実現可能であるとは、以下の全ての条件を満たすカードベース暗号プロトコルが存在することである。
%%\begin{itemize}
%%\item $S$に含まれるシャッフルのみを用いる。
%%\item プロトコル中に$\Gamma$に含まれるカードを一切オープンしない。
%%\item カード列$\Gamma = \back \, \back \, \back \, \cdots \, \back$をプロトコルに入力したときの出力結果のカード列の確率分布と、$\Gamma$に$s$を適用したときのカード列の確率分布が全く等しい。
%%\end{itemize}
%%
%%本稿では特に$S$としてPSSの族を考える。
%%以降では「シャッフル$X$をPSSから実現可能である」と書いた場合には、「シャッフル$X$はPSSの族から実現可能である」ということを意味する。
%
%\section{グラフシャッフルプロトコル}\label{s:graph}
%
%本章ではグラフシャッフルプロトコルを構成する。
%グラフシャッフルプロトコルとは、\ref{ss:graph}節で定義するグラフシャッフルに対するシャッフルプロトコルである。
%グラフシャッフルプロトコルの構成は\ref{ss:graphprotocol}節で与える。
%なお、グラフシャッフルプロトコルの中で用いられるシャッフルはPSSのみである。
%
%\subsection{グラフシャッフル}\label{ss:graph}
%
%$G = (V, E)$を頂点数$|V| = n$および辺数$|E| = m$の有向グラフとする。
%$G$の自己同型群$\aut(G)$とは、$G$の自己同型写像\footnote{全単射$\phi: V \ra V$が$G$の自己同型写像であるとは、辺$(i, j) \in E$が存在するならば辺$(\phi(i), \phi(j)) \in E$も存在し、その逆も成り立つことである。}からなる群である。
%$\aut(G)$は自然に$\mathfrak{S}_n$の部分群とみなせる。
%
%$G$のグラフシャッフルとは、$n$枚のカード列に対する一様閉シャッフル$(\shuffle, \aut(G))$である。
%
%\textcolor{red}{残タスク:ここに図とともにグラフシャッフルの例を書く。}
%
%なお、次節のプロトコルは有向グラフに対するグラフシャッフルプロトコルであるが、無向グラフを扱いたい場合は、無向辺$(i, j) \in E$を、二つの有向辺$(i, j), (j, i) \in E$と考えればよい。
%
%\subsection{グラフシャッフルプロトコル}\label{ss:graphprotocol}
%
%\begin{figure}[h]
%\begin{center}
% \includegraphics[width=0.2\textwidth,bb=0 0 446 336]{graph_example.pdf}
%  \caption{グラフ$G_0$}
%  \label{fig:graph1}
%\end{center}
%\end{figure}
%
%提案プロトコルでは、$n$枚の入力カード列に加えて、以下の補助カード列を用いる。
%\[
%\crd{$\stext{1}$} \, \crd{$\stext{2}$} \, \crd{$\stext{3}$} ~\cdots ~\crd{$\stext{n}$}~
%\crd{1} ~\cdots ~ \crd{2} ~\cdots ~ \crd{3}  ~ \cdots \cdots~ \crd{$n$}
%\]
%赤カード$\crd{$\stext{i}$}$は各頂点$i$につき一枚である。
%黒カード$\crd{$i$}$は各頂点$i$につき$\degree(i) = \indegree(i) + \outdegree(i)$枚である。
%($\indegree(i)$と$\outdegree(i)$はそれぞれ頂点$i$の入次数と出次数を表す。)
%$\sum_{i \in V}\indegree(i) = \sum_{i \in V}\outdegree(i) = m$だから、補助カード列のカード枚数は$n + 2m$枚である。
%
%具体例として図\ref{fig:graph1}のグラフ$G_0 = (V_0, E_0)$を考える。
%頂点集合は$V_0 = \{1, 2, 3, 4, 5\}$、辺集合は$E_0 = \{(1,2), (1,3), (2,1), (2,3), (3,4), (3,5)\}$である。
%この場合、補助カード列は以下の$17$枚である。
%\[
%\crd{$\stext{1}$} \, \crd{$\stext{2}$} \, \crd{$\stext{3}$} \, \crd{$\stext{4}$} \, \crd{$\stext{5}$}~
%\crd{1} \, \crd{1} \, \crd{1}  ~ \crd{2} \, \crd{2} \, \crd{2} ~ \crd{3} \, \crd{3} \, \crd{3} \, \crd{3} ~  \crd{4} ~ \crd{5}
%\]
%
%それでは、グラフシャッフルプロトコルの手続きを説明する。
%分かりやすさのため、図\ref{fig:graph1}のグラフ$G_0$を用いて説明を行うが、一般のグラフに対しても自然に拡張できる。
%%\footnote{\ref{sec:dihedral}章で述べるが、このグラフ$G_0$の場合、もっと効率的にグラフシャッフルを実現できる。}
%
%\begin{enumerate}
%
%\item 各頂点$i \in V$について、$\crd{$\stext{i}$} \, \crd{$i$} \,\crd{$i$} \,\cdots \,\crd{$i$}$を、一つの束とみなす。
%まず、$n$個の束を頂点の次数(=入次数と出次数の和)についてグループ分けをする。
%グラフ$G_0$の場合、頂点$1, 2$が次数$3$のグループ、頂点$3$のみが次数$4$のグループ、頂点$4, 5$が次数$1$のグループである。
%\[
%\underbrace{\crd{$\stext{1}$} \, \crd{1} \, \crd{1} \, \crd{1}~~\crd{$\stext{2}$} \, \crd{2} \, \crd{2} \, \crd{2}}_{\text{次数$3$のグループ}}~~
%\underbrace{\crd{$\stext{3}$} \, \crd{3} \, \crd{3} \, \crd{3} \, \crd{3}}_{\text{次数$4$のグループ}}~~
%\underbrace{\crd{$\stext{4}$} \, \crd{4}~~\crd{$\stext{5}$} \, \crd{5}}_{\text{次数$1$のグループ}}
%\]
%
%\item 各グループについて、グループに属する束たちを、パイルスクランブルシャッフルする。
%グラフ$G_0$の場合、次数$3$のグループに対しては、$\crd{$\stext{1}$} \, \crd{1} \, \crd{1} \, \crd{1}$と$\crd{$\stext{2}$} \, \crd{2} \, \crd{2} \, \crd{2}$のパイルスクランブルシャッフルを行い、$\crd{$\stext{\phi_1}$} \, \crd{$\phi_1$} \, \crd{$\phi_1$} \, \crd{$\phi_1$}$と$\crd{$\stext{\phi_2}$} \, \crd{$\phi_2$} \, \crd{$\phi_2$} \, \crd{$\phi_2$}$を得る(ただし$(\phi_1, \phi_2) \in \{(1,2), (2,1)\}$である)。
%次数$4$のグループは一つの束$\crd{$\stext{3}$} \, \crd{3} \, \crd{3} \, \crd{3} \, \crd{3}$しかないので、何もしなくてよい。
%次数$1$のグループに対しては、$\crd{$\stext{4}$} \, \crd{4}$と$\crd{$\stext{5}$} \, \crd{5}$のパイルスクランブルシャッフルを行い、$\crd{$\stext{\phi_4}$} \, \crd{$\phi_4$}$と$\crd{$\stext{\phi_5}$} \, \crd{$\phi_5$}$を得る(ただし$(\phi_4, \phi_5) \in \{(4,5), (5,4)\}$である)。
%\begin{align*}
%&\bigg|~
%\underset{\stext{1}}{\back} \, \underset{1}{\back} \, \underset{1}{\back} \, \underset{1}{\back}
%~\bigg|~
%\underset{\stext{2}}{\back} \, \underset{2}{\back} \, \underset{2}{\back} \, \underset{2}{\back}
%~\bigg|
%~~\ra~~
%\underset{\stext{\phi_1}}{\back} \, \underset{\phi_1}{\back} \, \underset{\phi_1}{\back} \, \underset{\phi_1}{\back}~~
%\underset{\stext{\phi_2}}{\back} \, \underset{\phi_2}{\back} \, \underset{\phi_2}{\back} \, \underset{\phi_2}{\back}\\
%%
%&\bigg|~
%\underset{\stext{4}}{\back} \, \underset{4}{\back}
%~\bigg|~
%\underset{\stext{5}}{\back} \, \underset{5}{\back}
%~\bigg|
%~~\ra~~
%\underset{\stext{\phi_4}}{\back} \, \underset{\phi_4}{\back}~~
%\underset{\stext{\phi_5}}{\back} \, \underset{\phi_5}{\back}
%\end{align*}
%$\phi_3 = 3$とすれば、現在のカード列は以下のように表せる。
%\[
%\crd{$\stext{\phi_1}$} \, \crd{$\phi_1$} \, \crd{$\phi_1$} \, \crd{$\phi_1$}~~
%\crd{$\stext{\phi_2}$} \, \crd{$\phi_2$} \, \crd{$\phi_2$} \, \crd{$\phi_2$}~~
%\crd{$\stext{\phi_3}$} \, \crd{$\phi_3$} \, \crd{$\phi_3$} \, \crd{$\phi_3$} \, \crd{$\phi_3$}~~
%\crd{$\stext{\phi_4}$} \, \crd{$\phi_4$}~~
%\crd{$\stext{\phi_5}$} \, \crd{$\phi_5$}
%\]
%なお、$\phi_1, \phi_2, \phi_4, \phi_5$の実際の値は誰も知らないことに注意する。
%
%\item 以下のようにカード列を二段に並べる。
%まず、$\crd{$\stext{\phi_1}$} \, \crd{$\stext{\phi_2}$} \, \crd{$\stext{\phi_3}$} \, \crd{$\stext{\phi_4}$} \, \crd{$\stext{\phi_5}$}$を上段に並べ、その下に入力カード列$\back\,\back\,\back\,\back\,\back$を並べる。
%次に、各辺$(i, j) \in E_0$に対して、$\crd{$\phi_i$}$を上段に並べ、その下に$\crd{$\phi_j$}$を並べる。
%\begin{align*}
%&\overset{\stext{\phi_1}}{\back} \, \overset{\stext{\phi_2}}{\back} \, \overset{\stext{\phi_3}}{\back} \, \overset{\stext{\phi_4}}{\back} \, \overset{\stext{\phi_5}}{\back} \,
%\overset{\phi_1}{\back} \, \overset{\phi_1}{\back} \,
%\overset{\phi_2}{\back} \, \overset{\phi_2}{\back} \,
%\overset{\phi_3}{\back} \, \overset{\phi_3}{\back} \\
%&\back \, \back \, \back \, \back \, \back \,
%\overset{\phi_2}{\back} \, \overset{\phi_3}{\back} \,
%\overset{\phi_1}{\back} \, \overset{\phi_3}{\back} \,
%\overset{\phi_4}{\back} \, \overset{\phi_5}{\back}
%\end{align*}
%
%\item 各列を一つの束として、パイルスクランブルシャッフルを適用する。
%\[
%\begin{tabular}{|c|c|c|c|c|c|c|c|c|c|c|}
%\back & \back & \back & \back & \back & \back & \back & \back & \back & \back & \back \\
%\back & \back & \back & \back & \back & \back & \back & \back & \back & \back & \back
%\end{tabular}
%\]
%
%\item 一段目のカードを全てめくる。すると$\stext{1}\stext{2}\stext{3}\stext{4}\stext{5}112233$を並べ替えた列が得られる。
%例えば、以下のようなカード列が得られる。
%\begin{align*}
%&\crd{$\stext{5}$}\,\crd{1}\,\crd{2}\,\crd{$\stext{4}$}\,\crd{$\stext{2}$}\,\crd{2}\,\crd{$\stext{1}$}\,\crd{3}\,\crd{3}\,\crd{1}\,\crd{$\stext{3}$}
%\,\\
%&\back\,\back\,\back\,\back\,\back\,\back\,\back\,\back\,\back\,\back\,\back\,
%\end{align*}
%
%\item 一段目が$\stext{1}\stext{2}\stext{3}\stext{4}\stext{5}112233$
%の順になるように、列ごとに(上下の$2$枚組をセットで)並べ替える。
%\begin{align*}
%&\crd{$\stext{1}$}\,\crd{$\stext{2}$}\,\crd{$\stext{3}$}\,\crd{$\stext{4}$}\,\crd{$\stext{5}$}\,\crd{1}\,\crd{1}\,\crd{2}\,\crd{2}\,\crd{3}\,\crd{3}\,\\
%&\back\,\back\,\back\,\back\,\back\,\back\,\back\,\back\,\back\,\back\,\back\,
%\end{align*}
%
%\item 一段目の黒カードの下のカードを全てめくる。
%例えば、以下のようなカード列が得られる。
%\begin{align*}
%&\crd{$\stext{1}$}\,\crd{$\stext{2}$}\,\crd{$\stext{3}$}\,\crd{$\stext{4}$}\,\crd{$\stext{5}$}\,\crd{1}\,\crd{1}\,\crd{2}\,\crd{2}\,\crd{3}\,\crd{3}\,\\
%&\back\,\back\,\back\,\back\,\back\,\crd{3}\,\crd{2}\,\crd{3}\,\crd{1}\,\crd{5}\,\crd{4}\,
%\end{align*}
%
%\item ステップ$7$のオープンの結果に従って、グラフ$G_0' = (V_0', E_0')$を以下のように定義する:
%\begin{itemize}
%\item 頂点集合を$V_0'$は$V_0$と同じである。
%\item 上段の黒カード$\crd{$i$}$の下に$\crd{$j$}$が現れているとき、またそのときに限り、$(i, j) \in E'$とする。
%\end{itemize}
%ステップ2の$\phi_i$の定義より、辺$(i, j) \in E_0$が存在することと辺$(\phi_i, \phi_j) \in E'_0$が存在することは同値である。
%したがって、グラフ$G_0'$は$G_0$の頂点の名前を付け替えたグラフであり、$G'_0$と$G_0$は同型である。
%$|\aut(G)|$個の同型写像の中から具体的に一つ$\psi: V'\ra V$を選ぶ(同型写像を求める計算のコストに関しては効率性のパラグラフで述べる)。
%$\psi_i := \psi(i)$とする。
%$\crd{$\stext{\psi_1}$}\, \crd{$\stext{\psi_2}$} \, \crd{$\stext{\psi_3}$} \, \crd{$\stext{\psi_4}$} \, \crd{$\stext{\psi_5}$}$の下にあるカードを順番に集め、その五枚のカード列$\back \, \back \, \back \, \back \, \back$を出力結果とする。
%ステップ7で図示したオープン結果の場合、$G'$は$1 \ra 3 \ra 4 \ra 2 \ra 1$という四角形なので、同型写像として$(\psi_1, \psi_2, \psi_3, \psi_4, \psi_5) = (1, 2, 3, 4, 5)$が取れるので、$\crd{$\stext{1}$}\,\crd{$\stext{2}$}\,\crd{$\stext{3}$}\,\crd{$\stext{4}$}\,\crd{$\stext{5}$}$の下のカードが出力結果である。
%\end{enumerate}
%
%\paragraph{\bf 正当性}
%\textcolor{red}{残タスク:正当性の証明を書く。出力結果のカード列は、入力カード列に対して一様ランダムな元$\pi \in \aut(G)$を適用したものと同じ分布であることを言えればよい。}
%
%\paragraph{\bf 安全性}
%\textcolor{red}{残タスク:安全性の証明を書く。ステップ7においてオープンされているシンボルの分布が、出力結果のカード列の分布と独立であることを言えればよい。}
%
%\paragraph{\bf 効率性}
%\textcolor{red}{残タスク:カード枚数とシャッフル回数について改めて書く。それと、二つのグラフの同型写像を求める計算コストについてBabaiの結果を引用する。}
%
%\section{具体的なグラフクラスに対する効率化}
%
%本章では、いくつかの具体的なグラフクラスに対して、グラフシャッフルプロトコルを効率化する。
%
%\subsection{有向閉路(ランダムカット)}
%
%\begin{figure}[h]
%\begin{center}
% \includegraphics[width=0.2\textwidth,bb=0 0 339 339]{graph_directed_cycle.pdf}
%  \caption{長さ$4$の有向閉路}
%  \label{fig:graph_directed_cycle}
%\end{center}
%\end{figure}
%
%有向閉路のグラフシャッフルは、ランダムカットと等しい。
%例えば、図\ref{fig:graph_directed_cycle}のグラフに対するグラフシャッフルは、$4$枚のランダムカットと等しい。
%長さ$n$の有向閉路に対してグラフシャッフルプロトコル(\ref{s:graph}章)を直接適用すると、$3n$枚の補助カードが必要である。
%本節では、これを効率化し、$2n$枚の補助カードを用いる方法を提案する。
%その手順は以下の通りである。
%
%\begin{enumerate}
%\item 用いる補助カードは$\crd{1}\,\crd{1}~\crd{2}\,\crd{2}~\crd{3}\,\crd{3}~\cdots\crd{$n$}\,\crd{$n$}$の$2n$枚である。
%まず、$n$個の二枚組(同じ数字のペア)に対してパイルスクランブルシャッフルを適用する。
%得られたカード列を$\crd{$\phi_1$}\,\crd{$\phi_1$}~\crd{$\phi_2$}\,\crd{$\phi_2$}~\crd{$\phi_3$}\,\crd{$\phi_3$}~\cdots\crd{$\phi_n$}\,\crd{$\phi_n$}$とする。
%\[
%\bigg|~
%\underset{1}{\back} \, \underset{1}{\back}
%~\bigg|~
%\underset{2}{\back} \, \underset{2}{\back}
%~\bigg|~
%\underset{3}{\back} \, \underset{3}{\back}
%~\bigg|~
%\cdots
%~\bigg|~
%\underset{n}{\back} \, \underset{n}{\back}
%~\bigg|
%~~\ra~~
%\underset{\phi_1}{\back} \, \underset{\phi_1}{\back}
%~
%\underset{\phi_2}{\back} \, \underset{\phi_2}{\back}
%~
%\underset{\phi_3}{\back} \, \underset{\phi_3}{\back}
%~
%\cdots
%~
%\underset{\phi_n}{\back} \, \underset{\phi_n}{\back}
%\]
%
%\item 入力カード列の各カードの間に以下のように二枚ずつ補助カードを挿入する。
%\[
%\overset{1}{\back} \,
%\overset{2}{\back} \,
%\overset{3}{\back} \,
%\cdots \,
%\overset{n}{\back}
%~~\ra~~
%\overset{1}{\back} \,
%\underset{\phi_1}{\back} \, \underset{\phi_2}{\back} ~
%\overset{2}{\back} \,
%\underset{\phi_2}{\back} \, \underset{\phi_3}{\back} ~
%\overset{3}{\back} \,
%\underset{\phi_3}{\back} \, \underset{\phi_4}{\back} ~
%\cdots ~
%\overset{n}{\back} \,
%\underset{\phi_n}{\back} \, \underset{\phi_1}{\back}
%\]
%カードの上に書いてある数字は、入力カード列の何番目かを表している。
%
%\item 三枚組を束として、$n$個の束に対してパイルスクランブルシャッフルを適用する。
%\[
%\begin{tabular}{|c|c|c|c|c|}
%$\back\, \back\, \back$ &
%$\back\, \back\, \back$ &
%$\back\, \back\, \back$ &
%$\cdots$ &
%$\back\, \back\, \back$
%\end{tabular}
%\]
%
%\item 各束の二枚目と三枚目を全てめくる。
%%以下のルールにしたがって束をソートする。
%以下のようなカード列が得られる。
%\[
%\back\,\crd{$a_1$}\,\crd{$b_1$}~~\back\,\crd{$a_2$}\,\crd{$b_2$}~~\back\,\crd{$a_3$}\,\crd{$b_3$}~~\cdots~~\back\,\crd{$a_n$}\,\crd{$b_n$}
%\]
%
%\item 以下のように束を並び替える。
%\[
%\back\,\crd{$c_1$}\,\crd{$d_1$}~~\back\,\crd{$c_2$}\,\crd{$d_2$}~~\back\,\crd{$c_3$}\,\crd{$d_3$}~~\cdots~~\back\,\crd{$c_n$}\,\crd{$d_n$}
%\]
%ただし$c_i, d_i$は、$(c_1, d_1) = (a_1, b_1)$、$d_i = c_{i+1}$($1 \leq i \leq n-1$)、および$d_n = c_1$という条件を満たすものとする。
%
%\item 全ての補助カードを取り除く。残ったカード列が出力カード列である。
%\end{enumerate}
%
%\subsection{無向閉路(二面体群)}\label{sec:dihedral}
%
%\begin{figure}[h]
%\begin{center}
% \includegraphics[width=0.2\textwidth,bb=0 0 339 339]{graph_undirected_cycle.pdf}
%  \caption{長さ$4$の無向閉路}
%  \label{fig:graph_undirected_cycle}
%\end{center}
%\end{figure}
%
%無向閉路の自己同型群は、二面体群である。
%長さ$n$の無向閉路に対してグラフシャッフルプロトコル(\ref{s:graph}章)を直接適用すると、$3n$枚の補助カードが必要である。
%本節では、これを効率化し、$2n$枚の補助カードを用いる方法を提案する。
%その手順は以下の通りである。
%
%\begin{enumerate}
%\item 用いる補助カードは$\crd{1}\,\crd{1}~\crd{2}\,\crd{2}~\crd{3}\,\crd{3}~\cdots\crd{$n$}\,\crd{$n$}$の$2n$枚である。
%まず、$n$個の二枚組(同じ数字のペア)に対してパイルスクランブルシャッフルを適用する。
%得られたカード列を$\crd{$\phi_1$}\,\crd{$\phi_1$}~\crd{$\phi_2$}\,\crd{$\phi_2$}~\crd{$\phi_3$}\,\crd{$\phi_3$}~\cdots\crd{$\phi_n$}\,\crd{$\phi_n$}$とする。
%\[
%\bigg|~
%\underset{1}{\back} \, \underset{1}{\back}
%~\bigg|~
%\underset{2}{\back} \, \underset{2}{\back}
%~\bigg|~
%\underset{3}{\back} \, \underset{3}{\back}
%~\bigg|~
%\cdots
%~\bigg|~
%\underset{n}{\back} \, \underset{n}{\back}
%~\bigg|
%~~\ra~~
%\underset{\phi_1}{\back} \, \underset{\phi_1}{\back}
%~
%\underset{\phi_2}{\back} \, \underset{\phi_2}{\back}
%~
%\underset{\phi_3}{\back} \, \underset{\phi_3}{\back}
%~
%\cdots
%~
%\underset{\phi_n}{\back} \, \underset{\phi_n}{\back}
%\]
%
%\item 二つのカード列$\crd{$\phi_1$}\,\crd{$\phi_2$}\,\crd{$\phi_3$}\,\cdots\,\crd{$\phi_n$}$と$\crd{$\phi_2$}\,\crd{$\phi_3$}\,\crd{$\phi_4$}\,\cdots\,\crd{$\phi_n$}\,\crd{$\phi_1$}$に対してPSSを適用する。
%得られたカード列を$\crd{$\psi_1$}\,\crd{$\psi_2$}\,\crd{$\psi_3$}\,\cdots\,\crd{$\psi_n$}$と$\crd{$\chi_1$}\,\crd{$\chi_2$}\,\crd{$\chi_3$}\,\cdots\,\crd{$\chi_n$}$とする。
%\[
%\bigg|~
%\underset{\phi_1}{\back} \, \underset{\phi_2}{\back} \, \underset{\phi_3}{\back} \, \cdots \, \underset{\phi_{n-1}}{\back} \, \underset{\phi_n}{\back}
%~\bigg|~
%\underset{\phi_2}{\back} \, \underset{\phi_3}{\back} \, \underset{\phi_4}{\back} \, \cdots \, \underset{\phi_n}{\back} \, \underset{\phi_1}{\back}
%~\bigg|
%~~\ra~~
%\underset{\psi_1}{\back} \, \underset{\psi_2}{\back} \, \underset{\psi_3}{\back} \, \cdots \, \underset{\psi_n}{\back}
%~~
%\underset{\chi_1}{\back} \, \underset{\chi_2}{\back} \, \underset{\chi_3}{\back} \, \cdots \, \underset{\chi_n}{\back}
%\]
%
%\item 入力カード列の各カードの間に以下のように二枚ずつ補助カードを挿入する。
%\[
%\overset{1}{\back} \,
%\overset{2}{\back} \,
%\overset{3}{\back} \,
%\cdots \,
%\overset{n}{\back}
%~~\ra~~
%\overset{1}{\back} \,
%\underset{\psi_1}{\back} \, \underset{\chi_1}{\back} ~
%\overset{2}{\back} \,
%\underset{\psi_2}{\back} \, \underset{\chi_2}{\back} ~
%\overset{3}{\back} \,
%\underset{\psi_3}{\back} \, \underset{\chi_3}{\back} ~
%\cdots ~
%\overset{n}{\back} \,
%\underset{\psi_n}{\back} \, \underset{\chi_n}{\back}
%\]
%カードの上に書いてある数字は、入力カード列の何番目かを表している。
%
%\item 三枚組を束として、$n$個の束に対してパイルスクランブルシャッフルを適用する。
%\[
%\begin{tabular}{|c|c|c|c|c|}
%$\back\, \back\, \back$ &
%$\back\, \back\, \back$ &
%$\back\, \back\, \back$ &
%$\cdots$ &
%$\back\, \back\, \back$
%\end{tabular}
%\]
%
%\item 各束の二枚目と三枚目を全てめくる。
%%以下のルールにしたがって束をソートする。
%以下のようなカード列が得られる。
%\[
%\back\,\crd{$a_1$}\,\crd{$b_1$}~~\back\,\crd{$a_2$}\,\crd{$b_2$}~~\back\,\crd{$a_3$}\,\crd{$b_3$}~~\cdots~~\back\,\crd{$a_n$}\,\crd{$b_n$}
%\]
%
%\item 以下のように束を並び替える。
%\[
%\back\,\crd{$c_1$}\,\crd{$d_1$}~~\back\,\crd{$c_2$}\,\crd{$d_2$}~~\back\,\crd{$c_3$}\,\crd{$d_3$}~~\cdots~~\back\,\crd{$c_n$}\,\crd{$d_n$}
%\]
%ただし$c_i, d_i$は、$(c_1, d_1) = (a_1, b_1)$、$d_i = c_{i+1}$($1 \leq i \leq n-1$)、および$d_n = c_1$という条件を満たすものとする。
%
%\item 全ての補助カードを取り除く。残ったカード列が出力カード列である。
%\end{enumerate}
%
%%$3$回のPSSと$2n$枚の補助カードを用いて$D_n$シャッフルを実装する。
%%以下の実行例は$n=4$のときである。
%%
%%\begin{enumerate}
%%
%%\item 補助カード$\crd{1} \, \crd{1}$と$\crd{2} \, \crd{2}$と$\crd{3} \, \crd{3}$と$\crd{4} \, \crd{4}$にパイルスクランブルシャッフルを適用し、$\crd{$\alpha$} \, \crd{$\alpha$}$と$\crd{$\beta$} \, \crd{$\beta$}$と$\crd{$\gamma$} \, \crd{$\gamma$}$と$\crd{$\delta$} \, \crd{$\delta$}$を得る。
%%\[
%%\bigg|~
%%\underset{1}{\back} \, \underset{1}{\back}
%%~\bigg|~
%%\underset{2}{\back} \, \underset{2}{\back}
%%~\bigg|~
%%\underset{3}{\back} \, \underset{3}{\back}
%%~\bigg|~
%%\underset{4}{\back} \, \underset{4}{\back}
%%~\bigg|
%%~~\ra~~
%%\underset{\alpha}{\back} \, \underset{\alpha}{\back}~
%%\underset{\beta}{\back} \, \underset{\beta}{\back}~
%%\underset{\gamma}{\back} \, \underset{\gamma}{\back}~
%%\underset{\delta}{\back} \, \underset{\delta}{\back}
%%\]
%%ここでパイルスクランブルシャッフルの性質より、$(\alpha, \beta, \gamma, \delta)$は$(1, 2, 3, 4)$の$4!$通りの並べ替えから一様ランダムに選ばれる。
%%
%%\item $\crd{$\alpha$} \,\crd{$\beta$} \, \crd{$\gamma$} \,\crd{$\delta$}$と$\crd{$\beta$} \, \crd{$\gamma$} \,\crd{$\delta$} \, \crd{$\alpha$}$に対してパイルスクランブルシャッフル(すなわちランダム二等分割カット)を適用し、$\crd{$\chi_0$} \, \crd{$\chi_1$} \,\crd{$\chi_2$} \, \crd{$\chi_3$}$と$\crd{$\psi_0$}\,\crd{$\psi_1$} \,\crd{$\psi_2$} \, \crd{$\psi_3$}$を得る。
%%\[
%%\bigg|~
%%\underset{\alpha}{\back}\, \underset{\beta}{\back} \, \underset{\gamma}{\back} \, \underset{\delta}{\back}
%%~\bigg|~
%%\underset{\beta}{\back} \, \underset{\gamma}{\back} \, \underset{\delta}{\back} \, \underset{\alpha}{\back}
%%~\bigg|
%%~~\ra~~
%%\underset{\chi_0}{\back} \, \underset{\chi_1}{\back}\, \underset{\chi_2}{\back} \, \underset{\chi_3}{\back}~~
%%\underset{\psi_0}{\back} \, \underset{\psi_1}{\back}\, \underset{\psi_2}{\back} \, \underset{\psi_3}{\back}
%%\]
%%ここで、$\chi_i = \psi_{i-1}$が全ての$i$で成り立つか、$\chi_i = \psi_{i+1}$が全ての$i$で成り立つか、どちらかが$1/2$の確率で生じている(添字は$\bmod 4$の計算)。
%%
%%\item 入力カード$\back \, \back \, \back \, \back$に補助カードを以下のように挿入する。
%%\[
%%\underset{\chi_0}{\back} \, \underset{\psi_0}{\back}  \, \back~
%%\underset{\chi_1}{\back} \, \underset{\psi_1}{\back}  \, \back~
%%\underset{\chi_2}{\back} \, \underset{\psi_2}{\back}  \, \back~
%%\underset{\chi_3}{\back} \, \underset{\psi_3}{\back}  \, \back~
%%\]
%%
%%\item 各三枚組を束として、パイルスクランブルシャッフルを適用する。
%%\[
%%\begin{tabular}{|c|c|c|c|}
%%$\back\, \back\, \back$ &
%%$\back\, \back\, \back$ &
%%$\back\, \back\, \back$ &
%%$\back\, \back\, \back$
%%\end{tabular}
%%\]
%%
%%\item 各束の一枚目と二枚目を全てめくる。例えば、以下のようなカード列が得られる。
%%\[
%%\crd{2}\,\crd{4}\,\back~~\crd{1}\,\crd{2}\,\back~~\crd{4}\,\crd{3}\,\back~~\crd{3}\,\crd{1}\,\back
%%\]
%%一枚目のカードを現在の番号、二枚目のカードを次の番号とみなすと、上の例の場合$2 \ra 4 \ra 3 \ra 1 \ra 2$というサイクルが表れている。(ステップ2の置き方から必ずサイクルになることに注意する。)サイクルの順番通りに各束から三枚目のカードを並べるたものが出力カード列である。
%%\end{enumerate}
%
%%\subsection{完全$n$部グラフ}
%%
%%\subsection{正則グラフ}
%%
%%\subsection{木}



\bibliographystyle{abbrv}
\bibliography{card}

\section*{Declarations}

\subsection*{Funding}
K. Miyamoto was partly supported by JSPS KAKENHI 20K14302. 
K. Shinagawa was partly supported by JSPS KAKENHI 21K17702. 
%\subsection*{Conflicts of interest/Competing interests}
%The second author has some co-authored papers with Takaaki Mizuki, Goichiro Hanaoka, Mitsugu Iwamoto, and Pascal Lafourcade within three years. The second author is a member of a research group organized by Goichiro Hanaoka. The second author had been visited Mitsugu Iwamoto's research group in UEC as a postdoc researcher last year. 
%\subsection*{Availability of data and material}
%Not applicable
%\subsection*{Code availability}
%Not applicable
%\subsection*{Authors' contributions}

\end{document}
