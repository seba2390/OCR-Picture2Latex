

\section{Preliminaries}

In this section, we collect some fundamentals in card-based cryptography; see \cite{MizukiIJISEC14} for example.


\subsection{Cards}\label{ss:card}
Throughout \md{this paper}, we deal with physical \textit{cards} with the symbol ``?"  on the backs. 
\md{We use two collections of cards: \textit{black-cards} $\crd{1}\,\crd{2}\,\crd{3}\,\cdots$ and \textit{red-cards} $\crd{$\stext{1}$}\,\crd{$\stext{2}$}\,\crd{$\stext{3}$}\,\cdots$ as follows: }
%A card with a natural number $\stext{i}$ (resp. $i$) on the front is called \textit{a red-card} (resp. \textit{a black-card}). 
\[ \begin{array}{ll}
\text{front:} & \overbrace{\crd{1}~\crd{2}~\crd{3}~\crd{4}~\crd{5}~\crd{6}~\cdots}^{\text{black-cards}}~\overbrace{\crd{$\stext{1}$}~\crd{$\stext{2}$}~\crd{$\stext{3}$}~\crd{$\stext{4}$}~\crd{$\stext{5}$}~\crd{$\stext{6}$}~\cdots}^{\text{red-cards}}\, \\
\text{back:} & \back~\back~\back~\back~\back~\back~\cdots~\back~\back~\back~\back~\back~\back~\cdots.
\end{array}
\]
We distinguish between the natural number $\stext{i}$ (written in red) and the natural number $i$ (written in black). 
We denote by $\mathbb{N}^{\f{red}}$ the set of all natural numbers written in red, i.e., $\mathbb{N}^{\f{red}} = \{\stext{1}, \stext{2}, \stext{3}, \ldots\}$.
The set $\mathbb{N}^{\f{red}}$ is a totally ordered set by using the natural order on $\mathbb{N}$.
We define a totally order $\preccurlyeq$ on $\mathbb{N}\cup\mathbb{N}^{\f{red}}$ by
$\alpha \preccurlyeq \beta$ if and only if 
\begin{itemize}
\item $x, y \in \mathbb{N}$ and $x\leq y$, where $\alpha = x$ and $\beta = y$, 
\item $\stext{x}, \stext{y}\in\mathbb{N}^{\f{red}}$ and $x\leq y$, where $\alpha = \stext{x}$ and $\beta = \stext{y}$, or
\item $\stext{x}\in \mathbb{N}^{\f{red}}$ and $y\in\mathbb{N}$, where $\alpha = \stext{x}$ and $\beta = y$.
\end{itemize}
\md{\textit{A deck} $D$} is a non-empty multiset such that $\{?\}\cap D=\varnothing$. 
Let $D$ be a deck.  
An expression $\dfrac{x}{?}$ $\left(\text{resp. } \dfrac{?}{x}\right)$ with $x\in D$ is said to be \textit{a face-up card} (resp.  \textit{a face-down card}) of $D$.
\textit{A lying card} $y$ of $D$ is the face-up card $y=\dfrac{x}{?}$ of $D$ or the face-down card $y=\dfrac{?}{x}$ of $D$, and in this case, we set $\f{atom}(y)=x$.
\textit{A card-sequence} from $D$ is a list of lying cards of $D$, say $(x_1,\ldots,x_n)$, such that $\{\f{atom}(x_i)\mid i=1,2,\ldots,n\}=D$ as multisets.
For a card-sequence $\f{x}$, we write $\f{x}_i$ for the $i$-th term. 
A face-up card $\dfrac{x}{?}$ is represented by \crd{$x$}~, and a face-down card $\dfrac{?}{x}$ is represented by $\back$~.
Given a card $x$ with the expression $\dfrac{y}{z}$, we write $\f{front}(x)=y$, $\f{back}(x)=z$, and $\f{swap}(x)=\dfrac{z}{y}$.
For a card-sequence $\f{x}=(\f{x}_1,\ldots,\f{x}_n)$ and a subset $T\md{\subseteq} \{1,2,\ldots, n\}$, we define an operator $\turn_T(-)$ by
\[ 
\f{turn}_T(\f{x})=(y_1,\ldots, y_n), \quad y_i=
\left\{\begin{array}{ll}
\f{swap}(\f{x}_i) & \text{if $i\in T$,}\\
\f{x}_i  &\text{if $i\notin T$.}
\end{array}\right.
\]
The card-sequence $\f{front}(\f{x}) = (\f{front}(\f{x}_1),\ldots, \f{front}(\f{x}_n))$  is called \textit{the visible sequence} of $\f{x}$.
Let $(\mathcal{T}, \mathcal{G})$ be a pair of a collection of subsets of $\{1,2,\ldots, n\}$ (i.e., $\mathcal{T} \md{\subseteq} 2^{\{1,2,\ldots, n\}}$) and a probability distribution on $\mathcal{T}$.
Now, we also define an operation $\f{rflip}_{(\mathcal{T},\mathcal{G})}(-)$ associated with the pair $(\mathcal{T}, \mathcal{G})$ by
\[ \f{rflip}_{(\mathcal{T},\mathcal{G})}(\f{x})= \f{turn}_T(\f{x}), \]
where $T$ is chosen from $\mathcal{T}$ depending on the probability distribution $\mathcal{G}$.
\md{Note that if $\mathcal{T}=\{T\}$ with a subset $T\md{\subseteq} \{1,2,\ldots, n\}$,} then $ \f{rflip}_{(\mathcal{T},\mathcal{G})}(-)=\f{turn}_T(-)$.




\subsection{Shuffles}\label{ss:shuffle}
For a natural number $n\in\mathbb{N}$, we denote by  $\mathfrak{S}_n$ the symmetric group of degree $n$, that is, the group whose elements are all bijective maps from $\{1,2,\ldots, n\}$ to itself, and whose group multiplication is the composition of functions. An element of the symmetric group is called \textit{a permutation}.

Given a card-sequence  $\mathsf{x}=(\f{x}_1,\ldots,\f{x}_n)$ and $\sigma\in\mathfrak{S}_n$, we have a card-sequence $\sigma(\f{x})$ in the natural way:
\[ \sigma(\mathsf{x})=(\f{x}_{\sigma^{-1}(1)},\ldots, \f{x}_{\sigma^{-1}(n)}).\]

Now, we recall an operation on a card-sequence which is called a ``shuffle".
Roughly speaking, a shuffle is a probabilistic reordering operation on a card-sequence. 
Let $(\Pi,\F)$ be a pair of a subset of $\mathfrak{S}_n$ and a probability distribution on $\Pi$. 
For a card-sequence $\f{x}=(\f{x}_1,\ldots,\f{x}_n)$, an operation $\shuffle_{(\Pi,\F)}(-)$ associated with the pair $(\Pi,\F)$ is defined by
\[ \shuffle_{(\Pi,\F)}(\f{x})= \sigma(\f{x}).  \]
\md{Here, $\sigma$ is chosen according to the probability distribution $\mathcal{F}$ on $\Pi$.}
Note that when we apply a shuffle to a card-sequence, no one knows which permutation was actually chosen.
We also note that if $\Pi=\{\sigma\}$ for some $\sigma\in\mathfrak{S}_{n}$, then $ \f{shuffle}_{(\Pi,\mathcal{F})}(-)=\sigma(-)$. 

\begin{definition}\label{def:ucshuffle}
A shuffle $\shuffle_{(\Pi,\F)}$ is said to be \textit{uniform closed} if $\Pi$ is closed under the multiplication of the symmetric group, and $\F$ is the uniform distribution on $\Pi$. 
\end{definition}

All shuffles dealt with \md{this paper} are uniform closed shuffles. 

\begin{example}
\begin{enumerate}
\item[(1)] \md{
For a sequence of $\ell$ cards, suppose that a subsequence of the sequence is divided into $n$ piles of $m$ cards. (It holds $\ell \geq nm$.) 
\textit{A pile-scramble shuffle} (PSS for short) is a uniform closed shuffle that completely randomly permutes $n$ piles.}
%Let $\ell, n, m, d$ be positive integers such that $\ell \geq nm + d - 1$. 
%For a sequence of $\ell$ cards, \textit{a pile-scramble shuffle} for $n$ piles of $m$ cards starting from the $d$-th card is a uniform closed shuffle defined as follows: 
%Suppose that $\ell$ cards are divided into $n$ piles of $m$ cards where the $i$-th pile ($1 \leq i \leq n$) consists of the $j$-th cards for $(i-1)m + 1 \leq j \leq im$. 
%A uniform closed shuffle $\shuffle_{(\Pi,\F)}$ for $\ell$ cards is called \textit{a pile-scramble shuffle} (PSS for short) if there exists a natural number $n \leq \ell$ such that $\Pi$ is isomorphic to $\mathfrak{S}_n$. 
The following shuffle is an example of a PSS:
\[\mathsf{PSS}_{(3,2)}:
\left(~\crd{1}~\crd{2}~,~\crd{3}~\crd{4}~,~\crd{5}~\crd{6}~ \right) \overset{\sigma}{\longmapsto}
\begin{cases}
\left(~\crd{1}~\crd{2}~,~\crd{3}~\crd{4}~,~\crd{5}~\crd{6}~ \right)& \text{if $\sigma=\mathsf{id}$,}\\
\left(~\crd{1}~\crd{2}~,~\crd{5}~\crd{6}~,~\crd{3}~\crd{4}~ \right)& \text{if $\sigma=\mathsf{(2~3)}$,}\\
\left(~\crd{3}~\crd{4}~,~\crd{5}~\crd{6}~,~\crd{1}~\crd{2}~ \right)&\text{if $\sigma=\mathsf{(1~3~2)}$,}\\
\left(~\crd{3}~\crd{4}~,~\crd{1}~\crd{2}~,~\crd{5}~\crd{6}~ \right)& \text{if $\sigma=\mathsf{(1~2)}$,}\\
\left(~\crd{5}~\crd{6}~,~\crd{1}~\crd{2}~,~\crd{3}~\crd{4}~ \right) &\text{if $\sigma=\mathsf{(1~2~3)}$,}\\
\left(~\crd{5}~\crd{6}~,~\crd{3}~\crd{4}~,~\crd{1}~\crd{2}~ \right) & \text{if $\sigma=\mathsf{(1~3)}$.}\\
\end{cases}
\]
%In the above example, since this rearranges $3$ piles with $2$ cards, one can take $\Pi \subset \mathfrak{S}_6$ which is isomorphic to $\mathfrak{S}_3$. 
% and the rearrangement permutation is chosen from $\mathfrak{S}_3$. 
%Although it is a rearrangement of $6$ cards, we regard $\mathfrak{S}_3$ to ; for example, a permutation $\sigma = (2~3) \in \mathfrak{S}_3$ denotes the second pile, and the third pile is swapped. 
We use $\mathsf{PSS}_{(n,m)}$ to denote a PSS for $n$ piles each having $m$ cards. 
\md{We remark that $\mathsf{PSS}_{(n,m)}$ can be easily implemented by putting each pile into each physical envelope
 and then permute them.}
%Since it consists of $3$ piles each having $2$ cards, this pile-scramble shuffle is denoted by $\mathsf{PSS}_{(3,2)}$. 
%each $\sigma\in \mathfrak{S}_3$ is chosen with probability $\dfrac{1}{6}$.
%In this study, PSS plays an important role.

\item[(2)] Let $\pi_k\in\mathfrak{S}_n$ be the permutation
\[\pi_k=\begin{pmatrix}
1 & 2 & \cdots &  k & k+1 & \cdots & n \\
n-k+1 & n-k+2 & \cdots & n & 1 & \cdots & n-k
\end{pmatrix},\]
and set $\Pi=\{\pi_k\mid k=1,2,\ldots,n\}$. \md{This} uniform closed shuffle $\shuffle_{(\Pi,\F)}$ is called \textit{a \md{random} cut} (RC for short). 
\end{enumerate}
\end{example}



\subsection{Procotols}\label{ss:protocol}
Mizuki and Shizuya \cite{MizukiIJISEC14} define the formal definition of a card-based protocol via an abstract machine.
In this section, we recall the definition of a card-based protocol and introduce a shuffle protocol, which is a particular card-based protocol realizing a shuffle. 

\subsubsection{Card-based protocols}\label{sss:reduction}

To put it briefly, a ``protocol" is a Turing machine that chooses one of the following operations to be applied to a card-sequence $\f{x}$: turning $(\f{x}\mapsto \f{rflip}_{(\mathcal{T},\mathcal{G})}(\f{x}))$ or shuffling $(\f{x}\mapsto \shuffle_{(\Pi,\F)}(\f{x}))$.

For a deck $D$, the set of all card-sequences from $D$ will be denoted by $\f{Seq}^D$.
Then \textit{the visible sequence set} $\f{Vis}^D$ is defined as the set of all sequences $\f{front(x)}$ for $\f{x}\in \f{Seq}^D$.
We also define the sets of the actions:
\begin{align*}
& \f{turn}^n=\{\f{turn}_T(-)\mid T\md{\subseteq} \{1,2,\ldots, n\}\}, \\
& \f{perm}^n=\{\sigma(-)\mid \sigma\in\mathfrak{S}_n\}, \\
& \f{SP}^n=\{\shuffle_{(\Pi,\mathcal{F})}(-)\mid \text{$\mathcal{F}$ is  a probability distribution on $\Pi\in 2^{\mathfrak{S}_n}\}, \text{ and}$}\\
& \f{TP}^n=\{\f{rflip}_{(\mathcal{T},\mathcal{G})}(-)\mid \text{$\mathcal{G}$ is a probability distribution on $\mathcal{T}\md{\subseteq} 2^{\{1,2,\ldots, n\}}\}$}.
\end{align*}
%Here, we use the symbol $\f{P}(S)$ to denote the power set of a set $S$. 

A protocol is a Markov chain, that  is, a stochastic model describing a sequence of possible actions in which the probability of each action depends only on the state attained in the previous event.
Let $Q$ be a finite set with two distinguished states, which are called \textit{an initial state} $q_0$ and \textit{a final state} $q_{\rm f}$.

\begin{definition}
\textit{A card-based protocol} is a quadruple $\mathcal{P}=(D,U,Q,\f{A})$, where $U\md{\subseteq} \f{Seq}^D$ is an input set and $\f{A}$ is a partial action function
\[ 
\begin{array}{cccc}
 \f{A}:  & (Q\setminus\{q_{\rm f}\})\times \f{Vis}^D &  \longrightarrow &  Q\times (\f{turn}^n\cup\f{perm}^n\cup \f{SP}^n\cup \f{TP}^n), \\
           & (q,\f{y})                                                    & \longmapsto & (q',\f{act}_{q,\f{y}})
 \end{array} \]
which depends only on the current state and visible sequence, specifying the next state and an operation on the card-sequence from $(\f{turn}^n\cup\f{perm}^n\cup \f{SP}^n\cup \f{TP}^n)$, such that $\f{A}(q_0,\f{front(x)})$ is defined if $\f{x}\in U$.
For a state $q\in Q\setminus\{q_{\rm f}\}$ and a visible sequence $\f{y}=\f{front(x)}\in \f{Vis}^D$ such that $\f{A}(q,\f{y})=(q',\f{act}_{q,\f{y}})$, we obtain the next state $(q' ,\f{front}(\act_{q,\f{y}}(\f{x})))$.
\md{By the above process, if we have $(q_{\rm f},\mathsf{X}) \in Q\times\f{Vis}^D$ for some $\mathsf{X}\in \f{Vis}^D$,}
the protocol $\mathcal{P}$ terminates. 
\end{definition}

Let $\mathcal{P}=(D,U,Q,\f{A})$ be a card-based protocol.
%\md{Recall that $Q$ is a finite set of states.}
For an execution of $\mathcal{P}$ with an input card-sequence $\f{x}^{(0)}\in U$, we obtain a sequence of results of actions as follows:
\[ (q_0,\f{x}^{(0)})\longmapsto (q_1,\f{x}^{(1)})\longmapsto (q_2,\f{x}^{(2)})\longmapsto(\md{q_3},\f{x}^{(3)})\longmapsto \cdots,\]
where $\f{x}^{(i)}=\act_{q_{i-1},\f{front}(\f{x}^{(i-1)})}(\md{\f{x}^{(i-1)}})$ for $i\geq 1$. \md{Here, $q_i$ ($i = 0, 1, 2, \ldots$) are not necessarily distinct}. 
If \md{the action function value} $\f{A}(q_i,\f{x}^{(i)})$ is undefined for some $i \in \N$, we say that ``$\mathcal{P}$ aborts at Step $i$ in the execution". 
Note that even for the same input card-sequence $\f{x}^{(0)}$, the obtained chains may be different for each execution.
If the protocol $\mathcal{P}$ terminates for an input card-sequence $\f{x}^{(0)}$, then we have a chain of results as follows:
\[ (q_0,\f{x}^{(0)})\longmapsto (q_1,\f{x}^{(1)})\longmapsto (q_2,\f{x}^{(2)})\longmapsto(q_2,\f{x}^{(3)})\longmapsto \cdots \longmapsto (q_{\rm f},\f{x}^{(\ell)}).\]
In this case, $\f{x}^{(0)}$ is called \textit{an initial sequence}, $\f{x}^{(\ell)}$ is called \textit{a final sequence}, and the sequence 
\[ (\f{y}^{(0)},\f{y}^{(1)},\ldots, \f{y}^{(\ell)}), \]
where $\f{y}^{(i)}=\f{front}(\f{x}^{(i)})$, is called \textit{a visible sequence-trace} of $\mathcal{P}$. 
We denote by $\f{Fin}(\mathcal{P})$ the set of  all \md{final} sequences, which is obtained by $\mathcal{P}$.

%For a protocol $\mathcal{P}=(D,U,Q,\f{A})$, let $\f{x}$ be a probability distribution on the set of input sequences $U$. 
%Let $\f{v}$ be a random variable for the visible sequence-trace of $\mathcal{P}$.
%Then $\mathcal{P}$ is \textit{secure} if $\f{x}$ and $\f{v}$ are stochastically independent for any $\f{x}$ and $\f{v}$.

\begin{example}
Let us consider the following. Take the deck $D=\{1,2,3,4\}$, and hence use as follows:
\[ \begin{array}{ccccc}
\text{front:} & \crd{1}~& \crd{2}~ & \crd{3}~ & \crd{4}~\;\\
\text{back:} & \back~ & \back~   &  \back~  & \back~.
\end{array}
\]
Now, we give a card-based protocol $\mathcal{P}=\left(D, \left\{\left(\dfrac{1}{?},\dfrac{2}{?},\dfrac{3}{?},\dfrac{4}{?}\right)\right\}, \{q_0,q_1,q_2,q_{\rm f}\}, \f{A}\right)$ such that
\begin{align*}
&\f{A}\left(q_0,1234\right) = (q_1, \f{turn}_{\{1,2,3,4\}}(-)), \\
&\f{A}\left(q_1,????\right) = (q_2,(1\;3)(-)), \\
&\f{A}\left(q_2,????\right) = (q_{\rm f}, \turn_{\{3\}}(-)).
\end{align*}
In this case, the card-sequence $\crd{1}~\crd{2}~\crd{3}~\crd{4}$ is changed by the protocol $\mathcal{P}$ as follows:
\[ \crd{1}~\crd{2}~\crd{3}~\crd{4}~ \longrightarrow \overset{1}{\back}~\overset{2}{\back}~\overset{3}{\back}~\overset{4}{\back}~ \longrightarrow\overset{3}{\back}~\overset{2}{\back}~\overset{1}{\back}~\overset{4}{\back}~ \longrightarrow \overset{3}{\back}~\overset{2}{\back}~\crd{1}~\overset{4}{\back}~.\]
Thus, the final sequence is $ \back~\back~\crd{1}~\back~$.
\end{example}

%\md{プロトコルの正当性とは, プロトコルが必ず正しい計算を行う性質である. (後に回す)}

\subsubsection{Shuffle protocols}\label{sss:reduction2}

A shuffle protocol\footnote{\md{Koch and Walzer \cite{KochFUN21} considered a similar notion and proposed a protocol for any uniform closed shuffles. The main difference of their model and our model is that their model allows a randomness generation in the head (see Section \ref{relatedworks} in Introduction). }} is a card-based protocol realizing a shuffle operation. 
%We newly introduce the concept of a shuffle protocol, which brings the same effect as a shuffle operation. 
It takes a card-sequence $\f{x} = (\f{x}_1, \f{x}_2, \ldots, \f{x}_n)$ such that $\f{back}(\f{x}) = (?, ?, \ldots, ?)$ as input and outputs $\f{y} = (\f{y}_1, \f{y}_2, \ldots, \f{y}_n)$ such that $\f{y} = \sigma(\f{x})$ for a permutation $\sigma$ is chosen from $\mathfrak{S}_n$ depending on some probability distribution:
\[
\underbrace{\back \, \back \, \cdots \, \back}_{\f{x}} ~ \underbrace{\crd{$h_1$} \, \crd{$h_2$} \, \cdots \, \crd{$h_k$}}_{\f{h}} ~ 
\lra~
\underbrace{\back \, \back \, \cdots \, \back}_{\f{y}} ~ \underbrace{\crd{$h_1$} \, \crd{$h_2$} \, \cdots \, \crd{$h_k$}}_{\f{h}}~,
\]
where $\f{h}$ is a card-sequence of helping cards. 
%\md{Here, the asterisk symbol ``$*$" denotes some natural number in $\mathbb{N}\cup\mathbb{N}^{\f{red}}$ and $\crd{$*$}$ means some face-up card.} 
\md{Informally speaking, the correctness requires $\mathsf{y}=\mathsf{shuffle}_{(\Pi,\mathcal{F})}(\mathsf{x})$ and the security requires that no one learns nothing about the chosen permutation $\sigma \in \Pi$.}

\begin{definition}
Let $D_{\f{inp}}, D_{\f{help}}$ be decks, $U_{\f{inp}}$ an input set from $D_{\f{inp}}$, and $\f{h} \in \f{Seq}^{D_{\f{help}}}$ a card-sequence from $D_{\f{help}}$. 
We define an input set $U$ from $D = D_{\f{inp}} \cup D_{\f{help}}$ by $U = \{(\f{x}, \f{h}) \mid \f{x} \in U_{\f{inp}}\}$. 
A card-based protocol $\mathcal{P}=(D, U,Q,\f{A})$ is said to be a \textit{shuffle protocol} if the following conditions are satisfied:
\begin{enumerate}
\item[(a)] $\mathcal{P}$ always terminates within a fixed number of steps, i.e., it is a finite-runtime protocol;
\item[(b)] for any input sequence $(\f{x}, \f{h}) \in U$ \md{and for any final sequence $\f{y} \in \f{Fin}(\mathcal{P})$ of the form $\f{y} = (\f{x'}, \f{h})$, there exists a permutation $\sigma \in \mathfrak{S}_{|D_{\f{inp}}|}$ such that $\f{x'} = \sigma(\f{x})$;}
\item[(c)] for any input sequence $(\f{x}, \f{h})\in U$, \md{any card contained in $\f{x}$ has not been turned} at any step of a protocol execution.
\end{enumerate}
\md{%Let $\shuffle_{(\Pi,\F)}$ be a shuffle. 
We say that $\mathcal{P}$ \md{implements} a shuffle $\shuffle_{(\Pi,\F)}$ if every permutation $\sigma$ in (b) belongs to $\Pi$ and it is chosen according to the distribution $\F$.}
%Let $\f{y}_{\f{x}}$ be a random variable of the first $n$ cards of the final sequence when $(\f{x}, \f{h}) \in U$ is given as an input card-sequence. 
%Let $\shuffle_{(\Pi,\F)}$ be a shuffle. 
%% and $\f{y'}_{\f{x}}$ be a random variable of the card-sequence which is the result of $\shuffle_{(\Pi,\F)}$ with $\f{x} \in U_{\f{inp}}$. 
%We say that $\mathcal{P}$ \md{implements} $\shuffle_{(\Pi,\F)}$ if for any $\f{x} \in U_{\f{inp}}$, $\f{y}_{\f{x}}$ and $\shuffle_{(\Pi,\F)}(\f{x})$ are the same. 
%Let $\shuffle_{(\Pi,\F)}$ be a shuffle and $\mathcal{F}_{\f{x}}$ be a probability distribution of the resultant card-sequence when $\shuffle_{(\Pi,\F)}$ is applied to a card-sequence $\f{x} \in U$. 
%Let $\mathcal{F'}_{\f{x}}$ be a probability distribution of the final sequence of $\mathcal{P}$ when $\f{x} \in U$ is an initial sequence. 
%We say that $\mathcal{P}$ realizes $\shuffle_{(\Pi,\F)}$ if for any input sequence $\f{x} \in U$, $\mathcal{F}_{\f{x}}$ and $\mathcal{F'}_{\f{x}}$ are the same. 
We say that $\mathcal{P}$ is secure if for any $\f{x} \in U_{\f{inp}}$, \md{a random variable of $\sigma$} is stochastically independent of the random variable of the visible sequence-trace of $\mathcal{P}$. 
% when $(\f{x}, \f{h}) \in U$ is an initial card-sequence. 
\end{definition}
