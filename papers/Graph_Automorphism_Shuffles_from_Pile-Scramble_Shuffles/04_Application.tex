\section{Efficiency improvements for graph shuffles for cycles}\label{s:app}

In this section, we \md{implement} efficient graph shuffle protocols for some \md{specific} graph classes. 
%\md{Since cyclic graphs have the same degree, we can remove the red cards. In particular, for an undirected $n$-cyclic graph, even though it has $2n$ edges, the number of cards corresponding to edges is reduced to $n$ pairs of cards using the symmetry of the graph.}
In particular, we improve the number of cards in our protocol.  

\subsection{The $n$-cycle graph} \label{4-1}
First, we consider the $n$-cycle graph $\overset{\ra}{C_n}$:
\[ \overset{\ra}{C_n}= \begin{xy}
                     (0,-2)*[o]+{1}="1",(15,-2)*[o]+{2}="2",(30,-2)*[o]+{\cdots}="3",(45,-2)*[o]+{n-1}="4",(60,-2)*[o]+{n.}="n",
                     \ar "1";"2"^{}
                     \ar "2";"3"^{}
                     \ar "3";"4"_{}
                     \ar "4";"n"_{}
                     \ar @(lu,ur)"n";"1"_{}
            \end{xy}\]
The graph shuffle for  $\overset{\ra}{C_n}$ is equivalent to a RC of $n$ cards since the automorphism group $\aut(\overset{\ra}{C_n})$ is isomorphic to the cyclic group of degree $n$.
If we apply our graph shuffle protocol for $\overset{\ra}{C_n}$ proposed in Section 3, we need $4n$ cards.
In this subsection, we propose a graph shuffle protocol for $\overset{\ra}{C_n}$ with $3n$ cards only. 

\md{Before describing the improved protocol, we shortly mention how to improve the number of cards. The idea\footnote{We remark that this idea works for every graphs such that all vertices have the same degree.} is to remove the red cards by making a pile of $(\f{x}_i, \alpha_i, \alpha_{i+1})$ instead of a pile of $(\f{x}_i, \stext{\alpha_i})$ and a pile of $(\alpha_i, \alpha_{i+1})$ in the previous protocol. Since all vertices of $\overset{\ra}{C_n}$ have the same degree, all piles of $(\f{x}_i, \alpha_i, \alpha_{i+1})$ have the same number of cards and thus the final randomization (corresponding to Step (5) in the previous protocol) can be done by a single PSS.}
%since all vertices of $\overset{\ra}{C_n}$ have the same degree, we can remove the red cards $\crd{$\stext{1}$}\, \crd{$\stext{2}$} \, \cdots \, \crd{$\stext{n}$}$ by associating each card $\f{x}_i$ with each edge. }

%We set $V_G=\{1,2,\ldots, n\}$ and $D_{\f{inp}}=\{x_1,x_2,\ldots, x_n\}$. 
%Assume\footnote{By Remark \ref{rem:shuffleprotocol}, our graph shuffle protocol works for any deck $D_{\f{inp}}$ and any input set $U_{\f{inp}}$.} that all symbols of $D_{\f{inp}}$ are distinct and $\f{front}(\f{x})=(?,?,\ldots, ?)$ for all $\f{x} \in U_{\f{inp}}$. 
%We set a card-sequence $\f{h}$ of helping cards as follows:
%\[
%\f{h} = \crd{$\stext{1}$} \, \crd{$\stext{2}$} \, \crd{$\stext{3}$} ~\cdots ~\crd{$\stext{n}$}~
%\overbrace{\crd{1} \, \cdots \, \crd{1}}^{\f{deg}(1)} ~ \overbrace{\crd{2} \, \cdots \, \crd{2}}^{\f{deg}(2)} ~ \overbrace{\crd{3} \, \cdots \, \crd{3}}^{\f{deg}(3)} ~ \cdots \cdots~ \overbrace{\crd{$n$} \, \cdots \, \crd{$n$}}^{\f{deg}(n)}~.
%\]
%Thus the deck of helping cards is $D_{\f{help}} = \{\stext{1}, \stext{2}, \ldots, \stext{n}, 1^{\f{deg}(1)}, 2^{\f{deg}(2)}, \ldots, n^{\f{deg}(n)}\}$, where the superscript denotes the number of the symbol in the deck $D_{\f{help}}$. 
%The deck $D$ is the union of $D_{\f{inp}}$ and $D_{\f{help}}$ as multisets and it consists of $2n+2m$ symbols. 

Let $D_{\f{inp}}=\{x_1,x_2,\ldots, x_n\}$ be an arbitrary deck and $D_{\f{help}} = \{1, 1, 2, 2, 3, 3, \ldots, n, n\}$ a deck of the symbols of $2n$ helping cards. 
The sequence of helping cards $\f{h}$ is defined as follows:
\[ 
\f{h} = \crd{1}\,\crd{1}~\crd{2}\,\crd{2}~\crd{3}\,\crd{3}~\cdots\crd{$n$}\,\crd{$n$}~. \]
For $i=1,2,\ldots ,n$, we set $\f{pile}[i]=\left(\dfrac{?}{i}, \dfrac{?}{i}\right)$.

\begin{enumerate}
\item[(1)] Place the $3n$ cards as follows:
\[
\underbrace{\back \, \back \, \back \, \cdots \, \back}_{\f{x}} ~ 
\underbrace{\crd{1}\,\crd{1}~\crd{2}\,\crd{2}~\crd{3}\,\crd{3}~\cdots\crd{$n$}\,\crd{$n$}}_{\f{h}}~. 
\]

\item[(2)] Arrange the card-sequence as follows:
\[
\underbrace{\back \, \back \, \back \, \cdots \, \back}_{\f{x}} ~ 
\underbrace{\underset{1}{\back} \, \underset{1}{\back}}_{\f{pile}[1]}~ 
\underbrace{\underset{2}{\back} \, \underset{2}{\back}}_{\f{pile}[2]}~ 
\underbrace{\underset{3}{\back} \, \underset{3}{\back}}_{\f{pile}[3]}~ 
~\cdots~ 
\underbrace{\underset{n}{\back} \, \underset{n}{\back}}_{\f{pile}[n]}~ .
\]
%\[
%\underbrace{\back \, \back \, \cdots \, \back}_{\f{x}} ~ 
%\underbrace{\crd{1}\,\crd{1}~\crd{2}\,\crd{2}~\crd{3}\,\crd{3}~\cdots\crd{$n$}\,\crd{$n$}}_{\f{h}}~. 
%\]
Apply $\PSS_{(n,2)}$ to $(\f{pile}[1], \f{pile}[2] ,\ldots,\f{pile}[n])$ and then we obtain the card-sequence as follows:
\[
\underbrace{\back \, \back \, \back \, \cdots \, \back}_{\f{x}} ~ 
\underset{\alpha_1}{\back} \, \underset{\alpha_1}{\back}
~
\underset{\alpha_2}{\back} \, \underset{\alpha_2}{\back}
~
\underset{\alpha_3}{\back} \, \underset{\alpha_3}{\back}
~
\cdots
~
\underset{\alpha_n}{\back} \, \underset{\alpha_n}{\back}~,
\]
where $\{\alpha_1,\alpha_2,\ldots,\alpha_n\}=\{1,2,\ldots, n\}$.

\item[(3)] Arrange the card-sequence as follows:
\[ 
\underset{\f{x}_1}{\back} \,
\underset{\alpha_1}{\back} \, \underset{\alpha_2}{\back} ~
\underset{\f{x}_2}{\back} \,
\underset{\alpha_2}{\back} \, \underset{\alpha_3}{\back} ~
\underset{\f{x}_3}{\back} \,
\underset{\alpha_3}{\back} \, \underset{\alpha_4}{\back} ~
\cdots ~
\underset{\f{x}_n}{\back} \,
\underset{\alpha_n}{\back} \, \underset{\alpha_1}{\back}\,.
\]

\item[(4)] Apply $\PSS_{(n,3)}$ to the card-sequence as follows:
\[
\begin{tabular}{|c|c|c|c|c|}
$\back\, \back\, \back$ &
$\back\, \back\, \back$ &
$\back\, \back\, \back$ &
$\cdots$ &
$\back\, \back\, \back$
\end{tabular}\,.
\]

\item[(5)] For all piles, turn over the second and third cards. Let $a_i, b_i \in \{1,2,\ldots, n\}$ be the opened symbols of the second and third cards, respectively, in the $i$-th pile as follows:
%Then we suppose that the following card-sequence is obtained.
\[
\back\,\crd{$a_1$}\,\crd{$b_1$}~~\back\,\crd{$a_2$}\,\crd{$b_2$}~~\back\,\crd{$a_3$}\,\crd{$b_3$}~~\cdots~~\back\,\crd{$a_n$}\,\crd{$b_n$}~\md{.}
\]

\item[(6)] Arrange $n$ piles so that $(c_1, d_1) = (a_1, b_1)$, $d_i = c_{i+1}$, $(1 \leq i \leq n-1)$, and $d_n = c_1$ as follows:
\[
\underset{\f{y}_1}{\back} \,\crd{$c_1$}\,\crd{$d_1$}~~\underset{\f{y}_2}{\back} \,\crd{$c_2$}\,\crd{$d_2$}~~\underset{\f{y}_3}{\back} \,\crd{$c_3$}\,\crd{$d_3$}~~\cdots~~\underset{\f{y}_n}{\back} \,\crd{$c_n$}\,\crd{$d_n$}~.
\]
After that, we arrange the card-sequence as follows:
\[
\underset{\f{y}_1}{\back} \, \underset{\f{y}_2}{\back} \, \underset{\f{y}_3}{\back} \, \cdots \, \underset{\f{y}_n}{\back} ~ 
\underbrace{\crd{1}\,\crd{1}~\crd{2}\,\crd{2}~\crd{3}\,\crd{3}~\cdots\crd{$n$}\,\crd{$n$}}_{\f{h}}~. 
\]
Then the output card-sequence is $\mathsf{y}=(\f{y}_1,\f{y}_2,\ldots, \f{y}_n)$.
\end{enumerate}


\md{We show the correctness of the protocol. Let $\mathsf{x}=(\mathsf{x}_1,\ldots, \mathsf{x}_n)$ be an input sequence. Assume that the protocol outputs the sequence $\mathsf{y}=(\mathsf{y}_1,\ldots, \mathsf{y}_n)$ when $\mathsf{x}$ is given as input.
First, we see that $\mathsf{y}=\sigma (\mathsf{x})$ for some $\sigma$ in the cyclic group of degree $n$. 
For $i=1,\ldots, n$, we set $P_i=(\mathsf{x}_i,\alpha_i,\alpha_{i+1})$, where $\alpha_{n+1}=\alpha_1$, in Step (3) and put $P=(P_1,P_2,\ldots, P_n)$. 
Let $Q=(Q_1,\ldots, Q_n)=(P_{\sigma^{-1}(1)},\ldots, P_{\sigma^{-1}(n)})$ for some $\sigma\in\mathfrak{S}_n$. Then, $Q$ is obtained by $\sigma$ in the cyclic group of degree $n$ if and only if the third entry of $Q_i$ and the second entry of $Q_{i+1}$ are same for any $1\leq i\leq n-1$. It follows that the components of obtained sequence in Step (6) are sorted in a cyclic fashion of $P$. Therefore, $\mathsf{y}$ is equal to $\sigma (\mathsf{x})$ for some $\sigma$ in the cyclic group of degree $n$. 
Note that each element $\sigma$ of the cyclic group is determined by $\sigma^{-1}(1)$, and it is determined by $d_1$.  
For each $k\in\{1,2,\ldots, n\}$, the probability that $k=d_1$ is $\dfrac{1}{n}$ since $d_1$ is dependent on the PSS in Step (4) only. Thus the distribution of $\sigma$ is uniformly random, and hence the protocol is correct.}

\md{We show the security of the protocol. Assume that $\sigma\in\mathfrak{S}_n$ and $\tau\in \mathfrak{S}_n$ are chosen in Steps (2) and (4), respectively. Then the first card in the $i$-th pile in Step (5) is $\mathsf{x}_{\tau^{-1}(i)}$. On the other hand, the second and third cards in the $i$-th pile in Step (5) are $a_i=\tau^{-1}\sigma^{-1}(i)$ and $b_i=\tau^{-1}\sigma^{-1}(i+1)$. Here, we consider $n+1$ as $1$. 
This implies that these opened symbols $a_1,\ldots, a_n$ and $b_1,\ldots, b_n$ do not allow us to guess the first card of any pile since $\sigma$ is chosen uniformly at random in Step (4). Therefore, the protocol is secure.}






\subsection{The undirected $n$-cycle}\label{sec:dihedral}
Next, we consider the undirected $n$-cycle graph $C_n$:
\[ C_n= \begin{xy}
                     (0,-2)*[o]+{1}="1",(15,-2)*[o]+{2}="2",(30,-2)*[o]+{\cdots}="3",(45,-2)*[o]+{n-1}="4",(60,-2)*[o]+{n.}="n",
                     \ar @{-} "1";"2"^{}
                     \ar @{-} "2";"3"^{}
                     \ar @{-} "3";"4"_{}
                     \ar @{-} "4";"n"_{}
                     \ar @{-} @(lu,ur)"n";"1"_{}
            \end{xy}\]
\md{Recall that we regard undirected edge as two directed edges with opposite directions (see the paragraph just before Definition \ref{def:graphshuffle}).} 
The automorphism group $\aut(C_n)$ is isomorphic to the dihedral group of degree $n$. 
\md{For example, the graph shuffle for $C_n$ when $n = 4$ is given as follows: 
\[
\crd{1}~\crd{2}~\crd{3}~\crd{4}~ \longmapsto
\begin{cases}
~\crd{1}~\crd{2}~\crd{3}~\crd{4}~\\
~\crd{2}~\crd{3}~\crd{4}~\crd{1}~\\
~\crd{3}~\crd{4}~\crd{1}~\crd{2}~\\
~\crd{4}~\crd{1}~\crd{2}~\crd{3}~\\
~\crd{4}~\crd{3}~\crd{2}~\crd{1}~\\
~\crd{3}~\crd{2}~\crd{1}~\crd{4}~\\
~\crd{2}~\crd{1}~\crd{4}~\crd{3}~\\
~\crd{1}~\crd{4}~\crd{3}~\crd{2}~,
\end{cases}
\]
where each sequence is obtained with probability $1/8$.}
If we apply the graph shuffle for $C_n$, we need $6n$ cards.
%To streamline our protocol for $C_n$, we give a procedure to realize it with $2n$ auxiliary cards. 
In this subsection, we propose a graph shuffle protocol for $C_n$ with $3n$ cards only. 

\md{
%Now we shortly mention how to improve the number of cards. 
For an undirected $n$-cyclic graph, even though it has $2n$ edges, the number of cards corresponding to edges is reduced to $n$ pairs of cards using the symmetry of the graph.
This improvement is done by the pile-scramble shuffle in Step (3) in the below protocol. }
%Besides the same idea in Subsection \ref\label{4-1}, 

Let $D_{\f{inp}}=\{x_1,x_2,\ldots, x_n\}$ be an arbitrary deck and $D_{\f{help}} = \{1, 1, 2, 2, 3, 3, \ldots, n, n\}$ a deck of the symbols of $2n$ helping cards. 
The sequence of helping cards $\f{h}$ is defined as follows:
\[ 
\f{h} = \crd{1}\,\crd{1}~\crd{2}\,\crd{2}~\crd{3}\,\crd{3}~\cdots\crd{$n$}\,\crd{$n$}~. \]
For $i=1,2,\ldots ,n$, we set $\f{pile}[i]=\left(\dfrac{?}{i}, \dfrac{?}{i}\right)$.

\begin{enumerate}
\item[(1)] Place the $3n$ cards as follows:
\[
\underbrace{\back \, \back \, \back \, \cdots \, \back}_{\f{x}} ~ 
\underbrace{\crd{1}\,\crd{1}~\crd{2}\,\crd{2}~\crd{3}\,\crd{3}~\cdots\crd{$n$}\,\crd{$n$}}_{\f{h}}~. 
\]

\item[(2)] Arrange the card-sequence as follows:
\[
\underbrace{\back \, \back \, \back \, \cdots \, \back}_{\f{x}} ~ 
\underbrace{\underset{1}{\back} \, \underset{1}{\back}}_{\f{pile}[1]}~ 
\underbrace{\underset{2}{\back} \, \underset{2}{\back}}_{\f{pile}[2]}~ 
\underbrace{\underset{3}{\back} \, \underset{3}{\back}}_{\f{pile}[3]}~ 
~\cdots~ 
\underbrace{\underset{n}{\back} \, \underset{n}{\back}}_{\f{pile}[n]}~ .
\]
%\[
%\underbrace{\back \, \back \, \cdots \, \back}_{\f{x}} ~ 
%\underbrace{\crd{1}\,\crd{1}~\crd{2}\,\crd{2}~\crd{3}\,\crd{3}~\cdots\crd{$n$}\,\crd{$n$}}_{\f{h}}~. 
%\]
Apply $\PSS_{(n,2)}$ to $(\f{pile}[1], \f{pile}[2] ,\ldots,\f{pile}[n])$ and then we obtain the card-sequence as follows:
\[
\underbrace{\back \, \back \, \back \, \cdots \, \back}_{\f{x}} ~ 
\underset{\alpha_1}{\back} \, \underset{\alpha_1}{\back}
~
\underset{\alpha_2}{\back} \, \underset{\alpha_2}{\back}
~
\underset{\alpha_3}{\back} \, \underset{\alpha_3}{\back}
~
\cdots
~
\underset{\alpha_n}{\back} \, \underset{\alpha_n}{\back}~,
\]
where $\{\alpha_1,\alpha_2,\ldots,\alpha_n\}=\{1,2,\ldots, n\}$.

\item[(3)] Arrange the card-sequence as follows:
\[ 
\underbrace{\back \, \back \, \back \, \cdots \, \back}_{\f{x}} ~
\underset{\alpha_1}{\back} \, \underset{\alpha_2}{\back} \, \underset{\alpha_3}{\back} \, \cdots \, \underset{\alpha_{n-1}}{\back} \, \underset{\alpha_n}{\back} ~~
\underset{\alpha_2}{\back} \, \underset{\alpha_3}{\back} \, \underset{\alpha_4}{\back} \, \cdots \, \underset{\alpha_n}{\back} \, \underset{\alpha_1}{\back}\,.
\]
Apply $\PSS_{(2,n)}$ to the rightmost card-sequence of $2n$ cards as follows:
\[
\bigg|~
\underset{\alpha_1}{\back} \, \underset{\alpha_2}{\back} \, \underset{\alpha_3}{\back} \, \cdots \, \underset{\alpha_{n-1}}{\back} \, \underset{\alpha_n}{\back}
~\bigg|~
\underset{\alpha_2}{\back} \, \underset{\alpha_3}{\back} \, \underset{\alpha_4}{\back} \, \cdots \, \underset{\alpha_n}{\back} \, \underset{\alpha_1}{\back}
~\bigg|.\]
Then we obtain the following card-sequence:
\[\underbrace{\back\, \back\,  \cdots\, \back}_{\f{x}} ~~\underset{\beta_1}{\back} \, \underset{\beta_2}{\back} \, \underset{\beta_3}{\back} \, \cdots \, \underset{\beta_n}{\back}
~~
\underset{\gamma_1}{\back} \, \underset{\gamma_2}{\back} \, \underset{\gamma_3}{\back} \, \cdots \, \underset{\gamma_n}{\back}~,
\]
where $\{(\alpha_1,\alpha_2,\ldots,\alpha_n), (\alpha_2,\ldots,\alpha_n,\alpha_1)\}=\{(\beta_1,\beta_2,\ldots, \beta_n), (\gamma_1,\gamma_2,\ldots,\gamma_n)\}$.

\item[(4)] Arrange the card-sequence as follows:
\[ 
\underset{\f{x}_1}{\back} \,
\underset{\beta_1}{\back} \, \underset{\gamma_1}{\back} ~
\underset{\f{x}_2}{\back} \,
\underset{\beta_2}{\back} \, \underset{\gamma_2}{\back} ~
\underset{\f{x}_3}{\back} \,
\underset{\beta_3}{\back} \, \underset{\gamma_3}{\back} ~
\cdots ~
\underset{\f{x}_n}{\back} \,
\underset{\beta_n}{\back} \, \underset{\gamma_n}{\back}\,.
\]

\item[(5)] Apply $\PSS_{(n,3)}$ to the card-sequence as follows:
\[
\begin{tabular}{|c|c|c|c|c|}
$\back\, \back\, \back$ &
$\back\, \back\, \back$ &
$\back\, \back\, \back$ &
$\cdots$ &
$\back\, \back\, \back$
\end{tabular}\,.
\]

\item[(6)] For all piles, turn over the second and third cards. Let $a_i, b_i \in \{1,2,\ldots, n\}$ be the opened symbols of the second and third cards, respectively, in the $i$-th pile as follows:
%Then we suppose that the following card-sequence is obtained.
\[
\back\,\crd{$a_1$}\,\crd{$b_1$}~~\back\,\crd{$a_2$}\,\crd{$b_2$}~~\back\,\crd{$a_3$}\,\crd{$b_3$}~~\cdots~~\back\,\crd{$a_n$}\,\crd{$b_n$}~\md{.}
\]

\item[(7)] Arrange $n$ piles so that $(c_1, d_1) = (a_1, b_1)$, $d_i = c_{i+1}$, $(1 \leq i \leq n-1)$, and $d_n = c_1$ as follows:
\[
\underset{\f{y}_1}{\back} \,\crd{$c_1$}\,\crd{$d_1$}~~\underset{\f{y}_2}{\back} \,\crd{$c_2$}\,\crd{$d_2$}~~\underset{\f{y}_3}{\back} \,\crd{$c_3$}\,\crd{$d_3$}~~\cdots~~\underset{\f{y}_n}{\back} \,\crd{$c_n$}\,\crd{$d_n$}~.
\]
Then arrange the card-sequence as follows:
\[
\underset{\f{y}_1}{\back} \, \underset{\f{y}_2}{\back} \, \underset{\f{y}_3}{\back} \, \cdots \, \underset{\f{y}_n}{\back} ~ 
\underbrace{\crd{1}\,\crd{1}~\crd{2}\,\crd{2}~\crd{3}\,\crd{3}~\cdots\crd{$n$}\,\crd{$n$}}_{\f{h}}~. 
\]
Then the output card-sequence is $(\f{y}_1,\f{y}_2,\ldots, \f{y}_n)$.
\end{enumerate}


\md{We first show the correctness of the protocol. 
%In the last of this subsection, we show that the correctness and the security.
Let $\mathsf{x}=(\mathsf{x}_1,\ldots, \mathsf{x}_n)$ be an input sequence. 
Assume that the protocol outputs the sequence $\mathsf{y}=(\mathsf{y}_1,\ldots, \mathsf{y}_n)$ when $\mathsf{x}$ is given as input. 
%We first show that $\mathsf{y}$ is equal to $\sigma(\f{x})$ for some $\sigma \in \aut_0(C_n)$. 
%Assume that $\mathsf{x}$ outputs the sequence $\mathsf{y}=(\mathsf{y}_1,\ldots, \mathsf{y}_n)$ by applying the above protocol. 
Observe that if we apply a graph shuffle for $C_n$ to $\mathsf{x}$, the output sequence is one of the following sequences}
\[ \md{(\mathsf{x}_k,\mathsf{x}_{k+1},\ldots, \mathsf{x}_n, \mathsf{x}_1,\mathsf{x}_2\ldots, \mathsf{x}_{k-1}),\quad  (\mathsf{x}_k,\mathsf{x}_{k-1},\ldots, \mathsf{x}_1, \mathsf{x}_n,\mathsf{x}_{n-1},\ldots, \mathsf{x}_{k+1})}\]
\md{for some $k\in \{1,2,\ldots, n \}$. 
We denote by $\mathsf{Cyc}(k)$ and $\mathsf{Rev}(k)$ the former sequence and the latter sequence, respectively. 
To show the correctness of the protocol, we see that $\mathsf{y}$ is one of $\mathsf{Cyc}(k)$ and $\mathsf{Rev}(k)$ for some $k=1,\ldots, n$.
For $i=1,\ldots, n$, we set $P_i=(\mathsf{x}_i, \beta_i, \gamma_i)$ and put $P=(P_1,\ldots, P_n)$. 
Suppose that $(\alpha_1,\ldots, \alpha_n)$ is equal to $(\beta_1,\ldots, \beta_n)$ in Step (3). 
In this case, it holds $\gamma_i=\beta_{i+1}$ for any $i\in\{1,2,\ldots,n\}$, where $\beta_{n+1}=\beta_1$. 
%, hold. 
It follows from the above equations and the argument in the proof of the correctness of the protocol in Subsection \ref{4-1} that $\mathsf{y}=\mathsf{Cyc}(k)$ for some $k$. Similarly, if $(\alpha_1,\ldots, \alpha_n)=(\gamma_1,\ldots, \gamma_n)$ in Step (3), the equations $\gamma_i=\beta_{n-i+1}$ hold for any $i\in\{1,2,\ldots,n\}$. This implies that $\mathsf{y}=\mathsf{Rev}(k)$ for some $k$.}

\md{Next, we show that the distribution of $\mathsf{y}$ is uniform. Assume that $n=2$. We note that $\mathsf{Cyc}(1)=\mathsf{Rev}(1)$ and $\mathsf{Cyc}(2)=\mathsf{Rev}(2)$. Then the candidates appearing as a result of Step (4) are: }
\[ \md{\underset{\f{x}_1}{\back} \,
\underset{1}{\back} \, \underset{2}{\back} ~
\underset{\f{x}_2}{\back} \,
\underset{2}{\back} \, \underset{1}{\back} ~}, \quad 
\md{\underset{\f{x}_1}{\back} \,
\underset{2}{\back} \, \underset{1}{\back} ~
\underset{\f{x}_2}{\back} \,
\underset{1}{\back} \, \underset{2}{\back} ~}\ ,\] 
\md{and these each have a probability of $\dfrac{1}{2}$. Thus, the probabilities that $\mathsf{y}=(\mathsf{x}_1,\mathsf{x}_2)$ and $\mathsf{y}=(\mathsf{x}_2,\mathsf{x}_1)$ are same. Now, we assume that $n\geq 3$. In this case, for any $k=1,\ldots, n$, all sequences $\mathsf{Cyc}(k)$ and $\mathsf{Rev}(k)$ are distinct. In order to get $\mathsf{y}=\mathsf{Cyc}(k)$,  it requires that $(\alpha_1,\ldots, \alpha_n)=(\beta_1,\ldots, \beta_n)$ in Step (3) and $\sigma^{-1}(1)=k$, where $\sigma$ is the chosen permutation in Step (5). 
Hence, the probability that $\mathsf{y}=\mathsf{Cyc}(k)$ is $\dfrac{1}{2k}$. Similarly, the probability that $\mathsf{y}=\mathsf{Rev}(k)$ is also $\dfrac{1}{2k}$. This shows that the protocol is correct.}

\md{We show the correctness of the protocol. 
%Lastly, we show that the security of the protocol.
Assume that $\sigma\in\mathfrak{S}_n$ and $\tau\in \mathfrak{S}_n$ are chosen in Step (2) and Step (5), respectively. Then the first card in the $i$-th pile in Step (6) is $\mathsf{x}_{\tau^{-1}(i)}$. On the other hand, the second and third cards in the $i$-th pile in Step (5) are depending on the result of Step (3), and they are determined as follows. 
If $(\alpha_1,\ldots, \alpha_n)=(\beta_1,\ldots, \beta_n)$, then $a_i=\tau^{-1}\sigma^{-1}(i)$ and $b_i=\tau^{-1}\sigma^{-1}(i+1)$, 
otherwise, $a_i=\tau^{-1}\sigma^{-1}(i+1)$ and $b_i=\tau^{-1}\sigma^{-1}(i)$. Here, we consider $n+1$ as $1$. 
In either case,  these open symbols $a_1,\ldots, a_n$ and $b_1,\ldots, b_n$ do not allow us to guess the first card of any pile since $\sigma$ is chosen uniformly at random in Step (5). Therefore, the protocol is secure.}


%Let $D$ be an arbitrary deck with $n$ cards, say $x_1,x_2,\ldots ,x_n$, and $D''$ the set of $2n$ cards such that their fronts are 
%\[ \crd{1}\,\crd{1}~\crd{2}\,\crd{2}~\crd{3}\,\crd{3}~\cdots\crd{$n$}\,\crd{$n$}~. \]
%Similarly, we set $\f{pile}[i]=\left(\dfrac{?}{i}, \dfrac{?}{i}\right)$.
%
%\begin{enumerate}
%\item[(1)] For an input card-sequence $\f{x}$, we take $(\f{x}, \f{pile}[1], \f{pile}[2] ,\ldots,\f{pile}[n])$~.
%Applying $\PSS_{(n,2)}$ to the card-sequence $(\f{pile}[1], \f{pile}[2], ,\ldots,\f{pile}[n])$, we obtain the following card sequence 
%\[
%\underbrace{\back\, \back\,  \cdots\, \back}_{\f{x}}
%\underset{\phi_1}{\back} \, \underset{\phi_1}{\back}
%~
%\underset{\phi_2}{\back} \, \underset{\phi_2}{\back}
%~
%\underset{\phi_3}{\back} \, \underset{\phi_3}{\back}
%~
%\cdots
%~
%\underset{\phi_n}{\back} \, \underset{\phi_n}{\back}~,
%\]
%where $\{\phi_1,\phi_2,\ldots,\phi_n\}=\{1,2,\ldots, n\}$.
%
%\item[(2)] Arrange the card-sequence as follows:
%\[\f{z}=\underbrace{\back\, \back\,  \cdots\, \back}_{\f{x}} \underset{\phi_1}{\back} \, \underset{\phi_2}{\back} \, \underset{\phi_3}{\back} \, \cdots \, \underset{\phi_{n-1}}{\back} \, \underset{\phi_n}{\back}\,
%\underset{\phi_2}{\back} \, \underset{\phi_3}{\back} \, \underset{\phi_4}{\back} \, \cdots \, \underset{\phi_n}{\back} \, \underset{\phi_1}{\back}~. \] 
%
%\item[(3)] Apply $\PSS_{(2,n)}$ to the rightmost card-sequence of $2n$ cards as follows:
%%Choose between $\f{id}$ and $\sigma$ with a probability of $\dfrac{1}{2}$, where 
%%\[ \sigma=\left(\begin{array}{cccccccccccccccc} 
%%1 & \cdots & n & n+1 &n+2 & \cdots &2n  & 2n+1 & 2n+2 & \cdots & 3n\\
%%1 & \cdots & n &  2n+1 & 2n+2  & \cdots & 3n  & n+1 &n+2 & \cdots & 2n
%%\end{array}\right).  \]
%%Then we apply the chosen substitution to the card-sequence $\f{z}$. 
%%In other words, we perform the PSS
%\[
%\bigg|~
%\underset{\phi_1}{\back} \, \underset{\phi_2}{\back} \, \underset{\phi_3}{\back} \, \cdots \, \underset{\phi_{n-1}}{\back} \, \underset{\phi_n}{\back}
%~\bigg|~
%\underset{\phi_2}{\back} \, \underset{\phi_3}{\back} \, \underset{\phi_4}{\back} \, \cdots \, \underset{\phi_n}{\back} \, \underset{\phi_1}{\back}
%~\bigg|.\]
%Suppose that we obtain the following card-sequence by the operation described above:
%\[\underbrace{\back\, \back\,  \cdots\, \back}_{\f{x}} ~~\underset{\psi_1}{\back} \, \underset{\psi_2}{\back} \, \underset{\psi_3}{\back} \, \cdots \, \underset{\psi_n}{\back}
%~~
%\underset{\chi_1}{\back} \, \underset{\chi_2}{\back} \, \underset{\chi_3}{\back} \, \cdots \, \underset{\chi_n}{\back}~,
%\]
%where $\{(\phi_1,\phi_2,\ldots,\phi_n), (\phi_2,\ldots,\phi_n,\phi_1)\}=\{(\psi_1,\psi_2,\ldots, \psi_n), (\chi_1,\chi_2,\ldots,\chi_n)\}$.
%
%
%\item[(4)] Arrange the card-sequence as follows:
%\[ 
%\f{z}' = 
%\overset{\f{x}_1}{\back} \,
%\underset{\psi_1}{\back} \, \underset{\chi_1}{\back} ~
%\overset{\f{x}_2}{\back} \,
%\underset{\psi_2}{\back} \, \underset{\chi_2}{\back} ~
%\overset{\f{x}_3}{\back} \,
%\underset{\psi_3}{\back} \, \underset{\chi_3}{\back} ~
%\cdots ~
%\overset{\f{x}_n}{\back} \,
%\underset{\psi_n}{\back} \, \underset{\chi_n}{\back}~.
%\]
%
%\item[(5)] Apply $\PSS_{(n,3)}$ to the card-sequence $\f{z}'$.
%\[
%\begin{tabular}{|c|c|c|c|c|}
%$\back\, \back\, \back$ &
%$\back\, \back\, \back$ &
%$\back\, \back\, \back$ &
%$\cdots$ &
%$\back\, \back\, \back$
%\end{tabular}
%\]
%
%\item[(6)] For all piles, turn over the second and third cards. Let $a_i, b_i \in \{1,2,\ldots, n\}$ be the opened symbols of the second and third cards, respectively, in the $i$-th pile as follows:
%%Turn over all the second and third cards of each pile.  Then we suppose that the following card-sequence is obtained.
%\[
%\back\,\crd{$a_1$}\,\crd{$b_1$}~~\back\,\crd{$a_2$}\,\crd{$b_2$}~~\back\,\crd{$a_3$}\,\crd{$b_3$}~~\cdots~~\back\,\crd{$a_n$}\,\crd{$b_n$}
%\]
%
%\item[(7)] Arrange $n$ piles so that $(c_1, d_1) = (a_1, b_1)$, $d_i = c_{i+1}$, $(1 \leq i \leq n-1)$, and $d_n = c_1$ as follows:
%%Sort the above sequence of piles of cards so that $(c_1, d_1) = (a_1, b_1)$, $d_i = c_{i+1}$, $(1 \leq i \leq n-1)$, and $d_n = c_1$ are satisfied:
%\[
%\overset{\f{y}_1}{\back} \,\crd{$c_1$}\,\crd{$d_1$}~~\overset{\f{y}_2}{\back} \,\crd{$c_2$}\,\crd{$d_2$}~~\overset{\f{y}_3}{\back} \,\crd{$c_3$}\,\crd{$d_3$}~~\cdots~~\overset{\f{y}_n}{\back} \,\crd{$c_n$}\,\crd{$d_n$}~.
%\]
%Then the output card-sequence is $\f{y}=(\f{y}_1,\f{y}_2,\ldots, \f{y}_n)$.
%\end{enumerate}
