Traditionally, topic modeling and discovery methods have focused on extracting high quality, interpretable topics that aim to succinctly represent the inherent latent structure within a corpus. Indicatively, there have been different schools of thought on topic extraction, ranging from factorization-based methods \cite{deerwester1990indexing, xu2003document}  to probabilistic graphical models \cite{blei2003latent,steyvers2004probabilistic}.

Recently, there has been significant interest in studying the evolution of topics over time, and this has found particular applications in \cite{wiredmovie} and \cite{he2009detecting}. In general, taking time into account offers the advantage of putting an extracted topic into historical context and can enable the analysis to link the topic to external events that may be related to it.

To the best of our knowledge, the state-of-the-art in time-evolving topic extraction has focused on a notion of ``time'' that pertains to the particular moment that a topic emerged and how it evolved throughout its history within a corpus. However, when we are dealing with topic extraction from community and crowd based platforms, such as \texttt{Stack Exchange}, an additional notion of ``time'' arises. This notion of time is related to the evolution of the user who contributes the content: a relatively new user is more likely to contribute ``entry-level'' content, whereas an experienced user who has already contributed a significant amount of posts, is more likely to create more advanced content. 
%\reminder{do we have a quick way to show that with data??} 
Consider the two following questions posted in \texttt{Stack Exchange} Physics forum. 

\begin{quote}
\textbf{1. Why does centripetal acceleration have a magnitude?}
Assuming that the magnitude of velocity is constant. Why does centripetal acceleration have a magnitude? Since acceleration is the rate of change for velocity and its magnitude remains the same shouldn't we express centripetal acceleration by the angle it changed in the vertical or horizontal over a period of time instead?
\end{quote}

\begin{quote}
\textbf{2. Will the Sun's fast (but subluminal) removal cause gravitational waves?}
 We cannot just remove the sun as it violates energy conservation. We can however let the Sun accelerate fast out of the solar system. Assuming this (unreasonable) scenario, will this fast disappearance of Sun cause any gravitational wave signature? Basically would an experiment such as LIGO be able to measure a gravitational signature of the Sun's removal. 
\end{quote}
The first post is written by a new user who never answered to any others question in the \texttt{Stack Exchange} forum. The question is about ``magnitude of velocity" which is not considered an advanced topic in physics. On the other hand the second post is written by an advanced user who has answer about 70 questions in the forum. The topic of the question is gravitational waves, an advanced topic in physics.
Therefore, information about the ``experience'' of the user who contributed a piece of content in our corpus (measured by the number of post they have already contributed) provides useful information about the level of the content.

Previous work on topic detection has overlooked this notion of time, which relates to user maturity and experience,  and which, as we showcase in this paper, can provide valuable insights on how advanced a particular topic is. In addition to being able to tease out latent concepts of varying levels, these insights are also useful in bootstrapping automated curriculum design approaches such as \cite{agrawal2016toward}  which require a set of concepts to be taught in a curriculum, as well as prerequisite relations for those concepts, which can be given via the user maturity dimension in our topic discovery.

In this paper we introduce an time-evolving topic discovery method, based on Constrained Coupled Matrix-Tensor Factorization, which effectively models time and user maturity/experience towards {\em extracting interpretable topics, their temporal evolution, as well as their level of difficulty}. Figure \ref{fig:ligo} shows a representative such topic detected by our algorithm. The topic corresponds to ``Gravitational Waves Detection by Ligo Lab"; it is clearly an advanced Physics topic, which is indicated by the ``level of difficulty'' aspect of our results, and the topic made its appearance in February of 2016 (as indicated by the ``time'' aspect), which was the date it was announced. 


\begin{figure}
 \begin{subfigure}{0.22\textwidth}  
  \centering  
  \includegraphics[width=\linewidth]{figures/ligo_parafac_0word}  
    \caption{Words}   
\end{subfigure}   
\begin{subfigure}{0.22\textwidth}  
  \centering  
  \includegraphics[width=\linewidth]{figures/ligo_parafac_0time}  
    \caption{Time in Weeks}   
\end{subfigure}  

\begin{subfigure}{0.45\textwidth}  
  \centering  
  \includegraphics[width=\linewidth]{figures/ligo_parafac_0post}  
    \caption{{\footnotesize``Level of difficulty''}\hide{(measured by Log Post Number, see Section \ref{sec:problem})}}   
\end{subfigure}  
 \caption{\label{fig:ligo}An example of a topic discussed by advances users at a specific time. This pattern indicates topics discussed in response to an external event. The peak of the  ``time'' mode corresponds to February of 2016 when detection of gravitational waves was announced by Ligo lab; furthermore, ``gravitational waves'' is an advanced Physics topic, and our method correctly infers its level of difficulty.}
 \vspace*{-12pt}
\end{figure}


Our contributions are summarized as follows:
\begin{itemize}
\item {\bf Novel Problem Formulation}: We introduce a novel method based on coupled matrix-tensor analysis to discover evolutionary topics and their level of difficulty in online communities. 
%We identify events triggered by external events which are not part of topics evolution and exclude them from the curriculum. We characterize the difficulty of topics and the order that they should appear in the curriculum. 
\item {\bf Constrained Coupled Matrix-Tensor Model}: We propose a novel flexible constrained coupled matrix-tensor factorization model which incorporates sparsity, non-negativity, and orthogonality constraints which are motivated by our topic discovery goal and, as we demonstrate in the experimental evaluation, are essential for the accurate discovery of topics. We derive an efficient Alternating Least Squares algorithm for our proposed factorization model, and in order to promote reproducibility and further research, we release our code at \url{https://github.com/ConCMTF/ConCMTF}.
\item {\bf Evaluation on Real Data}: We qualitatively evaluate our method in comparison to the baseline approaches on real and public data from the Physics and also Programming forum of Stack Exchange. In particular, we demonstrate the power of the proposed method in discovering easy-to-interpret time-evolving topics and their level of difficulty. 
\item {\bf User Study}: For the quantitative analysis of our method, we conduct a user study among 10 domain experts in Physics who judged the quality, interpretability, and coherence of our results.
%\item {\bf Other Applications}: Although this paper focuses on the application of our method for topic extraction, our proposed technique has proven effective on other datasets and other applications as well. Due to lack of space, we omitted the experiments on other datasets. Interested readers can find more about experiments on other datasets in  \url{http://cs-people.bu.edu/bahargam/cmtf/}.
\end{itemize}


%\reminder{It's good if we have a cool picture of our most impressive results here in the first page - usually the reviewer forms an opinion about the paper after reading the intro, so we should impress them right away! :)}




\spara{Roadmap}
The paper is structured as follows. We start with a discussion of related works in Section \ref{prelim}. In Section \ref{prelim},  we provide certain our notations and background on tensor factorization. 
We formally define our framework in Section \ref{problem}. Our algorithm for the problem is described
in Section \ref{algo}. 
Section \ref{results} presents the results and  empirical evaluations.
Finally, we conclude with a summary and directions for future work in Section \ref{conclusion}. A version of this paper appears in the Proceedings of the 2018 IEEE/ACM International Conference on Advances in Social Networks Analysis and Mining \cite{Bahargam:2018}. 


%%%%%%OLD INTRO

\hide{
In both online and traditional classrooms, the main goal the of the instructor is to  provide the best experience for students by different means including promoting effective study groups  \cite{bahargam2015personalized},  providing the pedagogical contents  \cite{agrawal2016toward}, and the way they are delivered \cite{mullaney2015staggered} to students. It is well studied that the content and the way that content is presented to learners is very important in learning efficiency. However, choosing the right curriculum (syllabus) is not an easy task. An important challenge for instructors is always to design the materials and the depth of each topic they think is necessary to be taught. Given the rapid increase of new course offering in online educational platforms; a crucial task is to synthesize  curriculum more easily with less human effort. Due to the massive amount of discussion generated, online discussions can be utilized as a good source to extract topics and generate curriculum. 
In this paper, we address the problem of  algorithmically synthesizing curriculum by finding topics and their relevant difficulty from existing online discussions. One important advantage of our proposed algorithm is that we will generate the curriculum completely automatically. 
%without any human intervention; whereas previous studies such as \cite{agrawaldata} need the topics of the interest as the input and only output an ordering of the topics. 
By all means, one can modify the suggested curriculum obtained by our algorithm to customize it for a specific group of students. 
We consider three mode of words, time and post numbers in the tensor. An important aspect of considering time along the topics is temporal information of the topics help to understand the topic better. As an example,  the word jobs relates to employment, but after  October 5, 2011 the word jobs may refer to 'Steve Jobs'.
  % We also address how to distinguish evolutionary topics from topic triggered from external events. 
%More specifically, we focus on physics question and answers in Stack Exchange. 
%contribution

\subsection{Illustrative example}
Figure \ref{fig:intro_example} illustrates three posts written by the same user in StackOverflow. user's topic of interest has change over time from a simple question on how to \textit{iterate over Numpy matrix in Python} to something very complicated such as \textit{using Torch deep learning model in Python}. 


Figure \ref{fig:intro_example2} illustrated three different posts written by different users about \textit{Detection of Gravitational Waves} around February $11^{th}$, after Logo lab announced the news on February $11^{th}$.


While there exist team coupled tensor factorization that try to form good teams to cover some skills while minimizing the cost, to the best of our knowledge, our work is the first to tackle this customized team formation problem. Nonethe- less, our work has ties to extant research on team formation and other problems, which we review in Section 2. In Sec- tion 3, we explore alternative formal definitions and study their hardness. We then present an efficient algorithmic framework in Section 4. In Section 5, we evaluate the effi- cacy of our framework via an experimental evaluation that includes both real and synthetic data, as well as competitive baselines. Finally, we conclude our work in Section 6.

}
