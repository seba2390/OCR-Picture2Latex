%\reminder{TODO for Vagelis: change}
\hide{Despite the rapid increase of new courses offered in online educational platforms, designing a curriculum is still done in traditional ways, manually  in a limited amount of time and based on the prior experience of the instructor. Although there is a massive amount of discussion data  that can  be utilized  to help the instructor design curriculum, little work has been done to take advantage of this data.
%Little work has been done to utilize the massive amount of data available from course discussions or 
In this paper, we introduce a novel designing framework for generating a curriculum from online discussions, which allows us to automatically uncover learning units, identify their sequence and produce a curriculum. Our model allows for efficient and automatic curriculum synthesizing and scales well from small to large discussions forums. Experiments on real data gathered from Physics Stack Exchange and also Programming Stack Exchange forum shows our model is able to discover and characterize topics, track their evolution over a time, and generate meaningful and desirable curriculum. 
}
Topic discovery has witnessed a significant growth as a field of data mining at large. In particular, time-evolving topic discovery, where the evolution of a topic is taken into account has been instrumental in understanding the historical context of an emerging topic in a dynamic corpus. Traditionally, time-evolving topic discovery has focused on this notion of time. However, especially in settings where content is contributed by a community or a crowd, an orthogonal notion of time is the one that pertains to the level of expertise of the content creator: the more experienced the creator, the more advanced the topic. 

In this paper, we propose a novel time-evolving topic discovery method which, in addition to the extracted topics, is able to identify the evolution of that topic over time, as well as the level of difficulty of that topic, as it is inferred by the level of expertise of its main contributors. Our method is based on a novel formulation of  Constrained Coupled Matrix-Tensor Factorization, which adopts constraints well-motivated for, and, as we demonstrate, are essential for high-quality topic discovery.  

We qualitatively evaluate our approach using real data from the Physics and also Programming Stack Exchange forum, and we were able to identify topics of varying levels of difficulty which can be linked to external events, such as the announcement of gravitational waves by the LIGO lab in Physics forum. We provide a quantitative evaluation of our method by conducting a user study where experts were asked to judge the coherence and quality of the extracted topics. 
Finally, our proposed method has implications for automatic curriculum design using the extracted topics, where the notion of the level of difficulty is necessary for the proper modeling of prerequisites and advanced concepts. 
