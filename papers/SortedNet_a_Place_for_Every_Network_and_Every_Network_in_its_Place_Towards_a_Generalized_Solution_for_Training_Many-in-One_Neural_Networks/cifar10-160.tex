%The matrix in numbers
%Horizontal target class
%Vertical output class
\def\myConfMat{{
{19.99,	21.07,	29.40,	34.20,	36.04,	34.88,	35.76,	35.84,	37.68,	37.91,	25.60,	31.67},  %row 1
{22.41,	24.91,	25.67,	31.08,	35.37,	32.67,	35.76,	36.41,	34.81,	33.97,	38.79,	32.18},  %row 2
{17.16,	28.59,	26.09,	34.96,	41.57,	40.68,	39.03,	41.03,	41.08,	43.23,	49.20,	33.68},  %row 3
{25.55,	30.68,	28.57,	35.60,	44.41,	45.73,	43.60,	48.44,	48.60,	49.56,	50.69,	37.25},  %row 4
{27.84,	30.36,	32.03,	38.83,	44.65,	46.97,	49.03,	52.59,	52.59,	52.96,	52.64,	38.18},  %row 5
{23.27,	30.77,	42.11,	46.47,	50.08,	52.03,	57.20,	58.87,	59.97,	61.40,	59.11,	40.96},  %row 6
{24.12,	38.20,	46.21,	53.41,	57.71,	57.07,	62.81,	65.53,	65.55,	67.85,	67.73,	46.12},  %row 7
{27.32,	39.17,	49.53,	57.43,	61.96,	61.83,	66.81,	69.52,	69.39,	70.35,	72.05,	48.75},  %row 8
{28.59,	43.31,	52.97,	59.52,	63.68,	64.40,	69.35,	71.27,	70.52,	72.19,	75.01,	50.44},  %row 9
{31.45,	47.65,	57.27,	61.85,	66.52,	69.33,	73.27,	76.61,	75.13,	76.47,	78.40,	46.25},  %row 10
{39.48,	49.77,	60.41,	65.41,	70.09,	72.61,	76.43,	79.31,	79.15,	80.44,	82.81,	49.8},  %row 11
{40.99,	51.59,	60.80,	66.31,	71.68,	74.80,	78.60,	81.13,	81.24,	82.48,	85.64,	52.92},  %row 12
{33.96,	51.87,	63.20,	71.65,	76.12,	79.57,	83.59,	85.23,	86.56,	87.55,	90.71,	56.16},  %row 13
{43.85,	54.73,	64.12,	73.55,	78.03,	81.23,	85.57,	86.99,	88.44,	89.57,	92.80,	64.60},  %row 14
{45.61,	55.33,	64.61,	74.52,	79.24,	81.65,	86.00,	87.85,	89.65,	90.89,	93.92,	80.89},  %row 15
{42.91,	56.55,	65.76,	74.75,	80.83,	82.79,	87.07,	89.09,	90.87,	91.67,	94.01,	95.45},  %row 16
{38.40,	53.88,	59.44,	69.31,	74.88,	81.99,	86.40,	89.21,	92.20,	94.37,0,0},  %row 17
}}

% \def\classNames{{"10\%","20\%","30\%","40\%","50\%",60\%","70\%","80\%","90\%","100\%"}} %class names. Adapt at will

\def\classNames{{"10\%","20\%","30\%","40\%","50\%","60\%","70\%","80\%","90\%","100\%"}}

\def\xNames{{"10\%","20\%","30\%","40\%","50\%","60\%","70\%","80\%","90\%","100\%","D. Only", "Baseline"}}

\def\yNames{{"1 B.","2 B.","3 B.","4 B.","5 B.","6 B.","7 B.","8 B.","9 B.","10 B.","11 B.","12 B.","13 B.","14 B.","15 B.","16 B.","W. Only"}}

\def\numX{12}
\def\numY{17}

\def\numClasses{10} %number of classes. Could be automatic, but you can change it for tests.

\def\myScale{1.5} % 1.5 is a good scale. Values under 1 may need smaller fonts!
\begin{tikzpicture}[
    scale = \myScale,
    %font={\scriptsize}, %for smaller scales, even \tiny may be useful
    font={\Large},
    ]

\tikzset{vertical label/.style={rotate=90,anchor=east}}   % usable styles for below
\tikzset{diagonal label/.style={rotate=45,anchor=north east}}

\foreach \y in {1,...,\numY} %loop vertical starting on top
{
    % Add class name on the left
    \node [anchor=east] at (0.4,-\y) {\pgfmathparse{\yNames[\y-1]}\pgfmathresult}; 
    
    \foreach \x in {1,...,\numX}  %loop horizontal starting on left
    {
% %---- Start of automatic calculation of totSamples for the column ------------   
%     \def\totSamples{0}
%     \foreach \ll in {1,...,\numClasses}
%     {
%         \pgfmathparse{\myConfMat[\ll-1][\x-1]}   %fetch next element
%         \xdef\totSamples{\totSamples+\pgfmathresult} %accumulate it with previous sum
%         %must use \xdef fro global effect otherwise lost in foreach loop!
%     }
%     \pgfmathparse{\totSamples} \xdef\totSamples{\pgfmathresult}  % put the final sum in variable
% %---- End of automatic calculation of totSamples ----------------
    
    \begin{scope}[shift={(\x,-\y)}]
        \def\mVal{\myConfMat[\y-1][\x-1]} % The value at index y,x (-1 because of zero indexing)
        \pgfmathsetmacro{\r}{\mVal}   %
        \pgfmathtruncatemacro{\p}{\r/95*100}
        \coordinate (C) at (0,0);
        \ifthenelse{\p<50}{\def\txtcol{black}}{\def\txtcol{white}} %decide text color for contrast
        \node[
            draw,                 %draw lines
            text=\txtcol,         %text color (automatic for better contrast)
            align=center,         %align text inside cells (also for wrapping)
            fill=black!\p,        %intensity of fill (can change base color)
            minimum size=\myScale*10mm,    %cell size to fit the scale and integer dimensions (in cm)
            inner sep=0,          %remove all inner gaps to save space in small scales
            ] (C) {\r\\\p\%};     %text to put in cell (adapt at will)
        %Now if last vertical class add its label at the bottom
        \ifthenelse{\y=\numY}{
        \node [] at ($(C)-(0,0.75)$) % can use vertical or diagonal label as option
        {\pgfmathparse{\xNames[\x-1]}\pgfmathresult};}{}
    \end{scope}
    }
}
%Now add x and y labels on suitable coordinates
\coordinate (yaxis) at (-0.3,0.5-\numY/2);  %must adapt if class labels are wider!
\coordinate (xaxis) at (0.5+\numX/2, -\numY-1.25); %id. for non horizontal labels!
\node [vertical label] at (yaxis) {Depth};
\node []               at (xaxis) {Width};
\end{tikzpicture}