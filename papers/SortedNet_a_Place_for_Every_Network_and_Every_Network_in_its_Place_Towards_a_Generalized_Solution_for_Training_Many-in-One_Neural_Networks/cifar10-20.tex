%The matrix in numbers
%Horizontal target class
%Vertical output class
\def\myConfMat{{
{72.77,	80.15,	82.64,	85.71,	86.41,	90.71,	56.16},  %row 1
{84.25,	88.88,	90.25,	92.32,	93.76,	92.80,	64.60},  %row 2
{86.25,	90.11,	91.71,	93.29,	94.84,	93.92,	80.89},  %row 3
{86.24,	90.48,	91.88,	93.41,	95.16,	94.01,	95.45},  %row 4
{81.99,	86.40,	89.21,	92.20,	94.37, 0, 0},  %row 5
}}

\def\xNames{{"60\%","70\%","80\%","90\%","100\%","D. Only", "Baseline"}}

\def\yNames{{"13 B.","14 B.","15 B.","16 B.","W. Only"}}

\def\numX{7}
\def\numY{5}

\def\myScale{1.5} % 1.5 is a good scale. Values under 1 may need smaller fonts!
\begin{tikzpicture}[
    scale = \myScale,
    %font={\scriptsize}, %for smaller scales, even \tiny may be useful
    font={\normalsize},
    ]

\tikzset{vertical label/.style={rotate=90,anchor=east}}   % usable styles for below
\tikzset{diagonal label/.style={rotate=45,anchor=north east}}

\foreach \y in {1,...,\numY} %loop vertical starting on top
{
    % Add class name on the left
    \node [anchor=east] at (0.4,-\y) {\pgfmathparse{\yNames[\y-1]}\pgfmathresult}; 
    
    \foreach \x in {1,...,\numX}  %loop horizontal starting on left
    {
% %---- Start of automatic calculation of totSamples for the column ------------   
%     \def\totSamples{0}
%     \foreach \ll in {1,...,\numClasses}
%     {
%         \pgfmathparse{\myConfMat[\ll-1][\x-1]}   %fetch next element
%         \xdef\totSamples{\totSamples+\pgfmathresult} %accumulate it with previous sum
%         %must use \xdef fro global effect otherwise lost in foreach loop!
%     }
%     \pgfmathparse{\totSamples} \xdef\totSamples{\pgfmathresult}  % put the final sum in variable
% %---- End of automatic calculation of totSamples ----------------
    
    \begin{scope}[shift={(\x,-\y)}]
        \def\mVal{\myConfMat[\y-1][\x-1]} % The value at index y,x (-1 because of zero indexing)
        \pgfmathsetmacro{\r}{\mVal}   %
        \pgfmathtruncatemacro{\p}{\r/95.45*100}
        \coordinate (C) at (0,0);
        \ifthenelse{\p<50}{\def\txtcol{black}}{\def\txtcol{white}} %decide text color for contrast
        \node[
            draw,                 %draw lines
            text=\txtcol,         %text color (automatic for better contrast)
            align=center,         %align text inside cells (also for wrapping)
            fill=black!\p,        %intensity of fill (can change base color)
            minimum size=\myScale*10mm,    %cell size to fit the scale and integer dimensions (in cm)
            inner sep=0,          %remove all inner gaps to save space in small scales
            ] (C) {\r\\\p\%};     %text to put in cell (adapt at will)
        %Now if last vertical class add its label at the bottom
        \ifthenelse{\y=\numY}{
        \node [] at ($(C)-(0,0.75)$) % can use vertical or diagonal label as option
        {\pgfmathparse{\xNames[\x-1]}\pgfmathresult};}{}
    \end{scope}
    }
}
%Now add x and y labels on suitable coordinates
\coordinate (yaxis) at (-0.5,0.0-\numY/2);  %must adapt if class labels are wider!
\coordinate (xaxis) at (0.5+\numX/2, -\numY-1.25); %id. for non horizontal labels!
\node [vertical label] at (yaxis) {Depth};
\node []               at (xaxis) {Width};
\end{tikzpicture}