%The matrix in numbers
%Horizontal target class
%Vertical output class


\def\myConfMat{{
{19.99, 21.07, 29.40, 34.20, 36.04, 34.88, 35.76, 35.84, 37.68, 37.91, 31.67},  %row 1
{22.41, 24.91, 25.67, 31.08, 35.37, 32.67, 35.76, 36.41, 34.81, 33.97, 32.18},  %row 2
{17.16, 28.59, 26.09, 34.96, 41.57, 40.68, 39.03, 41.03, 41.08, 43.23, 33.68},  %row 3
{25.55, 30.68, 28.57, 35.60, 44.41, 45.73, 43.60, 48.44, 48.60, 49.56, 37.25},  %row 4
{27.84, 30.36, 32.03, 38.83, 44.65, 46.97, 49.03, 52.59, 52.59, 52.96, 38.18},  %row 5
{23.27, 30.77, 42.11, 46.47, 50.08, 52.03, 57.20, 58.87, 59.97, 61.40, 40.96},  %row 6
{24.12, 38.20, 46.21, 53.41, 57.71, 57.07, 62.81, 65.53, 65.55, 67.85, 46.12},  %row 7
{27.32, 39.17, 49.53, 57.43, 61.96, 61.83, 66.81, 69.52, 69.39, 70.35, 48.75},  %row 8
{28.59, 43.31, 52.97, 59.52, 63.68, 64.40, 69.35, 71.27, 70.52, 72.19, 50.44},  %row 9
{31.45, 47.65, 57.27, 61.85, 66.52, 69.33, 73.27, 76.61, 75.13, 76.47, 46.25},  %row 10
{39.48, 49.77, 60.41, 65.41, 70.09, 72.61, 76.43, 79.31, 79.15, 80.44, 49.80},  %row 11
{40.99, 51.59, 60.80, 66.31, 71.68, 74.80, 78.60, 81.13, 81.24, 82.48, 52.92},  %row 12
{33.96, 51.87, 63.20, 71.65, 76.12, 79.57, 83.59, 85.23, 86.56, 87.55, 56.16},  %row 13
{43.85, 54.73, 64.12, 73.55, 78.03, 81.23, 85.57, 86.99, 88.44, 89.57, 64.60},  %row 14
{45.61, 55.33, 64.61, 74.52, 79.24, 81.65, 86.00, 87.85, 89.65, 90.89, 80.89},  %row 15
{42.91, 56.55, 65.76, 74.75, 80.83, 82.79, 87.07, 89.09, 90.87, 91.67, 95.45},  %row 16
}}

\def\myConfMatAdjusted{{
{31.50, 30.52, 32.52, 34.29, 37.80, 38.96, 40.21, 41.79, 42.47, 43.43, 31.67},  %row 1
{31.44, 33.05, 36.35, 36.44, 38.35, 38.91, 39.97, 41.81, 42.49, 43.40, 32.18},  %row 2
{31.40, 37.17, 38.69, 42.01, 47.21, 49.73, 49.65, 50.67, 50.80, 51.77, 33.68},  %row 3
{33.45, 39.97, 41.66, 47.59, 49.92, 52.25, 52.55, 54.21, 55.39, 55.69, 37.25},  %row 4
{34.36, 39.81, 41.67, 48.19, 51.12, 53.64, 54.19, 56.15, 57.56, 57.83, 38.18},  %row 5
{36.59, 41.71, 47.47, 51.89, 54.99, 57.56, 60.69, 62.48, 64.45, 65.33, 40.96},  %row 6
{39.71, 46.96, 52.92, 57.99, 61.23, 64.15, 67.24, 68.52, 71.29, 71.96, 46.12},  %row 7
{42.95, 50.11, 56.04, 62.04, 65.67, 68.95, 70.80, 72.43, 74.75, 75.29, 48.75},  %row 8
{43.71, 50.43, 57.87, 64.04, 67.92, 71.04, 73.11, 74.85, 77.23, 77.55, 50.44},  %row 9
{46.19, 50.95, 59.37, 66.37, 70.56, 73.59, 76.40, 78.80, 80.64, 81.29, 46.25},  %row 10
{47.20, 52.93, 61.53, 68.72, 73.53, 77.15, 79.97, 82.49, 84.73, 85.29, 49.80},  %row 11
{47.47, 53.92, 61.73, 70.61, 75.20, 79.77, 82.91, 84.56, 86.75, 87.52, 52.92},  %row 12
{46.89, 55.24, 63.63, 72.33, 77.81, 82.43, 85.49, 87.07, 89.27, 90.48, 56.16},  %row 13
{47.75, 57.15, 65.39, 75.11, 79.99, 84.28, 86.87, 88.99, 90.87, 91.83, 64.60},  %row 14
{48.52, 57.85, 66.59, 75.71, 81.15, 85.07, 87.64, 89.76, 91.57, 92.80, 80.89},  %row 15
{47.28, 57.44, 66.23, 75.88, 81.51, 84.79, 87.60, 90.03, 91.63, 93.07, 95.45},  %row 16
}}

% \def\classNames{{"10\%","20\%","30\%","40\%","50\%",60\%","70\%","80\%","90\%","100\%"}} %class names. Adapt at will

\def\classNames{{"10\%","20\%","30\%","40\%","50\%","60\%","70\%","80\%","90\%","100\%"}}

\def\xNames{{"10\%","20\%","30\%","40\%","50\%","60\%","70\%","80\%","90\%","100\%"}}

\def\yNames{{"1 B.","2 B.","3 B.","4 B.","5 B.","6 B.","7 B.","8 B.","9 B.","10 B.","11 B.","12 B.","13 B.","14 B.","15 B.","16 B."}}

\def\numX{10}
\def\numY{16}

\def\numClasses{10} %number of classes. Could be automatic, but you can change it for tests.

\def\myScale{1.5} % 1.5 is a good scale. Values under 1 may need smaller fonts!
\begin{tikzpicture}[
    scale = \myScale,
    %font={\scriptsize}, %for smaller scales, even \tiny may be useful
    font={\Large},
    ]

\tikzset{vertical label/.style={rotate=90,anchor=east}}   % usable styles for below
\tikzset{diagonal label/.style={rotate=45,anchor=north east}}

\foreach \y in {1,...,\numY} %loop vertical starting on top
{
    % Add class name on the left
    \node [anchor=east] at (0.4,-\y) {\pgfmathparse{\yNames[\y-1]}\pgfmathresult}; 
    
    \foreach \x in {1,...,\numX}  %loop horizontal starting on left
    {
% %---- Start of automatic calculation of totSamples for the column ------------   
%     \def\totSamples{0}
%     \foreach \ll in {1,...,\numClasses}
%     {
%         \pgfmathparse{\myConfMat[\ll-1][\x-1]}   %fetch next element
%         \xdef\totSamples{\totSamples+\pgfmathresult} %accumulate it with previous sum
%         %must use \xdef fro global effect otherwise lost in foreach loop!
%     }
%     \pgfmathparse{\totSamples} \xdef\totSamples{\pgfmathresult}  % put the final sum in variable
% %---- End of automatic calculation of totSamples ----------------
    
    \begin{scope}[shift={(\x,-\y)}]
        \def\mValAdjusted{\myConfMatAdjusted[\y-1][\x-1]} % The value at index y,x (-1 because of zero indexing)
        \def\mVal{\myConfMat[\y-1][\x-1]} 
        \pgfmathsetmacro{\r}{\mValAdjusted}   %
        \pgfmathsetmacro{\adjdiff}{\mValAdjusted-\mVal}
        \pgfmathtruncatemacro{\adjdiffpercentage}{\adjdiff/\mVal*100}
        \pgfmathtruncatemacro{\p}{\mValAdjusted}
        \coordinate (C) at (0,0);
        \ifthenelse{\adjdiffpercentage<50}{\def\txtcol{black}}{\def\txtcol{white}} %decide text color for contrast
        \node[
            draw,                 %draw lines
            text=\txtcol,         %text color (automatic for better contrast)
            align=center,         %align text inside cells (also for wrapping)
            fill=red!\adjdiffpercentage,        %intensity of fill (can change base color)
            minimum size=\myScale*10mm,    %cell size to fit the scale and integer dimensions (in cm)
            inner sep=0,          %remove all inner gaps to save space in small scales
            ] (C) {\r \\ \adjdiffpercentage \%};     %text to put in cell (adapt at will)
        %Now if last vertical class add its label at the bottom
        \ifthenelse{\y=\numY}{
        \node [] at ($(C)-(0,0.75)$) % can use vertical or diagonal label as option
        {\pgfmathparse{\xNames[\x-1]}\pgfmathresult};}{}
    \end{scope}
    }
}
%Now add x and y labels on suitable coordinates
\coordinate (yaxis) at (-0.3,0.5-\numY/2);  %must adapt if class labels are wider!
\coordinate (xaxis) at (0.5+\numX/2, -\numY-1.25); %id. for non horizontal labels!
\node [vertical label] at (yaxis) {Depth};
\node []               at (xaxis) {Width};
\end{tikzpicture}