%\documentclass[main.tex]{subfiles}

\begin{document}

\section{Transformations that preserve Nash Equilibrium Sets or Best Response Sets}
\label{sec:strat equiv preserving}
To our knowledge, the results of this section are all novel unless explicitly stated otherwise. The proofs can be found in \ref{sec:main proofs}.

We showed in Lemma~\ref{multiplayer PAT preserves} that the maps $H^i$ of a PAT universally preserve best responses. Note moreover that by the definition of a \NE{}, if for all player~$i$ the map $H^i$ universally preserves best responses, then the game transformation $H = \{H^i\}_{i \in [N]}$ universally preserves \NE{} sets. The main result of our paper states the reverse:
\begin{thm*}
Let $H = \{H^i\}_{i \in [N]}$ be a game transformation. Then:
\begin{align}
\label{mt1} \tag{i}
&\, H \text{ universally preserves \NE{} sets} \\
\label{mt2} \tag{ii}
&\iff \text{for each player~$i$, map $H^i$ universally preserves best responses} \\
\label{mt3} \tag{iii}
&\iff H \text{ is a positive affine transformation}.
\end{align}
\end{thm*}
In this section, we develop the directions (\ref{mt1})$\implies$(\ref{mt2})$\implies$(\ref{mt3}). The key property that enables us to derive this theorem is that the game transformation $H = \{H^i\}_{i \in [N]}$ of consideration needs to be \textit{universally} applicable, no matter the input game $G = \{u_i\}_{i \in [N]}$ we have at hand. 

\begin{prop}
\label{two conclusions of strat equiv preserving}
Let $H = \{H^i\}_{i \in [N]}$ be a game transformation that universally preserves \NE{} sets and consider the map $H^i$ of a player~$i$. Then $H^i$ only depends on the strategy choice of the opponents. Moreover, $H^i$ universally preserves best responses.
\end{prop}

We prove the second conclusion of Proposition \ref{two conclusions of strat equiv preserving} by analyzing $H$ on games where all other players $j \neq i$ receive constant utilities. In such games, the \NE{} profiles are exactly those in which player $i$ plays a best response. The first conclusion of Proposition \ref{two conclusions of strat equiv preserving} relies on the following intuition: If the maps $h_{\ks}^i$ from $H^i$ would depend on the strategy choice of player $i$, then in the transformed game $H(G)$, player $i$ must adjust her strategy choice to those $h_{\ks}^i$ that yield a high transformed payoff. This would affect the strategic decision making and therefore the \NE{} set. Similar reasoning provides us with a related result:

\begin{lemma}
\label{br implies opp dependent}
Suppose a map $H^i$ universally preserves best responses. Then $H^i$ only depends on the strategy choice of the opponents.
\end{lemma}

Due to Proposition~\ref{two conclusions of strat equiv preserving}, we can transition to the analysis of transformation maps $H^i$ that universally preserve best responses. Thus from now on, our results also become relevant for the game theory literature that focuses on best response sets, such as best response dynamics  or fictitious play.

Proposition~\ref{two conclusions of strat equiv preserving} also allows us to restrict our analysis to the map $H^1$ for player~$1$ w.l.o.g. because any results for $H^1$ will, by symmetry, also hold for maps $H^2,\ldots,H^N$. By Lemma~\ref{br implies opp dependent}, we can also drop the dependence of $H^1$ on $k_1$ and write
\[H^1 := \Big\{ h_{\ks_{-1}}^1 : \R \longrightarrow \R \Big\}_{\ks_{-1} \in S^{-1}} \, . \]
 For each (pure strategy)-specific map $h_{\ks_{-1}}^1$, we introduce its \textit{distance distortion} function which takes two utility values and measures their distance after a $h_{\ks_{-1}}^1$-transformation:
\begin{align}
\label{def dist dist fct}
\begin{aligned}
\Delta h_{\ks_{-1}}^1 : \R \times \R &\longrightarrow \R 
\\
(z,w) &\longmapsto h_{\ks_{-1}}^1(z) \, - \, h_{\ks_{-1}}^1(w) \, .
\end{aligned}
\end{align}

The next lemma relates the (pure strategy)-specific maps $h_{\ks_{-1}}^1$ to each other through their distance distortion functions. The subsequent lemma then shines some light on how those maps $h_{\ks_{-1}}^1$ behave. Both lemmas are based on another technical result that we present and prove in \ref{app:relating distance maps} due to its more complicated formulation and interpretation. In all those results it is also crucial that $H^1$ is assumed to preserve best responses \textit{universally}. Only due to that we are able to deduce those global properties of the $h_{\ks_{-1}}^1$ in $H^1$.

\begin{lemma}
\label{same distance distortion functions}
Suppose transformation $H^1$ universally preserves best responses. Then the (pure strategies)-specific maps within $H^1$ equally distort distances:
\[\forall \ks_{-1} \in S^{-1} \, : \, \Delta h_{\ks_{-1}}^1 = \Delta h_{\1_{-1}}^1 \]
where $\1_{-1} := (1, \ldots, 1) \in S^{-1}$.
\end{lemma}

\begin{lemma}
\label{strat specific maps behaviour}
Suppose transformation $H^1$ universally preserves best responses. Then we obtain for all $\ks_{-1} \in S^{-1}$ that
\begin{enumerate}
\item map $h_{\ks_{-1}}^1$ is strictly increasing, and that
\item map $h_{\ks_{-1}}^1$ distorts distances independently of their reference points:
\begin{align*}
\forall z,z',\lambda \in \R \, : \, \Delta h_{\ks_{-1}}^1(z + \lambda, z) = \Delta h_{\ks_{-1}}^1(z' + \lambda, z') \, .
\end{align*}
\end{enumerate}
\end{lemma}

With Lemmas~\ref{same distance distortion functions} and~\ref{strat specific maps behaviour}, we can finally show that positive affine transformations are the only game transformations that universally preserve best responses. Intuitively speaking, the second conclusion of Lemma~\ref{strat specific maps behaviour} states that a step of length $\lambda$ in the domain space consistently maps to a step of some (other) length in the range space independently of the base point $z$ from which we take that step. This brings us to a known result from the analysis literature. Recall that a function $h: \R \longrightarrow \R$ is called linear if there exists some $a \in \R$ such that $\forall z \in \R \, : \, h(z) = az$. A function $h: \R \longrightarrow \R$ is said to be additive if it satisfies $\forall x,y \in \R \, : \, h(x+y) = h(x) + h(y)$.
\begin{lemma}[Known result from \cite{BSMA_1875__9__281_1, Reem2017RemarksOT}]
\label{helping analysis lemma}
If a map $h: \R \longrightarrow \R$ is monotone and additive, then it is also linear.
\end{lemma}
\begin{proof}
Reproven in \ref{sec:analysislemma}.
\end{proof}
\begin{cor}
\label{affine linearity corollary}
Let $h: \R \longrightarrow \R$ be monotone and satisfy for all $z,z',\lambda \in \R$:
\begin{equation*}
h(z+\lambda) - h(z) = h(z' + \lambda) - h(z') \, .
\end{equation*} 
Then h is affine linear, i.e., there exist some $a,c \in \R$ such that for all $z \in \R\, : \, h(z) = az + c$.
\end{cor}

We can conclude with the main result:
\begin{thm}
\label{equiv charact of PAT}
Let $H = \{H^i\}_{i \in [N]}$ be a game transformation. Then:
\begin{align}
\label{second:mt1} \tag{i}
&\,H \text{ universally preserves \NE{} sets} \\
\label{second:mt2} \tag{ii}
&\iff \text{for each player~$i$, map $H^i$ universally preserves best responses} \\
\label{second:mt3} \tag{iii}
&\iff H \text{ is a positive affine transformation.}
\end{align}
\end{thm}

Let us additionally state the result for $2$-player games as a special case of interest:
\begin{cor}
\label{bimatrix consequence}
PATs are all the game transformations in the sense of (\ref{bimatrix game trafo}) that can be applied to any bimatrix game without changing the \NE{} set (resp., the best response sets).
\end{cor}


\end{document}