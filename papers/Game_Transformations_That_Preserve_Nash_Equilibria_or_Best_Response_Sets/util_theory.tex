%\documentclass[main.tex]{subfiles}

\begin{document}

\section{Utility Theory in Game Theory}
\label{sec:utility_theory}

Let us revise some utility theory, as to be found in e.g. Mas-Colell et al. \cite{MasCol}.

\paragraph{Preferences and Utility Functions} Suppose a decision maker can choose one outcome from a space $C$ of $N$-many outcomes (where $N$ finite). Moreover, the decision maker prefers some outcomes over others which is captured by her preference relation $\succeq$ on $C$.

We typically describe the preferences of the decision maker through utility functions:
\begin{defn}
\label{utility repr defn}
A utility function $u : C \longrightarrow \R$ is said to represent a preference relation $\succeq$ if for all $c,d \in C$, we have $c \succeq d \iff u(c) \geq u(d)$.
\end{defn}
Multiple utility functions can represent the same preference relation. Their practical use is that they translate the preference relation $\succeq$ into comparisons of numerical values.

On the other hand, starting with a utility function $u$ yields an induced preference relation $\succeq$ through
\[ \forall \, c,d \in C \quad :  \quad c \succeq d :\iff u(c) \geq u(d) \, . \] 

\paragraph{Lotteries and the Expected Utility}

Now suppose we want to allow the decision maker to choose each outcome in $C$ with some probability. Call such a probability distribution $L = (p_1,\ldots,p_N)$ over $C$ a lottery. The $i$-th outcome in C can then be represented by the lottery $e_i \in \R^n$. Thus, we extended the choice space of the decision maker from $C$ to the space $\mathcal{L}$ of lotteries. We can also extend Definition~\ref{utility repr defn} to preference relations $\succeq$ over $\mathcal{L}$ by requiring $u : \mathcal{L} \longrightarrow \R$ and $\forall \, L,M \in \mathcal{L} \, : \, L \succeq M \iff u(L) \geq u(M)$.

We will be especially interested in those utility functions that simply compute the expected utility of randomly choosing an outcome according to input $L$.
\begin{defn*}
    A \textit{von Neumann-Morgenstern (NM)} expected utility function is a map $U : \mathcal{L} \longrightarrow \R$ that is determined by its values $U(e_i)$ on the outcomes $e_i \in C, i \in [N],$ and by 
    \[\forall L = (p_1,\ldots,p_N) \,: \quad U(L) = \sum_{i = 1}^N p_i \cdot U(e_i) \, .\]
\end{defn*}

The following theorem describes the preference relations that can be represented by a NM expected utility function. The theorem relies on four properties - called \textit{axioms} - that a preference relation $\succeq$ can satisfy: Completeness\footnote{For all $L,M \in \mathcal{L}$, we have $L \succeq M$ or $L \preceq M$ (or both, in which case we write $L \sim M$).}, Transitivity\footnote{For all $L,M,N \in \mathcal{L}$, if $L \succeq M$ and $M \succeq N$, then $L \succeq N$.}, Continuity\footnote{For all $L,M,N \in \mathcal{L}$ with $L \succeq M \succeq N$, there exists probability $p \in [0,1]$ such that $p \cdot L + (1-p) \cdot N \sim M$.} and Independence\footnote{For all $L,M,N \in \mathcal{L}$ and $p \in [0,1]$, we have $L \succeq M$ if and only if $p \cdot L + (1-p) \cdot N \succeq p \cdot M + (1-p) \cdot N$.}.
\begin{thm}[Expected Utility Theorem]
\label{exp util thm}
Let preference relation $\succeq$ satisfy the four axioms mentioned above. Then $\succeq$ can be represented by a NM expected utility function $U$. Moreover, the representing $U$ is unique up to a positive affine transformation. That is, if $U$ and $U'$ are NM expected utility functions representing $\succeq$, then there exist $\alpha,c \in \R$ such that for all $L \in \mathcal{L}$, we have $U'(L) = \alpha \cdot U(L) + c$.
\end{thm}
\begin{proof}
See Proposition~6.B.2 and 6.B.3 from Mas-Colell et al. \cite{MasCol}.
\end{proof}
In contrast to Theorem~\ref{exp util thm}, suppose we start with an arbitrary NM expected utility function $U$. Then $U$ induces a preference relation $\succeq$ on $\mathcal{L}$ by 
\[ \forall L, L' \in \mathcal{L} \quad :  \quad L \succeq L' :\iff U(L) \geq U(L') \, . \] 
By construction, $U$ represents $\succeq$. One can also show that this induced preference relation $\succeq$ satisfies the four axioms. Therefore, by Theorem~\ref{exp util thm}, $U$ uniquely represents the induced $\succeq$ up to a PAT.

\paragraph{Connections to Game Theory} Take a multiplayer game $G = \big( N, \{S^i\}_{i \in [N]}, \{u_i\}_{i \in [N]} \big)$. Then, the utility functions $u_i$ induce each player's preferences through the following:

Consider a game that only allows for pure strategy play. Then, given some player~$i$ and the pure strategy profile $s^{-i}$ of the opponents, the ``sliced'' utility function $u_i(\cdot, s^{-i})$ induces a preference relation $\succeq$ for player~$i$ over her strategy set $S^i$.

Now suppose that the input game allows for mixed strategy play. In that case, each element in $\Delta(S^i)$ can be viewed as a lottery over the choice set $C := S^i$. Moreover, player~$i$'s utility payoff from a mixed strategy profile $\displaystyle s \in \bigtimes_{i=1}^N \Delta(S^i)$ is
\[u_i(s^i, s^{-i}) = \sum_{k_i=1}^{m_i} s_{k_i}^i \cdot u_i(e_{k_i}, s^{-i}) \, . \]
Therefore, $u_i(\cdot, s^{-i})$ has the form of a NM expected utility function. This induces a preference relation $\succeq_{i, s^{-i}}$ on the space of lotteries $\Delta(S^i)$ with $\succeq_{i, s^{-i}}$ satisfying the four axioms. Hence, $u_i(\cdot, s^{-i})$ represents the induced preference relation $\succeq_{i, s^{-i}}$ uniquely up to a PAT.

\end{document}