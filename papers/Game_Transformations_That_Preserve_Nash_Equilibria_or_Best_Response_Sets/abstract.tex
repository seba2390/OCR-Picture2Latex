%\documentclass[main.tex]{subfiles}
%\graphicspath{{\subfix{../images/}}}
\begin{document}

\begin{abstract}
\noindent
In the literature on simultaneous non-cooperative games, it is a widely used fact that a positive affine (linear) transformation of the utility payoffs neither changes the best response sets nor the \NE{} set. We investigate which other game transformations also possess one of these two properties when being applied to an arbitrary $N$-player game ($N \geq 2$):
\begin{enumerate}[label=(\roman*)]
\item The \NE{} set stays the same. 
\item The best response sets stay the same. 
\end{enumerate}
For game transformations that operate player-wise and strategy-wise, we prove that (i) implies (ii) and that transformations with property (ii) must be positive affine. The resulting equivalence chain gives an explicit description of all those game transformations that always preserve the \NE{} set (or, respectively, the best response sets). Simultaneously, we obtain two new characterizations of the class of positive affine transformations.

\end{abstract}

\end{document}