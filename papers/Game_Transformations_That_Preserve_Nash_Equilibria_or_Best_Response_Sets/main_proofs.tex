%\documentclass[main.tex]{subfiles}

\begin{document}

\section{Proofs of Novel Results}
\label{sec:main proofs}

To our knowledge, the results of this section (and therefore their proofs) are all novel. All proofs can be found in the appendix.
%\todo{Note that you should be able to simplify the notation difference between el and s-i here because we defined the strat sets as number choices now. \\}

Throughout this section, we write $u \equiv \lambda$ if we want to define a function $u: \mathcal{D} \longrightarrow \R$ as a constant function on its domain $\mathcal{D}$ set to the value $\lambda \in \R$.

\begin{prop*}
\label{app:two conclusions of strat equiv preserving}
Let $H = \{H^i\}_{i \in [N]}$ be a game transformation that universally preserves \NE{} sets and consider the map $H^i$ of a player~$i$. Then $H^i$ only depends on the strategy choice of the opponents. Moreover, $H^i$ universally preserves best responses.
\end{prop*}
\begin{proof} 
Take a game transformation $H = \{H^i\}_{i \in [N]}$ that universally preserves \NE{} sets and fix some player~$i$. %\\
%\\
%First conclusion: \\
\paragraph{1} Fix a pure strategy choice $\ks_{-i} \in S^{-i}$ of the opponent players and take some arbitrary value $z \in \R$. Consider the game $G = \{u_j\}_{j \in [N]}$ with constant utility functions $u_j \equiv z$ for all $j \in [N]$. Then, the \NE{} set will be the whole strategy space $\Delta(S)$. By assumption on $H$, the transformed game $H(G)$ also has the full strategy space as its set of \NEs{}. In particular, each of the strategy profiles $(1 , \ks_{-i}), \ldots,  (m_i , \ks_{-i})$ will be a \NE{} of the transformed game $H(G)$. Hence, for all $2 \leq l \leq m_i$:

\begin{align*}
h&_{1, k_{-i}}^i(z) \overset{u_i \equiv z}{=} h_{1, \ks_{-i}}^i \Big( u_i(1, \ks_{-i}) \Big) \overset{(\ref{transformed pure utilities evaluation})}{=} H^i(u_i)(1, \ks_{-i}) \\
\overset{\text{Nash-Eq}}&{=} \max_{t^i \in \Delta(S^i)} \Big\{ \, H^i(u_i)(t^i, \ks_{-i}) \Big\} \overset{\text{Nash-Eq}}{=} H^i(u_i)(l,\ks_{-i}) \\
&= h_{l, \ks_{-i}}^i \Big( u_i(l, \ks_{-i}) \Big) = h_{l, \ks_{-i}}^i(z) \, .
\end{align*}
Since $z$ and $l$ were chosen arbitrarily, we get $h_{1, l, \ks_{-i}}^i = \ldots = h_{m_i, l, \ks_{-i}}^i$. %\\
%\\
%Second conclusion: \\
\paragraph{2} Fix player~$i$'s utility function $u_i$ and the opponents' strategy choices $\strats^{-i} \in \Delta(S^{-i})$. Then by \ref{app:BR equal through PBR}, it suffices to identify the pure strategies in the best response sets $\BR_{u_i}(\strats^{-i})$ and $\BR_{H^i(u_i)}(\strats^{-i})$.

Complete the prefixed $u_i$ to a full game $G = \{u_j\}_{j \in [N]}$ by setting $u_j \equiv 0$ for the other players $j \neq i$. Then, the best response set of a player~$j \neq i$ is her whole strategy space $\Delta(S^j)$. By assumption on the game transformation~$H$, we get for a pure strategy $e_l = l \in S^i$:
\begin{align*}
&e_l \in \BR_{u_i}(\strats^{-i}) \\
&\iff (e_l, \strats^{-i}) \text{ is a \NE{} for the game } G \\
&\iff (e_l, \strats^{-i}) \text{ is a \NE{} for the game } H(G) \\
\overset{\text{def}}&{\iff} e_l \in \BR_{H^i(u_i)}(\strats^{-i}) \\
&\, \quad \quad \quad \text{ and } \forall j \neq i \, : \, s^j \in \BR_{H^j(u_j)}(s^1, \ldots, s^{j-1}, s^{j+1}, \ldots, s^{i-1}, e_l, s^{i+1}, \ldots, s^N) \\
\overset{(*)}&{\iff} e_l \in \BR_{H^i(u_i)}(\strats^{-i})
\end{align*}
Let us give some further explanation for step $(*)$. Recall the definition for a strategy $s^j$, $j \neq i$, to be a best response to the opponents' strategy choices $(s^1, \ldots, s^{j-1}, s^{j+1}, \ldots, s^{i-1}, s^i := e_l, s^{i+1}, \ldots s^N)$:
\[s^j \in \argmax_{t^j \in \Delta(S^j)} \Big\{ \, \sum_{\ks \in S} s_{k_1}^1 \cdot \ldots \cdot s_{k_{i-1}}^{i-1} \cdot t_{k_i}^i \cdot s_{k_{i+1}}^{i+1} \cdot \ldots \cdot s_{k_N}^N \cdot h_{\ks}^j \big( u_j(\ks) \big) \, \Big\} \, .\]
We can show that the term in the argmax is constant in $t^j$. First, note that the maps $h_{\ks}^j$ are independent of player $j$'s action, which, in particular, implies $h_{\ks}^j = h_{1, \ks_{-j}}^j$. Then, rearranging yields
\begin{align*}
&\sum_{\ks \in S} s_{k_1}^1 \cdot \ldots \cdot s_{k_{i-1}}^{i-1} \cdot t_{k_i}^i \cdot s_{k_{i+1}}^{i+1} \cdot \ldots \cdot s_{k_N}^N \cdot h_{\ks}^j \big( u_j(\ks) \big) \\
\overset{u_j \equiv 0}&{=} \sum_{\ks_{-j}} \Bigg( s_{k_1}^1 \cdot \ldots \cdot s_{k_{j-1}}^{j-1} \cdot s_{k_{j+1}}^{j+1} \cdot \ldots \cdot s_{k_N}^N \cdot h_{1, \ks_{-j}}^j ( 0 ) \cdot \sum_{k_j=1}^{m_j} t_{k_j}^j \, \Bigg) \\
\overset{(\dagger)}&{=} \sum_{\ks_{-j}} s_{k_1}^1 \cdot \ldots \cdot s_{k_{j-1}}^{j-1} \cdot s_{k_{j+1}}^{j+1} \cdot \ldots \cdot s_{k_N}^N \cdot h_{1, \ks_{-j}}^j ( 0 ) \, .
\end{align*}
Since the term in the argmax is constant in $t^j$, any strategy of player~$j$ is a best response to $(s^1, \ldots, s^{j-1}, s^{j+1}, \ldots, s^{i-1}, e_l, s^{i+1}, \ldots s^N)$. Therefore, we obtain the equivalence $(*)$ by removing/adding the redundant condition on each $s^j$, $j \neq i$, to be a best response.

All in all, we proved that the sets $\BR_{u_i}(\strats^{-i})$ and $\BR_{H^i(u_i)}(\strats^{-i})$ contain the same pure strategies. \ref{app:BR equal through PBR} therefore yields set equality.
\end{proof}

\begin{lemma*}
\label{app:br implies opp dependent}
Suppose a map $H^i$ universally preserves best responses. Then $H^i$ only depends on the strategy choice of the opponents.
\end{lemma*}
\begin{proof} 
Let the pure strategy choice of the opponents be $\ks_{-i} \in S^{-i}$. Pick some $z \in \R$ and set $u_i \equiv z$. Then we can reformulate
\begin{align*}
&h_{1, \ks_{-i}}^i(z) = \ldots = h_{m_i, \ks_{-i}}^i(z) \\
&\iff \forall l \in [m_i]: \, h_{l, \ks_{-i}}^i(z) = \max_{p \in [m_i]} h_{p, \ks_{-i} }^i(z) \\
\overset{u_i \equiv z}&{\iff} \forall l \in [m_i]: \, h_{l, \ks_{-i}}^i(u_i(l, \ks_{-i})) = \max_{p \in [m_i]} h_{p, \ks_{-i}}^i(u_i(p, \ks_{-i})) \\
&\iff \forall l \in [m_i]: \, H^i(u_i)(l, \ks_{-i}) = \max_{p \in [m_i]} H^i(u_i)(p, \ks_{-i}) \\
&\iff \forall l \in [m_i]: \, e_l = l \in \BR_{H^i(u_i)}(s^{-i} = \ks_{-i}) \\
\overset{(*)}&{\iff} \forall l \in [m_i]: \, e_l = l \in \BR_{u_i}(s^{-i} = \ks_{-i}) \\
&\iff \forall l \in [m_i]: \, u_i(l, \ks_{-i}) = \max_{p \in [m_i]} u_i(p \ks_{-i}) \\
\overset{u_i \equiv z}&{\iff} \forall l \in [m_i]: \, z = \max_{p \in [m_i]} z \, .
\end{align*}
In $(*)$, we use that $H^i$ is universally best response preserving.

With the last line of the equivalence chain above being a universally true statement, we obtain that the first line also holds true. Since $z$ was chosen arbitrarily, we can conclude $h_{1, k_{-i}}^i = \ldots = h_{m_i, k_{-i}}^i$.
\end{proof}

\begin{rem*}
A distance distortion function $\Delta h_{\ks_{-1}}^1$, as defined in (\ref{def dist dist fct}), is skew-symmetric:
\begin{align}
\label{app:skew-symm}
    \forall \, z,w \in \R \, : \quad \Delta h_{\ks_{-1}}^1(z,w) = \, -\Delta h_{\ks_{-1}}^1(w,z) \, .
\end{align}
\end{rem*}

The upcoming lemma reveals an important preliminary observation on how the distance distortion functions $\Delta h_{\ks_{-1}}^1$ relate to each other. It highlights how the distorted utility distances are affected by a strategy change of a player $j \neq 1$ from, e.g., some pure strategy $k_j \in [m_j] \setminus \{1\}$ to their pure strategy $1 \in [m_j]$.

We formulate the lemma with index variables $\ps_{-1} = (p_2, \ldots, p_N)$ instead of $\ks_{-1} = (k_2,\ldots,k_N)$ in order to avoid confusion in the proof of the subsequent Lemma~\ref{same distance distortion functions}.
\begin{lemma}
\label{app:relating distance maps}
Suppose transformation map $H^1$ universally preserves best responses. Take a player $r \in [N] \setminus \{1\}$ and let $\ps_{-1} \in S^{-1}$ be such that $p_r \neq 1$. Define $\ps_{-1}' \in S^{-1}$ to have $r$-th entry $p_r' = 1$ and, otherwise, to have the same entries as $\ps_{-1}$. Then, for all $z,z',w,w' \in \R$:
\begin{align*}
z-w \geq z'-w' \quad \iff \quad \Delta h_{\ps_{-1}}^1(z,w) \geq \Delta h_{\ps_{-1}'}^1(z',w') \, .
\end{align*}
\end{lemma}

\begin{proof} 
Take a transformation map $H^1$ that universally preserves the best response sets. Then by Lemma~\ref{br implies opp dependent}, its maps $h_{\ks}^1$ only depend on the strategy choices $\ks_{-1}$ of the opponents. Fix $r, \ps_{-1}, \ps_{-1}'$ and $z,z',w,w'$ as described in the lemma statement. We will construct a utility function $u_1$ for these parameters such that a universally best response preserving $H^1$ reveals to satisfies the property of this lemma.

Set $u_1(1, \ps_{-1}) := z$ and $u_1(1, \ps_{-1}') := w'$. Additionally, for all pure strategies $l \in [m_1] \setminus \{1\}$, set $u_1(l, \ps_{-1}) := w$ and $u_1(l, \ps_{-1}') := z'$. All these utility value assignments are possible because of $p_r \neq 1 = p_r'$. The utility payoffs of player~$1$ (i.e., the values of $u_1$) from other pure strategy outcomes $\ks \in S$ can be set arbitrarily. Finally, consider the opponents' mixed strategy profile $\strats^{-1} := \frac{1}{2} \ps_{-1} + \frac{1}{2} \ps_{-1}' \in \Delta(S^{-1})$.
%$(e_{l_2}, \ldots, e_{l_{r-1}}, \frac{1}{2}e_{p_r'} + \frac{1}{2}e_{p_r}, e_{l_{r+1}}, \ldots, e_{l_N})$. 
Then we derive:
\begin{flalign*}
&z-w \geq z'-w'&&
\\
&\iff \forall l \in [m_1] \setminus \{1\} \,: \, u_1(1, \ps_{-1}) - u_1(l, \ps_{-1}) \geq u_1(l, \ps_{-1}') - u_1(1, \ps_{-1}')&&
\end{flalign*}
\textit{Reorder and divide by $2$}
\begin{flalign*}
&\iff  \forall l \in [m_1] \setminus \{1\} \,: \, \frac{1}{2}u_1(1, \ps_{-1}) + \frac{1}{2} u_1(1, \ps_{-1}') \geq \frac{1}{2} u_1(l, \ps_{-1}) + \frac{1}{2} u_1(l, \ps_{-1}')&&
\\
&\iff  \forall l \in [m_1] \setminus \{1\} \,: \, u_1(e_1, \strats^{-1}) \geq u_1(e_l, \strats^{-1})&&
\\
&\iff  e_1 \in \BR_{u_i}(\strats^{-1})&&
\end{flalign*}
\textit{$H^1$ is universally preserves best responses}
\begin{flalign*}
&\iff e_1 \in \BR_{H^1(u_1)}(\strats^{-1})&&
\\         %\displaybreak
&\iff  \forall l \in [m_1] \setminus \{1\} \,: \, H^1(u_1)(e_1, \strats^{-1}) \geq H^1(u_1)(e_l, \strats^{-1})&&
\\
&\iff \forall l \in [m_1] \setminus \{1\} \,: \, \frac{1}{2} h_{1, \ps_{-1}}^1(u_1(1, \ps_{-1}))  + \frac{1}{2} h_{1, \ps_{-1}'}^1(u_1(1, \ps_{-1}'))&&
\\
&\, \quad \quad \quad \quad \quad \quad \quad \quad \quad \quad \quad \geq \frac{1}{2} h_{l, \ps_{-1}}^1(u_1(l, \ps_{-1})) + \frac{1}{2} h_{l, \ps_{-1}'}^1(u_1(l, \ps_{-1}'))&&
\\
&\iff \forall l \in [m_1] \setminus \{1\} \,: \, h_{1, \ps_{-1}}^1(z)  + h_{1, \ps_{-1}'}^1(w') \geq h_{l, \ps_{-1}}^1(w) + h_{l, \ps_{-1}'}^1(z')&&
\end{flalign*}
\textit{$H^1$ does not depend on the pure strategy choice of player 1}
\begin{flalign*}
&\iff h_{\ps_{-1}}^1(z) + h_{\ps_{-1}'}^1(w') \geq h_{\ps_{-1}}^1(w) + h_{\ps_{-1}'}^1(z')&&
\\
&\iff h_{\ps_{-1}}^1(z) - h_{\ps_{-1}}^1(w) \geq h_{\ps_{-1}'}^1(z') - h_{\ps_{-1}'}^1(w')&&
\\
&\iff \Delta h_{\ps_{-1}}^1(z,w) \geq \Delta h_{\ps_{-1}'}^1(z',w')&&
\end{flalign*}
\end{proof}

\begin{lemma*}
\label{app:same distance distortion functions}
Suppose transformation $H^1$ universally preserves best responses. Then the (pure strategies)-specific maps within $H^1$ equally distort distances:
\[\forall \ks_{-1} \in S^{-1} \, : \, \Delta h_{\ks_{-1}}^1 = \Delta h_{\1_{-1}}^1 \]
where $\1_{-1} := (1, \ldots, 1) \in S^{-1}$.
\end{lemma*}

\begin{proof} 
Take a transformation map $H^1$ that universally preserves the best response sets. Then by Lemma~\ref{br implies opp dependent}, its maps $h_{\ks}^1$ only depend on the strategy choices $\ks_{-1}$ of the opponents. Fix $\ks_{-1} \in S^{-1}$. Recall that the elements $r \geq 2$ and $\ps \in S^{-1}$ in \ref{app:relating distance maps} can be chosen arbitrarily\footnote{We required $p_r \neq 1$, but this is irrelevant for the argument we are making here.}. So we can apply \ref{app:relating distance maps} on a trivially true statement to get for all $z,w \in \R$:
\begin{align*}
&z-w \geq z-w \\
&\implies \forall r \in [N] \setminus \{1\}: \, \Delta h_{k_2, \ldots, k_{r-1}, k_r, 1, \ldots, 1}^1(z,w) \geq \Delta h_{k_2, \ldots, k_{r-1}, 1, 1, \ldots, 1}^1(z,w) \\
&\implies \Delta h_{k_2, \ldots, k_{N-1}, k_N}^1(z,w) \geq \Delta h_{k_2, \ldots, k_{N-1}, 1}^1(z,w) \geq \ldots \geq \Delta h_{1, \ldots, 1}^1(z,w) \, . 
\end{align*}
With skew-symmetry, we similarly obtain
\begin{align*}
&w-z \geq w-z \\
&\implies \forall r \in [N] \setminus \{1\}: \, \Delta h_{k_2, \ldots, k_{r-1}, k_r, 1, \ldots, 1}^1(w,z) \geq \Delta h_{k_2, \ldots, k_{r-1}, 1, 1, \ldots, 1}^1(w,z) \\
&\implies \Delta h_{k_2, \ldots, k_{N-1}, k_N}^1(w,z) \geq \Delta h_{k_2, \ldots, k_{N-1}, 1}^1(w,z) \geq \ldots \geq \Delta h_{1, \ldots, 1}^1(w,z) \\
\overset{\cdot \, (-1)}&{\implies} \Delta h_{k_2, \ldots, k_{N-1}, k_N}^1(z,w) \leq \Delta h_{1, \ldots, 1}^1(z,w) \, . 
\end{align*}
Putting both together, we have for all $z,w \in \R$:
\[ \Delta h_{\ks_{-1}}^1(z,w) = \Delta h_{k_2, \ldots, k_{N-1}, k_N}^1(z,w) = \Delta h_{1, \ldots, 1}^1(z,w) = \Delta h_{\1_{-1}}^1(z,w) \, . \]

\end{proof}



\begin{lemma*}
\label{app:strat specific maps behaviour}
Suppose transformation $H^1$ universally preserves best responses. Then we obtain for all $\ks_{-1} \in S^{-1}$ that
\begin{enumerate}
\item map $h_{\ks_{-1}}^1$ is strictly increasing, and that
\item map $h_{\ks_{-1}}^1$ distorts distances independently of their reference points:
\begin{align}
\label{app:distance distortion ind of point}
\forall z,z',\lambda \in \R \, : \, \Delta h_{\ks_{-1}}^1(z + \lambda, z) = \Delta h_{\ks_{-1}}^1(z' + \lambda, z') \, .
\end{align}
\end{enumerate}
\end{lemma*}

\begin{proof}
Take a transformation map $H^1$ that universally preserves the best response sets. Then by Lemma~\ref{br implies opp dependent}, its maps $h_{\ks}^1$ only depend on the strategy choices $\ks_{-1}$ of the opponents.

\paragraph{1} 
Let us first consider $h_{2,1, \ldots, 1}^1$ that is associated to the pure strategy profile $(2,1, \ldots, 1) \in S^{-1}$. Apply \ref{app:relating distance maps} in the upcoming line $(*)$ with parameters $r = 2$, $\ps_{-1} = (2,1, \ldots, 1)$, and $z' = w' \in \R$ to get for arbitrary $z,w \in \R$:
\begin{align*}
z \geq w &\iff z-w \geq 0 = z' - w' \\
\overset{(*)}&{\iff} \Delta h_{2,1, \ldots, 1}^1(z,w) \geq \Delta h_{\1_{-1}}^1(z',w') \overset{z' = w'}{=} 0 \\
&\iff h_{2,1, \ldots, 1}^1(z) \geq h_{2,1, \ldots, 1}^1(w) \, .
\end{align*}
Consequently, we have for arbitrary $\bar{z},\bar{w} \in \R$: 
\begin{align*}
\bar{z} > \bar{w} &\iff \bar{z} \geq \bar{w} \text{ and } \bar{w} \ngeq \bar{z} \\
\overset{\text{by above}}&{\iff} h_{2,1, \ldots, 1}^1(\bar{z}) \geq h_{2,1, \ldots, 1}^1(\bar{w}) \text{ and }  h_{2,1, \ldots, 1}^1(\bar{w}) \ngeq h_{2,1, \ldots, 1}^1(\bar{z}) \\
&\iff h_{2,1, \ldots, 1}^1(\bar{z}) > h_{2,1, \ldots, 1}^1(\bar{w}) \, .
\end{align*}
This shows that $h_{2,1, \ldots, 1}^1$ is strictly increasing.
\\
For arbitrary $\ks_{-1} \in S^{-1}$, we can then use Lemma~\ref{same distance distortion functions} to obtain
\begin{align*}
\bar{z} > \bar{w} &\iff h_{2,1, \ldots, 1}^1(\bar{z}) > h_{2,1, \ldots, 1}^1(\bar{w}) \\
&\iff \Delta h_{2,1, \ldots, 1}^1(\bar{z}, \bar{w}) > 0 \\
&\iff \Delta h_{\ks_{-1}}^1(\bar{z}, \bar{w}) = \Delta h_{\1_{-1}}^1(\bar{z}, \bar{w}) = \Delta h_{2,1, \ldots, 1}^1(\bar{z}, \bar{w}) > 0 \\
&\iff h_{\ks_{-1}}^1(\bar{z}) > h_{\ks_{-1}}^1(\bar{w}) \, .
\end{align*}
Thus, $h_{\ks_{-1}}^1$ is strictly increasing as well. 

\paragraph{2} 
Because of Lemma~\ref{same distance distortion functions}, we only need to show that the map $\Delta h_{\1_{-1}}^1$ satisfies property (\ref{app:distance distortion ind of point}), which would consequently imply the property for all maps $\Delta h_{\ks_{-1}}^1$.

Fix $z,z',\lambda \in \R$. Then the upcoming equivalence chain uses skew-symmetry (\ref{app:skew-symm}) in $(*)$, Lemma~\ref{same distance distortion functions} in $(\dagger)$, and \ref{app:relating distance maps} in $(\star)$ for parameters $r = 2$ and $\ps_{-1} = (2,1, \ldots, 1)$:
\begin{align*}
&\Delta h_{\1_{-1}}^1(z + \lambda, z) = \Delta h_{\1_{-1}}^1(z' + \lambda, z') 
\\
\overset{(*)}&{\iff} \Delta h_{\1_{-1}}^1(z + \lambda, z) \geq \Delta h_{\1_{-1}}^1(z' + \lambda, z') \\
&\, \quad \quad \quad \text{and} \, \, \Delta h_{\1_{-1}}^1(z, z + \lambda)\geq \Delta h_{\1_{-1}}^1(z', z' + \lambda) 
\\
\overset{(\dagger)}&{\iff} \Delta h_{2,\ldots, 1}^1(z + \lambda, z) \geq \Delta h_{\1_{-1}}^1(z' + \lambda, z') \\
&\, \quad \quad \quad \text{and} \, \, \Delta h_{2,\ldots, 1}^1(z, z + \lambda)\geq \Delta h_{\1_{-1}}^1(z', z' + \lambda) 
\\
\overset{(\star)}&{\iff} z + \lambda - z \geq z' + \lambda - z' \quad \text{and} \quad z - (z + \lambda) \geq z' - (z' + \lambda) \, .
\end{align*}
The last line is a true statement and thus, the first line as well. Because $z,z',\lambda \in \R$ were taken arbitrarily, map $h_{\1_{-1}}^1$ satisfies property (\ref{app:distance distortion ind of point}).
\end{proof}


\begin{thm*}
%\label{equiv charact of PAT}
Let $H = \{H^i\}_{i \in [N]}$ be a game transformation. Then:
\begin{align}
\label{app:mt1} \tag{i}
&\,H \text{ universally preserves \NE{} sets} \\
\label{app:mt2} \tag{ii}
&\iff \text{for each player~$i$, map $H^i$ universally preserves best responses} \\
\label{app:mt3} \tag{iii}
&\iff H \text{ is a positive affine transformation.}
\end{align}
\end{thm*}

\begin{proof} 
(\ref{app:mt3}) $\implies$ (\ref{app:mt1}): \\
By Lemma~\ref{multiplayer PAT preserves}. \\
\\
(\ref{app:mt1}) $\implies$ (\ref{app:mt2}): \\
By Proposition~\ref{two conclusions of strat equiv preserving}. \\
\\
(\ref{app:mt2}) $\implies$ (\ref{app:mt3}): \\
Consider map $H^1$ first, that is, the perspective of player 1. By Lemma~\ref{strat specific maps behaviour}, maps $h_{\ks_{-1}}^1$ are monotone functions and they satisfy the distance distortion property (\ref{distance preserving}). Hence, by Corollary~\ref{affine linearity corollary}, there exist parameters $a_{\ks_{-1}}^1, c_{\ks_{-1}}^1 \in \R$ for each $\ks_{-1} \in S^{-1}$ such that for all $z \in \R$:
\[h_{\ks_{-1}}^1(z) = a_{\ks_{-1}}^1 \cdot z + c_{\ks_{-1}}^1 \, .\]
Lemma~\ref{same distance distortion functions} implies $a_{\ks_{-1}}^1 = a_{\1_{-1}}^1$ for all $\ks_{-1} \in S^{-1}$. Therefore, we only have to keep track of one scaling parameter for all the maps within $H^1$; let us denote it with $\alpha^1$. With the first conclusion of Lemma~\ref{strat specific maps behaviour}, we obtain $\alpha^1 > 0$. Therefore, we have shown that $H^1$ is a positive affine transformation.

After Lemma~\ref{br implies opp dependent}, we mentioned that the same results as above can be analogously derived for the other player's transformation maps $H^2, \ldots, H^N$. There is only one difference for their analysis, in that for $H^i$, we don't index over $\ks_{-1} \in S^{-1}$ but over $\ks_{-i} \in S^{-i}$ instead.
\end{proof}


%It is key that $H^1$ preserves best responses \textit{universally}. Only due to that we are able to deduce global properties of $H^1$ in \ref{app:relating distance maps} and the upcoming Lemma~\ref{three implications}.

%\begin{lemma}
%\label{three implications}
%Suppose transformation $H^1$ universally preserves best responses. Then:
%\begin{enumerate}
%\item ``The (pure strategies)-specific maps within $H^1$ equally distort distances''
%\[\forall k_2 \in [m_2], \ldots, k_N \in [m_N] \, : \, \Delta h_{k_2, \ldots, k_N}^1 = \Delta h_{1, \ldots, 1}^1 \, . \]
%\item $\forall k_2 \in [m_2], \ldots, k_N \in [m_N]: $ \text{map} $h_{k_2, \ldots, k_N}^1$ \text{is strictly increasing.}
%\item ``Distances are distorted independently of their reference points''
%\\ $\forall k_2 \in [m_2], \ldots, k_N \in [m_N]$:
%\begin{align}
%\label{distance distortion ind of point}
%\forall z,z',\lambda \in \R \, : \, \Delta h_{k_2,\ldots, k_N}^1(z + \lambda, z) = \Delta h_{k_2,\ldots, k_N}^1(z' + \lambda, z') \, .
%\end{align}
%\end{enumerate}
%\end{lemma}

\end{document}