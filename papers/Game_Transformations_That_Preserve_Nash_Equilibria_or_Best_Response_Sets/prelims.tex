%\documentclass[main]{subfiles}

\begin{document}

\section{Preliminaries}
\label{sec:preliminaries}
%\paragraph{Notation} 
\noindent
For $n \in \N$, we denote $[n] := \{1,\ldots,n\}$ and the vector with all ones as $\1_n \in \R^n$.
%\todo{Any vector $x \in \R^n$ is a column vector and its transpose is denoted by $x^T$. The entries of $x \in \R^n$ will be $x_1, \ldots, x_n$. Inequalities like $x \geq 0$ abbreviate the statement $\forall i \in [n] \, : \, x_i \geq 0$. For $i \in [n]$, we will use $e_i \in \R^n$ to denote the $i$-th standard basis vector (with a $1$ in its $i$-th entry and $0$'s anywhere else).} 
 
%\\
%\todo{When working with sets $\{S^i\}_{i=1}^N$, we denote $\displaystyle \bigtimes_{i=1}^N S^i := S^1 \times \ldots \times S^N$.} \\


\subsection{Multiplayer Games} 
A multiplayer game $G$ specifies (a) the number of players $N \in \N, N \geq 2,$ (b) a set of pure strategies $S^i = [m_i]$ for each player~$i$ where $m_i \in \N, m_i \geq 2,$ and (c) the utility payoffs for each player~$i$ given as a function $u_i: S = S^1 \times \ldots \times S^N \longrightarrow \R$. Throughout this paper, all multiplayer games considered shall have the same number of players $N$ and the same strategy sets $S^1, \ldots, S^N$. Hence, any game $G$ will be determined by its utility functions $\{u_i\}_{i \in [N]}$. The players choose their strategies simultaneously and they cannot communicate with each other. A utility function $u_i$ can be summarized by its pure strategy outcomes for player~$i$, captured as an $N$-dimensional array $\big\{ u_i(\ks) \big\}_{\ks \in S}$.

\begin{ex}
$2$-player games are better known as bimatrix games because their $2$-dimensional payoff arrays in become matrices $A,B \in \R^{m \times n}$.
\end{ex}

As usual, we allow the players to randomize over their pure strategies. Then, player~$i$'s strategy space extends to the set of probability distributions over $S^i$. We identify this set with $\Delta(S^i) := \, \Big\{ s^i = (s_k^i)_k \, \in \R^{m_i} \, \Big| \, s_k^i \geq 0 \, \, \forall k \in [m_i] \, \, \text{and} \, \sum_{k \in [m_i]} s_k^i = 1 \Big\}$ and refer to the probability distributions as mixed strategies. A tuple $\strats = (s^1, \ldots, s^N) \in \Delta(S^1) \times \ldots \times \Delta(S^N) =: \Delta(S)$ of mixed strategies is called a strategy profile in $G$\footnote{Note that in our notation, $\Delta(S)$ is not a simplex of higher dimensions itself but only the product of $N$ simplices.}. The utility payoff of player~$i$ for the strategy profile $\strats$ is defined as the player's utility payoff in expectation 
\[ u_i(\strats) := \sum_{\ks \in S} s_{k_1}^1 \cdot \ldots \cdot s_{k_N}^N \cdot u_i(\ks) \, .\]
The goal of each player is to maximize her utility.


We will abbreviate with $S^{-i}$ the set that consists of all possible pure strategy choices $\ks_{-i} = (k_1, \ldots, k_{i-1},k_{i+1}, \ldots, k_N)$ of the opponent players (resp. $\Delta(S^{-i})$ for the set of mixed strategy choices $\strats^{-i} = (s^1, \ldots, s^{i-1},s^{i+1}, \ldots, s^N)$). We will also often use $u_i(k_i,\ks_{-i})$ instead of $u_i(\ks)$ to stress how player~$i$ can only influence her own strategy when it comes to her payoff (resp. $u_i(s^i,\strats^{-i})$ instead of $u_i(\strats)$).
\begin{defn}
The best response set of player~$i$ to the opponents' strategy choices $\strats^{-i}$ is defined as $\BR_{u_i}(\strats^{-i}) :=  \argmax_{t^i \in \Delta(S^i)} \big\{ \, u_i(t^i,\strats^{-i}) \, \big\}$. 
\end{defn} 
Best response strategies capture the idea of optimal play against the other player's strategy choices. The most popular equilibrium concept in non-cooperative games is based on best responses.
\begin{defn}
A strategy profile $\strats \in \Delta(S)$ to a game $G = \{u_i\}_{i \in [N]}$ is called a \NE{} if for all player~$i \in [N]$ we have $s^i \in \BR_{u_i}(\strats^{-i})$.
\end{defn}
\noindent
By \cite{Nash48}, any multiplayer game $G$ admits at least one \NE{}.

\subsection{Positive Affine Transformations} 

The following lemma is a well-known result for $2$-player games\footnote{ See \cite[Lemma 2.1]{heyman}, \cite[Theorem 5.35]{maschler_solan_zamir_2013}, \cite[Chapter 3]{harsanyi1988general} or \cite[Proposition 3.1]{DynGT}.}:
\begin{lemma}
\label{PAT preserves lemma}
Let $(A,B)$ be a $m \times n$ bimatrix game and take arbitrary $\alpha, \beta >0$ and $c \in \R^n, d \in \R^m$. Define $A' = \alpha A + \1_m c^T$ and $B' = \beta B + d \1_n^T$.

Then the game $(A', B')$ has the same best response sets as the game $(A,B)$. Consequently, both games have the same \NE{} set.
\end{lemma}
The intuition behind this lemma is as follows:
player~$1$ wants to maximize her utility given the strategy that player~$2$ chose. A positive rescaling of $u_1$ will change the utility payoffs but will not change the sets of best response strategies. The same holds true if we add utility payoffs to $u_1$ that are only dependent on the strategy choice $s^2$ of her opponent. In the notation of bimatrix games, this intuition yields that the transformation $A \mapsto \alpha A + \1_m c^T$ does not affect the best response sets of player~$1$. The analogous result holds for player~$2$ and the transformation $B \mapsto \beta B + d \1_n^T$.

Let us generalize PATs to multiplayer games.
\begin{defn}
\label{multiplayer PAT defn}
A positive affine transformation (PAT) specifies for each player~$i$ a scaling parameter $\alpha^i \in \R, \alpha^i >0,$ and translation constants $C^i := ( c_{\ks_{-i}})_{\ks_{-i} \in S^{-i}}$ for each choice of pure strategies from the opponents. 
The PAT $H_{\textnormal{PAT}} = \big\{ \alpha^i, C^i \big\}_{i \in [N]}$ applied to an input game $G = \{u_i\}_{i \in [N]}$ returns the transformed game $H_{\textnormal{PAT}}(G) = \{u_i'\}_{i \in [N]}$ in which (only) the utility functions changed to
\begin{align}
\label{PAT transformed utilities}
\begin{aligned}
u_i' : S &\longrightarrow \R \\
\ks &\longmapsto \alpha_i \cdot u_i(\ks) + c_{\ks_{-i}}^i \, .
\end{aligned}
\end{align}
\end{defn}
We could not find multiplayer PATs defined in the literature, so we came up with the natural generalization above. As shown in Section \ref{sec:bimatrix games}, they indeed generalize the 2-player PATs from Lemma~\ref{PAT preserves lemma} to multiplayer settings. Moreover, multiplayer PATs also preserve the best response sets and \NE{} set.
\begin{lemma}
\label{multiplayer PAT preserves}
Take a PAT $H_{\textnormal{PAT}} = \big\{ \alpha^i, C^i \big\}_{i \in [N]}$ and any game $G = \{u_i\}_{i \in [N]}$. Then, the transformed game $H_{\textnormal{PAT}}(G) = \{u_i'\}_{i \in [N]}$ has the same best response sets as the input game $G$. Consequently, $H_{\textnormal{PAT}}(G)$ also has the same \NE{} set as $G$.
\end{lemma}
\begin{proof}
See \ref{sec:helpinglemmas}.
\end{proof}
PATs have found much success as a tool for simplifying an input game precisely because of this property. We want to investigate which other game transformations also preserve the best response sets or the \NE{} set. If we found more of these transformations, we could use them to, e.g., further increase the class of efficiently solvable games.

\subsection{Game Transformations}

There are various ways in which we could define the concept of a game transformation. Section~\ref{literature review} gives an overview of some definitions from the literature that are useful for different purposes. A key component of PATs are that they operate player-wise and strategy-wise, that is, they do not change the player set nor the players' strategy sets. This allows for a direct comparison of the strategic structure between a game and its PAT-transform. We argue that this is a natural desideratum for a definition of more general game transformation.

\begin{defn}
\label{def game trafo}
A game transformation $H = \{H^i\}_{i \in [N]}$ specifies for each player~$i$ a collection of functions $H^i := \Big\{ h_{\ks}^i : \R \longrightarrow \R \Big\}_{\ks \in S}$, indexed by the different pure strategy profiles $\ks$. \\
The transformation $H$ can then be applied to any $N$-player game $G = \{u_i\}_{i \in [N]}$ to construct the transformed game $H(G) = \{H^i(u_i)\}_{i \in [N]}$ where 
\begin{align}
\label{transformed pure utilities evaluation}
    H^i(u_i) : S \to \R, \quad \ks \mapsto h_{\ks}^i \big( u_i(\ks) \big) \, .
\end{align}
\end{defn}
Observe that the utility payoff of player~$i$ in the transformed game $H(G)$ from the pure strategy outcome $\ks$ is only a function of the utility payoff from \textit{that same} player in \textit{that same} pure strategy outcome of the input game~$G$.

We extend the utility functions $H^i(u_i)$ to mixed strategy profiles $\strats \in \Delta(S)$ as usual through $H^i(u_i)(\strats) := \sum_{\ks \in S} s_{k_1}^1 \cdot \ldots \cdot s_{k_N}^N \cdot h_{\ks}^i \big( u_i(\ks) \big)$. To simplify future notation, we will often use $h_{k_i,\ks_{-i}}^i$ to refer to $h_{\ks}^i$.

\begin{rem}
A multiplayer positive affine transformation $H_{\textnormal{PAT}} = \big\{ \alpha^i, C^i \big\}_{i \in [N]}$ makes a game transformation $H = \{H^i\}_{i \in [N]}$ in the above sense by setting $h_{\ks}^i : \, \, \R \to \R$, $z \mapsto \alpha^i  \cdot z + c_{\ks_{-i}}^i$.
\end{rem}
\begin{defn}
\label{defn NE preserving}
Let $H = \{H^i\}_{i \in [N]}$ be a game transformation. Then we say that $H$ universally preserves \NE{} sets if for all input games $G = \{u_i\}_{i \in [N]}$, the transformed game $H(G) = \{H^i(u_i)\}_{i \in [N]}$ has the same \NE{} set as the input game $G$.
\end{defn}

\begin{defn}
\label{defn BR preserving}
Let map $H^i$ come from a game transformation $H$. Then we say that $H^i$ universally preserves best responses if for all utility functions $u_i : S \longrightarrow \R$ and for all opponents' strategy choices $\strats^{-i} \in \Delta(S^{-i})$:
\begin{equation*}
\BR_{H^i(u_i)}(\strats^{-i}) = \argmax_{t^i \in \Delta(S^i)} \big\{ H^i(u_i)(t^i,\strats^{-i}) \big\} = \argmax_{t^i \in \Delta(S^i)} \big\{ u_i(t^i,\strats^{-i}) \big\} = \BR_{u_i}(\strats^{-i}) \, .
\end{equation*}
\end{defn}

\begin{defn}
\label{defn opponent dependence}
Let map $H^i$ come from a game transformation $H$. Then we say that $H^i$ only depends on the strategy choice of the opponents if for all pure strategy choices $\ks_{-i} \in S^{-i}$ of the opponents, we have the map identities
    \[h_{1, \ks_{-i}}^i = \ldots = h_{m_i, \ks_{-i}}^i\,: \R \to \R \, .\]
\end{defn}

\end{document}