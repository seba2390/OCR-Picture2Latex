%\documentclass[main.tex]{subfiles}

\begin{document}

\section{Introduction}
\label{sec:introduction}

A classic tool in game theory is to transform some given game into another strategically equivalent game that is easier to handle \cite{10.2307/j.ctt1r2gkx}. Positive affine (linear) transformations (PATs) for normal-form games have been particularly useful in that regards \cite{Aumann1961AlmostSC, 10.1007/978-3-642-10841-9_44}. For a PAT example, take any two player game in which the players utility payoffs are measured in amounts of dollars. Then the best response strategies of player 1 do not change if her utility payoffs are multiplied by $5$. Moreover, they don't change if $10$ dollars are added to all outcomes that involve player $2$ playing his, say, third strategy. More generally, PATs have the power to rescale the utility payoffs of each player and to add constant terms to the utility payoffs for each player $i$ and strategy choice $k_{-i}$ of her opponents. PATs have been leveraged in the literature to significantly extend the applicability of efficient \NE{} solvers beyond the classes of zero-sum and rank-$1$ games\footnote{A $2$-player game, represented by its payoff matrices $A,B \in \R^{m \times n}$, is said to have rank-$1$ if $\rank(A+B) = 1$.} (see \cite{Neumann1928, Dantzig51, Adler2013, 2011, main} for the NE solvers and \cite{moulin, KONTOGIANNIS201264, heyman} for their extensions). The key to success of these extensions was the well-known property of PATs that they do not change the \NE{} set and best response sets when being applied to an arbitrary input game \cite{heyman, maschler_solan_zamir_2013, harsanyi1988general, DynGT}.

This paper addresses the question whether there are game transformations $H$ other than PATs with that same property. Such transformations are of key interest for two major reasons:
\begin{enumerate}
    \item They open up methods of generating (infinitely many) games that share key game-theoretic characteristics among themselves, and therefore,
    \item They also allow us to straight-forwardly extend any results we have about Nash equilibria and best response sets of normal-form games to any of their transformations.
\end{enumerate}

Throughout this paper, we restrict our attention to game transformations that transform utilities player-wise and strategy-wise.

\subsection{Our Contributions}
Our mathematical analysis starts with considering all those game transformations that preserve the \NE{} set when being applied to an arbitrary $N$-player game. Proposition~\ref{two conclusions of strat equiv preserving} shows that such game transformations must also preserve the best response sets when being applied to an arbitrary $N$-player game. We proceed to prove over a series of results that any best response preserving game transformation must (in particular) be a positive affine transformation (PAT). To close the proof cycle, we restate  in Lemma~\ref{multiplayer PAT preserves} the popular textbook result that PATs preserve the \NE{} set. Therefore, we have found two new equivalent definitions of PATs that point out their special status among game transformations: PATs are the only types of game transformations that always preserve the \NE{} set (or, respectively, the best response sets). This main result of our paper is summarized in Theorem~\ref{equiv charact of PAT}. Any proofs can be found in the appendix.

\paragraph{Setup} 
Let $N \in \N, N \geq 2,$ be the number of players and $S^i = \{1, \ldots, m_i\} =: [m_i]$ be the strategy sets for all player~$i$, where $m_i \in \N, m_i \geq 2$. These parameters shall be fixed for the rest of this paper. Denote all strategy profiles as $S := S^1 \times \ldots \times S^N$. A game $G$ is then determined by the utility functions $u_i: S \longrightarrow \R$ for each player~$i$; even after we allow the players to randomize over their pure strategies.

A \textit{positive affine transformation} $H_{\textnormal{PAT}}$ specifies a scaling parameter $\alpha^i \in \R$, $\alpha^i >0,$ for each player~$i$, as well as a translation constant $c_{\ks_{-i}}^i$ for each player~$i$ and each opponent's strategy choice $\ks_{-i} = (k_j \in S^j)_{j \neq i}$. The transformation $H_{\textnormal{PAT}}$ then takes a game $G = \{u_i\}_{i \in [N]}$ as an input and returns the transformed game $H_{\textnormal{PAT}}(G) = \{u_i'\}_{i \in [N]}$ where
\begin{align*}
u_i' : S &\longrightarrow \R \\
\ks &\longmapsto \alpha_i \cdot u_i(\ks) + c_{\ks_{-i}}^i \, . \,
\end{align*}
A \textit{game transformation} $H = \{H^i\}_{i \in [N]}$, on the other hand, specifies for each player~$i$ a collection of functions $H^i := \big\{ h_{\ks}^i : \R \longrightarrow \R \big\}_{\ks \in S}$ that are indexed by the different pure strategies $\ks = (k_1, \ldots, k_N) \in S$. The game transformation $H$ then takes a game $G = \{u_i\}_{i \in [N]}$ as an input and returns the transformed game $H(G) = \{H^i(u_i)\}_{i \in [N]}$ where 
\begin{align*}
H^i(u_i) : S &\longrightarrow \R \\
\ks &\longmapsto h_{\ks}^i \big( u_i(\ks) \big) \, .
\end{align*}
Intuitively speaking, $H$ operates player-wise and strategy-wise through its maps $h_{\ks}^i$.

\paragraph{Main Result} 
Abstractly speaking, we say that a transformation $H$ \textit{universally preserves} some special set $\mathcal{Z}$ if applying transformation $H$ to an arbitrary input $G$ does not change $G$'s special set, that is, $\mathcal{Z}_{H(G)} = \mathcal{Z}_G$. Definitions~\ref{defn NE preserving} and~\ref{defn BR preserving} make this concept precise for $\mathcal{Z}=$ \{\NEs{}\} and $\mathcal{Z}=$ \{Best responses\}. 

\begin{thm*}[Main Result, Restatement of Theorem~\ref{equiv charact of PAT}] 
\, \\
Let $H = \{H^i\}_{i \in [N]}$ be a game transformation. Then:
\begin{align*}
&\, H \text{ universally preserves the \NE{} set} \\
&\iff \text{for each player~$i$, map $H^i$ universally preserves best responses} \\
&\iff H \text{ is a positive affine transformation.}
\end{align*}
\end{thm*}

Therefore, if we intend to utilize game transformations to generate games with the same \NE{} set or the same best responses as some input game $G$, we must do at least one of the following: 
\begin{enumerate}
    \item rely on positive affine transformations only,
    \item take advantage of properties that are specific to the input game $G$, or
    \item consider other notions of transforming a game altogether.
\end{enumerate}

\subsection{Illustration of Game Transformations on Bimatrix Games}
\label{sec:bimatrix games}
In the case of two-player games ($N = 2$), a normal-form game can be represented by its pair of payoff matrices $(A,B) =  \Big( (a_{ij})_{ij}, (b_{ij})_{ij} \Big) \in \R^{m \times n} \times \R^{m \times n}$. A PAT of bimatrix games is then determined by positive scalars $\alpha \, := \, \alpha_1 >0$ and $\beta \, := \, \alpha_2 > 0$, and translation vectors $c \, := \, (c_{k_2}^1)_{k_2 \in [n]} \in \R^n$ and $d \, := \, (c_{k_1}^2)_{k_1 \in [m]} \in \R^m$. It transforms the game $(A,B)$ into the game
\[H_{\textnormal{PAT}}(A,B) = \Big( ( \alpha \cdot a_{ij} + c_j^1)_{ij}, ( \beta \cdot b_{ij} + c_i^2)_{ij} \Big) = ( \alpha A + \1_m c^T, \beta B + d \1_n^T ) \, .\]
A general game transformation $H$, on the other hand, specifies $H^1, H^2 : \R^{m \times n} \longrightarrow  \R^{m \times n}$ as matrices of single-variable functions $(h_{ij}^1)_{ij}$ and $(h_{ij}^2)_{ij}$. Transforming a game $(A,B)$ through $H$ yields the bimatrix game
\begin{align}
\label{bimatrix game trafo}
H(A,B) = \Big( \big( h_{ij}^1(a_{ij}) \big)_{ij}, \big( h_{ij}^2(b_{ij}) \big)_{ij} \Big) \, .
\end{align}

\begin{ex}
Consider the transformation $H_{\textnormal{Ex}}$ that takes a $2 \times 2$ bimatrix game $(A,B)$ as an input and returns the transformed game
\[A' = \begin{pmatrix}
-3 \cdot a_{11} +10  & a_{12}^5\\
e^{a_{21}} & 0
\end{pmatrix}
 \quad \text{and} \quad
B'= \begin{pmatrix}
|b_{11}| & \sign (b_{12})\\
\sqrt{|b_{22}|} & \arctan (b_{21})
\end{pmatrix} \, . \]
Then $H_{\textnormal{Ex}}$ is a game transformation\footnote{Note that $H_{\textnormal{Ex}}$ is not even continuous because of the sign function in $B'$. Nonetheless, our definition still considers it to be a game transformation.} that is not positive affine. Therefore, by our main result, there exists a $2 \times 2$ bimatrix game $(A,B)$ that has a different \NE{} set and different best response sets than its transformed game $H_{\textnormal{Ex}}(A,B)$.
\end{ex}

\subsection{Literature Review}
\label{literature review}
\paragraph{Strategic Similarity}
Much work has gone into identifying when two games are strategically similar/identical.

Strategic similarity, for example, is an important aspect of \textit{Potential Games} \cite{MONDERER1996124}. Morris and Ui \cite{MORRIS2004260} noted that a game $G$ is a weighted potential game if and only if it is the PAT transformation of an identical interest game\footnote{Definition of an identical interest game: Given an action profile $s$, each player receives the same utility from $s$.}. This characterization has been used to analyze the \NEs{} and solvers of potential games. The main contribution of Morris and Ui \cite{MORRIS2004260}, however, was to characterize when two given input games are \textit{best-response equivalent}, \textit{better-response equivalent} or \textit{von Neumann-Morgenstern equivalent}. Best-response equivalency essentially means that the two input games have the same best response sets
%Better-response equivalency requires that each player's induced preferences over her pure strategies\footnote{Given the mixed strategy choices of her opponents.} are the same in both input games. 
and von Neumann-Morgenstern equivalency requires that the input games only differ by a PAT.
%, both of which are also of interest in this work. 
An attempt from our side to prove (\ref{mt2}) $\implies$ (\ref{mt3}) in Theorem~\ref{equiv charact of PAT} based on their characterization for best-response equivalency \cite[Prop. 4-5, Cor. 6]{MORRIS2004260} failed because their characterization only holds for input games that satisfy specific properties. One of them (property G3) requires knowledge of the player's preferences over non-best-response strategies; knowledge that we do not have in our analysis.

Hammond \cite{hammond} described that the strategic decision-making in a game with mixed strategies does not depend on the player's numerical utility values, but solely on the preferences that the utility functions induce. This is based on the observation that the utility functions $\{u_i\}_{i \in [N]}$ of a game reveal the preferences of each player over her strategy set, given the strategy choices of her opponents. \ref{sec:utility_theory} gives some further background in utility theory that may help put Hammond's work into our context. Using the Expected Utility Theorem \cite[Prop 6.B.2-3]{MasCol}, he deduced that utility functions that induce the same preferences can only differ up to a positive affine transformation. From our perspective, Hammond therefore showed that PATs are the only game transformations that preserve the underlying preferences of each player. Note, however, that the property of preserving the player's preferences is - a priori - much harder to satisfy than preserving best responses (and, hence, Nash equilibria). Thus, our Theorem \ref{equiv charact of PAT} generalizes his result to broader questions of interest.
    
Du \cite{10.1016/j.tcs.2008.07.021} proved that it is $\mathbf{NP}$-hard to decide whether two input games share a common \NE{}, and that it is $\mathbf{co\text{-}NP}$-hard to decide whether two input games have the same \NE{} set.
    
On the more broader related work, Gabarró et al. \cite{GABARRO20116675, gabarro_garcia_serna_2013} gave several complexity-theoretic results to the problem of deciding whether two pure strategy games are \textit{isomorphic} w.r.t. a notion of game transformation that can help us understand the symmetries within a game \cite[Chapter 3]{harsanyi1988general}. McKinsey \cite{McKinsey} and Chang and Tijs \cite{CoopGamesIsomorphism} studied two notions of game equivalency specific to the research area of cooperative games.

\paragraph{Game Transformations} 
Another related line of research consists of work that utilizes different notions of transforming a game while preserving its ``strategic structure''. These papers work with similar notions of game transformations as we do, but their purpose and conclusions are non-comparable to ours.

Abbott et al. \cite{Abbott05onthe} and Brandt et al. \cite{10.5555/1597538.1597636} obtained complexity-theoretic results for computing \NEs{} and other solution concepts. Their methods are based on so-called \textit{Nash homomorphisms}. A Nash homomorphism resembles a reduction in the complexity-theoretic sense\footnote{Namely, ``a mapping~$h$ from a set of games $\mathcal{A}$ into a set of games $\mathcal{B}$, such that there exists a polynomial-time computable function $f$ that, when given a game $\Gamma$ and an \NE{} of $h(\Gamma)$, returns an \NE{} of $\Gamma$.''}. PATs would be a simple example of a Nash homomorphism. Note, however, that their complexity analysis requires their game transformations to always be efficiently computable.
    
Pottier and Nessah \cite{BergeVaismanEq} worked with a notion of a general game transformation that is close to ours, but their study of interest are game transformations that convert the Berge-Vaisman equilibria of an input game to the \NEs{} of the transformed game. 

Other transformations that preserve strategic features were also studied by Wu et al. \cite{9585431} for Bayesian games and in \cite{RM-759-PR, 10.2307/1912320, ELMES19941, CASAJUS2003267} for extensive form games.
\end{document}