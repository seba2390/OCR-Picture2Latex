%\documentclass[main.tex]{subfiles}

\begin{document}

\section{Conclusion}
First, we introduced game transformations and positive affine transformations (PATs) for multiplayer games. Next, we defined the properties (i) \textit{universally preserving \NE{} sets} and (ii) \textit{universally preserving best responses}. It is well-known that PATs universally preserve \NE{} sets. We showed that game transformations which universally preserve \NE{} sets also universally preserve  best responses. In the subsequent results, we derived further that if a game transformation universally preserves best responses then it is a positive affine transformation. Therefore, we gave two equivalent characterisations for a game transformation to be a PAT. They highlight how special PATs are with regard to \NE{} sets and best response sets.

\paragraph{Implications for future work}
The current literature is lacking in methods for generating strategically equivalent games to a given input game. Finding such methods would greatly benefit the field of game theory. 

Among the game transformations of our definition, and aside from PATs, such methods can only exist if we restrict our focus to smaller classes of $N$-player games (i.e., games of some specific characteristic). Preferably, such a utilized method would be computationally efficient, and the corresponding class of games would contain "most" games. 

Alternatively, we could relax the notion of a game transformation. Future work may, for example, examine game transformations that can freely change the strategy sets of the players or the number of players. Note however, that in that case, it is not straightforward anymore what it means for two games to be strategically equivalent.

\end{document}