\documentclass[a4paper]{amsart}

\usepackage{etex}
\usepackage{aliascnt}
\usepackage{bbm} 
\usepackage{pbox}
\usepackage{booktabs}
\usepackage{arydshln}
\usepackage{pinlabel, caption, subcaption}
\usepackage{esvect}
\usepackage{cals}
\usepackage{verbatim}
\setcounter{tocdepth}{2}

\newcommand{\bk}{\mathbbm{k}}
\newcommand{\KK}{K}
\DeclareMathOperator{\Mod}{Mod}
\DeclareMathOperator{\Tor}{Tor}
\DeclareMathOperator{\Span}{span}
\DeclareMathOperator{\VB}{VB} 
\DeclareMathOperator{\Ch}{Ch}
\DeclareMathOperator{\Sp}{Sp}
\DeclareMathOperator{\Ab}{Ab}
\DeclareMathOperator{\CB}{CB}

\DeclareMathOperator{\Top}{Top}
\DeclareMathOperator{\Set}{Set}
\DeclareMathOperator{\Fun}{Fun}
\DeclareMathOperator{\Ind}{Ind}
\newcommand{\simga}{\sigma}
\newcommand{\OO}{{\mathcal O}}
\newcommand{\OK}{{\mathcal O}_\KK}
\newcommand{\m}{\to}
\newcommand{\cA}{\mathcal{A}}\newcommand{\cB}{\mathcal{B}}
\newcommand{\cC}{\mathcal{C}}\newcommand{\cD}{\mathcal{D}}
\newcommand{\cE}{\mathcal{E}}\newcommand{\cF}{\mathcal{F}}
\newcommand{\cG}{\mathcal{G}}\newcommand{\cH}{\mathcal{H}}
\newcommand{\cI}{\mathcal{I}}\newcommand{\cJ}{\mathcal{J}}
\newcommand{\cK}{\mathcal{K}}\newcommand{\cL}{\mathcal{L}}
\newcommand{\cM}{\mathcal{M}}\newcommand{\cN}{\mathcal{N}}
\newcommand{\cO}{\mathcal{O}}\newcommand{\cP}{\mathcal{P}}
\newcommand{\cQ}{\mathcal{Q}}\newcommand{\cR}{\mathcal{R}}
\newcommand{\cS}{\mathcal{S}}\newcommand{\cT}{\mathcal{T}}
\newcommand{\cU}{\mathcal{U}}\newcommand{\cV}{\mathcal{V}}
\newcommand{\cW}{\mathcal{W}}\newcommand{\cX}{\mathcal{X}}
\newcommand{\cY}{\mathcal{Y}}\newcommand{\cZ}{\mathcal{Z}}

\providecommand{\Linkh}{\widehat{\ensuremath\mr{Link}}}
\providecommand{\Link}{\ensuremath\mr{Link}}
\newcommand{\hooklongrightarrow}{\lhook\joinrel\longrightarrow}
\newcommand{\vecq}[1]{{\vec{#1}^{\,\tiny \mathsf{Q}}}}
\newcommand{\vecs}[1]{{\vec{#1}^{\,\tiny S}}}

\newcommand{\gA}{\bold{A}}
\newcommand{\gB}{\bold{B}}
\newcommand{\gC}{\bold{C}}
\newcommand{\gD}{\bold{D}}
\newcommand{\gE}{\bold{E}}
\newcommand{\gF}{\bold{F}}
\newcommand{\gG}{\bold{G}}
\newcommand{\gH}{\bold{H}}
\newcommand{\gI}{\bold{I}}
\newcommand{\gJ}{\bold{J}}
\newcommand{\gK}{\bold{K}}
\newcommand{\gL}{\bold{L}}
\newcommand{\gM}{\bold{M}}
\newcommand{\gN}{\bold{N}}
\newcommand{\gO}{\bold{O}}
\newcommand{\gP}{\bold{P}}
\newcommand{\gQ}{\bold{Q}}
\newcommand{\gR}{\bold{R}}
\newcommand{\gS}{\bold{S}}
\newcommand{\gT}{\bold{T}}
\newcommand{\gU}{\bold{U}}
\newcommand{\gV}{\bold{V}}
\newcommand{\gW}{\bold{W}}
\newcommand{\gX}{\bold{X}}
\newcommand{\gY}{\bold{Y}}
\newcommand{\gZ}{\bold{Z}}

\newcommand{\bA}{\mathbb{A}}\newcommand{\bB}{\mathbb{B}}
\newcommand{\bC}{\mathbb{C}}\newcommand{\bD}{\mathbb{D}}
\newcommand{\bE}{\mathbb{E}}\newcommand{\bF}{\mathbb{F}}
\newcommand{\bG}{\mathbb{G}}\newcommand{\bH}{\mathbb{H}}
\newcommand{\bI}{\mathbb{I}}\newcommand{\bJ}{\mathbb{J}}
\newcommand{\bK}{\mathbb{K}}\newcommand{\bL}{\mathbb{L}}
\newcommand{\bM}{\mathbb{M}}\newcommand{\bN}{\mathbb{N}}
\newcommand{\bO}{\mathbb{O}}\newcommand{\bP}{\mathbb{P}}
\newcommand{\bQ}{\mathbb{Q}}\newcommand{\bR}{\mathbb{R}}
\newcommand{\bS}{\mathbb{S}}\newcommand{\bT}{\mathbb{T}}
\newcommand{\bU}{\mathbb{U}}\newcommand{\bV}{\mathbb{V}}
\newcommand{\bW}{\mathbb{W}}\newcommand{\bX}{\mathbb{X}}
\newcommand{\bY}{\mathbb{Y}}\newcommand{\bZ}{\mathbb{Z}}
\newcommand{\Z}{\mathbb{Z}}
\newcommand{\N}{\mathbb{N}_0}
\newcommand{\Q}{\mathbb{Q}}
\newcommand{\C}{\mathbb{C}}
\newcommand{\R}{\mathbb{R}}

\usepackage{lmodern}
\usepackage{booktabs}
\usepackage[dvipsnames,svgnames,x11names,hyperref]{xcolor}
\usepackage{mathtools,url,graphicx,verbatim,amssymb,enumerate,stmaryrd,nicefrac}
\usepackage[pagebackref,colorlinks,citecolor=blue,linkcolor=blue,urlcolor=blue,filecolor=blue]{hyperref}
\usepackage{microtype}
\usepackage[margin=1.25in]{geometry}
\usepackage{pdflscape}
%\usepackage{setspace}

\usepackage{tikz, tikz-cd}
\usetikzlibrary{matrix,arrows,decorations.pathmorphing}
\usetikzlibrary{arrows,decorations.pathmorphing,backgrounds,fit,positioning,shapes.symbols,chains,calc}



\newcounter{theoremcounter}
\numberwithin{theoremcounter}{section}
\newaliascnt{theoremauto}{theoremcounter}
\newcommand{\theoremautoautorefname}{Theorem}
\newcommand{\atheoremautorefname}{Theorem}
\newaliascnt{Defauto}{theoremcounter}
\newcommand{\Defautoautorefname}{Definition}
\newaliascnt{exampleauto}{theoremcounter}
\newcommand{\exampleautoautorefname}{Example}
\newaliascnt{lemmaauto}{theoremcounter}
\newcommand{\lemmaautoautorefname}{Lemma}
\newaliascnt{propositionauto}{theoremcounter}
\newcommand{\propositionautoautorefname}{Proposition}
\newaliascnt{corollaryauto}{theoremcounter}
\newcommand{\corollaryautoautorefname}{Corollary}
\newaliascnt{remarkauto}{theoremcounter}
\newcommand{\remarkautoautorefname}{Remark}
\newaliascnt{notationauto}{theoremcounter}
\newcommand{\notationautoautorefname}{Notation}
\newaliascnt{claimauto}{theoremcounter}
\newcommand{\claimautoautorefname}{Claim}
\newaliascnt{warningauto}{theoremcounter}
\newcommand{\warningautoautorefname}{Warning}
\newaliascnt{questionauto}{theoremcounter}
\newcommand{\questionautoautorefname}{Question}
\newaliascnt{discussionauto}{theoremcounter}
\newcommand{\discussionautoautorefname}{Discussion}
\newaliascnt{computationauto}{theoremcounter}
\newcommand{\computationautoautorefname}{Computation}
\newaliascnt{conjectureauto}{theoremcounter}
\newcommand{\conjectureautoautorefname}{Conjecture}
\newaliascnt{convauto}{theoremcounter}
\newcommand{\convautoautorefname}{Convention}
\renewcommand{\sectionautorefname}{Section}
\renewcommand{\subsectionautorefname}{Section}
\renewcommand{\itemautorefname}{Condition}

\newcounter{atheoremcounter}

\newtheorem{theorem}[theoremauto]{Theorem}
\newtheorem{lemma}[lemmaauto]{Lemma}
\newtheorem{proposition}[propositionauto]{Proposition}
\newtheorem{corollary}[corollaryauto]{Corollary}
\newtheorem*{corollary*}{Corollary}
\newtheorem{conjecture}[conjectureauto]{Conjecture}
\newtheorem{atheorem}{Theorem}
\renewcommand{\theatheorem}{\Alph{atheorem}}
\newtheorem{acorollary}[atheorem]{Corollary}
\renewcommand{\theacorollary}{\Alph{acorollary}}
\theoremstyle{definition}
\newtheorem{definition}[Defauto]{Definition}
\newtheorem{notation}[notationauto]{Notation}
\newtheorem{convention}[convauto]{Convention}
\theoremstyle{remark}
\newtheorem{example}[exampleauto]{Example}
\newtheorem{remark}[remarkauto]{Remark}
\newtheorem{claim}[claimauto]{Claim}
\newtheorem{warning}[warningauto]{Warning}
\newtheorem{question}[questionauto]{Question}
\newtheorem{discussion}[discussionauto]{Discussion}
\newtheorem{computation}[computationauto]{Computation}



\newcommand{\half}{\nicefrac{1}{2}}
\newcommand{\cat}[1]{\mathsf{#1}}
\newcommand{\mr}[1]{{\rm #1}}
\newcommand{\mon}[1]{\mathbf{#1}}
\newcommand{\ul}[1]{\underline{#1}}
\newcommand{\fS}{\mathfrak{S}}
\newcommand{\aut}{\mr{Aut}}
\newcommand{\diff}{\mr{Diff}}
\newcommand{\pl}{\mr{PL}}
\newcommand{\topo}{\mr{Top}}
\newcommand{\emb}{\mr{Emb}}
\newcommand{\rank}{\mr{rank}}
\newcommand{\Star}{\mr{Star}}
\newcommand{\lra}{\longrightarrow}
\newcommand{\rel}{\,\mr{rel}\,}
\newcommand{\relo}{\,\mr{rel}^0\,}
\newcommand{\relp}{\,\mr{rel}^+\,}
%\newcommand{\Sp}{\mathbf{Sp}}
%\newcommand{\OO}{\mathbf{O}}
\newcommand{\SO}{\mathbf{SO}}
\newcommand{\GG}{\mathbf{G}}

\DeclareMathOperator*{\hocolim}{hocolim}
\DeclareMathOperator*{\colim}{colim}
\DeclareMathOperator*{\tohofib}{tohofib}
\DeclareMathOperator*{\sgn}{sgn}
\DeclareMathOperator{\V}{Vert}
\newcommand{\rk}{\mr{rk}}
\newcommand{\GL}{\mr{GL}}
\newcommand{\SL}{\mr{SL}}
\newcommand{\St}{\mr{St}}

\newcommand{\Lk}{\mr{Link}}

\setlength\marginparwidth{1in}
\newcommand{\purple}[1]{\textcolor{purple}{#1}}
\newcommand{\orw}[1]{\marginpar{\tiny\textcolor{red}{orw: #1}}}
\newcommand{\ak}[1]{\marginpar{\tiny\textcolor{blue}{ak: #1}}}
\newcommand{\jm}[1]{\marginpar{\tiny\textcolor{teal}{jm: #1}}}
\newcommand{\jw}[1]{\marginpar{\tiny\textcolor{violet}{jw: #1}}}
\newcommand{\pp}[1]{\marginpar{\tiny\textcolor{olive}{pp: #1}}}

\renewcommand\labelitemi{{\boldmath$\cdot$}}

\setcounter{tocdepth}{2}

\title[On  rank filtrations of algebraic $K$-theory and Steinberg modules]{On  rank filtrations of algebraic $K$-theory\\ and Steinberg modules}




 
 \author{Jeremy Miller}\thanks{Jeremy Miller was supported in part by NSF Grant DMS-2202943 and a Simons Foundation Collaboration Grant}
 \email{jeremykmiller@purdue.edu}  
\address{Purdue University \\
Department of Mathematics \\
 	 150 N. University \\
 	 West Lafayette IN, 47907 \\USA}  

  
  
\author{Peter Patzt}
\email{ppatzt@ou.edu}
\address{University of Oklahoma\\
Department of Mathematics \\
 	 601 Elm Av \\
 	 Norman OK, 73019 \\USA}
	 \thanks{Peter Patzt was supported by the Danish National Research Foundation through the Copenhagen Centre for Geometry and Topology (DNRF151) and a Simons Foundation Collaboration Grant}
   
\author{Jennifer C. H. Wilson}
\email{jchw@umich.edu}
\address{University of Michigan \\ Department of Mathematics \\
 	 530 Church St\\
 	 Ann Arbor MI, 48109 \\USA}
\thanks{Jennifer Wilson was supported in part by NSF grant DMS-1906123 and NSF CAREER grant DMS-2142709}


\renewcommand{\shortauthors}{Miller, Patzt, and Wilson}

%Don't display MR-number in bibliography
\AtBeginDocument{%
	\def\MR#1{}}

\date{\today}

\begin{document}
	
\begin{abstract} Motivated by his work on the stable rank filtration of algebraic $K$-theory spectra, Rognes defined a simplicial complex called the common basis complex and conjectured that this complex is highly connected for local rings and Euclidean domains. We prove this conjecture in the case of fields. Our methods give a novel description of this common basis complex of a PID as an iterated bar construction on an equivariant monoid built out of Tits buildings. We also identify the Koszul dual of a certain equivariant ring assembled out of Steinberg modules.

 \end{abstract}

\maketitle



\tableofcontents

%\vspace{-.5cm}

\section{Introduction}




\subsection{The stable rank filtration and Rognes' connectivity conjecture} 



Given a PID $R$, let ${\CB}_n(R)$ denote Rognes' common basis complex. That is, ${\CB}_n(R)$ is the simplicial complex with vertices the proper nonzero summands of $R^n$ such that $\{V_0,\dots, V_p\}$ forms a simplex if there is a basis for $R^n$ such that each $V_i$ is a span of a subset of that basis. See \autoref{Figure-TvsD}.
\begin{figure}[h!]
\includegraphics[scale=3]{apartment-text} 
\caption{Let $R^n = L_1 \oplus L_2 \oplus \dots \oplus L_n$ be a decomposition into rank-1 summands. Consider the $(2^n-2)$ submodules of $R^n$ given by sums of proper nonzero subsets of these $n$ summands. The full subcomplex of the Tits building on the corresponding vertices is homeomorphic to an $(n-2)$-sphere (pictured for $n=3$). In contrast, in ${\CB}_n(R)$ these vertices span a $(2^n-3)$-dimensional simplex.} 
\label{Figure-TvsD}
\end{figure} 

One of the most popular models for algebraic $K$-theory is Waldhausen's iterated $S_\bullet$-construction \cite{WaldhausenSource}.  Filtering the Waldhausen $S_\bullet$-construction by rank yields a filtration of the algebraic $K$-theory spectrum: 
\[*=\mathcal{F}_0K(R) \subset \mathcal{F}_1K(R) \subset \dots \subset K(R).\]
Rognes  \cite{Rog1} proved that the associated graded of this filtration is the general linear group homotopy orbits of the suspension spectrum of the suspension  of the common basis complex: \[\mathcal{F}_{n}K(F)/\mathcal{F}_{n-1}K(F) \simeq \left(\Sigma^{\infty}\Sigma {\CB}_n(R) \right)_{h\GL_n(R)}.\] This stable rank filtration was a key ingredient in Rognes' proof that $K_4(\Z) \cong 0$ \cite{RognesK4}. Based on calculations for $n=2$ and $3$, Rognes \cite[Conjecture 12.3]{Rog1} made the following connectivity conjecture.

\begin{conjecture}[Rognes' connectivity conjecture]
For $R$ a local ring or Euclidean domain, ${\CB}_n(R)$ is $(2n-4)$-connected.
\end{conjecture}

We resolve this conjecture for $R=F$ a field.

\begin{theorem}\label{ConnectivityThm}
For $F$ a field, $ {\CB}_n(F)$ is $(2n-4)$-connected. 
\end{theorem}

Rognes  \cite{Rog1} proved that  $\widetilde{H}_i({\CB}_n(R))$ vanishes for $i>2n-3$. It follows that the reduced homology of ${\CB}_n(F)$ is concentrated in degree $2n-3$. Rognes named the conjectural single non-vanishing reduced homology group the \emph{stable Steinberg module} $\St^{\infty}_n(R)$ \cite[Definition 11.3]{Rognes96}. \autoref{ConnectivityThm} shows that the homology of the associated graded of the stable rank filtration of the $K$-theory of a field is the homology of $\GL_n(F)$ with coefficients in the stable Steinberg module:
$$H_i\big(\mathcal{F}_n K(F)/ \mathcal{F}_{n-1}K(F)\big) \cong H_{i-2n+2}\big(\GL_n(F); \St^{\infty}_n(F)\big).$$
In particular this result implies a vanishing line on the $E^1$ page of the spectral sequence associated to the stable rank filtration. 

We note that in the case that $F$ is an infinite field, Galatius--Kupers--Randal-Williams recently proved that $(\Sigma^{\infty} \Sigma {\CB}_n(F))_{h\GL_n(F)}$ is $(2n-3)$-connected \cite[Theorem C]{e2cellsIV}.  Their results imply the same vanishing line on the $E^1$ page of the spectral sequence associated to the stable rank filtration in the case of infinite fields. 

\subsection{Higher Tits buildings} 

The rank filtration not only gives a filtration of the $K$-theory spectrum but also gives a filtration of each of the spaces in the Waldhausen model of the K-theory  spectrum. Rognes proved that the associated graded of the rank filtration of the $k$th space is the reduced homotopy orbits of a $\GL_n(R)$ action on a space $D^k_n(R)$.\footnote{Rognes' notation differs from our notation having subscripts and superscripts flipped.} This space can be thought of as a $k$-dimensional version of the Tits building. The definition of $D^k_n(R)$ is somewhat involved (see \autoref{Dab}, following Rognes \cite[Definition 3.9]{Rog1}) so we will only describe a model of its desuspension here. Recall that the Tits building $T_n(R)$ is the realization of the poset of proper nontrivial summands of $R^n$ ordered by inclusion. Let $T^k_n(R)$ denote the subcomplex of the $k$-fold join \[\underbrace{T_n(R) * \dots * T_n(R)}_{k \text{ times}} \] of simplices that admit a common basis.  There is an equivalence $D_n^k(R) \simeq \Sigma^{k+1} T_n^k(R)$. 

The complexes $\{D^k_n(R)\}_n$ assemble to form a kind of equivariant monoid. Let $\GL(R)$ denote the groupoid of general linear groups $\{\GL_n(R)\}_n$ viewed as a symmetric monoidal category with block sum. 
The category $\Fun(\GL(R),{\Top_*}  )$ of functors to based spaces has  the structure of a symmetric monoidal category as well. The monoidal operation is given by Day convolution (\autoref{Day}). 
 Let $D^k(R)$ denote the functor $n \mapsto D^k_n(R)$. Galatius--Kupers--Randal-Williams \cite{e2cellsIV} observed that  $D^k(R)$ has the structure of an augmented graded-commutative monoid object in this category. Concretely, the multiplication map is the data of $(\GL_n(R) \times \GL_m(R))$-equivariant maps: \[D^k_n(R) \wedge D^k_m(R) \m D^k_{n+m}(R). \] It satisfies an equivariant version of the associativity and commutativity axioms. This augmented monoid structure allows us to make sense of bar constructions. Our main technical result is the following. 

\begin{lemma} \label{mainLemma} For $R$ a PID and $k \geq 1$, there is an equivalence $B D^k(R) \simeq D^{k+1}(R)$.
\end{lemma}

For $F$ a field, it is easy to determine the connectivity of $D^2_n(F)$. In fact, $T_n^2(F)$ is simply the join $T_n(F) * T_n(F)$. Since bar constructions increase connectivity by one and \[{\CB}_n(R) \simeq \colim_{k \m \infty} T_n^k(R),\] \autoref{ConnectivityThm} follows quickly from \autoref{mainLemma}.


\subsection{Steinberg modules} 

Recall that the Steinberg module is defined to be \[\St_n(R):=\widetilde H_{n-2}(T_n(R)) \cong H_n(D_n^1(R)),\] for $n>0$, and by convention $\St_0(R)=\Z$. This is an important object in representation theory, algebraic $K$-theory, and the cohomology of arithmetic groups. The assignment $n \mapsto \St_n(R)$ assembles to a functor $\St(R)$ from $\GL(R)$ to abelian groups. Miller--Nagpal--Patzt \cite{MNP} described a multiplication map \[\St_n(R) \otimes \St_m(R) \m \St_{n+m}(R) \] involving  ``apartment concatenation'' which makes $\St(R)$ into an augmented graded-commutative monoid object in $\Fun(\GL(R),\Ab)$, the category of functors to abelian groups. Using this structure, we can make sense of the groups $\Tor^{\St(F)}_i(\Z,\Z)_n$. Here $i$ denotes the homological grading and $n$ denotes the grading associated to the groupoid \[\GL(R)=\bigsqcup_n \GL_n(R).\] Miller--Nagpal--Patzt proved that, for $F$ a field, $\St(F)$ is Koszul in the sense that $\Tor^{\St(F)}_i(\Z,\Z)_n$ vanishes for $i \neq n$. They asked what the Koszul dual of $\St(F)$ is \cite[Question 3.11]{MNP} (in other words, what is $\Tor^{\St(F)}_n(\Z,\Z)_n$?). Since $\Tor$ groups can be computed using bar constructions, and
$$ \widetilde{H}_{2n-3}(T_n(F) * T_n(F) ) \cong \St_n(F) \otimes \St_n(F),$$ 
 \autoref{mainLemma} lets us answer this question. 


\begin{theorem} \label{KD}
For $F$ a field, the single nonvanishing $\Tor$ group is $$\Tor^{\St(F)}_n(\Z,\Z)_n \cong \St_n(F) \otimes \St_n(F).$$
\end{theorem}

\begin{question}
The Koszul dual of an associative ring naturally has a co-algebra structure. Since $\St(R)$ is graded-commutative, there is also an algebra structure on the Koszul dual. We invite the interested reader to try to describe these structures explicitly in terms of apartments. 
\end{question}



\begin{remark}
Recall that $k$-fold iterated bar constructions compute $E_k$-indecomposables in the sense of Galatius--Kupers--Randal-Williams \cite{e2cellsI} (also known as $E_k$-homology or $E_k$-Andr\'e--Quillen homology). From this viewpoint, \autoref{mainLemma} implies that $H_*(D^{j+k}(R))$ is the $E_k$-homology of $D^j(R)$. Since the reduced chain complex  $ \widetilde{C}_*(D^1_n(R))$ is equivalent to a shift of $\St_n(R)$, we can interpret $H_*(D^{k}(R))$ as the $E_{k-1}$-homology  of $\St(R)$. In particular, the homology of Rognes' common basis complex measures the $E_\infty$-indecomposables of $\St(R)$. When $R=F$ is a field, this implies that the stable Steinberg modules $\St^{\infty}_n(F)$ measure the $E_\infty$-indecomposables of $\St(F).$
\end{remark}




\subsection{Acknowledgments} We thank Alexander Kupers who contributed significantly to the this project but declined authorship. We also thank Søren Galatius, Manuel Rivera, John Rognes, and Randal-Williams for helpful conversations.



\section{Algebraic foundations} 

We begin by recalling some basic algebraic properties of modules over a PID. Then we give a characterization of when a collection of submodules has the common basis property in terms of an inclusion/exclusion condition (\autoref{CBPCriterionPID}). This characterization will be used in \autoref{Section-TheBuildings} to prove certain subcomplexes of higher Tits buildings are full. 

\subsection{Review of some algebraic preliminaries}

We summarize some statements about PIDs $R$. The following result is standard (see, for example, Kaplansky \cite{Kaplansky}). 


\begin{lemma}  \label{LemmaSplit} Let $U$ be an $R$-submodule of $R^n$. The following are equivalent. 
\begin{itemize}
\item There exist an $R$-submodule $C$ such that $R^n = U \oplus C$. 
\item There exists a basis for $U$ that extends to a basis for $R^n$. 
\item Any basis for $U$ extends to a basis for $R^n$. 
\item The quotient $R^n/U$ is torsion-free. 
\end{itemize} 
\end{lemma}

\begin{definition}
Let $U \subseteq R^n$ be a submodule satisfying the equivalent conditions of \autoref{LemmaSplit}. Then we call $U$ a \emph{split} submodule. We may also say that $U$ \emph{is  a  summand} or \emph{has a complement} in $R^n$.  
\end{definition} 

The following lemma is straight-forward. 
\begin{lemma} \label{LemmaNestedSplit} Let $W \subseteq R^n$ be a split submodule, and let $U \subseteq W$. Then $U$ has a complement in $W$ if and only if it has a complement in $R^n$. 
\end{lemma} 

\begin{lemma} \label{LemmaIntersectionSplit} Let $U,W \subseteq R^n$ be submodules. If $W$ is a summand of $R^n$, then $W \cap U$ is  summand of $U$.  If $U$ is a summand of $R^n$, then in addition $W \cap U$ is a summand of $R^n$. 
\end{lemma} 
\begin{proof} Since $W$ is split in $R^n$, the quotient $U / (U\cap W) \cong (U+W)/ W \subseteq R^n / W$ is torsion-free. Thus $U \cap W$ is a split submodule of $U$. If $U$ is split, then $U\cap W$ is split in $R^n$ by  \autoref{LemmaNestedSplit}. 
  \end{proof}
  
  \begin{example} \autoref{LemmaIntersectionSplit} states that the collection of split submodules of $R^n$ is closed under intersection. Observe that it is not closed under sum. For example, 
$$ U = \mathrm{span} \left( \begin{bmatrix} 1\\ 1 \end{bmatrix} \right)  \qquad \text{and} \qquad W = \mathrm{span} \left( \begin{bmatrix} 1\\-1 \end{bmatrix} \right)$$ are both split $\Z$-submodules of $\Z^2$, but their sum is index $2$ in $\Z^2$. 
  \end{example} 



\begin{definition} \label{DefnCompatible} Let $R$ be a PID and $n \geq 1$. We say a collection  of split submodules $\{U_0, U_1, \dots, U_p\}$  of $R^n$ has the \emph{common basis property} or is \emph{compatible} if there exists a basis $B$ for $R^n$ such that each $U_i$ is spanned by a subset of $B$.  We say that two or more collections of submodules are \emph{compatible} with each other if their union is compatible. 
\end{definition} 

\begin{example}
For example, the lines in $\Z^3$ spanned by $(1,1,0)$ and $(1, -1, 0)$ are not compatible. Split submodules $\{U_0, U_1, \dots, U_p\}$ that form a flag in $R^n$ are always compatible. By definition any subset of a compatible collection is compatible. 
\end{example}





We note that the question of compatibility for split $R$-modules $U$ and $W$ depends on the ambient space. For instance, $U$ and $W$ are always compatible as submodules of $U+W$. 

\begin{example} We note that for a collection of split submodules $U_1, U_2, \dots$ of $R^n$, pairwise compatibility does not imply compatibility. Even when $R$ is a field, if we take three distinct coplanar lines, any two will be compatible but all three will not. When $R$ is not a field there are more subtle phenomena. For example, the $\Z$-submodules of $\Z^3$
$$ U_1 = \mathrm{span} \left( \begin{bmatrix} 1\\ 1 \\0  \end{bmatrix} \right),   \qquad U_2 = \mathrm{span} \left( \begin{bmatrix} 1\\0 \\1  \end{bmatrix} \right),  \qquad U_1 = \mathrm{span} \left( \begin{bmatrix} 0\\ 1 \\1  \end{bmatrix} \right)$$ 
are pairwise-compatible, non-coplanar lines, but the sum $U_1 \oplus U_2 \oplus U_3$ is index $2$ in $\Z^3$. 
\end{example} 

The following lemmas reflect the fact that, for collections of submodules with the common basis property, the operations of sum and intersection simplify to the operations of union and intersection of sets of basis elements. Hence, in this context these operations have better properties than for arbitrary submodules, for example, intersection distributes over sums. The slogan is that modules with the common basis property behave like sets (of basis elements).

The next result is straight-forward. 

\begin{lemma}  \label{CommonBasisIntersection} Suppose that $U, W$ are split submodules of a free $R$-module $R^n$. Suppose there exists some basis $B$ of $R^n$ such that $U$ is spanned by a subset $B_U \subseteq B$ and $W$ is spanned by a subset $B_W \subseteq B$. Then $U + W$ is spanned by $B_U \cup B_W$, and $U \cap W$ is spanned by $B_U \cap B_W$.
\end{lemma}

\begin{lemma} \label{CommonBasisIntersectionSum} Suppose that $U,V, W \subseteq R^n$ have the common basis property. Then $$(U+V) \cap W = (U \cap W) + (V \cap W).$$ 
\end{lemma} 
\begin{proof} Let $B$ be a basis for $R^n$ with subsets $B_U, B_V, B_W$ spanning $U,V,W$ respectively. By \autoref{CommonBasisIntersection}, $(U+V) \cap W$ is spanned by the basis elements $(B_U \cup B_V) \cap B_W$ and $(U \cap W) + (V \cap W)$ is spanned by the same set $(B_U \cap B_W) \cup (B_V \cap B_W)$. 
\end{proof} 

\begin{lemma} \label{CommonBasisClosure} Let $\sigma=\{U_0, \dots, U_p\}$ be a collection of split submodules of $R^n$. Let $\overline{\sigma}$ be the union of $\sigma$ with modules constructed by all (iterated) sums and intersections of elements of $\sigma$.
\begin{enumerate}[(i)]
\item \label{i}  The set  $\sigma$ has the common basis property if and only if  $\overline{\sigma}$ does. 
\item  \label{ii} A submodule $U$ is compatible with $\sigma$ if and only if it is compatible with $\overline{\sigma}$. 
\end{enumerate}
In particular, if $\sigma=\{U_0, \dots, U_p\}$ has the common basis property, modules constructed by (iterated) sums and intersections of the modules $U_i$ are necessarily split. 
\end{lemma} 

\begin{proof} Part (\ref{i}) follows from \autoref{CommonBasisIntersection}.  
 Part  (\ref{ii}) follows from the observation that a common basis for $\{U\} \cup \sigma$ is also a common basis for $\overline{\{U\} \cup \sigma} \supseteq \{U\} \cup \overline{\sigma}$  by Part (\ref{i}), and vice versa. 
\end{proof}





\begin{example} Let $U$, $W$ be split $R$-submodules of $R^n$. If $W$ is contained in a complement of $U$, then $U$ is compatible with $W$, and moreover $U$ is compatible with any split submodule of $W$. In general, however, if $U$ is compatible with $W$ it need not be compatible with split submodules of $W$. For example, the $\Z$-span of $(1, 1, 0)$ in $\Z^3$ is compatible with $W=\Z^3$, but not with the submodule of $W$ spanned by $(1, -1, 0)$. 
\end{example} 




The following result will be used in \autoref{ContractibleTits} to compare certain subcomplexes of higher Tits buildings.

\begin{proposition}\label{mapfdefined}
Let $U,V, W \subseteq R^n$ be submodules with the property that $W+V = R^n$ and $U \subseteq W$. Then $U+V$ is a proper submodule of $R^n$ if and only if $U+(W \cap V)$ is a proper submodule of $W$. 
\end{proposition} 

\begin{proof} We prove the contrapositive. Suppose that $U+(W \cap V) =W.$ Then
\begin{align*}
U + V = U + \big((W \cap V)  +V\big) = \big(U +(W \cap V) ) + V = W+V = R^n 
\end{align*}
Conversely, suppose that  $U+V=R^n$. Then, given $w \in W$, we can write $w = u+v$ for some $u\in U, v \in V$. But then $v  =w-u$ must be contained in $W$, so $v \in W \cap V$.  We see that $w \in U+(W \cap V)$, so we conclude that $U+(W \cap V) = W$. 
\end{proof} 




\subsection{The common basis property and inclusion-exclusion} 

The first goal of this subsection is to prove that a collection of submodules has the common basis property if and only if they satisfy a splitting condition and their ranks satisfy an inclusion-exclusion formula. 

We will use this characterization of the common basis property to deduce a lemma on how the common basis property interacts with flags. This lemma will imply that certain subcomplexes of the higher Tits buildings are full subcomplexes. We will need this fullness result in the Morse theory arguments of \autoref{Section-TheBuildings}.  

\subsubsection{Posets and corank} 




\begin{notation}  Let $R$ be a PID. Let $U_1, U_2, \dots, U_k$ be a collection of submodules of $R^n$. Let $[k]=\{1, 2, 3, \dots, k\}$. For a subet $S \subseteq [k]$, let $U_S$ denote the intersection
$$ U_S = \bigcap_{i \in S} U_i, \quad U_{\varnothing} = R^n$$
\end{notation} 

We note that $U_S \cap U_T = U_{S \cup T}$ for all $S,T \subseteq [k]$. \\

Consider the following two posets. 
\begin{itemize}
\item The poset $(\{ S\}_{ S \subseteq [k]}, \supseteq)$ on subsets $S \subseteq [k]$ ordered by reverse inclusion. \item The poset $(\{ U_S\}_{ S \subseteq [k]}, \subseteq)$ on $R$-submodules $U_S$ ordered by containment. \end{itemize} 

The map 
\begin{align*}
\Phi\colon  \{ S\}_{ S \subseteq [k]} & \longrightarrow \{ U_S\}_{ S \subseteq [k]} \\ 
S & \longmapsto U_S
\end{align*} 
may not be injective. It is, however, order-preserving, as clearly $T \subseteq S$ implies $U_S \subseteq U_T$. 

Though the poset structure on the submodules $\{U_S\}$ is more natural to consider when studying the modules $\{U_i\}$, we will later use the poset structure on the subsets $S$ of $[k]$ to apply the M\"obius inversion formula. 


\begin{remark}\label{preimage} Consider a submodule $U_S$ and its $\Phi$-preimage $\Phi^{-1}(U_S) =\{S_i\}_i$. Observe that the subposet $\{S_i\}_i$ contains a unique minimal element $\bigcup_i S_i$: for if $U_{S_i} = U_{S}$ for all $i$, then 
$$ U_S = \bigcap_i U_{S_i}  = U_{\bigcup_i S_i}.$$ 
\end{remark} 

\begin{definition} Let $R$ be a PID. Let $U_1, U_2, \dots, U_k$ be a collection of submodules of $R^n$. On each poset we define a \emph{corank function}, 
\begin{align*}
F\colon  \{ S\}_{ S \subseteq [k]} & \longrightarrow \R \\ 
S  & \longmapsto \rank_R(U_S) - \rank_R\left(\sum_{S \subsetneq T}  U_T \right)
\end{align*} 
\begin{align*}
G\colon  \{ U_S\}_{ S \subseteq [k]} & \longrightarrow \R \\ 
U_S  & \longmapsto \rank_R(U_S) - \rank_R\left(\sum_{U_T \subsetneq U_S}  U_T \right)
\end{align*} 
\end{definition} 

The corank function is illustrated in two examples in \autoref{Poset-Example}. 

 \begin{figure}[h!]
    \begin{subfigure}[t]{.4\textwidth} \hspace{-2em}
\includegraphics[scale=3.3]{Poset-Example-CBP.pdf}  
      \caption{A collection of submodules of $R^3$ with the common basis property.}
    \end{subfigure} \hfill 
    \begin{subfigure}[t]{.4\textwidth} 
   \includegraphics[scale=3.3]{Poset-Example-nonCBP.pdf} 
      \caption{A collection of submodules of $R^3$ that does not have the common basis property.}
    \end{subfigure}
  \caption{ Let $e_1, e_2, e_3$ denote the standard basis of $R^3$. Two examples are shown of collections of summands $U_1, U_2, U_3$ of $R^3$ and their associated posets. Fibers of $\Phi$ are color-coded according to their image. The values of the corank functions are written next to each element in gray. } 
  \label{Poset-Example}
  \end{figure}


\begin{lemma} \label{FvsG}  For every fixed $S \subseteq [k]$, the corank functions satisfy
$$ F(S) = \left\{ \begin{array}{ll} G(U_S), & \text{$S$ is the unique minimal element in $\Phi^{-1}(U_S)$ (\autoref{preimage})} \\ 0, & \text{ otherwise}. \end{array} \right.
$$ 
\end{lemma} 
\begin{proof}  
Recall that, by definition, $F(S)$ is the corank of   $\sum_{S \subsetneq T}  U_T$ in $U_S$. 


The statement that $S$ is not the minimal element of $\Phi^{-1}(U_S)$ means that there is some $T \subseteq [k]$ such that $S \subsetneq T$ and $U_T = U_S$. In this case, $\sum_{S \subsetneq T}  U_T  = U_S$, and the corank of $\sum_{S \subsetneq T}  U_T$ in $U_S$ is zero as claimed. 

In contrast, suppose $S$ is the unique minimal element of the subposet $\Phi^{-1}(U_S)$. This means $U_T \neq U_S$ for all $T$ with $S \subsetneq T$. In this case 
$$ \sum_{S \subsetneq T}  U_T  = \sum_{U_T \subsetneq U_S} U_T $$ 
and so the values of $F(S)$ and $G(U_S)$ agree. 
\end{proof} 

\begin{corollary} \label{SumF=SumG} $$  \sum_{\{S\}_{S \subseteq [k]}} F(S) =  \sum_{\{U_S\}_{S \subseteq [k]}} G(U_S). $$ 
\end{corollary} 

\begin{proof} By \autoref{FvsG}, the sum $  \sum_{\{S\}_{S \subseteq [k]}} F(S)$  is  the sum $\sum_{\{U_S\}_{S \subseteq [k]}} G(U_S)$ plus additional terms equal to zero. 
\end{proof} 


Given a collection of split submodules $U_1, \dots, U_k$ in $R^n$, suppose we wish to find a minimal collection of spanning vectors of $R^n$ with the property that some subset of these vectors spans $U_i$ for each $i$. The following lemma relates the corank functions to the minimal number of vectors needed to achieve this. 

\begin{lemma} \label{CorankCondition} Let $R$ be a PID. Fix $n$ and let $U_1, U_2, \dots, U_k$ be a collection of split submodules of $R^n$. 
\begin{enumerate}[(a)]
\item  \label{ItemA} Then the integer $$  \sum_{\{S\}_{S \subseteq [k]}} F(S) =  \sum_{\{U_S\}_{S \subseteq [k]}} G(U_S) $$ is a lower bound on the size of any spanning set $B$ of $R^n$  with the property that each submodule $U_S$, $S \subseteq [k]$, is equal to the span of a subset of $B$. 
\item  \label{ItemC} Let $R$ be any PID.   Then $U_1, U_2, \dots, U_k$ has the common basis property if and only if
$$  \sum_{\{S\}_{S \subseteq [k]}} F(S) =  \sum_{\{U_S\}_{S \subseteq [k]}} G(U_S) =n $$
and for each $S \subseteq [k]$ the module $\sum_{i \notin S} (U_S \cap U_i)$ is a split submodule.  
\end{enumerate}  
\end{lemma} 

Note that the splitting condition in  Part (\ref{ItemC}) is automatic for fields. See \autoref{Poset-Example} for an illustration of Part (\ref{ItemC}). 

\begin{proof}[Proof of \autoref{CorankCondition}] 

Recall that we proved  the equality $  \sum_{\{S\}_{S \subseteq [k]}} F(S) =  \sum_{\{U_S\}_{S \subseteq [k]}} G(U_S) $
in \autoref{SumF=SumG}. 

We first address the case that $R$ is a field. We can build our `minimal spanning set' $B$ inductively, proceeding by the height in the poset $\{U_S\}$ of submodules under containment.  Recall that the height of an element $x$ in a poset is the maximal length $h$ of a chain $x_0 < x_1 < \dots <x_h=x$.  

We first choose a basis set for the element $U_{[k]} = \bigcap_i U_i$ of height 0. Then, for each subspace $U_S$ at height one, we choose a complement of $U_{[k]}$ and a basis for the complement; this basis will have $G(U_S)$ elements.  
Next, for each subspace $U_S$ of height $2$, we choose $G(U_S)$ elements of $U_S$ that span a complement of the sum $\sum_{U_T \subsetneq U_S} U_T$ in $U_S$. We continue this procedure inductively by height  until we have a spanning set for $R^n = U_\varnothing$. 

We claim that, at each step $h$, this procedure produces the minimum number of elements required to yield a spanning set for each subspace of height $\leq h$. To see this, suppose we had a set $A$ of vectors with the property that some subset of $A$ spans every subspace of height $\leq h$. Then $A$ must contain a set of vectors spanning all submodules of height $<h$. Moreover, $A$ must contain vectors that span a complement of  $\sum_{U_T \subsetneq U_S} U_T$ in $U_S$ for each subspace $U_S$ of height $h$. The sets of vectors spanning these complements must be disjoint:  if some element $v$ is contained in two distinct subspaces of height $h$, then $v$ is contained in their intersection, and therefore in a subspace of strictly smaller height in the poset. Thus the procedure in the previous paragraph constructs a minimal set with the desired properties, and we have proved Part (\ref{ItemA}) for fields. 

We also note that we could have equally well carried out this construction by induction by height in the poset $\{S\}_{S \subseteq [k]}$. For each subset $S$ of height $h$, we choose $F(S)$ elements that span a complement of $\sum_{S \subsetneq T} U_T$ in $U_S$. 

The total size of the `minimal spanning set' $B$ produced by this procedure is $$  \sum_{\{S\}_{S \subseteq [k]}} F(S) =  \sum_{\{U_S\}_{S \subseteq [k]}} G(U_S). $$  This set $B$ spans $R^n$. Thus $B$ is a basis for $R^n$ if and only if $|B|=n$. If $|B|>n$, then  by minimality of $B$ the subspaces $\{U_S\}_{S \subseteq [k]}$ fail to have the common basis property. \autoref{CommonBasisClosure} implies that the subspaces $\{U_i\}_i$ have the common basis property if and only if the subspaces $\{U_S\}_{S \subseteq [k]}$ do. This concludes the lemma in the case that $R$ is a field. 

Now let us address the case that $R$ is a PID. We consider the same procedure for constructing the `minimal spanning set' $B$. This time, because we want to relate the construction to submodules of the form $\sum_{i \notin S} (U_S \cap U_i)$, we proceed by induction on the poset $\{S\}_{S \subseteq [k]}$. Suppose $S$ is a subset of height $h$.   In this case, at step $h$, it is only possible to choose a complement for the submodule 
$\sum_{S \subsetneq T} U_T$ in $U_S$ if this submodule is split in $U_S$. Otherwise, more than $F(S)$ generators are needed to extend our spanning set for $\sum_{S \subsetneq T} U_T$ to a spanning set for $U_S$. Thus the set $B$ we produce has at least $  \sum_{\{S\}_{S \subseteq [k]}} F(S) =  \sum_{\{U_S\}_{S \subseteq [k]}} G(U_S) $ elements. We have completed the proof of Part (\ref{ItemA}). 

Now we consider the common basis property for $\{U_i\}$. We collect some observations: 
\begin{itemize} 
\item For each $S \subseteq [k]$,
$$\sum_{S \subsetneq T} U_T 
= \sum_{\substack{ i \in [k] \\ i \notin S}} U_S \cap U_i.$$
\item The submodules $U_i$ are split by assumption. Thus, for each $S \subseteq [k]$, the intersection of submodules $U_S$ is split by \autoref{LemmaIntersectionSplit}. It follows by \autoref{LemmaNestedSplit} that $\sum_{i \notin S} U_S \cap U_i$ is split in $U_S$ if and only if it is split in $R^n$. 
\end{itemize} 

By \autoref{CommonBasisClosure}, the set $\{U_i\}$ has the common basis property if and only if its closure under iterated sums and intersections does. In particular, it is a necessary condition that the sum $\sum_{i \notin S} U_S \cap U_i$ be split for every $S$. 

Now suppose the sums $\sum_{i \notin S} U_S \cap U_i$ are split. Then we proceed with our construction of the set $B$ as in the field case, and obtain a set  with $$ |B| = \sum_{\{S\}_{S \subseteq [k]}} F(S) =  \sum_{\{U_S\}_{S \subseteq [k]}} G(U_S) $$ elements. Again, our modules have the common basis property if and only if this value $|B|=n$.
\end{proof} 




\subsubsection{The inclusion-exclusion property} 
We will use \autoref{CorankCondition} to give a new characterization of the common basis property in terms of the inclusion-exclusion formula. The key tool to deduce this equivalence  is M\"obius inversion.

The M\"obius inversion theorem is due to Rota \cite{Rota}. For this version of the theorem, see (for example) Kung--Rota--Yan \cite[3.1.2. M\"obius Inversion Formula]{KungRotaYan}. Recall that a poset $P$ is \emph{locally finite} if for all $x,y$ in $P$ the interval $[x,y] = \{ z  \mid  x \leq z \leq y\}$ is finite. 

\begin{theorem}[Rota \cite{Rota}] \label{MobiusInversion}
 Let $P$ be a locally finite poset. Then there exists an invariant of the poset $P$, a function $\mu\colon  P\times P \to \R$, called a  \emph{M\"obius function}, with the following property. If $f$ and $g$ are real-valued functions defined on $P$, then 
$$ f(y) = \sum_{x \leq y} g(x) \; \; \text{ for all $y \in P$}  \qquad \text{if and only if} \qquad g(y) = \sum_{x \leq y} \mu(x,y) f(x)  \; \;  \text{ for all $y \in P$} .$$ 
\end{theorem} 

\begin{proposition}[Rota {\cite[Proposition 3 (Duality)]{Rota}}] \label{RotaDuality}  If $P$ is a locally finite poset and $P^{\star}$ is its opposite poset, then $\mu_P(x,y) = \mu_{P^{\star}}(y,x)$. 
\end{proposition} 

The following theorem is the statement of Rota's result \cite[Corollary (Principle if Inclusion-Exclusion)]{Rota} combined with  \cite[Proposition 3 (Duality)]{Rota}(here \autoref{RotaDuality}). 
 
\begin{theorem}[Rota \cite{Rota}] \label{MobiusSubsets}   When $P$ is the poset of subsets of a set $[k]$ under inclusion, then both $P$ and its opposite poset (ordered by reverse inclusion) have M\"obius function
$$ \mu(S,T) = (-1)^{|S|-|T|}.$$
\end{theorem} 


M\"obius inversion generalizes the inclusion-exclusion principle for a collection of sets. We will use this result to relate the inclusion-exclusion formula to the common basis property.  

\begin{definition} \label{DefnInclusionExclusionProperty} Let $R$ be a PID. Fix $n$ and let $U_1, U_2, \dots, U_k$ be a collection of  submodules of $R^n$. Then we say that these submodules  have the \emph{inclusion-exclusion property} if for all $S \subseteq [k]$, 
$$\text{ rank$\left(\sum_{i \notin S} U_S \cap U_i \right)$} = \sum_{T\supsetneq S} (-1)^{|S|-|T| +1} \; \rank(U_T). 
$$ 
Equivalently, for all $S \subseteq [k]$, 
 $$\Big(\text{corank of $\sum_{i \notin S} (U_S \cap U_i) $ in $U_S$}\Big)  = \sum_{T\supseteq S} (-1)^{|S|-|T|} \; \rank(U_T). 
$$ 
\end{definition}

In particular, the inclusion-exclusion implies, when $S=\varnothing$, that
$$ \rank(U_1 + U_2 + \dots +  U_k) = \sum_{\varnothing \neq T \subseteq [k]} (-1)^{|T|+1} \; \rank(U_T),$$ equivalently,   
$$ \Big(\text{corank of $(U_1 + U_2 + \dots +  U_k)$ in $R^n$}\Big)  = \sum_{T \subseteq [k]} (-1)^{|T|} \; \rank(U_T).$$

We now prove one of the main results of this section, a characterization of the common basis property in terms of the inclusion-exclusion property. 

\begin{theorem} \label{CBPCriterionPID} Let $R$ be a PID. Fix $n$ and let $U_1, U_2, \dots, U_k$ be a collection of split submodules of $R^n$.  The set $U_1, \dots, U_k$ has the common basis property if and only if it satisfies the inclusion-exclusion property in the sense  of \autoref{DefnInclusionExclusionProperty} and for each $S \subseteq [k]$ the submodule $\sum_{i \notin S} (U_S \cap U_i)$ of $R^n$ is split. 
\end{theorem} 

Observe that, if $R$ is a field, the split condition is automatic. In this case a collection of subspaces has the common basis property if and only it satisfies the inclusion-exclusion property. 

\begin{proof}[Proof of \autoref{CBPCriterionPID}] Consider the poset of subsets of $[k]$ under reverse inclusion, and recall that $F$ denotes the corank function on this poset.  Clearly if a set of modules has the common basis property then so do each subset. 
 Hence by \autoref{CorankCondition}
, $U_1, U_2, \dots, U_k$ has the common basis property if and only if 
\begin{itemize}
\item $\sum_{i \notin S} (U_S \cap U_i)$  is split for every $S$ (a condition that always holds when $R$ is a field). \\

\item   $ \displaystyle \; \rank(U_S) = \sum_{S \subseteq T \subseteq [k]} F(U_T)$ for all $S \subseteq [k]$.
\end{itemize} 
Note that the second item includes the statement that $\displaystyle n =  \sum_{\varnothing \subseteq T \subseteq [k]} F(U_T)$. 


By the M\"obius inversion theorems \autoref{MobiusInversion} and \autoref{MobiusSubsets}, this second condition holds if and only if 
$$ F(U_S) = \sum_{S \supseteq T} (-1)^{|S|-|T|} \rank(U_T) \qquad \text{for all $ \varnothing \subseteq S \subseteq [k].$}$$
This is precisely the inclusion-exclusion property. 
\end{proof} 

\begin{example} Consider a collection $U_1, \dots, U_k$ of $R$-submodules of $R^n$. In general, in the statement of \autoref{CBPCriterionPID} it is not enough to replace the inclusion-exclusion property of \autoref{DefnInclusionExclusionProperty} with the single condition that $$ \rank(U_1 + U_2 + \dots +  U_k) = \sum_{\varnothing \neq S \subseteq [k]} (-1)^{|S|+1} \; \rank(U_S). $$ 
Consider, for example, the free $R$-module $R^2 = \langle e_1, e_2\rangle$ and the submodules 
$$ U_1 = R^2, \quad U_2 =  \langle e_1 \rangle,  \quad U_3 =  \langle e_2\rangle,  \quad U_4 =  \langle e_1 + e_2\rangle. $$ 
These submodules fail to have the common basis property. However, the reader can verify that 
$$ \rank(U_1 + U_2 + U_3 +  U_4) = 2 =  \sum_{\varnothing \neq S \subseteq [k]} (-1)^{|S|+1} \; \rank(U_S). $$ 
Here the right-hand-side has nonzero terms 
\begin{align*} 
& \rank(U_1) + \rank(U_2) + \rank(U_3) + \rank(U_4)  - \rank(U_{12}) -  \rank(U_{13}) - \rank(U_{14})  \\ 
& = 2+1+1+1-1-1-1 \\ 
&=2 .
\end{align*} 
In keeping with \autoref{CBPCriterionPID}, the formula 
$$\text{ rank$\left(\sum_{i \notin S} U_S \cap U_i \right)$} = \sum_{T\supsetneq S} (-1)^{|S|-|T| +1} \; \rank(U_T). 
$$ 
does fail for $S=\{1\}$, as the submodules $U_{12},  U_{13},  U_{14}$ do not satisfy the inclusion-exclusion formula. 
\end{example} 


\subsubsection{Flags and the common basis property} 

In future sections, we will need the following lemma to prove that certain subcomplexes of our higher Tits buildings are full subcomplexes. This lemma is our main application of \autoref{CBPCriterionPID}.  

\begin{lemma} \label{BigFlagLemma}  Let $R$ be a PID.  Fix $n$. Suppose that $U_1, \dots, U_{\ell}$ is a collection of submodules of $R^n$, and $V_1 \subseteq V_2 \subseteq \dots \subseteq V_r$ is a flag in $R^n$.  If $\{U_1, \dots, U_{\ell}, V_i\}$ has the common basis properrty for every $1 \leq i \leq r$, then 
$\{U_1, \dots, U_{\ell}, V_1, \dots, V_r\}$ has the common basis property. 
\end{lemma}   

\begin{proof}   The claim is vacuous when $r=1$. We may proceed by induction on the length  $r$ of the flag. To perform the inductive step, it suffices (after reindexing) to consider a flag of length $r=2$. 

Let $r=2$. Assume $\{U_1, \dots, U_{\ell}, V_i\}$ has the common basis property for $i=1, 2$.  For notational convenience set $k={\ell}+2$, $U_{{\ell}+1}=V_1$ and $U_k = U_{{\ell}+2}=V_2$. 
By \autoref{CBPCriterionPID}, we must verify two things: that $\{U_1, \dots, U_{k-2}, V_1, V_2\}$ has the inclusion-exclusion property, and that for all $S \subseteq [k]$ the submodule $\sum_{i \notin S} (U_S \cap U_i)$  is split. 

We comment that verifying these two properties is a long but conceptually straight-forward computation. For completeness we will write it out explicitly. The calculation will involve direct manipulation of sums of intersections $U_T$ using the following shortlist of elementary observations: 
\begin{itemize}
\item Pairs of $R$-submodules $M,N$ of $R^n$ always satisfy the inclusion-exclusion formula; 
$$ \mathrm{rank}(M+N) =  \mathrm{rank}(M) +  \mathrm{rank}(N) -  \mathrm{rank}(M\cap N).$$ 
\item Given any $R$-submodules $M,N,P$, $$ \text{if $M \subseteq N$ then } (M+P)\cap N = M + (P \cap N).$$ 
\item If submodules $\{M_1, \dots, M_{\ell}, N\}$ have the common basis property, by \autoref{CommonBasisIntersectionSum}, $$(M_1 \cap N) + \dots + (M_{\ell} \cap N) = (M_1 + \dots  +M_{\ell}) \cap N.$$  
\item If a collection of $R$-submodules has the common basis property then all sums of intersections of these modules are split; see  \autoref{CommonBasisClosure}.
\item By assumption, the sets $\{U_1, \dots, U_{\ell}, V_i\}$ satisfy the common basis property for $i=1,2$.
\item By assumption, $V_1 \subseteq V_2,$ in particular $V_1 \cap V_2 = V_1$.
\end{itemize} 

To verify both the inclusion-exclusion property and the split condititon, we first observe that each subset $S \subseteq [k]$ falls into one of four cases. Let $A \subseteq [k-2]$. 
\begin{enumerate}
\item[(i)] $S = A \cup \{k-1, k\}$, i.e.,the intersection $U_S = U_A \cap V_1 \cap V_2 = U_A \cap V_1$. \\  Otherwise said, $U_{A \cup \{k-1, k\}} = U_{A \cup \{k-1\}}$. 

\item[(ii)]  $S=A \cup \{k-1\}$, i.e. $U_S = U_A \cap V_1$. \\
As above, $U_S = U_A \cap V_1 \cap V_2  = U_{S \cup \{k\}}$. 

\item[(iii)]  $S = A \cup \{k\}$, i.e., $U_S = U_A \cap V_2$.

\item[(iv)]  $S=A$, i.e., the intersection $U_S$ has neither factor $V_1$ or $V_2$.
\end{enumerate} 

First we check the inclusion-exclusion property.  
For the set $S = A \cup \{k-1, k\}$, morally, the inclusion-exclusion formula holds because we can delete $U_k=V_2$ from every intersection and appeal to the inclusion-exclusion formula associated to the set $\{U_1, \dots, U_{k-2}, V_1\}$ which by assumption satisfies the common basis property.  In detail, 

 \allowdisplaybreaks
\begin{align*} 
&  \text{ rank$\left(\sum_{i \notin S} U_S \cap U_i \right)$} \\ 
&=  \text{ rank$\left(\sum_{i \in [k-2]\setminus A} U_A \cap V_1 \cap V_2 \cap U_i \right)$} \\
& =  \text{ rank$\left(\sum_{i \in [k-2]\setminus A} U_A \cap V_1  \cap U_i \right)$} \\
& =  \text{ rank$\left(\sum_{i \in [k-1]\setminus (A\sqcup \{k-1\})} U_{A \sqcup \{k-1\} }  \cap U_i \right)$} \\
& = \sum_{ [k-1] \supseteq T \sqcup \{k-1\} \supsetneq A \sqcup \{k-1\} } (-1)^{(|A|+1)-(|T|+1)+1} \; \rank(U_T \cap V_1) \\
& \qquad \qquad \qquad  \text{(by \autoref{CBPCriterionPID}, since $\{U_1, \dots U_{k-2}, V_1\}$ satisfies inclusion-exclusion),} \\ 
& = \sum_{[k-2] \supseteq T\supsetneq A } (-1)^{|A|-|T|+1} \; \rank(U_T \cap V_1) \\
& = \sum_{[k-2] \supseteq T\supsetneq A  } (-1)^{|A|-|T|+1 } \; \rank(U_A \cap V_1 \cap V_2) \\
& = \sum_{[k] \supseteq T \sqcup \{k-1, k\} \supsetneq A \cup \{k-1, k\} } (-1)^{|A|-|T|+1 } \; \rank(U_A \cap V_1 \cap V_2) \\
& = \sum_{[k] \supseteq T' \supsetneq S } (-1)^{|A|-|T'|+1} \; \rank(U_{T'}).
\end{align*} 
 
 
 
 For the set $S =A \cup \{k-1\}$, morally, the inclusion-exclusion expression for the corank of $\sum_{i \notin S} (U_S \cap U_i) $ in $U_S$ vanishes since we have cancellation between the terms for $T$ and $T \cup \{k\}$ for each index $T \subseteq [k-1]$. Correspondingly, the corank does indeed vanish since  $U_S \cap U_k = U_S$. In detail, 
 

 \begin{align*}
& \sum_{[k] \supseteq T\supseteq S} (-1)^{|S|-|T|} \; \rank(U_T) \\ 
 = &  \left( \sum_{[k-1] \supseteq T\supseteq (A  \cup \{k-1\})} (-1)^{|A|+1-|T|} \; \rank(U_{T}) \right) + \left( \sum_{[k-1] \supseteq T\supseteq (A  \cup \{k-1\})} (-1)^{|A|+1-|T|-1} \; \rank(U_{T\cup \{k\}}) \right)  \\ 
  = &  \left( \sum_{[k-1] \supseteq T\supseteq (A  \cup \{k-1\})} (-1)^{|A|+1-|T|} \; \rank(U_{T}) \right) + \left( \sum_{[k-1] \supseteq T\supseteq (A  \cup \{k-1\})} (-1)^{|A|+1-|T|-1} \; \rank(U_{T}) \right)  \\ & \qquad \qquad \qquad  \text{ since $U_T = U_{T\cup \{k\} }$, } \\ 
    = &   \sum_{[k-1] \supseteq T\supseteq (A  \cup \{k-1\})} (-1)^{|A|+1-|T|} \; ( \rank(U_{T}) - \rank(U_T) )\\
    =&\, 0 \\
    =& \Big(\text{corank of $\sum_{i \notin S} (U_S \cap U_i) $ in $U_S$}\Big).
 \end{align*}

 
  For the set $S = A \cup \{k\}$,  
  
   \begin{align*} &  \text{ rank$\left(\sum_{i \notin S} U_S \cap U_i \right)$} \\ 
&=  \text{ rank$\left(U_A   \cap V_2 \cap V_1 +  \sum_{i \in [k-2]\setminus A} U_A  \cap V_2  \cap U_i \right)$} \\
& =  \text{ rank}\left(U_A  \cap V_1 \cap V_2 \right) +  \text{ rank}\left(\sum_{i \in [k-2]\setminus A} U_A  \cap V_2 \cap U_i \right) - \text{ rank} \left( U_A  \cap V_1 \cap V_2 \cap \left(\sum_{i \in [k-2]\setminus A} U_A  \cap V_2 \cap U_i \right)\right)   \\
& \qquad \qquad \qquad  \text{since rank$(C+ B)=$ rank$(C) + $ rank$(B) - $ rank$(C \cap B)$, } \\ 
& =  \text{ rank}\left(U_A  \cap V_1 \cap V_2 \right) +  \text{ rank}\left(\sum_{i \in [k-2]\setminus A} U_A  \cap V_2 \cap U_i \right) - \text{ rank} \left( U_A  \cap V_1\cap V_2  \cap \left(\sum_{i \in [k-2]\setminus A} U_A    \cap U_i \right)\right)   \\
& \qquad \qquad \qquad  \text{because $\sum_{i \in [k-2]\setminus A} U_A  \cap V_2 \cap U_i  =  V_2 \cap \left( \sum_{i \in [k-2]\setminus A} U_A \cap U_i \right)$ by \autoref{CommonBasisIntersectionSum} } \\  & \qquad \qquad \qquad  \text{since $\{U_1, \dots, U_{k-2}, V_2\}$ has the common basis property, } \\ 
& =  \text{ rank}\left(U_A  \cap V_1 \cap V_2 \right) +  \text{ rank}\left(\sum_{i \in [k-2]\setminus A} U_A  \cap V_2 \cap U_i \right) - \text{ rank} \left( U_A  \cap V_1   \cap \left(\sum_{i \in [k-2]\setminus A} U_A    \cap U_i \right)\right)   \\
& =  \text{ rank}\left(U_A  \cap V_1 \cap V_2 \right) +  \text{ rank}\left(\sum_{i \in [k-2]\setminus A} U_A  \cap V_2 \cap U_i \right) - \text{ rank} \left(\sum_{i \in [k-2]\setminus A} U_A    \cap V_1 \cap U_i \right)   \\
& \qquad \qquad \qquad  \text{ by \autoref{CommonBasisIntersectionSum} since $\{U_1, \dots, U_{k-2}, V_1\}$ has the common basis property, } \\ 
& =  \text{ rank}\left(U_A  \cap V_1 \cap V_2 \right)  +  \sum_{[k-2]\cup\{k\} \supseteq T\supsetneq A \cup \{k\}} (-1)^{|A|-|T|} \; \rank(U_T) -  \sum_{[k-1] \supseteq T'\supsetneq A \cup\{k-1\}} (-1)^{|A|-|T'| } \; \rank(U_{T'})\\
& \qquad \qquad \qquad  \text{since \autoref{CBPCriterionPID} applies to the second and third term,} \\ 
& =  \text{ rank}\left(U_A  \cap V_1 \cap V_2 \right)  +  \sum_{[k-2]\cup\{k\} \supseteq T\supsetneq A \cup \{k\}} (-1)^{|A|-|T|} \; \rank(U_T) -  \sum_{[k] \supseteq T\supsetneq A \cup\{k-1, k\}} (-1)^{|A|-|T|+1 } \; \rank(U_T)\\
& \qquad \qquad \qquad  \text{since $U_{T'} = U_{T'\cup \{k\}}$ when $k-1 \in T'$, } \\ 
& =  \sum_{[k] \supseteq T\supsetneq A \cup \{k\}} (-1)^{|A|-|T|} \; \rank(U_T) \\
& =  \sum_{[k] \supseteq T\supsetneq S} (-1)^{|S|-|T|+1} \; \rank(U_T).  
  \end{align*} 
  
 


 For the set $S=A \subseteq [k-2]$, morally, all intersections involving $V_1$ cancel, and the result follows since  $\{U_1, \dots, U_{k-2}, V_2\}$ satisfies the common basis property. In detail, 
    \begin{align*}
 & \sum_{[k] \supseteq T\supseteq S} (-1)^{|S|-|T|} \; \rank(U_T)  \\ 
  & =   \sum_{[k-2] \supseteq T\supseteq S} (-1)^{|S|-|T|} \; \rank(U_{T})  +  \sum_{[k-2] \supseteq T\supseteq S} (-1)^{|S|-|T|+1} \; \rank(U_{T \cup \{k\}})  \\ & \qquad +  \sum_{[k-2] \supseteq T\supseteq S} (-1)^{|S|-|T|+1} \; \rank(U_{T \cup \{k-1\}})  +  \sum_{[k-2] \supseteq T\supseteq S} (-1)^{|S|-|T|+2} \; \rank(U_{T \cup \{k-1, k\}}) \\ 
    & =   \sum_{[k-2] \supseteq T\supseteq S} (-1)^{|S|-|T|} \; \rank(U_{T})  +  \sum_{[k-2] \supseteq T\supseteq S} (-1)^{|S|-|T|+1} \; \rank(U_{T \cup \{k\}})  \\
    & \qquad \qquad  \text{since $U_{T \cup \{k-1, k\}} = U_{T \cup \{k-1\}}$ for all $T \subseteq [k-2]$}, \\ 
    & = \Big(\text{corank of $U_S \cap V_2 + \sum_{i \in [k-2]\setminus S} (U_S \cap U_i) $ in $U_S$}\Big) \\ & \qquad \qquad \text{since $\{U_1, \dots U_{k-2}, V_2\}$ has the common basis property hence satisfies inclusion-exclusion}, \\ 
        & = \Big(\text{corank of $U_S \cap V_2 + U_S \cap V_1 + \sum_{i \in [k-2]\setminus S} (U_S \cap U_i) $ in $U_S$}\Big)  \\ & \qquad \qquad \text{since $U_S \cap V_1 \subseteq U_S \cap V_2$}, \\
        &= \Big(\text{corank of $\sum_{i \in [k]\setminus S} (U_S \cap U_i) $ in $U_S$}\Big) 
  \end{align*}

   
 
 Next we check that for all $S \subseteq [k]$ the submodule $\sum_{i \notin S} (U_S \cap U_i)$ is split.  Let $A \subseteq [k-2]$ and consider the four cases. 
Take the case $S=A \cup \{k-1, k\}$. Then $$U_S = U_{A \cup \{k-1,k\} } = U_A \cap V_1 \cap V_2 = U_A \cap V_1,$$  so 
 \begin{align*}
 \sum_{i \notin S} (U_S \cap U_i)  = \sum_{ i \in [k-2] \setminus A } U_A \cap V_1 \cap U_i.
 \end{align*}
 This sum is split by \autoref{CommonBasisClosure} since $\{U_1, \dots, U_k, V_1\}$ has the common basis property. 
 
 In the case $S=A \cup \{k-1\}$, we see $U_S = U_{A \cup \{k-1\}} = U_A \cap V_1$, so 
 \begin{align*}
 \sum_{i \notin S} (U_S \cap U_i)  & = \left( \sum_{ i \in [k-2]\setminus A} U_A \cap V_1 \cap U_i \right) + U_A \cap V_1 \cap V_2 \\  
 & = \left( \sum_{ i \in [k-2]\setminus A} U_A \cap V_1 \cap U_i \right) + U_A \cap V_1 \\
 & = U_A \cap V_1
 \end{align*}
 which is split since the modules $U_1, \dots, U_k, V_1$ are split (\autoref{LemmaIntersectionSplit}). 
 
 
  In the case $S=A \cup \{k\}$ we have $U_S = U_{A \cup \{k\}} = U_A \cap V_2$, so 
 \begin{align*}
 \sum_{i \notin S} (U_S \cap U_i)  & = \left( \sum_{ i \in [k-2]\setminus A} U_A \cap V_2 \cap U_i \right) + U_A \cap V_1 \cap V_2  \\  
& = \left( \sum_{ i \in [k-2]\setminus A} U_A  \cap U_i \right)\cap V_2 + U_A \cap V_1  \cap V_2 \\ & \qquad \qquad \qquad \qquad \text{ by \autoref{CommonBasisIntersectionSum} since $\{U_1, \dots, U_{k-2}, V_2\}$ has the common basis property}, \\  
& = \left(\left( \sum_{ i \in [k-2]\setminus A} U_A  \cap U_i \right) + U_A \cap V_1 \right) \cap V_2  \\  & \qquad \qquad \qquad \qquad  \text{ since $U_A \cap V_1 \subseteq V_1 \subseteq V_2$.} 
 \end{align*}
But $\left(\left( \sum_{ i \in [k-2]\setminus A} U_A  \cap U_i \right) + U_A \cap V_1 \right)$ is split  by \autoref{CommonBasisClosure}  since  $\{U_1, \dots, U_k, V_1\}$ has the common basis property.  Since $V_2$ is split, the intersection is split by \autoref{LemmaIntersectionSplit}. 
 
 
  In the case $S=A$, 
 \begin{align*}
 \sum_{i \notin S} (U_S \cap U_i)  & = \left( \sum_{ i \in [k-2]\setminus A} U_A \cap U_i \right) + U_A \cap V_1 + U_A \cap V_2 \\  
 & = \left( \sum_{ i \in [k-2]\setminus A} U_A  \cap U_i \right)  + U_A \cap V_2 
 \end{align*}
 since $(U_A + V_1) \subseteq (U_A + V_2)$. The resultant sum is split by \autoref{CommonBasisClosure} because $\{U_1, \dots, U_k, V_2\}$ has the common basis property. 
\end{proof} 



\section{Comparison of higher Tits buildings} \label{Section-TheBuildings}

In this section, we recall the definitions of the classical Tits building and Charney's split Tits building \cite{Charney}.  We then introduce ``higher" Tits buildings generalizing constructions of Rognes \cite{Rog1} and Galatius--Kupers--Randal-Williams \cite{e2cellsIV}. The main result of this section, \autoref{SplitvsNotSplit}, is a comparison theorem between higher buildings with different splitting data. 


\subsection{Simplicial complex models of higher buildings}

In this subsection we recall the definitions of the (split) Tits buildings and introduce the higher variants. 

\begin{definition}[Tits building] 
Let $M$ be a finite-rank free $R$-module with $R$ a PID. Let $T(M)$ be the realization of the poset of proper nonzero split submodules of $M$ ordered by inclusion.  
\end{definition} 

In this context, we say that proper nonzero split submodules $U,V \subseteq M$ are \emph{comparable} if  $U \subseteq V$ or $V \subseteq U$. 



Charney introduced the following simplicial complex to prove homological stability for general linear groups of Dedekind domains \cite{Charney}.


\begin{definition}[Split Tits building]
Let $M$ be a finite-rank free $R$-module with $R$ a PID. A \emph{splitting of $M$ of size $k$}  is tuple $(V_1,\dots,V_k)$ with $M=V_1 \oplus \dots \oplus V_k$. Let $ST(M)$ be the realization of the poset of splittings of size $2$ with $(P,Q)<(P',Q')$ if $P \subsetneq P'$ and $Q' \subsetneq Q$. We call $ST(M)$ the \emph{split Tits building} associated to $M$. 
\end{definition}

We note that the name ``split Tits building" is intended to emphasize its relationship to the classical Tits building; however, these complexes do not satisfy the combinatorial definition of a building. 

\begin{proposition} \label{SplittingtVsSplitFlag} 
There is a bijection between the set of $p$-simplices of $ST(M)$ and the set of splittings of $M$ of size $p+2$ given by:

\begin{align*} (P_0,Q_0) < \dots <(P_p,Q_p) &\longmapsto (P_0,P_1 \cap Q_0,P_2 \cap Q_1, \dots, P_p \cap Q_{p-1},Q_p ).\\
(M_0, M_1 \oplus \dots \oplus M_{p+1}) 
< \dots <(M_0  \oplus \dots \oplus M_p,M_{p+1}) & \longmapsfrom (M_0, M_1, \dots, M_{p+1} ).
\end{align*}
\end{proposition}


See (for example) Hepworth \cite[Proposition 3.4]{Hepworth-Edge}. We now recall the definition of Rognes' common basis complex \cite{Rog1}.

\begin{definition}[Common basis complex] Let $R$ be a PID. 
 The \emph{common basis complex} ${\CB}_n(R)$ is the simplicial complex whose vertices $V$ are proper nonzero summands of $R^n$, and where a set of vertices $\sigma=\{V_0,\dots, V_p\}$ spans a simplex if and only if it has the common basis property in the sense of \autoref{DefnCompatible}. 
\end{definition}



\begin{definition}[Higher Tits buildings] \label{DefnTab}
Let $M$ be a finite-rank free $R$-module with $R$ a PID.  Let $\sigma=\{W_0,\dots,W_k \}$ be a collection of  submodules of $M$ with the common basis property. We define $T^{a,b}(M,\sigma)$ as a subcomplex of the join $$\underbrace{T(M) * \dots * T(M)}_{a\ \text{times}} * \underbrace{ST(M) * \dots * ST(M)}_{b\ \text{times}}.$$ 
Let $ (V_{0,i} < \dots < V_{n_{i},i}  ) $ denote a flag of proper nonzero submodules, representing a simplex in the $i$th factor $T(M)$. Let $(M_{0,i},\dots,M_{m_{i},i})$ denote a splitting of $R^n$ of size $(m_i+1)$ into proper nonzero subspaces, representing a simplex in the $i$th factor $ST(M)$. Then a simplex  
$$ (V_{0,1} < \dots < V_{n_1,1}  ) * \dots * (V_{0,a} < \dots < V_{n_a,a}  ) * (M_{0,1},\dots,M_{m_1,1}  ) * \dots * (M_{0,b},\dots,M_{m_b,b}  ) $$ 
of the join is contained in $T^{a,b}(M,\sigma)$ precisely when $$\{V_{0,1}, \dots, V_{n_a,a}, M_{0,1}, \dots,   M_{m_{b,b}} , W_0, \dots, W_k  \}$$ has the common basis property. For $\sigma$ empty, denote $T^{a,b}(M,\sigma)$ by $T^{a,b}(M)$. We write $T(M, \sigma)$ for $T^{1,0}(M, \sigma)$ and $ST(M, \sigma)$ for $T^{0,1}(M, \sigma)$. 
\end{definition}
Notably, in the definition, the submodules  appearing in different join factors can coincide with each other or with the submodules $W_i$. 

\begin{example} Let $M$ be a finite-rank free $R$-module. Recall that a \emph{frame} for $M$ is a decomposition $M= L_1\oplus L_2 \oplus \dots \oplus L_n$ of $M$ into a direct sum of lines. If $\sigma$ is a fixed frame $\{L_1, \dots, L_n\}$, then $T(M, \sigma) \cong S^{n-2}$  is the full subcomplex of $T(M)$ on vertices spanned by subsets of $\sigma$. This is called the \emph{apartment} associated to the frame $\sigma$. 
\end{example} 


\begin{example} Let $R=F$ be a field, and let $M$ be a finite-dimensional $F$-vector space. Let $\sigma$ be a flag in $M$. Then $T(M, \sigma) = T(M)$.  Since $T(M)$ is a \emph{building}, by definition any two simplices are contained in a common \emph{apartment}, i.e., any two flags in $M$ admit a common basis for $M$. 
See (e.g.) Abramenko--Brown \cite[Section 4]{AbramenkoBrown} for precise definitions and \cite[Section 4.3]{AbramenkoBrown} for a proof. 

This is not true for more general rings $R$. For example, let $M=R^2 = \langle e_1, e_2\rangle.$  If  $\sigma = \{ R e_1 \}$ then $T(R^2, \sigma)$ has vertices all spans of primitive vectors with second coordinate a unit or zero. If $R$ is not a field, this is strictly smaller than the vertex set of $T(R^2)$. 
\end{example} 



The following lemma is a consequence of \autoref{BigFlagLemma}. 

\begin{lemma} Let $M$ be a finite-rank free $R$-module, and $\sigma$ a set of proper nonzero submodules. Then $T(M, \sigma)$ is a full subcomplex of $T(M)$. 
\end{lemma} 




\subsection{Combinatorial Morse Theory}
To analyze these complexes we will employ variations on a technique sometimes called combinatorial Morse theory. See Bestvina \cite{Bestvina-MorseTheory} for exposition on this tool. This method has been used (for example) by Charney \cite{Charney} in her analysis of split Tits buildings. 
 

\begin{theorem}{\bf (Combinatorial Morse Theory).}  \label{Morse} Let $X$ be a simplicial complex and let $Y$ be a full subcomplex of $X$. Let $S$ be the set of  vertices of $X$ that are not in $Y$. Suppose there is no edge between any pair of vertices $s,t  \in S$. Then $$X/Y \simeq  \bigvee_{s \in S} \Sigma \big( \Link_X(s) \big).$$ In particular, if $Y$ is contractible and $  \Lk_X(s) $ is contractible for every $s \in S$, then $X$ is contractible. 

\end{theorem}


\autoref{Morse} allows us to construct the complex $X$ out of the complex $Y$ and the set $S$ of added vertices. For each vertex $s \in S$ we cone off the subcomplex $\Link_X(s) = \Link_X(s)  \cap Y$ in $Y$ and identify $s$ with the cone point. 

We can use \autoref{Morse} to prove a more general version of combinatorial Morse theory that allows us to add higher-dimensional simplices as well as vertices. 

\begin{theorem}{\bf (Generalized Combinatorial Morse Theory).}  \label{Morse2} Let $X$ be a simplicial complex and let $Y$ be a  subcomplex of $X$. Let $S$ be a set of  simplices of $X$ satisfying the following conditions:
\begin{enumerate}[(i)]
\item \label{Item-Morse2i} Suppose $\sigma$ is a simplex in $X$. Then $\sigma$ is a simplex in $Y$ if and only if no face of  $\sigma$ is in $S$.
\item \label{Item-Morse2ii} If $\tau$ and $\sigma$ are distinct simplices in $S$, then the union of their vertices does not form a simplex in $X$. 
\end{enumerate}
Then $$X/Y \simeq  \bigvee_{\sigma \in S} \Sigma^{\dim(\sigma)+1} \big( \Link_X(\sigma) \big).$$ 
\end{theorem}

\begin{proof} We will reduce this statement to \autoref{Morse}. We will construct a new simplicial complex $X'$ homeomorphic to $X$ by subdividing $X$. Heuristically, we will introduce a new vertex $v(s)$ to the barycenter of each positive-dimensional simplex in $s \in S$, and take the minimal necessary subdivision. Condition (\ref{Item-Morse2ii}) will ensure a canonical minimal subdivision exists. 

Let $\V(X)$ be the set of vertices of $X$. The vertices of $X'$ will be the disjoint union of $\V(X)$ and $S' = \{ v(s)  \mid  s \in S \text{ with } \dim(s)\geq 1\}$.  The following is a complete list of simplices of $X'$: 
\begin{enumerate}[(a)]
\item \label{SimplicesType(a)} A set of vertices $x_0, \dots, x_p \in \V(X)$ spans a simplex in $X'$ if and only if they span a simplex in $X$ that has no positive-dimensional face contained in $S$.
\item \label{SimplicesType(b)} Let $s =[s_0, \dots, s_q]$ be a positive-dimensional simplex in $S$. Then for  $x_0, \dots, x_p \in \V(X)$, the vertices $x_0, \dots, x_p, v(s)$ span a simplex in $X'$ if and only if the vertices in the (not necessarily disjoint) union  $\{s_0, \dots, s_q\} \cup \{x_0, \dots, x_p \}$ form a simplex in $X$ but $\{s_0, \dots, s_q\} \not\subseteq \{x_0, \dots, x_p \}$. 
\end{enumerate} 
We briefly verify that our construction of $X'$ is well-defined, by checking that all faces of simplices in $X'$ are themselves simplices. This is straightforward except for a simplex $[x_0, \dots, x_p, v(s)]$ of type (\ref{SimplicesType(b)}) and a face $\alpha=[x_{i_0}, \dots, x_{i_t}]$ not containing the vertex $v(s)$. By assumption $\alpha$ is a face of the simplex spanned by the vertices $\{s_0, \dots, s_q\} \cup \{x_0, \dots, x_p \}$ but $\alpha$ does not contain $s \in S$. By Condition (\ref{Item-Morse2ii}) $\alpha$ cannot have a face in $S$. Thus $\alpha$ is a simplex of type (\ref{SimplicesType(a)}). 

To argue that $X'$ is homeomorphic to $X$, we interpret $X'$ as a subdivision of $X$. Simplices with no positive-dimensional face in $S$ are not divided; these are simplices of type (\ref{SimplicesType(a)}).  So we consider simplices with faces in $S$. Note that Condition (\ref{Item-Morse2ii}) ensures that any given simplex in $X$ has at most one face contained in $S$. 

 Let $s =[s_0, \dots, s_q]$ be a positive-dimensional simplex of $S$. Let $\sigma$  be a simplex spanned by the disjoint union of the vertices $\{s_0, \dots, s_q\} \sqcup \{x_{q+1}, \dots, x_r \}$.  Item (\ref{SimplicesType(b)})  states that, to build $X'$ from $X$, we will replace $\sigma$ with a homeomorphic complex obtained by adding a vertex $v(s)$ to the barycenter of the face $s$. Specifically, we replace  the simplex $\sigma$ by the union of the simplices spanned by the vertices $x_{q+1}, \dots, x_r, v(s)$ and each proper subset of $\{s_0, \dots s_q\}$. 
 
 An example of complexes $X$ and $X'$ is shown in \autoref{MorseExample1}. 

\begin{figure}[h!]
\includegraphics[scale=3]{MorseLemma-Example1} 
\caption{In this example, the set $S$ consists of the edge $\sigma_1$, the 2-simplex $\sigma_2$, and the vertex $\sigma_3$ shown in dark gray in the first figure. The subcomplex $Y \subseteq X$ is highlighted in the second figure. The subdivided complex $X'$ is shown in the third, with the newly added simplices shown in black.} 
\label{MorseExample1}
\end{figure} 




Observe that (by Condition (\ref{Item-Morse2i})) simplices in $Y \subseteq X$ are not subdivided in $X'$. Thus we may view $Y$ as a subcomplex of $X'$ in a manner compatible with the homeomorphism $X \cong X'$. 

Let $T = S' \cup \{s  \mid   s \in S \text{ with $\dim(s)=0$}\}$ be the union of the new vertices $v(s)$ in $X'$ and the elements of $S$ that are vertices.  Condition (\ref{Item-Morse2i}) states that a vertex of $X$ is in $Y$ if and only if it is not an element of $S$. Hence $T$ is precisely the set of vertices of $X'$ not contained in $Y$. 

The set $T$ is in canonical bijection with $S$. For uniformity we write $v(s)$ for the vertex $s$ when $s$ is an element of $S$, so $T=\{v(s) \mid   s \in S\}$. 


We now verify that  there is no edge in $X'$ between any pair of vertices in $T$. From our description of the simplices of $X'$, there are no edges in $X'$ between ``new" vertices $v(s)$ in $S'$. Given a $0$-dimensional simplex $s \in S$, Condition (\ref{Item-Morse2ii})  states that $s$ is not contained in any simplex of $X$ with a face $\sigma$ in $S$ distinct from $s$. This condition implies there are no edges between distinct $0$-simplices of $S$ in $X$, and in our construction of $X'$ no new edges are added between vertices of $X$. Condition (\ref{Item-Morse2ii})  implies moreover, by our description of simplices of type (\ref{SimplicesType(b)}),  there is no edge between $s$ and $v(\sigma)$ for any positive-dimensional simplex $\sigma \in S$. We deduce that there are no edges between elements of $T$. 

Next we verify that $Y$ is a full subcomplex of $X'$ (specifically, the full subcomplex on vertices not in $T$). Let $x_0, \dots, x_p$ be a set of vertices in $Y$ that span a simplex $\sigma$ in $X'$. Since the vertices $x_i$ are in $Y$ they must 
\begin{itemize}
\item be vertices of $X$, and
\item not include any 0-simplex in $S$. 
\end{itemize} 
The first item implies that $\sigma$ is a simplex of type (\ref{SimplicesType(a)}). Condition (\ref{Item-Morse2i})  and the second item then imply that $\sigma$  is a simplex of $Y$. 


We can therefore apply \autoref{Morse} to the complex $X'$, its full subcomplex $Y$, and the vertex set $T$. This implies
$$X/Y \cong X'/Y \simeq  \bigvee_{s \in S} \Sigma\big( \Link_{X'}( v(s))  \big).$$ 
But
$$ \Link_{X'}( v(s))  \cong ( \Link_{X}(s) ) * \partial s \cong  \Sigma^{\dim(\sigma)} \big( \Link_X(\sigma)\big).  $$
The result follows.
\end{proof}



\begin{comment}
 \jw{This subsection was copy-pasted directly from the document Notes-AdaptingCharneyMorseTheory-AIMZ\textasciicircum4Case. Should edit it. May want to rephrase for simplicial complexes instead of posets}

\begin{theorem}{\bf (Combinatorial Morse Theory).}  \label{Morse} Let $X$ be a poset with $X=F_0 \cup X_1 \cup \dots \cup X_m$ as sets. Let $F_k=F_0 \cup X_1 \cup \dots \cup X_k$. Suppose the following: 
\begin{enumerate}
\item $|F_0|$ is contractible.
\item For $k \geq 1$ then any pair $s,t  \in X_k$ of distinct elements are not comparable.
\item For $k \geq 1$ and $t \in X_k$,  $  |\Lk_X(t) \cap F_{k-1}| \simeq \bigvee S^{d-1}.  $ 
\end{enumerate}
Then $|X|$ is homotopy equivalent to $\bigvee S^d$.
\end{theorem}


\begin{proof}[Proof idea for Theorem \ref{Morse}, from Charney:] Proceed by induction, showing that for $|F_k| \simeq \bigvee S^d$ for all $k \geq 1$. By assumption, 
\begin{align*}
|F_1| &= |F_0| \cup \left( \bigcup_{t \in X_1} \text{cone on }  \left| \Lk_{F_1}(t) \cap F_0  \right| \right) \quad \text{ where } |F_0| \simeq *, \quad \left|\Lk_{F_1}(t) \cap F_0 \right| \simeq S^{d-1},  \\ 
& = \bigvee_{t \in X_1} \text{ suspension on } S^{d-1} \\ 
& = \bigvee S^d
\end{align*}
Now, by inductive hypothesis,  $|F_k|=|F_0 \cup X_1 \cup \dots \cup X_k| \simeq \bigvee S^d$. Then 
\begin{align*}
|F_k| &= |F_{k-1}| \cup \left( \bigcup_{t \in X_k} \text{cone on }  \left| \Lk_{F_k}(t) \cap F_{k-1}  \right| \right) \qquad \text{ where }  \left|\Lk_{F_k}(t) \cap X_0 \right| \simeq S^{d-1}  \\ 
& =  \left( \bigvee S^d  \right) \cup \left( \bigvee_{t \in X_1} \text{ suspension on } S^{d-1} \right)  \qquad \text{ where $S^{d-1} \subseteq S^d$ is contractible,}\\ 
& = \bigvee S^d 
\end{align*}
In particular, $|X| = |F_m| \simeq \bigvee S^d$. 
\end{proof}
\end{comment} 


\subsection{Contractible subcomplexes}

In this subsection, we prove that certain subcomplexes of higher Tits buildings are contractible. In the next subsection we will use this result in our proof that the homology of $T^{a,b}(M)$ (\autoref{DefnTab}) depends only on the sum $a+b$ whenever $a \geq 1$. 

\begin{lemma} \label{sigmaClosure}
Let $M$ be a finite-rank free $R$-module with $R$ a PID.  Let $\sigma=\{W_0,\dots,W_k \}$ be a collection of split submodules of $M$. Let $\bar \sigma$ be the closure of $\sigma$ under intersections and sum. Then $T^{a,b}(M,\bar \sigma) \m T^{a,b}(M, \sigma)$ is an isomorphism.
\end{lemma}

\begin{proof} This lemma is a consequence of \autoref{CommonBasisClosure}. 
\end{proof} 


The map $ST(M) \m T(M)$ given by forgetting complements induces maps $$T^{a,b}(M,\sigma) \m T^{a',b'}(M,\sigma)$$ whenever $a+b=a'+b'$ and $a' \geq a$. The goal of this section is to prove this map is a homology equivalence whenever $a \geq 1$. The following proposition is the main technical result that we use to prove this comparison.

\begin{proposition} \label{ContractibleTits}
Let $M$ be a finite-rank free $R$-module with $R$ a PID.  Let $\tau=\{V_0, \dots,V_p\}$ be a nonempty simplex of $T(M)$ with $V_0 \subsetneq V_1 \subsetneq \dots \subsetneq V_p$. Let $\sigma$ be a simplex of ${\CB}(M)$ containing $\tau$. Let $\tau_1,\dots,\tau_r$ be distinct simplices of $ST(M,\sigma)$ in the preimage of $\tau$ under the map $ST(M,\sigma) \m T(M,\sigma)$.  Let $\sigma_i$ be the simplex obtained by adding the vertices of $\tau_i$ to $\sigma$. If $r \geq 2$ and $\rank (M) \geq 2$, then $$T(M,\sigma_1) \cap \dots \cap T(M,\sigma_r) $$ is contractible.
\end{proposition}

Let us parse the statement of the proposition. We fix a flag $\tau=\{V_0, \dots,V_p\}$ in $M$ and a set $\sigma = \{V_0, \dots, V_p, Z_0, \dots, Z_k\}$ with the common basis property.  Then $$\tau_i = \{(V_0, C_{0,i}), (V_1, C_{1,i}), \dots, (V_p, C_{p,i})\}$$ and $$\sigma_i = \{C_{0,i} , \dots, C_{p,i}\} \cup \sigma = \{ V_0, C_{0,i} , \dots, V_p,  C_{p,i},Z_0, \dots, Z_k\}  $$ are defined by the conditions:
\begin{itemize}
\item $V_{\ell} \oplus C_{{\ell,i}} = M$ for all $\ell$,
\item $C_{0,i} \supseteq C_{1,i} \supseteq \dots \supseteq C_{p,i} $,
\item The collection of submodules $\sigma_i=\{ V_0, C_{0,i} , \dots, C_{p,i},V_p, Z_0, \dots, Z_k\} $ has a common basis,
\item The flags $C_{0,i} \supseteq \dots \supseteq C_{p,i} $ and $C_{0,j} \supseteq  \dots \supseteq C_{p,j} $ are distinct for all $i\neq j$. This means there is at least one index $\ell$ with $C_{\ell, i} \neq C_{\ell, j}$. 
\end{itemize} 
A submodule $W$ is a vertex of $T(M,\sigma_1) \cap \dots \cap T(M,\sigma_r) $ precisely when $W$ is a proper nonzero submodule of $M$ compatible with $\sigma_i =  \{C_{0,i} , \dots, C_{p,i}\} \cup \sigma$ for each $i=1, \dots, r$.  This set of submodules $W$ contains  $\sigma$ and is closed under sum and intersection with the submodules in $\sigma$. 


\begin{proof}[Proof of \autoref{ContractibleTits}]

Let $n=\rank(M)$. Assume we have proven the claim for free $R$-modules of rank strictly less than $n$. Let $$X=T(M,\sigma_1) \cap \dots \cap T(M,\sigma_r). $$ 
%
By \autoref{BigFlagLemma}, $X$ is a full subcomplex of $T(M)$. In other words, given a flag $U_0 < U_1 < \dots <U_p$ in $M$, if each $U_i$ is compatible with each of $\sigma_1, \dots ,\sigma_r$, the lemma ensures that $\{U_0, \dots, U_p\}$ is compatible with each of $\sigma_1, \dots ,\sigma_r$. Thus a set of vertices in $X$ span a simplex if and only if they form a flag. By \autoref{sigmaClosure} we may assume without loss of generality that $\sigma$ is closed under intersection. Let $V \subseteq V_0$ be a minimal rank subspace which is a vertex of $\sigma$. (Recall that $\tau \subseteq \sigma$.) 

We will proceed by using two applications of combinatorial Morse theory (\autoref{Morse}), by partitioning the vertices of $X$ into 
\begin{itemize}
\item the set of vertices $W$ in $X$ comparable to $V$
\item the set of vertices $W$ not comparable to $V$ with $V + W \neq M$, further partitioned by rank
\item the set of vertices $W$ not comparable to $V$ with $V + W =M$, further partitioned by rank
\end{itemize}


Let $F_0=\Star_X(V)$, so $F_0$ is contractible by construction. 
For $k \geq 1$, define
$$S_k = \left\{ W \;  \middle |\; \begin{array}{l}  $W$ \text{ a vertex of $X$}, \, W \not\in F_0, \\  \rank(W)=k,  \,  W + V \subsetneq M \end{array} \right\}.$$ 

Inductively define $F_k$ to be the full subcomplex of $X$ obtained from $F_{k-1}$ by adding the vertices $S_k$. Assume by induction that $F_{k-1}$ is contractible. There are no edges in $X$ between vertices of $S_k$ since they have the same rank as $R$-modules.  By \autoref{Morse}, we can inductively prove that $F_k$ is contractible by verifying $ \Link_X(W) \cap F_{k-1} $ is contractible for all $W \in S_k$. 

So suppose $S_k$ is nonempty and let $W \in S_k$. We will show that the submodule $V +W$ is a cone point of $ \Link_X(W) \cap F_{k-1} $.  Because $X$ is a full subcomplex of $T(M)$, it suffices to show that $V+W$ is comparable to every vertex under containment. Since $V$ and $W$ are compatible by construction, their sum $V+W$ is split. 

Observe that $V \subseteq V+W \subsetneq M$, so $V+W \in F_0 \subseteq F_{k-1}$. Since $W \subseteq V+W$, we deduce  $V +W \in \Link_X(W) \cap F_{k-1} $. To prove that $V+W$ is a cone point we must verify that any  $U \in \Link_X(W) \cap F_{k-1}$ is comparable to $V+W$. 

Let $U \in \Link_X(W) \cap F_{k-1}$. Since $U \in \Link_X(W)$, either $U \subsetneq W$ or $U \supsetneq W$.
If $U \subseteq W$, then $U \subseteq V+W$ as desired. If $U \supsetneq W$, then $U \in F_0$ since rank$(U)$ is too large for $U$ to be contained in $S_1, S_2, \dots, S_{k-1}$. Thus, $U$ is comparable to $V$. If $U \subseteq V$, then $U \subseteq V+W$. If $U \supseteq V$, then $U \supseteq V+W$ as desired.  We conclude that $V+W$ is a cone point of $\Link_X(W) \cap F_{k-1}$, hence $\Link_X(W) \cap F_{k-1}$ is contractible.

Now, $$F_{n-2} = \text{the full subcomplex of $X$ on vertices } \{W \mid  V+W \neq M\}$$ (this includes the cases $W\subseteq V$ and $V \subseteq W$) and we have shown this subcomplex is contractible. We begin our second sequence of applications of \autoref{Morse}. Let $G_0=F_{n-2}$. For $k \geq 1$ let
$$T_k = \left\{ W \;  \middle | \; \begin{array}{l}  $W$ \text{ a vertex of $X$},  \\  \rank(W)=n-k,  \,  W + V = M \end{array} \right\}.$$ 

Let $G_k$ be the full subcomplex of $X$ obtained by adding vertices in $T_k$ to $G_{k-1}$. Let $d=\rank(V)$, so $X=G_{d}$. Assume by induction that $G_{k-1}$ is contractible.  Since there are no edges between vertices of $T_k$, we may apply \autoref{Morse}, and show $G_k$ is contractible by verifying $ \Link_X(W) \cap G_{k-1} $ is contractible for all $W \in T_k$. 
Define the shorthand 
$$ \Link^>_X(W)  = \text{full subcomplex of $X$ on } \{ U \in X  \mid   U \supsetneq W\}  $$ $$ \Link^<_X(W)  = \text{full subcomplex of $X$ on }\{ U \in X \mid   U \subsetneq W\} $$ so we may write $\Link_X(W) $ as the join
$$\Link_X(W) =  \Link^>_X(W)  * \Link^<_X(W) .$$ 
Since $G_{k-1}$ is a full subcomplex, 
$$ \Link_X(W) \cap G_{k-1} \cong \big( \Link^>_X(W) \cap G_{k-1} \big) * \big(\Link^<_X(W) \cap G_{k-1} \big).  $$
To show that $\Link_X(W) \cap G_{k-1}$  is contractible it then suffices to show $\Link^<_X(W) \cap G_{k-1}$ is contractible. We first assume that $k<d$. 
Let $$f\colon \Link^<_X(W) \cap G_{k-1} \m \Link^<_X(W) \cap G_{k-1}$$ be given by $$U \mapsto U + (W \cap V).  $$ 
To check this function is well-defined, we will verify that $U + (W \cap V)$ is in $\Link^<_X(W)$.
We note that $U$ must be compatible with $\{V, W\}$. We must check that $U + (W \cap V) \subsetneq W$. By  \autoref{mapfdefined}, it suffices to show that 
 $U+V  \subsetneq M$.  Since $U \subsetneq W$, we know $\rank(U)<\rank(W)$. This implies $U$ cannot be contained in the set $T_i$ for $i \leq k-1$; $U$ must contained in $F_0$ or $S_1, \dots, S_{n-2}$. It follows that  $U+V  \subsetneq M$ as claimed. 


Since $f$ is an order-preserving self-map of a poset, it is a homotopy equivalence onto its image (see Quillen \cite[1.3 ``Homotopy Property"]{Quillen-Poset}). Its image is contractible since $W \cap V$ is a cone point. Here we are using the assumption $k<d$ to ensure $W \cap V \neq 0$. 


Finally, consider the case $k=d$. Recall this proof is by induction on $n=\rank(M)$. We will first verify the case $k=d$ in the base case, $\rank(M)=2$, by showing in this case $T_d = \varnothing$. Note in general if $T_d$ is empty, we have shown $X$ is contractible. 

If $\rank(M)=2$ and there existed some $W \in T_d$, necessarily $\rank(W)=1$. Then $V=V_0$ and each $\sigma_i$ must contain a complement $L_i$ of $V_0$ of rank $1$. Since $r \geq 2$ and the lines $L_1$ and $L_2$ must be both distinct and equal to $W$, we have a contradiction. 


Now suppose $\rank(M) > 2$ and $T_d$ is nonempty. Fix $W \in T_d$. We claim, 
\begin{equation}\label{ContainmentClaim}
\text{if $W\in T_d$ and $H$ is a vertex in $\sigma_i$ contained in a complement of $V_0$, then $H \subseteq W$.}
\end{equation} 
To see this, observe that for $W$ to be in $T_d$ and thus in $X$, it has to be compatible with $V$, $V_0$, and $H$. Because $V \oplus W = R^n$, all elements of a common basis of these four spaces have to be either in $V$ or in $W$. Because $H$ is contained in a complement of $V_0 \supseteq V$, all basis elements of $H$ have to be in $W$ and thus $H \subseteq W$.
We will use this claim twice. 

First, let $\hat \sigma$ and $\hat \tau_i$, respectively, be the simplices of ${\CB}(W)$ obtained by intersecting $W$ with all of the subspaces corresponding to vertices of $\sigma$ and $\tau_i$, respectively, after removing duplicates and zeros. Let $\hat \sigma_i = \hat \sigma \cup \hat \tau_i$.  By assumption the $\tau_i$ pairwise differ in a complement of some $V_j$. By Claim (\ref{ContainmentClaim})  these complements are equal to their own intersections with $W$. Thus the simplices $ \hat \tau_i$ are distinct. 

We  will now show that 
$$\Link^{<}_X(W) \cap G_{d-1} = \Link^{<}_X(W) \cong  T(W,\hat \sigma_1) \cap \dots \cap T(W,\hat \sigma_k).$$ 
To see this, first let $U \in \Link^{<}_X(W)$, i.e., $U$ is a summand of $W$ such that $\{U,W\}\cup \sigma_i$ are compatible for all $i$. Hence $U$ is compatible with $\hat \sigma_i$ for all $i$ since taking closure under intersection preserves compatibility (\autoref{CommonBasisClosure}). It follows that $$\Link^{<}_X(W) \subseteq  T(W,\hat \sigma_1) \cap \dots \cap T(W,\hat \sigma_k).$$
%
Now suppose $U \in T(W,\hat \sigma_1) \cap \dots \cap T(W,\hat \sigma_k)$, i.e., $U$ is a summand of $W$ that is compatible with $\hat \sigma_i$ for all $i$. Fix $i$. We can find a basis of $W$ that is compatible with $\hat \sigma_i$ and $U$. Because $W$ is compatible with $\sigma_i$, we can also find a basis of a complement of $W$ that is compatible with $\sigma_i$. The union of these bases is a common basis of $\{U,W\} \cup \sigma_i$, proving that $U \in \Link^{<}_X(W)$. 

The only thing that remains to check is that $\rank(W) \geq 2$. If $\rank(W)=1$, then $V=V_0$ and each $\sigma_i$ must contain a complement $L_i$ of $V_0$ of rank $1$. For $W$ to be compatible with $V_0$ and $L_i$, we must have that $W=L_i$ for all $i$ which is a contradiction. 

Since $ 2 \leq \rank(W) < \rank(M)$, the complex $$ \Link^{<}_X(W) \cap G_{d-1} \cong  T(W,\hat \sigma_1) \cap \dots \cap T(W,\hat \sigma_k)$$ is contractible by the inductive hypothesis. By \autoref{Morse}, this concludes the proof. 
\end{proof}

We now prove the analogue of \autoref{ContractibleTits} for higher buildings. 

\begin{proposition} \label{higherContractible}
Let $M$ be a free $R$-module with $R$ a PID.  Let $\tau=\{V_0,\dots,V_p\}$ be a simplex of $T(M)$ and let $\sigma$ be a simplex of ${\CB}(M)$ containing $\tau$. Let $\tau_1,\dots,\tau_k$ be distinct simplicies of $ST(M,\sigma)$ in the preimage of $\tau$ under the map $ST(M,\sigma) \m T(M,\sigma)$.  Let $\sigma_i$ be the simplex obtained by adding the vertices of $\tau_i$ to $\sigma$. If $k \geq 2$, $a \geq 1$ and $\rank (M) \geq 2$, then $$T^{a,b}(M,\sigma_1) \cap \dots \cap T^{a,b}(M,\sigma_k) $$ is contractible.



\end{proposition}

\begin{proof}
We prove this by induction on $a+b$. The case that $a+b=1$ (so necessarily $a=1, b=0$) is \autoref{ContractibleTits}. Assume $a+b \geq 2$ and the statement holds for all $a'+b'<a+b$ with $a'\geq 1$. Let 
$$X=T^{a,b}(M,\sigma_1) \cap \dots \cap T^{a,b}(M,\sigma_k)$$
 and let $F_i(X)$ be the subcomplex of $X$ consisting of simplices: 
$$ (V_{0,1} < \dots< V_{n_1,1}  ) * \dots * (V_{0,a} < \dots< V_{n_a,a}  ) * (M_{0,1},\dots,M_{m_1,1}  ) * \dots * (M_{0,b},\dots,M_{m_b,b}  ) $$
 with $m_b \leq i-1$ if $b \geq 1$ or $n_a \leq i-1$ if $b=0$. In other words, we filter 
$$X \subseteq \underbrace{T(M) * \dots * T(M)}_{a\ \text{times}} * \underbrace{ST(M) * \dots * ST(M)}_{b\ \text{times}}$$ 
by the skeletal filtration of the last term in the join (the last $ST(M)$ factor, or the last $T(M)$ factor if $b=0$). Then 
$$F_0(X) \cong  \left\{ \begin{array}{l}  T^{a,b-1}(M,\sigma_1) \cap \dots \cap T^{a,b-1}(M,\sigma_k), \text{ if $b> 0$} \\ T^{a-1,0}(M,\sigma_1) \cap \dots \cap T^{a-1,0}(M,\sigma_k), \text{ if $b= 0$}, \end{array}\right. $$ 
so $F_0(X)$ is contractible by induction on $a+b$. 

Fix $i>0$ and assume by induction on $i$ that $F_{i-1}(M)$ is contractible. Let $S_i$ denote the set of $(i-1)$-simplices of the last term of the join, i.e., the $(i-1)$-simplices of
%
$$   \left\{ \begin{array}{ll}   ST(M,\sigma_1) \cap \dots \cap ST(M,\sigma_k) , &\text{ if $b> 0$} \\ 
T(M,\sigma_1) \cap \dots \cap T(M,\sigma_k) , & \text{ if $b= 0$} . 
 \end{array}\right. $$ 
 %
Apply \autoref{Morse2} to the complex $F_i(X)$, subcomplex $F_{i-1}(X)$, and set $S_i$. Then \begin{align*}F_i(X) &\simeq F_i(X)/F_{i-1}(X)   \qquad \qquad \text{since $F_{i-1}(X) $ is a contractible subcomplex,} 
\\& \simeq  \bigvee_{\rho \in S_i} \Sigma^{\dim(\rho)+1} \big( \Link_{F_i(X)}(\rho) \big)
\\& \simeq  \left\{ \begin{array}{ll}   \bigvee_{\rho \in S_i} \Sigma^{i}  \Big( T^{a,b-1}(M,\sigma_1 \cup \rho) \cap \dots \cap T^{a,b-1}(M,\sigma_k \cup \rho)\Big) &\text{ if $b> 0$} \\[1em] 
 \bigvee_{\rho \in S_i} \Sigma^{i} \Big( T^{a-1,0}(M,\sigma_1 \cup \rho) \cap \dots \cap T^{a-1,0}(M,\sigma_k \cup \rho) \Big)
& \text{ if $b= 0$} . 
 \end{array}\right. \end{align*} 
Here $\sigma_j \cup \rho$ denotes the simplex obtained by taking the union of the vertices of $\sigma_j$ and $\rho$. By induction on $a+b$, the complexes 
$$ T^{a,b-1}(M,\sigma_1 \cup \rho) \cap \dots \cap T^{a,b-1}(M,\sigma_k \cup \rho) \text{ and } T^{a-1,0}(M,\sigma_1 \cup \rho) \cap \dots \cap T^{a-1,0}(M,\sigma_k \cup \rho) $$ 
are contractible. Thus, by induction on $i$, the complex $F_i(X)$ is contractible  for all $i$. Since $F_{\rank(M)-1}(X)=X$, we conclude $X$ is contractible as claimed.  
\end{proof}


\subsection{Proof of the comparison theorem}


In this section, we prove that the homology of $T^{a,b}(M)$ only depends on the quantity $a+b$ if $a \geq 1$. We begin observing the following elementary topological lemma, which follows from the Mayer--Vietoris spectral sequence. 

\begin{lemma} \label{LemmaWedge}
Let $X$ be a $CW$ complex with $X=\cup_{\alpha \in S} X_\alpha$ for some subcomplexes $\{X_{\alpha}\}_{\alpha \in S}$.  Assume all intersections of the form $$X_{\alpha_1} \cap \dots \cap X_{\alpha_k}$$ are acyclic provided $k \geq 2$ and each $\alpha_i$ is distinct. Assume $\bigcap_{\alpha \in S} X_{\alpha}$ is not empty and let $p \in \cap_{\alpha \in S} X_{\alpha}$. View $X_\alpha$ as a based space with $p$ as the basepoint. Then the map $$\bigvee_{\alpha \in S} X_\alpha \m X $$ is a homology equivalence.
\end{lemma}


We now prove the main result of this section which can be viewed as an unstable version of Waldhausen's additivity theorem. 


\begin{theorem} \label{SplitvsNotSplit}
Let $R$ be a PID and $M$ a finite-rank free $R$-module. Let $\sigma$ be a (possibly empty) simplex of ${\CB}(M)$. Let $f\colon T^{a,b}(M, \sigma) \m T^{a+b,0}(M, \sigma)$ denote the map induced by forgetting complements. 
For $a\geq 1$, the map $f$ is a homology equivalence. 
\end{theorem}

The map $f$ is likely a homotopy equivalence, but for simplicity we only prove this weaker statement. 


\begin{proof}[Proof of \autoref{SplitvsNotSplit}]
We will prove this theorem by induction on $a+b$. The statement is immediate if $b=0$ (in particular if $a+b \leq 1$) so assume otherwise.  Let $F_i(T^{a,b}(M,\sigma) )$ and $F_i(T^{a+b,0}(M,\sigma) )$, respectively, be the filtrations induced by the skeletal filtration on the final join factor $ST(M)$ and $T(M)$, respectively, as in the proof of \autoref{higherContractible}. Explicitly, $F_i(T^{a,b}(M,\sigma) )$ is the subcomplex of $T^{a,b}(M,\sigma)$ consisting of simplices
$$ (V_{0,1} < \dots< V_{n_1,1}  ) * \dots * (V_{0,a} < \dots< V_{n_a,a}  ) * (M_{0,1},\dots,M_{m_1,1}  ) * \dots * (M_{0,b},\dots,M_{m_b,b}  ) $$
with $m_b \leq i-1$. The filtration $F_i(T^{a+b,0}(M,\sigma) )$ is indexed analogously. 


Let $S_i^{a,b}$ denote the set of $(i-1)$-simplices of $ST(M,\sigma)$ and let $S_i^{a+b,0}$ denote the set of $(i-1)$-simplices of $T(M,\sigma)$. By \autoref{Morse2},
$$F_i( T^{a,b}(M,\sigma)  )/ F_{i-1}( T^{a,b}(M,\sigma)  ) \simeq \bigvee_{\rho \in S_i^{a,b}} \Sigma^{i} \Big( T^{a,b-1}(M,\rho \cup \sigma) \Big)  \text{ and}$$
$$ F_i( T^{a+b,0}(M,\sigma)  )/ F_{i-1}( T^{a+b,0}(M,\sigma) \simeq \bigvee_{\rho \in S_i^{a+b,0}} \Sigma^{i} \Big( T^{a+b-1,0}(M,\rho \cup \sigma)\Big).$$ Note the distinct index sets of the wedges. 

We will prove that $f\colon T^{a,b}(M, \sigma) \m T^{a+b,0}(M, \sigma)$ is a homology equivalence by checking that $f$ induces a homology equivalence on the associated graded of this filtration.  To show $$ F_i( T^{a,b}(M,\sigma)  )/ F_{i-1}( T^{a,b}(M,\sigma)  ) \m F_i( T^{a+b,0}(M,\sigma)  )/ F_{i-1}( T^{a+b,0}(M,\sigma))$$ is a homology equivalence, it suffices to show  $$ f_{\tau}\colon  \bigvee_{\rho \in f^{-1}(\tau)} T^{a,b-1}(M,\rho \cup \sigma) \m T^{a+b-1,0}(M,\tau \cup \sigma) $$ is a homology equivalence for all $\tau \in S_i^{a+b,0}$. By our induction hypothesis, it suffices to show $$ \hat{f}_{\tau}\colon  \bigvee_{\rho \in f^{-1}(\tau)} T^{a+b-1,0}(M,\rho \cup \sigma) \m T^{a+b-1,0}(M,\tau \cup \sigma) $$ is a homology equivalence. We will show that $\hat{f}_{\tau}$ is an equivalence using \autoref{LemmaWedge} applied to a cover of $T^{a+b-1,0}(M,\tau \cup \sigma)$. We cover $T^{a+b-1,0}(M,\tau \cup \sigma)$ by subspaces of the form $T^{a+b-1,0}(M,\rho \cup \sigma) $ for $\rho \in f^{-1}(\tau)$. This is a cover because whenever a collection of subspaces are compatible with a flag $\tau$, then the collection is compatible with some splitting $\rho \in f^{-1}(\tau)$. In other words, we can choose complements to the terms in the flag $\tau$ compatible with the collection of subspaces. 


\autoref{higherContractible} implies that finite intersections of subcomplexes in the cover are contractible. To see that $$ \bigcap_{\rho \in f^{-1}(\tau)}  T^{a+b-1,0}(M,\rho \cup \sigma)   $$ is not empty, observe that $\tau$ is a simplex of the intersection. \autoref{LemmaWedge} now implies that $\hat{f}_{\tau}$ and hence $f$ is a homology equivalence. 
%
%
\end{proof}


\section{Algebraic properties of higher Tits buildings}


In this section, we describe spaces $D^{a,b}(M)$ which are homotopy equivalent to iterated suspensions of the complexes $T^{a,b}(M)$ studied in the previous section. The reason for considering two different models of higher Tits buildings is that the complexes $T^{a,b}(M)$ are smaller and hence more convenient for combinatorial Morse theory while the complexes $D^{a,b}(M)$ are larger and have better algebraic properties. We begin with categorical preliminaries.

\subsection{Categorical framework} 
 
In this subsection, we recall basic facts about Day convolution and bar constructions. Throughout, we will fix a PID $R$.

\begin{definition}

Let $\VB$ denote the groupoid with objects finite-rank free $R$-modules and morphisms all $R$-linear isomorphisms.  A \emph{$\VB$-module} is a functor from $\VB$ to the category $\Ab$ of $\Z$-modules. Let $\Mod_{\VB}$ denote the category of $\VB$-modules. Similarly define $\VB$-sets, $\VB$-based sets, $\VB$-chain complexes, and $\VB$-based spaces.
\end{definition}

The notation $\VB$ is used in the field of representation stability to connote ``vector spaces" and ``bijective maps" \cite{PutmanSam}. We sometimes use the notation $\VB(R)$ to stress the dependence of the groupoid on $R$.


Given a functor $V$ from $\VB$ to some category and a finite-rank free $R$-module $M$, let $V(M)$ denote the value of $V$ on $M$. Let $V_n$ denote $V(R^n)$. Let $\cC=\Ab$, ${\Set_*}$, $\Ch$, or ${\Top_*}$.  The category of functors $\Fun(\VB,\cC)$ has a symmetric monodial product called \emph{Day convolution} or induction product. 



\begin{definition} \label{Day}
 Let $\cC= \Ab$, ${\Set_*}$, $\Ch$, or ${\Top_*}$ and let $V,W \in \Fun(\VB,\cC)$. Let $\otimes_{\cC}$ denote tensor product if $\cC$ is $\Ab$ or $\Ch$, and smash product if $\cC$  is ${\Set_*}$ or ${\Top_*}$.   Let $V \otimes_{\VB} W \in \Fun(\VB,\cC)$ be defined by  
 $$(V \otimes_{\VB} W)(M):= \coprod_{A \oplus B=M} V(A) \otimes_{\cC} W(B)$$ on objects and defined on morphisms in the obvious way.
\end{definition}


In the case of $\VB$-modules,  $$(V \otimes_{\VB} W)_n \cong \bigoplus_{a +b=n} \Ind^{\GL_n(R)}_{\GL_a(R) \times \GL_b(R)}V_a \otimes W_b.$$ The product $\otimes_{\VB}$ gives a symmetric monodial structure on $\Fun(\VB,\cC)$ with the following object serving as the unit.


\begin{notation} Let $\cC= \Ab$, ${\Set_*}$, $\Ch$, or ${\Top_*}$. Given an object $X$ of $\mathcal{C}$, we may view $X$ as a functor in $\Fun(\VB, \mathcal{C})$ that takes the value $X$ on modules of rank $0$ and takes the zero object of $\mathcal{C}$ elsewhere. With this convention, the unit of $(\mathcal{C}, \otimes_{\mathcal{C}})$ defines a unit in $(\Fun(\VB,\cC), \otimes_{\VB})$. 
 
 We denote this unit generically by ${\bf 1}$. We may write $\Z$ for the unit when $\cC= \Ab$ or $\cC=\Ch$. We may write $S^0$ for the unit when $\cC={\Set_*}$ or $\cC={\Top_*}$. 
 \end{notation}



This symmetric monoidal structure  $(\Fun(\VB,\cC), \otimes_{\VB}, {\bf 1})$  allows us to make sense of monoid objects, right/left-modules over monoids, etc. Note that $\mathbf 1$ is a monoid object. If $\cC$ is $\Ab$ or $\Ch$, we often call monoid objects \emph{$\VB$-rings}.

\begin{definition}
Let $A$ be a monoid in $\Mod_{\VB}$ with $M$ a right $A$-module and $N$ a left $A$-module. Let $M \otimes_A N$ be the coequalizer of the two natural maps $$ M \otimes_{\VB} A \otimes_{\VB} N \rightrightarrows M \otimes_{\VB} N.$$ Let $\Tor^i_A(\;\cdot\;,N)$ denote the $i$th left-derived functor of $\cdot \otimes_A N$.
\end{definition}


These $\Tor$ groups can be used to formulate the notion of Koszul rings.

\begin{definition} \label{DefnAugmented}

Let $\cC= \Ab$, ${\Set_*}$, $\Ch$, or ${\Top_*}$. We say a monoid $A$ in $\Fun(\VB, \cC)$ is \emph{augmented} if it is equipped with a map of monoids $A \m {\bf 1}$ that is an isomorphism in $\VB$-degree $0$. 
\end{definition}


\begin{definition} \label{DefnKoszul}
An augmented $\VB$-ring $A$ in $\Mod_{\VB}$ is called \emph{Koszul} if $$\Tor^A_i(\Z,\Z)_n \cong 0 \qquad \text{ for $n \neq i$.} $$   In this case, we define the \emph{Koszul dual} of $A$ to be the $\VB$-module with value $\Tor^A_n(\Z,\Z)_M$ on a rank-$n$ free $R$-module $M$. 
\end{definition}





As in classical algebra, $\Tor$ groups can be computed via bar constructions.

\begin{definition}
Let $\cC= \Ab$, ${\Set_*}$, $\Ch$, or ${\Top_*}$. Let $A$ be a monoid in $\Fun(\VB,\cC)$, $M$ a right $A$-module and $N$ a left $A$-module. Let $B_\bullet(M,A,N)$ be the simplicial object in $\Fun(\VB,\cC)$ with $p$-simplices $$B_p(M,A,N):=M \otimes_{\VB} A^{\otimes_{\VB}p} \otimes_{\VB} N,$$ face maps induced by the monoid and module multiplication maps \begin{align*} M \otimes_{\VB} A &\m M, \\  A \otimes_{\VB} A& \m A,\\ A \otimes_{\VB} N &\m N, \end{align*} and degeneracies induced by the unit of $A$. 

For $\cC=\Ch$, let $B(M,A,N) \in \Fun(\VB,\Ch)$ denote the total complex of the double complex of normalized chains associated to the simplicial object  $B_\bullet(M,A,N)$. For $\cC=\Ab$, we make the same definition by viewing the abelian groups as chain complexes concentrated in homological degree 0. When $\cC$ is ${\Set_*}$  or ${\Top_*}$, let $B(M,A,N) \in \Fun(\VB,\cC)$ denote the thin geometric realization of $B_\bullet(M,A,N)$.

% See https://people.math.binghamton.edu/malkiewich/hocolim_bar.pdf page 9. 
% Geometric realizattion is the same in based and unbased spaces. The result is based. 

\end{definition}


The following result appears in Miller--Patzt--Petersen \cite[Proposition 2.33]{MillerPatztPetersen}. 

\begin{proposition}  Let $\cC=\Ab$. Then $$H_i(B(M,A,N)) \cong \Tor_i^A(M,N)$$ if either $M_n$ is a free abelian group for all $n$ or $N_n$ is a free abelian group for all $n$. 
\end{proposition}

\begin{definition}

 Let $\cC=\Ab$, ${\Set_*}$,  $\Ch$, or ${\Top_*}$.  Let $A$ be an augmented monoid in $\Fun(\VB,\cC)$. Let $BA_{\bullet} :=B_{\bullet}({\bf 1},A,{\bf 1})$ and $BA :=B({\bf 1},A,{\bf 1})$.
\end{definition} 




\begin{lemma} \label{Bconn}
Let $A$ be a monoid object in  $\VB$-based spaces. Let $\beta \geq -2$. If $A(R^n)$  is $(\alpha n+\beta)$-connected for all $n$ then $BA(R^n)$ is $(\alpha n+\beta+1)$--connected for all $n$.
\end{lemma}

\begin{proof} Observe $BA_{p}(R^n)$ is 
$$ \bigvee_{n_1 + n_2+ \dots + n_p=n} \; \; \bigwedge_{i=1, \dots, p} A_{n_i}.$$
When $p=0$, this is contractible, and hence $(\alpha n+\beta+1)$-connected. In general, $BA_{p}(R^n)$  is  $(\alpha n+p\beta+p-1)$-connected, and $(\alpha n+p\beta+p-1) \geq (\alpha n+\beta+1-p)$ for $p>0$. 
\end{proof} 


If $A$ is a commutative monoid object, then so is $BA$. Thus, we can iterate the bar construction.



\subsection{Simplicial models of higher buildings}

In this section, we introduce simplicial complexes $D^{a,b}(M)$, and prove they are homotopy equivalent to iterated suspensions of the complexes $T^{a,b}(M)$.
\begin{definition}
Let $L_\bullet$ be the simplicial $\VB$-set with value on a finite-rank free $R$-module $M$ given by the set of flags $V_0 \subseteq \dots \subseteq V_p $ of (not necessarily distinct, not necessarily proper, not necessarily nonzero) subspaces of $M$. Let $s_i\colon L_p \m L_{p+1}$ be the map that duplicates the $i$th subspace. Let $d_i\colon L_p \m L_{p-1}$ be the map that forgets the $i$th subspace.  Let $\mathring{L}_\bullet$ denote the sub-simplicial $\VB$-set of flags $V_0 \subseteq \dots \subseteq V_p $ with $V_0 \neq 0$ or $V_p \neq M$.  
\end{definition}

We use the notation $L_\bullet$ to connote ``lattices"  (in the sense of Rognes \cite{Rog1}), and $\mathring{L}_\bullet$  are referred to as ``not full lattices". We now define their split versions. 

\begin{definition}
Let $SL_\bullet$ be the simplicial $\VB$-set with value on a free $R$-module $M$ given by splittings $(M_0 , \dots , M_{p+1})$ of $M$ into (not necessarily proper or nonzero) subspaces. Let $s_i\colon SL_p \m SL_{p+1}$ be the map that inserts the zero subspace between $M_i$ and $M_{i+1}$. Let $d_i\colon SL_p \m SL_{p-1}$ be the map that replaces $M_i$ and $M_{i+1}$ with $M_i \oplus M_{i+1}$. Let $S\mathring{L}_\bullet$ denote the sub-simplicial $\VB$-set of splittings $(M_0, \dots,  M_{p+1} )$ with $M_0 \neq 0$ or $M_{p+1} \neq 0$.
\end{definition}

\begin{definition}
Let $L^{a,b}_{\bullet,\dots,\bullet}$ be the $(a+b)$-fold simplicial $\VB$-set  whose value on a free $R$-module $M$ and a tuple $(p_1,\dots,p_{a+b})$ is the subset of $$L_{p_1} \times \dots \times L_{p_a} \times SL_{p_{a+1}} \times \dots \times SL_{p_{a+b}}$$ of subspaces that together have a common basis. The simplicial structure is induced from the simplicial structures on $L_\bullet$ and $SL_{\bullet}$. 
\end{definition}

\begin{definition}
Let $\mathring{L}^{a,b}_{\bullet,\dots,\bullet}$ be the $(a+b)$-fold sub-simplicial $\VB$-set of $L^{a,b}_{\bullet,\dots,\bullet}$ whose $p$-simplices are contained in the $p$-simplices of 
$$L_1 \times \dots \times L_{n_{i-1}} \times \mathring{L}_{n_i} \times  L_{n_{i+1}} \times  \dots   \times L_{n_a} \times SL_{n_{a+1}} \times \dots \times SL_{n_{a+b}}$$ for some $1 \leq i \leq a$ or  $$L_1 \times \dots \times   L_{n_a} \times SL_{n_{a+1}} \times \dots  \times SL_{n_{i-1}} \times S\mathring{L}_{n_i} \times  SL_{n_{i+1}} \times \dots \times SL_{n_{a+b}}$$ for some $(a+1) \leq i \leq (a+b)$. Let $D^{a,b}_{\bullet,\dots,\bullet}$ be the $(a+b)$-fold simplicial $\VB$-based set $$L^{a,b}_{\bullet,\dots,\bullet} / \mathring{L}^{a,b}_{\bullet,\dots,\bullet}$$ with basepoint given by the image of $\mathring{L}^{a,b}_{\bullet,\dots,\bullet}$.



\end{definition}

Given a multi-simplicial space $X_{\bullet,\dots,\bullet}$, we let $X_\bullet$ denote the diagonal and let $X$ denote the thin geometric realization. 

\begin{definition}  \label{Dab}
Let $$\mu_{n_1,\dots,n_{a+b}}\colon  D^{a,b}_{n_1,\dots,n_{a+b}}(M) \wedge D^{a,b}_{n_1,\dots,n_{a+b}}(N) \longrightarrow D^{a,b}_{n_1,\dots,n_{a+b}}(M \oplus N)$$ be the map defined factorwise  by the formula 
$$(V_0 \subseteq \dots \subseteq V_p) \times (W_0 \subseteq \dots \subseteq W_p) \longmapsto (V_0 \oplus W_0 \subseteq \dots \subseteq V_p \oplus W_p)$$ on $L_\bullet$ factors and defined by $$(M_0, \dots, M_p) \times (N_0, \dots, N_p) \longmapsto (M_0 \oplus N_0, \dots, M_p \oplus N_p)$$ on $SL_\bullet$ factors.
\end{definition}

The complex $D^{a,0}$ (which we referred to as $D^a$ in the introduction) is Rognes' higher building \cite{Rog1}. The complex $D^{0,b}$ is Galatius--Kuper--Randal-Williams' split version \cite{e2cellsIV}. The fact that these complexes agree with those of Rognes and Galatius--Kuper--Randal-Williams is not immediate but follows quickly from Galatius--Kuper--Randal-Williams \cite[Lemma 5.6 and Lemma 5.7]{e2cellsIV}.






\begin{lemma}
The maps $\mu_{n_1,\dots,n_{a+b}} $ assemble to form a simplicial map:  $$\mu_{\bullet,\dots,\bullet}\colon  D^{a,b}_{\bullet,\dots,\bullet}(M) \wedge D^{a,b}_{\bullet,\dots,\bullet}(N) \m D^{a,b}_{\bullet,\dots,\bullet}(M \oplus N).$$
\end{lemma}





Observe that $D^{a,b}_{0,\dots,0}(M)$ is  the basepoint for all $M\neq 0$. By convention, $D^{a,b}_{0,\dots,0}(0)$ is $S^0$ since $L^{a,b}_{0,\dots,0}(0)$ is a point and $\mathring{L}^{a,b}_{0,\dots,0}(0)$ is empty; similarly for the split versions of these complexes.
In particular, there is a natural isomorphism $\iota \colon  {\bf 1} \m  D^{a,b}_{0,\dots,0}$. 

\begin{definition}
Let ${\bf 1}_{\bullet,\dots,\bullet} $ be the constant $(a+b)$-fold simplicial  $\VB$-based set on ${\bf 1}$.  Let $\iota_{\bullet,\dots,\bullet} \colon {\bf 1}_{\bullet,\dots,\bullet}  \m  D^{a,b}_{\bullet,\dots,\bullet}$ be the unique extension of the natural map $\iota \colon  {\bf 1} \m  D^{a,b}_{0,\dots,0}$. 
\end{definition}

 The next two lemmas are straight-forward to verify. Compare \autoref{GKRW6.6} to Galatius--Kupers--Randal-Williams \cite[Lemma 6.6]{e2cellsIV}. \autoref{GKRW6.6} follows from associativity and commutativity of direct sums. \autoref{Bsplit} follows from the fact that sum and intersection operation are well-behaved for modules with the common basis property; see (for example) \autoref{CommonBasisIntersectionSum}. 

\begin{lemma} \label{GKRW6.6}
The maps $\mu_{\bullet,\dots,\bullet}$ and $\iota_{\bullet,\dots,\bullet} $ give $D^{a,b}_{\bullet,\dots,\bullet}$ the structure of a commutative monoid object in the category of $(a+b)$-fold simplicial $\VB$-based sets.
\end{lemma}

\begin{lemma} \label{Bsplit}
There is a natural isomorphism of $(a+b+1)$-fold simplicial $\VB$-based sets $B_\bullet D^{a,b}_{\bullet,\dots,\bullet} \cong D^{a,b+1}_{\bullet,\dots,\bullet}$.
\end{lemma}





\subsection{Comparing models of higher buildings}

We now show $D^{a,b}(M)$ is an iterated suspension of $T^{a,b}(M)$. This fact appeared in Galatius--Kupers--Randal-Williams \cite{e2cellsIV} in the case $(a,b)$ is $(1,0)$ or $(2,0)$.


\begin{lemma} \label{suspend}
$D^{a,b}(M) \simeq \Sigma^{a+b+1} T^{a,b}(M) $.
\end{lemma}

\begin{proof}
Note that the diagonal of $L^{a,b}_{\bullet}(M)$ admits two extra degeneracies. These are induced by the extra degeneracies of $L_\bullet(M)$ given by $$(V_0 \subseteq \dots \subseteq V_p ) \mapsto (0 \subseteq V_0 \subseteq \dots \subseteq V_p )   $$ and $$(V_0 \subseteq \dots \subseteq V_p ) \mapsto ( V_0 \subseteq \dots\subseteq V_p \subseteq M)   $$  and the extra degeneracies of $SL_\bullet(M)$ given by $$(M_0  , \dots , M_p ) \mapsto (0 , M_0 , \dots , M_p )   $$ and $$(M_0  , \dots , M_p ) \mapsto ( M_0 , \dots , M_p,0 ).   $$ In particular, $L^{a,b}(M)$ is contractible (see e.g. Goerss--Jardine \cite[Lemma 5.1]{GoerssJardine}). Since $\mathring{L}^{a,b}(M) \m L^{a,b}(M)$ is a cofibration and $D^{a,b}(M)= L^{a,b}(M)/\mathring{L}^{a,b}(M)$, we conclude that $D^{a,b}(M) \simeq \Sigma (\mathring{L}^{a,b}(M))$.
 

There are inclusions of spaces $T(M) \hookrightarrow \mathring{L}(M)$ and $ST(M) \hookrightarrow S\mathring{L}(M)$.
For $0 \leq k \leq (a+b)$, let $F_k$ be the subcomplex of $\mathring{L}^{a,b}(M)$ where the first $k$ $L(M)$ and $SL(M)$ factors are in $T(M)$ or $ST(M)$, i.e., 
for each $i\leq k$ the $i$th factor has the form 
$$(V_0 \subseteq \dots \subseteq V_p)  \text{ with }  V_0 \neq 0 \text{ and } V_p \neq M \text{ (but inclusions may not be strict)}$$
or 
$$(M_0,\dots,M_p) \text{ with }  M_0, M_p \neq 0  \text{ (but other terms $M_i$ may be 0)}.$$

 We will show that $F_{k-1}$ is the suspension of $F_k$ by defining contractible subspaces $N_k$ and $S_k$ of $F_{k-1}$ that we view as the `northern' and `southern' hemispheres, which intersect in the `equator' $F_k$. Let $N_k$ be subcomplex of $F_{k-1}$ where the $k$th factor is of the form $$(V_0 \subseteq \dots \subseteq V_p) \text{ with } V_p \neq M $$ with $k \leq a$, and is the form $$(M_0,\dots,M_p) \text{ with } M_p \neq 0$$ if $k>a$.  Let $S_k$ be subcomplex of $F_{k-1}$ where the $k$th factor is of the form $$(V_0 \subseteq \dots \subseteq V_p)  \text{ with } V_0 \neq 0 $$ if $k \leq a$, and is the form $$(M_0,\dots,M_p) \text{ with } M_0 \neq 0$$ if $k>a$. Then $F_{a+b} \cong T^{a,b}(M)$, $F_0 \cong D^{a,b}(M)$,   $S_k \cap N_k \cong F_k$, and $S_k \cup N_k \cong F_{k-1}$. Note that $S_k$ and $N_k$ are contractible since they have cone points. Thus $F_k \simeq \Sigma F_{k+1}$. This implies $\mathring{L}^{a,b}(M) \simeq \Sigma^k T^{a,b}(M)$. Since $D^{a,b}(M) \simeq \Sigma (\mathring{L}^{a,b}(M))$, the claim follows. 
\end{proof}

Since the suspension of a homology equivalence gives a homotopy equivalence, \autoref{SplitvsNotSplit} and \autoref{suspend} imply the following. 

\begin{corollary} \label{SplitvsNotSplitD}
Let $a \geq 1$. Then the `forget the complement' map $D^{a,b}(M) \m D^{a+b,0}(M)$ is a homotopy equivalence. 
\end{corollary}

We now prove \autoref{mainLemma} which states that $B D^{k,0}(M) \simeq D^{k+1,0}(M)$.

\begin{proof}[Proof of \autoref{mainLemma}]  By \autoref{Bsplit}, there is a natural isomorphism of $(k+1)$-fold simplicial $\VB$-based sets $B_\bullet D^{k,0}_{\bullet,\dots,\bullet} \cong D^{k,1}_{\bullet,\dots,\bullet}$. \autoref{mainLemma} then follows from \autoref{SplitvsNotSplitD}. 
\end{proof}



\section{Rognes' connectivity conjecture}

In this section, we prove Rognes' connectivity conjecture in the case of fields, \autoref{ConnectivityThm}.


\subsection{Comparing $T^{k,0}_n(R)$ and ${\CB}_n(R)$}

In this subsection, we recall maps between $T^{k,0}_n(R)$ for different values of $k$ and also a map to ${\CB}_n(R)$.

\begin{definition}
Given an injection $f\colon \{1,\dots,k\} \m \{1,\dots,j\}$, let $f \colon  T_n^{k,0}(R) \m T_n^{j,0}(R)$ denote the map $$\underbrace{\emptyset * \dots * \emptyset}_{i-1} * V * \emptyset* \dots * \emptyset \longmapsto \underbrace{\emptyset * \dots * \emptyset}_{f(i)-1} * V * \emptyset* \dots * \emptyset.$$  Let $\iota \colon T_n^{k,0}(R) \m T_n^{k+1,0}(R) $ be the map induced by the standard injection.

\end{definition}

\begin{definition}
Let $\pi \colon  T_n^{k,0}(R) \m {\CB}_n(R)$ be the map  $$\emptyset * \dots * \emptyset * V * \emptyset* \dots * \emptyset \longmapsto V.$$  

\end{definition}

Note that $\pi$ commutes with $\iota$ so we obtain a map $$ \pi\colon    \colim_k T_n^{k,0}(R) \m {\CB}_n(R).$$ Since the maps $\iota$ are inclusions of simplicial complexes, this colimit agrees with the homotopy colimit. The following is a desuspension of Rognes' result \cite[Lemma 14.6]{Rog1}.

\begin{proposition} \label{colimTD}
The map $$ \pi\colon  \colim_k T_n^{k,0}(R) \m {\CB}_n(R) $$ is a homotopy equivalence.


\end{proposition}

\begin{proof}
Let  $\sigma=[V_0,\dots,V_p]$ in ${\CB}_n(R)$ be a simplex of ${\CB}_n(R)$. It suffices to show $\pi^{-1}(\sigma)$ is weakly contractible. Let $g\colon S^d \m \pi^{-1}(\sigma)$ be a simplicial map from some simplicial structure on the $d$-sphere $S^d$. Since $S^d$ is compact, the image of $g$ must be contained in the image of $\iota\colon  T^{m,0}_n(R) \m \colim_k T_n^{k,0}(R)$ for some $m$. The map $g$ has image in the star in $\pi^{-1}(\sigma)$ of $$\underbrace{\emptyset * \dots * \emptyset}_{m} * V_0 * \emptyset* \dots $$ and hence $g$ is nullhomotopic.
\end{proof}

\subsection{The fundamental group of $T^{k,0}_n(F)$ and ${\CB}_n(F)$}

Let $F$ be a field. Our goal is to show $T^{k,0}_n(F)$ and ${\CB}_n(F)$ are simply connected if $k \geq 2$ and $n \geq 3$. The following is implicit in Galatius--Kupers--Randal-Williams \cite{e2cellsIV}.

\begin{lemma} \label{lemJoin}
For $F$ a field, the inclusion $T^{2,0}_n(F) \m T_n(F) * T_n(F)$ is an isomorphism. 
\end{lemma}

\begin{proof}
This is just the classical fact that the union of any pair of flags admits a common basis, a fact used to prove the Tits building is a building. 
\end{proof}

\begin{corollary}\label{CorT2Conn}
For $F$ a field, $T^{2,0}_n(F)$ is $(2n-4)$-connected.
\end{corollary}
\begin{proof}
This follows from \autoref{lemJoin} and the Solomon--Tits theorem which states that $T_n(F)$ is $(n-3)$-connected.
\end{proof}

\begin{proposition} \label{badness} 
Let $F$ be a field. For $k \geq 2$ and $n \geq 3$, $T^{k,0}_n(F)$ and ${\CB}_n(F)$ are $1$-connected.  
\end{proposition}

\begin{proof}
By \autoref{colimTD}, it suffices to prove the claim for  $T^{k,0}_n(F)$.  Fix $k \geq 2$ and $n \geq 3$. Note that by the proof of  \autoref{lemJoin}, any two vertices in different join factors are connected by an edge. Since we assume $k \geq 2$, this implies $T^{k,0}_n(F)$ is connected. Now we will show that it is simply connected. Consider a simplicial map $$g \colon  X \m T^{k,0}_n(F)$$ with $X$ some simplicial complex structure on $S^1$. If we can show that $g$ is homotopic to a map that factors through $T^{2,0}_n(F)$, then by \autoref{CorT2Conn} we can conclude that $g$ is nullhomotopic and hence that every path component of $T^{k,0}_n(F)$ is $1$-connected. We will prove this  by showing that $g$ is homotopic to a map that factors through $T^{1,0}_n(F)$.

Call a simplex of $X$ \emph{bad} if  none of its vertices map to the image of $T^{1,0}_n(F)$. We are done if we can homotope $f$ to have no bad simplices. Suppose $\sigma=[x_0,x_1]$ is a bad edge. Let $\iota_j$ be the injection from $\{1\}$ to $\N$ with image $j$. Let $V_0, V_1$ be subspaces of $F^n$ and $n_0, n_1$ be numbers with $$g(x_0)=\iota_{n_0}(V_0) \qquad \text{and} \qquad g(x_1)=\iota_{n_1}(V_1) .$$ Since $\sigma$ is bad, $n_0$ and $n_1$ are larger than $1$. Observe that $g(\sigma)$ is contained in $\Link_{T^{k,0}_n(F)}(\iota_1(V_0))$. Let $X'$ be the subdivision of $X$ with a new vertex $t$ in the middle of $\sigma$. Let $g'\colon X' \m T^{k,0}_n(F)$ be the map that sends $t$ to $\iota_1(V_0)$ and that agrees with $g$ elsewhere. Since $[\iota_1(V_0),g(x_0),g(x_1)]$ is a simplex of $T^{k,0}_n(F)$, $g$ is homotopic to $g'$. By iterating this procedure, we can find a homotopic map $$g''\colon X'' \m T^{k,0}_n(F)$$ with $X''$ a new simplicial structure on $S^1$ and $g''$ having no bad $1$-simplices.  Let $y_1 \in X''$ be a bad vertex and let $g''(y_1)=\iota_j(V_1)$. Note that $j \geq 2$. Let $y_0$ and $y_2$ be the vertices adjacent to $y_1$ in $X''$. There are subspaces $V_0$ and $V_2$ with  $g''(y_0)=\iota_1(V_0)$ and $g''(y_2)=\iota_1(V_2)$. Since $n \geq 3$, the complex $T_n(F)$ is connected. Thus there is a simplicial structure $Y$ on $[0,1]$ and a simplicial map $h\colon Y \m T_n(F)$ with $h(0)=V_0$ and $h(1)=V_2$. Let $X'''$ be the subdivision of $X''$ where we replace $[y_0,y_1] \cup [y_1,y_2]$ with $Y$ and let $g'''\colon X''' \m T^{k,0}_n(F)$ be defined to agree with $g''$ on vertices of $X''$ and to equal $\iota_1 \circ h$ on $Y$. Since any pair of flags are compatible, the image of $\iota_1 \circ h$ is contained in $\Link_{T^{k,0}_n(F)} (\iota_j(V_1))$.  Thus, $g'''$ and $g''$ are homotopic. We have removed one bad vertex. Iterating this procedure produces the desired homotopy. \end{proof}

\subsection{High connectivity of ${\CB}_n(F)$}

We first prove $D_n^{k,0}(F)$ is highly connected.

\begin{proposition} \label{DRognes}
Let $F$ be a field. For $k \geq 2$, $D_n^{k,0}(F)$ is $(2n+k-3)$-connected. 
\end{proposition}

\begin{proof}
We will prove the claim by induction on $k$. By \autoref{suspend} and  \autoref{CorT2Conn},  $$D_n^{2,0}(F) \simeq \Sigma^3 (T^2_n(F))$$   is $(2n-1)$-connected. This proves the base case. Assume we have proven that $D_n^{k-1,0}(F)$ is $(2n+k-4)$-connected. By \autoref{Bconn}, $BD_n^{k-1,0}(F)$ is $(2n+k-3)$-connected. By \autoref{Bsplit}, $BD_n^{k-1,0}(F) \cong D_n^{k-1,1}(F)$. By \autoref{SplitvsNotSplitD},  $D_n^{k-1,1}(F) \m D_n^{k,0}(F)$ is a homotopy equivalence and so $D_n^{k,0}(F)$ is $(2n+k-3)$-connected. The claim follows by induction.
\end{proof}

We now prove the following which includes the statement of \autoref{ConnectivityThm}.

\begin{theorem}\label{maingeneral}
For $k \geq 2$ and $F$ a field, $T_n^k(F)$ and ${\CB}_n(F)$ are $(2n-4)$-connected.
\end{theorem}
\begin{proof}
Note that the claim is vacuous for $n=1$ and $n=0$. Since $D_n^{k,0}(F)$ is $(2n+k-3)$-connected by \autoref{DRognes}   and $\Sigma^{k+1} T_n^{k,0} \simeq D_n^{k,0}(F)$ by \autoref{suspend},  $$\widetilde H_i(T_n^{k,0}(F)) \cong 0 \qquad \text{ for $i \leq 2n-4$.}$$ This implies $T_2^k(F)$ is $0$-connected since reduced homology detects $0$-connectivity. Now consider the case $n \geq 3$. By \autoref{badness} the space $T_n^{k,0}(F)$ is simply-connected for $n \geq 3$ and $k \geq 2$, so the Hurewicz theorem implies $T_n^k(F)$ is $(2n-4)$-connected. Since ${\CB}_n(F)$ is the homotopy colimit over $k$ of the spaces $T_n^k(F)$, we deduce that ${\CB}_n(F)$ is also $(2n-4)$-connected.
\end{proof}

\begin{remark}
Using Galatius--Kupers--Randal-Williams \cite[Theorem 7.1 (ii)]{e2cellsIV} instead of \autoref{CorT2Conn}, it is possible to prove a version of \autoref{maingeneral} for fields replaced with semi-local PIDs with infinite residue fields (e.g. power series rings of infinite fields in one variable).



\end{remark}


\section{The Koszul dual of the Steinberg monoid} 


We now recall the definition of the Steinberg monoid and compute its Koszul dual.

\begin{definition}
For $R$ a PID, let $\St(R)$ be the $\VB(R)$-ring with $\St(R)(M)=\widetilde H_{\rank(M)}(D^{1,0}(M))$ and with ring structure induced by the monoid structure on $D^{1,0}$. 
\end{definition} 

Note that the ring structure on $\St(R)$ was original introduced by Miller--Nagpal--Patzt \cite[Section 2.3]{MNP}. The definition given here is due to Galatius--Kupers--Randal-Williams \cite[Lemma 6.6]{e2cellsIV} and agrees with that of Miller--Nagpal--Patzt  by Galatius--Kupers--Randal-Williams \cite[Remark 6.7]{e2cellsIV}.

The monoid $\St(R)$ is augmented in the sense of \autoref{DefnAugmented} since $\St(R)_0 \cong \Z$. 

Recall from \autoref{DefnKoszul} that, given an augmented $\VB$-ring $A$, we say $A$ is \emph{Koszul} if $\Tor_i^{A}(\Z,\Z)_n \cong 0$ for $i \neq n$, and its Koszul dual is the  $\VB$-module $$M \longmapsto \Tor_{\rank M}^{A}(\Z,\Z)_M.$$
It was shown by Miller--Nagpal--Patzt \cite[Theorem 1.4]{MNP} that $\St(F)$ is Koszul if $F$ is a field. We give a new proof of this and compute its Koszul dual. 

\begin{lemma} \label{TorD}
For $R$ a PID, there is a natural isomorphism $\Tor_i^{\St(R)}(\Z,\Z)_n \cong \widetilde H_{i+n}(D^{2,0}_n(R))$.    
\end{lemma}

\begin{proof}
Recall from \autoref{Bsplit} and \autoref{SplitvsNotSplitD}, $$B D^{1,0}_n(R) \cong D^{1,1}_n(R) \simeq D^{2,0}_n(R).$$  Since $D^{1,0}_n(R) \simeq \Sigma^2 T^{1,0}_n(R)$ by \autoref{suspend}, the Solomon--Tits theorem states that $\widetilde H_i ( D^{1,0}(R) ) =0$ for $i<n$ and $\widetilde H_i ( D^{1,0}_n(R) ) = \St(R)_n$. Hence the hypertor spectral sequence collapses and we see that $$\Tor_i^{\St(R)}(\Z,\Z)_n \cong \Tor_{n+i}^{\widetilde C_*(D^{1,0}_n(R)) }(\Z,\Z)_n.$$ The claim now follows from the fact that $$\Tor_{j}^{\widetilde C_*( A) }(\Z,\Z)_n \cong \widetilde H_j(B A)$$ for $A$ a monoid object in simplicial $\VB$-based spaces. 
\end{proof}

The following theorem combines the Koszulness result Miller--Nagpal--Patzt \cite[Theorem 1.4]{MNP}  and our \autoref{KD}, which states that for $F$ a field, $\Tor_n^{\St(F)}(\Z,\Z)_n \cong \St_n(F) \otimes \St_n(F)$. 
\begin{theorem} Let $F$ be a field. Let $n, i \geq 0$. There are isomorphisms
$$\Tor_i^{\St(F)}(\Z,\Z)_n \cong \left\{ \begin{array}{ll}  \St_n(F) \otimes \St_n(F), &  i=n \\ 0, & i \neq n. \end{array} \right. $$ 
\end{theorem} 

\begin{proof}
By \autoref{TorD}, $$\Tor_i^{\St(F)}(\Z,\Z)_n \cong \widetilde H_{i+n}(D^{2,0}_n(F)).$$ By \autoref{suspend}, $$\widetilde H_{i+n}(D^{2,0}_n(F)) \cong \widetilde H_{i+n-3}(T^{2,0}(F)).$$  By \autoref{lemJoin},  $$\widetilde H_{i+n-3}(T^{2,0}(F))  \cong \left\{ \begin{array}{ll}  \St_n(F) \otimes \St_n(F), &  i=n \\ 0, & i \neq n. \end{array} \right.$$ 
\end{proof}



\bibliographystyle{amsalpha}
\bibliography{refs}

\vspace{.5cm}


\end{document}

 