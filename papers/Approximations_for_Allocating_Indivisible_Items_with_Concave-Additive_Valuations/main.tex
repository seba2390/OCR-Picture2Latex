\documentclass[11pt,letterpaper]{article}
\usepackage{fullpage,cite}
\usepackage{amsmath,amsthm,amsfonts,commath,xcolor,graphicx,hyperref}

\usepackage[ruled,linesnumbered]{algorithm2e}
\usepackage{algpseudocode}

\newtheorem{theorem}{Theorem}[section]
\newtheorem{corollary}[theorem]{Corollary}
\newtheorem{lemma}[theorem]{Lemma}
\newtheorem{proposition}[theorem]{Proposition}
\newtheorem{definition}[theorem]{Definition}
\newtheorem{fact}[theorem]{Fact}
\newtheorem*{remark}{Remark}
\newtheorem{problem}{Problem}


\newcommand{\todo}[1]{{\color{red} todo: #1}}
\newcommand{\eat}[1]{}

\newcommand{\Brac}[1]{\left[#1\right]}
\newcommand{\Ex}[1]{\mathbb{E}\Brac{#1}}
\newcommand{\Var}[1]{\mathrm{Var}\Brac{#1}}
\newcommand{\poly}{\mathrm{poly}}

\newcommand{\OPT}{\mathrm{OPT}}
\newcommand{\NSW}{\mathsf{SNSW}}
\newcommand{\LNSW}{\mathsf{SNSW}\text{-}\mathsf{LOG}}
\newcommand{\POPT}{\mathrm{OPT}_\mathrm{prod}}
\newcommand{\LOPT}{\mathrm{OPT}_\mathrm{log}}
\newcommand{\DUAL}{\mathrm{DUAL}}
\newcommand{\ICA}{\mathsf{ICA}}
\newcommand{\SANW}{\mathsf{SANW}}
\newcommand{\SN}{{\small\mathsf{SN}}}
\newcommand{\compc}{\text{{\sccompute-}}c_{\ell}(F)}
\newcommand{\bmax}{b_{\max}}
\newcommand{\fmax}{f_{\max}}
\newcommand{\fpmax}{f'_{\max}}
\newcommand{\umax}{u_{\max}}
\newcommand{\rmax}{\rho_{\max}}
\newcommand{\bmaxi}{b_i^{\max}}
\newcommand{\fmaxi}{f_i^{\max}}
\newcommand{\umaxi}{u_i^{\max}}
\newcommand{\dvi}{D_{v_i}(u_i)}
\newcommand{\ct}{\tilde c_{\ell}}
\newcommand{\fpi}{f'_0}
\newcommand{\pub}{\mathrm{pub}}
\newcommand{\SNCP}{\mathsf{SN\text{-}CP}}
\newcommand{\SND}{\mathsf{SN\text{-}D}}
\newcommand{\ICACP}{\mathsf{ICA\text{-}CP}}
\newcommand{\ICAD}{\mathsf{ICA\text{-}D}}

% Choose a citation style by commenting/uncommenting the appropriate line:
%\setcitestyle{authoryear}
%\setcitestyle{acmnumeric}

% Title. Note the optional short title for running heads. In the interest of anonymization, please do not include any acknowledgements.
\title{Approximations for Allocating Indivisible Items with Concave-Additive Valuations}
\author{
Nathaniel Kell\thanks{Email: {\tt kelln@denison.edu}}\\Denison University
\and
Kevin Sun\thanks{Email: {\tt ksun@cs.duke.edu}}\\Duke University
}

\begin{document}
\maketitle
\begin{abstract} 
We study a general allocation setting where agent valuations are {\em concave additive}. In this model, a collection of items must  be uniquely distributed among a set of agents, where each agent-item pair has a specified utility. The objective is to maximize the sum of agent valuations, each of which is an arbitrary non-decreasing concave function of the agent's total additive utility.  This setting was studied by Devanur and Jain (STOC 2012) in the online setting for divisible items. In this paper, 
we obtain {\em both  tight multiplicative and additive approximations in the offline setting for  indivisible items.}
Our approximations depend on novel parameters that measure the local multiplicative/additive curvatures of each agent valuation, which we show correspond directly to the integrality gap of the natural assignment convex program of the problem. Furthermore, we extend our additive guarantees to obtain {\em constant multiplicative approximations for Asymmetric Nash Welfare Maximization} when agents have  {\em smooth valuations} (Fain et al.\ EC 2018, Fluschnik et al.\ AAAI 2019). This algorithm also yields an interesting tatonnement-style interpretation,  where agents adjust uniform prices and items are assigned according to maximum {\em weighted} bang-per-buck ratios.
\end{abstract}

Reinforcement learning has achieved great success in areas such as Game-playing \citep{silver2018general,vinyals2019grandmaster}, robotics \cite{kober2013reinforcement}, large language models \citep{ouyang2022training}, etc.
However, due to safety concerns or physical limitations, in some real-world reinforcement learning problems, we must consider additional constraints that may influence the optimal policy and the learning process \citep{garcia2015comprehensive}.
% For example, a robotic arm must not take actions that may cause harm to itself or the environments.
A standard framework to handle such cases is the constrained Markov Decision Process (CMDP) \citep{altman1999constrained}.
Within the CMDP framework, the agent has to maximize
the expected cumulative reward while
obeying a finite number of constraints, which are usually in the form of expected cumulative cost criteria.

However, we are sometimes concerned with the problem with a continuum of constraints.
For example,
the constraints we meet might be time-evolving or subject to uncertain parameters, which
cannot be formulated as an ordinary CMDP
(see Examples \ref{Example_Time_Evolving} and  \ref{Example_Uncertain}).
In this paper we would study a generalized CMDP  
to address the above problem.  Because the constraints are not only infinite-number but also lie
in a continuous set,
the generalization is not trivial. Fortunately, we find that we can borrow the idea behind semi-infinite programming (SIP) \citep{remez1934determination, hettich1993semi} to deal with the semi-infinite constraints.
Accordingly, we propose \emph{semi-infinitely constrained Markov decision processes} (SICMDPs)
as a novel complement to the ordinary CMDP framework.
%More specifically,  an SICMDP model %, we consider 
%contains a continuum of constraints whereas an ordinary CMDP contains a finite number of constraints. 

%This generalization is natural but not trivial. However, we can brows the idea  
%The idea is quite natural and can be backtracked
%to the practice of extending linear programming to linear semi-infinite programming (LSIP) %\cite{remez1934determination, GobernaLSIO1998}.
%In addition, 
%As a complementary approach to the ordinary CMDP framework, 
%SICMDP can be used to model these problems  which cannot be described by a finite number of constraints
%that are not covered by .
%For example,
%the restrictions we consider can be time-evolving or subject to uncertain parameters
%, thus
%cannot be described by a finite number of constraints but a continuum of constraints 
%(see Examples \ref{Example_Time_Evolving} and  \ref{Example_Uncertain}).

We also present two reinforcement learning algorithms to solve SICMDPs called SI-CRL and SI-CPO, respectively.
SI-CRL is a model-based reinforcement learning algorithm designed for tabular cases, and SI-CPO is a policy optimization algorithm for non-tabular cases.
% and analyze its performance both theoretically and empirically.
The main challenge is that we need to deal with a continuum of constraints, thus reinforcement learning algorithms for ordinary CMDPs do not work anymore.
In SI-CRL, we tackle this difficulty by first transforming the reinforcement learning problem to an equivalent LSIP problem, which can then be solved using methods in the LSIP literature like the dual exchange methods \citep{Hu1990,reemtsen1998numerical}.
In SI-CPO, we resort to the idea of cooperative stochastic approximation developed in \cite{lan2020algorithms, wei2020comirror}.
As far as we know, we are the first to introduce tools from semi-infinitely programming (SIP) into the reinforcement learning community for solving constrained reinforcement learning problems.

% To the best of our knowledge, we are the first to apply tools from semi-infinitely programming (SIP) to solve reinforcement learning problems.
Furthermore, we give theoretical analysis for both SI-CRL and SI-CPO.
We decompose the error of SI-CRL into two parts: the statistical error from approximating the true SICMDP with an offline dataset and the optimization error due to the fact that the solution of the LSIP problem obtained by the dual exchange method is inexact.
On the optimization side, we show that the iteration complexity of SI-CRL is $O\left(\left\{\mathrm{diam}(Y)L\sqrt{|\gS|^2|\gA|m}/\left[(1-\gamma)\epsilon\right]\right\}^m\right)$.
On the statistical side, we show that the sample complexity of SI-CRL is $\widetilde O\left(\frac{|S|^2|A|^2}{\epsilon^2(1-\gamma)^3}\right)$ if the offline dataset is generated by a generative model, and $\widetilde O\left(\frac{|S||A|}{\nu_{\min} \epsilon^2(1-\gamma)^3}\right)$ if the dataset is generated by a probability measure $\nu$ as considered in \cite{chen2019information}.
Here $\widetilde O$ means that all logarithm terms are discarded.
For SI-CPO, things become a little more complicated because other than the statistical error and the optimization error, we also need to consider the function approximation error, which comes from imperfect policy parametrizations.
It is shown if the function approximation error can be controlled to $O(\epsilon)$ order, the iteration complexity of SI-CPO is $\widetilde{O}\left(\frac{1}{\epsilon^2(1-\gamma)^6}\right)$ and the sample complexity of SI-CPO is $\widetilde{O}(\frac{1}{\epsilon^4(1-\gamma)^{10}})$.
Here our iteration complexity bound is equivalent to a typical $\widetilde O(1/\sqrt{T})$ global convergence rate.

We perform a set of numerical experiments to illustrate the SICMDP model and validate our proposed algorithms.
Specifically, we examine two numerical examples, namely the discharge of sewage and ship route planning.
Through the discharge of sewage example, we show the advantage of the SICMDP framework over the CMDP baseline obtained by naive discretization in modeling realistic sequential decision-making problems.
Moreover, we demonstrate the effectiveness of the SI-CRL and SI-CPO algorithms in such tabular environments. 
In the ship route planning example, we illustrate the benefits of the SICMDP framework and the ability of the SI-CPO algorithm to address complex continuous control tasks involving continuous state spaces with modern deep reinforcement learning techniques.

% In summary, our contributions are listed as follows.
% First, we present the SICMDP model, which can be viewed as a generalization of the ordinary CMDP model.
% Second, we propose an algorithm to perform reinforcement learning for SICMDPs, which is called SI-CRL, and we believe that we are the first to apply tools from SIP
% to solve reinforcement learning problems.
% Third, we give a theoretical analysis of SI-CRL and identify both its sample complexity and iteration complexity.
% In addition, we perform numerical experiments to illustrate the SICMDP model and validate the SI-CRL algorithm.
% \{This paragraph can be removed!!! \}






We study the problem of selling $m$ items to $n$ buyers. We denote a bundle of items as a quantity vector $\vec{q} \in \Z_{\geq 0}^m$. The number of units of item $i$ in the bundle is $q[i]$. The bundle consisting of only one copy of the $i^{th}$ item is denoted by the standard basis vector $\vec{e}_i$, where $e_i[i] = 1$ and $e_i[j] = 0$ for all $j \not= i$. Each buyer $j \in [n]$ has a valuation function $v_j$ over bundles of items. We denote an allocation as $Q = \left(\vec{q}_1, \dots, \vec{q}_n\right)$ where $\vec{q}_j$ is the bundle that buyer $j$ receives. The cost to produce $\vec{q}$ is $c\left(\vec{q}\right)$ and the cost to produce the allocation $Q$ is $c\left(Q\right)$.
Suppose there are $\kappa_i$ units available of item $i$. Let $K = \prod_{i = 1}^m \left(\kappa_i+1\right)$. We use $\vec{v}_j = \left(v_j\left(\vec{q}_1\right), \dots, v_j\left(\vec{q}_K\right)\right)$ to denote buyer $j$'s values for all of the $K$ bundles and we use $\vec{v} = \left(\vec{v}_1, \dots, \vec{v}_n\right)$ to denote a vector of buyer values. We use the notation $\cX$ to denote the set of all valuation vectors $\vec{v}$. Additive buyers have values $v_j\left(\vec{q}\right) = \sum_{i = 1}^m q[i] v_j\left(\vec{e}_i\right)$ and unit-demand buyers have values $v_j\left(\vec{q}\right) = \max_{i : q[i] \geq 1} v_j\left(\vec{e}_i\right)$. The mechanisms  we study are dominant strategy incentive compatible, so we assume that the bids equal the buyers' valuations.

There is an unknown distribution $\pazocal{D}$ over buyers' values. 
The notation $\profit_M\left(\vec{v}\right)$ denotes the profit of a mechanism $M$ on the valuation vector $\vec{v}$. We use the notation $\profit_{\dist}\left(M\right) = \E_{\vec{v} \sim \dist}\left[\profit_M\left(\vec{v}\right)\right]$ and for a set of samples $\sample$, we use the notation \[\profit_{\sample}\left(M\right) = \frac{1}{|\sample|}\sum_{\vec{v} \in \sample}\profit_M\left(\vec{v}\right).\]

We study real-valued functions parameterized by vectors $\vec{p}$ in $\R^d$, denoted as $f_{\vec{p}}:\domain \to \R.$ For a fixed $\vec{v} \in \domain$, we often consider $f_{\vec{p}}\left(\vec{v}\right)$ as a function of its parameters, which we denote as $f_{\vec{v}}\left(\vec{p}\right)$.
\begin{algorithm}[!ht]
\begin{algorithmic}[1]
\Require Query workload $Q$, event stream $I$, \app\ graph $G$, hash table of snapshots $S$
\Ensure Hash table of results $R$ 
\State $G \leftarrow \emptyset$, $S, R \leftarrow$ empty hash tables
\ForAll {event $e \in I$ with $e.type=E$} 
    \State $//$ \textbf{\app\ graph construction}
    \ForAll {$q \in Q$ \text{ with event types }T}
        \ForAll {$E' \in T,\ E' \neq E$}
            \State $G_{E'} \leftarrow \mathit{getGraphlet}(G,E')$,
            $G_{E'}.\mathit{active} \leftarrow \mathit{false}$
        \EndFor
    \EndFor
    \If {\textbf{not} $G_E.\mathit{active}$}
        \State $G_E \leftarrow \mathit{createGraphlet()}$, $G_{E}.\mathit{active} \leftarrow \mathit{true}$,
        $G \leftarrow G \cup G_E$
        \If {$G_E.\mathit{shared}$ by $Q_E \subseteq Q$}
            $x \leftarrow \mathit{createSnapshot()}$ 
            \ForAll {$q \in Q_E$}
                \ForAll{$E' \in \mathit{pt}(E,q), E' \neq E$}
                    \State $G_{E'} \leftarrow \mathit{getGraphlet}(G,E')$
                    \State $S(x,q) \leftarrow S(x,q) + sum(G_{E'},q)$ \hspace{0.5cm}$//$ Eq.~5
                \EndFor
            \EndFor
        \EndIf    
    \EndIf
    \State insert $e$ into $G_E$
    \State $//$ \textbf{Trend count computation}
    \If {$G_E.\mathit{shared}$ by $Q_E \subseteq Q$}
        \If {$\forall q \in Q_E\ pe(e,q)$ are identical}
            \State $count(e,Q_E) \leftarrow count(e,q)$ \hspace{2.3cm}$//$ Eq.~2
        \Else\ $y \leftarrow \mathit{createSnapshot()}$, $count(e,Q_E) = y$
            \ForAll {$q \in Q_E$}
                $S(y,q) \leftarrow count(e,q)$ \hspace{0.2cm}$//$ Eq.~2
            \EndFor
          \EndIf
    \Else\ $count(e,q)$ \hspace{5.2cm}$//$ Eq.~2
    \EndIf
    \ForAll{$q \in Q$}
  	    \If {$E \in \mathit{end}(q)$} 
  		    $R(q) \leftarrow R(q) + count(e,q)$ $//$ Eq.~3
        \EndIf
    \EndFor
\EndFor
\State \Return $R$
\end{algorithmic}
\caption{\app\ shared online trend aggregation}
\label{algo:snapshot-propagation}
\end{algorithm}

\section{Integrality Gap of $\ICA$ Convex Program}
\label{sec:int-gap}


We now prove Theorem \ref{thm:int-gap}. In particular, we show that for any fixed monotone concave value function $v(\cdot)$ and maximum bid value $u$ with local multiplicative curvature $\mu$, we can construct an instance of $\ICA$
such that the optimal fractional solution to $\ICACP$ has objective $\mu$ times that of any integral assignment. 

%Similarly to algorithm, the integrality gap easily extends extends to show a $n\alpha$ additive integrality gap (Theorem \ref{thm:add-alg-bound}). We omit the details of this extension here for sake of brevity. 

\paragraph{Instance Construction.} Let $z$ be the $\arg\min$
that defines $\mu$ for function  $v(\cdot)$, and let $z^*$ be the $\arg\max$ that defines $\mu_i(z, u)$ in the definition of $\mu_i$, as shown on the left side of Fig.~\ref{fig:int-gap}.
Without loss of generality, we can assume $z^*/u$ can be expressed as 
rational number $\beta/\gamma$ where $\beta, \gamma \in \mathbb{Z}^+$, since any irrational number has arbitrarily close rational  approximation. (In which case our instance construction can be taken such a limit to obtain the desired bound).  




We construct our instance as follows. 
The valuation function of every agent is $v(\cdot)$. 
There are $\gamma$ agents and $\beta + \gamma \cdot \lceil \frac{z}{u} \rceil$ items in total. Of the items, $\beta$ of them are ``public,'' i.e., for all agents $i$ we have have $u_{i,j} = u$. Call this subset of items $M_{\pub}$.  The remaining items $\gamma \lceil\frac{z}{u}\rceil$ items are partitioned among the $\gamma$ agents so that each agent $i$ receives a set of $\lceil\frac{z}{u}\rceil$ ``private'' items $M_i$. For each item $j$ in the first $\lfloor \frac{z}{u}\rfloor$ of these private items in $M_i$, we set $u_{i,j} = u$. 
For the remaining item (if there is one) $j'$ in $M_i$, we set $v_{i,j'} = z - \lfloor \frac{z}{u}\rfloor u < u$.
For all other agents $i' \neq i$, $u_{i',j} = 0$ for all $j \in M_i$. 
Based on this construction, the sum of $u_{i,j}$ over all $j \in M_i$ is equal to $z$, and the maximum bid value in the instance is indeed $u$. 
This completes the construction, which is illustrated in Figure \ref{fig:int-gap}. 

\begin{figure}
\centering
\includegraphics[height=5cm]{figures/int-gap-crop.pdf}
\caption{{\small Illustration of integrality gap construction. Squares correspond to agents and circles correspond to items.}}
\label{fig:int-gap}
\end{figure}

\paragraph{Analysis.} Let $\OPT_F$ and $\OPT_I$ denote 
the optimal fractional and integral solutions, respectively, to the $\ICACP$ convex program for the above instance construction. 
We first show that $\OPT_F$ is obtained by evenly splitting the spend of the $\gamma$ agents among the $\beta$ public items. 
To do this, we construct a feasible dual solution $\DUAL$ to the $\ICAD$ program that has objective equal to that of $\OPT_F$.
By Lemma \ref{lem:duality}, it  then follows $\OPT_F$ is an optimal fractional solution. 
Constructing this dual solution will also be useful for showing the desired integrality gap, as it will be easier to relate the objective of $\OPT_I$ to $\DUAL$. Note that to simplify notation, for the remainder of the section we denote $t^* := z + z^*$. 

Observe that in the solution $\OPT_F$ specified above, each agent spends a total of $u\beta/\gamma$ on public items.
By the definitions of $\beta$ and $\gamma$, we have $u\beta/\gamma = z^*$.  Each agent also receives a total spend of $z$ from from their private items.  
Therefore, since there are $\gamma$ agents in the total, the objective of $\OPT_F$ is the following:

\begin{equation}
\label{eq:int-gap-opt}
\OPT_F = \gamma v(z+z^*) = \gamma v(t^*).
\end{equation}
To construct $\DUAL$,  we set $t_i = t^*$ for all agents $i$. 
Since each agent has an identical valuation function $v(\cdot)$, setting variables $p_j = u v'(t^*)$ if $j$ is a public item and setting $p_j = u_{i,j}v'(t^*)$ if $j$ is a private item in $M_i$ produces a feasible solution to $\DUAL$.

To show that the objectives of $\OPT_F$ and $\DUAL$ are the same, first observe that the definitions of $y(\cdot)$ and $v'(\cdot)$, we have the following identities: 
\begin{gather} 
\label{eq:int-gap-dual-1}
y(t^*) + (z+u)v'(t^*) = v(t^*) + (u-z^*)v'(t^*) \\
\label{eq:int-gap-dual-2}
y(t^*) + zv'(t^*) = v(t^*) - z^*v'(t^*).
\end{gather}
 Equations \eqref{eq:int-gap-dual-1} and \eqref{eq:int-gap-dual-2} give two equivalent ways of expressing the value  
of the line tangent to $v(\cdot)$ at $t^*$ evaluated at $x$-coordinates $z+u$ and $z$, respectively. Geometrically, on the LHS, we start at the $y$-intercept of the line, 
and then follow the slope of the tangent for a total $x$-width of $z+u$ (resp.\ $z$). On the RHS, we instead start at $v(t^*) = v(z+z^*)$ (i.e., the tangent point)
and follow the tangent line to $z+u$ (resp.\ backwards to $z$). 



From the construction of the instance and the definition of $\DUAL$, the objective of $\DUAL$ can be characterized as follows:
\begin{align*}
\sum_i \left(y(t^*) + \sum_{j \in M_i} p_j\right) + \sum_{j \in M_{\pub}}p_j & = \gamma  (y(t^*) + z  v'(t^*)) +  \beta u v'(t^*) \\ 
& = \beta \cdot (y(t^*) + (z +u) v'(t^*)) +  (\gamma - \beta) \cdot (y(t^*) + zv'(t^*)).
\end{align*}
By applying Equations \eqref{eq:int-gap-dual-1} and \eqref{eq:int-gap-dual-2} to the RHS above, it follows the RHS is equal to: 
\begin{equation}
\label{eq:dual-gap-obj}
\beta \cdot (v(t^*) + (u-z^*)v'(t^*)) + (\gamma-\beta) \cdot (v(t^*)-z^* v'(t^*)), 
\end{equation}
which can simplified as follows:
\begin{align*} 
\gamma v(t^*) + v'(t^*) \cdot (\beta(u-z^*) -(\gamma-\beta)z^*)  & = \gamma v(t^*) + v'(t^*) \cdot (\beta u - \gamma z^*)  \\
& = \gamma v(t^*) \\
& = \OPT_F
\end{align*}
where the second equality follows since $\beta/\gamma = z^*/u$, and the last equality follows from \eqref{eq:int-gap-opt}.
Thus $\OPT_F$ and $\DUAL$ have equivalent objective values. 


Now consider $\OPT_I$, which is obtained by assigning a unique public item to $\beta$ of the $\gamma$ agents.
(By a simple exchange argument, the objective cannot increase by assigning multiple public items to the same agent, since $v(\cdot)$ is monotone and concave.)
Each agent that receives a public item spends a total of $z + u$. 
The remaining $\gamma - \beta$ agents spend only a  total of $z$ from solely their private items. 
Thus the objective of the optimal integral solution is:


\begin{equation}
\label{eq:gap-int-obj}
\OPT_I = \beta v(z+u) + (\gamma-\beta) v(z).
\end{equation}

By definition of $\mu$, the gap between $v(z+b)$ and $v(z)$, and the (respective) points 
characterized by Equations \eqref{eq:int-gap-dual-1} and \eqref{eq:int-gap-dual-2} are equal to $\mu$. More specifically, we have
\begin{equation}
\label{eq:gap-int-char}
 \mu = \frac{v(t^*) + (u-z^*)v'(t^*)}{v(z+u)} = \frac{v(t^*) - z^*v'(t^*)}{v(z)}.
\end{equation}

Since \eqref{eq:dual-gap-obj} expresses the objective value of $\DUAL$ (and therefore $\OPT_F$, as well), 
Equations \eqref{eq:dual-gap-obj}, \eqref{eq:gap-int-obj}, and \eqref{eq:gap-int-char} together imply $\OPT_F = \mu\OPT_I$, i.e., the integrality gap of this instance is $\mu$.



\section{Applications}
\label{sec:apps}


% https://arxiv.org/pdf/1912.12541.pdf
% https://arxiv.org/pdf/2112.10199.pdf
% https://arxiv.org/pdf/2201.01419.pdf
% https://www.ijcai.org/proceedings/2019/0042.pdf


\subsection{Constant Approximation for Asymmetric Nash Welfare with Smooth Valuations}
\label{subsec:smooth-anw}
We now apply our techniques to the problem of Nash Welfare Maximization for asymmetric agents with smooth additive additive valuations.
In this problem, each agent $i$ has a weight $\eta_i > 0$, and the goal is to find an allocation that maximizes $ \left(\prod_i (u_i + \omega)^{\eta_i}\right)^{1/\eta}$
where $\eta = \sum_i \eta_i$ is the sum of the agent weights and $\omega \in (0,1]$ denotes the smoothing parameter of the instance.

As discussed in the introduction, observe that we can scale the objective of each 
agent $i$ by $(\max_j u_{i,j})^{-\eta_i}$ without changing the approximation ratio of the algorithm. 
Therefore, wlog for the rest of the section we will assume that $\max_{j}u_{i,j} = 1$ for every agent $i$.
After this scaling, we can think of the smoothing parameter as giving each agent $i$ an initial utility of $\omega \max_{j}u_{i,j}$
at the beginning of the instance. Also for simplicity, throughout the section we assume weights are normalized by dividing them by $\eta$, so $\eta = 1$ (i.e., we bring the $1/\eta$ exponent into each term in the product objective).  





\subsubsection{Algorithm Definition}

Our algorithm has a natural combinatorial interpretation  which we call the {\em Weighted Bang-Per-Buck} (WBB) algorithm. To define the algorithm, we first explicitly define the additive curvature parameter $\alpha_i$ in the case where $v_i(u_i) = \eta_i \ln(u_i + \omega)$. Let $\sigma_i(z)$ denote the slope of the lower-bounding secant line that intersects the points $(z, \eta_i\ln(z + \omega))$ and $(z+1, \eta_i\ln(z+\omega + 1))$, given as:
\begin{equation}
\label{eq:sn-slope}
\sigma_{i}(z,1) := \eta_i\ln(z+ \omega + 1) - \eta_i\ln(z+\omega) =\eta_i\ln\left(1 + (z+\omega)^{-1}\right).
\end{equation} 
We then define the local additive curvature bound $\alpha_i$ at $z$ for agent $i$:
\begin{equation}
\label{eq:sn-curv}
\alpha_i(z) := \max_{z^* \in (0, 1)} \left[\eta_i\ln(z + z^* + \omega) - (\eta_i\ln(z + \omega) + z^*\sigma_{i}(z,1))\right]
\end{equation}
The  WBB Algorithm is given below in Algorithm 2.  
Throughout its execution, we adjust a uniform bid $b_i$ each agent $i$ makes for on every item in the instance. The algorithm starts with bids that are underestimates of the optimal dual bids, and thus proceeds by increasing the uniform bid of each agent one at a time, ensuring throughout every item is assigned to a maximum weighted bang-per-buck ratio agent, i.e., an agent that maximizes $(\eta_i u_{i,j})/b_i$. 
The algorithm stops increasing the bid of an agent according an exponential potential function proportional to agent's average {\em unweighted} MBB ratio (which we derive from the while-loop condition from the additive version of the $\ICA$ algorithm given in Section \ref{subsec:additive}).


\begin{algorithm}

\caption{Maximum Weighted Bang-per-buck Algorithm ({\sc WBB})}
Initialize fixed bid $b_i \leftarrow \omega$ for each agent $i$  \\
 \label{alg:WMBB}
Allocate each item $j$ to maximum WBB agent $\arg\max_{i} \left( \frac{ \eta_i u_{i,j}}{b_i} \right)$ \Comment{weighted greedy assignment} \\
\While{there exists an agent $i$ such that $\frac{u_i + \omega}{b_i}< \exp\left(\frac{u_i + \omega}{b_i} - 1 - \alpha_i\right)$}{
    \While{$\frac{u_i + \omega}{b_i}< \exp\left(\frac{u_i + \omega}{b_i} - 1 - \alpha_i\right)$}{
    \eIf{there is an item $j$ assigned to agent $i$ such that $i$ is not $j$'s maximum WBB agent}
    {
     Reassign $j$ to agent $\arg\max_{k}\left(\frac{\eta_{k} u_{k,j}}{b_{k}}\right)$
    }
    {
    Increase agent $i$'s bid to be $b_i \leftarrow \frac{\eta_imb_i}{\eta_i m - \epsilon b_i}$ \\
    }
}  
}
Output resulting allocation $u_i$ for all agents
\end{algorithm}






\subsubsection{Analysis}
To analyze the algorithm, we first argue that the WBB algorithm is equivalent to executing the ICA algorithm for an additive guarantee (as outlined in Section \ref{subsec:additive}). We then derive a closed-form for the local additive curvature $\alpha_i$ in terms of the smoothing parameter $\omega$.

\begin{lemma} 
The WBB algorithm (Algorithm 2) is equivalent to executing the ICA algorithm for an additive guarantee, where in the ICA instance $v_i(u_i) = \eta_i \ln(u_i + \omega)$.
\end{lemma} 

\begin{proof} 
By Lemma \ref{lem:duality}, the  primal and dual program for an ICA instance with $v_i(u_i) = \eta_i \ln(u_i + \omega)$ is given by the following (denoted {\sc ASN-CP} for ``Asymmetric Smooth Nash''):  

\begin{center}
\begin{tabular}{c  c  c | c  c} 
\hspace{5mm} & 
$
\begin{gathered}
    \textnormal{({\sc ASN-CP}):}  \max \sum_i \eta_i\ln(u_i+\omega) \\
    \forall i: u_i = \sum_j u_{i,j}x_{i,j} \\
    \forall j: \sum_i x_{i,j} \leq 1 \\
    \forall i,j: x_{i,j} \geq 0 \\
\end{gathered}
$
& \hspace{1mm} & \hspace{1mm} & 
\vspace{-4mm}
$
\begin{gathered}
    \textnormal{({\sc ASN-D}):} \min \sum_i \eta_i \left(\ln(t_i+ \omega) - \frac{t_i}{t_i+\omega}\right) + \sum_j \beta_j \\
    \forall i,j: \beta_j \geq \frac{\eta_i u_{i,j}}{t_i + \omega} \\
    \forall i,j: t_i, \beta_j \geq 0 \\
\end{gathered}
$
\end{tabular} \\
\end{center} 
\vspace{5mm}


% The earlier section can be directly applied to maximizing $\NSW'$, since $\log$ is a concave function. \todo{I couldn't do the calculus all the way, but the plots look convincing:} We can show that for $f(x) = \log(x+1)$, we have $R^* = \log 2 \approx 0.693$. But this is only a multiplicative guarantee for $\NSW'$, when the usual goal is to obtain a guarantee for $\NSW$.

%Now we show how we can obtain a multiplicative guarantee for $\NSW$. Our approach is to modify our general algorithm from Section~\ref{sec:c-ell} to obtain an \emph{additive} guarantee for $\NSW'$, and then we show that this translates into a multiplicative guarantee for $\NSW$.

Note that for this application, we denote the dual variable $p_j$ as $\beta_j$, since it is interpreted as the weighted MBB ratio, not the price. In particular, in the WBB algorithm, we substitute the $t_i + \omega$ terms in the {\sc ASN-D} program to be the uniform bid $b_i$ made by agent $i$ for all items. Thus the function $D(u_i)$ becomes: 

\begin{equation} 
\label{eq:Dlogdef}
D(u_i) = \eta_i\left[\frac{u_i + \omega}{b_i} + \ln(b_i)  -1 \right].
\end{equation}

Rearranging \eqref{eq:Dlogdef} the while-loop condition in Algorithm \ref{alg:pd} (which is $v_i(u_i) - D(u_i) > \alpha_i$ for the general additive $\ICA$ algorithm) and  exponentiating, we obtain the while-loop condition in Algorithm~\ref{alg:WMBB} after canceling $\eta_i$ terms. Furthermore, since WBB algorithm maintains an assignment where each item is assigned to the agent with maximum weighted MBB ratio, the variables $t_i = b_i - \omega$ and $\beta_j = \arg\max_{i} \left( \frac{ \eta_i u_{i,j}}{b_i} \right)$ form a feasible dual solution in a manner identical to the ICA algorithm. 
Finally, it is easily seen that the update to bid $b_i$ decreases $v'_i(t_i)$ by $\epsilon/m$:


\begin{equation*}
v_i'^{(2)}(t_i) = \frac{\eta_i}{\frac{\eta_imb_i}{\eta_i m - \epsilon b_i}} = \frac{\eta_i}{b_i} - \frac{\epsilon}{m} = v_i'^{(1)}(t_i) - \frac{\epsilon}{m},
\end{equation*}
where $v_i'^{(2)}(t_i)$ and $v_i'^{(1)}(t_i)$ denote the $v_i'(t_i)$ before and after the update to $b_i$ (respectively). 
\end{proof}


\begin{lemma}
\label{lem:nsw-diff-gap}
The local additive curvature $\alpha_i$ for agent $i$ is given by:

\begin{equation*} 
\alpha_i = \eta_i  \left[ \ln\left(\frac{1}{\omega \ln(1+1/\omega)}\right) + \omega \ln(1+1/\omega) - 1 \right]= O\left(\ln\left(\frac{\eta_i}{\omega\ln(1+\omega)}\right)\right),
\end{equation*}
when valuation function of agent $i$ is $v_i(u_i) = \eta_i \ln(u_i + \omega)$.
\end{lemma}

\begin{proof} 

For a fixed value $z$, let $\gamma = z + \omega$ and let $\Delta(z^*)$ be a function of $z^*$ defined by the expression inside the $\max$ in Equation \eqref{eq:sn-curv} for $\alpha(z)$, given as:


\begin{equation*}
\Delta(z^*) = \eta_i\left(\ln(z^* + \gamma) - \ln \gamma - z^*\ln(1+1/\gamma)\right).
\end{equation*}
Observe $\Delta(z^*)$ is a concave function in $z^*$, since the term $\ln(z^* + \gamma)$ is a concave function of $z^*$
and the remainder of the expression is a linear function in $z^*$. 
Therefore its maximum is obtained when $\frac{d}{dz^*} \Delta(z^*) = 0.$ 
By basic calculus, it follows the maximizer $z_{\max}$ for Equation \eqref{eq:sn-curv}
is given by:

\begin{equation} 
\label{eq:zmax}
z_{\max} = \arg\max_{z^* \in (0, 1)} \Delta(z^*)  =  \frac{1}{\ln\left(1 + 1/\gamma \right)} - \gamma.
\end{equation}
Thus $\alpha_i(z)$ can be then be expressed in a
closed form by plugging in $z_{\max}$ from Equation \eqref{eq:zmax}
in for $z^*$, which can be simplified as:

\begin{equation*} 
\alpha_i(z) = \eta_i\left[\ln\left(\frac{1}{\gamma \ln(1+1/\gamma)}\right) + \gamma \ln\left(1+\frac{1}{\gamma}\right) - 1\right].
\end{equation*}

It can be verified that the derivative of $\alpha_i(z)$ with respect to $\gamma$ is negative for all $\gamma > 0$. Therefore since $\gamma = \omega + z$, 
the derivative of $\alpha_i(z)$ is also negative for all $z \geq 0$, so $\alpha_i(z)$ is maximized at $z = 0$. Since $\gamma = \omega$ when $z= 0$, the lemma follows.  
\end{proof}






We can now prove Theorem \ref{thm:snsw}.
\begin{proof}[Proof of Theorem~\ref{thm:snsw}]





By Theorem \ref{thm:add-alg-bound}, the algorithm achieves the desired run-time bound, since $v_i'(0) = \eta_i/\omega \leq 1/\omega$ (recall we normalized agent weights to be $\eta_i/\eta$) and the update on Line 8 in WBB takes $O(1)$ time.
Thus, we are left with bounding the approximation ratio of the algorithm for the product objective.  

Let $\POPT$ and $\LOPT$ denote the objective value of the optimal solution for the product and log objective, respectively.  
Consider the dual variables $(\beta_j, t_i)$ corresponding to the allocation returned by the algorithm.
Also by Theorem \ref{thm:add-alg-bound}, $\beta_j + \epsilon/m$ is a feasible dual solution to the dual program {\sc ASN-D}. 
Thus by Lemma \ref{lem:duality} we have:

\begin{align}
    \text{{\sc ASN-D}}(t,\beta) & = \sum_i \eta_i \left(\ln(t_i+ \omega) - \frac{t_i}{t_i+\omega}\right)+ \sum_j \left(\beta_j + \frac{\epsilon}{m}\right) \notag  \\
    &= \sum_i D(u_i) + \epsilon \geq \LOPT \label{eq:sn-dual-bound}. 
\end{align}

When the algorithm terminates we have $\eta_i\ln(u_i + \omega) \geq D(u_i)  - \alpha_i$ for every agent $i$. Along with Inequality \eqref{eq:sn-dual-bound}, 
this implies the total objective of the algorithm is bounded by: 

\begin{equation*}
\sum_i \eta_i \ln(u_i+\omega) \geq \sum_i (D(u_i) - \alpha_i) \geq \LOPT  - \sum_i \alpha_i - \epsilon.
\end{equation*}
From this inequality, and the  fact that $\LOPT = \ln\left(\POPT\right)$ (when weights are scaled such $\eta = 1$), it follows the algorithm's objective on the product objective is bounded by: 
\begin{alignat}{2}
\prod_i (u_i + \omega)^{\eta_i} = 
\exp\left(\sum_i \ln(u_i + \omega)\right) & \geq \exp\left(\LOPT - \sum_i \alpha_i -\epsilon  \right) \notag \\
& = \exp\left(- \sum_i \alpha_i -\epsilon  \right)\POPT. \label{eq:nsw-final}
\end{alignat}
From Lemma \ref{lem:nsw-diff-gap}, we have that $\sum_i \alpha_i  = O\left(\ln\left(\frac{1}{\omega\ln(1+\omega)}\right)\right)$, it follows $\exp\left(\sum_i \alpha_i +\epsilon\right) = O(e^{\epsilon}/(\omega\ln(1+1/\omega)))$. Therefore by rearranging the above inequality \eqref{eq:nsw-final}, the theorem is established. 
\end{proof}


\subsubsection{Extension to Smooth Piecewise-Linear Valuation}
\label{subsub:anw-pl}
The results in this section can be also extended to  piecewise-linear valuation, i.e., maximizing $\left(\prod_i (f_i(u_i + \omega))^{\eta_i}\right)^{1/\eta}$ where $f_i(\cdot)$ is a piecewise-linear function.  In the next section (Section \ref{subsec:app-pl-welfare}) we argue the multiplicative curvature $\mu_i$ is at most $4/3$ for all such functions. Combining arguments in this section  with that of Lemma \ref{lem:nsw-diff-gap} we can obtain the following result.

\begin{theorem} 
\label{thm:anw-pl}
Consider an ICA instance with $v_i(u_i) = \eta_i \ln(f_i(u_i + \omega))$ where $f_i(u_i)$ is a piecewise-linear function concave-additive function with $\min_{k}\left(x_{i,k+1} - x_{i,k}\right) \geq \max_{j} u_{ij}$.
If $\omega = \Omega(1)$, then there exists a polynomial-time algorithm for the problem with an $O(1)$ approximation factor.
\end{theorem}

The extension works since Lemma \ref{lem:nsw-diff-gap} can be adapted to derive an additive curvature $\alpha_i$ that is a constant if $\omega = \Omega(1)$.
For example, when $f_i(u_i) = \min(u_i, c_i)$ is a budget-additive function, we obtain an approximation of $\approx  1.154$ as $\epsilon \rightarrow 0$ and $\omega = 1$.
We defer the proof of this result to the full version of the paper.





%%%% UNUSED SEPARABLE PIECE WISE LINEAR ALGORITHM %%%%%%

\subsection{Utilitarian Welfare Maximization for Piecewise-Linear Concave Utilities}
\label{subsec:app-pl-welfare}




%The bid value of $i_j$ is 1 for every item of type $j$, and 0 for all other items. This way, agent $i$ receiving $k$ copies of item type $j$ is equivalent to agent $i_j$ receiving $k$ items of type $j$.

In this section we establish Theorem \ref{thm:pl-mult}, we analyze our multiplicative algorithm for $\ICA$ in the  special case where every valuation function $v_i(\cdot)$ is monotone piecewise-linear concave function. 
In this setting, each $v_i(\cdot)$ is defined over series of conjoined {\em segments} $\ell_{i,1}, \ell_{i,2}, \ldots, \ell_{i, \lambda_i}$. Let
\[
0 = x_{i,0} < x_{i,1} < x_{i,2} < \cdots < x_{i,\lambda_i -1}
\]
denote the transition points between the segments of $v_i(\cdot)$, where $x_{i,k}$ denotes the transition point between segments $\ell_{k}$ and $\ell_{k+1}$ along the $x$-axis. Also, let 
\[
v'_{i,1} > v'_{i,2} > \cdots > v'_{i,\lambda_i } \geq 0
\]
denote the slopes of the segments, where $v'_{i,k}$ denotes the slope of segment $\ell_{i,k}$ (the inequalities follow since $v_i(\cdot)$ is monotone and concave).
Thus, if $u_i \in [x_{i,k}, x_{i,k+1}]$, then we can express agent $i$'s valuation formally as $v_i(u_i) = v_i(x_{i,k}) + v'_{i,k}\cdot(u_i - x_{i,k})$.
We further assume that $\min_{k \in [\lambda_i-1]}\left(x_{i,k+1} - x_{i,k}\right) \geq \max_{j} u_{ij}$, i.e., the maximum additive utility earned from any one item is at most the length of any segment. Recall that for the special case of $v_i(u_i) = \min(u_i, c_i)$ for some budget $c_i$, this corresponds to the standard assumption that every agent $i$'s utility for each item is at most her budget.

% Let $\ell_i$ denote the number linear segments in $v_i(\cdot)$. Let $v'_{ik} \geq 0$ denote the slope of the $k$th segment, and let $x_{i,k}$ denote the transition point on the $x$-axis between segments $k$ and $k-1$ (where for all functions and agents we let $x_{i,0} = 0$). 

In the definition of Algorithm \ref{alg:pd} in Section \ref{sec:ica-algo}, we assumed that every valuation $v_i$ is differentiable. Although a piecewise-linear $v_i(u_i)$ as defined above is not differentiable at each transition point $x_{i,k}$, Algorithm \ref{alg:pd} can be easily adapted to this setting by using supergradients in place of of tangents (we defer a complete definition of this adaptation to the full version).

Thus to establish Theorem \ref{thm:pl-mult}, it suffices to show the local multiplicative curvature $\mu_i$ for all such piecewise-linear functions is at most $4/3$. We first prove that the local multiplicative curvature of the budget-additive valuation $v_i(u_i) = \min(u_i, c_i)$ exactly $4/3$.
We then show the relaxing $v_i$ to be a general piecewise-linear function does not increase its value of $\mu_i$.

%As we will see, intuitively, this function is the ``worst case'' among all concave, piecewise-linear, monotone functions.

\begin{lemma} \label{fact:budget-3/4}
The local multiplicative curvature $\mu_i$ of $v_i(u_i) = \min(u_i,c_i)$ is exactly $4/3$.
\end{lemma}

\begin{proof}
%Recall that the width of the (lower-bounding) secant line $s$ must $w$. Without loss of generality, we can assume this width is exactly $w$, because longer secants result in larger values of $\mu$.
For a fixed value $z$, let $\Delta_z(z^*, u_{i,j})$ be a function of $z^*$ defined by the expression inside the $\max$ in Equation \eqref{eq:mult-lcb} for $\mu_i(z, u_{i,j})$. 
For all $z+z^* \leq c_i$, $\Delta_z(z^*, u_{i,j})$ is given by:
\begin{equation} \label{eq:budget-gap}
\Delta_z(z^*, u_{i,j})  = \frac{z+z^*}{z+ z^*\left(\frac{u_{i,j}-z}{u_{i,j}}\right)}
\end{equation}

One can verify the derivative of $\Delta_z(z^*, u_{i,j})$ with respect to $z^*$ is positive for $z^* + z \leq c_i$, and also that the derivative of the expression for $\Delta(z^*, u_{i,j})$ when $z + z^* \geq c_i$ is negative. Thus it follows for a fixed $z$, $\Delta_z(z^*, u_{i,j})$ is maximized when $z+z^* = c_i$. 
Therefore, 
we can express $\mu_i(z, u_{i,j})$ as the following function of $z$:
\begin{equation} \label{eq:budget-gap-2}
\mu_i(z,u_{i,j})   = \frac{c_i}{z+ (c_i - z)\left(\frac{u_{i,j}-z}{u_{i,j}}\right)}
\end{equation}

The above expression is maximized when $z = c_i/2$. The expression is also strictly increasing in $u_{i,j}$; therefore by our assumption that $\max_{j} u_{ij} \leq c_i$, the expression is maximized with respect to $u_{i,j}$ when $u_{i,j} = c_i$. 
Thus plugging in $z = c_i/2$ and $u_{i,j} = c_i$, we obtain $\max \mu_i(z,u_{i,j}) = c_i/(c_i/2 + c_i/4) = 4/3$, as desired. 
\end{proof}



\begin{lemma}
If $v_i(\cdot)$ is piecewise-linear, concave, non-decreasing, and satisfies $\min_{k}\left(x_{i,k+1} - x_{i,k}\right) \geq \max_{j} u_{ij}$, then its multiplicative curvature $\mu_i$ is at most $4/3$.
\end{lemma}




\begin{proof}
Let $w = \max_j u_{i,j}$, let $(z,z^*)$ be the maximizers that define $\mu_i$, and let 
$\sigma = \sigma_i(z,w) = v_i(z + w) - v_i(z))/w$ denote the slope of the lower-bounding 
secant line given by Equation \eqref{eq:lcb-slope}.
Thus the equation of the lower bounding secant line $s(x)$ that defines $\mu_i$ is given by 
$s(x) = \sigma \cdot(x-z) + v_i(z)$. Let  $\ell_1, \ell_2$ denote the  segments containing $z$ and $z+w$ fall in, respectively.
By our assumption $\min_{k}\left(x_{i,k+1} - x_{i,k}\right) \geq \max_{j} u_{ij}$, $\ell_1$ and $\ell_2$ must be adjacent segments of the function.


%First, note that we can assume that $\ell_1$ and $\ell_2$ are adjacent segments. This is because $v_i$ is concave, so if we replace the portion of $v_i$ between $z$ and $z+w$ by extending $\ell_1$ and $\ell_2$ until they intersect, the value of $v_i$ in this region can only increase while the secant line remains unchanged, so $\mu_i$ can only increase. 

To show the lemma, it suffices to show to show that for all $x \in [z, z+w]$, $v_i(x) / s(x) \leq 4/3$. Similar to proof of Lemma \ref{fact:budget-3/4}, we establish this by showing the largest gap occurs at the intersection of $\ell_1$ and $\ell_2$. Let $\ell_1, \ell_2$ be defined as functions of $x$ by $\ell_1(x) = v'_1x+b_1$ and $\ell_2(x) = v'_2x+b_2$, respectively.
Let $\overline{x}$ denote the $x$ coordinate of their intersection point, i.e., $\ell_1(\overline{x}) = \ell_2(\overline{x})$. Since $v_i(\cdot)$ is concave, we have $v'_1 < \sigma < v'_2$ and $b_1 < v_i(z) - zx < b_2$. For $x \in [z, \overline{x}]$, the multiplicative curvature at $x$ is the ratio between $\ell_1(x)$ and $s(x)$, which is
\[
\Delta(x) = \frac{v'_1x+b_1}{\sigma \cdot (x-z) + v_i(z)}.
\]
Since the derivative of $\Delta(x)$ is positive with respect to $x$, $\Delta(x)$ increases as $x$ increases.
We can similarly show that for $x \in [\overline{x}, z+z^*]$, $\Delta(x)$ decreases as $x$ increases. Thus, $\Delta(x)$ is maximized at $x = \overline{x}$.

To complete the argument, we can transform $\ell_2(x)$ so that conjoining segments $\ell_1$ and $\ell_2$ resemble a budget-additive function. In particular, we can ``flatten'' $\ell_2(x)$ by decreasing its slope $v'_2 = 0$, which  decreases the value of $\ell_2(z+z^*)$ from $v_i(\overline{x}) + v'_2(w-\overline{x})$ to now be $v_i(\overline{x})$. Since $z$ remains fixed, the slope of $s(x)$ decreases when adjust for this change, which means the curvature (evaluated at $\overline{x}$) can only increase. Applying Lemma \ref{fact:budget-3/4}, it follows that $\mu_i \leq 4/3$.


%Similarly, we steepen $\ell_1(x)$ by setting raising slope to 1 while ensuring that it still contains $\alpha$, and the curvature again increases.

% Next, we translate the entire picture (i.e., $v_i$ and $s$) down until the left endpoint of $\ell_1(x)$ (which now has slope 1) hits the $x$-axis. Doing so decreases the values of $v_i$ and $s$ by the same amount, which increases the value of the curvature.

%Finally, recall that the width of $s$ beneath $v_i$ (i.e., $w_i$) is at most the width of $\ell_1$, and the function now under consideration is a standard budget function. By Lemma~\ref{fact:budget-3/4}, this proves that $v_i(x)/s(x) \leq 4/3$ for all $x\in [p_1, q_1]$, as desired.
\end{proof}




\bibliographystyle{plainurl}
\bibliography{bibliography}

% Appendix
% \appendix
% \section{Switching times}

\end{document}
