\subsection{Utilitarian Welfare Maximization for Piecewise-Linear Concave Utilities}
\label{subsec:app-pl-welfare}




%The bid value of $i_j$ is 1 for every item of type $j$, and 0 for all other items. This way, agent $i$ receiving $k$ copies of item type $j$ is equivalent to agent $i_j$ receiving $k$ items of type $j$.

In this section we establish Theorem \ref{thm:pl-mult}, we analyze our multiplicative algorithm for $\ICA$ in the  special case where every valuation function $v_i(\cdot)$ is monotone piecewise-linear concave function. 
In this setting, each $v_i(\cdot)$ is defined over series of conjoined {\em segments} $\ell_{i,1}, \ell_{i,2}, \ldots, \ell_{i, \lambda_i}$. Let
\[
0 = x_{i,0} < x_{i,1} < x_{i,2} < \cdots < x_{i,\lambda_i -1}
\]
denote the transition points between the segments of $v_i(\cdot)$, where $x_{i,k}$ denotes the transition point between segments $\ell_{k}$ and $\ell_{k+1}$ along the $x$-axis. Also, let 
\[
v'_{i,1} > v'_{i,2} > \cdots > v'_{i,\lambda_i } \geq 0
\]
denote the slopes of the segments, where $v'_{i,k}$ denotes the slope of segment $\ell_{i,k}$ (the inequalities follow since $v_i(\cdot)$ is monotone and concave).
Thus, if $u_i \in [x_{i,k}, x_{i,k+1}]$, then we can express agent $i$'s valuation formally as $v_i(u_i) = v_i(x_{i,k}) + v'_{i,k}\cdot(u_i - x_{i,k})$.
We further assume that $\min_{k \in [\lambda_i-1]}\left(x_{i,k+1} - x_{i,k}\right) \geq \max_{j} u_{ij}$, i.e., the maximum additive utility earned from any one item is at most the length of any segment. Recall that for the special case of $v_i(u_i) = \min(u_i, c_i)$ for some budget $c_i$, this corresponds to the standard assumption that every agent $i$'s utility for each item is at most her budget.

% Let $\ell_i$ denote the number linear segments in $v_i(\cdot)$. Let $v'_{ik} \geq 0$ denote the slope of the $k$th segment, and let $x_{i,k}$ denote the transition point on the $x$-axis between segments $k$ and $k-1$ (where for all functions and agents we let $x_{i,0} = 0$). 

In the definition of Algorithm \ref{alg:pd} in Section \ref{sec:ica-algo}, we assumed that every valuation $v_i$ is differentiable. Although a piecewise-linear $v_i(u_i)$ as defined above is not differentiable at each transition point $x_{i,k}$, Algorithm \ref{alg:pd} can be easily adapted to this setting by using supergradients in place of of tangents (we defer a complete definition of this adaptation to the full version).

Thus to establish Theorem \ref{thm:pl-mult}, it suffices to show the local multiplicative curvature $\mu_i$ for all such piecewise-linear functions is at most $4/3$. We first prove that the local multiplicative curvature of the budget-additive valuation $v_i(u_i) = \min(u_i, c_i)$ exactly $4/3$.
We then show the relaxing $v_i$ to be a general piecewise-linear function does not increase its value of $\mu_i$.

%As we will see, intuitively, this function is the ``worst case'' among all concave, piecewise-linear, monotone functions.

\begin{lemma} \label{fact:budget-3/4}
The local multiplicative curvature $\mu_i$ of $v_i(u_i) = \min(u_i,c_i)$ is exactly $4/3$.
\end{lemma}

\begin{proof}
%Recall that the width of the (lower-bounding) secant line $s$ must $w$. Without loss of generality, we can assume this width is exactly $w$, because longer secants result in larger values of $\mu$.
For a fixed value $z$, let $\Delta_z(z^*, u_{i,j})$ be a function of $z^*$ defined by the expression inside the $\max$ in Equation \eqref{eq:mult-lcb} for $\mu_i(z, u_{i,j})$. 
For all $z+z^* \leq c_i$, $\Delta_z(z^*, u_{i,j})$ is given by:
\begin{equation} \label{eq:budget-gap}
\Delta_z(z^*, u_{i,j})  = \frac{z+z^*}{z+ z^*\left(\frac{u_{i,j}-z}{u_{i,j}}\right)}
\end{equation}

One can verify the derivative of $\Delta_z(z^*, u_{i,j})$ with respect to $z^*$ is positive for $z^* + z \leq c_i$, and also that the derivative of the expression for $\Delta(z^*, u_{i,j})$ when $z + z^* \geq c_i$ is negative. Thus it follows for a fixed $z$, $\Delta_z(z^*, u_{i,j})$ is maximized when $z+z^* = c_i$. 
Therefore, 
we can express $\mu_i(z, u_{i,j})$ as the following function of $z$:
\begin{equation} \label{eq:budget-gap-2}
\mu_i(z,u_{i,j})   = \frac{c_i}{z+ (c_i - z)\left(\frac{u_{i,j}-z}{u_{i,j}}\right)}
\end{equation}

The above expression is maximized when $z = c_i/2$. The expression is also strictly increasing in $u_{i,j}$; therefore by our assumption that $\max_{j} u_{ij} \leq c_i$, the expression is maximized with respect to $u_{i,j}$ when $u_{i,j} = c_i$. 
Thus plugging in $z = c_i/2$ and $u_{i,j} = c_i$, we obtain $\max \mu_i(z,u_{i,j}) = c_i/(c_i/2 + c_i/4) = 4/3$, as desired. 
\end{proof}



\begin{lemma}
If $v_i(\cdot)$ is piecewise-linear, concave, non-decreasing, and satisfies $\min_{k}\left(x_{i,k+1} - x_{i,k}\right) \geq \max_{j} u_{ij}$, then its multiplicative curvature $\mu_i$ is at most $4/3$.
\end{lemma}




\begin{proof}
Let $w = \max_j u_{i,j}$, let $(z,z^*)$ be the maximizers that define $\mu_i$, and let 
$\sigma = \sigma_i(z,w) = v_i(z + w) - v_i(z))/w$ denote the slope of the lower-bounding 
secant line given by Equation \eqref{eq:lcb-slope}.
Thus the equation of the lower bounding secant line $s(x)$ that defines $\mu_i$ is given by 
$s(x) = \sigma \cdot(x-z) + v_i(z)$. Let  $\ell_1, \ell_2$ denote the  segments containing $z$ and $z+w$ fall in, respectively.
By our assumption $\min_{k}\left(x_{i,k+1} - x_{i,k}\right) \geq \max_{j} u_{ij}$, $\ell_1$ and $\ell_2$ must be adjacent segments of the function.


%First, note that we can assume that $\ell_1$ and $\ell_2$ are adjacent segments. This is because $v_i$ is concave, so if we replace the portion of $v_i$ between $z$ and $z+w$ by extending $\ell_1$ and $\ell_2$ until they intersect, the value of $v_i$ in this region can only increase while the secant line remains unchanged, so $\mu_i$ can only increase. 

To show the lemma, it suffices to show to show that for all $x \in [z, z+w]$, $v_i(x) / s(x) \leq 4/3$. Similar to proof of Lemma \ref{fact:budget-3/4}, we establish this by showing the largest gap occurs at the intersection of $\ell_1$ and $\ell_2$. Let $\ell_1, \ell_2$ be defined as functions of $x$ by $\ell_1(x) = v'_1x+b_1$ and $\ell_2(x) = v'_2x+b_2$, respectively.
Let $\overline{x}$ denote the $x$ coordinate of their intersection point, i.e., $\ell_1(\overline{x}) = \ell_2(\overline{x})$. Since $v_i(\cdot)$ is concave, we have $v'_1 < \sigma < v'_2$ and $b_1 < v_i(z) - zx < b_2$. For $x \in [z, \overline{x}]$, the multiplicative curvature at $x$ is the ratio between $\ell_1(x)$ and $s(x)$, which is
\[
\Delta(x) = \frac{v'_1x+b_1}{\sigma \cdot (x-z) + v_i(z)}.
\]
Since the derivative of $\Delta(x)$ is positive with respect to $x$, $\Delta(x)$ increases as $x$ increases.
We can similarly show that for $x \in [\overline{x}, z+z^*]$, $\Delta(x)$ decreases as $x$ increases. Thus, $\Delta(x)$ is maximized at $x = \overline{x}$.

To complete the argument, we can transform $\ell_2(x)$ so that conjoining segments $\ell_1$ and $\ell_2$ resemble a budget-additive function. In particular, we can ``flatten'' $\ell_2(x)$ by decreasing its slope $v'_2 = 0$, which  decreases the value of $\ell_2(z+z^*)$ from $v_i(\overline{x}) + v'_2(w-\overline{x})$ to now be $v_i(\overline{x})$. Since $z$ remains fixed, the slope of $s(x)$ decreases when adjust for this change, which means the curvature (evaluated at $\overline{x}$) can only increase. Applying Lemma \ref{fact:budget-3/4}, it follows that $\mu_i \leq 4/3$.


%Similarly, we steepen $\ell_1(x)$ by setting raising slope to 1 while ensuring that it still contains $\alpha$, and the curvature again increases.

% Next, we translate the entire picture (i.e., $v_i$ and $s$) down until the left endpoint of $\ell_1(x)$ (which now has slope 1) hits the $x$-axis. Doing so decreases the values of $v_i$ and $s$ by the same amount, which increases the value of the curvature.

%Finally, recall that the width of $s$ beneath $v_i$ (i.e., $w_i$) is at most the width of $\ell_1$, and the function now under consideration is a standard budget function. By Lemma~\ref{fact:budget-3/4}, this proves that $v_i(x)/s(x) \leq 4/3$ for all $x\in [p_1, q_1]$, as desired.
\end{proof}

