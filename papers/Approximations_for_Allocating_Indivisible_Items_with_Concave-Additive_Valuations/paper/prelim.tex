\section{Preliminaries}
\label{sec:prelim}

\subsection{Convex Program Formulation and Dual Program}
\label{subsec:prelim-convex-dual}

Recall that every agent $i$ has a non-decreasing concave valuation function $v_i(\cdot)$. Our algorithm utilizes the natural assignment convex program for the problem, which we will refer to as 
$\ICA$-CP: 
\begin{gather*}
    \textnormal{($\ICACP$)}: \max \sum_i v_i(u_i) \\
    \forall i: u_i = \sum_j u_{i,j}x_{ij} \\
    \forall j: \sum_i x_{i,j} \leq 1 \\
    \forall i,j: x_{i,j} \geq 0 
\end{gather*}
The algorithm is primal-dual in nature, and uses the dual program which was defined for the online variant of the problem in \cite{devanur2012online}:
\begin{gather*}
    \textnormal{($\ICAD$)}: \min \sum_i y_i(t_i) + \sum_j p_j \\
    \forall i,j: p_j \geq u_{i,j}v_i'(t_i) \\
    \forall i,j: t_i, p_j \geq 0 
\end{gather*}
where $y_i(t_i) = v_i(t_i) - t_iv_i'(t_i)$ is as defined $y$-intercept of the tangent to $v_i$ at $t_i$. Thus we have the following lemma. 


\begin{lemma}[shown in \cite{devanur2012online}] \label{lem:duality}
The above convex programs form a primal-dual pair. That is, any feasible solution to $\ICAD$ has objective at least that of any feasible solution to $\ICACP$.
\end{lemma}

\subsection{Local Curvature Parameters}
\label{subsec:lcb}
As stated in the introduction, both the definition and guarantees provided by our algorithm depend on parameters that measure the local curvatures of each agent valuation function, both in multiplicative (we will use $\mu$) and additive senses ($\alpha$). To define these parameters, let
\begin{equation}
\label{eq:lcb-slope}
\sigma_{i}(z,w) := \frac{v_i(z+w) - v_i(z)}{w},
\end{equation}
be the slope of the lower-bounding  secant line that intersects $v_i$ at points $(z,v_i(z))$ and $(z+w, v_i(z+w))$. Define the  {\em local multiplicative curvature}\footnote{For some functions, a fixed $z^*$ that is an $\arg\max$ in \eqref{eq:mult-lcb} may not exist, and therefore in such cases the $\max$ in the definition should be replaced by a supremum. Such cases can be handled in our analysis by introducing limits when necessary.} of a function $v_i$ at point $z$ with $x$-width $w > 0$ to be:
\begin{equation}
\label{eq:mult-lcb}
\mu_{i}(z, w) := \max_{z^* \in \left(0,w\right)} \left[\frac{v_i(z+z^*)}{v_i(z) + z^*\sigma_i(z,w)} \right].
\end{equation}
Informally, $\mu_{i}(z, w)$ measures the largest multiplicative gap
%\footnote{Throughout the rest of the paper, it will be more natural to work with the factor as a value that is at most one. Therefore. Therefore, since we work with reciprocal values, 
%the ``largest" curvature corresponds to the lowest multiplicative factor
%Hence taking  the minimums in above the definitions}
between a point $z+ z^*$ on the lower bounding secant line and the function evaluated at $z+ z^*$. The definition of $\mu_{i}(z, w)$
is illustrated in Figure \ref{fig:lcb}. The overall local multiplicative curvature for agent $i$ is then defined to be $\mu_i := \max_{z, u_{i,j}} \mu_{i}(z, u_{i,j})$. 

Similarly, we define the {\em local additive curvature} for an agent at point $z$ with $x$-width $w$ to be: 

\begin{equation}
\label{eq:add-lcb}
\alpha_{i}(z, w) := \max_{z^* \in \left(0,w\right)} \left[v_i(z+z^*) - (v_i(z) + z^* \sigma_i(z,w))\right],
\end{equation}
where we again let $\alpha_i := \max_{z, u_{i,j}} \alpha_{i}(z, u_{i,j})$.


\begin{figure}
\centering
\includegraphics[scale=0.7]{figures/lcb-crop.pdf}
\caption{{\small Illustration of the definition of the multiplicative local curvature at point $z$ with width $w$ for function $v_i$ (denoted $\mu_{i}(z,w)$).}}
\label{fig:lcb}
\end{figure}




%Since $c_\ell$ is defined according to the marginal change that can occur over {\em all} points of the function (in some sense, one can think of $c_\ell$  as a bound on the multiplicative gap between the function and a ``low-resolution" derivative) intuitively it should capture a more accurate measure of a valuation function's curvature. For example, for MBA $c_\ell = 3/4$ (where $z = B/2$ and $z^* = B/2$ for a budget of $B$), capturing the fact that portions of the function do not exhibit diminishing returns. Figure \ref{fig:curv-compare} illustrates the value of $c_{\ell}$ for functions with varying degrees of curvature and maximum bid values.