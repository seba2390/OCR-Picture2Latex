\section{Integrality Gap of $\ICA$ Convex Program}
\label{sec:int-gap}


We now prove Theorem \ref{thm:int-gap}. In particular, we show that for any fixed monotone concave value function $v(\cdot)$ and maximum bid value $u$ with local multiplicative curvature $\mu$, we can construct an instance of $\ICA$
such that the optimal fractional solution to $\ICACP$ has objective $\mu$ times that of any integral assignment. 

%Similarly to algorithm, the integrality gap easily extends extends to show a $n\alpha$ additive integrality gap (Theorem \ref{thm:add-alg-bound}). We omit the details of this extension here for sake of brevity. 

\paragraph{Instance Construction.} Let $z$ be the $\arg\min$
that defines $\mu$ for function  $v(\cdot)$, and let $z^*$ be the $\arg\max$ that defines $\mu_i(z, u)$ in the definition of $\mu_i$, as shown on the left side of Fig.~\ref{fig:int-gap}.
Without loss of generality, we can assume $z^*/u$ can be expressed as 
rational number $\beta/\gamma$ where $\beta, \gamma \in \mathbb{Z}^+$, since any irrational number has arbitrarily close rational  approximation. (In which case our instance construction can be taken such a limit to obtain the desired bound).  




We construct our instance as follows. 
The valuation function of every agent is $v(\cdot)$. 
There are $\gamma$ agents and $\beta + \gamma \cdot \lceil \frac{z}{u} \rceil$ items in total. Of the items, $\beta$ of them are ``public,'' i.e., for all agents $i$ we have have $u_{i,j} = u$. Call this subset of items $M_{\pub}$.  The remaining items $\gamma \lceil\frac{z}{u}\rceil$ items are partitioned among the $\gamma$ agents so that each agent $i$ receives a set of $\lceil\frac{z}{u}\rceil$ ``private'' items $M_i$. For each item $j$ in the first $\lfloor \frac{z}{u}\rfloor$ of these private items in $M_i$, we set $u_{i,j} = u$. 
For the remaining item (if there is one) $j'$ in $M_i$, we set $v_{i,j'} = z - \lfloor \frac{z}{u}\rfloor u < u$.
For all other agents $i' \neq i$, $u_{i',j} = 0$ for all $j \in M_i$. 
Based on this construction, the sum of $u_{i,j}$ over all $j \in M_i$ is equal to $z$, and the maximum bid value in the instance is indeed $u$. 
This completes the construction, which is illustrated in Figure \ref{fig:int-gap}. 

\begin{figure}
\centering
\includegraphics[height=5cm]{figures/int-gap-crop.pdf}
\caption{{\small Illustration of integrality gap construction. Squares correspond to agents and circles correspond to items.}}
\label{fig:int-gap}
\end{figure}

\paragraph{Analysis.} Let $\OPT_F$ and $\OPT_I$ denote 
the optimal fractional and integral solutions, respectively, to the $\ICACP$ convex program for the above instance construction. 
We first show that $\OPT_F$ is obtained by evenly splitting the spend of the $\gamma$ agents among the $\beta$ public items. 
To do this, we construct a feasible dual solution $\DUAL$ to the $\ICAD$ program that has objective equal to that of $\OPT_F$.
By Lemma \ref{lem:duality}, it  then follows $\OPT_F$ is an optimal fractional solution. 
Constructing this dual solution will also be useful for showing the desired integrality gap, as it will be easier to relate the objective of $\OPT_I$ to $\DUAL$. Note that to simplify notation, for the remainder of the section we denote $t^* := z + z^*$. 

Observe that in the solution $\OPT_F$ specified above, each agent spends a total of $u\beta/\gamma$ on public items.
By the definitions of $\beta$ and $\gamma$, we have $u\beta/\gamma = z^*$.  Each agent also receives a total spend of $z$ from from their private items.  
Therefore, since there are $\gamma$ agents in the total, the objective of $\OPT_F$ is the following:

\begin{equation}
\label{eq:int-gap-opt}
\OPT_F = \gamma v(z+z^*) = \gamma v(t^*).
\end{equation}
To construct $\DUAL$,  we set $t_i = t^*$ for all agents $i$. 
Since each agent has an identical valuation function $v(\cdot)$, setting variables $p_j = u v'(t^*)$ if $j$ is a public item and setting $p_j = u_{i,j}v'(t^*)$ if $j$ is a private item in $M_i$ produces a feasible solution to $\DUAL$.

To show that the objectives of $\OPT_F$ and $\DUAL$ are the same, first observe that the definitions of $y(\cdot)$ and $v'(\cdot)$, we have the following identities: 
\begin{gather} 
\label{eq:int-gap-dual-1}
y(t^*) + (z+u)v'(t^*) = v(t^*) + (u-z^*)v'(t^*) \\
\label{eq:int-gap-dual-2}
y(t^*) + zv'(t^*) = v(t^*) - z^*v'(t^*).
\end{gather}
 Equations \eqref{eq:int-gap-dual-1} and \eqref{eq:int-gap-dual-2} give two equivalent ways of expressing the value  
of the line tangent to $v(\cdot)$ at $t^*$ evaluated at $x$-coordinates $z+u$ and $z$, respectively. Geometrically, on the LHS, we start at the $y$-intercept of the line, 
and then follow the slope of the tangent for a total $x$-width of $z+u$ (resp.\ $z$). On the RHS, we instead start at $v(t^*) = v(z+z^*)$ (i.e., the tangent point)
and follow the tangent line to $z+u$ (resp.\ backwards to $z$). 



From the construction of the instance and the definition of $\DUAL$, the objective of $\DUAL$ can be characterized as follows:
\begin{align*}
\sum_i \left(y(t^*) + \sum_{j \in M_i} p_j\right) + \sum_{j \in M_{\pub}}p_j & = \gamma  (y(t^*) + z  v'(t^*)) +  \beta u v'(t^*) \\ 
& = \beta \cdot (y(t^*) + (z +u) v'(t^*)) +  (\gamma - \beta) \cdot (y(t^*) + zv'(t^*)).
\end{align*}
By applying Equations \eqref{eq:int-gap-dual-1} and \eqref{eq:int-gap-dual-2} to the RHS above, it follows the RHS is equal to: 
\begin{equation}
\label{eq:dual-gap-obj}
\beta \cdot (v(t^*) + (u-z^*)v'(t^*)) + (\gamma-\beta) \cdot (v(t^*)-z^* v'(t^*)), 
\end{equation}
which can simplified as follows:
\begin{align*} 
\gamma v(t^*) + v'(t^*) \cdot (\beta(u-z^*) -(\gamma-\beta)z^*)  & = \gamma v(t^*) + v'(t^*) \cdot (\beta u - \gamma z^*)  \\
& = \gamma v(t^*) \\
& = \OPT_F
\end{align*}
where the second equality follows since $\beta/\gamma = z^*/u$, and the last equality follows from \eqref{eq:int-gap-opt}.
Thus $\OPT_F$ and $\DUAL$ have equivalent objective values. 


Now consider $\OPT_I$, which is obtained by assigning a unique public item to $\beta$ of the $\gamma$ agents.
(By a simple exchange argument, the objective cannot increase by assigning multiple public items to the same agent, since $v(\cdot)$ is monotone and concave.)
Each agent that receives a public item spends a total of $z + u$. 
The remaining $\gamma - \beta$ agents spend only a  total of $z$ from solely their private items. 
Thus the objective of the optimal integral solution is:


\begin{equation}
\label{eq:gap-int-obj}
\OPT_I = \beta v(z+u) + (\gamma-\beta) v(z).
\end{equation}

By definition of $\mu$, the gap between $v(z+b)$ and $v(z)$, and the (respective) points 
characterized by Equations \eqref{eq:int-gap-dual-1} and \eqref{eq:int-gap-dual-2} are equal to $\mu$. More specifically, we have
\begin{equation}
\label{eq:gap-int-char}
 \mu = \frac{v(t^*) + (u-z^*)v'(t^*)}{v(z+u)} = \frac{v(t^*) - z^*v'(t^*)}{v(z)}.
\end{equation}

Since \eqref{eq:dual-gap-obj} expresses the objective value of $\DUAL$ (and therefore $\OPT_F$, as well), 
Equations \eqref{eq:dual-gap-obj}, \eqref{eq:gap-int-obj}, and \eqref{eq:gap-int-char} together imply $\OPT_F = \mu\OPT_I$, i.e., the integrality gap of this instance is $\mu$.


