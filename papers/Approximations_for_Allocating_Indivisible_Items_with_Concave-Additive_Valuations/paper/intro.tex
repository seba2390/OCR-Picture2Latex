\section{Introduction}
\label{sec:intro}

The study of {\em indivisible} allocation has received increasing attention in recent years: given a collection of indivisible items and a set of agents --- each with a specified valuation function --- how should items be uniquely distributed among the agents as to maximize a specified measure of overall welfare? 
Many classic allocation and market models consider divisible goods that can be split fractionally among agents. However in the context of optimization,  there are many methods for computing optimal fractional assignments in polynomial time (e.g. ellipsoid methods), while indivisible variants are often NP-hard and  more algorithmically challenging.  
The consideration of other important model features and quality measures adds further complexity to the indivisible setting, such as {\em diminishing returns} in agent valuations and {\em ensuring fairness}. 
How do we allocate items uniquely when agents value additional items less on the margins? 
How do we find allocations that are efficient with respect to utilitarian welfare, while not completely starving some subsets of the agents? 
The focus of much recent work has been aimed at addressing such questions \cite{anari2018nash, barman2018finding, chakrabarty2010approximability, garg2018approximating, kalaitzis2015configuration,li2021constant}.

In this paper, we study a general yet natural allocation model that lies at the intersection of these algorithmic challenges, which we call {\em Indivisible Allocation with Concave-Additive Valuations} ($ICA$).
As input, we are given a set of $n$ agents and $m$ indivisible items, where each agent $i$ has a specified utility $u_{i,j}$ for an item $j$. An algorithm must partition the items into $n$ disjoint sets $(A_1, A_2, \ldots A_n$), one for each agent. The overall valuation agent $i$ has for her set is  $v_i(u_i)$, where her valuation $v_i: {\mathbf R}_{+} \rightarrow {\mathbf R}$ is permitted to be any monotone (non-decreasing) concave function of her total additive utility $u_i = \sum_{j \in A_i} u_{i,j}$. The objective is to maximize the utilitarian welfare of the allocation, i.e., the sum of agent valuations $\sum_iv_i( u_{i})$.  For ease of comparison to prior work, we refer to such valuations $v_i(\cdot)$ as {\em concave additive}.

\vspace{3mm} 
\noindent {\em Motivation and Background.} $\ICA$ is an indivisible variant of the fractional online problem studied by Devanur and Jain \cite{devanur2012online}. This model was primarily motivated by applications in internet advertising, where agents correspond to advertisers and items correspond to so-called ``impressions" (opportunities to show ads to users). In this setting, $u_{i,j}$ would  translate to the bid value an advertiser $i$ is willing pay for impression $j$.  The concave objective is then used to capture common contract features in internet ad systems such as under-delivery penalties and soft budgets. 

However, given the special role concave functions have played in the economics literature, we believe $\ICA$ is a natural problem to consider in its own right. 
For example, a standard class of valuations is that of {\em separable concave valuations} \cite{anari2018nash, chaudhury2018fair, chen2009settling, vazirani2011market}, which can be decomposed into the sum of monotone concave functions over the amount received from each good (or in the indivisible settings, the numbers of {\em copies} of a single good).
Thus, such valuations can  express varying degrees of diminishing returns for receiving more of the same item. 

But in many applications, agent valuations are in fact {\em inseparable}, since how much an agent values an additional item will likely depend on the other items she has already received. 
A canonical example of inseparable valuations are {\em budget-additive} functions, i.e., each valuation $v_i(u_i) = \min\left(u_i, c_i\right)$ where $c_i$ denotes the utility cap for agent $i$. 
One can view concave-additive valuations as a general class of inseparable functions that can express diminishing returns beyond simply a global cap on an agent's overall utility. 
There is a line work that has extensively studied approximation algorithms for welfare maximization of budget-additive valuations \cite{garg2001approximation, andelman2004auctions, azar2008improved, chakrabarty2010approximability, kalaitzis2015configuration}. To the best of our knowledge, the approximability of the concave-additive case has yet to be considered for indivisible items. 

\vspace{3mm} 
\noindent {\em Extension to Nash Welfare Maximization.} Another key motivation for considering concave-additive functions is that an {\em additive} approximation for $\ICA$
translates to a standard multiplicative approximation for the problem of {\em Nash Welfare Maximization}. 
Here, the objective is to maximize the weighed product of valuations $\left(\prod_i (v_i(u_i))^{\eta_i}\right)^{1/\eta}$, where $\eta_i > 0$ is the weight of each agent and $\eta = \sum_i \eta_i$ is the sum of weights.  
The main appeal of Nash welfare is that it naturally strikes a balance between optimizing utilitarian welfare versus max-min fairness (e.g., completely starving even just one agent will result in an overall objective of zero).

Observe that we can convert the objective to an instance of $\ICA$ by taking the logarithm, giving us the objective $\frac{1}{\eta}\sum_i \eta_i\ln(v_i(u_i))$. As long as each valuation $v_i(\cdot)$ is monotone and concave, we obtain an instance of $\ICA$ since the logarithm of a monotone concave function is still monotone and concave. Furthermore, an additive approximation of $\alpha$ for this log objective translates to a multiplicative approximation of $e^{\alpha}$ for the original product objective.


It has been known for over eighty years that the optimal Nash-welfare assignment for divisible goods can be obtained by solving the famous Eisenberg-Gale convex program \cite{eisenberg1959consensus}. 
However, the indivisible case was less-understood until the recent work of Cole and Gkatzelis~\cite{cole2015approximating}, who obtained the first constant-approximation algorithm for {\em symmetric agents} with {\em additive valuations} (i.e., for all agents $i$ we have $\eta_i =1$ and  $v_i(u_i) = u_i$).
Since this result, the problem has been heavily studied for indivisible items, both in terms of its approximability and its appealing fairness properties. 
Much of the work on approximations has focused on the symmetric  case, which  has produced a line of results obtaining constant approximations for subsequently more general valuation functions, such as budget additive \cite{garg2018approximating}, separable piecewise-linear \cite{anari2018nash}, Rado valuations \cite{garg2021approximating}, and then recently for general submodular valuations \cite{li2021constant}. 


Unfortunately even for additive valuations, the case of asymmetric agents (general weights $\eta_i > 0$) still remains a key open problem. The best known approximation for additive valuations is currently $O(n)$ by a result of Garg et al.~\cite{garg2001approximation}, which also extends to budget-additive functions. Very recently, polynomial-time approximation schemes were given by Garg et al.~\cite{garg2021tractable} for the special cases of identical agents ($u_{i,j} = u_j$ for all $i$) and two-value agents (for all agents $i$, $u_{i,j} > 0$ for at most two items $j$).




\subsection{Contributions and Techniques} 
In this paper, we obtain {\em tight multiplicative and additive approximations for the $ICA$ problem}.
Our approximations depend on novel curvature parameters that measure the degree of local change that can occur in the concave valuation of each agent, which may be of independent interest. 
For a given agent valuation  $v_i(\cdot)$ and maximum item utility  $\max_{j}u_{i,j}$ for agent $i$, we denote the {\em local multiplicative and additive curvatures} of $v_i(\cdot)$ as $\mu_i$ and $\alpha_i$, respectively, where higher/lower values of $\mu_i$ and $\alpha_i$ correspond to higher/lower degrees of local curvature  in $v_i(\cdot)$ with respect to a local change of $\max_{j}u_{i,j}$.
These parameters are formally defined and diagrammed in our preliminaries section (Section \ref{subsec:lcb}). 

Furthermore, we show that our additive guarantees for $\ICA$ extend to {\em new approximations for Asymmetric Nash Welfare with smoothed agent valuations}. The details and applications of each approximation type are stated formally below.



\subsubsection{Multiplicative Guarantees} 
\label{subsub:mult-guar}
For our main result, we give a  polynomial-time algorithm for any general instance of $\ICA$ that achieves an approximation ratio of $(1+\epsilon)\max_{i}\mu_i$.
Note that in the setting of multiplicative guarantees, we assume that each agent valuation is non-negative ($v_i:{\mathbf R}_+\rightarrow{\mathbf R}_+$) and differentiable.

\begin{theorem} 
\label{thm:pd-alg}
Consider an instance of $\ICA$ such that each $v_i:{\mathbf R}_+ \rightarrow {\mathbf R}_+$ has well-defined first derivative.  
Let $\umaxi = \sum_j u_{ij}$, and let $\rmax = \max_i(v_i'(0)\umaxi/v_i(\umaxi))$. Then there exists an algorithm that achieves an approximation of  $(1+\epsilon)\max_{i}\mu_i$ for $\ICA$ that runs in time $O(mn\ln(\rmax/\epsilon)/\epsilon)$.
\end{theorem}

We emphasize that our approach is efficient, easy to implement, and does not require  convex program solvers. Instead, it leverages the duality techniques introduced  in \cite{chakrabarty2010approximability} and \cite{devanur2012online} to guide a simple tatonnement style local-search algorithm. 
In fact, one can view our algorithm
as a convex generalization of the LP primal-dual $(4/3)$-approximation algorithm for budget-additive functions in \cite{chakrabarty2010approximability}. This algorithm begins by initializing dual prices greedily and then continuously lowers prices, defecting items to the highest bidder throughout. 
The $4/3$ approximation ratio is then derived from a set of algebraically obtained equations whose solution express the minimum ratio ensuring that as items defect, no agent ends up overspending on an allocation where they received less than what they are given in an optimal fractional solution (which guarantees prices never need to be raised). Clearly  it would be difficult (or impossible, as $v_i(\cdot)$ need not have a closed form)  to derive such a ratio for a set of generic concave-additive functions. Thus for our main technical insight, we show this algebraic approach can be bypassed via more elegant geometric arguments that coincide  directly with the definition of our curvature parameter $\mu_i$.





Additionally, we show our approximation is tight among algorithms that  utilize the natural assignment convex program. 
In particular, we establish the integrality gap of the $\ICA$ convex programming relaxation is precisely our multiplicative curvature parameter.

\begin{theorem}
\label{thm:int-gap}
Consider an instance $\ICA$ where each agent has valuation function $v(\cdot)$ and $u = \max_{i,j}u_{i, j}$. Then the integrality gap of the $\ICA$ assignment convex program is the multiplicative curvature of the instance $\mu$ (determined by $v(\cdot)$ and $u$). 
\end{theorem}

As an application of this result, we note that concave valuations are often examined in the special case {\em piecewise-linear functions} (e.g., see \cite{anari2018nash, chaudhury2018fair, vazirani2011market}), since any continuous concave function can be closely approximated by one that is piecewise-linear. 
In the context of $\ICA$, this corresponds to the following definition: each agent valuation function $v_i(\cdot)$ is defined over on a sequence of conjoined $\lambda_i$ line segments.  %Let $v'_{ik} \geq 0$ denote the slope of the $k$th segment.
Let $x_{i,k}$ denote the transition point on the $x$-axis between the $k$-th and $(k+1)$-th segments (where $x_{i,0} = 0$). For any such function, we give an algorithm that achieves an approximation ratio of at most $(1+\epsilon)4/3$, as long as the maximum utility gained for a single item is at most the length (along the $x$-axis) of any segment of the piecewise-linear function.


\begin{theorem} 
\label{thm:pl-mult}
Consider an ICA instance where $v_i(u_i)$ is a linear piecewise function such that $\min_{k \in [\lambda_i-1]}\left(x_{i,k+1} - x_{i,k}\right) \geq \max_{j} u_{ij}$.
Then there exists an algorithm whose approximation ratio is $(1+\epsilon)\max_{i}\mu_i \leq (1+\epsilon)4/3$. 
\end{theorem}

For the special case of a budget-additive function where agent $i$ has utility cap $c_i$, the condition required by Theorem~\ref{thm:pl-mult} is equivalent to the standard assumption that $\max_{j} u_{i,j} \leq c_i$. Thus, our bound essentially (i.e., barring the $4/3 - \delta$ approximation for a small constant $\delta$ via the configuration LP in \cite{kalaitzis2015configuration}) matches the state-of-the-art approximation for budget-additive valuations but in the more general setting of piecewise-linear functions. 

We also note that as $\max_{j}u_{ij} \rightarrow 0$, $\mu_i$ tends towards 1. Therefore, for so-called ``small bids" instances, our algorithm outputs near-optimal allocations. To the best of our knowledge, the best algorithm for the small-bids setting is  the online fractional algorithm given in \cite{devanur2012online}, which
for many concave-additive functions achieves a ratio $\gg 1$.\footnote{For example, when $f(x) = x^\alpha$ for $\alpha \in (0,1]$, the algorithm's approximation is $\alpha^{-\alpha}$ (e.g., when $\alpha = 1/2$, $ \alpha^{-\alpha} \approx 1.414$).}


\subsubsection{Additive Guarantees}
\label{subsub:add-guar}


Our multiplicative algorithm and integrality gap example extend to obtaining a tight additive guarantee for $\ICA$ (outlined in Section 
\ref{subsec:additive}, Theorem \ref{thm:add-alg-bound}).
For the main application of our techniques, we show these additive bounds provide new multiplicative approximations for the Nash-welfare objective with asymmetric agents.

As was the case for our multiplicative results,  our additive bound also scales with the integrality gap of the assignment convex program relaxation. 
%(Again, the gap is determined by the local curvature parameters $\max_i\alpha_i$). 
It has been known that the Nash-welfare objective has an integrality gap of $\Omega(n)$, even for symmetric agents. In fact, the original techniques of Cole and Gkatzelis \cite{cole2015approximating} were directly aimed at circumventing the linear integrality gap. Given that our analysis competes against the optimal fractional objective of the standard assignment CP, the $\Omega(n)$ integrality gap remains a barrier to directly applying our techniques. 

%(which then laid the foundation for obtaining constant approximation for more general valuations functions). 

However, another proposed alternative for naturally handling the large gap is to examine agents with {\em smooth valuations} \cite{fain2018fair, fluschnik2019fair}.
In the smooth valuation setting, we give each agent a (potentially fractional) copy of her favorite item at the outset of the allocation, which relaxes the degree to which the objective penalizes under allocations.
Smoothing the valuations essentially ``shifts'' them to the more well-behaved portions of the log objective, but this setting still captures part of the technical challenge of standard (non-smooth) asymmetric objective. 
For example, the algorithm in \cite{garg2020approximating} still has an approximation ratio of $\Omega(n)$ even for smooth valuations \cite{garg2020approximating}.\footnote{One can verify that algorithm in  \cite{garg2020approximating} still has an approximation of $\Omega(n)$ for the tight example instance (Section 6.3) when modified so that each agent gets an additional copy of her most-valued item at the outset.}


More formally, we define an mooth instance with asymmetric agents as follows. First observe that for additive valuations $v_i(u_i) = u_i$, we can scale the objective of each agent $i$ by $(\max_j u_{i,j})^{-\eta_i}$ without changing the approximation factor of the algorithm. 
Therefore, wlog we can assume that $\max_{j}u_{i,j} = 1$. We then define the smooth version of the objective to be $\left(\prod_i (u_i + \omega )^{\eta_i}\right)^{1/\eta}$, where $\omega \in (0,1]$ denotes the {\em smoothing parameter} for the instance. By applying our techniques, we obtain the following result.

\begin{theorem}
\label{thm:snsw}
Consider an instance of Asymmetric  Nash Welfare Maximization with smooth additive valuations. Then there exists an algorithm that runs in time $O(nm^2/(\epsilon \omega))$ that achieves an approximation of $O(e^{\epsilon}/(\omega\ln(1+1/\omega)))$ for smoothing parameter $\omega \in (0,1]$. 
\end{theorem}

Thus for any constant smoothing parameter $\omega$, we obtain a constant approximation for the problem. For example when $\omega = 1$, the approximation ratio of the algorithm is $\approx 1.061$ as $\epsilon \rightarrow 0$. This result also extends to piecewise-linear valuations, yielding an $O(1)$ approximation under the same assumptions as Theorem \ref{thm:pl-mult} when $\omega = \Omega(1)$ (stated formally in Section \ref{subsub:anw-pl}, Theorem \ref{thm:anw-pl}).


Furthermore for additive valuations, the resulting algorithm has an interesting combinatorial interpretation, 
which we call the {\em Weighted Bang-Per-Buck} (WBB) algorithm. 
Many of the aforementioned approximations for the symmetric case (e.g., \cite{anari2018nash, barman2018finding, chaudhury2018fair, garg2018approximating}) also use tatonnement-style algorithms that adjust prices $p_j$ for each item $j$, maintaining throughout that
each item always assigned to a maximum ``bang-per-buck'' (MBB) agent, i.e., agents $i$ such that the ratio $u_{i,j}/p_j$ is maximized. 
In our WBB algorithm, we instead adjust a uniform bid $b_i$ that agent $i$ makes for {\em all} items, but then each item is assigned based on maximum {\em weighted} bang-per-buck ratios, i.e., item $j$ is assigned to the agent that maximizes $(\eta_i u_{i,j})/b_i$. To the best of our knowledge, weighted bang-per-buck ratios have yet to be considered in the context of algorithm design for asymmetric Nash welfare. We hope this concept and interpretation proves useful for making progress in the very challenging non-smooth case. 

\subsection{Related Work}

%In the computational economics literature, concave-additive valuations have been primarily studied 
%in the context of market equilibrium. 

There is an extensive body of work that has examined the approximability of  welfare maximization for the special case of budget-additive functions. 
A series of results \cite{garg2001approximation, andelman2004auctions, azar2008improved, chakrabarty2010approximability, kalaitzis2015configuration} improved the best approximation to $4/3 - \delta$ for a small constant $\delta$, while the hardness lower bound currently sits at 16/15 \cite{chakrabarty2010approximability}. 
Budget-additive valuations have also been heavily studied in the online setting (typically called the {\em Adwords} problem in this context, motivated by applications in internet advertising), often for divisible items or with the small bids assumption ($u_{i,j} \rightarrow 0)$. 
Please see the excellent survey by Mehta \cite{mehta2013online} for an overview of this area.
We note the algorithm of Devanur and Jain \cite{devanur2012online} for concave-additive valuations with fractional items is viewed as the state-of-the-art approximation in the online setting. 
Their analysis also expresses its approximation as a parameter that depends on the curvature of the concave functions for a  given instance. 
(In the fractional setting, this parameter is instead determined by the solution to a differential equation that tends towards 1 as the function becomes more linear.)

As discussed earlier, there has been an explosion of work on the prolem of Nash Welfare Maximization for indivisible items since the seminal result of Cole and Gkatzelis~\cite{cole2015approximating}, 
who gave the first constant approximation for symmetric agents with additive valuations, achieving a ratio of $2e^{1/e} \approx 2.889$. Later, Cole et al.~\cite{cole2017convex} gave a tight factor 2 analysis of the same algorithm.
 Barman et al.~\cite{barman2018finding} improved the approximation to $e^{1/e} \approx 1.444$, matching the integrality gap example in \cite{cole2017convex}. Since these results, the symmetric case has been examined for a variety of more general valuation functions \cite{garg2018approximating, anari2018nash, garg2021approximating, li2021constant, chaudhury2018fair, barman2021approximating}.
 
 The Nash-welfare objective has also been extensively studied for its appealing fairness properties (e.g.,  Caragiannis et al.~\cite{caragiannis2019unreasonable} call the objective ``unreasonably fair'').  Conitzer et al.~\cite{conitzer2017fair} first gave three relaxations of the proportionality notion of fairness and showed that an optimal solution to the objective satisfies/approximates all three relaxations. For additional results on fairness in relation to Nash welfare, see \cite{barman2018finding, plaut2020almost, mcglaughlin2020improving}. The case of smooth agent valuations was first considered by Fain et al.~\cite{fain2018fair}, also in the context of fairness. Fluschnik et al.~\cite{fluschnik2019fair} showed that this objective is NP-hard to maximize.



Finally, we mention that $\ICA$ is a special case of {\em Submodular Welfare Maximization}, 
where each agent's valuation is given as a general submodular set function.
The optimal approximation in this general setting is $e/(e-1) \approx 1.581$ \cite{vondrak2008optimal, khot2005inapproximability}.
A recent line of work also examined  submodular maximization with {\em bounded curvature}  \cite{conforti1984submodular, Sviridenko2015OptimalAF, vondrak2010submodularity}. 
In this setting, Sviridenko et al.\ showed an optimal approximation of $e/(e-c)$~\cite{Sviridenko2015OptimalAF}, where $c$ is a measure of the {\em total curvature} of the submodular valuation function of the instance.\footnote{Intuitively, $c \in [0,1]$ ($c \rightarrow 1$ implies more curvature) measures the multiplicative gap between the marginal gain of receiving an item after first receiving all other items, versus the marginal gain from only receiving the item.} 
We note that since $c$ is a total curvature parameter (in contrast to our local curvature parameters), many natural concave-additive functions will have a total of curvature of $c=1$.\footnote{For example, for any sufficient large $\ICA$ instance where $v_i'(u_i) \rightarrow 0$ as $u_i \rightarrow \infty$, the total curvature $c$ will be 1.}  Thus in such instances, this bound does not improve upon the $e/(e-1)$ approximation obtained in the general submodular function case.

\subsection{Paper Organization}
The organization of the paper is as follows. In our preliminaries  (Section \ref{sec:prelim}), we outline the convex program relaxation  and dual program of $\ICA $ defined by Devanur and Jain in \cite{devanur2012online} (Section \ref{subsec:prelim-convex-dual}), and then formally define our multiplicative and additive curvature parameters $\mu_i$ and $\alpha_i$ (Section \ref{subsec:lcb}).
For our results, we first present our multiplicative (Sections \ref{sec:pd-alg-def} and \ref{subsec:pd-analysis}) and additive (Section \ref{subsec:additive}) guarantees for general instances of $\ICA$ in Section \ref{sec:ica-algo}.  
Next, we present our tight integrality gap example in Section \ref{sec:int-gap}. 
We conclude in Section \ref{sec:apps} with the two key applications of our techniques, showing our additive bounds for $\ICA$ translate to constant approximations for smooth-asymmetric Nash Welfare Maximization with additive valuations (Section \ref{subsec:smooth-anw}), and outlining the extension to smooth Nash welfare with piecewise-linear valuations (Section \ref{subsub:anw-pl}). 
Finally, we examine our multiplicative guarantees in the special case of piecewise-linear valuations, obtaining an approximation of $(1+\epsilon)4/3$ (Section \ref{subsec:app-pl-welfare}).
