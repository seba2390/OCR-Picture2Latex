\documentclass[journal=nalefd, manuscript=letter, layout=twocolumn]{achemso}

%% packages
\usepackage{amssymb, physics, titlesec}
\usepackage[version=4]{mhchem}
\usepackage[separate-uncertainty = true]{siunitx}
\usepackage[hidelinks]{hyperref}
\usepackage{xcolor}
\usepackage[normalem]{ulem}

\let\titlefont\undefined
\makeatletter
\let\l@addto@macro\relax
\makeatother
\usepackage[fontsize=10pt]{scrextend}
\mciteErrorOnUnknownfalse

%% apperance
\titleformat{\subsection}[runin]{\bfseries}{}{}{}[]
\let\oldmaketitle\maketitle
\let\maketitle\relax
\renewcommand{\topfraction}{0.9}
\renewcommand{\dbltopfraction}{0.5}
   
%% macros
\newcommand{\supp}[1]{Supporting Section~S#1}
\newcommand{\hl}[1]{\textcolor{blue}{#1}}
\newcommand{\del}[1]{\textcolor{red}{\sout{#1}}}
\newcommand{\comment}[1]{\textcolor{red}{\footnote{\textcolor{red}{#1}}}}
%% front matter
\title{Resolving the controversy in biexciton binding energy of cesium lead halide perovskite nanocrystals through heralded single-particle spectroscopy}

\author{Gur Lubin}
\affiliation{Deptartment of Physics of Complex Systems, Weizmann Institute of Science, Rehovot, Israel}

\author{Gili Yaniv}
\author{Miri Kazes}
\affiliation{Deptartment of Molecular Chemistry and Materials Science, Weizmann Institute of Science, Rehovot, Israel}

\author{Arin Can Ulku}
\author{Ivan Michel Antolovic}
\author{Samuel Burri}
\affiliation{School of Engineering, École polytechnique fédérale de Lausanne (EPFL), Neuchâtel, Switzerland}

\author{Claudio Bruschini}
\author{Edoardo Charbon}
\affiliation{School of Engineering, École polytechnique fédérale de Lausanne (EPFL), Neuchâtel, Switzerland}

\author{Venkata Jayasurya Yallapragada}
\affiliation{Deptartment of Physics of Complex Systems, Weizmann Institute of Science, Rehovot, Israel}
\email{venkata-jayasurya.yallapragada@weizmann.ac.il}

\author{Dan Oron}
\affiliation{Deptartment of Molecular Chemistry and Materials Science, Weizmann Institute of Science, Rehovot, Israel}
\email{dan.oron@weizmann.ac.il}

%% main
\begin{document}
\twocolumn[
\begin{@twocolumnfalse}
\oldmaketitle
\begin{abstract}
Understanding exciton-exciton interaction in multiply-excited nanocrystals is crucial to their utilization as functional materials. Yet, for lead halide perovskite nanocrystals, which are promising candidates for nanocrystal-based technologies, numerous contradicting values have been reported for the strength and sign of their exciton-exciton interaction. In this work we unambiguously determine the biexciton binding energy in single cesium lead halide perovskite nanocrystals at room temperature. This is enabled by the recently introduced SPAD array spectrometer, capable of temporally isolating biexciton-exciton emission cascades while retaining spectral resolution. We demonstrate that \ce{CsPbBr3} nanocrystals feature an attractive exciton-exciton interaction, with a mean biexciton binding energy of \SI{10}{meV}. For \ce{CsPbI3} nanocrystals we observe a mean biexciton binding energy that is close to zero, and individual nanocrystals show either weakly attractive or weakly repulsive exciton-exciton interaction. We further show that within ensembles of both materials, single-nanocrystal biexciton binding energies are correlated with the degree of charge-carrier confinement.
\\
\textbf{Keywords:} perovskite nanocrystals, quantum dots, biexciton binding energy, single-particle spectroscopy, SPAD arrays
\end{abstract}
\end{@twocolumnfalse}
]

% -----------------------------------------------------------------------------
% --------------------------------- Figures -----------------------------------
% -----------------------------------------------------------------------------
\begin{figure}[t]
    \centering
    \includegraphics[width=\linewidth]{figures/particles.pdf}
    \caption{
    \textbf{Particles investigated in this work.} 
    \textbf{a)} Transmission electron micrograph of the \ce{CsPbBr3} NCs investigated in this work. 
    \textbf{b)} Transmission electron micrograph of the \ce{CsPbI3} NCs investigated in this work. Both scale bars are \SI{20}{nm}.
    \textbf{c)} Ensemble emission (solid lines) and absorption (dashed dotted lines) of the \ce{CsPbBr3} (green) and \ce{CsPbI3} (red) NCs. Blue line marks the excitation wavelength (\SI{470}{nm}).
    }
    \label{fig:particles}
\end{figure}

\begin{figure*}[t]
    \centering
    \includegraphics[width=.9\linewidth]{figures/single.pdf}
    \caption{
    \textbf{Heralded spectroscopy of a single particle.} 
    \textbf{a)} A schematic illustration of the heralded spectroscopy scheme. A linear SPAD array is placed at the output of a grating spectrometer such that each SPAD pixel detects a different wavelength. The data from each SPAD pixel consists of the absolute arrival times of photons. By identifying the first and second arriving photons in each coincidence detection (BX and 1X, respectively), their corresponding energies can be extratced ($E_{BX}$ and $E_{1X}$). 
    \textbf{b)} 2D histogram of photon pairs following the same excitation pulse, from a 5-minute measurement of a single \ce{CsPbBr3} NC. Green dashed line is a guide to the eye marking both photons with the same energy (undetectable by the system).
    \textbf{c)} BX spectrum (red dots) and 1X spectrum (blue circles) extracted by full horizontal and full vertical binning of panel (b), respectively. Grey area is the 1X spectrum (normalized) extracted by summing over all detected photons. Red solid line and blue dashed line represent fit of the BX and 1X spectra, respectively, to Cauchy-Lorentz distributions. Binding energy for this specific NC, estimated as the difference between the spectral peaks of the two fits, is $\varepsilon_b = 13.5\pm\SI{1.8}{meV}$.
    }
    \label{fig:single}
\end{figure*}

\begin{figure}[t]
    \centering
    \includegraphics[width=.9\linewidth]{figures/br.pdf}
    \caption{
    \textbf{\ce{CsPbBr3} binding energy.} 
    \textbf{a)} Binding energy histogram for 60 NCs. Mean single-particle error is $\pm\SI{3.1}{meV}$. 
    \textbf{b)} Binding energy as a function of 1X emission peak. 
    \textbf{c)} Binding energy as a function of $g^{(2)}(0)$. 
    \textbf{std} - standard deviation, \textbf{CC coeff} - cross-correlation coefficient. \textbf{p-value} - p-value of the cross-correlation.
    }
    \label{fig:br}
\end{figure}

\begin{figure}[t]
    \centering
    \includegraphics[width=.9\linewidth]{figures/iod.pdf}
    \caption{
    \textbf{\ce{CsPbI3} binding energy.} 
    \textbf{a)} Binding energy histogram for 20 NCs. Mean single-particle error is $\pm\SI{4.8}{meV}$. 
    \textbf{b)} Binding energy as a function of 1X emission peak. 
    \textbf{c)} Binding energy as a function of $g^{(2)}(0)$. 
    \textbf{std} - standard deviation, \textbf{CC coeff} - cross-correlation coefficient. \textbf{p-value} - p-value of the cross-correlation.
    }
    \label{fig:iod}
\end{figure}
% -----------------------------------------------------------------------------
% ------------------------------- main text -----------------------------------
% -----------------------------------------------------------------------------

\section*{Introduction}
Colloidal semiconductor nanocrystals (NC’s) have been extensively studied over the last three decades, owing to the ease of their synthesis and tunability of their photo-physical properties\cite{Efros2021}. Absorption of a photon by a NC leads to the formation of an exciton, a bound electron-hole pair, whose energy can be precisely tuned by varying the physical dimensions of the NC\cite{Brus1984}. In well passivated direct gap NCs, the dominant relaxation route of excitons is via photoluminescence (PL). Additional complexity is introduced when NCs are further excited, by absorbing multiple photons, to generate mutli-excitonic states. In the simplest of these states, the biexciton (BX), two excitons are confined within the NC.

PL from the BX state can serve as a probe to investigate exciton-exciton interaction within the NC. Relaxation from the BX to the singly-excited (1X) state can occur through radiative PL process or via non-radiative Auger processes\cite{Melnychuk2021}. Hence, the probability of radiative relaxation from the BX state, the BX quantum yield, is indicative of the relative rates of the two processes. A cascaded radiative relaxation from BX to 1X and further to the ground (G) state, results in the emission of two photons in rapid succession. The energy of the first photon ($E_{BX}$) will be shifted from the second ($E_{1X}$), according to exciton-exciton interaction. The value of this shift, the BX binding energy ($\varepsilon_b \equiv E_{1X} - E_{BX}$), is defined to be positive for attractive interaction. In intrinsic homogeneous or type-I NCs, where all charge-carriers are confined to the same volume, this interaction is typically attractive and stronger than in the bulk, due to the correlation energy of the confined excitons. In type-II heterostructure NCs, where electrons and holes are spatially separated, Coulombic repulsion of like-charged carriers can overwhelm this correlation energy, and result in an overall repulsive interaction\cite{Oron2007}. Significant effort has been directed at the evaluation and control of this value in II-VI and III-V semiconductor NCs, as it is critical to enhance their performance in various applications, such as sources of quantum light\cite{Senellart2017}, lasing media and LEDs\cite{Melnychuk2021} and photovoltaics\cite{Kramer2011}.

In recent years, there has been a surge of interest in lead halide perovskites (LHP) NCs of the form \ce{APbX3}, where A is a monovalent cation and X a halide anion. Their prominent features: near unity PL quantum yield, defect tolerance and tunable emission across the visible spectrum, have made them a promising candidate for various optoelectronic applications\cite{Protesescu2015,Kovalenko2017}. Additionally, at cryogenic temperatures, they exhibit long PL coherence times, which are desirable for emerging quantum optical technologies such as generation of coherent single-photons\cite{Utzat2019} and entangled photon pairs\cite{Akopian2006}. As in their II-VI and III-V counterparts, many of these applications stand to benefit, or even depend on reliable estimation of the BX binding energy.

However, the value of the BX binding energy in LHP NCs is a subject of current debate. Reported values for the prototypical all-inorganic \ce{CsPbBr3} NCs vary from +\SI{100}{meV}\cite{Castaneda2016} to \SI{-100}{meV}\cite{Dana2021}, while other experimental and theoretical works suggest a much lower bound of $\abs{\varepsilon_b}<\SI{20}{meV}$\cite{Shulenberger2019,Nguyen2020}. Common to all previous experimental works is their reliance on ensemble measurements. While these techniques proved invaluable in studying multiexcitonic states in NC, their ensemble nature introduces several possible sources for the estimation errors which may underlie the existing discrepancies. First, ensemble methods require fitting data to a model, and quantitative results often depend on the model chosen to analyze and interpret the data\cite{Makarov2016,Ashner2019}. In particular, the BX contribution might be hard to disentangle from other photo-physical or chemical processes such as charging or sintering\cite{Shulenberger2019,Lubin2021}, leading to ambiguities. Additionally, most methods require resolving the BX and 1X peaks spectrally, which might prove challenging when $\varepsilon_b$ is much smaller than the 1X homogeneous and inhomogeneous spectral broadening\cite{Shulenberger2019,Lubin2021}. Finally, the size heterogeneity, inherent to colloidally synthesized NC ensembles, can introduce systematic biases due to size dependent absorption cross-section of the 1X and BX states.

Room temperature single-particle heralded spectroscopy has been recently introduced as a way to overcome these shortcomings of ensemble approaches.\cite{Lubin2021,Vonk2021} This is achieved by temporally isolating photon pairs originating from the $BX{\rightarrow}1X{\rightarrow}G$ cascade of single-particles, and is hence free of all the aforementioned biases and ambiguities. In this letter, we utilize this technique to unambiguously determine the BX binding energies of the prototypical LHP NCs  \ce{CsPbBr3} and \ce{CsPbI3}. All \ce{CsPbBr3} single-particles measured featured an attractive exciton-exciton interaction ($\varepsilon_b=10\pm\SI{6}{meV}$), and a clear correlation of the BX binding energy with charge-carrier confinement was observed. Interestingly, \ce{CsPbI3} showed either weakly attractive or weakly repulsive exciton-exciton interaction with an average response around zero binding energy ($\varepsilon_b=1\pm\SI{9}{meV}$).

\section*{Results and Discussion}

\subsection*{Nanocrystals in this work}
Perovskite NCs investigated in this work were synthesised according to references\cite{Cao2020,Ahmed2018} (\ce{CsPbBr3}), and reference\cite{Pan2020} (\ce{CsPbI3}), with minor modifications (see \supp{1}). \ce{CsPbBr3} NCs, seen in \autoref{fig:particles}a, feature an edge size distribution of $5.9\pm\SI{1.3}{nm}$, \SI{2.44}{eV} ensemble emission peak and ${\sim}100\%$ quantum yield. For \ce{CsPbI3}, seen in \autoref{fig:particles}b, two size populations are visible. Smaller NCs (${\sim}80\%$ of the particles) with an edge size distribution of $7.2\pm\SI{1.9}{nm}$, and larger NCs with an edge size distribution of $15.4\pm\SI{3.3}{nm}$. The ensemble emission peak is at \SI{1.84}{eV} and the quantum yield is ${\sim}42\%$. Samples of isolated nanocrystals were prepared by spin coating a dilute solution of the NCs dispersed in a 3wt\% solution of poly(methylmetacrylate) (PMMA) in toluene on a glass coverslip.

\subsection*{Single-particle heralded spectroscopy.}
In order to measure the BX binding energy in single NCs, we use heralded spectroscopy - a technique that utilizes the temporal correlation of photon detections to unambiguously resolve the BX and 1X emission spectra. 
This technique was recently introduced and utilized to measure the BX binding energy of single CdSe/CdS/ZnS quantum dots at room temperature\cite{Lubin2021}. Briefly, an inverted microscope with a high numerical aperture objective is used to focus pulsed laser illumination on a single NC, and collect the emitted fluorescence. The collected fluorescence is then passed through a Czerny-Turner spectrometer with a single-photon avalanche diode (SPAD) array detector, so that each detected photon is time-stamped according to its arrival time, and energy-stamped according to the array pixel it was detected in (\autoref{fig:single}a). Post-selecting only photon pairs that follow the same excitation pulse, robustly isolates BX-1X emission cascades from emission of other overlapping emitting states, such as 1X or trions. The pump power is adjusted so that the average number of photons absorbed by a NC per pump pulse ($\ev{N}$) is low ($<0.4$, see \supp{2}). This helps prevent rapid deterioration of the NCs and minimize excitation of higher multiexcitonic states. A thorough description of the system and technique is given in reference\cite{Lubin2021}, and some modifications made to accommodate the different fluorescence parameters of the LHP NCs are described in \supp{3}.

\autoref{fig:single}b is a 2D-histogram of such post-selected photon pairs from a 5-minute measurement of a single \ce{CsPbBr3} NC. It shows the energy of the first arriving photon ($E_{BX}$) as a function of the second arriving photon ($E_{1X}$) of the pair. The green dashed line is a guide to the eye, marking same energy for both photons (undetectable by the system due to pixel dead time). The asymmetry of the histogram around this diagonal is indicative of an attractive exciton-exciton interaction ($E_{BX}$ is typically smaller than $E_{1X}$). This energy difference is quantified in \autoref{fig:single}c where the BX (red dots) and 1X (blue rings) spectra are extracted by full horizontal and full vertical binning, respectively, of \autoref{fig:single}b. This identification is corroborated by the good agreement between the 1X spectrum, and the spectrum of all detected photons (grey area, normalized).
The emission peaks of the BX and 1X spectra are estimated from fits to Cauchy-Lorentz distributions (matching color lines), and the BX binding energy is estimated as the difference in peak positions. For this specific NC, $\varepsilon_b = 13.5\pm\SI{1.8}{meV}$ (All errors in this paper are estimated as the 68\% confidence interval of the fit).

Two further insights were extracted from the same data-set. First, the normalized second order correlation of photon arrival times ($g^{(2)}(0)$), was calculated by the method described in reference\cite{Lubin2019}. This value is defined as the ratio between the number of detection pairs following the same excitation pulse and the expected number for a classical Poissonian emitter. The presence of the additional exciton in a doubly-excited NC increases the probability of nonradiative BX to 1X decay via Auger recombination. As a consequence, fewer photon pairs are emitted leading to $g^{(2)}(0)<1$, a phenomena termed photon anti-bunching. Hence, the value of $g^{(2)}(0)$ helps quantify the PL quantum yield of the BX state\cite{Nair2011}. 

Second, the \emph{time-gated} second order correlation of photon arrival times ($\Hat{g}^{(2)}(0)$) was calculated. This is performed by post-selecting only detections arriving within a time window of 1 to \SI{30}{ns} after any excitation pulse, and applying the same $g^{(2)}(0)$ calculation procedure to the resulting filtered data. Most multiexciton emission processes occur at timescales shorter than 1 ns (see \supp{4}), and are therefore filtered out by this time window. In single NCs, multi-exciton states are the only source for multiple photon detections following the same excitation pulse. Therefore, a low $\Hat{g}^{(2)}(0)$ is a good indication of whether the observed emission originates from a single particle or not\cite{Mangum2013,Benjamin2020}. ($g^{(2)}(0)$ and $\Hat{g}^{(2)}(0)$ are further discussed in \supp{5}). For this specific NC, $g^{(2)}(0) = 0.175\pm0.008$ and $\Hat{g}^{(2)}(0) = 0.012\pm0.003$.

\subsection*{\ce{CsPbBr3 NCs}.}
\autoref{fig:br}a shows the BX binding energy of 60 single \ce{CsPbBr3} NCs, determined using the procedure illustrated in \autoref{fig:single}. In our measurements, we maintain $\ev{N}\!\sim\!0.1$, and obtain a mean single-particle $\varepsilon_b$ error of $\pm\SI{3.1}{meV}$. To filter out accidental measurements of non-isolated NCs, only measurements where $\Hat{g}^{(2)}(0)<0.2$ are considered. All particles feature an attractive exciton-exciton interaction ($\varepsilon_b>0$), and the distribution is centered around $\varepsilon_b=10\pm\SI{6}{meV}$. \autoref{fig:br}b displays the binding energy of each NC as a function of the 1X emission peak position. A clear correlation between the two values is evident. This can be interpreted as the effect stronger charge-carrier confinement has on both the 1X emission peak (stronger confinement is correlated with higher energy emission peak) and the binding energy (stronger confinement is correlated with stronger interaction of the two excitons). The trend and magnitude are in reasonable agreement with theoretical predictions recently made by Nugyen \textit{et al.}\cite{Nguyen2020}, and bounds suggested by Shulenberger \textit{et al.}\cite{Shulenberger2019}.

The suggested interpretation is further corroborated in \autoref{fig:br}c. Here the BX binding energy is plotted as a function of $g^{(2)}(0)$, another value indicative of charge-carrier confinement. Namely, tighter confinement increases the rate of Auger processes\cite{Melnychuk2021}, and consequently reduces the yield of the competing radiative BX decay process, evident in lower $g^{(2)}(0)$. Therefore, the inverse correlation of $\varepsilon_b$ with $g^{(2)}(0)$ evident in \autoref{fig:br}c, can be seen as pointing to the same underlying correlation of the BX binding energy with charge-carrier confinement.

\subsection*{\ce{CsPbI3 NCs}.}
\ce{CsPbI3} NCs BX binding energies were measured by the same technique (\autoref{fig:iod}). Results feature $\varepsilon_b$ values distributed around zero ($\varepsilon_b=1\pm\SI{9}{meV}$, $\ev{N}\!\sim\!0.3$, mean single-particle error $\pm\SI{4.8}{meV}$). The trends observed for \ce{CsPbBr3} are visible here as well, where higher 1X emission peak energy and lower $g^{(2)}(0)$, or stronger confinement, are correlated with stronger attractive interaction (\autoref{fig:iod}b-c). As a result, while $\varepsilon_b$ values are mostly within reasonable error from zero, NCs featuring 1X emission peak lower (higher) than \SI{1.845}{eV} or $g^{(2)}(0)$ higher (lower) than 0.25 typically display a small negative (positive) $\varepsilon_b$ value. While the results are not as conclusive as for \ce{CsPbBr3}, they suggest that the weak exciton-exciton interaction in \ce{CsPbI3} NCs can be either repulsive or attractive, depending on the exact parameters of the single-particle.

It is noteworthy that \ce{CsPbI3} measurements were significantly more challenging than their \ce{CsPbBr3} counterparts. This is due to two factors. First, \ce{CsPbI3} NCs synthesized were typically less emissive and less stable under the conditions of our measurements. That resulted in many NCs deteriorating during the measurement (PL declines to near zero), before enough photon pairs were detected to extract reliable spectra fits. Second, current SPAD array technology is less sensitive at these longer wavelengths\cite{Antolovic2018}. The SPAD array detector used in this work has roughly twice the photon detection efficiency at the \ce{CsPbBr3} emission peak compared to the \ce{CsPbI3} emission peak. These two factors resulted in the smaller statistics and larger errors for \ce{CsPbI3} NCs in this work.

\subsection*{Discussion.}
The BX binding energies presented in this paper are at the lower range of values previously reported in the literature for these NCs (see a table of previously reported values in \supp{6}). While in some cases this might be attributed to the potential pitfalls associated with ensemble measurements discussed in the Introduction, it is also important to consider the possibility that heralded spectroscopy and ensemble measurements probe the NC in a qualitatively different excitation state.
For example, one widely adopted ensemble technique for estimating BX binding energy, involves recording the transient absorption (TA) spectrum of a probe pulse at very short ($<\SI{1}{ps}$) delays from a pump pulse i.e.\ before the relaxation of hot carriers to the band edge\cite{Klimov2007,Makarov2016,Aneesh2017}. The hot carrier pair generated by the pump shifts the spectral position of the absorption resonance for the probe photon, and this shift is recorded as the exciton-exciton interaction energy. In contrast, results presented in this paper rely on measurements of photon pairs emitted from individual $BX{\rightarrow}1X{\rightarrow}G$ cascades following the excitation pulse. Since the PL decay lifetimes of the BX and 1X states are significantly longer than the timescales of hot carrier relaxation in the NCs (see \supp{4}), our measurements probe the NC only after the hot carrier pairs have relaxed to the band edge.

Since the wavefunction of the hot exciton differs from that of a band-edge 1X state, the interaction energies may be different in the two cases. Studies on PbS nanocrystals indicate that the magnitude of interaction between a hot exciton and a band-edge exciton is larger than between two band-edge excitons \cite{Geiregat2014}. For \ce{CsPbI3} NCs, a recent study indicates that the estimated $\varepsilon_b$ increases as the pump wavelength decreases in short-delay TA experiments\cite{Yumoto2018}. Similar trends have been demonstrated for \ce{CsPbBr3} at cryogenic temperatures using two dimensional electron spectroscopy\cite{Huang2020}. In addition, analyses of TA measurements by Ashner \textit{et al.}, that do not employ short-delay spectra, did not result in large positive values (but rather in small negative values of few meV)\cite{Ashner2019}. Together, these observations suggest that $\varepsilon_b$ measured when both excitons are at the band edge would be lower than that measured when the first exciton is still hot. In this sense, heralded spectroscopy and short-delay TA are complementary measurements of band-edge and hot exciton-exciton interaction, respectively, and a careful comparison of the two can help uncover new insights into dynamics of exciton interactions in NCs. 

Negative $\varepsilon_b$ values, observed only for \ce{CsPbI3} NCs in this work, are less often reported in the literature for similar NCs (see \supp{6}). The origin of this repulsive interaction in not immediately apparent from existing theoretical models of intrinsic homogeneous semiconductor NCs\cite{Nguyen2020,Hu1990}. One possible explanation is a modification of the charge-carrier wavefunctions induced by surface ligands\cite{Baker2010}. This can result in a type-II potential landscape, where the electrons or the holes are localized at the NC surface, and Coulomb repulsion might dominate the exciton-exciton interaction. Alternatively, the electrostatic field generated by charge carriers trapped in the ligand-induced trap states can result in charge-separation, and a similar repulsive interaction. Another possibility, suggested by Ashner \textit{et al.}\cite{Ashner2019}, is the formation of polarons, supported by the deformable nature of the perovskite lattice. The results presented in this work cannot pinpoint a certain mechanism, and present limited statistics of negative $\varepsilon_b$ values. However, the apparent observation of a repulsive exciton-exciton interaction in a homogeneous nanocrystal highlights the importance of further investigating the effect of surface chemistry, environment and perovskite lattice on charge-carrier interaction in LHP NCs.

\section*{Conclusions}
Heralded spectroscopy enables us to unambiguously determine the biexciton binding energy ($\varepsilon_b$) of single lead halide perovskite nanocrystals. Using this technique, we demonstrate that \SI{{\sim}6}{nm} edge \ce{CsPbBr3} nanocrystals feature an attractive exciton-exciton interaction of $\varepsilon_b=10\pm\SI{6}{meV}$, which lies at the lower range of previously reported values. Interestingly, within the ensemble of \SI{{\sim}7}{nm} edge  \ce{CsPbI3} nanocrystals, some exhibit weak attractive interactions whereas in others weak exciton-exciton repulsion is observed. This rarely observed phenomenon in homogeneous nanocrystals warrants further investigation of charge-carrier interactions in these particles. In nanocrystals of both materials, the strength of attractive interaction exhibits a clear correlation with single-exciton emission peak position and photoluminescence anti-bunching ($g^{(2)}(0)$), highlighting the dependence of $\varepsilon_b$ on charge-carrier confinement. These insights into the physics of exciton interactions in lead halide perovskite nanocrystals can enable the developement of better engineered nanocrystals for future optoelectronic technologies. Moreover, the unprecedented ability to determine biexciton binding energy of single nanocrystals at room temperature is instrumental to their utilization in quantum technologies.

% -----------------------------------------------------------------------------
% ------------------------------- end matter ----------------------------------
% -----------------------------------------------------------------------------

\subsection*{Supporting Information}
Nanocrystal synthesis protocol; details of supporting analyses $\ev{N}$, $g^{(2)}(0)$ and $\Hat{g}^{(2)}(0)$; photoluminecence decay lifetime estimation; system parameters; list of previously reported biexciton binding energies.

\begin{acknowledgement}
The authors wish to thank Ron Tenne for his part in designing the SPAD array spectrometer. Financial support by the Crown center of Photonics and by the Minerva Foundation is gratefully acknowledged. DO is the incumbent of the Harry Weinrebe professorial chair of laser physics.
\end{acknowledgement}

\bibliography{main.bib}

\end{document}
