%%
%% This is file `sample-sigconf.tex',
%% generated with the docstrip utility.
%%
%% The original source files were:
%%
%% samples.dtx  (with options: `sigconf')
%% 
%% IMPORTANT NOTICE:
%% 
%% For the copyright see the source file.
%% 
%% Any modified versions of this file must be renamed
%% with new filenames distinct from sample-sigconf.tex.
%% 
%% For distribution of the original source see the terms
%% for copying and modification in the file samples.dtx.
%% 
%% This generated file may be distributed as long as the
%% original source files, as listed above, are part of the
%% same distribution. (The sources need not necessarily be
%% in the same archive or directory.)
%%
%%
%% Commands for TeXCount
%TC:macro \cite [option:text,text]
%TC:macro \citep [option:text,text]
%TC:macro \citet [option:text,text]
%TC:envir table 0 1
%TC:envir table* 0 1
%TC:envir tabular [ignore] word
%TC:envir displaymath 0 word
%TC:envir math 0 word
%TC:envir comment 0 0
%%
%%
%% The first command in your LaTeX source must be the \documentclass command.
\documentclass[acmsmall]{acmart}

%%
%% \BibTeX command to typeset BibTeX logo in the docs
\AtBeginDocument{%
  \providecommand\BibTeX{{%
      Bib\TeX}}}

%% Rights management information.  This information is sent to you
%% when you complete the rights form.  These commands have SAMPLE
%% values in them; it is your responsibility as an author to replace
%% the commands and values with those provided to you when you
%% complete the rights form.
%\copyrightyear{2023}

\setcopyright{acmlicensed}
\acmJournal{PACMMOD}
\acmYear{2023} \acmVolume{1} \acmNumber{1} \acmArticle{22} \acmMonth{5} \acmPrice{15.00}\acmDOI{10.1145/3588702}

%% These commands are for a PROCEEDINGS abstract or paper.
% \acmConference[ACM SIGMOD '23]{Proceedings of the '23 International Conference on the Management of Data}{June 18--23, 2023}{Seattle, WA}
%\acmISBN{978-1-4503-XXXX-X/18/06}


%%
%% Submission ID.
%% Use this when submitting an article to a sponsored event. You'll
%% receive a unique submission ID from the organizers
%% of the event, and this ID should be used as the parameter to this command.
%%\acmSubmissionID{123-A56-BU3}

%%
%% For managing citations, it is recommended to use bibliography
%% files in BibTeX format.
%%
%% You can then either use BibTeX with the ACM-Reference-Format style,
%% or BibLaTeX with the acmnumeric or acmauthoryear sytles, that include
%% support for advanced citation of software artefact from the
%% biblatex-software package, also separately available on CTAN.
%%
%% Look at the sample-*-biblatex.tex files for templates showcasing
%% the biblatex styles.
%%

%%
%% The majority of ACM publications use numbered citations and
%% references.  The command \citestyle{authoryear} switches to the
%% "author year" style.
%%
%% If you are preparing content for an event
%% sponsored by ACM SIGGRAPH, you must use the "author year" style of
%% citations and references.
%% Uncommenting
%% the next command will enable that style.
%%\citestyle{acmauthoryear}

%%%%%%%%%%%%%%%

%\usepackage[OT1,T1]{fontenc}

\usepackage[numbers,sort&compress]{natbib}
\renewcommand{\bibfont}{\footnotesize}
%\usepackage{cite}
%\usepackage{mystyle}
%%%%%%%%%%%%%%%%%%%%%%%%%%%%%%%%%%%%
\makeatletter

\usepackage{etex}

%%% Review %%%

\usepackage{zref-savepos}

\newcounter{mnote}%[page]
\renewcommand{\themnote}{p.\thepage\;$\langle$\arabic{mnote}$\rangle$}

\def\xmarginnote{%
  \xymarginnote{\hskip -\marginparsep \hskip -\marginparwidth}}

\def\ymarginnote{%
  \xymarginnote{\hskip\columnwidth \hskip\marginparsep}}

\long\def\xymarginnote#1#2{%
\vadjust{#1%
\smash{\hbox{{%
        \hsize\marginparwidth
        \@parboxrestore
        \@marginparreset
\footnotesize #2}}}}}

\def\mnoteson{%
\gdef\mnote##1{\refstepcounter{mnote}\label{##1}%
  \zsavepos{##1}%
  \ifnum20432158>\number\zposx{##1}%
  \xmarginnote{{\color{blue}\bf $\langle$\arabic{mnote}$\rangle$}}% 
  \else
  \ymarginnote{{\color{blue}\bf $\langle$\arabic{mnote}$\rangle$}}%
  \fi%
}
  }
\gdef\mnotesoff{\gdef\mnote##1{}}
\mnoteson
\mnotesoff








%%% Layout %%%

% \usepackage{geometry} % override layout
% \geometry{tmargin=2.5cm,bmargin=m2.5cm,lmargin=3cm,rmargin=3cm}
% \setlength{\pdfpagewidth}{8.5in} % overrides default pdftex paper size
% \setlength{\pdfpageheight}{11in}

\newlength{\mywidth}

%%% Conventions %%%

% References
\newcommand{\figref}[1]{Fig.~\ref{#1}}
\newcommand{\defref}[1]{Definition~\ref{#1}}
\newcommand{\tabref}[1]{Table~\ref{#1}}
% general
%\usepackage{ifthen,nonfloat,subfigure,rotating,array,framed}
\usepackage{framed}
%\usepackage{subfigure}
\usepackage{subcaption}
\usepackage{comment}
%\specialcomment{nb}{\begingroup \noindent \framed\textbf{n.b.\ }}{\endframed\endgroup}
%%\usepackage{xtab,arydshln,multirow}
% topcaption defined in xtab. must load nonfloat before xtab
%\PassOptionsToPackage{svgnames,dvipsnames}{xcolor}
\usepackage[svgnames,dvipsnames]{xcolor}
%\definecolor{myblue}{rgb}{.8,.8,1}
%\definecolor{umbra}{rgb}{.8,.8,.5}
%\newcommand*\mybluebox[1]{%
%  \colorbox{myblue}{\hspace{1em}#1\hspace{1em}}}
\usepackage[all]{xy}
%\usepackage{pstricks,pst-node}
\usepackage{tikz}
\usetikzlibrary{positioning,matrix,through,calc,arrows,fit,shapes,decorations.pathreplacing,decorations.markings,decorations.text}

\tikzstyle{block} = [draw,fill=blue!20,minimum size=2em]

% allow prefix to scope name
\tikzset{%
	prefix node name/.code={%
		\tikzset{%
			name/.code={\edef\tikz@fig@name{#1 ##1}}
		}%
	}%
}


\@ifpackagelater{tikz}{2013/12/01}{
	\newcommand{\convexpath}[2]{
		[create hullcoords/.code={
			\global\edef\namelist{#1}
			\foreach [count=\counter] \nodename in \namelist {
				\global\edef\numberofnodes{\counter}
				\coordinate (hullcoord\counter) at (\nodename);
			}
			\coordinate (hullcoord0) at (hullcoord\numberofnodes);
			\pgfmathtruncatemacro\lastnumber{\numberofnodes+1}
			\coordinate (hullcoord\lastnumber) at (hullcoord1);
		}, create hullcoords ]
		($(hullcoord1)!#2!-90:(hullcoord0)$)
		\foreach [evaluate=\currentnode as \previousnode using \currentnode-1,
		evaluate=\currentnode as \nextnode using \currentnode+1] \currentnode in {1,...,\numberofnodes} {
			let \p1 = ($(hullcoord\currentnode) - (hullcoord\previousnode)$),
			\n1 = {atan2(\y1,\x1) + 90},
			\p2 = ($(hullcoord\nextnode) - (hullcoord\currentnode)$),
			\n2 = {atan2(\y2,\x2) + 90},
			\n{delta} = {Mod(\n2-\n1,360) - 360}
			in 
			{arc [start angle=\n1, delta angle=\n{delta}, radius=#2]}
			-- ($(hullcoord\nextnode)!#2!-90:(hullcoord\currentnode)$) 
		}
	}
}{
	\newcommand{\convexpath}[2]{
		[create hullcoords/.code={
			\global\edef\namelist{#1}
			\foreach [count=\counter] \nodename in \namelist {
				\global\edef\numberofnodes{\counter}
				\coordinate (hullcoord\counter) at (\nodename);
			}
			\coordinate (hullcoord0) at (hullcoord\numberofnodes);
			\pgfmathtruncatemacro\lastnumber{\numberofnodes+1}
			\coordinate (hullcoord\lastnumber) at (hullcoord1);
		}, create hullcoords ]
		($(hullcoord1)!#2!-90:(hullcoord0)$)
		\foreach [evaluate=\currentnode as \previousnode using \currentnode-1,
		evaluate=\currentnode as \nextnode using \currentnode+1] \currentnode in {1,...,\numberofnodes} {
			let \p1 = ($(hullcoord\currentnode) - (hullcoord\previousnode)$),
			\n1 = {atan2(\x1,\y1) + 90},
			\p2 = ($(hullcoord\nextnode) - (hullcoord\currentnode)$),
			\n2 = {atan2(\x2,\y2) + 90},
			\n{delta} = {Mod(\n2-\n1,360) - 360}
			in 
			{arc [start angle=\n1, delta angle=\n{delta}, radius=#2]}
			-- ($(hullcoord\nextnode)!#2!-90:(hullcoord\currentnode)$) 
		}
	}
}

% circle around nodes

% typsetting math
\usepackage{qsymbols,amssymb,mathrsfs}
\usepackage{amsmath}
\usepackage[standard,thmmarks]{ntheorem}
\theoremstyle{plain}
\theoremsymbol{\ensuremath{_\vartriangleleft}}
\theorembodyfont{\itshape}
\theoremheaderfont{\normalfont\bfseries}
\theoremseparator{}
\newtheorem{Claim}{Claim}
\newtheorem{Subclaim}{Subclaim}
\theoremstyle{nonumberplain}
\theoremheaderfont{\scshape}
\theorembodyfont{\normalfont}
\theoremsymbol{\ensuremath{_\blacktriangleleft}}
\newtheorem{Subproof}{Proof}

\theoremnumbering{arabic}
\theoremstyle{plain}
\usepackage{latexsym}
\theoremsymbol{\ensuremath{_\Box}}
\theorembodyfont{\itshape}
\theoremheaderfont{\normalfont\bfseries}
\theoremseparator{}
\newtheorem{Conjecture}{Conjecture}

\theorembodyfont{\upshape}
\theoremprework{\bigskip\hrule}
\theorempostwork{\hrule\bigskip}
\newtheorem{Condition}{Condition}%[section]


%\RequirePckage{amsmath} loaded by empheq
\usepackage[overload]{empheq} % no \intertext and \displaybreak
%\usepackage{breqn}

\let\iftwocolumn\if@twocolumn
\g@addto@macro\@twocolumntrue{\let\iftwocolumn\if@twocolumn}
\g@addto@macro\@twocolumnfalse{\let\iftwocolumn\if@twocolumn}

%\empheqset{box=\mybluebox}
%\usepackage{mathtools}      % to polish math typsetting, loaded
%                                % by empeq
\mathtoolsset{showonlyrefs=false,showmanualtags}
\let\underbrace\LaTeXunderbrace % adapt spacing to font sizes
\let\overbrace\LaTeXoverbrace
\renewcommand{\eqref}[1]{\textup{(\refeq{#1})}} % eqref was not allowed in
                                       % sub/super-scripts
\newtagform{brackets}[]{(}{)}   % new tags for equations
\usetagform{brackets}
% defined commands:
% \shortintertext{}, dcases*, \cramped, \smashoperator[]{}

\usepackage[Smaller]{cancel}
\renewcommand{\CancelColor}{\color{Red}}
%\newcommand\hcancel[2][black]{\setbox0=\hbox{#2}% colored horizontal cross
%  \rlap{\raisebox{.45\ht0}{\color{#1}\rule{\wd0}{1pt}}}#2}



\usepackage{graphicx,psfrag}
\graphicspath{{figure/}{image/}} % Search path of figures

% for tabular
\usepackage{diagbox} % \backslashbox{}{} for slashed entries
%\usepackage{threeparttable} % threeparttable, \tnote{},
                                % tablenotes, and \item[]
%\usepackage{colortab} % \cellcolor[gray]{0.9},
%\rowcolor,\columncolor,
%\usepackage{colortab} % \LCC \gray & ...  \ECC \\

% typesetting codes
%\usepackage{maple2e} % need to use \char29 for ^
\usepackage{algorithm2e}
\usepackage{listings} 
\lstdefinelanguage{Maple}{
  morekeywords={proc,module,end, for,from,to,by,while,in,do,od
    ,if,elif,else,then,fi ,use,try,catch,finally}, sensitive,
  morecomment=[l]\#,
  morestring=[b]",morestring=[b]`}[keywords,comments,strings]
\lstset{
  basicstyle=\scriptsize,
  keywordstyle=\color{ForestGreen}\bfseries,
  commentstyle=\color{DarkBlue},
  stringstyle=\color{DimGray}\ttfamily,
  texcl
}
%%% New fonts %%%
\DeclareMathAlphabet{\mathpzc}{OT1}{pzc}{m}{it}
\usepackage{upgreek} % \upalpha,\upbeta, ...
%\usepackage{bbold}   % blackboard math
\usepackage{dsfont}  % \mathds

%%% Macros for multiple definitions %%%

% example usage:
% \multi{M}{\boldsymbol{#1}}  % defines \multiM
% \multi ABC.                 % defines \MA \MB and \MC as
%                             % \boldsymbol{A}, \boldsymbol{B} and
%                             % \boldsymbol{C} respectively.
% 
%  The last period '.' is necessary to terminate the macro expansion.
%
% \multi*{M}{\boldsymbol{#1}} % defines \multiM and \M
% \M{A}                       % expands to \boldsymbol{A}

\def\multi@nostar#1#2{%
  \expandafter\def\csname multi#1\endcsname##1{%
    \if ##1.\let\next=\relax \else
    \def\next{\csname multi#1\endcsname}     
    %\expandafter\def\csname #1##1\endcsname{#2}
    \expandafter\newcommand\csname #1##1\endcsname{#2}
    \fi\next}}

\def\multi@star#1#2{%
  \expandafter\def\csname #1\endcsname##1{#2}
  \multi@nostar{#1}{#2}
}

\newcommand{\multi}{%
  \@ifstar \multi@star \multi@nostar}

%%% new alphabets %%%

\multi*{rm}{\mathrm{#1}}
\multi*{mc}{\mathcal{#1}}
\multi*{op}{\mathop {\operator@font #1}}
% \multi*{op}{\operatorname{#1}}
\multi*{ds}{\mathds{#1}}
\multi*{set}{\mathcal{#1}}
\multi*{rsfs}{\mathscr{#1}}
\multi*{pz}{\mathpzc{#1}}
\multi*{M}{\boldsymbol{#1}}
\multi*{R}{\mathsf{#1}}
\multi*{RM}{\M{\R{#1}}}
\multi*{bb}{\mathbb{#1}}
\multi*{td}{\tilde{#1}}
\multi*{tR}{\tilde{\mathsf{#1}}}
\multi*{trM}{\tilde{\M{\R{#1}}}}
\multi*{tset}{\tilde{\mathcal{#1}}}
\multi*{tM}{\tilde{\M{#1}}}
\multi*{baM}{\bar{\M{#1}}}
\multi*{ol}{\overline{#1}}

\multirm  ABCDEFGHIJKLMNOPQRSTUVWXYZabcdefghijklmnopqrstuvwxyz.
\multiol  ABCDEFGHIJKLMNOPQRSTUVWXYZabcdefghijklmnopqrstuvwxyz.
\multitR   ABCDEFGHIJKLMNOPQRSTUVWXYZabcdefghijklmnopqrstuvwxyz.
\multitd   ABCDEFGHIJKLMNOPQRSTUVWXYZabcdefghijklmnopqrstuvwxyz.
\multitset ABCDEFGHIJKLMNOPQRSTUVWXYZabcdefghijklmnopqrstuvwxyz.
\multitM   ABCDEFGHIJKLMNOPQRSTUVWXYZabcdefghijklmnopqrstuvwxyz.
\multibaM   ABCDEFGHIJKLMNOPQRSTUVWXYZabcdefghijklmnopqrstuvwxyz.
\multitrM   ABCDEFGHIJKLMNOPQRSTUVWXYZabcdefghijklmnopqrstuvwxyz.
\multimc   ABCDEFGHIJKLMNOPQRSTUVWXYZabcdefghijklmnopqrstuvwxyz.
\multiop   ABCDEFGHIJKLMNOPQRSTUVWXYZabcdefghijklmnopqrstuvwxyz.
\multids   ABCDEFGHIJKLMNOPQRSTUVWXYZabcdefghijklmnopqrstuvwxyz.
\multiset  ABCDEFGHIJKLMNOPQRSTUVWXYZabcdefghijklmnopqrstuvwxyz.
\multirsfs ABCDEFGHIJKLMNOPQRSTUVWXYZabcdefghijklmnopqrstuvwxyz.
\multipz   ABCDEFGHIJKLMNOPQRSTUVWXYZabcdefghijklmnopqrstuvwxyz.
\multiM    ABCDEFGHIJKLMNOPQRSTUVWXYZabcdefghijklmnopqrstuvwxyz.
\multiR    ABCDEFGHIJKL NO QR TUVWXYZabcd fghijklmnopqrstuvwxyz.
\multibb   ABCDEFGHIJKLMNOPQRSTUVWXYZabcdefghijklmnopqrstuvwxyz.
\multiRM   ABCDEFGHIJKLMNOPQRSTUVWXYZabcdefghijklmnopqrstuvwxyz.
\newcommand{\RRM}{\R{M}}
\newcommand{\RRP}{\R{P}}
\newcommand{\RRe}{\R{e}}
\newcommand{\RRS}{\R{S}}
%%% new symbols %%%

%\newcommand{\dotgeq}{\buildrel \textstyle  .\over \geq}
%\newcommand{\dotleq}{\buildrel \textstyle  .\over \leq}
\newcommand{\dotleq}{\buildrel \textstyle  .\over {\smash{\lower
      .2ex\hbox{\ensuremath\leqslant}}\vphantom{=}}}
\newcommand{\dotgeq}{\buildrel \textstyle  .\over {\smash{\lower
      .2ex\hbox{\ensuremath\geqslant}}\vphantom{=}}}

\DeclareMathOperator*{\argmin}{arg\,min}
\DeclareMathOperator*{\argmax}{arg\,max}

%%% abbreviations %%%

% commands
\newcommand{\esm}{\ensuremath}

% environments
\newcommand{\bM}{\begin{bmatrix}}
\newcommand{\eM}{\end{bmatrix}}
\newcommand{\bSM}{\left[\begin{smallmatrix}}
\newcommand{\eSM}{\end{smallmatrix}\right]}
\renewcommand*\env@matrix[1][*\c@MaxMatrixCols c]{%
  \hskip -\arraycolsep
  \let\@ifnextchar\new@ifnextchar
  \array{#1}}



% sets of number
\newqsymbol{`N}{\mathbb{N}}
\newqsymbol{`R}{\mathbb{R}}
\newqsymbol{`P}{\mathbb{P}}
\newqsymbol{`Z}{\mathbb{Z}}

% symbol short cut
\newqsymbol{`|}{\mid}
% use \| for \parallel
\newqsymbol{`8}{\infty}
\newqsymbol{`1}{\left}
\newqsymbol{`2}{\right}
\newqsymbol{`6}{\partial}
\newqsymbol{`0}{\emptyset}
\newqsymbol{`-}{\leftrightarrow}
\newqsymbol{`<}{\langle}
\newqsymbol{`>}{\rangle}

%%% new operators / functions %%%

\newcommand{\sgn}{\operatorname{sgn}}
\newcommand{\Var}{\op{Var}}
\newcommand{\diag}{\operatorname{diag}}
\newcommand{\erf}{\operatorname{erf}}
\newcommand{\erfc}{\operatorname{erfc}}
\newcommand{\erfi}{\operatorname{erfi}}
\newcommand{\adj}{\operatorname{adj}}
\newcommand{\supp}{\operatorname{supp}}
\newcommand{\E}{\opE\nolimits}
\newcommand{\T}{\intercal}
% requires mathtools
% \abs,\abs*,\abs[<size_cmd:\big,\Big,\bigg,\Bigg etc.>]
\DeclarePairedDelimiter\abs{\lvert}{\rvert} 
\DeclarePairedDelimiter\norm{\lVert}{\rVert}
\DeclarePairedDelimiter\ceil{\lceil}{\rceil}
\DeclarePairedDelimiter\floor{\lfloor}{\rfloor}
\DeclarePairedDelimiter\Set{\{}{\}}
\newcommand{\imod}[1]{\allowbreak\mkern10mu({\operator@font mod}\,\,#1)}

%%% new formats %%%
\newcommand{\leftexp}[2]{{\vphantom{#2}}^{#1}{#2}}


% non-floating figures that can be put inside tables
\newenvironment{nffigure}[1][\relax]{\vskip \intextsep
  \noindent\minipage{\linewidth}\def\@captype{figure}}{\endminipage\vskip \intextsep}

\newcommand{\threecols}[3]{
\hbox to \textwidth{%
      \normalfont\rlap{\parbox[b]{\textwidth}{\raggedright#1\strut}}%
        \hss\parbox[b]{\textwidth}{\centering#2\strut}\hss
        \llap{\parbox[b]{\textwidth}{\raggedleft#3\strut}}%
    }% hbox 
}

\newcommand{\reason}[2][\relax]{
  \ifthenelse{\equal{#1}{\relax}}{
    \left(\text{#2}\right)
  }{
    \left(\parbox{#1}{\raggedright #2}\right)
  }
}

\newcommand{\marginlabel}[1]
{\mbox[]\marginpar{\color{ForestGreen} \sffamily \small \raggedright\hspace{0pt}#1}}


% up-tag an equation
\newcommand{\utag}[2]{\mathop{#2}\limits^{\text{(#1)}}}
\newcommand{\uref}[1]{(#1)}


% Notation table

\newcommand{\Hline}{\noalign{\vskip 0.1in \hrule height 0.1pt \vskip
    0.1in}}
  
\def\Malign#1{\tabskip=0in
  \halign to\columnwidth{
    \ensuremath{\displaystyle ##}\hfil
    \tabskip=0in plus 1 fil minus 1 fil
    &
    \parbox[t]{0.8\columnwidth}{##}
    \tabskip=0in
    \cr #1}}


%%%%%%%%%%%%%%%%%%%%%%%%%%%%%%%%%%%%%%%%%%%%%%%%%%%%%%%%%%%%%%%%%%%
% MISCELLANEOUS

% Modification from braket.sty by Donald Arseneau
% Command defined is: \extendvert{ }
% The "small versions" use fixed-size brackets independent of their
% contents, whereas the expand the first vertical line '|' or '\|' to
% envelop the content
\let\SavedDoubleVert\relax
\let\protect\relax
{\catcode`\|=\active
  \xdef\extendvert{\protect\expandafter\noexpand\csname extendvert \endcsname}
  \expandafter\gdef\csname extendvert \endcsname#1{\mskip-5mu \left.%
      \ifx\SavedDoubleVert\relax \let\SavedDoubleVert\|\fi
     \:{\let\|\SetDoubleVert
       \mathcode`\|32768\let|\SetVert
     #1}\:\right.\mskip-5mu}
}
\def\SetVert{\@ifnextchar|{\|\@gobble}% turn || into \|
    {\egroup\;\mid@vertical\;\bgroup}}
\def\SetDoubleVert{\egroup\;\mid@dblvertical\;\bgroup}

% If the user is using e-TeX with its \middle primitive, use that for
% verticals instead of \vrule.
%
\begingroup
 \edef\@tempa{\meaning\middle}
 \edef\@tempb{\string\middle}
\expandafter \endgroup \ifx\@tempa\@tempb
 \def\mid@vertical{\middle|}
 \def\mid@dblvertical{\middle\SavedDoubleVert}
\else
 \def\mid@vertical{\mskip1mu\vrule\mskip1mu}
 \def\mid@dblvertical{\mskip1mu\vrule\mskip2.5mu\vrule\mskip1mu}
\fi

%%%%%%%%%%%%%%%%%%%%%%%%%%%%%%%%%%%%%%%%%%%%%%%%%%%%%%%%%%%%%%%%

\makeatother

%%%%%%%%%%%%%%%%%%%%%%%%%%%%%%%%%%%%

\usepackage{ctable}
\usepackage{fouridx}
%\usepackage{calc}
\usepackage{framed}
\usetikzlibrary{positioning,matrix}

\usepackage{paralist}
%\usepackage{refcheck}
\usepackage{enumerate}

\usepackage[normalem]{ulem}
\newcommand{\Ans}[1]{\uuline{\raisebox{.15em}{#1}}}



\numberwithin{equation}{section}
\makeatletter
\@addtoreset{equation}{section}
\renewcommand{\theequation}{\arabic{section}.\arabic{equation}}
\@addtoreset{Theorem}{section}
\renewcommand{\theTheorem}{\arabic{section}.\arabic{Theorem}}
\@addtoreset{Lemma}{section}
\renewcommand{\theLemma}{\arabic{section}.\arabic{Lemma}}
\@addtoreset{Corollary}{section}
\renewcommand{\theCorollary}{\arabic{section}.\arabic{Corollary}}
\@addtoreset{Example}{section}
\renewcommand{\theExample}{\arabic{section}.\arabic{Example}}
\@addtoreset{Remark}{section}
\renewcommand{\theRemark}{\arabic{section}.\arabic{Remark}}
\@addtoreset{Proposition}{section}
\renewcommand{\theProposition}{\arabic{section}.\arabic{Proposition}}
\@addtoreset{Definition}{section}
\renewcommand{\theDefinition}{\arabic{section}.\arabic{Definition}}
\@addtoreset{Claim}{section}
\renewcommand{\theClaim}{\arabic{section}.\arabic{Claim}}
\@addtoreset{Subclaim}{Theorem}
\renewcommand{\theSubclaim}{\theTheorem\Alph{Subclaim}}
\makeatother

\newcommand{\Null}{\op{Null}}
%\newcommand{\T}{\op{T}\nolimits}
\newcommand{\Bern}{\op{Bern}\nolimits}
\newcommand{\odd}{\op{odd}}
\newcommand{\even}{\op{even}}
\newcommand{\Sym}{\op{Sym}}
\newcommand{\si}{s_{\op{div}}}
\newcommand{\sv}{s_{\op{var}}}
\newcommand{\Wtyp}{W_{\op{typ}}}
\newcommand{\Rco}{R_{\op{CO}}}
\newcommand{\Tm}{\op{T}\nolimits}
\newcommand{\JGK}{J_{\op{GK}}}

\newcommand{\diff}{\mathrm{d}}

\newenvironment{lbox}{
  \setlength{\FrameSep}{1.5mm}
  \setlength{\FrameRule}{0mm}
  \def\FrameCommand{\fboxsep=\FrameSep \fcolorbox{black!20}{white}}%
  \MakeFramed {\FrameRestore}}%
{\endMakeFramed}

\newenvironment{ybox}{
	\setlength{\FrameSep}{1.5mm}
	\setlength{\FrameRule}{0mm}
  \def\FrameCommand{\fboxsep=\FrameSep \fcolorbox{black!10}{yellow!8}}%
  \MakeFramed {\FrameRestore}}%
{\endMakeFramed}

\newenvironment{gbox}{
	\setlength{\FrameSep}{1.5mm}
\setlength{\FrameRule}{0mm}
  \def\FrameCommand{\fboxsep=\FrameSep \fcolorbox{black!10}{green!8}}%
  \MakeFramed {\FrameRestore}}%
{\endMakeFramed}

\newenvironment{bbox}{
	\setlength{\FrameSep}{1.5mm}
\setlength{\FrameRule}{0mm}
  \def\FrameCommand{\fboxsep=\FrameSep \fcolorbox{black!10}{blue!8}}%
  \MakeFramed {\FrameRestore}}%
{\endMakeFramed}

\newenvironment{yleftbar}{%
  \def\FrameCommand{{\color{yellow!20}\vrule width 3pt} \hspace{10pt}}%
  \MakeFramed {\advance\hsize-\width \FrameRestore}}%
 {\endMakeFramed}

\newcommand{\tbox}[2][\relax]{
 \setlength{\FrameSep}{1.5mm}
  \setlength{\FrameRule}{0mm}
  \begin{ybox}
    \noindent\underline{#1:}\newline
    #2
  \end{ybox}
}

\newcommand{\pbox}[2][\relax]{
  \setlength{\FrameSep}{1.5mm}
 \setlength{\FrameRule}{0mm}
  \begin{gbox}
    \noindent\underline{#1:}\newline
    #2
  \end{gbox}
}

\newcommand{\gtag}[1]{\text{\color{green!50!black!60} #1}}
\let\labelindent\relax
\usepackage{enumitem}

%%%%%%%%%%%%%%%%%%%%%%%%%%%%%%%%%%%%
% fix subequations
% http://tex.stackexchange.com/questions/80134/nesting-subequations-within-align
%%%%%%%%%%%%%%%%%%%%%%%%%%%%%%%%%%%%

\usepackage{etoolbox}

% let \theparentequation use the same definition as equation
\let\theparentequation\theequation
% change every occurence of "equation" to "parentequation"
\patchcmd{\theparentequation}{equation}{parentequation}{}{}

\renewenvironment{subequations}[1][]{%              optional argument: label-name for (first) parent equation
	\refstepcounter{equation}%
	%  \def\theparentequation{\arabic{parentequation}}% we patched it already :)
	\setcounter{parentequation}{\value{equation}}%    parentequation = equation
	\setcounter{equation}{0}%                         (sub)equation  = 0
	\def\theequation{\theparentequation\alph{equation}}% 
	\let\parentlabel\label%                           Evade sanitation performed by amsmath
	\ifx\\#1\\\relax\else\label{#1}\fi%               #1 given: \label{#1}, otherwise: nothing
	\ignorespaces
}{%
	\setcounter{equation}{\value{parentequation}}%    equation = subequation
	\ignorespacesafterend
}

\newcommand*{\nextParentEquation}[1][]{%            optional argument: label-name for (first) parent equation
	\refstepcounter{parentequation}%                  parentequation++
	\setcounter{equation}{0}%                         equation = 0
	\ifx\\#1\\\relax\else\parentlabel{#1}\fi%         #1 given: \label{#1}, otherwise: nothing
}

% hyperlink
\PassOptionsToPackage{breaklinks,letterpaper,hyperindex=true,backref=false,bookmarksnumbered,bookmarksopen,linktocpage,colorlinks,linkcolor=BrickRed,citecolor=OliveGreen,urlcolor=Blue,pdfstartview=FitH}{hyperref}
\usepackage{hyperref}

% makeindex style
\newcommand{\indexmain}[1]{\textbf{\hyperpage{#1}}}

\newcommand{\haoyue}[1]{}
\newcommand{\julia}[1]{\textcolor{red}{(Julia) {#1}}}
\newcommand{\rev}[1]{{#1}}
\newcommand{\revv}[1]{{#1}}
\newcommand{\revvv}[1]{{#1}}

%%
%% end of the preamble, start of the body of the document source.
\begin{document}
  
%%
%% The "title" command has an optional parameter,
%% allowing the author to define a "short title" to be used in page headers.
\title{\mEw: An Interpretation of Winners over Uncertain Voter Preferences}

%%
%% The "author" command and its associated commands are used to define
%% the authors and their affiliations.
%% Of note is the shared affiliation of the first two authors, and the
%% "authornote" and "authornotemark" commands
%% used to denote shared contribution to the research.
\author{Haoyue Ping}
\affiliation{%
  \institution{New York University}
 \country{New York, NY USA}
 }
 \email{hp1326@nyu.edu}
 \orcid{0000-0002-2694-3301}

\author{Julia Stoyanovich}\authornote{This research was supported in part by NSF awards No. 1916647, 1934464, and 1916505.}
 \affiliation{%
   \institution{New York University}
  \country{New York, NY USA}
 }
 \email{stoyanovich@nyu.edu}
 \orcid{0000-0002-1587-0450}

%%
%% By default, the full list of authors will be used in the page
%% headers. Often, this list is too long, and will overlap
%% other information printed in the page headers. This command allows
%% the author to define a more concise list
%% of authors' names for this purpose.
% \renewcommand{\shortauthors}{Ping and Stoyanovich}

%%
%% The abstract is a short summary of the work to be presented in the
%% article.
  In this paper, we explore the connection between secret key agreement and secure omniscience within the setting of the multiterminal source model with a wiretapper who has side information. While the secret key agreement problem considers the generation of a maximum-rate secret key through public discussion, the secure omniscience problem is concerned with communication protocols for omniscience that minimize the rate of information leakage to the wiretapper. The starting point of our work is a lower bound on the minimum leakage rate for omniscience, $\rl$, in terms of the wiretap secret key capacity, $\wskc$. Our interest is in identifying broad classes of sources for which this lower bound is met with equality, in which case we say that there is a duality between secure omniscience and secret key agreement. We show that this duality holds in the case of certain finite linear source (FLS) models, such as two-terminal FLS models and pairwise independent network models on trees with a linear wiretapper. Duality also holds for any FLS model in which $\wskc$ is achieved by a perfect linear secret key agreement scheme. We conjecture that the duality in fact holds unconditionally for any FLS model. On the negative side, we give an example of a (non-FLS) source model for which duality does not hold if we limit ourselves to communication-for-omniscience protocols with at most two (interactive) communications.  We also address the secure function computation problem and explore the connection between the minimum leakage rate for computing a function and the wiretap secret key capacity.
  
%   Finally, we demonstrate the usefulness of our lower bound on $\rl$ by using it to derive equivalent conditions for the positivity of $\wskc$ in the multiterminal model. This extends a recent result of Gohari, G\"{u}nl\"{u} and Kramer (2020) obtained for the two-user setting.
  
   
%   In this paper, we study the problem of secret key generation through an omniscience achieving communication that minimizes the 
%   leakage rate to a wiretapper who has side information in the setting of multiterminal source model.  We explore this problem by deriving a lower bound on the wiretap secret key capacity $\wskc$ in terms of the minimum leakage rate for omniscience, $\rl$. 
%   %The former quantity is defined to be the maximum secret key rate achievable, and the latter one is defined as the minimum possible leakage rate about the source through an omniscience scheme to a wiretapper. 
%   The main focus of our work is the characterization of the sources for which the lower bound holds with equality \textemdash it is referred to as a duality between secure omniscience and wiretap secret key agreement. For general source models, we show that duality need not hold if we limit to the communication protocols with at most two (interactive) communications. In the case when there is no restriction on the number of communications, whether the duality holds or not is still unknown. However, we resolve this question affirmatively for two-user finite linear sources (FLS) and pairwise independent networks (PIN) defined on trees, a subclass of FLS. Moreover, for these sources, we give a single-letter expression for $\wskc$. Furthermore, in the direction of proving the conjecture that duality holds for all FLS, we show that if $\wskc$ is achieved by a \emph{perfect} secret key agreement scheme for FLS then the duality must hold. All these results mount up the evidence in favor of the conjecture on FLS. Moreover, we demonstrate the usefulness of our lower bound on $\wskc$ in terms of $\rl$ by deriving some equivalent conditions on the positivity of secret key capacity for multiterminal source model. Our result indeed extends the work of Gohari, G\"{u}nl\"{u} and Kramer in two-user case.

%%
%% The code below is generated by the tool at http://dl.acm.org/ccs.cfm.
%% Please copy and paste the code instead of the example below.
%%
\begin{CCSXML}
<ccs2012>
   <concept>
       <concept_id>10003752.10010070.10010111.10011736</concept_id>
       <concept_desc>Theory of computation~Incomplete, inconsistent, and uncertain databases</concept_desc>
       <concept_significance>500</concept_significance>
       </concept>
   <concept>
       <concept_id>10010405.10010476.10010936.10003590</concept_id>
       <concept_desc>Applied computing~Voting / election technologies</concept_desc>
       <concept_significance>500</concept_significance>
       </concept>
 </ccs2012>
\end{CCSXML}

\ccsdesc[500]{Theory of computation~Incomplete, inconsistent, and uncertain databases}
\ccsdesc[500]{Applied computing~Voting / election technologies}

\keywords{computational social choice; voting and elections; positional scoring rules; uncertain preferences}


\maketitle

% !TEX root = ../arxiv.tex

Unsupervised domain adaptation (UDA) is a variant of semi-supervised learning \cite{blum1998combining}, where the available unlabelled data comes from a different distribution than the annotated dataset \cite{Ben-DavidBCP06}.
A case in point is to exploit synthetic data, where annotation is more accessible compared to the costly labelling of real-world images \cite{RichterVRK16,RosSMVL16}.
Along with some success in addressing UDA for semantic segmentation \cite{TsaiHSS0C18,VuJBCP19,0001S20,ZouYKW18}, the developed methods are growing increasingly sophisticated and often combine style transfer networks, adversarial training or network ensembles \cite{KimB20a,LiYV19,TsaiSSC19,Yang_2020_ECCV}.
This increase in model complexity impedes reproducibility, potentially slowing further progress.

In this work, we propose a UDA framework reaching state-of-the-art segmentation accuracy (measured by the Intersection-over-Union, IoU) without incurring substantial training efforts.
Toward this goal, we adopt a simple semi-supervised approach, \emph{self-training} \cite{ChenWB11,lee2013pseudo,ZouYKW18}, used in recent works only in conjunction with adversarial training or network ensembles \cite{ChoiKK19,KimB20a,Mei_2020_ECCV,Wang_2020_ECCV,0001S20,Zheng_2020_IJCV,ZhengY20}.
By contrast, we use self-training \emph{standalone}.
Compared to previous self-training methods \cite{ChenLCCCZAS20,Li_2020_ECCV,subhani2020learning,ZouYKW18,ZouYLKW19}, our approach also sidesteps the inconvenience of multiple training rounds, as they often require expert intervention between consecutive rounds.
We train our model using co-evolving pseudo labels end-to-end without such need.

\begin{figure}[t]%
    \centering
    \def\svgwidth{\linewidth}
    \input{figures/preview/bars.pdf_tex}
    \caption{\textbf{Results preview.} Unlike much recent work that combines multiple training paradigms, such as adversarial training and style transfer, our approach retains the modest single-round training complexity of self-training, yet improves the state of the art for adapting semantic segmentation by a significant margin.}
    \label{fig:preview}
\end{figure}

Our method leverages the ubiquitous \emph{data augmentation} techniques from fully supervised learning \cite{deeplabv3plus2018,ZhaoSQWJ17}: photometric jitter, flipping and multi-scale cropping.
We enforce \emph{consistency} of the semantic maps produced by the model across these image perturbations.
The following assumption formalises the key premise:

\myparagraph{Assumption 1.}
Let $f: \mathcal{I} \rightarrow \mathcal{M}$ represent a pixelwise mapping from images $\mathcal{I}$ to semantic output $\mathcal{M}$.
Denote $\rho_{\bm{\epsilon}}: \mathcal{I} \rightarrow \mathcal{I}$ a photometric image transform and, similarly, $\tau_{\bm{\epsilon}'}: \mathcal{I} \rightarrow \mathcal{I}$ a spatial similarity transformation, where $\bm{\epsilon},\bm{\epsilon}'\sim p(\cdot)$ are control variables following some pre-defined density (\eg, $p \equiv \mathcal{N}(0, 1)$).
Then, for any image $I \in \mathcal{I}$, $f$ is \emph{invariant} under $\rho_{\bm{\epsilon}}$ and \emph{equivariant} under $\tau_{\bm{\epsilon}'}$, \ie~$f(\rho_{\bm{\epsilon}}(I)) = f(I)$ and $f(\tau_{\bm{\epsilon}'}(I)) = \tau_{\bm{\epsilon}'}(f(I))$.

\smallskip
\noindent Next, we introduce a training framework using a \emph{momentum network} -- a slowly advancing copy of the original model.
The momentum network provides stable, yet recent targets for model updates, as opposed to the fixed supervision in model distillation \cite{Chen0G18,Zheng_2020_IJCV,ZhengY20}.
We also re-visit the problem of long-tail recognition in the context of generating pseudo labels for self-supervision.
In particular, we maintain an \emph{exponentially moving class prior} used to discount the confidence thresholds for those classes with few samples and increase their relative contribution to the training loss.
Our framework is simple to train, adds moderate computational overhead compared to a fully supervised setup, yet sets a new state of the art on established benchmarks (\cf \cref{fig:preview}).

\section{Related Work}
%\mz{We lack a comparison to this paper: https://arxiv.org/abs/2305.14877}
%\anirudh{refine to be more on-topic?}
\iffalse
\paragraph{In-Context Learning} As language models have scaled, the ability to learn in-context, without any weight updates, has emerged. \cite{brown2020language}. While other families of large language models have emerged, in-context learning remains ubiquitous \cite{llama, bloom, gptneo, opt}. Although such as HELM \cite{helm} have arisen for systematic evaluation of \emph{models}, there is no systematic framework to our knowledge for evaluating \emph{prompting methods}, and validating prompt engineering heuristics. The test-suite we propose will ensure that progress in the field of prompt-engineering is structured and objectively evaluated. 

\paragraph{Prompt Engineering Methods} Researchers are interested in the automatic design of high performing instructions for downstream tasks. Some focus on simple heuristics, such as selecting instructions that have the lowest perplexity \cite{lowperplexityprompts}. Other methods try to use large language models to induce an instruction when provided with a few input-output pairs \cite{ape}. Researchers have also used RL objectives to create discrete token sequences that can serve as instructions \cite{rlprompt}. Since the datasets and models used in these works have very little intersection, it is impossible to compare these methods objectively and glean insights. In our work, we evaluate these three methods on a diverse set of tasks and models, and analyze their relative performance. Additionally, we recognize that there are many other interesting angles of prompting that are not covered by instruction engineering \cite{weichain, react, selfconsistency}, but we leave these to future work.

\paragraph{Analysis of Prompting Methods} While most prompt engineering methods focus on accuracy, there are many other interesting dimensions of performance as well. For instance, researchers have found that for most tasks, the selection of demonstrations plays a large role in few-shot accuracy \cite{whatmakesgoodicexamples, selectionmachinetranslation, knnprompting}. Additionally, many researchers have found that even permuting the ordering of a fixed set of demonstrations has a significant effect on downstream accuracy \cite{fantasticallyorderedprompts}. Prompts that are sensitive to the permutation of demonstrations have been shown to also have lower accuracies \cite{relationsensitivityaccuracy}. Especially in low-resource domains, which includes the large public usage of in-context learning, these large swings in accuracy make prompting less dependable. In our test-suite we include sensitivity metrics that go beyond accuracy and allow us to find methods that are not only performant but reliable.

\paragraph{Existing Benchmarks} We recognize that other holistic in-context learning benchmarks exist. BigBench is a large benchmark of 204 tasks that are beyond the capabilities of current LLMs. BigBench seeks to evaluate the few-shot abilities of state of the art large language models, focusing on performance metrics such as accuracy \cite{bigbench}. Similarly, HELM is another benchmark for language model in-context learning ability. Rather than only focusing on performance, HELM branches out and considers many other metrics such as robustness and bias \cite{helm}. Both BigBench and HELM focus on ranking different language model, while fix a generic instruction and prompt format. We instead choose to evaluate instruction induction / selection methods over a fixed set of models. We are the first ever evaluation script that compares different prompt-engineering methods head to head. 
\fi

\paragraph{In-Context Learning and Existing Benchmarks} As language models have scaled, in-context learning has emerged as a popular paradigm and remains ubiquitous among several autoregressive LLM families \cite{brown2020language, llama, bloom, gptneo, opt}. Benchmarks like BigBench \cite{bigbench} and HELM \cite{helm} have been created for the holistic evaluation of these models. BigBench focuses on few-shot abilities of state-of-the-art large language models, while HELM extends to consider metrics like robustness and bias. However, these benchmarks focus on evaluating and ranking \emph{language models}, and do not address the systematic evaluation of \emph{prompting methods}. Although contemporary work by \citet{yang2023improving} also aims to perform a similar systematic analysis of prompting methods, they focus on simple probability-based prompt selection while we evaluate a broader range of methods including trivial instruction baselines, curated manually selected instructions, and sophisticated automated instruction selection.

\paragraph{Automated Prompt Engineering Methods} There has been interest in performing automated prompt-engineering for target downstream tasks within ICL. This has led to the exploration of various prompting methods, ranging from simple heuristics such as selecting instructions with the lowest perplexity \cite{lowperplexityprompts}, inducing instructions from large language models using a few annotated input-output pairs \cite{ape}, to utilizing RL objectives to create discrete token sequences as prompts \cite{rlprompt}. However, these works restrict their evaluation to small sets of models and tasks with little intersection, hindering their objective comparison. %\mz{For paragraphs that only have one work in the last line, try to shorten the paragraph to squeeze in context.}

\paragraph{Understanding in-context learning} There has been much recent work attempting to understand the mechanisms that drive in-context learning. Studies have found that the selection of demonstrations included in prompts significantly impacts few-shot accuracy across most tasks \cite{whatmakesgoodicexamples, selectionmachinetranslation, knnprompting}. Works like \cite{fantasticallyorderedprompts} also show that altering the ordering of a fixed set of demonstrations can affect downstream accuracy. Prompts sensitive to demonstration permutation often exhibit lower accuracies \cite{relationsensitivityaccuracy}, making them less reliable, particularly in low-resource domains.

Our work aims to bridge these gaps by systematically evaluating the efficacy of popular instruction selection approaches over a diverse set of tasks and models, facilitating objective comparison. We evaluate these methods not only on accuracy metrics, but also on sensitivity metrics to glean additional insights. We recognize that other facets of prompting not covered by instruction engineering exist \cite{weichain, react, selfconsistency}, and defer these explorations to future work. 
%!TEX root = hopfwright.tex
%

In this section we systematically recast the Hopf bifurcation problem in Fourier space. 
We introduce appropriate scalings, sequence spaces of Fourier coefficients and convenient operators on these spaces. 
To study Equation~\eqref{eq:FourierSequenceEquation} we consider Fourier sequences $ \{a_k\}$ and fix a Banach space in which these sequences reside. It is indispensable for our analysis that this space have an algebraic structure. 
The Wiener algebra of absolutely summable Fourier series is a natural candidate, which we use with minor modifications. 
In numerical applications, weighted sequence spaces with algebraic and geometric decay have been used to great effect to study periodic solutions which are $C^k$ and analytic, respectively~\cite{lessard2010recent,hungria2016rigorous}. 
Although it follows from Lemma~\ref{l:analytic} that the Fourier coefficients of any solution decay exponentially, we choose to work in a space of less regularity. 
The reason is that by working in a space with less regularity, we are better able to connect our results with the global estimates in \cite{neumaier2014global}, see Theorem~\ref{thm:UniqunessNbd2}.


%
%
%\begin{remark}
%	Although it follows from Lemma~\ref{l:analytic} that the Fourier coefficients of any solution decay exponentially, we choose to work in a space of less regularity, namely summable Fourier coefficients. This allows us to draw SOME MORE INTERESTING CONCLUSION LATER.
%	EXPLAIN WHY WE CHOOSE A NORM WITH ALMOST NO DECAY!
%	% of s Periodic solutions to Wright's equation are known to be real analytic and so their  Fourier coefficients must decay geometrically [Nussbaum].
%	% We do not use such a strong result;  any periodic solution must be continuously differentiable, by which it follows that $ \sum | c_k| < \infty$.
%\end{remark}


\begin{remark}\label{r:a0}
There is considerable redundancy in Equation~\eqref{eq:FourierSequenceEquation}. First, since we are considering real-valued solutions $y$, we assume $\c_{-k}$ is the complex conjugate of $\c_k$. This symmetry implies it suffices to consider Equation~\eqref{eq:FourierSequenceEquation} for $k \geq 0$.
Second, we may effectively ignore the zeroth Fourier coefficient of any periodic solution \cite{jones1962existence}, since it is necessarily equal to $0$. 
%In \cite{jones1962existence}, it is shown that if $y \not\equiv -1$ is a periodic solution of~\eqref{eq:Wright} with frequency $\omega$, then $ \int_0^{2\pi/\omega} y(t) dt =0$. 
		The self contained argument is as follows. 
		As mentioned in the introduction, any periodic solution to Wright's equation must satisfy $ y(t) > -1$ for all $t$. 
	By dividing Equation~\eqref{eq:Wright} by $(1+y(t))$, which never vanishes, we obtain
	\[
	\frac{d}{dt} \log (1 + y(t)) = - \alpha y(t-1).
	\]  
	Integrating over one period $L$ we derive the condition 
	$0=\int_0^L y(t) dt $.
	Hence $a_0=0$ for any periodic solution. 
	It will be shown in Theorem~\ref{thm:FourierEquivalence1} that a related argument implies that we do not need to consider Equation~\eqref{eq:FourierSequenceEquation} for $k=0$.
\end{remark}

%%%
%%%
%%%\begin{remark}\label{r:c0} 
%%%In \cite{jones1962existence}, it is shown that if $y \not\equiv -1$ is a periodic solution of~\eqref{eq:Wright} with frequency $\omega$, then $ \int_0^{2\pi/\omega} y(t) dt =0$. 
%%%PERHAPS TOO MUCH DETAIL HERE. The self contained argument is as follows.
%%%If $y \not\equiv -1$ then $y(t) \neq -1$ for all $t$, since if $y(t_0)=-1$ for some $t_0 \in \R$ then $y'(t_0)=0$ by~\eqref{eq:Wright} and in fact by differentiating~\eqref{eq:Wright} repeatedly one obtains that all derivatives of $y$ vanish at $t_0$. Hence $y \equiv -1$ by Lemma~\ref{l:analytic}, a contradiction. Now divide~\eqref{eq:Wright} by $(1+y(t))$, which never vanishes, to obtain
%%%\[
%%%  \frac{d}{dt} \log |1 + y(t)| = - \alpha y(t-1).
%%%\]  
%%%Integrating over one period we obtain $\int_0^L y(t) dt =0$.
%%%\end{remark}



%Furthermore, the condition that $y(t)$ is real forces $\c_{-k} = \overline{\c}_{k}$.  
%
We define the spaces of absolutely summable Fourier series
\begin{alignat*}{1}
	\ell^1 &:= \left\{ \{ \c_k \}_{k \geq 1} : 
    \sum_{k \geq 1} | \c_k| < \infty  \right\} , \\
	\ell^1_\bi &:= \left\{ \{ \c_k \}_{k \in \Z} : 
    \sum_{k \in \Z} | \c_k| < \infty  \right\} .
\end{alignat*} 
We identify any semi-infinite sequence $ \{ \c_k \}_{k \geq 1} \in \ell^1$ with the bi-infinite sequence $ \{ \c_k \}_{k \in \Z} \in \ell^1_\bi$ via the conventions (see Remark~\ref{r:a0})
\begin{equation}
  \c_0=0 \qquad\text{ and }\qquad \c_{-k} = \c_{k}^*. 
\end{equation}
In other word, we identify $\ell^1$ with the set
\begin{equation*}
   \ell^1_\sym := \left\{ \c \in \ell^1_\bi : 
	\c_0=0,~\c_{-k}=\c_k^* \right\} .
\end{equation*}
On $\ell^1$ we introduce the norm
\begin{equation}\label{e:lnorm}
  \| \c \| = \| \c \|_{\ell^1} := 2 \sum_{k = 1}^\infty |\c_k|.
\end{equation}
The factor $2$ in this norm is chosen to have a Banach algebra estimate.
Indeed, for $\c, \tilde{\c} \in \ell^1 \cong \ell^1_\sym$ we define
the discrete convolution 
\[
\left[ \c * \tilde{\c} \right]_k = \sum_{\substack{k_1,k_2\in\Z\\ k_1 + k_2 = k}} \c_{k_1} \tilde{\c}_{k_2} .
\]
Although $[\c*\tilde{\c}]_0$ does not necessarily vanish, we have $\{\c*\tilde{\c}\}_{k \geq 1} \in \ell^1 $ and 
\begin{equation*}
	\| \c*\tilde{\c} \| \leq \| \c \| \cdot  \| \tilde{\c} \| 
	\qquad\text{for all } \c , \tilde{\c} \in \ell^1, 
\end{equation*}
hence $\ell^1$ with norm~\eqref{e:lnorm} is a Banach algebra.

By Lemma~\ref{l:analytic} it is clear that any periodic solution of~\eqref{eq:Wright} has a well-defined Fourier series $\c \in \ell^1_\bi$. 
The next theorem shows that in order to study periodic orbits to Wright's equation we only need to study Equation~\eqref{eq:FourierSequenceEquation} 
for $k \geq 1$. For convenience we introduce the notation 
\[
G(\alpha,\omega,\c)_k=
( i \omega k + \alpha e^{ - i \omega k}) \c_k + \alpha \sum_{k_1 + k_2 = k} e^{- i \omega k_1} \c_{k_1} \c_{k_2} \qquad \text{for } k \in \N.
\]
We note that we may interpret the trivial solution $y(t)\equiv 0$ as a periodic solution of arbitrary period.
\begin{theorem}
\label{thm:FourierEquivalence1}
Let $\alpha>0$ and $\omega>0$.
If $\c \in \ell^1 \cong \ell^1_{\sym}$ solves
$G(\alpha,\omega,\c)_k =0$  for all $k \geq 1$,
then $y(t)$ given by~\eqref{eq:FourierEquation} is a periodic solution of~\eqref{eq:Wright} with period~$2\pi/\omega$.
Vice versa, if $y(t)$ is a periodic solution of~\eqref{eq:Wright} with period~$2\pi/\omega$ then its Fourier coefficients $\c \in \ell^1_\bi$ lie in $\ell^1_\sym \cong \ell^1$ and solve $G(\alpha,\omega,\c)_k =0$ for all $k \geq 1$.
\end{theorem}

\begin{proof}	
	If $y(t)$ is a periodic solution of~\eqref{eq:Wright} then it is real analytic by Lemma~\ref{l:analytic}, hence its Fourier series $\c$ is well-defined and $\c \in \ell^1_{\sym}$ by Remark~\ref{r:a0}.
	Plugging the Fourier series~\eqref{eq:FourierEquation} into~\eqref{eq:Wright} one easily derives that $\c$ solves~\eqref{eq:FourierSequenceEquation} for all $k \geq 1$.

To prove the reverse implication, assume that $\c \in \ell^1_\sym$ solves
Equation~\eqref{eq:FourierSequenceEquation} for all $k \geq 1$. Since $\c_{-k}
= \c_k^*$, Equation \eqref{eq:FourierSequenceEquation} is also satisfied for
all $k \leq -1$. It follows from the Banach algebra property and
\eqref{eq:FourierSequenceEquation} that $\{k \c_k\}_{k \in \Z} \in \ell^1_\bi$,
hence $y$, given by~\eqref{eq:FourierEquation}, is continuously differentiable.
% (and by bootstrapping one infers that $\{k^m c_k \} \in \ell^1_\bi$, 
% hence $y \in C^m$ for any $m \geq 1$).
	Since~\eqref{eq:FourierSequenceEquation} is satisfied for all $k \in \Z \setminus \{0\}$ (but not necessarily for $k=0$) one may perform the inverse Fourier transform on~\eqref{eq:FourierSequenceEquation} to conclude that
	$y$ satisfies the delay equation 
\begin{equation}\label{eq:delaywithK}
   	y'(t) = - \alpha y(t-1) [ 1 + y(t)] + C
\end{equation}
	for some constant $C \in \R$. 
   Finally, to prove that $C=0$ we argue by contradiction.
   Suppose $C \neq 0$. Then $y(t) \neq -1$ for all $t$.
   Namely, at any point where $y(t_0) =-1$ one would have $y'(t_0) = C$
   which has fixed sign,   hence it would follow that $y$ is not periodic
   ($y$ would not be able to cross $-1$ in the opposite direction, 
   preventing $y$  from being periodic).  
  We may thus divide~\eqref{eq:delaywithK} through by $1 + y(t)$ and obtain 
\begin{equation*}
	\frac{d}{dt} \log | 1 + y(t) | = - \alpha y(t-1) + \frac{C}{1+y(t)} .
\end{equation*}
	By integrating both sides of the equation over one period $L$ and by using that $\c_0=0$, we 
	obtain
	\[
	 C \int_0^L \frac{1}{1+y(t)} dt =0.
	\]
	Since the integrand is either strictly negative or strictly positive, this implies that $C=0$. Hence~\eqref{eq:delaywithK} reduces to~\eqref{eq:Wright},
	and $y$ satisfies Wright's equation. 
\end{proof}






To efficiently study Equation~\eqref{eq:FourierSequenceEquation}, we introduce the following linear operators on $ \ell^1$:
\begin{alignat*}{1}
   [K \c ]_k &:= k^{-1} \c_k  , \\ 
   [ U_\omega \c ]_k &:= e^{-i k \omega} \c_k  .
\end{alignat*}
The map $K$ is a compact operator, and it has a densely defined inverse $K^{-1}$. The domain of $K^{-1}$ is denoted by
\[
  \ell^K := \{ \c \in \ell^1 : K^{-1} \c \in \ell^1 \}.  
\]
The map $U_{\omega}$ is a unitary operator on $\ell^1$, but
it is discontinuous in $\omega$. 
With this notation, Theorem~\ref{thm:FourierEquivalence1} implies that our problem of finding a SOPS to~\eqref{eq:Wright} is equivalent to finding an $\c \in \ell^1$ such that
\begin{equation}
\label{e:defG}
  G(\alpha,\omega,\c) :=
  \left( i \omega K^{-1} + \alpha U_\omega \right) \c + \alpha \left[U_\omega \, \c \right] * \c  = 0.
\end{equation}


%In order for the solutions of Equation \ref{eq:FHat} to be isolated we need to impose a phase condition. 
%If there is a sequence $ \{ c_k \} $ which satisfies  Equation \ref{eq:FHat}, then $ y( t + \tau) = \sum_{k \in \Z} c_k e^{ i k \omega (t + \tau)}$ satisfies Wright's equation at parameter $\alpha$. 
%Fix $ \tau = - Arg[c_1] / \omega$ so that $ c_1  e^{ i \omega \tau} $ is a nonnegative real number. 
%By Proposition \ref{thm:FourierEquivalence1} it follows that $\{ c'_k \} =  \{c_k e^{ i \omega k \tau }   \}$ is a solution to Equation \ref{eq:FHat}, and furthermore that $ c'_1 = \epsilon$ for some $ \epsilon \geq 0$. 


Periodic solutions are invariant under time translation: if $y(t)$ solves Wright's equation, then so does $ y(t+\tau)$ for any $\tau \in \R$. 
We remove this degeneracy by adding a phase condition. 
Without loss of generality, if $\c \in \ell^1$ solves Equation~\eqref{e:defG}, we may assume that $\c_1 = \epsilon$ for some 
\emph{real non-negative}~$\epsilon$:
\[
  \ell^1_{\epsilon} := \{\c \in \ell^1 : \c_1 = \epsilon \} 
  \qquad \text{where } \epsilon \in \R,  \epsilon \geq 0.
\]
In the rest of our analysis, we will split elements $\c \in \ell^1$ into two parts: $\c_1$ and $\{\c_{k}\}_{k \geq 2}$.  
We define the basis elements $\e_j \in \ell^1$ for $j=1,2,\dots$ as
\[
  [\e_j]_k = \begin{cases}
  1 & \text{if } k=j, \\
  0 & \text{if } k \neq j.
  \end{cases}
\]
We note that $\| \e_j \|=2$. 
Then we can decompose
% We define
% \[
%   \tilde{\epsilon} := (\epsilon,0,0,0,\dots) \in \ell^1
% \]
% and
% For clarity when referring to sequences $\{c_{k}\}_{k \geq 2}$, we make the following definition:
% \[
% \ell^1_0  := \{ \tc \in \ell^1 : \tc_1 = 0 \}.
% \]
% With the
any $\c \in \ell^1_\epsilon$ uniquely as
\begin{equation}\label{e:aepsc}
  \c= \epsilon \e_1 + \tc \qquad \text{with}\quad 
  \tc \in \ell^1_0 := \{ \tc \in \ell^1 : \tc_1 = 0 \}.
\end{equation}
We follow the classical approach in studying Hopf bifurcations and consider 
$\c_1 = \epsilon$ to be a parameter, and then find periodic solutions with Fourier modes in $\ell^1_{\epsilon}$.
This approach rewrites the function $G: \R^2 \times \ell^K \to \ell^1$ as a function $\tilde{F}_\epsilon : \R^2 \times \ell^K_0 \to \ell^1$, where 
we denote 
\[
\ell^K_0 := \ell^1_0 \cap \ell^K.
\]
% I AM ACTUALLY NOT SURE IF YOU WANT TO DEFINE THIS WITH RANGE IN $\ell^1$
% OR WITH DOMAIN IN $\ell^1_0$ ?? IT SEEMS TO DEPEND ON WHICH GLOBAL STATEMENT YOU WANT/NEED TO MAKE!?
\begin{definition}
We define the $\epsilon$-parameterized family of  functions $\tilde{F}_\epsilon: \R^2 \times \ell^K_0  \to \ell^1$ 
by 
\begin{equation}
\label{eq:fourieroperators}
\tilde{F}_{\epsilon}(\alpha,\omega, \tc) := 
\epsilon [i \omega + \alpha e^{-i \omega}] \e_1 + 
( i \omega K^{-1} + \alpha U_{\omega}) \tc + 
\epsilon^2 \alpha e^{-i \omega}  \e_2  +
\alpha \epsilon L_\omega \tc + 
\alpha  [ U_{\omega} \tc] * \tc ,
\end{equation}
where
$L_\omega : \ell^1_0 \to \ell^1$ is given by
\[
   L_{\omega} := \sigma^+( e^{- i \omega} I + U_{\omega}) + \sigma^-(e^{i \omega} I + U_{\omega}),
\]
with $I$ the identity and  $\sigma^\pm$ the shift operators on $\ell^1$:
\begin{alignat*}{2}
\left[ \sigma^- a \right]_k &:=  a_{k+1}  , \\
\left[ \sigma^+ a \right]_k &:=  a_{k-1}  &\qquad&\text{with the convention } \c_0=0.
\end{alignat*}
The operator $ L_\omega$ is discontinuous in $\omega$ and $ \| L_\omega \| \leq 4$. 
\end{definition} 

%The maps $ \sigma^{+}$ and $ \sigma^-$ are shift up and shift down operators respectively. 
We reformulate Theorem~\ref{thm:FourierEquivalence1}  in terms of the map  $\tilde{F}$. 
We note that it follows from Lemma~\ref{l:analytic} and 
%\marginpar{Reformulate}
%one's choice of  
Equation~\eqref{eq:FourierSequenceEquation}  
%or Equation ~\eqref{eq:fourieroperators},
that the Fourier coefficients of any periodic solution of~\eqref{eq:Wright} lie in $\ell^K$.
These observations are summarized in the following theorem.
\begin{theorem}
\label{thm:FourierEquivalence2}
	Let $ \epsilon \geq 0$,  $\tc \in \ell^K_0$, $\alpha>0$ and $ \omega >0$. 
	Define $y: \R\to \R$ as 
\begin{equation}\label{e:ytc}
	y(t) = 
	\epsilon \left( e^{i \omega t }  + e^{- i \omega t }\right) 
	+  \sum_{k = 2}^\infty   \tc_k e^{i \omega k t }  + \tc_k^* e^{- i \omega k t } .
\end{equation}
%	and suppose that $ y(t) > -1$. 
	Then $y(t)$ solves~\eqref{eq:Wright} if and only if $\tilde{F}_{\epsilon}( \alpha , \omega , \tc) = 0$. 
	Furthermore, up to time translation, any periodic solution of~\eqref{eq:Wright} with period $2\pi/\omega$ is described by a Fourier series of the form~\eqref{e:ytc} with $\epsilon \geq 0$ and $\tc \in \ell^K_0$.
\end{theorem}


%We note that for $\epsilon>0$ such solutions are truly periodic, while for $\epsilon=0$ a zero of $\tilde{F}_\epsilon$ may either correspond to a periodic solution or to the trivial solution $y(t) \equiv 0$. 



% \begin{proof}
%  By Proposition \ref{thm:FourierEquivalence1}, it suffices to show that $\tilde{F}(\alpha,\omega,c) =0$ is equivalent to Equation \ref{eq:FourierSequenceEquation} being satisfied for $k \geq 1$.
%  Since Equation \ref{eq:FourierSequenceEquation} is equivalent to Equation \ref{eq:FHat}, we expand  Equation \ref{eq:FHat} by writing $ \hat{c} = \hat{\epsilon } + c$  where $ \hat{\epsilon} := (\epsilon,0,0,\dots) \in \ell^1$ as below:
%  \begin{equation}
%  0=  \left( i \omega K^{-1} + \alpha U_\omega \right) (\hat{\epsilon}+ c) + \alpha \left[U_\omega \, (\hat{\epsilon}+ c) \right] * (\hat{\epsilon}+ c) \label{eq:Intial}
%  \end{equation}
%  The RHS of Equation \ref{eq:Intial} is $ \tilde{F}(\alpha,\omega,c)$, so the theorem is proved.
% \end{proof}



Since we want to analyze a Hopf bifurcation, we will want to solve $\tilde{F}_\epsilon = 0$ for small values of~$\epsilon$. 
However, at the bifurcation point, $ D \tilde{F}_0(\pp  ,\pp , 0)$ is not invertible.
In order for our asymptotic analysis to be non-degenerate,
we work with a rescaled version of the problem. To this end, for any $\epsilon >0$, we rescale both $\tc$ and $\tilde{F}$ as follows. Let $\tc = \epsilon c$ and 
\begin{equation}\label{e:changeofvariables}
  \tilde{F}_\epsilon (\alpha,\omega,\epsilon c) = \epsilon F_\epsilon (\alpha,\omega,c).
\end{equation}
For $\epsilon>0$ the problem then reduces to finding zeros of 
\begin{equation}
\label{eq:FDefinition}
	F_\epsilon(\alpha,\omega, c) := 
	[i \omega + \alpha e^{-i \omega}] \e_1 + 
	( i \omega K^{-1} + \alpha U_{\omega}) c + 
	\epsilon \alpha e^{-i \omega} \e_2  +
	\alpha \epsilon L_\omega c + 
	\alpha \epsilon [ U_{\omega} c] * c.
\end{equation}
We denote the triple $(\alpha,\omega,c) \in \R^2 \times \ell^1_0$ by $x$.
To pinpoint the components of $x$ we use the projection operators
\[
   \pi_\alpha x = \alpha, \quad \pi_\omega x = \omega, \quad 
  \pi_c x = c \qquad\text{for any } x=(\alpha,\omega,c).
\]

After the change of variables~\eqref{e:changeofvariables} we now have an invertible Jacobian $D F_0(\pp  ,\pp , 0)$ at the bifurcation point.
On the other hand, for $\epsilon=0$ the zero finding problems for $\tilde{F}_\epsilon$ and $F_\epsilon$ are not equivalent. 
However, it follows from the following lemma that any nontrivial periodic solution having $ \epsilon=0$ must have a relatively large size when $ \alpha $ and $ \omega $ are close to the bifurcation point. 

\begin{lemma}\label{lem:Cone}
	Fix $ \epsilon \geq 0$ and $\alpha,\omega >0$. 
	Let
	\[
	b_* :=  \frac{\omega}{\alpha} - \frac{1}{2} - \epsilon  \left(\frac{2}{3}+ \frac{1}{2}\sqrt{2 + 2 |\omega-\pp| } \right).
	\]
Assume that $b_*> \sqrt{2} \epsilon$. 
Define
% \begin{equation*}%\label{e:zstar}
% 	z^{\pm}_* :=b_* \pm \sqrt{(b_*)^2- \epsilon^2 } .
% \end{equation*}
% \note[J]{Proposed change to match Lemma E.4}
\begin{equation}\label{e:zstar}
z^{\pm}_* :=b_* \pm \sqrt{(b_*)^2- 2 \epsilon^2 } .
\end{equation}
If there exists a $\tc \in \ell^1_0$ such that $\tilde{F}_\epsilon(\alpha, \omega,\tc) = 0$, then \\
\mbox{}\quad\textup{(a)} either $ \|\tc\| \leq  z_*^-$ or $ \|\tc\| \geq z_*^+  $.\\
\mbox{}\quad\textup{(b)} 
$ \| K^{-1} \tc \| \leq (2\epsilon^2+ \|\tc\|^2) / b_*$. 
\end{lemma}
\begin{proof}
	The proof follows from Lemmas~\ref{lem:gamma} and~\ref{lem:thecone} in Appendix~\ref{appendix:aprioribounds}, combined with the observation that
$\frac{\omega}{\alpha} - \gamma \geq b_*$,
% \[
%   \frac{\omega}{\alpha} - \gamma \geq b_*
%  \qquad\text{for all }
% | \alpha - \pp| \leq r_\alpha \text{ and } 
%   | \omega - \pp| \leq r_\omega.
% \]
with $\gamma$ as defined in Lemma~\ref{lem:gamma}.
\end{proof}

\begin{remark}\label{r:smalleps}
We note that for $\alpha < 2\omega$
\begin{alignat*}{1}
z^+_* &\geq   \frac{2 \omega - \alpha}{\alpha} 
- \epsilon \left(4/3+\sqrt{2 + 2 |\omega-\pp| } \, \right) + \cO(\epsilon^2)
\\[1mm]
z^-_* & \leq   \cO(\epsilon^2)
\end{alignat*}
for small $\epsilon$. 
Hence Lemma~\ref{lem:Cone} implies that for values of $(\alpha,\omega)$ near $(\pp,\pp)$ any solution has either $\|\tc\|$ of order 1 or $\|\tc\| =  \cO(\epsilon^2)$. 
The asymptotically small term bounding $z_*^-$ is explicitly calculated in Lemma~\ref{lem:ZminusBound}. 
A related consequence is that for $\epsilon=0$ there are no nontrivial solutions 
of $\tilde{F}_0(\alpha,\omega,\tc)=0$ with 
$\| \tc \| < \frac{2 \omega - \alpha}{\alpha} $. 
\end{remark}

\begin{remark}\label{r:rhobound}
In Section~\ref{s:contraction} we will work on subsets of $\ell^K_0$ of the form
\[
  \ell_\rho := \{ c \in \ell^K_0 : \|K^{-1} c\| \leq \rho \} .
\]
Part (b) of Lemma~\ref{lem:Cone} will be used in Section~\ref{s:global} to guarantee that we are not missing any solutions by considering $\ell_\rho$ (for some specific choice of $\rho$) rather than the full space $\ell^K_0$.
In particular, we infer from Remark~\ref{r:smalleps} that  small solutions (meaning roughly that $\|\tc\| \to 0$ as $\epsilon \to 0$)
satisfy $\| K^{-1} \tc \| = \cO(\epsilon^2)$.
\end{remark}

The following theorem guarantees that near the bifurcation point the problem of finding all periodic solutions is equivalent to considering the rescaled problem $F_\epsilon(\alpha,\omega,c)=0$.
\begin{theorem}
\label{thm:FourierEquivalence3}
\textup{(a)} Let $ \epsilon > 0$,  $c \in \ell^K_0$, $\alpha>0$ and $ \omega >0$. 
	Define $y: \R\to \R$ as 
\begin{equation}\label{e:yc}
	y(t) = 
	\epsilon \left( e^{i \omega t }  + e^{- i \omega t }\right) 
	+ \epsilon  \sum_{k = 2}^\infty   c_k e^{i \omega k t }  + c_k^* e^{- i \omega k t } .
\end{equation}
%	and suppose that $ y(t) > -1$. 
	Then $y(t)$ solves~\eqref{eq:Wright} if and only if $F_{\epsilon}( \alpha , \omega , c) = 0$.\\
\textup{(b)}
Let $y(t) \not\equiv 0$ be a periodic solution of~\eqref{eq:Wright} of period $2\pi/\omega$
 with Fourier coefficients $\c$.
Suppose $\alpha < 2\omega$ and $\| \c \| < \frac{2 \omega - \alpha}{\alpha} $.
Then, up to time translation, $y(t)$ is described by a Fourier series of the form~\eqref{e:yc} with $\epsilon > 0$ and $c \in \ell^K_0$.
\end{theorem}

\begin{proof}
Part (a) follows directly from Theorem~\ref{thm:FourierEquivalence2} and the  change of variables~\eqref{e:changeofvariables}.
To prove part (b) we need to exclude the possibility that there is a nontrivial solution with $\epsilon=0$. The asserted bound on the ratio of $\alpha$ and $\omega$ guarantees, by Lemma~\ref{lem:Cone} (see also Remark~\ref{r:smalleps}), that indeed $\epsilon>0$ for any nontrivial solution. 
\end{proof}

We note that in practice (see Section~\ref{s:global}) a bound on $\| \c \|$ is derived from a bound on $y$ or $y'$ using Parseval's identity.

\begin{remark}\label{r:cone}
It follows from Theorem~\ref{thm:FourierEquivalence3} and Remark~\ref{r:smalleps} that for values of $(\alpha,\omega)$ near $(\pp,\pp)$ any reasonably bounded solution satisfies $\| c\| =  O(\epsilon)$ as well as $\|K^{-1} c \| = O(\epsilon)$ asymptotically (as $\epsilon \to 0$).
These bounds will be made explicit (and non-asymptotic) for specific choices of the parameters in Section~\ref{s:global}.
\end{remark}

% We are able to rule out such large amplitude solutions using global estimates such as those in \cite{neumaier2014global}.
% Hence, near the bifurcation point, the problem of describing periodic solutions of~\eqref{eq:Wright} reduces to studying the family of zeros finding problems $F_\epsilon=0$.





%Specifically, if a solution having $ \epsilon = 0$ does in fact correspond to a nontrivial periodic solution and $\alpha  < 2\omega $, then $ \| \tilde{c} \| > 2 \omega \alpha^{-1} -1$. 
%%PERHAPS THIS NEEDS A FORMULATION AS A THEOREM AS WELL?
%%IN OTHER WORDS: ARE WE SURE WE HAVE FOUND ALL ZEROS OF $\tilde{F}_0$, I.E. ALL SOLUTIONS WITH $\epsilon=0$ NEAR THE BIFURCATION POINT? AFTER RESCALING THESE ARE INVISIBLE?
%%THERE SHOULD BE A STATEMENT ABOUT THIS SOMEWHERE! EITHER HERE OR SOME





We finish this section by defining a curve of approximate zeros $\bx_\epsilon$ of $F_\epsilon$ 
(see \cite{chow1977integral,hassard1981theory}). 
%(see \cite{chow1977integral,morris1976perturbative,hassard1981theory}). 


\begin{definition}\label{def:xepsilon}
Let
\begin{alignat*}{1}
	\balpha_\epsilon &:= \pp + \tfrac{\epsilon^2}{5} ( \tfrac{3\pi}{2} -1)  \\
	\bomega_\epsilon &:= \pp -  \tfrac{\epsilon^2}{5} \\
	\bc_\epsilon 	 &:= \left(\tfrac{2 - i}{5}\right) \epsilon \,  \e_2 \,.
\end{alignat*}
We define the approximate solution 
$ \bx_\epsilon := \left( \balpha_\epsilon , \bomega_\epsilon  , \bc_\epsilon \right)$
for all $\epsilon \geq 0$.
\end{definition}

We leave it to the reader to verify that both 
 $F_\epsilon(\pp,\pp,\bc_{\epsilon})=\cO(\epsilon^2)$ and $F_\epsilon(\bx_\epsilon)=\cO(\epsilon^2)$.
%%%	
%%%	
%%%	}{Better like this?}
%%%\annote[J]{ $F_\epsilon(\bx_0)=\cO(\epsilon^2)$ and $F_\epsilon(\bx_\epsilon)=\cO(\epsilon^2)$.}{I think we'd still need the $ \bar{c}_\epsilon$ term in $\bar{x}_0$ to be of order $ \epsilon$.}
%%%\remove[JB]{We show in Proposition A.1
%%%%\ref{prop:ApproximateSolutionWorks} 
%%% that any $ x \in \R^2 \times \ell^1_0$ which is $ \cO(\epsilon^2)$ close to $ \bar{x}_\epsilon $ will yield the estimate $F_\epsilon(x) = \cO(\epsilon^2)$.
%%%Hence choosing $\{ \pp , \pp, \bar{c}_\epsilon\}$ as our approximate solution would also have been a natural choice for performing an $\cO(\epsilon^2)$ analysis and would have simplified several of our calculations.
%%%However,} 
%%%
We choose to use the more accurate approximation 
for the $ \alpha$ and $ \omega $ components to improve our final quantitative results. 














%
% Values for $ (\alpha, \omega,c)$ which approximately solve $\tilde{F}(\alpha,\omega,c) = 0$  are computed in  \cite{chow1977integral,morris1976perturbative,hassard1981theory} and are as follows:
%  \begin{eqnarray}
%  \tilde{\alpha}( \epsilon) &:=& \pi /2 + \tfrac{\epsilon^2}{5} ( \tfrac{3\pi}{2} -1) \nonumber \\
%  \tilde{\omega}( \epsilon) &:=& \pi /2 -  \tfrac{\epsilon^2}{5} \label{eq:ScaleApprox} \\
%  \tc(\epsilon) 	  &:=& \{ \left(\tfrac{2 - i}{5}\right)  \epsilon^2 , 0,0, \dots \} \nonumber
%  \end{eqnarray}
% In Appendix \ref{sec:OperatorNorms} we illustrate an alternative method for deriving this approximation.
%
%
%
%
% We want to solve $ \tilde{F}(\alpha , \omega, \hat{c}) =0$ for small values of $ \epsilon$.
% However $ D \tilde{F}(\alpha , \omega , c)$ is not invertible at $ ( \pp , \pp , 0)$ when $ \epsilon = 0$.
% In order for our asymptotic analysis to be non-degenerate, we need to make the change of variables $ c \mapsto \epsilon c$.
% Under this change of variables, we define the function $ F$ below so that $ \tilde{F}(\alpha , \omega , \epsilon c) =\epsilon  F( \alpha , \omega , c)$.
%
%
%
% \begin{definition}
% Construct an $\epsilon$-parameterized family of densely defined functions  $F : \R^2 \oplus \ell^1 / \C \to \ell^1$ by:
% \begin{equation}
% \label{eq:FDefinition}
% 	F(\alpha,\omega, c) :=
% 	[i \omega + \alpha e^{-i \omega}]_1 +
% 	( i \omega K^{-1} + \alpha U_{\omega}) c +
% 	[\epsilon \alpha e^{-i \omega}]_2  +
% 	\alpha \epsilon L_\omega c +
% 	\alpha \epsilon [ U_{\omega} c] * c.
% \end{equation}
% \end{definition}

%%
%%
%%\begin{corollary}
%%	\label{thm:FourierEquivalence3}
%%	Fix $ \epsilon > 0$, and $ c \in \ell^1 / \C $, and $ \omega >0$. Define $y: \R\to \R$ as 
%%	\[
%%	y(t) = 
%%	\epsilon \left( e^{i \omega t }  + e^{- i \omega t }\right) 
%%	+  \epsilon  \left( \sum_{k = 2}^\infty   c_k e^{i \omega k t }  + \overline{c}_k e^{- i \omega k t } \right) 
%%	\]
%%	and suppose that $ y(t) > -1$. 
%%	Then $y(t)$ solves Wright's equation at parameter $ \alpha > 0 $ if and only if $ F( \alpha , \omega , c) = 0$ at parameter $ \epsilon$. 
%%	
%%	
%%	
%%\end{corollary}
%%
%%
%%\begin{proof}
%%	Since $ \tilde{F}(\alpha,\omega, \epsilon c) = \epsilon F( \alpha , \omega , c)$, the result follows from Theorem \ref{thm:FourierEquivalence2}.
%%\end{proof}

% If we can find $(\alpha , \omega, c)$ for which $ F( \alpha , \omega,c)=0$ at parameter $\epsilon$, then $ \tilde{F}(\alpha ,\omega, c)=0$.
% By Theorem \ref{thm:FourierEquivalence2} this amounts to finding a periodic solution to Wright's equation.
% Lastly, because we have performed the change of variables $ c \mapsto \epsilon c$, we need to  apply this change of variables to our approximate solution as well.
%
% \begin{definition}
% 	Define the approximate solution $ x( \epsilon) = \left\{ \alpha(\epsilon ) , \omega ( \epsilon ) , c(\epsilon) \right\}$ as below,  where $c(\epsilon) = \{ c_2( \epsilon) , 0 ,0 , \dots\} $.
% 	We may also write $ x_\epsilon = x(\epsilon) $.
% 	\begin{eqnarray}
% 	\alpha( \epsilon) &:=& \pi /2 + \tfrac{\epsilon^2}{5} ( \tfrac{3\pi}{2} -1) \nonumber \\
% 	\omega( \epsilon) &:=& \pi /2 -  \tfrac{\epsilon^2}{5} \label{eq:Approx} \\
% 	c_2(\epsilon) 	  &:=& \left(\tfrac{2 - i}{5}\right) \epsilon \nonumber
% 	\end{eqnarray}
%
% \end{definition}

\section{Uncertain Voting Profiles}
\label{sec:voting_profiles}

In classical voting theory, voters give complete rankings over candidates.
However, in practice only partial preferences may be observed, due to the voting mechanism (\eg when approval ballots or a ranking of at most $k<m$ candidates are elicited), the uncertainty in preferences themselves~\cite{marden1995analyzing}, or both.  Figure~\ref{fig:venn_diagrams_of_voting_profiles}  represents uncertainty as two distinct steps: preference generation (Figure~\ref{fig:venn_generation}) and preference observation (Figure~\ref{fig:venn_observation}). Important special cases of voting profiles are discussed next.

\begin{figure}[b!]
	\centering
	\subfloat[Generation step]{
		\label{fig:venn_generation}
		\includegraphics[width=0.22\linewidth]{figs/venn_diagram_of_probabilistic_voting_profiles}
	}
	\subfloat[Observation step]{
		\label{fig:venn_observation}
		\includegraphics[width=0.34\linewidth]{figs/venn_diagram_of_incomplete_voting_profiles}
	}~
	\caption{Uncertain voting profiles. (a) Uncertainty in the preferences by the voters themselves is represented by the ranking models in the generation step. (b) Uncertainty introduced by the preference elicitation mechanism is represented by the incomplete voting profiles in the observation step.}
	\label{fig:venn_diagrams_of_voting_profiles}
\end{figure}

\subsection{Uncertainty in profile generation}
The most general form of voter preferences over rankings is a non-parametric probability distribution such as that given in Figure~\ref{tab:profile}. 
Let $\probaVP=(\enum[n]{\model})$ denote a \e{probabilistic voting profile} where $\model_i$ is the ranking model of voter $v_i$.
A \e{possible world} of $\probaVP$ is a complete voting profile $\completeVP = (\enum[n]{\ranking})$ where each $\ranking_i$ is sampled from $\model_i$.
It is assumed that the voters cast their ballots independently, \ie $\Pr(\completeVP \mid \probaVP) = \prod_{i=1}^{n} {\Pr(\ranking_i \mid \model_i)}$.
So $\probaVP$ represents a probability distribution of its possible worlds $\Omega(\probaVP) = \set{\enum[\numPW]{\completeVP}}$.

Let $\score(c, \model)$ and $\score(c, \probaVP)$ denote the scores assigned to candidate $c$ by model $\model$ and  probabilistic voting profile $\probaVP$, respectively.
(Note that both are random variables.)
%
Partial voting profiles are a special case of probabilistic voting profiles, since they are based on the assumption that all completions are equally likely. 
Below are a few other cases.

In a \e{uniform voting profile}, denoted by $\uniVP$, no voter has any preference between any two candidates. Voter preferences are sampled from the uniform distribution over the rankings of all candidates.

There are specific important ranking models that we will describe in Section~\ref{sec:algorithms}.
The Mallows model~\cite{Mallows1957} is the best known, and it is generalized by the Repeated Insertion Model (RIM)~\cite{Doignon2004} and the ranking version of the Repeated Selection Model (RSM)~\cite{DBLP:journals/tdasci/ChakrabortyDKKR21}.
Voting profiles consisting of Mallows, RIMs, and RSMs (ranking version) are denoted by $\mallowsVP$, $\rimVP$, and $\rsmVP$, respectively.
We will see that, when the generation step is based on these models, we have a computational advantage for computing the \mew.

\subsection{Uncertainty in profile observation}

A \e{partial voting profile}, denoted by $\partialVP$, consists of partial orders and represents a uniform distribution of its completions.
What follows are important special cases.

A \e{(fully) partitioned voting profile}, denoted by $\fullparVP$, records preferences in the form of partitions: linear orders of item buckets, with no preference among the items within a bucket.

A \e{partially partitioned voting profile}, denoted by $\parparVP$, generalizes the fully partitioned preferences by only considering a subset of items, with no preference information for the missing ones.

A \e{partial chain voting profile}, denoted by $\pchainVP$, records preferences in the form of partial chains. A partial chain is a linear order of a subset of items, with no preference over the remaining items. The partial chains are a special case of the partially partitioned preferences, where each partition only consists of one item.

A \e{truncated voting profile}, denoted by $\trunVP$, records preferences in the form of truncated rankings.
Let $\ranking^{(t, b)}$ denote a truncated ranking with $t$ items at the top and $b$ items at the bottom, with no preference specified for the middle part of the ranking.
The $\ranking^{(t, b)}$ is a special case of the partitioned preferences, with $t+b+1$ partitions.

\paragraph{Combined voting profiles.}

Most research works have assumed that a partial voting profile represents a uniform distribution over its completions. However, the assumption that all completions are equally likely may not hold in practice.
We propose \e{combined voting profiles} $\VP^{\model+P}$, where each voter is associated with both her original incomplete preferences $P$ and a ranking model $\model$.  The ranking model $\model$ is her prior preference distribution, which may be obtained from historical data.
But after observing new evidence $P$, the probabilities of rankings that violate $P$ collapse to zero, while the remaining rankings that satisfy $P$ have the same relative probabilities among each other.
Formally speaking, the preferences of this voter become the posterior distribution of $\model$ conditioned on $P$. 

In another sense, the combined voting profiles are also the most general form of voting profiles (same as $\probaVP$) that unify all voting profiles so far.
For example, a partitioned voting profile $\fullparVP$ is essentially $\VP^{\text{U+FP}}$, and a RIM voting profile $\rimVP$ is essentially $\VP^{\RIM+\emptyset}$ where $\emptyset$ means that the partial orders are the empty ones.

Among the tractable cases in Section~\ref{sec:algorithms}, there are also combined voting profiles, which means that applying combined voting profiles both brings the benefit of more customized ranking distributions, and can also be practical.

\section{\mEw}
\label{sec:mew}

For a general voting profile and a positional scoring rule, the performance of a candidate can be quantified by the expectation of her score in a random possible world.  We define this formally below.

\begin{definition} [\mew] \label{def:mew}
  Given a general voting profile $\probaVP$ and a positional scoring rule $\vr_m$, candidate $w$ is a \mEw, \ifff, $\mathds{E}(\score(w, \probaVP)) = \max_{c \in C} \mathds{E}(\score(c, \probaVP))$.
\end{definition}

We denote the set of \mEws by $\mew(\vr_m, \probaVP)$.

\subsection{Alternative Interpretations}
\label{sec:mew:alternative_interpretations}

To gain an intuition for \mEw~(\mew), we will now give two winner definitions that are equivalent to \mew. Proofs can be found in Appendix~\ref{sec:appendix:proofs}.

\paragraph{Least Expected Regret Winner.}

The MEW can also be regarded as the candidate who minimizes the expected regret in a random possible world.
Let $\txt{Regret}(w, \completeVP)$ denote the regret value of choosing $w \in C$ as the winner given a complete voting profile $\completeVP$.
\[
\txt{Regret}(w, \completeVP) = \max_{c \in C} \score(c, \completeVP) - \score(w, \completeVP)
\]
Accordingly, the regret value $\txt{Regret}(w, \probaVP)$ over a probabilistic voting profile becomes a random variable.
\[
\mathds{E}(\txt{Regret}(w, \probaVP)) = \sum_{\completeVP \in \Omega(\probaVP)} \txt{Regret}(c, \completeVP) \cdot \Pr(\completeVP \mid \probaVP)
\]

Candidate $w$ is a \e{Least Expected Regret Winner}, \ifff, she minimizes the expected regret $\mathds{E}(\txt{Regret}(w, \probaVP))$.

\def\theoremLeastExpectedRegretWinner{
  Least Expected Regret Winner is equivalent to \mew.
}

\begin{theorem}
  \label{theorem:least_expected_regret_winner}
  \theoremLeastExpectedRegretWinner
\end{theorem}

\revv{Our use of expected regret as an alternative interpretation of MEW is inspired by Lou and Boutillier~\cite{DBLP:conf/ijcai/LuB11a}, who were the first to use regret in winner determination.  They proposed  MMR, the winner that minimized regret in the worst-case completion.  In contrast, MEW minimizes regret in expectation, across all completions.}

\paragraph{Meta-Election Winner.}
Recall that a voting profile $\probaVP$ represents a probability distribution of possible worlds $\Omega(\probaVP) = \set{\completeVP_1, \ldots, \completeVP_\numPW}$.
The Meta-Election Winner is defined as the candidate who wins a meta election with a large meta profile $\completeVP_M = (\completeVP_1, \ldots, \completeVP_\numPW)$ where rankings in $\completeVP_i$ are weighted by $\Pr(\completeVP_i \mid \probaVP)$.

\def\theoremGiganticElectionWinner{
  Meta-Election Winner is equivalent to \mew.
}

\begin{theorem}
  \label{theorem:gigantic_election_winner}
  \theoremGiganticElectionWinner
\end{theorem}

\subsection{Problem Statement}
\label{sec:problem}

The \mew is determined based on the expected performance of the candidates. \rev{Thus, the winner determination problem of \mew can be reduced to the problem of \ESC (\esc), stated below and addressed in the remainder of the paper.}

\begin{definition} [\esc] \label{def:esc}
  Given a general voting profile $\probaVP$, a positional scoring rule $\vr_m$, and a candidate $c \in C$, compute $\mathds{E}(\score(c, \probaVP))$ the expected score of the candidate $c$.
\end{definition}

\section{Hardness of \esc}
\label{sec:complexity}

This section investigates the complexity of the \esc problem. We first prove the hardness of two closely related problems, the Fixed-rank Counting Problem and the Rank Estimation Problem, and then demonstrate that the \esc problem is hard as well. Proofs can be found in Appendix~\ref{sec:appendix:proofs}.

\subsection{Fixed-rank Counting Problem}

Counting the number of linear extensions of a partial order is well-known to be \#P-complete~\cite{DBLP:conf/stoc/BrightwellW91}.
The Fixed-rank Counting Problem (FCP) is interested in the number of linear extensions where an item is placed at a specific rank.
Let $\Omega(\partialOrder)$ denote the set of linear extensions of $\partialOrder$, and $N(c {\rightarrow} j \mid \partialOrder)$ denote the number of linear extensions in $\Omega(\partialOrder)$ where item $c$ is placed at rank $j$.

\begin{definition}[FCP]
    Given a partial order $\partialOrder$ over $m$ items, an item $c$, and an integer $j \in [1, m]$, calculate $N(c {\rightarrow} j \mid \partialOrder)$, the number of linear extensions of $\partialOrder$ where item $c$ is placed at rank $j$.
\end{definition}

De Loof's doctoral dissertation~\cite{Loof2009EfficientCO} discusses this problem (Section 4.2.1), proposing exact and approximate algorithms, but does not provide proof of complexity.

\def\theoremFCPhardnessOverPartialOrders{
    The FCP is \#P-complete.
}

\begin{theorem} \label{theorem:FCP_shaPcomplete_over_partialOrders}
    \theoremFCPhardnessOverPartialOrders
\end{theorem}

The hardness of FCP facilitates the hardness proofs for the Rank Estimation Problem.

\subsection{Rank Estimation Problem}

Now we move on to the Rank Estimation Problem (REP).
This problem can be regarded as the probabilistic version of the FCP.
It calculates the probability that a given item is placed at a specific rank.
But the REP is generalized from partial orders to arbitrary ranking distributions.

\begin{definition}[REP]
    Given a ranking model $\model$ over $m$ items, an item $c$, and an integer $j \in [1, m]$, calculate $\Pr(c {\rightarrow} j \mid \model)$, the probability that item $c$ is placed at rank $j$, by $\model$.
\end{definition}

For the convenience of discussions in the rest of this Section, we also define two special cases of the REP where items are placed at the top and bottom of the linear extensions.

\begin{definition}[REP-$t$]
    Given a ranking model $\model$ of $m$ items and an item $c$, calculate $\Pr(c {\rightarrow} 1 \mid \model)$, the probability that item $c$ is placed at the top of a linear extension, by $\model$.
\end{definition}

\begin{definition}[REP-$b$]
	Given a ranking model $\model$ of $m$ items and an item $c$, calculate $\Pr(c {\rightarrow} m \mid \model)$, the probability that item $c$ is placed at the bottom of a linear extension, by $\model$.
\end{definition}

Lerche and S{\o}rensen~\cite{lerche2003evaluation} proposed an approximation for the REP over partial orders, but did not provide a formal complexity proof.
Bruggemann and Annoni~\cite{bruggemann2014average} and De Loof \etal \cite{de2011approximation} considered a related problem, calculating the expected rank of an item in the linear extensions of a partial order.
These works focused on approximation, lacking complexity proofs as well.

With the help of Theorem~\ref{theorem:FCP_shaPcomplete_over_partialOrders}, it turns out that the REP and its two variants are all FP$^{\#P}$-complete over partial orders.

\def\lemmaREPtHardnessOverPartialOrders{
    If ranking model $\model$ is a partial order $\partialOrder$ of $m$ items representing a uniform distribution of $\Omega(\partialOrder)$, the REP-t is FP$^{\#P}$-complete.
}

\begin{lemma} \label{lemma:REPt_shaPcomplete_over_partialOrders}
    \lemmaREPtHardnessOverPartialOrders
\end{lemma}

\def\lemmaREPbHardnessOverPartialOrders{
    If ranking model $\model$ is a partial order $\partialOrder$ of $m$ items representing a uniform distribution of $\Omega(\partialOrder)$, the REP-b is FP$^{\#P}$-complete.
}

\begin{lemma} \label{lemma:REPb_shaPcomplete_over_partialOrders}
	\lemmaREPbHardnessOverPartialOrders
\end{lemma}

\def\theoremRepHardnessOverPartialOrders{
    If ranking model $\model$ is a partial order $\partialOrder$ of $m$ items representing a uniform distribution of $\Omega(\partialOrder)$, the REP is FP$^{\#P}$-complete.
}

\begin{theorem}
    \label{theorem:REP_shaPcomplete_over_partialOrders}
    \theoremRepHardnessOverPartialOrders
\end{theorem}

\subsection{Complexity of \esc}

\esc is closely related to REP.  Firstly,  \esc  is no harder than REP over general voting profiles (Theorem~\ref{theorem:reduction}), which lays the foundation for the identification of tractable cases in Section~\ref{sec:algorithms}.

\def\theoremReductionEscToRep{
    Given a general voting profile $\probaVP$ and a positional scoring rule $\vr_m$, the \esc problem can be reduced to the REP.
}

\begin{theorem} \label{theorem:reduction}
    \theoremReductionEscToRep
\end{theorem}

\def\theoremRepMewEquivalentUnderKapproval{
    The REP for rank $k$ is equivalent to the \esc problem over either one or both of the $(k-1)$-approval and $k$-approval rules.
}

\begin{theorem} \label{theorem:REP_MEW_equivalent_under_k_approval}
    \theoremRepMewEquivalentUnderKapproval
\end{theorem}

Theorem~\ref{theorem:REP_MEW_equivalent_under_k_approval} provides an insight into the relation between REP and \esc in terms of computational complexity.
If a solver is available for the \esc problem over a collection of $k$-approval votes, this solver is computationally equivalent to the REP solver.
Note that Theorem~\ref{theorem:REP_MEW_equivalent_under_k_approval} is not limited to partial voting profiles.

\def\theoremHardnessForEscGivenPartialOrderAndPlurality{
    Given a partial voting profile $\partialVP$, a distinguished candidate $c$, and plurality rule $\vr_m$, the \esc problem of calculating $\mathds{E}(\score(c \mid \partialVP, \vr_m))$ is FP$^{\#P}$-complete.
}

\begin{theorem} \label{theorem:shaPcomplete_for_expected_score_given_partialOrder_and_plurality}
    \theoremHardnessForEscGivenPartialOrderAndPlurality
\end{theorem}

\def\theoremHardnessForEscGivenPartialOrderAndVeto{
    Given a partial voting profile $\partialVP$, a distinguished candidate $c$, and veto rule $\vr_m$, the \esc problem of calculating $\mathds{E}(\score(c \mid \partialVP, \vr_m))$ is FP$^{\#P}$-complete.
}

\begin{theorem} \label{theorem:shaPcomplete_for_expected_score_given_partialOrder_and_veto}
	\theoremHardnessForEscGivenPartialOrderAndVeto
\end{theorem}

\def\theoremHardnessForExpectedScoreGivenPartialOrderAndKApproval{
    Given a partial voting profile $\partialVP$, a distinguished candidate $c$, and $k$-approval rule $\vr_m$, the \esc problem of calculating $\mathds{E}(\score(c \mid \partialVP, \vr_m))$ is FP$^{\#P}$-complete.
}

\begin{theorem} \label{theorem:shaPcomplete_for_expected_score_given_partialOrder_and_k_approval}
    \theoremHardnessForExpectedScoreGivenPartialOrderAndKApproval
\end{theorem}

The three theorems above demonstrate the hardness of the \esc problem over partial voting profiles, under plurality, veto, and $k$-approval, respectively.
In particular, \esc is FP$^{\#P}$-complete even under plurality (Theorem~\ref{theorem:shaPcomplete_for_expected_score_given_partialOrder_and_plurality}).

\section{Most Expected Winner Solvers}
\label{sec:algorithms}

The problem of determining \mew can be reduced to the \esc problem (Definition~\ref{def:esc}), then further reduced to the REP by Theorem~\ref{theorem:reduction}.
This section will solve the \mew problem by solving the REP.
Proofs can be found in Appendix~\ref{sec:appendix:proofs}.

\subsection{Solver for RIM-based partial orders}
\label{sec:algorithms:exact}

We adopt the RIMDP algorithm~\cite{DBLP:conf/aaai/KenigIPKS18}, discussed in Section~\ref{sec:preliminaries:pref}, to develop an exact solver for the REP over posets, and more generally, for RIM combined with posets. The exact solver only modifies the RIMDP by always tracking the target candidate, such that when RIM insertions finish, RIMDP obtains the probabilities of the target candidate being placed at each rank.

The RIMDP has exponential complexity $O(m^{\text{cw}(\bsigma, \partialOrder)+2})$, where $\text{cw}(\bsigma, \partialOrder)$ is the cover width, given the reference ranking $\bsigma$ and the partial order $\partialOrder$ ~\cite{DBLP:conf/aaai/KenigIPKS18}. It takes $O(nm^{\text{cw}(\bsigma, \partialOrder)+3})$ to calculate \mew by executing RIMDP for all $m$ candidates and $n$ voters. 

\subsection{Solvers for Partitioned Preferences, Truncated Rankings,  and Partial Chains} 

It turns out that the \mew problem can be solved efficiently for all special cases of partial voting profiles in Figure~\ref{fig:venn_observation}, \ie the partitioned voting profile, 

\def\theoremTractabilityOfFP{
  Given a positional scoring rule $\vr_m$, a fully partitioned voting profile $\fullparVP {=} (\enum[n]{\fullpar})$, and candidate $w$, determining $w \in \mew(\vr_m, \fullparVP)$ is in $O(nm^2)$.
}

\begin{theorem} \label{theorem:tractability_of_fullparVP}
     \theoremTractabilityOfFP
\end{theorem}

The truncated voting profiles can be solved by the same approach above, since they are a special case of the fully partitioned profile.

\def\theoremTractabilityOfPC{
  Given a positional scoring rule $\vr_m$, a partial chain voting profile $\pchainVP = (\enum[n]{\pchain})$, and candidate $w$, determining $w \in \mew(\vr_m, \pchainVP)$ is in $O(nm^2)$.
}

\begin{theorem} \label{theorem:tractability_of_pchainVP}
    \theoremTractabilityOfPC
\end{theorem}

The \mew over partially partitioned preferences can be solved by extending the above approach for partial chain voting profiles.

\def\theoremTractabilityOfPP{
  Given a positional scoring rule $\vr_m$, a partially partitioned voting profile $\parparVP = (\enum[n]{\parpar})$, and candidate $w$, determining $w \in \mew(\vr_m, \parparVP)$ is in $O(nm^2)$.
}

\begin{theorem} \label{theorem:tractability_of_parparVP}
  \theoremTractabilityOfPP
\end{theorem}

\begin{algorithm}[tb!]
\small 
  \raggedright
  \caption{REP solver for RIM}
  \label{alg:rim_rank_estimation}
  \textbf{Input}: Item $c$, $\RIM(\bsigma, \Pi)$ where $|\bsigma| = m$ \\
  \textbf{Output}: $\set{k \rightarrow \Pr(c {\rightarrow} k \mid \bsigma, \Pi) \mid k \in [1, m]}$
  
  \begin{algorithmic}[1] %[1] enables line numbers
    \STATE $\delta_0 \defeq \emptyset$, $\mathcal{P}_0 \defeq \set{\delta_0}$ and $q_0(\delta_0) \defeq 1$
    \FOR {$i=1, \ldots, m$}
    \STATE $\mathcal{P}_i \defeq \set{}$
    \FOR {$\delta \in \mathcal{P}_{i-1}$}
    \FOR {$j=1, \ldots, i$} \label{alg:rim_rank_estimation:j}
    \IF {$\bsigma(i)$ is $c$}
    \STATE $\delta' \defeq \set{c \rightarrow j}$
    \ELSIF {$\delta = \set{c \rightarrow k}$ and $j \leq k$}
    \STATE $\delta' \defeq \set{c \rightarrow k + 1}$
    \ELSE
    \STATE $\delta' \defeq \delta$
    \ENDIF
    \IF {$\delta' \notin \mathcal{P}_i$}
    \STATE $\mathcal{P}_i.add(\delta')$
    \STATE $q_i(\delta') \defeq 0$
    \ENDIF
    \STATE $q_i(\delta') \pluseq q_{i-1}(\delta) \cdot \Pi(i, j)$
    \ENDFOR
    \ENDFOR
    \ENDFOR
    \STATE $\forall k \in [1, m], \Pr(c {\rightarrow} k \mid \bsigma, \Pi) = q_m(\set{c \rightarrow k})$.
    \RETURN $\set{k \rightarrow \Pr(c {\rightarrow} k \mid \bsigma, \Pi) \mid k \in [1, m]}$
  \end{algorithmic}
\end{algorithm}

\subsection{Solver for Probabilistic Voting Profiles}

While the problem of \mew determination is hard in general, it is tractable over specific ranking models such as the Mallows~\cite{Mallows1957} and its generalizations RIM~\cite{Doignon2004} and RSM~\cite{DBLP:journals/tdasci/ChakrabortyDKKR21}.

\def\theoremTractabilityOfRIM{
  Given positional scoring rule $\vr_m$, a RIM voting profile $\rimVP = (\enum[n]{\RIM})$, and candidate $w$, determining $w \in \mew(\vr_m, \rimVP)$ is in $O(nm^4)$.
}

\begin{theorem} \label{theorem:tractability_of_rimVP}
    \theoremTractabilityOfRIM
\end{theorem}

The complexity of \mew determination over RSM voting profiles is $O(nm^4)$ as well.
Proof can be found in Appendix~\ref{sec:appendix:proofs}.

\subsection{Solver for Combined Voting Profiles}

It is usually harder to compute the expected scores over combined voting profiles.
Below are some cases where this problem is tractable.
The first case is the RIMs combined with truncated rankings.

\def\theoremTractabilityOfRimTrun{
  Given a positional scoring rule $\vr_m$, a voting profile $\rimTrunVP {=} \big( (\RIM_1, \ranking_1^{(t_1, b_1)}), \ldots, \\(\RIM_n, \ranking_n^{(t_n, b_n)}) \big)$, and candidate $w$, determining $w \in \mew(\vr_m, \rimTrunVP)$ is in $O(nm^4)$.
}

\begin{theorem} \label{theorem:tractability_of_rimTrunVP}
    \theoremTractabilityOfRimTrun
\end{theorem}

Mallows combined with partitioned preferences is also tractable.

\def\theoremTractabilityOfMallowsFP{
  Given a positional scoring rule $\vr_m$, a voting profile $\mallowsPartitionVP=\big( (\mallows_1, \fullpar_1), \ldots, \\(\mallows_n, \fullpar_n) \big)$, and candidate $w$, determining $w \in \mew(\vr_m, \mallowsPartitionVP)$ is in $O(nm^4)$.
}

\begin{theorem} \label{theorem:tractability_of_mallowsPartitionVP}
    \theoremTractabilityOfMallowsFP
\end{theorem}


\begin{table}[tb!]
  \caption{
    Complexity of exact \mew solvers under general positional scoring rules for various voting profiles, including partial orders (PO), partially partitioned preferences (PP), fully partitioned preferences (FP), partial chains (PC), RIMs, rRSMs, Mallows, and combined profiles. \rev{A dagger ($\dagger$) refers to results presented in Appendix~\ref{sec:appendix:rsm_profile}.}
    An asterisk (*) means that the RIM and rRSM models are effectively the Mallows models in the experimental benchmarks.
  }
\small 
  \begin{tabular}{@{}llll}
    \toprule
    Profile                           & Complexity            & Solver                                                                                   & Experiments                               \\ \midrule
    $\partialVP$                      & $O(nm^{\text{cw}+3})$ & RIMDP                                                                                    & Figures~\ref{fig:posets} and ~\ref{fig:cw} \\
    $\parparVP$                       & $O(nm^2)$             & Theorem~\ref{theorem:tractability_of_parparVP}                                           & Figure~\ref{fig:pp}                       \\
    $\fullparVP$                      & $O(nm^2)$             & Theorem~\ref{theorem:tractability_of_fullparVP}                                          & Figure~\ref{fig:fp}                       \\
    $\pchainVP$                       & $O(nm^2)$             & Theorem~\ref{theorem:tractability_of_pchainVP}                                           & Figure~\ref{fig:pc}                       \\
    \rev{$\trunVP$}                   & \rev{$O(nm^2)$}       & \rev{Theorem~\ref{theorem:tractability_of_fullparVP}, as $\fullparVP$}                   & Figure~\ref{fig:tr}                       \\ \midrule
    $\rimVP$                          & $O(nm^4)$             & Theorem~\ref{theorem:tractability_of_rimVP}                                              &                                           \\
    \rev{$\mallowsVP$}                & \rev{$O(nm^4)$}       & \rev{Theorem~\ref{theorem:tractability_of_rimVP}, as $\rimVP$}                           & Figure~\ref{fig:mallows}                  \\
    $\rsmVP$                          & $O(nm^4)$             & Theorem~\ref{theorem:tractability_of_rsmVP}$^\dagger$                                    & Figure~\ref{fig:rsm}*                     \\ \midrule
    $\rimPartialVP$                   & $O(nm^{\text{cw}+3})$ & RIMDP                                                                                    & Figure~\ref{fig:mallows|po}               \\
    \rev{$\VP^{\RIM \text{+PP}}$}     & $O(nm^{\text{cw}+3})$ & \rev{RIMDP, as $\rimPartialVP$}                                                          &                                           \\
    \rev{$\rimPChainVP$}              & $O(nm^{\text{cw}+3})$ & \rev{RIMDP, as $\rimPartialVP$}                                                          &                                           \\
    \rev{$\rimPartitionVP$}           & $O(nm^{\text{cw}+3})$ & \rev{RIMDP, as $\rimPartialVP$}                                                          &                                           \\
    $\rimTrunVP$                      & $O(nm^4)$             & Theorem~\ref{theorem:tractability_of_rimTrunVP}                                          & Figure~\ref{fig:mallows|tr}*              \\
    $\mallowsPartitionVP$             & $O(nm^4)$             & Theorem~\ref{theorem:tractability_of_mallowsPartitionVP}                                 & Figure~\ref{fig:mallows|fp}               \\
    \rev{$\VP^{\mallows \text{+TR}}$} & $O(nm^4)$             & \rev{Theorem~\ref{theorem:tractability_of_mallowsPartitionVP}, as $\mallowsPartitionVP$} &                                           \\
    \rev{$\VP^{\mallows \text{+PP}}$} & $O(nm^{\text{cw}+3})$ & \rev{RIMDP, as $\VP^{\RIM \text{+PP}}$}                                                  &                                           \\
    \rev{$\VP^{\mallows \text{+PC}}$} & $O(nm^{\text{cw}+3})$ & \rev{RIMDP, as $\rimPChainVP$}                                                           &                                           \\
    \rev{$\VP^{\mallows \text{+PO}}$} & $O(nm^{\text{cw}+3})$ & \rev{RIMDP, as $\rimPartialVP$}                                                          &                                           \\ \bottomrule
  \end{tabular}
  \centering
  \label{tab:complexity}
\end{table}

\subsection{Summary}
\label{sec:algorithms:summary}

\rev{Table~\ref{tab:complexity} summarizes all solvers in this Section, including additional conclusions for a large number of specialized voting profiles.}
For example, the complexity over Mallows voting profiles is bounded by $O(nm^4)$, since they are a special case of $\rimVP$.
An interesting observation is that the \mew complexity over $\rimPartitionVP$ is not tractable, but its special case $\mallowsPartitionVP$ has polynomial complexity.
Although this table demonstrates that evaluating \mew over probabilistic voting profiles has higher complexity than that over incomplete voting profiles (\eg $O(nm^4)$ for $\mallowsPartitionVP$ but only $O(nm^2)$ for $\fullparVP$), the tractability results over a collection of probabilistic and combined voting profiles give the \mew a computation advantage in practice.
\rev{Note that the combined voting profiles with RSM models are not covered by Table~\ref{tab:complexity}, since the complexity of determining the \mew for them is still an open question.}

\section{Optimization strategies}
\label{sec:optimize}

The algorithms in Section~\ref{sec:algorithms} find the MEW by computing candidate scores for a single voter at a time, and then aggregating the scores across the voting profile.
We now present \rev{two optimization strategies} for a voting profile solver in Algorithm~\ref{alg:pruning}.

\begin{algorithm}[tb!]
  \small 
  \raggedright
  \caption{Solver for voting profiles}
  \label{alg:pruning}
  \textbf{Input}: general voting profile $\probaVP=(\enum[n]{\model})$, voting rule $\vr_m$ \\
  \textbf{Output}: $\mew(\vr_m, \probaVP)$
  
  \begin{algorithmic}[1]
    \STATE \rev{Group identical votes in $\probaVP$, sort them by the group size in descending order, to generate a weighted voting profile $\VP_w$ where each vote $\model_i$ is associated with a weight $w_i$}
    \STATE Obtain candidates $C = \set{\enum{c}}$ from $\VP_w$
    \STATE $\forall c \in C, UB(c) {\defeq} 0, LB(c) {\defeq} 0$ \label{alg:pruning:bounds:1}
    \FOR {\rev{$(\model, w) \in \VP_w$}}
    \FOR {$c \in C$}
    \STATE Compute the best possible rank $a$ of $c$ over $\model$
    \STATE Compute the worst possible rank $b$ of $c$ over $\model$
    \STATE \rev{$UB(c) \pluseq w \cdot \vr_m(a)$, $LB(c) \pluseq w \cdot \vr_m(b)$} \label{alg:pruning:sum}
    \ENDFOR
    \ENDFOR \label{alg:pruning:bounds:2}
    \STATE Compute $max(LB)$, prune any candidate $c$ if $UB(c) < max(LB)$ \label{alg:pruning:prune:1}
    \FOR {\rev{$(\model, w) \in \VP_w$}} \label{alg:pruning:for:1}
    \STATE Let $C_t$ denote the candidates currently tracked by $UB$ and $LB$
    \FOR {$c \in C_t$}
    \STATE Compute the exact score of $c$ over $\model$
    \STATE Refine $UB(c)$ and $LB(c)$ with its exact score \rev{and the weight $w$}
    \ENDFOR
    \STATE Compute $max(LB)$, prune any candidate $c$ if $UB(c) < max(LB)$ \label{alg:pruning:prune:2}
    \IF{Only one candidate remains in $UB$ and $LB$}
    \STATE Let $c'$ denote the only remaining candidate
    \RETURN $\mew(\vr_m, \probaVP) = c'$ \label{alg:pruning:ret:1}
    \ENDIF
    \ENDFOR \label{alg:pruning:for:2}
    \STATE Let $C'$ denote the set of remaining candidates in $UB$ and $LB$
    \RETURN $\mew(\vr_m, \probaVP) = C'$
  \end{algorithmic}
\end{algorithm}

\paragraph{Candidate pruning}
We can save computation by iteratively pruning candidates who cannot be the MEW.
The key idea is to quickly compute score upper- and lower-bounds for all candidates across the voting profile,  progressively refine these scores during per-voter computation, and prune candidates who cannot be the MEW based on the upper- and lower-bounds computed so far.

Algorithm~\ref{alg:pruning} presents the candidate pruning procedure. The first step is to compute $UB(c)$ and $LB(c)$, the upper- and lower-bounds of the score of each candidate $c$ by quickly computing their best possible and worst possible ranks assigned by each voter (lines~\ref{alg:pruning:bounds:1} - \ref{alg:pruning:bounds:2}).  For example, in a given partial order $M$ (voter) in a partial voting profile, the best possible rank of candidate $c$ is bounded by the number of its ancestors in $M$, while its worst possible rank is bounded by the number of its descendants.  Combing this information with the voting rule, we can efficiently obtain score upper- and lower-bounds of each candidate for each voter.

\begin{example}
  Let $\partialOrder=\set{\sigma_3 \succ \sigma_5, \sigma_3 \succ \sigma_8}$ be a partial order, over $m=10$ items, and assume the voting rule is 2-approval.
  Item $\sigma_5$ has 1 ancestor $\sigma_3$ and no descendant in $\partialOrder$.
  So its best possible rank is 2 and its worst possible rank is $m$, which correspond to scores 1 and 0, respectively, under 2-approval.  Thus, 1 is the score upper-bound of $\sigma_5$, and 0 the score lower-bound, for $\partialOrder$.
\end{example}

Per-voter score upper-bounds and lower-bounds can be aggregated (summed up) across all voters in the profile (line~\ref{alg:pruning:sum} of Algorithm~\ref{alg:pruning}).  The pruning strategy can then confidently determine the losing candidates as those whose score upper-bounds are lower than the best observed score lower-bound (line~\ref{alg:pruning:prune:1} of Algorithm~\ref{alg:pruning}).  


Score upper- and lower-bounds are refined as the algorithm iterates over the voters (lines~\ref{alg:pruning:for:1}-~\ref{alg:pruning:for:2}), computing the exact scores of viable (tracked) candidates, refining $UB(c)$ and $LB(c)$, and pruning any candidates that are no longer viable (line~\ref{alg:pruning:prune:2}).

If, at any point during the iteration, only a single tracked candidate $w$ remains, then it is returned as the $\mew$ (line~\ref{alg:pruning:ret:1}).  Once all voters $M$ have been iterated over, return all remaining tracked candidates as the winners.  (Note that if after all voters are iterated, and there are still multiple tracked items, then their upper- and lower-bounds are precisely their exact scores, and they all share the same exact score, otherwise the candidates with lower exact scores would have been pruned during voter iteration.)

\begin{example}
  Assume $UB=\set{\sigma_1 \rightarrow 15, \sigma_2 \rightarrow 6}$ and $LB = \set{\sigma_1 \rightarrow 5, \sigma_2 \rightarrow 2}$ at a time point during refining the $UB$ and $LB$.
  Assume that after computing the exact scores of $\sigma_1$ and $\sigma_2$ over the latest voter, the refined $UB=\set{\sigma_1 \rightarrow 12, \sigma_2 \rightarrow 5}$ and the refined $LB = \set{\sigma_1 \rightarrow 7, \sigma_2 \rightarrow 3}$.
  Now, candidate $\sigma_2$ can be pruned, since $UB(\sigma_2) < LB(\sigma_1)$.
  After pruning $\sigma_2$, only $\sigma_1$ is tracked by $UB$ and $LB$, thus we can confidently declare $\sigma_1$ as the winner.
\end{example}

\paragraph{\rev{Voter grouping}}
\rev{As another optimization, we observe that when multiple voters cast the same vote, we only need to compute the expected candidate scores for this vote once.  When computing the \mew over a voting profile, the first step is to group identical votes and sort them by their frequencies in descending order, which generates a new voting profile that is weighted and sorted.}

In this section we conduct comprehensive experiments to emphasise the effectiveness of DIAL, including evaluations under white-box and black-box settings, robustness to unforeseen adversaries, robustness to unforeseen corruptions, transfer learning, and ablation studies. Finally, we present a new measurement to test the balance between robustness and natural accuracy, which we named $F_1$-robust score. 


\subsection{A case study on SVHN and CIFAR-100}
In the first part of our analysis, we conduct a case study experiment on two benchmark datasets: SVHN \citep{netzer2011reading} and CIFAR-100 \cite{krizhevsky2009learning}. We follow common experiment settings as in \cite{rice2020overfitting, wu2020adversarial}. We used the PreAct ResNet-18 \citep{he2016identity} architecture on which we integrate a domain classification layer. The adversarial training is done using 10-step PGD adversary with perturbation size of 0.031 and a step size of 0.003 for SVHN and 0.007 for CIFAR-100. The batch size is 128, weight decay is $7e^{-4}$ and the model is trained for 100 epochs. For SVHN, the initial learinnig rate is set to 0.01 and decays by a factor of 10 after 55, 75 and 90 iteration. For CIFAR-100, the initial learning rate is set to 0.1 and decays by a factor of 10 after 75 and 90 iterations. 
%We compared DIAL to \cite{madry2017towards} and TRADES \citep{zhang2019theoretically}. 
%The evaluation is done using Auto-Attack~\citep{croce2020reliable}, which is an ensemble of three white-box and one black-box parameter-free attacks, and various $\ell_{\infty}$ adversaries: PGD$^{20}$, PGD$^{100}$, PGD$^{1000}$ and CW$_{\infty}$ with step size of 0.003. 
Results are averaged over 3 restarts while omitting one standard deviation (which is smaller than 0.2\% in all experiments). As can be seen by the results in Tables~\ref{black-and_white-svhn} and \ref{black-and_white-cifar100}, DIAL presents consistent improvement in robustness (e.g., 5.75\% improved robustness on SVHN against AA) compared to the standard AT 
%under variety of attacks 
while also improving the natural accuracy. More results are presented in Appendix \ref{cifar100-svhn-appendix}.


\begin{table}[!ht]
  \caption{Robustness against white-box, black-box attacks and Auto-Attack (AA) on SVHN. Black-box attacks are generated using naturally trained surrogate model. Natural represents the naturally trained (non-adversarial) model.
  %and applied to the best performing robust models.
  }
  \vskip 0.1in
  \label{black-and_white-svhn}
  \centering
  \small
  \begin{tabular}{l@{\hspace{1\tabcolsep}}c@{\hspace{1\tabcolsep}}c@{\hspace{1\tabcolsep}}c@{\hspace{1\tabcolsep}}c@{\hspace{1\tabcolsep}}c@{\hspace{1\tabcolsep}}c@{\hspace{1\tabcolsep}}c@{\hspace{1\tabcolsep}}c@{\hspace{1\tabcolsep}}c@{\hspace{1\tabcolsep}}c}
    \toprule
    & & \multicolumn{4}{c}{White-box} & \multicolumn{4}{c}{Black-Box}  \\
    \cmidrule(r){3-6} 
    \cmidrule(r){7-10}
    Defense Model & Natural & PGD$^{20}$ & PGD$^{100}$  & PGD$^{1000}$  & CW$^{\infty}$ & PGD$^{20}$ & PGD$^{100}$ & PGD$^{1000}$  & CW$^{\infty}$ & AA \\
    \midrule
    NATURAL & 96.85 & 0 & 0 & 0 & 0 & 0 & 0 & 0 & 0 & 0 \\
    \midrule
    AT & 89.90 & 53.23 & 49.45 & 49.23 & 48.25 & 86.44 & 86.28 & 86.18 & 86.42 & 45.25 \\
    % TRADES & 90.35 & 57.10 & 54.13 & 54.08 & 52.19 & 86.89 & 86.73 & 86.57 & 86.70 &  49.50 \\
    $\DIAL_{\kl}$ (Ours) & 90.66 & \textbf{58.91} & \textbf{55.30} & \textbf{55.11} & \textbf{53.67} & 87.62 & 87.52 & 87.41 & 87.63 & \textbf{51.00} \\
    $\DIAL_{\ce}$ (Ours) & \textbf{92.88} & 55.26  & 50.82 & 50.54 & 49.66 & \textbf{89.12} & \textbf{89.01} & \textbf{88.74} & \textbf{89.10} &  46.52  \\
    \bottomrule
  \end{tabular}
\end{table}


\begin{table}[!ht]
  \caption{Robustness against white-box, black-box attacks and Auto-Attack (AA) on CIFAR100. Black-box attacks are generated using naturally trained surrogate model. Natural represents the naturally trained (non-adversarial) model.
  %and applied to the best performing robust models.
  }
  \vskip 0.1in
  \label{black-and_white-cifar100}
  \centering
  \small
  \begin{tabular}{l@{\hspace{1\tabcolsep}}c@{\hspace{1\tabcolsep}}c@{\hspace{1\tabcolsep}}c@{\hspace{1\tabcolsep}}c@{\hspace{1\tabcolsep}}c@{\hspace{1\tabcolsep}}c@{\hspace{1\tabcolsep}}c@{\hspace{1\tabcolsep}}c@{\hspace{1\tabcolsep}}c@{\hspace{1\tabcolsep}}c}
    \toprule
    & & \multicolumn{4}{c}{White-box} & \multicolumn{4}{c}{Black-Box}  \\
    \cmidrule(r){3-6} 
    \cmidrule(r){7-10}
    Defense Model & Natural & PGD$^{20}$ & PGD$^{100}$  & PGD$^{1000}$  & CW$^{\infty}$ & PGD$^{20}$ & PGD$^{100}$ & PGD$^{1000}$  & CW$^{\infty}$ & AA \\
    \midrule
    NATURAL & 79.30 & 0 & 0 & 0 & 0 & 0 & 0 & 0 & 0 & 0 \\
    \midrule
    AT & 56.73 & 29.57 & 28.45 & 28.39 & 26.6 & 55.52 & 55.29 & 55.26 & 55.40 & 24.12 \\
    % TRADES & 58.24 & 30.10 & 29.66 & 29.64 & 25.97 & 57.05 & 56.71 & 56.67 & 56.77 & 24.92 \\
    $\DIAL_{\kl}$ (Ours) & 58.47 & \textbf{31.19} & \textbf{30.50} & \textbf{30.42} & \textbf{26.91} & 57.16 & 56.81 & 56.80 & 57.00 & \textbf{25.87} \\
    $\DIAL_{\ce}$ (Ours) & \textbf{60.77} & 27.87 & 26.66 & 26.61 & 25.98 & \textbf{59.48} & \textbf{59.06} & \textbf{58.96} & \textbf{59.20} & 23.51  \\
    \bottomrule
  \end{tabular}
\end{table}


% \begin{table}[!ht]
%   \caption{Robustness comparison of DIAL to Madry et al. and TRADES defense models on the SVHN dataset under different PGD white-box attacks and the ensemble Auto-Attack (AA).}
%   \label{svhn}
%   \centering
%   \begin{tabular}{llllll|l}
%     \toprule
%     \cmidrule(r){1-5}
%     Defense Model & Natural & PGD$^{20}$ & PGD$^{100}$ & PGD$^{1000}$ & CW$_{\infty}$ & AA\\
%     \midrule
%     $\DIAL_{\kl}$ (Ours) & $\mathbf{90.66}$ & $\mathbf{58.91}$ & $\mathbf{55.30}$ & $\mathbf{55.12}$ & $\mathbf{53.67}$  & $\mathbf{51.00}$  \\
%     Madry et al. & 89.90 & 53.23 & 49.45 & 49.23 & 48.25 & 45.25  \\
%     TRADES & 90.35 & 57.10 & 54.13 & 54.08 & 52.19 & 49.50 \\
%     \bottomrule
%   \end{tabular}
% \end{table}


\subsection{Performance comparison on CIFAR-10} \label{defence-settings}
In this part, we evaluate the performance of DIAL compared to other well-known methods on CIFAR-10. 
%To be consistent with other methods, 
We follow the same experiment setups as in~\cite{madry2017towards, wang2019improving, zhang2019theoretically}. When experiment settings are not identical between tested methods, we choose the most commonly used settings, and apply it to all experiments. This way, we keep the comparison as fair as possible and avoid reporting changes in results which are caused by inconsistent experiment settings \citep{pang2020bag}. To show that our results are not caused because of what is referred to as \textit{obfuscated gradients}~\citep{athalye2018obfuscated}, we evaluate our method with same setup as in our defense model, under strong attacks (e.g., PGD$^{1000}$) in both white-box, black-box settings, Auto-Attack ~\citep{croce2020reliable}, unforeseen "natural" corruptions~\citep{hendrycks2018benchmarking}, and unforeseen adversaries. To make sure that the reported improvements are not caused by \textit{adversarial overfitting}~\citep{rice2020overfitting}, we report best robust results for each method on average of 3 restarts, while omitting one standard deviation (which is smaller than 0.2\% in all experiments). Additional results for CIFAR-10 as well as comprehensive evaluation on MNIST can be found in Appendix \ref{mnist-results} and \ref{additional_res}.
%To further keep the comparison consistent, we followed the same attack settings for all methods.


\begin{table}[ht]
  \caption{Robustness against white-box, black-box attacks and Auto-Attack (AA) on CIFAR-10. Black-box attacks are generated using naturally trained surrogate model. Natural represents the naturally trained (non-adversarial) model.
  %and applied to the best performing robust models.
  }
  \vskip 0.1in
  \label{black-and_white-cifar}
  \centering
  \small
  \begin{tabular}{cccccccc@{\hspace{1\tabcolsep}}c}
    \toprule
    & & \multicolumn{3}{c}{White-box} & \multicolumn{3}{c}{Black-Box} \\
    \cmidrule(r){3-5} 
    \cmidrule(r){6-8}
    Defense Model & Natural & PGD$^{20}$ & PGD$^{100}$ & CW$^{\infty}$ & PGD$^{20}$ & PGD$^{100}$ & CW$^{\infty}$ & AA \\
    \midrule
    NATURAL & 95.43 & 0 & 0 & 0 & 0 & 0 & 0 &  0 \\
    \midrule
    TRADES & 84.92 & 56.60 & 55.56 & 54.20 & 84.08 & 83.89 & 83.91 &  53.08 \\
    MART & 83.62 & 58.12 & 56.48 & 53.09 & 82.82 & 82.52 & 82.80 & 51.10 \\
    AT & 85.10 & 56.28 & 54.46 & 53.99 & 84.22 & 84.14 & 83.92 & 51.52 \\
    ATDA & 76.91 & 43.27 & 41.13 & 41.01 & 75.59 & 75.37 & 75.35 & 40.08\\
    $\DIAL_{\kl}$ (Ours) & 85.25 & $\mathbf{58.43}$ & $\mathbf{56.80}$ & $\mathbf{55.00}$ & 84.30 & 84.18 & 84.05 & \textbf{53.75} \\
    $\DIAL_{\ce}$ (Ours)  & $\mathbf{89.59}$ & 54.31 & 51.67 & 52.04 &$ \mathbf{88.60}$ & $\mathbf{88.39}$ & $\mathbf{88.44}$ & 49.85 \\
    \midrule
    $\DIAL_{\awp}$ (Ours) & $\mathbf{85.91}$ & $\mathbf{61.10}$ & $\mathbf{59.86}$ & $\mathbf{57.67}$ & $\mathbf{85.13}$ & $\mathbf{84.93}$ & $\mathbf{85.03}$  & \textbf{56.78} \\
    $\TRADES_{\awp}$ & 85.36 & 59.27 & 59.12 & 57.07 & 84.58 & 84.58 & 84.59 & 56.17 \\
    \bottomrule
  \end{tabular}
\end{table}



\paragraph{CIFAR-10 setup.} We use the wide residual network (WRN-34-10)~\citep{zagoruyko2016wide} architecture. %used in the experiments of~\cite{madry2017towards, wang2019improving, zhang2019theoretically}. 
Sidelong this architecture, we integrate a domain classification layer. To generate the adversarial domain dataset, we use a perturbation size of $\epsilon=0.031$. We apply 10 of inner maximization iterations with perturbation step size of 0.007. Batch size is set to 128, weight decay is set to $7e^{-4}$, and the model is trained for 100 epochs. Similar to the other methods, the initial learning rate was set to 0.1, and decays by a factor of 10 at iterations 75 and 90. 
%For being consistent with other methods, the natural images are padded with 4-pixel padding with 32-random crop and random horizontal flip. Furthermore, all methods are trained using SGD with momentum 0.9. For $\DIAL_{\kl}$, we balance the robust loss with $\lambda=6$ and the domains loss with $r=4$. For $\DIAL_{\ce}$, we balance the robust loss with $\lambda=1$ and the domains loss with $r=2$. 
%We also introduce a version of our method that incorporates the AWP double-perturbation mechanism, named DIAL-AWP.
%which is trained using the same learning rate schedule used in ~\cite{wu2020adversarial}, where the initial 0.1 learning rate decays by a factor of 10 after 100 and 150 iterations. 
See Appendix \ref{cifar10-additional-setup} for additional details.

\begin{table}[ht]
  \caption{Black-box attack using the adversarially trained surrogate models on CIFAR-10.}
  \vskip 0.1in
  \label{black-box-cifar-adv}
  \centering
  \small
  \begin{tabular}{ll|c}
    \toprule
    \cmidrule(r){1-2}
    Surrogate (source) model & Target model & robustness \% \\
    % \midrule
    \midrule
    TRADES & $\DIAL_{\ce}$ & $\mathbf{67.77}$ \\
    $\DIAL_{\ce}$ & TRADES & 65.75 \\
    \midrule
    MART & $\DIAL_{\ce}$ & $\mathbf{70.30}$ \\
    $\DIAL_{\ce}$ & MART & 64.91 \\
    \midrule
    AT & $\DIAL_{\ce}$ & $\mathbf{65.32}$ \\
    $\DIAL_{\ce}$ & AT  & 63.54 \\
    \midrule
    ATDA & $\DIAL_{\ce}$ & $\mathbf{66.77}$ \\
    $\DIAL_{\ce}$ & ATDA & 52.56 \\
    \bottomrule
  \end{tabular}
\end{table}

\paragraph{White-box/Black-box robustness.} 
%We evaluate all defense models using Auto-Attack, PGD$^{20}$, PGD$^{100}$, PGD$^{1000}$ and CW$_{\infty}$ with step size 0.003. We constrain all attacks by the same perturbation $\epsilon=0.031$. 
As reported in Table~\ref{black-and_white-cifar} and Appendix~\ref{additional_res}, our method achieves better robustness compared to the other methods. Specifically, in the white-box settings, our method improves robustness over~\citet{madry2017towards} and TRADES by 2\% 
%using the common PGD$^{20}$ attack 
while keeping higher natural accuracy. We also observe better natural accuracy of 1.65\% over MART while also achieving better robustness over all attacks. Moreover, our method presents significant improvement of up to 15\% compared to the the domain invariant method suggested by~\citet{song2018improving} (ATDA).
%in both natural and robust accuracy. 
When incorporating 
%the double-perturbation mechanism of 
AWP, our method improves the results of $\TRADES_{\awp}$ by almost 2\%.
%and reaches state-of-the-art results for robust models with no additional data. 
% Additional results are available in Appendix~\ref{additional_res}.
When tested on black-box settings, $\DIAL_{\ce}$ presents a significant improvement of more than 4.4\% over the second-best performing method, and up to 13\%. In Table~\ref{black-box-cifar-adv}, we also present the black-box results when the source model is taken from one of the adversarially trained models. %Then, we compare our model to each one of them both as the source model and target model. 
In addition to the improvement in black-box robustness, $\DIAL_{\ce}$ also manages to achieve better clean accuracy of more than 4.5\% over the second-best performing method.
% Moreover, based on the auto-attack leader-board \footnote{\url{https://github.com/fra31/auto-attack}}, our method achieves the 1st place among models without additional data using the WRN-34-10 architecture.

% \begin{table}
%   \caption{White-box robustness on CIFAR-10 using WRN-34-10}
%   \label{white-box-cifar-10}
%   \centering
%   \begin{tabular}{lllll}
%     \toprule
%     \cmidrule(r){1-2}
%     Defense Model & Natural & PGD$^{20}$ & PGD$^{100}$ & PGD$^{1000}$ \\
%     \midrule
%     TRADES ~\cite{zhang2019theoretically} & 84.92  & 56.6 & 55.56 & 56.43  \\
%     MART ~\cite{wang2019improving} & 83.62  & 58.12 & 56.48 & 56.55  \\
%     Madry et al. ~\cite{madry2017towards} & 85.1  & 56.28 & 54.46 & 54.4  \\
%     Song et al. ~\cite{song2018improving} & 76.91 & 43.27 & 41.13 & 41.02  \\
%     $\DIAL_{\ce}$ (Ours) & $ \mathbf{90}$  & 52.12 & 48.88 & 48.78  \\
%     $\DIAL_{\kl}$ (Ours) & 85.25 & $\mathbf{58.43}$ & $\mathbf{56.8}$ & $\mathbf{56.73}$ \\
%     \midrule
%     $\DIAL_{\kl}$+AWP (Ours) & $\mathbf{85.91}$ & $\mathbf{61.1}$ & - & -  \\
%     TRADES+AWP \cite{wu2020adversarial} & 85.36 & 59.27 & 59.12 & -  \\
%     % MART + AWP & 84.43 & 60.68 & 59.32 & -  \\
%     \bottomrule
%   \end{tabular}
% \end{table}


% \begin{table}
%   \caption{White-box robustness on MNIST}
%   \label{white-box-mnist}
%   \centering
%   \begin{tabular}{llllll}
%     \toprule
%     \cmidrule(r){1-2}
%     Defense Model & Natural & PGD$^{40}$ & PGD$^{100}$ & PGD$^{1000}$ \\
%     \midrule
%     TRADES ~\cite{zhang2019theoretically} & 99.48 & 96.07 & 95.52 & 95.22 \\
%     MART ~\cite{wang2019improving} & 99.38  & 96.99 & 96.11 & 95.74  \\
%     Madry et al. ~\cite{madry2017towards} & 99.41  & 96.01 & 95.49 & 95.36 \\
%     Song et al. ~\cite{song2018improving}  & 98.72 & 96.82 & 96.26 & 96.2  \\
%     $\DIAL_{\kl}$ (Ours) & 99.46 & 97.05 & 96.06 & 95.99  \\
%     $\DIAL_{\ce}$ (Ours) & $\mathbf{99.49}$  & $\mathbf{97.38}$ & $\mathbf{96.45}$ & $\mathbf{96.33}$ \\
%     \bottomrule
%   \end{tabular}
% \end{table}


% \paragraph{Attacking MNIST.} For consistency, we use the same perturbation and step sizes. For MNIST, we use $\epsilon=0.3$ and step size of $0.01$. The natural accuracy of our surrogate (source) model is 99.51\%. Attacks results are reported in Table~\ref{black-and_white-mnist}. It is worth noting that the improvement margin is not conclusive on MNIST as it is on CIFAR-10, which is a more complex task.

% \begin{table}
%   \caption{Black-box robustness on MNIST and CIFAR-10 using naturally trained surrogate model and best performing robust models}
%   \label{black-box-mnist-cifar}
%   \centering
%   \begin{tabular}{lllllll}
%     \toprule
%     & \multicolumn{3}{c}{MNIST} & \multicolumn{3}{c}{CIFAR-10} \\
%     \cmidrule(r){2-4} 
%     \cmidrule(r){5-7}  
%     Defense Model & PGD$^{40}$ & PGD$^{100}$ & PGD$^{1000}$ & PGD$^{20}$ & PGD$^{100}$ & PGD$^{1000}$ \\
%     \midrule
%     TRADES ~\cite{zhang2019theoretically} & 98.12 & 97.86 & 97.81 & 84.08 & 83.89 & 83.8 \\
%     MART ~\cite{wang2019improving} & 98.16 & 97.96 & 97.89  & 82.82 & 82.52 & 82.47 \\
%     Madry et al. ~\cite{madry2017towards}  & 98.05 & 97.73 & 97.78 & 84.22 & 84.14 & 83.96 \\
%     Song et al. ~\cite{song2018improving} & 97.74 & 97.28 & 97.34 & 75.59 & 75.37 & 75.11 \\
%     $\DIAL_{\kl}$ (Ours) & 98.14 & 97.83 & 97.87  & 84.3 & 84.18 & 84.0 \\
%     $\DIAL_{\ce}$ (Ours)  & $\mathbf{98.37}$ & $\mathbf{98.12}$ & $\mathbf{98.05}$  & $\mathbf{89.13}$ & $\mathbf{88.89}$ & $\mathbf{88.78}$ \\
%     \bottomrule
%   \end{tabular}
% \end{table}



% \subsubsection{Ensemble attack} In addition to the white-box and black-box settings, we evaluate our method on the Auto-Attack ~\citep{croce2020reliable} using $\ell_{\infty}$ threat model with perturbation $\epsilon=0.031$. Auto-Attack is an ensemble of parameter-free attacks. It consists of three white-box attacks: APGD-CE which is a step size-free version of PGD on the cross-entropy ~\citep{croce2020reliable}. APGD-DLR which is a step size-free version of PGD on the DLR loss ~\citep{croce2020reliable} and FAB which  minimizes the norm of the adversarial perturbations, and one black-box attack: square attack which is a query-efficient black-box attack ~\citep{andriushchenko2020square}. Results are presented in Table~\ref{auto-attack}. Based on the auto-attack leader-board \footnote{\url{https://github.com/fra31/auto-attack}}, our method achieves the 1st place among models without additional data using the WRN-34-10 architecture.

%Additional results can be found in Appendix ~\ref{additional_res}.

% \begin{table}
%   \caption{Auto-Attack (AA) on CIFAR-10 with perturbation size $\epsilon=0.031$ with $\ell_{\infty}$ threat model}
%   \label{auto-attack}
%   \centering
%   \begin{tabular}{lll}
%     \toprule
%     \cmidrule(r){1-2}
%     Defense Model & AA \\
%     \midrule
%     TRADES ~\cite{zhang2019theoretically} & 53.08  \\
%     MART ~\cite{wang2019improving} & 51.1  \\
%     Madry et al. ~\cite{madry2017towards} & 51.52    \\
%     Song et al.   ~\cite{song2018improving} & 40.18 \\
%     $\DIAL_{\ce}$ (Ours) & 47.33  \\
%     $\DIAL_{\kl}$ (Ours) & $\mathbf{53.75}$ \\
%     \midrule
%     DIAL-AWP (Ours) & $\mathbf{56.78}$ \\
%     TRADES-AWP \cite{wu2020adversarial} & 56.17 \\
%     \bottomrule
%   \end{tabular}
% \end{table}


% \begin{table}[!ht]
%   \caption{Auto-Attack (AA) Robustness (\%) on CIFAR-10 with $\epsilon=0.031$ using an $\ell_{\infty}$ threat model}
%   \label{auto-attack}
%   \centering
%   \begin{tabular}{cccccc|cc}
%     \toprule
%     % \multicolumn{8}{c}{Defence Model}  \\
%     % \cmidrule(r){1-8} 
%     TRADES & MART & Madry & Song & $\DIAL_{\ce}$ & $\DIAL_{\kl}$ & DIAL-AWP  & TRADES-AWP\\
%     \midrule
%     53.08 & 51.10 & 51.52 &  40.08 & 47.33  & $\mathbf{53.75}$ & $\mathbf{56.78}$ & 56.17 \\

%     \bottomrule
%   \end{tabular}
% \end{table}

% \begin{table}[!ht]
% \caption{$F_1$-robust measurement using PGD$^{20}$ attack in white-box and black-box settings on CIFAR-10}
%   \label{f1-robust}
%   \centering
%   \begin{tabular}{ccccccc|cc}
%     \toprule
%     % \multicolumn{8}{c}{Defence Model}  \\
%     % \cmidrule(r){1-8} 
%     Defense Model & TRADES & MART & Madry & Song & $\DIAL_{\kl}$ & $\DIAL_{\ce}$ & DIAL-AWP  & TRADES-AWP\\
%     \midrule
%     White-box & 0.659 & 0.666 & 0.657 & 0.518 & $\mathbf{0.675}$  & 0.643 & $\mathbf{0.698}$ & 0.682 \\
%     Black-box & 0.844 & 0.831 & 0.846 & 0.761 & 0.847 & $\mathbf{0.895}$ & $\mathbf{0.854}$ &  0.849 \\
%     \bottomrule
%   \end{tabular}
% \end{table}

\subsubsection{Robustness to Unforeseen Attacks and Corruptions}
\paragraph{Unforeseen Adversaries.} To further demonstrate the effectiveness of our approach, we test our method against various adversaries that were not used during the training process. We attack the model under the white-box settings with $\ell_{2}$-PGD, $\ell_{1}$-PGD, $\ell_{\infty}$-DeepFool and $\ell_{2}$-DeepFool \citep{moosavi2016deepfool} adversaries using Foolbox \citep{rauber2017foolbox}. We applied commonly used attack budget 
%(perturbation for PGD adversaries and overshot for DeepFool adversaries) 
with 20 and 50 iterations for PGD and DeepFool, respectively.
Results are presented in Table \ref{unseen-attacks}. As can be seen, our approach  gains an improvement of up to 4.73\% over the second best method under the various attack types and an average improvement of 3.7\% over all threat models.


\begin{table}[ht]
  \caption{Robustness on CIFAR-10 against unseen adversaries under white-box settings.}
  \vskip 0.1in
  \label{unseen-attacks}
  \centering
%   \small
  \begin{tabular}{c@{\hspace{1.5\tabcolsep}}c@{\hspace{1.5\tabcolsep}}c@{\hspace{1.5\tabcolsep}}c@{\hspace{1.5\tabcolsep}}c@{\hspace{1.5\tabcolsep}}c@{\hspace{1.5\tabcolsep}}c@{\hspace{1.5\tabcolsep}}c}
    \toprule
    Threat Model & Attack Constraints & $\DIAL_{\kl}$ & $\DIAL_{\ce}$ & AT & TRADES & MART & ATDA \\
    \midrule
    \multirow{2}{*}{$\ell_{2}$-PGD} & $\epsilon=0.5$ & 76.05 & \textbf{80.51} & 76.82 & 76.57 & 75.07 & 66.25 \\
    & $\epsilon=0.25$ & 80.98 & \textbf{85.38} & 81.41 & 81.10 & 80.04 & 71.87 \\\midrule
    \multirow{2}{*}{$\ell_{1}$-PGD} & $\epsilon=12$ & 74.84 & \textbf{80.00} & 76.17 & 75.52 & 75.95 & 65.76 \\
    & $\epsilon=7.84$ & 78.69 & \textbf{83.62} & 79.86 & 79.16 & 78.55 & 69.97 \\
    \midrule
    $\ell_{2}$-DeepFool & overshoot=0.02 & 84.53 & \textbf{88.88} & 84.15 & 84.23 & 82.96 & 76.08 \\\midrule
    $\ell_{\infty}$-DeepFool & overshoot=0.02 & 68.43 & \textbf{69.50} & 67.29 & 67.60 & 66.40 & 57.35 \\
    \bottomrule
  \end{tabular}
\end{table}


%%%%%%%%%%%%%%%%%%%%%%%%% conference version %%%%%%%%%%%%%%%%%%%%%%%%%%%%%%%%%%%%%
\paragraph{Unforeseen Corruptions.}
We further demonstrate that our method consistently holds against unforeseen ``natural'' corruptions, consists of 18 unforeseen diverse corruption types proposed by \citet{hendrycks2018benchmarking} on CIFAR-10, which we refer to as CIFAR10-C. The CIFAR10-C benchmark covers noise, blur, weather, and digital categories. As can be shown in Figure \ref{corruption}, our method gains a significant and consistent improvement over all the other methods. Our method leads to an average improvement of 4.7\% with minimum improvement of 3.5\% and maximum improvement of 5.9\% compared to the second best method over all unforeseen attacks. See Appendix \ref{corruptions-apendix} for the full experiment results.


\begin{figure}[h]
 \centering
  \includegraphics[width=0.4\textwidth]{figures/spider_full.png}
%   \caption{Summary of accuracy over all unforeseen corruptions compared to the second and third best performing methods.}
  \caption{Accuracy comparison over all unforeseen corruptions.}
  \label{corruption}
\end{figure}


%%%%%%%%%%%%%%%%%%%%%%%%% conference version %%%%%%%%%%%%%%%%%%%%%%%%%%%%%%%%%%%%%

%%%%%%%%%%%%%%%%%%%%%%%%% Arxiv version %%%%%%%%%%%%%%%%%%%%%%%%%%%%%%%%%%%%%
% \newpage
% \paragraph{Unforeseen Corruptions.}
% We further demonstrate that our method consistently holds against unforeseen "natural" corruptions, consists of 18 unforeseen diverse corruption types proposed by \cite{hendrycks2018benchmarking} on CIFAR-10, which we refer to as CIFAR10-C. The CIFAR10-C benchmark covers noise, blur, weather, and digital categories. As can be shown in Figure  \ref{spider-full-graph}, our method gains a significant and consistent improvement over all the other methods. Our approach leads to an average improvement of 4.7\% with minimum improvement of 3.5\% and maximum improvement of 5.9\% compared to the second best method over all unforeseen attacks. Full accuracy results against unforeseen corruptions are presented in Tables \ref{corruption-table1} and \ref{corruption-table2}. 

% \begin{table}[!ht]
%   \caption{Accuracy (\%) against unforeseen corruptions.}
%   \label{corruption-table1}
%   \centering
%   \tiny
%   \begin{tabular}{lcccccccccccccccccc}
%     \toprule
%     Defense Model & brightness & defocus blur & fog & glass blur & jpeg compression & motion blur & saturate & snow & speckle noise  \\
%     \midrule
%     TRADES & 82.63 & 80.04 & 60.19 & 78.00 & 82.81 & 76.49 & 81.53 & 80.68 & 80.14 \\
%     MART & 80.76 & 78.62 & 56.78 & 76.60 & 81.26 & 74.58 & 80.74 & 78.22 & 79.42 \\
%     AT &  83.30 & 80.42 & 60.22 & 77.90 & 82.73 & 76.64 & 82.31 & 80.37 & 80.74 \\
%     ATDA & 72.67 & 69.36 & 45.52 & 64.88 & 73.22 & 63.47 & 72.07 & 68.76 & 72.27 \\
%     DIAL (Ours)  & \textbf{87.14} & \textbf{84.84} & \textbf{66.08} & \textbf{81.82} & \textbf{87.07} & \textbf{81.20} & \textbf{86.45} & \textbf{84.18} & \textbf{84.94} \\
%     \bottomrule
%   \end{tabular}
% \end{table}


% \begin{table}[!ht]
%   \caption{Accuracy (\%) against unforeseen corruptions.}
%   \label{corruption-table2}
%   \centering
%   \tiny
%   \begin{tabular}{lcccccccccccccccccc}
%     \toprule
%     Defense Model & contrast & elastic transform & frost & gaussian noise & impulse noise & pixelate & shot noise & spatter & zoom blur \\
%     \midrule
%     TRADES & 43.11 & 79.11 & 76.45 & 79.21 & 73.72 & 82.73 & 80.42 & 80.72 & 78.97 \\
%     MART & 41.22 & 77.77 & 73.07 & 78.30 & 74.97 & 81.31 & 79.53 & 79.28 & 77.8 \\
%     AT & 43.30 & 79.58 & 77.53 & 79.47 & 73.76 & 82.78 & 80.86 & 80.49 & 79.58 \\
%     ATDA & 36.06 & 67.06 & 62.56 & 70.33 & 64.63 & 73.46 & 72.28 & 70.50 & 67.31 \\
%     DIAL (Ours) & \textbf{48.84} & \textbf{84.13} & \textbf{81.76} & \textbf{83.76} & \textbf{78.26} & \textbf{87.24} & \textbf{85.13} & \textbf{84.84} & \textbf{83.93}  \\
%     \bottomrule
%   \end{tabular}
% \end{table}


% \begin{figure}[!ht]
%   \centering
%   \includegraphics[width=9cm]{figures/spider_full.png}
%   \caption{Accuracy comparison with all tested methods over unforeseen corruptions.}
%   \label{spider-full-graph}
% \end{figure}
% %%%%%%%%%%%%%%%%%%%%%%%%% Arxiv version %%%%%%%%%%%%%%%%%%%%%%%%%%%%%%%%%%%%%
%%%%%%%%%%%%%%%%%%%%%%%%% Arxiv version %%%%%%%%%%%%%%%%%%%%%%%%%%%%%%%%%%%%%

\subsubsection{Transfer Learning}
Recent works \citep{salman2020adversarially,utrera2020adversarially} suggested that robust models transfer better on standard downstream classification tasks. In Table \ref{transfer-res} we demonstrate the advantage of our method when applied for transfer learning across CIFAR10 and CIFAR100 using the common linear evaluation protocol. see Appendix \ref{transfer-learning-settings} for detailed settings.

\begin{table}[H]
  \caption{Transfer learning results comparison.}
  \vskip 0.1in
  \label{transfer-res}
  \centering
  \small
\begin{tabular}{c|c|c|c}
\toprule

\multicolumn{2}{l}{} & \multicolumn{2}{c}{Target} \\
\cmidrule(r){3-4}
Source & Defence Model & CIFAR10 & CIFAR100 \\
\midrule
\multirow{3}{*}{CIFAR10} & DIAL & \multirow{3}{*}{-} & \textbf{28.57} \\
 & AT &  & 26.95  \\
 & TRADES &  & 25.40  \\
 \midrule
\multirow{3}{*}{CIFAR100} & DIAL & \textbf{73.68} & \multirow{3}{*}{-} \\
 & AT & 71.41 & \\
 & TRADES & 71.42 &  \\
%  \midrule
% \multirow{3}{}{SVHN} & DIAL &  &  & \multirow{3}{}{-} \\
%  & Madry et al. &  &  &  \\
%  & TRADES &  &  &  \\ 
\bottomrule
\end{tabular}
\end{table}


\subsubsection{Modularity and Ablation Studies}

We note that the domain classifier is a modular component that can be integrated into existing models for further improvements. Removing the domain head and related loss components from the different DIAL formulations results in some common adversarial training techniques. For $\DIAL_{\kl}$, removing the domain and related loss components results in the formulation of TRADES. For $\DIAL_{\ce}$, removing the domain and related loss components results in the original formulation of the standard adversarial training, and for $\DIAL_{\awp}$ the removal results in $\TRADES_{\awp}$. Therefore, the ablation studies will demonstrate the effectiveness of combining DIAL on top of different adversarial training methods. 

We investigate the contribution of the additional domain head component introduced in our method. Experiment configuration are as in \ref{defence-settings}, and robust accuracy is based on white-box PGD$^{20}$ on CIFAR-10 dataset. We remove the domain head from both $\DIAL_{\kl}$, $\DIAL_{\awp}$, and $\DIAL_{\ce}$ (equivalent to $r=0$) and report the natural and robust accuracy. We perform 3 random restarts and omit one standard deviation from the results. Results are presented in Figure \ref{ablation}. All DIAL variants exhibits stable improvements on both natural accuracy and robust accuracy. $\DIAL_{\ce}$, $\DIAL_{\kl}$, and $\DIAL_{\awp}$ present an improvement of 1.82\%, 0.33\%, and 0.55\% on natural accuracy and an improvement of 2.5\%, 1.87\%, and 0.83\% on robust accuracy, respectively. This evaluation empirically demonstrates the benefits of incorporating DIAL on top of different adversarial training techniques.
% \paragraph{semi-supervised extensions.} Since the domain classifier does not require the class labels, we argue that additional unlabeled data can be leveraged in future work.
%for improved results. 

\begin{figure}[ht]
  \centering
  \includegraphics[width=0.35\textwidth]{figures/ablation_graphs3.png}
  \caption{Ablation studies for $\DIAL_{\kl}$, $\DIAL_{\ce}$, and $\DIAL_{\awp}$ on CIFAR-10. Circle represent the robust-natural accuracy without using DIAL, and square represent the robust-natural accuracy when incorporating DIAL.
  %to further investigate the impact of the domain head and loss on natural and robust accuracy.
  }
  \label{ablation}
\end{figure}

\subsubsection{Visualizing DIAL}
To further illustrate the superiority of our method, we visualize the model outputs from the different methods on both natural and adversarial test data.
% adversarial test data generated using PGD$^{20}$ white-box attack with step size 0.003 and $\epsilon=0.031$ on CIFAR-10. 
Figure~\ref{tsne1} shows the embedding received after applying t-SNE ~\citep{van2008visualizing} with two components on the model output for our method and for TRADES. DIAL seems to preserve strong separation between classes on both natural test data and adversarial test data. Additional illustrations for the other methods are attached in Appendix~\ref{additional_viz}. 

\begin{figure}[h]
\centering
  \subfigure[\textbf{DIAL} on natural logits]{\includegraphics[width=0.21\textwidth]{figures/domain_ce_test.png}}
  \hspace{0.03\textwidth}
  \subfigure[\textbf{DIAL} on adversarial logits]{\includegraphics[width=0.21\textwidth]{figures/domain_ce_adversarial.png}}
  \hspace{0.03\textwidth}
    \subfigure[\textbf{TRADES} on natural logits]{\includegraphics[width=0.21\textwidth]{figures/trades_test.png}}
    \hspace{0.03\textwidth}
    \subfigure[\textbf{TRADES} on adversarial logits]{\includegraphics[width=0.21\textwidth]{figures/trades_adversarial.png}}
  \caption{t-SNE embedding of model output (logits) into two-dimensional space for DIAL and TRADES using the CIFAR-10 natural test data and the corresponding PGD$^{20}$ generated adversarial examples.}
  \label{tsne1}
\end{figure}


% \begin{figure}[ht]
% \centering
%   \begin{subfigure}{4cm}
%     \centering\includegraphics[width=3.3cm]{figures/domain_ce_test.png}
%     \caption{\textbf{DIAL} on nat. examples}
%   \end{subfigure}
%   \begin{subfigure}{4cm}
%     \centering\includegraphics[width=3.3cm]{figures/domain_ce_adversarial.png}
%     \caption{\textbf{DIAL} on adv. examples}
%   \end{subfigure}
  
%   \begin{subfigure}{4cm}
%     \centering\includegraphics[width=3.3cm]{figures/trades_test.png}
%     \caption{\textbf{TRADES} on nat. examples}
%   \end{subfigure}
%   \begin{subfigure}{4cm}
%     \centering\includegraphics[width=3.3cm]{figures/trades_adversarial.png}
%     \caption{\textbf{TRADES} on adv. examples}
%   \end{subfigure}
%   \caption{t-SNE embedding of model output (logits) into two-dimensional space for DIAL and TRADES using the CIFAR-10 natural test data and the corresponding adversarial examples.}
%   \label{tsne1}
% \end{figure}



% \begin{figure}[ht]
% \centering
%   \begin{subfigure}{6cm}
%     \centering\includegraphics[width=5cm]{figures/domain_ce_test.png}
%     \caption{\textbf{DIAL} on nat. examples}
%   \end{subfigure}
%   \begin{subfigure}{6cm}
%     \centering\includegraphics[width=5cm]{figures/domain_ce_adversarial.png}
%     \caption{\textbf{DIAL} on adv. examples}
%   \end{subfigure}
  
%   \begin{subfigure}{6cm}
%     \centering\includegraphics[width=5cm]{figures/trades_test.png}
%     \caption{\textbf{TRADES} on nat. examples}
%   \end{subfigure}
%   \begin{subfigure}{6cm}
%     \centering\includegraphics[width=5cm]{figures/trades_adversarial.png}
%     \caption{\textbf{TRADES} on adv. examples}
%   \end{subfigure}
%   \caption{t-SNE embedding of model output (logits) into two-dimensional space for DIAL and TRADES using the CIFAR-10 natural test data and the corresponding adversarial examples.}
%   \label{tsne1}
% \end{figure}



\subsection{Balanced measurement for robust-natural accuracy}
One of the goals of our method is to better balance between robust and natural accuracy under a given model. For a balanced metric, we adopt the idea of $F_1$-score, which is the harmonic mean between the precision and recall. However, rather than using precision and recall, we measure the $F_1$-score between robustness and natural accuracy,
using a measure we call
%We named it
the
\textbf{$\mathbf{F_1}$-robust} score.
\begin{equation}
F_1\text{-robust} = \dfrac{\text{true\_robust}}
{\text{true\_robust}+\frac{1}{2}
%\cdot
(\text{false\_{robust}}+
\text{false\_natural})},
\end{equation}
where $\text{true\_robust}$ are the adversarial examples that were correctly classified, $\text{false\_{robust}}$ are the adversarial examples that were miss-classified, and $\text{false\_natural}$ are the natural examples that were miss-classified.
%We tested the proposed $F_1$-robust score using PGD$^{20}$ on CIFAR-10 dataset in white-box and black-box settings. 
Results are presented in Table~\ref{f1-robust} and demonstrate that our method achieves the best $F_1$-robust score in both settings, which supports our findings from previous sections.

% \begin{table}[!ht]
%   \caption{$F_1$-robust measurement using PGD$^{20}$ attack in white and black box settings on CIFAR-10}
%   \label{f1-robust}
%   \centering
%   \begin{tabular}{lll}
%     \toprule
%     \cmidrule(r){1-2}
%     Defense Model & White-box & Black-box \\
%     \midrule
%     TRADES & 0.65937  & 0.84435 \\
%     MART & 0.66613  & 0.83153  \\
%     Madry et al. & 0.65755 & 0.84574   \\
%     Song et al. & 0.51823 & 0.76092  \\
%     $\DIAL_{\ce}$ (Ours) & 0.65318   & $\mathbf{0.88806}$  \\
%     $\DIAL_{\kl}$ (Ours) & $\mathbf{0.67479}$ & 0.84702 \\
%     \midrule
%     \midrule
%     DIAL-AWP (Ours) & $\mathbf{0.69753}$  & $\mathbf{0.85406}$  \\
%     TRADES-AWP & 0.68162 & 0.84917 \\
%     \bottomrule
%   \end{tabular}
% \end{table}

\begin{table}[ht]
\small
  \caption{$F_1$-robust measurement using PGD$^{20}$ attack in white and black box settings on CIFAR-10.}
  \vskip 0.1in
  \label{f1-robust}
  \centering
%   \small
  \begin{tabular}{c
  @{\hspace{1.5\tabcolsep}}c @{\hspace{1.5\tabcolsep}}c @{\hspace{1.5\tabcolsep}}c @{\hspace{1.5\tabcolsep}}c
  @{\hspace{1.5\tabcolsep}}c @{\hspace{1.5\tabcolsep}}c @{\hspace{1.5\tabcolsep}}|
  @{\hspace{1.5\tabcolsep}}c
  @{\hspace{1.5\tabcolsep}}c}
    \toprule
    % \cmidrule(r){8-9}
     & TRADES & MART & AT & ATDA & $\DIAL_{\ce}$ & $\DIAL_{\kl}$ & $\DIAL_{\awp}$ & $\TRADES_{\awp}$ \\
    \midrule
    White-box & 0.659 & 0.666 & 0.657 & 0.518 & 0.660 & \textbf{0.675} & \textbf{0.698} & 0.682 \\
    Black-box & 0.844 & 0.831 & 0.845 & 0.761 & \textbf{0.890} & 0.847 & \textbf{0.854} & 0.849 \\ 
    \bottomrule
  \end{tabular}
\end{table}

\section{Discussion} \label{sec:discussions}
\subsection{Categorical Attributes}\label{subsec:categoricalAttributes}
\subsection{Comparison of TopDown vs TASKY}

\bibliographystyle{ACM-Reference-Format}
\bibliography{references}

\newpage 
\chapter{Supplementary Material}
\label{appendix}

In this appendix, we present supplementary material for the techniques and
experiments presented in the main text.

\section{Baseline Results and Analysis for Informed Sampler}
\label{appendix:chap3}

Here, we give an in-depth
performance analysis of the various samplers and the effect of their
hyperparameters. We choose hyperparameters with the lowest PSRF value
after $10k$ iterations, for each sampler individually. If the
differences between PSRF are not significantly different among
multiple values, we choose the one that has the highest acceptance
rate.

\subsection{Experiment: Estimating Camera Extrinsics}
\label{appendix:chap3:room}

\subsubsection{Parameter Selection}
\paragraph{Metropolis Hastings (\MH)}

Figure~\ref{fig:exp1_MH} shows the median acceptance rates and PSRF
values corresponding to various proposal standard deviations of plain
\MH~sampling. Mixing gets better and the acceptance rate gets worse as
the standard deviation increases. The value $0.3$ is selected standard
deviation for this sampler.

\paragraph{Metropolis Hastings Within Gibbs (\MHWG)}

As mentioned in Section~\ref{sec:room}, the \MHWG~sampler with one-dimensional
updates did not converge for any value of proposal standard deviation.
This problem has high correlation of the camera parameters and is of
multi-modal nature, which this sampler has problems with.

\paragraph{Parallel Tempering (\PT)}

For \PT~sampling, we took the best performing \MH~sampler and used
different temperature chains to improve the mixing of the
sampler. Figure~\ref{fig:exp1_PT} shows the results corresponding to
different combination of temperature levels. The sampler with
temperature levels of $[1,3,27]$ performed best in terms of both
mixing and acceptance rate.

\paragraph{Effect of Mixture Coefficient in Informed Sampling (\MIXLMH)}

Figure~\ref{fig:exp1_alpha} shows the effect of mixture
coefficient ($\alpha$) on the informed sampling
\MIXLMH. Since there is no significant different in PSRF values for
$0 \le \alpha \le 0.7$, we chose $0.7$ due to its high acceptance
rate.


% \end{multicols}

\begin{figure}[h]
\centering
  \subfigure[MH]{%
    \includegraphics[width=.48\textwidth]{figures/supplementary/camPose_MH.pdf} \label{fig:exp1_MH}
  }
  \subfigure[PT]{%
    \includegraphics[width=.48\textwidth]{figures/supplementary/camPose_PT.pdf} \label{fig:exp1_PT}
  }
\\
  \subfigure[INF-MH]{%
    \includegraphics[width=.48\textwidth]{figures/supplementary/camPose_alpha.pdf} \label{fig:exp1_alpha}
  }
  \mycaption{Results of the `Estimating Camera Extrinsics' experiment}{PRSFs and Acceptance rates corresponding to (a) various standard deviations of \MH, (b) various temperature level combinations of \PT sampling and (c) various mixture coefficients of \MIXLMH sampling.}
\end{figure}



\begin{figure}[!t]
\centering
  \subfigure[\MH]{%
    \includegraphics[width=.48\textwidth]{figures/supplementary/occlusionExp_MH.pdf} \label{fig:exp2_MH}
  }
  \subfigure[\BMHWG]{%
    \includegraphics[width=.48\textwidth]{figures/supplementary/occlusionExp_BMHWG.pdf} \label{fig:exp2_BMHWG}
  }
\\
  \subfigure[\MHWG]{%
    \includegraphics[width=.48\textwidth]{figures/supplementary/occlusionExp_MHWG.pdf} \label{fig:exp2_MHWG}
  }
  \subfigure[\PT]{%
    \includegraphics[width=.48\textwidth]{figures/supplementary/occlusionExp_PT.pdf} \label{fig:exp2_PT}
  }
\\
  \subfigure[\INFBMHWG]{%
    \includegraphics[width=.5\textwidth]{figures/supplementary/occlusionExp_alpha.pdf} \label{fig:exp2_alpha}
  }
  \mycaption{Results of the `Occluding Tiles' experiment}{PRSF and
    Acceptance rates corresponding to various standard deviations of
    (a) \MH, (b) \BMHWG, (c) \MHWG, (d) various temperature level
    combinations of \PT~sampling and; (e) various mixture coefficients
    of our informed \INFBMHWG sampling.}
\end{figure}

%\onecolumn\newpage\twocolumn
\subsection{Experiment: Occluding Tiles}
\label{appendix:chap3:tiles}

\subsubsection{Parameter Selection}

\paragraph{Metropolis Hastings (\MH)}

Figure~\ref{fig:exp2_MH} shows the results of
\MH~sampling. Results show the poor convergence for all proposal
standard deviations and rapid decrease of AR with increasing standard
deviation. This is due to the high-dimensional nature of
the problem. We selected a standard deviation of $1.1$.

\paragraph{Blocked Metropolis Hastings Within Gibbs (\BMHWG)}

The results of \BMHWG are shown in Figure~\ref{fig:exp2_BMHWG}. In
this sampler we update only one block of tile variables (of dimension
four) in each sampling step. Results show much better performance
compared to plain \MH. The optimal proposal standard deviation for
this sampler is $0.7$.

\paragraph{Metropolis Hastings Within Gibbs (\MHWG)}

Figure~\ref{fig:exp2_MHWG} shows the result of \MHWG sampling. This
sampler is better than \BMHWG and converges much more quickly. Here
a standard deviation of $0.9$ is found to be best.

\paragraph{Parallel Tempering (\PT)}

Figure~\ref{fig:exp2_PT} shows the results of \PT sampling with various
temperature combinations. Results show no improvement in AR from plain
\MH sampling and again $[1,3,27]$ temperature levels are found to be optimal.

\paragraph{Effect of Mixture Coefficient in Informed Sampling (\INFBMHWG)}

Figure~\ref{fig:exp2_alpha} shows the effect of mixture
coefficient ($\alpha$) on the blocked informed sampling
\INFBMHWG. Since there is no significant different in PSRF values for
$0 \le \alpha \le 0.8$, we chose $0.8$ due to its high acceptance
rate.



\subsection{Experiment: Estimating Body Shape}
\label{appendix:chap3:body}

\subsubsection{Parameter Selection}
\paragraph{Metropolis Hastings (\MH)}

Figure~\ref{fig:exp3_MH} shows the result of \MH~sampling with various
proposal standard deviations. The value of $0.1$ is found to be
best.

\paragraph{Metropolis Hastings Within Gibbs (\MHWG)}

For \MHWG sampling we select $0.3$ proposal standard
deviation. Results are shown in Fig.~\ref{fig:exp3_MHWG}.


\paragraph{Parallel Tempering (\PT)}

As before, results in Fig.~\ref{fig:exp3_PT}, the temperature levels
were selected to be $[1,3,27]$ due its slightly higher AR.

\paragraph{Effect of Mixture Coefficient in Informed Sampling (\MIXLMH)}

Figure~\ref{fig:exp3_alpha} shows the effect of $\alpha$ on PSRF and
AR. Since there is no significant differences in PSRF values for $0 \le
\alpha \le 0.8$, we choose $0.8$.


\begin{figure}[t]
\centering
  \subfigure[\MH]{%
    \includegraphics[width=.48\textwidth]{figures/supplementary/bodyShape_MH.pdf} \label{fig:exp3_MH}
  }
  \subfigure[\MHWG]{%
    \includegraphics[width=.48\textwidth]{figures/supplementary/bodyShape_MHWG.pdf} \label{fig:exp3_MHWG}
  }
\\
  \subfigure[\PT]{%
    \includegraphics[width=.48\textwidth]{figures/supplementary/bodyShape_PT.pdf} \label{fig:exp3_PT}
  }
  \subfigure[\MIXLMH]{%
    \includegraphics[width=.48\textwidth]{figures/supplementary/bodyShape_alpha.pdf} \label{fig:exp3_alpha}
  }
\\
  \mycaption{Results of the `Body Shape Estimation' experiment}{PRSFs and
    Acceptance rates corresponding to various standard deviations of
    (a) \MH, (b) \MHWG; (c) various temperature level combinations
    of \PT sampling and; (d) various mixture coefficients of the
    informed \MIXLMH sampling.}
\end{figure}


\subsection{Results Overview}
Figure~\ref{fig:exp_summary} shows the summary results of the all the three
experimental studies related to informed sampler.
\begin{figure*}[h!]
\centering
  \subfigure[Results for: Estimating Camera Extrinsics]{%
    \includegraphics[width=0.9\textwidth]{figures/supplementary/camPose_ALL.pdf} \label{fig:exp1_all}
  }
  \subfigure[Results for: Occluding Tiles]{%
    \includegraphics[width=0.9\textwidth]{figures/supplementary/occlusionExp_ALL.pdf} \label{fig:exp2_all}
  }
  \subfigure[Results for: Estimating Body Shape]{%
    \includegraphics[width=0.9\textwidth]{figures/supplementary/bodyShape_ALL.pdf} \label{fig:exp3_all}
  }
  \label{fig:exp_summary}
  \mycaption{Summary of the statistics for the three experiments}{Shown are
    for several baseline methods and the informed samplers the
    acceptance rates (left), PSRFs (middle), and RMSE values
    (right). All results are median results over multiple test
    examples.}
\end{figure*}

\subsection{Additional Qualitative Results}

\subsubsection{Occluding Tiles}
In Figure~\ref{fig:exp2_visual_more} more qualitative results of the
occluding tiles experiment are shown. The informed sampling approach
(\INFBMHWG) is better than the best baseline (\MHWG). This still is a
very challenging problem since the parameters for occluded tiles are
flat over a large region. Some of the posterior variance of the
occluded tiles is already captured by the informed sampler.

\begin{figure*}[h!]
\begin{center}
\centerline{\includegraphics[width=0.95\textwidth]{figures/supplementary/occlusionExp_Visual.pdf}}
\mycaption{Additional qualitative results of the occluding tiles experiment}
  {From left to right: (a)
  Given image, (b) Ground truth tiles, (c) OpenCV heuristic and most probable estimates
  from 5000 samples obtained by (d) MHWG sampler (best baseline) and
  (e) our INF-BMHWG sampler. (f) Posterior expectation of the tiles
  boundaries obtained by INF-BMHWG sampling (First 2000 samples are
  discarded as burn-in).}
\label{fig:exp2_visual_more}
\end{center}
\end{figure*}

\subsubsection{Body Shape}
Figure~\ref{fig:exp3_bodyMeshes} shows some more results of 3D mesh
reconstruction using posterior samples obtained by our informed
sampling \MIXLMH.

\begin{figure*}[t]
\begin{center}
\centerline{\includegraphics[width=0.75\textwidth]{figures/supplementary/bodyMeshResults.pdf}}
\mycaption{Qualitative results for the body shape experiment}
  {Shown is the 3D mesh reconstruction results with first 1000 samples obtained
  using the \MIXLMH informed sampling method. (blue indicates small
  values and red indicates high values)}
\label{fig:exp3_bodyMeshes}
\end{center}
\end{figure*}

\clearpage



\section{Additional Results on the Face Problem with CMP}

Figure~\ref{fig:shading-qualitative-multiple-subjects-supp} shows inference results for reflectance maps, normal maps and lights for randomly chosen test images, and Fig.~\ref{fig:shading-qualitative-same-subject-supp} shows reflectance estimation results on multiple images of the same subject produced under different illumination conditions. CMP is able to produce estimates that are closer to the groundtruth across different subjects and illumination conditions.

\begin{figure*}[h]
  \begin{center}
  \centerline{\includegraphics[width=1.0\columnwidth]{figures/face_cmp_visual_results_supp.pdf}}
  \vspace{-1.2cm}
  \end{center}
	\mycaption{A visual comparison of inference results}{(a)~Observed images. (b)~Inferred reflectance maps. \textit{GT} is the photometric stereo groundtruth, \textit{BU} is the Biswas \etal (2009) reflectance estimate and \textit{Forest} is the consensus prediction. (c)~The variance of the inferred reflectance estimate produced by \MTD (normalized across rows).(d)~Visualization of inferred light directions. (e)~Inferred normal maps.}
	\label{fig:shading-qualitative-multiple-subjects-supp}
\end{figure*}


\begin{figure*}[h]
	\centering
	\setlength\fboxsep{0.2mm}
	\setlength\fboxrule{0pt}
	\begin{tikzpicture}

		\matrix at (0, 0) [matrix of nodes, nodes={anchor=east}, column sep=-0.05cm, row sep=-0.2cm]
		{
			\fbox{\includegraphics[width=1cm]{figures/sample_3_4_X.png}} &
			\fbox{\includegraphics[width=1cm]{figures/sample_3_4_GT.png}} &
			\fbox{\includegraphics[width=1cm]{figures/sample_3_4_BISWAS.png}}  &
			\fbox{\includegraphics[width=1cm]{figures/sample_3_4_VMP.png}}  &
			\fbox{\includegraphics[width=1cm]{figures/sample_3_4_FOREST.png}}  &
			\fbox{\includegraphics[width=1cm]{figures/sample_3_4_CMP.png}}  &
			\fbox{\includegraphics[width=1cm]{figures/sample_3_4_CMPVAR.png}}
			 \\

			\fbox{\includegraphics[width=1cm]{figures/sample_3_5_X.png}} &
			\fbox{\includegraphics[width=1cm]{figures/sample_3_5_GT.png}} &
			\fbox{\includegraphics[width=1cm]{figures/sample_3_5_BISWAS.png}}  &
			\fbox{\includegraphics[width=1cm]{figures/sample_3_5_VMP.png}}  &
			\fbox{\includegraphics[width=1cm]{figures/sample_3_5_FOREST.png}}  &
			\fbox{\includegraphics[width=1cm]{figures/sample_3_5_CMP.png}}  &
			\fbox{\includegraphics[width=1cm]{figures/sample_3_5_CMPVAR.png}}
			 \\

			\fbox{\includegraphics[width=1cm]{figures/sample_3_6_X.png}} &
			\fbox{\includegraphics[width=1cm]{figures/sample_3_6_GT.png}} &
			\fbox{\includegraphics[width=1cm]{figures/sample_3_6_BISWAS.png}}  &
			\fbox{\includegraphics[width=1cm]{figures/sample_3_6_VMP.png}}  &
			\fbox{\includegraphics[width=1cm]{figures/sample_3_6_FOREST.png}}  &
			\fbox{\includegraphics[width=1cm]{figures/sample_3_6_CMP.png}}  &
			\fbox{\includegraphics[width=1cm]{figures/sample_3_6_CMPVAR.png}}
			 \\
	     };

       \node at (-3.85, -2.0) {\small Observed};
       \node at (-2.55, -2.0) {\small `GT'};
       \node at (-1.27, -2.0) {\small BU};
       \node at (0.0, -2.0) {\small MP};
       \node at (1.27, -2.0) {\small Forest};
       \node at (2.55, -2.0) {\small \textbf{CMP}};
       \node at (3.85, -2.0) {\small Variance};

	\end{tikzpicture}
	\mycaption{Robustness to varying illumination}{Reflectance estimation on a subject images with varying illumination. Left to right: observed image, photometric stereo estimate (GT)
  which is used as a proxy for groundtruth, bottom-up estimate of \cite{Biswas2009}, VMP result, consensus forest estimate, CMP mean, and CMP variance.}
	\label{fig:shading-qualitative-same-subject-supp}
\end{figure*}

\clearpage

\section{Additional Material for Learning Sparse High Dimensional Filters}
\label{sec:appendix-bnn}

This part of supplementary material contains a more detailed overview of the permutohedral
lattice convolution in Section~\ref{sec:permconv}, more experiments in
Section~\ref{sec:addexps} and additional results with protocols for
the experiments presented in Chapter~\ref{chap:bnn} in Section~\ref{sec:addresults}.

\vspace{-0.2cm}
\subsection{General Permutohedral Convolutions}
\label{sec:permconv}

A core technical contribution of this work is the generalization of the Gaussian permutohedral lattice
convolution proposed in~\cite{adams2010fast} to the full non-separable case with the
ability to perform back-propagation. Although, conceptually, there are minor
differences between Gaussian and general parameterized filters, there are non-trivial practical
differences in terms of the algorithmic implementation. The Gauss filters belong to
the separable class and can thus be decomposed into multiple
sequential one dimensional convolutions. We are interested in the general filter
convolutions, which can not be decomposed. Thus, performing a general permutohedral
convolution at a lattice point requires the computation of the inner product with the
neighboring elements in all the directions in the high-dimensional space.

Here, we give more details of the implementation differences of separable
and non-separable filters. In the following, we will explain the scalar case first.
Recall, that the forward pass of general permutohedral convolution
involves 3 steps: \textit{splatting}, \textit{convolving} and \textit{slicing}.
We follow the same splatting and slicing strategies as in~\cite{adams2010fast}
since these operations do not depend on the filter kernel. The main difference
between our work and the existing implementation of~\cite{adams2010fast} is
the way that the convolution operation is executed. This proceeds by constructing
a \emph{blur neighbor} matrix $K$ that stores for every lattice point all
values of the lattice neighbors that are needed to compute the filter output.

\begin{figure}[t!]
  \centering
    \includegraphics[width=0.6\columnwidth]{figures/supplementary/lattice_construction}
  \mycaption{Illustration of 1D permutohedral lattice construction}
  {A $4\times 4$ $(x,y)$ grid lattice is projected onto the plane defined by the normal
  vector $(1,1)^{\top}$. This grid has $s+1=4$ and $d=2$ $(s+1)^{d}=4^2=16$ elements.
  In the projection, all points of the same color are projected onto the same points in the plane.
  The number of elements of the projected lattice is $t=(s+1)^d-s^d=4^2-3^2=7$, that is
  the $(4\times 4)$ grid minus the size of lattice that is $1$ smaller at each size, in this
  case a $(3\times 3)$ lattice (the upper right $(3\times 3)$ elements).
  }
\label{fig:latticeconstruction}
\end{figure}

The blur neighbor matrix is constructed by traversing through all the populated
lattice points and their neighboring elements.
% For efficiency, we do this matrix construction recursively with shared computations
% since $n^{th}$ neighbourhood elements are $1^{st}$ neighborhood elements of $n-1^{th}$ neighbourhood elements. \pg{do not understand}
This is done recursively to share computations. For any lattice point, the neighbors that are
$n$ hops away are the direct neighbors of the points that are $n-1$ hops away.
The size of a $d$ dimensional spatial filter with width $s+1$ is $(s+1)^{d}$ (\eg, a
$3\times 3$ filter, $s=2$ in $d=2$ has $3^2=9$ elements) and this size grows
exponentially in the number of dimensions $d$. The permutohedral lattice is constructed by
projecting a regular grid onto the plane spanned by the $d$ dimensional normal vector ${(1,\ldots,1)}^{\top}$. See
Fig.~\ref{fig:latticeconstruction} for an illustration of the 1D lattice construction.
Many corners of a grid filter are projected onto the same point, in total $t = {(s+1)}^{d} -
s^{d}$ elements remain in the permutohedral filter with $s$ neighborhood in $d-1$ dimensions.
If the lattice has $m$ populated elements, the
matrix $K$ has size $t\times m$. Note that, since the input signal is typically
sparse, only a few lattice corners are being populated in the \textit{slicing} step.
We use a hash-table to keep track of these points and traverse only through
the populated lattice points for this neighborhood matrix construction.

Once the blur neighbor matrix $K$ is constructed, we can perform the convolution
by the matrix vector multiplication
\begin{equation}
\ell' = BK,
\label{eq:conv}
\end{equation}
where $B$ is the $1 \times t$ filter kernel (whose values we will learn) and $\ell'\in\mathbb{R}^{1\times m}$
is the result of the filtering at the $m$ lattice points. In practice, we found that the
matrix $K$ is sometimes too large to fit into GPU memory and we divided the matrix $K$
into smaller pieces to compute Eq.~\ref{eq:conv} sequentially.

In the general multi-dimensional case, the signal $\ell$ is of $c$ dimensions. Then
the kernel $B$ is of size $c \times t$ and $K$ stores the $c$ dimensional vectors
accordingly. When the input and output points are different, we slice only the
input points and splat only at the output points.


\subsection{Additional Experiments}
\label{sec:addexps}
In this section, we discuss more use-cases for the learned bilateral filters, one
use-case of BNNs and two single filter applications for image and 3D mesh denoising.

\subsubsection{Recognition of subsampled MNIST}\label{sec:app_mnist}

One of the strengths of the proposed filter convolution is that it does not
require the input to lie on a regular grid. The only requirement is to define a distance
between features of the input signal.
We highlight this feature with the following experiment using the
classical MNIST ten class classification problem~\cite{lecun1998mnist}. We sample a
sparse set of $N$ points $(x,y)\in [0,1]\times [0,1]$
uniformly at random in the input image, use their interpolated values
as signal and the \emph{continuous} $(x,y)$ positions as features. This mimics
sub-sampling of a high-dimensional signal. To compare against a spatial convolution,
we interpolate the sparse set of values at the grid positions.

We take a reference implementation of LeNet~\cite{lecun1998gradient} that
is part of the Caffe project~\cite{jia2014caffe} and compare it
against the same architecture but replacing the first convolutional
layer with a bilateral convolution layer (BCL). The filter size
and numbers are adjusted to get a comparable number of parameters
($5\times 5$ for LeNet, $2$-neighborhood for BCL).

The results are shown in Table~\ref{tab:all-results}. We see that training
on the original MNIST data (column Original, LeNet vs. BNN) leads to a slight
decrease in performance of the BNN (99.03\%) compared to LeNet
(99.19\%). The BNN can be trained and evaluated on sparse
signals, and we resample the image as described above for $N=$ 100\%, 60\% and
20\% of the total number of pixels. The methods are also evaluated
on test images that are subsampled in the same way. Note that we can
train and test with different subsampling rates. We introduce an additional
bilinear interpolation layer for the LeNet architecture to train on the same
data. In essence, both models perform a spatial interpolation and thus we
expect them to yield a similar classification accuracy. Once the data is of
higher dimensions, the permutohedral convolution will be faster due to hashing
the sparse input points, as well as less memory demanding in comparison to
naive application of a spatial convolution with interpolated values.

\begin{table}[t]
  \begin{center}
    \footnotesize
    \centering
    \begin{tabular}[t]{lllll}
      \toprule
              &     & \multicolumn{3}{c}{Test Subsampling} \\
       Method  & Original & 100\% & 60\% & 20\%\\
      \midrule
       LeNet &  \textbf{0.9919} & 0.9660 & 0.9348 & \textbf{0.6434} \\
       BNN &  0.9903 & \textbf{0.9844} & \textbf{0.9534} & 0.5767 \\
      \hline
       LeNet 100\% & 0.9856 & 0.9809 & 0.9678 & \textbf{0.7386} \\
       BNN 100\% & \textbf{0.9900} & \textbf{0.9863} & \textbf{0.9699} & 0.6910 \\
      \hline
       LeNet 60\% & 0.9848 & 0.9821 & 0.9740 & 0.8151 \\
       BNN 60\% & \textbf{0.9885} & \textbf{0.9864} & \textbf{0.9771} & \textbf{0.8214}\\
      \hline
       LeNet 20\% & \textbf{0.9763} & \textbf{0.9754} & 0.9695 & 0.8928 \\
       BNN 20\% & 0.9728 & 0.9735 & \textbf{0.9701} & \textbf{0.9042}\\
      \bottomrule
    \end{tabular}
  \end{center}
\vspace{-.2cm}
\caption{Classification accuracy on MNIST. We compare the
    LeNet~\cite{lecun1998gradient} implementation that is part of
    Caffe~\cite{jia2014caffe} to the network with the first layer
    replaced by a bilateral convolution layer (BCL). Both are trained
    on the original image resolution (first two rows). Three more BNN
    and CNN models are trained with randomly subsampled images (100\%,
    60\% and 20\% of the pixels). An additional bilinear interpolation
    layer samples the input signal on a spatial grid for the CNN model.
  }
  \label{tab:all-results}
\vspace{-.5cm}
\end{table}

\subsubsection{Image Denoising}

The main application that inspired the development of the bilateral
filtering operation is image denoising~\cite{aurich1995non}, there
using a single Gaussian kernel. Our development allows to learn this
kernel function from data and we explore how to improve using a \emph{single}
but more general bilateral filter.

We use the Berkeley segmentation dataset
(BSDS500)~\cite{arbelaezi2011bsds500} as a test bed. The color
images in the dataset are converted to gray-scale,
and corrupted with Gaussian noise with a standard deviation of
$\frac {25} {255}$.

We compare the performance of four different filter models on a
denoising task.
The first baseline model (`Spatial' in Table \ref{tab:denoising}, $25$
weights) uses a single spatial filter with a kernel size of
$5$ and predicts the scalar gray-scale value at the center pixel. The next model
(`Gauss Bilateral') applies a bilateral \emph{Gaussian}
filter to the noisy input, using position and intensity features $\f=(x,y,v)^\top$.
The third setup (`Learned Bilateral', $65$ weights)
takes a Gauss kernel as initialization and
fits all filter weights on the train set to minimize the
mean squared error with respect to the clean images.
We run a combination
of spatial and permutohedral convolutions on spatial and bilateral
features (`Spatial + Bilateral (Learned)') to check for a complementary
performance of the two convolutions.

\label{sec:exp:denoising}
\begin{table}[!h]
\begin{center}
  \footnotesize
  \begin{tabular}[t]{lr}
    \toprule
    Method & PSNR \\
    \midrule
    Noisy Input & $20.17$ \\
    Spatial & $26.27$ \\
    Gauss Bilateral & $26.51$ \\
    Learned Bilateral & $26.58$ \\
    Spatial + Bilateral (Learned) & \textbf{$26.65$} \\
    \bottomrule
  \end{tabular}
\end{center}
\vspace{-0.5em}
\caption{PSNR results of a denoising task using the BSDS500
  dataset~\cite{arbelaezi2011bsds500}}
\vspace{-0.5em}
\label{tab:denoising}
\end{table}
\vspace{-0.2em}

The PSNR scores evaluated on full images of the test set are
shown in Table \ref{tab:denoising}. We find that an untrained bilateral
filter already performs better than a trained spatial convolution
($26.27$ to $26.51$). A learned convolution further improve the
performance slightly. We chose this simple one-kernel setup to
validate an advantage of the generalized bilateral filter. A competitive
denoising system would employ RGB color information and also
needs to be properly adjusted in network size. Multi-layer perceptrons
have obtained state-of-the-art denoising results~\cite{burger12cvpr}
and the permutohedral lattice layer can readily be used in such an
architecture, which is intended future work.

\subsection{Additional results}
\label{sec:addresults}

This section contains more qualitative results for the experiments presented in Chapter~\ref{chap:bnn}.

\begin{figure*}[th!]
  \centering
    \includegraphics[width=\columnwidth,trim={5cm 2.5cm 5cm 4.5cm},clip]{figures/supplementary/lattice_viz.pdf}
    \vspace{-0.7cm}
  \mycaption{Visualization of the Permutohedral Lattice}
  {Sample lattice visualizations for different feature spaces. All pixels falling in the same simplex cell are shown with
  the same color. $(x,y)$ features correspond to image pixel positions, and $(r,g,b) \in [0,255]$ correspond
  to the red, green and blue color values.}
\label{fig:latticeviz}
\end{figure*}

\subsubsection{Lattice Visualization}

Figure~\ref{fig:latticeviz} shows sample lattice visualizations for different feature spaces.

\newcolumntype{L}[1]{>{\raggedright\let\newline\\\arraybackslash\hspace{0pt}}b{#1}}
\newcolumntype{C}[1]{>{\centering\let\newline\\\arraybackslash\hspace{0pt}}b{#1}}
\newcolumntype{R}[1]{>{\raggedleft\let\newline\\\arraybackslash\hspace{0pt}}b{#1}}

\subsubsection{Color Upsampling}\label{sec:color_upsampling}
\label{sec:col_upsample_extra}

Some images of the upsampling for the Pascal
VOC12 dataset are shown in Fig.~\ref{fig:Colour_upsample_visuals}. It is
especially the low level image details that are better preserved with
a learned bilateral filter compared to the Gaussian case.

\begin{figure*}[t!]
  \centering
    \subfigure{%
   \raisebox{2.0em}{
    \includegraphics[width=.06\columnwidth]{figures/supplementary/2007_004969.jpg}
   }
  }
  \subfigure{%
    \includegraphics[width=.17\columnwidth]{figures/supplementary/2007_004969_gray.pdf}
  }
  \subfigure{%
    \includegraphics[width=.17\columnwidth]{figures/supplementary/2007_004969_gt.pdf}
  }
  \subfigure{%
    \includegraphics[width=.17\columnwidth]{figures/supplementary/2007_004969_bicubic.pdf}
  }
  \subfigure{%
    \includegraphics[width=.17\columnwidth]{figures/supplementary/2007_004969_gauss.pdf}
  }
  \subfigure{%
    \includegraphics[width=.17\columnwidth]{figures/supplementary/2007_004969_learnt.pdf}
  }\\
    \subfigure{%
   \raisebox{2.0em}{
    \includegraphics[width=.06\columnwidth]{figures/supplementary/2007_003106.jpg}
   }
  }
  \subfigure{%
    \includegraphics[width=.17\columnwidth]{figures/supplementary/2007_003106_gray.pdf}
  }
  \subfigure{%
    \includegraphics[width=.17\columnwidth]{figures/supplementary/2007_003106_gt.pdf}
  }
  \subfigure{%
    \includegraphics[width=.17\columnwidth]{figures/supplementary/2007_003106_bicubic.pdf}
  }
  \subfigure{%
    \includegraphics[width=.17\columnwidth]{figures/supplementary/2007_003106_gauss.pdf}
  }
  \subfigure{%
    \includegraphics[width=.17\columnwidth]{figures/supplementary/2007_003106_learnt.pdf}
  }\\
  \setcounter{subfigure}{0}
  \small{
  \subfigure[Inp.]{%
  \raisebox{2.0em}{
    \includegraphics[width=.06\columnwidth]{figures/supplementary/2007_006837.jpg}
   }
  }
  \subfigure[Guidance]{%
    \includegraphics[width=.17\columnwidth]{figures/supplementary/2007_006837_gray.pdf}
  }
   \subfigure[GT]{%
    \includegraphics[width=.17\columnwidth]{figures/supplementary/2007_006837_gt.pdf}
  }
  \subfigure[Bicubic]{%
    \includegraphics[width=.17\columnwidth]{figures/supplementary/2007_006837_bicubic.pdf}
  }
  \subfigure[Gauss-BF]{%
    \includegraphics[width=.17\columnwidth]{figures/supplementary/2007_006837_gauss.pdf}
  }
  \subfigure[Learned-BF]{%
    \includegraphics[width=.17\columnwidth]{figures/supplementary/2007_006837_learnt.pdf}
  }
  }
  \vspace{-0.5cm}
  \mycaption{Color Upsampling}{Color $8\times$ upsampling results
  using different methods, from left to right, (a)~Low-resolution input color image (Inp.),
  (b)~Gray scale guidance image, (c)~Ground-truth color image; Upsampled color images with
  (d)~Bicubic interpolation, (e) Gauss bilateral upsampling and, (f)~Learned bilateral
  updampgling (best viewed on screen).}

\label{fig:Colour_upsample_visuals}
\end{figure*}

\subsubsection{Depth Upsampling}
\label{sec:depth_upsample_extra}

Figure~\ref{fig:depth_upsample_visuals} presents some more qualitative results comparing bicubic interpolation, Gauss
bilateral and learned bilateral upsampling on NYU depth dataset image~\cite{silberman2012indoor}.

\subsubsection{Character Recognition}\label{sec:app_character}

 Figure~\ref{fig:nnrecognition} shows the schematic of different layers
 of the network architecture for LeNet-7~\cite{lecun1998mnist}
 and DeepCNet(5, 50)~\cite{ciresan2012multi,graham2014spatially}. For the BNN variants, the first layer filters are replaced
 with learned bilateral filters and are learned end-to-end.

\subsubsection{Semantic Segmentation}\label{sec:app_semantic_segmentation}
\label{sec:semantic_bnn_extra}

Some more visual results for semantic segmentation are shown in Figure~\ref{fig:semantic_visuals}.
These include the underlying DeepLab CNN\cite{chen2014semantic} result (DeepLab),
the 2 step mean-field result with Gaussian edge potentials (+2stepMF-GaussCRF)
and also corresponding results with learned edge potentials (+2stepMF-LearnedCRF).
In general, we observe that mean-field in learned CRF leads to slightly dilated
classification regions in comparison to using Gaussian CRF thereby filling-in the
false negative pixels and also correcting some mis-classified regions.

\begin{figure*}[t!]
  \centering
    \subfigure{%
   \raisebox{2.0em}{
    \includegraphics[width=.06\columnwidth]{figures/supplementary/2bicubic}
   }
  }
  \subfigure{%
    \includegraphics[width=.17\columnwidth]{figures/supplementary/2given_image}
  }
  \subfigure{%
    \includegraphics[width=.17\columnwidth]{figures/supplementary/2ground_truth}
  }
  \subfigure{%
    \includegraphics[width=.17\columnwidth]{figures/supplementary/2bicubic}
  }
  \subfigure{%
    \includegraphics[width=.17\columnwidth]{figures/supplementary/2gauss}
  }
  \subfigure{%
    \includegraphics[width=.17\columnwidth]{figures/supplementary/2learnt}
  }\\
    \subfigure{%
   \raisebox{2.0em}{
    \includegraphics[width=.06\columnwidth]{figures/supplementary/32bicubic}
   }
  }
  \subfigure{%
    \includegraphics[width=.17\columnwidth]{figures/supplementary/32given_image}
  }
  \subfigure{%
    \includegraphics[width=.17\columnwidth]{figures/supplementary/32ground_truth}
  }
  \subfigure{%
    \includegraphics[width=.17\columnwidth]{figures/supplementary/32bicubic}
  }
  \subfigure{%
    \includegraphics[width=.17\columnwidth]{figures/supplementary/32gauss}
  }
  \subfigure{%
    \includegraphics[width=.17\columnwidth]{figures/supplementary/32learnt}
  }\\
  \setcounter{subfigure}{0}
  \small{
  \subfigure[Inp.]{%
  \raisebox{2.0em}{
    \includegraphics[width=.06\columnwidth]{figures/supplementary/41bicubic}
   }
  }
  \subfigure[Guidance]{%
    \includegraphics[width=.17\columnwidth]{figures/supplementary/41given_image}
  }
   \subfigure[GT]{%
    \includegraphics[width=.17\columnwidth]{figures/supplementary/41ground_truth}
  }
  \subfigure[Bicubic]{%
    \includegraphics[width=.17\columnwidth]{figures/supplementary/41bicubic}
  }
  \subfigure[Gauss-BF]{%
    \includegraphics[width=.17\columnwidth]{figures/supplementary/41gauss}
  }
  \subfigure[Learned-BF]{%
    \includegraphics[width=.17\columnwidth]{figures/supplementary/41learnt}
  }
  }
  \mycaption{Depth Upsampling}{Depth $8\times$ upsampling results
  using different upsampling strategies, from left to right,
  (a)~Low-resolution input depth image (Inp.),
  (b)~High-resolution guidance image, (c)~Ground-truth depth; Upsampled depth images with
  (d)~Bicubic interpolation, (e) Gauss bilateral upsampling and, (f)~Learned bilateral
  updampgling (best viewed on screen).}

\label{fig:depth_upsample_visuals}
\end{figure*}

\subsubsection{Material Segmentation}\label{sec:app_material_segmentation}
\label{sec:material_bnn_extra}

In Fig.~\ref{fig:material_visuals-app2}, we present visual results comparing 2 step
mean-field inference with Gaussian and learned pairwise CRF potentials. In
general, we observe that the pixels belonging to dominant classes in the
training data are being more accurately classified with learned CRF. This leads to
a significant improvements in overall pixel accuracy. This also results
in a slight decrease of the accuracy from less frequent class pixels thereby
slightly reducing the average class accuracy with learning. We attribute this
to the type of annotation that is available for this dataset, which is not
for the entire image but for some segments in the image. We have very few
images of the infrequent classes to combat this behaviour during training.

\subsubsection{Experiment Protocols}
\label{sec:protocols}

Table~\ref{tbl:parameters} shows experiment protocols of different experiments.

 \begin{figure*}[t!]
  \centering
  \subfigure[LeNet-7]{
    \includegraphics[width=0.7\columnwidth]{figures/supplementary/lenet_cnn_network}
    }\\
    \subfigure[DeepCNet]{
    \includegraphics[width=\columnwidth]{figures/supplementary/deepcnet_cnn_network}
    }
  \mycaption{CNNs for Character Recognition}
  {Schematic of (top) LeNet-7~\cite{lecun1998mnist} and (bottom) DeepCNet(5,50)~\cite{ciresan2012multi,graham2014spatially} architectures used in Assamese
  character recognition experiments.}
\label{fig:nnrecognition}
\end{figure*}

\definecolor{voc_1}{RGB}{0, 0, 0}
\definecolor{voc_2}{RGB}{128, 0, 0}
\definecolor{voc_3}{RGB}{0, 128, 0}
\definecolor{voc_4}{RGB}{128, 128, 0}
\definecolor{voc_5}{RGB}{0, 0, 128}
\definecolor{voc_6}{RGB}{128, 0, 128}
\definecolor{voc_7}{RGB}{0, 128, 128}
\definecolor{voc_8}{RGB}{128, 128, 128}
\definecolor{voc_9}{RGB}{64, 0, 0}
\definecolor{voc_10}{RGB}{192, 0, 0}
\definecolor{voc_11}{RGB}{64, 128, 0}
\definecolor{voc_12}{RGB}{192, 128, 0}
\definecolor{voc_13}{RGB}{64, 0, 128}
\definecolor{voc_14}{RGB}{192, 0, 128}
\definecolor{voc_15}{RGB}{64, 128, 128}
\definecolor{voc_16}{RGB}{192, 128, 128}
\definecolor{voc_17}{RGB}{0, 64, 0}
\definecolor{voc_18}{RGB}{128, 64, 0}
\definecolor{voc_19}{RGB}{0, 192, 0}
\definecolor{voc_20}{RGB}{128, 192, 0}
\definecolor{voc_21}{RGB}{0, 64, 128}
\definecolor{voc_22}{RGB}{128, 64, 128}

\begin{figure*}[t]
  \centering
  \small{
  \fcolorbox{white}{voc_1}{\rule{0pt}{6pt}\rule{6pt}{0pt}} Background~~
  \fcolorbox{white}{voc_2}{\rule{0pt}{6pt}\rule{6pt}{0pt}} Aeroplane~~
  \fcolorbox{white}{voc_3}{\rule{0pt}{6pt}\rule{6pt}{0pt}} Bicycle~~
  \fcolorbox{white}{voc_4}{\rule{0pt}{6pt}\rule{6pt}{0pt}} Bird~~
  \fcolorbox{white}{voc_5}{\rule{0pt}{6pt}\rule{6pt}{0pt}} Boat~~
  \fcolorbox{white}{voc_6}{\rule{0pt}{6pt}\rule{6pt}{0pt}} Bottle~~
  \fcolorbox{white}{voc_7}{\rule{0pt}{6pt}\rule{6pt}{0pt}} Bus~~
  \fcolorbox{white}{voc_8}{\rule{0pt}{6pt}\rule{6pt}{0pt}} Car~~ \\
  \fcolorbox{white}{voc_9}{\rule{0pt}{6pt}\rule{6pt}{0pt}} Cat~~
  \fcolorbox{white}{voc_10}{\rule{0pt}{6pt}\rule{6pt}{0pt}} Chair~~
  \fcolorbox{white}{voc_11}{\rule{0pt}{6pt}\rule{6pt}{0pt}} Cow~~
  \fcolorbox{white}{voc_12}{\rule{0pt}{6pt}\rule{6pt}{0pt}} Dining Table~~
  \fcolorbox{white}{voc_13}{\rule{0pt}{6pt}\rule{6pt}{0pt}} Dog~~
  \fcolorbox{white}{voc_14}{\rule{0pt}{6pt}\rule{6pt}{0pt}} Horse~~
  \fcolorbox{white}{voc_15}{\rule{0pt}{6pt}\rule{6pt}{0pt}} Motorbike~~
  \fcolorbox{white}{voc_16}{\rule{0pt}{6pt}\rule{6pt}{0pt}} Person~~ \\
  \fcolorbox{white}{voc_17}{\rule{0pt}{6pt}\rule{6pt}{0pt}} Potted Plant~~
  \fcolorbox{white}{voc_18}{\rule{0pt}{6pt}\rule{6pt}{0pt}} Sheep~~
  \fcolorbox{white}{voc_19}{\rule{0pt}{6pt}\rule{6pt}{0pt}} Sofa~~
  \fcolorbox{white}{voc_20}{\rule{0pt}{6pt}\rule{6pt}{0pt}} Train~~
  \fcolorbox{white}{voc_21}{\rule{0pt}{6pt}\rule{6pt}{0pt}} TV monitor~~ \\
  }
  \subfigure{%
    \includegraphics[width=.18\columnwidth]{figures/supplementary/2007_001423_given.jpg}
  }
  \subfigure{%
    \includegraphics[width=.18\columnwidth]{figures/supplementary/2007_001423_gt.png}
  }
  \subfigure{%
    \includegraphics[width=.18\columnwidth]{figures/supplementary/2007_001423_cnn.png}
  }
  \subfigure{%
    \includegraphics[width=.18\columnwidth]{figures/supplementary/2007_001423_gauss.png}
  }
  \subfigure{%
    \includegraphics[width=.18\columnwidth]{figures/supplementary/2007_001423_learnt.png}
  }\\
  \subfigure{%
    \includegraphics[width=.18\columnwidth]{figures/supplementary/2007_001430_given.jpg}
  }
  \subfigure{%
    \includegraphics[width=.18\columnwidth]{figures/supplementary/2007_001430_gt.png}
  }
  \subfigure{%
    \includegraphics[width=.18\columnwidth]{figures/supplementary/2007_001430_cnn.png}
  }
  \subfigure{%
    \includegraphics[width=.18\columnwidth]{figures/supplementary/2007_001430_gauss.png}
  }
  \subfigure{%
    \includegraphics[width=.18\columnwidth]{figures/supplementary/2007_001430_learnt.png}
  }\\
    \subfigure{%
    \includegraphics[width=.18\columnwidth]{figures/supplementary/2007_007996_given.jpg}
  }
  \subfigure{%
    \includegraphics[width=.18\columnwidth]{figures/supplementary/2007_007996_gt.png}
  }
  \subfigure{%
    \includegraphics[width=.18\columnwidth]{figures/supplementary/2007_007996_cnn.png}
  }
  \subfigure{%
    \includegraphics[width=.18\columnwidth]{figures/supplementary/2007_007996_gauss.png}
  }
  \subfigure{%
    \includegraphics[width=.18\columnwidth]{figures/supplementary/2007_007996_learnt.png}
  }\\
   \subfigure{%
    \includegraphics[width=.18\columnwidth]{figures/supplementary/2010_002682_given.jpg}
  }
  \subfigure{%
    \includegraphics[width=.18\columnwidth]{figures/supplementary/2010_002682_gt.png}
  }
  \subfigure{%
    \includegraphics[width=.18\columnwidth]{figures/supplementary/2010_002682_cnn.png}
  }
  \subfigure{%
    \includegraphics[width=.18\columnwidth]{figures/supplementary/2010_002682_gauss.png}
  }
  \subfigure{%
    \includegraphics[width=.18\columnwidth]{figures/supplementary/2010_002682_learnt.png}
  }\\
     \subfigure{%
    \includegraphics[width=.18\columnwidth]{figures/supplementary/2010_004789_given.jpg}
  }
  \subfigure{%
    \includegraphics[width=.18\columnwidth]{figures/supplementary/2010_004789_gt.png}
  }
  \subfigure{%
    \includegraphics[width=.18\columnwidth]{figures/supplementary/2010_004789_cnn.png}
  }
  \subfigure{%
    \includegraphics[width=.18\columnwidth]{figures/supplementary/2010_004789_gauss.png}
  }
  \subfigure{%
    \includegraphics[width=.18\columnwidth]{figures/supplementary/2010_004789_learnt.png}
  }\\
       \subfigure{%
    \includegraphics[width=.18\columnwidth]{figures/supplementary/2007_001311_given.jpg}
  }
  \subfigure{%
    \includegraphics[width=.18\columnwidth]{figures/supplementary/2007_001311_gt.png}
  }
  \subfigure{%
    \includegraphics[width=.18\columnwidth]{figures/supplementary/2007_001311_cnn.png}
  }
  \subfigure{%
    \includegraphics[width=.18\columnwidth]{figures/supplementary/2007_001311_gauss.png}
  }
  \subfigure{%
    \includegraphics[width=.18\columnwidth]{figures/supplementary/2007_001311_learnt.png}
  }\\
  \setcounter{subfigure}{0}
  \subfigure[Input]{%
    \includegraphics[width=.18\columnwidth]{figures/supplementary/2010_003531_given.jpg}
  }
  \subfigure[Ground Truth]{%
    \includegraphics[width=.18\columnwidth]{figures/supplementary/2010_003531_gt.png}
  }
  \subfigure[DeepLab]{%
    \includegraphics[width=.18\columnwidth]{figures/supplementary/2010_003531_cnn.png}
  }
  \subfigure[+GaussCRF]{%
    \includegraphics[width=.18\columnwidth]{figures/supplementary/2010_003531_gauss.png}
  }
  \subfigure[+LearnedCRF]{%
    \includegraphics[width=.18\columnwidth]{figures/supplementary/2010_003531_learnt.png}
  }
  \vspace{-0.3cm}
  \mycaption{Semantic Segmentation}{Example results of semantic segmentation.
  (c)~depicts the unary results before application of MF, (d)~after two steps of MF with Gaussian edge CRF potentials, (e)~after
  two steps of MF with learned edge CRF potentials.}
    \label{fig:semantic_visuals}
\end{figure*}


\definecolor{minc_1}{HTML}{771111}
\definecolor{minc_2}{HTML}{CAC690}
\definecolor{minc_3}{HTML}{EEEEEE}
\definecolor{minc_4}{HTML}{7C8FA6}
\definecolor{minc_5}{HTML}{597D31}
\definecolor{minc_6}{HTML}{104410}
\definecolor{minc_7}{HTML}{BB819C}
\definecolor{minc_8}{HTML}{D0CE48}
\definecolor{minc_9}{HTML}{622745}
\definecolor{minc_10}{HTML}{666666}
\definecolor{minc_11}{HTML}{D54A31}
\definecolor{minc_12}{HTML}{101044}
\definecolor{minc_13}{HTML}{444126}
\definecolor{minc_14}{HTML}{75D646}
\definecolor{minc_15}{HTML}{DD4348}
\definecolor{minc_16}{HTML}{5C8577}
\definecolor{minc_17}{HTML}{C78472}
\definecolor{minc_18}{HTML}{75D6D0}
\definecolor{minc_19}{HTML}{5B4586}
\definecolor{minc_20}{HTML}{C04393}
\definecolor{minc_21}{HTML}{D69948}
\definecolor{minc_22}{HTML}{7370D8}
\definecolor{minc_23}{HTML}{7A3622}
\definecolor{minc_24}{HTML}{000000}

\begin{figure*}[t]
  \centering
  \small{
  \fcolorbox{white}{minc_1}{\rule{0pt}{6pt}\rule{6pt}{0pt}} Brick~~
  \fcolorbox{white}{minc_2}{\rule{0pt}{6pt}\rule{6pt}{0pt}} Carpet~~
  \fcolorbox{white}{minc_3}{\rule{0pt}{6pt}\rule{6pt}{0pt}} Ceramic~~
  \fcolorbox{white}{minc_4}{\rule{0pt}{6pt}\rule{6pt}{0pt}} Fabric~~
  \fcolorbox{white}{minc_5}{\rule{0pt}{6pt}\rule{6pt}{0pt}} Foliage~~
  \fcolorbox{white}{minc_6}{\rule{0pt}{6pt}\rule{6pt}{0pt}} Food~~
  \fcolorbox{white}{minc_7}{\rule{0pt}{6pt}\rule{6pt}{0pt}} Glass~~
  \fcolorbox{white}{minc_8}{\rule{0pt}{6pt}\rule{6pt}{0pt}} Hair~~ \\
  \fcolorbox{white}{minc_9}{\rule{0pt}{6pt}\rule{6pt}{0pt}} Leather~~
  \fcolorbox{white}{minc_10}{\rule{0pt}{6pt}\rule{6pt}{0pt}} Metal~~
  \fcolorbox{white}{minc_11}{\rule{0pt}{6pt}\rule{6pt}{0pt}} Mirror~~
  \fcolorbox{white}{minc_12}{\rule{0pt}{6pt}\rule{6pt}{0pt}} Other~~
  \fcolorbox{white}{minc_13}{\rule{0pt}{6pt}\rule{6pt}{0pt}} Painted~~
  \fcolorbox{white}{minc_14}{\rule{0pt}{6pt}\rule{6pt}{0pt}} Paper~~
  \fcolorbox{white}{minc_15}{\rule{0pt}{6pt}\rule{6pt}{0pt}} Plastic~~\\
  \fcolorbox{white}{minc_16}{\rule{0pt}{6pt}\rule{6pt}{0pt}} Polished Stone~~
  \fcolorbox{white}{minc_17}{\rule{0pt}{6pt}\rule{6pt}{0pt}} Skin~~
  \fcolorbox{white}{minc_18}{\rule{0pt}{6pt}\rule{6pt}{0pt}} Sky~~
  \fcolorbox{white}{minc_19}{\rule{0pt}{6pt}\rule{6pt}{0pt}} Stone~~
  \fcolorbox{white}{minc_20}{\rule{0pt}{6pt}\rule{6pt}{0pt}} Tile~~
  \fcolorbox{white}{minc_21}{\rule{0pt}{6pt}\rule{6pt}{0pt}} Wallpaper~~
  \fcolorbox{white}{minc_22}{\rule{0pt}{6pt}\rule{6pt}{0pt}} Water~~
  \fcolorbox{white}{minc_23}{\rule{0pt}{6pt}\rule{6pt}{0pt}} Wood~~ \\
  }
  \subfigure{%
    \includegraphics[width=.18\columnwidth]{figures/supplementary/000010868_given.jpg}
  }
  \subfigure{%
    \includegraphics[width=.18\columnwidth]{figures/supplementary/000010868_gt.png}
  }
  \subfigure{%
    \includegraphics[width=.18\columnwidth]{figures/supplementary/000010868_cnn.png}
  }
  \subfigure{%
    \includegraphics[width=.18\columnwidth]{figures/supplementary/000010868_gauss.png}
  }
  \subfigure{%
    \includegraphics[width=.18\columnwidth]{figures/supplementary/000010868_learnt.png}
  }\\[-2ex]
  \subfigure{%
    \includegraphics[width=.18\columnwidth]{figures/supplementary/000006011_given.jpg}
  }
  \subfigure{%
    \includegraphics[width=.18\columnwidth]{figures/supplementary/000006011_gt.png}
  }
  \subfigure{%
    \includegraphics[width=.18\columnwidth]{figures/supplementary/000006011_cnn.png}
  }
  \subfigure{%
    \includegraphics[width=.18\columnwidth]{figures/supplementary/000006011_gauss.png}
  }
  \subfigure{%
    \includegraphics[width=.18\columnwidth]{figures/supplementary/000006011_learnt.png}
  }\\[-2ex]
    \subfigure{%
    \includegraphics[width=.18\columnwidth]{figures/supplementary/000008553_given.jpg}
  }
  \subfigure{%
    \includegraphics[width=.18\columnwidth]{figures/supplementary/000008553_gt.png}
  }
  \subfigure{%
    \includegraphics[width=.18\columnwidth]{figures/supplementary/000008553_cnn.png}
  }
  \subfigure{%
    \includegraphics[width=.18\columnwidth]{figures/supplementary/000008553_gauss.png}
  }
  \subfigure{%
    \includegraphics[width=.18\columnwidth]{figures/supplementary/000008553_learnt.png}
  }\\[-2ex]
   \subfigure{%
    \includegraphics[width=.18\columnwidth]{figures/supplementary/000009188_given.jpg}
  }
  \subfigure{%
    \includegraphics[width=.18\columnwidth]{figures/supplementary/000009188_gt.png}
  }
  \subfigure{%
    \includegraphics[width=.18\columnwidth]{figures/supplementary/000009188_cnn.png}
  }
  \subfigure{%
    \includegraphics[width=.18\columnwidth]{figures/supplementary/000009188_gauss.png}
  }
  \subfigure{%
    \includegraphics[width=.18\columnwidth]{figures/supplementary/000009188_learnt.png}
  }\\[-2ex]
  \setcounter{subfigure}{0}
  \subfigure[Input]{%
    \includegraphics[width=.18\columnwidth]{figures/supplementary/000023570_given.jpg}
  }
  \subfigure[Ground Truth]{%
    \includegraphics[width=.18\columnwidth]{figures/supplementary/000023570_gt.png}
  }
  \subfigure[DeepLab]{%
    \includegraphics[width=.18\columnwidth]{figures/supplementary/000023570_cnn.png}
  }
  \subfigure[+GaussCRF]{%
    \includegraphics[width=.18\columnwidth]{figures/supplementary/000023570_gauss.png}
  }
  \subfigure[+LearnedCRF]{%
    \includegraphics[width=.18\columnwidth]{figures/supplementary/000023570_learnt.png}
  }
  \mycaption{Material Segmentation}{Example results of material segmentation.
  (c)~depicts the unary results before application of MF, (d)~after two steps of MF with Gaussian edge CRF potentials, (e)~after two steps of MF with learned edge CRF potentials.}
    \label{fig:material_visuals-app2}
\end{figure*}


\begin{table*}[h]
\tiny
  \centering
    \begin{tabular}{L{2.3cm} L{2.25cm} C{1.5cm} C{0.7cm} C{0.6cm} C{0.7cm} C{0.7cm} C{0.7cm} C{1.6cm} C{0.6cm} C{0.6cm} C{0.6cm}}
      \toprule
& & & & & \multicolumn{3}{c}{\textbf{Data Statistics}} & \multicolumn{4}{c}{\textbf{Training Protocol}} \\

\textbf{Experiment} & \textbf{Feature Types} & \textbf{Feature Scales} & \textbf{Filter Size} & \textbf{Filter Nbr.} & \textbf{Train}  & \textbf{Val.} & \textbf{Test} & \textbf{Loss Type} & \textbf{LR} & \textbf{Batch} & \textbf{Epochs} \\
      \midrule
      \multicolumn{2}{c}{\textbf{Single Bilateral Filter Applications}} & & & & & & & & & \\
      \textbf{2$\times$ Color Upsampling} & Position$_{1}$, Intensity (3D) & 0.13, 0.17 & 65 & 2 & 10581 & 1449 & 1456 & MSE & 1e-06 & 200 & 94.5\\
      \textbf{4$\times$ Color Upsampling} & Position$_{1}$, Intensity (3D) & 0.06, 0.17 & 65 & 2 & 10581 & 1449 & 1456 & MSE & 1e-06 & 200 & 94.5\\
      \textbf{8$\times$ Color Upsampling} & Position$_{1}$, Intensity (3D) & 0.03, 0.17 & 65 & 2 & 10581 & 1449 & 1456 & MSE & 1e-06 & 200 & 94.5\\
      \textbf{16$\times$ Color Upsampling} & Position$_{1}$, Intensity (3D) & 0.02, 0.17 & 65 & 2 & 10581 & 1449 & 1456 & MSE & 1e-06 & 200 & 94.5\\
      \textbf{Depth Upsampling} & Position$_{1}$, Color (5D) & 0.05, 0.02 & 665 & 2 & 795 & 100 & 654 & MSE & 1e-07 & 50 & 251.6\\
      \textbf{Mesh Denoising} & Isomap (4D) & 46.00 & 63 & 2 & 1000 & 200 & 500 & MSE & 100 & 10 & 100.0 \\
      \midrule
      \multicolumn{2}{c}{\textbf{DenseCRF Applications}} & & & & & & & & &\\
      \multicolumn{2}{l}{\textbf{Semantic Segmentation}} & & & & & & & & &\\
      \textbf{- 1step MF} & Position$_{1}$, Color (5D); Position$_{1}$ (2D) & 0.01, 0.34; 0.34  & 665; 19  & 2; 2 & 10581 & 1449 & 1456 & Logistic & 0.1 & 5 & 1.4 \\
      \textbf{- 2step MF} & Position$_{1}$, Color (5D); Position$_{1}$ (2D) & 0.01, 0.34; 0.34 & 665; 19 & 2; 2 & 10581 & 1449 & 1456 & Logistic & 0.1 & 5 & 1.4 \\
      \textbf{- \textit{loose} 2step MF} & Position$_{1}$, Color (5D); Position$_{1}$ (2D) & 0.01, 0.34; 0.34 & 665; 19 & 2; 2 &10581 & 1449 & 1456 & Logistic & 0.1 & 5 & +1.9  \\ \\
      \multicolumn{2}{l}{\textbf{Material Segmentation}} & & & & & & & & &\\
      \textbf{- 1step MF} & Position$_{2}$, Lab-Color (5D) & 5.00, 0.05, 0.30  & 665 & 2 & 928 & 150 & 1798 & Weighted Logistic & 1e-04 & 24 & 2.6 \\
      \textbf{- 2step MF} & Position$_{2}$, Lab-Color (5D) & 5.00, 0.05, 0.30 & 665 & 2 & 928 & 150 & 1798 & Weighted Logistic & 1e-04 & 12 & +0.7 \\
      \textbf{- \textit{loose} 2step MF} & Position$_{2}$, Lab-Color (5D) & 5.00, 0.05, 0.30 & 665 & 2 & 928 & 150 & 1798 & Weighted Logistic & 1e-04 & 12 & +0.2\\
      \midrule
      \multicolumn{2}{c}{\textbf{Neural Network Applications}} & & & & & & & & &\\
      \textbf{Tiles: CNN-9$\times$9} & - & - & 81 & 4 & 10000 & 1000 & 1000 & Logistic & 0.01 & 100 & 500.0 \\
      \textbf{Tiles: CNN-13$\times$13} & - & - & 169 & 6 & 10000 & 1000 & 1000 & Logistic & 0.01 & 100 & 500.0 \\
      \textbf{Tiles: CNN-17$\times$17} & - & - & 289 & 8 & 10000 & 1000 & 1000 & Logistic & 0.01 & 100 & 500.0 \\
      \textbf{Tiles: CNN-21$\times$21} & - & - & 441 & 10 & 10000 & 1000 & 1000 & Logistic & 0.01 & 100 & 500.0 \\
      \textbf{Tiles: BNN} & Position$_{1}$, Color (5D) & 0.05, 0.04 & 63 & 1 & 10000 & 1000 & 1000 & Logistic & 0.01 & 100 & 30.0 \\
      \textbf{LeNet} & - & - & 25 & 2 & 5490 & 1098 & 1647 & Logistic & 0.1 & 100 & 182.2 \\
      \textbf{Crop-LeNet} & - & - & 25 & 2 & 5490 & 1098 & 1647 & Logistic & 0.1 & 100 & 182.2 \\
      \textbf{BNN-LeNet} & Position$_{2}$ (2D) & 20.00 & 7 & 1 & 5490 & 1098 & 1647 & Logistic & 0.1 & 100 & 182.2 \\
      \textbf{DeepCNet} & - & - & 9 & 1 & 5490 & 1098 & 1647 & Logistic & 0.1 & 100 & 182.2 \\
      \textbf{Crop-DeepCNet} & - & - & 9 & 1 & 5490 & 1098 & 1647 & Logistic & 0.1 & 100 & 182.2 \\
      \textbf{BNN-DeepCNet} & Position$_{2}$ (2D) & 40.00  & 7 & 1 & 5490 & 1098 & 1647 & Logistic & 0.1 & 100 & 182.2 \\
      \bottomrule
      \\
    \end{tabular}
    \mycaption{Experiment Protocols} {Experiment protocols for the different experiments presented in this work. \textbf{Feature Types}:
    Feature spaces used for the bilateral convolutions. Position$_1$ corresponds to un-normalized pixel positions whereas Position$_2$ corresponds
    to pixel positions normalized to $[0,1]$ with respect to the given image. \textbf{Feature Scales}: Cross-validated scales for the features used.
     \textbf{Filter Size}: Number of elements in the filter that is being learned. \textbf{Filter Nbr.}: Half-width of the filter. \textbf{Train},
     \textbf{Val.} and \textbf{Test} corresponds to the number of train, validation and test images used in the experiment. \textbf{Loss Type}: Type
     of loss used for back-propagation. ``MSE'' corresponds to Euclidean mean squared error loss and ``Logistic'' corresponds to multinomial logistic
     loss. ``Weighted Logistic'' is the class-weighted multinomial logistic loss. We weighted the loss with inverse class probability for material
     segmentation task due to the small availability of training data with class imbalance. \textbf{LR}: Fixed learning rate used in stochastic gradient
     descent. \textbf{Batch}: Number of images used in one parameter update step. \textbf{Epochs}: Number of training epochs. In all the experiments,
     we used fixed momentum of 0.9 and weight decay of 0.0005 for stochastic gradient descent. ```Color Upsampling'' experiments in this Table corresponds
     to those performed on Pascal VOC12 dataset images. For all experiments using Pascal VOC12 images, we use extended
     training segmentation dataset available from~\cite{hariharan2011moredata}, and used standard validation and test splits
     from the main dataset~\cite{voc2012segmentation}.}
  \label{tbl:parameters}
\end{table*}

\clearpage

\section{Parameters and Additional Results for Video Propagation Networks}

In this Section, we present experiment protocols and additional qualitative results for experiments
on video object segmentation, semantic video segmentation and video color
propagation. Table~\ref{tbl:parameters_supp} shows the feature scales and other parameters used in different experiments.
Figures~\ref{fig:video_seg_pos_supp} show some qualitative results on video object segmentation
with some failure cases in Fig.~\ref{fig:video_seg_neg_supp}.
Figure~\ref{fig:semantic_visuals_supp} shows some qualitative results on semantic video segmentation and
Fig.~\ref{fig:color_visuals_supp} shows results on video color propagation.

\newcolumntype{L}[1]{>{\raggedright\let\newline\\\arraybackslash\hspace{0pt}}b{#1}}
\newcolumntype{C}[1]{>{\centering\let\newline\\\arraybackslash\hspace{0pt}}b{#1}}
\newcolumntype{R}[1]{>{\raggedleft\let\newline\\\arraybackslash\hspace{0pt}}b{#1}}

\begin{table*}[h]
\tiny
  \centering
    \begin{tabular}{L{3.0cm} L{2.4cm} L{2.8cm} L{2.8cm} C{0.5cm} C{1.0cm} L{1.2cm}}
      \toprule
\textbf{Experiment} & \textbf{Feature Type} & \textbf{Feature Scale-1, $\Lambda_a$} & \textbf{Feature Scale-2, $\Lambda_b$} & \textbf{$\alpha$} & \textbf{Input Frames} & \textbf{Loss Type} \\
      \midrule
      \textbf{Video Object Segmentation} & ($x,y,Y,Cb,Cr,t$) & (0.02,0.02,0.07,0.4,0.4,0.01) & (0.03,0.03,0.09,0.5,0.5,0.2) & 0.5 & 9 & Logistic\\
      \midrule
      \textbf{Semantic Video Segmentation} & & & & & \\
      \textbf{with CNN1~\cite{yu2015multi}-NoFlow} & ($x,y,R,G,B,t$) & (0.08,0.08,0.2,0.2,0.2,0.04) & (0.11,0.11,0.2,0.2,0.2,0.04) & 0.5 & 3 & Logistic \\
      \textbf{with CNN1~\cite{yu2015multi}-Flow} & ($x+u_x,y+u_y,R,G,B,t$) & (0.11,0.11,0.14,0.14,0.14,0.03) & (0.08,0.08,0.12,0.12,0.12,0.01) & 0.65 & 3 & Logistic\\
      \textbf{with CNN2~\cite{richter2016playing}-Flow} & ($x+u_x,y+u_y,R,G,B,t$) & (0.08,0.08,0.2,0.2,0.2,0.04) & (0.09,0.09,0.25,0.25,0.25,0.03) & 0.5 & 4 & Logistic\\
      \midrule
      \textbf{Video Color Propagation} & ($x,y,I,t$)  & (0.04,0.04,0.2,0.04) & No second kernel & 1 & 4 & MSE\\
      \bottomrule
      \\
    \end{tabular}
    \mycaption{Experiment Protocols} {Experiment protocols for the different experiments presented in this work. \textbf{Feature Types}:
    Feature spaces used for the bilateral convolutions, with position ($x,y$) and color
    ($R,G,B$ or $Y,Cb,Cr$) features $\in [0,255]$. $u_x$, $u_y$ denotes optical flow with respect
    to the present frame and $I$ denotes grayscale intensity.
    \textbf{Feature Scales ($\Lambda_a, \Lambda_b$)}: Cross-validated scales for the features used.
    \textbf{$\alpha$}: Exponential time decay for the input frames.
    \textbf{Input Frames}: Number of input frames for VPN.
    \textbf{Loss Type}: Type
     of loss used for back-propagation. ``MSE'' corresponds to Euclidean mean squared error loss and ``Logistic'' corresponds to multinomial logistic loss.}
  \label{tbl:parameters_supp}
\end{table*}

% \begin{figure}[th!]
% \begin{center}
%   \centerline{\includegraphics[width=\textwidth]{figures/video_seg_visuals_supp_small.pdf}}
%     \mycaption{Video Object Segmentation}
%     {Shown are the different frames in example videos with the corresponding
%     ground truth (GT) masks, predictions from BVS~\cite{marki2016bilateral},
%     OFL~\cite{tsaivideo}, VPN (VPN-Stage2) and VPN-DLab (VPN-DeepLab) models.}
%     \label{fig:video_seg_small_supp}
% \end{center}
% \vspace{-1.0cm}
% \end{figure}

\begin{figure}[th!]
\begin{center}
  \centerline{\includegraphics[width=0.7\textwidth]{figures/video_seg_visuals_supp_positive.pdf}}
    \mycaption{Video Object Segmentation}
    {Shown are the different frames in example videos with the corresponding
    ground truth (GT) masks, predictions from BVS~\cite{marki2016bilateral},
    OFL~\cite{tsaivideo}, VPN (VPN-Stage2) and VPN-DLab (VPN-DeepLab) models.}
    \label{fig:video_seg_pos_supp}
\end{center}
\vspace{-1.0cm}
\end{figure}

\begin{figure}[th!]
\begin{center}
  \centerline{\includegraphics[width=0.7\textwidth]{figures/video_seg_visuals_supp_negative.pdf}}
    \mycaption{Failure Cases for Video Object Segmentation}
    {Shown are the different frames in example videos with the corresponding
    ground truth (GT) masks, predictions from BVS~\cite{marki2016bilateral},
    OFL~\cite{tsaivideo}, VPN (VPN-Stage2) and VPN-DLab (VPN-DeepLab) models.}
    \label{fig:video_seg_neg_supp}
\end{center}
\vspace{-1.0cm}
\end{figure}

\begin{figure}[th!]
\begin{center}
  \centerline{\includegraphics[width=0.9\textwidth]{figures/supp_semantic_visual.pdf}}
    \mycaption{Semantic Video Segmentation}
    {Input video frames and the corresponding ground truth (GT)
    segmentation together with the predictions of CNN~\cite{yu2015multi} and with
    VPN-Flow.}
    \label{fig:semantic_visuals_supp}
\end{center}
\vspace{-0.7cm}
\end{figure}

\begin{figure}[th!]
\begin{center}
  \centerline{\includegraphics[width=\textwidth]{figures/colorization_visuals_supp.pdf}}
  \mycaption{Video Color Propagation}
  {Input grayscale video frames and corresponding ground-truth (GT) color images
  together with color predictions of Levin et al.~\cite{levin2004colorization} and VPN-Stage1 models.}
  \label{fig:color_visuals_supp}
\end{center}
\vspace{-0.7cm}
\end{figure}

\clearpage

\section{Additional Material for Bilateral Inception Networks}
\label{sec:binception-app}

In this section of the Appendix, we first discuss the use of approximate bilateral
filtering in BI modules (Sec.~\ref{sec:lattice}).
Later, we present some qualitative results using different models for the approach presented in
Chapter~\ref{chap:binception} (Sec.~\ref{sec:qualitative-app}).

\subsection{Approximate Bilateral Filtering}
\label{sec:lattice}

The bilateral inception module presented in Chapter~\ref{chap:binception} computes a matrix-vector
product between a Gaussian filter $K$ and a vector of activations $\bz_c$.
Bilateral filtering is an important operation and many algorithmic techniques have been
proposed to speed-up this operation~\cite{paris2006fast,adams2010fast,gastal2011domain}.
In the main paper we opted to implement what can be considered the
brute-force variant of explicitly constructing $K$ and then using BLAS to compute the
matrix-vector product. This resulted in a few millisecond operation.
The explicit way to compute is possible due to the
reduction to super-pixels, e.g., it would not work for DenseCRF variants
that operate on the full image resolution.

Here, we present experiments where we use the fast approximate bilateral filtering
algorithm of~\cite{adams2010fast}, which is also used in Chapter~\ref{chap:bnn}
for learning sparse high dimensional filters. This
choice allows for larger dimensions of matrix-vector multiplication. The reason for choosing
the explicit multiplication in Chapter~\ref{chap:binception} was that it was computationally faster.
For the small sizes of the involved matrices and vectors, the explicit computation is sufficient and we had no
GPU implementation of an approximate technique that matched this runtime. Also it
is conceptually easier and the gradient to the feature transformations ($\Lambda \mathbf{f}$) is
obtained using standard matrix calculus.

\subsubsection{Experiments}

We modified the existing segmentation architectures analogous to those in Chapter~\ref{chap:binception}.
The main difference is that, here, the inception modules use the lattice
approximation~\cite{adams2010fast} to compute the bilateral filtering.
Using the lattice approximation did not allow us to back-propagate through feature transformations ($\Lambda$)
and thus we used hand-specified feature scales as will be explained later.
Specifically, we take CNN architectures from the works
of~\cite{chen2014semantic,zheng2015conditional,bell2015minc} and insert the BI modules between
the spatial FC layers.
We use superpixels from~\cite{DollarICCV13edges}
for all the experiments with the lattice approximation. Experiments are
performed using Caffe neural network framework~\cite{jia2014caffe}.

\begin{table}
  \small
  \centering
  \begin{tabular}{p{5.5cm}>{\raggedright\arraybackslash}p{1.4cm}>{\centering\arraybackslash}p{2.2cm}}
    \toprule
		\textbf{Model} & \emph{IoU} & \emph{Runtime}(ms) \\
    \midrule

    %%%%%%%%%%%% Scores computed by us)%%%%%%%%%%%%
		\deeplablargefov & 68.9 & 145ms\\
    \midrule
    \bi{7}{2}-\bi{8}{10}& \textbf{73.8} & +600 \\
    \midrule
    \deeplablargefovcrf~\cite{chen2014semantic} & 72.7 & +830\\
    \deeplabmsclargefovcrf~\cite{chen2014semantic} & \textbf{73.6} & +880\\
    DeepLab-EdgeNet~\cite{chen2015semantic} & 71.7 & +30\\
    DeepLab-EdgeNet-CRF~\cite{chen2015semantic} & \textbf{73.6} & +860\\
  \bottomrule \\
  \end{tabular}
  \mycaption{Semantic Segmentation using the DeepLab model}
  {IoU scores on the Pascal VOC12 segmentation test dataset
  with different models and our modified inception model.
  Also shown are the corresponding runtimes in milliseconds. Runtimes
  also include superpixel computations (300 ms with Dollar superpixels~\cite{DollarICCV13edges})}
  \label{tab:largefovresults}
\end{table}

\paragraph{Semantic Segmentation}
The experiments in this section use the Pascal VOC12 segmentation dataset~\cite{voc2012segmentation} with 21 object classes and the images have a maximum resolution of 0.25 megapixels.
For all experiments on VOC12, we train using the extended training set of
10581 images collected by~\cite{hariharan2011moredata}.
We modified the \deeplab~network architecture of~\cite{chen2014semantic} and
the CRFasRNN architecture from~\cite{zheng2015conditional} which uses a CNN with
deconvolution layers followed by DenseCRF trained end-to-end.

\paragraph{DeepLab Model}\label{sec:deeplabmodel}
We experimented with the \bi{7}{2}-\bi{8}{10} inception model.
Results using the~\deeplab~model are summarized in Tab.~\ref{tab:largefovresults}.
Although we get similar improvements with inception modules as with the
explicit kernel computation, using lattice approximation is slower.

\begin{table}
  \small
  \centering
  \begin{tabular}{p{6.4cm}>{\raggedright\arraybackslash}p{1.8cm}>{\raggedright\arraybackslash}p{1.8cm}}
    \toprule
    \textbf{Model} & \emph{IoU (Val)} & \emph{IoU (Test)}\\
    \midrule
    %%%%%%%%%%%% Scores computed by us)%%%%%%%%%%%%
    CNN &  67.5 & - \\
    \deconv (CNN+Deconvolutions) & 69.8 & 72.0 \\
    \midrule
    \bi{3}{6}-\bi{4}{6}-\bi{7}{2}-\bi{8}{6}& 71.9 & - \\
    \bi{3}{6}-\bi{4}{6}-\bi{7}{2}-\bi{8}{6}-\gi{6}& 73.6 &  \href{http://host.robots.ox.ac.uk:8080/anonymous/VOTV5E.html}{\textbf{75.2}}\\
    \midrule
    \deconvcrf (CRF-RNN)~\cite{zheng2015conditional} & 73.0 & 74.7\\
    Context-CRF-RNN~\cite{yu2015multi} & ~~ - ~ & \textbf{75.3} \\
    \bottomrule \\
  \end{tabular}
  \mycaption{Semantic Segmentation using the CRFasRNN model}{IoU score corresponding to different models
  on Pascal VOC12 reduced validation / test segmentation dataset. The reduced validation set consists of 346 images
  as used in~\cite{zheng2015conditional} where we adapted the model from.}
  \label{tab:deconvresults-app}
\end{table}

\paragraph{CRFasRNN Model}\label{sec:deepinception}
We add BI modules after score-pool3, score-pool4, \fc{7} and \fc{8} $1\times1$ convolution layers
resulting in the \bi{3}{6}-\bi{4}{6}-\bi{7}{2}-\bi{8}{6}
model and also experimented with another variant where $BI_8$ is followed by another inception
module, G$(6)$, with 6 Gaussian kernels.
Note that here also we discarded both deconvolution and DenseCRF parts of the original model~\cite{zheng2015conditional}
and inserted the BI modules in the base CNN and found similar improvements compared to the inception modules with explicit
kernel computaion. See Tab.~\ref{tab:deconvresults-app} for results on the CRFasRNN model.

\paragraph{Material Segmentation}
Table~\ref{tab:mincresults-app} shows the results on the MINC dataset~\cite{bell2015minc}
obtained by modifying the AlexNet architecture with our inception modules. We observe
similar improvements as with explicit kernel construction.
For this model, we do not provide any learned setup due to very limited segment training
data. The weights to combine outputs in the bilateral inception layer are
found by validation on the validation set.

\begin{table}[t]
  \small
  \centering
  \begin{tabular}{p{3.5cm}>{\centering\arraybackslash}p{4.0cm}}
    \toprule
    \textbf{Model} & Class / Total accuracy\\
    \midrule

    %%%%%%%%%%%% Scores computed by us)%%%%%%%%%%%%
    AlexNet CNN & 55.3 / 58.9 \\
    \midrule
    \bi{7}{2}-\bi{8}{6}& 68.5 / 71.8 \\
    \bi{7}{2}-\bi{8}{6}-G$(6)$& 67.6 / 73.1 \\
    \midrule
    AlexNet-CRF & 65.5 / 71.0 \\
    \bottomrule \\
  \end{tabular}
  \mycaption{Material Segmentation using AlexNet}{Pixel accuracy of different models on
  the MINC material segmentation test dataset~\cite{bell2015minc}.}
  \label{tab:mincresults-app}
\end{table}

\paragraph{Scales of Bilateral Inception Modules}
\label{sec:scales}

Unlike the explicit kernel technique presented in the main text (Chapter~\ref{chap:binception}),
we didn't back-propagate through feature transformation ($\Lambda$)
using the approximate bilateral filter technique.
So, the feature scales are hand-specified and validated, which are as follows.
The optimal scale values for the \bi{7}{2}-\bi{8}{2} model are found by validation for the best performance which are
$\sigma_{xy}$ = (0.1, 0.1) for the spatial (XY) kernel and $\sigma_{rgbxy}$ = (0.1, 0.1, 0.1, 0.01, 0.01) for color and position (RGBXY)  kernel.
Next, as more kernels are added to \bi{8}{2}, we set scales to be $\alpha$*($\sigma_{xy}$, $\sigma_{rgbxy}$).
The value of $\alpha$ is chosen as  1, 0.5, 0.1, 0.05, 0.1, at uniform interval, for the \bi{8}{10} bilateral inception module.


\subsection{Qualitative Results}
\label{sec:qualitative-app}

In this section, we present more qualitative results obtained using the BI module with explicit
kernel computation technique presented in Chapter~\ref{chap:binception}. Results on the Pascal VOC12
dataset~\cite{voc2012segmentation} using the DeepLab-LargeFOV model are shown in Fig.~\ref{fig:semantic_visuals-app},
followed by the results on MINC dataset~\cite{bell2015minc}
in Fig.~\ref{fig:material_visuals-app} and on
Cityscapes dataset~\cite{Cordts2015Cvprw} in Fig.~\ref{fig:street_visuals-app}.


\definecolor{voc_1}{RGB}{0, 0, 0}
\definecolor{voc_2}{RGB}{128, 0, 0}
\definecolor{voc_3}{RGB}{0, 128, 0}
\definecolor{voc_4}{RGB}{128, 128, 0}
\definecolor{voc_5}{RGB}{0, 0, 128}
\definecolor{voc_6}{RGB}{128, 0, 128}
\definecolor{voc_7}{RGB}{0, 128, 128}
\definecolor{voc_8}{RGB}{128, 128, 128}
\definecolor{voc_9}{RGB}{64, 0, 0}
\definecolor{voc_10}{RGB}{192, 0, 0}
\definecolor{voc_11}{RGB}{64, 128, 0}
\definecolor{voc_12}{RGB}{192, 128, 0}
\definecolor{voc_13}{RGB}{64, 0, 128}
\definecolor{voc_14}{RGB}{192, 0, 128}
\definecolor{voc_15}{RGB}{64, 128, 128}
\definecolor{voc_16}{RGB}{192, 128, 128}
\definecolor{voc_17}{RGB}{0, 64, 0}
\definecolor{voc_18}{RGB}{128, 64, 0}
\definecolor{voc_19}{RGB}{0, 192, 0}
\definecolor{voc_20}{RGB}{128, 192, 0}
\definecolor{voc_21}{RGB}{0, 64, 128}
\definecolor{voc_22}{RGB}{128, 64, 128}

\begin{figure*}[!ht]
  \small
  \centering
  \fcolorbox{white}{voc_1}{\rule{0pt}{4pt}\rule{4pt}{0pt}} Background~~
  \fcolorbox{white}{voc_2}{\rule{0pt}{4pt}\rule{4pt}{0pt}} Aeroplane~~
  \fcolorbox{white}{voc_3}{\rule{0pt}{4pt}\rule{4pt}{0pt}} Bicycle~~
  \fcolorbox{white}{voc_4}{\rule{0pt}{4pt}\rule{4pt}{0pt}} Bird~~
  \fcolorbox{white}{voc_5}{\rule{0pt}{4pt}\rule{4pt}{0pt}} Boat~~
  \fcolorbox{white}{voc_6}{\rule{0pt}{4pt}\rule{4pt}{0pt}} Bottle~~
  \fcolorbox{white}{voc_7}{\rule{0pt}{4pt}\rule{4pt}{0pt}} Bus~~
  \fcolorbox{white}{voc_8}{\rule{0pt}{4pt}\rule{4pt}{0pt}} Car~~\\
  \fcolorbox{white}{voc_9}{\rule{0pt}{4pt}\rule{4pt}{0pt}} Cat~~
  \fcolorbox{white}{voc_10}{\rule{0pt}{4pt}\rule{4pt}{0pt}} Chair~~
  \fcolorbox{white}{voc_11}{\rule{0pt}{4pt}\rule{4pt}{0pt}} Cow~~
  \fcolorbox{white}{voc_12}{\rule{0pt}{4pt}\rule{4pt}{0pt}} Dining Table~~
  \fcolorbox{white}{voc_13}{\rule{0pt}{4pt}\rule{4pt}{0pt}} Dog~~
  \fcolorbox{white}{voc_14}{\rule{0pt}{4pt}\rule{4pt}{0pt}} Horse~~
  \fcolorbox{white}{voc_15}{\rule{0pt}{4pt}\rule{4pt}{0pt}} Motorbike~~
  \fcolorbox{white}{voc_16}{\rule{0pt}{4pt}\rule{4pt}{0pt}} Person~~\\
  \fcolorbox{white}{voc_17}{\rule{0pt}{4pt}\rule{4pt}{0pt}} Potted Plant~~
  \fcolorbox{white}{voc_18}{\rule{0pt}{4pt}\rule{4pt}{0pt}} Sheep~~
  \fcolorbox{white}{voc_19}{\rule{0pt}{4pt}\rule{4pt}{0pt}} Sofa~~
  \fcolorbox{white}{voc_20}{\rule{0pt}{4pt}\rule{4pt}{0pt}} Train~~
  \fcolorbox{white}{voc_21}{\rule{0pt}{4pt}\rule{4pt}{0pt}} TV monitor~~\\


  \subfigure{%
    \includegraphics[width=.15\columnwidth]{figures/supplementary/2008_001308_given.png}
  }
  \subfigure{%
    \includegraphics[width=.15\columnwidth]{figures/supplementary/2008_001308_sp.png}
  }
  \subfigure{%
    \includegraphics[width=.15\columnwidth]{figures/supplementary/2008_001308_gt.png}
  }
  \subfigure{%
    \includegraphics[width=.15\columnwidth]{figures/supplementary/2008_001308_cnn.png}
  }
  \subfigure{%
    \includegraphics[width=.15\columnwidth]{figures/supplementary/2008_001308_crf.png}
  }
  \subfigure{%
    \includegraphics[width=.15\columnwidth]{figures/supplementary/2008_001308_ours.png}
  }\\[-2ex]


  \subfigure{%
    \includegraphics[width=.15\columnwidth]{figures/supplementary/2008_001821_given.png}
  }
  \subfigure{%
    \includegraphics[width=.15\columnwidth]{figures/supplementary/2008_001821_sp.png}
  }
  \subfigure{%
    \includegraphics[width=.15\columnwidth]{figures/supplementary/2008_001821_gt.png}
  }
  \subfigure{%
    \includegraphics[width=.15\columnwidth]{figures/supplementary/2008_001821_cnn.png}
  }
  \subfigure{%
    \includegraphics[width=.15\columnwidth]{figures/supplementary/2008_001821_crf.png}
  }
  \subfigure{%
    \includegraphics[width=.15\columnwidth]{figures/supplementary/2008_001821_ours.png}
  }\\[-2ex]



  \subfigure{%
    \includegraphics[width=.15\columnwidth]{figures/supplementary/2008_004612_given.png}
  }
  \subfigure{%
    \includegraphics[width=.15\columnwidth]{figures/supplementary/2008_004612_sp.png}
  }
  \subfigure{%
    \includegraphics[width=.15\columnwidth]{figures/supplementary/2008_004612_gt.png}
  }
  \subfigure{%
    \includegraphics[width=.15\columnwidth]{figures/supplementary/2008_004612_cnn.png}
  }
  \subfigure{%
    \includegraphics[width=.15\columnwidth]{figures/supplementary/2008_004612_crf.png}
  }
  \subfigure{%
    \includegraphics[width=.15\columnwidth]{figures/supplementary/2008_004612_ours.png}
  }\\[-2ex]


  \subfigure{%
    \includegraphics[width=.15\columnwidth]{figures/supplementary/2009_001008_given.png}
  }
  \subfigure{%
    \includegraphics[width=.15\columnwidth]{figures/supplementary/2009_001008_sp.png}
  }
  \subfigure{%
    \includegraphics[width=.15\columnwidth]{figures/supplementary/2009_001008_gt.png}
  }
  \subfigure{%
    \includegraphics[width=.15\columnwidth]{figures/supplementary/2009_001008_cnn.png}
  }
  \subfigure{%
    \includegraphics[width=.15\columnwidth]{figures/supplementary/2009_001008_crf.png}
  }
  \subfigure{%
    \includegraphics[width=.15\columnwidth]{figures/supplementary/2009_001008_ours.png}
  }\\[-2ex]




  \subfigure{%
    \includegraphics[width=.15\columnwidth]{figures/supplementary/2009_004497_given.png}
  }
  \subfigure{%
    \includegraphics[width=.15\columnwidth]{figures/supplementary/2009_004497_sp.png}
  }
  \subfigure{%
    \includegraphics[width=.15\columnwidth]{figures/supplementary/2009_004497_gt.png}
  }
  \subfigure{%
    \includegraphics[width=.15\columnwidth]{figures/supplementary/2009_004497_cnn.png}
  }
  \subfigure{%
    \includegraphics[width=.15\columnwidth]{figures/supplementary/2009_004497_crf.png}
  }
  \subfigure{%
    \includegraphics[width=.15\columnwidth]{figures/supplementary/2009_004497_ours.png}
  }\\[-2ex]



  \setcounter{subfigure}{0}
  \subfigure[\scriptsize Input]{%
    \includegraphics[width=.15\columnwidth]{figures/supplementary/2010_001327_given.png}
  }
  \subfigure[\scriptsize Superpixels]{%
    \includegraphics[width=.15\columnwidth]{figures/supplementary/2010_001327_sp.png}
  }
  \subfigure[\scriptsize GT]{%
    \includegraphics[width=.15\columnwidth]{figures/supplementary/2010_001327_gt.png}
  }
  \subfigure[\scriptsize Deeplab]{%
    \includegraphics[width=.15\columnwidth]{figures/supplementary/2010_001327_cnn.png}
  }
  \subfigure[\scriptsize +DenseCRF]{%
    \includegraphics[width=.15\columnwidth]{figures/supplementary/2010_001327_crf.png}
  }
  \subfigure[\scriptsize Using BI]{%
    \includegraphics[width=.15\columnwidth]{figures/supplementary/2010_001327_ours.png}
  }
  \mycaption{Semantic Segmentation}{Example results of semantic segmentation
  on the Pascal VOC12 dataset.
  (d)~depicts the DeepLab CNN result, (e)~CNN + 10 steps of mean-field inference,
  (f~result obtained with bilateral inception (BI) modules (\bi{6}{2}+\bi{7}{6}) between \fc~layers.}
  \label{fig:semantic_visuals-app}
\end{figure*}


\definecolor{minc_1}{HTML}{771111}
\definecolor{minc_2}{HTML}{CAC690}
\definecolor{minc_3}{HTML}{EEEEEE}
\definecolor{minc_4}{HTML}{7C8FA6}
\definecolor{minc_5}{HTML}{597D31}
\definecolor{minc_6}{HTML}{104410}
\definecolor{minc_7}{HTML}{BB819C}
\definecolor{minc_8}{HTML}{D0CE48}
\definecolor{minc_9}{HTML}{622745}
\definecolor{minc_10}{HTML}{666666}
\definecolor{minc_11}{HTML}{D54A31}
\definecolor{minc_12}{HTML}{101044}
\definecolor{minc_13}{HTML}{444126}
\definecolor{minc_14}{HTML}{75D646}
\definecolor{minc_15}{HTML}{DD4348}
\definecolor{minc_16}{HTML}{5C8577}
\definecolor{minc_17}{HTML}{C78472}
\definecolor{minc_18}{HTML}{75D6D0}
\definecolor{minc_19}{HTML}{5B4586}
\definecolor{minc_20}{HTML}{C04393}
\definecolor{minc_21}{HTML}{D69948}
\definecolor{minc_22}{HTML}{7370D8}
\definecolor{minc_23}{HTML}{7A3622}
\definecolor{minc_24}{HTML}{000000}

\begin{figure*}[!ht]
  \small % scriptsize
  \centering
  \fcolorbox{white}{minc_1}{\rule{0pt}{4pt}\rule{4pt}{0pt}} Brick~~
  \fcolorbox{white}{minc_2}{\rule{0pt}{4pt}\rule{4pt}{0pt}} Carpet~~
  \fcolorbox{white}{minc_3}{\rule{0pt}{4pt}\rule{4pt}{0pt}} Ceramic~~
  \fcolorbox{white}{minc_4}{\rule{0pt}{4pt}\rule{4pt}{0pt}} Fabric~~
  \fcolorbox{white}{minc_5}{\rule{0pt}{4pt}\rule{4pt}{0pt}} Foliage~~
  \fcolorbox{white}{minc_6}{\rule{0pt}{4pt}\rule{4pt}{0pt}} Food~~
  \fcolorbox{white}{minc_7}{\rule{0pt}{4pt}\rule{4pt}{0pt}} Glass~~
  \fcolorbox{white}{minc_8}{\rule{0pt}{4pt}\rule{4pt}{0pt}} Hair~~\\
  \fcolorbox{white}{minc_9}{\rule{0pt}{4pt}\rule{4pt}{0pt}} Leather~~
  \fcolorbox{white}{minc_10}{\rule{0pt}{4pt}\rule{4pt}{0pt}} Metal~~
  \fcolorbox{white}{minc_11}{\rule{0pt}{4pt}\rule{4pt}{0pt}} Mirror~~
  \fcolorbox{white}{minc_12}{\rule{0pt}{4pt}\rule{4pt}{0pt}} Other~~
  \fcolorbox{white}{minc_13}{\rule{0pt}{4pt}\rule{4pt}{0pt}} Painted~~
  \fcolorbox{white}{minc_14}{\rule{0pt}{4pt}\rule{4pt}{0pt}} Paper~~
  \fcolorbox{white}{minc_15}{\rule{0pt}{4pt}\rule{4pt}{0pt}} Plastic~~\\
  \fcolorbox{white}{minc_16}{\rule{0pt}{4pt}\rule{4pt}{0pt}} Polished Stone~~
  \fcolorbox{white}{minc_17}{\rule{0pt}{4pt}\rule{4pt}{0pt}} Skin~~
  \fcolorbox{white}{minc_18}{\rule{0pt}{4pt}\rule{4pt}{0pt}} Sky~~
  \fcolorbox{white}{minc_19}{\rule{0pt}{4pt}\rule{4pt}{0pt}} Stone~~
  \fcolorbox{white}{minc_20}{\rule{0pt}{4pt}\rule{4pt}{0pt}} Tile~~
  \fcolorbox{white}{minc_21}{\rule{0pt}{4pt}\rule{4pt}{0pt}} Wallpaper~~
  \fcolorbox{white}{minc_22}{\rule{0pt}{4pt}\rule{4pt}{0pt}} Water~~
  \fcolorbox{white}{minc_23}{\rule{0pt}{4pt}\rule{4pt}{0pt}} Wood~~\\
  \subfigure{%
    \includegraphics[width=.15\columnwidth]{figures/supplementary/000008468_given.png}
  }
  \subfigure{%
    \includegraphics[width=.15\columnwidth]{figures/supplementary/000008468_sp.png}
  }
  \subfigure{%
    \includegraphics[width=.15\columnwidth]{figures/supplementary/000008468_gt.png}
  }
  \subfigure{%
    \includegraphics[width=.15\columnwidth]{figures/supplementary/000008468_cnn.png}
  }
  \subfigure{%
    \includegraphics[width=.15\columnwidth]{figures/supplementary/000008468_crf.png}
  }
  \subfigure{%
    \includegraphics[width=.15\columnwidth]{figures/supplementary/000008468_ours.png}
  }\\[-2ex]

  \subfigure{%
    \includegraphics[width=.15\columnwidth]{figures/supplementary/000009053_given.png}
  }
  \subfigure{%
    \includegraphics[width=.15\columnwidth]{figures/supplementary/000009053_sp.png}
  }
  \subfigure{%
    \includegraphics[width=.15\columnwidth]{figures/supplementary/000009053_gt.png}
  }
  \subfigure{%
    \includegraphics[width=.15\columnwidth]{figures/supplementary/000009053_cnn.png}
  }
  \subfigure{%
    \includegraphics[width=.15\columnwidth]{figures/supplementary/000009053_crf.png}
  }
  \subfigure{%
    \includegraphics[width=.15\columnwidth]{figures/supplementary/000009053_ours.png}
  }\\[-2ex]




  \subfigure{%
    \includegraphics[width=.15\columnwidth]{figures/supplementary/000014977_given.png}
  }
  \subfigure{%
    \includegraphics[width=.15\columnwidth]{figures/supplementary/000014977_sp.png}
  }
  \subfigure{%
    \includegraphics[width=.15\columnwidth]{figures/supplementary/000014977_gt.png}
  }
  \subfigure{%
    \includegraphics[width=.15\columnwidth]{figures/supplementary/000014977_cnn.png}
  }
  \subfigure{%
    \includegraphics[width=.15\columnwidth]{figures/supplementary/000014977_crf.png}
  }
  \subfigure{%
    \includegraphics[width=.15\columnwidth]{figures/supplementary/000014977_ours.png}
  }\\[-2ex]


  \subfigure{%
    \includegraphics[width=.15\columnwidth]{figures/supplementary/000022922_given.png}
  }
  \subfigure{%
    \includegraphics[width=.15\columnwidth]{figures/supplementary/000022922_sp.png}
  }
  \subfigure{%
    \includegraphics[width=.15\columnwidth]{figures/supplementary/000022922_gt.png}
  }
  \subfigure{%
    \includegraphics[width=.15\columnwidth]{figures/supplementary/000022922_cnn.png}
  }
  \subfigure{%
    \includegraphics[width=.15\columnwidth]{figures/supplementary/000022922_crf.png}
  }
  \subfigure{%
    \includegraphics[width=.15\columnwidth]{figures/supplementary/000022922_ours.png}
  }\\[-2ex]


  \subfigure{%
    \includegraphics[width=.15\columnwidth]{figures/supplementary/000025711_given.png}
  }
  \subfigure{%
    \includegraphics[width=.15\columnwidth]{figures/supplementary/000025711_sp.png}
  }
  \subfigure{%
    \includegraphics[width=.15\columnwidth]{figures/supplementary/000025711_gt.png}
  }
  \subfigure{%
    \includegraphics[width=.15\columnwidth]{figures/supplementary/000025711_cnn.png}
  }
  \subfigure{%
    \includegraphics[width=.15\columnwidth]{figures/supplementary/000025711_crf.png}
  }
  \subfigure{%
    \includegraphics[width=.15\columnwidth]{figures/supplementary/000025711_ours.png}
  }\\[-2ex]


  \subfigure{%
    \includegraphics[width=.15\columnwidth]{figures/supplementary/000034473_given.png}
  }
  \subfigure{%
    \includegraphics[width=.15\columnwidth]{figures/supplementary/000034473_sp.png}
  }
  \subfigure{%
    \includegraphics[width=.15\columnwidth]{figures/supplementary/000034473_gt.png}
  }
  \subfigure{%
    \includegraphics[width=.15\columnwidth]{figures/supplementary/000034473_cnn.png}
  }
  \subfigure{%
    \includegraphics[width=.15\columnwidth]{figures/supplementary/000034473_crf.png}
  }
  \subfigure{%
    \includegraphics[width=.15\columnwidth]{figures/supplementary/000034473_ours.png}
  }\\[-2ex]


  \subfigure{%
    \includegraphics[width=.15\columnwidth]{figures/supplementary/000035463_given.png}
  }
  \subfigure{%
    \includegraphics[width=.15\columnwidth]{figures/supplementary/000035463_sp.png}
  }
  \subfigure{%
    \includegraphics[width=.15\columnwidth]{figures/supplementary/000035463_gt.png}
  }
  \subfigure{%
    \includegraphics[width=.15\columnwidth]{figures/supplementary/000035463_cnn.png}
  }
  \subfigure{%
    \includegraphics[width=.15\columnwidth]{figures/supplementary/000035463_crf.png}
  }
  \subfigure{%
    \includegraphics[width=.15\columnwidth]{figures/supplementary/000035463_ours.png}
  }\\[-2ex]


  \setcounter{subfigure}{0}
  \subfigure[\scriptsize Input]{%
    \includegraphics[width=.15\columnwidth]{figures/supplementary/000035993_given.png}
  }
  \subfigure[\scriptsize Superpixels]{%
    \includegraphics[width=.15\columnwidth]{figures/supplementary/000035993_sp.png}
  }
  \subfigure[\scriptsize GT]{%
    \includegraphics[width=.15\columnwidth]{figures/supplementary/000035993_gt.png}
  }
  \subfigure[\scriptsize AlexNet]{%
    \includegraphics[width=.15\columnwidth]{figures/supplementary/000035993_cnn.png}
  }
  \subfigure[\scriptsize +DenseCRF]{%
    \includegraphics[width=.15\columnwidth]{figures/supplementary/000035993_crf.png}
  }
  \subfigure[\scriptsize Using BI]{%
    \includegraphics[width=.15\columnwidth]{figures/supplementary/000035993_ours.png}
  }
  \mycaption{Material Segmentation}{Example results of material segmentation.
  (d)~depicts the AlexNet CNN result, (e)~CNN + 10 steps of mean-field inference,
  (f)~result obtained with bilateral inception (BI) modules (\bi{7}{2}+\bi{8}{6}) between
  \fc~layers.}
\label{fig:material_visuals-app}
\end{figure*}


\definecolor{city_1}{RGB}{128, 64, 128}
\definecolor{city_2}{RGB}{244, 35, 232}
\definecolor{city_3}{RGB}{70, 70, 70}
\definecolor{city_4}{RGB}{102, 102, 156}
\definecolor{city_5}{RGB}{190, 153, 153}
\definecolor{city_6}{RGB}{153, 153, 153}
\definecolor{city_7}{RGB}{250, 170, 30}
\definecolor{city_8}{RGB}{220, 220, 0}
\definecolor{city_9}{RGB}{107, 142, 35}
\definecolor{city_10}{RGB}{152, 251, 152}
\definecolor{city_11}{RGB}{70, 130, 180}
\definecolor{city_12}{RGB}{220, 20, 60}
\definecolor{city_13}{RGB}{255, 0, 0}
\definecolor{city_14}{RGB}{0, 0, 142}
\definecolor{city_15}{RGB}{0, 0, 70}
\definecolor{city_16}{RGB}{0, 60, 100}
\definecolor{city_17}{RGB}{0, 80, 100}
\definecolor{city_18}{RGB}{0, 0, 230}
\definecolor{city_19}{RGB}{119, 11, 32}
\begin{figure*}[!ht]
  \small % scriptsize
  \centering


  \subfigure{%
    \includegraphics[width=.18\columnwidth]{figures/supplementary/frankfurt00000_016005_given.png}
  }
  \subfigure{%
    \includegraphics[width=.18\columnwidth]{figures/supplementary/frankfurt00000_016005_sp.png}
  }
  \subfigure{%
    \includegraphics[width=.18\columnwidth]{figures/supplementary/frankfurt00000_016005_gt.png}
  }
  \subfigure{%
    \includegraphics[width=.18\columnwidth]{figures/supplementary/frankfurt00000_016005_cnn.png}
  }
  \subfigure{%
    \includegraphics[width=.18\columnwidth]{figures/supplementary/frankfurt00000_016005_ours.png}
  }\\[-2ex]

  \subfigure{%
    \includegraphics[width=.18\columnwidth]{figures/supplementary/frankfurt00000_004617_given.png}
  }
  \subfigure{%
    \includegraphics[width=.18\columnwidth]{figures/supplementary/frankfurt00000_004617_sp.png}
  }
  \subfigure{%
    \includegraphics[width=.18\columnwidth]{figures/supplementary/frankfurt00000_004617_gt.png}
  }
  \subfigure{%
    \includegraphics[width=.18\columnwidth]{figures/supplementary/frankfurt00000_004617_cnn.png}
  }
  \subfigure{%
    \includegraphics[width=.18\columnwidth]{figures/supplementary/frankfurt00000_004617_ours.png}
  }\\[-2ex]

  \subfigure{%
    \includegraphics[width=.18\columnwidth]{figures/supplementary/frankfurt00000_020880_given.png}
  }
  \subfigure{%
    \includegraphics[width=.18\columnwidth]{figures/supplementary/frankfurt00000_020880_sp.png}
  }
  \subfigure{%
    \includegraphics[width=.18\columnwidth]{figures/supplementary/frankfurt00000_020880_gt.png}
  }
  \subfigure{%
    \includegraphics[width=.18\columnwidth]{figures/supplementary/frankfurt00000_020880_cnn.png}
  }
  \subfigure{%
    \includegraphics[width=.18\columnwidth]{figures/supplementary/frankfurt00000_020880_ours.png}
  }\\[-2ex]



  \subfigure{%
    \includegraphics[width=.18\columnwidth]{figures/supplementary/frankfurt00001_007285_given.png}
  }
  \subfigure{%
    \includegraphics[width=.18\columnwidth]{figures/supplementary/frankfurt00001_007285_sp.png}
  }
  \subfigure{%
    \includegraphics[width=.18\columnwidth]{figures/supplementary/frankfurt00001_007285_gt.png}
  }
  \subfigure{%
    \includegraphics[width=.18\columnwidth]{figures/supplementary/frankfurt00001_007285_cnn.png}
  }
  \subfigure{%
    \includegraphics[width=.18\columnwidth]{figures/supplementary/frankfurt00001_007285_ours.png}
  }\\[-2ex]


  \subfigure{%
    \includegraphics[width=.18\columnwidth]{figures/supplementary/frankfurt00001_059789_given.png}
  }
  \subfigure{%
    \includegraphics[width=.18\columnwidth]{figures/supplementary/frankfurt00001_059789_sp.png}
  }
  \subfigure{%
    \includegraphics[width=.18\columnwidth]{figures/supplementary/frankfurt00001_059789_gt.png}
  }
  \subfigure{%
    \includegraphics[width=.18\columnwidth]{figures/supplementary/frankfurt00001_059789_cnn.png}
  }
  \subfigure{%
    \includegraphics[width=.18\columnwidth]{figures/supplementary/frankfurt00001_059789_ours.png}
  }\\[-2ex]


  \subfigure{%
    \includegraphics[width=.18\columnwidth]{figures/supplementary/frankfurt00001_068208_given.png}
  }
  \subfigure{%
    \includegraphics[width=.18\columnwidth]{figures/supplementary/frankfurt00001_068208_sp.png}
  }
  \subfigure{%
    \includegraphics[width=.18\columnwidth]{figures/supplementary/frankfurt00001_068208_gt.png}
  }
  \subfigure{%
    \includegraphics[width=.18\columnwidth]{figures/supplementary/frankfurt00001_068208_cnn.png}
  }
  \subfigure{%
    \includegraphics[width=.18\columnwidth]{figures/supplementary/frankfurt00001_068208_ours.png}
  }\\[-2ex]

  \subfigure{%
    \includegraphics[width=.18\columnwidth]{figures/supplementary/frankfurt00001_082466_given.png}
  }
  \subfigure{%
    \includegraphics[width=.18\columnwidth]{figures/supplementary/frankfurt00001_082466_sp.png}
  }
  \subfigure{%
    \includegraphics[width=.18\columnwidth]{figures/supplementary/frankfurt00001_082466_gt.png}
  }
  \subfigure{%
    \includegraphics[width=.18\columnwidth]{figures/supplementary/frankfurt00001_082466_cnn.png}
  }
  \subfigure{%
    \includegraphics[width=.18\columnwidth]{figures/supplementary/frankfurt00001_082466_ours.png}
  }\\[-2ex]

  \subfigure{%
    \includegraphics[width=.18\columnwidth]{figures/supplementary/lindau00033_000019_given.png}
  }
  \subfigure{%
    \includegraphics[width=.18\columnwidth]{figures/supplementary/lindau00033_000019_sp.png}
  }
  \subfigure{%
    \includegraphics[width=.18\columnwidth]{figures/supplementary/lindau00033_000019_gt.png}
  }
  \subfigure{%
    \includegraphics[width=.18\columnwidth]{figures/supplementary/lindau00033_000019_cnn.png}
  }
  \subfigure{%
    \includegraphics[width=.18\columnwidth]{figures/supplementary/lindau00033_000019_ours.png}
  }\\[-2ex]

  \subfigure{%
    \includegraphics[width=.18\columnwidth]{figures/supplementary/lindau00052_000019_given.png}
  }
  \subfigure{%
    \includegraphics[width=.18\columnwidth]{figures/supplementary/lindau00052_000019_sp.png}
  }
  \subfigure{%
    \includegraphics[width=.18\columnwidth]{figures/supplementary/lindau00052_000019_gt.png}
  }
  \subfigure{%
    \includegraphics[width=.18\columnwidth]{figures/supplementary/lindau00052_000019_cnn.png}
  }
  \subfigure{%
    \includegraphics[width=.18\columnwidth]{figures/supplementary/lindau00052_000019_ours.png}
  }\\[-2ex]




  \subfigure{%
    \includegraphics[width=.18\columnwidth]{figures/supplementary/lindau00027_000019_given.png}
  }
  \subfigure{%
    \includegraphics[width=.18\columnwidth]{figures/supplementary/lindau00027_000019_sp.png}
  }
  \subfigure{%
    \includegraphics[width=.18\columnwidth]{figures/supplementary/lindau00027_000019_gt.png}
  }
  \subfigure{%
    \includegraphics[width=.18\columnwidth]{figures/supplementary/lindau00027_000019_cnn.png}
  }
  \subfigure{%
    \includegraphics[width=.18\columnwidth]{figures/supplementary/lindau00027_000019_ours.png}
  }\\[-2ex]



  \setcounter{subfigure}{0}
  \subfigure[\scriptsize Input]{%
    \includegraphics[width=.18\columnwidth]{figures/supplementary/lindau00029_000019_given.png}
  }
  \subfigure[\scriptsize Superpixels]{%
    \includegraphics[width=.18\columnwidth]{figures/supplementary/lindau00029_000019_sp.png}
  }
  \subfigure[\scriptsize GT]{%
    \includegraphics[width=.18\columnwidth]{figures/supplementary/lindau00029_000019_gt.png}
  }
  \subfigure[\scriptsize Deeplab]{%
    \includegraphics[width=.18\columnwidth]{figures/supplementary/lindau00029_000019_cnn.png}
  }
  \subfigure[\scriptsize Using BI]{%
    \includegraphics[width=.18\columnwidth]{figures/supplementary/lindau00029_000019_ours.png}
  }%\\[-2ex]

  \mycaption{Street Scene Segmentation}{Example results of street scene segmentation.
  (d)~depicts the DeepLab results, (e)~result obtained by adding bilateral inception (BI) modules (\bi{6}{2}+\bi{7}{6}) between \fc~layers.}
\label{fig:street_visuals-app}
\end{figure*}


\received{April 2022}
\received[revised]{July 2022}
\received[accepted]{August 2022}
% \haoyue{Comment out the Appendix when generating pdf for submission, so that "Received April 2022; revised July 2022; accepted August 2022" is at the end of the paper.}
  
\end{document}
\endinput
%%
%% End of file `sample-sigconf.tex'.
