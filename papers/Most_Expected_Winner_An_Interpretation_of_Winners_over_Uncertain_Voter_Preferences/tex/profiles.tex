\section{Uncertain Voting Profiles}
\label{sec:voting_profiles}

In classical voting theory, voters give complete rankings over candidates.
However, in practice only partial preferences may be observed, due to the voting mechanism (\eg when approval ballots or a ranking of at most $k<m$ candidates are elicited), the uncertainty in preferences themselves~\cite{marden1995analyzing}, or both.  Figure~\ref{fig:venn_diagrams_of_voting_profiles}  represents uncertainty as two distinct steps: preference generation (Figure~\ref{fig:venn_generation}) and preference observation (Figure~\ref{fig:venn_observation}). Important special cases of voting profiles are discussed next.

\begin{figure}[b!]
	\centering
	\subfloat[Generation step]{
		\label{fig:venn_generation}
		\includegraphics[width=0.22\linewidth]{figs/venn_diagram_of_probabilistic_voting_profiles}
	}
	\subfloat[Observation step]{
		\label{fig:venn_observation}
		\includegraphics[width=0.34\linewidth]{figs/venn_diagram_of_incomplete_voting_profiles}
	}~
	\caption{Uncertain voting profiles. (a) Uncertainty in the preferences by the voters themselves is represented by the ranking models in the generation step. (b) Uncertainty introduced by the preference elicitation mechanism is represented by the incomplete voting profiles in the observation step.}
	\label{fig:venn_diagrams_of_voting_profiles}
\end{figure}

\subsection{Uncertainty in profile generation}
The most general form of voter preferences over rankings is a non-parametric probability distribution such as that given in Figure~\ref{tab:profile}. 
Let $\probaVP=(\enum[n]{\model})$ denote a \e{probabilistic voting profile} where $\model_i$ is the ranking model of voter $v_i$.
A \e{possible world} of $\probaVP$ is a complete voting profile $\completeVP = (\enum[n]{\ranking})$ where each $\ranking_i$ is sampled from $\model_i$.
It is assumed that the voters cast their ballots independently, \ie $\Pr(\completeVP \mid \probaVP) = \prod_{i=1}^{n} {\Pr(\ranking_i \mid \model_i)}$.
So $\probaVP$ represents a probability distribution of its possible worlds $\Omega(\probaVP) = \set{\enum[\numPW]{\completeVP}}$.

Let $\score(c, \model)$ and $\score(c, \probaVP)$ denote the scores assigned to candidate $c$ by model $\model$ and  probabilistic voting profile $\probaVP$, respectively.
(Note that both are random variables.)
%
Partial voting profiles are a special case of probabilistic voting profiles, since they are based on the assumption that all completions are equally likely. 
Below are a few other cases.

In a \e{uniform voting profile}, denoted by $\uniVP$, no voter has any preference between any two candidates. Voter preferences are sampled from the uniform distribution over the rankings of all candidates.

There are specific important ranking models that we will describe in Section~\ref{sec:algorithms}.
The Mallows model~\cite{Mallows1957} is the best known, and it is generalized by the Repeated Insertion Model (RIM)~\cite{Doignon2004} and the ranking version of the Repeated Selection Model (RSM)~\cite{DBLP:journals/tdasci/ChakrabortyDKKR21}.
Voting profiles consisting of Mallows, RIMs, and RSMs (ranking version) are denoted by $\mallowsVP$, $\rimVP$, and $\rsmVP$, respectively.
We will see that, when the generation step is based on these models, we have a computational advantage for computing the \mew.

\subsection{Uncertainty in profile observation}

A \e{partial voting profile}, denoted by $\partialVP$, consists of partial orders and represents a uniform distribution of its completions.
What follows are important special cases.

A \e{(fully) partitioned voting profile}, denoted by $\fullparVP$, records preferences in the form of partitions: linear orders of item buckets, with no preference among the items within a bucket.

A \e{partially partitioned voting profile}, denoted by $\parparVP$, generalizes the fully partitioned preferences by only considering a subset of items, with no preference information for the missing ones.

A \e{partial chain voting profile}, denoted by $\pchainVP$, records preferences in the form of partial chains. A partial chain is a linear order of a subset of items, with no preference over the remaining items. The partial chains are a special case of the partially partitioned preferences, where each partition only consists of one item.

A \e{truncated voting profile}, denoted by $\trunVP$, records preferences in the form of truncated rankings.
Let $\ranking^{(t, b)}$ denote a truncated ranking with $t$ items at the top and $b$ items at the bottom, with no preference specified for the middle part of the ranking.
The $\ranking^{(t, b)}$ is a special case of the partitioned preferences, with $t+b+1$ partitions.

\paragraph{Combined voting profiles.}

Most research works have assumed that a partial voting profile represents a uniform distribution over its completions. However, the assumption that all completions are equally likely may not hold in practice.
We propose \e{combined voting profiles} $\VP^{\model+P}$, where each voter is associated with both her original incomplete preferences $P$ and a ranking model $\model$.  The ranking model $\model$ is her prior preference distribution, which may be obtained from historical data.
But after observing new evidence $P$, the probabilities of rankings that violate $P$ collapse to zero, while the remaining rankings that satisfy $P$ have the same relative probabilities among each other.
Formally speaking, the preferences of this voter become the posterior distribution of $\model$ conditioned on $P$. 

In another sense, the combined voting profiles are also the most general form of voting profiles (same as $\probaVP$) that unify all voting profiles so far.
For example, a partitioned voting profile $\fullparVP$ is essentially $\VP^{\text{U+FP}}$, and a RIM voting profile $\rimVP$ is essentially $\VP^{\RIM+\emptyset}$ where $\emptyset$ means that the partial orders are the empty ones.

Among the tractable cases in Section~\ref{sec:algorithms}, there are also combined voting profiles, which means that applying combined voting profiles both brings the benefit of more customized ranking distributions, and can also be practical.
