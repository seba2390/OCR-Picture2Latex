\section{Introduction}
\label{sec:intro}

Voting is a mechanism to determine winners among the candidates in an election by aggregating voter preferences. In classical voting theory, each voter gives a complete preference (most frequently, a ranking) of all candidates.   How voter preferences are aggregated is determined by a voting rule~\cite{DBLP:journals/scw/PritchardW07,DBLP:conf/aaai/GoldsmithLMP14,DBLP:journals/jasss/Seidl18}.
A prominent class of voting rules, which assign scores to candidates based on their positions in the rankings and then sum up the scores for each candidate, are positional scoring rules, on which we focus in this paper.


In practice, voter preferences may well be incomplete and represented by partial orders.
Since voting rules are defined over (complete) rankings, the solution is to replace each partial order with all of its linear extensions, each of which is a \e{completion} or a \e{possible world}\footnote{We use \e{completion} and \e{possible world} interchangeably.} of the partial order.
The preferences of all voters in an election are referred to as a \e{voting profile}.
A voting profile of complete rankings is a \e{complete voting profile}, while that of incomplete preferences is an \e{incomplete voting profile}.
Voters are assumed to cast their preferences independently.
A \e{completion} of an incomplete voting profile is a complete voting profile obtained by replacing each voter's partial order with one of its completions.

There have been various interpretations of winners proposed for this setting.
For example, Konczak \etal \cite{konczak2005voting} proposed the necessary winners (NWs) and possible winners (PWs): the
NWs are the candidates winning in all possible worlds, while the PWs are the ones winning in at least one possible world. Bachrach \etal \cite{DBLP:conf/aaai/BachrachBF10} assumed that the partial orders correspond to the uniform distribution over their completions, and evaluated the candidates by the number of possible worlds in which they win. We refer to this winner semantics as the \e{\mPw} (\mpw).

In this paper, we propose the \e{\mEw} (\mew) as an alternative winner interpretation for incomplete voter preferences under positional scoring rules. Like \mpw, it adopts the  possible world semantics of incomplete voting profiles.  However, in contrast to \mpw that determines a winner by a (weighted) count of the possible worlds in which she wins, \mew follows the principle of score-based rules that high-scoring candidates should be favored.  Specifically, \e{an \mew is the candidate who has the highest expected score in a random possible world}.

\mew and \mpw are similar in that they both aggregate election results over all possible worlds and give a balanced evaluation of the candidates.  Their difference lies in the aggregation methods, which will be discussed in detail in Section~\ref{sec:mew_vs_mpw}.
In practice, \mew and \mpw  may yield the same result, but this is not always the case.

\begin{figure}[tb!]
	\centering
	\small 
	\subfloat[A probabilistic voting profile of voters $x$ and $y$.]{
		\label{tab:profile}
		\begin{tabular}{@{}c|cccc@{}}
			\toprule
			Voter & $\ranking_1 {=} \angs{a, b, c}$ & $\ranking_2 {=} \angs{b, a, c}$ & $\ranking_3 {=} \angs{b, c, a}$ & $\ranking_4 {=} \angs{c, b, a}$ \\ \midrule
			$x$  &              $0.7$              &              $0.3$              &               $0$               &               $0$               \\
			$y$  &               $0$               &               $0$               &              $0.5$              &              $0.5$              \\ \bottomrule
		\end{tabular}
	}\\
	\subfloat[Each voter casts her vote independently of the other, leading to 4 possible worlds and 4 possible outcomes for (co-)winners under the plurality rule. The probability of a possible world is $\Pr(\ranking_x, \ranking_y)$. Under the plurality rule, candidate $a$ is the \mpw with a winning probability of $\Pr (\ranking_1, \ranking_3) {+} \Pr (\ranking_1, \ranking_4) {=} 0.7$, while candidate $b$ is the \mew with expected score $\Pr(\ranking_2 \mid x){+}\Pr(\ranking_3 \mid y) {=} 0.3{+}0.5{=}0.8$.]{
		\label{tab:possible_worlds}
		\begin{tabular}{@{}c|c|c@{}}
			\toprule
			Possible world                 & (Co-)winners by Plurality & $\Pr(\ranking_x, \ranking_y)$ \\ \midrule
			$x$ casts $\ranking_1$, $y$ casts $\ranking_3$ &        $\{a, b\}$         &    $0.7 \times 0.5 = 0.35$    \\
			$x$ casts $\ranking_1$, $y$ casts $\ranking_4$ &        $\{a, c\}$         &    $0.7 \times 0.5 = 0.35$    \\
			$x$ casts $\ranking_2$, $y$ casts $\ranking_3$ &          $\{b\}$          &    $0.3 \times 0.5 = 0.15$    \\
			$x$ casts $\ranking_2$, $y$ casts $\ranking_4$ &        $\{b, c\}$         &    $0.3 \times 0.5 = 0.15$    \\ \bottomrule
		\end{tabular}
	}~
	\caption{A probabilistic voting profile of 2 voters.}
	\label{fig:example_profile}
\end{figure}

\begin{example}\label{ex:1}
Figure~\ref{fig:example_profile} gives an example where \mew and \mpw select different winners in an election with two voters and three candidates, under the plurality rule. 
In this election, each voter produces a full ranking drawn from a probability distribution over  $\ranking_1 = \langle a, b, c \rangle$, $\ranking_2 = \langle b, a, c \rangle$,  $\ranking_3 = \langle b, c, a \rangle$, $\ranking_4 = \langle c, b, a \rangle$, with probabilities of each ranking for each voter given in Figure~\ref{tab:profile}.
Since voter preferences are probability distributions over rankings, the corresponding voting profile is named a \e{probabilistic voting profile}.

Let $\Pr(\ranking_x, \ranking_y)$ denote the probability that voters $x$ and $y$ cast rankings $\ranking_x$ and $\ranking_y$, respectively.  If we assume that $x$ and $y$ cast their votes independently, then $\Pr(\ranking_x, \ranking_y) = \Pr(\ranking_x \mid x) \cdot \Pr(\ranking_y \mid y)$. For example, $\Pr (\ranking_1, \ranking_3) = 0.7 \cdot 0.5 = 0.35$.
The voting profile in Figure~\ref{tab:profile} generates 4 possible worlds in Figure~\ref{tab:possible_worlds}.
The plurality rule rewards a candidate with 1 point every time she is ranked at the top of a ranking in the profile.
So, in the possible world of $\Pr (\ranking_1, \ranking_3)$, candidate $a$ obtains 1 point from $\ranking_1$, candidate $b$ obtains 1 point from $\ranking_3$, and both of them become (co-)winners in this possible world.
After enumerating all 4 possible worlds, we find that candidate $a$ is the \mpw with probability 0.7 to be a (co-)winner, while candidate $b$ is the \mew with an expected score of 0.8 points.

If the Borda rule, which rewards a candidate with the number of candidates ranked below her, is used, then, in the possible world of $\Pr (\ranking_1, \ranking_3)$, candidate $a$ would obtain 2 points from $\ranking_1$, while candidate $b$ would obtain 1 point from $\ranking_1$ and 2 points from $\ranking_3$. Candidate $b$ would then be the only NW of this profile, and she would also be the \mew with an expected score of 2.8 points.
\end{example}

{\bf Contributions and Roadmap.} We present related work in Section~\ref{sec:related}, and provide the necessary background on preferences and voting in Section~\ref{sec:preliminaries}.  

Then, in Section~\ref{sec:voting_profiles}, we present our \emph{first contribution: a unified framework for representing uncertainty in voter preferences}, where we distinguish between the uncertainty from voters themselves (\eg their preferences are actually probability distributions~\cite{marden1995analyzing}) and that due to the voting mechanism (\eg approval ballots do not allow the voters to fully reveal their preferences). We classify voting profiles into probabilistic profiles, where uncertainty is due to the voters themselves, incomplete profiles, where uncertainty is due to the voting mechanism, and combined  profiles, with both kinds of uncertainty. This classification gives us a framework within which to study the complexity of identifying \mew, by computing expected scores of the candidates.

In Section~\ref{sec:mew}, we discuss several alternative interpretations of \mew and formally state the \mew problem.  Then, we present our \emph{second contribution: an investigation of the computational complexity of \mew, and its related \ESC (\esc) problem}, in Section~\ref{sec:complexity}. The \mew problem turns out to be FP$^{\#P}$-complete under plurality, veto, and $k$-approval rules for the general case of uncertain profiles.\footnote{Recall that FP$^{\#P}$ is a class of functions efficiently solvable with an oracle to some \#P problem. A function $f$ is FP$^{\#P}$-hard if there is a polynomial-time Turing reduction from any FP$^{\#P}$ function to $f$.}

In Sections~\ref{sec:algorithms} and~\ref{sec:optimize}, we present our \emph{third contribution:  exact solvers for \mew computation over different voting profiles}. We first give a solver with exponential complexity for the partial voting profiles studied in Section~\ref{sec:complexity}. We then identify many interesting cases, such as the RIM~\cite{Doignon2004} voting profiles and the combined profiles of Mallows~\cite{Mallows1957} and partitioned preferences, where it is tractable to compute the expected scores and determine the \mew (Section~\ref{sec:algorithms}). In Section~\ref{sec:optimize}, we present \rev{performance optimizations to speed up the computation of MEW for large voting profiles.} 

In Section~\ref{sec:exp} we present our \emph{fourth contribution: an extensive experimental evaluation of the proposed solvers}, demonstrating that our proposed methods are practical. 

\rev{In Section~\ref{sec:mew_vs_mpw} we present a thorough comparison between \mew and \mpw, describing cases where \mew treats candidates differently from \mpw, and highlighting the computational advantages of \mew that can lead to more efficient computation in practice.}

We conclude and propose future directions in Section~\ref{sec:conc}.