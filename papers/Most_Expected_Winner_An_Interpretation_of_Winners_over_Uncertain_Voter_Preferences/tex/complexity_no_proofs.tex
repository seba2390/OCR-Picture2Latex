\section{Hardness of \esc}
\label{sec:complexity}

This section investigates the complexity of the \esc problem. We first prove the hardness of two closely related problems, the Fixed-rank Counting Problem and the Rank Estimation Problem, and then demonstrate that the \esc problem is hard as well. Proofs can be found in Appendix~\ref{sec:appendix:proofs}.

\subsection{Fixed-rank Counting Problem}

Counting the number of linear extensions of a partial order is well-known to be \#P-complete~\cite{DBLP:conf/stoc/BrightwellW91}.
The Fixed-rank Counting Problem (FCP) is interested in the number of linear extensions where an item is placed at a specific rank.
Let $\Omega(\partialOrder)$ denote the set of linear extensions of $\partialOrder$, and $N(c {\rightarrow} j \mid \partialOrder)$ denote the number of linear extensions in $\Omega(\partialOrder)$ where item $c$ is placed at rank $j$.

\begin{definition}[FCP]
    Given a partial order $\partialOrder$ over $m$ items, an item $c$, and an integer $j \in [1, m]$, calculate $N(c {\rightarrow} j \mid \partialOrder)$, the number of linear extensions of $\partialOrder$ where item $c$ is placed at rank $j$.
\end{definition}

De Loof's doctoral dissertation~\cite{Loof2009EfficientCO} discusses this problem (Section 4.2.1), proposing exact and approximate algorithms, but does not provide proof of complexity.

\def\theoremFCPhardnessOverPartialOrders{
    The FCP is \#P-complete.
}

\begin{theorem} \label{theorem:FCP_shaPcomplete_over_partialOrders}
    \theoremFCPhardnessOverPartialOrders
\end{theorem}

The hardness of FCP facilitates the hardness proofs for the Rank Estimation Problem.

\subsection{Rank Estimation Problem}

Now we move on to the Rank Estimation Problem (REP).
This problem can be regarded as the probabilistic version of the FCP.
It calculates the probability that a given item is placed at a specific rank.
But the REP is generalized from partial orders to arbitrary ranking distributions.

\begin{definition}[REP]
    Given a ranking model $\model$ over $m$ items, an item $c$, and an integer $j \in [1, m]$, calculate $\Pr(c {\rightarrow} j \mid \model)$, the probability that item $c$ is placed at rank $j$, by $\model$.
\end{definition}

For the convenience of discussions in the rest of this Section, we also define two special cases of the REP where items are placed at the top and bottom of the linear extensions.

\begin{definition}[REP-$t$]
    Given a ranking model $\model$ of $m$ items and an item $c$, calculate $\Pr(c {\rightarrow} 1 \mid \model)$, the probability that item $c$ is placed at the top of a linear extension, by $\model$.
\end{definition}

\begin{definition}[REP-$b$]
	Given a ranking model $\model$ of $m$ items and an item $c$, calculate $\Pr(c {\rightarrow} m \mid \model)$, the probability that item $c$ is placed at the bottom of a linear extension, by $\model$.
\end{definition}

Lerche and S{\o}rensen~\cite{lerche2003evaluation} proposed an approximation for the REP over partial orders, but did not provide a formal complexity proof.
Bruggemann and Annoni~\cite{bruggemann2014average} and De Loof \etal \cite{de2011approximation} considered a related problem, calculating the expected rank of an item in the linear extensions of a partial order.
These works focused on approximation, lacking complexity proofs as well.

With the help of Theorem~\ref{theorem:FCP_shaPcomplete_over_partialOrders}, it turns out that the REP and its two variants are all FP$^{\#P}$-complete over partial orders.

\def\lemmaREPtHardnessOverPartialOrders{
    If ranking model $\model$ is a partial order $\partialOrder$ of $m$ items representing a uniform distribution of $\Omega(\partialOrder)$, the REP-t is FP$^{\#P}$-complete.
}

\begin{lemma} \label{lemma:REPt_shaPcomplete_over_partialOrders}
    \lemmaREPtHardnessOverPartialOrders
\end{lemma}

\def\lemmaREPbHardnessOverPartialOrders{
    If ranking model $\model$ is a partial order $\partialOrder$ of $m$ items representing a uniform distribution of $\Omega(\partialOrder)$, the REP-b is FP$^{\#P}$-complete.
}

\begin{lemma} \label{lemma:REPb_shaPcomplete_over_partialOrders}
	\lemmaREPbHardnessOverPartialOrders
\end{lemma}

\def\theoremRepHardnessOverPartialOrders{
    If ranking model $\model$ is a partial order $\partialOrder$ of $m$ items representing a uniform distribution of $\Omega(\partialOrder)$, the REP is FP$^{\#P}$-complete.
}

\begin{theorem}
    \label{theorem:REP_shaPcomplete_over_partialOrders}
    \theoremRepHardnessOverPartialOrders
\end{theorem}

\subsection{Complexity of \esc}

\esc is closely related to REP.  Firstly,  \esc  is no harder than REP over general voting profiles (Theorem~\ref{theorem:reduction}), which lays the foundation for the identification of tractable cases in Section~\ref{sec:algorithms}.

\def\theoremReductionEscToRep{
    Given a general voting profile $\probaVP$ and a positional scoring rule $\vr_m$, the \esc problem can be reduced to the REP.
}

\begin{theorem} \label{theorem:reduction}
    \theoremReductionEscToRep
\end{theorem}

\def\theoremRepMewEquivalentUnderKapproval{
    The REP for rank $k$ is equivalent to the \esc problem over either one or both of the $(k-1)$-approval and $k$-approval rules.
}

\begin{theorem} \label{theorem:REP_MEW_equivalent_under_k_approval}
    \theoremRepMewEquivalentUnderKapproval
\end{theorem}

Theorem~\ref{theorem:REP_MEW_equivalent_under_k_approval} provides an insight into the relation between REP and \esc in terms of computational complexity.
If a solver is available for the \esc problem over a collection of $k$-approval votes, this solver is computationally equivalent to the REP solver.
Note that Theorem~\ref{theorem:REP_MEW_equivalent_under_k_approval} is not limited to partial voting profiles.

\def\theoremHardnessForEscGivenPartialOrderAndPlurality{
    Given a partial voting profile $\partialVP$, a distinguished candidate $c$, and plurality rule $\vr_m$, the \esc problem of calculating $\mathds{E}(\score(c \mid \partialVP, \vr_m))$ is FP$^{\#P}$-complete.
}

\begin{theorem} \label{theorem:shaPcomplete_for_expected_score_given_partialOrder_and_plurality}
    \theoremHardnessForEscGivenPartialOrderAndPlurality
\end{theorem}

\def\theoremHardnessForEscGivenPartialOrderAndVeto{
    Given a partial voting profile $\partialVP$, a distinguished candidate $c$, and veto rule $\vr_m$, the \esc problem of calculating $\mathds{E}(\score(c \mid \partialVP, \vr_m))$ is FP$^{\#P}$-complete.
}

\begin{theorem} \label{theorem:shaPcomplete_for_expected_score_given_partialOrder_and_veto}
	\theoremHardnessForEscGivenPartialOrderAndVeto
\end{theorem}

\def\theoremHardnessForExpectedScoreGivenPartialOrderAndKApproval{
    Given a partial voting profile $\partialVP$, a distinguished candidate $c$, and $k$-approval rule $\vr_m$, the \esc problem of calculating $\mathds{E}(\score(c \mid \partialVP, \vr_m))$ is FP$^{\#P}$-complete.
}

\begin{theorem} \label{theorem:shaPcomplete_for_expected_score_given_partialOrder_and_k_approval}
    \theoremHardnessForExpectedScoreGivenPartialOrderAndKApproval
\end{theorem}

The three theorems above demonstrate the hardness of the \esc problem over partial voting profiles, under plurality, veto, and $k$-approval, respectively.
In particular, \esc is FP$^{\#P}$-complete even under plurality (Theorem~\ref{theorem:shaPcomplete_for_expected_score_given_partialOrder_and_plurality}).
