\section{Comparing \mew and \mpw}
\label{sec:mew_vs_mpw}

\rev{We now conduct a thorough comparison between \mew and \mpw to better demonstrate their differences.}
\mew and \mpw can be interpreted as different aggregation approaches across possible worlds.
Recall that given a general voting profile $\probaVP$, $\Omega(\probaVP) = \set{\completeVP_1, \ldots, \completeVP_\numPW}$ is the set of its possible worlds of $\probaVP$, and each $\completeVP_i$ is associated with a probability $p_i = \Pr(\completeVP_i \mid \probaVP)$ and $\sum_{i=1}^{\numPW} p_i = 1$.
Now let's see how the performance of a candidate $c$ is aggregated across possible worlds. Let $\mathds{1}()$ be the indicator function.

\begin{itemize}
  \itemsep -0.1em
  \item \mew: $\mathds{E}(\score(c, \probaVP)) = \sum_{i=1}^{\numPW} \score(c, \completeVP_i) \cdot p_i$
  \item \mpw: $\Pr(c \textsf{ wins} \mid \probaVP) = \sum_{i=1}^{\numPW} \mathds{1}(c \textsf{ wins} \mid \completeVP_i) \cdot p_i$
\end{itemize}

\mew estimates the average performance of a candidate, while \mpw estimates the probability that she wins.
As a result, \mpw ignores the possible worlds in which the candidate cannot win, \rev{putting certain candidates at a disadvantage}.

\begin{figure}[b!]
	\centering
	\subfloat[\revv{10 voters}, varying \#candidates]{
		\label{fig:mpw:candidates}
		\includegraphics[width=0.3\linewidth]{figs/synthetic_mpw_parallel_pluarlity_10_voters__time_vs_m}
	}\hspace{5em}
	\subfloat[\revv{9 candidates}, varying \#voters]{
		\label{fig:mpw:voters}
		\includegraphics[width=0.3\linewidth]{figs/synthetic_mpw_parallel_pluarlity_9_candidates__time_vs_n}
	}
	\caption{\revv{Average time of parallel MPW and MEW, using 48 worker processes,  over partial voting profiles, fixing $\phi = 0.5$ and $p_{max}=0.1$, using the plurality rule. \mew scales much better than \mpw,  with both \#candidates and \# voters.}}
	\label{fig:mpw}
\end{figure}

\begin{example}
  \rev{Let $\probaVP = \set{\model_0}$ be a single-voter voting profile over 4 candidates $\set{a, b, c, d}$. 
  Assume $\model_0$ is a ranking model where $\Pr(\angs{b,a,c,d}) = \Pr(\angs{c,a,b,d}) = \Pr(\angs{d,a,b,c}) = 1/3$.
  Under Borda rule, \mew favors candidate $a$ who, despite losing in every possible world, enjoys the highest expected score of 2, while the MPWs are $\set{b, c, d}$, each winning in one possible world.}
\end{example}

\rev{\mew also quantifies the performance of candidates more granularly than \mpw, as illustrated in the next example.}

\begin{example}
  \rev{Let $\partialVP = \set{\partialOrder_0}$ be a single-voter voting profile over 4 candidates $\set{a, b, c, d}$. 
    Assume $\partialOrder_0 = \set{a \succ b, b \succ c, b \succ d}$.
    Under the Borda rule, \mew and \mpw agree that candidate $a$ is the winner, but they disagree on the performance of candidate $b$. \mpw cannot differentiate between $b$, $c$, and $d$, since none of them wins in any possible world, but \mew believes that $b$ outperforms $c$ and $d$, since $b$ has a higher expected score than $c$ and $d$.}
\end{example}

\rev{\mew also has a computational advantage over \mpw, making it practical for real-world applications: Although both are intractable in the general case, \mew enjoys linear complexity in the number of voters, while the running time of \mpw grows exponentially. We implemented the \mpw solver based on the \e{VotingResult} algorithm by Hazon \etal \cite{DBLP:journals/ai/HazonAKW12}, with the modification that the solver iterates over the completions of each voter's partial order, while the original algorithm assumes that each voter gives an explicit probability distribution over the rankings.} \revv{We compared performance of the parallel versions of \mpw and \mew under plurality, fixing the number of voters to 10 and varying the number of candidates from 3 to 9 (Figures~\ref{fig:mpw:candidates}), then fixing the number of candidates to 9 and varying the number of voters from 1 to 10 (Figure~\ref{fig:mpw:voters}). Observe that \mew scales much better than \mpw.}
\revv{The corresponding results under the Borda rule are available in Appendix~\ref{sec:appendix:experiments}.}

\section{Concluding Remarks}
\label{sec:conc}
 
In this paper we modeled uncertainty in voter preferences with the help of a framework that distinguishes between uncertainty in preference generation and uncertainty in preference observation, unifying incomplete and probabilistic voting profiles.  
We then proposed the \mEw (\mew) semantics for  positional scoring rules and established the theoretical hardness of this problem.  We identified tractable cases with the help of the uncertainty framework for voting profiles, and developed solvers.

Much exciting future work remains.  For example, the hardness of \esc is proved over only plurality, veto, and $k$-approval, which calls for investigation of other positional scoring rules such as Borda. When  \mew is intractable, it may be necessary to develop approximate solvers. \mew can also be extended to other score-based rules, such as Simpson and Copeland.
Another direction is to consider voter preferences represented by additional ranking models~\cite{marden1995analyzing} such as the Plackett-Luce (PL)~\cite{luce1959individual,plackett1975analysis} and  Thurstone-Mosteller (TM)~\cite{Thurstone1927-THUALO-2,RePEc:spr:psycho:v:16:y:1951:i:1:p:3-9}. For example, others~\cite{DBLP:conf/aaai/NoothigattuGADR18,DBLP:conf/uai/ZhaoLWKMSX18} have studied a preference aggregation method over PL models that is closely related to the \mew over PL models, and we plan to investigate this connection further in the future.