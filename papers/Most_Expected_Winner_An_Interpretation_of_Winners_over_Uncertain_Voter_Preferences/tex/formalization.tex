\section{\mEw}
\label{sec:mew}

For a general voting profile and a positional scoring rule, the performance of a candidate can be quantified by the expectation of her score in a random possible world.  We define this formally below.

\begin{definition} [\mew] \label{def:mew}
  Given a general voting profile $\probaVP$ and a positional scoring rule $\vr_m$, candidate $w$ is a \mEw, \ifff, $\mathds{E}(\score(w, \probaVP)) = \max_{c \in C} \mathds{E}(\score(c, \probaVP))$.
\end{definition}

We denote the set of \mEws by $\mew(\vr_m, \probaVP)$.

\subsection{Alternative Interpretations}
\label{sec:mew:alternative_interpretations}

To gain an intuition for \mEw~(\mew), we will now give two winner definitions that are equivalent to \mew. Proofs can be found in Appendix~\ref{sec:appendix:proofs}.

\paragraph{Least Expected Regret Winner.}

The MEW can also be regarded as the candidate who minimizes the expected regret in a random possible world.
Let $\txt{Regret}(w, \completeVP)$ denote the regret value of choosing $w \in C$ as the winner given a complete voting profile $\completeVP$.
\[
\txt{Regret}(w, \completeVP) = \max_{c \in C} \score(c, \completeVP) - \score(w, \completeVP)
\]
Accordingly, the regret value $\txt{Regret}(w, \probaVP)$ over a probabilistic voting profile becomes a random variable.
\[
\mathds{E}(\txt{Regret}(w, \probaVP)) = \sum_{\completeVP \in \Omega(\probaVP)} \txt{Regret}(c, \completeVP) \cdot \Pr(\completeVP \mid \probaVP)
\]

Candidate $w$ is a \e{Least Expected Regret Winner}, \ifff, she minimizes the expected regret $\mathds{E}(\txt{Regret}(w, \probaVP))$.

\def\theoremLeastExpectedRegretWinner{
  Least Expected Regret Winner is equivalent to \mew.
}

\begin{theorem}
  \label{theorem:least_expected_regret_winner}
  \theoremLeastExpectedRegretWinner
\end{theorem}

\revv{Our use of expected regret as an alternative interpretation of MEW is inspired by Lou and Boutillier~\cite{DBLP:conf/ijcai/LuB11a}, who were the first to use regret in winner determination.  They proposed  MMR, the winner that minimized regret in the worst-case completion.  In contrast, MEW minimizes regret in expectation, across all completions.}

\paragraph{Meta-Election Winner.}
Recall that a voting profile $\probaVP$ represents a probability distribution of possible worlds $\Omega(\probaVP) = \set{\completeVP_1, \ldots, \completeVP_\numPW}$.
The Meta-Election Winner is defined as the candidate who wins a meta election with a large meta profile $\completeVP_M = (\completeVP_1, \ldots, \completeVP_\numPW)$ where rankings in $\completeVP_i$ are weighted by $\Pr(\completeVP_i \mid \probaVP)$.

\def\theoremGiganticElectionWinner{
  Meta-Election Winner is equivalent to \mew.
}

\begin{theorem}
  \label{theorem:gigantic_election_winner}
  \theoremGiganticElectionWinner
\end{theorem}

\subsection{Problem Statement}
\label{sec:problem}

The \mew is determined based on the expected performance of the candidates. \rev{Thus, the winner determination problem of \mew can be reduced to the problem of \ESC (\esc), stated below and addressed in the remainder of the paper.}

\begin{definition} [\esc] \label{def:esc}
  Given a general voting profile $\probaVP$, a positional scoring rule $\vr_m$, and a candidate $c \in C$, compute $\mathds{E}(\score(c, \probaVP))$ the expected score of the candidate $c$.
\end{definition}
