\section{Related Work}
\label{sec:related}

\e{Winner semantics under incompleteness.} Among the winner interpretations for incomplete preferences, the most thoroughly studied are the necessary and possible winners~\cite{konczak2005voting}. A candidate is a necessary winner (NW) if she wins in every possible world;  she is a possible winner (PW) if she wins in at least one possible world.
Chakraborty \etal~\cite{DBLP:journals/tdasci/ChakrabortyDKKR21} recently developed practical techniques for NW and PW computation. NW and PW have substantial shortcomings: The requirement for NW is so strict that there are often no  winners available in a voting profile under this interpretation, while the requirement for PW does not differentiate between a candidate who only wins in one possible world and another candidate who only loses in one possible world. To address these limitations, alternative winner semantics for uncertain preferences have been proposed in the literature, discussed next.

Bachrach \etal \cite{DBLP:conf/aaai/BachrachBF10} assume that an incomplete voting profile of partial orders represents a uniform distribution over its completions, and prefer the candidates who enjoy victory in more possible worlds. This winner semantics is named the \e{\mPw} (\mpw).
While this semantics is well defined under any voting rule, and while it can be extended in a straight-forward way to incorporate the probability of a completion of a voting profile, computing a winner under \mpw is known to be intractable already under plurality~\cite{DBLP:conf/aaai/BachrachBF10}. Kenig and Kimelfeld~\cite{DBLP:conf/aaai/KenigK19} study the probability of the complement event, namely, losing an election, and devise an approach based on the Karp-Luby-Madras algorithm~\cite{DBLP:journals/jal/KarpLM89} for multiplicative polynomial-time approximations.

Hazon \etal \cite{DBLP:journals/ai/HazonAKW12} also investigate \mpw but in a different setting, where voter preferences are specified explicitly by rankings and their associated probabilities. They prove that it is \#P-hard to compute the winning probabilities under plurality, k-approval, Borda, Copeland, and Bucklin, and provide an approximation algorithm.

Imber and Kimelfeld~\cite{DBLP:conf/atal/ImberK21a} investigate the minimal and maximal possible ranks of candidates after rank aggregation of partial voting profiles, and prove intractability for every positional scoring rule.

\e{Preference models.} In this paper we model the uncertainty in voter preferences using a distance-based model known as the Mallows~\cite{Mallows1957}, and its generalizations, the Repeated Insertion Model~\cite{Doignon2004} and the Repeated Selection Model~\cite{DBLP:journals/tdasci/ChakrabortyDKKR21}.
Others have considered uncertainty in voter preferences under the Random Utility Models (RUMs) such as the Thurstone-Mosteller (TM)~\cite{Thurstone1927-THUALO-2,RePEc:spr:psycho:v:16:y:1951:i:1:p:3-9} and the Plackett-Luce (PL)~\cite{plackett1975analysis,luce1959individual}.
RUMs quantify the preferences over each item with a modal utility randomized with noise (e.g., Gaussian noise for TM and Gumbel distributions for PL), and the modal utilities are regarded as the ground truth.

Two recent papers proposed preference aggregation semantics for PL and TM models that are similar to \mew, in that they evaluate the candidates in an election based on their expected utility. 
Noothigattu \etal \cite{DBLP:conf/aaai/NoothigattuGADR18} first aggregate  preferences over the TM or PL models that correspond to each voter into a summary model to represent the entire voter base (without explicit use of a scoring rule at this step), and then select the highest modal utility candidate.  The authors show that their aggregation method is equivalent to Borda and Copeland.
Zhao \etal \cite{DBLP:conf/uai/ZhaoLWKMSX18}  apply randomized voting rules that sample a winner from the candidates with a probability proportional to their expected scores.  They demonstrate that the expected utilities of the candidates  can be determined efficiently under plurality and Borda, for PL models.  These papers share motivation with our work, but they do not consider distance-based preference models such as the Mallows, or their popular generalization like RIM, and do not study the complexity of winner determination for specific kinds of uncertain voting profiles.  Understanding the complexity of evaluation of \mew for TM and PL models for different kinds of incomplete voting profiles and voting rules is an interesting direction for future work.

\e{Querying probabilistic preferences.}
Voter preferences are a special case of preference data that has been studied in the database community~\cite{DBLP:journals/pvldb/JacobKS14}. Kenig \etal \cite{DBLP:conf/pods/KenigKPS17} propose RIMPPD, a database framework to incorporate probabilistic preferences represented by RIM models, and identify a class of tractable queries. Then Kenig \etal \cite{DBLP:conf/aaai/KenigIPKS18} further optimize the query engine with lifted inference. For a more general class of queries that are intractable in RIMPPD, Ping \etal \cite{DBLP:journals/pvldb/PingSK20} develop a number of exact solvers, as well as approximate techniques based on Multiple Importance Sampling.  In this work, we build on some of the technical insights of these papers, and develop solvers for \mew.