\section{Discussions and conclusions}
\label{sect:discuz}

In this work, together with a twin paper \citep{Hernandez-Aguayo:2021_twin_paper}, we have introduced a new, fast and accurate modified gravity simulation code, \textsc{mg-glam}, which is based on the highly-optimised parallel particle-mesh $N$-body code \textsc{glam} \citep{Klypin:2017iwu}.
We have focused on the numerical implementation of three representative classes of conformally coupled scalar field models, including two thin-shell screening models, $f(R)$ gravity and symmetrons, and a coupled quintessence model with no screening. In the case of $f(R)$ gravity, we have extended earlier simulation studies to include more general parameter choices, e.g., $n=0,2$, by generalising an efficient algorithm developed for the $n=1$ case in \cite{Bose:2016wms} to these new cases. The twin paper \citep{Hernandez-Aguayo:2021_twin_paper} explores \ac{MG} models with derivative coupling terms, including the {DGP} and K-mouflage models. Altogether, the \textsc{mg-glam} code not only covers several of the most popular \ac{MG} models in the literature, but can also serve as prototypes that can be %directly extended
easily extended to work for other leading classes of \ac{MG} models, such as chameleons, Galileon gravity and coupled quintessence models with other user-specified potentials and coupling functions.


%We studied three representative classes of conformally coupled models, including two thin-shell screening models, $f(R)$ gravity and symmetrons, and a coupled quintessence model. 
We have performed a series of tests to check that our implementation of the multigrid solvers works correctly, using different density configurations for which we can obtain analytical or independent numerical solutions of the scalar field, and found that the numerical solutions given by \textsc{mg-glam} agree very well with them in all cases. We have shown that, using only two V-cycles, we can achieve convergence for the nonlinear equations in the \ac{MG} models considered. 
Also, we have compared the solutions of the background scalar field and the modified expansion rate in the coupled quintessence model obtained using \textsc{mg-glam} and \href{https://camb.info/}{\textsc{camb}}, finding excellent agreement between both codes. 
Finally, we have compared the power spectrum enhancement and the abundance of dark matter haloes for the $f(R)$ model predicted by \textsc{mg-glam} and the \textsc{mg-arepo} code. 
In general, \textsc{mg-glam} is able to reproduce the predictions of these quantities by \textsc{mg-arepo} and \textsc{ecosmog} simulations with sufficiently high accuracy for the cosmological applications of interest to us, in spite of taking only a tiny fraction of the time needed for the latter codes. For example, with $1024^3$ particles in a box of size $512h^{-1}\mathrm{Mpc}$, \textsc{mg-glam} simulations can accurately predict $\Delta P/P_{\rm GR}$ at $k\lesssim3h\mathrm{Mpc}^{-1}$ and the HMF down to $10^{12.5}h^{-1}M_\odot$, with about $1\%$ of the computational costs for \textsc{mg-arepo} and \textsc{ecosmog}. 

We have run a large suite of $f(R)$ cosmological simulations for 10 models with $|f_{R0}|$ logarithmically spaced in $[-6.00, -4.50]$ and $n = 0, 1, 2$, and carried out the simulations for five symmetron models and three coupled quintessence models. 
With this large suite of \ac{MG} simulations we are able to study in great detail the modified gravity effects, including that of the screening mechanisms, on structure formation, as we have shown in the nonlinear matter power spectra and halo mass functions.
In particular, the large number of $f(R)$ gravity runs demonstrate, with fine details, how the effect of the chameleon screening mechanism depends on not only the present-day scalar field value, $f_{R0}$, but also the parameter $n$ which has been less explored to date.


The development of \textsc{mg-glam} will help in the construction of a large number of galaxy mock catalogues in MG theories for Stage-IV galaxy surveys, such as DESI and Euclid. 
Owing to its high efficiency and accuracy, this code can be used to perform $\mathcal{O} (100)$ large ($L > 1.0 \, h^{-1} \, \mathrm{Gpc}$ at least) and high-resolution ($m_{\rm p} < 10^{10} \, h^{-1} \, M_{\odot}$) simulations for each modified gravity model, with minimal computational cost. 
These will allow for variations of not only the gravitational but also cosmological parameters, and subsequently the construction of accurate emulators for various physical quantities in different gravity models. 
This will open up a wide range of possibilities for future works to test gravity using cosmological observations. 

One of the main potential applications of \textsc{mg-glam} simulations is the study of various galaxy clustering statistics \cite[e.g.][]{Alam:2020jdv} based on the mock galaxy catalogues mentioned above. \textsc{mg-glam} will have the flexibility to be run at different resolutions, tailored to the different observables and/or galaxy types. It will also have the advantage of allowing different classes of \ac{MG}, as well as dynamical dark energy \cite{Klypin:2020tud} models, to be studied with equal depths and fine details. In a series of upcoming papers, we will visit this topic, starting with the prescriptions to populate dark matter haloes with galaxies, as well as a more detailed study of halo properties, including halo clustering.