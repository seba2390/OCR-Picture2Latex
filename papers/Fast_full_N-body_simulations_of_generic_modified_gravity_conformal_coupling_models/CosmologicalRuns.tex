\section{Cosmological runs}
\label{sect:cos_runs}

The objective of \textsc{mg-glam} is the very fast generation of $N$-body simulations for a wide range of modified gravity models. In this section, we will present some examples of cosmological runs using this code. In particular, we will run a very large suite of $f(R)$ simulations with different parameter values of $n$ and $f_{R0}$. These simulations only take a small fraction of time of a single high-resolution run of \textsc{mg-arepo} or \textsc{ecosmog} for the box size and mass resolution.


The inventory of the cosmological runs we have performed is 
\begin{itemize}
    \item $f(R)$ gravity runs with $n = 0, 1$ and $2$ and $\log_{10}(|f_{R0}|)$ in $10$ bins linearly spaced in the range $[-6.00, -4.50]$. One realisation for each model.
    \item Ten realisations of $f(R)$ gravity runs with $n=0$ and 1 and $-\log_{10} (|f_{R0}|) = 5.00$.
    \item five symmetron models with fixed $a_\ast=0.33$ and $\beta_\ast=1$, with different values of $\xi$ given by $c\xi/H_0=0.5, 1, 2, 2.5, 3$.
    \item three variants of the coupled quintessence model described in Sect.~\ref{subsect:csf}, with $(\alpha,\beta)$ equal to $(0.1,-0.1)$, $(0.1,-0.2)$ and $(0.5,-0.2)$ respectively. 
    \item For each \ac{MG} simulation, we have a  counterpart $\Lambda$CDM run with the same simulation specifications of cosmological parameters. We will label these runs as `GR' runs, to contrast with `\ac{MG}' runs, even though none of our simulations is really general relativisic.
\end{itemize}
For all simulations, we followed the evolution of $1024^3$ particles in a cubic box with size $512 \, h^{-1}\mathrm{Mpc}$ using a grid with $2048^3$ cells. The non-MG cosmological parameters are from the Planck 2015 \citep{Ade:2015xua} best-fitting  $\Lambda$CDM parameters: $$\{\Omega_{\rm b}, \Omega_{\rm m}, h, n_s, \sigma_8\} = \{0.0486,0.3089,0.6774,0.9667,0.8159\}.$$ The ICs of both the \ac{GR} and \ac{MG} runs are generated on the fly from the same $\Lambda$CDM linear perturbation theory power spectrum at $z_{\rm init} = 100$, which itself is generated using the \href{https://camb.info/}{\textsc{camb} code}. We have used the same ICs for \ac{GR} and \ac{MG} simulations (for the same realisation),  since the \ac{MG} effect is very weak at $z > 100$, so that the linear matter power spectrum at $z_{\rm ini}=100$ is nearly identical to that of $\Lambda$CDM.


\subsection{$f(R)$ gravity}
\label{subsect:fR_runs}

We have measured the matter power spectra $P_{\rm mm} (k)$ and halo mass functions (HMF) at $z = 0$. The results are shown in Fig.~\ref{fig:Pk_mm_z0_fRn012} for the matter power spectra and Fig.~\ref{fig:hmf_z0_fRn012} for the halo mass functions. The relative differences for $P_{\rm mm}(k)$ and HMF between $f(R)$ and \ac{GR} are displayed in the lower subpanels.

Fig.~\ref{fig:Pk_mm_z0_fRn012} shows that the matter clustering is boosted by $3$-$40\%$ due to the fifth force, but the boost is scale-dependent and is weak on very large scales ($k \lesssim 0.03 \, h \, \mathrm{Mpc}^{-1}$).
The $P_{\rm mm}(k)$ enhancement, $\Delta P / P_{\mathrm{GR}}$, depends qualitatively on the value of $|f_{R0}|$.
When $|f_{R0}|$ is small so that the \ac{MG} effect is weak, $\Delta P / P_{\mathrm{GR}}$ increases monotonically with $k$. On the other hand, when the \ac{MG} effect is strong, the fractional difference of matter power spectra no longer monotonically increases with $k$, but goes down at small scales after reaching some peak value at $k \sim 1 \, h \,\mathrm{Mpc}^{-1}$ (although on even smaller scales the $P_{\rm mm}(k)$ enhancement increases again for some models, we only focus on the scales $k \lesssim 3 \, h\,\mathrm{Mpc}^{-1}$ given the fixed simulation resolution, cf.~Sec.~\ref{subsect:comparions}).
This behaviour can be explained in the context of the halo model \citep{Cooray:2002PhR...372....1C}, which assumes that on small scales the matter power spectrum is determined mainly by the matter distribution inside dark matter haloes (the one-halo term).
\begin{itemize}
    \item In the regime of weak \ac{MG} effect, haloes are well screened inside so that particles do not feel the fifth force during most of their evolution. 
    When the haloes become unscreened at late times, the total gravitational potential rapidly becomes $1/3$ deeper, but the particle kinetic energy requires more time to respond, so that these particles tend to fall towards the halo centre, increasing the halo density profile and therefore the one-halo contribution to $P_{\rm mm}(k)$.
    \item When the \ac{MG} effect is strong, particles have been accelerated for a long time (both well before and after they fall into haloes, as the latter are unscreened or less screened) due to the relatively strong fifth force. This means that the accelerations and velocities of particles can be boosted by a similar fraction as the enhancement in the depth of the gravitational potential, and hence the partice kinetic energy can be increased by a larger factor than the deepening of the potential, so that the particles are less likely to be trapped towards the centre of the potential. The small-scale structure can thus be erased out to a certain degree.
    This behaviour of $f(R)$ gravity has been discussed in previous works such as \citep[][]{2011PhRvD..83d4007Z,Li:2013MNRAS.428..743L,Mitchell:2019qke}.
    The explanation also works for other models in which screening has always been weak or absent, such as the coupled quintessence model (the left panel of Fig.~\ref{fig:Pk_HMF_csf}) and the K-mouflage model \citep{Hernandez-Aguayo:2021_twin_paper}; in both cases we see a decay of $\Delta P/P_{\rm GR}$ at $k\gtrsim 1 \, h \, \mathrm{Mpc}^{-1}$. 
\end{itemize}

We note that the parameter $n$ of the $f(R)$ model also has a considerable influence on structure formation. For fixed $f_{R0}$, the larger the value of $n$, the more efficiently the fifth force is screened, as can be seen from Fig.~\ref{fig:Pk_mm_z0_fRn012}, which shows that the matter clustering enhancement is strongest in the $n=0$ while weakest in the $n=2$ case. We have found similar behaviour when we checked the solution of scalar field around a top-hat overdensity in Sec.~\ref{subsubsect:3D_tests}, see the middle panel Fig.~\ref{fig:CodeTests3D}: the $n=2$ case has the strongest screening efficiency.



\begin{figure}
    \centering 
    \includegraphics[width=\textwidth]{./fig/Pk_fR_1x3_z0.0.pdf}
    \caption{\textit{Upper Panels:} The non-linear matter power spectra at redshift $z=0$, from \textsc{mg-glam} simulations of the $f(R)$ model for $n=0$ (left panel), $n=1$ (middle) and $n=2$ (right), each with $10$ values of $|f_{R0}|$ logarithmically spaced between $10^{-6}$ and $10^{-4.5}$, i.e., $-\log_{10}|f_{R0}| = 4.50, 4.67, \dots, 6.00$. These are indicated with different colours given in the legends. 
    \textit{Lower Panels:} The fractional difference, $\Delta P / P_{\mathrm{GR}}$, between the $f(R)$ and $\Lambda$CDM results, where $\Delta P \equiv P_{\rm MG} - P_{\rm GR}$. The $n=0,1$, $-\log_{10} |f_{R0}| = 5.00$ and $\Lambda$CDM results are the mean of ten independent realisations, while other models only have one realisation.}
    \label{fig:Pk_mm_z0_fRn012}
\end{figure}


 
In \ac{MG} theories, the dark matter halo populations are also affected. One of the elementary halo properties is their abundance, which we quantify using the differential halo mass function (dHMF), $\dd{n}(M)/\dd{\log_{10}M}$, which is defined as the halo number density per unit logarithmic halo mass.
The dHMF result for the $f(R)$ gravity runs at $z=0$ is shown in Fig.~\ref{fig:hmf_z0_fRn012}, where the lower subpanels show the enhancements with respect to $\Lambda$CDM.


Firstly, we note that the abundance of haloes is enhanced due to the enhancement of total gravity. Secondly, for the weaker $f(R)$ models, the relative difference from $\Lambda$CDM is suppressed for massive haloes, where the fifth force is efficiently screened; going to smaller haloes the enhancement increases first, which is due to the less efficient screening and stronger \ac{MG} effect for these objects; but for even smaller haloes the HMF enhancement decreases after reaching a peak, which is due to smaller haloes experiencing more mergers to form larger haloes. Apparently, this trend is not seen for the strong \ac{MG} cases, such as F4.50 (purple) and F4.67 (dark blue), where the HMF enhancement seems to increase monotonoically with halo mass. However, we speculate that the qualitative behaviour for the weaker $f(R)$ models should also hold even in these cases: note that our halo catalogues have been cut off for $M_{\rm vir} \gtrsim 10^{14.7} \, h^{-1} M_{\odot}$ due to the relatively small box size; should the simulations be run with larger box sizes (while keeping the same resolution), we expect the HMF enhancement to dacay to zero for large enough haloes even in the strong \ac{MG} cases.
%Thirdly, the dHMF enhancement rises to a peak at a critical halo mass whose value is larger for larger $|f_{R0}|$. For the high-mass end, this can be explained by the screening effect. For the low-mass end, the small haloes are accreted and merged more frequently in the $f(R)$ gravity than \ac{GR}.
Finally, we note that the dHMFs are less sensitive to the model parameter $n$ than to $f_{R0}$, compared to the matter power spectra. The shapes and amplitude of dHMFs are similar for $n = 0, 1$ and $2$, though we can still see that they are enhanced slightly more in the case of $n=0$ than the cases of $n=1,2$, for F4.50 and F4.67.



\begin{figure}
    \centering 
    \includegraphics[width=\textwidth]{./fig/dHMF_fR_1x3_z0.0.pdf}
    \caption{\textit{Upper Panels:} Differential halo mass functions (HMFs) of $f(R)$ gravity for $n=0$ (left panel), $n=1$ (middle) and $n=2$ (right), each with $10$ values of $|f_{R0}|$ logarithmically spaced between $10^{-6}$ and $10^{-4.5}$, i.e., $-\log_{10}|f_{R0}| = 4.50, 4.67, \dots, 6.00$, at redshift $z=0$, from \textsc{mg-glam} cosmological runs.
    \textit{Lower Panels:} The fractional difference $\Delta \text{HMF} / \text{HMF}_{\mathrm{GR}}$ between $f(R)$ and $\Lambda$CDM results, where $\Delta \text{HMF}  \equiv \text{HMF}_{\rm MG} - \text{HMF}_{\rm GR}$.
    The $n=0,1$, $-\log_{10} |f_{R0}| = 5.00$ and $\Lambda$CDM results come from ten realisations (the standard deviation of which is shown as the error bars in the bottom left/central panels), while other models only have one realisation.}
    \label{fig:hmf_z0_fRn012}
\end{figure}


\subsection{Symmetrons and coupled quintessence}
\label{subsect:sym_csf_runs}

We now present the measured matter power spectra and halo mass functions from our symmetron and coupled quintessence runs in Figs.~\ref{fig:Pk_hmf_z0_sym} and \ref{fig:Pk_HMF_csf}, respectively.

Fig.~\ref{fig:Pk_hmf_z0_sym} presents the symmetron model results with $a_\ast = 0.33$, $\beta_\ast = 1.0$ and five $c\xi/H_0$ values of $0.5, 1.0, 2.0, 2.5$ and $3.0$.
The behaviour of the symmetron model is qualitatively similar to that of the $f(R)$ model since both of them are thin-shell screened models \citep{Brax:2012gr}. This agrees with the middle panel of Fig.~\ref{fig:CodeTests3D}, which shows that these two models have qualitatively very similar scalar field profiles for a given spherical tophat overdensity. A smaller value of $c\xi/H_0$ means $m_\ast$, the `mass' of the symmetron field, is larger, which subsequently implies that the field can more easily settle to the potential minimum (which corresponds to $\varphi=0$) in dense regions, and therefore be screened. 



\begin{figure}
    \centering 
    \includegraphics[width=\textwidth]{./fig/Pk_HMF_sym.pdf}
    \caption{The matter power spectra (left panel) and differential halo mass functions (right) of the symmetron models at $z=0$ for 5 different values of $c\xi/H_0$ as labelled. In all cases the remaining symmetron parameters are fixed as $a_\ast=0.33$ and $\beta_\ast=1.0$. As in Figs.~\ref{fig:Pk_mm_z0_fRn012} and \ref{fig:hmf_z0_fRn012}, the upper subpanels present the absolute measurements from simulations, while the lower subpanels show the relative differences from $\Lambda$CDM.
    }
    \label{fig:Pk_hmf_z0_sym}
\end{figure}


In Fig.~\ref{fig:Pk_HMF_csf} we show the $P_{\rm mm}(k)$ and dHMF from our three coupled quintessence models with $(\alpha, \beta) = (0.1, -0.1), (0.1, -0.2)$ and $(0.5, -0.2)$. 
The power spectrum enhancement remains approximately constant at $k\lesssim0.1h\mathrm{Mpc}^{-1}$, which is the linear perturbation regime. This is different from the behaviour seen in the $f(R)$ and symmetron models above, where $\Delta P/P_{\rm GR}$ increases with $k$ in this range, and the difference is because in coupled quintessence there is no screening, so that the fifth force is long ranged, with a ratio to the strength of Newtonian gravity that is almost constant in space. At small scales, $k \gtrsim 1 \, h \, \mathrm{Mpc}^{-1}$, $\Delta P/P_{\rm GR}$ decays with $k$, as we found in the stronger $f(R)$ models in Fig.~\ref{fig:Pk_mm_z0_fRn012}, and the physical reason behind this is the same as there: different from the weaker $f(R)$ models, even inside dark matter haloes the particles still feel a strong fifth force, which is almost in constant proportion to the strength of Newtonian force, and on top of this the direction-dependent force can also speed up the particles; the result of the two forces is that the particles gain higher kinetic energy and tend to move to and stay in the outer region of haloes, thereby reducing matter clustering on small scales compared to $\Lambda$CDM.


The right panel of Fig.~\ref{fig:Pk_HMF_csf} presents the dHMF results. 
We find that the coupled quintessence models studied here produce more high-mass haloes and fewer low-mass haloes than \ac{GR}, which is the consequence of the competition between the four effects discussed in Sect.~\ref{subsect:bg_tests}. Because these effects strongly entangle with each other through the complicated structure formation process, it is difficult to know quantitatively how they lead to the observed behaviour above, except by running simulations with different combinations of them switched on or off. 
%Therefore, to sort out these four effects and rank their relative importance, we need to switch them on and off separately to observe the impact on observable quantities. 
Although this is obviously an interesting and important thing to do, it is beyond the scope of this paper, so we will leave it to future works.


\begin{figure}
    \centering 
    \includegraphics[width=\textwidth]{./fig/Pk_HMF_csf.pdf}
    \caption{The matter power spectra (left panel) and the differential halo mass functions (right) of the coupled quintessence models at $z=0$, for three different $\alpha$ and $\beta$ values as labelled. As in Figs.~\ref{fig:Pk_mm_z0_fRn012}, \ref{fig:hmf_z0_fRn012} and \ref{fig:Pk_hmf_z0_sym}, the upper subpanels present the absolute measurements from \textsc{mg-glam} simuations, while the lower subpanels show the relative differences from $\Lambda$CDM.
    }
    \label{fig:Pk_HMF_csf}
\end{figure}


\subsection{Summary}
\label{subsect:cosmo_runs_summary}

In this section we have had an initial taste of the \textsc{mg-glam} code, by running a large suite of simulations covering all three classes of models studied in this paper.

One particularly relevant aspect of the \textsc{mg-glam} code is its fast speed. The 40 $f(R)$ simulations described in this section have been run using 56 threads with \textsc{openmp} parallelisation, and we find that the run times vary randomly between $\simeq17,000$ and $\simeq33,000$ seconds, apparently depending on the real-time performance of the computer nodes used. The majority of them took $\sim24,000$ seconds, or equivalently $\simeq375$ core hours. This is roughly 100 times faster than \textsc{mg-arepo}, and $300$ times faster than \textsc{ecosmog}, for the same simulation specifications. With such a high efficiency, we can easily ramp up the simulation programme to include many more models and parameter choices, and increase the size and/or resolution of the runs, e.g., using box size of at least $1 \, h^{-1} \, \mathrm{Gpc}$. For the symmetron and coupled quintessence runs we have found similar speeds, though the run time for coupled quintessence models can perhaps be dramatically reduced if we do not explicitly solve the scalar field and the fifth force, by instead assume that the latter is proportional to the Newtonian force. We have also run a few even larger simulations for $\Lambda$CDM, F5n0 and F5n1 with $L = 512 \, h^{-1}\mathrm{Mpc}$, $N_{\rm p}=2048$ and $N_{\rm g}=4096$ (for the same cosmology as above), and some of the results are presented in Appendix \ref{appendix:resolution} --- these runs took around $42,000$ seconds for $\Lambda$CDM, $80,000$ seconds for F5n0 and $125,000$ seconds (wallclock time) for F5n1 with 128 threads using the SKUN8@IAA supercomputer at the IAA-CSIC in Spain, suggesting that a single run of specification L1000Np2048Ng4096, which would be useful for cosmological (e.g., galaxy clustering and galaxy clusters) analyses should take at most $1$--$1.5$ days to complete and is therefore easily affordable with existing computing resources.

On the other hand, efficiency should not be achieved at the cost of a significant loss of accuracy. For the runs used here, we have used a mesh resolution of $0.25 \, h^{-1} \, \mathrm{Mpc}$, which is sufficient to achieve percent-level accuracy of the matter power spectrum at $k\lesssim 1 \, h \, \mathrm{Mpc}^{-1}$ \cite{Klypin:2017iwu}, matter power spectrum enhancement at $k \lesssim 3 \, h \, \mathrm{Mpc}^{-1}$, and (main) halo mass function down to $\sim \, 10^{12.5} \, h^{-1}M_\odot$. The particle number, $N_{\rm p}^3$, in \textsc{glam} simulations is normally set according to $N_{\rm p}=N_{\rm g}/2$, so that in the simulations here we have used $1024^3$ particles. However, we have checked that increasing the particle number to $2048^3$ has little impact on the halo mass function. We notice that the completeness level of the HMFs here is similar to \textsc{ecosmog} runs with the same simulation specifications, suggesting that \textsc{mg-glam} is capable of striking an optimal balance between cost and accuracy. In Appendix \ref{appendix:resolution}, we present more detailed tests of \textsc{mg-glam}'s power spectrum and HMF predictions at different force and mesh resolutions, including our highest-resolution runs for $\Lambda$CDM and F5n1 with $L = 512 \, h^{-1}\mathrm{Mpc}$, $N_{\rm p}=2048$ and $N_{\rm g}=4096$ (for the same cosmology as above). There we demonstrate that the increase of force resolution can lead to further improvement of the small-scale and low-mass predictions.


Before concluding this paper, let us briefly describe some tests we have performed to understand how well the parallelisation of \textsc{mg-glam} works. This consists of a series of runs (taking $f(R)$ gravity F5n1 as a representative) to demonstrate the scaling of \textsc{mg-glam}, and these runs were all done on the SKUN6/SKUN8 facility managed by the IAA-CSIC in Spain.The strong scaling tests are presented in the left panel of Fig.~\ref{fig:scaling}. The test simulations employed a fixed resolution of $256^3$ particles and $512^3$ grids, with the same Planck 2015 cosmology as used in the main \textsc{mg-glam} runs of this paper. This plot shows that, when the number of \textsc{openmp} threads ranges between 1 and $\sim 30$, the wallclock time scales linearly with the thread number. The deviation from a perfect linear scaling (black dashed line) when the number of threads exceeds $30$ is possibly due to the small size of the test run. In addition, we have also run a set of simulations of different sizes by varying the resolutions and keeping the number of threads fixed. The wallclock time used is shown in the right panel of Fig.~\ref{fig:scaling}. We see that the time consumption again scales nearly perfectly linearly with the considered resolutions (up to $N_{\rm g} = 4096$ and $N_{\rm p} = N_{\rm g}/2$). These tests demonstrate that \textsc{mg-glam} is well scalable.


\begin{figure}
    \centering 
    \includegraphics[width=\textwidth]{./fig/scaling.pdf}
    \caption{
        \textit{Left panel:} The wallclock time of the \textsc{mg-glam} test runs for the F5n1 model, with fixed simulation size/resolution ($L=512h^{-1}\mathrm{Mpc}$, $N_{\rm p} = 256$ and $N_{\rm g} = 512$), as a function of the number of threads used in \textsc{openmp} parallelisation.
        The scaling between run time and thread number is very close to be perfectly linear for number of threads up to $\sim 30$.
        \textit{Right panel:} The wallclock times of the \textsc{mg-glam} runs for F5n1 with varying simulation sizes and resolutions (from left to right: $N_{\rm g} = 256, 512, 1024, 2048, 4096$, and $N_{\rm p} = N_{\rm g}/2$), while the number of threads is chosen as $128$. Again, the scaling is nearly perfectly linear. In both cases, the symbols denote the times taken by the test runs, and the lines denote the expected results with a `perfect linear scaling'. We find similarly good scaling properties for $\Lambda$CDM and F5n0, but those are not shown here.
    }
    \label{fig:scaling}
\end{figure}