\section{Code tests}
\label{sect:code_tests}

In this section, we present various code test results to demonstrate the reliability of the equations, algorithms and implementations described in the previous sections. We follow the code test framework of the \textsc{ecosmog} code papers \citep{Li:2011_ECOSMOG_code_paper,Li:2013_ECOSMOGV_code_paper}. Apart from the background cosmology test, all the tests shown in this section were performed on a cubic box with size $256 \, h^{-1} \mathrm{Mpc}$ and $512$ grid cells in each direction, and all background quantities are calculated at $a = 1$. 
%{\bf ??? isn't it the same for ALL redshifts?} \Ruan{Some factors in the code test equations involve background quantities such as $\bar{f}_R(a), \bar{R}(a)$ in Eq.~(4.2), so we need to specify the redshift. }\ca{}{[I think maybe the sentence is a bit confusing and implies that the box size and particle number change with redshift while it actually means that the results are presented at $a=1$. Maybe this could be made more clear.]}
%The other parameters were: the $f(R)$ gravity model parameters $n = 0, 1, 2$, $f_{R0} = -10^{-5}$; symmetron model parameters $a_* = 0.5, \xi = 10^{-3}$, $\beta_* = 0.1$; coupled quintessence model parameters $\beta = -0.2$, $\alpha = 0.1$.




\subsection{Background cosmology tests}
\label{subsect:bg_tests}

Of the models considered in this work, only the coupled quintessence model can substantially affect the background expansion history, while for (viable) $f(R)$ gravity and symmetron models the expansion rate is practically indistinguishable from that of $\Lambda$CDM. In \textsc{mg}-\textsc{glam}, the background cosmology in the coupled quintessence model is solved numerically, as described in Sect.~\ref{subsubsect:csf_imp}. 

To check the numerical implementation, we have compared the predictions of certain background quantities by \textsc{mg}-\textsc{glam} with the results produced by the modified \href{https://camb.info/}{\textsc{camb}} code, for the same coupled quintessence model, described in \cite{Li:2010re}. The results are presented in Fig~\ref{fig:code_test_background_csf}, where the left panel shows the ratio between the background expansion rates of three coupled quintessence models and that of a $\Lambda$CDM model with the same (non-MG) cosmological parameters, while the right panel shows the background evolution of the scalar field, $\bar{\varphi}(a)$, for the same three models. Lines are from the modified \href{https://camb.info/}{\textsc{camb}} code and symbols are for \textsc{mg-glam}. We see that the background cosmology solver of \textsc{mg-glam} agrees with the \href{https://camb.info/}{\textsc{camb}} code very well in all cases.


There are two additional interesting features displayed in Fig.~\ref{fig:code_test_background_csf}. First, the results are much more sensitive to $\beta$ than to $\alpha$, as can be observed by comparing the closeness between the black vs red lines, and the large difference between the black vs blue lines. This shows that the coupling to matter has a stronger impact on the scalar field background evolution than the potetial itself.

Second, as discussed in Sect.~\ref{subsect:csf}, the scalar field affects structure formation through a combination of the following four effects:
\begin{itemize}
    \item \textit{modified expansion rate}: in the models studied here, the expansion rate is slowed down, which can lead to enhancement of structure formation.
    \item \textit{fifth force}: the fifth-force-to-Newtonian-gravity ratio is a constant $2\beta^2$, and this boosts structure formation.
    \item \textit{velocity-dependent force}: from the right panel of Fig.~\ref{fig:code_test_background_csf}, we see that the scalar field is positive and grows over time such that, with $\beta<0$, the term $\left({\rm d}\ln A(\bar{\varphi})/{\rm d}\varphi\right)\bar{\varphi}'<0$, which means that the velocity-dependent force is in the same direction as the particle velocity, i.e., it is essentially an `anti-friction' force which tends to strengthen structure formation.
    \item \textit{time variation of effective particle mass}: since the particle mass effectively depends on $\exp(\beta\bar{\varphi})$, with $\beta<0$ and $\bar{\varphi}>0$, at late times the effective mass decreases, which tends to weaken structure formation.
\end{itemize}
Therefore, the 4 effects work in different directions, and the net effect on structure formation---whether it is boosted or weakened---will need to be calculated numerically for specific models.


\begin{figure}
    \centering
    \includegraphics[width=\textwidth]{./fig/background_test_csf.pdf}
\caption{
    Cosmological background evolution tests.
    \textit{Left panel:} The ratio of the Hubble parameters between the coupled quintessence and the \ac{GR} models, from the modified \href{https://camb.info/}{\textsc{camb}} (lines) and \textsc{mg-glam} (dots) codes for three kinds of model parameter values as labeled.  
    \textit{Right panel:} The evolution of the background scalar field in the coupled quintessence from \href{https://camb.info/}{\textsc{camb}} (lines) and \textsc{mg-glam} codes.
}
    \label{fig:code_test_background_csf}
\end{figure}

\subsection{Density tests}

This subsection is devoted to the tests of the multigrid solvers for the $f(R)$, symmetron and coupled quintessence models, using different density configurations for which the scalar field solution can be solved analytically or using a different numerical code.

%In this subsection \textit{only}, we will simply denote the code unit coordinates $\tilde{x}$ as $x$ for brevity.

\subsubsection{Homogeneous matter density field}
\label{subsubsec:homogeneous_rhom_test}

In a homogeneous density field the \ac{MG} scalar field should also be homogeneous and exactly equal to its background value if the matter field is homogeneous, i.e.,
\begin{align}\label{eq:field_uniform_dens_test}
    \tilde{\delta}(\tilde{x}) \equiv 0 \longrightarrow \begin{cases}
        f_R(\tilde{x}) / \bar{f}_R \equiv 1, & \text{$f(R)$ gravity;} \\
        \varphi(\tilde{x}) / \varphi_\ast = \bar{\varphi}/\varphi_\ast \leq 1, & \text{symmetron;} \\
        \tilde{c}^2\delta\varphi(\tilde{x}) \equiv 0, & \text{coupled quintessence.}
    \end{cases}
\end{align}
This offers a very simple test for the relaxation solvers described above, that is particularly useful for checking the implementation of multigrid.

We show the test results for a homogeneous density field in the left-hand panels of Fig.~\ref{fig:code_test_1D_3x2}, where we display the scalar field values along the $\tilde{x}$ direction for fixed $\tilde{y}, \tilde{z}$ coordinates before (symbols) and after (lines) the multigrid relaxation, for two initial guesses (black and red). The three rows, from top to bottom, are respectively for the $f(R)$, symmetron and coupld quintessence models. For $f(R)$ gravity, the initial guesses are randomly generated from a uniform distribution within $\xi=f_R(\tilde{x}) / \bar{f}_R\in[0,2]$, and the model parameters used are $n=1$ and $f_{R0}=-10^{-5}$; for the symmetron model, the random initial guesses are generated from a uniform distribution $\varphi(\tilde{x})/\varphi_\ast\in[0,1]$ and the model parameters adopted are $a_\ast = 0.5, \xi = 10^{-3}$, $\beta_\ast = 0.1$; for coupled quintessence we consider the model parameters $\alpha = 0.1, \beta = -0.2$, and the initial guesses are from a uniformation distribution $\delta\varphi(\tilde{x})\in[-0.5,0.5]$.

In all cases, we find that the solutions after relaxation agree very well to the analytical predictions given in Eq.~\eqref{eq:field_uniform_dens_test}.


\begin{figure}
    \centering
    \includegraphics[width=0.7\textwidth]{./fig/OneDimTests_3x2.pdf}
    \caption{The uniform and one-dimensional code test results. The two columns show the cases with homogeneous (left) and sine (right) scalar fields, whilst the different rows represent the $f(R)$ gravity (upper), symmetron (middle) and coupled quintessence (bottom) models. \textit{Left Panels:} Uniform density test, where the symbols represent the random initial guesses of the \ac{MG} scalar field in the ranges of $[0, 2]$ ($f(R)$ gravity), $[0, 1]$ (symmetron) and $[-0.5, 0.5]$ (coupled quintessence), respectively. The solid lines show the numerical solutions after multigrid relaxation. Two random initialisations have been displayed in red and black. \textit{Right Panels:} Sine field tests. The squares show the numerical results and the lines show the analytical solutions. The upper right panel shows the $f(R)$ gravity test results with $n = 0, 1$ and $2$ as labeled.
    }
    \label{fig:code_test_1D_3x2}
\end{figure}

\subsubsection{One-dimensional code tests}
\label{subsubsect:1D_tests}

In the case of one spatial dimension, the scalar field satisfies ordinary differential equations. Therefore, we can construct a density field that has a known analytical solution of the scalar field, to check if the code returns the correct numerical solution, according to the scalar field equations of $f(R)$ gravity (Eq.~\eqref{eq:fR_eom_codeunit2}), symmetron (Eq.~\eqref{eq:sym_eom_code_unit}) and coupled quintessence (Eq.~\eqref{eq:csf_eom_code}) in code units. In practice this can be achieved by choosing a functional form of the scalar field in 1D, and applying the above equations to derive analytical expressions for $\delta(\tilde{x})$. For example, we can design density configurations in $f(R)$ gravity by manipulating Eq.~\eqref{eq:fR_eom_codeunit2} in the 1D case as 
\begin{align}
    \tilde{\delta}(\tilde{x}) = -\frac{a}{\Omega_{\rm m}} \qty{ -\tilde{c}^2 \tilde{\nabla}^2 \qty[u^{n+1}] - \frac{1}{3} \tilde{\bar{R}}(a) a^2 \qty[-\bar{f}_R (a)]^{1/(n+1)} \frac{1}{u} + \frac{1}{3} \tilde{\bar{R}}(a) a^2 } \ . \label{eq:fR_eom_codeunit2_CodeTestForm}
\end{align}
We have designed such tests where the scalar field solution is a sine function. 

For $f(R)$ gravity, the scalar field takes the following sine-function form, 
\begin{align}\label{eq:sine_field_fR}
    \frac{f_R(\tilde{x})}{\bar{f}_R} = 1 + A \sin\frac{2\pi \tilde{x}}{N_{\rm g}}\,,
\end{align}
if the density field is given by 
%\begin{align}
%    \delta (\tilde{x}) = \begin{cases}\displaystyle
%        \frac{a}{\Omega_{\rm m}} \qty[\tilde{c}^2\bar{f}_R\left(\frac{2\pi}{N_{\rm g}}\right)^2A \sin\frac{2\pi \tilde{x}}{N_{\rm g}}], & n=0 \\
%        \displaystyle \frac{a}{\Omega_{\rm m}} \qty[ \tilde{c}^2 \bar{f}_R \qty(\frac{2\pi}{N_{\rm g}})^2 A \sin\frac{2\pi x}{N_{\rm g}} + \frac{1}{3} \tilde{\bar{R}}(a) a^2 \qty(1 + A \sin\frac{2\pi x}{N_{\rm g}})^{-1/2} - \frac{1}{3} \tilde{\bar{R}}(a) a^2], & n=1, \\
%        \displaystyle   \frac{a}{\Omega_{\rm m}} \qty[\tilde{c}^2 \bar{f}_R \qty(\frac{2\pi}{N_{\rm g}})^2A \sin\frac{2\pi \tilde{x}}{N_{\rm g}} + \frac{1}{3} \tilde{\bar{R}}(a) a^2 \qty(1 + A \sin\frac{2 \pi \tilde{x}}{N_{\rm g}})^{-1/3} - \frac{1}{3} \tilde{\bar{R}}(a) a^2], & n=2 
%    \end{cases}
%\end{align}
\begin{equation}
    \tilde{\delta}(\tilde{x}) = \frac{a}{\Omega_{\rm m}} \qty[\tilde{c}^2 \bar{f}_R \qty(\frac{2\pi}{N_{\rm g}})^2A \sin\frac{2\pi \tilde{x}}{N_{\rm g}} + \frac{1}{3} \tilde{\bar{R}}(a) a^2 \qty(1 + A \sin\frac{2 \pi \tilde{x}}{N_{\rm g}})^{-\frac{1}{n+1}} - \frac{1}{3} \tilde{\bar{R}}(a) a^2],
\end{equation}
where $A$ is a constant and $|A| < 1$ as $f_R/\bar{f}_R$ should be positive. We have again adopted $f_{R0}=-10^{-5}$ and considered the three cases of $n=0,1,2$ respectively. 

For the symmetron model, we have taken the following form of the scalar field
\begin{equation}\label{eq:sine_field_sym}
    u(\tilde{x}) = \frac{\varphi(\tilde{x})}{\varphi_\ast} = \frac{1}{2}+A\sin\frac{2\pi\tilde{x}}{N_{\rm g}},
\end{equation}
which corresponds to the following overdensity field,
\begin{equation}
    \tilde{\delta}\left(\tilde{x}\right) = \frac{a^3}{a^3_\ast}\left[1 - \left(\frac{1}{2}+A\sin\frac{2\pi\tilde{x}}{N_{\rm g}}\right)^2 - \frac{2}{a^2}\xi^2\tilde{c}^2A\left(\frac{2\pi}{N_{\rm g}}\right)^2\frac{\sin\frac{2\pi\tilde{x}}{N_{\rm g}}}{\frac{1}{2}+A\sin\frac{2\pi\tilde{x}}{N_{\rm g}}}\right]-1.
\end{equation}
The model parameter used here are the same as in the uniform density test above.

For the coupled quintessence model, we have taken the following form of the scalar field
\begin{equation}\label{eq:sine_field_cs}
    \tilde{\varphi}(\tilde{x}) = \tilde{c}^2\delta\varphi(\tilde{x}) = A\sin\frac{2\pi\tilde{x}}{N_{\rm g}},
\end{equation}
which corresponds to the following overdensity field,
\begin{equation}
    \tilde{\delta}\left(\tilde{x}\right) = -\frac{a}{3\beta\Omega_{\rm m}}\exp(\beta\bar{\varphi})\left[A\left(\frac{2\pi}{N_{\rm g}}\right)^2\sin\frac{2\pi\tilde{x}}{N_{\rm g}}-\frac{\lambda^2a^2}{\left(\bar{\varphi}+\tilde{c}^{-2}A\sin\frac{2\pi\tilde{x}}{N_{\rm g}}\right)^\alpha} + \frac{\lambda^2a^2}{\bar{\varphi}^\alpha}\right].
\end{equation}
The model parameter used here are the same as in the uniform density test above.

The panels in the right column of Fig.~\ref{fig:code_test_1D_3x2} present the sine field test results for the three classes of models, in the same order as in the left column. The numerical solutions from \textsc{mg-glam} (squares) agree well with the analytical solutions of Eqs.~(\ref{eq:sine_field_fR}, \ref{eq:sine_field_sym}, \ref{eq:sine_field_cs}), shown by lines, indicating that the code works accurately to solve the scalar field equations. In all the tests shown here we have taken $A=0.5$, but we have checked other values of $A$, as well as sine functions with more than one oscillation period, and found similar agreements in all cases.

%\subsubsection{Gaussian field}
%\label{subsubsec:Gaussian_test}

%Similarly, we can construct a Gaussian type density configuration from Eq.~\eqref{eq:fR_eom_codeunit2_CodeTestForm}. The scalar field and the corresponding density field are given by 
%\begin{align}
%    \frac{u^{n+1}(x)}{-\bar{f}_R} \equiv \frac{f_R(x)}{\bar{f}_R}  = A \exp \qty[-\frac{1}{2} \qty(\frac{x}{N_{\rm g}})^2], 
%\end{align}
%and 
%\begin{align}
%\delta (x) = \frac{a}{\Omega_{\rm m}} \qty{\frac{1}{3} \tilde{\bar{R}}(a) a^2 \qty[A^{-1/2} \exp\qty(\frac{x^2}{4N_{\rm g}^2}) - 1] - \tilde{c}^2 \bar{f}_R A \exp\qty(-\frac{x^2}{2N_{\rm g}^2}) \qty[\qty(\frac{x}{N_{\rm g}})^2 - 1] } \ ,
%\end{align}
%respectively, where $A$ is a constant. The numerical and the analytical solutions are shown in the lower right panel of Fig.~\ref{fig:code_test_1D_3x2} for $a=1, n=1, A=0.1$ and $f_{R0} = -10^{-4}, -10^{-5}$ and $-10^{-6}$. We can see that the numerical results agree with the analytical solution very well.

\subsubsection{Three-dimensional density tests}
\label{subsubsect:3D_tests}

As the final part of our tests of the multigrid relaxation solver, we consider slightly more complicated density configurations than the uniform and 1D density fields used previously. In order to get analytical and numerical solutions that can be compared with the predictions by \textsc{mg-glam}, we still would like to use density fields that have certain symmetries. To this end, we have done tests using a point mass (for $f(R)$ gravity) and spherical tophat overdensity (for the symmetron and coupled quintessence models). These tests will see the scalar field values vary in $x, y$ and $z$ directions, and they are therefore proper 3D tests.

\subsubsection*{Point mass}

For the first test in 3D space, we consider the solution of the scalar field around a point mass placed at the origin, for which we have approximated analytical solution that is valid in the regions far from the mass. This test has been widey performed in previous MG code papers such as \citep{2008PhRvD..78l3523O,2011PhRvD..83j4026B,Li:2011_ECOSMOG_code_paper,Arnold:2019_MGAREPO_code_paper}. The matter overdensity array is constructed as
\begin{align}
    \tilde{\delta}_{i,j,k} = \begin{cases}
        10^{-4} (N_{\rm g}^3 - 1) , & i=j=k=1; \\
        -10^{-4} , & \text{otherwise.}
    \end{cases} \label{eqn:deltaijk_point_mass}
\end{align}
where $i,j,k = 1, ..., N_{\rm g}$ are the cell indices in $x,y,z$ directions, respectively.

In $f(R)$ gravity, with this density configuration, the scalaron equation Eq.~\eqref{eq:fR_eom_qsa} in regions far from the point mass simplifies to 
\begin{align}
    \boldsymbol{\nabla}^2 \delta f_R \approx m_{\rm eff}^2 \delta f_R \ , \label{eqn:fR_eom_pointmass_far}
\end{align}%\label{eq:linear_fR_eqn}
where $\delta f_R (\bm{x}) \equiv f_R(\bm{x}) - \bar{f}_R$, and the effective mass of the scalar field, $m_{\rm eff}$, is given by
\begin{equation}
    m^2_{\rm eff} \equiv -\frac{1}{3(n+1)}\frac{\bar{R}_0}{c^2\bar{f}_{R0}}\left(\frac{\bar{R}}{\bar{R}_0}\right)^{n+2} = \frac{H_0^2\Omega_{\rm m}}{c^2(n+1) (-\bar{f}_{R0})} \frac{\qty(a^{-3} + 4\frac{\Omega_{\Lambda}}{\Omega_{\rm m}})^{n+2}}{\qty(1 + 4\frac{\Omega_{\Lambda}}{\Omega_{\rm m}})^{n+1}}\,.
\end{equation}
At $a=1$, this only depends on the combination $(n+1)\bar{f}_{R0}$. For a sphericially symmetric case such as the one considered here, the equation can be recast in the following form,
\begin{equation}
    \frac{1}{r^2}\frac{{\rm d}}{{\rm d}r}\left[r^2\frac{{\rm d}\delta{f}_R}{{\rm d}r}\right] = m^2_{\rm eff}\delta{f}_R,
\end{equation}
or equivalently 
\begin{equation}
    \frac{{\rm d}^2}{{\rm d}r^2}\left[r\delta{f}_R(r)\right] = m^2_{\rm eff}\cdot r\delta{f}_R(r),
\end{equation}
where $r$ is the distance from the central point mass. This equation has the solution 
\begin{equation}
    r\delta{f}_R(r) = \alpha_1\exp\left(-m_{\rm eff}r\right) + \alpha_2\exp\left(m_{\rm eff}r\right),
\end{equation}
where $\alpha_{1}, \alpha_2$ are constants of integral, and we must have $\alpha_2=0$ to prevent the solution from diverging at $r\rightarrow\infty$. This leads to the following solution 
\begin{equation}\label{eq:fR_point_mass_soln}
    \delta{f}_R(r) \propto \frac{1}{r}\exp\left(-m_{\rm eff}r\right),
\end{equation}
which in code unit can be rewritten as
\begin{equation}\label{eq:fR_point_mass_soln_code_unit}
    \delta{f}_R\left(\tilde{r}\right) \propto \frac{1}{\tilde{r}}\exp\left(-\tilde{m}_{\rm eff}\tilde{r}\right),
\end{equation}
where the $\tilde{m}_{\rm eff}$ is the scalar field mass $m_{\rm eff}$ in code unit, given by
\begin{equation}
    \tilde{m}_{\rm eff}^2 \equiv \frac{\Omega_{\rm m}}{\tilde{c}^2(n+1) (-\bar{f}_{R0})} \frac{\qty(a^{-3} + 4\frac{\Omega_{\Lambda}}{\Omega_{\rm m}})^{n+2}}{\qty(1 + 4\frac{\Omega_{\Lambda}}{\Omega_{\rm m}})^{n+1}}\,.
\end{equation}
Note that we have neglected the tilde for $\delta{f}_R$ since in our code units $\tilde{f}_R = f_R$ and $\tilde{\bar{f}}_{R0} = \bar{f}_{R0} \equiv f_{R0}$.

In the left panel of Fig.~\ref{fig:CodeTests3D}, we show the numerical solutions from \textsc{mg-glam} and the analytical results given in Eq.~\eqref{eq:fR_point_mass_soln_code_unit}. Notice that the latter has an unknown coefficient, which we have tuned to match the amplitude of the \textsc{mg-glam} solution. Once that is done, the two agree very well for all three $f(R)$ gravity models with $f_{R0}=-10^{-5}$ for $n = 0, 1$ and $2$ resepctively, except on scales smaller than $\simeq 5 \, h^{-1} \mathrm{Mpc}$ since Eq.~\eqref{eqn:fR_eom_pointmass_far} is not valid near the point mass, and far from the point mass where the \textsc{mg-glam} solution starts to see the effect of periodic boundary condition, which is absent in Eq.~\eqref{eq:fR_point_mass_soln_code_unit}.

\begin{figure}
    \centering 
    \includegraphics[width=\textwidth]{./fig/ThreeDimTests_1x3.pdf}
    \caption{The three-dimensional code test results. 
    \textit{Left Panel:} The numerical (squares) and analytical (lines) solutions to $\delta f_R \equiv f_R - \bar{f}_R$ around a point mass located at $(x,y,z)=(0,0,0)$, for three $f(R)$ gravity models with $f_{R0} = -10^{-5}$ and $n = 0$ (red), $1$ (black) and $2$ (blue), respectively. The analytical approximations are only valid far from the point mass. Only the solutions along the $x$-axis are shown. 
    \textit{Middle Panel:} Top-hat overdensity test for the symmetron (black) and $f(R)$ gravity models (dashed lines). 
    The lines correspond to the analytical solutions and the dots represent the numerical results. 
    The quantities shown on the $y$-axis are $\varphi (r) / \varphi_*$ for the symmetron model, and $f_R (r) / \bar{f}_R$ for the $f(R)$ model.
    \textit{Right Panel:} The same as the middle panel but for the coupled quintessence model.
    }
    \label{fig:CodeTests3D}
\end{figure}


\subsubsection*{Spherical tophat overdensity}

For the symmetron and coupled quintessence models, instead of a point mass test, we have considered a spherical tophat overdensity with radius $\tilde{R}_{\rm TH}$ located at the centre of the simulation box $(\tilde{x}, \tilde{y}, \tilde{z}) = (N_{\rm g}/2, N_{\rm g}/2, N_{\rm g}/2)$. Note that code units are used here. The overdensity field is given by 
\begin{align}
    \tilde{\delta}_{\rm TH} \left(\tilde{r}\right) = \begin{cases}
        \tilde{\delta}_{\rm in}, & \tilde{r} \le \tilde{R}_{\rm TH} \\
        \tilde{\delta}_{\rm out}, & \tilde{r} > \tilde{R}_{\rm TH}
    \end{cases},
\end{align}
where $\tilde{r} \equiv \sqrt{(\tilde{x}-N_{\rm g}/2)^2 + (\tilde{y}-N_{\rm g}/2)^2 + (\tilde{z}-N_{\rm g}/2)^2}$ is the distance from the tophat centre, and we have adopted $\tilde{R}_{\rm TH} = 0.1 N_{\rm g}, \tilde{\delta}_{\rm in} = 5000$ and $\tilde{\delta}_{\rm out} = 0$ in our tests. In spherical symmetry, the scalar field equations for the symmetron  (Eq.~\eqref{eq:sym_eom_code_unit}) and coupled quintessence (Eq.~\eqref{eq:csf_eom_code}) models reduce to the following 1D ordinary differential equations,
\begin{align}
    \tilde{c}^2 \frac{1}{\tilde{r}^2} \frac{\dd}{\dd{\tilde{r}}} \qty[\tilde{r}^2 \frac{\dd{u}}{\dd{\tilde{r}}}] &= \frac{a^2}{2\xi^2}\left[ \qty(1 + \tilde{\delta}_{\rm TH}(\tilde{r}))\frac{a_\ast^3}{a^3}-1\right]u + \frac{a^2}{2\xi^2}u^3 \label{eq:sym_eom_code_unit_spherical}
\intertext{and}
    \frac{1}{\tilde{r}^2} \frac{\dd}{\dd{\tilde{r}}} \qty[\tilde{r}^2 \frac{\dd{\tilde{\varphi}}}{\dd{\tilde{r}}}] &= \frac{3\beta\Omega_{\rm m}}{a} e^{\beta\bar{\varphi}}\left[\exp\left(\beta\frac{\tilde{\varphi}}{\tilde{c}^2}\right) \qty(1+\tilde{\delta}_{\rm TH}(r))-1\right] \notag \\
    &\phantom{=} -\alpha\lambda^2a^2\left[\frac{1}{\left(\bar{\varphi}+\tilde{c}^{-2}\tilde{\varphi}\right)^{1+\alpha}}-\frac{1}{\bar{\varphi}^{1+\alpha}}\right] , \label{eq:csf_eom_code_spherical}
\end{align}
respectively.

These two 1D equations are numerically solved using the \textsc{maple} software, with the following boundary conditions on the interval $r \in [0, N_{\rm g}/2]$,
\begin{align}
    u(\tilde{r} = N_{\rm g}/2) = \sqrt{1 - \qty(\frac{a_*}{a})^3}, \quad \frac{\dd{u}}{\dd{\tilde{r}}}(\tilde{r}=0) = 0\,, \\
\intertext{and}
    \tilde{\varphi}(\tilde{r} = N_{\rm g}/2) = 0, \quad \frac{\dd{\tilde{\varphi}}}{\dd{\tilde{r}}}(\tilde{r}=0) = 0\,,
\end{align}
for the symmetron and coupled quintessence models respectively. Note that rigorously speaking the first boundary condition should really have been set at $\tilde{r}\rightarrow\infty$, but for numerical implementation this is impractical and we instead use $N_{\rm g}/2$ as an approximation to $\infty$. 

We have obtained the numerical solutions of these ODEs for $u(\tilde{r})$ and $\tilde{\varphi}(\tilde{r})$, but still call them `analytical' to distinguish from the numerical solutions directly solved from the original PDEs solved by \textsc{mg-glam}. The model parameters are the same as in the uniform and 1D density tests: for the symmetron model we have used $a_\ast = 0.5, \xi = 10^{-3}$, $\beta_\ast = 0.1$, while for coupled quintessence we have used $\alpha = 0.1, \beta = -0.2$.

The analytical and numerical solutions for the symmetron and coupled quintessence models are displayed in the middle and right panels of Fig.~\ref{fig:CodeTests3D}, respectively as the black solid line and black symbols. They agree very well.

As a comparison, in the middle panel of Fig.~\ref{fig:CodeTests3D} we have also shown, with coloured symbols, the \textsc{mg-glam} solutions for the $f(R)$ model with $f_{R0}=-10^{-5}$ and $n=0$ (blue), $1$ (orange) and $2$ (green). This can serve as a quick comparison of the screening efficiencies in these four models. First, we note that in all three $f(R)$ models the solution, $f_R(\tilde{r})/\bar{f}_{R0}$, tends to $1$ far from the spherical tophat, which is expected because the scalar field approaches its background value far from the matter perturbation at the centre. Second, for all four models, the scalar field is strongly suppressed inside the tophat (grey shaded region), but increases sharply immediately outside $\tilde{R}_{\rm TH}$ such that within some small distance from the edge of the tophat it already reaches $\gtrsim50\%$ of the background value: this is what one would expect from the $f(R)$ and symmetron models---both of which are examples of the so-called \textit{thin-shell screened} models \cite{Brax:2012gr}. Third, comparing the solutions of the three $f(R)$ models with the same $f_{R0}$, it seems that increasing the value of $n$ increases the screening efficency, implying that the $n=2$ case has the strongest screening amongst them; we shall see the consequence of this in the cosmological simulations in the next section. Finally, comparing the tested symmetron model with the $f(R)$ ones, it seems that the solution of the former lies somewhere in between the $n=0$ and $n=1$ cases (at least near the tophat); however, we caution that the fifth forces in the two models are obtained in different ways: in $f(R)$ gravity it is directly proportional to $\boldsymbol{\nabla}f_{R}$, while for symmetrons it is proportional to $\boldsymbol{\nabla}\left(u^2\right)$, cf.~Eq.~\eqref{eq:F_5_codeunit}, rather than $\boldsymbol{\nabla}u$. 



\subsection{Convergence tests}
\label{subsect:convergence_tests}

As mentioned in Sect.~\ref{subsubsect:relaxation}, we have implemented three different arrangements of the multigrid solver --- V-, F- and W-cycles. To compare them we have run a series of small cosmological simulations for the $f(R)$ gravity model with $f_{R0} = -10^{-5}$ and $n=1$, the symmetron model with $a_\ast = 0.3, \xi = 10^{-3}$ and $\beta_\ast = 0.1$ and the coupled quintessence model with $\alpha = 0.1$ and $\beta = -0.2$. These runs all use $L = 256 \, h^{-1}\mathrm{Mpc}$, $N_{\rm p}^3 = 512^3$ and $N_{\rm g}^3 = 1024^3$ for the smaller simulations. We consider $10$ and $2$ V-cycles (V10 and V2), $1$ F-cycle (F1) and 1 W-cycle (W1) to test the convergence of the \ac{MG} scalar field solutions. The V10 simulation results are used as the benchmark of our test. 
For F- and W-cycles we only conisder one cycle because, as will be shown below, this already gives excellently converged results.

Figure \ref{fig:ConvergenceTest} shows the relative differences
of the matter power spectra at $z=0$ between the test simulations described above and the benchmark case (V10), for $f(R)$ gravity (left), and the symmetron (middle) and coupled quintessence (right) models. We find that all the different schemes and different numbers of cycles used to solve the partial differential equations have good agreement on almost all scales probed by the simulations ($\lesssim0.4\%$). However, when more cycles are used, the run time gets longer, and the slowest simulations are those using V10. 
F-cycles and W-cycles are more effective in reducing residuals, both agreeing with V10 by $\lesssim0.05\%$ after only one cycle, which is not surprising since they walk more times across the fine and coarse multigrid levels. As a result, Both F1 and W1 are slower than V2. Therefore, as a compromise between accuracy and cost, we decide to always use V2 in our cosmological runs. 

It is a great achievement for the multigrid solver to reach convergence after just 2 V-cycles (and 2 Gauss-Seidel passings of the entire mesh in each cycle), for nonlinear equations in
the $f(R)$ gravity and symmetron models.

\begin{figure}
    \centering 
    \includegraphics[width=\textwidth]{./fig/ConvergenceTest_1x3.pdf}
    \caption{A comparison of the convergence with different multigrid arrangements and numbers of cycles of the Gauss-Seidel relaxation. The fractional differences in the matter power spectra are plotted at $z=0$, obtained for different multigrid schemes (V2, V10, F1, W1) using V10 as the reference. 
    The cases shown are for the $f(R)$ model with $f_{R0} = -10^{-5}$ and $n=1$ (left panel), the symmetron model with $a_* = 0.3, \xi=10^{-3}$ and $\beta_* = 0.1$ (middle panel), and the coupled quintessence model with $\alpha = 0.1, \beta = -0.2$ (right panel).}
    \label{fig:ConvergenceTest}
\end{figure}

\subsection{Comparisons with previous simulations}
\label{subsect:comparions}

\begin{figure}
    \centering 
    \includegraphics[width=\textwidth]{./fig/Pk_HMF_MGGLAMvsArepo.pdf}
    \caption{Comparison of matter power spectra (left panel) and halo mass functions (right panel) predicted by simulations with the same box size and particle number, using the \textsc{mg-glam} (black dashed lines with symbols) and \textsc{mg-arepo} (red solid lines) codes for the same $f(R)$ model, $n=1$ and $-f_{R0}=10^{-5}$. The upper subpanels show the absolute measurements from the simulations, while the lower subpanels show the relative differences from the counterpart $\Lambda$CDM runs. The vertical dashed line in the right panel denotes $10^{12.5}h^{-1}M_\odot$. The two codes agree very well above this mass.
    }
    \label{fig:Pk_hmf_MGGLAMvsArepo}
\end{figure}

As a final test of the \textsc{mg-glam} code, we compare its predictions from cosmological simulations with those by other modified gravity codes in the literature. We do this for the $f(R)$ and symmetron models only, since the coupled quintessence model is more trivial: the fifth force in this model is unscreened, and has a nearly constant ratio with the strength of Newtonian gravity in space \cite{Li:2010re}. 

For $f(R)$ gravity, we have run two \textsc{mg-glam} simulations for the model $f_{R0}=-10^{-5}, n=1$, using a box size $L=512 \, h^{-1}\mathrm{Mpc}$ with $N_{\rm p}^3 = 1024^3$ particles and  $N_{\rm g}^3=2048^3$ mesh cells. These are compared with the predictions from a simulation using \textsc{mg-arepo}, with $L = 500 \, h^{-1}\mathrm{Mpc}$ and $N_{\rm p}^3 = 1024^3$. All these simulations have the same cosmological parameters, but they started from different realisations of initial conditions (ICs). Since \textsc{mg-arepo} uses adaptive mesh refinement for the modified gravity force and trees for the Newtonian force with a softening length of $\approx 15 \, h^{-1}\mathrm{kpc}$, it achieves better force resolution as compared with the \textsc{mg-glam} simulations that use a regular mesh with $N_{\rm g}=2048$ giving a force resolution of $0.25 \, h^{-1}\mathrm{Mpc}$. Despite this, we will see that \textsc{mg-glam} can reproduce the \textsc{mg-arepo} results on scales of interest.



In the left panel of Fig.~\ref{fig:Pk_hmf_MGGLAMvsArepo} we compare the matter power spectra, $P_{mm}(k)$, from the \textsc{mg-glam} (lines) and \textsc{mg-arepo} (symbols) simulations. The upper subpanel shows the absolute $P(k)$, where the two codes agree down to $k \approx 1 \, h\,\mathrm{Mpc}^{-1}$. As shown in \cite{Klypin:2017iwu}, with a mesh resolution of $0.25 \, h^{-1} \mathrm{Mpc}$ the \textsc{glam} code is capable of predicting $P_{mm}(k)$ with percent-level accuracy down to $k \approx 1 \, h\,\mathrm{Mpc}^{-1}$. The lower subpanel shows the enhancements of the matter power spectrum due to $f(R)$ gravity. To obtain this, we have also run a counterpart $\Lambda$CDM simulation for each of the $f(R)$ simulations, using the same box size, grid number, particle number, cosmological parameters and initial conditions; we then take the relative difference between an $f(R)$ run and its counterpart $\Lambda$CDM run. We can see an excellent agreement between the two codes, down to $k \approx 3 \, h\,\mathrm{Mpc}^{-1}$ (even though the power spectra themselves agree only down to $k \approx 1 \, h\,\mathrm{Mpc}^{-1}$).  

The right panel of Fig.~\ref{fig:Pk_hmf_MGGLAMvsArepo} extends the comparison to the differential halo mass function (dHMF). The dHMF is a description of the halo abundance; more accurately, it quantifies the number density of haloes, in a spatial volume, that falls into a given halo mass bin. In the upper subpanel we present the dHMFs measured from the \textsc{mg-glam} and \textsc{mg-arepo} simulations, while in the lower subpanel we show the enhancements with respect to their counterpart $\Lambda$CDM runs. As we mentioned above, \textsc{mg-glam} uses the spherical overdensity halo mass definition with the virial halo overdensity, $M_{\rm vir}$. On the other hand, \textsc{mg-arepo} by default uses the $M_{200c}$ halo mass difinition, which is defined by requiring the mean overdensity within the halo radius $R_\Delta$, to be $\Delta=200\rho_{\rm crit}(z)$. To be self-consistent, we have rerun \textsc{mg-arepo}'s halo finder, \textsc{subfind} \cite{Springel2001}, using the $M_{\rm vir}$ definition. %\footnote{We have checked explicitly that, if we use $M_{\rm vir}$ for \textsc{mg-glam} and $M_{200c}$ for \textsc{mg-arepo}, then no surprisingly there can be a substantial discrepancy between their dHMF predictions.}. 
The upper subpanel shows that, at this specific mesh resolution, the dHMF predicted by \textsc{mg-glam} is complete down to $10^{12.5}h^{-1}M_\odot$, and agrees with \textsc{mg-arepo} for $M>10^{12.5}h^{-1}M_\odot$. In addition, the dHMF enhancements due to $f(R)$ gravity predicted by these two codes also agree very well, despite being noisy at the high-mass end due to the small box sizes used here.

Overall, for the $f(R)$ version, we find very good agreement between \textsc{mg-glam} and \textsc{mg-arepo}. We have also compared the \textsc{mg-glam} simulation results with predictions by \textsc{ecosmog} (although the results are not presented here), and obtained as good agreements as shown in Fig.~\ref{fig:Pk_hmf_MGGLAMvsArepo} for both $P_{mm}(k)$ and the HMF.



\begin{figure}
    \centering 
    \includegraphics[width=0.7\textwidth]{./fig/Pk_sym_MGGLAMvsECOSMOG.pdf}
    \caption{\revision{Comparison of the matter power spectrum enhancements with respect to $\Lambda$CDM, for the symmetron model with $a_\ast=0.33, \beta_\ast=1$ and $\xi=3.34 \times 10^{-4}$ at $z=0$, from two previous simulations run with \textsc{ecosmog} \citep{2012JCAP...10..002B} (red symbols) and \textsc{mlapm} \citep{2012ApJ...756..166W} (black symbols) respectively, and two groups of \textsc{mg-glam} runs with box sizes of $128$ (thin grey lines) and $512 \, h^{-1}\mathrm{Mpc}$ (thick blue line).  The box sizes used in the \textsc{mlapm} and \textsc{ecosmog} runs are respectively $64\,h^{-1}\mathrm{Mpc}$ and $128\,h^{-1}\mathrm{Mpc}$. The $25$ small-box \textsc{mg-glam} realisations have substantial sample variance (as shown by the strong scatters) on small scales due to the box size, and these curves are also much noisier compared with the large-box result. The \textsc{ecosmog} and \textsc{mlapm} results are close to the two limits of the scatters in the 25 \textsc{mg-glam} runs. The large scale behaviour of the \textsc{mlapm} result is likely due to its very small box size.}}
    \label{fig:Pk_sym_MGGLAMvsECOSMOG}
\end{figure}


For the symmetron model, we have three \textsc{mg-glam} runs for the parameter values $a_\ast=0.33, \beta_\ast=1$ and $\xi = 3.34 \times 10^{-4}$, and we compare the measured matter power spectra with those presented in \citep{2012JCAP...10..002B} and \citep{2012ApJ...756..166W} using the adaptive mesh refinements codes \textsc{ecosmog} and \textsc{mlapm}, respectively.
The \textsc{ecosmog} symmetron run followed the evolution of  $N_{\rm p}^3 = 256^3$ particles in a box of size $L = 128 \, h^{-1}\mathrm{Mpc}$, and the domain grid (defined as the finest uniform grid which covers the whole simulation box) has $N_{\rm g}^3 = 256^3$ cells. For the \textsc{mlapm} simulation, the box size is $64 \, h^{-1} \mathrm{Mpc}$, the particle number is $256^3$ and the domain grid cell number is $128^3$. 

\revision{Fig.~\ref{fig:Pk_sym_MGGLAMvsECOSMOG} shows the matter power spectrum enhancement, $\Delta P / P_{\mathrm{GR}}$, between a pair of $\Lambda$CDM and MG simulations starting from the same initial conditions, for the \textsc{mg-glam} runs (solid lines) and the data taken from Ref.~\citep{2012JCAP...10..002B} (red symbols) and \citep{2012ApJ...756..166W} (black symbols). 
For the power spectrum itself, the sample variance should be smaller on smaller scales, which have more $k$~modes than large scales. For $\Delta P / P_{\mathrm{GR}}$, however, we see the opposite behaviour: the sample variance is substantially suppressed on large scales ($k < 0.1 \, h^{-1}\mathrm{Mpc}$) where the evolution is largely linear and different $k$~modes uncoupled to each other. On small scales, different Fourier modes are coupled together, and two different gravity models that have different strengths of gravity would lead to different levels of such coupling.
Hence, the difference between the power spectra in these two models at the same high $k$ can be substantial, especially when the box size is small and therefore the result is more susceptible to rare large objects present in the box.
An example to illustrate this point is the bottom panel of Fig.~5 of Ref.~\cite{2011PhRvD..83d4007Z}, which compares the $\Delta P / P_{\mathrm{GR}}$ from an $f(R)$ simulation and a `linearised' (no-chameleon) $f(R)$ simulation which has its chameleon screening effects removed --- the latter case corresponds to a much stronger gravity and a much larger power spectrum enhancement, and therefore much more significant scatters.  }

\revision{In Fig.~\ref{fig:Pk_sym_MGGLAMvsECOSMOG}, we find that the \textsc{mg-glam} simulations for the $25$ independent realisations with a small box ($L = 128 \, h^{-1}\mathrm{Mpc}$) have strong scatters in $\Delta P / P_{\mathrm{GR}}$ on small scales while not on large scales.
The previous simulation results from \textsc{ecosmog} and \textsc{mlapm} follow into this range of scatters.
Note that the \textsc{mlapm} simulation has a box size of $64 \, h^{-1} \mathrm{Mpc}$ which can likely explain its behaviour on large scales.
We also show the result from a single $512 \, h^{-1} \mathrm{Mpc}$ box.
Within the uncertainties allowed by sample variance, all three codes seem to agree with each other.}



\subsection{Summary}
\label{subsect:code_test_summary}

To quickly sum up this section: we have done a number of tests of different aspects of the \textsc{mg-glam} code. These include the test of the background cosmology solver for the coupled quintessence model (cf.~Sect.~\ref{subsect:bg_tests}), tests of the multigrid relaxation solver of the scalar field equations for different density configurations (cf.~Sects.~\ref{subsubsec:homogeneous_rhom_test}, \ref{subsubsect:1D_tests}, \ref{subsubsect:3D_tests}), convergence property tests of the relaxation solvers with three different multigrid arrangements (V-cycle, F-cycle and W-cycle), and additionally comparisons of \textsc{mg-glam} cosmological simulations with runs using other codes. We see that \textsc{mg-glam} satisfactorily passes all these tests, and gives reasonable results. 

