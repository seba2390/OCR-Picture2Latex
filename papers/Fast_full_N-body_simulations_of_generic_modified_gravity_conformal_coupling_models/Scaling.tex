\section{Scalability Test}
\label{sec:scaling}

% We run a series of tests (taking $f(R)$ gravity as a representative) to demonstrate the scaling performace of \textsc{mg-glam} on the SKUN6@IAA facility managed by the IAA-CSIC in Spain.


% The strong scaling tests of \textsc{mg-glam} are presented in the left panel of Fig.~\ref{fig:scaling}. The test simulations employed a fixed resolution of $256^3$ particles and $512^3$ grids, with the same Planck 2015 cosmology as used in the main \textsc{mg-glam} runs of this paper. The plot shows that, when the number of \textsc{openmp} threads ranges between 1 and $\sim 30$, the wallclock time roughly scales linearly with the thread number. The deviation from a linear scaling when the number of threads exceeds $30$ is possibly due to the small size of the test run.

% We also ran a set of simulations of different sizes by varying the resolutions and keeping the number of threads fixed. The wallclock time used is shown in the right panel of Fig.~\ref{fig:scaling}. We see that the time consumption scales nearly perfectly linearly with the considered resolutions (up to $N_{\rm g} = 4096$ and $N_{\rm p} = N_{\rm g} / 2$).


% \begin{figure}
%     \centering 
%     \includegraphics[width=\textwidth]{./fig/scaling.pdf}
%     \caption{
%         \textit{Left panel:} The wallclock time of the \textsc{mg-glam} test runs for the F5n1 model, with fixed simulation size and resolution ($L=128h^{-1}\mathrm{Mpc}$, $N_{\rm p} = 256$, $N_{\rm g} = 512$), as a function of the number of threads used in \textsc{openmp} parallelisation.
%         The time-thread number scaling is approximately linear for thread numbers up to $\sim 30$.
%         \textit{Right panel:} The wallclock time of the \textsc{mg-glam} runs for F5n1 with varying simulation sizes and resolutions (from left to right, $N_{\rm g} = 256, 512, 1024, 2048, 4096$, and $N_{\rm p} = N_{\rm g}/2$), while the number of threads is fixed to \textcolor{red}{$56$}. Again the scaling is linear. In both cases, the symbols denote the times taken by the test runs, and the lines denote the expected results with a `perfect scaling'.
%     }
%     \label{fig:scaling}
% \end{figure}


