
\section{Theories}
\label{sec:theories}

In this section we will describe several classes of theoretical models which will later be implemented in the modified {\sc glam} code. The main purpose of this description is to make this paper self-contained, and so we will keep it concise. Interested readers can find more details in the literature elsewhere. 

Consider a general model where a scalar field, $\phi$, couples to matter, described by the following action
\begin{equation}
    S = \int{\rm d}^4x\sqrt{-g}\left[\frac{M^2_{\rm Pl}}{2}R - \frac{1}{2}\nabla^\mu\phi\nabla_\mu\phi-V(\phi)\right] + \sum_i\int{\rm d}^4x\sqrt{-\hat{g}}\mathcal{L}^m\left[\psi_i,\hat{g}_{\mu\nu}\right].
\end{equation}
Here the first term is the gravitational action, where $g$ is the determinant of the metric tensor $g_{\mu\nu}$, $M_{\rm Pl}$ the reduced Planck mass, $R$ the Ricci scalar, $\nabla^\mu$ the covariant derivative, and $V(\phi)$ the potential energy of the scalar field $\phi$. The second term is the matter action, which sums over all matter species labelled by $i$, with $\psi$ being the matter field and $\hat{g}_{\mu\nu}$ the metric that couples to it. In principle, $\hat{g}_{\mu\nu}$ can be different for different matter species, but we consider the universal $\hat{g}_{\mu\nu}$ here for simplicity. 

The Jordan-frame metric $\hat{g}_{\mu\nu}$ and the Einstein-frame metric $g_{\mu\nu}$ are related to each other by the following conformal scaling
\begin{equation}\label{eq:conformal_metric}
    \hat{g}_{\mu\nu} = A^2(\phi)g_{\mu\nu}.
\end{equation}
Here we work in the \textit{Einstein frame}, in which the effect of the scalar field is in the matter sector, i.e., modified geodesics for matter particles, while the left-hand sides of the Einstein equations keep their standard form. This is in contrast to the \textit{Jordan frame}, where the scalar field manifestly modifies the curvature terms on the left side of the Einstein equation. However, in a classical sense the physics is the same in these two frames. Note that the relation between $g_{\mu\nu}$ and $\hat{g}_{\mu\nu}$ can be more complicated, e.g., including a disformal term, but these possibilities are beyond the scope of the present work.

The scalar field is a dynamical and physical degree of freedom in this model, which is governed by the following equation of motion
\begin{equation}\label{eq:csf_eom_general}
    \nabla^\mu\nabla_\mu\phi = \frac{{\rm d}A(\phi)}{{\rm d}\phi}\left[\rho_{\rm m}-3P_m\right] + \frac{{\rm d}V(\phi)}{{\rm d}\phi},
\end{equation}
where $\rho_{\rm m}$ and $P_m$ are respectively the density and pressure of non-relativistic matter (radiation species do not contribute due to the conformal nature of Eq.~\eqref{eq:conformal_metric}). We also define the coupling strength $\beta(\phi)$ as a dimensionless function of $\phi$:
\begin{equation}\label{eq:csf_beta_general}
    \beta(\phi) \equiv M_{\rm Pl}\frac{{\rm d}\ln A(\phi)}{\phi}.
\end{equation}
Note the $M_{\rm Pl}$ in this definition, which is because $\phi$ has dimensions of mass. For later convenience, we shall define a dimensionless scalar field as
\begin{equation}
    \varphi \equiv \frac{\phi}{M_{\rm Pl}}. 
\end{equation}
We can see from Eq.~\eqref{eq:csf_eom_general} that, in addition to the self-interaction of the scalar field $\phi$, described by its potential energy, $V(\phi)$, the matter coupling means that the dynamics of $\phi$ is also affected by the presence of matter. We can therefore define an \textit{effective total potential} of the scalar field, $V_{\rm eff}(\phi)$, as
\begin{equation}\label{eq:Veff}
    V_{\rm eff}(\phi) \equiv A(\phi)\rho_{\rm m} + V(\phi),
\end{equation}
where we have used $P_m=0$ for matter. With appropriate choices of $V(\phi)$ and $A(\phi)$, the effective potential $V_{\rm eff}(\phi)$ may have one or more minima, i.e., ${\rm d}V_{\rm eff}/{\rm d}\phi=0$ at $\phi=\phi_{\rm min}$. Provided that the shape of $V_{\rm eff}(\phi)$ is sufficiently steep around $\phi_{\rm min}$, as in some classes of models to be studied below, the scalar field can oscillate around it, and we can define a scalar field mass, $m$, as
\begin{equation}
    m^2 \equiv \frac{{\rm d}^2V_{\rm eff}\left(\phi_{\rm min}\right)}{{\rm d}\phi^2}.
\end{equation}

For non-relativistic matter particles, the interaction with the scalar field introduces new terms in their geodesic equations,
\begin{equation}\label{eq:csf_particle_geodesic_general}
    \dot{u}^\mu + \frac{\dot{\phi}}{M_{\rm Pl}}u^\mu = -c\frac{\beta(\phi)}{M_{\rm Pl}}\nabla^\mu\phi,
\end{equation}
where $u^\mu\equiv{\rm d}x^{\mu}/{\rm d}\tau$ is the 4-velocity, and overdot denotes the time derivative.

In the weak-field limit where the metric $g_{\mu\nu}$ can be written through the following line element,
\begin{equation}
    {\rm d}s^2 = -(1+2\Phi)c^2{\rm d}t^2 + (1-2\Phi){\rm d}x^i{\rm d}x_i,
\end{equation}
where $\Phi$ is the Newtonian potential, we can approximately write Eq.~\eqref{eq:csf_particle_geodesic_general} as
\begin{equation}\label{eq:csf_particle_geodesic_qsa}
    \ddot{\boldsymbol{r}} = -\boldsymbol{\nabla}\Phi - c^2\frac{\beta(\phi)}{M_{\rm Pl}}\boldsymbol{\nabla}\phi - \frac{\beta(\phi)}{M_{\rm Pl}}\dot{\phi}\dot{\boldsymbol{r}},
\end{equation}
where $\boldsymbol{r}$ is the physical coordinate of the particle and $\boldsymbol{\nabla}$ is the gradient with respect to the physical coordinate. 

\revision{The gravitational potential $\Phi$ and the \textcolor{black}{perturbation to the} MG scalar field have small values in Newtonian $N$-body simulations.}  
% Relativistic $N$-body codes, such as \textit{gevolution} \cite{Adamek2016JCAP...07..053A}, $k$-evolution \cite{Hassani:2019JCAP...12..011H} and \textsc{gramses} \cite{Barrera-Hinojosa:2020JCAP...04..056B,Barrera-Hinojosa:2020JCAP...01..007B}, take a leap beyond the weal-field approximation. 
%However, the relativistic effects are expected to be significant only on horizon scales and negligible on nonlinear scales.}
\textcolor{black}{Some relativistic cosmological simulation codes, such as \textsc{gramses} \citep{Barrera-Hinojosa:2020JCAP...01..007B,Barrera-Hinojosa:2020JCAP...04..056B}, go beyond the weak-field approximation by including higher-order terms of the gravitational potentials, but find the effect on small scales is indeed small.}

Eq.~\eqref{eq:csf_particle_geodesic_qsa} summarises three of the key effects that a coupled scalar field can have on cosmic structure formation: (1) a \textit{fifth force}, as given by the gradient of $\phi$, (2) a \textit{frictional force} that is proportional to $\dot{\phi}$ and the particle's velocity $\dot{\boldsymbol{r}}$ -- this is similar to the usual \revision{`frictional'} force caused by the Hubble expansion $H$, but because $H$ can be modified by the coupled scalar field too, we have a third effect through a modified $H$, which is implicit in Eq.~\eqref{eq:csf_particle_geodesic_qsa}.


In the same limit, the scalar field equation of motion, Eq.~\eqref{eq:csf_eom_general}, can be simplified as
\begin{equation}\label{eq:csf_eom_qsa}
    c^2\boldsymbol{\nabla}^2\phi \approx V_\phi(\phi) - V_\phi(\bar{\phi}) + A_\phi(\phi)\rho_{\rm m} - A_\phi(\bar{\phi})\bar{\rho}_{\rm m},
\end{equation}
where an overbar denotes the background value of a quantity, and $V_\phi\equiv{\rm d}V(\phi)/{\rm d}\phi$, $A_\phi\equiv{\rm d}A(\phi)/{\rm d}\phi$. In deriving Eq.~\eqref{eq:csf_eom_qsa} we have used the weak field approximation, as well as the quasi-static approximation which enables use to neglect the time derivative of the scalar field perturbation, $\delta\phi\equiv\phi-\bar{\phi}$, compared with its spatial gradient, i.e., $|\ddot{\delta\phi}|\simeq|H\dot{\delta\phi}|\ll|\boldsymbol{\nabla}^2\delta\phi|=|\boldsymbol{\nabla}^2\phi|$, where $H\equiv\dot{a}/a$ is the Hubble expansion rate. \revision{It is important to note that we do not assume that $\ddot{\bar{\phi}} \ll |\boldsymbol{\nabla}^2 \phi|$, because $\ddot{\bar{\phi}}$ and $H \dot{\bar{\phi}}$ can be significant in certain models such as coupled quintessence, where $\bar{\phi}$ can evolve by a large amount throughout the cosmic history.}

\revision{The quasi-static approximation has been tested for the modified gravity theories considered in this paper, such as $f(R)$ gravity \cite{Oyaizu:2008PhRvD..78l3523O,Bose:2015JCAP...02..034B} and symmetron \cite{Llinares:2014PhRvD..89h4023L}. Ref.~\cite{Oyaizu:2008PhRvD..78l3523O} performed a consistency check of this approximation for Hu-Sawicki $f(R)$ gravity \cite{Hu:2007nk}, where the simulations were run in the quasi-static limit but it was checked that the time derivative of the scalar field perturbation is generally $5$--$6$ orders of magnitude smaller than its spatial derivative in amplitude. Ref.~\cite{Bose:2015JCAP...02..034B} directly examined this approximation by running full simulations including the time derivative terms, and found that the effects of the scalar field time derivative terms can be safely ignored in  Hu-Sawicki $f(R)$ gravity. For the symmetron model, the quasi-static approximation has also been widely used in previous literature, e.g., \citep{Davis:2011pj,2012JCAP...10..002B}. 
Ref.~\cite{Llinares:2014PhRvD..89h4023L} ran simulations with non-static terms and found very little difference in the matter power spectrum with the quasi-static simulations. However, the local power spectrum (defined as the $P(k)$ for the filtered matter field) shows deviations of the order of $1\%$. Therefore, it is expected that the quasi-static approximation is valid for usual cosmological probes such as power spectra which we are interested in, but other properties may be affected.}

{According to these researches, the quasi-static approximation is valid for our cosmological analyses.
The effects of the scalar field time derivatives are small enough that can be safely ignored for the nonlinear evolution of dark matter fields.
}

Finally, the Newtonian potential $\Phi$ is governed by the following Poisson equation, again written under the weak-field and quasi-static approximations,
\begin{equation}\label{eq:csf_poisson_qsa}
    \boldsymbol{\nabla}^2\Phi \approx 4\pi{G}A(\bar{\phi})\left(\rho_{\rm m} - \bar{\rho}_{\rm m}\right),
\end{equation}
where we note the presence of $A(\bar{\phi})$ in front of $\rho_{\rm m}$, which is because the coupling to the scalar field $\phi$ can cause a time evolution of the particle masses of non-relativistic species, therefore affecting the depth of the resulting potential well $\Phi$. This is the fourth key effect a coupled scalar field can have on cosmic structure formation. In the models considered in this paper, either the scalar field perturbation is small such that $A(\phi)\simeq A(\bar{\phi})$, or the scalar field has a small amplitude ($|{\varphi}|\ll1$) in the entire cosmological regime so that $A(\phi)\simeq1$ and $A(\bar{\phi})\simeq1$. 

Eqs.~(\ref{eq:csf_particle_geodesic_qsa}, \ref{eq:csf_eom_qsa}, \ref{eq:csf_poisson_qsa}) are the three key equations to be solved in our $N$-body simulations.

\subsection{Coupled quintessence}
\label{subsect:csf}

The behaviour of the coupled scalar field, as well as its effect on the cosmological evolution, is fully specified with concrete choices of the coupling function $A(\phi)$ and scalar potential $V(\phi)$. Such models are known as \textit{coupled quintessence} \cite{Amendola:1999er}, and have been studied extensively in the literature, including simulation analyses.

With some choices of $A(\phi)$ and $V(\phi)$, the scalar field dynamics can become highly nonlinear, such as in the symmetron and chameleon models described below. These models are often display very little evolution of the background scalar field ($|\Delta\varphi|\ll1$) throughout the cosmic history so that the background expansion rate closely mimics that of $\Lambda$CDM; the spatial perturbations of $\varphi$ can reach $|\delta\varphi|\simeq|\bar{\varphi}|\ll1$. In other, more general, cases, the scalar field can have a substantial dynamical evolution, $|\Delta\varphi|\sim\mathcal{O}(1)$ and $|\delta\varphi|\ll|\bar{\varphi}|$, which allows deviations from the $\Lambda$CDM expansion history, and the fifth force behaves in a less nonlinear way. This latter case is the focus in this subsection. 

We consider an exponential coupling function
\begin{equation}\label{eq:csf_coupling_function}
    A(\phi) = \exp\left(\beta\frac{\phi}{M_{\rm Pl}}\right) = \exp\left(\beta\varphi\right),
\end{equation}
and an inverse power-law potential
\begin{equation}\label{eq:csf_potential}
    V(\phi) = \frac{M^\alpha_{\rm Pl}\Lambda^4}{\phi^\alpha} = \frac{\Lambda^4}{\varphi^\alpha},
\end{equation}
where $\alpha,\beta$ are dimensionless model parameters, and $\Lambda$ is a model parameter with mass dimension 1 which represents a new energy scale related to the cosmic acceleration. For convenience, we define a dimensionless order-unity parameter $\lambda$ as
\begin{equation}\label{eq:csf_param_lambda}
    \frac{\Lambda^4}{M_{\rm Pl}^2} = H_0^2\lambda^2.
\end{equation}
We consider parameters $\alpha>0$, so that $V(\phi)$ is a runaway potential and the scalar field rolls down $V(\phi)$, and $\beta<0$ so that the effective potential $V_{\rm eff}(\phi)$ has no minimum and the scalar field can keep rolling down $V_{\rm eff}(\phi)$ if not stopped by other effects. This means that we can have $|\bar{\varphi}|\sim\mathcal{O}(1)$ at late times (as mentioned in the previous paragraph) and kinetic energy makes up a substantial fraction of the scalar field's total energy (so that its equation of state $w_\phi$ can deviate substantially from $-1$). 

While we specialise to Eqs.~(\ref{eq:csf_coupling_function}, \ref{eq:csf_potential}) for the coupled quintessence models in this paper, the {\sc mg}-{\sc glam} code that we will illustrate below using this model can be applied to other choices of $A(\phi)$ and $V(\phi)$ with minor changes in a few places, to allow fast, inexpensive and accurate simulations for generic coupled quintessence models.

For completeness and convenience of later discussions, we also present here the linear growth equation for matter density contrast $\delta$ (or equivalently the linear growth factor $D_+$ itself) in the above coupled quintessence model:
\begin{equation}\label{eq:csf_lin_growth}
    \delta'' + \left[\frac{a'}{a}+\frac{{\rm d}\ln{A}(\bar{\varphi})}{{\rm d}\varphi}\bar{\varphi}'\right]\delta' - 4\pi{G}\bar{\rho}_{\rm m}(a){a}^2A(\bar{\varphi})\left(1+2\beta^2\right)\delta = 0,
\end{equation}
where $'$ denotes the derivative with respect to the conformal time $\tau$. According to this equation, there are 4 effects that the coupled scalar field has on structure formation: (\textit{i}) a modified expansion history, $a'/a$; (\textit{ii}) a fifth force whose ratio with respect to the strength of the standard Newtonian force is given by $2\beta^2$; (\textit{iii}) a rescaling of the matter density field by $A(\varphi)\neq1$ in the Poisson equation, implying that the matter particle mass is effectively modified; and (\textit{iv}) a velocity-dependent force described by the term involving $\left({\rm d}\ln A/{\rm d}\varphi\right)\bar{\varphi}'\delta'$. The ratio between the fifth and Newtonian forces can be derived as follows: Eq.~\eqref{eq:csf_eom_qsa} can be approximately rewritten as
\begin{equation}\label{eq:csf_eom_qsa_approx}
    \boldsymbol{\nabla}^2\left(c^2\delta\varphi\right) \approx 8\pi G \beta A(\bar{\varphi})\left[\rho_{\rm m} - \bar{\rho}_{\rm m}\right],
\end{equation}
where we have used $A_\phi=\frac{\beta}{M_{\rm Pl}}\exp(\beta\varphi)$, $M_{\rm Pl}^{-2}=8\pi G$, and neglected the contribution the scalar field potetial $V(\phi)$ in the field perturbation $\delta\varphi$. Then, from Eqs.~\eqref{eq:csf_poisson_qsa} and \eqref{eq:csf_particle_geodesic_qsa}, it follows that the ratio of the two forces is $2\beta^2$, which means that the fifth force always boosts the total force experienced by matter particles in this model. In addition, since $\beta$ is a constant, from Eq.~\eqref{eq:csf_lin_growth} we can conclude that the enhancement to linear matter growth, i.e., in the linear growth factor and matter power spectrum, will be scale-independent.

\subsection{Symmetrons}
\label{subsect:sym}

The \textit{symmetron} \cite{Hinterbichler:2010es,Hinterbichler:2011ca} model features the following potential $V(\phi)$ and coupling function $A(\phi)$ for the scalar field:
\begin{eqnarray}\label{eq:sym_potential}
    V(\phi) &=& V_0 - \frac{1}{2}\mu^2\phi^2 + \frac{1}{4}\zeta\phi^4,\\
    \label{eq:sym_coupling_function} A(\phi) &=& 1+ \frac{1}{2}\frac{\phi^2}{M^2},
\end{eqnarray}
where $\mu, M$ are model parameters of mass dimension 1, $\zeta$ is a dimensionless model parameter and $V_0$ is a constant parameter of mass dimension 4, which represents vacuum energy and acts to accelerate the Hubble expansion rate.

We can define 
\begin{equation}
    \phi_\ast \equiv \frac{\mu}{\sqrt{\zeta}},
\end{equation}
which represents the local minimum of the Mexican-hat-shaped symmetron potential $V(\phi)$. The total effective potential of the scalar field, however, is given in Eq.~\eqref{eq:Veff}. Because $A(\phi)$ is a quadratic function of $\phi$, when $\rho_{\rm m}$ is large, the effective potential is dominated by $A(\phi)\rho_{\rm m}$, with single global minimum at $\phi=0$; but when $\rho_{\rm m}$ is small, the effective potential is dominated by $V(\phi)$ and has two minima, $\pm\phi_{\rm min}$.  Explicitly, it can be shown that $\phi_{\rm min}=0$ when $\bar{\rho}_{\rm m}>\mu^2M^2\equiv\rho_\ast$ in background cosmology, while otherwise the symmetry in $V(\phi)$ is broken and the symmetron field solutions are given by
\begin{equation}\label{eq:phi_min_sym}
    \pm\phi_{\rm min} = \sqrt{\frac{1}{\zeta{M}^2}\left(\rho_\ast-\bar{\rho}_{\rm m}\right)},
\end{equation}
from which we can confirm the above statement that as $\bar{\rho}_{\rm m}\rightarrow0$ we have $\phi_{\rm min}\rightarrow\phi_\ast$. Because $\rho_\ast$ has the dimension of density, it is more convenient to express it in terms of a characteristic scale factor $a_\ast$ or redshift $z_\ast$ corresponding to the time of symmetry breaking in $V_{\rm eff}(\phi)$:
\begin{equation}
    \rho_\ast = \bar{\rho}_{m0}a^{-3}_\ast,
\end{equation}
where $\bar{\rho}_{m0}$ is the background matter density today. According to Eq.~\eqref{eq:phi_min_sym}, as $\rho_{\rm m}\rightarrow0$, $\phi_{\rm min}\rightarrow\phi_\ast$, i.e., $\phi_{\rm min}$ approaches the minimum of $V(\phi)$. Therefore, we must have $\phi_{\rm min}\in[0,\phi_\ast]$. For this reason we can define the following dimensionless variable
\begin{equation}
    u \equiv \frac{\phi}{\phi_\ast}\in[0,1),
\end{equation}
Note that this is only true for background $u$, while in the perturbed case it is possible to have $u>1$ in certain regions. Also, $u>0$ is just a choice, because the symmetron field has two physically identical branches of solutions which differ by sign, and we choose the positive branch for simplicity\footnote{Indeed, it is possible that $u$ can have different signs in different regions of the Universe, which are separated by domain walls, but we do not consider this more realistic possibility in this paper, as it does not have a big impact on the observables of interest to us.}. In terms of the dimensionless scalar field $\varphi$, we have \cite{Brax:2012a}
\begin{equation}\label{eq:varphi_min_symm}
    \varphi_{\rm min}(a) = \varphi_\ast\sqrt{1-\left(\frac{a_\ast}{a}\right)^3},
\end{equation}
with
\begin{equation}\label{eq:varphi_ast}
    \varphi_\ast \equiv \frac{\phi_\ast}{M_{\rm Pl}} =  6\Omega_m\beta_\ast\xi^2a_{\ast}^{-3},
\end{equation}
where $\Omega_m$ is the matter density parameter today, $\xi\equiv H_0/m_\ast$ with $m_\ast$ being the `mass' of the scalar field at $\phi_\ast$, given by
\begin{equation}
    m_\ast^2 \equiv \frac{{\rm d}^2V(\phi_\ast)}{{\rm d}^2\phi} = -\mu^2 + 3\zeta\phi^2_\ast = 2\mu^2,
\end{equation}
and $\beta_\ast$ is a dimensionless parameter defined through
\begin{equation}
    M_{\rm Pl}\frac{{\rm d}A}{{\rm d}\phi} = \frac{M_{\rm Pl}\phi}{M^2} \equiv \beta_\ast\frac{\phi}{\phi_\ast},
\end{equation}
which can be further expressed as
\begin{equation}
    \beta_\ast \equiv \frac{M_{\rm Pl}}{M^2}\frac{\mu}{\sqrt{\zeta}} = \frac{M_{\rm Pl}m_\ast^2}{2\rho_\ast}\phi_\ast.
\end{equation}

Therefore, the model can be fully specified by three dimensionless parameters -- $\beta_\ast$, $a_\ast$ (or $z_\ast$) and $\xi$ -- as opposed to the original, dimensional, parameters $\mu, \zeta$, $M$. We are interested in the regime of $\beta_\ast, a_\ast\sim\mathcal{O}(0.1)$ and $\xi\sim\mathcal{O}\left(10^{-3}\right)$. It is then evident from Eqs.~(\ref{eq:varphi_ast}) that $\varphi_\ast\ll1$ and therefore $\varphi_{\rm min}(a)\ll1$, confirming our claim above that in this model the scalar field has little evolution throughout the cosmic history. For simplicity we will assume that in the background the scalar field always follows $\varphi_{\rm min}$, namely $\bar{\varphi}(a)=\varphi_{\rm min}(a)$\footnote{In practice, because $\varphi_{\rm min}(a)$ evolves with time, when trying to track it, $\bar{\varphi}$ can have oscillations around $\varphi_{\rm min}$ because $m_\ast\gg H(a)\simeq H_0$. Following most literature on the symmetron model, we will neglect these oscillations.}. Further, because $\varphi_{\rm min}\simeq\varphi_\ast\ll1$, we have
\begin{equation}
    A(\phi) = 1 + \frac{1}{2}\beta_\ast\frac{\varphi}{\varphi_\ast}\varphi \simeq 1,
\end{equation}
which implies that the time variation of particle mass is negligible in this model, and 
\begin{equation}\label{eq:sym_coupling_strength}
    \beta(\phi) = M_{\rm Pl}\frac{{\rm d\ln A(\phi)}}{{\rm d}\phi} \simeq \frac{{\rm dA(\varphi)}}{{\rm d\varphi}} = \beta_\ast\frac{\varphi}{\varphi_\ast} = \beta_\ast u,
\end{equation}
so that $\beta_\ast$ characterises the coupling strength between the scalar field and matter in this model. 

With all the newly-defined variables, the scalar field equation of motion, Eq.~\eqref{eq:csf_eom_general}, in this model can be simplified as 
\begin{equation}
    c^2\nabla^2\frac{\varphi}{\varphi_\ast} = \frac{1}{2}\xi^{-2}H_0^2a^2\frac{\varphi}{\varphi_\ast}\left(\frac{\varphi^2}{\varphi_\ast^2}-1\right) + \frac{1}{2}\xi^{-2}H_0^2a_\ast^3\frac{\varphi}{\varphi_\ast}\frac{{\rho}_m}{\bar{\rho}_{\rm m}}a^{-1},
\end{equation}
or equivalently
\begin{equation}\label{eq:sym_eom}
    c^2\nabla^2u = \frac{1}{2}\xi^{-2}H_0^2a^2u\left(u^2-1\right) + \frac{1}{2}\xi^{-2}H_0^2a_\ast^3u(1+\delta)a^{-1},
\end{equation}
where the density contrast is defined as
\begin{equation}
    \delta \equiv \frac{\rho_{\rm m}}{\bar{\rho}_{\rm m}}-1.
\end{equation}

The symmetron model and its extensions have been studied with the help of numerical simulations in several works \cite{Davis:2011pj,Brax:2012a}, but the large computational cost has so far made it impossible to run large, high-resolution simulations for a very large number of parameter combinations, which is why we are implementing it in {\sc mg}-{\sc glam}. This model features the \textit{symmetron screening mechanism} \cite{Hinterbichler:2010es}, which helps to suppress the fifth force in dense environments by driving $\varphi\rightarrow0$ so that the coupling strength $\beta(\phi)\rightarrow0$, cf.~Eq.~\eqref{eq:sym_coupling_strength}. This essentially decouples the scalar field from matter and therefore eliminates the fifth force in these environments, such that the model could evade stringent local and Solar System constraints. The \textit{dilaton screening mechanism} \cite{Brax:2010gi} is another class of coupled scalar field models with a screening mechanism that works similarly, so in this paper we shall focus on the symmetron model only.

\subsection{Chameleon $f(R)$ gravity}
\label{subsect:fR}

$f(R)$ gravity \cite{Sotiriou:2008rp,DeFelice:2010aj} is a very popular class of modified gravity models, which can be described by the following gravitational action
\begin{equation}\label{eq:fR_action}
    S = \frac{M^2_{\rm Pl}}{2}\int{\rm d}^4x\sqrt{-g}\left[R+f(R)\right],
\end{equation}
simply replacing the cosmological constant $\Lambda$ with an algebraic function of the Ricci scalar, $f(R)$. It is well known that this theory can be equivalently rewritten as a scalar-tensor theory after a change of variable, and is therefore mathematically and physically equivalent to a coupled scalar field model in which the scalar field has a universal coupling to different matter species. Therefore it belongs to the general models introduced in the beginning of this section. The model is fully specified by fixing the function $f(R)$, with different choices of $f(R)$ equivalent to coupled scalar field models with different forms of the scalar potential $V(\phi)$. Meanwhile, the coupling strength of the scalar field is a constant $\beta=1/\sqrt{6}$ for all $f(R)$ models\footnote{This means that the ratio between the strengths of the fifth and the standard Newtonian forces is at most $1+2\beta^2=1/3$. For more details see below.}, independent of $f(R)$. Despite this limitation, this model still has very rich phenomenology, and in this paper we will study it in the original form given by Eq.~\eqref{eq:fR_action}, instead of studying its equivalent coupled scalar field model.

With certain choices of the function $f(R)$, the model can have the so-called \textit{chameleon screening} mechanism \cite{Khoury:2003aq,Khoury:2003rn,Mota:2006fz,Brax:2008hh}, which can help the fifth force to hide from experimental detections in dense environments where $\rho_{\rm m}$ is high and the scalar field acquires a large mass $m$ and therefore its strength decays exponentially and essentially vanishes beyond a typical distance of order $m^{-1}$. Of course, not all choices of $f(R)$ can lead to a viable chameleon screening mechanism, and in this paper we will focus only on those where the chameleon mechanism works, and we call the latter \textit{chameleon $f(R)$ gravity}. 

In $f(R)$ gravity, the Einstein equation is modified to
\begin{equation}\label{eq:fR_modified_einstein_eqn}
    G_{\mu\nu} - X_{\mu\nu} = 8\pi{G}T_{\mu\nu},
\end{equation}
where $T_{\mu\nu}$ is the energy-momentum tensor, $G_{\mu\nu}\equiv{R}_{\mu\nu}-\frac{1}{2}g_{\mu\nu}R$ is the Einstein tensor with $R_{\mu\nu}$ being the Ricci tensor, and $X_{\mu\nu}$ is defined as
\begin{equation}\label{eq:Xmn}
    X_{\mu\nu} \equiv -f_RR_{\mu\nu} + \frac{1}{2}\left[f(R)-\nabla^\lambda\nabla_\lambda f_R\right]g_{\mu\nu} + \nabla_\mu\nabla_\nu f_R,
\end{equation}
where $f_R\equiv{\rm d}f(R)/{\rm d}R$ is a new dynamical scalar degree of freedom, with the following equation of motion
\begin{equation}\label{eq:fR_eom}
    \nabla^\mu\nabla_\mu{f}_R = \frac{1}{3}\left[R-f_RR+2f(R)-8\pi{G}\rho_{\rm m}\right].
\end{equation}

One of the leading choices of the function $f(R)$ was the one proposed by Hu \& Sawicki \cite{Hu:2007nk}. In this paper, instead of using the original function form provided in \cite{Hu:2007nk}, we present it in an approximate form which will allow us to generalise it. Let's start with the following expression of $f_R(R)$,
\begin{equation}\label{eq:fR_HS}
    f_R(R) = -\left|f_{R0}\right|\left(\frac{\bar{R}_0}{R}\right)^{n+1},
\end{equation}
where $f_{R0}$ is the present-day value of the background $f_R$, $\bar{R}_0$ is the background Ricci scalar today, and $n\geq0$ is an integer. For $n>0$, the functional form $f(R)$ can be written as
\begin{equation}\label{eq:fR_HS_original}
    f(R) \approx -6H_0^2\Omega_\Lambda + \frac{1}{n}\left|f_{R0}\right|\left(\frac{\bar{R}_0}{R}\right)^{n+1}R,
\end{equation}
where $\Omega_\Lambda=1-\Omega_m$ and the first term represents a cosmological constant that is responsible for the cosmic acceleration. For $n=0$, we have
\begin{equation}\label{eq:fR_log}
    f(R) \approx -6H_0^2\Omega_\Lambda + \left|f_{R0}\right|\bar{R}_0\ln\left(\frac{\bar{R}_0}{R}\right).
\end{equation}
Most of the simulation works to date have been performed for the case of $n=1$, while the cases of $n=2$ and $n=0$ are not as well explored. In this paper we implement all three cases into {\sc mg}-{\sc glam}. 

In general, for the model to have a viable chameleon screening, the parameter $f_{R0}$ in Eq.~\eqref{eq:fR_HS} should satisfy $|f_{R0}|\ll1$. At late times, when $\bar{R}(a)\simeq\bar{R}_0$, we can see from Eqs.~(\ref{eq:fR_HS_original}, \ref{eq:fR_log}) that the relation $f(R)\simeq-6H_0^2\Omega_\Lambda$ holds. On the other hand, from Eq.~\eqref{eq:fR_HS} we have $|f_R|\ll1$ throughout the cosmic history, i.e., it has a negligible evolution in time. This implies that all the terms in $X_{\mu\nu}$ in Eq.~\eqref{eq:Xmn} other than $f(R)$ can be neglected compared with the $f(R)$ term, and so the model behaves approximately like $\Lambda$CDM in the background expansion rate, with the background Ricci scalar given by
\begin{equation}\label{eq:R_bar}
    \bar{R}(a) = 3\mathbb{M}^2\left(a^{-3}+4\frac{\Omega_\Lambda}{\Omega_m}\right),
\end{equation}
and $\mathbb{M}^2\equiv{H}_0^2\Omega_m$. This is compatible with what we mentioned above, i.e., in the coupled scalar field model that is equivalent to these $f(R)$ models, the scalar field $\phi$ has little time evolution and therefore has an equation of state which is very close to $-1$. It also implies that the weak-field approximation, where we can neglect the time evolution of the scalar degree of freedom $f_R$, is a good approximation, so that in an inhomogeneous Universe we have
\begin{equation}\label{eq:fR_eom_qsa}
    \boldsymbol{\nabla}^2f_R \approx \frac{1}{3c^2}\left[\delta R-8\pi G\delta\rho_{\rm m}\right]a^2,
\end{equation}
where $\boldsymbol{\nabla}$ is the gradient with respect to the comoving coordinate, as before, $\delta\rho_{\rm m}\equiv\rho_{\rm m}-\bar{\rho}_{\rm m}=\bar{\rho}_{\rm m}\delta$, and 
\begin{equation}
    \delta R = R - \bar{R}.
\end{equation}
By realising that Eq.~\eqref{eq:fR_HS} can be inverted to give
\begin{equation}\label{eq:R_fR}
    R = \bar{R}_0\left(\frac{f_{R0}}{f_R}\right)^{\frac{1}{n+1}}.
\end{equation}
With Eq.~\eqref{eq:R_fR}, Eq.~\eqref{eq:fR_eom_qsa} becomes a nonlinear dynamical equation for $f_R$. 

Also under the quasi-static and weak-field approximations, the Poisson equation takes the following modified form
\begin{equation}\label{eq:fR_Poisson_qsa}
    \boldsymbol{\nabla}^2\Phi \approx \frac{16\pi{G}}{3}\delta\rho_{\rm m}a^2 - \frac{1}{6}\delta{R}a^2 = 
    4\pi{G} \bar{\rho}_{\rm m} a^2 \delta - \frac{1}{2} c^2\boldsymbol{\nabla}^2 f_R \,,
\end{equation}
where in the second step we have used Eq.~\eqref{eq:fR_eom_qsa}.

One can have a quick peek into two opposite regimes of solutions for Eqs.~(\ref{eq:fR_eom_qsa}, \ref{eq:fR_Poisson_qsa}). In the large field limit, when $|f_R|$ is relatively large (e.g., in the case of large $|f_{R0}|$), the perturbation $\delta f_R$ is small compared to the background field $|\bar{f}_R|$, and $|\delta R|\ll8\pi{G}\delta\rho_{\rm m}$, so that the Poisson equation \eqref{eq:fR_Poisson_qsa} can be approximated as 
\begin{equation}
    \boldsymbol{\nabla}^2\Phi \approx \frac{16\pi{G}}{3}\delta\rho_{\rm m}a^2.
\end{equation}
Comparing this with the standard Poisson equation in $\Lambda$CDM,
\begin{equation}\label{eq:GR_Poisson_qsa}
    \boldsymbol{\nabla}^2\Phi \approx {4\pi{G}}\delta\rho_{\rm m}a^2,
\end{equation}
we confirm that the fifth force, i.e., the enhancement of gravity, is $1/3$ of the strength of the standard Newtonian force. In the opposite, small-field, limit where $|f_R|$ takes very small values, the left-hand side of Eq.~\eqref{eq:fR_eom_qsa} is negligible and so we have $\delta{R}\approx8\pi{G}\delta\rho_{\rm m}$, and plugging this into Eq.~\eqref{eq:fR_Poisson_qsa} we recover Eq.~\eqref{eq:GR_Poisson_qsa}: this is the screened regime where the fifth force is strongly suppressed.

\subsection{Summary and comments}
\label{subsect:model_summary}

In this section we have briefly summarised the essentials of the three classes of scalar field modified gravity models to be considered in this work. Among these, coupled quintessence is technically more trivial, because the fifth force is unscreened nearly everywhere, while $f(R)$ gravity and symmetrons are both representative thin-shell screening models \cite{Brax:2012gr} featuring two of the most important screening mechanisms respectively. Compared with previous simulation work, we will consider $f(R)$ models with more values of the parameter $n$: as discussed below, instead of the common choice of $n=1$, we will also look at $n=0,2$ to see how the phenomenology of the model varies.  

We remark that, even with the additional modified gravity models implemented in this paper, as well as the models implemented in the twin paper \cite{Hernandez-Aguayo:2021_twin_paper}, we are still far from covering all possible models. Changing the coupling function $A(\varphi)$ or the scalar field potential $V(\varphi)$, as an example, will lead to new models. However, our objective is to have an efficient simulation code that covers different \textit{types} of models, which serves as a `prototype' that can be very easily modified for any other models belonging to the same type. This differs from the model-independent \cite{Srinivasan:2021gib} or parameterised modified gravity \cite{Hassani:2020rxd} approaches adopted elsewhere, and we perfer this approach since there is a direct link to some fundamental Lagrangian here, and because, any parameterisation of models, one its parameters specified, also corresponds to a fixed model.