\section{Effects of mass and force resolution}
\label{appendix:resolution}

The original \textsc{glam} code has been well tested (cf.~\citep{Klypin:2017iwu}) by examining the effects of time-stepping and force resolution and comparing with the high-resolution MultiDark simulations \citep{Klypin:2016MNRAS.457.4340K} which were performed using the \textsc{gadget} code \citep{Springel:2005_Gadget_code_paper}. In addition, Ref.~\citep{Klypin:2020tud} compared the \textsc{glam} results of halo mass functions and matter power spectra with those of the \textsc{quijote} simulations \citep{Villaescusa-Navarro:2020ApJS..250....2V}, and found good agreement. Denote $k_{1\%}$ as the wavenumber above which the \textsc{glam} matter power spectrum begins to deviate by more than $1\%$ from those of the high-resolution simulations. Based on the comparison with the MultiDark simulations, the authors found that $k_{1\%}$ is related to the force resolution $\Delta x = L_{\rm box}/N_{\rm g}$ (cf. Eq.~\eqref{eqn:force_resolution_def}) as 
\begin{align}
    k_{1\%} = \frac{0.25 \pm 0.05}{(\Delta x) / (h^{-1} \mathrm{Mpc})} \, h \, \mathrm{Mpc}^{-1} \ .
\end{align}
If this relation also works for the \textsc{mg-glam} code, the power spectra of the modified gravity cosmological runs presented in the main text are reliable down to $k_{1\%} \sim 1 \, h \, \mathrm{Mpc}^{-1}$. However, as mentioned in the main text above, the results of the power spectrum enhancement with respect to $\Lambda$CDM are empirically reliable down to larger $k$.


To test the effect of mass and force resolutions, we have performed four $f(R)$ gravity simulations for $f_{R0} = -10^{-5}, n = 1$ with fixed box size $512 \, h^{-1}\mathrm{Mpc}$ and varying particle and grid numbers
\begin{align*}
    (N_{\rm p}, N_{\rm g}) = \begin{Bmatrix}
    (1024, 2048),& (2048, 2048) \\ 
    (1024, 4096),& (2048, 4096)
    \end{Bmatrix}\ ,
\end{align*} 
which correspond to mass and force resolutions of 
\begin{align*}
    \qty( \displaystyle \frac{m_{\rm p}}{10^9 \, h^{-1} M_{\odot}}, \frac{\Delta x}{h^{-1} \mathrm{Mpc}} ) = \begin{Bmatrix}
    (11.0, 0.25\phantom{1}), &(1.37, 0.25\phantom{1}), \\
    (11.0, 0.125), &(1.37, 0.125)
    \end{Bmatrix} \ .
\end{align*}
Here the two runs in the same row (column) have the same force/mesh (mass) resolution. The adopted cosmological parameters are the same as the simulations used in the main text.

We compare the \textsc{mg-glam} matter power spectrum and halo mass function enhancement $\Delta P / P_{\rm GR}$ and $\Delta \mathrm{HMF} / \mathrm{HMF}_{\rm GR}$ with those of the \textsc{mg-arepo} simulations.
We focus on these quantities instead of comparing the absolute $P(k)$ and HMF from (\textsc{mg}-)\textsc{glam} and other codes, for the following reasons: (1) as mentioned above, the reliability of the $\Lambda$CDM results from \textsc{glam} has been carefully tested and established; (2) comparisons between different codes usually suffer from cosmic variance and different implementation details (such as the IC set up, force calculation, time stepping and halo finding), and as a result a large number of runs are needed to make reliable comparisons, after carefully calibrating simulation specifications of the different codes --- such an effort is unnecessary and beyond the scope of this work given (1); (3) in MG simulations, people are often more interested in the enhancement with respect to $\Lambda$CDM, and this is indeed what has been tested in the code papers of the previous MG simulation codes. Also, we note that the \textsc{mg-glam} and \textsc{mg-arepo} simulations presented in this work use slightly different cosmological parameters: we have checked explicitly (by running test simulations with \textsc{mg-glam} using identical cosmological parameters as the \textsc{mg-arepo} runs) that the effect is small (few percent level), but this nevertheless still makes it difficult to justify directly comparing the absolute $P(k)$ or HMF from them; the enhancement, on the other hand, is known empirically to be less sensitive to cosmological parameter values and differences between simulation codes.

%which  may be different in the base codes \textsc{glam} and \textsc{arepo}. Comparisons of the \ac{MG} code predictions for the relative difference to $\Lambda$CDM should be  independent of these aspects and mostly determined by the \ac{MG} equation solvers.


\begin{figure}
    \centering 
    \includegraphics[width=\textwidth]{./fig/PkEnhancement_ResolutionTest_MGGLAMvsMGArepo_noeps.pdf}
    \caption{
        Comparison of matter power spectra from \textsc{mg-glam} (lines), \textsc{mg-gadget} (squares and circles) and \textsc{mg-arepo} (crosses) simulations at $z=0$. The left panel shows the matter spectrum enhancement, $\Delta P/P_{\rm GR}$, from the different codes and resolutions, as the legend labels. The \textsc{mg-arepo} data are the same as in Fig.~\ref{fig:Pk_hmf_MGGLAMvsArepo}, while the \textsc{mg-gadget} data are from the two \textsc{lightcone} simulations, at higher ($L=768h^{-1}\mathrm{Mpc}$ and $N_{\rm p}^3=2048^3$) and lower ($L=1536h^{-1}\mathrm{Mpc}$ and $N_{\rm p}^3=2048^3$) resolutions, described in Ref.~\cite{Arnold:2018nmv}. The upper right panel shows the absolute values of $P(k)$ from \textsc{mg-glam} simulations with the four combinations of mass and force resolutions. In the lower right panel, the ratios of the power spectrum in each simulation to that of the highest resolution run ($N_{\rm p} = 2048$ and $N_{\rm g} = 4096$) are displayed, where the dark and light grey shaded regions denote respectively $\pm1\%$ and $\pm2\%$ differences from the benchmark. The vertical lines represent $k=1h\mathrm{Mpc}^{-1}$.
    }
    \label{fig:PkEnhancement_ResolutionTest_MGGLAMvsMGArepo}
\end{figure}


The left panel of Fig.~\ref{fig:PkEnhancement_ResolutionTest_MGGLAMvsMGArepo} presents the matter power spectrum enhancements at $z = 0$ from \textsc{mg-glam} and \textsc{mg-arepo}, as well as two \textsc{mg-gadget} simulations. We see that $\Delta P / P_{\rm GR}$ is relatively insensitive to the mass and force resolution variations considered here; this is consistent with previous experiences. However, increasing the mesh resolution from $0.25$ to $0.125h^{-1}\mathrm{Mpc}$ does improve the agreement between \textsc{mg-glam} and \textsc{mg-arepo}, by reducing $\Delta P / P_{\rm GR}$ (see, e.g., \cite{Li:2013MNRAS.428..743L} for a discussion of how a lower resolution simulation gives higher $\Delta P / P_{\rm GR}$). The highest resolution \textsc{mg-glam} run ($N_{\rm p} = 2048$ and $N_{\rm g} = 4096$) agrees with \textsc{mg-arepo} nearly perfectly down to $k \sim 1 \, h \,\mathrm{Mpc}^{-1}$, and the agreement is at the level of a couple percent down to $k\approx5h\mathrm{Mpc}^{-1}$ (ignoring the dip in $\Delta P / P_{\rm GR}$ at $k\approx4h\mathrm{Mpc}^{-1}$, which is apparently not physical). The slightly larger deviations at $k > 1 \, h \,\mathrm{Mpc}^{-1}$ can be still due to the lower force resolutions in the \textsc{glam} simulations, but we note that the agreement between the \textsc{mg-gadget} and \textsc{mg-arepo} runs (which have similar force resolutions) is at a comparable level, so the difference is likely also partly due to the different codes (or simulation realisations).

In the upper right panel of Fig.~\ref{fig:PkEnhancement_ResolutionTest_MGGLAMvsMGArepo}, we present the absolute matter power spectra from the \textsc{mg-glam} simulations at different resolutions. As expected, increasing the mesh/force resolution leads to a $P(k)$ curve that decays much more slowly at small scales (orange and green lines), while increasing the mass resolution (blue) gives little improvement. The lower right panel of Fig.~\ref{fig:PkEnhancement_ResolutionTest_MGGLAMvsMGArepo} shows the ratio of the matter spectrum in each simulation to that from the highest resolution run. The figures indicate $\approx1 \%$ convergence for $k \lesssim 1 \, h \,\mathrm{Mpc}^{-1}$, which is consistent with the convergence test of the original \textsc{glam} code presented in \citep{Klypin:2017iwu}.

The comparisons of halo mass functions are shown in Fig.~\ref{fig:dHMF_ResTest_MGGLAMvsArepo}, where note that we used different halo finders for the \textsc{mg-glam} and \textsc{mg-arepo} results, but the same halo mass definition, as described in Sect.~\ref{subsect:comparions}. The HMFs of \textsc{mg-glam} simulations are accurate in the range of $M_{\rm vir} \gtrsim 10^{12.5} \, h^{-1}M_{\odot}$ for $\Delta x = 0.25 \, h^{-1}\mathrm{Mpc}$ ($N_{\rm g} = 2048$), and $M_{\rm vir} \gtrsim 10^{12}\, h^{-1}M_{\odot}$ for $\Delta x = 0.125 \, h^{-1}\mathrm{Mpc}$ ($N_{\rm g} = 4096$). There is excellent agreement between \textsc{mg-glam}'s higher-resolution runs and \textsc{mg-arepo}, in both the HMF and its enhancement, down to $10^{12}h^{-1}M_\odot$.



\begin{figure}
    \centering 
    \includegraphics[width=\textwidth]{./fig/dHMF_ResTest_MGGLAMvsArepo.pdf}
    \caption{Comparison of halo mass functions of \textsc{mg-glam} and \textsc{mg-arepo} simulations at $z=0$.
    The relative enhancements with respect to $\Lambda$CDM and the absolute values of the HMFs are shown in the left and right panels, respectively. The two vertical lines in the right panel denote respectively the masses $10^{12}$ and $10^{12.5}h^{-1}M_\odot$. There is excellent agreement between \textsc{mg-glam}'s higher-resolution runs and \textsc{mg-arepo}, in both the HMF and its enhancement, down to $10^{12}h^{-1}M_\odot$.
    }
    \label{fig:dHMF_ResTest_MGGLAMvsArepo}
\end{figure}


