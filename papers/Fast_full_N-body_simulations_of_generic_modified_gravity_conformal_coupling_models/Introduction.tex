\section{Introduction}
\label{sec:intro}

The accelerated expansion of our Universe \cite{SupernovaCosmologyProject:1998vns,SupernovaSearchTeam:1998fmf} is one of the most challenging problems in modern physics, and after decades of attempts to find its origin, we are still far from reaching a clear conclusion. 
While the current standard cosmological model --- $\Lambda$ Cold Dark Matter ($\Lambda$CDM), which assumes that this accelerated expansion is caused by the cosmological constant, $\Lambda$ --- is in excellent agreement with most observational data to \revision{date}, this model suffers from the well-known coincidence and fine-tuning problems. This suggests that a more fundamental theory is yet to be developed which can naturally explain the small value of $\Lambda$ inferred from observations. The alternative theoretical models proposed so far can be roughly classified into two categories: those that involve some exotic new matter species beyond the standard model of particle physics, the so-called \textit{dark energy} \cite{Copeland:2006wr}, which usually has non-trivial dynamics; and the other which involve modifications to Einstein's \ac{GR} on certain (usually cosmic) scales \citep{Clifton:2011jh,2015PhR...568....1J,Koyama:2020zce}, or introduces new fundamental forces between matter particles\footnote{The two classes of models can not always be clearly distinguished, and some of the modified gravity models studied here can also considered as coupled dark energy.}. Leading examples include: quintessence \cite{Ratra:1988_quintessence,Wetterich:1988_quintessence,Zlatev:1998tr_quintessence,Steinhardt:1999nw_quintessence}, k-essence \cite{Armendariz-Picon:2000nqq:kessence,Armendariz-Picon:2000ulo_kessense}, coupled quintessence \cite{Amendola:1999er}, $f(R)$ gravity \cite{Sotiriou:2008rp,DeFelice:2010aj} and chameleon model  \cite{Khoury:2003aq,Khoury:2003rn,Mota:2006fz,Brax:2008hh}, symmetron model \cite{Hinterbichler:2010es,Hinterbichler:2011ca}, the Dvali-Gabadadze-Porrati braneworld (DGP) model \cite{Davis:2011pj}, scalar \cite{Nicolis:2008in,Deffayet:2009wt} and vector \cite{Heisenberg:2014_Proca,Allys:2015sht_Proca,BeltranJimenez:2016rff_Proca} Galileons, K-mouflage \cite{Babichev:2009ee}, and massive gravity \cite[e.g.,][]{Hinterbichler:2011tt_massive_gravity}.

In \ac{MG} models, in addition to a modified, and accelerated,  expansion rate that could explain observations, often the law of gravity is also different from \ac{GR}, which can further affect the evolution of the large-scale structure (LSS) of the Universe. This suggests that we can use various cosmological observations to constrain and test these models \citep[e.g.,][]{Koyama:2015vza,Ferreira:2019xrr,Baker:2019gxo}.
In this sense, the study of \ac{MG} models can be used as a testbed to verify the validity of \ac{GR} on cosmological scales, hence going beyond the usual small-scale or local tests of \ac{GR} \cite{Will:2014_GR_LRR}.


In the last two decades, there have been substantial progresses in the size and quality of cosmological observations, many of which can be excellent probes of dark energy and modified gravity \citep[e.g.,][]{Albrecht:2006um,Weinberg_2013}. 
Some of the leading probes include cosmic microwave background (CMB) \cite{Hinshaw:WMAP9,Hou:SPT2014,Planck2018,Aiola:ACT2020}, supernovae \cite{SupernovaCosmologyProject:1998vns,SupernovaSearchTeam:1998fmf,Astier:2006_SNLS_SN,Wood-Vasey:2007_ESSENCE_SN,Sullivan:2011_SNLS_SN,Scolnic:2013_Pan_STARRS_SN,Rest:2013_Pan_STARRS_SN,Abbott:2018_DES_SN,Betoule:2014_SDSS_SN,Jones:2018_Pan_STARRS_SN}, galaxy clustering \citep{Percival:2004_2dF_Galaxy_Clustering,Guzzo:2008_Galaxy_Clustering,Blake:2011_wiggleZ_Galaxy_Clustering,Beutler:2012_6dF_Galaxy_Clustering,Pezzotta:2016_VIMOS_Galaxy_Clustering,Alam:2017_BOSS_Galaxy_Clustering,Zarrouk:2018_SDSS_Galaxy_Clustering} and baryonic acoustic oscillations (BAO)  \cite{Cole2005_BAO,Eisenstein2005_BAO,Beutler2011_BAO,Blake2011_BAO,Anderson2012_BAO,eBOSS2020_BAO}, gravitational lensing \cite{Heymans2013:CFHTLenS_WL,Abbott:2020_DES_WL_Clusters,Hamana2020:HSC_WL,Amon2021:DES_WL,Secco2021:DES_WL}, and the properties of galaxy clusters \cite{Vikhlinin:2009_Xray_Cluster_count,Planck:2013_SZ_Cluster,Mantz:2014a_Cluster,Mantz:2014b_Cluster,SPT:2016_SZ_Cluster,SPT:2018_SZ_Cluster,Abbott:2020_DES_WL_Clusters,Giocoli:2021_KiDS_WL_Cluster}.
In the near future, a number of large, Stage-IV, galaxy and cluster surveys, such as DESI \citep{DESI:2016zmz}, Euclid \citep{Laureijs:2011gra,EuclidTheoryWorkingGroup:2012gxx}, Vera Rubin observatory \cite{lsst} and eROSITA \cite{erosita:2012arXiv1209.3114M}, are expected to revolutionise our knowledge about the Universe and our understanding of the cosmic acceleration, by providing cutting-edge observational data with unprecedented volume and much better controlled systematic errors. Further down the line, experiments such as CMB-S4 \cite{CMBS4} and LISA \cite{LISA} will offer other independent tests of models by using  improved CMB observables, such as CMB lensing and the kinetic Sunyaev-Zel'dovich effect, and gravitational waves.
% We expect that a combination of these probes will 


To exploit the next generation of observational data, we need to develop accurate theoretical tools to predict the cosmological implications of various models, in particular their behaviour on small scales which encode a great wealth of information. However, predicting LSS formation on small scales is a non-trivial task because structure evolution is in the highly non-linear regime on these scales, with a lot of complicated physical processes, such as gravitational collapse and baryonic interactions, in play. The only tool that could accurately predict structure formation in this regime is cosmological simulations, which follow the evolution of matter through the cosmic time, from some initial, linear, density field all the way down to the highly-clustered matter distribution on small, sub-galactic, scales at late times. Modern cosmological simulation codes, e.g., \textsc{ramses} \cite{Teyssier:2001_RAMSES_code_paper}, \textsc{gadget} \cite{Springel:2005_Gadget_code_paper,Springel:2020plp}, \textsc{arepo} \cite{Springel:2010_AREPO_code_paper}, \textsc{pkdgrav} \cite{Potter:2016_PKDGRAV_code_paper}, \textsc{swift} \cite{Schaller:2016_SWIFT_code_paper}, \revision{\textsc{co\textsl{n}cept}\cite{Dakin2021arXiv211201508D}, \textit{gevolution}\cite{Adamek2016JCAP...07..053A}}, have been able to employ hundreds of billions or trillions of particles in giga-parsec volumes \cite[e.g.,][]{Angulo:2012_MXXL_sim_paper,Kim:2015_Horizon4_sim_paper,Potter:2016_PKDGRAV_code_paper}, and are nowdays indispensable in the confrontation of theories with observational data. In particular, to achieve the high level of precision required by galaxy surveys, one can generate hundreds or thousands of independent galaxy mocks that cover the expected survey volume, using these simulations. However, this has so far been impossible for \ac{MG} models, which usually involve highly non-linear partial differential equations that govern the new physics, solving which has proven to be very expensive even with the latest codes, e.g., \textsc{ecosmog} \cite{Li:2011_ECOSMOG_code_paper,Li:2013_ECOSMOGV_code_paper,2012JCAP...10..002B,Brax:2013mua}, \textsc{mg-gadget} \cite{Puchwein:2013_MGGADGET_code_paper}, \textsc{isis} \cite{Llinares:2013_ISIS_code_paper} and \textsc{mg-arepo} \cite{Arnold:2019_MGAREPO_code_paper,Hernandez-Aguayo:2020_MGAREPO_code_paper} (see \cite{Winther:2015_MG_code_comparison} for a comparison of several MG codes). For example, current \ac{MG} simulations can take between $2$ to $\mathcal{O}(10)$ times longer than standard $\Lambda$CDM simulations with the same specifications. Obviously, to best explore future observations for testing \ac{MG} models, we need a new simulation code for these models with greatly improved efficiency compared with the current generation of codes.

Here, we present such a code, {\sc mg-glam}, which is an extension of the parallel particle-mesh (PPM) $N$-body code {\sc glam}\footnote{{\sc glam} stands for GaLAxy Mocks, which is a pipeline for massive production of galaxy catalogues in the $\Lambda$CDM (GR) model.} \citep{Klypin:2017iwu}, in which various important classes of modified gravity models have been implemented. Efficiency is the main feature of \textsc{mg-glam}, which is partly thanks to the efficiency and optimisations it inherits from its base code,  \textsc{glam}\footnote{The \textsc{glam} code has been shown to be $1.6$--$4$ times faster than similar codes such as {\sc cola} \cite{Koda_15}, {\sc icecola} \cite{Izard_15} and {\sc fastpm} \cite{Feng_16}, while still achieving high resolution and accuracy.}, partly due to optimised numerical algorithms tailored to solve the nonlinear equations of motion in these modified gravity models, and partly thanks to a careful design of the code and data structures to reduce the memory footprint of the simulations.

Modified gravity models can be classified according to the fundamental properties of their new dynamical degrees of freedom, and the interactions the latter have. Here, we study three classes of MG models which introduce scalar-type degrees of freedom that have conformal-coupling interactions: coupled quintessence \cite{Amendola:1999er}, chameleon \cite{Khoury:2003aq,Khoury:2003rn} $f(R)$ gravity \citep{Hu:2007nk}, and symmetron models \citep{Hinterbichler:2010es,Hinterbichler:2011ca}. These models generally introduce a new force (\textit{fifth force}) between matter particles, and the latter two can be considered as special examples of the former, but differ in that they can both employ screening  mechanisms to evade Solar System constraints on the fifth force. These models have been widely studied in recent years and, as we argue below, the implementation of them can lead to prototype MG codes that can be modified to work with minimal effort for other classes of interesting models. %especially in the context of clusters of galaxies \citep[see e.g.,][]{Brax:2015lra,Mitchell:2020fnj,Mitchell:2021aex}, and make them ideal to be implemented in our code. 
In a twin paper \citep{Hernandez-Aguayo:2021_twin_paper}, we will describe the implementation and analysis of two classes of derivative-coupling MG models, including the DGP and K-mouflage models.

As we will demonstrate below, the inclusion of modified gravity solvers in \textsc{mg-glam} adds an overhead to the computational cost of \textsc{glam}, and for the models considered in this paper and its twin paper \cite{Hernandez-Aguayo:2021_twin_paper}, a \textsc{mg-glam} run takes about $3$-$5$ times (depending on the resolution) the computing time of an equivalent $\Lambda$CDM simulation using default \textsc{glam}. All in all, this makes this new code at least $100$ times faster than other modified gravity simulation codes such as \textsc{ecosmog} \cite{Li:2011_ECOSMOG_code_paper,Li:2013_ECOSMOGV_code_paper,2012JCAP...10..002B,Brax:2013mua} and \textsc{mg-arepo} \cite{Arnold:2019_MGAREPO_code_paper,Hernandez-Aguayo:2020_MGAREPO_code_paper} for the same simulation boxsize and particle number. In spite of such a massive improvement in speed over those latter codes, it is worthwhile to note that \textsc{mg-glam} is \textit{not} an approximate code: it solves the full Poisson and MG equations, and its accuracy is only limited by the resolution of the PM grid used, which can be specified by users based on their particular scientific objectives. This makes it different from fast approximate simulation codes such as those  \cite{Winther:2017j_MGCOLA_code_paper,Wright:2017_NUCOLA_code_paper,Valogiannis:2016_MGCOLA_code_paper,Fiorini:2021_MGCOLA_application_haloes} based on the COmoving Lagrangian Acceleration method (\textsc{cola}) \cite{Tassev:2013_COLA_code_paper}.



This paper is organised as follows. In Section~\ref{sec:theories}, we present a brief description of the conformally coupled \ac{MG} models covered in this work, which aims at offering a self-contained overview of the key theoretical properties which are relevant for the numerical code.  In Section~\ref{sect:numerics}, we present the details of our numerical implementations to solve the \ac{MG} scalar field equations, including the code and data structure, the implementation of the multigrid relaxation method to solve the MG equations, and the tailored relaxation alogorithms for each model. In Section~\ref{sect:code_tests}, we show various code test results, which help us to verify the accuracy and reliability of the code. Section~\ref{sect:cos_runs} shows the cosmological simulation results for a large suite of MG models, which serve to showcase the potential power of the new code. Finally we summarise and conclude in Section~\ref{sect:discuz}.

Throughout this paper, we adopt the usual conventions that Greek indices label all space-time coordinates ($\mu,\nu,\cdots=0,1,2,3$), while Latin indices label the space coordinates only ($i,j,k,\cdots=1,2,3$). Our metric signature is $(-,+,+,+)$. We will strive to include the speed of light $c$ explicitly in relevant equations, rather than setting it to $1$, given that in numerical implementations $c$ must be treated carefully. Unless otherwise stated, the symbol $\approx$ means `approximately equal' or `equal under certain approximations as detailed in the text', while the symbol $\simeq$ means that two quantities are of a similar order of magnitude. An overdot denotes the derivative with respect to (wrt) the cosmic time $t$, e.g., $\dot{a} \equiv {\dd{a}}/{\dd{t}}$ and the Hubble expansion rate $H(a)$ is defined as $H=\dot{a}/a$, while a prime ($'$) denotes the derivative wrt the conformal time $\tau$, e.g., $a'=\dd{a}/\dd{\tau}$, $\mathcal{H}(a)\equiv{a}'/a=aH(a)$. Unless otherwise stated, we use a subscript $0$ to denote the present-day value of a physical quantity, an overbar for the background value of a quantity, and a tilde for quantities written in code units.

We note that, since they have a lot in common, including the motivation and the design of code structure and algorithms, this paper has identical or similar texts with its twin paper \cite{Hernandez-Aguayo:2021_twin_paper} in the Introduction section, as well as in Sections~\ref{subsect:glam}, \ref{sec:glam_units}, \ref{subsect:extradof} until \ref{subsubsect:relaxation}, \ref{subsubsect:relaxation}, \ref{subsubsect:code_struc}, the last paragraph of \ref{subsubsect:csf_imp}, and part of \ref{subsect:bg_tests}. 
