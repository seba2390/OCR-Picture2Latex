\section{Applications}\label{sec:applications}
We discuss two applications and their restrictions for the general SGP~\eqref{eq:min_rand}--\eqref{eq:rewcomputation}.\vspace{-.2cm}
%
\paragraph{Model repair.}
%Bartocci \emph{et al.}~\cite{bartocci2011model} introduced the notion of \emph{model repair} for MCs. 
For MC $\dtmc$ and specification $\varphi$ with $\dtmc\not\models\varphi$, the \emph{model repair} problem~\cite{bartocci2011model} is to transform $\dtmc$ to $\dtmc'$ such that $\dtmc'\models\varphi$. The transformation involves a change of transition probabilities. Additionally, a cost function measures the change of probabilities. The natural underlying model is a pMC where parameters are added to probabilities. The cost function is minimized subject to constraints that induce satisfaction of $\varphi$. In~\cite{bartocci2011model}, the problem is given as NLP. Heuristic~\cite{pathak-et-al-nfm-2015} and simulation-based methods~\cite{chen2013model} (for MDPs) were presented.

Leveraging our results, one can readily encode model repair problems for MDPs, multiple objectives, and restrictions on probability or cost changes directly as NLPs. The encoding as in~\cite{bartocci2011model} is  handled by our method in Section~\ref{sec:approximation} as it involves signomial constraints. We now propose a more efficient approach, which encodes the change of probabilities using monomial functions.
%
Consider an MDP $\MdpInit$ and specifications $\varphi_1,\ldots,\varphi_q$ with $\mdp\not\models\varphi_1\land\ldots\land\varphi_q$. For each probability $\pmdp(s,\act,s')=a\in\R$ that may be changed, introduce a parameter $p$, forming the parameter set $\Paramvar$. We define a parametric transition probability function by $\pmdp'(s,\act,s')=p\cdot a\in\monos[V]$. The quadratic cost function is for instance  $f=\sum_{p\in V} p^2\in\posys[V]$. 
%Quadratic cost functions are also used in~\cite{chen2013model}.
By minimizing the sum of squares of the parameters (with some regularization), the change of probabilities is minimized.

By incorporating these modifications into SGP~\eqref{eq:min_rand}--\eqref{eq:rewcomputation}, our approach is directly applicable. Either we restrict the cost function $f$ to an upper bound, and efficiently solve a feasibility problem (Section~\ref{sec:geometric}), or we compute a local minimum of the cost function (Section~\ref{sec:approximation}).
%
In contrast to~\cite{bartocci2011model}, our approach works for MDPs and has an efficient solution. While~\cite{chen2013model} uses fast simulation techniques, we can directly incorporate multiple objectives and restrictions on the results while offering an efficient numerical solution of the problem.

\paragraph{Parameter space partitioning.}
For pMDPs, 
%one is interested in \emph{synthesizing} well-defined parameter valuations that induce satisfaction or violation of the given specifications~\cite{param_sttt}. In particuluar, 
tools like \tool{PRISM}~\cite{KNP11} or \tool{PROPhESY}~\cite{dehnert-et-al-cav-2015} aim at partitioning the parameter space into regions with respect to a specification.
A \emph{parameter region} is given by a convex polytope defined by linear inequalities over the parameters, restricting valuations to a region. Now, for pMDP $\mdp$ a region is \emph{safe} regarding a specification $\varphi$, if no valuation $u$ inside this region and no scheduler $\sched$ induce $\mdp^\sched[u]\not\models\varphi$. Vice versa, a region is unsafe, if there is no valuation and scheduler such that the specification is satisfied. 
In~\cite{dehnert-et-al-cav-2015}, this certification is performed using SMT solving.
More efficiency is achieved by using an approximation method~\cite{quatmann-et-al-atva-2016}.

Certifying regions to be unsafe is directly possible using our approach.
%Our NLP~\eqref{eq:nl_obj}--\eqref{eq:rewcomputation} aims at finding one scheduler and one valuation that induce optimality and satisfaction of multiple specifications, which are upper bounded reachability or cost properties.
Assume pMDP $\mdp$, specifications $\varphi_1,\ldots,\varphi_q$, and a region candidate defined by a set of linear inequalities.
%The region is \emph{unsafe} iff there is no scheduler and valuation that satisfy the specifications. 
We incorporate the inequalities in the NLP~\eqref{eq:min_rand}--\eqref{eq:rewcomputation}. If the feasibility problem (Section~\ref{sec:geometric}) has no solution, the region is unsafe. This yields the \emph{first efficient numerical method} for this problem of which we are aware.
Proving that a region is safe is more involved. Given one specification $\varphi=\reachProplT$, we maximize the probability to reach $T$. If this probability is at most $\lambda$, the region is safe. For using our method from Section~\ref{sec:approximation}, one needs domain specific knowledge to show that a local optimum is a global optimum.
%
%For computing that \emph{global optimum} by the method described in Section~\ref{sec:approximation}, one needs domain specific knowledge to show that a local optimum is a global optimum, which cannot always be assumed. 


%Formally, let a \emph{half-space} for parameters $V=\{p_1, \ldots, p_n\}$ be given by the linear inequality $a_1 p_1 + \ldots + a_n p_n \leq b$ with $a_1, \ldots, a_n, b \in \Q$. A \emph{region} is a \emph{convex polytope} defined by $m$ half-spaces, \ie, a system of linear inequalities $A\mathbf{p} \leq b$ with $A \in \Q^{m \times n}$, $\mathbf{p} = (p_1 \ldots p_n)^{T} \in V^{n \times 1}$ and $b \in \Q^{m \times 1}$. Assume a rational function $f^r\in\ratfunc[V]$, $f^c\in\ratfunc[V]$, or $f^e\in\ratfunc[V]$ according to Definition~\ref{def:param_model_check} to be computed for a pMC $\pdtmc$ as explained in the previous section.
