\section{Convexification}\label{sec:geometric}
%
We investigate how to transform the SGP~\eqref{eq:min_rand}--\eqref{eq:rewcomputation} into a convex program by relaxing equality constraints and a lifting of variables of the SGP. 
%We first describe a special convex form of an SGP and then show how to convexify the SGP to obtain a feasible solution to the original problem.
%\subsection{Geometric programming}
A certain subclass of SGPs called \emph{geometric programs} (GPs) can be transformed into convex programs~\cite[\S{}2.5]{boyd2007tutorial} and solved efficiently.
A GP is an SGP of the form~\eqref{eq:nl_obj}--\eqref{eq:nl_eq} where $f,g_i\in \posys[\Var]$ and $h_j\in\monos[\Var]$.
We will refer to a constraint with posynomial or monomial function as a posynomial or monomial constraint, respectively.
%
%Basically, the objective and inequality constraints are posynomial while the equality constraints are monomials. Note there is an implicit constraint that all variables are positive. 
%
%Let $\Var=\{x_1,\ldots,x_n\}$ be a set of real-valued \emph{positive} variables. 
%A GP is of the form
%\begin{align}
%	\text{minimize} 		&\quad \posy_0\label{eq:geo_objective}\\
%	\text{subject to} 		&\notag\\
%	\forall i.\, 1\leq i\leq m 	&\quad \posy_i\leq 1\label{eq:geo_posy}\\
%	\forall i.\, 1\leq i\leq p 	&\quad \mono_i= 1\label{eq:geo_mono}
%\end{align}
%where $\mono_i$ for $1\leq i\leq m$ are \emph{monomials} over $\Var$ and $\posy_i$ for $0\leq i\leq p$ a \emph{posynomials} over $\Var$.
% All posynomials are upper bounded by one, all monomials are equal to one, and the objective function is a posynomial. Note that the objective can also be maximized in case it is a monomial. 
% \nj{Say something about the log transformation and how GPs can be solved efficiently by which solver.}
%
\subsection{Transformation and relaxation of equality constraints}
As discussed before, the SGP~\eqref{eq:min_rand}--\eqref{eq:rewcomputation} is not convex because of the presence of non-affine equality constraints. 
First observe the following transformation~\cite{boyd2007tutorial}:
\begin{align}
	f\leq h\Longleftrightarrow \frac{f}{h}\leq 1,\label{eq:geo_trans}
\end{align}
for $f\in\posys[\Var]$ and $h\in\monos [\Var]$. Note that monomials are strictly positive (Def.~\ref{def:posy}). This \emph{(division-)transformation} of $f\leq h$ yields a \emph{posynomial inequality constraint}.

We \emph{relax} all equality constraints of SGP~\eqref{eq:min_rand}--\eqref{eq:rewcomputation} that are not monomials to inequalities, then we apply the division-transformation wherever possible. Constraints \eqref{eq:lambda}, \eqref{eq:kappa}, \eqref{eq:well-defined_sched_rand}, \eqref{eq:well-defined_probs_rand}, \eqref{eq:probcomputation_rand}, and \eqref{eq:rewcomputation} are transformed to
\begin{align}
%			\text{minimize } &\quad f\label{eq:min_rand}\\
%			\text{subject to}\notag \\
							 &\quad \frac{p_{\sinit}}{\lambda}\leq 1\label{eq:lambda_geo},\\
							 &\quad \frac{c_{\sinit}}{\kappa}\leq 1\label{eq:kappa_geo},\\
			\forall s\in S.	&\quad \sum_{\act\in\Act(s)}\sched^{s,\act}\leq 1\label{eq:welldefined_sched_rand_geo},\\			
%			\forall s\in T.	 &\quad p_s=1\label{eq:targetprob_rand}\\
			\forall s\in S\, \forall\act\in\Act(s).	 &\quad \sum_{s'\in S}\probmdp(s,\act,s')\leq 1\label{eq:well-defined_probs_rand_geo},\\
			\forall s\in S\setminus T.	&\quad \frac{\sum\limits_{\act\in\Act(s)}\sigma^{s,\act}\cdot\sum\limits_{s'\in S}	\probmdp(s,\act,s')\cdot p_{s'}}{p_s}\leq 1\label{eq:probcomputation_rand_geo},\\
%			\forall s\in G.	 &\quad r_s=0\label{eq:targetrew}\\
			\forall s\in S\setminus G.	&\quad \frac{\sum\limits_{\act\in\Act(s)} \sigma^{s,\act} \cdot \Bigl(c(s,\act) + \sum\limits_{s'\in S}	\probmdp (s,\act,s') \cdot c_{s'}\Bigr)}{c_s}\leq 1 \label{eq:rewcomputation_geo}.
		\end{align}
%
These constraints are not necessarily posynomial inequality constraints because (as in Def.~\ref{def:pmdp}) we allow signomial expressions in the transition probability function $\pmdp$. Therefore, replacing \eqref{eq:lambda}, \eqref{eq:kappa}, \eqref{eq:well-defined_sched_rand}, \eqref{eq:well-defined_probs_rand}, \eqref{eq:probcomputation_rand}, and \eqref{eq:rewcomputation} in the SGP with~\eqref{eq:lambda_geo}--\eqref{eq:rewcomputation_geo} does not by itself convert the SGP to a GP.

\subsection{Convexification by lifting}\label{sec:lifting}
%
\begin{figure}[t]\centering
	\subfigure[signomial transition functions]{
		\scalebox{0.7}{\input{pics/pmc_signomials}}
	\label{fig:pmc_signomials}
	}\hspace{1cm}
	\subfigure[posynomial transition functions]{
		\scalebox{0.7}{\begin{tikzpicture}[scale=1, die/.style={inner sep=0,outer sep=0},nodestyle/.style={draw,circle},baseline=(s0)]
%    \draw[white, use as bounding box] (-1.7,-4.5) rectangle (1.7,0.3); 
    \node [die,state] (s0) at (0,0) {\large$s$};    
    \node [die] (s1) [state, on grid, left=3cm of s0] {\large$s_1$};
    \node [die] (s2) [state, on grid, above left=3cm of s0] {\large$s_2$};
    \node [die] (s3) [state, on grid, above right=3cm of s0] {\large$s_{n-1}$};
    \node [die] (s4) [state, on grid, right=3cm of s0] {\large$s_n$};
    \draw (s0) edge[->] node[auto] {\large$f_1$} (s1);
    \draw (s0) edge[->] node[auto] {\large$f_2$} (s2);
    \draw (s0) edge[->] node[right] {\large$f_{n-1}$} (s3);
    \draw (s0) edge[->] node[below] {\large$\bar f$} (s4);
%    \node [] (s1dots) [on grid, above=.3cm of s1] {$\cdots$};
%    \node [] (s2dots) [on grid, right=.4cm of s2] {$\ldots$};
%    \node [] (s3dots) [on grid, below=.3cm of s3] {$\cdots$};
%    \node [] (s4dots) [on grid, left=.4cm of s4] {$\ldots$};
%    \node [] (transitiondots) [on grid, above=1cm of s0] {\large$\ldots$};
	\node [] (dummy1) [on grid, right=0.7cm of s2] {};
	\node [] (dummy2) [on grid, right=0.7cm of dummy1] {};
	\node [] (dummy3) [on grid, right=0.7cm of dummy2] {};
	\node [] (dummy4) [on grid, right=0.7cm of dummy3] {};
	\node [] (dummy5) [on grid, right=0.7cm of dummy4] {};
    \node [] (transitiondots) [on grid, above=.1cm of dummy3] {\large$\ldots$};
    \draw (s0) edge[->] (dummy1);
	\draw (s0) edge[->] (dummy2);	
	\draw (s0) edge[->] (dummy3);
	\draw (s0) edge[->] (dummy4);
	\draw (s0) edge[->] (dummy5);
%    \draw [loosely dotted, very thick] (s2) -- (s3);
    \node[draw=white, rectangle, fit=(current bounding box), inner sep=3pt, outer sep=0pt] {};    
\end{tikzpicture}
}
	\label{fig:pmc_monomials}
	}
%	\subfigure[Knuth--Yao die with monomial transition probability function]{
%		\scalebox{0.7}{\vspace{-5ex}%\begin{tikzpicture}[scale=1, die/.style={inner sep=0,outer sep=0},nodestyle/.style={draw,circle},baseline=(s0)]
\begin{tikzpicture}[scale=1]
%    \draw[white, use as bounding box] (-1.7,-4.5) rectangle (1.7,0.3); 
	\tikzstyle{nodestyle}=[draw,circle]
	\tikzstyle{die}=[inner sep=0,outer sep=0]
    
    \node [nodestyle] (s0) at (0,0) {$s_0$};
    \node [] (leftdummy)  [on grid, left=1.2cm of s0] {};
    \node [] (rightdummy) [on grid, right=1.2cm of s0] {};
    \node [nodestyle] (s1) [on grid, below=1.1cm of leftdummy] {$s_1$};
    \node [nodestyle] (s2) [on grid, below=1.1cm of rightdummy] {$s_2$};
    \node [nodestyle] (s3) [on grid, below=1.3cm of s1, xshift=-0.5cm] {$s_3$};
        \node [nodestyle] (s4) [on grid, below=1.3cm of s1, xshift=0.5cm] {$s_4$};
    \node [nodestyle] (s5) [on grid, below=1.3cm of s2, xshift=-0.5cm] {$s_5$};
    
    \node [nodestyle] (s6) [on grid, below=1.3cm of s2, xshift=0.5cm] {$s_6$};
    
    
    \node[die, scale=1.5, below=0.6cm of s3, xshift=-0.3cm] (X1) {\drawdie{1}};
    \node[die, scale=1.5, right=0.21cm of X1, inner sep=0pt] (X2) {\drawdie{2}};
    \node[die, scale=1.5, right=0.21cm of X2] (X3) {\drawdie{3}};
    \node[die, scale=1.5, right=0.21cm of X3] (X4) {\drawdie{4}};
    \node[die, scale=1.5, right=0.21cm of X4] (X5) {\drawdie{5}};
    \node[die, scale=1.5, right=0.21cm of X5] (X6) {\drawdie{6}};
    
    \draw ($(s0)-(0.7,0)$) edge[->] (s0);
    \draw (s0) edge[->] node[right] {\scriptsize$p$} (s1);
    \draw (s0) edge[->] node[right] {\scriptsize$\bar p$} (s2);
    \draw (s1) edge[bend left, ->] node[left] {\scriptsize$q$} (s3);
    \draw (s1) edge[->] node[right] {\scriptsize$\bar q$} (s4);
    \draw (s3) edge[bend left, ->] node[left] {\scriptsize$p$} (s1);
    \draw (s3) edge[->] node[left] {\scriptsize$\bar p$} (X1);
    \draw (s4) edge[->] node[left] {\scriptsize$\bar p$} (X2);
    \draw (s4) edge[->] node[right] {\scriptsize$p$} (X3);
    \draw (s2) edge[bend left, ->] node[left] {\scriptsize$q$} (s5);
    \draw (s2) edge[->] node[right] {\scriptsize$\bar q$} (s6);
    \draw (s5) edge[bend left, ->] node[left] {\scriptsize$p$} (s2);
    \draw (s5) edge[->] node[left] {\scriptsize$\bar p$} (X4);
    \draw (s6) edge[->] node[left] {\scriptsize$\bar p$} (X5);
    \draw (s6) edge[->] node[right] {\scriptsize$p$} (X6);
    
    \node[draw=white, rectangle, fit=(current bounding box), inner sep=3pt, outer sep=0pt] {};
    
\end{tikzpicture}
}
%	\label{fig:die_mon}
\caption{Lifting of signomial transition probability function.}
\end{figure}

%
The relaxed equality constraints~\eqref{eq:well-defined_probs_rand_geo}--\eqref{eq:rewcomputation_geo} involving $\probmdp$ are signomial, rather than posynomial, because the parameters enter Problem~\ref{prob:pmdpsyn} in signomial form. Specifically, consider the relaxed equality constraint~\eqref{eq:probcomputation_rand_geo} at $s_0$ in Example~\ref{ex:die},
\begin{align}
	\label{eq:kys0_relaxed}
	\frac{p\cdot p_{s_1} + (1-p)\cdot p_{s_2}}{p_{s_0}} \leq 1.
\end{align}
The term $(1-p)\cdot p_{s_2}$ is signomial in $p$ and $p_{s_2}$. We \emph{lift} by introducing a new variable $\bar p = 1-p$, and rewrite~\eqref{eq:kys0_relaxed} as a posynomial inequality constraint and an equality constraint in the lifted variables:
\begin{align}
	\frac{p\cdot p_{s_1} + \bar{p} \cdot p_{s_2}}{p_{s_0}} \leq 1,
	\quad
	\bar p = 1-p.
\end{align}
%
%
We relax the (non-monomial) equality constraint to $p + \bar p \leq 1$. 
%\ip{is this what we do, or do we approximate $p + \bar p = 1$ as a monomial equality constraint?}
More generally, we restrict the way parameters occur in $\probmdp$ as follows. Refer to Fig.~\ref{fig:pmc_signomials}. For every state $s\in S$ and every action $\act\in\Act(s)$ we require that there exists at most one state $\bar s\in S$ such that $\probmdp(s,\act,\bar s)\in\signomis[\Paramvar]$ and $\probmdp(s,\act,s')\in\posys[\Paramvar]$ for all $s'\in S\setminus\{\bar s\}$. In particular, we require that
	\begin{align*}
		\underbrace{\probmdp(s,\act,\bar s)}_{\in\signomis[\Paramvar]} = 1 - \sum_{s'\in S\setminus\{\bar s\}} \underbrace{\probmdp({s,\act,s'})}_{\in\posys[\Paramvar]}\ .
	\end{align*}
This requirement is met by all benchmarks available at the \tool{PARAM}--webpage~\cite{param_website}.
In general, we lift by introducing a new variable $\bar p_{s,\act,\bar s}=\probmdp(s,\act,\bar s)$ for each such state $s\in S$; refer to Fig.~\ref{fig:pmc_monomials}.
We denote this set of \emph{lifting variables} by $L$. Lifting as explained above then creates a new transition probability function $\bar \probmdp$ where for every $s,s'\in S$ and $\act\in \Act$ we have $\bar\probmdp(s,\act,s')\in\posys[\Paramvar\cup L]$. 

We call the set of constraints obtained through transformation, relaxation, and lifting of every constraint of the SGP~\eqref{eq:lambda}--\eqref{eq:rewcomputation} as shown above the \emph{convexified constraints}. 
Any posynomial objective subject to the convexified constraints forms by construction a GP over the pMDP parameters $\Paramvar$, the SGP additional variables $W$, and the lifting variables $L$. 

\subsection{Tightening the constraints}
A solution of the GP as obtained in the previous section does not have a direct relation to the original SGP~\eqref{eq:min_rand}--\eqref{eq:rewcomputation}.
In particular, a solution to the GP may not have the relaxed constraints satisfied with equality. 
For \eqref{eq:welldefined_sched_rand_geo} and~\eqref{eq:well-defined_probs_rand_geo}, the induced parameter valuation and the scheduler are not well-defined, \ie, the probabilities may not sum to one. \sj{In particular, this makes it trivial to fulfill the GP, just assign 0 to all schedulers}
We need to relate the relaxed and lifted GP to Problem~\ref{prob:pmdpsyn}. By defining a \emph{regularization function} $F$ over all parameter and scheduler variables, we ensure that the constraints are satisfied with equality; enforcing well-defined probability distributions.
%\begin{align*}
%	F = \sum_{p\in \Paramvar} \frac{1}{p} + \sum_{\bar p\in L} \frac{1}{\bar p} + \sum_{s\in S,\act\in\Act} \frac{1}{\sigma_{s,\act}}
%\end{align*}
%  that is used to enforce well-defined probability distributions:
% : the pMDP parameters, the lifting variables, and the scheduler variables:. The function is given by:
%
%Let $M$ denote the set of variables that are used to enforce well-defined probability distributions. Specifically, $M$ consists of the scheduler and transition probability variables,
%\begin{itemize}
%	\item $\{ \sched^{s,\act} \mid s \in S \}$
%	\item $\{ \probmdp(s,\act,s') \mid s \in S, \act\in\Act \}$.
%\end{itemize}
%Define a regularization function
\begin{align}
	F = \sum_{p\in \Paramvar} \frac{1}{p} + \sum_{\bar p\in L} \frac{1}{\bar p} + \sum_{s\in S,\act\in\Act(s)} \frac{1}{\sigma_{s,\act}}\label{eq:regularization} \ .
\end{align}
The function $F$ is monotone in all its variables. We discard the original objective $f$ in~\eqref{eq:min_rand} and form a GP with the regularization objective $F$~\eqref{eq:regularization}:
\begin{align}
			\text{minimize } &\quad F\label{eq:min_rand_geofull}\\
			\text{subject to}\notag \\
							 &\quad \frac{p_{\sinit}}{\lambda}\leq 1\label{eq:lambda_geofull},\\
							 &\quad \frac{c_{\sinit}}{\kappa}\leq 1\label{eq:kappa_geofull},\\
			\forall s\in S.	&\quad \sum_{\act\in\Act(s)}\sched^{s,\act}\leq 1\label{eq:welldefined_sched_rand_geofull},\\
			\forall s\in S\, \forall\act\in\Act(s). &\quad \sched^{s,\act} \leq 1			\label{eq:sched_is_dist_geofull},\\					%			\forall s\in T.	 &\quad p_s=1\label{eq:targetprob_rand}\\
			\forall s\in S\, \forall\act\in\Act(s).	 &\quad \sum_{s'\in S}\bar\probmdp(s,\act,s')\leq 1\label{eq:well-defined_probs_rand_geofull},\\
				\forall s, s'\in S\, \forall\act\in\Act(s).	 &\quad\bar{\probmdp}(s,\act,s') \leq 1,	\label{eq:probs_is_prob_geofull}\\
	\forall s\in T.	&\quad p_s=1,															\label{eq:targetprob_rand_geofull}\\
			\forall s\in S\setminus T.	&\quad \frac{\sum\limits_{\act\in\Act(s)}\sigma^{s,\act}\cdot\sum\limits_{s'\in S}	\bar\probmdp(s,\act,s')\cdot p_{s'}}{p_s}\leq 1\label{eq:probcomputation_rand_geofull},\\
%			\forall s\in G.	 &\quad r_s=0\label{eq:targetrew}\\
			\forall s\in S\setminus G.	&\quad \frac{\sum\limits_{\act\in\Act(s)} \sigma^{s,\act} \cdot \Bigl(c(s,\act) + \sum\limits_{s'\in S}	\bar\probmdp (s,\act,s') \cdot c_{s'}\Bigr)}{c_s}\leq 1 \label{eq:rewcomputation_geofull}.
\end{align}
%
%We make a few remarks about the GP~\eqref{eq:min_rand_geofull}--\eqref{eq:rewcomputation_geofull}. 
Since the objective $F$~\eqref{eq:regularization} and the inequality constraints~\eqref{eq:welldefined_sched_rand_geofull} and~\eqref{eq:well-defined_probs_rand_geofull} are monotone in $\Paramvar$, $L$, and the scheduler variables, each optimal solution for a feasible problem satisfies them with equality. We obtain a well-defined scheduler $\sched$ and a valuation $u$ as in Problem 1. Note that variables from~\eqref{eq:targetrew} are explicitly excluded from the GP by treating them as constants.

The reachability probability constraints~\eqref{eq:probcomputation_rand_geofull} and cost constraints \eqref{eq:rewcomputation_geofull} need not be satisfied with equality. However, \eqref{eq:probcomputation_rand_geofull} is equivalent to 
\begin{align*}
	{p_s} \geq {\sum\limits_{\act\in\Act(s)}\sigma^{s,\act}\cdot\sum\limits_{s'\in S} \bar\probmdp(s,\act,s')\cdot p_{s'}}
\end{align*}
for all $s\in S\setminus T$ and $\act\in \Act$.
The probability variables $p_s$ are assigned upper bounds on the actual probability to reach the target states $T$ under scheduler $\sched$ and valuation $u$. Put differently, the $p_s$ variables cannot be assigned values that are lower than the actual probability; ensuring that $\sched$ and $u$ induce satisfaction of the specification given by~\eqref{eq:lambda_geofull} if the problem is feasible and $\sched$ and $u$ are well-defined. An analogous reasoning applies to the expected cost computation~\eqref{eq:rewcomputation_geofull}. 
A solution consisting of a scheduler or valuation that are not well-defined occurs only if Problem~\ref{prob:pmdpsyn} itself is infeasible. Identifying that such a solution has been obtained is easy.
%Only if Problem~\ref{prob:pmdpsyn} itself is infeasible, a solution inducing not well-defined scheduler or valuation might be computed. This can easily be checked.
These facts allow us to state the main result of this section.
\begin{theorem}
	\label{thm:main}
	A solution of the GP~\eqref{eq:min_rand_geofull}--\eqref{eq:rewcomputation_geofull} inducing well-defined scheduler $\sched$ and valuation $u$ is a feasible solution to Problem 1.
\end{theorem}
Note that the actual probabilities induced by $\sched$ and $u$ for the given pMDP $\mdp$ are given by the MC $\mdp^\sched[u]$ induced by $\sched$ and instantiated by $u$.
Since all variables are implicitly positive in a GP, no transition probability function will be instantiated to probability zero. 
%Thereby, the topology of the pMDP is always preserved. 
The case of a scheduler variable being zero to induce the optimum can be excluded by a previous graph analysis.
%






