\section{Experiments}\label{sec:experiments}
%
We implemented a prototype using the \tool{Python} interfaces of the probabilistic model checker \tool{STORM}~\cite{DBLP:journals/corr/DehnertJK016} and the optimization solver \tool{MOSEK}~\cite{mosek}.
 All experiments were run on a 2.6 GHz machine with 32 GB RAM. 
 We used \tool{PRISM}~\cite{KNP11} to correct approximation errors as explained before. 
%
We evaluated our approaches using mainly examples from the \tool{PARAM}--webpage~\cite{param_website} and from \tool{PRISM}~\cite{KNP12b}.
We considered several parametric instances of the \emph{Bounded Retransmission Protocol} (BRP)~\cite{HSV94},  \emph{NAND Multiplexing}~\cite{HJ02}, and the \emph{Consensus} protocol (CONS)~\cite{consensus}. For BRP, we have a pMC and a pMDP version, NAND is a pMC, and CONS is a pMDP.
For obtaining feasibility solutions, we compare to the SMT solver \tool{Z3}~\cite{demoura_nlsat}. For additional optimality criteria, there is no comparison to another tool possible as \tool{IPOPT}~\cite{ipopt} already fails for the smallest instances we consider.

Fig.~\ref{tab:scalability} states for each benchmark instance the number of states (\#{}states) and the number of parameters (\#{}par). We defined two specifications consisting of a expected cost property ($\er$) and a reachability property ($\p$). For some benchmarks, we also maximized the probability to reach a set of ``good states'' ($*$). We list the times taken by \tool{MOSEK}; for optimality problems we also list the times \tool{PRISM} took to compute precise probabilities or costs (Section~\ref{sec:approximation}). For feasibility problems we list the times of \tool{Z3}. The timeout (\emph{TO}) is $90$ minutes.
%
%
%\begin{minipage}{0.7\textwidth}
%\begin{table*}[t]
%\centering
%\caption{Experimental results.}
%\label{tab:scalability}
\begin{figure}[t]
	\centering
	\subfigure[Benchmark results]{\scalebox{0.75}{\begin{center}
\centering
\begin{table*}[ht]
\caption{Average rate in bpov of proposed method and percentage gains compared with MPEG G-PCC (negative percentages mean bitrate reduction).}
\resizebox{0.99\linewidth}{!}{ \begin{tabular}{|P{0.80cm}|l|R{0.9cm}||R{0.9cm}|R{1.2cm}|R{0.9cm}|R{1.2cm}|R{0.9cm}|R{1.2cm}|R{0.9cm}|R{1.2cm}|}
\cline{3-11}
% \multicolumn{2}{|c||}{\begin{bf} Test PC \end{bf}}
\multicolumn{2}{c|}{}
& \begin{bf}G-PCC\end{bf}
& \multicolumn{2}{c|}{\begin{bf} Baseline \end{bf}}
& \multicolumn{2}{c|}{\begin{bf} Baseline + DA \end{bf}}
& \multicolumn{2}{c|}{\begin{bf} Baseline + CE \end{bf}}
& \multicolumn{2}{c|}{\begin{bf} Baseline + DA + CE\end{bf}}
\\
\hline
Dataset&Point Cloud&bpov&bpov&Gain over G-PCC &bpov&Gain over G-PCC &bpov&Gain over G-PCC&bpov&Gain over G-PCC\\
\hline

\multirow{3}{*}{MVUB}&Phil & 1.1617 &0.8205 &-29.37\% &0.8954 & -22.92\%&0.7601 & -34.57\%&  0.8252&-28.97\% \\ 
\cline{2-11}
&Ricardo & 1.0672 & 0.7440&-30.28\% &  0.8235&-22.84\% &0.6874&-35.59\% &  0.7572&-29.05\% \\
\cline{2-11}
&\textbf{Average} &\textbf{1.1145}  &\textbf{0.7823}&\textbf{-29.83\%} &\textbf{0.8595 }&\textbf{-22.88\%} &\textbf{07238. }&\textbf{-35.06\%}& \textbf{0.7912}&\textbf{-29.01\% } \\
\cline{2-11}
\hline



\multirow{5}{*}{8i}&Redandblack &1.0899  & 0.7190&-34.3\% &0.7772 &-28.69\% & 0.6645&-39.03\%& 0.7003&-35.75\% \\
\cline{2-11}
&Loot & 0.9524 &0.6271 & -34.16\%&0.6282 &-34.04\% &0.5766 &-39.46 \% &0.6084 &-36.12\%\\
\cline{2-11}
&Thaidancer &0.9985 &0.7297 &-26.92\% &0.7253 &-27.36\% &0.6769 &-32.21\%&0.6627   &-33.63\%\\
\cline{2-11}
&Boxer&0.9479  & 0.5900&-37.76\% &0.6573 &-30.66\%&0.5503 &-41.95\% &0.5906 &-37.69\% \\
\cline{2-11}
&\textbf{Average} &\textbf{0.9972} &\textbf{0.6665} &\textbf{-33.22\%} &\textbf{0.6870} &\textbf{-30.19\% }&\textbf{0.6171} &\textbf{-38.12\% }& \textbf{ 0.6405}&\textbf{-35.77\% }\\
\hline



\multirow{4}{*}{CAT1}&Frog&1.9085   & 1.8214&-4.56\% &1.7662 &-7.64\% & 1.6971&-11.08\%& 1.7071&-10.55\% \\
\cline{2-11}
&Arco Valentino & 4.8119 & 5.2050&+8.17\% &5.0639 &+5.24\% & 4.9923&+3.75\%& 4,9900&+3.70\% \\
\cline{2-11}
&Shiva&3.6721 & 3.6403&-0.87\% &3.5838 &-2.04\% & 3.4619&-5.72\%& 3.5135&-4.32\% \\
\cline{2-11}
&\textbf{Average} &\textbf{3.4642} &\textbf{3.5556} &\textbf{+0.91\%} &\textbf{3.7413} &\textbf{-1.54\% }&\textbf{3.3838} &\textbf{-2.32\% }& \textbf{3.4035 }&\textbf{-3.72\% }\\
\hline



\multirow{3}{*}{USP}&BumbaMeuBoi & 5.4522   & 5.7305&+5.10\% &5.3831 &-1.27\% & 5.3580&-1.73\%& 5.066&-7.08\% \\
\cline{2-11}
&RomanOiLight & 1.8604  & 1.7030&-8.46\% &1.7319 &-6.91\% & 1.6130&-13.30\%& 1.6231&-12.76\% \\
\cline{2-11}
&\textbf{Average}  &\textbf{3.6563}  &\textbf{3.7168}&\textbf{-1.68\%} &\textbf{3.5575}&\textbf{-4.09\%} &\textbf{3.4855 }&\textbf{-7.51\%}& \textbf{3.4855}&\textbf{-9.91\% } \\
\cline{2-11}
\hline
\hline
\end{tabular}}
\label{table:result table}
\end{table*}
\end{center}

}\label{tab:scalability}}
	\subfigure[Sensitivity to \#par]{\scalebox{0.95}{\pgfplotsset{every axis/.append style={
                    legend style={font=\tiny, at={((0.5,-0.19))}, align=left, anchor=north,draw=none ,mark size=2pt},
                    }}%
\pgfplotsset{footnotesize}
 \begin{tikzpicture}
% \selectcolormodel{gray}
\draw[black, use as bounding box] (-1.2,-1.6) rectangle (2.7,4.2);
 \iftrue
\begin{axis}[width=3.9cm, height=5.5cm, ylabel={Time (s)}, xlabel={Number of parameters},
axis x line=bottom,mark size=0.8pt,
 axis y line=left,
 ymode = log,
 ymin = 0,
 ymax = 100,
 ytick = {0.1,0.2,0.5,1,2,5,10,20,50,100},
 yticklabels = {0.1,0.2,0.5,1,2,5,10,20,50,$\mathit{TO}$},
 xtick={2,3,4,5,6,7,8},
% xticklabels = {0.01, 0.1, 1, 2},
 x label style={font=\scriptsize,at={(axis description cs:0.5,0.05)},anchor=north},
 y label style={font=\scriptsize, at={(axis description cs:0.25,.5)},anchor=south},
% cycle list name=mycolor, 
 legend columns=2,
 legend entries={MOSEK,Z3,PROPhESY},
 legend image post style={scale=0.5},
% legend style={font=\tiny, at={(0,-0.3)}, anchor=center},
% legend pos=outer north east,
% legend to name=named,
legend cell align=left,
 yticklabel style={font=\tiny},
 xticklabel style={font=\tiny}]
]
    \addplot+[mark=*] file {data/die_convex.dat};
    \addplot+[mark=square] file {data/die_smt.dat};
    \addplot+[mark=square*] file {data/die_stateelim.dat};
\end{axis}  
\fi
\end{tikzpicture}


\label{plot:param}}}
%	\vspace{-0.3cm}
\caption{Experiments.}
%\vspace{-0.7cm}
%\label{tab:scalability}
\end{figure}

%\end{table*}
We observe that both for feasibility with optimality criteria we can handle most benchmarks of up to $10^5$ states within the timeout, while we ran into a timeout for CONS. The number of iterations $N$ in the sequential convex programming is less than $12$ for all benchmarks with $\epsilon=10^{-3}$.
As expected, simply solving feasibility problems is faster by at least one order of magnitude. Raising the number of parameters from $2$ to $4$ for BRP does not cause a major performance hit, contrary to existing tools. For all benchmarks except NAND, \tool{Z3} only delivered results for the smallest instances within the timeout. 
%As mentioned before, this is expected due to the complexity of SMT solving.

To demonstrate the insensitivity of our approach to the number of parameters, we considered a pMC of rolling multiple  Knuth--Yao dice with $156$ states, $522$ transitions and considered instances with up to $8$ different parameters. The timeout is $100$ seconds.
In Fig.~\ref{plot:param} we compare our encoding in \tool{MOSEK} for this benchmark to the mere computation of a rational function using \tool{PROPhESY}~\cite{dehnert-et-al-cav-2015} and again to \tool{Z3}. \tool{PROPhESY} already runs into a timeout for $4$ parameters\footnote{Due to the costly computation of greatest common divisors employed in \prophesy.}.
%; this is due to the fact that \tool{PROPhESY} uses state elimination which involves the costly computation of the greatest common divisor of polynomials. 
\tool{Z3} needs around $15$ seconds for most of the tests. Using GPs with \tool{MOSEK} proves far more efficient  as it needs less than one second for all instances.

In addition, we test model repair (Section~\ref{sec:applications}) on a BRP instance with $17415$ states for $\varphi=\reachProp{0.9}{T}$. The initial parameter instantiation violates $\varphi$. We performed model repair towards satisfaction of $\varphi$. The probability of reaching $T$ results in $0.79$ and the associated cost is $0.013$. The computation time is $21.93$ seconds. We compare our result to an implementation of~\cite{chen2013model}, where the probability of reaching $T$ is $0.58$ and the associated cost is $0.064$. However, the time for the simulation-based method is only $2.4$ seconds, highlighting the expected trade-off between optimality and computation times for the two methods.
	
Finally, we encode model repair for the small pMC from Example~\ref{ex:die} in \tool{IPOPT}, see~\cite{bartocci2011model}. For $\psi=\reachProp{0.125}{T}$ where $T$ represents the outcome of the die being $2$, the initial instantiation induces probability $1/6$. With our method, the probability of satisfying $\psi$ is $0.1248$ and the cost is $0.0128$. With \tool{IPOPT}, the probability is $0.125$ with cost $0.1025$, showing that our result is nearly optimal.
%	For the instance with two die, the reachability specification is given as $\reachProplT$ where $\lambda =0.01$ and The initial parameter instantiation satisfies the reachability property with probability $1/36.$ With our method, the propability of satisfying the specification is $0.0022$ and the associated cost is 0.0134. With \tool{IPOPT}, the propability of satisfying the specification is $0.01$ and the associated cost is 0.0136




%\begin{itemize}
%%\item for Knuth Yao Die, report in \tool{IPOPT} results in comparison
%%		\item We also compared our approach to bertocci for Knuth Yao Die in two instances. 
%%	For the instance with one die, the reachability specification is given as $\reachProplT$ where $\lambda =0.125$ and The initial parameter instantiation satisfies the reachability property with probability $1/6.$ With our method, the propability of satisfying the specification is $0.1248$ and the associated cost is 0.0128. With \tool{IPOPT}, the propability of satisfying the specification is $0.125$ and the associated cost is 0.1025.
%%	For the instance with two die, the reachability specification is given as $\reachProplT$ where $\lambda =0.01$ and The initial parameter instantiation satisfies the reachability property with probability $1/36.$ With our method, the propability of satisfying the specification is $0.0022$ and the associated cost is 0.0134. With \tool{IPOPT}, the propability of satisfying the specification is $0.01$ and the associated cost is 0.0136.
%	\item for BRP with 4 parameters, list running times of \prophesy to just generate the rational function in comparison to solve the whole feasibility problem with our approach
%	\item mention that no tool can directly handle multi-objective properties.
%	\item discuss that if this approach would be part of a mature tool like \tool{PRISM} or \tool{PROPhESY}, eg bisimulation minimization would enable way larger benchmarks.
%	\item add benchmarks with more parameters
%\end{itemize}
