\section{Introduction}
\label{sec:introduction}
%
We study the applicability of \emph{convex optimization} to the formal verification of systems that exhibit randomness or stochastic uncertainties. Such systems are formally represented by so-called parametric Markov models. 

In fact, many real-world systems exhibit random behavior and stochastic uncertainties. One major example is in the field of \emph{robotics}, where the presence of measurement noise or input disturbances requires special controller synthesis techniques~\cite{thrun2005probabilistic} that achieve robustness of robot actions against uncertainties in the robot model and the environment.
On the other hand, formal verification offers methods for rigorously proving or disproving properties about the system behavior, and synthesizing strategies that satisfy these properties.
In particular, \emph{model checking}~\cite{BK08} is a well-studied technique that provides guarantees on appropriate behavior for all possible events and scenarios.
%Other applications of model checking include randomized distributed algorithms, security, systems biology, and embedded systems.

Model checking can be applied to systems with stochastic uncertainties, including discrete-time Markov chains (MCs), Markov decision processes (MDPs), and their continuous-time counterparts~\cite{katoen2016probabilistic}.
Probabilistic model checkers are able to verify reachability properties like ``the probability of reaching a set of unsafe states is $\leq 10\%$'' and expected costs properties like ``the expected cost of reaching a goal state is $\leq 20$.''
A rich set of properties, specified by linear- and branching-time logics, reduces to such properties~\cite{katoen2016probabilistic}.
Tools like \tool{PRISM}~\cite{KNP11}, \tool{STORM}~\cite{DBLP:journals/corr/DehnertJK016}, and 
\tool{iscasMc}~\cite{iscasmc} are probabilistic model checkers capable of handling a wide range of large-scale problems.

Key requirements for applying model checking are a reliable system model and formal specifications of desired or undesired behaviors. 
As a result, most approaches assume that models of the stochastic uncertainties are precisely given. For example, if a system description includes an environmental disturbance, the mean of that disturbance should be known \emph{before} formal  statements are made about expected system behavior.
However, the desire to treat many applications where uncertainty measures (\eg, faultiness, reliability, reaction rates, packet loss ratio) are not exactly known at design time 
% due to possible changes in the real-world or at early design stages.
%The task of robustifying systems against such variable uncertainties gives rise to \emph{parametric} probabilistic models, where the transition probabilities are expressed as functions over system parameters.
gives rise to \emph{parametric} probabilistic models~\cite{DBLP:journals/ior/SatiaL73,DBLP:journals/ai/DelgadoBDS16}. Here, transition probabilities are expressed as functions over system parameters, \ie, \emph{descriptions of uncertainties}.
In this setting, \emph{parameter synthesis} addresses the problem of computing parameter instantiations leading to satisfaction of system specifications. 
More precisely, parameters are mapped to concrete probabilities inducing the resulting \emph{instantiated} model to satisfy  specifications.
%An even harder task is to find an instantiation that satisfies an optimality criterion, for instance defined on reachability probabilities.
A direct application is \emph{model repair}~\cite{bartocci2011model}, where a concrete model (without parameters) is changed (repaired) such that specifications \emph{are} satisfied.
%
%
%The model is changed (repaired) such that the specifications \emph{are} satisfied. 
%The repair is subject to a cost function, which penalizes deviations from some original instantiation; the underlying model is parametric. 

Dedicated tools like \tool{PARAM}~\cite{PARAM10}, \tool{PRISM}~\cite{KNP11}, or \prophesy~\cite{dehnert-et-al-cav-2015} compute rational functions over parameters that express reachability probabilities or expected costs in a parametric Markov chain (pMC). These optimized tools work with millions of states but are restricted to a few parameters, as the necessary computation of greatest common divisors does not scale well with the number of parameters.
	Moreover, the resulting functions are inherently \emph{nonlinear} and often of high degree. Evaluation by an SMT solver over nonlinear arithmetic such as \tool{Z3}~\cite{demoura_nlsat} suffers from the fact that the solving procedures are \emph{exponential in the degree of polynomials and the number of variables}. 
%	In practice, this poor scalability can be even worse for parametric Markov decision processes (pMDPs) that involve nondeterminism~\cite{DBLP:journals/ior/SatiaL73,DBLP:journals/ai/GivanLD00,DBLP:journals/ai/DelgadoSB11,DBLP:journals/ai/DelgadoBDS16}. A recent approach partly overcomes these problems by computing lower and upper bounds on probabilities by a relaxation of parameter dependencies~\cite{quatmann-et-al-atva-2016}.

This paper takes an alternative perspective. 
%Many problems concerning pMDPs can be formulated as nonlinear programs (NLPs). 
We discuss a general nonlinear programming formulation for the verification of parametric Markov decision processes (pMDPs).
 The powerful modeling capabilities of nonlinear programs (NLPs) enable incorporating multi-objective properties and penalties on the parameters of the pMDP.
However, because of their generality, solving NLPs to find a global optimum is difficult. Even feasible solutions (satisfying the constraints) cannot always be computed efficiently~\cite{bertsekas1999nonlinear,Las01}. 
In contrast, for the class of NLPs called \emph{convex optimization} problems, efficient methods to compute feasible solutions and global optima even for large-scale problems are available~\cite{boyd_convex_optimization}. 

We therefore propose a novel automated method of utilizing convex optimization for pMDPs.
%
%First, we restrict the way in which parameters are allowed to enter the pMDP verification problem to signomials
Many NLP problems for pMDPs belong to the class of \emph{signomial programs} (SGPs), a certain class of nonconvex optimization problems.
For instance, all benchmarks available at the \tool{PARAM}--webpage~\cite{param_website} belong to this class.
%First, we restrict functions over parameters such that the problems we consider are formulated 
%While SGPs do not belong to the class of convex optimization problems, we show that under certain conditions a convex \emph{relaxation} is available. 
%
%
Restricting the general pMDP problem accordingly yields a direct and efficient synthesis method---formulated as an NLP---for a large class of pMDP problems. 
%
 We list the two main technical results of this paper:
\begin{enumerate}
	\item We relax nonconvex constraints in SGPs and apply a simple transformation to the parameter functions. The resulting programs are \emph{geometric programs} (GPs)~\cite{boyd2007tutorial}, a class of \emph{convex programs}. We show that a solution to the relaxed GP induces feasibility (satisfaction of all specifications) in the original pMDP problem. Note that solving GPs is \emph{polynomial} in the number of variables. 
	\smallskip
	\item Given an initial feasible solution, we use a technique called \emph{sequential convex programming}~\cite{boyd2007tutorial} to improve a signomial objective. This local optimization method for nonconvex problems leverages convex optimization by solving a sequence of convex approximations (GPs) of the original SGP. \end{enumerate}
%
Sequential convex programming is known to efficiently find a feasible solution with
good, though not necessarily globally optimal, objective values~\cite{boyd2007tutorial,boyd2008sequential}.
We initialize the sequence with a feasible solution (obtained from the GP) of the original problem and compute a \emph{trust region}. Inside this region, the optimal value of the approximation of the SGP is at least as good as the objective value at the feasible solution of the GP. 
The optimal solution of the approximation is then the initial point of the next iteration with a new trust region.
This procedure is iterated to approximate a local optimum of the original problem. 
%As part of trust region calculations, we make use of the probabilistic model checker \tool{PRISM}~\cite{KNP11} to correct approximation errors. 
%

Utilizing our results, we discuss the concrete problems of parameter synthesis and model repair for multiple specifications for pMDPs.
Experimental results with a prototype implementation show the applicability of our optimization methods to benchmarks of up to $10^5$ states.
%
As solving GPs is polynomial in the number of variables, our approaches are relatively insensitive to the number of parameters in pMDPs. This is an improvement over state-of-the-art approaches that leverage SMT, which---for our class of problems---scale exponentially in variables and the degree of polynomials. 
This is substantiated by our experiments.


\paragraph{Related work.}
Several approaches exist for pMCs~\cite{PARAM10,dehnert-et-al-cav-2015,param_sttt,jansen-et-al-qest-2014} while the number of approaches for pMDPs~\cite{param_sttt,quatmann-et-al-atva-2016} is limited.
Ceska \emph{et al.}~\cite{DBLP:conf/cmsb/CeskaDKP14} synthesize rate parameters in stochastic biochemical networks. 
Multi-objective model checking of non-parametric MDPs~\cite{DBLP:journals/lmcs/EtessamiKVY08} is a convex problem~\cite{DBLP:conf/tacas/ForejtKNPQ11}. 
Bortolussi \emph{et al.}~\cite{DBLP:journals/iandc/BortolussiMS16} developed a Bayesian statistical algorithm for properties on stochastic population models.
%First, heuristic~\cite{pathak-et-al-nfm-2015} and simulation-based~\cite{chen2013model} approaches address the scalability of model repair.
Convex uncertainties in MDPs without parameter dependencies are discussed in~\cite{seshia_et_al_cav_13}.
%In quality-of-service analysis of software, parameters model the quantified estimation errors in log data~\cite{calinescu_ieee_tr_2016}.
Parametric probabilistic models are used to rank patches in the repair of software~\cite{DBLP:conf/popl/LongR16} and to compute perturbation bounds~\cite{rosenblum-et-al-atva-2014,su-et-al-icse-2016-qosevaluation}. 



%Moreover, the size of the obtained functions often limits the practicability as analysing the (potentially large) rational function via SMT solving~\cite{dehnert-et-al-cav-2015} is often not feasible.



%(variants of) the state elimination approach by Daws~\cite{Daw04} to obtain such a function which conceptually allows for many types of analysis.
%

%
%Parametric probabilistic models have various applications as witnessed by several recent works.
%Model repair~\cite{bartocci2011model} exploits parametric Markov chains (MCs) to tune the parameters of the model.
%In quality-of-service analysis of software, parameters are used to model the unquantified estimation errors in log data~\cite{calinescu_ieee_tr_2016}.
%Ceska \emph{et al.}~\cite{DBLP:conf/cmsb/CeskaDKP14} consider the problem of synthesising rate parameters in stochastic biochemical networks.
%Parametric probabilistic models are also used to rank patches in the repair of software~\cite{DBLP:conf/popl/LongR16} and for computing perturbation bounds~\cite{rosenblum-et-al-atva-2014,su-et-al-icse-2016-qosevaluation}. 
%The main problem though is that current parametric probabilistic model checking algorithms cannot cope with the complexity of these applications.
%Their scalability is restricted to a couple of thousands of states and a few (preferably independent) parameters, and models with nondeterminism are out of reach.
%(The only existing algorithm~\cite{DBLP:conf/nfm/HahnHZ11} for parametric MDPs uses an unsound heuristic in its implementation to improve scalability.)
%
%We present an algorithm that overcomes all these limitations: It is scalable to millions of states, several (dependent) parameters, and---perhaps most importantly---provides the first sound and feasible technique to do parameter synthesis of parametric MDPs.
%
%The key technique used so far is computing a rational function (in terms of the parameters) expressing the reachability probability in a parametric MC.
%Tools like \tool{PARAM}~\cite{PARAM10}, \tool{PRISM}~\cite{KNP11}, and \prophesy~\cite{dehnert-et-al-cav-2015} exploit (variants of) the state elimination approach by Daws~\cite{Daw04} to obtain such a function which conceptually allows for many types of analysis.
%While state elimination is feasible for millions of states~\cite{dehnert-et-al-cav-2015}, it does not scale well in the number of different parameters.
%Moreover, the size of the obtained functions often limits the practicability as analysing the (potentially large) rational function via SMT solving~\cite{dehnert-et-al-cav-2015} is often not feasible.
%The problem of finding a \emph{randomized scheduler} that induces satisfaction of the specifications for an MDP without parameters, referred to as \emph{multi-objective model checking}, was studied in~\cite{DBLP:journals/lmcs/EtessamiKVY08,DBLP:conf/atva/ForejtKP12}. It was shown that randomized schedulers are strictly more powerful than deterministic ones.
%



%%% probabilistic model checking, what else?!
%This has led to the development of different automata- and tableau-based  \emph{probabilistic model checking} techniques to prove model properties specified by, e.g., probabilistic $\omega$-regular languages or probabilistic branching-time logics such as pCTL and pCTL$^*$.
%Probabilistic model checking is applicable to a plethora of probabilistic models, ranging from discrete-time Markov chains to continuous-time Markov decision processes and probabilistic timed automata, possibly extended with notions of resource consumption (such as memory footprint and energy usage) using rewards (or prices).
%\tool{PRISM}~\cite{KNP11}, \tool{MRMC}~\cite{mrmc}, 
%\tool{CADP}~\cite{DBLP:journals/sttt/GaravelLMS13} and 
%\tool{iscasMc}~\cite{iscasmc} are mature probabilistic model checkers and have been applied successfully to a wide range of benchmarks.
%Recently, Alur \emph{et al.}~\cite{alur_siglog} identified probabilistic model checking as a promising new direction as it establishes correctness \emph{and} evaluates performance aspects; see also~\cite{baier-et-al-cacm}.



