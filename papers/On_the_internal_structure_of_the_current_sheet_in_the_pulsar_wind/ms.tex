\documentclass[useAMS,usenatbib]{mn2e}
\usepackage{bm}
\usepackage{amssymb}
\usepackage{comment}
\usepackage{macros}
\voffset-.6in
\hoffset0.2in
\usepackage[usenames,dvipsnames,svgnames,table]{xcolor}
\usepackage{hyperref}
\definecolor{darkblue}{rgb}{0.0,0.0,0.7}
\hypersetup{colorlinks,breaklinks,
            linkcolor=darkblue,urlcolor=darkblue,
            anchorcolor=darkblue,citecolor=darkblue}
\usepackage{graphicx}
\usepackage{amsmath}
\usepackage[outdir=./]{epstopdf}

\newcommand\cmtla[1]{{\color{red}[LA: #1]}}
\newcommand\cmtvp[1]{{\color{blue}[VP: #1]}}
\newcommand\cmtvsb[1]{{\color{OliveGreen}[VSB: #1]}}

\def\etal{et al.\ \rm}
\def\vph{\varphi}
\def\th{\theta}

\def\b{B_{L \rm} R_{L \rm}}
\def\td{\tanh\left(\frac{r-\beta c t'}{\Delta_{\rm lab}}\right)}

\title[On the internal structure of the current sheet in the pulsar wind]{On the internal
structure of the current sheet in the pulsar wind}
\author [V. V. Prokofev, L. I. Arzamasskiy and V. S. Beskin]{V. V. Prokofev$^1$, L. I. Arzamasskiy$^{2}$\thanks{E-mails:
beskin@lpi.ru, leva@astro.princeton.edu},
V. S. Beskin$^{1,3}$\footnotemark[1]\\
$^{1}$Moscow Institute of Physics and Technology, Dolgoprudny, Institutsky per., 9,
Moscow region, 141700, Russia\\
$^{2}$Department of Astrophysical Sciences, Peyton Hall, Princeton University, Princeton, NJ 08544, USA \\
$^{3}$P.N.Lebedev Physical Institute, Leninsky prosp., 53, Moscow,
119991, Russia}

\begin{document}

\date{Accepted. Received; in original form}

\pagerange{\pageref{firstpage}--\pageref{lastpage}} \pubyear{2017}

\maketitle

\label{firstpage}

\begin{abstract}
We investigate { the internal structure of the current sheet in the pulsar
wind within} force-free and two-fluid MHD approximations. Within the 
force-free approximation we obtain { general} asymptotic solution of the
Grad-Shafranov equation for quasi-spherical pulsar wind { up to the second
order in small parameter $\varepsilon = (\Omega r/c)^{-1}$. The solution allows an arbitrary latitudinal structure of the radial magnetic field, including that obtained in the numerical simulations of oblique rotators. 
%In particular, it can describe the latitudinal structure of the radial magnetic field obtained numerically for oblique rotator. 
It is also shown that the shape of the current sheet does not depend on the 
latitudinal structure.} 
For the internal region of the current sheet { outside the fast magnetosonic 
surface} where the force-free approximation is not valid we use two-fluid MHD approximation.  
{ Carrying out calculations in the 
comoving reference frame we succeed in determining intrinsic electric and magnetic fields 
of a sheet. It allows us to analyze time-dependent effects which were not investigated
up to now. In particular, we estimate} the efficiency of the particle 
acceleration inside the sheet. Finally, after investigating the motion of individual 
particles in the time-dependent current sheet, we find self-consistently the width of the 
sheet and its time evolution.
\end{abstract}

\begin{keywords}
Neutron stars -- radio pulsars
\end{keywords}

%%%%%%%%%%%%%%%%%%%%%%%%%%%%%%%%%%%%%%%%%%%%%%%%%%%%%%%%%%%

\section{Introduction}
\label{sect:intro}

Particle acceleration in compact astrophysical objects is the classical
problem of modern astrophysics. Indeed, high energy radiation in the GeV
and even TeV energy band is the smoking gun of the existence of the relativistic
particles with characteristic Lorentz-factors $\gamma$ up to $10^4$--$10^5$ 
\citep{2000PhRvL..85..912L, 2007A&A...464..235A}.

Radio pulsars are thought to be the most effective accelerators in space.
Indeed, fast rotation of a neutron star (radius $R \sim 10$--$15$ km and
periods $P \sim 1$ s for ordinary pulsars) with surface magnetic field
$B_{0} \sim 10^{11}$--$10^{13}$ G (and even  $10^{13}$--$10^{15}$ G for
magnetars) inevitably results in the generation of the large enough electric
field $E \sim (\Omega R_{0}/c) B_{0}$. Here $\Omega = 2 \pi/P$ is the
neutron star angular velocity, and $R_{0}$ is the effective radius of the 
`central engine'; for radio pulsars $R_{0} \approx (\Omega R/c)^{1/2} R$
is the polar cap radius. The energetics of radio pulsars is determined by the
potential drop $V_{\rm tot} \sim e E R_{0}$.

To determine self-consistently the specific realization of the effective
particle acceleration, it is necessary to know in detail the structure of the
neutron star magnetosphere and the pulsar wind. Important analytical results
were already obtained in the first quarter of the century since radio pulsars
were discovered. In particular, the role of the process of the quantum-mechanical
particle creation was clarified~\citep{sturrock71, rs75, 1981ApJ...248.1099A}. 
It was shown the very possibility of the quasi-radial magnetized wind transporting 
electromagnetic energy to infinity~\citep{1973ApJ...180L.133M, 1975MNRAS.170..551B, 
1999A&A...349.1017B}. It was also predicted the full screening of magnetodipole 
radiation by plasma filling the neutron star magnetosphere~\citep{BGI}. Later 
these results were confirmed with numerical simulations of axisymmetric~\citep{ckf99, 
2006MNRAS.368.1055T, 2005PhRvL..94b1101G, 2006MNRAS.367...19K} as well as inclined 
magnetosphere~\citep{2012MNRAS.420.2793K, SashaMHD}.

As a result, at large enough distance from a neutron star $r \gg R_{\rm L}$
($R_{\rm L} = c/\Omega$ is the radius of the light cylinder) the theory predicts
the quasi-radial outflow of the relativistic electron-positron plasma along the
poloidal magnetic field. As the total flux of the magnetic field
through the whole sphere is to vanish, such a structure is to contain the current
sheet separating outgoing and ingoing magnetic fluxes. Up to the light cylinder the
energy is mainly transported by the electromagnetic field (i.e., by the flux of the
Poynting vector ${\bf S} = (c/4\pi){\boldsymbol E} \times {\boldsymbol B}$), but somewhere at larger
distances the electromagnetic energy flux is to be transferred into the particle energy.
Actually, the mechanism of this transformation is the main subject of the particle
acceleration theory. Up to now it remains the open question as the ideal MHD predicts
very ineffective particle acceleration for quasi-radial outflow~\citep{1994PASJ...46..123T, bes98}.

Clearly, for the MHD outflow to exist, the electric field ${\boldsymbol E}$ has to be smaller
 than the magnetic field ${\boldsymbol B}$. In the pulsar wind it can take place only if the
total electric current $I$ flowing in the magnetosphere (and producing toroidal
magnetic field $B_{\varphi}$) is large enough so that at the light cylinder the
toroidal magnetic field $B_{\varphi} \approx 2I/R_{\rm L}c$ becomes as large
as the electric field. Indeed, both $B_{\varphi}$ and $E$ diminish with
the distance as $r^{-1}$ (and, accordingly, $S \propto r^{-2}$). As the poloidal
magnetic field $B_{\rm p}$ for quasi-spherical structure decreases much faster
($B_{\rm p} \propto r^{-2}$), the very MHD approximation $E < B$ can be fulfilled
only if the total electric current  $I$ circulating in the pulsar magnetosphere is
as large as the so-called Goldreich-Julian current
\begin{equation}
I_{\rm GJ} = \pi R_{0}^2 j_{\rm GJ}^{\rm A},
\label{IGJ}
\end{equation}
where
\begin{equation}
j_{\rm GJ}^{\rm A} = \frac{\Omega B}{2 \pi}
\label{jGJA}
\end{equation}
is the amplitude of the Goldreich-Julian current density.

Here it is necessary to stress one important point. The definition of the Goldreich-Julian
charge density
\begin{equation}
\rho_{\rm GJ} = -\frac{\bf{\Omega} {\boldsymbol B}}{2 \pi c}
\label{rhoGJ}
\end{equation}
contains the factor $\cos \theta_{\rm m}$, where $\theta_{\rm m}$
is the angle between the vectors ${\bf \Omega}$ and ${\boldsymbol B}$. As a result,
for the condition (\ref{IGJ}) to be fulfilled for large enough inclination
angles $\chi$ between ${\bf \Omega}$ and the neutron star magnetic moment
${\bf m}$, the current density $j$ is to be larger than the local GJ current
density $j_{\rm GJ} = \rho_{\rm GJ} c$ near magnetic poles where 
$\theta_{\rm m} \sim \chi $, and
\begin{equation}
j_{\rm GJ} \approx \frac{3}{2} \, \frac{\Omega B}{2 \pi} \, \cos\theta.
\label{jjGJ}
\end{equation}

Thus, for quasi-radial MHD outflow to exist, the pair creation mechanism
has to support large enough longitudinal current $j > j_{\rm GJ}$ (\ref{jjGJ}).
At present, the most scientists believe in this model~\citep[see, e.g.,][]{2013MNRAS.429...20T},
and here we follow this point of view as well. On the other hand, effective
particle acceleration can be realized if the particle generation mechanism
cannot support large enough longitudinal current $j$. It can take place
in the vicinity of the so-called light surface $E = B$ which inevitable
appears outside the light cylinder for $j < j_{\rm GJ}$ ~\citep{BGI, 2000MNRAS.313..433B}.
Recently, the possible existence of such effective particle acceleration region was demonstrated by 
analyzing the TeV time-dependent radiation of the Crab pulsar~\citep{2012Natur.482..507A}.

Returning to standard model of the pulsar wind, let us recall that it contains `striped' 
current sheet separating ingoing and outgoing magnetic fluxes. This current sheet was predicted 
analytically~\citep{coroniti_striped_wind_1990, michel_pulsar_wind_1994, 1999A&A...349.1017B}
and later was confirmed in numerous force-free~\citep{ckf99, 2006ApJ...648L..51S, 2012MNRAS.420.2793K}, 
MHD~\citep{2006MNRAS.367...19K, 2009ApJ...699.1789T} and even PIC~\citep{2015MNRAS.448..606C}  
simulations\footnote{Nowhere any restrictions on the longitudinal current were imposed.}.
And it has long been understood that this current sheet can be the domain of very effective particle
acceleration~\citep{2001ApJ...547..437L,2007ApJ...670..702Z,2012SSRv..173..341A,2014ApJ...781...46C} 
On the other hand, self-consistent model of the internal structure of this 'striped' current sheet
was not constructed. 

The paper is organized as follows. We start with the discussion of force-free asymptotic 
behavior of the pulsar wind in Sect.~\ref{sect:rest}.  Here we obtain simple asymptotic 
solution of the Grad-Shafranov equation for quasi-spherical pulsar wind. { In 
particular, this solution can describe the latitudinal structure of the radial magnetic field obtained 
numerically for oblique rotator~\citep{2016MNRAS.457.3384T}}. We also show that the shape 
of the current sheet does not depend on the latitudinal structure. Then in Sect.~\ref{sect:fields} 
we determine the main properties of the internal regions of a current sheet where the force-free
approximation is not valid.  Using two-fluid MHD approximation and carrying out calculations 
in the comoving reference frame we determine electric and magnetic field structure as well as 
the velocity component perpendicular to the sheet. This allows us to estimate the efficiency 
of particle acceleration. After that we find the self-consistent solution for the current sheet evolution. 
Sect.~\ref{sect:IntStruct} is devoted to numerical simulation which, as will be shown, 
fully support our analytical asymptotic solutions. Finally, in Sect.~\ref{sect:conc} we discuss 
possible astrophysical applications of our consideration.


%%%%%%%%%%%%%%%%%%%%%%%%%%%%%%%%%%%%%%%%%%%%%%%%%%%%%%%%%%%
\section{Force-free asymptotic behavior of the pulsar wind}
\label{sect:rest}

\subsection{Basic equations}

In this section we discuss the asymptotic behavior of the pulsar wind
using well-known approach of the so-called \textit{pulsar equation}
~\citep{1973Ap&SS..24..289M, 1974MNRAS.167..457O}
\begin{equation}
\begin{split}
-\left(1-\frac{\Omega_{\rm F}^2\varpi^2}{c^2}\right)\nabla^2\Psi
+2\frac{1}{\varpi}\frac{\partial\Psi}{\partial\varpi}
+\frac{\varpi^2\Omega_{\rm F}}{c^2}(\nabla\Psi)^2\frac{\mathrm{d}\Omega_{\rm F}}{\mathrm{d}\Psi}-
\\
-\frac{16\pi^2}{c^2}I\frac{\mathrm{d}I}{\mathrm{d}\Psi} = 0.
\label{G-SH}
\end{split}
\end{equation}
This equation resulting directly from Maxwell equations
describes axisymmetric stationary electromagnetic fields
\begin{eqnarray}
\bm{E} & = & -\frac{\Omega_{\rm F}}{2\pi c}\nabla\Psi,
\label{fieldsE} \\
\bm{B} & = & \frac{\nabla\Psi\times {\bf e}_{\varphi}}{2\pi r\sin\theta}
-\frac{2I}{cr\sin\theta} {\bf e}_{\varphi}
\label{fieldsB}
\end{eqnarray}
within the force-free approximation. Here $\Psi=\Psi(r,\theta)$ is the magnetic
flux, and two integrals of motion $\Omega_{\rm F}=\Omega_{\rm F}(\Psi)$ 
and $I=I(\Psi)$ are the so-called field angular velocity (more exactly, the angular
velocity of a test charged particle drifting in the crossed electromagnetic fields) 
and the total current inside the magnetic tube respectively. Below for simplicity we  
consider the case $\Omega_{\rm F} = \Omega$ only. It corresponds to the fast enough 
rotation of the neutron star when the potential drop in the inner gap $V_{\rm gap}$ 
is much smaller than the maximum value $V_{\rm tot}$.

The first solution of the pulsar equation (\ref{G-SH}) containing radial wind
was obtained by \citet{1973ApJ...180L.133M}. He demonstrated that the `split 
monopole' magnetic field corresponding to magnetic flux
\begin{eqnarray}
\Psi(r, \theta) & = & \Psi_{\rm tot}(1 - \cos\theta), \qquad \theta < \pi/2, \\
\Psi(r, \theta) & = & \Psi_{\rm tot}(1 + \cos\theta), \qquad \theta > \pi/2
\label{Michel}
\end{eqnarray}
is the exact solution of the pulsar equation if the additional relation
\begin{equation}
4 \pi I(\Psi) = \Omega_{\rm F} \left(2\Psi - \Psi^2/\Psi_{\rm tot}\right)
\label{EeqB1}
\end{equation}
holds. In this case
\begin{eqnarray}
B_{r} & = & B_{\rm L}\frac{R_{\rm L}^2}{r^2} \, {\rm sign}(\cos\theta),
\label{m1973'} \\
B_{\varphi} & = & E_{\theta} = -B_{\rm L}\frac{R_{\rm L}}{r}\sin\theta 
\, {\rm sign}(\cos\theta),
\label{m1973}
\end{eqnarray}
where $B_{\rm L} = B_{\rm p}(R_{\rm L}, \pi/2)$. 
This solution describing the axisymmetric case is called the `split-monopole' one as it contains 
the current sheet in the equatorial plane separating ingoing and outgoing magnetic fluxes.

Later~\citet{1973ApJ...186..625I} has found more general asymptotic solution, i.e., 
the solution in the limit $r \rightarrow \infty$. He has shown that in this limit 
an arbitrary function $\Psi = \Psi(\theta)$ remains the solution of the pulsar 
equation (\ref{G-SH}) if
\begin{equation}
4 \pi I(\theta) = \Omega_{\rm F} \sin\theta \, \frac{{\rm d}\Psi}{{\rm d}\theta}.
\label{EeqB2}
\end{equation}
According to (\ref{fieldsB}), this implies that in the asymptotic limit
$r \rightarrow \infty$ any $\theta$-dependence of the poloidal magnetic field
$B_{\rm p} = B_{\rm p}(\theta)$ can be realized. Here we have to stress that 
relation (\ref{rhoGJ}) for charge density remains true for monopole structure 
(\ref{m1973'})--(\ref{m1973}) only. In general case
\begin{equation}
\rho_{\rm e} = -\frac{\Omega_{\rm F}}{4 \pi \sin\theta} \,
\frac{{\rm d}}{{\rm d}\theta}\left(\sin\theta\frac{{\rm d}\Psi}{{\rm d}\theta}\right).
\end{equation}
As one can easily check using Eqns. (\ref{fieldsE})--(\ref{fieldsB}), both conditions 
(\ref{EeqB1}) and (\ref{EeqB2}) correspond to the clear relation $E_{\theta} = B_{\varphi}$.

Finally, it was found that the appropriate solution can be constructed for the oblique
rotator as well. According to~~\citet{1999A&A...349.1017B}, the `inclined split monopole'
\begin{eqnarray}
B_{\rm p} & = & B_{\rm L}\frac{R_{\rm L}^2}{r^2} \, {\rm sign}(\Phi), \\
B_{\varphi} & = & E_{\theta} = -B_{\rm L}\frac{R_{\rm L}}{r}\sin\theta \, {\rm sign}(\Phi),
\label{b1999}
\end{eqnarray}
where now
\begin{eqnarray}
\Phi = \cos\theta \cos\chi - \sin\theta \sin\chi \cos\left[\varphi - \Omega \left(t - r/c\right)\right],
\label{Phi}
\end{eqnarray}
and $\chi$ is the inclination angle, is the exact solution of the pulsar equation as well.
In the polar regions \mbox{$\theta < \pi/2 - \chi$} and \mbox{$\theta > \pi/2 + \chi$}
this solution coincides with the time-independent Michel solution (\ref{m1973'})--(\ref{m1973}),
but in the equatorial region $\pi/2 - \chi < \theta < \pi/2 + \chi$ all the field components
change the signs at the current sheet locating at the position $\Phi = 0$.

%%%%%%%%%%%%%%%%%%%%%%%%%%%%%%%%%%%%%%%%%%%%%%%%%%%%%%%%%%%
\subsection{Asymptotic behavior}
\label{sect:Axis}

In this section we are going to generalize the solutions mentioned above to substantiate
the result of numerical simulation. Indeed, as was shown by~~\citet{SashaMHD}, for large enough
inclination angle $\chi > 30^{\circ}$ the $\varphi$-average poloidal magnetic field depends
on the angle $\theta$ as
\begin{eqnarray}
\langle B_{\rm p} \rangle_{\varphi} \, \propto \, \sin\theta.
\label{Bpp}
\end{eqnarray}
For this reason, the structure of the solution with $\theta$-dependent poloidal magnetic field
(\ref{EeqB2}) is to be considered in more detail.

As we are mainly interested in the asymptotic behavior $r \gg R_{\rm L}$, one can search
a solution of the pulsar equation (\ref{G-SH}) in the form
\begin{equation}
\Psi(r,\theta) = \sum\limits_{n=0}^{\infty}\Psi_{n}(\theta)
\left(\dfrac{R_{\rm L}}{r}\right)^{2n}\mathrm{sign}(\cos\theta).
\label{dPsi}
\end{equation}
As was already stressed, the function $\Psi_{0}(\theta)$ describing the asymptotic 
magnetic field can be arbitrary if the condition (\ref{EeqB2})
$4 \pi I(\theta) = \Omega_{\rm F} \sin\theta \, {\rm d}\Psi_{0}/{\rm d}\theta$ holds.
According to (\ref{fieldsE})--(\ref{fieldsB}), we obtain in the zero approximation
\begin{eqnarray}
&&B_r  =  B_{\rm L}\left(\frac{R_{\rm L}}{r}\right)^2 \,
\frac{F(\theta)}{\sin\theta} \, \mathrm{sign}(\cos\theta),
\label{BRr} \\
&&B_{\varphi} =  E_{\theta} = -B_{\rm L}\frac{R_s}{r}F(\theta) \mathrm{sign}(\cos\theta),
\label{Ett} \\
&&B_{\theta}  = E_{\varphi}=E_{r}=0,
\end{eqnarray}
where $F(\theta) = (4/\pi)\Psi_0'/\Psi_{\rm tot}$, $\Psi_{\rm tot}=\Psi_{0}(\pi/2)$, 
and primes indicate the $\theta$-derivatives, e.g., $\Psi_0' = {\rm d}\Psi_0/{\rm d}\theta$. 
%and $\Psi_{0}'' = {\rm d}^2\Psi_0/{\rm d}\theta^2$. 
The Michel monopole solution corresponds to $F(\theta) = \sin\theta$. As to equation for 
the first disturbance $\Psi_{1}(\theta)$, it looks like
\begin{align}
\frac{1}{\sin\theta}\frac{\mathrm{d}}{\mathrm{d}\theta}
\left(\sin\theta\frac{\mathrm{d}\Psi_1}{\mathrm{d}\theta}\right)
&- \left(\cot^2\th-3 +
 3\frac{F'}{F}\cot\theta+\frac{F''}{F}\right)\Psi_1 =\nonumber\\ 
&= \Psi_{\rm tot}\frac{1}{\sin\theta}
\frac{\mathrm{d}}{\mathrm{d}\theta}\left(\frac{F}{\sin\theta}\right).
\end{align}
In particular, for $F(\theta) = \sin\theta$ we obtain $\Psi_{1}(\theta) = 0$, i.e., 
pure radial flow. On the other hand, for $\Psi_{0}'(\theta) =\Psi_{\rm tot} \sin^2\theta$ (and, hence, 
$B_{r} \propto \sin\theta$) we obtain
\begin{eqnarray}
\frac{1}{\sin\theta}\frac{\mathrm{d}}{\mathrm{d}\theta}
\left(\sin\theta\frac{\mathrm{d}\Psi_1}{\mathrm{d}\theta}\right)
- \left(9\cot^2\th-5\right)\Psi_1
= 2\Psi_{\rm tot}\cot\theta.
\label{Psi1}
\end{eqnarray}

%%%%%%%%%%%%%%%%%%%%%%%%%%%%%%%%%%%%%%%%%%%%%%%%%%%%%%%%%%%%%%%%
%\subsection{Oblique case}
%\label{sect:oblique}

\begin{figure}
\centering
\includegraphics[scale=0.7]{fig1-eps-converted-to.pdf}
\caption{The only solution  $\Psi_{1}(\theta)$ of equation (\ref{Psi1}) which has no singularity at $\theta = 0$ and \mbox{$\theta = \pi/2$}. The presence of finite solution implies that the disturbance of magnetic flux function decreases as $R^2_{\rm L}/r^2$.}
\label{fig:Ing}
\end{figure}

On Fig.~\ref{fig:Ing} we show the only solution $\Psi_{1}(\theta)$ of Eqn. (\ref{Psi1}) which 
has no singularity at $\theta = 0$ and $\theta = \pi/2$; these two conditions just determine 
the solution of this equation. As we see, this function is finite and, hence, the disturbance of the
radial poloidal magnetic field decreases as $R_{\rm L}^2/r^2$. We will use this property in what follows.

Now, to obtain the fields in oblique case we use the procedure similar to one
applied in~~\citet{1999A&A...349.1017B}. Instead of multiplying our axisymmetric solution by
${\rm Sign}(\cos\theta)$ we will multiply it by ${\rm sign}(\Phi)$. As one can easily check,
the appropriate fields satisfy Maxwell equations as well except for the current
sheet position $\Phi(r, \theta, \varphi) = 0$. Thus, one can conclude that the shape
of the current sheet does not depend on the latitudinal structure of the magnetic field.

Finally, using the definitions (\ref{BRr}) and (\ref{Ett}), one can obtain the
following very simple relation between the latitudinal structure of the radial
magnetic field $B_{r}(\theta)$ and the pulsar wind energy flux
$S(\theta)=cB_{\varphi}E_{\theta}/4\pi$ 
\begin{equation}
S(\theta) \propto \sin^2\theta B_{r}^2(\theta).
\end{equation}
The same expression for Pointing flux was also obtained by~\citet{2016MNRAS.457.3384T}. For $B_{r} =$ const we return to the expression $S(\theta) = \sin^2\theta$
which was widely used in the literature~\citep{2002AstL...28..373B, 2003MNRAS.344L..93K}.
On the other hand, for
$B_{r}(\theta)  \propto  \sin\theta$ we have
\begin{equation}
S(\theta) \propto \sin^4\theta,
\end{equation}
\\
i.e., exactly what was obtained numerically by~\citet{SashaMHD}. Of course, as in Eqn.
(\ref{Bpp}), it concerns $\varphi$-averaging values. Nevertheless, this result
implies that even very simple analytical consideration provide good enough description of
the main characteristics of the pulsar wind obtained numerically.

\section{Current sheet in the comoving reference frame}
\label{sect:fields}

\subsection{MHD approximation}

There are two reasons why the force-free model considered above is
too simple to describe the main properties of the current sheet. First, in
this model the sheet is infinitely thin. Second, within the force-free
approximation massless particles move with the velocity of light. As one
can see from (\ref{Phi}), the current sheet moves with the same velocity as
well, which prevents us considering in detail its internal structure.

For this reason below we try to pass to the reference frame comoving with the
outflowing plasma. It help us to separate the intrinsic processes inside the
current sheet from the common outflowing motion. Certainly, within the force-free
approximation this boost is impossible. For this reason in what follows we use
more general MHD approximation formulated in ~\citet{2000MNRAS.313..433B} 
\citep[see also][]{beszak04}.

Remember that to describe magnetically dominated MHD outflow it is very convenient
to introduce two dimensionless parameters, namely Michel magnetization parameter
$\sigma_{\rm M}$ and multiplicity parameter $\lambda$ ~\citep{2010mfca.book.....B}. Here
\begin{equation}
\sigma_{\rm M} = \frac{e\Omega \Psi_{\rm tot}}{8 \pi \lambda m_{\rm e}c^3},
\label{sigma}
\end{equation}
where $\Psi_{\rm tot} = \Psi(\pi/2)$ is the total magnetic flux through the upper
hemisphere, and
\begin{equation}
\lambda = \frac{n_{\rm e}}{n^{0}_{\rm GJ}},
\end{equation}
where $n^{0}_{\rm GJ} = \Omega B_{\rm p}/2\pi c |e|$ is the amplitude of the Goldreich-Julian 
number density. For ordinary pulsars \mbox{$\sigma_{\rm M} \sim 10^3$--$10^4$,} and 
$\lambda \sim 10^3$--$10^4$, and only for the fast young pulsars (Crab, Vela)
\mbox{$\sigma_{\rm M} \sim 10^5$--$10^6$,} and $\lambda \sim 10^4$--$10^5$. In the force-free 
limit $m_{\rm e} \rightarrow 0$ we have $\sigma_{\rm M} \rightarrow \infty$. On the other hand, 
for finite $\sigma_{\rm M}$ the particle velocity is smaller than that of light (see below). In 
addition, we suppose that the injection Lorentz-factor $\gamma_{\rm in} \sim 10^2$ is constant 
for all outflowing region.

As a result, according both to the theory~\citep{bes98, beam2015} and
numerical simulation~\citep{1999MNRAS.305..211B, 2006MNRAS.368.1717B},
the quasi-radial MHD flow is to intersect the fast magnetosonic surface at the distance
\begin{equation}
r_{\rm F} =  {\rm min}(\sigma_{\rm M}^{1/3} \sin^{-1/3}\theta, 
\sqrt{\sigma_{\rm M}/\gamma_{\rm in}}) \, R_{\rm L},
\label{rF}
\end{equation}
the Lorentz-factor at this surface being
\begin{equation}
\gamma_{\rm F} = {\rm max}(\sigma_{\rm M}^{1/3}\sin^{2/3}\theta, \gamma_{\rm in}).
\label{gF}
\end{equation}
For $\gamma_{\rm in} \gg \sigma_{\rm M}^{1/3}$ (slow rotation) it is necessary to use the
second expressions, while for  $\gamma_{\rm in} \ll \sigma_{\rm M}^{1/3}$ (fast rotation)
they realized in the narrow cone $\theta < \gamma_{\rm in}^{3/2}\sigma_{\rm M}^{-1/2}$ near
the rotational axis only; for most angles the first expressions are to be used.

It is very important that for both fast and slow rotation outside the fast magnetosonic
surface there are actually no collimation and particle acceleration. More exactly,
\begin{equation}
\gamma \approx \sigma_{\rm M}^{1/3}\sin^{2/3}\theta \log^{1/3}(r/r_{\rm F})
\end{equation}
for fast, and $\gamma_{\rm F} \approx \gamma_{\rm in}$ for slow 
rotation~\citep{1994PASJ...46..123T, bes98, beam2015}. This implies that 
with the logarithmic precision one can believe that at large 
distances $r \gg r_{\rm F}$ the particles move with constant 
velocity~\mbox{$v_{r} < c$} exactly corresponding to the drift 
velocity in~\mbox{$U_{\rm dr}/c = E/B$}~\citep{2009ApJ...699.1789T, 2010mfca.book.....B}. 
This property just helps us to move into the reference frame comoving with a particular 
part of the wind (at some constant $\theta$).

\subsection{Fields in the comoving reference frame}

\subsubsection{Introductory remarks}
\label{sect:rem}

As was already stressed, the first attempts in describing the internal structure 
of the `striped' pulsar wind were done by~\citet{coroniti_striped_wind_1990} and ~\citet{michel_pulsar_wind_1994}. They based their analysis on magnetic reconnection. 
%
%But they did not include into consideration radial dependence of toroidal magnetic and electric fields.
%Accordingly, Bogovalov solution (\ref{m1973'})--(\ref{m1973})
%corresponds to infinitely thin sheet and, hence, says nothing about its internal structure
%as well. In all cases it was rather difficult to separate the zero order electromagnetic
%fields resulting in general radial drift motion with the relativistic velocity $v \approx c$
%from the much smaller values which really control the internal structure of the current sheet.

The next step was made by~\citet{2001ApJ...547..437L} describing the internal structure 
of the moving current sheet by introducing `fast' and `slow' variables allowing to consider 
the current sheet moving radially with the velocity \mbox{$v_{\rm sh} < c$.} { But they 
did not consider internal structure of a sheet postulating actually zero magnetic field 
inside it.} Later, the following electromagnetic fields were considered~\citep{2013MNRAS.434.2636P}
\begin{eqnarray}
B_{r}(r, \theta, \varphi, t')   & = &   B_{\rm L}\frac{R_{\rm L}^2}{r^2} \, 
\tanh\left(\dfrac{R_{\rm L}}{\Delta_{\rm lab}}\Phi_{1}\right), \\
B_{\varphi}(r, \theta, \varphi, t')  & = &   -\frac{B_{\rm L}}{\beta}
\frac{R_{\rm L}}{r}\sin\theta \, \tanh\left(\dfrac{R_{\rm L}}{\Delta_{\rm lab}}\Phi_{1}\right), \\
E_{\theta}(r, \theta, \varphi, t')   & = & = 
-B_{\rm L}\frac{R_{\rm L}}{r}\sin\theta \, \tanh\left(\dfrac{R_{\rm L}}{\Delta_{\rm lab}}\Phi_{1}\right).
\label{b1999}
\end{eqnarray}
Here now
\begin{eqnarray}
\Phi_{1} =\cos\theta \cos\chi
- \sin\theta \sin\chi \cos\left[\varphi - \Omega \left(t' - r/\beta c\right)\right],
\label{Phi1}
\end{eqnarray}
where $t'$  is the time in the laboratory reference frame, $\beta = v_{\rm sh}/c <1$ and 
the fuction $\tanh(...)$ was taken for clear historical reason{\footnote{This function 
corresponds to well-known \citet{harris1962} solution.}} 
(this function can be arbitrary).

Unfortunately, these fields cannot be considered as good enough zeroth approximation as
they have no force-free limit inside the sheet with finite thickness. Indeed, as can
be easily checked, the $\theta$ and $\varphi$ dependencies of the function $\Phi_{1}$
(\ref{Phi1}) denies the existence of $\theta$ and $\varphi$ components of the current
density inside the sheet even in the force-free approximation. On the other hand, in
this limit $j_r = \rho_{\rm e}c$, and, hence, to support the components $j_{\theta}$
and $j_{\varphi}$ the particle velocity is to be larger than that of light. By the way,
~\citet{2001ApJ...547..437L} did not include into consideration the radial component of 
the Maxwell equation $\nabla \times {\boldsymbol B} = \dots$  (it was postulated that the radial
component of the current density $j_{r} = 0$ in spite of $[\nabla \times {\boldsymbol B}]_{r} \neq 0$),
so their analysis cannot be considered as self-consistent as well.

Here we present another approach to this problem { carrying out calculations 
in the reference frame moving radially with the current sheet}. It allows us to avoid the
leading components of electromagnetic fields resulting in the common drift motion.
Our consideration is based on the another exact force-free solution obtained 
by~\citet{2011PhRvD..83l4035L}
\begin{eqnarray}
&&B_r= B_{\rm L} \left(\frac{R_{\rm L}}{r} \right)^2;
\label{eqn2_1}\\
&&B_\vph= E_\th =-\frac{\b}{r} \sin \th f(r- c t'),
\label{eqn2_2}
\end{eqnarray}
where $f(...)$ is again an arbitrary function. Having no dependence on $\theta$ and $\varphi$,
this fields are in agreement with the condition ${\boldsymbol j} = \rho_{\rm e}c \, {\bf e}_{r}$. We can
use this solution for inclined rotator because the shape of the current sheet in this case is
similar to the spherical wave, see, e.g.,~\citet{2012MNRAS.420.2793K}.

Of course, it is necessary to stress that this solution contains no sign change of the radial
component of the magnetic field $B_r$. On the other hand, as was already mentioned above, 
outside the fast magnetosonic surface $r \gg r_{\rm F} \sim \sigma_{\rm M}^{1/3} R_{\rm L}$ 
(\ref{rF}) both the disturbance of the monopole poloidal magnetic field resulting from the 
MHD disturbances~\citep{bes98} and the disturbances (\ref{dPsi}) connecting with 
$\theta$-dependence of the poloidal field have the same smallness $\sim \sigma_{\rm M}^{-2/3}$ 
at the fast surface and, hence, can be neglected.

For this reason, in what follows we put $B_r = 0$. As is well-known, this approximation is 
good enough outside the fast magnetosonic surface $r > r_{\rm F}$ and widely used in analysis 
of the pulsar wind~\citep{2001ApJ...547..437L, 2014PhRvD..89j3013B}. Indeed, according to 
(\ref{rF})--(\ref{gF}), for fast rotator in the comoving reference frame the toroidal 
magnetic field $B'_{\varphi} = B_{\varphi}/\gamma$ becomes larger than poloidal one 
$B'_{r} = B_{r}$ just outside the fast surface; for slow rotator it takes place even 
at smaller distances. 

As a result, we can modify now this solution for arbitrary
$\theta$-dependence of the poloidal magnetic field $B_{\rm r}(\theta)$ which, as was 
already stressed, better corresponds to the real structure of the pulsar wind.
For $f\equiv - \tanh$ the fields in the laboratory frame 
($r, \theta, \varphi, t^{\prime}$) can be presented as
\begin{eqnarray}
&&B_\vph (r, \theta, t') = \frac{1}{\beta} \frac{\b}{r} F(\theta) \td,
\label{F1}\\
&&E_\th (r, \theta, t') = \frac{\b}{r} F(\theta) \td\label{F2},
\end{eqnarray}
where $\Delta_{\rm lab}$ is a current sheet thickness in the laboratory reference frame, 
and $F(\theta)$ now is the arbitrary function. As was already stressed, the parameter 
$\beta = E_{\theta}/B_{\varphi}$ can be considered here as a constant.

%One of the key property of this solution is that the parameter $\beta$ can be considered 
%as $r$-independent. Indeed, as was shown by~\citep{1994PASJ...46..123T, bes98}, outside 
%the fast magnetosonic surface $r \gg r_{\rm F}$ both the disturbance of quasi-conical 
%magnetic surfaces and hydrodynamical Lorentz factor $\gamma$ increase very slowly; 
%$\propto \ln^{1/3}(r/r_{\rm F})$. Hence, in zero approximation the particle energy can 
%be considered as a constant alongradial magnetic field line. As for $r \gg r_{\rm F}$ 
%the particle energy is fully determinedby the drift 
%motion~\citep{2008MNRAS.388..551T, 2010mfca.book.....B}, one can put 
%$\beta = E_{\theta}/B_{\varphi} = $ const. 

Accordingly, the charge density in this frame is equal to Goldreich-Julian charge density:
\begin{equation}
\rho_{\rm e} = 
- \frac{\b}{4\pi r^2 \sin \theta} \frac{{\rm d}[F(\theta)\sin\theta]}{{\rm d}\theta} \td 
=  n_{GJ \rm} e,
\end{equation}
but the current density is now equal to
\begin{equation}
j = \frac{\rho_{\rm e}}{\beta}.
\end{equation}
This implies that within the MHD approximation the velocities of electron and positron
components are to be different. For magnetically dominated case one can seek the first 
order corrections in the following form:
\begin{equation}
v_r^\pm/c = 1 - \xi_r^\pm;~v_\th^\pm/c=\xi_\th^\pm;~v_\vph^\pm /c= \xi_\vph^\pm.
\end{equation}
As the particle number densities can be now written as
\begin{equation}
n^\pm = n_{GJ\rm} \left[\lambda \mp \frac{1}{4} {\hat D}_{\theta}F(\theta) \right],
\end{equation}
where ${\hat D}_{\theta}F(\theta)=F'(\theta) + F(\theta)\cot\theta$ 
(e.g., ${\hat D}_{\theta} \sin\theta = 2 \cos\theta$),
one can obtain using the definition of the current density 
\mbox{$j = e n^+ v_r^+ - e n^- v_r^-$}
\begin{equation}
\xi_r^\pm = 1-\beta \pm \frac{1}{2 \lambda \gamma^2 \beta},
\end{equation}
or
\begin{equation}
v_r^\pm/c = \beta \mp \frac{1}{2 \lambda \gamma^2 \beta}.
\end{equation}
Here $\gamma = (1-v^2/c^2)^{-1/2}$ is the Lorentz factor and $\lambda$ again is the
multiplicity parameter. 
%Finally, using the general approach described in 
%Appendix A, one can find the following expression
%\begin{equation}
%|E_{\theta}|\simeq\beta|B_{\varphi}|
%\label{zeta}
%\end{equation}
%which can be applied outside the fast magnetosonic surface. As was already stressed, 
%toroidal magnetic field $B_{\varphi}$  is larger than the electric field $E_{\theta}$.

\subsubsection{Comoving reference frame --- orthogonal case}
\label{sect:cmframe}

As was already stressed, important property of the solution (\ref{F1})--(\ref{F2}) 
presented above is that the parameter $\beta$ can be considered as a constant. Thus, the
reference frame moving radially with the velocity $V = \beta c$ is the inertial one. 
In order to study the field structure in this reference frame moving with the current sheet,
we need to express the fields in Cartesian coordinates and make a Lorentz transform:
\begin{eqnarray}
B_x'=B_x;~B_y'=\Gamma (B_y+ \beta E_z);~B_z' = \Gamma (B_z - \beta E_y), \\
E_x'=E_x;~E_y'=\Gamma (E_y - \beta B_z);~E_z'=\Gamma(E_z+\beta B_y).
\end{eqnarray}
Here and below $\Gamma = (1-\beta^2)^{-1/2}$ is the boost Lorentz-factor, and the values 
without prime correspond to the reference frame moving radially with the $x$-axis  
directed along the radius-vector ${\bf e}_{r}$ and the $y$-axis along ${\bf e}_{\varphi}$ 
(see Fig.~\ref{Fig02}). As a result, near the origin of the reference frame moving radially with 
the velocity $V = \beta c$ (and for $f \equiv -\tanh$) the first order fields look like 
\begin{eqnarray}
B_{y}(x, y, z, t) & = & B_{0}\frac{R_{\rm L}}{c t}
\left(1 - \frac{z^2}{c^2\,t^2}\right)
\tanh \left(\frac{x}{\Gamma \Delta_{\rm lab}} \right),
\label{BYinitial} \\
E_{x}(x, y, z, t) & = & B_{0}\frac{R_{\rm L} z}{c^2t^2} \tanh
\left(\frac{x}{\Gamma \Delta_{\rm lab}} \right).
\label{EXinitial}
\end{eqnarray}
Here $x = \Gamma(r-\beta c t')$, $t = t'/\Gamma$ is the time in the comoving 
reference frame, and
\begin{equation}
B_{0} = \frac{B_{\rm L}}{\beta^2 \Gamma^2} F(\theta),
\label{B00}
\end{equation}
which now can be considered as a constant. Note that this expression 
has $\Gamma^2$ instead of $\Gamma$ in denominator since $B_0$ actually
represents magnetic field at fast magnetosonic surface in comoving 
frame. Fast magnetosonic surface corresponds to the time 
$t_0' = R_{\rm F}/\beta c \Rightarrow t_0 \approx R_{\rm L}/c$.
{ Finally, as $z \ll ct$, we do not include below the factor
$(1 - z^2/c^{2}t^{2})$ into consideration.}

\begin{figure}
\centering
\includegraphics[scale=0.45]{fig2-eps-converted-to.pdf}
\caption{Inertial reference frame ($x, y, z, t$) moving radially with velocity $V =\beta c$.}
\label{Fig02}
\end{figure}

Further, as one can directly check, Maxwell equation 
$c \nabla \times {\boldsymbol E} = - \partial {\boldsymbol B}/\partial t$ is automatically
fulfilled. Finally, for another Maxwell equation 
$c \nabla \times {\boldsymbol B} =  \partial {\boldsymbol E}/\partial t + 4 \pi {\boldsymbol j}$ to 
be valid, two-fluid MHD consideration is necessary. This question will be
considered in detail in Sect.~\ref{sect:IntStruct}. At the moment we only 
stress that the charge density along the boost axis (i.e., $x$-axis) is equal 
to zero, and the particle velocities after the Lorentz transform become
\begin{equation}
\frac{v_x^{\pm}}{c} = \mp \frac{1}{(2\lambda\pm 1) \beta}\approx \mp \frac{1}{2\lambda \beta}.
\end{equation}

Expressions (\ref{BYinitial})--(\ref{EXinitial}) are our main intermediate result 
giving zero approximation for the internal structure of the quasi-spherical 
current sheet in its comoving reference frame. As we see, this solution is
essentially time-dependent. Indeed, $r^{-1}$ diminishing of the toroidal 
magnetic field $B_{\varphi}$ in the pulsar wind transforms into $t^{-1}$ time
dependence in the comoving reference frame. As will be shown below, this results 
in a set of new effects which do not exist in classical time-independent
configurations.

%Corresponding to such fields velocities are:
%\be
%\xi^+_x = - \xi^-_x = - \frac{1}{4 \pi n e \Gamma\lambda} \frac{z}{t^3} \frac{\b \sin^m \th}{\Gamma^2 \beta^2}\tanh \left(\frac{x}{\Gamma \Delta_{\rm lab}} \right);
%\ee
%\be
%\xi^+_z = - \xi^-_z =  \frac{1}{8 \pi n e \gamma^2\lambda\Delta_{\rm lab}} \frac{1}{t} \frac{\b \sin^m \th}{\gamma^2 \beta^2}\cosh^{-2} \left(\frac{x}{\gamma \Delta_{\rm lab}} \right).
%\ee
%%%%%%%%%%%%%%%%%%%%%%%%%%%%%%%%%%%%%%%%%%%%%%%%%%%%%%%%%%%

\subsubsection{Comoving reference frame --- aligned case}
\label{sect:align}

Similarly, on can consider the internal structure of the current sheet for the
axisymmetric case when the current sheet locates in the equatorial plane. 
%Aligned case differs from orthogonal only in the form of the current sheet.
%For sheet located in the equatorial plane 
After passing into comoving reference frame (the calculations are quite similar 
to Section~\ref{sect:cmframe}) we obtain for the leading components of 
electromagnetic  field ($\theta = \pi/2$):
\begin{eqnarray}
B_{y}(x, y, z, t) & = & B_{0} \frac{R_{\rm L}}{ct} 
\tanh \left(\frac{z}{\Gamma \Delta_{\rm lab}} \right), \\
E_{x}(x, y, z, t) & = & B_{0} \frac{R_{\rm L}z}{c^2 t^2} 
\tanh \left(\frac{z}{\Gamma \Delta_{\rm lab}} \right).
\end{eqnarray}
This fields differ from Eqns. (\ref{BYinitial})--(\ref{EXinitial}) by $z$
(not $x$) coordinate perpendicular to the sheet plane. 

It is necessary to stress that recent PIC simulations of the axisymmetric
pulsar magnetosphere~\citep{2015MNRAS.448..606C} demonstrate non-stationarity of the equatorial
sheet so that a high amplitude wave is generated just outside the light cylinder. 
In other words, axisymmetric equatorial sheet actually cannot exist. Nevertheless,
we consider this case as well.

%%%%%%%%%%%%%%%%%%%%%%%%%%%%%%%%%%%%%%%%%%%%%%%%%%%%%%%%%%%
\section{Internal structure of time-dependent current sheet}
\label{sect:IntStruct}

In this section we discuss in detail the structure of electromagnetic fields 
and particle drift motion for time-dependent current sheet { not far 
from the fast magnetosonic surface. In this domain the time-dependent effects 
are most pronounced. Most of the section is devoted to discussion of orthogonal 
case, while aligned case is discussed in Appendix~\ref{Appendix.align}.}

{ In our analysis, we do not consider the important effects related to different instabilities (e.g., tearing and drift-kink) which can drastically disturb the structure of a sheet. The inclusion of these effects is beyond the scope of this study and we leave it for the following paper.}
%\cmtla{DONE I don't think we can neglect them. E.g. the rough estimate for the drift-kink instability growth rate is $\omega_{\rm B}^{-1}$ in the comoving frame. It means that the current sheet will move a distance less than a light cylinder (in the lab frame) before becoming unstable. The best thing we can do is saying that we understand it, but still want to solve simplified problem of stable sheet.} 

Note, that for simplicity we use $f(x) = \tanh(x)$ form of the current sheet 
(Harris sheet),  everywhere except for Sect. \ref{sect.A.E.F.}. The results 
can be easily applied to any physically reasonable form, i.e. odd function with 
\begin{equation}
\lim\limits_{x\rightarrow \pm \infty}f(x)\rightarrow \pm 1 
\nonumber
\end{equation}
and  $f(0)=0$. 

\subsection{Accelerating electric field}
\label{sect.A.E.F.}

As was already stressed, the solution (\ref{BYinitial})--(\ref{EXinitial}) 
constructed above corresponds to the constant width $\Delta_{\rm lab}$ of the current 
sheet separating time-dependent magnetic fluxes. To describe the time evolution 
of the  current sheet width $\Delta_{\rm lab}(t)$ (which is one of the main goals of this paper),
it is necessary to include into accelerating another component of the electric
field $E_{z}$.

To show this, it is convenient to rewrite the relations (\ref{BYinitial})--(\ref{EXinitial}) 
in the form
\begin{eqnarray}
B_{y}(x, y, z, t) & = & B_{0}\frac{R_{\rm L}}{c t} h(x,t),
\label{BYin} \\
E_{x}(x, y, z, t) & = & B_{0}\frac{R_{\rm L} z}{c^2 t^2} h(x,t).
\label{EXin}
\end{eqnarray}
where $h(x,t)=f[x/\Delta(t)]$. Then, we can rewrite Maxwell equation 
$c \nabla \times {\boldsymbol E} = - \partial {\boldsymbol B}/\partial t$  as
\begin{equation}
\frac{\partial E_{z}}{\partial x} 
= B_{0}\frac{R_{\rm L}}{c^2 t} \frac{\partial h}{\partial t}. 
\end{equation}
It gives
\begin{equation}
E_{z}
= B_{0}\frac{R_{\rm L}}{c^2 t} \frac{\partial}{\partial t}\left(\int\limits^{x}_{\infty} h(x',t){\rm d}x'\right). 
\end{equation}
Further, as $f(x)$ and $h(x,t)$ are both odd functions of $x$, 
%$\int f(x) {\rm d} x$,
one can choose the integrating constant for $E_{z} \propto \int h(x',t){\rm d}x'$
to be even function with clear boundary conditions $E_{z}(\pm \infty) = 0$. Clearly, in this case 
we obtain $E_{z}(x=0)\neq 0$ near the center plane of the current sheet. Finally, Maxwell equation 
$c \nabla \times {\boldsymbol B} =  \partial {\boldsymbol E}/\partial t + 4 \pi {\boldsymbol j}$ 
now looks like
\begin{equation}
\label{eqn:ez}
B_{0}\frac{R_{\rm L}}{c t} \frac{\partial h}{\partial x} 
= \frac{1}{c \, }\frac{\partial E_{z}}{\partial t} + \frac{4 \pi}{c} \, j_{\rm z}. 
\end{equation}
For Harris current sheet one can obtain the following expressions:
\begin{align}
B_{y} & =  B_{0} \frac{t_0}{t}\tanh\dfrac{x}{\Delta(t)},
\label{ByOrt}\\
E_{x}  & =   B_{0} \frac{t_0z}{ct^2}\tanh\dfrac{x}{\Delta(t)},
\label{ExOrt}\\
E_{z} & =  B_{0} \frac{t_0\Delta'(t)}{ct}\left\{\log\left[2\cosh\dfrac{x}{\Delta(t)}\right]-
\right. \label{EzOrt} \\
&- \left.\dfrac{x}{\Delta(t)}\tanh\dfrac{x}{\Delta(t)}\right\}.
\notag
\end{align}

As we see, time-dependence of the current sheet thickness $\Delta(t)$ inevitably results in 
the appearance of the nonzero electric field $E_{z}$ along the sheet which is larger 
than magnetic one near the zero surface. Thus, the evolution of the current sheet
width and the problem of particle acceleration cannot be considered separately.

Acceleration of particles could be estimated from considering particles trapped deep inside 
the current sheet $x\ll \Delta$. For such particles, the change of $z$-component of momentum 
due to $E_z$ could be found from the solution of 
\begin{equation}
\dot p_z \approx e B_0 \frac{t_0 \Delta'(t)}{ct} \log(2),
\end{equation}
giving the acceleration
\begin{equation}
\delta p_z(t) = \int\limits_{t_0}^{t} \dot p_z {\rm d}t = \frac{\kappa \log(2)}{\kappa - 1} \frac{e B_0 \Delta_0}{c}[(t/t_0)^{\kappa -1} - 1].  
\label{eq:deltapz}
\end{equation}
Here $\Delta'(t) = {\rm d}\Delta/{\rm d} t$, and we assume power-law dependence 
for the current sheet thickness 
$\Delta(t) \propto t^\kappa$. Equation (\ref{eq:deltapz}) is valid for 
$\kappa \ne 1$. For $\kappa = 1$ integration (\ref{eq:deltapz}) gives 
logarithmic divergence. For such case we assume 
$\delta p_z \approx e B_0 \Delta_0/c = {\rm const}$. 
For $\kappa < 1$, $\delta p_z$ asymptotically goes to constant. In this 
case we also set $\delta p_z \approx e B_0 \Delta_0/c = {\rm const}$. 
On the other hand, for $\kappa > 1$ momentum grows as $t^{\kappa - 1}$
with time. Combining all the expressions, one can approximate the 
acceleration of particles in the current sheet as
\begin{align}
\delta p_z(t)  &= e B_0 \Delta_0/c, ~~~~\qquad \kappa \le 1 \label{eq:pz1}\\
\delta p_z(t)  &= e B(t) \Delta(t)/c, \qquad \kappa > 1 \label{eq:pz2}
\end{align}
where $B(t)$ is the value of magnetic field outside of the current sheet $B(t) = B_0 (t_0/t)$.

In Figure \ref{fig:p(t)}, expression (\ref{eq:deltapz}) is compared with the exact 
solution of particle equations of motion in the fields (\ref{ByOrt})--(\ref{EzOrt})
for different $\kappa$. The numerical solution is in good agreement with analytical 
prediction, except for $\kappa = 0.5$. The small difference between analytical and 
numerical solution in $\kappa = 0.5$ case is examined in Figure \ref{fig:p(x)}. 
All particles examined for this Figure start with $ z_0 = 0$, $v_x = 0$, 
$v_z = \omega_{\rm B}\Delta_0$ and with different initial $x_0$. An analytical value 
$\delta v_z = \log 2 \omega_{\rm B}\Delta_0$ corresponds to the limit of this curve 
at $x_0 \rightarrow 0$. For larger $x_0$, acceleration is larger, but stays order of
magnitude the same. We thus conclude that estimations (\ref{eq:pz1})--(\ref{eq:pz2})
are in good agreement with numerical solution.

\begin{figure}
\centering
\includegraphics[scale=0.55]{acceleration-eps-converted-to.pdf}
\caption{The acceleration of particles along the sheet by an additional electric 
field for different expansion parameters $\kappa$. Solid lines correspond to the 
time dependence of particle velocity, and dashed lines represent analytical prediction 
(\ref{eq:deltapz}).}
\label{fig:p(t)}
\end{figure}
\begin{figure}
\centering
\includegraphics[scale=0.55]{dvz_tot-eps-converted-to.pdf}
\caption{Dependence of the total acceleration in the sheet with $\kappa = 0.5$ on the 
initial coordinate of the particle. Acceleration increases until $x_0/\Delta_0\sim 2.5$. 
It corresponds to the last initially trapped orbit. Particles with larger $x_0$ move 
outside the sheet until getting trapped, and the total change in velocity decreases.}
\label{fig:p(x)}
\end{figure}

\subsection{Particle drift outside the current sheet}
  \label{part.drift}

{ In this section we discuss the main manifestations of time-dependent
effects in the proper reference frame of the current sheet. First of all,} in the 
comoving reference frame the particles experience drift motion outside the current 
sheet ($x \gg \Delta$ for orthogonal case and $z \gg \Delta$ for aligned case). The 
fastest component of such motion is $z$-component of electric drift 
\begin{equation}
\label{eqn.drift.vel.}
    U_z = c\frac{E_x B_y}{B_y^2} = \frac{z}{t}.
\end{equation}
This velocity exactly corresponds to the radial divergence of the flow due to 
radial expansion. Since within our approximation $z \ll ct $, the drift velocity 
of particles remains non-relativistic.

The drift velocity $U_z$ is directed along the current sheet in orthogonal case, 
and out of the sheet in aligned case. For orthogonal case it implies that the 
particle, which is outside of the current sheet initially, will move along the 
sheet until it's orbit starts to intersect the midplane of the sheet. After 
this moment the particle is trapped and no longer drifts. On the other hand, 
for aligned case, particles drift away from the sheet. If current sheet expands 
slowly enough, such particles will never get trapped. Similarly, the particles, 
which are initially trapped might escape the sheet and start to drift with 
velocity $U_{\rm esc}^{\rm align} = \Delta(t_{\rm esc})/t_{\rm esc}$, where 
$t_{\rm esc}$ is the time when particle escapes the sheet. One can easily find 
that escape can only occurred if $\Delta(t)$ expands slower than linear.

As was shown in Sect.~\ref{sect.A.E.F.}, in orthogonal case the additional time 
dependence of a sheet width leads to additional electric field diminishing 
exponentially outside the sheet: $E_{\rm add}\propto\exp(-x/\Delta)$. Since for 
particles outside the sheet $\Delta\ll x$, we can neglect this electric field. 
The same remains true for gradient drift.
 
Further, Larmor radius of particles changes due to the conservation of the first 
adiabatic invariant $p_{\perp}^2/B_y \approx$ const. This connection allows 
us to evaluate the dependence of gyroradius on time (cf.~\citealt{2001ApJ...547..437L})
\begin{equation}
\label{eqn.L.R.}
\mathcal{R}_{\rm L}(t) = \mathcal{R}_{\rm L}(t_0)\left(\dfrac{t}{t_0}\right)^{1/2}.
\end{equation}
Such a dependence shows that the gradient drift (or any other drifts except electric) 
is not just significantly smaller than $z$-component of electric drift, but 
also smaller compared to Larmor radius growth. For orthogonal case 
Eqns. (\ref{eqn.drift.vel.}) and (\ref{eqn.L.R.}) indicate that any particle that 
initially is outside the sheet eventually gets trapped inside the sheet.

Now we can compare an average gyro-radius of particles with the thickness of sheet. 
If gyroradius of a particle is much smaller than the width 
of the current sheet, it is possible to use the drift approximation. The only component 
of the drift velocity having $x$-component is the electric drift 
\mbox{$U_{x} = -c E_{z}B_{y}/B^2$} 
\begin{equation}
U_{x} = -\Delta'(t)
\left\{\dfrac{\log\left[2\cosh( x/\Delta)\right]}{\tanh (x/\Delta)}  - x/\Delta\right\},
\label{eq:Ux}
\end{equation}
 In particular, for \mbox{$x \ll \Delta_0$} we have
\begin{equation}
U_{x} = -\log(2)\frac{\Delta'(t)\Delta(t)}{x}.
\end{equation}
Certainly, it is impossible to use this evaluation in the very center of a sheet $x \rightarrow 0$
where $E_{z} > B_{y}$.

As one can see, for $\Delta \gg {\cal R}_{\rm L}$, when there are a lot of particles 
with $\mathcal{R}_{\rm L} \ll |x| < \Delta$ which do not cross the null surface, these 
particles will drift towards $x = 0$. This results in slow collapse of the sheet until 
the current sheet becomes thin enough so $\mathcal{R}_{\rm L} \sim \Delta$.

On the other hand, it is impossible to sustain the Harris current sheet if 
$\mathcal{R}_L \gg \Delta$ as the particles in such a sheet spend most of 
their time in the region $|x| > \Delta$, i.e. outside of the sheet. This 
leads to the conclusion that in realistic current sheet
$\mathcal{R}_{\rm L} \sim \Delta$.

\begin{figure}
\centering
\includegraphics[scale=1]{trapping-eps-converted-to.pdf}
\caption{Trapping of particles in the orthogonal current sheet for different starting 
positions outside the sheet. The points correspond to the positions of particles at 
the same moment in time. As $x$-positions of these points are close, the time until
particles get trapped is the same for particles with the same initial $x$.}
\label{fig:trap}
\end{figure}

{ Finally, time-dependence of the current sheet results in the trapping
of particles in the vicinity of the null surface. We illustrate this effect in
Figure~\ref{fig:trap} where we show} the trajectories of drifting particles outside the 
sheet. All particles start with the same velocity and $x_0$, but with different $z_0$. 
Points correspond to the positions of particles at the same moment in time showing that 
the time required for particle to get trapped is almost independent on $z_0$, and is 
determined by the $x$-component of drift velocity (\ref{eq:Ux}).

\subsection{Self-consistent solutions for non-relativistic particles}
\label{sect:solAn}

In this section we find exact solution for a particle trapped deep inside the orthogonal
current sheet $x \ll \Delta$. Surprisingly, the equations for non-relativistic particles 
could be integrated exactly. For relativistic particles not only equations could not be 
solved analytically, but also the actual approximation $x,z \ll ct$ breaks down and the 
fields (\ref{ByOrt})-(\ref{EzOrt}) could not be used.

In the orthogonal case we can use expressions (\ref{ByOrt})-(\ref{EzOrt}) for electromagnetic 
fields:
\begin{align}
B_y &= B_0 \frac{t_0}{t} \tanh\frac{x}{\Delta(t)},\\
E_x &= B_0\dfrac{t_0z}{ct^2} \tanh\frac{x}{\Delta(t)}\\
E_z &= \kappa \dfrac{B_0t_0\Delta(t)}{ct^2} 
\left\{\log\left[2\cosh\dfrac{x}{\Delta(t)}\right]-\right.\\
&-\left. \dfrac{x}{\Delta(t)} \tanh\dfrac{x}{\Delta(t)}\right\},
\end{align}
where we again assume $\Delta(t)\propto t^{\kappa}$. After substitution these expression into equation 
of motion we obtain the following system of equations:
\begin{align}
\ddot{z} &= \dfrac{eB_0t_0}{m_{\rm e} c t}\left\{\dot{x}\tanh\dfrac{x}{\Delta(t)}+\right.\\
&+\kappa\dfrac{\Delta(t)}{t}\log\left[2\cosh\dfrac{x}{\Delta(t)}\right]
-\left.\kappa\dfrac{x}{t}\tanh\dfrac{x}{\Delta(t)}\right\},
\nonumber \\
\ddot{x} &= \dfrac{eB_0t_0}{m_{\rm e} ct^2}\tanh\dfrac{x}{\Delta(t)}(z-\dot{z}t).\label{eq:ddotx}
\end{align}

Integrating now the first equation one can obtain
\begin{equation}
z-\dot{z}t=-\Delta(t)\Lambda\log\left[2\cosh\dfrac{x}{\Delta(t)}\right]+C
\label{zaz}
\end{equation}
corresponding to conversation of the following invariant 
\begin{equation}
I = m_{\rm e}z - P_{z}t.
\label{Iinv}
\end{equation}
Here $\Lambda =  \omega_{B} t_0$ ($\omega_{B}= e B_0/m_{\rm e} c$ is cyclotron frequency),
and $P_z = p_z + e A_z$ is generalized momentum. This integral remains constant for non-relativistic 
case only. Further, substituting (\ref{zaz}) into (\ref{eq:ddotx}) we obtain
\begin{equation}
\label{eq.x.full.ort}
\ddot{x}=-\Lambda^2\frac{\Delta(t)}{t^2}\tanh\dfrac{x}{\Delta(t)}\log\left[2\cosh\dfrac{x}{\Delta(t)}\right].
\end{equation}
Here we neglect the constant $C$ in comparison with $\Delta(t)$ which is possible for expanding sheet.

To evaluate the asymptotic behaviour of nonlinear equation (\ref{eq.x.full.ort}) (which has 
no exact analytical solution) we present the coordinate $x(t)$ in the form $x(t) = \Delta(t)S(t)$ 
where $S(t)$ is a restricted function. After substituting this into equation (\ref{eq.x.full.ort}) 
we obtain
\begin{equation}
t^{2} S''+2 \kappa t S' + \kappa(\kappa-1)S
= -\Lambda^2\tanh S\log[2\cosh S],
\end{equation} 
where primes indicate derivatives over $t$. 

Expanding now r.h.s. into Maclaurin series (which is possible for $S \ll 1$) we obtain for the
leading term
\begin{equation}
\label{eqn.S.ort}
t^{2}S''+ 2\kappa t S' + \log (2) \Lambda^2S = 0.
\end{equation}
Here we suppose that $\Lambda \simeq 4\lambda\sigma_{\rm M} \gg 1$.
Equation (\ref{eqn.S.ort}) is linear and have the following solutions
\begin{equation}
\label{sol.S.ort}
S_{\pm} = S_0 \, t^{-(2\kappa-1)/2} 
\cos\left[\sqrt{\log (2)}\Lambda \log (t) + \varphi_{\pm}\right].
\end{equation}
%\begin{equation}
%\label{sol.S.ort2}
%K = \frac{2\kappa-1}{2}. 
%\end{equation}
As a result, it becomes clear that unless $\kappa\neq 1/2$ the expansion of the sheet 
doesn't follow the expansion of particle trajectory, for which we have
\begin{equation}
    x(t)\approx 
x_{0}\left(\frac{t}{t_0}\right)^{1/2}\cos\left[\alpha (x_0) \Lambda\log (t)
+\varphi_{0}\right]
\label{eq:orb_orth}
\end{equation}
for arbitrary oscillation amplitude $x_0$. The constant $\alpha(x_0)$ slowly varies
between $\sqrt{\log (2)}$ at $x_0 \ll \Delta$ and unity for \mbox{$x_0 \gg \Delta$.} 
An analytical solution (\ref{eq:orb_orth}) is compared with numerical solution in 
Figure~\ref{fig:orb_orth} demonstrating their good agreement. { It is also clear 
from (\ref{sol.S.ort}), that the growth rate proportional to square root of time is 
a solution for arbitrary expanding sheet.}

\begin{figure}
\centering
\includegraphics[scale=0.55]{single_particle-eps-converted-to.pdf}
\caption{Comparison of numerical solution for motion of particle inside the
sheet (blue line) with analytical prediction (\ref{eq:orb_orth}). Red line 
shows the dependence of amplitude of particle oscillations on time
$x_{\rm max} = x_0 t^{1/2}/t_0^{1/2}$.}
\label{fig:orb_orth}
\end{figure}
%%%%%%%%%%%%%%%%%%%%%%%%%%%%%%%%%%%%%%%%%%%%%%%%%%%%%%%%%%%%%%%%%%

\subsection{Current sheet thickness and particle acceleration}
\label{sect.Est}

In previous section we have found self-consistent solution for internal 
structure of current sheet, however, in this section we will show, that 
such solution cannot be asymptotic one. Below we evaluate the current 
sheet thickness based on its global structure. We will explore MHD equations 
in order to find asymptotic behaviour of current sheet.

Starting from the Faraday's law
\begin{equation}
    \dfrac{B}{\Delta}=8\pi n_{\rm in} e \dfrac{v_z}{c}
\end{equation}
and combining it with the definition of multiplicity $\lambda$, we can express 
thickness of the current sheet through $v_{z}(t)/c$ and $n_{\rm out}/n_{\rm in}$ 
parameters:
\begin{equation}
\Delta=R_{L}\dfrac{t}{t_0}\dfrac{\Gamma}{4\lambda}\dfrac{c}{v_{z}}\dfrac{n_{\rm out}}{n_{\rm in}}.
\label{DD}
\end{equation}
Here $n_{\rm out}$ and $n_{\rm in}$ refer to particle number density outside a
nd inside the current sheet respectfully.

It is clear that asymptotically $v_{z}/c$ growths up to a constant value. For 
$v_{z}(\infty)/c<1$ we get non-relativistic case, which correspond to particle
motion discussed above. It is possible to determine value of 
$\delta p_{z}\simeq eB\Delta/c$:
 \begin{equation}
 \delta p_z = m_{\rm e}c \Gamma^2 \dfrac{c}{v_{z}}\dfrac{n_{\rm out}}{n_{\rm in}}.
 \label{eq:pz_final}
 \end{equation}
 This expression could be used for $\kappa \ge 1$ to get $\delta p_z(t)$, or for 
 $\kappa < 1$ to get $\delta p_z(\infty)$. For the latter case, the values of 
 $v_z$, $n_{\rm in}$ and $n_{\rm out}$ should be taken at $t = t_0$. 
 
 From equation (\ref{eq:pz_final}) one can see that if the sheet is relativistic 
 ($v_z \sim c$) and $n_{\rm in}\sim n_{\rm out}$, the particle can reach a Lorentz 
 factor $\gamma_{\rm e} \sim \Gamma^2 = \sigma_{\rm M}^{2/3}$. It is obvious that 
 such rapid acceleration breaks some of the assumption made in the beginning (e.g. 
 the Lorentz factor of the current sheet in the laboratory frame is only $\Gamma$).
 On the other hand, one may expect that in realistic sheet $n_{\rm in} \gg n_{\rm out}$, 
 so $\gamma_{\rm e} \ll \Gamma^2$. 
 
 { Further, as was shown earlier, the sheet with $\Delta \gg \mathcal{R}_{\rm L}$ 
 collapses. Under this condition the force balance perpendicular to the sheet plane 
 can be written as
 \begin{equation}
 \frac{B^2}{8\pi} = n_{\rm in} \mathcal{E}_{\rm k},
 \label{balance}
 \end{equation}
 where $\mathcal{E}_{\rm k}(t)$ is the  particle thermal kinetic energy inside the sheet. 
 Using now the definition (\ref{sigma}) for magnetization parameter $\sigma_{\rm M}$, 
 one can rewrite this relation in a simple form
 \begin{equation}
 \frac{\mathcal{E}_{\rm k}}{m_{\rm e} c^2} = 
 \frac{\sigma_{\rm M}}{\Gamma} \, \frac{n_{\rm out}}{n_{\rm in}}.
 \label{bbalance}
 \end{equation}
As we see, non-relativistic approximation inside the sheet is valid for high enough 
number density 
 \begin{equation}
  \frac{n_{\rm in}}{n_{\rm out}} > \sigma_{\rm M}^{2/3}.
 \label{nonrel}
 \end{equation}

As a result, comparing relations (\ref{DD}) and (\ref{bbalance}) one can conclude 
that linear increase of the sheet thickness $\Delta \propto t$ predicted by~\citet{coroniti_striped_wind_1990} and~\citet{michel_pulsar_wind_1994} and 
recently reproduced in numerical simulation~\citep{Q} can be realized for constant
particle energy $\mathcal{E}_{\rm k}$ and constant ratio $n_{\rm in}/n_{\rm out}$
only. The first condition $\mathcal{E}_{\rm k} \approx$ const is in agreement with 
time-independent acceleration energy along the sheet. On the other hand, the second
one $n_{\rm in}/n_{\rm out} \approx$ const can be realized only if the particle
inflow significantly increases the number of particles inside the sheet. Such as
inflow was also reproduced in numerical simulations (see e.g.~\citealt{2015MNRAS.448..606C}).
 
 Besides, if the plasma is hot, $\mathcal{E}_{\rm k} > m_{\rm e} c^2$, we obtain for 
 the Larmor radius
 \begin{equation}
 \langle\mathcal{R}_{\rm L} \rangle = \frac{c\langle p_\perp\rangle }{eB} =
 \frac{\mathcal{E}_{\rm k}}{eB} =
 R_{L}\dfrac{t}{t_0}\dfrac{\Gamma}{4\lambda}\dfrac{n_{\rm out}}{n_{\rm in}}.
 \end{equation}
As one can see, $\langle\mathcal{R}_{\rm L}\rangle = \Delta\cdot v_z/c$. Thus, for 
the condition  $\Delta \sim \mathcal{R}_{\rm L}$ to be fulfilled, relativistically 
hot current sheet requires $v_z \sim c$. 
 }
 
 { On the other hand, if the number density outside the sheet is much 
 smaller than inside, the number of particles inside the sheet cannot increase. This 
 will lead, to $n_{\rm in}\propto t^{-3}$, but $n_{\rm out}$ 
 will be proportional to $t^{-2}$. So the ratio between $n_{\rm out}$ and $n_{\rm in}$ 
 should be of order unity.}
 In this case particle acceleration, can be estimated as
\begin{equation}
    \delta p_z\simeq m_{\rm e}c \Gamma^2,
\end{equation}
or, in laboratory frame,
\begin{equation}
    \delta p_{z}^{\rm lab} \simeq m_{\rm e}c \Gamma.
\end{equation}
 Which is the same order as mean thermal momentum of particles inside current sheet.
 
% In order to determine the average Larmor radius of particles, one can use pressure balance
 %\begin{equation}
 %\frac{B^2}{8\pi} = n_{\rm in} T,
 %\end{equation}
% where $T(t)$ is the temperature of the sheet. If the plasma is hot, $T > m_{\rm e} c^2$, we obtain
% \begin{equation}
% \langle\mathcal{R}_{\rm L} \rangle = \frac{c\langle p_\perp\rangle }{eB} = \frac{T}{eB} = R_{L}\dfrac{t}{t_0}\dfrac{\Gamma}{4\lambda}\dfrac{n_{\rm out}}{n_{\rm in}}.
% \end{equation}
%As one can see, $\langle\mathcal{R}_{\rm L}\rangle = \Delta\cdot v_z/c$. On the other hand, 
%as was shown earlier, the sheet with $\Delta \gg \mathcal{R}_{\rm L}$ collapses. Thus, 
%relativistically hot current sheet requires $v_z \sim c$. 
 
%As for the case of cold sheet $T\ll m_{\rm e}c^2$,
%\begin{equation}
%\langle\mathcal{R}_{\rm L}\rangle = \frac{c\langle p_\perp\rangle }{eB} = \frac{\sqrt{2m_{\rm e}T}c}{eB} = R_{L}\dfrac{t}{t_0}\dfrac{1}{2\sqrt{2}\lambda}\dfrac{n_{\rm out}^{1/2}}{n_{\rm in}^{1/2}}.
%\end{equation}
%In this case $\langle\mathcal{R}_{\rm L}\rangle = \Delta\cdot\sqrt{2} v_z/c \sqrt{n_{\rm in}/n_{\rm out}}\Gamma^{-1}$. The approximate equality $\mathcal{R}_{\rm L} \simeq \Delta$ translates to $v_z/c \simeq \Gamma \sqrt{n_{\rm out}/n_{\rm in}}$. It is clear that this case can only work if $n_{\rm in}/n_{\rm out} > \Gamma^2$. So it is necessary for the cold current sheet to have number density inside the sheet much greater than outside. One can also show that in this case if $v_z \ll c$, than $v_z \sim \sqrt{2m_{\rm e}T}$, i.e. mean velocity is close to thermal one.


\section{Conclusions and discussion}
\label{sect:conc}

{ This paper provides the formalism to describe essentially time-dependent 
evolution of the current sheet in the pulsar wind. As the first step in this direction, 
we carry out the calculations in the comoving reference frame and successfully determine 
intrinsic electromagnetic fields of the current sheet. In our opinion, this approach 
allows to describe more vividly the physical processes in a sheet.}

%{\color{green} In this paper we have tried to make the next step in understanding the
%internal structure of the current sheet in the pulsar wind. Carrying out calculations 
%in the comoving reference frame we succeeded in determining intrinsic electromagnetic
%fields of the current sheet which is shown to be essentially time-dependent. In our 
%opinion, this approach allows us to describe more vividly the physical processes in a 
%sheet.}

In the first part of this paper we investigate asymptotic structure of the wind from 
rotating oblique neutron star using force-free approximation. {  General asymptotic 
solution of the Grad-Shafranov equation for quasi-spherical pulsar wind up to the second 
order in small parameter \mbox{$\varepsilon = (\Omega r/c)^{-1}$} was  obtained. We have 
shown that the wind can have arbitrary latitude dependence of the energy flux. In particular, 
our solution describes the latitudinal structure of the radial magnetic field obtained 
numerically for oblique rotator. The form of current sheet in asymptotic does not depend on latitudinal structure of and matches the one in Bogovalov solution (\citealt{1999A&A...349.1017B}) } 
%Pointing vector with Bogovalov-like solution such that shape of current sheet does
%not depend on the latitudinal structure of the radial magnetic field.

{ As the force-free approximation does not allow us to discuss the inner structure 
of the current sheet, we use MHD approximation in which the velocity of the pulsar wind 
is assumed to be less than that of light. Indeed, as it is well known (see, 
e.g.,~\citealt{bes98}), outside the fast magnetosonic surface the velocity of 
quasi-spherical MHD flow becomes almost constant. Using this property (and carrying 
out calculations in the comoving reference frame) we estimate the efficiency of the 
particle acceleration inside the sheet.}

{ The main conclusion of our consideration is that the intrinsic time-dependence 
of a sheet in the comoving reference frame (especially the increase of the sheet thickness 
$\Delta$) inevitably results in the appearance of the electric field which is larger than
magnetic one inside the sheet. It is this electric field that controls the electric current
of a sheet.}

{ Finally, after investigating the motion of individual particles in the time-dependent 
current sheet, we { evaluate} the width of the sheet and its time evolution. In particular, 
we considered both relativistic and non-relativistic temperatures inside the sheet.} 

{ In particular, it was shown, that while the individual particle orbit grows as 
$t^{1/2}$, the sheet as whole should growth linearly with time. This contradiction 
can be solved by using methods of kinetic equation. Since the particle inflow inside 
a sheet due to its expansion is considerable, the evolution of current sheet is 
determined by incoming particles and not by the evolution of individual particles 
inside the sheet.}

%For orthogonal pulsar in the case of non-relativistic particle equation of motion inside 
%arbitrary expanding current sheet have been found and solved numerically (even analytically 
%for particles deep inside current sheet). The amplitude of oscillations of particles inside 
%the sheet grows proportional to $t^{1/2}$. However such growth rate of the sheet thickness 
%cannot globally satisfied Maxwell equations without constant growth of number of particle.

%Analysis of the global current sheet structure leads to two possible solutions: one is 
%non-relativistic which requires linear growth of the mean particle velocity along the sheet.
%Another solution is relativistic, and predicts linear growth of the current sheet thickness. 
%Such a solution predicts effective particle acceleration along the sheet. Linear expansion 
%can only continue until the thickness of the current sheet becomes comparable to the distance 
%between two sheets in stripped wind: $\Delta_{\rm lab} \sim \pi R_{\rm L}$. As one can see, 
%this happens at
%\begin{equation}
%r = 4\pi\lambda r_{\rm F} = 4\pi\lambda\sigma_{\rm M}^{1/3} R_{\rm L}.
%\end{equation}
%For Crab parameters ($\lambda\simeq 10^4$, $\sigma\simeq 10^6$)  $r\simeq 10^7R_{\rm L}$ which 
%is less then termination shock distance ($ 10^9 R_{\rm L}$).  

{ As for particle acceleration, we show that in relativistic case when number
density inside the sheet is similar to one outside the sheet, particle gains additional
$m_{\rm e}c\sigma^{1/3}$ momentum (in laboratory frame) due to expansion of a sheet. It is
important to notice that although the acceleration region $|{\boldsymbol E}| > |{\boldsymbol B}|$ is
narrow, the trajectories of particles are adjusted in such a way that they accelerate 
in MHD region $|{\boldsymbol E}| < |{\bf B}|$ every half period in one direction, cross 
$|{\bf E}| > |{\boldsymbol B}|$ zone, and then accelerate in the same direction again 
(see Figure~\ref{fig:trap}). }

Finally, it is important to highlight the difference between present article and~\citet{2001PASA...18..415K} in which authors use global equation in order to 
determine growth of current sheet thickness. In our work we do not consider 
reconnection, so our result should be interpreted as pre-reconnection sheet structure.
{ However, the electric field, which results from time-dependence of sheet 
thickness in case of linear growth, has the same structure as reconnection field  
and it is possible, that even in case of reconnection, our description of 
individual particle motion remains relevant.}

\section{Acknowledgments}

We thank Ya.N.~Istomin and A.Philippov for their interest and useful discussion 
{ and the anonymous referee for instructive comments which helped us to improve 
the manuscript}. This research was partially supported by the government of the 
Russian Federation (agreement No. 05.Y09.21.0018) and by Russian Foundation for 
Basic Research (Grant no. 15-02-03063). 


%%%%%%%%%%%%%%%%%%%%%%%%%%%%%%%%%%%%%%%%%%%%%%%%%%%%%%%%%%%%%%%%%%%%%%%%%%%%%%%%%%%%

{\small
\bibliographystyle{mn2e}
\bibliography{mybib}
%\documentclass[usenatbib,fleqn]{mnras}
\usepackage{newtxtext,newtxmath}  % more compact font
\usepackage{graphicx,hyperref,amsmath,mathrsfs,xspace}
\usepackage[dvipsnames]{xcolor}

\title[Milky Way mass measurement with LMC]{Measuring the Milky Way mass distribution in the presence of the LMC}

\author[Correa Magnus \& Vasiliev]{
Lilia Correa Magnus$^{1}$\thanks{E-mail: lilimagnus@hotmail.es},
Eugene Vasiliev$^{2,3}$\thanks{E-mail: eugvas@lpi.ru}\\
$^1$University of Edinburgh, Peter Guthrie Tait road, Edinburgh, EH9 3FD, UK \\
$^2$Institute of Astronomy, Madingley road, Cambridge, CB3 0HA, UK\\
$^3$Lebedev Physical Institute, Leninsky prospekt 53, Moscow, 119991, Russia
}

\newcommand{\Gaia}{\textit{Gaia}\xspace}
\newcommand{\kms}{km\:s$^{-1}$\xspace}
\newcommand{\masyr}{mas\:yr$^{-1}$\xspace}
%\renewcommand{\floatpagefraction}{.8}
\date{Accepted 2021 December 17. Received 2021 December 10; in original form 2021 September 30}

\begin{document}
\label{firstpage}
\pagerange{2610--2630}\volume{511}\pubyear{2022}
\setcounter{page}{2610}
\maketitle

\begin{abstract}
The ongoing interaction between the Milky Way (MW) and its largest satellite -- the Large Magellanic Cloud (LMC) -- creates a significant perturbation in the distribution and kinematics of distant halo stars, globular clusters and satellite galaxies, and leads to biases in MW mass estimates from these tracer populations. We present a method for compensating these perturbations for any choice of MW potential by computing the past trajectory of LMC and MW and then integrating the orbits of tracer objects back in time until the influence of the LMC is negligible, at which point the equilibrium approximation can be used with any standard dynamical modelling approach. We add this orbit-rewinding step to the mass estimation approach based on simultaneous fitting of the potential and the distribution function of tracers, and apply it to two datasets with the latest \Gaia EDR3 measurements of 6d phase-space coordinates: globular clusters and satellite galaxies. We find that models with LMC mass in the range $(1-2) \times 10^{11}\,M_\odot$ better fit the observed distribution of tracers, and measure MW mass within 100~kpc to be $(0.75\pm0.1)\times 10^{12}\,M_\odot$, while neglecting the LMC perturbation increases it by $\sim15$\%.
\end{abstract}

\begin{keywords}
Galaxy: kinematics and dynamics -- Magellanic Clouds -- globular clusters: general -- Local Group
\end{keywords}


%%%%%%%%
\section{Introduction}   \label{sec:intro}

The study of stellar dynamics in the Milky Way (MW) and its neighbourhood flourishes in recent years thanks to the vast amount of observational data provided by the \Gaia satellite and numerous ground-based spectroscopic surveys. One aspect of this analysis is the measurement of the mass distribution up to the virial radius%
\footnote{here $r_\text{vir}$ is defined as the radius within which the mean density is 102 times higher than the cosmic density of matter, and is related to the virial mass as $r_\text{vir} = (M_\text{vir}/10^{12}\,M_\odot)^{1/3} \times 260$~kpc.}
of the MW, primarily based on the kinematics of various tracer populations: distant halo stars, globular clusters and satellite galaxies. Commonly used methods rely on the Jeans equations or on simultaneous modelling of the tracer distribution function (DF) and the MW gravitational potential with simple analytic expressions, such as the power-law mass estimator of \citet{Watkins2010}, variants of which have been applied to globular clusters or satellites by \citet{Eadie2019} and \citet{Fritz2020}, respectively. More complicated DF families in action space were employed by \citet{Posti2019} and \citet{Vasiliev2019b} for the globular cluster population, all based on the previous \Gaia data release (DR2). The most recent Early data release 3 \citep{Brown2021} improved the accuracy of proper motions (PM) by a factor of two or better, and the new measurements of cluster and satellite PM by \citet{Vasiliev2021a}, \citet{McConnachie2020}, \citet{Li2021}, \citet{Battaglia2021} are awaiting application to the MW mass modelling.

At the same time, a new challenge has emerged in dynamical analysis of the MW outskirts: the Large Magellanic Cloud (LMC), which just passed the pericentre of its orbit, appears to be a massive enough satellite to cause a significant perturbation in the motion of stars and other objects at distances beyond a few tens kpc. Various recent estimates of the LMC mass put it in the range $(1-2)\times 10^{11}\,M_\odot$ (see \citealt{Shipp2021} and references therein), i.e., only $5-10$ times smaller than the MW itself.

The effects of the LMC on the MW system are manifold. First, it has brought a host of its own satellites with it, some of which are now stripped and became MW satellites, as discussed, e.g., in \citet{Battaglia2021}. Second, it deflects MW stars, satellites and stellar streams that pass in its vicinity, sometimes dramatically changing their orbits (e.g., the southern portion of the Orphan--Chenab stream, \citealt{Erkal2019}). The MW halo stars deflected by the LMC form a density wake behind its orbit, as per the classical dynamical friction scenario, which might have been detected as the Pisces overdensity \citep{Belokurov2019}. However, another equally important and perhaps less obvious effect is caused by the global perturbation induced by the LMC on the outer parts of the MW, which arises from a combination of two factors. On the one hand, the LMC and the MW move under mutual gravitational forces as a binary system, so the reflex motion of the MW produces an acceleration of the associated Galactocentric reference frame, as first stressed by \citet{Gomez2015}. The non-inertial frame is not necessarily a concern: the fact that the entire MW is also pulled towards the Andromeda galaxy and towards the Virgo cluster has little effect on the motion of test bodies in the Galaxy because the acceleration is spatially uniform, and the Galactocentric reference frame is free-falling, as in the Einstein's elevator \textit{gedankenexperiment}. But given the proximity of the LMC, distant objects in the MW halo feel a different acceleration from it than the MW centre -- in other words, the MW becomes deformed by the LMC (and, of course, vice versa). One could view this from a different perspective: the dynamical time in the inner Galaxy (e.g., in the Solar neighbourhood) is short enough that the LMC passage is an adiabatic perturbation for these objects, and their Galactic orbits are little affected. But the outer halo objects are not able to adjust their orbits rapidly enough and thus are displaced w.r.t.\ the MW centre -- or, better to say, the central part of the MW rapidly swings towards the LMC on its pericentre passage, while its outer parts largely stay put. In the end, the resulting dipole perturbation is manifested both in the density and in the kinematics of the outer MW halo \citep[e.g.,][]{Cunningham2020, GaravitoCamargo2020}, and both effects have been detected recently: the former in \citet{Conroy2021}, the latter in \citet{Erkal2021} and \citet{Petersen2021}.

Naturally, these perturbations invalidate the assumption of dynamical equilibrium that lies at the heart of most dynamical modelling methods. \citet{Erkal2020b}, using a simple power-law mass estimator from \citet{Watkins2010}, concluded that the neglect of this perturbation causes an upward bias in the inferred MW mass by up to 50\%, depending on the LMC mass and the distance within which the MW mass is measured. More recently, \citet{Deason2021} measured the MW mass out to 100~kpc from the kinematics of halo stars, while approximately correcting for the LMC-induced bias by shifting the stellar velocities by a distance-dependent offset calibrated in $N$-body simulations, although this changed the resulting mass estimate by $\lesssim 2\%$.

In this work, we take a step further and develop an orbit-rewinding method for compensating the perturbation from the LMC in any given MW potential model, which can be used with any classical dynamical modelling approach based on the equilibrium assumption. In subsequent analysis, we employ the method for measuring the gravitational potential of the MW by fitting a parametric DF to the population of tracer objects, and apply it to two classes of tracers: globular clusters and satellite galaxies, using the latest \Gaia EDR3 measurements. Section~\ref{sec:method} describes our modelling method, including the orbit-rewinding step and another novel aspect -- the treatment of possible outliers in the kinematic sample.  Section~\ref{sec:tests} demonstrates the performance of the method on mock datasets. Section~\ref{sec:MW} specifies the model setup for the actual MW and the observational catalogues that we use in the analysis, and Section~\ref{sec:results} presents the results of our fits: the MW mass profile, the properties of the tracer populations (primarily satellite galaxies), and discusses the role of the LMC in their kinematics. Section~\ref{sec:summary} wraps up.


%%%%%%%%
\section{Method}   \label{sec:method}

Our approach falls within the class of likelihood-based forward modelling methods. For any choice of parameters describing the DF of the tracer population, the (initial, unperturbed) potential of the MW, and the LMC mass, we evaluate the likelihood of the observed dataset in this model, taking into account the observational errors and the perturbation from the LMC. We then explore the parameter space with the Monte Carlo Markov Chain method. Below we describe these steps in more detail. 

Our analysis pipeline relies on the \textsc{Agama} stellar-dynamical framework \citep{Vasiliev2019a}, which provides a wide choice of gravitational potentials, DFs, orbit integration and other tools, but the general idea can be applied in a broader context and with other modelling platforms.

%%%%%%%%%%%
\subsection{Orbit rewinding}  \label{sec:orbit_rewinding}

The key novel aspect of our work is the orbit rewinding procedure for compensating the LMC perturbation. It is applied to the tracer population for each choice of model parameters, and consists of two steps: first we reconstruct the past trajectories of the MW and the LMC under mutual gravitational forces, and then integrate the orbits of tracers backward in time up to the point when the LMC perturbation was negligible.

Although the LMC was usually considered as just one of many MW satellites, whose orbit is determined entirely by the gravitational potential of the Galaxy and the dynamical friction, the relatively large mass ratio (likely between $1:10$ and $1:5$) between the two galaxies makes their interaction far more complex. Both galaxies exert gravitational force onto each other, but are also deforming in the process of interaction, so the forces acting on the central parts of both galaxies have additional contribution from their own distorted mass distribution in the outer parts. As discussed in \citet{Vasiliev2021c}, the ``self-gravity'' is actually the dominant mechanism in the evolution of orbital angular momentum: the commonly used approximation based on the Chandrasekhar dynamical friction formula produces qualitatively different evolution than found in $N$-body simulations. Nevertheless, these nonlinear phenomena kick in largely after the first pericentre passage. In the case of the LMC, we are fortunate that it just passed its pericentre very recently, likely for the first time \citep[e.g.,][]{Kallivayalil2013}, and therefore its past trajectory might be reasonably well approximated by a simple kinematic model with two rigid mutually gravitating galaxies, without worrying too much about their internal deformations. This approach adequately captures the most important effect of the MW reflex motion, as highlighted by \citet{Gomez2015}, and was used in a number of subsequent papers (e.g., \citealt{Jethwa2016,Erkal2019, Vasiliev2021b, Shipp2021}); here we briefly summarize it.

The positions and velocities of the MW and the LMC evolve according to the coupled ODE system:
\begin{equation}  \label{eq:lmc_mw_orbit}
\begin{aligned}
\dot{\boldsymbol{x}}_\text{MW} &= \boldsymbol{v}_\text{MW}, \\
\dot{\boldsymbol{v}}_\text{MW} &= -\nabla\Phi_\text{LMC}(\boldsymbol{x}_\text{MW}-\boldsymbol{x}_\text{LMC}), \\
\dot{\boldsymbol{x}}_\text{LMC} &= \boldsymbol{v}_\text{LMC}, \\
\dot{\boldsymbol{v}}_\text{LMC} &= -\nabla\Phi_\text{MW}(\boldsymbol{x}_\text{LMC}-\boldsymbol{x}_\text{MW}) + \boldsymbol{a}_\text{DF}, 
\end{aligned}
\end{equation}
where $\Phi_\text{MW}$ and $\Phi_\text{LMC}$ are static, non-deforming potentials of both galaxies, and $\boldsymbol{a}_\text{DF}$ is the Chandrasekhar dynamical friction acceleration \citep[Equation~8.6]{Binney2008}:
\begin{equation}  \label{eq:dynfric}
\begin{aligned}
\boldsymbol{a}_\text{DF} &\equiv \frac{-4\pi\, \rho_\text{MW}\, G^2\, M_\text{LMC} \ln \Lambda}{v^2}\left[\text{erf}(X) - \frac{2X\,\exp(-X^2)}{\sqrt{\pi}}\right] \frac{\boldsymbol{v}}{v} , \\
X &\equiv v/ \sqrt{2}\sigma_\text{MW} .
\end{aligned}
\end{equation}
Here $\rho_\text{MW}$ and $\sigma_\text{MW}$ are the density and velocity dispersion of the host galaxy at the given position%
\footnote{The velocity dispersion could be obtained from the Jeans equation for each choice of MW potential, but for simplicity, we adopt a universal profile $\sigma_\text{MW} = 150 / \big(1 + |\boldsymbol x_\text{LMC}-\boldsymbol x_\text{MW}| / 100\,\text{kpc}\big)$~\kms, having checked that the results are insensitive to it.}%
, $M_\text{LMC}$ is the LMC mass, $\ln\Lambda$ is the Coulomb logarithm, which we discuss further in Section~\ref{sec:test_lmc_orbit}. In other words, the centres of both galaxies move as point masses in each other's gravitational potential, but these potentials are rigidly attached to the moving points. Since the interaction is manifestly non-symmetric, these equations do not conserve any of the classical integrals of motion, but nevertheless approximate the actual trajectories reasonably well, as we demonstrate in the later section. The equations are integrated back in time from the present-day position and velocity of the LMC in the Galactocentric frame for a time $T_\text{rewind}$. The choice of this rewinding time is somewhat arbitrary, but the LMC needs to recede sufficiently far away so that its perturbation is negligible. We find that for $T_\text{rewind}=2$~Gyr, the LMC moves from its present-day distance of $\sim 50$~kpc to $200-300$~kpc, and even for the most massive MW potentials, still remains near the apocentre of its orbit (for lower-mass MW, the orbit is actually unbound).

In the second step, the orbits of all tracer objects are integrated backward in time for the same interval $T_\text{rewind}$ in the combined MW+LMC potential. Of course, the MW centre as computed from Equation~(\ref{eq:lmc_mw_orbit}) moves away from origin, but it is more convenient to carry out subsequent steps in the non-inertial reference frame pinned to the MW centre, so we initialize a composite time-dependent potential of the two galaxies as follows. The MW potential is fixed at origin, the LMC is represented by a rigid analytic potential moving along the pre-computed trajectory $\boldsymbol{x}_\text{LMC}(t)-\boldsymbol{x}_\text{MW}(t)$, and we add a spatially uniform but time-dependent acceleration $\boldsymbol{a}_\text{non-inertial}(t) \equiv -\ddot{\boldsymbol{x}}_\text{MW}(t)$. The positions and velocities of tracers at the moment $T_\text{rewind}$ in the past are then assumed to represent the original, unperturbed state of the MW before the LMC arrival. If desired, we may again integrate the orbits from these past coordinates forward up to present time in a static potential of the MW alone, obtaining the ``compensated'' present-day coordinates that the objects would have had if the LMC did not exist. However, since the past and the compensated coordinates belong to the same orbit and differ only in phase angles, they are equivalent for the purpose of dynamical modelling, which uses only the integrals of motion of tracer objects. As discussed in Section~\ref{sec:test_orbit_rewinding}, orbits in the inner part of the Galaxy are almost unperturbed by the LMC, so we may save a considerable effort by not rewinding them at all.

%%%%%%%%%%%
\subsection{Distribution function fitting approach} \label{sec:df_fit}

The assumption of dynamical equilibrium underpins almost all classical methods for measuring the gravitational potential. According to the Jeans theorem \citep[Chapter~4.2 in][]{Binney2008}, in the this case the DF $f$ of any tracer population may depend only on the integrals of motion $\mathcal I(\boldsymbol x, \boldsymbol v ;\; \Phi)$ in the given potential $\Phi$ (the latter represents the total mass distribution and needs not be related to the density profile of the tracers). Although many studies prefer to constrain the gravitational potential from the DF moments using the Jeans equations, in this work we choose to work with the DF itself. Namely, we optimize the parameters $\boldsymbol{\Pi}$ of the potential $\Phi(\boldsymbol{x};\;\boldsymbol\Pi)$ and the DF $f(\mathcal{I};\;\boldsymbol\Pi)$ simultaneously to maximize the log-likelihood of drawing the observed 6d phase-space coordinates $\boldsymbol w \equiv \{\boldsymbol x, \boldsymbol v\}$ of tracer objects in the catalogue:
\begin{equation}  \label{eq:likelihood}
\ln\mathcal{L} = \sum_{i=1}^{N_\text{tracers}} \ln f\big( \mathcal I(\boldsymbol w_i; \; \Phi )\big).
\end{equation}

The DF-fitting approach has been shown to perform well in a variety of contexts, even with missing data dimensions, as explored e.g.\ by \citet{Read2021} on mock datasets representing stars in dwarf galaxies with resolved stellar kinematics. In practice, we need to choose a suitable family of parametric DFs and potentials, which are detailed below.

A flexible model for the density profile of spherical or ellipsoidal stellar systems is a double-power-law \citet{Zhao1996} profile, augmented with an optional exponential cutoff (\texttt{Spheroid} model in \textsc{Agama}):
\begin{equation}  \label{eq:spheroid_density}
\begin{aligned}
\rho &= \rho_0 \left (\frac{\tilde r}{r_\text{scale}} \right)^{-\gamma}
\left [1+\left (\frac{\tilde r}{r_\text{scale}} \right)^{\alpha} \right]^{\frac{\gamma - \beta}{\alpha}} \!\!
\exp\left[-\left (\frac{\tilde r}{r_\text{cut}}\right)^\xi \right] , \\
\tilde r &\equiv \sqrt{R^2+(z/q)^2}\quad \mbox{(ellipsoidal radius)} ,
\end{aligned}
\end{equation}
where $r_\text{scale}$ is the scale radius, $\gamma$ is the asymptotic inner slope $-\mathrm{d}\ln\rho/\mathrm{d}\ln r$, $\beta$ is the outer slope, $\alpha$ is the steepness of transition between the two regimes, $r_\text{cut}$ is the optional outer cutoff radius, and $\xi$ is the cutoff strength. We use this profile for the MW dark halo, the LMC (represented by a spherical NFW profile, i.e., $\gamma=1, \beta=3, \alpha=1$), and for the density profile of tracers in some cases. In particular, our primary model for the MW dark halo has no cutoff and 5 free parameters ($\rho_0$, $r_\text{scale}$, $\gamma$, $\beta$, $\alpha$), but we additionally consider the \citet{Einasto1965} profile with 3 free parameters ($\rho_0$, $r_\text{cut}$, $\xi$, setting $\beta=\gamma=0$); both Zhao and Einasto models are special cases of the \texttt{Spheroid} model, but the full 7-parameter model has too much flexibility to be practical. The baryonic component of the MW is represented by an exponential disc model and a spheroid bulge, as described in \citet{McMillan2017}.

For the tracer DF, we have two alternative choices. The first one is the \texttt{QuasiSpherical} DF constructed from the given tracer density $\rho_\text{tr}$ in the given total potential $\Phi$, with two tunable parameters specifying the velocity anisotropy: the value of anisotropy coefficient $\beta_0 \equiv 1 - \sigma_\text{tan}^2 / 2\sigma_\text{rad}^2$ at small radii, and the anisotropy radius $r_\text{a}$, beyond which the distribution gradually becomes dominated by radial orbits. Its functional form is $f(E,L) = L^{-2\beta_0}\,f_Q\big(E + L^2/2r_\text{a}^2\big)$, with the function $f_Q$ constructed numerically from the provided $\rho_\text{tr}$ and $\Phi$ using the generalization of the Eddington inversion formula by \citet{Cuddeford1991}. For some combinations of parameters, this integral inversion may result in an unphysical DF attaining negative values for some $E,L$, in which case these parameters are discarded from further consideration. 
This model can be used even with non-spherical potentials, but is limited to systems whose equidensity contours match the equipotential contours and without net rotation. 

A more flexible alternative is offered by DFs in the space of actions $\boldsymbol{J} \equiv \{J_r,\, J_\phi,\, J_z\}$ (Section~3.5 in \citealt{Binney2008}), which are a convenient choice for integrals of motion $\mathcal I$, and can be reasonably accurately computed from $\{\boldsymbol x, \boldsymbol v\}$ in any axisymmetric potential, using the St\"ackel fudge method \citep{Binney2012}. We use the following DF family with 9 free parameters, which is adapted from \citet{Posti2015}:
\begin{equation}  \label{eq:DPL_DF}
\begin{aligned}
f(\boldsymbol{J}) &= f_0\;
\left[ 1 + \left(\frac{J_0}{h(\boldsymbol{J})}\right)^\eta \right]^{\frac{\Gamma}{\eta}}
\left[ 1 + \left(\frac{g(\boldsymbol{J})}{J_0}\right)^\eta \right]^{\frac{\Gamma-\mathrm{B}}{\eta}} \\
 &\times \left[1 + \tanh\frac{\varkappa J_\phi}{J_r+J_z+|J_\phi|}\right] , \\
g(\boldsymbol{J}) &\equiv g_r J_r + g_z J_z\, + (3-g_r-g_z)\, |J_\phi|, \\
h(\boldsymbol{J}) &\equiv h_r J_r + h_z J_z   + (3-h_r-h_z)   |J_\phi|.
\end{aligned}   
\end{equation}
Here the dimensionless coefficients $\Gamma$ and B are related to the density slope at small and large radii, respectively, $\eta$ is the steepness of transition between the two asymptotic regimes,  $J_0$ sets the dimensional scale action, $\varkappa$ defines the amount of streaming motion (rotation), and the mixing coefficients $g_{r,z}$ and $h_{r,z}$ control the velocity anisotropy and spatial flattening at large and small radii, respectively. The normalization $f_0$ is determined from the condition that the total mass (the integral of the DF over the 3d action space) is unity. This DF is very similar to the built-in \texttt{DoublePowerLaw} DF in \textsc{Agama}, with an important difference regarding the rotation: the parameter $\varkappa$ controls the ratio of mean azimuthal velocity to its dispersion at all radii, instead of large radii only, as in the original definition; this better matches the kinematics of observed tracer populations in our study. The density generated by this DF depends on its parameters and on the potential in a rather non-trivial way, but we do not need to know it explicitly: the likelihood of the model is computed from the 6d phase-space coordinates translated into action space, not from the 3d density profile of tracers. In practice, it is flexible enough to represent approximately the two-power-law \texttt{Spheroid} profile and to have shape and velocity anisotropy adjustable separately at small and large radii.

%%%%%%%%%%%
\subsection{Treatment of observational errors}  \label{sec:obs_errors}

Any observational data come with associated uncertainties, which should be taken into account in model fitting. The likelihood of the given datapoint in the model is given by the convolution of the model DF with the error distribution of this point:
\begin{equation}  \label{eq:error_convolution_integral}
f\big( \boldsymbol w^\text{(obs)}_i \big) =
\int f\big( \boldsymbol w^\text{(true)} \big)\, E\big( \boldsymbol w^\text{(obs)}_i \,|\, \boldsymbol w^\text{(true)}, \epsilon_i \big)\; \mathrm{d}\boldsymbol w^\text{(true)} ,
\end{equation}
where $E$ is the normalized probability of measuring $w^\text{(obs)}_i$ given the true phase-space coordinates $\boldsymbol w^\text{(true)}$ and observational uncertainties $\epsilon_i$. In practice, the uncertainties on distance, line-of-sight velocity and PM can be assumed to follow independent normal distributions with the standard deviation (or the full $2\times2$ covariance matrix for the two PM components) quoted in the catalogue. For our sample of tracers (globular clusters and satellite galaxies), the relative uncertainty in the distance is typically at the level of a few percent, in velocity -- a few \kms, and in PM -- $0.02-0.1$~\masyr, corresponding to the transverse velocity error of $10-50$~\kms at 100~kpc (sometimes with a significant correlation between the two PM components, which is fully taken into account). We evaluate the above four-dimensional convolution integral with the Monte Carlo method, replacing it with a sum over $N_\text{samp}$ random samples from the joint error distribution of measured values translated into the Galactocentric position and velocity:
\begin{equation}  \label{eq:error_convolution_sum}
f\big( \boldsymbol w^\text{(obs)}_i \big) \approx \frac{1}{N_{\text{samp},i}}
\sum_{s=1}^{N_{\text{samp},i}} f\big( \boldsymbol w^{(s)}_i \big).
\end{equation}
To reduce the impact of the Poisson error in Monte Carlo integration, we use the same set of random samples for all model evaluations, following \citet{McMillan2013}. This approach can be used even in the case that some dimensions of data, such as line-of-sight velocity, are missing: in this case, we need to integrate over a flat distribution of possible velocities in a wide enough range (greater than the escape velocity). However, if the model DF varies widely across this range, the Monte Carlo sum will be dominated by only a few sampling points, thus increasing the discreteness noise; to mitigate this, a more sophisticated importance sampling strategy could be designed (e.g., as described in Section~3.4 of \citealt{Read2021} and Section~4.3.2 of \citealt{Hattori2021}). In our case, we restrict the input dataset to objects with all 6d phase-space coordinates known, and achieve sufficient accuracy with $N_\text{samp}=100$ for globular clusters and $N_\text{samp}=1000$ for satellites.

%%%%%%%%%%%
\subsection{Treatment of outliers}  \label{sec:outliers}

The modelling approach based on the Jeans theorem assumes that the tracer population is fully virialized; however, this might not be true for all Galactic satellites. Even a single non-virialized object with a significantly higher-than-average energy, such as Leo~I, may drive the mass estimate up by a few tens percent \citep[e.g.,][]{Watkins2010}. To mitigate this complication, we allow for a possibility that some objects may not belong to the main population, but instead are described by another DF of the ``unmixed'' population, which is an umbrella term for objects that are unbound, infalling for the first time, or have passed pericentre once but are on orbits with apocentres beyond the virial radius (``splashback'' orbits). Based on a large number of cosmological simulations, \citet{Li2020b} found that the velocities of infalling satellites at the moment of crossing the virial radius of the host halo are well described by a log-normal distribution:
\begin{equation}  \label{eq:velocity_distribution_unmixed}
p(v)\,\text{d}v = \frac{1}{\sqrt{2\pi}\,\sigma_\text{un}} \exp\left[-\frac{\ln^2(v/v_\text{un})}{2\sigma_\text{un}^2}\right] \frac{\text{d}v}{v},
\end{equation}
with $v_\text{un} = 1.2 v_\text{circ}(r_\text{vir}) = 1.2\sqrt{G M_\text{vir}/r_\text{vir}}$ and $\sigma_\text{un} = 0.2$ (dimensionless). The above expression is for the velocity magnitude, and the DF in the 6d phase space is $\displaystyle f(\boldsymbol{x},\boldsymbol{v})\big|_{|\boldsymbol x|=r_\text{vir}} = \rho_\text{un}(r_\text{vir})\, p\big(|\boldsymbol v|\big) \big/ (4\pi v^2)$, where $\rho_\text{un}(r_\text{vir})$ is the density of infalling objects at virial radius. Under the approximation of a stationary host galaxy potential, we may use the Jeans theorem to express this DF as a function of energy (an integral of motion) and evaluate it anywhere in space, not only at the virial radius:
\begin{equation}  \label{eq:df_unmixed}
f_\text{un}(E) = \frac{\displaystyle\rho_\text{un}(r_\text{vir})\, \exp\left[
-\frac{\ln^2\Big\{2\big[E-\Phi(r_\text{vir})\big]/v_\text{un}^2\Big\}}
{8\sigma_\text{un}^2}\right]}
{16\pi^{3/2}\,\sigma_\text{un}\,\big[E-\Phi(r_\text{vir})\big]^{3/2}} .
\end{equation}
Note that the DF is still defined as a probability distribution in the 6d phase space, not in energy space. For instance, the density generated by it at a radius $r$ is computed as
\begin{equation}
\begin{aligned}  \label{eq:density_unmixed}
\rho_\text{un}(r) &= \iiint f_\text{un} \big(E=\Phi(r)+v^2/2\big)\;\text{d}^3v \\
&= \int_0^\infty 4\pi\, v^2\,f_\text{un}\big(\Phi(r)+v^2/2\big)\;\text{d}v \\
&= \int_{\Phi(r_\text{vir})}^\infty 4\pi\,\sqrt{2\big[E-\Phi(r)\big]}\; f_\text{un}(E)\;\text{d}E .
\end{aligned}
\end{equation}
The density increases rather slowly towards small radii at $r<r_\text{vir}$ and decreases more rapidly at larger radii, but the integral of $\rho_\text{un}(r)$ over the entire space is infinite -- this is not unexpected, since the velocity distribution contains a tail of objects with $E>0$ that can travel anywhere. However, since our tracer population has limited extent, we only need to normalize $\rho_\text{un}$ to a unit total mass within some fiducial radius $r_\text{out}$, which we fix to 300~kpc (the most remote object in our sample is Leo~I at a Galactocentric distance $262\pm12$~kpc). For each choice of the MW potential, we compute the virial mass and radius, then evaluate the normalization factor $\rho_\text{un}(r_\text{vir})$ that produces a unit mass within $r_\text{out}$; this fully defines the DF of outliers. A suitable approximation for our range $0.8\le r_\text{vir}/r_\text{out}\le 1.2$ is $\rho_\text{un}(r_\text{vir}) = 0.81\,\frac{3}{4\pi\,r_\text{out}^3}\,(r_\text{vir}/r_\text{out})^{-1/2}$ almost independently of the potential. Note that we do not explicitly need $\rho_\text{un}(r)$ to compute the likelihood of an object in the model: it is given by $f_\text{un}(\boldsymbol x,\, \boldsymbol{v})$, which depends only on its energy in the given potential.

The possibly unmixed outliers are thus treated in a standard mixture model approach. We assume that any object may belong to either of the two populations described by their unit-normalized DFs: $f_\text{bound}\big(\boldsymbol J(\boldsymbol x,\,\boldsymbol v) \big)$ is the action-space DF of bound satellites, and $f_\text{un}\big(E(\boldsymbol x,\,\boldsymbol v)\big)$ is the phase-space DF of interlopers. Let $\eta$ be the overall fraction of the first population in the sample. The mixture DF 
\begin{equation}  \label{eq:df_mixture}
f_\text{mix} = \eta\, f_\text{bound} + (1-\eta)\, f_\text{un},
\end{equation}
evaluated at the phase-space coordinates of each object, now gives the likelihood of this object in the model. The posterior probability of belonging to the first population for $i$-th object is 
\begin{equation}  \label{eq:posterior_outlier_probability}
\eta_i = \frac{\eta\, f_\text{bound}(\boldsymbol x_i,\,\boldsymbol v_i)}
{\eta\, f_\text{bound}(\boldsymbol x_i,\,\boldsymbol v_i) + (1-\eta)\, f_\text{un}(\boldsymbol x_i,\,\boldsymbol v_i)},
\end{equation}
and although we have not specified the overall fraction $\eta$ in advance, it is easy to see that the likelihood is maximized when $\sum_{i=1}^{N_\text{obj}} \eta_i = N_\text{obj}\, \eta$. Indeed, taking the derivative of the total log-likelihood w.r.t.\ $\eta$ and equating it to zero, we find
\begin{equation}
\begin{aligned}
0 &= \frac{\partial}{\partial \eta} \sum\nolimits_{i=1}^{N_\text{obj}} \ln \big[ \eta\, f_{\text{bound},i} + (1-\eta)\, f_{\text{un},i} \big] \\
&= \sum\nolimits_{i=1}^{N_\text{obj}}  \frac{f_{\text{bound},i} - f_{\text{un},i}}{\eta\, f_{\text{bound},i} + (1-\eta)\, f_{\text{un},i}} \\
&= \frac{1}{\eta-1} \sum\nolimits_{i=1}^{N_\text{obj}} \frac{(\eta-1)\,f_{\text{bound},i} + (1-\eta)\,f_{\text{un},i}}{\eta\, f_{\text{bound},i} + (1-\eta)\, f_{\text{un},i}} \\
&= \frac{1}{\eta-1} \left[ N_\text{obj} - \sum\nolimits_{i=1}^{N_\text{obj}} \frac{f_{\text{bound},i}}{\eta\, f_{\text{bound},i} + (1-\eta)\, f_{\text{un},i}} \right] \\
&= \frac{1}{\eta(\eta-1)} \left[ \eta N_\text{obj} - \sum\nolimits_{i=1}^{N_\text{obj}} \eta_i \right].
\end{aligned}
\end{equation}
This simple equation for the auxiliary parameter $\eta$ is easy to solve for each choice of parameters defining the two DFs, thus $\eta$ is not a free parameter and does not increase the complexity of the model (recall that $f_\text{un}$ is fully specified by the host potential).

The addition of this extra ingredient in the model stabilizes it against outliers at very little extra cost. Of course, its performance depends on the validity of the assumption that outliers are reasonably well described by the DF (Equation~\ref{eq:df_unmixed}), but in practice it seems adequate. As a side benefit, the posterior probability of being an outlier (Equation~\ref{eq:posterior_outlier_probability}) can be computed for all objects in our sample and plotted against other model parameters (e.g., virial mass). We use the mixture DF only for the satellite population, since all globular clusters have safely negative energies for any reasonable potential.

%%%%%%%%%%%
\subsection{Monte Carlo simulations}  \label{sec:mcmc}

The above sections describe the steps for constructing a single dynamical model and evaluating its likelihood against the chosen tracer dataset(s). We then explore the parameter space of models with the Markov Chain Monte Carlo (MCMC) approach, using the \textsc{emcee} package \citep{ForemanMackey2013}. We evolve 50 walkers for a few thousand steps, monitoring the convergence by analyzing the evolution of parameters and the overall distribution of likelihoods in the chain, and take the last 500 steps to construct the posterior distribution of model parameters. In the most complete setup (two tracer populations, marginalizing over observational uncertainties and performing orbit rewinding for each choice of model parameters), the evaluation of a single model takes less than a second on a 32-core workstation, and a few times faster without rewinding. Thus the entire analysis pipeline runs in less than a day of wall-clock time.


%%%%%%%%
\section{Tests on mock datasets}  \label{sec:tests}

We test the performance of the method on several mock datasets constructed as follows.

%%%%%%%%%%%
\subsection{Variants of mock models}  \label{sec:mock_variants}

First we choose the structural parameters of the MW, using either a simplified spherical isotropic model with a mass profile closely resembling a realistic galaxy, or more complicated models containing a stellar disc and a dark halo; the latter could be spherical or non-spherical. The LMC is always represented by a spherical truncated NFW profile, with the mass and scale radius related by $r_\text{scale}\propto M_\text{LMC}^{0.6}$ (this keeps the mass profile in the inner part of the LMC nearly independent of the total mass and satisfies the observational constraints, as illustrated by Figure~3 in \citealt{Vasiliev2021b}). We then construct equilibrium models of both galaxies, using the Eddington inversion formula for the DF in a spherical case, or the Schwarzschild orbit-superposition method for non-spherical disc+halo models (both approaches are provided by \textsc{Agama}). The MW is sampled with $10^6$ particles and the LMC -- with $0.5\times10^6$, regardless of the actual mass ratio. In disc+halo MW models, 80\% particles are allocated to the halo component. We add a few thousand tracer particles to the MW snapshot, which are sampled from a particular DF in equilibrium with the total potential, and represent the globular cluster and satellite populations, with parameters chosen to closely mimic the observed datasets. Table~\ref{tab:mock_tracers} lists the parameters of the fiducial model, which we discuss in more detail below. We considered a few additional models with the same tracer populations, LMC masses between 1 and $2\times10^{11}\,M_\odot$, and lowering the MW halo mass or adding a stellar disc; the parameters of one such model fitted to the Sagittarius stream are listed in Table~1 of \citet{Vasiliev2021b}.

%%%%%%%%%%%%%
\begin{table}
\caption{Parameters of the fiducial mock dataset. First two columns define the tracer populations (globular clusters and satellite galaxies), which are the same in other mock datasets; third column defines the MW (spherical halo-only model), last column defines the LMC.
The density of all components follows a tapered double-power-law profile (Equation~\ref{eq:spheroid_density}), and the DF is given by the \citet{Cuddeford1991} inversion formula with the central anisotropy coefficient $\beta_0$ and anisotropy radius $r_\text{a}$.
}  \label{tab:mock_tracers}
\begin{tabular}{lrrrr}
& GC & sat & MW halo & LMC \\
\hline
total mass [$10^{12}\,M_\odot$] & \multicolumn{2}{c}{negligible} & 1.1 & 0.15 \\
scale radius $r_\text{scale}$ [kpc] & 5 & 100 & 5 & 10.8 \\
cutoff radius $r_\text{cut}$ [kpc] & \multicolumn{2}{c}{$\infty$} & 290 & 108 \\
cutoff strength $\xi$ & \multicolumn{2}{c}{n/a} & 2 & 2 \\
inner slope $\gamma$ & 0.0 & 0.5 & 1.0 & 1.0 \\
outer slope $\beta$ & 6.0 & 6.0 & 3.0 & 3.0 \\
transition steepness $\alpha$ & 0.5 & 2.0 & 0.5 & 1.0 \\
central anisotropy $\beta_0$ & 0.0 & $-0.4$ & 0.0 & 0.0 \\
anisotropy radius $r_\text{a}$ [kpc] & 25 & 200 & \multicolumn{2}{c}{\makebox[6mm]{}$\infty$} \\
\hline
\end{tabular}
\end{table}
%%%%%%%%%%%

We then run a conventional $N$-body simulation of the encounter between the MW and the LMC, using the code \textsc{gyrfalcON} \citep{Dehnen2000}. We choose the initial position and velocity of the LMC in such a way that it arrives at the observed point%
\footnote{We use the present-day position and velocity of the LMC determined by \citet{Luri2021} and references therein: $\alpha^\text{LMC}=81.28^\circ$, $\delta^\text{LMC}=-69.78^\circ$, $\mu_\alpha^\text{LMC}=1.858\pm0.02$~\masyr, $\mu_\delta^\text{LMC}=0.385\pm0.02$~\masyr, distance $D^\text{LMC}=49.5\pm0.5$~kpc, line-of-sight velocity $v_\text{los}^\text{LMC}=262.2\pm3.4$~\kms. \label{footnote:LMCcoords}}
after 2~Gyr of evolution. The initial guess for the starting point is given by integrating the equations of motion of two galaxies (Equation~\ref{eq:lmc_mw_orbit}), and then we iteratively refine it by running a suite of 6 simulations with slightly different initial conditions, computing the Jacobian of transformation between the start and end points, and using the Gauss--Newton method to update the initial point, as described in Section~3.2 of \citet{Vasiliev2021b}. The final LMC position and velocity matches the actual observations to $\sim0.1$~kpc and $\sim0.5$~\kms. We extract the trajectories of the MW and LMC centres from the $N$-body simulation (using centre-of-mass position and velocity iteratively determined from particles within the central 10~kpc for each galaxy), fit a smooth spline curve to the former and differentiate it twice to obtain the acceleration of the MW-centered non-inertial reference frame. The positions and velocities of tracer particles at the final snapshot are used to construct the ``present-day'' mock dataset, and we also use their initial coordinates to test the method on the unperturbed MW system.

%%%%%%%%%%%
\subsection{LMC orbit reconstruction}  \label{sec:test_lmc_orbit}

%%%%%%%%%%%%%%%
\begin{figure*}
\includegraphics{fig_lmc_orbit_reconstruction.pdf}
\caption{Reconstruction of LMC's past orbit with the approximate solution of two extended bodies' equations of motion. Top row show the three cartesian components of relative distance between the MW and LMC centres, bottom row shows the components of Milky Way's acceleration. Solid lines are taken from a live $N$-body simulation, dashed lines are the orbits reconstructed via Equation~\ref{eq:lmc_mw_orbit}. Thinner and thicker lines show two cases: $M_\text{LMC}=10^{11}\,M_\odot$ and $2\times10^{11}\,M_\odot$ respectively; the MW potential is the same in both cases. Overall, the approximate solution reproduces the actual orbits reasonably well over the relevant range of LMC masses, at least over the last Gyr when the effect of the LMC flyby is most significant, although the MW acceleration is slightly overestimated by the approximation.
}  \label{fig:lmc_orbit_reconstruction}
\end{figure*}
%%%%%%%%%%%%%

The first question to be explored quantitatively is the accuracy of the LMC orbit reconstruction with our orbit-rewinding procedure (Equation~\ref{eq:lmc_mw_orbit}). Figure~\ref{fig:lmc_orbit_reconstruction} shows the time evolution of three components of the relative separation between LMC and MW centres (top row) and three components of the Galactocentric reference frame acceleration (bottom row), for two simulations with the same MW model but LMC masses of $10^{11}$ and $2\times10^{11}\,M_\odot$. The only adjustable parameter in this rewinding procedure is the Coulomb logarithm $\ln\Lambda$. By examining these and a few other simulations, we settle on the following expression, which provides an adequate match to $N$-body orbits in all cases: $\ln\Lambda = \ln \big[ D_\text{LMC}(t) / \epsilon_\text{LMC} \big]$. Here $D_\text{LMC}(t)$ is the instantaneous distance between the LMC and the MW centre, and $\epsilon_\text{LMC}$ is the effective ``softening radius'', for which the value $2\times r_\text{scale}$ appears to be optimal (it varies between 17 and 26 kpc for the two limiting values of LMC mass we considered). This functional form was also used by \citet{Jethwa2016} in the same context, although they found that their $N$-body trajectories were better matched by a somewhat lower $\epsilon_\text{LMC}$ in the Coulomb logarithm.

As seen from the above figure, the LMC rewinding prescription adequately recovers the time evolution of the MW--LMC separation in $N$-body simulations, but overestimates the MW acceleration around the its peak by $\sim20$\% (40\%) for the $x$ ($y$) components. This could be attributed to the neglect of the developing deformation of the MW: we implicitly assume that the entire Galaxy is pulled towards the LMC with spatially uniform acceleration, but in reality the swinging motion of the inner MW towards the LMC is partly counteracted by the gravitational force from the outer MW halo, which reacts more slowly (see \citealt{Vasiliev2021c} for an in-depth discussion). Although not perfectly reconstructed, the $x$- and $y$-components of MW acceleration change sign during the interaction, and the accumulated changes in the MW reflex velocity are smaller than in the $z$-component ($\sim20$~\kms vs. $\sim 50$~\kms, see Figure~10 in \citealt{Vasiliev2021b}); the latter is the dominant cause of the kinematic asymmetry illustrated later in Figure~\ref{fig:deltavz}, and is recovered to within a few \kms.

It is also interesting to note that the effect of changing the LMC mass is sublinear, and that the more massive LMC actually needs to start at a smaller distance in order to arrive to the same present-day point. This contrasts a simple picture used in earlier studies, in which the back-reaction of the LMC on the MW would be neglected, and LMC would move in a fixed external potential subject to additional dynamical friction force: in this case, more massive LMC would lose more energy and would have to start at a less bound orbit (i.e. further out and with a higher inward velocity), as seen, for instance, in Figure~9 of \citet{Kallivayalil2013}. Our orbit-rewinding scheme follows the mutual forces and motion of both galaxies, and is thus able to reproduce this nonlinear effect, but will likely break down for even higher LMC masses. For this reason, we put an upper limit of $2.5\times10^{11}\,M_\odot$ for $M_\text{LMC}$ in the fit, but fortunately, the best-fit models do not hit this boundary.

%%%%%%%%%%%
\subsection{Orbit rewinding in the time-dependent potential}  \label{sec:test_orbit_rewinding}

%%%%%%%%%%%%%%
\begin{figure}
\includegraphics{fig_orbit_rewinding.pdf}
\caption{Illustration of the accuracy of orbit rewinding for several representative orbits. Black cross is the MW centre; black dots mark the present-day positions; blue solid lines show the trajectories of particles in the original $N$-body simulation; red dashed lines are orbits integrated back in time from the present-day snapshot in the evolving MW+LMC potential; and green short-dashed lines are the orbits integrated in a static MW potential. The latter are often very different from the true ones, at least in the outer parts of the Galaxy, while the rewinding in a time-dependent potential reproduces the original trajectories $\sim 5\times$ better.
}  \label{fig:orbit_rewinding}
\end{figure}

\begin{figure}
\includegraphics{fig_orbit_rewinding_energy.pdf}
\caption{Illustration of the accuracy of orbit rewinding. Shown are density plots of particle energy changes as a function of radius; contours are logarithmically spaced by factors of two in density. Blue contours show the changes in particle energies due to the LMC perturbation in the $N$-body simulation. Red contours show the difference between the energy of the reconstructed initial snapshot (rewinding orbits from the present-day state backward in time in the evolving analytic MW+LMC potential) and the actual initial energy of the same particles in the simulation. In the ideal case, this difference would be zero; in practice, it has some residual scatter, but much narrower than the actual energy changes. The imperfect reconstruction is likely caused by inaccuracy of the original $N$-body simulation (numerical heating in the inner few kpc, where the LMC plays no role) and the neglect of MW deformation in the orbit rewinding scheme at larger radii. 
}  \label{fig:orbit_rewinding_energy}
\end{figure}
%%%%%%%%%%%%

After confirming that the LMC trajectory can be reconstructed reasonably well, we now turn to the second part of the orbit rewinding scheme -- integrating the tracer orbits back in time from their present-day coordinates in an evolving potential composed of a moving LMC, fixed analytic potential of the MW, and time-dependent but spatially uniform acceleration of the Galactocentric reference frame. Figure~\ref{fig:orbit_rewinding} shows a few typical orbits: the original particle trajectories in the $N$-body simulation are reasonably well approximated by the time-dependent orbit rewinding, but much less well by orbits in a static MW potential alone. Although some phase differences between the original and the reconstructed orbit inevitably accumulate over time, these are unimportant for the purpose of DF fitting, which uses only the integrals of motion but not phases. The relative error in position grows approximately linearly with lookback time and is at the level $\sim 0.1$ (median) for the time-dependent orbit reconstruction, while being $\sim 5$ times higher for the static potential.

Figure~\ref{fig:orbit_rewinding_energy} presents a different view on the LMC perturbation, showing the energy changes of particles in the actual $N$-body simulation at different radii (blue contours). These changes are relatively small and likely caused by numerical relaxation effects in the inner part of the system, but are much more prominent beyond $20-30$~kpc (here the horizontal axis is not the instantaneous radius of a particle, but the radius of a circular orbit with the same energy, i.e., close to a time-averaged orbit size). The energy of individual particles may increase or decrease, but on average it increases more often (note the asymmetry of the 1d histogram of $\Delta E$ in the right panel). The orbit rewinding in a time-dependent potential recovers the initial particle energies fairly well, as shown by a much tighter and symmetric distribution of the difference between reconstructed and actual initial energies (the r.m.s.\ error in energy is $\sim 10^3\,\big[\text{km\,s}^{-1}\big]^2$ for the time-dependent reconstruction and $\sim 5$ times higher in the static potential). Since the most bound orbits are little affected by the LMC, we may save effort by performing rewinding only for orbits with energies corresponding to the radii of circular orbits $\ge 10$~kpc, safely including the entire region of significant perturbations shown by blue contours in that figure.

Having demonstrated the accuracy of the orbit rewinding scheme, we now scrutinize the performance of the DF fitting approach itself.

%%%%%%%%%%%
\subsection{Measurement of the MW potential}  \label{sec:test_potential_inference}

%%%%%%%%%%%%%%%
\begin{figure*}
\includegraphics{fig_vcirc_mock2000.pdf}
\includegraphics{fig_vcirc_mock200.pdf}
\caption{Tests of the method on mock datasets, described in Section~\ref{sec:mock_variants}: top row shows large datasets (1000 objects in each), bottom row -- realistic sizes (150 globular clusters and 50 satellites). Each panel shows the inferred circular-velocity profiles $v_\text{circ}(r) \equiv \sqrt{G\,M(<r)/r}$ for two mock catalogues, representing globular clusters (red) and satellites (blue), and in the bottom row, the combined sample (green); the true profile is plotted by a black dashed curve. Solid lines show the medians, while darker and lighter shaded regions indicate 16/84 and 2.3/97.7 percentiles (``1$\sigma$'' and ``2$\sigma$'' confidence intervals). Naturally, the constraints are tightest in different radial ranges for the two datasets: around $5-10$~kpc for clusters and 100~kpc for dSph. \textbf{Left panel} shows the initial, unperturbed system before the arrival of the LMC, which illustrates excellent performance of the method in the ideal case. \textbf{Centre panel} shows the fit for the present-day snapshot perturbed by the LMC, but without accounting for it in the model, which leads to an overestimate of enclosed mass beyond a few tens kpc. \textbf{Right panel} demonstrates that with the orbit rewinding scheme in place, the model is able to recover the true potential.
}  \label{fig:vcirc_mock}
\end{figure*}
%%%%%%%%%%%%%

We ran the fitting routine for several choices of the MW potential and LMC mass, which all displayed similar performance. In this section we discuss the fiducial case of a $1.5\times10^{11}\,M_\odot$ LMC in a spherical MW potential with the total mass $1.1\times10^{12}\,M_\odot$, using either the globular cluster sample, the satellite sample, or both. For each mock dataset, we consider three scenarios: (1) initial, unperturbed MW potential; (2) present-day MW affected by the LMC, but ignoring it during the fit; (3) present-day MW with orbit rewinding compensating the LMC perturbation. 

Figure~\ref{fig:vcirc_mock}, top row, shows the results of these experiments on the full sample of tracers (1000 objects in each dataset). These ideal conditions illustrate that each of the two datasets produces tight constraints around the median radius of tracer points ($\sim 5$~kpc for globular clusters and $\sim 100$~kpc for satellites), and uncertainties increase considerably outside the range spanned by tracers, which is natural to expect if the model for the potential is flexible enough. In the unperturbed case, the mass distribution is recovered very well, and applying the vanilla DF-fitting method to a perturbed snapshot overestimates the mass at radii $r\gtrsim 30$~kpc by $\sim20-30$\%, in agreement with \citet{Erkal2020b}. Most interestingly, the addition of the orbit-rewinding step brings the inferred mass profile quite close to truth, which is very encouraging. We note that the inference is still slightly biased at $r\lesssim 10$~kpc, but at these radii the MW potential is better measured with alternative methods. 

As expected, the uncertainties grow considerably when we reduce the tracer sample to a more realistic size: 150 clusters and 50 satellites. The inferred mass profile becomes rather sensitive to the choice of the random sample (i.e., to the Poisson noise), although still remains statistically consistent between different samples due to large uncertainty intervals. The bottom row of Figure~\ref{fig:vcirc_mock} shows one of the least biased samples, illustrating the same trends as for the large sample test; in other cases, the median inferred profiles may shift up or down by a few tens percent, but the relative effect of adding the LMC perturbation and accounting for it during the fit remains the same. When we perform the fit simultaneously for both datasets, the uncertainties ``take the best of both worlds'' and become rather small in the entire radial range, also stabilizing the results against random fluctuations caused by a small sample size.

Finally, we consider the effect of adding measurement errors: 0.1 for the distance modulus (4.7\% relative distance error), 0.05~\masyr for PM, and 2~\kms for the line-of-sight velocity. These are relatively small errors, but if we ignore them in the fit, the inferred mass is biased up by 12\% at 100~kpc and by 30\% at 200~kpc. On the other hand, taking them into account restores the unbiased inference at large radii, whine increasing the uncertainty intervals only slightly (by $10-20$\%): in other words, the finite sample size (Poisson noise) is far more important than measurement errors. The green curves in the bottom panel show the fits to the combined dataset with measurement errors, which still have fairly small uncertainty intervals and recover the enclosed mass profile to within $\sim10\%$ at all radii.

The average log-likelihood of models without orbit rewinding (i.e., with biased potential inference) is lower than models with rewinding by $\Delta\ln\mathcal L\sim 5-10$ for the cluster sample and $10-15$ for the satellite sample, i.e., the non-zero LMC mass is preferred at high significance. In fact, the relative uncertainty on $M_\text{LMC}$ is at the level of 30\% when using only satellites, and twice larger when using only clusters; in the combined fit, satellites dominate. The best-fit LMC mass agrees with the true value within uncertainties.

The above results were obtained for a spherical MW model and using a \texttt{QuasiSpherical} DF of tracers. We also ran the analysis pipeline on the same mock dataset, but using \texttt{DoublePowerLaw} action-based DF (Equation~\ref{eq:DPL_DF}) in the fit (thus different from the actual DF used to construct the sample), obtaining very similar constraints for the potential and recovering the kinematic profiles of the tracer populations. Finally, additional experiments were run with non-spherical MW models taken from \citet{Vasiliev2021b} and \texttt{DoublePowerLaw} tracer DFs, again recovering the potential with good accuracy (typically within 10\% at all radii).


%%%%%%%%
\section{Application to the Milky Way}  \label{sec:MW}

%%%%%%%%%%%
\subsection{Input data}  \label{sec:input_catalogues}

We use two classes of dynamical tracers: globular clusters and satellite galaxies. The former are more likely to be orbiting in the MW for many Gyr and thus are expected to be well mixed and satisfy the assumptions of equilibrium dynamical models (in absence of the LMC), but are mostly concentrated in the inner $10-20$~kpc of the Galaxy, where the LMC does not cause a significant perturbation. Satellite galaxies, on the other hand, are located much further out, at a typical distance of $\sim100$~kpc, but present an additional complication as some of them might have been accreted into the MW system only recently and thus not yet fully virialized. This certainly applies to objects at distances comparable to or exceeding the virial radius ($250-300$~kpc), such as Eridanus~II, Leo~T or Phoenix, which are still on the way towards the MW, but also possibly to galaxies that have already passed their pericentre, but still have high energy, such as Leo~I. The latter object is especially troublesome, with its distance of $\sim250$~kpc and heliocentric reflex-corrected (``Galactic standard of rest'') line-of-sight velocity of 167~\kms making it difficult to reconcile with being a long-term bound satellite, unless the MW mass exceeds $\sim1.5\times10^{12}\,M_\odot$ \citep[e.g.,][]{Kulessa1992, BoylanKolchin2013}.

For the globular cluster population, we use the most recent distance, PM and line-of-sight velocity measurements from \citet{Baumgardt2021}, \citet{Vasiliev2021a} and \citet{Baumgardt2019}, respectively. The entire catalogue contains 170 objects, but 9 of them (all at distances below 50~kpc) lack line-of-sight velocity measurements. Although datapoints with missing dimensions can still be used in model fitting, they add very little to model constraints, so we exclude them for simplicity. Furthermore, we exclude the 7 clusters associated with the Sagittarius dSph and its stream: NGC~6715 (M~54), Arp~2, Terzan~7, Terzan~8, Pal~12, Whiting~1, and NGC~2419. 

For the satellite population, we primarily use the most complete compilation of PM measurements and other properties by \citet[tables B1 and B2]{Battaglia2021}, but also rerun some of our fits with alternative PM catalogues of \citet{McConnachie2020} and \citet{Li2021}, which are also based on \Gaia~EDR3: the results were robust w.r.t.\ the choice of the PM dataset. These catalogues are not limited to the MW satellites, but contain other galaxies from the Local Group; we limit the sample to galaxies within 300~kpc, and likewise exclude objects without line-of-sight velocity measurements (although we examine their possible orbits in the potential models determined in our fit). We also remove the Small Magellanic Cloud (SMC) and other galaxies possibly associated with the LMC, as discussed by \citet{Battaglia2021}: Carina~II, Carina~III, Horologium~I, Horologium~II, Hydrus~I, Phoenix~II, Reticulum~II; again, we revisit their orbits after running the fits. Finally, we exclude a few objects that have unreliable PM measurements or are possibly unbound: Columba~I, Pegasus~III, Pisces~II, Reticulum~III. This leaves us with a sample of 36 galaxies: Antlia~II, Aquarius~II, Bo\"otes~I, Bo\"otes~II, Bo\"otes~III, Canes Venatici~I, Canes Venatici~II, Carina, Coma Berenices, Crater~II, Draco, Draco~II, Fornax, Grus~I, Grus~II, Hercules, Hydra~II, Leo~I, Leo~II, Leo~IV, Leo~V, Sagittarius, Sagittarius~II, Sculptor, Segue~1, Segue~2, Sextans, Triangulum~II, Tucana~II, Tucana~III, Tucana~IV, Tucana~V, Ursa Major~I, Ursa Major~II, Ursa Minor, Willman~1.

We propagate the measurement errors into Galactocentric positions and velocities, drawing 100 (for globular clusters) or 1000 (for satellites) Gaussian random samples from the quoted uncertainties, imposing a lower limit of 0.02~\masyr for the PM error, and 0.05 mag (for clusters) or 0.1 mag (for satellites) on the distance modulus. The results are fairly insensitive to the number of samples, since the observational uncertainties are typically quite small, with a few exceptions discussed later. However, for objects lacking line-of-sight velocities, we would have to use a larger number of samples covering a wide range of possible values, or design a more sophisticated importance sampling scheme, as in \citet{Read2021}, which is the main reason for excluding them. We also add uncertainties on the Solar position and velocity: $x_\odot=-8.12\pm0.1$~kpc, $v_\odot=\{12.9\pm1,\, 245.6\pm3,\, 7.8\pm1\}$~\kms \citep{Drimmel2018}.

%%%%%%%%%%%
\subsection{Model specifications}  \label{sec:MW_model}

We first consider the two datasets independently, running separate fits for globular clusters and satellites. In each case, we use the \texttt{DoublePowerLaw} DF family for the tracers (Equation~\ref{eq:DPL_DF}), which has 9 free parameters; simpler \texttt{QuasiSpherical} DF models are less flexible and produce noticeably worse fits (with a difference in log-likelihood $\Delta \ln \mathcal L \gtrsim 5$ for dSph alone), justifying the extra 3 parameters responsible for a non-spherical tracer density distribution and rotation. The MW halo is represented by a spherical \citet{Zhao1996} model (\texttt{Spheroid} density profile, Equation~\ref{eq:spheroid_density}) with 5 free parameters: inner and outer slopes $\gamma$, $\beta$, transition steepness $\alpha$, scale radius $r_0$ and corresponding density $\rho_0$. The outer slope $\beta$ is allowed to be shallower than 3 (the value for the NFW profile), which formally corresponds to an infinite mass (we do not impose exponential truncation in these models); as long as $\beta>2$, the potential still tends to zero at infinity, and the entire procedure remains valid.
In contrast to many other studies \citep[e.g.,][]{Eadie2019,Deason2021}, we do not artificially narrow down the prior range on any of the model parameters based on some cosmological simulations, but consider widest possible range of models and let the data speak for itself. However, to explore the sensitivity of results to the chosen parametrization of the halo density, we additionally ran another series of models with the exponential \citet{Einasto1965} profile, which has 3 free parameters: cutoff radius $r_\text{cut}$, steepness $\xi$ and normalization $\rho_0$.
We experimented with non-spherical, constant-axis-ratio models, but found that the axis ratio $q=z/x$, although not well constrained, prefers values closer to unity. Since the St\"ackel fudge is designed only for oblate or spherical potentials, we could not check if prolate models would provide a better fit%
\footnote{Note that \citet{Posti2019} also used St\"ackel fudge in conjunction with action-based DF for the globular cluster population, but did not impose the restriction $q\le 1$ on the axis ratio; thus their inferred value of $q\simeq1.3$ violated the applicability of the method and cannot be trusted.}.

We fix the parameters of the baryonic components to the values taken from the \citet{McMillan2017} best-fit potential. The bulge is a flattened truncated power-law model with $\gamma=\beta=1.8$, $\alpha=1$, $r_\text{cut}=2.1$~kpc, axis ratio $q=0.5$ and total mass $M=0.9\times10^{10}\,M_\odot$. We use a single exponential disc component in place of two stellar and two gas discs to simplify and speed up computations; it has a total mass $5.6\times10^{10}\,M_\odot$, scale radius 3~kpc and scale height 0.3~kpc. In principle, one could allow these parameters to vary, but doing so introduces degeneracy in the total potential, which should be balanced by additional observational constraints, e.g., the dependence of vertical force on $z$ inferred from stellar kinematics in the Solar neighbourhood. Since our primary goal is to explore the potential at large scales, and because the axisymmetric assumption of the St\"ackel fudge is violated in the inner Galaxy anyway, we ignore these complications, and instead pin the circular-velocity curve at the Solar radius to $235\pm10$~\kms \citep{McMillan2017}, adding a corresponding term to the likelihood function (although the results change very little if we remove this constraint). This model setup is nearly identical to the one used in \citet{Vasiliev2019b}. We also explored the effect of adding measurements of the circular velocity from \citet{Eilers2019}: these provide very tight constraints on the circular velocity profile at radii $5\le R \le 25$~kpc, but otherwise little affect the inferred mass distribution at large radii.

The LMC mass is another free parameter with a log-flat prior and an upper limit of $3\times10^{11}\,M_\odot$, and we also consider models without LMC separately, since these run a few times faster in absence of the orbit rewinding step. The uncertainty on the present-day LMC position and velocity (see footnote \ref{footnote:LMCcoords}) is propagated into the fit, adding four more parameters ($D^\text{LMC}$, $v_\text{los}^\text{LMC}$, $\mu_\alpha^\text{LMC}$, $\mu_\delta^\text{LMC}$) with Gaussian priors centered on the measured values.
We then perform a joint fit of both population (each one described by its own DF, but with a common potential), which has 28 or 26 parameters in total (5 or 3 for the Zhao or Einasto halo potential, 1 for the LMC mass, 4 for its position/velocity, and 9 for each of the two tracer DFs).
The large number of free parameters in the model may seem excessive, and indeed not all of them are well constrained (e.g., $\alpha, \beta$ and $\gamma$ span almost the entire allowed ranges $0.2-4$, $2.1-6$ and $0-2$, respectively). However, these parameters do not have an intuitively clear physical interpretation by themselves; what matters is the overall mass distribution (for the potential) and the density and velocity dispersion profiles (for the tracer DF) that they produce. We illustrate the relation between fitted potential parameters and the enclosed mass profiles in the \hyperref[sec:covplots]{supplementary material}.

Globular clusters are not expected to provide strong constraints on the Galactic potential in the outer part (beyond $\sim100$~kpc), while satellites likewise cannot constrain the inner Galaxy well. On the other hand, the combined fit, performed here for the first time, is able to put relatively tight limits on the mass distribution in the wide range of Galactocentric distances, $10-200$~kpc. We quantify it by the enclosed mass $M(<r)$ within spherical radii $r=50$, 100 and 200~kpc, or equivalently by the circular-velocity curve in the equatorial plane: $v_\text{circ}(R)\equiv \sqrt{R\,\partial\Phi/\partial R}$ (note that in a non-spherical potential, $v_\text{circ}(r)\ne \sqrt{G\,M(<r)/r}$, but the difference between them is fairly small at large radii). To facilitate comparison with other studies, we also compute the virial mass $M_\text{vir}$ and virial radius $r_\text{vir}$; however, since we do not impose a fixed functional form such as the NFW profile, the extrapolated virial mass has a much larger uncertainty than the enclosed mass within smaller radii constrained by the tracer population kinematics.


%%%%%%%%%%%%%%%%%
\section{Results}  \label{sec:results}

\subsection{MW potential and LMC mass}  \label{sec:mw_potential}

%%%%%%%%%%%%%%%
\begin{figure*}
\includegraphics{fig_mwpot.pdf}
\caption{Results of DF fits with two tracer populations (clusters and satellites), taking into account the LMC perturbation. \textbf{Left panel} shows the enclosed mass profile, \textbf{centre panel} -- the corresponding circular velocity profile, and \textbf{right panel} -- the past LMC trajectory (distance from MW centre). Median values are plotted by black solid lines, dark and light-shaded regions show the 16/84 and 2.3/97.7 percentiles. For comparison, red dashed lines and red-shaded region show the best-fit potential and an ensemble of plausible potentials from \citet{McMillan2017}, the best-fit potential being very close to the median potential in our series of models without the LMC. Likewise, blue short-dashed lines and blue-shaded region show the median and the ensemble of potentials from the Tango simulation \citep{Vasiliev2021b}, which was fit to the Sagittarius stream in the presence of the LMC.
}  \label{fig:mwpot}
\end{figure*}

%%%%%%%%%%%%%
\begin{table}
\caption{Constraints on the MW mass profile at different radii from various combinations of tracers (globular clusters or satellites), with or without accounting for the LMC perturbation. We list the enclosed mass at 50, 100 and 200 kpc, as well as the virial mass (for the overdensity $\Delta=102$ w.r.t.\ the cosmological matter density). The last two rows are our preferred values from both datasets with orbit rewinding, using either the baseline Zhao halo profile or the alternative Einasto profile. Distances are given in kpc and masses -- in $10^{12}\,M_\odot$, quoted as median and 16/84 percentiles.
}  \label{tab:MWpotential}
\begin{tabular}{lp{11mm}p{11mm}p{11mm}p{11mm}}
& \makebox{$M(<50)$} & \makebox{$M(<100)$} & \makebox{$M(<200)$} & $M_\text{vir}$ \\
\hline
GC, no LMC             & $0.51^{+0.12}_{-0.09}$ & $0.79^{+0.33}_{-0.20}$ & $1.08^{+0.69}_{-0.39}$ & $1.2^{+1.0}_{-0.5}$ \\[2mm]
sat, no LMC            & $0.50^{+0.07}_{-0.07}$ & $0.85^{+0.12}_{-0.10}$ & $1.31^{+0.29}_{-0.20}$ & $1.6^{+0.6}_{-0.4}$ \\[2mm]
both, no LMC           & $0.54^{+0.07}_{-0.06}$ & $0.85^{+0.11}_{-0.09}$ & $1.17^{+0.27}_{-0.23}$ & $1.3^{+0.6}_{-0.3}$ \\[2mm]
%Einasto, no LMC       & $0.57^{+0.07}_{-0.06}$ & $0.86^{+0.13}_{-0.09}$ & $1.06^{+0.25}_{-0.21}$ & $1.1^{+0.4}_{-0.2}$ \\
{\bf both w/LMC}       & $0.45^{+0.04}_{-0.04}$ & $0.73^{+0.09}_{-0.08}$ & $1.10^{+0.27}_{-0.22}$ & $1.3^{+0.6}_{-0.4}$ \\[2mm]
%{\bf --"-- excl.2 sat} & $0.45^{+0.04}_{-0.04}$ & $0.69^{+0.08}_{-0.06}$ & $0.97^{+0.22}_{-0.16}$ & $1.1^{+0.4}_{-0.2}$ \\[2mm]
{\bf --"--, Einasto}   & $0.46^{+0.05}_{-0.04}$ & $0.74^{+0.09}_{-0.07}$ & $1.00^{+0.23}_{-0.17}$ & $1.1^{+0.4}_{-0.3}$ \\
\hline
\end{tabular}
\end{table}
%%%%%%%%%%%

We first perform separate fits for the globular cluster and satellite datasets, either neglecting the LMC perturbation or performing the orbit rewinding to compensate for it.
The MW potential inferred from the cluster population without taking LMC into account is very similar to the results of \citet{Vasiliev2019b}, which used essentially the same method but with lower-precision data from \Gaia DR2, confirming that the accuracy of observations is no longer a limiting factor. This potential is very close to the best-fit model from \citet{McMillan2017} out to a distance 100~kpc, beyond which the constraints are very weak due to absence of tracers.
The enclosed mass at 100~kpc is $0.79^{+0.33}_{-0.20}\times10^{12}\,M_\odot$ (compared to $0.85^{+0.33}_{-0.20}\times10^{12}\,M_\odot$ in the previous analysis). The fit to the satellite population provides tighter constraints at large distances, despite the smaller overall number of tracers (though we use a prior on the circular velocity at 8~kpc to pin the potential in the inner MW, which have no satellite tracers); the enclosed mass at 100~kpc is found to be $0.86^{+0.11}_{-0.09}\times10^{12}\,M_\odot$, and the median $v_\text{circ}(r)$ has a very similar profile for both datasets. Given the agreement between the two datasets, it makes sense to fit them simultaneously; unsurprisingly, the combined fit produces somewhat tighter constraints at all radii.

If we now turn on the orbit rewinding, the inferred MW mass decreases by $\sim 15\%$ at $r\gtrsim 100$~kpc, as expected from the test runs. Again the profiles are reasonably consistent between the two populations, and the combined fit to both yields the enclosed mass $0.73^{+0.09}_{-0.07}\times10^{12}\,M_\odot$ within 100~kpc; the values at other radii and for other combinations of parameters are listed in Table~\ref{tab:MWpotential}. The enclosed mass and circular velocity profiles for the fiducial series of models with LMC are shown in the left and centre panels of Figure~\ref{fig:mwpot}.

As mentioned earlier, the flexibility of the Zhao density profile means that not all of its parameters can be usefully constrained by observations, and consequently the extrapolation of the halo density profile to larger radii (and thus the inferred virial mass) depend on the allowed range of shape parameters $\alpha$ and $\beta$. Appendix~\ref{sec:covplots} shows the covariance plots between the potential parameters used in the fit and the enclosed masses at different radii (50, 100, 200 kpc and $M_\text{vir}$). We find that the outer slope $\beta$ most significantly affects the virial mass, and to a slightly lesser extent $M(<200)$, with $M_\text{vir}$ at the lower end of the allowed range of $\beta$ (2.1) being 0.1 dex (i.e., 25\%) higher than at $\beta=3$. Nevertheless, the mean likelihoods of models restricted to the range $\beta\ge 3$ is the same as for the entire range of $\beta$; thus we cannot say that there is any real preference for lower $\beta$ warranted by the data. We also examined an alternative family of Einasto profiles and found that they generally have more steeply falling density at large radii, and consequently lower virial masses (but still compatible with Zhao models within uncertainties). We conclude that in absence of more selective priors on the halo density profile, we cannot constrain the total MW mass to better than a factor of two; however, the enclosed mass at $r\lesssim 200$~kpc is better constrained for any choice of halo model, and exhibits a systematic difference of $\sim 15\%$ between the models with and without the LMC.

The median circular velocity curve in the fiducial Zhao series of models lies roughly halfway between the commonly used ``best-fit model'' for the MW potential by \citet{McMillan2017} and the potential from the ``Tango simulation'' \citep{Vasiliev2021b}, which was fit to the properties of the Sagittarius stream in the presence of the LMC perturbation. Both reference models have considerable uncertainties, which are often overlooked in the case of McMillan's potential. The difference between these two is more pronounced at large radii, and results in a qualitative difference in the inferred past orbit of the LMC (right panel of Figure~\ref{fig:mwpot}): while in the Tango simulations the LMC was nearly always on the first approach to the MW, in the heavier McMillan's potential its past trajectory would have an apocentre distance $\lesssim 200$~kpc and a period $\sim 3$~Gyr. The LMC orbits in our series of fits typically have a longer period (4--6~Gyr) and larger apocentre distance, but $\sim 90\%$ of them are still bound to the MW (i.e., had at least one additional pericentre passage in the last 7~Gyr).
However, there is a strong evidence that the LMC is on its first encounter with the MW \citep[e.g.,][and references therein]{Kallivayalil2013}. This argument alone implies a relatively light MW, on the lower end of our inferred mass range, though we caution that these models do not take into account that the MW mass was likely lower in the past, resulting in a less bound LMC trajectory for the same present mass (see Figure~11 in \citealt{Kallivayalil2013}). The gradual increase of the MW mass due to cosmological accretion of matter over many Gyr would not invalidate the DF fitting approach, since the orbital actions are conserved under slow variation of potential. Thus it is possible that the current MW mass lies in the range preferred by our fits, but the LMC is still on its first approach.

Our inferred MW mass profile agrees within uncertainties with various recent estimates (see \citealt{Wang2020} for a compilation of results). 
Before \Gaia DR2, very few satellites had PM measurements, and consequently the mass estimates varied widely between studies or even within a single study, but using different tracer subsets. For instance, \citet{Watkins2010}, using a power-law mass estimator, found $M(<300) = (0.9\pm 0.3) \times 10^{12}\,M_\odot$ using only line-of-sight velocities and assuming isotropy, or $(0.7-3.4) \times 10^{12}\,M_\odot$ under more general assumptions about anisotropy, while  the sample of 8 satellites (including LMC and SMC) with measured PM indicated a strong tangential anisotropy and hence implied $M\gtrsim 2 \times 10^{12}\,M_\odot$. The more recent PM measurements instead suggest only a mild tangential anisotropy for satellites, as discussed in the next section. With \Gaia DR2 and \textit{HST} measurements of satellite PM, and using the same power-law mass estimator, \citet{Fritz2020} measured $M_\text{vir}=(1.5\pm0.4) \times 10^{12}\,M_\odot$, or $M(<100)=(0.8\pm0.2) \times 10^{12}\,M_\odot$. The choice of method may significantly affect the result: for instance, \citet{Eadie2019}, using the same catalogue of globular clusters as \citet{Vasiliev2019b}, but employing a variant of power-law mass estimator, determined $M(<100)=0.53^{+0.21}_{-0.12}\times10^{12}\,M_\odot$, almost 40\% lower than the latter study. Recently \citet{Slizewski2021} applied the method of \citet{Eadie2019} to the satellite tracers (still using PM from \Gaia DR2) and found $M(<100)=(0.88\pm0.1)\times10^{12}\,M_\odot$. On the other hand, \citet{Wang2021} used a nearly identical method to \citet{Vasiliev2019b} with the updated \Gaia EDR3-based PM measurements of globular clusters, and obtained a rather low virial mass $0.83^{+0.36}_{-0.21}\times10^{12}\,M_\odot$ (quoted for the overdensity of 200; to compare with our definition, these values should be increased by $\sim 16\%$) when using our PM measurements with the Zhao density profile, and even considerably smaller when using the Einasto profile. They additionally imposed tight constraints on the circular velocity at $5<r<25$~kpc taken from \citet{Eilers2019}, which are based on kinematics of stars in the Galactic disc. We repeated our analysis while imposing the same prior on the inner circular-velocity curve and got very similar results to \citet{Wang2021} when using only globular clusters; however, the fits to the combined dataset of clusters and satellites instead yielded higher virial masses, more consistent with our default setup (which does not use additional priors from stellar kinematics). As we do not expect that clusters alone can provide useful constraints for the enclosed mass at $r\gtrsim 100$~kpc, we tend to trust the combined fits more, but note that even these do not tightly constrain the virial mass without additional priors on the behaviour of halo density at large radii.

Since the assumption of virial equilibrium may be questionable for the satellite population, numerous studies instead estimated the MW mass by comparing the satellite distribution with cosmological simulations: in particular, \citet{Cautun2020} find $M(<100) = 0.65^{+0.08}_{-0.06}\times10^{12}\,M_\odot$, and \citet{Li2020a} find $M(<100) = (0.72\pm0.1) \times10^{12}\,M_\odot$; these values are well compatible with our measurements. 
The MW mass profile may also be constrained by other dynamical tracers such as distant halo stars or stellar streams. A recent study of \citet{Shen2021}, using $\sim 170$ stars from the H3 survey and the \citet{Eadie2019} modelling method, measured $M(<100) = (0.69\pm 0.05)\times10^{12}\,M_\odot$, but noted that the simple power-law model cannot simultaneously fit well the density and velocity distribution of tracers. With a compilation of $\sim500$ MW halo stars from several surveys and again using a power-law mass estimator, \citet{Deason2021} found $M(<100) = 0.61 \pm 0.03\,\text{(stat.)} \pm 0.12\,\text{(sys.)} \times 10^{12}\,M_\odot$, somewhat lower than we measure. Interestingly, although they approximately accounted for the LMC influence by adding a velocity offset to the measured values, this had very little effect on the results; their systematic uncertainty rather reflects the scatter in assembly histories of galaxies in cosmological simulations.  Finally, we note that the ``Tango'' MW+LMC model fitted to the Sagittarius stream \citep{Vasiliev2021b} also has a lower mass: $M(<100) = (0.56 \pm 0.04) \times 10^{12}\,M_\odot$. Nevertheless, these values are still within the 95\% confidence interval of models in this study (in particular, if we run our fits with a fixed Tango potential, varying only the parameters of tracer DFs, the log-likelihood is decreased by $\Delta\ln\mathcal L\simeq 3$).

%%%%%%%%%%%%%%
\begin{figure}
\includegraphics{fig_mass.pdf}
\caption{
Posterior distribution of the enclosed MW mass at 100~kpc in models without LMC (blue) and with LMC rewinding (red), and the LMC mass in the latter series of models (green).
}  \label{fig:mass}
\end{figure}

\begin{figure}
\includegraphics{fig_deltavz.pdf}
\caption{
Illustration of the significant kinematic asymmetry in the satellite distribution induced by the LMC. Blue dots show the $z$-component of velocity plotted against Galactocentric distance for the 36 satellites selected for the fit; three quarters of them have $v_z>0$, with a mean value of 60~\kms for the entire population (see also Figure~6 in \citealt{Erkal2020b}). Red crosses show the same quantities that these object would have in absence of the LMC perturbation, obtained by integrating the present-day positions and velocities backward in time for 2~Gyr in the time-dependent MW+LMC potential and then integrating them forward in the static MW-only potential. Almost all galaxies would have had lower $v_z$ in this case, bringing the overall distribution closer to symmetry (compare the cumulative distributions in the right panel) with a mean $v_z$ of only 20~\kms.
}  \label{fig:deltavz}
\end{figure}
%%%%%%%%%%%%

Although the inferred MW mass profile and the virial mass in models with or without LMC is compatible within uncertainties, the former produce materially better fits. The difference in log-likelihood is $\Delta\ln\mathcal L\simeq 12$, split roughly equally between clusters and satellites: in other words, a nonzero LMC mass is preferred with high significance. Figure~\ref{fig:mass} shows the posterior distribution of LMC mass values, which lies between $10^{11}$ and $2\times10^{11}\,M_\odot$ -- same range as inferred from various independent pieces of evidence in recent works, e.g., perturbations to stellar streams \citep{Erkal2019,Vasiliev2021b,Shipp2021}, the census of its satellites \citep{Erkal2020a,Battaglia2021}, the requirement to gravitationally bind the SMC \citep{Kallivayalil2013}, and finally, the density and kinematic asymmetries in the MW stellar halo \citep{Erkal2021,Petersen2021,Conroy2021}. The physical reason why our fits are able to constrain the LMC mass is illustrated by Figure~\ref{fig:deltavz}, which shows the measured $z$-component of velocities of satellite galaxies (for simplicity, without uncertainties). As clear from this plot and from a similar Figure~6 in \citet{Erkal2020b}, there is a significant kinematic asymmetry: the mean velocity is $\sim 60$~\kms, with three quarters of objects having positive $v_z$. If we rewind their orbits back in time in the presence of the LMC perturbation, and then integrate forward up to present time in a static MW potential, almost all of them would have lower $v_z$, decreasing the mean value threefold. Since the DF fitting (or any other method based on the equilibrium assumption) implicitly assumes a uniform distribution in orbital phases, it naturally returns a higher likelihood for a more symmetric distribution of $v_z$, and since the dispersion of values also decreases, so does the inferred MW mass.

%%%%%%%%%%%
\subsection{Properties of tracer populations}  \label{sec:satellite_orbits}

%%%%%%%%%%%%%%%
\begin{figure*}
\includegraphics{fig_tracerdf.pdf}
\caption{Properties of the two tracer populations: globular clusters (solid lines) and satellites (dashed lines). \textbf{Left panel} shows the sphericalized 3d density multiplied by $r^3$, i.e., the number of objects per logarithmic interval of radii (normalized to unity for each population; the actual number of objects used in the fit is 154 clusters and 36 satellites). \textbf{Centre panel} shows the three components of velocity dispersion in spherical coordinates, and the mean rotation velocity. \textbf{Right panel} shows the velocity anisotropy coefficient $\beta \equiv 1 - (\sigma_\theta^2 + \sigma_\phi^2) / (2 \sigma_r^2)$ and the axis ratio $z/R$ of the axisymmetric density profile: clusters are flattened in the central part of the Galaxy and more spherical in the outer part, while satellites have a slightly prolate distribution overall. Lines show the median and shaded regions -- 16/84 percentiles in the MCMC ensemble of models.
}  \label{fig:tracerdf}
\end{figure*}

\begin{figure*}
\includegraphics{fig_orbital_poles_nolmc.pdf}\qquad\quad
\includegraphics{fig_orbital_poles_lmc.pdf}
\caption{Orbital poles of selected satellite galaxies (excluding the LMC and its likely satellites), shown on the celestial sphere in the same coordinates as Figure~3 in \citet{Fritz2018} or Figure~2 in \citet{Li2021}. Each object is shown by a cloud of points representing its posterior distribution of phase-space coordinates, with distance uncertainty usually the dominant one (hence the clouds are stretched along great circles). \textbf{Left panel} shows the actual present-day measurements, \textbf{right panel} -- the orientation of orbital poles that the objects would have if not perturbed by the LMC. The angular momentum direction of the MW disc is at the bottom of this sphere, and the largest concentration of orbital poles is around the VPOS, marked by a gray circle in the centre, which is roughly orthogonal to the MW disc.
}  \label{fig:orbital_poles}
\end{figure*}

As a by-product of our fits, we can also examine the properties of the tracer DFs manifested as the density and velocity dispersion profiles. Figure~\ref{fig:tracerdf} shows both populations on the same scale: they are clearly separated in radius by more than a factor of 10, which is beneficial for the determination of the potential, as discussed in \citet{Walker2011}. The density profiles (left panel) are well approximated by a \texttt{Spheroid} model (Equation~\ref{eq:spheroid_density}) with parameters listed in Table~\ref{tab:mock_tracers}, though both populations are non-spherical (clusters are more oblate in the inner Galaxy, and satellites are slighly prolate, as shown in the right panel). Their kinematic properties are also quite different: although both are moderately rotating in central parts ($v_\phi\sim 50-70$~\kms, see, e.g., \citealt{Frenk1980} for an early analysis of cluster kinematics), clusters are close to isotropic in the inner Galaxy and become significantly radially anisotropic further out, with $\sigma_r/\sigma_{\{\theta,\phi\}} \simeq 2$, whereas for satellites, $\sigma_r$ lies between $\sigma_\phi$ and $\sigma_\theta$, with a mild tangental anisotropy.
However, we caution that these properties are determined by the DF under the assumption of dynamical equilibrium. If we instead fit simple parametric density and velocity dispersion profiles to the actual satellite catalogue, without linking them to each other in any way, the spatial distribution of objects is significantly more extended in the $z$ direction, with axis ratio $z/R$ closer to two, but the velocity anisotropy changes from mildly tangential in the inner part to nearly isotropic or even weakly radially dominated in the outer part (see e.g. \citealt{Riley2019}). These two properties apparently cannot be reconciled in a steady-state model, hence the configuration represented by the DF is a tradeoff between these conflicting requirements, and neither the shape nor the velocity anisotropy resemble the present-day configuration particularly well.

In fact, these features are likely caused by a spatially and kinematically coherent feature known as the Vast Polar Structure (VPOS; e.g., \citealt{Kroupa2005}, \citealt{Pawlowski2020}), roughly perpendicular to the Galactic disc. We recall that we excluded the LMC and its likely satellites from the sample, but even the remaining galaxies have a significantly anisotropic distribution of orbital poles, as illustrated in Figure~\ref{fig:orbital_poles}, left panel. It has been suggested by \citet{GaravitoCamargo2021} that the gravitational effect of the LMC may cause or enhance the orbital pole clustering. However, if we ``undo'' the LMC perturbation in the same way as for $v_z$, i.e., rewinding orbits in a time-dependent MW+LMC potential and then bringing them back to present time in a static MW potential, the resulting distribution of orbital poles does not significantly change (right panel) and still remains rather non-uniform; thus the LMC perturbation cannot be the main cause of the orbital pole clustering (the same conclusion is independently reached by \citealt{Pawlowski2021}). \citet{Li2021} estimated that around a half of the entire satellite population may be part of VPOS, including the LMC itself and its satellites.

In the above figure and in subsequent ones, each object is rendered as a cloud of points representing the uncertainty in its orbit parameters, which stems both from the measurement uncertainties and from the variation in the potential in our ensemble of models (including the LMC mass). However, it is important to stress that one should not sample uniformly from the measurement error distribution, but rather from the posterior distribution (i.e., weighted by the DF value). As a simple illustration, consider an object moving on a nearly circular orbit at a certain radius, and imagine that its PM is measured with a relatively large uncertainty comparable to the value itself. Then most samples from this error distribution will produce a higher velocity and hence place the object at the pericentre of an eccentric orbit with a larger apocentre radius, or even on an unbound orbit. Thus if one assigns equal probability to all these samples, one would arrive at an incorrect conclusion that the object is likely near the pericentre of a highly eccentric orbit. If, on the other hand, one accounts for the fact that lower velocity corresponding to a more tightly bound orbit is more likely to be found among the ensemble of all orbits generated by the DF, the inference will be unbiased. This resolves the apparent ``double convolution'' paradox: assuming that we have a given intrinsic distribution of some quantity (say, velocity), and that measured values are equally likely to be smaller or larger than the true value for any object, the observed distribution will be broadened by errors. Now by sampling from a Gaussian error distribution around the \textit{measured} value, we effectively broaden the true distribution a second time, but by assigning higher weight to samples that have smaller velocities, we may recover the true width of the intrinsic distribution, and the posterior distribution of possible true velocity values for each object will be shifted towards lower-than-measured values. Of course, this requires a model for the intrinsic distribution to be constructed from the measured values, which is precisely what we do in our fits. A caveat is that if the model contains only the equilibrium DF in the given potential, all posterior orbits will necessarily be gravitationally bound to the MW. However, since we introduced a second, ``unmixed'' population of object containing both positive and mildly negative energies, we do not artificially enforce the orbits to be bound.

To construct our posterior sample of orbit parameters $\boldsymbol w_i$ for the $i$-th object, we first draw samples $\boldsymbol w_i^{(s)}, s=1..N_\text{samp}$ from the Gaussian error distribution of this object. Then we iterate over a large number of model parameters from the MCMC ensemble, and for each model $m$ of the potential and the DF $f_m(\boldsymbol w)$, evaluate the likelihood of all samples:
\begin{equation}
f_m(\boldsymbol w_i^{(s)}),\quad s=1..N_\text{samp}\,,
\end{equation}
then pick up one or more samples with probability
\begin{equation}
p_i^{(s)} \equiv \frac{f_m(\boldsymbol w_i^{(s)})}{\sum_{s'=1}^{N_\text{samp}} f_m(\boldsymbol w_i^{(s')})} \,,
\end{equation}
and finally stack together the posterior samples from all examined models (note that the denominator in the above equation is the error-convolved DF value as in Equation~\ref{eq:error_convolution_sum}). For the majority of objects, the posterior distribution averaged over many models in the MCMC ensemble is actually very close to a normal distribution centered on the measured values with quoted uncertainties, i.e., the measurement errors are effectively negligibly small. One may quantify the reduction in the posterior width by the entropy, defined as 
\begin{equation}
S_i \equiv -\!\!\sum_{s=1}^{N_\text{samp}} p_i^{(s)}\; \ln\big(p_i^{(s)}\,N_\text{samp}\big) \le 0.
\end{equation}
If the measurement errors are small, all $p_i^{(s)}$ are close to $1/N_\text{samp}$, and hence the entropy is close to zero; if a single sample dominates the sum, $S\approx -\ln N_\text{samp}$. Objects with the largest negative entropy show the largest reduction in the posterior width, and unsurprisingly, typically have relatively large PM uncertainties or no line-of-sight velocity measurements. These include a few faint globular clusters, namely, Crater, AM~1, AM~4 and Mu\~noz~1: the first two are also the most distant ones, while the latter two are very faint clusters at distances $\sim30$ and $\sim45$~kpc respectively, which have no stars on the giant branch. Among satellite galaxies with $S<-0.5$ are Columba~I, Eridanus~II, Hydra~II, Leo~IV, Leo~V, Pegasus~III, Pisces~II, Reticulum~III (most of them are rather distant), and some (but not all) objects with no line-of-sight velocities: Bo\"otes~IV, Cetus~II, Cetus~III, Indus~I, Pictor~I.

In a mixture model for the satellite sample, we additionally evaluate the posterior probability of belonging to the unmixed (infalling, splashback and unbound) population for each object (Equation~\ref{eq:posterior_outlier_probability}). It turns out that in models without the LMC, Leo~I has a high probability of being unmixed, which is not surprising, given its high velocity and large distance. However, when accounting for the LMC perturbation, its pre-LMC orbit actually has a lower energy and is likely to belong to the main sample, except if the MW mass is small enough ($M_\text{vir}\lesssim 0.9\times10^{12}\,M_\odot$). Hercules, on the other hand, is confidently bound in all models without LMC, but also has a significant probability of being unmixed in a low-mass MW+LMC scenario. Although the mixture model is much less sensitive to an occasional outlier than a standard model with only bound population, we repeated the fits excluding these two galaxies from the sample. The inferred mass within 100 kpc was $\sim 10\%$ lower in this case, but well within uncertainties of the fiducial scenario.
We also looked at the probability of belonging to the unmixed population for other objects not included in the main sample. Apart from the likely LMC satellites, which are examined in the next section, the following objects were found not to be among virialized satellites: Columba~I, Eridanus~II, Leo~T, Phoenix (which are all distant and still infalling), and possibly Cetus~III and Bo\"otes~IV (which have no line-of-sight velocity measurements).

%%%%%%%%%%%%%%
\begin{figure*}
\includegraphics{fig_orbital_phase_nolmc.pdf}\qquad\quad
\includegraphics{fig_orbital_phase_lmc.pdf}
\caption{Distribution of satellites in the space of energy and orbital phase. In the polar coordinates, radius corresponds to the radius of a circular orbit with the given energy, and polar angle -- the radial phase angle $\theta_r$; each galaxy is rendered by a cloud of points sampling from the joint posterior distribution of MW and LMC potential models and measurement uncertainties. Only a subset of all satellites is shown, excluding the galaxies associated with the LMC and a few likely non-virialized objects. \textbf{Left panel} shows the ensemble of models without the LMC, so that the location of points corresponds to the actual present-day phase-space coordinates of these objects. \textbf{Right panel} depicts the ensemble of models with the LMC: the energy and orbital phase are obtained by integrating orbits backward from the present-day coordinates in the time-dependent MW+LMC potential, and then integrating forward to present time in a static MW potential, thus the scatter of points is increased by the variation in the MW potential parameters.
}  \label{fig:orbital_phase}
\end{figure*}
%%%%%%%%%%%%

With the posterior samples, we construct the diagram of orbital phases of satellite galaxies: for each choice of potential and each sample from the position/velocity distribution, we compute the radial phase angle $\theta_r$ -- a canonically conjugate variable to the radial action, which uniformly increases with time. A well-mixed population is expected to have a uniform distribution of $\theta_r$. Figure~\ref{fig:orbital_phase} shows the 36 galaxies from the main dataset used in the fit, plus a few objects with poor PM measurements that were excluded from the sample (Pegasus~III, Pisces~II, Ridiculum~III). This plot does not include the LMC itself, its likely satellites, and objects with high posterior probability of belonging to the unmixed population or without line-of-sight velocities. We observe that there is no significant bias towards close-to-pericentre orbital phases, in contrast to the analyses of \citet{Simon2018}, \citet{Fritz2018} and \citet{Li2021}, which, however, used raw measurements rather than posterior distribution, possibly leading to a bias as explained above. The inclusion of the LMC does not significantly change this distribution (we again perform a backward-then-forward rewinding to undo the LMC perturbation, but caution that the reconstruction of angle variables may be less accurate than the integrals of motion). 

However, we caution that the DF fitting method (or any other approach invoking the Jeans theorem) implicitly assumes a uniform distribution in angle space, and hence prefers potentials that make the observed population consistent with uniform random sampling of orbital phases. If the MW mass were lower, the same measured velocities of objects would be relatively larger compared to the equilibrium velocity distribution at a given radius, and hence one would infer that many of the objects are near the pericentres of their orbits. However, since there is no corresponding population of more distant and slower objects near their apocentres, this potential would be assigned a lower likelihood in the model. On the other hand, if there actually exist distant undiscovered satellites (for instance, \citealt{Koposov2008} estimate that up to $50-100$ faint satellites are still undetected), the inferred MW potential would need to be revised downwards, as illustrated by a toy experiment in the \hyperref[sec:potential_bias]{Appendix}.

%%%%%%%%%%%%%
\begin{figure*}
\includegraphics{fig_orbits1.pdf}
\caption{
Orbits of satellite galaxies in the models neglecting the LMC (blue) or including it (red). For each object, we show the evolution of the Galactocentric radius for a variety of orbits integrated in different potentials with parameters drawn from the MCMC ensemble of models, and the orbital initial conditions (present-day positions and velocities) drawn from the posterior distribution of each model (i.e., are sampled from measurement uncertainties and weighted with the probability of each sample in the given DF and potential model).  The top left panel shows the trajectories of the LMC itself (neglecting observational uncertainties). In the remaining panels, cyan lines additionally show the evolution of energy for each orbit with respect to its present-day value (secondary axis) for the case of the time-dependent MW+LMC potential. When the energy does not change significantly (e.g., for Crater~II or Fornax), the orbits in the time-dependent potential are somewhat more extended since the MW is on average less massive in this series of models. But the recent LMC-induced energy kicks may be both positive or negative and can significantly alter the inferred orbits (e.g., for Carina or Grus~I), or the object might end up being bound to the LMC over its past orbit, in which case the energy in the MW potential varies wildly (e.g., Carina~III).
}  \label{fig:orbits}
\end{figure*}

\begin{figure*}
\includegraphics{fig_orbits2.pdf}
\contcaption{}
\end{figure*}

We now examine more systematically the effect of the LMC on the orbits of clusters and satellite galaxies, using the posterior samples in two MCMC ensembles of models, with and without orbit rewinding. As expected, we do not find any significant changes in cluster orbits with or without the LMC, in agreement with \citet{Garrow2020}, nor are any cluster orbits associated with the LMC, confirming \citet{Boldrini2021}; thus we concentrate on the satellites in what follows. Figure~\ref{fig:orbits} shows the time evolution of Galactocentric distance over the last 3~Gyr%
\footnote{In this section, we reconstruct the LMC trajectory and rewind the orbits of tracers for a longer period of 3~Gyr, even though the likelihood of models is still computed based on the positions and velocities of tracers 2~Gyr ago.}
for models without LMC in blue, and with LMC in red. Green lines additionally show the evolution of orbit energy with respect to its current value in the latter series of models: if it stays around zero, the LMC has little effect on the orbit, but it can also be larger in the past (i.e., the galaxy was moved by the LMC onto a more tightly bound orbit) or vice versa. We observe that for some objects (e.g., Segue~1 or Willman~1), especially in the inner Galaxy, the red and the blue series of orbits are rather similar, but for the majority of them, the orbits look quite different. There are, of course, satellites of the LMC, which have been affected by it over the entire time interval, as evidenced by the large spread of green curves (e.g., Carina~III or Hydrus~I); we discuss them in more detail in the next section. For others, the influence of the LMC is mainly limited to the last 0.5~Gyr. We stress that this needs not be a direct gravitational scattering by the LMC itself (though this likely happened for Aquarius~II, Grus~II, Sagittarius~II, Tucana~III, Tucana~IV, Tucana~V), but rather the change in MW-centric orbit parameters as a result of the MW reflex motion caused by the LMC. As illustrated by Figure~\ref{fig:orbit_rewinding_energy}, objects beyond a few tens kpc are likely to be moved up or down in energy, but the former occurs more often. For some galaxies, this dramatically alters their inferred past orbit; a prime example is Leo~I, which had a more tightly bound orbit in the past until the arrival of the LMC, as already noted by \citet[their section 3.2]{Erkal2020b}. A similar conclusion is reached for Leo~V, Sextans and Tucana~II. On the other hand, a surprisingly large fraction of satellites have been moved to more tightly bound orbits recently, including Canes Venatici~II, Draco, Fornax, Hercules, Sculptor and Ursa~Minor. We note that in absence of any energy change (e.g., Crater~II or Ursa Major~I), orbits in the series of models with LMC would be somewhat larger than in models without LMC, on account of a lower MW mass in the former case, but the difference in orbits for the above named galaxies is far more pronounced. Among them, only Hercules is affected to such an extent that it might have been on a splashback orbit prior to the LMC-induced perturbation, but the orbits of Carina and Sculptor are also dramatically changed and had a higher eccentricity in the past. We note that the orbit of Sagittarius dSph also increased its eccentricity due to the LMC perturbation (see Figure~9 in \citealt{Vasiliev2021b}), but these changes are still minor compared to what many other satellites have experienced. It is clear that the LMC cannot be neglected when analyzing the past trajectories, tidal forces and mass loss history of most MW satellites.

%%%%%%%%%%%
\subsection{Satellites of the LMC}  \label{sec:lmc_satellites}

%%%%%%%%%%%%%%%
\begin{figure*}
\includegraphics{fig_orbits_wrt_lmc.pdf}
\caption{History of interaction with the LMC for selected galaxies over the past 3 Gyr. For each object, red lines show the distances from the LMC as a function of time for an ensemble of orbits sampled from the uncertainties in the present-day position/velocity, variations in the MW potential, and the LMC mass; cyan lines show the relative energy w.r.t.\ the LMC (ignoring the MW). A substantial fraction of green lines below zero indicates that the object is likely to be an LMC satellite (though it might not be bound at present time): this is the case for the galaxies in the first two rows, though Horologium~II and Reticulum~II have only a moderate overall probability of being bound in the past. The galaxies in the third row lack line-of-sight velocity measurements, but could have been bound to the LMC for a range of possible velocities. The last two rows show galaxies that have experienced a close passage with the LMC recently, but were not bound to it earlier, and with a possible exception of Grus~II and Tucana~IV, are unlikely to be bound at present time. 
}  \label{fig:orbits_wrt_lmc}
\end{figure*}

\begin{figure*}
\includegraphics{fig_boundprob_lmc.pdf}
\caption{Probability of past association with the LMC for selected galaxies. Each point represents an orbit sampled from the uncertainties in the present-day position/velocity, variations in the MW potential, and the LMC mass (\textit{without} posterior reweighting); the colour indicates the fraction of time that this orbit had a negative energy w.r.t.\ LMC in the interval between $-3$ and $-1$ Gyr in the past. Orbits are plotted against LMC mass and either the heliocentric distance (top two rows) or the presently unknown line-of-sight velocity (bottom row).
}  \label{fig:boundprob_lmc}
\end{figure*}
%%%%%%%%%%%%%

We now turn attention to the objects suspected to be LMC satellites and therefore excluded from the main sample during the fit (marked by green frames in Figure~\ref{fig:orbits}). Their past orbits in models with the LMC are indeed concentrated around the past trajectory of the LMC itself, and their energy in the MW-centered frame has large fluctuations at all times, not just over the last 0.5~Gyr. For a better view, Figure~\ref{fig:orbits_wrt_lmc} shows the evolution of the distance from the LMC for these and a few other galaxies, and the evolution of energy in the LMC-centered frame $E_\text{LMC}$ (i.e., using the relative velocity). Although this energy may change rather dramatically in the last 0.5~Gyr due to the complex three-body interaction between the MW, the LMC and the satellite galaxy, it often ends up being negative for a large fraction of orbits (recall that the ensemble of orbits encompasses the uncertainties in the current coordinates, MW potentials and LMC masses, but in this case we do not additionally reweigh orbits by their posterior probability in the MW-bound DF).

To quantify the probability of being an LMC satellite, we use the fraction of orbits with $E_\text{LMC}<0$ over the time interval between $-3$ and $-1$~Gyr. Figure~\ref{fig:boundprob_lmc} shows this fraction as a function of the LMC mass and a second parameter -- either the heliocentric distance or the line-of-sight velocity. We find that Carina~II, Carina~III, Horologium~I, Hydrus~I, Phoenix~II and SMC have high probability of being bound to the LMC in the past regardless of the LMC mass (though it generally increases with mass), at least in most of the range of heliocentric distances compatible with observations. Reticulum~II has a lower probability overall, unless its distance is underestimated (typically it would be bound for $D\gtrsim 35$~kpc for $M_\text{LMC} \ge 10^{11}\,M_\odot$ or $D\gtrsim33$~kpc for $M_\text{LMC} \ge 2\times10^{11}\,M_\odot$, while the observed distance is $31.6\pm1.5$~kpc). Horologium~II also has a large spread in probabilities at all LMC masses, possibly due to relatively large PM uncertainties.

In addition, we find a few LMC satellite candidates among objects with no line-of-sight velocity measurements, shown in the third row of Figures~\ref{fig:orbits_wrt_lmc} and \ref{fig:boundprob_lmc}. Pictor~II, Delve~2 and Eridanus~III are currently located within 10, 30 and 40 kpc from the LMC, respectively, but their PM are compatible with the past association with the LMC (marginally, in the case of Eridanus~III). Naturally, this is conditioned on the currently unknown velocity, namely: $v_\text{los}$ needs to be $110\pm80$~\kms for Delve~2, $200\pm70$~\kms for Eridanus~III, or $270\pm120$~\kms for Pictor~II. The association between these objects and the LMC has been previously proposed in \citet{Cerny2021}, \citet{Kallivayalil2018}, \citet{Erkal2020a} and \citet{DrlicaWagner2016}, with similar predictions for the missing velocity.

Galaxies shown in the lower two rows of Figure~\ref{fig:orbits_wrt_lmc} had a close passage with the LMC in the last few hundred Myr, but were not among its satellites in the past. Two of them (Grus~II and Tucana~IV) currently may have velocities below the escape velocity of the LMC, but might not remain bound in the future on account of strongly dynamic interactions with both the LMC and the MW. 
For the remaining galaxies, we do not find any significant probability of LMC association in the past, although some of the brightest satellites (e.g., Carina, Fornax) share the orbital plane with the LMC and might be accreted from the same filament on cosmological timescales \citep[e.g.,][]{Pardy2020}.

Our assessment of the LMC satellite population broadly agrees with other studies (\citealt{Jethwa2016}, \citealt{Kallivayalil2018}, \citealt{Erkal2020a}, \citealt{Patel2020}, \citealt{Battaglia2021}), which typically limited their analysis to one or a few choices of MW potential. For Carina~II, Carina~III, Horologium~I, Hydrus~I, SMC there is a general concensus that they have been associated with (a massive enough) LMC in the past. Horologium~II, Phoenix~II and Reticulum~II are more debatable; we find respectively moderate, high and low probability of them being former LMC satellites.


%%%%%%%%
\section{Discussion and conclusions}  \label{sec:summary}

The role of the LMC as a major factor affecting the dynamics in the outer reaches of our Galaxy is gradually becoming appreciated by the community, but the tools for incorporating this factor into analysis are still scarce. Here we introduced a simple but effective approach for ``restoring order'' by first computing the past orbits of the MW and the LMC under mutual gravity, and then rewinding the orbits of stars or any other MW-bound objects in a time-dependent potential of the two galaxies, thus reconstructing the unperturbed state of the MW before the arrival of the LMC.
This opens up the possibility to apply well-developed classical methods for dynamical modelling and measurement of gravitational potential based on the equilibrium approximation. 

We then invoked one such method, simultaneously constraining the tracer DF, the MW potential, and the LMC mass, using two complementary datasets of objects with 6d phase-space coordinates: globular clusters and satellite galaxies. We demonstrated that the method is able to recover the true potential in the presence of the LMC perturbation, and that the neglect of this perturbation biases the inferred MW mass up by $\sim20$\%, in agreement with estimates by \citet{Erkal2020b}. Applying our method to the most recent measurements based on \Gaia EDR3, we measured the MW mass distribution up to the virial radius with a relative uncertainty of $\sim20\%$ at 200~kpc (see Table~\ref{tab:MWpotential}), although we do not obtain tight constraints on the virial mass. The most likely range for the LMC mass is $(1-2)\times10^{11}\,M_\odot$; models without the LMC perturbation are noticeably worse in fitting the observed dataset (the difference in log-likelihood is $\sim 12$).

We stress that the effect of the LMC cannot be simply reduced to a deflection of orbits that pass close to it. Equally, if not more important, is the reflex motion of the central region of the MW in response to the LMC's gravitational pull, displacing it spatially and kinematically w.r.t.\ the outer part. The more distant objects are not able to keep up with this recent swinging motion of the MW-centered reference frame, and thus find themselves on significantly different orbits than without the LMC perturbation: this is clearly manifested in the energy variations in the last 0.5~Gyr in a large fraction of satellite galaxies (green curves in Figure~\ref{fig:orbits}) and the corresponding differences in inferred past orbits with and without the LMC (red and blue curves in that figure). Nevertheless, the global features of the entire population of satellites, such as the orientations of angular momentum vectors (Figure~\ref{fig:orbital_poles}) or distribution of orbital phases (Figure~\ref{fig:orbital_phase}), are robust w.r.t.\ the inclusion of the LMC.

In contrast to some other studies (e.g., \citealt{Fritz2018}, \citealt{Li2021}), we do not find any evidence for a concentration of satellites near the pericentres of their orbits, nor for a strong tangential anisotropy of their velocities. We stress that this assessment is based on the limited selection of galaxies that does not include the LMC and its likely satellites, which are, of course, mostly near their pericentres, but should not count as independent samples. However, there are two further reasons for our different conclusion. One is that we use posterior-weighted uncertainties rather than raw measurements, which downweights the high-velocity tail of the observational error distribution. The other is that since we let the potential vary in the fit, the models choose the MW mass profile that prefers a more uniform distribution of orbital phases by construction. Nevertheless, we highlight a possible caveat in this line of reasoning: if the observed catalogue of satellites is incomplete and there exist yet undiscovered objects at large distances (hence more likely to be at apocentres), our inference on the MW mass would be biased up, as illustrated in the \hyperref[sec:potential_bias]{Appendix}, and in the true (lower-mass) potential, the observed distribution of satellites should indeed be concentrated near pericentres. If the selection function of the observed sample could be reliably estimated, this effect may be taken into account in our modelling scheme, as illustrated e.g.\ by \citet{Hattori2021} in a similar context.

Inevitably, our approach has a number of limitations. For simplicity, we neglect the measurement uncertainties on the LMC position and velocity, which are fairly small compared to uncertainties of many other objects. We also ignore the gravitational interaction between LMC and SMC, which might deflect the LMC and thus affect its past orbit reconstruction. More fundamentally, we assume a fixed (non-deforming) potential of both MW and LMC; note that the deformation of the tracer density profile is implicitly accounted for by the orbit rewinding procedure. Nevertheless, even this rather simplified reconstruction of the LMC orbit is able to compensate, in the first approximation, its perturbation on MW tracers.

The orbit rewinding step can be easily added to any forward-modelling approach constrained by kinematics of individual tracers, such as halo stars or tidal streams. Thus the significant gravitational perturbation from the LMC does not invalidate the classical dynamical modelling tools, but should be taken into account if we aim for a 10\% accuracy level that matches the quality of modern observational datasets.


%%%%%%%%%
\section*{Acknowledgements}
We thank V.Belokurov, D.Erkal, W.Evans and S.Koposov for valuable discussions, and the anonymous referee for expedient and comprehensive reports, which prompted us to extend our analysis and clarify various issues, improving the presentation. EV acknowledges support from STFC via the Consolidated grant to the Institute of Astronomy.\\
\textit{Software used}: \textsc{Numpy}, \textsc{Matplotlib}, \textsc{Emcee} (\href{http://ascl.net/1303.002}{ascl:1303.002}), \textsc{GyrfalcON} (\href{http://ascl.net/1402.031}{ascl:1402.031}), \textsc{Nemo} (\href{http://ascl.net/1010.051}{ascl:1010.051}).

\section*{Data availability}
We provide samples of the MW and LMC potential parameters from our MCMC analysis and corresponding satellite orbits, together with example \textsc{python} scripts for orbit rewinding, at the following repository: \url{https://zenodo.org/record/5541971}.

\appendix

\begin{figure*}
\includegraphics{fig_potential_bias_incomplete_sample.pdf}
\caption{Illustration of the potential bias caused by incomplete tracer sample (see text for details).
}  \label{fig:potential_bias_incomplete_sample}
\end{figure*}

%%%%%%
\begin{thebibliography}{}

\bibitem[Battaglia et al.(2022)]{Battaglia2021}
Battaglia G., Taibi S., Thomas G., Fritz T., 2022, A\&A, 657, 54
%dSph PM from Gaia EDR3, LMC effect on satellite orbits

\bibitem[Baumgardt \& Vasiliev(2021)]{Baumgardt2021}
Baumgardt H., Vasiliev E., 2021, MNRAS, 505, 5957
%GC distances

\bibitem[Baumgardt et al.(2019)]{Baumgardt2019}
Baumgardt H., Hilker M., Sollima A., Bellini A., 2019, MNRAS, 482, 5138
%GC line-of-sight velocities

\bibitem[Belokurov et al.(2019)]{Belokurov2019}
Belokurov V., Deason A., Erkal D., Koposov S., Carballo-Bello J., Smith M., Jethwa P., Navarrete C., 2019, MNRAS, 488, L47
%Magellanic wake in Pisces

\bibitem[Binney(2012)]{Binney2012}
Binney J., 2012, MNRAS, 426, 1324
%Staeckel fudge

\bibitem[Binney \& Tremaine(2008)]{Binney2008}
Binney J., Tremaine S., 2008, Galactic Dynamics, 2nd edn., Princeton Univ. Press, Princeton, NJ
%the bible

%\bibitem[Binney \& Wong(2017)]{Binney2017}
%Binney J., Wong L.K., 2017, MNRAS, 467, 2446
%DF for GC

\bibitem[Boldrini \& Bovy(2021)]{Boldrini2021}
Boldrini P., Bovy J., 2021, arXiv:2106.09419
%no GC coming from dSph

\bibitem[Boylan-Kolchin et al.(2013)]{BoylanKolchin2013}
Boylan-Kolchin M., Bullock J., Sohn S.~T., Besla G., van der Marel R., 2013, ApJ, 768, 140
%Leo I role in MW mass estimates

%\bibitem[Callingham et al.(2019)]{Callingham2019}
%Callingham T., Cautun M., Deason A., et al., 2019, MNRAS, 484, 5453
%MW mass from satellites / EAGLE&AURIGA sims

\bibitem[Cautun et al.(2020)]{Cautun2020}
Cautun M., Ben\'\i tez-Llambay A., Deason A., et al., 2020, MNRAS, 494, 4291
%MW mass from satellites / AURIGA,APOSTLE&EAGLE sims

\bibitem[Cerny et al.(2021)]{Cerny2021}
Cerny W., Pace A., Drlica-Wagner A., et al., 2021, ApJ, 910, 18
%Delve 2 satellite

\bibitem[Conroy et al.(2021)]{Conroy2021}
Conroy C., Naidu R., Garavito-Camargo N., Besla G., Zaritsky D., Bonaca A., Johnson B., 2021, Nature, 592, 534
%MW halo asymmetry due to LMC

\bibitem[Cuddeford(1991)]{Cuddeford1991}
Cuddeford P., 1991, MNRAS, 253, 414
%anisotropic DF

\bibitem[Cunningham et al.(2020)]{Cunningham2020}
Cunningham E., Garavito-Camargo N., Deason A., et al., 2020, ApJ, 898, 4
%MW halo distortion due to LMC

\bibitem[Deason et al.(2021)]{Deason2021}
Deason A., Erkal D., Belokurov V., et al., 2021, MNRAS, 501, 5964
%MW mass from kinematics of halo stars taking LMC into account

\bibitem[Dehnen(2000)]{Dehnen2000}
Dehnen W., 2000, ApJ, 536, L9; ascl:1402.031
%gyrfalcON (N-body simulation code)

\bibitem[Drimmel \& Poggio(2018)]{Drimmel2018}
Drimmel R., Poggio E., 2018, RNAAS, 2, 210
%solar velocity

\bibitem[Drlica-Wagner et al.(2016)]{DrlicaWagner2016}
Drlica-Wagner A., Bechtol K., Allam S., et al., 2016, ApJL, 833, L5
%Pictor II satellite

\bibitem[Eadie \& Juric(2019)]{Eadie2019}
Eadie G., Juri\'c M., 2019, ApJ, 875, 159
%MW mass model from GC kinem

\bibitem[Einasto(1965)]{Einasto1965}
Einasto J., 1965, Trudy Inst. Astrofiz. Alma-Ata, 51, 87
%density profile

\bibitem[Erkal et al.(2019)]{Erkal2019}
Erkal D., Belokurov V., Laporte C., et al., 2019, MNRAS, 487, 2685
%LMC mass from Orphan stream

\bibitem[Erkal \& Belokurov(2020)]{Erkal2020a}
Erkal D., Belokurov V., 2020, MNRAS, 495, 2554
%LMC satellites census

\bibitem[Erkal et al.(2020)]{Erkal2020b}
Erkal D., Belokurov V., Parkin D., 2020, MNRAS, 498, 5574
%LMC biases MW mass

\bibitem[Erkal et al.(2021)]{Erkal2021}
Erkal D., Deason A., Belokurov V., et al., 2021, MNRAS, 506, 2677
%MW halo kinematic asymmetries due to LMC

%\bibitem[Evans et al.(1997)]{Evans1997}
%Evans N.~W., H\"afner R., de Zeeuw T., 1997, MNRAS, 286, 315
%power-law mass estimator

\bibitem[Eilers et al.(2019)]{Eilers2019}
Eilers A.-C., Hogg D., Rix H.-W., Ness M., 2019, ApJ, 871, 120
%MW circular-velocity curve up to 25 kpc

\bibitem[Foreman-Mackey et al.(2013)]{ForemanMackey2013}
Foreman-Mackey D., Hogg D., Lang D., Goodman J., 2013, PASP, 125, 306
%MCMC

\bibitem[Frenk \& White(1980)]{Frenk1980}
Frenk C., White S., 1980, MNRAS, 193, 295
%kinematics of GC population

\bibitem[Fritz et al.(2018)]{Fritz2018}
Fritz T., Battaglia G., Pawlowski M., et al. 2018, A\&A, 619, A103
%dSph PM and orbits from Gaia DR2

\bibitem[Fritz et al.(2020)]{Fritz2020}
Fritz T., Di Cintio A., Battaglia G., Brook C., Taibi S., 2020, MNRAS, 494, 5178
%MW mass estimate from satellites

\bibitem[Gaia Collaboration(2021a)]{Brown2021}
\Gaia Collaboration (Brown et al.), 2021a, A\&A, 649, 1
%Gaia EDR3 overview

\bibitem[Gaia Collaboration(2021b)]{Luri2021}
\Gaia Collaboration (Luri et al.), 2021b, A\&A, 649, 7
%LMC in Gaia EDR3

\bibitem[Garavito-Camargo et al.(2021a)]{GaravitoCamargo2020}
Garavito-Camargo N., Besla G., Laporte C., Price-Whelan A., Cunningham E., Johnston K., Weinberg M., G{\'o}mez F., 2021, ApJ, 919, 109
%MW distortion due to LMC

\bibitem[Garavito-Camargo et al.(2021b)]{GaravitoCamargo2021}
Garavito-Camargo N., Patel E., Besla G., Price-Whelan A., Gomez F., Laporte C., Johnston K., 2021, ApJ, 923, 140
%orbital pole clustering due to LMC

\bibitem[Garrow et al.(2020)]{Garrow2020}
Garrow T., Webb J., Bovy J., 2020, MNRAS, 499, 804
%perturbations of GC orbits by MW satellites

\bibitem[G\'omez et al.(2015)]{Gomez2015}
G\'omez F., Besla G., Carpintero D., Villalobos \'A., O'Shea B., Bell E., 2015, ApJ, 802, 128
%MW and Sgr stream response to LMC

%\bibitem[Harris et al.(2020)]{Numpy}
%Harris C., et al., 2020, Nature, 585, 357

\bibitem[Hattori et al.(2021)]{Hattori2021}
Hattori K., Valluri M., Vasiliev E., 2021, MNRAS, 508, 5468
%DF modelling with selection function

%\bibitem[Hunter(2007)]{Matplotlib}
%Hunter J., 2007, Comput. Sci. Eng., 9, 90

\bibitem[Jethwa et al.(2016)]{Jethwa2016}
Jethwa P., Erkal D., Belokurov V., 2016, MNRAS, 461, 2212
%LMC satellites and orbit rewinding

\bibitem[Kallivayalil et al.(2013)]{Kallivayalil2013}
Kallivayalil N., van der Marel R., Besla G., Anderson J., Alcock C., 2013, ApJ, 764, 161
%LMC PM and orbit reconstructions

\bibitem[Kallivayalil et al.(2018)]{Kallivayalil2018}
Kallivayalil N., Sales L., Zivick P., et al., 2018, ApJ, 867, 19
%LMC satellites in DR2

%\bibitem[Kochanek(1996)]{Kochanek1996}
%Kochanek C., 1996, 457, 228
%MW mass estimate

\bibitem[Koposov et al.(2008)]{Koposov2008}
Koposov S., Belokurov V., Evans N.W., et al., 2008, ApJ, 686, 279
%MW satellite luminosity function and prediction for the total number of sats

\bibitem[Kroupa et al.(2005)]{Kroupa2005}
Kroupa P., Theis C., Boily C., 2005, A\&A, 431, 517
%VPOS

\bibitem[Kulessa \& Lynden-Bell(1992)]{Kulessa1992}
Kulessa A., Lynden-Bell D., 1992, MNRAS, 255, 105
%maximum-likelihood estimate of MW mass in a scale-free model

\bibitem[Li et al.(2020a)]{Li2020a}
Li Z.-Z., Qian Y.-Z., Han J., Li T.~S., Wang W., Jing Y.~P., 2020a, ApJ, 894, 10
%MW mass from satellites / EAGLE sims

\bibitem[Li et al.(2020b)]{Li2020b}
Li Z.-Z., Zhao D.~H., Jing Y.~P., Han J., Dong F.~Y., 2020b, ApJ, 905, 177
%infalling satellite orbits from sims

\bibitem[Li et al.(2021)]{Li2021}
Li H., Hammer F., Babusiaux C., Pawlowski M., Yang Y., Arenou F. Du C., Wang J., 2021, ApJ, 916, 8
%dSph PM from Gaia EDR3

\bibitem[McConnachie \& Venn(2020)]{McConnachie2020}
McConnachie A., Venn K., 2020, RNAAS, 4, 229
%dSph PM from Gaia EDR3

\bibitem[McMillan \& Binney(2013)]{McMillan2013}
McMillan P., Binney J., 2013, MNRAS, 433, 1411
%Monte Carlo marginalization over obs errors

\bibitem[McMillan(2017)]{McMillan2017}
McMillan P., 2017, MNRAS, 465, 76
%MW potential

\bibitem[Pardy et al.(2020)]{Pardy2020}
Pardy S., D'Onghia E., Navarro J., et al., 2020, MNRAS, 492, 1543
%suggest that Carina and Fornax are LMC sats

%\bibitem[Patel et al.(2018)]{Patel2018}
%Patel E., Besla G., Mandel K., Sohn S.~T., 2018, ApJ, 857, 78
%MW mass from satellites / Illustris sims

\bibitem[Patel et al.(2020)]{Patel2020}
Patel E., Kallivayalil N., Garavito-Camargo N., et al., 2020, ApJ, 893, 121
%orbits and classification of LMC satellites

\bibitem[Pawlowski \& Kroupa(2020)]{Pawlowski2020}
Pawlowski M., Kroupa P., 2020, MNRAS, 491, 3042
%VPOS

\bibitem[Pawlowski et al.(2021)]{Pawlowski2021}
Pawlowski M., Oria P.-A., Taibi S., Famaey B., Ibata R., 2021, arXiv:2111.05358
%no effect of LMC on VPOS

\bibitem[Petersen \& Pe\~narrubia(2021)]{Petersen2021}
Petersen M., Pe\~narrubia J., 2021, Nature Astronomy, 5, 251
%detection of LMC-induced MW reflex motion in halo kinematics

\bibitem[Posti \& Helmi(2019)]{Posti2019}
Posti L., Helmi A., 2019, A\&A, 621, 56
%MW mass model from globular clusters

\bibitem[Posti et al.(2015)]{Posti2015}
Posti L., Binney J., Nipoti C., Ciotti L., 2015, MNRAS, 447, 3060
%Double power law action-based DF

\bibitem[Read et al.(2021)]{Read2021}
Read J., Mamon G., Vasiliev E., et al., 2021, MNRAS, 501, 978
%Gaia Challenge test suite

\bibitem[Riley et al.(2019)]{Riley2019}
Riley A., Fattahi A., Pace A., et al., 2019, MNRAS, 486, 2679
%anisotropy of satellites

\bibitem[Shen et al.(2022)]{Shen2021}
Shen J., Eadie G., Murray N., et al., 2022, ApJ, 925, 1
%MW mass from halo stars

\bibitem[Shipp et al.(2021)]{Shipp2021}
Shipp N., Erkal D., Drlica-Wagner A., et al., 2021, ApJ, 923, 149
%LMC mass from streams

\bibitem[Simon(2018)]{Simon2018}
Simon J., 2018, ApJ, 863, 89
%dSph PM and orbits from Gaia DR2

\bibitem[Slizewski et al.(2022)]{Slizewski2021}
Slizewski A., Dufresne X., Murdock K., Eadie G., Sanderson R., Wetzel A., Juric M., 2022, ApJ, 924, 131
%MW mass from satellites / FIRE sims

\bibitem[Vasiliev(2019a)]{Vasiliev2019a}
Vasiliev E., 2019a, MNRAS, 482, 1525
%AGAMA

\bibitem[Vasiliev(2019b)]{Vasiliev2019b}
Vasiliev E., 2019b, MNRAS, 484, 2832
%mean PM of clusters in Gaia DR2 and MW mass estimate from GC

\bibitem[Vasiliev \& Baumgardt(2021)]{Vasiliev2021a}
Vasiliev E., Baumgardt H., 2021, MNRAS, 505, 5978
%GC PM in EDR3

\bibitem[Vasiliev et al.(2021)]{Vasiliev2021b}
Vasiliev E., Belokurov V., Erkal D., 2021, MNRAS, 501, 2279
%Sgr stream + LMC

\bibitem[Vasiliev et al.(2022)]{Vasiliev2021c}
Vasiliev E., Belokurov V., Evans N.~W., 2022, ApJ, 926, 203
%radialization of massive satellite orbits, implications for LMC rewinding

%\bibitem[Virtanen et al.(2020)]{Scipy}
%Virtanen P., et al., Nature Methods, 17, 261

\bibitem[Walker \& Pe\~narrubia(2011)]{Walker2011}
Walker M, Pe\~narrubia J., 2011, ApJ, 742, 20
%two-population dynamical modelling

\bibitem[Wang et al.(2020)]{Wang2020}
Wang W., Han J., Cautun M., Li Z., Ishigaki M., 2020, SCPMA, 63, 109801
%review of MW mass profile estimates

\bibitem[Wang et al.(2022)]{Wang2021}
Wang J., Hammer F., Yang Y., 2022, MNRAS, 510, 2242
%MW mass from DF-based models fitted to globular clusters

%\bibitem[Watkins et al.(2019)]{Watkins2019}
%Watkins L., van der Marel R., Sohn S.~T., Evans N.~W., 2019, ApJ, 873, 118
%MW mass from GC, power-law estimator

\bibitem[Watkins et al.(2010)]{Watkins2010}
Watkins L., Evans N.~W., An J., 2010, MNRAS, 406, 264
%power-law mass estimator for MW satellites

\bibitem[Zhao(1996)]{Zhao1996}
Zhao H.-S., 1996, MNRAS, 278, 488
%double-power-law models

\end{thebibliography}

\vspace*{-5mm}
\section{Bias in potential from incomplete tracer samples}  \label{sec:potential_bias}

Here we present a simple illustration of the bias in potential inference resulting from an incomplete tracer sample. The true distribution of tracers is sampled from an isotropic \texttt{QuasiSpherical} DF corresponding to the density profile shown in blue in the left panel of Figure~\ref{fig:potential_bias_incomplete_sample}. The limited sample is obtained by removing some of the more distant objects, so that the spatial distribution of the remaining ones follows the red curve (it matches the original sample at $r\lesssim 50$~kpc and becomes progressively more incomplete at larger radii). Naturally, this limited sample is also more likely to contain objects near pericentres of their orbits -- the distribution of radial phase angles of the limited sample in the actual potential is more peaked around $\theta_r=0,2\pi$, unlike the uniform distribution of the full sample (both shown by solid hustograms in the right panel). Although there is no explicit kinematic selection, the remaining objects have on average larger velocities than if we observed them equally spread over all orbital phases.
Centre panel shows the circular-velocity curve obtained by simultaneously fitting the DF and the potential to the full sample (blue shaded region) and to the limited sample (red shaded region). The former one recovers the actual circular velocity (black dashed line) to within 3\%, while the latter one is biased up by 7\% at 100~kpc and by 15\% at 300~kpc (the biases in the enclosed mass are twice larger). The reason is that the Jeans theorem, which underlies any equilibrium modelling approach, implies a uniform distribution in phase (red shaded regions in the right panel), and hence compensates for the pericentre bias in the limited sample by increasing the host galaxy mass, so that the higher velocities in the limited sample are representative of the average orbit velocity rather than the pericentre velocity. The DF fitted to the limited sample is also slightly more tangentially biased ($\beta\simeq -0.05$) for the same reason (the actual velocities of remaining objects near pericentre are preferentially tangential, and the assumption of uniform phase distribution propagates this to the entire orbit).

\vspace*{-4mm}
\section{Covariance plots for potential parameters}  \label{sec:covplots}

\begin{figure*}
\includegraphics{covar_both_zhao.pdf}
\caption{Covariances and marginalized posterior distributions of potential parameters used in the fit (panels on the right of the green line) and derived enclosed masses at different radii (50, 100, 200 kpc and $r_\text{vir}$, panels on the left) for the Zhao family of halo profiles. We used the density $\rho_\odot$ at the Solar radius $r_\odot\equiv 8.2$~kpc instead of the normalization factor $\rho_0$ in the fit, since it is less strongly correlated with $r_\text{scale}$ and other parameters than $\rho_0$. Blue histograms and contours show the models without the LMC, and red -- with the LMC perturbation and orbit rewinding. The upper inset panel shows the enclosed mass profiles for both series of models: solid lines, darker and lighter shaded regions correspond to the median, 16/84 and 2.3/97.7 percentiles, and the green dashed line shows the relation between virial mass and radius. The right inset panel shows the histograms of likelihood values in both series of models, offset horizontally by a constant value (same for all models); overplotted thinner curves show the expected $\chi^2$ distributions with 28 degrees of freedom (the overall number of free parameters).
}  \label{fig:covplot_zhao}
\end{figure*}

\begin{figure*}
\includegraphics{covar_both_ein.pdf}
\caption{Same as the previous figure, but for the Einasto family of models. Similarly to the Zhao family, we replace the intrinsic profile parameters $\rho_0$ and $r_\text{cut}$ by the equivalent pair $\rho_\odot$ (density at the Solar radius) and $\gamma_\odot$ (corresponding logarithmic density slope), since they are much less correlated than the intrinsic parameters. The enclosed mass profiles are very similar to the Zhao models for $r\lesssim 200$~kpc, but the density drops more steeply at larger radii, and therefore the extrapolated virial mass is somewhat lower (but with a large uncertainty in either case). The distribution of likelihoods is almost identical between both series of models (i.e., neither is statistically more preferred, although the Einasto family has fewer free potential parameters -- 3 instead of 5).
}  \label{fig:covplot_einasto}
\end{figure*}

\end{document}

}
\bsp
\appendix
\section{Fields in a comoving frame for align case}
\label{Appendix.align}

In this Appendix we investigate the particle motion inside and outside of
the time-dependent current sheet with magnetic field
\begin{align}
\label{eqn:B1}\\
B_y & = B_0 \frac{R_{\rm L}}{ct} \tanh\left[\frac{z}{\Delta(t)}\right]. 
\label{eqn:B2}
\end{align}

For the aligned case we consider the same functional form of the time dependence of the sheet thickness on time 
$\Delta(t)=\Delta_0(t/t_0)^{\kappa}$. Then, the expression for the electric field
\begin{eqnarray}
E_x & = & \dfrac{B_0 R_{\rm L} z}{c^2t^2} \tanh\left[\frac{z}{\Delta(t)}\right], 
\label{eqnB:exO2d}
\\
\label{Appendix:ez.al}
E_z & = & 0,
\end{eqnarray}
again does not satisfy Maxwell equations for arbitrary $\kappa$, and remains valid only for $\kappa = 1$. This corresponds to the linear growth of the sheet thickness due to the radial expansion with constant opening angle of a sheet.
 
Similarly to the orthogonal case there are two possibilities to make the fields satisfy Maxwell equations:

 (i) Changing $x$-component of electric field, without any change of $z$-component

 (ii) Changing $z$-component without any change of $x$-component.

In contrast to the orthogonal case we change $x$-component of the
electric field instead of adding $E_{z}$ one. This implies that Hamiltonian for aligned case 
will be independent of $x$:
\begin{equation}
\begin{split}
\mathcal{H}^{\rm al}
=c \left(m^2c^2+ P^{2}_{z} + p^{2}_{z}\right)^{1/2},
\end{split}
\end{equation}
where
\begin{equation}
P_{x} = p_{x} - \dfrac{eB_0R_L\Delta(t)}{c^2t}
\log\left[\cosh\left(\dfrac{z}{\Delta(t)}\right)\right].
\end{equation}

Finally, similarly to the orthogonal case, we can find the solution of Maxwell equations by choosing 
the integration constant such that electric field vanishes at infinity.
\begin{equation}
\label{Appendix:ex.al}
\begin{split}
E_{x} = \dfrac{B_0R_L\Delta(t)}{c^2t^2}
\left\{(1-\kappa)\log\left[2\cosh\left(\frac{z}{\Delta(t)}\right)\right]+\right.\\
\left.+\kappa\dfrac{z}{\Delta(t)}\tanh\left(\frac{z}{\Delta(t)}\right)\right\}.
\end{split}
\end{equation}
As a result, an estimation of the current sheet width based on the equality between 
Larmor radius and the sheet width is still valid for the aligned case. But here the drift leads  
to the rarefaction of sheet, since it is directed perpendicular to the sheet. On the other hand, since 
$\partial\mathcal{H}/\partial x =0$, we have $P_{x} =$ const. Hence, the particle can gain acceleration along the sheet only due to electromagnetic part of momentum
\begin{equation}
\begin{split}
\delta p_x=\dfrac{eB_0R_L}{c^2}\left\{\dfrac{\Delta_0}{t_0}
\log\left[\cosh\dfrac{z_0}{\Delta_0}\right]\right.
-\\-\left.
\dfrac{\Delta(t)}{t}
\log\left[\cosh\dfrac{z}{\Delta(t)}\right]\right\}
\end{split}
\end{equation}
 
\subsection{Particle motion}
\label{sect:al}

\begin{figure}
\centering
\includegraphics[scale=0.55]{align-eps-converted-to.pdf}
\caption{Numerical solution to equation (\ref{App.case1}). The self consistency obtained for $\kappa=0.35$. }
\label{fig:App.sol}
\end{figure}

For aligned current sheet we use expressions (\ref{eqn:B2}), (\ref{Appendix:ez.al}), and
(\ref{Appendix:ex.al}) for electric and magnetic fields:
\begin{align}
& B_y   = B_0 \frac{t_0}{t} \tanh\frac{z}{\Delta(t)},\\
\begin{split}
 E_x = B_0\dfrac{R_{\rm L}\Delta(t)}{(ct)^2} 
\left\{(1-\kappa)\log\left[2\cosh\dfrac{z}{\Delta(t)}\right]+\right.\\
\left.+\kappa\dfrac{z}{\Delta(t)}\tanh\frac{x}{\Delta(t)}\right\},
\end{split}
\\
&E_z = 0.
\end{align}
Here again $\Delta(t)\propto t^{\kappa}$. After writing down equations of motion and some 
algebra we obtain the following equation for $z$ coordinate of the particle:
\begin{equation}
\ddot{z}=\dot{x}\dfrac{\Lambda}{t}\tanh\dfrac{z}{\Delta(t)}
\end{equation} 
 and equality for for $x$-component of velocity
 \begin{equation}
 \dot{x}=v_0+\Lambda\dfrac{\Delta_0}{t_0}\ln[2\cosh(x_0/\Delta_0)]-\Lambda\dfrac{\Delta}{t}\ln[2\cosh(x/\Delta)],
 \end{equation}
which corresponds to conservation of $x$-component of generalize momentum. 
After that we should consider three cases:
 \begin{enumerate}
 \item 
 $\kappa < 1$. In this case on large time-scales the first term dominates, 
so we can neglect the second one. The same takes place when $\kappa=1$ but $\dot{x}\sim {\rm const}\neq0$
\item
 $\kappa > 1$. In this case we neglect the first term.
\item
$\kappa = 1$, $\dot{x}\rightarrow 0$. This scenario is possible if $\Delta\propto t$. 
In this case $z(t)=z_0(t)+\delta z(t)$, where $z_0/\Delta =$ const.
Accordingly, $\delta z\ll z_0$ $z_0''(t)=0$.  
\end{enumerate}  
In the first case we have
  \begin{equation}
  \label{App.case1}
  \ddot{z}=\Lambda\dfrac{v_0+V}{t}\tanh(z/\Delta),
  \end{equation}
  where $V=\Lambda\dfrac{\Delta_0}{t_0}\ln[2\cosh(x/0/t_0)]$.
This equation cannot be solved analytically for arbitrary $\kappa$ even in $z/\Delta\ll 1$ limit. For $v+V_0<0$ the numerical solution of this equation has shown at (\ref{fig:App.sol}).
%However if almost all particle $v_0+V$ and, since $v_0=V_J+V_T$, where $V_{x}=\frac{1}{2}j_x/(en)$-current velocity which is mean velocity of one of the component and $V_T$ is thermal velocity. Since $j_x<0$, $V_J<0$ and we should required, that $V_J+V<V_T$ in order to make $v_0+V<0$ for almost all particles. The numerical simulations show, that in this condition particles will move inside sheet. When $v_0+V>0$ the particle motion apparently it can be described by equations, from case 3.

In second case we obtain following equation:
  \begin{equation}
\ddot{z} = -\Lambda^2\dfrac{\Delta}{t^2}\tanh(z/\Delta)\log[2\cosh(z/\Delta)].
\end{equation} 

This equation is the similar to (\ref{eq.x.full.ort}). As we already 
establish above, it describes particle orbit with $t^{1/2}$ radius growth rate.
%Thus, one can conclude that the only solution with particle orbit expanding at the same rate as the sheet thickness corresponds to $\kappa = 1$ when 
%$\dot{x}\rightarrow 0$. In this case we obtain 
%following equation for $\delta z$:

Finally for $\kappa = 1$ we have
\begin{equation}
\ddot{\delta z}=-\Lambda^2\dfrac{\delta z}{t^2}\tanh^2(x_0/\Delta).
\end{equation}
The solution of this equation is 
\begin{equation}
\delta z=\sqrt{t}\cos\left[\Lambda
\tanh\left(\dfrac{x_0}{\Delta(t)}\right)\log t\right].
\end{equation}
  This solution describes radial motion of particle experiencing Larmor's motion in declining field.
%As we see, for align current sheet the have the same scaling 
%$\delta z \propto \sqrt{t}$ as for orthogonal one.

%So there are two types of particles in align case: one of them has $v_0+\Lambda\dfrac{\Delta_0}{t_0}\ln[2\cosh(z_0/\Delta_0)]>0$ and the other has $v_0+\Lambda\dfrac{\Delta_0}{t_0}\ln[2\cosh(z_0/\Delta_0)]<0$

%It is obvious to consider the firs one particles inside current sheet and the others outside it. So for self-consistent solution we should require $V_J>\dfrac{eB_0\Delta_0}{m_ec}$ or that particles will gain less momentum, that they initially had:
%\begin{equation}
%\delta p_z<p_z
%\end{equation}

\bsp
\label{lastpage}

\end{document}



%%%%%%%%%%%%%%%%%%%%%%%%%%%%%%%%%%%%%%%%%%%%%%%%%%%%%%%%%%%%%%%%%%%
%%%%%%%%%%%%%%%%%%%%%%%%%%%%%%%%%%%%%%%%%%%%%%%%%%%%%%%%%%%%%%%%%%%
%%%%%%%%%%%%%%%%%%%%%%%%%%%%%%%%%%%%%%%%%%%%%%%%%%%%%%%%%%%%%%%%%%%
%%%%%%%%%%%%%%%%%%%%%%%%%%%%%%%%%%%%%%%%%%%%%%%%%%%%%%%%%%%%%%%%%%%
%%%%%%%%%%%%%%%%%%%%%%%%%%%%%%%%%%%%%%%%%%%%%%%%%%%%%%%%%%%%%%%%%%%



\subsection{Particle acceleration and the value of current sheet thickness}

In this Section, we estimate current sheet thickness and particle acceleration. We try to use as less number of assumptions as possible. Our assumptions include the following:

1. Maxwell equation:
\begin{equation}
    \frac{B}{\Delta} = \frac{8\pi}{c}n_0 e v_z;
\end{equation}

2. Definition of multiplicity outside the current sheet
\begin{equation}
    \sigma_{\rm M} = \frac{\Omega e B_{\rm LC} R_{\rm LC}^2}{4 \lambda m_e c^3};
\end{equation}

3. Concentration of particles inside the current sheet in laboratory frame:
\begin{equation}
    n_{\rm lab}^{\rm cs} (r) = \lambda_{\rm cs} n_{\rm GJ} (R_{\rm LC}) (r/R_{\rm LC})^{-2};
\end{equation}

4. Lorenz transforms of magnetic field and concentration into comoving reference frame
\begin{equation}
    n_0 = n_{\rm lab} \Gamma^{-1}, \qquad B = B_{\rm LC} (t_0/t) \Gamma^{-2};
\end{equation}

5. Approximate amplitude of particle acceleration along the current sheet:
\begin{equation}
p_z \sim e B \Delta/c.
\end{equation}

If one sets $r = r_{\rm F} = \sigma_{\rm M}^{1/3} R_{\rm LC}$, and $t = t_0$, then the unknown parameters of the system are $\Delta_0$, $v_z$, $B_0$, $B_{\rm LC}$, $n_0$, $n_{\rm lab}$. The known parameters are $\lambda$, $\lambda_{\rm cs}$, $\sigma_{\rm M} = \Gamma^3$, $R_{\rm LC}$. Having 6 equations for 6 unknowns, the system can be solved. 

As a result, current sheet thickness equals
\begin{equation}
\label{eq:delta0}
\Delta = R_{\rm LC} \frac{t}{t_0} \frac{\Gamma}{4 \lambda_{\rm cs}} \frac{c}{v_z},
\end{equation}
while the acceleration estimate yields 
\begin{equation}
\label{eq:pz}
p_z v_z \sim m_e c^2 \sigma_{\rm M}^{2/3} \frac{\lambda}{\lambda_{\rm cs}}.
\end{equation}
The total number of particles in the current sheet
\begin{equation}
\label{eq:ntot}
    N_{\rm tot} (t) \propto n(t) \Delta(t) t^2 \propto \Delta(t)\lambda_{\rm cs}(t) \propto t/v_z(t).
\end{equation}

The analysis above lacks the value of $\lambda_{\rm cs}$ which is treated as a free parameter. However, we can make some conclusions based on the limiting case.

\subsubsection{Case 1: low multiplicity, ultra-relativistic sheet}

If we assume that the multiplicity of the current sheet is low, of the order of the multiplicity outside the sheet $\lambda_{\rm cs} \sim \lambda$, then $p_z v_z \sim m_e c^2 \sigma_{\rm M}^{2/3} \gg m_e c^2$ and $v_z \sim c$. Then, from equations (\ref{eq:delta0}) and (\ref{eq:ntot}), one gets
\begin{equation}
    N_{\rm tot} \propto t, \qquad \Delta_0 \propto t/\lambda_{\rm cs}(t).
\end{equation}
As $N_{\rm tot}$ does not depend on the $\lambda_{\rm cs}$, it does not constrain the multiplicity of the current sheet. However, if we assume that the total number of particles in the sheet can only be changed by trapping the particles in the expanding current sheet, $N_{\rm tot} \propto \Delta_0$, then $\Delta_0 \propto t$ and $\lambda_{\rm cs} = \rm const$. This is the case when current sheet multiplicity equals to the multiplicity outside and particles just get trapped in the sheet. The parameters of such sheet are:
\begin{eqnarray}
B &\propto& t^{-1};\\
n &\propto& t^{-2};\\
\Delta_0 &\propto& t;\\
v_z &\sim& c \propto {\rm const};\\
p_z &\sim& m c \sigma_{\rm M}^{2/3} \propto {\rm const};\\
\lambda_{\rm cs} &=& \lambda = {\rm const};\\
N_{\rm tot} &\propto& t.
\end{eqnarray}

One needs to note, that in such ultra-relativistic current sheet Lorenz factor of particles $\gamma \sim \sigma_{\rm M}^{2/3} \gg \Gamma =\sigma_{\rm M}^{1/3}\gg 1$. Such motion breaks the assumptions we made when deriving the fields in comoving reference frame.

The typical Larmor radius of particles in this case
\begin{equation}
r_{\rm L} \propto p_z/B \propto t \propto \Delta_0.
\end{equation}

\subsubsection{Case 2: high multiplicity, non-relativistic sheet}

The situation is quite different in high multiplicity case $p_z v_z \ll m_e c^2$. This requires $\lambda_{\rm cs}\gg\lambda \sigma_{\rm M}^{2/3}$. For such case one gets
\begin{equation}
v_z \propto \lambda_{\rm cs}^{-1/2},
\end{equation}
which implies
\begin{equation}
N_{\rm tot}\propto  t \lambda_{\rm cs}^{1/2}, \qquad \Delta_0 \propto t \lambda_{\rm cs}^{-1/2}.
\end{equation}
One can see that the case $N_{\rm tot} \propto \Delta_0$ is realised if $\lambda_{\rm cs} = \rm const$. However, if the multiplicity if the current sheet is so large, trapping a small number of outside particles does not influence the total number of particles. So, the case $N_{\rm tot} \propto \Delta_0$ does not really make sense. Instead, $N_{\rm tot} \propto \rm const$ is more reasonable. For such case $\Delta_0 \propto t^2$. The parameters of such current sheet
\begin{eqnarray}
B &\propto& t^{-1};\\
n &\propto& t^{-4};\\
\Delta_0 &\propto& t^2;\\
v_z & \propto& t;\\
p_z &\propto& t;\\
\lambda_{\rm cs} &\propto& t^{-2};\\
N_{\rm tot} &\propto& \rm const.
\end{eqnarray}

Such solution does not violate the assumptions made when deriving the fields in the comoving frame. However, such solution can only be realised during the linear stage of particle acceleration.

The typical Larmor radius of particles in this case
\begin{equation}
r_{\rm L} \propto p_z/B \propto t^2 \propto \Delta_0.
\end{equation}
%%%%%%%%%%%%%%%%%%%%%%%%%%%%%%%%%%%%%%%%%%%%%%%%%%%%%%%%%%%%%%%%%%%%%%%%%%%%%%%%%%%%%%%%%%%%%%%%%%%%%%%%%%%%%%%%%%%%%%%%%%%%%%%%%%%%%%%%%%%%%%%%%%%%%%%%%%%%%%%%%%%%%%%%%%%%%%%%%%%%%%%%%%%%%%%%%%%%%%%%%%%%%%%%%%%%%%%%%%%%%%%%%%%%%%%%%%%%%%%%%%%%%%%%%%%%%%%%%%%%%%%%%%%%%%%%%%%%%%%%%%%%%%%%%%%%%%%%%%%%%%%%%%%%%%%%%%%%%%%%%%%%%%%%%%%%%%%%%%%%%%%%%%%%%%%%%%%%%%%%%%%%%%%%%%%%%%%%%%%%%%%%%%%%%%%%%%%%%%%%%%%%%%%%%%%%%%%%%%%%%%%%%%%%%%%%%%%%%%%%%%%%%%%%%%%%%%%%%%%%%%%%%%%%%%%%%%%%%%%%%%%%%%%%%%%%%%%%%%%%%%%%%%%%%%%%%%%%%%%%%%%%%%%%%%%%%%%%%%%%%%%%%%%%%%%%%%%%%%%%%%%%%%%%%%%%%%%%%%%%%%%%%%%%%%%%%%%%%%%%%%%%%%%%%%%%%%%%%%%%%%%%%%%%%%%%%%%%%%%%%%%%%%%%%%%%%%%%%%%%%%%%%%%%%%%%%%%%%%%%%%%%%%%%%%%%%%%%%%%%%%%%%%%%%%%%%%%%%%%%%%%%%%%%%%%%%%%%%%%%%%%%%%%%%%%%%%%%%%%%%%%%%%%%%%%%%%%%%%%%%%%%%%%%%%%%%%%%%%%%%%%%%%%%%%%%%%%%%%%%%%%%%%%%%%%%%%%%%%%%%%%%%%%%%%%%%%%%%%%%%%%%%%%%%%%%%%%%%%%%%%%%%%%%%%%%%%%%%%%%%%%%%%%%%%%%%%%%%%%%%%%%%%%%%%%%%%%%%%%%%%%%%%%%%%%%%%%%%%%%%%%%%%%%%%%%%%%%%%%%%%%%%%%%%%%%%%%%%%%%%%%%%%%%%%%%%%%%%%%%%%%%%%%%%%%%%%%%%%%%%%%%%%%%%%%%%%%%%%%%%%%%%%%%%%%%%%%%%%%%%%%%%%%%%%%%%%%%%%%%%%%%%%%%%%%%%%%%%%%%%%%%%


******************************************

On the other hand, we can stick with the Michel fields without
coefficients (orthogonal case):
\begin{eqnarray}
B_y &=& B_0 \frac{t_0}{t} h(x,t),\\
E_x &=& B_0 \frac{z}{ct} \frac{t_0}{t} h(x,t).
\end{eqnarray}
As was shown in Sect.~\ref{sect:width}, the $z$-component of the
electric field arises and is equal to
\begin{equation}
E_{z} 
= B_{0}\frac{t_0}{c t} \int\frac{\partial h}{\partial t} {\rm d}x.
\end{equation}

In realistic case 
\begin{equation}
h(x,t) = \tanh\left(\frac{x}{L_0 (t/t_0)^d} \right),
\end{equation}
the condition $E_z(\infty) = 0$ fixes the electric field in the center
of the sheet:
\begin{equation}
E_z(0) = B_0 d \frac{L_0}{c t_0} \left(\frac{t}{t_0} \right)^{d-2}\log 2.
\end{equation}
As a result, for $\kappa < 1$ the total energy gain 
$\Delta {\cal E} = \int {\bf E  v}{\rm d}t$ can be evaluated as
\begin{equation}
\Delta {\cal E} \sim e B_0 L_0.
\end{equation}

In aligned case the solution has different form:
\begin{equation}
E_z = B_0\frac{t_0}{t} \left(\frac{1}{c} \frac{\partial h}{\partial t}
+ \frac{z}{ct}\frac{\partial h}{\partial z}\right)x.
\end{equation}

For
\begin{equation}
h(z,t) = \tanh\left(\frac{z}{L_0 (t/t_0)^d} \right),
\end{equation}
the solution is zero at $z=0$ and $z=\infty$, and has a maximum value
of:
\begin{equation}
E_{z}^{\rm max} = 0.48 B_0 (1-d) \frac{x}{c t} \left(\frac{t_0}{t}
\right)\cdot C,
\end{equation}
where numerical coefficient $0.48$ is, approximately,
maximum of a function $z \cosh(z)^{-2}$. For $x\sim L_0$ it has the
same order of magnitude, as a field in orthogonal case.



*********************************

\section{Two-fluid MHD effects}
\label{2F}

Let us consider axisymmetric outflow of a two-component plasma. The structure 
of the flow is described by Maxwell equations and equation of motion
\begin{equation}
\label{Appendix: Maxwell+Motion common}
\begin{split}
&\nabla\cdot\bm{E}=4\pi\rho,\hspace{10pt}\nabla\times\bm{E}=0,\\
\hspace{3pt}
&\nabla\cdot\bm{B}=0,
\nabla\times\bm{B}=\displaystyle\frac{4\pi}{c}\bm{J},\\
&(\bm{v}^{\pm},\nabla)\bm{p}^{\pm} 
= \pm e\left[\bm{E} + \frac{\bm{v}^{\pm}}{c} \times \bm{B}\right].
\end{split}
\end{equation}
Here we consider the fields far from the current sheet, so we put down $\Theta(...)$ factor 
in further calculations.

Asymptotic behaviour of fields in the force-free approximation ($\gamma=\infty$) can be obtained 
from general formula (\ref{dPsi}) by taking only first term: $\Psi(r,\theta)=\Psi_0(\theta)$. 
After that one could obtain following expressions for electromagnetic field:
\begin{align}
&B_r^0=B_s\left(\displaystyle\frac{R_s}{r}\right)^2\displaystyle\frac{F(\theta)}{\sin\theta}
,\\
&B_{\theta}^0=0
,\\
&B_{\varphi}^0=-B_s\dfrac{\Omega R_s}{c}\displaystyle\frac{R_s}{r} F(\theta)
,\\
&\rho_{\rm e}^0=\displaystyle\frac{\nabla\cdot\bm{E}}{4\pi}=-\displaystyle\frac{B_s}{4\pi r}
\dfrac{\Omega R_s}{c}\displaystyle\frac{R_s}{r} D_{\theta}^+F(\theta)
,\\
&J_r^0=\displaystyle\frac{c}{4\pi}(\nabla\times\bm{B})_r
=\displaystyle\frac{cB_s}{4\pi r}\dfrac{\Omega R_s}{c}D_{\theta}^+F(\theta),
\\
&J_{\theta}^0=0,
\\
&J^0_{\varphi}=\displaystyle\frac{cB_s}{4\pi r}\left(\displaystyle\frac{R_s}{r}\right)^2D_{\theta}^-F(\theta),
\end{align}
where $D_{\theta}^{\pm}F(\theta)=F'(\theta)\pm F(\theta)\cot\theta$. The choise $F(\theta)=\sin\theta$ 
corresponds to Michel solution.

It is convenient to introduce the electric field potential $\Phi(r,\theta)$, so that 
$\bm{E}=-\nabla \Phi$ and
\begin{equation}
\Phi^{0} = -\dfrac{\Omega R_s^2B_s}{c}\Psi_0(\theta).
\end{equation}
Finally, the number densities can be written as
\begin{eqnarray}
&n^{+} = \dfrac{B_s}{4\pi R_s e}\dfrac{R_s\Omega}{c}
\left(\dfrac{R_s}{r}\right)^2\left[\lambda-\dfrac{1}{2}D_{\theta}^+F(\theta)\right],
\\
&n^{-} = \dfrac{B_s}{4\pi R_s e}\dfrac{R_s\Omega}{c}
\left(\dfrac{R_s}{r}\right)^2\left[\lambda+\dfrac{1}{2}D_{\theta}^+F(\theta)\right],
\end{eqnarray}
where again $\lambda \sim 10^{3}$--$10^{5}$ is the multiplication parameter. 
%In what follows we consider $\lambda={\rm const}$. 
In general case we  have the $\varphi$-component of current, so we cannot neglect 
toroidal particle motion. Thus, in force-free limit the particle velocity can be 
written as
\begin{equation}
\frac{v_{r}^{0}}{c} = 
1-\dfrac{1}{2}\left(\dfrac{c}{\Omega r}\right)^2
\left[\dfrac{D^-_{\theta}F(\theta)}{\lambda}\right]^2.
\end{equation}
%Or we may put $v_r^0=c$ since our solution correct only up to $r^{-2}$ terms.
%
%\begin{align}
%&v_{\theta}^0=0
%,\\
%&v_{\phi}^0=c\dfrac{c}{\Omega r}\dfrac{D_{\theta}^-F(\theta)}{\lambda}.
%\end{align}

For further convenience we will denote $\lambda^{-1} \, D_{\theta}^-F(\theta)$ as $b(\theta)$.
Then one can seek the first-order correction in the following form
\begin{align}
\label{Appendix: Corrections.first}
&n^{\pm}=\dfrac{B_s}{4\pi R_s e}\dfrac{R_s\Omega}{c}
\left(\dfrac{R_s}{r}\right)^2
\left[\lambda\mp\dfrac{1}{2}D_{\theta}^+F(\theta)+\eta^{\pm}(r,\theta)\right],
\\
&\Phi(r,\theta)=\dfrac{\Omega R_s^2B_s}{c}[\Psi_0(\theta)+ q(r,\theta)],
\\
&\Psi(r,\theta)=\dfrac{\Omega R_s^2B_s}{c}[\Psi_0(\theta)+\varepsilon f(r,\theta)],
\\
&v_{r}^{\pm}=c\left[1+\xi^{\pm}_{r}(r,\theta)\right],
\\
&v_{\theta}^{\pm}=c\xi^{\pm}_{\theta}(r,\theta),
\\
&v_{\varphi}^{\pm}=c\left[b(\theta)\dfrac{c}{r\Omega}+\xi^{\pm}_{\varphi}(r,\theta)\right],
\\
&B_{r}=B_s\dfrac{R_s^2}{\sin\theta r^2}
\left[F(\theta)+\varepsilon\dfrac{\partial f}{\partial\theta}\right],
\\
&B_{\theta}=-\varepsilon\dfrac{B_sR_s^2}{r\sin\theta}\dfrac{\partial f}{\partial \theta},
\\
&B_{\varphi}=-B_s\dfrac{\Omega R_s}{c}\dfrac{R_s}{r}\left[F(\theta)+\zeta(r,\theta)\right],
\\
&E_{r}=-\dfrac{\Omega B_s R_s^2}{c}\dfrac{\partial q}{\partial r},
\\
\label{Appendix: Correction.last}
&E_{\theta}=-\dfrac{\Omega R_s^2B_s}{cr}\left[F(\theta)+\dfrac{\partial q}{\partial \theta}\right].
\end{align}

 Substituting now expansions (\ref{Appendix: Corrections.first})--(\ref{Appendix: Correction.last}) 
into (\ref{Appendix: Maxwell+Motion common}) we obtain the following system of equations
\begin{align}
\label{Appendix: Faradey law.phi}
&D_{\theta}^{+}\zeta=-2\delta\eta
+2\left[\lambda\delta\xi_r-\dfrac{1}{2}D_{\theta}^+F(\theta)(\xi^+_r+\xi_r^-)\right], \\
\label{Appendix: Gauss law}
&2\delta\eta + \dfrac{\partial}{\partial r}\left(r^2\dfrac{\partial q}{\partial r}\right)
+D^+_{\theta}\dfrac{\partial q}{\partial\theta}=0,\\
\label{Appendix: Faradey law.theta}
&\dfrac{\partial \zeta}{\partial r} = \dfrac{2}{r}\left[\lambda\delta\xi_{\theta}-\dfrac{1}{2}D_{\theta}^+F(\theta)(\xi^+_{\theta}+\xi_{\theta}^-)\right],\\
\label{Appendix: Faradey law.r}
&\dfrac{\varepsilon}{\sin\theta}\left(\dfrac{\partial^2 f}{\partial r^2}+\dfrac{1}{r^2}D_{\theta}^-\dfrac{\partial f}{\partial\theta}\right)=\nonumber\\
&=-2\dfrac{\Omega}{rc}\left[\lambda\delta\xi_{\varphi}-\dfrac{1}{2}D_{\theta}^+F(\theta)(\xi^+_{\varphi}+\xi_{\varphi}^-)\right],\\
\label{Appendix: Motion.theta}
&\dfrac{\partial}{\partial r}\left(\xi^{\pm}_{\theta}\gamma^{\pm}\right)+\dfrac{\xi_{\theta}^{\pm}\gamma^{\pm}}{r}=\nonumber\\
&=\pm 4\lambda\sigma_{\rm M}\left[-\dfrac{1}{r}\dfrac{\partial q}{\partial\theta}+\dfrac{\zeta}{r}-\dfrac{F(\theta)}{r}\xi^{\pm}_r+\dfrac{c}{\Omega r^2}\dfrac{F(\theta)}{\sin\theta}\xi^{\pm}_{\varphi}\right],\\
\label{Appendix: Motion.r}
&\dfrac{\partial}{\partial r}(\gamma^{\pm})=\mp 4\lambda\sigma_{\rm M}\left[\dfrac{\partial q}{\partial r}+\dfrac{F(\theta)}{r}\xi^{\pm}_{\theta}\right],\\
\label{Appendix: Motion.phi}
&\dfrac{\partial}{\partial r}\left(\xi^{\pm}_{\varphi}\gamma^{\pm}\right)+\dfrac{\xi_{\varphi}^{\pm}\gamma^{\pm}}{r}=
\mp 4\lambda\sigma_{\rm M}\left[\varepsilon\dfrac{c}{\Omega r\sin\theta}\dfrac{\partial f}{\partial r}+\dfrac{c}{\Omega r^2}\dfrac{F(\theta)}{\sin\theta}\xi^{\pm}_{\theta}\right],
\end{align}
where $\delta A=A^{+}-A^{-}$. In these equations all the deflecting function 
are supposed to be $\ll 1$, and we have neglected the terms $R_{\rm L}^{2}/r^{2}$ 
times less than the leading ones. We also have neglected $b(\theta)$ in the left-hand 
side of equations (\ref{Appendix: Motion.theta})--(\ref{Appendix: Motion.phi}) 
since $b\simeq \lambda^{-1} \ll1$.
 
General system of equations (\ref{Appendix: Gauss law})--(\ref{Appendix: Motion.phi}) has several 
integrals. First, Eqns. (\ref{Appendix: Faradey law.theta}) and (\ref{Appendix: Motion.r}) lead to 
\begin{equation}
\label{Appendix: Energy Cons.}
\zeta-\dfrac{2D^+_{\theta} F(\theta)}{F(\theta)} q 
+\dfrac{2 \lambda(\gamma^{+}+\gamma^{-})
-D^{+}_{\theta}F(\theta)\Delta\gamma}{4\sigma_{\rm M}\lambda F(\theta)}=M(\theta),
\end{equation}
 which is simply the energy conservation law. On the other hand, Eqns. 
(\ref{Appendix: Motion.r}) and (\ref{Appendix: Motion.phi}) result in 
\begin{equation}
q - \varepsilon f\pm\dfrac{1}{4\lambda\sigma_{\rm M}}\gamma^{\pm}
\left(1-\dfrac{\Omega r\sin\theta}{c}\xi^{\pm}_{\varphi} \right)=C^{\pm}(\theta),
\end{equation}
which can be consider as a conservation of $z$-component of the angular momentum.
Besides, as $\sigma_{\rm M}\lambda\gg 1$, one can neglect in Eqns. 
(\ref{Appendix: Motion.theta})--(\ref{Appendix: Motion.phi}) their left-hand sides. 
In this approximation  we have $\xi^+=\xi^-=\xi$ and $\gamma^+=\gamma^-=\gamma$, 
so that  
\begin{equation}
\label{Appendix:zeta.approx}
\zeta = \dfrac{\partial q}{\partial\theta}
+F(\theta)\xi_r-\dfrac{c}{\Omega r}\dfrac{F(\theta)}{\sin\theta}\xi_{\varphi},
\end{equation}  
\begin{equation}
\dfrac{\partial q}{\partial r}+\dfrac{F(\theta)}{r}\xi_{\theta}=0
\end{equation}
 and
 \begin{equation}
\varepsilon f = q. 
\end{equation}
The last expression implies that within this approximation magnetic field lines
remain equi-potential. Finally, one can find that
\begin{equation}
\xi_{\varphi}<\dfrac{A}{r\sin\theta}.
\end{equation}
Hence, we can neglect the last term in Eqn. (\ref{Appendix:zeta.approx}) since it 
becomes proportional to $r^{-2}$ compared to the others terms. Thus, we have
\begin{equation}
\zeta = \dfrac{\partial q}{\partial\theta}+F(\theta)\xi_r,
\end{equation}
which can be rewritten as $B_{\varphi}-E_{\theta}=(1-\beta)B_{\varphi}$, or as
$E_{\theta}=\beta B_{\varphi}$.

\section{Numerical simulation}
\label{Appendix.sect.NS}

It is interesting to see, how do particles move in such a configuration of a time-dependent
current sheet. For this, we produce simple numerical simulations where we set external
electromagnetic field and look at the particle motion without investigating the influence 
of the particles on the field.

It is naturally to express momentum in $m_{\rm e} c$, coordinates in units of $\Delta_0$ and time in 
units of $t_0$. Then, it is convenient to express energy in $m_{\rm e} c^2$ and velocity in units of
$\Delta_0/t_0$. It allows us to move the the following dimensionless variables
\begin{align}
\tau &\equiv t/t_0,\\
\xi &\equiv x/\Delta_0,\\
\zeta &\equiv z/\Delta_0,\\
\pi_x &\equiv P_x/(m_{\rm e} c)\\
\pi_z &\equiv P_z/(m_{\rm e} c)\\
h\ &\equiv \mathcal{H}/(m_{\rm e} c^2)
\end{align}
In particular, we can rewrite dimensionless Hamiltonians as
\begin{equation}
\begin{split}
h_{\rm ort}=\sqrt{1+(\pi_z-Q\tau^{d-1}\log[2\cosh(\xi\tau^{-d})])+\pi_x^2}-
\\ - Q\dfrac{\zeta\tau^{d-2}}{\lambda\gamma}\log[2\cosh(\xi\tau^{-d})].
\end{split}
\end{equation}
\begin{equation}
\begin{split}
h_{\rm al}=\sqrt{1+(\pi_x + Q\tau^{d-1}\log[2\cosh(\zeta\tau^{-d})])+\pi_z^2}.
\end{split}
\end{equation}
Here $Q$ is the charge of the particles ($-1$ for electrons and $1$ for positrons).

As to dimensionless equation of motion, they look like
\begin{align}
\pi'_x=-\lambda\gamma\dfrac{\partial h}{\partial\xi}& 
& \pi_z'=-\lambda\gamma\dfrac{\partial h}{\partial\zeta}\\
\xi'=\lambda\gamma\dfrac{\partial h}{\partial\pi_x}& 
&\zeta'=-\lambda\gamma\dfrac{\partial h}{\partial\pi_z}\\
\end{align}
But actually we have the system of four differential equations for orthogonal case only
as for aligned case $\pi_x =$ const. Both of systems have only one numerical parameter 
$\lambda \gamma\simeq 10^6-10^8$, which is rpm for characteristic time. 

******************************
******************************
******************************
******************************
******************************
******************************
******************************
******************************
******************************
\section{Comparison with numerical simulation}
\label{sect:Num}

%To check the theoretical predictions 
In this section we present results of numerical simulations of particle motion inside self-consistent current sheet. Moreover we show that parameter $N$ (which is simply ratio of average Larmor radius of particles for $t=t_0$ to $\Delta_0$-initial sheet thickness), which wasn't determine for both cases should be equal to 1.

Appropriate equations of motion was formulated previously~\ref{sect:solAn}.
 We start with orthogonal case.
 
First of all, Fig.~\ref{fig:momentum} shows dependence 
of the momentum gain via additional electric field for orthogonal pulsar for 
presented which was estimated in Sect.~\ref{sect.Est} on initial  $x$ coordinate. Since the period of motion along $x$ axis is much smaller characteristic time scale, we consider all particles moving along $z$ axes. As one can see, the result for arbitrary initial position of particle is slightly differs from result from one for particle moving along $z$ axis.

\begin{figure}
\centering
\includegraphics[scale=0.15]{dP(x).jpeg}
\caption{Particle momentum gained by additional electric field against initial particle coordinate. }
\label{fig:momentum}
\end{figure}

Next we will numerically investigate two cases of initial condition from (\ref{sect.Est}) For $N=1$ and $N=10$ (\ref{fig:sheetort}), (\ref{fig:sheetort1}).

From Fig.\ref{fig:sheetort} one can see that there is a breaking point where amplitude of particle oscillation change it growth rate. In order to explain it we should take a closer look at definition of integral of motion (\ref{Iinv}):
\begin{equation}
    I=m_ez-\dot{z}t+\Delta(t)\dfrac{eB_0R_{\rm L}}{c^2m}\log\left[2\cosh\left(\dfrac{x}{\Delta}\right)\right]
\end{equation}
For particle which starts it's trajectory from $(0,0)$ :
\begin{equation}
    I=-m_ev_{z}(t_0)+\Delta(t)\dfrac{eB_0R_{\rm L}}{c^2m}\log2
\end{equation}
When $N\gg1$ it is defines by first term:
\begin{equation}
I\simeq \dfrac{m_eR_{L}}{\lambda^*}
\end{equation}
Thus the condition required to neglect this constant compared to $\Delta(t)\frac{eB_0R_{\rm L}}{c^2m}\log\left[2\cosh\left(\frac{x}{\Delta}\right)\right]$  is following:
\begin{equation}
(t/t_0)^{\kappa}>N^2
\end{equation}
Which is well-agreed with position of breaking point on (\ref{fig:sheetort}). 

Fig.\ref{fig:sheetort1}  shows that in case of $N=1$ particle trajectory will expand by power law from beginning.

\begin{figure}
\centering
\includegraphics[scale=0.15]{Ort.jpeg}
\caption{$x$-coordinate of the particle motion (oscillating blue line) and the thickness of the sheet  
 with $\kappa = 0.5$ $N=10$
 for orthogonal case. Though asymptotically sheet expanses as predicted for some time it experienced additional growth}
\label{fig:sheetort}
\end{figure}
\begin{figure}
\centering
\includegraphics[scale=0.15]{OrtlessN.jpeg}
\caption{$x$-coordinate of the particle motion (oscillating blue line) and the thickness of the sheet  
 with $\kappa = 0.5$ $N=1$ for orthogonal case. In this case power growth of sheet thickness is valid at every timescale }
\label{fig:sheetort1}
\end{figure}

Investigation of kinetic effects is beyond scope of this article. However we can introduce "typical particle" in following sense:

From analytical consideration (\ref{sect:solAn}) it is clear, that motion along axis $z$ doesn't effect motion along $x$ axis. So we can introduce two different parameters $V_z$ which is mean particle velocity and $V_T$ which is thermal velocity. From (\ref{sect.Est}) one can obtain, that $V_z/(eB_0R_{\pm L}\Delta_0c^2t_0m_e)=4N^2$ and $V_T/(eB_0R_{\pm L}\Delta_0c^2t_0m_e)=2N$ So our typical particle will start with $\dot{z}=V_Z(N)$, $\dot{x}=V_T(N)$ and we put $x=z-=0$. It is possible now to investigate dependence of amplitude of such particle on $N$ and determine N for which $Amp(t)/\Delta=1$. As we can see from Fig.\ref{fig:amport} $N\simeq 1$ 

\begin{figure}
\centering
\includegraphics[scale=0.15]{Amp(N)ort.jpeg}
\caption{Amplitude of particle oscillations to sheet thickness for different $N$,  for orthogonal case. Particles starts with following initial conditions $x=z=0$  $\dot{z}=V_Z(N)$, $\dot{x}=V_T(N)$}
\label{fig:amport}
\end{figure}

Now we move to align case. 

It is clear, that "typical particle" method, which we introduced for orthogonal case works in align case as well. 

As it was stressed out in section (\ref{sect:al}) analytical consideration doesn't exclude values of $N<1$ (ratio between particle gyroradius and sheet thickness). However for such values it became impossible to construct self-consistent solution for both positrons and electrons. Fig.\ref{fig:sheetalp} and Fig.\ref{fig:sheetale} show trajectories for typical positron and electron with the same thermal velocity and opposite current velocity. Positron is moving along border of sheet in a way we expected for self-consistent solution. However electron starts to oscillate in side the sheet which is not self consistent. For $N>1$ this problem doesn't occur and  self-consistent solution can be obtained.

Determination of $N$ in align case are similar to one in orthogonal case finding dependence of mean $z(t)/t$ on $N$ and compare it to thickness of sheet determines the value of $N$ Fig.\ref{fig:ampal}
\begin{figure}
\centering
\includegraphics[scale=0.15]{sheetal}
\caption{Dependence $z$ component of the particle trajectory (blue) for aligned case and sheet 
boundary (red) on time. $N=0.1$ for positron}
\label{fig:sheetalp}
\end{figure}
\begin{figure}
\centering
\includegraphics[scale=0.15]{sheetalsmallN}
\caption{Dependence $z$ component of the particle trajectory (blue) for aligned case and sheet 
boundary (red) on time. $N=0.1$ for electon}
\label{fig:sheetale}
\end{figure}

\begin{figure}
\centering
\includegraphics[scale=0.15]{Amp(N)al}
\caption{Dependence of mean $z/\Delta$ on $N$ }
\label{fig:ampal}
\end{figure}

As result of numerical simulations we establish that in both orthogonal and align cases $N\simeq1$ or initial particle gyroradius is the same order as sheet's width.he same order as sheet's width.
