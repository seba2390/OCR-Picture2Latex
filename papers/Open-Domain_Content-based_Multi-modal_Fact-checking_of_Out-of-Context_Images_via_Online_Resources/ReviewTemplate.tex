% CVPR 2022 Paper Template
% based on the CVPR template provided by Ming-Ming Cheng (https://github.com/MCG-NKU/CVPR_Template)
% modified and extended by Stefan Roth (stefan.roth@NOSPAMtu-darmstadt.de)

\documentclass[10pt,twocolumn,letterpaper]{article}

%%%%%%%%% PAPER TYPE  - PLEASE UPDATE FOR FINAL VERSION
\usepackage{cvpr}      % To produce the REVIEW version
%\usepackage{cvpr}              % To produce the CAMERA-READY version
%\usepackage[pagenumbers]{cvpr} % To force page numbers, e.g. for an arXiv version

% Include other packages here, before hyperref.
\makeatletter
\@namedef{ver@everyshi.sty}{}
\makeatother
\usepackage{tikz}

\usepackage{graphicx}
\usepackage{amsmath}
\usepackage{amssymb}
\usepackage{booktabs}
\usepackage{mdframed}
\usepackage{xcolor,soul}
\usepackage{enumitem}
\usepackage{pifont}% 
\usepackage{makecell}
\usepackage{varwidth}
\usepackage{multirow}
\usepackage[super]{nth}
\pdfimageresolution=300

%\documentclass[twocolumn]{article}

%\usepackage{tcolorbox}



% It is strongly recommended to use hyperref, especially for the review version.
% hyperref with option pagebackref eases the reviewers' job.
% Please disable hyperref *only* if you encounter grave issues, e.g. with the
% file validation for the camera-ready version.
%
% If you comment hyperref and then uncomment it, you should delete
% ReviewTempalte.aux before re-running LaTeX.
% (Or just hit 'q' on the first LaTeX run, let it finish, and you
%  should be clear).
\usepackage[pagebackref,breaklinks,colorlinks,hypertexnames=false]{hyperref}


% Support for easy cross-referencing
\usepackage[capitalize]{cleveref}
\crefname{section}{Sec.}{Secs.}
\Crefname{section}{Section}{Sections}
\Crefname{table}{Table}{Tables}
\crefname{table}{Tab.}{Tabs.}



%%%%%%%%% PAPER ID  - PLEASE UPDATE
\def\cvprPaperID{7457} % *** Enter the CVPR Paper ID here
\def\confName{CVPR}
\def\confYear{2022}


\begin{document}
%\definecolor{myOrange}{rgb}{1.0, 0.49, 0.0}
\definecolor{myOrange}{rgb}{1.00, 0.60, 0}
%{1.0, 0.55, 0.0}
\definecolor{myblue}{rgb}{0.0,0.40,0.80}
\newcommand{\model}{\textit{CCN}}
\pdfoutput=1
%{0.216, 0.494, 0.722}
%{0.19, 0.55, 0.91}



\newcommand{\cmark}{\ding{51}}%
\newcommand{\xmark}{\ding{55}}%

\newcommand{\vcenteredinclude}[1]{\begingroup
\setbox0=\hbox{\includegraphics[scale=0.17]{#1}}%
\parbox{\wd0}{\box0}\endgroup}

\newcommand{\mycircle}[2][black,fill=black]{\tikz[baseline=-0.5ex]\draw[#1,radius=#2] (0,0) circle ;}%

\definecolor{applegreen}{rgb}{0.55, 0.71, 0.0}
\definecolor{harlequin}{rgb}{0.25, 1.0, 0.0}
\def\boxit#1{%
  \smash{\color{harlequin}\fboxrule=1pt\relax\fboxsep=4pt\relax%
  \llap{\rlap{\fbox{\vphantom{0}\makebox[#1]{}}}~}}\ignorespaces
}

%%%%%%%%% TITLE - PLEASE UPDATE
\title{Open-Domain, Content-based, Multi-modal Fact-checking of\\Out-of-Context Images via Online Resources}

\author{Sahar Abdelnabi, Rakibul Hasan, and Mario Fritz\\
CISPA Helmholtz Center for Information Security\\
{\tt\small \{sahar.abdelnabi,rakibul.hasan,fritz\}@cispa.de}}
% For a paper whose authors are all at the same institution,
% omit the following lines up until the closing ``}''.
% Additional authors and addresses can be added with ``\and'',
% just like the second author.
% To save space, use either the email address or home page, not both
%\and
%Second Author\\
%Institution2\\
%First line of institution2 address\\
%{\tt\small secondauthor@i2.org}
%}
%\maketitle
\twocolumn[{%
\renewcommand\twocolumn[1][]{#1}%
\maketitle
\begin{center}
    \centering
    \captionsetup{type=figure}
    \includegraphics[width=0.90\textwidth]{figs/teaser.pdf}
    \captionof{figure}{To evaluate the veracity of \textbf{\textcolor{myOrange}{image}}-\textbf{\textcolor{myblue}{caption}} pairings, we leverage \textbf{\textcolor{myOrange}{visual}} and \textbf{\textcolor{myblue}{textual}} evidence gathered by querying the Web. We propose a novel framework to detect the consistency of the claim-evidence (\textbf{\textcolor{myblue}{text}}-\textbf{\textcolor{myblue}{text}} and \textbf{\textcolor{myOrange}{image}}-\textbf{\textcolor{myOrange}{image}}), in addition to the \textbf{\textcolor{myOrange}{image}}-\textbf{\textcolor{myblue}{caption}} pairing. Highlighted evidence represents the model's highest attention, showing a difference in location compared to the query \textbf{\textcolor{myblue}{caption}}.} \label{fig:teaser}
\end{center}%
}]


%%%%%%%%% ABSTRACT

\begin{abstract}
Misinformation is now a major problem due to its potential high risks to our core democratic and societal values and orders. \textbf{Out-of-context} misinformation is one of the easiest and effective ways used by adversaries to spread viral false stories. In this threat, a real image is \textbf{re-purposed} to support other narratives by misrepresenting its context and/or elements. 
The internet is being used as the go-to way to verify information using different sources and modalities. Our goal is an inspectable method that automates this time-consuming and reasoning-intensive process by fact-checking the \textbf{\textcolor{myOrange}{image}}-\textbf{\textcolor{myblue}{caption}} pairing using Web evidence. 
To integrate evidence and cues from both modalities, we introduce the concept of \textbf{`multi-modal cycle-consistency check'} \vcenteredinclude{figs/icon3.pdf} / \vcenteredinclude{figs/icon4.pdf}; starting from the \textbf{\textcolor{myOrange}{image}}/\textbf{\textcolor{myblue}{caption}}, we gather \textbf{\textcolor{myblue}{textual}}/\textbf{\textcolor{myOrange}{visual}} evidence, which will be compared against the other paired \textbf{\textcolor{myblue}{caption}}/\textbf{\textcolor{myOrange}{image}}, respectively. 
%
Moreover, we propose a novel architecture, \textbf{Consistency-Checking Network (CCN)}, that mimics the layered human reasoning across the same and different modalities: the \textbf{\textcolor{myblue}{caption}} vs. \textbf{\textcolor{myblue}{textual evidence}}, the \textbf{\textcolor{myOrange}{image}} vs. \textbf{\textcolor{myOrange}{visual evidence}}, and the \textbf{\textcolor{myOrange}{image}} vs. \textbf{\textcolor{myblue}{caption}}. Our work offers the first step and benchmark for \textbf{open-domain, content-based, multi-modal fact-checking}, and significantly outperforms previous baselines that did not leverage external evidence\footnote{For code, checkpoints, and dataset, check: \url{https://s-abdelnabi.github.io/OoC-multi-modal-fc/}}.


\end{abstract}

%%%%%%%%% BODY TEXT
\vspace{-5mm}
% !TEX root = ../arxiv.tex

Unsupervised domain adaptation (UDA) is a variant of semi-supervised learning \cite{blum1998combining}, where the available unlabelled data comes from a different distribution than the annotated dataset \cite{Ben-DavidBCP06}.
A case in point is to exploit synthetic data, where annotation is more accessible compared to the costly labelling of real-world images \cite{RichterVRK16,RosSMVL16}.
Along with some success in addressing UDA for semantic segmentation \cite{TsaiHSS0C18,VuJBCP19,0001S20,ZouYKW18}, the developed methods are growing increasingly sophisticated and often combine style transfer networks, adversarial training or network ensembles \cite{KimB20a,LiYV19,TsaiSSC19,Yang_2020_ECCV}.
This increase in model complexity impedes reproducibility, potentially slowing further progress.

In this work, we propose a UDA framework reaching state-of-the-art segmentation accuracy (measured by the Intersection-over-Union, IoU) without incurring substantial training efforts.
Toward this goal, we adopt a simple semi-supervised approach, \emph{self-training} \cite{ChenWB11,lee2013pseudo,ZouYKW18}, used in recent works only in conjunction with adversarial training or network ensembles \cite{ChoiKK19,KimB20a,Mei_2020_ECCV,Wang_2020_ECCV,0001S20,Zheng_2020_IJCV,ZhengY20}.
By contrast, we use self-training \emph{standalone}.
Compared to previous self-training methods \cite{ChenLCCCZAS20,Li_2020_ECCV,subhani2020learning,ZouYKW18,ZouYLKW19}, our approach also sidesteps the inconvenience of multiple training rounds, as they often require expert intervention between consecutive rounds.
We train our model using co-evolving pseudo labels end-to-end without such need.

\begin{figure}[t]%
    \centering
    \def\svgwidth{\linewidth}
    \input{figures/preview/bars.pdf_tex}
    \caption{\textbf{Results preview.} Unlike much recent work that combines multiple training paradigms, such as adversarial training and style transfer, our approach retains the modest single-round training complexity of self-training, yet improves the state of the art for adapting semantic segmentation by a significant margin.}
    \label{fig:preview}
\end{figure}

Our method leverages the ubiquitous \emph{data augmentation} techniques from fully supervised learning \cite{deeplabv3plus2018,ZhaoSQWJ17}: photometric jitter, flipping and multi-scale cropping.
We enforce \emph{consistency} of the semantic maps produced by the model across these image perturbations.
The following assumption formalises the key premise:

\myparagraph{Assumption 1.}
Let $f: \mathcal{I} \rightarrow \mathcal{M}$ represent a pixelwise mapping from images $\mathcal{I}$ to semantic output $\mathcal{M}$.
Denote $\rho_{\bm{\epsilon}}: \mathcal{I} \rightarrow \mathcal{I}$ a photometric image transform and, similarly, $\tau_{\bm{\epsilon}'}: \mathcal{I} \rightarrow \mathcal{I}$ a spatial similarity transformation, where $\bm{\epsilon},\bm{\epsilon}'\sim p(\cdot)$ are control variables following some pre-defined density (\eg, $p \equiv \mathcal{N}(0, 1)$).
Then, for any image $I \in \mathcal{I}$, $f$ is \emph{invariant} under $\rho_{\bm{\epsilon}}$ and \emph{equivariant} under $\tau_{\bm{\epsilon}'}$, \ie~$f(\rho_{\bm{\epsilon}}(I)) = f(I)$ and $f(\tau_{\bm{\epsilon}'}(I)) = \tau_{\bm{\epsilon}'}(f(I))$.

\smallskip
\noindent Next, we introduce a training framework using a \emph{momentum network} -- a slowly advancing copy of the original model.
The momentum network provides stable, yet recent targets for model updates, as opposed to the fixed supervision in model distillation \cite{Chen0G18,Zheng_2020_IJCV,ZhengY20}.
We also re-visit the problem of long-tail recognition in the context of generating pseudo labels for self-supervision.
In particular, we maintain an \emph{exponentially moving class prior} used to discount the confidence thresholds for those classes with few samples and increase their relative contribution to the training loss.
Our framework is simple to train, adds moderate computational overhead compared to a fully supervised setup, yet sets a new state of the art on established benchmarks (\cf \cref{fig:preview}).

\section{Related Work}
%\mz{We lack a comparison to this paper: https://arxiv.org/abs/2305.14877}
%\anirudh{refine to be more on-topic?}
\iffalse
\paragraph{In-Context Learning} As language models have scaled, the ability to learn in-context, without any weight updates, has emerged. \cite{brown2020language}. While other families of large language models have emerged, in-context learning remains ubiquitous \cite{llama, bloom, gptneo, opt}. Although such as HELM \cite{helm} have arisen for systematic evaluation of \emph{models}, there is no systematic framework to our knowledge for evaluating \emph{prompting methods}, and validating prompt engineering heuristics. The test-suite we propose will ensure that progress in the field of prompt-engineering is structured and objectively evaluated. 

\paragraph{Prompt Engineering Methods} Researchers are interested in the automatic design of high performing instructions for downstream tasks. Some focus on simple heuristics, such as selecting instructions that have the lowest perplexity \cite{lowperplexityprompts}. Other methods try to use large language models to induce an instruction when provided with a few input-output pairs \cite{ape}. Researchers have also used RL objectives to create discrete token sequences that can serve as instructions \cite{rlprompt}. Since the datasets and models used in these works have very little intersection, it is impossible to compare these methods objectively and glean insights. In our work, we evaluate these three methods on a diverse set of tasks and models, and analyze their relative performance. Additionally, we recognize that there are many other interesting angles of prompting that are not covered by instruction engineering \cite{weichain, react, selfconsistency}, but we leave these to future work.

\paragraph{Analysis of Prompting Methods} While most prompt engineering methods focus on accuracy, there are many other interesting dimensions of performance as well. For instance, researchers have found that for most tasks, the selection of demonstrations plays a large role in few-shot accuracy \cite{whatmakesgoodicexamples, selectionmachinetranslation, knnprompting}. Additionally, many researchers have found that even permuting the ordering of a fixed set of demonstrations has a significant effect on downstream accuracy \cite{fantasticallyorderedprompts}. Prompts that are sensitive to the permutation of demonstrations have been shown to also have lower accuracies \cite{relationsensitivityaccuracy}. Especially in low-resource domains, which includes the large public usage of in-context learning, these large swings in accuracy make prompting less dependable. In our test-suite we include sensitivity metrics that go beyond accuracy and allow us to find methods that are not only performant but reliable.

\paragraph{Existing Benchmarks} We recognize that other holistic in-context learning benchmarks exist. BigBench is a large benchmark of 204 tasks that are beyond the capabilities of current LLMs. BigBench seeks to evaluate the few-shot abilities of state of the art large language models, focusing on performance metrics such as accuracy \cite{bigbench}. Similarly, HELM is another benchmark for language model in-context learning ability. Rather than only focusing on performance, HELM branches out and considers many other metrics such as robustness and bias \cite{helm}. Both BigBench and HELM focus on ranking different language model, while fix a generic instruction and prompt format. We instead choose to evaluate instruction induction / selection methods over a fixed set of models. We are the first ever evaluation script that compares different prompt-engineering methods head to head. 
\fi

\paragraph{In-Context Learning and Existing Benchmarks} As language models have scaled, in-context learning has emerged as a popular paradigm and remains ubiquitous among several autoregressive LLM families \cite{brown2020language, llama, bloom, gptneo, opt}. Benchmarks like BigBench \cite{bigbench} and HELM \cite{helm} have been created for the holistic evaluation of these models. BigBench focuses on few-shot abilities of state-of-the-art large language models, while HELM extends to consider metrics like robustness and bias. However, these benchmarks focus on evaluating and ranking \emph{language models}, and do not address the systematic evaluation of \emph{prompting methods}. Although contemporary work by \citet{yang2023improving} also aims to perform a similar systematic analysis of prompting methods, they focus on simple probability-based prompt selection while we evaluate a broader range of methods including trivial instruction baselines, curated manually selected instructions, and sophisticated automated instruction selection.

\paragraph{Automated Prompt Engineering Methods} There has been interest in performing automated prompt-engineering for target downstream tasks within ICL. This has led to the exploration of various prompting methods, ranging from simple heuristics such as selecting instructions with the lowest perplexity \cite{lowperplexityprompts}, inducing instructions from large language models using a few annotated input-output pairs \cite{ape}, to utilizing RL objectives to create discrete token sequences as prompts \cite{rlprompt}. However, these works restrict their evaluation to small sets of models and tasks with little intersection, hindering their objective comparison. %\mz{For paragraphs that only have one work in the last line, try to shorten the paragraph to squeeze in context.}

\paragraph{Understanding in-context learning} There has been much recent work attempting to understand the mechanisms that drive in-context learning. Studies have found that the selection of demonstrations included in prompts significantly impacts few-shot accuracy across most tasks \cite{whatmakesgoodicexamples, selectionmachinetranslation, knnprompting}. Works like \cite{fantasticallyorderedprompts} also show that altering the ordering of a fixed set of demonstrations can affect downstream accuracy. Prompts sensitive to demonstration permutation often exhibit lower accuracies \cite{relationsensitivityaccuracy}, making them less reliable, particularly in low-resource domains.

Our work aims to bridge these gaps by systematically evaluating the efficacy of popular instruction selection approaches over a diverse set of tasks and models, facilitating objective comparison. We evaluate these methods not only on accuracy metrics, but also on sensitivity metrics to glean additional insights. We recognize that other facets of prompting not covered by instruction engineering exist \cite{weichain, react, selfconsistency}, and defer these explorations to future work. 
\section{Dataset}
\label{sec:dataset}
%\sarah{add statistics about distribution of merge patterns}
%\alexey{I added some numbers in the section 4 (around line 270). Detailed numbers are in Appendix. We can move it up here if needed...}
%To create a dataset for self-supervised pretraining, we clone all non-fork repositories with more than 20 stars in GitHub that have C, C++, C\#, Python, Java, JavaScript, TypeScript, PHP, Go, and Ruby as their top language. The resulting dataset comprises over 64 million source code files. 
%\chris{why do we list languages here that we don't ever evaluate on?  A reviewer will find this confusing and ask about it.  We found that language specific models work better than multi-lingual models, right?}

The finetuning dataset is mined from over 100,000 open source software repositories in multiple programming languages with merge conflicts. It contains commits from git histories with exactly two parents, which resulted in a merge conflict.  We replay \texttt{git merge} on the two parents to see if it generates any conflicts. Otherwise, we ignore the merge from our dataset. We use the approach introduced by~\citet{Dinella2021} to extract resolution regions---however, we do not restrict ourselves to conflicts with less than 30 lines only.  Lastly, we extract token-level conflicts and conflict resolution classification labels (introduced in Section \ref{formulation}) from line-level conflicts and resolutions. Tab.~\ref{tab:fintuning_dataset} provides a summary of the finetuning dataset.

\begin{table}[htb]
\centering
\caption{Number of merge conflicts in the dataset.}
\begin{tabular}{llllllllllll} \toprule
\textbf{Programming language} & \textbf{Development set}  & \textbf{Test set} \\ \midrule
C\# & 27874 & 6969 \\ 
JavaScript & 66573 & 16644\\ 
TypeScript & 22422 & 5606\\ 
Java & 103065 & 25767 \\ 
\bottomrule
\end{tabular}
\label{tab:fintuning_dataset}
\end{table}
The finetuning dataset is split into development and test sets in the proportion 80/20 at random at the file-level. The development set is further split into training and validation sets in 80/20 proportion at the merge conflict level.    

%\vspace{-5mm}

\textbf{Dataset decomposition.}
We summarize the dataset components and task as follows:
\definecolor{cosmiclatte}{rgb}{1.0, 0.97, 0.91}
\definecolor{darkchampagne}{rgb}{0.76, 0.7, 0.5}
\definecolor{gray(x11gray)}{rgb}{0.75, 0.75, 0.75}
\definecolor{aliceblue}{rgb}{0.94, 0.97, 1.0}
\definecolor{lightgray}{rgb}{0.93, 0.93, 0.93}
\begin{mdframed}[linecolor=gray(x11gray),backgroundcolor=lightgray,roundcorner=20pt,linewidth=1pt]
\textbf{Dataset.} Unless no search results were found, a single example in the dataset consists of the following:
\begin{itemize}[noitemsep,topsep=0pt]
\item A query \textbf{\textcolor{myOrange}{image}} $I^q$.
\item A query \textbf{\textcolor{myblue}{caption}} $C^q$.
\item \textbf{\textcolor{myOrange}{Visual evidence}}: 
    \begin{itemize}[noitemsep,topsep=0pt] 
    \item A list of \textbf{images}: $I^e = [I^e_1, ..., I^e_K]$.
    \end{itemize}
\item \textbf{\textcolor{myblue}{Textual evidence}}:
    \begin{itemize}[noitemsep,topsep=0pt]
        \item A list of \textbf{entities}: $\textit{ENT} = [\textit{E}_1, ..., \textit{E}_M]$. 
        \item A list of \textbf{captions/sentences}:
        
        $S = [S_1, ..., S_N]$.
    \end{itemize}
\end{itemize}
\textbf{Task.} Classify $\{I^q,C^q\}$ to: $\textit{Pristine}$ or $\textit{Falsified}$.
\end{mdframed}












\section{Proposed Approach} \label{sec:method}

Our goal is to create a unified model that maps task representations (e.g., obtained using task2vec~\cite{achille2019task2vec}) to simulation parameters, which are in turn used to render synthetic pre-training datasets for not only tasks that are seen during training, but also novel tasks.
This is a challenging problem, as the number of possible simulation parameter configurations is combinatorially large, making a brute-force approach infeasible when the number of parameters grows. 

\subsection{Overview} 

\cref{fig:controller-approach} shows an overview of our approach. During training, a batch of ``seen'' tasks is provided as input. Their task2vec vector representations are fed as input to \ours, which is a parametric model (shared across all tasks) mapping these downstream task2vecs to simulation parameters, such as lighting direction, amount of blur, background variability, etc.  These parameters are then used by a data generator (in our implementation, built using the Three-D-World platform~\cite{gan2020threedworld}) to generate a dataset of synthetic images. A classifier model then gets pre-trained on these synthetic images, and the backbone is subsequently used for evaluation on specific downstream task. The classifier's accuracy on this task is used as a reward to update \ours's parameters. 
Once trained, \ours can also be used to efficiently predict simulation parameters in {\em one-shot} for ``unseen'' tasks that it has not encountered during training. 


\subsection{\ours Model} 


Let us denote \ours's parameters with $\theta$. Given the task2vec representation of a downstream task $\bs{x} \in \mc{X}$ as input, \ours outputs simulation parameters $a \in \Omega$. The model consists of $M$ output heads, one for each simulation parameter. In the following discussion, just as in our experiments, each simulation parameter is discretized to a few levels to limit the space of possible outputs. Each head outputs a categorical distribution $\pi_i(\bs{x}, \theta) \in \Delta^{k_i}$, where $k_i$ is the number of discrete values for parameter $i \in [M]$, and $\Delta^{k_i}$, a standard $k_i$-simplex. The set of argmax outputs $\nu(\bs{x}, \theta) = \{\nu_i | \nu_i = \argmax_{j \in [k_i]} \pi_{i, j} ~\forall i \in [M]\}$ is the set of simulation parameter values used for synthetic data generation. Subsequently, we drop annotating the dependence of $\pi$ and $\nu$ on $\theta$ and $\bs{x}$ when clear.

\subsection{\ours Training} 


Since Task2Sim aims to maximize downstream accuracy after pre-training, we use this accuracy as the reward in our training optimization\footnote{Note that our rewards depend only on the task2vec input and the output action and do not involve any states, and thus our problem can be considered similar to a stateless-RL or contextual bandits problem \cite{langford2007epoch}.}.
Note that this downstream accuracy is a non-differentiable function of the output simulation parameters (assuming any simulation engine can be used as a black box) and hence direct gradient-based optimization cannot be used to train \ours. Instead, we use REINFORCE~\cite{williams1992simple}, to approximate gradients of downstream task performance with respect to model parameters $\theta$. 

\ours's outputs represent a distribution over ``actions'' corresponding to different values of the set of $M$ simulation parameters. $P(a) = \prod_{i \in [M]} \pi_i(a_i)$ is the probability of picking action $a = [a_i]_{i \in [M]}$, under policy $\pi = [\pi_i]_{i \in [M]}$. Remember that the output $\pi$ is a function of the parameters $\theta$ and the task representation $\bs{x}$. To train the model, we maximize the expected reward under its policy, defined as
\begin{align}
    R = \E_{a \in \Omega}[R(a)] = \sum_{a \in \Omega} P(a) R(a)
\end{align}
where $\Omega$ is the space of all outputs $a$ and $R(a)$ is the reward when parameter values corresponding to action $a$ are chosen. Since reward is the downstream accuracy, $R(a) \in [0, 100]$.  
Using the REINFORCE rule, we have
\begin{align}
    \nabla_{\theta} R 
    &= \E_{a \in \Omega} \left[ (\nabla_{\theta} \log P(a)) R(a) \right] \\
    &= \E_{a \in \Omega} \left[ \left(\sum_{i \in [M]} \nabla_{\theta} \log \pi_i(a_i) \right) R(a) \right]
\end{align}
where the 2nd step comes from linearity of the derivative. In practice, we use a point estimate of the above expectation at a sample $a \sim (\pi + \epsilon)$ ($\epsilon$ being some exploration noise added to the Task2Sim output distribution) with a self-critical baseline following \cite{rennie2017self}:
\begin{align} \label{eq:grad-pt-est}
    \nabla_{\theta} R \approx \left(\sum_{i \in [M]} \nabla_{\theta} \log \pi_i(a_i) \right) \left( R(a) - R(\nu) \right) 
\end{align}
where, as a reminder $\nu$ is the set of the distribution argmax parameter values from the \name{} model heads.

A pseudo-code of our approach is shown in \cref{alg:train}.  Specifically, we update the model parameters $\theta$ using minibatches of tasks sampled from a set of ``seen'' tasks. Similar to \cite{oh2018self}, we also employ self-imitation learning biased towards actions found to have better rewards. This is done by keeping track of the best action encountered in the learning process and using it for additional updates to the model, besides the ones in \cref{ln:update} of \cref{alg:train}. 
Furthermore, we use the test accuracy of a 5-nearest neighbors classifier operating on features generated by the pretrained backbone as a proxy for downstream task performance since it is computationally much faster than other common evaluation criteria used in transfer learning, e.g., linear probing or full-network finetuning. Our experiments demonstrate that this proxy evaluation measure indeed correlates with, and thus, helps in final downstream performance with linear probing or full-network finetuning. 






\begin{algorithm}
\DontPrintSemicolon
 \textbf{Input:} Set of $N$ ``seen'' downstream tasks represented by task2vecs $\mc{T} = \{\bs{x}_i | i \in [N]\}$. \\
 Given initial Task2Sim parameters $\theta_0$ and initial noise level $\epsilon_0$\\
 Initialize $a_{max}^{(i)} | i \in [N]$ the maximum reward action for each seen task \\
 \For{$t \in [T]$}{
 Set noise level $\epsilon = \frac{\epsilon_0}{t} $ \\
 Sample minibatch $\tau$ of size $n$ from $\mc{T}$  \\
 Get \ours output distributions $\pi^{(i)} | i \in [n]$ \\
 Sample outputs $a^{(i)} \sim \pi^{(i)} + \epsilon$ \\
 Get Rewards $R(a^{(i)})$ by generating a synthetic dataset with parameters $a^{(i)}$, pre-training a backbone on it, and getting the 5-NN downstream accuracy using this backbone \\
 Update $a_{max}^{(i)}$ if $R(a^{(i)}) > R(a_{max}^{(i)})$ \\
 Get point estimates of reward gradients $dr^{(i)}$ for each task in minibatch using \cref{eq:grad-pt-est} \\
 $\theta_{t,0} \leftarrow \theta_{t-1} + \frac{\sum_{i \in [n]} dr^{(i)}}{n}$ \label{ln:update} \\
 \For{$j \in [T_{si}]$}{ 
    \tcp{Self Imitation}
    Get reward gradient estimates $dr_{si}^{(i)}$ from \cref{eq:grad-pt-est} for $a \leftarrow a_{max}^{(i)}$ \\
    $\theta_{t, j}  \leftarrow \theta_{t, j-1} + \frac{\sum_{i \in [n]} dr_{si}^{(i)}}{n}$
 }
 $\theta_{t} \leftarrow \theta_{t, T_{si}}$
 }
 \textbf{Output}: Trained model with parameters $\theta_T$. 
 \caption{Training Task2Sim}
 \label{alg:train}  
\end{algorithm}

%!TEX ROOT = ../../centralized_vs_distributed.tex

\section{{\titlecap{the centralized-distributed trade-off}}}\label{sec:numerical-results}

\revision{In the previous sections we formulated the optimal control problem for a given controller architecture
(\ie the number of links) parametrized by $ n $
and showed how to compute minimum-variance objective function and the corresponding constraints.
In this section, we present our main result:
%\red{for a ring topology with multiple options for the parameter $ n $},
we solve the optimal control problem for each $ n $ and compare the best achievable closed-loop performance with different control architectures.\footnote{
\revision{Recall that small (large) values of $ n $ mean sparse (dense) architectures.}}
For delays that increase linearly with $n$,
\ie $ f(n) \propto n $, 
we demonstrate that distributed controllers with} {few communication links outperform controllers with larger number of communication links.}

\textcolor{subsectioncolor}{Figure~\ref{fig:cont-time-single-int-opt-var}} shows the steady-state variances
obtained with single-integrator dynamics~\eqref{eq:cont-time-single-int-variance-minimization}
%where we compare the standard multi-parameter design 
%with a simplified version \tcb{that utilizes spatially-constant feedback gains
and the quadratic approximation~\eqref{eq:quadratic-approximation} for \revision{ring topology}
with $ N = 50 $ nodes. % and $ n\in\{1,\dots,10\} $.
%with $ N = 50 $, $ f(n) = n $ and $ \tau_{\textit{min}} = 0.1 $.
%\autoref{fig:cont-time-single-int-err} shows the relative error, defined as
%\begin{equation}\label{eq:relative-error}
%	e \doteq \dfrac{\optvarx-\optvar}{\optvar}
%\end{equation}
%where $ \optvar $ and $ \optvarx $ denote the the optimal and sub-optimal scalar variances, respectively.
%The performance gap is small
%and becomes negligible for large $ n $.
{The best performance is achieved for a sparse architecture with  $ n = 2 $ 
in which each agent communicates with the two closest pairs of neighboring nodes. 
This should be compared and contrasted to nearest-neighbor and all-to-all 
communication topologies which induce higher closed-loop variances. 
Thus, 
the advantage of introducing additional communication links diminishes 
beyond}
{a certain threshold because of communication delays.}

%For a linear increase in the delay,
\textcolor{subsectioncolor}{Figure~\ref{fig:cont-time-double-int-opt-var}} shows that the use of approximation~\eqref{eq:cont-time-double-int-min-var-simplified} with $ \tilde{\gvel}^* = 70 $
identifies nearest-neighbor information exchange as the {near-optimal} architecture for a double-integrator model
with ring topology. 
This can be explained by noting that the variance of the process noise $ n(t) $
in the reduced model~\eqref{eq:x-dynamics-1st-order-approximation}
is proportional to $ \nicefrac{1}{\gvel} $ and thereby to $ \taun $,
according to~\eqref{eq:substitutions-4-normalization},
making the variance scale with the delay.

%\mjmargin{i feel that we need to comment about different results that we obtained for CT and DT double-intergrator dynamics (monotonic deterioration of performance for the former and oscillations for the latter)}
\revision{\textcolor{subsectioncolor}{Figures~\ref{fig:disc-time-single-int-opt-var}--\ref{fig:disc-time-double-int-opt-var}}
show the results obtained by solving the optimal control problem for discrete-time dynamics.
%which exhibit similar trade-offs.
The oscillations about the minimum in~\autoref{fig:disc-time-double-int-opt-var}
are compatible with the investigated \tradeoff~\eqref{eq:trade-off}:
in general, 
the sum of two monotone functions does not have a unique local minimum.
Details about discrete-time systems are deferred to~\autoref{sec:disc-time}.
Interestingly,
double integrators with continuous- (\autoref{fig:cont-time-double-int-opt-var}) ad discrete-time (\autoref{fig:disc-time-double-int-opt-var}) dynamics
exhibits very different trade-off curves,
whereby performance monotonically deteriorates for the former and oscillates for the latter.
While a clear interpretation is difficult because there is no explicit expression of the variance as a function of $ n $,
one possible explanation might be the first-order approximation used to compute gains in the continuous-time case.
%which reinforce our thesis exposed in~\autoref{sec:contribution}.

%\begin{figure}
%	\centering
%	\includegraphics[width=.6\linewidth]{cont-time-double-int-opt-var-n}
%	\caption{Steady-state scalar variance for continuous-time double integrators with $ \taun = 0.1n $.
%		Here, the \tradeoff is optimized by nearest-neighbor interaction.
%	}
%	\label{fig:cont-time-double-int-opt-var-lin}
%\end{figure}
}

\begin{figure}
	\centering
	\begin{minipage}[l]{.5\linewidth}
		\centering
		\includegraphics[width=\linewidth]{random-graph}
	\end{minipage}%
	\begin{minipage}[r]{.5\linewidth}
		\centering
		\includegraphics[width=\linewidth]{disc-time-single-int-random-graph-opt-var}
	\end{minipage}
	\caption{Network topology and its optimal {closed-loop} variance.}
	\label{fig:general-graph}
\end{figure}

Finally,
\autoref{fig:general-graph} shows the optimization results for a random graph topology with discrete-time single integrator agents. % with a linear increase in the delay, $ \taun = n $.
Here, $ n $ denotes the number of communication hops in the ``original" network, shown in~\autoref{fig:general-graph}:
as $ n $ increases, each agent can first communicate with its nearest neighbors,
then with its neighbors' neighbors, and so on. For a control architecture that utilizes different feedback gains for each communication link
	(\ie we only require $ K = K^\top $) we demonstrate that, in this case, two communication hops provide optimal closed-loop performance. % of the system.}

Additional computational experiments performed with different rates $ f(\cdot) $ show that the optimal number of links increases for slower rates: 
for example, 
the optimal number of links is larger for $ f(n) = \sqrt{n} $ than for $ f(n) = n $. 
\revision{These results are not reported because of space limitations.}
% \vspace{-0.5em}
\section{Conclusion}
% \vspace{-0.5em}
Recent advances in multimodal single-cell technology have enabled the simultaneous profiling of the transcriptome alongside other cellular modalities, leading to an increase in the availability of multimodal single-cell data. In this paper, we present \method{}, a multimodal transformer model for single-cell surface protein abundance from gene expression measurements. We combined the data with prior biological interaction knowledge from the STRING database into a richly connected heterogeneous graph and leveraged the transformer architectures to learn an accurate mapping between gene expression and surface protein abundance. Remarkably, \method{} achieves superior and more stable performance than other baselines on both 2021 and 2022 NeurIPS single-cell datasets.

\noindent\textbf{Future Work.}
% Our work is an extension of the model we implemented in the NeurIPS 2022 competition. 
Our framework of multimodal transformers with the cross-modality heterogeneous graph goes far beyond the specific downstream task of modality prediction, and there are lots of potentials to be further explored. Our graph contains three types of nodes. While the cell embeddings are used for predictions, the remaining protein embeddings and gene embeddings may be further interpreted for other tasks. The similarities between proteins may show data-specific protein-protein relationships, while the attention matrix of the gene transformer may help to identify marker genes of each cell type. Additionally, we may achieve gene interaction prediction using the attention mechanism.
% under adequate regulations. 
% We expect \method{} to be capable of much more than just modality prediction. Note that currently, we fuse information from different transformers with message-passing GNNs. 
To extend more on transformers, a potential next step is implementing cross-attention cross-modalities. Ideally, all three types of nodes, namely genes, proteins, and cells, would be jointly modeled using a large transformer that includes specific regulations for each modality. 

% insight of protein and gene embedding (diff task)

% all in one transformer

% \noindent\textbf{Limitations and future work}
% Despite the noticeable performance improvement by utilizing transformers with the cross-modality heterogeneous graph, there are still bottlenecks in the current settings. To begin with, we noticed that the performance variations of all methods are consistently higher in the ``CITE'' dataset compared to the ``GEX2ADT'' dataset. We hypothesized that the increased variability in ``CITE'' was due to both less number of training samples (43k vs. 66k cells) and a significantly more number of testing samples used (28k vs. 1k cells). One straightforward solution to alleviate the high variation issue is to include more training samples, which is not always possible given the training data availability. Nevertheless, publicly available single-cell datasets have been accumulated over the past decades and are still being collected on an ever-increasing scale. Taking advantage of these large-scale atlases is the key to a more stable and well-performing model, as some of the intra-cell variations could be common across different datasets. For example, reference-based methods are commonly used to identify the cell identity of a single cell, or cell-type compositions of a mixture of cells. (other examples for pretrained, e.g., scbert)


%\noindent\textbf{Future work.}
% Our work is an extension of the model we implemented in the NeurIPS 2022 competition. Now our framework of multimodal transformers with the cross-modality heterogeneous graph goes far beyond the specific downstream task of modality prediction, and there are lots of potentials to be further explored. Our graph contains three types of nodes. while the cell embeddings are used for predictions, the remaining protein embeddings and gene embeddings may be further interpreted for other tasks. The similarities between proteins may show data-specific protein-protein relationships, while the attention matrix of the gene transformer may help to identify marker genes of each cell type. Additionally, we may achieve gene interaction prediction using the attention mechanism under adequate regulations. We expect \method{} to be capable of much more than just modality prediction. Note that currently, we fuse information from different transformers with message-passing GNNs. To extend more on transformers, a potential next step is implementing cross-attention cross-modalities. Ideally, all three types of nodes, namely genes, proteins, and cells, would be jointly modeled using a large transformer that includes specific regulations for each modality. The self-attention within each modality would reconstruct the prior interaction network, while the cross-attention between modalities would be supervised by the data observations. Then, The attention matrix will provide insights into all the internal interactions and cross-relationships. With the linearized transformer, this idea would be both practical and versatile.

% \begin{acks}
% This research is supported by the National Science Foundation (NSF) and Johnson \& Johnson.
% \end{acks}


%-------------------------------------------------------------------------



%%%%%%%%% REFERENCES
{\small
\bibliographystyle{ieee_fullname}
\bibliography{egbib}
}
\clearpage
\appendix 
\section*{Supplementary Materials}
\section{Background: Standard ADMM Training of DNNs} \label{sec:admm_nn}

Alternating Direction Method of Multipliers (ADMM) \cite{gabay1975dual,boyd2011distributed} is a class of optimization methods belonging to  \textit{operator splitting techniques} which borrows benefits from both dual decomposition and augmented Lagrangian methods for constrained optimization. %To show the potentials of standard ADMM, we first revisit a general formulation of ADMM in DNN training, similar to those used in prior work. Then, we propose our stochastic block-ADMM in the next subsection.

To formulate training an $L$-layer DNN in a general supervised setting, we would have the following non-convex constrained optimization problem \cite{zeng2018global}:
% \vspace{-0.1in}
\begin{align} \label{eq:obj}
	\minimize_{ \mathcal{W}, \mathcal{A}, \mathcal{Z}} \quad &\mathcal{J}\left(\mY, \mZ_{L} \right) + \sum_{\ell = 1}^{L} \lambda_{\ell}  {\bf r}_{\ell} (\mW_{\ell}) \\
	 {\rm subject~to} \quad & \mA_{\ell} - {\bm \phi}_{\ell } \left( \mZ_{\ell} \right) = {\bf 0}, \quad \ell = 1,\dots, L-1   \nonumber \\
	 {\rm subject~to} \quad & \mZ_{\ell} - \mW_{\ell} \mA_{\ell-1} = {\bf 0}, \quad \ell = 1, \dots , L \nonumber 
\end{align}
where $\mathcal{J}$ is the main objective (\textit{e.g.}, cross-entropy, mean-squared-error loss functions) that needs to be minimized. The subscript $\ell$ denotes the $\ell$-th layer in the network. The optimization variables are $\mathcal{W} = \{ \mW_\ell\}_{\ell=1}^{L}$, $\mathcal{A} = \{ \mA_{\ell}\}_{\ell=1}^{L-1}$, and $\mathcal{Z} = \{ \mZ_{\ell}\}_{\ell=1}^{L}$ where $\mW_\ell$, $\mZ_{\ell}$, $\mA_\ell$, and ${\bm \phi}_\ell (.)$ are the weight matrix, output matrix, activation matrix, and the activation function (\textit{e.g.}, ReLU) at the $\ell$-th layer, respectively. Note that $\mA_{0} = \mX$ where $\mX = \{ \vx_1,\dots, \vx_N \} \in  \R^{M \times N}$ is the input data matrix containing $N$ samples with input dimensionality $M$; $\mY = \{\vy_1,\dots, \vy_N \} \in \R^{C \times N}$ is the target matrix pair comprised of $N$ one-hot vector label of dimension $C$, representing number of prediction classes. Also, ${\bf r(.)}$ is the regularization term with (\textit{e.g.}, Frobenius norm $\|.\|_F^2$) corresponding penalty weight $\lambda_{\ell}$. Note that the regularization term can be simply ignored by setting $\lambda_\ell$ to zero. In this formulation, the intercept in each layer is ignored for simplicity as it can be simply be added by slightly modifying the $\mW_\ell$ and the input to each layer. The formulation in Eq. (\ref{eq:obj}) breaks the the conventional multi-layer backpropagation optimization of DNNs into simpler sub-problems that can be solved efficiently (e.g. reducing to least-squares problem). This also facilitates training in a distributed manner --- as the layers of the DNN are decoupled and the variables can be updated in parallel across layers ($\mW_\ell$) and data points (\ $\mW_\ell, \mZ_\ell, \mA_\ell$).



To enforce the constraints in problem (\ref{eq:obj}) and solve the optimization using ADMM, we would have the following augmented Lagrangian problem:

\begin{eqnarray} \label{eq:augmented}
	\minimize_{ \mathcal{W}, \mathcal{A}, \mathcal{Z}} \quad &\mathcal{J}\left(\mY, \mZ_{L} \right) + \sum_{\ell = 1}^{L} \lambda_{\ell}  {\bf r}_{\ell} (\mW_{\ell}) \\
	& + \sum_{\ell=1}^{L} \frac{\beta_{\ell}}{2} \| \mZ_{\ell} - \mW_{\ell} \mA_{\ell-1} + \mU_{\ell}\|_{F}^{2} \nonumber\\
	& + \sum_{\ell=1}^{L-1} \frac{\gamma_{\ell}}{2} \| \mA_{\ell} - {\bm \phi}_{\ell}(\mZ_{\ell}) + \mV_{\ell}\|_{F}^{2}\nonumber
\end{eqnarray}
where $\beta_{\ell}, \gamma_\ell >0$ are the step sizes, $\mU_{\ell}$ and $\mV_{\ell}$ are the \textit{(scaled) dual variables} \cite{boyd2011distributed} for the equality constraint at the layer $\ell$. 
Algorithm \ref{alg:admm} shows a standard ADMM scheme for optimizing Eq. (\ref{eq:augmented}). Note, the parameters are updated in a closed-form as analytical solution can be simply derived. For simplicity of the equations, we denote $\gP_\ell (.) = \frac{\beta_{\ell}}{2} \| \mZ_{\ell} - \mW_{\ell} \mA_{\ell-1} + \mU_{\ell}\|_{F}^{2} $ and $\gQ_\ell (.) = \frac{\gamma_{\ell}}{2} \| \mA_{\ell} - {\bm \phi}_{\ell}(\mZ_{\ell}) + \mV_{\ell}\|_{F}^{2}$. This algorithm is similar to \cite{taylor2016training,wang2019admm} with the difference that all the equality constraints in problem (\ref{eq:obj}) are enforced using multipliers, while previous work only enforced the constraints on the last layer $L$ while other constraints were only loosely enforced using quadratic penalty. 

\begin{algorithm}[htb]
  \caption{Standard ADMM for DNN Training}
  \label{alg:admm}
\begin{algorithmic}
  {\STATE \scalebox{1}{\bfseries Input:} data $\mX$, labels $\mY$}
  \STATE  \scalebox{1}{{\bfseries Params:} $\beta_\ell >0, \gamma_\ell >0,\lambda_\ell > 0$ }
  \STATE  \scalebox{0.8}{{\bfseries Initialize:} $\{\mW_\ell^0\}_{\ell=1}^{L}, \{ \mU_\ell^0\}_{\ell=1}^{L}, \{ \mV_\ell^0\}_{\ell=1}^{L-1}, \{\mZ^0_\ell\}_{\ell=1}^{L}, \{\mA^0_\ell\}_{\ell=1}^{L-1}\; k \leftarrow 0$ }
  \REPEAT
  \FOR{$\ell=1$ {\bfseries to} $L$}
  \STATE \scalebox{1}{$\mW_\ell^{k+1} \leftarrow \argmin\; \{ \gP_\ell (.) +  \lambda_{\ell}  {\bf r}_{\ell} (\mW_{\ell}^{k})\}$}
  \ENDFOR
  \FOR{$\ell=1$ {\bfseries to} $L-1$}
  \STATE \scalebox{1}{ $\mZ_\ell^{k+1} \leftarrow \argmin\; \{ \gP_\ell (.) +  \gQ_\ell (.) \}$ }
  \STATE \scalebox{1}{$\mA_\ell^{k+1} \leftarrow \argmin\; \{ \gP_{\ell+1} (.) +  \gQ_\ell (.) \} $}
  \ENDFOR
    \STATE \scalebox{1}{ $\mZ_{L}^{k+1} \leftarrow \argmin\; \{ \mathcal{J}\left(\mY, \mZ_{L}^{k} \right) + \gP_L (.) \}$ }
  \FOR{$\ell=1$ {\bfseries to} $L-1$}
  \STATE \scalebox{1}{$\mU_\ell^{k+1} \leftarrow \mU_\ell^{k} + \mZ_{\ell}^{k+1} - \mW_{\ell}^{k+1} \mA_{\ell-1}^{k+1}$}
  \STATE \scalebox{1}{$\mV_\ell^{k+1} \leftarrow \mV_\ell^{k} + \mA_{\ell}^{k+1} - {\bm \phi}_{\ell}(\mZ_{\ell}^{k+1})$}
  \ENDFOR
  \STATE \scalebox{1}{$\mU_L^{k+1} \leftarrow \mU_L^{k} + \mZ_{L}^{k+1} - \mW_{L}^{k+1} \mA_{L-1}^{k+1}$}
  \UNTIL{some stopping criterion is reached.}
\end{algorithmic}
\end{algorithm}


While the standard ADMM Algorithm \ref{alg:admm} has potentials in training (simple) DNNs \cite{taylor2016training}, there exists hurdles that confines extending ADMM to more complex problems --- the global convergence proof of the ADMM \cite{deng2016global} assumes that $\mathcal{J}$ is deterministic and the global solution is calculated at each iteration of the cyclic parameter updates.
% and during each iteration of the cyclic parameter updates, all the data samples are visited.
This makes standard ADMM computationally expensive thus impractical for training of many large-scale optimization problems. Specifically, for  deep learning, this would impose a severe restriction on training set size when limited computational resources are available. In addition, since the variable updates in standard ADMM are analytically driven, the extent of its applications is limit to trivial tasks \cite{taylor2016training}, making it incompetent to perform on par with the recent complex architectures introduced in deep learning (e.g. \cite{he2016deep}).


\section{Proof for Proposition 1}\label{sec:proof}

We follow the steps in the proof for similar problems in \cite{fu2018anchor} and \cite{shi2017penalty} with deterministic primal updates. Proper modifications are made to cover the stochastic primal update in our proof.


Note that we have
              \[     \nabla{\cal L}_{\rho_k}(\X^k)= \nabla f(\X^k) + \nabla h(\X^k)^T\bm \mu^k,          \]
              where 
              \[      \bm \mu^k = (1/\rho_k)h(\bm X^k)+\bm \lambda^k.   
              \]
              Our first step is to show that $\{\bm \mu^k\}$ is a convergent sequence. To see this, we define 
              \[ \bm \bar{\bm \mu}^k = \frac{\bm \mu^k}{\|{\bm \mu}^k\|}. \]
              Since $\bm \bar{\bm \mu}^k$ is bounded, it converges to a limit point $\bm \bar{\bm \mu}$. Also let $\x^\star$ be a limit point of $\x^k$.
              Because we have assumed that 
              $$\varepsilon_k\rightarrow 0,\quad \sigma_k^2\rightarrow 0,$$ 
              it means that the mean and variance of the stochastic gradient of our primal update goes to zero.
              Since our stochastic gradient is unbiased, we have
              \[       {\cal G}(\X^k) \rightarrow \nabla {\cal L}_{\rho_k}(\X^\star). \]  
              This also means that  we must have ${\cal G}(\x^k)\rightarrow \bm 0$ and $$\nabla L_{\rho_k}(\bm x^k)\rightarrow \bm 0.$$
     Hence, the following holds when $k\rightarrow \infty$:
              \begin{equation}\label{eq:approxkkt}
                 \nabla L_{\rho_k}(\bm X^\star)=\nabla f(\X^\star)+\nabla h(\X^\star)^T\bm {\bm \mu}^\infty = 0,
              \end{equation}           
               
               
              Suppose $\bm \mu^k$ is unbounded. By dividing \eqref{eq:approxkkt} by the above $\|\bm \mu^k\|$ and considering $k\rightarrow \infty$, we must have 
              \begin{equation}\label{eq:key}
                \nabla h(\X^\star)^T\bm \bar{\bm \mu}= 0,\quad \forall \X.    
              \end{equation}               
              The term $\nabla f(\bm X^\star)/\|\bm \mu\|$ is zero since we assumed $\bar{\bm \mu}$ is unbounded.
              Since $h(\bm X)=\bm 0$ satisfies the Robinson's condition, then, for any $\bm w$, there exists $\beta>0$ and $\bm x$ such that
              \[      \bm w = \beta \nabla h(\X^\star)(\X-\X^\star).        \]
              This together with \eqref{eq:key} says that $\bar{\bm \mu}=\bm 0$. This contradicts to the fact $\|\bar{\bm \mu}\|=1$. Hence, $\{ \bm \mu^k \}$ must be a bounded sequence and thus admits a limit point. Denote $\bm \mu^\star$ as this limit point, and take limit of both sides of \eqref{eq:approxkkt}. We have:
              \begin{equation}
              \nabla f(\X^\star)+\nabla h(\X^\star)^T\bm \mu^\star= \bm 0,\quad \forall \X.
              \end{equation}
               
              In addition, since $$\rho_k(\bm \mu^k-\bm \lambda^k) = h(\mathbf{\X^k})$$ with $\rho_k \rightarrow 0$ or $\bm \mu_k-\bm \lambda_k \rightarrow 0$ (per our updating rule and $\eta_k\rightarrow 0$), the constraints will be enforced in the limit.      $\mbox{     } \square$   \\
              

% \subsection*{{\uppercase\expandafter{\romannumeral D}. Supervised training on Fashion-Mnist}}\label{fmnist}


% To compare our method with dlADMM \citet{wang2019admm}, we evaluated the performance of our method on the Fashion-MNIST dataset \citep{xiao2017/online} with 60,000 training samples and 10,000 testing samples. We followed the settings in \citet{wang2019admm} by having 2 hidden layers with 1000 neurons each, and Cross-Entropy loss at the final layer. Also, the batch size is set to 128, $\beta_t = 1$, and the updates for $\mZ_t$ and $\Theta_t$ (eq. 6a) are performed 3 times at each epoch. Figure \ref{fig:fmnist_acc} shows the test set accuracy results over 200 epochs of training. It can be noticed that Stochastic Block ADMM is converging at lower epochs and reaching a higher test accuracy while performing efficient mini-batch updates. Further, in section C., it will be demonstrated that Stochastic Block ADMM converges drastically faster than dlADMM in terms of wall clock time.

   
% \begin{figure}[ht]
% \begin{center}
% \centerline{
% \includesvg[width=\columnwidth]{img/fmnist_acc.svg}
% }
% \caption{Test accuracy comparison of Stochastic Block ADMM and dlADMM \citep{wang2019admm} on Fashion-MNIST dataset using a network with 3 fully-connected layers: $784-1000-1000-10$. Final test accuracy: "Stochastic Block ADMM": $\bf 90.39\%$, "Wang \etal":$84.67 \%$ (averaged over 5 runs).}
% \vskip -0.25in
% \label{fig:fmnist_acc}
% \end{center}
% \end{figure}


\begin{figure}[ht]
\begin{center}
\centerline{
\includegraphics[width=\columnwidth]{imgs/fmnist_acc.pdf}
}
\caption{Test accuracy comparison of Stochastic Block ADMM and dlADMM on Fashion-MNIST dataset using a network with 3 fully-connected layers: $784-1000-1000-10$. Final test accuracy: "Stochastic Block ADMM": $\bf 90.39\%$, "Wang \textit{et al.}":$84.67 \%$ (averaged over 5 runs).}
% \vskip -0.25in
\label{fig:fmnist_acc}
\end{center}
\end{figure}




%----------------------------
\section{Supervised training of DNNs}\label{sec:sup_train}

\textbf{Fashion-MNIST.}
To compare our method with dlADMM \cite{wang2019admm}, we evaluated the performance of our method on the Fashion-MNIST dataset \cite{xiao2017/online} with 60,000 training samples and 10,000 testing samples. We followed the settings in \cite{wang2019admm} by having 2 hidden layers with 1000 neurons each, and Cross-Entropy loss at the final layer. Also, the batch size is set to 128, $\beta_t = 1$, and the updates for $\mZ_t$ and $\Theta_t$ (eq. 6a) are performed 3 times at each epoch. Figure \ref{fig:fmnist_acc} shows the test set accuracy results over 200 epochs of training. It can be noticed that Stochastic Block ADMM is converging at lower epochs and reaching a higher test accuracy while performing efficient mini-batch updates. Further, in section C., it will be demonstrated that Stochastic Block ADMM converges drastically faster than dlADMM in terms of wall clock time.



\textbf{CIFAR-10.}
The previous works on training deep netowrks using ADMM have been limited to trivial networks and datasets (e.g. MNIST) \cite{taylor2016training,wang2019admm}. However, our proposed method does not have many of the existing restrictions and assumptions in the network architecture, as in previous works do, and can easily be extended to train non-trivial applications. It is critical to validate stochastic block-ADMM in settings where deep and modern architectures such as deep residual networks, convolutional layers, cross-entropy loss function, etc., are used. To that end, we validate the ability of our method is a supervised setting (image classification) on the CIFAR-10 dataset \cite{cifar} using ResNet-18 \cite{he2016deep}. To best of our knowledge, this is the first attempt of using ADMM for training complex networks such as ResNets. 


For this purpose, we used 50,000 samples for training and the remaining 10,000 for evaluation. 
To have a fair comparison, we followed the configuration suggested in \cite{gotmare2018decoupling} by converting Resnet-18 network into two blocks $(T=2)$, with the splitting point located at the end of {\sc conv3\_x} layer. We used the Adam optimizer to update both the blocks and the decoupling variables with the learning rates of $\eta_t = 5e^{-3}$ and $\zeta_t = 0.5$. We noted since the auxiliary variables $\mZ_t$ are not "shared parameters" across data samples, they usually require a higher learning rate in Algorithm \ref{alg:blockadmm}. Also, we found the ADMM step size $\beta_t = 1$ to be sufficient for enforcing the block's coupling. 


Figure. \ref{fig:cifar} shows the results from our method compared with two baselines: \cite{gotmare2018decoupling}, and conventional end-to-end neural network training using back-propagation and SGD. Our algorithm consistently outperformed ~\cite{gotmare2018decoupling} however cannot match the conventional SGD results. There are several factors that we hypothesize that might have contributed to the performance difference: 1) in a ResNet the residual structure already partially solved the vanishing gradient problem, hence SGD/Adam performs significantly better than a fully-connected version; 
% 2) The common data augmentation in CIFAR will end up sending a different training example to the optimization algorithm at each iteration, which does not seem to affect SGD but seem to affect ADMM convergence somewhat; 
2) we noticed decreasing the learning rate for $\Theta_t$ updates does not impact the performance as it does for an end-to-end back-propagation using SGD. Still, we obtained the best performance of ADMM-type methods on both MNIST and CIFAR datasets, showing the promise of our approach.
% As illustrated, ADMM gets to a good performance fast and then slowly progress to higher accuracy..


%---------------------------- fig cifar  ------------------------------

\begin{figure}[htb]
% \vskip 0.15in
\begin{center}
\centerline{
\includegraphics[width=\columnwidth]{imgs/cifar.pdf}
}
% \vskip -0.05in
\caption{Test set accuracy on CIFAR-10 dataset. Final accuracy "Block ADMM": $89.66\%$, "Gotmare \etal":$87.12 \%$, "SGD": $\bf 92.70\%$. (Best viewed in color.)}
\label{fig:cifar}
\end{center}
% \vskip -0.2in
\end{figure}

 
 
 
% \subsection*{{\uppercase\expandafter{\romannumeral C}. Wall Clock Time Comparison}} \label{time_cmp}

% In this section, we setup a experiment to further analyse the efficiency of Stochastic Block ADMM and compare its training wall clock time against other baselines: \citet{gotmare2018decoupling,zeng2018global} (BCD), and \citet{wang2019admm} (ADMM). 
% For this purpose, we follow the similar settings as in section 4.1 for a supervised Deep Neural Network (DNN) training over MNIST dataset. Figure \ref{fig:time} shows the test set accuracy v.s. the training wall clock time from different methods. All the experiments are run on a machine with a single NVIDIA GeForce RTX 2080 Ti GPU. The methods are implemented in PyTorch framework -- except for dlADMM \citep{wang2019admm} that is implemented\footnote{code taken from \url{https://github.com/xianggebenben/dlADMM}} in "cupy", a NumPy-compatible matrix library accelerated by CUDA. \citet{gotmare2018decoupling} and Stochastic Block ADMM are trained with a mini-batch size of 128 and \citet{zeng2018global,wang2019admm} are trained in a batch setting. Note that in Figure \ref{fig:time}, the time recorded merely shows the \emph{training time} and excludes the time taken for initialization, data loading, etc. It can be observed that \citet{gotmare2018decoupling} and dlADMM are showing much slower convergence behaviors than Stochastic Block ADMM. We speculate that enforcing all the constraints by dual variables along with the efficient and cheap mini-batch updates in our method highly contributes to the convergence speed as well as its performance superiority over the other methods, including \citet{zeng2018global}.


% \begin{figure}[ht]
% \begin{center}
% \centerline{
% \includesvg[width=\columnwidth]{img/time_comparison.svg}
% }
% \caption{Test set accuracy v.s. training wall clock time comparison of different alternating optimization methods for training DNNs on MNIST dataset. Our method (blue) shows superior performance while presenting comparable convergence speed against \citet{zeng2018global} (green).}
% \vskip - 0.15in
% \label{fig:time}
% \end{center}
% \end{figure}


\begin{table*}[htb]
\caption{Prediction accuracy (\%) of individual attributes in LFWA dataset. DeepFacto with other weakly-supervised and supervised baselines.}
\label{table:attr_lfw}
\vskip 0.15in
\begin{center}
\begin{small}
\begin{sc}
\begin{tabular}{lcccccc}
\toprule
{Attributes} & \multicolumn{3}{c}{\small DeepFacto} & \small \cite{liu2015deep} & \small \cite{liu2018exploring} & \small \cite{zhang2014panda}\\
 {} & \multicolumn{3}{c}{\tiny (Weakly-Supervised)} & {\tiny (Weakly-Supervised)} & {\tiny (Supervised)} & {\tiny (Supervised)} \\
 {} & $r= $256 & 32 & 4 \\
\midrule
‘5 o Clock Shadow’ & 83.3 & 80.0 & 68.7 & 78.8 & \bf84 & \bf84\\
‘Arched Eyebrows’ & \bf86.6 & 83.9 & 79.2 & 78.1 & 82 & 79\\
‘Attractive’ & \bf84.3 & 79.8 & 73.3 & 79.2 & 83 & 81\\
‘Bags Under Eyes’ & \bf83.9 & 72.5 & 64.5 & 83.1 & 83 & 80 \\
‘Bald’ & \bf94.3 & 93.3 & 89.3 & 84.8 & 88 & 84\\
‘Bangs’ & \bf93.2 & 88.4 & 84.4 & 86.5 & 88 & 84\\
‘Big Lips’ & \bf83.2 & 77.0 & 71.9 & 75.2 & 75 & 73\\
‘Big Nose' & 80.1 & 68.7 & 61.4 & \bf81.3 & 81 & 79\\
‘Black Hair’ & \bf92.7 & 91.4 & 87.4 & 87.4 & 90 & 87\\
‘Blond Hair’ & \bf97.9 & 97.3 & 93.2 & 94.2 & 97 & 94\\
‘Blurry’ & \bf90.4 & 90.5 & 86.5 & 78.4 & 74 & 74\\
‘Brown Hair’ & \bf78.4 & 74.4 & 70.2 & 72.9 & 77 & 74\\
‘Bushy Eyebrows’ & \bf84.0 & 78.6 & 63.4 & 83.0 & 82 & 79\\
‘Chubby’ & \bf80.5 & 75.2 & 71.1 & 74.6 & 73 & 69\\
‘Double Chin’ & \bf86.0 & 77.9 & 72.3 & 80.2 & 78 & 75\\
‘Eyeglasses’ & 94.3 & 89.6 & 84.8 & 89.5 & \bf95 & 89\\
‘Goatee’ & \bf89.1 & 85.4 & 80.0 & 78.6 & 78 & 75\\
‘Gray Hair’ & \bf91.9 & 90 & 85.6 & 86.9 & 84 & 81\\
‘Heavy Makeup’ & \bf96.3 & 91.5 & 87.4 & 94.5 & 95 & 93\\
‘High Cheekbones’ & \bf90.4 & 79.0 & 72.1 & 88.8 & 88 & 86\\
‘Male’ & 81.3 & 76.6 & 70.5 & \bf94.3 & 94 & 92\\
‘Mouth Slightly Open’ & \bf85.4 & 78.0 & 73.3 & 81.7 & 82 & 78 \\
‘Mustache’ & \bf96.6 & 93.2 & 91.3 & 83.3 & 92 & 87\\
‘Narrow Eyes’ & \bf78.3 & 69.3 & 58.4 & 77.5 & 81 & 73\\
‘No Beard’ & \bf79.5 & 73.0 & 65.5 & 77.7 & 79 & 75\\
‘Oval Face’ & \bf80.6 & 73.2 & 66.1 & 78.7 & 74 & 72\\
‘Pale Skin’ & 75.1 & 66.7 & 60.6 & \bf89.8 & 84 & 84\\
‘Pointy Nose'& \bf81.6 & 73.7 & 62.2 & 79.8 & 80 & 76\\
‘Receding Hairline’ & 84.0 & 80.9 & 73.8 & \bf88.0 & 85 & 84 \\
‘Rosy Cheeks’ & \bf87.3 & 87.4 & 83.4 & 79.9 & 78 & 73\\
‘Sideburns’ & \bf85.4 & 81.5 & 75.8 & 80.5 & 77 & 76\\
‘Smiling’ & \bf92.6 & 78.7 & 69.8 & 92.2 & 91 & 89\\
‘Straight Hair’ & \bf82.8 & 77.0 & 72.1 &  73.6 & 76 & 73\\
‘Wavy Hair’ & 80.4 & 77.0 & 68.3 & \bf81.7 & 76 & 75\\
‘Wearing Earrings’ & \bf95.4 & 91.6 & 87.1 & 89.7 & 94 & 92\\
‘Wearing Hat’ & \bf93.0 & 90.2 & 87.0 & 80.5 & 88 & 82\\
‘Wearing Lipstick’ & \bf95.8 & 92.8 & 89.0 & 91.4 & 95 & 93\\
‘Wearing Necklace’ & \bf93.0 & 89.8 & 85.1 & 84.0 & 88 & 86\\
‘Wearing Necktie’ & \bf79.8 & 75.2 & 70.6 & 78.7 & 79 & 79\\
‘Young’ & \bf91.0 & 88.4 & 84.4 & 79.2 & 86 & 82\\
\midrule
Average & \bf87.0 & 81.4 & 74.8 &  83.1 & 84 & 81\\
\bottomrule
\end{tabular}
\end{sc}
\end{small}
\end{center}
\vskip -0.25in
\end{table*}




\section{Weakly Supervised Attribute Prediction}\label{sec:weakly_sup}


\subsection*{Factorizing the activations}\label{sec:factor_layer} 

With the assumption that the observations are formed by a linear combination of few basis vectors, one can approximate a given matrix $\mX \in \R^{m \times n}$ into a \textit{basis} matrix $\mM \in \R^{m \times r}$ and an \textit{score} matrix $\mS \in \R^{r \times n}$ such that $\mX \approx \mM \mS$ where $r$ is the (reduced) \textit{rank} of the factorized matrices -- commonly $r \ll \min(m, n)$.
Methods such as NMF would restrict the entries of $\mM$ and $\mS$ to be non-negative $(\forall i,j \;  \mM_{ij} \ge 0,\; \mS_{ij} \ge 0)$ which forces the decomposition to be only \textit{additive}. This has been shown to result in a parts-based representation that is intuitively more close to human perception. It is also worth mentioning that obviously, the matrix $\mX$ needs to be positive $({\forall i,j} \;  \mX_{ij} \ge 0)$. For non-negative factorization on the activations of the DNNS, due to the common use of activation functions such as \textit{ReLU}, this would not impose any constraints in most of the problems.

Activations of the CNN networks are generally tensors of the shape $\tZ_{\ell} \in \R^{(N, C, H, W)}$ which namely represent the batch size of the input, the number of the channels, the height of each channel, and the corresponding width. To adapt such tensors for the NMF problem, we reshape the tensor into the matrix $\mZ_{\ell} \in \R^{ C \times (N * H * W)}$ by stacking it over its channels while flattening the other dimensions. This way, the channels would be embedded into a pre-defined small dimension $r$ while keeping each sample and pixels information. For the weakly-supervised problem of attribute classification using DeepFacto, we attached the DeepFacto module to the last convolutional layer of the Inception-Resnet-V1 architecture followed by a \emph{ReLU}. This layer has 1792 channels and, for a given input of the size $160 \times 160$ pixels (the original input size from the LFWA dataset), the height and the width are both equal to 3. 

\begin{figure}[htb]
\vskip -0.05in
\begin{center}
\centerline{
\includegraphics[width=\columnwidth]{imgs/heatmap.jpg}
}
\caption{Heat map visualizations from three different dimensions of the score matrix $\mS$ (rows) trained by DeepFacto-32 over different samples (columns) in LFWA dataset. These dimensions can capture interpretable representations over different faces identities: \emph{eyes} (top), \emph{forehead} (middle), and \emph{nose} (bottom).}
\label{fig:heatmap}
\end{center}
\vskip -0.15in
\end{figure}

% Table \ref{table:attr_lfw} shows the prediction accuracy of each attribute in LFWA dataset and compares DeepFacto with different ranks ($r=4,32,256$) against other supervised and weakly-supervised baselines. It can be noted that our method can generate highly informative representation of the LFWA attributes without accessing their labels. This supports our conjecture that DeepFacto, by non-negatively factorizing the activations of the DNNs in and end-to-end training, can lead to an interpretable decomposition of the DNN activations.



\subsection*{Heat maps}\label{sec:heatmap}
To qualitatively investigate the interpretability of the factorized representations learned from DeepFacto, similar to \cite{collins2018deep}, one can visualize the score matrix $\mS$. Each dimension of the score matrix $\mS$ can be reshaped back to the original activation size and be up-sampled to the size of the input using bi-linear interpolation. In Figure \ref{fig:heatmap}, the score matrix learned form the DeepFacto with $r=32$ (average attribute prediction of 81.4\%) is used where three different heat maps (out of 32) are depicted over different samples from LFWA dataset. We have found $r=4$ to be very low to represent interpretable heat maps for the attributes and $r=256$ to contain redundant heat maps. It can be seen, that the heat maps can show local and persistent attention over different face identities: \emph{eyes}, \emph{forehead}, \emph{nose}, etc.




\end{document}
