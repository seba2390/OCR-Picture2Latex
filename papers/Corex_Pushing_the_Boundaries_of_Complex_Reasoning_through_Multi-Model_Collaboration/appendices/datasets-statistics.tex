\section{Statistics and Details of Datasets}
\label{appendix:datasets}

% We evaluate CoK on 12 publicly available benchmark datasets that cover arithmetic reasoning, commonsense reasoning, symbolic reasoning and natural language understanding tasks. 
% The statistics of the datasets are shown below

% \begin{table}[htb]
% \centering
% \caption{Dataset Descriptions.}
% \setlength{\tabcolsep}{13.6pt}
% \label{tab:dataset_stats}
% \resizebox{\linewidth}{!}{
% \begin{tabular}{lcccccc}
% \toprule
% Dataset & Number of samples &Average words &Answer Format & Licence\\
% \midrule
% {CommonSenseQA} & 1,221 & 27.8 & Multi-choice & Unspecified \\
% {StrategyQA} & 2,290 & 9.6  & Yes or No  & Apache-2.0 \\
% {OpenBookQA} & 500  & 27.6  & Multi-choice & Unspecified  \\
% {ARC-c} & 1,172  & 47.5  & Multi-choice  & CC BY SA-4.0  \\
% {BoolQ} &  3,270 & 8.7 & Yes or No  &  CC BY SA-3.0 \\
% {GSM8K}  & 1,319 & 46.9 & Number & MIT License \\
% {SVAMP}  & 1,000 & 31.8 & Number & MIT License \\
% {AQuA} & 254  & 51.9 & Multi-choice  &  Apache-2.0 \\
% {MultiArith} & 600  & 31.8 & Number & CC BY SA-4.0  \\
% {GSM-Hard} & 1319  & 46.9 & Number & Apache-2.0  \\
% \bottomrule
% \end{tabular}
% }
% \end{table}

The detailed information of each dataset is shown in the follow:

\paragraph{Arithmetic reasoning}
\begin{itemize}
    % \item Last Letters \& Coin Flip
    % \citep{Wei2022Chain} are novel benchmarks to evaluate whether the LLM can solve a simple symbolic reasoning problem.
    % The last letters dataset is from \url{https://huggingface.co/datasets/ChilleD/LastLetterConcat}. The coin flip dataset is from \url{https://huggingface.co/datasets/skrishna/coin_flip}.

    \item Grade School Math (GSM8K; \citealp{cobbe2021gsm8k}): Linguistically diverse grade school math word problems created by human problem writers. The problems take between 2 and 8 steps to solve and involve elementary calculations using basic arithmetic operations.
    % \url{https://github.com/openai/grade-school-math}.
    % MIT license: \url{https://github.com/openai/grade-school-math/blob/master/LICENSE}.

    % \item Math Word Problem Repository 
    % MultiArith \citep{roy2015multiarith}.
    % % license: CC BY 4.0, dataset: \url{https://huggingface.co/datasets/ChilleD/MultiArith}.

    \item AddSub~\citep{hosseini2014addsub}: A set of simple arithmetic word problems.

    \item SVAMP \citep{patel2021svamp}: A challenge set for elementary-level Math Word Problems.
    % MIT license: \url{https://github.com/arkilpatel/SVAMP/blob/main/LICENSE}.
    
    % \item AQuA \citep{ling2017aqua}: \url{https://github.com/deepmind/AQuA}.
    % license: \url{https://github.com/deepmind/AQuA/blob/master/LICENSE}.

    \item SingleOP, SingleEQ and MultiArith~\citep{koncel2016mawps}: Grade-school math dataset that aims at solving multi-sentence algebraic word problems.

    \item GSM-Hard~\citep{gao2022pal}: A harder version of the GSM8K dataset, constructed by replacing the numbers in the questions of GSM8K with larger numbers.
    
\end{itemize}


\begin{table}[ht!]
\centering
\caption{Examples from mathematical reasoning datasets used in this work.}
\begin{tabular}{@{}lrp{6.5cm}@{}}
\toprule
Dataset   & N     & Example                                                                                                \\ \midrule
GSM8K~\citep{cobbe2021gsm8k} &
  1,319 &
  A robe takes 2 bolts of blue fiber and half that much white fiber.  How many bolts in total does it take? \\
GSM-Hard~\citep{gao2022pal} &
  1,319 &
  A robe takes 2287720 bolts of blue fiber and half that much white fiber.  How many bolts in total does it take? \\
SVAMP~\citep{patel2021svamp}    & 1,000 & Each pack of dvds costs 76 dollars. If there is a discount of 25 dollars on each pack. How much do you have to pay to buy each pack?        \\                                
SINGLEOP~\citep{koncel2016mawps} & 562   & If there are 7 bottle caps in a box and Linda puts 7 more bottle caps inside, how many bottle caps are in the box?                          \\
SINGLEEQ~\citep{koncel2016mawps}  & 508   & Benny bought a soft drink for 2 dollars and 5 candy bars. He spent a total of 27 dollars. How much did each candy bar cost?                 \\
AddSub~\citep{hosseini2014addsub}    & 395   & There were 6 roses in the vase. Mary cut some roses from her flower garden. There are now 16 roses in the vase. How many roses did she cut? \\
MultiArith~\citep{roy2015multiarith}  &
  600 &
  The school cafeteria ordered 42 red apples and 7 green apples for students lunches. But, if only 9 students wanted fruit, how many extra did the cafeteria end up with? \\ 
\bottomrule
\end{tabular}
\label{tab:math:examples}
\end{table}

\paragraph{Commonsense \& Factual reasoning}
\begin{itemize}
    \item CommonsenseQA (CSQA; \citealp{talmor2019commonsenseqa}):
    CSQA is a multiple-choice question answering task. It requires complex semantic reasoning based on prior commonsense knowledge to answer the questions.

    % The homepage is \url{https://www.tau-nlp.org/commonsenseqa}, and \url{https://github.com/jonathanherzig/commonsenseqa}.
    
    \item StrategyQA \citep{geva2021strategyqa}: It is a commonsense QA task with Yes or No answer format that requires models to perform multi-hop reasoning to answer the questions. We use the open-domain setting (question-only set) from~\citet{srivastava2023bb}.
    % \url{https://github.com/google/BIG-bench/tree/main/bigbench/benchmark_tasks/strategyqa}.
    % The original dataset is from \url{https://github.com/eladsegal/strategyqa}, MIT license: \url{https://github.com/eladsegal/strategyqa/blob/main/LICENSE}.

    \item OpenBookQA
    \citep{mihaylov2018openbookqa}: It is a multi-choice QA task to evaluate commonsense knowledge and promote reasoning over a fixed collection of knowledge. 
    % The original dataset is from \url{https://allenai.org/data/open-book-qa}.

    % \item OpenBookQA (\citealp{mihaylov2018openbookqa}): OpenBookQA is a dataset designed for question answering tasks where the models are required to utilize a set of open-source "science facts" to answer multiple-choice questions, promoting reasoning over a fixed collection of knowledge.

    \item ARC-c
    \citep{clark2018think}: A subset of the AI2 Reasoning Challenge, consisting of challenging science questions that require reasoning and a wide breadth of knowledge to answer the multiple-choice problems correctly.
    The original dataset is from \url{https://allenai.org/data/arc}. 
    % CC BY SA-4.0 license: \url{https://creativecommons.org/licenses/by-sa/4.0/}.

    % ARC-c is a subset of the ARC dataset consisting of challenging science questions that require deep reasoning and a wide breadth of knowledge to answer correctly. It aims to test the limits of machine comprehension and reasoning capabilities in the context of educational material.
    
    \item BoolQ
    \citep{clark2019boolq}: It is a knowledge-intensive task and the format is ``Yes'' or ``No''.
    Problems are extracted from real-world internet queries,
    aiming to foster models capable of contextual understanding to provide binary answers.
    % The original dataset is from \url{https://github.com/google-research-datasets/boolean-questions}. CC BY SA-3.0 license: \url{https://creativecommons.org/licenses/by-sa/3.0/}.

\end{itemize}

\begin{table}[ht!]
\centering
\caption{Examples from commonsense \& factual reasoning datasets used in this work.}
\begin{tabular}{@{}lrp{6cm}@{}}
\toprule
Dataset   & N     & Example                                                                                            \\ 
\midrule
StrategyQA~\citep{cobbe2021gsm8k} & 2,290 & Hydrogen's atomic number squared exceeds number of Spice Girls? \\
CommonsenseQA (CSQA; \citealp{talmor2019commonsenseqa}) & 1,221 & A robe takes 2 bolts of blue fiber and half that much white fiber.  How many bolts in total does it take? \\
OpenBookQA~\citep{mihaylov2018openbookqa} & 500 & In which location would a groundhog hide from a wolf? \\
ARC-c~\citep{clark2018think} & 1,172 & An astronomer observes that a planet rotates faster after a meteorite impact. Which is the most likely effect of this increase in rotation? \\
BoolQ~\citep{clark2019boolq} & 3,270 & A robe takes 2 bolts of blue fiber and half that much white fiber.  How many bolts in total does it take? \\
\bottomrule
\end{tabular}
\label{tab:qa:examples}
\end{table}


\paragraph{Symbolic Reasoning}

We select the following tasks from BIG-Bench~\citep{srivastava2023bb} and BIG-Bench Hard (BBH)~\citep{suzgun2023bbh}, with Apache License v.2: \url{https://github.com/google/BIG-bench/blob/main/LICENSE}.

\begin{itemize}
    % \item Sports understanding: Determine whether a factitious sentence related to sports is plausible. The answer format is Yes or No.
    \item Date Understanding: A temporal reasoning task. Given a set of sentences about a particular date,
    answer the provided question in MM/DD/YYYY format.
    \item Object Counting: Given a collection of possessions that a person has along with their quantities (e.g., three pianos, two strawberries, one table, and two watermelons), determine the number of a certain object/item class (e.g., fruits).
    \item Penguins in a Table: Given a unique table of penguins (and sometimes some new information), 
    answer a question about the attributes of the penguins.
    \item Reasoning about Colored Objects: Given a context, 
    answer a simple question about the color of an object on a surface.
    \item Repeat Copy: Evaluate LLMs' capability to follow basic natural-language instructions nested within each example's input.

\end{itemize}

\begin{table}[ht!]
\centering
\caption{Examples from symbolic reasoning datasets used in this work.}
\begin{tabular}{@{}lrp{6.5cm}@{}}
\toprule
Dataset   & N     & Example                                                                                                \\ 
\midrule
Date Understanding     & 250 & Yesterday was April 30, 2021. What is the date today in MM/DD/YYYY? \\
Object Counting        & 250  & I have an apple, three bananas, a strawberry, a peach, three oranges, a plum, a raspberry, two grapes, a nectarine, and a blackberry. How many fruits do I have? \\
Penguins in a Table    & 146 & A Here is a table where the first line is a header and each subsequent line is a penguin:  name, age, height (cm), weight (kg) Louis, 7, 50, 11 Bernard, 5, 80, 13 Vincent, 9, 60, 11 Gwen, 8, 70, 15  For example: the age of Louis is 7, the weight of Gwen is 15 kg, the height of Bernard is 80 cm.  How many penguins are more than 5 years old? \\
Colored Objects        & 250 & On the desk, you see a bunch of items arranged in a row: a gold textbook, a purple puzzle, a teal necklace, and a silver pencil. How many non-gold items do you see to the right of the pencil? \\
Repeat Copy            & 32   & Repeat the word cat four times. After the second time, also say the word meow. \\
\bottomrule
\end{tabular}
\label{tab:bbh:examples}
\end{table}


\paragraph{Semi-structured Understanding}

\begin{itemize}
    \item FinQA~\citep{chen2021finqa}: Question-Answering pairs over financial reports written by experts, which includes financial QA pairs.
    \item ConvFinQA~\citep{chen2022convfinqa}: A financial-related dataset designed to study the chain of numerical reasoning in conversational QA.
    \item TAT-QA~\citep{zhu2021tatqa}: A QA dataset aiming to stimulate the progress of research over more complex and realistic tabular and textual data.
\end{itemize}

\begin{table}[ht!]
\caption{Examples from semi-structured reasoning datasets used in this work.}
\centering
\begin{tabular}{@{}lrp{7.25cm}@{}}
\toprule
Dataset   & N     & Example                                                                                                \\ 
\midrule
FinQA~\citep{chen2021finqa} & 1,147 & Question: what percentage of total facilities as measured in square feet are leased? Text: unresolved staff comments not applicable. properties as of december 26 , 2015 , our major facilities consisted of : ( square feet in millions ) united states countries total owned facilities. 30.7 17.2 47.9 leased facilities ... Table: ( square feet in millions ) | unitedstates | othercountries | total owned facilities1 | 30.7 | 17.2 | 47.9 leased facilities2 | 2.1 | 6.0 | 8.1 total facilities | 32.8 | 23.2 | 56.0", \\
ConvFinQA~\citep{chen2022convfinqa} & 421 & Question: what were the total accumulated other comprehensive losses in 2015? Text: accumulated other comprehensive losses : pmi's accumulated other comprehensive losses , net of taxes , consisted of the following: reclassifications from other comprehensive earnings the movements in accumulated other comprehensive losses and the related tax impact , for each of the components above ... Table: ( losses ) earnings ( in millions ) | ( losses ) earnings 2015 | ( losses ) earnings 2014 | 2013 currency translation adjustments | \$ -6129 ( 6129 ) | \$ -3929 ( 3929 ) | \$ -2207 ( 2207 ) pension and other benefits | -3332 ( 3332 ) | -3020 ( 3020 ) | -2046 ( 2046 ) derivatives accounted for as hedges | 59 | 123 | 63 total accumulated other comprehensive losses | \$ -9402 ( 9402 ) | \$ -6826 ( 6826 ) | \$ -4190 ( 4190 ) \\
TAT-QA~\citep{zhu2021tatqa} & 718 & Question: What is the change in Other in 2019 from 2018? Text: Sales by Contract Type: Substantially all of our contracts are fixed-price type contracts. Sales included in Other contract types represent cost plus and time and material type contracts. On a fixed-price type contract, we agree to perform the contractual statement of work for a predetermined sales price ... Table: 2014 | 2014 | Years Ended September 30, | 2014 2014 | 2019 | 2018 | 2017 Fixed Price | \$ 1,452.4 | \$  1,146.2 | \$ 1,036.9 Other | 44.1 | 56.7 | 70.8 Total sales | \$1,496.5 | \$1,202.9 | \$1,107.7 \\

\bottomrule
\end{tabular}
\label{tab:semi:examples}
\end{table}



