\begin{table}[htbp]
\centering
\begin{tabular}{p{0.9\textwidth}}
\toprule
\underline{\textbf{\textsc{FinQA: NL Prompts}}} \\
Read the following text and table, and then answer a question:\\
\$ in millions | year ended December 2014 | year ended December 2013 | year ended December 2012 \\
fixed income currency and commodities client execution | \$ 8461 | \$ 8651 | \$ 9914 \\
equities client execution1 | 2079 | 2594 | 3171 \\
% commissions and fees | 3153 | 3103 | 3053 \\
... \\
% securities services | 1504 | 1373 | 1986 \\
% total equities | 6736 | 7070 | 8210 \\
% total net revenues | 15197 | 15721 | 18124 \\
% operating expenses | 10880 | 11792 | 12490 \\
pre-tax earnings | \$ 4317 | \$ 3929 | \$ 5634 \\
\textbf{Q}: what was the percentage change in pre-tax earnings for the institutional client services segment between 2012 and 2013? \\
        \textbf{A}: The pre-tax earnings for the institutional client services segment in 2012 was \$ 5634 million, and in 2013 was \$ 3929 million.\\
        The net change in pre-tax earnings was \$ 1705 million, \\
        and the percentage change was 30.3\%. So the answer is 30.3\%. \\
        \midrule
        \underline{\textbf{\textsc{FinQA: PAL Prompts}}} \\
\textbf{Q}: what percentage of total cash and investments as of dec . 29 2012 was comprised of available-for-sale investments? \\
Read the following text and table, and then write code to answer a question: \\
( in millions ) | dec 282013 | dec 292012 \\
available-for-sale investments | \ 18086 | \$ 14001 \\
% cash | 854 | 593 \\
% equity method investments | 1038 | 992 \\
... \\
% loans receivable | 1072 | 979 \\
% non-marketable cost method investments | 1270 | 1202 \\
% reverse repurchase agreements | 800 | 2850 \\
trading assets | 8441 | 5685 \\
total cash and investments | \$ 31561 | \$ 26302 
\begin{minted}[breaklines, fontsize=\scriptsize]{python}
#Python
available_for_sale_investments_dec_29_2012 = 14001
total_cash_and_investments_dec_29_2012 = 26302
percent_available_for_sale_investments_dec_29_2012 = available_for_sale_investments_dec_29_2012 / total_cash_and_investments_dec_29_2012
ans = percent_available_for_sale_investments_dec_29_2012
\end{minted}
\\
\midrule
\underline{\textbf{\textsc{TAT-QA: NL Prompts}}} \\
Read the following text and table, and then answer a question:\\
The average monthly number of employees (including Executive Directors but excluding third-party contractors) employed by the Group was as follows: — | 2019 | 2018 \\
— | Number | Number \\
Customer operations | 370 | 380 \\
Product and technology | 317 | 312 \\
Corporate | 115 | 130 \\
Total | 802 | 822 \\
\textbf{Q}: What are the categories of employees listed in the table? \\
        \textbf{A}: The answer can be found directly in the table above. \\
        So the answer is ['Customer operations', 'Product and technology', 'Corporate']. \\
\midrule
\underline{\textbf{\textsc{TAT-QA: PAL Prompts}}} \\
\textbf{Q}: What are the categories of employees listed in the table? \\
Read the following text and table, and then write code to answer a question: \\
The average monthly number of employees (including Executive Directors but excluding third-party contractors) employed by the Group was as follows:
— | 2019 | 2018 \\
— | Number | Number \\
Customer operations | 370 | 380 \\
Product and technology | 317 | 312 \\
Corporate | 115 | 130 \\
Total | 802 | 822 
\begin{minted}[breaklines, fontsize=\footnotesize]{python}
#Python
ans = ['Customer operations', 'Product and technology', 'Corporate']
\end{minted}
\\
        \bottomrule
    \end{tabular}
     \caption{Examples of few-shot exemplars for FinQA and TAT-QA. For CoT, CoT-SC and PAL,
     4-shots are adopted for FinQA and 8-shots are adopted for TAT-QA.}
    \label{tab:exp-tables}
\end{table}

