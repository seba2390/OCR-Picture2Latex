
\begin{table}[htbp]
\centering
\begin{tabular}{p{0.9\textwidth}}
\toprule
\underline{\textbf{\textsc{Penguins: NL Prompts}}} \\
\textbf{Q}: Here is a table where the first line is a header and each subsequent line is a penguin:  name, age, height (cm), weight (kg) Louis, 7, 50, 11 Bernard, 5, 80, 13 Vincent, 9, 60, 11 Gwen, 8, 70, 15  For example: the age of Louis is 7, the weight of Gwen is 15 kg, the height of Bernard is 80 cm.  We now add a penguin to the table:
James, 12, 90, 12 \\
How many penguins are less than 8 years old? \\
\textbf{Options:}
(A) 1
(B) 2
(C) 3
(D) 4
(E) 5 \\
\textbf{A}: Let's think step by step. \\
This question focuses on age. We know the following: Louis is 7 years old, Bernard is 5 years old, Vincent is 9 years old, and Gwen is 8 years old.
Now, we add James to this table: James is 12 years old. \\
The penguins that are less than 8 years old are Louis and Bernard. \\
There are 2 penguins less than 8 years old. So the answer is (B).\\
\midrule
\underline{\textbf{\textsc{Penguins: PAL Prompts}}} \\
\textbf{Q}: Here is a table where the first line is a header and each subsequent line is a penguin:  name, age, height (cm), weight (kg) Louis, 7, 50, 11 Bernard, 5, 80, 13 Vincent, 9, 60, 11 Gwen, 8, 70, 15  For example: the age of Louis is 7, the weight of Gwen is 15 kg, the height of Bernard is 80 cm.  We now add a penguin to the table:
James, 12, 90, 12 \\
How many penguins are less than 8 years old? 
\\
\textbf{Options:}
(A) 1
(B) 2
(C) 3
(D) 4
(E) 5 \\
\begin{minted}[breaklines, fontsize=\footnotesize]{python}
"""
Q: Here is a table where the first line is a header and each subsequent line is a penguin:
name, age, height (cm), weight (kg) 
Louis, 7, 50, 11
Bernard, 5, 80, 13
Vincent, 9, 60, 11
Gwen, 8, 70, 15
For example: the age of Louis is 7, the weight of Gwen is 15 kg, the height of Bernard is 80 cm. 
We now add a penguin to the table:
James, 12, 90, 12
How many penguins are less than 8 years old?
"""

# Put the penguins into a list.
penguins = []
penguins.append(('Louis', 7, 50, 11))
penguins.append(('Bernard', 5, 80, 13))
penguins.append(('Vincent', 9, 60, 11))
penguins.append(('Gwen', 8, 70, 15))

# Add penguin James.
penguins.append(('James', 12, 90, 12))

# Find penguins under 8 years old.
penguins_under_8_years_old = [penguin for penguin in penguins if penguin[1] < 8]

# Count number of perguins under 8.
num_penguin_under_8 = len(penguins_under_8_years_old)
answer = num_penguin_under_8
\end{minted}
\\
\bottomrule
% \midrule
% \underline{\textbf{\textsc{TAT-QA: NL Prompts}}} \\
% Read the following text and table, and then answer a question:\\
% The average monthly number of employees (including Executive Directors but excluding third-party contractors) employed by the Group was as follows: \\
% — | 2019 | 2018 \\
% — | Number | Number \\
% Customer operations | 370 | 380 \\
% Product and technology | 317 | 312 \\
% Corporate | 115 | 130 \\
% Total | 802 | 822 \\
% \textbf{Q}: What are the categories of employees listed in the table? \\
%         \textbf{A}: The answer can be found directly in the table above. \\
%         So the answer is ['Customer operations', 'Product and technology', 'Corporate']. \\
% \midrule
% \underline{\textbf{\textsc{TAT-QA: PAL Prompts}}} \\
% \textbf{Q}: What are the categories of employees listed in the table? \\
% Read the following text and table, and then write code to answer a question: \\
% The average monthly number of employees (including Executive Directors but excluding third-party contractors) employed by the Group was as follows:
% — | 2019 | 2018 \\
% — | Number | Number \\
% Customer operations | 370 | 380 \\
% Product and technology | 317 | 312 \\
% Corporate | 115 | 130 \\
% Total | 802 | 822 
% \begin{minted}[breaklines, fontsize=\footnotesize]{python}
% #Python
% ans = ['Customer operations', 'Product and technology', 'Corporate']
% \end{minted}
% \\
%         \bottomrule
    \end{tabular}
     \caption{Examples of few-shot exemplars for Penguins. 3-shots are adopted for CoT, CoT-SC, and PAL.}
    \label{tab:exp-penguins}
\end{table}



