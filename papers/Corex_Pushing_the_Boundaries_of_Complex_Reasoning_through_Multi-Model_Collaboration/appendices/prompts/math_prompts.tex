\begin{table}[htbp]
    \centering
    \begin{tabular}{p{0.9\textwidth}}
        \toprule
        \underline{\textbf{\textsc{Mathematical Reasoning: NL Prompts}}} \\
\textbf{Q}: Olivia has \$23. She bought five bagels for \$3 each. How much money does she have left?
\textbf{A}: Olivia had 23 dollars. 5 bagels for 3 dollars each will be 5 x 3 = 15 dollars. \\
So she has 23 - 15 dollars left.  \\
23 - 15 is 8. So the answer is 8. \\
        \midrule
        \underline{\textbf{\textsc{Mathematical Reasoning: PAL Prompts}}} \\
 \textbf{Q}: Olivia has \$23. She bought five bagels for \$3 each. How much money does she have left?
\begin{minted}[breaklines, fontsize=\footnotesize]{python}
# Q: Olivia has $23. She bought five bagels for $3 each. How much money does she have left?

# solution in Python:

def solution():
    """Olivia has $23. She bought five bagels for $3 each. How much money does she have left?"""
    money_initial = 23
    bagels = 5
    bagel_cost = 3
    money_spent = bagels * bagel_cost
    money_left = money_initial - money_spent
    result = money_left
    return result
\end{minted}
        \\
        \midrule
\underline{\textbf{\textsc{Mathematical Reasoning: Complexity-based Prompts}}} \\
\textbf{Q}: Angelo and Melanie want to plan how many hours over the next week they should study together for their test next week. They have 2 chapters of their textbook to study and 4 worksheets to memorize. They figure out that they should dedicate 3 hours to each chapter of their textbook and 1.5 hours for each worksheet. If they plan to study no more than 4 hours each day, how many days should they plan to study total over the next week if they take a 10-minute break every hour, include 3 10-minute snack breaks each day, and 30 minutes for lunch each day? \\
\textbf{A}: Let's think step by step. \\
Angelo and Melanie think they should dedicate 3 hours to each of the 2 chapters, 3 hours x 2 chapters = 6 hours total. \\
For the worksheets they plan to dedicate 1.5 hours for each worksheet, 1.5 hours x 4 worksheets = 6 hours total. \\
Angelo and Melanie need to start with planning 12 hours to study, at 4 hours a day, 12 / 4 = 3 days. \\
However, they need to include time for breaks and lunch. Every hour they want to include a 10-minute break, so 12 total hours x 10 minutes = 120 extra minutes for breaks. \\
They also want to include 3 10-minute snack breaks, 3 x 10 minutes = 30 minutes. \\
And they want to include 30 minutes for lunch each day, so 120 minutes for breaks + 30 minutes for snack breaks + 30 minutes for lunch = 180 minutes, or 180 / 60 minutes per hour = 3 extra hours. \\
So Angelo and Melanie want to plan 12 hours to study + 3 hours of breaks = 15 hours total. \\
They want to study no more than 4 hours each day, 15 hours / 4 hours each day = 3.75 \\
They will need to plan to study 4 days to allow for all the time they need. \\
So the answer is 4. \\
        \bottomrule
    \end{tabular}
     \caption{Examples of few-shot exemplars for mathematical reasoning tasks. 8-shots are adopted for all the datasets covered in this paper when using CoT, CoT-SC, and ComplexCoT. 3-shots are adopted when using PAL.}
    \label{tab:exp-math}
\end{table}


