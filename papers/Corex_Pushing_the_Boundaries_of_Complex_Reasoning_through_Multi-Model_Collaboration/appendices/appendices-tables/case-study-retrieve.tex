\begin{table}[ht]
\begin{tabular}{p{13cm}}
\caption{An illustrative example of the retrieve process of \ours among five agents (four candidates and one retriever) over a mathematical problem from SingleEQ.} 
\label{table:retrieve-case-study} \\
\toprule
\textbf{Q}: Melanie is selling 4 gumballs for eight cents each. How much money can Melanie get from selling the gumballs? \\
\midrule
\begin{minipage}[b]{0.05\columnwidth}
    \centering
    \raisebox{-.1in}{\includegraphics[width=\linewidth]{icons/player1_down.png}}
\end{minipage}
Melanie is selling 4 gumballs for eight cents each. That means each gumball costs 8/4 = 2 cents. So from selling the 4 gumballs, Melanie can get 4 x 2 = 8 cents. The answer is 8 cents.  \\
\vspace{-1mm}
\begin{minipage}[b]{0.05\columnwidth}
    \centering
    \raisebox{-.1in}{\includegraphics[width=\linewidth]{icons/player2_down.png}}
\end{minipage}
Melanie is selling 4 gumballs for 8 cents each. So she can get 4 x 8 = 32 cents from selling the gumballs. The answer is 32 cents. \\
\vspace{-1mm}
\begin{minipage}[b]{0.05\columnwidth}
    \centering
    \raisebox{-.1in}{\includegraphics[width=\linewidth]{icons/player3_down.png}}
\end{minipage}
Melanie is selling 4 gumballs for eight cents each. So she can get 4 x 8 = 32 cents from selling the gumballs. The answer is 32 cents. \\
\vspace{-1mm}
\begin{minipage}[b]{0.05\columnwidth}
    \centering
    \raisebox{-.1in}{\includegraphics[width=\linewidth]{icons/player5_down.png}}
\end{minipage}
Melanie is selling 4 gumballs for 8 cents each. So for each gumball, she can get 8 / 4 = 2 cents. Therefore, she can get a total of 2 x 4 = 8 cents from selling the gumballs. The answer is 8 cents. \\
\midrule
\begin{minipage}[b]{0.05\columnwidth}
    \centering
    \raisebox{-.1in}{\includegraphics[width=\linewidth]{icons/player4_down.png}}
\end{minipage}
Candidates 2 and 3 are accurate in their computations, emphasizing that each gumball costs 8 cents and that Melanie would earn a total of 32 cents from the sale of all gumballs. In contrast, Candidates 1 and 4 mistakenly deduce the price per gumball, thereby providing an incorrect total earning. Therefore, the most faithful answers are provided by candidate 2 and candidate 3 with a confidence score of 1.\\
\bottomrule
\end{tabular}
\end{table}
