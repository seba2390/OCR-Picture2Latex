\begin{table}[ht]
\begin{tabular}{p{13cm}}
\caption{An illustrative example of the debate process of \ours with a group between Tom \begin{minipage}[b]{0.045\columnwidth}
    \centering
    \raisebox{-.1in}{\includegraphics[width=\linewidth]{icons/player1_down.png}}
\end{minipage} and Jerry \begin{minipage}[b]{0.045\columnwidth}
    \centering
    \raisebox{-.1in}{\includegraphics[width=\linewidth]{icons/player2_down.png}}
\end{minipage} over a mathematical problem from GSM8K.} 
\label{table:debate-case-study} \\
\toprule
\textbf{Q}: Mark took a test yesterday that consisted of 75 questions. He completed the test at a rate of 5 questions per hour. Today, he took another test of 100 questions at the same rate. If Mark had 8 hours to complete the first test and 6 hours to complete the second one, how many questions did he leave incomplete? \\
\midrule
\begin{minipage}[b]{0.05\columnwidth}
    \centering
    % \raisebox{-.5\height}{\includegraphics[width=\linewidth]{icons/player1_down.png}}
    \raisebox{-.1in}{\includegraphics[width=\linewidth]{icons/player1_down.png}}
\end{minipage}
Mark completed the first test at a rate of 5 questions per hour with a total of 8 hours, so he completed 5 * 8 = 40 questions on the first test.He completed the second test at the same rate of 5 questions per hour with a total of 6 hours, so he completed 5 * 6 = 30 questions on the second test. Therefore, Mark completed a total of 40 + 30 = 70 questions from both tests.Since the first test consisted of 75 questions and Mark completed 40 questions, he left 75 - 40 = 35 questions incomplete in the first test. Similarly, the second test consisted of 100 questions and Mark completed 30 questions, so he left 100 - 30 = 70 questions incomplete in the second test. Therefore, Mark left a total of 35 + 70 = 105 questions incomplete.  \\
\vspace{-1mm}
\begin{minipage}[b]{0.05\columnwidth}
    \centering
    \raisebox{-.1in}{\includegraphics[width=\linewidth]{icons/player2_down.png}}
\end{minipage}
Mark completed the first test at a rate of 5 questions per hour, and he had 8 hours to complete it. So he completed 8 * 5 = 40 questions. He completed the second test at the same rate, 5 questions per hour, and he had 6 hours to complete it. So he completed 6 * 5 = 30 questions. Therefore, Mark completed a total of 40 + 30 = 70 questions. Since the first test had 75 questions and Mark completed 70 questions, he left 75 - 70 = 5 questions incomplete. Therefore, Mark left 5 questions incomplete. \\
\vspace{-1mm}
\begin{minipage}[b]{0.05\columnwidth}
    \centering
    \raisebox{-.1in}{\includegraphics[width=\linewidth]{icons/player1_down.png}}
\end{minipage}
Jerry's analysis is mostly correct, but there is a small mistake in the final calculation. Let's break down the solution step by step: From the given information: Mark completed the first test at a rate of 5 questions per hour, and he had 8 hours to complete it. So he completed 8 * 5 = 40 questions on the first test. He completed the second test at the same rate, 5 questions per hour, and he had 6 hours to complete it. So he completed 6 * 5 = 30 questions on the second test. Therefore, Mark completed a total of 40 + 30 = 70 questions. However, the mistake was made in the next step: Since the first test had 75 questions and Mark completed 40 questions, the number of questions he left incomplete on the first test is 75 - 40 = 35 questions. Similarly, the second test had 100 questions and Mark completed 30 questions, so he left 100 - 30 = 70 questions incomplete on the second test. To find the total number of questions left incomplete, we need to add the number of questions left incomplete on both tests: 35 + 70 = 105. So, the answer is Mark left 105 questions incomplete. \\
\vspace{-1mm}
\begin{minipage}[b]{0.05\columnwidth}
    \centering
    \raisebox{-.1in}{\includegraphics[width=\linewidth]{icons/player2_down.png}}
\end{minipage}
You are correct, Tom. I apologize for the mistake in my previous calculation. Thank you for pointing it out and providing the correct solution. To recap: Mark completed 8 * 5 = 40 questions on the first test. He completed 6 * 5 = 30 questions on the second test. Therefore, Mark completed a total of 40 + 30 = 70 questions. Since the first test had 75 questions and Mark completed 40 questions, the number of questions he left incomplete on the first test is 75 - 40 = 35 questions. Similarly, the second test had 100 questions and Mark completed 30 questions, so he left 100 - 30 = 70 questions incomplete on the second test. To find the total number of questions left incomplete, we need to add the number of questions left incomplete on both tests: 35 + 70 = 105. So the answer is Mark left 105 questions incomplete. \\
\midrule
The debate ends when both two players reach an agreement: answer=105.
Tom and Jerry got the correct answer of 105 instead of Jerry's incorrect answer of 5. \\
\bottomrule
\end{tabular}
\end{table}

% \begin{table}[ht]
% \begin{tabular}{p{13cm}}
% \toprule
% \textbf{Q}: Marissa is hiking a 12-mile trail. She took 1 hour to walk the first 4 miles, then another hour to walk the next two miles. If she wants her average speed to be 4 miles per hour, what speed (in miles per hour) does she need to walk the remaining distance? \\
% \midrule
% \begin{minipage}[b]{0.05\columnwidth}
%     \centering
%     \raisebox{-.5\height}{\includegraphics[width=\linewidth]{icons/player1_down.png}}
% \end{minipage}
% To find the speed Marissa needs to walk the remaining distance, we need to first calculate the time it took her to walk the first 6 miles (4 miles + 2 miles). Given that it took her 1 hour to walk the first 4 miles and another hour to walk the next 2 miles, the total time she spent walking the first 6 miles is 1 hour + 1 hour = 2 hours. Now, let's find the time it would take Marissa to walk the remaining 6 miles at the desired average speed of 4 miles per hour. We can use the formula: time = distance / speed. The distance is 6 miles and the speed is 4 miles per hour. Plugging these values into the formula, we get: time = 6 miles / 4 miles per hour = 1.5 hours. Therefore, Marissa needs to walk the remaining distance at a speed of 1.5 hours per 6 miles, or simplifying further, 1 hour per 4 miles."  \\
% \vspace{-1mm}
% \begin{minipage}[b]{0.05\columnwidth}
%     \centering
%     \raisebox{-.5\height}{\includegraphics[width=\linewidth]{icons/player2_down.png}}
% \end{minipage}
% Marissa is hiking a 12-mile trail and she has already walked a total of 4 + 2 = 6 miles. To find her average speed of 4 miles per hour, she needs to complete the remaining distance at a certain speed. Let's assume she needs to walk x miles per hour for the remaining distance. The remaining distance is 12 - 6 = 6 miles. The time it takes for her to walk the remaining distance is given by 6 / x.The total time it took for her to walk the 6 miles she has already walked is 1 + 1 = 2 hours. The total time it should take for her to complete the entire 12-mile trail is 12 / 4 = 3 hours. Therefore, the equation for the total time is 2 + (6 / x) = 3. To solve for x, we need to isolate it on one side of the equation. Subtracting 2 from both sides gives us 6 / x = 3 - 2 = 1. Multiplying both sides by x gives us 6 = x. Therefore, Marissa needs to walk the remaining distance at a speed of 6 miles per hour. \\
% \vspace{-1mm}
% \begin{minipage}[b]{0.05\columnwidth}
%     \centering
%     \raisebox{-.5\height}{\includegraphics[width=\linewidth]{icons/player1_down.png}}
% \end{minipage}
% Jerry's solution is mostly correct, but there is a small error in the last step of the calculation. Let's go through the solution together: Jerry correctly identified that Marissa has already walked 6 miles and needs to complete the remaining distance of 6 miles at a certain speed. The equation for the total time is 2 + (6 / x) = 3, where x is the speed at which Marissa needs to walk the remaining distance. To solve for x, we need to isolate it on one side of the equation. By subtracting 2 from both sides of the equation, we get (6 / x) = 3 - 2 = 1. Jerry made a mistake in the last step by stating that multiplying both sides by x gives 6 = x. However, this is incorrect. To solve for x, we need to multiply both sides of the equation by x, not divide. Multiplying both sides by x gives us x * (6 / x) = x * 1. This simplifies to 6 = x. So the correct answer is that Marissa needs to walk the remaining distance at a speed of 6 miles per hour. \\
% \vspace{-1mm}
% \begin{minipage}[b]{0.05\columnwidth}
%     \centering
%     \raisebox{-.5\height}{\includegraphics[width=\linewidth]{icons/player2_down.png}}
% \end{minipage}
% wait \\
% \vspace{-1mm}
% \begin{minipage}[b]{0.05\columnwidth}
%     \centering
%     \raisebox{-.5\height}{\includegraphics[width=\linewidth]{icons/player1_down.png}}
% \end{minipage}
% wait\\
% \vspace{-1mm}
% \begin{minipage}[b]{0.05\columnwidth}
%     \centering
%     \raisebox{-.5\height}{\includegraphics[width=\linewidth]{icons/player2_down.png}}
% \end{minipage}
% wait \\
% \bottomrule
% \caption{The debate process of \ours between Tom and Jerry over a problem from GSM8K.}
% \end{tabular}
% \label{table:debate-case}
% \end{table}
