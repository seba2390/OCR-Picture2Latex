\begin{figure}
 \centering
 \includegraphics[width=12cm]{cr3.png}
  \caption{{\bf Bifurcation diagram and sample graphs of the logistic map.} 
%   The logistic map has a top node, the point 0, that for simplicity is not shown.
  This picture shows the bifurcation diagram of the logistic map in the range of parameter values $[2.9,4]$.
  For each value of $\mu$, the attracting set is painted in shades of gray, depending on the density of the attractor, repelling periodic orbits in green and repelling Cantor sets in red. Below the $\mu$ axis we show seven samples of the graphs illustrating some of the possible variability. In these graphs, each colored disk is a node. Each black disk represents an attractor, each green disk represents a repelling periodic orbit and red represents a chaotic Cantor set repellor. For simplicity we always omit the top node, which is the point 0. Graph T4 represents the infinite tower at the first Feigenbaum point. It has infinitely many unstable periodic orbit nodes.
  }
  \label{fig:full}
\end{figure}

%%%%%%%%%%%%%%%

\begin{figure}
 \centering
 \includegraphics[width=13cm]{crsmall}
 \caption{{\bf Towers of nodes shown below the period-3 window of the logistic map bifurcation diagram.}
    This figure is a blow-up from Fig.~\ref{fig:full}, and uses the color-coding from that figure. Graph T8 has two levels of nodes that are Cantor sets repellors and the second is painted in blue.
 %The period-3 window of the logistic map starts at and ends at ...
 %$\mu\INT$ (\ref{muInt}).
 %=1.5[ 1 + (19-3\sqrt{33})^{1/3} + (19+3\sqrt{33})^{1/3} ]\simeq3.67857$,
% The black dots on the $\mu$ axis are sample values sampling some of the possible towers near the value where the period three orbit first appears, $\mu = 1+\sqrt 8\simeq3.8284$.
%Red nodes denote fractal chaotic saddles. Blue and green nodes are periodic orbits. Black denotes attracting sets which occur at the bottom of the tower. 
In the bifurcation diagram, the chain recurrent sets have the same coloring as their nodes. 
Graph T5 represents the infinite tower at the first Feigenbaum point of the main cascade of the period-3 window. 
%It has infinitely many unstable periodic orbit nodes.
%Notice that we do not include in the pictures of the towers the zero node, which is always on top.
}
 \label{fig:p3}
\end{figure} 
%%%%%%%%%%%%%%%
