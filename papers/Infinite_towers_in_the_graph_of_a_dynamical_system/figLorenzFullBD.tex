\begin{figure}
 \centering
 \includegraphics[width=12.21cm]{lorBD}
% \includegraphics[width=12.21cm,height=6cm]{l1}\\
% \includegraphics[width=12cm,height=6cm]{l1x}\\
% \includegraphics[width=12cm,height=6cm]{l1a}\\
 \caption{{\bf Bifurcation Diagram}. 
    These are the projections onto the $(r,y)$ plane (Top panel), $(r,x)$ plane (Middle panel) and $(y-x,r)$ plane (Bottom panel) of the bifurcation diagram for the Poincar\'e Return map of the Lorenz equations~(\ref{eq:Lor}) using the plane $\pi_r$ defined by $z=r-1$. 
    A dot is plotted in the $(r,y)$ (resp. $(r,x)$) plane when a trajectory crosses downward past $\pi_r$ through the point $(x,y,r-1)$. 
    The regions where there is speckled white and magenta dots is where the attractor is low density. 
    The Lorenz attractor typically has great variations in density, so extremely long trajectories would be needed to reveal such parts of the attractor. 
    In Fig.~\ref{fig:p4}, we show details of the period-4 window that is centered around $r=150$. 
    The period-3 window shown in Fig.~\ref{fig:lcr} is an enlargement of the black rectangle shown in the top projection.
    For some parameter values, there are two attractors.
    They are shown in red and blue.
    When there is a single attractor, it is shown in magenta.
    }
 \label{fig:lfull}
\end{figure} 