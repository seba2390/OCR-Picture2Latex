\begin{figure}
 \centering
 \includegraphics[width=12cm]{l6.png}
 \caption{{\bf A small region from the Lorenz system Poincar\'e return map {\BF $P_\rhoo$} for {\BF $\rhoo=209.2$}.} 
 The region shown corresponds to the small yellow region on the right-hand side of Fig.~\ref{fig:l3d}.
 In that figure, a periodic orbit is shown piercing the yellow rectangle in three points.
 Those points are shown here as the centers of three circles. 
 Almost all points in the colored region are in the basin of attraction of the periodic orbit. Yellow indicates rapid convergence to the periodic orbit. Red indicates slow convergence. Red points are close to points that are attracted to the Cantor set on the blue line.
 %Figure~\ref{fig:l3d} shows, in blue and orange, the two attracting periodic orbits of the Lorenz system for $\rhoo=209.2$. The two period-6 attractors of the corresponding Poincar\'e map are their intersection with the large gray rectangle in Fig.~\ref{fig:l3d}. 
 %Each orbit corresponds to a period-6 orbit of the Poincar\'e return map, $P_\rhoo$.  
 %Correspondingly, $P^2_\rhoo$ has four period-3 attractors, only one of which is shown in this figure. 
 %Its three points are circled. (In Fig.~\ref{fig:l3d}, these are the three intersections of the blue orbit with the yellow rectangle, which contains the region shown here). 
 The blue curve is the unstable manifold of a Cantor set that lies within it.
 Points in the white region are attracted to the other off-screen attractor. 
 %(in Fig.~\ref{fig:l3d}, it is the second set of three points where the blue attractor intersects the gray rectangle). 
 The blue curve includes a chain-recurrent Cantor set of saddle points and the unstable manifolds of all of the periodic orbits in the Cantor set. 
 }
 \label{fig:l2dtm}
\end{figure} 