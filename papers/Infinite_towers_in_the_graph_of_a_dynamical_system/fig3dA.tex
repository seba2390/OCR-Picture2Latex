\begin{figure}
 \centering
\includegraphics[width=11cm]{l3d2}
% \includegraphics[width=5cm]{l3d-r208} \includegraphics[width=5cm]{l3d-r209_2}
 \caption{{\bf Attractors of the Lorenz system}. 
    This picture shows the attractors of the Lorenz system for $\rhoo=208$ (left) and $\rhoo=209.2$ (right) together with the rectangle $-20\leq x\leq20$, $-100\leq y\leq100$ in the plane $z=\rhoo-1$. 
    If $z$ is thought of as the vertical coordinate, both pictures are viewed from below.
    The Poincar\'e map for the Lorenz system is built out of the intersections of the Lorenz orbits crossing this rectangle downwards. The yellow rectangle shown for $\rhoo=209.2$ is the one shown (not in scale) in Fig.~\ref{fig:l2dtm} and the three intersections of the blue orbit are at the center of the three little circles shown in that picture. 
 %The intersections of the attractors with the planes $z=\rhoo-1$, which are the attractor for the corresponding Poincar\'e map, are shown in Fig.~\ref{fig:la} for several values of $\rhoo$. 
 }
 \label{fig:l3d}
\end{figure} 