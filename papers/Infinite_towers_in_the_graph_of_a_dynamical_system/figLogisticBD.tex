%%%%%%%%%%%%%%%
\begin{figure}
 \centering
 \includegraphics[width=12cm]{lm}
 \caption{{\bf The Bifurcation Diagram of the Logistic Map.} To the left of the Feigenbaum-Myrberg parameter value $\mu_{FM}\simeq3.56994567$, we see the well-known period-doubling cascade. To its right, we see lots of chaos but also many windows, \ie, intervals in parameter space that begin with a periodic attractor which evolves through period-doubling into small intervals of chaos. This picture is created by plotting trajectories. More frequently visited regions are darker. Points on attracting periodic orbits of period less than 26 are indicated by black dots. Notice, in particular, that many of these points are near where $x=0.5$. In colors are highlighted, besides $\mu_{FM}$, the largest period-6 window, the intersection parameter value $\mu\INT=1.5[ 1 + (19-3\sqrt{33})^{1/3} + (19+3\sqrt{33})^{1/3} ]\simeq3.67857$, the largest period-5 window and the largest period-3 window. Notice that many high-density lines intersect at $(\mu\INT,x\INT)$ (see figure). Each of the high-density lines is the image of the $x=0.5$ line under $\ell^n_\mu$ for some $n$. In the bottom right box it is shown a detail of the cascade about $x=0.5$ inside the period-5 window.
 }
 \label{fig:lmbd}
\end{figure} 
%%%%%
