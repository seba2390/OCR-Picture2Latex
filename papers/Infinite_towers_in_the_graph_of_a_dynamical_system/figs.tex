%%%%%%%%%%%%%%%
\newpage
\begin{figure}
 \centering
 \includegraphics[width=12.21cm,height=6cm]{l1}\\
 \includegraphics[width=12cm,height=6cm]{l1x}\\
 \includegraphics[width=12cm,height=6cm]{l1a}\\
 \caption{{\bf Bifurcation Diagram}. These are the projections onto the $(\rho,y)$ (Top panel) and $(\rho,x)$ (Bottom panel) planes of the bifurcation diagram for the Poincar\'e Return map of the Lorenz equations~(\ref{eq:Lor}) using the plane $\Pi$ defined by $z=\rho-1$. A dot is plotted in the $(\rho,y)$ (resp. $(\rho,x)$) plane when a trajectory crosses downward past $\Pi$ through the point $(x,y,\rho-1)$. The regions where there is speckled white and magenta dots is where the attractor is low density. The Lorenz attractor typically has great variations in density, so extremely long trajectories would be needed to reveal such parts of the attractor.
 }
 \label{fig:l1}
\end{figure} 
%%%%%%%%%%%%%%%
\begin{figure}
 \centering
 \includegraphics[width=12cm]{l2}
 \caption{{\bf Bifurcation Diagram}. This is a detail, in the $(\rho,y)$ projection, of the large period-4 window of the bifurcation diagram of the Lorenz system. The window starts at $\rho\simeq142.26$ and ends at $\rho\simeq162.12$.}
 \label{fig:l2}
\end{figure} 
%%%%%%%%%%%%%%%
\begin{figure}
 \centering
 \includegraphics[width=12cm]{l3.png}
 \caption{{\bf Bifurcation Diagram}. This is a detail, in the $(\rho,y)$ projection, of the upper cascade of the main period-4 window of the bifurcation diagram for the Poincar\'e Return map of the Lorenz system.}
 \label{fig:l3}
\end{figure} 
%%%%%%%%%%%%%%%
\begin{figure}
 \centering
 \includegraphics[width=12cm]{l4}
 \caption{{\bf Bifurcation Diagram}. This is a detail, in the $(\rho,y)$ projection, of the middle cascade in the main period-3 window of the upper cascade in the main period-4 window (Fig.~\ref{fig:l3}) of the bifurcation diagram for the Poincar\'e Return map of the Lorenz system (Fig.~\ref{fig:l1}).}
 \label{fig:l4}
\end{figure} 
%%%%%%%%%%%%%%%
\begin{figure}
 \centering
 \includegraphics[width=12cm]{l5}
 \caption{{\bf Bifurcation Diagram}. This picture shows the attractor (in black/gray) and the rest of the chain-recurrent set (in red and blue) in the largest period-3 window of the Lorenz system bifurcation diagram. The window starts at about $\rho=208.520$ and ends at about $\rho=209.453$.}
 \label{fig:l5}
\end{figure} 
%%%%%%%%%%%%%%%
\begin{figure}
 \centering
 \includegraphics[width=12cm]{l6.png}
 \caption{{\bf The Lorenz Poincare return map for $\rho=209.2$, which has a period-3 attractor}. A period-3 attractor is shown with black dots on the blue curve. The blue curve is a global attractor for the $(x,y)$ region shown. The region in white are points that are attracted to another attractor, that is off-screen. The blue curve includes the period-3 attractor and a chain-recurrent Cantor set of saddle points and the unstable manifolds of all of the periodic orbits in the Cantor set. Almost all points in the colored region are in the basin of attraction of the period-3 orbit. Yellow indicates rapid convergence to the period-3 orbit, while red indicates slow convergence. Red points are close to points that are attracted to the Cantor set on the blue line.}
 \label{fig:l6}
\end{figure} 
%%%%%%%%%%%%%%%
\begin{figure}
 \centering
 \includegraphics[width=5cm]{lorenzOrbit-r208-tf5000-dT11.png} \includegraphics[width=5cm]{lorenzOrbit-r209_2-tf5000-dT11.png}
 \caption{{\bf Attractors of the Poincar\'e Map of the Lorenz system.}. This picture shows two attractors of the Lorenz system for $\rho=208$ (left) and $\rho=209$ (right) together with the plane $z=\rho-1$. The plane is translucent so that it is possible to see also the portion of the attractor on the other side (shaded in blue). The intersections of the attractor with the plane $z=\rho-1$ is shown in Fig.~\ref{fig:l7} for several values of $\rho$. }
 \label{fig:l7}
\end{figure} 
%%%%%%%%%%%%%%%
\begin{figure}
 \centering
 \includegraphics[width=11.8cm]{la.png}
 \caption{{\bf Attractors of the Poincar\'e Map of the Lorenz system.} The pictures above show the attractor of the Poincar\'e map of the Lorenz system  in the $(x,y)$ plane for five different values of $\rho$. In all pictures $x$ ranges from $-20$ to $20$ and $y$ ranges from $-100$ to $100$. Each panel is the plot of a single trajectory, Hence low density regions of the attractor may not be represented.
 }
 \label{fig:la}
\end{figure} 
%%%%%%%%%%%%%%%
%%%%%%%%%%%%%%%
\begin{figure}
 \centering
 \includegraphics[width=12cm]{lm}
 \caption{{\bf Bifurcation Diagram}. This is the bifurcation diagram of the Logistic map}
 \label{fig:lmbd}
\end{figure} 
%%%%%
\begin{figure}
 \centering
 \includegraphics[width=12cm]{cr3.png}
  \caption{{\bf \allblue The bifurcations diagram and sample graphs of the logistic map.} 
  The logistic map has a top node, the point 0, that for simplicity is not shown.
  For each value of $\mu$, the attracting set is painted in gray or black, depending on the density of the attractor, repelling periodic orbits in green, and repelling Cantor sets in red. 
  \allred WE NEED TO DISCUSS THE T$_J$ GRAPHS SOMEPLACE.
  }
  \label{fig:full}
\end{figure}

%%%%%%%%%%%%%%%
\begin{figure}
 \centering
 \includegraphics[width=13cm]{crsmall}
 \caption{{\bf Towers of nodes, \ie, chain recurrent sets, in the Logistic map diagram.}
 \allblue
 The black dots on the $\mu$ axis are sample values sampling some of the possible towers near the value where the period three orbit first appears, $\mu = 1+\sqrt 8\simeq3.8284$.
Red nodes denote fractal chaotic saddles. Blue and green nodes are periodic orbits. Black denotes attracting sets which occur at the bottom of the tower. 
In the bifurcation diagram, the chain recurrent sets have the same coloring as their nodes. Notice that we do not include in the pictures of the towers the zero node, which is always on top.}
 \label{fig:p3}
\end{figure} 
%%%%%%%%%%%%%%%

