% CVPR 2024 Paper Template; see https://github.com/cvpr-org/author-kit

\documentclass[10pt,twocolumn,letterpaper]{article}
\usepackage{wrapfig}
\usepackage{amsmath}
\usepackage{stmaryrd} 
\newcommand{\btheta}{\boldsymbol{\theta}}
\newcommand{\bbeta}{\boldsymbol{\beta}}
\newcommand{\btau}{\boldsymbol{\tau}}
\newcommand{\bphi}{\boldsymbol{\phi}}
\newcommand{\bPhi}{\boldsymbol{\Phi}}
\newcommand{\bmu}{\boldsymbol{\mu}}
\newcommand{\bX}{\mathbf{X}}
\newcommand{\bO}{\mathbf{O}}
\newcommand{\bj}{\mathbf{j}}
\newcommand{\bJ}{\mathbf{J}} 
\newcommand{\bI}{\mathbf{I}}
\newcommand{\bS}{\mathbf{S}}
\newcommand{\bV}{\mathbf{V}}
\newcommand{\bP}{\mathbf{P}}
\newcommand{\bv}{\mathbf{v}}
\newcommand{\ba}{\mathbf{a}}
\newcommand{\bp}{\mathbf{p}}
\newcommand{\bZ}{\mathbf{Z}}
\newcommand{\bz}{\mathbf{z}}
\newcommand{\vf}{\mathbf{f}}
\newcommand{\vT}{\mathbf{T}}
\newcommand{\vF}{\mathbf{F}}
\newcommand{\bc}{\mathbf{c}}
\newcommand{\bw}{\mathbf{w}}
\newcommand{\ie}{\textit{i.e., }}
\newcommand{\eg}{\textit{e.g., }}
\newcommand{\etal}{\textit{et al. }}

\newcommand{\mcol}{m_{\text{col}}}
\newcommand{\mdist}{m_{\text{dist}}}
\newcommand{\mtouch}{m_{\text{touch}}}
\newcommand{\ours}{\textit{Ours}}
\newcommand{\vae}{\textit{VAE}}
\newcommand{\vaegan}{\textit{VAEGAN}}
\newcommand{\method}{\textit{MACS}}
\newcommand{\hideablecomment}[1]{#1}
\newcommand{\hidecomments}{%
    \renewcommand{\hideablecomment}[1]{}%
}%
 
\newcommand\blfootnote[1]{%
  \begingroup
  \renewcommand\thefootnote{}\footnote{#1}%
  \addtocounter{footnote}{-1}%
  \endgroup
}
%%%%%%%%%%%%%%%%%%%%%%%
%%%%%%%%%%%%%%% pls define your own tag with an unique color when you write comments 
%%%%%%%%%%%%%
\newcommand{\todo}[1]{\hideablecomment{\textcolor{red}{#1}}}
\newcommand{\SOS}[1]{\hideablecomment{\textcolor{red}{[\textbf{SS:} #1]}}}
\newcommand{\VG}[1]{\hideablecomment{\textcolor{red}{[\textbf{VG:} #1]}}}
\newcommand{\hl}[1]{\hideablecomment{\textcolor{red}{\textbf{#1}}}}
\newcommand{\BD}[1]{\hideablecomment{\textcolor{cyan}{[\textbf{BD:} #1]}}}
\newcommand{\fm}[1]{\hideablecomment{\textcolor{blue}{[\textbf{FM:} #1]}}}
\newcommand{\jb}[1]{\hideablecomment{\textcolor{magenta}{[\textbf{JB:} #1]}}}
\newcommand{\JB}[1]{\hideablecomment{\textcolor{magenta}{#1}}}
\newcommand{\bb}[1]{\hideablecomment{\textcolor{green}{[\textbf{BB:} #1]}}}
\newcommand{\dt}[1]{\hideablecomment{\textcolor{yellow}{[\textbf{DT:} #1]}}}
\newcommand{\tb}[1]{\hideablecomment{\textcolor{red}{[\textbf{TB:} #1]}}}
 
\newcommand{\GT}{GT}
%%%%%%%%% PAPER TYPE  - PLEASE UPDATE FOR FINAL VERSION
 \usepackage{cvpr}              % To produce the CAMERA-READY version
%\usepackage[review]{cvpr}      % To produce the REVIEW version
\usepackage[pagenumbers]{} % To force page numbers, e.g. for an arXiv version 
% Import additional packages in the preamble file, before hyperref
\pagestyle{plain}
% OLD PREAMBLE:

% \usepackage{jsen}
% \usepackage{cite}
% \usepackage{amsmath,amssymb,amsfonts, bbm, mathtools}
% \usepackage{algorithm,algorithmic}
% \usepackage{graphicx}
% \usepackage{textcomp}
% \usepackage{wrapfig}
% \usepackage{xfrac}
% \usepackage{stackengine}
% \usepackage{subfigure}
% \def\delequal{\mathrel{\ensurestackMath{\stackon[1pt]{=}{\scriptstyle\Delta}}}}



% \usepackage{color, soul}
% \newcommand{\hlt}[1]{\hl{#1}}
% \newcommand{\red}[1]{\textcolor{red}{#1}}

% \def\BibTeX{{\rm B\kern-.05em{\sc i\kern-.025em b}\kern-.08em
%     T\kern-.1667em\lower.7ex\hbox{E}\kern-.125emX}}
% \markboth{\journalname, VOL. XX, NO. XX, XXXX 2017}
% {Author \MakeLowercase{\textit{et al.}}: Preparation of Papers for IEEE TRANSACTIONS and JOURNALS (February 2017)}
% \definecolor{abstractbg}{rgb}{0.89804,0.94510,0.83137}
% \setlength{\fboxrule}{0pt}
% \setlength{\fboxsep}{0pt}

% NEW PREAMBLE:


\usepackage{amsmath,amsfonts,amssymb,bbm, amsthm, xfrac}
\usepackage{algorithmic}
\usepackage{algorithm}
\usepackage{array, multirow}
% \usepackage[caption=false,font=normalsize,labelfont=sf,textfont=sf]{subfig}
\usepackage{caption, subcaption}
\usepackage{textcomp}
\usepackage{stfloats}
\usepackage{url}
\usepackage{verbatim}
\usepackage{graphicx}
\usepackage{cite}
\usepackage{caption}
\usepackage{subcaption}
\hyphenation{}

\theoremstyle{plain}
\newtheorem{theorem}{Theorem}

\usepackage{color, soul}
\newcommand{\hlt}[1]{\hl{#1}}
\newcommand{\red}[1]{\textcolor{red}{#1}}


% It is strongly recommended to use hyperref, especially for the review version.
% hyperref with option pagebackref eases the reviewers' job.
% Please disable hyperref *only* if you encounter grave issues, 
% e.g. with the file validation for the camera-ready version.
%
% If you comment hyperref and then uncomment it, you should delete *.aux before re-running LaTeX.
% (Or just hit 'q' on the first LaTeX run, let it finish, and you should be clear).
\definecolor{cvprblue}{rgb}{0.21,0.49,0.74}
\usepackage[pagebackref,breaklinks,colorlinks,citecolor=cvprblue]{hyperref}

%%%%%%%%% PAPER ID  - PLEASE UPDATE
\def\paperID{323} % *** Enter the Paper ID here
\def\confName{3DV\xspace}
\def\confYear{2024\xspace}

%%%%%%%%% TITLE - PLEASE UPDATE
\title{MACS: Mass Conditioned 3D Hand and Object Motion Synthesis}

%%%%%%%%% AUTHORS - PLEASE UPDATE

\author{
Soshi Shimada$^{1,2,*}$  $\;\;\;\;$  
Franziska Mueller$^{3}$   $\;\;\;\;$  
Jan Bednarik$^{3}$   $\;\;\;\;$  
Bardia Doosti$^{3}$   $\;\;\;\;$  
Bernd Bickel$^{3}$    \\
Danhang Tang$^{3}$   $\;$  
Vladislav Golyanik$^{1}$   $\;$  
Jonathan Taylor$^{3}$   $\;$  
Christian Theobalt$^{1,2}$   $\;$  
Thabo Beeler$^{3}$  \\\\ 
$^{1}$MPI for Informatics, SIC$\;\;$
$^{2}$ VIA Research Center $\;\;$ 
$^{3}$ Google  
}
 
 
\begin{document}
\twocolumn[{
\vspace{-1cm}
\maketitle
\vspace{-0.8cm}
\begin{center}
  \centering
  \includegraphics[width=\textwidth]{Figures/teaser2.pdf} %  
  \captionof{figure}{Example visualizations of 3D object manipulation synthesized by our method \method. Conditioning object mass values of $0.2$kg (left) and $5.0$kg (right) are given to the model for the action type "passing from one hand to another". \method\ plausibly reflects the mass value in the synthesized 3D motions.}\label{fig:teaser}
\end{center}
% 
}]

 
\blfootnote{*Work done while at Google.}
\begin{abstract}
%\medskip
%\centering \textcolor{red}{Write the abstract last}
Silicon-compatible short- and mid-wave infrared emitters are highly sought-after for on-chip monolithic integration of electronic and photonic circuits to serve a myriad of applications in sensing and communication. To address this longstanding challenge, GeSn semiconductors have been proposed as versatile building blocks for silicon-integrated optoelectronic devices. In this regard, this work demonstrates light-emitting diodes (LEDs) consisting of a vertical PIN double heterostructure  p-Ge$_{0.94}$Sn$_{0.06}$/i-Ge$_{0.91}$Sn$_{0.09}$/n-Ge$_{0.95}$Sn$_{0.05}$ grown epitaxially on a silicon wafer using germanium interlayer and multiple GeSn buffer layers. The emission from these GeSn LEDs at variable diameters in the 40-120 $\mu$m range is investigated under both DC and AC operation modes. The fabricated LEDs exhibit a room temperature emission in the extended short-wave range centered around 2.5 $\mu$m under an injected current density as low as 45 A/cm$^2$.  By comparing the photoluminescence and electroluminescence signals, it is demonstrated that the LED emission wavelength is not affected by the device fabrication process or heating during the LED operation. Moreover, the measured optical power was found to increase monotonically as the duty cycle increases indicating that the DC operation yields the highest achievable optical power. The LED emission profile and bandwidth are also presented and discussed. 
\end{abstract}     
\section{Introduction}

Many problems in econometrics, statistics, causal inference, and finance involve linear functionals of unknown functions:
\begin{equation}
\theta(g)=\E[m(Z; g)]
\end{equation}
where $Z$ denotes a random vector, and $g: \mcX\to \R$ is a function in some space $ \mcG$. A continuous linear functional that is mean square continuous with respect to $\ell_2$ norm can be written in a more benign and useful manner. Formally, for a given linear functional $\theta(\cdot)$, there exists a function $a_0$ such that for any $g\in \mcG$:\footnote{For simplicity of exposition, throughout the paper we consider scalar-valued functions $g$. All our results naturally extend to vector-valued functions $g$, and estimate a vector valued Riesz representer that satisfies that $\theta(g)=\E[a(X)'g(X)]$.}
\begin{equation}
    \theta(g) = \E[a_0(X)\, g(X)]
\end{equation}
This result is known as the Riesz representation theorem, and the function $a_0$ is the Riesz representer of the linear functional. Evaluation of a linear functional $\theta(g)$ can be achieved by simply taking the inner product between $a_0$ and $g$.

Knowing the Riesz representation of a linear functional is a critical building block in a variety of learning problems. For instance, in semi-parametric models, $g_0$ is an unknown regression function and $\theta(g_0)$ is a causal or structural parameter of interest. The Riesz representer $a_0$ of the functional $\theta(\cdot)$ can be used to debias the plug-in estimator and construct semi-parametrically efficient estimators of the parameter $\theta(g_0)$. In asset pricing applications, the Riesz representer corresponds to the stochastic discount factor, which is of primary interest when pricing financial derivatives.

Irrespective of the downstream application, the goal of this paper is to derive an estimator for the Riesz representer of any linear functional, when given access to $n$ samples of the random vector $Z$ and a target function space $\mcA$ that can well approximate the function $a_0$. We propose and analyze an estimator $\hat{a}$, with small mean-squared-error. Formally, with probability (w.p.) $1-\zeta$:
\begin{equation}
    \|\hat{a}-a_0\|_2 = \sqrt{\E\left[\left(\hat{a}(X) - a_0(X)\right)^2\right]} \leq \epsilon_{n,\zeta}
\end{equation}

We consider estimation of the Riesz representer within some function space $\mcA$ and propose an adversarial estimator based on regularized variants of the following min-max criterion:
\begin{equation}
    \hat{a} = \argmin_{a\in \mcA} \max_{f\in \mcF} \frac{1}{n}\sum_{i=1}^n \left(m(Z_i;f) - a(X_i)\cdot f(X_i) - f(X_i)^2\right)
\end{equation}
We derive oracle inequalities for this estimator as a function of the localized Rademacher complexity of the function space $\mcA$ and the approximation error $\epsilon = \min_{a\in \mcA} \|a-a_0\|_{2}$.

We show that as long as the function class $\mcF$ contains the star-hull of differences of functions in $\mcA$, i.e. $\mcF:= \{r(a-a'): a, a'\in \mcA, r\in [0, 1]\}$, then the estimation rate of the adversarial estimator achieves w.p. $1-\zeta$:
\begin{equation}
    \|\hat{a} - a_0\|_2 = O\left(\epsilon + \delta_n + \sqrt{\frac{\log(1/\zeta)}{n}}\right)
\end{equation}
where $\delta_n$ is the critical radius of the function classes $\mcF$ and $m\circ \mcF=\{Z\to m(Z; f): f\in \mcF\}$. The critical radius of a function class is a widely used quantity in statistical learning theory that allows one to argue fast estimation rates that are nearly optimal. For instance, for parametric function classes, the critical radius is of order $n^{-1/2}$, leading to fast parametric rates (as compared to $n^{-1/4}$ which would be achievable via looser uniform deviation bounds).

Moreover, the critical radius has been analyzed and derived for a variety of function spaces of interest, such as neural networks, high-dimensional linear functions, reproducing kernel Hilbert spaces, and VC-subgraph classes. Thus our general theorem allows us to appeal to these characterizations and provide oracle rates for a family of Riesz representer estimators. Prior work on estimating Riesz representers only considered particular high-dimensional parametric classes and derived specialized estimators for the function space of interest. Our adversarial estimator provides a single approach that tackles generic function spaces in a uniform manner.

We also examine the computational aspect of our estimator. We provide examples of how estimation can be achieved in a computationally efficient manner for several function spaces of interest.

Finally, we show how our estimator can be used in the context of estimating causal or structural parameters in semi-parametric models. Specifically, our mean square rate for the Riesz representer is sufficiently fast to achieve semi-parametric efficiency and asymptotic normality of the causal or structural parameter.

\subsection{Applications: Causal Inference and Asset Pricing}\label{sec:intro_examples}

This learning problem arises in two important domains for economic research: causal inference and asset pricing.

\paragraph{Automated De-biasing of Causal Estimates.} In causal inference, a variety of treatment effects and policy effects can be formulated as functionals--i.e., scalar summaries--of an underlying regression \cite{chernozhukov2016locally}. Formally, the causal parameter $\theta_0=\theta(g_0)=\mathbb{E}[m(Z;g_0)]$ is a functional $\theta(\cdot)$ of the nuisance parameter $g_0(x):=\mathbb{E}[Y|X=x]$. In this paper, we consider a variety of treatment and policy effects including
\begin{enumerate}
    \item Average treatment effect (ATE): $\theta_0=\mathbb{E}[g_0(1,W)-g_0(0,W)]$, where $X=(D,W)$ consists of treatment and covariates.
    \item Average policy effect: $\theta_0=\int g_0(x)d\mu(x)$ where $\mu(x)=F_1(x)-F_0(x)$
    \item Policy effect from transporting covariates: $\theta_0=\mathbb{E}[g_0(t(X))-g_0(X)]$
    \item Cross effect: $\theta_0=\mathbb{E}[Dg_0(0,W)]$, where $X=(D,W)$ consists of treatment and covariates.
    \item Regression decomposition: $\mathbb{E}[Y|D=1]-\mathbb{E}[Y|D=0]=\theta_0^{response}+\theta_0^{composition}$
    where
    \begin{align}
        \theta_0^{response}&=\mathbb{E}[g_0(1,W)|D=1]-\mathbb{E}[g_0(0,W)|D=1] \\
        \theta_0^{composition}&=\mathbb{E}[g_0(0,W)|D=1]-\mathbb{E}[g_0(0,W)|D=0]
    \end{align}
    \item Average treatment on the treated (ATT): $\theta_0=\mathbb{E}[g_0(1,W)|D=1]-\mathbb{E}[g_0(0,W)|D=1]$, where $X=(D,W)$ consists of treatment and covariates.
    \item Local average treatment effect (LATE): $\theta_0=\frac{\mathbb{E}[g_0(1,W)-g_0(0,W)]}{\mathbb{E}[h_0(1,W)-h_0(0,W)]}$, where $X=(V,W)$ consists of instrument and covariates and $h_0(x):=\mathbb{E}[D|X=x]$ is a second regression.
\end{enumerate}
More generally, our results extend to parameters defined implicitly by $0=\mathbb{E}[m(Z;g_0;\theta_0)]$, such as partially linear regression and partially linear instrumental variable regression.

    If the regression $g_0$ is learned by a regularized estimator $\hat{g}$, then estimation of the causal parameter $\theta_0$  by a plug-in estimator $\mathbb{E}_n[m(Z;\hat{g})]$ is badly biased. The solution is to use a de-biased formulation of the causal parameter instead: $\theta_0=\mathbb{E}[m(Z;g_0)+a_0(X)\{Y-g_0(X)\}]$. Observe that $a_0$ arises in the bias correction term. We re-visit this example in Section~\ref{sec:debiasing}.

%

\paragraph{Fundamental Asset Pricing Equation.} In asset pricing, a variety of financial models deliver the same fundamental asset pricing equation. This equation is of both theoretical and practical interest. Theoretically, it elucidates why asset prices or returns are what they are. Practically, it can be used to identify trading opportunities when assets are mis-priced. The asset pricing equation follows from two weak assumptions: free portfolio formation, and the law of one price.  In Appendix~\ref{sec:finance}, we review the derivation for a general audience.\footnote{The same asset pricing equation can be derived from either a model of complete markets for contingent claims, or a model of investor utility maximization. Free portfolio formation is a weaker assumption on market structure than the existence of complete markets for contingent claims. The law of one price is a weaker assumption on preference structure than investor utility maximization. We present these additional derivations in Appendix~\ref{sec:finance}.}

Formally, the fundamental asset pricing equation is $p_{t,i}=\mathbb{E}_t[m_{t+1}x_{t+1,i}]$ where $p_{t,i}$ is the price of asset $i$ at time $t$, $x_{t+1,i}$ is payoff of asset $i$ at time $t+1$, and $m_{t+1}$ is the market-wide stochastic discount factor (SDF) at time $t+1$.\footnote{The SDF has many additional names: marginal rate of substitution, state price density, and pricing kernel. Each name corresponds to a different derivation of the asset pricing equation, starting from different first principles.} The expectation is conditional on information $(I_t,I_{t,i})$ known at time $t$:  $I_t$ are macroeconomic conditioning variables that are not asset specific, e.g. inflation rates and market return; $I_{t,i}$ are asset-specific characteristics, e.g. the size or book-to-market ratio of firm $i$ at time $t$. The asset pricing equation encompasses stocks, bonds, and options. We clarify its many instantiations below, where $d_{t+1}$ is dividend, $C$ is the call price, $S_T$ is the stock price at expiration, $K$ is the strike price. 

\begin{table}[H]
       \centering
       \begin{tabular}{|c||c|c|}
        \hline 
            Asset & Price $p_t$ & Payoff $x_{t+1}$ \\
             \hline 
            \hline
            Stock &$p_t$& $p_{t+1}+d_{t+1}$ \\
              Bond &$p_t$&$1$\\
             Option &$C$&$\max\{S_T-K,0\}$ \\
             \hline 
            Return & $1$& $R_{t+1}$ \\
            Excess return &0&$R^e_{t+1}$ \\
            \hline 
       \end{tabular}
       \caption{Generality of asset pricing equation}
       \label{tab:my_label}
   \end{table}
 
 The fundamental asset pricing equation can also be parametrized in terms of returns. If an investor pays one dollar for an asset $i$ today, the gross rate of return $R_{t+1,i}$ is how many dollars the investor receives tomorrow; formally, the price is $p_{t,i}=1$ and the payoff is $x_{t+1,i}=R_{t+1,i}$ by definition. Next consider what happens when an investor borrows a dollar today at the interest rate $R_{t+1}^f$ and buys an asset $i$ that gives the gross rate of return $R_{t+1,i}$ tomorrow. From the perspective of the investor, who paid nothing out-of-pocket, the price is $p_{t,i}=0$ while the payoff is the excess rate of return $R_{t+1,i}^e:=R_{t+1,i}-R_{t+1}^f$, leading to the asset pricing equation: $0=\mathbb{E}_t[m_{t+1}R^e_{t+1,i}]$.
 
 
 Following \cite{chen2019deep}, we focus on the latter excess return parametrization of the asset pricing equation. Taking expectations yields the unconditional moment restriction
$$
0=\mathbb{E}[m_{t+1}R^e_{t+1,i}z(I_t,I_{t,i})]=\mathbb{E}[\mathbb{E}[m_{t+1}|R^e_{t+1,i},I_t,I_{t,i}]R^e_{t+1,i}z(I_t,I_{t,i})],\quad \forall z(\cdot)
$$
Our framework nests this final expression. Specifically,
$$
\theta(g)=0,\quad g(R^e_{t+1,i},I_t,I_{t,i})=R^e_{t+1,i}z(I_t,I_{t,i}),\quad a_0(R^e_{t+1,i},I_t,I_{t,i})=\mathbb{E}[m_{t+1}|R^e_{t+1,i},I_t,I_{t,i}]
$$
By estimating $a_0$, which is the projection of the SDF onto excess returns and other available information, one can pin down the price of any hypothetical asset. 

%
%
%
%

\subsection{Related Work}

\textbf{Classical Semi-parametric Statistics.} Classical semi-parametric statistical theory studies the asymptotic properties of statistical quantities that are functionals of a density or a regression over a low-dimensional domain \cite{levit1976efficiency,hasminskii1979nonparametric,ibragimov1981statistical,pfanzagl1982lecture,klaassen1987consistent,robinson1988root,van1991differentiable,bickel1993efficient,newey1994asymptotic,robins1995semiparametric,vaart,bickel1988estimating,newey1998undersmoothing,ai2003efficient,newey2004twicing,ai2007estimation,tsiatis2007semiparametric,kosorok2007introduction,ai2012semiparametric}. Any continuous linear functional has a Riesz representer. In this classical theory, the Riesz representer appears in the influence function and therefore in the asymptotic variance of semi-parametric estimators \cite{newey1994asymptotic}. We depart from classical theory by considering the high-dimensional setting.

\textbf{De-biased Machine Learning and Targeted Maximum Likelihood.} Because the Riesz representer appears in the asymptotic variance of semi-parametric estimators, it can be incorporated into estimation to ensure semi-parametric efficiency. In practice, this can be achieved by introducing a de-biasing term into the estimating equation \cite{hasminskii1979nonparametric,bickel1988estimating,zhang2014confidence,belloni2011inference,belloni2014inference,belloni2014uniform,belloni2014pivotal,javanmard2014confidence,javanmard2014hypothesis,javanmard2018debiasing,van2014asymptotically,ning2017general,chernozhukov2015valid,neykov2018unified,ren2015asymptotic,jankova2015confidence,jankova2016confidence,jankova2018semiparametric,bradic2017uniform,zhu2017breaking,zhu2018linear}. In doubly robust estimating equations for regression functionals, the de-biasing term is the product between the Riesz representer and the regression residual \cite{robins1995analysis,robins1995semiparametric,van2006targeted,van2011targeted,luedtke2016statistical,toth2016tmle}. The more general principle at play is Neyman orthogonality: the learning problem for the functional of interest becomes orthogonal to the learning problems for both the regression and the Riesz representer \cite{neyman1959,neyman1979c,vaart,robins2008higher,zheng2010asymptotic,belloni2014uniform,belloni2014pivotal,chernozhukov2016locally,belloni2017program,chernozhukov2018double,foster2019orthogonal}.

De-biased machine learning and targeted maximum likelihood combine the algorithmic insight of doubly-robust moment functions with the algorithmic insight of sample splitting \cite{bickel1982adaptive,schick1986asymptotically,klaassen1987consistent,vaart,robins2008higher}.  In doing so, these frameworks facilitate a general analysis of residuals such that the target functional is $\sqrt{n}$-consistent under minimal assumptions on the estimators used for the regression and Riesz representer \cite{scharfstein1999adjusting,rubin2005general,rubin2006extending,van2006targeted,zheng2010asymptotic,van2011targeted,diaz2013targeted,van2014targeted,kennedy2017nonparametric,kennedy2020optimal}. In particular, any machine learning estimators are permitted that satisfy $\sqrt{n}\|\hat{g}-g_0\|_2\cdot\|\hat{a}-a_0\|_2\rightarrow 0$ \cite{chernozhukov2018double,chernozhukov2016locally}.

The Riesz representer may be a difficult object to estimate. Even for simple regression functionals such as policy effects, its closed form involves ratios of densities. In restricted models, where the regression is known to belong to a certain function class, there is the further difficulty of projecting the Riesz representer accordingly. A recent literature explores the possibility of directly estimating the Riesz representer, without estimating its components or even knowing its functional form \cite{robins2007comment,newey2018cross,athey2018approximate,chernozhukov2018global,chernozhukov2018learning,hirshberg2018debiased,hirshberg2019augmented,singh2019biased,rothenhausler2019incremental}. A crucial insight, on which we build, is that the Riesz representer is directly identified from data. 

\cite{hirshberg2019augmented} observe that to debias an average moment, it is sufficient to estimate an empirical analogue of the Riesz representer that approximately satisfies the Riesz representer moment equation on the $n$ samples. They propose a parametric min-max criterion to estimate $n$ parameters corresponding to the $n$ evaluations of the empirical Riesz representer. Unlike \cite{hirshberg2019augmented}, we provide a guarantee on learning the true Riesz representer, we approximate the Riesz representer within non-parametric function spaces, and our result therefore has broader application beyond causal inference. Importantly, \cite{hirshberg2019augmented} require that the same sample used to estimate the $n$ parameters is used in final stage estimation of the causal parameter. As such, the analysis requires that the regression function $g$ lies in a Donsker class--a restriction that precludes many machine learning estimators. By contrast, our adversarial estimator provides fast estimation rates with respect to the true Reisz representer and hence can be used in combination with cross-fitting and sample splitting to eliminate the Donsker assumption.


\textbf{Adversarial Estimation.} Riesz representation theorem can be viewed as a continuum of unconditional moment restrictions. The non-parametric instrumental variable problem, based on a conditional moment restriction, also implies a continuum of unconditional moment restrictions \cite{newey2003instrumental,hall2005nonparametric,blundell2007semi,chen2009efficient,darolles2011nonparametric,chen2012estimation,chen2015sieve,chen2018optimal}. A central insight of this work is that the min-max approach for conditional moment models may be adapted to the problem of learning the Riesz representer. In a min-max approach, the continuum of unconditional moment restrictions is enforced adversarially over a set of test functions \cite{goodfellow2014generative,arjovsky2017wasserstein,dikkala2020minimax}. 

The fundamental advantage of the min-max approach is its unified analysis over arbitrary function classes. In particular, via local Rademacher analysis, one can derive an abstract bound that encompasses sparse linear models, neural networks, and RKHS methods \cite{koltchinskii2000rademacher,bartlett2005local}. As such, the min-max approach is actually a family of algorithms adaptive to a variety of data settings with a unified guarantee \cite{negahban2012,lecue2017regularization,Lecue2018}. 

\textbf{Machine Learning Approaches to Causal Inference and Asset Pricing.} By pursuing a min-max approach, our work relates to previous work that incorporates a variety of machine learning methods into causal inference. Much work on de-biased machine learning focuses on sparse and approximately sparse models \cite{chernozhukov2018global,chernozhukov2018learning,chernozhukov2018plug}. A neural network estimator with mean square rate has been successfully used to learn the nuisance regression in semiparametric estimation \cite{chen1999improved,farrell2018deep} and to learn the structural function in nonparametric instrumental variable regression \cite{deepiv,bennett2019deep,dikkala2020minimax}. A more recent literature incorporates RKHS methods into causal inference due to their convenient closed form solutions and strong performance on smooth designs \cite{nie2017quasi,singh2019kernel,muandet2019dual,singh2020kernel,muandet2020kernel}.

Finally, our works provides a theoretical foundation for a growing literature that incorporates machine learning into asset pricing. We follow the asset pricing literature in framing the problem of learning a stochastic discount factor as the problem of learning a Riesz representer \cite{hansen1997assessing}. Specifically, we propose a deep min-max approach based on free portfolio formation and the law of one price \cite{bansal1993no,chen2019deep}. This approach differs from deep learning approaches that predict asset prices via nonparametric regression \cite{messmer2017deep,feng2018deep,gu2020autoencoder,bianchi2020bond}. Unlike previous work, we prove mean square rates for the stochastic discount factor, and we prove $\sqrt{n}$-consistency and semiparametric efficiency for expected asset prices.
\section{Related Work}

Online meta-learning brings together ideas from online learning, meta learning, and continual learning, with the aim of adapting quickly to each new task while \emph{simultaneously} learning how to adapt even more quickly in the future. We discuss these three sets of approaches next.


\noindent \textbf{Meta Learning:} Meta learning methods try to learn the high-level context of the data, to behave well on new tasks (\emph{Learning to learn}). These methods involve learning a metric space~\citep{koch2015siamese, vinyals2016matching, snell2017prototypical, yang2017learning}, gradient based updates~\citep{finn2017model, li2017meta, park2019meta, nichol2018first, nichol2018reptile}, or some specific architecture designs~\citep{santoro2016meta, munkhdalai2017meta, ravi2016optimization}.
In this work, we are mainly interested in gradient based meta learning methods for online learning. MAML~\citep{finn2017model} and its variants~\citep{nichol2018first, nichol2018reptile, li2017meta, park2019meta, antoniou2018train} first meta train the models in such a way that the meta parameters are close to the optimal task specific parameters (good initialization). This way, adaptation becomes faster when fine tuning from the meta parameters. However, directly adapting this approach into an online setting will require more relaxation on online learning assumptions, such as access to task boundaries and resetting back and froth from meta parameters. Our method does not require knowledge of task boundaries.




\noindent \textbf{Online Learning:} Online learning methods update their models based on the stream of data sequentially. There are various works on online learning using linear models~\citep{cesa2006prediction}, non-linear models with kernels~\citep{kivinen2004online, jin2010online}, and deep neural networks~\citep{zhou2012online}. Online learning algorithms often simply update the model on the new data, and do not consider the past knowledge of the previously seen data to do this online update more efficiently. However, the online meta learning framework, allow us to keep track of previously seen data and with the ``meta'' knowledge we can update the online weights to the new data more faster and efficiently.
\noindent \textbf{Continual Learning:} 
A number of prior works on continual learning have addressed catastrophic forgetting~\citep{mccloskey1989catastrophic,li2017learning,ratcliff1990connectionist, rajasegaran2019random, rajasegaran2020itaml}, removing the need to store all prior data during training. Our method does not address catastrophic forgetting for the meta-training phase, because we must still store all data so as to ``replay'' it for meta-training, though it may be possible to discard or sub-sample old data (which we leave to future work). However, our adaptation process is fully online. A number of works perform meta-learning for better continual learning, i.e. learning good continual learning strategies~\citep{al2017continuous,nagabandi2018deep,javed2019meta,harrison2019continuous,he2019task,beaulieu2020learning}. However, these prior methods still perform batch-mode meta-training, while our method also performs the meta-training itself incrementally online, without task boundaries.


The closest work to ours is the follow the meta-leader (FTML) method~\citep{finn19a} and other online meta-learning methods~\citep{yao2020online}. FTML is a varaint of MAML that finetunes to each new task in turn, resetting to the meta-trained parameters between every task. While this effectively accelerates acquisition of new tasks, it requires ground truth knowledge of task boundaries and, as we show in our experiments, our approach outperforms FTML \emph{even when FTML has access to task boundaries and our method does not}. Note that the memory requirements for such methods increase with the number of adaptation gradient steps, and this limitation is also shared by our approach. Online-within-online meta-learning~\cite{denevi2019online} also aims to accelerate online updates by leveraging prior tasks, but still requires knowledge of task boundaries. MOCA~\cite{harrison2020continuous} instead aims to \emph{infer} the task boundaries. In contrast, our method does not even attempt to find the task boundaries, but directly adapts without them. A number of related works also address continual learning via meta-learning, but with the aim of minimizing catastrophic forgetting~\cite{gupta2020maml, caccia2020online}. Our aim is not to address catastrophic forgetting. Our method also meta-trains from small datasets for thousands of tasks, whereas prior continual learning approaches typically focus on settings with fewer larger tasks (e.g., 10-100 tasks).
 
\section{Method}
Fig.~\ref{fig:framework} presents the illustration of the proposed \frameworkName.
In this section,  
we start by providing the problem definition of online CIL. Then, we describe the definition of the online prototype, the proposed online prototype equilibrium, and the proposed adaptive prototypical feedback. Finally, we propose an online prototype learning framework.

\subsection{Problem Definition}
Formally, online CIL considers a continuous sequence of tasks from a single-pass data stream $\mathfrak{D}=\left\{\mathcal{D}_1, \ldots, \mathcal{D}_T \right\} $, where $\mathcal{D}_t = \left\{ x_{i}, y_{i} \right\} ^{N_t}_{i=1} $ is the dataset of task $t$, and $T$ is the total number of tasks. Dataset $\mathcal{D}_t$ contains $N_t$ labeled samples, $y_{i}$ is the class label of sample $x_{i}$ and $y_{i} \in \mathcal{C}_t$, where $\mathcal{C}_t$ is the class set of task $t$ and the class sets of different tasks are disjoint. 
For replay-based methods, a memory bank is used to store a small subset of seen data, and we also maintain a memory bank $\mathcal{M}$ in our method.
At each time step of task $t$, the model receives a mini-batch data $X \cup X^\mathrm{b}$ for training, where $X$ and $X^\mathrm{b}$ are drawn from the i.i.d distribution $\mathcal{D}_t$ and the memory bank $\mathcal{M}$, respectively. 
Moreover, we adopt the single-head evaluation setup~\cite{EWC}, where a unified classifier must choose labels from all seen classes at inference due to unavailable task identifiers. 
The goal of online CIL is to train a unified model on data seen only once while predicting well on both new and old classes.

\subsection{Online Prototype Definition}
Prior to introducing the online prototypes, we first present the network architecture of our \frameworkName. Suppose that the model consists of three components: an encoder network $f$, a projection head $g$, and a classifier $\varphi$. Each sample $x$ in incoming data $X$ (a mini-batch data from new classes) is mapped to a projected vectorial embedding (representation) $\mathbf{z}$ by encoder $f$ and projector $g$:
\begin{align}
\label{eq:cal_z}
    \mathbf{z} = g(f(\operatorname{aug}(x);\theta_f);\theta_g),
\end{align}
where $\operatorname{aug}$ represents the data augmentation operation, $\theta_f$ and $\theta_g$ represent the parameters of $f$ and $g$, respectively, and $\mathbf{z}$ is $\ell_2$-normalized. 
Similar to Eq.~\eqref{eq:cal_z}, we use $\mathbf{z}^\mathrm{b}$ to denote the representation of replay data $X^\mathrm{b}$ (a mini-batch data from seen classes in the memory bank). 

At each time step of task $t$, the online prototype of each class is defined as the mean representation in a mini-batch:
\begin{align}
\label{eq:cal_p}
    \mathbf{p}_i = \frac{1}{n_i}\sum\nolimits_j\mathbf{z}_j\cdot \mathbbm{1}\{y_j = i\},
\end{align}
where $n_i$ is the number of samples for class $i$ in a mini-batch, and $\mathbbm{1}$ is the indicator function. 
We can get a set of $K$ online prototypes  in $X$, $\mathcal{P} = \left\{ \mathbf{p}_{i} \right\} ^{K}_{i=1}$, and a set of $K^\mathrm{b}$ online prototypes in $X^\mathrm{b}$, $\mathcal{P}^\mathrm{b} = \left\{ \mathbf{p}_i^\mathrm{b} \right\} ^{K^\mathrm{b}}_{i=1}$.
Note that $K = |\mathcal{P}| \leq |\mathcal{C}_t|$ and $K^\mathrm{b} = |\mathcal{P}^\mathrm{b}| \leq \sum_{i=1}^{t}|\mathcal{C}_i| $, where $|\cdot|$ denotes the cardinal number.



\subsection{Online Prototype Equilibrium}
The introduced online prototypes can provide representative features and avoid class-unrelated information.  
These characteristics are exactly the key to counteracting shortcut learning in online CL.
Besides, maintaining the discrimination among seen classes is also essential to mitigate catastrophic forgetting.
Based on these, we attempt to learn representative features of each class by pulling online prototypes $\mathcal{P}$ and their augmented views $\widehat{\mathcal{P}}$ closer in the embedding space, and learn discriminative features between classes by pushing online prototypes of different classes away, formally defined as a contrastive loss:
\begin{align}
\label{eq:proto_infoNCE}
    \ell(\mathcal{P},\widehat{\mathcal{P}})\!=\!
    % \frac{-1}{K}
    \frac{-1}{|\mathcal{P}|}\sum_{i=1}^{|\mathcal{P}|}\!\log\! 
    \tfrac
    {\exp \big(\tfrac{{\mathbf{p}_i^\mathrm{T} \widehat{\mathbf{p}}_i}}{\tau}\big)}
    {
    \sum\limits_{j} \exp \big(\tfrac{{\mathbf{p}_i^\mathrm{T} \widehat{\mathbf{p}}_j}}{\tau}\big)
    +\!
    \sum\limits_{\substack{j \neq i}} \exp \big(\tfrac{{\mathbf{p}_i^\mathrm{T} \mathbf{p}_j}}{\tau}\big) 
    },
\end{align}
where 
$\tau$ is the temperature hyper-parameter, 
$\mathcal{P}$ and $\widehat{\mathcal{P}}$ are $\ell_2$-normalized. To compute the contrastive loss across all positive pairs in both $(\mathcal{P}, \widehat{\mathcal{P}})$ and $(\widehat{\mathcal{P}}, \mathcal{P})$, we define $\mathcal{L}_{\mathrm{pro}}$ as the final contrastive loss over online prototypes:
\begin{align}
    \mathcal{L}_{\mathrm{pro}}(\mathcal{P},\widehat{\mathcal{P}}) = 
    \frac{1}{2}
    \left[\ell(\mathcal{P}, \widehat{\mathcal{P}}) + \ell(\widehat{\mathcal{P}}, \mathcal{P})\right].
\end{align}



Considering the learning of new classes and the consolidation of learned knowledge simultaneously in online CL, we propose Online Prototype Equilibrium (\methodname) to 
learn representative and discriminative features on both new and seen classes by employing $\mathcal{L}_{\mathrm{pro}}$:
\begin{equation}
    \begin{aligned}
    \mathcal{L}_{\mathrm{\methodname}}
    &=
    \mathcal{L}^{\mathrm{new}}_{\mathrm{pro}}(\mathcal{P},\widehat{\mathcal{P}}) + \mathcal{L}^{\mathrm{seen}}_{\mathrm{pro}}(\mathcal{P}^\mathrm{b},\widehat{\mathcal{P}}^\mathrm{b}),
    \end{aligned}
\end{equation}
where
$\mathcal{L}^{\mathrm{new}}_{\mathrm{pro}}$ focuses on learning knowledge from \emph{new} classes, and $\mathcal{L}^{\mathrm{seen}}_{\mathrm{pro}}$ is dedicated to preserving learned knowledge of all \emph{seen} classes.
\textit{This process is similar to a zero-sum game, 
and \methodname aims to achieve the equilibrium to play a win-win game.}
Concretely,
as the model learns, the knowledge of new classes is gained and added to the prototypes over the memory bank $\mathcal{M}$, causing $\mathcal{L}^{\mathrm{seen}}_{\mathrm{pro}}$ gradually changes to the equilibrium that separates all seen classes well, including new ones. 
This variation is crucial to mitigate forgetting and is consistent with the goal of CIL.



\subsection{Adaptive Prototypical Feedback} 
Although \methodname can bring an overall equilibrium, it tends to treat each class \emph{equally}. 
In fact, the degree of confusion varies among classes, 
and the model should focus purposefully on confused classes to consolidate learned knowledge. 
To this end, we propose Adaptive Prototypical Feedback (\dataaugname) with the feedback of online prototypes to sense the classes that are prone to be misclassified and then enhance their decision boundaries.
 
For each class pair in the memory bank $\mathcal{M}$, \dataaugname calculates the distances between online prototypes of all seen classes from the previous time step, showing the class confusion status by these distances. The closer the two prototypes are, the easier to be misclassified. 
Based on this analysis, 
our idea is to enhance the boundaries for those classes. Therefore, we convert the prototype distance matrix to a probability distribution $P$ over the classes via a symmetric Gaussian kernel, defined as follows:
\begin{align}
\label{eq:cal_d}
    P_{i, j} \propto \exp (-\| \mathbf{p}_i^\mathrm{b} - \mathbf{p}_j^\mathrm{b} \|_2^2),
\end{align}
where $i,j \in \{ 1, \ldots, |\mathcal{P}^\mathrm{b}| \}$ and $i \neq j$. 
Then, 
all probabilities are normalized to a probability mass function that sums to one.
\dataaugname returns probabilities to $\mathcal{M}$ for guiding the next sampling process and enhancing decision boundaries of easily misclassified classes. 


Our adaptive prototypical feedback is implemented as a sampling-based mixup. Specifically, 
\dataaugname adaptively selects more samples from easily misclassified classes in $\mathcal{M}$ for mixup~\cite{Mixup} according to the probability distribution $P$. 
Considering not over-penalizing the equilibrium of current online prototypes, we introduce a two-stage sampling strategy for replay data $X^\mathrm{b}$ of size $m$. 
First, we select $n_{\mathrm{\dataaugname}}$ samples  
with $P$, and a larger $P_{a,b}$ means more sampling from classes $a$ and $b$. Here, $n_{\mathrm{\dataaugname}} = \alpha \cdot m$, and $\alpha$ is the ratio of \dataaugname.
Second, the remaining $m-n_{\mathrm{\dataaugname}}$ samples are uniformly randomly selected from the entire memory bank to avoid the model only focusing on easily misclassified classes and disrupting the established equilibrium. 




\subsection{Overall Framework of \frameworkName}
The overall structure of \frameworkName is shown in Fig.~\ref{fig:framework}. \frameworkName comprises two key components based on proposed online prototypes: Online Prototype Equilibrium (\methodname) and Adaptive Prototypical Feedback (\dataaugname). 
With the two components, 
the model can learn representative features against shortcut learning, and 
all seen classes maintain discriminative when learning new classes. 
However, classes may not be compact, because the online prototypes cannot cover full instance-level information.
To further achieve intra-class compactness, 
we employ supervised contrastive learning~\cite{SupCL} to learn instance-wise representations:
\begin{equation}
\begin{aligned}
    \mathcal{L}_{\mathrm{INS}}
    &=
    \sum_{i=1}^{2N} \frac{-1}{\left|I_i\right|} \sum_{j \in I_i} \log \frac{\exp \left(\mathrm{sim}(\mathbf{z}_i, \mathbf{z}_j) / \tau^{\prime}\right)}{\sum\limits_{k \neq i} \exp \left(\mathrm{sim}(\mathbf{z}_i, \mathbf{z}_k) / \tau^{\prime}\right)}
    \\
    &+
    \sum_{i=1}^{2m} \frac{-1}{\left|I_i^{\mathrm{b}}\right|} \sum_{j \in I_i^{\mathrm{b}}} \log \frac{\exp (\mathrm{sim}(\mathbf{z}_i^{\mathrm{b}}, \mathbf{z}_j^{\mathrm{b}}) / \tau^{\prime})}{\sum\limits_{k \neq i} \exp \left(\mathrm{sim}(\mathbf{z}_i^{\mathrm{b}}, \mathbf{z}_k^{\mathrm{b}}) / \tau^{\prime}\right)},
\end{aligned}
\end{equation}
where $I_i=\left\{j \in\{1, \ldots, 2 N\} \mid j \neq i, y_j=y_i\right\}$ and $I_i^\mathrm{{b}}=\left\{j \in\{1, \ldots, 2m\} \mid j \neq i, y_j^\mathrm{b}=y_i^\mathrm{b}\right\}$ are the set of positive samples indexes to $\mathbf{z}_i$ and $\mathbf{z}_i^\mathrm{{b}}$, respectively. $y_i^\mathrm{b}$ is the class label of input $x_i^\mathrm{b}$ from $X^\mathrm{b}$. $N$ is the batch size of $X$. $\tau^{\prime}$ is the temperature hyperparameter.
The similarity function $\mathrm{sim}$ is computed in the same way as Eq.~(9) in OCM~\cite{OCM}.

Thus, the total loss of our \frameworkName framework is given as:
\begin{align}
    \mathcal{L}_{\mathrm{\frameworkName}}=\mathcal{L}_{\mathrm{\methodname}} + \mathcal{L}_{\mathrm{INS}} + \mathcal{L}_{\mathrm{CE}},
\end{align}
where $\mathcal{L}_{\mathrm{CE}} = \mathrm{CE}(y^\mathrm{b}, \varphi(f(\operatorname{aug}(x^\mathrm{b}))))$ is the cross-entropy loss; see Appendix~\ref{appendix:algorithm} for detailed training algorithms.

Following other replay-based methods~\cite{ER, SCR, OCM}, we update the memory bank in each time step by uniformly randomly selecting samples from $X$ to push into $\mathcal{M}$ and, if $\mathcal{M}$ is full, pulling an equal number of samples out of $\mathcal{M}$.


\section{Experiments}
\subsection{Datasets and Metrics}

%dialogue task分四种, intent prediction, slot-filling, semantic parsing, and dialogue state tracking. 然后我们完成的数据集任务是relation extraction, emotional recognition, speech act classification,不太确定是不是都能解决

\textbf{DialogRE} \citep{yu-etal-2020-dialogue} is a relation extraction task based on 1,788 dialogues from the Friends transcript. Each pair of arguments can be classified as one of 36 possible relation types. For each of the 10,168 human-annotated entity pairs, the trigger words are also provided. % such as neighbours or siblings. (subject, object, relation type) triplets

\textbf{EmoryNLP} \citep{zahiri:18a} is an emotion detection task based on 12,606 utterances from the Friends transcript. Each utterance can be classified as one of seven emotions, e.g., joyful, scared. 
%The label is human-annotated based on the dialogue context. 

\textbf{DailyDialog} \citep{DailyDialog} is a dialogue database containing 13,118 simple English dialogues. Each utterance can be assigned an emotion label from seven categories (anger, surprise, etc.). 
%targets both emotion detection and act classification. DailyDialog The label is human-annotated based on the dialogue context.  (Inform, Questions, Directives, Commissive) 

\textbf{MELD} \citep{poria-etal-2019-meld} is an emotion detection task based on 13,000 sentences from the Friends transcript. Each utterance can be classified as one of eight emotions, such as sad, disgust. % or neutral. 

%The label is human-annotated based on the multimodal cues of the dialogue, including visual, audio and textual information. 
\begin{table}[t]
  % \setlength{\tabcolsep}{3.5pt}
% \footnotesize
% \scriptsize
  \centering
  \vspace{-0.35cm}
  \begin{tabular}{lllllll}
    \toprule
  \multirow{2}{*}{\textbf{Method}} &\multirow{2}{*}{\textbf{MELD}} & \multirow{2}{*}{\textbf{ENLP}} & \multirow{2}{*}{\textbf{DDialog}} & \multirow{2}{*}{\textbf{MRDA}} & \multicolumn{2}{c}{\textbf{DialogRE-Test}}      \\
    % \textbf{Method} &\textbf{MELD} & \textbf{ENLP} & \textbf{DDialog} &\textbf{MRDA} & \multicolumn{2}{c}{\textbf{DialogRE}}      \\
    %\cmidrule(r){2-5}
    \cmidrule(l){6-7}   
    % \cmidrule(l){2-2}  \cmidrule(l){3-3} \cmidrule(l){4-4} \cmidrule(l){5-5}  \cmidrule(l){6-7}  
      &  & & & & $F1$  & $F1_c$  \\
    \midrule
    % BERT &61.50	& 34.17	& 54.09& 91.0 & 58.5	& 53.2\\
    % +HiDialog & + 1.78	& +0.63 & +5.55 & +0.3 & + &\textbf{59.8}\\
    PHT  &61.90&-&	60.14&	\textbf{92.4}&- &- \\
    DialogXL  & 62.41 & 34.73 & 54.93 & -& - & -  \\
    % \midrule
    RoBERTa$_s$& 64.19&38.03	&61.65	&91.3	&71.3 & 63.7\\
+Intra-turn &\textbf{65.64}\textsubscript{\textcolor{green}{+1.45}}& \textbf{38.13}\textsubscript{\textcolor{green}{+0.1}} &\textbf{61.83}\textsubscript{\textcolor{green}{+0.28}}&91.5\textsubscript{\textcolor{green}{+0.2}} & \textbf{74.4}\textsubscript{\textcolor{green}{+3.1}}&\textbf{66.6}\textsubscript{\textcolor{green}{+2.9}}\\
    \bottomrule
  \end{tabular}
  % \vspace{-0.2cm}
  % \caption{All methods performance on 5 multi-turn dialogue-based understanding datasets: MELD, EmoryNLP (Weighted-F1), DailyDialog (Micro-F1), MRDA (Top-1 Acc.), DialogRE (F1 and F1$_c$), averaged over five runs. Performance gains over the RoBERTa$_s$ are highlighted in green.}
    \caption{All methods performance on 5 multi-turn dialogue-based understanding datasets: MELD, EmoryNLP, DailyDialog, MRDA, DialogRE, averaged over five runs. Performance gains over the RoBERTa$_s$ are highlighted in green.}
  \label{tab:exp-mtr}
  \vspace{-0.5cm}
\end{table}


\begin{minipage}[]{0.48\linewidth}
\footnotesize
\setlength{\tabcolsep}{7pt}
\begin{center}
\begin{tabular}{l c c} 
 \toprule
  {\textbf{Method}} & \textbf{F1} & \textbf{F1$_c$} \\
 \midrule
HiDialog                      & 77.1        & 68.2       \\ 
w/o attention mask & 76.5 (-0.6) & 67.9 (-0.3) \\
w/o special tokens & 75.6 (-1.5) & 67.4 (-0.8) \\
only intra-turn     & 74.4 (-2.7) & 66.6 (-1.6) \\
\bottomrule
\end{tabular}
\end{center}
\captionof{table}{Ablation Study on HiDialog components on DialogRE to evaluate the individual effect of turn-level attention, turn-level special tokens, and graph module. } %\textit{Ablation study. Turn-level} is omitted for brevity.}
\label{tab:ablation_structure}
\end{minipage}
\hspace*{0.1cm}
\begin{minipage}[]{0.48\linewidth}
\footnotesize
\setlength{\tabcolsep}{7pt}
\begin{center}
% \vspace{-0.5cm}
\begin{tabular}{lccc} 
\toprule
\textbf{Method} & \textbf{I} &  \textbf{II} &  \textbf{III}\\
\midrule
BERT & 42.5  & 60.7 & 65.6 \\
% BERT$_s$ & 46.5 & 61.5 & 69.4 \\
GDPNet  & 47.4  &59.8  &68.1 \\
RoBERTa$_s$ & 57.4 & 69.3 & 79.6 \\
TUCORE-GCN  &62.3  &\textbf{71.3}  &79.9 \\
\midrule
HiDialog & \textbf{76.6}  & 70.5	& \textbf{80.9}  \\
w/o graph module &65.5 & 69.9 & 79.4\\
\bottomrule
\end{tabular}
\end{center}
% \vspace{-0.3cm}
\captionof{table}{All methods performance on DialogRE. We break down the performance into three groups (I) asymmetric inverse relations, (II) symmetric inverse relations, and (III) others.}
\label{tab:symmetric}
\vspace{0.5cm}
\end{minipage}

\textbf{MRDA} \citep{MRDA} is a dialogue act task based on 75 hours of real-life meeting transcript. Each sentence is assigned a general dialogue act (topic change, repeat, etc.) and a specific dialogue act (apology, suggestion, etc.).  % explanation sympathy

\textbf{Metrics}. For DialogRE, F1 and F1$_c$ are used as evaluation metrics. F1$_c$ modifies F1 by taking an early part of the dialogue as input \cite{yu-etal-2020-dialogue}. For MELD and EmoryNLP, we use weighted-F1 as metrics. For DailyDialog, the Micro-F1 score excluding the neutral class is used as the metric. 
\subsection{Results and Analysis}
\textbf{Overall Performance}. We first evaluated HiDialog on the Dialogue Relation Extract (DRE) dataset, DialogRE \citep{yu-etal-2020-dialogue} and the Emotion Recognition in Conversation (ERC) dataset, MELD \citep{poria-etal-2019-meld}. We selected BERT \citep{bertbase}, GDPNet \citep{xue2021gdpnet}, RoBERTa$_s$ \citep{yu-etal-2020-dialogue}, SimpleRE \citep{SimpleRE}, and TUCORE-GCN \citep{lee2021graph} as baselines. As reported in Table \ref{tab:exp-re}, HiDialog established new state-of-the-art results on both datasets. 
On the DialogRE test set, HiDialog surpassed the previous SOTA, TUCORE-GCN, by 4\% in F1 and 2.3\% in F1$_c$. On the MELD dataset, HiDialog outperformed TUCORE-GCN by 1.5\% in weighted F1. 
%, surpassing the previous SOTA by 4\% in F1 and 2.3\% in F1$_c$ on the DialogRE test set, by 1.5\% in weighted F1 score on the MELD test set. 


\textbf{Towards Generality}.
% In view of the simplicity and effectiveness of the intra-turn modeling, it is expected to have a general use for further work on dialogue understanding. To validate this idea, we incorporate our intra-turn modeling into the baseline encoder, without any extra components such as the inter-turn module or speaker embeddings. For a fair comparison, only the global $[CLS]$ token embedding from the encoder output is fed into a softmax classifier to make a prediction. 
Our intra-turn modeling's simplicity suggests its potential as a valuable solution for enhancing dialogue understanding without the need for extra pre-training. To assess this claim, we integrated it into the baseline encoder without any additional components, such as an inter-turn module or speaker embeddings. For fair comparisons, only the encoder's global $[CLS]$ token was used in a softmax classifier for prediction.
%Our intra-turn modeling's simplicity and effectiveness suggest its potential as a valuable solution for enhancing dialogue understanding, without additional pre-training. 

% \begin{wraptable}{r}{8.2cm}
%   \setlength{\tabcolsep}{3.5pt}
% % \footnotesize
% \scriptsize
%   \centering
%   \vspace{-0.35cm}
%   \begin{tabular}{lllllll}
%     \toprule
%   \multirow{2}{*}{\textbf{Method}} &\multirow{2}{*}{\textbf{MELD}} & \multirow{2}{*}{\textbf{ENLP}} & \multirow{2}{*}{\textbf{DDialog}} & \multirow{2}{*}{\textbf{MRDA}} & \multicolumn{2}{c}{\textbf{DialogRE}}      \\
%     % \textbf{Method} &\textbf{MELD} & \textbf{ENLP} & \textbf{DDialog} &\textbf{MRDA} & \multicolumn{2}{c}{\textbf{DialogRE}}      \\
%     %\cmidrule(r){2-5}
%     \cmidrule(l){6-7}   
%     % \cmidrule(l){2-2}  \cmidrule(l){3-3} \cmidrule(l){4-4} \cmidrule(l){5-5}  \cmidrule(l){6-7}  
%       &  & & & & $F1$  & $F1_c$  \\
%     \midrule
%     % BERT &61.50	& 34.17	& 54.09& 91.0 & 58.5	& 53.2\\
%     % +HiDialog & + 1.78	& +0.63 & +5.55 & +0.3 & + &\textbf{59.8}\\
%     PHT  &61.90&-&	60.14&	\textbf{92.4}&- &- \\
%     DialogXL  & 62.41 & 34.73 & 54.93 & -& - & -  \\
%     % \midrule
%     RoBERTa$_s$& 64.19&38.03	&61.65	&91.3	&71.3 & 63.7\\
% +Intra-turn &\textbf{65.64}\textsubscript{\textcolor{green}{+1.45}}& \textbf{38.13}\textsubscript{\textcolor{green}{+0.1}} &\textbf{61.83}\textsubscript{\textcolor{green}{+0.28}}&91.5\textsubscript{\textcolor{green}{+0.2}} & \textbf{74.4}\textsubscript{\textcolor{green}{+3.1}}&\textbf{66.6}\textsubscript{\textcolor{green}{+2.9}}\\
%     \bottomrule
%   \end{tabular}
%   \vspace{-0.2cm}
%   % \caption{All methods performance on 5 multi-turn dialogue-based understanding datasets: MELD, EmoryNLP (Weighted-F1), DailyDialog (Micro-F1), MRDA (Top-1 Acc.), DialogRE (F1 and F1$_c$), averaged over five runs. Performance gains over the RoBERTa$_s$ are highlighted in green.}
%     \caption{All methods performance on 5 multi-turn dialogue-based understanding datasets: MELD, EmoryNLP, DailyDialog, MRDA, DialogRE, averaged over five runs. Performance gains over the RoBERTa$_s$ are highlighted in green.}
%   \label{tab:exp-mtr}
%   \vspace{-0.35cm}
% \end{wraptable}



% \begin{minipage}[]{0.5\linewidth}
% % \begin{figure}[h]
% % \vspace{-0.2cm}
% \footnotesize
% \centering
% \includegraphics[width=7cm]{sections/length.pdf}
% \vspace{-0.5cm}
% \captionof{figure}{Analysis of robustness of HiDialog tackling increasing utterance length compared to baseline TUCORE-GCN on DialogRE dataset.}
% \vspace{0.6cm}
% \label{fig:length}
% % \end{figure}
% \end{minipage}



We conducted the experiment on 5 datasets from 3 different tasks: DRE (DialogRE), ERC (MELD, EmoryNLP \citep{zahiri:18a}, DailyDialog \citep{DailyDialog}), and Dialogue Act Classification (MRDA \citep{MRDA}). We chose RoBERTa$_s$, Pretrained Hierarchical Transformer (PHT) \citep{chapuis2020hierarchical}, and DialogXL \citep{DialogXL} as baselines. Compared to PHT and DialogXL, both of which require additional pre-training to address the domain adaption gap, the performance of proposed intra-turn modeling is surprisingly good in all 5 datasets (Table \ref{tab:exp-mtr}). 

% Moreover, we conducted an ablation study and analysis, which further reveals HiDialog is good at handling asymmetric relations and robust against increasing utterance length (see Appendix \ref{ap:ablation}). 

% \begin{wraptable}{r}{7cm}
% % \scriptsize
% % \vspace{-0.5cm}
% % \setlength{\tabcolsep}{3.5pt}
% \begin{center}

% \begin{tabular}{l c c} 
%  \toprule
%   {\textbf{Method}} & \textbf{F1} & \textbf{F1$_c$} \\
%  \midrule
% HiDialog                      & 77.1        & 68.2       \\ 
% w/o attention mask & 76.5 (-0.6) & 67.9 (-0.3) \\
% w/o special tokens & 75.6 (-1.5) & 67.4 (-0.8) \\
% only intra-turn     & 74.4 (-2.7) & 66.6 (-1.6) \\
% \bottomrule
% \end{tabular}
% \end{center}
% % \vspace{-0.2cm}
% \caption{Ablation Study on HiDialog components on DialogRE to evaluate the individual effect of turn-level attention, turn-level special tokens, and graph module. } %\textit{Ablation study. Turn-level} is omitted for brevity.}
% \label{tab:ablation_structure}
% % \vspace{-0.2cm}
% \end{wraptable}



\textbf{Ablation study on components.} We conducted an ablation study on DialogRE to evaluate key components in HiDialog: turn-level attention, turn-level special tokens, and inter-turn module (Table \ref{tab:ablation_structure}). First, after we removed the turn-level attention mask, the performance slightly dropped. In this case, these special tokens are able to aggregate information from the entire sequence, thus they are not context-aware at the turn level. We experimented with removing intra-turn modeling, resulting in only one difference from the final HiDialog: here we used an average of corresponding token embeddings for initialization. The $F1$ score decreases by 1.5\% and the $F1_c$ score declines by 0.8\%.

\textbf{Analysis of relations}. We grouped the test set of DialogRE according to the relation types into three subsets: (I) asymmetric, when a relation type differs from its inversion (e.g. \textit{children} and \textit{parents});  (II) symmetric, when a relation type is the same as its inversion (e.g. \textit{spouse}); (III) other, when a relation type does not have inversion (e.g. \textit{age}). We compared the performance of our model with baselines and report the results in Table \ref{tab:symmetric}. As we can observe, there is a great performance increase in the asymmetric subset while the F1 score drops moderately for symmetric relations. This trend reverses when we remove the graph module in our method (i.e. symmetric $>$ asymmetric). 
% \clearpage

\textbf{Analysis of robustness against increasing utterance length.} With the hierarchical aggregation in HiDialog, each turn-level special token is enforced to capture intra-turn critical information regardless of the whole dialogue. This nature enables our method to handle dialogues of various lengths. Thus, we further divided the samples in the DialogRE test set into six groups according to their lengths and compared HiDialog against the previous SOTA, TUCORE-GCN. As shown in Figure \ref{fig:length}, our method consistently outperforms TUCORE-GCN in all groups, where the largest performance gap can be found in the group with less than 100 tokens. Moreover, TUCORE-GCN shows a great drop with an increase of length (i.e., from $[400,500)$ to $[500,+\infty)$), while HiDialog maintains decent performance for long sequences.

 \begin{figure}[t]
\centering
% \vspace{-0.2cm}
\footnotesize
\centering
\includegraphics[width=7cm]{sections/length.pdf}
% \vspace{-0.5cm}
\captionof{figure}{Analysis of robustness of HiDialog tackling increasing utterance length compared to baseline TUCORE-GCN on DialogRE dataset.}
% \vspace{0.6cm}
\label{fig:length}
\end{figure}


%Note that both PHT and DialogXL are pre-trained methods that require extra computation
%Moreover, it outperforms the current SOTA by 1.3\% in $F1$ and 0.7\% in $F1_c$ on the DialogRE dataset, by 0.3\% in weighted F1 on MELD.
%HiDialog performs well in managing asymmetric relations and handling longer utterances, as revealed in our ablation study (see Appendix \ref{ap:ablation}). It provides an effective solution for bridging the gap between pre-training on general corpora and dialogue understanding without additional computational costs or training data while maintaining good performance. 
%Considering that our turn-level attention is easy to adopt and does not introduce any parameters to the base encoder, we believe it can be used as a strong baseline or plug-in module for future work in the community.
%Our turn-level attention mechanism can be effortlessly integrated into the base encoder without introducing any new parameters. Thus, we anticipate that it will serve as a compelling baseline or plug-in module for upcoming research in this field.
\section{CONCLUSIONS}
We consider the map prediction problem where the measurements are noisy, sparse and partially observed. We first show that many maps possess low-rank and incoherent structures. Then we propose to use LRMC to perform the map prediction. Our extensive experiments validate that the LRMC method can effectively perform map prediction (interpolation and extrapolation), and the LRMC method outperforms  state-of-the-art Bayesian Hilbert Mapping in terms of mapping accuracy and computation time and could update the entire map in real-time. Lastly we show that the proposed real-time map prediction method can be easily combined with popular coverage planning methods and can significantly improve their coverage convergence rates.
 
{
    \small
    \bibliographystyle{ieeenat_fullname}
    \bibliography{main}
}
%\ 
% !TEX root = ../supp.tex
% !TEX spellcheck = en-US

\section{Physical Rendering of SwissCube}
\label{sec:appendix}

Although the European Space Agency has organized a satellite pose estimation challenge and released the SPEED satellite dataset, the unavailability of the target 3D model makes the pose accuracy not depending on the pose estimation method alone. Furthermore, the limited varieties of lighting also make it soon saturated and less discriminative, as discussed in Section~\ref{sec:related}.

To fully demonstrate the effectiveness of our method in space, we introduce the Swisscube satellite dataset. Swisscube is a Cubesat-type satellite which was designed at EPFL and launched in 2009. Given the accurate CAD files and material properties of each component of it, we synthesize photorealistic images using physically based rendering~\cite{xx}.

 
% !TEX root = ../top.tex
% !TEX spellcheck = en-US

\begin{table}
    \centering
    \begin{small}
    % \rowcolors{2}{white}{gray!10}
    \begin{tabular}{lcc}
        \toprule
        &	SPEED & {\bf CubeSat}\\
        \midrule
        % Synthetic  &12k & 40k \\
        % Real       & 5 & 300 \\
        % \midrule
        Size & 12k & 40k \\
        Accurate 3D model   &\xmark  & \cmark  \\
        Complex lighting    &\xmark  & \cmark \\
        Physical modeling    &\xmark  & \cmark \\
        Colors        &\xmark  & \cmark \\
        Sequences         &\xmark  & \cmark  \\
        Depth distribution & non-uniform & uniform \\
        \bottomrule
    \end{tabular}
    \end{small}
    % \vspace{-3mm}
    \caption{{\bf CubeSat dataset.} bla bla bla bla bla bla bla bla bla bla bla bla bla bla bla bla bla bla bla bla bla bla bla bla bla bla bla bla bla bla bla bla bla bla bla bla bla bla bla bla bla bla bla bla bla bla bla bla bla bla bla bla bla bla bla bla bla bla bla bla bla bla bla bla bla bla bla bla bla bla bla bla bla bla bla bla bla bla bla bla }
    \label{tab:swisscube_vs_speed}
\end{table} 

% -----------------------------------
% Physically-based spectral rendering

\subsubsection{Physically-based spectral rendering}

In this section we provide a high-level description of the Swisscube satellite dataset which collects 40'000  physically-based synthetic images. While the SPEED satellite dataset images were produced using an OpenGL-based RGB rendering framework, we opted for a physically-based approach, where every element of the rendering pipeline were carefully modeled to mimic reality.

While the RGB model is often used to render color images, using tristimulus RGB colors in the rendering simulation generally yields non-physical results. For instance, surface reflectance properties of an object can be highly dependent on the wavelengths, which won't be accurately reproduced with RGB values. In a spectral renderer, colors are represented as spectral power distributions, resulting in improved accuracy especially when measured spectral data is available. For the Swisscube dataset, using a spectral renderer was a necessity as it was a requirement to correctly model the spectral responses of the solar irradiance emission, material reflectance properties and Earth surface radiance. Although the rendering simulation uses spectral colors, the resulting images will be converted to RGB images.

Relying on a physically-based rendering pipeline also gives us more control on the dynamic range of the output images. Thus we were able to accurately reproduce highlights orders of magnitude brighter than darker region of the images. An appropriate gamma curve could then be applied to produce images that can be viewed on regular displays.

To achieve all of this, we build our pipeline around the Mitsuba 2 renderer [???] which is a highly modular open-source framework that supports spectral rendering.

% -------------------
% The 3D / CAD model

\subsubsection{Accurate 3D model from CAD data}

For this dataset, we modelled every mechanical parts of the SwissCube, such as solar panels, antennas, and screws based directly on the raw CAD files. We carefully assign material reflection properties to each part. Given the physically-based nature of the pipeline, it would be possible to use efficient material acquisition technique such as [???] in the future for better results. Due to confidential reasons, we only release the mesh geometry data of the combined SwissCube without separable pieces to the public, which is enough for perfect registration.

% Citation for material acquistion technique: https://rgl.epfl.ch/publications/Dupuy2018Adaptive

% --------------------
% Physically-based Sun

\subsubsection{Modeling a physically-based Sun emitter}

In order to correctly model the illumination from the Sun, we leveraged the vast literature in astrophysics. As the target object will be placed above the Earth atmosphere, it is not enough to use specular solar irradiance measurements made at ground surface [???] as those will be be affected by the highly variable and absorbing constituents of the Earth atmosphere. Instead, we rely on the air mass zero reference spectrum [???], also known as extraterrestrial solar irradiance, mainly based on data from satellites and space shuttle missions. Figure \ref{fig:swisscube_sun_spectrum} show its spectral power distribution. We then use a point light source to represent the Sun, placed at the correct distance to the Earth. Note that is was necessary to scale the Sun irradiance to account for its surface area.

% Groud surface citation: https://www.osapublishing.org/ao/abstract.cfm?uri=ao-21-3-554
% ASTM Standard Extraterrestrial Spectrum Reference: https://www.nrel.gov/grid/solar-resource/spectra-astm-e490.html

% !TEX root = ../top.tex
% !TEX spellcheck = en-US

\begin{figure}[t]
    \begin{center}
    \includegraphics[width=1.0\linewidth]{./fig/swisscube_sun_spectrum/sun_spectrum_plot.jpeg}
    % \fbox{\rule{0pt}{2in} \rule{0.25\linewidth}{0pt}}
    \end{center}
    \vspace{-6mm}
    \caption{{\bf Air mass zero solar spectral power distribution.} 
    }
    \label{fig:swisscube_sun_spectrum}
\end{figure}


% ---------------------------
% Modeling the stars / galaxy

\subsubsection{Modeling the stars and galaxies}

We also added galaxies and other astronomical objects to our pipeline as we believe those could distract the learning algorithm. Based on the HYG database star catalogue [???], we could generate a high-resolution environment map that we later used as an second emitter. The HYG database contains around 220 thousands astronomical objects, mostly galaxies but also star clusters and Nebulae along with information regarding their position and brightness. Figure \ref{fig:swisscube_stars_envmap} shows the astronomical object projected on a spherical coordinate 2D map with their respective physically-based brightness.

Compared to the sun illumination, the irradiance coming from the stars is orders of magnitude lower. On the other hand, to maximize the diversities of the generated data, we decided to increase the actual brightness of each star in the galaxies to make them more apparent in rendering, which we think is a beneficial perturbation for a dataset.

% !TEX root = ../top.tex
% !TEX spellcheck = en-US

\begin{figure}[t]
    \begin{center}
    \includegraphics[width=1.0\linewidth]{./fig/swisscube_stars_envmap/stars_envmap.jpeg}
    % \fbox{\rule{0pt}{2in} \rule{0.25\linewidth}{0pt}}
    \end{center}
    \vspace{-6mm}
    \caption{{\bf Astronomical objects environment map based on the HYG dataset.} 
    }
    \label{fig:swisscube_stars_envmap}
\end{figure}


% HYG database link: http://www.astronexus.com/hyg

% ---------------------------------------------------
% Modeling the Earth radiance using the VIIRS dataset

\subsubsection{Spectral Earth radiance using the VIIRS dataset}

Properly modeling the Earth is very important here as it often occupies a large portion of the images. Moreover, the Sun light reflecting off the Earth is drastically affecting the illumination of the target object. In our pipeline, the Earth is a represented as a very large sphere, reflecting light coming from the Sun emitter onto the target object or directly towards the camera. Based on the NASA Visible Infrared Imaging Radiometer Suite (VIIRS) Level-1B data products [????], we generated a spectral radiance texture to model the reflectance of the Earth and its atmosphere. The VIIRS data products are produced by whiskbroom scanning radiometers on satellites orbiting around the Earth at a nominal altitude of 829 km, providing a full daily coverage of the Earth. These data products include 6 bands in the visible spectrum with high spatial resolution which we could use to generate a spectral reflectance texture. Figure \ref{fig:swisscube_earth_renders} shows the result of this process compared to the use of a simple Earth albedo texture available from the NASA website [????].

% !TEX root = ../supp.tex
% !TEX spellcheck = en-US

\begin{figure}[t]
    \centering
    \begin{tabular}{ccc}
    \includegraphics[height=2.4cm]{fig/swisscube_earth_renders_L1/earth_L1_albedo_texture.png}&
    \includegraphics[height=2.4cm]{fig/swisscube_earth_renders_L1/earth_L1_spectral_texture.png}&
    \includegraphics[height=2.4cm]{fig/swisscube_earth_renders_L1/earth_L1_DSCOVR.png}\\
    (a)&(b)&(c)\\
    \end{tabular}
    \vspace{-3mm}
    \caption{\small {\bf Rendered Earth compared to DSCOVR photograph.} (a) Ground-level albedo texture doesn't account for the scattering effects introduced by the atmosphere, resulting in over saturated colors when viewed from space. (b) Rendering of the Earth using our spectral texture based on the VIIRS data products. (c) Ground-truth real photograph taken by the DSCOVR satellite at the L1 Lagrange point.}
    \label{fig:swisscube_earth_renders}
\end{figure}

% VIIRS website: https://earthdata.nasa.gov/earth-observation-data/near-real-time/download-nrt-data/viirs-nrt#ed-corrected-reflectance
% Earth albedo texture: https://visibleearth.nasa.gov/images/57735/the-blue-marble-land-surface-ocean-color-sea-ice-and-clouds/57737l
% Real image DSCOVR website: https://www.nesdis.noaa.gov/content/dscovr-deep-space-climate-observatory

\subsubsection{Bring everything together at real scale}

We placed all those elements in a virtual scene at the real scale to ensure high fidelity images and produce a more comprehensive dataset. We could then generate sequences of images, simulating various docking procedures by varying camera and target vehicle poses and respective speed. Each sequence contains 100 consecutive images.



To achieve highest reflection of the real world, we place the SwissCube in the actual-working orbits about 700 km above the Earth's surface during rendering and model most of the space-borne items, such as the Sun, the Earth, and galaxies, physically. 

Depth range.

\subsubsection{Dataset generation and specs}

We generate 400 sequences in total, each sequence contains 100 consecutive images with random angular speed between 0.xx to 0.xx.

Fig.~\ref{fig:swisscube_vs_speed} shows some examples of the generated SwissCube dataset.

bla bla bla bla bla bla bla bla bla bla bla bla bla bla bla bla bla bla bla bla bla bla bla bla bla bla bla bla bla bla bla bla bla bla bla bla bla bla bla bla bla bla 

As discussed Section~\ref{sec:related}. Table~\ref{tab:swisscube_vs_speed} shows the comparison between SwissCube and SPEED dataset. Note that, in SPEED dataset, there are 2998 more synthetic images and also 300 more real images in the test set. However, Their ground truth labels are not accessible, so here we do not take them into account.

\YH{TODO}

From the total 400 sequences of images in the SwissCube dataset, we take all the images from the first 300 sequences as our training set and that from the last 100 as the test. In this subsection, we will evaluate the methods' performance in different depth ranges, that is, the whole depth range will be divided to three regions denoted as {\it Near}, {\it Medium} and {\it Far}, which correspond to depth range [1d-4d], [4d-7d] and [7d-10d], respectively.

\end{document}
