\documentclass[journal]{vgtc}                % final (journal style)
% \documentclass[review,journal]{vgtc}         % review (journal style)
%\documentclass[widereview]{vgtc}             % wide-spaced review
%\documentclass[preprint,journal]{vgtc}       % preprint (journal style)

%% Uncomment one of the lines above depending on where your paper is
%% in the conference process. ``review'' and ``widereview'' are for review
%% submission, ``preprint'' is for pre-publication, and the final version
%% doesn't use a specific qualifier.

%% Please use one of the ``review'' options in combination with the
%% assigned online id (see below) ONLY if your paper uses a double blind
%% review process. Some conferences, like IEEE Vis and InfoVis, have NOT
%% in the past.

%% Please use the ``preprint''  option when producing a preprint version
%% for sharing your article on an open access repository

%% Please note that the use of figures other than the optional teaser is not permitted on the first page
%% of the journal version.  Figures should begin on the second page and be
%% in CMYK or Grey scale format, otherwise, colour shifting may occur
%% during the printing process.  Papers submitted with figures other than the optional teaser on the
%% first page will be refused. Also, the teaser figure should only have the
%% width of the abstract as the template enforces it.

%% These few lines make a distinction between latex and pdflatex calls and they
%% bring in essential packages for graphics and font handling.
%% Note that due to the \DeclareGraphicsExtensions{} call it is no longer necessary
%% to provide the the path and extension of a graphics file:
%% \includegraphics{diamondrule} is completely sufficient.
%%
% \ifpdf%                                % if we use pdflatex
%   \pdfoutput=1\relax                   % create PDFs from pdfLaTeX
%   \pdfcompresslevel=9                  % PDF Compression
%   \pdfoptionpdfminorversion=7          % create PDF 1.7
%   \ExecuteOptions{pdftex}
%   \usepackage{graphicx}                % allow us to embed graphics files
%   \DeclareGraphicsExtensions{.pdf,.png,.jpg,.jpeg} % for pdflatex we expect .pdf, .png, or .jpg files
% \else%                                 % else we use pure latex
%   \ExecuteOptions{dvips}
%   \usepackage{graphicx}                % allow us to embed graphics files
%   \DeclareGraphicsExtensions{.eps}     % for pure latex we expect eps files
% \fi%

%% it is recomended to use ``\autoref{sec:bla}'' instead of ``Fig.~\ref{sec:bla}''
\graphicspath{{figures/}{pictures/}{images/}{./}} % where to search for the images

% \usepackage{microtype}                 % use micro-typography (slightly more compact, better to read)
% \PassOptionsToPackage{warn}{textcomp}  % to address font issues with \textrightarrow
% \usepackage{textcomp}                  % use better special symbols
% \usepackage{mathptmx}                  % use matching math font
% \usepackage{times}                     % we use Times as the main font
% \renewcommand*\ttdefault{txtt}         % a nicer typewriter font
% \usepackage{cite}                      % needed to automatically sort the references
% \usepackage{tabu}                      % only used for the table example
% \usepackage{booktabs}                  % only used for the table example


\usepackage{subfiles}
\usepackage{xspace}
% \usepackage{enumitem}
% \usepackage{hyperref}
% \usepackage{amssymb}
\usepackage{multicol}
\usepackage{multirow}
\usepackage{booktabs}
% \usepackage{nicefrac}
% \usepackage{graphics}
% \usepackage{subcaption}
% \usepackage{enumitem}
\usepackage[svgnames]{xcolor}
\usepackage{makecell}

\def\measure{CLAMS\xspace}

\def\Autoss{AutoScatterplotSampling\xspace}
\def\autoss{AutoSS\xspace}

\def\ambreducer{AmbReducer\xspace}

\newcommand{\fix}[1]{\textcolor{red}{\textbf{\textit{#1}}}}
\newcommand{\hj}[1]{\textcolor{blue}{\textbf[PR: #1]}}
\newcommand{\gh}[1]{\textcolor{green}{\textbf[[GH: #1]]}}


\newcommand{\subscript}[2]{$#1 _ #2$}


%% We encourage the use of mathptmx for consistent usage of times font
%% throughout the proceedings. However, if you encounter conflicts
%% with other math-related packages, you may want to disable it.

%% In preprint mode you may define your own headline. If not, the default IEEE copyright message will appear in preprint mode.
%\preprinttext{To appear in IEEE Transactions on Visualization and Computer Graphics.}

%% In preprint mode, this adds a link to the version of the paper on IEEEXplore
%% Uncomment this line when you produce a preprint version of the article 
%% after the article receives a DOI for the paper from IEEE
%\ieeedoi{xx.xxxx/TVCG.201x.xxxxxxx}

%% If you are submitting a paper to a conference for review with a double
%% blind reviewing process, please replace the value ``0'' below with your
%% OnlineID. Otherwise, you may safely leave it at ``0''.
\onlineid{1026}

%% declare the category of your paper, only shown in review mode
\vgtccategory{Research}
%% please declare the paper type of your paper to help reviewers, only shown in review mode
%% choices:
%% * algorithm/technique
%% * application/design study
%% * evaluation
%% * system
%% * theory/model
\vgtcpapertype{Theoretical \& Empirical}

%% Paper title.
\title{
% Measuring Cluster Ambiguity: Study on the Variance of Human Cluster Perception in Monochrome Scatterplots \\ 
\textit{\measure}: A Cluster Ambiguity Measure for Estimating \texorpdfstring{\\}{} Perceptual Variability in Visual Clustering}



%% This is how authors are specified in the journal style

%% indicate IEEE Member or Student Member in form indicated below
\author{Hyeon Jeon*, Ghulam Jilani Quadri*, Hyunwook Lee, Paul Rosen, Danielle Albers Szafir, and Jinwook Seo}
\authorfooter{

\item Hyeon Jeon and Jinwook Seo are with Seoul National University. E-mail: hj@hcil.snu.ac.kr, jseo@snu.ac.kr
\item Ghulam Jilani Quadri and Danielle Albers Szafir are with the University of
North Carolina, Chapel Hill. E-mail: \{jiquad, danielle.szafir\}@cs.unc.edu
\item Hyunwook Lee is with UNIST. E-mail: gusdnr0916@unist.ac.kr
\item Paul Rosen is with the University of Utah. E-mail: prosen@sci.utah.edu
\item Hyeon Jeon and Ghulam Jilani Quadri equally contributed to the work.
%% insert punctuation at end of each item
% \item
%  Roy G. Biv is with Starbucks Research. E-mail: roy.g.biv@aol.com.
% \item
%  Ed Grimley is with Grimley Widgets, Inc.. E-mail: ed.grimley@aol.com.
% \item
%  Martha Stewart is with Martha Stewart Enterprises at Microsoft
%  Research. E-mail: martha.stewart@marthastewart.com.
}

%other entries to be set up for journal
\shortauthortitle{Biv \MakeLowercase{\textit{et al.}}: Global Illumination for Fun and Profit}
%\shortauthortitle{Firstauthor \MakeLowercase{\textit{et al.}}: Paper Title}

%% Abstract section.
\abstract{
  % Visual 
  % Clustering is a key visual analytic tasks for scatterplots. 
  % Compelling 
  % Scatterplots support the understanding of data by leveraging visual perception to boost awareness while performing visual analytic tasks.
Visual clustering is a common perceptual task in scatterplots that supports diverse analytics tasks (e.g., cluster identification).
However, even with the same scatterplot, the ways of perceiving clusters (i.e., conducting visual clustering) can differ due to the differences among individuals and ambiguous cluster boundaries. 
% 
Although such perceptual variability casts doubt on the reliability of data analysis based on visual clustering, we lack a systematic way to efficiently assess this variability.
% 
In this research, we study perceptual variability in conducting visual clustering, which we call \textit{Cluster Ambiguity}.
To this end, we introduce \textit{\measure}, a data-driven visual quality measure for automatically predicting cluster ambiguity in monochrome scatterplots. 
% 
We first conduct a qualitative study to identify key factors that affect the visual separation of clusters (e.g., proximity or size difference between clusters).
Based on study findings, we deploy a regression module that estimates the human-judged separability of two clusters.
Then, \measure predicts cluster ambiguity by analyzing the aggregated results of all pairwise separability between clusters that are generated by the module. 
% The model is used by \measure to predict cluster ambiguity as the potential inter-subject variability in visual clustering.
\measure outperforms widely-used clustering techniques in predicting ground truth cluster ambiguity. Meanwhile, \measure exhibits performance on par with human annotators.
We conclude our work by presenting two applications for optimizing and benchmarking data mining techniques using \measure. The \textbf{interactive demo} of \measure is available at \texttt{\href{http://www.clusterambiguity.dev.s3-website.ap-northeast-2.amazonaws.com/}{clusterambiguity.dev}.}
} % end of abstract

%% Keywords that describe your work. Will show as 'Index Terms' in journal
%% please capitalize first letter and insert punctuation after last keyword
\keywords{Cluster, scatterplot, perception, cluster analysis, cluster ambiguity, visual quality measure.}

%% ACM Computing Classification System (CCS). 
%% See <http://www.acm.org/class/1998/> for details.
%% The ``\CCScat'' command takes four arguments.

% \CCScatlist{ % not used in journal version
%  \CCScat{K.6.1}{Management of Computing and Information Systems}%
% {Project and People Management}{Life Cycle};
%  \CCScat{K.7.m}{The Computing Profession}{Miscellaneous}{Ethics}
% }


\definecolor{myblue}{RGB}{7, 115, 193}
\definecolor{myred}{RGB}{255, 10, 9}


\newcommand{\blue}[1]{{\color{myblue}#1}}
\newcommand{\red}[1]{{\color{myred}#1}}

\newcommand{\rev}[1]{{#1}}



\newcommand{\figureautorefname}{Fig.}
\newcommand{\sectionautorefname}{Sect.}
\newcommand{\subsectionautorefname}{\sectionautorefname}
\newcommand{\subsubsectionautorefname}{\sectionautorefname}



%% A teaser figure can be included as follows
\teaser{
  \centering
  \includegraphics[width=\linewidth]{figures/teaser.pdf}
  \caption{The comparison between the way of estimating the perceptual variability in conducting visual clustering, i.e., cluster ambiguity, of monochrome scatterplots. 
  Without \measure ($\beta$), we need to ask people to conduct visual clustering (colored scatterplots with $\beta$; color labels depict visual clustering results) and check whether the results are \blue{consistent (i.e., clear)} or \red{not (i.e., ambiguous)}. 
  This approach requires extensive time and human resources. 
  In contrast, \measure ($\alpha$), which is constructed over perceptual data and a feature engineering based on a user study, 
  automatically produces a score representing cluster ambiguity of an input scatterplot, 
  where \blue{low} and \red{high} scores correspond to \blue{clear} and \red{ambiguous} cluster structure. 
  }
  \label{fig:teaser}
}

%% Uncomment below to disable the manuscript note
%\renewcommand{\manuscriptnotetxt}{}

%% Copyright space is enabled by default as required by guidelines.
%% It is disabled by the 'review' option or via the following command:
% \nocopyrightspace


% \vgtcinsertpkg

%%%%%%%%%%%%%%%%%%%%%%%%%%%%%%%%%%%%%%%%%%%%%%%%%%%%%%%%%%%%%%%%
%%%%%%%%%%%%%%%%%%%%%% START OF THE PAPER %%%%%%%%%%%%%%%%%%%%%%
%%%%%%%%%%%%%%%%%%%%%%%%%%%%%%%%%%%%%%%%%%%%%%%%%%%%%%%%%%%%%%%%%

\begin{document}

%% The ``\maketitle'' command must be the first command after the
%% ``\begin{document}'' command. It prepares and prints the title block.

%% the only exception to this rule is the \firstsection command
\firstsection{Introduction\label{sec:intro}}
\maketitle

\subfile{sections/01_introduction}
\subfile{sections/02_related_works}
\subfile{sections/03_measure}
\subfile{sections/04_evaluations}
\subfile{sections/05_applications}
\subfile{sections/06_conclusion}

% %% \section{Introduction} %for journal use above \firstsection{..} instead
% This template is for papers of VGTC-sponsored conferences such as IEEE VIS, IEEE VR, and ISMAR which are published as special issues of TVCG. The template does not contain the respective dates of the conference/journal issue, these will be entered by IEEE as part of the publication production process. Therefore, \textbf{please leave the copyright statement at the bottom-left of this first page untouched}.

% \section{Using the Style Template}

% \begin{itemize}
% \item If you receive compilation errors along the lines of ``\texttt{Package ifpdf Error: Name clash, \textbackslash ifpdf is already defined}'' then please add a new line ``\texttt{\textbackslash let\textbackslash ifpdf\textbackslash relax}'' right after the ``\texttt{\textbackslash documentclass[journal]\{vgtc\}}'' call. Note that your error is due to packages you use that define ``\texttt{\textbackslash ifpdf}'' which is obsolete (the result is that \texttt{\textbackslash ifpdf} is defined twice); these packages should be changed to use ifpdf package instead.
% \item Note that each author's affiliations have to be provided in the author footer on the bottom-left corner of the first page. It is permitted to merge two or more people from the same institution as long as they are shown in the same order as in the overall author sequence on the top of the first page. For example, if authors A, B, C, and D are from institutions 1, 2, 1, and 2, respectively, then it is ok to use 2 bullets as follows:
% \begin{itemize}
% \item A and C are with Institution 1. E-mail: \{a\,$|$\,c\}@i1.com\,.
% \item B and D are with Institution 2. E-mail: \{b\,$|$\,d\}@i2.org\,.
% \end{itemize}
% \item The style uses the hyperref package, thus turns references into internal links. We thus recommend to make use of the ``\texttt{\textbackslash autoref\{reference\}}'' call (instead of ``\texttt{Figure\~{}\textbackslash ref\{reference\}}'' or similar) since ``\texttt{\textbackslash autoref\{reference\}}'' turns the entire reference into an internal link, not just the number. Examples: \autoref{fig:sample} and \autoref{tab:vis_papers}.
% \item The style automatically looks for image files with the correct extension (eps for regular \LaTeX; pdf, png, and jpg for pdf\LaTeX), in a set of given subfolders (figures/, pictures/, images/). It is thus sufficient to use ``\texttt{\textbackslash includegraphics\{CypressView\}}'' (instead of ``\texttt{\textbackslash includegraphics\{pictures/CypressView.jpg\}}'').
% \item For adding hyperlinks and DOIs to the list of references, you can use ``\texttt{\textbackslash bibliographystyle\{abbrv-doi-hyperref-narrow\}}'' (instead of ``\texttt{\textbackslash bibliographystyle\{abbrv\}}''). It uses the doi and url fields in a bib\TeX\ entry and turns the entire reference into a link, giving priority to the doi. The doi can be entered with or without the ``\texttt{http://dx.doi.org/}'' url part. See the examples in the bib\TeX\ file and the bibliography at the end of this template.\\[1em]
% \textbf{Note 1:} occasionally (for some \LaTeX\ distributions) this hyper-linked bib\TeX\ style may lead to \textbf{compilation errors} (``\texttt{pdfendlink ended up in different nesting level ...}'') if a reference entry is broken across two pages (due to a bug in hyperref). In this case make sure you have the latest version of the hyperref package (i.\,e., update your \LaTeX\ installation/packages) or, alternatively, revert back to ``\texttt{\textbackslash bibliographystyle\{abbrv-doi-narrow\}}'' (at the expense of removing hyperlinks from the bibliography) and try ``\texttt{\textbackslash bibliographystyle\{abbrv-doi-hyperref-narrow\}}'' again after some more editing.\\[1em]
% \textbf{Note 2:} the ``\texttt{-narrow}'' versions of the bibliography style use the font ``PTSansNarrow-TLF'' for typesetting the DOIs in a compact way. This font needs to be available on your \LaTeX\ system. It is part of the \href{https://www.ctan.org/pkg/paratype}{``paratype'' package}, and many distributions (such as MikTeX) have it automatically installed. If you do not have this package yet and want to use a ``\texttt{-narrow}'' bibliography style then use your \LaTeX\ system's package installer to add it. If this is not possible you can also revert to the respective bibliography styles without the ``\texttt{-narrow}'' in the file name.\\[1em]
% DVI-based processes to compile the template apparently cannot handle the different font so, by default, the template file uses the \texttt{abbrv-doi} bibliography style but the compiled PDF shows you the effect of the \texttt{abbrv-doi-hyperref-narrow} style.
% \end{itemize}

% \section{Bibliography Instructions}

% \begin{itemize}
% \item Sort all bibliographic entries alphabetically but the last name of the first author. This \LaTeX/bib\TeX\ template takes care of this sorting automatically.
% \item Merge multiple references into one; e.\,g., use \cite{Max:1995:OMF,Kitware:2003} (not \cite{Kitware:2003}\cite{Max:1995:OMF}). Within each set of multiple references, the references should be sorted in ascending order. This \LaTeX/bib\TeX\ template takes care of both the merging and the sorting automatically.
% \item Verify all data obtained from digital libraries, even ACM's DL and IEEE Xplore  etc.\ are sometimes wrong or incomplete.
% \item Do not trust bibliographic data from other services such as Mendeley.com, Google Scholar, or similar; these are even more likely to be incorrect or incomplete.
% \item Articles in journal---items to include:
%   \begin{itemize}
%   \item author names
% 	\item title
% 	\item journal name
% 	\item year
% 	\item volume
% 	\item number
% 	\item month of publication as variable name (i.\,e., \{jan\} for January, etc.; month ranges using \{jan \#\{/\}\# feb\} or \{jan \#\{-{}-\}\# feb\})
%   \end{itemize}
% \item use journal names in proper style: correct: ``IEEE Transactions on Visualization and Computer Graphics'', incorrect: ``Visualization and Computer Graphics, IEEE Transactions on''
% \item Papers in proceedings---items to include:
%   \begin{itemize}
%   \item author names
% 	\item title
% 	\item abbreviated proceedings name: e.\,g., ``Proc.\textbackslash{} CONF\_ACRONYNM'' without the year; example: ``Proc.\textbackslash{} CHI'', ``Proc.\textbackslash{} 3DUI'', ``Proc.\textbackslash{} Eurographics'', ``Proc.\textbackslash{} EuroVis''
% 	\item year
% 	\item publisher
% 	\item town with country of publisher (the town can be abbreviated for well-known towns such as New York or Berlin)
%   \end{itemize}
% \item article/paper title convention: refrain from using curly brackets, except for acronyms/proper names/words following dashes/question marks etc.; example:
% \begin{itemize}
% 	\item paper ``Marching Cubes: A High Resolution 3D Surface Construction Algorithm''
% 	\item should be entered as ``\{M\}arching \{C\}ubes: A High Resolution \{3D\} Surface Construction Algorithm'' or  ``\{M\}arching \{C\}ubes: A high resolution \{3D\} surface construction algorithm''
% 	\item will be typeset as ``Marching Cubes: A high resolution 3D surface construction algorithm''
% \end{itemize}
% \item for all entries
% \begin{itemize}
% 	\item DOI can be entered in the DOI field as plain DOI number or as DOI url; alternative: a url in the URL field
% 	\item provide full page ranges AA-{}-BB
% \end{itemize}
% \item when citing references, do not use the reference as a sentence object; e.\,g., wrong: ``In \cite{Lorensen:1987:MCA} the authors describe \dots'', correct: ``Lorensen and Cline \cite{Lorensen:1987:MCA} describe \dots''
% \end{itemize}

% \section{Example Section}

% Lorem\marginpar{\small You can use the margins for comments while editing the submission, but please remove the marginpar comments for submission.} ipsum dolor sit amet, consetetur sadipscing elitr, sed diam
% nonumy eirmod tempor invidunt ut labore et dolore magna aliquyam erat,
% sed diam voluptua. At vero eos et accusam et justo duo dolores et ea
% rebum. Stet clita kasd gubergren, no sea takimata sanctus est Lorem
% ipsum dolor sit amet. Lorem ipsum dolor sit amet, consetetur
% sadipscing elitr, sed diam nonumy eirmod tempor invidunt ut labore et
% dolore magna aliquyam erat, sed diam
% voluptua~\cite{Kitware:2003,Max:1995:OMF}. At vero eos et accusam et
% justo duo dolores et ea rebum. Stet clita kasd gubergren, no sea
% takimata sanctus est Lorem ipsum dolor sit amet. Lorem ipsum dolor sit
% amet, consetetur sadipscing elitr, sed diam nonumy eirmod tempor
% invidunt ut labore et dolore magna aliquyam erat, sed diam
% voluptua. At vero eos et accusam et justo duo dolores et ea
% rebum. Stet clita kasd gubergren, no sea takimata sanctus est.

% \section{Exposition}

% Duis autem vel eum iriure dolor in hendrerit in vulputate velit esse
% molestie consequat, vel illum dolore eu feugiat nulla facilisis at
% vero eros et accumsan et iusto odio dignissim qui blandit praesent
% luptatum zzril delenit augue duis dolore te feugait nulla
% facilisi. Lorem ipsum dolor sit amet, consectetuer adipiscing elit,
% sed diam nonummy nibh euismod tincidunt ut laoreet dolore magna
% aliquam erat volutpat~\cite{Kindlmann:1999:SAG}.

% \begin{equation}
% \sum_{j=1}^{z} j = \frac{z(z+1)}{2}
% \end{equation}

% Lorem ipsum dolor sit amet, consetetur sadipscing elitr, sed diam
% nonumy eirmod tempor invidunt ut labore et dolore magna aliquyam erat,
% sed diam voluptua. At vero eos et accusam et justo duo dolores et ea
% rebum. Stet clita kasd gubergren, no sea takimata sanctus est Lorem
% ipsum dolor sit amet. Lorem ipsum dolor sit amet, consetetur
% sadipscing elitr, sed diam nonumy eirmod tempor invidunt ut labore et
% dolore magna aliquyam erat, sed diam voluptua. At vero eos et accusam
% et justo duo dolores et ea rebum. Stet clita kasd gubergren, no sea
% takimata sanctus est Lorem ipsum dolor sit amet.

% \subsection{Lorem ipsum}

% Lorem ipsum dolor sit amet (see \autoref{tab:vis_papers}), consetetur sadipscing elitr, sed diam
% nonumy eirmod tempor invidunt ut labore et dolore magna aliquyam erat,
% sed diam voluptua. At vero eos et accusam et justo duo dolores et ea
% rebum. Stet clita kasd gubergren, no sea takimata sanctus est Lorem
% ipsum dolor sit amet. Lorem ipsum dolor sit amet, consetetur
% sadipscing elitr, sed diam nonumy eirmod tempor invidunt ut labore et
% dolore magna aliquyam erat, sed diam voluptua. At vero eos et accusam
% et justo duo dolores et ea rebum. Stet clita kasd gubergren, no sea
% takimata sanctus est Lorem ipsum dolor sit amet. Lorem ipsum dolor sit
% amet, consetetur sadipscing elitr, sed diam nonumy eirmod tempor
% invidunt ut labore et dolore magna aliquyam erat, sed diam
% voluptua. At vero eos et accusam et justo duo dolores et ea
% rebum. 

% \begin{table}[tb]
%   \caption{VIS/VisWeek accepted/presented papers: 1990--2016.}
%   \label{tab:vis_papers}
%   \scriptsize%
% 	\centering%
%   \begin{tabu}{%
% 	r%
% 	*{7}{c}%
% 	*{2}{r}%
% 	}
%   \toprule
%    year & \rotatebox{90}{Vis/SciVis} &   \rotatebox{90}{SciVis conf} &   \rotatebox{90}{InfoVis} &   \rotatebox{90}{VAST} &   \rotatebox{90}{VAST conf} &   \rotatebox{90}{TVCG @ VIS} &   \rotatebox{90}{CG\&A @ VIS} &   \rotatebox{90}{VIS/VisWeek} \rotatebox{90}{incl. TVCG/CG\&A}   &   \rotatebox{90}{VIS/VisWeek} \rotatebox{90}{w/o TVCG/CG\&A}   \\
%   \midrule
% 	2016 & 30 &   & 37 & 33 & 15 & 23 & 10 & 148 & 115 \\
%   2015 & 33 & 9 & 38 & 33 & 14 & 17 & 15 & 159 & 127 \\
%   2014 & 34 &   & 45 & 33 & 21 & 20 &   & 153 & 133 \\
%   2013 & 31 &   & 38 & 32 &   & 20 &   & 121 & 101 \\
%   2012 & 42 &   & 44 & 30 &   & 23 &   & 139 & 116 \\
%   2011 & 49 &   & 44 & 26 &   & 20 &   & 139 & 119 \\
%   2010 & 48 &   & 35 & 26 &   &   &   & 109 & 109 \\
%   2009 & 54 &   & 37 & 26 &   &   &   & 117 & 117 \\
%   2008 & 50 &   & 28 & 21 &   &   &   & 99 & 99 \\
%   2007 & 56 &   & 27 & 24 &   &   &   & 107 & 107 \\
%   2006 & 63 &   & 24 & 26 &   &   &   & 113 & 113 \\
%   2005 & 88 &   & 31 &   &   &   &   & 119 & 119 \\
%   2004 & 70 &   & 27 &   &   &   &   & 97 & 97 \\
%   2003 & 74 &   & 29 &   &   &   &   & 103 & 103 \\
%   2002 & 78 &   & 23 &   &   &   &   & 101 & 101 \\
%   2001 & 74 &   & 22 &   &   &   &   & 96 & 96 \\
%   2000 & 73 &   & 20 &   &   &   &   & 93 & 93 \\
%   1999 & 69 &   & 19 &   &   &   &   & 88 & 88 \\
%   1998 & 72 &   & 18 &   &   &   &   & 90 & 90 \\
%   1997 & 72 &   & 16 &   &   &   &   & 88 & 88 \\
%   1996 & 65 &   & 12 &   &   &   &   & 77 & 77 \\
%   1995 & 56 &   & 18 &   &   &   &   & 74 & 74 \\
%   1994 & 53 &   &   &   &   &   &   & 53 & 53 \\
%   1993 & 55 &   &   &   &   &   &   & 55 & 55 \\
%   1992 & 53 &   &   &   &   &   &   & 53 & 53 \\
%   1991 & 50 &   &   &   &   &   &   & 50 & 50 \\
%   1990 & 53 &   &   &   &   &   &   & 53 & 53 \\
%   \midrule
%   \textbf{sum} & \textbf{1545} & \textbf{9} & \textbf{632} & \textbf{310} & \textbf{50} & \textbf{123} & \textbf{25} & \textbf{2694} & \textbf{2546} \\
%   \bottomrule
%   \end{tabu}%
% \end{table}

% \subsection{Mezcal Head}

% Lorem ipsum dolor sit amet (see \autoref{fig:sample}), consetetur sadipscing elitr, sed diam
% nonumy eirmod tempor invidunt ut labore et dolore magna aliquyam erat,
% sed diam voluptua. At vero eos et accusam et justo duo dolores et ea
% rebum. Stet clita kasd gubergren, no sea takimata sanctus est Lorem
% ipsum dolor sit amet. Lorem ipsum dolor sit amet, consetetur
% sadipscing elitr, sed diam nonumy eirmod tempor invidunt ut labore et
% dolore magna aliquyam erat, sed diam voluptua. At vero eos et accusam
% et justo duo dolores et ea rebum. Stet clita kasd gubergren, no sea
% takimata sanctus est Lorem ipsum dolor sit amet. 

% \subsubsection{Duis Autem}

% Lorem ipsum dolor sit amet, consetetur sadipscing elitr, sed diam
% nonumy eirmod tempor invidunt ut labore et dolore magna aliquyam erat,
% sed diam voluptua. At vero eos et accusam et justo duo dolores et ea
% rebum. Stet clita kasd gubergren, no sea takimata sanctus est Lorem
% ipsum dolor sit amet. Lorem ipsum dolor sit amet, consetetur
% sadipscing elitr, sed diam nonumy eirmod tempor invidunt ut labore et
% dolore magna aliquyam erat, sed diam voluptua. At vero eos et accusam
% et justo duo dolores et ea rebum. Stet clita kasd gubergren, no sea
% takimata sanctus est Lorem ipsum dolor sit amet. Lorem ipsum dolor sit
% amet, consetetur sadipscing elitr, sed diam nonumy eirmod tempor
% invidunt ut labore et dolore magna aliquyam erat, sed diam
% voluptua. At vero eos et accusam et justo duo dolores et ea
% rebum. Stet clita kasd gubergren, no sea takimata sanctus est. Lorem
% ipsum dolor sit amet.

% \begin{figure}[tb]
%  \centering % avoid the use of \begin{center}...\end{center} and use \centering instead (more compact)
%  \includegraphics[width=\columnwidth]{paper-count-w-2015-new}
%  \caption{A visualization of the 1990--2015 data from \autoref{tab:vis_papers}. The image is from \cite{Isenberg:2017:VMC} and is in the public domain.}
%  \label{fig:sample}
% \end{figure}

% \subsubsection{Ejector Seat Reservation}

% Duis autem~\cite{Lorensen:1987:MCA}\footnote{The algorithm behind
% Marching Cubes \cite{Lorensen:1987:MCA} had already been
% described by Wyvill et al. \cite{Wyvill:1986:DSS} a year
% earlier.} vel eum iriure dolor in hendrerit
% in vulputate velit esse molestie consequat,\footnote{Footnotes
% appear at the bottom of the column.} vel illum dolore eu
% feugiat nulla facilisis at vero eros et accumsan et iusto odio
% dignissim qui blandit praesent luptatum zzril delenit augue duis
% dolore te feugait nulla facilisi. Lorem ipsum dolor sit amet,
% consectetuer adipiscing elit, sed diam nonummy nibh euismod tincidunt
% ut laoreet dolore magna aliquam erat volutpat.


% \paragraph{Confirmed Ejector Seat Reservation}

% Ut wisi enim ad minim veniam, quis nostrud exerci tation ullamcorper
% suscipit lobortis nisl ut aliquip ex ea commodo
% consequat~\cite{Nielson:1991:TAD}. Duis autem vel eum iriure dolor in
% hendrerit in vulputate velit esse molestie consequat, vel illum dolore
% eu feugiat nulla facilisis at vero eros et accumsan et iusto odio
% dignissim qui blandit praesent luptatum zzril delenit augue duis
% dolore te feugait nulla facilisi.

% \paragraph{Rejected Ejector Seat Reservation}

% Ut wisi enim ad minim veniam, quis nostrud exerci tation ullamcorper
% suscipit lobortis nisl ut aliquip ex ea commodo consequat. Duis autem
% vel eum iriure dolor in hendrerit in vulputate velit esse molestie

% \subsection{Vestibulum}

% Vestibulum ut est libero. Suspendisse non libero id massa congue egestas nec at ligula. Donec nibh lorem, ornare et odio eu, cursus accumsan felis. Pellentesque habitant morbi tristique senectus et netus et malesuada fames ac turpis egestas. Donec dapibus quam vel eros mattis, id ornare dolor convallis. Donec at nisl sapien. Integer fringilla laoreet tempor. Fusce accumsan ante vel augue euismod, sit amet maximus turpis mattis. Nam accumsan vestibulum rhoncus. Aenean quis pellentesque augue. Suspendisse sed augue et velit consequat bibendum id nec est. Quisque erat purus, ullamcorper ut ex vel, dapibus dignissim erat.

% Quisque sit amet orci quam. Lorem ipsum dolor sit amet, consectetur adipiscing elit. Aliquam pharetra, nunc non efficitur convallis, tellus purus iaculis lorem, nec ultricies dolor ligula in metus. Class aptent taciti sociosqu ad litora torquent per conubia nostra, per inceptos himenaeos. Aenean eu ex nulla. Morbi id ex interdum, scelerisque lorem nec, iaculis dui. Phasellus ultricies nunc vitae quam fringilla laoreet. Quisque sed dolor at sem vestibulum fringilla nec ac augue. Ut consequat, velit ac mattis ornare, eros arcu pellentesque erat, non ultricies libero metus nec mi. Sed eget elit sed quam malesuada viverra. Quisque ullamcorper, felis ut convallis fermentum, purus ligula varius ligula, sit amet tempor neque dui non neque. Donec vulputate ultricies tortor in mollis.

% Integer sit amet dolor sit amet turpis ullamcorper varius. Cras volutpat bibendum scelerisque. Maecenas mauris dolor, gravida eu elit et, sodales consequat tortor. Integer id commodo elit. Pellentesque sollicitudin ex non nulla molestie eleifend. Mauris sagittis metus nec turpis imperdiet, vel ullamcorper nibh tincidunt. Sed semper tempus ex, ut aliquet erat hendrerit id. Maecenas sit amet dolor sollicitudin, luctus nunc sit amet, malesuada justo.

% Mauris ut sapien non ipsum imperdiet sodales sit amet ac diam. Nulla vel convallis est. Etiam dapibus augue urna. Aenean enim leo, fermentum quis pulvinar at, ultrices quis enim. Sed placerat porta libero et feugiat. Phasellus ullamcorper, felis id porta sollicitudin, dolor dui venenatis augue, vel fringilla risus massa non risus. Maecenas ut nulla vitae ligula pharetra feugiat non eu ante. Donec quis neque quis lorem cursus pretium ac vulputate quam. Cras viverra tellus vitae sapien pretium laoreet. Pellentesque fringilla odio venenatis ex viverra, quis eleifend tortor ornare. Ut ut enim nunc. Vivamus id ligula nec est dignissim eleifend.

% Nunc ac velit tellus. Donec et venenatis mauris. Lorem ipsum dolor sit amet, consectetur adipiscing elit. Ut vitae lectus vel ante mollis congue. Vestibulum at cursus velit. Curabitur in facilisis enim. Vestibulum eget dui aliquet risus laoreet laoreet. Phasellus et est id magna interdum venenatis. Donec luctus vehicula justo sed laoreet. Quisque tincidunt suscipit augue, in molestie sem accumsan sed.
% \section{Conclusion}

% Lorem ipsum dolor sit amet, consetetur sadipscing elitr, sed diam
% nonumy eirmod tempor invidunt ut labore et dolore magna aliquyam erat,
% sed diam voluptua. At vero eos et accusam et justo duo dolores et ea
% rebum. Stet clita kasd gubergren, no sea takimata sanctus est Lorem
% ipsum dolor sit amet. Lorem ipsum dolor sit amet, consetetur
% sadipscing elitr, sed diam nonumy eirmod tempor invidunt ut labore et
% dolore magna aliquyam erat, sed diam voluptua. At vero eos et accusam
% et justo duo dolores et ea rebum. Stet clita kasd gubergren, no sea
% takimata sanctus est Lorem ipsum dolor sit amet. Lorem ipsum dolor sit
% amet, consetetur sadipscing elitr, sed diam nonumy eirmod tempor
% invidunt ut labore et dolore magna aliquyam erat, sed diam
% voluptua. At vero eos et accusam et justo duo dolores et ea
% rebum.


%% if specified like this the section will be committed in review mode
\acknowledgments{
This work was supported by NAVER Corporation (Cloud Data Box), by the National Research Foundation of Korea (NRF) grant funded by the Korean government (MSIT) (No. 2023R1A2C200520911), and by the National Science Foundation under grant No. 2127309 to the Computing Research Association for the CIFellows project, NSF IIS-2046725, NSF IIS-1764089, NSF IIS-2320920, NSF IIS-2233316, and NSF III-2316496.
% and by the National Science Foundation (NSF III-2316496).
The authors wish to thank Sungbok Shin, Yun-Hsin Kuo, Seokhyeon Park and SNU-HCIL / UNC-VisuaLab members for their valuable feedback. We would also like to appreciate Micha\"el Aupetit for providing ClustMe dataset.}

%\bibliographystyle{abbrv}
\bibliographystyle{abbrv-doi}
%\bibliographystyle{abbrv-doi-narrow}
%\bibliographystyle{abbrv-doi-hyperref}
%\bibliographystyle{abbrv-doi-hyperref-narrow}

\bibliography{ref}
\end{document}

