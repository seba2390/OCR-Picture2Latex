\section{Background and Related Work}
We survey past work about visual clustering in scatterplots,  visual quality measures, and perceptual variability. 
%Based on such backgrounds, we briefly discuss the 
These works collectively illustrate the necessity of \measure.

% \fix{Need to change the verbatum}
\subsection{Cluster Perception in Scatterplots}

\label{sec:cps}


%With the 
Given the 
% number of 
broad applications for visual clustering,
%in practice,
%and attention in the research area, 
several previous works have concentrated on understanding and modeling cluster perception.
% 
For example, eye-tracking 
%is used to 
was used to detect what aspects of a dataset people use for
%analyze user perception in
cluster identification, highlighting the role of Gestalt principles, especially proximity and closure~\cite{etemadpour2014eye}. 
% Matute et al.\ proposed to quantify and represent scatterplots through skeleton-based descriptors, including the ones for cluster structure, mimicking human-perceived scatterplot similarity~\cite{matute2017skeleton}. 
ScatterNet captures perceptual similarities between scatterplots to emulate human clustering decisions~\cite{ma2018scatternet}. 
% However, as  a deep learning model, ScatterNet lacks explainability
%The Scagnostics technique focused on identifying 
Scagnostics identify scatterplot patterns, including cluster structure~\cite{dang2014transforming}, but 
%Pandey et al.\ later showed they do 
cannot reliably reproduce human perception~\cite{pandey2016towards}. 
In ClustMe~\cite{abbas19cgf},
%study, 
Abbas et al. 
%focused on modeling 
\rev{computationally} modeled \rev{human perception in judging the} complexity of the cluster structure in scatterplots, which affects by the number of clusters and the nontriviality of patterns.
% ClustMe performed well in reproducing human decisions for clustering patterns.  
Quadri \& Rosen studied various factors that influence the perception of clusters. Relying on these factors, they built explainable models of how humans perceive cluster separation based on merge tree data structures from topological data analysis~\cite{quadri21tvcg}.

These studies aim to 
%understand the scatterplot's 
characterize cluster perception processes, 
%what 
the factors that influence cluster perception, and the relation between these factors and scatterplot design. However, 
%the 
visual clustering in scatterplots 
%comes with 
lacks any ground truth; we cannot always categorically determine which group of points forms a cluster. 
%, and u
Underlying data characteristics lead to ambiguity in cluster structure (i.e., cluster ambiguity), causing intrinsic variability in 
%user analysis.
the clusters people detect.
%In our study, w
We thus develop a VQM that automatically estimates cluster ambiguity
for a wide range of cluster patterns. We then explore and rank the cluster ambiguity of scatterplots.

%%%%%%%%%%%%%%%%%%%%%%%%%%%%%%%%%%%%%%%%%%%%%%%%%%%%%%%%%%%%%%%    % 
\subsection{Visual Quality Measures on Cluster Perception}

\label{sec:vqmcp}

% Abbas et al. introduced ClustMe, a visual quality measure to rank scatterplots~\cite{abbasclustme} by estimating the perceived complexity of cluster structure. 

The visualization community has proposed and developed various \textit{Visual Quality Measures} (VQMs)~\cite{bertini2011quality, behrisch2018quality} 
that metricize the specific characteristics of cluster patterns in scatterplots.
These metrics enable analysts to rapidly evaluate their scatterplots in terms of supporting general pattern exploration \cite{brehmer2013multi} or a specific perceptual task \cite{amar2005low}, and can also be used to optimize scatterplots\cite{quadri22tvcg}. 
%of diverse scatterplots, and moreover to optimize scatterplots, generating a numeric representation that properly reflects the characteristics of scatterplots is essential.
%Here, we list and describe VQMs that specifically focus on cluster patterns. 
% 

A na{\"i}ve approach in designing VQMs is to reuse existing algorithms (e.g., clustering techniques), 
%. This is done 
where the performance of the algorithm is evaluated against the 
% by evaluating the algorithms' performance in reproducing 
%the visual clustering results.
results of human experiments.
Aupetit et al.\ investigated how well clustering techniques can mimic (i.e., model) visual clustering by testing the extent to which 1,400 variants of clustering techniques can reproduce the human separability judgments~\cite{aupetit19vis}. 
Sedlmair \& Aupetit proposed a framework for evaluating class separability measures based on human-judged class separability data and 
%tested
used the framework to test 15 class separation measures~\cite{sedlmair2015data}.
%based on the framework~\cite{sedlmair2015data}. 
The framework was later extended to test a large set of class separability measures, which were constructed as combinations of 17 neighborhood graph functions and 14 class purity functions~\cite{aupetit2016sepme}. However, reusing existing algorithms does not directly reflect human perception, thus often performing poorly in imitating human perception~\cite{abbas19cgf} (\autoref{sec:mainstudy}).

%VQMs that mainly target cluster perception were thus developed. 
VQMs can be instead designed to target a specific pattern or a perceptual task. For example, 
%in the aforementioned study made by 
Quadri \& Rosen~\cite{quadri21tvcg}
%, they 
developed a threshold plot based on their modeling of cluster perception.
Threshold plots tell the extent to which clusters are salient in an input scatterplot.
%Later,  an optimization model using threshold plots as a saliency measure is proposed to 
Designers can use these plots to 
%select an optimal design to 
enhance the effectiveness and efficiency of a scatterplot for visual clustering~\cite{quadri22tvcg}. 
ClustMe~\cite{abbas19cgf} also produced a VQM
%, which is also called ClustMe, 
simulating human perception in judging the complexity of cluster patterns. Specifically, ClustMe models human perception based on 
%perceptual data on 
experimental data on perceived cluster separability judged by multiple subjects. 
%, rather than simulating it in a purely algorithmic way. In our study, 
We 
%borrowed both such a
apply this modeling approach and the perceptual data from ClustMe to guide our model.
However, ClustMe regards the judgments of multiple subjects as a binary decision (i.e., separated or not separated), estimating a \textit{general} complexity perceived. In contrast,
% however, unlike the binary (i.e., separated or not separated) model from ClustMe, 
%study that modeled cluster separability in a binary manner (i.e., separated or not separated), we tried to 
\measure summarizes the judgments made by subjects as a continuous probabilistic function for the sake of estimating the \textit{variability} of the judgments  (i.e., cluster ambiguity).
\rev{Estimating variability helps us understand the distribution of human visual percepts, enabling related applications to more accurately reflect perception.}
% by training a regression model
%, for the sake of precise estimation of cluster ambiguity.
% in measuring cluster ambiguity: a more fine-grained VQM for applications where...





\subsection{Perceptual Variability}
% 
Due to perceptual variability among individuals, the perception of clusters can differ even with the same scatterplot.  Perceptual variability refers to individuals’ ``traits or stable tendencies to respond to certain stimuli or situations in predictable ways''~\cite{dillon1996user}. Prior work on perceptual variability demonstrated that people exhibit observable differences in task-solving and behavioral patterns~\cite{woolhouse2000personality}. 
%An abundance of work from 
Psychology and vision research 
%especially showed 
have shown that 
%the 
people exhibit substantial variability 
%in 
%user 
%people's responses 
on tasks such as pattern identification, judgments, or adjustments~\cite{hoskin2019sensitivity, huang2010distortions, ons2011subjectively}. 
%With growing interest in extending these findings and 
Given the significance of perceptual variability in information visualization~\cite{davis2022risks,ziemkiewicz2009preconceptions}, 
%a recent work developed a comprehensive survey~\cite{liu2020survey} on ongoing research that studies the impact of individual differences on data visualization.
 recent research explores when, where, and how perceptual variability influences visualization use (see Liu et al.~\cite{liu2020survey} for a survey). 



\begin{figure*}
    \centering
    \includegraphics[width=\linewidth]{figures/pipeline.pdf}
    \vspace{-6mm}
    \caption{The 
    %illustration of the 
    \measure pipeline (\autoref{sec:pipeline}). (Step 1) Given a scatterplot, we first apply a Gaussian Mixture Model (GMM) to abstract 
    a scatterplot into a set of Gaussian components. (Step 2) We estimate component-pairwise separability scores by applying a predefined regression module (\autoref{sec:regmodel}) and convert the scores into local ambiguity scores by applying the binary Shannon entropy function. (Step 3) Finally, we obtain the cluster ambiguity score of the input scatterplot by averaging the local ambiguity scores.
    \vspace{-4mm}
    }
    \label{fig:pipeline}
\end{figure*}
%

Perceptual variability can influence the reliability of any generalized model of perception~\cite{zaman21pbr}. 
%In their work on the ``
The Axiom of Perceptual Variability
%'', Ashby and Lee suggested
suggests that the success of such generalized perceptual theories depends on their ability to account for variance in perceptual representations~\cite{ashby1993perceptual}. 
Moreover, the perceptual variability of a certain task may degrade the credibility of the applications relying on visual clustering (\autoref{sec:intro}).
Visual clustering is an ill-posed problem, where 
%the 
there is no "ground truth" for clusters 
%come with the 
 (i.e., it is not always possible to determine a ``correct'' clustering). 
% Several factors play a role in how they are perceived~\cite{quadri2020modeling, sedlmair2012taxonomy, xia21cgna}. 
Generalized models and applications of visual clustering are thus likely to be more vulnerable to perceptual variability. We believe that our modeling of perceptual variability in visual clustering (i.e., cluster ambiguity) not only enhances our understanding of cluster perception but also resolves \rev{the} vulnerabilities \rev{of such models and applications}. 


% . Human perception combined with inherent characteristics of datasets generates variability in cluster identification or cluster structure perception. Also, based on our interviews and preliminary studies, people's concept of cluster ambiguity varies. 


% % 
% Modeling the perceptual variability of cluster separation in the scatterplot shows how perceptual variability can influence generalized responding of visual clustering. 
% % 



% \subsection{Why CLAMS?}
% Bring together cluster percetion, VQM and Individual difference to demonstrate the need of CLAMS
% 
% Previous perceptual models and VQMs focused on modeling the cluster separation and identification of clusters in scatterplots. Additionally, VQMs such as ClustMe~\cite{abbasclustme}, and ThresholdPlot~\cite{quadri2020modeling} proposed measures to predict human judgments of cluster count/structure in the visualized data. However, the individual differences among the visual analyst, users, and viewers demonstrate observed differences on tasks (i.e., cluster identification). The individual difference leads to intrinsic variability in cluster structure among people, called \textit{cluster ambiguity}. Therefore, we need a proxy for human perception that measures and predicts the ambiguity level in the data visualized on the scatterplot. In our proposed \measure - \textit{\textbf{C}luster \textbf{A}mbiguity \textbf{M}easure}, we established a novel VQM to automatically estimate the cluster ambiguity of the monochrome scatterplots for wide range of datasets. 

% Cluster perception plays a key role in conducting visual clustering, which is a key task that reveal data characteristics and allow further exploration of underlying data patterns~\cite{sarikaya2018scatterplots, sarikaya2018design}.
% % 
% Recent studies have thus tried to understand and model the cluster perception in scatterplots. 
% % 
% Aupetit et al.\ investigated how well clustering algorithms cand mimic (i.e., model) cluster perception by testing how well 1,400 variants of clustering techqniues captures the aspects of human perception~\cite{abbas19vis}. 
% % 
% % Matute et al. proposed to quantify and represent scatterplots through skeleton-based descriptors measuring scatterplot similarity~\cite{matute2017skeleton} and showed that the modeling well matches with human perception. 
% % Sedlmair and Aupetit's approach mimic human judgment of cluster separation by using machine learning on 15 class separation measures on scatterplots~\cite{sedlmair2015data}. 
% % 
% ScatterNet, a deep learning model, captures perceptual similarities between scatterplots that could be used to emulate human clustering decisions~\cite{ma2018scatternet}. Scagnostics focused on identifying patterns in scatterplots, including clusters~\cite{dang2014transforming}, but Pandey et al.\ later showed they do not reliably reproduce human judgments~\cite{pandey2016towards}. 
% % 
% Recently, Quadri and Rosen studied and demonstrated cluster perception variability on visual encoding, and data aspects using topological data structure\cite{quadri2020modeling}, which was further prototyped into automatic optimization of cluster structure~\cite{quadri21tvcg}. 

% \textbf{Taxonomies of Clustering Factors} \fix{Should be removed. Seems less relevant} Identifying clusters is directly influenced by the perception of cluster separation, and much of our understanding has come from studying dimension reduction (DR) techniques.  Lewis et al.\ compared the effectiveness of DR techniques using human judgments and concluded that T-SNE performs better than other commonly used methods when expecting clusters in the data~\cite{lewis2012behavioral}.  Etemadpour et al.\ showed, however, the performance of DR techniques also depends on data characteristics~\cite{etemadpour2014perception}, e.g., the separability of clusters, and later created a user-centric taxonomy of visual tasks related to clustering in DR techniques~\cite{etemadpour2015user}.  A taxonomy of visual cluster separation in scatterplots used a qualitative evaluation to identify 4 important factors---scale, point distance, shape, and position~\cite{sedlmair2012taxonomy}. The taxonomy gives a context to our visual factor selection. Sedlmair and Aupetit later evaluated 15 class separation measures for assessing the quality of DR using human input for building a machine learning framework~\cite{sedlmair2015data} and later extended the framework to include an even greater number of measures~\cite{aupetit2016sepme}.