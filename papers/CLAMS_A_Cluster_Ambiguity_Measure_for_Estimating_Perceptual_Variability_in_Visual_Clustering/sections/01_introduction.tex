

Clustering is a key analytic task in scatterplots~\cite{quadri21tvcg, xia22tvcg, xia21cgna}.
It occurs when we infer structure in data by identifying groups (i.e., clusters) based on the pairwise proximity between data points~\cite{amar2005low}.
Visual clustering occurs when people infer these groups visually, such as finding neighborhoods of points in a scatterplot.
It is among the most common perceptual tasks people conduct with scatterplots~\cite{sarikaya2018scatterplots}.
% 
Diverse domains, such as bioinformatics~\cite{shi10cbm, shannon03phar} and machine learning~\cite{kahng18tvcg}, leverage visual clustering for data analysis. These applications include the identification of ground truth clusters for benchmarking data mining techniques, such as automatic clustering~\cite{aupetit19vis, xia21cgna} and dimensionality reduction techniques~\cite{etemadpour2014perception, xia22tvcg}.
% 


% We introduced an approach to help people quickly and easily identify and rank the \textit{cluster ambiguity} of scatterplots, which stands for the intrinsic perceptual variability in visual clustering. 
While visual clustering supports a range of applications, 
%these applications can be obstructed due to the \textit{cluster ambiguity} of scatterplots, which stands for the 
\textit{cluster ambiguity}---the intrinsic perceptual variability in visual clustering due to unclear cluster boundaries---can introduce uncertainty or even errors in applications relying on visual clustering. 
% 
% Cluster ambiguity occurs as people have different expectations of what constitutes a cluster boundary~\cite{quadri21tvcg}. 
For example, 
cluster ambiguity can make data analysis unreliable. If a scatterplot is highly ambiguous,
it is easy to make multiple conclusions about cluster structure within the data. 
% 
% multiple /analysts 
%are likely to 
% will likely reach significantly different conclusions about the cluster structures in the data, which costs an additional consensus process. 
% 
%Regarding benchmarking studies, 
Ambiguity also reduces the reliability of establishing ground truth clusters based on human perception for benchmarking data mining techniques: if each person perceives cluster structure differently, we cannot know which structure is ``correct.''
%how can a single ground truth be established?
% To properly deal with such threats of high cluster ambiguity, we
% Therefore, cluster ambiguity should be revised, filtered, or at least informed prior to data analysis.
% Regarding benchmarking unsupervised learning techniques, ambiguity makes us hardly believe the ground truth clusters set by human perception: if each person perceives cluster structure differently, how can a single ground truth be established?

%Despite such challenges of cluster ambiguity, studies and modelings 
However, most studies and models of visual clustering focus on how people perceive clusters in general~\cite{abbas19cgf, xia21cgna, wertheimer22pf}. For example, Gestalt principles~\cite{wertheimer22pf} explain how people ``commonly'' or ``generally'' group visual objects within a complex scene~\cite{pinna2010new}. 
Studies examining average cluster perception support visualization designers in building effective systems 
that most people can use~\cite{lu20tvcg, wang19tvcg, nonato19tvcg}. However, they do not offer solutions to deal with the challenges in data analysis and benchmarking imparted by cluster ambiguity. Designers thus need a more systematic way to assess cluster ambiguity.


A reliable method to evaluate the ambiguity of a scatterplot is directly measuring perceptual variability via human experiments~\cite{hartwig2023clusternet}.
% We can ask people to perform visual clustering and examine the extent to which their results vary.
However, this process is costly and not scalable.
As an alternative, we can use clustering techniques to mimic human perception \cite{aupetit2016sepme, sedlmair2015data} (\autoref{sec:vqmcp}). 
Assuming that a clustering technique with a specific hyperparameter setting represents a human’s perception, this method mimics human variability by running the technique under various hyperparameter settings.
 This approach is scalable but at the cost of 
 % sacrificing 
accuracy due to the lack of 
%involving 
human input (\autoref{sec:mainstudy}).


% Another strategy is to use clustering techniques' abilities to capture clusters as a proxy for human cluster perception~\cite{aupetit2016sepme, sedlmair2015data} (\autoref{sec:vqmcp}). 
% For example, we can check the extent to which clustering results vary due to hyperparameter settings to mimic human variability. The strategy is based on the notion that a technique ``perceives'' cluster structure differently depending on hyperparameter settings. Each hyperparameter setting mimics an individual participant in a user study. 
% This automation is scalable; however, it does not take into account human perceptual variability
% and fails to accurately estimate cluster ambiguity .


This research presents a scalable and accurate method to evaluate cluster ambiguity through a visual quality measure (VQM) called \textit{\measure}. 
We design \measure based on a dataset gathered from human input about the separability of clusters~\cite{abbas19cgf}. 
We construct \measure in two steps: first, we conduct a user study to investigate important factors that influence visual clustering. 
Second, based on the study findings, we train a regression module estimating how human subjects separate clusters.
Given a scatterplot, \measure computes the separability of every pair of identified clusters using the regression module. 
The measure then aggregates the computed pairwise separabilities of clusters to predict how ambiguously the clusters are portrayed by human subjects. 

% a visual quality measure (VQM) called \textit{\measure} that uses a statistical model derived from perceptual data to automatically assess cluster ambiguity in monochrome scatterplots.
% % 
% Our metric is developed using a dataset from a past study ~\cite{abbas19cgf} that
% %; the dataset 
% represents scatterplot cluster separability judgements 
% %of scatterplots 
% made by multiple subjects.
% % that represents human-judged cluster separability in scatterplots, which was obtained from a user study involving 34 participants~\cite{abbas19cgf}. 
% %In this context, cluster ambiguity is determined by 
% We compute cluster ambiguity as the degree of variability in separability judgments among subjects.
% %The dataset is used 
% We use this dataset to build a regression module that estimates the cluster separability judgement on an input scatterplot. We use this module to compute ambiguity using features derived in a qualitative study with ten participants completing 240 total separability judgments. 
% %made by multiple human subjects.
% %We conducted a qualitative user study to reveal important factors related to visual clustering, where our model is trained relying on a feature engineering based on the study findings. 
% Utilizing the module, \measure accurately predicts the potential variability of visual clustering among human subjects as a proxy for cluster ambiguity (\autoref{sec:mainstudy}).

Our quantitative experiments show that \measure is more accurate than existing models in predicting cluster ambiguity.
First, an ablation study verifies the accuracy of the regression module, validating our user-study-driven approach in constructing the module.
Moreover, we find that the ranking of scatterplots set by \measure has a strong correlation with the ground truth ranking constructed from 20 participants in our study.
Furthermore, \measure outperforms an average human annotator in estimating cluster ambiguity.
We also present two applications of \measure in optimizing and benchmarking data mining algorithms. 
First, we propose \ambreducer, an optimization system that reduces the ambiguity of dimensionality reduction embeddings while maintaining accuracy.
\ambreducer helps analysts effectively interpret high-dimensional data by informing cluster ambiguity. 
Second, we show how our measure can help select reliable benchmark datasets for  
comparing different clustering techniques. 
Findings from our experiments and applications open up discussions on leveraging perceptual variability in 
% future applications and 
visualization research. 