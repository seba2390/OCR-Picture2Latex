\documentclass[a4paper,10pt]{amsart}
\usepackage[utf8]{inputenc}
\usepackage{indentfirst, amsfonts, amsmath, amsthm, amssymb, amscd}
\usepackage{amsmath,amsfonts,amscd,bezier}
\usepackage{graphicx}
\usepackage{color}

\usepackage{hyperref}
%  \usepackage{refcheck}
\usepackage[normalem]{ulem}

\usepackage{thmtools}
\usepackage{thm-restate}

\newtheorem{maintheorem}{Theorem}
\renewcommand{\themaintheorem}{\Alph{maintheorem}}
% \newtheorem{maintheorem}{Theorem}
% \renewcommand{\themaintheorem}{}
\newtheorem{maincorollary}[maintheorem]{Corollary}
\newtheorem{mainlemma}[maintheorem]{Lemma}
\newtheorem{theorem}{Theorem}[section]
\newtheorem{conjecture}[theorem]{Conjecture}
\newtheorem{corollary}[theorem]{Corollary}
\newtheorem{proposition}[theorem]{Proposition}
\newtheorem{lemma}[theorem]{Lemma}
\newtheorem{definition}[theorem]{Definition}

\newtheorem{Sublemma}[theorem]{Sublemma}
\newtheorem{example}{Example}
\theoremstyle{remark}
\newtheorem{remark}[theorem]{Remark}
\newtheorem{problem}{Problem}
\newtheorem{question}{Question}

\newtheorem{pensar}{PENSAR}

\newcommand{\tco}{\textcolor}

%opening
\title[Measure Rigidity and Disintegration]{Measure Rigidity and Disintegration: \\ Time-one map of flows.}
\author{Gabriel Ponce}
\address{Departamento de Matem\'atica, Estat\'istica e Computa\c c\~ao Cient\'ifica,
  IMECC-UNICAMP, Campinas-SP, Brazil.}
  \email{gaponce@ime.unicamp.br}
\author{R\'egis Var\~{a}o} 
\address{Departamento de Matem\'atica, Estat\'istica e Computa\c c\~ao Cient\'ifica,
  IMECC-UNICAMP, Campinas-SP, Brazil.}% {\&  Department of Mathematics, University of Chicago, Chicago, USA.}}
\email{regisvarao@ime.unicamp.br}


\begin{document}

\maketitle


\begin{abstract}
An invariant measure for a flow is, of course, an invariant measure for any of its time-t maps. But the converse is far from being true. Hence, one may naturally ask: What is the obstruction for an invariant measure for the time-one map to be invariant for the flow itself? We give an answer in terms of measure disintegration. Surprisingly all it takes is the measure not to be ``too much pathological in the orbits". We prove the following rigidity result. If $\mu$ is an ergodic probability for the time-one map of a flow, then it is either highly pathological in the orbits, or it is highly regular (i.e invariant for the flow). In particular this measure rigidity result is also true for measurable flows by the classical Ambrose-Kakutani's representation theorem for measurable flows.
\end{abstract}


\setcounter{tocdepth}{1}
\tableofcontents


\section{Introduction}

A basic question answered on an introductory ergodic theory course is that we may always find invariant probability measures for many dynamical systems. Two particularly large classes of such dynamical systems are homeomorphisms and continuous flows on a compact manifold. We say that a measurable map $h:X \rightarrow X$ preserves a probability $\mu$ if for every measurable set $A \subset X$, then $\mu(h^{-1}(A))=\mu(A)$. If $\phi:\mathbb R \times X \rightarrow X$ is a flow, we say that the flow $\phi$ preserves the measure $\mu$ if it is preserved for every time $t$-map $\phi_t:=\phi(t,.)$. We say that a measure $\mu$ is ergodic for a certain dynamical system if any invariant set has either full measure or zero measure. It is not at all expected that an ergodic probability for the time-$t$ map to be ergodic (in particular invariant) for the flow itself. Hence a natural question arises:


\textbf{Question:} \textit{What is the obstruction for an ergodic measure for the time-one map to be ergodic (in particular invariant) for the flow itself?} 

To the best of our knowledge, even though this seems to be a natural question it has not been treated in the literature. We are able to give a precise answer to this question in terms of measure disintegration. Surprisingly all it takes is the measure not to be ``too much pathological in the orbits". That is, we prove a measure rigidity result. If $\mu$ is an ergodic probability for the time-one map of a flow, then it is either highly pathological in the orbits, or it is highly regular (i.e invariant for the flow). This is our main result:


\begin{maintheorem}\label{theo:continuous.flow}
Let $X$ be a separable metric space and $\phi:\mathbb R \times X \rightarrow X$ be a continuous flow. Denote by $\mathcal F=\{\mathcal F(x)\}_{x\in X}$ the foliation of $X$ by orbits of the flow $\phi$. Given any ergodic Borel measure $\mu$ for the time$-1$ map $\phi_1:= \phi(1,\cdot):X\rightarrow X$, then 
\begin{enumerate}
\item  either there is a set $A \subset X$ of full $\mu$-measure such that $\mathcal F(x) \cap A$ is a discrete set of $\mathcal F(x)$. 
Moreover, there is a natural number $k\geq 1$ such that $\mathcal F(x) \cap A$ is the $\phi_1-$orbit of exactly $k$ points; or
\item for $\mu$-almost every $x\in X$ there is a measure $\mu_{\mathcal F(x)}$ on $\mathcal F(x)$ such that
% given any arc $[x,\phi_t(x)] \subset \{\phi_t(x):t \in \mathbb R\}$ we have 
\[ \mu_{\mathcal F(x)}(\phi([0,t] \times \{x\})) = 2^{-1}t, \]
as long as $\tau \mapsto \phi(\tau,x)$ is injective on $[0,t]$ and where $\mu_{\mathcal F(x)}$ is a $\phi_1$-invariant measure which normalized and restricted on a foliated chart of the orbit is a disintegration of $\mu$. In particular $\mu$ is invariant for the flow.
% And $\mu_y=\mu_x$ if $y \in \mathcal F(x)$.
% $[\mu_x]$ such that $\mu^1_x((T_{-1}(x), T_1(x)))=1$.
\end{enumerate}
\end{maintheorem}




 In \cite{Ambrose, AmbroseKakutani} W. Ambrose and S. Kakutani proved a remarkable representation theorem for measurable flows which can be summarized as follows: every measurable  measure preserving flow on a Lebesgue space is isomorphic to a flow built under a function (see \cite{AmbroseKakutani} for definition). This result was latter strengthened and extended to larger classes of measurable flows (e.g. non-singular flows) by D. Rudolph \cite{Rudolph}, S. Dani \cite{Dani}, U. Krengel \cite{Krengel, Krengel2} and I. Kubo \cite{Kubo}. In \cite{Wagh} V. Wagh gave a descriptive version of Ambrose-Kakutani's theorem and more recently D. McClendon \cite{McClendon} proved a version of Ambrose-Kakutani's theorem for Borel countable-to-one semi-flows.
 
 \begin{corollary}\label{theorem:A}
Let $\phi$ be a measurable flow defined on a Lebesgue space $X$ and $\mathcal F(x)$ be the $\phi-$orbit of the point $x$. Then, given any $\phi_1-$ergodic invariant measure either
\begin{enumerate}
\item  there is a set $A \subset X$ of full $\mu$-measure such that $\mathcal F(x) \cap A$ is a discrete subset of $\mathcal F(x)$. 
Moreover, there is a natural number $k\geq 1$ such that $\mathcal F(x) \cap A$ is the $\phi_1-$orbit of exactly $k$ points; or
\item for $\mu$-almost every $x\in X$ there is a measure $\mu_{\mathcal F(x)}$ on $\mathcal F(x)$ such that
% given any arc $[x,\phi_t(x)] \subset \{\phi_t(x):t \in \mathbb R\}$ we have 
\[ \mu_{\mathcal F(x)}(\phi([0,t] \times \{x\})) = 2^{-1}t, \]
as long as $\tau \mapsto \phi(\tau,x)$ is injective on $[0,t]$, $t\geq 0$, and where $\mu_{\mathcal F(x)}$ is a $\phi_1-$invariant measure which normalized and restricted on a foliated chart of the orbit is a disintegration of $\mu$. 
%And $\mu_y=\mu_x$ if $y \in \mathcal F(x)$.
% $[\mu_x]$ such that $\mu^1_x((T_{-1}(x), T_1(x)))=1$.
\end{enumerate}
 \end{corollary}


This corollay follows as a direct consequence of the classical Ambrose-Kakutani's Theorem (see Theorem ~\ref{theorem:ambrose.kakutani} from Subsection ~\ref{subsec:Mflows}).

\section{Preliminaries on measure theory}\label{sec:preliminaries}

\subsection{Measurable partitions and Rohklin's Theorem}


Let $(X, \mu, \mathcal B)$ be a probability space, where $X$ is a compact metric space, $\mu$ a probability measure and $\mathcal B$ the Borelian $\sigma$-algebra of $X$.
Given a partition $\mathcal P$ of $X$ by measurable sets, we associate the probability space $(\mathcal P, \widetilde \mu, \widetilde{\mathcal B})$ by the following way. Let $\pi:X \rightarrow \mathcal P$ be the canonical projection, that is, $\pi$ maps a point $x$ of $X$ to the partition element of $\mathcal P$ that contains it. Then we define $\widetilde \mu := \pi_* \mu$ and 
% $ \widetilde{\mathcal B}:= \pi_*\mathcal B$, therefore 
 $\widetilde B \in \widetilde{\mathcal B}$ if and only if $\pi^{-1}(\widetilde B) \in \mathcal B$.

\begin{definition} \label{definition:conditionalmeasure}
 Given a partition $\mathcal P$. A family $\{\mu_P\}_{P \in \mathcal P} $ is a \textit{system of conditional measures} for $\mu$ (with respect to $\mathcal P$) if
\begin{itemize}
 \item[i)] given $\phi \in C^0(X)$, then $P \mapsto \int \phi \mu_P$ is measurable;
\item[ii)] $\mu_P(P)=1$ $\widetilde \mu$-a.e.;
\item[iii)] if $\phi \in C^0(X)$, then $\displaystyle{ \int_X \phi d\mu = \int_{\mathcal P}\left(\int_P \phi d\mu_P \right)d\widetilde \mu }$.
\end{itemize}
\end{definition}

When it is clear which partition we are referring to, we say that the family $\{\mu_P\}$ \textit{disintegrates} the measure $\mu$ or that it is the \textit{disintegration of $\mu$ along $\mathcal P$}.  

\begin{proposition} \label{prop:uniqueness} \cite{EW, Ro52} 
 Given a partition $\mathcal P$, if $\{\mu_P\}$ and $\{\nu_P\}$ are conditional measures that disintegrate $\mu$ on $\mathcal P$, then $\mu_P = \nu_P$ $\widetilde \mu$-a.e.
\end{proposition}

%\begin{corollary} \label{cor:same.disintegration}
% If $T:M \rightarrow M$ preserves a probability $\mu$ and the partition $\mathcal P$, then  $T_*\mu_P = \mu_{T(P)}, \widetilde \mu$-a.e.
%\end{corollary}
%\begin{proof}
% It follows from the fact that $\{T_*\mu_P\}_{P \in \mathcal P}$ is also a disintegration of $\mu$.
%\end{proof}

\begin{definition} \label{def:mensurable.partition}
We say that a partition $\mathcal P$ is measurable (or countably generated) with respect to $\mu$ if there exist a measurable family $\{A_i\}_{i \in \mathbb N}$ and a measurable set $F$ of full measure such that 
if $B \in \mathcal P$, then there exists a sequence $\{B_i\}$, depending on $B$, where $B_i \in \{A_i, A_i^c \}$ such that $B \cap F = \bigcap_i B_i \cap F$.
\end{definition}




\begin{theorem}[Rokhlin's disintegration \cite{Ro52}] \label{theo:rokhlin} 
 Let $\mathcal P$ be a measurable partition of a compact metric space $X$ and $\mu$ a Borel probability. Then there exists a disintegration of $\mu$ along $\mathcal P$.
\end{theorem}


\subsection{Souslin Theory} 

We list some basic properties of Souslin sets. All the results cited here can be found in  \cite[Chapter $6$]{BogachevII}.

\begin{definition}
Given a Hausdorff space $X$, a subset $A \subset X$ is called Souslin if it is the image of a complete separable metric space under a continuous mapping. We say that the Hausdorff space $X$ is a Souslin space if it is a Souslin set. By convention, we define the empty set to be a Souslin set.
\end{definition}

Observe that by definition, if $X$ and $Y$ are Hausdorff spaces and $A \subset X$ is Souslin, then given any continuous function $f:X \rightarrow Y$, the image $f(A) \subset Y$ is Souslin. Given Souslin spaces $X$ and $Y$, the product $X\times Y$ is a Souslin space and the images of a Souslin set $A\subset X\times Y$ by the projections $\pi_1:X\times Y \rightarrow X , \pi_2:X\times Y \rightarrow Y$ are Souslin sets. Notice that the image of a Borel sets even by a well behaved continuous function such as the projection may not be a Borel set. In fact this was result of a classical mistake committed by Lebesgue \cite{Lebesgue} and corrected by Souslin \cite{Souslin}.

% as the projection does  do not have these properties, the projection of a Borel set may not be a Borel setminus
% One of the advantages of working with Souslin sets instead of Borel sets in our setting is that the previous property is not true for Borel sets, that is, the image of a Borel set by a projection is not necessarily Borel.


%(\cite[Corollary $6.6.7$]{BogachevII}) (\cite[Theorem $6.7.3$]{BogachevII})
Every Borel subset of a Souslin space is itself a Souslin space. Also, Souslin sets of Souslin spaces are preserved under Borel maps, that is, given $X$ and $Y$ Souslin spaces, $f:X\rightarrow Y$ a Borel map and $A\subset X$, $B\subset Y$ Souslin sets, then $f(A)$ and $f^{-1}(B)$ are Souslin sets. Although the complement of a Souslin set may not be a Souslin set, if the base space is Hausdorff and the complement of a Souslin set is a Souslin set it turns out that the original set is in fact a Borel set. Another interesting property of Souslin sets is that they are universally measurable sets.

% It is not hard to show (see Lemma $6.6.5$, Theorem $6.6.6$ and Corollary $6.6.7$ of \cite{BogachevII}) that every Borel subset of a Souslin space is itself a Souslin space. Also, Souslin sets of Souslin spaces are preserved under Borel maps (Theorem $6.7.3$ in \cite{BogachevII}) , that is, given $X$ and $Y$ Souslin spaces, $f:X\rightarrow Y$ a Borel map and $A\subset X$, $B\subset Y$ Souslin sets, then $f(A)$ and $f^{-1}(C)$ are Souslin sets. 
% Souslin sets are universally measurable. \textcolor{red}{Procurar ref. no boga. pra isso}

\subsection{Measurable flows} \label{subsec:Mflows}
Let $(X,\mathcal B, \mu)$ be a Lebesgue space. 
%A Borel map $T:X\rightarrow X$ is a nonsingular transformation of $X$ if for any $A \in \mathcal B$, $\mu(T^{-1}(A))=0 \Leftrightarrow \mu(A)=0$. If $\mu(A) = \mu(T^{-1](A))$ for all $A\in \mathcal B$ then we say that $T$ is measure-preserving or that $\mu$ is $T$-invariant. Clearly any measure-preserving transformation $T:X\rightarrow X$ is non-singular. 
In this section we briefly recall some basic notions on the structure of measurable flows and we refer the reader to \cite{Ambrose,AmbroseKakutani} for more on the subject.


\begin{definition}
A flow $\phi$ in $X$ is a one-parameter group $\{\phi_t\}$ , $-\infty < t <+\infty$, of measure preserving transformations $\phi_t :X \rightarrow X$. If $x\in X$ and $\phi$ is a flow on $X$ we say that the set $\{\phi_t(x): t\in \mathbb R\}$ is the trajectory of $x$ or the orbit of $x$ by the flow.
A flow $\phi : \mathbb R \times X \rightarrow X$ is said to be measurable if $\phi$ is a measurable function, that is, for any measurable set $Y \subset X$ the set $\{(x,t) : \phi_t(x) \in Y\}$ is a measurable set in the product space $\mathbb R \times X$ where the measure is the product of the Lebesgue measure on $\mathbb R$ and $\mu$.  
\end{definition}
%
%\begin{definition}
%We say that a measurable transformation $T:X\rightarrow X$ on $(X,\mathcal B,\mu)$ is a nonsingular isomorphism if it is a measurable isomorphism and satisfies: $\mu(A) = 0 \Leftrightarrow \mu(T^{-1}(A))=0$ for $A\in \mathcal B$.
%%Let (X,B,?) be a standard Borel space with a ?-finite measure ? on B. A nonsingular flow on (X,?) is a Borel map S:X�R?(x,t) ?Stx?XsuchthatStSs =St+s foralls,t?RandeachSt isa nonsingular transformation of (X, ?).
%\end{definition}

%\begin{definition}[Suspension flow or flow built under a function] \label{defi:built}
%Let $S:X \rightarrow X$ be a measure preserving transformation. Let $H: X\rightarrow \mathbb R$ be a positive real valued function which is measurable and integrable on $X$. Assume that
%\[\sum_{n=0}^{\infty}H(S^n(x)) = +\infty \quad \text{ and } \quad \sum_{n=1}^{\infty}H(S^{-n}(x)) = +\infty\]
%for every $x\in X$. Let $Y = X \times \mathbb R$ and define a the measure $\widehat{\mu}$ on $Y$ by taking the product $\widehat{\mu} = \mu \times m$ where $m$ denotes the Lebesgue measure on $\mathbb R$. Denote by $\widehat{X}$ the subset of $Y$ given by the portion under the graph of $H$, that is,
%\[\widehat{X} = \{ (x,s) : 0\leq s <H(x) \},\]
%and let $\widehat{\mathcal B}$ be the collection of all $\widehat{\mu}$ measurable subsets of $\widehat{X}$. Then, $(\widehat{X}, \widehat{\mathcal B},\widehat{\mu} )$ is a measurable space. Define the flow $\phi:\mathbb R \times \widehat{X} \rightarrow \widehat{X}$ by
%\begin{itemize}
%\item if $-s \leq t < -s+H(x)$ then 
%\[\phi_t(x,s) = (x,s+t);\]
%\item if $-s+\sum_{k=0}^{n-1}H(S^k(x)) \leq t \leq -s+\sum_{k=0}^{n} H(S^k(x))$ then,
%\[\phi_t(x,s) = (S^n(x), s+t-H(x)-...-H(S^{n-1}(x))), \quad n=1,2,...;\]
%\item if $ -s-\sum_{k=1}^{n}H(S^{-k}(x)) \leq t < -s -\sum_{k=1}^{n-1}H(S^{-k}(x))$ then,
%\[\phi_t(x,s) = (S^{-n}(x), s+t+H(S^{-1}(x)) + ... + H(S^{-n}(x)), n=1,2,... .\]
%\end{itemize}
%We call $\{\phi_t\}$ the suspension flow of the function $S$ with height function $H$, or the suspension of $S$ with ceiling function $H$. We say that $S$ is the base transformation and that $H$ is the ceiling function. On the literature, this construction is also called the flow built under the function $H$ on the measure preserving transformation $S$.
%\end{definition}


%\begin{definition}[Properly separable space]
%A measure space $(X,\mathcal B, \mu)$ is said to be properly separable if there exists a countable collection $\mathcal U$ of measurable sets such that the sigma algebra generated by $\mathcal U$, when completed with respect to the measure $\mu$, is equal to $\mathcal B$ the sigma algebra of all measurable sets of $X$.
%\end{definition}

Two flows $\phi = (\phi_t)_{t\in \mathbb R}$ on the Lebesgue space $(X,\mathcal B, \mu)$  and $\psi = (\psi_t)_{t\in \mathbb R}$ on the Lebesgue space $(Y,\mathcal C,\nu)$ are said to be \textit{isomorphic} if there exist invariant full measure sets $X_0 \subset X$, $Y_0 \subset Y$ and an invertible measure preserving transformation $\rho:X_0 \rightarrow Y_0$ such that
\[ \rho \circ \phi_t = \psi_t \circ \rho\]
for all $t\in \mathbb R$.

The following classical result of Ambrose-Kakutani shows that measure preserving flows on Lebesgue spaces can be represented as continuous flows on metric spaces.

\begin{theorem}[Ambrose-Kakutani \cite{AmbroseKakutani}] \label{theorem:ambrose.kakutani}
Let $\{\phi_t\}$ be a measure preserving measurable flow defined on a Lebesgue space $(X,\mathcal B, \mu)$. Then $\{\phi_t\}$ is isomorphic to a continuous flow on a separable metric space $M$ endowed with a measure $\lambda$ such that 
\begin{itemize}
\item[1)] every open set has positive $\lambda$-measure;
\item[2)] $\lambda$ is a regular measure.
\end{itemize}
\end{theorem}

%Classifications of measurable flows was also obtained for other authors generalizing Theorem \ref{theorem:ambrose.kakutani} to other contexts \cite{Kubo, Dani, Krengel, Krengel2}. For the case of non-singular flows (see \cite{Kubo} for definitions) on Lebesgue spaces we can summarize the results of I. Kubo and U. Krengel in the following theorem.
%
%\begin{theorem}\cite{Kubo, Krengel, Krengel2}
%Let $\{\phi_t\}$ be a free non-singular flow on a Lebesgue space $(X,\mathcal B,\mu)$. Then it is isomorphic to a flow built under a function $\varphi$ which takes at most two values.
%\end{theorem}

\subsection{Measurable choice}
We finish this preliminary section with a result by R. J. Aumann \cite{Aumann}, which although comes from the Decision Theory in Economics, lies in the realm of measure theory. This result will be used in the study of some atomic case.
% , as it has been used in \cite{PTV2}.

\begin{theorem}[Measurable Choice Theorem \cite{Aumann}] \label{theo:MCT}
Let $(T,\mu)$ be a $\sigma$-finite measure space, let $S$ be a Lebesgue space, and let $G$ be a measurable subset of $T\times S$ whose projection on $T$ is all of $T$. Then there is a measurable function $g:T\rightarrow S$, such that $(t,g(t)) \in G$ for almost all $t \in T$.
\end{theorem}








\section{Fibered spaces and disintegration} \label{sec:FiberedSpaces}
Given a continuous foliation $\mathcal F$ of a non-atomic Lebesgue probability space $X$, it is useful to look at $\mathcal F$ as fibers over a certain base space. It is not true that we can always choose a measurable set intersecting each plaque $\mathcal F(x)$ in exactly one point (the simplest example being the irrational linear foliation on the $2$-torus), so the quotient space $X/ \mathcal F$ is not always a good candidate for a base of a fibered space. In the light of this observation, instead of taking the quotient by the plaques we construct a fibered-type space over $X$ by literally attaching over each $x\in X$ the plaque $\mathcal F(x)$.

\begin{definition}\label{defi:fibered.type.space}
Given a space $X$ and a family $\mathcal P$ of subsets of $X$. We can construct a natural fibered-type space over $X$ where the fibers are given by the elements of the family $\mathcal P$. More precisely, we define the space
\[X^{\mathcal P} = \bigcup_{x\in X}\{x\}\times \mathcal P(x) \subset X\times X,\]
endowed with the $\sigma$-algebra induced by the product $\sigma$-algebra on $X\times X$.

We call $X^{\mathcal P}$ the $(X,\mathcal P)$-fibered space or simply the $\mathcal P$-fibered space. Each subset $\{x\}\times \mathcal P(x) \subset X^{\mathcal P}$ is called the fiber of $x$ on $X^{\mathcal P}$. 
\end{definition}


Given a continuous foliation $\mathcal F$ of a non-atomic Lebesgue probability space $(X,\mu)$, consider a local chart 
\[\varphi_x:(0,1)\times (0,1)^k \rightarrow U\]
of $\mathcal F$. The partition $\mathcal V = \{ \{x\} \times (0,1)^k \}$ is a measurable partition of $(0,1)\times (0,1)^k$ with respect to any Borel measure $\mu$ due to the separability of $(0,1)^k$. Hence on a local chart the partition given by the segments of leaves $\varphi_x(\{x\} \times (0,1)^k )$ forms a measurable partition on $U$ for any Borel measure on $U$. That means we can always disintegrate a measure on a local chart. Although the partition by the leaves of a foliation is not necessarily a measurable partition the next result allow us to say that the disintegration of a measure is atomic on the leaves, or it is absolutely continuous to Lebesgue on the leaves, since these properties persists independent of the foliated box one uses to disintegrate.

% Take $\nu := (\varphi_x^{-1})_{*}\mu$. By considering the closure $\overline{U}$ with the natural measure $\overline{\nu}$ which has zero measure on the boundary of $U$ we have, by Theorem \ref{theo:rokhlin}, a system of conditional measures $\{ \overline{\nu}_y\}$ where each $ \overline{\nu}_y$ is supported on  $\{y\}\times [0,1]^k$. By taking the standard projection, we have a system of conditional measures $\nu_y$ for $\nu$ with respect to the partition $\mathcal V$. By the definition of $\nu$ it follows that $\mu_{y}:=(\varphi_x)_*\nu_{\varphi_x^{-1}(y)} $ is a probability measure supported on the plaque $\mathcal F(x) \cap \varphi_x(U)$, where $y\in U$, and $\{\mu_{y}\}$ is a system of conditional measures for $\mu$ with respect to the partition $\{\mathcal F(x) \cap U\}$ of $U$. \tco{red}{RV: pra que fizemos isso?????} \textcolor{blue}{Para poder falar de uma medida desintegrada em uma folheacao - mas na verdade precisamos restringir o contexto}

\begin{proposition} \label{prop:disintegration.unbounded}
If $U_1$ and $U_2$ are described by the local charts $\varphi_{x_1}$ and $\varphi_{x_2}$ of $\mathcal F$ respectively, then the conditional measures $\mu_x^{U_1}$ and $\mu_x^{U_2}$, of $\mu$ on $U_1$ and $U_2$ respectively, coincide up to a constant on $U_1 \cap U_2$.
\end{proposition}
\begin{proof}
It follows from \cite[Proposition 5.17]{El.pisa}.
\end{proof}

\begin{definition}
We say that a probability $\mu$ has atomic disintegration with respect to a foliation if its conditional measures on any foliated box are sum of Dirac measures.
\end{definition}


\begin{remark}\label{remark:class.measures}
Consider the classical volume preserving Kronecker irrational flow on the the torus $\mathbb T^2$. Let $\mathcal F$ be the continuous foliation given by the orbits of this flow, it follows that this is not a measurable partition in the sense of Definition \ref{def:mensurable.partition}. Hence, we cannot apply the Rohklin's disintegration Theorem \ref{theo:rokhlin} even on the apparently well-behaved continuous foliations. But we may always disintegrate locally and compare two local disintegrations by the above result. The proposition above implies that we can talk about disintegration of a measure over a foliation even if it does not form a measurable partition, as long as we have in mind that for a disintegration we understand that on a plaque there is a class of conditional measures which differ up to a multiplication of a constant, we denote this system of conditional measures as $\{[\mu_x]\}$. More precisely for a given foliation $\mathcal F$ on each plaque $\mathcal F(x)$ there is a family of measures $[\mu_x]$ defined on $\mathcal F(x)$ such that if $\eta \in [\mu_x]$ then $\mu_x = \alpha \eta$ for some positive constant $\alpha \in \mathbb R$. And on a foliated box if one normalizes these measures they form a disintegration of the measure $\mu$ in this foliated box. 
\end{remark}




\subsection{Disintegration on the unitary fibered space}
In this section the foliation $\mathcal F$ comes from the orbits of a continuous flow $\phi$ on a separable metric space $X$. Hence $\mathcal F(x)$ is the orbit of $x$ through the flow $\phi$. Denote $B_{\mathcal F}(x,r):=\phi((-r,r)\times \{x\})$ and consider the family of sets \[\mathcal F^1 = \{ \mathcal F^1(x):= \{x\} \times B_{\mathcal F}(x,1)\}_x.\] 
For convenience, denote by $X_1^{\mathcal F}$ the $(X,\mathcal F^1)-$fibered space, that is,
 \[X_1^{\mathcal F} = \bigcup_{x \in X}\mathcal F^1(x).\]


\begin{lemma} \label{lemma:measurable.for.unbounded}
The partition of $X_1^{\mathcal F}$ by the verticals $\mathcal F^1(x)$ is a measurable partition with respect to any measure on $X_1^\mathcal{F}$.
\end{lemma}
\begin{proof}
Let $\{U_i\} \subset X$ be a countable basis of open sets of $X$. By the definition of $\mathcal F$, the $\mathcal F$ saturation of $U_i$ is given by $ \phi((-\infty,+\infty) \times U_i)$, which is a measurable set since the flow is continuous. Let $V_i:= (U_i \times \mathcal F(U_i)) \cap X_1^{\mathcal F}$. Each $V_i$ is a measurable set in $X^{\mathcal F}_1$. Now, it is easy to see that each fiber can be written as intersection of sets of the countable family of sets $\{V_i\}$ or its complement.

\end{proof}




\begin{proposition} \label{prop:unbounded.measurable}
% Let $\mathcal F$ be a continuous foliation of $(X,\mu)$ which is induced by the orbits of a continuous flow $\phi_t$. 
For each $x\in X$ denote by $\mu^1_x$ the measure on the equivalence class  $[\mu_x]$ (as defined on Remark \ref{remark:class.measures}) such that $\mu^1_x$ is a probability measure when restricted to $B_{\mathcal F}(x,1)$. Then 
\[x\mapsto \mu^1_x\]
is a measurable map, that is, given any measurable set $W\subset X$ the function
\[x\mapsto \mu^1_x(W)\]
is a measurable function.
\end{proposition}
\begin{proof}
On the fibered space $X^{\mathcal F}_1$ consider the measure $\widetilde{\mu}$ defined by
$$\widetilde \mu (\widetilde{A}) = \int_X \mu^1_x(\widetilde{A}_x) d\mu(x), $$
 for any measurable set $\widetilde{A} \subset X^{\mathcal F}_1$, where $\widetilde{A}_x = \{y\in B_{\mathcal F}(x,1): (x,y) \in \widetilde{A}\}$.
Since the vertical partition on $X^{\mathcal F}_1$ is a measurable partition by Lemma \ref{lemma:measurable.for.unbounded} the probability measure $\widetilde \mu$ has a Rohklin disintegration along the leaves for which the conditional measures varies measurably on the base point. By uniqueness and by the definition of $\widetilde \mu$ we have that the conditional measure on the plaque $\{x\} \times B_{\mathcal F}(x,1)$ is exactly $\mu^1_x$. By the properties of the Rohklin disintegration it follows that given any measurable set $\widetilde{W} \subset X^{\mathcal F}_1$ we have that $x\mapsto \mu^1_x(\widetilde{W}_x)$ is a measurable function.
Given any measurable set $W\subset X$ let 
\[\widetilde{W}:= \bigcup_{x\in W} \{x\}\times [W\cap B_{\mathcal F}(x,1)].\]
Thus $\widetilde{W}_x = W\cap B_{\mathcal F}(x,1)$ and then we have that
\[x\mapsto \mu^1_x(W \cap B_{\mathcal F}(x,1)) = \mu^1_x(W)\]
is a measurable function on $x$ as we wanted to show.
\end{proof}

%Using the same arguments as for the bounded plaques case we have the following.

\begin{proposition} \label{prop:disinmeasurable2}
For each $r\in (0,\infty)$ the function
\[x\mapsto \mu^1_x(B_{\mathcal F}(x,r))\]
is a measurable function.
\end{proposition}
\begin{proof}
For each fixed $0\leq r \leq 1$, define the following $r$-top function
\[f_r: X \rightarrow X_1^{\mathcal F},  \quad f_r(z) = (z,\phi_r(z)).\]
Let
\[W := \bigcup_{z\in X} [f_{-r}(z),f_r(z)]_z,\]
where $[x, y]_z$ denotes the closed vertical segment connecting $x$ and $y$ on $\mathcal F(z)$. Observe that $W$ is Borel since it is a compact set. By the measurability of $x\mapsto \mu^1_x(W)$ we have that $z\mapsto \mu^1_z(B_{\mathcal F}(z,r))$ is a measurable function. Consequently
\[z\mapsto \mu^1_z(B_{\mathcal F}(z,r))\]
is a measurable function.
Since $x\mapsto \mu_{x}^1(B_{\mathcal F}(x,r_0))$, for every $0\leq r_0 \leq 1$ fixed, is $\phi_1-$invariant we have conclude that for any $r\in (0,\infty)$ the function $x\mapsto \mu_{x}^1(B_{\mathcal F}(x,r))$ is measurable.
\end{proof}

\begin{corollary} \label{cor:jointlymeasurable2}
If $\{[\mu_x]\}$ is a non-atomic system of conditional measures then, for each typical $x\in X$ the function
\[r\mapsto \mu^1_x(B_{\mathcal F}(x,r))\]
is continuous. Furthermore the function
\[(x,r) \mapsto \mu^1_x(B_{\mathcal F}(x,r))\]
is jointly measurable.
\end{corollary}
\begin{proof}
Let $x\in X$ be a $\mu$-typical point, hence $\mu^1_x$ is a non-atomic measure on $\mathcal F(x)$. First, let us prove that $r \mapsto \mu^1_x(B_{\mathcal F}(x,r))$ is a continuous function. Let $y_n \in \mathcal F(x)$ and $\varepsilon_n \searrow \varepsilon \in (0,\infty)$, hence  $\mu^1_{x}(B_{\mathcal F}(x,\varepsilon_n)) = \mu^1_x(B_{\mathcal F}(x,\varepsilon)) + \mu^1_x(B_{\mathcal F}(x,\varepsilon_n)\setminus B_{\mathcal F}(x,\varepsilon))$. Because $\mu^1_x$ is nonatomic 
$$\lim_{n\rightarrow \infty}\mu^1_x(B_{\mathcal F}(x,\varepsilon_n)\setminus B_{\mathcal F}(x,\varepsilon)) =0.$$ 
Then,
$$\mu^1_{x}(B_{\mathcal F}(x,\varepsilon_n))  \rightarrow  \mu^1_x(B_{\mathcal F}(x,\varepsilon)).$$
By Proposition \ref{prop:disinmeasurable2} we know that $x \mapsto \mu^1_x(B_{\mathcal F}(x,r))$ is a measurable function, therefore the function $(x,r) \mapsto \mu^1_{x}(B_{\mathcal F}(x,r))$ is a Carath\'eodory function (i.e. measurable in one variable and continuous in the other, see \cite[Definition 4.50]{InfDimAna}), in particular it is a jointly measurable function \cite[Lemma 4.51]{InfDimAna}.
\end{proof}



\subsection{The leafwise measure distortion}

The last concept we will introduce in this section is the concept of leafwise measure distortion.


\begin{definition} \label{defi:distortion2}
Let $(X,\mu)$ be a non-atomic Lebesgue space and $\mathcal F$ be a continuous foliation of $X$ induced by the orbits of a continuous flow $\phi_t$. Let $\{[\mu_{x}]\}$ denote the system of equivalence classes of conditional measures along $\mathcal F$. We define the upper and lower $\mu$-distortion at $x$ respectively by
\[\overline{\Delta(\mu)}(x):= \limsup_{\varepsilon \rightarrow 0}\frac{\mu^1_x(B_{\mathcal F}(x,\varepsilon))}{\varepsilon} ,\quad \underline{\Delta(\mu)}(x):= \liminf_{\varepsilon \rightarrow 0}\frac{\mu^1_x(B_{\mathcal F}(x,\varepsilon))}{\varepsilon},\]
where $\mu^1_x$ is taken to be the measure on the class of $[\mu_{x}]$ which gives weight one to $B_{\mathcal F}(x,1)$.
If the upper and lower distortions at $x$ are equal then we just call it the $\mu$-distortion at $x$ and denote by
\[\Delta(\mu)(x):= \lim_{\varepsilon \rightarrow 0}\frac{\mu^1_x(B_{\mathcal F}(x,\varepsilon))}{\varepsilon}.\]
\end{definition}

\section{Proof of the main result} \label{sec:DDL}

We proceed to the proof of our main result, Theorem \ref{theo:continuous.flow}, but first we provide a sketch of its proof.

\subsection{Sketch of the proof of Theorem \ref{theo:continuous.flow}}
The proof will be made in two steps. The first, and easy case, is the atomic case. The second case, the non-atomic case is the one where the main ideas appear. 

The first observation is that ergodicity implies that the upper (resp. lower) $\mu-$distortion at $x$ is constant almost everywhere. Then, using the $\varphi_1-$invariance of the family $\{B_{\mathcal F}(x,r)\}$ and the ergodicity of the measure, we obtain some uniformity on the upper (resp. lower) $\mu-$distortion in the sense that along a certain sequence $(\varepsilon_k)_k$, $\varepsilon_k \rightarrow 0$, the ratios appearing in Definition \ref{defi:distortion2} converge to the upper (resp. lower) $\mu-$distortion with the same rate for almost every point $x \in X$ . This is proven in Lemmas \ref{lema:sequenciaboa} and  \ref{lema:sequenciaboa2}. Once proven this uniformity of the upper (resp. lower) distortion, we turn our attention to the set of all points $\overline{\Pi}$ (resp. $\underline{\Pi}$) where such uniformity occurs and its topological characteristics when restricted to a plaque. To be more precise, we prove in Lemma \ref{lemma:closed} that the set of points for which the uniforme distortions occurs is closed in each plaque intersecting it.
The last step consists of analyzing the set $D$ of points $x$ for which $\overline{\Pi}$ is dense in $\mathcal F(x)$, that is, $\overline{\Pi}\cap \mathcal F(x)=\mathcal F(x)$. $D$ is $\varphi_1$-invariant thus it has full or zero measure. If it has full measure then the denseness of $\overline{\Pi}$ on the plaques $\mathcal F(x), x\in D$, allows us to extend the uniform upper distortion to every point on the respective plaque (i.e. orbit). Using the uniformity at every point we prove that the upper distortion is a constant times the $\mu_x$ measure of the set $B_{\mathcal F}(x,1)$ on the plaque $\mathcal F(x)$. Applying the same argument for the set $\underline{\Pi}$ where the lower distortion is uniform we get to the same equality and conclude that the upper and lower distortion are equal, thus the limit converges and we actually have a well defined distortion. Using this fact we prove in Lemma \ref{lemma:equallebesgue} that $\mu^1_x$ is a constant times the natural measure induced by the flow on the orbits.
If $D$ has zero measure then almost every plaque has pieces of open intervals in it which are in the complement of the set $\overline{\Pi}$. We use this holes to show that atoms should appear, which yields an absurd.




\subsection{Proof of Theorem \ref{theo:continuous.flow}}

To simplify notation we denote $f:=\phi_1$.

First let us deal with the case where $\mu$ itself has atoms, that is, there is a countable subset $Z\subset X$ such that $\mu(\{z\})>0$ for any $z\in Z$. Since $f$ is ergodic and $Z$ is $f$-invariant we have $\mu(Z)=1$. Hence the second item of the theorem is satisfied. We may now assume the measure $\mu$ itself is atomless.

Let $Per(\phi)$ to be the set of periodic orbits of the flow $\phi$. First let us assume that $\mu(Per(\phi))=0$ and break the proof in two cases (the \textit{atomic case} and the \textit{non-atomic case}). We deal with $\mu(Per(\phi))>0$ by the end of the proof. Also recall that $\mathcal F$ is the foliation whose plaques are the orbits of the flow and  $B_{\mathcal F}(x,r):=\phi((-r,r)\times \{x\})$. 

% First let us deal with the case where $\mu$ itself has atoms, that is, there is a countable subset $Z\subset X$ such that $\mu(\{z\})>0$ for any $z\in Z$. Since $f$ is ergodic and $Z$ is $f$-invariant we have $\mu(Z)=1$ and since the weight of each atom is also an $f$-invariant function we have that there exists $k_0 \in \mathbb N$ such that $Z$ has $s$ elements $a_1,...,a_{s}$ and $\mu(a_i) = 1/s$ for every $1\leq i \leq s$. Consider $\mathcal F_i := \mathcal F(a_i)$. By the invariance of the cardinality of $\mathcal F_i \cap  Z$ and ergodicity of $f$, each $\mathcal F(a_i)$ has exactly the same number of atoms $k_0 \leq s$. Thus we fall in the second case of the statement.




\vspace{0.3cm}
\textbf{The atomic case:} Assume that $\mu$ has atomic disintegration over $\mathcal F$.

Consider the measurable function $g_r: x \mapsto \mu_x^1(B_{\mathcal F}(x,r))$. Now define the weight map
$$w: x \mapsto \mu_x^1(\{x\}).$$
This is a measurable map because $w(x) = \lim_{r \rightarrow 0} g_r(x)$ and pointwise limit of measurable functions is a measurable function.

Now consider the invariant set $w^{-1} ((0,\delta) )$ of atoms whose weight is less then $\delta$. Ergodicity implies that this set has zero or one measure. Thus, there exists a real number $\delta_0>0$ such that each atom has weight $\delta_0$ and, consequently, each plaque has the same number of atoms $k_0 = 1/\delta_0$. 

Hence we have proved statement $(2)$ of Theorem \ref{theo:continuous.flow}.

\vspace{0.3cm}
\textbf{Non-atomic case}: We now assume that the disintegration is not atomic.

Let $\{[\mu_x]\}$, as in Remark \ref{remark:class.measures}, be the equivalence classes of the conditional measures coming from the Rokhlin disintegration of $\mu$ along the leaves of $\mathcal F$.
% Recall that the upper and lower unitary distortion defined on definition \ref{defi:distortion} by $\overline{\Delta}$ and $\underline{\Delta}$ respectively, that is,
%
%\[\overline{\Delta}(x):= \limsup_{\varepsilon \rightarrow 0}\frac{\mu_x(B_{d_x}(x,\varepsilon))}{\varepsilon} ,\quad \underline{\Delta}(x):= \liminf_{\varepsilon \rightarrow 0}\frac{\mu_x(B_{d_x}(x,\varepsilon))}{\varepsilon}. \]
%Recall that $B_{d_x}(x,\varepsilon)$ is the ball inside $\mathcal F(x)$, centered in the point $x$ and with radius $\varepsilon$ with respect to the metric $d_x$.
Observe that $\mu_x^1(B_{\mathcal F}(x,\varepsilon)) >0 $ for every $x \in \operatorname{Supp}_{\mathcal F}(\mu^1_x)$ (where the support here is inside $\mathcal F(x)$). Thus, it makes sense to evaluate the upper and lower unitary distortions. Also observe that, a priori, $\overline{\Delta}(x)$ and $\underline{\Delta}(x)$ could be infinity. In any case these functions are well-known to be measurable functions.
Also note that both $\overline{\Delta}(x)$ and $\underline{\Delta}(x)$ are $f$-invariant maps because
\[f_{*}\mu^1_x = \mu^1_{f(x)} \;\text{ and }\;f(B_{\mathcal F}(x,\varepsilon)) = B_{\mathcal F}(f(x),\varepsilon).\]
%since
%\[d_{f(x)}(f(x),f(y))=d_x(x,y).\] 
By ergodicity of $f$ it follows that both are constant almost everywhere, let us call these constants by $\overline{\Delta}$ and $\underline{\Delta}$. That is, for almost every $x$:
\begin{equation}\label{eq:delta}
\overline{\Delta}(x)  = \overline{\Delta} ,\quad \text{ and } \underline{\Delta}(x)  = \underline{\Delta}.
\end{equation}
Let $D$ be a (full measure) set of points $x$ for which \eqref{eq:delta} occurs.

\begin{lemma} \label{lema:sequenciaboa} 
If $\overline{\Delta}$ is finite, there exists a sequence $\varepsilon_k\rightarrow 0$, as $k\rightarrow +\infty$, and a full measure subset $R \subset D$ such that
\begin{itemize}
\item[i)] $R$ is $f$-invariant;
\item[ii)]for every $x \in R$, then
\begin{equation}\label{eq:uniform}
\left| \frac{ \mu^1_x(B_{\mathcal F}(x,\varepsilon_k))}{\varepsilon_k} - \overline{\Delta}  \right|   \leq \frac{1}{k};\end{equation}
%\item[iii)]on a fiber $\mathcal F(x)$ the set of points which satisfies the inequality from item (ii) forms a closed set inside $\mathcal F(x)$.
\end{itemize}
% $R \cap \mathcal F(x)$ is closed inside $\mathcal F(x)$ for each $x \in R$;
An analogous result holds if instead of $\overline{\Delta}$ we consider $\underline{\Delta}$.
\end{lemma}
%\tco{red}{G:Pro caso n\~ao compacto basta escolher $m_x$ que da peso $1$ para $B(x,1)$}. \tco{red}{RV: Temos s\'o que nos certificar que est\'a tudo certo aquela coisaa de que podemos encontrar classes de medidas que desintegram no nosso caso tamb\'em.}
\begin{proof}
Since $\overline{\Delta}(x) = \overline{\Delta}$ for every $x \in D$ and $k\in \mathbb N^{*}$ define
\[\varepsilon_k(x):= \sup \left\{\varepsilon: \left| \frac{ \mu^1_x(B_{\mathcal F}(x,\varepsilon))}{ \varepsilon} - \overline{\Delta}  \right| +\varepsilon  \leq \frac{1}{k} \right\}.\]
Observe that $\varepsilon_k(x)$ exists because since the $\limsup$ is $\overline{\Delta}$ we can take a sequence $\varepsilon_l(x) \rightarrow 0$ such that the ratio given approaches $\overline{\Delta}$.


\noindent {\bf Claim:}
The function $\varepsilon_k(x)$ is a measurable for all $k \in \mathbb N$.
\begin{proof}
Observe that since $\mu_x$ is not atomic we have
\[\varepsilon_k(x) = \lim_{n\rightarrow \infty} \varepsilon^n_k(x)\]
where
\[\varepsilon^n_k(x) = \sup \left\{\varepsilon: \left| \frac{ \mu_x^1(B_{\mathcal F}(x,\varepsilon))}{ \varepsilon} - \overline{\Delta}  \right|  +\varepsilon < \frac{1}{k} +\frac{1}{n} \right\}.\]
So, it is enough to prove that $\varepsilon_k^n(x)$ is measurable on $x$. 

Define \[g(x,\varepsilon) = \left| \frac{ \mu_x^1(B_{\mathcal F}(x,\varepsilon))}{ \varepsilon} - \overline{\Delta}  \right| + \varepsilon.\]
%Observe that since $\mu_x$ is not atomic, $g(x,\cdot)$ is continuous. Also, since $x\mapsto \mu_x$ is measurable, for any fixed $\varepsilon$ the function $g(\cdot, \varepsilon)$ is \textcolor{red}{measurable.} Thus, $g$ is jointly measurable.
By Corollary \ref{cor:jointlymeasurable2}, for any typical $x\in M$ the function $g(x,\cdot):(0,\infty) \rightarrow \mathbb (0,\infty)$ is continuous.
%be a typical point (that is, $\mu_x$ is not atomic in $\mathcal F(x)$) and lets prove that $g(x,\cdot):(0,\infty) \rightarrow \mathbb (0,\infty)$ is continuous. Let $y_n \in \mathcal F(x)$, $\varepsilon_n\rightarrow \varepsilon \in (0,\infty)$. Because $\mu_x$ is not atomic we have
%\[\mu_{x}(B_{d_x}(x,\varepsilon_n)) \rightarrow  \mu_x(B_{d_x}(x,\varepsilon)) \Rightarrow g(x,\varepsilon_n) \rightarrow g(x,\varepsilon). \]
%Therefore $g(x,\cdot)$ is indeed continuous.  
Let $\varepsilon>0$ be fixed and let us prove that $g(\cdot, \varepsilon):M \rightarrow \mathbb (0,\infty)$ is a measurable function. By Proposition \ref{prop:disinmeasurable2} we know that $x\mapsto  \mu_x^1(B_{\mathcal F}(x,\varepsilon))$ is a measurable function, therefore $g(\cdot, \varepsilon)$ is measurable function.

%Since $\varepsilon$ is fixed, we just need to show that
%$x\mapsto \mu_x(B_{d_x}(x,\varepsilon))$ is \textcolor{red}{measurable} on a full measure set. Take any interval $(a,b) \subset (0,\infty)$.
%
%
%Similar to the argument made on Proposition XXXX, define the roof and floor functions $f$ and $g$ which are measurable functions. Take a sequence of compact sets $K_1, K_2, ...$
%such that $F=\bigcup K_i$ has full measure and $f$ and $g$ are continuous on each $K_i$. On each $K_i$ we have
%\[x\mapsto \mu_x(B_{d_x}(x,\varepsilon))\]
%is continuous thus $ x\mapsto \mu_x(B_{d_x}(x,\varepsilon))$ is measurable on $F$ as we wanted.

%\[W\setminus Y = \bigcup_{(x,y)\in K\cap L} (g(x,y), f(x,y)],\]
%which is a Borel set whose projection has positive measure and, consequently, has positive measure. Consider the set $T = \bigcup f^k(W\seminus Y)$. $T$ is a Borel $f\times f$-invariant set. Consequently $T$ has full measure and full projected measure. Since $x\mapsto \mu_x$ is measurable, we have
%\[x\mapsto \mu_x()\]

 
%We know that, since $M$ is a Souslin space and $\{d_x\}$ is a Souslin metric system, the set \textcolor{red}{Argument equal to proposition. Must justify properly using trivializations}
%\[W:=\bigcup_{x\in M} \{x\}\times B_{d_x}(x,\varepsilon)\]
%is Souslin. Since $l: x\mapsto \mu_x(W)$ is measurable then $l^{-1}(a,b)) = \{x: \mu_x(B_{d_x}(x,r)) \in (a,b)\}$ is measurable as we wanted.

Given any $k\in \mathbb N$, $k>0$, the continuity of $g(x,\cdot)$ implies that 
\[\varepsilon_k^{-1}((0,\beta))=\{x: \varepsilon_{k}(x) \in (0,\beta)\}  = \bigcap_{r\geq b, r\in \mathbb Q} g(\cdot, r)^{-1}([1/k, +\infty)). \]
Therefore $\varepsilon_k^{-1}((0,\beta))$ is measurable and consequently $\varepsilon_k$ is a measurable function for every $k$.
\end{proof}

Note that $\varepsilon_k(x)$ is $f$-invariant. Thus, by ergodicity, let $R_k$ be a full measure set such that $\varepsilon_k(x)$ is constant equal to $\varepsilon_k$.
It is easy to see that the sequence $\varepsilon_k$ goes to $0$ as $k$ goes to infinity. Take $\widetilde{R}:= \bigcap_{k=1}^{+\infty} R_k$.
Since each $R_k$ has full measure, $\widetilde{R}$ has full measure and clearly satisfies what we want for the sequence $\{\varepsilon_k\}_{k}$.
Finally, take $R = \bigcap_{-\infty}^{+\infty} f^i(\widetilde{R})$. $R$ is $f$-invariant, has full measure and satisfies $(i)$ and $(ii)$. 

\end{proof}

% We define the set $\Pi$ of all points where the uniform convergence condition \eqref{eq:uniform} holds, that is:
Now consider the following set
\[\overline{\Pi} := \bigcup_{x \in R} \overline{\Pi}_x.\] 
where
\[\overline{\Pi}_x:= \left\{y \in \mathcal F(x):\left| \frac{ \mu^1_x(B_{\mathcal F}(y,\varepsilon_k))}{\varepsilon_k} - \overline{\Delta}  \right|   \leq \frac{1}{k}, \forall k\geq 1 \right\},\]
similarly we define $\underline{\Pi}_x$ and $\underline{\Pi}$ with $\underline \Delta$ in the role of $\overline \Delta$.

\begin{lemma} \label{lemma:closed}
For every $x \in R$ the set $\overline{\Pi}_x$ is closed in the plaque $\mathcal F(x)$.
\end{lemma}
\begin{proof}
Let $y_n \rightarrow y$, $y_n \in \overline{\Pi}_x$, $y\in \mathcal F(x)$. To prove that $y\in \overline{\Pi}_x$ it is enough to show that 
 \[\lim_{n\rightarrow \infty} \mu_x^1(B_{\mathcal F}(y_n,\varepsilon_k)) = \mu_x^1(B_{\mathcal F}(y,\varepsilon_k)). \]
Given any $k\in \mathbb N$, since $\mu_x$ is not atomic we have that 
\begin{eqnarray*}
 \mu^1_x(\partial B_{\mathcal F}(y,\varepsilon_k)) = \mu^1_x(\phi(-\varepsilon_k,y) \cup \phi(\varepsilon_k,y)) =0
\end{eqnarray*}
and
\begin{eqnarray*}
 \mu^1_x(\partial B_{\mathcal F}(y_n,\varepsilon_k)) = \mu^1_x(\phi(-\varepsilon_k,y_n) \cup \phi(\varepsilon_k,y_n)) =0, \forall n \in \mathbb N,
\end{eqnarray*}
where $\partial B_{\mathcal F}$ denotes the boundary of the set inside the leaf.
% \[\mu^1_x(\partial B_{\mathcal F}(y,\varepsilon_k)) = \mu^1_x(\partial B_{\mathcal F}(y_n,\varepsilon_k)) = 0. \]
% Thus 
%\textcolor{red}{observar argumento do infinito},

Now, let $B_n:=B_{\mathcal F}(y_n,\varepsilon_k) \Delta B_{\mathcal F}(y,\varepsilon_k)$ where $Y\Delta Z$ denotes the symmetric diference of the sets $Y$ and $Z$. Observe that, by passing to a subsequence of $y_n$ if necessary, we have $B_n \supset B_{n+1}$, for every $n \geq 1$. Thus
\begin{align*}
  \lim_{n\rightarrow \infty}  \mu^1_x(B_n) & =  \lim_{n\rightarrow \infty}  \mu^1_x \left(\bigcap_{n} B_n \right) \\
 &=   \lim_{n\rightarrow \infty}  \mu^1_x(\{\phi(-\varepsilon_k, y) , \phi(\varepsilon_k,y) \}) \\
 &=  0.\end{align*}
 Therefore $\lim_{n\rightarrow \infty} \mu_x^1(B_{\mathcal F}(y,\varepsilon_k) \setminus B_{\mathcal F}(y_n,\varepsilon_k)) = \lim_{n\rightarrow \infty} \mu_x^1(B_{\mathcal F}(y_n,\varepsilon_k) \setminus B_{\mathcal F}(y,\varepsilon_k))= 0$ and consequently 
 \[\lim_{n\rightarrow \infty} \mu_x^1(B_{\mathcal F}(y_n,\varepsilon_k)) = \mu_x^1(B_{\mathcal F}(y,\varepsilon_k)), \]
 as we wanted to show.
 
%We now claim that \begin{eqnarray}\label{eq:PI.closed.on.plaque}
%               \lim_{n\rightarrow \infty}    \mu^1_x(B_{\mathcal F}(y_n,\varepsilon_k)) = \mu^1_x( B_{\mathcal F}(y,\varepsilon_k)),
%                  \end{eqnarray}
%(\tco{red}{RV: Gabriel, eu nao sei colocar n a infinito embaixo da seta})
%which implies $y \in \overline{\Pi}_x$. 
%
%
%Equation (\ref{eq:PI.closed.on.plaque}) can be verified by the fact that the set $B_{\mathcal F}(y_n,\varepsilon_k) \Delta B_{\mathcal F}(y,\varepsilon_k)$ (i.e. the symmetric difference of $B_{\mathcal F}(y_n,\varepsilon_k)$ and $B_{\mathcal F}(y,\varepsilon_k)$) are two intervals that tends to points and the non-atomicity implies they have zero measure. (\tco{red}{RV: Gabriel, veja se como eu escrevi deixou bem explicado})
\end{proof}
An analogous result is true for $\overline{\Delta}$.
\begin{lemma} \label{lema:sequenciaboa2}
If $\overline{\Delta}$ is infinity, there exists a sequence $\varepsilon_k\rightarrow 0$, as $k\rightarrow +\infty$, and a full measure subset $R^{\infty} \subset D$ such that
\begin{itemize}
\item[i)] $R^{\infty}$ is $f$-invariant;
%\item[ii)] $R \cap \mathcal F(x)$ is closed inside $\mathcal F(x)$ for each $x \in R$;
\item[ii)]for every $x \in R^{\infty}$ we have 
\begin{equation}\label{eq:infity}
 \frac{\mu^1_x(B_{\mathcal F}(x,\varepsilon_k))}{\varepsilon_k} \geq k .\end{equation}
\end{itemize} 
An analogous result holds if instead of $\overline{\Delta}$ we consider $\underline{\Delta}$.
\end{lemma}

Analogously to what we have done for the finite case, define
\[\overline{\Pi}^{\infty}_x:= \left\{y \in \mathcal F(x): \frac{ \mu^1_x(B_{\mathcal F}(x,\varepsilon_k))}{\varepsilon_k} \geq k , \forall k\geq 1\right\},\]
and
\[\overline{\Pi}^{\infty} := \bigcup \overline{\Pi}^{\infty}_x.\]
Similarly we define $\underline{\Pi}^{\infty}_x$ and $\underline{\Pi}^{\infty}$.

\begin{lemma}
If $\overline{\Delta}$ (resp. $\underline{\Delta}$) is infinity then for every $x \in R$ the set $\overline{\Pi}^{\infty}_x$ (resp. $\underline{\Pi}^{\infty}_x$) is closed on the plaque $\mathcal F(x)$.
\end{lemma}
\begin{proof}
Analogous to the proof of Lemma \ref{lemma:closed}.
\end{proof}

\begin{lemma}\label{lemma:two.sets}
 If $\overline{\Delta}$ is finite, then there are Borel sets $Q$ and $G$ such that
 \begin{itemize}
  \item[i)] $f(Q)=Q$ and $f(G)=G$;
  \item[ii)] $Q \cap G = \emptyset$;
  \item[iii)] $\mu(Q\cup G)=1$;
  \item[iv)] if $x \in Q$, then for $\varepsilon_k$ as in Lemma \ref{lema:sequenciaboa} then
  \[\left| \frac{ \mu_x^1(B_{\mathcal F}(x,\varepsilon_k))}{\varepsilon_k} - \overline{\Delta}  \right|   \leq \frac{1}{k};\]  
  \item[v)] if $x \in G$, then there exists $k_0 \in \mathbb N$ such that 
    \[\left| \frac{ \mu^1_x(B_{\mathcal F}(x,\varepsilon_{k_0}))}{\varepsilon_{k_0}} - \overline{\Delta}  \right|   > \frac{1}{k_0}.\]  
 \end{itemize}
\end{lemma}
\begin{proof}
Consider $\overline{\Pi}$ as defined above. Take any $x\in \overline{\Pi}^c$, that is, there exists $k\geq 1$ such that
 \[\left| \frac{ \mu^1_x(B_{\mathcal F}(x,\varepsilon_{k}))}{\varepsilon_{k}} - \overline{\Delta}  \right|   > \frac{1}{k}.\]  
By the measurability of $x\mapsto \mu^1_x(B_{\mathcal F}(x,\varepsilon_k))$ proved in Proposition \ref{prop:disinmeasurable2} and Lusin's Theorem we can take a compact set $G_1$ where this function varies continuously. Thus, there exists an open set $G_2$ such that for every $y\in G_2\cap G_1$ we have
 \[\left| \frac{ \mu^1_x(B_{\mathcal F}(x,\varepsilon_{k}))}{\varepsilon_{k}} - \overline{\Delta}  \right|   > \frac{1}{k}.\]  
Define $G = \bigcup_{n\in \mathbb Z}f^n(G_2\cap G_1)$.

Let $x \in \overline{\Pi}$. For each $n\in \mathbb N$ we have
 \[\left| \frac{ \mu^1_x(B_{\mathcal F}(x,\varepsilon_k))}{\varepsilon_k} - \overline{\Delta}  \right| < \frac{1}{k}+\frac{1}{n}.\]
 Using again Proposition \ref{prop:disinmeasurable2}, Lusin's Theorem and the invariance of $\mu_x$ by $f$, we find a sequence of nested Borel sets $ \ldots Q_{n+1} \subset Q_n \subset Q_{n-1} \subset ...\subset Q_1$ such that $f(Q_n)=Q_n$, $n\geq 1$ and for all $y\in Q_n$ we have
 \[\left| \frac{ \mu^1_y(B_{\mathcal F}(y,\varepsilon_k))}{\varepsilon_k} - \overline{\Delta}  \right| < \frac{1}{k}+\frac{1}{n}.\]
By Lemma \ref{lema:sequenciaboa} we have $\mu(Q_n)=1$ for every $n$.
Take $Q:=\bigcap_{n=1}^{\infty}Q_n$. Then $Q$ is an $f$-invariant Borel set and $\mu(Q)=1$.
Therefore $\mu(Q\cup G) = 1$. Also, it is clear that $Q\cap G = \emptyset$ and we conclude the proof of the lemma.
\end{proof}
Consider the following measurable set 
$$D := \mathcal F(Q) \setminus \mathcal F(G).$$
Equivalently
\[D = \{x \in \mathcal F(G \cup Q) : \overline{\Pi}_x \cap \mathcal F(x) = \mathcal F(x)\},\]
that is, $D$ is the set of all points whose plaque is fully inside $\overline{\Pi}_x$.

In the sequel of the proof we will need the following counting lemma.

\begin{lemma} \label{lemma:aux}
Let $r>0$ be a fixed real number and  $x\in D$ an arbitrary point. Let $a_i := \varphi_{2i r}(\varphi_{-1}(x))$ and $b_i := \varphi_{2ir}(x)$ for $i=1,2,..., l$ where $l=\left \lfloor \frac{1}{2}\left(\frac{1}{r}-1\right)\right \rfloor$. Then
\begin{equation}\label{eq:statement}
\sum_{i=1}^{l}\mu_{x}^{1}(B_{\mathcal F}[a_i,1]) + \sum_{i=1}^{l}\mu_{x}^{1}(B_{\mathcal F}[b_i,1]) = 2l.\end{equation}
\end{lemma}
\begin{proof}
To simplify the notation, for $s>0$ we will write $[x,\varphi_s(x)]$ to denote the set $\{\varphi_t(x): 0\leq t \leq s\}$. With this notation we can write
\begin{equation*}
[\varphi_{-1}(x),x] = [\varphi_{-1}(x), a_1] \cup [a_1,a_2] \cup \ldots \cup [a_{l-1}, a_l] \cup [a_l, \varphi_{2(l+1)r-1}(x)] \cup [\varphi_{2(l+1)r-1}(x),x] 
\end{equation*}
Denote $J_0 := [\varphi_{-1}(x), a_1] $, $J_i:=[a_i,a_{i+1}]$ for $1\leq i \leq l-1$, $J_l :=  [a_l, \varphi_{2(l+1)r-1}(x)] $ and $J_{l+1} =  [\varphi_{2(l+1)r-1}(x),x] $. Thus we can rewrite
\begin{equation}\label{eq:part1}
[\varphi_{-1}(x),x] = J_0 \cup \ldots J_{l+1}.
\end{equation}
Now, by applying $\varphi_1$ to \eqref{eq:part1} we can write
\begin{align}\label{eq:part2}
[x,\varphi_1(x)] = & [x, b_1] \cup [b_1,b_2] \cup \ldots \cup [b_{l-1}, b_l] \cup [b_l, \varphi_{2(l+1)r}(x)] \cup [\varphi_{2(l+1)r}(x),\varphi_1(x)] \\
=& \varphi_1(J_0) \cup \ldots \varphi_1(J_{l+1}).
\end{align}
Also as a consequence of \eqref{eq:part1} we can write
\begin{equation} \label{eq:part3}
[\varphi_{-2}(x), \varphi_2(x)] = \varphi_{-1}(J_0) \cup \ldots \cup \varphi_{-1}(J_{l+1}) \cup [\varphi_{-1}(x),x] \cup  [x,\varphi_1(x)] \cup \varphi_{2}(J_0) \cup \ldots \cup \varphi_{2}(J_{l+1}) . \end{equation}

Now, observe that each term involved in the sums on the left side of \eqref{eq:statement} can be written as the sum of the $\mu_x^1-$measure of sets of the forms involved on the equations \eqref{eq:part1}, \eqref{eq:part2} and \eqref{eq:part3}. Lets count how many times each of this sets appears on the left side of \eqref{eq:statement}. 
\begin{itemize}
\item Observe that the set $\varphi_{-1}(J_0)$ is not contained in any of the sets $B_{\mathcal F}[a_i,1]$, $B_{\mathcal F}[b_i,1]$, thus it does not appears on \eqref{eq:statement}. However, the set $\varphi_1(J_0)=[x,\varphi_1(a_1)]$ is contained in all of the sets $B_{\mathcal F}[a_i,1], B_{\mathcal F}[b_i,1]$, thus is appears on $2l$ times on the equation \eqref{eq:statement}. Thus, $\mu_x^1(\varphi_1(J_0))$ appears exactly $2l$ times on \eqref{eq:statement}. 
%Since $\mu_x^1$ is preserved by the time-one map $\varphi_1$, we can say that both, $\mu_x^1(\varphi_{-1}(J_0))$ and $\mu_x^1(\varphi_1(J_0))$ appears exactly $l$ times each.
\item For any $1\leq i \leq l+1$ the set $\varphi_{-1}(J_i)$ appears on each of the terms $B_{\mathcal F}[a_j,1]$, $j=1,...,i$, that is, it appears $i$ times on \eqref{eq:statement}. On the other hand the set $\varphi_1(J_i)$ appears $2l-i$ times as it does not belong only to the sets $B_{\mathcal F}[a_j,1]$, $j=1,...,i$. By the fact that $\varphi_1$ preserves $\mu_x^1$ we know that $\mu_x^1(\varphi_{-1}(J_i)) =\mu_x^1(\varphi_1(J_i))$ and then we can say that $\mu_x^1(\varphi_1(J_i))$ appears exactly $2l$ times on \eqref{eq:statement}
\item By symmetry we can see that the terms $\mu_x^1(J_i)$ also appears exactly $2l$ times each. 
\end{itemize}
Thus we have 
\begin{align*}
\sum_{i=1}^{l}\mu_{x}^{1}(B_{\mathcal F}[a_i,1]) +  \sum_{i=1}^{l}\mu_{x}^{1}(B_{\mathcal F}[b_i,1]) = & \\
= & l \cdot \left(2l\cdot \sum_{i=0}^{l+1} \mu_x^1(J_i) + 2l\cdot \sum_{i=0}^{l+1} \mu_x^1(\varphi_{1}(J_i)) \right) \\
= & 2l \cdot \mu_{x}^1(B_{\mathcal F}[x,1]) = 2l.\end{align*}
as we wanted to show.
\end{proof}

\begin{lemma} \label{lemma:aux2}
Let $r>0$ be a fixed real number and  $x\in D$ an arbitrary point. Let $a_i := \varphi_{2i r}(\varphi_{-1}(x))$ and $b_i := \varphi_{2ir}(x)$ for $i=1,2,..., l+1$ where $l=\left \lfloor \frac{1}{2}\left(\frac{1}{r}-1\right)\right \rfloor$. Then
\begin{equation}\label{eq:statement}
\sum_{i=1}^{l+1}\mu_{x}^{1}(B_{\mathcal F}[a_i,1]) + \sum_{i=1}^{l}\mu_{x}^{1}(B_{\mathcal F}[b_i,1]) = 2l+2.\end{equation}
\end{lemma}
\begin{proof}
The proof is identical to the proof of Lemma \ref{lemma:aux}.
\end{proof}




\vspace{.3cm}
\textbf{Case 1: $D$ has full measure.} First of all, we will prove that in this case we must have $\underline{\Delta}\leq \overline{\Delta}<\infty$. Assume that $\overline{\Delta}=\infty$. Consider a typical fiber $\mathcal F(x)$ with $x\in D$ and take any $k\geq 1$ fixed. On Lemma \ref{lemma:aux} take $r:=\varepsilon_k$ and let $l, a_i, b_i$, $1\leq i\leq l$ be as in the statement of the respective lemma. 
For each $1\leq i \leq l$ we have
\begin{equation}
 \frac{\mu^1_{a_i}(B_{\mathcal F}[a_i,\varepsilon_k]))}{\varepsilon_k} \geq k  \Rightarrow  \mu^1_{a_i}(B_{\mathcal F}[a_i,\varepsilon_k]) \geq k \varepsilon_k,
\end{equation}
and similarly we obtain
\begin{equation}
\mu^1_{b_i}(B_{\mathcal F}[b_i,\varepsilon_k]) \geq k \varepsilon_k.\end{equation}
Now observe that 
\begin{align} \label{eq:xai} 
\mu_x^1(B_{\mathcal F}[a_i,\varepsilon_k])) = & \mu_x^1(B_{\mathcal F}[a_i,1])) \cdot \mu_{a_i}^1(B_{\mathcal F}[a_i,\varepsilon_k]) \\
\mu_x^1(B_{\mathcal F}[b_i,\varepsilon_k])) = & \mu_x^1(B_{\mathcal F}[b_i,1])) \cdot \mu_{b_i}^1(B_{\mathcal F}[b_i,\varepsilon_k]) \label{eq:xbi}
\end{align}
Taking the sum over $i$ we have
\begin{align*}
 \mu_x^1(B_{\mathcal F}[x,1]) \geq \sum_{i=1}^{l} \mu_x^1(B_{\mathcal F}[a_i,\varepsilon_k]) +& \sum_{i=1}^{l} \mu_x^1(B_{\mathcal F}[b_i,\varepsilon_k]) \\
= \sum_{i=1}^{l} \mu_x^1(B_{\mathcal F}[a_i,1])) \cdot \mu_{a_i}^1(B_{\mathcal F}[a_i,\varepsilon_k]) +& \sum_{i=1}^{l} \mu_x^1(B_{\mathcal F}[b_i,1])) \cdot \mu_{b_i}^1(B_{\mathcal F}[b_i,\varepsilon_k]),
\end{align*}
using ~\eqref{eq:ai} and ~\eqref{eq:bi} we get,
\[ \mu_x^1(B_{\mathcal F}[x,1]) \geq \left( \sum_{i=1}^{l} \mu_x^1(B_{\mathcal F}[a_i,1])) + \sum_{i=1}^{l} \mu_x^1(B_{\mathcal F}[b_i,1])) \right) \cdot k \varepsilon_k.\]
Thus, from the conclusion of Lemma ~\ref{lemma:aux} we have that
\[ \mu_x^1(B_{\mathcal F}[x,1]) \geq 2 \left \lfloor \frac{1}{2}\left(\frac{1}{\varepsilon_k}-1\right)\right \rfloor \cdot \varepsilon_k \cdot k.\]
As the left side is finite and the right side goes to infinity as $k$ goes to infinity we obtain a contradiction. Thus indeed $\overline{\Delta}$ is finite.

%
%We can take at least $i(k) = \lfloor{1/\varepsilon_k}\rfloor$ disjoint balls of radius $\varepsilon_k$ inside $B_{\mathcal F}[x,1]$. Let $b_1,b_2,...,b_{i(k)}$ be the centers of such balls. Then, for each $1\leq i \leq i(k)$
%\[k\cdot \varepsilon_k \leq \mu^1_x(B_{\mathcal F}[b_i,\varepsilon_k]) \Rightarrow i(k) \cdot k\cdot \varepsilon_k < \mu^1_x(B_{\mathcal F}[x,1]).\]
%Since $i(k)\cdot \varepsilon_k$ goes to $1$ as $k$ goes to infinity we have that $ \mu^1_x(B_{\mathcal F}[x,1]) =\infty$ yielding a contradiction. Thus, indeed $\overline{\Delta} <\infty$.


\begin{lemma} \label{lemma:uniformDelta}
\[\overline{\Delta} = \underline{\Delta} = 1. \]
\end{lemma}
\begin{proof}
For a given $k \in \mathbb N^{*}$, we know that for any $x \in \overline{\Pi}$
\begin{equation}\label{eq:all}
\left| \frac{\mu^1_x(B_{\mathcal F}(x,\varepsilon_k))}{\varepsilon_k} - \overline{\Delta}  \right|   \leq \frac{1}{k} .\end{equation} 

Consider the closed ball $B=B_{\mathcal F}[x,1] \subset \mathcal F(x)$. Given $\epsilon > 0$ take $k_0 \in \mathbb N$ such that $k_{0}^{-1} < \epsilon$. Let $r=\varepsilon_k$ and let $a_i, b_i$ be as in Lemma \ref{lemma:aux}. Thus, we have a family of disjoint balls inside $B_{\mathcal F}[x,1]$ centered at the points $a_i$ and $b_i$, $1\leq i \leq l:=\left \lfloor \frac{1}{2}\left( \frac{1}{\varepsilon_k}-1\right) \right \rfloor$.
%Now, take $k\in \mathbb N$ such that $\frac{1}{k} < \varepsilon_{k_0}$.  
%We need at least $s(k):=\lceil{1/\varepsilon_k}\rceil \leq 1/\varepsilon_k +1$ points, say $a_1,a_2,...,a_{s(k)}$, to cover $B$ with balls of radius $\varepsilon_k$. 
%As in ~\eqref{eq:ai} and ~\eqref{eq:bi} , for each $i$ we have
For each $1\leq i \leq l$ we have
\begin{equation}\label{eq:ai}
 \mu^1_{a_i}(B_{\mathcal F}[a_i,\varepsilon_k])) - \overline{\Delta}\varepsilon_k>- \epsilon \cdot \varepsilon_k \Rightarrow  \mu^1_{a_i}(B_{\mathcal F}[a_i,\varepsilon_k]) > \varepsilon_k(\overline{\Delta}-\epsilon),
\end{equation}
and similarly we obtain
\begin{equation}\label{eq:bi}
\mu^1_{b_i}(B_{\mathcal F}[b_i,\varepsilon_k]) > \varepsilon_k(\overline{\Delta}-\epsilon).\end{equation}
%\[ \mu^1_{a_i}(B_{\mathcal F}[a_i,\varepsilon_k])) - \overline{\Delta}\varepsilon_k>- \epsilon \cdot \varepsilon_k \Rightarrow  
%\begin{align*}
%\mu^1_{a_i}(B_{\mathcal F}[a_i,\varepsilon_k]) > & \varepsilon_k(\overline{\Delta}-\epsilon) \\
%\mu^1_{b_i}(B_{\mathcal F}[b_i,\varepsilon_k]) > & \varepsilon_k(\overline{\Delta}-\epsilon).\end{align*}
%Now observe that 
%\[\mu_x^1(B_{\mathcal F}[a_i,\varepsilon_k])) = \mu_x^1(B_{\mathcal F}[a_i,1])) \cdot \mu_{a_i}^1(B_{\mathcal F}[a_i,\varepsilon_k])\]
%\[\mu_x^1(B_{\mathcal F}[b_i,\varepsilon_k])) = \mu_x^1(B_{\mathcal F}[b_i,1])) \cdot \mu_{b_i}^1(B_{\mathcal F}[b_i,\varepsilon_k]).\]
Therefore, by ~\eqref{eq:xai} and \eqref{eq:xbi}, 
\begin{eqnarray*}
 1=\mu^1_x(B_{\mathcal F}[x,1]) & > & \left( \sum_{i=1}^{l}\mu_{x}^{1}(B_{\mathcal F}[a_i,1]) + \sum_{i=1}^{l}\mu_{x}^{1}(B_{\mathcal F}[b_i,1]) \right) \cdot \varepsilon_k\cdot ({\overline{\Delta}-\epsilon}) .\\
\end{eqnarray*}
By Lemma \ref{lemma:aux} we have
\begin{eqnarray*}
 1=\mu^1_x(B_{\mathcal F}[x,1]) & > & 2 \cdot \left\lfloor \frac{1}{2}\left( \frac{1}{\varepsilon_k}-1\right) \right \rfloor \cdot \varepsilon_k\cdot ({\overline{\Delta}-\epsilon}),\\
\end{eqnarray*}
for every $k\geq 1$. Taking $k\rightarrow \infty$ we have that $$\overline{\Delta}\leq 1.$$
Similarly, by taking $r:=\varepsilon_k$ and  $a_i$, $b_i$, $1\leq i \leq l:=\left \lfloor \frac{1}{2}\left( \frac{1}{\varepsilon_k}-1\right) \right \rfloor$, as in Lemma \ref{lemma:aux2} we cover $B_{\mathcal F}[x,1]$ with $2l+2$ balls of radius $\varepsilon_k$. Now, we know that
\begin{align*}\mu^1_{a_i}(B_{\mathcal F}[a_i,\varepsilon_k])< & \epsilon\cdot \varepsilon_k + \overline{\Delta}\varepsilon_k \\
\mu^1_{b_i}(B_{\mathcal F}[b_i,\varepsilon_k])< & \epsilon\cdot \varepsilon_k + \overline{\Delta}\varepsilon_k,\end{align*}
%and that 
%\[\mu_x^1(B_{\mathcal F}[a_i,\varepsilon_k])) = \mu_x^1(B_{\mathcal F}[a_i,1])) \cdot \mu_{a_i}^1(B_{\mathcal F}[a_i,\varepsilon_k])\]
%\[\mu_x^1(B_{\mathcal F}[b_i,\varepsilon_k])) = \mu_x^1(B_{\mathcal F}[b_i,1])) \cdot \mu_{b_i}^1(B_{\mathcal F}[b_i,\varepsilon_k]),\]
for $1\leq i \leq \left \lfloor \frac{1}{2}\left( \frac{1}{\varepsilon_k}-1\right) \right \rfloor$.
Consequently, again using ~\eqref{eq:xai} and ~\eqref{eq:xbi}, we have
\[1= \mu^1_x(B_{\mathcal F}[x,1]) < \left( \sum_{i=1}^{l+1}\mu_{x}^{1}(B_{\mathcal F}[a_i,1]) + \sum_{i=1}^{l+1}\mu_{x}^{1}(B_{\mathcal F}[b_i,1]) \right) \cdot \varepsilon_k(\epsilon+\overline{\Delta}). \]
By Lemma \ref{lemma:aux2}
\[1= \mu^1_x(B_{\mathcal F}[x,1]) < (2l +2) \cdot \varepsilon_k(\epsilon+\overline{\Delta}) \Rightarrow \overline{\Delta}\leq 1.\]
Consequently we have that $\overline{\Delta} = 1$. Repeating the same argument with $\underline{\Delta}$ we conclude that 
\[\overline{\Delta} = \underline{\Delta} =1.\]
\end{proof}

Next we are able to conclude that $\mu^1_x$ is equivalent to the measure induced by the flow $\phi$ on the orbits 

\begin{lemma} \label{lemma:equallebesgue}
For almost every $x\in X$
\[\mu^1_x(B) = 2^{-1} \cdot \lambda_{\mathcal F(x)}(B)\]
where $\lambda_{\mathcal F(x)}$ is the measure on $\mathcal F(x)$ induced by the flow $\phi$ (i.e. $\lambda_{\mathcal F(x)}([x,y])=|t|$ if $y = \phi_t(x)$)
\end{lemma}
\begin{proof}
Take any typical plaque $\mathcal F(x)$ and any point $a\in \mathcal F(x)$. For each $r>0$ we can write the set $[a,\phi_r(a)]$ as a disjoint union as below
\[[a,\phi_r(a)] = \left(\bigcup_{j=0}^n [\phi_{2j\varepsilon_k}(a),\phi_{2(j+1)\varepsilon_k}(a)] \right) \cup J_k, \quad n:= \left \lfloor r/2\varepsilon_k \right \rfloor \]
where $J_k = [\phi_{2(n+1)\varepsilon_k}(a),\phi_r(a)]$. Each of the terms appearing on the right side of the previous equality, except for $J_k$, is a closed $\mathcal F$-ball of radius $\varepsilon_k$. By Lemma \ref{lemma:uniformDelta}, $\overline{\Delta} = \underline{\Delta}=1$ so
\[\varepsilon_k(1-1/k)<\mu_{c_j}^1([\phi_{2j\varepsilon_k}(a),\phi_{2(j+1)\varepsilon_k}(a)]) < \varepsilon_k(1+1/k),\]
where $c_j:=\phi_{2j\varepsilon_k}(a)+\varepsilon_k$, $j=0,1,\ldots, n$.
Also, we know that 
\[\mu_x^1(B_{\mathcal F}[c_j,\varepsilon_k]) = \mu_x^1(B_{\mathcal F}[c_j,1]) \cdot \mu_{c_j}^1(B_{\mathcal F}[c_j,\varepsilon_k]).\]
Therefore we have
\begin{align*}
\left( \sum_{j=0}^{n} \mu_x^1(B_{\mathcal F}[c_j,1]) \right) \cdot \varepsilon_k(1-1/k)  \leq &\quad \mu_x^1[a,\phi_r(a)] \leq   \\
\leq & \left( \sum_{j=0}^{n} \mu_x^1(B_{\mathcal F}[c_j,1]) \right)  \cdot \varepsilon_k(1+1/k) + \mu_x^1(J_k). \end{align*}
Repeating the argument of the proof of Lemma \ref{lemma:aux}, we see that $ \sum_{j=0}^{n} \mu_x^1(B_{\mathcal F}[c_j,1]) =n+1$. Thus
\[ \lfloor r/2\varepsilon_k \rfloor \cdot \varepsilon_k(1-1/k)  \leq  \mu_x^1([a,\phi_r(a)]) \leq \lfloor r/2\varepsilon_k \rfloor \cdot \varepsilon_k(1+1/k) + \mu_x^1(J_k).\]
Taking $k\rightarrow +\infty$ we have
\[\mu_x^1([a,\phi_r(a)]) = r/2\]
as we wanted to show.
%
%
%
%
%since the estimative \tco{red}{(RV: checar palavra)} holds for every point in $R$, we have
%\[\varepsilon_k\cdot (\overline{\Delta}-1/k) + \ldots + \varepsilon_k\cdot (\overline{\Delta}-1/k)+\mu_x(J_k)<\mu_x([a,\phi_{r}(a)]) \]
%and
%\[\mu_x([a,\phi_r(a)]) < \varepsilon_k\cdot (\overline{\Delta}+1/k) + \ldots + \varepsilon_k\cdot (\overline{\Delta}+1/k) + \mu_x(J_k) \]
%which implies 
% \[\mu_x([a,\phi_r(a)]) = 2^{-1}r.\]
%% \textcolor{red}{$2C$ ou $C$?}
% By the definition of $\lambda_x$ the result follows.
\end{proof}


\textbf{Case 2: $D$ has null measure.}
Since $\overline \Pi_x$ is closed in the plaque $\mathcal F(x)$ for all $x \in D$ and $\mu(\overline \Pi)=1$, it is true that for a full measurable set $\mathfrak D$, if $ x\in \mathfrak D$ then $x\notin \overline{\Pi}$ if, and only if, there is $r>0$ with $\mu^1_x(B_{\mathcal F}(x,r)) = 0$. Now consider $\{q_1,q_2,...\}$ to be an enumeration of the rationals.
 
For each $i\geq 1$ let us define the function $S_i$ as
\[S_i(x) = \max\{q_j: 1\leq j \leq i \text{ and } \mu^1_y(B_{\mathcal F}(y,q_j))=0 \text{ for some } y \in \mathcal F(x) \}.\]

\begin{lemma}
 $S_i$ is an invariant measurable function for all $i \in \mathbb N$.
\end{lemma}
\begin{proof}
For each $i \in \mathbb N$ define the function $Q_i:  \mathfrak D \rightarrow [0,\infty)$ by
\[Q_i(x) = \mu^1_x(B_{\mathcal F}(x,q_i)).\]
By proposition \ref{prop:disinmeasurable2} $Q_i(x)$ is a measurable function for every $i$ and, by an standard measure theory argument, we may take a compact set $K \subset \mathfrak D$ of positive measure such that $Q_i | K$ is continuous for every $i$. Now, given $j \in \mathbb N$, let $\sigma$ be a permutation of $\{1,...,j\}$ such that $q_{\sigma(1)}<q_{\sigma(2)}<...<q_{\sigma(j)}$. 
Observe that for $$\mathfrak K = \bigcup_{n\in \mathbb Z} f^n(K)$$ we have
\[S_j^{-1}(\{q_{\sigma(j)}\}) \cap \mathfrak K = \bigcup_{n\in \mathbb Z} f^n ( \mathcal F(Q_{\sigma(j)}^{-1}(\{0\}) \cap K )),\]
which is a measurable set since $Q_{\sigma(n)}^{-1}(\{0\}) \cap K $ is a Borel set. Now,
\[S_j^{-1}(\{q_{\sigma(j-1)}\}) \cap \mathfrak K = \bigcup_{n\in \mathbb Z} f^n ( \mathcal F(Q_{\sigma(j-1)}^{-1}(\{0\}) \cap K )) \setminus (S_j^{-1}(\{q_{\sigma(j)}\}) \cap \mathfrak K),\]
which is also a measurable set. Inductively we prove that $S_j^{-1}(\{q_{\sigma(i)}\}) \cap \mathfrak K$ is measurable for all $1\leq i \leq j$. Since, by ergodicity, the set $\mathfrak K \subset \mathfrak D$ has full measure we conclude that $S_j(x)$ is measurable for every $j \geq 1$.
\end{proof}

Let $S(x) := \lim_{i\rightarrow \infty} S_i(x)$. $S$ is measurable and $f$-invariant thus it is constant almost everywhere, call this constant $r_0$. This means that for a full measure set $Y \subset \mathfrak D$, for every $x\in Y$ the plaque $\mathcal F(x)$ has a finite number of intervals of radius $r_0$ outside $\overline \Pi_x$. Let us call these intervals as ``bad" intervals.

Now consider the set $\mathfrak M$ formed by the median points of these ``bad" intervals of radius $r_0$. Notice that $\mathfrak M$ is a measurable set, since it is inside a set of zero measure and also that $f(\mathfrak M)=\mathfrak M$. 
%because of the invariance of the Souslin metric system.

Let $\varphi:(0,1) \times (0,1)^k \rightarrow U$ be a local chart for $\mathcal F$ such that the $\mathcal F(\mathfrak M \cap U)$, the $\mathcal F$ saturation of these ``bad" intervals inside $U$, has positive measure. Set $\Sigma := \pi_1(\varphi^{-1}(\mathfrak M\cap U))$, where $\pi_1:(0,1) \times (0,1)^k \rightarrow (0,1)$ is the projection onto the first coordinate. Now we may apply The Measurable Choice Theorem \ref{theo:MCT} to obtain a measurable function $ \mathfrak F: \Sigma \rightarrow (0,1)$ such that $(x,\mathfrak F(x)) \in  \varphi^{-1}(\mathfrak M\cap U)$ for all $x \in \Sigma$. Again by standard arguments, using Lusin's theorem, we may assume $\Sigma$ to be compact and such that $\mathfrak F$ is a continuous function.

Now consider the set $\mathfrak M_0:= \varphi(\text{graph }\mathfrak F)$, which is a Borel set since the graph of $\mathfrak F$ is a compact set. Notice that our construction implies that $\mathcal F(\mathfrak M_0)$ has positive measure. Now define the following $f$ invariant set $$\mathfrak M_1:= \bigcup_{n \in \mathbb Z} f^n(\mathfrak M_0).$$ By ergodicity the set $\mathcal F (\mathfrak M_1)$ has full measure.

% \begin{itemize}
% \item For a full measure set, $x\notin \overline{\Pi}$ if, and only if, there is $r>0$ with $\mu_x(B_{d_x}(x,r)) = 0$;
% \item Take an enumeration of $\mathbb Q = \{q_1,q_2,...\}$.
% \item Let $S_i$ be the function:
% \[S_i(x) = \max\{q_j: 1\leq j \leq i \text{ and } \mu_y(B_{d_y}(y,q_j))=0 \text{ for some } y \in \mathcal F(x) \}.\]
% This function is measurable since $(x,r) \mapsto \mu_x(B_{d_x}(x,r))$ is measurable.
% \item  Let $S(x) = \lim_{i\rightarrow \infty} S_i(x)$. $S$ is measurable and clearly $f$-invariant. Thus it is constant almost everywhere. This means that for a full measure set $Y$ , for every $x\in Y$ the plaque $\mathcal F(x)$ has (a finite number of) bad intervals.
% \item We now use the measurable choice theorem to take a measurable function $g: Y/\mathcal F \rightarrow Y$ such that $g(y)$ is inside the union of bad intervals. Since $g$ is  measurable, it is continuous on a compact set $K\subset Y/\mathcal F$ of positive (projected) measure. Consequently, the image $g(K)$ is a Borel set. Take $K' = \bigcup f^n(K)$ and we work with the set of points $P = \bigcup f^n(g(K))$. Since it is a Borel set, we can flow it and obtain a Souslin set for every $r$.
% We take the first $r$ where it jumps from zero to positive measure and we finish the argument.
% \end{itemize}

The set $\mathfrak M_1$ intersects almost each plaque in a finite (constant) number of points. Notice that, for each $r \in \mathbb R_+$ the invariant set $$\mathfrak M_1^r:= \bigcup_{x \in \mathfrak M_1} B_{\mathcal F}(x,r)$$ has zero or full measure. Let $\alpha_0$ such that $\mu(\mathfrak M_1^r)=0$ if $r < \alpha_0$ and $\mu(\mathfrak M_1^r)=1$ if $r \geq \alpha_0$. This implies that the extreme points of $B_{\mathcal F}(x,\alpha_0)$ for $x \in \mathfrak M_1$ forms a set of atoms.


Which is an absurd, because we are assuming we are in the non-atomic case. The case $\overline{\Delta}=\infty$ is similar. 

The measure $\mu_{\mathcal F(x)}$ of the statement of the result is given, due to Lemma \ref{lemma:equallebesgue}, as $2^{-1}\lambda_{\mathcal F(x)}$.

Let us now work with the case $\mu(Per(\phi))>0$. By ergodicity of $f$ and the $f$ invariance of $Per(\phi)$ we have $\mu(Per(\phi))=1$. Hence the partition of $X$ given by each periodic orbit of $\phi$ and the set $X\setminus Per(\phi)$ forms a measurable partition (e.g. \cite[Proposition 2.5]{PTV}). We can then disintegrate $\mu$ on this partition. Denote the family of conditional measures by $\{\mu_{\mathcal F(x)}\}$. If the set of singularities for the flow $\phi$ have positive measure then it is clear that one should have full measure, in particular the measure is atomic and we fall on the first item. If not then we repeat the prove but instead of working with $\mu_x^1$ we can simply work with the disintegrated measures $\mu_{\mathcal F(x)}$ and the theorem follows.

$\hfill \square$


% \section{Proofs of Theorem \ref{theo:ergodic.measure.flows} and Corollary \ref{cor:rigidity.conjugation}}\label{sec:proof.thm.intro}
% 
% \subsection{Proof of Theorem \ref{theo:ergodic.measure.flows}}
% 
% We apply Theorem \ref{theo:continuous.flow} and because the disintegration is not atomic, then there is a class of disintegration described on item 2 of Theorem \ref{theo:continuous.flow}. Because the measures $\mu_{\mathcal F(x)}$ are flow invariant the measure $\mu$ is also flow invariant.
% 
% % If $\mu(Per(\phi))>0$, by ergodicity $\mu(Per(\phi))=1$. Also the ergodicity and because $f(\mathcal F(x))=\mathcal F(f(x))$, then $\mu$ gives full measure for some periodic orbit. Now the proof of Theorem \ref{theo:continuous.flow} follow directly for this case where instead of considering the disintegration $\mu^1_x$ as in the proof one repeat the whole proof but works directly with $\mu$.
% 
% $\hfill \square$
% 
% \subsection{Proof of Corollary \ref{cor:rigidity.conjugation}}
% 
% Use the conjugacy $h:X \rightarrow Y$ to induce a flow $\widetilde \phi$ on $Y$ as the push forward of $\phi$, that is $\widetilde \phi(t,h(x))=h(\phi(t,x))$. We want to prove that the flow $\widetilde \phi$ and $\psi$ are the same: 
% % The theorem is reduced assuming that $\phi$ and $\psi$ are on $Y$ and $\phi_1=\psi_1$. We want to prove that these are the same flows, 
% that is $\widetilde \phi_t = \psi_t$ for all $t \in \mathbb R$.
% 
% Because $h$ is a conjugacy of the $\phi_1$ and $\psi_1$ and $h$ sends orbits of the flow $\phi$ to the orbits of the flow $\psi$, then $\widetilde \phi_1=\psi_1$ and their orbits to be the same. Also because $\mu$ is ergodic for $\psi_1$ it is, of course, ergodic for $\widetilde \phi_1$ and Theorem \ref{theo:ergodic.measure.flows} implies that $\mu$ is invariant for the flows $\phi$ and $\psi$, since its disintegration on the foliation given by the orbits is not atomic. But notice that the disintegration determines the flow, hence the flows $\phi$ and $\psi$ are the same in a set of full $\mu$ measure, but since $\mu$ has full support these flows coincide in a dense set, since they are continuous flows they must be the same.
% 
% 
% % 
% % item two to both of these measures, them 
% % 
% % The topological conjugacy of the time-1 maps implies that the $h$ sends orbits of $\phi$ into orbits of $\psi$. Applying the second item of Theorem \ref{theo:continuous.flow} to the measures $\mu$ and $\nu$, we obtain respectively the disintegration measures given by item 2: $\{\mu_x\}_{x \in X}$ and $\{\nu_y\}_{y \in Y}$. Notice that $\psi_1(h(x)) = h(\phi_1(x))$, this implies that $$\psi_t(h(x)) = h(\phi_t(x)),$$
% 
% $\hfill \square$




% \subsection{Proof of Theorem \ref{theo:rigidity.flows}}
% 
% Because the time
% 
% $\hfill \square$

\section*{Acknowledgements}

We would like to thank Ali Tahzibi for usefull conversations. G.P. was partially supported by FAPESP grant 2016/05384-0 and R.V. was partially supported by FAPESP grant 2016/22475-9.
\bibliographystyle{plain}
\bibliography{Referencias.bib}
\end{document}
