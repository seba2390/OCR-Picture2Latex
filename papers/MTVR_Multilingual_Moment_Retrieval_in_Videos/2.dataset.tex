\section{Dataset}\label{sec:dataset}
The TVR~\cite{lei2020tvr} dataset contains 108,965 high-quality English queries from 21,793 videos from 6 long-running TV shows (provided by TVQA~\cite{Lei2018TVQALC}). The videos are associated with English dialogues in the form of subtitle text. \DsetName~extends this dataset with translated dialogues and queries in Chinese to support multilingual multimodal research.



\subsection{Data Collection}

\paragraph{Dialogue Subtitles.}
We crawl fan translated Chinese subtitles from subtitle sites.\footnote{\url{https://subhd.tv}, \url{http://zimuku.la}} 
All subtitles are manually checked by the authors to ensure they are of good quality and are aligned with the videos.
The original English subtitles come with speaker names from transcripts that we map to the Chinese subtitles, to ensure that the Chinese subtitles have the same amount of information as the English version. 


\paragraph{Query.}
To obtain Chinese queries, we hire human translators from Amazon Mechanical Turk (AMT).
Each AMT worker is asked to write a Chinese translation of a given English query.
Languages are ambiguous, hence we also present the original videos to the workers at the time of translation to help clarify query meaning via spatio-temporal visual grounding. The Chinese translations are required to have the exact same meaning as the original English queries and the translation should be made based on the aligned video content.
To facilitate the translation process, we provide machine translated Chinese queries from Google Cloud Translation\footnote{\url{https://cloud.google.com/translate}} as references, similar to~\cite{wang2019vatex}. 
To find qualified bilingual workers in AMT, we created a qualification test with 5 multiple-choice questions designed to evaluate workers' Chinese language proficiency and their ability to perform our translation task.
We only allow workers that correctly answer all 5 questions to participate our annotation task.
In total, 99 workers finished the test and 44 passed, earning our qualification.
To further ensure data quality, we also manually inspect the submitted results during the annotation process and disqualify workers with poor annotations.
We pay workers \$0.24 every three sentences, this results in an average hourly pay of \$8.70. 
The whole annotation process took about 3 months and cost approximately \$12,000.00. 



\begin{table}[!t]
\centering
\small
\setlength{\tabcolsep}{2pt}
\renewcommand{\arraystretch}{1.3}
\scalebox{0.87}{
\begin{CJK*}{UTF8}{gbsn}
\begin{tabular}{ll}
\toprule
QType (\%) & \multicolumn{1}{c}{Query Examples (in English and Chinese)} \\
\midrule
video-only & Howard places his plate onto the coffee table. \\
(74.2)  & 霍华德将盘子放在咖啡桌子上。\\
\midrule
sub-only & Alexis and Castle talk about the timeline of the murder. \\
 (9.1) & 
亚历克西斯和卡塞尔谈论谋杀的时间顺序。
 \\
\midrule
video+sub & Joey waives his hand when he asks for his food.  \\
(16.6) & 
乔伊催餐时摆了摆手。\\
\bottomrule
\end{tabular}
\end{CJK*} 
}
\caption{
\DsetName~English and Chinese query examples in different query types. 
The percentage of the queries in each query type is shown in brackets.
}
\label{tab:qtype_examples}
\end{table}


\subsection{Data Analysis}
In Table~\ref{tab:en_zh_data_comparison}, we compare the average sentence lengths and the number of unique words under different part-of-speech (POS) tags, between the two languages, English and Chinese, and between query and subtitle text.
For both languages, dialogue subtitles are linguistically more diverse than queries, i.e., they have more unique words in all categories. 
This is potentially because the language used in subtitles are unconstrained human dialogues while the queries are collected as declarative sentences referring to specific moments in videos~\cite{lei2020tvr}. 
Comparing the two languages, the Chinese data is typically more diverse than the English data.\footnote{
\begin{CJK*}{UTF8}{gbsn}
The differences might be due to the different morphemes in the languages.
E.g., the Chinese word 长发\; (`long hair') is labeled as a single noun, but as an adjective (`long') and a noun (`hair') in English~\cite{wang2019vatex}.
\end{CJK*}
}
In Table~\ref{tab:qtype_examples}, we show English and their translated Chinese query examples in Table~\ref{tab:qtype_examples}, by query type.
In the appendix, we compare \DsetName~with existing video and language datasets.






\begin{table}[]
\centering
\small
\scalebox{0.86}{
\begin{tabular}{lrrrrrr}
\toprule
\multirow{2}{*}{ Data } & \multicolumn{1}{c}{Avg} & \multicolumn{5}{c}{ \#unique words by POS tags } \\ \cmidrule(l){3-7}
& \multicolumn{1}{c}{Len} & \multicolumn{1}{c}{all} & \multicolumn{1}{c}{verb}  & \multicolumn{1}{c}{noun} & \multicolumn{1}{c}{adj.} & \multicolumn{1}{c}{adv.} \\
\midrule
\multicolumn{2}{l}{\textbf{English}} & & & &  &  \\
Q & 13.45 & 15,201 & 3,015 & 7,143 & 2,290 & 763 \\
Sub & 10.78 & 49,325 & 6,441 & 19,223 & 7,504 & 1,740 \\
Q+Sub & 11.27 & 52,545 & 7,151 & 20,689 & 8,021 & 1,976 \\
\midrule
\multicolumn{2}{l}{\textbf{Chinese}} & & & &  &  \\
Q & 12.55 & 34,752 & 12,773 & 18,706 & 1,415 & 1,669 \\
Sub & 9.04 & 101,018 & 36,810 & 53736 & 4,958 & 5,568 \\
Q+Sub & 9.67 & 117,448 & 42,284 & 62,611 & 5,505 & 6,185 \\
\bottomrule
\end{tabular}
}
\vspace{-3pt}
\caption{Comparison of English and Chinese data in~\DsetName. We show average sentence length, and number of unique tokens by POS tags, for Query (\textit{Q}) and or Subtitle (\textit{Sub}).}
\label{tab:en_zh_data_comparison}
\vspace{-8pt}
\end{table}

