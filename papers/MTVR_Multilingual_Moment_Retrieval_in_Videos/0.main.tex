%
% File acl2021.tex
%
%% Based on the style files for EMNLP 2020, which were
%% Based on the style files for ACL 2020, which were
%% Based on the style files for ACL 2018, NAACL 2018/19, which were
%% Based on the style files for ACL-2015, with some improvements
%%  taken from the NAACL-2016 style
%% Based on the style files for ACL-2014, which were, in turn,
%% based on ACL-2013, ACL-2012, ACL-2011, ACL-2010, ACL-IJCNLP-2009,
%% EACL-2009, IJCNLP-2008...
%% Based on the style files for EACL 2006 by 
%%e.agirre@ehu.es or Sergi.Balari@uab.es
%% and that of ACL 08 by Joakim Nivre and Noah Smith

\documentclass[11pt,a4paper]{article}
\usepackage[hyperref]{acl2021}
\usepackage{times}
\usepackage{latexsym}
\renewcommand{\UrlFont}{\ttfamily\small}


% This is not strictly necessary, and may be commented out,
% but it will improve the layout of the manuscript,
% and will typically save some space.
\usepackage{microtype}

\aclfinalcopy % Uncomment this line for the final submission
%\def\aclpaperid{***} %  Enter the acl Paper ID here

%\setlength\titlebox{5cm}
% You can expand the titlebox if you need extra space
% to show all the authors. Please do not make the titlebox
% smaller than 5cm (the original size); we will check this
% in the camera-ready version and ask you to change it back.


%%%%%%%%%%%%%%%% Customized Area [START]
\usepackage{CJKutf8} % use Chinese, see https://www.overleaf.com/learn/latex/chinese

\usepackage{times}
\usepackage{epsfig}
\usepackage{graphicx}
\usepackage{amsmath}
\usepackage{amssymb}
\usepackage{array}

\usepackage{longtable}

\definecolor{demphcolor}{RGB}{144,144,144}
\newcommand{\demph}[1]{\textcolor{demphcolor}{#1}}

%%% set row text color [START]
% https://tex.stackexchange.com/questions/26360/how-to-color-the-font-of-a-single-row-in-a-table
\makeatletter
\newcommand*{\@rowstyle}{}

\newcommand*{\rowstyle}[1]{% sets the style of the next row
  \gdef\@rowstyle{#1}%
  \@rowstyle\ignorespaces%
}

\newcolumntype{=}{% resets the row style
  >{\gdef\@rowstyle{}}%
}

\newcolumntype{+}{% adds the current row style to the next column
  >{\@rowstyle}%
}
\makeatother
% Usage: note the '=l', '+l', and rowstyle
% \begin{tabular}{ =l | +l +l +l +l }
%   \rowstyle{\color{red}}
%       & 1 & 2 & 3 & 4 \\
%   \hline
%   1   & A & B & C & D \\
%   2   & A & B & C & D \\
%   3   & A & B & C & D \\
%   4   & A & B & C & D \\
% \end{tabular} 

%%% set row text color [END]


%%%%%%%%%%%% define thicker \hline with \thickhline [START]
\makeatletter
\def\thickhline{%
  \noalign{\ifnum0=`}\fi\hrule \@height \thickarrayrulewidth \futurelet
   \reserved@a\@xthickhline}
\def\@xthickhline{\ifx\reserved@a\thickhline
               \vskip\doublerulesep
               \vskip-\thickarrayrulewidth
             \fi
      \ifnum0=`{\fi}}
\makeatother

\newlength{\thickarrayrulewidth}
\setlength{\thickarrayrulewidth}{2\arrayrulewidth}
% define thicker \hline with \thickhline [END]

\usepackage{multirow}
\usepackage{booktabs}

\newcommand{\rulesep}{\unskip\ \vrule\ }
\usepackage{pifont}% http://ctan.org/pkg/pifont
\newcommand{\cmark}{\ding{51}}%
\newcommand{\xmark}{\ding{55}}%

% down arrow $\downarrow$, used in math mode
\newcommand{\xdownarrow}[1]{%
  {\left\downarrow\vbox to #1{}\right.\kern-\nulldelimiterspace}
}

\usepackage{xcolor}
% define colors from here: http://latexcolor.com/
\definecolor{ForestGreen}{rgb}{0.13, 0.55, 0.13}
\definecolor{bittersweet}{rgb}{1.0, 0.44, 0.37}
\definecolor{orchid}{rgb}{0.478, 0.506, 1.0}
\definecolor{salmon}{rgb}{0.968, 0.503, 0.408}
\definecolor{green}{rgb}{0.438, 0.678, 0.408}
\definecolor{orange}{rgb}{1.0, 0.75, 0.}


\usepackage{colortbl}
\definecolor{Gray}{gray}{0.85}

\def\DsetName{\textsc{mTVR}}
\def\ModelName{mXML}

%%%%%%%%%%%%% Customized Area [END]


\newcommand\BibTeX{B\textsc{ib}\TeX}

\title{\DsetName: Multilingual Moment Retrieval in Videos}


\author{
  Jie Lei $\;\;\;\;\;$ 
  Tamara L. Berg $\;\;\;\;\;$ Mohit Bansal \\
  Department of Computer Science \\ University of North Carolina at Chapel Hill \\
  {\tt \{jielei, tlberg, mbansal\}@cs.unc.edu} \\
}


% \date{}

\begin{document}
\maketitle
\begin{abstract}
We introduce \DsetName, a large-scale multilingual video moment retrieval dataset, containing 218K English and Chinese queries from 21.8K TV show video clips.
The dataset is collected by extending the popular TVR dataset (in English) with paired Chinese queries and subtitles. 
Compared to existing moment retrieval datasets, \DsetName~is multilingual, larger, and comes with diverse annotations.
We further propose \ModelName, a multilingual moment retrieval model that learns and operates on data from both languages, via encoder parameter sharing and language neighborhood constraints.
We demonstrate the effectiveness of \ModelName~on the newly collected \DsetName~ dataset, where \ModelName~outperforms strong monolingual baselines while using fewer parameters.
In addition, we also provide detailed dataset analyses and model ablations.
Data and code are publicly available at \url{https://github.com/jayleicn/mTVRetrieval}
\end{abstract}

%%%%%%%%% BODY TEXT
\section{Introduction}
\label{sec:intro}
Object localization aims to find the area of a target object in a given image~\cite{ren2015faster,russakovsky2015imagenet,duan2019centernet,lin2017feature,tan2020efficientdet}. However, fully supervised approaches require accurate bounding box annotations, which require a tremendous cost. Weakly supervised object localization (WSOL) has been a great alternative because it requires only image-level labels to train a localization model~\cite{singh2017hide,choe2019attention,choe2020evaluation,pan2021unveiling, xue2019danet}.

The most commonly used approach for WSOL is a class activation map (CAM)~\cite{zhou2016learning}. CAM-based methods employ a global average pooling (GAP) layer~\cite{lin2013network} followed by a fully connected (FC) layer, and generate a CAM with the feature maps prior to the GAP layer.
A highly activated area in a CAM is predicted to be an object location.
However, it is widely observed that CAM identifies only the most discriminative parts of an object rather than the entire object area, resulting in low localization performance~\cite{mai2020erasing,lee2021anti,zhang2018adversarial}.

\begin{figure}[t]
	\centering
    \includegraphics[width=0.95\columnwidth]{figures/fig_first.pdf}
    \vspace{-0.5em}
    \caption{(a) Examples of CAM and decomposed terms from the classifier trained with the vanilla method~\cite{zhou2016learning} and with EIL~\cite{mai2020erasing}. (b) Visualization of the changes of CAM and decomposed terms as training with our method progresses.}
    \label{fig:first}
\end{figure}

We ask the question, ``\textit{Why does CAM generated from an accurate classifier fail to highlight the entire object area?}''
To answer this, we provide a new perspective of decomposing CAM into two terms: (1) activation in a feature map and (2) cosine similarity between the feature vector at each spatial location and the class-specific weight in the FC layer.
Fig.~\ref{fig:first}(a) shows that only the bird's body is highly activated in the CAM of the vanilla model, leaving the wing less activated. However, looking at the activation in the feature map, the wing as well as the body is highly activated.
The low similarity of the wing region offsets the activation in the feature map, making the region invisible in the CAM.
Here, we find that the low cosine similarity, \ie, misalignment of feature directions to the class-specific weights, prevents the less discriminative part belonging to a target object from being highly activated in a CAM.
This is because training for classification only considers the feature averaged over all locations, not the feature at each spatial location.
This brings the gap between classification and localization.

Although various approaches have been proposed to expand the activated region to the entire object area in a CAM~\cite{zhang2018adversarial,choe2019attention,mai2020erasing,yun2019cutmix,xue2019danet,zhang2018self}, none of them discovered or mitigated the misalignment. Fig.~\ref{fig:first}(a) shows that EIL~\cite{mai2020erasing}, one of those approaches, expands the activated region in the feature map. However, it fails to increase the similarity in the object region; hence, the expansion effect is not as large in the CAM as in the activation of the feature map.

To bridge the gap between classification and localization, we propose feature direction alignment, a method to enhance the alignment of feature directions in the entire object region to the directions of class-specific weights while discouraging the alignment in the background region.
We also introduce consistency with attentive dropout, which ensures that the target object region has uniformly high activation in the feature map.
Fig.~\ref{fig:first}(b) shows that our method gradually aligns the feature directions to the class-specific weight as the training progresses.
The alignment results in high activation of less discriminative regions, \eg, wing, in the CAM, enabling accurate localization of the entire object.
We evaluate our method on the most widely used WSOL benchmark datasets: CUB-200-2011~\cite{welinder2010caltech} and ImageNet-1K~\cite{russakovsky2015imagenet}.
Our method achieves a state-of-the-art localization performance for both datasets.

The contributions of this paper can be summarized as follows:
\begin{itemize}
\setlength{\itemsep}{2pt}
\vspace{-3pt}
	\item[$\bullet$] We interpret a CAM in terms of the degree of alignment between the direction of input features and the direction of class-specific vectors, and find the gap between classification and localization.
	\vspace{-2pt}
	\item[$\bullet$] We propose a method to bridge the gap between classification and localization by aligning feature directions with class-specific weights.
	\vspace{-2pt}
	\item[$\bullet$] We demonstrate that our proposed method outperforms other state-of-the-art WSOL methods on the CUB-200-2011 and ImageNet-1K datasets.
\end{itemize}
\section{Dataset}\label{sec:dataset}
The TVR~\cite{lei2020tvr} dataset contains 108,965 high-quality English queries from 21,793 videos from 6 long-running TV shows (provided by TVQA~\cite{Lei2018TVQALC}). The videos are associated with English dialogues in the form of subtitle text. \DsetName~extends this dataset with translated dialogues and queries in Chinese to support multilingual multimodal research.



\subsection{Data Collection}

\paragraph{Dialogue Subtitles.}
We crawl fan translated Chinese subtitles from subtitle sites.\footnote{\url{https://subhd.tv}, \url{http://zimuku.la}} 
All subtitles are manually checked by the authors to ensure they are of good quality and are aligned with the videos.
The original English subtitles come with speaker names from transcripts that we map to the Chinese subtitles, to ensure that the Chinese subtitles have the same amount of information as the English version. 


\paragraph{Query.}
To obtain Chinese queries, we hire human translators from Amazon Mechanical Turk (AMT).
Each AMT worker is asked to write a Chinese translation of a given English query.
Languages are ambiguous, hence we also present the original videos to the workers at the time of translation to help clarify query meaning via spatio-temporal visual grounding. The Chinese translations are required to have the exact same meaning as the original English queries and the translation should be made based on the aligned video content.
To facilitate the translation process, we provide machine translated Chinese queries from Google Cloud Translation\footnote{\url{https://cloud.google.com/translate}} as references, similar to~\cite{wang2019vatex}. 
To find qualified bilingual workers in AMT, we created a qualification test with 5 multiple-choice questions designed to evaluate workers' Chinese language proficiency and their ability to perform our translation task.
We only allow workers that correctly answer all 5 questions to participate our annotation task.
In total, 99 workers finished the test and 44 passed, earning our qualification.
To further ensure data quality, we also manually inspect the submitted results during the annotation process and disqualify workers with poor annotations.
We pay workers \$0.24 every three sentences, this results in an average hourly pay of \$8.70. 
The whole annotation process took about 3 months and cost approximately \$12,000.00. 



\begin{table}[!t]
\centering
\small
\setlength{\tabcolsep}{2pt}
\renewcommand{\arraystretch}{1.3}
\scalebox{0.87}{
\begin{CJK*}{UTF8}{gbsn}
\begin{tabular}{ll}
\toprule
QType (\%) & \multicolumn{1}{c}{Query Examples (in English and Chinese)} \\
\midrule
video-only & Howard places his plate onto the coffee table. \\
(74.2)  & 霍华德将盘子放在咖啡桌子上。\\
\midrule
sub-only & Alexis and Castle talk about the timeline of the murder. \\
 (9.1) & 
亚历克西斯和卡塞尔谈论谋杀的时间顺序。
 \\
\midrule
video+sub & Joey waives his hand when he asks for his food.  \\
(16.6) & 
乔伊催餐时摆了摆手。\\
\bottomrule
\end{tabular}
\end{CJK*} 
}
\caption{
\DsetName~English and Chinese query examples in different query types. 
The percentage of the queries in each query type is shown in brackets.
}
\label{tab:qtype_examples}
\end{table}


\subsection{Data Analysis}
In Table~\ref{tab:en_zh_data_comparison}, we compare the average sentence lengths and the number of unique words under different part-of-speech (POS) tags, between the two languages, English and Chinese, and between query and subtitle text.
For both languages, dialogue subtitles are linguistically more diverse than queries, i.e., they have more unique words in all categories. 
This is potentially because the language used in subtitles are unconstrained human dialogues while the queries are collected as declarative sentences referring to specific moments in videos~\cite{lei2020tvr}. 
Comparing the two languages, the Chinese data is typically more diverse than the English data.\footnote{
\begin{CJK*}{UTF8}{gbsn}
The differences might be due to the different morphemes in the languages.
E.g., the Chinese word 长发\; (`long hair') is labeled as a single noun, but as an adjective (`long') and a noun (`hair') in English~\cite{wang2019vatex}.
\end{CJK*}
}
In Table~\ref{tab:qtype_examples}, we show English and their translated Chinese query examples in Table~\ref{tab:qtype_examples}, by query type.
In the appendix, we compare \DsetName~with existing video and language datasets.






\begin{table}[]
\centering
\small
\scalebox{0.86}{
\begin{tabular}{lrrrrrr}
\toprule
\multirow{2}{*}{ Data } & \multicolumn{1}{c}{Avg} & \multicolumn{5}{c}{ \#unique words by POS tags } \\ \cmidrule(l){3-7}
& \multicolumn{1}{c}{Len} & \multicolumn{1}{c}{all} & \multicolumn{1}{c}{verb}  & \multicolumn{1}{c}{noun} & \multicolumn{1}{c}{adj.} & \multicolumn{1}{c}{adv.} \\
\midrule
\multicolumn{2}{l}{\textbf{English}} & & & &  &  \\
Q & 13.45 & 15,201 & 3,015 & 7,143 & 2,290 & 763 \\
Sub & 10.78 & 49,325 & 6,441 & 19,223 & 7,504 & 1,740 \\
Q+Sub & 11.27 & 52,545 & 7,151 & 20,689 & 8,021 & 1,976 \\
\midrule
\multicolumn{2}{l}{\textbf{Chinese}} & & & &  &  \\
Q & 12.55 & 34,752 & 12,773 & 18,706 & 1,415 & 1,669 \\
Sub & 9.04 & 101,018 & 36,810 & 53736 & 4,958 & 5,568 \\
Q+Sub & 9.67 & 117,448 & 42,284 & 62,611 & 5,505 & 6,185 \\
\bottomrule
\end{tabular}
}
\vspace{-3pt}
\caption{Comparison of English and Chinese data in~\DsetName. We show average sentence length, and number of unique tokens by POS tags, for Query (\textit{Q}) and or Subtitle (\textit{Sub}).}
\label{tab:en_zh_data_comparison}
\vspace{-8pt}
\end{table}


% \vspace{-0.1in}

\section{Method} \label{sec:method}

%\subsection{ADMM Training of DNNs}\label{sec:admm_nn}

%Alternating Direction Method of Multipliers (ADMM) \cite{gabay1975dual,boyd2011distributed} is a class of optimization methods belonging to  \textit{operator splitting techniques} which borrows benefits from both dual decomposition and augmented Lagrangian methods for constrained optimization. To show the potentials of standard ADMM, we first revisit a general formulation of ADMM in DNN training, similar to those used in prior work. Then, we propose our stochastic block-ADMM in the next subsection.

%To formulate training an $L$-layer DNN in a general supervised setting, we would have the following non-convex constrained optimization problem \cite{zeng2018global}:
% \vspace{-0.1in}
%\begin{align} \label{eq:obj}
	%\minimize_{ \mathcal{W}, \mathcal{A}, \mathcal{Z}} \quad &\mathcal{J}\left(\mY, \mZ_{L} \right) + \sum_{\ell = 1}^{L} \lambda_{\ell}  {\bf r}_{\ell} (\mW_{\ell}) \\
	% {\rm subject~to} \quad & \mA_{\ell} - {\bm \phi}_{\ell } \left( \mZ_{\ell} \right) = {\bf 0}, \quad \ell = 1,\dots, L-1   \nonumber \\
	 %{\rm subject~to} \quad & \mZ_{\ell} - \mW_{\ell} \mA_{\ell-1} = {\bf 0}, \quad \ell = 1, \dots , L \nonumber 
%\end{align}
%where $\mathcal{J}$ is the main objective (\textit{e.g.}, cross-entropy, mean-squared-error loss functions) that needs to be minimized. The subscript $\ell$ denotes the $\ell$-th layer in the network. The optimization variables are $\mathcal{W} = \{ \mW_\ell\}_{\ell=1}^{L}$, $\mathcal{A} = \{ \mA_{\ell}\}_{\ell=1}^{L-1}$, and $\mathcal{Z} = \{ \mZ_{\ell}\}_{\ell=1}^{L}$ where $\mW_\ell$, $\mZ_{\ell}$, $\mA_\ell$, and ${\bm \phi}_\ell (.)$ are the weight matrix, output matrix, activation matrix, and the activation function (\textit{e.g.}, ReLU) at the $\ell$-th layer, respectively. Note that $\mA_{0} = \mX$ where $\mX = \{ \vx_1,\dots, \vx_N \} \in  \R^{M \times N}$ is the input data matrix containing $N$ samples with input dimensionality $M$; $\mY = \{\vy_1,\dots, \vy_N \} \in \R^{C \times N}$ is the target matrix pair comprised of $N$ one-hot vector label of dimension $C$, representing number of prediction classes. Also, ${\bf r(.)}$ is the regularization term with (\textit{e.g.}, Frobenius norm $\|.\|_F^2$) corresponding penalty weight $\lambda_{\ell}$. %Note that the regularization term can be simply ignored by setting $\lambda_\ell$ to zero. 
%In this formulation, the intercept in each layer is ignored for simplicity as it can be simply be added by slightly modifying the $\mW_\ell$ and the input to each layer. 
%The formulation in Eq. (\ref{eq:obj}) breaks the the conventional multi-layer backpropagation optimization of DNNs into simpler sub-problems that can be solved efficiently (e.g. reducing to least-squares problem). %This also facilitates training in a distributed manner --- as the layers of the DNN are decoupled and the variables can be updated in parallel across layers ($\mW_\ell$) and data points (\ $\mW_\ell, \mZ_\ell, \mA_\ell$).



%To enforce the constraints in problem (\ref{eq:obj}) and solve the optimization using ADMM, we would have the following augmented Lagrangian problem:

%\begin{eqnarray} \label{eq:augmented}
%	\minimize_{ \mathcal{W}, \mathcal{A}, \mathcal{Z}} \quad &\mathcal{J}\left(\mY, \mZ_{L} \right) + \sum_{\ell = 1}^{L} \lambda_{\ell}  {\bf r}_{\ell} (\mW_{\ell}) \\
%	& + \sum_{\ell=1}^{L} \frac{\beta_{\ell}}{2} \| \mZ_{\ell} - \mW_{\ell} \mA_{\ell-1} + \mU_{\ell}\|_{F}^{2} \nonumber\\
%	& + \sum_{\ell=1}^{L-1} \frac{\gamma_{\ell}}{2} \| \mA_{\ell} - {\bm \phi}_{\ell}(\mZ_{\ell}) + \mV_{\ell}\|_{F}^{2}\nonumber
%\end{eqnarray}
%where $\beta_{\ell}, \gamma_\ell >0$ are the step sizes, $\mU_{\ell}$ and $\mV_{\ell}$ are the \textit{(scaled) dual variables} \cite{boyd2011distributed} for the equality constraint at the layer $\ell$. 
%Algorithm \ref{alg:admm} shows a standard ADMM scheme for optimizing Eq. (\ref{eq:augmented}). Note, the parameters are updated in a closed-form as analytical solution can be simply derived. For simplicity of the equations, we denote $\gP_\ell (.) = \frac{\beta_{\ell}}{2} \| \mZ_{\ell} - \mW_{\ell} \mA_{\ell-1} + \mU_{\ell}\|_{F}^{2} $ and $\gQ_\ell (.) = \frac{\gamma_{\ell}}{2} \| \mA_{\ell} - {\bm \phi}_{\ell}(\mZ_{\ell}) + \mV_{\ell}\|_{F}^{2}$. 
%This is solved in \cite{taylor2016training,wang2019admm} with the difference that they only enforced the constraints on the last layer $L$ with dual variables while other constraints were only loosely enforced using quadratic penalty. 

%\begin{algorithm}[htb]
%   \caption{Standard ADMM for DNN Training}
%   \label{alg:admm}
%\begin{algorithmic}
%   {\STATE \scalebox{1}{\bfseries Input:} data $\mX$, labels $\mY$}
%   \STATE  \scalebox{1}{{\bfseries Params:} $\beta_\ell >0, \gamma_\ell >0,\lambda_\ell > 0$ }
%   \STATE  \scalebox{0.8}{{\bfseries Initialize:} $\{\mW_\ell^0\}_{\ell=1}^{L}, \{ \mU_\ell^0\}_{\ell=1}^{L}, \{ \mV_\ell^0\}_{\ell=1}^{L-1}, \{\mZ^0_\ell\}_{\ell=1}^{L}, \{\mA^0_\ell\}_{\ell=1}^{L-1}\; k \leftarrow 0$ }
%   \REPEAT
%   \FOR{$\ell=1$ {\bfseries to} $L$}
%   \STATE \scalebox{1}{$\mW_\ell^{k+1} \leftarrow \argmin\; \{ \gP_\ell (.) +  \lambda_{\ell}  {\bf r}_{\ell} (\mW_{\ell}^{k})\}$}
%   \ENDFOR
%   \FOR{$\ell=1$ {\bfseries to} $L-1$}
%   \STATE \scalebox{1}{ $\mZ_\ell^{k+1} \leftarrow \argmin\; \{ \gP_\ell (.) +  \gQ_\ell (.) \}$ }
%   \STATE \scalebox{1}{$\mA_\ell^{k+1} \leftarrow \argmin\; \{ \gP_{\ell+1} (.) +  \gQ_\ell (.) \} $}
%   \ENDFOR
%     \STATE \scalebox{1}{ $\mZ_{L}^{k+1} \leftarrow \argmin\; \{ \mathcal{J}\left(\mY, \mZ_{L}^{k} \right) + \gP_L (.) \}$ }
%   \FOR{$\ell=1$ {\bfseries to} $L-1$}
%   \STATE \scalebox{1}{$\mU_\ell^{k+1} \leftarrow \mU_\ell^{k} + \mZ_{\ell}^{k+1} - \mW_{\ell}^{k+1} \mA_{\ell-1}^{k+1}$}
%   \STATE \scalebox{1}{$\mV_\ell^{k+1} \leftarrow \mV_\ell^{k} + \mA_{\ell}^{k+1} - {\bm \phi}_{\ell}(\mZ_{\ell}^{k+1})$}
%   \ENDFOR
%   \STATE \scalebox{1}{$\mU_L^{k+1} \leftarrow \mU_L^{k} + \mZ_{L}^{k+1} - \mW_{L}^{k+1} \mA_{L-1}^{k+1}$}
%   \UNTIL{some stopping criterion is reached.}
% \end{algorithmic}
% \end{algorithm}


There were many hurdles in using ADMMs for deep learning --- the global convergence proof of the ADMM \cite{deng2016global} assumes that the optimization objective is deterministic and the global solution is calculated at each iteration of the cyclic parameter updates.
% and during each iteration of the cyclic parameter updates, all the data samples are visited.
This typically requires matrix inversion and makes standard ADMM computationally expensive thus impractical for training of many large-scale optimization problems. To see a formulation of standard ADMM for training DNNs refer to the supplementary materials \ref{sec:admm_nn}.  %Specifically, for  deep learning, this would impose a severe restriction on training set size when limited computational resources are available. 
%In addition, since the variable updates in standard ADMM require matrix inversion, the extent of its applications is limit to trivial tasks \cite{taylor2016training}, making it incompetent to perform on par with the recent complex architectures introduced in deep learning (e.g. \cite{he2016deep}).

In this section, we present stochastic Block-ADMM which does not require global solution as well as an online version which further reduces the communication load. We prove the convergence of these algorithms in Sec. \ref{sec:convergence} and present its application in supervised disentanglement in Sec.~\ref{sec:deepfacto}.

\begin{figure*}[t]
\begin{center}
\subfigure[] { \label{fig:block_admm}
\includegraphics[width=0.75\linewidth]{imgs/block_admm.pdf}
}
\subfigure[] { \label{fig:block}
\includegraphics[width=0.15\linewidth]{imgs/block.pdf}
}
\end{center}
% \vspace{-.2in}
\caption{\small a) General Architecture for training DNNs proposed in Stochastic block-ADMM. b) A few differential layers selected from a parent network are stacked inside a block. The parameters $\Theta_t$ are updated by SGD in a forward-backward pass.}
%  \vskip -0.1in
\end{figure*}




% \vspace{-0.025in}

%-----------------------------------------
\subsection{Stochastic Block-ADMM}\label{sec:block_admm}
% \vspace{-0.025in}

In this section, we introduce a novel variant of ADMM for training DNNs, the stochastic block-ADMM. We first split the conventional multi-layer network architectures into an arbitrary number of \emph{blocks}, each containing only a part of the network. To make the parameters of each block independent from its neighbors, \emph{decoupling variables} \{$\mZ_t, \; t=1, \dots, T$\} are introduced as shown in Fig.~\ref{fig:block_admm}. These variables pass the information forward and backward in the architecture to train blocks in a cyclic manner until consensus is reached. Each $block_t$  consists of one or multiple differentiable layers (e.g., convolutional layers, activation layers, etc.) that are detached from the rest of the network via coupling variables. Denote the set of all learnable parameters of each $block_t$ as $\Theta_t$. As an example, a $block_t$ wrapping multiple layers can be seen in Figure \ref{fig:block}. Our formulation is:
\begin{align} \label{eq:ourformulation}
	\minimize_{ {\bm \Theta}, \mathcal{Z}}\; &\mathcal{J}\left(\mY, \mZ_{T} \right) 
	 \\
 {\rm subject~to} ~ &\bm Z_t = \mathrm{block}_{\bm \Theta_t}(\bm Z_{t-1}), \quad \mZ_{0} = \mX \nonumber
\end{align}
where ${\bf \Theta} = \{\Theta_t\}_{t=1}^{T} \text{and } \mathcal{Z} = \{\mZ_t\}_{t=1}^{T}$. $\mathcal{J}$ is the desired cost to be minimized (\textit{e.g.}, cross-entropy loss), $T$ is the total number of blocks, $\mX = \{ \vx_1,\dots, \vx_N \} \in  \R^{M \times N}$ is the input data, and $\mY = \{\vy_1,\dots, \vy_N \} \in \R^{C \times N}$ is the target label -- for $C$ classes. Note that the number of blocks $T$ can be different than the number of layers in the network $L$.

To train DNNs with this new approach, 
%similar to problem (\ref{eq:augmented}), 
we would have the following augmented Lagrangian minimization problem to enforce the equality constraints needed for training,
\begin{align} \label{eq:block_admm_unconstrained}
	\min_{ {\bf \Theta}, \mathcal{Z}} \; &\mathcal{J}\left(\mY, \mZ_{T} \right) 
	+ \sum_{t=1}^{T} \frac{\beta_t}{2} \| \mZ_t - \mathrm{block}_{\Theta_t}(\mZ_{t-1}) + \mU_t\|_F^2 \nonumber \\
	& {\rm subject~to} \quad \mZ_{0} = \mX 
\end{align}
where $\beta_t$ and $\mU_t$ are the (scaled) step size  and the Lagrange multiplier corresponding to the $t$-th Block, respectively. Our proposed Stochastic block-ADMM method for training problem (\ref{eq:block_admm_unconstrained}) is presented in Algorithm \ref{alg:blockadmm}. %In stochastic block-ADMM%, parameters of the $t$-th block, $\Theta_t$ are updated using the \emph{Stochastic Gradient Descent} optimizer or its adaptive learning rate variants (e.g. \emph{Adam} \cite{kingma2014adam}). %or second-order optimizers including Newton's method and \emph{(L)BFGS}. 
%We have found Adam to consistently outperform other counterparts, particularly in updating the decoupling variables $\mZ_t$. 
$\zeta_t$ and $\eta_t$ are the learning rates in each update step for $\mZ_t$ and $\Theta_t$, respectively. Similar to training conventional neural networks, each block is updated by first going in a forward pass through the block and update the parameters using back-propagation. Update of the block parameters $\Theta_{t}$ is done using mini-batch stochastic gradient descent or Adam. The same goes for the decoupling variables $\mZ_t$. Note, in each cycle of the parameter update in Algorithm \ref{alg:blockadmm}, all the samples of $\mZ$ are updated, while $\Theta_{t}$ is updated stochastically. In addition, due to non-convexity of primal sub-problem (Eq. \ref{eq:primal}), one can perform the primal updates for multiple steps. %However, we found one step update to be sufficient in our experiments.
In Algorithm \ref{alg:blockadmm}, we take the reverse order for updating  the decoupling variables $\mZ_t$, which we have empirically found more efficient, as analogous to backpropagation where gradient flows backwards as well.
%Although any ordering should converge in theory, a few may converge faster in practice (e.g., in Algorithm \ref{alg:admm}, we found it more stable to update the  weights $\mW_\ell$ first). 

Note that in this formulation, %no gradient is backpropagated through the entire network. To be more precise, 
backpropagation stops at each auxiliary variable $\mZ_t$ . Hence, our method can readily mitigate the long-known vanishing gradient problem by splitting a conventional DNN into arbitrary sized blocks. 
%In section \ref{exp:mnist}, the results from our proposed method, Block-ADMM, compared with baselines in dealing with vanishing gradient and performance in supervised learning are presented. Keep in mind, the strategy of splitting a conventional networks into the Blocks is completely optional and relies on the task to be accomplished e.g., for the heterogeneous problem of activation factorization as illustrated in section \ref{exp:weakly}, the activation layer to be factorized is a splitting point, $\mZ_t$.\\
During testing time, one could follow Eq. (\ref{eq:block_admm_unconstrained}) to solve an optimization problem. But in practice, it suffices to use a straight-through estimator by removing the decoupling variables and simply pass the output of each layer to the next, equivalent of doing a forward pass in a conventional DNN. %We have taken this approach in our experiments as the error induced by ignoring the decoupling variables is negligible. 
%It should be noted that the update step for $\mZ_\ell$ is dependent on the adjacent blocks, hence can only be parallelized across the data points. However, fixing $\mZ$, all the block parameters $\Theta_t$ are independent of each other, hence can be updated in parallel across blocks as well as data points.


%---------------------------- algorithm block admm ------------------------------
%  \vspace{-0.05in}
\begin{algorithm}[htb]
   \caption{Stochastic Block-ADMM}
   \label{alg:blockadmm}
\begin{algorithmic}
   {\STATE \scalebox{1}{\bfseries Input:} data $\mX$, labels $\mY$}
   \STATE  \scalebox{1}{{\bfseries Params:} $\beta_t >0, \; \zeta_t >0, \eta_t >0$ }
   \STATE  {\bfseries Define:} \scalebox{0.80}{ $\mathcal{T}({\mZ_{t}, \mZ_{t-1}, \mU_{t}, \Theta_t}) = \frac{\beta_t}{2} \| \mZ_t - \mathrm{block}_{\Theta_t}(\mZ_{t-1}) + \mU_t\|_F^2$ }
   \STATE  \scalebox{1}{{\bfseries Initialize:} $\{{\Theta_t^0}\}_{t=1}^{T}, \{ \mU_t^0\}_{t=1}^{T} ,\; k \leftarrow 0$ }
   \STATE  \scalebox{1}{{\bfseries Initialize:} $\{\mZ_t\}_{t=1}^{T}$ in a forward pass. }
   \REPEAT
   \STATE \scalebox{1}{ $\mZ_{T}^{k+1} \leftarrow \mZ_{T}^{k} - \zeta_T \nabla_{\mZ_{T}^k} ( \mathcal{J}\left(\mY_{i}, \mZ_{T}^{k} \right)$ }
   \STATE \scalebox{1}{$ + \mathcal{T}({\mZ_{T}^k, \mZ_{T-1}^k, \mU_{T}^k, \Theta_L^k})) \;  $}
   %\forall i \in \{1,\dots,N\} $}
   \FOR{$t=T-1$ {\bfseries to} $1$}
   \STATE \scalebox{1}{ $\mZ_{t}^{k+1} \leftarrow\mZ_{t}^{k} - \zeta_t \nabla_{\mZ_{t}^k} ( \mathcal{T}({\mZ_{t}^k, \mZ_{t-1}^k, \mU_{t}^k, \Theta_{t}^k})$} \\
   \STATE  \scalebox{1}{$+ \mathcal{T}({\mZ_{t+1}^{k+1}, \mZ_{t}^k, \mU_{t+1}^k, \Theta_{t+1}^k})) \; $}
%   \forall i \in \{1,\dots,N\}  $}
   \ENDFOR
   \FOR{$t=1$ {\bfseries to} $T$}
%   \STATE{${\rm draw} \; i \subset \{1,\dots,N\}$}
   \STATE \scalebox{0.9}{${\Theta_t}^{k+1} \leftarrow {\Theta_t}^{k} - \eta_t  \nabla_{\Theta_t} \mathcal{T}({\mZ_{t,i}^{k+1}, \mZ_{t-1,i}^{k+1}, \mU_{t,i}^k, \Theta_{t}^k}),$}
   \STATE \scalebox{0.9}{$draw \; i \subset \{1,\dots,N\} \; $}
   \STATE \scalebox{1}{$\mU_t^{k+1} \leftarrow \mU_t^{k} + \mZ_{t}^{k+1} - \mathrm{block}_{\Theta_t}^{k+1}(\mZ_{t-1}^{k+1})$}
   \ENDFOR
   \UNTIL{some stopping criterion is reached.}
\end{algorithmic}
\end{algorithm}
% \vspace{-0.05in}


%----------------------------
\subsection{Online Stochastic Block-ADMM}\label{sec:onlineadmm}
% \vspace{-0.025in}

The stochastic block-ADMM formulation in section \ref{sec:block_admm} is still a batch mode algorithm, in the sense that the entire training set is updated at once. This imposes restrictions on the size of the input and the number of parameters in the network when limited resources are available. %As a result, the extension to extremely large datasets, often the case in deep learning applications, can be problematic. 
Also, it does not readily accommodate to settings where data is constantly changing, such as data augmentation on the input or reinforcement learning. To overcome such limitations, we propose an \textit{online} variant of the stochastic block-ADMM in Algorithm \ref{alg:online_admm} which alternatively solves the unconstrained problem, 
\begin{align}\label{eq:scalar_dual}
	\min_{ {\bf \Theta}, \mathcal{Z}} \; &\mathcal{J}\left(\vy, \vz_{T} \right) 
	+ \sum_{t=1}^{T} \frac{\beta_t}{2} \big( \|\vz_t - \mathrm{block}_{\Theta_t}(\vz_{t-1}) \|_F^2 + u_t\big) \nonumber\\
	& {\rm subject~to} \quad \vz_{0} = \vx 
\end{align}
Although similar to the Eq. (\ref{eq:block_admm_unconstrained}), the dual variable in the online Block-ADMM is a \textit{scalar}. The benefits of this are two-folded: First, this substantially reduces the memory size needed for storing the dual variables as the optimization proceeds. Second, this considerably reduces the variance in the gradient induced by re-initializing the auxiliary variables $\vz_{\ell,i}$ when updating the block parameters at each iteration. %On the other hand, when the dual variable has the same size as the auxiliary variables as in the batch stochastic block-ADMM, the algorithm easily diverges.

% In Algorithm \ref{alg:online_admm}, as the data pair $(\vx_i, \vy_i)$ comes, first, the auxiliary variables are initialized in a forward pass though the blocks. Then, one iteration of the alternating update by ADMM is performed.

%---------------------------- algorithm block admm ------------------------------
% \vspace{-0.1in}
\begin{algorithm}[htb]
   \caption{Online Stochastic Block-ADMM }
   \label{alg:online_admm}
\begin{algorithmic}
   {\STATE \scalebox{1}{\bfseries Input:} data $\mX$, labels $\mY$}
   \STATE  \scalebox{1}{{\bfseries Params:} $\beta_t >0, \; \zeta_t >0, \eta_t >0$ }
   \STATE  {\bfseries Define:} \scalebox{0.8}{ $\mathcal{T}({\vz_{t}, \vz_{t-1}, u_{t}, \Theta_t}) = \frac{\beta_t}{2} (\| \vz_t - \mathrm{block}_{\Theta_t}(\vz_{t-1}) \|_2 + u_t)^2$ }
   \STATE  \scalebox{1}{{\bfseries Initialize:} $\{{\Theta_t^0}\}_{t=1}^{T}, \{ u_t^0\}_{t=1}^{T} ,\; k \leftarrow 0$ }
%   \STATE  \scalebox{1}{{\bfseries Initialize:} $\{\mZ_t\}_{t=1}^{T}$ in a forward pass. }
   \REPEAT 
   \FOR{$(\vx_i, \vy_i) \text{\bfseries in} (\mX,\mY)$}
   \STATE \scalebox{1}{{\bfseries Initialize:} $\{\vz_{t,i}\}_{t=1}^{T}$ in a forward pass $(\vz_{0,i} = \vx_i)$.}
   \STATE \scalebox{1}{$\vz_{T,i} \leftarrow \vz_{T,i} - \zeta_T \nabla_{\vz_{T,i}} ( \mathcal{J}\left(\vy_{i}, \vz_{T,i} \right)$ }
   \STATE \scalebox{1}{$+\mathcal{T}({\vz_{T}, \vz_{T-1}, u_{T}^k, \Theta_T^k})) \;  $}
   %\forall i \in \{1,\dots,N\} $}
   \FOR{$t=T-1$ {\bfseries to} $1$}
   \STATE \scalebox{1}{ $\vz_{t,i} \leftarrow\vz_{t,i} - \zeta_t \nabla_{\vz_{t,i}} ( \mathcal{T}({\vz_{t,i}, \vz_{t-1,i}, u_{t}^k, \Theta_{t}^k})$} \\
   \STATE  \scalebox{1}{$+ \mathcal{T}({\vz_{t+1,i}, \vz_{t,i}, u_{t+1}^k, \Theta_{t+1}^k})) \; $}
%   \forall i \in \{1,\dots,N\}  $}
   \ENDFOR
   \FOR{$t=1$ {\bfseries to} $T$}
%   \STATE{${\rm draw} \; i \subset \{1,\dots,N\}$}
   \STATE \scalebox{1}{${\Theta_t}^{k+1} \leftarrow {\Theta_t}^{k} - \eta_t  \nabla_{\Theta_t} \mathcal{T}({\vz_{t,i}, \vz_{t-1,i}, u_{t}^k, \Theta_{t}})$}
%   \STATE \scalebox{0.9}{$draw \; i \subset \{1,\dots,N\} \; $}
   \STATE \scalebox{1}{$u_t^{k+1} \leftarrow u_t^{k} + \| \vz_{t}^{k} - \mathrm{block}_{\Theta_t^{k+1}}(\vz_{t-1,i})\|_2$}
   \ENDFOR
   \ENDFOR
   \UNTIL{some stopping criterion is reached.}
\end{algorithmic}
\end{algorithm}
% \vspace{-0.125in}




%----------------------------
\subsection{Convergence of the Algorithm}\label{sec:convergence}
Let us consider the following general problem:
\begin{align}\label{eq:main}
	\minimize_{\mathcal{Z},\bf \Theta} & f(\mathcal{Z})\\
	{\rm subject to} & h(\mathcal{Z},\bm \Theta)=\bm 0,\nonumber
\end{align}
where $\mathcal{Z}$ and $\bf \Theta$ are as defined in Sec.~\ref{sec:block_admm}, and $f(\cdot)$ represents the training objective, and $h(\cdot)$ represents the layer coupling equalities as in \eqref{eq:ourformulation}.
We also assume that both $f(\cdot)$ and $h(\cdot)$ are differentiable functions. Note that both $f$ and $h$ can be non-convex.

Let us consider the following augmented Lagrangian:
\[          {\cal L}_{\rho_k}(\mathcal{Z},{\bf \Theta},{\bm \lambda})=f(\mathcal{Z}) + \langle \bm \lambda, h(\mathcal{Z},{\bf \Theta})\rangle + \frac{1}{2\rho_k}\|h(\bm Z,\bf \Theta)\|_2^2,  \]
where $\bm \lambda$ collects all the dual variables $\bm U_1,\ldots,\bm U_T$ that correspond to different layers. The standard primal-dual updates can be summarized as follows:
\begin{subequations}\label{eq:stopdd}
\begin{align}
    (\bm Z^{k+1},\Theta^{k+1}) &\leftarrow  \arg\min_{\mathcal{Z},\bf \Theta}  {\cal L}_{\rho_k}(\mathcal{Z},\bf \Theta,\bm \lambda^k), \label{eq:primal}\\
    \bm \lambda^{k+1} &\leftarrow \bm \lambda^k + \frac{1}{\rho_k}h(\bm Z^{k+1}, \Theta^{k+1}),
\end{align}
\end{subequations}
%In our case, since the sub-problem in \eqref{eq:primal} is non-convex, exactly minimizing this function may not be possible. In the previous section, the primal update is carried out by stochastic optimization w.r.t. $\mathcal{Z}$ and $\mathbf{\Theta}$ in an alternating fashion---which is a computationally lightweight algorithm that converges to a stationary point of ${\cal L}_{\rho_k}(\mathcal{Z},\mathbf{ \Theta},\bm \lambda^k)$ under certain conditions \cite{bottou2012stochastic,xu2015block}.
%The convergence of this type of primal-dual algorithm with inexact stochastic solution for the primal problem is unclear. In this work, we offer convergence support for our designed deep network training algorithm. Our idea follows recent work in \cite{shi2017penalty} that handles deterministic primal problems under non-convex equality constraints. 
We employ the trick in \cite{shi2017penalty} for adaptively adjusting the parameter $\rho_k$. We assume that $\rho_k$ is adjusted by
\begin{align}\label{eq:rho}
    \rho_{k+1} \leftarrow \begin{cases}  \rho_k,&\quad \|h(\bm Z^{k},\bm \Theta^k)\|\leq \eta_k,\\
                                         c\rho_k,~0<c<1,&\quad {\rm o.w.}
    \end{cases}
\end{align}
where $\eta_k$ for $k=1,2,\ldots$ is a pre-specified sequence that bounds the equality-enforcing error.

Our analysis shows the following convergence result:
% \vspace{-0.1in}
\begin{Prop}\label{prop:convergence}
    Assume $h({\cal Z},\bm \Theta)=\bm 0$ satisfies the Robinson's condition.
    Also assume for each update in \eqref{eq:primal}, the sub-problem solution solved by stochastic alternating optimization satisfies
   \begin{equation}
       \mathbb{E}\left[ \left\| {\cal G}(\x^k) \right\|^2\right]\leq \varepsilon_k, ~ \mathbb{V}\left[  {\cal G}(\x^k) \right]\leq \sigma_k^2,
   \end{equation}  
   where $\x=({\cal Z},\bm \Theta)$ is a vector that collects all the optimization variables and ${\cal G}(\x^k)$ collects the stochastic gradients that we used for updating $({\cal Z},\bm \Theta)$.
   Assume that the stochastic gradient for the primal update is unbiased, i.e.,
   \begin{equation}\label{eq:unbiasedness}
       \mathbb{E}[{\cal G}(\x^k)] = \nabla {\cal L}_{\rho_k}(\x_k),~\forall k. 
   \end{equation}         
   Then, every limit point of the solution sequence produced by the algorithm in \eqref{eq:stopdd} converges to a KKT point of the problem in~\eqref{eq:main}, if $\eta_k\rightarrow 0$, $\sigma_k^2 \rightarrow 0$ and $\varepsilon_k\rightarrow 0$.	
\end{Prop}
% \vspace{-0.05in}
% {\it \bf Proof}: 
The proof for Proposition~\ref{prop:convergence} is presented in the supplementary materials \ref{sec:proof}.
Proposition~\ref{prop:convergence} asserts that the algorithm converges to a KKT point under some conditions. 
   There are a number of remarks regarding implementation.
   To begin with, the condition $\varepsilon_k\rightarrow 0$ means that the primal problem needs to be solved more and more accurately when $k$ grows, in terms of approaching the stationary point of the sub-problem using block stochastic gradient. This can be achieved via gradually increasing the number of iterations for the primal updates. Note that stochastic block gradient can provably attain $\mathbb{E}[\|{\cal G}(\X^k)\|^2]\leq \varepsilon_k$; see \cite{xu2015block}. 
%   \comment{
%   In addition, the condition $\sigma_k^2\rightarrow 0$ means that the variance of the stochastic gradient needs to shrink when $k$ increases. This can be achieved by increasing the batch size when $k$ grows; see discussions in \cite{xu2015block}. 
%   Hence, {\it in theory}, to satisfy both conditions, the complexity for carrying out each iteration $k$ may grow.
%   Nonetheless, our empirical experience shows that using a fixed number of iterations for stochastic primal optimization and a fixed batch size in general does not hurt convergence. We hypothesize that the momentum may play an important role in increasing the effective batch size since with momentum gradients from previous batches are remembered and utilized in the subsequent steps. We leave further analysis with momentum to future research. Another remark is that the unbiasedness of the primal stochastic gradient [cf. \eqref{eq:unbiasedness}] is not always easy to establish under block coordinate descent settings \cite{xu2015block}. Nonetheless, this can be fixed via a simple randomization strategy among the blocks \cite{fu2019block}. }
   
   %The third challenge is that the hyperparameter $c$ and the sequence $\{\eta_k\}$ are not necessarily easy to select in some cases~\cite{shi2017penalty,fu2018anchor}. These parameters control how quickly one should adjust the update settings to accommodate the current iteration, which varies from case to case. However, interestingly, we find that these hyperparameters are relatively easy to tune in our case. In particular, our extensive experiments show that fixed $\rho_k$ and $\eta_k$ work reasonably well for our deep learning problems.
   
   %The slightly stringent conditions in Proposition~\ref{prop:convergence} and the relatively `benign' convergence behavior observed in practice pose a gap between theory and practice---and an interesting direction for future research.



%----------------------------
\subsection{DeepFacto: Factorization of DNN Activations }\label{sec:deepfacto}

%To show the power and flexibility of our proposed method in training heterogeneous networks, we will 
Here, we investigate a task for supervised disentanglement, which can provide insights for explaining DNNs to humans. Supervised disentanglement aims to find disentangled factors that decide the CNN output, yet are human-understandable and distinct from each other. One approach to learn a disentangled representation is through adding  non-negative matrix factorization (NMF)\cite{lee1999learning} layers to the network \cite{collins2018deep}. Note that NMF imposes non-differentiable constraints into the network where conventional end-to-end training using backpropagation would not be applicable. Hence, prior work were mostly running NMF after the training, where the network might have already learned highly entangled features. In this work, aided with our stochastic block-ADMM, we attempt to perform training with NMF layers in the intermediate layers of DNNs.

Figure \ref{fig:deepfacto} shows an \emph{NMF module} with \emph{rank $r$} incorporated between two arbitrary neighboring blocks. The output from the $block_t$ is factorized into $\mM_t$ and $\mS_t$, namely, the basis and score matrices. In this configuration, only the score matrix $\mS_t$ is passed to the next blocks. The score matrix is low-rank, sparse and non-negative hence can possibly represent features that are more disentangled than the original network. 
Exploring this architecture is one attempt of us in making deep networks more explainable to humans. Humans would not be able to interpret conventional deep network weights which are both positive and negative and sometimes cancels out each other. The sparse and non-negative feature from NMF would be much more preferable to interpret~\cite{collins2018deep}.

However, the NMF module breaks the gradient path from $\mS_t$ to $Z_t$, hence conventional backpropagation would not be applicable in this problem. We extend the ADMM framework (\ref{eq:block_admm_unconstrained}) into having non-negative factorization constraints over its activations and formulate the following optimization problem:
\begin{eqnarray} \label{eq:block_admm_nmf}
	\min_{ {\bf \Theta}, \mathcal{Z}, \mS, \mM} \; &\mathcal{J}\left(\mY, \mZ_{T} \right) \nonumber \\
	+ & \sum_{k=1, k\neq t+1}^{T} \frac{\beta_k}{2} \| \mZ_k - block_k(\mZ_{k-1}) + \mU_k\|_F^2 \nonumber \\
	+ & \frac{\beta_{t+1}}{2} \| \mZ_{t+1} - block_{t+1}(\mS_{t}) + \mU_{t+1}\|_F^2 \nonumber \\
	+ &  {\frac{\gamma_t}{2} \| \mZ_t - \mM_t \mS_t + \mV_t\|_F^2} \nonumber \\
	  & {\forall i,j} \;  \mM_{\ell,ij} \ge 0,\; \mS_{\ell,ij} \ge 0 
\end{eqnarray}
where $\gamma_t$ is the step-size and $\mV_t$ is the corresponding multipliers to enforce the matrix factorization equality $\mZ_t = \mM_t \mS_t$. The NMF module adds a nonconvex term to the optimization. However, in the alternating optimization scheme, while keeping either $\mM_t$ or $\mS_t$ constant, solving for the other term would reduce to a normal convex least-squares problem. The rest of the updates are the same as in section \ref{sec:block_admm}. Note that, trivially to not change the input dimension of the next block after the NMF module, one can simply add an affine layer to increase the dimensions without changing the formulation.
% , as the linear layer can be easily regarded in the parameter space of the next block. 

At testing time, one only needs to perform a non-negative projection since the basis matrix $M$ will be given, which can be solved using a convex solver such as LBFGS. Note that for simplicity, we only formulated adding \emph{one} NMF module in the middle of the blocks. This can be simply extended to as many NMF modules as needed in the architecture.

%---------------------------- Figure admm nmf  ------------------------------
\begin{figure}[t!]
% \vskip -0.1in
\begin{center}
\centerline{
\includegraphics[width=1 \columnwidth]{imgs/block_admm_nmf.pdf}
}
%  \vskip -0.1in
 \caption{General architecture for Deepfacto: an NMF module with rank $r$ is added in the middle of two arbitrary blocks. Note, only $\mS_t$ is passed to the next blocks.}
%\vspace{-0.05in}
\label{fig:deepfacto}
\end{center}
%  \vskip -0.3in
\end{figure}

\begin{table*}[!ht]\centering
\small
\begin{tabular}{l|c|c|c|ccc|c}\toprule
\multirow{2}{*}{ Model } &VQA2 &NLVR2 &VE & \multicolumn{3}{c|}{RefCOCO+} & \multirow{2}{*}{ Meta-Ave } \\
&Test-Dev &Test-P &Test &Dev &TestA &TestB & \\\cmidrule{1-8}
ViLBERT\cite{lu2019vilbert} &70.6 &- &- &72.3 &78.5 &62.6 & - \\
VL-BERT\cite{su2019vl} &71.2 &- &- &71.6 &77.7 &61.0 & - \\
$\text{UNITER}_{\text{CC}}$\cite{chen2020uniter} &71.2 &- &- &72.5 &79.4 &63.7 & - \\
VisualBERT \cite{li2019visualbert,li2020unsupervised} &70.9 &73.9 &- &73.7 &79.5 &64.5 & - \\
Aligned VLP &\textbf{72.5} &\textbf{75.9} &\textbf{78.7} &\textbf{82.1} &\textbf{86.6} &\textbf{75.0} & \textbf{77.3} \\
\midrule
Base &70.1 &51.2 &73.2 &69.4 &74.8 &60.3 & 65.9 \\
\uvisualbert \cite{li2020unsupervised} &71.8 &53.2 &76.8 &78.2 &83.6 &69.9 & 70.0\\
$\text{\ModelName}_{\text{CC}}$ &\textbf{72.1} &\textbf{73.4} &\textbf{77.3} &\textbf{80.3} &\textbf{85.5} & \textbf{73.7} & \textbf{75.8} \\
$\text{\ModelName}_{\text{BC}}$ &71.2 &67.1 &77.1 &79.7 &85.0 &72.7 & 73.8 \\
\bottomrule
\end{tabular}
\caption{Evaluation results on four V+L downstream tasks. Our model trained with un-aligned data ($\text{\ModelName}_{\text{CC}}$, $\text{\ModelName}_{\text{BC}}$) achieves comparable performance with the supervised model trained with aligned data (Aligned VLP). $\text{\ModelName}_{\text{CC}}$ and $\text{\ModelName}_{\text{BC}}$ also outperform {\uvisualbert } on nearly all tasks.}
\label{tab:main}
\end{table*}

\begin{table*}[!ht]\centering
\small
\begin{tabular}{l|c|c|c|ccc|c}\toprule
\multirow{2}{*}{V+L Alignment} &VQA &NLVR2 &VE &\multicolumn{3}{c|}{RefCOCO+} & \multirow{2}{*}{Meta-Ave}\\
&Test-Dev &Test-P &Test &Dev &TestA &TestB & \\\cmidrule{1-8}
$\text{\ModelName}_{\text{CC}}$ (R-T)  &71.7 &52.0 &75.6 &78.7 &83.3 &70.0 & 69.5\\
$\text{\ModelName}_{\text{CC}}$ (R-N)  &71.4 &69.4 &76.5 &77.4 &81.5 & 68.7 & 73.7 \\
$\text{\ModelName}_{\text{CC}}$ (I-S)  &71.6 &71.5 &76.8 &75.7 &80.3 &67.9 & 73.9\\
$\text{\ModelName}_{\text{CC}}$ (R-T + R-N) &71.9 &72.4 &76.4 &79.3 & 84.5 & 71.7 & 75.0 \\
$\text{\ModelName}_{\text{CC}}$ (R-T + R-N + I-S) &\textbf{72.1} & \textbf{73.4} &\textbf{77.3} &\textbf{80.3} & \textbf{85.0} &\textbf{73.7} & \textbf{75.8}\\
\bottomrule
\end{tabular}
\caption{Effect of cross-modal alignment on the three types of granularities: region-tag alignment(R-T), region-noun phrase alignment(R-N), and image-sentence alignment(I-S)}\label{tab:ablation_align}
\end{table*}


\section{Experiments}
In this section, we provide the detailed experimental set up to evaluate our proposed {\ModelName } against previous supervised and unsupervised VLP models. More specifically, we introduce our pre-training dataset, baselines, and our pre-training setting.

\subsection{Pre-training Datasets}
We prepare the un-aligned data under two different settings: (1) We use images and text separately from Conceptual Captions (CC) \cite{sharma2018conceptual} ignoring the alignment information; (2) We use images from Conceptual Captions (CC) \cite{sharma2018conceptual} and text from BookCorpus (BC) \cite{Zhu_2015_ICCV}. 
Setting (1) sets up a fair comparison with previous supervised methods by keeping the domain and the quality of training data consistent. Our proposed model trained in this setting is called \ModelName$_{CC}$. 
Setting (2) mimics a more realistic challenge where we have large-scale images and text data from different domains, in particular the text sources are not similar to captions of the images. \ModelName$_{BC}$ has been trained in this setting.

As introduced in section \ref{section:data_aug}, for each image we retrieve 5 text data points (captions from CC or sentences from BC) from the text corpus that are semantically similar to the detected objects in the image. 
This creates weakly-aligned image-text pairs for our pre-training models. 

\subsection{Baselines}
We compare the performance of our proposed {\ModelName } to the following baselines: 

\head{Base Model} VisualBERT that is initialized from BERT. It does not undergo any pre-training but is directly fine-tuned on the downstream tasks. 

\head{Supervised Pre-trained Models} Supervised pre-trained VLP models that are trained only on CC, including VILBERT\cite{lu2019vilbert}, VL-BERT\cite{su2019vl}, and UNITER\cite{chen2020uniter}. We also report the numbers on the Supervised VisualBERT implemented in \uvisualbert\cite{li2020unsupervised} that is trained on CC and an additional 2.5 Million text segments from BC. 
For fair comparison with our proposed method, we also introduce the aligned vision-language pre-training model (Aligned VLP) that is pre-trained on the 3M (image, caption) pairs from CC and 3M (image, object tag) pairs. 

\head{Unsupervised Pre-trained Models} 
{\uvisualbert } is pre-trained on individual image or text corpus in a round-robin fashion and captures the cross-modal alignment by using detected object tags as the anchor point. 
For fair comparison, we re-implemented this method to pre-train with the VinVL object features\cite{zhang2021vinvl} and BC. 
\subsection {Training Setup}\label{sec:training_setup}
Our transformer architecture consists of 12 layers of transformer blocks, where each block has 768 hidden units and 12 self-attention heads. 
We initialize the model from $\text{BERT}_{base}$ and pre-train for 20 epochs on their respective pre-training datasets with a batch size of 480. The region features for images are obtained from the pre-trained VinVL object detectors \cite{zhang2021vinvl}. We use Adam optimizer \cite{ADAM} with a linear warm-up for the first 10\% of training steps, and set the peak learning rate as 6e-5. After warm up, a linear-decayed learning-rate scheduler gradually drops the learning rate for the rest of training steps. 
All models were trained on 4 NVIDIA A100 GPUs, with 40GB of memory per GPU using MMF\cite{singh2020mmf}.
The pre-training takes 3 days.
We evaluate our pre-trained models on four downstream tasks: Visual Question Answering (VQA 2.0)\cite{anderson2018bottom}, Natural Language for Visual reasoning\cite{suhr2018corpus} ($\text{NLVR}^2$), Visual Entailment\cite{xie2019visual} (VE), and Referring Expression\cite{yu2016modeling} (RefCOCO+). 
% To validate our proposed sentence-image alignment pre-training, we also conduct a zero-shot evaluation with image-text retrieval task on Flickr30K\cite{young-etal-2014-image}. 
Detailed training settings for each task can be found in our supplementary material. 
\subsection{Experimental Results}
We first compare {\ModelName } to various supervised models that are pre-trained on CC and to the state-of-the-art unsupervised V+L pre-training method, {\uvisualbert } on the four downstream tasks. Besides reporting scores for each individual task, we also compute the meta-average score to reflect the overall performance across all tasks. 
The results are summarized in Table \ref{tab:main}.

\myparagraph{Compared to Base.} It is clear from Table~\ref{tab:main} that both \ModelName$_{CC}$ and \ModelName$_{BC}$ outperform the Base model by a large margin on all benchmarks. 


\myparagraph{Compared to Aligned VLP.} It also achieves better performance than existing supervised models like VilBERT\cite{lu2019vilbert}, which is potentially due to the usage of better visual regional features of VinVL~\cite{Zhang_2021_CVPR}. When compared to Aligned VLP, which is trained with the same architecture and visual features, our model is only slightly worse. 
This shows the effectiveness of our proposed pre-training curriculum which can learn comparable universal representation across vision and language as the supervised models without any parallel image-text corpus. 

\myparagraph{Compared to UVLP. }Our {\ModelName } also achieves consistently better performance than the previous UVLP method: \uvisualbert. 
This improvement shows how our proposed cross-modal alignment pre-training curriculum effectively bridges the gap across the two modalities.
In particular, our model outperforms {\uvisualbert } in the task of NLVR2 by more than 20\%. 
As NLVR2 is known to benefit more from image-sentence cross-modal alignment from previous supervised V+L pre-training research \cite{chen2020uniter}, this observation indicates that our model is able to capture the instance-level cross-modal alignment without parallel data. 
When {\ModelName } is trained on BC text and CC images \ie \ModelName$_{BC}$, it still achieves comparable or better performance than {\uvisualbert } except for VQA.  
The slight advantage  {\uvisualbert } has over \ModelName$_{BC}$ in VQA is potentially due the similar style between the VQA text and the pre-trained CC captions. 
However, this does not overshadow the overall better performance of {\ModelName}. 
It shows that our proposed method is more robust than {\uvisualbert } training on the uni-modal datasets collected from separate domains, which makes it more useful in practical settings.    
\begin{figure}[h!]
\centering
\includegraphics[width=0.8\linewidth]{figures/number_candidate_plot.pdf}
\caption{Meta average scores of non-parallel V+L pre-training with different number of retrieved candidate sentences.}
\label{fig:ablation_ncandidate}
\end{figure}


\begin{figure*}[h!]
\centering
\includegraphics[width=16cm]{figures/retrieved_positives.pdf}
\caption{Examples of retrieved text from both CC and BC. The covered grounded noun phrases in retrieved sentences are highlighted in green bar for positive examples.}
\label{fig:visualization}
\end{figure*}

\subsubsection{Ablation Study on Multi-Granular Alignment}
We conduct ablation study to verify the effectiveness of the three types of visual-language alignment for unsupervised V+L pre-training, namely region-tag alignment (R-T), region-noun Phrase alignment (R-N), and image-sentence alignment (I-S). We first evaluate each individual type of alignment to measure its usefulness for different downstream tasks. Then, we gradually add each type of alignment into the UVLP.  For this ablation study we pre-train {\ModelName } on CC images and text, and the results are summarized in Table \ref{tab:ablation_align}. 

From Table~\ref{tab:ablation_align}, we can see that aligning local regions to object tags (R-T) and noun phrases (R-N) are especially helpful for the task of RefCOCO+, which requires the model to understand specific objects that natural expressions describe. Meanwhile, aligning the image and sentence at instance-level (I-S) benefits NLVR2 and VE. Especially on NLVR2, the model that captures the global vision and language alignment \ModelName$_{CC}$ (I-S) obtains 19.5\% gain over the model that only learns the local alignments between regions and object tags \ModelName$_{CC}$ (R-T). This observation is consistent with previous research \cite{chen2020uniter}, where the performance of model on NLVR2 is boosted after introducing pre-training objectives that capture the cross-modal alignment in the image-text pairs. Our results demonstrate that even with just weakly-aligned sentences, we can still effectively learn the instance-level cross-modal alignment. 
% \mingyang{Maybe also show zero-shot Image-Sentence Alignment performance to further proves the learned cross-modal alignment on vision and language;} 
Combining the region-tag and region-noun phrase alignment (R-T+R-N) for UVLP, we observe that these two types of grounding and matching compensate each other. \ModelName$_{CC}$ (R-T+R-N) shows a marginal but consistent improvement over models that only learn a single type of local region-to-language alignment (R-T, R-N). After adding object-phrase level alignment we can further improve the performance on NLVR2 and VE, which gives us our best performing model \ModelName$_{CC}$ (R-T + R-N + I-S). 

\subsubsection{Ablation Study on Number of Retrieved Candidates}
We conduct experiments to verify the impact the number of retrieved candidate text for each image has on the performance. We create three variants of pre-training corpus, where the number of retrieved candidate are 1, 5, and 10 based on the rank of the similarity of each candidate text to the query image's detected object tags. The candidate text is sampled from CC. We pre-train our {\ModelName } model with only the pre-training objectives to capture the sentence-image alignment (I-S). For each variant of pre-training corpus, we train the model for the same number of steps. We compute the meta average score for the three resulting pre-trained models and visualize them in Fig.~\ref{fig:ablation_ncandidate}. 

Fig.~\ref{fig:ablation_ncandidate} shows that retrieving more than one candidate text for an image greatly benefits the pre-trained model to learn a better joint representation between vision and language, demonstrated by stronger performance in the downstream tasks. 
We suspect this is because the closeness between the candidate caption and the detected object tags in language embedding space does not always mean a better alignment between the candidate caption and the image. A better and more semantically similar caption candidate for the image could be found in the other caption candidates. However, when we increase the number of candidate captions to 10, we observe a slight drop on the overall performance compared to the model that is pre-trained on corpus with 5 candidate captions. This indicates that having too many candidate captions to form the weakly-aligned pairs with the query image for V+L pre-training may also introduce unnecessary noise. Hence, we set the number of retrieved captions in our experiments to 5.
 

\subsubsection{Visualization}
To get a sense of the quality of the retrieved sentences, we show some examples of retrieved text from both CC and BC in Fig.~\ref{fig:visualization}. The first row demonstrates a positive case of retrieved captions from CC, where we observe a good coverage of the objects in the image such as ``young woman", ``sofa", and ``couch" in the top retrieved sentences. Similarly, our retrieval method can also retrieve good candidates from BC that describe many visual objects from the image as depicted in row 2. This observation demonstrate the effectiveness of picking candidates based on their closeness to the object list in the language embedding space. 

We also compare the text-to-image attention between the pre-trained \uvisualbert~and \ModelName~without task-specific fine-tuning as~\cite{chen2020uniter,Zhou_2021_CVPR}.
As shown in Fig.~\ref{fig:attention}, we feed into the models an aligned pair whose caption is ``young woman seated on the beach", we visualize the local cross-modality alignment between regions and tokens.
we found our full model \ModelName~can better attend on the described regions, showing higher-quality alignment is learned through the proposed pre-training.
More visualizations are in the supplementary file.



\begin{figure}[h!]
\centering
\includegraphics[width=0.8\linewidth]{figures/attention.pdf}
\vspace{-0.5cm}
\caption{Text-to-image attention given the aligned pair whose caption is ``young woman seated on the beach".}
\label{fig:attention}
\end{figure}



\section{Conclusion}
In this paper, we find the gap between classification and localization by decomposing CAM from a new perspective. We claim that the misalignment between the feature vector at each location and class-specific weight causes CAM to be activated only in a small discriminative region.
To bridge this gap, we propose a method of aligning feature directions with class-specific weights. We also introduce a strategy to enhance the effect of feature direction alignment.
Extensive experiments demonstrate the effectiveness of the proposed method, which outperforms existing WSOL methods by a large margin.

\noindent\textbf{Limitation.}
There are several hyperparameters to decide in our method. To alleviate the search burden, we discuss a rationale for hyperparameter selection.
%%%%%%%%% BIB
\bibliographystyle{acl_natbib}
\bibliography{anthology,acl2021}


\appendix

\section{Appendix}

\paragraph{Data Analysis.}
In Table~\ref{tab:dataset_comparison} we show a comparison of \DsetName~with existing moment retrieval datasets and related video and language datasets. 
Compared to other moment retrieval datasets, \DsetName~is significantly larger in scale, and comes with query type annotations that allows in-depth analyses for the models trained on it.
Besides, it is also the only moment retrieval dataset with multilingual annotations, which is vital in studying the moment retrieval problem under the multilingual context. 
Compared to the existing multilingual video and language datasets, \DsetName~is unique as it has a more diverse set of context and annotations, i.e., dialogue, query type, and timestamps.


\paragraph{Training and Inference Details.}
In Figure~\ref{fig:mxml_overview} we show an overview of the \ModelName~model.
We compute video retrieval score as:
\begin{align}
    s^{vr} = \frac{1}{2}\sum_{m \in \{v, s\}} \mathrm{max}(\frac{H^{m}_{vr}}{\left\Vert H^{m}_{vr}\right\Vert} \frac{\boldsymbol{q}^{m}}{\left\Vert \boldsymbol{q}^{m}\right\Vert}).
\end{align}
The subscript $lang \in \{en, zh\}$ is omitted for simplicity.
It is optimized using a triplet loss similar to main text Equation (1).
For moment retrieval, we first compute the query-clip similarity scores $S^{q,c} \in \mathbb{R}^{l}$ as:
\begin{align}
    S^{q,c} = \frac{1}{2}(H^{s}_{mr}\boldsymbol{q}^{s} + H^{v}_{mr}\boldsymbol{q}^{v}).
\end{align}
Next, we apply Convolutional Start-End Detector (ConvSE module)~\cite{lei2020tvr} to obtain start, end probabilities $P_{st}, P_{ed} \in \mathbb{R}^{l}$. These scores are optimized using a cross-entropy loss. The single video moment retrieval score for moment $[t_{st}, t_{ed}]$ is computed as:
\begin{align}
    s^{mr}(t_{st}, t_{ed}) = P_{st}(t_{st}) P_{ed}(t_{ed}), \, t_{st} \leq t_{ed}.
\end{align}

\noindent
Given a query $q_i$, the retrieval score for moment [$t_{st}$:$t_{ed}$] in video $v_j$ is computed following the aggregation function as in~\cite{lei2020tvr}:
\begin{align}
    s^{vcmr}&(v_j,t_{st}, t_{ed}|q_i) = \nonumber \\ &s^{mr}(t_{st}, t_{ed})\mathrm{exp}(\alpha s^{vr}(v_j|q_i)),
\end{align}


\noindent
where $\alpha{=}20$ is used to assign higher weight to the video retrieval scores.
The overall loss is a simple summation of video and moment retrieval loss across the two languages, and the language neighborhood constraint loss. 








\paragraph{Implementation Details.}
\ModelName~is implemented in PyTorch~\cite{paszke2017automatic}.
We use Adam~\cite{kingma2014adam} with initial learning rate 1e-4, $\beta_1{=}0.9$, $\beta_2{=}0.999$, L2 weight decay 0.01, learning rate warm-up over the first 5 epochs. 
We train \ModelName~for at most 100 epochs at batch size 128, with early stop based on the sum of R@1 (IoU=0.7) scores for English and Chinese.
The experiments are conducted on a NVIDIA RTX 2080Ti GPU. 
Each run takes around 7 hours.



\begin{table}[!t]
\centering
\small
\setlength{\tabcolsep}{3.5pt}
\renewcommand{\arraystretch}{1.05}
\scalebox{1.0}{
\begin{tabular}{lcccc}
\toprule
& \multicolumn{2}{c}{English R@1} & \multicolumn{2}{c}{Chinese R@1} \\  \cmidrule(l){2-3} \cmidrule(l){4-5}
Setting & IoU=0.5 & IoU=0.7 & IoU=0.5 & IoU=0.7 \\
\midrule
unseen & 1.68 & 0.79 & 1 & 0.54 \\
seen & 4.82 & 2.79 & 4.18 & 2.32 \\
\bottomrule
\end{tabular}
}
\caption{\ModelName~performance on the \DsetName~val split \textit{Friends} examples, in both \textit{unseen} and \textit{seen} settings. 
}
\label{tab:ablation_unseen}
\end{table}



\begin{figure*}[!t]
  \includegraphics[width=\linewidth]{res/mXML_overview.pdf}
  \caption{
  \ModelName~overview. For brevity, we only show the modeling process for a single language (Chinese). The cross-language modifications, i.e., parameter sharing and neighborhood constraint are illustrated in Figure~\ref{fig:tvrm_encoding}. This figure is edited from the Figure 4 in~\citep{lei2020tvr}. 
  }
  \label{fig:mxml_overview}
\end{figure*}



\begin{table*}[ht]
\centering
\small
\setlength{\tabcolsep}{5pt}
\scalebox{0.96}{
\begin{tabular}{lcccccc}
\toprule
Dataset & Domain & \#Q/\#videos & Multilingual & Dialogue & QType & Timestamp \\
\midrule
\bf QA datasets with temporal annotation &  &  &  &  &  &  \\
TVQA~\cite{Lei2018TVQALC} & TV show & 152.5K/21.8K & - & \checkmark & - & \checkmark \\
How2QA~\cite{li2020hero} & Instructional & 44K/22K & - & \checkmark & - & \checkmark \\
\bf Multilingual video description datasets &  &  &  &  &  &  \\
MSVD~\cite{chen2011collecting} & Open & 70K/2K & \checkmark & - & - & - \\
VATEX~\cite{wang2019vatex} & Activity & 826K/41.3K & \checkmark & - & - & - \\
\bf Moment retrieval datasets &  &  &  &  &  &  \\
TACoS~\cite{regneri2013grounding} & Cooking & 16.2K/0.1K & - & - & - & \checkmark \\
DiDeMo~\cite{anne2017localizing} & Flickr & 41.2K/10.6K & - & - & - & \checkmark \\
ActivityNet Captions~\cite{Krishna2017DenseCaptioningEI} & Activity & 72K/15K & - & - & - & \checkmark \\
CharadesSTA~\cite{gao2017tall} & Activity & 16.1K/6.7K & - & - & - & \checkmark \\
How2R~\cite{li2020hero} & Instructional & 51K/24K & - & \checkmark & - & \checkmark \\
TVR~\cite{lei2020tvr} & TV show & 109K/21.8K & - & \checkmark & \checkmark & \checkmark \\
\midrule
\DsetName & TV show & 218K/21.8K & \checkmark & \checkmark & \checkmark & \checkmark \\ 
\bottomrule
\end{tabular}
}
\caption{
Comparison of~\DsetName~with related video and language datasets.   
}
\label{tab:dataset_comparison}
\end{table*}

 

\paragraph{Generalization to Unseen TV shows.} 
To investigate whether the learned model can be transferred to other TV shows, we conduct an experiment by using the TV show `\textit{Friends}' as an `\textit{unseen}' TV show for testing, and train the model on all the other 5 TV shows. 
For comparison, we also include a model trained on `\textit{seen}' setting, where we use all the 6 TV shows including \textit{Friends} for training. 
To ensure the models on these two settings are trained on the same number of examples, we downsample the examples in the \textit{seen} setting to match the \textit{unseen} setting.
The results are shown in Table~\ref{tab:ablation_unseen}.
We notice our \ModelName~achieves a reasonable performance even though it does see a single example from the TV show \textit{Friends}.
Meanwhile, the gap between \textit{unseen} and \textit{seen} settings are still large, we encourage future work to further explore this direction.


\paragraph{Prediction Examples}
We show \ModelName~prediction examples in Figure~\ref{fig:pred_examples}. 
We show both Chinese (\textit{top}) and English (\textit{bottom}) prediction examples, and correct (\textit{left}) and incorrect (\textit{right}) examples.


\begin{figure*}[!t]
  \includegraphics[width=\linewidth]{res/pred_examples.pdf}
  \caption{
  Qualitative examples of \ModelName. \textit{Top:} examples in Chinese. \textit{Bottom:} examples in English. \textit{Left:} correct predictions. \textit{Right:} incorrect predictions.
  We show top-3 retrieved moments for each query. \textcolor{salmon}{salmon bar} shows the predictions, \textcolor{ForestGreen}{green box} indicates the ground truth.
  }
  \label{fig:pred_examples}
\end{figure*}



\end{document}
