%
% File acl2021.tex
%
%% Based on the style files for EMNLP 2020, which were
%% Based on the style files for ACL 2020, which were
%% Based on the style files for ACL 2018, NAACL 2018/19, which were
%% Based on the style files for ACL-2015, with some improvements
%%  taken from the NAACL-2016 style
%% Based on the style files for ACL-2014, which were, in turn,
%% based on ACL-2013, ACL-2012, ACL-2011, ACL-2010, ACL-IJCNLP-2009,
%% EACL-2009, IJCNLP-2008...
%% Based on the style files for EACL 2006 by 
%%e.agirre@ehu.es or Sergi.Balari@uab.es
%% and that of ACL 08 by Joakim Nivre and Noah Smith

\documentclass[11pt,a4paper]{article}
\usepackage[hyperref]{acl2021}
\usepackage{times}
\usepackage{latexsym}
\renewcommand{\UrlFont}{\ttfamily\small}


% This is not strictly necessary, and may be commented out,
% but it will improve the layout of the manuscript,
% and will typically save some space.
\usepackage{microtype}

\aclfinalcopy % Uncomment this line for the final submission
%\def\aclpaperid{***} %  Enter the acl Paper ID here

%\setlength\titlebox{5cm}
% You can expand the titlebox if you need extra space
% to show all the authors. Please do not make the titlebox
% smaller than 5cm (the original size); we will check this
% in the camera-ready version and ask you to change it back.


%%%%%%%%%%%%%%%% Customized Area [START]
\usepackage{CJKutf8} % use Chinese, see https://www.overleaf.com/learn/latex/chinese

\usepackage{times}
\usepackage{epsfig}
\usepackage{graphicx}
\usepackage{amsmath}
\usepackage{amssymb}
\usepackage{array}

\usepackage{longtable}

\definecolor{demphcolor}{RGB}{144,144,144}
\newcommand{\demph}[1]{\textcolor{demphcolor}{#1}}

%%% set row text color [START]
% https://tex.stackexchange.com/questions/26360/how-to-color-the-font-of-a-single-row-in-a-table
\makeatletter
\newcommand*{\@rowstyle}{}

\newcommand*{\rowstyle}[1]{% sets the style of the next row
  \gdef\@rowstyle{#1}%
  \@rowstyle\ignorespaces%
}

\newcolumntype{=}{% resets the row style
  >{\gdef\@rowstyle{}}%
}

\newcolumntype{+}{% adds the current row style to the next column
  >{\@rowstyle}%
}
\makeatother
% Usage: note the '=l', '+l', and rowstyle
% \begin{tabular}{ =l | +l +l +l +l }
%   \rowstyle{\color{red}}
%       & 1 & 2 & 3 & 4 \\
%   \hline
%   1   & A & B & C & D \\
%   2   & A & B & C & D \\
%   3   & A & B & C & D \\
%   4   & A & B & C & D \\
% \end{tabular} 

%%% set row text color [END]


%%%%%%%%%%%% define thicker \hline with \thickhline [START]
\makeatletter
\def\thickhline{%
  \noalign{\ifnum0=`}\fi\hrule \@height \thickarrayrulewidth \futurelet
   \reserved@a\@xthickhline}
\def\@xthickhline{\ifx\reserved@a\thickhline
               \vskip\doublerulesep
               \vskip-\thickarrayrulewidth
             \fi
      \ifnum0=`{\fi}}
\makeatother

\newlength{\thickarrayrulewidth}
\setlength{\thickarrayrulewidth}{2\arrayrulewidth}
% define thicker \hline with \thickhline [END]

\usepackage{multirow}
\usepackage{booktabs}

\newcommand{\rulesep}{\unskip\ \vrule\ }
\usepackage{pifont}% http://ctan.org/pkg/pifont
\newcommand{\cmark}{\ding{51}}%
\newcommand{\xmark}{\ding{55}}%

% down arrow $\downarrow$, used in math mode
\newcommand{\xdownarrow}[1]{%
  {\left\downarrow\vbox to #1{}\right.\kern-\nulldelimiterspace}
}

\usepackage{xcolor}
% define colors from here: http://latexcolor.com/
\definecolor{ForestGreen}{rgb}{0.13, 0.55, 0.13}
\definecolor{bittersweet}{rgb}{1.0, 0.44, 0.37}
\definecolor{orchid}{rgb}{0.478, 0.506, 1.0}
\definecolor{salmon}{rgb}{0.968, 0.503, 0.408}
\definecolor{green}{rgb}{0.438, 0.678, 0.408}
\definecolor{orange}{rgb}{1.0, 0.75, 0.}


\usepackage{colortbl}
\definecolor{Gray}{gray}{0.85}

\def\DsetName{\textsc{mTVR}}
\def\ModelName{mXML}

%%%%%%%%%%%%% Customized Area [END]


\newcommand\BibTeX{B\textsc{ib}\TeX}

\title{\DsetName: Multilingual Moment Retrieval in Videos}


\author{
  Jie Lei $\;\;\;\;\;$ 
  Tamara L. Berg $\;\;\;\;\;$ Mohit Bansal \\
  Department of Computer Science \\ University of North Carolina at Chapel Hill \\
  {\tt \{jielei, tlberg, mbansal\}@cs.unc.edu} \\
}


% \date{}

\begin{document}
\maketitle
\begin{abstract}
We introduce \DsetName, a large-scale multilingual video moment retrieval dataset, containing 218K English and Chinese queries from 21.8K TV show video clips.
The dataset is collected by extending the popular TVR dataset (in English) with paired Chinese queries and subtitles. 
Compared to existing moment retrieval datasets, \DsetName~is multilingual, larger, and comes with diverse annotations.
We further propose \ModelName, a multilingual moment retrieval model that learns and operates on data from both languages, via encoder parameter sharing and language neighborhood constraints.
We demonstrate the effectiveness of \ModelName~on the newly collected \DsetName~ dataset, where \ModelName~outperforms strong monolingual baselines while using fewer parameters.
In addition, we also provide detailed dataset analyses and model ablations.
Data and code are publicly available at \url{https://github.com/jayleicn/mTVRetrieval}
\end{abstract}

%%%%%%%%% BODY TEXT
\section{Introduction}\label{introuction}
The number of videos available online is growing at an unprecedented speed.
Recent work~\cite{escorcia2019temporal,lei2020tvr} introduced the Video Corpus Moment Retrieval (VCMR) task: given a natural language query, a system needs to retrieve a short moment from a large video corpus. 
Figure~\ref{fig:data_example} shows a VCMR example.
Compared to the standard text-to-video retrieval task~\cite{xu2016msr,yu2018joint}, it allows more fine-grained moment-level retrieval, as it requires the system to not only retrieve the most relevant videos, but also localize the most relevant moments inside these videos. 
Various datasets~\cite{Krishna2017DenseCaptioningEI,anne2017localizing,gao2017tall,lei2020tvr} have been proposed or adapted for the task. 
However, they are all created for a single language (English), though the application could be useful for users speaking other languages as well. 
Besides, it is also unclear whether the progress and findings in one language generalizes to another language~\cite{bender2009linguistically}.
While there are multiple existing multilingual image datasets~\cite{gao2015you,elliott-etal-2016-multi30k,shimizu2018visual,pappas2016multilingual,lan2017fluency,li2019coco}, the availability of multilingual video datasets~\cite{Wang_2019_ICCV,chen2011collecting} is still limited.


\begin{figure}[!t]
\begin{center}
  \includegraphics[width=0.99\columnwidth]{res/example_vcmr.pdf}
  \vspace{-12pt}
  \caption{
  A \DsetName~example in the Video Corpus Moment Retrieval (VCMR) task. Ground truth moment is shown in \textit{\textcolor{green}{green}} box. Colors in the query text indicate whether the words are more related to video (\textcolor{orchid}{orchid}) or  subtitle (\textcolor{salmon}{salmon}) or both (\textcolor{orange}{orange}). 
  The query and the subtitle text are presented in both English and Chinese. 
  The video corpus typically contains thousands of videos, for brevity, we only show 3 videos here.
  }
  \label{fig:data_example}
  \end{center}
\end{figure}


Therefore, we introduce~\DsetName, a large-scale, multilingual moment retrieval dataset, with 218K human-annotated natural language queries in two languages, English and Chinese. 
\DsetName~extends the TVR~\cite{lei2020tvr} dataset by collecting paired Chinese queries and Chinese subtitle text (see Figure~\ref{fig:data_example}).
We choose TVR over other moment retrieval datasets~\cite{Krishna2017DenseCaptioningEI,anne2017localizing,gao2017tall} because TVR is the largest moment retrieval dataset, and also has the advantage of having dialogues (in the form of subtitle text) as additional context for retrieval, in contrast to pure video context in the other datasets.
We further propose \ModelName, a compact, multilingual model that learns jointly from both English and Chinese data for moment retrieval. 
Specifically, on top of the state-of-the-art monolingual moment retrieval model XML~\cite{lei2020tvr}, we enforce encoder parameter sharing~\cite{sachan2018parameter,dong2015multi} where the queries and subtitles from the two languages are encoded using shared encoders. 
We also incorporate a language neighborhood constraint~\cite{wang2018learning,kim2020mule} to the output query and subtitle embeddings. 
It encourages sentences of the same meaning in different languages to lie close to each other in the embedding space.
Compared to separately trained monolingual models, \ModelName~substantially reduces the total model size while improving retrieval performance (over monolingual models) as we show in Section~\ref{sec:experiments}. 
Detailed dataset analyses and model ablations are provided.







\section{Dataset}\label{sec:dataset}
The TVR~\cite{lei2020tvr} dataset contains 108,965 high-quality English queries from 21,793 videos from 6 long-running TV shows (provided by TVQA~\cite{Lei2018TVQALC}). The videos are associated with English dialogues in the form of subtitle text. \DsetName~extends this dataset with translated dialogues and queries in Chinese to support multilingual multimodal research.



\subsection{Data Collection}

\paragraph{Dialogue Subtitles.}
We crawl fan translated Chinese subtitles from subtitle sites.\footnote{\url{https://subhd.tv}, \url{http://zimuku.la}} 
All subtitles are manually checked by the authors to ensure they are of good quality and are aligned with the videos.
The original English subtitles come with speaker names from transcripts that we map to the Chinese subtitles, to ensure that the Chinese subtitles have the same amount of information as the English version. 


\paragraph{Query.}
To obtain Chinese queries, we hire human translators from Amazon Mechanical Turk (AMT).
Each AMT worker is asked to write a Chinese translation of a given English query.
Languages are ambiguous, hence we also present the original videos to the workers at the time of translation to help clarify query meaning via spatio-temporal visual grounding. The Chinese translations are required to have the exact same meaning as the original English queries and the translation should be made based on the aligned video content.
To facilitate the translation process, we provide machine translated Chinese queries from Google Cloud Translation\footnote{\url{https://cloud.google.com/translate}} as references, similar to~\cite{wang2019vatex}. 
To find qualified bilingual workers in AMT, we created a qualification test with 5 multiple-choice questions designed to evaluate workers' Chinese language proficiency and their ability to perform our translation task.
We only allow workers that correctly answer all 5 questions to participate our annotation task.
In total, 99 workers finished the test and 44 passed, earning our qualification.
To further ensure data quality, we also manually inspect the submitted results during the annotation process and disqualify workers with poor annotations.
We pay workers \$0.24 every three sentences, this results in an average hourly pay of \$8.70. 
The whole annotation process took about 3 months and cost approximately \$12,000.00. 



\begin{table}[!t]
\centering
\small
\setlength{\tabcolsep}{2pt}
\renewcommand{\arraystretch}{1.3}
\scalebox{0.87}{
\begin{CJK*}{UTF8}{gbsn}
\begin{tabular}{ll}
\toprule
QType (\%) & \multicolumn{1}{c}{Query Examples (in English and Chinese)} \\
\midrule
video-only & Howard places his plate onto the coffee table. \\
(74.2)  & 霍华德将盘子放在咖啡桌子上。\\
\midrule
sub-only & Alexis and Castle talk about the timeline of the murder. \\
 (9.1) & 
亚历克西斯和卡塞尔谈论谋杀的时间顺序。
 \\
\midrule
video+sub & Joey waives his hand when he asks for his food.  \\
(16.6) & 
乔伊催餐时摆了摆手。\\
\bottomrule
\end{tabular}
\end{CJK*} 
}
\caption{
\DsetName~English and Chinese query examples in different query types. 
The percentage of the queries in each query type is shown in brackets.
}
\label{tab:qtype_examples}
\end{table}


\subsection{Data Analysis}
In Table~\ref{tab:en_zh_data_comparison}, we compare the average sentence lengths and the number of unique words under different part-of-speech (POS) tags, between the two languages, English and Chinese, and between query and subtitle text.
For both languages, dialogue subtitles are linguistically more diverse than queries, i.e., they have more unique words in all categories. 
This is potentially because the language used in subtitles are unconstrained human dialogues while the queries are collected as declarative sentences referring to specific moments in videos~\cite{lei2020tvr}. 
Comparing the two languages, the Chinese data is typically more diverse than the English data.\footnote{
\begin{CJK*}{UTF8}{gbsn}
The differences might be due to the different morphemes in the languages.
E.g., the Chinese word 长发\; (`long hair') is labeled as a single noun, but as an adjective (`long') and a noun (`hair') in English~\cite{wang2019vatex}.
\end{CJK*}
}
In Table~\ref{tab:qtype_examples}, we show English and their translated Chinese query examples in Table~\ref{tab:qtype_examples}, by query type.
In the appendix, we compare \DsetName~with existing video and language datasets.






\begin{table}[]
\centering
\small
\scalebox{0.86}{
\begin{tabular}{lrrrrrr}
\toprule
\multirow{2}{*}{ Data } & \multicolumn{1}{c}{Avg} & \multicolumn{5}{c}{ \#unique words by POS tags } \\ \cmidrule(l){3-7}
& \multicolumn{1}{c}{Len} & \multicolumn{1}{c}{all} & \multicolumn{1}{c}{verb}  & \multicolumn{1}{c}{noun} & \multicolumn{1}{c}{adj.} & \multicolumn{1}{c}{adv.} \\
\midrule
\multicolumn{2}{l}{\textbf{English}} & & & &  &  \\
Q & 13.45 & 15,201 & 3,015 & 7,143 & 2,290 & 763 \\
Sub & 10.78 & 49,325 & 6,441 & 19,223 & 7,504 & 1,740 \\
Q+Sub & 11.27 & 52,545 & 7,151 & 20,689 & 8,021 & 1,976 \\
\midrule
\multicolumn{2}{l}{\textbf{Chinese}} & & & &  &  \\
Q & 12.55 & 34,752 & 12,773 & 18,706 & 1,415 & 1,669 \\
Sub & 9.04 & 101,018 & 36,810 & 53736 & 4,958 & 5,568 \\
Q+Sub & 9.67 & 117,448 & 42,284 & 62,611 & 5,505 & 6,185 \\
\bottomrule
\end{tabular}
}
\vspace{-3pt}
\caption{Comparison of English and Chinese data in~\DsetName. We show average sentence length, and number of unique tokens by POS tags, for Query (\textit{Q}) and or Subtitle (\textit{Sub}).}
\label{tab:en_zh_data_comparison}
\vspace{-8pt}
\end{table}


\section{Finding the Gap with CAM Decomposition}
Given an input image $x$ and a typical image classifier comprising convolutional layers and a GAP followed by an FC layer, a CAM for target class $c$ is computed as follows:
\begin{equation}\label{eq:cam}
\texttt{CAM}(x) = \mathbf{w}^\intercal_c F(x).
\end{equation}
$F(x)\in\mathbb{R}^{H \times W \times D}$ is the feature map before the GAP, and $\mathbf{w}_c\in\mathbb{R}^{D}$ is the weight of the FC layer connected to class $c$, where $H$, $W$, and $D$ are the height, width, and dimension, respectively.
Eq.~\ref{eq:cam} implies that the value of CAM at each spatial location is the dot product of two vectors, $\mathbf{w}_c$ and $F_u(x)$, where $u\in\{1, ..., HW\}$ is the index of spatial location.
It can be decomposed as follows:
\begin{equation}\label{eq:cam_each}
\begin{aligned}
\texttt{CAM}_u(x) = & \mathbf{w}_c \cdot F_u(x) \\
= & \|\mathbf{w}_c\|\|F_u(x)\| \underbrace{\frac{\mathbf{w}_c \cdot F_u(x)}{\|\mathbf{w}_c\|\|F_u(x)\|}}_{\textstyle S(\mathbf{w}_c,F_u(x))},
\end{aligned}
\end{equation}
where $S(\mathbf{a},\mathbf{b})$ is the cosine similarity between the two vectors, $\mathbf{a}$ and $\mathbf{b}$.
When generating a CAM, target class $c$ is fixed and $\|\mathbf{w}_c\|$ is the same for every $u$.
The CAM value at each position can now be interpreted as the product of the norm of the feature vector at the corresponding location and the similarity between the feature vector and class-specific weight vector.
Let $\mathcal{F}\in\mathbb{R}^{H \times W}$ and $\mathcal{S}\in\mathbb{R}^{H \times W}$ be the norm map and the similarity map, respectively, where $\mathcal{F}_u=\|F_u\|$ and $\mathcal{S}_u=S(\mathbf{w}_c, F_u(x))$. Subsequently, CAM can be rewritten as
\begin{equation}\label{eq:cam_abb}
\texttt{CAM}(x) = \|\mathbf{w}_c\|\cdot\mathcal{F}\odot\mathcal{S}.
\end{equation}
To localize the target object accurately, both $\mathcal{F}_u$ and $\mathcal{S}_u$ should be large for $u$ belonging to the object.

Likewise, the classification score can be interpreted with the output of the GAP, $f(x)=\text{GAP}(F(x))\in\mathbb{R}^{\rm{D}}$.
\begin{equation}\label{eq:logit}
\begin{aligned}
\texttt{logit}_c(x) = & \mathbf{w}_c \cdot f(x) \\
= & \|\mathbf{w}_c\|\left\lVert f(x)\right\rVert S\left( \mathbf{w}_c,f(x)\right).
\end{aligned}
\end{equation}
Because $\left\lVert f(x)\right\rVert$ is fixed for $x$, $\|\mathbf{w}_c\|$ and $S\left( \mathbf{w}_c,f(x)\right)$ determine the logit score of each class $c$.
The scale variation of $\|\mathbf{w}_c\|$ across classes is not very large.
Therefore, to classify $x$ correctly, $S(\mathbf{w}_c, f(x))$ must be large for the ground truth class $c$.
Here exists the gap between classification and localization.
The classifier is trained to increase $S(\mathbf{w}_c, f(x))$, not $S(\mathbf{w}_c, F_u(x))$ for $u$ belonging to an object region. Cosine similarity is interpreted as the degree of alignment between the directions of the two vectors, meaning that the input feature vector at the object region and class-specific weight vector are not ensured to be aligned with training only for classification.
This causes the model to fail to localize the entire object in a CAM.

Fig.~\ref{fig:cam_norm_sim} shows some examples of norm map $\mathcal{F}$, similarity map $\mathcal{S}$, and CAM from a vanilla model.
The less discriminative but object-belonging regions also have noticeably high activation in $\mathcal{F}$, including wings and bodies of birds. 
However, those regions are not activated in the final CAMs, due to the small values in $\mathcal{S}$.
Although $\mathcal{F}$ contains considerable information for localization, its effect diminishes because of the misalignment of the feature directions with the class-specific weight.

In the next section, we propose a method to bridge the gap between classification and localization by aligning feature directions: adjusting the cosine similarity between input features and class-specific weights. 


\section{Bridging the Gap through Alignment}
We describe how to align feature directions in Sec.~\ref{sec:feature_directions}. An additional strategy to enhance the effect of the feature direction alignment, consistency with attentive dropout, is introduced in Sec.~\ref{sec:consistency_drop}. In Sec.~\ref{sec:training_scheme}, we describe the overall training scheme.
Fig.~\ref{fig:overall} shows the overview of our proposed method.


\subsection{Alignment of Feature Directions}\label{sec:feature_directions}
To enhance the activation of the entire object region in CAM, we want the cosine similarity between $F_u$ and $\mathbf{w}_{c}$ to be high for $u$ belonging to the target object and low for the background region.
Because high activation in $\mathcal{F}$ implies that there is a cue for classification at the corresponding location, we divide the region of the feature map into coarse foreground region $\mathcal{R}^\text{norm}_\text{fg}$ and background region $\mathcal{R}^\text{norm}_\text{bg}$ based on a normalized $\mathcal{F}$.
\begin{equation}
\begin{aligned}
&\mathcal{R}^\text{norm}_\text{fg}=\{u|\hat{\mathcal{F}}_u>\tau_\text{fg}\},\\
&\mathcal{R}^\text{norm}_\text{bg}=\{u|\hat{\mathcal{F}}_u<\tau_\text{bg}\},\\
&\text{where}~\hat{\mathcal{F}}=\frac{\mathcal{F}-\min_{i}{\mathcal{F}_i}}{\max_{i}{\mathcal{F}_i}-\min_{i}{\mathcal{F}_i}}.
\end{aligned}
\end{equation}
$\tau_\text{fg}$ and $\tau_\text{bg}$ are constant thresholds that determine the foreground and background regions, respectively. Note that $\tau_\text{fg}$ and $\tau_\text{bg}$ are not the same; therefore, there is an unknown region that is not included in either $\mathcal{R}^\text{norm}_\text{fg}$ or $\mathcal{R}^\text{norm}_\text{bg}$.
To increase $\mathcal{S}_u$ in $\mathcal{R}^\text{norm}_\text{fg}$ and suppress it in $\mathcal{R}^\text{norm}_\text{bg}$, we define the similarity loss as follows:
\begin{equation}\label{eq:loss_sim}
\begin{aligned}
\mathcal{L}_\text{sim} = -\frac{1}{|\mathcal{R}^\text{norm}_\text{fg}|}\sum_{u\in \mathcal{R}^\text{norm}_\text{fg}}{\mathcal{S}_u} +\frac{1}{|\mathcal{R}^\text{norm}_\text{bg}|}\sum_{u\in \mathcal{R}^\text{norm}_\text{bg}}{\mathcal{S}_u}.
\end{aligned}
\end{equation}

There still remains a possibility that some parts of the object region have low activation in $\hat{\mathcal{F}}$.
In this case, $\mathcal{L}_\text{sim}$ may not be sufficient for the alignment.
Therefore, we introduce an additional loss term to increase $\hat{\mathcal{F}}$ in every candidate region belonging to the target object.
Because a positive $\mathcal{S}_u$ indicates that $u$ is making a positive contribution to increasing the classification logit, the regions with positive similarity can be treated as candidates for the object region. Therefore, we force this area to be activated.
We estimate the object region, $\mathcal{R}^\text{sim}_\text{fg}$, and background region, $\mathcal{R}^\text{sim}_\text{bg}$, based on $\mathcal{S}_u$ as
\begin{equation}
\begin{aligned}
&\mathcal{R}^\text{sim}_\text{fg}=\{u|\mathcal{S}_u>0\},\\
&\mathcal{R}^\text{sim}_\text{bg}=\{u|\mathcal{S}_u<0\}.
\end{aligned}
\end{equation}
With each estimated region, we define the norm loss in a manner similar to Eq.~\ref{eq:loss_sim}, as follows:
\begin{equation}\label{eq:loss_norm}
\begin{aligned}
\mathcal{L}_\text{norm} = -\frac{1}{|\mathcal{\mathcal{R}^\text{sim}_\text{fg}}|}\sum_{u\in \mathcal{R}^\text{sim}_\text{fg}}{\hat{\mathcal{F}}_u} +\frac{1}{|\mathcal{R}^\text{sim}_\text{bg}|}\sum_{u\in \mathcal{R}^\text{sim}_\text{bg}}{\hat{\mathcal{F}}_u}.
\end{aligned}
\end{equation}

For fine-grained classification, such as bird species classification, the object to be recognized is the same across classes. In this case, we define the region with a non-positive similarity with any class as $\mathcal{R}^\text{sim}_\text{bg}$ and the other as $\mathcal{R}^\text{sim}_\text{fg}$. In general, the regions $\mathcal{R}^\text{sim}_\text{bg}$ and $\mathcal{R}^\text{sim}_\text{fg}$ are defined with a similarity with a target class.

The two loss terms $\mathcal{L}_\text{sim}$ and $\mathcal{L}_\text{norm}$ operate complementary.
Through the minimization of $\mathcal{L}_\text{sim}$, the value of $\mathcal{S}$ in the region that is highly activated in $\hat{\mathcal{F}}$ increases.
Through the minimization of $\mathcal{L}_\text{norm}$, the value of $\hat{\mathcal{F}}$ in the region with high similarity increases.
After the joint minimization of $\mathcal{L}_\text{sim}$ and $\mathcal{L}_\text{norm}$, the activated region in $\hat{\mathcal{F}}$  and that in $\mathcal{S}$ become similar.


\subsection{Consistency with Attentive Dropout}\label{sec:consistency_drop}
We can expect the successful alignment by $\mathcal{L}_\text{sim}$ when the estimation of $\mathcal{R}^\text{norm}_\text{fg}$ and $\mathcal{R}^\text{norm}_\text{bg}$ is accurate: $\hat{\mathcal{F}}$ is consistently large over the entire object region and small over the background region.
\begin{figure}[t]
	\centering
    \includegraphics[width=\columnwidth]{figures/fig_dropconsistency.pdf}
    \vspace{-1.8em}
    \caption{Dropout mechanism of consistency with attentive dropout}
    \label{fig:dropconsistency}
\end{figure}
Because the value of $\mathcal{F}$ at the most discriminative region is significantly larger than that at the other region, the value of the normalized map $\hat{\mathcal{F}}$ at the less discriminative part but belonging to the object region becomes small.

We introduce consistency with attentive dropout, a method to distribute the activation to the target object region.
We adopt $L_1$ loss between the two feature maps $F$ and $F_\text{drop}$: $F$ is the feedforward result of an intermediate feature map $F'$, and $F_\text{drop}$ is the feedforward result of $F'_\text{drop}$ obtained by intentionally dropping large activations from $F'$.
Fig.~\ref{fig:dropconsistency} shows the overall process of obtaining $F'_\text{drop}$ for consistency with attentive dropout.
In $F'$, the activation at the spatial location whose channel-wise averaged activation is larger than $\gamma$ is dropped with probability $p$. The stochastic dropout prevents all information in the highly activated area from being eliminated. The loss for consistency with attentive dropout is as follows:
\begin{equation}\label{eq:loss_er}
\mathcal{L}_\text{drop} = \|F(x)- F_\text{drop}(x)\|_{1}.
\end{equation}

There have been several attempts that utilize a similar erasing mechanism~\cite{mai2020erasing,choe2019attention,zhang2018adversarial}.
They train a classifier to preserve the predicted labels before and after erasing highly activated features.
In contrast, our method explicitly regularizes a model to yield a similar feature map even after the highly activated features are dropped.
This decreases the dependency on the dropped features, resulting in more evenly distributed activation compared to the other methods.

\subsection{Training Scheme}\label{sec:training_scheme}
With cross-entropy loss for classification, $\mathcal{L}_\text{CE}$, the total cost function is defined as follows:
\begin{equation}\label{eq:loss_tot_step2}
\mathcal{L_\text{total}} = \mathcal{L}_\text{CE} + \lambda_\text{drop} \mathcal{L}_\text{drop} + \lambda_\text{sim} \mathcal{L}_\text{sim} + \lambda_\text{norm} \mathcal{L}_\text{norm},
\end{equation}
where $\lambda_\text{drop}$, $\lambda_\text{sim}$, and $\lambda_\text{norm}$ are hyperparameters for balancing the losses.
The feature direction alignment is better applied after training the classifier to some extent to obtain a suitable feature map for classification.
Thus, for the first few epochs (\ie, the warm stage), we train a model only with $\mathcal{L}_\text{CE}$ and $\mathcal{L}_\text{drop}$:
\vspace{-1pt}
\begin{equation}\label{eq:loss_tot_step1}
\mathcal{L_\text{warm}} = \mathcal{L}_\text{CE} + \lambda_\text{drop} \mathcal{L}_\text{drop}.
\end{equation}
% \vspace{-0.1in}
\section{Experiments} \label{sec:exp}


%Our implementation uses the PyTorch framework \cite{paszke2017automatic}. 
All the experiments are run on a machine with a single NVIDIA GeForce RTX 2080 Ti GPU. The results presented for each of the following experiments are selected from their best performance after grid search over the hyper-parameters, both for our method and the baselines. %Note, in the following Figures (\ref{fig:mnist_acc}, \ref{fig:mnist_deep}, \ref{fig:cifar}), 
Each algorithm is ran five times with different initialization and the average test set accuracy is reported. The shaded area corresponds to $\pm1$ standard deviation. We will make our code available online. %for reproducibility. %and a more precise access to the set of parameters used in the experiments.

% \vspace{-0.05in}
\subsection{Supervised Deep Network Training}\label{exp:conv}
In this section, we present the experiment results from training conventional neural networks in a supervised setting on the MNIST, Fashion-MNIST, and CIFAR-10 datasets. For experiments results on Fashion-MNIST and CIFAR-10, see supplementary materials \ref{sec:sup_train}. %The results from the proposed methods in section \ref{sec:method} are compared with baselines including training a conventional neural network in an end-to-end setting using SGD. 
% For the CIFAR-10 dataset, our formulation enables training complicated networks such as ResNets \cite{he2016deep} using ADMM which has never been done before.


\subsubsection{MNIST}\label{exp:mnist}
For the first supervised learning experiment, the MNIST dataset of handwritten digits \cite{mnist}, is used for the evaluation of ADMM/BCD methods for training DNNs. We use the standard train/test split. %Throughout the experiments, 60,000 samples are used during training and %
The performance on the testing set of 10,000 samples is reported in Figure \ref{fig:mnist_acc}. The architecture of the \emph{shallow} network used for the experiments incorporates three fully-connected layers with 128-neuron hidden layers $(784-128-128-10)$ and \emph{ReLU} nonlinearity. In order to make a fair comparison with ~\cite{taylor2016training} which can only work with Mean Squared Error (MSE), we utilize MSE as the training objective ($\mathcal{J}$) while the more common Cross-Entropy (CE) is applicable in our block-ADMM formulation and utilized in the experiments in the supplementary materials. 


In training standard ADMM and \cite{taylor2016training} as baselines, all the parameters are initialized by sampling from the uniform distribution $x \sim {U}(0, 10^{-4})$.% and are down scaled by a factor of $1e^{-4}$. 
We set $\beta_l = \gamma_l = 10$ for all of the layers. 
%Note that Algorithm \ref{alg:admm} can be converted to the formulation in \cite{taylor2016training} by setting dual variables $\forall \ell \neq L \; \mU_\ell = \bm 0;$ and discarding their updates. 
%To regularize the weights, $\mW_l$ during the training, \emph{$L_2$} norm 
Weight decay is used with $\lambda_l = 5 \times 10^{-5}$. %We observed that the regularization term significantly improves the optimization behavior in standard ADMM and without it, the training is not stable. 
For baselines with backpropagation in Fig. \ref{fig:mnist_acc}, a  learning rate of $5 \times 10^{-3}$ is used. 
 
 
Further, for the training of the batch and online Stochastic Block-ADMM algorithms presented in Algorithm \ref{alg:blockadmm} and \ref{alg:online_admm}, the aforementioned three-layer architecture is split into 3 one-layer blocks. $\beta_t$ is set to 1 for all layers, the weights are initialized using the normal distribution, dual variables $\mU_t$ are initialized using a uniform distribution, and auxiliary variables $\mZ_t$ are initialized in a forward pass. During training, the block parameters ($\Theta_t$) are updated stochastically, and both of sub-problem updates for the $\text{block}_{\Theta_t}$ and $\mZ_t$ are performed using \textit{Adam}. In our experiments in the batch mode, we performed the primal updates for $3$ steps during each iteration. For the online version, we set the batch size to 64 and auxiliary variables are re-initialized at each iteration (see Algorithm \ref{alg:online_admm}). 

Figure \ref{fig:mnist_acc} shows that Stochastic Block-ADMM outperforms the baselines by reaching $97.61 \%$ average test accuracy. Note the accuracy for all methods is lower than normal because of the MSE loss function that is used --- which is not the best choice for classification yet chosen for fair comparison with previous ADMM methods. The online version performs slightly worse with a $93.88 \%$ test accuracy. However, this comes with enormous advantage in terms of memory utilization, e.g. given the configuration for training on MNIST, the online version uses \~ 10$\times$ less memory to store training variables compared to the batch version.


%---------------------------- fig mnist acc ------------------------------
\begin{figure}[ht]
%  \vskip -0.05in
\begin{center}
\centerline{
\includegraphics[width=\columnwidth]{imgs/mnist_acc_new.pdf}
}
%  \vskip -0.05in
\caption{Test set accuracy on MNIST using network with 3 fully-connected layers: $784-128-128-10$. 
Final test accuracy: ``Stochastic Block-ADMM'': {\bf 97.61\%}, 
``Online Stochastic Block-ADMM'': 93.88\%, 
``Standard ADMM'': 95.02\%, 
%``Taylor et al.,  
\protect \cite{taylor2016training}
: 87.52\%, 
%``Wang et al. 
\protect \cite{wang2019admm}: 83.89\% ,
%``Zeng et al. 
\protect \cite{zeng2018global}: 83.28\% , 
``SGD'': 95.29\% 
(Best viewed in color)}
\label{fig:mnist_acc}
\end{center}
%  \vskip -0.4in
\end{figure}


\subsubsection{Vanishing Gradient}\label{exp:vanish}

Since no gradient is backpropagated through the entire network in our proposed algorithm, stochastic block-ADMM is robust against vanishing gradients. We run the previous experiments on an unconventional architecture with 10 fully-connected layers --- this is to make the vanishing gradient problem obvious. Note that normally this will not be adopted because of the severe overfitting and gradient vanishing problems, but here we utilized this setting to test our resistance to these problems. Figure \ref{fig:mnist_deep} illustrates the experiment results. Stochastic Block-ADMM reaches final test accuracy of $94.43\%$ while SGD and ADAM only reach to $10.28\%$ and $58\%$, respectively. As it can be seen in Figure \ref{fig:mnist_deep}, we also compared our method with the recent work of \cite{zeng2018global}. We observed the BCD in \cite{zeng2018global} %\footnote{code taken from \url{https://github.com/timlautk/BCD-for-DNNs-PyTorch}} 
to be unstable, sensitive to network architectures, and eventually, not converging after 300 epochs. Although we still exhibited some overfitting, we can see our approach is significantly better in handling of the vanishing gradient problem, and performs reasonably well. We further tested our performance with 20 fully-connected layers. Results show that although there is slightly more overfitting, our algorithm can still find a reasonable solution (Fig.~\ref{fig:mnist_deep}), showing its potential in helping with training scenarios with vanishing gradients.



%---------------------------- fig deep mnist ------------------------------
\begin{figure}[ht]
% \vskip -0.05in
\begin{center}
\centerline{
\includegraphics[width=\columnwidth]{imgs/mnist_deep.pdf}
}
% \vskip -0.05in
\caption{Test accuracies from deep architectures on MNIST. Block-ADMM demonstrates stable convergence and obtains final test accuracy of $\bf 94.43\%$ (10 layers), and $91.75\%$ (20 layers) respectively, while SGD and Adam (10 layers) fail due to vanishing gradients (Best viewed in color)}
\label{fig:mnist_deep}
\end{center}
% \vskip -0.35in
\end{figure}



%----------------------------
\subsubsection{Wall Clock Time Comparison}\label{time_cmp}

In this section, we analyze the batch and online versions of stochastic block-ADMM in training wall clock time and compare them against other baselines as illustrated in Figure \ref{fig:time}.  %The methods are implemented in PyTorch framework -- except for \cite{wang2019admm} that is implemented\footnote{code taken from \url{https://github.com/xianggebenben/dlADMM}} in "cupy", a NumPy-compatible matrix library accelerated by CUDA. 
Note Gotmare \etal and SGD are trained with a mini-batch size of 64 and \cite{zeng2018global,wang2019admm} are trained in a batch setting. Only the time taken for the \emph{training} was plotted in Fig.~\ref{fig:time} and stages such as initialization, data loading, etc were excluded. The online version shows faster convergence than \cite{gotmare2018decoupling} and simple SGD. Although \cite{zeng2018global} and \cite{wang2019global} have been convergence rates due to being batch methods, our approach achieves higher performance later on.% It can be also observed that our stochastic block-ADMM approach has comparable convergence speed with \cite{zeng2018global} while having noticeably superior performance over other baselines. We speculate that enforcing all the constraints by dual variables along with the efficient and cheap mini-batch updates in our method highly contributes to the convergence speed as well as its performance superiority over the other methods, including \cite{zeng2018global}. 


%---------------------------- fig time cmp mnist ------------------------------

\begin{figure}[ht]
\begin{center}
\centerline{
\includegraphics[width=\columnwidth]{imgs/time_comparison_new.pdf}
}
% \vskip -0.1in
\caption{Test set accuracy v.s. training wall clock time comparison of different alternating optimization methods for training DNNs on the MNIST dataset. Our methods (blue and orange) show superior performance vs. \protect\cite{zeng2018global} and \protect\cite{wang2019global} while converge faster than all other methods}
%  \vskip - 0.15in
\label{fig:time}
\end{center}
% \vskip -0.15in
\end{figure}



% %----------------------------
% \subsubsection{CIFAR-10}\label{exp:cifar}

% The previous works on training deep netowrks using ADMM have been limited to trivial networks and datasets (e.g. MNIST) \cite{taylor2016training,wang2019admm}. However, our proposed method does not have many of the existing restrictions and assumptions in the network architecture, as in previous works do, and can easily be extended to train non-trivial applications. It is critical to validate stochastic block-ADMM in settings where deep and modern architectures such as deep residual networks, convolutional layers, cross-entropy loss function, etc., are used. To that end, we validate the ability of our method is a supervised setting (image classification) on the CIFAR-10 dataset \cite{cifar} using ResNet-18 \cite{he2016deep}. To best of our knowledge, this is the first attempt of using ADMM for training complex networks such as ResNets. 


% For this purpose, we used 50,000 samples for training and the remaining 10,000 for evaluation. 
% To have a fair comparison, we followed the configuration suggested in \cite{gotmare2018decoupling} by converting Resnet-18 network into two blocks $(T=2)$, with the splitting point located at the end of {\sc conv3\_x} layer. We used the Adam optimizer to update both the blocks and the decoupling variables with the learning rates of $\eta_t = 5e^{-3}$ and $\zeta_t = 0.5$. We noted since the auxiliary variables $\mZ_t$ are not "shared parameters" across data samples, they usually require a higher learning rate in Algorithm \ref{alg:blockadmm}. Also, we found the ADMM step size $\beta_t = 1$ to be sufficient for enforcing the block's coupling. 


% Figure. \ref{fig:cifar} shows the results from our method compared with two baselines: \cite{gotmare2018decoupling}, and conventional end-to-end neural network training using back-propagation and SGD. Our algorithm consistently outperformed ~\cite{gotmare2018decoupling} however cannot match the conventional SGD results. There are several factors that we hypothesize that might have contributed to the performance difference: 1) in a ResNet the residual structure already partially solved the vanishing gradient problem, hence SGD/Adam performs significantly better than a fully-connected version; 
% % 2) The common data augmentation in CIFAR will end up sending a different training example to the optimization algorithm at each iteration, which does not seem to affect SGD but seem to affect ADMM convergence somewhat; 
% 2) we noticed decreasing the learning rate for $\Theta_t$ updates does not impact the performance as it does for an end-to-end back-propagation using SGD. Still, we obtained the best performance of ADMM-type methods on both MNIST and CIFAR datasets, showing the promise of our approach.
% % As illustrated, ADMM gets to a good performance fast and then slowly progress to higher accuracy..


% %---------------------------- fig cifar  ------------------------------

% \begin{figure}[htb]
% % \vskip 0.15in
% \begin{center}
% \centerline{
% \includesvg[width=\columnwidth]{imgs/cifar.svg}
% }
% % \vskip -0.05in
% \caption{Test set accuracy on CIFAR-10 dataset. Final accuracy "Block ADMM": $89.66\%$, "Gotmare \etal":$87.12 \%$, "SGD": $\bf 92.70\%$. (Best viewed in color.)}
% \label{fig:cifar}
% \end{center}
% % \vskip -0.2in
% \end{figure}

%----------------------------
\subsection{Supervised Disentangling on LFWA}\label{exp:hetero}


%----------------------------
In this section, we showcase the flexibility of stochstic block-ADMM in trainig deep networks with non-differentiable layers where conventional backpropagation cannot be used. For that purpose, we evaluate our proposed method in a supervised disentanglement problem where we used DeepFacto \ref{sec:deepfacto} to learn a nonnegative factorized representation of the DNN activations while training end-to-end on the LFWA dataset \cite{LFWTech}. Next, similar to \cite{liu2018exploring}, linear SVMs are used over the factorized space to predict face attributes. This setup examines the capability of the network to extract a disentangled representation that linearly corresponds to human-marked attributes that the network does not have prior knowledge of.

%This is to show the discriminative power of a disentangled representation. Note that there is no supervision over the attributes during the training of DeepFacto. 
% LFWA \cite{LFWTech} is a face verification dataset that contains 13,233 images with 72 attribute tags from 5,749 distinct people. 
We used the Inception-Resnet architecture from \cite{schroff2015facenet}, pre-trained on the VGGFace-2 \cite{Cao18} dataset as the back-bone. To incorporate an NMF, we follow the same approach as in Fig.~\ref{fig:deepfacto} where the pretrained DNN is the first block, and we add a simple fully-connected layer over the score matrix $\mS_t$ to train a face-verification network with a triplet loss~\cite{hoffer2015deep}.
%The choice of a simple fully-connected layer is two-folded. First, to lift the dimensions of the embedding needed for training, particularly when $\mS_t$ is low rank. Second, the embedding would be only a linear combination of the score matrix $\mS_t$, directly guiding it using the supervised signal coming from the Triplet Loss . 
We conjecture the score matrix $\mS_t$ will be guided to learn an disentangled factorization due to the nonnegativity constraint \cite{collins2018deep}. 
%Note that the latest activation in the network that is followed by a ReLU is selected from the Resnet-Inception network. This is due to the nonnegative constraint in the NMF, i.e. the input to the NMF should be also positive.
To have a warm start for an end-to-end training of DeepFacto, we first pre-train the NMF module having the Inception-Resnet block freezed. Then, we fine-tune the block parameters as well as the NMF module in an alternating fashion, similar to Algorithm \ref{alg:blockadmm}. Note, the rank of the NMF in DeepFacto is a hyperparameter and we selected three different values ($r=4, 32, 256$) in the experiments. The final $r=256$ is also the latent space dimensionality in \cite{liu2018exploring}.
Table. \ref{table:lfw} illustrates average prediction accuracy over LFWA attributes
% \footnote{The common 40 attributes with Celeb-A dataset \cite{liu2015faceattributes}} 
from DeepFacto and other supervised and weakly supervised baselines. This validates that DeepFacto has learned a meaningful representation of the attributes by disentangling the activations. To see visualization for individual dimensions learned by DeepFacto see supplementary materials \ref{sec:weakly_sup}.%, and the methodolgy to reshape the activations tensors into a matrix, 


% More details about the experiments are presented in the supplementary material.


\begin{table}[t]
\caption{Average prediction accuracy on 40 attributes from LFWA dataset. Weakly-supervised methods train the network without access to attribute labels. Final classification then comes from a linear SVM on their latent representations.}
\label{table:lfw}
\begin{center}
\begin{small}
\begin{sc}
% \vskip -0.15in
\begin{tabular}{lcccr}
\toprule
LFWA & Accuracy \\
\midrule
\cite{zhang2014panda} {\tiny (supervised)}                      &  81.00\%\\
\cite{liu2015deep} {\tiny (supervised)}                         &  84.00\%\\
\cite{liu2018exploring}  {\tiny (weakly-supervised)}             &  83.16\%\\
Deepfacto - rank 4 {\tiny (weakly-supervised)}                   & 74.80\%\\
Deepfacto - rank 32 {\tiny (weakly-supervised)}                  & 81.39\%\\
Deepfacto - rank 256 {\tiny (weakly-supervised)}                 & \textbf{87.03}\%\\
\bottomrule
\end{tabular}
\end{sc}
\end{small}
\end{center}
% \vskip -0.25in
\end{table}




% \textsc{Factorize the latent space\;} 
% {\textsc{Heatmaps.}} \cite{collins2018deep}
% {\subsection{all positive network.}}

\section{Conclusion}\label{conclusion}
In this work, we collect \DsetName, a new large-scale, multilingual moment retrieval dataset. 
It contains 218K queries in English and in Chinese from 21.8K video clips from 6 TV shows. 
We also propose a multilingual moment retrieval model \ModelName~as a strong baseline for the \DsetName~dataset. 
We show in experiments that \ModelName~outperforms monolingual models while using fewer parameters.


\section*{Acknowledgements}
We thank the reviewers for their helpful feedback. This research is supported by NSF Award \#1562098, DARPA KAIROS Grant \#FA8750-19-2-1004, and ARO-YIP Award \#W911NF-18-1-0336. The views contained in this article are those of the authors and not of the funding agency.

%%%%%%%%% BIB
\bibliographystyle{acl_natbib}
\bibliography{anthology,acl2021}



\section{Details of Motivation Study}
As introduced in Section~\ref{section:intro}, we try to answer two questions: $(i)$ whether presenting a joint image-text data from non-parallel sources would improve the learned joint embedding space than alternatively presenting uni-modal data during pre-training. $(ii)$ If we fed joint image-text data to the model, how does its existing latent alignment affect the cross-modal representation learning. 

We conduct the unsupervised vision and language pre-training on Conceptual Captions (CC) by shuffling the image-text pairs. 
For pre-training objectives, we apply standard MLM + MRM. 
All other pre-training setup is the same as introduced in Section~\ref{sec:training_setup}. 
We first compare the round-robin and joint MLM + MRM pre-training, whose results are shown in Table~\ref{tab:data-fedding}.
We then evaluate how the alignment degree of the pre-training dataset affects the model performance, where the degree is controlled by the ratio of originally aligned image-text data in Conceptual Captions.
Table~\ref{tab:paired-ratio} shows the detailed results of each downstream task.
Their Meta-Ave scores are also plotted in Fig.~\ref{fig:intro}.
From these results, we obtained two important messages: 
$(i)$ joint image-and-text input is more optimal for UVLP than alternatively presenting uni-modal data from unparallel image and text corpus. 
$(ii)$ The more the latent semantic alignment exists in the image-text data the better the pre-trained model performs. 

We further explore the realistic unsupervised V+L pre-training, where the images and texts are from two different sources.
Specifically, we sample the images from Conceptual Captions and the texts from Book Corpus respectively.
Table~\ref{tab:bc_alignment} shows that the pre-trained model on our weakly aligned CC image and BC sentence corpus far outperforms that on random pairs, indicating it also holds that better latent image-text alignment leads to better pre-trained model's performance under realistic setting.
\begin{table}[!h]
\centering
\small
\tablestyle{5pt}{0.80}
\begin{tabular}{l|ccccc}\toprule
\multirow{2}{*}{} &VQA2 &NLVR2 &VE & RefCOCO+ & \multirow{2}{*}{ Meta-Ave } \\
&Test-Dev &Test-P &Test &Devs & \\\cmidrule{1-6}
random &70.3 &51.2 &75.3 &76.5 & 68.3 \\
proposed & \bf 71.2 & \bf 67.1 & \bf 77.1 & \bf 79.7 & \bf 73.8 \\
\bottomrule
\end{tabular}
\vspace{-0.3cm}
\caption{Pre-training on realistic CC + BC data}
\label{tab:bc_alignment}
\end{table}

\section{Effectiveness of Weighted ITM}
We compared the performance of pre-training our model with or without weighted ITM. 
The models are pre-trained on CC images and texts. 
As shown in Table~\ref{tab:WITM}, weighted ITM are consistently better than treating all the retrieved pairs with the same weight. 


\begin{table}[!htp]\centering
\footnotesize
\tablestyle{3pt}{0.80}
\begin{tabular}{l|ccccc}\toprule
\multirow{2}{*}{} &VQA2 &NLVR2 &VE & RefCOCO+ & \multirow{2}{*}{ Meta-Ave } \\
&Test-Dev &Test-P &Test &Devs & \\\cmidrule{1-6}
w/o $w_{\text{ITM}}$ &71.9 &72.6 &77.0 &79.7 & 75.3 \\
$w_{\text{ITM}}$ & \bf 72.1 & \bf 73.4 & \bf 77.3 & \bf 80.3 & \bf 75.8 \\
\bottomrule
\end{tabular}
\vspace{-0.3cm}
\caption{Ablation Study on weighted ITM}
\label{tab:WITM}
\end{table}


\begin{table*}[!ht]\centering
\small
\begin{tabular}{l|c|c|c|ccc|c}\toprule
\multirow{2}{*}{ Pre-training } &VQA2 &NLVR2 &VE & \multicolumn{3}{c|}{RefCOCO+} & \multirow{2}{*}{ Meta-Ave } \\
&Test-Dev &Test-P &Test &Dev &TestA &TestB & \\\cmidrule{1-8}
Round-Robin MLM+MRM &70.4 &51.1 &74.8 &73.3 &78.3 &\textbf{67.4} & 67.4 \\
Joint MLM+MRM &\textbf{70.6} &\textbf{52.4} &\textbf{74.9} &\textbf{74.5} &\textbf{79.4} &66.8 & \textbf{68.1} \\
\bottomrule
\end{tabular}
\vspace{-0.3cm}
\caption{Detailed evaluation results on four V+L downstream tasks with two different data feeding strategy for UVLP: (1) joint image-text data (joint MLM+MRM); (2) alternative uni-modal data (round-robin MLM+MRM).}
\label{tab:data-fedding}
\end{table*}

\begin{table*}[!ht]\centering
\small
\begin{tabular}{l|c|c|c|ccc|c}\toprule
\multirow{2}{*}{ Paired Ratio } &VQA2 &NLVR2 &VE & \multicolumn{3}{c|}{RefCOCO+} & \multirow{2}{*}{ Meta-Ave } \\
&Test-Dev &Test-P &Test &Dev &TestA &TestB & \\\cmidrule{1-8}
0\% &70.6 &52.4 &74.9 &74.5 &79.4 &66.8 & 68.1 \\
20\% &71.1 &70.0 &76.4 & 76.3 &80.3 &67.5 & 73.5 \\
40\% &71.4 &71.6 &77.2 &77.9 &82.4 &68.8 & 74.5 \\
60\% &71.9 &74.5 &77.8 &79.9 &84.4 &69.9 & 76.0 \\
80\% &72.2 &75.7 &78.4 &80.9 &85.7 &71.8 & 76.8 \\
100\% &72.5 &75.9 &78.7 &82.1 &86.6 &75.0 & 77.3 \\
\bottomrule
\end{tabular}
\vspace{-0.3cm}
\caption{Detailed evaluation results on four V+L downstream tasks with 6 sets of image and text corpus of different latent cross-modal alignment degree. The alignment degree is controlled by changing the ratio of original aligned image-text data from 0\% to 100\%.}
\label{tab:paired-ratio}
\end{table*}


\begin{figure*}[h!]
\centering
\includegraphics[width=14cm]{figures/Pos_Retrieve.png}
\vspace{-0.3cm}
\caption{Examples of retrieved text from both CC and BC. The covered grounded noun phrases in retrieved sentences are highlighted in green bar for positive examples.}
\label{fig:pos-ret}
\end{figure*}

\section{Downstream Task Details}
We describe the details of fine-tuning on the four different downstream tasks: Visual Question Answering (VQA2), Natural Language for Visual Reasoning (NLVR2), Visual Entailment (VE), and Referring Expression (RefCOCO+). We mainly follow the setup of UNITER\cite{chen2020uniter} for each downstream task with minor adjustments.  

\noindent\textbf{VQA2}
Given a question about an image, the task is to predict the answer to the question. Following \cite{yu2019mcan}, we take 3,129 most frequent answers as answer candidates. We use both training and validation sets from VQA 2.0 for fine-tuning. Following UNITER, we also leverage data from Visual Genome\cite{krishna2017visualgenome} to augment the best performance on the test-dev split. We fine-tune the model with a binary cross-entropy loss with a peak learning rate of $6\times10^{-5}$ for 20 epochs. The training batch size is set as 480. 

\noindent\textbf{NLVR2}
NLVR2 is a task for visual reasoning. The objective is to determine whether a natural language statement is true or not given a pair of input images. 
We follow UNITER's setup treating each data point as two text-image pairs with repeated text. 
The two [CLS] outputs from the model are then concatenated as the joint embedding for the example. We apply a multi-layer perceptron (MLP) classifier on top of this joint embedding for the final classification. Unlike~\cite{li2020unsupervised} that conducts additional ``pre-training" on NLVR2 datasets, we only fine-tune the model with the task-specific objective to maintain the same setting as all other downstream tasks. We train the model for 8 epochs with a batch size of 60 and a peak learning rate of $3\times10^{-5}$. 

\noindent\textbf{VE}
Visual Entailment is a task built on Flickr30k Images\cite{young-etal-2014-image}, where the goal is to determine the logical relationship between a natural language statement and an image. Similar to the Natural Language Inference problem in NLP, this task is formatted as a 3-way classification problem to predict if the language statement entails, contradicts, or is undetermined with respect to the given image. An MLP transformer classifier is applied to the output of the $\text{[CLS]}$ token to make the final prediction. The model is fine-tuned using cross-entropy loss. We set the batch size as 480 and the peak learning rate as $8\times10^{-5}$. The model is fine-tuned for 4 epochs for this downstream task. 

\noindent\textbf{RefCOCO+}
The referring expression task involves locating an image region given a natural language phrase. We use RefCOCO+ \cite{yu2016modeling} as the evaluation dataset. Bounding box proposals from VinVL object detectors are used for fine-tuning. A proposal box is considered correct if it has an IoU with a gold box larger than 0.5. We add an MLP layer on top of the region outputs from the Transformer to compute the alignment score between the language phrase and each proposed region. We fine-tune our model for 20 epochs with a peak-learning rate of $2\times10^{-4}$.


\begin{figure*}[h!]
\centering
\includegraphics[width=14cm]{figures/Neg_Retrieve.png}
\vspace{-0.1cm}
\caption{Examples of retrieved text from both CC and BC. The mistakenly covered grounded noun phrases in retrieved sentences are highlighted in red bar for negative examples.}
\label{fig:neg-ret}
\end{figure*}

\begin{figure}[h!]
\centering
\includegraphics[width=0.7\linewidth]{figures/attention_viz_1.jpg}
\vspace{-0.2cm}
\caption{Text-to-image attention given the aligned pair whose caption is ``person in a leather jacket riding a motorcycle on the road".}
\label{fig:attn_viz_1}
\end{figure}

\begin{figure}[h!]
\centering
\includegraphics[width=0.7\linewidth]{figures/attention_viz_2.jpg}
\vspace{-0.2cm}
\caption{Text-to-image attention given the aligned pair whose caption is ``girl in a blue kayak floating on the picturesque river at sunset".}
\label{fig:attn_viz_2}
\end{figure}

\begin{figure}[h!]
\centering
\includegraphics[width=0.7\linewidth]{figures/attention_viz_3.jpg}
\vspace{-0.2cm}
\caption{Text-to-image attention given the aligned pair whose caption is ``people walking by the christmas tree and stage area".}
\label{fig:attn_viz_3}
\end{figure}

\begin{figure}[h!]
\centering
\includegraphics[width=0.7\linewidth]{figures/attention_viz_4.jpg}
\vspace{-0.2cm}
\caption{Text-to-image attention given the aligned pair whose caption is ``single cowboy guiding a line of horses through the desert".}
\label{fig:attn_viz_4}
\end{figure}
\section{Additional Visualization}
We present additional examples of retrieved text from both CC and BookCorpus. Specifically, we demonstrate more positive examples in Fig \ref{fig:pos-ret} that covers the appropriate grounded noun phrases. We also share some negative examples in Fig \ref{fig:pos-ret}. As analyzed in the limitation section, the current language embedding model weighs all the object tags equally to generate the joint embedding representation. This can lead to mistakenly focused object tags when retrieving the text. In row 1 of Fig \ref{fig:neg-ret}, texts retrieved cover less important noun phrases such as ``finger" and ``hair" instead of the more important noun phrase "baby". Row 2 of Fig \ref{fig:neg-ret} demonstrate mistakenly retrieved texts due to the limitation of the pre-defined object categories in the object detector. In this example,  the students in the image are detected as ``person" or ``man", which leads to the failure of retrieving any valid text.    

We also demonstrate more examples on text-to-image attention between the pre-trained U-VisualBert and {\ModelName } on the Conceptual Captions Validation set in Fig \ref{fig:attn_viz_1}, \ref{fig:attn_viz_2}, \ref{fig:attn_viz_3}, \ref{fig:attn_viz_4}. These examples provide additional evidence on the better local alignment captured by \ModelName. 


\end{document}
