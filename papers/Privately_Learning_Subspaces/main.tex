\def\epsdeltastuff{0}
\def\comments{0}
\def\supp{0}
\documentclass[letterpaper,11pt]{article}
\usepackage[utf8]{inputenc}
\usepackage{times}
\usepackage{comment}
\usepackage{preamble}
\allowdisplaybreaks
\newcommand{\dr}[3]{\mathrm{D}_{#1}\left(#2\middle\|#3\right)}
\newcommand{\nope}[1]{}
\newcommand{\stcomp}[1]{\overline{#1}}
\newcommand{\iprod}[1]{\langle #1 \rangle}
\newcommand{\llnorm}[1]{\left\lVert#1\right\rVert_2}
\newcommand{\fnorm}[1]{\left\lVert#1\right\rVert_F}
\newcommand{\norm}[1]{\left\lVert#1\right\rVert}
\newcommand{\sgnorm}[1]{\left\lVert#1\right\rVert_{\Psi^2}}
\newcommand{\abs}[1]{\left| #1 \right|}
\newcommand{\chisq}[1]{\chi^2\left( #1 \right)}
\renewcommand{\epsilon}{\varepsilon}
\newcommand{\ball}[2]{\mathit{B}_{#2}\left( #1 \right)}
\newcommand{\myeqand}{\quad \textrm{and} \quad}

\newcommand{\X}{\mathcal{X}}
\newcommand{\Q}{\mathcal{Q}}
\newcommand{\M}{\mathcal{M}}

\newcommand{\EE}{\mathbb{E}}
\newcommand{\RR}{\mathbb{R}}

\newcommand{\ONE}{\mathbbm{1}}

\newcommand{\vars}{\mathit{Var}}

\newcommand{\IN}{\mathsf{IN}}
\newcommand{\OUT}{\mathsf{OUT}}
\newcommand{\NULL}{\mathsf{NULL}}

\newcommand{\Sym}{\mathsf{Sym}}

\renewcommand{\k}{\kappa}
\newcommand{\id}{\mathbb{I}}
\newcommand{\Lap}{\mathrm{Lap}}
\newcommand{\TLap}{\mathrm{TLap}}
\newcommand{\opt}{\mathrm{opt}}
\newcommand{\OPT}{\mathrm{OPT}}
\newcommand{\Score}{\textsc{Score}}
\newcommand{\Match}{\mathrm{Match}}
\newcommand{\Median}{\mathrm{Median}}
\newcommand{\pcount}{\mathrm{Count}}

\newcommand{\DPASE}{\mathrm{DPASE}}
\newcommand{\DPASEB}{\mathrm{DPASEB}}
\newcommand{\DPESE}{\mathrm{DPESE}}
\newcommand{\GAPMAX}{\mathrm{GAP\text{-}MAX}}

\newcommand{\tnote}[1]{}%\textcolor{red}{[Thomas: #1]}}
\newcommand{\vnote}[1]{}%\textcolor{orange}{[Vikrant: #1]}}

\usepackage[style=alphabetic,backend=bibtex,maxalphanames=10,maxbibnames=20,maxcitenames=10,giveninits=true,doi=false,url=true]{biblatex}
\newcommand*{\citet}[1]{\AtNextCite{\AtEachCitekey{\defcounter{maxnames}{2}}}\textcite{#1}}
\newcommand*{\citetall}[1]{\AtNextCite{\AtEachCitekey{\defcounter{maxnames}{999}}}\textcite{#1}}
\newcommand*{\citep}[1]{\citep{#1}}
\newcommand{\citeyearpar}[1]{\citep{#1}}
\usepackage{hyperref}

\addbibresource{biblio.bib}

\title{Privately Learning Subspaces}
\iftrue
\author{Vikrant Singhal
    \thanks{Northeastern University. Part of this work was done during an internship at IBM Research -- Almaden.~\dotfill~\texttt{singhal.vi@northeastern.edu}}
    \and
    Thomas Steinke\thanks{Google Research, Brain Team. Part of this work was done at IBM Research --
    Almaden.~\dotfill~\texttt{subspace@thomas-steinke.net}}}
\date{}
\fi
%\coltauthor{%
% \Name{Vikrant Singhal} \Email{singhal.vi@northeastern.edu}\\
% \addr Northeastern University
% \AND
% \Name{Thomas Steinke} \Email{subspace@thomas-steinke.net}\\
% \addr Google Research, Brain Team
%}

\begin{document}
\maketitle

\begin{abstract}
    Private data analysis suffers a costly curse of dimensionality. However, the data often has an underlying low-dimensional structure. For example, when optimizing via gradient descent, the gradients often lie in or near a low-dimensional subspace. If that low-dimensional structure can be identified, then we can avoid paying (in terms of privacy or accuracy) for the high ambient dimension. 
    
    We present differentially private algorithms that take input data sampled from a low-dimensional linear subspace (possibly with a small amount of error) and output that subspace (or an approximation to it). These algorithms can serve as a pre-processing step for other procedures.
\end{abstract}
%\newpage

% !TEX root = ../arxiv.tex

Unsupervised domain adaptation (UDA) is a variant of semi-supervised learning \cite{blum1998combining}, where the available unlabelled data comes from a different distribution than the annotated dataset \cite{Ben-DavidBCP06}.
A case in point is to exploit synthetic data, where annotation is more accessible compared to the costly labelling of real-world images \cite{RichterVRK16,RosSMVL16}.
Along with some success in addressing UDA for semantic segmentation \cite{TsaiHSS0C18,VuJBCP19,0001S20,ZouYKW18}, the developed methods are growing increasingly sophisticated and often combine style transfer networks, adversarial training or network ensembles \cite{KimB20a,LiYV19,TsaiSSC19,Yang_2020_ECCV}.
This increase in model complexity impedes reproducibility, potentially slowing further progress.

In this work, we propose a UDA framework reaching state-of-the-art segmentation accuracy (measured by the Intersection-over-Union, IoU) without incurring substantial training efforts.
Toward this goal, we adopt a simple semi-supervised approach, \emph{self-training} \cite{ChenWB11,lee2013pseudo,ZouYKW18}, used in recent works only in conjunction with adversarial training or network ensembles \cite{ChoiKK19,KimB20a,Mei_2020_ECCV,Wang_2020_ECCV,0001S20,Zheng_2020_IJCV,ZhengY20}.
By contrast, we use self-training \emph{standalone}.
Compared to previous self-training methods \cite{ChenLCCCZAS20,Li_2020_ECCV,subhani2020learning,ZouYKW18,ZouYLKW19}, our approach also sidesteps the inconvenience of multiple training rounds, as they often require expert intervention between consecutive rounds.
We train our model using co-evolving pseudo labels end-to-end without such need.

\begin{figure}[t]%
    \centering
    \def\svgwidth{\linewidth}
    \input{figures/preview/bars.pdf_tex}
    \caption{\textbf{Results preview.} Unlike much recent work that combines multiple training paradigms, such as adversarial training and style transfer, our approach retains the modest single-round training complexity of self-training, yet improves the state of the art for adapting semantic segmentation by a significant margin.}
    \label{fig:preview}
\end{figure}

Our method leverages the ubiquitous \emph{data augmentation} techniques from fully supervised learning \cite{deeplabv3plus2018,ZhaoSQWJ17}: photometric jitter, flipping and multi-scale cropping.
We enforce \emph{consistency} of the semantic maps produced by the model across these image perturbations.
The following assumption formalises the key premise:

\myparagraph{Assumption 1.}
Let $f: \mathcal{I} \rightarrow \mathcal{M}$ represent a pixelwise mapping from images $\mathcal{I}$ to semantic output $\mathcal{M}$.
Denote $\rho_{\bm{\epsilon}}: \mathcal{I} \rightarrow \mathcal{I}$ a photometric image transform and, similarly, $\tau_{\bm{\epsilon}'}: \mathcal{I} \rightarrow \mathcal{I}$ a spatial similarity transformation, where $\bm{\epsilon},\bm{\epsilon}'\sim p(\cdot)$ are control variables following some pre-defined density (\eg, $p \equiv \mathcal{N}(0, 1)$).
Then, for any image $I \in \mathcal{I}$, $f$ is \emph{invariant} under $\rho_{\bm{\epsilon}}$ and \emph{equivariant} under $\tau_{\bm{\epsilon}'}$, \ie~$f(\rho_{\bm{\epsilon}}(I)) = f(I)$ and $f(\tau_{\bm{\epsilon}'}(I)) = \tau_{\bm{\epsilon}'}(f(I))$.

\smallskip
\noindent Next, we introduce a training framework using a \emph{momentum network} -- a slowly advancing copy of the original model.
The momentum network provides stable, yet recent targets for model updates, as opposed to the fixed supervision in model distillation \cite{Chen0G18,Zheng_2020_IJCV,ZhengY20}.
We also re-visit the problem of long-tail recognition in the context of generating pseudo labels for self-supervision.
In particular, we maintain an \emph{exponentially moving class prior} used to discount the confidence thresholds for those classes with few samples and increase their relative contribution to the training loss.
Our framework is simple to train, adds moderate computational overhead compared to a fully supervised setup, yet sets a new state of the art on established benchmarks (\cf \cref{fig:preview}).


%!TEX root = hopfwright.tex
%

In this section we systematically recast the Hopf bifurcation problem in Fourier space. 
We introduce appropriate scalings, sequence spaces of Fourier coefficients and convenient operators on these spaces. 
To study Equation~\eqref{eq:FourierSequenceEquation} we consider Fourier sequences $ \{a_k\}$ and fix a Banach space in which these sequences reside. It is indispensable for our analysis that this space have an algebraic structure. 
The Wiener algebra of absolutely summable Fourier series is a natural candidate, which we use with minor modifications. 
In numerical applications, weighted sequence spaces with algebraic and geometric decay have been used to great effect to study periodic solutions which are $C^k$ and analytic, respectively~\cite{lessard2010recent,hungria2016rigorous}. 
Although it follows from Lemma~\ref{l:analytic} that the Fourier coefficients of any solution decay exponentially, we choose to work in a space of less regularity. 
The reason is that by working in a space with less regularity, we are better able to connect our results with the global estimates in \cite{neumaier2014global}, see Theorem~\ref{thm:UniqunessNbd2}.


%
%
%\begin{remark}
%	Although it follows from Lemma~\ref{l:analytic} that the Fourier coefficients of any solution decay exponentially, we choose to work in a space of less regularity, namely summable Fourier coefficients. This allows us to draw SOME MORE INTERESTING CONCLUSION LATER.
%	EXPLAIN WHY WE CHOOSE A NORM WITH ALMOST NO DECAY!
%	% of s Periodic solutions to Wright's equation are known to be real analytic and so their  Fourier coefficients must decay geometrically [Nussbaum].
%	% We do not use such a strong result;  any periodic solution must be continuously differentiable, by which it follows that $ \sum | c_k| < \infty$.
%\end{remark}


\begin{remark}\label{r:a0}
There is considerable redundancy in Equation~\eqref{eq:FourierSequenceEquation}. First, since we are considering real-valued solutions $y$, we assume $\c_{-k}$ is the complex conjugate of $\c_k$. This symmetry implies it suffices to consider Equation~\eqref{eq:FourierSequenceEquation} for $k \geq 0$.
Second, we may effectively ignore the zeroth Fourier coefficient of any periodic solution \cite{jones1962existence}, since it is necessarily equal to $0$. 
%In \cite{jones1962existence}, it is shown that if $y \not\equiv -1$ is a periodic solution of~\eqref{eq:Wright} with frequency $\omega$, then $ \int_0^{2\pi/\omega} y(t) dt =0$. 
		The self contained argument is as follows. 
		As mentioned in the introduction, any periodic solution to Wright's equation must satisfy $ y(t) > -1$ for all $t$. 
	By dividing Equation~\eqref{eq:Wright} by $(1+y(t))$, which never vanishes, we obtain
	\[
	\frac{d}{dt} \log (1 + y(t)) = - \alpha y(t-1).
	\]  
	Integrating over one period $L$ we derive the condition 
	$0=\int_0^L y(t) dt $.
	Hence $a_0=0$ for any periodic solution. 
	It will be shown in Theorem~\ref{thm:FourierEquivalence1} that a related argument implies that we do not need to consider Equation~\eqref{eq:FourierSequenceEquation} for $k=0$.
\end{remark}

%%%
%%%
%%%\begin{remark}\label{r:c0} 
%%%In \cite{jones1962existence}, it is shown that if $y \not\equiv -1$ is a periodic solution of~\eqref{eq:Wright} with frequency $\omega$, then $ \int_0^{2\pi/\omega} y(t) dt =0$. 
%%%PERHAPS TOO MUCH DETAIL HERE. The self contained argument is as follows.
%%%If $y \not\equiv -1$ then $y(t) \neq -1$ for all $t$, since if $y(t_0)=-1$ for some $t_0 \in \R$ then $y'(t_0)=0$ by~\eqref{eq:Wright} and in fact by differentiating~\eqref{eq:Wright} repeatedly one obtains that all derivatives of $y$ vanish at $t_0$. Hence $y \equiv -1$ by Lemma~\ref{l:analytic}, a contradiction. Now divide~\eqref{eq:Wright} by $(1+y(t))$, which never vanishes, to obtain
%%%\[
%%%  \frac{d}{dt} \log |1 + y(t)| = - \alpha y(t-1).
%%%\]  
%%%Integrating over one period we obtain $\int_0^L y(t) dt =0$.
%%%\end{remark}



%Furthermore, the condition that $y(t)$ is real forces $\c_{-k} = \overline{\c}_{k}$.  
%
We define the spaces of absolutely summable Fourier series
\begin{alignat*}{1}
	\ell^1 &:= \left\{ \{ \c_k \}_{k \geq 1} : 
    \sum_{k \geq 1} | \c_k| < \infty  \right\} , \\
	\ell^1_\bi &:= \left\{ \{ \c_k \}_{k \in \Z} : 
    \sum_{k \in \Z} | \c_k| < \infty  \right\} .
\end{alignat*} 
We identify any semi-infinite sequence $ \{ \c_k \}_{k \geq 1} \in \ell^1$ with the bi-infinite sequence $ \{ \c_k \}_{k \in \Z} \in \ell^1_\bi$ via the conventions (see Remark~\ref{r:a0})
\begin{equation}
  \c_0=0 \qquad\text{ and }\qquad \c_{-k} = \c_{k}^*. 
\end{equation}
In other word, we identify $\ell^1$ with the set
\begin{equation*}
   \ell^1_\sym := \left\{ \c \in \ell^1_\bi : 
	\c_0=0,~\c_{-k}=\c_k^* \right\} .
\end{equation*}
On $\ell^1$ we introduce the norm
\begin{equation}\label{e:lnorm}
  \| \c \| = \| \c \|_{\ell^1} := 2 \sum_{k = 1}^\infty |\c_k|.
\end{equation}
The factor $2$ in this norm is chosen to have a Banach algebra estimate.
Indeed, for $\c, \tilde{\c} \in \ell^1 \cong \ell^1_\sym$ we define
the discrete convolution 
\[
\left[ \c * \tilde{\c} \right]_k = \sum_{\substack{k_1,k_2\in\Z\\ k_1 + k_2 = k}} \c_{k_1} \tilde{\c}_{k_2} .
\]
Although $[\c*\tilde{\c}]_0$ does not necessarily vanish, we have $\{\c*\tilde{\c}\}_{k \geq 1} \in \ell^1 $ and 
\begin{equation*}
	\| \c*\tilde{\c} \| \leq \| \c \| \cdot  \| \tilde{\c} \| 
	\qquad\text{for all } \c , \tilde{\c} \in \ell^1, 
\end{equation*}
hence $\ell^1$ with norm~\eqref{e:lnorm} is a Banach algebra.

By Lemma~\ref{l:analytic} it is clear that any periodic solution of~\eqref{eq:Wright} has a well-defined Fourier series $\c \in \ell^1_\bi$. 
The next theorem shows that in order to study periodic orbits to Wright's equation we only need to study Equation~\eqref{eq:FourierSequenceEquation} 
for $k \geq 1$. For convenience we introduce the notation 
\[
G(\alpha,\omega,\c)_k=
( i \omega k + \alpha e^{ - i \omega k}) \c_k + \alpha \sum_{k_1 + k_2 = k} e^{- i \omega k_1} \c_{k_1} \c_{k_2} \qquad \text{for } k \in \N.
\]
We note that we may interpret the trivial solution $y(t)\equiv 0$ as a periodic solution of arbitrary period.
\begin{theorem}
\label{thm:FourierEquivalence1}
Let $\alpha>0$ and $\omega>0$.
If $\c \in \ell^1 \cong \ell^1_{\sym}$ solves
$G(\alpha,\omega,\c)_k =0$  for all $k \geq 1$,
then $y(t)$ given by~\eqref{eq:FourierEquation} is a periodic solution of~\eqref{eq:Wright} with period~$2\pi/\omega$.
Vice versa, if $y(t)$ is a periodic solution of~\eqref{eq:Wright} with period~$2\pi/\omega$ then its Fourier coefficients $\c \in \ell^1_\bi$ lie in $\ell^1_\sym \cong \ell^1$ and solve $G(\alpha,\omega,\c)_k =0$ for all $k \geq 1$.
\end{theorem}

\begin{proof}	
	If $y(t)$ is a periodic solution of~\eqref{eq:Wright} then it is real analytic by Lemma~\ref{l:analytic}, hence its Fourier series $\c$ is well-defined and $\c \in \ell^1_{\sym}$ by Remark~\ref{r:a0}.
	Plugging the Fourier series~\eqref{eq:FourierEquation} into~\eqref{eq:Wright} one easily derives that $\c$ solves~\eqref{eq:FourierSequenceEquation} for all $k \geq 1$.

To prove the reverse implication, assume that $\c \in \ell^1_\sym$ solves
Equation~\eqref{eq:FourierSequenceEquation} for all $k \geq 1$. Since $\c_{-k}
= \c_k^*$, Equation \eqref{eq:FourierSequenceEquation} is also satisfied for
all $k \leq -1$. It follows from the Banach algebra property and
\eqref{eq:FourierSequenceEquation} that $\{k \c_k\}_{k \in \Z} \in \ell^1_\bi$,
hence $y$, given by~\eqref{eq:FourierEquation}, is continuously differentiable.
% (and by bootstrapping one infers that $\{k^m c_k \} \in \ell^1_\bi$, 
% hence $y \in C^m$ for any $m \geq 1$).
	Since~\eqref{eq:FourierSequenceEquation} is satisfied for all $k \in \Z \setminus \{0\}$ (but not necessarily for $k=0$) one may perform the inverse Fourier transform on~\eqref{eq:FourierSequenceEquation} to conclude that
	$y$ satisfies the delay equation 
\begin{equation}\label{eq:delaywithK}
   	y'(t) = - \alpha y(t-1) [ 1 + y(t)] + C
\end{equation}
	for some constant $C \in \R$. 
   Finally, to prove that $C=0$ we argue by contradiction.
   Suppose $C \neq 0$. Then $y(t) \neq -1$ for all $t$.
   Namely, at any point where $y(t_0) =-1$ one would have $y'(t_0) = C$
   which has fixed sign,   hence it would follow that $y$ is not periodic
   ($y$ would not be able to cross $-1$ in the opposite direction, 
   preventing $y$  from being periodic).  
  We may thus divide~\eqref{eq:delaywithK} through by $1 + y(t)$ and obtain 
\begin{equation*}
	\frac{d}{dt} \log | 1 + y(t) | = - \alpha y(t-1) + \frac{C}{1+y(t)} .
\end{equation*}
	By integrating both sides of the equation over one period $L$ and by using that $\c_0=0$, we 
	obtain
	\[
	 C \int_0^L \frac{1}{1+y(t)} dt =0.
	\]
	Since the integrand is either strictly negative or strictly positive, this implies that $C=0$. Hence~\eqref{eq:delaywithK} reduces to~\eqref{eq:Wright},
	and $y$ satisfies Wright's equation. 
\end{proof}






To efficiently study Equation~\eqref{eq:FourierSequenceEquation}, we introduce the following linear operators on $ \ell^1$:
\begin{alignat*}{1}
   [K \c ]_k &:= k^{-1} \c_k  , \\ 
   [ U_\omega \c ]_k &:= e^{-i k \omega} \c_k  .
\end{alignat*}
The map $K$ is a compact operator, and it has a densely defined inverse $K^{-1}$. The domain of $K^{-1}$ is denoted by
\[
  \ell^K := \{ \c \in \ell^1 : K^{-1} \c \in \ell^1 \}.  
\]
The map $U_{\omega}$ is a unitary operator on $\ell^1$, but
it is discontinuous in $\omega$. 
With this notation, Theorem~\ref{thm:FourierEquivalence1} implies that our problem of finding a SOPS to~\eqref{eq:Wright} is equivalent to finding an $\c \in \ell^1$ such that
\begin{equation}
\label{e:defG}
  G(\alpha,\omega,\c) :=
  \left( i \omega K^{-1} + \alpha U_\omega \right) \c + \alpha \left[U_\omega \, \c \right] * \c  = 0.
\end{equation}


%In order for the solutions of Equation \ref{eq:FHat} to be isolated we need to impose a phase condition. 
%If there is a sequence $ \{ c_k \} $ which satisfies  Equation \ref{eq:FHat}, then $ y( t + \tau) = \sum_{k \in \Z} c_k e^{ i k \omega (t + \tau)}$ satisfies Wright's equation at parameter $\alpha$. 
%Fix $ \tau = - Arg[c_1] / \omega$ so that $ c_1  e^{ i \omega \tau} $ is a nonnegative real number. 
%By Proposition \ref{thm:FourierEquivalence1} it follows that $\{ c'_k \} =  \{c_k e^{ i \omega k \tau }   \}$ is a solution to Equation \ref{eq:FHat}, and furthermore that $ c'_1 = \epsilon$ for some $ \epsilon \geq 0$. 


Periodic solutions are invariant under time translation: if $y(t)$ solves Wright's equation, then so does $ y(t+\tau)$ for any $\tau \in \R$. 
We remove this degeneracy by adding a phase condition. 
Without loss of generality, if $\c \in \ell^1$ solves Equation~\eqref{e:defG}, we may assume that $\c_1 = \epsilon$ for some 
\emph{real non-negative}~$\epsilon$:
\[
  \ell^1_{\epsilon} := \{\c \in \ell^1 : \c_1 = \epsilon \} 
  \qquad \text{where } \epsilon \in \R,  \epsilon \geq 0.
\]
In the rest of our analysis, we will split elements $\c \in \ell^1$ into two parts: $\c_1$ and $\{\c_{k}\}_{k \geq 2}$.  
We define the basis elements $\e_j \in \ell^1$ for $j=1,2,\dots$ as
\[
  [\e_j]_k = \begin{cases}
  1 & \text{if } k=j, \\
  0 & \text{if } k \neq j.
  \end{cases}
\]
We note that $\| \e_j \|=2$. 
Then we can decompose
% We define
% \[
%   \tilde{\epsilon} := (\epsilon,0,0,0,\dots) \in \ell^1
% \]
% and
% For clarity when referring to sequences $\{c_{k}\}_{k \geq 2}$, we make the following definition:
% \[
% \ell^1_0  := \{ \tc \in \ell^1 : \tc_1 = 0 \}.
% \]
% With the
any $\c \in \ell^1_\epsilon$ uniquely as
\begin{equation}\label{e:aepsc}
  \c= \epsilon \e_1 + \tc \qquad \text{with}\quad 
  \tc \in \ell^1_0 := \{ \tc \in \ell^1 : \tc_1 = 0 \}.
\end{equation}
We follow the classical approach in studying Hopf bifurcations and consider 
$\c_1 = \epsilon$ to be a parameter, and then find periodic solutions with Fourier modes in $\ell^1_{\epsilon}$.
This approach rewrites the function $G: \R^2 \times \ell^K \to \ell^1$ as a function $\tilde{F}_\epsilon : \R^2 \times \ell^K_0 \to \ell^1$, where 
we denote 
\[
\ell^K_0 := \ell^1_0 \cap \ell^K.
\]
% I AM ACTUALLY NOT SURE IF YOU WANT TO DEFINE THIS WITH RANGE IN $\ell^1$
% OR WITH DOMAIN IN $\ell^1_0$ ?? IT SEEMS TO DEPEND ON WHICH GLOBAL STATEMENT YOU WANT/NEED TO MAKE!?
\begin{definition}
We define the $\epsilon$-parameterized family of  functions $\tilde{F}_\epsilon: \R^2 \times \ell^K_0  \to \ell^1$ 
by 
\begin{equation}
\label{eq:fourieroperators}
\tilde{F}_{\epsilon}(\alpha,\omega, \tc) := 
\epsilon [i \omega + \alpha e^{-i \omega}] \e_1 + 
( i \omega K^{-1} + \alpha U_{\omega}) \tc + 
\epsilon^2 \alpha e^{-i \omega}  \e_2  +
\alpha \epsilon L_\omega \tc + 
\alpha  [ U_{\omega} \tc] * \tc ,
\end{equation}
where
$L_\omega : \ell^1_0 \to \ell^1$ is given by
\[
   L_{\omega} := \sigma^+( e^{- i \omega} I + U_{\omega}) + \sigma^-(e^{i \omega} I + U_{\omega}),
\]
with $I$ the identity and  $\sigma^\pm$ the shift operators on $\ell^1$:
\begin{alignat*}{2}
\left[ \sigma^- a \right]_k &:=  a_{k+1}  , \\
\left[ \sigma^+ a \right]_k &:=  a_{k-1}  &\qquad&\text{with the convention } \c_0=0.
\end{alignat*}
The operator $ L_\omega$ is discontinuous in $\omega$ and $ \| L_\omega \| \leq 4$. 
\end{definition} 

%The maps $ \sigma^{+}$ and $ \sigma^-$ are shift up and shift down operators respectively. 
We reformulate Theorem~\ref{thm:FourierEquivalence1}  in terms of the map  $\tilde{F}$. 
We note that it follows from Lemma~\ref{l:analytic} and 
%\marginpar{Reformulate}
%one's choice of  
Equation~\eqref{eq:FourierSequenceEquation}  
%or Equation ~\eqref{eq:fourieroperators},
that the Fourier coefficients of any periodic solution of~\eqref{eq:Wright} lie in $\ell^K$.
These observations are summarized in the following theorem.
\begin{theorem}
\label{thm:FourierEquivalence2}
	Let $ \epsilon \geq 0$,  $\tc \in \ell^K_0$, $\alpha>0$ and $ \omega >0$. 
	Define $y: \R\to \R$ as 
\begin{equation}\label{e:ytc}
	y(t) = 
	\epsilon \left( e^{i \omega t }  + e^{- i \omega t }\right) 
	+  \sum_{k = 2}^\infty   \tc_k e^{i \omega k t }  + \tc_k^* e^{- i \omega k t } .
\end{equation}
%	and suppose that $ y(t) > -1$. 
	Then $y(t)$ solves~\eqref{eq:Wright} if and only if $\tilde{F}_{\epsilon}( \alpha , \omega , \tc) = 0$. 
	Furthermore, up to time translation, any periodic solution of~\eqref{eq:Wright} with period $2\pi/\omega$ is described by a Fourier series of the form~\eqref{e:ytc} with $\epsilon \geq 0$ and $\tc \in \ell^K_0$.
\end{theorem}


%We note that for $\epsilon>0$ such solutions are truly periodic, while for $\epsilon=0$ a zero of $\tilde{F}_\epsilon$ may either correspond to a periodic solution or to the trivial solution $y(t) \equiv 0$. 



% \begin{proof}
%  By Proposition \ref{thm:FourierEquivalence1}, it suffices to show that $\tilde{F}(\alpha,\omega,c) =0$ is equivalent to Equation \ref{eq:FourierSequenceEquation} being satisfied for $k \geq 1$.
%  Since Equation \ref{eq:FourierSequenceEquation} is equivalent to Equation \ref{eq:FHat}, we expand  Equation \ref{eq:FHat} by writing $ \hat{c} = \hat{\epsilon } + c$  where $ \hat{\epsilon} := (\epsilon,0,0,\dots) \in \ell^1$ as below:
%  \begin{equation}
%  0=  \left( i \omega K^{-1} + \alpha U_\omega \right) (\hat{\epsilon}+ c) + \alpha \left[U_\omega \, (\hat{\epsilon}+ c) \right] * (\hat{\epsilon}+ c) \label{eq:Intial}
%  \end{equation}
%  The RHS of Equation \ref{eq:Intial} is $ \tilde{F}(\alpha,\omega,c)$, so the theorem is proved.
% \end{proof}



Since we want to analyze a Hopf bifurcation, we will want to solve $\tilde{F}_\epsilon = 0$ for small values of~$\epsilon$. 
However, at the bifurcation point, $ D \tilde{F}_0(\pp  ,\pp , 0)$ is not invertible.
In order for our asymptotic analysis to be non-degenerate,
we work with a rescaled version of the problem. To this end, for any $\epsilon >0$, we rescale both $\tc$ and $\tilde{F}$ as follows. Let $\tc = \epsilon c$ and 
\begin{equation}\label{e:changeofvariables}
  \tilde{F}_\epsilon (\alpha,\omega,\epsilon c) = \epsilon F_\epsilon (\alpha,\omega,c).
\end{equation}
For $\epsilon>0$ the problem then reduces to finding zeros of 
\begin{equation}
\label{eq:FDefinition}
	F_\epsilon(\alpha,\omega, c) := 
	[i \omega + \alpha e^{-i \omega}] \e_1 + 
	( i \omega K^{-1} + \alpha U_{\omega}) c + 
	\epsilon \alpha e^{-i \omega} \e_2  +
	\alpha \epsilon L_\omega c + 
	\alpha \epsilon [ U_{\omega} c] * c.
\end{equation}
We denote the triple $(\alpha,\omega,c) \in \R^2 \times \ell^1_0$ by $x$.
To pinpoint the components of $x$ we use the projection operators
\[
   \pi_\alpha x = \alpha, \quad \pi_\omega x = \omega, \quad 
  \pi_c x = c \qquad\text{for any } x=(\alpha,\omega,c).
\]

After the change of variables~\eqref{e:changeofvariables} we now have an invertible Jacobian $D F_0(\pp  ,\pp , 0)$ at the bifurcation point.
On the other hand, for $\epsilon=0$ the zero finding problems for $\tilde{F}_\epsilon$ and $F_\epsilon$ are not equivalent. 
However, it follows from the following lemma that any nontrivial periodic solution having $ \epsilon=0$ must have a relatively large size when $ \alpha $ and $ \omega $ are close to the bifurcation point. 

\begin{lemma}\label{lem:Cone}
	Fix $ \epsilon \geq 0$ and $\alpha,\omega >0$. 
	Let
	\[
	b_* :=  \frac{\omega}{\alpha} - \frac{1}{2} - \epsilon  \left(\frac{2}{3}+ \frac{1}{2}\sqrt{2 + 2 |\omega-\pp| } \right).
	\]
Assume that $b_*> \sqrt{2} \epsilon$. 
Define
% \begin{equation*}%\label{e:zstar}
% 	z^{\pm}_* :=b_* \pm \sqrt{(b_*)^2- \epsilon^2 } .
% \end{equation*}
% \note[J]{Proposed change to match Lemma E.4}
\begin{equation}\label{e:zstar}
z^{\pm}_* :=b_* \pm \sqrt{(b_*)^2- 2 \epsilon^2 } .
\end{equation}
If there exists a $\tc \in \ell^1_0$ such that $\tilde{F}_\epsilon(\alpha, \omega,\tc) = 0$, then \\
\mbox{}\quad\textup{(a)} either $ \|\tc\| \leq  z_*^-$ or $ \|\tc\| \geq z_*^+  $.\\
\mbox{}\quad\textup{(b)} 
$ \| K^{-1} \tc \| \leq (2\epsilon^2+ \|\tc\|^2) / b_*$. 
\end{lemma}
\begin{proof}
	The proof follows from Lemmas~\ref{lem:gamma} and~\ref{lem:thecone} in Appendix~\ref{appendix:aprioribounds}, combined with the observation that
$\frac{\omega}{\alpha} - \gamma \geq b_*$,
% \[
%   \frac{\omega}{\alpha} - \gamma \geq b_*
%  \qquad\text{for all }
% | \alpha - \pp| \leq r_\alpha \text{ and } 
%   | \omega - \pp| \leq r_\omega.
% \]
with $\gamma$ as defined in Lemma~\ref{lem:gamma}.
\end{proof}

\begin{remark}\label{r:smalleps}
We note that for $\alpha < 2\omega$
\begin{alignat*}{1}
z^+_* &\geq   \frac{2 \omega - \alpha}{\alpha} 
- \epsilon \left(4/3+\sqrt{2 + 2 |\omega-\pp| } \, \right) + \cO(\epsilon^2)
\\[1mm]
z^-_* & \leq   \cO(\epsilon^2)
\end{alignat*}
for small $\epsilon$. 
Hence Lemma~\ref{lem:Cone} implies that for values of $(\alpha,\omega)$ near $(\pp,\pp)$ any solution has either $\|\tc\|$ of order 1 or $\|\tc\| =  \cO(\epsilon^2)$. 
The asymptotically small term bounding $z_*^-$ is explicitly calculated in Lemma~\ref{lem:ZminusBound}. 
A related consequence is that for $\epsilon=0$ there are no nontrivial solutions 
of $\tilde{F}_0(\alpha,\omega,\tc)=0$ with 
$\| \tc \| < \frac{2 \omega - \alpha}{\alpha} $. 
\end{remark}

\begin{remark}\label{r:rhobound}
In Section~\ref{s:contraction} we will work on subsets of $\ell^K_0$ of the form
\[
  \ell_\rho := \{ c \in \ell^K_0 : \|K^{-1} c\| \leq \rho \} .
\]
Part (b) of Lemma~\ref{lem:Cone} will be used in Section~\ref{s:global} to guarantee that we are not missing any solutions by considering $\ell_\rho$ (for some specific choice of $\rho$) rather than the full space $\ell^K_0$.
In particular, we infer from Remark~\ref{r:smalleps} that  small solutions (meaning roughly that $\|\tc\| \to 0$ as $\epsilon \to 0$)
satisfy $\| K^{-1} \tc \| = \cO(\epsilon^2)$.
\end{remark}

The following theorem guarantees that near the bifurcation point the problem of finding all periodic solutions is equivalent to considering the rescaled problem $F_\epsilon(\alpha,\omega,c)=0$.
\begin{theorem}
\label{thm:FourierEquivalence3}
\textup{(a)} Let $ \epsilon > 0$,  $c \in \ell^K_0$, $\alpha>0$ and $ \omega >0$. 
	Define $y: \R\to \R$ as 
\begin{equation}\label{e:yc}
	y(t) = 
	\epsilon \left( e^{i \omega t }  + e^{- i \omega t }\right) 
	+ \epsilon  \sum_{k = 2}^\infty   c_k e^{i \omega k t }  + c_k^* e^{- i \omega k t } .
\end{equation}
%	and suppose that $ y(t) > -1$. 
	Then $y(t)$ solves~\eqref{eq:Wright} if and only if $F_{\epsilon}( \alpha , \omega , c) = 0$.\\
\textup{(b)}
Let $y(t) \not\equiv 0$ be a periodic solution of~\eqref{eq:Wright} of period $2\pi/\omega$
 with Fourier coefficients $\c$.
Suppose $\alpha < 2\omega$ and $\| \c \| < \frac{2 \omega - \alpha}{\alpha} $.
Then, up to time translation, $y(t)$ is described by a Fourier series of the form~\eqref{e:yc} with $\epsilon > 0$ and $c \in \ell^K_0$.
\end{theorem}

\begin{proof}
Part (a) follows directly from Theorem~\ref{thm:FourierEquivalence2} and the  change of variables~\eqref{e:changeofvariables}.
To prove part (b) we need to exclude the possibility that there is a nontrivial solution with $\epsilon=0$. The asserted bound on the ratio of $\alpha$ and $\omega$ guarantees, by Lemma~\ref{lem:Cone} (see also Remark~\ref{r:smalleps}), that indeed $\epsilon>0$ for any nontrivial solution. 
\end{proof}

We note that in practice (see Section~\ref{s:global}) a bound on $\| \c \|$ is derived from a bound on $y$ or $y'$ using Parseval's identity.

\begin{remark}\label{r:cone}
It follows from Theorem~\ref{thm:FourierEquivalence3} and Remark~\ref{r:smalleps} that for values of $(\alpha,\omega)$ near $(\pp,\pp)$ any reasonably bounded solution satisfies $\| c\| =  O(\epsilon)$ as well as $\|K^{-1} c \| = O(\epsilon)$ asymptotically (as $\epsilon \to 0$).
These bounds will be made explicit (and non-asymptotic) for specific choices of the parameters in Section~\ref{s:global}.
\end{remark}

% We are able to rule out such large amplitude solutions using global estimates such as those in \cite{neumaier2014global}.
% Hence, near the bifurcation point, the problem of describing periodic solutions of~\eqref{eq:Wright} reduces to studying the family of zeros finding problems $F_\epsilon=0$.





%Specifically, if a solution having $ \epsilon = 0$ does in fact correspond to a nontrivial periodic solution and $\alpha  < 2\omega $, then $ \| \tilde{c} \| > 2 \omega \alpha^{-1} -1$. 
%%PERHAPS THIS NEEDS A FORMULATION AS A THEOREM AS WELL?
%%IN OTHER WORDS: ARE WE SURE WE HAVE FOUND ALL ZEROS OF $\tilde{F}_0$, I.E. ALL SOLUTIONS WITH $\epsilon=0$ NEAR THE BIFURCATION POINT? AFTER RESCALING THESE ARE INVISIBLE?
%%THERE SHOULD BE A STATEMENT ABOUT THIS SOMEWHERE! EITHER HERE OR SOME





We finish this section by defining a curve of approximate zeros $\bx_\epsilon$ of $F_\epsilon$ 
(see \cite{chow1977integral,hassard1981theory}). 
%(see \cite{chow1977integral,morris1976perturbative,hassard1981theory}). 


\begin{definition}\label{def:xepsilon}
Let
\begin{alignat*}{1}
	\balpha_\epsilon &:= \pp + \tfrac{\epsilon^2}{5} ( \tfrac{3\pi}{2} -1)  \\
	\bomega_\epsilon &:= \pp -  \tfrac{\epsilon^2}{5} \\
	\bc_\epsilon 	 &:= \left(\tfrac{2 - i}{5}\right) \epsilon \,  \e_2 \,.
\end{alignat*}
We define the approximate solution 
$ \bx_\epsilon := \left( \balpha_\epsilon , \bomega_\epsilon  , \bc_\epsilon \right)$
for all $\epsilon \geq 0$.
\end{definition}

We leave it to the reader to verify that both 
 $F_\epsilon(\pp,\pp,\bc_{\epsilon})=\cO(\epsilon^2)$ and $F_\epsilon(\bx_\epsilon)=\cO(\epsilon^2)$.
%%%	
%%%	
%%%	}{Better like this?}
%%%\annote[J]{ $F_\epsilon(\bx_0)=\cO(\epsilon^2)$ and $F_\epsilon(\bx_\epsilon)=\cO(\epsilon^2)$.}{I think we'd still need the $ \bar{c}_\epsilon$ term in $\bar{x}_0$ to be of order $ \epsilon$.}
%%%\remove[JB]{We show in Proposition A.1
%%%%\ref{prop:ApproximateSolutionWorks} 
%%% that any $ x \in \R^2 \times \ell^1_0$ which is $ \cO(\epsilon^2)$ close to $ \bar{x}_\epsilon $ will yield the estimate $F_\epsilon(x) = \cO(\epsilon^2)$.
%%%Hence choosing $\{ \pp , \pp, \bar{c}_\epsilon\}$ as our approximate solution would also have been a natural choice for performing an $\cO(\epsilon^2)$ analysis and would have simplified several of our calculations.
%%%However,} 
%%%
We choose to use the more accurate approximation 
for the $ \alpha$ and $ \omega $ components to improve our final quantitative results. 














%
% Values for $ (\alpha, \omega,c)$ which approximately solve $\tilde{F}(\alpha,\omega,c) = 0$  are computed in  \cite{chow1977integral,morris1976perturbative,hassard1981theory} and are as follows:
%  \begin{eqnarray}
%  \tilde{\alpha}( \epsilon) &:=& \pi /2 + \tfrac{\epsilon^2}{5} ( \tfrac{3\pi}{2} -1) \nonumber \\
%  \tilde{\omega}( \epsilon) &:=& \pi /2 -  \tfrac{\epsilon^2}{5} \label{eq:ScaleApprox} \\
%  \tc(\epsilon) 	  &:=& \{ \left(\tfrac{2 - i}{5}\right)  \epsilon^2 , 0,0, \dots \} \nonumber
%  \end{eqnarray}
% In Appendix \ref{sec:OperatorNorms} we illustrate an alternative method for deriving this approximation.
%
%
%
%
% We want to solve $ \tilde{F}(\alpha , \omega, \hat{c}) =0$ for small values of $ \epsilon$.
% However $ D \tilde{F}(\alpha , \omega , c)$ is not invertible at $ ( \pp , \pp , 0)$ when $ \epsilon = 0$.
% In order for our asymptotic analysis to be non-degenerate, we need to make the change of variables $ c \mapsto \epsilon c$.
% Under this change of variables, we define the function $ F$ below so that $ \tilde{F}(\alpha , \omega , \epsilon c) =\epsilon  F( \alpha , \omega , c)$.
%
%
%
% \begin{definition}
% Construct an $\epsilon$-parameterized family of densely defined functions  $F : \R^2 \oplus \ell^1 / \C \to \ell^1$ by:
% \begin{equation}
% \label{eq:FDefinition}
% 	F(\alpha,\omega, c) :=
% 	[i \omega + \alpha e^{-i \omega}]_1 +
% 	( i \omega K^{-1} + \alpha U_{\omega}) c +
% 	[\epsilon \alpha e^{-i \omega}]_2  +
% 	\alpha \epsilon L_\omega c +
% 	\alpha \epsilon [ U_{\omega} c] * c.
% \end{equation}
% \end{definition}

%%
%%
%%\begin{corollary}
%%	\label{thm:FourierEquivalence3}
%%	Fix $ \epsilon > 0$, and $ c \in \ell^1 / \C $, and $ \omega >0$. Define $y: \R\to \R$ as 
%%	\[
%%	y(t) = 
%%	\epsilon \left( e^{i \omega t }  + e^{- i \omega t }\right) 
%%	+  \epsilon  \left( \sum_{k = 2}^\infty   c_k e^{i \omega k t }  + \overline{c}_k e^{- i \omega k t } \right) 
%%	\]
%%	and suppose that $ y(t) > -1$. 
%%	Then $y(t)$ solves Wright's equation at parameter $ \alpha > 0 $ if and only if $ F( \alpha , \omega , c) = 0$ at parameter $ \epsilon$. 
%%	
%%	
%%	
%%\end{corollary}
%%
%%
%%\begin{proof}
%%	Since $ \tilde{F}(\alpha,\omega, \epsilon c) = \epsilon F( \alpha , \omega , c)$, the result follows from Theorem \ref{thm:FourierEquivalence2}.
%%\end{proof}

% If we can find $(\alpha , \omega, c)$ for which $ F( \alpha , \omega,c)=0$ at parameter $\epsilon$, then $ \tilde{F}(\alpha ,\omega, c)=0$.
% By Theorem \ref{thm:FourierEquivalence2} this amounts to finding a periodic solution to Wright's equation.
% Lastly, because we have performed the change of variables $ c \mapsto \epsilon c$, we need to  apply this change of variables to our approximate solution as well.
%
% \begin{definition}
% 	Define the approximate solution $ x( \epsilon) = \left\{ \alpha(\epsilon ) , \omega ( \epsilon ) , c(\epsilon) \right\}$ as below,  where $c(\epsilon) = \{ c_2( \epsilon) , 0 ,0 , \dots\} $.
% 	We may also write $ x_\epsilon = x(\epsilon) $.
% 	\begin{eqnarray}
% 	\alpha( \epsilon) &:=& \pi /2 + \tfrac{\epsilon^2}{5} ( \tfrac{3\pi}{2} -1) \nonumber \\
% 	\omega( \epsilon) &:=& \pi /2 -  \tfrac{\epsilon^2}{5} \label{eq:Approx} \\
% 	c_2(\epsilon) 	  &:=& \left(\tfrac{2 - i}{5}\right) \epsilon \nonumber
% 	\end{eqnarray}
%
% \end{definition}


%removed preliminaries
%\medskip
%We provide definitions and background results on
%linear algebra and privacy in Appendix \ref{sec:preliminaries}.

\section{Exact case}

Here, we discuss the case, where all $n$ points lie \emph{exactly} in a subspace $s_*$ of dimension $k$ of $\RR^d$. Our goal
is to privately output that subspace. We do it under the
assumption that all strict subspaces of $s_*$ contain at most $\ell$
points.
If the points are in general position, then $\ell=k-1$, as any strictly smaller subspace has dimension $<k$ and cannot contain more points than its dimension.
Let $\mathcal{S}_d^k$ be the set of all $k$-dimensional
subspaces of $\mathbb{R}^d$. Let $\mathcal{S}_d$ be the
set of all subspaces of $\mathbb{R}^d$. We formally define
that problem as follows.

\begin{problem}\label{prob:exact}
    Assume (i) all but at most $\ell$, input points are in some
    $s_* \in \mathcal{S}_d^k$, and (ii)  every subspace
    of dimension $<k$ contains at most $\ell$ points. (If the points
    are in general position -- aside from being contained in
    $s_*$ -- then $\ell=k-1$.) The goal is to output a representation
    of $s_*$.
\end{problem}

We call these $\leq \ell$ points that do not lie in
$s_*$, ``adversarial points''.
With the problem defined in Problem~\ref{prob:exact}, we
will state the main theorem of this section.

\begin{theorem}\label{thm:exact}
    For any $\eps,\delta>0$, $\ell \ge k-1 \ge 0$, and
    $$n \geq O\left(\ell + \frac{\log(1/\delta)}{\eps}\right),$$ there
    exists an $(\eps,\delta)$-DP algorithm
    $M : \mathbb{R}^{d \times n} \to \mathcal{S}_d^k$, such that if
    $X$ is a dataset of $n$ points satisfying the conditions
    in Problem~\ref{prob:exact},
    then $M(X)$ outputs a representation of $s_*$ with probability $1$.
\end{theorem}

We prove Theorem \ref{thm:exact} by proving the privacy and
the accuracy guarantees of Algorithm~\ref{alg:exact}.
The algorithm performs a $\GAPMAX$ (cf.~Lemma~\ref{lem:gap-max}).
It assigns a score to all the relevant
subspaces, that is, the subspaces spanned by the points
of the dataset $X$. We show that the only subspace that
has a high score is the true subspace $s_*$, and the rest
of the subspaces have low scores. Then $\GAPMAX$ outputs
the true subspace successfully because of the gap between
the scores of the best subspace and the second to the best
one. For $\GAPMAX$ to work all the time, we define a default
option in the output space that has a high score, which we
call $\NULL$. Thus, the output space is now
$\cY = \cS_d \cup \{\NULL\}$. Also, for $\GAPMAX$ to run in
finite time, we filter $\cS_d$ to select finite number of subspaces
that have at least $0$ scores on the basis of $X$. Note that
this is a preprocessing step, and does not violate privacy as,
we will show, all other subspaces already have $0$ probability
of getting output.
We define the score function
$u : \mathcal{X}^n \times \cY \to \mathbb{N}$
as follows.
\[ u(x,s) :=
\begin{cases}
    |x \cap s| - \sup \{ |x \cap t| : t \in \mathcal{S}_d, t \subsetneq s \} &
        \text{if $s \in \cS_d$}\\
    \ell + \frac{4\log(1/\delta)}{\eps} + 1 & \text{if $s = \NULL$}
\end{cases}
\]
Note that this score function can be computed in finite
time because for any $m$ points and $i>0$, if the points
are contained in an $i$-dimensional subspace, then the
subspace that contains all $m$ points must lie within
the set of subspaces spanned by ${m \choose i+1}$ subsets
of points.

\begin{algorithm}[h!] \label{alg:exact}
\caption{DP Exact Subspace Estimator
    $\DPESE_{\eps, \delta, k, \ell}(X)$}
\KwIn{Samples $X \in \R^{d \times n}$.
    Parameters $\eps, \delta, k, \ell > 0$.}
\KwOut{$\hat{s} \in \cS_d^k$.}
\vspace{5pt}

Set $\cY \gets \{\NULL\}$ and sample noise $\xi(\NULL)$ from $\TLap(2,\eps,\delta)$.\\
Set score $u(X,\NULL) = \ell + \frac{4\log(1/\delta)}{\eps} + 1$.
\vspace{5pt}

\tcp{Identify candidate outputs.}
\For{each subset $S$ of $X$ of size $k$}{
    Let $s$ be the subspace spanned by $S$.\\
    $\cY \gets \cY \cup \{s\}$.\\
    Sample noise $\xi(s)$ from $\TLap(2,\eps,\delta)$.\\
    Set score $u(X,s) = |x \cap s| - \sup \{ |x \cap t| : t \in \mathcal{S}_d, t \subsetneq s \} $.
}
\vspace{5pt}

\tcp{Apply $\GAPMAX$.}
Let $s_1 = \argmax_{s \in \cY} u(X,s)$ be the candidate with the largest score.\\
Let $s_2 = \argmax_{s \in \cY \setminus \{s_1\}} u(X,s)$ be the candidate with the second-largest score.\\
Let $\hat s = \argmax_{s \in \cY} \max\{ 0 , u(X,s) - u(X,s_2) -1\} + \xi(s)$.\\
\tcp{Truncated Laplace noise $\xi \sim \TLap(2,\eps,\delta)$; see Lemma \ref{lem:truncated-laplace}}

\vspace{5pt}
\Return $\hat{s}.$
\vspace{5pt}
\end{algorithm}

We split the proof of Theorem~\ref{thm:intro-main-exact} into sections
for privacy (Lemma~\ref{lem:exact-privacy}) and accuracy (Lemma~\ref{lem:exact-accuracy}).

\subsection{Privacy}

\begin{lemma}\label{lem:exact-privacy}
    Algorithm~\ref{alg:exact} is $(\eps,\delta)$-differentially
    private.
\end{lemma}
The proof of Lemma \ref{lem:exact-privacy} closely follows the
privacy analysis of $\GAPMAX$ by \cite{BunDRS18}. The only novelty
is that Algorithm \ref{alg:exact} may output $\NULL$ in the case
that the input is malformed (i.e., doesn't satisfy the assumptions
of Problem~\ref{prob:exact}).

The key is that the score $u(X,s)$ is low sensitivity. Thus
$\max\{ 0 , u(X,s) - u(X,s_2) -1\}$ also has low sensitivity.
What we gain from subtracting the second-largest score and
taking this maximum is that these values are also sparse -- only
one ($s=s_1$) is nonzero. This means we can add noise to all
the values without paying for composition. We now prove
Lemma~\ref{lem:exact-privacy}.

\begin{proof}
    First, we argue that the sensitivity of $u$ is
    $1$. %Consider any $s \in \cS_d$, and neighbouring datasets $X,X'$.
    The quantity $\abs{X \cap s}$ has sensitivity $1$ and so does
    $\sup \{ |X \cap t| : t \in \mathcal{S}_d, t \subsetneq s \}$.
    This implies sensitivity $2$ by the triangle inequality.
    However, we see that it is not possible to change one point
    that simultaneously increases $\abs{X \cap s}$ and decreases
    $\sup \{ |X \cap t| : t \in \mathcal{S}_d, t \subsetneq s \}$
    or vice versa. Thus the sensitivity is actually $1$.

    We also argue that $u(X,s_2)$ has sensitivity $1$, where $s_2$
    is the candidate with the second-largest score.
    Observe that the second-largest score is a monotone function of
    the collection of all scores -- i.e., increasing scores cannot
    decrease the second-largest score and vice versa.
    Changing one input point can at most increase all the scores by
    $1$, which would only increase the second-largest score by $1$.

    This implies that $\max\{ 0, u(X,s) - u(X,s_2) -1 \}$ has sensitivity
    $2$ by the triangle inequality and the fact that the maximum does
    not increase the sensitivity.

    Now we observe that for any input $X$ there is at most one $s$ such
    that $\max\{ 0, u(X,s) - u(X,s_2) -1 \} \ne 0$, namely $s=s_1$.
    We can say something even stronger: Let $X$ and $X'$ be neighbouring
    datasets with $s_1$ and $s_2$ the largest and second-largest scores
    on $X$ and $s_1'$ and $s_2'$ the largest and second-largest scores
    on $X'$. Then there is at most one $s$ such that
    $\max\{ 0, u(X,s) - u(X,s_2) -1 \} \ne 0$ or $\max\{ 0, u(X',s) - u(X',s_2') -1 \} \ne 0$.
    In other words, we cannot have both $u(X,s_1) - u(X,s_2) >1$ and
    $u(X',s_1') - u(X',s_2') >1$ unless $s_1=s_1'$. This holds because
    $u(X,s) - u(X,s_2)$ has sensitivity $2$.

    With these observations in hand, we can delve into the privacy
    analysis. Let $X$ and $X'$ be neighbouring datasets with $s_1$
    and $s_2$ the largest and second-largest scores on $X$ and $s_1'$
    and $s_2'$ the largest and second-largest scores on $X'$. Let $\cY$
    be the set of candidates from $X$ and let $\cY'$ be the set of
    candidates from $X'$. Let $\check \cY = \cY \cup \cY'$ and
    $\hat \cY = \cY \cap \cY'$.

    We note that, for $s \in \check \cY$, if $u(X,s) \le \ell$, then
    there is no way that $\hat s = s$. This is because
    $|\xi(s)|\le \frac{2 \log(1/\delta)}{\varepsilon}$ for all $s$ and
    hence, there is no way we could have
    $\argmax_{s \in \cY} \max\{ 0 , u(X,s) - u(X,s_2) -1\} +
    \xi(s) \ge \argmax_{s \in \cY} \max\{ 0 , u(X,\NULL) - u(X,s_2) -1\} + \xi(\NULL)$.

    If $s \in \check \cY \setminus \hat \cY$, then
    $u(X,s) \le |X \cap s| \le k+1 \le \ell$ and $u(X',s) \le \ell$.
    This is because $s \notin \hat \cY$ implies $|X \cap s| < k$ or
    $|X' \cap s| < k$, but $|X \cap s| \le |X' \cap s| +1$. Thus,
    there is no way these points are output and, hence, we can ignore
    these points in the privacy analysis. (This is the reason for adding
    the $\NULL$ candidate.)

    Now we argue that the entire collection of noisy values
    $\max\{ 0 , u(X,s) - u(X,s_2) -1\} + \xi(s)$ for $s \in \hat \cY$
    is differentially private.
    This is because we are adding noise to a vector where (i) on the
    neighbouring datasets only $1$ coordinate is potentially different
    and (ii) this coordinate has sensitivity $2$.
\end{proof}

\subsection{Accuracy}

We start by showing that the true subspace $s_*$ has a
high score, while the rest of the subspaces have low scores.

\begin{lemma}\label{lem:scores}
    Under the assumptions of Problem~\ref{prob:exact}, $u(x,s_*) \ge n - 2\ell$
    and $u(x,s')\le 2\ell$ for $s' \ne s_*$.
\end{lemma}
\begin{proof}
    We have $u(x,s_*) = |x \cap s_*| - |x \cap s'|$ for some
    $s' \in \mathcal{S}_d$ with $s' \subsetneq s_*$. The
    dimension of $s'$ is at most $k-1$ and, by the assumption
    (ii), $|x \cap s'| \le \ell$. 
    
    Let $s' \in \mathcal{S}_d \setminus \{s_*\}$.
    There are three cases to analyse:
    \begin{enumerate}
        \item Let $s' \supsetneq s_*$. Then
            $u(x,s') \le |x \cap s'| - |x \cap s_*| \leq \ell$
            because the $\leq \ell$ adverserial points and the $\geq n-\ell$
            non-adversarial points may not together lie in a subspace
            of dimension $k$.

        \item Let $s' \subsetneq s_*$. Let
            $k'$ be the dimension of $s'$. Clearly $k'<k$.
            By our assumption (ii), $|s' \cap x| \le \ell$.
            Then $u(x,s') = |x \cap s'| - |x \cap t| \le \ell$
            for some $t$ because the $\leq \ell$ adversarial points
            already don't lie in $s_*$, so they will not lie in
            any subspace of $s_*$.

        \item Let $s'$ be incomparable to $s_*$.
            Let $s'' = s' \cap s_*$. Then
            $u(x,s') \le |x \cap s'| - |x \cap s''| \leq \ell$
            because the adversarial points may not lie in $s_*$,
            but could be in $s'\setminus s''$.
    \end{enumerate}
    This completes the proof.
\end{proof}

Now, we show that the algorithm is accurate.

\begin{lemma}\label{lem:exact-accuracy}
    If
    $n \geq 3\ell + \frac{8\log(1/\delta)}{\eps} + 2,$
    then Algorithm~\ref{alg:exact} outputs $s_*$ for Problem~\ref{prob:exact}.
\end{lemma}
\begin{proof}
    From Lemma~\ref{lem:scores}, we know that $s_*$ has
    a score of at least $n-2\ell$, and the next best subspace
    can have a score of at most $\ell$. Also, the score of
    $\NULL$ is defined to be $\ell + \tfrac{4\log(1/\delta)}{\eps} + 1$.
    This means that the gap satisfies $\max\{ 0 , u(X,s_*) - u(X,s_2) -1\} \ge n - 3\ell - \tfrac{4\log(1/\delta)}{\eps}  -1$.
    Since the noise is bounded by $\tfrac{2\log(1/\delta)}{\eps}$, our bound on $n$ implies that $\hat s = s_*$
\end{proof}

%\subsection{Putting It All Together}
%
%\begin{proof}[Proof of Theorem~\ref{thm:exact}]
%    By Lemmata~\ref{lem:exact-privacy} and \ref{lem:exact-accuracy},
%    the algorithm is differentially private and accurate.
%\end{proof}

\begin{comment}
    
    Key questions: Remove/weaken assumption (ii) general
    position. Approximate case (or discretization).
    
    \subsection{Generalizing beyond general position}
    
    Still assuming $x \subset s_*$ for some $s_* \in \mathcal{S}_d^k$. Also assume $0 \notin x$.
    
    Claim: There exists some $s \in \mathcal{S}_d$ with $u(x,s) \ge n/k$.
    \begin{proof}
    Let $s_k=s_*$. For $i=k,k-1,k-2,\cdots,1$, inductively choose $s_{i-1} \in \mathcal{S}_d$ such that $s_{i-1} \subsetneq s_i$ and $|x \cap s_{i-1}|$ is maximal. By construction, $u(x,s_i) = |x \cap s_i| - |x \cap s_{i-1}|$ for all $i \in [k]$. By assumption, $|x \cap s_k| = n$. Also, $\mathsf{dimension}(s_i) \le i$. Thus $|x \cap s_0| = 0$.
    Since $\sum_{i=1}^k u(x,s_i) = |x \cap s_k| - |x \cap s_0| = n$, we must have $u(x,s_i) \ge n/k$ for some $i \in [k]$.
    \end{proof}

\end{comment}

\subsection{Lower Bound}

Here, we show that our upper bound is optimal up to constants for
the exact case.

\begin{theorem}
     Any $(\eps,\delta)$-DP algorithm that takes a dataset of $n$ points satisfying the conditions
    in Problem~\ref{prob:exact} and outputs $s_*$ with probability $>0.5$ requires
    $n \geq \Omega\left(\ell + \frac{\log(1/\delta)}{\eps}\right).$
\end{theorem}
\begin{proof}
    First, $n \ge \ell + k$. This is because we need at least
    $k$ points to span the subspace, and $\ell$ points could be corrupted.
    Second, $n \ge \Omega(\log(1/\delta)/\varepsilon)$ by group
    privacy. Otherwise, the algorithm is $(10,0.1)$-differentially
    private with respect to changing the \emph{entire} dataset and
    it is clearly impossible to output the subspace under this condition.
\end{proof}


\section{Approximate Case}

In this section, we discuss the case, where the data
``approximately'' lies in a $k$-dimensional subspace of
$\R^d$. %An alternate perspective of this problem is that
%the data lies in a $k$-dimensional subspace of $\R^d$,
%but has very small noise in the orthogonal directions.
We make a Gaussian distributional assumption, where the
covariance is approximately $k$-dimensional, though the
results could be extended to distributions with heavier
tails using the right inequalities. We formally define
the problem:

\begin{problem}\label{prob:gaussians}
    Let $\Sigma \in \R^{d \times d}$ be a symmetric, PSD
    matrix of rank $\geq k \in \{1,\dots,d\}$, and let $0 < \gamma \ll 1$,
    such that $\tfrac{\lambda_{k+1}}{\lambda_k} \leq \gamma^2$.
    Suppose $\Pi$ is the projection matrix corresponding
    to the subspace spanned by the eigenvectors of $\Sigma$
    corresponding to the eigenvalues $\lambda_1,\dots,\lambda_k$.
    Given sample access to $\cN(\vec{0},\Sigma)$,
    and $0 < \alpha < 1$, output a projection matrix $\wh{\Pi}$,
    such that $\|\Pi-\wh{\Pi}\| \leq \alpha$.
\end{problem}

We solve Problem~\ref{prob:gaussians} under the constraint
of $(\eps,\delta)$-differential privacy. Throughout this section,
we would refer to the subspace spanned by the top $k$ eigenvectors
of $\Sigma$ as the ``true'' or ``actual'' subspace.

Algorithm \ref{alg:approximate} solves Problem~\ref{prob:gaussians} and proves Theorem \ref{thm:intro-main-approx}.
Here $\|\cdot\|$ is the operator norm.

\begin{remark}\label{rem:gamma}
    We scale the eigenvalues of $\Sigma$
    so that $\lambda_k=1$ and $\lambda_{k+1} \leq \gamma^2$.
    %We will be adopting this notation throughout this text.
    Also, for the purpose of the analysis, we will be splitting
    $\Sigma = \Sigma_k + \Sigma_{d-k}$, where $\Sigma_k$ is the
    covariance matrix formed by the top $k$ eigenvalues and
    the corresponding eigenvectors of $\Sigma$ and $\Sigma_{d-k}$
    is remainder.
\end{remark}

%\begin{comment}

Also, we assume the
knowledge of $\gamma$ (or an upper bound on $\gamma$). Our solution
is presented in Algorithm~\ref{alg:approximate}. The following
theorem is the main result of the section.

\begin{theorem}\label{thm:approximate}
    Let $\Sigma \in \R^{d \times d}$ be an arbitrary, symmetric, PSD
    matrix of rank $\geq k \in \{1,\dots,d\}$, and let $0 < \gamma < 1$.
    Suppose $\Pi$ is the projection matrix corresponding
    to the subspace spanned by the vectors of $\Sigma_k$.
    Then given
    $$\gamma^2 \in
        O\left(\frac{\eps\alpha^2n}{d^{2}k\ln(1/\delta)}\cdot
        \min\left\{\frac{1}{k},
        \frac{1}{\ln(k\ln(1/\delta)/\eps)}
        \right\}\right),$$
    such that $\lambda_{k+1}(\Sigma) \leq \gamma^2\lambda_k(\Sigma)$,
    for every $\eps,\delta>0$, and $0 < \alpha < 1$,
    there exists and $(\eps,\delta)$-DP algorithm that takes
    $$n \geq O\left(\frac{k\log(1/\delta)}{\eps} +
        \frac{\log(1/\delta)\log(\log(1/\delta)/\eps)}{\eps}\right)$$
    samples from $\cN(\vec{0},\Sigma)$, and outputs a projection matrix $\wh{\Pi}$,
    such that $\|\Pi-\wh{\Pi}\| \leq \alpha$ with probability
    at least $0.7$.
    \tnote{Eigenvalue gap assumption missing. Also order of quantifiers is ambiguous -- ``for every $\Sigma$ there exists an algorith.''}
\end{theorem}
%\end{comment}

Algorithm~\ref{alg:approximate} is a type of
``Subsample-and-Aggregate'' algorithm \cite{NissimRS07}.
Here, we consider multiple subspaces formed by the points
from the same Gaussian, and privately find a subspace that
is close to all those subspaces. Since the subspaces formed
by the points would be close to the true subspace, the privately
found subspace would be close to the true subspace.

A little more formally, we first sample $q$ public data points
(called ``reference points'') from $\cN(\vec{0},\id)$. Next,
we divide the original dataset $X$ into disjoint datasets of $m$ samples
each, and project all reference points on the subspaces spanned
by every subset. Now, for every reference point, we do the
following. We have $t=\tfrac{n}{m}$ projections of the reference
point. Using DP histogram over $\R^d$, we aggregate those
projections in the histogram cells; with high probability
all those projections will be close to one another, so they
would lie within one histogram cell. We output a random point
from the histogram cell corresponding to the reference point.
With a total of $q$ points output in this way, we finally
output the projection matrix spanned by these points. In
the algorithm $C_0$, $C_1$, and $C_2$ are universal constants.

We divide the proof of Theorem~\ref{thm:approximate}
into two parts: privacy (Lemma \ref{coro:privacy}) and
accuracy (Lemma~\ref{lem:final-projection}).

\begin{algorithm}[h!] 
\caption{\label{alg:approximate}DP Approximate Subspace Estimator
    $\DPASE_{\eps, \delta, \alpha, \gamma, k}(X)$}
\KwIn{Samples $X_1,\dots,X_{n} \in \R^d$.
    Parameters $\eps, \delta, \alpha, \gamma, k > 0$.}
\KwOut{Projection matrix $\wh{\Pi} \in \R^{d \times d}$ of rank $k$.}
\vspace{5pt}

Set parameters:
    $t \gets \tfrac{C_0\ln(1/\delta)}{\eps}$ \qquad
    $m \gets \lfloor n/t \rfloor$ \qquad $q \gets C_1 k$
    \qquad $\ell \gets \tfrac{C_2\gamma\sqrt{dk}(\sqrt{k}+\sqrt{\ln(kt)})}{\sqrt{m}}$
\vspace{5pt}

Sample reference points $p_1,\dots,p_q$ from $\cN(\vec{0},\id)$ independently.
\vspace{5pt}

\tcp{Subsample from $X$, and form projection matrices.}
\For{$j \in 1,\dots,t$}{
    Let $X^j = (X_{(j-1)m+1},\dots,X_{jm}) \in \mathbb{R}^{d \times m}$.\\
    Let $\Pi_j \in \mathbb{R}^{d \times d}$ be the projection matrix onto the subspace spanned by the eigenvectors of $X^j (X^j)^T \in \mathbb{R}^{d \times d}$ corresponding to the largest $k$ eigenvalues.\\
    \For{$i \in 1,\dots,q$}{
        $p_{i}^j \gets \Pi_j p_i$
    }
}
\vspace{5pt}

\tcp{Create histogram cells with random offset.}
Let $\lambda$ be a random number in $[0,1)$.\\
Divide $\R^{qd}$ into $\Omega =
    \{\dots,[\lambda\ell+i\ell,\lambda\ell+(i+1)\ell),\dots\}^{qd}$,
    for all $i \in \Z$.\\
Let each disjoint cell of length $\ell$ be a histogram bucket.
\vspace{5pt}

\tcp{Perform private aggregation of subspaces.}
For each $i \in [q]$, let $Q_i \in \RR^{d \times t}$ be the
    dataset, where column $j$ is $p_i^j$.\\
Let $Q \in \RR^{qd \times t}$ be the vertical concatenation
    of all $Q_i$'s in order.\\
Run $(\eps,\delta)$-DP histogram over $\Omega$ using $Q$
    to get $\omega \in \Omega$ that contains at least $\tfrac{t}{2}$ points.\\
\If{no such $\omega$ exists}{
    \Return $\bot$
}
\vspace{5pt}

\tcp{Return the subspace.}
Let $\wh{p}=(\wh{p}_1,\dots,\wh{p}_d,\dots,\wh{p}_{(q-1)d+1},\dots,\wh{p}_{qd})$
    be a random point in $\omega$.\\
\For{each $i \in [q]$}{
    Let $\wh{p}_i = (\wh{p}_{(i-1)d+1},\dots,\wh{p}_{id})$.
}
    Let $\wh{\Pi}$ be the projection matrix of the top-$k$ subspace of $(\wh{p}_1,\dots,\wh{p}_q)$.\\
\Return $\wh{\Pi}.$
\vspace{5pt}
\end{algorithm}

\subsection{Privacy}

We analyse the privacy by understanding the sensitivities
at the only sequence of steps invoking a differentially
private mechanism, that is, the sequence of steps involving
DP-histograms.

\begin{lemma}\label{lem:histogram-sensitivity}\label{coro:privacy}
    Algorithm~\ref{alg:approximate} is $(\eps,\delta)$-differentially
    private.
\end{lemma}
\begin{proof}
    Changing one point in $X$ can change only
    one of the $X^j$'s. This can
    only change one point in $Q$, which in turn can only
    change the counts in two histogram cells by $1$.
    Therefore, the sensitivity is $2$. % Since the choice
    %of $i$ was arbitrary, this is true for all $i$.
    %For a reference
    %point $p_i$, changing a point in $X^{j^*}$ can either move
    %its projection on to the subspace spanned by $X^{j^*}$ to a
    %different histogram cell, or keep it in the same cell.
%    Privacy now follows from the guarantees of DP-histogram (Lemma~\ref{lem:priv-hist}).
    Because the sensitivity of the histogram step is bounded
    by $2$ (Lemma~\ref{lem:histogram-sensitivity}), an application
    of DP-histogram, by Lemma~\ref{lem:priv-hist}, is $(\eps,\delta)$-DP.
    Outputting a random
    point in the privately found histogram cell preserves privacy
    by post-processing (Lemma~\ref{lem:post-processing}).
    Hence, the claim.
\end{proof}

\subsection{Accuracy}

\begin{comment}

We begin by showing a technical result that says that
any two matrices, whose difference is bounded in operator
norm, and which have significant eigenvalue gap, span
similar subspaces.

\begin{lemma}\label{lem:projections-close}
    Let $A, \tilde A \in \mathbb{R}^{d \times d}$ be symmetric
    matrices. Suppose $\|A-\tilde A\| \le \varepsilon$.
    Let $\lambda_1 \ge \lambda_2 \ge \cdots \ge \lambda_d$
    be the eigenvalues of $A$. Suppose $\lambda_k-\lambda_{k+1} > \varepsilon$.
    Let $\Pi_k, \tilde \Pi_k \in \mathbb{R}^{d \times d}$ be
    the projections onto the eigenspaces corresponding to
    the largest $k$ eigenvalues of $A$ and $\tilde A$ respectively.
    Then
    \[\left\| \Pi_k - \tilde \Pi_k \right\| \le
        \frac{\varepsilon}{\lambda_k-\lambda_{k+1}-\varepsilon}.\]
\end{lemma}
\begin{proof}
    Let $A,\tilde{A},U,\tilde{U},\Lambda,\tilde\Lambda
    \in \mathbb{R}^{d \times d}$ satisfy the following.
    (i) $A=U\Lambda U^T$,
    $\tilde{A} = \tilde{U} \tilde{\Lambda} \tilde{U}^T$,
    (ii) $U^TU=I_d=\tilde{U}^T\tilde{U}$, and (iii) $\Lambda$
    and $\tilde\Lambda$ are diagonal matrices with entries
    in descending order. Let $\lambda_i, \tilde{\lambda}_i$
    denote the $i^\text{th}$ diagonal entry of $\Lambda, \tilde\Lambda$.
    Denote $U_{a:b}, \tilde{U}_{a:b} \in \mathbb{R}^{d \times (b-a+1)}$
    to be the matrix formed by columns $a, a+1, \cdots, b$ of $U$
    and $\tilde{U}$ respectively (i.e., corresponding to
    $\lambda_a \ge \lambda_{a+1} \ge \cdots \ge \lambda_b$
    and $\tilde\lambda_a \ge \tilde\lambda_{a+1} \ge \cdots \ge \tilde\lambda_b$).

    Lemma~\ref{lem:weyl} tells us
    $|\tilde\lambda_k - \lambda_k| \le \|\tilde{A}-A\|$
    for all $k \in [d]$.
    Note that the operator norm is submultiplicative -- i.e.,
    $\|M M'\| \le \|M\| \cdot \|M'\|$.
    It immediately follows from Lemma~\ref{lem:davis-kahan}
    that, for all $k \in [d-1]$,
    we have
    \begin{align*}
        \|\tilde{U}_{1:k}^T {U}_{k+1:d}\| &\leq
                \frac{\|\tilde{A}-A\|}{\lambda_k-\tilde{\lambda}_{k+1}}\\
            &=\frac{\|\tilde{A}-A\|}{\lambda_k-\lambda_{k+1}-\lambda_{k+1}-\tilde{\lambda}_{k+1}}\\
            &\leq\frac{\|\tilde{A}-A\|}{\lambda_k - \lambda_{k+1} - \|\tilde{A}-A\|}.
    \end{align*}
    Now,
    \begin{align*}
        \tilde{\lambda}_k - \lambda_{k+1} &=
                \tilde{\lambda}_k - \lambda_k + \lambda_k - \lambda_{k+1}\\
            &> -\abs{\tilde{\lambda}_k - \lambda_k} + \lambda_k - \lambda_{k+1}\\
            &> -\eps + \eps\\
            &= 0.
    \end{align*}
    Therefore, we can apply Lemma~\ref{lem:davis-kahan}
    again
    to get $\|{U}_{1:k}^T \tilde{U}_{k+1:d}\| \le
    \frac{\|\tilde{A}-A\|}{\lambda_k - \lambda_{k+1} - \|\tilde{A}-A\|}$,
    and $\| \tilde{U}_{k+1:d}^T {U}_{1:k}\| \le
    \frac{\|\tilde{A}-A\|}{\lambda_k - \lambda_{k+1} - \|\tilde{A}-A\|}$
    follows because all eigenvalues of $U$ and $\tilde{U}$
    are non-negative.

    The ultimate quantity of interest for us is the
    difference between the projections $U_{1:k} U_{1:k}^T$
    and $\tilde{U}_{1:k} \tilde{U}_{1:k}^T$. We can
    apply the Lemma~\ref{lem:davis-kahan} to bound this:
    Note that $\tilde U_{1:k} \tilde U_{1:k}^T +
    \tilde U_{k+1:d} \tilde U_{k+1:d}^T = \tilde U \tilde U^T = I_d$.
    For any $A \in \mathbb{R}^{n \times m}$, $\|A^TA-I_m\|=\|AA^T-I_n\|$,
    as these are symmetric matrices with the same nonzero
    eigenvalues. Thus,
    \begin{align*}
        \left\| U_{1:k} U_{1:k}^T - \tilde{U}_{1:k} \tilde{U}_{1:k}^T \right\|
            &= \left\| U_{1:k} U_{1:k}^T + \tilde{U}_{k+1:d}
                \tilde{U}_{k+1:d}^T - I_d \right\| \\
            &= \left\|\left( U_{1:k} , \tilde{U}_{k+1:d} \right)
                \left( \begin{array}{c} U_{1:k}^T \\
                \tilde{U}_{k+1:d}^T \end{array} \right) - I_d \right\|\\
            &= \left\|\left( U_{1:k} , \tilde{U}_{k+1:d} \right)
                \left( U_{1:k} , \tilde{U}_{k+1:d} \right)^T - I_d \right\|\\
            &= \left\|\left( U_{1:k} , \tilde{U}_{k+1:d} \right)^T
                \left( U_{1:k} , \tilde{U}_{k+1:d} \right) - I_d \right\|\\
            &= \left\| \left( \begin{array}{cc} U_{1:k}^T U_{1:k}
                & U_{1:k}^T \tilde{U}_{k+1:d} \\ \tilde{U}_{k+1:d}^T U_{1:k}
                & \tilde{U}_{k+1:d}^T \tilde{U}_{k+1:d} \end{array} \right)
                - \left( \begin{array}{cc} I_k & 0_{k \times d-k} \\ 0_{d-k \times k}
                & I_{d-k} \end{array} \right)\right\|\\
            &= \left\| \left( \begin{array}{cc} 0_{k \times k}
                & U_{1:k}^T \tilde{U}_{k+1:d} \\ \tilde{U}_{k+1:d}^T U_{1:k}
                & 0_{d-k \times d-k} \end{array} \right) \right\|\\
            &= \max\left\{ \left\| U_{1:k}^T \tilde{U}_{k+1:d} \right\|,
                \left\| \tilde{U}_{k+1:d}^T U_{1:k} \right\| \right\}\\
            &\le \frac{\|\tilde{A}-A\|}{\lambda_k - \lambda_{k+1} - \|\tilde{A}-A\|}.
    \end{align*}
    This concludes the proof.
\end{proof}

Now we delve into the utility analysis of the algorithm.
Note that any matrix can be represented by its singular
value decomposition (SVD), that is, any matrix $A=UDV^T$,
where $D$ is a diagonal matrix containing its singular
values in decreasing order, $U$ is the matrix with left
singular vectors, and $V$ is the matrix with right singular
vectors, and $XX^T$ is a symmetric matrix $UDD^TU^T$,
where $U$ is the matrix containing the eigenvectors of
$XX^T$. Hence, to work with the projection matrix of
the subspace spanned by the columns of $X$, we can directly
work with the subspace spanned by the columns vectors
of $XX^T$ because they are equivalent. For $1 \leq j \leq t$,
let $X^j$ be the subsets of $X$ as defined in
Algorithm~\ref{alg:approximate}, and $\Pi_j$ be the
projection matrices of their respective subspaces. We
now show that $\Pi_j$ and the projection matrix of the
subspace spanned by $\Sigma_k$ are close in operator norm.

\end{comment}

Now we delve into the utility analysis of the algorithm.
For $1 \leq j \leq t$,
let $X^j$ be the subsets of $X$ as defined in
Algorithm~\ref{alg:approximate}, and $\Pi_j$ be the
projection matrices of their respective subspaces. We
now show that $\Pi_j$ and the projection matrix of the
subspace spanned by $\Sigma_k$ are close in operator norm.

\begin{lemma}\label{lem:empirical-subspaces-close}
    Let $\Pi$ be the projection matrix of the subspace
    spanned by the vectors of $\Sigma_k$, and for each
    $1 \leq j \leq t$, let $\Pi_j$ be the projection
    matrix as defined in Algorithm~\ref{alg:approximate}.
    If $m \geq O(k + \ln(qt))$, then
    $$\pr{}{\forall j, \|\Pi-\Pi_j\| \leq
        O\left(\frac{\gamma\sqrt{d}}{\sqrt{m}}\right)} \geq 0.95$$
\end{lemma}
\begin{proof}
    We show that the subspaces spanned by $X^j$ and
    the true subspace spanned by $\Sigma$ are close.
    Formally, we invoke
    Lemmata \ref{lem:sin-theta} and \ref{lem:sin-theta-property}.
    This closeness follows from standard matrix concentration
    inequalities.
    %which we discuss in Appendix \ref{sec:preliminaries}.
    
    Fix a $j \in [t]$. Note that $X^j$ can be written
    as $Y^j + H$, where $Y^j$ is the matrix of vectors
    distributed as $\cN(\vec{0},\Sigma_k)$, and $H$ is
    a matrix of vectors distributed as $\cN(\vec{0},\Sigma_{d-k})$,
    where $\Sigma_k$ and $\Sigma_{d-k}$ are defined as
    in Remark~\ref{rem:gamma}.
    By Corollary~\ref{coro:normal-spectrum}, with probability at least $1-\tfrac{0.02}{t}$,
    $s_k(Y^j) \in \Theta((\sqrt{m}+\sqrt{k})(\sqrt{s_k(\Sigma_k)})) = \Theta(\sqrt{m}+\sqrt{k})> 0$.
    Therefore, the subspace spanned by
    $Y^j$ is the same as the subspace spanned by $\Sigma_k$.
    So, it suffices to look at the subspace spanned
    by $Y^j$.

    Now, by Corollary~\ref{coro:normal-spectrum}, we know
    that with probability at least $1-\tfrac{0.02}{t}$,
    $\|X^j-Y^j\| = \|H\| \leq O((\sqrt{m}+{\sqrt{d}})\sqrt{s_1(\Sigma_{d-k})})
    \leq O(\gamma(\sqrt{m}+\sqrt{d})\sqrt{s_k(\Sigma_k)}) \leq O(\gamma(\sqrt{m}+\sqrt{d}))$.
    
    We wish to invoke Lemma~\ref{lem:sin-theta}. Let $UDV^T$
    be the SVD of $Y^j$, and let $\hat{U}\hat{D}\hat{V}^T$ be
    the SVD of $X^j$. Now, for a matrix $M$, let $\Pi_M$ denote
    the projection matrix of the subspace spanned by the columns
    of $M$. Define quantities $a,b,z_{12},z_{21}$ as follows.
    \begin{align*}
        a &= s_{\min}(U^TX^jV)\\
            &= s_{\min}(U^TY^jV + U^THV)\\
            &= s_{\min}(U^TY^jV) \tag{Columns of $U$ are orthogonal to columns of $H$}\\
            &= s_k(Y^j)\\
            &\in \Theta(\sqrt{m}+\sqrt{k})\\
            &\in \Theta(\sqrt{m})\\
        b &= \|U_{\bot}^TX^jV_{\bot}\|\\
            &= \|U_{\bot}^TY^jV_{\bot} + U_{\bot}^THV_{\bot}\|\\
            &= \|U_{\bot}^THV_{\bot}\|
                \tag{Columns of $U_{\bot}$ are orthogonal to columns of $Y^j$}\\
            &\leq \|H\|\\
            &\leq O(\gamma(\sqrt{m}+\sqrt{d}))\\
        z_{12} &= \|\Pi_U H \Pi_{V_{\bot}}\|\\
            &= 0\\
        z_{21} &= \|\Pi_{U_{\bot}}H\Pi_V\|\\
            &= \|\Pi_{U_{\bot}}\Sigma_{d-k}^{1/2}(\Sigma_{d-k}^{-1/2}H)\Pi_V\|
    \end{align*}
    Now, in the above, $\Sigma_{d-k}^{-1/2}H \in \RR^{d\times m}$,
    such that each of its entry is an independent sample from $\cN(0,1)$.
    Right-multiplying it by $\Pi_V$ makes it a matrix
    in a $k$-dimensional subspace of $\RR^m$, such that
    each row is an independent vector from a spherical
    Gaussian. Using Corollary~\ref{coro:normal-spectrum},
    $\|\Sigma_{d-k}^{-1/2}H\| \leq O(\sqrt{d}+\sqrt{k}) \leq O(\sqrt{d})$
    with probability at least $1-\tfrac{0.01}{t}$.
    Also, $\|\Pi_{U_{\bot}}\Sigma_{d-k}^{1/2}\| \leq O(\gamma\sqrt{s_k(\Sigma_k)}) \leq O(\gamma)$.
    This gives us:
    $$z_{21} \leq O(\gamma\sqrt{d}).$$

    Since $a^2 > 2b^2$, we get the following by
    Lemma~\ref{lem:sin-theta}.
    \begin{align*}
        \|\text{Sin}(\Theta)(U,\hat{U})\| &\leq \frac{az_{21} + bz_{12}}
                {a^2-b^2-\min\{z_{12}^2,z_{21}^2\}}\\
            &\leq O\left(\frac{\gamma\sqrt{d}}{\sqrt{m}}\right)
    \end{align*}

    Therefore, using Lemma~\ref{lem:sin-theta-property},
    and applying the union bound over all $j$, we get the
    required result.
\end{proof}

Let $\xi = O\left(\tfrac{\gamma\sqrt{d}}{\sqrt{m}}\right)$. We
show that the projections of any
reference point are close.

\begin{corollary}\label{coro:reference-projections-close}
    Let $p_1,\dots,p_q$ be the reference points as
    defined in Algorithm~\ref{alg:approximate}, and
    let $\Pi$ and $\Pi_j$ (for $1 \leq j \leq t$) be
    projections matrices as defined in Lemma~\ref{lem:empirical-subspaces-close}.
    Then
    $$\pr{}{\forall i,j, \|(\Pi-\Pi_j)p_i\| \leq O(\xi(\sqrt{k}+\sqrt{\ln(qt)}))} \geq 0.9.$$
\end{corollary}
\begin{proof}
    We know from Lemma~\ref{lem:empirical-subspaces-close}
    that $\|\Pi-\Pi_j\| \leq \xi$ for all $j$ with
    probability at least $0.95$. For $j \in [t]$, let
    $\wh{\Pi}_j$ be the projection matrix for the union
    of the $j^{\text{th}}$ subspace and the subspace
    spanned by $\Sigma_k$. Lemma~\ref{lem:gauss-vector-norm}
    implies that with probability at least $0.95$,
    for all $i,j$, $\|\wh{\Pi}_j p_i\| \leq O(\sqrt{k}+\sqrt{\ln(qt)})$.
    Therefore,
    \begin{align*}
        \|(\Pi-\Pi_j)p_i\| &= \|(\Pi-\Pi_j)\wh{\Pi}_jp_i\|
            \leq \|\Pi-\Pi_j\|\cdot\|\wh{\Pi}_jp_i\|
            \leq O(\xi(\sqrt{k}+\sqrt{\ln(qt)})).
    \end{align*}
    Hence, the claim.
\end{proof}

The above corollary shows that the projections of
each reference point lie in a ball of radius $O(\xi\sqrt{k})$.
Next, we show that for each reference point, all the
projections of the point lie inside a histogram cell
with high probability. For notational convenience, since
each point in $Q$ is a concatenation of the projection
of all reference points on a given subspace, for all
$i,j$, we refer to
$(0,\dots,0,Q_{(i-1)d+1}^j,\dots,Q_{id}^j,0,\dots,0) \in R^{qd}$
(where there are $(i-1)d$ zeroes behind $Q_{(i-1)d+1}^j$,
and $(q-i)d$ zeroes after $Q_{id}^j$) as $p_i^j$.

\begin{lemma}\label{lem:histogram-cell-points}
    Let $\ell$ and $\lambda$ be the length of a histogram
    cell and the random offset respectively, as defined in
    Algorithm~\ref{alg:approximate}. Then
    $$\pr{}{|\omega \cap Q| = t} \geq 0.8.$$
    Thus there exists $\omega \in \Omega$ that,
    such that all points in $Q$ lie within $\omega$.
\end{lemma}
\begin{proof}
    Let $r = O(\xi(\sqrt{k}+\sqrt{\ln(qt)}))$. This implies that $\ell = 20r\sqrt{q}$.
    The random offset could also be viewed as moving along a
    diagonal of a cell by $\lambda\ell\sqrt{dq}$. We know that
    with probability at least $0.8$, for each $i$, all projections
    of reference point $p_i$ lie in a ball of radius $r$.
    This means that all the points in $Q$ lie in a ball of
    radius $r\sqrt{q}$. Then
    $$\pr{}{|\omega \cap Q| = t} \leq \pr{}{\frac{1}{20} \geq
        \lambda \vee \lambda \geq \frac{19}{20}} = \frac{1}{10}.$$
    Taking the union bound over all $q$ and the failure
    of the event in Corollary~\ref{coro:reference-projections-close},
    we get the claim.
\end{proof}

Now, we analyse the sample complexity due
to the private algorithm, that is,
DP-histograms.

\begin{lemma}\label{lem:dp-histogram-cost}
    Let $\omega$ be the histogram cell as defined in
    Algorithm~\ref{alg:approximate}. Suppose $\pcount(\omega)$
    is the noisy count of $\omega$ as a result of applying
    the private histogram. If
    $t \geq O\left(\frac{\log(1/\delta)}{\eps}\right),$
    then
    $$\pr{}{\abs{\pcount(\omega)} \geq \frac{t}{2}} \geq 0.75.$$
\end{lemma}
\begin{proof}
    Lemma~\ref{lem:histogram-cell-points} implies that
    with probability at least $0.8$, for each $i$, all
    projections of $p_i$ lie in a histogram cell, that is,
    all points of $Q$ lie in a histogram cell in $\Omega$.
    Because of the error bound in Lemma~\ref{lem:priv-hist}
    and our bound on $t$, we see at least $\tfrac{t}{2}$
    points in that cell with probability at least $1-0.05$.
    Therefore, by taking the union bound, the proof is complete.
\end{proof}

We finally show that the error of the projection matrix
that is output by Algorithm~\ref{alg:approximate} is small.

\begin{lemma}\label{lem:final-projection}
    Let $\wh{\Pi}$ be the projection matrix as defined in
    Algorithm~\ref{alg:approximate}, and $n$ be the total
    number of samples. If
    $$\gamma^2 \in
        O\left(\frac{\eps\alpha^2n}{d^{2}k\ln(1/\delta)}\cdot
        \min\left\{\frac{1}{k},
        \frac{1}{\ln(k\ln(1/\delta)/\eps)}
        \right\}\right),$$
    $n \geq O(\frac{k\log(1/\delta)}{\eps}+\frac{\ln(1/\delta)\ln(\ln(1/\delta)/\eps)}{\eps})$,
    and $q \geq O(k)$
    the with probability at least $0.7$, $\|\wh{\Pi}-\Pi\| \leq \alpha$.
\end{lemma}
\begin{proof}
    For each $i \in [q]$, let $p_i^*$ be the projection
    of $p_i$ on to the subspace spanned by $\Sigma_k$,
    $\wh{p}_i$ be as defined in the algorithm, and $p_i^j$
    be the projection of $p_i$ on to the subspace spanned
    by the $j^{\mathrm{th}}$ subset of $X$. From Lemma~\ref{lem:dp-histogram-cost},
    we know that all $p_i^j$'s are contained in a histogram
    cell of length $\ell$. This implies that $p_i^*$ is also
    contained within the same histogram cell.

    Now, let $P=(p_1^*,\dots,p_q^*)$ and $\wh{P}=(\wh{p}_1,\dots,\wh{p}_q)$.
    Then by above, $\wh{P}=P+E$, where $\|E\|_F \leq 2\ell\sqrt{dq}$. Therefore,
    $\|E\| \leq 2\ell\sqrt{dq}$. Let $E=E_P+E_{\wb{P}}$,
    where $E_P$ is the component of $E$ in the subspace
    spanned by $P$, and $E_{\wb{P}}$ be the orthogonal
    component. Let $P' = P + E_P$. We will be analysing
    $\wh{P}$ with respect to $P'$.

    Now, with probability
    at least $0.95$, $s_k(P) \in \Theta(\sqrt{k})$ due to our
    choice of $q$ and using Corollary~\ref{coro:normal-spectrum},
    and $s_{k+1}(P) = 0$. So, $s_{k+1}(P') = 0$ because $E_P$ is
    in the same subspace as $P$. Now, using Lemma~\ref{lem:least-singular},
    we know that $s_k(P') \geq s_k(P) - \|E_P\| \geq \Omega(\sqrt{k}) > 0$.
    This means that
    $P'$ has rank $k$, so the subspaces spanned by $\Sigma_k$
    and $P'$ are the same.

    As before, we will try to
    bound the distance between the subspaces spanned
    by $P'$ and $\wh{P}$. Note that using Lemma~\ref{lem:weyl-singular},
    we know that $s_k(P') \leq s_k(P) + \|E_P\| \leq O(\sqrt{k})$.

    We wish to invoke Lemma~\ref{lem:sin-theta} again. Let $UDV^T$
    be the SVD of $P'$, and let $\hat{U}\hat{D}\hat{V}^T$ be
    the SVD of $\wh{P}$. Now, for a matrix $M$, let $\Pi_M$ denote
    the projection matrix of the subspace spanned by the columns
    of $M$. Define quantities $a,b,z_{12},z_{21}$ as follows.
    \begin{align*}
        a &= s_{\min}(U^T\wh{P}V)\\
            &= s_{\min}(U^TP'V + U^TE_{\wb{P}}V)\\
            &= s_{\min}(U^TP'V) \tag{Columns of $U$ are orthogonal to columns of $E_{\wb{P}}$}\\
            &= s_k(P')\\
            &\in \Theta(\sqrt{k})\\
        b &= \|U_{\bot}^T\wh{P}V_{\bot}\|\\
            &= \|U_{\bot}^TP'V_{\bot} + U_{\bot}^TE_{\wb{P}}V_{\bot}\|\\
            &= \|U_{\bot}^TE_{\wb{P}}V_{\bot}\|
                \tag{Columns of $U_{\bot}$ are orthogonal to columns of $P'$}\\
            &\leq \|E_{\wb{P}}\|\\
            &\leq O(\ell\sqrt{dq})\\
        z_{12} &= \|\Pi_U E_{\wb{P}} \Pi_{V_{\bot}}\|\\
            &= 0\\
        z_{21} &= \|\Pi_{U_{\bot}}E_{\wb{P}}\Pi_V\|\\
            &\leq \|E_{\wb{P}}\|\\
            &\leq O(\ell{\sqrt{dq}})
    \end{align*}

    Using Lemma~\ref{lem:sin-theta}, we get the following.
    \begin{align*}
        \|\text{Sin}(\Theta)(U,\hat{U})\| &\leq \frac{az_{21} + bz_{12}}
                {a^2-b^2-\min\{z_{12}^2,z_{21}^2\}}\\
            &\leq O\left(\ell\sqrt{dk}\right)\\
            &\leq \alpha
    \end{align*}

    This completes our proof.
\end{proof}

\subsection{Boosting}

In this subsection, we discuss boosting of error
guarantees of Algorithm~\ref{alg:approximate}.
The approach we use is very similar to the well-known
Median-of-Means method: we run the algorithm multiple
times, and choose an output that is close to all
other ``good'' outputs. We formalise this in
Algorithm~\ref{alg:approximate-boosted}.

\begin{algorithm}[h!]
\caption{\label{alg:approximate-boosted}DP Approximate Subspace Estimator Boosted
    $\DPASEB_{\eps, \delta, \alpha, \beta, \gamma, k}(X)$}
\KwIn{Samples $X_1,\dots,X_{n} \in \R^d$.
    Parameters $\eps, \delta, \alpha, \beta, \gamma, k > 0$.}
\KwOut{Projection matrix $\wh{\Pi} \in \R^{d \times d}$ of rank $k$.}
\vspace{5pt}

Set parameters:
    $t \gets C_3 \log(1/\beta)$ \qquad $m \gets \lfloor n/t \rfloor$
\vspace{5pt}

Split $X$ into $t$ datasets of size $m$: $X^1,\dots,X^t$.
\vspace{5pt}

\tcp{Run $\DPASE$ $t$ times to get multiple projection matrices.}
\For{$i \gets 1,\dots,t$}{
    $\wh{\Pi}_i \gets \DPASE_{\eps,\delta,\alpha,\gamma,k(X^i)}$
}
\vspace{5pt}

\tcp{Select a good subspace.}
\For{$i \gets 1,\dots,t$}{
    $c_i \gets 0$\\
    \For{$j \in [t]\setminus\{i\}$}{
        \If{$\|\wh{\Pi}_i-\wh{\Pi}_j\| \leq 2\alpha$}{
            $c_i \gets c_i + 1$
        }
    }
    \If{$c_i \geq 0.6t-1$}{
        \Return $\wh{\Pi}_i$.
    }
}
\vspace{5pt}

\tcp{If there were not enough good subspaces, return $\bot$.}
\Return $\bot.$
\vspace{5pt}
\end{algorithm}

Now, we present the main result of this subsection.

\begin{theorem}\label{thm:approximate-boosted}
    Let $\Sigma \in \R^{d \times d}$ be an arbitrary, symmetric, PSD
    matrix of rank $\geq k \in \{1,\dots,d\}$, and let $0 < \gamma < 1$.
    Suppose $\Pi$ is the projection matrix corresponding
    to the subspace spanned by the vectors of $\Sigma_k$.
    Then given
    $$\gamma^2 \in
        O\left(\frac{\eps\alpha^2n}{d^{2}k\ln(1/\delta)}\cdot
        \min\left\{\frac{1}{k},
        \frac{1}{\ln(k\ln(1/\delta)/\eps)}
        \right\}\right),$$
    such that $\lambda_{k+1}(\Sigma) \leq \gamma^2\lambda_k(\Sigma)$,
    for every $\eps,\delta>0$, and $0 < \alpha,\beta < 1$,
    there exists and $(\eps,\delta)$-DP algorithm that takes
    $$n \geq O\left(\frac{k\log(1/\delta)\log(1/\beta)}{\eps} +
        \frac{\log(1/\delta)\log(\log(1/\delta)/\eps)\log(1/\beta)}{\eps}\right)$$
    samples from $\cN(\vec{0},\Sigma)$, and outputs a projection matrix $\wh{\Pi}$,
    such that $\|\Pi-\wh{\Pi}\| \leq \alpha$ with probability
    at least $1-\beta$.
\end{theorem}
\begin{proof}
    Privacy holds trivially by Theorem~\ref{thm:approximate}.

    We know by Theorem~\ref{thm:approximate} that
    for each $i$, with probability at least $0.7$,
    $\|\wh{\Pi}_i-\Pi\| \leq \alpha$. This means
    that by Lemma~\ref{lem:chernoff-add}, with probability
    at least $1-\beta$, at least $0.6t$ of all
    the computed projection matrices are accurate.

    This means that there has to be at least one projection
    matrix that is close to $0.6t-1>0.5t$ of these
    accurate projection matrices. So, the algorithm
    cannot return $\bot$.

    Now, we want to argue that the returned projection
    matrix is accurate, too. Any projection matrix
    that is close to at least $0.6t-1$ projection
    matrices must be close to at least one accurate
    projection matrix (by pigeonhole principle). Therefore,
    by triangle inequality,
    it will be close to the true subspace. Therefore,
    the returned projection matrix is also accurate.
\end{proof}

%\subsection{Putting It All Together}

%Now, we are ready to finish the main theorem of the section.

%\begin{proof}[Proof of Theorem~\ref{thm:approximate}]
%    By Corollary~\ref{coro:privacy} and Lemma~\ref{lem:final-projection},
%    Algorithm~\ref{alg:approximate} is differentially private
%    and accurate.
%\end{proof}

%\break
\addcontentsline{toc}{section}{References}
%\bibliographystyle{alpha}
%\bibliography{biblio}

\printbibliography

%\appendix

%\begin{appendices}

\section{Simple versions of the algorithms}
\label{appendix:simple_algs}

\begin{algorithm}
\caption{Online algorithm}\label{online_simple}
\begin{algorithmic}
\State Initialize \textsc{Student} learning algorithm
\State Initialize expected return $Q(a)=0$ for all $N$ tasks
\For{t=1,\ldots,T}
\State Choose task $a_t$ based on $|Q|$ using $\epsilon$-greedy or Boltzmann policy
\State Train \textsc{Student} using task $a_t$ and observe reward $r_t = x_t^{(a_t)} - x_{t'}^{(a_t)}$
\State Update expected return $Q(a_t) = \alpha r_t + (1 - \alpha) Q(a_t)$
\EndFor
\end{algorithmic}
\end{algorithm}

\begin{algorithm}
\caption{Naive algorithm}\label{naive_simple}
\begin{algorithmic}
\State Initialize \textsc{Student} learning algorithm
\State Initialize expected return $Q(a)=0$ for all $N$ tasks
\For{t=1,...,T}
\State Choose task $a_t$ based on $|Q|$ using $\epsilon$-greedy or Boltzmann policy
\State Reset $D=\emptyset$
\For{k=1,...,K}
\State Train \textsc{Student} using task $a_t$ and observe score $o_t = x_t^{(a_t)}$
\State Store score $o_t$ in list $D$
\EndFor
\State Apply linear regression to $D$ and extract the coefficient as $r_t$
\State Update expected return $Q(a_t) = \alpha r_t + (1 - \alpha) Q(a_t)$
\EndFor
\end{algorithmic}
\end{algorithm}

\begin{algorithm}
\caption{Window algorithm}\label{window_simple}
\begin{algorithmic}
\State Initialize \textsc{Student} learning algorithm
\State Initialize FIFO buffers $D(a)$ and $E(a)$ with length $K$ for all $N$ tasks
\State Initialize expected return $Q(a)=0$ for all $N$ tasks
\For{t=1,\ldots,T}
\State Choose task $a_t$ based on $|Q|$ using $\epsilon$-greedy or Boltzmann policy
\State Train \textsc{Student} using task $a_t$ and observe score $o_t = x_t^{(a_t)}$
\State Store score $o_t$ in $D(a_t)$ and timestep $t$ in $E(a_t)$
\State Use linear regression to predict $D(a_t)$ from $E(a_t)$ and use the coef. as $r_t$
%\State Update expected return $Q(a_t) := r_t$
\State Update expected return $Q(a_t) = \alpha r_t + (1 - \alpha) Q(a_t)$
\EndFor
\end{algorithmic}
\end{algorithm}

\begin{algorithm}
\caption{Sampling algorithm}\label{sampling_simple}
\begin{algorithmic}
\State Initialize \textsc{Student} learning algorithm
\State Initialize FIFO buffers $D(a)$ with length $K$ for all $N$ tasks
\For{t=1,\ldots,T}
\State Sample reward $\tilde{r}_a$ from $D(a)$ for each task (if $|D(a)|=0$ then $\tilde{r}_a=1$)
\State Choose task $a_t = \argmax_a |\tilde{r}_a|$
\State Train \textsc{Student} using task $a_t$ and observe reward $r_t = x_t^{(a_t)} - x_{t'}^{(a_t)}$
\State Store reward $r_t$ in $D(a_t)$
\EndFor
\end{algorithmic}
\end{algorithm}

\newpage
\section{Batch versions of the algorithms}
\label{appendix:batch_algs}

\begin{algorithm}
\caption{Online algorithm}\label{online_batch}
\begin{algorithmic}
\State Initialize \textsc{Student} learning algorithm
\State Initialize expected return $Q(a)=0$ for all $N$ tasks
\For{t=1,\ldots,T}
\State Create prob. dist. $\vec{a_t}=(p_t^{(1)}, ..., p_t^{(N)})$ based on $|Q|$ using $\epsilon$-greedy or Boltzmann policy
\State Train \textsc{Student} using prob. dist. $\vec{a_t}$ and observe scores $\vec{o_t} = (x_t^{(1)}, ..., x_t^{(N)})$
\State Calculate score changes $\vec{r_t} = \vec{o_t} - \vec{o_{t-1}}$
%\State Calculate score change $\hat{r}_t = o_t - o_{t-1}$
%\State Calculate corrected reward $r_t = \hat{r}_t / a_t$ ($a_t$ is prob. dist.)
\State Update expected return $\vec{Q} = \alpha \vec{r_t} + (1 - \alpha) \vec{Q}$
\EndFor
\end{algorithmic}
\end{algorithm}

\begin{algorithm}
\caption{Naive algorithm}\label{online_naive}
\begin{algorithmic}
\State Initialize \textsc{Student} learning algorithm
\State Initialize expected return $Q(a)=0$ for all $N$ tasks
\For{t=1,\ldots,T}
\State Create prob. dist. $\vec{a_t}=(p_t^{(1)}, ..., p_t^{(N)})$ based on $|Q|$ using $\epsilon$-greedy or Boltzmann policy
\State Reset $D(a)=\emptyset$ for all tasks
\For{k=1,\ldots,K}
\State Train \textsc{Student} using prob. dist. $\vec{a_t}$ and observe scores $\vec{o_t} = (x_t^{(1)}, ..., x_t^{(N)})$
\State Store score $o_t^{(a)}$ in list $D(a)$ for each task $a$
\EndFor
\State Apply linear regression to each $D(a)$ and extract the coefficients as vector $\vec{r_t}$
%\State Apply linear regression to each $D(a)$ and extract the coefficients as $\hat{r}_t$
%\State Calculate corrected rewards $r_t = \hat{r}_t / a_t$ ($a_t$ is prob. dist.)
\State Update expected return $\vec{Q} = \alpha \vec{r_t} + (1 - \alpha) \vec{Q}$
\EndFor
\end{algorithmic}
\end{algorithm}

\begin{algorithm}
\caption{Window algorithm}\label{online_window}
\begin{algorithmic}
\State Initialize \textsc{Student} learning algorithm
\State Initialize FIFO buffers $D(a)$ with length $K$ for all $N$ tasks
\State Initialize expected return $Q(a)=0$ for all $N$ tasks
\For{t=1,\ldots,T}
\State Create prob. dist. $\vec{a_t}=(p_t^{(1)}, ..., p_t^{(N)})$ based on $|Q|$ using $\epsilon$-greedy or Boltzmann policy
\State Train \textsc{Student} using prob. dist. $\vec{a_t}$ and observe scores $\vec{o_t} = (x_t^{(1)}, ..., x_t^{(N)})$
\State Store score $o_t^{(a)}$ in $D(a)$ for all tasks $a$
\State Apply linear regression to each $D(a)$ and extract the coefficients as vector $\vec{r_t}$
%\State Apply linear regression to each $D(a)$ and extract the coefficients as $\hat{r}_t$
%\State Calculate corrected rewards $r_t = \hat{r}_t / a_t$ ($a_t$ is prob. dist.)
\State Update expected return $\vec{Q} = \alpha \vec{r_t} + (1 - \alpha) \vec{Q}$
%\State Update expected return $Q = r_t$
\EndFor
\end{algorithmic}
\end{algorithm}

\begin{algorithm}
\caption{Sampling algorithm}\label{online_sampling}
\begin{algorithmic}
\State Initialize \textsc{Student} learning algorithm
\State Initialize FIFO buffers $D(a)$ with length $K$ for all $N$ tasks
\For{t=1,\ldots,T}
\State Sample reward $\tilde{r}_a$ from $D(a)$ for each task (if $|D(a)|=0$ then $\tilde{r}_a=1$)
\State Create one-hot prob. dist. $\vec{\tilde{a}_t}=(p_t^{(1)}, ..., p_t^{(N)})$ based on $\argmax\nolimits_a |\tilde{r}_a|$
\State Mix in uniform dist. : $\vec{a_t} = (1 - \epsilon) \vec{\tilde{a}_t} + \epsilon/N$
\State Train \textsc{Student} using prob. dist. $\vec{a_t}$ and observe scores $\vec{o_t} = (x_t^{(1)}, ..., x_t^{(N)})$
\State Calculate score changes $\vec{r_t} = \vec{o_t} - \vec{o_{t-1}}$
%\State Calculate score change $\hat{r}_t = o_t - o_{t-1}$
%\State Calculate corrected rewards $r_t = \hat{r}_t / a_t$ ($a_t$ is prob. dist.)
\State Store reward $r_t^{(a)}$ in $D(a)$ for each task $a$
\EndFor
\end{algorithmic}
\end{algorithm}

\clearpage
\section{Decimal Number Addition Training Details}
\label{appendix:addition}

Our reimplementation of decimal addition is based on Keras \citep{chollet2015keras}. The encoder and decoder are both LSTMs with 128 units. In contrast to the original implementation, the hidden state is not passed from encoder to decoder, instead the last output of the encoder is provided to all inputs of the decoder. One curriculum training step consists of training on 40,960 samples. Validation set consists of 4,096 samples and 4,096 is also the batch size. Adam optimizer \citep{kingma2014adam} is used for training with default learning rate of 0.001. Both input and output are padded to a fixed size.

In the experiments we used the number of steps until 99\% validation set accuracy is reached as a comparison metric. The exploration coefficient $\epsilon$ was fixed to 0.1, the temperature $\tau$ was fixed to 0.0004, the learning rate $\alpha$ was 0.1, and the window size $K$ was 10 in all experiments.
 
\section{Minecraft Training Details}
\label{appendix:minecraft}

The Minecraft task consisted of navigating through randomly generated mazes. The maze ends with a target block and the agent gets 1,000 points by touching it. Each move costs -0.1 and dying in lava or getting a timeout yields -1,000 points. Timeout is 30 seconds (1,500 steps) in the first task and 45 seconds (2,250 steps) in the subsequent tasks.

For learning we used the \textit{proximal policy optimization} (PPO) algorithm \citep{schulman2017proximal} implemented using Keras \citep{chollet2015keras} and optimized for real-time environments. The policy network used four convolutional layers and one LSTM layer. Input to the network was $40\times 30$ color image and outputs were two Gaussian actions: move forward/backward and turn left/right. In addition the policy network had state value output, which was used as the baseline. Figure \ref{f14} shows the network architecture.

\begin{figure}[h]
  \includegraphics[scale=0.4]{figures/minecraft_network}
\caption{Network architecture used for Minecraft.}
\label{f14}
\end{figure}

For training we used a setup with 10 parallel Minecraft instances. The agent code was separated into runners, that interact with the environment, and a trainer, that performs batch training on GPU, similar to \cite{babaeizadeh2016reinforcement}. Runners regularly update their snapshot of the current policy weights, but they only perform prediction (forward pass), never training. After a fixed number of steps they use FIFO buffers to send collected states, actions and rewards to the trainer. Trainer collects those experiences from all runners, assembles them into batches and performs training. FIFO buffers shield the runners and the trainer from occasional hiccups. This also means that the trainer is not completely on-policy, but this problem is handled by the importance sampling in PPO.

\begin{figure}[h]
  \includegraphics[scale=0.4]{figures/minecraft_training}
\caption{Training scheme used for Minecraft.}
\label{f14}
\end{figure}

During training we also used frame skipping, i.e. processed only every 5th frame. This sped up the learning considerably and the resulting policy also worked without frame skip. Also, we used auxiliary loss for predicting the depth as suggested in \citep{mirowski2016learning}. Surprisingly this resulted only in minor improvements.

For automatic curriculum learning we only implemented the Window algorithm for the Minecraft task, because other algorithms rely on score change, which is not straightforward to calculate for parallel training scheme. Window size was defined in timesteps and fixed to 10,000 in the experiments, exploration rate was set to 0.1.

The idea of the first task in the curriculum was to make the agent associate the target with a reward. In practice this task proved to be too simple - the agent could achieve almost the same reward by doing backwards circles in the room. For this reason we added penalty for moving backwards to the policy loss function. This fixed the problem in most cases, but we occasionally still had to discard some unsuccessful runs. Results only reflect the successful runs.

We also had some preliminary success combining continuous (Gaussian) actions with binary (Bernoulli) actions for "jump" and "use" controls, as shown on figure \ref{f14}. This allowed the agent to learn to cope also with rooms that involve doors, switches or jumping obstacles, see \url{https://youtu.be/e1oKiPlAv74}.

\end{appendices}

\end{document}
