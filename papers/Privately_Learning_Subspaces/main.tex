\def\epsdeltastuff{0}
\def\comments{0}
\def\supp{0}
\documentclass[letterpaper,11pt]{article}
\usepackage[utf8]{inputenc}
\usepackage{times}
\usepackage{comment}
\usepackage{preamble}
\allowdisplaybreaks
\newcommand{\dr}[3]{\mathrm{D}_{#1}\left(#2\middle\|#3\right)}
\newcommand{\nope}[1]{}
\newcommand{\stcomp}[1]{\overline{#1}}
\newcommand{\iprod}[1]{\langle #1 \rangle}
\newcommand{\llnorm}[1]{\left\lVert#1\right\rVert_2}
\newcommand{\fnorm}[1]{\left\lVert#1\right\rVert_F}
\newcommand{\norm}[1]{\left\lVert#1\right\rVert}
\newcommand{\sgnorm}[1]{\left\lVert#1\right\rVert_{\Psi^2}}
\newcommand{\abs}[1]{\left| #1 \right|}
\newcommand{\chisq}[1]{\chi^2\left( #1 \right)}
\renewcommand{\epsilon}{\varepsilon}
\newcommand{\ball}[2]{\mathit{B}_{#2}\left( #1 \right)}
\newcommand{\myeqand}{\quad \textrm{and} \quad}

\newcommand{\X}{\mathcal{X}}
\newcommand{\Q}{\mathcal{Q}}
\newcommand{\M}{\mathcal{M}}

\newcommand{\EE}{\mathbb{E}}
\newcommand{\RR}{\mathbb{R}}

\newcommand{\ONE}{\mathbbm{1}}

\newcommand{\vars}{\mathit{Var}}

\newcommand{\IN}{\mathsf{IN}}
\newcommand{\OUT}{\mathsf{OUT}}
\newcommand{\NULL}{\mathsf{NULL}}

\newcommand{\Sym}{\mathsf{Sym}}

\renewcommand{\k}{\kappa}
\newcommand{\id}{\mathbb{I}}
\newcommand{\Lap}{\mathrm{Lap}}
\newcommand{\TLap}{\mathrm{TLap}}
\newcommand{\opt}{\mathrm{opt}}
\newcommand{\OPT}{\mathrm{OPT}}
\newcommand{\Score}{\textsc{Score}}
\newcommand{\Match}{\mathrm{Match}}
\newcommand{\Median}{\mathrm{Median}}
\newcommand{\pcount}{\mathrm{Count}}

\newcommand{\DPASE}{\mathrm{DPASE}}
\newcommand{\DPASEB}{\mathrm{DPASEB}}
\newcommand{\DPESE}{\mathrm{DPESE}}
\newcommand{\GAPMAX}{\mathrm{GAP\text{-}MAX}}

\newcommand{\tnote}[1]{}%\textcolor{red}{[Thomas: #1]}}
\newcommand{\vnote}[1]{}%\textcolor{orange}{[Vikrant: #1]}}

\usepackage[style=alphabetic,backend=bibtex,maxalphanames=10,maxbibnames=20,maxcitenames=10,giveninits=true,doi=false,url=true]{biblatex}
\newcommand*{\citet}[1]{\AtNextCite{\AtEachCitekey{\defcounter{maxnames}{2}}}\textcite{#1}}
\newcommand*{\citetall}[1]{\AtNextCite{\AtEachCitekey{\defcounter{maxnames}{999}}}\textcite{#1}}
\newcommand*{\citep}[1]{\citep{#1}}
\newcommand{\citeyearpar}[1]{\citep{#1}}
\usepackage{hyperref}

\addbibresource{biblio.bib}

\title{Privately Learning Subspaces}
\iftrue
\author{Vikrant Singhal
    \thanks{Northeastern University. Part of this work was done during an internship at IBM Research -- Almaden.~\dotfill~\texttt{singhal.vi@northeastern.edu}}
    \and
    Thomas Steinke\thanks{Google Research, Brain Team. Part of this work was done at IBM Research --
    Almaden.~\dotfill~\texttt{subspace@thomas-steinke.net}}}
\date{}
\fi
%\coltauthor{%
% \Name{Vikrant Singhal} \Email{singhal.vi@northeastern.edu}\\
% \addr Northeastern University
% \AND
% \Name{Thomas Steinke} \Email{subspace@thomas-steinke.net}\\
% \addr Google Research, Brain Team
%}

\begin{document}
\maketitle

\begin{abstract}
    Private data analysis suffers a costly curse of dimensionality. However, the data often has an underlying low-dimensional structure. For example, when optimizing via gradient descent, the gradients often lie in or near a low-dimensional subspace. If that low-dimensional structure can be identified, then we can avoid paying (in terms of privacy or accuracy) for the high ambient dimension. 
    
    We present differentially private algorithms that take input data sampled from a low-dimensional linear subspace (possibly with a small amount of error) and output that subspace (or an approximation to it). These algorithms can serve as a pre-processing step for other procedures.
\end{abstract}
%\newpage

Reinforcement learning has achieved great success in areas such as Game-playing \citep{silver2018general,vinyals2019grandmaster}, robotics \cite{kober2013reinforcement}, large language models \citep{ouyang2022training}, etc.
However, due to safety concerns or physical limitations, in some real-world reinforcement learning problems, we must consider additional constraints that may influence the optimal policy and the learning process \citep{garcia2015comprehensive}.
% For example, a robotic arm must not take actions that may cause harm to itself or the environments.
A standard framework to handle such cases is the constrained Markov Decision Process (CMDP) \citep{altman1999constrained}.
Within the CMDP framework, the agent has to maximize
the expected cumulative reward while
obeying a finite number of constraints, which are usually in the form of expected cumulative cost criteria.

However, we are sometimes concerned with the problem with a continuum of constraints.
For example,
the constraints we meet might be time-evolving or subject to uncertain parameters, which
cannot be formulated as an ordinary CMDP
(see Examples \ref{Example_Time_Evolving} and  \ref{Example_Uncertain}).
In this paper we would study a generalized CMDP  
to address the above problem.  Because the constraints are not only infinite-number but also lie
in a continuous set,
the generalization is not trivial. Fortunately, we find that we can borrow the idea behind semi-infinite programming (SIP) \citep{remez1934determination, hettich1993semi} to deal with the semi-infinite constraints.
Accordingly, we propose \emph{semi-infinitely constrained Markov decision processes} (SICMDPs)
as a novel complement to the ordinary CMDP framework.
%More specifically,  an SICMDP model %, we consider 
%contains a continuum of constraints whereas an ordinary CMDP contains a finite number of constraints. 

%This generalization is natural but not trivial. However, we can brows the idea  
%The idea is quite natural and can be backtracked
%to the practice of extending linear programming to linear semi-infinite programming (LSIP) %\cite{remez1934determination, GobernaLSIO1998}.
%In addition, 
%As a complementary approach to the ordinary CMDP framework, 
%SICMDP can be used to model these problems  which cannot be described by a finite number of constraints
%that are not covered by .
%For example,
%the restrictions we consider can be time-evolving or subject to uncertain parameters
%, thus
%cannot be described by a finite number of constraints but a continuum of constraints 
%(see Examples \ref{Example_Time_Evolving} and  \ref{Example_Uncertain}).

We also present two reinforcement learning algorithms to solve SICMDPs called SI-CRL and SI-CPO, respectively.
SI-CRL is a model-based reinforcement learning algorithm designed for tabular cases, and SI-CPO is a policy optimization algorithm for non-tabular cases.
% and analyze its performance both theoretically and empirically.
The main challenge is that we need to deal with a continuum of constraints, thus reinforcement learning algorithms for ordinary CMDPs do not work anymore.
In SI-CRL, we tackle this difficulty by first transforming the reinforcement learning problem to an equivalent LSIP problem, which can then be solved using methods in the LSIP literature like the dual exchange methods \citep{Hu1990,reemtsen1998numerical}.
In SI-CPO, we resort to the idea of cooperative stochastic approximation developed in \cite{lan2020algorithms, wei2020comirror}.
As far as we know, we are the first to introduce tools from semi-infinitely programming (SIP) into the reinforcement learning community for solving constrained reinforcement learning problems.

% To the best of our knowledge, we are the first to apply tools from semi-infinitely programming (SIP) to solve reinforcement learning problems.
Furthermore, we give theoretical analysis for both SI-CRL and SI-CPO.
We decompose the error of SI-CRL into two parts: the statistical error from approximating the true SICMDP with an offline dataset and the optimization error due to the fact that the solution of the LSIP problem obtained by the dual exchange method is inexact.
On the optimization side, we show that the iteration complexity of SI-CRL is $O\left(\left\{\mathrm{diam}(Y)L\sqrt{|\gS|^2|\gA|m}/\left[(1-\gamma)\epsilon\right]\right\}^m\right)$.
On the statistical side, we show that the sample complexity of SI-CRL is $\widetilde O\left(\frac{|S|^2|A|^2}{\epsilon^2(1-\gamma)^3}\right)$ if the offline dataset is generated by a generative model, and $\widetilde O\left(\frac{|S||A|}{\nu_{\min} \epsilon^2(1-\gamma)^3}\right)$ if the dataset is generated by a probability measure $\nu$ as considered in \cite{chen2019information}.
Here $\widetilde O$ means that all logarithm terms are discarded.
For SI-CPO, things become a little more complicated because other than the statistical error and the optimization error, we also need to consider the function approximation error, which comes from imperfect policy parametrizations.
It is shown if the function approximation error can be controlled to $O(\epsilon)$ order, the iteration complexity of SI-CPO is $\widetilde{O}\left(\frac{1}{\epsilon^2(1-\gamma)^6}\right)$ and the sample complexity of SI-CPO is $\widetilde{O}(\frac{1}{\epsilon^4(1-\gamma)^{10}})$.
Here our iteration complexity bound is equivalent to a typical $\widetilde O(1/\sqrt{T})$ global convergence rate.

We perform a set of numerical experiments to illustrate the SICMDP model and validate our proposed algorithms.
Specifically, we examine two numerical examples, namely the discharge of sewage and ship route planning.
Through the discharge of sewage example, we show the advantage of the SICMDP framework over the CMDP baseline obtained by naive discretization in modeling realistic sequential decision-making problems.
Moreover, we demonstrate the effectiveness of the SI-CRL and SI-CPO algorithms in such tabular environments. 
In the ship route planning example, we illustrate the benefits of the SICMDP framework and the ability of the SI-CPO algorithm to address complex continuous control tasks involving continuous state spaces with modern deep reinforcement learning techniques.

% In summary, our contributions are listed as follows.
% First, we present the SICMDP model, which can be viewed as a generalization of the ordinary CMDP model.
% Second, we propose an algorithm to perform reinforcement learning for SICMDPs, which is called SI-CRL, and we believe that we are the first to apply tools from SIP
% to solve reinforcement learning problems.
% Third, we give a theoretical analysis of SI-CRL and identify both its sample complexity and iteration complexity.
% In addition, we perform numerical experiments to illustrate the SICMDP model and validate the SI-CRL algorithm.
% \{This paragraph can be removed!!! \}






\section{Preliminaries}\label{chpt:preliminiaries}
In this chapter we will introduce some of the mathematical background and notation needed for this thesis. In particular, we will shortly introduce the differential geometric description of spacetime in Section \ref{sec:spacetime_geometry} and give an introduction to the notion of global hyperbolicity and its connection to Green- and normally-hyperbolic operators in Section \ref{sec:global_hyperbolicity}. In a bit more detail, we will introduce the notion of differential forms and give explicit definitions, also in terms of an index based notation, in Section \ref{sec:differential_forms}. For completeness, in Section \ref{sec:cat-theory}, we present basic definitions of category theory. The reader familiar with these topics can safely skip this chapter and refer to it when interested in the chosen conventions.
%
%
%
%
%%%%%%
%%SPACTIME GEOMETRY
%%%%%
%
%
%
\subsection{Spacetime geometry}\label{sec:spacetime_geometry}
In GR, the universe is mathematically described as a four dimensional \emph{spacetime}, consisting of a smooth, four dimensional manifold \gls{M} (assumed to be Hausdorff, connected, oriented, time-oriented and para-compact) and a Lorentzian metric $g$. We will assume the signature of the Lorentzian metric $g$ to be $(-,+,+,+)$. The Levi-Civita connection on $(\M,g)$ is as usual denoted by \gls{nabla}.
Throughout this thesis, we treat spacetime as fixed, implementing a gravitational background determined classically by Einstein's field equations. Hence, we neglect any back-reaction of the fields on the metric, both in the quantum and the classical case. In that sense, we treat the fields as \emph{test fields}.\par
For the basic mathematical theory regarding Lorentzian manifolds, we refer to the literature: An introduction to the topic with an emphasis on the physical application in GR is for example given in \cite{wald_GR} and \cite{carroll_spacetime-and-gr}.
Here, we will shortly recap the notion of a tangent space and tangent bundle and generalize to the notion of a vector bundle which we will use in the general description of normally hyperbolic operators and differential forms.
In the following, we generalize the setting to an arbitrary smooth manifold $\N$ of dimension $N$ with either Lorentzian or Riemannian metric $k$.\par
%
%
A \emph{tangent vector} $v_x$ at point $x \in \N$ is a linear map $v_x : C^\infty(\N , \IR) \to \IR$ that obeys the Leibniz rule, that is, for $f,g \in C^\infty (\N,\IR)$ it holds $v_x(fg) = f(x)v_x(g) + v_x(f)g(x)$.
We define the \emph{tangent space} \gls{TxN} of $\N$ at $x$ as the real $N$-dimensional vector space of all tangent vectors at point $x$.
The disjoint union of all tangent spaces is called the \emph{tangent bundle} \gls{TN} of $\N$ and is itself a manifold of dimension $2N$. A \emph{vector field} is a map $v: \N \to T\N$ such that $v(x) \in T_x\N$.
The respective dual spaces, that is the space of all linear functionals, the \emph{co-tangent space} and the \emph{co-tangent bundle}, are denoted by \gls{TsxN} and \gls{TsN} respectively.\par
%
For Lorentzian manifolds, we call a tangent vector $v$ at $x \in \N$ \emph{timelike} if $k_{\mu \nu} v^\mu v^\nu < 0$, \emph{spacelike} if $k_{\mu \nu} v^\mu v^\nu > 0$ and \emph{null} (or lightlike) if $k_{\mu \nu} v^\mu v^\nu = 0$. At every point $x \in \N$, we define the set of all \emph{causal}, that is, either timelike or null, tangent vectors in the tangent space at $x$. This set is called the \emph{light cone} at $x$ and it is split up into two distinct parts, one that we call the future light cone, and one that we call the past light cone at $x$. Since we assume the manifold to be time orientable, there exists a smooth vector field $t$ that is timelike at every $x \in \N$. Given this time orientation, we identify the future (past) light cone with the set of tangent vectors $v \in T_x\N$ such that $k_{\mu\nu} v^\mu t^\nu < 0$ (respectively $> 0$). Therefore, a tangent vector $v$ at $x$ is called \emph{future directed} (past directed) if it lies in the future (past) light cone at $x$.\\
Accordingly, a curve $\gamma : I \to \N$ is called timelike (spacelike, null, causal, future or past directed) if its tangent vector $\dot{\gamma}$ is timelike (spacelike, null, causal, future or past directed) at every $x \in \N$.  For every point $x \in \N$ we define the \emph{causal future/past} \gls{causalfuturepast} of $x$ as the set of all points $q \in \N$ that can be reached by a future directed causal curve originating in $x$. For any subset $S \in \N$ we define $J^\pm (S) = \bigcup_{x \in S} J^\pm(x)$ and $J(S) = J^+(S) \cup J^- (S)$. Finally, the future/past domain of dependence $\gls{futurepastdomainofdependence}$ of a set $S \subset \N$ is the set of all points $x \in \N$ such that every inextendible causal curve through $x$ intersects $S$. The \emph{domain of dependence} \gls{domainofdependence} of $S$ is the union of the future and past domain of dependence of the set $S$.
For more details on the causal structure of spacetime we refer to for example \cite[Chapter 8]{wald_GR}.\par
%
%
%
The notion of tangent bundles can be generalized to the notion of a vector bundle. Instead of ``attaching'' the vector spaces $T_x \N$ to every point $x$ of the manifold, we allow for the occurrence of arbitrary vector spaces, called the fibres of the vector bundle. A vector bundle then consists of the base manifold, in our case $\N$, the total space and a map $\pi$ from the total space to the base manifold, that can be locally trivialized. At each point of the base manifold, the pre-image of $\pi$ is the fibre of the vector bundle. To be precise we define, following \cite{rudolph_schmidt}:
\begin{definition}[Vector bundle]
	A smooth \emph{vector bundle} over $\N$ is a tuple $\gls{vectorbundle} = (E,\N, \pi)$, where $E$ is a smooth manifold and $\pi : E \to \N$ is a smooth surjective map satisfying:
	\begin{enumerate}
		\item For every $x \in \N$, $\pi^{-1}(x)$ is a vector space, called the fibre of the bundle at point $x$.
		\item There exists a finite dimensional vector space $F$, an open covering $\left\{ U_\alpha\right\}_\alpha$ of $\N$ and a family of diffeomorphisms $\chi_\alpha : \pi^{-1}(U_\alpha) \to U_\alpha \times F$ such that for all $\alpha$ it holds $\chi_\alpha \comp \text{pr}_1 =  \restr{\pi}{\pi^{-1}(U_\alpha)}$ and for every $x \in \N$ the map $\text{pr}_2 \comp \restr{\chi_\alpha}{\pi^{-1}(x)} : \pi^{-1}(x) \to F$ is linear.
	\end{enumerate}
\end{definition}
Here, the maps $\text{pr}_1$ and $\text{pr}_2$ denote the projection onto the first respectively second component of an element in $U_\alpha \times F$. The properties graphically mean that \emph{locally}, the vector bundle ``looks like" the product of the base manifold with the fibre. The tuples $(U_\alpha, \chi_\alpha)$ are called \emph{local trivializations} of the vector bundle. Like for vector spaces, we can define the sum and product of vector bundles, by using the according vector space definitions on the fibres of the bundle.\par
Let $\mathfrak{X}, \mathfrak{Y}$ be vector bundles over $\N$ with fibres $X_x$ and $Y_x$ at $x \in \N$. We denote by \gls{whitneysum} the \emph{Whitney sum} of the two vector bundles - the vector bundle over $\N$ whose fibres are given by the direct sum $X_x \oplus Y_x$. Similarly, one obtains the local trivializations of the Whitney sum from the trivializations of $\mathfrak{X}, \mathfrak{Y}$ and direct sums.\par
Accordingly, let $\mathfrak{X}, \mathfrak{Y}$ be vector bundles over $\N$ and $\widetilde{\N}$, with fibres $X_x$ and $Y_{\tilde{x}}$ at $x \in \N$, $\tilde{x} \in \widetilde{\N}$ respectively. We denote by \gls{outerproductbundle} the \emph{outer product} of the two vector bundles - the vector bundle over $\N \times \widetilde{\N}$ whose fibres are given by the tensor products $X_x \otimes Y_x$. Similarly, one obtains the local trivializations of the outer product from the trivializations of $\mathfrak{X}, \mathfrak{Y}$ and tensor products. \par
%
Finally, we generalize the notion of vector fields:
\begin{definition}[Sections of vector bundles]
Let $\mathfrak{X}=(E,\N,\pi)$ be a vector bundle with fibres $X_x=\pi^{-1}(x)$ at $x \in \N$. A \emph{smooth section} of the vector bundle is a smooth map $\gamma : \N \to E$ such that $\gamma(x) \in X_x$ for all $x \in \N$. The \emph{vector space of smooth sections} of $\mathfrak{X}$ is denoted by \gls{gammax}, the one with compactly supported sections is as usual denoted by \gls{gammaxzero}.
\end{definition}
In this language, a vector field $v$ is just a smooth section of the tangent bundle of a manifold, $v \in \Gamma(T\N)$. One may therefore identify the physical notion of fields with smooth sections of vector bundles. This point of view will be used to define the notion of differential forms in Section \ref{sec:differential_forms}.\par
In this thesis, we usually are interested in complex valued functions (or sections in general). Therefore, we view all occurring vector bundles as complex, in the sense that we take two distinct copies of the vector bundle, one representing the real, one the imaginary part of the bundle. A section of that complex vector bundle is just a pair of two sections of the real vector bundle under consideration. From now, if not specified explicitly, we will view all vector bundles, including the tangent bundle $T\N$, as complex vector bundles. Accordingly, smooth sections of those bundles will in general be complex valued.
%
%
%
%
%
%
%
%
%%%%%%%
%%PARTIAL DIFFERENTIAL OPERATORS AND GLOBAL HYPERBOLICITY
%%%%%%%
%
%
%
\subsection{Partial differential operators and global hyperbolicity}\label{sec:global_hyperbolicity}
When dealing with field theories, whether classical or quantum, one is, of course, interested in the dynamics of the fields. These are usually described by some partial differential equation, often of second order. In the following, we give a short introduction to the theory of certain partial differential operators acting on smooth sections of a vector bundle over the spacetime $(\M,g)$.\par
%
As we have seen, these smooth sections are generalizations of the notion of a field.  In the following, let $\mathfrak{X}$ denote a vector bundle over the manifold $\M$ and let $P: \Gamma(\mathfrak{X}) \to \Gamma(\mathfrak{X})$ be a partial differential operator acting on smooth sections of the bundle. As in the case of flat spacetime, we are interested in basic questions regarding the differential equation $Pf = j$, for example: Can we formulate a (globally) well posed initial value problem? Does the differential equation possess (unique) solutions? To answer these questions, we will now restrict to the case where $P$ is linear and of second order, as it is often the case in physical applications. One can show that for a certain class of such operators, namely normally hyperbolic partial differential operators of second order, we can rigorously treat these questions.\par
Choosing local coordinates $x=(x_\mu)$ on $\M$ and a local trivialization of $\mathfrak{X}$, a linear partial differential operator of second order is called \emph{normally hyperbolic} if it takes the form
\begin{align}
	P = - \sum_{\mu,\nu} g^{\mu \nu} \partial_\mu \partial_\nu + \sum_{\alpha} A_\alpha (x) \partial_\alpha + B(x) \formspace,
\end{align}
where $A_\alpha$ and $B$ are matrix-valued coefficients depending smoothly on the coordinate $x$ (see. \cite[Chapter 1.5]{baer_ginoux_pfaeffle}). One can also formulate a coordinate independent definition in terms of the principal symbol, which we will not present here (see for example \cite[Section 1.5]{baer_ginoux_pfaeffle} ). \par
%
Normally hyperbolic operators possess unique fundamental solutions (see for example the fundamental solutions to the wave operator as noted in Lemma \ref{lem:fundamental_solution_wave_operator}). These fundamental solutions fulfill certain physically important properties, such as a finite propagation speed smaller than the speed of light. Furthermore, specifying the initial data on some space-like hypersurface $X \in  \M$ specifies a unique solution on the domain of dependence $D(X)$ of $X$. Due to these properties, one often calls normally hyperbolic operators just \emph{wave operators}. But to state a \emph{globally} well posed initial value problem for a wave equation, we need to restrict the class of spacetimes $\M$ under consideration to those that possess space-like hypersurfaces $X$ whose domain of dependence is all of the spacetime, $D(X) = \M$. This leads to the notion of \emph{globally hyperbolic} spacetimes:
\begin{definition}[Global Hyperbolicity]
	A spacetime $\M$ is called \emph{globally hyperbolic} if there exists a Cauchy surface $\gls{sigma}$ in $\M$.
\end{definition}
\noindent Here, a Cauchy surface is a space-like hypersurface $\Sigma \subset \M$ such that every inextendible causal curve $\gamma$ intersects $\Sigma$ exactly once. One can show that Cauchy surfaces fulfill the desired property mentioned above, that is,  $D(\Sigma) = \M$. Furthermore, one can show that any globally hyperbolic spacetime $\M$ is foliated by a one-parameter family $\left\{ \Sigma_t \right\}_t$ of Cauchy surfaces (see for example \cite[Theorem 8.3.14]{wald_GR}). \par
In physical applications, one often finds the dynamics of a theory to be described by wave operators. Most prominently, the Klein-Gordon operator $(\square + m^2)$ acting on scalar fields, or its generalization, the wave operator acting on differential forms introduced in Section \ref{sec:differential_forms}, is normally hyperbolic. But there are also important physical field theories that are not described by wave operators, such as the Proca field treated in this thesis. It turns out that the Proca operator (see Definition \ref{def:proca_operator}) is a so called \emph{Green-hyperbolic} operator. These are again partial differential operators $P$ of second order acting on smooth sections of some vector bundle, such that $P$ (and its dual $P'$) posses fundamental solutions. Obviously, normally hyperbolic operators are Green-hyperbolic, but the opposite is not true. One can generalize some results obtained by studying normally hyperbolic operators to Green-hyperbolic operators. An introduction to this topic is given in \cite{baer_green-hyperbolic}, where it is also shown that the Proca operator is Green-hyperbolic but not normally hyperbolic.\par
For our application, the notion of Green-hyperbolicity is not of vast importance, but it is worth mentioning that there exists a more detailed mathematical background on the treatment of such operators.
A very detailed description of normally hyperbolic operators on Lorentzian manifolds, including proofs of the above statements regarding the initial value problem and the existence of fundamental solutions, is given in \cite{baer_ginoux_pfaeffle}, also with an overview of quantization. A shorter introduction to the topic is for example treated in \cite{baer-ginoux_classical-and-quantum-fields}, also with a description of quantization.
%
%
%
%
%
%
%%%
%
%
%
%%
%%%%%%%%%
%%%DIFFERENTIAL FORMS
%%%%%%%%
%
%
%
\subsection{Differential forms}\label{sec:differential_forms}
%
%
Differential forms provide an elegant, coordinate independent description of calculus on smooth manifolds. In particular, they generalize the notion of line- and volume-integrals that are known from analysis. Differential forms play a remarkable role in physics, as one can argue that they indeed describe fundamental physical entities. As an example, instead of viewing a classical force as a vector, one can think of it, more closely related to experiments, as a differential one-form that assigns a scalar to a tangent vector of a curve. This scalar is the (infinitesimal) work associated with the force along the curve. Also, differential forms allow for an elegant geometric description of field theories, for example the Maxwell and Proca field theories that we encounter in this thesis. In Maxwell's classical theory of electromagnetism, instead of viewing the electric and magnetic field (which are conceptually just forces) as the fundamental physical entities, one introduces the \emph{vector potential}, a one-form, consisting of the scalar electric potential and the vector potential associated with the magnet field. Experiments like the Aharonov-Bohm experiment allow for an interpretation of the vector potential as the fundamental physical object, rather than the associated electromagnetic field. \\
Even more fundamentally, the two main theories of physics, General Relativity and the Standard Model of particle physics, are field theories. They are deeply connected to a geometric interpretation and can be elegantly described using differential forms. \par
%
%
Despite of all this, differential forms are usually not part of the standard curriculum of physicists. We shall therefore introduce the basic aspects and definitions regarding differential forms that are used in this thesis. For a more detailed introduction we refer to the literature: For example \cite[Chapter 2 and 4]{rudolph_schmidt} or \cite[Appendix B]{wald_GR} provide introductions to the topic.\par
%
%
In the following, let $\N$ denote a smooth $N$-dimensional manifold, assumed to be Hausdorff, connected, oriented and para-compact, with either Lorentzian or Riemannian metric $k$ and Levi-Civita connection $\nabla$. For a Lorentzian manifold we use the sign convention $(-,+,\dots,+)$ of the metric $k$. The number of negative eigenvalues of $k$ is denoted by $s$, so $s=0$ for a Riemannian manifold and, in our convention, $s=1$ for a Lorentzian manifold.
Later, we will specify to a four dimensional (globally hyperbolic) spacetime consisting of a four dimensional manifold $\M$ with Lorentzian metric $g$ and Cauchy surface $\Sigma$ with induced Riemannian metric $h$.
%
We define:
\begin{definition}[Differential form]
	Let $p\in \{0,1,\dots,N\}$. A \emph{differential form} $\omega$ of degree $p$, or $p$-form for short, on the manifold $\N$ is an anti-symmetric tensor field of rank $(0,p)$. That is, at every point $x \in \N$, $\omega_x$ is an anti-symmetric multi-linear map
	\begin{align}
	\omega_x : \underbrace{T_x \N \times T_x \N \times \cdots \times T_x \N}_{p\text{-times}} \to \IR \formspace.
	\end{align}
	We denote the vector space\footnote{Naturally, addition and scalar multiplication are defined point-wise.} of $p$-forms on $\N$ by $\gls{omegap}$, the space with compactly supported ones by \gls{omegapz}.
\end{definition}
As an example, a zero-form $f \in \Omega^0(\N)$ is just a $C^\infty$-function from $\N$ to $\IR$, hence we can identify $\Omega^0(\N) = C^\infty (\N, \IR)$. A one-form $A \in \Omega^1(\N)$ is nothing more than a co-vector field and in a physical context usually denoted in local coordinates by $A_\mu$. Note, that alternatively one can directly define a $p$-form as a smooth section of the $p$-th exterior product of the co-tangent bundle and hence identify $\Omega^p(\N) = \Gamma \big( \largewedge^k T^*\N\big)$. As mentioned in Section \ref{sec:spacetime_geometry}, we view the tangent bundle as a complex bundle. Therefore, the sections of that bundle will be complex valued functionals. In that fashion, we will usually view the spaces $\Omega^p(\N)$ as complex valued differential forms.\par
%
Next we define the basic operations, besides addition and scalar multiplication, that one can perform on differential forms.
%
\begin{definition}[Exterior product]
	Let $A \in \Omega^p(\N)$ be a $p$-form and  $B\in \Omega^q(\N)$ a $q$-form on $\N$. \\
	The \emph{exterior product} $\gls{wedge}:\Omega^p(\N) \times \Omega^q(\N) \to \Omega^{p+q} (\N)$ is defined by
	\begin{align}
	(A \wedge B)_{\mu_1\dots\mu_p \nu_1\dots\nu_q} = \frac{(p+q)!}{p!q!}\, A_{[\mu_1 \dots \mu_p} B_{\nu_1\dots\nu_q]} \formspace,
	\end{align}
	where the anti-symmetrization of a tensor $T$ is given through
	\begin{align}
	T_{[\mu_1\dots\mu_p]} = \frac{1}{p!} \sum\limits_{\sigma\in S_N }\textrm{sgn}(\sigma) T_{\sigma(\mu_1)\dots\sigma(\mu_p)} \formspace.
	\end{align}
\end{definition}
Here, $S_N$ denotes the symmetric group\footnote{Usually the symmetric group is defined as the set of permutations of $\{1,2,\dots,N\}$ but we chose the index to run over $\{0,1,\dots,N-1\}$, identifying the time component with zero rather then one.} of degree $N$, consisting of permutations of the set $\{0,1,\dots,N-1\}$.
With this notion of multiplication, point-wise addition and scalar multiplication, the space $\gls{omega} \coloneqq \bigoplus_{p = 0}^\infty \Omega^p(\N) = \bigoplus_{p = 0}^N \Omega^p(\N)$ becomes an algebra, usually called the Grassmann- or \emph{exterior algebra} of differential forms on $\N$. We have used that obviously $\Omega^k(\N) =0$ for $k >N$ due to the anti-symmetrization.\par
Furthermore, we find a notion of how to \emph{pullback} differential forms on manifolds to another manifold, for example the pullback of a differential form on the spacetime $\M$ to differential forms on its Cauchy surface $\Sigma$. Given a $C^\infty$-map $\psi: \widetilde{\N} \to \N$, where $\N, \widetilde{\N}$ are manifolds, we can naturally define the pullback of a function $f \in \Omega^0(\N)$ to a function $(\psi^* f) \in \Omega^0(\widetilde{\N})$ by composing $f$ with $\psi$:
\begin{align}
\psi^* f \coloneqq f \comp \psi \formspace.
\end{align}
\newpage
With the pullback of functions defined, we can define how to \emph{push forward}, or carry along, vector fields on $\widetilde{\N}$ to vector fields on $\N$: Let $f\in \Omega^0(\N)$ and $\tilde{v} \in \Gamma(T\widetilde{\N})$ and $\tilde{x} \in \widetilde{\N}$. Then
\begin{align}
(\psi_* \tilde{v})_{\psi(\tilde{x})} (f) \coloneqq \tilde{v}_{\tilde{x}}(\psi^* f)
\end{align}
defines the vector field $(\psi_* v) \in \Gamma(T\N)$. With these basic operations at hand, we can generalize to define the pullback of differential forms:
\begin{definition}[Pullback]\label{def:pullback}
	Let $\N, \widetilde{\N}$ be manifolds of dimension $N,\widetilde{N}$ respectively, and let $\psi: \widetilde{\N} \to \N$ be a smooth map. Then, $\psi$ defines an algebra homomorphism $\psi^* : \Omega(\N) \to  \Omega(\widetilde{\N})$,
	called the \emph{pullback} of differential forms. For $\omega \in \Omega^p(\N)$, $\tilde{x} \in \widetilde{\N}$ and $\tilde{v}_i \in T_x \widetilde{\N}$, $i=1,2,\dots,p$, it is defined by
	\begin{align}
	\left( \psi^* \omega \right)_{\tilde{x}}  (\tilde{v}_1,\tilde{v}_2,\dots,\tilde{v}_p) \coloneqq \omega_{\psi(\tilde{x})} (\psi_* \tilde{v}_1, \dots , \psi_* \tilde{v}_p) \formspace.
	\end{align}
\end{definition}
%
%
%
%
On the exterior algebra we find a duality, provided by the Hodge operator:
\begin{definition}[Hodge dual]
	The hodge star operator $\gls{hodge}: \Omega^p(\N) \to \Omega^{N-p}(\N)$ is defined through
	\begin{align}
	B \wedge *A = \frac{1}{p!} B^{\mu_1\dots\mu_p}A_{\mu_1\dots\mu_p} \dvolk \formspace,
	\end{align}
	which yields the coordinate representation
	\begin{align}
	(*A)_{\mu_{p+1}\dots\mu_N} = \frac{\detk}{p!} \, \epsilon_{\mu_1\dots\mu_N} A^{\mu_1\dots\mu_p} \formspace.
	\end{align}
\end{definition}
Here, \gls{levicivita} denotes the fully antisymmetric tensor of rank $N$ (Levi-Civita symbol) satisfying $\epsilon_{12,\dots,N} =1$ and the \emph{volume element} \gls{dvolk} is defined by
\begin{align}
\left( \gls{dvolk} \right)_{\alpha_1\dots\alpha_N} = \detk \, \epsilon_{\alpha_1\dots\alpha_N} \formspace.
\end{align}
In a sense, the volume element describes how the curvature of the manifold deforms a unit volume.
The duality follows from the important property of the Hodge operator as stated in the following lemma:
\begin{lemma}
	Let $*$ denote the Hodge star operator on the exterior algebra $\Omega(\N) $. It holds that
	\begin{align}
	** = (-1)^{s+p(N-p)} \, \mathbbm{1} \formspace,
	\end{align}
	which is trivially equivalent to $*^{-1} = (-1)^{s+p(N-p)} \, *$.
\end{lemma}
\begin{proof}
	Let $A \in \Omega^p(\N)$ be a $p$-form on $\N$. Then:
	\begin{align}
	(*{*A})_{\mu_1 \dots \mu_p}
	&= \frac{\detk \, \detk}{p! \, (N-p)!} \; \epsilon_{\alpha_{p+1}\dots\alpha_N \mu_1 \dots \mu_p}\;\epsilon^{\alpha_{1}\dots\alpha_N}\;A_{\alpha_1\dots\alpha_p} \notag\\
	&= (-1)^{p(N-p)} \frac{\detk \, \detk}{p! \, (N-p)!} \; \epsilon_{\alpha_{p+1}\dots\alpha_N \mu_1 \dots \mu_p}\;\epsilon^{\alpha_{p+1}\dots\alpha_{N}\alpha_1\dots\alpha_p}\;A_{\alpha_1\dots\alpha_p}  \notag\\
	&= (-1)^{s+p(N-p)} \delta\indices{^{[\alpha_{1}}_{\mu_{1}}}\, \dots \, \delta\indices{^{\alpha_p ] }_{\mu_p}} \;A_{\alpha_1\dots\alpha_p} \notag\\
	&=  (-1)^{s+p(N-p)}\;A_{\mu_1\dots\mu_p} \formspace
	\end{align}
	We have used Lemma \ref{lem:epsilon_contraction} and, in the last step, that the anti-symmetrization is absorbed by contraction because $A$ is antisymmetric.
\end{proof}
%
%
%
%
%
Furthermore, we can equip the exterior algebra with a differentiable structure, introducing the notion of the exterior derivative.
\begin{definition}[Exterior derivative]
	The \emph{exterior derivative} $\gls{d}:\Omega^p(\N) \to \Omega^{p+1} (\N)$ is defined by the following properties:
	\begin{enumerate}
		\item $d$ is linear
		\item $d$ obeys a graded Leibniz rule: Let $A \in \Omega^p(\N)$ and  $B\in \Omega^q(\N)$, then
		\begin{align}
		d(A \wedge B) = dA \wedge B + (-1)^p \, A \wedge dB
		\end{align}
		\item $d$ is nilpotent, that is,  $d^2 = 0$.
	\end{enumerate}
	In local coordinates, this is equivalent to the representation
	\begin{align}
	(dA)_{\mu \alpha_1\dots\alpha_p} = (p+1)\, \nabla_{[\mu}A_{\alpha_1\dots\alpha_p]} \formspace.
	\end{align}
\end{definition}
An important property of the exterior derivative is that it commutes (or rather intertwines its action) with pullbacks (see \cite[Proposition 4.1.7]{rudolph_schmidt}).
A $p$-form $\omega \in \Omega^p(\N)$ is called \emph{exact} if there is a $(p-1)$-form $\alpha \in \Omega^{p-1}(\N)$ such that $\omega = d\alpha$. We call $\omega$ \emph{closed} if $d \omega =0$. Accordingly, the space of closed $p$-forms is denoted by \gls{omegapd}, the space of exact ones by \gls{domegap}. As usual, the ones with compact support are denoted by a subscript zero. Note, that every exact form is closed, using that $d$ is by definition nilpotent, but the reverse is in general not true. It does hold, however, on certain manifolds with trivial topology, such as Minkowski spacetime. This is expressed in the so called Poincar\'e-Lemma (see for example \cite[Chapter 4]{bott_tu}) based on the study of de Rham cohomology.\par
%
Moreover, $N$-forms can naturally be integrated. Using local coordinates and a partition of unity, we define the integral of $N$-forms via the well known integration on $\IR^N$:
\begin{definition}[Integration on manifolds]
	Let $\left\{U_\alpha, \psi_\alpha\right\}_\alpha$ be an atlas of the manifold $\N$ and $\left\{\chi_\alpha\right\}_\alpha$ a partition of unity subordinate to the locally finite open cover $\left\{U_\alpha\right\}_\alpha$. Let $x^\mu_{(\alpha)}$ be a coordinate basis of $\psi$ on $U_\alpha$. For any $N$-form $\omega \in \Omega^N_0(\M)$ we define the integral
	\begin{align}
	\int\limits_{\N} \omega &\coloneqq \sum_{\alpha} \int\limits_{\psi_\alpha (U_\alpha)} w(x_{(\alpha)}^0,\dots,x_{(\alpha)}^1)\; dx_{(\alpha)}^0 \cdots dx_{(\alpha)}^{N-1} \formspace,
	\end{align}
	where $w$ are the components of $\omega$ in the coordinates $x_{(\alpha)}^\mu$, that is $\omega = w dx_{(\alpha)}^0 \wedge \cdots \wedge dx_{(\alpha)}^{N-1}$.
	This definition is independent of the choice of the atlas and the partition of unity (see \cite[Proposition 3.3]{bott_tu}).
\end{definition}
With integration at our disposal, we present an important theorem regarding the integration of exact differential forms:
\begin{theorem}[Stoke's Theorem]\label{thm:stokes}
	Let $\N$ be an oriented manifold of dimension $N$ and let its boundary $\partial \N$ be endowed with the induced orientation. Let $\gls{inclusionmap} : \partial \N \hookrightarrow \N$ be the inclusion operator.
	Let $\omega \in \Omega^{N-1}_0(\N)$ be a compactly supported $(N-1)$-form on $\N$. Then it holds
	\begin{align}
	\int\limits_\N d\omega = \int\limits_{\partial \N} i^*\omega \formspace.
	\end{align}
\end{theorem}
\begin{proof}
	A proof is given in most of the introductory literature on differential geometry (see for example \cite[Chapter 17, Theorem 2.1]{lang}).
	Note that one can equivalently formulate Stoke's theorem on a \emph{compact} manifold but for {arbitrary} (that is, in general not compactly supported) $(N-1)$-forms on the manifold (see for example \cite[Theorem 4.2.14]{rudolph_schmidt}). This will be of importance in later calculations.
\end{proof}
%
Furthermore, we can define a bilinear map on $\Omega^p(\N)$ using the integration of $N$-forms:
\begin{definition}
	Let $A,B \in \Omega^p(\N)$ such that their supports have a compact intersection. Define the bilinear map $\gls{innerprod} : \Omega^p(\N) \times \Omega^p(\N) \to \IC$ by
	\begin{align}
	\langle A, B \rangle_\N \coloneqq  \int_{\N } A \wedge * B = \int_{\N } A_{\mu_1 \dots \mu_p}B^{\mu_1 \dots \mu_p}\,\dvolk \formspace.
	\end{align}
\end{definition}
Since by definition $A \wedge * B$ is a compactly supported $N$-form, this is well defined. We may sometimes refer to $\langle \cdot , \cdot \rangle_\N$ as an inner product for simplicity, even though it is not positive definite.
%
%
%
%
%
Using the exterior derivative, we define the interior or co-derivative:
\begin{definition}[Interior derivative]
	The \emph{interior derivative} $\gls{delta} : \Omega^p(\N) \to \Omega^{p-1}(\N)$ is defined by
	\begin{align}
	\delta \coloneqq (-1)^{s+1+N(p-1)}\, {*{d*}} \formspace.
	\end{align}
	From the defining properties of $d$ and $*$ it follows $\delta^2 =0$.
\end{definition}
Here, $s$ again denotes the number of negative eigenvalues of the metric $k$ of $\N$. In accordance with our nomenclature, we call a $p$-form $\omega$ co-exact if there exists a $\alpha \in \Omega^{p+1}(\N)$ such that $\omega = \delta \alpha$ and co-closed if $\delta \omega = 0$. Accordingly, the spaces of co-closed and co-exact $p$-forms are denoted by \gls{omegapdelta} and \gls{deltaomegap} respectively.\par
Using the exterior and interior derivative we define the partial differential operator:
\begin{definition}[D'Alembert Operator]
	The d'Alembert (or Laplace - de Rham) operator $\gls{dalembert}: \Omega^p(\N) \to \Omega^{p}(\N)$ is defined by
	\begin{align}
	\square \coloneqq \delta d +d \delta \formspace.
	\end{align}
\end{definition}
By definition of the exterior and interior derivative, it is easy to show that $\square$ commutes with both $d$ and $\delta$:
\begin{align}
\square d &= (\delta d + d \delta )d \notag \\
&= d \delta d \notag \\
&= d (\delta d + d \delta) \formspace,
\end{align}
and analogously for $\delta$.
The d'Alembert operator, and its generalization to $(\square + m^2)$ for some constant $m > 0$, are important examples for a normally hyperbolic differential operators (see Section \ref{sec:global_hyperbolicity}) and we may therefore sometimes just refer to them as \emph{wave operators}.\par
The sign convention in the definition of the exterior derivative is chosen such that on any Lorentzian or Riemannian manifold the interior derivative is formally adjoint to the exterior derivative, that is,  for $A \in \Omega^{p}(\N)$ and $B \in \Omega^{p+1}(\N)$ it holds that
\begin{align}
\langle dA , B \rangle_{\N} = \langle A , \delta B \rangle_\N \formspace,
\end{align}
which leads to a representation in local coordinates of the Manifold given by:
\begin{align}
(\delta A)_{\mu_2\dots\mu_p} = - \nabla^{\mu_1}A_{\mu_1\dots\mu_p} \formspace.
\end{align}
To see that this is consistent, let $A \in \Omega^{p-1}(\N)$ and $B \in \Omega^{p}(\N)$ such that their supports have compact intersection.
We obtain, using Stoke's Theorem \ref{thm:stokes}:
\begin{align}
0 &= \int \limits_{\partial \N} i^* (A \wedge *B) \notag\\
&= \int \limits_{\N} d(A \wedge *B)  \notag\\
&= \int \limits_{\N} dA \wedge *B + (-1)^{p-1} A \wedge d{*B} \notag\\
&= \int \limits_{\N} dA \wedge *B + (-1)^{p-1} A \wedge *{*^{-1}}\underbrace{d{*B}}_{\textrm{is a } (N-p+1) \textrm{ form.}} \notag\\
&= \int \limits_{\N} dA \wedge *B + (-1)^{p-1}(-1)^{s+(N-p+1)(N-N+p-1)} A \wedge *{*d{*B}} \notag\\
&= \int \limits_{\N} dA \wedge *B + (-1)^{p+(1-p)(p-1)} A \wedge *\delta B \formspace.
\end{align}
It can easily be proven by induction that $\big(p+(1-p)(p-1)\big)$ is odd for any $p \in \IN$, which yields the result
\begin{align}
\langle dA , B \rangle_{\N} = \langle A , \delta B \rangle_\N \formspace.
\end{align}
The definitions stated above thus fulfill the requirement of formal adjointness of the exterior and interior derivate on an arbitrary Lorentzian or Riemannian manifold $\N$.
In local coordinates we use a partial integration to obtain
\begin{align}
\langle dA , B \rangle_\N &= \int \limits_{\N} dA \wedge * B \notag\\
%&= \int \limits_{\N} \frac{1}{p!} (dA)^{\alpha_1\dots\alpha_p}\,B_{\alpha_1 \dots \alpha_p} \, \dvolk \notag\\
&= \int \limits_{\N}  \frac{p}{p!} \nabla^{[\alpha_1}A^{\alpha_2\dots\alpha_p]}\,B_{\alpha_1 \dots \alpha_p} \, \dvolk \notag\\
&= \int \limits_{\N}  \frac{1}{(p-1)!} \nabla^{\alpha_1}A^{\alpha_2\dots\alpha_p}\,B_{\alpha_1 \dots \alpha_p} \, \dvolk \notag\\
&= - \int \limits_{\N}  \frac{1}{(p-1)!} A^{\alpha_2\dots\alpha_p}\, \nabla^{\alpha_1}B_{\alpha_1 \dots \alpha_p} \, \dvolk \notag\\
&= \langle A, \delta B \rangle_\N \formspace,
\end{align}
which yields
\begin{align}
-\nabla^{\alpha_1}B_{\alpha_1 \dots \alpha p} = (\delta B)_{\alpha_2 \dots \alpha_p}\formspace.
\end{align}
On the four dimensional spacetime $(\M,g)$ the definitions of the Hodge star operator and the interior derivative simplify, such that
\begin{align}
*_{(\M)}*_{(\M)} &= (-1)^{p+1} \mathbbm{1} \\
\delta_{(\M)} &= *_{(\M)}{d_{(\M)}*_{(\M)}} \formspace ,
\end{align}
holds on the spacetime $(\M,g)$ and
\begin{align}
*_{(\Sigma)}*_{(\Sigma)} &= \mathbbm{1} \\
\delta_{(\Sigma)} &= (-1)^p *_{(\Sigma)}{d_{(\Sigma)}*_{(\Sigma)}}
\end{align}
holds on  $(\Sigma,h)$. In the following we will drop the subscript ${(\M)}$, since we will perform all the calculations on a four dimensional spacetime, except when explicitly noted (for example with a subscript $(\Sigma)$).
%
%
%
%
%
%
%
%
%%%%%%
%%CATEGORY THEORY
%%%%%%
\subsection{Category theory}\label{sec:cat-theory}
The description of Quantum Field Theory on Curved Spacetimes (QFTCS) in the framework of \name{Brunetti}, \name{Fredenhagen} and \name{Verch} \cite{Brunetti_Fredenhagen_Verch} is based on category theory. In this thesis, we will not go into detail on those categorical aspects, however we will need some basic definitions to formulate the theory rigorously, that is namely the notion of a category and that of covariant functors, since, in the used framework, the generally covariant QFTCS is a functor.\par
Here, we present definitions given in \cite[Appendix A.1]{baer_ginoux_pfaeffle} and refer to the appropriate literature for details. We define:
\begin{definition}[Category]
	A \emph{category} $\mathsf{Cat}$ consists of the following:
	\begin{enumerate}
		\item a class $\mathsf{Obj}_\mathsf{Cat}$ whose members are called \emph{objects},
		\item a set $\mathsf{Mor}_\mathsf{Cat}(A,B)$, for any two objects $A,B \in \mathsf{Obj}_\mathsf{Cat}$, whose elements are called \emph{morphisms},
		\item for any three objects $A,B,C \in \mathsf{Obj}_\mathsf{Cat}$ there is a map
		\begin{align}
\mathsf{Mor}_\mathsf{Cat}(B,C) \times \mathsf{Mor}_\mathsf{Cat}(A,B) &\to \mathsf{Mor}_\mathsf{Cat}(A,C) \notag\\
(\psi,\phi) &\mapsto \psi \comp \phi
		\end{align}
		called the composition of morphisms subject to the relations:\vspace{4mm}
		\begin{enumerate}[label=(\arabic*)]
			\item for non equal pairs $(A,B)$, $(A',B')$ of objects, the sets $\mathsf{Mor}_\mathsf{Cat}(A,B)$ and $\mathsf{Mor}_\mathsf{Cat}(A',B')$ are disjoint,
			\item for every object $A$ there exists a morphism $\text{id}_A \in \mathsf{Mor}_\mathsf{Cat}(A,A)$ such that it holds for all objects $B$, morphisms $\psi \in \mathsf{Mor}_\mathsf{Cat}(B,A)$ and $\phi \in \mathsf{Mor}_\mathsf{Cat}(A,B)$
			\begin{align}
				\text{id}_A \comp \psi &= \psi \quad \text{and}\\
				\phi \comp \text{id}_A &= \phi \quad,
			\end{align}
			\item the composition law is associative, that is for an objects $A,B,C,D$ and any morphisms $\psi \in \mathsf{Mor}_\mathsf{Cat}(A,B)$, $\phi \in \mathsf{Mor}_\mathsf{Cat}(B,C)$ and $\chi \in \mathsf{Mor}_\mathsf{Cat}(C,D)$ it holds
			\begin{align}
				(\chi \comp \phi) \comp \psi = \chi \comp (\phi \comp \psi) \formspace.
			\end{align}
		\end{enumerate}
	\end{enumerate}
\end{definition}
%
%
%
\begin{definition}[Functor]
	Let $\mathsf{Cat1}$ and $\mathsf{Cat2}$ be categories. A \emph{covariant functor} $\mathscr{A}: \mathsf{Cat1} \to \mathsf{Cat2}$ consists of the map $\mathscr{A} : \mathsf{Obj}_\mathsf{Cat1} \to \mathsf{Obj}_\mathsf{Cat2}$ and maps $\mathscr{A}: \mathsf{Mor}_\mathsf{Cat1}(A,B) \to \mathsf{Mor}_\mathsf{Cat2}\big(\mathscr{A}(A),\mathscr{A}(B)\big)$ for any two objects $A,B \in \mathsf{Obj}_\mathsf{Cat1}$ such that
	\begin{enumerate}
		\item {the composition is preserved, that is for all objects $A,B,C \in \mathsf{Obj}_\mathsf{Cat1}$ and for any morphisms $\psi \in \mathsf{Mor}_\mathsf{Cat1}(A,B)$ and $\phi \in \mathsf{Mor}_\mathsf{Cat1}(B,C)$ it holds
		\begin{align}
			\mathscr{A}(\phi \comp \psi) = \mathscr{A}(\phi) \comp \mathscr{A}(\psi) \formspace,
		\end{align}}
		\item{
			$\mathscr{A}$ maps identities to identities, that is for any object $A \in \mathsf{Obj}_\mathsf{Cat1}$ it holds
			\begin{align}
				\mathscr{A}(\text{id}_\mathsf{A}) = \text{id}_{\mathscr{A}(A)} \formspace.
			\end{align}
			}
	\end{enumerate}
\end{definition}
%
%
%
%
%
%
%
%
%
%
%
%
%%%%%%
%%SIGN CONVENTIONS
%%%%%%
%
%
\subsection{Sign conventions}\label{sec:sign_conventions}
At certain points throughout this chapter we have had a freedom of choice regarding the signs of some entities, in particular the sign of the signature of the Lorentzian metric $g$ and that of the interior derivative $\delta$. Though at this stage the choice can be made arbitrarily, we want to make it in a way that in the end allows us to make certain physical interpretations on some parameters. More precisely, we want to interpret the parameter $m$ of the Klein-Gordon equation\footnote{or its generalization on $p$-forms} $(\square + m^2) f = 0$ for a zero-form $f \in \Omega^0(\M)$ as a mass in the physical sense. With the chosen sign convention for $\delta$ we find, using ${\delta}f = 0$:
\begin{align}
	\square f
	&= (\delta d + d \delta) f \notag\\
	&= \delta d f \notag\\
	&= - \nabla^\mu \nabla_\mu f \formspace.
\end{align}
In the following heuristic (local) argument we see
\begin{align}
	\square + m^2
	&= -\nabla^\mu \nabla_\mu + m^2 \notag\\
	&\sim \partial_t^2 + \sum_i \partial_i^2 + m^2\notag\\
	&\sim -E^2 + \abs{\vector{p}}^2 + m^2
\end{align}
which yields the correct relativistic relation of energy, momentum and mass according to $E^2 = \abs{\vector{p}}^2 + m^2$.
A similar calculation holds for the Klein-Gordon operator generalized to act on one-forms. If we had found a ``wrong'' relation between energy, momentum and mass, we would have had to adapt the chosen signs. Usually one chooses the sign of the metric and the interior derivative such that they are in some sense mathematically convenient (although one might disagree with another one's choice). We have made the choice of the metric, such that the Cauchy surfaces become Riemannian rather that ``anti-Riemannian'' (with an all minus signature), which seems more natural to some. Also, a lot of the used references on spacetime geometry (in particular the book by \name{Wald} \cite{wald_GR}) use this sign convention, which makes the application of certain formulas easier. As mentioned, the sign of the interior derivative was chosen such that it is formally adjoint to the exterior derivative (with respect the specified inner product) on all Lorentzian and Riemannian manifolds. It seemed convenient for the actual calculations to fix the sign regardless of the signature of the metric of the underlying manifold. One could equivalently have fixed the opposite sign, yielding the two derivatives to be skew-adjoint, which is also done in the literature. However, in the end, one has one freedom left to make the energy-momentum-mass relation work: that is the sign in front of the mass in the Klein-Gordon equation and all other wave equations accordingly. Hence, one regularly also finds the Klein-Gordon equation to be defined with a flipped sign of the mass term. But for our case, we want the mass $m$ in any wave equation to appear with a positive sign.
%
%


%removed preliminaries
%\medskip
%We provide definitions and background results on
%linear algebra and privacy in Appendix \ref{sec:preliminaries}.

\section{Exact case}

Here, we discuss the case, where all $n$ points lie \emph{exactly} in a subspace $s_*$ of dimension $k$ of $\RR^d$. Our goal
is to privately output that subspace. We do it under the
assumption that all strict subspaces of $s_*$ contain at most $\ell$
points.
If the points are in general position, then $\ell=k-1$, as any strictly smaller subspace has dimension $<k$ and cannot contain more points than its dimension.
Let $\mathcal{S}_d^k$ be the set of all $k$-dimensional
subspaces of $\mathbb{R}^d$. Let $\mathcal{S}_d$ be the
set of all subspaces of $\mathbb{R}^d$. We formally define
that problem as follows.

\begin{problem}\label{prob:exact}
    Assume (i) all but at most $\ell$, input points are in some
    $s_* \in \mathcal{S}_d^k$, and (ii)  every subspace
    of dimension $<k$ contains at most $\ell$ points. (If the points
    are in general position -- aside from being contained in
    $s_*$ -- then $\ell=k-1$.) The goal is to output a representation
    of $s_*$.
\end{problem}

We call these $\leq \ell$ points that do not lie in
$s_*$, ``adversarial points''.
With the problem defined in Problem~\ref{prob:exact}, we
will state the main theorem of this section.

\begin{theorem}\label{thm:exact}
    For any $\eps,\delta>0$, $\ell \ge k-1 \ge 0$, and
    $$n \geq O\left(\ell + \frac{\log(1/\delta)}{\eps}\right),$$ there
    exists an $(\eps,\delta)$-DP algorithm
    $M : \mathbb{R}^{d \times n} \to \mathcal{S}_d^k$, such that if
    $X$ is a dataset of $n$ points satisfying the conditions
    in Problem~\ref{prob:exact},
    then $M(X)$ outputs a representation of $s_*$ with probability $1$.
\end{theorem}

We prove Theorem \ref{thm:exact} by proving the privacy and
the accuracy guarantees of Algorithm~\ref{alg:exact}.
The algorithm performs a $\GAPMAX$ (cf.~Lemma~\ref{lem:gap-max}).
It assigns a score to all the relevant
subspaces, that is, the subspaces spanned by the points
of the dataset $X$. We show that the only subspace that
has a high score is the true subspace $s_*$, and the rest
of the subspaces have low scores. Then $\GAPMAX$ outputs
the true subspace successfully because of the gap between
the scores of the best subspace and the second to the best
one. For $\GAPMAX$ to work all the time, we define a default
option in the output space that has a high score, which we
call $\NULL$. Thus, the output space is now
$\cY = \cS_d \cup \{\NULL\}$. Also, for $\GAPMAX$ to run in
finite time, we filter $\cS_d$ to select finite number of subspaces
that have at least $0$ scores on the basis of $X$. Note that
this is a preprocessing step, and does not violate privacy as,
we will show, all other subspaces already have $0$ probability
of getting output.
We define the score function
$u : \mathcal{X}^n \times \cY \to \mathbb{N}$
as follows.
\[ u(x,s) :=
\begin{cases}
    |x \cap s| - \sup \{ |x \cap t| : t \in \mathcal{S}_d, t \subsetneq s \} &
        \text{if $s \in \cS_d$}\\
    \ell + \frac{4\log(1/\delta)}{\eps} + 1 & \text{if $s = \NULL$}
\end{cases}
\]
Note that this score function can be computed in finite
time because for any $m$ points and $i>0$, if the points
are contained in an $i$-dimensional subspace, then the
subspace that contains all $m$ points must lie within
the set of subspaces spanned by ${m \choose i+1}$ subsets
of points.

\begin{algorithm}[h!] \label{alg:exact}
\caption{DP Exact Subspace Estimator
    $\DPESE_{\eps, \delta, k, \ell}(X)$}
\KwIn{Samples $X \in \R^{d \times n}$.
    Parameters $\eps, \delta, k, \ell > 0$.}
\KwOut{$\hat{s} \in \cS_d^k$.}
\vspace{5pt}

Set $\cY \gets \{\NULL\}$ and sample noise $\xi(\NULL)$ from $\TLap(2,\eps,\delta)$.\\
Set score $u(X,\NULL) = \ell + \frac{4\log(1/\delta)}{\eps} + 1$.
\vspace{5pt}

\tcp{Identify candidate outputs.}
\For{each subset $S$ of $X$ of size $k$}{
    Let $s$ be the subspace spanned by $S$.\\
    $\cY \gets \cY \cup \{s\}$.\\
    Sample noise $\xi(s)$ from $\TLap(2,\eps,\delta)$.\\
    Set score $u(X,s) = |x \cap s| - \sup \{ |x \cap t| : t \in \mathcal{S}_d, t \subsetneq s \} $.
}
\vspace{5pt}

\tcp{Apply $\GAPMAX$.}
Let $s_1 = \argmax_{s \in \cY} u(X,s)$ be the candidate with the largest score.\\
Let $s_2 = \argmax_{s \in \cY \setminus \{s_1\}} u(X,s)$ be the candidate with the second-largest score.\\
Let $\hat s = \argmax_{s \in \cY} \max\{ 0 , u(X,s) - u(X,s_2) -1\} + \xi(s)$.\\
\tcp{Truncated Laplace noise $\xi \sim \TLap(2,\eps,\delta)$; see Lemma \ref{lem:truncated-laplace}}

\vspace{5pt}
\Return $\hat{s}.$
\vspace{5pt}
\end{algorithm}

We split the proof of Theorem~\ref{thm:intro-main-exact} into sections
for privacy (Lemma~\ref{lem:exact-privacy}) and accuracy (Lemma~\ref{lem:exact-accuracy}).

\subsection{Privacy}

\begin{lemma}\label{lem:exact-privacy}
    Algorithm~\ref{alg:exact} is $(\eps,\delta)$-differentially
    private.
\end{lemma}
The proof of Lemma \ref{lem:exact-privacy} closely follows the
privacy analysis of $\GAPMAX$ by \cite{BunDRS18}. The only novelty
is that Algorithm \ref{alg:exact} may output $\NULL$ in the case
that the input is malformed (i.e., doesn't satisfy the assumptions
of Problem~\ref{prob:exact}).

The key is that the score $u(X,s)$ is low sensitivity. Thus
$\max\{ 0 , u(X,s) - u(X,s_2) -1\}$ also has low sensitivity.
What we gain from subtracting the second-largest score and
taking this maximum is that these values are also sparse -- only
one ($s=s_1$) is nonzero. This means we can add noise to all
the values without paying for composition. We now prove
Lemma~\ref{lem:exact-privacy}.

\begin{proof}
    First, we argue that the sensitivity of $u$ is
    $1$. %Consider any $s \in \cS_d$, and neighbouring datasets $X,X'$.
    The quantity $\abs{X \cap s}$ has sensitivity $1$ and so does
    $\sup \{ |X \cap t| : t \in \mathcal{S}_d, t \subsetneq s \}$.
    This implies sensitivity $2$ by the triangle inequality.
    However, we see that it is not possible to change one point
    that simultaneously increases $\abs{X \cap s}$ and decreases
    $\sup \{ |X \cap t| : t \in \mathcal{S}_d, t \subsetneq s \}$
    or vice versa. Thus the sensitivity is actually $1$.

    We also argue that $u(X,s_2)$ has sensitivity $1$, where $s_2$
    is the candidate with the second-largest score.
    Observe that the second-largest score is a monotone function of
    the collection of all scores -- i.e., increasing scores cannot
    decrease the second-largest score and vice versa.
    Changing one input point can at most increase all the scores by
    $1$, which would only increase the second-largest score by $1$.

    This implies that $\max\{ 0, u(X,s) - u(X,s_2) -1 \}$ has sensitivity
    $2$ by the triangle inequality and the fact that the maximum does
    not increase the sensitivity.

    Now we observe that for any input $X$ there is at most one $s$ such
    that $\max\{ 0, u(X,s) - u(X,s_2) -1 \} \ne 0$, namely $s=s_1$.
    We can say something even stronger: Let $X$ and $X'$ be neighbouring
    datasets with $s_1$ and $s_2$ the largest and second-largest scores
    on $X$ and $s_1'$ and $s_2'$ the largest and second-largest scores
    on $X'$. Then there is at most one $s$ such that
    $\max\{ 0, u(X,s) - u(X,s_2) -1 \} \ne 0$ or $\max\{ 0, u(X',s) - u(X',s_2') -1 \} \ne 0$.
    In other words, we cannot have both $u(X,s_1) - u(X,s_2) >1$ and
    $u(X',s_1') - u(X',s_2') >1$ unless $s_1=s_1'$. This holds because
    $u(X,s) - u(X,s_2)$ has sensitivity $2$.

    With these observations in hand, we can delve into the privacy
    analysis. Let $X$ and $X'$ be neighbouring datasets with $s_1$
    and $s_2$ the largest and second-largest scores on $X$ and $s_1'$
    and $s_2'$ the largest and second-largest scores on $X'$. Let $\cY$
    be the set of candidates from $X$ and let $\cY'$ be the set of
    candidates from $X'$. Let $\check \cY = \cY \cup \cY'$ and
    $\hat \cY = \cY \cap \cY'$.

    We note that, for $s \in \check \cY$, if $u(X,s) \le \ell$, then
    there is no way that $\hat s = s$. This is because
    $|\xi(s)|\le \frac{2 \log(1/\delta)}{\varepsilon}$ for all $s$ and
    hence, there is no way we could have
    $\argmax_{s \in \cY} \max\{ 0 , u(X,s) - u(X,s_2) -1\} +
    \xi(s) \ge \argmax_{s \in \cY} \max\{ 0 , u(X,\NULL) - u(X,s_2) -1\} + \xi(\NULL)$.

    If $s \in \check \cY \setminus \hat \cY$, then
    $u(X,s) \le |X \cap s| \le k+1 \le \ell$ and $u(X',s) \le \ell$.
    This is because $s \notin \hat \cY$ implies $|X \cap s| < k$ or
    $|X' \cap s| < k$, but $|X \cap s| \le |X' \cap s| +1$. Thus,
    there is no way these points are output and, hence, we can ignore
    these points in the privacy analysis. (This is the reason for adding
    the $\NULL$ candidate.)

    Now we argue that the entire collection of noisy values
    $\max\{ 0 , u(X,s) - u(X,s_2) -1\} + \xi(s)$ for $s \in \hat \cY$
    is differentially private.
    This is because we are adding noise to a vector where (i) on the
    neighbouring datasets only $1$ coordinate is potentially different
    and (ii) this coordinate has sensitivity $2$.
\end{proof}

\subsection{Accuracy}

We start by showing that the true subspace $s_*$ has a
high score, while the rest of the subspaces have low scores.

\begin{lemma}\label{lem:scores}
    Under the assumptions of Problem~\ref{prob:exact}, $u(x,s_*) \ge n - 2\ell$
    and $u(x,s')\le 2\ell$ for $s' \ne s_*$.
\end{lemma}
\begin{proof}
    We have $u(x,s_*) = |x \cap s_*| - |x \cap s'|$ for some
    $s' \in \mathcal{S}_d$ with $s' \subsetneq s_*$. The
    dimension of $s'$ is at most $k-1$ and, by the assumption
    (ii), $|x \cap s'| \le \ell$. 
    
    Let $s' \in \mathcal{S}_d \setminus \{s_*\}$.
    There are three cases to analyse:
    \begin{enumerate}
        \item Let $s' \supsetneq s_*$. Then
            $u(x,s') \le |x \cap s'| - |x \cap s_*| \leq \ell$
            because the $\leq \ell$ adverserial points and the $\geq n-\ell$
            non-adversarial points may not together lie in a subspace
            of dimension $k$.

        \item Let $s' \subsetneq s_*$. Let
            $k'$ be the dimension of $s'$. Clearly $k'<k$.
            By our assumption (ii), $|s' \cap x| \le \ell$.
            Then $u(x,s') = |x \cap s'| - |x \cap t| \le \ell$
            for some $t$ because the $\leq \ell$ adversarial points
            already don't lie in $s_*$, so they will not lie in
            any subspace of $s_*$.

        \item Let $s'$ be incomparable to $s_*$.
            Let $s'' = s' \cap s_*$. Then
            $u(x,s') \le |x \cap s'| - |x \cap s''| \leq \ell$
            because the adversarial points may not lie in $s_*$,
            but could be in $s'\setminus s''$.
    \end{enumerate}
    This completes the proof.
\end{proof}

Now, we show that the algorithm is accurate.

\begin{lemma}\label{lem:exact-accuracy}
    If
    $n \geq 3\ell + \frac{8\log(1/\delta)}{\eps} + 2,$
    then Algorithm~\ref{alg:exact} outputs $s_*$ for Problem~\ref{prob:exact}.
\end{lemma}
\begin{proof}
    From Lemma~\ref{lem:scores}, we know that $s_*$ has
    a score of at least $n-2\ell$, and the next best subspace
    can have a score of at most $\ell$. Also, the score of
    $\NULL$ is defined to be $\ell + \tfrac{4\log(1/\delta)}{\eps} + 1$.
    This means that the gap satisfies $\max\{ 0 , u(X,s_*) - u(X,s_2) -1\} \ge n - 3\ell - \tfrac{4\log(1/\delta)}{\eps}  -1$.
    Since the noise is bounded by $\tfrac{2\log(1/\delta)}{\eps}$, our bound on $n$ implies that $\hat s = s_*$
\end{proof}

%\subsection{Putting It All Together}
%
%\begin{proof}[Proof of Theorem~\ref{thm:exact}]
%    By Lemmata~\ref{lem:exact-privacy} and \ref{lem:exact-accuracy},
%    the algorithm is differentially private and accurate.
%\end{proof}

\begin{comment}
    
    Key questions: Remove/weaken assumption (ii) general
    position. Approximate case (or discretization).
    
    \subsection{Generalizing beyond general position}
    
    Still assuming $x \subset s_*$ for some $s_* \in \mathcal{S}_d^k$. Also assume $0 \notin x$.
    
    Claim: There exists some $s \in \mathcal{S}_d$ with $u(x,s) \ge n/k$.
    \begin{proof}
    Let $s_k=s_*$. For $i=k,k-1,k-2,\cdots,1$, inductively choose $s_{i-1} \in \mathcal{S}_d$ such that $s_{i-1} \subsetneq s_i$ and $|x \cap s_{i-1}|$ is maximal. By construction, $u(x,s_i) = |x \cap s_i| - |x \cap s_{i-1}|$ for all $i \in [k]$. By assumption, $|x \cap s_k| = n$. Also, $\mathsf{dimension}(s_i) \le i$. Thus $|x \cap s_0| = 0$.
    Since $\sum_{i=1}^k u(x,s_i) = |x \cap s_k| - |x \cap s_0| = n$, we must have $u(x,s_i) \ge n/k$ for some $i \in [k]$.
    \end{proof}

\end{comment}

\subsection{Lower Bound}

Here, we show that our upper bound is optimal up to constants for
the exact case.

\begin{theorem}
     Any $(\eps,\delta)$-DP algorithm that takes a dataset of $n$ points satisfying the conditions
    in Problem~\ref{prob:exact} and outputs $s_*$ with probability $>0.5$ requires
    $n \geq \Omega\left(\ell + \frac{\log(1/\delta)}{\eps}\right).$
\end{theorem}
\begin{proof}
    First, $n \ge \ell + k$. This is because we need at least
    $k$ points to span the subspace, and $\ell$ points could be corrupted.
    Second, $n \ge \Omega(\log(1/\delta)/\varepsilon)$ by group
    privacy. Otherwise, the algorithm is $(10,0.1)$-differentially
    private with respect to changing the \emph{entire} dataset and
    it is clearly impossible to output the subspace under this condition.
\end{proof}


\section{Approximate Case}

In this section, we discuss the case, where the data
``approximately'' lies in a $k$-dimensional subspace of
$\R^d$. %An alternate perspective of this problem is that
%the data lies in a $k$-dimensional subspace of $\R^d$,
%but has very small noise in the orthogonal directions.
We make a Gaussian distributional assumption, where the
covariance is approximately $k$-dimensional, though the
results could be extended to distributions with heavier
tails using the right inequalities. We formally define
the problem:

\begin{problem}\label{prob:gaussians}
    Let $\Sigma \in \R^{d \times d}$ be a symmetric, PSD
    matrix of rank $\geq k \in \{1,\dots,d\}$, and let $0 < \gamma \ll 1$,
    such that $\tfrac{\lambda_{k+1}}{\lambda_k} \leq \gamma^2$.
    Suppose $\Pi$ is the projection matrix corresponding
    to the subspace spanned by the eigenvectors of $\Sigma$
    corresponding to the eigenvalues $\lambda_1,\dots,\lambda_k$.
    Given sample access to $\cN(\vec{0},\Sigma)$,
    and $0 < \alpha < 1$, output a projection matrix $\wh{\Pi}$,
    such that $\|\Pi-\wh{\Pi}\| \leq \alpha$.
\end{problem}

We solve Problem~\ref{prob:gaussians} under the constraint
of $(\eps,\delta)$-differential privacy. Throughout this section,
we would refer to the subspace spanned by the top $k$ eigenvectors
of $\Sigma$ as the ``true'' or ``actual'' subspace.

Algorithm \ref{alg:approximate} solves Problem~\ref{prob:gaussians} and proves Theorem \ref{thm:intro-main-approx}.
Here $\|\cdot\|$ is the operator norm.

\begin{remark}\label{rem:gamma}
    We scale the eigenvalues of $\Sigma$
    so that $\lambda_k=1$ and $\lambda_{k+1} \leq \gamma^2$.
    %We will be adopting this notation throughout this text.
    Also, for the purpose of the analysis, we will be splitting
    $\Sigma = \Sigma_k + \Sigma_{d-k}$, where $\Sigma_k$ is the
    covariance matrix formed by the top $k$ eigenvalues and
    the corresponding eigenvectors of $\Sigma$ and $\Sigma_{d-k}$
    is remainder.
\end{remark}

%\begin{comment}

Also, we assume the
knowledge of $\gamma$ (or an upper bound on $\gamma$). Our solution
is presented in Algorithm~\ref{alg:approximate}. The following
theorem is the main result of the section.

\begin{theorem}\label{thm:approximate}
    Let $\Sigma \in \R^{d \times d}$ be an arbitrary, symmetric, PSD
    matrix of rank $\geq k \in \{1,\dots,d\}$, and let $0 < \gamma < 1$.
    Suppose $\Pi$ is the projection matrix corresponding
    to the subspace spanned by the vectors of $\Sigma_k$.
    Then given
    $$\gamma^2 \in
        O\left(\frac{\eps\alpha^2n}{d^{2}k\ln(1/\delta)}\cdot
        \min\left\{\frac{1}{k},
        \frac{1}{\ln(k\ln(1/\delta)/\eps)}
        \right\}\right),$$
    such that $\lambda_{k+1}(\Sigma) \leq \gamma^2\lambda_k(\Sigma)$,
    for every $\eps,\delta>0$, and $0 < \alpha < 1$,
    there exists and $(\eps,\delta)$-DP algorithm that takes
    $$n \geq O\left(\frac{k\log(1/\delta)}{\eps} +
        \frac{\log(1/\delta)\log(\log(1/\delta)/\eps)}{\eps}\right)$$
    samples from $\cN(\vec{0},\Sigma)$, and outputs a projection matrix $\wh{\Pi}$,
    such that $\|\Pi-\wh{\Pi}\| \leq \alpha$ with probability
    at least $0.7$.
    \tnote{Eigenvalue gap assumption missing. Also order of quantifiers is ambiguous -- ``for every $\Sigma$ there exists an algorith.''}
\end{theorem}
%\end{comment}

Algorithm~\ref{alg:approximate} is a type of
``Subsample-and-Aggregate'' algorithm \cite{NissimRS07}.
Here, we consider multiple subspaces formed by the points
from the same Gaussian, and privately find a subspace that
is close to all those subspaces. Since the subspaces formed
by the points would be close to the true subspace, the privately
found subspace would be close to the true subspace.

A little more formally, we first sample $q$ public data points
(called ``reference points'') from $\cN(\vec{0},\id)$. Next,
we divide the original dataset $X$ into disjoint datasets of $m$ samples
each, and project all reference points on the subspaces spanned
by every subset. Now, for every reference point, we do the
following. We have $t=\tfrac{n}{m}$ projections of the reference
point. Using DP histogram over $\R^d$, we aggregate those
projections in the histogram cells; with high probability
all those projections will be close to one another, so they
would lie within one histogram cell. We output a random point
from the histogram cell corresponding to the reference point.
With a total of $q$ points output in this way, we finally
output the projection matrix spanned by these points. In
the algorithm $C_0$, $C_1$, and $C_2$ are universal constants.

We divide the proof of Theorem~\ref{thm:approximate}
into two parts: privacy (Lemma \ref{coro:privacy}) and
accuracy (Lemma~\ref{lem:final-projection}).

\begin{algorithm}[h!] 
\caption{\label{alg:approximate}DP Approximate Subspace Estimator
    $\DPASE_{\eps, \delta, \alpha, \gamma, k}(X)$}
\KwIn{Samples $X_1,\dots,X_{n} \in \R^d$.
    Parameters $\eps, \delta, \alpha, \gamma, k > 0$.}
\KwOut{Projection matrix $\wh{\Pi} \in \R^{d \times d}$ of rank $k$.}
\vspace{5pt}

Set parameters:
    $t \gets \tfrac{C_0\ln(1/\delta)}{\eps}$ \qquad
    $m \gets \lfloor n/t \rfloor$ \qquad $q \gets C_1 k$
    \qquad $\ell \gets \tfrac{C_2\gamma\sqrt{dk}(\sqrt{k}+\sqrt{\ln(kt)})}{\sqrt{m}}$
\vspace{5pt}

Sample reference points $p_1,\dots,p_q$ from $\cN(\vec{0},\id)$ independently.
\vspace{5pt}

\tcp{Subsample from $X$, and form projection matrices.}
\For{$j \in 1,\dots,t$}{
    Let $X^j = (X_{(j-1)m+1},\dots,X_{jm}) \in \mathbb{R}^{d \times m}$.\\
    Let $\Pi_j \in \mathbb{R}^{d \times d}$ be the projection matrix onto the subspace spanned by the eigenvectors of $X^j (X^j)^T \in \mathbb{R}^{d \times d}$ corresponding to the largest $k$ eigenvalues.\\
    \For{$i \in 1,\dots,q$}{
        $p_{i}^j \gets \Pi_j p_i$
    }
}
\vspace{5pt}

\tcp{Create histogram cells with random offset.}
Let $\lambda$ be a random number in $[0,1)$.\\
Divide $\R^{qd}$ into $\Omega =
    \{\dots,[\lambda\ell+i\ell,\lambda\ell+(i+1)\ell),\dots\}^{qd}$,
    for all $i \in \Z$.\\
Let each disjoint cell of length $\ell$ be a histogram bucket.
\vspace{5pt}

\tcp{Perform private aggregation of subspaces.}
For each $i \in [q]$, let $Q_i \in \RR^{d \times t}$ be the
    dataset, where column $j$ is $p_i^j$.\\
Let $Q \in \RR^{qd \times t}$ be the vertical concatenation
    of all $Q_i$'s in order.\\
Run $(\eps,\delta)$-DP histogram over $\Omega$ using $Q$
    to get $\omega \in \Omega$ that contains at least $\tfrac{t}{2}$ points.\\
\If{no such $\omega$ exists}{
    \Return $\bot$
}
\vspace{5pt}

\tcp{Return the subspace.}
Let $\wh{p}=(\wh{p}_1,\dots,\wh{p}_d,\dots,\wh{p}_{(q-1)d+1},\dots,\wh{p}_{qd})$
    be a random point in $\omega$.\\
\For{each $i \in [q]$}{
    Let $\wh{p}_i = (\wh{p}_{(i-1)d+1},\dots,\wh{p}_{id})$.
}
    Let $\wh{\Pi}$ be the projection matrix of the top-$k$ subspace of $(\wh{p}_1,\dots,\wh{p}_q)$.\\
\Return $\wh{\Pi}.$
\vspace{5pt}
\end{algorithm}

\subsection{Privacy}

We analyse the privacy by understanding the sensitivities
at the only sequence of steps invoking a differentially
private mechanism, that is, the sequence of steps involving
DP-histograms.

\begin{lemma}\label{lem:histogram-sensitivity}\label{coro:privacy}
    Algorithm~\ref{alg:approximate} is $(\eps,\delta)$-differentially
    private.
\end{lemma}
\begin{proof}
    Changing one point in $X$ can change only
    one of the $X^j$'s. This can
    only change one point in $Q$, which in turn can only
    change the counts in two histogram cells by $1$.
    Therefore, the sensitivity is $2$. % Since the choice
    %of $i$ was arbitrary, this is true for all $i$.
    %For a reference
    %point $p_i$, changing a point in $X^{j^*}$ can either move
    %its projection on to the subspace spanned by $X^{j^*}$ to a
    %different histogram cell, or keep it in the same cell.
%    Privacy now follows from the guarantees of DP-histogram (Lemma~\ref{lem:priv-hist}).
    Because the sensitivity of the histogram step is bounded
    by $2$ (Lemma~\ref{lem:histogram-sensitivity}), an application
    of DP-histogram, by Lemma~\ref{lem:priv-hist}, is $(\eps,\delta)$-DP.
    Outputting a random
    point in the privately found histogram cell preserves privacy
    by post-processing (Lemma~\ref{lem:post-processing}).
    Hence, the claim.
\end{proof}

\subsection{Accuracy}

\begin{comment}

We begin by showing a technical result that says that
any two matrices, whose difference is bounded in operator
norm, and which have significant eigenvalue gap, span
similar subspaces.

\begin{lemma}\label{lem:projections-close}
    Let $A, \tilde A \in \mathbb{R}^{d \times d}$ be symmetric
    matrices. Suppose $\|A-\tilde A\| \le \varepsilon$.
    Let $\lambda_1 \ge \lambda_2 \ge \cdots \ge \lambda_d$
    be the eigenvalues of $A$. Suppose $\lambda_k-\lambda_{k+1} > \varepsilon$.
    Let $\Pi_k, \tilde \Pi_k \in \mathbb{R}^{d \times d}$ be
    the projections onto the eigenspaces corresponding to
    the largest $k$ eigenvalues of $A$ and $\tilde A$ respectively.
    Then
    \[\left\| \Pi_k - \tilde \Pi_k \right\| \le
        \frac{\varepsilon}{\lambda_k-\lambda_{k+1}-\varepsilon}.\]
\end{lemma}
\begin{proof}
    Let $A,\tilde{A},U,\tilde{U},\Lambda,\tilde\Lambda
    \in \mathbb{R}^{d \times d}$ satisfy the following.
    (i) $A=U\Lambda U^T$,
    $\tilde{A} = \tilde{U} \tilde{\Lambda} \tilde{U}^T$,
    (ii) $U^TU=I_d=\tilde{U}^T\tilde{U}$, and (iii) $\Lambda$
    and $\tilde\Lambda$ are diagonal matrices with entries
    in descending order. Let $\lambda_i, \tilde{\lambda}_i$
    denote the $i^\text{th}$ diagonal entry of $\Lambda, \tilde\Lambda$.
    Denote $U_{a:b}, \tilde{U}_{a:b} \in \mathbb{R}^{d \times (b-a+1)}$
    to be the matrix formed by columns $a, a+1, \cdots, b$ of $U$
    and $\tilde{U}$ respectively (i.e., corresponding to
    $\lambda_a \ge \lambda_{a+1} \ge \cdots \ge \lambda_b$
    and $\tilde\lambda_a \ge \tilde\lambda_{a+1} \ge \cdots \ge \tilde\lambda_b$).

    Lemma~\ref{lem:weyl} tells us
    $|\tilde\lambda_k - \lambda_k| \le \|\tilde{A}-A\|$
    for all $k \in [d]$.
    Note that the operator norm is submultiplicative -- i.e.,
    $\|M M'\| \le \|M\| \cdot \|M'\|$.
    It immediately follows from Lemma~\ref{lem:davis-kahan}
    that, for all $k \in [d-1]$,
    we have
    \begin{align*}
        \|\tilde{U}_{1:k}^T {U}_{k+1:d}\| &\leq
                \frac{\|\tilde{A}-A\|}{\lambda_k-\tilde{\lambda}_{k+1}}\\
            &=\frac{\|\tilde{A}-A\|}{\lambda_k-\lambda_{k+1}-\lambda_{k+1}-\tilde{\lambda}_{k+1}}\\
            &\leq\frac{\|\tilde{A}-A\|}{\lambda_k - \lambda_{k+1} - \|\tilde{A}-A\|}.
    \end{align*}
    Now,
    \begin{align*}
        \tilde{\lambda}_k - \lambda_{k+1} &=
                \tilde{\lambda}_k - \lambda_k + \lambda_k - \lambda_{k+1}\\
            &> -\abs{\tilde{\lambda}_k - \lambda_k} + \lambda_k - \lambda_{k+1}\\
            &> -\eps + \eps\\
            &= 0.
    \end{align*}
    Therefore, we can apply Lemma~\ref{lem:davis-kahan}
    again
    to get $\|{U}_{1:k}^T \tilde{U}_{k+1:d}\| \le
    \frac{\|\tilde{A}-A\|}{\lambda_k - \lambda_{k+1} - \|\tilde{A}-A\|}$,
    and $\| \tilde{U}_{k+1:d}^T {U}_{1:k}\| \le
    \frac{\|\tilde{A}-A\|}{\lambda_k - \lambda_{k+1} - \|\tilde{A}-A\|}$
    follows because all eigenvalues of $U$ and $\tilde{U}$
    are non-negative.

    The ultimate quantity of interest for us is the
    difference between the projections $U_{1:k} U_{1:k}^T$
    and $\tilde{U}_{1:k} \tilde{U}_{1:k}^T$. We can
    apply the Lemma~\ref{lem:davis-kahan} to bound this:
    Note that $\tilde U_{1:k} \tilde U_{1:k}^T +
    \tilde U_{k+1:d} \tilde U_{k+1:d}^T = \tilde U \tilde U^T = I_d$.
    For any $A \in \mathbb{R}^{n \times m}$, $\|A^TA-I_m\|=\|AA^T-I_n\|$,
    as these are symmetric matrices with the same nonzero
    eigenvalues. Thus,
    \begin{align*}
        \left\| U_{1:k} U_{1:k}^T - \tilde{U}_{1:k} \tilde{U}_{1:k}^T \right\|
            &= \left\| U_{1:k} U_{1:k}^T + \tilde{U}_{k+1:d}
                \tilde{U}_{k+1:d}^T - I_d \right\| \\
            &= \left\|\left( U_{1:k} , \tilde{U}_{k+1:d} \right)
                \left( \begin{array}{c} U_{1:k}^T \\
                \tilde{U}_{k+1:d}^T \end{array} \right) - I_d \right\|\\
            &= \left\|\left( U_{1:k} , \tilde{U}_{k+1:d} \right)
                \left( U_{1:k} , \tilde{U}_{k+1:d} \right)^T - I_d \right\|\\
            &= \left\|\left( U_{1:k} , \tilde{U}_{k+1:d} \right)^T
                \left( U_{1:k} , \tilde{U}_{k+1:d} \right) - I_d \right\|\\
            &= \left\| \left( \begin{array}{cc} U_{1:k}^T U_{1:k}
                & U_{1:k}^T \tilde{U}_{k+1:d} \\ \tilde{U}_{k+1:d}^T U_{1:k}
                & \tilde{U}_{k+1:d}^T \tilde{U}_{k+1:d} \end{array} \right)
                - \left( \begin{array}{cc} I_k & 0_{k \times d-k} \\ 0_{d-k \times k}
                & I_{d-k} \end{array} \right)\right\|\\
            &= \left\| \left( \begin{array}{cc} 0_{k \times k}
                & U_{1:k}^T \tilde{U}_{k+1:d} \\ \tilde{U}_{k+1:d}^T U_{1:k}
                & 0_{d-k \times d-k} \end{array} \right) \right\|\\
            &= \max\left\{ \left\| U_{1:k}^T \tilde{U}_{k+1:d} \right\|,
                \left\| \tilde{U}_{k+1:d}^T U_{1:k} \right\| \right\}\\
            &\le \frac{\|\tilde{A}-A\|}{\lambda_k - \lambda_{k+1} - \|\tilde{A}-A\|}.
    \end{align*}
    This concludes the proof.
\end{proof}

Now we delve into the utility analysis of the algorithm.
Note that any matrix can be represented by its singular
value decomposition (SVD), that is, any matrix $A=UDV^T$,
where $D$ is a diagonal matrix containing its singular
values in decreasing order, $U$ is the matrix with left
singular vectors, and $V$ is the matrix with right singular
vectors, and $XX^T$ is a symmetric matrix $UDD^TU^T$,
where $U$ is the matrix containing the eigenvectors of
$XX^T$. Hence, to work with the projection matrix of
the subspace spanned by the columns of $X$, we can directly
work with the subspace spanned by the columns vectors
of $XX^T$ because they are equivalent. For $1 \leq j \leq t$,
let $X^j$ be the subsets of $X$ as defined in
Algorithm~\ref{alg:approximate}, and $\Pi_j$ be the
projection matrices of their respective subspaces. We
now show that $\Pi_j$ and the projection matrix of the
subspace spanned by $\Sigma_k$ are close in operator norm.

\end{comment}

Now we delve into the utility analysis of the algorithm.
For $1 \leq j \leq t$,
let $X^j$ be the subsets of $X$ as defined in
Algorithm~\ref{alg:approximate}, and $\Pi_j$ be the
projection matrices of their respective subspaces. We
now show that $\Pi_j$ and the projection matrix of the
subspace spanned by $\Sigma_k$ are close in operator norm.

\begin{lemma}\label{lem:empirical-subspaces-close}
    Let $\Pi$ be the projection matrix of the subspace
    spanned by the vectors of $\Sigma_k$, and for each
    $1 \leq j \leq t$, let $\Pi_j$ be the projection
    matrix as defined in Algorithm~\ref{alg:approximate}.
    If $m \geq O(k + \ln(qt))$, then
    $$\pr{}{\forall j, \|\Pi-\Pi_j\| \leq
        O\left(\frac{\gamma\sqrt{d}}{\sqrt{m}}\right)} \geq 0.95$$
\end{lemma}
\begin{proof}
    We show that the subspaces spanned by $X^j$ and
    the true subspace spanned by $\Sigma$ are close.
    Formally, we invoke
    Lemmata \ref{lem:sin-theta} and \ref{lem:sin-theta-property}.
    This closeness follows from standard matrix concentration
    inequalities.
    %which we discuss in Appendix \ref{sec:preliminaries}.
    
    Fix a $j \in [t]$. Note that $X^j$ can be written
    as $Y^j + H$, where $Y^j$ is the matrix of vectors
    distributed as $\cN(\vec{0},\Sigma_k)$, and $H$ is
    a matrix of vectors distributed as $\cN(\vec{0},\Sigma_{d-k})$,
    where $\Sigma_k$ and $\Sigma_{d-k}$ are defined as
    in Remark~\ref{rem:gamma}.
    By Corollary~\ref{coro:normal-spectrum}, with probability at least $1-\tfrac{0.02}{t}$,
    $s_k(Y^j) \in \Theta((\sqrt{m}+\sqrt{k})(\sqrt{s_k(\Sigma_k)})) = \Theta(\sqrt{m}+\sqrt{k})> 0$.
    Therefore, the subspace spanned by
    $Y^j$ is the same as the subspace spanned by $\Sigma_k$.
    So, it suffices to look at the subspace spanned
    by $Y^j$.

    Now, by Corollary~\ref{coro:normal-spectrum}, we know
    that with probability at least $1-\tfrac{0.02}{t}$,
    $\|X^j-Y^j\| = \|H\| \leq O((\sqrt{m}+{\sqrt{d}})\sqrt{s_1(\Sigma_{d-k})})
    \leq O(\gamma(\sqrt{m}+\sqrt{d})\sqrt{s_k(\Sigma_k)}) \leq O(\gamma(\sqrt{m}+\sqrt{d}))$.
    
    We wish to invoke Lemma~\ref{lem:sin-theta}. Let $UDV^T$
    be the SVD of $Y^j$, and let $\hat{U}\hat{D}\hat{V}^T$ be
    the SVD of $X^j$. Now, for a matrix $M$, let $\Pi_M$ denote
    the projection matrix of the subspace spanned by the columns
    of $M$. Define quantities $a,b,z_{12},z_{21}$ as follows.
    \begin{align*}
        a &= s_{\min}(U^TX^jV)\\
            &= s_{\min}(U^TY^jV + U^THV)\\
            &= s_{\min}(U^TY^jV) \tag{Columns of $U$ are orthogonal to columns of $H$}\\
            &= s_k(Y^j)\\
            &\in \Theta(\sqrt{m}+\sqrt{k})\\
            &\in \Theta(\sqrt{m})\\
        b &= \|U_{\bot}^TX^jV_{\bot}\|\\
            &= \|U_{\bot}^TY^jV_{\bot} + U_{\bot}^THV_{\bot}\|\\
            &= \|U_{\bot}^THV_{\bot}\|
                \tag{Columns of $U_{\bot}$ are orthogonal to columns of $Y^j$}\\
            &\leq \|H\|\\
            &\leq O(\gamma(\sqrt{m}+\sqrt{d}))\\
        z_{12} &= \|\Pi_U H \Pi_{V_{\bot}}\|\\
            &= 0\\
        z_{21} &= \|\Pi_{U_{\bot}}H\Pi_V\|\\
            &= \|\Pi_{U_{\bot}}\Sigma_{d-k}^{1/2}(\Sigma_{d-k}^{-1/2}H)\Pi_V\|
    \end{align*}
    Now, in the above, $\Sigma_{d-k}^{-1/2}H \in \RR^{d\times m}$,
    such that each of its entry is an independent sample from $\cN(0,1)$.
    Right-multiplying it by $\Pi_V$ makes it a matrix
    in a $k$-dimensional subspace of $\RR^m$, such that
    each row is an independent vector from a spherical
    Gaussian. Using Corollary~\ref{coro:normal-spectrum},
    $\|\Sigma_{d-k}^{-1/2}H\| \leq O(\sqrt{d}+\sqrt{k}) \leq O(\sqrt{d})$
    with probability at least $1-\tfrac{0.01}{t}$.
    Also, $\|\Pi_{U_{\bot}}\Sigma_{d-k}^{1/2}\| \leq O(\gamma\sqrt{s_k(\Sigma_k)}) \leq O(\gamma)$.
    This gives us:
    $$z_{21} \leq O(\gamma\sqrt{d}).$$

    Since $a^2 > 2b^2$, we get the following by
    Lemma~\ref{lem:sin-theta}.
    \begin{align*}
        \|\text{Sin}(\Theta)(U,\hat{U})\| &\leq \frac{az_{21} + bz_{12}}
                {a^2-b^2-\min\{z_{12}^2,z_{21}^2\}}\\
            &\leq O\left(\frac{\gamma\sqrt{d}}{\sqrt{m}}\right)
    \end{align*}

    Therefore, using Lemma~\ref{lem:sin-theta-property},
    and applying the union bound over all $j$, we get the
    required result.
\end{proof}

Let $\xi = O\left(\tfrac{\gamma\sqrt{d}}{\sqrt{m}}\right)$. We
show that the projections of any
reference point are close.

\begin{corollary}\label{coro:reference-projections-close}
    Let $p_1,\dots,p_q$ be the reference points as
    defined in Algorithm~\ref{alg:approximate}, and
    let $\Pi$ and $\Pi_j$ (for $1 \leq j \leq t$) be
    projections matrices as defined in Lemma~\ref{lem:empirical-subspaces-close}.
    Then
    $$\pr{}{\forall i,j, \|(\Pi-\Pi_j)p_i\| \leq O(\xi(\sqrt{k}+\sqrt{\ln(qt)}))} \geq 0.9.$$
\end{corollary}
\begin{proof}
    We know from Lemma~\ref{lem:empirical-subspaces-close}
    that $\|\Pi-\Pi_j\| \leq \xi$ for all $j$ with
    probability at least $0.95$. For $j \in [t]$, let
    $\wh{\Pi}_j$ be the projection matrix for the union
    of the $j^{\text{th}}$ subspace and the subspace
    spanned by $\Sigma_k$. Lemma~\ref{lem:gauss-vector-norm}
    implies that with probability at least $0.95$,
    for all $i,j$, $\|\wh{\Pi}_j p_i\| \leq O(\sqrt{k}+\sqrt{\ln(qt)})$.
    Therefore,
    \begin{align*}
        \|(\Pi-\Pi_j)p_i\| &= \|(\Pi-\Pi_j)\wh{\Pi}_jp_i\|
            \leq \|\Pi-\Pi_j\|\cdot\|\wh{\Pi}_jp_i\|
            \leq O(\xi(\sqrt{k}+\sqrt{\ln(qt)})).
    \end{align*}
    Hence, the claim.
\end{proof}

The above corollary shows that the projections of
each reference point lie in a ball of radius $O(\xi\sqrt{k})$.
Next, we show that for each reference point, all the
projections of the point lie inside a histogram cell
with high probability. For notational convenience, since
each point in $Q$ is a concatenation of the projection
of all reference points on a given subspace, for all
$i,j$, we refer to
$(0,\dots,0,Q_{(i-1)d+1}^j,\dots,Q_{id}^j,0,\dots,0) \in R^{qd}$
(where there are $(i-1)d$ zeroes behind $Q_{(i-1)d+1}^j$,
and $(q-i)d$ zeroes after $Q_{id}^j$) as $p_i^j$.

\begin{lemma}\label{lem:histogram-cell-points}
    Let $\ell$ and $\lambda$ be the length of a histogram
    cell and the random offset respectively, as defined in
    Algorithm~\ref{alg:approximate}. Then
    $$\pr{}{|\omega \cap Q| = t} \geq 0.8.$$
    Thus there exists $\omega \in \Omega$ that,
    such that all points in $Q$ lie within $\omega$.
\end{lemma}
\begin{proof}
    Let $r = O(\xi(\sqrt{k}+\sqrt{\ln(qt)}))$. This implies that $\ell = 20r\sqrt{q}$.
    The random offset could also be viewed as moving along a
    diagonal of a cell by $\lambda\ell\sqrt{dq}$. We know that
    with probability at least $0.8$, for each $i$, all projections
    of reference point $p_i$ lie in a ball of radius $r$.
    This means that all the points in $Q$ lie in a ball of
    radius $r\sqrt{q}$. Then
    $$\pr{}{|\omega \cap Q| = t} \leq \pr{}{\frac{1}{20} \geq
        \lambda \vee \lambda \geq \frac{19}{20}} = \frac{1}{10}.$$
    Taking the union bound over all $q$ and the failure
    of the event in Corollary~\ref{coro:reference-projections-close},
    we get the claim.
\end{proof}

Now, we analyse the sample complexity due
to the private algorithm, that is,
DP-histograms.

\begin{lemma}\label{lem:dp-histogram-cost}
    Let $\omega$ be the histogram cell as defined in
    Algorithm~\ref{alg:approximate}. Suppose $\pcount(\omega)$
    is the noisy count of $\omega$ as a result of applying
    the private histogram. If
    $t \geq O\left(\frac{\log(1/\delta)}{\eps}\right),$
    then
    $$\pr{}{\abs{\pcount(\omega)} \geq \frac{t}{2}} \geq 0.75.$$
\end{lemma}
\begin{proof}
    Lemma~\ref{lem:histogram-cell-points} implies that
    with probability at least $0.8$, for each $i$, all
    projections of $p_i$ lie in a histogram cell, that is,
    all points of $Q$ lie in a histogram cell in $\Omega$.
    Because of the error bound in Lemma~\ref{lem:priv-hist}
    and our bound on $t$, we see at least $\tfrac{t}{2}$
    points in that cell with probability at least $1-0.05$.
    Therefore, by taking the union bound, the proof is complete.
\end{proof}

We finally show that the error of the projection matrix
that is output by Algorithm~\ref{alg:approximate} is small.

\begin{lemma}\label{lem:final-projection}
    Let $\wh{\Pi}$ be the projection matrix as defined in
    Algorithm~\ref{alg:approximate}, and $n$ be the total
    number of samples. If
    $$\gamma^2 \in
        O\left(\frac{\eps\alpha^2n}{d^{2}k\ln(1/\delta)}\cdot
        \min\left\{\frac{1}{k},
        \frac{1}{\ln(k\ln(1/\delta)/\eps)}
        \right\}\right),$$
    $n \geq O(\frac{k\log(1/\delta)}{\eps}+\frac{\ln(1/\delta)\ln(\ln(1/\delta)/\eps)}{\eps})$,
    and $q \geq O(k)$
    the with probability at least $0.7$, $\|\wh{\Pi}-\Pi\| \leq \alpha$.
\end{lemma}
\begin{proof}
    For each $i \in [q]$, let $p_i^*$ be the projection
    of $p_i$ on to the subspace spanned by $\Sigma_k$,
    $\wh{p}_i$ be as defined in the algorithm, and $p_i^j$
    be the projection of $p_i$ on to the subspace spanned
    by the $j^{\mathrm{th}}$ subset of $X$. From Lemma~\ref{lem:dp-histogram-cost},
    we know that all $p_i^j$'s are contained in a histogram
    cell of length $\ell$. This implies that $p_i^*$ is also
    contained within the same histogram cell.

    Now, let $P=(p_1^*,\dots,p_q^*)$ and $\wh{P}=(\wh{p}_1,\dots,\wh{p}_q)$.
    Then by above, $\wh{P}=P+E$, where $\|E\|_F \leq 2\ell\sqrt{dq}$. Therefore,
    $\|E\| \leq 2\ell\sqrt{dq}$. Let $E=E_P+E_{\wb{P}}$,
    where $E_P$ is the component of $E$ in the subspace
    spanned by $P$, and $E_{\wb{P}}$ be the orthogonal
    component. Let $P' = P + E_P$. We will be analysing
    $\wh{P}$ with respect to $P'$.

    Now, with probability
    at least $0.95$, $s_k(P) \in \Theta(\sqrt{k})$ due to our
    choice of $q$ and using Corollary~\ref{coro:normal-spectrum},
    and $s_{k+1}(P) = 0$. So, $s_{k+1}(P') = 0$ because $E_P$ is
    in the same subspace as $P$. Now, using Lemma~\ref{lem:least-singular},
    we know that $s_k(P') \geq s_k(P) - \|E_P\| \geq \Omega(\sqrt{k}) > 0$.
    This means that
    $P'$ has rank $k$, so the subspaces spanned by $\Sigma_k$
    and $P'$ are the same.

    As before, we will try to
    bound the distance between the subspaces spanned
    by $P'$ and $\wh{P}$. Note that using Lemma~\ref{lem:weyl-singular},
    we know that $s_k(P') \leq s_k(P) + \|E_P\| \leq O(\sqrt{k})$.

    We wish to invoke Lemma~\ref{lem:sin-theta} again. Let $UDV^T$
    be the SVD of $P'$, and let $\hat{U}\hat{D}\hat{V}^T$ be
    the SVD of $\wh{P}$. Now, for a matrix $M$, let $\Pi_M$ denote
    the projection matrix of the subspace spanned by the columns
    of $M$. Define quantities $a,b,z_{12},z_{21}$ as follows.
    \begin{align*}
        a &= s_{\min}(U^T\wh{P}V)\\
            &= s_{\min}(U^TP'V + U^TE_{\wb{P}}V)\\
            &= s_{\min}(U^TP'V) \tag{Columns of $U$ are orthogonal to columns of $E_{\wb{P}}$}\\
            &= s_k(P')\\
            &\in \Theta(\sqrt{k})\\
        b &= \|U_{\bot}^T\wh{P}V_{\bot}\|\\
            &= \|U_{\bot}^TP'V_{\bot} + U_{\bot}^TE_{\wb{P}}V_{\bot}\|\\
            &= \|U_{\bot}^TE_{\wb{P}}V_{\bot}\|
                \tag{Columns of $U_{\bot}$ are orthogonal to columns of $P'$}\\
            &\leq \|E_{\wb{P}}\|\\
            &\leq O(\ell\sqrt{dq})\\
        z_{12} &= \|\Pi_U E_{\wb{P}} \Pi_{V_{\bot}}\|\\
            &= 0\\
        z_{21} &= \|\Pi_{U_{\bot}}E_{\wb{P}}\Pi_V\|\\
            &\leq \|E_{\wb{P}}\|\\
            &\leq O(\ell{\sqrt{dq}})
    \end{align*}

    Using Lemma~\ref{lem:sin-theta}, we get the following.
    \begin{align*}
        \|\text{Sin}(\Theta)(U,\hat{U})\| &\leq \frac{az_{21} + bz_{12}}
                {a^2-b^2-\min\{z_{12}^2,z_{21}^2\}}\\
            &\leq O\left(\ell\sqrt{dk}\right)\\
            &\leq \alpha
    \end{align*}

    This completes our proof.
\end{proof}

\subsection{Boosting}

In this subsection, we discuss boosting of error
guarantees of Algorithm~\ref{alg:approximate}.
The approach we use is very similar to the well-known
Median-of-Means method: we run the algorithm multiple
times, and choose an output that is close to all
other ``good'' outputs. We formalise this in
Algorithm~\ref{alg:approximate-boosted}.

\begin{algorithm}[h!]
\caption{\label{alg:approximate-boosted}DP Approximate Subspace Estimator Boosted
    $\DPASEB_{\eps, \delta, \alpha, \beta, \gamma, k}(X)$}
\KwIn{Samples $X_1,\dots,X_{n} \in \R^d$.
    Parameters $\eps, \delta, \alpha, \beta, \gamma, k > 0$.}
\KwOut{Projection matrix $\wh{\Pi} \in \R^{d \times d}$ of rank $k$.}
\vspace{5pt}

Set parameters:
    $t \gets C_3 \log(1/\beta)$ \qquad $m \gets \lfloor n/t \rfloor$
\vspace{5pt}

Split $X$ into $t$ datasets of size $m$: $X^1,\dots,X^t$.
\vspace{5pt}

\tcp{Run $\DPASE$ $t$ times to get multiple projection matrices.}
\For{$i \gets 1,\dots,t$}{
    $\wh{\Pi}_i \gets \DPASE_{\eps,\delta,\alpha,\gamma,k(X^i)}$
}
\vspace{5pt}

\tcp{Select a good subspace.}
\For{$i \gets 1,\dots,t$}{
    $c_i \gets 0$\\
    \For{$j \in [t]\setminus\{i\}$}{
        \If{$\|\wh{\Pi}_i-\wh{\Pi}_j\| \leq 2\alpha$}{
            $c_i \gets c_i + 1$
        }
    }
    \If{$c_i \geq 0.6t-1$}{
        \Return $\wh{\Pi}_i$.
    }
}
\vspace{5pt}

\tcp{If there were not enough good subspaces, return $\bot$.}
\Return $\bot.$
\vspace{5pt}
\end{algorithm}

Now, we present the main result of this subsection.

\begin{theorem}\label{thm:approximate-boosted}
    Let $\Sigma \in \R^{d \times d}$ be an arbitrary, symmetric, PSD
    matrix of rank $\geq k \in \{1,\dots,d\}$, and let $0 < \gamma < 1$.
    Suppose $\Pi$ is the projection matrix corresponding
    to the subspace spanned by the vectors of $\Sigma_k$.
    Then given
    $$\gamma^2 \in
        O\left(\frac{\eps\alpha^2n}{d^{2}k\ln(1/\delta)}\cdot
        \min\left\{\frac{1}{k},
        \frac{1}{\ln(k\ln(1/\delta)/\eps)}
        \right\}\right),$$
    such that $\lambda_{k+1}(\Sigma) \leq \gamma^2\lambda_k(\Sigma)$,
    for every $\eps,\delta>0$, and $0 < \alpha,\beta < 1$,
    there exists and $(\eps,\delta)$-DP algorithm that takes
    $$n \geq O\left(\frac{k\log(1/\delta)\log(1/\beta)}{\eps} +
        \frac{\log(1/\delta)\log(\log(1/\delta)/\eps)\log(1/\beta)}{\eps}\right)$$
    samples from $\cN(\vec{0},\Sigma)$, and outputs a projection matrix $\wh{\Pi}$,
    such that $\|\Pi-\wh{\Pi}\| \leq \alpha$ with probability
    at least $1-\beta$.
\end{theorem}
\begin{proof}
    Privacy holds trivially by Theorem~\ref{thm:approximate}.

    We know by Theorem~\ref{thm:approximate} that
    for each $i$, with probability at least $0.7$,
    $\|\wh{\Pi}_i-\Pi\| \leq \alpha$. This means
    that by Lemma~\ref{lem:chernoff-add}, with probability
    at least $1-\beta$, at least $0.6t$ of all
    the computed projection matrices are accurate.

    This means that there has to be at least one projection
    matrix that is close to $0.6t-1>0.5t$ of these
    accurate projection matrices. So, the algorithm
    cannot return $\bot$.

    Now, we want to argue that the returned projection
    matrix is accurate, too. Any projection matrix
    that is close to at least $0.6t-1$ projection
    matrices must be close to at least one accurate
    projection matrix (by pigeonhole principle). Therefore,
    by triangle inequality,
    it will be close to the true subspace. Therefore,
    the returned projection matrix is also accurate.
\end{proof}

%\subsection{Putting It All Together}

%Now, we are ready to finish the main theorem of the section.

%\begin{proof}[Proof of Theorem~\ref{thm:approximate}]
%    By Corollary~\ref{coro:privacy} and Lemma~\ref{lem:final-projection},
%    Algorithm~\ref{alg:approximate} is differentially private
%    and accurate.
%\end{proof}

%\break
\addcontentsline{toc}{section}{References}
%\bibliographystyle{alpha}
%\bibliography{biblio}

\printbibliography

%\appendix

%
	\section{Proof of Proposition~\ref{prop-quasi-order}}
\begin{proof}
	We will prove the result for relation $\sqsubseteq$, the proof for $\preceq$ being similar. We need to prove that $\sqsubseteq$ is reflexive and transitive. For reflexivity, it is obvious that since $G\subseteq G$ for any goal $G$, we have $G\sqsubseteq_\theta G$ for the empty substitution $\theta$. For transitivity, suppose that for goals $G_1$, $G_2$ and $G_3$, it holds that $G_1 \sqsubseteq_{\theta_1} G_2$ and $G_2 \sqsubseteq_{\theta_2} G_3$. Then by Definition~\ref{def-generalization}, there exist sets of atoms $\Delta_1$ and $\Delta_2$ such that $G_1\theta_1 \cup \Delta_1 = G_2$ and $G_2\theta_2\cup\Delta_2 = G_3$. In other words it holds that $(G_1\theta_1\cup\Delta_1)\theta_2\cup\Delta_2 = G_3$ or equivalently, $(G_1\theta_1)\theta_2 \cup (\Delta_1\theta_2 \cup\Delta_2) = G_3$. As the composition of two substitutions is a substitution, by defining $\theta_3 = \theta_2\circ\theta_1$ and $\Delta_3 = \Delta_1\theta_2 \cup\Delta_2$, we have $G_1\theta_3 \cup \Delta_3 = G_3$, so $G_1\sqsubseteq_{\theta_3} G_3$, which concludes the proof.
\end{proof}
	

	\section{Proof of Proposition~\ref{prop-msg-lcg}}
	
		First, observe the following property that holds for both relations, essentially stating that a common generalization that is not a lcg has a direct extension obtained by the addition of one atom. 
		
		\begin{proposition}\label{prop-lcg-extensible}
			Let $G_1, \dots, G_n$ and $G$ be goals such that $G$ is a $\leqslant$-common generalization, but not a $\leqslant$-lcg, of $\{G_1, \dots, G_n\}$. Then there exists an atom $A\notin G$ such that $G\cup\{A\}$ is a $\leqslant$-common generalization of $\{G_1, \dots, G_n\}$.
		\end{proposition} 
	
	
		\begin{proof}
		
		Let us suppose the existence of some goal $G$, a $\leqslant$-common generalization that is not a $\leqslant$-lcg of $\{G_1, \dots, G_n\}$, and let us try and extend $G$ into a $\leqslant$-common generalization $G\cup\{A\}$ with $A\notin G$ an atom. As $G$ is not a lcg, there must exist another goal $G'$ being a $\leqslant$-lcg of $G_1$ and $G_2$ and obviously we have $|G'|>|G|$. As a consequence at this point there are three groups of atoms that can be identified: let us denote by $\hat{A}_1, \dots, \hat{A}_p$ the $p (\ge 0)$ atom(s) that are both in $G$ and in $G'$; by $A_1, \dots, A_m$ the $m (\ge 0)$ atom(s) that are part of $G$ but not of $G'$; and by $B_1, \dots, B_l$ the $l (\ge 1)$ atom(s) that are part of $G'$ but not of $G$. For an element $A$ of any of these sets, we denote by $A^1, \dots, A_n$ the atom in respectively $G_1, \dots, G_n$ whose anti-unification led to having $A$ as part of the generalizations.
		
		From the fact that $|G'|>|G|$ it follows that $l>m$. Now each $A_i (i \in 1..m)$ is such that $\exists  h\in 1..n : A_i^h\in \{B_i^h|i\in 1..l\}$: if not, it would be possible to add an atom generalizing $\{A_i^1, \dots, A_i^n\}$ (such as $A_i$) in $G'$ and get a larger generalization, which is impossible given that $G'$ is a lcg. Also note that for two atoms $B_i$ and $B_j (1\le i < j \le l)$, for $g, h \in 1..n : g\neq h$, if $B_i^g$ is anti-unifiable with an atom $B_j^h$ then $B_j^g$ is also anti-unifiable with $B_i^h$ (as it means that all four base atoms are a call to one and the same predicate (with relation $\sqsubseteq$) or have the exact same inner structure save for variables (with relation $\preceq$)), so that it is possible to switch the atoms $B_i^g$ and $B_j^h$, compute the anti-unification of $\{B_i^g, B_j^h)$ and $(B_j^g, B_i^h)$, and get an equally valid anti-unification. Thanks to this we can, where necessary, perform switches so as to rearrange the atoms $B_i (1\le i \le l)$ into $\{\hat{B}_i|i \in 1..l\}$ in such a way that $\{A_i|i\in 1..m\} \subset \{\hat{B}_i|i \in 1..l\}$ and for each atom $\hat{B}_i$, either $\hat{B}_i \in \{A_k|k\in 1..m\}$ or $\exists g, h \in 1..n : g\neq h \wedge \hat{B}_i^g\notin \{A_k^g|k\in 1..m\}\wedge \hat{B}_i^h\notin \{A_k^h|k\in 1..m\}$. We can now define a new generalization $\hat{G}$ defined as the union of these rearranged atoms and those that are common to $G$ and $G'$, i.e. $\hat{G} = \{\hat{A_i}|i\in 1..p\}\cup\{\hat{B}_i|i\in 1..l\}$. Since $|\hat{G}| = |G'|$ and $G\subset \hat{G}$, it suffices to add one of the atoms $A \in \hat{G}\setminus G$ to $G$ in order to obtain $G\cup\{A\}$, a $\leqslant$-common generalization of $\{G_1, \dots, G_n\}$ by construction.
	\end{proof}


		Next, we prove Proposition~\ref{prop-msg-lcg}.
		
	\begin{proof}
	We prove that any $\leqslant$-msg is a $\leqslant$-lcg by contradiction. Let us suppose that some goal $G$ is both a $\leqslant$-msg and not a $\leqslant$-lcg of the set of $\{G_1, \dots, G_n\}$. According to Proposition~\ref{prop-lcg-extensible} it must then be possible to select an atom $A \notin G$ such that $G\cup\{A\}$ is a $\leqslant$-common generalization of $\{G_1, \dots, G_n\}$. Since $A\notin G$ and any atom has a $\tau$-value of at least 1, it follows that $|\tau(G\cup\{A\})|>|\tau(G)|$. Consequently $G$ cannot be a $\leqslant$-most specific generalization of $G_1$ and $G_2$: a contradiction.
	
	As for the fact that any $\preceq$-lcg is a $\preceq$-msg, we prove this also by contradiction. Let $G$ represent a $\preceq$-lcg of the set of goals $\{G_1, \dots, G_n\}$ and let us suppose that $G$ is not a $\preceq$-msg. Then there must exist another goal that is a $\preceq$-msg of $\{G_1, \dots, G_n\}$, say $G'$, such that $|\tau(G')|>|\tau(G)|$ and, according to the first part of the proposition, $|G'|=|G|$. 
	Now, observe that for a set of atoms $\{A_1, \dots, A_n\}$ to be anti-unified with $\preceq$ into an atom $A$, necessarily all $A_i (1\le i\le n)$ must have the same $\terms$-value. Indeed relation $\preceq$ is defined upon renamings so that only variables (having a $\terms$-value of zero) are impacted by the generalization process. Therefore, the only possibility for the inequality $|\tau(G')|>|\tau(G)|$ to be true is that some atoms $B_1, \dots, B_n$ of respective goals $G_1,\dots,G_n$ appear in a generalized form (say $B$) in $G'$, while these atoms have not been generalized in $G$. This means that it is possible to add a (possibly renamed) version of $B$ in $G$ and obtain $G\cup\{B\}$, also a $\preceq$-common generalization and larger than $G$: a contradiction.
	\end{proof}

	
	\section{Detailed proof of Lemma~\ref{lemma-au-op}}
	\begin{proof}
The lemma will be shown correct by the definition of three anti-unification operators. A first anti-unification operator, based on $\sqsubseteq$ is the following. 

\begin{definition}%[Simple anti-unification operator]
	\label{def-atoms-au}
	Given a variabilization function $\Phi$, let $\au^\Phi_\sqsubseteq$ (or simply $\au_\sqsubseteq$ if $\Phi$ is clear from the context) denote the anti-unification operator such that for any two atoms $A = a(t^A_1, \dots, t^A_n)$ and $B = b(t^B_1, \dots, t^B_m)$, it holds that \[\au^{\Phi}_\sqsubseteq(A,B)=\left\{\begin{array}{l}
		a\big(\Phi(t^A_1, t^B_1), \dots, \Phi(t^A_n, t^B_n)\big) \\ \qquad  \mbox{if } a = b \mbox{ and } n = m\\
		\bot \\
		\qquad \mbox{otherwise}\\
	\end{array}\right. \]
\end{definition} 

\begin{example}\label{ex-au-sq}
	In Table~\ref{table:sqsubseteq}, we show three atomic anti-unification results obtained by the application of $\au_\sqsubseteq^\Phi$ with $\Phi$ a given variabilization function. Note how in the first example, the predicates used in $A_1$ and $A_2$ differ (resp. $p/2$ and $p/3$), leading to an impossible anti-unification.
\end{example}

\begin{table*}
	\caption{Example results for $au_\sqsubseteq^\Phi$}
	\label{table:sqsubseteq}
	\centering
	\begin{tabular}{l|l|l}
		%\hline 
		$\bm{A_1}$ & $\bm{A_2}$ & $\bm{\au_\sqsubseteq^\Phi(A_1, A_2)}$\\\hline 
		$p(X, 5, q(Y,4))$ & $p(W,t(Z))$ & $\bot$\\\hline 
		$p(r(X,3), t(5))$ & $p(W, t(Z))$ & $p(\Phi(r(X,3),W), \Phi(t(5), t(Z)))$\\\hline 
		$p(r(X,3), t(Y))$ & $p(r(W,3),t(Z))$ & $p(\Phi(r(X,3),r(W,3)), \Phi(t(Y),t(Z)))$ %\\\hline 
	\end{tabular} 
\end{table*}

Note that the anti-unification operator defined in Definition~\ref{def-atoms-au} differs from the traditional subsumption operator in the ordered case (i.e. when goals are ordered sequences of atoms). The difference comes from the fact that our goals being sets, all the possible couples of atoms have to be considered, whereas traditional subsumption must handle one atom at the time, making the anti-unification operator more straigtforward.     

Let us now introduce a second anti-unification operator that will allow to compute a $\preceq$-lcg. 
%and to prove Theorem~\ref{thm-preceq-lcg}. 
Since the result of this operator should be a $\preceq$-common generalization, the operator need only to anti-unify the \textit{variables} occurring at the corresponding positions in the atoms under investigation. 
The operator must thus go deeper into the term structure of the atoms than $\au_\sqsubseteq$ does, as it needs to only anti-unify those atoms that harbor the exact same structure at the level of their non-variable terms.
\begin{definition}%[Variable anti-unification operator]
	\label{def-term-au-through-variables}
	Given some variabilization function $\Phi$, let $\au^\Phi_\preceq$ (or simply $\au_\preceq$ if $\Phi$ is clear from the context) denote the function such that for any two terms $T = t(t_1, \dots, t_n)$ and $U = u(u_1, \dots, u_m)$ it holds that
	\[\au^\Phi_\preceq(T,U)=\left\{\begin{array}{l}
		\Phi(T,U) 
		\\ \qquad \mbox{if } T\in\mathcal{V}\mbox{ and } U\in\mathcal{V}
		\\t\big(\au^\Phi_\preceq(t_1,u_1), \dots, \au^\Phi_\preceq(t_n, u_n)\big) 
		\\ \qquad \mbox{if } t = u \mbox{ and } n = m 
		\\ \qquad \mbox{and } \forall i \in 1..n: \au^\Phi_\preceq(t_i,u_i)\neq\bot
		\\ \bot
		\\ \qquad  \mbox{otherwise}
	\end{array}\right.\]
	and for any two atoms $A = a(t^A_1, \dots, t^A_n)$ and $B = b(u^B_1, \dots, u^B_m)$, it holds that
	\[\au^\Phi_\preceq(A,B)=\left\{\begin{array}{l}
		a\big(\au^\Phi_\preceq(t^A_1, u^B_1),\dots, \au^\Phi_\preceq(t^A_n, u^B_n)\big) 
		\\ \qquad \mbox{if } a = b \mbox{ and } n = m 
		\\ \qquad \mbox{and } \forall i \in 1..n: \au^\Phi_\preceq(t^A_i, u^B_i) \neq\bot
		\\ \bot  
		\\ \qquad \mbox{otherwise}
	\end{array}\right.\]
\end{definition}

\begin{example}
	In Table~\ref{table:preceq}, we treat the anti-unification of the same atoms as above, this time with the use of $\au_\preceq^\Phi$ with $\Phi$ a given variabilization function. Note how $\au_\preceq$ behaves differently than $\au_\sqsubseteq$ on the second and third couple of atoms as it requires its arguments to exhibit a similar structure in order to be anti-unifiable.
\end{example}

\begin{table*}
	\caption{Example results for $au_\preceq^\Phi$}
	\label{table:preceq}
	\centering
	\begin{tabular}{l|l|l}
		%\hline
		$\bm{A_1}$ & $\bm{A_2}$ & $\bm{\au_\preceq^\Phi(A_1, A_2)}$\\\hline 
		$p(X, 5, q(Y,4))$ & $p(W,t(Z))$ & $\bot$\\\hline 
		$p(r(X,3), t(5))$ & $p(W, t(Z))$ & $\bot$\\\hline 
		$p(r(X,3), t(Y))$ & $p(r(W,3),t(Z))$ & $p(r(\Phi(X,W),3), t(\Phi(Y,Z)))$ %\\\hline
	\end{tabular}
\end{table*}

Now, in order to compute $\sqsubseteq$-msgs, we need a more precise anti-unification operator: one that goes deeper into detail when comparing atoms so as not to miss their maximal common structure. 
\begin{definition} %[Deep anti-unification operator]\label{def-deep-operator}
	Given some variabilization function $\Phi$, let $\dau^\Phi_\sqsubseteq$ (or simply $\dau_\sqsubseteq$ if $\Phi$ is clear from the context) denote the function such that for any two terms $T = t(t_1, \dots, t_n)$ and $U = u(u_1, \dots, u_m)$ it holds that 
	\[\dau^\Phi_\sqsubseteq(T,U)=\left\{\begin{array}{l}
		
		t\big(\dau^\Phi_\sqsubseteq(t_1,u_1), \dots, \dau^\Phi_\sqsubseteq(t_n, u_n)\big) 
		\\ \qquad \mbox{if } t = u \mbox{ and } n = m 
		\\ \qquad \mbox{and } T \notin \mathcal{V} \mbox{ and } U \notin \mathcal{V}
		\\ \Phi(T,U) 
		\\ \qquad \mbox{otherwise}
	\end{array}\right.\]
	and for any two atoms $A = a(t^A_1, \dots, t^A_n)$ and $B = b(u^B_1, \dots, u^B_m)$, it holds that
	\[\dau^\Phi_\sqsubseteq(A,B)=\left\{\begin{array}{l}
		a\big(\dau^\Phi_\sqsubseteq(t^A_1, u^B_1),\dots, \dau^\Phi_\sqsubseteq(t^A_n, u^B_n)\big) 
		\\ \qquad \mbox{if } a = b \mbox{ and } n = m 
		\\ \bot
		\\ \qquad \mbox{otherwise}
	\end{array}\right.\]
\end{definition}

When applied on atoms, it is easy to see that $\dau_\sqsubseteq$ is an anti-unification operator based on relation $\sqsubseteq$.

\begin{example}
	Let us once more consider the anti-unification of the atoms introduced in Example~\ref{ex-au-sq}. This time we make use of $\dau_\sqsubseteq^\Phi$ with $\Phi$ a given variabilization function, to anti-unify the three pairs of atoms. The result is shown in Table~\ref{table:dau}. Notice how the operator preserves as much non-variable atomic structure as possible in the process.
\end{example}
\begin{table*}
	\caption{Example results for $dau_\sqsubseteq^\Phi$}
	\label{table:dau}
	\centering
	\begin{tabular}{l|l|l}
		%\hline 
		$\bm{A_1}$ & $\bm{A_2}$ & $\bm{\dau_\sqsubseteq^\Phi(A_1, A_2)}$\\\hline 
		$p(X, 5, q(Y,4))$ & $p(W,t(Z))$ & $\bot$\\\hline 
		$p(r(X,3), t(5))$ & $p(W, t(Z))$ & $p(\Phi(r(X,3),W),t(\Phi(5,Z)))$\\\hline 
		$p(r(X,3), t(Y))$ & $p(r(W,3),t(Z))$ & $p(r(\Phi(X,W),3), t(\Phi(Y,Z)))$ %\\\hline 
	\end{tabular} 
\end{table*}
The existence of these operators proves Lemma~\ref{lemma-au-op}.
	\end{proof}
	
	\section{Proof of Theorem~\ref{thm-ausqsubseteq}}
	\begin{proof}
	Obviously the $\au_\sqsubseteq(A_1,A_2)$ operation can be achieved in a time linear with respect to the arity $n$ of $A_1$. In the worst case, the operation needs to be performed for each atom in $G_1$ with respect to each atom in $G_2$. Hence the first result.
	
	
	It is also easy to see that the $\au_\preceq(A_1,A_2)$ operation can be achieved in linear time with respect to the maximum number of function applications in the argument terms of the atom $A_1$ under scrutiny. In the worst case, the operation needs to be performed for each atom in $G_1$ with respect to each atom in $G_2$. Hence the second result.
	\end{proof}
		
%	
%	\section{Maximum Weight Matching of Example~\ref{example-mwm}}
%	See Fig.~\ref{fig:mwm}.
	
	
	\section{Proof of Theorem~\ref{thm-sqsubseteq-msg}}
\begin{proof}
	First note how the atomic anti-unifications and the weights of the associated bipartite graph's edges can be computed simultaneously, by working out $\dau_\sqsubseteq(A_1,A_2)$ for each possible couple $(A_1,A_2)$ in $G_1\times G_2$ and keeping account of the number of non-variable terms encountered during the operation (or $-1$). Given that $\dau_\sqsubseteq(A_1,A_2)$ can obviously operate linearly in the number of terms appearing in $A_1$ (denoted $N$), the computation of all weights is carried out in a time not exceeding $\mathcal{O}(|G_1|.|G_2|.N)$.
	
	Now the obtained assignment problem can be solved by existing algorithms (such as the Hungarian method~\cite{assignment}) that compute a MWM in $\mathcal{O}(n^3)$, where $n$ is the number of vertexes appearing on the side of the bipartite graph that has the most vertexes. In our case, there are $|G_1|$ left vertexes and $|G_2|$ right vertexes so that a MWM algorithm can be ran in $\mathcal{O}(max(|G_1|,|G_2|)^3)$.
\end{proof}
	
	\section{Proof of Theorem~\ref{thm-dataflow-np-complete}}
\begin{proof}
	First, let us consider MSG-MIN. It clearly belongs to NP. Indeed, given an arbitrary generalization $G$, we can verify in polynomial time whether it is a most specific generalization. The procedure is as follows. We can compute at least one $\leqslant$-msg, say $G'$, in polynomial time (see Theorem~\ref{thm-sqsubseteq-msg}). It suffices then to compare the $\tau$-value of $G'$ with that of $G$ in order to decide whether $G$ is a msg. Next, verifying whether the number of variables in $G$ is bounded by a constant is obviously achieved in polynomial time as well.
	
	In order to prove NP-hardness, we will construct a reduction from the well-known set cover problem (known to be NP-complete~\cite{karp}) to MSG-MIN. The set cover problem in its decision-problem version (denoted SCP), can be formulated as follows. Given a constant $p \in \mathbb{N}_0$, a universe $U$ of values and a collection $S$ composed of $n$ sets $\{S_1, \dots, S_n\}$ that cover $U$, i.e. $U = \underset{i=1}{\overset{n}{\cup}}S_i$, the problem is to decide whether there exists $p$ subsets from $S$ that still cover $U$.
	
	We can transform an arbitrary instance of SCP into MSG-MIN as follows. Let us consider without loss of generality a universe $U$ where the elements are lowercase strings and $p \in \mathbb{N}_0$ a constant. Given a collection of sets $S=\{S_1, \dots, S_n\}$ we construct an instance of MSG-MIN as follows. In our construction we use $n+1$ different variables, namely $V$ and $(W_i)_{i\in1..n}$. We use $x_j$ to denote some element of $U$; these elements being strings, we can easily use them as predicate names. The construction of goals $G_1$ and $G_2$ proceeds then as follows:
	
	\begin{algorithmic}
		\State $G_1 = \{\}$ 
		\State $G_2 = \{\}$ 
		\For {each ($S_i \in S$)}
		\For {each ($x_j \in S_i$)}
		\State $G_1 \gets G_1\cup \{x_j(V)\}$				
		\State $G_2 \gets G_2\cup \{x_j(W_i)\}$
		\EndFor
		\EndFor
	\end{algorithmic}
	Note that all the atoms in $G_1$ have the same argument (namely the variable $V$) and there are as many atoms in $G_1$ as there are distinct elements in $S$. In $G_2$, however, there is an atom of the form $x_j(W_i)$ for each element $x_j$ occurring in $S_i$.
	
	The construction is such that any $\leqslant$-msg of $G_1$ and $G_2$ will be a version of $G_1$ where each occurrence of a variable $V$ is replaced by $\Phi(V, W_k)$ for some $W_k\in\vars(G_2)$ (where $\Phi$ is a variabilization function). Now, introducing such a variable $\Phi(V, W_k)$ in the generalization will allow to reuse the same variable for all the atoms $x_j(V)$ in $G_1$ that have a corresponding $x_j(W_k)$ in $G_2$. In other words, choosing to have variable $\Phi(V,W_k)$ in the $\leqslant$-msg is the same as selecting the subset $S_k$ to be part of the solution of the set cover problem. Consequently, using this transformation MSG-MIN can be used to decide SCP. Since the transformation can clearly be done in polynomial time, and since SCP is known to be NP-complete, we conclude that MSG-MIN is NP-complete as well.
	
	Now let us prove the result for LCG-MIN. We know that a $\leqslant$-lcg can be computed in polynomial time, so that a positive instance of LCG-MIN can be verified just like it can be for MSG-MIN. Moreover, the absence of non-variable terms in the transformation from SCP to MSG-MIN above allows us to reuse said transformation as-is to prove that LCG-MIN is NP-hard. Indeed, since the obtained anti-unification problem doesn't harbor terms other than variables, it is both an instance of MSG-MIN and LCG-MIN. LCG-MIN is therefore also NP-complete.
\end{proof}
	
	\section{Proof of Theorem~\ref{thm-inj-np-complete}}
\begin{proof}
	INJ is in NP: given a relation $\leqslant^\iota$, goals $G_1$ and $G_2$ and a substitution (or renaming) $\theta$, it is possible to verify in polynomial time whether the application of $\theta$ on $G_1$ results on a subset of $G_2$ or not.
	As for the proof of NP-hardness, we refer to~\cite{gen} in which the problem ``is $G_1$ a $\preceq^\iota$-lcg of $G_1$ and $G_2$?'' has been proved to be NP-complete using a polynomial reduction from the Induced Subgraph Isomorphism Problem~\cite{SYSLO198291}. The same reduction can be used for the other cases, leading to the conclusion that INJ is NP-complete.
\end{proof}
	

\end{document}
