\begin{thebibliography}{99.}%

\bibitem{StA} Aguila-Camacho, N. and Duarte-Mermoud, M.A., and
Gallegos, J.A.: Lyapunov functions for fractional order systems. 
Commun. Nonlin. Sci. Numer. Simul. \textbf{19}, 2951�-2957 (2014) 


\bibitem
{Anderson}  Anderson, J.R.: Learning and memory: An integrated approach. 
Wiley, New York 1(995)

\bibitem
{Anastas} Anastassiou, G.A.:
Nabla discrete fractional calculus and nabla inequalities.
Math. Comput. Modelling \textbf{51}, 562--571 (2010) 


%\bibitem{T2no}
%Area, I., Losada, J., and Nieto, J.J.: 
%On Fractional Derivatives and Primitives of Periodic Functions.
%Abstr. Appl. Anal. \textbf{2014},  392598 (2014)

\bibitem
{Atici1}  Atici, F. and Eloe P.:
Initial value problems in discrete fractional calculus.
Proc. Am. Math. Soc. \textbf{137}, 981--989 (2009) 

\bibitem
{Atici2}  Atici, F. and Eloe P.:
Discrete fractional calculus with the nabla operator.
Electron. J. Qual. Theory Differ. Equ. Spec. Ed. \textbf{I3}, 1--12 (2009) 

\bibitem
{StB} Baleanu, D., Wu, G.-C., Bai, Y.-R., and Chen, F.-L.: 
Stability analysis of Caputo-like discrete fractional systems.
Commun. Nonlin. Sci. Numer. Simul. \textbf{48}, 520--530 (2017) 

\bibitem 
{hdif1} Bastos, N.R.O., Ferreira, R.A.C., and Torres D.F.M.: Discrete-time fractional variational
problems. Signal Processing \textbf{91}, 513--524 (2011) 
 
\bibitem 
{hdif2} Bastos, N.R.O., Ferreira, R.A.C., and Torres D.F.M.:
Necessary optimality conditions
for fractional difference problems of the calculus of variations. Discrete Contin. Dyn. Syst.
\textbf{29}, 417--437 (2011)

\bibitem{DifSum} Chen, F., Luo, X., and Zhou, Y.:
Existence Results for Nonlinear Fractional Difference Equation.
Adv.Differ.Eq. \textbf{2011}, 713201, (2011)


\bibitem{Cvi} Cvitanovic, P.:  Universality in Chaos.
Adam Hilger, Bristol and New York (1989) 

\bibitem{ME2} Edelman,M.: 
Fractional Standard Map: Riemann-Liouville vs. Caputo.
Commun. Nonlin. Sci. Numer. Simul. \textbf{16}, 4573--4580 (2011)

\bibitem {ME3}  Edelman, M.: 
Fractional Maps and Fractional Attractors. Part I:
$\alpha$-Families of Maps.
Discontinuity, Nonlinearity, and Complexity
\textbf{1}, 305--324 (2013)

\bibitem{ME4} Edelman, M.:
Universal Fractional Map and Cascade of
  Bifurcations Type Attractors.
Chaos \textbf{23}, 033127 (2013) 

\bibitem{ME6}  Edelman, M.: Universality in fractional dynamics.
International Conference on Fractional Differentiation and 
Its Applications (ICFDA), 2014,
DOI: 10.1109/ICFDA.2014.6967376, (2014), Page(s): 1--6

\bibitem
{ME7} Edelman, M.:
Fractional Maps as Maps with Power-Law Memory.
In: Eds.: Afraimovich, A., Luo, A.C.J., Fu, X. (eds.):
Nonlinear Dynamics and Complexity;
Series: Nonlinear Systems and Complexity,    
79--120, New York, Springer (2014)

\bibitem {ME8} Edelman, M.:
Caputo standard $\alpha$-family of maps: 
Fractional difference vs. fractional.
Chaos \textbf{24}, 023137 (2014)

\bibitem{ME9} Edelman, M.:
Fractional Maps and Fractional Attractors. Part II: Fractional
Difference $\alpha$-Families of Maps.
Discontinuity, Nonlinearity, and Complexity \textbf{4}, 391--402 (2015)

\bibitem{Chaos2015} Edelman, M.: 
On the fractional Eulerian numbers and equivalence of 
maps with long  term power-law memory (integral Volterra 
equations of the second kind) to
Gr$\ddot{u}$nvald-Letnikov fractional difference (differential) equations.
Chaos \textbf{25}, 073103 (2015).

\bibitem{ME1} Edelman, M.,  Tarasov, V.E.: 
Fractional standard map.
Phys. Lett. A \textbf{374}, 279--285 (2009)

\bibitem{ME5} Edelman, M., Taieb, L.A.:   
New types of solutions of non-linear fractional differential
equations. In:  Almeida, A., Castro, L.,   Speck F.-O. (eds.)
 Advances in Harmonic Analysis and Operator Theory;   
Series: Operator Theory: Advances and Applications.
\textbf{229}, 139--155 Springer, Basel (2013)

\bibitem{Adaptation3} Fairhall, A.L., Lewen, G.D., Bialek, W.,   
de Ruyter van Steveninck R.R.: 
Efficiency and Ambiguity in an Adaptive Neural Code.
Nature  787--792 (2001)

\bibitem 
{hdif3} Ferreira, R.A.C. and Torres D.F.M.:
Fractional h-difference equations arising
from the calculus of variations. Appl. Anal. Discrete Math. \textbf{5},
110--121 (2011)

\bibitem 
{hdif3n} Frederico, G.S.F. and Torres D.F.M.:
A formulation of Noether's theorem for fractional problems of the calculus 
of variations.
J. Math. Appl. Anal. Appl. \textbf{334},
834--846 (2007)


\bibitem{GZ} Gray, H.L. and Zhang, N.-F.: On a new definition
of the fractional difference. Math. Comput.  \textbf{50}, 513--529 (1988) 


\bibitem
{Kahana}  Kahana, M.J.: Foundations of human memory.
Oxford University Press, New York (2012)


\bibitem 
{KBT1}  Kilbas, A.A., Bonilla, B.,  and Trujillo, J.J.:
Nonlinear differential equations of fractional order is space of
integrable functions.
Dokl. Math. \textbf{62}, 222--226 (2000)

\bibitem 
{KBT2}   Kilbas, A.A., Bonilla, B.,  and Trujillo, J.J.:
Existence and uniqueness theorems for nonlinear fractional
differential equations. 
Demonstratio Math. \textbf{33}, 583--602 (2000)


\bibitem
{KST} Kilbas, A.A., Srivastava, H.M., and Trujillo, J.J.: 
Theory and Application of Fractional Differential Equations.
Elsevier, Amsterdam (2006)


\bibitem{Adaptation4}  Leopold D.A., Murayama, Y,,  Logothetis, N.K.: 
Very slow activity fluctuations in monkey visual cortex: 
implications for functional brain imaging. 
Cerebr Cortex
\textbf{413}, 422--433 (2003)

\bibitem{StL} Li, Y., Chen, Y.Q., and Podlubny, I.: 
Stability of fractional-order nonlinear dynamic systems: Lyapunov direct
method and generalized Mittag-Leffler stability. 
Comput. Math. Appl. \textbf{59}, 1810�-21 (2010) 


\bibitem{Neuron3}  Lundstrom, B.N., Fairhall, A.L.,  Maravall, M.: 
Multiple time scale encoding of slowly varying whisker stimulus
envelope incortical and thalamic neurons in vivo.
J. Neurosci \textbf{30}, 5071--5077 (2010)

\bibitem{Neuron4}  Lundstrom, B.N., Higgs, M.H.,  Spain, W.J.,  
 Fairhall, A.L.:
Fractional differentiation by neocortical pyramidal neurons. 
Nat Neurosci \textbf{11},  1335--1342 (2008)

\bibitem{Mach} Machado,J.A. Tenreiro,  Pinto, Carla M.A.,
Lopes, A. Mendes: 
A review on the characterization of signals and systems
by power law distributions.
Signal Process \textbf{107}, 246--253 (2015)


\bibitem{StM} Matignon, D.: 
Stability properties for generalized fractional differential systems. 
ESAIM Proc. \textbf{5}, 145--58 (1998) 


\bibitem{May} May, R.M.: Simple mathematical models with very complicated
dynamics. Nature \textbf{261}, 459--467 (1976)



\bibitem{MR} Miller, K.S. and Ross, B.:
Fractional Difference Calculus.
In:  Srivastava, H.M. and Owa, S. (eds.) Univalent Functions, Fractional Calculus, and Their
Applications. 139--151 Ellis Howard, Chichester,  (1989)

\bibitem 
{hdif4} Mozyrska, D. and Girejko, E.: 
Overview of the fractional h-difference operators. 
In:  Almeida, A., Castro, L.,   Speck F.-O. (eds.)
 Advances in Harmonic Analysis and Operator Theory;   
Series: Operator Theory: Advances and Applications.
\textbf{229}, 253--267 Springer, Basel (2013)

\bibitem 
{hdif4n} Mozyrska, D. and Girejko, E., and Wirwas, M.: 
Fractional nonlinear systems with sequential  operators. 
Cent. Eur. J. Phys. \textbf{11}, 1295--1303 (2013)

\bibitem 
{hdif5} Mozyrska, D. and Pawluszewicz E.:
Local controllability of nonlinear discrete-time
fractional order systems. Bull. Pol. Acad. Sci. Techn. Sci. \textbf{61}, 251--256 (2013)

\bibitem 
{hdif6} Mozyrska, D., Pawluszewicz E., and Girejko, E.:
Stability of nonlinear h-difference systems with N fractional orders. Kibernetica \textbf{51}, 112--136 (2015)

\bibitem{Petras} Petras, I.: 
Fractional-Order Nonlinear Systems. 
Springer, Berlin, (2011)

\bibitem
{Podlubny} Podlubny, I.: 
Fractional Differential Equations.
Academic Press, San Diego (1999)

\bibitem{Neuron5} Pozzorini, C., Naud, R., Mensi, S., Gerstner, W.:
Temporal whitening by power-law adaptation in neocortical neurons.
Nat Neurosci \textbf{16},  942--948 (2013)	

\bibitem{StRev2013} Rivero, M., Rogozin, S.V., Machado, J.A.T., and Trujilo, J.J.: Stability of fractional order systems.
Math. Probl. Eng. \textbf{2013}, 356215 (2013)

\bibitem
{Rubin} Rubin, D.C., Wenzel, A.E.:  
One Hundred Years of 
Forgetting: A Quantitative Description of Retention.
Psychol Rev \textbf{103}, 743--760 (1996)

\bibitem
{SKM} Samko, S.G., Kilbas, A.A., and Marichev, O.I.:
Fractional Integrals and Derivatives Theory and Applications.
Gordon and Breach, New York (1993)

\bibitem{StanislavskyMaps} 
Stanislavsky, A.A: Long-term memory contribution as applied to the motion of discrete dynamical system. 
Chaos \textbf{16}, 043105 (2006) 


\bibitem
{T2009a} Tarasov, V.E.:
Differential equations with fractional derivative and universal map with memory.
J. Phys. A \textbf{42}, 465102 (2009)

\bibitem
{T2009b} Tarasov, V.E.:
Discrete map with memory from fractional differential 
equation of arbitrary positive order
J. Math. Phys. \textbf{50}, 122703 (2009)


\bibitem{T2}  Tarasov, V.E.:  
Fractional Dynamics:
Application of Fractional Calculus to Dynamics of Particles, Fields and
Media. HEP, Springer, Beijing, Berlin, Heidelberg (2011)

\bibitem{T1} Tarasov, V.E., Zaslavsky, G.M.:
Fractional equations of kicked systems and discrete maps.
J. Phys. A \textbf{41}, 435101 (2008) 

\bibitem {Adaptation2} Toib, A., Lyakhov, V.,   Marom, S.:
Interaction between duration of activity and recovery from slow 
inactivation in mammalian brain Na+ channels. 
J Neurosci \textbf{18}, 1893--1903 (1998)

\bibitem{Adaptation5}  Ulanovsky, N.,  Las, L.,  Farkas, D., Nelken, I.:   
Multiple time scales of adaptation in auditory cortex neurons. 
J Neurosci \textbf{24}, 10440--10453 (2004)

\bibitem
{Wixted1}  Wixted, J.T.:
Analyzing the empirical course of forgetting. 
J Exp Psychol Learn Mem 
Cognit  \textbf{16}, 927--935 (1990) 

\bibitem
{Wixted2} Wixted, J.T., Ebbesen, E.:   
On the form of forgetting. 
Psychol Sci  \textbf{2},  409--415 (1991). 

\bibitem{Adaptation1}  Wixted, J.T.,  Ebbesen, E.: 
Genuine power curves in forgetting. 
Mem Cognit  \textbf{25}, 731--739 (1997) 

\bibitem{FallC} Wu, G.-C., Baleanu, D.:
Discrete fractional logistic map and its chaos.
Nonlin. Dyn. \textbf{75}, 283--287 (2014)

\bibitem{Fall} Wu, G.-C., Baleanu, D., Zeng, S.-D.:
Discrete chaos in fractional sine and standard maps.
Phys. Lett. A \textbf{378}, 484--487 (2014)


\bibitem{StDis1} Wyrwas, M., Pawluszewicz, E., and Girejko, E.:
Stability of nonlinear $h$-difference systems with $N$ fractional 
orders.
Kybernetika \textbf{15}, 112--136 (2015)




\bibitem{ZasBook} Zaslavsky, G.M.: 
Hamiltonian Chaos and Fractional Dynamics.
Oxford University Press, Oxford (2005)

\bibitem{ZSE} Zaslavsky, G.M., Stanislavsky, A.A., and Edelman, M: Chaotic and pseudochaotic attractors of perturbed fractional oscillator. 
Chaos \textbf{16}, 013102 (2006) 


\bibitem{Adaptation6}  Zilany, M.S., Bruce, I.C., Nelson, P.C., Carney, L.H.:  
A phenomenological model of the synapse between the inner hair cell and
auditory nerve: long-term adaptation with power-law dynamics.
J. Acoust. Soc. Am. \textbf{126}, 2390--2412 (2009)












\end{thebibliography}


















