% \documentclass[main.tex]{subfiles}

% \begin{document}

\chapter{Tangent categories}%
\label{ch:tangent_categories}

Tangent categories are an example of convergent evolution in mathematics, in which two unrelated lines of research with very different aims have arrived at a common endpoint, in this case the same formal setting for abstract differential geometry. The older line of research has its roots in differential geometry proper and, in particular, Weil's algebraic characterization of the tangent bundle of a smooth manifold (\cite{Weil1953}). Weil's work motivated Kock and Lawvere's development of synthetic differential geometry, presented in the book of the same name \cite{Lawvere1979,Kock2006} as well as the Weil functor formalism of \cite{Kolar1993}. The second, more recent line of research has its foundations in theoretical computer science following the publication of \emph{Linear Logic} by \cite{girard1987linear}. Ehrhard and Regnier noticed that some models of linear logic have a notion of the ``Taylor series approximation'' of a proof; this led to the development of differential linear logic by \cite{ehrhard2003differential}. Blute, Cockett, and Seely studied the categorical semantics of models of linear logic equipped with the derivative operation - that is, they identified those categories whose internal language are were models of differential linear logic - developing a categorical theory of differentiation in \cite{blute2006differential,Blute2009}.


Tangent categories arise naturally in each line of research: on the first path with the distillation of synthetic differential geometry into abstract tangent functors in \cite{Rosicky1984}, and more recently when Cockett and Cruttwell refined abstract tangent functors following their investigations into the manifold categories of cartesian differential restriction categories in \cite{MR2861119,Cockett2014}. In a sense, they are categories that axiomatize Weil's characterization of the tangent bundle as an endofunctor in a way that captures the combinatorics of higher-order derivatives when looking at a certain class of internal commutative monoids (\cite{cockett2011faa}), as will be made precise in Chapter \ref{chap:weil-nerve}. Tangent categories also pull Kock and Lawvere's synthetic differential geometry into the framework of enriched category theory, which is explored in Chapter \ref{ch:inf-nerve-and-realization}.

Instances of tangent structure abound throughout mathematics and computer science. For example,  many categories of geometric spaces have natural tangent structure, such as the category of \emph{convenient manifolds} (the category of manifolds modelled locally by \emph{convenient vector spaces} \cite{Kriegl1997}) and the category of schemes (see point (ii) in Example 2 of \cite{Garner2018}). An example from mathematical logic is the category of K\"{o}the sequence spaces( \cite{MR1934421}), and categorical models of the differential lambda-calculus (\cite{MR4037417}). More recently, tangent and differential categories have found applications in differentiable programming and machine learning (\cite{wilson2021reverse}), and to understanding Johnson and McCarthy's \emph{functor calculus} \cite{Bauer2016}.

This thesis studies differential geometric structures using the language of tangent categories, following the tradition of synthetic differential geometry. As such, this chapter will develop tangent categories with a focus on the category of smooth manifolds. Extending the study of these formal structures in the context of novel tangent categories is a significant endeavour and should be treated as a direction for future research. The first section introduces Cartesian differential categories as the categorical axiomatization of multivariable calculus. The second section introduces the category of smooth manifolds and two characterizations of its tangent bundle (kinematic versus operational),  while the third section identifies the structure of the kinematic tangent bundle that characterizes abstract tangent structures. The fourth section presents a pair of structures that allow for ``local-coordinate calculations'' in the tangent category of differential objects and connections. The final section introduces \emph{tangent submersions}. A submersion is a differentiable map between differentiable manifolds whose differential is everywhere surjective; as a preview of the work in Chapters \ref{ch:differential_bundles} and \ref{ch:involution-algebroids}, this section shows that in the category of smooth manifolds, a tangent submersion is precisely a submersion. Section \ref{sec:submersions} first appeared in \cite{MacAdam2021}, and is the only original work in this chapter.


\section{Differential calculus}%
\label{sec:differential-calculus}


As with most treatments of synthetic differential geometry, e.g. \cite{Kock2006}, it makes sense to begin with the differential calculus - in this case, an introduction to the categorical theory of differentiation. Categorical differentiation has recently gained quite a bit of attention due to its relationship with machine learning \cite{Cockett2019}, and applications to homotopy theory \cite{Bauer2016}. This section will just consider the basic structures introduced in \cite{Blute2009}, and the canonical example of a cartesian differential category (the category of finite-dimensional real vector spaces and smooth maps between them).
% Note: This section introduction seems reasonable based on the chapter introduction rewrite

\begin{definition}\label{def:clac}[Definition 1.2.1 \cite{Blute2009}]
    % NOTE: footnote was added to define cartesian category notation, addition & 0 notation was clarified
    A cartesian left additive category is a cartesian category\footnote{We use the standard notation where $1$ is the terminal object, $\x$ is product, and $\pi_i$ is the $i^{th}$ projection.} $\C$ so that:
    \begin{enumerate}[(i)]
        \item Each hom-set $\C(A,B)$ is a commutative monoid with addition $+_{AB}$ and zero map $0_{AB}:A \to B$ (the subscript $AB$ will be suppressed when the context is clear).
        \item The composition operation $\o$ preserves addition \textit{on the left}: \[(g+h) \o f= g \o f + h \o f\]
        \item Projection is an additive map (preserves addition): \[\pi_i \o (f + g) = (\pi_i \o g \o f) + (\pi_i \o g)\]
        Where $\pi_i$ denotes the projection from the $i^{th}$ component of a product or pullback.
j    \end{enumerate}
\end{definition}
There are various examples of cartesian left additive categories - they all fit the same pattern of a category where each object is equipped with a non-natural, but coherent, choice of linear structure:
\begin{example}\label{ex:clacs}
    ~\begin{enumerate}[(i)]
        \item Any category with biproducts is a cartesian left additive category where every map is additive. 
        \item The category of cartesian spaces $\mathsf{CartSp}$, whose objects are finite-dimensional real vector spaces and morphisms are smooth maps between them, is a cartesian left additive category."
        Clearly smooth maps from $A \to B$ are closed under addition, projection is an additive map, and $(g+h) \o f= g \o f + h \o f$.
        \item   The category of topological vector spaces and continuous morphisms is a cartesian left additive category. 
        % We note that each object has chosen commutative monoid structure satisfying the coherences in point (2) of Proposition \ref{prop:clac-defs}.     
    \end{enumerate}
\end{example}
In fact, cartesian left additive categories may equivalently be described as cartesian categories where each object has a coherent choice of commutative monoid structure.
\begin{proposition}[\cite{Blute2009}]%
    \label{prop:clac-defs}
    The following are equivalent:
    \begin{enumerate}[(i)]
        \item $\C$ is a cartesian left additive category
        \item $\C$ is a cartesian category so that each object has a chosen commutative monoid structure $(A,+_A,0_A)$ where the following coherence holds: \input{TikzDrawings/Ch1/clac-coh.tikz} (where $\tau = ((\pi_0\o \pi_0,\pi_0\o \pi_1),(\pi_1\o\pi_0, \pi_1\o\pi_1))$).
        \item There is a category with biproducts $\C^+$ and a bijective-on-objects subcategory inclusion $i: \C^+ \hookrightarrow\C$ that creates products.
    \end{enumerate}
\end{proposition}

Cartesian left additive categories provide an appropriate to define a differentiation operation. Recall that the usual derivative of a map $f: \R \to \R$ from elementary calculus can be written
\[
    \frac{\partial f}{\partial x}: \R \to \R. 
\]
More generally, for a map $f: \R^n \to \R$, one writes the \emph{Jacobian} of $f$ at $x$:
\[
    J[f]: \R^n \to (\R^n \multimap \R^m) :=
    \begin{bmatrix}
        \frac{\partial f_1}{\partial x_1} & \dots & \frac{\partial f_1}{\partial x_n} \\
        \vdots & \ddots & \vdots \\
        \frac{\partial f_1}{\partial x_1} & \dots & \frac{\partial f_1}{\partial x_n} 
    \end{bmatrix}
\]\pagenote{tidied notation, removed jargon}
The Jacobian, however, requires some notion of a ``matrix''\footnote{This would be called an \emph{internal hom} in the 
 categorical logic literature.} representing a linear map from $\mathbb{R}^n$ to $\mathbb{R}^m$---not every category that has a notion of differentiation supports that operation. Instead, the \emph{directional derivative}:
\[
    D[f](x,v) := \lim_{d \to 0} \frac{f(x + t\cdot v)}{t}
\]
gives an appropriately general notion of differentiation that extends to categories where the space of linear maps $A \to B$ is not representable by an object in the category \footnote{In the case of automatic differentiation, it is also worth noticing that computing the directional derivative of a map $\mathbb{R}^n \to \mathbb{R}^m$ has complexity $2\mathcal{O}(f)$, while forming the Jacobian has complexity $n\mathcal{O}(f)$, so the directional derivative is a more appropriate primitive for purely practical computational reasons (see Section 5 of \cite{hoffmann2016hitchhiker} for a discussion of the computational complexity of forward-mode automatic differentiation).}.\pagenote{Spelled out AD.} 
A cartesian differential category axiomatizes the directional derivative as a combinator on a cartesian left-additive category.


\begin{definition}\label{def:cdc}[Definition 2.1.1 in \cite{Blute2009}]
    A cartesian differential category is a cartesian left additive category equipped with a combinator (e.g. a function on hom-sets)\pagenote{
    I have added a precise reference to definition of a CDC, and included a brief definition for what a combinator is.}
    %I have never seen "combinators" formally defined...
    \[
        \infer{A \x A \xrightarrow[{D[f]}]{} B}{A \xrightarrow{f} B}
    \]
    satisfying the following axioms:
  \begin{enumerate}[{[CD.1]}]
      \item Additive:
        \[D[f+g] = D[f] + D[g] \hspace{1cm} D[0] = 0\]
      \item Additive in the second variable:
        \[D[f] \o (g, h+k) = D[f]\o (g,h) + D[f]\o (g,k) \hspace{1cm} D[f] \o (g,0) = 0\]
      \item Projection is linear:
        \[D[\pi_i] =  \pi_i \o \pi_1 \hspace{1cm} D[id] = \pi_1\]
      \item Pairing:
        \[D[(f,g)] = (D[f],D[g])\]
      \item Chain rule:
        \[D[g\o f] = D[g] \o (\pi_0, D[f])\]
      \item Linear in the second variable:
        \[D[D[f]] \o ((a,0),(0,d)) = D[f] \o (a,d)\]
      \item Symmetry of partial differentiation:
        \[D[D[f]] \o ((a,b),(c,d)) = D[D[f]] \o ((a,c),(b,d))\]
  \end{enumerate}
\end{definition}
\begin{example}
    The category of cartesian spaces (Example \ref{ex:clacs}(ii)), $\mathsf{CartSp}$, is the canonical cartesian differential category. Let $f: \mathbb{R}^n \to \mathbb{R}^m $, and consider its Jacobian at $x \in \mathbb{R}^n , J[f](x) \in \mathbb{R}^{n \x m}$.  Define the differential combinator:
	    \[
	        D[f]\o (u,v) = J[f](v) \cdot u = \lim_{t \to 0} \frac{f(x + t\cdot v)}{t}.
	    \]\pagenote{Removed extra examples, since they distracted from the original point.}
	\item In \cite{Bauer2016}, the authors construct a cartesian differential category based on the Abelian functor calculus of \cite{MR1451606}.
	\item In \cite{wilson2021reverse}, the authors consider a cartesian differential category whose objects are $\mathbb{Z}_2$-modules to apply gradient-based methods to learn the parameters of of Boolean circuits. 
	\pagenote{
	We have added to 
	}
	% \pagenote{added some extra examples w/references.}
\end{example}
Every cartesian differential category comes with a notion of linearity.
This notion of linearity is strictly stronger than additivity - there do exist examples of non-linear additive maps.
\begin{definition}[Definition 2.2.1 of \cite{Blute2009}]\label{def:linear-map-in-cdc}
    A map $f:A \to B$ is \textit{linear} whenever $D[f] \o (0_{AB},id) = f$. 
    \pagenote{
        I have added a precise reference to the definition of a linear map, and the $0$ has now been defined in the definition of a cartesian left additive category.
    }
\end{definition}
We denote the category of linear maps in a cartesian differential category $\C$ as $\mathsf{Lin}(\C)$.
The category $\mathsf{Lin}(\C)$ will have biproducts, and will be a cartesian differential subcategory of $\C$.
\begin{lemma}[Corollary 2.2.3 in \cite{Blute2009}]\label{lem:collected-remarks-lin}\pagenote{
    I have added the reference to this lemma, this was originally missing.
}
    Let $\C$ be a cartesian differential category, and denote its category of linear maps as $\mathsf{Lin}(\C)$.
    \begin{enumerate}[(i)]
        \item Linear maps preserve addition.
        \item The category $\mathsf{Lin}(\C)$ is a bijective-on-objects subcategory of $\C$ with biproducts, and the inclusion $\mathsf{Lin}(\C) \hookrightarrow \C$ creates products. 
        \item Every category with biproducts is a cartesian differential category, where
        \[
            D[f] = f \o \pi_1,
        \]
        and this differential structure makes the inclusion $\mathsf{Lin}(\C) \hookrightarrow \C$ preserve the left additive structure and differential combinator (that is, it is a cartesian differential functor). 
    \end{enumerate}
\end{lemma}


\section{The category of smooth manifolds}%
\label{sec:smooth-manifolds}

% {\color{red}
%     This thesis applies abstract methods to structures arising in differential geometry using the tangent bundle.
%     It is important, then, to set a working definition of the category of smooth manifolds and their tangent bundle construction, along with the universal properties satisfied by this construction.
% }

Tangent categories axiomatize a more general structure than differential calculus, one in which spaces are only ``locally linear.''
The category of smooth manifolds gives the historically canonical example, and a good portion of this thesis relates to structures internal to that category,
so it seems worthwhile to set a working definition for that context.
We follow \cite{Tu2011}, and allow for disconnected components of a manifold to have different dimensions.
% The material in this section is classical and may be found in \cite{Kolar1993,Tu2011}.

\begin{definition}%
\label{def:smooth-manifold}[Definitions 5.5--5.7 in \cite{Tu2011}]
    A \emph{chart} on a topological space $M$ is pair $(U_i, \phi_i:U_i \hookrightarrow \R^n)$, where $U_i$ is an open subset $U_i \subseteq M$ and $\phi_i:U_i \to \R^n$ is a local homeomorphism.
    An \emph{atlas} is a collection of charts $\{ (U_i,\phi_i:U_i \to \R^n) | i \in I \}$ (where $n$ is fixed for each connected component of $M$) so that for each $i,j \in I$, the transition function $\psi_{i,j}$ that completes the diagram
    % https://q.uiver.app/?q=WzAsNCxbMCwxLCJVX3tpfSBcXGNhcCBVX2oiXSxbMSwxLCJVX2oiXSxbMCwwLCJcXHBoaV97aX1eey0xfShVX3tpfSBcXGNhcCBVX2opIl0sWzEsMCwiXFxSXm4iXSxbMiwwLCJcXHBoaV9pIl0sWzAsMSwiXFxzdWJzZXRlcSJdLFsxLDMsIlxccGhpX2oiXSxbMiwzLCJcXHBzaV97aSxqfSIsMCx7InN0eWxlIjp7ImJvZHkiOnsibmFtZSI6ImRhc2hlZCJ9fX1dXQ==
    \[\begin{tikzcd}
        {\phi_{i}^{-1}(U_{i} \cap U_j)} & {\R^n} \\
        {U_{i} \cap U_j} & {U_j}
        \arrow["{\phi_i}", from=1-1, to=2-1]
        \arrow["\subseteq", from=2-1, to=2-2]
        \arrow["{\phi_j}", from=2-2, to=1-2]
        \arrow["{\psi_{i,j}}", dashed, from=1-1, to=1-2]
    \end{tikzcd}\]
    is a smooth map.
    \begin{figure}
        \centering
        \input{TikzDrawings/Ch1/chart-diagram.tikz}     
        \caption{Overlapping charts in an atlas (Credit for this tex code belongs to user Cragfelt \url{https://tex.stackexchange.com/a/388493/101171}.)}
        \label{fig:overlapping-charts}
    \end{figure}
    A \emph{smooth manifold} is a topological space equipped with a (maximal\footnote{With respect to subset inclusion.}) atlas. \pagenote{This definition was originally incomplete, and it lacked a reference. I have added the missing details and included a reference.
    } A morphism of smooth manifolds is a topological map $f: M \to N$ that is locally smooth - for each chart pair of charts $(U_i, \phi_i)$ on $M$ and $(V_j, \theta_j)$, see Figure \ref{fig:overlapping-charts} for an illustration. The map $f$ is smooth whenever each map $f_{i,j}$ that completes the diagram is smooth 
    % https://q.uiver.app/?q=WzAsNCxbMCwwLCJVX2kiXSxbMCwxLCJNIl0sWzEsMSwiTiJdLFsxLDAsIlZfaiJdLFswLDFdLFszLDJdLFsxLDIsImYiXSxbMCwzLCJmX3tpLGp9IiwwLHsic3R5bGUiOnsiYm9keSI6eyJuYW1lIjoiZGFzaGVkIn19fV1d
    \[\begin{tikzcd}
        {U_i} & {V_j} \\
        M & N
        \arrow[from=1-1, to=2-1]
        \arrow[from=1-2, to=2-2]
        \arrow["f", from=2-1, to=2-2]
        \arrow["{f_{i,j}}", dashed, from=1-1, to=1-2]
    \end{tikzcd}\]

    The \emph{category of smooth manifolds}, $\mathsf{SMan}$, is the category of smooth manifolds and their morphisms.
\end{definition}
\begin{remark}
    In \Cref{ch:differential_bundles}, some results will implicitly use partition of unity arguments, which require that the underlying topological space for a manifold is \emph{Hausdorff} and has a countable basis (i.e. it is \emph{second-countable}). We will avoid any direct reference to these properties, and so we omit them from the definition of a smooth manifold.
\end{remark}

\begin{example}
    ~\begin{enumerate}[(i)]
        \item Each vector space $\R^n$, for $n\in \N$, has a canonical smooth manifold structure whose atlas is a single chart (the identity map $\mathbb{R}^n \to \mathbb{R}^n$).
        \item Most geometric shapes that do not have any singularities or sharp edges can be equipped with an atlas without any issue.
        For example, consider the circle:  $\{ (\cos(x), \sin(x)) : x \in [-\pi, \pi) \}$
        \[\input{TikzItDrawings/circle-plain.tikz}\]
        For any appropriately small $\epsilon>0$, there are two charts from $I_\epsilon = (-\epsilon, \pi+\epsilon)$; 
        \[
            \phi_0(t) = (\cos(t), \sin(t)), \hspace{0.25cm} \phi_1(t) = (\cos(t-\pi), \sin(t-\pi))
        \]
        making the circle a smooth manifold. \pagenote{
           I have added more details to the chart maps here.
        }
    \end{enumerate}
\end{example}
Category theory has not seen as many applications in differential geometry as it has in topology or algebra, likely because the category of smooth manifolds (Definition \ref{def:smooth-manifold}) is somewhat poorly behaved.
The two main reasons that the category of smooth manifolds is ``inconvenient'' are as follows:
\begin{itemize}
    \item The set of smooth maps between two manifolds $M, N$ fails, in general, to form a smooth manifold; thus, the category is not cartesian closed.
    \item The category of manifolds does not have quotients or arbitrary fibre products.
    \pagenote{
       As per Kristine's recommendation, we give more specific limits and colimits that fail to exist in the category of smooth manifolds.
    }
\end{itemize}
However, the category of smooth manifolds admits some limits, for example finite products.
\begin{proposition}[1.12 \cite{Kolar1993}]
    The category \emph{SMan} of smooth manifolds has finite products.
\end{proposition}
\begin{proof}
    Given two manifolds $M, N$, take the product of their underlying topological spaces together with the product charts
    \[
        (\phi_i \x \psi_j): U_i \x V_j \to \R^n \x \R^m.
    \]
\end{proof}
The category of smooth manifolds---using our definition where disconnected components of a manifold may have different dimensions---does have a class of (co)limits used throughout this paper, namely idempotent splittings.\pagenote{
   I have added the definition of an idempotent splitting.
}
\begin{definition}[\cite{MR850528}]
    An \emph{idempotent} is an endomorphism $e:E \to E$ so that $e \o e = e$. The \emph{splitting} of an idempotent is given by a pair of maps $e = s\o r$ so that $r \o s = id$. The existence of a splitting of an idempotent $e$ is equivalent to asking that the following pair of parallel arrows has a (co)equalizer:
    % https://q.uiver.app/?q=WzAsMixbMCwwLCJFIl0sWzEsMCwiRSJdLFswLDEsImUiLDAseyJvZmZzZXQiOi0xfV0sWzAsMSwiIiwyLHsib2Zmc2V0IjoxLCJsZXZlbCI6Miwic3R5bGUiOnsiaGVhZCI6eyJuYW1lIjoibm9uZSJ9fX1dXQ==
    \[\begin{tikzcd}
        E & E
        \arrow["e", shift left=1, from=1-1, to=1-2]
        \arrow[shift right=1, Rightarrow, no head, from=1-1, to=1-2]
    \end{tikzcd}\]
    The idempotent splitting $\C$ of a category (also known as the \emph{Cauchy completion} of the category), is the full subcategory of presheaves $[\C^{op}, \s]$ that are retracts of representable functors.
    Any functor into a category with idempotent splittings will factor through this inclusion of categories, so it is the \emph{free cocompletion} of $\C$ under idempotent splittings.
\end{definition}
\begin{proposition}[\cite{MR1003203}]
    The category of smooth manifolds is the idempotent splitting of the category whose objects are open subsets of Cartesian spaces and whose morphisms are smooth maps $f: U \to V$.
\end{proposition}
% A natural benefit to this result is that the category of smooth manifolds is closed to idempotent splittings.
An idempotent in the idempotent splitting of a category is also an idempotent in the base category, and thus admits a splitting.
\begin{corollary}
    The category of smooth manifolds is closed to idempotent splittings: for every map $e:M \to M$ so that $e = e \o e$ there exists a pair of maps $r:Q \to M, s:M \to Q$ so that $e = s\o r, r \o s = id$.\pagenote{
       I have added the coherences on $r,s$, so that $r \o s = id, s \o r = e$.    }
\end{corollary}

The main construction of interest on the category of smooth manifolds, for tangent categories at least, is the \emph{tangent bundle}. Given  a smooth map $f:M \to N$, restrict it to a morphism between coordinate patches, so it may be regarded as a map $f|_U: U \to V$, where $U \subseteq \R^m, V \subseteq \R^n$. This gives a local derivative operation (remembering that $\pi_i$ denotes the $i^{th}$ projection from a product $\prod^n A_i$)\footnote{
The wording here was originally muddled, it has since been corrected.
}
% Recall that the derivative of a map $f: U \to V$, for $U \subseteq \R^m, V \subset \R^n$ is given by:
\begin{gather*}
    \infer{(f|_U \o \pi_0, D[f|_U]):U \x \R^m \to V \x \R^n.}{ D[f|_U]: U \x \R^m \to \R^n,}
\end{gather*}
% By the definition of a smooth map, for each $m \in M$ there is a local homeomorphism $m \in U_m \hookrightarrow \R^m$, and $f(m) \in V_{f(m)} \hookrightarrow \R^n$. Thus there is a derivative map:
% \[
%     T(f|_m) = (f|_m \o \pi_0, D[f]): U \x \R^m \to V \x \R^m
% \]
The tangent bundle makes this construction global; that is, there is a functor $T:\mathsf{SMan \to SMan}$ giving an assignment \[T.M \xrightarrow{T.f} T.N\] that agrees with the local derivative on coordinate patches of $M, N$.

\begin{definition}[\cite{Kolar1993}]%
    \label{def:tang-vector}
    Write the algebra of smooth functions on a manifold as $C^\infty(M):=\mathsf{SMan}(M,\R)$.
    The set $\mathsf{SMan}(\R, M)/\cong$ of tangent vectors on a smooth manifold $M$ comprises the curves $\R \to M$ subject to the equivalence relation that for a pair of curves $\phi, \theta:\R \to M$, $\phi \cong \theta$ if and only if $\phi(0) = \theta(0)$ and for every $f \in C^\infty(M)$,
    \[
        \frac{\partial f \o \phi}{\partial x}(0) = \frac{\partial f \o \theta}{\partial x}(0).
        % D[f \o \phi]\o (0,id) = D[f \o \theta] \o (0,id))
    \] 
    The set $\mathsf{SMan}(\R,M)/\cong$ has a naturally determined smooth manifold structure which we call the \emph{tangent bundle} over $M$, $TM$.\pagenote{
    The original definition forgot to set the notation $TM$ for the tangent bundle of $M$.}
\end{definition}

\begin{example}
    ~\begin{enumerate}[(i)]
        \item For a vector space, the space of linear paths crossing through a point $v \in V$ is isomorphic to $V$, so $TV \cong V \x V$. 
        \item The tangent bundle above the circle is diffeomorphic to the cylinder. This is follows from the classical result that a tangent vector on the circle must be perpendicular to its position vector.
    \end{enumerate}
\end{example}

The tangent bundle lifts the ``local derivative'' into a globally defined construction, so the tangent bundle construction is functorial.\pagenote{
   This proof has been cleaned up, it was originally quite messy.
}
\begin{proposition}
    The tangent bundle is a product-preserving endofunctor on the category of smooth manifolds.
\end{proposition}
\begin{proof}
    Functoriality follows by showing that a morphism of smooth manifolds preserves the equivalence relation on curves that defines a tangent vector: 
    \[
        \forall g \in C^\infty(M),  \frac{\partial g \o \phi}{\partial x}(0) = \frac{\partial g \o \theta}{\partial x}(0)
    \]
    Note that if $f:N \to M \in C^\infty(N)$, so that $g \o f \in C^\infty(M)$, the chain rule ensures that
    \[
        \forall g \in C^\infty(N), %D[(f \o g) \o \phi] \cong   D[(f \o g) \o \theta]
        \frac{\partial f\o g \o \phi}{\partial x}(0) = \frac{\partial f\o g \o \theta}{\partial x}(0)
    \]
    % Given $\theta \cong \phi$, look at an open chart $U$ containing $\theta(0) = \phi(0)$, $(f|_U \o \theta)'(0) = (f|_U \o \phi)'(0)$ follows by the chain rule.

    To show that $T$ is product-preserving, it suffices to show that the equivalence classes of curves are stable under pairing. First, note that for any $M$ and $\phi \cong \theta:\R \to M$, 
    \[
        (\phi,id) \cong (\theta,id): \R \to M\x\R
    \] 
    Given a pair of curves $\theta_M, \psi_M:\R \to M$, where $\theta_M \cong \psi_M$ and similarly $\theta_N \cong \psi_N$ for $N$, this implies that
    \pagenote{Cleaned up proof per Kristine's comments.}
    \begin{gather*}
        f \o (\phi_M, \phi_N) \\
        = f \o (\phi_M \x id) \o (id, \phi_N) \\
        = f \o (\phi_M \x id) \o (id, \theta_N) \\
        = f \o (id, \theta_N) \o (\phi_M \x id) \\
        = f \o (id, \theta_N) \o (\theta_M \x id) \\
        = f \o (\theta_M \x \theta_N)
    \end{gather*}
    so $(\phi_M, \phi_N) \cong (\theta_M, \theta_N)$.
    % \[
    %     \left(\forall f_M \in C^\infty(M), \frac{\partial f_M \o \phi_M}{\partial x}(0) = \frac{\partial f_M \o \theta_M}{\partial x}(0) \right)
    %     %D[f_M \o \theta_M] = D[f_M \o \phi_M]\right)
    %     \wedge
    %     \left(\forall f_N \in C^\infty(N),\frac{\partial f_N \o \phi_N}{\partial x}(0) = \frac{\partial f_N \o \theta_N}{\partial x}(0) \right)
    %         %\forall f_N \in C^\infty(N), D[f_N \o \theta_N] = D[f_N \o \phi_N]\right)
    % \]
    % Suppose there exists a function $f \in C^\infty(M \x N)$ so that 
    % \[
    %     \frac{\partial f \o (\phi_M,\phi_N)}{\partial x}(0) = \frac{\partial f \o (\theta_M,\theta_N)}{\partial x}(0)
    %     % D[f \o (\theta_M, \theta_N)](0,(x,y)) \not= D[f \o (\phi_M, \phi_N)](0,(x,y))
    % \] 
    % But this would mean that $f \o (id, ) 
    % then for $x=0$, in particular, this means:
    % \[
    %     D[f \o (\theta_M, \theta_N)](0,(0,y)) \not= D[f \o (\phi_M, \phi_N)](0,(0,y))
    % \] 
    % This is the partial derivative of $f \o (\theta_M, \theta_N)$ in the second variable at $(0,0)$, which may be rewritten:
    % \[
    %     D[f \o (\theta_M\o 0, \theta_N)] \o (0,y) \not= D[f\o (\phi_M\o 0, \phi_N)] \o (0,y)
    % \]
    % so that:
    % \[
    %     D[f \o (m, id) \o \theta_N]\o (0,id) \not= D[f \o (m, id) \o \phi_N] \o (0,id)
    % \]
    % then, $f' := f \o (m,id) \in C^\infty(M)$ would violate the original hypothesis that $\theta_N \cong \phi_N$. Therefore, the equivalence relation is stable under pairing.
\end{proof}


The scalar action by $\R$ on tangent vectors and a partially defined addition additionally give the tangent bundle the structure of a fibered $\R$-module (that is, an $\R$-module in the slice category $\mathsf{SMan}/M$ whose objects are morphisms into $M$, $f:X \to M$, and morphisms are commuting triangles).
\begin{proposition}
    The tangent bundle over $M$ is an $\R$-module in $\mathsf{SMan}/M$, as follows:
    let $\gamma, \omega$ be tangent vectors on $M$, and define
    \begin{itemize}
        \item $p: TM \to M; p(\gamma) = \gamma(0)$.
        \item $0: M \to TM; 0(m) =  [r \mapsto m]$ (the constant map $\R \to M$ sending all $r\in \R$ to $m\in M$)
        \item $\cdot_p: TM \x \R \to TM; \gamma \cdot_p r = [x \mapsto \gamma(r \cdot x)]$
        \item $+: TM \ts{p}{p} TM \to TM := [\gamma],[\omega] \mapsto [\gamma + \omega]$ (where addition around $\gamma(0)=\omega(0)$ is defined using local coordinates). \pagenote{The notation in this proposition has been changed to be more clear, and it was clarified that addition is defined using local coordinate charts around points in $TM$.}
    \end{itemize}
\end{proposition}

The second derivative is involved in the more nuanced axioms for a cartesian differential category, namely linearity in the vector argument and the symmetry of mixed partial derivatives.  First, set $f|_U$ to be the restriction of $f:M \to N$ to a map between local coordinate patches $U \subseteq M, V \subseteq N$, and then define
\[
  f_0 = f|_U \o \pi_0 \o \pi_1, \hspace{0.25cm},
\]
and
\[
   f_1 = D[f] \o (\pi_0 \o \pi_0, \pi_1 \o \pi_0), \hspace{0.25cm}
   f_2 = D[f] \o (\pi_0 \o \pi_0, \pi_0 \o \pi_1) \hspace{0.25cm}.
\]
The axioms $[CDC.6],[CDC.7]$ then give:\pagenote{
    I have added the definition of the maps ahead of the diagrams to avoid any confusion, and made a more direct reference to the CDC axioms.
}
% (setting \[f_0 = f|_U\o\pi_0\o\pi_0, f_1 = D[f|_U]\o \pi_0, f_2 = D[f|_U]\o (\pi_0\o\pi_0, \pi_0 \o \pi_1)\] so the diagrams fit on a single page)
\begin{center}
    \input{TikzDrawings/Ch1/local-coord-lift-diagram.tikz}
    \input{TikzDrawings/Ch1/local-coord-flip-diagram.tikz}
\end{center}
% Similarly, a tangent vector in $T^3M$ is then:
% \begin{align*}
%     \phi \cong \theta :\R^3 \to M
%     \iff   &  \left(\phi(0,0,0) = \theta(0,0,0) \right) \\
%     \wedge &  \forall f \in C^\infty(M), \frac{\partial f \o \phi}{\partial x_i}(0,0,0) = \frac{\partial f \o \phi}{\partial x_i}(0,0,0), i = 0,1,2
% \end{align*}
% A tangent vector in $T^3M$ is then:
% \begin{align*}
%     \phi \cong \theta :\R^3 \to M
%     \iff   &  \left(\phi(0,0) = \theta(0,0)\right) \\
%     % \wedge &  \left(\forall f \in C^\infty(M), D[f \o \phi] \o (0, x) = D[f \o theta] \o (0,x) \right) \\
%     \wedge &  \left(\forall f \in C^\infty(M), D[D[f \o \phi]] \o ((0,x),(y,z)) = D[D[f \o theta]] \o ((0,x),(y,z)) \right) \\
% \end{align*}

% A tangent vector in $T^3M$ is a map $\R^3 \to M$ satisfying similar coherences.
The two natural transformations $\ell$ and $c$---the vertical lift and canonical flip---capture these coherences.
Locally, $\ell$ is the map inserting zeros into the second and third coordinates, while $c$ flips the second and third arguments, leading to the coherences established in the next proposition.
To capture these coherences on the tangent bundle, first note that a tangent vector on $TM$ is equivalent to an equivalence class of surfaces on $M$,\pagenote{I have clarified the equivalence relation that defines the second tangent bundle by using normal calculus notation.}
\begin{align*}
    \phi \cong \theta :\R^2 \to M
    \iff   &  \phi(0,0) = \theta(0,0) \\
    % \wedge &  \left(\forall f \in C^\infty(M), D[f \o \phi] \o (0, x) = D[f \o theta] \o (0,x) \right) \\
    \text{and } &  \forall f \in C^\infty(M), \frac{\partial f \o \phi}{\partial x_i}(0,0) = \frac{\partial f \o \theta}{\partial x_i}(0,0), i = 0,1
    % \wedge &  \forall f \in C^\infty(M), D^2[f \o \phi] = D^2[f \o \theta] \big) \\
\end{align*}

\begin{proposition}[\cite{Cockett2014}]
    There are two natural transformations\pagenote{I have fixed the notation in this proposition.}
    \[
        \ell: TM \to T^2M; \ell([\gamma]) = [\gamma \o (\pi_0 \cdot_\R \pi_1)] \hspace{0.5cm}
        c: T^2M \to T^2M; c([\gamma]) = [\gamma \o (\pi_1, \pi_0)]
    \]
    satisfying the following coherences:
    \begin{enumerate}[(i)]
        \item $\ell.T \o \ell = T.\ell \o \ell$\footnote{Recall that we are using the 2-categorical notation described in the front-matter}
        \item The following maps are morphisms of fibred $\R$-modules. 
        % https://q.uiver.app/?q=WzAsNCxbMCwwLCJUTSJdLFswLDEsIk0iXSxbMSwwLCJUXjJNIl0sWzEsMSwiVE0iXSxbMCwyLCJcXGVsbCJdLFswLDEsInAiXSxbMiwzLCJwLlQiXSxbMSwzLCIwIl1d
        \[\begin{tikzcd}
            TM & {T^2M} \\
            M & TM
            \arrow["\ell", from=1-1, to=1-2]
            \arrow["p", from=1-1, to=2-1]
            \arrow["{p.T}", from=1-2, to=2-2]
            \arrow["0", from=2-1, to=2-2]
        \end{tikzcd}% https://q.uiver.app/?q=WzAsNCxbMCwwLCJUTSJdLFswLDEsIk0iXSxbMSwwLCJUXjJNIl0sWzEsMSwiVE0iXSxbMCwyLCJcXGVsbCJdLFswLDEsInAiXSxbMiwzLCJULnAiXSxbMSwzLCIwIl1d
        \begin{tikzcd}
            TM & {T^2M} \\
            M & TM
            \arrow["\ell", from=1-1, to=1-2]
            \arrow["p", from=1-1, to=2-1]
            \arrow["{T.p}", from=1-2, to=2-2]
            \arrow["0", from=2-1, to=2-2]
        \end{tikzcd}\]
        \item $c \o c = id$
        \item $T.c \o c.T \o T.c = c.T \o T.c \o c.T$
        \item $c \o \ell = \ell$
        \item $T.c \o c \o T.\ell = \ell.T \o c$.
    \end{enumerate}
\end{proposition}
\pagenote{Removed the proofs - they are in the literature and they proved to be a distraction.}
% \begin{proof}
%     We sketch some of the proofs.
%     \begin{itemize}
%         \item[(i)] This is a consequence of the associativity of addition - we have:
%         \[
%             \ell.T \o \ell [\gamma] = \ell.T() =  [\gamma(\pi_0 \cdot (\pi_1\cdot \pi_2) )] = T.\ell \o \ell [\gamma]
%         \]
%         \item[(iv)] Note that $T.c \o ([\gamma]) = [\gamma(\pi_0, \pi_2,\pi_1)]$ and $c.T \o ([\gamma]) = [\gamma(\pi_1, \pi_0,\pi_2)]$
%         Each side of the equation becomes $\gamma(\pi_2, \pi_1, \pi_0)$, so the equation holds.
%     \end{itemize}
% \end{proof}
\begin{observation}
\label{obs:yb-eq}
    Equation (iv) is known as the \emph{Yang--Baxter} equation. 
    It is one of the coherences for a symmetric monoidal category, and states that the twisting operation between two variables is coherent.
    We may regard the category of endofunctors on a category as a strict monoidal category and use string diagram notation (see e.g. \cite{selinger2010survey}).
    Interpreting the map $c$ as twisting two strings, the coherence becomes \pagenote{I added a reference to string calculus.}
    \begin{center}
        \input{TikzDrawings/Ch1/YangBaxterString.tikz}
    \end{center}
\end{observation}

The projection $p:TM \to M$ is \emph{locally trivial}: for each connected component of $M$ that is modeled on $\R^n$, each point $m$ lies in an open subset $U_m$ so that $p^{-1}(U_m) \cong U_m \x \R^n$. 
This local triviality property leads to the following universality condition.
\begin{proposition}\label{prop:ell-universal}
    The following diagram is an equalizer:
    % https://q.uiver.app/?q=WzAsMyxbMCwwLCJUTSJdLFsxLDAsIlReMk0iXSxbMiwwLCJUTSJdLFsxLDIsInAiLDFdLFsxLDIsIlQucCIsMCx7Im9mZnNldCI6LTJ9XSxbMSwyLCIwXFxvIHAgXFxvIHAiLDIseyJvZmZzZXQiOjJ9XSxbMCwxLCJcXGVsbCIsMl1d
    \[\begin{tikzcd}
        TM & {T^2M} & TM
        \arrow["p"{description}, from=1-2, to=1-3]
        \arrow["{T.p}", shift left=2, from=1-2, to=1-3]
        \arrow["{0\o p \o p}"', shift right=2, from=1-2, to=1-3]
        \arrow["\ell"', from=1-1, to=1-2]
    \end{tikzcd}\]
\end{proposition}
Therefore the diagram in the following corollary is a pullback: \pagenote{In the original draft it was unclear whether there was a missing diagram or I was referring to the diagram in the following lemma. This change addresses that ambiguity.}
\begin{corollary}\label{cor:mu-universal}
    Write the map $\mu:TM \ts{p}{p} TM \to T^2M$ to be $T.+ \o (\ell \x 0)$. The following diagram is a pullback:
    % https://q.uiver.app/?q=WzAsNCxbMCwwLCJUXzJNIl0sWzEsMCwiVF4yTSJdLFswLDEsIk0iXSxbMSwxLCJUTSJdLFsyLDMsIjAiLDJdLFsxLDMsInAiXSxbMCwxLCJcXG11Il0sWzAsMl0sWzAsMywiIiwxLHsic3R5bGUiOnsibmFtZSI6ImNvcm5lciJ9fV1d&macro_url=https%3A%2F%2Fraw.githubusercontent.com%2Fbenjamin-macadam%2Ftex-preamble%2Fmain%2Fpreamble.sty
    \[\begin{tikzcd}
        {TM \ts{p}{p} TM} & {T^2M} \\
        M & TM
        \arrow["0"', from=2-1, to=2-2]
        \arrow["T.p", from=1-2, to=2-2]
        \arrow["\mu", from=1-1, to=1-2]
        \arrow[from=1-1, to=2-1]
        \arrow["\lrcorner"{anchor=center, pos=0.125}, draw=none, from=1-1, to=2-2]
    \end{tikzcd}\]
\end{corollary}

%NOTE: To fix later comments re: setting notation for Lie algebroids, it makes sense to better introduce the operational tangent bundle at this point.
\pagenote{I now introduce the definition of the operational tangent bundle at this point addresses confusions about notation that comes up later in the chapter/thesis regarding the $C^\infty(M)$-module structure on $\chi(M)$}
The five maps $(p, 0, +, c, \ell)$, along with their coherences and universal properties, characterize the \emph{kinematic} tangent bundle, axiomatized as a tangent structure in the next section. However, there is an equivalent characterization of the tangent bundle for a finite-dimensional smooth manifold that will be important throughout this thesis: the \emph{operational} tangent bundle. We first need to define the module of vector fields on a manifold, where a vector field is essentially an ordinary differential equation defined on a manifold rather than a cartesian space. 
\begin{definition}[3.1,3.3 of \cite{Kolar1993}]\label{def:operational-tang}
    A \emph{vector field} on a manifold $M$ is a section $X:M \to TM$ of the projection $p:TM \to M$ so that $p \o X = id_M$. 
    The set of vector fields on a manifold $M$ is written $\chi(M)$ and carries a $C^\infty(M)$-module structure using the fibered $\R$-module structure on $p:TM \to M$:
    \[
        X +_{\chi(M)} Y := +.M \o (X,Y), \hspace{0.5cm} 0_{\chi(M)} := 0.M,\hspace{0.5cm} f \cdot_{\chi(M)} X (m) := f(m) \cdot_{TM} X(m)
    \]
    where $X,Y \in \chi(M), f \in C^\infty(M)$.
\end{definition}
The module $\chi(M)$ has an important universal property as a $C^\infty(M)$-module---it is precisely the module of \emph{derivations} of $C^\infty(M)$:
 \[
        \chi(M) = \{ X: C^\infty(M) \to C^\infty(M) : \forall f, g \in C^\infty(M), X(f\cdot g) = X(f)\cdot g + f \cdot X(g) \}
\]
\begin{proposition}[3.4 of \cite{Kolar1993}]\label{prop:derivations-tangent-bundle}
    There is an isomorphism of $C^\infty(M)$ modules:
    \[
        \mathsf{Der}(C^\infty(M)) \cong \chi(M).  
    \]
\end{proposition}
Finally, we observe that there is a $C^\infty(M)$-Lie algebra structure on $\chi(M)$, with two equivalent definitions. First, there is the kinematic definition of the bracket, which is induced using the universality of the vertical lift. Given vector fields $X,Y$ on $M$, note that
\[
   X = p.T \o (T.Y \o X -_{TM} c \o T.X \o Y), \hspace{0.25cm}
   0 = T.p \o (T.Y \o X -_{TM} c \o T.X \o Y)
\]
so by Corollary \ref{cor:mu-universal}, there is a unique map $[X,Y]:M \to TM$ so that
\begin{equation}\label{eq:hard}
    (T.Y \o X -_{TM} c \o T.X \o Y) = \mu([X,Y], X).
\end{equation}
Similarly, there is a bracket defined on $\mathsf{Der}(M)$ using the \emph{anticommutator} of derivations:
\begin{equation}\label{eq:anticom}
    [X,Y](f) = X(Y(f)) - Y(X(f)).
\end{equation}
These brackets are equivalent for finite-dimensional smooth manifolds.
\begin{proposition}\label{prop:anti-commm-lie}[\cite{Mackenzie2013}]
    Recall that a Lie algebra over a ring $R$ is an $R$-module $A$ equipped with a bilinear map
    \[
        [-.-]: A \ox A \to A   
    \]%fill this in
    that is alternating and satisfies the Jacobi identity:
    \[
        [X,[Y,Z]] + [Z,[X,Y]] + [Y, [Z,X]] = 0.  
    \]
    The two brackets on $\chi(M)$ (viewed as a Lie algebra) from Equations \ref{eq:hard} and \ref{eq:anticom} coincide.
\end{proposition}
% \begin{observation}%
%     \label{obs:operational-tangent-bundle}
    % {\color{red}
    %     This observation must be merged back into the main section
    %     \begin{itemize}
    %         \item A \emph{vector field} is a section of tangent projection (is an ODE).
    %         \item Vector fields have an evident $\R$-module structure, but can be extended to a $C^\infty(M)$-module structure
    %         \item Prop: eq. char. of $\chi(M)$ as derivations (operational tangent bundle)
    %         \item Note that there is a Lie bracket on the $\R$-module of sections - this can be defined the ``hard way'' using the canonical flip (b.c. the Jacobi identity is a truly difficult technical proof) or the easy way (derivations) - note that both definitions give same Lie algebra.
    %     \end{itemize}
    % }
    % In the category of \emph{finite-dimensional} smooth manifolds, there is an equivalent characterization of the tangent bundle, called the \emph{operational} tangent bundle (this is in contrast to the \emph{kinematic} tangent bundle described in this section, where kinematics refers to the evolution of a physical system as a path through a configuration space that is represented as a manifold). 
    % This construction of the tangent bundle is more algebraic in nature - the primitive structure is the ring of functionals $C^\infty(M)$. In this case, a vector field on a manifold $M$ is a derivation on the ring of functionals on $M$:
    % \[
    %     \chi(M) \{ X: C^\infty(M) \to C^\infty(M) : \forall f, g \in C^\infty(M): X(f\cdot g) = X(f)\cdot g + f \cdot X(g) \}
    % \]
    % % The derivative of $f$ along $X$, then, is written $X(f)$. When $X$ is regarded as a map $X:M \to TM$, this is exactly:
    % % % This equips functions $f:M \to \R$ with a notion of directional derivative:
    % % \[
    % %     X(f) : M \xrightarrow[]{X} TM \xrightarrow[]{T.f} T\R \xrightarrow[]{\phat} \R  
    % % \]
    % And there is a \emph{Lie bracket} on the space of vector fields defined using the anti-commutator:
    % \[
    %     [X,Y] = X\o Y - Y \o X  
    % \]
    % and this bracket satisfies the Liebniz law:
    % \[
    %     [X \cdot f,Y]  = X \cdot Y(f) + [X,Y]\cdot f
    % \]
% \end{observation}

\section{Tangent structures}%
\label{sec:tangent-structure}
This section develops the categorical framework to study more general categories of smooth manifolds by axiomatizing the tangent bundle, tangent categories, which first appeared in \cite{Rosicky1984}. An arbitrary tangent category is significantly more general than smooth manifolds and captures examples of categories with ``tangent bundle'' from computer science and logic, as developed in \cite{Cockett2014}. First, observe that tangent categories forget the base ring from the previous section and only consider the fibered commutative monoid structure of the tangent bundle.
\begin{definition}\label{def:add-bun}
    An additive bundle in a category $\C$ is a triple $E \xrightarrow{q} M, +:E\ts{q}{q}E \to E, \xi:M \to E$ which gives $(q,+,\xi)$ the structure of a commutative monoid in the slice category $\C/M$.
    If $(q,\xi,+), (q',\xi',+')$ are both additive bundles, a bundle morphism
    \[
        \begin{tikzcd}
            E \dar[swap]{q} \rar{f} & E' \dar{q} \\
            M \rar{f_0} & M'
        \end{tikzcd}
    \]
    is \textit{additive} if $f \o + = +' \o (f\o \pi_0,f\o \pi_1)$ and $f \o \xi = \xi' \o f_0$. The category of additive bundles in a a category $\C$ is given by additive bundles in $\C$ and additive bundle morphisms.
\end{definition}
We will often write pullback powers of an additive bundle $E \ts{q}{q} \dots \ts{q}{q} E$ as $E_n$, and use infix notation to write addition so $+ \o (a,b)$ becomes $a + b$.
In the category of smooth manifolds, the tangent bundle functor gives a well-behaved functorial vector bundle. 
A tangent category has a functorial additive bundle and axiomatizes the coherences and universal properties of the tangent bundle from the category of smooth manifolds.
\begin{definition}[\cite{Rosicky1984,Cockett2014}]\label{def:tangent-cat}
 A tangent structure consists of a functor $T:\C \to \C$ equipped with natural transformations
  \begin{gather*}
    p:T \Rightarrow id, \hspace{0.15cm} 0:id \Rightarrow T, +: T\ts{p}{p} T \Rightarrow T \\
    \ell: T \Rightarrow T.T \hspace{0.30cm} c:T.T \Rightarrow T.T
  \end{gather*}
  satisfying the following axioms; we call a category equipped with a tangent structure a \emph{tangent category}.\pagenote{I have added "."'s to the diagrams to keep my notation consistent.}
  \begin{enumerate}[{[TC.1]}]
  \item Additive bundle axioms:
    \begin{enumerate}[(i)]
        \item Pullback powers of $p$ exist and are preserved by $T$; write these $T_n$.
        \item Each triple $(p.M: TM \to M, 0.M:M \to TM, +.M:T_2M \to TM)$ is an additive bundle.
        \begin{equation}
            \label{eq:abun-axioms}
            \input{TikzDrawings/Ch1/additivity-axioms.tikz}
        \end{equation}
        % \item $(\ell,0):p \to p.T$ is additive
        % \item $(c,1): p_T \to T.p$ is additive
    \end{enumerate}
  \item Symmetry axioms:
    \begin{enumerate}[(i)]
        \item Involution: 
        \begin{equation}\label{eq:tc-2-1}
            \input{TikzDrawings/Ch1/sym-1.tikz}
        \end{equation}
        \item Yang--Baxter: $c.T \o T.c \o c.T = T.c \o c.T \o T.c$
        \begin{equation}\label{eq:tc-2-2}
            \input{TikzDrawings/Ch1/sym-2.tikz}
        \end{equation}
        \item Naturality equations:
        \begin{equation}\label{eq:tc-2-3}
            \input{TikzDrawings/Ch1/sym-3.tikz}
        \end{equation}
    \end{enumerate}
  \item Lift axioms:
    \begin{enumerate}[(i)]
        \item Naturality with addition:
        \begin{equation}
            \label{eq:tc-3-0}
            \input{TikzDrawings/Ch1/ell-0.tikz}
        \end{equation}
        \item Coassociativity:
        \begin{equation}
            \label{eq:tc-3-1}
            \input{TikzDrawings/Ch1/ell-1.tikz}
        \end{equation}
        \item Symmetric co-multiplication:
        \begin{equation}
            \label{eq:tc-3-2}
            \input{TikzDrawings/Ch1/ell-2.tikz}
        \end{equation}
        \item Universality: for $\mu:= T.+ \o (0\o \pi_0, \ell \o  \pi_1)$, the following diagram is a pullback for all $X$:
        \[\input{TikzDrawings/Ch1/univ-lift.tikz}\]
    \end{enumerate}
  \end{enumerate}
\end{definition}
%NOTE: Pulled out definition of cartesian tangent category to make it clear that it was defined, added a note re: notation to handle any confusion
\begin{definition}[\cite{Cockett2014}]
    \pagenote{I pulled out the definition of a cartesian tangent category to make it clear that it had been defined, and added a note regarding notation to handle any confusion.}
    A tangent category is \textit{monoidal} whenever $\C$ is a monoidal category, $T$ is a monoidal functor, and $+,p,c,\ell$ are monoidal natural transformations. A tangent category is \textit{cartesian} whenever $\C$ is cartesian and is a strict monoidal tangent category for products.  
\end{definition}
\begin{notation}
Throughout this thesis, two key pieces of notation apply:
    \begin{itemize}
        \item $T_n$ denotes pullback powers of $p:T \Rightarrow id$ (and more generally $E_n$ for $q:E \to M$); iterated powers of $T$ are written $T^n$.
        \item There will often be long strings of $T_n$ pullbacks and functors $F:A \to B$, so a 2-categorical notation where functor composition is written with a period, $T_n.F.T'_m$, will often be adopted. While the natural transformation $c$ at $T_n.T_m.M$ under the image of the functor $T$ would often be written $T(c_{T_n.T_m.M})$, in this thesis it will be written as
        \[
            T.c.T_n.T_m.M:T.T^2.T_n.T_m \Rightarrow T.T^2.T_n.T_m
        \]  
        while the composition of 2-cells will be written using $\o$ in applicative order, rather than the diagrammatic order typically used in the tangent category literature. That is, the composition
        \[
            A \xrightarrow[]{g} B \xrightarrow[]{f} C
        \]
        would be written $f \o g$ rather than $gf$.
    \end{itemize}
\end{notation}
\begin{example}\label{example:tangcat-sman}
    Applying the results from \cref{sec:smooth-manifolds}, the category of smooth manifolds is a tangent category. 
    Recall that if $M$ is a smooth manifold, there is a coordinate patch $m \in U \hookrightarrow M$ around each point $m \in M$ so that $U \cong U' \subseteq \mathbb{R}^n$.
    fibre above $U$, $p^{-1}(U)$, is locally isomorphic to $U' \x \mathbb{R}^n$ and similarly $(p \o p)^{-1}(U) \cong U' \x (\R^n)^3$, so that:
    \begin{center}
        \begin{tabular}{|l|l|}
            \hline
            $p:U \x \R^n \to U$  & $(m,x) \mapsto m$ \\ \hline
            $0: U \to U \x \R^n$ & $m \mapsto (m,0)$ \\ \hline
            $+:U \x \R^n \x R^n \to U \x \R^n$                         & $(m,x,y) \mapsto (m, x+y)$      \\ \hline
            $\ell:U \x \R^n \to U \x \R^n \x \R^n\x \R^n$              & $(m,x) \mapsto (m,0,0,x)$       \\ \hline
            $c: U \x \R^n \x \R^n\x \R^n \to U \x \R^n \x \R^n\x \R^n$ & $(m, x,y,z) \mapsto (m, y,x,z)$ \\ \hline
        \end{tabular}
    \end{center}
\end{example}
\pagenote{removed the example of Lex - there are some coherence issues to sort out, and that would further distract from the main point of this chapter (introducing the tangent categories and smooth manifolds).}

% \begin{example}
%     There is a folklore example of tangent structure, on the category of categories with finite limits and finite-limit-preserving functors, $\mathsf{Lex}$, due to Quillen\footnote{Finding a precise reference has been challenging}. The ``tangent category'' to $\C$ is the category of abelian group bundles in a category $\C$ (they are often called Beck modules, and first appear in \cite{Beck2003}). Several mathematicians have observed that Quillen's tangent category behaves like the tangent bundle on $\mathsf{Lex}$, the  - see, for example, \cite{Frankland2010, Stel2013}.

%     The endofunctor $T$ sends a category $\c$ to Beck modules in $\c$. The projection, then, sends an additive bundle to its base space, and the zero maps send an object to the trivial Beck module $(id,id,id)$. The category $T_2\c$, then, is the category of pairs of Beck modules over a fixed space, and the addition map sends a pair of Beck modules to its biproduct:
%     \input{TikzDrawings/Ch1/BeckModules/addition.tikz}
%     The canonical flip sends a double Beck module to its ``flipped'' bundle
%     \input{TikzDrawings/Ch1/BeckModules/flip.tikz}
%     Furthermore, the vertical lift sends an additive bundle its ``vertical'' bundle:
%     \input{TikzDrawings/Ch1/BeckModules/lift.tikz}
%     Note that for any finitely-complete additive category $\a$ in $\mathsf{Lex}$, any Beck module splits as:
%     \[
%         A \xrightarrow[]{q} B \cong B \oplus A' 
%     \]
%     where $A' = \mathsf{Ker}(q)$ - so there is an equivalence of categories $\a \x \a \cong T\a$. This equivalence of categories proves the universality of the vertical lift - the pullback:
%     \input{TikzDrawings/Ch1/BeckModules/univ-v-lift.tikz}
%     is the category of Beck modules of the form:
%     \input{TikzDrawings/Ch1/BeckModules/ob-in-univ.tikz}
%     This forces $D \to A$ to be a Beck module in the category of Beck modules over $M$, which is an additive category with finite limits, so $D \cong A \x V$, and double Beck module is isomorphic to one of the form:
%     \input{TikzDrawings/Ch1/BeckModules/splitting-of-vlift-ob.tikz}
%     This subcategory is precisely the image of $A\to M, V \to M$ under the map $\mu:T_2\c \to T^2\c$. Thus, this pullback is a limit up to an equivalence of categories. The coherences on addition only holds up to a natural isomorphism (e.g. the associator map), and the universality of the lift holds only up to equivalence of categories, so this is a \emph{2-tangent category} in some appropriate sense.
% \end{example}

The study of tangent categories is closely related to Lawvere's \emph{synthetic differential geometry}, first introduced in \cite{Lawvere1979} and later developed in \cite{Kock2006, Lavendhomme1996, Moerdijk1991}. The setting of synthetic differential geometry is a topos $\e$ (for our purposes, we need only a complete, cartesian closed category) equipped with a chosen ring object $R$ that satisfies the \emph{Kock-Lawvere axiom}: given the object of nilpotent elements in $R$, $D = [d : R | d^2 = 0]$, the following map is an isomorphism:
\[
    \alpha: R \x R \to [D,R]; \hspace{0.25cm} \alpha(a,b) = (d \mapsto a + db).  
\]
% The full subcategory of $R$-modules in $\e$ satisfying the Kock-Lawvere axiom:
% \[
%    \alpha_E: E \x E \to [D,E];  \hspace{0.25cm} \alpha_E(a,b) = d \mapsto a + db 
% \]
% satisfies the axioms of a cartesian differential category (when using the full subcategory of maps in $\C$). 
One can find a class of objects that form a tangent category: the infinitesimally linear objects.\pagenote{
   This passage has been edited substantially to clear up potential confusion about the definition of the Kock-Lawvere axiom (it originally seemed like it was defined twice), and the notions of infinitesimal linearity (the language of ``perceiving'' a diagram as a colimit is a needlessly confusing bit of jargon in synthetic differential geometry that doesn't need to be in this thesis). 
}
\begin{definition}%
\label{def:inf-linear}
    Let $\C$ be a model of synthetic differential geometry where the ring object is $(R, \cdot, 1,+, 0)$.
    An object $M$ in a model of synthetic differential geometry is \emph{infinitesimally linear} when it satisfies the following axioms.
    \begin{enumerate}[(i)]
        \item The following natural morphism must be an isomorphism: 
            \[
                [D(2),M] \cong [D,M]\ts{0^*}{0^*} [D,M];
                \text{ where } 
                D(2) := [(d_0,d_1) \in D^2 : d_i\cdot d_j = 0].
            \]
        \item $M$ satisfies \emph{property W} (credited by  \cite{Kock2006} to Gavin Wraith), namely that the following diagram is a ternary equalizer:
            \input{TikzDrawings/Ch1/InfObj/triple-eq.tikz}
        (We will often use $M$ as shorthand for $id_M$ in diagrams).
    \end{enumerate}
\end{definition}
\pagenote{The treatment of the operational tangent bundle at the end of the last section made the observation regarding the Lie bracket here unnecessary.}
% \begin{observation}%
%     \label{obs:the-lie-bracket-in-a-tang-cat}
%     Recall the ``operational'' description of the tangent bundle at the end of Section \ref{sec:smooth-manifolds}. In a tangent category with negatives, there is a Lie bracket on the set of vector fields:
%     \[
%           \ell \o [X,Y] = (c \o T.X \o Y -_{p.T} T.Y \o X) -_{T.p}  0 \o X
%     \]
%     \cite{Mackenzie2013} showed this bracket is equivalent to the Lie bracket of vector fields in the category of smooth manifolds from Observation \ref{obs:operational-tangent-bundle}, and gave a direct proof that this bracket that it satisfies the Jacobi identity. In \cite{Cockett2015}, the authors provided a proof that the Jacobi identity holds for this Lie bracket in any tangent category - this proof used a string calculus (in some ways anticipating \cite{Leung2017}). In a tangent category with a base ring (e.g. Definition \ref{def:inf-linear}), \cite{Lavendhomme1996} proved that the Liebniz law holds.
% \end{observation}

An \emph{infinitesimal object} generalizes the object $D$ in the category of infinitesimally linear objects in a model of synthetic differential geometry. The definition adopted in this thesis is a strict generalization of the definition given in \cite{Cockett2014}: here, we work with a symmetric monoidal closed category rather than a cartesian closed category in order to capture some examples from logic.\pagenote{I have expanded on this definition to fix  ambiguities in the original draft.}
\begin{definition}[Definition 5.6 \cite{Cockett2014}]
    \label{def:inf-object}
    An \emph{infinitesimal object} in a symmetric monoidal category \[(\C, \ox, I, \alpha, \rho, \sigma)\] is a tuple $(\odot:D \ox D \to D, 0:I \to D, \epsilon: D \to I, \delta: D \to D(2))$ (where $D(n)$ denotes pushout powers of $0$) so that:
    \begin{enumerate}[{[{I}O.1]}]
        \item Pushout powers $D(n)$ of $\hspace{0.05cm} 0:I \to D$ exist, and $\epsilon \o 0 = id_I$.
        \item $\odot$ is a commutative semigroup with zero, so that the following diagrams commute:
        \input{TikzDrawings/Ch1/InfSemiGroupCoh.tikz}
        The third diagram shows that $0$ is an absorbing element rather than a unit (think of $0$ as being in the commutative semigroup on the set $[0,1)$ given by multiplication).
        \item The map $\delta: D \to D(2)$ makes $(0:I \to D, \delta, \epsilon)$ into a commutative comonoid in the coslice $I/\C$:
        \input{TikzDrawings/Ch1/inf-co-add-coh.tikz}
        \item The following diagram commutes ($\odot$ is coadditive):
        \input{TikzDrawings/Ch1/InfObj/odot-coadditive.tikz}
        The notation $(f | g):A + B \to C$ for pushouts/coproducts is dual to pairing $(f',g'):C \to A \x B$ for pullbacks/products, just as $\iota_i$ is dual to projection. Therefore, $(f | g) \o \iota_0 = f$ just as $\pi_0 \o (f,g) = f$.
        \item The following diagram is a coequalizer:
        \input{TikzDrawings/Ch1/InfObj/coeq.tikz}
    \end{enumerate} 
\end{definition}
There are two tangent structures associated to an infinitesimal object in a symmetric monoidal category $\C$. The first relies on the \emph{exponentiability} of the infinitesimal object, while the second tangent structure is on the opposite category of $\C$. The enriched perspective on tangent structure in Section \ref{sec:tang-cats-enrichment} will clarify the relationship between these two tangent structures.
\begin{proposition}%
    \label{prop:inf-object-tangent-structures}
    Let $(\C, \ox, I, \alpha, \rho, \sigma)$ be a symmetric monoidal category, and \[\odot:D \ox D \to D, 0:I \to D, \epsilon: D \to I, \delta: D \to D(2)\] define an infinitesimal object in $\C$.\pagenote{
        The original proposition was unclear, so this passage has been changed to fix some ambiguities. The full structure of the symmetric monoidal category was added, the order of (i) and (ii) are swapped, the wording in part (ii) has been clarified, as has the proof itself.
    }
    \begin{enumerate}[(i)]
        \item There is a tangent structure on $\C^{op}$, where $T = D \ox (-)$.
        \item If $\C$ is also a symmetric monoidal closed category, then it is a tangent category with $T = [D, -]$.
    \end{enumerate}
\end{proposition}
\begin{proof}
   ~\begin{enumerate}[(i)]
        \item The opposite category of $\C$ is a symmetric monoidal category:
        \[
            (\C^{op}, \ox, I, \alpha^{-1}, \rho^{-1}, \sigma).
        \]
        The tangent functor is $D \ox (-)$, and the projection is \[p = D \ox (-) \xrightarrow[]{0^{op}} I \ox (-) \xrightarrow[]{\rho^{-1}} (-).\] The zero map is given by
        \[
            (-) \xrightarrow[]{\rho} I \ox (-) \xrightarrow[]{\epsilon^{op}} D \ox (-).
        \]
        Addition in $\C^{op}$ is given by co-addition in $\C$:
        \[
            D(2) \ox (-) \xrightarrow[]{\delta^{op} \ox (-)} D \ox (-). 
        \]
        The lift is given by the semigroup structure (along with the monoidal coherences):
        \[
            D \ox (-) \xrightarrow[]{\odot \ox (-)} (D \ox D) \ox (-) \xrightarrow[]{\alpha^{-1}} D \ox (D \ox (-)).
        \]
        The flip is also given by monoidal coherences, combined with the symmetry on the monoidal category:
        \[
            D \ox (D \ox (-))  \xrightarrow[]{\alpha} (D \ox D) \o (-) \xrightarrow[]{\sigma \ox (-)}
            (D \ox D) \ox (-) \xrightarrow[]{\alpha^{-1}} D \ox (D \ox (-)).
        \]
      \item First, we identify the following natural isomorphisms:
        \[
            u: id \Rightarrow [I,-],\hspace{0.5cm}
            b: [A,[B,-]] \Rightarrow [A \ox B, -].
        \]
        \begin{itemize}
            \item The tangent functor is $[D,-]$, and the triple $(0:I \to D, \delta:D \to D(2), \linebreak \epsilon:D \to I)$ gives the additive bundle structure
            \begin{gather*}
                p:[D,-] \xrightarrow{0^*} [I,-] \xrightarrow{u} id, \hspace{1cm}
                0: id \xrightarrow{u} [I,-] \xrightarrow{[\epsilon, -]} [D,-] \\
                +:[D(2),-] \xrightarrow{\delta^*} [D,-].
            \end{gather*}
    
            Note that by the continuity of $[-, M]: \C^{op} \to \C$, we have $[D(2),M] \cong [D,M] \ts{p}{p} [D,M]$.
            \item The lift is given by
                \[
                    \ell: [D, -] \xrightarrow{\odot^*} [D \ox D, -] \xrightarrow{b^{-1}} [D,[D,-]]. 
                \]
            \item The canonical flip is given by
                \[
                    c: [D,[D,-]] \xrightarrow{b} [D\ox D, -] \xrightarrow{\sigma^*} [D\ox D, -]\xrightarrow{b^{-1}} [D,[D,-]].
                \]
        \end{itemize}
        The coherences and couniversality properties of an infinitesimal object, along with the continuity of 
        \[
            [-,M]: \C^{op} \to \C 
        \]
        then induce the coherences and universality properties for the tangent bundle. This completes the proof.
    \end{enumerate}
\end{proof}
The tangent structure on $\C^{op}$ induced by the infinitesimal object is the \emph{dual} tangent structure on $\C$. This will be revisited in Chapters \ref{chap:weil-nerve} and \ref{ch:inf-nerve-and-realization} when looking at tangent categories from the enriched perspective.
% \begin{observation}
%     The definition of a infinitesimal object does not strictly necessitate the monoidal category being symmetric, as the object can carry a symmetry $\sigma: D \ox D \to D\ox D$ that is coherent with the other structure maps of an infinitesimal object. Given this, we see that an infinitesimal object in a left-closed monoidal category will give rise to a tangent structure, but chasing through the coherences for a complete proof is best left for future work. 
    % Similarly, we can require that each of the endofunctors \[ \ox^k D(n_i) \ox \_: \C \to \C  \] have left adjoints (e.g. they are exponentiable), so that this result applies to general monoidal categories.
% \end{observation}

Returning to synthetic differential geometry, the co-universality conditions on an infinitesimal object corresponds to infinitesimal linearity (Definition \ref{def:inf-linear}). In some sense, the category of infinitesimally linear objects is the largest subcategory of $\e$ for which $D$ is an infinitesimal object. 
\begin{corollary}
    In a model of synthetic differential geometry $(\e, R)$, the object $D = [d \in R | d^2 = 0]$ is an infinitesimal object in the category of infinitesimally linear objects in $\e$.
\end{corollary}

This thesis makes use of the 2-category of tangent categories (Section 2.3 of \cite{Cockett2014}), which formalizes the notion of a morphism of tangent structure and 2-cells between them. This 2-categorical framework is a departure from the classical theory of synthetic differential geometry, where the literature only really addresses morphisms of tangent structure in the form of fully faithful embeddings $\mathsf{SMan} \hookrightarrow \mathsf{Microl}(\e)$.\pagenote{Removed ``the formal theory of tangent categories, added Microl to $\e$.''}

\begin{definition}\label{def:tang-functor}
    Let $(\C, \mathbb{T}), (\D, \mathbb{T}')$ be a pair of tangent categories. A pair $(F: \C \to \D, \alpha:F.T \Rightarrow T'.F)$ is a \emph{tangent functor} if the following diagrams commute:\pagenote{I have fixed up the notation in this definition by adding a "." to $F.T$, and by setting the notation $(F,\alpha)$ for tangent functors. }
    \begin{center}
        \input{TikzDrawings/Ch1/TangFunctor.tikz}
    \end{center}
    A tangent functor is \emph{strong} whenever $\tnat$ is a natural isomorphism; for a sub-(tangent category) inclusion, $\alpha = id$.  A tangent functor $(F, \alpha)$ between cartesian tangent categories is a \emph{cartesian tangent functor} if $F$ is an isomonoidal functor and $\tnat$ is a monoidal natural transformation. 
\end{definition}
\begin{example}\label{ex:composition-of-tangent-functors}
    ~\begin{enumerate}[(i)]
        \item The coherences on the canonical flip $c$ guarantee that $(T:\C \to \C,c:T.T \Rightarrow T.T)$ is a strong tangent endofunctor on any tangent category. 
        \item Given a pair of tangent functors $(A,\alpha):\C \to \D, (B, \beta): \D \to \mathbb{E}$, the composition 
        \[
            (B.A:\C \to \mathbb{E}, \beta.A \o B.\alpha: % https://q.uiver.app/?q=WzAsMyxbMCwwLCJCLkEuVCJdLFsxLDAsIkIuVC5BIl0sWzIsMCwiQi5BLlQiXSxbMCwxLCJCLlxcYWxwaGEiLDAseyJsZXZlbCI6Mn1dLFsxLDIsIlxcYmV0YS5BIiwwLHsibGV2ZWwiOjJ9XV0=
            \begin{tikzcd}
                {B.A.T^C} & {B.T^D.A} & {T^E.B.A}
                \arrow["{B.\alpha}", Rightarrow, from=1-1, to=1-2]
                \arrow["{\beta.A}", Rightarrow, from=1-2, to=1-3]
            \end{tikzcd})
        \]
        is a tangent functor.
        \item A model of synthetic differential geometry $(\e,R)$ is \emph{well-adapted} whenever there is a fully faithful, strict tangent functor from $\mathsf{SMan}$ to the category of microlinear spaces of $\e$,  $\mathsf{SMan} \hookrightarrow \mathsf{Microl}(\e)$. The original development of well-adapted models for synthetic differential geometry may be found in \cite{Dubuc1981}, and the reader may check \cite{Bunge2018} for a recent account of the construction of such models, or section 3 of \cite{Kock2006}.\pagenote{Clarified that the inclusion into $\mathsf{Microl}(\e)$ is a tangent functor.}
    \end{enumerate}
\end{example}


\begin{definition}\label{def:tang-nat}
    A \emph{tangent natural transformation} $\gamma$ between tangent functors $(F,\alpha), (G,\beta)$ is a natural transformation so that the following diagram commutes:
    \[
        \input{TikzDrawings/Ch1/tang-nat.tikz}
    \] 
    If $F, G$ are cartesian tangent functors, then $\gamma$ is cartesian whenever it is an isomonoidal natural transformation.
\end{definition}
\begin{definition}\label{def:tang-2cat}
    We will call the 2-category of tangent categories, tangent functors, and tangent natural transformations $\mathsf{TangCat}$. The 2-category of cartesian tangent categories is $\mathsf{CartTangCat}$.
\end{definition}


%


\section{Local coordinates in a tangent category}%
\label{sec:diff-and-tang-struct}

This section develops some structures to facilitate reasoning about higher tangent bundles, using ``local coordinates.'' In the case of a cartesian differential category,  $T^n(A) = \prod_{2^n} A$, whereas for an open subset $U \subseteq \R^m$ the tangent bundle decomposes as $T^nU \cong U \x (\R^m)^{2^n - 1}$. This section introduces two structures that allow for these arguments in an arbitrary tangent category: differential objects and connections.\pagenote{
   When I originally wrote this thesis, differential objects figured into the story more prominently. I have cut a few results here that were messily written and were no longer used in subsequent chapters.
}

\begin{definition}[\cite{Cockett2018}]
    \label{def:differential-object}
    A\pagenote{
        This definition was very unclear/inconsistent. I have since removed the term ``linear'', and fixed the notation on the commutative monoid structure on $A$, which addresses the main issued raised. 
    } \emph{ differential object}\footnote{Not to be confused with 4.1 from \cite{Barr2002}.} in a cartesian tangent category is a commutative monoid $(A, +_A,0_A)$ such that there is a section-retract pair $A \xrightarrow{\lambda} TA \xrightarrow{\hat{p}} A$ which exhibits $T(A)$ as a biproduct in the category of commutative monoids:
        \[A \oplus A \cong TA.\]
    % For convenience, denote the isomorphism $T(A) \cong A \times A$ as $\nabla$. 
    Concretely, a differential object is a commutative monoid equipped with $\lambda:A \to TA, \phat:TA \to A, \phat \o \lambda = id$ so that the following axioms hold:
    \begin{enumerate}[DO.1]
    \item Coherence between $+$ and $\lambda, \phat$:
        \[\input{TikzDrawings/Ch1/DiffOb-axioms/addition-coh.tikz}\]
    \item Coherence between $0_A$ and $0.A$:
      \[ \input{TikzDrawings/Ch1/DiffOb-axioms/zero-coh.tikz} \]    
    \item Coherence between $+_A$ and $+.A$:
        \[\input{TikzDrawings/Ch1/DiffOb-axioms/addition-coh-2.tikz}\]
    \item Coherence between $\lambda$ and $\phat$ with $\ell$:
      \[\input{TikzDrawings/Ch1/DiffOb-axioms/conn.tikz}\]
  \end{enumerate}
  A map $f:A \to B$ between differential objects $(A,\lambda_A, \phat_A, +_A,0_B) \to (B,\lambda_B,\phat_B, +_B, 0_B)$ is \emph{linear} whenever $f$ preserves the lifts and projections
  \[
      (T.f \o \lambda_A = \lambda_B \o f) \text{ and } (\phat_B \o T.f = f \o \phat_A).
  \]
\end{definition}
 Following the work in Section 3 of \cite{Cockett2018}, it is sufficient to check that $f$ preserves $\lambda$ or $\phat$ (each condition implies the other).
\begin{example}\label{ex:diffob-sman}
    ~\begin{enumerate}[(i)]
        \item In the category of smooth manifolds, the real numbers are a differential object, as $T.\R = \R[x]/x^2$, so the lift map in this case is
        \[
            \lambda(a) = 0 + a\cdot x,  \hspace{1cm} \phat(a + b\cdot x) = b. 
        \]
        More generally, finite-dimensional real vector spaces, with their canonical smooth manifold structure, are exactly differential objects in the category of smooth manifolds. The lift map is defined using $T\R \cong \R[x]/x^2$, so that
        \[ V \xrightarrow[]{(0,1^\R)} T(V \x \R) \xrightarrow[]{T\cdot_V} TV. \]
        This is equivalent to the isomorphism $T.V = R[x]/x^2 \ox V$.
        % \item Convenient vector spaces are differential objects in the category of convenient manifolds.
        %NOTE: Removed convenient manifolds because I had not introduced that example tangent category
        \pagenote{I have removed the convenient manifolds example because I had not introduced that example tangent category.}
        \item Every object in a cartesian differential category has a canonical differential object structure, as $TA := A \x A$.
        % \item In $\mathsf{Lex}$, a differential object is exactly a finitely-complete additive category. This follows from Beck's result that in an additive category, every additive bundle splits as $E \to M \cong A \x M$, and so $T\a \cong \a \x \a$. The map $\lambda$ sends $A$ to the Beck module $!:A \to 1$, and the map $\lambda$
    \end{enumerate}
\end{example}

Classically, the tangent space above each point of a smooth manifold is a vector space. 
We have a similar result for differential objects from \cite{Cockett2018}.
\begin{lemma}\label{lem:tang-space-diff-ob}
    Suppose we have the following pullback in a cartesian tangent category $\C$, and all powers of $T$ preserve it.
    \[
        \begin{tikzcd}
            E \rar{\iota} \dar{!} & TM \dar{p} \\
            1 \rar{m} & M
        \end{tikzcd}
    \]
    There is a unique differential object structure $(E,\lambda,\phat)$ so that $\ell \o \iota = T.\iota\o \lambda$.\footnote{Some diagrammatic notation had creeped into the original draft here, I have fixed it.}
\end{lemma}
% We now collect some facts about differential objects in a tangent category.
% In particular, we show that differential objects are closed under products and the tangent functor.
% % Furthermore, if a map $f$ preserves $\lambda$ it preserves $\phat$, and $Tf$ splits as a biproduct.
% \begin{lemma}[\cite{Cockett2018}]\label{lem:preserve-lambda-iff-phat}
%     Let $(A,\lambda,\phat),(B,\lambda',\phat')$ be differential objects in a cartesian tangent category $\C$.
%     \begin{enumerate}[(i)]
%         \item $f:A \to B$ preserves $\lambda$ if and only if it preserves $\phat$.
%         % \item $f:A \to B$ preserves $\lambda$ if and only if $(p,\phat)^{-1} \o T(f) (p \o \phat) = f \oplus f$.
%         % \item $(A \x B, \lambda \x \lambda', \phat \x \phat')$ is a differential object.
%         \item $(T(A), c\o T.\lambda, T.\phat \o c)$ is a differential object.
%         % \item A cartesian tangent functor $(F,\tnat,m)$ will preserve differential objects. 
%     \end{enumerate}
% \end{lemma}

There are two natural classes of morphisms between differential objects, linear and ``smooth'' (that is, arbitrary morphisms).
\begin{definition}[\cite{Cockett2018}]\label{def:diff-dlin}
    Let $\C$ be a cartesian tangent category. 
    % A morphism of differential objects is \textit{linear} if it preserves the lift (or, equivalently, $\phat$).
    We define the following categories:
    \begin{enumerate}[(i)]
        \item $\Diff(\C)$ is the category of differential objects and arbitrary morphisms, so for any differential objects $A,B$, we have $\Diff(\C)(A,B) := \C(A,B)$.
        \item $\Dlin(\C)$ is the category of differential objects and linear morphisms.
    \end{enumerate}
\end{definition}

The category of differential objects and smooth maps is a cartesian differential category, exhibiting differential calculus as a specialized logic in a tangent category. 
\begin{proposition}[Section 3.5 of \cite{Cockett2018}]\label{prop:diff-obs-cdc}
    Let $\C$ be a cartesian tangent category.
    Then:
    \begin{enumerate}[(i)]
        \item $\Diff(\C)$ is a cartesian differential category, where we define the differential combinator $D$ to be
        \[
            \infer{A \x A \xrightarrow{\lambda \x 0} T(A \x A) \xrightarrow[]{T.+_A} T(A) \xrightarrow{T(f)} T(B) \xrightarrow{\phat_B} B}{A \xrightarrow{f} B}
        \]
        (where $\lambda_A, +_A$ are the lift and addition for the differential object structure on $A$, and $\phat_B$ is the projection map on the differential object structure on $B$).
        \item There is an equality of categories, $\mathsf{Lin}(\Diff(\C)) = \Dlin(\C)$, meaning that a morphism between differential objects in the cartesian tangent category $\C$ is linear if and only if it is linear in the cartesian differential category $\Diff(\C)$.
    \end{enumerate}
\end{proposition}
% \begin{remark}
%     In the original proof of Proposition \ref{prop:diff-obs-cdc}, Cockett and Cruttwell used the notion of a \emph{coherent differential structure} on objects with a differential object structure associated to them. The phrasing of this result is slightly different, but allows for that subtlety to be ignored as on can show that $\Diff(\C)$ is a cartesian tangent category 
% \end{remark}
Recall that if $M$ is an open subset of $\R^n$, the second tangent bundle of an open subset $U \subseteq \R^n$ splits as
\[
    T^2(U) = T(U \x \R^n) = (U \x \R^n) \x (\R^n \x \R^n) = T_3U.
\]
Every smooth manifold admits such a decomposition on its second tangent bundle; these are known as a \emph{affine connections}\footnote{The prefix ``affine'' differentiates these from more general connections on differential bundles, which are introduced in Section \ref{sec:connections-on-a-differential-bundle}.} and provide a way to reason about an object as though it has local coordinates in an arbitrary tangent category.\pagenote{Added a footnote to explain the prefix ``affine''.}
\pagenote{slightly changed the passage between differential objects -> affine connections}
% However, this is not the case in a general tangent category - not every object has a decomposition $T_3M \cong T^2M$.
%  While every smooth manifold has such a decomposition, it is not natural. These decompositions are known as \emph{connections}

%NOTE: slightly changed the passage between differential objects -> affine connections

\begin{definition}[\cite{Cockett2017}]
    In a tangent category $\C$, define the following:
    \begin{enumerate}[(i)]
        \item An affine vertical connection is a map $\kappa:T^2M \to TM$ so that 
        \begin{enumerate}
            \item $\kappa$ is a \emph{vertical descent}, namely a section of the vertical lift $\ell:TM \to T^2M$, so $\kappa \o \ell = id$;
            \item $\kappa$ is compatible with both lifts on $T^2M$: $T.\kappa \o \ell.T = T.\kappa \o T.\ell = \ell\o \kappa$.
        \end{enumerate}
        \item An affine horizontal connection is a map $\nabla:T_2 \to T^2M$ so that
        \begin{enumerate}
            \item $\nabla$ is a \emph{horizontal lift}, namely a section to the horizontal descent $(p.T, T.p): T^2 \Rightarrow T_2$, so that $(p.T,T.p) \o \nabla = id$;
            \item $\nabla$ is compatible with the linear structures on $T^2, T_2$:$ T.\nabla\o (\ell\x 0) \linebreak = \ell \o \nabla and T.\nabla(0\x \ell) = T.\ell \o \nabla$.
        \end{enumerate}
        \item An \emph{affine connection} is a pair $(\kappa, \nabla)$ comprising a vertical and horizontal connection on $M$ satisfying the compatibility conditions: 
        \begin{enumerate}
            \item $T.+ \o (+.T \o (\ell \o \kappa, T.0 \o p.T), \nabla(p.T, T.p)) = id_{T^2M}$, 
            \item $\kappa \o \nabla = 0 \o p \o \pi_i$.
        \end{enumerate}
        An affine connection is \emph{torsion-free} if $\kappa \o c = \kappa$.
    \end{enumerate}
\end{definition}

% Note that the full data of an affine connection is equivalent to a universality condition on the vertical part of the connection.
% \begin{proposition}
%     An affine vertical connection $\kappa:T^2M \to TM$ determines a full connection on $M$ if and only if the following diagram is a limit.
%     % https://q.uiver.app/?q=WzAsNSxbMCwxLCJUXjJNIl0sWzEsMCwiVE0iXSxbMSwyLCJUTSJdLFsxLDEsIlRNIl0sWzIsMSwiTSJdLFswLDEsIlQucCJdLFswLDIsInAuVCIsMl0sWzAsMywiXFxrYXBwYSIsMV0sWzEsNCwicCJdLFszLDQsInAiLDFdLFsyLDQsInAiLDJdXQ==
%     \[\begin{tikzcd}
%         & TM \\
%         {T^2M} & TM & M \\
%         & TM
%         \arrow["{T.p}", from=2-1, to=1-2]
%         \arrow["{p.T}"', from=2-1, to=3-2]
%         \arrow["\kappa"{description}, from=2-1, to=2-2]
%         \arrow["p", from=1-2, to=2-3]
%         \arrow["p"{description}, from=2-2, to=2-3]
%         \arrow["p"', from=3-2, to=2-3]
%     \end{tikzcd}\]
% \end{proposition}
% It can also be shown that a morphism $f:M \to N$ preserves an affine connection if and only if it preserves the vertical part of 

The data of a vertical connection is sufficient to define a full connection, as observed in \cite{LucyshynWright2018}.
\begin{lemma}[\cite{LucyshynWright2018}]
    A full connection is equivalent to a vertical connection in which the following diagram is a fiber product:
    \input{TikzDrawings/Ch1/full-con.tikz}
\end{lemma}



\begin{example}
    ~\begin{enumerate}[(i)]
        \item Every differential object in a tangent category has a canonical vertical connection given by $(p \o p.T, \phat \o T.\phat)$. A morphism of differential objects will preserve this vertical connection.
        \pagenote{Fixed the notation for the affine vertical connection on a differential object induced by a differential object's $\phat$ map. }
        \item Every smooth manifold has a non-natural choice of Riemannian metric. By the fundamental theorem of Riemannian geometry, there is a torsion-free connection associated with the metric. (See any standard reference on Riemannian geometry, e.g. \cite{Carmo1992}.)
        % \item The category of commutative rings is has a full connection in $\mathsf{Lex}$. An observation due to Quillen states that $TC_{ring} \cong Mod$, where $Mod$ is the category given by:
        % \begin{itemize}
        %     \item $(S, N)$ where $S$ is a commutative ring and $N$ an $S$-module.
        %     \item $(R,M) \to (S,N)$ is a pair $\phi:R \to M$ and $f: M \to \phi^*N$.
        % \end{itemize}
        % Implicit here is the result that a Beck module over $S$ is an $S$-module.
    
        % A double Beck module in the category of commutative rings is then a Beck module in $Mod$.
        % Note that a pullback of modules of $(\phi,f): (S,Q) \to (R,N)$ is given by:
        % \[
        %     (R \ts{\phi}{\phi} R, Q \ts{f}{f} Q) \cong (R_2, Q_2)
        % \]
        % % So $Mod$ is itself the category of models for a multi-sorted algebraic theory.
    
        % \begin{itemize}
        %     \item The ring part of the Beck module is exactly a Beck module in the category of rings, so it splits as $S \cong R \oplus N$.
        %     The commutative ring $S$, then, is the square-zero extension of $R$ by $N$: 
        %     \[
        %         (R \oplus N) \x (R \oplus N) \to (R\oplus N); \hspace{0.15cm}
        %         (r_1,n_1),(r_2,n_2) \mapsto (r_1r_2, r_1n_2 + r_2n_1)
        %     \]
        %     so the map $\phi$ is actually $\pi_0: R \oplus N \to R$.
        %     \item The module part $Q$, then, is a module over the square-zero extension of $R$ by $N$. Moreover, the module $\phi^*(M)$ has the action:
        %     \[
        %         (R \oplus N) \x M \to M; \hspace{0.15cm} (r,n), m \mapsto rm
        %     \]
        %     so that we have 
        %     \[
        %         f((r,n)\cdot m) = r\cdot f(m)
        %     \]
        %     meaning that $f$ is in fact an $R$-module $f:\iota^*(Q) \to M$, and this is a Beck module in the category of $R$-modules and of course splits as $Q \cong N \oplus P$.
        % \end{itemize}
        % Therefore, that there is a natural isomorphism $id \Rightarrow \mu(p.T,\kappa) + \nabla(p.T, T.p)$ on the category of commutative rings.
        % Therefore, the core connection on the category of commutative rings is a full connection.
    \end{enumerate}
\end{example}

Connections allow for arguments on higher powers of the tangent bundle to be pushed down to pullback powers of $T$.
\begin{observation}
    Suppose $M, N$ each have connections $(\kappa_{-}, \nabla_{-})$. 
    The map $T^2.f: T^2M \to T^2N$ can be written using local coordinates $\widehat{T^2.f}: T_3M \to T_3N$ as
    % https://q.uiver.app/?q=WzAsNCxbMCwxLCJUXzNNIl0sWzEsMSwiVF8zTiJdLFswLDAsIlReMk0iXSxbMSwwLCJUXjJOIl0sWzAsMiwiXFx0aGV0YV9NIl0sWzIsMywiVF4yLmYiXSxbMywxLCIocC5ULCBULnAsIFxca2FwcGFfTikiXSxbMCwxLCIiLDIseyJzdHlsZSI6eyJib2R5Ijp7Im5hbWUiOiJkYXNoZWQifX19XV0=
    \[\begin{tikzcd}
        {T^2M} & {T^2N} \\
        {T_3M} & {T_3N}
        \arrow["{T.+ \o (+.T \o (\ell \o \pi_2, T.0 \o \pi_0), \nabla(\pi_0, \pi_1))}", from=2-1, to=1-1]
        \arrow["{T^2.f}", from=1-1, to=1-2]
        \arrow["{(p.T, T.p, \kappa_N)}", from=1-2, to=2-2]
        \arrow[dashed, from=2-1, to=2-2]
    \end{tikzcd}\]
    %T.+ \o (+.T \o (\ell \o \kappa, T.0 \o p.T), \nabla(p.T, T.p))
    where $\widehat{T^2.f}$ is given by
    \[
        (T.f\o\pi_0, T.f \o\pi_1, T.f \o \pi_2 +_N \nabla[f](\pi_0,\pi_1))
    \]
    with $\nabla[f] := \kappa_N \o T^2.f \o \nabla_M$.
\end{observation}
Note that in the case that $f$ preserves the connections, $\nabla[f] = 0$, as 
\[
    \kappa_N \o T^2.f \o \nabla_M = Tf \o \kappa_M \o \nabla_M = Tf \o 0 \o p = 0 \o f \o p.
\]

% The tangent functor preserves connections:
% \begin{lemma}
%     Let $\kappa$ be a vertical connection on $M$. Then $\kappa_T := c \o T.\kappa \o c.T \o T.c$ is a connection on $TM$.
%     If $\kappa$ is universal then so to is $\kappa_T$.
% \end{lemma}
% \begin{proof}
%     Observe that
%     \[
%         c \o T.\kappa \o c.T \o T.c \o \ell = c \o T.\kappa \o T.\ell \o c = c\o c = id
%     \]
%     Linearity follows by composition, and universality follows as $T$ preserves the universal property.
% \end{proof}

\section{Submersions}\label{sec:submersions}

The category of smooth manifolds is incomplete: there are cospans\footnote{Following a general convention in category theory, where the prefix "co"-X means an X in the opposite category, a cospan in $\C$ is a span in the dual category of $\C$.} $X \xrightarrow[]{f} M \xleftarrow[]{g} Y$ for which the pullback fails to exist. Following \cite{thom1954}, the pullback of a cospan exists and is preserved by $T$ (i.e. it is a \emph{T-limit}) whenever for each point $f(x) = g(y)$, the direct sum of the images of $T_xf$ and $T_yg$ is the full vector space $T_{f(x)}M$, such cospans are called \emph{transverse}. Submersions, then, form a convenient class of maps, as any cospan where one map is a submersion will be transverse. More precisely:
\begin{definition}
    \label{def:submersion-sman}
    If $A$ and $B$ are smooth manifolds, a smooth function $f:\linebreak A \rightarrow B$ is a \emph{submersion} if and only if the derivative $Df|_a$ of $f$ at every point $a\in A$ is a surjective linear map.
\end{definition}
With this definition we have the following result:
\begin{proposition}%
    \label{prop:submersion-properties}
    In the category of smooth manifolds, let the class of submersions be denoted by $\sh$.
    \begin{enumerate}[(i)]
        \item Submersions are closed under the tangent functor: $f \in \sh \Rightarrow T.f \in \sh$.
        \item Submersions are closed to pullback along arbitrary maps:
        \input{TikzDrawings/Ch1/submersion-closure.tikz} 
        This will often be referred to as \emph{$T$-stability under reindexing}, as it induces a functor between slice categories: 
        \[g^*: \mathsf{Submersions}/M \to \mathsf{Submersions}/N.\]
    \end{enumerate}
    \pagenote{ Based on a comment from Michael, it seemed appropriate to move the T-prefix to stability, as the reindexing operation is unchanged.}
    %NOTE: Based on a comment from Michael, it seemed appropriate to move the T-prefix to stability, as the reindexing operation is unchanged.
\end{proposition}
The properties of the class of submersions in the category of smooth manifolds were studied in \cite{Cockett2018} and axiomatized as a \emph{tangent display system}.

\begin{definition}\label{def:display-system}
    A \emph{tangent display system} in a tangent category $\C$ is a class of maps $\mathcal{D}$ in $\C$ that is
    \begin{itemize}
        \item stable under the tangent functor, $d \in \d \Rightarrow T.d \in \d$,
        \item $T$-stable under reindexing (as in Proposition \ref{prop:submersion-properties}).
    \end{itemize}
    We call any tangent display system that is closed to retracts in the arrow category a \textit{retractive} display system. If for all $M$, $p_M \in \mathcal{D}$, we call $\mathcal{D}$ a proper (retractive)\pagenote{Brackets are a common notation for optional prefixes, e.g. (co)limits.} display system. 
\end{definition}
This section will show that the submersions in the category of smooth manifolds give a retractive display system, yielding a general construction of retractive display systems from display systems.


The definition of a submersion may be rephrased as follows: $f$ is a submersion if and only if for all $a\in A$ and all $v\in T(B)$ such that $fa = pv$, there exists a $w\in T(A)$ such that $T.f \o w = v$.
This is a \textit{weakly} universal cone over $A \xrightarrow{f} B \xleftarrow{p} TB$: there exists \textit{at least} one morphism into it for any other cone over the diagram. 
\begin{definition}\label{def:weak-pullback}
    A commuting square is a \textit{weak pullback} if for any $x:X \to A$ and $y:X \to B$ so that $fx = gy$, there exists a map $X \to W$ making the following diagram commute:
    \[
        \begin{tikzcd}
        X \ar[bend left]{rrd}{x} \ar[bend right]{ddr}[swap]{y} \ar[dashed]{rd}{\exists} \\
            & W \rar{a} \dar[swap]{b} & A \dar{f} \\
            & B \rar[swap]{g} & C
        \end{tikzcd}
    \]
    If the above diagram is a weak pullback for each $T^n$, then it is a \emph{weak $T$-pullback}.
\end{definition}
\begin{lemma}\label{lem:pb-retract}
    Should the pullback of $A \xrightarrow{f} C \xleftarrow{g} B$ exist, Definition \ref{def:weak-pullback} is equivalent to asking that the induced map $(a, b):W \to A \ts{f}{g} B$ be a split epimorphism.
\end{lemma}
\begin{proof}
    Let $r$ be a retract of $(a,b):W \to A\ts{f}{g} B$. For any $X \xrightarrow[]{(x,y)}  A\ts{f}{g} B$, the map $ r \o (x,y)$ exhibits the diagram as a weak pullback. For the converse, the unique map $(a,b):W \to A \ts{f}{g} B$ will be a section of any map $A \ts{f}{g} B \to W$ induced by weak univerality.\pagenote{
    This result was originally stated without proof.
    }
\end{proof}
We now restate the submersion property for a map $f$ using global elements (for all $a\in A$ and all $v\in T(B)$ such that $fa = pv$, there exists a $w\in T(A)$ such that $T(f)w = v$) using generalized elements.
\begin{definition}\label{def:tangent-submersion}
    An arrow $f:A \rightarrow B$ in a tangent category is a \emph{tangent submersion} if and only if the naturality diagram
    \[
        \begin{tikzcd}
            TA \rar{Tf} \dar[swap]{p} & TB \dar{p} \\
            A \rar[swap]{f} & B
        \end{tikzcd}
    \] 
    is a weak $T$-pullback. 
\end{definition}{}
Following Lemma \ref{lem:pb-retract}, in the case that the pullback exists this is equivalent to asking for a section $h: A \ts{f}{p} TB \to TA$ of the horizontal descent $(p,Tf): TA \to A \ts{f}{p} TB$ (this section is sometimes called a \textit{horizontal lift} in differential geometry literature \cite{Cordero1989}). 
In smooth manifolds, the $T$-pullback along the projection $p:T \Rightarrow id$ always exists, so to prove that every submersion is a tangent submersion it suffices to show the existence of a horizontal lift.
\begin{proposition}
    In the category of smooth manifolds, the tangent submersions are precisely the submersions (Definition \ref{def:submersion-sman})
\end{proposition}
\begin{proof}
    There is an explicit construction of a horizontal lift for a classical smooth submersion in VII.1 of \cite{Cushman2015}.
\end{proof}
It is possible to show that the $T$-stability properties for submersions in the category of smooth manifolds follow from the general theory of weak pullbacks.
We begin by showing that weak pullbacks satisfy a weakened version of the pullback lemma and then show that the retract of a weak pullback is a weak pullback (the second lemma is Lemma 2.1 of \cite{Adamek2010}).
\begin{lemma}[Pullback lemma]\label{lem:weak-pullback}
    Consider the diagram
    \[
        \begin{tikzcd}
            \bullet \rar \dar \ar[rd, phantom, "(A)"] & \bullet \rar{f}\dar{g} \ar[rd, phantom, "(B)"] & \bullet\dar \\
            \bullet \rar & \bullet \rar & \bullet 
        \end{tikzcd}
    \]
    \begin{enumerate}[(i)]
        \item If $f,g$ are jointly monic and $(A)+(B)$ is a weak pullback, then $(A)$ is a weak pullback.
        % \item If $(B)$ is a pullback and $(A)$ is a weak pullback, then $(A+B)$ is a weak pullback.
        \item If $(A),(B)$ are weak pullbacks then $(A)+(B)$ is a weak pullback.
    \end{enumerate}
\end{lemma}    
\begin{proof}
    ~\begin{enumerate}[(i)]
        \item If $(A)+(B)$ is a weak pullback, a map can be induced for a cone over $(A)$ by concatenating it with $(B)$; the jointly monic condition on $f,g$ guarantees that the map induced for $(A)+(B)$ will commute for $(A)$. \pagenote{caught a broken sentence here.}
        % \item Given a cone for $(A+B)$, induce the unique map for $(B)
        \item Given a cone for $(A)+(B)$ induce a map for $(B)$, which then induces a cone for $(A)$.
    \end{enumerate}
\end{proof}

\begin{lemma}\label{lem:wpb-retract}
    (Weak) pullbacks are closed to retracts.
\end{lemma}
\begin{proof}
    Suppose that $S'$ is a weak pullback, and $S$ is a retract of it in the category of commuting squares. Consider the following diagram (suppressing the subscripts for $s,r$):
    \input{TikzDrawings/Ch1/submersion-cube-diagram.tikz}
    Given a cone for $S$, there is a corresponding cone for $S'$ which induces a map $Z \to A'$ and postcomposition with $r_A$ gives the desired map into $A$.
\end{proof}
Using these lemmas, it is straightforward to prove that the following $T$-stability properties hold for tangent submersions.
\begin{lemma}\label{lem:submersion}
    In any tangent category $\mathbb{X}$,
    \begin{enumerate}[(a)]
        \item tangent submersions are closed to composition;
        \item tangent submersions are closed to retracts; 
        \item any $T$-pullback of a tangent submersion is a tangent submersion.
    \end{enumerate}
\end{lemma}
\begin{proof}
    (a) follows from Lemma \ref{lem:weak-pullback} while (b) follows from Lemma \ref{lem:wpb-retract}. It remains to prove (c).     Consider a $T$-pullback, where $u$ is a tangent submersion:
    \[
    \begin{tikzcd}
        A \rar{f} \dar{v} & M \dar{u} \\
        B \rar{g} & N
    \end{tikzcd}
    \]
    By naturality, the outer paths of the following two diagrams are equal: 
    \pagenote{The equality of the two diagrams was originally asserted without explaining why they were equal.}
    %NOTE: clarified a point raised by Kristine
    \[
        \begin{tikzcd}
            TA \rar{Tf} \dar{Tv} & TM \rar{p} \dar{Tu} & M \dar{u} \\
            TB \rar{Tg} & TN \rar{p} & N 
        \end{tikzcd}
        =
        \begin{tikzcd}
            TA \rar{p} \dar{Tv} & A \rar{f} \dar{v} & M \dar{u} \\
            TB \rar{p} & B \rar{g} & N
        \end{tikzcd}
    \]
   Note that the left diagram is a weak pullback by composition.  Therefore the outer perimeter of the right diagram is a weak pullback, and the right square is a pullback, so the left square is a weak pullback by Lemma \ref{lem:weak-pullback}, as desired.
\end{proof}
The class of tangent submersions is closed to retracts in the arrow category and is conditionally $T$-stable under reindexing (if the $T$-pullback of a tangent submersion exists, it is a tangent submersion). This stability property leads to the following result:
\begin{proposition}\label{prop:display-submersions-are-r-display}
    Let $\mathbb{X}$ be a tangent category that allows for reindexing of the class of tangent submersions $\mathcal{R}$. Then the class of tangent submersions is a display system.
\end{proposition}
\begin{proof}
    Any class of maps that is closed to reindexing is a tangent display system, and the class of submersions is closed to retracts in the arrow category.
\end{proof}
% In the category of smooth manifolds, where the class of smooth submersions is the canonical example of a proper tangent display system, this gives the following:
\begin{corollary}\label{cor:sman-r-display}
    The class of submersions in the category of smooth manifolds is a  proper retractive display system.
\end{corollary}

