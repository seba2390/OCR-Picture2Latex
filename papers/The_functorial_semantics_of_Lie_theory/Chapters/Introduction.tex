\chapter*{Introduction}

% \begin{itemize}
%     \item The study of Lie groupoids and Lie algebroids goes back to Ehresmann's initial investigations into sketch theory, in particular ``differentiable groupoids'', which are internal groupoids in the category of smooth manifolds.  His student Pradines identified that differentiable groupoids admit an ``infinitesimal approximation'', just as the single-object version of a groupoid (a Lie group) is approximated by a Lie algebra. This situated the study of differentiable groupoids and their infinitesimal approximations, now called Lie groupoids and Lie algebroids, as the ``many-object'' analogue of classical Lie theory, which had been developed throughout the early 1900's by Poincare, Cartan.
%     \item 
% \begin{itemize}
%     \item The study of Lie groupoids and algebroids goes back to Ehresmann and his student Pradines in the late 50's, while 
%     \item Ehresmann introduced the notion of a differentiable groupoid to model point-dependent symmetries in a smooth manifold, extending the notion of a Lie group action (which models global symmetries on a manifold). 
%     \item His student, Pradines, identified the correct notion of an \emph{infinitesimal approximation} of a differentiable groupoid, and modified the definition of a differentiable groupoid to properly construct a many-object analogue of the 
% \end{itemize}
% \end{itemize}

The study of Lie groupoids and Lie algebroids goes back to Charles Ehresmann and his student Jean Pradines in the late 1950s, building upon Sophus Lie's original research into the application of groups of smooth symmetries to solving ordinary differential equations (\cite{lie1893theorie}). Motivated by partial differential equations,  \cite{ehresmann1959categories} introduced the notion of a \emph{differentiable groupoid}, which models the \emph{internal} symmetries of a smooth manifold, in contrast to the \emph{external} symmetries given by a Lie group (a group object in the category of smooth manifolds) or Lie group action.  \cite{Pradines1967} extended the Lie functor (which sends Lie groups and Lie group actions to Lie algebras and Lie algebra actions) from external to internal symmetries and introduced the notion of a \emph{Lie algebroid}. In doing so, he identified some shortcomings in Ehresmann's original definition of differentiable groupoids, introducing the modern notion of a \emph{Lie groupoid}. 

Ehresmann's investigations into differentiable groupoids initiated one of the major differential geometry research programmes of the second half of the twentieth century, the study of Lie groupoids and Lie algebroids (which one may refer to as \emph{many-object} Lie theory to distinguish it from the ``single-object'' Lie groups and Lie algebras that had classically been studied). Research into many-object Lie theory in the 80s and 90s focused on extending Lie's second theorem and the Cartan--Lie theorem, which in modern terms state that Lie algebras form a \emph{coreflective subcategory} of Lie groups; that is, the Lie functor has a fully faithful left adjoint (\cite{MACKENZIE2000445,moerdijk2002integrability,nistor2000groupoids}). The left adjoint is often called \emph{Lie integration}, and in their famous paper \cite{Crainic2003} found the exact conditions governing whether a Lie algebroid integrates to a Lie groupoid. \cite{Weinstein1996} initiated a line of research into classical mechanics on Lie algebroids and groupoids, extending Poincar\`{e}'s development of mechanics on a space with a Lie group action and the Euler--Poincar\`{e} equations (\cite{poincare1901}; see \cite{marle2013} for a modern treatment);  this has been further developed by Eduardo Martinez and his collaborators (\cite{de2005lagrangian,Martinez2001,Martinez2018,fusca2018}). 
% Problem of integration
% Weinstein's work extending mechanics (very close to Poincare's original EP-eqs)
% Multiplicative structures

However, Ehresmann's work in differentiable groupoids signalled a change in focus for his own research, as he increasingly focused within the then-new area of category theory. Over the course of the 60s and 70s Ehresmann published a string of influential papers in the nascent area of \emph{functorial semantics}, developing the formalism of sketch theory. 
Sketch theory has proven to be highly influential in mathematical logic, and has been active for some 40 years, being extended to syntax/semantics adjunctions \cite{gabriel2006lokal},  enriched category theory \cite{Kelly1982} and generalized limit doctrines \cite{ADAMEK20027}.
The influence of sketch theory can still be seen on Lie theory in the work of Kirill Mackenzie and his collaborators to develop the theory of double Lie algebroids, Lie-algebroid groupoids, double-vector bundles, and other tensor-product theories (\cite{Mackenzie1992, Mackenzie2011}).

This thesis aims to provide a structural account of the Lie functor from Lie groupoids to Lie algebroids, using tangent categories \cite{Cockett2014} to unify Ehresmann's many-object Lie theory and sketch theory. Tangent categories provide a syntactic description of tangent structure based on Kock and Lawvere's synthetic differential geometry (\cite{Kock2006}, \cite{Lawvere1979}) and the Weil functor formalism (a comprehensive account may be found in \cite{Kolar1993}, while the explicit link to abstract tangent structure is found in \cite{Leung2017}, another line of research in differential geometry that has run parallel to modern Lie theory (albeit with some exchange, e.g. \cite{kolar2007functorial}). Recent work has recast tangent categories as a class of \emph{enriched categories} (\cite{Garner2018}), making it possible for modern techniques from sketch theory and functorial semantics to be applied to differential geometry. In doing so, we demonstrate that the language of tangent categories sheds light on the study of classical mechanics on Lie algebroids and groupoids, as well as Mackenzie's investigations into ``Ehresmann doubles'' of vector bundles and Lie algebroids.

\subsection*{Overview}
The first three chapters of this thesis build on previous work in the tangent category literature (for example, \cite{Cockett2017,Cockett2018,LucyshynWright2018}), providing tangent-categorical sketches of differential-geometric structures.  These structures follow Ehresmann's original notion of a sketch quite closely; they are specified as graphs (a collection of objects and arrows in the category) with a set of diagrams that must commute and cones that must be universal, except that the data may now include the tangent functor $T$ and the tangent natural transformations $p$, 0, +, $\ell$, and $c$. Each sketch is accompanied by a proof that its category of models in smooth manifolds (which we shall often write $\mathsf{SMan}$) is precisely the category of geometric structures it seeks to model. 

The first chapter reviews the basic theory of tangent categories, paying particular attention to the category of smooth manifolds. The first examples of ``sketches'' from the tangent categories literature are covered, namely differential objects and affine connections, which model vector spaces and connections respectively (\cite{Cockett2017,Cockett2018}). The chapter concludes with a study of \emph{tangent submersions}, which model submersions from classical differential geometry and are a useful example of the sort of work that occurs in Chapters \ref{ch:differential_bundles} and \ref{ch:involution-algebroids}.

The second and third chapters develop tangent categorical sketches for vector bundles and Lie algebroids respectively. Chapter \ref{ch:differential_bundles} extends the observation due to \cite{Grabowski2009} that the category of vector bundles is a full subcategory of multiplicative monoid actions by the non-negative reals $\R^+$, and then applies the \emph{Euler vector field} construction (Definition \ref{def:evf}). The tangent categorical sketch for a vector bundle, called a \emph{differential bundle}, is then developed based on a morphism $\lambda:E \to TE$,  and an isomorphism of categories between differential bundles in $\mathsf{SMan}$ and the category of smooth vector bundles is proved. Chapter \ref{ch:involution-algebroids} introduces \emph{involution algebroids}, which replace the bracket of a Lie algebroid with an involution map
\[
   \sigma: \prolong \to \prolong
\]
(where $\anc:A \to TM$ is the \emph{anchor} of the Lie algebroid). Using Martinez's presentation of the \emph{structure equations} for a Lie algebroid (\cite{Martinez2001}), we are once again able to prove an isomorphism of categories, this time that the category of Lie algebroids is isomorphic to that of involution algebroids in $\mathsf{SMan}$. This provides the initial bridge between differential geometric structures and tangent categorical sketches, making it possible to apply more sophisticated techniques in Chapters \ref{chap:weil-nerve} and \ref{ch:inf-nerve-and-realization}.

% and introduces differential objects, affine connections, and tangent submersions, which are tangent-categorical sketches for vector spaces, affine connections, and submersions, respectively. 


% The most important part of each result is that it proves an \emph{isomorphism} of categories between differential geometric structures and the category of models for a tangent-categorical sketch. This provides the initial bridge between the theory of enriched sketches and differential geometric structures, particularly into the work of Mackenzie.

The fourth chapter constructs a syntactic tangent category for Lie algebroids, and demonstrates that Lie algebroids are precisely generalized tangent bundles. We call this result the \emph{Weil nerve}, as it follows the same structure as Grothendieck's original nerve theorem \cite{Segal1974}, in this case using the \emph{categories} presentation of tangent categories due to \cite{Leung2017}. This result has useful implications for the study of generalized mechanics and geometric structures on Lie algebroids, as it introduces a novel tangent structure on the category of Lie algebroids. This novel tangent structure corresponds to Poincar\`{e}/Weinstein/Martinez's characterization of classical mechanics on a Lie algebroid.


The fifth and final chapter introduces the enriched categories perspective on tangent categories from \cite{Garner2018}, so that the work in Chapters \ref{ch:differential_bundles} and \ref{ch:involution-algebroids} may be rephrased using the enriched sketches of \cite{Kelly2005}.  We construct a functor from the syntactic category of involution algebroids to the syntactic category of a groupoid-in-a-tangent-category, thus giving a presentation of the Lie functor in the spirit of Ehresmann's sketch theory. The syntactic version of the Lie functor is built by constructing another novel tangent structure on the category of Lie groupoids (or more generally, groupoids in a tangent category), which also agrees with previous investigations into classical mechanics on a Lie groupoid. As a final result, we demonstrate that in a \emph{locally presentable tangent category} we may use a left Kan extension to construct a left adjoint to the Lie functor, which we call the the \emph{Lie realization}.

% signalled a change in focus for his research programme, as he increasing worked in the then-new area of category theory and 

% Functorial semantics, first introduced in the thesis \cite{Lawvere1963}, remains one of category theory's most important contributions to the broader fields of mathematics and theoretical computer science. Classically, a Lie group is a group whose underlying set has a smooth structure and whose multiplication map is a smooth map; a homomorphism would be a set map that is both smooth and preserves the multiplication. Lawvere's insight was that there should be a syntactic category $\mathsf{Grp}$ so that a group is precisely a product preserving functor
% \[
%     G: \mathsf{Grp} \to \s.
% \]
% Moreover, a group homomorphism is a natural transformation between these functors.
% A Lie group, then, is a product preserving functor into the category of smooth manifolds
% \[
%     H: \mathsf{Grp} \to \mathsf{SMan}.
% \]
% Lawvere's notion of an algebraic theory was extended to \emph{sketches} by \cite{Ehresmann1968b}, which are small categories equipped with a small set of limit cones $(\T, \th)$. A \emph{model} then, is a functor into a category
% \[
%     M: \T \to \C
% \]
% that sends every limit cone in $\th$ to a limit in $\C$. Sketches allow for greater flexibility in describing categorical structures than Lawvere theories, capturing theories with partially-defined operations such as groupoids. Thus, using sketches on may regard an internal groupoid in $\C$ as a functor $\th_{\mathsf{Gpd}} \to \C$, internal functors are precisely natural transformations. Whenever a category $\C$ is \emph{locally finitely presentable} (see \cref{sec:enriched-nerve-constructions}), a morphism of sketches will induce an adjunction between the categories of models of the two sketches in $\C$ - this is the syntax/semantics adjunction of \cite{gabriel2006lokal}.
 
% Lie theory, then, occupies a unique position in the worldview of functorial semantics (see \cite{Duistermaat2000,Mackenzie2005} for relevant background). A Lie algebra is an $\R$-vector space equipped with a bracket operation:
% \[
%     [-,-]: V \x V \to V
% \]
% that satisfies similar axioms to the set of vector fields on a smooth manifold: it is bilinear, it is alternating, and it satisfies the \emph{Jacobi identity}
% \[
%     [a,[b,c]] + [c, [a,b]] + [b, [c,a]] = 0    
% \]
% Lie's second theorem and the Cartan-Lie theorem (sometimes called Lie's third theorem) combine to give the fundamental theorem of Lie theory, 
% Lie algebras are a coreflective subcategory of Lie groups.
% % https://q.uiver.app/?q=WzAsMixbMCwwLCJcXG1hdGhzZntMaWVHcH0iXSxbMiwwLCJcXG1hdGhzZntMaWVBbGd9Il0sWzEsMCwiIiwxLHsib2Zmc2V0IjoyLCJzdHlsZSI6eyJ0YWlsIjp7Im5hbWUiOiJob29rIiwic2lkZSI6InRvcCJ9fX1dLFswLDEsIkxpZSIsMix7Im9mZnNldCI6MiwiY3VydmUiOjJ9XSxbMiwzLCIiLDEseyJsZXZlbCI6MSwic3R5bGUiOnsibmFtZSI6ImFkanVuY3Rpb24ifX1dXQ==&macro_url=https%3A%2F%2Fraw.githubusercontent.com%2Fbenjamin-macadam%2Ftex-preamble%2Fmain%2Fpreamble.sty
% \[\begin{tikzcd}
%     {\mathsf{LieGp}} && {\mathsf{LieAlg}}
%     \arrow[""{name=0, anchor=center, inner sep=0}, shift right=2, hook, from=1-3, to=1-1]
%     \arrow[""{name=1, anchor=center, inner sep=0}, "Lie"', shift right=2, curve={height=12pt}, from=1-1, to=1-3]
%     \arrow["\dashv"{anchor=center, rotate=-90}, draw=none, from=0, to=1]
% \end{tikzcd}\]
% This adjunction, however, is not induced by a morphism between the syntactic categories of groups and Lie algebras, as the construction takes the \emph{infinitesimal approximation} of the group: its tangent bundle above the identity element. There is no obvious morphism from the theory of Lie algebras into the theory of groups that induces the functor $Lie: \mathsf{LieGp} \to \mathsf{LieAlg}$, so this adjunction seems as though it does not fit into the syntax/semantics adjunction.

% The study of Lie theory was further developed by Ehresmann and his school as part of his research into sketches, with the notion of a \emph{Lie groupoid}, a groupoid internal to the category of smooth manifolds \cite{Ehresmann1970}. A Lie groupoid is a ``multi-object'' Lie group, and Lie groups are a full subcategory of Lie groupoids, so this motivated the introduction of a multi-object Lie algebra in \cite{Pradines1967}, Lie \emph{algebroids}. There is a Lie functor, a straightforward generalization of the single-object case, from the category of Lie groupoids to Lie algebroids, but Lie's theorem fails in this case: there are very specific conditions when a Lie algebroid ``integrates'' to a groupoid satisfying the hypotheses of Lie II and the Cartan-Lie theorem that are due to \cite{Crainic2003}.
% Furthermore, Lie algebroids do not seem to be the models of a sketch; they are vector bundles (these are not sketchable due to the local triviality condition on the projection) equipped with a Lie algebra on the space of sections (sketches cannot access the set of sections of a map). Thus, one might think that the Lie adjunction is a purely geometric phenomenon rather than an instance of functorial semantics.

% However, functorial semantics permeate the work of Kirill MacKenzie and his collaborators, who developed the theory of double Lie groupoid and double algebroids \cite{Mackenzie1992}, VB-algebroids and VB-groupoids \cite{Bursztyn2016}, for example.  These structures all follow the intuition of the tensor product of sketches, introduced in \cite{Lair1975} and discussed in \cite{Gray1987,Gray1989}. Thus the problem becomes finding an appropriate framework for functorial semantics that include ``infinitesimal'' structure. Recent work in \emph{tangent categories}, introduced in \cite{Rosicky1984} and redeveloped in \cite{Cockett2014}, provide a natural solution. 

% A tangent category axiomatizes the structure of the tangent bundle from the category of smooth manifold using an endofunctor $T$ along with five natural transformations: the projection $p:T \Rightarrow id$, the zero section $0:id \Rightarrow T$, the addition map $+:T \ts{p}{p} T \Rightarrow T$, the vertical lift $\ell:T \Rightarrow T^2$, and the canonical flip $c: T^2 \Rightarrow T^2$.
% In a tangent category, algebraic structures can include the tangent bundle and its structure maps in their definition (a first example may be found in \Cref{sec:diff-and-tang-struct}).
% The significant work, then, is in axiomatizing a notion of ``vector bundle'' (\Cref{ch:differential_bundles}) and ``Lie algebroid'' (\Cref{ch:involution-algebroids}) that may be expressed as sort of tangent-categorical sketches - in fact, we may show that Lie algebroids are \emph{algebraic} in an appropriately general sense in \Cref{ch:inf-nerve-and-realization}.
% The enriched perspective makes it clear that there is a morphism from the theory of Lie algebroids to the theory of groupoids-in-a-tangent-category, inducing the Lie functor via functorial semantics.

% \section*{Overview}%
% \label{sec:overview}

% Roughly speaking, Chapters \ref{ch:tangent_categories}, \ref{ch:differential_bundles}, and \ref{ch:involution-algebroids} focus on the category of smooth manifolds, using the same technology that may be found in \cite{Cockett2014,Cockett2018,Cockett2019}.
% In Chapters \ref{chap:weil-nerve} and \ref{ch:inf-nerve-and-realization}, the constructions follow more closely to techniques from formal category theory - in particular \emph{nerve} constructions. Following \cite{Kan1958}, the nerve of a functor $K:\a \to \C$ is the functor 
% \[
%     N_K: \C \to \widehat{\a}; \hspace{1cm} C\mapsto \C(K-, C): \a \to \s
% \]
% (where $\widehat{\a}$ is the category of presheaves on $\a$.)
% Chapter \ref{ch:tangent_categories} introduces tangent categories following the classical development found in most treatments of differential geometry, starting with the classical differential calculus - in this case, axiomatized as \emph{cartesian differential categories} following \cite{Blute2009} - and then moves on to the category of smooth manifolds. 
% Section \ref{sec:smooth-manifolds} outlines the tangent bundle construction for smooth manifolds and then identifies the structure maps of a tangent category and their universal properties. 
% Section \ref{sec:tangent-structure} introduces abstract tangent structure and the 2-category of tangent categories. The chapter concludes by considering submersions in the category of smooth manifolds and showing they are precisely \emph{tangent submersions}.

% Chapter \ref{ch:differential_bundles} is a study of \emph{differential bundles} that addresses the first issue of sketching Lie algebroids - vector bundles are not sketchable.
% \cite{Cockett2018} introduced differential bundles as a tangent categorical axiomatization of vector bundles. This chapter drastically simplifies the definition of a differential bundle and demonstrates an isomorphism of categories between differential bundles and vector bundles in smooth manifolds.

% Chapter \ref{ch:involution-algebroids} introduces \emph{involution algebroids}, the categorical axiomatization of Lie algebroids. The chapter begins with an overview of prior work characterizing Lie algebroids, particularly Martinez's construction of the prolongation of a Lie algebroid and the canonical involution map on the prolongation of a Lie algebroid. Involution algebroids, in a sense, axiomatize Martinez's involution map, and Section \ref{sec:the-isomorphism-of-categories} exhibits an isomorphism of categories between Lie algebroids and involution algebroids in smooth manifolds.
% % The chapter concludes by revisiting the Lie functor from groupoids to Lie algebroids. However, it follows an idea due to \cite{Burke2018} of instead looking at \emph{cubical objects} - this extends the Lie functor to a much larger class of cubical objects. 

% Chapter \ref{chap:weil-nerve} contains the main technical result of this thesis - the characterization of involution algebroids in $\C$ as a full subcategory of the tangent functors from the category of Weil algebras into $\C$. In the category of smooth manifolds, this provides functorial semantics for Lie algebroids. The proof of the ``Weil nerve'' construction follows similar to Segal's construction to the nerve of an internal category - of course, the category of Weil algebras carries more structure than the simplex category, so this construction is significantly more technical. This section introduces the \emph{actegory} perspective on tangent categories of \cite{Leung2017}, so that a tangent category is a category equipped with a monoidal action by the category of Weil algebras. This framework makes it clear that there is a second tangent structure on the category of Lie algebroids. This new tangent structure on Lie algebroids agrees with the construction of Lagrangian mechanics on Lie algebroids found in \cite{Martinez2001}, where the prolongation plays the role of the tangent bundle.

% Chapter \ref{ch:inf-nerve-and-realization} makes the Weil nerve a rigorous construction, using the language of enriched categories. The first four sections are spent translating the previous four chapters into the enriched framework, so that the nerve theorem of Chapter 4 is translated into a monadicity result. The final section exhibits the infinitesimal approximation of a groupoid as an approximation in the sense of Kan's simplicial approximation - the functor is given as a nerve:
% \[
%     N_\partial: \mathsf{Gpd}(\w) \to \mathsf{Inf}(\w)  
% \]
% so that there is a ``geometric realization'' left adjoint from $\mathsf{Inv}(\w) \to \mathsf{Gpd}(\w)$.
% % C chapter first introduces the formal machinery for enriched nerve constructions in \cref{sec:enriched-nerve-constructions} - including the generalization of Kan's simplicial nerve construction and the nervous monad-theories framework of \cite{Bourke2019}. \Cref{sec:tang-cats-enrichment}, reviews Garner's work characterizing tangent categories as a variety of enriched categories, so that the formal machinery of enriched category theory as in \cite{Kelly2005} may be applied to tangent categories. \Cref{sec:revisiting-segal-conds} then exhibits the category of pre-differential bundles in $\C$ as an enriched functor category, the category of differential bundles as a reflective subcategory of this functor category, the category of anchored bundles as monadic over this functor category, and the category of involution algebroids as monadic over the category of anchored bundles (this agrees with a result due to \cite{Kapranov2007}). In \Cref{sec:revisiting-segal-conds}, it is shown that the Weil nerve is an instance of the nervous theory formalism. Finally, the thesis concludes with \cref{sec:inf-nerve-of-a-gpd}, where the Lie functor for groupoids is exhibited as a nerve in $\w$ (\Cref{thm:inf-gpd}) and a weighted limit in an arbitrary tangent category $\C$ (Observation \ref{obs:derivative-weighted-limit}). The presentation of the Lie functor for groupoids in $\w$ as a nerve allows for the \emph{realization} result, giving an adjunction between groupoids and algebroids in $\w$ (\Cref{thm:lie-realization}). 

% % \Cref{sec:inf-nerve-of-a-gpd} sharpens the cubical nerve result from \cref{sec:gpds-and-cub}, and constructs a functor $\wone^{op} \to \mathsf{LocGpd}$ that acts like the simplices from Kan's simplicial nerve construction - the nerve of this functor is exactly the Lie functor to algebroids, thus exhibiting the Lie functor as a construction in functorial semantics \Cref{sec:realization-of-inf-nerve} shows that in a locally presentable tangent category, this nerve functor admits a left adjoint, and leaves open the question as to how this compares to the integration procedure introduced in \cite{Crainic2003}.

