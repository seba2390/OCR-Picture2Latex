% \documentclass[main.tex]{subfiles} 

% \begin{document}
\chapter{Conclusions and future work}

\section{Conclusions}

We now take stock of what this thesis has accomplished.

\paragraph{Enriched essentially algebraic presentations of geometric structures}

This thesis has been concerned with categories of vector bundles and Lie algebroids in smooth manifolds.
While these are not models of a limit sketch in the category of smooth manifolds, classical results such as the Serre--Swann theorem or Vaintrob's presentation of Lie algebroids indicate that these categories do have some algebraic description.
By combining the results in Chapters \ref{ch:differential_bundles} and \ref{ch:involution-algebroids}, we can see that vector bundles and Lie algebroids are characterized by tangent categorical gadgets, differential bundles (Theorem \ref{iso-of-cats-dbun-sman}) and involution algebroids (Theorem \ref{thm:iso-of-cats-Lie}).
By combining these observations with the enriched perspective on tangent categories, the results in Section \ref{sec:enriched-structures} exhibit vector bundles and Lie algebroids as models of \emph{enriched sketches} (Proposition \ref{prop:Lambda-is-refl-subcat} and Theorem \ref{thm:pullback-in-cat-of-cats-inv-algd}), as can also be found in Chapter 6 of \cite{Kelly2005}.
Note that from this perspective, the construction of the Weil nerve of an involution algebroid is mostly important as an intermediate step.

\paragraph{New tangent structures for Lie algebroids and groupoids}

It is well known that the categories of Lie algebroids and groupoids have tangent structures induced by post-composition with the tangent functor, yielding the tangent algebroid and tangent groupoid, respectively.
The key result in Chapter \ref{chap:weil-nerve} exhibiting the category of Lie algebroids as transverse-limit-preserving cartesian-tangent functors $\wone \to \mathsf{SMan}$, then, gives the category of Lie algebroids a second tangent structure that corresponds to Martinez's \emph{prolongation} Lie algebroid. 
Similarly, the construction of the infinitesimal nerve of a local groupoid restricts to a new tangent structure on the category of Lie groupoids; in this case, the base space is the Lie algebroid of the groupoid $G$, and the total space is the Lie algebroid of the arrow groupoid $G^2$. 

While this may seem like a piece of formal category theory, it is closely related to symmetry reduction in classical mechanics using Lie theory. This goes back to Poincar\'{e}'s celebrated note \cite{poincare1901}, which introduced the \emph{Euler-Poincare} formulation of Lagrangian mechanics on a manifold equipped with a Lie group action, where the tangent bundle is replaced with a Lie algebra action (see \cite{marle2013} for a modern exposition). Poincar\'{e}'s formalism has been extended to Lie groupoids and Lie algebroids in \cite{Weinstein1996} and \cite{Martinez2001}, and the thesis \cite{fusca2018} investigates fluid mechanics through this lens. This work may be interpreted as using the novel tangent structures on Lie groupoids and Lie algebroids given in Chapters \ref{chap:weil-nerve} and \ref{ch:inf-nerve-and-realization}, and warrants further investigation.

\paragraph{Functorial semantics of Lie theory}
The work in Section \ref{sec:inf-nerve-of-a-gpd} puts the Lie derivative on a new formal grounding:
\begin{enumerate}[(i)]
    \item In the enriching category $\w$, the Lie derivative functor now arises as a nerve/realization context. This guarantees the existence of a left adjoint---the \emph{realization}---that constructs a Lie groupoid from an algebroid. The realization preserves products and is stable over the base space.
    \item It is only a small extension of the work in Section \ref{sec:inf-nerve-of-a-gpd} to show that the Lie derivative of a groupoid in a tangent category may generally be regarded as a weighted limit, where $A_V = \{\partial(V), G\}$ for each Weil prolongation of the involution algebroid of the groupoid.
\end{enumerate}
These appear to be new results, although in a nerve/realization approach they had previously appeared in Lie theory in the form of Sullivan's construction (\cite{Sullivan1977}), which takes the differential graded algebra of a Lie algebroid and constructs a simplicial set:
\[
    \mathsf{DGA}^{op}(\Omega(\Delta_n), A)
\]
where $\Omega(\Delta_n)$ is the de Rham cohomology of the $n$-simplex in cartesian space.

\section{Future work}

This section outlines various lines of research that were either cut from the thesis while writing due to time and/or space constraints, or new lines of research that the thesis-writing process has motivated but which have not yet been pursued.

\paragraph{Enriched sketches and Mackenzie theory}

Following \cite{Voronov2012}, Mackenzie theory refers to the body of research developed by Kirill Mackenzie and his collaborators into Lie theory, particularly structures like double Lie algebroids (\cite{Mackenzie1992}), VB-Lie algebroids (\cite{Bursztyn2016}), double Lie groupoids and so on.
Intuitively, a double Lie algebroid is a Lie algebroid in the category of Lie algebroids, while a VB-Lie algebroid is a vector bundle in the category of Lie algebroids.
These structures have an intuitive relationship with tensor products of sketches; that is, for limit sketches $A,B$ there is a chain of isomorphisms of categories:
\[
    \mathsf{Mod}(A, \mathsf{Mod}(B,\s)) \cong \mathsf{Mod}(A \ox B, \s) \cong \mathsf{Mod}(B, \mathsf{Mod}(A, \s)).
\]
We have already demonstrated that the Lie algebroids and vector bundles are $\w$-sketchable, so the natural next step is to revisit Mackenzie theory via this lens. Certain results such as the symmetry of partial Lie derivatives from \cite{Mackenzie1992} (given a double Lie groupoid, the order in which the Lie functor is applied doesn't matter) should be immediate.

\paragraph{A tangent categorical formalization of mechanics}

Several papers in synthetic differential geometry have translated aspects of Lagrangian and Hamiltonian mechanics into the synthetic setting (\cite{Bunge1984,Nishimura1997a}), and to a degree this has made it difficult to build enthusiasm for a tangent categorical presentation of mechanics. These approaches generally emphasize the ability to construct function spaces, or the use of a topos-theoretic internal language. The novel tangent structures for Lie algebroids and groupoids presented here, however, provide a new line of inquiry: to develop a unified framework for Lagrangian or Hamiltonian mechanics in a tangent category that agrees with the algebroid and groupoid approaches to mechanics.

% \paragraph{}
% \paragraph{The enriched nerve/realization paradigm and differential linear logic}
% A modest extension to the Bourke-Garner framework for nervous theories \cite{Bourke2019} allows for a \emph{monoidal} dense subcategory to act as a system of arities for a commutative algebraic theory. A monoidally dense functor is a strict monoidal functor:
% \[
%     m: \a^\ox \hookrightarrow \C^{\boxtimes}
% \]
% where $(-)\boxtimes(=)$ is cocontinuous in each argument,  and $N_m$ a fully faithful monoidal functor into $\vv$-presheaves on $\a$ with the Day convolution monoidal structure:
% \[
%     \C^{\boxtimes} \hookrightarrow \widehat{\a}^{\ox[Day]}
% \]
% (note that this means the reflection to $\C^\boxtimes$ is monoidal).
% Now, observe that the finite coproduct closure of infinitesimals in $\w$ is a monoidal dense subcategory, $D^+$, and furthermore differential object may be regarded as the concrete models of the $D^+$-theory:
% \begin{itemize}
%     \item Objects:$i \in D^+$,
%     \item Homs: $\d(i,j) = \N^{i \x j}$
%     \item 
% \end{itemize}




% \paragraph{The relationship between local Lie groupoids and local groupoids}

% Throughout \cref{ch:inf-nerve-and-realization} we identified cubical objects in $\w$ satisfying a certain universal property as \emph{local groupoids}.
% The definition of a local groupoid can apply to any tangent category, following the observation at the end of \Cref{sec:linear-approx-of-a-cubical-object}.
% It would useful to know if this does capture the notion of a local Lie groupoid, which is an involutive, reflexive graph equipped with an open set $U \subset G \ts{t}{s} G$ where multiplication is well defined. This would involve checking the local Lie groupoids admit a ``cubical nerve'' construction, and that the resulting cubical object satisfies the Segal conditions for local groupoids.


% \paragraph{Sharpening the Lie functor} The adjunction between $\w$-groupoids and involution algebroids should be compared with other adjunctions constructed w

% \paragraph{Lie integration}
% As stated in \cref{sec:Lie_algebroids}, the theory of Lie groupoids and algebroids is a generalization of the theory of Lie groups and Lie algebras.
% Two of the major theorems for Lie groups and algebras, called Lie II and Lie III (although Lie III is properly credited as the Cartan-Lie theorem), state that the category of finite-dimensional real Lie algebras is a coreflective subcategory of Lie groups.
% However, this result famously fails to generalize to Lie groupoids and algebroids, and exact conditions for when a Lie groupoid \emph{integrating} a Lie algebroid $A$ were provided in \cite{Crainic2003}.
% The original motivation for the development in \cref{chap:microlinear-nerve,ch:inf-nerve-and-realization} was to exhibit Lie integration via pure abstract nonsense - while that construction does warrant further development, at this point the idea of inducing a coreflection via purely abstract results seems to be a dead end.
% However, the cubical approach used in \cref{sec:inf-nerve-of-a-gpd} seems like a good fit with the Crainic and Fernandes approach of building a cubical set:
% \[
%     \mathsf{LieAlgd}((TI)(-), A):\mathsf{LieAlgd} \to \widehat{\square}
% \]
% and quotienting out a groupoid.
% ($I$ in this case is the unit interval, and $TI$ is the tangent Lie algebroid above it).
% Thus, revisiting the Weinstein construction combined with the enriched categorical machinery developed in \cref{ch:inf-nerve-and-realization} seems to be a viable path forwards.


% \end{document}