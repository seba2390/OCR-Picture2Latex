% \documentclass[main.tex]{subfiles}

% \begin{document}

\chapter{The infinitesimal nerve and its realization}%
\label{ch:inf-nerve-and-realization}

This chapter makes use of \emph{enriched nerve} constructions, transferring a classical construction in categorical topology to categorical Lie theory. Recall simplicial localization functor of \cite{Kan1958}, which sends a topological space to $X$ to the simplicial set:
\[
    N_\Delta(X): \Delta^{op} \to \s; [n] \mapsto \mathsf{Top}(\Delta_n, X)
\]
where $\Delta_n$ is the subset of $\R^n$ spanned by:
\[
    \{
        (x_1, \dots, x_n) \in \R^n | \sum x_i \le 1
    \}    
\]
That is, the geometric $n$-simplex. The simplicial set $N_\partial(X)$ can be considered the simplicial approximation of the topological space $X$.
Kan showed that the \emph{topological realization} functor, induced by the left Kan extension:
\[\input{TikzDrawings/Ch5/top-real.tikz}\]
Kan's construction induces an adjunction between simplicial sets and topological spaces, and is essential in relating the homotopy structure on simplicial sets to that of topological spaces.
Our goal, then, is to construct an ``infinitesimal'' version of this story on the category of groupoids in a (suffiently well-behaved) tangent category, so that we induce an adjunction:
\[\input{TikzDrawings/Ch5/lie-adj.tikz}\]
This will similarly construct an adjoint pair between the Lie functor:
\[
    \mathsf{Lie}: \mathsf{Gpd} \to \mathsf{Inv}  
\]
and a \emph{Lie realization} functor, that associates a groupoid to an involution algebroid.


\section{Tangent categories via enrichment}%
\label{sec:tang-cats-enrichment}

In this section demonstrates how tangent categories fit inside the framework of enriched categories.  This was first demonstrated in \cite{Garner2018}, building on the actegory perspective on tangent categories introduced in \cite{Leung2017} and a result relating monoidal actions and enrichment from \cite{Wood1978}. This is mostly expository, however we make time to consider the differential objects in the enriching tangent category, the category of Weil spaces.


Weil spaces are closely related to the \emph{Weil topos}, introduced by Dubuc in his original study of models of synthetic differential geometry \cite{Dubuc1981}. The Weil topos is not a well-adapted model of synthetic differential geometry, and was mostly treated as a toy model, although a deeper study of the topos may be found in \cite{Bertram2014}. Recall that the category $\wone$ is the free tangent category over a single object. The category of Weil spaces may be identified as the \emph{cofree} tangent category over $\s$, which is $\mathsf{Mod}(\wone, \s)$ from Observation \ref{obs:cofree-tangent-cat}. Call this the category of \emph{Weil spaces}, and write it $\w$. Just as a simplicial set $S: \delta \to \s$ can be thought of as a gadget recording homotopical data, a Weil space is a gadget recording \emph{infinitesimal} data.

\begin{definition}
	A \emph{Weil space} is a functor $\wone \to \s$ that preserves transverse limits - that is, the $\ox$-closure of the set of limits:
    \[
        \left\{
            \input{TikzDrawings/Ch5/wone-pb.tikz}
            ,
            \input{TikzDrawings/Ch5/wone-univ-lift.tikz}
            ,
            \input{TikzDrawings/Ch5/wone-id.tikz}
        \right\}
    \]  
	A morphism of Weil spaces is a natural transformation. The category of Weil spaces is written $\w$.
\end{definition}
\begin{example}
	~\begin{enumerate}[(i)]
		\item Every commutative monoid may be regarded as a Weil space in a canonical way.
		Observe that for every $V \in \wone$ and commutative monoid $M$, there is a free $V$-module structure on $M$ given by $|V| \ox_{\mathsf{CMon}} M$.
		The underlying commutative monoid of $V$ is exactly $\N^{\mathsf{\dim V}}$, so that:
		\[
			V \bullet M \cong \oplus^{\dim V} M 	
		\]
		The microlinearity of $V \mapsto |V| \ox M$ follows from basic properties of finite biproducts.
        \item Following (i), any tangent category $\C$ that is \emph{concrete} - it admits a faithful functor $U_\s: \C \to \s$ - will have a natural functor into tangent sets (copresheaves on $\wone$).
        Every object $A$ will have an underlying tangent set $V \mapsto U_{\s}(T^V(A))$, and whenever $U_\s$ preserves connected limits (such as the forgetful functor from commutative monoids to sets), each of the underlying tangent sets will be a propert Weil space.
		\item Consider a symmetric monoidal category with an infinitesimal object, which by \Cref{cor:rewrite-rep-functor} is a transverse colimit preserving symmetric monoidal functor $D:\wone \to \C$. Then for every object, the nerve (\Cref{def:nerve}) $N_D(X): \C(D-,X):\wone \to \s$ is a Weil space. 
		\item For any pair of objects $A,B$ in a tangent category, $\C(A,T^{(-)}B):\wone \to \s$ is a Weil space by the continuity of $\C(B,-):\C \to \s$.
	\end{enumerate}
\end{example}
Unlike the category of simplicial sets, the category of Weil spaces is not a topos due to the restriction to transverse-limit-preserving functors.
The category of Weil spaces does, however, inherit some nice properties from the topos of copresheaves on $\wone$ by applying results from \cref{sec:enriched-nerve-constructions}.
\begin{proposition}[\cite{Garner2018}]
	The category of Weil spaces is a cartesian-monoidal reflective subcategory of $[\wone,\s]$.
\end{proposition}
\begin{corollary}
	The category of Weil spaces is:
	\begin{enumerate}[(i)]
		\item A cartesian closed category.
		\item Locally finitely presentable as a cartesian monoidal category (as in \cref{def:v-lfp}).
		\item A representable tangent category, using $\yon: \wone^{op} \to \w \hookrightarrow [\wone,\s]$.
	\end{enumerate}
\end{corollary}
The cofree tangent structure on $\w$ is given by precomposition, that is:
\[
	T^U.M.(V) = M.(U \ox V) = M.T^U.V
\]
This tangent coincides with the representable tangent structure induced by the Yoneda embedding, which can be seen using a simple application of the Yoneda lemma. The proof is a simple application of the Yoneda lemma. Observe:
\[
    [D,M](V) \cong [\wone,\s](D \x \yon(V), M) \cong [\wone, \s](\yon(T.V)), M) \cong M(T.V)
\]
where $D \x \yon(V) = \yon(T.V)$ as the tensor product in $\wone$ is cocartesian.

% Thus, we set the following piece of notation/terminology:
% \begin{notation}
% 	For a structure in a tangent category (e.g. a differential object, involution algebroid, etc), we call a model of $X$ in $\w$ a tangent $X$.
% \end{notation}

% % \paragraph{The enrichment of a tangent category in $\w$}%
% % \label{sub:enrichment-of-tang-cat}
% The basic idea in enriched category theory is that the definition of a category involves very little about the category of sets.
% Thus, we can define for any monoidal $\vv^\ox$ the notion of a $\vv$-category.


In general, any actegory over a monoidal $\C$ will give rise to $[\C,\s]$-enriched category, using the tensor structure induced by Day convolution, was proved by \cite{Wood1978}, that tangent categories are certain class of enriched category is a consequence of that result when combined with Leung's actegory result.
% We will only look at the result for tangent categories.
\begin{proposition}
	Every tangent structure $(\C, \T)$ equips $\C$ with an enrichment in Weil spaces.
\end{proposition}
\begin{proof}
	The presheaf is constructed:
	\[
		\underline{\C}(A,B) = \C(A, T^{}B): \wone \to \s 
	\]
	The functor $\C(A,-)$ is continuous and $T^{-}B$ is a microlinear complex, so this is a Weil space. The composition map is constructed as:
	\input{TikzDrawings/Ch5/enrichment-in-W-comp.tikz}
	which is natural in $U,V$.
\end{proof}
The following coherence on powers was defined by \cite{LucyshynWright2016}.
\begin{definition}
	Let $\j \hookleftarrow \vv$ be a full monoidal subcategory of $\vv$. A $\vv$-category $\C$ has \emph{coherently chosen powers} by $\j$ if there is a choice of $\j$-powers $(-)^J$ so that:
	\[
		(C^J)^K \cong C^{J \ox K}
	\]
	and coherence chosen copowers whenever there is a choice of $\j$-copowers so that:
	\[
		K \bullet (J \bullet C) \cong (K \ox J) \bullet C	
	\]
	The sub-2-category of $\vv$-categories equipped with coherently chosen powers and copowers are $\vv\cat^\j$ and $\vv\cat_\j$, respectively.
\end{definition}

Now, returning to the specific enrichment in $\w$, remember that the hom-object between $A,B$ is given by:
\[	
	V \mapsto \C(A,T^V B)
\]
For a representable functor $\yon U$, use the fact that the representable tangent structure induced by $\yon$ coincides with the cofree tangent structure, and find:
\[
	\w(\yon U, V \mapsto \C(A, T^VB)) \cong V \mapsto \C(A, T^V.T^U(B))
\]
Then, a tangent category (and any actegory, really) does not simply have an enrichment in $\w$, it also has coherently chosen powers by representables. In fact, $\w$-categories with coherently chosen powers by representables are \emph{exactly} the $\w$ categories that are induced by a tangent structure. Thus we have the following:
\begin{proposition}[\cite{Garner2018}]
	A tangent category is exactly an $\w$-category with coherently chosen powers by representables.
\end{proposition}
Now the original notions of (lax, strong, strict) tangent functors can be defined as $\w$-functors between $\w$-categories with coherently chosen powers, where the lax/strong/strict adjective now relates to the preservation of the coherently chosen powers.
\begin{theorem}[\cite{Garner2018}]
	We have the following equivalences of 2-categories.
	\begin{enumerate}[(i)]
		\item The 2-category of $\w$-categories with coherently chosen powers and $\mathsf{TangCat}_{\mathsf{Lax}}$
		\item The 2-category of $\w$-categories with coherently chosen powers and power-preserving $\w$ functors  and $\mathsf{TangCat}_{\mathsf{Strong}}$
		\item The 2-category of $\w$-categories with coherently chosen powers and chosen-power-preserving $\w$ functors $\mathsf{TangCat}_{\mathsf{Strict}}$
	\end{enumerate}
\end{theorem}
Now, the Yoneda lemma applies of $\w$-categories. So a tangent category may be embedded into its $\w$-category of tangent presheaves:
\[
	F: \C^{op} \to \w 
\]
The powers by representables are computed pointwise, and so they inherit the coherently choice. Thus the following holds:
\begin{corollary}
	Every tangent category embeds into a representable tangent category
	\[
		\C \hookrightarrow [\C^{op},\w] = \widehat\C
	\]
\end{corollary}

% \paragraph{Differential objects and commutative monoids}
% As an application of the enriched perspective of tangent categories, consider the category differential objects in $\w$.
% We find that the category of differential objects is isomorphic to the replete image of $\mathsf{CMon}$ into the category of Weil spaces. 
% \begin{lemma}\label{lem:cmon-tang-inclusion}
% 	There is a cartesian $\w$-inclusion $d:\mathsf{CMon} \hookrightarrow \w$, where $\mathsf{CMon}$ is regarded as a tangent category using the biproduct structure.
% \end{lemma}
% \begin{proof}
% 	Note that because $\mathsf{CMon}$ is a cartesian tangent category, we have powers by representables given by:
% 	\[
% 		A^{W^{n_1} \dots W^{n_k}} = T_{n_1}\circ\dots\circ T_{n_k}(A)
% 	\]
% 	thus we can send $A$ to the presheaf $V \mapsto A^V$.
% \end{proof}
% \begin{remark}
%     It is worth noting that every category with biproducts has this tangent structure, and recalling that product-preserving tangent functors preserve differential objects. Since a cartesian differential category is equivalently a cartesian tangent category with a coherent choice of differential objects, this leads to the idea that ``a CDC is a CLAC in $\w\cat$'' - that is, a cartesian differential category is exactly a $\w$-category $\C$ where there is a category with biproducts $\C^+$ (regarded as a tangent category) and a bijective-on-objects $\w$-functor:
%     \[
%       \C^+ \hookrightarrow \C   
%     \] that creates products and powers by representable functors - this is a straightforward generalization of \Cref*{prop:clac-defs} from cartesian left addtive categories to cartesian differential categories.
% \end{remark}

% \begin{lemma}\label{lem:diff-ob-w-determines-cmon}
% 	A differential object in $\w$ uniquely determines a commutative monoid in $\s$.
% \end{lemma}
% \begin{proof}
% 	First, note that by the symmetry of theories, a commutative monoid in $\w$ is a model of Weil algebras and transverse limits into commutative monoids. This means that for a differential object $(A,+',0')$, we have a commutative monoid
% 	\[
% 		A(R) \times A(R) \xrightarrow{+'_R} A(R) \hspace{0.5cm} 0': 1 \to A(R)
% 	\]
% 	The coherence that $TA \cong A \times A$, and that $+_A$ may be rewritten as $+'$, means that 
% 	\[
%     \input{TikzDrawings/Ch5/wsp-cmon-plus.tikz}
% 	\]
% 	Thus we can see that $A(W^{n_1} \dots W^{n_k}) \cong A(R)^{\prod (n_i + 1)}$. 
% 	Therefore $A$ is determined, up to a unique isomorphism, by the inclusion $(A(R),+'_R,0'_R)$ from $\mathsf{CMon}$ into $\w$.
% \end{proof}
% \Cref{lem:cmon-tang-inclusion} and \cref{lem:diff-ob-w-determines-cmon} yield the following equivalence of categories. 
% \begin{proposition}\label{prop:diff-ob-is-CMon}
% 	The category of differential objects in $\w$ is $\w$-equivalent to the category of commutative monoids.
% \end{proposition}

% We now look at some characterizations of the natural numbers $d(\N)$ in $\w$ - which we will often call $\N$.
% We may associate the ring $\N$ with a certain pullback involving the infinitesimal $D$, and the underlying Weil algebra functor.
% \begin{proposition}\label{prop:ring_properties}
%     Denote the inclusion $I: \mathbf{CMon} \hookrightarrow \w$. The following are equivalent:
%     \begin{enumerate}[(i)]
%         \item $\mathbb{N}$
%         \item The underlying set functor $U:\mathbf{Weil}_1 \to \mathbf{Set}$
%         \item The pullback
%             \input{TikzDrawings/Ch5/N-as-pb.tikz}
%             with $\sigma_A$ induced by $+_D$ and multiplication given by composition.
%     \end{enumerate}
%     Call this universal ring object $R$.
% \end{proposition}
% \begin{proof}
%     We first observe that $\mathcal{U} \cong d(\mathbb{N})$, as $\N$ sends a Weil algebra $W^{n_1} \dots W^{n_k}$ to the underlying set of the commutative monoid $\N^{W^{n_1} \dots W^{n_k}}$.
%     The underlying commutative monoid of $W^{n_1} \dots W^{n_k}$ is 
%     \[ \N^{n_1 + 1} \ox \dots \ox\N^{n_k +1} \cong \N^{\prod n_i + 1} \cong T_{n_1} \dots \circ T_{n_k}\N\]
% 	Thus the isomorphism holds.
    
%     To show $d(\N) \cong [D,D]_0$, first calculate the presheaf $[D,D]$:
%     \begin{align*}
%     [D,D](V)   &\simeq [\yon.T,\yon.T](V) \\
%                &\simeq \mathsf{Nat}(\yon.T \x \yon.T^V,\yon(x^2))\\ 
%                &\simeq \mathsf{Nat}(\yon.T.T^V,\yon.T) \\
%                &\simeq \mathsf{Weil}(T,T.V)
%   \end{align*}
%   Then we may compute the pullback (2) point-wise:
%   \[
%      \input{TikzDrawings/Ch5/two-as-pointwise.tikz}
%   \]
%   First, note that $R(0) = \mathsf{Weil}(x^2,x^2) = \mathbb{N}$ as the only endomorphisms for $\N[x]/x^2$ in $\mathsf{Weil}$ are scalar multiplication $x \mapsto nx$. 
%   $R$ is a differential object by \cref{lem:tang-space-diff-ob}, so by Proposition \ref{lem:diff-ob-w-determines-cmon}, $R(W) \cong \mathbb{N}^W$ and the ring isomorphism $[D,D]_0 \cong I(\mathbb{N})$ is immediate.
% \end{proof}
% \begin{lemma}\label{lem:action_on_T}
%     In a tangent category $\mathbb{C}$, the action of $\mathbb{N}$ on $(p_M, +_m, 0_M)$ coincides with the $ \mathcal{W}$-natural transformations induced by the inclusion \[\N \hookrightarrow [D,D] \to \w\mathbf{nat}(T,T)\]
% \end{lemma} 

% Now, we observe a particularly useful Weil space:
% \begin{definition}
%     Let $E$ be a commutative monoid. The free vector over $E$ is the Weil space $E_0$ where:
%     \[
%         E_0(V) = E^{\dim(V)-1}
%     \]
%     Each $V$ is the fiber of $E(V)$ over $E(\N) \to^{0^V} E(V)$, and the structure transformations are defined accordingly.
% \end{definition}


\section{Enriched nerve constructions}%
\label{sec:enriched-nerve-constructions}

This chapter will be making use of enriched nerve constructions, as developed in \cite{Bourke2019}. 
These constructions will make formal the intuitive notions of the ``Weil nerve'' and the ``Segal conditions for involution algebroids'' developed in \Cref{chap:weil-nerve}. In order to keep this chapter relatively self-contained, this section includes a basic introduction to locally presentable enriched categories, the nerve/realization paradigm, and the notion of a nervous theory.

% \paragraph{A review of locally finitely presentable $\vv$-categories}



% \paragraph{Monads and theories}
The ``nerve'' of an internal category and involution algebroid were discussed in \Cref{sec:nerve-of-a-category,sub:microlinear-nerve-of-an-involution-algebroid}, although the notion of a nerve or Segal condition was not made precise. This section concludes by giving the formal theory of nerves/Segal-condition paradigm from  \cite{Berger2012,Bourke2019}, and the the nerve/realization paradigm in \cite{loregian2015}.

\begin{definition}%
    \label{def:nerve}
    The \emph{nerve} of a $\vv$-functor $K: \a \to \C$ is the functor
    \[
        N_K: \C \to \widehat{\a}; \hspace{0.5cm} C \mapsto \C(K-, C)    
    \]
    (where $\widehat{A}$ is the $\vv$-category of $\vv$-presheaves on small $\vv$-category $A$).
    Whenever $\C$ is cocomplete, the nerve has a left adjoint called the \emph{realization}, computed by the left Kan extension/coend:
    \[
        \input{TikzDrawings/Ch5/realization.tikz}
        \hspace{0.5cm}
        |C|_K := \int^A \C(KA, C) \cdot KA
    \]
    A \emph{nerve/realization context} is $\vv$-functor $K:\a \to \C$ from a small $A$ to a cocomplete $\C$.
\end{definition}
% We can see that the nerve itself may be represented as a left Kan extension:
% \begin{lemma}
The nerve functor itself is equivalently the left Kan extension of $K$ along $\yon$:
\[\input{TikzDrawings/Ch5/nerve-is-lany.tikz}\]
For a proof see \cite{loregian2015}. This gives the unusual property that the nerve/realization adjunction is of the form:
\[
    Lan_{\yon}K \vvdash Lan_K\yon
\]

\begin{example}
    ~\begin{enumerate}[(i)]
        \item Recall the original construction of the simplicial nerve of a topological space $X$, where $X_n = \mathsf{Top}(\Delta_n,X)$.
        This is exactly the nerve of the functor $\bar{\Delta} \to \mathsf{Top}$ that sends $n$ to the $n$-simplex:
        \[
            \{
                x \in \R^n | \sum x_i = 1
            \}
        \]
        \item The theory of locally finitely presentable categories gives a useful class of examples.
        Because we may represent a locally presentable $\C$ as $\mathsf{Lex}(\C_{fp}^{op}, \vv)$ (whenever $\vv$ itself is locally presentable as a closed category), we can see that the nerve of $\C_{fp} \hookrightarrow \C$ is a fully faithful functor, and the realization is the reflection from presheaves on $\C_{fp}$ to models.
    \end{enumerate}
\end{example}

The situation where the nerve of $K$ is fully-faithful particularly important. Kelly called these subcategories \emph{dense}, by metaphor with the notion of density found in Hausdorff spaces. A dense subcategory models the same situation, except we replace ``continuous functions'' with ``cocontinuous functors''. That is to say, a functor $K:\a \to \c$ into a cocomplete $\c$ is \textit{dense} whenever each object $C \in \c_0$ is canonically written as a coend:
\[
	\int^A \C(KA, C)\cdot KA
\]
so that a cocontinuous functor $F:\c \to \d$ is determined by $F.K$.
The general definition of density does not require a cocomplete category.
\begin{definition}
	A functor $K: \a \to \C$ is \textit{dense} whenever one of the following conditions holds
	\begin{itemize}
		\item The identity functor $id:\C \to \C$ is the left Kan extension of $K$ along itself, so: $Lan_K K = id$.
		\item Every object is given as the weighted colimit $C = N_K(C) \star K$.
		\item The nerve of $K$ is fully faithful.
	\end{itemize}
\end{definition}
Dense functors are poorly behaved under composition, but there is a useful cancellativity results from Chapter 5.2 of \cite{Kelly2005}.
\begin{proposition}
    Consider a diagram of $\vv$-categories
    \[\input{TikzDrawings/Ch5/lan-result.tikz}\]
    where $\alpha$ is a natural isomorphism, and $K$ is dense. 
    If this diagram exhibits $(\alpha, J)$ as the left Kan extension of $K$ along $F$, then $J$ is dense.
\end{proposition}
\begin{corollary}%
    \label{cor:dense-ff-result}
    If $K$ is dense, $F$ is fully faithful, and  $J.F = K$, then $F$ is a dense subcategory.s
\end{corollary}
Generally speaking, we will often refer to dense \textit{subcategories} rather than dense \textit{functors}. Through a naive application of the (bo,ff)-factorization system on $\vv$-categories, every dense functor induces a dense subcategory, as $K$ factors as
\[
    K = \a \xrightarrow[]{K'} \mathsf{im}(K) \hookrightarrow \C
\]
so the inclusion of $\mathsf{im}(K)$ into $\C$ is dense.
% \begin{proposition}
% 	Let $K:A \to \c$ be a dense functor. Then the full subcategory $im(K)$ is a dense subcategory.
% \end{proposition}
% \begin{proof}
% 	Use \cref{cor:dense-ff-result}.
% \end{proof}
% \begin{example}
% 	Consider the category of transverse-limit preserving copresheaves of $\wone$ algebras $\w = \mathsf{Mod}(\wone, \s)$, the category of Weil spaces. We have the following dense subcategories:
% 	\begin{enumerate}
% 		\item The unit: $K_1: 1 \hookrightarrow \w$
% 		\item The infinitesimals: $D: \wone^{op} \hookrightarrow \w$
% 		\item The finite-coproduct closure of infinitesimals $\{ \sum D_v | v \in \wone \} \hookrightarrow \w$ 
% 	\end{enumerate} 
% \end{example}
The category of finite sets is dense in $\s$, as $\s$ is locally finitely presentable and the finitely presentable sets are exactly finite sets. In particular, this means that the skeleton of finite sets, where each object is $[n] = \{1, \dots, n\}$, is dense in $\s$. The general idea of density as a generalization of this setting, and find the abstract notion of a Segal condition.
\begin{example}
    ~\begin{enumerate}[(i)]
        \item Consider a Lawvere theory $\th$, regarded as a cartesian category equipped with a bijective-on-objects, product-preserving functor $t:\mathsf{FinSet}_{\mathsf{skel}}^{op} \to \th$. Every object in $\mathsf{FinSet}_{\mathsf{skel}}$ is equivalent to a $[n] \cong \sum^n 1$. Then the category of models of $\th$ is equivalent to the pullback in $\mathsf{CAT}$:
        \[\input{TikzDrawings/Ch5/mod-of-th.tikz}\]
        \item Recall that the Segal condition states that a simplicial set $X$ is the nerve of a category if and only if each $X_n$ is the set of $n$-composible arrows of the underlying graph $s,t:X_1 \to X_0$. Now, recall that the category of graphs is exactly the category of presheaves over the free category generated by a pair of parallel arrows $s,t: 0 \to 1$, so any full subcategory of graphs containing $\yon 0, \yon 1$ is dense. We construct a dense subcategory of $\mathsf{Gph}$ where:
       \[ \input{TikzDrawings/Ch5/dense-in-graphs.tikz}\]
        so that $\mathsf{Gph}([n],G)$ picks out the set of paths of length $n$ in a graph $G$. We call this subcategory $P:\mathsf{Pth} \to \mathsf{Gph}$, and see that $N_P$ sends a graph to its $\mathsf{Pth}$-presheave of composible paths.
        Observe that we have a bijective-on-objects functor $t:\mathsf{Pth} \to \Delta$, so precomposition $t^*$ sends a simplicial set to the underlying $\mathsf{Pth}$-complex - the Segal conditions then state that the category of small categories is exactly the following pullback in the category of (possibly large) categories $\mathsf{CAT}$:
        \[\input{TikzDrawings/Ch5/pb-in-cat.tikz}\]
    \end{enumerate}
\end{example}


In the usual setting for Lawvere theories, the arities for our function symbols are given by natural numbers (e.g. finite sets), and the category of finite sets is a dense subcategory of $\s$. One can generalize this from $\s$ all the way to an arbitrary locally presentable $\vv$-category $\C$, where a dense subcategory $\a \hookrightarrow \C$ plays the role of the (skeleton of) category of finite sets. Power was the first to extend this to an arbitrary $\vv$-LFP $\C$, where the arguments for a Lawvere theory over $\C$ would then be the finitely presentable objects $\C_{fp}$. Berger-Mellies-Weber generalized this to dense subcategories of $\s$-LFPs $\C$, and Garner-Bourke's paradigm moved to dense sub-$\vv$-categories of LFP $\vv$-categories.

\begin{definition}%
    \label{def:nerve-theory}
	Let $K:\a \to \C$ be a dense $\vv$-subcategory of a locally finitely presentable $\c$.
    We call the replete image of $N_K$ in $\widehat{\a}$ the category of $K$-nerves.
	An \textit{$\a$-theory} is a bijective-on-object $\vv$-functor $J: \a \to \th$, where each
    \[
        \th(J-,a): \a \to \vv
    \]
    is a $K$-nerve.
	The category of concrete models for an $\a$-theory is given by the pullback in $\vv\mathsf{CAT}$:
    \[\input{TikzDrawings/Ch5/conc-models.tikz}\]
    We say that an $\a$-presheaf $X$ satisfies the $K$-Segal conditions whenever $t^*X$ is a $K$-nerve (that is, it is a concrete model of the theory).
\end{definition}
\begin{example}
    ~\begin{enumerate}[(i)]
        \item A functor $t:\mathsf{FinSet_{Skel}} \hookrightarrow \th$ is a $\mathsf{FinSet_{Skel}}$-theory if and only if $\th$ is a Lawvere theory. The Segal conditions in this case identify the models of the Lawvere theory.
        \item The inclusion of $\mathsf{Pth} \hookrightarrow \Delta$ is a $\mathsf{Pth}$-theory for the category of graphs. The Segal conditions in this case capture the classical Segal conditions for categories.
    \end{enumerate}
\end{example}
We do not need the full monad/theory correspondence from Bourke-Garner. However, the following is a useful result:
\begin{proposition}
    Let $K: \a \hookrightarrow \C$ be a dense $\vv$-subcategory, and $J: \a \to \th$ an $\a$-theory.
    The category of concrete models of $J$ is monadic over $\C$.
\end{proposition}
\begin{proof}
    This proof relies on a technical lemma - note that $N_K$ is fully faithful and monadic (as it is the inclusion of full reflective subcategory), and $J^*$ is monadic as $J$ is a bijective-on-objects functor. The pullback along $N_K$ will be fully faithful (as fully faithful functors are the right class of maps for the (bo,ff)-factorization system on $\vv$-categories). 
\end{proof}

This chapter makes use of enriched nerves in a few ways. First, this theory allows for the intuition of ``Segal conditions'' for involution algebroids to be made concrete, by exhibiting the category of involution algebroids as the category of models of a theory over the category of anchored bundles. Second, the Lie derivative from \cref{sec:linear-approx-of-a-cubical-object} may be exhibited as a nerve construction, this puts the Lie derivative into the domain of functorial semantics and in a presentable tangent category will guarantee the existence of a left adjoint - the \emph{realization} of the Lie algebroid. These results make use of the enriched perspective on tangent categories, explained below.

\section{Differential objects as enriched structures}%
\label{sec:diff-obs-enrichment}

Starting with differential bundles, note that a morphism $E \to TE$ is represented by a map:
\[
	\infer{\hat{\lambda}:D \to \C(E,E)}{\lambda: 1 \to \C(E, E^D)}
\]
The commutativity condition, then, is equivalent to asking that the following diagram commutes:
\[\input{TikzDrawings/Ch5/dbun-commutativity.tikz}\]
That is, $\hat{\lambda}$ is a semigroup morphism $D \to \C(E,E)$. 
In any cartesian closed category with coproducts, a semigroup may be freely lifted to a monoid using the "maybe monad" $(-) + 1$.
\begin{definition}
	Define the one-object $\w$-category $\Lambda$ to be given by the monoid 
	\[
		\infer{m: (D+1) \x (D+1) \to D+1}{D \x D + D + D + 1 \xrightarrow{(\iota_L\o m | \iota_L | \iota_L |\iota_R)} D + 1}
	\]
\end{definition}
\begin{lemma}
	The category of lifts in a tangent category $\C$ is isomorphic to the category of $\w$-functors and $\w$-natural transformations $\Lambda \to \C$.
\end{lemma}
\begin{proof}
	We check that a $\w$-natural transformation is exactly morphism of lifts $f: \lambda \to l$.
	Starting with the $\w$-naturality square:
    \[\input{TikzDrawings/Ch5/w-nat.tikz}\]
	Now, rewriting $D+1 \to \C(A,B)$ as a semigroup map $D \to \C(A,B)$, we have:
	\[
		\infer{
			\infer{ 
				1 \xrightarrow{Tf \o \lambda'}\C(A,TB) 
			}{ 
				1 \xrightarrow{(\lambda',f)} \C(A,TA) \x \C(A,B) \xrightarrow{1 \x T} \C(A,TA) \x \C(TA,TB) \xrightarrow{m_{A,TA,TB}} \C(A,TB)
			}}{	
			D \xrightarrow{(\lambda,f\o !)} \C(A,A) \x \C(A,B) \xrightarrow{m_{AAB}} \C(A,B)
		}
	\]
	similarly, we see the other path is exactly $l \o f$. So a $\w$-natural transformation is exactly a morphism of lifts.
\end{proof}
We may also see that there is a natural idempotent on the category of lifts in a tangent category $\C$, given by $p \o \lambda: (E,\lambda) \to (E,\lambda)$.
This idempotent has a natural representation in the monoid $\Lambda$ as the map $1 \xrightarrow{\iota_L \o 0} D + 1$.
In fact we note that this idempotent is an absorbing element of the monoid $D+1$, so for any $f:X \to D+1$, we have that $m(f,0\o !) = m(0\o !, f) = 0\o !$. 
A pre-differential bundle is exactly a lift with a chosen splitting of the natural idempotent $p\o\lambda$.
Way may take the idempotent splitting of $\Lambda$ and call it $\Lambda^+$, and observe that a predifferential bundle is exactly a functor $\Lambda^+ \to \C$.
\begin{definition}
	We define the $\w$-category $\Lambda^+$ to be given by the set of objects $\{0,1\}$ with hom-spaces:
	\begin{itemize}
		\item $\Lambda^+(1,1) = D+1$, otherwise $\Lambda^+(i,j) = 1$.
		\item $m_{111} = m, m_{i11} = \iota_L \o 0 \o !, m_{i0j} = !$
	\end{itemize}
\end{definition}
The following lemma lets us simplify the composition in $\Lambda^+$.
\begin{lemma}
	Associativity of composition in $\Lambda^+$ can be written by embedding each hom-space into $D+1$
	$\w$-category, where we use inclusions:
	\begin{gather*}
		\Lambda^+(0,0) = 1 \xrightarrow{\iota_L \o 0 \o !} D+1  \hspace{0.25cm}
		\Lambda^+(0,1) = 1 \xrightarrow{\iota_L \o 0 \o !} D+1  \\
		\Lambda^+(1,0) = 1 \xrightarrow{\iota_L \o 0 \o !} D+1  \hspace{0.25cm}
		\Lambda^+(1,1) = D+1 \xrightarrow{id} D+1  
	\end{gather*}
    \[\input{TikzDrawings/Ch5/assoc.tikz}\]
\end{lemma}
Basic facts about idempotent splittings yield the following theorem:
\begin{lemma}
	The category of pre-differential bundles is exactly the category of $\w$-functors $\Lambda^+ \to \C$. 
\end{lemma}
Now that we have exhibited pre-differential bundles as the category of $\C$-valued presheaves on $\Lambda^+$, it is straightforward to exhibit the category of differential bundles as a reflective subcategory of pre-differential bundles in $[\Lambda^+, \C]$.
\begin{proposition}
    The category of differential bundles in a locally finitely presentable tangent category $\C$ is a reflective subcategory of $[\Lambda^+, \C]$, and is therefore a locally finitely presentable tangent category.
\end{proposition}
\begin{proof}
    The category of pre-differential bundles in $\C$ is isomorphic to $[\Lambda^+, \C]$. 
    By \Cref{cor:idemp-dbun}, the category of differential bundles is isomorphic to the category of algebras for a canonical idempotent monad on $\C$, where a pre-differential bundle is sent to the $T$-equalizer: 
    \input{TikzDrawings/Ch5/reflector.tikz}
    This equalizer will always exist if $\C$ has equalizers, so this gives a left-exact idempotent monad on $[\Lambda^+, \C]$ whose algebras are differential bundles. 
\end{proof}
The notion of comodels for differential bundles is relatively underdeveloped.
Recall that because $(D+1, m', \iota_L)$ is a commutative monoid, there is a bijective correspondence between functors
\[
	\Lambda = \Lambda^{op} \Rightarrow \infer{\Lambda \to \C}{\Lambda^{op} \to \C}
\]
and this extends to pre-differential bundles due to basic facts about idempotent splittings.
In a cartesian category equipped with an infinitesimal object, we have an alternative presentation of pre-differential bundles.
When considering the dual of Rosicky's universality diagram, one finds a co-additive bundle structure.
\begin{lemma}
	Let $\C$ be a cartesian category with an infinitesimal object.
	\begin{enumerate}[(i)]
		\item A pre-differential bundle may be presented as a pair $(y:D \x A \to A, a:1 \to A)$ where $a \o ! = y(0\o !, id)$.
		\item If the following diagram is a pushout and pushout powers of $a$ exist, there is a commutative comonoid in $(1/\C, +_0, id_1)$.
            \[\input{TikzDrawings/Ch5/coadditivite-bundle.tikz}\]
	\end{enumerate}
\end{lemma}
\begin{proof}
	For the first part, note that this is exactly asking that $a$ splits the idempotent $y(0,id)$.
	For the second part, we induce co-addition using:
    \[\input{TikzDrawings/Ch5/induce-coadd.tikz}\]
\end{proof}

The notion of a \emph{comodel} of differential bundles is relatively underdeveloped, but will be useful for our purposes.
A codifferential bundle will determine a functor
\[
    Q: (\Lambda^{+})^{op} \to \C
\]
whose nerve will land in the category of differential bundles in $\wone$.
Because $D$ is a commutative cosemigroup, a codifferential bundle is still determined by a functor $\Lambda^+ = (\Lambda^+)^{op} \to \C$.
\begin{definition}
    Let $\C$ be a $\w$-category with copowers by representables.
    Then a codifferential bundle in $\C$ is exactly an object $Q$ equipped with an algebra:
    \[
        y:Q \cdot D \to Q
    \]
    that is associative with respect to the semigroup multiplication on $D$, and a chosen splitting of the idempotent 
    \[
        Q \xrightarrow{Q \cdot 0} Q \cdot D \xrightarrow{y} Q
    \]
    We require that the following diagram is a coequalizer:
    \[\input{TikzDrawings/Ch5/comodel-coeq.tikz}\]
\end{definition}
\begin{lemma}
    Every codifferential bundle in a $\w$-category with copowers by representables determines a nerve functor 
    \[
        N_Q: \C \to [\Lambda^+, \w]
    \]
    that sends any object $C$ to differential bundle in $\w$.
\end{lemma}
There is one construction in particular that will be useful later on:
\begin{lemma}
    Let $\C$ be a $\w$ category with copowers, coequalizers, and products.
    If $(y:Q \cdot D \to A, M)$ is a codifferential bundle, and $a:A \to A$ is an idempotent. Then the pushout:
    \[\input{TikzDrawings/Ch5/po-det-cobun.tikz}\]
    determines a codifferential bundle on $S$.
\end{lemma}
\begin{proof}
    Note that $D \x (-)$ is the copower by $D$ functor, and copowers preserve colimits, so we have the following morphism:
    \[\input{TikzDrawings/Ch5/induced-morphism.tikz}\]
    Associativity follows by couniversality. Idempotent splittings are coequalizers, so the triple exists.
    The commutativity of colimits ensures that the action $y^s$ is couniversal.
\end{proof}

Finally, we observe the following:
\begin{proposition}
    $\Lambda^+$ is a dense subcategory of differential bundles in $\w$. 
\end{proposition}


\section{Segal conditions for involution algebroids}%
\label{sec:revisiting-segal-conds}

In this section we revisit the ``Segal conditions'' for involution algebroids using the enriched categories perspective. 
There are three theories presented in this section - the theory of differential bundles, the theory of anchored bundles, and the theory of involution algebroids.
We can observe that:
\begin{enumerate}[(i)]
    \item The category of differential bundles in any tangent category with equalizers is a reflexive subcategory of a functor category.
    \item The category of anchored bundles is a full subcategory of the functors from the category of Weil algebras of width 1 (e.g. $\{ \N, T_n \}$). The Segal conditions for an anchored bundles is that its underlying pre-differential bundle (the precomposition functor $\Lambda^+ \to \wone^1$) is a differential bundle.
    \item The category of involution algebroids a full subcategory of  $\w$-functors $[\wone^3, \C]$, and it has Segal conditions for the width-three prolongations of anchored bundles. This will exhibit the category of involution/Lie algbebroids as monadic over the category of anchored bundles, which was first noticed in \cite{Kapranov2007}.
\end{enumerate}
% 
% \paragraph{Differential bundles as an enriched structure}
% We take an old fact about $M$-sets, and apply it to the enriched setting (and forget the unit axiom)

% \paragraph{Anchored bundles}
Recall that an anchored bundle is a differential bundle $(q:E \to M, \xi, \lambda)$ equipped with a linear morphism to $TM$.
That is to say, there is a natural transformation:
\[
    \input{TikzDrawings/Ch5/nat-trans-anc.tikz}
    \input{TikzDrawings/Ch5/nat-trans-anc-2.tikz}
\]
Note that the $TM$ is a limit, and has the universal property that for any $C$:
\[
    \C(C, TM) \cong \w(D, \C(C,M))    
\]
To construct a syntactic category for $\w$, then, we must augment $\Lambda^+$ so that the object $1$ is a cylinder for that limit diagram.
This agrees with previous results showing that the $\C$ is a reflective subcategory of the category of anchored bundles in $\C$, where $\rho$ acts as the reflector.
Recall that a cylinder for a diagram $G: \d \to \C$ weighted by $F:\d \to \w$ is given by an object $C$ and a $\w$-natural transformation:
\[
	F \Rightarrow \C(B, G-)
\]
so a cylinder for the tangent bundle of an object $C$ in $\C$ is given by the pair $K_D:1 \to \w, K_C: 1 \to \Lambda^+$, and the natural transformation is a map:
\[
	\infer{D \to \C(B,C) }{K_D \Rightarrow \C(B, C)}
\]
In the category $\Lambda^+$, we have $\Lambda^+(0,1) = 1$ so there is a unique cylinder
\[
	\w(D, \Lambda^+(0,1)) \cong \w(D, 1) = \{ !:D \to 1\}
\]
This corresponds with the linear morphism:
\[\input{TikzDrawings/Ch5/cor-to-lin-mor.tikz}\]
The category of \emph{truncated} Weil algebras solves this problem. 
\begin{definition}
    Define $\wone^1$ to be the full subcategory of $\wone$ spanned by $\{ \N, T \}$ (regarding $\wone$ as a $\w$-category).
\end{definition}
The object $T$ is a differential bundle in $\wone$, so we have there is a functor from $\Lambda^+ \to \wone^1$.
\begin{proposition}
    Every anchored differential bundle in $\C$ determines a functor $\wone^1 \to \C$, a natural transformation is exactly a tangent natural transformation.
\end{proposition}
\begin{proof}
    Start with an anchored bundle $(q:E \to M, \xi, \lambda, \anc)$. Then we have for hom-objects with domain $\N$:
    \[
        \wone^1(\N,\N) = 1 \mapsto id_M \hspace{0.5cm} \wone^1(\N,T) = 1 \mapsto \xi
    \]
    The hom-objects with domain $T$, the problem is slightly more difficult as $\wone^1(T,\N)(T^V) = \wone(T, T^V)$ and $\wone^1(T, T)(T) = \wone(T, T.T^V)$.
    This amounts to constructing maps:
    \[
         \wone(T, T^V) \to \C(E, T^V.M) \hspace{0.5cm}  \wone(T,T^V.T) \to \C(E, T^V.E)
    \]
    The first mapping is straightforward, send $f$ to $f.M \o \anc$. The second map relies on the fact that every map:
    \[
        T \to T^V
    \]
    in $\wone$ may be constructed using only $\{p,+,0,\ell\}$ closed under composition, pairing, and maps induced the universality $\ell$ (since there is only a single source variable).
    Thus for each $f:T \to T^V$ we can construct $f':E \to TE$ using $\{ q, +_q, \xi, \lambda\}$, pairing, and the universality of $\lambda$.

    The inclusion $\Lambda^+ \to \wone$ ensures that any tangent natural transformation will be a linear morphism on the underlying differential bundle, and the tangent natural transformations coherences ensures that a tangent natural transformation will preserve the anchor.
    Conversely, an anchored bundle morphism will preserve each of the constructed morphisms $E \to T^V.E, E \to T^V.M$ (as it preserves each of $\{ q, +_q, \xi, \lambda\}$), giving a tangent natural transformation. Thus there is a faithful embeddings $\mathsf{Anc}(\C) \hookrightarrow [\wone^1, \C]$.
\end{proof}

We use our first Segal condition. Observe that there is a bijective-on-objects functor
\[
    a: \Lambda^+ \to \wone^1
\]
The category $\Lambda^+$ is a dense subcategory of differential bundles, and each object $\N, 1$ have the structure of a differential bundle (e.g. they have a lift satisfying Rosicky's universality condition).
Thus we have the following:
\begin{lemma}
    The category $\wone^1$ is a $\Lambda^+$ theory on the category of differential bundles in $\w$.
\end{lemma}
The category of anchored bundles in $\w$ is precisely the category of presheaves on $(\wone^{1})^{op}$ whose underlying pre-differential bundle is a differential bundle. Thus we characterize the category of anchored bundles using the following pullback in $\w-CAT$:
\[\input{TikzDrawings/Ch5/anc-as-pb-in-cat.tikz}\]
That is, the Segal conditions for an anchored bundle is exactly a co-presheaf on $\wone^1$ whose underlying pre-differential bundle is a differential bundle.
\begin{proposition}
    The category of anchored bundles (in $\w$) is monadic over the category of differential bundles in $\w$.
\end{proposition}
For any tangent category $\C$, then, the following holds (using the embedding theorem).
\begin{corollary}
    For any tangent category $\C$, we have the following characterization of the anchored bundles in $\C$.
    \[\input{TikzDrawings/Ch5/anc-as-pb-in-cat2.tikz}\]
\end{corollary}


% \paragraph{Segal conditions for involution algebroids}
In order to give Segal conditions for involution algebroids, we must develop the appropriate set of arities.
This involves developing the \emph{coprolongations} of the initial co-anchored bundle given by $\yon: (\wone^1)^{op} \hookrightarrow \mathsf{Anc}(\w)$.

The idea of a coprolongation is to take the limit defining the $UV$-prolongation:
\[\input{TikzDrawings/Ch4/Sec3/span-comp.tikz}\]
and turn it into a representable functor $\mathsf{Anc}(\w) \to \w$.

\begin{definition}
    Let $(y:D \x A \to A, a:1 \to A, \delta: D \to A)$ be a co-anchored codifferential bundle. The coprolongation of $A$ by a Weil algebra $V$ is the cospan defined inductively by:
    \begin{enumerate}[(i)]
        \item $A(R): 1 \xrightarrow{=} 1 \xleftarrow{=} 1$
        \item $A(W^n):1 \xrightarrow{a(n)} A(n) \xleftarrow{\delta(n)} D(n)$
        \item $A(V \ox W^n)$ is given by cospan composition:
        \input{TikzDrawings/Ch5/cospan-comp.tikz}
    \end{enumerate}
\end{definition}

\begin{proposition}
    In the category of anchored bundles in $\w$, the representable functor:
    \[
        [\wone^1, \w](A_V, -)        
    \]
    sends an anchored bundle $B$ to its prolongation $\prol(V, B)$.
\end{proposition}
\begin{proof}
    This uses the Yoneda embedding, and the continuity of \[[\wone^1, \w](-, X): \mathsf{Anc}(\w) \to \w \] 
    This functor sends the diagram in question to the exact span composition defining the prolongation of the anchored bundle $X$.
\end{proof}

In the category of microlinear anchored bundles, there is the initial representable comodel given by the Yoneda embedding.
\[
	\yon(0) = 1, \yon(1) = L
\] 
For any anchored bundle $(\pi:A \to M, \xi, \lambda, \anc)$ in $\w$, the Yoneda lemma ensures that:
\[
	\mathsf{Anc}(\w)(1, A) = M, \mathsf{Anc}(\w)(L, A) = A
\]
so by the above proposition, we have $\mathsf{Anc}(\w)(L(V), A) = \prol(A, V)$.
This means that the category of ``prolongation complexes'' are naturally defined as presheaves on the coprolongations of $(y:D \x L \to L, 0, \delta)$ in $\mathsf{Anc}(\w)$.

\begin{definition}
	The $\w$-category ${\prol}$ is defined to be the full subcategory of $\mathsf{Anc}(\w)$ spanned by the coprolongations of the comodel $\yon:\wone^1 \to \mathsf{Anc}(\w)$.
\end{definition}


\begin{proposition}
	For any tangent category $\C$, there is a fully faithful inclusion:
	\[
		\mathsf{Anc}^*(\C) \to [{\prol}, \C]
	\]
	where $\mathsf{Anc}^*$ is the full subcategory of anchored bundles where all prolongations by $\wone$ exists.
\end{proposition}
\begin{proof}
	By the lemmas, we have that $\mathsf{Anc}(\w) \to [{\prol}, \w]$ is fully faithful.
	It then follows that the following functor is fully faithful:
	\[
		[\C^{op}, \mathsf{Anc}(\w)] \to [\C^{op}, [{\prol}, \w]]
	\]
	By the symmetry of theories, we can see that:
	\[
		\mathsf{Anc}(\widehat{\C}) \hookrightarrow [{\prol}, \widehat{\C}]
	\]
	Note that $\mathsf{Anc}(\C)$ is the full subcategory of $\mathsf{Anc}(\widehat{\C})$ where the anchored pre-differential bundle factors through the representables $\wone^1 \to \C \hookrightarrow \widehat{\C}$. 
	Because the inclusion $\C \to \hat{C}$ is flat, if an anchored bundle in $\C$ is complete then so to is its prolongation complex ${\prol} \to \widehat{C}$ factors through $\C$, so that we have the desired inclusion: 
	\[
		\mathsf{Anc}^*(\C) \to [{\prol}, \C]
	\]
\end{proof}

We finish by proving the main result in this section, that involution algebroids are models of a $\prol$-theory. 
The first step is giving a bijective-on-objects functor from the prolongation complex category and Weil algebras.
\begin{lemma}
	There is a bijective-on-object $\w$-functor ${\prol}^{op} \to \wone$.
\end{lemma}
\begin{proof}
	This follows from the dual result that $\w$ is a reflective subcategory of $\mathsf{Anc}(\w)$; we have that $\w$ is coreflective in the category of anchored codifferential bundles; the coreflector sends $L \mapsto D$, and so the pushouts $L(V) \mapsto D(v)$.
%	The trick here is to observe that $\w \hookrightarrow \mathsf{Anc}$ is a reflective subcategory of anchored bundles, sending an anchored bundle to its ``discrete'' anchored bundle $\lambda = \anc = 0:A \to TA$, and the coreflector sends $L(V) \mapsto D^V$.%TODO check this 
\end{proof}
\begin{proposition}
	Involution algebroids in $\w$ are models of the ${\prol}$-theory ${\prol} \to \wone^{op}$.
\end{proposition}
\begin{proof}
	The category of microlinear involution algebroids is exactly the category of cartesian models of $\wone$ in $\w$. 
	These are precisely the category of $\wone$-copresheaves where 
	\[
		(A.p,A.0, \alpha \o A.\ell, \alpha)
	\]
	are anchored differential bundles, where $A.T^U = \prol(A,U)$, which is the subcategory of $\widehat{\wone^{op}}$ that lands in the image of $\iota: \mathsf{Anc}(\w) \hookrightarrow \widehat\prol$.
\end{proof}
We can use embedding theorems and the symmetry of theories to show that the this holds more generally for the category of complete involution algebroids in any tangent category.
\begin{corollary}
	The category of complete involution algebroids in a tangent category $\C$ is precisely the pullback in $\w$-cat:
	\begin{equation}\label{eq:prol2}
        \input{TikzDrawings/Ch5/prol2.tikz}
	\end{equation}
\end{corollary}
\begin{corollary}
    The category of involution algebroids in a tangent category $\C$ is exactly the pullback:
    \begin{equation}\label{eq:prol3}
        \input{TikzDrawings/Ch5/prol3.tikz}
    \end{equation}
    where we write the category of width-three prolongations of an anchored bundle $\prol^3$.
\end{corollary}
In particular, this means that we can write the category of Lie algebroids as a pullback in $\w$-cat:
\begin{corollary}
	The category of Lie algebroids is the pullback in $\w\mathsf{Cat}$:
    \[\input{TikzDrawings/Ch5/lie-algd-monadic.tikz}\]
\end{corollary}
And finally, observe that in any locally presentable tangent category the category of involution algebroids is monadic over the category of anchored bundles, as previously observed in \cite{Kapranov2007}.
\begin{corollary}
	In a locally presentable tangent category $\C$, the category of involution algebroids is monadic over the category of anchored bundles in $\C$.
\end{corollary}

There are still some important properties to check in the category of involution algebroids. 
In any tangent category $\C$, the product in $[\wone,\C]$ of involution algebroids is an involution algebroid, so to show that $\mathsf{Inv}(\w)$ is a cartesian closed category, it suffices to show that it is an exponentiable ideal of $[\wone,\w]$ using the cartesian closed structure on $\w$-presheaves.
\begin{proposition}
	The category of involution algebroids in $\w$ is cartesian closed. 
\end{proposition}
\begin{proof}
	We need only show that involution algebroids are an exponential ideal in $[\wone,\w]$.
	This is equivalent to checking that for any involution algebroid $A$ and copresheaf $Q$, that $[Q,A](v)$ is the prolongation of the underlying anchored bundle.
	However, this follows as as simple calculation yields:
	\[
		[Q,A](v) = [\wone, \w](Q, [\yon (v), A])
	\]
	and $[\wone,\w](Q,-)$ is a $\w$-continuous functor, so it sends the limit defining $\prol(A,V)$ to the limit defining $\prol([Q,A],V)$.
%	\begin{align*}
%		[Q,A](v) &= [\wone, \w](Q \x \yon (v), A) \\
%				 &= [\wone, \w](Q, [\yon (v), A]) \\
%				 &= [\wone, \w](Q, \{ \prol_v, A\}) \\
%				 &= \{ \prol_v, [\wone, \w](Q, A) \}
%	\end{align*}
\end{proof}


\section{The infinitesimal nerve of a groupoid and its realization}%
\label{sec:inf-nerve-of-a-gpd}

The idea of this category is to construct a cartesian $\w$-functor:
\[
    \partial: \wone \to \mathsf{Gpd}(\w)    
\]
so that the nerve $N_\partial$ will coincide with the Lie functor from groupoids to algebroids. 
This has two major consequences, the first it equips the category of groupoids in $\mathsf{Gpd}(\w)$, and as a consequence the category of internal groupoids in most tangent categories, with a new tangent structure. The second consequence is that it puts the Lie functor into the nerve/realization paradigm, allowing for the construction of a groupoid from an algebroid.

% \paragraph{Revisiting groupoids and cubical objects}
Following \cref{sec:gpds-and-cub}, internal groupoids will be treated as a full subcategory of cubical objects. The language of locally presentable categories and density allows us to make simplify some of our previous results.
Recall that by \cref{prop:gpd-2cub-ff}: 
\begin{lemma}
    The arrow groupoid is a cubical monoid, and 2-truncation:
    \[
        \square_2 \hookrightarrow \square \xrightarrow[]{I} \mathsf{Gpd}
    \]
    is a dense functor.
    % \begin{enumerate}[(i)]
    %     \item $\square_2 \to \mathsf{Gpd}$
    %     \item $\square^{\mathsf{Conn}}_2 \to \mathsf{Gpd}$
    %     \item $\square^{\mathsf{SymConn}}_2 \to \mathsf{Gpd}$
    % \end{enumerate}
\end{lemma}
Now we can use the cancellativity result from \Cref{cor:dense-ff-result} to get the following:
\begin{proposition}%
    \label{prop:I-is-dense}
    The following functors arrow groupoid determines determiens a dense functor $\square \to \mathsf{Gpd}$.
\end{proposition}
\begin{proof}
    In every case, observe that the composite $\square_2 \hookrightarrow \square \to \mathsf{Gpd}$ is a dense functor, while the first map is fully faithful, thus the second map is a dense functor.
\end{proof}
This result was proved more concretely in \cref{cor:gpd-cub-ff}, but is a useful demonstration of density. 
\begin{corollary}
    The category of groupoids in $\w$ is a cartesian monoidal reflective subcategory of cubical objects in $\w$. 
\end{corollary}

Following the work in \Cref{sec:revisiting-segal-conds}, we wish to construct a cubical object that represents the linear approximation. Recall that given a cubical object $C$, the objects $C^\partial, C^{\partial \x \partial}, \dots$ are defined using $n$-fold equalizers:
\[\input{TikzDrawings/Ch5/nfold-eqs.tikz}\]
This is a cocomplete $\w$-category so the a cubical object $C \mapsto T.C.2$ is represented by $D \x I$. Thus, by taking the $n$-fold coequalizers:
\[\input{TikzDrawings/Ch5/nfold-coeq.tikz}\]
which is equivalent to the $n$-fold product of the equalizer diagram:
\[\input{TikzDrawings/Ch5/prod-coeqs.tikz}\]
the functor sending $C$ to $C^{n\cdot \partial}$ is represented by $n\cdot \partial$.
\begin{proposition}
    The object $\partial \cdot n$ is $n$-fold coequalizer:
    \[\input{TikzDrawings/Ch5/prod-coeqs.tikz}\]
    in cubical Weil spaces, and groupoids in $\w$.
\end{proposition}
\begin{proof}
    This follows from the fact that the category of cubical Weil spaces and groupoids in $\w$ are both cartesian closed, so that $(-) \x (-)$ is cocontinuous in each variable.
\end{proof}
The object $\partial$ is very nearly an infinitesimal object in the category of cubical Weil spaces.
\begin{definition}\label{def:structure-maps-del}
    Define the following maps on $\partial$.
    \begin{enumerate}[(i)]
        \item There is a point $0 \to \partial$ induced by:
        \[\input{TikzDrawings/Ch5/partial0.tikz}\]
        \item Write pushout powers of $0$ as $\partial(n)$. 
        \[\input{TikzDrawings/Ch5/delta-coadd.tikz}\]
        There is a map $\delta: \partial \to \partial(2)$ induced by:
        \[\input{TikzDrawings/Ch5/delta-coadd2.tikz}\]
        \item There is a cosemigroup structure $\odot: \partial \x \partial \to \partial$ induced by:
        \[\input{TikzDrawings/Ch5/cosemi.tikz}\]
        \item There is a co-anchor map $d:D \to \partial$ induced by the target:
        \[\input{TikzDrawings/Ch5/coanchor.tikz}\]
    \end{enumerate}
\end{definition}
\begin{lemma}%
\label{lem:structure-coherences}
    We have the following properties on $\partial$, where we write the pushout powers of 
    \begin{enumerate}[(i)]
        \item $\partial$ has a co-additive bundle structure $(0:1 \to \partial, !:\partial \to 1, \delta: \partial \to \partial(2))$, and $d$ preserves the co-addition.
        \item $\partial$ is a commutative semigroup with degeneracy $0$, and $d$ preserves the cosemigroup morphism and degeneracy.
        \item The following diagram commutes:
        \[\input{TikzDrawings/Ch5/coherence-for-structure-thing.tikz}\]
    \end{enumerate}
\end{lemma}
\begin{proof}
~\begin{enumerate}[(i)]
    \item This follows from co-associativity on $D$. Co-additivity of $d$ follows by precomposition:
    \input{TikzDrawings/Ch5/Coadditivity.tikz}
    \item The commutative cosemigroup axioms with degeneracy axioms all follow from the axioms on $D$. That $d$ is a cosemigroup homomorphism comes from precomposition (note that $I.\delta^+$ is the unit for $I.\gamma^+$)
    \input{TikzDrawings/Ch5/cosemigp.tikz}
    \item Consider the following diagram, where the maps into the center square are the relevent quotient maps.
    \input{TikzDrawings/Ch5/diag-commutes.tikz}
    The outer square commutes, and each of the maps commutes with the coequalizers diagrams. Thus, we may conclude that the center diagram commutes.
\end{enumerate} 
\end{proof}
\begin{proposition}
    The infinitesimal object $\delta$ represents the linear approximation functor $\widehat{\square^L} \to \w$.
\end{proposition}

So far we have represented the linear approximation functor, constructing an object that behaves like a Lie groupoid (but does not satisfy the necessary universality conditions). 
Recall that a cubical object with symmetric connections had an associated involution algebroid whenever the two morphisms:
\[
    z: C^{\partial \x \partial} \to \prol(C^\partial), \hspace{0.5cm}
    z_2: C^{\partial\x\partial\x\partial} \to \prol^2(C^\partial)
\]
are isomorphisms. 
We can formalize this as a natural transformation to the category of prolongations:
\[\input{TikzDrawings/Ch5/nat-in-prols.tikz}\]
Where:
\begin{enumerate}[(i)]
    \item The nerve $L$ refers to the functor $L: \prol \to \widehat{\square^L}$ induced by regarding $\partial$ as a co-anchored bundles and taking its coprolongations.
    \item The functor $Q^*$ identifies a functor from the full subcategory spanned by $\partial$ to the category $\prol$.
\end{enumerate}
\begin{proposition}
    The infinitesimal approximation object $\partial$ is a co-anchored bundle.
\end{proposition}
% \begin{proof}
%     The action is given by $y:D \x \partial \xrightarrow{} \partial \x \partial \to \partial$, and so it has the original unit $0:1 \to \partial$. %TODO this needs more
% \end{proof}

\begin{notation}
    Write the coprolongation of $D \to \partial$ as $L$, and more generally write the coprolongation of $\partial$ by the Weil algebra $V$ as $L_V$.
\end{notation}


\begin{proposition}
    There is a functor $Q:\prol \to \widehat{\square^L}$ that sends $v \mapsto \prol_v$, and a unique natural transformation:
    \[\input{TikzDrawings/Ch5/QFunct.tikz}\]
\end{proposition}
\begin{proof}
    The coprolongation defining $L_v$ has $\partial_v$ as a cocylinder, by observing that the following diagram commutes:
    \[\input{TikzDrawings/Ch5/coprol-inductive.tikz}\]
    and the result follows by induction.
\end{proof}

\begin{definition}
    A microlinear cartesian cubical object is a \emph{local groupoid} precisely whenever the map $z^*: N_Q(C) \to N_L(C)$ is an isomorphism.
\end{definition}
\begin{proposition}
    In the category of local groupoids, $\partial$ is an infinitesimal object. 
    (In this case, we mean the appropriate limit computing $\partial$ in the infiniteismal objects).
\end{proposition}
\begin{proof}
    The only axiom that fails for $\partial$ in the category of cartesian cubical objects in $\w$ was the universality of the vertical lift. 
    However, we have that the diagram:
    \[\input{TikzDrawings/Ch5/couniversal-colift.tikz}\]
    is a coequalizer for every cubical object, and this diagram corresponds to the coequalizer:
    \input{TikzDrawings/Ch1/InfObj/coeq.tikz}
\end{proof}
\begin{corollary}
    The category of local groupoids has a tangent structure that is represented by $\partial$.
\end{corollary}
Returning to \Cref{sec:linear-approx-of-a-cubical-object}, the result \Cref{prop:gpd-is-local-gpd} guarantees that groupoids are local groupoids.
Thus the following holds:
% \begin{proof}
%     We use that groupoids are an exponential ideal, and we have shown in the original treatment of groupoids that $[L_{WW},G] \cong [\partial \x \partial, G]$ - applying this fact inductively yields the result. More concretely, just consider 
% \end{proof}


\begin{theorem}[The infinitesimal nerve]
    In the category of (local) groupoids in $\w$ has a $\w$-functor $\partial: \wone \to \mathsf{LocGpd}$, and the nerve of $\partial$ factors through the category of involution algebroids.
    \[\input{TikzDrawings/Ch5/inf-nerve.tikz}\]
\end{theorem}
\begin{corollary}
    Note that a (local) groupoid in the category of $\w$-presheaves on $\C$ may be regarded as a $\w$-functor from $\C$ into the category of (local) groupoids. Similarly, an involution algebroid in the category of $\w$-presheaves on $\C$ may be regarded as a $\w$-functor $\C \to \mathsf{Inv}(\w)$.
    The infinitesimal nerve agrees with the Lie functor - that is to say, the following diagram commutes:
    \[\input{TikzDrawings/Ch5/inf-nerve-commutes.tikz}\]
\end{corollary}

\begin{observation}
    Note that the object $\partial_V$ in $\mathsf{LocGpd, Gpd}$ is constructed as a finite colimit of representable objects.
    This means, in particular, that the $\partial_V$ is a finitely presentable object, so that is is a co-model of $\mathsf{Inv}$ in the category of finitely presentable objects in $\mathsf{LocGpd, Gpd}$.
    This induces left-exact $\w$-morphisms from $\mathsf{Inv}_{fp}^{op} \to \mathsf{Gpd}_{fp}^{op}$, exhibiting the Lie functor as pre-composition (so it is induced via the Gabriel-Ulmer syntax/semantics adjunction).
\end{observation}
% This ends up making use of a fairly obscure technical condition on $\s$, that we can then extend to cartesian cubical objects and then microlinear groupoids.
% \begin{proposition}
%     % In the category of microlinear cartesian cubical objects, and microlinear groupoids, pushouts of the following form are 
%     In any cartesian monoidal reflective subcategory of a pre-sheaf tangent category, pushouts of the form:

%     are closed under products.
% \end{proposition}
% \begin{proof}
    
% \end{proof}
% \begin{corollary}
%     The $\partial^n$ is the colimit of the $n$-fold coequalizer:

% \end{corollary}





% \begin{proposition}
%     Write the arrow groupoid $I$. The full category spanned by $I$ in $\mathsf{Gpd}(\s)$ is dense.
% \end{proposition}
% \begin{proof}
%     Recall that for the construction of the cubical object of a groupoid, we had $G_n = \mathsf{Gpd}(\s)(I^n, G)$. 
%     This means that the functor $I^{(-)}: \square_{\mathsf{SymCon}} \to \mathsf{Gpd}(\s)$ is dense, and by -, this factors as two dense functors.%Add this lemma
%     This means that the full subcategory of $\mathsf{Gpd}(\s)$ spanned by $I^n$ is dense.
% \end{proof}
% \begin{corollary}
%     Consider the free Lawvere theory generated by the structure maps of a cubical monoid
%     %TODO include diagram here
%     satisfying the necessary maps (note that in this case, $\epsilon$ is now $!$). 
%     We call this the cartesian cube category $\square_{\mathsf{Cart}}$ and call the corresponding category of presheaves the category of cartesian cubical set,.
%     Observe the arrow groupoid $I$ is a model of this Lawvere theory.
%     Now use the the cancellation for dense functors from ,%TODO add reference
%     We can see that the category of internal groupoids is a full subcategory of the category of cartesian cubical sets.  
% \end{corollary}
% \begin{corollary}
%     The category of microlinear groupoids is cartsian closed, as it is a reflective subcategory of cartesian cubical sets. 
% \end{corollary}

% \begin{corollary}
%     The category of microlinear groupoids (e.g. groupoids in $\w$) is a reflective sub-tangent-category of $\widehat\square^{\mathsf{Cart}}$, where we treat $\square^{\mathsf{Cart}}$ as a trivial $\w$-category.
% \end{corollary}

% We begin by constructing an object $\partial$ that represents the $C^\partial$ construction from the previous chapter.
% \begin{definition}
%     Define the cartesian cubical object: $\partial(n)$ to be $\dots$ 
% \end{definition}
% \begin{proposition}
%     The object $\prod \partial(n_k)$ represents the functor blah
% \end{proposition}

% Now, we also note that $\partial$ acts as a co-differential bundle under $1$.
% \begin{itemize}
%     \item There is an action:
%     \item The action splits as:
%     \item There is a co-anchor:
% \end{itemize}
% Thus, we can construct the coprolongations of $\partial$, and write these $L_V$.
% \begin{definition}
%     The in the category of microlinear cartesian cubical objects, we have the object $L_V$ given by
% \end{definition}

% %TODO something about the d maps.

% \begin{proposition}
%     $\partial$ is an infinitesimal object.
% \end{proposition}

% \begin{proposition}
%     Each $\partial_v$ represents the functor $\prol(V,-)$.
% \end{proposition}


\section{The realization of the infinitesimal nerve}%
\label{sec:realization-of-inf-nerve}

As the final new result of this thesis, we construct an adjunction between (local) groupoids and involution algebroids in $\w$ using the nerve/realization paradigm. This adjunction is well-behaved: it is stable over the base-space of the cubical 

\begin{definition}
	Let $K: \D \to \C$ be a $\vv$-functor from a small $D$ into a cocomplete $\C$.
	The \textit{realization} of $K$ is the left adjoint to the $K$-nerve $N_K$, constructed as the following left Kan extension:
    \[\input{TikzDrawings/Ch5/real-kan-ext.tikz}\]
\end{definition}
\begin{lemma}
	The $K$-realization may be constructed as the following coend:
	\[
		|A|_K = \int^{D \in \D} A(V) \bullet K(V)
	\]
	where $A(V) \bullet(-)$ denotes the copower by the object $A(V)$ in $\w$.
\end{lemma}
We have constructed a $\w$-functor $\partial: \wone^{op} \to \mathsf{Gpd}(\w)$ that represents the $U$-approximation functor from groupoids into $\w$, so that the the nerve lands in involution algebroids. If we consider the Yoneda embedding of $\wone^{op}$ into the category of involution algebroids in $\w$, each $V$ determines a model $\delta^V$, and this $\delta^V$ represents the ``evaluate at $V$'' functor (so that it picks out the representables for the \textit{Lie} tangent structure). 
\[\input{TikzDrawings/Ch5/lie-tang.tikz}\]

The realization itself is computed as a coend - this has the funny effect that we are using the coend calculus integral to denote what (one hopes to be) the Lie integration functor :
\[
	|A|_\partial = \int ^{v \in \wone} A(v)\bullet \partial v 
\]
We see that for any groupoid $G$ and involution algebroid $A$:
\begin{align*}
	\mathsf{Gpd}(\w)(|A|_\partial, G)
	&= \mathsf{Gpd}(\w)(\int ^{v \in \wone} A(v)\bullet \partial v, G) \\
	&= \int_{v \in \wone} \mathsf{Gpd}(\w)( A(v)\bullet \partial v, G) \\
	&= \int_{v \in \wone} \w(A(v), \mathsf{Gpd}(\w)( \partial v, G)) \\
	&= \int_{v \in \wone} \w(A(v), N_{\partial}(G)(v)) \\
	&= [\wone,\w](A, N_{\partial}(G))
\end{align*}
All that remains to show is that $|-|_\partial$ is fully faithful - this requires a more careful analysis of the realization functor.
\begin{lemma}
	The realization functor preserves products.
\end{lemma}
\begin{proof}
	This is a consequence of the fact that $|-|_\partial$ is the left Kan extension of a cartesian functor along a cartesian functor.
	Note that this means $\int^v\partial(v) = 1$.
\end{proof}

\begin{lemma}\label{lem:base-of-groupoid}
	The base space of the groupoid $\partial$ is $D^v$.
\end{lemma}
\begin{proof}
	When constructing the colimit of groupoids, the colimits base space is constructed as the ordinary colimit for the diagram of the base spaces.
	The reflector from simplicial objects to groupoids preserves products, so it suffices to check that the base space of $\partial(n)$ is given by $D(n)$.
	
    The base space of $I$ is $1+1$, as:
    \[
        I(0) = \widehat{\square}(1,I) \cong \square(0,I) = 1 + 1
    \]
    and the map $e^-_0$ is given by $\delta^- \o !$. 
    $D$ is a discrete cubical object, so the base space is $D \x 1$.
    Thus the coequalizer defining $\partial$ is:
    \[\input{TikzDrawings/Ch5/coeq-def-partial.tikz}\]
    A map $\gamma: D + D \to M$ is a pair of maps $\gamma_0, \gamma_1: D \to M$. 
    We can see that $\gamma_0, \gamma_1$ agree at $0$ (they are both $\gamma_0(0)$), and $\gamma(0)$ is a constant tangent vector. It follows then, that $\gamma_1 \o (id | id) = \gamma$.
    % This result holds as a consequence of microlinearity. 
    % Note that $\square(0,1) = \{ \delta^\pm \}$, so $e^-$ sends $
	% The groupoid $D(n)$ is discrete, and the base space of $\yon(n)$ is $n\bullet 1 = \coprod_n 1$.
	% The diagram becomes:
	% \begin{equation*}
	% 	\begin{tikzcd}
	% 		(1 \x D(n)) + 2 \rar{(\iota_L \x 1 | 2\bullet 0)} \dar{((1\x !) | (id | id)) } &[3em] 2 \x D(n) \dar \\
	% 		1 \rar & (\partial(n))_0 
	% 	\end{tikzcd}
	% 	\cong
	% 	\begin{tikzcd}
	% 		D(n) + 2 \rar{i_L + (2\bullet0)} \dar[swap]{!} &[3em] D(n) + D(n) \dar \\
	% 		1 \rar & (\partial v(n))_0 
	% 	\end{tikzcd}
	% \end{equation*}
	% Now, we claim the following diagram is a pushout:
	% \begin{equation*}
	% 	\begin{tikzcd}
	% 		D(n) + 2 \rar{i_L + (2\bullet0)} \dar[swap]{!} &[3em] D(n) + D(n) \dar{(0! | id)} \ar[bend left]{rdd}{\gamma} \\
	% 		1 \rar{0} \ar[bend right]{rrd}[swap]{x} & D(n) \ar[dashed]{rd}{\gamma \o \iota_R} \\
	% 		& & X
	% 	\end{tikzcd}
	% \end{equation*}
	% First, notice that $\gamma$ is a pair of tangent vectors on $X$, and look at each injection into $D(n) + 2$.
	% \[
	% 	\gamma \o (i_L + (2 \bullet 0)) \o \iota_L = \gamma \o \iota_L, 
	% 	\hspace{0.3cm}
	% 	\gamma \o (i_L + (2 \bullet 0)) \o \iota_R = \gamma \o (2 \bullet 0)
	% \]
	% We can see that each tangent vector $\gamma \o \iota_L, \gamma \o \iota_R$ has the same base point, $x$.
	% \[ \gamma \o \iota_L \o 0 = \gamma \o (2 \bullet 0) \o \iota_L = x\o ! = \gamma \o (2 \bullet 0) \o \iota_R = \gamma \o \iota_R \o 0\]
	% Next observe that $\gamma \o \iota_L$ must be a constant tangent vector, as:
	% \[
	% 	\gamma \o \iota_L = \gamma \o (\iota_L + (2 \bullet 0)) \o \iota_L = x \o ! \o \iota_L = x \o !
	% \]
	% in particular, we have that $\gamma \o \iota_L = \gamma \o \iota_R \o 0 \o !$, 
	% so then a pair $(x,\gamma)$ uniquely determines a $D(n)$-tangent vector $\gamma \o \iota_R$.
%	%TODO uniqueness?
%	
%	{\color{red} I think this proof shows that this is a pushout in the entire presheaf category, since microlinearity on $X$ is never used}.
\end{proof}

%\begin{remark}
%	By the same argument as the lemma, we have
%	\begin{equation*}
%		\begin{tikzcd}
%			D^V + (n+1) \rar{i_0 + ((n+1)\bullet0)} \dar[swap]{!} &[3em] (n+1)\bullet D^V \dar{(0! | id)} \ar[bend left]{rdd}{\gamma} \\
%			1 \rar{0} \ar[bend right]{rrd}[swap]{x} & D^V(n) \ar[dashed]{rd}{\gamma \o \iota_R} \\
%			& & X
%		\end{tikzcd}
%	\end{equation*}
%	Therefore, we have that the simplicial object representing $\partial_V$ is given by:
%	\[
%		n \mapsto D^V(n)
%	\]
%	and because groupoids are a reflective subcategory of simplicial objects, we may compute
%	\[
%		[\partial^V, G](n) = [\Delta^{op},\w](\hat{\partial}^V \x \yon(n),G)
%	\]
%	However, it suffices for our purposes to have the base space characterized.
%\end{remark}

Now, recall the co-Yoneda lemma, for any $\vv$-presheaf $A: \C^{op} \to \vv$
\[
	F(C) = \int^{C' \in \C} \C(C,C') \ox F(C')
\]
so in particular, for an involution algebroid in $\w$ (or any $\w$-presheaf on $\wone^{op}$)
\[
	A(U) = \int^{V \in \wone} \wone^{op}(U,V) \x A(V) = \int^{V \in \wone} \wone(V,U) \x A(V) 
\]
\begin{lemma}\label{lem:as-a-presheaf}
	For any involution algebroid $A$:
	\[ A(R) = \int^{V \in \wone} A(V) \x D^V \]
\end{lemma}
\begin{proof}
	Use the tangent structure on $\wone$ to regard it as a $\w$-enriched category:
	\begin{gather*}
		D^V = \yon(V) = \wone(V,-) = (U \mapsto \wone(V, U))  \\= (U \mapsto \wone(V, U\ox R)) = \wone(V,R)
	\end{gather*}
	so that we may compute:
	\[
		A(R) = \int^{V} \wone(V,R) \x A(V) = \int^{V} D^V \x A(V) 
	\]
\end{proof}
This allows us to relate the base space of an involution algebroid and its associated groupoid. 

\begin{proposition}\label{prop:lie-int-first-part}
	Let $A$ be an involution algebroid in $\w$. Then $|A|([0]) = A(R)$.
\end{proposition}
\begin{proof}
	Use the Yoneda lemma, and that $\yon[0] = 1$ is a small projective so $\mathsf{Gpd}(1,-)$ is $\w$-cocontinuous, then apply \cref{lem:base-of-groupoid} and \cref{lem:as-a-presheaf}
	\begin{align*}
		|A|([0])&= \mathsf{Gpd}(1, |A|) \\
			    &= \mathsf{Gpd}\left(1, \int^{v \in \wone} A(v)\bullet \partial v \right) \\
			    &= \int^{v} A(v)\x \mathsf{Gpd}(1,  \partial v ) \\
			    &= \int^{v}  A(v)\x D^v = A(R)
	\end{align*}
\end{proof}

Note, then, there are two tangent structures on each of the category of (local) groupoids and the category of involution algebroids $\w$.
In order to prove that the realization functor into (local) groupoids is a coreflection, it would suffice to prove that the realization functor is a tangent functor for the second tangent structures, as:
\[
    \left(T_v(|A|) = [\partial_v, |A|] = |[\partial_v,A]| \right) \Rightarrow 
    \left(\mathsf{LocGpd}(\partial_v, |A|) = A(v)\right) \Rightarrow
    N_\partial(|A|) \cong A
\]
% This is equivalent to asking that the coend commute with the following pullback:
% \[
%     \int^v A(wv)\cdot \partial_v \cong \int^v (A(v) \ts{}{} T^v.A(w) ) \cdot \partial_v
%     \cong \int^v (A(v) \cdot \partial_v)\ts{\anc'}{T^v.\pi} (T^v.A(w)\cdot\partial_v)
% \]
%\begin{lemma}
%	There is a functor $\wone \to \w$ defined:
%	\[
%		v \mapsto \coprod^{\dim(v)} 1
%	\]
%	This is an involution algebroid, and its realization is the terminal groupoid.
%\end{lemma}
%\begin{proof}
%	
%\end{proof}
%
%\begin{proposition}
%	Any differential object, when regarded as an abelian involution algebroid, integrates to the pair groupoid.
%\end{proposition}
%\begin{proof}
%	\begin{align*}
%		|V|(2) &= \mathsf{Gpd}(\yon 2, |V|) \\
%			   &= \mathsf{Gpd}(\yon 2, \int^{u} T^uV \cdot \partial v ) \\
%			   &= \mathsf{Gpd}(\yon 2, \int^{u} V(R) \cdot [\dim(v)] \cdot \partial v ) \\
%			   &= \mathsf{Gpd}(\yon 2, V(R) \cdot \int^{u} [\dim(v)] \cdot \partial v ) \\
%			   &= V(R) \cdot \mathsf{Gpd}(\yon 2,  \int^{u} [\dim(v)] \cdot \partial v ) \\
%			   &= V(R) \cdot 2
%	\end{align*}
%\end{proof}

% \paragraph{How is the realization related to Lie integration?}
% {\color{blue}
% 	It is still unclear how Lie integration is related to this adjunction. 
% 	In particular, we may write the comonad on smooth groupoids as:
% 	\[
% 		S(\g) = \int^{v \in \wone} G.K(V) \cdot K(v)	
% 	\]
% 	where we observe that as each $\partial v$ is finitely presentable, we may factor our map as: 
% 	\[K: \wone \hookrightarrow \mathsf{Gpd}_{fp}^{op} \hookrightarrow \mathsf{Gpd}\]
% 	So, if we restrict our attention to the idempotent core of the adjunction, we have the counit $\epsilon$ is an isomorphism if and only if:
% 	\[
% 		\int^{v \in \wone} G.K(v) \cdot K(v) = \int^{v \in \mathsf{Gpd}^{op}_{fp}}G(v) \cdot \hat{v}
% 	\]
% }


% \end{document}