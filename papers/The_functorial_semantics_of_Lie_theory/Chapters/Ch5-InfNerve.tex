\chapter{The infinitesimal nerve and its realization}%
\label{ch:inf-nerve-and-realization}

The main thrust of Chapters \ref{ch:differential_bundles}, \ref{ch:involution-algebroids}, and \ref{chap:weil-nerve} has been that the tangent categories framework allows for Lie algebroids to be regarded as tangent functors
\[
    \wone \to \mathsf{SMan} 
\] which satisfy certain universality conditions. 
This chapter, which is more experimental than the previous four chapters and represents work still in progress, puts Lie algebroids into the framework of \emph{enriched functorial semantics}. This new perspective on algebroids uses Garner's enriched perspective on tangent categories (\cite{Garner2018}) and the enriched theories paradigm from \cite{Bourke2019}. The functorial-semantics presentation of the Lie functor will generalize the Cartan--Lie theorem (that the category of Lie algebras is a coreflective subcategory of Lie groups) into a statement within the general theory of functorial semantics. 

The goal is to show that the infinitesimal approximation of a groupoid, as discussed in Example \ref{ex:lie-algebroids}, may be constructed as a \emph{nerve}, just like Kan's original simplical approximation of a topological space. The nerve of a functor $K:\a \to \C$ approximates objects and morphisms in $\C$ by $\a$-presheaves, so it sends an object in $\C$ to the $\a$-presheaf
\[
    N_K: \a \to \C; C \mapsto \C(K-, C). 
\]
Thus there will be an infinitesimal object,
\[
    \partial: \wone^{op} \to \mathsf{Gpd}(\w)   
\] where $\mathsf{Gpd}(\w)$ denotes groupoids in the category $\W$ of Weil spaces (formally defined in Section \ref{sec:tang-cats-enrichment}). The nerve of $\partial$ has a left adjoint, the \emph{Lie realization}, given by the left Kan extension (just as Kan's geometric approximation of a simplicial set is, in \cite{Kan1958}):
\begin{equation}\label{eq:lan-lie}
    % https://q.uiver.app/?q=WzAsMyxbMCwwLCJcXHdvbmVee29wfSJdLFswLDEsIlxcd2lkZWhhdHtcXHdvbmVee29wfX0iXSxbMSwwLCJcXG1hdGhzZntHcGR9KFxcdykiXSxbMCwxLCJcXHlvbiIsMl0sWzAsMiwiXFxwYXJ0aWFsIl0sWzEsMiwiTGFuX1xceW9uXFxwYXJ0aWFsIiwyLHsic3R5bGUiOnsiYm9keSI6eyJuYW1lIjoiZGFzaGVkIn19fV1d
    \begin{tikzcd}
        {\wone^{op}} & {\mathsf{Gpd}(\w)} \\
        {[\wone,\w]}
        \arrow["\yon"', from=1-1, to=2-1]
        \arrow["\partial", from=1-1, to=1-2]
        \arrow["{Lan_\yon\partial}"', dashed, from=2-1, to=1-2]
    \end{tikzcd}
\end{equation}

\begin{restatable*}[The Lie Realization]{theorem}{lie}
    \label{thm:lie-realization}
    There is a tangent adjunction between the category of involution algebroids and groupoids in $\w$, where each functor preserves products and the base spaces.
\end{restatable*}
\[% https://q.uiver.app/?q=WzAsMixbMCwwLCJcXG1hdGhzZntHcGR9KFxcQykiXSxbMSwwLCJcXG1hdGhzZntJbnZ9KFxcQykiXSxbMCwxLCJOX1xccGFydGlhbCIsMCx7ImN1cnZlIjotMn1dLFsxLDAsInwtfF9cXHBhcnRpYWwiLDAseyJjdXJ2ZSI6LTJ9XSxbMiwzLCIiLDAseyJsZXZlbCI6MSwic3R5bGUiOnsibmFtZSI6ImFkanVuY3Rpb24ifX1dXQ==
\begin{tikzcd}
    {\mathsf{Gpd}(\w)} & {\mathsf{Inv}(\w)}
    \arrow[""{name=0, anchor=center, inner sep=0}, "{N_\partial}", curve={height=-12pt}, from=1-1, to=1-2]
    \arrow[""{name=1, anchor=center, inner sep=0}, "{|-|_\partial}", curve={height=-12pt}, from=1-2, to=1-1]
    \arrow["\dashv"{anchor=center, rotate=-90}, draw=none, from=0, to=1]
\end{tikzcd}\]

Note, however, that the left Kan extension in Equation \ref{eq:lan-lie} does not immediately give the desired adjunction of Theorem \ref{thm:lie-realization}, as there is no guarantee that the nerve functor $N_\partial$ lands in the category of algebroids. To prove this we must revisit the work in Section \ref{sec:weil-nerve} presenting involution algebroids as those functors $A:\wone \to \C$ for which each $A(V)$ is the prolongation of its underlying anchored bundle; this leads naturally to the formalism for enriched theories developed in \cite{Bourke2019}.

The presentation of algebroids as models of an enriched theory requires situating the categories of differential bundles, anchored bundles, and involution algebroids in $\w$ as full subcategories of $\w$-presheaf categories on small $\w$-categories. 
% Remove enum here!
% \begin{enumerate}
Section \ref{sec:tang-cats-enrichment} reviews the work in \cite{Garner2018} characterizing tangent categories as categories enriched in $\w = \mathsf{Mod}(\wone, \s)$ (the cofree tangent category on $\s$, by Observation \ref{obs:cofree-tangent-cat}).
    % \item 

Section \ref{sec:enriched-structures} reconfigures the content of Chapter \ref{ch:differential_bundles} using the enriched perspective, so that lifts are $\w$-functors from the $\w$-monoid $D + 1$, while differential bundles are a reflective subcategory of functors from its idempotent splitting. Similarly, the category of anchored bundles are a reflective subcategory of the category $[\wone^1,\C]$, where $\wone^1$ is the category of 1-truncated Weil algebras (Definition \ref{def:truncated-wone}).

Section \ref{sec:enriched-nerve-constructions} reviews the basic idea of an enriched nerve/approximation. A particularly important example is the linear approximation of a reflexive graph, the functor introduced in Example \ref{ex:prolongations}, which is the nerve of a functor
\[
    \partial: (\wone^1)^{op} \to \mathsf{Gph}(\w).
\]

Section \ref{sec:enriched-theories} applies the enriched theories framework introduced in \cite{Bourke2019}, where a \emph{dense} subcategory of a locally presentable $\a \hookrightarrow \C$ forms the ``arities'', and a bijective-on-objects functor $\a \to \th$ is the theory. The category of models is the pullback of (enriched) categories:
% https://q.uiver.app/?q=WzAsNCxbMCwxLCJcXEMiXSxbMSwxLCJbXFxhXntvcH0sIFxcdl0iXSxbMSwwLCJbXFx0aCwgXFx2XSJdLFswLDAsIlxcQ157XFx0aH0iXSxbMCwxLCIiLDEseyJzdHlsZSI6eyJ0YWlsIjp7Im5hbWUiOiJob29rIiwic2lkZSI6InRvcCJ9fX1dLFsyLDFdLFszLDBdLFszLDJdXQ==
\[\begin{tikzcd}
	{\C^{\th}} & {[\th, \vv]} \\
	\C & {[\a^{op}, \vv]}
	\arrow[hook, from=2-1, to=2-2]
	\arrow[from=1-2, to=2-2]
	\arrow[from=1-1, to=2-1]
	\arrow[from=1-1, to=1-2]
\end{tikzcd}\]
The first step is to freely complete the $\w$-category $\wone^1$ of truncated Weil algebras (Definition \ref{def:truncated-wone}) so that its base anchored bundle has all prolongations; call this $\w$-category $\prol$. Then, in every tangent category $\C$, the category of anchored bundles with chosen prolongations (Definition \ref{def:monoidal-category}) in $\C$ embeds fully and faithfully into the functor category:
\[
    \mathsf{Anc}^\prol(\C) \hookrightarrow [\prol, \C]  
\]
(here, $\mathsf{Anc}^\prol(\C)$ denotes the category of anchored bundles with chosen prolongations).
In particular, the tangent bundle on $\N$ in $\wone$ determines a bijective-on-objects functor
\[
    \prol \to \wone 
\]
so that the category of involution algebroids in any tangent category $\C$ is the pullback in $\w$Cat:
\[\begin{tikzcd}
    {\mathsf{Inv}^*(\C)} & {[\wone,\C]} \\
    {\mathsf{Anc}^*(\C)} & {[{\prol}^{op},\C]}
    \arrow[hook, from=2-1, to=2-2]
    \arrow[from=1-2, to=2-2]
    \arrow[from=1-1, to=2-1]
    \arrow[hook, from=1-1, to=1-2]
    \arrow["\lrcorner"{anchor=center, pos=0.125}, draw=none, from=1-1, to=2-2]
\end{tikzcd}\]
This means that the category of involution algebroids in $\w$ is monadic over the category of anchored bundles in $\w$ using the monad-theory correspondence from \cite{Bourke2019}.

The final section looks at the category of $\w$-groupoids. Essentially, the free groupoid over the linear approximation of a graph will now give an infinitesimal object $\wone \to \mathsf{Gpd}(\w)$. The nerve of  $\partial: \wone^{op} \to \mathsf{Gpd}(\w)$---the \emph{infinitesimal approximation}---has a right adjoint via the \emph{realization} of the nerve functor from Definition \ref{def:realization}. This is used to prove the culminating Theorem \ref{thm:lie-realization}.

The individual pieces of categorical machinery used in this chapter are not new (the enriched perspective on tangent categories, enriched nerve constructions, enriched theories). However, all of the results dealing with the application of enriched nerve constructions and enriched theories to tangent categories is original work of the author.

\section{Tangent categories via enrichment}%
\label{sec:tang-cats-enrichment}

This section gives a quick introduction to Garner's enriched perspective on tangent categories. The enriched approach to tangent categories first appeared in \cite{Garner2018} and builds on the category perspective on tangent categories introduced in \cite{Leung2017}. Garner was able to exhibit some of the major results from synthetic differential geometry as pieces of enriched category theory; for example, the Yoneda lemma implies the existence of a well-adapted model of synthetic differential geometry.

The category of Weil spaces is the site of enrichment for tangent categories and is closely related to Dubuc's \emph{Weil topos} from his original work on models of synthetic differential geometry \cite{Dubuc1981}; a deeper study of this topos may be found in \cite{Bertram2014}. Recall that the category $\wone$ is the free tangent category over a single object. The category of Weil spaces is the \emph{cofree} tangent category over $\s$, which is the category of transverse-limit-preserving functors $\wone \to \s$ by Observation \ref{obs:cofree-tangent-cat}. Call this the category of \emph{Weil spaces}, and write it $\w$. Just as a simplicial set $S: \Delta \to \s$ is a gadget recording homotopical data, a Weil space records \emph{infinitesimal} data.

\begin{definition}%
    \label{def:weil-space}
    A \emph{Weil space} is a functor $\wone \to \s$ that preserves transverse limits (Definition \ref{def:transverse-limit}): that is, the $\ox$-closure of the set of limits
    \[
        \left\{
        \input{TikzDrawings/Ch5/wone-pb.tikz}
        ,
        \input{TikzDrawings/Ch5/wone-univ-lift.tikz}
        ,
        \input{TikzDrawings/Ch5/wone-id.tikz}
        \right\}
    \]
    A morphism of Weil spaces is a natural transformation. Write the category of Weil spaces as $\w$.
\end{definition}
\begin{example}
    ~\begin{enumerate}[(i)]
        \item Every commutative monoid may be regarded as a Weil space canonically. Observe that for every $V \in \wone$ and commutative monoid $M$, one has the free $V$-module structure on $M$ given by $|V| \ox_{\mathsf{CMon}} M$ (here $|V|$ is the underlying commutative monoid of $V$). The commutative monoid $|V|$ is exactly $\N^{\mathsf{\dim V}}$, so that
        \[
            |V| \ox_{\mathsf{CMon}} M \cong \oplus^{\dim V} M.
        \]
        This agrees with the usual tangent structure on a category with biproducts.
        % (this is the \emph{copower} of $M$ by $V$, )
        % The infinitesimal linearity of $V \mapsto |V| \ox M$ follows from fundamental properties of finite biproducts.
        \item Following (i), any tangent category $\C$ that is \emph{concrete}---that is, admitting a faithful functor $U: \C \to \s$---will have a natural functor into Weil spaces (copresheaves on $\wone$). Every object $A$ will have an underlying Weil space $V \mapsto U_{\s}(T^V(A))$, and whenever $U$ preserves connected limits (such as the forgetful functor from commutative monoids to sets), each of the underlying copresheaves will be a Weil space.
        \item Consider a symmetric monoidal category with an infinitesimal object, which by Proposition \ref{prop:monoidal-functor-inf-obj} is a transverse-colimit-preserving symmetric monoidal functor $D:\wone \to \C$. Then for every object $X$, the nerve (Definition \ref{def:nerve-of-a-functor}) $N_D(X): \C(D-,X):\wone \to \s$ is a Weil space.
        \item For any pair of objects $A,B$ in a tangent category, $\C(A,T^{(-)}B):\wone \to \s$ is a Weil space by the continuity of $\C(B,-):\C \to \s$.
    \end{enumerate}
\end{example}
Unlike the category of simplicial sets, the category of Weil spaces is not a topos.
The category of Weil spaces does, however, inherit some nice properties from the topos of copresheaves on $\wone$ by applying results from \Cref{sec:enriched-nerve-constructions}, as it is \emph{locally presentable}. The basics of locally presentable categories can be found in the Appendix \ref{appendix}. Roughly speaking, a cocomplete category $\C$ has a subcategory of finitely presentable objects $\C_{fp}$, those $C$ so that
\[
    \C(C,-): \C \to \s  
\]
preserves \emph{filtered} colimits (i.e. those colimit diagrams that commute with all finite limits in $\s$). $\C$ is locally finitely presentable whenever every object is given by the coend
\[
    C \cong \int^{X \in \C_{fp}} \C(X, C) \cdot X
\]
where $S\cdot X$ is the (possibly infinite) product $X^{|S|}$ with $|S|$ the cardinality of the set $S$. This means that $\C = \mathsf{Lex}(\C_{fp}^{op}, \s)$, where $\mathsf{Lex}$ means the category of finite-limit-preserving functors. 

The third point of Corollary \ref{cor:properties-of-w} below, that $\w$ is locally finitely presentable as a cartesian monoidal category, means that the category $\w_{fp}$ is closed under products. This implies that $\w$ is locally presentable \emph{as a} $\w$-category (this ends up being an important technical condition whereby locally presentable $\w$-categories make sense). Whenever we discuss an arbitrary $\vv$-category, we will assume that $\vv$ is locally presentable as a monoidal category.

\begin{proposition}[\cite{Garner2018}]
    The category of Weil spaces is a cartesian-monoidal reflective subcategory of\, $[\wone,\s]$.
\end{proposition}
\begin{corollary}%
    \label{cor:properties-of-w}
    The category of Weil spaces is
    \begin{enumerate}[(i)]
        \item a cartesian closed category;
        \item a representable tangent category, where the infinitesimal object is given by the restricted Yoneda embedding $\yon: \wone^{op} \to \w$;
        \item Locally finitely presentable as a cartesian monoidal category.
    \end{enumerate}
\end{corollary}
% \begin{observation}%
% \label{obs:v-cat-stuff}
% \end{observation}
The cofree tangent structure on $\w$ is given by precomposition, that is:
\[
    T^U.M.(V) = M.(U \ox V) = M.T^U.V.
\]
This tangent coincides with the representable tangent structure induced by the Yoneda embedding. The proof is an application of the Yoneda lemma. Observe that
\[
    [D,M](V) \cong [\wone,\s](D \x \yon(V), M) \cong [\wone, \s](\yon(WV), M) \cong M(WV)
\]
where $D \x \yon(V) = \yon(WV)$ follows because the tensor product in $\wone$ is cocartesian and the reflector is cartesian monoidal.
\begin{notation}
    The tangent category $\w$ is a representable tangent category, where $D = \yon W$ (Definition \ref{def:inf-object}). 
    We will write the Yoneda functor $\yon: \wone^{op} \to \w$ as $D(-)$, so it is closer to the usual notation used in representable tangent categories or synthetic differential geometry (a single $D$ may be used as shorthand for $D(W)$, $D(n)$ for $D(W_n)$, etc.).
    Note that
    \[
        D(V) = D(\ox^k W_{n(i)}) = \prod^K D(n_i)  
    \]
    and that $D(\ell) = \otimes, D(c) = (\pi_1, \pi_0)$, $D(+) = \delta$, and so on.
\end{notation}

At this point, we are ready to move to the enriched perspective on tangent categories. The basics of enriched category theory may be found in the Appendix \ref{appendix}, but one definition in particular is important to include here.
\begin{definition}\label{def:power}
    Let\, $\C$ be a $\vv$-category for $\vv$ a closed symmetric monoidal category. For $J \in \vv$, the \emph{power} by $J$ of an object $C \in \C$ is an object $C^J$ so that the following is an isomorphism:
    \[
        \forall D: \vv(J, \C(D,C)) \cong \C(D,C^J)
    \]
    whereas the \emph{copower} is given by
    \[
        \forall D: \vv(J, \C(C,D)) \cong \C(J \bullet C, D).
    \]
    Let $\j \hookleftarrow \vv$ be a full monoidal subcategory of $\vv$. A $\vv$-category\, $\C$ has \emph{coherently chosen powers} by $\j$ if there is a choice of $\j$-powers\, $(-)^J$ so that
    \[
        (C^J)^K = C^{J \ox K}.
    \]
    Likewise, \emph{coherently chosen copowers} are a choice of $\j$-copowers so that
    \[
        K \bullet (J \bullet C) = (K \ox J) \bullet C.
    \]
    The sub-2-categories of $\vv$-categories equipped with coherently chosen powers and copowers are $\vv\cat^\j$ and $\vv\cat_\j$, respectively.
\end{definition}

\cite{Wood1978} proved that the 2-category of actegories over a monoidal $\C$ is equivalent to the 2-category of $[\C,\s]$-enriched categories with powers by representable functors (Definition \ref{def:power}), using the monoidal structure on $[\C, \s]$ induced by Day convolution\footnote{
    The \emph{Day convolution} tensor product of presheaves $X,Y: \C^{op} \to \s$ is given by $X \widehat\ox Y := \mathsf{Lan}_{\ox}(X \boxtimes Y)$, where $(X \x Y)(V) =X(V) \x Y(V)$ (\cite{Day1970}).
    }. 
Moreover, \cite{Garner2018} showed that a monoidal reflective subcategory $\vv \hookrightarrow \hat \a$ exhibits the 2-category of $\vv$-categories as a reflective sub-2-category of $\a$-categories; this proves that tangent categories are equivalent to a particular class of enriched category.
\begin{proposition}[\cite{Garner2018}]
    A tangent category is exactly a $\w$-category with powers by representables.
\end{proposition}
\begin{proof}
    For every $A, B \in \C_0$, the Weil space is defined as
    \[
        \underline{\C}(A,B) := U \mapsto \C(A, T^{U}B).
    \]
    The functor $\C(A,-)$ is continuous and $T^{-}B$ is an infinitesimally linear functor, so this is a Weil space. The following diagram gives the composition map :
    \input{TikzDrawings/Ch5/enrichment-in-W-comp.tikz}
    Note that it is natural in $U$ and $V$.

    By the Yoneda lemma (as the internal hom in $\w$ is the internal hom of copresheaves on $\wone$), \[\C(A, T^VB) =  (U \mapsto \C(A, B)(V \ox U)) = [D(V), \C(A,B)]\]
    so this category has coherently chosen powers by representable functors.
\end{proof}
% Enriched category theory introduces a new family of limits, \emph{powers}\footnote{Recall that the entire class of weighted limits is equivalent to conical (ordinary) limits and powers.}. Powers are essential in tangent category theory and satisfy a coherence due to \cite{LucyshynWright2016}.

% Now, returning to the specific enrichment in $\w$, remember that the hom-object between $A, B$ is given by:
% \[
%     V \mapsto \C(A,T^V B)
% \]
% For a representable functor $\yon U$, use the fact that the representable tangent structure induced by $\yon$ coincides with the cofree tangent structure and find:
% \[
%     \w(\yon U, V \mapsto \C(A, T^VB)) \cong V \mapsto \C(A, T^V.T^U(B))
% \]
% Then, a tangent category does not simply have enrichment in $\w$; it also has coherently chosen powers by representable functors (this applies to any actegory - the original insight bridging actegories and presheaf-enriched categories \cite{Wood1978}). $\w$-categories with coherently chosen powers by representables are \emph{exactly} tangent categories. Thus we have the following:
% \begin{proposition}[\cite{Garner2018}]
%     A tangent category is exactly an $\w$-category with coherently chosen powers by representables.
% \end{proposition}
Now the original notions of (lax, strong, strict) tangent functors can be shown to correspond to power-preservation properties of $\w$-functors between $\w$-categories with coherently chosen powers:
\begin{theorem}[\cite{Garner2018}]\label{thm:2-cat-tangcat}
    We have the following equivalences of 2-categories:
    \begin{enumerate}[(i)]
        \item the 2-category of $\w$-categories with coherently chosen powers and $\mathsf{TangCat}_{\mathsf{Lax}}$;
        \item the 2-category of $\w$-categories with coherently chosen powers and power-preserving $\w$ functors  and $\mathsf{TangCat}_{\mathsf{Strong}}$;
        \item The 2-category of $\w$-categories with coherently chosen powers and chosen-power-preserving $\w$ functors and $\mathsf{TangCat}_{\mathsf{Strict}}$.
    \end{enumerate}
\end{theorem}
Note that for a lax tangent functor $(F,\alpha): \C \to \D$, the map
\[
    \alpha.X: F.T.X \to T.F.X
\] can be seen as the unique morphism induced by universality. Conversely, for a strong tangent functor, the natural isomorphism $\alpha$ is the isomorphism from \emph{a} power $F.T.X$ to the \emph{coherently chosen} power $T.F.X$. In contrast, a strict tangent functor preserves the coherent choice of powers.

The ability to work with $\w$-categories that do not have powers by representables allows for significant flexibility. It is useful to observe that there are $\w$-categories which are not tangent categories.
\begin{example}%
    \label{ex:w-cats}
    ~\begin{enumerate}[(i)]
        \item Every monoid $(M, m, e)$ in $\w$ gives rise to a one-object $\w$-category whose hom-object is $M$, composition is $m$, and unit is $e$.
        \item Given a tangent category\, $\C$, it is possible to take the full\, $\w$-category over some set of objects\, $\D$, even though $D$ may not be closed under iterated applications of the tangent functor.
        \item The dual of a\, $\w$-category is a\, $\w$ category, where
        \[
            \C^{op}(A,B) = \C(B,A): \wone \to \s.    
        \]
        Dually, in the case that\, $\C$ is a tangent category, $\C^{op}$ will have coherent \emph{copowers} by representables.
        \item If a cartesian category\, $\C$ has an infinitesimal object $D$ (Definition \ref{def:inf-object}), it has a natural $\w$-category structure (with coherent \emph{copowers} by representables)
        \[
            \C(A,B):\wone \to \s := \underline{\C}(A \x D(-), B).
        \]
        Using the dual tangent structure on $\C^{op}$ from Proposition \ref{prop:inf-object-tangent-structures}, the enrichment on $\C$ is exactly the enrichment found by regarding $\C$ as the dual $\w$-category of the tangent category\, $\C^{op}$.
    \end{enumerate}
\end{example}

As a final remark, note that the Yoneda lemma applies to $\w$-categories, so there is an embedding
\[
    \C \hookrightarrow [\C^{op},\s].
\]
The powers and copowers by representables are computed pointwise in a presheaf category, so they inherit the coherent choice. Thus the following holds:
\begin{corollary}[\cite{Garner2018}]
    Every tangent category embeds into a $\w$-cocomplete representable tangent category.
\end{corollary}
(In fact, showing that the Yoneda embedding applies to tangent categories is the main theorem of \cite{Garner2018}.)


\section{Differential and anchored bundles as enriched structures}%
\label{sec:enriched-structures}

This section gives an enriched-categorical reinterpretation of the work in Chapter \ref{ch:differential_bundles} regarding differential bundles and Chapters \ref{ch:involution-algebroids} and \ref{chap:weil-nerve} regarding anchored bundles.

\subsection*{Differential bundles as enriched structures}%
\label{sub:differential-bundles-as-enriched-structures}

As a first case study using the enriched perspective for tangent categories, consider differential bundles. Most of the work in \Cref{ch:differential_bundles} uses the intuition that differential bundles are some sort of tangent-categorical algebraic theory; this section will make that intuition concrete. Recall that a \emph{lift} (Definition \ref{def:lift}) is a map $\lambda:E \to TE$. Using the enriched perspective and treating $TE$ as a power (recall Definition \ref{def:power}), this gives the following correspondence:
\[
    \infer{\hat{\lambda}:D \to \C(E,E)}{\lambda: 1 \to \C(E, E^D)}
\]
The commutativity condition for $\hat{\lambda}$, then, is translated as follows:
\[\input{TikzDrawings/Ch5/dbun-commutativity.tikz}\]
That is, $\hat{\lambda}$ is a semigroup morphism $D \to \C(E,E)$. In any cartesian closed category with coproducts, a semigroup may be freely lifted to a monoid using the "exception monad" $(-) + 1$ from functional programming (see, for example, \cite{Seal2013}).
\begin{definition}%
    \label{def:lambda-monoid}
    Regard the following monoid as the one-object $\w$-category $\Lambda$:
    \[
        \infer{m: (D+1) \x (D+1) \to D+1}{D \x D + D + D + 1 \xrightarrow{(\iota_L\o m | \iota_L | \iota_L |\iota_R)} D + 1}
    \]
\end{definition}
Thus, a lift $\lambda$ is exactly a functor $\hat{\lambda}:\Lambda \to \C$. Now check that morphisms are tangent natural transformations. Note that the semigroup $D$ is commutative, so $\Lambda = \Lambda^{op}$; the choice of using $\Lambda^{op}$ in the next lemma is to be consistent with conventions used in \Cref{sec:enriched-nerve-constructions}.

\begin{lemma}%
    \label{lem:cat-of-lifts-iso}
    The category of lifts in a tangent category $\C$ is isomorphic to the category of $\w$-functors and $\w$-natural transformations\, $\Lambda^{op} \to \C$.
\end{lemma}
\begin{proof}
    Check that a $\w$-natural transformation is exactly a morphism of lifts $f: \lambda \to \lambda'$. Start with the $\w$-naturality square:
    \[\input{TikzDrawings/Ch5/w-nat.tikz}\]
    Now, rewriting $D+1 \to \C(A,B)$ as a semigroup map $D \to \C(A,B)$, we have:
    \[
        \infer{
            \infer{
                1 \xrightarrow{Tf \o \lambda'}\C(A,TB)
            }{
                1 \xrightarrow{(\lambda',f)} \C(A,TA) \x \C(A,B) \xrightarrow{1 \x T} \C(A,TA) \x \C(TA,TB) \xrightarrow{m_{A,TA,TB}} \C(A,TB)
            }}{
            D \xrightarrow{(\lambda,f\o !)} \C(A,A) \x \C(A,B) \xrightarrow{m_{AAB}} \C(A,B)
        }
    \]
    Similarly, the other path is exactly $\lambda' \o f$. Thus, a $\w$-natural transformation is exactly a morphism of lifts. %TODO make sure the change l \Rightarrow \lambda' makes sense
\end{proof}

It is a classical result in synthetic differential geometry that the object $D$ has only one point. In the case of $\w$, this follows from the Yoneda lemma (regarding $\w$ as a $\s$-category):
\[
    \w(1,D) = \wone(\N[x]/x^2, \N) = \{ !: \N[x]/x^2 \to \N \}
\]
Note that the natural idempotent $e:id \Rightarrow id$ from Proposition \ref{prop:idempotent-natural}, then, must be the point $0:1 \to D$. Note that this idempotent is an absorbing element of the monoid $D+1$, so for any $f:X \to D+1$, it follows that $m(f,0\o !) = m(0\o !, f) = 0\o !$.  A pre-differential bundle is exactly a lift with a chosen splitting of the natural idempotent $p\o\lambda$.
\begin{lemma}%
    \label{lem:splitting-of-idemp-is-lambdaplus}
    For every lift $\bar{\lambda}:\Lambda^{op} \to \C$, the natural idempotent $e = p \o \lambda$ is exactly
    \[1 \xrightarrow[]{0} D \xrightarrow[]{\iota_L} D+1 \to \C(E,E).\]
\end{lemma}
Now that the natural idempotent is understood as a map in $\Lambda$, that idempotent splits to give the theory of a pre-differential object.
\begin{definition}%
    \label{def:lambda-plus}
    The $\w$-category $\Lambda^+$ is given by the set of objects $\{0,1\}$ with hom-Weil-spaces. Specifically,
    \begin{itemize}
        \item The hom-spaces are $\Lambda^+(1,1) = D+1$, otherwise $\Lambda^+(i,j) = 1$.
        \item Composition (writing the original composition from $\Lambda$ as $m$) is given by
        \begin{gather*}
            m_{111}: (D+1) \x (D+1) \xrightarrow[]{m} (D+1) \\
            m_{101}: 1 \x 1 \xrightarrow[]{\iota_R} (D + 1) \\
            \text{otherwise: } m_{ijk} = ! 
        \end{gather*}  
    \end{itemize}
\end{definition}
% The following lemma lets simplifies the composition in $\Lambda^+$.
% \begin{lemma}
%   Associativity of composition in $\Lambda^+$ can be written by embedding each hom-space into $D+1$
%   $\w$-category, where we use inclusions:
%   \begin{gather*}
%       \Lambda^+(0,0) = 1 \xrightarrow{\iota_L \o 0 \o !} D+1  \hspace{0.25cm}
%       \Lambda^+(0,1) = 1 \xrightarrow{\iota_L \o 0 \o !} D+1  \\
%       \Lambda^+(1,0) = 1 \xrightarrow{\iota_L \o 0 \o !} D+1  \hspace{0.25cm}
%       \Lambda^+(1,1) = D+1 \xrightarrow{id} D+1  
%   \end{gather*}
%     \[\input{TikzDrawings/Ch5/assoc.tikz}\]
% \end{lemma}
Idempotent splittings are absolute (co)limits, and are preserved by all functors; as this is a limit completion, we have the following.
\begin{lemma}%
    \label{lem:lambda-plus-is-pdb}
    The category of pre-differential bundles is exactly the category of $\w$-functors $\Lambda^+\to \C$ (that is, $\C$-valued presheaves).
\end{lemma}
% ($(\Lambda^+)^{op} = \Lambda^+$, so this is purely conventional).
It is straightforward to exhibit the category of differential bundles as a reflective subcategory of pre-differential bundles in $[\Lambda^+, \C]$ (so long as $\C$ has equalizers).
\begin{proposition}%
    \label{prop:Lambda-is-refl-subcat}
    The category of differential bundles in a tangent category\, $\C$ with $T$-equalizers and $T$-pullbacks is a reflective subcategory of\, $[\Lambda^+, \C]$.
\end{proposition}
\begin{proof}
    The category of pre-differential bundles in $\C$ is isomorphic to $[\Lambda^+, \C]$. By Corollary \ref{cor:idemp-dbun}, the category of differential bundles is the category of algebras for an idempotent monad on the category of pre-differential bundles in $\C$. The reflector sends a pre-differential bundle to the $T$-equalizer: \input{TikzDrawings/Ch5/reflector.tikz} This equalizer will always exist if $\C$ has equalizers, and pullbacks of the projection will exist if $\C$ has pullbacks, so the pullback is a differential bundle. This reflection gives a left-exact idempotent monad on $[\Lambda^+, \C]$ whose algebras are differential bundles.
\end{proof}
Now, in the case that $\C$ is a \emph{locally presentable} tangent category (such as $\w$), $[\Lambda^+, \C]$ is locally presentable and so is the reflective subcategory of differential bundles; thus the following holds.
\begin{corollary}%
    \label{cor:Lambda-dense}
    If\, $\C$ is locally presentable, then $\mathsf{DBun}(\C)$ is a locally presentable category.
\end{corollary}

\subsection*{Anchored bundles}

There are two ways to think about anchored bundles:
\begin{enumerate}[(i)]
    \item an anchored bundle is a differential bundle with an anchor $A \to TM$, or
    \item an anchored bundle is an involution algebroid without an involution.
\end{enumerate} 
These two perspectives can be unified by regarding $A$ as a cylinder for the weighted limit $TM$, so that the anchor is induced by the unique map $A.T.0 \to T.A.0$:
\[\input{TikzDrawings/Ch5/anc-as-nat.tikz}\]
That is, the syntactic category for anchored bundles is constructed as a full $\w$-category of $\wone$ that doesn't include the map $c$. This may be found by taking the full subcategory of $\wone$ whose objects are constructed out of $W_n, n\in \mathbb{N}$.
\begin{definition}%
    \label{def:truncated-wone}
    A Weil algebra has \emph{width} $k \in \N$ if it can be written
    \[
        V = \ox^{0 \le i < k} W_{n(i)},\; n(i) \in \N
    \]
    The category of $k$-truncated Weil algebras, written\, $\wone^k$, is the full $\w$-subcategory of\, $\wone$ of Weil algebras of width $k$ or less.

    The full subcategory whose objects are $\{ \N, W \}$ will be given the special notation $\wone^*$.
\end{definition}
Note that for each $V$ in $\wone^k$, the enrichment is given by
\[
    \wone^k(U,V) := (X \mapsto \underline{\wone}(U, X \ox V) ).
\]
The maps in $\wone^1$---that is, the full subcategory of $\wone$ whose objects are
\[
    \{ W_n | n \in \mathbb{N} \}
\]
---have the useful property that they may be written without the flip $c$. This makes $\wone^1$ a natural candidate for the syntactic category of anchored bundles. 
\begin{lemma}\label{lem:writing-maps-in-wone}
    Every morphism
    \[
        W_n \to V \in \wone
    \]
    may be written without $c$; that is, it is generated by the set of maps $\{p,+,0,\ell\}$ closed under tensor, composition, and maps induced by transverse limits.
\end{lemma}
\begin{proof}
    Every map $W_n \to V$ in $\wone$ is the finite sum
    \[
        v \mapsto \sum_{x} (A_x v)\bullet x
    \]
    where $x \in \mathsf{var}(V)$, and $A_x \in \N^n$ so that $A_n v$ is the ordinary dot product.
    Each term can be written without $c$, and the whole term is then constructed by adding each component using the appropriate $U.+.V$-symbols. \footnote{This may also be regarded as a consequence of the graphical notation for maps in $\wone$ in Table 1 on page 308 of \cite{Leung2017}.}
\end{proof}
 It is possible, then, to show that the category of anchored bundles in $\C$ is a full sub-tangent-category of functors $\wone^1 \to \C$; note that $W$ acts as a cone for $(-)^{D}$, so this induces a map
\[
    \anc: F(W) \to T.F(\N).
\]
\begin{proposition}
    \label{prop:nerve-anc-work}
    Every anchored differential bundle in $\C$ determines a functor $\wone^1 \to \C$; an anchored bundle morphism is exactly an enriched natural transformation.
\end{proposition}
\begin{proof}
    Start with an anchored bundle $(q:E \to M, \xi, \lambda, \anc)$; then for hom-objects with domain $\N$,
    \[
        \wone^1(\N,\N) = 1 \mapsto id_M, \hspace{0.5cm} \wone^1(\N,T) = 1 \mapsto \xi.
    \]
    For the hom-objects with domain $T$, the problem is slightly more difficult, as $\wone^1(T,\N)(T^V) = \wone(T, T^V)$ and $\wone^1(T, T)(T) = \wone(T, T.T^V)$. This part of the  proof amounts to constructing maps
    \[
        \wone(T, T^V) \to \C(E, T^V.M), \hspace{0.5cm}  \wone(T,T^V.T) \to \C(E, T^V.E)/
    \]
    The first mapping is straightforward: send $\theta$ to $\theta.M \o \anc$. For the second map, observe that the following diagrams commute:
    \[
        \input{TikzDrawings/Ch5/anc-atlas.tikz}
    \]
    The idea is to take an anchored bundle morphism $f$ and rewrite it as a string of compositions that does not include $c$, switching out every occurrence of
    \[
        T^V.\theta.M, \theta \in \{p,0,+,\ell\}
    \]
    and replacing it with $T^V.(\theta')$, where $\theta'$ is the corresponding map in $\{q,\xi,+_q,\lambda\}$. This induces a map \[\wone^1(W,W) \to \C(E,E) \in \w\] and the $\anc$ is exactly the unique $\alpha: F.W \to T.F$ induced by universality.

    For morphisms, the inclusion $(\Lambda^+)^{op} \to \wone$ ensures that any tangent natural transformation will be a linear morphism on the underlying differential bundle, and the tangent natural transformations coherences ensure that a tangent natural transformation will preserve the anchor. Conversely, an anchored bundle morphism will preserve each of the constructed morphisms $E \to T^V.E, E \to T^V.M$ (as it preserves each of $\{ q, +_q, \xi, \lambda\}$), giving a natural tangent transformation. Thus there is a faithful embedding $\mathsf{Anc}(\C) \hookrightarrow [\wone^1, \C]$.
\end{proof}
The converse, identifying those functors $\wone^1 \to \C$ that determine anchored bundles, is immediate.
% It is possible to restrict attention to $\wone^*$ from.
\begin{corollary}%
    \label{cor:anchored-bundle-as-nerve}
    The category of anchored bundles comprises precisely the $\w$-functors and $\w$-natural transformations
    $
        A: \wone^* \to \C    
    $ (Definition \ref{def:truncated-wone})
    so that the precomposition $\Lambda^+ \to \wone^* \to \C$ determines a differential bundle.
    That is to say, the category of anchored bundles in $\C$ is the following pullback in $\w$Cat:
    % https://q.uiver.app/?q=WzAsNCxbMCwwLCJcXG1hdGhzZntBbmN9KFxcQykiXSxbMSwwLCJbXFx3b25lXjEsIFxcQ10iXSxbMSwxLCJbXFxMYW1iZGFeKywgXFxDXSJdLFswLDEsIlxcbWF0aHNme0RCdW59KFxcQykiXSxbMCwxLCIiLDAseyJzdHlsZSI6eyJ0YWlsIjp7Im5hbWUiOiJob29rIiwic2lkZSI6InRvcCJ9fX1dLFsxLDJdLFszLDIsIiIsMix7InN0eWxlIjp7InRhaWwiOnsibmFtZSI6Imhvb2siLCJzaWRlIjoidG9wIn19fV0sWzAsM10sWzAsMiwiIiwxLHsic3R5bGUiOnsibmFtZSI6ImNvcm5lciJ9fV1d
    \[\begin{tikzcd}
        {\mathsf{Anc}(\C)} & {[\wone^1, \C]} \\
        {\mathsf{DBun}(\C)} & {[\Lambda^+, \C]}
        \arrow[hook, from=1-1, to=1-2]
        \arrow[from=1-2, to=2-2]
        \arrow[hook, from=2-1, to=2-2]
        \arrow[from=1-1, to=2-1]
        \arrow["\lrcorner"{anchor=center, pos=0.125}, draw=none, from=1-1, to=2-2]
    \end{tikzcd}\]
\end{corollary}

%  
\section{Enriched nerve constructions}%
\label{sec:enriched-nerve-constructions}

Nerve constructions present a powerful generalization of the Yoneda functor that sends an object $C \in \C$ to the representable presheaf $\C(-, C) \in[\C^{op},\vv]$ (for a general reference on nerve constructions and their realizations, see Chapter 3 of \cite{Loregian2015}). 
The Yoneda lemma states that this functor is an \emph{embedding} (that is, it is fully faithful), so no information is lost when embedding a category into its category of presheaves. 
Nerve constructions, then, move this towards an \emph{approximation} of the original category $\C$ by some subcategory $\D \hookrightarrow \C$, or more generally by some functor $K:\a \to \C$.
In Section \ref{sec:inf-nerve-of-a-gpd}, the ``infinitesimal'' approximation of a Lie groupoid as a Lie algebroid will be exhibited as approximation by a $\w$-functor $\partial: \wone^{op} \to \mathsf{Gpd}(\w)$.

We work with a $\vv$ that is locally presentable as a monoidal category, such as $\w, \s$, or the category of commutative monoids.
\begin{definition}%
    \label{def:nerve-of-a-functor}
    The \emph{nerve} of a $\vv$-functor $K: \a \to \C$ is the functor
    \[
        N_K: \C \to [\a^{op}, \vv]; \hspace{0.5cm} C \mapsto \C(K-, C)
    \]
    Any presheaf $A:\a \to \vv$ that is in the image of $N_K$ is a \emph{$K$-nerve}.
\end{definition}
\begin{remark}
    The ``$K$-nerve'' terminology seems to go back to Grothendieck/Segals's original intuition for the nerve construction of a category\footnote{Segal published the result, but seems to have credited the theorem to Grothendieck.} (\cite{Segal1974}) and appears in, for example, \cite{Berger2012} and \cite{Bourke2019}. However, the ``approximation'' of a category by a functor $K:\a \to \C$ originally used in topology seems to be a more intuitive description of the functor $N_K$.
\end{remark}


\begin{example}%
    \label{ex:nerve-functors}
    ~\begin{enumerate}[(i)]
        \item The nerve of the identity functor $\C = \C$ is the usual Yoneda embedding $\C \hookrightarrow[\C^{op}, \w]$.
        \item 
        The first example of a nerve construction in the mathematical literature is the simplicial approximation of a topological space by \cite{Kan1958}. Recall the original construction of the simplicial nerve of a topological space $X$, where $X_n = \mathsf{Top}(\Delta_n,X)$. This is exactly the nerve of the functor $\Delta \to \mathsf{Top}$ that sends $n$ to the $n$-simplex
        \[
            \left\{
                x \in \R^n | \sum x_i = 1
            \right\}.
        \]
        \item The following example figures into Segal's original nerve construction for a category, and is revisited in Section \ref{sec:enriched-theories}. Define a reflexive graph to be a presheaf over the full subcategory of $\mathsf{Cat}$ whose objects are the two preorders
        \[
            [0] = 0, \hspace{0.2cm} [1] =  0 < 1.    
        \]
        This is equivalent to the free category with two parallel arrows and a common retract,
        % https://q.uiver.app/?q=WzAsMixbMCwwLCIxIl0sWzEsMCwiMCJdLFsxLDAsInQiLDIseyJvZmZzZXQiOjJ9XSxbMSwwLCJzIiwwLHsib2Zmc2V0IjotMn1dLFswLDEsImUiLDFdXQ==
        \[\begin{tikzcd}
            {[1]} & {[0]}
            \arrow["t"', shift right=2, from=1-2, to=1-1]
            \arrow["s", shift left=2, from=1-2, to=1-1]
            \arrow["e"{description}, from=1-1, to=1-2]
        \end{tikzcd}\]
        so that a graph in a category has an object-of-vertices $V$ and an object-of-edges $E$, along with source and target maps $s,t:E \to V$; morphisms of graphs are maps $(f_E,f_V)$ that commute with the source and target maps.

        By Corollary \ref{cor:dense-ff-result}, any full subcategory containing the representables will be dense.
        Define the category of \emph{paths}, $\mathsf{Pth}$, to be full subcategory of graphs generated by
        \[ \input{TikzDrawings/Ch5/dense-in-graphs.tikz}, n \in \mathbb{N} \]
        so that $\mathsf{Gph}([n],G)$ picks out the set of paths of length $n$ in a graph $G$. We call this subcategory $P:\mathsf{Pth} \to \mathsf{Gph}$, and see that $N_P$ sends a graph to its $\mathsf{Pth}$-presheaf of composable paths:
        % https://q.uiver.app/?q=WzAsNyxbMCwxLCJFIl0sWzIsMSwiXFxkb3RzIl0sWzEsMiwiViJdLFszLDIsIlYiXSxbNCwxLCJFIl0sWzIsMiwiXFxkb3RzIl0sWzIsMCwiRV9uIl0sWzAsMiwidCJdLFsxLDIsInMiLDJdLFsxLDMsInQiXSxbNCwzLCJzIiwyXSxbNiwwXSxbNiw0XSxbNiwxLCIiLDAseyJzdHlsZSI6eyJuYW1lIjoiY29ybmVyIn19XV0=
        \[\begin{tikzcd}
            && {E_n} \\
            E && \dots && E \\
            & V & {} & V
            \arrow["t", from=2-1, to=3-2]
            \arrow["s"', from=2-3, to=3-2]
            \arrow["t", from=2-3, to=3-4]
            \arrow["s"', from=2-5, to=3-4]
            \arrow[from=1-3, to=2-1]
            \arrow[from=1-3, to=2-5]
            \arrow["\lrcorner"{anchor=center, pos=0.125, rotate=-45}, draw=none, from=1-3, to=2-3]
        \end{tikzcd}\]
        The pushout $[n]$ is precisely the graph
        \[
            0 \xrightarrow[]{} 1 \xrightarrow[]{}\dots \xrightarrow[]{} (n-1) \xrightarrow[]{} n    
        \]
        where the reflexive map $e$ at each vertex is suppressed.
    \end{enumerate}
\end{example}
Recall the functor sending reflexive graphs to anchored bundles from Example \ref{ex:prolongations}(iv).
This may be restated as a nerve construction.
\begin{proposition}%
    \label{prop:lin-approx-gph}
    There is a functor
    \[
        \partial: \wone^* \to \mathsf{Gph}(\w)  
    \]
    so that $N_\partial: \mathsf{Gpd}(\w) \to [\wone^*, \w]$ is the linear approximation of a reflexive graph as described in Example \ref{ex:prolongations} (iv).
\end{proposition}
\begin{proof}
    Let $s,t:C \to M, e:M \to C$ be a reflexive graph, and recall that the anchored bundle $C^\partial \to C$ is induced by the equalizer
    \[\input{TikzDrawings/Ch3/Sec9/eq-of-idem.tikz}\]
    The category $\mathsf{Gph}$ is a representable tangent category, where $D$ is the discrete graph $D = D$. By the Yoneda lemma, we have that
    \[
        [\mathsf{Gph}, \w](\yon 1, G) = G_1,
        [\mathsf{Gph}, \w](\yon 0, G) = G_0,
        [\mathsf{Gph}, \w](D \x \yon(i), G) = T.G_i,
    \]
    so the limit diagram defining the infinitesimal approximation of a graph becomes
    % https://q.uiver.app/?q=WzAsMyxbMSwwLCJcXG1hdGhzZntHcGh9KEQgXFx4IEksIEcpIl0sWzIsMCwiXFxtYXRoc2Z7R3BofShEIFxceCBJLCBHKSJdLFswLDAsIlxcbWF0aHNme0dwaH0oXFxwYXJ0aWFsLCBHKSJdLFsyLDBdLFswLDEsIihlIFxceCBJKV4qIiwwLHsib2Zmc2V0IjotMX1dLFswLDEsIihJIFxceCBlXnMpXioiLDIseyJvZmZzZXQiOjF9XV0=
    \[\begin{tikzcd}
        {C^\partial} & {\mathsf{Gph}(D \x I, C)} & {\mathsf{Gph}(D \x I, C).}
        \arrow[from=1-1, to=1-2]
        \arrow["{(e \x I)^*}", shift left=1, from=1-2, to=1-3]
        \arrow["{(D \x e^-)^*}"', shift right=1, from=1-2, to=1-3]
    \end{tikzcd}\]
    Now observe that $\mathsf{Gph}(\w)^{op}$ with the dual tangent structure from Proposition \ref{prop:inf-object-tangent-structures} has a reflexive graph object \[s,t:1 \to I, !:I \to 1.\] Construct its linear approximation as in Example \ref{ex:prolongations}(iv) (e.g. take the coequalizer of $\w$-graphs):
    % https://q.uiver.app/?q=WzAsMyxbMCwwLCJEXFx4IEkiXSxbMSwwLCJEXFx4IEkiXSxbMiwwLCJcXHBhcnRpYWwiXSxbMCwxLCJlIFxceCBJIiwwLHsib2Zmc2V0IjotMX1dLFswLDEsIkQgXFx4IGVecyIsMix7Im9mZnNldCI6MX1dLFsxLDJdXQ==
    \[\begin{tikzcd}
        {D\x I} & {D\x I} & \partial.
        \arrow["{e \x I}", shift left=1, from=1-1, to=1-2]
        \arrow["{D \x e^s}"', shift right=1, from=1-1, to=1-2]
        \arrow[from=1-2, to=1-3]
    \end{tikzcd}\]
    By Proposition \ref{prop:nerve-anc-work}, this determines a functor
    \[
        \partial: \wone^1 \to \mathsf{Gph}(\w)^{op}. 
    \]
    By the continuity of the hom-functor, the nerve $N_\partial: \mathsf{Gph}(\w) \to [\wone^*, \w]$ lands in the category of anchored bundles. The continuity of the hom-functor ensures that this is indeed the linear approximation from Example \ref{ex:prolongations} (iv).
\end{proof}


The simplicial localization in \cite{Kan1958} has a left adjoint, the \emph{geometric realization}, that constructs a topological space using the data of a simplicial set. This realization may be constructed using a left Kan extension.
\begin{definition}%
    \label{def:realization}
    Let $K: \a \to \C$ be a $\vv$-functor for a cocomplete $\,\C$.
    The \emph{realization} of $K$ is the left Kan extension
    \[
        \input{TikzDrawings/Ch5/realization.tikz}
        \hspace{0.5cm}
        |C|_K := \int^A \C(KA, C) \bullet KA
    \]
    A \emph{nerve/realization context} is a $\vv$-functor $K:\a \to \C$ from a small $A$ to a cocomplete $\C$.   
\end{definition}
The adjunction between simplicial sets and topological spaces follows from general categorical machinery as a nerve/realization context:
\begin{lemma}
    For every nerve/realization context $K: \a \to \C$, the realization of $K$ is left adjoint to the nerve of $K$.
\end{lemma}


\section{Nervous monads and algebroids}%
\label{sec:enriched-theories}

Transposing the characterization of algebroids from Theorem \ref{thm:weil-nerve} to the enriched perspective introduces the challenge of finding an appropriate framework to describe algebroids as enriched structures. Kelly's theory of enriched sketches (see Chapter 6 of \cite{Kelly2005}), small $\vv$-categories equipped with a chosen class of limits cones, seems to be a natural candidate. However, when regarding a tangent functor as a $\w$-functor, the natural part 
\[
    \alpha: F.T \Rightarrow T.F 
\]
is the unique morphism induced by the universality of $T$ as a weighted limit, so it becomes unclear how to translate the condition that $\alpha$ is a cartesian natural transformation (Definition \ref{def:cart-nat}).

A clue for how to proceed may be found in \cite{Kapranov2007}, which proved that Lie algebroids are monadic over anchored bundles (when allowing for infinite-dimensional vector bundles). 
Recent work in \cite{Bourke2019} and \cite{Berger2012} has developed the appropriate notion of (enriched) theories that correspond to monads over general locally presentable categories. % using the notion of \emph{dense} functors $K: \a \to \C$. 
A critical insight is that for a filtered-colimit-preserving monad $\mathbb{T}$ on $\s$, the opposite category of the Lawvere theory $\th$ is precisely the full subcategory of the Kleisli category whose objects are $[n] = \coprod_n 1$, and the \emph{nerve} of the inclusion
\[
    \th^{op} \hookrightarrow \s^{\mathbb{T}}
\]
is fully faithful; that is, when the functor $K$ is dense.
Formally, a functor is \emph{dense} whenever its nerve behaves like the Yoneda embedding. The theory of dense functors is developed in Chapter 5 of \cite{Kelly2005}. We continue to assume that an arbitrary site of enrichment $\vv$ is locally presentable as a $\vv$-category.
\begin{definition}%
    \label{def:dense}
    A functor $K: \a \to \C$ is \emph{dense} whenever the nerve of $K$ is fully faithful, and a subcategory inclusion that is dense will be called a \emph{dense subcategory}.
    % \begin{itemize}
    %   \item The nerve of $K$ is fully faithful.
    %   \item The identity functor $id:\C \to \C$ is the left Kan extension of $K$ along itself, so: $Lan_K K = id$.
    %   \item Every object is given as the weighted colimit $C = N_K(C) \star K$.
    % \end{itemize}
    % Note that a locally finitely presentable category is precisely\footnote{Assuming that Vopenka's principle holds \cite{Adamek1994}.}
    %  a cocomplete category $\C$ with a dense functor $K: \a \to \C$.
\end{definition}

Dense functors are poorly behaved under composition, but there is a useful cancellativity result from Section 5.2 of \cite{Kelly2005}.
\begin{proposition}
    Consider a diagram of $\vv$-categories
    \[\input{TikzDrawings/Ch5/lan-result.tikz}\]
    where $\alpha$ is a natural isomorphism and $K$ is dense.
    If this diagram exhibits $(\alpha, J)$ as the left Kan extension of $K$ along $F$, then $J$ is also dense.
\end{proposition}
\begin{corollary}%
    \label{cor:dense-ff-result}
    If $K$ is dense, $F$ is fully faithful, and  $J.F = K$, then $F$ is a dense subcategory.
\end{corollary}
Generally speaking, we will often refer to dense \emph{subcategories} rather than dense \emph{functors}. 
This is achieved by factoring the dense functor
\[
    K = \a \xrightarrow[]{K'} \mathsf{im}(K) \hookrightarrow \C
\]
where $\mathsf{im}(K)$ is the $\vv$-category whose objects are those of $\a$ and whose hom-objects are given by $\mathsf{im}(K)(A,B) = \C(KA, KB)$,  so that the inclusion of $\mathsf{im}(K)$ into $\C$ is fully faithful and therefore dense by Corollary \ref{cor:dense-ff-result}.

\begin{example}\label{ex:fin-card-def}
    ~\begin{enumerate}[(i)]
        \item The category of finite cardinals $\Sigma$ is the full subcategory of $\,\s$ whose objects are given by finite coproducts of the terminal object, $[n] = \coprod_n 1$. This is a skeleton of the category of finite sets, and a dense subcategory of $\,\s$ (see e.g. \cite{Bourke2019}).
        \item Recall that the category of $\vv$-presheaves on $\a$, $[\a^{op}, \vv]$ is the free colimit completion of $\a$ (see e.g. \cite{Kelly2005}). 
        If $\C$ is cocomplete, and $K:\a \to \C$ is dense, then the realization $|-|_K$ exhibits $\C$ as a reflective subcategory of $[\a^{op}, \vv]$, and is this a locally presentable category\footnote{In fact, an equivalent definition of a locally presentable category is as a cocomplete category with a dense subcategory.}. Conversely, if $\,\C$ is a reflective subcategory of $\,[\a^{op}, \vv]$ that contains the representable functors, then $\a$ is a dense subcategory of $\,\C$. 
        \item By Corollary \ref{cor:Lambda-dense}, the category of differential bundles in $\w$ is a reflective subcategory of $[{\Lambda^+}{op}, \vv]$, so $\yon: \Lambda^+ \hookrightarrow [{\Lambda^+}{op}, \vv]$ is a dense subcategory of differential bundles in $\w$ following the argument in the above example.
    \end{enumerate}
\end{example}

Nervous theories (\cite{Bourke2019}) are generalizations of Lawvere theories that extend to arbitrary locally finitely presentable $\vv$-categories. Recall that a classical Lawvere theory is a bijective-on-objects, product-preserving functor
\[
    t: \Sigma \to \th   
\]
where $\th$ is a cartesian category. Nervous theories replace $\Sigma$ with a dense subcategory of some locally presentable $\vv$-category, and the product preservation condition with conditions on the nerve of the theory map.
\begin{definition}%
    \label{def:nerve-theory}
    Let $K:\a \to \C$ be a dense $\vv$-subcategory of a locally finitely presentable $\,\C$. We call the replete image of $N_K$ in $[\a^{op}, \vv]$ the category of $K$-nerves. An \emph{$\a$-theory} is a bijective-on-objects $\vv$-functor $J: \a \to \th$, where each
    \[
        \th(J-,a): \a \to \vv
    \] is a $K$-nerve.
    The category of models for an $\a$-theory is the pullback in $\vv\mathsf{CAT}$:
    \[\input{TikzDrawings/Ch5/conc-models.tikz}\]
    (These are called \emph{concrete} models in \cite{Bourke2019}.)
\end{definition}
\begin{remark}
    The category of models for a theory is monadic. The core of the argument is due to Weber's \emph{nerve theorem}, found in \cite{Weber2007}, but an exposition on that result is beyond the scope of this thesis. 
\end{remark}
\begin{example}%
    \label{ex:theories}
    ~\begin{enumerate}[(i)]
        \item A functor $t:\Sigma \hookrightarrow \th$ is a $\Sigma$-theory if and only if $\th$ is a Lawvere theory, where  we use the fact that $\Sigma$ (Example \ref{ex:fin-card-def}) a  skeletal subcategory of finite sets. The nerve conditions in this case identify the models of the Lawvere theory, as the nerve of $\,\Sigma \hookrightarrow \s$ sends a set to the strict product-preserving functor $[n] \mapsto A^n$.
        \item As shown in \cite{Berger2012} and \cite{Bourke2019}, the original nerve construction from \cite{Segal1974} may be restated as saying that small categories arise as models of a $\mathsf{Pth}$-theory (Example \ref{ex:nerve-functors}). The set $\mathsf{Gph}([n], G)$ is the set of paths of through the graph $G$ that have $n$ non-identity elements, as
        \[
            G \ts{t}{s} G \cong G \ts{t}{id} M \ts{id}{s} G \cong G \ts{t}{s \o e \o s} G \ts{t \o e \o t}{s} G.     
        \]

        Next, recall that the category $\Delta$ may be regarded as follows:
        \begin{itemize}
            \item Objects: A strict order $[n] = 0 < 1 < \dots < n$, for $n \ge 0$, regarded as a category.
            \item Maps: Functors.
        \end{itemize}
        There is a bijective-on-objects functor from $\mathsf{Pth} \to \Delta$ that sends the graph $[n]$ to the pre-order $[n]$, as every graph homomorphism between paths will be order-preserving. Now observe that a model is precisely a simplicial set, where 
        \[
            X([0]) = M, 
            X([1]) = C,
            X([2]) = C \ts{t}{s} C,
            X([3]) = C \ts{t}{s} C \ts{t}{s} C.  
        \]
        Furthermore, the map
        % https://q.uiver.app/?q=WzAsNSxbMCwxLCIwIl0sWzEsMCwiMCJdLFswLDMsIjEiXSxbMSw0LCIyIl0sWzEsMiwiMSJdLFswLDEsIiIsMCx7InN0eWxlIjp7InRhaWwiOnsibmFtZSI6Im1hcHMgdG8ifX19XSxbMiwzLCIiLDAseyJzdHlsZSI6eyJ0YWlsIjp7Im5hbWUiOiJtYXBzIHRvIn19fV0sWzAsMl0sWzEsNF0sWzQsM11d
        \[\begin{tikzcd}[row sep = tiny]
            & 0 \\
            0 \\
            & 1 \\
            1 \\
            & 2
            \arrow[maps to, from=2-1, to=1-2]
            \arrow[maps to, from=4-1, to=5-2]
            \arrow[from=2-1, to=4-1]
            \arrow[from=1-2, to=3-2]
            \arrow[from=3-2, to=5-2]
        \end{tikzcd}\]
        in $\Delta$ becomes a composition map: 
        \[ 
            \infer{C \ts{t}{s} C \to C}{X([2]) \to X([1])}
        \] 
        Associativity and unitality (for the section $e: M \to C$) follow by functoriality. Thus, a model of $\mathsf{Pth} \to \Delta$ is a small category, and a morphism is exactly a functor.
    \end{enumerate}
\end{example}
Corollary \ref{cor:anchored-bundle-as-nerve}, then, gives a monadicity result for $\w$-anchored bundles over $\w$-differential bundles.
\begin{proposition}
    $\w$-anchored bundles are models of the theory $(\Lambda^+)^{op} \to (\wone^*)^{op}$ (see Section \ref{sec:enriched-structures}).
\end{proposition}
\begin{proof}
    Note that the inclusion $\Lambda^+ \to \wone^1$ is a differential bundle, so it is a nerve. Then the diagram:
    % https://q.uiver.app/?q=WzAsNCxbMCwwLCJcXG1hdGhzZntBbmN9KFxcQykiXSxbMSwwLCJbXFx3b25lXiosXFxDXSJdLFsxLDEsIltcXExhbWJkYV4rLCBcXENdIl0sWzAsMSwiXFxtYXRoc2Z7REJ1bn0oXFxDKSJdLFsxLDJdLFszLDIsIiIsMix7InN0eWxlIjp7InRhaWwiOnsibmFtZSI6Imhvb2siLCJzaWRlIjoidG9wIn19fV0sWzAsM10sWzAsMV0sWzAsMiwiIiwxLHsic3R5bGUiOnsibmFtZSI6ImNvcm5lciJ9fV1d
    \[\begin{tikzcd}
        {\mathsf{Anc}(\w)} & {[\wone^*,\w]} \\
        {\mathsf{DBun}(\w)} & {[\Lambda^+, \w]}
        \arrow[from=1-2, to=2-2]
        \arrow[hook, from=2-1, to=2-2]
        \arrow[from=1-1, to=2-1]
        \arrow[from=1-1, to=1-2]
        \arrow["\lrcorner"{anchor=center, pos=0.125}, draw=none, from=1-1, to=2-2]
    \end{tikzcd}\]
    exhibits $\w$-anchored bundles as models of a $(\Lambda)^{op}$-theory.
\end{proof}
As a corollary, we see that the category of anchored bundles is locally presentable.
\begin{corollary}
    The category of anchored bundles is locally presentable, and $(\wone^*)^{op} \hookrightarrow \mathsf{Anc}(\w)$ is dense.
\end{corollary}
In \cite{Bourke2019}, the authors identify exactly those monads on a locally presentable $\vv$-category that correspond to the models of a theory.
Recall the notation that the category of algebras for a monad is $\C^T$ and the category of free coalgebras is $\C_T$. For a dense subcategory $\a \hookrightarrow \C$, use $\a_T$ for the category of free algebras over objects in $\a$. Recall that the Lawvere theory \[K:\mathsf{FinSet} \to \th\] for a filtered-colimit-preserving monad $\mathbb{T}$ on $\s$ may be re-derived as the full subcategory of free algebras over finite sets, $K_T: \mathsf{FinSet}_{\mathbb{T}} \hookrightarrow \s_{\mathbb{T}} \hookrightarrow \s^{\mathbb{T}}$. Nervous monads abstract this property.
\begin{definition}
    Let $\C$ be a locally presentable $\vv$-category, with $K:\a \to \C$ a dense sub-$\vv$-category. 
    A $\vv$-monad $\mathbb{T}$ over $\C$ is \emph{$K$-nervous} if
    \begin{enumerate}
        \item the inclusion $K_T: \a_T \hookrightarrow \C^T$ is dense;
        \item the following diagram is a pullback in $\vv$CAT:
        % https://q.uiver.app/?q=WzAsNCxbMCwxLCJcXEMiXSxbMSwxLCJbXFxhXntvcH0sIFxcdl0iXSxbMSwwLCJbXFxhX1Ree29wfSwgXFx2XSJdLFswLDAsIlxcQ15UIl0sWzAsMSwiIiwxLHsic3R5bGUiOnsidGFpbCI6eyJuYW1lIjoiaG9vayIsInNpZGUiOiJ0b3AifX19XSxbMiwxXSxbMywwXSxbMywyXV0=
        \[\begin{tikzcd}
            {\C^T} & {[\a_T^{op}, \vv]} \\
            \C & {[\a^{op}, \vv]}
            \arrow[hook, from=2-1, to=2-2]
            \arrow[from=1-2, to=2-2]
            \arrow[from=1-1, to=2-1]
            \arrow[from=1-1, to=1-2]
        \end{tikzcd}\]
    \end{enumerate}
\end{definition}

\begin{theorem}[\cite{Bourke2019}]
    Let $K:\a \hookrightarrow \mathbb{C}$ be a dense sub-$\vv$-category of a cocomplete $\vv$-category $\,\C$.
    There is an equivalence of categories between $\a$-theories and $\a$-nervous monads on $\,\C$.
\end{theorem}

The first step in showing that algebroids are monadic over anchored bundles is to construct a dense subcategory of $\mathsf{Anc}(\w)$ that plays the role of the $V$-prolongations from Definition \ref{def:monoidal-category}. The enriched framework makes this straightforward: prolongations are weighted limits, with which we may freely complete $\wone^*$.

\begin{definition}
    \label{def:prol}
    Consider the category $\mathsf{Anc}(\w)$ as a representable tangent category, and take the dual tangent structure on $\mathsf{Anc}(\w)^{op}$.
    The category $\prol$ is defined as the full subcategory of $\mathsf{Anc}(\w)^{op}$ whose objects are generated by the prolongations of the Yoneda embedding $\yon:\wone^* \to \mathsf{Anc}(\w)$.
\end{definition}

For any small $\w$-category $\C$, the free completion of $\C$ is the opposite $\w$-category of the category of copresheaves on $\C$.\footnote{This is the dual statement of the classical theorem that the category of presheaves is the free cocompletion of a small category $\C$.} Thus, $\prol$ is the free completion of $\wone^1$ with prolongations. Therefore, a choice of prolongations on an anchored bundle $A$ determines a unique functor $\prol \to \C$ given by right Kan extension (see e.g. \cite{Kelly1982}). The continuity of 
\[
    \mathsf{Anc}(\w)(-, A): \mathsf{Anc}(\w)^{op} \to \w 
\]
ensures that $\mathsf{Anc}(\w)(L_V, A)$ is $A_V$. 

By Theorem \ref{thm:weil-nerve}, the category of involution algebroids in $\C$ is a full sub-$\w$-category of $[\wone, \C]$ that has a forgetful functor down to the category of anchored bundles. A consequence of Theorem \ref{thm:iso-of-cats-inv-emcs} is that a functor $\wone \to \C$ is an involution algebroid if and only if the precomposition \[\wone^* \hookrightarrow \wone \to \C\] restricts to an anchored bundle, with each $V$ sent to the $V$-prolongation of this anchored bundle. In other words, algebroids are models of the following enriched theorem (as in Definition \ref{def:nerve-theory}).

\begin{definition}
    \label{def:weil-theory}
    Define the $\prol^{op}$-theory of algebroids as the functor
    \[
        a: \prol \to \wone.
    \]
    This functor is bijective-on-objects by definition, and satisfies the nerve condition because the $V$-prolongation of the tangent bundle is $T^V$, so each presheaf
    \[
        \wone(a-, V):\prol \to \w   
    \] 
    is an $\prol$-nerve.
\end{definition}

The Weil nerve, then, translates to the following characterization of the category of involution algebroids in a tangent category $\C$. 
\begin{theorem}%
    \label{thm:pullback-in-cat-of-cats-inv-algd}
    The category of involution algebroids with chosen prolongations in a tangent category $\C$ is precisely the pullback in $\w\mathsf{CAT}$:
    \begin{equation}\label{eq:prol2}
        \input{TikzDrawings/Ch5/prol2.tikz}
    \end{equation}
    % where $w$ is the reflector from $\mathsf{Anc}(\w)$ to $\w$ restricted to the full subcategory $\prol$. 
    Consequently, in $\w$ involution algebroids are monadic over anchored bundles.
\end{theorem}
\begin{proof}
    Recall the correspondence
    \[
        \infer{(\hat A, \alpha): \wone \to \C \in \mathsf{TangCat}_{lax}}
        {\bar A: \wone \to \C \in \w\cat}
    \]
    If
    \[
        \wone^* \to \prol \to \wone \to \C  
    \]
    determines an anchored bundle $(\pi:A \to M, \xi, \lambda, \anc)$ whose $V$-coprolongation is $\hat A.V$, so that $\hat A$ is the nerve of an involution algebroid by Corollary \ref{cor:the-prolongation-description}.
\end{proof}

Thus, we may characterize $\w$-involution algebroids as algebras for a $\prol$-nervous monad on the category of $\w$-anchored bundles.
\begin{corollary}
    The category of involution algebroids in $\w$ is equivalent to the category of algebras for the $\prol$-nervous monad on $\mathsf{Anc}(\w)$ generated by the theory
    \[
        a: \prol \to \wone
    \]
    from Definition \ref{def:weil-theory}, using Theorem 19 from \cite{Bourke2019}.
\end{corollary}

\begin{remark}
    The construction of involution algebroids as the category of models for a $\prol$-theory is remarkably similar to the original nerve construction for categories in \cite{Segal1974}, with the symmetric nerve construction for groupoids replacing $\Delta$ with the category of finite sets $\Sigma$ (see Example 44 (iv) of \cite{Bourke2019}). Each construction truncates the original category to two objects and builds a new category with a bijective set of objects by freely adding limits to the two-object truncation.
\end{remark}


\section{The infinitesimal approximation of a groupoid}%
\label{sec:inf-nerve-of-a-gpd}

This section contains the main theorem of the chapter, namely that there is an adjunction
% https://q.uiver.app/?q=WzAsMixbMCwwLCJcXG1hdGhzZntHcGR9KFxcdykiXSxbMSwwLCJcXG1hdGhzZntJbnZ9KFxcdykiXSxbMCwxLCIiLDAseyJjdXJ2ZSI6LTJ9XSxbMSwwLCIiLDAseyJjdXJ2ZSI6LTJ9XSxbMiwzLCIiLDAseyJsZXZlbCI6MSwic3R5bGUiOnsibmFtZSI6ImFkanVuY3Rpb24ifX1dXQ==
\[\begin{tikzcd}
    {\mathsf{Gpd}(\w)} & {\mathsf{Inv}(\w)}
    \arrow[""{name=0, anchor=center, inner sep=0}, curve={height=-12pt}, from=1-1, to=1-2]
    \arrow[""{name=1, anchor=center, inner sep=0}, curve={height=-12pt}, from=1-2, to=1-1]
    \arrow["\dashv"{anchor=center, rotate=-90}, draw=none, from=0, to=1]
\end{tikzcd}\]
This adjunction follows by inducing a nerve/realization context on the category of groupoids in $\w$:
\[
    \partial: \wone^{op} \to \mathsf{Gpd}(\w).   
\]
A simplified version of this functor appeared in Proposition \ref{prop:lin-approx-gph}, the linear approximation of a reflexive graph. 
In fact, $\partial$ will be exactly the free groupoid over the graph $\partial$ from Proposition \ref{prop:lin-approx-gph} (if we had worked in the category of reflexive graphs equipped with an involution).
Recall that groupoids and Weil spaces  are both models of a sketch (recall Section \ref{sec:tang-cats-enrichment}). 
The symmetry of the theories gives two equivalent presentations of groupoids in $\w$:
\begin{definition}%
    \label{def:w-gpd}
    The tangent category of $\w$-groupoids is equivalently
    \begin{enumerate}[(i)]
        \item the tangent category of internal groupoids in $\w$, where the tangent structure is computed pointwise;
        \item the tangent category of transverse-limit-preserving functors $\wone \to \mathsf{Gpd}(\s)$ with the cofree tangent structure from Observation \ref{obs:cofree-tangent-cat}.
    \end{enumerate}
\end{definition}
It is important to note that the tangent structure on $\w$-groupoids is representable, as it is a cartesian closed category with an infinitesimal object given by $\wone \hookrightarrow \w \hookrightarrow \mathsf{Gpd}(\w)$ (a proof that $\mathsf{Gpd}(\C)$ is cartesian closed for a cartesian closed category $\C$ with sufficient limits a may be found in Section B2.3 of \cite{Johnstone2002}).

There are two classes of ``trivial'' $\w$-groupoids that will be useful.
\begin{example}
    ~\begin{enumerate}[(i)]
        \item Every small groupoid $G$ may be regarded as a groupoid in $\w$ as the constant functor $\wone \to \mathsf{Gpd}(\s)$ sending $V$ to the groupoid $G$ preserves transverse limits, we may then apply the symmetry of theories to find a groupoid in the category of Weil spaces. 
        \item Every object $M$ in a category has a cofree groupoid given by $s,t:M = M$, this is the \emph{discrete} groupoid whose only morphisms are the identity maps. Thus, every Weil space $M \in \w$ has a corresponding discrete groupoid (this will often be written $M = M$ to denote the discrete groupoid over $M$).
    \end{enumerate}
\end{example}

Using the cartesian closure of $\w$ and the full subcategories of trivial and discrete groupoids, we have the following:
\begin{observation}
    The category of $\w$-groupoids is both
    \begin{enumerate}[(i)]
        \item a $\w$-category, with powers by $\w$. The power of a $\w$-groupoid $\mathcal{G} = s,t:G \to M$ by a Weil space $E$ is given by the $\w$-groupoid
        \[
            [E, \mathcal{G}] = \{[E,s], [E, t]: [E,G] \to [E, M], \hspace{0.15cm} [E,e]: [E, M] \to [E, G]  \};
        \]
        \item a $\mathsf{Gpd}$-enriched category with powers by small groupoids. The power of a $\w$-groupoid $\mathcal{G}:\mathsf{Gpd} \to \w$ by a groupoid $\mathcal{H}$ is the $\w$-groupoid
        \[
            [\mathcal{H}, \mathcal{G}](V) = [\mathcal{H}, \mathcal{G}(V)]_{\mathsf{Gpd}}.
        \]
    \end{enumerate}
\end{observation}

The following two $\w$-groupoids form the basic building block for the main result.
\begin{example}
    ~\begin{enumerate}[(i)]
        \item Set $D := \yon W$ in $\w$. The discrete groupoid $D = D$ represents the tangent functor internally, and the copresheaf $\mathsf{Gpd}(D, -): \mathsf{Gpd}(\w) \to \w$ sends a groupoid $s,t:G \to M$ to $TM$.
        \item The arrow groupoid is the free groupoid generated by the graph:
        \[\bullet \to \bullet\]
        Write the trivial $\w$-groupoid on this groupoid as $I$. Note that the power by $I$ will send a groupoid to its ``arrow groupoid'' $\mathcal{G}^\to$, the groupoid whose objects are arrows in $\mathcal{G}$, with a map $u \to v$ being a commuting square.
        It follows that $\mathsf{Gpd}(I, \mathcal{G})$ is the space of arrows of the groupoid, $\mathsf{Gpd}(I \x I, \mathcal{G})$ the space of commuting squares, and so on.
    \end{enumerate}
\end{example}

The next proposition gives the necessary properties of $I$ to construct the Lie derivative.
\begin{lemma}%
    \label{lem:arrow-gpd-facts}
    ~\begin{enumerate}[(i)]
        \item For every groupoid $\mathcal{G} = s,t:G \to M$, there is an isomorphism
        \[
            G \ts{t}{t} G \ts{s}{t} G \cong \mathsf{Gpd}(\w)(I \x I, \mathcal{G});
        \]
        this corresponds to the unique filler for the diagram
        % https://q.uiver.app/?q=WzAsNCxbMCwxLCJcXGJ1bGxldCJdLFswLDAsIlxcYnVsbGV0Il0sWzEsMCwiXFxidWxsZXQiXSxbMSwxLCJcXGJ1bGxldCJdLFswLDEsInUiXSxbMSwyLCJ2Il0sWzMsMiwidyIsMl0sWzAsMywiXFxleGlzdHMhIHdeey0xfSBcXG8gdiBcXG8gdSIsMix7InN0eWxlIjp7ImJvZHkiOnsibmFtZSI6ImRhc2hlZCJ9fX1dXQ==
        \[\begin{tikzcd}
            \bullet & \bullet \\
            \bullet & \bullet
            \arrow["u", from=2-1, to=1-1]
            \arrow["v", from=1-1, to=1-2]
            \arrow["w"', from=2-2, to=1-2]
            \arrow["{\exists! w^{-1} \o v \o u}"', dashed, from=2-1, to=2-2]
        \end{tikzcd}\]
        \item The arrow groupoid has a semigroup with a zero structure (like an infinitesimal object $D$), and the multiplication is the coequalizer
        % https://q.uiver.app/?q=WzAsNSxbMCwxLCJJIFxceCBJIl0sWzIsMSwiSSBcXHggSSJdLFszLDEsIkkiXSxbMSwwLCIxIFxceCBJIl0sWzEsMiwiSSBcXHggMSJdLFswLDMsIiEgXFx4IEkiXSxbMywxLCJzIFxceCBJIl0sWzAsNCwiSSBcXHggISIsMl0sWzQsMSwiSSBcXHggcyIsMl0sWzEsMiwibSJdXQ==
        \[\begin{tikzcd}
            & {1 \x I} \\
            {I \x I} && {I \x I} & I \\
            & {I \x 1}
            \arrow["{! \x I}", from=2-1, to=1-2]
            \arrow["{s \x I}", from=1-2, to=2-3]
            \arrow["{I \x !}"', from=2-1, to=3-2]
            \arrow["{I \x s}"', from=3-2, to=2-3]
            \arrow["m", from=2-3, to=2-4]
        \end{tikzcd}\]

    \end{enumerate}
\end{lemma}
The dual result of part $(ii)$ is somewhat more obvious to see, as the following fork is always an equalizer in $\w$ for a groupoid $G$:
% https://q.uiver.app/?q=WzAsMyxbMSwwLCJHXlxcc3F1YXJlIl0sWzIsMCwiR15cXHNxdWFyZSJdLFswLDAsIkciXSxbMCwxLCJHXntlXnMgXFx4IEl9IiwwLHsib2Zmc2V0IjotMn1dLFswLDEsIkdee0kgXFx4IGVec30iLDIseyJvZmZzZXQiOjF9XSxbMiwwLCJHXntcXGJveGRvdH0iXV0=
\[\begin{tikzcd}
    G & {G^\square} & {G^\square}
    \arrow["{G^{e[s] \x I}}", shift left=2, from=1-2, to=1-3]
    \arrow["{G^{I \x e[s]}}"', shift right=1, from=1-2, to=1-3]
    \arrow["{G^{m}}", from=1-1, to=1-2]
\end{tikzcd}\]
(where we write $G$ as the object of arrows and $\mathsf{Gpd}(\w)(I\x I, G)$ as $G^\square$). 
Note that $G^{e[s] \x I}$ and $G^{I \x e[s]}$ correspond to the idempotents on $G^\square$:
% https://q.uiver.app/?q=WzAsMTIsWzIsMCwiXFxidWxsZXQiXSxbMiwxLCJcXGJ1bGxldCJdLFszLDAsIlxcYnVsbGV0Il0sWzMsMSwiXFxidWxsZXQiXSxbNCwwLCJcXGJ1bGxldCJdLFs1LDAsIlxcYnVsbGV0Il0sWzQsMSwiXFxidWxsZXQiXSxbNSwxLCJcXGJ1bGxldCJdLFswLDAsIlxcYnVsbGV0Il0sWzAsMSwiXFxidWxsZXQiXSxbMSwwLCJcXGJ1bGxldCJdLFsxLDEsIlxcYnVsbGV0Il0sWzAsMSwicSIsMl0sWzAsMiwidSJdLFsyLDMsInYiXSxbMSwzLCJ3IiwyXSxbNCw1LCJ1Il0sWzQsNiwiIiwwLHsibGV2ZWwiOjIsInN0eWxlIjp7ImhlYWQiOnsibmFtZSI6Im5vbmUifX19XSxbNSw3LCIiLDIseyJsZXZlbCI6Miwic3R5bGUiOnsiaGVhZCI6eyJuYW1lIjoibm9uZSJ9fX1dLFs2LDcsInUiLDJdLFs4LDksInEiLDJdLFsxMCwxMSwicSJdLFs4LDEwLCIiLDEseyJsZXZlbCI6Miwic3R5bGUiOnsiaGVhZCI6eyJuYW1lIjoibm9uZSJ9fX1dLFs5LDExLCIiLDEseyJsZXZlbCI6Miwic3R5bGUiOnsiaGVhZCI6eyJuYW1lIjoibm9uZSJ9fX1dLFsxNCwxNywiR157SSBcXHggZV5zfSIsMCx7InNob3J0ZW4iOnsic291cmNlIjozMCwidGFyZ2V0IjozMH0sInN0eWxlIjp7InRhaWwiOnsibmFtZSI6Im1hcHMgdG8ifX19XSxbMTIsMjEsIkdee2VecyBcXHggSX0iLDIseyJzaG9ydGVuIjp7InNvdXJjZSI6MzAsInRhcmdldCI6MzB9LCJzdHlsZSI6eyJ0YWlsIjp7Im5hbWUiOiJtYXBzIHRvIn19fV1d
\[\begin{tikzcd}
    \bullet & \bullet & \bullet & \bullet & \bullet & \bullet \\
    \bullet & \bullet & \bullet & \bullet & \bullet & \bullet
    \arrow[""{name=0, anchor=center, inner sep=0}, "q"', from=1-3, to=2-3]
    \arrow["u", from=1-3, to=1-4]
    \arrow[""{name=1, anchor=center, inner sep=0}, "v", from=1-4, to=2-4]
    \arrow["w"', from=2-3, to=2-4]
    \arrow["u", from=1-5, to=1-6]
    \arrow[""{name=2, anchor=center, inner sep=0}, Rightarrow, no head, from=1-5, to=2-5]
    \arrow[Rightarrow, no head, from=1-6, to=2-6]
    \arrow["u"', from=2-5, to=2-6]
    \arrow["q"', from=1-1, to=2-1]
    \arrow[""{name=3, anchor=center, inner sep=0}, "q", from=1-2, to=2-2]
    \arrow[Rightarrow, no head, from=1-1, to=1-2]
    \arrow[Rightarrow, no head, from=2-1, to=2-2]
    \arrow["{G^{I \x e[s]}}", shorten <=10pt, shorten >=10pt, Rightarrow, maps to, from=1, to=2]
    \arrow["{G^{e[s] \x I}}"', shorten <=10pt, shorten >=10pt, Rightarrow, maps to, from=0, to=3]
\end{tikzcd}\]
Also note that a commuting square is equalized by these two maps if and only if $u = q = id$, which forces $w = v$; these commuting squares are exactly the image of $G^m$.

The semigroup structure on $I$ represents the map sending
% https://q.uiver.app/?q=WzAsNixbMCwxLCJYIl0sWzAsMCwiWSJdLFsxLDAsIlgiXSxbMiwwLCJZIl0sWzIsMSwiWCJdLFsxLDEsIlgiXSxbMCwxLCJ1IiwxXSxbMiwzLCJ1IiwxXSxbNCwzLCJ1IiwxXSxbMiw1LCIiLDAseyJsZXZlbCI6Miwic3R5bGUiOnsiaGVhZCI6eyJuYW1lIjoibm9uZSJ9fX1dLFs1LDQsIiIsMCx7ImxldmVsIjoyLCJzdHlsZSI6eyJoZWFkIjp7Im5hbWUiOiJub25lIn19fV0sWzYsOSwiIiwwLHsic2hvcnRlbiI6eyJzb3VyY2UiOjQwLCJ0YXJnZXQiOjMwfSwic3R5bGUiOnsidGFpbCI6eyJuYW1lIjoibWFwcyB0byJ9fX1dXQ==
\[\begin{tikzcd}
    Y & X & Y \\
    X & X & X
    \arrow[""{name=0, anchor=center, inner sep=0}, "u"{description}, from=2-1, to=1-1]
    \arrow["u"{description}, from=1-2, to=1-3]
    \arrow["u"{description}, from=2-3, to=1-3]
    \arrow[""{name=1, anchor=center, inner sep=0}, Rightarrow, no head, from=1-2, to=2-2]
    \arrow[Rightarrow, no head, from=2-2, to=2-3]
    \arrow[shorten <=13pt, shorten >=10pt, Rightarrow, maps to, from=0, to=1]
\end{tikzcd}\]
% Multiplication by the zero $id_0$ corresponds to the map sending a morphism to the identity on its source.
% \[
%   G \xrightarrow[]{s} M \xrightarrow[]{e} G   
% \]
% This makes the morphisms defining the Lie derivative of a groupoid representable:
% % https://q.uiver.app/?q=WzAsNyxbMSwwLCJURyJdLFsyLDAsIlRHIl0sWzAsMCwiQSJdLFszLDBdLFswLDEsIkEiXSxbMSwxLCJcXG1hdGhzZntHcGR9KFxcdykoRCBcXHggSSwgRykiXSxbMiwxLCJcXG1hdGhzZntHcGR9KFxcdykoRCBcXHggSSwgRykiXSxbMCwxLCJULihpIFxcbyBzKSIsMCx7Im9mZnNldCI6LTF9XSxbMCwxLCIoMCBcXG8gcCkuRyIsMix7Im9mZnNldCI6MX1dLFsyLDBdLFs1LDYsIihEIFxceCAoXFxpb3RhIFxcbyBzKSleKiIsMCx7Im9mZnNldCI6LTF9XSxbNSw2LCIoKFxceGkgXFxvICEpIFxceCBJKV4qIiwyLHsib2Zmc2V0IjoxfV0sWzQsNV1d
% \[\begin{tikzcd}
%   A & TG & TG & {} \\
%   A & {\mathsf{Gpd}(\w)(D \x I, G)} & {\mathsf{Gpd}(\w)(D \x I, G)}
%   \arrow["{T.(i \o s)}", shift left=1, from=1-2, to=1-3]
%   \arrow["{(0 \o p).G}"', shift right=1, from=1-2, to=1-3]
%   \arrow[from=1-1, to=1-2]
%   \arrow["{(D \x (\iota \o s))^*}", shift left=1, from=2-2, to=2-3]
%   \arrow["{((\xi \o !) \x I)^*}"', shift right=1, from=2-2, to=2-3]
%   \arrow[from=2-1, to=2-2]
% \end{tikzcd}\]
% The category $\mathsf{Gpd}(\w)$ is cocomplete, so it is possible to represent this functor.
Now look at the $\partial$ defined in Proposition \ref{prop:lin-approx-gph}, and take the same colimit in $\mathsf{Gpd}(\w)$. 
Groupoids are monadic over reflexive graphs with an involution, so this is essentially the free groupoid over that graph (as the free functor is a left adjoint and therefore preserves colimits).
\begin{definition}%
    \label{def:partial-gpd}
    Define the $\w$-groupoid $\partial$ to be the $\w$-coequalizer
    % https://q.uiver.app/?q=WzAsMyxbMCwwLCJEXFx4IEkiXSxbMSwwLCJEXFx4IEkiXSxbMiwwLCJcXHBhcnRpYWwiXSxbMCwxLCJlIFxceCBJIiwwLHsib2Zmc2V0IjotMX1dLFswLDEsIkQgXFx4IGVecyIsMix7Im9mZnNldCI6MX1dLFsxLDJdXQ==
    \[\begin{tikzcd}
        {D\x I} & {D\x I} & \partial
        \arrow["{e \x I}", shift left=1, from=1-1, to=1-2]
        \arrow["{D \x e[s]}"', shift right=1, from=1-1, to=1-2]
        \arrow[from=1-2, to=1-3]
    \end{tikzcd}\]
    where $e = 0 \o !, e[s] = i \o s$. Note that just as in Proposition \ref{prop:lin-approx-gph}, $\partial$ is an anchored bundle.
\end{definition}

The properties of the arrow groupoid make it possible to prove the following theorem.

\begin{theorem}%
    \label{thm:inf-nerve}
    The object $\partial$ determines a cartesian $\w$-functor
    \[
        \partial(-): \wone^{op} \to \mathsf{Gpd}(\w)    
    \]
    that is both an infinitesimal object in $\w$-groupoids and an involution algebroid in $\mathsf{Gpd}(\w)^{op}$.
\end{theorem}
\begin{proof}
    Write the pushout powers of $\xi:1 \to \partial$ as $\partial(n)$, and for any Weil algebra $V = W_{n(1)} \ox \dots \ox W_{n(k)}$, write $\partial(V) = \partial(n_1) \x \dots \x \partial(n_k)$. Composition for internal categories will be written in the diagramattic order, writing composition as an infix semicolon ``;'' to keep it distinct from composition of morphisms in $\w$. 
    
    The proof has two main steps:
    \begin{enumerate}
        \item For every Weil algebra $V$, there is an isomorphism
        \[
            \partial_V \cong \partial(V)    
        \]
        where $\partial_V$ is the $V$-prolongation of the anchored bundle in $\mathsf{Gpd}(\w)^{op}$.
        \item The universal lift for the anchored bundle, given as a coequalizer
        % https://q.uiver.app/?q=WzAsMyxbMCwwLCJcXHBhcnRpYWwgXFx4IFxccGFydGlhbCAiXSxbMSwwLCJcXHBhcnRpYWxcXHggXFxwYXJ0aWFsIl0sWzIsMCwiXFxwYXJ0aWFsIl0sWzAsMSwiZV5cXHBhcnRpYWwgXFx4IFxccGFydGlhbCIsMCx7Im9mZnNldCI6LTJ9XSxbMCwxLCJlXlxccGFydGlhbCBcXHggXFxwYXJ0aWFsICIsMix7Im9mZnNldCI6Mn1dLFsxLDIsIlxcYm94ZG90Il1d
        \[\begin{tikzcd}
            {\partial \x \partial } & {\partial\x \partial} & \partial
            \arrow["{e^\partial \x \partial}", shift left=2, from=1-1, to=1-2]
            \arrow["{e^\partial \x \partial }"', shift right=2, from=1-1, to=1-2]
            \arrow["\boxdot", from=1-2, to=1-3]
        \end{tikzcd}\]
        is a semigroup.
    \end{enumerate}
    From (1) and (2), we may infer that the object $\partial$ is an infinitesimal object. The semigroup map $\boxdot$ is commutative and has a zero by (2), and satisfies all of the couniversality axioms by (1). 
    \begin{enumerate}
        \item   The proof of this step follows by induction on the width (Definition \ref{def:truncated-wone}) of the Weil algebra $V$, where the cases $0, 1$ both hold by definition.
        In the case $n=2$, we need only check this holds for $\partial \x \partial \cong \partial_{WW}$:
        %   For $L_2$ and $\partial \x \partial$, first note there is the injection:
        \[\input{TikzDrawings/Ch5/induce-map-from-coprol.tikz}\]
        For any $G$, a map $X: \partial \x \partial \to G$ corresponds to a commuting square in $T^2G$ of the form
        \[\input{TikzDrawings/Ch5/comm-square-T2G.tikz}\]
        so that $T.e \o v = id$ and $e.T \o u = id$.
        It follows that $u = (e.T \o v)^{-1};(T.e \o u);v$ (Lemma \ref{lem:arrow-gpd-facts} (i)), and that precomposition with the uniquely induced map determines $(\bar{u},v):X \to G^{L(2)}$, where
        \[ T.0 \o \bar{u} = u.\]
        Observe that any pair $(\bar{u}, v):X \to  G^{L(2)}$ determines a square
        \[
            \input{TikzDrawings/Ch5/inverse-of-prol.tikz}
        \]
        where $T.e \o v = id$ and $T.e \o \bar{u} = id$.
        Note that $T.0 \o T.p \o T.0 \o \bar{u} = T.0 \o \bar{u}$, and that the bottom horn is $u = (e.T \o v)^{-1};(T.0 \o \bar{u});v$; now check
        \begin{align*}
            T.e \o u
            % &= T.e \o \((e.T \o v)^{-1};(T.0 \o \bar{u});v \) \\
                & = (T.e \o e.T \o v)^{-1};(T.e \o T.0 \o \bar{u});(T.e \o v) \\
                & = id; T.0 \o \bar{u}; T.e \o v = T.0 \o \bar{u}
        \end{align*}
        and
        \begin{align*}
            e.T \o u
                & = e.T \o (e.T \o v)^{-1};(T.0 \o \bar{u});v                 \\
                & = (e.T \o e.T \o v)^{-1};(e.T \o T.0 \o \bar{u});(e.T \o v) \\
                & = (e.T \o v)^{-1};id;(e.T \o v) = id
        \end{align*}
        thus determining a map $X \to G^{\partial \x \partial}$.
        The two maps are inverse to each other, giving an isomorphism.
    
        For the inductive case, look at the prolongations of anchored bundles and recall that
        \[
            A_{UV} \boxtimes_M A_Z  =
            A_{UV} \boxtimes_{\prol(U,A)} A_{UZ}
        \]
        Where $A_{UV}$ and $A_{UZ}$ are treated as spans over $A_U$, so that
        \begin{gather*}
            A_U \xleftarrow[]{id \boxtimes \pi^V} A_{UV} \xrightarrow[]{anc^U \boxtimes A_V} T^U.A_V\\
            A_U \xleftarrow[]{id \boxtimes \pi^Z} A_{UZ} \xrightarrow[]{anc^U \boxtimes A_Z} T^U.A_Z
        \end{gather*}
        and observe that their span composition is
        \[
            \input{TikzDrawings/Ch5/pullback-uvz.tikz}
        \]
        % Now, assume that the hypothesis holds for any two of $U, V, Z \in \wone$ and use the co-continuity of $X \x (-)$ in a cartesian closed category.
        % \begin{align*}
        %   L_{UVZ} & = L_{UV} \boxplus_U L_{UZ}
        %   % \\
        %   % &= (\partial(U) \x \partial(V)) \po_{(id)}{\pi_0} (D(V) \x \partial(U) \x \partial(Z)) \\
        %   % &= \partial(U) \x (\partial(U) \po_{}{} \partial(Z)) \\
        %   % &= \partial(V) \x \partial(U)\x \partial(Z)
        % \end{align*}
        So it now suffices to prove that the diagram
        \[\input{TikzDrawings/Ch5/partial-u-po.tikz}\]
        is a pushout. But by the inductive hypothesis,
        \[\input{TikzDrawings/Ch5/partial-no-u-po.tikz}\]
        so the result follows by the cocontinuity of $\partial_U \x (-)$, and $\mathsf{Gpd}(\w)$ is a cartesian closed category (so $X \x (-)$ is cocontinuous).
        \item Recall that the fork
        % https://q.uiver.app/?q=WzAsMyxbMCwwLCJJIFxceCBJIl0sWzEsMCwiSSBcXHggSSJdLFsyLDAsIkkiXSxbMSwyLCJtIl0sWzAsMSwiZV5zIFxceCBJIiwwLHsib2Zmc2V0IjotMX1dLFswLDEsIkkgXFx4IGVecyIsMix7Im9mZnNldCI6MX1dXQ==
        \[\begin{tikzcd}
            {I \x I} & {I \x I} & I
            \arrow["m", from=1-2, to=1-3]
            \arrow["{e[s] \x I}", shift left=1, from=1-1, to=1-2]
            \arrow["{I \x e[s]}"', shift right=1, from=1-1, to=1-2]
        \end{tikzcd}\]
        is a coequalizer. 
        Also note that $e^\partial$ is the coequalizer of $e, e[s]$. The commutativity of colimits then ensures that there is a multiplication map induced by the following diagram:
        % https://q.uiver.app/?q=WzAsOSxbMCwwLCIoRCBcXHggSSleMiJdLFsxLDAsIihEIFxceCBJKV4yIl0sWzIsMCwiXFxwYXJ0aWFsXjIiXSxbMCwxLCIoRCBcXHggSSleMiJdLFsxLDEsIihEIFxceCBJKV4yIl0sWzIsMSwiXFxwYXJ0aWFsXjIiXSxbMiwyLCJcXHBhcnRpYWwiXSxbMSwyLCJEXFx4IEkiXSxbMCwyLCJEXFx4IEkiXSxbMCwxLCIiLDAseyJvZmZzZXQiOi0xfV0sWzAsMSwiIiwyLHsib2Zmc2V0IjoxfV0sWzEsMl0sWzAsMywiIiwyLHsib2Zmc2V0IjoxfV0sWzAsMywiIiwyLHsib2Zmc2V0IjotMX1dLFsxLDQsIiIsMix7Im9mZnNldCI6MX1dLFsxLDQsIiIsMix7Im9mZnNldCI6LTF9XSxbMiw1LCIiLDIseyJvZmZzZXQiOjF9XSxbMiw1LCIiLDIseyJvZmZzZXQiOi0xfV0sWzMsNCwiIiwxLHsib2Zmc2V0IjotMX1dLFszLDQsIiIsMSx7Im9mZnNldCI6MX1dLFs0LDVdLFs1LDYsIiIsMSx7InN0eWxlIjp7ImJvZHkiOnsibmFtZSI6ImRhc2hlZCJ9fX1dLFs0LDddLFs3LDZdLFszLDhdLFs4LDcsIiIsMSx7Im9mZnNldCI6MX1dLFs4LDcsIiIsMSx7Im9mZnNldCI6LTF9XV0=
        \[\begin{tikzcd}
            {(D \x I)^2} & {(D \x I)^2} & {\partial^2} \\
            {(D \x I)^2} & {(D \x I)^2} & {\partial^2} \\
            {D\x I} & {D\x I} & \partial
            \arrow[shift left=1, from=1-1, to=1-2]
            \arrow[shift right=1, from=1-1, to=1-2]
            \arrow[from=1-2, to=1-3]
            \arrow[shift right=1, from=1-1, to=2-1]
            \arrow[shift left=1, from=1-1, to=2-1]
            \arrow[shift right=1, from=1-2, to=2-2]
            \arrow[shift left=1, from=1-2, to=2-2]
            \arrow[shift right=1, from=1-3, to=2-3]
            \arrow[shift left=1, from=1-3, to=2-3]
            \arrow[shift left=1, from=2-1, to=2-2]
            \arrow[shift right=1, from=2-1, to=2-2]
            \arrow[from=2-2, to=2-3]
            \arrow[dashed, from=2-3, to=3-3]
            \arrow[from=2-2, to=3-2]
            \arrow[from=3-2, to=3-3]
            \arrow[from=2-1, to=3-1]
            \arrow[shift right=1, from=3-1, to=3-2]
            \arrow[shift left=1, from=3-1, to=3-2]
        \end{tikzcd}\]
        Thus the multplication is associative, is commutative, and has a zero given by $1 \xrightarrow[]{0 \x s} D \x I$.
    \end{enumerate}
\end{proof}

Aan immediate corollary of Theorem \ref{thm:inf-nerve} is that the $\partial$ determines a nerve/ realization where the nerve factors through the category of involution algebroids.
This puts the Lie functor into a nerve/realization context (recall Definition \ref{def:realization}).
\begin{definition}%
    \label{def:lie-realization}
    The \emph{Lie realization} is the left Kan extension
    \[
        |-|_\partial = Lan_\partial: \mathsf{Inv}(\w) \to \mathsf{Gpd}(\w).  
    \]
\end{definition}

This functor is well behaved. First, ote that it preserved products:
\begin{lemma}
    The realization functor preserves products.
\end{lemma}
\begin{proof}
    Product preservation is a consequence of $|-|_\partial$ being the left Kan extension of a cartesian functor along a cartesian functor (\cite{Day1995}).
    Note that this implies $\int^v\partial(v) = 1$. 
\end{proof}
Next, we see that the realization of an involution algebroid has the same base space.

\begin{lemma}\label{lem:base-of-groupoid}
    The base space of the groupoid $\partial$ is $D^v$.
\end{lemma}
\begin{proof}
    When constructing the colimit of groupoids, the colimit's base space is the ordinary colimit for the diagram of the base spaces. The reflector from simplicial objects to groupoids preserves products, so it suffices to check that the base space of $\partial(n)$ is $D(n)$.

    The base space of $I$ is $1+1$, and the map $e^s_0$ is given by $\delta^- \o !$.
    Since $D$ is a discrete cubical object, its base space is $D \x 1$.
    Thus the coequalizer defining $\partial$ is
    \[\input{TikzDrawings/Ch5/coeq-def-partial.tikz}.\]
    A map $\gamma: D + D \to M$ is a pair of maps $\gamma_0, \gamma_1: D \to M$.
    We can see that $\gamma_0$ and $\gamma_1$ agree at $0$ (they are both $\gamma_0(0)$), and $\gamma(0)$ is a constant tangent vector. It follows that $\gamma_1 \o (id | id) = \gamma$.
\end{proof}

Now recall the co-Yoneda lemma: for any $\vv$-presheaf $A: \C^{op} \to \vv$,
\[
    F(C) = \int^{C' \in \C} \C(C,C') \ox F(C').
\]
In particular, for an involution algebroid in $\w$ (or any $\w$-presheaf on $\wone^{op}$),
\[
    A(U) = \int^{V \in \wone} \wone^{op}(U,V) \x A(V) = \int^{V \in \wone} \wone(V,U) \x A(V).
\]
\begin{lemma}\label{lem:as-a-presheaf}
    For any involution algebroid $A$,
    \[ A(R) = \int^{V \in \wone} A(V) \x D^V. \]
\end{lemma}
\begin{proof}
    Use the tangent structure on $\wone$ to regard it as a $\w$-enriched category:
    \begin{gather*}
        D^V = \yon(V) = \wone(V,-) = (U \mapsto \wone(V, U))  \\= (U \mapsto \wone(V, U\ox R)) = \wone(V,R).
    \end{gather*}
    The following computation gives the result:
    \[
        A(R) = \int^{V} \wone(V,R) \x A(V) = \int^{V} D^V \x A(V).
    \]
\end{proof}
We can now see that the base space of an involution algebroid is isomorphic to the base space of its realization. That is to say, the realization sends an involution algebroid over a Weil space $M$ to a groupoid over the Weil space $M$ (up to isomorphism).

\begin{proposition}\label{prop:lie-int-first-part}
    Let $A$ be an involution algebroid in $\w$. Then $|A|([0]) = A(R)$.
\end{proposition}
\begin{proof}
    Use the Yoneda lemma, and the fact that $\yon[0] = 1$ is a small projective so that $\mathsf{Gpd}(1,-)$ is $\w$-cocontinuous. Now apply Lemma \ref{lem:base-of-groupoid} and Lemma \ref{lem:as-a-presheaf}:
    \begin{align*}
        |A|([0]) & = \mathsf{Gpd}(1, |A|)                                                   \\
                 & = \mathsf{Gpd}\left(1, \int^{v \in \wone} A(v)\bullet \partial v \right) \\
                 & = \int^{v} A(v)\x \mathsf{Gpd}(1,  \partial v )                          \\
                 & = \int^{v}  A(v)\x D^v = A(R).
    \end{align*}
\end{proof}

Thus, as a final result, we have achieved an adjunction between the category of involution algebroids and groupoids in $\w$ that is product-preserving and stable over the base spaces.
\pagebreak 
\lie
\[% https://q.uiver.app/?q=WzAsMixbMCwwLCJcXG1hdGhzZntHcGR9KFxcQykiXSxbMSwwLCJcXG1hdGhzZntJbnZ9KFxcQykiXSxbMCwxLCJOX1xccGFydGlhbCIsMCx7ImN1cnZlIjotMn1dLFsxLDAsInwtfF9cXHBhcnRpYWwiLDAseyJjdXJ2ZSI6LTJ9XSxbMiwzLCIiLDAseyJsZXZlbCI6MSwic3R5bGUiOnsibmFtZSI6ImFkanVuY3Rpb24ifX1dXQ==
\begin{tikzcd}
    {\mathsf{Gpd}(\w)} & {\mathsf{Inv}(\w)}
    \arrow[""{name=0, anchor=center, inner sep=0}, "{N_\partial}", curve={height=-12pt}, from=1-1, to=1-2]
    \arrow[""{name=1, anchor=center, inner sep=0}, "{|-|_\partial}", curve={height=-12pt}, from=1-2, to=1-1]
    \arrow["\dashv"{anchor=center, rotate=-90}, draw=none, from=0, to=1]
\end{tikzcd}\]

\begin{remark}
    Just as the introduction to this thesis begins with the work of Charles Ehresmann, we should take a moment to see how the Lie realization relates to his original research into sketch theory. Rather than sketches, we use their closely related cousins, \emph{essentially algebraic theories}, which in this case are small, finitely complete $\w$-categories. We may regard $\mathsf{Gpd}(\w)$ and $\mathsf{Inv}(\w)$ as $\w$-functor categories
    \[
        \mathsf{Lex}(\th_{\mathsf{Gpd}}, \w), \hspace{0.5cm}\mathsf{Inv}(\th_{\mathsf{Gpd}}, \w),
    \]
    respectively (where $\mathsf{Lex}$ means finite-limit-preserving $\w$-functors). The functor $\partial:\wone^{op} \to \mathsf{Gpd}(\w)$ induces a left-exact $\w$-functor
    \[
        \hat{\partial}: \th_{\mathsf{Inv}} \to \th_{\mathsf{Gpd}}.
    \]
    This means the functor from Lie groupoids to Lie algebroids is induced by a morphisms of essentially algebraic theories, and may thus be presented as a morphism of sketches.  
\end{remark}
