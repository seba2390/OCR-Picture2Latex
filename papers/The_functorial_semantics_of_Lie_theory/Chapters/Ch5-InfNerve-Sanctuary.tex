\chapter{The infinitesimal nerve and its realization}%
\label{ch:inf-nerve-and-realization}




\section{Tangent categories via enrichment}%
\label{sec:tang-cats-enrichment}

This section demonstrates how tangent categories fit inside the framework of enriched categories. This connection  was first made by \cite{Garner2018}, building on the actegory perspective on tangent categories introduced in \cite{Leung2017} and a result relating monoidal actions and enrichment from \cite{Wood1978}. Garner was able to exhibit some of the major results from synthetic differential geometry as pieces of enriched category theory, for example the Yoneda lemma implies the existence of a well-adapted model of synthetic differential geometry.

The category of Weil spaces is the site of enrichment for tangent categories and are closely related to Dubuc's \emph{Weil topos} from his original work on models of synthetic differential geometry \cite{Dubuc1981}, and a deeper study of the topos may be found in \cite{Bertram2014}. Recall that the category $\wone$ is the free tangent category over a single object. The category of Weil spaces is the \emph{cofree} tangent category over $\s$, which is the category of transverse-limit-preserving functors $\wone \to \s$ by Observation \ref{obs:cofree-tangent-cat}. Call this the category of \emph{Weil spaces}, and write it $\w$. Just as a simplicial set $S: \delta \to \s$ is a gadget recording homotopical data, a Weil space is a gadget recording \emph{infinitesimal} data.

\begin{definition}%
	\label{def:weil-space}
	A \emph{Weil space} is a functor $\wone \to \s$ that preserves transverse limits - that is, the $\ox$-closure of the set of limits:
	\[
		\left\{
		\input{TikzDrawings/Ch5/wone-pb.tikz}
		,
		\input{TikzDrawings/Ch5/wone-univ-lift.tikz}
		,
		\input{TikzDrawings/Ch5/wone-id.tikz}
		\right\}
	\]
	A morphism of Weil spaces is a natural transformation. The category of Weil spaces is written $\w$.
\end{definition}
\begin{example}
	~\begin{enumerate}[(i)]
		\item Every commutative monoid may be regarded as a Weil space in a canonical way.
		      Observe that for every $V \in \wone$ and commutative monoid $M$, there is a free $V$-module structure on $M$ given by $|V| \ox_{\mathsf{CMon}} M$. The underlying commutative monoid of $V$ is exactly $\N^{\mathsf{\dim V}}$, so that:
		      \[
			      V \bullet M \cong \oplus^{\dim V} M
		      \]
		      The microlinearity of $V \mapsto |V| \ox M$ follows from fundamental properties of finite biproducts.
		\item Following (i), any tangent category $\C$ that is \emph{concrete} - it admits a faithful functor $U_\s: \C \to \s$ - will have a natural functor into tangent sets (copresheaves on $\wone$). Every object $A$ will have an underlying tangent set $V \mapsto U_{\s}(T^V(A))$, and whenever $U_\s$ preserves connected limits (such as the forgetful functor from commutative monoids to sets), each of the underlying tangent sets will be a Weil space.
		\item Consider a symmetric monoidal category with an infinitesimal object, which by Corollary \ref{cor:rewrite-rep-functor} is a transverse colimit preserving symmetric monoidal functor $D:\wone \to \C$. Then for every object, the nerve (Definition \ref{def:nerve}) $N_D(X): \C(D-,X):\wone \to \s$ is a Weil space.
		\item For any pair of objects $A,B$ in a tangent category, $\C(A,T^{(-)}B):\wone \to \s$ is a Weil space by the continuity of $\C(B,-):\C \to \s$.
	\end{enumerate}
\end{example}
Unlike the category of simplicial sets, the category of Weil spaces is not a topos.
The category of Weil spaces does, however, inherit some nice properties from the topos of copresheaves on $\wone$ by applying results from \Cref{sec:enriched-nerve-constructions}.
\begin{proposition}[\cite{Garner2018}]
	The category of Weil spaces is a cartesian-monoidal reflective subcategory of $[\wone,\s]$.
\end{proposition}
\begin{corollary}
	The category of Weil spaces is:
	\begin{enumerate}[(i)]
		\item A cartesian closed category.
		\item Locally finitely presentable as a cartesian monoidal category (as in Definition \ref{def:v-lfp})\footnote{
			      That is to say, the category of Weil spaces is a co-complete category with a dense subcategory $\yon:\wone \to \w$ (see Definition \ref{def:dense-subcat}).
		      }.
		\item A representable tangent category, where the infinitesimal object is given by the restricted Yoneda embedding $\yon: \wone^{op} \to \w$.
	\end{enumerate}
\end{corollary}
The cofree tangent structure on $\w$ is given by precomposition, that is:
\[
	T^U.M.(V) = M.(U \ox V) = M.T^U.V
\]
This tangent coincides with the representable tangent structure induced by the Yoneda embedding. The proof is an application of the Yoneda lemma. Observe:
\[
	[D,M](V) \cong [\wone,\s](D \x \yon(V), M) \cong [\wone, \s](\yon(WV), M) \cong M(WV)
\]
where $D \x \yon(V) = \yon(WV)$ as the tensor product in $\wone$ is cocartesian and the reflector is cartesian monoidal.

% Thus, we set the following piece of notation/terminology:
% \begin{notation}
% 	For a structure in a tangent category (e.g. a differential object, involution algebroid, etc), we call a model of $X$ in $\w$ a tangent $X$.
% \end{notation}

% % \paragraph{The enrichment of a tangent category in $\w$}%
% % \label{sub:enrichment-of-tang-cat}
% The basic idea in enriched category theory is that the definition of a category involves very little about the category of sets.
% Thus, we can define for any monoidal $\vv^\ox$ the notion of a $\vv$-category.


In general, any actegory over a monoidal $\C$ will give rise to $[\C,\s]$-enriched category, using the tensor structure induced by Day convolution, was proved by \cite{Wood1978}. \cite{Garner2018} showed that a monoidal reflective subcategory $\vv \hookrightarrow \hat \a$ exhibits the 2-category of $\vv$-categories as a reflective sub-2-category of $\a$-categories - this allow for tangent categories to be regarded as a particular class of enriched category.
\begin{proposition}[\cite{Garner2018}]
	Every tangent structure $(\C, \T)$ equips $\C$ with enrichment in Weil spaces.
\end{proposition}
\begin{proof}
	The presheaf is constructed:
	\[
		\underline{\C}(A,B) = \C(A, T^{}B): \wone \to \s
	\]
	The functor $\C(A,-)$ is continuous and $T^{-}B$ is a microlinear complex, so this is a Weil space. The following diagram gives the composition map :
	\input{TikzDrawings/Ch5/enrichment-in-W-comp.tikz}
	Note that it is natural in $U, V$.
\end{proof}
Enriched category theory introduces a new family of limits, \emph{powers}\footnote{Recall that the entire class of weighted limits is equivalent to conical (ordinary) limits and powers.}. Powers are essential in tangent category theory and satisfy a coherence due to \cite{LucyshynWright2016}.
\begin{definition}\label{def:power}
	Let $\C$ be a $\vv$-category for $\vv$ a closed symmetric monoidal category. For $J \in \vv$, the \emph{power} by $J$ for an object $C \in \C$ is an object $C^J$ so that the following is an isomorphism:
	\[
		\forall D: \vv(J, \C(D,C)) \cong \C(D,C^J)
	\]
	whereas the \emph{copower} is given by:
	\[
		\forall D: \vv(J, \C(C,D)) \cong \C(J \cdot C, D)
	\]
	Let $\j \hookleftarrow \vv$ be a full monoidal subcategory of $\vv$. A $\vv$-category $\C$ has \emph{coherently chosen powers} by $\j$ if there is a choice of $\j$-powers $(-)^J$ so that:
	\[
		(C^J)^K = C^{J \ox K}
	\]
	Moreover, coherently chosen copowers are a choice of $\j$-copowers so that:
	\[
		K \bullet (J \bullet C) = (K \ox J) \bullet C
	\]
	The sub-2-category of $\vv$-categories equipped with coherently chosen powers and copowers are $\vv\cat^\j$ and $\vv\cat_\j$, respectively.
\end{definition}

Now, returning to the specific enrichment in $\w$, remember that the hom-object between $A, B$ is given by:
\[
	V \mapsto \C(A,T^V B)
\]
For a representable functor $\yon U$, use the fact that the representable tangent structure induced by $\yon$ coincides with the cofree tangent structure and find:
\[
	\w(\yon U, V \mapsto \C(A, T^VB)) \cong V \mapsto \C(A, T^V.T^U(B))
\]
Then, a tangent category (and any actegory, really) does not simply have enrichment in $\w$; it also has coherently chosen powers by representable functors\footnote{The terms ``representable'' and ``representable functor'' will be used interchangeably in this thesis.}. $\w$-categories with coherently chosen powers by representables are \emph{exactly} tangent categories. Thus we have the following:
\begin{proposition}[\cite{Garner2018}]
	A tangent category is exactly an $\w$-category with coherently chosen powers by representables.
\end{proposition}
Now the original notions of (lax, strong, strict) tangent functors can be defined as $\w$-functors between $\w$-categories with coherently chosen powers, where the lax/strong/strict adjective now relates to the preservation of the coherently chosen powers.
\begin{theorem}[\cite{Garner2018}]\label{thm:2-cat-tangcat}
	We have the following equivalences of 2-categories.
	\begin{enumerate}[(i)]
		\item The 2-category of $\w$-categories with coherently chosen powers and $\mathsf{TangCat}_{\mathsf{Lax}}$
		\item The 2-category of $\w$-categories with coherently chosen powers and power-preserving $\w$ functors  and $\mathsf{TangCat}_{\mathsf{Strong}}$
		\item The 2-category of $\w$-categories with coherently chosen powers and chosen-power-preserving $\w$ functors $\mathsf{TangCat}_{\mathsf{Strict}}$
	\end{enumerate}
\end{theorem}
Note that the for a lax tangent functor $(F,\alpha): \C \to \D$, the map
\[
	\alpha.X: F.T.X \to T.F.X
\] can be seen as the unique morphism induced by universality. Conversely, for a strong tangent functor, the natural isomorphism $\alpha$ is the isomorphism from the power $F.T.X$ to the \emph{coherently chosen} power $T.F.X$, while a strict tangent functor preserves the coherent choice of powers.

The following observation is somewhat dual \Cref*{thm:2-cat-tangcat}, taking copowers by representables as the basic structure. It would be an unnecessary digression to develop the entire theory; however, it will be helpful to relate representable tangent structures.
\begin{observation}\label{obs:second-enrichment}
	Regard a cartesian category with an infinitesimal object $D$ (Definition \ref{def:inf-object}) as a $\w$-category, where copowers induce the enrichment (use that $D$ determines a functor $D:\wone \to \C$)
	\[
		\C(A,B):\wone \to \s := \underline{\C}(A \x D(-), B)
	\]
	Note, then, that a second infinitesimal object in $\C$, this time a monoidal $F: \wone^{op} \to \C$, may be regarded as an enriched functor \emph{for $D$'s enrichment} if there is monoidal natural transformation:
	\[
		d: D \Rightarrow F
	\]
	This natural transformation allows for the construction of the following map in $\w$ that is natural for $X \in \wone$:
	\[
		(X \mapsto \wone^{op}(U \x X,V)) \xrightarrow[]{F_{U\x X,V}} (X \mapsto \C(F.U\x F.X, F.V)) \xrightarrow[]{(id \x d)^*} (X \mapsto \C(F.U\x D.X, F.V))
	\]
	This exhibits $F$ as a $\w$-functor.
\end{observation}

As a final remark, note that the Yoneda lemma applies of $\w$-categories, so there is an embedding:
\[
	\C \hookrightarrow \widehat{C} = [\C^{op},\s]
\]
The powers and copowers by representables are computed pointwise in a presheaf category, so they inherit the coherent choice. Thus the following holds:
\begin{corollary}[\cite{Garner2018}]
	Every tangent category embeds into a $\w$-cocomplete representable tangent category.
\end{corollary}
(In fact, this result is the main theorem of \cite{Garner2018}).
\begin{corollary}
	The category of anchored bundles in a locally presentable tangent category is itself locally presentable, and in Weil spaces, the inclusion $\wone^* \hookrightarrow \mathsf{Anc}(\w)$ is a dense subcategory.
\end{corollary}



\subsection*{Differential bundles as enriched structures}%
\label{sub:differential-bundles-as-enriched-structures}

As a first case study using the enriched perspective for tangent categories, consider differential bundles. Most of the work in \Cref{ch:differential_bundles} uses the intuition that differential bundles were some tangent-categorical algebraic theory - this section will make that intuition concrete. Recall that a lift (Definition \ref{def:lift}) is a map $\lambda:E \to TE$. Using the enriched perspective treating $TE$ as a power (recall Definition \ref{def:power}), this gives the following correspondence:
\[
	\infer{\hat{\lambda}:D \to \C(E,E)}{\lambda: 1 \to \C(E, E^D)}
\]
The commutativity condition, then, is equivalent to asking that the following diagram commutes:
\[\input{TikzDrawings/Ch5/dbun-commutativity.tikz}\]
That is, $\hat{\lambda}$ is a semigroup morphism $D \to \C(E,E)$. In any cartesian closed category with coproducts, a semigroup may be freely lifted to a monoid using the "maybe monad" $(-) + 1$ from functional programming (see, for example, \cite{Seal2013}).
\begin{definition}%
	\label{def:lambda-monoid}
	Regard the following monoid the one-object $\w$-category $\Lambda$:
	\[
		\infer{m: (D+1) \x (D+1) \to D+1}{D \x D + D + D + 1 \xrightarrow{(\iota_L\o m | \iota_L | \iota_L |\iota_R)} D + 1}
	\]
\end{definition}
Thus, a lift $\lambda$ is exactly a functor from $\hat{\lambda}:\Lambda \to \C$. Now check that morphisms are tangent natural transformations.
\begin{lemma}%
	\label{lem:cat-of-lifts-iso}
	The category of lifts in a tangent category $\C$ is isomorphic to the category of $\w$-functors and $\w$-natural transformations $\Lambda^{op} \to \C$.
\end{lemma}
\begin{proof}
	Check that a $\w$-natural transformation is exactly morphism of lifts $f: \lambda \to l$. Start with the $\w$-naturality square:
	\[\input{TikzDrawings/Ch5/w-nat.tikz}\]
	Now, rewriting $D+1 \to \C(A,B)$ as a semigroup map $D \to \C(A,B)$, we have:
	\[
		\infer{
			\infer{
				1 \xrightarrow{Tf \o \lambda'}\C(A,TB)
			}{
				1 \xrightarrow{(\lambda',f)} \C(A,TA) \x \C(A,B) \xrightarrow{1 \x T} \C(A,TA) \x \C(TA,TB) \xrightarrow{m_{A,TA,TB}} \C(A,TB)
			}}{
			D \xrightarrow{(\lambda,f\o !)} \C(A,A) \x \C(A,B) \xrightarrow{m_{AAB}} \C(A,B)
		}
	\]
	Similarly, the other path is exactly $l \o f$. Thus, a $\w$-natural transformation is exactly a morphism of lifts.
\end{proof}
\begin{observation}
	The semigroup $D$ is commutative, so $\Lambda = \Lambda^{op}$ - the choice of using $\Lambda^{op}$ is to remain consistent with conventions used in \Cref{sec:enriched-nerve-constructions}.
\end{observation}
It is a classical result in synthetic differential geometry that the object $D$ has only one point - in the case of $\w$, this follows from the Yoneda lemma (regarding $\w$ as a $\s$-category)
\[
	\w(1,D) = \wone(\N[x]/x^2, \N) = \{ !: \N[x]/x^2 \to \N \}
\]
Note that this forces the natural idempotent $e:id \Rightarrow id$ from Proposition \ref{prop:idempotent-natural}, then, must be the point $0:1 \to D$. Note that this idempotent is an absorbing element of the monoid $D+1$, so for any $f:X \to D+1$, it follows that $m(f,0\o !) = m(0\o !, f) = 0\o !$.  A pre-differential bundle is exactly a lift with a chosen splitting of the natural idempotent $p\o\lambda$.
\begin{lemma}%
	\label{lem:splitting-of-idemp-is-lambdaplus}
	For every lift $\bar{\lambda}:\Lambda^{op} \to \C$, the natural idempotent $e = p \o \lambda$ is exactly
	\[1 \xrightarrow[]{0} D \xrightarrow[]{\iota_L} D+1 \to \C(E,E)\]
\end{lemma}
Now that the natural idempotent is a map in $\Lambda$, that idempotent splits to give the theory of a pre-differential object.
\begin{definition}%
	\label{def:lambda-plus}
	The $\w$-category $\Lambda^+$ is given by the set of objects $\{0,1\}$ with hom-Weil-spaces:
	\begin{itemize}
		\item $\Lambda^+(1,1) = D+1$, otherwise $\Lambda^+(i,j) = 1$.
		\item $m_{111} = m, m_{i11} = \iota_L \o 0 \o !, m_{i0j} = !$
	\end{itemize}
\end{definition}
% The following lemma lets simplifies the composition in $\Lambda^+$.
% \begin{lemma}
% 	Associativity of composition in $\Lambda^+$ can be written by embedding each hom-space into $D+1$
% 	$\w$-category, where we use inclusions:
% 	\begin{gather*}
% 		\Lambda^+(0,0) = 1 \xrightarrow{\iota_L \o 0 \o !} D+1  \hspace{0.25cm}
% 		\Lambda^+(0,1) = 1 \xrightarrow{\iota_L \o 0 \o !} D+1  \\
% 		\Lambda^+(1,0) = 1 \xrightarrow{\iota_L \o 0 \o !} D+1  \hspace{0.25cm}
% 		\Lambda^+(1,1) = D+1 \xrightarrow{id} D+1  
% 	\end{gather*}
%     \[\input{TikzDrawings/Ch5/assoc.tikz}\]
% \end{lemma}
Basic facts about idempotent splittings yield the following theorem:
\begin{lemma}%
	\label{lem:lambda-plus-is-pdb}
	The category of pre-differential bundles is exactly the category of $\w$-functors $(\Lambda^+)^{op} \to \C$ (that is, $\C$-valued presheaves).
\end{lemma}
($(\Lambda^+)^{op} = \Lambda^+$, so this is purely conventional).
It is straightforward to exhibit the category of differential bundles as a reflective subcategory of pre-differential bundles in $[(\Lambda^+)^{op}, \C]$ (so long as $\C$ has equalizers).
\begin{proposition}%
	\label{prop:Lambda-is-refl-subcat}
	The category of differential bundles in a tangent category $\C$ with $T$-equalizers and $T$-pullbacks is a reflective subcategory of $[(\Lambda^+)^{op}, \C]$, and is therefore a locally finitely presentable tangent category.
\end{proposition}
\begin{proof}
	The category of pre-differential bundles in $\C$ is isomorphic to $[(\Lambda^+)^{op}, \C]$. By Corollary \ref{cor:idemp-dbun}, the category of differential bundles is isomorphic to the category of algebras for a canonical idempotent monad on $\C$. The reflector sends a pre-differential bundle to the $T$-equalizer \input{TikzDrawings/Ch5/reflector.tikz}. This equalizer will always exist if $\C$ has equalizers, and pullbacks of the projection will exist if $\C$ has pullbacks, so the pullback is a differential bundle. This reflection gives a left-exact idempotent monad on $[(\Lambda^+)^{op}, \C]$ whose algebras are differential bundles.
\end{proof}
Now, in case $\C$ is a \emph{locally presentable} tangent category (such as $\w$), basic facts about locally presentable categories yields the following:
\begin{corollary}%
	\label{cor:Lambda-dense}
	If $\C$ is locally presentable, then $\mathsf{DBun}(\C)$ is a locally presentable category.
\end{corollary}
% \begin{corollary}
%     $\Lambda^+$ is a dense subcategory of differential bundles in $\w$. 
% \end{corollary}
% This corollary, the subcategory inclusion $\yon:\Lambda^+ \hookrightarrow \widehat{\Lambda^+}$ is dense in the category of differential bundles, means $\Lambda^+$ can act as the arities for a theory - this this case, the theory of anchored bundles.


\section{Enriched nerve constructions}%
\label{sec:enriched-nerve-constructions}

Nerve constructions present a powerful generalization of the Yoneda functor that sends an object $C \in \C$ to the representable presheaf $\C(-, C) \in \widehat{\C}$ (for reference material, see Chapter -- of \cite{Loregian2015}). %TODO add chapter
The Yoneda lemma states that this functor is an \emph{embedding} (that is, it is fully faithful), so no information is lost when embedding a category into its category of presheaves. 
Nerve constructions, then, move this towards an \emph{approximation} of the original category $\C$ by some subcategory $\D \hookrightarrow \C$, or more generally by some functor $K:\a \to \C$.

\begin{definition}%
	\label{def:nerve-of-a-functor}
		The \emph{nerve} of a $\vv$-functor $K: \a \to \C$ is the functor
	\[
		N_K: \C \to \widehat{\a}; \hspace{0.5cm} C \mapsto \C(K-, C)
	\]
	(where $\widehat{A}$ is the $\vv$-category of $\vv$-presheaves on small $\vv$-category $A$).
\end{definition}

\begin{example}%
	\label{ex:nerve-functors}
	~\begin{enumerate}[(i)]
		\item The nerve of the identity functor $\C = \C$ is the usual Yoneda embedding $\C \hookrightarrow \widehat{\C}$.
		\item 
		The first example of a nerve construction in the mathematical literature is the simplicial approximation of a topological space by \cite{Kan1958}.
		Recall the original construction of the simplicial nerve of a topological space $X$, where $X_n = \mathsf{Top}(\Delta_n,X)$.
		This is exactly the nerve of the functor $\bar{\Delta} \to \mathsf{Top}$ that sends $n$ to the $n$-simplex:
		\[
			\{
				x \in \R^n | \sum x_i = 1
			\}
		\]
		\item Recall that in -, a functor sending reflexive graphs to anchored bundles was constructed. This ``infinitesimal approximation'' of a reflexive graph may also be regarded as a nerve construction, first by looking at the category of reflexive graphs in $\w$, which is the presheaf category $[\mathsf{Refl}, \w]$ for the trivial category 
		\[
			blah % reflexive graph diagram
		\]
		This will be a representable tangent category, where $D$ is the disconnected graph $D = D$. By the Yoneda lemma, we have that:
		\[
			[\mathsf{Refl}, \w](\yon 1, G) = G_1,
			[\mathsf{Refl}, \w](\yon 0, G) = G_0,
			[\mathsf{Refl}, \w](D \x \yon(i), G) = T.G_i,
		\]
		So the limit diagram defining the infinitesimal approximation of a graph becomes:
		\[
			blah % TODO put the two graphs here	
		\]

		% \item The theory of locally finitely presentable categories gives a useful class of examples.
		%       Because we may represent a locally presentable $\C$ as $\mathsf{Lex}(\C_{fp}^{op}, \vv)$ (whenever $\vv$ itself is locally presentable as a closed category), we can see that the nerve of $\C_{fp} \hookrightarrow \C$ is a fully faithful functor, and the realization is the reflection from presheaves on $\C_{fp}$ to models.
		% \item In general, a tangent category $\C$ is locally finitely presentable whenever it is the reflective subcategory of $\widehat{\a}$ for some small $\w$-category $\a$ - if the representables $\a \hookrightarrow \widehat{\a}$ belong to $\C$, then the inclusion $\C \hookrightarrow \widehat{\a}$ is induced by the nerve of $\a \hookrightarrow \C$.
	\end{enumerate}
\end{example}

Recall that the simplicial localization in \cite{Kan1958} has a left adjoint, the \emph{geometric realization}, that constructs a topological space using the data of a simplicial set. 
This realization may be constructed using a left Kan extension.
\begin{definition}%
	\label{def:realization}
	Let $K: \a \to \C$ be a $\vv$-functor for a cocomplete $\C$.
	The \emph{realization} of $K$ is the left Kan extension:
	\[
		\input{TikzDrawings/Ch5/realization.tikz}
		\hspace{0.5cm}
		|C|_K := \int^A \C(KA, C) \cdot KA
	\]
	A \emph{nerve/realization context} is $\vv$-functor $K:\a \to \C$ from a small $A$ to a cocomplete $\C$.   
\end{definition}
The adjunction between simplicial sets and topological spaces fits into general categorical machinery as a nerve/realization context.
\begin{lemma}
	For every nerve/realization context $K: \a \to \C$, the realization of $K$ is left adjoint to the nerve of $K$.
\end{lemma}

% Note that in Example \ref{ex:blah}, each of the approximations above loses some data (except for the identity functor). It will be useful to identify those functors $K: \a \to \C$ whose nerve behaves like the Yoneda embedding - that is, they are fully faithful. This condition is known as \emph{density}, and is explored in Chapter 5 of \cite{Kelly2005}.

% \begin{definition}%
% 	\label{def:dense}
% 	A functor $K: \a \to \C$ is \textit{dense} whenever its nerve:
% 	\[
% 		N_K: \C \to \widehat{A}; C \mapsto \C(K-,C)	
% 	\]
% 	If $K$ is a subcategory inclusion, we call this a \emph{dense subcategory}.
% \end{definition}

% \begin{example}
% 	\begin{itemize}
% 		\item For any small category $\C$, the identity functor $\C = \C$ is dense, as is the Yoneda embedding $\yon: \C \to \widehat{\C}$ is a dense functor. 
% 		\item The category of finite sets is a dense subcategory of $\s$, it follows that the category of finite cardinals - that is, the full subcategory of sets whose objects are $[n] = \coprod_n 1$ - is a dense subcategory of $\s$.
% 		\item Recall that a monad on $\s$ is a
% 	\end{itemize}
% \end{example}
% \begin{itemize}
% 	\item Nerve constructions are a powerful generalization of the Yoneda embedding that sends an object $C$ to the presheaf $\C(-, C)$.
% 	\item The Yoneda lemma states that the functor assigning a an object to its representable presheaf is \emph{fully faithful} (i.e. an embedding) - so it completely preserves the data of the category.
% 	\item Nerve constructions, then, move towards \emph{approximations} of the original category by a small subcategory, or more generally by a functor from some small subcategory $K: \a \to \C$.
% 	\item Definition of nerve here.
% 	\item Example: (simplicial localization, sets + Nats, Signpost towards the infinitesimal approximation using reflexive graphs)
% 	\item Parenthetical remark about how it's unclear where the ``nerve'' language came from - but it seems that Kan's original ``simplicial approximation'' captures the intuition most clearly, but in the interest of connecting to the relevant literature we stick to their language.
% 	\item Next, talk about \emph{density} - this determines when a category is completely approximated by a functor.
% 	% \item A first step towards a nerve construction is to think about subcategories - for example, one consequence of the Yoneda lemma is that every presheaf may be written as the coend:
% 	% \item If one wanted to ``aprox''
% 	\item Some neat little results. 
% 	\item Example: the anchored bundle one.
% \end{itemize}



% % The nerve construction 
% % This section gives a short review the \emph{nervous theories} paradigm of \cite{Bourke2019}, along with the \emph{nerve/realization} contexts of \cite{Kan1958} (a recent exposition may be found in \cite{Loregian2015}).The common denominator here is the notion of a nerve. The ``nerve'' of an internal category and involution algebroid were discussed in \Cref{sec:weil-nerve}, although the notion of a nerve or Segal condition was not made precise.
% \begin{definition}%
% 	\label{def:nerve}
% 	The \emph{nerve} of a $\vv$-functor $K: \a \to \C$ is the functor
% 	\[
% 		N_K: \C \to \widehat{\a}; \hspace{0.5cm} C \mapsto \C(K-, C)
% 	\]
% 	(where $\widehat{A}$ is the $\vv$-category of $\vv$-presheaves on small $\vv$-category $A$).
% 	Whenever $\C$ is cocomplete, the nerve has a left adjoint called the \emph{realization}, computed by the left Kan extension/coend:
% 	\[
% 		\input{TikzDrawings/Ch5/realization.tikz}
% 		\hspace{0.5cm}
% 		|C|_K := \int^A \C(KA, C) \cdot KA
% 	\]
% 	A \emph{nerve/realization context} is $\vv$-functor $K:\a \to \C$ from a small $A$ to a cocomplete $\C$.
% \end{definition}
% % We can see that the nerve itself may be represented as a left Kan extension:
% % \begin{lemma}
% \begin{observation}
% 	The nerve functor itself is equivalently the left Kan extension of $K$ along $\yon$:
% 	\[\input{TikzDrawings/Ch5/nerve-is-lany.tikz}\]
% 	For proof, see \cite{Loregian2015}. This presentation gives the unusual property that the nerve/realization adjunction is of the form:
% 	\[
% 		Lan_{\yon}K \vvdash Lan_K\yon
% 	\]
% \end{observation}

% \begin{example}
% 	~\begin{enumerate}[(i)]
% 		\item Recall the original construction of the simplicial nerve of a topological space $X$, where $X_n = \mathsf{Top}(\Delta_n,X)$.
% 		      This is exactly the nerve of the functor $\bar{\Delta} \to \mathsf{Top}$ that sends $n$ to the $n$-simplex:
% 		      \[
% 			      \{
% 			      	x \in \R^n | \sum x_i = 1
% 			      \}
% 		      \]
% 		\item The theory of locally finitely presentable categories gives a useful class of examples.
% 		      Because we may represent a locally presentable $\C$ as $\mathsf{Lex}(\C_{fp}^{op}, \vv)$ (whenever $\vv$ itself is locally presentable as a closed category), we can see that the nerve of $\C_{fp} \hookrightarrow \C$ is a fully faithful functor, and the realization is the ref$lection from presheaves on $\C_{fp}$ to models.
% 		\item In general, a tangent category $\C$ is locally finitely presentable whenever it is the reflective subcategory of $\widehat{\a}$ for some small $\w$-category $\a$ - if the representables $\a \hookrightarrow \widehat{\a}$ belong to $\C$, then the inclusion $\C \hookrightarrow \widehat{\a}$ is induced by the nerve of $\a \hookrightarrow \C$.
% 	\end{enumerate}
% \end{example}

% The situation where the nerve of $K$ is fully faithful is particularly important - Kelly called these functors \emph{dense}, by metaphor with the notion of density found in Hausdorff spaces. A dense subcategory models the same situation, except we replace ``continuous functions'' with ``co-continuous functors''. That is to say, a functor $K:\a \to \c$ into a cocomplete $\c$ is \textit{dense} whenever each object $C \in \c_0$ is canonically written as a coend:
% \[
% 	\int^A \C(KA, C)\cdot KA
% \]
% so that a cocontinuous functor $F:\c \to \d$ is determined by $F.K$.
% The general definition of density does not require a cocomplete category.
% \begin{definition}%
% 	\label{def:dense}
% 	A functor $K: \a \to \C$ is \textit{dense} whenever one of the following conditions holds
% 	\begin{itemize}
% 		\item The identity functor $id:\C \to \C$ is the left Kan extension of $K$ along itself, so: $Lan_K K = id$.
% 		\item Every object is given as the weighted colimit $C = N_K(C) \star K$.
% 		\item The nerve of $K$ is fully faithful.
% 	\end{itemize}
% 	Note that a locally finitely presentable category is precisely\footnote{Assuming that Vopenka's principle holds \cite{Adamek1994}.}
% 	 a cocomplete category $\C$ with a dense functor $K: \a \to \C$.
% \end{definition}
% Dense functors are poorly behaved under composition, but there is a useful cancellativity results from Chapter 5.2 of \cite{Kelly2005}.
% \begin{proposition}
% 	Consider a diagram of $\vv$-categories
% 	\[\input{TikzDrawings/Ch5/lan-result.tikz}\]
% 	where $\alpha$ is a natural isomorphism, and $K$ is dense.
% 	If this diagram exhibits $(\alpha, J)$ as the left Kan extension of $K$ along $F$, then $J$ is dense.
% \end{proposition}
% \begin{corollary}%
% 	\label{cor:dense-ff-result}
% 	If $K$ is dense, $F$ is fully faithful, and  $J.F = K$, then $F$ is a dense subcategory.s
% \end{corollary}
% Generally speaking, we will often refer to dense \textit{subcategories} rather than dense \textit{functors}. Through a naive application of the (bo,ff)-factorization system on $\vv$-categories, every dense functor induces a dense subcategory, as $K$ factors as
% \[
% 	K = \a \xrightarrow[]{K'} \mathsf{im}(K) \hookrightarrow \C
% \]
% so the inclusion of $\mathsf{im}(K)$ into $\C$ is dense.
% % \begin{proposition}
% % 	Let $K:A \to \c$ be a dense functor. Then the full subcategory $im(K)$ is a dense subcategory.
% % \end{proposition}
% % \begin{proof}
% % 	Use Corollary \ref{cor:dense-ff-result}.
% % \end{proof}
% % \begin{example}
% % 	Consider the category of transverse-limit preserving copresheaves of $\wone$ algebras $\w = \mathsf{Mod}(\wone, \s)$, the category of Weil spaces. We have the following dense subcategories:
% % 	\begin{enumerate}
% % 		\item The unit: $K_1: 1 \hookrightarrow \w$
% % 		\item The infinitesimals: $D: \wone^{op} \hookrightarrow \w$
% % 		\item The finite-coproduct closure of infinitesimals $\{ \sum D_v | v \in \wone \} \hookrightarrow \w$ 
% % 	\end{enumerate} 
% % \end{example}
% The category of finite sets is dense in $\s$, as $\s$ is locally finitely presentable, and the finitely presentable sets are exactly finite sets. In particular, this means that the skeleton of finite sets, where each object is $[n] = \{1, \dots, n\}$, is dense in $\s$.  A dense subcategory $\a$ of $\C$, then, acts as a generalization of this context - the objects $A \in \a$ may act as the \emph{arities} for a theory in $\C$.


% % \subsection*{Anchored bundles are concrete models of a $\Lambda^+$-theory}

% % Anchored bundles provide a good first application of nervous theories to tangent categories. Recall that an anchored bundle is a differential bundle $(\pi:A \to M, \xi, \lambda)$ equipped with an anchor map $\anc: A \to TM$. Regarding a differential bundle as a functor $\Lambda^+ \to \C$ (Proposition \ref{prop:Lambda-is-refl-subcat}), this corresponds to a natural transformation:
% % \[\input{TikzDrawings/Ch5/anc-as-nat.tikz}\]
% % Now, it is always the case that $TM$ is the power $M^{D}$, so a reasonable first step is to find a category $\mathsf{Anchor}$ so that there is a bijective on objects functor $\Lambda^+ \to \mathsf{Anchor}$ that sends $1$ to a cone for $T.0$ - this would induce an anchor map as follows:
% % \[\input{TikzDrawings/Ch5/anc-as-nat-2.tikz}\]
% % The obvious solution works best in this case - take some full subcategory of $\w$.
% % \begin{definition}%
% % 	\label{def:truncated-wone}
% % 	A Weil algebra has \emph{width} $k \in \N$ if it is written:
% % 	\[
% % 		V = \ox^{0 \le i < k} W_{n(i)}, n(i) \in \N
% % 	\]
% % 	The category of $k$-truncated Weil algebras, written $\wone^k$, is the full $\w$-subcategory of $\wone$ of Weil algebras of width $k$ or less.
% % \end{definition}
% % Note that for each $V$ in $\wone^k$, the enrichment is given by:
% % \[
% % 	\wone^k(U,V) := (X \mapsto \underline{\wone}(U, X \ox V) )
% % \]
% % The maps in $\wone^1$ - that is, the full subcategory of $\wone$ spanned by:
% % \[
% % 	\N, W, W_2, \dots
% % \]
% % has a useful property:
% % \begin{lemma}\label{lem:writing-maps-in-wone}
% % 	Every morphism:
% % 	\[
% % 		W_n \to V \in \wone
% % 	\]
% % 	may be written using only $\{p,+,0,\ell\}$ (closed under $\ox, \o$ and maps induced by transverse limits).
% % \end{lemma}
% % \begin{proof}
% % 	This is a consequence of the fact that every map $W_n \to V$ in $\wone$ may be written:
% % 	\[
% % 		v \mapsto \sum_{x} (A_x v)\cdot x
% % 	\]
% % 	where $x \in \mathsf{var}(V)$, and $A_x$ is in $\N^n$ so that $A_n v$ is the ordinary dot product.
% % 	Each individual term may be formed without $c$, and the whole term is then constructed by adding each component using the appropriate $U.+.V$-symbols. (This may also be regarded as a consequence of the graphical notation for maps in $\wone$ in Table 1 on page 308 of \cite{Leung2017}.)
% % \end{proof}
% %  It is possible, then, to show that the category of anchored bundles in $\C$ is a full sub-tangent-category of functors $\wone^1 \to \C$ - note that $W$ acts as a cone for $(-)^{\yon(W)}$, so this induces a map:
% % \[
% % 	\anc: F(W) \to T.F(\N)
% % \]
% % \begin{proposition}
% % 	\label{prop:nerve-anc-work}
% % 	Every anchored differential bundle in $\C$ determines a functor $\wone^1 \to \C$; a morphism is exactly a tangent natural transformation.
% % \end{proposition}
% % \begin{proof}
% % 	Start with an anchored bundle $(q:E \to M, \xi, \lambda, \anc)$, then for hom-objects with domain $\N$:
% % 	\[
% % 		\wone^1(\N,\N) = 1 \mapsto id_M \hspace{0.5cm} \wone^1(\N,T) = 1 \mapsto \xi
% % 	\]
% % 	The hom-objects with domain $T$, the problem is slightly more difficult, as $\wone^1(T,\N)(T^V) = \wone(T, T^V)$ and $\wone^1(T, T)(T) = \wone(T, T.T^V)$. This part of the  proof amounts to constructing maps:
% % 	\[
% % 		\wone(T, T^V) \to \C(E, T^V.M) \hspace{0.5cm}  \wone(T,T^V.T) \to \C(E, T^V.E)
% % 	\]
% % 	The first mapping is straightforward, send $\theta$ to $\theta.M \o \anc$. For the second map, observe that the following diagrams commute:
% % 	\[
% % 		\input{TikzDrawings/Ch5/anc-atlas.tikz}
% % 	\]
% % 	The idea, then, is to take $f$ and rewrite it as a string of compositions that does not include $c$ and switch out every occurrence of
% % 	\[
% % 		T^V.\theta.M, \theta \in \{p,0,+,\ell\}
% % 	\]
% % 	replace it with $T^V.(\theta')$, where $\theta'$ is the corresponding map in $\{q,\xi,+_q,\lambda\}$. This induces a map $\wone^1(W,W) \to \C(E,E) \in \w$, and the $\anc$ is exactly the unique $\alpha: F.W \to T.F$ induced by universality.

% % 	For morphisms, the inclusion $\Lambda^+ \to \wone$ ensures that any tangent natural transformation will be a linear morphism on the underlying differential bundle, and the tangent natural transformations coherences ensures that a tangent natural transformation will preserve the anchor. Conversely, an anchored bundle morphism will preserve each of the constructed morphisms $E \to T^V.E, E \to T^V.M$ (as it preserves each of $\{ q, +_q, \xi, \lambda\}$), giving a tangent natural transformation. Thus there is a faithful embedding $\mathsf{Anc}(\C) \hookrightarrow [\wone^1, \C]$.
% % \end{proof}
% % Now restrict to the category $\wone^*$, the full subcategory of $\wone$ spanned by $\{\N, W\}$, the density result \Cref{cor:dense-ff-result} yields the following.
% % \begin{corollary}%
% % 	\label{cor:dense-subcat}
% % 	The category of anchored bundles is a full subcategory of $[{\wone^*},\C]$.
% % \end{corollary}
% % Observe that there is a bijective-on-objects functor
% % \[
% % 	a: \Lambda^+ \to \wone^*
% % \]
% % The category $\Lambda^+$ is a dense subcategory of differential bundles, and each object $\N, 1$ have the structure of a differential bundle.
% % \begin{lemma}
% % 	The category $\wone^1$ is a $\Lambda^+$-theory on the category of differential bundles in $\w$.
% % \end{lemma}
% % The category of anchored bundles in $\w$ is the category of presheaves on $(\wone^{1})^{op}$ whose underlying pre-differential bundle is a differential bundle. Thus we characterize the category of anchored bundles as a following pullback in $\w\mathsf{CAT}$.
% % \begin{proposition}\label{prop:models-in-C-anc}
% % 	The category of anchored bundles in $\C$ is equivalent to the pullback of $\w$-categories:
% % 	\[\input{TikzDrawings/Ch5/anc-as-pb-in-cat2.tikz}\]
% % \end{proposition}
% % \begin{proof}
% % 	A functor $\wone^* \to \C$ is an anchored bundle precisely whenever $\Lambda^+ \to \wone^* \to \C$ is a differential bundle, and the uniquely induced map:
% % 	\[
% % 		\hat A.W \to T.\hat A.\N	
% % 	\] gives the anchor map (naturality ensures linearity). Fully faithfulness was proved in \Cref*{prop:nerve-anc-work}.
% % \end{proof}
% % Now restricting to Weil spaces, the diagram:
% % \[\input{TikzDrawings/Ch5/anc-as-pb-in-cat.tikz}\]
% % yields the following corollary.
% % \begin{corollary}\label{cor:models-in-w-anc}
% % 	Anchored bundles are monadic over differential bundles in $\w$.
% % \end{corollary}
% % This result gives a Segal condition for anchored bundles, although it is quite simplistic: a $\w$-functor $A:\wone^1 \to \C$ is an anchored bundle if and only if $A.a:\Lambda^+ \to \C$ is a differential bundle.

\section{Nervous Theories}%
\label{sec:nervous_theories}

\begin{itemize}
	\item Lifting the characterization of algebroids from Theorem \ref{thm:weil-nerve} to the enriched perspective introduces the challenge of finding the appropriate framework to describe them.
	\item A natural candidate would be Kelly's enriched sketches, which are small $\vv$-categories equipped with a chosen class of limit cones - however there is no simple way to require the natural part $\alpha: F.T \Rightarrow T.F$ is cartesian when $\alpha$ is the map induced by a choice of weighed limit.
	\item The nervous theories of \cite{Bourke2019} provide a natural generalization of -, where blah is an anchored bundle and blahhh 
\end{itemize}

\begin{itemize}
	\item Nervous theories are generalizations of Lawvere theories, which here are product-preserving functors:
	\item Models of Lawvere theories are, of course, strictly product preserving functors. But they may also be thought of as functors $\th \to \s$ so that $\Sigma^{op} \o \th \to \s$ is strictly product-preserving - that means, it must be the nerve of $A$ for blah
	\item This generalizes as follows:
\end{itemize}

Then: 
\begin{itemize}
	\item original exposition on theories.
	\item add the comodel stuff.
\end{itemize}
% Recall that a Lawvere theory is a bijective-on-objects, product-preserving functor:
% \[
% 	\mathsf{FinSet_{Skel}} \to \mathbb{T}
% \]
% There is a generalized notion of Lawvere theory based on the idea that a dense subcategory can play the role of $ \mathsf{FinSet_{Skel}}$. \cite{Power1999} was the first to extend this to an arbitrary $\vv$-LFP $\C$, where the arguments for a Lawvere theory over $\C$ would then be the finitely presentable objects $\C_{fp}$. \cite{Berger2012} generalized this to dense subcategories of $\s$-LFPs $\C$, and Garner-Bourke's paradigm moved to dense sub-$\vv$-categories of LFP $\vv$-categories, and the presentation in this section follows the refinement of these ideas in \cite{Bourke2019}.

% \begin{definition}%
% 	\label{def:nerve-theory}
% 	Let $K:\a \to \C$ be a dense $\vv$-subcategory of a locally finitely presentable $\c$. We call the replete image of $N_K$ in $\widehat{\a}$ the category of $K$-nerves. An \textit{$\a$-theory} is a bijective-on-object $\vv$-functor $J: \a \to \th$, where each
% 	\[
% 		\th(J-,a): \a \to \vv
% 	\] is a $K$-nerve.
% 	The category of concrete models for an $\a$-theory is given by the pullback in $\vv\mathsf{CAT}$:
% 	\[\input{TikzDrawings/Ch5/conc-models.tikz}\]
% 	We say that an $\a$-presheaf $X$ satisfies the $K$-Segal conditions whenever $t^*X$ is a $K$-nerve (that is, it is a concrete model of the theory).
% \end{definition}
% \begin{example}
% 	~\begin{enumerate}[(i)]
% 		\item A functor $t:\mathsf{FinSet_{Skel}} \hookrightarrow \th$ is a $\mathsf{FinSet_{Skel}}$-theory if and only if $\th$ is a Lawvere theory. The Segal conditions in this case identify the models of the Lawvere theory.
% 		\item The inclusion of $\mathsf{Pth} \hookrightarrow \Delta$ is a $\mathsf{Pth}$-theory for the category of graphs. The Segal conditions in this case capture the classical Segal conditions for categories.
% 	\end{enumerate}
% \end{example}
% \cite{Bourke2019} give an idempotent adjunction between pre-theories and monads over a category, that is refined to an equivalence of categories between theories and \emph{nervous} monads. For the purposes of this thesis, the following suffices:
% \begin{proposition}
% 	Let $K: \a \hookrightarrow \C$ be a dense $\vv$-subcategory, and $J: \a \to \th$ an $\a$-theory.
% 	The category of concrete models of $J$ is monadic over $\C$.
% \end{proposition}
% \begin{proof}
%     This proof relies on a technical lemma - note that $N_K$ is fully faithful and monadic (as it is the inclusion of full reflective subcategory), and $J^*$ is monadic as $J$ is a bijective-on-objects functor. The pullback along $N_K$ will be fully faithful (as fully faithful functors are the right class of maps for the (bo,ff)-factorization system on $\vv$-categories). 
% \end{proof}

\section{The Weil nerve and Segal conditions for involution algebroids}%
\label{sec:revisiting-segal-conds}

This section proves one of the main theorems of this thesis - the category of algebroids is algebraic. That is, the category of involution algebroids is algebraic over the category of anchored bundles. \cite{Kapranov2007} showed that something like this held in the category of vector bundles over a smooth manifold, if the vector bundles are allowed to be infinite-dimensional. All of the actual work to prove this theorem was completed in \Cref{chap:weil-nerve}, particularly by Theorem \ref{thm:weil-nerve,cor:the-prolongation-description}.

% \begin{itemize}
	
% \end{itemize}


In $\mathsf{Gph}(\s)$, the path complex at $n$ for a graph is given by a choice of pullback
\[\input{TikzDrawings/Ch5/graph-comp.tikz}\]
This is represented by the following pushout:
\[ \input{TikzDrawings/Ch5/dense-in-graphs.tikz}\]
A construction of this flavour works for any ``co-graph'' in a category $\C$, so that the nerve of the co-graph lands in graphs, and the nerve of the path-complex sends the graph $N\dots$ to its path complex. Repeating this for the Weil nerve then requires a notion of a co-anchored bundle and its co-prolongation - however, all of the necessary results from \Cref{sec:weil-nerve} translate over by duality.

\begin{definition}%
	\label{def:coanc-bundle}
	Let $\C$ be a $\w$-category with coherently chosen copowers by representables (so that $\C^{op}$ is a tangent category). A \emph{co}-anchored bundle in $\C$ is an anchored bundle in $\C^{op}$.
	\begin{enumerate}[(i)]
		\item A co-differential bundle, so a triple:
		      \[
			      (y:D \cdot L \to L, q:M \to L, \xi:L \to M)
		      \]
		      so that $y$ is a semigroup action:
		      \[\input{TikzDrawings/Ch5/semigroup-action.tikz}\]
		      and $q \o \xi$ splits the idempotent $e = y \o (0\o !, id)$ and the following diagram is a coequalizer:\[\input{TikzDrawings/Ch5/coeq-coanc.tikz}\]
		      and pushout powers $L(n)$ of $q:M \to L$ exist and are preserved by $D \x (-)$.
		\item There is a co-anchor: a map $d:D\x M \to L$ so that
		      \[\input{TikzDrawings/Ch5/coanc.tikz}\]
	\end{enumerate}
\end{definition}
This definition has an immediate consequence, as $\C(-, A):\C^{op} \to \w$ is continuous for every $A \in \C$ (so it sends colimits in $\C$ to limits in $\w$). (It is important to remember here that powers in $\C$ are copowers in $\C^{op}$, and vice versa).
\begin{lemma}%
	\label{lem:nerve-of-coanc}
	Let $\C$ be $\w$-category with coherently chosen copowers by representables.
	For any co-anchored bundle $K: (\wone^*)^{op} \to \C$, the nerve of $K$ lands in anchored bundles in $\w$.
	\[
		N_K: \C \to \mathsf{Anc}(\w)
	\]
\end{lemma}

The construction of the co-prolongations of a co-differential bundle is dual to the prolongations from Definition \ref{def:prolongation-of-anchored-bundle} (where span composition is replaced with the dual notion of \emph{cospan} composition).
\begin{definition}%
	\label{def:coprol}
	Let $(y:D \cdot L \to L, q:M \to L, \xi:L \to M)$ be a co-anchored codifferential bundle in a $\w$-category with an coherently chosen copowers by representables. The coprolongation of $A$ by a Weil algebra $V$ is the cospan defined by the dual of $\boxtimes$, written $\boxplus$. So for cospans of the form:
	\[
		M \xrightarrow[]{l_X} X \xleftarrow[]{r_X} D_U \cdot M,
		M \xrightarrow[]{l_Y} Y \xleftarrow[]{r_Y} D_V \cdot M
	\]
	The cospan $(X:M \to D_U \cdot M)\boxplus (Y:M \to D_V\cdot M)$ is given by the pushout:
	\[\input{TikzDrawings/Ch5/cospan-compuv.tikz}\]
	% \begin{enumerate}[(i)]
	%     \item $L_{\N}: M \xrightarrow{=} M \xleftarrow{=} M$
	%     \item $L_{W_n}:1 \xrightarrow{\prod^n q} L(n) \xleftarrow{\delta(n)} D(n)$
	%     \item $L_{UV}$ is given by cospan composition:
	%     \[\input{TikzDrawings/Ch5/cospan-comp.tikz}\]
	% \end{enumerate}
	Then for any $V = W_{n(1)} \ox \dots \ox W_{n(k)}$, the object $L_V$ is the apex of the cospan:
	\[
		L_V := L(n_1) \boxplus \dots \boxplus L(n_k)
	\]
	This pushout is the \emph{$V$-coprolongation} of $L$.
\end{definition}
\begin{example}
	The $V$-coprolongation of the infinitesimal object $D$ in a representable tangent category is $D_V$.
\end{example}
The following results follow by duality.
\begin{proposition}%
	\label{prop:coanc-facts}
	Let $(y:D \cdot L \to L, q:M \to L, \xi:L \to M)$ be a co-anchored codifferential bundle in a $\w$-category with an coherently chosen copowers by representables. Then
	\begin{enumerate}
		\item There is a co-additive bundle $(q:M \to L, \xi:L \to M, \delta_L:L \to L(2))$, and $d$ is a co-additive bundle morphism.
		\item There is a morphism $\boxdot: L \boxplus L \to L$ (\Cref{def:coprol}), defined:
		      \[
			      L \po{d}{D \cdot q} D \cdot L \xrightarrow[]{(e | y)} L
		      \]
		\item The following diagram is a coequalizer.
		      \[
			      \input{TikzDrawings/Ch5/coanc-couni.tikz}
		      \]
	\end{enumerate}
\end{proposition}

Now, consider the co-anchored bundle induced by the fully faithful functor:
\[
	\yon: \wone^* \to \mathsf{Anc}(\w)
\]
So in this case, $L_V$ represents the functor that sends an anchored bundle $A$ to $\prol(V,A)$. Furthermore, $d:D \to L$ is the unique map induced by the couniversality of $D$. Now define the following.
\begin{definition}%
	\label{def:prol-cat}
	Define the full subcategory of $\mathsf{Anc}(\w)$ spanned by $L_V, V \in \wone$, to be $\prol$. Furthermore, write the full subcategory of $\mathsf{Anc}(\w)$ spanned by $V \in \wone^k$ as $\prol^k$.
\end{definition}
\begin{proposition}%
	\label{prop:nerve-thm-for-coprol}
	Every anchored bundle with chosen prolongations in a tangent category $\C$ determines a $\w$-functor $\prol^{op} \to \C$, and the embedding of $\mathsf{Anc}^* \to [\prol^{op}, \C]$ is fully faithful.
\end{proposition}

The category $\prol$, then, will play the role of the arities for involution algebroids, which is unsurprising based on the wording of Corollary \ref{cor:the-prolongation-description}. The construction of the functor:
\[
	w: \prol \to \wone^{op}
\]
is surprisingly straightforward.
\begin{observation}
	Recall that for any tangent category $\C$, Proposition \ref{prop:c-refl-in-anc} states that the category $\C$ is a reflective subcategory of $\mathsf{Anc}(\C)$. Because the reflection sends $A \mapsto TM$, it will map $L \mapsto D$ - then it must map:
	\[
		L_V \mapsto D_V
	\] as the coprolongation of $D_V$ for the co-anchored bundle $D$ is $D_V$, and left adjoints preserve colimits. This reflection restricts to a functor:
	\[
		\prol \to \wone^{op}
	\]
	Note that this sends $\boxplus^k_{i=1} L(n_i) \mapsto \prod^k_{i=1}D(n_i)$.
\end{observation}

\begin{theorem}%
	\label{thm:pullback-in-cat-of-cats-inv-algd}
	The category of involution algebroids with chosen prolongations in a tangent category $\C$ is precisely the pullback in $\w\mathsf{CAT}$:
	\begin{equation}\label{eq:prol2}
		\input{TikzDrawings/Ch5/prol2.tikz}
	\end{equation}
	where $w$ is the reflector from $\mathsf{Anc}(\w)$ to $\w$ restricted to the full subcategory $\prol$. Consequently, in $\w$ involution algebroids are monadic over anchored bundles.
\end{theorem}
\begin{proof}
	Recall the correspondence:
	\[
		\infer{(\hat, \alpha): \wone \to \C \in \mathsf{TangCat}_{lax}}
		{\bar A: \wone \to \C \in \w\cat}
	\]
	Then, if:
	\[
		\wone^* \to \prol \to \wone \to \C	
	\]
	determines an anchored bundle $(\pi:A \to M, \xi, \lambda, \anc)$ whose $V$-coprolongation is $\hat A.V$, so that $\hat A$ is the nerve of an involution algebroid by \Cref{cor:the-prolongation-description}.
\end{proof}
Now, recall that the basic definition of an involution algebroid only really requires prolongations for algebras of width three or less, $T_{i}.T_j.T_k$ so we can use truncation to get the following:
\begin{corollary}
	The category of involution algebroids in a tangent category $\C$ is exactly the pullback:
	\begin{equation}\label{eq:prol3}
		\input{TikzDrawings/Ch5/prol3.tikz}
	\end{equation}
	using the truncated categories from Definition \ref{def:prol-cat} and Definition \ref{def:truncated-wone}
\end{corollary}
In particular, this means that we can write the category of Lie algebroids as a pullback in $\w$-cat:
\begin{corollary}
	The category of Lie algebroids is the pullback in $\w\mathsf{Cat}$:
	\[\input{TikzDrawings/Ch5/lie-algd-monadic.tikz}\]
\end{corollary}



\section{The infinitesimal nerve of a groupoid}%
\label{sec:inf-nerve-of-a-gpd}


The Lie functor, discussed in \Cref{sec:Lie_algebroids} and Example \ref{ex:inv-algds}, is often called the \emph{infinitesimal approximation} of a Lie groupoid (\cite{nlab:lie_group}). This section demonstrates how the enriched categorical perspective on Lie algebroids can make this concrete. Recall that Kan's simplicial approximation functor:

\[
	\mathsf{Top} \to \widehat{\Delta}
\]
Is constructed as a \emph{nerve}, so there is a functor that sends $[n] \in \Delta$ to the $n$-simplex:
\[
	K: \Delta \to \mathsf{Top}; [n] \mapsto \{ x \in \R^n | \sum x \le 1 \}
\] 
The simplicial set approximating a topological space $X$ is exactly $N_K(X)$. The infinitesimal approximation of a groupoid then replaces the simplex category that models homotopical data (the category $\Delta$) with the category of Weil algebras that models infinitesimal data ($\wone$). Thus the infinitesimal approximation of a groupoid will be the nerve of an infinitesimal object:
\[
	\partial: \wone \to \mathsf{Gpd}(\w)
\] so that the nerve $N_\partial$ will coincide with the Lie functor from groupoids to algebroids. This infinitesimal object will also equip the category of groupoids with a second tangent structure, similar to that of Lie algebroids.

This section deals with internal groupoids in the cartesian closed, locally presentable category $\w$. We may use the symmetry of theories to treat:
\[
	\mathsf{Gpd}(\w) \cong \mathsf{Mod}(\wone, \mathsf{Gpd}(\s))
\]
Note that the cofree tangent structure on $\mathsf{Gpd}(\s)$ (recall Observation \ref{obs:cofree-tangent-cat}) agrees with the pointwise tangent structure on internal groupoids.
% A groupoid of particular interest is the \emph{arrow groupoid}.
\begin{definition}%
	\label{def:arrow-gpd}
	The arrow groupoid in $\mathsf{Gpd}(\s)$ is the free groupoid over the graph with a single arrow:
	\[
		\bullet \to \bullet
	\]
	The arrow groupoid in $\w$ is the constant functor:
	\[
		\wone \to \mathsf{Gpd}(\s)
	\]
	Write this internal groupoid $I$, and write the full sub-$\w$-category of $\mathsf{Gpd}(\w)$ spanned by product powers $I^n$ as $\mathbb{I}$.
\end{definition}
\begin{lemma}%
	\label{lem:arr-gpd-cub-monoid}
	The arrow groupoid has two commutative monoid structures on it,
	\[
		\input{TikzDrawings/Ch3/Sec8/cub-mon-gen-maps.tikz}
	\]
	where the multiplications correspond to the maps:
	\[\input{TikzDrawings/Ch3/Sec8/connections-gpd.tikz}\]
	and the units are $s,t:1 \to I$. Note that $t$ is a zero\footnote{A zero for a semigroup is an absorbing element, following the convention from Definition \ref{def:inf-object}.} for the $s$ monoid, and vice versa. The inverse map:
	\[
		i: I \to I
	\]
	exchanges these to monoids: $i \o \gamma^s = \gamma^t$.
\end{lemma}
\begin{remark}
	The arrow groupoid is a \emph{symmetric cubical monoid} (see Section 6 of \cite{Grandis2003}), and the reversion operator makes it a cubical monoid with reversion (see Section 9 of \cite{Grandis2003}, and 8.2 of \cite{Mauri2017}). The cubical perspective presents internal groupoids using commuting squares rather than a composition operation.
\end{remark}
It is well known that the category of internal groupoids is cartesian closed in any locally presentable, cartesian closed category $\C$. Density arguments and the arrow groupoid give way to see this:
\begin{proposition}
	The category of internal groupoids in $\w$ is a cartesian monoidal reflective sub-$\w$-category of $\widehat{\mathbb{I}}$, so it is a cartesian closed, locally presentable $\w$-category.
\end{proposition}
\begin{proof}[(Sketch)]
	This seems to be a piece of mathematical folklore - a proof that groupoids embed into cubical sets may be found in \cite{nlab:fundamental_groupoid_of_a_cubical_set_and_the_cubical_nerve_of_a_groupoid}, where the cubical set is induced using the nerve of the functor $\square \to \mathsf{Gpd}$ induced by cubical monoid $I$, then use the fact that groupoids are an exponential ideal of cubical sets, and apply \Cref{cor:dense-ff-result} to get the desired result that $\mathbb{I}$ is dense in $\mathsf{Gpd}$.
\end{proof}
\begin{corollary}
	The category of internal groupoids in any tangent category $\C$ is a full sub-$\w$-category of $[\mathbb{I}, \C]$.
\end{corollary}

The diagram constructing the linear approximation of a reflexive graph $(s,t:G \to M, i:M \to G)$ is equivalent to an equalizer. Observe that the universality of the following two diagrams is equivalent.
\[\input{TikzDrawings/Ch3/Sec9/Lie-diff-as-eq.tikz}\]
<<<<<<< .mine
A similar argument to Proposition \ref{prop:evf-vbun-is-nonsingular} lets this diagram be presented as an equalizer.
=======
were the target map is $\anc = A \hookrightarrow TG \xrightarrow[]{T.t} TM$.
A similar argument to \Cref{prop:evf-vbun-is-nonsingular} lets this diagram be presented as an equalizer.
>>>>>>> .r7303
\[\input{TikzDrawings/Ch3/Sec9/Lie-diff-as-eq-idempotents.tikz}\]
% \begin{definition}
%     Define $\partial$ to be the coequalizer
%     \[\input{TikzDrawings/Ch5/coeq-partial.tikz}\]
%     Define the following maps on $\partial$.
%     \begin{enumerate}[(i)]
%         \item There is a point $0 \to \partial$ induced by:
%         \[\input{TikzDrawings/Ch5/partial0.tikz}\]
%         \item Write pushout powers of $0$ as $\partial(n)$. 
%         \[\input{TikzDrawings/Ch5/delta-coadd.tikz}\]
%         There is a map $\delta: \partial \to \partial(2)$ induced by:
%         \[\input{TikzDrawings/Ch5/delta-coadd2.tikz}\]
%         \item There is a cosemigroup structure $\odot: \partial \x \partial \to \partial$ induced by:
%         \[\input{TikzDrawings/Ch5/cosemi.tikz}\]
%     \end{enumerate}
% \end{definition}
\begin{definition}%
	\label{def:partial-gpd}
	Define $\partial$ and the map $d:D \to \partial$ to be the following coequalizer in $\mathsf{Gpd}(\w)$.
	\[\input{TikzDrawings/Ch3/Sec9/Lie-diff-as-eq-idempotents.tikz}\]
	where the co-anchor is given by:
	\[
		d := D \xrightarrow[]{(id, t)} D\x I \to \partial
	\]	
\end{definition}

\begin{proposition}
	For any tangent category $\C$, the infinitesimal approximation of the groupoid $G$ (if it exists)
	\[\input{TikzDrawings/Ch3/Sec9/Lie-diff-as-eq.tikz}\]
	Is equivalent to the weighted limit:
	\[
		\{\partial, G\};\hspace{0.25cm} 	\w \xleftarrow[]{\partial} \mathbb{I} \xrightarrow[]{G} \C
	\]
\end{proposition}
\begin{proof}
	Note that by the definition of a weighed limit, for every object $C$ in $\C$
	\[
		\C(C, \{\partial, G\}) \cong [\mathbb{I}, \w](\partial -, \C(C, G-))
	\]
	By the continuity of $ [\mathbb{I}, \w](\partial -, x):\C^{op} \to \w$, it follows that we have the following equalizer diagram for every object $C \in \C_0$:
	\[\input{TikzDrawings/Ch3/Sec9/Lie-diff-as-eq-idempotents.tikz}\]
	which is equivalent to asking that $\{\partial, G\}$ be the equalizer
	\[\input{TikzDrawings/Ch3/Sec9/Lie-diff-as-eq-2.tikz}\]
\end{proof}
\begin{remark}
	The same coequalizer $\partial$ can be constructed in the category of reflexive graphs to exhibit the infinitesimal approximation of a reflexive graph as a weighted colimits.
\end{remark}



\begin{proposition}%
	\label{prop:coeq-of-semigroup-w-z}
	The groupoid $\partial$ in $\mathsf{Gpd}(\w)$ is the coequalizer in the category of semigroups with zeros.
\end{proposition}
\begin{proof}
	First, observe that $(-)\x (=)$ is cocontinuous in each variable, so $\partial \x \partial$ is a double coequalizer - use this to induce a multiplication $\boxdot$.
	\[\input{TikzDrawings/Ch5/induce-obox.tikz}\]
	Where $\tau$ is the natural transformation: \[((\pi_0\o \pi_0, \pi_1 \o \pi_0),(\pi_0\o\pi_1, \pi_1 \o\pi_1))\]
	There is a map:
	\[
		0^\partial:= 1 \xrightarrow[]{(0,s)} D \x I \to \partial	
	\]
	that acts as a zero for the semigroup. 
	The co-continuity of $(-)\x (=)$ in each variable also guarantess that for any semigroup-with-a-zero map:
	\[
		f: D \x I \to X, \hspace{0.15cm}f \o (D \x e^-) = f \o (e \x I)
	\]
	The induced map $f': \partial \to X$ will preserve the multiplication and zero (where the zero is induced by $(0, s): 1 \to D \x I$).
\end{proof}
\begin{corollary}%
	\label{cor:boxdot-coeq}
	The following diagram:
	\[
		\input{TikzDrawings/Ch5/boxdot-coeq.tikz}
	\]
	is a coequalizer.
\end{corollary}
\begin{proof}
	This follows by the commutativity of colimits. The following two diagrams are coequalizers:
	\[\input{TikzDrawings/Ch5/coeqs-mult.tikz}\]
	where $G.2$ is the object of commuting squares, $G_2 \ts{m}{m} G_2$.
	The first couniversality condition is a consequence of the couniversality of $\odot$.
	For the second one, consider the diagram:
	\[\input{TikzDrawings/Ch5/eq-gpd-univ.tikz}\]
	The two equalizer maps send a square to:
	\[\input{TikzDrawings/Ch5/G-univ-maps.tikz}\]
	That is,
	\[
		((v,t), (u,w)): G_2 \ts{m}{m} G_2 \mapsto ((id,u),(u,id)), ((v,id),(id,v))
	\]
	So for a square to $((v,t), (u,w))$ be equalized by these two maps, it follows that:
	\[
		v = id, u = id
	\]
	so the square must be of the form $((id,t),(id,w)):G_2 \ts{m}{m} G_2$, thus $t = w$, and the square is $G^{\square +}(t)$.
\end{proof}

\begin{lemma}%
	\label{lem:co-anchored-bundle-gpd}
	The tuple:
	\[
		(y: D \x \partial \xrightarrow[]{d \x \partial} \partial \x \partial \xrightarrow[]{\boxdot}\partial, 0: 1 \to \partial, !:\partial \to 1, d:D \to \partial)
	\]
	is a co-anchored bundle under $1$, where
	\[
		y: D \x \partial \xrightarrow[]{d \x \partial} \partial \x \partial \xrightarrow[]{\boxdot} \partial
	\]
\end{lemma}

\begin{proposition}
	For every Weil algebra $V$, there is an isomorphism:
	\[
		L_V \cong \partial_V
	\]

	Where$L_V$ is the $V$-prolongation of the co-anchored bundle in Lemma \ref{lem:co-anchored-bundle-gpd}.
\end{proposition}

\begin{proof}
	Without any loss in generality, it is sufficient to prove this holds for $L_{n} = L(W_n)$ and $n\cdot \partial$. The proof follows by induction, first note that $L = \partial$ by definition.

	For $L_2$ and $\partial \x \partial$, first note there is the injection:
	\[\input{TikzDrawings/Ch5/induce-map-from-coprol.tikz}\]
	Note that for any $G$, a map $X: \partial \x \partial \to G$ corresponds to a commuting square in $T^2G$ of the form:
	\[\input{TikzDrawings/Ch5/comm-square-T2G.tikz}\]
	so that $T.e \o v = id, e.T \o u = id$.
	Note that $u = (e.T \o v)^{-1};(T.e \o u);v$, and that precomposition with the uniquely induces map determine a $(\bar{u},v):X \to G^{L(2)}$, where
	\[ T.0 \o \bar{u} = u\]
	Observe that any pair $(\bar{u}, v):X \to  G^{L(2)}$ determines a square:
	\[
		\input{TikzDrawings/Ch5/inverse-of-prol.tikz}
	\]
	where $T.e \o v = id$, and $T.e \o \bar{u} = id$.
	Note that $T.0 \o T.p \o T.0 \o \bar{u} = T.0 \o \bar{u}$, and that the bottom horn is $u = (e.T \o v)^{-1};(T.0 \o \bar{u});v$, now check:
	\begin{align*}
		T.e \o u
		% &= T.e \o \((e.T \o v)^{-1};(T.0 \o \bar{u});v \) \\
		 & = (T.e \o e.T \o v)^{-1};(T.e \o T.0 \o \bar{u});(T.e \o v) \\
		 & = id; T.0 \o \bar{u}; T.e \o v = T.0 \o \bar{u}
	\end{align*}
	and
	\begin{align*}
		e.T \o u
		 & = e.T \o (e.T \o v)^{-1};(T.0 \o \bar{u});v                 \\
		 & = (e.T \o e.T \o v)^{-1};(e.T \o T.0 \o \bar{u});(e.T \o v) \\
		 & = (e.T \o v)^{-1};id;(e.T \o v) = id
	\end{align*}
	Thus determining a map $X \to G^{\partial \x \partial}$.
	The two maps are inverse to each other, giving an isomorphism.

	% For the inductive case, use the zig-zag presentation of $G^{L(n+2)}$ to see the following isomorphism holds:
	% \[
	%     \input{TikzDrawings/Ch5/inductive-step.tikz}
	% \]
	% Thus, $L(n) = D(n)$.

	For the inductive case, look at the prolongations of anchored bundles and recall that:
	\[
		\prol(UV,A) \boxtimes_M \prol(Z,A)  =
		\prol(UV,A) \boxtimes_{\prol(U,A)} \prol(UZ,A)
	\]
	Where you treat $\prol(UV,A), \prol(UZ,A)$ as spans over $\prol(U,A)$, so that:
	\begin{gather*}
		\prol(U,A) \xleftarrow[]{id \boxtimes \pi^V} \prol(UV,A) \xrightarrow[]{anc^U \boxtimes \prol(V,A)} T^U.\prol(V,A)\\
		\prol(U,A) \xleftarrow[]{id \boxtimes \pi^Z} \prol(UZ,A) \xrightarrow[]{anc^U \boxtimes \prol(Z,A)} T^U.\prol(Z,A)
	\end{gather*}
	And observe that their span composition is:
	\[
		\input{TikzDrawings/Ch5/pullback-uvz.tikz}
	\]
	Now, assume that the hypothesis holds for any two of $U, V, Z \in \wone$ and use the co-continuity of $X \x (-)$ in a cartesian closed category.
	\begin{align*}
		L_{UVZ} & = L_{UV} \boxplus_U L_{UZ}
		% \\
		% &= (\partial(U) \x \partial(V)) \po_{(id)}{\pi_0} (D(V) \x \partial(U) \x \partial(Z)) \\
		% &= \partial(U) \x (\partial(U) \po_{}{} \partial(Z)) \\
		% &= \partial(V) \x \partial(U)\x \partial(Z)
	\end{align*}
	So it now suffices to prove that the diagram:
	\[\input{TikzDrawings/Ch5/partial-u-po.tikz}\]
	is a pushout - but by the inductive hypothesis,
	\[\input{TikzDrawings/Ch5/partial-no-u-po.tikz}\]
	so the result follows by the cocontinuity of $\partial_U \x (-)$.
	% (where you use the hypothesis for the co-anchored bundle $\boxplus_U$ )
\end{proof}
\begin{corollary}%
	\label{cor:rewrite-boxdot}
	Note that, by construction, the $\hat{y}:L_{WW} \to L$ map is isomorphic to the $\boxdot$ multiplication.
	\[
		\hat{y}: L_{WW} \cong \partial \x \partial \xrightarrow[]{\boxtimes} \partial
	\]
\end{corollary}

\begin{theorem}%
	\label{thm:inf-gpd}
	The object $\partial$ determines a monoidal, transverse-colimit-preserving $\w$-functor:
	\[
		\partial: \wone \to \mathsf{Gpd}(\w)
	\]
	with a monoidal natural transformation $d:\partial \Rightarrow D$.
\end{theorem}
\begin{proof}
	First, check that $(\partial, \boxtimes, 0, !, \delta)$ is an infinitesimal object Definition \ref{def:inf-object}.
	\begin{enumerate}[{[{I}O.1]}]
		\item Pushout powers of $0:1 \to \partial$ exist as $\mathsf{Gpd}(\w)$ is cocomplete.
		\item The commutative-semigroup-with-a-zero axiom follows by Proposition \ref{prop:coeq-of-semigroup-w-z}
		\item The additivity follows from Corollary \ref{cor:rewrite-boxdot}
		\item The couniversality axiom follows from the couniversality of the underlying coanchored bundle and Corollary \ref{cor:rewrite-boxdot}.
	\end{enumerate}
	Note that $d: D \to \partial$ commutes will all of the structure morphisms, so it induces a monoidal natural transformation $D \Rightarrow \partial$, exhibiting $\partial$ as an enriched functor.
\end{proof}


\begin{definition}
	By Theorem \ref{thm:inf-gpd}, there is a $\w$-functor:
	\[
		\partial: \wone^{op} \to \mathsf{Gpd}(\w)
	\]
	The nerve of $\partial$ is the \emph{infinitesimal nerve} functor (the nerve of an infinitesimal object).
\end{definition}
\begin{theorem}
	The infinitesimal nerve of $\partial: \wone \to \mathsf{Gpd}$ lands in the category of involution algebroids.
\end{theorem}
\begin{proof}
	Observe that the underlying co-anchored bundle structure on $\partial$ induces a functor:
	\[
		L: \prol \to \mathsf{Gpd}(\w)
	\]
	so that $N_L$ lands in the $\prol$-nerves for $\mathsf{Anc}(\w)$. Therefore, the following diagram commutes:
	\[\input{TikzDrawings/Ch5/inf-nerve-wonky.tikz}\]
\end{proof}
\begin{observation}%
	\label{obs:derivative-weighted-limit}
	If all limits weighted by $W:\a \to \w$ exist in $\C$, then this gives a functor:
	\[
		[\a, \C ] \xrightarrow[]{\{W,-\}} \C; F \mapsto \{ W, F\}
	\]
	which may also be regarded as a functor:
	\[
		\{-,-\}: [\a, \w] \x [\a, \C] \to \C 	
	\]
	Now, regard the infinitesimal object $\partial$ as the weight:
	\[
		\partial: \wone \x \square^{op} \to \w; (V, [n]) \mapsto \partial_V([n])
	\]
	And a groupoid $\g$ as a strict tangent functor:
	\[
		G:\wone \x \square^{op} \to \C	
	\]
	Then the Lie functor is equivalently given as the weighted limit:
	\[
		`G^\partial: \wone^{op} \xrightarrow[]{\widehat{(\partial, G)}} 
		[\square^{op}, \w] \x [\square^{op}, \C] \xrightarrow[]{\{-,-\}} \C
	\]	
	% (regarding $\wone \x \square^{op}$ as the free tangent category over $\square^{op}$).
	% Then for any groupoid $G: \wone \x \square^{op} \to \C$, the Lie derivative is computed as the partial weighted-limit 
	% \[
	% 	\{\partial_{\wone}, -\}: \mathsf{Gpd}(\C) \to \mathsf{Inv}(\C)
	% \]
\end{observation}

Now that the Lie derivative in $\w$ has been characterized as the nerve of 
\[
	\partial: \wone \to \w
\]	
This has put the Lie functor into a nerve/realization context (recall \Cref{def:nerve}).
\begin{definition}
	The \emph{Lie realization} is the left Kan extension:
	\[
		|-|_\partial = Lan_\partial: \mathsf{Inv}(\w) \to \mathsf{Gpd}(\w)	
	\]
\end{definition}

This functor is well behaved - first see that it preserved products:
\begin{lemma}
	The realization functor preserves products.
\end{lemma}
\begin{proof}
	Product-preservation is a consequence of $|-|_\partial$ being the left Kan extension of a cartesian functor along a cartesian functor.
	Note that this means $\int^v\partial(v) = 1$.
\end{proof}
Next, we see that the realization of an involution algebroid has the same base space.

\begin{lemma}\label{lem:base-of-groupoid}
	The base space of the groupoid $\partial$ is $D^v$.
\end{lemma}
\begin{proof}
	When constructing the colimit of groupoids, the colimit's base space is the ordinary colimit for the diagram of the base spaces. The reflector from simplicial objects to groupoids preserves products, so it suffices to check that the base space of $\partial(n)$ is $D(n)$.

	The base space of $I$ is $1+1$, as:
	\[
		I(0) = \widehat{\square}(1,I) \cong \square(0,I) = 1 + 1
	\]
	and the map $e^-_0$ is given by $\delta^- \o !$.
	$D$ is a discrete cubical object, so the base space is $D \x 1$.
	Thus the coequalizer defining $\partial$ is:
	\[\input{TikzDrawings/Ch5/coeq-def-partial.tikz}\]
	A map $\gamma: D + D \to M$ is a pair of maps $\gamma_0, \gamma_1: D \to M$.
	We can see that $\gamma_0, \gamma_1$ agree at $0$ (they are both $\gamma_0(0)$), and $\gamma(0)$ is a constant tangent vector. It follows then, that $\gamma_1 \o (id | id) = \gamma$.
\end{proof}

Now, recall the co-Yoneda lemma, for any $\vv$-presheaf $A: \C^{op} \to \vv$
\[
	F(C) = \int^{C' \in \C} \C(C,C') \ox F(C')
\]
In particular, for an involution algebroid in $\w$ (or any $\w$-presheaf on $\wone^{op}$)
\[
	A(U) = \int^{V \in \wone} \wone^{op}(U,V) \x A(V) = \int^{V \in \wone} \wone(V,U) \x A(V)
\]
\begin{lemma}\label{lem:as-a-presheaf}
	For any involution algebroid $A$:
	\[ A(R) = \int^{V \in \wone} A(V) \x D^V \]
\end{lemma}
\begin{proof}
	Use the tangent structure on $\wone$ to regard it as a $\w$-enriched category:
	\begin{gather*}
		D^V = \yon(V) = \wone(V,-) = (U \mapsto \wone(V, U))  \\= (U \mapsto \wone(V, U\ox R)) = \wone(V,R)
	\end{gather*}
	The following computation gives the result:
	\[
		A(R) = \int^{V} \wone(V,R) \x A(V) = \int^{V} D^V \x A(V)
	\]
\end{proof}
We can now see that the base space of an involution algebroid and its realization are isomorphic.

\begin{proposition}\label{prop:lie-int-first-part}
	Let $A$ be an involution algebroid in $\w$. Then $|A|([0]) = A(R)$.
\end{proposition}
\begin{proof}
	Use the Yoneda lemma, and that $\yon[0] = 1$ is a small projective so $\mathsf{Gpd}(1,-)$ is $\w$-cocontinuous, then apply Lemma \ref{lem:base-of-groupoid} and Lemma \ref{lem:as-a-presheaf}
	\begin{align*}
		|A|([0]) & = \mathsf{Gpd}(1, |A|)                                                   \\
		         & = \mathsf{Gpd}\left(1, \int^{v \in \wone} A(v)\bullet \partial v \right) \\
		         & = \int^{v} A(v)\x \mathsf{Gpd}(1,  \partial v )                          \\
		         & = \int^{v}  A(v)\x D^v = A(R)
	\end{align*}
\end{proof}

Thus, as a final result, we have achieved an adjunction between the category of involution algebroids and groupoids in $\w$, that is product-preserving and stable over the base spaces.
\begin{theorem}[The Lie Realization]%
	\label{thm:lie-realization}
	There is an adjunction between the category of involution algebroids and groupoids internal to $\w$, where each functor preserves products and the base spaces.
	\[% https://q.uiver.app/?q=WzAsMixbMCwwLCJcXG1hdGhzZntHcGR9KFxcdykiXSxbMSwwLCJcXG1hdGhzZntJbnZ9KFxcdykiXSxbMCwxLCJOX1xccGFydGlhbCIsMCx7ImN1cnZlIjotMn1dLFsxLDAsInwtfF9cXHBhcnRpYWwiLDAseyJjdXJ2ZSI6LTJ9XSxbMiwzLCIiLDAseyJsZXZlbCI6MSwic3R5bGUiOnsibmFtZSI6ImFkanVuY3Rpb24ifX1dXQ==
	\begin{tikzcd}
		{\mathsf{Gpd}(\w)} & {\mathsf{Inv}(\w)}
		\arrow[""{name=0, anchor=center, inner sep=0}, "{N_\partial}", curve={height=-12pt}, from=1-1, to=1-2]
		\arrow[""{name=1, anchor=center, inner sep=0}, "{|-|_\partial}", curve={height=-12pt}, from=1-2, to=1-1]
		\arrow["\dashv"{anchor=center, rotate=-90}, draw=none, from=0, to=1]
	\end{tikzcd}\]
\end{theorem}

