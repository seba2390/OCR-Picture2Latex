% \documentclass[main.tex]{subfiles}

% \begin{document}

\chapter{Involution algebroids}%
\label{ch:involution-algebroids}

This chapter accomplishes the first major goal of this thesis by providing a tangent-categorical axiomatization of Lie algebroids, namely involution algebroids. Much like vector bundles, Lie algebroids are a highly non-algebraic notion (in the sense of \cite{Freyd1972}), being vector bundles equipped with a Lie algebra structure on the set of sections of the projection. Furthermore, the bracket on sections must satisfy a product rule with respect to $\mathbb{R}$-valued functions on the base space (the \emph{Leibniz law}), introducing another piece of non-algebraic structure to the definition. The tangent-categorical definition of Lie algebroids will treat the tangent bundle as the ``prototypical Lie algebroid'' in which the vertical lift $\ell:T \Rightarrow T^2$ identifies the vector bundle structure and the canonical flip $c:T^2 \Rightarrow T^2$ plays the role of the Lie bracket.

Lie algebroids are the natural many-object analogue to Lie algebras, in the same way that Lie groupoids are the many-object analogue of Lie groups. In the single-object case, a Lie group is classically thought of as a space of symmetries for some smooth manifold (one often identifies a group action $G \x M \to M$), and a Lie algebra may similarly be thought of as a space of \emph{derivations} (often identified as a sub-Lie-algebra of $\chi(M)$ for a manifold $M$).  The extension of groups to groupoids is natural; in fact, Brandt's introduction of groupoids in \cite{brandt1927verallgemeinerung} predates MacLane and Eilenberg's invention of category theory in \cite{eilenberg1945general} by nearly two decades.  The translation of Lie algebras to the many-object case is not as straightforward. The first step is to replace the vector space underlying a Lie algebra with a vector bundle $(\pi:A \to M, \xi,\lambda)$. The idea is to axiomatize this vector bundle so that each section in $\Gamma(\pi)$ corresponds to a derivation on $C^\infty(M)$. The \emph{anti-commutator} operation on derivations from Proposition \ref{prop:anti-commm-lie} suggests there should be a Lie bracket $[-,-]:\Gamma(\pi)\ox\Gamma(\pi) \to \Gamma(\pi)$ (similar to the partially-defined multiplication for a groupoid), while the correspondence with derivations on $C^\infty(M)$ suggests there be a vector bundle morphism $\anc:A \to TM$ satisfying the \emph{Leibniz law}:
\begin{equation}\label{eq:bracket-f}
    [X, f\cdot Y] = f\cdot [X, Y] + [X,f]\cdot Y; \hspace{0.5cm} [X,f] := \phat \o T.f \o \anc \o X\footnote{Recall the notation from Lemma \ref{lem:Cinfty-module-vbun}.}
\end{equation}
(the full definition of Lie algebroids may be found in \ref{def:lie-algd}). This ``operational'' definition of Lie algebroids makes it difficult to describe their morphisms, and furthermore it essentially fails to be an algebraic structure in the classical sense, as it axiomatizes structure on the \emph{set of sections} of a map rather than a morphism in the category itself.

Involution algebroids were introduced to provide a tangent-categorical presentation of Lie algebroids, similar to the relationship between differential bundles and vector bundles. Chapter \ref{ch:differential_bundles} focused on the Euler vector field construction on a vector bundle, showing that this induced a fully-faithful functor from vector bundles to associative coalgebras (lifts) of the weak comonad $(T,\ell)$, and identified vector bundles with a subcategory of $\mathsf{Lift}(\mathsf{SMan})$ satisfying a universal property. The corresponding construction for Lie algebroids, then, is the \emph{canonical involution}, which was identified by Eduardo Martinez and his collaborators (a clearly written exposition may be found in Section 4 of \cite{de2005lagrangian}). Given a Lie algebroid $(\pi:A \to M, \anc:A \to TM, [-,-]:\Gamma(\pi)\ox\Gamma(\pi) \to \Gamma(\pi))$, its canonical involution is a map
\[
    \sigma: \prolong \to \prolong.
\]
Using this $\sigma$ map, there is a straightforward characterization of Lie algebroid morphisms: a Lie algebroid morphism is precisely a vector bundle morphism $(f,m):A \to B$ that preserves the anchor and involution maps:
\[% https://q.uiver.app/?q=WzAsOCxbMCwxLCJUTSJdLFswLDAsIkEiXSxbMSwwLCJCIl0sWzEsMSwiVE4iXSxbMiwwLCJcXHByb2xvbmciXSxbMiwxLCJcXHByb2xvbmciXSxbMywwLCJCXFx0c3tcXGFuY15CfXtULlxccGleQn1UQiJdLFszLDEsIkJcXHRze1xcYW5jXkJ9e1QuXFxwaV5CfVRCIl0sWzEsMCwiXFxhbmNeQSIsMl0sWzIsMywiXFxhbmNeQiJdLFsxLDIsImYiXSxbMCwzLCJULm0iLDJdLFs2LDcsIlxcc2lnbWFeQiJdLFs0LDYsImYgXFx4IFQuZiJdLFs0LDUsIlxcc2lnbWFeQSIsMl0sWzUsNywiZlxceCBULmYiLDJdXQ==
\begin{tikzcd}
	A & B & \prolong & {B\ts{\anc^B}{T.\pi^B}TB} \\
	TM & TN & \prolong & {B\ts{\anc^B}{T.\pi^B}TB}
	\arrow["{\anc^A}"', from=1-1, to=2-1]
	\arrow["{\anc^B}", from=1-2, to=2-2]
	\arrow["f", from=1-1, to=1-2]
	\arrow["{T.m}"', from=2-1, to=2-2]
	\arrow["{\sigma^B}", from=1-4, to=2-4]
	\arrow["{f \x T.f}", from=1-3, to=1-4]
	\arrow["{\sigma^A}"', from=1-3, to=2-3]
	\arrow["{f\x T.f}"', from=2-3, to=2-4]
\end{tikzcd}\]
Furthermore, it is implicit in Martinez's work (\cite{Martinez2001}) that $\sigma$ satisfies axioms corresponding to the Lie algebroid axioms. Thus, involutivity corresponds to antisymmetry of the Lie bracket
\[
    \sigma \o \sigma = id \iff [X,Y] + [Y,X] = 0, 
\]
while the Leibniz law holds if and only if the $\anc$ map sends the algebroid involution to the canonical flip on $M$,
\[
    T.\anc \o \pi_1 \o \sigma = c \o T.\pi \o \anc \iff 
    \forall f \in C^\infty(M), [X, f\cdot Y] = f\cdot [X, Y] + [X,f]\cdot Y
\]
(using the same definition as before for $[X,f]$).

The idea of an involution algebroid, then, is to axiomatize the canonical involution directly, just as differential bundles axiomatize the Euler vector field of a differential bundle. An involution algebroid is a differential bundle equipped with a pair of structure maps
\[
    \anc:A \to TM, \hspace{0.15cm} \sigma: \prolong \to \prolong
\]
satisfying a collection of axioms. Some of them are straightforward translations of the structure equations for Lie algebroids given in \cite{Martinez2001}, for instance
\[
  T.\anc \o \lambda = \ell \o \anc, \hspace{0.15cm}
  \sigma \o \sigma = id,\hspace{0.15cm}
  T.\anc \o \pi_1 \o \sigma = c \o T.\anc \o \pi_1.
\]
However, this requires a new coherence between the Euler vector field of the underlying vector bundle and the involution map:
\[
    \sigma \o (\xi\o\pi,\lambda) = (\xi\o\pi,\lambda).
\]
The most striking new fact about this coherence is that the Jacobi identity on the bracket $[-,-]$ corresponds to the \emph{Yang--Baxter} equation on $\sigma$:
% https://q.uiver.app/?q=WzAsNixbMCwwLCJcXHByb2xvbmcgXFx0c3tULlxcYW5jfXtUXjIuXFxwaX0gVF4yQSJdLFsxLDAsIlxccHJvbG9uZyBcXHRze1QuXFxhbmN9e1ReMi5cXHBpfSBUXjJBIl0sWzEsMSwiXFxwcm9sb25nIFxcdHN7VC5cXGFuY317VF4yLlxccGl9IFReMkEiXSxbMSwyLCJcXHByb2xvbmcgXFx0c3tULlxcYW5jfXtUXjIuXFxwaX0gVF4yQSJdLFswLDEsIlxccHJvbG9uZyBcXHRze1QuXFxhbmN9e1ReMi5cXHBpfSBUXjJBIl0sWzAsMiwiXFxwcm9sb25nIFxcdHN7VC5cXGFuY317VF4yLlxccGl9IFReMkEiXSxbMCwxLCJcXHNpZ21hIFxceCBjLkEiLDJdLFsxLDJdLFsyLDMsIlxcc2lnbWEgXFx4IGMuQSIsMl0sWzAsNCwiMSBcXHggVC5cXHNpZ21hIl0sWzQsNSwiXFxzaWdtYSBcXHggYy5BIl0sWzUsMywiMSBcXHggVC5cXHNpZ21hIl1d
\[\begin{tikzcd}
	{\prolong \ts{T.\anc}{T^2.\pi} T^2A} & {\prolong \ts{T.\anc}{T^2.\pi} T^2A} \\
	{\prolong \ts{T.\anc}{T^2.\pi} T^2A} & {\prolong \ts{T.\anc}{T^2.\pi} T^2A} \\
	{\prolong \ts{T.\anc}{T^2.\pi} T^2A} & {\prolong \ts{T.\anc}{T^2.\pi} T^2A}
	\arrow["{\sigma \x c.A}"', from=1-1, to=1-2]
	\arrow[from=1-2, to=2-2]
	\arrow["{\sigma \x c.A}"', from=2-2, to=3-2]
	\arrow["{1 \x T.\sigma}", from=1-1, to=2-1]
	\arrow["{\sigma \x c.A}", from=2-1, to=3-1]
	\arrow["{1 \x T.\sigma}", from=3-1, to=3-2]
\end{tikzcd}\]
This is both surprising (it is a new characterization of a central object of study in differential geometry and mathematical physics) and yet in a way expected (the work in \cite{Cockett2015}, \cite{Mackenzie2013} indicates that the Lie algebra structure on the set of vector fields over a manifold follows from the Yang--Baxter equation on $c$).  This vector-field-free presentation of the Jacobi identity allows for a structural approach to Lie algebroids that drives the work presented in Chapters \ref{chap:weil-nerve} and \ref{ch:inf-nerve-and-realization}.

As with \Cref{ch:differential_bundles}, the first section is expository, and is concerned with introducing the category of Lie algebroids. 
The second section introduces anchored bundles, together with the space of prolongations of an anchored bundle. The relationship between anchored bundles and involution algebroids is equivalent to that between reflexive graphs and groupoids (the subject of Chapter \ref{ch:inf-nerve-and-realization}), the space of prolongations of an anchored bundle being equivalent to the set of composable arrows for a groupoid. This section is mostly a translation of Martinez's prolongation construction to a general tangent category. The rest of the chapter contains new results, developed in collaboration with Matthew Burke and Richard Garner.

Section 3 introduces involution algebroids, which are anchored bundles equipped with an involution map on their space of prolongations. Section 4 considers an anchored bundle in a tangent category with negatives that is equipped with a connection. The connection gives an involution algebroid a ``local coordinates'' presentation (in the sense of Section \ref{sec:diff-and-tang-struct}) that is equivalent to the local characterization of Lie algebroids from Section 1. The final section of this chapter establishes the main result: the category of Lie algebroids is isomorphic to that of involution algebroids in smooth manifolds.\pagenote{
   The introduction has been substantially expanded to better explain what is happening in this chapter and why we are doing it. 
}

% The Lie functor from Lie groups to Lie algebras has a natural extension to the many-object case. 
% The Lie algebra of a Lie group $G$ is the space of the \emph{left-invariant} vector fields on $G$.
% \[
%    ---
% \]
% It can be show that the space of left-invariant vector fields is isomorphic to the tangent space on $G$ over the identity element $e:1 \to G$. 
% Left-invariance of a vector field on the space of arrows $G$ of a groupoid, the, corresponds to %cite UofT notes
% \[
%     dfd
% \]
% This corresponds to the space of tangent vectors above the embedded submanifold $M \hookrightarrow G$, that are \emph{source-constant}
% \[
%   dd    
% \]
% This is equivalent to the pullback:
% \[
%  ff    
% \]




\section{Lie algebroids}\label{sec:Lie_algebroids}

This section reviews the basic theory of Lie algebroids: their definition and that of their morphisms, along with some introductory examples. The classical definition will not appear elsewhere in this chapter, however, as we quickly introduce Martinez's \emph{structure equations} for a Lie algebroid (\cite{Martinez2001}), then translate them into tangent-categorical terms using a connection. \pagenote{
   The substantial revisions to the introduction made the original preamble to this section redundant. There have also been substantial revisions to the thesis up until this point, and those changes have percolated into this section. In particular, the notation for the $C^\infty(M)$-module of sections $\Gamma(\pi)$ was set in the previous chapter, and we stick to those conventions.
}

\begin{definition}\label{def:lie-algd}
    A \emph{Lie algebroid} is a vector bundle $\pi:A \rightarrow M$ equipped with an anchor $\anc:A \to TM$ and a bracket $[-,-]:\Gamma(\pi)\ox\Gamma(\pi) \rightarrow \Gamma(\pi)$ satisfying the following axioms:
    \begin{itemize}
        \item bilinear: $[aX_1+bX_2, Y] = a[X_1, Y]+ b[X_2, Y]$ and $[X, aY_1+bY_2] = a[X, Y_1]+b[X, Y_2]$
        % \item alternating: $[X, X] = 0$
        \item anti-symmetric: $[X, Y]+[Y, X] = 0$
        \item Jacobi: $[X, [Y, Z]] = [[X, Y], Z] + [Y, [X, Z]]$
        \item Leibniz: $[X, f\cdot Y] = f\cdot [X, Y] + [X,f]\cdot Y$\pagenote{Changed the notation of the Lie derivative to this bracket version.}
    \end{itemize}
    (where $[X,f]$ is defined as in Equation \ref{eq:bracket-f}).
\end{definition}

\begin{example}%
    \label{ex:lie-algebroids}
    ~\begin{enumerate}[(i)]
        \item The canonical example of a Lie algebroid is, of course, the tangent bundle using the operational tangent bundle from Definition \ref{def:operational-tang}.
        % The tangent bundle is the linear approximation of the pair groupoid:
        \item A \emph{Lie algebra} is a Lie algebroid over the terminal object: for a group $G$, the bundle of source-constant tangent vectors is the usual Lie functor from Lie groups to Lie algebras, because a groupoid is a one-object group.
        \item The bundle of \emph{source-constant} tangent vectors $s,t: G \to M$ of a Lie groupoid forms a Lie algebroid. This bundle is defined by the pullback
        \input{TikzDrawings/Ch3/Sec1/Lie-Functor-Of-G.tikz}
        where the projection is $\pi$ and the target is given by $\anc$ in the diagram.
        There is an injective $\R$-module morphism from sections of $\pi$, $\Gamma(\pi)$ to vector fields on $G$, $\chi(G)$, and the Lie bracket on $G$ is closed over the image of this lift, putting a Lie bracket on $\Gamma(\pi)$. In particular, we can see that $TM$ is the bundle of source-constant tangent vectors for the pair groupoid on a manifold $M$:
        \input{TikzDrawings/Ch3/Sec1/LieFuncPairGpd.tikz}
        Given $(u,v):X \to T(M \x M)$ above $(m,m):X \to TM$ with $u = 0 \o m$, it follows that $u = 0 \o p \o u$, so $v$ is the unique map induced into $TM$.
        \item Every group is a groupoid over a single object. The Lie algebroid associated with a group $G$, then, is the usual Lie algebra.
        \item From the Hamiltonian formalism of mechanics, every \emph{Poisson manifold} has an associated Lie algebroid.
        A Poisson manifold is a manifold $M$ equipped with a \emph{Poisson algebra} structure on $C^\infty(M)$, namely a Lie algebra that is also a derivation:
        \[
            [f\cdot g, h] = f\cdot [g,h] + [f,h]\cdot g
        \]
        where $\cdot$ is the multiplication in the algebra $C^\infty(M)$ as in Lemma \ref{lem:Cinfty-module-vbun}. The cotangent bundle over a Poisson manifold $M$ is canonically a Lie algebroid, called a Poisson Lie algebroid \cite{Courant1994}. 
        \item   Any Lie algebra bundle---that is, a vector bundle equipped with a Lie bracket on its space of sections---is a Lie algebroid, with anchor map $\xi \o \pi$.     
    \end{enumerate}
\end{example}
Morphisms of Lie algebroids are notoriously difficult to work with, and have an involved definition.
\begin{definition}\label{def:lie-algd-morphism}
    Let $A, B$ be a pair of anchored bundles over $M,N$, and $\Phi: A \to B$ an anchored bundle morphism over a map $\phi:M \to N$. A $\Phi$-decomposition of $X \in \Gamma(\pi_A)$ is a set of $X_i \in \Gamma(\pi_B)$ and $f_i \in C^\infty(M)$ so that
    \[
        \Phi \o X = \sum_i f_i \cdot X_i \o \phi.
    \]
    An anchor-preserving vector bundle morphism $\Phi$ is a Lie algebroid morphism if and only if for any $X,Y \in \Gamma(\pi_A)$ and $\Phi$-decompositions $\{X_i, f_i\}, \{Y_j, g_j\}$ of $X,Y$, the following equation holds:
    \[        
        \Phi \o [X,Y] 
        = 
        \sum f_i\cdot g_j \cdot ([X_i,Y_j] \o \phi)
        + \sum [X, f_i] \cdot (X_i \o \phi)
        - \sum [Y, g_j](Y_j \o \phi).
    \]
\end{definition}
The equation defining a Lie algebroid morphism holds independently of the choice of $\Phi$-decomposition (see \cite{Higgins1990} for a proof).\pagenote{Here I have tidied up the definition by explaining what a $\Phi$-decomposition is (in particular, quantifying over the $X_i, f_i$, and also clarified some wording.}
% When working over a fixed base, this definition is tractable:
% \begin{example}\label{ex:la-morphism}
%     If $A, B$ are Lie algebroids over a base manifold $M$, $f: A \to B$ is a Lie algebroid morphism if and only if it preserves the anchor and Lie bracket.
%     In particular, the anchor of a Lie algebroid $\an c: A \to TM$ is a Lie algebroid morphism.
% \end{example}

\begin{example}
    ~\begin{enumerate}[(i)]
        \item If $A, B$ are Lie algebroids over a base manifold $M$, $\Phi: A \to B$ is a Lie algebroid morphism if and only if it preserves the anchor and Lie bracket.\pagenote{I removed the remark about tangent Lie algebroids, as it refers to De Rham cohomologies which have disappeared from this thesis.}
        % \item Every map $f: M \to N$ gives rise to a morphism of the tangent Lie algebroids, as this induces a cochain morphism between their De Rham cohomologies.
        \item Given two Lie algebroid morphisms $f: A \to B, g: B \to C$, their composition $g \o f$ is also a Lie algebroid morphism.
        \item A \emph{Poisson Sigma model} (see \cite{Bojowald2005}) is a morphism of Lie algebroids 
        \[
            \phi: T\Sigma \to T^*M
        \]
        for which $\Sigma$ is a 2-dimensional manifold and $T\Sigma$ denotes its tangent Lie algebroid, while $M$ is a Poisson manifold and $T^*M$ denotes the Lie algebroid structure on its cotangent bundle .
    \end{enumerate}
\end{example}

% % This section moves away from bracket-based notation for Lie algebroids - the first step is using Martinez's local-coordinates presentation of the Lie bracket. 
% This bracket-based presentation of Lie algebroids is ill-suited for functoria
Lie algebroids are a natural generalization of Lie algebras to the ``multi-object'' setting, but they are ill-suited for a functorial presentation of the theory. A step in this direction is to consider the coordinate-based presentation of the Lie bracket and its coherences due to  \cite{Martinez2001}.
Let $A$ be an anchored bundle over $M$ equipped with a bilinear bracket on its space of sections, and choose a pair of bases for $\Gamma(\pi)$ and $\chi(M)$: for $\Gamma(\pi)$ write $\{ e_\alpha\}$, and for $\chi(M)$ write $\{ \frac{\partial}{\partial x^i}\}$. The anchor and bracket then have a presentation in local coordinates: 
\[
    \anc(e_\alpha) = \sum_i \anc^i_\alpha \frac{\partial}{\partial x^i}
    \hspace{0.5cm}
    [e_\alpha, e_\beta] = \sum_\gamma C^{\gamma}_{\alpha\beta}e_\gamma
\]
(from here on out, we use Einstein summation notation to simplify our calculations, so instead write
\[
    \anc(e_\alpha) = \anc^i_\alpha \frac{\partial}{\partial x^i}
    \hspace{0.5cm}
    [e_\alpha, e_\beta] = C^{\gamma}_{\alpha\beta}e_\gamma
\]
with $\sum$ suppressed).
The following characterization of the Lie algebroid axioms uses Martinez's \emph{structure equations}.\pagenote{This is better called the "structure equations" for a Lie algebroid.}
\begin{proposition}\label{prop:la-iff-structure-morphisms}[\cite{Martinez2001}]
    An anchored bundle $A$ over $M$ equipped with a bracket is a Lie algebroid if and only if  $\anc$ and $[-,-]$ satisfy the following structure equations:
    \begin{enumerate}[(i)]
        \item Alternating: \[C^\nu_{\alpha\beta} + C^\nu_{\beta\alpha} = 0\]
        \item Leibniz: \[\anc^j_\alpha \frac{\partial \anc^i_\beta}{\partial x^j} 
            = \anc^i_\gamma C^\gamma_{\alpha\beta} + \anc^j_\beta \frac{\partial \anc^i_\alpha}{\partial x^j}\]
        \item Bianchi: \[ 0
            = \anc^i_\alpha \frac{\partial C^{\nu}_{\beta\gamma}}{\partial x^i}
            + \anc^i_\beta \frac{\partial C^{\nu}_{\gamma\alpha}}{\partial x^i}
            + \anc^i_\gamma \frac{\partial C^{\nu}_{\alpha\beta}}{\partial x^i}
            + C^{\mu}_{\beta\gamma}C^{\nu}_{\alpha\mu}
            + C^{\mu}_{\gamma\alpha}C^{\nu}_{\beta\mu}
            + C^{\mu}_{\alpha\beta}C^{\nu}_{\gamma\mu}\]
    \end{enumerate}
\end{proposition}
This proposition is a straightforward translation of the Lie algebroid axioms into local coordinates using a covariant derivative.
First, recall that by the smooth Serre--Swan theorem (11.33 \cite{Nestruev2003}), the bilinearity of the bracket \pagenote{Clarified where this new bracket comes from/what category it lives in.}
\[
    [-,-]:\Gamma(\pi) \x \Gamma(\pi) \to \Gamma(\pi)  
\] as a morphism of $C^\infty(M)$-modules guarantees that it corresponds to a bilinear morphism $A_2 \to A$ of vector bundles, meaning that there exists a globally defined bilinear map $A_2 \to A$ that is equal to the Lie bracket when applied to sections of the projection. We record this as a lemma: \pagenote{Clarified the point that this theorem gives a global definition of the Lie bracket, extending from from $\Gamma(\pi)$ to general elements in the category}
\begin{lemma}\label{lem:ang-brack-lie}
    For every Lie algebroid $\a$, there is a bilinear morphism
    \[
       \langle -,-\rangle: A_2 \to A 
    \]
    so that for any sections $X,Y\in \Gamma(\pi)$, $\langle X,Y\rangle = [X,Y]$.
\end{lemma}

There are two structure maps derived from $\< -, -\>$ that encode the coherences of a Lie algebroid. The first map measures the extent to which the anchor maps fail to preserve the chosen connections on the vector bundle $\pi:A \to M$ and the tangent bundle $p:TM \to M$.\pagenote{Added some explanation of what the curly bracket map does.}
\begin{definition}\label{def:curly-bracket}
    Let $A$ be a Lie algebroid, and for a chosen horizontal connection $\nabla$ on $A$ and vertical connection $\kappa'$ on $TM$, set
    \[
        \{ v, x\}_{\kappa',\nabla} := TM \ts{p}{q} A \xrightarrow[]{\nabla} TA \xrightarrow[]{T.\anc} T^2M \xrightarrow[]{\kappa'} TM.
    \]
    When the choice of connection is evident by context, we will suppress the subscript.
\end{definition}
\begin{observation}
    The above parentheses bracket corresponds to the symbol
    \[
        \{-,-\} = \anc^j_\beta \frac{\partial \anc^i_\alpha}{\partial x^j}.  
    \]
\end{observation}
The Leibniz coherence may be rewritten as follows:
\begin{lemma}\label{lem:curly-bracket-coh}
    Let $A$ be an anchored bundle with a bilinear bracket (inducing an involution $\sigma$).
    Choose connections $(\kappa, \nabla), (\kappa', \nabla')$ on $A$ and $TM$ respectively.
    The bracket and anchor map satisfy the Leibniz law if and only if
    \[
        \anc \o \< x, y\>_{(\kappa, \nabla)} 
        + \{ \anc x, y\}_{(\kappa', \nabla)}
        = \{ \anc y, x\}_{(\kappa', \nabla)}.
    \]
\end{lemma}
\begin{proof}
    The condition is equivalent to the identity
    \[\anc^j_\alpha \frac{\partial \anc^i_\beta}{\partial x^j} 
            = \anc^i_\gamma C^\gamma_{\alpha\beta} + \anc^j_\beta \frac{\partial \anc^i_\alpha}{\partial x^j}.\]
\end{proof}
The Bianchi axiom measures the failure of the Jacobi identity in local coordinates, and states that it must be corrected for by the curvature of the brackets. 
We see that
\[
    C^{\nu}_{\alpha\mu} C^{\mu}_{\beta\gamma} = [e_\alpha, [e_\beta, e_\gamma]]
\]
while 
\[
    \frac{\partial C^{\nu}_{\beta\gamma}}{\partial x^i} e_\nu
    = \kappa \o T(\langle -,-\rangle_{(\kappa, \nabla)})\o \nabla^{A_2} \o ( \frac{\partial}{\partial x^i}, e_\beta, e_\gamma)
\]
determines a trilinear map
\[
    \{\frac{\partial}{\partial x^i}, e_\beta, e_\gamma\}_{(\kappa, \nabla)} 
    := \kappa \o T(\langle -,-\rangle_{(\kappa, \nabla)})\o \nabla^{A_2} \o ( \frac{\partial}{\partial x^i}, e_\beta, e_\gamma).
\]
This is the second derived map used in the structure equations for a Lie algebroid.
\begin{definition}
    Let $(\pi:A \to M, \xi, \lambda, \anc)$ be an anchored bundle equipped with a bilinear map
    \[
        \langle -, - \rangle: A_2 \to A.
    \]
    The derived ternary bracket $\{ -, -, - \}: TM \ts{p}{\pi} A \ts{\pi}{\pi} A \to A$ is defined as
    \[
        \{v,x,y\}_{(\kappa, \nabla)} := 
        TM\ts{p}{\pi} A \ts{\pi}{\pi} A \xrightarrow[]{\nabla[A2]} TA_2 \xrightarrow[]{T.\<-,-\>} TA \xrightarrow{\kappa} A
        % \kappa \o T(\langle -,-\rangle_{(\kappa, \nabla)})\o \nabla^{A_2} \o (v,x,y)
    \]
    where $\nabla[A2]$ is the pairing $(\nabla(\pi_0,\pi_1), \nabla(\pi_0,\pi_2))$.
\end{definition}
\begin{observation}
    The ternary bracket corresponds to the following symbol:
    \[
        \{-,-,-\}_{(\kappa, \nabla)} :=
        \anc^i_\alpha \frac{\partial C^{\nu}_{\beta\gamma}}{\partial x^i}.
    \]
\end{observation}
\begin{lemma}\label{lem:bianchi-connection}
    Let $A$ be an anchored bundle over $M$ with a bilinear bracket and denote the induced involution by $\sigma$.
    Choose a pair of connections and write the derived maps $\langle-,-\rangle, \{-,-,-\}$.
    Then
    \begin{enumerate}[(i)]
        \item the bracket is antisymmetric if and only if its globalization is; that is, $\langle e_\alpha, e_\beta\rangle + \langle e_\beta, e_\alpha\rangle = 0$;
        \item the bracket satisfies the Jacobi identity if and only if it is alternating in the last two arguments and
            \[
                0
                = \sum_{\gamma \in \mathsf{Cy}(3)} \< x_{\gamma_0}, \< x_{\gamma_1}, x_{\gamma_2}\> \>
                + \sum_{\gamma \in \mathsf{Cy}(3)} \{ \anc x_{\gamma_0}, x_{\gamma_1}, x_{\gamma_2}\}.
            \]
    \end{enumerate}
\end{lemma}
\begin{proof}~
    \begin{enumerate}[(i)]
        \item This is equivalent to $C^\nu_{\alpha\beta} + C^\nu_{\beta\alpha} = 0$.
        \item This is equivalent to \[ 0
            =   C^{\mu}_{\beta\gamma}C^{\nu}_{\alpha\mu}
            + C^{\mu}_{\gamma\alpha}C^{\nu}_{\beta\mu}
            + C^{\mu}_{\alpha\beta}C^{\nu}_{\gamma\mu}
            + \anc^i_\alpha \frac{\partial C^{\nu}_{\beta\gamma}}{\partial x^i}
            + \anc^i_\beta \frac{\partial C^{\nu}_{\gamma\alpha}}{\partial x^i}
            + \anc^i_\gamma \frac{\partial C^{\nu}_{\alpha\beta}}{\partial x^i}. \] 
    \end{enumerate}
\end{proof}

\begin{proposition}%
    \label{prop:lie-alg-bil-defn}
    A Lie algebroid is exactly an anchored vector bundle $(\pi:A \to M, \xi, \lambda, \anc)$ with a bilinear, alternating map
    \[
        \langle -, -\rangle: A_2 \to A; \hspace{0.5cm} \< x, y \> + \< y, x\> = 0
    \]
    so that for any connection $(\nabla,\kappa)$ on $A$, the maps 
    \begin{equation}
        \label{eq:lie-algd-structure-maps}
        \{ v, x\}_{(\kappa, \nabla)} := \kappa' \o T.\anc \o \nabla(v, x), \hspace{0.5cm}
        \{v,x,y\}_{(\kappa, \nabla)} := \kappa \o T(\langle -,-\rangle_{(\kappa, \nabla)})\o \nabla^{A_2} \o (v,x,y)
    \end{equation}
    satisfy the equations
    \begin{enumerate}[(i)]
        \item $ \anc \o \< x, y\> 
        + \{ \anc x, y\}_{(\kappa', \nabla)}
        = \{ \anc y, x\}_{(\kappa', \nabla)}$,
        \item $\sum_{\gamma \in \mathsf{Cy}(3)} \< x_{\gamma_0}, \< x_{\gamma_1}, x_{\gamma_2}\> \>
        + \sum_{\gamma \in \mathsf{Cy}(3)} \{ \anc x_{\gamma_0}, x_{\gamma_1}, x_{\gamma_2}\}_{(\kappa, \nabla)} = 0$.
    \end{enumerate}
\end{proposition}

There is also a local coordinates presentation of morphisms as in Section 2 of \cite{Martinez2018}. An anchored bundle morphism $A \to B$ is a Lie algebroid morphism whenever
\[
    f^\beta_\gamma A^\gamma_{\alpha \delta} + \anc^i_\delta \frac{\partial f^\beta_\alpha}{\partial x^i}
    = B^\beta_{\theta\sigma}f^\theta_{\alpha}f^{\sigma}_{\delta} + \anc^i_\alpha \frac{\partial f^\beta_\delta}{\partial x^i}.
\]
The $A$ and $B$ arguments are understood as the brackets, so this condition can be rewritten as
\[
    f^\beta_\gamma A^\gamma_{\alpha\delta} = f \o \< \alpha, \delta \>, \hspace{0.25cm}
    B^\beta_{\theta\sigma}f^\theta_{\alpha}f^{\sigma}_{\delta} = \< f \o \alpha, f \o \delta \>.
\]
Set the following notation for maps between vector bundles with connection:\pagenote{Reintroduce the $\nabla$ notation here after removing it.}
\begin{equation}\label{eq:nabla-notation}
    \infer{\nabla[f]:A_2 \to B := \kappa^B \o T.f \o \nabla^A}{f:A \to B & (\kappa^A,\nabla^A,A, \lambda^A) & (\kappa^B,\nabla^B, B, \lambda^B)}
    % 
\end{equation}
The $\anc$ terms are understood to be the torsion, so that
\begin{gather*}
        \anc^i_\delta \frac{\partial f^\beta_\alpha}{\partial x^i} 
    = \kappa \o T.f \o \nabla(\anc e_\delta, e_\alpha) = \nabla[f](\anc e_\delta, e_\alpha),\\
    \anc^i_\alpha \frac{\partial f^\beta_\delta}{\partial x^i} 
    = \kappa \o T.f \o \nabla(\anc \alpha, \delta) = \nabla[f](\anc e_\alpha, e_\delta),
\end{gather*}
using the notation set up in Equation \ref{eq:nabla-notation}.
The notion of a Lie algebroid morphism, then, has the following presentation:
\begin{proposition}[\cite{Martinez2018}]%
    \label{prop:lie-algd-morphism-defn}
    Let $(\pi:A \to M,\anc^A, \<-,-\>^A), (q:B \to N,\anc^B,\<-,-\>^B)$ be a pair of Lie algebroids with chosen connections $(\kappa^{-},\nabla^{-})$.
    An anchor-preserving vector bundle morphism $f:A \to B$ is a Lie algebroid morphism if and only if
    \[
        f \o \< e_\alpha, e_\delta \> + \nabla[f] \o (\anc e_\delta, e_\alpha)
        = \< f \o \alpha, f \o \delta \> + \nabla[f] \o(\anc e_\alpha, e_\delta).
    \]
\end{proposition}

\section{Anchored bundles}%
\label{sec:anchored-bundles}
Anchored bundles are to Lie algebroids what reflexive graphs are to groupoids. Each theory has (mostly) the same structure, but while a reflexive graph is missing a groupoid's composition operation, an anchored bundle lacks a Lie algebroid's bracket operation. This section reviews the basic theory of anchored bundles and their prolongations (see \cite{Mackenzie2005} for more details).
\begin{definition}\label{def:anchored_bundles}
    An anchored bundle in a tangent category is a differential bundle $(A \xrightarrow{\pi} M, \xi, \lambda)$ equipped with a linear morphism 
    \begin{equation*}
        \input{TikzDrawings/Ch3/Sec4/ancbun.tikz}
    \end{equation*}
    A morphism of anchored bundles is a linear bundle morphism $(f,v)$ that preserves the anchors
    \[\input{TikzDrawings/Ch3/Sec4/ancbun-mor.tikz}\]
    The category of anchored bundles and anchored bundle morphisms in a tangent category $\C$ is written $\mathsf{Anc}(\C)$, and a generic anchored bundle is written $(A \xrightarrow{\pi} M, \xi, \lambda,\anc)$.
    % Furthermore, the first and second \textit{prolongations} must exist:
    % 
\end{definition}
There are two pullbacks that are associated with every anchored bundle.
These play the role of the spaces of composable arrows $G_2 := G \ts{t}{s} G, G_3 := G \ts{t}{s} G \ts{t}{s} G$ for a reflexive graph $s,t:G \to M$.
\begin{definition}\label{def:prolongations}
    Let $(A \xrightarrow{\pi} M, \xi, \lambda, \anc)$ be an anchored bundle. Its \emph{first and second prolongations} are given by the limits
    \[\input{TikzDrawings/Ch3/Sec4/prols.tikz}\]
    (The notation for the fibre product is slightly non-standard, as it is not technically a pullback.)
    Throughout this chapter, it will always be assumed that the first and second prolongations of an anchored bundle exist (although no choice of prolongation is explicitly made).\pagenote{I have explicitly defined the category of anchored bundles, and split the definition of prolongations from the definition of an anchored bundle. Also clarified that we are simply assuming these first two $T$-limits always exist, but we do not make a specific choice of prolongation.}
\end{definition}


\begin{remark}
    It is not strictly necessary that the prolongations of an anchored bundle exist; this condition is primarily a matter of convenience when discussing involution algebroids. Every result in this section, that does not explicitly mention prolongations, holds for an anchored bundle independently of their existence.
\end{remark}
\begin{example}%
    \label{ex:anchored-bundles}\pagenote{
       This set of examples has been substantially revised to fix some clumsy wording in the original version. It made sense to included the full result about $\C$ be a reflective subcategory of $\mathsf{Anc}(\C)$ and the coreflection from the category of differential bundles to anchored bundles rather than having them as separate results, as they each describe classes of anchored bundles.
    }
    ~\begin{enumerate}[(i)]
        \item For any object $M$, $id: TM \to TM$ is an anchor for the tangent differential bundle, and every $f: M \to N$ yields a morphism of anchored bundles. Moreover, for any anchored bundle over $M$ the anchor is itself a morphism of anchored bundles $(A,\pi,\xi,\lambda,\anc) \to (TM, p, 0, \ell, id)$. This induces a fully faithful functor:
        \[
            \C \hookrightarrow \mathsf{Anc}(\C).
        \]
        This inclusion has a left adjoint, which sends an anchored bundle $(\pi:A \to M, \xi,\lambda,\anc)$ to its base space $M$ (the unit is the anchor map $\anc$), so that $\C$ is a reflective subcategory of $\mathsf{Anc}(\C)$.
        \item For any differential bundle $(A, \pi, \xi, \lambda)$, the map $0\o\pi:A \to TM$ is an anchor and every morphism $f: A \to B$ of differential bundles again yields a morphism of anchored bundles. The naturality of $0$ ensures that every differential bundle morphism will preserve this trivial anchor map, giving a fully faithful functor
        \[
            \mathsf{DBun}(\C) \hookrightarrow \mathsf{Anc}(\C).
        \]
        This functor has a right adjoint that replaces the anchor map with the trivial anchor map
        \[
            (\pi:A \to M, \xi, \lambda, \anc) \mapsto (\pi:A \to M, \xi, \lambda, 0 \o \pi)  
        \]
        where the counit is given by the natural idempotent $e = \xi \o \pi:(A,\lambda) \to (A,\lambda)$. It is trivial to check that the anchor map is preserved by the bundle morphism $(e,id)$:
        \[
            \anc \o \xi \o \pi = 0 \o T.id \o \pi 
        \]
        and so the following diagram commutes:\pagenote{I took the opportunity to clarify a point brought up in my defense about this relationship.}
        % https://q.uiver.app/?q=WzAsNCxbMCwxLCJBIl0sWzEsMSwiQSJdLFswLDAsIlRNIl0sWzEsMCwiVE0iXSxbMiwzLCIiLDAseyJsZXZlbCI6Miwic3R5bGUiOnsiaGVhZCI6eyJuYW1lIjoibm9uZSJ9fX1dLFswLDIsIjAgXFxvIFxccGkiXSxbMCwxLCJcXHhpIFxcbyBcXGxhbWJkYSIsMl0sWzEsMywiXFxhbmMiLDJdXQ==
        \[\begin{tikzcd}
            TM & TM \\
            A & A
            \arrow[Rightarrow, no head, from=1-1, to=1-2]
            \arrow["{0 \o \pi}", from=2-1, to=1-1]
            \arrow["{p \o \lambda}"', from=2-1, to=2-2]
            \arrow["\anc"', from=2-2, to=1-2]
        \end{tikzcd}\]
        This means that  differential bundles are a \emph{coreflective} subcategory of anchored bundles.
        \item Any reflexive graph $(s,t: C \to M, e:M \to C)$ in a tangent category has an anchored bundle (when sufficient limits exist), constructed as
        \[\input{TikzDrawings/Ch3/Sec9/eq-of-idem.tikz}\]
        (where $e^s:C \to C = e \o s$). Construct a lift on $C^\partial$:
        \[\input{TikzDrawings/Ch3/Sec9/lift-of-lin-approx.tikz}\]    
        This lift will be non-singular by the commutativity of $T$-limits. The pre-differential bundle data is given by the projection
        \[ C^\partial \hookrightarrow T.C \xrightarrow[]{p.e^s} M.\]
        The section is induced by
        \[
            M \xrightarrow[]{(0.e)}  T.C
        \]
        while post-composition with $T.t$ gives the anchor map:
        \[
            C^\partial \hookrightarrow T.C \xrightarrow[]{T.t} T.C. 
        \]
        The diagram is a pullback by composition, and the outer perimeter defines the pullback $\prol(A)$. Note that any reflexive graph morphism will give rise to an anchored bundle morphism by naturality, making the construction of an anchored bundle from a reflexive graph functorial.\pagenote{This was originally in the examples of prolongations, which was not correct.}
        \item For any object $M$ in a tangent category, $c_M: T^2M \to T^2M$ is an anchor on $(T.p, T.0, c \o T.\ell)$, and every map $f: M \to N$ gives a morphism of anchored bundles.
        \item For any anchored bundle $(q:E \to M, \xi, \lambda, \anc)$, the differential bundle $(T.q, T.\xi, \newline c \o T.\lambda)$ has an anchor 
        \[\anc_T: TE \xrightarrow[]{T.\anc} T^2M \xrightarrow[]{c}T^2M.\]
    \end{enumerate} 
\end{example}
The first prolongation of an anchored bundle $\prol(A)$ behaves similarly to the second tangent bundle, except that it does not have a canonical flip. In the definition of an affine connection, the tangent bundle played a similar role to the ``arities'' of a theory. There is a lift map that makes this connection stronger:
\begin{definition}\label{def:lhat}
    Let $(\pi:A \to M, \xi, \lambda, \anc)$ be an anchored bundle in a tangent category $\C$.
    We define a \emph{generalized lift}:
    \[
        \hat{\lambda}: A \to \prolong := (\xi \o \pi, \lambda).   
    \]
\end{definition}
This generalized lift satisfies the same coherences as the lift on the second tangent bundle:
\begin{proposition}\label{prop:lift-axioms-anchor}
    Let $(\pi:A \to M, \xi, \lambda, \anc)$ be an anchored bundle in a tangent category $\C$.
    It follows that
    \begin{enumerate}[(i)]
        \item (Coassociativity of $\hat{\lambda}$) $(\hat{\lambda}\x\ell)\o\hat{\lambda} = (id \x T.\hat{\lambda}) \o \hat{\lambda}$;
        \item (Universality) the following diagram is a $T$-pullback:
        \[\input{TikzDrawings/Ch3/Sec5/inv-algd-universality.tikz}\]
        where $\hat \mu := (\xi \o \pi \o \pi_0, \mu)$.
    \end{enumerate}
\end{proposition}
\begin{proof}
    ~\begin{enumerate}[(i)]
        \item Compute
            \begin{align*}
                (\hat{\lambda}\x \ell)\o \hat{\lambda} 
                &= (\xi\o\pi\o\xi\o\pi, \lambda\o\xi\o\pi, \ell\o\lambda) \\
                &= (\xi\o\pi, T.(\xi\o\pi)\o\lambda, T.\lambda \o \lambda) \\
                &= (\pi_0, T.(\xi\o\pi)\o\pi_1, T.\lambda \o \pi_1)\o (\xi\o\pi, \lambda) \\
                &= (id \x T(\hat{\lambda}))\o\hat{\lambda}.
            \end{align*}
        \item  Use the pullback lemma to observe that the following diagram is universal for any anchored bundle:
            % https://q.uiver.app/?q=WzAsNixbMCwwLCJBXzIiXSxbMCwxLCJNIl0sWzEsMCwiXFxwcm9sKEEpIl0sWzEsMSwiQSJdLFsyLDAsIlRBIl0sWzIsMSwiVE0iXSxbMSwzLCJcXHhpIl0sWzMsNSwiXFxhbmMiXSxbMSw1LCIwIiwyLHsiY3VydmUiOjJ9XSxbNCw1LCJUXFxwaSJdLFsyLDMsIlxccGlfMCJdLFswLDEsIlxccGlcXG9cXHBpXzEiLDJdLFswLDIsIlxcaGF0e1xcbXV9IiwyXSxbMiw0LCJcXHBpXzEiLDJdLFswLDQsIlxcbXUiLDAseyJjdXJ2ZSI6LTJ9XV0=
            \input{TikzDrawings/Ch3/Sec5/inv-algd-mu-universal-proof.tikz}
            The top triangle of the diagram commutes by definition\pagenote{Removed a redundancy pointed out by Kristine.}. 
            The right square and outer perimeter are pullbacks by definition, and the bottom triangle also commutes by definition. The pullback lemma ensures that the left square is a pullback, so for every anchored bundle, the general lift is universal for $\prol(A)$. Now post-compose with the involution:
            \input{TikzDrawings/Ch3/Sec5/inv-algd-nu-universal.tikz}
            It suffices to check that the top triangle commutes, so $\sigma \o \hat{\mu} = \nu$:
            \[
                \sigma \o \hat{\mu}\o (a,b) = \sigma \o   ((\xi\o \pi,0)\o a +_{\pi_0} (\xi\o \pi,\lambda)\o b) = 
                 (id, T.\xi \o \anc \o a) +_{p\pi_1} (\xi\o \pi,\lambda)\o b.
            \]
            Thus, the lift $(\xi\pi,\lambda)$ involution algebroid is universal for $\prol(A)$.
    \end{enumerate}
\end{proof}


% A similar notion of anchored bundle connection exists using the prolongation as an argument in a similar manner.
\begin{example}%
    \label{ex:prolongations}
    ~\begin{enumerate}[(i)]
        \item For $T(M) = T(M)$, the space of prolongations is $T(M) \ts{id}{Tp} T^2M = T^2M$, and the second prolongation is given by $T^2M$.
        \item   In a tangent category with a tangent display system, if $\pi \in \d$ then the prolongations for $(\pi, \xi, \lambda, \anc)$ automatically exist.
        In particular, for every anchored bundle in the category of smooth manifolds, all prolongations exist because the projection is a submersion (see \Cref{sec:submersions}).
        \item   For any differential bundle with the anchor $0\o \pi$, it follows that $\prol(A) \cong A\ts{\pi}{\pi\o\pi_i} (A_2) \cong A_3$. 
        %   Consider $(a_0,b):X \to A \ts{0\o \pi}{T\pi} TA$. 
            The universality of the vertical lift factors $b$ into $(a_1,a_2)$, as the following diagram is a pullback:
            \input{TikzDrawings/Ch3/Sec4/zero-anc-prol.tikz}
        \pagenote{The functor from reflexive graphs to anchored bundles was here for some reason, rather than in the examples of anchored bundles piece.}
        \item Returning to the anchored bundle constructed from a graph, the space of prolongations $\prol(A)$ embeds into the second tangent bundle of the space of composable arrows:
        \[ \prolong \hookrightarrow T^2(G \ts{t}{s} G).\]
        % \item   For $c: T.p \to p.T$, the space of prolongations is $T^2M \ts{c}{Tp} T^3M$, which is a ``twisting'' of the third tangent bundle.
    \end{enumerate}
\end{example}
\pagenote{It seemed cleaner to include this proposition in the above examples, since this is really just a remark about }
% First, note that for any tangent category, $\C$ is a reflective subcategory of the category of anchored bundles in $\C$ - the functor sends an anchored bundle to the base object.
% \begin{proposition}%
%     \label{prop:c-refl-in-anc}
%     The category $\C$ is a reflective subcategory of $\mathsf{Anc}(\C)$.
%     \[
%         S: (\pi: A \to M, \xi, \lambda, \anc) \mapsto (p:TM \to M, 0,\ell, id)  
%     \]
% \end{proposition}
% \begin{proof}
%     The endofunctor $S: \mathsf{Anc}(\C) \to \mathsf{Anc}(\C)$ is strictly idempotent $S.S = S$, and $\anc: id \Rightarrow S$ is a unit as $S.\anc = \anc.S = id$. 
% \end{proof}



The category of anchored bundles is, in a sense, ``tangent monadic'' over the category of differential bundles: the forgetful functor from anchored bundles to differential bundles ``creates'' $T$-limits and the tangent structure (this is all made precise in \Cref{ch:inf-nerve-and-realization} using an enriched perspective on tangent categories).
\begin{observation}%
    \label{obs:t-limits-of-anc}
    A limit of anchored bundles is the limit of the underlying differential bundles. Because the anchor is preserved by every map in the diagram, this induces a natural transformation for any $D: \d \to \mathsf{Anc}(C)$:
    \[\input{TikzDrawings/Ch3/limit-of-anc.tikz}\]
    Thus, $\lim U.S.D$ computes the limit of the underlying objects, while $\lim U.D$ computes the limit of the underlying differential bundles in the diagram. The anchor/unit map, then, induces a differential bundle map:
    \[
        \lim \anc_i: (\lim A_i, \lim \lambda_i) \to (T.(\lim_i M_i), \ell.(\lim_i M_i)).
    \]
    This is the limit in the category of anchored bundles (so long as pullback powers of the limit projection and the two prolongations of the limit anchor bundle exist).
    % \[\lim \anc_i: \lim A_i \to \lim T.M_i \cong T.(\lim M_i)\]
\end{observation}
The tangent structure on lifts defined in Proposition \ref{prop:lifts-is-tangent} lifts to a tangent structure on anchored bundles. 
\begin{lemma}\label{lem:tangent-anchored-bundle}
    The category of anchored bundles in a tangent category has a tangent structure that maps objects as follows:
    \[
        (\pi:A \to M, \xi, \lambda, \anc) \mapsto (T.\pi:TA \to TM, T.\xi, c \o T.\lambda, c \o T.\anc)  
    \]
    where the structure maps are all defined using the pointwise structure maps in $\C$. 
    \pagenote{There exists multiplie tangent-like structures on anchored bundles and involution algebroids, so I have added more detail in the statements of those results.}
\end{lemma}
\begin{proof}
    Given an anchored bundle $(q:A \to M,\xi, \lambda, \anc)$, there is an anchor on the differential bundle $(Tq, T\xi, c \o T\lambda)$ given by $c \o T\anc$; the diagram commutes by naturality and the coherences on $c, \ell$.
    \[\input{TikzDrawings/Ch3/Sec4/T-pres-anc.tikz}\]
    The tangent structure maps and universality properties all follow from the forgetful property of the functor from anchored bundles to differential bundles as a consequence of Observation \ref{obs:t-limits-of-anc}.
\end{proof}

For any anchored bundle $A$, there are two differential bundles associated to $\prol(A)$. The first is the usual pullback differential bundle given by pulling back $T.\pi$ along $\anc$ as in Lemma \ref{lem:reindex-db}:
\begin{equation*}
    \input{TikzDrawings/Ch3/Sec4/pullback-prol-dbun.tikz}
\end{equation*}
This gives the differential bundle structure 
\[
    (\prolong \xrightarrow{\pi_0} A, A \xrightarrow{(id, T.\xi \o \anc)} \prolong, \prolong \xrightarrow{(0, c \o T.\lambda)} T(\prolong)).
\]
Taking the fibre product in the category of anchored bundles as in Observation \ref{obs:t-limits-of-anc} yields the second differential bundle structure:
\[
    (A, \pi, \xi, \lambda) \xrightarrow{(\anc, id)} (TM, p, 0, \ell) \xleftarrow{(T\pi, \pi)} (TA, p, 0, \ell).
\]
The two lifts behave similarly to the pair $(T.\ell, \ell.T)$ on the second tangent bundle, and $(\ell,\lambda_T)$ on the tangent bundle of a pre-differential bundle.
\begin{lemma}\label{prop:anc-prol-fun}
% \pagenote{
%     This lemma has been restructured to address possible ambiguities. The two pullback structures on $\prol(A)$ are derived by taking two different pullback in the category of differential bundles. Also there was a tiz
% }
    Let $\C$ be a tangent category and $\mathsf{Anc}(\C)$ the category of anchored bundles in $\C$. If $T$-pullback powers of $p \o \pi_0, \pi_1: \prolong \to A$ exist, then there are two differential bundles on $\prol(A)$, with lifts induced as
    \[\input{TikzDrawings/Ch3/pullbacks-of-lifts-for-prol.tikz}\]
    with structure maps
    \begin{enumerate}
        \item $(\prol(A) \xrightarrow{p\o\pi_1} A, A \xrightarrow{(\xi\pi,0)} \prol(A), \prol(A) \xrightarrow{\lambda \x \ell} T \o \prol(A))$,
        \item $(\prol(A) \xrightarrow{\pi_0} A, A \xrightarrow{(id, T.\xi\o\anc)} \prol(A), \prol(A) \xrightarrow{0 \x c \o T.\lambda} T.\prol(A))$.
    \end{enumerate}
    % \begin{gather*}
    %     (\prol(A) \xrightarrow{p\o\pi_1} A, A \xrightarrow{(\xi\pi,0)} \prol(A), \prol(A) \xrightarrow{\lambda \x \ell} T \o \prol(A)) \\
    %     (\prol(A) \xrightarrow{\pi_0} A, A \xrightarrow{(id, T.\xi\o\anc)} \prol(A), \prol(A) \xrightarrow{0 \x c \o T.\lambda} T.\prol(A))
    % \end{gather*}
    Furthermore, the two lifts $\lambda_\prol$ and $\lambda_\anc$ commute: 
    \[c \o T.(\lambda \x \ell) \o ( 0 \x c \o T.\lambda ) = T.( 0 \x c \o T.\lambda ) \o (\lambda \x \ell).\]
\end{lemma}
\begin{proof} 
    % Consider the following diagram:
    % \input{TikzDrawings/Ch3/Sec4/two-lifts-on-prol.tikz}
    % Regarding the diagram as a pullback in the category of differential bundles gives the first differential bundles structure - look at Proposition \ref{prop:T-limits-of-lifts} and Observation \ref{obs:T-limits-pdbs} to see how limits of differential bundles are computed. 
    % Note that the pullback along a differential bundle projection has the canonical pullback differential bundle structure associated to it by Lemma \ref{lem:reindex-db}, which gives the second differential bundle structure.
    The two differential bundles exist as a consequence of Observation \ref{obs:T-limits-pdbs} and Lemma \ref{lem:reindex-db}, respectively. The commutativity of limits follows by postcomposition, as both $c \o T.\lambda, \ell$ and $\lambda, 0$ commute by the differential bundle axioms.
    % To check that the lifts commute, compute:
    % \begin{gather*}
    %     T.(0\o \pi_0, c \o T.\lambda \o \pi_1) \o (\lambda\o \pi_0, \ell\o \pi_1)
    %     = (T.0 \o \lambda \o \pi_0, T.c \o T.\lambda \o \ell \o \pi_1)\\
    %     = (c \o T.\lambda \o \pi_0, T.c \o \ell \o T.\lambda \o \pi_1) 
    %     = (c \o T.\lambda \o \pi_0, c \o T.\ell \o c \o T.\lambda \o \pi_1) 
    % \end{gather*}
\end{proof}
The above lemma determines a functor, which we denote $\prol$, that sends an anchored bundle in $\C$ to an anchored bundle in $\mathsf{Lift}(\C)$ (equipped with the tangent structure from Proposition \ref{prop:lifts-is-tangent}).
\begin{proposition}%
    \label{cor:prol-functor}
    There is a functor $\prol$ from anchored bundles in $\C$ to anchored bundles in the category of lifts\pagenote{added more context for what $\mathsf{Lift}(\C)$ is} $\mathsf{Lift}(\C)$, that sends an anchored bundle $(\pi:A \to M, \xi, \lambda, \anc)$ to the tuple in $\mathsf{Lift}(\C)$
    \begin{align*}
        &\pi^\prol:= (\prol(A), 0 \x c \o T.\lambda) \xrightarrow[]{p \o \pi_1} (A, 0),\\
        &\xi^\prol := (A,0) \xrightarrow[]{(\xi\o\pi,0)} (\prol(A),0 \x c \o T.\lambda)\\ 
        &\lambda^\prol := (\prol(A), 0\x c\o T.\lambda) \xrightarrow[]{(\lambda \x \ell)} (T(\prol(A)), c \o T(0 \x c \o T.\lambda), \\
        &\anc^\prol := (\prol(A), 0 \x c \o T.\lambda) \xrightarrow[]{\pi_1} (TA, c \o T.\lambda) 
    \end{align*}
    (note that this functor lands in anchored bundles of \emph{non-singular} lifts; see Definition \ref{def:non-singular-lift}).
    Morphisms of anchored bundles \[(f,m):(\pi:A \to M, \xi, \lambda, \anc) \to (q:E \to N, \zeta, l, \delta)\] are sent to
    \[
        \prol(f,m) := f \ts{Tm}{Tm} Tf: \prolong \to E \ts{\delta}{Tq} TE
    \]
\end{proposition}
\begin{proof}
    First, note that the tuple
    \[
        (\pi^\prol: \prol(A) \to A, \xi^\prol, \lambda^\prol, \anc^\prol)  
    \]
    is an anchored bundle, and that each morphism is a lift morphism:
    \begin{itemize}
        \item $\pi^\prol: (\prol(A), 0 \x c \o T.\lambda) \to (A,0)$ follows because
        \begin{align*}
            T.p \o T.\pi_1 \o (\lambda \x \ell) = T.p \o \ell \o \pi_1 = 0 \o p \o \pi_1
        \end{align*}
        \item $\xi^\prol: (A,0) \to (\prol(A), 0 \x c \o T.\lambda)$ follows since
        \[
           ( T.\xi \o T.\pi, T.0) \o 0 = (0 \o \xi \o \pi, T.0 \o 0) = (0 \o \xi \o \pi, c \o T.\lambda \o 0) = (0 \x c \o T.\lambda) \o (\xi \o \pi, 0)
        \]
        \item $\lambda^\prol: (\prol(A), 0 \x c \o T.\lambda) \to (T.\prol(A), 0 \x c \o T.(c \o T.\lambda))$
        follows by the commutativity of $0 \x c \o T.\lambda)$ and $\lambda \x \ell$, and
        \item $\anc^\prol: (\prol(A), 0 \x c \o T.\lambda) \to (TA, c \o T.\lambda)$ is a lift by definition of $\anc^\prol = \pi_1$.
    \end{itemize}
    This gives an anchored bundle in the category of non-singular lifts in $\C$.
    
    Next, check that the mapping is functorial. To see that $\prol(f,m)$ preserves the lifts, note that $(f,m)$ gives a morphism of diagrams for each pullback in $\mathsf{Lift}(\C)$ defining the two lifts, so the induced map $\prol(f,m)$\pagenote{There was some ambiguity in map names here, so $(f,v)$ was switched to $(f,m)$.} preserves each lift. To see that $\prol(f,m)$ preserves the anchor, check that
    \[
        \anc^{\prol B} \o \prol(f,m) = \pi_1 \o (f \x T.f)  = T.f \o \pi_1 = T.f \o \anc^{\prol A}.  
    \] 
\end{proof}
\begin{corollary}%
    \label{cor:prol-endofunctor}
    In a tangent category $\C$ where pullbacks along differential bundle projections exist, such as a tangent category equipped with a proper retractive display system (Definition \ref{def:display-system}) like the category of smooth manifolds, there is an endofunctor on the category of anchored bundles in $\C$,  \[\prol': \mathsf{Anc}(\C) \to \mathsf{Anc}(\C) = U^{\mathsf{Lift}}.\prol.\]
\end{corollary}
 \begin{observation}%
    \label{obs:computation-of-lim-prol}
     $T$-Limits in $\mathsf{Anc}(\mathsf{Lift}(\C))$ are computed as pointwise limits in $\C$ by Observations \ref{obs:t-limits-of-anc} and \ref{obs:T-limits-pdbs}, and $\prol$ is constructed as a limit, so it follows that $\prol$ preserves $T$-limits.
 \end{observation}



\begin{proposition}
    \label{prop:anc-almost-tang-cat}
        The prolongation endofunctor $\prol': \mathsf{Anc}(\C) \to \mathsf{Anc}(\C)$ has natural transformations
    \begin{itemize}
        \item $p': \prol' \Rightarrow id$
        \item $0': id \Rightarrow \prol'$
        \item $+': \prol' \ts{p'}{p'} \prol' \Rightarrow \prol'$
        \item $\ell': \prol' \Rightarrow \prol'.\prol'$
    \end{itemize}
    satisfying all of the axioms of a tangent category that do not incvolve the canonical flip (Definition \ref{def:tangent-cat}).
\end{proposition}
\begin{proof}[(Sketch)]
    Note that the full argument is given in \Cref{sec:weil-nerve}, but it is not difficult to sketch it out here. For the projection, zero map, and addition, use the differential bundle structure induced by Corollary \ref{cor:prol-functor}. To see the lift axiom, note that up to a choice of pullback, we have:
    \[
        \prol'.\prol'(\pi:A \to M, \xi, \lambda, \anc) = 
        \begin{cases}
            \quad \quad \quad \quad (p \o \pi_1, p \o \pi_2):& \prol^2(A) \to \prol(A) \\
            (\xi \o \pi \o \pi_0, 0 \o \pi_1, 0 \o \pi_2):& \prol(A) \to \prol^2(A) \\
            \quad \quad \quad \quad \quad \quad (\lambda \x \ell \x \ell):&\prol^2(A) \to T.\prol^2(A) \\
            \quad \quad \quad \quad \quad \quad \quad (\pi_1, \pi_2):& \prol^2(A) \to T.\prol(A)
        \end{cases}  
    \]
    The ``lift map'' is then
    \[
        \ell': \prol' \Rightarrow \prol'.\prol'; \hspace{0.15cm}
        \prol(A) \xrightarrow[]{A \x T.\hat\lambda} \prol^2(A) 
    \]
    (where we recall that $\hat\lambda = (\xi\o\pi,\lambda):A \to \prol(A)$).
\end{proof}

\begin{remark}
    The structure described in Proposition \ref{cor:prol-functor} leads to the theory of \emph{double vector bundles}, developed by MacKenzie and his collaborators \cite{flari2019warps,Mackenzie1992}. A double vector bundle is a commuting square
    \[\input{TikzDrawings/Ch3/dvb.tikz}\]
    where each projection is a vector bundle projection, and ``vertical'' and ``horizontal'' orientations of the square are each vector bundle homomorphisms. It was observed by \cite{Grabowski2009} that this is equivalent to a pair of commuting $\R^+$-actions on the total space $E$; following the development in \Cref{ch:differential_bundles}, this is a \emph{commuting} pair of non-singular lifts on $E$,
    \[
        \lambda^A,\lambda^B:E \to TE, \hspace{0.15cm}
        T.\lambda^B \o \lambda^A = c \o T.\lambda^A \o \lambda^B.  
    \]
    A proper exposition of the so-called ``Ehresmann doubles'' (\cite{Mackenzie2011}) for the structures in Lie theory would substantially expand the scope of this thesis, and so it has been relegated to the margins.
\end{remark}


Finally, observe that a connection (see Section \ref{sec:connections-on-a-differential-bundle}) on an anchored bundle's underlying differential bundle behaves similarly to an affine connection.
\begin{lemma}\label{lem:anchored-bundle-conn}
    Let $(\pi:A \to M, \xi, \lambda, \anc)$ be an anchored bundle with $\prol(A)$ existing in a tangent category $\C$.\pagenote{The original draft had a typo - it was meant to be ``anchored bundle equipped with a connection''.}
    If $(\pi, \xi, \lambda)$ has a connection, then define \[\hat{\kappa} := \kappa \pi_1, \quad \hat{\nabla} := \nabla(\pi_0, \anc\pi_1)\] and note that
    \begin{enumerate}[(i)]
        \item $\hat{\kappa}: \prol(A) \to A$ is a retract of $(\xi\pi, \lambda)$, and $\hat{\kappa}: (\prol(A),l) \to (A, \lambda)$ is linear for both $l = (\lambda \x \ell), (0 \x c \o T.\lambda)$;
        \item $\hat{\nabla}:A_2 \to \prol(A)$ is a section of $(\pi_0, p \o \pi_1)$ and is bilinear;
        \item there is an isomorphism $\prol(A) \cong A_3$:\pagenote{added the explicit isomorphism}
        \[
            \nabla(p \o \pi_1, \anc \o \pi_0) +_{(\pi, T.\pi)} \hat{\mu}(p \o \pi_1, \hat{\kappa}) = id. 
        \]
    \end{enumerate}
\end{lemma}
\begin{example}
    Every vector bundle in the category of smooth manifolds has a connection; it follows that  Lemma \ref{lem:anchored-bundle-conn} holds for every anchored bundle in the category of smooth manifolds.
\end{example}

\begin{remark}
    The category of anchored bundles in a tangent category is almost a tangent category, except that it lacks a symmetry map. The ``differential objects'' in such a category will act like a cartesian differential category, except that the symmetry of mixed partial derivatives fails. There has been some interest in settings for differentiable programming where the symmetry of mixed partial derivatives need not hold (see Definition 3.4 along with the discussion at the end of Section 6 in \cite{cruttwell2021categorical});\pagenote{I have included an actual reference to quantify some interest.} the category of anchored bundles in a tangent category appears to be a source of examples.
\end{remark}
\section{Involution algebroids}%
\label{sec:involution-algebroids}

% In  \Cref{sec:Lie_algebroids}, a formalism for describing Lie algebroids and Lie algebroid morphisms that would work in an arbitrary tangent category (Definition \ref{def:tangent-cat}). However, this presentation relies on the existence of a connection on the differential bundle and base space of the Lie algebroid, and there are objects in tangent categories that do not have affine connections (see, for example, the category $\wone$ outlined in \cref{sec:weil-algebras-tangent-structure}), and this would break the intuition that a Lie algebroid is a generalized tangent bundle. 

Involution algebroids are a tangent-categorical axiomatization of Lie algebroids. Recall that a Lie algebroid has almost all of the structure of the operational tangent bundle, in particular a Lie bracket on sections that satisfies the Leibniz law so that it gives a directional derivative. % (see Observation \ref{obs:operational-tangent-bundle} for discussion of the operational tangent bundle). 
In Proposition \ref{prop:anc-almost-tang-cat}, it was demonstrated that the category of anchored bundles have almost all of the same maps as the tangent bundle, the only structure map missing being the canonical flip
\[
    c:T^2M \Rightarrow T^2M
\]
which, we may recall from the chapter introduction (and \cite{Cockett2015}, \cite{Mackenzie2013}), is used in constructing the Lie bracket of vector fields in a tangent category with negatives. Thus, a natural next step is to add an involution map to an anchored bundle and require that it satisfies the same coherences as $c$ from the tangent bundle.


\begin{definition}\label{def:involution-algd}
    An involution algebroid is an anchored bundle $(A, \pi,\xi,\lambda, \anc)$ equipped with a map $\sigma: \prol(A) \to \prol(A)$ satisfying the following axioms:
    \begin{enumerate}[(i)]
        \item Involution: \[\input{TikzDrawings/Ch3/inv-is-inv.tikz}\]
        \item Double linearity:\[% https://q.uiver.app/?q=WzAsNCxbMCwxLCJcXHByb2woQSkiXSxbMSwxLCJcXHByb2woQSkiXSxbMCwwLCJULlxccHJvbChBKSJdLFsxLDAsIlQuXFxwcm9sKEEpIl0sWzAsMSwiXFxzaWdtYSJdLFsxLDMsIjAgXFx4IGMgXFxvIFQuXFxsYW1iZGEiLDJdLFsyLDMsIlQuXFxzaWdtYSJdLFswLDIsIlxcbGFtYmRhIFxceCBcXGVsbCJdXQ==
        \begin{tikzcd}
            {T.\prol(A)} & {T.\prol(A)} \\
            {\prol(A)} & {\prol(A)}
            \arrow["\sigma", from=2-1, to=2-2]
            \arrow["{0 \x c \o T.\lambda}"', from=2-2, to=1-2]
            \arrow["{T.\sigma}", from=1-1, to=1-2]
            \arrow["{\lambda \x \ell}", from=2-1, to=1-1]
        \end{tikzcd}\]
        \item Symmetry of lift: \[\input{TikzDrawings/Ch3/symmety-sig-l.tikz}\]
        \item Target: \[\input{TikzDrawings/Ch3/anc-is-inva-morphism.tikz}\]
        \item Yang--Baxter: \[\input{TikzDrawings/Ch3/yb.tikz}\]
    \end{enumerate}
    $A$ is an \emph{almost}-involution algebroid if the involution does not satisfy the Yang--Baxter equation.
    A morphism of involution algebroids is a morphism of anchored bundles, so that $\prol(f)\o \sigma_A = \sigma_B\o\prol(f)$.
    (Note that because $\sigma$ is an isomorphism and $\sigma = \sigma^{-1}$, 
    \[\sigma: (\prol(A), 0 \x c\o T.\lambda) \to (\prol(A),\lambda\x \ell)\]
    is linear as well.) Write the category of involution algebroids and involution algebroid morphisms in $\C$ is written $\mathsf{Inv}(\C)$.\pagenote{
    The category $\mathsf{Inv}(\C)$ has been explicitly defined - this addresses a few later remarks in this section.
}
\end{definition}

\begin{observation}
    It is not immediately clear that the Yang--Baxter equation is well-typed. This follows from the target axiom (iv) and the double linearity axiom (ii). Starting with $(u,v,w): \prolong \ts{T.\anc}{T^2.\pi} T^2A$, we see that $\sigma \x c$ is well-typed if and only if
    \[
       T.\anc \o \pi_1 \o \sigma(u,v) = T^2.\pi \o c \o w = c \o T^2.\pi \o w = c \o T.\anc \o v.
    \]
    Similarly, $1 \x T.\sigma$ is well-typed if $T.\anc \o u = T.\pi \o \pi_0 \o T.\sigma(v,w)$; then use the double linearity axiom to compute
    \begin{gather*}
        T.\pi \o \pi_0 \o T.\sigma(v,w) = T.\pi \o T.p \o T.\pi_1 (v,w) = T.\pi \o T.p \o w 
        \\ = T.p \o T^2.\pi \o w = T.p \o T.\anc \o v = T.\pi \o v.
    \end{gather*}
\end{observation}
This perspective on Lie algebroids has already appearad in the work of Martinez and his collaborators in \cite{Leon2005}, where a ``canonical involution'' was derived on space of prolongations of a Lie algebroid using the formula
\[
    \sigma: \prol(A) \to \prol(A); \sigma(x,y,z) = \sigma(y,x,z + \< x, y\>).
\]
The structure of this map has been largely unexplored; helpfully, involution algebroids succeed in reverse-engineering axioms for an involution map that will induce a Lie bracket on the sections of the projection map. \pagenote{I removed the remark about the Yang-Baxter equation, as that is discussed more thoroughly in the introduction of this chapter. Identifying axioms on the involution to identify Lie brackets is original, I think I may have muddied the waters with my original remark.} 
The bracket from the original Lie algebroid is induced using the same formula as for the bracket of vector fields on the tangent bundle:
\[
    \lambda \o [X,Y]^* 
    =
    \left( 
        (\pi_1 \o \sigma \o (id, T.X \o \anc) \o Y -_p T.Y \o X \o \anc) -_{T.\pi} 0Y
    \right).
\]
% Choosing a connection on $A$,  this map is equivalent to (see equation (4.1) in \cite{Leon2005})
% \[
%    (u,v,w):A_3 \mapsto (v,u,w + \< u,v \>):A_3
% \]
Furthermore, a morphism of anchored bundles is a Lie algebroid morphism if and only if it preserves the derived involution map.

\begin{example}\label{ex:inv-algds}
    \pagenote{I have updated this example following to clarify the relationships between $\C, \mathsf{Anc}(\C), \mathsf{Inv}(\C)$.}
    ~\begin{enumerate}[(i)]
        \item For any $M$ in $\C$, $(TM, p, 0, id, c)$ is an involution algebroid. Furthermore, for any involution algebroid anchored on $M$, $(\anc, id)$ is a morphism of involution algebroids (by the target axiom). This defines a fully faithful functor $\C \hookrightarrow \mathsf{Inv}(\C)$. The same construction as the anchored bundle case in Example \ref{ex:anchored-bundles}(i) exhibits $\C$ as a reflective subcategory of $\mathsf{Inv}(\C)$.
        \item For any differential bundle, the trivial anchored bundle $(\pi:A \to M,\xi,\lambda, 0\o\pi)$ has an involution using the isomorphism $\prol(A) \cong A_3$, given by $\sigma := (\pi_1,\pi_0,\pi_2)$ (proving this map satisfies the involution algebroid axioms is just an exercise in combinatorics). It follows that every differential bundle morphism gives rise to a morphism of these \emph{trivial} involution algebroids. The same construction from the anchored bundle case (Example \ref{ex:anchored-bundles}(ii)) exhibits $\mathsf{Diff}(\C)$ as a \emph{coreflective} subcategory of involution algebroids.
        \item Consider a groupoid 
        \[s,t:G \to M, \hspace{0.15cm} e: M \to G, \hspace{0.15cm} (-)^{-1}:G \to G, \hspace{0.15cm} m:G_2 \to G.\] 
        The underlying reflexive graph has an associated anchored bundle, constructed in Example \ref{ex:anchored-bundles}, and the space of prolongations of this anchored bundle includes into $(T^2.G_2)$. Note that there is a well-formed involution map:
        \[
            \sigma(u,v) = 
            c \o ((0 \o p \o v)^{-1}; (T.0 \o u);v)
        \] 
        The direct proof of this involves a more conceptual construction, which is the focus of Section \ref{sec:inf-nerve-of-a-gpd}.
    \end{enumerate}
\end{example}

An involution algebroid resembles a generalized tangent bundle, and so the lift $(\xi\o\pi, \lambda):A \to \prol(A)$ satisfies the same universality conditions as $\ell:T \Rightarrow T^2$. The double linearity condition is equivalent to the naturality condition for $c, \ell$ in Definition \ref{def:tangent-cat}:
\begin{equation}
    \label{eq:symmetry-of-lift}
    \input{TikzDrawings/Ch3/Sec5/twist-axiom-prol.tikz}
\end{equation}
which, using string diagrams for monoidal categories (\cite{selinger2010survey}), is the equation
\begin{equation*}
    % \label{eq:symmety-of-lift}
    \input{TikzItDrawings/twist-axiom.tikz}
\end{equation*}
where the circle denotes the lift and $c$ is the crossing of two lines. 
\begin{proposition}%
    \label{prop:nat-of-sigma-ell}
    Let $(\pi:A \to M, \xi, \lambda, \anc)$ be an anchored bundle, and suppose that
    \[
        \sigma: \prol(A) \to \prol(A)  
    \]
    satisfies the involution axiom (i) and the symmetry of lift axiom (iii), and furthermore that the involution ``exchanges'' the idempotents associated to the two lifts $(\lambda \x \ell)$ and $(0 \x c \o T.\lambda)$ on $\prol(A)$ (Proposition \ref{prop:idempotent-natural})
    \[
        \sigma \o ((p \o \lambda) \x (p \o \ell)) = (p \o 0) \x (p \o c \o T.\lambda) = id \x (T.p \o T.\lambda)
    \]
    Then the double linearity axiom is equivalent to the left-hand commuting diagram in Diagram \ref{eq:symmetry-of-lift}:
    \[
      (\hat{\lambda} \x \ell) \o \sigma = (id \x T.\sigma) \o (\sigma \x c) \o (id \x T.\hat{\lambda}).  
    \]
\end{proposition}
\begin{proof}
    Starting with the left-hand side of the equation, 
    \begin{align*}
        &\quad (id \x T.\sigma) \o (\sigma \x c) \o (id \x T.\hat\lambda) \o (u,v) \\
        &= (id \x T.\sigma) \o (\sigma \x c) \o (u, T.\xi \o T.\pi \o v, T.\lambda v) \\
        &= (id \x T.\sigma) \o (\sigma \o (u, T.\xi \o \anc \o u), T.\lambda v) \\
        &= (id \x T.\sigma) \o (\xi \o \pi \o u, 0 \o \anc \o u, T.\lambda v) \\
        &= (id \x T.\sigma) \o (\xi \o \pi \o u, 0 \o T.\pi \o v, T.\lambda v) \\
        &= (\xi \o \pi \o u, T.\sigma \o \hat \lambda \o v) .
    \end{align*}
    For the right-hand side:
    \begin{align*}
        &\quad (\hat \lambda, \ell) \o \sigma \o (u,v) \\
        &= (\xi \o \pi \o \pi_0 \o \sigma\o (u,v), (\lambda \x \ell) \o \sigma\o (u,v)) \\
        &= (\xi \o \pi \o p \o v, (\lambda \x \ell) \o \sigma\o (u,v)) \\
        &= (\xi \o p \o T.\pi \o v, (\lambda \x \ell) \o \sigma\o (u,v)) \\
        &= (\xi \o p \o \anc \o u, (\lambda \x \ell) \o \sigma\o (u,v)) \\
        &= (\xi \o \pi \o u, (\lambda \x \ell) \o \sigma\o (u,v)) \\
        &= (\xi \o \pi \o u, T.\sigma \o (0 \x c \o T.\lambda) \o (u,v)).
    \end{align*}\pagenote{Added a step to clarify the calculation.}
    So the naturality equation \ref*{eq:symmetry-of-lift} is equivalent to
    \[
        T.\sigma \o (0 \x c \o T.\lambda) = (\lambda \x \ell) \o \sigma.
    \]
\end{proof}



The category of involution algebroids is ``tangent monadic'' over the category of anchored bundles, in the same sense that anchored bundles are tangent monadic over the category of differential bundles, or internal categories over reflexive graphs in a category. The ``tangent monadicity'' leads to a similar observation about $T$-limits of involution algebroids as in Observation \ref{obs:t-limits-of-anc}.
\begin{observation}%
    \label{obs:inv-tmonad-anc}
    The forgetful functor from involution algebroids to anchored bundles creates limits; that is to say, the $T$-limits of the underlying anchored bundles give the limits of involution algebroids. Recall that by Observation \ref{obs:computation-of-lim-prol}, the limit $\prol(\lim A_i) = \lim \prol(A_i)$ in the category of anchored bundles, and this induces a map between objects in $\C$
    \[
        \lim \sigma_i: \lim \prol(A_i) \to \lim \prol(A_i).  
    \]
    The axioms for an involution algebroid are induced by universality. 
\end{observation}
Note that a point-wise tangent structure may be defined following the anchored bundles example from Lemma \ref{lem:tangent-anchored-bundle}:\pagenote{Added some preamble to this result to help differentiate it from the prolongation tangent structure.}
\begin{proposition}%
    \label{prop:pointwise-tangent-structure-inv}
    For a tangent category $\C$, the category of involution algebroids has a ``point-wise'' tangent structure that maps objects as follows:
    \[
        (\pi:A \to M, \xi, \lambda, \anc, \sigma) \mapsto
        (T.\pi, T.\xi, c \o T.\lambda, T.\anc, \sigma_T)  
    \]
    where $\sigma_T$ is defined as
    \begin{equation}\label{eq:sigmaTdeff}
        \sigma_T := (1 \x_c c)\o T.\sigma\o (1 \x_c c): \prol(A_T) \to \prol(A_T).
    \end{equation}
\end{proposition}
\begin{proof}    
    The tangent structure on involution algebroids is inherited from the functor $\mathsf{Inv}(\C) \to \mathsf{Anc}(\C)$. Thus, it suffices to give the involution map for the tangent involution algebroid.

    Note that given $(\pi:A \to M, \xi, \lambda, \anc, \sigma)$, the space of prolongations on $(T.\pi, T.\xi, \newline c\o T.\lambda, c\o T.\anc)$ is $TA \ts{c\o T.\anc}{T^2\pi} T^2A$. We can construct an isomorphism between the objects $\prol(TA)$ and $T(\prol(A))$ in $\C$ using the cospan isomorphism
    \input{TikzDrawings/Ch3/Sec5/prol-via-cospan.tikz}
    thus inducing a map
    \[
       TA \ts{c\o T.\anc}{T^2\pi} T^2A
       \xrightarrow[]{id \x c} T(\prolong)
       \xrightarrow[]{T.\sigma} T(\prolong)
       \xrightarrow[]{id \x c} TA \ts{c\o T.\anc}{T^2\pi} T^2A
    \]
    which we call $\sigma_T$. The linearity and involution axioms follow by construction.
    
    Now check the rest of the axioms. For the unit:
    \begin{gather*}
        (1 \x_c c)\o T.\sigma\o (1 \x_c c) \o (T.(\xi\o\pi), c \o T.\lambda)\\
        = (1 \x_c c) \o T.\sigma \o (T(\xi\pi), T\lambda) 
        = (1 \x_c c) \o (T.(\xi\o\pi), T.\lambda) = (T.(\xi\o\pi), c \o T.\lambda).
    \end{gather*}
    For the anchor:
    \begin{gather*}
        T.\anc_T \o \pi_1 \o \sigma_T = T(c\o T\anc) \o \pi_1 \o (1 \x_c c)\o T.\sigma\o (1 \x_c c)\\
        = T.c\o c \o  T^2.\anc \o T.\pi_1 \o T.\sigma\o (1 \x_c c)
        = T.c \o c \o T.c \o T^2.\anc \o T.\pi_1 \o (1 \x_c c)\\
        = c \o T.c \o c \o T^2\anc \o c \o \pi_1  = c \o Tc \o T^2.\anc \o \pi_1
        = c \o T.\anc_T \o \pi_1.
    \end{gather*}
    The Yang--Baxter equation is straightforward to check.
\end{proof}
The tangent bundle is a canonical involution algebroid on every object in a tangent category, and the anchor induces a morphism from an involution algebroid to the tangent involution algebroid on its base space.
The anchor acts as a reflector from involution algebroids in $\C$ to $\C$ itself.
\begin{proposition}%
    \label{prop:c-refl-in-inv}
    Any tangent category $\C$ is a reflective subcategory of the category of involution algebroids in $\C$.
\end{proposition}
\begin{proof}
    First, observe that the inclusion of $\C$ into $\mathsf{Inv}(\C)$ (Definition \ref{def:involution-algd})\pagenote{The notation $\mathsf{Inv}(\C)$ had not yet been introduced at this point, it has been added to the definition of involution algebroids and a reference has been made to that definition.} is fully faithful because the anchor on the tangent involution algebroid is $id: TM \to TM$. This means that the only involution algebroid morphisms $TA \to TB$ are pairs $(Tf,f)$, $f: A \to B$.
    Now consider the functor $\mathsf{Inv}(\C) \to \C$ that sends $(\pi:A \to M, \xi, \lambda, \anc, \sigma)$ to $M$: this gives an endofunctor $S: \mathsf{Inv}(\C) \to \mathsf{Inv}(\C)$ along with a natural transformation $\anc: id \Rightarrow S$, so that $S.\anc = \anc.S = id$, given by the anchor map.
    Thus,  the category $\C$ is the category of algebras for an idempotent monad on $\mathsf{Inv}(\C)$.
\end{proof}
\begin{corollary}
    Let \[ \widehat{A} = (\pi:A \to M, \xi, \lambda, \anc, \sigma) \]   be an involution algebroid in a tangent category $\C$.
    \begin{enumerate}[(i)]
        % \item The category of involution algebroids is a tangent category, so that given an involution algebroid:
        % \[ \widehat{}{A} = (\pi:A \to M, \xi, \lambda, \anc, \sigma) \]   
        \item The morphism
        \[(T.\pi,\pi): TA \to TM\]
        is an involution algebroid morphism from the tangent involution algebroid on $A$ to the tangent involution algebroid on $M$.
        \item The morphism
        \[(\anc,id):\widehat{A} \to TM\]  
        is an involution algebroid morphism.
    \end{enumerate}
    If pullback powers of $p \o \pi_1: \prol(A) \to A$ exist, then
    \[
        (p\o \pi_1: \prol(A) \to A, (\xi \o \pi, 0), (\lambda \x \ell))
    \] is a differential bundle\pagenote{A ``differentiable bundle'' slipped in.}; note that $\pi_0$ acts as an anchor. If the prolongations exist, then 
    \[
        (p \o \pi_1:\prol(A) \to A, (\xi \o \pi, 0), (\lambda \x \ell), \pi_0, 
        \sigma')
    \] is an involution algebroid; this follows from computing the pullback in the category of involution algebroids ($\sigma'$ is induced as in Observation \ref{obs:t-limits-of-anc}).
\end{corollary}
The above corollary puts an involution algebroid structure on the differential bundle $(\prol(A), \lambda \x \ell)$. Note that the map $\sigma$ gives an isomorphism of differential bundles
\[
  (\prol(A),\lambda \x \ell) \to (\prol(A),0 \x c \o T.\lambda).
\]
Martinez observed that the canonical involution $\sigma$ puts a unique Lie algebroid structure on $(\prol(A),0 \x c \o T.\lambda)$, and $\sigma$ is a uniquely determined isomorphism of involution algebroids (see \cite{Leon2005}):
\begin{corollary}\label{cor:second-lie-algd-unique-map}
    For an involution algebroid $(\pi:A \to M, \xi, \lambda, \anc, \sigma)$, the isomorphism of differential bundles
    \[
        \sigma: (\prol(A), 0 \x c \o T.\lambda) \to (\prol(A), \lambda \x \ell)
    \]
    induces a second involution algebroid structure on $\prol(A)$.
\end{corollary}

Recall that Proposition \ref{prop:anc-almost-tang-cat} sketched out a proof that the category of anchored bundles in $\C$ has an endofunctor $\prol'$ and natural transformations $p', 0', +', \ell'$ satisfying the axioms of a tangent structure; this endofunctor and the natural transformations all lift to $\mathsf{Inv}(\C)$. The involution map $\sigma$ is the missing piece that gives a tangent structure on $\mathsf{Inv}(\C)$. The construction may be spelled out here at a big-picture level, but the actual proof brings up tricky coherence issues that make up the bulk of Chapter \ref{chap:weil-nerve}.
\begin{proposition}%
    \label{prop:second-tangent-structure-inv-algds}
    The category of involution algebroids in a tangent category $\C$ has a second tangent structure, where the tangent functor is given by\pagenote{Changed wording to help differentiate this structure against the pointwise tangent structure defined beforehand.}
    \[
        \prol': \mathsf{Inv}(\C) \to \mathsf{Inv}(\C)  
    \]
    and the tangent natural transformations are given as in Proposition \ref{prop:anc-almost-tang-cat}, with the canonical flip
    \[
        \sigma': \prol'.\prol' \to \prol'.\prol' := \prol^2(A) \xrightarrow[]{1 \x T.\sigma} \prol^2(A).  
    \]
    Starting with an involution algebroid, $\prol'(A)$ and $\prol'.\prol'(A)$ are given by
    \[
       \prol'(A) 
       \begin{cases}
            \pi':& \prol(A) \xrightarrow[]{p \o \pi_1} A \\
            \xi':& A \xrightarrow[]{(\xi \o \pi, 0)} \prol(A) \\
            \lambda':& \prol(A) \xrightarrow[]{\lambda \x \ell} T.\prol(A) \\
            \anc':& \prol(A) \xrightarrow[]{\pi_1} TA \\
            \sigma':& \prol^2(A) \xrightarrow{\sigma \x c} \prol^2(A)
        \end{cases}  
        \hspace{0.15cm}
        \prol'.\prol'(A)
        \begin{cases}
            \pi''& \prol^2(A) \xrightarrow[]{(p \o \pi_1, p \o \pi_2):} \prol(A) \\
            \xi'':& \prol(A) \xrightarrow[]{(\xi \o \pi \o \pi_0, 0 \o \pi_1, 0 \o \pi_2)} \prol^2(A) \\
            \lambda'':&\prol^2(A) \xrightarrow[]{(\lambda \x \ell \x \ell)} T.\prol^2(A) \\
            \anc'':& \prol^2(A) \xrightarrow[]{(\pi_1, \pi_2)} T.\prol(A) \\
            \sigma'':& \prol^3(A) \xrightarrow[]{(\sigma \x c \x c)}\prol^3(A) 
        \end{cases}  
    \]
    (where $\prol^3(A) = \prolong \ts{T.\anc}{T^2.\pi} T.(\prolong)$).
    The tangent natural transformations are given by
    \begin{itemize}
        \item $p: \prol' \Rightarrow id; \hspace{0,15cm} \prol(A) \xrightarrow[]{\pi_0} A$ 
        \item $0: id \Rightarrow \prol'; \hspace{0,15cm} A \xrightarrow[]{(id,T.\xi \o T.\pi)} \prol'(A)$
        \item $+: \prol'_2 \Rightarrow \prol'; \hspace{0,15cm} A \ts{\anc}{T.\pi \o \pi_i} T_2A \xrightarrow[]{id \x +} \prol(A)$
        \item $\ell:\prol'(A) \Rightarrow \prol'.\prol'; \hspace{0,15cm} \prol(A) \xrightarrow[]{1 \x T.(\xi\o\pi, \lambda)} \prol^2(A)$
        \item $\anc': \prol' \Rightarrow T$
        \item $c:\prol'.\prol'(A) \Rightarrow \prol'.\prol';\hspace{0,15cm}  \prol^2(A) \xrightarrow[]{1 \x T.\sigma} \prol^2(A)$.
    \end{itemize}
\end{proposition}
\begin{proof}
    Deferred to \Cref{sec:prol_tang_struct}.
\end{proof}
% \begin{remark}
%     This construction of pulling back a map $(T.f,f):TN \to TM$ along the anchor map $\anc: A \to TM$ in the category of involution algebroids is called the \emph{prolongation of $f$ along $A$} in the Lie algebroid literature (see, for example, \cite{Leon2005,Martinez2018}).
% \end{remark}

% \begin{lemma}
%     Let $(\pi:A \to M,\xi,\lambda,\anc,\xi)$ be an involution algebroid in a tangent category $\C$, and $f:N \to M$ a map.
%     If the prolongation $\prol_f(A)$ exists, it is an involution algebroid over $N$.
% \end{lemma}
% \begin{proof}
%     By Observation \ref{obs:inv-tmonad-anc}, \cref{eq:prol-by-f} induces a pullback in the category of involution algebroids where the base square is the pullback of $f$ along $id$.
% \end{proof}


% \begin{definition}
%     Let $(\pi:A \to M,\xi,\lambda,\anc,\xi)$ be an involution algebroid in a tangent category $\C$, and $f:N \to M$ a map.
%     The prolongation of $f$ along $A$ is the $T$-pullback:
%     % https://q.uiver.app/?q=WzAsNCxbMSwwLCJUTiJdLFsxLDEsIlRNIl0sWzAsMSwiQSJdLFswLDAsIlxccHJvbF9mQSJdLFsyLDEsIlxcYW5jIiwyXSxbMCwxLCJUZiJdLFszLDIsIlxccGlfMCIsMl0sWzMsMCwiXFxwaV8xIl1d
%     \begin{equation}\label{eq:prol-by-f}
%         \input{TikzDrawings/Ch3/Sec5/prol-by-f.tikz}
%     \end{equation}
% \end{definition}
% Using this lemma and the fact that $\anc$ is an involution algebroid morphism, there is an involution algebroid structure on $\prol(A)$.
% \begin{proposition}
%     Let $(\pi:A \to M,\xi,\lambda,\anc,\xi)$ be an involution algebroid in a tangent category $\C$ so that $\pi$ is $T$-display.
%     Then the anchored bundle $(p\o\pi_1:\prol(A) \to A, (\xi\pi,0),(\lambda, \ell),\pi_0)$ has an involution algebroid structure.
% \end{proposition}

\section{Connections on an involution algebroid}%
\label{sec:connections_on_an_involution_algebroid}
In this section we take an involution algebroid with a chosen linear connection on its underlying anchored bundle (Definition \ref{def:involution-algd}, \ref{def:lin-connection}), and rederive Martinez's structure equations for a Lie algebroid (Proposition \ref{prop:la-iff-structure-morphisms}). 
% This section relates the coherences of an involution algebroid (Definition \ref{def:involution-algd}) with Martinez's \emph{structure equations} for a Lie algebroid  Recall that in Section \ref{sec:Lie_algebroids} the structure morphisms and equations were translated into tangent-categorical syntax using an explicit choice of connection on the underlying anchored bundle. By making an explicit choice of a connection on an \emph{involution} algebroid, we may rederive the structure equations in a more general setting.
% This section relati
% Martinez's structure equations for a Lie algebroid (--) rely upon the fact that every vector bundle in the category of smooth manifolds admits a connection. 
% In the category of smooth manifolds, every vector bundle has a connection, so the space of prolongations for an anchored bundle admits a decomposition $\prol(A) \cong A_3$.  In practice, this means the simplest way to define a geometric structure is  multilinear operations on an anchored bundle and a covariant derivative are . 
% Thus, in comparing involution algebroids to Lie algebroids, it will be helpful to characterize involution algebroids in terms of multilinear operations on $A$ using a connection. We work in in an arbitrary tangent category with negatives, where appropriate connections exist when stated.

In a tangent category $\C$  with negatives, there is a natural transformation\pagenote{Clarifying notation for this section as there is substantially more fibered linear algebra, including the use of substraction, than is present in other sections/chapters.}
\[
    n:T \Rightarrow T  
\]
making each fibred commutative monoid $(p:TM \Rightarrow M, 0, +, n)$ a fibred abelian group. Because the additive bundle structure on differential bundles is induced via universality (Proposition \ref{prop:induce-abun}), in a tangent category with negatives the additive bundle structure for differential bundles will likewise have negatives.
We adopt the following notation for the ``fibred linear algebra'' used in this section, as there are a significant number of computations done on bundles with multiple choices of additive bundle structure (e.g. the second tangent bundle of a differential bundle has three additive bundles structures).
\begin{notation}
    Let $E$ be an object with multiple differential bundle structures $(\pi^i:E \to M^i, \xi^i, \lambda^i)$ in a tangent category with negatives. Addition over a specific differential bundle is written using infix notation, where a subscript is added to the symbol denoting the projection of the differential bundle. Letting $x,y:X \to E$ denote a pair of generalized elements for which the $\pi^i$-addition operation is well-defined; we set
    \[
        x +_{\pi[i]} y = X \xrightarrow[]{(x,y)} E \ts{\pi[i]}{\pi[i]} E \xrightarrow[]{+^i} E.
    \]
    Similar notation is used for subtraction:
    \[
        x -_{\pi[i]} y  = X 
        \xrightarrow[]{(x,y)} E \ts{\pi[i]}{\pi[i]} E 
        \xrightarrow[]{id \x n[i]} E \ts{\pi[i]}{\pi[i]} E 
        \xrightarrow[]{+^i} E.
    \]
\end{notation}

Throughout this section, we work in a tangent category $\C$ with negatives\pagenote{I had  originally forgotten to add the ``with negatives'' caveat} and a chosen anchored bundle $(\pi:A \to M, \xi, \lambda, \anc)$, equipped with a connection $(\kappa,\nabla)$, whose base object has a torsion-free connection $(\kappa',\nabla')$ and a morphism
\[
    \sigma: \prol(A) \to \prol(A)
\]
that exchanges the two projection maps $p \o \pi_1, \pi_0$, meaning that the following diagram commutes:
\begin{equation*}
    \input{TikzDrawings/Ch3/Sec6/connection-notation.tikz}
\end{equation*}
The following notation will be useful when working with local coordinates.
\begin{notation}
    First, recall the $\nabla$-notation for morphisms of differential bundle where each morphism has a choice of connection from Equation \ref{eq:nabla-notation}:
    \[
        \infer{\nabla[f] := \kappa^B \o T.f \o \nabla^A:A_2 \to B}{f:A \to B & (\kappa^A,\nabla^A,A, \lambda^A) & (\kappa^B,\nabla^B, B, \lambda^B)}
        % 
    \]
    Let \[(\pi:A \to M, \xi, \lambda, \anc), (q:B \to N, \zeta, l, \rho)\] be a pair of anchored bundles equipped with connections.
    ``Hatting'' a map $f:\prol(A) \to \prol(B)$ refers to the map:
    \[
        \widehat{f}:A_3 \xrightarrow[]{\hat\nu(\pi_0,\pi_2) + \hat\nabla(\pi_0,\pi_1)} \prol(A) \xrightarrow[]{f} \prol(B) \xrightarrow[]{(\pi_0,p \o \pi_1, \kappa \o \pi_1)} B_3.  
    \]
    Similarly, for $f:TA \to TB$,
    \[
        \widehat{f}: TM \ts{p}{\pi} A \ts{\pi}{\pi} A \xrightarrow[]{\nu(\pi_0,\pi_2) + \nabla(\pi_0,\pi_1)} TA \xrightarrow[]{f} TB \xrightarrow[]{(\pi_0,p \o \pi_1, \kappa \o \pi_1)} TM \ts{p}{q} B \ts{\pi}{q} B  
    \]
    Clearly, $\widehat{f \o g} = \widehat{f} \o \widehat{g}$. Similarly, a map $A_3 \to B_3$ may be ``barred'' to form a map $\prol(A) \to \prol(B)$:
    \[
        \overline{g}:\prol(A) \xrightarrow[]{(\pi_0,p \o \pi_1, \kappa \o \pi_1)} A_3  \xrightarrow[]{g} B_3 \xrightarrow[]{\hat\nu(\pi_0,\pi_2) + \hat\nabla(\pi_0,\pi_1)} \prol(B).
    \]
    It is straightforward to see that $\overline{\widehat{f}} = f, \widehat{\overline{g}} = g$.
\end{notation}



\begin{lemma}\label{lem:intertwining-induces-bracket}
    $\sigma: \prol(A) \to \prol(A)$ induces a bracket on $\Gamma(\pi)$:
    \[
        \lambda \o [X,Y]
        = (\sigma \o (id, TX \o \anc) \o Y -_{T\pi} (id, TY \o \anc) \o X) - 0\o Y.
    \]    
\end{lemma}
\begin{proof}
    Let $X,Y \in \Gamma(\pi)$ and compute:
    \begin{align*}
         &\quad p \o \pi_1 \o (\sigma \o (id, TX \o \anc) \o Y -_{\pi_0} (id, TY \o \anc) \o X) \\
        &= p \o (\pi_1 \o \sigma \o (id, TX \o \anc) \o Y -_{T\pi} TY \o \anc \o X) \\
        &= p \o \pi_1 \o \sigma \o (id, TX \o \anc) \o Y -_{\pi} p\o TY \o \anc \o X \\
        &= \pi_0 \o (id, TX \o \anc) \o Y - Y \o p \o \anc \o X \\
        &= Y - Y = \xi, \\
         & \pi_0 \o ( \sigma \o (id, TX \o \anc) \o Y -_{\pi_0} (id, TY \o \anc) \o X) \\
        &= \pi_0 \o \sigma \o (id, TX \o \anc) \o Y -_{\pi} \pi_0 \o (id, TY \o \anc) \o X \\
        &= p \o \pi_1 \o (id, TX \o \anc) \o Y -_{\pi} X \\
        &= p \o TX \o \anc \o Y -_{\pi} X = X -_{\pi} X = \xi. 
    \end{align*}
    The universality of the lift induces a new section $[X,Y]$ so that
    \[
        \lambda \o [X,Y]
        = (\sigma \o (id, TX \o \anc) \o Y -_{T\pi} (id, TY \o \anc) \o X) - 0Y.
    \]    
\end{proof}

\begin{definition}\label{def:linear-intertwining}
    The map $\sigma:\prol(A) \to \prol(A)$ is \emph{linear} whenever the two bundle morphisms
    \[
        \sigma: (\prol(A),\lambda \x \ell) \to (\prol(A),0 \x c \o T\lambda)
        \hspace{0.25cm}
        \sigma: (\prol(A),0 \x c \o T\lambda) \to (\prol(A),\lambda \x \ell)
    \]
    are linear, and \emph{cosymmetric} if \[\sigma \o \widehat{\lambda} = \widehat{\lambda} = (\xi\pi,\lambda).\]
    Note that whenever $\sigma$ is linear and cosymmetric, $\sigma \o \hat\mu (u,v) = \hat\nu (u,v)$.
\end{definition}
Linearity and unit axioms, along with the connection on the differential bundle, force the existence of a bilinear bracket $\<-,-\>$ as in the definition of a Lie algebroid in Proposition \ref{prop:lie-alg-bil-defn}.
\begin{proposition}\label{prop:intertwining-linear-iff-bracket-linear}
    For an anchored bundle $(\pi:A \to M, \xi, \lambda, \anc)$ with connection $(\kappa, \nabla)$, a cosymmetric and bilinear $\sigma$ is equivalent to a bilinear $\<-,-\>:A_2 \to A$, with the correspondence given by
    \[
        \infer{\sigma: \prol(A)\xrightarrow[\overline{(\pi_1,\pi_0, \pi_2 + \< \pi_0,\pi_1 \>)}]{} \prol(A)}
        {\< -,-\>: \kappa \o \sigma \o \nabla -_\pi \kappa \o \nabla}
    \]
\end{proposition}
\begin{proof}
    Derive
    \begin{align*}
    \widehat{\sigma} \o (x,y,z)
        &= (y,x, \kappa \o \sigma \o (\nabla(\anc x,y) +_{p\o\pi_1} \mu(y, z)))\\
        &= (y,x, \kappa \o \sigma \o \nabla(\anc x, y) +_\pi \kappa \o \sigma \o \mu (y,z))\\
        &= (y,x, \kappa \o \sigma \o \nabla(\anc x, y) +_\pi \kappa \o \nu \o (y,z))\\
        &= (y,x, \kappa \o \sigma \o \nabla(\anc x, y) +_\pi z)\\
        &=: (y,x, \langle x, y\rangle +_\pi z )
    \end{align*}
    where $\<-,-\>$ is certainly bilinear. The converse is immediate: take
    \[
        \widehat{\sigma}(u,v,w) := (v,u,w + \< u,v\>).
    \]
    It is easy to see that $\sigma$ is linear and cosymmetric.    
\end{proof}
The linear bracket must be involutive for the bilinear bracket to be alternating.
\begin{proposition}\label{prop:intertwining-involutive}
    If $\sigma$ is cosymmetric and linear, then the bilinear bracket $\< -,-\>$ is alternating if and only if $\sigma\o \sigma = id$.
\end{proposition}
\begin{proof}
    First, note that $\sigma \o \sigma = id$ if and only if $\widehat{\sigma} \o \widehat{\sigma} = id$.
    Then check that
    \[
        \widehat{\sigma} \o \widehat{\sigma} (u,v,w) = \widehat{\sigma}(v,u,w+\<u,v\>) = (u,v,w + \< u,v \> + \< v, u\>).
    \]
    By the cancellativity of addition on $A$, 
    \[
    w = w+ \< u,v \> + \< v,u \> \iff 0 = \< u,v \> + \< v,u \>.
    \]
\end{proof}
\begin{observation}
    A bilinear $\<-,-\>$ on an anchored bundle with a connection induces the maps
    \[
        \{ v, x\}_{(\kappa, \nabla)} := \kappa' \o T.\anc \o \nabla(v, x), \quad
        \{v,x,y\}_{(\kappa, \nabla)} := \kappa \o T(\langle -,-\rangle_{(\kappa, \nabla)})\o \nabla^{A_2} \o (v,x,y)
    \]
    from \Cref{eq:lie-algd-structure-maps} in Proposition \ref{prop:lie-alg-bil-defn}.
\end{observation}
\begin{proposition}\label{prop:cosymmetric-leibniz}
    Let $\sigma$ be cosymmetric and bilinear, with the associated bracket $\<-,-\>$. Then
    \[
        T\anc \o \pi_1 \o \sigma = c \o T\anc \o \pi_1
    \] 
    if and only if the Leibniz equation is satisfied:
    \begin{equation}
        \label{eq:liebniz-local}
        \anc \o \langle u,v \rangle + \{ \anc v, u\} = \{ \anc u, v.\}
    \end{equation}
\end{proposition}
\begin{proof}
    Following the given notation and using the hypothesis that the connection is torsion-free on $M$, 
    \[
        \widehat{T\anc \o \pi_1}(u,v,w) = (\anc u, \anc v, w + \{u,v\})
        \hspace{0.5cm}
        \widehat{c}(x,y,z) = (y,x,z). 
    \]
    Computing each side,
    \begin{align*}
        \widehat{T\anc} \o \widehat{\pi_1} \o \widehat{\sigma} \o (u,v,w)
        &= \widehat{T\anc} \o \widehat{\pi_1} \o (v,u,w + \langle u,v\rangle) \\
        &= \widehat{T\anc} \o (\anc v,u,w + \langle u,v\rangle) \\
        &= (\anc v,\anc u, \anc w + \anc \langle u,v\rangle + D[\anc](\anc v, u)) \\
        &= (\anc v,\anc u, \anc w + \anc \langle u,v\rangle + \{ v, u\} ),\\
        \widehat{c} \o \widehat{T\anc} \o \widehat{\pi_1}(u,v,w)
        &= \widehat{c} (\anc u, \anc v, \anc w + \{u, v\}) \\
        &= (\anc v, \anc u, \anc w + \{ u, v\}),
    \end{align*}
    so it follows that the two terms are equal if and only if the desired equality holds.
\end{proof}

\begin{lemma}\label{lem:tang-of-sigma}
    Let $\sigma$ be linear and cosymmetric.
    % \[
    %     \{-,-,-\}: TM \ts{p}{\pi \o\pi_i} A_2 \to A; \{a, u_1, u_2\} := \kappa \o T(\<-,-\>) \o (\nabla(a,u_1), \nabla(a,u_2))
    % \]
    Then
    \begin{enumerate}[(i)]
        \item $T(\<-,-\>):TA_2 \to TA$ satisfies 
        \begin{align*}
            &\quad \widehat{T(\<-,-\>)}(a_x, u_y, u_z,u_{xy}, u_{xz})\\
            &= (a_x, \< u_y, u_z \>, \{a_x, u_y, y_z\} + \<u_y, u_{xz}\> + \langle u_{xy}, u_{z}\rangle); 
        \end{align*}
        \item $T.\sigma$ satisfies
        \begin{align*}
                &\quad \widehat{(id \x T(\widehat{\sigma}))}(u_{x}, u_{y}, u_{xy}, u_{z}, u_{xz}, u_{yz}, u_{xyz})\\
                &= (u_x, u_z, u_{xz}, u_y, u_{xy},  u_{yz} + \< u_y, u_z\>, \\&\quad u_{xyz} + \{a_x, u_y, y_z\} + \<u_y, u_{xz}\> 
                + \langle u_{xy}, u_{z}\rangle).
        \end{align*}
    \end{enumerate}
\end{lemma}
\begin{proof}~
    \begin{enumerate}[(i)]
        \item By the universality of the vertical lift and bilinearity of $\<-,-\>$, the outer squares below are pullbacks:
            \input{TikzDrawings/Ch3/Sec6/t-of-pair.tikz}
            so that 
               \[T(\langle,\rangle)(0,u_y, u_z, u_{xy},u_{xz}) 
               = \mu^A(\< u_y,u_z \>, \<u_y, u_{xz}\> + \langle u_{xy}, u_{z}\rangle).\]
               Now we compute
               \begin{align*}
                   &\quad T(\langle,\rangle)(\mu^{A_2}((u_y,y_z), (u_{xy}, u_{xz})) +_p \nabla^{A_2}(a_x, (u_y,u_z))) \\
                   &= T(\langle,\rangle)\mu^{A_2}((u_y, u_z),(u_{xy}, u_{xz})) +_p T(\langle,\rangle)\o\nabla(a_x, (u_y,u_z))\\
                   &= \mu^A(\< u_y,u_z \>, \<u_y, u_{xz}\> + \langle u_{xy}, u_{z}\rangle) 
                   +_p T(\langle,\rangle)\o\nabla(a_x, (u_y,u_z)) 
%                   T(\langle,\rangle)( \mu^{A_2}((u_x, u_y), (u_{xz}, u_{yz})) +_p \nabla^{A_2}(u_x, u_y, a_z))
%                   &= T(\langle,\rangle)\o \mu^{A_2}((u_x, u_y), (u_{xz}, u_{yz})) +_p T(\langle,\rangle)\o\nabla^{A_2}(u_x, u_y, a_z)\\
%                   &= \mu^{A}(\langle u_x, u_y\rangle, \langle u_{xz}, u_{yz}\rangle) +_p T(\langle,\rangle)\o\nabla^{A_2}(u_x, u_y, a_z)\\
               \end{align*}                   
               and then postcompose this with $(T\pi, p, \kappa)$ to obtain
               \begin{align*}
                   &\quad (0, \< u_y,u_z \>, \ \<u_y, u_{xz}\> + \langle u_{xy}, u_{z}\rangle) 
                   +_p (a_x, \< u_y, u_z \>, \{a_x, u_y, y_z\})\\
                   &=  
                   (a_x, \< u_y, u_z \>, \{a_x, u_y, y_z\} +  \<u_y, u_{xz}\> + \< u_{xy}, u_{z}\>). 
               \end{align*}
        \item Consider the following diagram:
               \input{TikzDrawings/Ch3/Sec6/rewriting-t-sigma.tikz}
            We want to find $\widehat{T\widehat{\sigma}} = \widehat{T(\pi_1, \pi_0, \pi_2 + \< \pi_0, \pi_1 \> )}$.
            Note that
            \[
                T(\pi_1, \pi_0, \pi_2 +_\pi \< \pi_0, \pi_1 \>)
                = T(\pi_1, \pi_0, \pi_2) +_{T\pi} T(\xi\pi\pi_i, \pi_0, \< \pi_0, \pi_1\>).
            \]
            The left term is straightforward:
            \begin{align*}
                &\widehat{T(\pi_0, \pi_1, \pi_2)}(a_x, (u_y, u_{xy}), (u_z, u_{xz}), (u_{yz}, u_{xyz}))\\
                &=(a_x,  (u_z, u_{xz}), (u_y, u_{xy}),(u_{yz}, u_{xyz}))
            \end{align*}
            and for the right term, use part (i) of this lemma:
            \begin{align*}
                &\quad \widehat{T\<-,-\>}\o \widehat{T(\pi_0, \pi_1))}(a_x, (u_y, u_{xy}), (u_z, u_{xz}), (u_{yz}, u_{xyz})) \\
                &= 
                \widehat{T\<-,-\>}(a_x, u_y, u_{xy}, u_z, u_{xyz}) \\
                &= (a_x, \< u_y, u_z \>, 
                \{a_x, u_y, y_z\} + \<u_y, u_{xz}\> + \< u_{xy}, u_{z}\>) 
            \end{align*}

            Then compute
            \begin{align*}
                &\quad \widehat{T\widehat{\sigma}}(a_x, u_y, u_{xy}, u_z, u_{xz}, u_{yz}, u_{xyz}) \\
                &= (a_x, u_z, u_{xz}, u_y, u_{xy}, u_{yz} + \< u_y, u_z\>, u_{xyz} + q)
            \end{align*}
            where $q = \{a_x, u_y, y_z\} + \<u_y, u_{xz}\> + \langle u_{xy}, u_{z}\rangle$, 
            giving the desired equation.
    \end{enumerate}
\end{proof}

\begin{proposition}\label{prop:yang-baxter-iff-bianchi}
    % Let $(\pi:A \to M, \xi, \lambda, \anc, \sigma)$ be an almost involution algebroid in a tangent category $\C$ with negatives, with anchored connection $(\nabla,\kappa)$ on $A$ and torsion-free affine connection $(\nabla',\kappa')$ on $M$.
    Let $\sigma$ be cosymmetric, doubly linear, and involutive, and satisfy the target axiom.
    Then $\sigma$ satisfies the Yang--Baxter equation if and only if $\<-,-\>$ and $\{-,-,-\}$ satisfy the Bianchi identity: 
    \begin{equation}
        \label{eq:bianchi-identity}
        0
        =\sum_{\gamma \in \mathsf{Cy}(3)} \langle x_{\gamma_0}, \langle x_{\gamma_1}, x_{\gamma_2}\rangle\rangle 
        + \sum_{\gamma \in \mathsf{Cy}(3)} \{ \anc x_{\gamma_0}, x_{\gamma_1}, x_{\gamma_2}.\}
    \end{equation}
\end{proposition}
\begin{proof}
    We expand $\widehat{id \x T\sigma}$ and  $\widehat{\sigma \x c}$.
    Start with $T(id \x \sigma)$, which was derived in Lemma \ref{lem:tang-of-sigma}:
    \begin{align*}
        \sigma_1(u) &= \widehat{(id \x T(\widehat{\sigma}))}(u_{x}, u_{y}, u_{xy}, u_{z}, u_{xz}, u_{yz}, u_{xyz})\\
        &= (u_x, u_z, u_{xz}, u_y, u_{xy}
        ,  u_{yz} + \< u_y, u_z\>, \\
        &\quad u_{xyz} + \langle u_y, u_{xz} \rangle  + \langle u_{xy}, u_z \rangle  + \{  \anc \o u_x, u_y, u_z \}).
    \end{align*}
    Using the fact that $\kappa'$ is torsion free, so 
    $\hat{c}(u_{xz}, u_{yz}, u_{xzy}) = (u_{yz}, u_{xz}, u_{xyz})$, it follows that
    \[
       \sigma_2(u_x, u_y, u_{xy}, u_z, u_{xz}, u_{yz}, u_{xyz})
        = (u_y, u_x, u_{xy}+ \< u_x, u_y \>, u_z, u_{yz}, u_{xz}, u_{xyz}).
    \]
    Then compute
    \begin{align*}
         &\quad \sigma_2\sigma_1\sigma_2(u) \\
        &= \big( u_z
                , u_y
                , u_{yz} + \langle u_y, u_z \rangle 
                , u_x
                , u_{xz} + \langle u_x, u_z \rangle 
                , u_{xy} + \langle u_x, u_y \rangle , z_1 \big),
%               (u_{xyz} + \langle u_y, u_{xy} \rangle  + \langle u_z, u_{xz} \rangle  + \{  \anc \o u_x, u_y, u_z \} + \langle u_x, u_{xz} + \langle u_x, u_z \rangle  \rangle  
%               \\& + \langle u_y, u_{yz} + \langle u_y, u_z \rangle  \rangle  + \{  \anc \o u_z, u_x, u_y \}) \
                \\
        &\quad  \sigma_1\sigma_2\sigma_1(u) \\
        &= \big( u_z
              , u_y
              , u_{yz} + \langle u_y, u_z \rangle 
              , u_x
              , u_{xz} + \langle u_x, u_z \rangle 
              , u_{xy} + \langle u_x, u_y \rangle , z_2\big).
%             u_{xyz} + \langle u_y, u_{xz} \rangle  + \langle u_{xy}, u_z \rangle  + \{  \anc \o u_x, u_y, u_z \} + \langle u_x, u_{yz}+ \langle u_y, u_z \rangle  \rangle \\
%             &  + \langle u_{xz} + \langle u_x, u_z \rangle , u_y \rangle  + \{  \anc \o u_z, u_x, u_y \}      
    \end{align*}
   Note that the first five terms are equal, so it suffices to check $z_1 = z_2$ for $z_1 = \pi_6\sigma_1\sigma_2\sigma_1(u), z_2 = \pi_6\sigma_2\sigma_1\sigma_2(u)$.
    \begin{align*}
    z_1 &= u_{xyz} + \langle u_y, u_{xz} \rangle  + \langle u_{xy}, u_z \rangle  
    +    \{  \anc \o u_x, u_y, u_z \} + \{  \anc \o u_z, u_x, u_y \}  \\
    &\quad +\langle u_x, u_{yz} + \langle u_y, u_z \rangle  \rangle  
    + \langle u_{xz} + \langle u_x, u_z \rangle , u_y \rangle \\ 
    &= u_{xyz} + \langle u_y, u_{xz} \rangle  + \langle u_{xy}, u_z \rangle  + \{  \anc \o u_x, u_y, u_z \} + \{  \anc \o u_z, u_x, u_y \}  \\
    &\quad +\langle u_x, u_{yz} \rangle + \langle u_x, \langle u_y, u_z \rangle  \rangle
    + \langle u_{xz}, u_y \rangle + \langle \langle u_x, u_z \rangle , u_y \rangle \\ 
    &= u_{xyz} + \langle u_{xy}, u_z \rangle  + \{  \anc \o u_x, u_y, u_z \} + \{  \anc \o u_z, u_x, u_y \}  \\
    &\quad +\langle u_x, u_{yz} \rangle + \langle u_x, \langle u_y, u_z \rangle  \rangle
    + \langle \langle u_x, u_z \rangle , u_y \rangle, \\ 
    z_2 &= u_{xyz} 
    + \langle u_x, u_{yz} \rangle + \{  \anc \o u_y, u_x, u_z \} + \langle u_{xy} + \langle u_x, u_y \rangle , u_z \rangle \\
    &= u_{xyz} + \langle u_x, u_{yz} \rangle + \{  \anc \o u_y, u_x, u_z \}    + \langle u_{xy} , u_z \rangle + \langle \langle u_x, u_y \rangle , u_z \rangle .
    \end{align*}    
    So $z_1 = z_2$ is equivalent to requiring
    \begin{align*}
     &\quad  u_{xyz} + \langle u_{xy}, u_z \rangle + \{  \anc \o u_x, u_y, u_z \} + \{  \anc \o u_z, u_x, u_y \}  \\
     &\quad +\langle u_x, u_{yz} \rangle + \langle u_x, \langle u_y, u_z \rangle  \rangle
    + \langle \langle u_x, u_z \rangle , u_y \rangle \\ 
    &= u_{xyz} + \langle u_x, u_{yz} \rangle + \{  \anc \o u_y, u_x, u_z \}     + \langle u_{xy} , u_z \rangle + \langle \langle u_x, u_y \rangle , u_z \rangle;
    \end{align*}
    cancelling alike terms, this is equivalent to
%   \[
%   \langle u_x, \langle u_y, u_z \rangle  \rangle + \langle \langle u_x, u_z \rangle , u_y \rangle + \{  \anc \o u_x, u_y, u_z \} 
%   + \{  \anc \o u_z, u_x, u_y \}
%   \]
    \begin{align*}
     &\quad \langle u_x, \langle u_y, u_z \rangle  \rangle + \langle \langle u_x, u_z \rangle , u_y \rangle + \{  \anc \o u_x, u_y, u_z \} 
    + \{  \anc \o u_z, u_x, u_y \} \\
    &=  \{  \anc \o u_y, u_x, u_z \}   +  \langle \langle u_x, u_y \rangle , u_z \rangle.
    \end{align*}
    Using the fact that $\< -, -\>$ and $\{-,-,-\}$ are alternating in the last two arguments, this is equivalent to
    \begin{align*}
        0&=\langle u_x, \langle u_y, u_z \rangle  \rangle + \langle \langle u_x, u_z \rangle , u_y \rangle +  \langle u_z , \langle u_x, u_y \rangle \rangle \\
         &\quad + \{  \anc \o u_x, u_y, u_z \} + \{  \anc \o u_z, u_x, u_y \} + \{  \anc \o u_y, u_z, u_x \} \\
         &=\langle u_x, \langle u_y, u_z \rangle  \rangle + \langle u_y, \langle u_z, u_x \rangle\rangle +  \langle u_z , \langle u_x, u_y \rangle \rangle \\
         &\quad + \{  \anc \o u_x, u_y, u_z \} + \{  \anc \o u_z, u_x, u_y \} + \{  \anc \o u_y, u_z, u_x \} 
    \end{align*}
%    \[
%       0 =
%       \langle u_x, \langle u_y, u_z \rangle  \rangle + \langle \langle u_x, u_z \rangle , u_y \rangle +  \langle \langle u_x, u_y \rangle , u_z \rangle
%       + \{  \anc \o u_x, u_y, u_z \} + \{  \anc \o u_z, u_x, u_y \} + \{  \anc \o u_y, u_z, u_x \}
%    \]
    giving the desired identity.
\end{proof}

\begin{corollary}\label{cor:inv-con-def}
    Let $(\pi:A \to M, \xi, \lambda, \anc)$ be an anchored bundle in a tangent category with negatives, with anchored connection $(\nabla,\kappa)$ on $A$ and torsion-free affine connection $(\nabla',\kappa')$ on $M$.
    An involution algebroid structure on $A$ is equivalent to a bilinear map
    \[
        \<-,-\>:A_2 \to A 
    \]
    with derived maps 
    \begin{gather*}
        \{-,-\}: A \ts{\pi}{p} TM \to TM := \kappa' \o T\anc \o (\pi_0,\pi_1), \\
        \{-,-,-\}: TM \ts{p}{\pi \o\pi_i} A_2 \to A; \{a, u_1, u_2\} := \kappa \o T(\<-,-\>) \o (\nabla(a,u_1), \nabla(a,u_2))
    \end{gather*}
    satisfying 
    \begin{enumerate}[(i)]
        \item $\< -, -\>$ is linear and cosymmetric,
        \item $\< -, -\>$ is alternating,
        \item $\< -, - \>$ and $\{-,-\}$ satisfy the Leibniz equation, Equation \ref{eq:liebniz-local}.
        \item $\< -, -\>, \{-,-\},$ and $\{-,-,-\}$ satisfy the Bianchi identity, \Cref{eq:bianchi-identity}.
    \end{enumerate}
\end{corollary}
Morphisms of involution algebroids may also be characterized by preservation of the tensor.
\begin{proposition}%
    \label{prop:inv-algd-mor-conn}
    Let $A, B$ be a pair of involution algebroids with chosen connections in a tangent category with negatives.
    Then an anchored bundle morphism $f: A \to B$ is an involution algebroid morphism if and only if (recalling the notation from Equation \ref{eq:nabla-notation})
    \[
        \nabla[f](x,y) + \< f\o x, f\o y\> = \nabla[f](y,x) + f \o \< x, y\>.
    \]
\end{proposition}
\begin{proof}
    Note that
    \begin{gather*}
        (\sigma \o \prol(f) = \prol(f) \o \sigma) \\
        \iff (\pi_1,\pi_0, \pi_2 + \< \pi_0, \pi_1 \> ) \o (f\o x, f\o y, f \o z + \nabla[f](x,y))
        \\
        = (f, f, f \o \pi_2 + \nabla[f](\pi_0,\pi_1)) \o (y,x, z + \< x, y \> ) 
    \end{gather*}
    while the second condition reduces to
    \[
        \nabla[f](x,y) + \< f\o x, f\o y\> = \nabla[f](y,x) + f \o \< x, y\>.
    \]
\end{proof}


\section{The isomorphism of Lie and involution algebroid categories}
\label{sec:the-isomorphism-of-categories}

Sections \ref{sec:Lie_algebroids} and \ref{sec:connections_on_an_involution_algebroid} have made the relationship between involution algebroids and Lie algebroids clear. It is important to note that while the proofs used connections as a tool to identify the local coherences satisfied by involution and Lie algebroids, the construction of a Lie algebroid from an involution algebroid (or vice versa) is independent of the choice of connection.

\begin{theorem}%
    \label{thm:iso-of-cats-Lie}
    There is an isomorphism of categories between Lie algebroids and involution algebroids in smooth manifolds.
\end{theorem}
\begin{proof}
    For the equivalence of categories, note that by Propositions \ref{prop:lie-alg-bil-defn} and \ref{prop:lie-algd-morphism-defn}, Corollary \ref{cor:inv-con-def}, and Proposition \ref{prop:inv-algd-mor-conn} there is an isomorphism of categories between involution algebroids with a choice of connection and Lie algebroids with a choice of connection (morphisms are \emph{not} restricted to connection preserving morphisms). This allows us to chain together isomorphisms
    \[
        \mathsf{Inv}(\mathsf{SMan}) 
        \cong \mathsf{Inv}(\mathsf{SMan})_{\mathsf{ChosenConn}}
        \cong \mathsf{LieAlgd}_{\mathsf{ChosenConn}}
        \cong \mathsf{LieAlgd}.
    \]
    \pagenote{One thing that was unclear in the original proof of this statement is that while connections are helpful for the nuts and bolts of this proof, the actual constructions do not rely on connections.}

    
    To complete the proof, we must show that the assignment that sends an involution algebroid to a Lie algebroid whose bracket is given by
    \begin{equation}\label{eq:inv-to-lie}
                \lambda \o [X,Y]^* = 
        \left( (\pi_1 \o \sigma \o (id, T.X \o \anc) \o Y -_p T.Y \o \anc \o) -_{T.\pi} 0 \o Y
        \right),
    \end{equation}
    is a bijection on objects, which brings up some subtleties. First, while an involution map 
    \[
        \sigma:\prolong \to \prolong
    \]
    is defined with respect to a particular choice of pullback $\prolong$, the category of involution algebroids \emph{does not} distinguish between different choices of this pullback (and therefore different representations of the map $\sigma$), and it is not part of the data of an involution algebroid. It is immediate by universality that the left-hand-side of Equation \ref{eq:inv-to-lie} is independent of the choice of pullback $\prolong$.
    
    Now, recall that the canonical involution of a Lie algebroid is uniquely by Theorem 4.7 of \cite{de2005lagrangian} (this was also mentioned in Corollary \ref{cor:second-lie-algd-unique-map}). Once we make a choice of prolongation $\prolong$, we have made a choice of pullback $\prol(A)$ in $\mathsf{LieAlgd}$, which uniquely determines the canonical involution
    \[
    \sigma: \prolong \to \prolong.
    \]
    While the exact map $\sigma$ depends on the choice of pullback $\prolong$, they all determine the same involution algebroid, thus proving the bijection correspondence of objects.
\end{proof}




