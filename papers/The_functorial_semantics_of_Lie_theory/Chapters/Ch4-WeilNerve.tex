% \documentclass[main.tex]{subfiles}

% \begin{document}

% Here talk about how the prolongation acts like a tangent functor. 
% 

\chapter{The Weil nerve of an algebroid}%
\label{chap:weil-nerve}\pagenote{
   This chapter has been given a new introduction to help with its exposition.
   Various typos have been fixed, but the most substantial changes are in Section \ref{sec:weil-nerve}, where the proof has been restructured to address some concerns brought up in Michael's comments.
}

The first three chapters of this thesis demonstrated that tangent categories allow for an essentially algebraic description of Lie algebroids by axiomatizing the behaviour of the tangent bundle, and showing that a Lie algebroid over a manifold $M$ is a ``generalized tangent bundle'', namely an \emph{involution algebroid}. This chapter will make precise the sense in which an involution algebroid is a generalized tangent bundle, by showing that the category of involution algebroids in a tangent category $\C$ is equivalent to a certain tangent-functor category from the free tangent category over a single object to $\C$, or more generally that there is a fully faithful functor
\[
	\mathsf{Inv}(\C) \hookrightarrow \mathsf{Tang_{Lax}}(\mathsf{FreeTangCat}(*),\C).
\]
This functor, the \emph{Weil nerve} of an involution algebroid, builds a functor from the free tangent category over a single object to $\C$ using a span construction. This chapter primarily builds on two pieces of work: Leung's construction of the free tangent category $\wone$ (\cite{Leung2017}) , and Grothendieck's original nerve construction (first published in \cite{Segal1974}).

To understand Leung's construction of the free tangent category, and more generally his actegory-theoretic presentation of tangent categories (Section \ref{sec:tang-struct-as-wone}), we first look at Weil's original insight relating the kinematic and operational descriptions of the tangent bundle in $\mathsf{SMan}$. The definition of a tangent vector on a manifold $M$ as an equivalence class of curves (Definition \ref{def:tang-vector})  puts a bijective correspondence between tangent vectors and $\R$-algebra homomorphisms from the ring of smooth functions $C^\infty(M)$ to the ring of dual numbers:
\[
	C^\infty(M) \to \R[x]/x^2.
\] The hom-set $\R\mathsf{Alg}(C^\infty(M),\R[x]/x^2)$ is precisely the set of \emph{derivations} on $C^\infty(M)$, which defines the operational tangent bundle discussed in Definition \ref{def:operational-tang}: there is a natural smooth manifold structure on this set. The Weil functor formalism, most notably developed in \cite{Kolar1993}, extends this observation to a general class of endofunctors on $\mathsf{SMan}$. For example, the fibre product  $T_2M$ corresponds to $\R$-algebra morphisms,
\[
C^\infty(M) \to R[x,y]/(x^2, y^2, xy), 
\] while the second tangent bundle corresponds to $\R$-algebra morphisms into the tensor product,
\[
C^\infty(M) \to R[x]/(x^2) \ox R[y]/(y^2) = R[x,y]/(x^2,y^2).
\] By applying Milnor's exercise (Problem 1-C \cite{Milnor1974}), which states that the $C^\infty$ functor
\[
    \mathsf{SMan} \to \R\mathsf{Alg^{op}};\hspace{0.15cm} M \mapsto C^\infty(M)=\mathsf{SMan}(M,\R)
\] is fully faithful, the structure maps occur as natural transformations. For example, the tangent projection is induced by the morphism
\[
	{p}: \R[x]/x^2 \xrightarrow[]{a + bx \mapsto a} \R,  
\] 
so that 
\[
    TM \xrightarrow[]{p} M = [C^\infty(M),\R[x]/x^2] \xrightarrow[]{(p)_*} [C^\infty(M), \R].
\]
The zero map and addition are similarly induced by
\[
	{0}: \R \xrightarrow[]{a \mapsto a + 0x} R[x]/x^2 \text{ and } +:R[x,y]/(x^2,y^2,xy) \xrightarrow[\mapsto a + (b+c)x]{a + bx + cy} R[x]/x^2,
\] 
respectively, while the lift and flip are induced by the morphisms
\[
	{\ell}: \R[x]/x^2 \xrightarrow[\mapsto a + bxy]{a+bx} \R[x,y]/(x^2,y^2) 
\]
and
\[
    {c}: \R[x,y]/(x^2,y^2) \xrightarrow[\mapsto a + cx + by + dxy]{a + bx + cy + dxy} \R[x,y]/(x^2,y^2).
\] 
More generally, there is a monoidal category of \emph{Weil algebras} (Definition \ref{def:weil-algebra-and-prol}) which has a monoidal action on the category of smooth manifolds. The Weil functor formalism, then, studies differential geometric structures from the perspective of the endofunctors and natural transformations induced by this action. Leung's insight is that there is an analogous category of commutative rigs\footnote{A rig is a ri\emph{n}g without \emph{n}egatives, i.e. a commutative monoid equipped with a bilinear multiplication.} built by replacing $\R[x]/x^2$ with $\N[x]/x^2$, called $\wone$ (Definition \ref{def:Weil-algebra}); a tangent structure is precisely a monoidal action by $\wone$ satisfying some universal properties. In particular, this category $\wone$ is precisely the free tangent category over a point, $\mathsf{FreeTang}(*)$, so that every object $A$ in a tangent category $\C$ determines a strict tangent functor
\[
    T^{(-)}A: \wone \to \C; V \mapsto T^VA
\]
and morphisms $f:A \to B$ are in bijective correspondence with tangent-natural transformations $T^{(-)}A \Rightarrow T^{(-)}(B)$.

The axioms of an involution algebroid in a tangent category $\C$ correspond bijectively with those of the tangent bundle - this suggests that an involution algebroid should determine a tangent functor from $\wone$ to $\C$. A first guess would lead one to think that $p:\N[x]/x^2 \to \N$ is sent to $\pi:A \to M$, $0$ to $\xi$, and $+$ to $+_A$. As the space of prolongations $\prol(A) = \prolong$ plays the role of the second tangent bundle, we can see that
\[
  \ell:N[x]/x^2 \to N[x,y]/(x^2,y^2) \mapsto (\xi\o\pi,\lambda): A \to \prol(A), 
\]
and 
\[
  c:N[x,y]/(x^2,y^2) \to N[x,y]/(x^2,y^2) \mapsto \sigma \prol(A) \to \prol(A).
\]
This pattern may be neatly summed up using span composition - we will construct a functor that sends $\N[x]/x^2$ to the span
% https://q.uiver.app/?q=WzAsMyxbMCwxLCJNIl0sWzEsMCwiQSJdLFsyLDEsIlRNIl0sWzEsMiwiXFxhbmMiLDFdLFsxLDAsIlxccGkiLDFdXQ==
\[\begin{tikzcd}
	& A \\
	M && TM
	\arrow["\anc"{description}, from=1-2, to=2-3]
	\arrow["\pi"{description}, from=1-2, to=2-1]
\end{tikzcd}\]
and the tensor product $\N[x]/x^2 \ox \N[x]/x^2$ to the span composition (e.g. the pullback)
\[\begin{tikzcd}
	&& {\prol(A)} \\
	& A && TA \\
	M && TM && {T^2M}
	\arrow["\anc"{description}, from=2-2, to=3-3]
	\arrow["\pi"{description}, from=2-2, to=3-1]
	\arrow["{T.\pi}"{description}, from=2-4, to=3-3]
	\arrow["{T.\anc}"{description}, from=2-4, to=3-5]
	\arrow[from=1-3, to=2-2]
	\arrow[from=1-3, to=2-4]
	\arrow["\lrcorner"{anchor=center, pos=0.125, rotate=-45}, draw=none, from=1-3, to=3-3]
\end{tikzcd}\]
which is the space of prolongations. This leads to the first major result of this chapter, the \emph{Weil Nerve} (Theorem \ref{thm:weil-nerve}), which states there is a fully faithful embedding
\[
    N_{\mathsf{Weil}}:\mathsf{Inv}(\C) \hookrightarrow [\wone, \C].
\]
This bears a strong similarity to Grothendieck's original nerve theorem, which takes an internal category $s,t:C \to M$ and constructs a functor $\Delta^{op} \to \C$ (where $\Delta^{op}$ is the monoidal theory of an internal monoid) by sending tensor to span composition, and the composition and unit maps given span morphisms
% https://q.uiver.app/?q=WzAsOCxbMCwxLCJNIl0sWzEsMCwiQ18yIl0sWzIsMSwiTSJdLFsxLDIsIkMiXSxbMywxLCJNIl0sWzQsMiwiQyJdLFs1LDEsIk0iXSxbNCwwLCJNIl0sWzEsMCwicyBcXG8gXFxwaV8wIiwxXSxbMSwyLCJ0IFxcbyBcXHBpXzEiLDFdLFszLDAsInMiLDFdLFszLDIsInQiLDFdLFsxLDMsIm0iLDFdLFs3LDQsIiIsMSx7ImxldmVsIjoyLCJzdHlsZSI6eyJoZWFkIjp7Im5hbWUiOiJub25lIn19fV0sWzcsNiwiIiwxLHsibGV2ZWwiOjIsInN0eWxlIjp7ImhlYWQiOnsibmFtZSI6Im5vbmUifX19XSxbNSw0LCJzIiwxXSxbNSw2LCJ0IiwxXSxbNyw1LCJlIiwxXV0=
\[\begin{tikzcd}
	& {C_2} &&& M \\
	M && M & M && M \\
	& C &&& C
	\arrow["{s \o \pi_0}"{description}, from=1-2, to=2-1]
	\arrow["{t \o \pi_1}"{description}, from=1-2, to=2-3]
	\arrow["s"{description}, from=3-2, to=2-1]
	\arrow["t"{description}, from=3-2, to=2-3]
	\arrow["m"{description}, from=1-2, to=3-2]
	\arrow[Rightarrow, no head, from=1-5, to=2-4]
	\arrow[Rightarrow, no head, from=1-5, to=2-6]
	\arrow["s"{description}, from=3-5, to=2-4]
	\arrow["t"{description}, from=3-5, to=2-6]
	\arrow["e"{description}, from=1-5, to=3-5]
\end{tikzcd}\]
while the unit and associativity axioms for a category are exactly the unit and associativity laws for a monoid in this setting. The \emph{Segal conditions} identify exactly the functors $C:\Delta^{op} \to \C$ that lie in the image of the nerve functor $N$ as those whose $[n]^{th}$ object is sent to the wide pullback $C([n]) = C[2] \ts{t}{s} C[2]\dots \ts{t}{s} C[2]$. The corresponding result for involution algebroids is found in Theorem \ref{thm:iso-of-cats-inv-emcs}, which states that a tangent functor $(A,\alpha):\wone \to \C$ is the nerve of an involution algebroid if and only if $A$ preserves tangent limits and $\alpha$ is a $T$-cartesian natural transformation (this forces $A.(W^{\ox n}) = A \ts{\anc}{T.\pi} TA \dots \ts{T^{n-1}\anc}{T^n\pi} T^nA$). The similarity between the Weil nerve and Grothendieck/Segal's nerve runs deep, and in Chapter \ref{ch:inf-nerve-and-realization} we demonstrate that the \emph{enriched} perspective on tangent categories puts these both into the same formal framework.
% There is a natural analogy between involution algebroids and the tangent bundle, as there is a bijective correspondence between the structure maps and the axioms they satisfy, and so it is a natural candidate for the ``classifying tangent category'' of an involution algebroid. However, the construction of a non-monoidal tangent functor $(A,\alpha):\wone \to \C$ from an involution algebroid in $\C$ poses a technical challenge. Luckily, inspiration may be drawn from a cornerstone result in category theory, Grothendieck's \emph{nerve theorem}, which shows that the internal categories in $\C$ embed into the category of simplicial objects in $\C$ $[\Delta^{op}, \C]$. The simplex category $\Delta$, a full subcategory of $\mathsf{Cat}$ whose objects are preorders $[n] = 1 < \dots < n$ (preorder morphisms are exactly functors), and its opposite category is the monoidal theory of a monoid; that is to say, any monoid in a monoidal category is given by a monoidal functor
% \[
%     (\Delta^{op}, +, [0]) \to (\C, \ox, I).
% \]
% Given an internal category $C$ in a finitely complete category $\C$, there is a base span given by the underlying graph:
% % https://q.uiver.app/?q=WzAsMyxbMCwxLCJNIl0sWzEsMCwiQyJdLFsyLDEsIk0iXSxbMSwwLCJzIiwxXSxbMSwyLCJ0IiwxXV0=
% \[\begin{tikzcd}
% 	& C \\
% 	M && M
% 	\arrow["s"{description}, from=1-2, to=2-1]
% 	\arrow["t"{description}, from=1-2, to=2-3]
% \end{tikzcd}\]
% The tensor product $[1]+[1]$ is sent to the \emph{span composition} (e.g. the pullback)
% % https://q.uiver.app/?q=WzAsNixbMCwyLCJNIl0sWzEsMSwiQyJdLFsyLDIsIk0iXSxbMywxLCJDIl0sWzQsMiwiTSJdLFsyLDAsIkNfMiJdLFsxLDAsInMiLDFdLFsxLDIsInQiLDFdLFszLDIsInMiLDFdLFszLDQsInQiLDFdLFs1LDFdLFs1LDNdLFs1LDAsIiIsMSx7ImN1cnZlIjoxfV0sWzUsNCwiIiwxLHsiY3VydmUiOi0xfV0sWzUsMiwiIiwxLHsic3R5bGUiOnsibmFtZSI6ImNvcm5lciJ9fV1d
% \[\begin{tikzcd}
% 	&& {C_2} \\
% 	& C && C \\
% 	M && M && M
% 	\arrow["s"{description}, from=2-2, to=3-1]
% 	\arrow["t"{description}, from=2-2, to=3-3]
% 	\arrow["s"{description}, from=2-4, to=3-3]
% 	\arrow["t"{description}, from=2-4, to=3-5]
% 	\arrow[from=1-3, to=2-2]
% 	\arrow[from=1-3, to=2-4]
% 	\arrow[curve={height=10pt}, from=1-3, to=3-1]
% 	\arrow[curve={height=-10pt}, from=1-3, to=3-5]
% 	\arrow["\lrcorner"{anchor=center, pos=0.125, rotate=-45}, draw=none, from=1-3, to=3-3]
% \end{tikzcd}\]
% where the monoid composition is sent to the internal category's composition map, and the identity map is sent by the unit map for the category:

% For the Weil nerve construction of an anchored bundle construction, we will generalize the prolongations of the underlying anchored bundle. An anchored bundle $(\pi:A \to M, \xi, \lambda, \anc)$ determines a span, where the object $\mathbb{N}[x]/x^2$ is sent to

% The tensor product of commutative rigs $\mathbb{N}[x]/x^2 \ox \mathbb{N}[x]/x^2$ will be sent to the following span composition:
% % https://q.uiver.app/?q=WzAsNixbMCwyLCJNIl0sWzEsMSwiQSJdLFsyLDIsIlRNIl0sWzMsMSwiVEEiXSxbNCwyLCJUXjJNIl0sWzIsMCwiXFxwcm9sKEEpIl0sWzEsMiwiXFxhbmMiLDFdLFsxLDAsIlxccGkiLDFdLFszLDIsIlQuXFxwaSIsMV0sWzMsNCwiVC5cXGFuYyIsMV0sWzUsMV0sWzUsM10sWzUsMiwiIiwxLHsic3R5bGUiOnsibmFtZSI6ImNvcm5lciJ9fV1d
% \[\begin{tikzcd}
% 	&& {\prol(A)} \\
% 	& A && TA \\
% 	M && TM && {T^2M}
% 	\arrow["\anc"{description}, from=2-2, to=3-3]
% 	\arrow["\pi"{description}, from=2-2, to=3-1]
% 	\arrow["{T.\pi}"{description}, from=2-4, to=3-3]
% 	\arrow["{T.\anc}"{description}, from=2-4, to=3-5]
% 	\arrow[from=1-3, to=2-2]
% 	\arrow[from=1-3, to=2-4]
% 	\arrow["\lrcorner"{anchor=center, pos=0.125, rotate=-45}, draw=none, from=1-3, to=3-3]
% \end{tikzcd}\]
% The morphisms in $\wone$ are then constructed using the structure maps for the involution algebroid.

Sections \ref{sec:weil-algebras-tangent-structure} and \ref{sec:tang-struct-as-wone} give a detailed introduction to the Weil functor formalism (\cite{Kolar1993},\cite{Bertram2014a}) and Leung's unification of Weil functors with tangent categories \cite{Leung2017}. The rest of the chapter contains contains new results developed by the author. Section \ref{sec:weil-nerve} proves the embedding part of the Weil nerve theorem, that the category of involution algebroids embeds into the category of tangent functors and tangent natural transformations $[\wone, \C]$. Section \ref{sec:identifying-involution-algebroids} identifies exactly those tangent functors $(A,\alpha):\wone \to \C$ that are the nerve of an involution algebroid, completing the proof of the Weil nerve theorem. Section \ref{sec:prol_tang_struct} uses the Weil nerve to develop a novel tangent structure on the category of involution algebroids in a tangent category (in particular, the category of Lie algebroids will have this novel tangent structure).


\section{Weil algebras and tangent structure}%
\label{sec:weil-algebras-tangent-structure}

This section gives a more thorough introduction to the Weil functor formalism of \cite{Kolar1993}, and in particular how the structure maps of a tangent category may be teased out of it. We begin by introducing  Weil algebras, and the \emph{prolongation} of a smooth manifold by a Weil algebra. (The relationship with prolongations of involution algebroids from Definition \ref{def:anchored_bundles} will be made clear in Section \ref{sec:weil-nerve}.)

\begin{definition}%
    \label{def:weil-algebra-and-prol}
    An $\R$-Weil algebra\footnote{Not to be confused with the normal usage of ``Weil algebra'' in Lie theory, e.g. \cite{Meinrenken2019}.} is a finite-dimensional $\R$-algebra $V$ so that $V = \R \oplus \dot{V}$ as $\R$-modules and $\dot{V}$ is a nilpotent ideal. The category $\R\mathsf{Weil}$ is the full subcategory of $\R\mathsf{Alg}$ spanned by the $\R$-Weil algebras. The \emph{prolongation} of a manifold by a Weil algebra $V$ is given by the manifold
    \[
        T^VM := \R\mathsf{Alg}(C^\infty(M,\R), V).
    \]
\end{definition}
\cite{Eck1986} showed that every product-preserving endofunctor on the category of smooth manifolds is constructed as the Weil prolongation by some Weil algebra. Consequently, $\R$-algebra homomorphisms induce natural transformations between these product-preserving  endofunctors on the category of smooth manifolds.
\begin{example}\label{ex:weil-algebras-and-maps}
    Consider the following $\R$-Weil algebras and their associated prolongation functors.
    \begin{enumerate}[(i)]
        \item Prolongation by $\R$ induces the identity functor, and the tangent bundle is given by $\R[x]/x^2$. The tangent projection, then, is equivalent to the $\R$-algebra morphism
        \[
            p: \R[x]/x^2 \to \R; \hspace{0.2cm}
            p(a + bx) = a
        \]
        while the 0-map induces the zero vector field:
        \[
            0: \R \to \R[x]/x^2; \hspace{0.2cm}
            0(a) = a + 0x.
        \]
        \item The algebra $\R[x_i]_{1 \le i \le n}/(x_ix_j)_{1 \le i \le j \le n} = (\R[x]/x^2)^n$ is the wide pullback $T_nM = TM \ts{p}{p} TM \dots \ts{p}{p} TM$. 
        In particular, prolongation by $\R[x,y]/(x^2,y^2,xy)$ gives the bundle $T_2M = TM \ts{p}{p} TM$. The $\R$-algebra morphism
        \[
            +:\R[x,y]/(x^2,y^2,xy) \to R[x]/x^2; \hspace{0.2cm}
            +(a_0 + a_1x + a_2y) = a_0 + (a_1 + a_2)x
        \]corresponds to the addition of tangent vectors.
        \item The algebra $\R[x,y]/(x^2,y^2) = (\R[x]/x^2)\ox (\R[x]/x^2)$ is the second tangent bundle $T^2M = TTM$. The vertical lift $T \to T^2$ is induced by the morphism 
        \[
            \ell: \R[x]/x^2 \to \R[x,y]/(x^2,y^2);\hspace{0.2cm}
            \ell(a + bx) = a + bxy.
        \]
        \item The monoidal symmetry map induces $c: T^2 \Rightarrow T^2$, as follows: 
        \begin{gather*}
                        c: (\R[x]/x^2)\ox (\R[y]/y^2) \to (\R[y]/y^2)\ox (\R[x]/x^2);\\ c(a + b_1x + b_2y + b_3xy) = a + b_2x + b_1y + b_3xy.        
        \end{gather*}
        %by exchanging the $x,y$ variables.
        \item For $n \ge 2$, the algebra $\R[x]/x^n$ gives the $n$-jet bundle.
        Note that this is the equalizer of $\ox^n \R[x]/x^2$ by the symmetry actions of $S_n$.
    \end{enumerate}
\end{example}
Further examples may be found in the monograph \cite{Kolar1993}. Tangent categories bridge the gap between the Weil functor approach to studying the differential geometry of smooth manifolds and the synthetic differential geometry approach of axiomatizing a tangent bundle using nilpotent infinitesimals. The main structure axiomatized here is that of \emph{monoidal action} of a symmetric monoidal category on a category $M \x \C \to \C$, or equivalently, a lift from a category to the category of complexes $\C \to [M,\C]$, which involves translating a bit of classical category theory to the 2-categorical setting.

Example  \ref{ex:weil-algebras-and-maps} leads to the classical theorem that the category of smooth manifolds has an action by the category of $\R$-Weil algebras that preserves all connected limits that exist. These ``natural'' universal properties (in the sense of \cite{Kolar1993}) is foundational to synthetic differential geometry; see, for example, Chapter Two of \cite{Lavendhomme1996}. Unfortunately, Weil algebras are not an ideal syntactic presentation: they are not a finitely presentable category, and it is not immediately clear when a diagram is a connected limit.\footnote{It should be noted that \cite{Nishimura2007} made progress applying techniques from computer algebra to latter problem.} Moving from $\R$-algebras to commutative rigs and restricting to an appropriate subcategory solves this problem.\pagenote{Clarified a point raised by Michael}

\begin{definition}[Definition 3.1 \cite{Leung2017}]
    \label{def:Weil-algebra}
    The category $\wone$ is defined to be the full subcategory of commutative rigs, $\mathsf{CRig}$, generated by the rig of dual numbers $W := \N[x]/x^2$, constructed as follows:
    \begin{enumerate}
        \item Start with finite product powers of $W$ in $\mathsf{CRig}$, and make a strict choice of presentation:
        \[
            W_0 = \N, \hspace{0.15cm} W_n := \N[x_i]/(x_ix_j)_{i \le j}, {0 \le i < n}.
        \]
        \item Then take the closure of $W_n$ under coproduct of commutative rigs, written $\ox$. Again, make a strict choice of presentation:
        \[
            W_{n(0)} \ox \dots \ox W_{n(m-1)} :=
            \N[x_{i,j}]/(x_{ij}x_{ik})_{j \le k}, 0 < i < m, 0 < j < n(i).
        \]
    \end{enumerate}
    Note that we will often suppress the tensor product $\ox$ and simply write\pagenote{explained notation used throughout this chapter}
    \[
        UV := U\ox V.
    \]
\end{definition}
\begin{proposition}[Definition 3.3 \cite{Leung2017} ]
    ~\begin{enumerate}[(i)]
        \item The category $\wone$ is a symmetric strict monoidal category with unit $\N$ and coproduct $\ox$.
        \item $\N$ is a terminal object in $\wone$.
    \end{enumerate}
\end{proposition} 
Note that there is a forgetful functor
\[
    \wone \to (\mathsf{CMon}/\N) \to \mathsf{CMon}    
\]
that reflects connected limits. This gives the following class of limits, identified in \cite{Leung2017}.
\begin{definition}%
    \label{def:transverse-limit}
    We say the following pullback diagrams in $\wone$ are \emph{transverse}:\pagenote{switched superscript to subscript in second diagram}
    \input{TikzDrawings/Ch4/Sec2/transverse-limits.tikz}
    where $\mu(a + a_1x + a_2y) = a + a_1x + a_2xy$. The $\ox$-closure of these three pullbacks is the set of \emph{transverse squares}, and they are also pullback squares by \cite{Leung2017}.
\end{definition}
To see that each transverse square in the $\ox$-closure is a pullback diagram, take the two non-identity squares and rewrite them in $\mathsf{CMon}$:
\[\input{TikzDrawings/Ch4/cmon-pb.tikz}\]
The coproduct of Weil algebras is the tensor product of the underlying commutative monoids, which are finite-dimensional and free, so these limits are closed under $\ox$.
\begin{proposition}[\cite{Leung2017} Proposition 4.1]
    The category $\wone$ is a tangent category, where the tangent functor is
    \[
        T := W \ox (\_): \wone \to \wone  
    \]
    and the natural transformations are given by 
    \begin{gather*}
        p: W \ox (\_) \xrightarrow[]{p \ox (\_)} (\_), \hspace{0.25cm}
        0: (\_) \xrightarrow[]{0 \ox (\_)} W \ox (\_), \hspace{0.25cm}
        +: W_2 \ox (\_) \xrightarrow[]{+ \ox (\_)} W \ox (\_), \\
        \ell: W \ox (\_) \xrightarrow[]{\ell \ox (\_)} W \ox W \ox (\_), \hspace{0.25cm}
        c: W \ox W \ox (\_) \xrightarrow[]{p \ox (\_)} W \ox W \ox (\_)).
    \end{gather*}
\end{proposition}

The category $\wone$ is, in some sense, a finitely presented theory. It is precisely the free tangent category on a single object:
\begin{proposition}[Proposition 9.5, \cite{Leung2017}]
    \label{thm:leung}
    The category $\wone$ is generated by the maps $\{p, 0, +, \ell, c\}$ from Example  \ref{ex:weil-algebras-and-maps}, closed under composition, tensor, and maps induced by transverse limits.
\end{proposition}
\begin{corollary}
    The category $\wone$ is the \emph{free} tangent category over a single object: every object $C$ in a tangent category $\C$ determines a strict tangent functor $T_{-}.C: \wone \to \C$, mapping
    \[
        V = W^{n{1}} \ox \dots \ox W^{n(k)}  
        \mapsto 
        T_{n(1)}.(\dots).T_{n(k)}.C = T^VC
    \]
    so that there is an isomorphism of categories between $\C$ and the category of strict tangent functors $\wone \to \C$ with tangent natural transformations as morphisms.\pagenote{Original statement was incomplete.}
\end{corollary}
\begin{notation}
    Throughout this section, the notation $T^VC$ will refer to the action of the Weil algebra $V$ on an object $C$ in a tangent category.
    In particular, we will make use of the isomorphism $T^U.T^VC = T^{UV}C$.
\end{notation}
% In some sense, then, a tangent structure is somehow a functorial choice of a ``tangent complex'' $\C \mapsto [\wone,\C]$.


\section{Tangent structures as monoidal actions}%
\label{sec:tang-struct-as-wone}

The presentation of $\wone$ as the free tangent category situates the formal theory of tangent categories as an instance of more general categorical machinery, namely monoidal actions. Recall that in a symmetric monoidal category $(\C, \ox, I)$, an internal monoid $(C, \bullet, i)$ determines a monad:
\[
    (C \ox \_: \C \to \C, \mu: \C \ox (\C \ox \_) \xrightarrow{\bullet \ox \_} \C \ox \_, 
    \eta: \_ \xrightarrow{\rho} I \ox \_ \xrightarrow{e \ox \_} C \ox \_).
\]
The category of algebras for this monad is exactly the category of $C$-modules, objects with an associative and unital action by $C$. A morphism will be a map on the base object that preserves the action:
\input{TikzDrawings/Ch4/Sec2/monoidal-action.tikz}
Strict actegories are the 2-categorical generalization of modules over a monoid. The coherences for a 2-monad follow from the coherences from a strict monoidal category in the 2-category of categories. The following proposition relies on a few facts from enriched category theory (treating the cartesian closed category $\mathsf{Cat}$ as a $\mathsf{Cat}$-enriched category, per \cite{Kelly2005}) but a more general treatment of non-strict actegories may be found in \cite{janelidze2001note}:\pagenote{Since I'm not referencing any sort of weakness, this is just enriched category theory so I think Kelly is a fair reference. However, I have included the reference to Janelidze and Kelly's work on actegories.} 
\begin{itemize}
    \item A 2-functor and 2-natural transformations are exactly a functor and natural transformations that satisfy extra coherences. These coherences follow for free by constructing the monad and comonad $\m \x \_, [\m, \_]$.
    \item An algebra of the underlying 1-monad is exactly an algebra of the 2-monad (the same result holds for comonads).
\end{itemize}
When working with algebras of a 2-monad, four different notions of morphisms can come into play (\cite{Lack2009}). These arise through using the 2-categorical data to weaken the notion of a morphism:
\begin{enumerate}[(i)]
    \item Strict: this is exactly a morphism of the underlying algebras. Write the 2-category of strict $\m$-actegories.
    \item Strong: the morphisms preserve the action to an isomorphism:
    \[\input{TikzDrawings/Ch4/Sec2/act-mor-strong.tikz}\]
    \item Lax: the 2-cell is no longer an isomorphism:
    \[\input{TikzDrawings/Ch4/Sec2/act-mor-lax.tikz}\]
    \item Oplax: the 2-cell travels in the opposite direction (these will not figure into this account).
\end{enumerate}
2-cells between actegory morphisms must satisfy a coherence between the natural transformation parts of the actegory morphisms.
\begin{definition}\label{def:actegory-natural}
    In the case of strict, strong, and lax tangent functors, the same notion of a 2-cell applies: a natural transformation $\gamma:F \Rightarrow G$ satisfying the following coherences with the natural transformations $\alpha$ and $\beta$:
% https://q.uiver.app/?q=WzAsOCxbMCwwLCJcXG1cXHhcXEMiXSxbMSwwLCJcXG1cXHhcXEQiXSxbMSwxLCJcXEQiXSxbMCwxLCJcXEMiXSxbMiwwLCJcXG1cXHhcXEMiXSxbMywwLCJcXG1cXHhcXEQiXSxbMiwxLCJcXEMiXSxbMywxLCJcXEQiXSxbMCwxLCJcXG1cXHggRiJdLFsxLDIsIlxccHJvcHRvXlxcRCJdLFswLDMsIlxccHJvcHRvXlxcQyIsMl0sWzMsMiwiRiIsMV0sWzEsMywiXFxhbHBoYSIsMSx7ImxldmVsIjoyfV0sWzMsMiwiRyIsMix7ImN1cnZlIjozfV0sWzQsNSwiXFxtXFx4IEciLDIseyJsYWJlbF9wb3NpdGlvbiI6MjB9XSxbNCw1LCJcXG1cXHggRiIsMCx7ImN1cnZlIjotM31dLFs0LDYsIlxccHJvcHRvXlxcQyIsMl0sWzUsNywiXFxwcm9wdG9eXFxEIl0sWzYsNywiRyIsMl0sWzUsNiwiXFxiZXRhIiwwLHsibGV2ZWwiOjJ9XSxbMTEsMTMsIlxcZ2FtbWEiLDAseyJzaG9ydGVuIjp7InNvdXJjZSI6MjAsInRhcmdldCI6MjB9fV0sWzE1LDE0LCJcXG1cXHggXFxnYW1tYSIsMix7Im9mZnNldCI6LTIsInNob3J0ZW4iOnsic291cmNlIjoyMCwidGFyZ2V0IjoyMH19XSxbOSwxNiwiPSIsMSx7InNob3J0ZW4iOnsic291cmNlIjoyMCwidGFyZ2V0IjoyMH0sInN0eWxlIjp7ImJvZHkiOnsibmFtZSI6Im5vbmUifSwiaGVhZCI6eyJuYW1lIjoibm9uZSJ9fX1dXQ==
\[\begin{tikzcd}
	\m\x\C & \m\x\D & \m\x\C & \m\x\D \\
	\C & \D & \C & \D
	\arrow["{\m\x F}", from=1-1, to=1-2]
	\arrow[""{name=0, anchor=center, inner sep=0}, "{\propto^\D}", from=1-2, to=2-2]
	\arrow["{\propto^\C}"', from=1-1, to=2-1]
	\arrow[""{name=1, anchor=center, inner sep=0}, "F"{description}, from=2-1, to=2-2]
	\arrow["\alpha"{description}, Rightarrow, from=1-2, to=2-1]
	\arrow[""{name=2, anchor=center, inner sep=0}, "G"', curve={height=18pt}, from=2-1, to=2-2]
	\arrow[""{name=3, anchor=center, inner sep=0}, "{\m\x G}"'{pos=0.2}, from=1-3, to=1-4]
	\arrow[""{name=4, anchor=center, inner sep=0}, "{\m\x F}", curve={height=-18pt}, from=1-3, to=1-4]
	\arrow[""{name=5, anchor=center, inner sep=0}, "{\propto^\C}"', from=1-3, to=2-3]
	\arrow["{\propto^\D}", from=1-4, to=2-4]
	\arrow["G"', from=2-3, to=2-4]
	\arrow["\beta", Rightarrow, from=1-4, to=2-3]
	\arrow["\gamma", shorten <=2pt, shorten >=2pt, Rightarrow, from=1, to=2]
	\arrow["{\m\x \gamma}"', shift left=2, shorten <=2pt, shorten >=2pt, Rightarrow, from=4, to=3]
	\arrow["{=}"{description}, Rightarrow, draw=none, from=0, to=5]
\end{tikzcd}\]
We call these \emph{actegory natural transformations}. 
\end{definition}
Note that for strict actegory morphisms, this condition holds for any natural transformation $\gamma:F \Rightarrow G$. Now consider the following three 2-categories.\pagenote{Added the actual definition of the 2-categories referenced later on. I also added the coherence for actegory-natural transformation}
\begin{definition}\label{def:2-categories}
    Let $(\m,\ox,I)$ be a strict monoidal category. Define the following three 2-categories.
    \begin{enumerate}
        \item $\m\mathsf{Act}_{\mathsf{strict}}$: the 2-category of strict $\m$-actegories, strict actegory morphisms, and natural transformations.
        \item $\m\mathsf{Act}_{\mathsf{strong}}$: the 2-category of strict $\m$-actegories, strong actegory morphisms, and actegory natural transformations.
        \item $\m\mathsf{Act}_{\mathsf{lax}}$: the 2-category of strict $\m$-actegories, lax actegory morphisms, and actegory natural transformations.
    \end{enumerate}
    Note that the inclusions of these 2-categories are \emph{locally} fully faithful, so only the 1-cells differ.
\end{definition}

The case where the action preserves certain limits in the monoidal category is of particular interest. A small category equipped with a class of chosen limits is known as a \emph{sketch}. The previous correspondence restricts to the class of limit-preserving actions in this case.
\begin{definition}
    A \emph{sketch} is a small category with a class of chosen limits, and a sketch morphism is a functor sending chosen limits to the chosen limits in the domain (up to isomorphism). The category of models of a sketch $\c$ in a category $\C$, $\mathsf{Mod}(\c, \C)$, is the full subcategory $[\c, \C]$ whose functors preserve the chosen limits.
    A \emph{monoidal sketch}, then, is a sketch $(\c, \prol)$ equipped with a symmetric monoidal category structure on $\c$ so that $\_ \ox \_$ preserves limits in each argument.
\end{definition}

Now use the fact that the category $\wone$ is a monoidal sketch, since it is a small, strict monoidal category equipped with a class of limits stable under the tensor product.
\begin{theorem}[Theorem 14.1, \cite{Leung2017}]
    Let $\C$ be a category. The following are equivalent:
    \begin{enumerate}[(i)]
        \item A tangent structure on $\C$,
        \item A sketch action $\propto: \wone \x \C \to \C$.
        % \item A sketch lift $\phi: \C \to \mathsf{Mod}(\wone, \C)$.
    \end{enumerate}
\end{theorem}

\begin{observation}%
    \label{obs:cofree-tangent-cat}
    There is a \emph{coalgebraic} perspective on tangent categories, coming from the equivalence between algebras of the 2-monad $(\wone \x (\_), \ox, I:1 \to \wone)$ and the 2-comonad $([\wone,\_], [\ox,\_], [I,\_])$.
    For any  category $\C$, there is a free tangent category given by
    \[
        \wone \x \C
    \] and this agrees with the free $\wone$-actegory. However, for the \emph{cofree} tangent category, take
    \[
        \mathsf{Mod}(\wone, \C),
    \]
    the category of transverse-limit-preserving functors $\wone \to \C$.
\end{observation}

We can use Leung's theorem to induce a monoidal functor $\wone \to \C$ when a tangent structure is induced by a single object.
\begin{corollary}\label{cor:using-leung-thm}
    Let $(\C,\ox,I)$ be a strict monoidal category.\pagenote{I've added this intermediary result to make the Weil nerve a bit more explicit.}
    If an additive bundle $(p:A \to I, +:A_2 \to A, 0:I \to A)$ equipped with morphisms
    \[
       A \ox A \xrightarrow[]{c} A \ox A \hspace{0.5cm} A \xrightarrow[]{\ell} A \ox A
    \]
    determines a tangent structure on $\C$ using the endofunctor $A \ox (-)$, then $A$ determines a strict, transverse-limit-preserving, monoidal functor 
    \[
       A(-):\wone \to \C; W_{n[1]}\ox \dots \ox W_{n[k]} \mapsto A_{n[1]}\ox \dots \ox A_{n[k]} = T_{n[1]}\dots T_{n[k]}.I
    \]
\end{corollary}
Note that this allows for a more conceptual description of representable tangent structure.
\begin{proposition}\label{prop:monoidal-functor-inf-obj}
    In a symmetric monoidal closed category, an infinitesimal object is exactly a strict symmetric monoidal functor $D:\wone \to \C$.\pagenote{The other statement was probably more general than necessary - this statement is more to the point.}
\end{proposition}
This presentation of an infinitesimal object makes it tautological that $\C^{op}$ has a tangent structure.
\begin{corollary}%
    \label{cor:dual-tangent-structure}
    Given a strict symmetric monoidal functor
    \[
        D:\wone \to \C^{op}
    \]
    there is a strict action of $\wone$ on $\C^{op}$ given by
    \[
        \wone \x \C^{op} \xrightarrow[]{D^{op} \ox \C} \C^{op} \x \C^{op} \xrightarrow[]{\otimes} \C^{op}. 
    \]
\end{corollary}


There is a clear correspondence between the notions of a (strict, strong, lax) tangent functor and a (strict, strong, lax) actegory morphism. This proposition extends to the following equivalence of 2-categories.
\begin{corollary}[Theorem 14.1 \cite{Leung2017}]
    The following pairs of 2-categories are equivalent.
    \begin{enumerate}[(i)]
        \item The 2-category of tangent categories and strict tangent functors is the full sub-2-category of $\wone\mathsf{Act_{Strict}}$ spanned by sketch actions.
        \item The 2-category of tangent categories and strong tangent functors is the full sub-2-category of $\wone\mathsf{Act_{strong}}$ spanned by sketch actions.
        \item The 2-category of tangent categories and lax tangent functors is the full sub-2-category of $\wone\mathsf{Act_{Lax}}$ spanned by sketch actions.
    \end{enumerate}
\end{corollary}



% Make sure you use 
\section{The Weil nerve of an involution algebroid}
\label{sec:weil-nerve}

The construction in this section is analagous to the nerve of an internal category---hence the ``Weil nerve'' construction---and deals with similar technical issues. In particular, the construction in this section will mimic the nerve construction for internal categories by replacing the tensor product of $\wone$ with span composition in the domain category. Recall that every anchored bundle or internal category has a canonical span associated with it:
\[\input{TikzDrawings/Ch4/Sec3/anc-bundle.tikz}\]
In any category $\C$, there is a category of spans in $\C$ as well as span composition.
\begin{definition}%
    \label{def:span-stuff}
    A \emph{span from $A$ to $B$} in a category $\C$ is a diagram of the form 
    \[\input{TikzDrawings/Ch4/Sec3/span.tikz}\]
    There is a notion of \emph{span composition}, so given a span $X:A \to B$ and $Y:B \to C$, then the composition of $X$ and $Y$ is the pullback (if it exists):
    \[\input{TikzDrawings/Ch4/Sec3/span-hor-comp.tikz}\]
    A morphism of spans is a commuting diagram of the form
    \[\input{TikzDrawings/Ch4/Sec3/span-morp.tikz}\]
    Note that if $f$ and $g$ are span morphisms with $f_r = g_l$, then the horizontal composition may be formed if each respective span composition exists:
    \[\input{TikzDrawings/Ch4/Sec3/span-hor-comp-morp.tikz}\]
    When discussing span composition in a tangent category, it is assumed that the pullback is a $T$-pullback.
\end{definition}
\begin{observation}%
    \label{obs:lim-of-spans}
    Note that the category of spans in $\C$ is a functor category, so that limits are computed pointwise in $\C$. This also means that the horizontal composition operation, when it exists, preserves limits in either argument.
\end{observation}
These span constructions can be helpful in constructing functors from a monoidal category into a non-monoidal category $\C$, by forming a monoidal category from $\C$ using spans. In the case of an internal category over $M$, one takes the slice category $\C/(M \x M)$ where the tensor product is span composition. An internal category $s,t:C \to M$ is a monoid in this category of spans over $M$, so that it determines a monoidal functor
\[
    C: \Delta^{op} \to \C/(M \x M),    
\] 
remembering that $\Delta^{op}$ is the monoidal theory for monoid (every monoid in a monoidal category $\C$ determines a monoidal functor $\Delta \to \C$). The construction of the corresponding monoidal category for spans is more nuanced, as the category $\wone$ is not $\N$-indexed. Observe that the prolongation of an anchored bundle is constructed as a span composition:
\input{TikzDrawings/Ch4/Sec3/prol-span-comp.tikz}
The third prolongation is given by span composition as well:
\[\input{TikzDrawings/Ch4/Sec3/prol-span-comp-2.tikz}\]
This horizontal composition will play the same role as the tensor product in $\C/(M\x M)$.
\begin{definition}%
    \label{def:boxtimes-span}
    In a tangent category $\C$, consider a pair of spans
    \[
        X:M \to T^UM, \hspace{0.15cm} Y:M \to T^VM.
    \]
    Define $X \boxtimes Y$ to be the horizontal composition (when it exists):
    \[\input{TikzDrawings/Ch4/boxtimes-span.tikz}\]
    (recall that we will often suppress the $\ox$ in $\wone$ to save space).\pagenote{clarifying notation}
    So the span composition is
    \[
        M \xrightarrow[]{X} T^UM \xrightarrow[]{T^U.Y} T^U.T^VM    
    \]
    \begin{equation}
        \label{eq:span-form}
        \input{TikzDrawings/Ch4/span-form.tikz}
    \end{equation}
    The horizontal composition $f \boxtimes g$ is defined as $f \x_{\theta.M} \theta.g$:
    \[\input{TikzDrawings/Ch4/boxtimes-diag.tikz}\]
\end{definition}
% \begin{lemma}
%     Consider an object $M$ in a tangent category $\C$, and define the span:
%     \[
%         p^U := M \xleftarrow[]{p^U} T^UM = T^UM  
%     \]
%     This object behaves like the unit
% \end{lemma}
% \begin{observation}
In any tangent category with a tangent display system (Definition\pagenote{there was no reason to pull this out as an observation} \ref{def:display-system}), the category of spans on $M$ whose maps are of the form given by Equation \ref{eq:span-form} with $l \in \d$ is a monoidal category.
% \end{observation}
Any anchored bundle in a tangent category gives rise to a monoidal category after a strict choice of $T$-pullbacks (assuming those $T$-pullbacks exist).
\begin{definition}\label{def:monoidal-category}
    Let $(\pi:A \to M,\xi,\lambda,\anc)$ be an anchored bundle in a tangent category $\C$. Write the span
    \[
        \widehat{A}.W_n := (M \xleftarrow[]{\pi \o \pi_i} A_n \xrightarrow[]{\anc^n} T_nM),
        \hspace{0.5cm}
        \widehat{A}.\N := (M = M = M).
    \]
    A \emph{choice of prolongations} for $(\pi,\xi,\lambda,\anc)$ is a strict choice of horizontal composition for each $V \in \wone$:
    \[
        \widehat{A}.V = \widehat{A}.(W_{n[1]}\dots W_{n[k]}) :=
        \widehat{A}.W_{n[1]} \boxtimes \dots \boxtimes \widehat{A}.W_{n[k]}.
    \]
    We will write the span as follows:
    % https://q.uiver.app/?q=WzAsMyxbMCwxLCJNIl0sWzEsMCwiQS5WIl0sWzIsMSwiVF5WLk0iXSxbMSwwLCJcXHBpXlYiLDJdLFsxLDIsIlxcYW5jXlYiXV0=
\[\begin{tikzcd}
	& {A.V} \\
	M && {T^V.M}
	\arrow["{\pi^V}"', from=1-2, to=2-1]
	\arrow["{\anc^V}", from=1-2, to=2-3]
\end{tikzcd}\]
    (notice that the apex is not hatted).
    Given a choice of prolongations for an anchored bundle $(\pi,\xi,\lambda,\anc)$, the category $\mathsf{Span}(\pi,\xi,\lambda,\anc)$ is defined as follows:
    \begin{itemize}
        \item Objects are $\widehat{A}.V$ for $V \in \wone$.
        \item Morphisms are given by pairs
        \[
            (f,\phi):\widehat{A}.V \to \widehat{A}.U
        \]
        where $f:A.V \to A.U$ and $\phi:V \to U$ determine a span morphism of the form
        % https://q.uiver.app/?q=WzAsNixbMCwwLCJNIl0sWzEsMCwiQS5WIl0sWzIsMCwiVF5WLk0iXSxbMiwxLCJUXlUuTSJdLFswLDEsIk0iXSxbMSwxLCJBLlYiXSxbMSwwLCJcXHBpXlYiLDJdLFsxLDIsIlxcYW5jXlYiXSxbMiwzLCJcXHBoaS5NIl0sWzAsNCwiIiwyLHsibGV2ZWwiOjIsInN0eWxlIjp7ImhlYWQiOnsibmFtZSI6Im5vbmUifX19XSxbNSw0LCJcXHBpXlYiXSxbNSwzLCJcXGFuY15VIiwyXSxbMSw1LCJmIl1d
\[\begin{tikzcd}
	M & {A.V} & {T^V.M} \\
	M & {A.U} & {T^U.M}
	\arrow["{\pi^V}"', from=1-2, to=1-1]
	\arrow["{\anc^V}", from=1-2, to=1-3]
	\arrow["{\phi.M}", from=1-3, to=2-3]
	\arrow[Rightarrow, no head, from=1-1, to=2-1]
	\arrow["{\pi^V}", from=2-2, to=2-1]
	\arrow["{\anc^U}"', from=2-2, to=2-3]
	\arrow["f", from=1-2, to=2-2]
\end{tikzcd}\]
        as discussed in Definition \ref{def:span-stuff}.
        \item Tensor structure: The tensor product is defined using the horizontal composition $\boxtimes$ as defined in Definition \ref{def:span-stuff}.
    \end{itemize}
\end{definition}
The idea is to show that an involution algebroid structure on an anchored bundle induces a tangent structure on the monoidal category of prolongations, and then to apply Leung's theorem. 
The following two lemmas will simplify this proof.
\begin{lemma}\label{lemma:pullback-part-of-theorem}
    Let $(\pi:A \to M, \xi, \lambda, \anc)$ be an anchored bundle with chosen prolongations in a tangent category $\C$, and identify the monoidal category $\mathsf{Span}(\pi,\xi,\lambda,\anc)$.
    \begin{enumerate}[(i)]
        \item There is a functor 
        \[
           U^\anc: \mathsf{Span}(\pi,\xi,\lambda,\anc) \to \C
        \]
        constructed by sending a span morphism to the morphism between the objects at its apex:
% https://q.uiver.app/?q=WzAsOCxbMCwwLCJNIl0sWzEsMCwiQS5WIl0sWzIsMCwiVF5WLk0iXSxbMiwxLCJUXlUuTSJdLFswLDEsIk0iXSxbMSwxLCJBLlUiXSxbMywwLCJBLlYiXSxbMywxLCJBLlUiXSxbMSwwLCJcXHBpXlYiLDJdLFsxLDIsIlxcYW5jXlYiXSxbMiwzLCJcXHBoaS5NIiwyXSxbMCw0LCIiLDIseyJsZXZlbCI6Miwic3R5bGUiOnsiaGVhZCI6eyJuYW1lIjoibm9uZSJ9fX1dLFs1LDQsIlxccGleViJdLFs1LDMsIlxcYW5jXlUiLDJdLFsxLDUsImYiXSxbNiw3LCJmIl0sWzEwLDE1LCIiLDIseyJzaG9ydGVuIjp7InNvdXJjZSI6NDAsInRhcmdldCI6NDB9LCJsZXZlbCI6MSwic3R5bGUiOnsidGFpbCI6eyJuYW1lIjoibWFwcyB0byJ9fX1dXQ==
\[\begin{tikzcd}
	M & {A.V} & {T^V.M} & {A.V} \\
	M & {A.U} & {T^U.M} & {A.U}
	\arrow["{\pi^V}"', from=1-2, to=1-1]
	\arrow["{\anc^V}", from=1-2, to=1-3]
	\arrow[""{name=0, anchor=center, inner sep=0}, "{\phi.M}"', from=1-3, to=2-3]
	\arrow[Rightarrow, no head, from=1-1, to=2-1]
	\arrow["{\pi^V}", from=2-2, to=2-1]
	\arrow["{\anc^U}"', from=2-2, to=2-3]
	\arrow["f", from=1-2, to=2-2]
	\arrow[""{name=1, anchor=center, inner sep=0}, "f", from=1-4, to=2-4]
	\arrow[shorten <=14pt, shorten >=14pt, maps to, from=0, to=1]
\end{tikzcd}\]
        \item Suppose we have a square
% https://q.uiver.app/?q=WzAsNCxbMCwxLCJcXHdpZGVoYXR7QX0uWCJdLFsxLDEsIlxcd2lkZWhhdHtBfS5aIl0sWzEsMCwiXFx3aWRlaGF0e0F9LlkiXSxbMCwwLCJcXHdpZGVoYXR7QX0uVSJdLFswLDEsIihmLFxccGhpKSIsMl0sWzIsMSwiKGcsXFxwc2kpIl0sWzMsMCwiKGwsXFxhbHBoYSkiLDJdLFszLDIsIihyLFxcYmV0YSkiXV0=
\[\begin{tikzcd}
	{\widehat{A}.U} & {\widehat{A}.Y} \\
	{\widehat{A}.X} & {\widehat{A}.Z}
	\arrow["{(f,\phi)}"', from=2-1, to=2-2]
	\arrow["{(g,\psi)}", from=1-2, to=2-2]
	\arrow["{(l,\alpha)}"', from=1-1, to=2-1]
	\arrow["{(r,\beta)}", from=1-1, to=1-2]
\end{tikzcd}\]
        whose image under $U^\anc$ is a $T$-pullback in $\C$, and so that the square in $\wone$ is a transverse $T$-pullback:
% https://q.uiver.app/?q=WzAsOCxbMCwxLCJ7QX0uWCJdLFsxLDEsIntBfS5aIl0sWzEsMCwie0F9LlkiXSxbMCwwLCJ7QX0uVSJdLFsyLDAsIlUiXSxbMywwLCJZIl0sWzIsMSwiWCJdLFszLDEsIloiXSxbMCwxLCJmIiwyXSxbMiwxLCJnIl0sWzMsMCwibCIsMl0sWzMsMiwiciJdLFs2LDcsIlxccGhpIiwyXSxbNSw3LCJcXHBzaSJdLFs0LDUsIlxcYmV0YSJdLFs0LDYsIlxcYWxwaGEiLDJdLFszLDEsIiIsMSx7InN0eWxlIjp7Im5hbWUiOiJjb3JuZXIifX1dLFs0LDcsIiIsMSx7InN0eWxlIjp7Im5hbWUiOiJjb3JuZXIifX1dXQ==
\[\begin{tikzcd}
	{{A}.U} & {{A}.Y} & U & Y \\
	{{A}.X} & {{A}.Z} & X & Z
	\arrow["f"', from=2-1, to=2-2]
	\arrow["g", from=1-2, to=2-2]
	\arrow["l"', from=1-1, to=2-1]
	\arrow["r", from=1-1, to=1-2]
	\arrow["\phi"', from=2-3, to=2-4]
	\arrow["\psi", from=1-4, to=2-4]
	\arrow["\beta", from=1-3, to=1-4]
	\arrow["\alpha"', from=1-3, to=2-3]
	\arrow["\lrcorner"{anchor=center, pos=0.125}, draw=none, from=1-1, to=2-2]
	\arrow["\lrcorner"{anchor=center, pos=0.125}, draw=none, from=1-3, to=2-4]
\end{tikzcd}\]
      Then $U^\anc$ reflects the limit; that is, the original square in $\mathsf{Span}(\pi,\xi,\lambda,\anc)$ is a $T$-pullback.
      \item $T$-pullbacks of the form described in (ii) are closed under $\boxtimes$.
    \end{enumerate}
\end{lemma}
\begin{proof}
    The functor in in (i) is straightforward to construct, as it simply forgets the left and right legs of the spans. For (ii), note that because the $\wone$ part of the diagram is a transverse $T$-pullback, then given a pair of maps
    % https://q.uiver.app/?q=WzAsNSxbMSwyLCJcXHdpZGVoYXR7QX0uWCJdLFsyLDIsIlxcd2lkZWhhdHtBfS5aIl0sWzIsMSwiXFx3aWRlaGF0e0F9LlkiXSxbMSwxLCJcXHdpZGVoYXR7QX0uVSJdLFswLDAsIlxcd2lkZWhhdHtBfS5WIl0sWzAsMSwiKGYsXFxwaGkpIiwyXSxbMiwxLCIoZyxcXHBzaSkiXSxbMywwLCIobCxcXGFscGhhKSIsMl0sWzMsMiwiKHIsXFxiZXRhKSJdLFs0LDAsIih4LFxcb21lZ2EpIiwyLHsiY3VydmUiOjJ9XSxbNCwyLCIoeSxcXGdhbW1hKSIsMCx7ImN1cnZlIjotMn1dXQ==
\[\begin{tikzcd}
	{\widehat{A}.V} \\
	& {\widehat{A}.U} & {\widehat{A}.Y} \\
	& {\widehat{A}.X} & {\widehat{A}.Z}
	\arrow["{(f,\phi)}"', from=3-2, to=3-3]
	\arrow["{(g,\psi)}", from=2-3, to=3-3]
	\arrow["{(l,\alpha)}"', from=2-2, to=3-2]
	\arrow["{(r,\beta)}", from=2-2, to=2-3]
	\arrow["{(x,\omega)}"', curve={height=12pt}, from=1-1, to=3-2]
	\arrow["{(y,\gamma)}", curve={height=-12pt}, from=1-1, to=2-3]
\end{tikzcd}\]
    a unique span morphism $\widehat{A}.V \to \widehat{A}.U$ may be induced using the apex map from $\C$, and the unique map induced in $\wone$ by the universality of transverse squares (this square is also universal in $\C$), so the span morphism diagram will commute by universality.
    
    For (iii), $T$-pullback squares of the form in (ii) are closed under $\boxtimes$ as transverse squares in $\wone$ are closed under $\ox$, so the result follows by the commutativity of limits and by applying part (ii) of this lemma. 
\end{proof}
% \begin{definition}%
% \label{def:prolongation-of-anchored-bundle}
%     Let $(\pi:A \to M, \xi, \lambda, \anc)$ be an anchored bundle in a tangent category $\C$. 
%     This anchored bundle defines a mapping on objects:
%     \[
%         \hat A: \mathsf{Ob}(\wone) \to \mathsf{Ob}(\C); V \mapsto \hat A .V
%     \]  
%     defined inductively using a choice of span compositions from Definition \ref{def:boxtimes-span}.

%     For $W_n$, the object is $A.W_n$. There is a canonical span:
%     \[
%       A_n: M \to T_nM := M \xleftarrow[]{\pi \o \pi_i} A_n \xrightarrow[]{\anc_n} T_n.M    
%     \]
%     The \emph{prolongation of $A$ by $V \in \wone$} is the span composition:
%     \[
%       V = W_{n(1)} \ox \dots W_{n(k)} \mapsto
%       \hat A. W_{n(1)} \boxtimes \dots \boxtimes\hat A. W_{n(k)} =: \hat . V
%     \]
%     Write the span maps:
%     \[
%         \hat A.V : M \to T^VM :=
%         M \xleftarrow{\pi^V} \hat A.V  \xrightarrow{\anc^V} T^VM  
%     \]
%     An anchored bundle has \emph{chosen prolongations} when all $\hat A. V$ exist and for each $U,V$ the equation:
%     \[
%         \hat A. U \boxtimes \hat A. V = \hat A. UV   
%     \]
%     holds. The $V$-prolongation of $(\pi, \xi, \lambda, \anc)$ is the apex of the $V$-span. 
% \end{definition}

% Whenever an anchored bundle has chosen prolongations, it is possible to identify the following monoidal category.
%TODO fill in this definition
%TODO fill in this proposition
% The tensor product 
% The monoidal category to be construction for an involution algebroid, then, takes the span determined by its underlying anchored bundle, and uses span composition to construct a span for each Weil algebra $V \in \wone$. The construction is inductive, and is outlined below:
% A ``higher order'' prolongation may be constructed for each Weil algebra $U \in \wone$, following the metaphor that the space of prolongations $\prol(A)$ is like the second tangent bundle, and $\prol^2(A)$ the third tangent bundle.



\begin{observation}%
    \label{obs:concrete-desc-zigzag}
    It will be useful to have a ``flat'' presentation of the prolongation $A.W_{n(1)}\dots W_{n(k)}$. 
    % Inductively, look at the prolongation $\prol(W_mW_nV,A)$:
    % \[\input{TikzDrawings/Ch4/prolong-ind-case.tikz}\]
    % But apply the same construction for $T_m.\prol(W_nV,A)$ to get the double pullback:
    % \[\input{TikzDrawings/Ch4/prolong-ind-case-2.tikz}\]
    The higher prolongations of an anchored bundle may be concretely described as the $T$-pullback of the zig-zag below:
    \[\input{TikzDrawings/Ch4/Sec3/prol-zigzag.tikz}\]
    so that the prolongation $A.W^{n[1]}\dots W^{n[k]}$ may be written concretely as
    \[
      (u_1,\dots, u_{k}) : A_{n[1]} \ts{\anc'}{T.\pi'}  T_{n[1]}.A_{n[2]} \ts{T.\anc}{T^2.\pi} \dots \ts{\anc'}{T.\pi'} T_{n[1]\dots n[k-1]}A_{n[k]}.
    \]
    Furthermore, the choice of prolongation identifies the following limits:
    \[\input{TikzDrawings/Ch4/prol-abuse.tikz}\input{TikzDrawings/Ch4/prol-abuse-2.tikz}\]
    so that
    \[
        A. UV
        = A.U \boxtimes A.V
        = A.U \ \boxtimes id_M \boxtimes A.V
    \]
    where $id_M$ is the span $M = M = M$.
\end{observation}

% \begin{example}
%     ~\begin{enumerate}[(i)]
%         \item For the tangent bundle, $\widehat{TM}.V = T^VM$. 
%         \item For a trivial bundle, $(\pi:A \to M, \xi, \lambda, 0 \o \pi)$, then $A_V = A_{|V|}$ where $|V| \in \N$ denotes the dimension of the underlying free commutative monoid of $V$.
%     \end{enumerate}
% \end{example}

%
% \begin{definition}
%     \label{def:boxtimes}
%     Let $(\pi:A \to M, \xi, \lambda, \anc)$ be an anchored bundle with chosen prolongations. 
%     For any pair of span morphisms $f:U \to U', g:V \to V'$ of the form:
%     \begin{equation}
%         \label{eq:span-form}
%         \input{TikzDrawings/Ch4/span-form.tikz}
%     \end{equation}
%     The horizontal composition $f \boxtimes g$ is defined:
%     \[\input{TikzDrawings/Ch4/boxtimes-diag.tikz}\]
% \end{definition}
% \begin{lemma}
%     Let $(\pi:A \to M, \xi, \lambda, \anc)$ be an anchored bundle with chosen prolongations.
%     Consider a commuting cospan in $\mathsf{Span}(\C)$ of the form:

%     Now the 
% \end{lemma}
% Given an anchored bundle $(\pi:A \to M, \xi,\lambda, \anc)$, a choice of prolongations allows for the construction of a strict monoidal category with a bijective correspondence on objects with the category of Weil algebras.
                   
Note that a $\anc$ sends the involution algebroid structure map to its corresponding tangent structure map. Each of the structure maps, then, gives a span morphism where $\boxtimes$ is well defined:
\begin{definition}%
    \label{def:generators-for-wone-in-anc}
    Let $(\pi:A \to M, \xi, +_q, \lambda, \anc, \sigma)$ be an involution algebroid in $\C$ with chosen prolongations.
    Then define the following maps in $\mathsf{Span}(\pi,\xi,\lambda,\anc)$:
    \begin{itemize}
        \item The projection $p: \hat{A}.W \to \hat{A}.\N$,
        \[\input{TikzDrawings/Ch4/Sec4/proj.tikz}\]
        \item The zero map $0: \hat{A}.\N \to \hat{A}.W$,
        \[\input{TikzDrawings/Ch4/Sec4/zero.tikz}\]
        \item The addition map $+:  \hat{A}.W_2 \to \hat{A}.W$,
        \[\input{TikzDrawings/Ch4/Sec4/add.tikz}\]
        \item The lift map $\ell: \hat{A}.W \to \hat{A}.WW$,
        \[\input{TikzDrawings/Ch4/Sec4/lift.tikz}\]
        \item The flip map $c: \hat{A}.WW \to \hat{A}.WW$,
        \[\input{TikzDrawings/Ch4/Sec4/flip.tikz}\]
    \end{itemize}
\end{definition}
The idea is to show that the monoidal category of chosen prolongations $\mathsf{Span}(\pi,\xi,\lambda,\anc)$ for an involution algebroid has a tangent structure generated by the structure maps in Definition \ref{def:generators-for-wone-in-anc} and the endofunctor $\widehat{A}.W \boxtimes (-)$. Using the flat presentation, we can then show that $U^\anc$ will determine a tangent functor in $\C$. The following lemma about $\anc$ will be useful in constructing the natural transformation part of a tangent functor.
% The idea is to prove that the generators define a functor $\wone \to \C$. 
% First, following table associates maps in $\wone$ to the structure maps of an involution algebroid:
% \begin{center}
%     \begin{tabular}{|l|l|l|l|l|l|l|}
%     \hline
%     Tangent bundle & $T^VM$ & $p$   & $0$   & $+$     & $\ell$    & $c$      \\ \hline
%     Involution algebroid & $\hat{A}.V$     & $\pi$ & $\xi$ & $+_\pi$ & $\hat\lambda$ & $\sigma$ \\ \hline
%     \end{tabular}
% \end{center}
% Having a flat presentation of these maps will prove useful.
% \begin{proposition}\label{def:higher-morphisms}
%     Let $(\pi:A \to M, \xi, \lambda, \sigma)$ be an involution algebroid with a chosen prolongations. By the earlier observation, identify $\prolong$ with $A \ts{\anc}{id} TM \ts{id}{T\pi} TA$. For any pair of Weil algebras $U,V$, define the following morphisms (note that $\anc^V_\N$ is the original $\anc^V$ map)
%     \begin{enumerate}[(i)]
%         \item \input{TikzDrawings/Ch4/LocMaps/proj.tikz}
%         \item \input{TikzDrawings/Ch4/LocMaps/zero.tikz}
%         \item \input{TikzDrawings/Ch4/LocMaps/plus.tikz}
%         \item \input{TikzDrawings/Ch4/LocMaps/lift.tikz}
%         \item \input{TikzDrawings/Ch4/LocMaps/flip.tikz}
%         % \item \input{TikzDrawings/Ch4/LocMaps/anc.tikz}
%     \end{enumerate}
% \end{proposition}
\begin{definition}\label{def:anc-nat}
    Let $(\pi:A \to M, \xi, \lambda, \anc)$ be an anchored bundle with chosen prolongations. 
    Recall that by Definition \ref{def:monoidal-category}, the right leg of $A.U$ is written $\anc$, so it induces a span map:\[\input{TikzDrawings/Ch4/anc-uv.tikz}\]
    This map has a flat presentation as
    \[
        \input{TikzDrawings/Ch4/LocMaps/anc.tikz}        
    \]
    We write the map
    \[
        \anc^U.V := \anc^U \boxtimes (\widehat A.V)
    \]
    which corresponds to the following span morphism:
    % https://q.uiver.app/?q=WzAsOSxbMCwyLCJNIl0sWzEsMSwiQS5VIl0sWzIsMiwiVF5VTSJdLFszLDEsIlReVS5BLlYiXSxbNCwyLCJUXntVVn1NIl0sWzEsMywiVF5VTSJdLFszLDMsIlReVS5BLlYiXSxbMiwwLCJBLlVWIl0sWzIsNCwiVF5VLkEuViJdLFsxLDUsIlxcYW5jXlUiXSxbMyw2LCIiLDAseyJsZXZlbCI6Miwic3R5bGUiOnsiaGVhZCI6eyJuYW1lIjoibm9uZSJ9fX1dLFszLDQsIlReVS5cXGFuY15WIiwxXSxbMywyLCJUXlUuXFxwaV5WIiwxXSxbMSwyLCJcXGFuY15VIiwxXSxbMSwwLCJcXHBpXlUiLDFdLFs4LDVdLFs4LDYsIiIsMSx7ImxldmVsIjoyLCJzdHlsZSI6eyJoZWFkIjp7Im5hbWUiOiJub25lIn19fV0sWzcsMV0sWzcsM10sWzcsMiwiIiwxLHsic3R5bGUiOnsibmFtZSI6ImNvcm5lciJ9fV0sWzUsMCwicF5VLk0iLDFdLFs1LDIsIiIsMSx7ImxldmVsIjoyLCJzdHlsZSI6eyJoZWFkIjp7Im5hbWUiOiJub25lIn19fV0sWzYsMiwiVF5VLlxccGleViIsMV0sWzYsNF1d
\[\begin{tikzcd}
	&& {A.UV} \\
	& {A.U} && {T^U.A.V} \\
	M && {T^UM} && {T^{UV}M} \\
	& {T^UM} && {T^U.A.V} \\
	&& {T^U.A.V}
	\arrow["{\anc^U}", from=2-2, to=4-2]
	\arrow[Rightarrow, no head, from=2-4, to=4-4]
	\arrow["{T^U.\anc^V}"{description}, from=2-4, to=3-5]
	\arrow["{T^U.\pi^V}"{description}, from=2-4, to=3-3]
	\arrow["{\anc^U}"{description}, from=2-2, to=3-3]
	\arrow["{\pi^U}"{description}, from=2-2, to=3-1]
	\arrow[from=5-3, to=4-2]
	\arrow[Rightarrow, no head, from=5-3, to=4-4]
	\arrow[from=1-3, to=2-2]
	\arrow[from=1-3, to=2-4]
	\arrow["\lrcorner"{anchor=center, pos=0.125, rotate=-45}, draw=none, from=1-3, to=3-3]
	\arrow["{p^U.M}"{description}, from=4-2, to=3-1]
	\arrow[Rightarrow, no head, from=4-2, to=3-3]
	\arrow["{T^U.\pi^V}"{description}, from=4-4, to=3-3]
	\arrow[from=4-4, to=3-5]
\end{tikzcd}\]
\end{definition}

% \begin{proposition}
%     Let $(\pi:A \to M, \xi, \lambda, \sigma)$ be an involution algebroid with a chosen prolongations.
%     The following coherences hold:
%     \begin{enumerate}
%         \item $\anc^{U}_V \o (\theta \boxtimes \phi) = 
%     \end{enumerate}
% \end{proposition}

% The following proposition establishes some coherences and universality conditions for the tangent bundle that are not directly axiomatized Definition \ref{def:involution-algd} but are necessary to prove Theorem  \ref{thm:weil-nerve}.
% \begin{proposition}\label{prop:higher-tangent-bundle-construction}
%     Let $(\pi:A \to M, \xi, \lambda, \anc, \sigma)$ be an involution algebroid in a tangent category $\C$.
%     It follows that:
%     \begin{enumerate}[(i)]
%         \item Coassociativity of $\hat{\lambda}$: $(\hat{\lambda}\x\ell)\o\hat{\lambda} = (id \x T.\hat{\lambda}) \o \hat{\lambda}$
%         \item Coherence between $\sigma$ and $\hat{\lambda}$, $(\sigma \x c) \o (1\x T\sigma) \o (\hat{\lambda},\ell) = (1 \x T\hat{\lambda}) \o \sigma$
%         \item Both of the diagrams are $T$-pullbacks:
%         \input{TikzDrawings/Ch3/Sec5/inv-algd-universality.tikz}
%     \end{enumerate}
% \end{proposition}
% \begin{proof}
%     ~\begin{enumerate}[(i)]
%         \item Compute:
%             \begin{align*}
%                 (\hat{\lambda}\x \ell)\o \hat{\lambda} 
%                 &= (\xi\o\pi\o\xi\o\pi, \lambda\o\xi\o\pi, \ell\o\lambda) \\
%                 &= (\xi\o\pi, T.(\xi\o\pi)\o\lambda, T.\lambda \o \lambda) \\
%                 &= (\pi_0, T.(\xi\o\pi)\o\pi_1, T.\lambda \o \pi_1)\o (\xi\o\pi, \lambda) \\
%                 &= (id \x T(\hat{\lambda}))\o\hat{\lambda}
%             \end{align*}
%         \item Compute:
%             \begin{align*}
%                 (1\x T.\sigma) \o (\sigma \x c) \o (1 \x T.\hat{\lambda}) 
%                 &= (1 \x T.\sigma) \o (\sigma \o (\pi_0, T.(\xi\o\pi)\o\pi_1), c \o T.\lambda) \\
%                 &= (1 \x T.\sigma) \o (\xi\o\pi\o\pi_0, 0\o\pi_0, c \o T.\lambda\o\pi_1) \\
%                 &= (\xi\o\pi\o\pi_0, T.\sigma\o(0\o\pi_0,c\o T.\lambda \o \pi_1) \\
%                 &= (\xi\o\pi\o\pi_0, \lambda\pi_0, \ell\o\pi_1) \o \sigma = (\hat{\lambda} \x \ell) \o \sigma
%             \end{align*}
%         \item  Use the pullback lemma to observe that the following diagram is universal for any anchored bundle
%             % https://q.uiver.app/?q=WzAsNixbMCwwLCJBXzIiXSxbMCwxLCJNIl0sWzEsMCwiXFxwcm9sKEEpIl0sWzEsMSwiQSJdLFsyLDAsIlRBIl0sWzIsMSwiVE0iXSxbMSwzLCJcXHhpIl0sWzMsNSwiXFxhbmMiXSxbMSw1LCIwIiwyLHsiY3VydmUiOjJ9XSxbNCw1LCJUXFxwaSJdLFsyLDMsIlxccGlfMCJdLFswLDEsIlxccGlcXG9cXHBpXzEiLDJdLFswLDIsIlxcaGF0e1xcbXV9IiwyXSxbMiw0LCJcXHBpXzEiLDJdLFswLDQsIlxcbXUiLDAseyJjdXJ2ZSI6LTJ9XV0=
%             \input{TikzDrawings/Ch3/Sec5/inv-algd-mu-universal-proof.tikz}
%             Define the map $\hat{\mu}(a,b) := (\xi\o \pi a, \lambda \o a) +_{\pi_0} (\xi\o \pi\o  b, 0\o b)$ so the top triangle of the diagram commutes. The right square and outer perimeter are pullbacks by definition, and the bottom triangle also commutes by definition. The pullback lemma ensures that the left square is a pullback - this means that for every anchored bundle, the general lift is universal for $\prol(A)$. Now post-compose with the involution:
%             \input{TikzDrawings/Ch3/Sec5/inv-algd-nu-universal.tikz}
%             It suffices to check that the top triangle commutes, so $\sigma \o \hat{\mu} = \nu$:
%             \[
%                 \sigma \o \hat{\mu}\o (a,b) = \sigma \o   ((\xi\o \pi,0)\o a +_{\pi_0} (\xi\o \pi,\lambda)\o b) = 
%                  (id, T.\xi \o \anc \o a) +_{p\pi_1} (\xi\o \pi,\lambda)\o b
%             \]
%             The lift $(\xi\pi,\lambda)$ involution algebroid is universal for $\prol(A)$.
%     \end{enumerate}
% \end{proof}


% \begin{proposition}%
%     \label{prop:wone-anc-strict-moncat}
%     The category $\wone^\anc$ is a strict monoidal category, where the tensor product on objects $\prol(U,A),\prol(V,A)$ is the chosen span prolongation $\prol(UV,A)$ and the unit is $\prol(\N, A)$. On morphisms, the tensor product:
%     \[
%         \infer{
%             (f,\theta)\ox (g,\phi) = (f \ts{\anc^U}{\theta.\pi^V} T^U.g, f.g.M )
%         }{
%             (f,\theta):U \to V & (g,\phi): X \to Y
%         }  
%     \]
%     induced by the span composition:
%     \[\input{TikzDrawings/Ch4/Sec3/tensor.tikz}\]
%     And the unit is the identity span.
% \end{proposition}
% \begin{observation}%
%     \label{obs:apex-functor}
%     % The category $\wone^\anc$ is a subcategory of the category of spans in $\C$. Suppose a diagram $D: \d \to \wone^\anc$
%     There is a functor $U^\anc: \wone^\anc \to \C$ so that projects out the apex map, sending:
%     \[
%         (f,\theta): \prol(A,U) \to \prol(A,V) \in \wone^\anc  
%     \] 
%     to 
%     \[
%         f: \prol(A,U) \to \prol(A,V) \in \C
%     \]
%     Similarly, there is a functor from $U^\C: \wone^\anc \to \wone$ that sends:
%     \[
%         (f,\theta): \prol(A,U) \to \prol(A,V) \in \wone^\anc  
%     \]
%     to 
%     \[
%         \theta:U \to V  
%     \]
% \end{observation}
% \begin{proposition}%
%     \label{prop:limits-in-wanc}
%     Let $(\pi:A \to M, \xi, \lambda, \anc)$ be anchored bundle with chosen prolongations in a tangent category $\C$. Consider a commuting square $D$ 
%     \[\input{TikzDrawings/Ch4/pb-wanc.tikz}\]
%     so that $U^\C(D) \in \wone$ is a transverse limit, and $U^\anc(D)\in \C$ is a $T$ limit.
%     Then this square is a pullback in $\wone^\anc$ that is sent to a $T$-limit by $U^\anc$.
% \end{proposition}
% \begin{proof}
%     Start with a pair:
%     \[\input{TikzDrawings/Ch4/pb-wanc-2.tikz}\]
%     This square is a pullback in the category of spans by Observation \ref{obs:lim-of-spans}, so it suffices to prove that the induced map is in $\wone^\anc$. Consider the diagram:
%     \[\input{TikzDrawings/Ch4/pb-wanc-3.tikz}\]
%     The map $(\alpha,\beta):Q \to U \in \wone$ by transversality, and the diagram commutes because it is the limit in $\mathsf{Span}(\C)$. 
% \end{proof}
% \begin{corollary}
%     The tensor product in $\wone$ is continuous in each variable for limits of the form \Cref*{prop:limits-in-wanc}.
% \end{corollary}
% \begin{corollary}%
%     \label{cor:tang-limits-in-wone}
%     For any anchored bundle with chosen prolongations $(\pi:A \to M, \xi, \lambda, \anc)$ in a tangent category $\C$, the following diagrams are limits in $\wone^\anc$ and are sent to $T$-limits in $\anc$.
%     \begin{equation}
%         \label{eq:univ-lift}
%         \input{TikzDrawings/Ch4/univ-pb.tikz}
%     \end{equation}
%     \begin{equation}
%         \label{eq:span-limit}
%         \input{TikzDrawings/Ch4/span-pullback-diag.tikz}
%     \end{equation}
% \end{corollary}
% \begin{proof}
%     The base pullback in \Cref{eq:univ-lift} was proved to be a $T$-limit in part (iii) of Proposition  \ref{prop:higher-tangent-bundle-construction}, whereas \Cref{eq:span-limit} is part of the axioms of a differential bundle. Note that \Cref{eq:univ-lift} is mapped to the universality of the vertical lift in $\wone$, and \Cref{eq:span-limit} is sent to the pullback powers defining $W_n$.
% \end{proof}


% \begin{remark}
%     The simplicial object of an internal category requires a choice of $n$-fold span compositions $M \xleftarrow[]{s} C \xrightarrow[]{t} M$. The construction in Definition \ref{def:weil-anc-cat}, then, takes the full subcategory of $\C/(M \x M)$ and notes that the category of span compositions  $M \xleftarrow[]{s} C \xrightarrow[]{t} M$ is still a monoidal category.
% \end{remark}

% The construction of the structure maps builds on the prolongation construction in Definition \ref{def:prolongation-of-anchored-bundle}. First, following table associates maps in $\wone$ to the structure maps of an involution algebroid:
% \begin{center}
%     \begin{tabular}{|l|l|l|l|l|l|l|}
%     \hline
%     Tangent bundle & $T^2M,TM,M$ & $p$   & $0$   & $+$     & $\ell$    & $c$      \\ \hline
%     Involution algebroid & $\prol(A), A, M$     & $\pi$ & $\xi$ & $+_\pi$ & $\lambda$ & $\sigma$ \\ \hline
%     \end{tabular}
% \end{center}

% Rather that constructing an internal monoid in the span category, the Weil nerve induces a tangent structure on the monoidal category.
% \begin{proposition}%
%     \label{prop:wone-is-tang-cat}
%     Given an involution algebroid $(\pi:A \to M, \xi, \lambda, \anc, \sigma)$, the monoidal category $\wone^\anc$ has a tangent structure given by:
%     \begin{itemize}
%         \item $T := \prol(W,A) \ox (-), p := (\pi,p), 0:= (\xi,0), +:= (+_q,+)$.
%         \item $\ell := (\xi\o\pi,\lambda), c := (\sigma, c)$
%     \end{itemize}
% \end{proposition}
% \begin{proof}
%     Functoriality of $T$ and the naturality of $(p,0,+,\ell,c)$ are by construction. Now check each of the involution algebroid axioms:
%     \begin{enumerate}[{[TC.1]}]
%         \item Additive bundle axioms:
%         \begin{enumerate}[(i)]
%             \item By Corollary  \ref{cor:tang-limits-in-wone}, pullback powers of $p$ exist and are preserved by $T$.
%             \item The additive bundle structure comes from the addition induced 
%         \end{enumerate} 
%         \item Symmetry axioms:
%         \begin{enumerate}[(i)]
%             \item $c \o c$ follows from the involution axiom.
%             \item For Yang-Baxter, note that: 
%             \[c.T \o T.c \o c.T\] 
%             is equivalent to the Yang-Baxter equation on an involution algebroid:
%             \[
%               (\sigma \x c)  \o (id \x T.\sigma) \o (\sigma \x c) =
%               (id \x T.\sigma) \o (\sigma \x c)\o (id \x T.\sigma)
%             \]
%             and 
%             \[\sigma \x c = (c,\anc) \ox \prol(W,A) = c.T,
%             \hspace{0.15cm}
%             id \x T.\sigma = \prol(W,A) \ox (\sigma,c) = T.c\]
%             \item For the naturality conditions:
%             \begin{enumerate}[(a)]
%                 \item The interchange of $+,0,p$ all follow from the fact that $\sigma:(\prol(A),\lambda \x \ell) \to (\prol(A),id \x c \o T.\lambda)$ is linear, so it is an additive bundle morphism.
%                 \item The axiom 
%                 \[\ell.T \o c = T.c \o c.T \o T.\ell\] 
%                 is equivalent to the equation:
%                 \[
%                     (\sigma \x c) \o (1\x T.\sigma) \o (\hat{\lambda} \x \ell) = (1 \x T\hat{\lambda}) \o \sigma
%                 \]
%                 proved in (ii) of Proposition  \ref{prop:higher-tangent-bundle-construction} (which requires that $\sigma$ be bilinear).
%             \end{enumerate}
%         \end{enumerate}
%         \item The lift axioms:
%             \begin{enumerate}[(i)]
%                 \item The additive bundle equations are a consequence of $\lambda$ being a lift, and $+$ being the addition induced by the non-singularity of $\lambda$.
%                 \item The coassociativity axiom \[\ell.T \o \ell = T.\ell \o \ell\] 
%                 is equivalent to 
%                 \[(\hat{\lambda}\x\ell)\o\hat{\lambda} = (id \x T.\hat{\lambda}) \o \hat{\lambda}\] proved in (i) of Proposition  \ref{prop:higher-tangent-bundle-construction}.
%                 \item The symmetry of comultiplication, $c \o \ell = \ell$ is given by the unique equation for an involution algebroid, so that $\sigma \o (\xi\o\pi,\lambda) = (\xi\o\pi,\lambda)$.
%                 \item The universality of the lift follows from part (iii) of Proposition  \ref{prop:higher-tangent-bundle-construction} and the coherence on limits in Corollary  \ref{cor:tang-limits-in-wone}.
%             \end{enumerate}
%     \end{enumerate}
%     Observe that just as in the case for the equivalence between involution algebroids and Lie algebroids proved in Theorem  \ref{thm:iso-of-cats-Lie} and \Cref{sec:connections_on_an_involution_algebroid}, this used each equation and universality condition on an involution algebroid (the anchor is used in the construction of $\wone^\anc$).
%     % For [TC.1], that the pullback powers of $p$ exist and are preserved by $T$ follows by Corollary  \ref{cor:tang-limits-in-wone}, the additive bundle coherences follow from the involution algebroid axioms. The coherences in [TC.2] follow by the involution algebroid axioms and parts (i) and (ii) Proposition  \ref{prop:higher-tangent-bundle-construction}. The universality of the vertical lift axiom [TC.3] follows by Corollary  \ref{cor:tang-limits-in-wone}, so all of the coherences of a tangent structure are satisfied.
%     % To see that pullback powers of $p$ exist and are preserved by $T$, note that by the commutativity of pullbacks each $\prol(UW_nV,A)$ is the $n$-fold pullback of: 
%     % \[\input{TikzDrawings/Ch4/pullback-diag.tikz}\]
%     % This is also a pullback in the category $\wone^\anc$ - given a family of maps:
%     % \[
%     %     (f_i, \theta_i): \prol(X,A) \to   \prol(UWV,A), \hspace{.5cm}
%     %     (id_U \ox (\pi,p) \ox id_V) \o (f_i, \theta_i) = (f,\theta)
%     % \]
%     % then the following diagram commutes: 
%     % \begin{equation}
%     %     \label{eq:span-limit}
%     %     \input{TikzDrawings/Ch4/span-pullback-diag.tikz}
%     % \end{equation}
%     % The commutativity of limits in $\C$ ensures that this pullback is preserved by $\ox$, and consequently by $T^\anc$. Thus, this gives the additive bundle coherence [TC.1].
%     % The rest of the coherences in [TC.2] follow by the involution algebroid axioms and parts (i) and (ii) Proposition  \ref{prop:higher-tangent-bundle-construction}.
%     % For the universality of the vertical lift, the diagram:
%     % \[\input{TikzDrawings/Ch1/univ-lift.tikz}\]
%     % is a pullback by part (iii) of Proposition  \ref{prop:higher-tangent-bundle-construction}. 
%     % Then observe that by the commutativity of $T$-limits, the following diagram is a pullback in $\C$:
%     % \begin{equation}
%     %     \label{eq:univ-lift}
%     %     \input{TikzDrawings/Ch4/univ-pb.tikz}
%     % \end{equation}
%     % By the same argument as in \Cref{eq:span-limit}, the uniquely induced map
%     % \[\input{TikzDrawings/Ch4/univ-pb.tikz}\]
%     % will be a morphism in $\wone^\anc$ - exhibiting \Cref{eq:univ-lift} as a pullback in $\wone^\anc$ - this induces 
% \end{proof}

\begin{theorem}[The Weil Nerve]
    \label{thm:weil-nerve}
    There is a fully faithful functor
    \[
        \mathsf{N}_{\weil}: \mathsf{Inv}(\C) \to [\wone, \C]  
    \]
    that sends an involution algebroid to the transverse-limit-preserving tangent functor:
    \[
        (\widehat{A},\alpha): \wone \to \C 
    \]
\end{theorem}
\begin{proof}
    % Starting with objects, note that $V \mapsto \prol(V)$. %
    % \begin{center}
    %     \begin{tabular}{|l|l|}
    %         \hline
    %         $\wone$           & $\widehat{A}:\wone \to \C$                    \\ \hline
    %         $W_n$             & $\hat{A}.W_n$                                \\ \hline
    %         $U \ox V$         & $\hat{A}.U \boxtimes\hat{A}.V$             \\ \hline
    %         $+:T_2 \to T$     & $\hat{A}.+: \hat{A}.W_2 \to  \hat{A}.W$ \\ \hline
    %         $p:T \to \N$      & $\hat{A}.p: \hat{A}.W \to \hat{A}.\N$  \\ \hline
    %         $0: I \to T$      & $\hat{A}.0: \hat{A}.\N \to  \hat{A}.W)$  \\ \hline
    %         $\ell:T \to T^2$  & $\hat{A}.\ell:  \hat{A}.W \to  \hat{A}.WW$ \\ \hline
    %         $c:T^2 \to T^2$   & $\hat{A}.WW \to  \hat{A}.WW$                  \\ \hline
    %         $\phi \ox \theta$ & $ \hat{A}.\phi \boxtimes  \hat{A}.\theta$        \\ \hline
    %     \end{tabular}    
    % \end{center}
    For the first step of this proof, we show that an involution algebroid structure on an anchored bundle $(\pi:A \to M, \xi, \lambda, \anc)$ determines a tangent category structure on the monoidal category $\mathsf{Span}(\pi:A \to M, \xi, \lambda, \anc)$.
    
    We check that the endofunctor $\hat{A} \boxtimes (-)$ determines a tangent structure, with the structure maps given by Definition \ref{def:generators-for-wone-in-anc}:
    
    % The $\boxtimes$ operation preserves equations on the left and right argument and is continuous. Therefore, it suffices to check that equations hold on the basic generators of $\wone$, and that the three generating transverse limits (identity, $W_2 = W \ts{p}{p} W$, and universality of the lift) are sent to $T$-limits.
    \begin{enumerate}[{[TC.1]}]
        \item Additive bundle axioms:
        \begin{enumerate}[(i)]
            \item Use Lemma \ref{lemma:pullback-part-of-theorem} to see that \[ \hat{A}.W_2 = \hat{A}.W \ts{p}{p} \hat{A}.W  A \ts{\pi}{\pi} A;\] this is preserved by $\hat A.V \boxtimes (-)$.
            \item The triple $(\hat{A}.+,\hat{A}.p,\hat{A}.0) = (+_q, \pi, \xi)$ is an additive bundle induced by Proposition  \ref{prop:induce-abun}, and $\boxtimes$ preserves pullbacks (and therefore additive bundles), so the additive bundle axioms hold.
        \end{enumerate} 
        \item Symmetry axioms:
        \begin{enumerate}[(i)]
            \item $\hat{A}.c \o \hat{A}.c = id$ follows from the involution axiom $\sigma \o \sigma = id$.
            \item For Yang--Baxter, note that 
            \[ 
                (\hat{A} \boxtimes c) \o (c \boxtimes \hat{A}) \o (\hat{A} \boxtimes c)  = 
                (c \boxtimes \hat{A})  \o (\hat{A} \boxtimes c)  \o (c \boxtimes \hat{A}) 
            \] 
            follows from the Yang--Baxter equation on an involution algebroid
            \[
              (\sigma \x c)  \o (id \x T.\sigma) \o (\sigma \x c) =
              (id \x T.\sigma) \o (\sigma \x c)\o (id \x T.\sigma),
            \]
            since 
            \[\sigma \x c.A = (\hat{A}.c) \boxtimes (\hat{A}.W )
                \text{ and }
            id \x T.\sigma = (\hat{A}.W ) \boxtimes c. \]
            % Note that $\boxtims$ preserves equations, so this holds for $U \boxtimes (W \boxtimes c) \boxtimes V$ and $U \boxtimes (c \boxtimes W) \boxtimes V$.
            \item For the naturality conditions:
            \begin{enumerate}[(a)]
                \item The interchanges of $+,0,p$ all follow from the fact that 
                \[\sigma:(A.WW,\lambda \x \ell) \to (A.WW,id \x c \o T.\lambda)\] 
                is linear, and so is an additive bundle morphism.
                \item The axiom 
                \[\ell.T \o c = T.c \o c.T \o T.\ell\] 
                is equivalent to the equation
                \[
                    (\sigma \x c) \o (1\x T.\sigma) \o (\hat{\lambda} \x \ell) = (1 \x T\hat{\lambda}) \o \sigma
                \]
                which is equivalent to the double linearity axiom on $\sigma$ by Proposition  \ref{prop:nat-of-sigma-ell}.
            \end{enumerate}
        \end{enumerate}
        \item The lift axioms:
            \begin{enumerate}[(i)]
                \item The additive bundle equations are a consequence of $\lambda$ being a lift and $+$ being the addition induced by the non-singularity of $\lambda$.
                \item The coassociativity axiom 
                \[\ell.T \o \ell = T.\ell \o \ell\] 
                is equivalent to 
                \[(\hat{\lambda}\x\ell)\o\hat{\lambda} = (id \x T.\hat{\lambda}) \o \hat{\lambda}\] 
                proved in (i) of Proposition  \ref{prop:lift-axioms-anchor}.
                \item The symmetry of comultiplication, $c \o \ell = \ell$, is given by the unique equation for an involution algebroid, so that $\sigma \o (\xi\o\pi,\lambda) = (\xi\o\pi,\lambda)$.
                \item The universality of the lift follows from part (ii) of Proposition  \ref{prop:lift-axioms-anchor}; Lemma \ref{lemma:pullback-part-of-theorem} ensures that for any $V \in \wone$, $\widehat{A}.V \boxtimes \mu$ and $\mu \boxtimes \widehat{A}.V$ are universal.
            \end{enumerate}
    \end{enumerate}
    This lemma puts a tangent structure on $\mathsf{Span}(\pi,\xi,\lambda,\anc)$. Now consider the functor sending spans to the apex map, 
    \[
        U^\anc: \mathsf{Span}(\pi,\xi,\lambda,\anc) \to \C.
    \]
    The family of maps
    \[
        \{\anc^U.V: \anc^U \boxtimes \widehat A.V | U,V \in \wone\}
    \]
    gives a family of natural transformations
    \[
        \anc^{U}:A.T^U \Rightarrow T^U.A,
    \]
    so that the following pair constitute a tangent functor
    \[
      (U^\anc,\anc): \mathsf{Span}(\pi,\xi,\lambda,\anc) \to \C.
    \]
    Because the universality conditions on $\mathsf{Span}(\pi,\xi,\lambda,\anc)$ followed by reflecting limits in $\C$ using Lemma \ref{lemma:pullback-part-of-theorem}, it follows that $(U^\anc,\anc)$ will preserve the tangent-natural limits in $\mathsf{Span}(\pi,\xi,\lambda,\anc)$ corresponding to transverse limits in $\wone$.
    
    By Leung's Theorem  \ref{thm:leung} (by way of Corollary \ref{cor:using-leung-thm}), the tangent structure on $\mathsf{Span}(\pi,\xi,\lambda,\anc)$ induces a strict, monoidal, transverse-limit-preserving functor
    \[
        \bar{A}: \wone \to \mathsf{Span}(\pi,\xi,\lambda,\anc)    
    \]
    that sends the tensor product $\ox$ to the span composition $\boxtimes$. By composing the strict tangent functor $(\bar{A},id)$ and $(U^\anc,\anc)$, we have a lax, transverse-limit-preserving, tangent functor:
    \[
        (A,\anc):\wone \to \C; V \mapsto A.V
    \]
    

    % The natural transformation is defined \[\alpha: \hat{A}.UV \Rightarrow T^U.\hat{A}.V = \anc^U \boxtimes \hat{A}.V \] using the definition of $\anc$ given in Definition \ref{def:anc-nat}. First look at $\anc^U \boxtimes \prol(V,A) \o (f \boxtimes g)$:
    % \[\input{TikzDrawings/Ch4/Sec4/anc-is-nat.tikz}\]
    % Then look at $\theta.g \o (\anc^U \boxtimes \prol(V,A))$:
    % \[\input{TikzDrawings/Ch4/Sec4/anc-is-nat-2.tikz}\]
    % The two maps induced from $\hat{A}.UV \to T^{U'}.\hat{A}.V'$ are equal, so $\alpha$ is natural.
    % The composite span is given by $\theta.g $
    % The assignment on objects was induced by Corollary  \ref{cor:actual-nerve}, note that a strict tangent functor $\wone \to \wone^\anc$ will preserve all transverse limits, and that $U^\anc:\wone^\anc \to \C$ will preserve tangent limits (as tangent limits in $\wone^\anc$ are computed ``pointwise'' in $\C$), all that remains is the bijection on morphisms.

    Now, check the bijection on morphisms. Starting with an involution algebroid morphism $(f,m):A \to B$, note that this gives a span morphism $\hat f$:
    % https://q.uiver.app/?q=WzAsNixbMCwwLCJNIl0sWzEsMCwiQSJdLFsyLDAsIlRNIl0sWzAsMSwiTiJdLFsxLDEsIkIiXSxbMiwxLCJUTiJdLFsyLDUsIlQubSJdLFswLDMsIm0iXSxbMSwwXSxbMSwyXSxbNCw1XSxbNCwzXSxbMSw0LCJmIiwxXV0=
\[\begin{tikzcd}[ampersand replacement=\&]
	M \& A \& TM \\
	N \& B \& TN
	\arrow["{T.m}", from=1-3, to=2-3]
	\arrow["m", from=1-1, to=2-1]
	\arrow[from=1-2, to=1-1]
	\arrow[from=1-2, to=1-3]
	\arrow[from=2-2, to=2-3]
	\arrow[from=2-2, to=2-1]
	\arrow["f", from=1-2, to=2-2]
\end{tikzcd}\]
    This gives a natural definition of $\hat{f}.V$ using the horizontal composition of span morphism, so that 
    \begin{equation}\label{eq:boxtimes-def-of-hat-f}
        \hat{f}.(UV) = \hat{f}.U \boxtimes \hat{f}.V \text{ and } \hat{f}.\N = m,
    \end{equation}
    giving a family of maps $\{\hat{f}_{U}: U \in \mathsf{objects}(\wone)\}$. Because $f$ will commute with the structure maps $\{ \pi,\xi,+,(\xi\o\pi,\lambda),\sigma\}$, it follow immediately that $\hat{f}$ is a natural transformation, because the following calculation holds for each $\theta:X \to Y \in \{ p,0,+,\ell,c\}$:
    \begin{align*}
        & \quad \hat{f}.UYV \o (\hat{A}.U \boxtimes \theta \boxtimes \hat{A}.V) \\
        &= (\hat{f}.U \boxtimes \hat{f}.Y \boxtimes \hat{f}.V) \o  (\hat{A}.U \boxtimes \theta \boxtimes \hat{A}.V) \\
        &= \hat{f}.U \boxtimes( \hat{f}.Y \o \theta)\boxtimes \hat{f}.V \\
        &= \hat{f}.U \boxtimes( \theta \o \hat{f}.X) \boxtimes \hat{f}.V \\
        &= (\hat{A}.U \boxtimes \theta \boxtimes \hat{A}.V) \o \hat{f}.UXV
        % =& (\prol(U,B) \ts{}{} T^U.\theta \ts{}{} T^{UY}.\prol(U,B)) \o (\prol(U,f) \ts{}{} T^U.\prol(X,f) \ts{}{} T^{UY}.\prol(V,f)) \\
        % =& (\prol(U,B) \boxtimes \theta \boxtimes \prol(U,B)) \o\prol(UXV,f) 
    \end{align*}
    Tangent naturality will follow by the preservation of the anchor map by $f$. 
    The equality, for any Weil algebra $U$, of the diagrams
% https://q.uiver.app/?q=WzAsMTgsWzAsMCwiTSJdLFsxLDAsIkEuVSJdLFsyLDAsIlReVU0iXSxbMCwxLCJOIl0sWzEsMSwiQi5VIl0sWzIsMSwiVF5VTiJdLFswLDIsIk4iXSxbMSwyLCJUXlVOIl0sWzIsMiwiVF5VTiJdLFs0LDEsIlReVU0iXSxbNCwwLCJBLlUiXSxbMywwLCJNIl0sWzUsMCwiVF5VTSJdLFszLDEsIk0iXSxbNSwxLCJUXlVNIl0sWzMsMiwiTiJdLFs0LDIsIlReVk4iXSxbNSwyLCJUXlZOIl0sWzIsNSwiVF5VLm0iXSxbMCwzLCJtIl0sWzEsMCwiXFxwaV5VIiwyXSxbMSwyLCJcXGFuY15VIl0sWzQsNSwiXFxhbmNeVSJdLFs0LDNdLFsxLDQsIlxcaGF0e2Z9LlUiLDFdLFs0LDcsIlxcYW5jXlUiXSxbNyw2LCJwXlUiXSxbMyw2LCIiLDAseyJsZXZlbCI6Miwic3R5bGUiOnsiaGVhZCI6eyJuYW1lIjoibm9uZSJ9fX1dLFs1LDgsIiIsMCx7ImxldmVsIjoyLCJzdHlsZSI6eyJoZWFkIjp7Im5hbWUiOiJub25lIn19fV0sWzcsOCwiIiwwLHsibGV2ZWwiOjIsInN0eWxlIjp7ImhlYWQiOnsibmFtZSI6Im5vbmUifX19XSxbMTAsMTEsIlxccGleVSIsMl0sWzksMTMsInBeVSIsMl0sWzEwLDEyLCJcXGFuY15VIiwyXSxbOSwxNCwiIiwyLHsibGV2ZWwiOjIsInN0eWxlIjp7ImhlYWQiOnsibmFtZSI6Im5vbmUifX19XSxbMTYsMTcsIiIsMix7ImxldmVsIjoyLCJzdHlsZSI6eyJoZWFkIjp7Im5hbWUiOiJub25lIn19fV0sWzE2LDE1LCJwXlUiXSxbMTIsMTQsIiIsMSx7ImxldmVsIjoyLCJzdHlsZSI6eyJoZWFkIjp7Im5hbWUiOiJub25lIn19fV0sWzE0LDE3LCJUXlUubSJdLFs5LDE2LCJUXlUubSJdLFsxMywxNSwibSIsMl0sWzExLDEzLCIiLDIseyJsZXZlbCI6Miwic3R5bGUiOnsiaGVhZCI6eyJuYW1lIjoibm9uZSJ9fX1dLFsxMCw5LCJcXGFuY15VIl0sWzUsMTMsIj0iLDEseyJzdHlsZSI6eyJib2R5Ijp7Im5hbWUiOiJub25lIn0sImhlYWQiOnsibmFtZSI6Im5vbmUifX19XV0=
\[\begin{tikzcd}[ampersand replacement=\&]
	M \& {A.U} \& {T^UM} \& M \& {A.U} \& {T^UM} \\
	N \& {B.U} \& {T^UN} \& M \& {T^UM} \& {T^UM} \\
	N \& {T^UN} \& {T^UN} \& N \& {T^VN} \& {T^VN}
	\arrow["{T^U.m}", from=1-3, to=2-3]
	\arrow["m", from=1-1, to=2-1]
	\arrow["{\pi^U}"', from=1-2, to=1-1]
	\arrow["{\anc^U}", from=1-2, to=1-3]
	\arrow["{\anc^U}", from=2-2, to=2-3]
	\arrow[from=2-2, to=2-1]
	\arrow["{\hat{f}.U}"{description}, from=1-2, to=2-2]
	\arrow["{\anc^U}", from=2-2, to=3-2]
	\arrow["{p^U}", from=3-2, to=3-1]
	\arrow[Rightarrow, no head, from=2-1, to=3-1]
	\arrow[Rightarrow, no head, from=2-3, to=3-3]
	\arrow[Rightarrow, no head, from=3-2, to=3-3]
	\arrow["{\pi^U}"', from=1-5, to=1-4]
	\arrow["{p^U}"', from=2-5, to=2-4]
	\arrow["{\anc^U}"', from=1-5, to=1-6]
	\arrow[Rightarrow, no head, from=2-5, to=2-6]
	\arrow[Rightarrow, no head, from=3-5, to=3-6]
	\arrow["{p^U}", from=3-5, to=3-4]
	\arrow[Rightarrow, no head, from=1-6, to=2-6]
	\arrow["{T^U.m}", from=2-6, to=3-6]
	\arrow["{T^U.m}", from=2-5, to=3-5]
	\arrow["m"', from=2-4, to=3-4]
	\arrow[Rightarrow, no head, from=1-4, to=2-4]
	\arrow["{\anc^U}", from=1-5, to=2-5]
	\arrow["{=}"{description}, draw=none, from=2-3, to=2-4]
\end{tikzcd}\]
is precisely the tangent-naturality condition from Definitions \ref{def:tang-nat}, \ref{def:actegory-natural}.
    
    For the inverse of this mapping, consider a tangent natural transformation (Definition \ref{def:tang-nat})
    \[
        \gamma: ({A},\alpha) \to ({B}, \beta),  \hspace{0.15cm}
            \input{TikzDrawings/Ch1/tang-nat-AB.tikz}
    \]
    where $(A,\alpha)$ and $(B,\beta)$ are tangent functors $\wone \to \C$ built out of involution algebroids with chosen prolongations. For any $U,V$, the map $\gamma.UV$ decomposes as $\gamma.U \boxtimes \gamma.V$:
    \[
        \input{TikzDrawings/Ch4/cube-map-nat.tikz}   
    \]
    Applying this relationship inductively, it is clear that the base maps $\gamma.W$ and $\gamma.\N$ determine the entire morphism $\gamma.V$:
% https://q.uiver.app/?q=WzAsMTYsWzAsMiwiTSJdLFsyLDIsIlRfe25bMV19Lk0iXSxbMywyLCJcXGRvdHMiXSxbNCwyLCJUX3tuWzFdfVxcZG90cyBUX3tuW2tdfS5NIl0sWzQsMywiVF97blsxXX1cXGRvdHMgVF97bltrXX0uTiJdLFsyLDMsIlRfe25bMV19Lk0iXSxbMCwzLCJOIl0sWzEsMiwiQV97blsxXX0iXSxbMSwzLCJCX3tuWzFdfSJdLFszLDMsIlxcZG90cyJdLFsxLDAsIk0iXSxbMSwxLCJOIl0sWzMsMCwiVF5VTSJdLFszLDEsIlReVU4iXSxbMiwwLCJBLlUiXSxbMiwxLCJCLlUiXSxbMyw0LCJUX3tuWzFdfVxcZG90cyBUX3tuW2tdfS5cXGdhbW1hLlxcTiJdLFswLDYsIlxcZ2FtbWEuXFxOIl0sWzcsMSwiXFxhbmNfe25bMV19Il0sWzcsMCwiXFxwaVxcb1xccGlfaSIsMl0sWzgsNiwiXFxwaVxcb1xccGlfaSJdLFs4LDUsIlxcYW5jX3tuWzFdfSIsMl0sWzcsOCwiXFxnYW1tYS5XX3tuWzFdfSJdLFsxLDUsIlRfe25bMV19LlxcZ2FtbWEuXFxOIl0sWzIsMSwiVF97blsxXX0uKFxccGkgXFxvIFxccGlfaSkiLDJdLFsyLDNdLFs5LDUsIlRfe25bMV19LihcXHBpIFxcbyBcXHBpX2kpIl0sWzksNF0sWzIsOSwiXFxkb3RzIiwxLHsic3R5bGUiOnsiYm9keSI6eyJuYW1lIjoibm9uZSJ9LCJoZWFkIjp7Im5hbWUiOiJub25lIn19fV0sWzEyLDEzLCJUXlUuXFxnYW1tYS5cXE4iLDJdLFsxNCwxMCwiXFxwaV5VIiwyXSxbMTUsMTEsIlxccGleVSJdLFsxNSwxMywiXFxhbmNeVSIsMl0sWzE0LDEyLCJcXGFuY15VIl0sWzEwLDExLCJcXGdhbW1hLlxcTiIsMl0sWzE0LDE1LCJcXGdhbW1hLlUiLDJdLFszNSwyMywiPSIsMSx7InNob3J0ZW4iOnsic291cmNlIjoyMCwidGFyZ2V0IjoyMH0sInN0eWxlIjp7ImJvZHkiOnsibmFtZSI6Im5vbmUifSwiaGVhZCI6eyJuYW1lIjoibm9uZSJ9fX1dXQ==
\[\begin{tikzcd}
	& M & {A.U} & {T^UM} \\
	& N & {B.U} & {T^UN} \\
	M & {A_{n[1]}} & {T_{n[1]}.M} & \dots & {T_{n[1]}\dots T_{n[k]}.M} \\
	N & {B_{n[1]}} & {T_{n[1]}.M} & \dots & {T_{n[1]}\dots T_{n[k]}.N}
	\arrow["{T_{n[1]}\dots T_{n[k]}.\gamma.\N}", from=3-5, to=4-5]
	\arrow["{\gamma.\N}", from=3-1, to=4-1]
	\arrow["{\anc_{n[1]}}", from=3-2, to=3-3]
	\arrow["{\pi\o\pi_i}"', from=3-2, to=3-1]
	\arrow["{\pi\o\pi_i}", from=4-2, to=4-1]
	\arrow["{\anc_{n[1]}}"', from=4-2, to=4-3]
	\arrow["{\gamma.W_{n[1]}}", from=3-2, to=4-2]
	\arrow[""{name=0, anchor=center, inner sep=0}, "{T_{n[1]}.\gamma.\N}", from=3-3, to=4-3]
	\arrow["{T_{n[1]}.(\pi \o \pi_i)}"', from=3-4, to=3-3]
	\arrow[from=3-4, to=3-5]
	\arrow["{T_{n[1]}.(\pi \o \pi_i)}", from=4-4, to=4-3]
	\arrow[from=4-4, to=4-5]
	\arrow["\dots"{description}, draw=none, from=3-4, to=4-4]
	\arrow["{T^U.\gamma.\N}"', from=1-4, to=2-4]
	\arrow["{\pi^U}"', from=1-3, to=1-2]
	\arrow["{\pi^U}", from=2-3, to=2-2]
	\arrow["{\anc^U}"', from=2-3, to=2-4]
	\arrow["{\anc^U}", from=1-3, to=1-4]
	\arrow["{\gamma.\N}"', from=1-2, to=2-2]
	\arrow[""{name=1, anchor=center, inner sep=0}, "{\gamma.U}"', from=1-3, to=2-3]
	\arrow["{=}"{description}, Rightarrow, draw=none, from=1, to=0]
\end{tikzcd}\]
    Thus, every tangent-natural transformation is constructed out of a pair
    \[
        (\gamma.\N: M \to N, \gamma.W:A \to B)
    \]
    using the $\boxtimes$ construction from Equation \ref{eq:boxtimes-def-of-hat-f}.
    All that remains to show is that this pair is an involution algebroid morphism. 

    Tangent naturality gives the following two coherences:
    \[
        \anc^B \o \gamma.W = T.\gamma.\N \o \anc^A \text{ and } 
        \sigma^B \o \gamma.WW = \gamma.WW \o \sigma^A
    \]
    since $\anc^B = \beta.W, \anc^A = \alpha.W, \sigma^B = B.c$, and $\sigma^A = A.c$ by construction. The following diagram proves that $\gamma.W$ preserves the lifts, so that $(\gamma.W, \gamma.\N)$ is an involution algebroid morphism:
    \[\input{TikzDrawings/Ch4/Sec4/tang-nat-pres-lifts.tikz}\]
    % and anchors:
    % \input{TikzDrawings/Ch4/Sec4/tang-nat-pres-anchors.tikz}
    % And the $\gamma$ commutes with the involution by naturality. 
    Thus, a tangent natural transformation $(\widehat{A},\alpha) \to (\widehat{B}, \beta)$ is exactly a morphism of involution algebroids $A \to B$, proving the theorem.
    
    % The bijection follows from the fact that by tangent naturality, every
    % \[
    %     f.V: \widehat{A}.V \to \widehat{B}.V  
    % \]
    % decomposes as a pullback power of the base maps $f.\N, f.W$, which are exactly involution algebroid morphisms.
\end{proof} 
Now, the projection for a Lie algebroid is a submersion, as we may make a choice of prolongations for each $U \in \wone$. These prolongations lead to a new observation about Lie algebroids: they embed into a category of functors into smooth manifolds.
\begin{corollary}%
    \label{cor:SMan-embedding}
    Using the Weil nerve construction, the category of Lie algebroids embeds into the tangent-functor category:
    \[
        \mathsf{LieAlgd} \hookrightarrow [\wone, \mathsf{SMan}].  
    \]
\end{corollary}





\section{Identifying involution algebroids}%
\label{sec:identifying-involution-algebroids}

This section identifies those tangent functors
\[
    (A,\alpha): \wone \to \C  
\] 
that are involution algebroids as precisely those where $A$ preserves transverse limits and $\alpha$ is a \emph{$T$-cartesian} natural transformation (Definition \ref{def:cart-nat}). 
These conditions will force each $A.V$ to be the $V$-prolongation of the underlying anchored pre-differential bundle:
\[
    (A.p: A.T \to A,\;\ A.0: A \to A.T,\;\ A.T \xrightarrow[]{A.\ell} A.T.T \xrightarrow[]{\alpha.T} T.A.T,\;\ \alpha:A.T \to T.A )  
\]
(these conditions also ensure that this tuple is an anchored differential bundle). 

Initially, it is only clear that $\alpha$ is $T$-cartesian for the projection $p$. Indeed, recall that the prolongation $A.UV$ is defined to be the $T$-pullback of the cospan:
\[ \widehat A .U \xrightarrow[]{\alpha^U} T^U.A.\N \xleftarrow[]{T^U.A.p^V} T^U.\widehat{ A}.V\]
Then consider the following diagram:
\[\input{TikzDrawings/Ch4/anc-cart-p.tikz}\]
This means that every naturality square of $\alpha$ for $p$ is a $T$-pullback; natural transformations satisfying this property for every map in the domain category are called $T$-cartesian.
\begin{definition}%
    \label{def:cart-nat}
    A natural transformation $\gamma: F \Rightarrow G$ is \emph{cartesian} whenever each naturality square
    \[\input{TikzDrawings/Ch4/Sec4/equifibred.tikz}\]
    is a pullback. A natural transformation between functors into a tangent category is \emph{$T$-cartesian} whenever each component square is a $T$-pullback (we will generally suppress the $T$ when the context is clear). 
\end{definition}
% \begin{example}
%     For the Weil nerve of an involution algebroid, $\alpha$ is $T$-cartesian for
% \end{example}

Now, recall that the Weil complex determined by an involution algebroid has $A.U.V$ determined by the following $T$-pullback squares:
\[\input{TikzDrawings/Ch4/Sec5/weil-nerve-pb.tikz}\]
Then it is not difficult to show that the $T$-cartesian condition on a Weil complex forces it to be an involution algebroid.  We first need:
\begin{definition}
    A $T$-cartesian Weil complex in $\C$ is a tangent functor
    \[
        (A,\alpha): \wone \to \C  
    \]
    for which $A$ sends transverse limits to $T$-limits and $\alpha$ is a $T$-cartesian natural transformation.
\end{definition}
The first condition to check is that a $T$-cartesian Weil complex gives a natural anchored bundle $\hat{A}$ whose Weil prolongations coincide with the functor assignments on objects.
\begin{proposition}%
    \label{prop:anc-bun-cw-complex}
    Let $(A,\alpha)$ be a $T$-cartesian Weil complex. Then we have an anchored bundle
    \[
        (M  := A.\N, \hspace{0.15cm}
        \hat{A} := A.W , \hspace{0.15cm}
        \pi := A.\pi, \hspace*{0.15cm}
        \xi := A.\xi, \hspace*{0.15cm}
        \lambda := \alpha.T \o A.\ell).
    \]
    Furthermore, 
    \[
        \prol(\hat{A}) = A.WW, \hspace*{0.30cm}
        \prol^2(\hat{A}) = A.WWW. 
    \]
\end{proposition}
\begin{proof}
    Suppose we have a tangent functor $(F,\alpha): \C \to \D$ and a differential bundle $(\pi,\xi,\lambda)$ in $\C$. If $F$ preserves $T$-pullbacks of $\pi$, it preserves the additive bundle structure on $(\pi,\xi,+)$, so to show $(F.\pi, F.\xi, \alpha \o F.\lambda)$ is universal it suffices to show that the following diagram is a $T$-pullback in $\D$:
    \input{TikzDrawings/Ch4/Sec5/mu-anc-lambda.tikz}
    Expand this to
    \input{TikzDrawings/Ch4/Sec5/mu-anc-lambda-expanded.tikz}
    In this case, it restricts to the diagram
    \input{TikzDrawings/Ch4/Sec5/mu-anc-lambda-restricted-diagram.tikz}
    Each square is a $T$-pullback by hypothesis, so the universality of the lift follows by the $T$-pullback lemma. Because the complex is $T$-cartesian, the assignment $A.V$ gives a coherent choice of prolongations by the $T$-pullback
    \input{TikzDrawings/Ch4/Sec5/mu-anc-equifibered-prol.tikz}
\end{proof}
There is, of course, a natural candidate for the involution map.
\begin{corollary}
    Let $(\pi:A \to M, \xi, \lambda, \anc)$ be the anchored bundle induced by a $T$-cartesian Weil complex in a tangent category $\C$. Then we have an involution map
    \[
        \sigma: \prol(A) \xrightarrow{A.c} \prol(A).
    \]
\end{corollary}
The equations for an involution algebroid should follow immediately by functoriality; one need only ensure that the maps take the correct form.
\begin{lemma}\label{lem:cwm-map-structure}
    Let $A$ be a $T$-cartesian Weil complex in a tangent category $\C$, with $(\pi:A \to M, \xi, \lambda, \sigma)$ its underlying anchored bundle.
    Then we have:
    \begin{enumerate}[(i)]
        \item $A.c.T = \sigma \x c$,
        \item $A.T.c = 1 \x T.\sigma$,
        \item $A.\ell.T = \hat{\lambda} \x \ell.A$,
        \item $A.T.\ell = id \x T.\hat{\lambda}$.
    \end{enumerate}
\end{lemma}
\begin{proof}
    ~\begin{enumerate}[(i)]
        \item Consider the diagram
        \input{TikzDrawings/Ch4/Sec5/rewrite-act.tikz}
        Observe that this forces $A.c.T = A.c \x c.A.T = \sigma \x c$.
        \item Likewise, the diagram
        \input{TikzDrawings/Ch4/Sec5/rewrite-atc.tikz}
        forces $A.T.c = id \x T.A.c = id \x T.\sigma$.
        \item The diagram
        \input{TikzDrawings/Ch4/Sec5/rewrite-alt.tikz}
        forces $A.\ell.T = A.\ell \x \ell.A = \hat{\lambda} \x \ell$.
        \item As $\alpha$ is $T$-cartesian, the following diagram is a $T$-pullback:
        \input{TikzDrawings/Ch4/Sec5/rewrite-atl-1.tikz}
        Using previous results, this means that $A.\ell.T$ is the unique map making the following diagram commute:
        \input{TikzDrawings/Ch4/Sec5/rewrite-atl-2.tikz}
        which we can see is $id \x \hat\lambda$.
    \end{enumerate}
\end{proof}

Pulling together this lemma and the previous proposition, the following is now clear:
\begin{proposition}
    A $T$-cartesian Weil complex determines an involution algebroid.
\end{proposition}
However, we have not yet exhibited an isomorphism of categories between the image of the Weil nerve functor and $T$-cartesian Weil complexes. 
At first glance, the Weil nerve construction only gives a Weil complex that is $T$-cartesian for the tangent projection $p \in \wone$. Being $T$-cartesian for $p$ is, however, sufficient: a Weil complex that is $T$-cartesian for tangent projections will be $T$-cartesian for every map in $\wone$ (a similar result appears in the context of differentiable programming languages; see \cite{Cruttwell2019}).

\begin{proposition}\label{prop:mod-is-cart-if-p}
    A lax tranverse-limit-preserving tangent functor $(F,\alpha):\wone \to \C$ for which $F$ preserves pullback powers of each $T^U.p$ is $T$-cartesian if and only if each
    \begin{equation*}
        \input{TikzDrawings/Ch4/Sec4/equifibred-iff-p.tikz}
    \end{equation*}
    is a $T$-pullback.
\end{proposition}
\begin{proof}
    We only check the converse since the forward implication is trivial. We make use of the $T$-pullback lemma.
    \begin{enumerate}[(i)]
        \item $c$ is an isomorphism, so its naturality square is a $T$-pullback.
        \item For projections $T_2 \to T$,  the retract of a $T$-pullback diagram is a $T$-pullback, so the following diagram is universal:
        \input{TikzDrawings/Ch4/Sec5/pcartiff-proj.tikz}
        \item For $0$, observe that the following two diagrams are equal:
        \input{TikzDrawings/Ch4/Sec5/pcartiff-zero.tikz}
        The right diagram is a $T$-pullback, and the right square of the left diagram is a $T$-pullback by hypothesis.
        By the $T$-pullback lemma, the left square of the left diagram is a $T$-pullback.
        \item For $\ell$, observe that
        \input{TikzDrawings/Ch4/Sec5/pcart-iff-ell.tikz}
        The outer perimeter of the right diagram is a $T$-pullback (left square by hypothesis, right square by (ii)), as is the right square of the left diagram (by hypothesis). 
        By the $T$-pullback lemma, the left square of the left diagram is a $T$-pullback.
        \item For $+$, observe that
        \input{TikzDrawings/Ch4/Sec5/pcart-iff-add.tikz}
        The outer diagram on the right is a $T$-pullback by composition, and the right square on the left diagram is a $T$-pullback by hypothesis, so the result follows. 
    \end{enumerate}
    To check that the naturality square is a $T$-pullback for \emph{every} map in $\wone$, we once again use Leung's characterization of maps in $\wone$ from Proposition \ref{thm:leung}. Inductively, the set of maps generated by $\{p,0,+,\ell,c\}$ closed under $\ox$ and $\o$ follows as $T$-pullback squares are closed to composition. For maps induced by a tranverse limit in $\wone$, $F$ preserves transverse limits so this follows by the commutativity of limits.
\end{proof}

\begin{theorem}\label{thm:iso-of-cats-inv-emcs}
    For any tangent category $\C$, the replete image of the Weil nerve functor
    \[
        \mathsf{Inv}(\C) \hookrightarrow [\wone, \C]  
    \]
    is precisely the category of $T$-cartesian Weil complexes.
\end{theorem}
\begin{corollary}\label{cor:the-prolongation-description}
    That $\alpha:A.T \Rightarrow T.A$ is $T$-cartesian is equivalent to requiring that the tangent functor
    \[
        (A,\alpha): \wone \to \C  
    \]
    restricts to an anchored bundle
    \[(\pi: A.T \xrightarrow[]{A.p} A.\N,\;\ \xi: A \xrightarrow[]{A.0} A.T,\;\ \lambda: A.T \xrightarrow[]{A.\ell} A.TT \xrightarrow[]{\alpha.T}T.A.T,\;\ \anc:A.T \xrightarrow[]{\alpha} T.A)\]
    and each $A.T^V$ is the $V$-prolongation of this anchor bundle.
\end{corollary}
\begin{remark}
    The condition in Corollary  \ref{cor:the-prolongation-description} is analogous to the Segal conditions identifying those simplicial complexes
    \[\Delta \to \C \]
    that are internal categories. Note that every simplicial object has an underlying reflexive graph
    \[
        \mathsf{tr}_1(X) := (s,t:X([1]) \to X([0]), i:X([0]) \to X([1]))
    \]
    where $X([n])$ is isomorphic to the object of $n$-composable arrows for the underlying reflexive graph.
\end{remark}
\begin{remark}
    Notably, being $T$-cartesian for $p$ is enough to force that a natural transformation is $T$-cartesian for the other tangent-structural natural transformations. This has consequences when one uses partial maps to combine \emph{topological} notions with tangent categories.
    In this context, a partial map $N \to X$ with domain $M \hookrightarrow N$  is a span
    \[\input{TikzDrawings/Ch4/Sec4/span-remark.tikz}\]
    whose right leg is monic. The intuition is that the map $f$ is defined on the subobject $M$ of $N$, which introduces a new problem: what is the proper notion of a \emph{subobject} in a tangent category?
    Such a notion should give rise to a \emph{stable class of monics}: one that is closed under horizontal span composition.
    One answer is the notion of etale monics: a morphism is \emph{etale} whenever the naturality square for $p$ is a $T$-pullbacks:
    \[\input{TikzDrawings/Ch4/Sec4/etale-cond.tikz}\]
    Geometrically, this means that the morphism is a local diffeomorphism; for example, an etale subobject of $\R^n$ in the Dubuc topos is precisely an open subset in the usual sense.
    An endofunctor lifts to the partial map category whenever it preserves the class of monics. A natural transformation lifts to endofunctors on the partial map category whenever it is $T$-cartesian for the class of monics, and the same proof will show that this property holds for etale monics \cite{Cruttwell2019}. 
\end{remark}
  


\section{The prolongation tangent structure}
\label{sec:prol_tang_struct}

One of the most important consequences of the Weil Nerve Theorem  \ref{thm:weil-nerve} is that the category of involution algebroids (with chosen prolongations) may be equipped with two tangent structures. The first tangent structure is the pointwise tangent structure described in Proposition \ref{prop:pointwise-tangent-structure-inv}.
%TODO add reference to the tangent structure on involution algebroids in chapter 3
The tangent functor sends
\[
    (A,\alpha) \mapsto (T.A:\wone \to \C, c.A \o T.\alpha: T.A.T \Rightarrow T.T.A)  
\]
(recall the composition of tangent functors given in Example \ref{ex:composition-of-tangent-functors} (ii)). The structure morphisms will be given by whiskering, so in this case $\theta.A, \theta \in \{ p,0,+,\ell,c\}$. The restriction to tangent functors that preserve transverse limits along with the fact that the natural part $\alpha$ is $T$-cartesian, however, ensures that precomposition with the tangent functor
\[
    (A,\alpha) \mapsto (A.T: \wone \to \C, T.\alpha \o A.c: A.T.T \Rightarrow T.A.T)  
\]
returns an involution algebroid. The structure maps are once again given by whiskering, with the pre-composition tangent structure $A.\theta, \theta \in \{ p,0,+,\ell,c\}$. Preservation of transverse limits guarantees that this tangent structure will satisfy the necessary universality conditions. 

\begin{proposition}[Proposition \ref{prop:second-tangent-structure-inv-algds}]
\label{prop:second-tangent-structure-inv-algds-2}
    The category of involution algebroids with chosen prolongations in a tangent category $\C$ has a second tangent structure, where the action by $\wone$ is given by
    \[
        (A,\alpha) \mapsto ( A.T: \wone \to \C, \alpha.T \o \hat A.c: \hat A.T.T \Rightarrow T.\hat A.T).  
    \]
\end{proposition}
\begin{proof}
    The proposition statement means that the structure morphisms for this new involution algebroid are given by
    \[
        ( A.T, \alpha.T \o  A.c) \cong
        \begin{cases}
            \alpha.T \o  A.c = \anc':& \prol(A) \xrightarrow[]{\pi_1} TA \\
             \quad \;\; A.T.p = \pi':& \prol(A) \xrightarrow[]{p \o \pi_1} A \\
             \quad \;\;\; A.T.0 = \xi':& A \xrightarrow[]{(\xi \o \pi, 0)} \prol(A) \\
            \anc' \o  A.T.\ell = \lambda':& \prol(A) \xrightarrow[]{\lambda \x \ell} T.\prol(A) \\
             \quad \;\ A.T.c = \sigma':& \prol^2(A) \xrightarrow{\sigma \x c} \prol^2(A)
        \end{cases}  
    \]
    Similarly, we can see that
    \begin{gather*}
        ( A.T.T, \alpha.T.T \o  A.c.T \o  A.T.c) \\ = 
        \begin{cases}
            \alpha.T.T \o  A.c.T \o  A.T.c = \anc'':& \prol^2(A) \xrightarrow[]{(\pi_1, \pi_2)} T.\prol(A) \\
             \hspace{2.5cm} A.T.p = \pi'':& \prol^2(A) \xrightarrow[]{(p \o \pi_1, p \o \pi_2):} \prol(A) \\
             \hspace{2.6cm} A.T.0 = \xi'':& \prol(A) \xrightarrow[]{(\xi \o \pi \o \pi_0, 0 \o \pi_1, 0 \o \pi_2)} \prol^2(A) \\
            \hspace{1.85cm} \anc''\o  A.T.\ell = \lambda'':&\prol^2(A) \xrightarrow[]{(\lambda \x \ell \x \ell)} T.\prol^2(A) \\
             \hspace{2.15cm} A.T.T.c = \sigma'':& \prol^3(A) \xrightarrow[]{(\sigma \x c \x c)}\prol^3(A) 
        \end{cases}  
    \end{gather*}
    These coincide with the involution algebroids $\prol'(A), \prol'.\prol'(A)$ in Proposition \ref{prop:second-tangent-structure-inv-algds}: the second tangent structure follows from the fact that the natural transformations for the tangent structure there are given by
    \[
        A.\phi: A.U \Rightarrow A.V, \phi:U \to V \in \{p,0,+, \ell, c\}.  
    \]
    The result follows as a corollary of Theorem \ref{thm:weil-nerve}.
\end{proof}



% This section exposits the second tangent structure on the category of involution algebroids (with chosen prolongations) in some tangent category $\C$. This tangent structure is a natural consequence of the Weil Nerve from \Cref*{thm:weil-nerve}, but is worth giving a more concrete presentation. The Jacobi identity on the sections of an involution algebroid follows as a consequence of this tangent structure.

% The category of cartesian tangent functors $[\wone, \C]$ into any tangent category $\C$ has two actions by $\wone$. The first is post-composition of the tangent functor in $\C$:
% \[
%     \mathsf{CartTang}[\wone, \C] \x \wone \to \mathsf{CartTang}[\wone, \C] ;   \hspace{0.15cm} ((A,\alpha), V) \mapsto (T^V.A, T^V.\alpha)  
% \]
% sends transverse limits in $\wone$ to limits in the functor category $\mathsf{CartTang}(\wone, \C)$ - this tangent structure was described in \Cref{obs:inv-tmonad-anc}. %TODO add the reference
% However, the restriction to transverse-limit-preserving cartesian tangent functors also guarantees that the \emph{pre-composition} action:
% \[
%     \mathsf{CartTang}[\wone, \C] \x \wone \to \mathsf{CartTang}[\wone, \C] ;   \hspace{0.15cm} ((A,\alpha), V) \mapsto (A.T^V, \alpha.T^V)  
% \]
% sends all the transverse limits in $\wone$ to tangent limits in the tangent category $\C$. This will induce a second tangent structure on the category, where each of the structure maps are tangent natural transformations with respect to the pointwise tangent structure.


% One of the original sticking points in the development of abstract tangent structure is quite surprising:  the proof of the Jacobi identity for the Lie bracket of vector fields (see section 3.4 of \cite{Cockett2014} for the history). The eventual proof made use of a string calculus, and a collection of identities \cite{Cockett2015}.  It should come as no surprise, then, that showing the bracket on the sections of an involution algebroid satisfies the Jacobi identity is complex. This section describes the second tangent structure on the category of    involution algebroids, corresponding to the prolongation functor.
% When a tangent category has negatives, there is a natural bracket associated with the abelian group of vector fields on a given $M$, discussed in Lemma  \ref{lem:intertwining-induces-bracket}. Proving that the Jacobi identity holds for this bracket - a reasonably trivial result in the category of smooth manifolds - proves to be a significant challenge (see \cite{Cockett2014} for the history). The eventual proof of this result in \cite{Cockett2015} developed a string calculus to facilitate calculations.

% It should not be surprising, then, that proving the Jacobi identity holds for an involution algebroid also proves to be a challenge. Using the new characterization of involution algebroids as a functor category $[\wone, \C]$, Cockett and Cruttwell's result may be applied directly to an involution algebroid.

% \begin{definition}\label{def:two-tang-on-inv}
%     Let $\C$ be a tangent category. The category of complete involution algebroids in $\C$ has two tangent structures:
%     \begin{enumerate}[(i)]
%         \item Post-composition: This is given by $T_{post}(A)(V) = T.A(V)$ - this is, the cartesian Weil complex is post-composed by $T$. We write this simply as $T$. This tangent structure was expounded on in %TODO look up tangent structure for involution algebroids in chapter three
%         \item Pre-composition: Pre-composition with the tangent functor preserves Cartesian models of $\wone$. This tangent functor will be written as $\prol$. 
%     \end{enumerate}
% \end{definition}
%IDEA: make this more concrete by building the second tangent structure on the category of involution algebroids, where the functor is a tangent functor each of the structure maps will be a tangent natural transformation. This gives a tangent category in the category of tangent categories, e.g. the action by $\wone$ is itself a tangent functor.
% The second tangent structure on involution algebroids can be confusing (proving that the prolongation operation is a tangent bundle on the category of involution algebroids is equivalent to the Weil Nerve Theorem  \ref{thm:weil-nerve}). It is instructive to have a concrete description of $\prol.\hat A, \prol.\prol.\hat A$. 
% \begin{example}%
%     \label{ex:prol-inv-algds}
%     Let $A = (\pi:A \to M, \xi, \lambda, \anc, \sigma)$ be an involution algebroid with chosen prolongations in a tangent category $\C$ (note that this induces an addition map $+_q$). It is important to realize here that the anchored bundle comes with a choice of prolongations - this gives a coherent choice for the prolongations of the anchored bundles (see -).%TODO add reference to the exact tangent structure.
    
%     % The anchored bundle $\prol \hat{A}$ is defined to be:
%     % \begin{gather}
%     %     p \o \pi_1: \prol(A) \to A, (\xi\o \pi, 0):A \to \prol(A), \lambda \x \ell: \prol(A) \to T.\prol(A), \pi_1: \prol(A) \to TA \\
%     %     % p \o (\pi_1,\pi_2): \prol^2(A) \to \prol(A), (\xi\o\pi \o \pi_1, 0 \o \pi_1, 0 \o \pi_2): \prol(A) \to \prol^2(A), \lambda \x \ell \x \ell: \prol^2(A) \to T.\prol^2(A), (\pi_1,\pi_2): \prol^2(A) \to T.\prol(A)
%     % \end{gather}%TODO format
%     Recall that by -, pullback of anchored bundles are computed as pullbacks of the underlying differential bundle, and the anchor is induced universality. %TODO cross reference limits of anchored bundles
%     The structure maps are then given by the following diagram:
%     \[\input{TikzDrawings/Ch4/dbun-pullback-abun.tikz}\]
%     It can be show that the prolongation of $\prol \hat{A}$ is $\prol^2(A)$ using the following diagram:
%     \[\input{TikzDrawings/Ch4/prol-2-pp.tikz}\]
%     The diagram commutes as:
%     \begin{gather*}
%         \anc \o g_1 = T.\pi \o g_2 = T.\pi \o T.p \o f_1 = T.p \o T^2.\pi \o f_1 \\
%         = T.p \o T.\anc \o f_2 = T.\pi \o f_2
%     \end{gather*}
%     This gives the induced involution:
%     \[\input{TikzDrawings/Ch4/induced-involution.tikz}\]
%     as state in -. %TODO fill in cross reference.
%     This gives the pullback involution algebroid:
%     \begin{gather*}
%         p \o \pi_1: \prol(A) \to A, (\xi\o \pi, 0):A \to \prol(A), \lambda \x \ell: \prol(A) \to T.\prol(A), \\
%          \pi_1: \prol(A) \to TA, \sigma \x c: \prol^2(A) \to \prol^2(A)
%         % p \o (\pi_1,\pi_2): \prol^2(A) \to \prol(A), (\xi\o\pi \o \pi_1, 0 \o \pi_1, 0 \o \pi_2): \prol(A) \to \prol^2(A), \lambda \x \ell \x \ell: \prol^2(A) \to T.\prol^2(A), (\pi_1,\pi_2): \prol^2(A) \to T.\prol(A)
%     \end{gather*}%TODO format
%     The involution algebroid axioms all follow by universality, but we check the axioms on this involution map (Definition \ref{def:involution-algd})
%     \begin{itemize}
%         \item The two differential bundle structures on $\prol^2(A)$ are then $\lambda \x \ell \x \ell.T, 0 \x c \o T.\lambda \x T.\ell$, so bilinearity follows.
%         \item The involution axiom is straightforward to check:
%         \[( \sigma \x c )\o (\sigma \x c) = (\sigma \o \sigma) \x (c \o c)  = id\]
%         \item The target axiom follows:
%         \[
%             \pi_1 \o (\pi_1,\pi_2) \o (\sigma \x c) = c \o \pi_2
%         \]
%         \item The unit axiom follows by:  
%         \[
%             (\sigma \x c) \o (\xi\o \pi \o x, \lambda x, \ell \o y) = 
%             (\sigma \o (\xi \o \pi \o x, \lambda x), c \o \ell \o y) =   (\xi\o \pi \o x, \lambda x, \ell \o y)
%         \]          
%         \item In this case, the Yang-Baxter equation becomes:
%         \begin{gather*}
%             (\sigma \x c \x c) \o (A \x T.\sigma \x T.c) \o (\sigma \x c \x c) \\
%             = 
%             (A \x T.\sigma \x T.c) \o (\sigma \x c \x c)\o (A \x T.\sigma \x T.c)
%         \end{gather*}
%         And it follows by parallel application of the Yang-Baxter equation:
%         \begin{itemize}
%             \item $ (\sigma \x c ) \o (A \x T.\sigma) \o (\sigma \x c) 
%             = 
%             (A \x T.\sigma ) \o (\sigma \x c) \o (A \x T.\sigma)$
%             \item $c.T \o T.c \o c.T = T.c \o c.T \o T.c$
%         \end{itemize}            
%     \end{itemize}
%     Applying an identical argument to this anchored bundle show that the involution algebroid $\prol^2 \hat{A}$ is given by the tuple:
%     \begin{gather*}
%         \pi \x p \x p: \prol^2(A) \to M \ts{id}{\pi}\prolong, \hspace{0.15cm}
%         \xi \x 0 \x 0: M \ts{id}{\pi} \prolong \to \prol^2(A), \\
%         \lambda \x \ell: \prolong \to T.(\prolong), \hspace{0.15cm}
%         (\pi_1,\pi_2): \prol^2(A) \to T.\prol(A) \\
%         \sigma \x c \x c: \prol^3(A) \to \prol^3(A)
%     \end{gather*}


%     % Similarly, the anchored bundle $\prol.\prol \hat{A}$ is induced by the pullback of anchored bundles:

%     % Giving it the structure maps:
%     % %TODO put in the itemized list

%     % The chosen prolongations for each $\prol(A), \prol^2(A)$ is given by:
%     % \begin{gather*}
%     %     \prol(U,\prol.\hat A) = \prol(UW,A) \\
%     %     \prol(U,\prol.\prol. \hat A) = \prol(UWW,A)
%     % \end{gather*}
%     % Where $\prol(U,-)$ refers the total object of the differential bundle.
%     % The induced involution maps are given by:
%     % %TODO insert span diagrams

%     % Then Yang-Baxter equation for the prolongation anchored bundles follows by parallel applications of the basic Yang-Baxter equation for $\hat{A}$. 
%     %TODO insert diagrams
%     % \begin{enumerate}
%     %     \item     The ``prolongation'' Lie algebroid, $\prol(A)$, is given by $f \boxtimes W$, for $f \in \{ \pi,\xi,+_q,\hat \lambda, \sigma\}$. It has been discussed in different pieces throughout \Cref*{ch:involution-algebroids} - the anchored bundle structure in -, the tangent involution algebroid structure in -, and then pullback involution algebroid structure in -. %TODO insert references
%     %     Here it is all pulled together and clarified based on insights provided by the Weil nerve construction for convenience. 
%     %     \begin{itemize}
%     %         \item The projection map is given by:
%     %         \[ \infer{p \o \pi_1: \prol(A) \to A}{\pi \x p: \prolong \to M \ts{id}{\pi} A}\]
%     %         Note that this means the $n$-fold pullback powers of the projection are given by $A_n \ts{\anc_n}{T_n.\pi} T_nA$.
%     %         \item The zero map is given by:
%     %         \[
%     %             \infer{(\xi \o \pi, 0): A \to \prol(A)}{\xi \x 0: M \ts{id}{\pi} \to \prolong}
%     %         \]
%     %         \item The addition map is given:
%     %         \[
%     %            ( +_q \x +.A ): A_2 \ts{\anc_2}{T_2.\pi} T_2A \to \prolong
%     %         \]
%     %         \item The (differential bundle) lift map is given:
%     %         \[
%     %             (\lambda \x \ell):\prolong \to T.(\prolong)
%     %         \]
%     %         \item The anchor map is given by:
%     %         \[
%     %             \infer{\pi_1: \prol(A) \to TA}{\anc \x id: \prolong \to TA}  
%     %         \]
%     %         \item The involution map is given by:
%     %         \[
%     %             \sigma \x c: \prolong \ts{T.\anc}{T^2.\pi} T^2A \to \prolong \ts{T.\anc}{T^2.\pi} T^2A 
%     %         \]
%     %     \end{itemize}
%     %     The involution proves to be the most difficult part of this proof. This involution algebroid is induced by the pullbac of the cospan of involution algeboids:
%     %     \[
%     %         A \xrightarrow[]{\anc} TM \xleftarrow[]{T.\pi} TA
%     %     \]
%     %     where $A$ is the involution algebroid, and $TM, TA$ are the tangent involution algebroids on $M, A$. This induces an involution map:
%     %     \[
%     %         (\prolong) \ts{\pi_1}{T.p \o \pi_1} (\prolong)
%     %         \to (\prolong) \ts{\pi_1}{T.p \o \pi_1} (\prolong)
%     %     \]
%     %     Upon identifying 
%     %     \[
%     %         (\prolong) \ts{\pi_1}{T.p \o \pi_1} (\prolong) \cong  \prolong \ts{T.\anc}{T^2.\pi} T^2A
%     %     \]
%     %     Then one can instead check that the involution $\sigma \x c$ satisfies the universal property of the induced map 
%     %     \[
%     %         \prol^2(A) \to \prol^2(A)  
%     %     \]

%     %     \item The second prolongation's involution algebroid structure follows by the same argument. \dots
%     % \end{enumerate}
% \end{example}
% %TODO this paragraph is messy as hell
% Using the Weil nerve (Theorem  \ref{thm:weil-nerve}), involution algebroids have a natural tangent structure given by pre-composing an involution algebroid $\hat A: \wone \to \C$ with the tangent functor in $\wone$ - that is, sending $\hat A \mapsto \hat A . T$. 
% The Weil nerve statement, then, identifies the strict tangent functors:
% \[
%     \wone \to \mathsf{Inv}^*(\C)  
% \]
% and identifies morphisms with tangent natural transformations. It is worth giving a concrete description of each of the tangent structure morphisms.

% \begin{lemma}
%     The prolongation and the anchor $(\prol, anc)$ is a tangent endofunctor on the category of involution algebroids.
% \end{lemma}
% \begin{proof}
%     The assignment on objects was demonstrated in Example  \ref{ex:prol-inv-algds}. On morphisms, note that an involution algebroid morphism $(f,u):\hat{A} \to \hat{B}$ gives a linear morphism:
%     \[
%         f \x T.f: (A \ts{\anc}{T.\pi} TB, \lambda \x \ell) \to (B \ts{\anc}{T.\pi} TB, \lambda \x \ell)   
%     \]
%     as $f$ is linear for the $\lambda$ part and $T.f$ is linear for the $\ell$ part. The base part of this map is $f$, and the anchor map is preserved as the anchor maps are $\pi_1$, so:
%     \[
%         \pi_1 \o (f \x T.f ) = T.f  
%     \]
%     Similarly, the involution is preserved as:
%     \begin{gather*}
%         (f \x T.f \x T^2.f) \o (\sigma \x c) = ((f \x T.f) \o \sigma, T^2.f \o c)  \\= ((\sigma \o (f \x T.f), c \o T.f)) = (\sigma \x c) \o (f \x T.f \x T^2.f)
%     \end{gather*}
%     Thus the map $(f,v)$ maps to $(f \x T.f, f)$, giving functoriality (composition is straightforward to check).

%     To see that this is a tangent functor with respect to the pointwise tangent structure on involution algebroids, first observe that .%TODO finish this proof
% \end{proof}
% Now given the functor part of the tangent structure, check each of the structure maps:
% \begin{lemma}
%     Regard the category of involution algebroids with chosen prolongations in $\C$ as a tangent category, using the pointwise tangent structure. 
%     Then there are tangent natural transformations:
%     \begin{enumerate}[(i)]
%         \item Projection:
%         \[
%             \hat p : (\prol,\anc) \Rightarrow id  
%         \]
%         where $\hat p = (\pi_1, \pi): \prol \hat{A} \to A$.  Note that the pullback powers of $\hat p$ has the underlying differential bundle:
%         \begin{gather*}
%             A \ts{\anc}{T.\pi \o T.\pi_i} T.A_n \xrightarrow[]{p \o \pi_1} A_n, \hspace{0.15cm}
%             A_N \xrightarrow[]{(\xi \o \pi \o \pi_i, 0)} A \ts{\anc}{T.\pi \o T.\pi_i} T.A_n \\
%             A \ts{\anc}{T.\pi \o T.\pi_i} T.A_n \xrightarrow[]{\lambda \x \ell.A_n}  T.(A \ts{\anc}{T.\pi \o T.\pi_i} T.A_n)
%         \end{gather*}
%         \item Addition:
%         \[
%             \hat +: (\prol_2, \anc_2) \Rightarrow (\prol, \anc); \hspace{0.15cm}
%             A \ts{\anc}{T.\pi \o T.\pi_i} T.A_2 \xrightarrow[]{id \x T.+_q} \prolong
%         \]
%         \item Zero:
%         \[
%             \hat 0: id \Rightarrow (\prol,\anc); \hspace{0.15cm}
%             \mathbf{A} \xrightarrow[]{(id, T.(\xi \o \pi))} \mathbf{A}
%         \]
%         \item Lift: 
%         \[
%             \hat \ell: (\prol, \anc) \Rightarrow (\prol, \anc)^2; \hspace{0.15cm}
%             {\hat \ell: \prol.A \xrightarrow[]{id \x (T.\xi \o T.\pi, T.\lambda)} \prol.\prol(A)}
%         \]
%         \item Flip:
%         \[
%             \hat c: (\prol,\anc)^2 \Rightarrow (\prol, \anc)^2; \hspace{0.15cm}
%             \prol^2(A) \xrightarrow[]{(1 \x T.\sigma)} \prol^2(A)
%         \]
%     \end{enumerate}
% \end{lemma}
% \begin{proof}
%     ~\begin{enumerate}[(i)]
%         \item 
%     \end{enumerate}
% \end{proof}

% \begin{observation}
%     Relating this to the $\boxtimes$ operation (Definition \ref{def:boxtimes-span}), the involution algebroid maps for $\prol(V,A)$ are given by $\theta \boxtimes \prol(V,A)$, where as the structure maps for the tangent category of involution algebroids is given by $\prol(V,A) \boxtimes \theta$.
% \end{observation}

% \begin{theorem}
%     There is a tangent structure on the category of involution algebroids with chosen prolongations in $\C$, given by
%     \[
%         (\prol, \hat p, \hat +, \hat 0, \hat \ell, \hat c)  
%     \]
% \end{theorem}





% \begin{theorem}%
%     \label{thm:second-tangent-structure}
%     For any tangent category $\C$, the prolongation endofunctor determines a tangent structure on the category of involution algebroids with chosen prolongations in $\C$. The tangent natural transformations are given by:
%     \begin{itemize}
%         \item The projection $p$ is given by:
%         \[
%             \hat p: 
%             \prol.\mathbf{A} \xrightarrow[]{\pi_0} \mathbf{A}   
%         \]
%         Note that the pullback powers of $\hat p$ has the underlying differential bundle:
%         \begin{gather*}
%             A \ts{\anc}{T.\pi \o T.\pi_i} T.A_n \xrightarrow[]{p \o \pi_1} A_n, \hspace{0.15cm}
%             A_N \xrightarrow[]{(\xi \o \pi \o \pi_i, 0)} A \ts{\anc}{T.\pi \o T.\pi_i} T.A_n \\
%             A \ts{\anc}{T.\pi \o T.\pi_i} T.A_n \xrightarrow[]{\lambda \x \ell.A_n}  T.(A \ts{\anc}{T.\pi \o T.\pi_i} T.A_n)
%         \end{gather*}
%         \item The zero map is givem by:
%         \[
%             \hat 0: 
%             \mathbf{A} \xrightarrow[]{(id, T.(\xi \o \pi))} \mathbf{A}              
%         \]
%         \item The addition map is given by:
%         \[
%             \hat +:A \ts{\anc}{T.\pi \o T.\pi_i} T.A_2 \xrightarrow[]{id \x T.+_q} \prolong
%         \]
%         \item The lift map is given by:
%         \[
%             \infer{A \ts{\anc}{id} TM \ts{id}{T.\pi} TA \xrightarrow[]{id \x T.\xi \x T.\lambda} \prolong \ts{T.\anc}{T^2.\pi} T^2A}
%             {\hat \ell: \prol.A \xrightarrow[]{id \x (T.\xi \o T.\pi, T.\lambda)} \prol.\prol(A)}
%         \]
%         \item The involution map is given by:
%         \[
%             \hat c: \prolong \ts{T.\anc}{T^2.\pi} T^2A
%             \xrightarrow[]{A \x T.\sigma}
%             \prolong \ts{T.\anc}{T^2.\pi} T^2A
%         \]
%     \end{itemize}
% \end{theorem}
% \begin{proof}
    
% \end{proof}
% The ``tangent involution'' algebroid of 
% \[\bar{A} = (\pi:A \to M, \xi, \lambda, \anc, \sigma)\] 
% is the prolongation involution algeboid 
% \begin{gather*}
%     (\pi\x p): \prol(A) \to M \ts{id}{\pi} A, \hspace{0.15cm}
%     (\xi \x 0): TM \ts{id}{\pi} A \to \prol(A), \hspace{0.15cm}
%     (\lambda \x \ell):\prol(A) \to T.\prol(A), \\
%     \pi_1: \prol(A) \to T.A,\hspace{0.15cm}
%     \sigma \x c: \prol^2(A) \to \prol^2(A)
% \end{gather*}
% One of the main sticking point in providing a direct proof that 
% \[\prol: \mathsf{Inv}(\C) \to \mathsf{Inv}(\C)\] 
% is a tangent structure is showing coherences like:
% \[
%     \prol(\prol(\bar{A})) = \prol^2(A)  
% \] and keeping the isomorphism maps straight as one checks the involution algebroid axioms. The coherence theorem Theorem  \ref{thm:weil-nerve} allows us to sidestep this difficulty and describe the involution $\prol(A)$ directly as:
% \begin{gather*}
%     \sigma \x c: \prol^2(A) \to \prol^2(A)
% \end{gather*}
% In this case, the Yang-Baxter equation becomes:
% \[
%     (\sigma \x c \x c) \o (A \x T.\sigma \x T.c) \o (\sigma \x c \x c) 
%     = 
%     (A \x T.\sigma \x T.c) \o (\sigma \x c \x c)\o (A \x T.\sigma \x T.c)
% \]
% And it follows by parallel application of the Yang-Baxter equation:
% \begin{itemize}
%     \item $ (\sigma \x c ) \o (A \x T.\sigma) \o (\sigma \x c) 
%     = 
%     (A \x T.\sigma ) \o (\sigma \x c) \o (A \x T.\sigma)$
%     \item $c.T \o T.c \o c.T = T.c \o c.T \o T.c$
% \end{itemize}
% The ``tangent differential bundle'' structure maps are then the second differential bundle structure on $\prol(A)$ from Proposition  \ref{prop:anc-prol-fun}
% \[
%     \pi_0: \prol(A) \to A    
% \]
% The zero map is given by:
% \[
%     id \x T.\xi \o \anc : A \ts{\anc}{id} TM \to \prol(A)  
% \]
% The addition map by:
% \[
%     (+_q \x +.A): A_2 \ts{\anc_2}{T_2.(\pi)} T_2A \to \prolong
% \]
% The second-prolongation involution algebroid may is:
% \begin{gather*}
%     (\pi \x p): \prol^2(A) \to M \ts{id}{\pi} \prol(A),\hspace{0.15cm}
%     (\xi \x 0): M \ts{id}{\pi} \prol(A) \to \prol^2(A),\hspace{0.15cm}
%     (\lambda \x \ell): \prol(A) \to T.\prol(A), \\
%     (\pi_1,\pi_2): \prol^2(A) \to T.\prol(A) \hspace{0.15cm}
%     (\sigma \x c \x c): \prol^3(A) \to \prol^3(A)
% \end{gather*}

% The lift map, then, is given by:
% \[
%     (id \x T.\xi \x T.\lambda): A \ts{\anc}{id} TM \ts{id}{T.\pi} TA \to \prolong \ts{T.\anc}{T^2.\pi} T^2A  
% \]
% and the involution is given by:
% \[
%     (id \x T.\sigma):
%     \prolong \ts{T.\anc}{T^2.\pi} T^2A \to \prolong \ts{T.\anc}{T^2.\pi} T^2A 
% \]
% Note that in this case, the Yang-Baxter equation reduces to:
% \begin{gather*}
%     (A \x T.\sigma \x T.c \x T.c) \o (A \x TA \x T^2.\sigma \x T^2.c) \o (A \x T.\sigma \x T.c \x T.c)
%     \\ = 
%     (A \x TA \x T^2.\sigma \x T^2.c) \o (A \x T.\sigma \x T.c \x T.c) \o (A \x TA \x T^2.\sigma \x T^2.c)
% \end{gather*}
% which, once again, boils down parallel applications of the Yang-Baxter equation:
% \begin{itemize}
%     \item $id \o id \o id = id = id \o id \o id$,
%     \item $T.((\sigma \x c) \o (1 \x T.\sigma) \o (\sigma \x c))  =  T.((1 \x T.\sigma) \o (\sigma \x c)\o (1 \x T.\sigma))$
%     \item $T.(c.T \o T.c \o c.T) = T.(T.c \o c.T \o T.c)$
% \end{itemize}
% Relating this back to the tangent structure on $\wone^\anc$ induced by an involution algebroid - the structure maps for the anchored bundle are given by tensoring by the identity on the \emph{right}, while the structure maps for the tangent structure on the category of involution algebroids is given by applying the tensor product to the \emph{left}.



\subsection*{The Jacobi identity for involution algebroids}
Classically, the theory of Lie algebroids uses the algebra of sections $\Gamma(\pi)$.  One key observation is that when using the Lie tangent structure $(\mathsf{Inv}(\C), \prol)$, sections of $\pi$ are in bijective correspondence with $\chi_\prol(A)$.
This observation allows for different statements about Lie algebroids to be translated into formal statements about the tangent bundle in $(\mathsf{Inv}(\C), \prol)$.
\begin{proposition}\label{prop:section-morphism}
    Let $A$ be an involution algebroid in $\C$.
    There is a bijection between the sections of\, $\pi$ in $\C$ and the vector fields on $A$ in $\mathsf{Inv}(\C)$:
    \[
        X \in \Gamma(\pi) \mapsto ((id, TX \o \anc), X): A \to T_L(A);
        \hspace{.2cm}
        \hat{X} \in \chi_\prol(A) \mapsto \hat{X}_R: A.R \to A.T.
    \]
\end{proposition}
\begin{proof}
    Recall the coherence for tangent natural transformations $\gamma: (H,\phi) \Rightarrow (G,\psi)$:
    \begin{equation*}
        \input{TikzDrawings/Ch4/Sec5/tangnat.tikz}
    \end{equation*}
    We specify this to a morphism $\hat{X}: (A, \alpha) \Rightarrow T_L(A,\alpha) = (A.T, \alpha.T)$ at $\N, W$:
    \begin{equation*}
        \input{TikzDrawings/Ch4/Sec5/hat-thing.tikz}
    \end{equation*}
    and so  infer that, if we set $X := X_R$, we have $\pi_1 \o X.T = T(X) \o \anc$.
    Furthermore, the condition that $p_L \o X = id$ forces $id = \pi_0 \o X.T$; thus, we can see that every section $X$ of $p_L$ is given by a morphism of the form $((id, TX \o \anc), X))$ on the underlying involution algebroids, where $\pi \o X = id$. 
    
    We now show that every $X \in \Gamma(\pi)$ gives rise to a section of $\pi_L$.
    Observe that the following is a morphism of involution algebroids:
    \begin{equation*}
        \input{TikzDrawings/Ch4/Sec5/inv-algd-mor.tikz}
    \end{equation*}
    Note that it is well typed, as $T.\pi \o T.X \o \anc = \anc \o id$. Check that it is a bundle morphism:
    \[
        p \o \pi_1 \o (id, T.X \o \anc) = p \o T.X \o \anc = X \o p \o \anc = X \o \pi
    \]
    and that it is linear:
    \begin{gather*}
        (\lambda \o \pi_0, \ell\o\pi_1)\o(id, T.X\o \anc) = (\lambda, \ell \o T.X \o \anc) \\= (\lambda, T^2.X \o \ell \o \anc) = (\lambda, T^2.X \o \lambda) = T(id, TX)\o\lambda. 
    \end{gather*}
    Then check that it preserves the anchor:
    \[
        \pi_1 \o (id, TX \o \anc) = TX \o \anc
    \]
    and the involution:
    \begin{gather*}
        (\pi_0, \pi_1, T^2X \o T\anc \pi_1)\o \sigma = (\sigma, T^2X \o T\anc \o \pi_1 \sigma)\\ 
        = (\sigma, T^2X \o c T\anc\o \pi_1)
        = (\sigma(\pi_0, \pi_1), c\pi_3) \o (id, T^2X \o T\anc \o \pi_1).
    \end{gather*}

    Thus we have that $(id, TX \o \anc \pi_1)$ is a morphism of involution algebroids, inducing a morphism of $T$-cartesian Weil complexes.
    Lastly, we check that it is a section of $p^L_A$, but this is clear, since
    \[
        \pi_0 \o (id, TX \o \anc) = id;
    \]
    thus we have the desired bijection.
\end{proof}

Recall that given an involution on an anchored bundle, there is a bracket on its set of sections (see the explicit construction in Section \ref{sec:connections_on_an_involution_algebroid}). 
Given an $X,Y \in \Gamma(\pi)$, there is a bracket defined as follows:
\[
    \hat{\lambda} \o [X,Y]_A +_1 (\xi\pi,0)\o Y = ((\sigma \o (id, TY \o \anc) \o X -_2 (id, TX \o \anc)\o Y)).
\]
A direct proof of the Jacobi identity is a detailed calculation (see the original preprint on involution algebroids \cite{Burke2019}) and still relies on Cockett and Cruttwell's result for an arbitrary tangent category with negatives. As a result of Proposition  \ref{prop:section-morphism}, we can instead use Cockett and Cruttwell's result directly:
\begin{corollary}\label{cor:lie-bracket}
    Let $A$ be a complete involution algebroid in a tangent category $\C$ with negatives. 
    There is a Lie bracket defined on $\Gamma(\pi)$, $[-,-]$ induced by
    \[
        \hat{\lambda} \o [X,Y]_A +_1 (\xi\pi,0)\o Y = ((\sigma \o (id, TY \o \anc) \o X -_2 (id, TX \o \anc)\o Y)).
    \]
\end{corollary}
\begin{proof}
    The bracket is induced by Rosicky's universality diagram, as
    \begin{align*}
        0&=p \o  ((\sigma \o (id, TY \o \anc) \o X - (id, TX \o \anc)\o Y)) - 0Y \\
        &= Tp \o  ((\sigma \o (id, TY \o \anc) \o X - (id, TX \o \anc)\o Y)) - 0Y. 
    \end{align*}
    We look at the Lie tangent structure for $\mathsf{Inv}^*(A)$; this is precisely the vector field induced by
    \[
        ev_R( [\hat{X}, \hat{Y}] ).
    \]
    We complete the proof by using the result that for any $A$ in a tangent category with negatives, the bracket on $\chi(A)$ that is defined by
    \[
        \ell \o [X,Y] = (c\o T.X \o Y - T.Y \o X) - 0X
    \]
    satisfies the Jacobi identity.
\end{proof}

\subsection*{Identifying categories of involution algebroids}%
\label{sub:identifying-cats-of-inv}

Section \ref{sec:identifying-involution-algebroids} identified whenever a functor $\wone \to \C$ is an involution algebroid, whereas this section identifies tangent categories $\C$ that embed into the category of involution algebroids in some tangent category $\C$. We call this structure an \emph{abstract category} of involution algebroids. This notion involves some 2-category theory, using a modified notion of \emph{codescent} (see \cite{Bourke2010} for a development of codescent).

Recall that for any tangent category $\C$, the category of involution algebroids has $\C$ as a reflective subcategory. Furthermore, because limits of involution algebroids are computed pointwise, this reflector is left-exact. This left-exact reflection is the main structure we axiomatize.
\begin{definition}
    An \emph{abstract category of involution algebroids} is a tangent category $\C$ with a left-exact $T$-cartesian tangent localization $(Z,\anc): \C \to \D$, where $L$ satisfies a \emph{codescent} condition:
    \[
        \mathsf{TangCat_{Strict}}(\wone, \C)  \hookrightarrow \mathsf{TangCat_{Lax}}(\wone, \C) \xrightarrow{L_*} \mathsf{TangCat_{Lax}}(\wone, \D)
    \]
    (where $L_*$ denotes post-composition by $L$) is fully faithful. 
\end{definition}
\begin{example}
    The category of involution algebroids in any tangent category $\C$ is an abstract category of involution algebroids using $(\mathsf{Inv}(C), \prol)$. The reflector is the functor sending an involution algebroid to its base space; the $T$-cartesian natural transformation is the anchor map. Any tangent subcategory of $\mathsf{Inv}(\C)$ that contains $\C$ as a full subcategory will give rise to an abstract category of involution algebroids.
\end{example}

\begin{proposition}
    Let $\Z \hookrightarrow \C$ be an abstract category of involution algebroids.
    Then there is an embedding $\C \hookrightarrow \mathsf{Inv}(\Z)$.
\end{proposition}
\begin{proof}
    The proof follows by treating objects in $\C$ as strict tangent functors $\wone \to \C$ and morphisms as tangent natural transformations.
    \[\input{TikzDrawings/Ch4/Sec5/coherence-thing.tikz}\]
    The natural part of $Z$ is $T$-cartesian, and the functor part preserves limits, so $Z.\prol(-,A) =: Z[A]$ determines an involution algebroid in $\Z$, and $f$ a morphism of involution algebroids.
    The embedding is guaranteed by the codescent condition so that the post-composition functor is fully faithful.
\end{proof}
\begin{corollary}
    An abstract category of involution algebroids $\Z \hookrightarrow \C$ is exactly a full subcategory $\Z \hookrightarrow \C \hookrightarrow \mathsf{Inv}(\Z)$.
\end{corollary}





