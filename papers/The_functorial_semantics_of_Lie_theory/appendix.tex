% Sample University of Calgary Thesis
% This file contains the APPENDIX

% If there is just one appendix, it must be called ``Appendix.'' For
% multiple appendices, use \chapter and a descriptive title,
% e.g., \chapter{Questionnaires}
\appendix
\chapter*{Appendix: Background on locally presentable $\vv$-categories}\label{appendix}

% \chapter* doesn't include it in the TOC, so we have to do that by
% hand.  If you have multiple appendices, use \chapter instead and
% remove the following line. 

\addcontentsline{toc}{chapter}{Appendix}
\chaptermark{Appendix}
\markboth{Appendix}{Appendix}
% An appendix is a way to include important information that would
% otherwise clutter up your thesis. It should be included when there is
% additional relevant information that won't fit in the body of your
% thesis. Any Appendix must also be mentioned in the body of your thesis
% (e.g., ``For a full list of interview questions used, please see the
% \hyperref[appendix]{Appendix}''). If your thesis only has one appendix, it
% must be titled ``Appendix.'' If your thesis has more than one appendix,
% add alphabetized letters, starting with ``Appendix~A.'' The following
% are examples of things you might include in appendices:
% \begin{enumerate}
%   \item Copyright permissions with signatures removed
%   \item Additional details of methodology and/or data
%   \item Diagrams of equipment that you developed
%   \item Digital files and/or artwork digital models
%   \item Blank copies of questionnaires or surveys used in your research
% \end{enumerate}

% \section*{Background on groupoids and cubical objects}%
% \label{sec:gpds-and-cub}

% This section develops a formalism for internal groupoids optimized for constructing the Lie functor from groupoids to algebroids.
% As in \cref{sec:Lie_algebroids}, consider the space of source constant tangent vectors:
% \[\input{TikzDrawings/Ch3/Sec8/sor-con.tikz}\]
% Now, there is a well-formed map that sends $(u,v):\prol(A)$ to $T.A$, defined:
% \[
%    (c \o 0 \o p (v))^{-1}; (c \o (T.0(u);v) = (T.0 \o p \o v)^{-1} ; (0 \o u) ; (c \o v)
% \]
% It is not immediately clear where this map comes from or that it satisfies the Yang-Baxter equation.
% Using the simplicial characterization of internal groupoids, note that $(u,v):\prolong$ may be identified with the following horn:
% \[\input{TikzDrawings/Ch3/Sec8/sor-con.tikz}\]
% However, while this may explain the involution property, it fails to give the Yang-Baxter equation for general reasons.
% This section departs from previous literature in Lie theory and follows Burke's idea \cite{Burke2018} of treating groupoids as \emph{cubical objects} rather than simplicial objects. The category of cubical objects in $\C$ may be regarded as a category of $\C$-valued co-presheaves on a cartesian category - this is closely related to the perspective on tangent categories that has been developed in \cite{Garner2018}. The presentaiton of cubical objects used in this chapter is a mix of \cite{Buchholtz2017} and \cite{Grandis2003}.

% Cubical objects were initially considered in the study of homotopy theory, giving a natural algebraic model of ``path spaces''. The zero cells of a cubical object are the points of the base space, while the 1-cells are paths with a source and target:
% \[
%     \delta^\pm: G_1 \to G_0; \hspace{0.5cm}
%     \input{TikzDrawings/Ch3/Sec8/1-cell-draw.tikz}
% \]
% There is also a degeneracy that sends a point to a loop:
% \[
%     \epsilon: G_0 \to G_1; 
%     \input{TikzDrawings/Ch3/Sec8/loop.tikz}
% \]
% This pattern repeats itself for higher paths so that $G_2$ is a set of commuting squares with four projections:
% \input{TikzDrawings/Ch3/Sec8/cub-proj-map.tikz}
% Moreover, the degeneracy, then, creates two squares that are either vertically or horizontal degenerate.
% \begin{definition}
%     The theory of a cubical monoid is the Lawvere theory generated by the maps:
%     \[
%         \input{TikzDrawings/Ch3/Sec8/cub-mon-gen-maps.tikz}
%     \]
%     so that:
%     \begin{itemize}
%         \item Each triple $(A,\delta^{\alpha}, \gamma^{\alpha})$ is a commutative monoid.
%         % \item $\epsilon$ is a degeneracy: $\epsilon \o \delta^{\alpha} = id$
%         % \[ \epsilon \o \delta^{\alpha} = id, \hspace{0.15cm} \epsilon \o \gamma^{\alpha} = \epsilon \o (\epsilon \ox A) = \epsilon \o (A \ox \epsilon)\]
%         \item $\delta^{\beta}$ is an absorbing element for $\gamma^\alpha$ ($\beta \not=\alpha$)
%         \[
%             \gamma^{\beta} \o (\delta^\alpha \ox A) = \delta^\alpha \o \epsilon = \gamma^{\beta} \ox (A \ox \delta^\alpha)
%         \]
%     \end{itemize}
%     The category of cubical objects with symmetric connections in $\C$ is exactly the category of $\C$-valued presheaves on $\square$. Write the truncated cubical category $\square_k$ as the full subcategory of $\square$ spanned by $\{0,\dots, k\}$ - then the category of $k$-truncated cubical objects in $\C$ is the category of $\C$-valued presheaves of $\square_k$.
% \end{definition}
% \begin{example}%
%     \label{ex:cub-examples}
%     % ~\begin{enumerate}[(i)]
%     %     \item 
%         The unit interval $[0,1]$ is a cubical monoid, where $\delta^{\pm}$ picks out the negative or positive polarity ($0$ or $1$). 
%         The positive connection, where $\delta^+$ is the meet (min), while the negative connection is the joint (max). The unit interval generates the cubical approximation of a space, by sending a topological space to the cubical set:
%         \[
%             n \mapsto \mathsf{Top}(I^n,X)
%         \]
%         This induces a functor $\mathsf{Top} \to \widehat{\square}$.
%         % \item 
%     % \end{enumerate}
% \end{example}
% % \begin{definition}
% % \end{definition}
% % \begin{example}
% %    ~\begin{enumerate}[(i)]
% %         \item For every topological space $X$, the unit interval determines a cubical set $n \mapsto [I(n), X]$.
% %         \item Suppose the category of internal categories in $\C$ admits the \emph{arrow category} construction. The arrow category is the internal category whose objects are arrows in $C_1$ and whose arrows are commuting squares.
% %         Then every internal category has an associated cubical object in $\C$, where $C_n$ is just the object-of-objects for the internal category $\C$ after applying the arrow category construction $n$-times. Note that as the category of internal groupoids is a 2-category, this is essentially the power (see \cref{sec:enriched-nerve-constructions}) by $I^n$.
% %     \end{enumerate}
% % \end{example}
% The arrow groupoid is a cubical monoid, where the negative polarity is the source map, positive polarity is the target map, and the unit corresponds to the identity map. The negative and positive monoid maps correspond to sending an arrow $f:x \to y$ to one of the commuting squares:
% \[
%     \input{TikzDrawings/Ch3/Sec8/connections-gpd.tikz}
% \]
% The hom-set $\mathsf{Gpd}(1, G)$ is the set of objects in $G$, and $\mathsf{Gpd}(I,G)$ the set of arrows, $\mathsf{Gpd}(I\x I, G)$ the set of commuting squares, and so on. This determines a functor from groupoids in $\s$ to the category of cubical sets. Note that the \emph{arrow groupoid}, $[I,G]$, is the internal groupoid satisfying the universal property that:
% \[
%     \mathsf{Gpd}(\s)(I, \mathsf{Gpd}(\C)(A,B)) \cong \mathsf{Gpd}(\C)(A,B^I)
% \]
% (this is a $\mathsf{Gpd}$-enriched weighted limit, where the hom $\mathsf{Gpd}(\C)(A,B)$ is treated as a groupoid, see \cref{sec:enriched-nerve-constructions} for weighted limits).
% The object-of-objects of $B^I$ will be the object-of-arrows for $B$, while the object-of-arrows for $B^I$ is the pullback:
% \[\input{TikzDrawings/Ch3/Sec8/g2.tikz}\]
% A commuting square in the arrow groupoid of $B$ is a commuting cube in $B$. Repeating this construction for each $n$ yields a cubical object where:
% \[
%     B_n = \mathsf{ob}(B^{n\cdot I})  
% \]

% \begin{proposition}%
%     \label{prop:gpd-2cub-ff}
%     Consider a full subcategory of groupoids in $\C$ so that for each groupoid $G$ the pullback: \[\input{TikzDrawings/Ch3/Sec8/g2.tikz} \] exists. Then this subcategory of internal groupoids in $\C$ is a full subcategory of 2-truncated cubical objects in $\C$.
% \end{proposition}
% \begin{proof}
%     First, note that a groupoid $s,t:G \to M, e:M \to G$ gives rise to 2-truncated cubical object. 
%     \[
%         G.0 := M, G.1 := G, \input{TikzDrawings/Ch3/Sec8/g2.tikz}  
%     \]
%     Any morphism of 2-truncated cubical objects generated from groupoids will have the map $f_2:G.2 \to F.2$ decompose as 
%     \[
%         \input{TikzDrawings/Ch3/Sec8/functor-diag.tikz}  
%     \]
%     This forces the inclusion of groupoids into 2-truncated cubical objects to be fully faithful.
% \end{proof}
% \begin{corollary}%
%     \label{cor:gpd-cub-ff}
%     If a full subcategory of internal groupoids in $\C$ admits the \emph{arrow groupoid} construction - that is for every internal groupoid $s,t:G \to M, e: M \to G$ the limit:
%     \[
%         \input{TikzDrawings/Ch3/Sec8/g2.tikz}  
%     \]
%     exists so that $G.2$ is a groupoid over $G$ in the category, then this category of groupoids embeds into the category of cubical objects in $\C$.
% \end{corollary}
% \begin{proof}
%     A simple proof may be found in \cref{prop:I-is-dense}. Intuitively, every morphism of cubical objects truncates to a 2-cubical object morphism. The functoriality of the arrow groupoid construction ensures that every internal functor extends to a cubical morphism, giving an inclusion of groupoids to cubical objects. The coherences that force each map to be a pairing of $f_1$ in the proof of \cref{prop:gpd-2cub-ff} still applies, so that each $f_n$ must decompose similarly, giving the bijection on homs.
% \end{proof}
% In the category of smooth manifolds, the composition map of a Lie groupoid is a submersion, this follows by a cancellativity result on surjective submersions in smooth manifolds (see Section 2 of \cite{Behrend2017}): if $f$ is a surjective submersion and $g\o f$ is a surjective submersion, then $g$ is a surjective submersion. In this case, $s$ is a submersion, and by composition $s \o \pi_0$ is as well. Then, as $s\o \pi_0 = s \o m$, it follows that $s \o m$ is a submersion, so then $m$ must be a submersion. Lie groupoids, then, allow for the arrow groupoid construction.
% \begin{corollary}
%     In the category of smooth manifolds, Lie groupoids admit the arrow-groupoid construction and are therefore a full subcategory of cubical manifolds. 
% \end{corollary}
% \begin{observation}%
%     \label{obs:horn-iff-gpd}
%     In any category $\C$, a 2-truncated object is a groupoid if and only if every horn of the form \[\input{TikzDrawings/Ch3/Sec8/gpd-ehorn.tikz}\] has a unique filler \[\input{TikzDrawings/Ch3/Sec8/gpd-cub-horn.tikz}\]. That is, there is an isomorphism: $C_2 \cong C_1 \ts{s}{s} C_1 \ts{t}{s} C_1$.
% \end{observation}

% \section{The linear approximation of a cubical object}%
% \label{sec:linear-approx-of-a-cubical-object}

% This section develops the linear approximation functor that sends an internal groupoid to its associated involution algebroid.
% \[
%     \mathsf{Lie}: \mathsf{Gpd} \to \mathsf{Inv}  
% \]
% The linear approximation functor uses the same basic construction for a Lie algebroid in the category of smooth manifolds described in \Cref{sec:Lie_algebroids} and the beginning of \Cref{sec:gpds-and-cub}. This section extends that construction to a category of so-called ``local groupoids'', which is a full subcategory of cubical objects in $\C$. The first step will be to simplify the diagram definining the linear approximation to an equalizer.

% The diagram constructing the linear approximation of a reflexive graph $(s,t:G \to M, i:M \to G)$ is equivalent to an equalizer. Observe that the universality of the following two diagrams is equivalent.
% \[\input{TikzDrawings/Ch3/Sec9/Lie-diff-as-eq.tikz}\]
% % Post-composing each fork with a monic which preserves the universality of the diagram, so the universality of each diagram is once again equivalent:
% % \[\input{TikzDrawings/Ch3/Sec9/Lie-diff-as-eq-postcompose.tikz}\]
% % This double equalizer is then equivalent to the ternary equalizer:
% % \[\input{TikzDrawings/Ch3/Sec9/Lie-diff-as-ternary-eq.tikz}\]
% % Note, however, that $T.i \o T.s$ and $0 \o p$ are a pair of commuting idempotents, a bit of algebra shows this ternary equalizer is equivalent to:
% A similar argument to \Cref{prop:evf-vbun-is-nonsingular} lets this diagram be presented as an equalizer.
% \[\input{TikzDrawings/Ch3/Sec9/Lie-diff-as-eq-idempotents.tikz}\]
% The map $\delta^-\o\epsilon \in \square$ warrants its own name similar to $0 \o p$.
% \begin{definition}
%     Recall that the map $0 \o p:T \Rightarrow T$ is written $e:T \Rightarrow T$.
%     Write the idempotent $e^-: \delta^- \o \epsilon$ in $\square$.
% \end{definition}
% The linear approximation of a groupoid, then, is the following equalizer of idempotents:
% \[\input{TikzDrawings/Ch3/Sec9/eq-of-idem.tikz}\]
% This can be extended to a functor from reflexive graphs (equivalently, 1-truncated cubical objects) to anchored bundles as follows:
% \begin{lemma}%
%     \label{lem:anc-of-refl}
%     There is a functor from cubical objects to anchored bundles, defined:
%     \[
%         \partial . \mathsf{tr}_1: [\square, \C] \to \mathsf{Anc}(\C)
%     \]
%     in a complete tangent category $\C$.
% \end{lemma}
% \begin{proof}
%     First truncate the cubical object with $\mathsf{tr}_1$ to get a reflexive graph. Then construct a lift on $C^\partial$:
%     \[\input{TikzDrawings/Ch3/Sec9/lift-of-lin-approx.tikz}\]    
%     This will be regular by the commutativity of $T$-limits. The the pre-differential bundle data is given by the projection
%     \[ C^\partial \hookrightarrow T.C_1 \xrightarrow[]{p.C.\delta^-} C_0\]
%     The section is induced by:
%     \[
%         C \xrightarrow[]{(0.C.\delta^-)}  T.C_1
%     \]
%     The anchor map is given by postcomposition with $\delta^+$.
%     \[
%         C^\partial \hookrightarrow T.C_1 \xrightarrow[]{T.\delta^+} T.C_0  
%     \]
% \end{proof}
% The elements of $C^\partial$, then, are 1-cells in $C$ of the form:
% \input{TikzDrawings/Ch3/Sec1/source-constant-tang.tikz}

% The prolongations, then, have a natural interpretation as composible arrows in $T.C$, so $A \ts{\anc}{T.\pi} TA$ embeds into the span composition:
% \input{TikzDrawings/Ch3/Sec9/span-comp-of-anc.tikz}
% This exhbits the a natural inclusion $\prol(A) \hookrightarrow T^2G \ts{T^2.\delta^+}{T^2.\delta^-} T^2C$.
% %TODO 
% % However, $\prol(A)$ is pairs of composible arrows rather than commuting squares. The idea of $C^{\partial\x \partial}$ is to construct the subobject of squares $T^2.C_2$ that 
% % The subobject of $T^2C$ corresponding to $\prol(A)$ is constructed as a doube equalizer.
% \begin{definition}%
%     \label{def:lin-approx}
%     Given a cubical object $C$ in a tangent category $\C$, the $n^{th}$ linear approximation is given by the $n$-fold equalizer:
%     \[\input{TikzDrawings/Ch3/Sec9/n-fold-eq.tikz}\]
%     where:
%     \begin{itemize}[(i)]
%         \item $e_i: T^n \xrightarrow{T^i.e.T^{n-(i+1)}} T^n$
%         \item $e^-_i: n \xrightarrow{i.e^-.(n-(i+1))} n$
%     \end{itemize}
%     For smaller $n$, it makes sense to just write $C^\partial, C^{\partial\partial}$.
% \end{definition}
% Intuitively, $C^{\partial\partial}$ is the space of squares in $T^2.C.[2]$ so that:
% \input{TikzDrawings/Ch3/Sec9/c-pp-illustration.tikz}
% The top arrow, then, is in the image of the inclusion
% \[
%     0: C^\partial \to T.C^\partial
% \]
% whereas a right arrow belongs to $T.C^\partial$ - thus the image of projecting off the top and bottom arrows corresponds to:
% \[
%     C^\partial \ts{T^2.\delta^+ \o 0 \o \iota}{T^2.\delta^- \o \iota} T.C^\partial  
%     \cong C^\partial \ts{T.\delta^+ \o \iota}{T^2.\delta^- \o \iota} T.C^\partial 
%     \cong \prol(C^\partial)
% \]
% Thus, this induces a map $C^{\partial\partial} \to \prol(C^\partial)$ - a similar map exists for $C^{3\partial}$.
% \begin{definition}
%     For every cubical object $C$ in a tangent category admitting the first three linear approximations, define the maps:
%     \[
%     d: C^{\partial \x \partial}
%     \xrightarrow{(C.\delta^-_0 \o \iota, C.\delta^+_1 \o \iota)} C^\partial \ts{C.\delta^+ \o \iota}{T.C.\delta^- \o \iota} T.C^\partial = \prol(C^\partial)
%     \]
%     This map picks off the top and right arrows. 
%     \[\input{TikzDrawings/Ch3/Sec9/cpp-to-L.tikz}\]
%     For the second prolongation, there is the map:
%     \[
%     d^2: C^{\partial \x \partial \x \partial} 
%     \xrightarrow{}  C^{\partial \x \partial} \ts{}{}  T.C^{\partial \x \partial}
%     \xrightarrow{} \prol^2(C^\partial)
%     \]
%     that starts with a cube and chooses a single path:
%     \input{TikzDrawings/Ch3/Sec9/cpp-to-L2.tikz}
% \end{definition}
% In case $G$ is the cubical object induced by a groupoid, the $d$ maps are isomorphisms using the unique horn-filler condition from Observation \ref{obs:horn-iff-gpd}.
% \begin{proposition}%
%     \label{prop:gpd-is-local-gpd}
%     Let $G$ be a groupoid in a tangent category regarded as a cubical object admitting the first three linear approximations. Then $d,d^2$ are isomorphisms.
% \end{proposition}
% \begin{proof}
%     For $d$, first start with an element $(u,v)$ of $\prol(G^\partial)$, and form the following horn in $T^2G$:
%     \[
%         \input{TikzDrawings/Ch3/Sec9/horn-prol.tikz}
%     \]
%     The unique filler is then $(e.T \o v)^{-1};u;v$, observe that:
%     \[
%         T.e \o((e.T \o v)^{-1};u;v) = id; (T.e \o u); (T.e \o v) = id; T.e \o u; id = u
%     \]
%     as $T.e \o u = u, T.e \o v = id$.
%     For $d^2$, then, start with the path:
%     \[\input{TikzDrawings/Ch3/Sec9/horn-one.tikz}\]
%     Use the idempotents $e.T.T, e.e.T$ to get the following vertical arrows and construct the unique fillers:
%     \[\input{TikzDrawings/Ch3/Sec9/horn-prol-filled-partial.tikz}\hspace{0.35cm} \input{TikzDrawings/Ch3/Sec9/prol-horns-filled.tikz}\]
%     This is a partial horn in the arrow groupoid, as in the starting point for $d$, so the result follows inductively.
% \end{proof}


% % Our goal in this section is to construct maps that play the role of the structure maps from an involution algebroid using the objects $C^{n\partial}$, as these $d$-maps satisfy the following coherence:
% % \begin{proposition}
% %     Let $C$ be the cubical object generated from an internal groupoid in a tangent category $\C$.
% %     Then $d^2:C^{2\partial} \to \prol(\C^\partial)$ and $d^3: C^{3\partial} \to \prol^2(C^{\partial})$ are isomorphisms.
% % \end{proposition}
% % \begin{proof}
% %     The isomorphism is an immediate consequence of the unique filler condition for an internal groupoid. Given $(u,v):\prol(C^\partial)$, we have that $\pi \o p \o v = \pi \o u$, as:
% %     \[
% %         \pi \o p \o v = p \o T.\pi \o v = p \o \anc \o u = \pi \o u
% %     \]
% %     The horn induces a unique filler:
% %     \input{TikzDrawings/Ch3/Sec9/unique-horn-filler.tikz}
% %     The unique filler is then $(0 \o p (v))^{-1};(T.0 \o u);v$, and observe that 
% %     \[
% %         T.p \o ((0 \o p (v))^{-1};(T.0 \o u);v) = e;u;e = u 
% %     \]
% %     so the square belongs to $C^{\partial \x \partial}$. A similar argument holds for $C^{\partial \x \partial \x \partial}$.
% %     Starting with the three arrows, we build horns using the idempotent $e = 0 \o p$ and draw them using dotted lines, and draw the uniquely induced filler using squiggly lines:
% %     \input{TikzDrawings/Ch3/Sec9/build-cube-from-L.tikz}
% % \end{proof}
% % \begin{observation}
% %     Note that given the isomorphism $\prol(A) \cong C^{\partial \x \partial}$, the map from the target axiom:
% %     \[
% %         \pi_1: \prolong \to TA
% %     \]
% %     is induced using $\delta^+$, giving:
% %     \[
% %         C^{\partial \x \partial} \to T.C^\partial
% %     \]
% %     This is in line with the intuition of $\pi_1$ as a higher-order anchor map.
% %     \input{TikzDrawings/Ch3/Sec9/pi-1-higher-order-anc.tikz}
% % \end{observation}


% \begin{definition}%
%     \label{def:c-part-structure-maps}
%     Let $C$ be a cubical object in a tangent category $\C$ where the first three linear approximations exist.
%     The following morphisms can be formed:
%     \begin{enumerate}[(i)]
%         \item The degeneracy maps are defined by pairing $0,\epsilon$.
%         \[\input{TikzDrawings/Ch3/Sec9/c-00.tikz}\]
%         \item The lift maps are defined by pairing $\ell, \gamma^+$.  
%         \[\input{TikzDrawings/Ch3/Sec9/cll.tikz}\]
%         % \begin{enumerate}[(a)]
%         %     \item $C^\ell: C^\partial \to C^{\partial \x \partial}$ 
%         %     \input{TikzDrawings/Ch3/Sec9/cl.tikz}
%         %     \item $C^{\partial \x \ell}$
%         %     
%         %     \item $C^{\partial \x \ell}$
%         %     \input{TikzDrawings/Ch3/Sec9/clll.tikz}
%         % \end{enumerate}
%         \item The involution maps are defined by pairing $c, \sigma$ 
%         \[\input{TikzDrawings/Ch3/Sec9/csig.tikz}\]
%         % \begin{enumerate}[(a)]
%         %     \item $C^c: C^{\partial \x \partial} \to C^{\partial \x \partial}$
%         %     
%         %     where $\sigma$ is the permutation sending $(1,2) \mapsto (2,1)$.
%         %     \item $C^{c \x \partial}:C^{\partial \x \partial \x \partial}\to C^{\partial \x \partial \x \partial}$
%         %     \input{TikzDrawings/Ch3/Sec9/csp.tikz}
%         %     where $\sigma$ is the permutation sending $(1,2,3) \mapsto (2,1,3)$.
%         %     \item $C^{c \x \partial}:C^{\partial \x \partial \x \partial}\to C^{\partial \x \partial \x \partial}$
%         %     \input{TikzDrawings/Ch3/Sec9/cps.tikz}
%         %     where $\sigma$ is the permutation sending $(1,2,3) \mapsto (1,3,2)$.
%         % \end{enumerate}
%     \end{enumerate}
% \end{definition}

% % \begin{definition}
% %     Let $C$ be a cubical object in a tangent category that allows the formation of $C^\partial$.
% %     Then define the following degeneracy maps:
% %     \begin{enumerate}[(i)]
% %         \item $C^0: C \to C^\partial$  
% %         \input{TikzDrawings/Ch3/Sec9/c0.tikz}
% %         \item Writing $0_i:T \xrightarrow{T^i.0} T^2$, then set: $C^{0_0} = C^{0 \x \partial}, C^{0_1} = C^{\partial \x 0}$.
% %         \input{TikzDrawings/Ch3/Sec9/c-00.tikz}
% %         \item Writing $0_i:T^2 \xrightarrow{T^i.0} T^3$, then set: $C^{0_0} = C^{0 \x \partial}, C^{0_1} = C^{\partial \x 0\x \partial}, C^{0_2} = C^{\partial \x \partial \x 0}$.
% %         \input{TikzDrawings/Ch3/Sec9/c-000.tikz}
% %     \end{enumerate}
% % \end{definition}
% % \begin{definition}
% %     Let $C$ be a cubical object in a tangent category, define the following projection maps by pairing together $\delta^-$ and $p$.
% %     \begin{enumerate}[(i)]
% %         \item $C^p: C^\partial \to C$
% %         \input{TikzDrawings/Ch3/Sec9/cq.tikz}
% %         \item $C^{p_0} = C^{p \x \partial}, C^{p_1} = C^{\partial \x p}$
% %         \input{TikzDrawings/Ch3/Sec9/cqq.tikz}
% %         \item $C^{p_0} = C^{p \x \partial \x \partial}, C^{p_1} = C^{\partial \x p \x \partial}, C^{p_2} = C^{\partial \x \partial \x p}$
% %         \input{TikzDrawings/Ch3/Sec9/cqqq.tikz}
% %     \end{enumerate}
% % \end{definition}

% % The morphism $\ell.C.\gamma^+: TC_1 \to T^2C_2$ should induce the lift, and the coherences should follow by post-composition.
% % Note that the map $\ell.C.\gamma^+$ is the following map:
% % \input{TikzDrawings/Ch3/Sec9/cl-illustration.tikz}
% % \begin{definition}
% %     Let $C$ be a cubical object with connections in a tangent category $\C$. 
% %     Then we can form the following three maps:
% %     \begin{enumerate}[(i)]
% %         \item $C^\ell: C^\partial \to C^{\partial \x \partial}$ 
% %         \input{TikzDrawings/Ch3/Sec9/cl.tikz}
% %         \item $C^{\partial \x \ell}$
% %         \input{TikzDrawings/Ch3/Sec9/cll.tikz}
% %         \item $C^{\partial \x \ell}$
% %         \input{TikzDrawings/Ch3/Sec9/clll.tikz}
% %     \end{enumerate}
% % \end{definition}
% % Note that coassociativity follows by post-composition.
% % \begin{lemma}
% %     For any cubical object in a tangent category admitting the first three linear approximations:
% %     \[
% %         C^{\partial \x \ell} \o C^{\ell} = C^{\ell \x \partial} \o C^\ell
% %     \]
% % \end{lemma}
% % \begin{proof}
% %     Coassociativity follows by post-composition with the monic inclusion into $T^n.C.n$.
% % \end{proof}
% % One may also induce a universality property that is similar to Grabowski's non-singularity condition - that is the diagram:
% % \[\input{TikzDrawings/Ch3/Sec9/univ-ell.tikz}\]
% % Where the map $C^{e \x \partial}$ is given by $C^{p \x \partial} \o C^{\ell \x \partial} = C^{0 \x \partial} \o C^{p \o \partial}$.
% % This requires the unit map behave similar to the unit from a groupoid.
% % \begin{definition}
% %     The transverse coforks in $\square_{Conn}$ are generated by the commuting cofork: 
% %     \[\input{TikzDrawings/Ch3/Sec9/transverse-coforks.tikz}\]
% %     closed under tensor with copies of itself and the identity cofork.
% %     A cubical object has unital degeneracies if and only if it sends transverse coforks to equalizers in $\C$.
% % \end{definition}
% % \begin{example}
% %     Every internal category has unital degeneracies, as $id;f = f$. 
% % \end{example}
% % \begin{lemma}
% %     Let $C$ be a cubical object with connections and epic degeneracies in a tangent category $\C$.
% %     Then the following diagrams are equalizers:
% %     \input{TikzDrawings/Ch3/Sec9/c-ros.tikz}
% % \end{lemma}

% % Finally, one may pair the two symmetry maps to get the following map:
% % \input{TikzDrawings/Ch3/Sec9/csig-illustration.tikz}

% % \begin{definition}
% %     Given a cubical object, form the maps:
% %     \begin{enumerate}[(i)]
% %         \item $C^c: C^{\partial \x \partial} \to C^{\partial \x \partial}$
% %         \input{TikzDrawings/Ch3/Sec9/csig.tikz}
% %         where $\sigma$ is the permutation sending $(1,2) \mapsto (2,1)$.
% %         \item $C^{c \x \partial}:C^{\partial \x \partial \x \partial}\to C^{\partial \x \partial \x \partial}$
% %         \input{TikzDrawings/Ch3/Sec9/csp.tikz}
% %         where $\sigma$ is the permutation sending $(1,2,3) \mapsto (2,1,3)$.
% %         \item $C^{c \x \partial}:C^{\partial \x \partial \x \partial}\to C^{\partial \x \partial \x \partial}$
% %         \input{TikzDrawings/Ch3/Sec9/cps.tikz}
% %         where $\sigma$ is the permutation sending $(1,2,3) \mapsto (1,3,2)$.
% %     \end{enumerate}
% % \end{definition}


% % The following results all hold by post-composition.
% % \begin{lemma}
% %     Let $C$ be a cubical object in a tangent category with the first three linear approximations.
% %     \begin{enumerate}[(i)]
% %         \item $C^c$ is an involution.
% %         \item Bilinearity: For the lifts $C^{\ell \x \partial}, C^{c \x \partial} \o C^{\partial \x \ell}$,  
% %         \[C^{\ell \x \partial} \o C^{c} = C^{\partial \x c} \o C^{c \x \partial} \o C^{\partial \x \ell}\]
% %         \item The target law: $C^{\delta[+] \x \delta[+]} \o C^{c} = c.C$
% %         \item The symmetry satisfies the Yang-Baxter equation   
% %         $C^{c \x \partial} \o C^{\partial \x c} \o C^{c \x \partial} = C^{\partial \x c} \o C^{c \x \partial} \o C^{\partial \x c}$
% %     \end{enumerate}
% % \end{lemma}





% % The results in this section indicate there is a close relationship between the linear approximation of a cubical object in a tangent category and the axioms of an involution algebroid.
% Using post-composition, it is clear that some version for each axiom of an involution algebroid is satisfied by the maps constructed in \cref{def:c-part-structure-maps}.
% \begin{proposition}
%     Consider a cubical obect with symmetric connections in a tangent category $\C$. 
%     The induced maps $C^{f}$ satisfy the following coherences.
%     \begin{enumerate}[(i)]
%         \item Involution: $C^{c} \o C^c = id_C$.
%         \item Bilinearity: $C^{\ell \x \partial} \o C^{c} = C^{\partial \x c} \o C^{c \x \partial} \o C^{\partial \x \ell}$
%         \item Unit: $C^c \o C^\ell = C^\ell$
%         \item Target $C^{\delta[+] \x \delta[+]} \o C^{c} = c.C$
%         \item Yang-Baxter: $C^{c \x \partial} \o C^{\partial \x c} \o C^{c \x \partial} = C^{\partial \x c} \o C^{c \x \partial} \o C^{\partial \x c}$
%     \end{enumerate}
% \end{proposition}

% \begin{theorem}%
%     \label{thm:functor-cub-to-inv}
%     Consider the full subcategory of cubical objects in a tangent category $\C$ so that the first three linear approximations exist and $(d,d^2)$ are isomorphisms. The linear approximation $C^{n\cdot \partial}$ determines a functor into the category of involution algebroids in $\C$.
% \end{theorem}
% \begin{proof}
%     Given that $d'$ is an isomorphism, it suffices to show that:
%     \[
%         G^\ell = (\xi\o\pi,\lambda),        
%         \hspace{0.5cm} 
%         G^{id \ox c} = 1 \x T.\sigma,        
%         \hspace{0.5cm}
%         G^{c \ox id} = \sigma \x c,
%     \]
%     (where $\lambda$ is the lift on $C^\partial$ induced in \cref{lem:anc-of-refl}).
%     This follows from the coherences:
%     \input{TikzDrawings/Ch3/Sec9/d-coherences.tikz}
%     So:
%     \begin{align*}
%         \pi_0 \o d' \o G^{id \ox c} &= id \\
%         (\pi_1, \pi_2) \o d' \o G^{id \ox c} &= T.\sigma \\
%         (\pi_0, \pi_1) \o d' \o G^{c \ox id} &= \sigma \\
%         \pi_2 \o d' \o G^{c \ox id} &= c 
%     \end{align*} %TODO \ell map
% \end{proof}
% \begin{corollary}
%     There is a functor from the category of internal groupoids in a tangent category to the category of involution algebroids.
% \end{corollary}

% In classical Lie theory, there is the notion of a ``local'' Lie groupoid $G$ \cite{Crainic2003}, where the composition is only defined on an open subset:
% \[
%     U \subseteq G\ts{t}{s} G  
% \]
% where $U$ contains the open set around $e(m)$ for every point $m \in M$. The corresponding notion in a tangent category, then, should be those cubical objects where $d$ is an isomorphism, as this would appear to correspond to an ``infinitesimal'' notion of composition. This perspective is developed in \Cref{sec:inf-nerve-of-a-gpd}.

% \section{}

Synthetic differential geometry uses a topos-theoretic framework, such as can be found in the final chapters of \cite{Lavendhomme1996}.
The internal language of a topos is a key tool, and models of synthetic differential geometry are constructed using sheaf constructions.
Tangent categories have an analagous toolbox, drawing from \emph{enriched locally presentable categories}. We begin with locally presentable categories; basic material on these may be found in \cite{Adamek1994}, and on enriched categories in \cite{Kelly2005}.
The following class of colimits is foundational in the theory of locally presentable categories.
% In the case of $\s$ we have a clear description of filtered colimits, as a colimits are conical.
\begin{definition}\label{def:filtered-colimit}
	A category is \emph{filtered} whenever any finite diagram in $\C$ has a cocone. A colimit whose diagram category is filtered is called a \emph{filtered colimit}. Note that a category $\d$ is filtered if and only if every colimit diagram $\d \to \s$ commutes with every finite limit in $\s$.
\end{definition}
\begin{example}
    ~\begin{enumerate}[(i)]
        \item Any finite category $\c$ with a terminal object is filtered: for any diagram $D:\d \to \c$, the natural transformation $!:id \Rightarrow K_1$ is a cocone via whiskering.\[\input{TikzDrawings/Ch5/cocone-whiskering.tikz}\]
        \item Any category with finite colimits is filtered, because each diagram has a colimit (and therefore a cocone). 
    \end{enumerate}
\end{example}
\begin{definition}%
    \label{def:lfp}
	An object in a cocomplete $\,\C$ is \emph{finitely presentable} whenever $\C(C,-)$ preserves filtered colimits. A cocomplete category $\,\C$ is \emph{locally finitely presentable} whenever it has a small subcategory $\,\C_{fp}$ of finitely presentable objects so that every object in $\C$ is given by the coend
	\[
		C \cong \int^{X \in \C_{fp}} \C(X, C) \cdot X.
	\]
\end{definition}
\begin{example}
	In the category $\s$, the subcategory of finitely presentable objects is precisely the category of finite sets. 
	Every set is a filtered colimit of finite sets, so that $\s$ is locally finitely presentable. 
\end{example}
Intuitively, this means that every object in $\C$ is generated by gluing together finitely presentable objects in a canonical way; hence the name locally finitely presentable categories. 

The subcategory of finitely presentable objects in a cocomplete category is itself finitely cocomplete.
This follows from the fact that $\C(-,X):\C^{op} \to \s$ is a continuous functor (it sends colimits in $\C$ to limits in $\s$) and also that filtered colimits commute with finite limits in $\s$.  
Thus, for any filtered colimit of representables,
\[
    \C(\mathsf{col}D, \mathsf{lim}F) \cong \mathsf{lim}_X \mathsf{lim}_Y  \C(D(X),F(Y)) \cong \mathsf{lim}_Y \mathsf{lim}_X  \C(D(X),F(Y)). 
\]
% A locally finitely presentable category is a cocomplete category where each object is a filtered colimit of finitely presentable objects. 
% This means that every object in the category comes with a simple ``recipe'' to construct it as a filtered colimit of finitely presentable objects. 
There are several ways to characterize locally finitely presentable categories; these may be found in \cite{Adamek1994}.
\begin{proposition}%
    \label{prop:lfp-defs}
    For a category $\,\C$ the following are equivalent:
    \begin{enumerate}[(i)]
        \item $\C$ is a locally finitely presentable category.
        \item For a small cocomplete category $\c$, $\C = \mathsf{Lex}(\c, \s)$ (where $\mathsf{Lex}$ means finite-limit-preserving) and $\c = \C^{op}_{fp}$.
        \item $\C$ is the category of models for some \emph{limit sketch} (e.g. a small category $\c$ equipped with a class of cones $\prol$, where a model is a functor into $\s$ sending cones to limit diagrams).
        \item $\C$ is a reflective subcategory of a presheaf topos for some small category $\c$ (this holds tautologically for presheaf topoi).
        \item $\C$ is a full subcategory of a locally presentable category that is closed under limits (this relies on a large cardinal axiom known as \emph{Vopenka's principle}).
    \end{enumerate}
\end{proposition}

One of the remarkable aspects of the theory of locally finitely presentable categories is that its results translate over to enrichment in a monoidal category $\vv$ without any significant changes.
\begin{definition}%
\label{def:v-cat}
    Let $\vv$ be a monoidal category. 
    A $\vv$-graph $\C$ is given by a (large) set of objects $\C_0$, and a map
    \[
        \C: \C_0 \x \C_0 \to \vv. 	
    \]
    Given a collection of maps
    \[
        m_{ABC}: \C(B,C) \x \C(A,B) \to \C(A,C), \hspace{0.25cm}
        j_A: I \to \C(A,A)	
    \]
    we say $\,\C$ is a $\vv$-category whenever the following associativity and unitality diagrams commute:
    \input{TikzDrawings/Ch5/vcat-axioms-assoc.tikz}
    \input{TikzDrawings/Ch5/vcat-axioms-unit.tikz}

\end{definition}
Examples of enrichment abound throughout category theory.
\begin{example}
    ~\begin{enumerate}[(i)]
        \item Any category with biproducts is enriched in the category of commutative monoids.
        \item A symmetric monoidal closed category is self-enriched using the internal hom $[-,-]$.
        \item A 2-category is a category enriched in $\mathsf{Cat}$. 
        \item The unit $\vv$ category is the 1-object $\vv$ category whose hom-object is $I$.
        \item When $\vv$ is symmetric monoidal, we can construct the opposite $\vv$-category of $\c$, where $\C^{op}(A,B) = \C(B, A)$.
        The new composition is defined by
        \[m^{op}: \C^{op}(B,C) \ox \C^{op}(A,B) \xrightarrow{\sigma} \C(B, A) \ox  \C(C,B) \xrightarrow{m} \C(C,A) = \C^{op}(A,C) \]
    \end{enumerate}
\end{example}
    
There is a 2-category of $\vv$-categories for any $\vv$.
\begin{definition}%
\label{def:vcat}
	A $\vv$-functor $F: \C \to \D$ is an assignment on objects $F_0: \C_0 \to \D_0$ and a collection of maps $F_{A,B}: \C(A,B) \to \D(F_0A,F_0B)$ so that the following diagrams are satisfied:
    \[\input{TikzDrawings/Ch5/v-functor.tikz}\]
	A $\vv$-natural transformation $F \Rightarrow G$ is a collection of maps $\gamma_{A}: I \to \D(F_0A,G_0A)$ so that the following diagram commutes: \[\input{TikzDrawings/Ch5/v-nat.tikz}\]
    $\vv\cat$ is the 2-category of $\vv$-categories, $\vv$-functors and $\vv$-natural transformations.
\end{definition}
	
Enriched categories have a richer notion of (co)limit than ordinary categories.
A limit for a functor is defined as a universal cone; that is, a right Kan extension along $!$:
\[\input{TikzDrawings/Ch5/limit-universal-cone.tikz}\]
so that the universal property
\[
    \forall C:
    \hspace{0.15cm}
    \C(C, \lim F) \cong [\C,\s](K_C, F-)
\]
is satisfied.
Dually, a colimit is a universal cocone, and so a left Kan extension:
\[\input{TikzDrawings/Ch5/universal-cocone.tikz}\]
where the couniversal property
\[
    \forall C:
    \hspace{0.15cm}
    [\C^{op},\s](F-, K_C) \cong \C(\mathsf{colim}F, C) 
\]
is satisfied.
This does not translate over to $\vv$-categories, for a few reasons. 
An obvious one is that $\vv$ need not be cartesian monoidal, so there is no reason a $\vv$-category should have a functor $\C \to 1$.
The notion of a \emph{weighted} colimit is more appropriate:
\begin{definition}
	Let $\d$ be a $\vv$-category, and call $W: \d \to \vv$ the weight and $D: \d \to \C$ the diagram.
	The limit of $D$ weighted by $W$, or the weighted limit $\{ W, \d \}$, is an object satisfying the following universal property:
	\[
        \forall C: \hspace{0.15cm} \vv(W, [\C,\vv](F-, K_C)) \cong \C({F,W}, C).
	\]
	Dually, given a diagram $P: \d^{op} \to \C$, the colimit of $P$ weighted by $W$, $W*P$, is an object in $\C$ satisfying the universal property
	\[
        \forall C: \hspace{0.15cm}  \vv(W, [\C,\vv](K_C, F-)) \cong \C(C, F*W ).
	\]
\end{definition}
\begin{example}
	~\begin{enumerate}[(i)]
		\item A conical (co)limit is weighted by the constant functor into the unit $I$, $K_I: \d \to w$.
		This coincides with the definition of ordinary (co)limits.
		\item The power of $C$ by an object $V \in \vv$ is an object $C^V$ satisfying the universal property that
		\[
			\C(B, C^V) \cong \vv(V, \C(B,C)).
		\]
		Dually, the copower by $V$ is given by 
		\[
			\C(B\bullet V, C) \cong \vv(V, \C(B, C)). 
		\]
		\item In ordinary categories, the power of an object $C$ by $V$ is given by the (possibly infinitary) $V$-indexed coproduct:
		\[
			C^V \cong \coprod_{v\in V} C	
		\]
		whereas the copower is the possibly infinitary $V$-index product of $C$:
		\[
			V\bullet C \cong \prod_{v \in V} C.
		\]
		\item In a symmetric monoidal closed category $\vv$, the power by $\vv$ is given by $[V,-]$ and the copower by $V \ox \_$.
	\end{enumerate}
\end{example}

Following Kelly's work, we observe that every finite weighted limit is composed of powers and conical limits.
\begin{proposition}
	Every weighted (co)limit is the composition of conical (co)limits and (co)powers.
\end{proposition}
% \begin{proof}
%     Note that every $W$-weighted limit may be rewritten as:
%     \[
%         \forall C: \hspace{0.15cm} \vv(W, [\C,\vv](F-, K_C)) \cong [\C,\vv](F-, K_C^W) \cong \C(\lim F, C^W)
%     \]
% \end{proof}
\begin{corollary}
	Weighted limits are exactly conical/ordinary limits in ordinary categories.
\end{corollary}

The previous definitions relating to locally presentable categories transfer over mostly unchanged.
\begin{definition}%
    \label{def:v-lfp}
	~\begin{enumerate}[(i)]
		\item A monoidal closed $\vv$ is \emph{presentable as a closed category} if $\vv_{fp}$ is a monoidal subcategory.
		\item Given $\vv$ as in (i), a weight $W: \d \to \vv$ is \emph{filtered} whenever $W\star (-): \widehat\d \to \vv$ preserves finite limits.
		\item An object in a cocomplete $\vv$ category is \emph{finitely presentable} when $\C(C,-)$ preserves filtered weighted colimits.
		\item A cocomplete $\vv$-category $\c$ is \emph{locally finitely presentable} whenever every object $C$ is the coend 
			\[C \cong \int^{X \in \C_{fp}}\C(X,C)\cdot X. \]
	\end{enumerate}
\end{definition}
\begin{proposition}[\cite{Freyd1972, Kelly2005}]%
    \label{prop:v-lfp}
	All of the previous results about locally presentable categories lift up to locally presentable $\vv$-categories, provided that $\vv$ is presentable as a closed category.
\end{proposition}

