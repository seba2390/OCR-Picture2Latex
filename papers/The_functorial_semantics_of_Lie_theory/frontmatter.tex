% Sample University of Calgary Thesis
% This file contains the FRONT MATTER other thna the title page

\chapter{Abstract}

% The abstract is a concise and accurate summary of the research
% contained in your thesis. It gives the reader a snapshot of your
% research by highlighting key points. It includes the problem, method of
% study, and general conclusion; relevant key words that will help
% people find your research. It must be no longer than 350 words (Note:
% hyphenated words or words separated by a slash are counted as two
% words). It must be double-spaced or one-half spaced, and not contain
% graphs, tables, or illustrations.

Ehresmann's introduction of differentiable groupoids in the 1950s may be seen as a starting point for two diverging lines of research, many-object Lie theory (the study of Lie algebroids and Lie groupoids) and sketch theory. This thesis uses tangent categories to build a bridge between these two lines of research, providing a structural account of Lie algebroids and the Lie functor. 

To accomplish this, we develop the theory of involution algebroids, which are a tangent-categorical sketch of Lie algebroids. We show that the category of Lie algebroids is precisely the category of involution algebroids in smooth manifolds, and that the category of Weil algebras is precisely the classifying category of an involution algebroid. This exhibits the category of Lie algebroids as a tangent-categorical functor category, and the Lie functor via precomposition with a functor
\[
    \partial: \wone \to \th_{\mathsf{Gpd}},
\]bringing Lie algebroids and the Lie functor into the realm of \emph{functorial semantics}.

% This thesis aims to provide a structural account of Lie algebroids and the Lie functor from groupoids to algebroids. 
% This thesis studies Lie algebroids through the lens of tangent categories,
% using involution algebroids as their tangent-categorical axiomatization.
% Using Leung and Garner's perspectives of tangent categories as certain actegories \cite{Leung2017} or enriched categories \cite{Garner2018}, respectively, involution algebroids are exactly generalized tangent bundles.
% This new perspective allows for a functorial semantics presentation of Lie algebroids and the classical Lie functor from Lie groupoids to Lie algebroids \cite{Pradines1967}.

\chapter{Preface}

This thesis is the original work of the author.

This thesis project arose from attempts to broadly understand differential geometry, and in particular the differential geometry of mechanics. First steps were taken with Jonathan Gallagher in understanding how the enriched perspective on category theory introduced in \cite{Garner2018} might relate to differential geometric structures. The basic structures in this thesis, however, came out of discussions and collaboration with Matthew Burke in the Lie theory context.

\Cref{ch:tangent_categories} is an introduction to the theory of tangent categories, and contains no new results.  
\Cref{ch:differential_bundles} began as a joint project with Matthew, who ultimately had to leave the project due to time constraints, although by that time we had already found the basic structure of the proof that differential bundles are vector bundles in the category of smooth manifolds (proved in Theorem \ref{iso-of-cats-dbun-sman}).
The current structure of the chapter, particularly with the emphasis on associative coalgebras of the weak tangent comonad $(T,\ell)$ and the tight connection to Grabowski's previous work on the Euler Vector Field construction (\cite{Grabowski2009}), is original work.
The basic results in that chapter appear in \cite{MacAdam2021}; however, the results have been streamlined (there was originally a notion of a \emph{strong} differential bundle, in Section \ref{sec:lifts-pdbs-dbs} it is observed that all differential bundles are strong), and there are some new observations about linear connections (Theorem \ref{thm:linear-connection-from-total-space}).

\Cref{ch:involution-algebroids} is a rewrite of a preprint written with Matthew on involution algebroids (\cite{Burke2019}), while \Cref{sec:connections_on_an_involution_algebroid} is due to conversations with Richard Garner (we expect to release a new paper on involution algebroids based on these results as a joint work). The original idea of augmenting an anchored bundle with an involution map is entirely due to Matthew, and the isomorphism on objects between involution and Lie algebroids follows calculations shared by Richard Garner (Proposition \ref{prop:inv-algd-mor-conn}). My own contribution in this section comes from connecting this to the work of \cite{Martinez2001}, as well as giving the bijection on morphisms for involution and Lie algebroids (Theorem \ref{thm:iso-of-cats-Lie}).

The first three chapters provide the background required for the deeper results in \Cref{chap:weil-nerve}. 
In the work of \cite{Weinstein1996,Martinez2001,Leon2005} on Lie algebroids, it was observed that the ``prolongation'' of a Lie algebroid acted like a tangent bundle. Proposition \ref{prop:second-tangent-structure-inv-algds-2} makes this intuition precise by showing the prolongation is a second tangent structure on the category of Lie algebroids. The main theorem in this chapter (Theorem \ref{thm:iso-of-cats-inv-emcs}) shows that involution algebroids in a tangent category $\C$ are equivalent to tangent functors $(A,\alpha)$ from $\wone$ to $\C$ so that the functor $A$ preserves transverse limits and the natural transformation $\alpha$ is $T$-cartesian (Definition \ref{def:cart-nat}). These are entirely new observations about Lie algebroids and are original work of the author. 

% The idea that the prolongation of a Lie algebroid acts like a generalized tangent bundle is explicit in the work of \cite{Weinstein1996,Martinez2001,Leon2005}, and Proposition \ref{prop:second-tangent-structure-inv-algds-2} makes that intuition concrete. The main theorem in this section, that involution algebroids in a tangent category $\C$ are equivalent to tangent functors from $\wone$ to $\C$ satisfying certain conditions, Theorems \ref{thm:weil-nerve} and \ref{thm:iso-of-cats-inv-emcs}, is an entirely new observation about Lie algebroids and is original work of the author. It also draws a connection to Grothendieck's nerve construction (that proves internal categories in $\C$ embed into the category of simplicial objects in $\C$) and the Segal conditions (which identify precisely those simplicial objects that are the nerves of internal categories) which has not previously been identified in the Lie algebroids literature (\cite{Segal1974}).

Finally, \Cref{ch:inf-nerve-and-realization} puts the first four chapters into the language of enriched category theory, using Garner's enriched perspective on tangent categories (\cite{Garner2018}). The first original results in this chapter demonstrate that differential bundles and anchored bundles are models of $\w$-sketches (Propositions \ref{prop:Lambda-is-refl-subcat} and \ref{prop:nerve-anc-work}), where $\w$ is the site of enrichment for tangent categories.
Next, the enriched theories framework of \cite{Bourke2019} is used to prove Theorem \ref{thm:pullback-in-cat-of-cats-inv-algd}, that involution algebroids are models of a nervous theory, which is the enriched version of Theorem \ref{thm:iso-of-cats-inv-emcs}. 
% By Theorem \ref{thm:pullback-in-cat-of-cats-inv-algd}, involution algebroids in the tangent category $\w$ are monadic over anchored bundles (\cite{Kapranov2007} had already anticipated the result that Lie algebroids are algebraic over anchored bundles in the scheme-theoretic language, but here it follows as a corollary). 
The thesis concludes with the \emph{Lie Realization}, Theorem \ref{thm:lie-realization}, which is a new characterization of Lie differentiation and introduces an entirely new way to construct adjunctions between categories of ``smooth groupoids'' and categories of ``Lie algebroids'' using purely enriched-categorical methods (this is in contrast to the geometric approach used in \cite{Crainic2003} and the homotopy theoretic approach in \cite{Sullivan1977}). 


% As a first piece of original work, it is demonstrated that differential bundles and anchored bundles are models of $\w$-sketches in Propositions \ref{prop:Lambda-is-refl-subcat} and \ref{prop:nerve-anc-work} (the theory of enriched sketches may be found \cite{Kelly2005a}, in this case $\w$ is the site of enrichment for tangent categories).
% Next, we use the enriched theories framework of \cite{Bourke2019} to prove Theorem \ref{thm:pullback-in-cat-of-cats-inv-algd}, that involution algebroids are models of a nervous theory, which is an enriched version of Theorem \ref{thm:iso-of-cats-inv-emcs} that makes a new connection to Grothendieck and Segal's nerve construction (\cite{Segal1974}).  Thus, the tangent-categorical approach leads to new results that extend structure previously observed via geometric methods. The thesis concludes with the \emph{Lie Realization}, Theorem \ref{thm:lie-realization}, which is a new characterization of Lie differentiation and introduced an entirely new way to construct adjunctions between categories of ``smooth groupoids'' and categories of ``Lie algebroids'' using purely enriched-categorical methods (this is in contrast, to, say the geometric approach used in \cite{Crainic2003} or the homotopy theoretic approach in \cite{Sullivan1977}). 

% This basic work in this thesis arose from a collaboration with Matthew Burke, and the basic definitions and results may be found in the pre-print \cite{Burke2019}. 

% The original intuition of augmenting an anchored bundle with an involution map, and treating this structure the primitive belongs entirely to Matthew. The proof in \cref{sec:connections_on_an_involution_algebroid} came from a letter from Richard Garner. My own contribution to that chapter would be the translation of Martinez's construction of an involution from a Lie algebroid, completing the proof of an isomorphism of categories, and the results in \cref{sec:gpds-and-cub,sec:linear-approx-of-a-cubical-object} extending the Lie functor to a class of cubical objects in a tangent category. \Cref{ch:involution-algebroids} will be collected and polished into a new paper on involution algbebroids with Garner and Bourke, superceding the results in \cite{Burke2019}. The work in \cref{ch:differential_bundles} also began as a collaboration with Matthew, however he had to exit the project due to time constraints - most of those results appeared in the paper \cite{MacAdam2021}, although the results of \cref{sec:revisiting-defn-diff-bundle} are new. The results in \Cref{chap:weil-nerve,ch:inf-nerve-and-realization} are entirely new.
This thesis touches on relatively advanced topics in two areas of math: differential geometry (Lie algebroids) and enriched category theory (enriched nerve constructions).
I have striven to keep it as self-contained as possible, introducing the category of smooth manifolds and tangent categories and including an appendix with the basics of enriched category theory and locally presentable category theory. Material on foundational category theory (which is to say, anything that can be found in \cite{MacLane1988}) and basic differential calculus (see any calculus textbook) is used without citation or introduction; this includes limit, adjunction, monads and monadicity theorems, and the calculus of Kan extensions and coends. Some facts about horizontal composition spans are used in \Cref{chap:weil-nerve}, but nothing that goes beyond the basic definition.

% The preface gives a statement of where the information included in
% your thesis came from. It gives credit to the authors who informed
% your work.  A manuscript-based thesis must include an explanation of
% which parts of your thesis were already published and details of the
% publication. For example:
% \begin{quote}
% Portions of the introductory text of Chapter~1 are used with
% permission from Smith et al. (2015) of which I am an author. Table~1.1
% is modified from Supplementary Table~3 in Smith et
% al. (2014). Chapter~2 of this thesis has been published as J.~Smith
% and J.~Doe, ``Title of Article''. Journal Name, vol.~1, issue~1.
% \end{quote}
% For a traditional thesis requiring ethics approval, include the name
% of the board that approved the research project, the title of the
% project, and the number of the approval certificate. For example:
% \begin{quote}
% This thesis is original, unpublished, independent work by the author,
% J. Doe.  The experiments reported in Chapters 2--4 were covered by
% Ethics Certificate number 007, issued by the University of Calgary
% Conjoint Health Ethics Board for the project ``Project Title'' on
% December 15, 2016.
% \end{quote}
% For a traditional thesis with no ethics approval required, simply
% state something along the lines of:
% \begin{quote}
% This thesis is original, unpublished, independent work by the author,
% Jane M. Doe.
% \end{quote}
  
\chapter{Acknowledgments}
%This doesn't seem very urgent
% This is probably the last thing on the list
% \begin{itemize}
%     \item Funding (PIMS/NSERC)
%     \item Robin (supervising)
%     \item Rory: spending a summer teaching me enriched category theory
%     \item Matthew: including me in the project in the first place, walking me through the basics of Lie theory.
%     \item Jonathan: collaborators
%     \item Kristine: helpful conversations
%     \item rest of lab/peripatetic seminar/tangent categories crowd
%     \item Family
% \end{itemize}
% First and foremost, I would like to thank my examiners for agreeing to read this work. 

I would first like to thank my supervisor, Robin Cockett, for guiding this project and for his investment of time in reading various preprints and walking through proofs with me. I also owe a special debt of thanks to Matthew Burke, who was a close collaborator on this project and spearheaded the application of tangent categories to Lie theory, and to Jonathan Gallagher for extensive discussions about tangent categories. I am also grateful to Rory Lucyshyn-Wright for spending a summer teaching me enriched category theory, and to Kristine Bauer for helpful conversations over the years. I would also like to thank all of the members of the Peripatetic Seminar and others over the years for friendly discussion and feedback: Chad Nester, Prashant Kumar, J.S. Lemay, Priyaa Srinivasan, Cole Comfort, Daniel Satanove, Rachel Hardeman, and Geoff Vooys. %TODO more?

Finally, I am grateful to my parents and family, and to my wife Niloofar for her support as I wrote this thesis.

\dedication{For my wife, Niloofar.}

\chapter{Notation}\label{ch:notation}

We start with a table of symbols:
\begin{center}
    \begin{tabular}{|p{2cm}|p{13cm}|}
    \hline
    \textbf{Notation}  &
       \\\hline
    $\C, \D, \dots$  &
      A (usually tangent) category, treated as a general context for mathematics, denoted using mathbb \\ \hline
    $\a, \c, \dots$  &
     A small category treated as a mathematical object, denoted by mathcal \\ \hline
    $A, B, \dots$ &
     Objects in a category and also functors. written using capital letters \\ \hline
    $f, \phi$ &
     Morphisms in a category and natural transformations: lower-case Roman and Greek letters \\ \hline
    $f \o g$ &
     The composition of two maps $g:A \to B, f:B \to C$ (applicative notation) \\ \hline
    $\pi_i$ &
     The projection from the $i^{th}$ component of an $n$-fold pullback $A_0 \ts{q_0}{q_1} \dots \ts{q_{n-1}}{q_n}A_n$ or a product $\prod^n A_i$ \\ \hline
    $F.G$ &
     Composition of two functors, $F:\mathbb{D} \to \mathbb{E}, G: \mathbb{C} \to \mathbb{D}$ (applicative notation) \\ \hline
    $\phi.G$ &
     Whiskering of a natural transformation \\ \hline
    $\ox$ &
     Tensor product in a monoidal category \\ \hline
    $\boxtimes$ &
     A restricted notion of span composition, introduced in Definition \ref{def:boxtimes-span} \\ \hline
     $T, p, 0, +, \ell, c$ &
    The data for an arbitrary tangent category, introduced in Definition \ref{def:tangent-cat} \\ \hline
    $D, \odot, 0, !, \delta$ &
     The data for an infinitesimal object, introduced in Definition \ref{def:inf-object} \\ \hline
    % $\boxplus$ &
    %  A restricted form of cospan compsosition introduced in \Cref{def:coprol}. \\ \hline
    % $\delta, \boxtimes, 0', d'$ &
    %  An infinitesimal object in the category of groupoids in a tangent category introduced in \Cref{def:partial-gpd}. \\ \hline
    $T^n$ &
     Iterated application of an endofunctor $T$ \\ \hline
    $T_n$ &
     The $n$-fold pullback power of $p:T \to id$ for a tangent category, $T \ts{p}{p} \dots \ts{p}{p} T$ \\ \hline
    %  $\nabla[f]$ &
    %  Notation for maps between objects with connections, see Definition \ref{def:nabla-notation}. \\ \hline
    \end{tabular}
\end{center}

\newpage 

We also provide a table of categories:
% Please add the following required packages to your document preamble:
% \usepackage{booktabs}
\begin{center}
    \begin{tabular}{|p{2cm}|p{13cm}|}
    \hline
    \textbf{Notation}  &
       \\ \hline
    $\mathsf{SMan}$ &
     The category of smooth manifolds \\ \hline
    $\C^\ell$ &
     The category of lifts in a tangent category $\C$, introduced in \Cref{def:lift} \\ \hline
    $\mathsf{NonSing}(\C)$ &
     The category of non-singular lifts in a tangent category $\C$, introduced in \Cref{def:non-singular-lift} \\ \hline
    $\mathsf{PDiff}(\C)$ &
     The category of pre-differential bundles in a tangent category $\C$, introduced in \Cref{def:pdb}(i) \\ \hline
    $\mathsf{Diff}(\C)$ &
     The category of differential bundles in a tangent category $\C$, introduced in \Cref{def:pdb}(ii) \\ \hline
    $\mathsf{LieAlgd}$ &
     The category of Lie algebroids, introduced in \Cref{sec:Lie_algebroids} \\ \hline
    $\mathsf{InvAlgd}(\C)$ &
    The category of anchored bundles in a tangent category $\C$, introduced in \Cref{def:anchored_bundles} \\ \hline
    $\mathsf{Anc}(A)$ &     
    The category of anchored bundles in a tangent category $\C$, introduced in \Cref{def:anchored_bundles} \\ \hline
    $\mathsf{Anc}^\prol(A)$ &
     The category of involution algebroids with chosen prolongations in a tangent category $\C$ \\ \hline
    $\wone$ &
     The category of Weil algebras, introduced in \Cref{sec:weil-algebras-tangent-structure} \\ \hline
    $W$ &
     Notation for the Weil algebra $\N[x]/x^2$ \\ \hline
    $\w$ &
     The category of transverse-limit-preserving functors $\wone \to \s$, introduced in \Cref{def:weil-space} \\ \hline
    $\wone^n$ & The category of Weil algebras with width $n$, used in \Cref{def:truncated-wone} \\ \hline
    $\wone^*$ & The full subcategory of $\wone$ spanned by $\{ \N, W\}$ \\ \hline
    $\prol$ &
      The classifying $\w$-category of an anchored bundles and all of its prolongations, introduced in Definition \ref{def:prol} \\ \hline
      % $\prol^n$ &
      % The classifying $\w$-category of an anchored bundles the prolongations corresponding to $\wone^n$. \\ \hline
    \end{tabular}
\end{center}
\tableofcontents

% If you have no tables, delete the next line

% \listoftables

% If you have no figures, delete the next line

% \listoffigures

% Consult Ch 9 of the memoir class manual on how to set up other
% content lists. Note that memoir does not automatically clear the
% page for these. ucalgmthesis fixes this for the default table of
% contents and lists of tables and figures, but not for anything you
% define
