% A Sample Thesis for the University of Calgary
% =============================================

% This is a sample LaTeX document to build a University of Calgary
% graduate thesis according to the guidelines of the Faculty of
% Graduate Studies, available here:
% https://grad.ucalgary.ca/current-students/thesis-based-students/thesis/building-thesis

% To use this sample for your own thesis, rename this file, make any changes
% necessary, then add the content of your thesis to the included files
% frontmatter.tex, chapter1.tex, etc.

% First, we load the UCalgary Memoir Thesis class ucalgmthesis,
% available at https://github.com/rzach/ucalgmthesis

% By default (without options), this produces a 1-1/2 spaced thesis in
% 11 point font without running heads. See the README file for a
% description of all package options.

% In our sample we give three options: Option utopia sets the thesis in a
% nice font. Option headers produces running heads. Option manuscript
% formats the page in a way suitable for reading and commenting: 12 pt type,
% double spaced, approx. 25 lines per page, with approx. 72 characters
% per line. For filing in the Vault, remove that option to produce a
% more compact thesis with a slightly better layout.


\documentclass[utopia
              % ,headers
              ,singlespaced
              ,fullpage
              % ,manuscript
              ]{ucalgmthesis}
%Swap singlespaced and fullpage for manuscript
% \usepackage[toc,page]{appendix}
% PGF plots
\usepackage{pgfplots}
\usepgfplotslibrary{patchplots}
\usetikzlibrary{patterns, positioning, arrows}

\usepackage{tikz}
\usepackage{tikz-cd}

%%TIKZ-IT
\usetikzlibrary{backgrounds}
\usetikzlibrary{arrows}
\usetikzlibrary{shapes,shapes.geometric,shapes.misc}

% this style is applied by default to any tikzpicture included via \tikzfig
\tikzstyle{tikzfig}=[baseline=-0.25em,scale=0.5]

% these are dummy properties used by TikZiT, but ignored by LaTex
\pgfkeys{/tikz/tikzit fill/.initial=0}
\pgfkeys{/tikz/tikzit draw/.initial=0}
\pgfkeys{/tikz/tikzit shape/.initial=0}
\pgfkeys{/tikz/tikzit category/.initial=0}

% standard layers used in .tikz files
\pgfdeclarelayer{edgelayer}
\pgfdeclarelayer{nodelayer}
\pgfsetlayers{background,edgelayer,nodelayer,main}

% style for blank nodes
\tikzstyle{none}=[inner sep=0mm]

% include a .tikz file
\newcommand{\tikzfig}[1]{%
{\tikzstyle{every picture}=[tikzfig]
\IfFileExists{#1.tikz}
  {\input{#1.tikz}}
  {%
      \IfFileExists{./figures/#1.tikz}
        {\input{./figures/#1.tikz}}
        {\tikz[baseline=-0.5em]{\node[draw=red,font=\color{red},fill=red!10!white] {\textit{#1}};}}%
    }
  }%
}

% the same as \tikzfig, but in a {center} environment
\newcommand{\ctikzfig}[1]{%
\begin{center}\rm
  \tikzfig{#1}
\end{center}}

% fix strange self-loops, which are PGF/TikZ default
\tikzstyle{every loop}=[]

% TiKZ style file generated by TikZiT. You may edit this file manually,
% but some things (e.g. comments) may be overwritten. To be readable in
% TikZiT, the only non-comment lines must be of the form:
% \tikzstyle{NAME}=[PROPERTY LIST]

% Node styles
\tikzstyle{new style 0}=[fill=white, draw=black, shape=circle]

% Edge styles

% \usepackage{quiver}
%%%% QUIVER
\usetikzlibrary{calc}
% `pathmorphing` is necessary to draw squiggly arrows.
\usetikzlibrary{decorations.pathmorphing}

% A TikZ style for curved arrows of a fixed height, due to AndréC.
\tikzset{curve/.style={settings={#1},to path={(\tikztostart)
    .. controls ($(\tikztostart)!\pv{pos}!(\tikztotarget)!\pv{height}!270:(\tikztotarget)$)
    and ($(\tikztostart)!1-\pv{pos}!(\tikztotarget)!\pv{height}!270:(\tikztotarget)$)
    .. (\tikztotarget)\tikztonodes}},
    settings/.code={\tikzset{quiver/.cd,#1}
        \def\pv##1{\pgfkeysvalueof{/tikz/quiver/##1}}},
    quiver/.cd,pos/.initial=0.35,height/.initial=0}

% TikZ arrowhead/tail styles.
\tikzset{tail reversed/.code={\pgfsetarrowsstart{tikzcd to}}}
\tikzset{2tail/.code={\pgfsetarrowsstart{Implies[reversed]}}}
\tikzset{2tail reversed/.code={\pgfsetarrowsstart{Implies}}}
% TikZ arrow styles.
\tikzset{no body/.style={/tikz/dash pattern=on 0 off 1mm}}


% Using LaTeX? Then you're probably using math, and so you want to use
% the AMS math commands and define some theorem environments! But you
% can take these out or use your own favorite theorem package.

\usepackage{amsmath,amsthm}

\usepackage{thmtools} 
\usepackage{thm-restate}

\theoremstyle{plain}
\newtheorem{theorem}{Theorem}[section]
\newtheorem{definition}[theorem]{Definition}
\newtheorem{observation}[theorem]{Observation}
\newtheorem{proposition}[theorem]{Proposition}
\newtheorem{corollary}[theorem]{Corollary}
\newtheorem{remark}[theorem]{Remark}
\newtheorem{lemma}[theorem]{Lemma}
\newtheorem{conjecture}[theorem]{Conjecture}
\newtheorem{example}[theorem]{Example}
\newtheorem{notation}[theorem]{Notation}

%
% microtype makes everything look better

\usepackage{microtype}

% We'll need some colored links, so we load xcolor and hyperref. But
% you can take that out if you don't want links at all.

% \usepackage[dvipsnames]{xcolor}

% You can turn off the boxes around links made by hyperref. Then links
% will appear in a different color, and per guidelines, all links must
% be blue or black. For blue links say

\usepackage[colorlinks]{hyperref} 

% For black links, 
% \usepackage[hidelinks]{hyperref}

% If you prefer hyperref's boxes around links (which don't print), you
% can also change their color. With boxes around links, you probably
% don't want everything in the table of contents to be a link, so we
% only make the page numbers links.
%
% \usepackage[allbordercolors=Periwinkle,linktocpage]{hyperref}

% The table of contents in your PDF reader's sidebar is just titles by
% default, but it's nice to also have chapter and section numbers for
% easy navigation.

\usepackage[numbered]{bookmark}

% For author-year references, you probably want to use natbib with a
% bibliography style appropriate for your discipline; or check out
% latexbib!

\usepackage[round]{natbib}
\bibliographystyle{plainnat}

% The blindtext package produces the ``lorem ipsum''
% texts in this sample and can safely be removed.

\usepackage{blindtext}


% \usepackage{booktabs}

% Now we put in the information for the thesis title page.

% Full Name

\author{Benjamin MacAdam}

% Full Title

\title{The functorial semantics of Lie theory}

% Official name of the degree

\degree{Doctor of Philosophy}

% The name of the graduate program (not the department!)

\prog{Graduate Program in Computer Science}

% The month (for the final version: when you file, not when you defended)

\monthname{JUNE}

% The year

\thesisyear{2022}

% Tell hyperref to put author and title into the PDF metadata

\hypersetup{pdfinfo={Title={\thetitle},Author={\theauthor}}}

% If you want memoir to produce endnotes, turn them on here

% \makepagenote

% Often you only want to output a single chapter so you can send it to
% your supervisor. Use includeonly and make sure everything you don't
% always want compiled to PDF is include'd from a separate file. For
% instance, to produce a PDF only of chapter 1, endnotes and
% bibliography, say

% \includeonly{
%     frontmatter 
%   , Chapters/Introduction
%   , Chapters/Ch1-TangentStructure
%   , Chapters/Ch2-DifferentialBundles
%   , Chapters/Ch3-LieAlgebroids
%   , Chapters/Ch4-WeilNerve
%   , Chapters/Ch5-InfNerve
%   , backmatter
%    }

% To compile only the title page, which you need when submitting your
% thesis, say

% \includeonly{titlepage}

% and then copy the resulting PDF to a separate file.


%%%%%%%%%%%%%%%%%%%%%%%%%%%%%%%%%%%%%%%%%%%%%
%%%%%%%%%%%%%%% Commands
%%%%%%%%%%%%%%%%%%%%%%%%%%%%%%%%%%%%%%%%%%%%%

\renewcommand{\o}{\circ}
\newcommand{\anc}{\varrho}
\newcommand{\tot}[1]{\overline{#1}}
\newcommand{\bas}[1]{\underline{#1}}
\newcommand{\prolong}{A \ts{\anc}{T\pi} TA}
\newcommand{\prol}{\mathcal{L}}
\renewcommand{\d}[0]{\mathcal{D}}
\newcommand{\s}[0]{\mathsf{Set}}
\newcommand{\wone}[0]{\mathsf{Weil}_1}
\newcommand{\weil}[0]{\mathsf{Weil}}
% Weil spaces...used a single letter because it gets used a LOT
\newcommand{\w}[0]{{\mathcal{W}}}
\newcommand{\W}[0]{{\mathcal{W}}}
\renewcommand{\k}[0]{\mathcal{K}}
\renewcommand{\c}[0]{\mathcal{C}}
\renewcommand{\a}[0]{\mathcal{A}}
\newcommand{\e}[0]{\mathcal{E}}
\newcommand{\m}[0]{\mathcal{M}}
\newcommand{\g}[0]{\mathcal{G}}
\renewcommand{\b}[0]{\mathcal{B}}
\renewcommand{\j}[0]{\mathcal{J}}
\renewcommand{\l}[0]{\mathcal{L}}
\renewcommand{\u}[0]{\mathcal{U}}
\newcommand{\vv}[0]{\mathcal{V}}
\newcommand{\q}[0]{\mathcal{Q}}

\newcommand{\C}[0]{\mathbb{C}}
\newcommand{\D}[0]{\mathbb{D}}
\newcommand{\N}[0]{\mathbb{N}}
\newcommand{\R}[0]{\mathbb{R}}
\newcommand{\T}[0]{\mathbb{T}}
\newcommand{\Z}[0]{\mathbb{Z}}
%theories
\newcommand{\sh}[0]{\mathcal{S}}
\renewcommand{\th}[0]{\mathcal{T}}

%%%%%%%%%%%%%%%
%complicated prolong
%%%%%%%%%%%%%%%
\newcommand{\uvprolong}[2]{\prol({#1},A) \ts{\anc^{#1}}{T^{#1}.\pi^{{#2}}} T^{#1}.\prol({#2},A)}
% \newcommand{\uvwprolong}[3]{
%     \prol({#1},A) \ts{\anc^{#1}}{T^{}} \prol() \ts{}{} T^U.T.\prol(V,A)
%   }


%%%%%%%%%%%%%% Natural transformations
\newcommand{\assoc}[0]{a} % For a monoidal cat
\newcommand{\tnat}[0]{\alpha} % The standard notation for the (F,a) tangent functor
\newcommand{\x}[0]{\times}
\newcommand{\ox}[0]{\otimes}
\newcommand{\phat}[0]{\hat{p}}
%\DeclareFontFamily{U}{min}{}
%\DeclareFontShape{U}{min}{m}{n}{<-> udmj30}{}
%\newcommand\yon{\!\text{\usefont{U}{min}{m}{n}\symbol{'210}}\!}
\newcommand{\yon}[0]{\mathcal{Y}}
\newcommand{\cat}[0]{\mathsf{Cat}}

\newcommand{\<}{\langle}
\renewcommand{\>}{\rangle}
\newcommand{\ts}[2]{{_{#1}}\!\times_{#2}\!}
\newcommand{\po}[2]{{_{#1}}\!+_{#2}\!}

%Other packages
\usepackage{todonotes}
\usepackage{proof}
\usepackage{cleveref}
\newcommand{\Diff}[0]{\mathsf{Diff}}
\newcommand{\Dlin}[0]{\mathsf{DLin}}


% \setcounter{chapter}{-1}

\pgfplotsset{compat=1.16}

%%%%%%%%%%%%%%%%%%%%%%%%%%%%%%%%%%%%%%%%%%%%%%%%%%%%%%%%%%%%%%%%%%%%%%%%%%%%%%
%%%%%%%%%%%%%%%%%%%%%%%%%%%%%%%%%%%%%%%%%%%%%%%%%%%%%%%%%%%%%%%%%%%%%%%%%%%%%%
% %Comment out this line for normal margins
% \fullpagethesis
% %%%%%%%%%%%%%%%%%%%%%%%%%%%%%%%%%%%%%%%%%%%%%%%%%%%%%%%%%%%%%%%%%%%%%%%%%%%%%%
%%%%%%%%%%%%%%%%%%%%%%%%%%%%%%%%%%%%%%%%%%%%%%%%%%%%%%%%%%%%%%%%%%%%%%%%%%%%%%

\begin{document}

\frontmatter

% titlepage.tex just makes the titlepage; it's in its own file so you
% can typeset it alone using includeonly.

\begin{titlingpage}

  \small \centering{\textsf{Dissertation zur Erlangung des Doktorgrades der Technischen Fakult{\"a}t der Albert-Ludwigs-Universit{\"a}t Freiburg im Breisgau}}

  \vspace{0.5cm}

  \rule{\textwidth}{0.4mm}\\%

  \huge \centering{\textbf{Symbolic Search for Optimal Planning with Expressive Extensions}}\\
  %\huge \centering{\textbf{Symbolic Search for Optimal Planning with Expressive Extensions and Plan Requirements}}\\
  %\huge \centering{\textbf{Extending the Scope of Symbolic Search for Optimal Planning}}\\

  \vspace{-0.4cm}

  \rule{\textwidth}{0.4mm}\\%

  \vspace{2.5cm}

  \centering{\textbf{David Speck}}

  %\vspace{3.8cm}
  \vspace{1cm}
  \centering{\resizebox*{0.7\textwidth}{!}{
      \includegraphics{figures/logo_neu}}}

  \vspace{1.5cm}

  \LARGE \centering{\textbf{2022}}

  \normalsize

  \clearpage

  \vspace*{\fill}
  \begin{flushleft}
    \noindent
    \textbf{Dean:}\\
    Prof. Dr. Roland Zengerle, \emph{University of Freiburg, Germany}\\

    \bigskip

    \noindent
    \textbf{PhD advisor and first reviewer:}\\
    Prof. Dr. Bernhard Nebel, \emph{University of Freiburg, Germany}\\

    \bigskip

    \noindent
    \textbf{Second reviewer:}\\
    Prof. Dr. {\'A}lvaro Torralba, \emph{Aalborg University, Denmark}%\\

    \bigskip

    \noindent
    \textbf{Date of defense:}\\
    February 23, 2022
    

  \end{flushleft}
\end{titlingpage}


% frontmatter.tex contains the abstract, preface, acknowledgments, and
% the commands to produce the table of contents, list of tables, etc.

%% Front matter for for FnT Optimization
%% (c) 2013 Neal Parikh (Stanford University)
%% 
%% Based on earlier NOW style files by S. Kumar,
%% J. Sebastian, T. Hoekwater, and B.V. Elvenkind
%%
%% Version 1.01 2014-03-19 Dimitri P. Bertsekas (MIT) 
%% added to EB.
%% Modified by M. Casey

\RequirePackage{lastpage}
\RequirePackage{multicol}

% base journal information definitions
\def\doi#1{\def\@doi{#1}}
\def\copyrightowner#1{\def\@copyrightowner{#1}}
\def\firstpage#1{\def\@firstpage{#1}}
\def\isbn#1{\def\@isbn{#1}}
\def\issue#1{\def\@issue{#1}}
\def\journalinfofont{\fontsize{8}{10}\selectfont\rmfamily}
\def\journallibraryinfo#1{\def\@libraryinfo{#1}}
\def\journaltitle#1{\def\@journaltitle{#1}}
\def\journaltitleprefix#1{\def\@journaltitleprefix{#1}}
\def\@lastpage{\pageref{LastPage}}
\def\lastpage#1{\def\@lastpage{#1}}
\def\copyrightyear#1{\def\@copyrightyear{#1}}
\edef\@copyrightyear{\the\year}
\def\pubyear#1{\def\@pubyear{#1}}
\edef\@pubyear{\the\year}
\def\volume#1{\def\@volume{#1}}

\def\journaltopicintro#1{\def\@journaltopicintro{#1}}
\def\journaltopics#1{\def\@journaltopics{#1}}

% editors
\newcount\editorcount
\def\@editors{}
\def\journaleditor#1{%
  \advance\editorcount1 
  \ifx\empty\@editors
    \def\@editors{#1}%
  \else
    \expandafter\def\expandafter\@editors\expandafter{\@editors\\[0.3em]#1}%
  \fi
}

% editors in chief
\newcount\chiefeditorcount
\def\@chiefeditors{}
\def\journalchiefeditor#1{%
  \advance\chiefeditorcount1 
  \ifx\empty\@chiefeditors
    \def\@chiefeditors{#1}%
  \else
    \expandafter\def\expandafter\@chiefeditors\expandafter{\@chiefeditors\\[0.3em]#1}%
  \fi
}

%% title page

\renewcommand\maketitle{\begin{titlepage}%

\if@openright % -- for 'book' version
\parindent \z@

% title page
\thispagestyle{empty}
\vspace*{\fill}
\begin{flushright}
\begin{minipage}{0.8\textwidth}
\begin{flushright}
\renewcommand{\baselinestretch}{1.1}
\sffamily\bfseries\huge 
\@title
\end{flushright}
\end{minipage}
\end{flushright}
\vspace*{\fill}

% title+author page
\cleardoublepage
\thispagestyle{empty}

\begin{flushright}
{\renewcommand{\baselinestretch}{1.2}
\sffamily\Huge\bfseries \@title\par}
\vspace{3ex}
\noindent\rule{\textwidth}{2pt}
\vspace{1ex}\par

\begingroup
\def\and{%
  \end{tabular}%
  \par\vspace*{2ex}
  \begin{tabular}[t]{r@{}}
  \bfseries}

\Large\sffamily
\begin{tabular}[t]{r@{}}
    \bfseries
    \@author
\end{tabular}
\endgroup
\vfill
\includegraphics[width=53bp]{now_logo}\\
\includegraphics[width=81bp]{essence_logo}\\
Boston --- Delft
\end{flushright}

% publisher page
\newpage
\thispagestyle{empty}
\begingroup

\parskip6pt

{\bfseries\sffamily\Large \@journaltitleprefix\ \@journaltitle}
\footnotesize

\vfill

{\itshape Published, sold and distributed by:}\\
now Publishers Inc.\\
PO Box 1024\\
Hanover, MA\ 02339\\
United States\\
Tel.\ +1-781-985-4510\\
www.nowpublishers.com\\
sales@nowpublishers.com

{\itshape Outside North America:}\\
now Publishers Inc.\\
PO Box 179\\
2600 AD Delft\\
The Netherlands\\
Tel.\ +31-6-51115274

The preferred citation for this publication is

{\raggedright
\@copyrightowner. \emph{\@title}.  \@journaltitleprefix\
\@journaltitle, vol.~\@volume, no.~\@issue, pp.~\@firstpage--\@lastpage,
\@pubyear.\par}

{\itshape This Foundations and Trends\textsuperscript{\textregistered} issue
was typeset in \LaTeX\ using a class file designed by Neal Parikh. Printed on
acid-free paper.}

ISBN: \@isbn\\ \copyright\ \@copyrightyear\ \@copyrightowner

\vfill

\fontsize{8}{9}\selectfont
All rights reserved. No part of this publication may be
reproduced, stored in a retrieval system, or transmitted in any form
or by any means, mechanical, photocopying, recording or otherwise,
without prior written permission of the publishers.

Photocopying. In the USA: This journal is registered at the Copyright
Clearance Center, Inc., 222 Rosewood Drive, Danvers, MA 01923. 
Authorization to photocopy items for
internal or personal use, or the internal or personal use of specific
clients, is granted by now Publishers Inc for users registered with 
the Copyright Clearance Center (CCC).
The `services' for users can be found on the internet at:
www.copyright.com

For those organizations that have been granted a photocopy license, a
separate system of payment has been arranged. Authorization does not
extend to other kinds of copying, such as that for general
distribution, for advertising or promotional purposes, for creating
new collective works, or for resale.  In the rest of the world:
Permission to photocopy must be obtained from the copyright
owner. Please apply to now Publishers Inc., PO Box 1024, Hanover, MA
02339, USA; Tel. +1 781 871 0245; www.nowpublishers.com;
sales@nowpublishers.com

now Publishers Inc. has an exclusive license to publish this
material worldwide. Permission to use this content must be obtained from
the copyright license holder. Please apply to now Publishers, PO Box 179, 2600
AD Delft, The Netherlands, www.nowpublishers.com; e-mail:
sales@nowpublishers.com\par
\endgroup

% editorial board
\newpage
\thispagestyle{empty}
\begin{center}
\sffamily\LARGE
{\bfseries \@journaltitleprefix\ \@journaltitle}\\
{Volume \@volume, Issue \@issue, \@pubyear}\\
{\bfseries Editorial Board}
\end{center}

\vspace{12pt}

\ifnum\chiefeditorcount>1
{\sffamily\bfseries Editors-in-Chief}
{\footnotesize
\begin{multicols}{2}
\@chiefeditors
\end{multicols}}
\else
{\sffamily\bfseries Editor-in-Chief}\par
\vspace*{10pt}
\begin{footnotesize}
\begin{tabular}[t]{@{}l}
\@chiefeditors \\
\end{tabular}
\end{footnotesize}
\vspace*{10pt}
\fi

{\sffamily \bfseries Editors}
{\footnotesize
\begin{multicols}{2}
\@editors
\end{multicols}}%

% editorial scope
\newpage % page vi -- blank
\thispagestyle{empty}
\begin{center}
{\sffamily\bfseries\LARGE Editorial Scope}
\end{center}

\vspace{18pt}

{\bfseries\large\sffamily Topics}\\

\@journaltopicintro

\fontsize{10}{13}\selectfont
{\raggedright \@journaltopics}

{\bfseries\large\sffamily Information for Librarians}\\

\@libraryinfo\par

\else\fi % -- no book front matter for "journal" version

% -- end book front matter

% title+author page
\newpage
\thispagestyle{empty}
\hbox to \textwidth{%
    \vbox{\raggedright
    \hsize=.6\textwidth \journalinfofont
{\hangindent=3.5ex
\hangafter=1
\@journaltitleprefix\ \@journaltitle\par}%
Vol.~\@volume, No.~\@issue\ (\@pubyear) 
\@firstpage--\@lastpage\\
{\hangindent=3.5ex
\hangafter=1
\copyright\ \@copyrightyear\ \@copyrightowner\par}%
DOI: \@doi}\hss
\vbox{\hsize=.4\textwidth 
    \raggedleft
    \includegraphics[width=53bp]{now_logo}\\
    \includegraphics[width=81bp]{essence_logo}\\
    \par
}}\par
\let\footnoterule\relax
\let\footnote\thanks
\null\vfil
\vskip 60\p@
\begin{center}%
{\LARGE \sffamily \bfseries \@title \par}%
\vskip 3em%
{
\large
\lineskip .75em%
     \lineskip .75em%
      \begin{tabular}[t]{c}%
        \@author
      \end{tabular}\par}%
\end{center}\par
\@thanks
\vfill
\end{titlepage}
\setcounter{footnote}{0}%
}

%% defaults and settings for this FnT

\def\@copyrightowner{now Publishers Inc.}
\journaltitleprefix{Foundations and Trends\textsuperscript{\textregistered} in}
\volume{XX}
\issue{XX}
\firstpage{1}
\isbn{XXX-X-XXXXX-XXX-X}
\doi{10.1561/XXXXXXXXXX}

\journaltitle{Optimization}

\journalchiefeditor{\textbf{Stephen Boyd} \\
Stanford University \\
United States}

\journalchiefeditor{\textbf{Yinyu Ye} \\
Stanford University \\
United States}

\journaleditor{Dimitris Bertsimas \\ \emph{Massachusetts Institute of Technology}}
\journaleditor{Dimitri P. Bertsekas \\ \emph{Massachusetts Institute of Technology}}
\journaleditor{John R. Birge \\ \emph{University of Chicago}}
\journaleditor{Robert E. Bixby \\ \emph{Rice University}}
\journaleditor{Emmanuel Cand\`{e}s \\ \emph{Stanford University}}
\journaleditor{David Donoho \\ \emph{Stanford University}}
\journaleditor{Laurent El Ghaoui \\ \emph{University of California, Berkeley}}
\journaleditor{Donald Goldfarb \\ \emph{Columbia University}}
\journaleditor{Michael I. Jordan \\ \emph{University of California, Berkeley}}
\journaleditor{Zhi-Quan (Tom) Luo \\ \emph{University of Minnesota, Twin Cites}}
\journaleditor{George L. Nemhauser \\ \emph{Georgia Institute of Technology}}
\journaleditor{Arkadi Nemirovski \\ \emph{Georgia Institute of Technology}}
\journaleditor{Yurii Nesterov \\ \emph{UC Louvain}}
\journaleditor{Jorge Nocedal \\ \emph{Northwestern University}}
\journaleditor{Pablo A. Parrilo \\ \emph{Massachusetts Institute of Technology}}
\journaleditor{Boris T. Polyak \\ \emph{Institute for Control Science, Moscow}}
\journaleditor{Tam\'{a}s Terlaky \\ \emph{Lehigh University}}
\journaleditor{Michael J. Todd \\ \emph{Cornell University}}
\journaleditor{Kim-Chuan Toh \\ \emph{National University of Singapore}}
\journaleditor{John N. Tsitsiklis \\ \emph{Massachusetts Institute of Technology}}
\journaleditor{Lieven Vandenberghe \\ \emph{University of California, Los Angeles}}
\journaleditor{Robert J. Vanderbei \\ \emph{Princeton University}}
\journaleditor{Stephen J. Wright \\ \emph{University of Wisconsin}}

\journaltopicintro{\@journaltitleprefix\ \@journaltitle\ publishes
survey and tutorial articles on methods for and applications of mathematical optimization, including the following topics:}

\journaltopics{
\begin{itemize}
\item{Algorithm design, analysis, and implementation (especially on
modern computing platforms)}
\item{Models and modeling systems} 
\item{New optimization formulations for practical problems}
\item{Applications of optimization in:}
	\begin{itemize}
	\item Machine learning
	\item Statistics
	\item Data analysis
	\item Signal and image processing
	\item Computational economics and finance
	\item Engineering design
	\item Scheduling and resource allocation
	\item and other areas
	\end{itemize}
\end{itemize}
}

\journallibraryinfo{%
  Foundations and Trends\textsuperscript{\textregistered} in \@journaltitle, \@pubyear,
  Volume~\@volume, 4 issues. ISSN paper version 2167-3888.
  ISSN online version 2167-3918.
  Also available as a combined paper and online
  subscription.
}

\endinput


% The main matter of the thesis contains the actual content, separated
% into chapters.

\mainmatter


\section{Introduction}
\label{sec:Introduction}


The goal in top-$\size$ recommendation is to recommend to each
consumer a small set of $\size$ items from a large collection of
items~\cite{cremonesi2010performance}.  For example, Netflix may want
to recommend $\size$ appealing movies to each consumer.  Collaborative
Filtering (CF)~\cite{herlocker2002empirical,lee2012comparative} is a
common top-$\size$ recommendation method.  CF infers user interests by
analyzing partially observed user-item interaction data, such as user
ratings on movies or historical purchase
logs~\cite{kanagal2012supercharging}. The main assumption in CF is that
users with similar interaction patterns have similar interests.


Standard CF methods for top-$\size$ recommendation focus on making  suggestions  that accurately reflect the user's preference history. However, as  observed in previous work,  CF recommendations are generally biased toward  popular items, leading to a rich get richer effect~\cite{vargas2014improving,steck2011item}.  The major reasons for this are \textit{popularity bias} and \textit{sparsity} of CF interaction data (detailed in Section~\ref{sec:related-work}). In a nutshell, to maintain  accuracy, recommendations are generated from the dense regions of the data,  where the popular items lie.  

However,  accurately suggesting popular items, may not be satisfactory for the consumers. For example, in Netflix, an accuracy-focused movie recommender may recommend ``Star Wars: The Force Awakens'' to users who have seen ``Star Wars: Rogue One''.  But, those users are probably already aware of ``The Force Awakens''. Considering additional factors, such as novelty of recommendations,  can lead to more effective suggestions~\cite{cremonesi2010performance,Castells2015,zhang2008avoiding,ziegler2005improving,zhang2012auralist}. 
%Second, accuracy-focused models typically achieve a   overall item-space coverage across their recommendations,  whereas high item-space coverage helps providers of the items increase revenue
%, users satisfaction since they are  likely already aware of or can find these items on their own.  

Focusing on popular items also adversely affects the satisfaction of  the providers of the items. This is because  accuracy-focused models typically achieve a  low overall item space coverage across their recommendations, whereas   high item space coverage helps providers of the items increase their revenue~\cite{vargas2014improving,Castells2015,adomavicius2011maximizing,anderson2006thelongtail, yin2012challenging,adomavicius2012improving}.
%accuracy-focused models typically achieve a

In contrast to the relatively small number of popular items, there are copious  {\it long-tail\/} items that have fewer observations (e.g., ratings) available. More precisely,  using the Pareto  principle (i.e.,~the $80/20$ rule),  long-tail items can be defined as items that generate the lower $20\%$ of observations~\cite{yin2012challenging}. Experimentally we found that these items correspond to almost $85\%$ of the items in several datasets (Sections~\ref{sec:Notation} and \ref{sec:Experiments}). %Table~\ref{tab:DatasetStatsticsSmall})


As previously shown, one way to improve the novelty of top-$\size$ sets is to recommend interesting long-tail items~\cite{cremonesi2010performance,ge2010beyond}.  The intuition  is that since they have fewer observations available,  they are more likely to be unseen~\cite{Kaminskas:2016:DSN:3028254.2926720}.  
 %For example, in online commerce,  newly added items are long-tail items that are yet to be discovered.  
Moreover, long-tail item promotion also results in higher overall coverage of the item space%, which increases profits for providers of the items
~\cite{vargas2014improving,Castells2015,zhang2008avoiding,zhang2012auralist,adomavicius2011maximizing,anderson2006thelongtail,yin2012challenging,jambor2010optimizing}. Because long-tail promotion reduces accuracy~\cite{steck2011item}, there are trade-offs to be explored.


%original submitted to ICDE
%This work studies three aspects of top-$\size$ recommendation: accuracy, novelty, and item-space coverage, and examines their trade-offs. In most previous work, predictions of a base recommendation system are re-ranked to handle their trade-offs~\cite{adomavicius2012improving,jambor2010optimizing,zhang2013personalize,wang2009portfolio}. Due to performance considerations, however, these techniques are not customized per user. For example,  parameters that balance the trade-off between novelty and accuracy are cross-validated at a global level.  This can be detrimental since users have varying preferences for  objectives such as long-tail novelty. We explore how to  automatically infer  user  preference for long-tail novelty, and how to leverage  it to correct  the popularity bias in standard recommender models. Our work does not rely on any additional contextual data, although such data, if available, can help promote newly-added long-tail items~\cite{agarwal2009regression,Saveski:2014:ICR:2645710.2645751}.

This work studies three aspects of top-$\size$ recommendation: accuracy, novelty, and item space coverage, and examines their trade-offs. In most previous work, predictions of a base recommendation algorithm are \textit{re-ranked} to handle these trade-offs~\cite{adomavicius2012improving,jambor2010optimizing,zhang2013personalize,wang2009portfolio}. The re-ranking models are computationally efficient but suffer from two drawbacks. First, due to performance considerations,  parameters that balance the trade-off between novelty and accuracy  are not customized per user. Instead they are cross-validated at a global level.  This can be detrimental since users have varying preferences for  objectives such as long-tail novelty. Second,  the re-ranking methods are often limited to a specific base recommender  that may be sensitive to dataset density. 
As a result, the datasets are pruned and the problem is studied in dense settings~\cite{adomavicius2012improving,ho2014likes}; but real world  scenarios are often sparse~\cite{kanagal2012supercharging,liu2017experimental}.   
% Because  dataset density can impact the performance of most base recommenders (like R-SVD), which in turn affects the performance of the re-ranking model, 

\iffalse
We address these limitations by directly inferring  user  preference for long-tail novelty  from interaction data.  This  allows us to customize the re-ranking  per user, and design a \textit{generic} framework, which resolves the second problem. In particular, since the long-tail novelty preferences are estimated independently of any base  recommender model, we can  plug-in an appropriate base recommender w.r.t. the dataset sparsity.% including ones that are more suitable for sparse settings.  

Modelling  user  preference for  long-tail novelty using only item popularity statistics, e.g., the average popularity of rated items as in~\cite{jugovac2017efficient}, disregards additional information like whether the user found the item interesting and the long-tail preferences of other users  of the items. \iffalse To incorporate them, we introduce the notion of  \emph{item long-tail importance}. Both  user long-tail preferences and item long-tail importance are dependent:  a user has high preference for discovering long-tail items if she is interested in important long-tail items, and an item that is associated with many of these kinds of users is likely to be more important.  We propose a joint optimization framework to directly learn,  from interaction data, both the users' long-tail preferences and the  items' long-tail importance. \fi
We propose an optimization approach that  incorporates  this information and  directly learns,  from interaction data, the users' long-tail novelty preferences.

Next, we use these learned preferences  to design a  top-$\size$ recommendation framework thats is generic, and provides customized balance between accuracy, novelty, and coverage. We refer to it as framework as GANC.  Using GANC, we design a novel algorithm, {\it Ordered Sampling-based Locally Greedy (OSLG)\/}, that relies on the learned long-tail novelty preferences  to scalably correct for popularity bias. Our work does not rely on any additional contextual data, although such data, if available, can help promote newly-added long-tail items~\cite{agarwal2009regression,Saveski:2014:ICR:2645710.2645751}. In summary:
\fi

We address the first limitation by directly inferring  user  preference for long-tail novelty  from interaction data.   Estimating these  preferences  using only item popularity statistics, e.g., the average popularity of rated items as in~\cite{jugovac2017efficient}, disregards additional information, like whether the user found the item interesting or the long-tail preferences of other users  of the items. We propose an approach that  incorporates  this information and  learns the users' long-tail novelty preferences from interaction data.

This approach allows us to customize the re-ranking  per user, and  design a \textit{generic} re-ranking framework, which resolves the second limitation of prior work. In particular, since the long-tail novelty preferences are estimated independently of any base recommender, we can  plug-in an appropriate one w.r.t. different factors, such as the dataset sparsity.

Our top-$\size$ recommendation framework, \textbf{GANC}, is \textbf{G}eneric, and provides customized balance between \textbf{A}ccuracy, \textbf{N}ovelty, and \textbf{C}overage. % Moreover, based on the learned long-tail novelty preferences, we also design a novel algorithm, {\it Ordered Sampling-based Locally Greedy (OSLG)\/}, that relies on the learned long-tail novelty preferences  to scalably correct for popularity bias. 
Our work does not rely on any additional contextual data, although such data, if available, can help promote newly-added long-tail items~\cite{agarwal2009regression,Saveski:2014:ICR:2645710.2645751}. In summary:

%Consider  the following toy example:
\vspace{-0.2cm}
\begin{table}[htb]
\centering
\scriptsize
%\small
\begin{tabular}{ccccccc} 
%\toprule
%&\multirow{2}{*}{}&\multicolumn{7}{c}{Ratings}\\
& & \cellcolor{blue!35}$w_1$ &\cellcolor{blue!18} $w_2$ & $\dots$ &\cellcolor{blue!8} $w_{89}$  &\cellcolor{blue!8} $w_{99}$   
\\
&   &$i_1$&$i_2$&$\dots$&$i_{89}$&$i_{90}$\\ 
\cmidrule(r){3-7} 	 
%\midrule
\cellcolor{red!35}$\theta_1$  &$u_1 $   &5 &   & $\dots$ &  &   \\
\cellcolor{red!28}$\theta_2$  &$u_2$     &5 &    & $\dots$ &  &  \\
 $\theta_3=?$  &$\bf u_3$  &5 &  &   $\dots$ &  &  \\
\cellcolor{red!10}$\theta_4$ & $u_4$  &  &5   & $\dots$ & &\\ 
\cellcolor{red!10}$\theta_5$ & $u_5$  &  & 5  & $\dots$ & &\\ 
$\theta_6=?$  & $\bf u_6$ & &5  &      $\dots$& &  \\ 
 & & $\hdots$  &$\hdots$   &$\hdots$   &$\hdots$   &$\hdots$  \\
%\midrule 
\cmidrule(r){3-7} 	 
\multicolumn{2}{c}{item pop.}  & 3  & 3  & $\dots$ &50&60\\  
%\bottomrule
%$ f_i$    &3  &3  &1  &3  &1  &2  \\  \hline
\end{tabular}
%#.
\caption{Simplified user-item interaction data. The user long-tail novelty preference ($\theta_u$), item long-tail importance weight ($w_i$) are highlighted. Darker colors indicate larger values. } \label{tab:example}
\end{table} 
\vspace{-0.2cm}
\begin{example}  
In Table~\ref{tab:example}, we are interested in estimating $\theta_3$ and $\theta_6$,  the long-tail preference of users $u_3$ and $u_6$ who have each rated a single movie. Additional ratings for other users  are not included here.  Considering only rating information, we observe $i_1$ and $i_2$ are  equally popular $|\mathcal{U}_{i_1}^{\trainset}| = |\mathcal{U}_{i_2}^{\trainset}|=3$, and $r_{31}=5$ and $r_{62}=5$. Using Eq.~\ref{eq:tfidf-risk}  we have $\theta_3 = \theta_6$. However, if we were given the long-tail preferences of the each item's user set, specifically that $u_1$ and $u_2$ have high long-tail preference (darker red), while $u_4$ and $u_5$ have lower long-tail preference (lighter red), we could conclude $i_1$ is a more important long-tail item compared to $i_2$ (indicated by a darker blue shade for $w_1$), and we expect  $\theta_3 \geq \theta_6$.

% On the other hand, if we knew that $u_4$ and $u_5$ have lower long-tail preference, we could conclude $i_2$ is a  less significant long-tail item. Therefore, However, if we  consider the long-tail preferences of other users, we may reason differently.    We need another variable $w_i$ which captures this information. 
%we would conclude that $u_3$ has higher long-tail preference compared to $u_6$, since the users $i_1$ is a more prominent long-tail item. 

% Relying only  on item popularity information, we would  conclude   $u_3$ and $u_6$ have equal long-tail preference, since $i_1$ and $i_2$ are  equally popular. However, considering  the second column,  long-tail preference of users,  long-tail importance for each item,  which captures the long-tail preference of its users. Since  that  both users of $i_1$ have high long-tail preference while  the users of $i_2$ have lower preference,  we may conclude $i_1$ is a more important long-tail item compared to $i_2$. Therefore, $u_3$'s long-tail preference should be at least as large as $u_6$'s preference. Specifically, consider two  items $i_1$ and $i_2$, with the following rating data: $i_1=\{u_1:5, u_2:5, u_3:5 \}$, $i_2=\{u_4:5, u_5:5, u_6:5\}$.  

%Table~\ref{tab:example} shows  simplified rating data. We want an estimate of the long-tail preference of $u_3$ and $u_6$, who have each  rated a single movie.  Relying only  on movie popularity information, we would  conclude   $u_3$ and $u_6$ have similar long-tail preference, since $m_1$ and $m_2$ are  equally popular. However, considering the long-tail preferences of other users of those movies, we may reason differently: since $u_1$ and $u_2$ have high long-tail preference, and $u_4$ and $u_5$ have low long-tail preference, $m_1$ is a more prominent long-tail item compared to $m_2$. Therefore, it is likely that $u_3$ has higher long-tail preference compared to $u_6$.considering the long-tail preferences of other users of those movies, we may reason differently.  For example, 
\label{ex:running}
\end{example}



%------------------------------

\iffalse
\begin{example}
Table~\ref{tab:example} shows rating data for a simplified system. %Note the user-item interaction matrix is sparse.
For this example, we define popular movies as those that have received  three or more ratings; $\{m_1, m_2, m_4\}$ are popular and  $\{m_3, m_5, m_6\}$ are niche movies. We observe $u_1$ and $u_3$  have rated relatively popular movies (risk-averse) while $u_2$ and $u_4$ have rated niche movies (risk-loving). 
\label{ex:running}
\end{example}

\begin{table}[htb]
\centering
\scriptsize
\begin{tabular}{ccccccc} 
\toprule
			&$m_1$ &$m_2$   &$m_3$    &$m_4$   &$m_5$ &$m_6$  \\ \hline 
$u_1 $ &5  &4  & - &-  &-  &-   \\
$u_2$  &-  &-  &-  &-  &5  &5   \\
$u_3$  &-  &4  &-  &5  &-  &-   \\
$u_4$  &-  &-  &3  &-  &-  &4   \\ 
$u_5$  &5  &-  &-  &3  &-  &-   \\ 
$u_6$  &4  &2  &-  &4  &-  &-   \\ 
\bottomrule
%$ f_i$    &3  &3  &1  &3  &1  &2  \\  \hline
\end{tabular}
\caption{User-Movie rating data} \label{tab:example}
\end{table}

It is essential to consider consumer characteristics in designing recommender systems so that they promote long-tail items to the right group of users and spread demand evenly between hit and niche items.  

\fi





%------------------------------
\iffalse
\begin{table}[htb]
\centering
\scriptsize
\begin{tabular}{ccccccc} 
\toprule
			&$m_1$ &$m_2$   &$m_3$    &$m_4$   &$m_5$ &$m_6$  \\ \hline 
$u_1 $ &\textbf{5}  & \textbf{4}  &\textcolor{gray}{ 1.2} &-  &-  &-   \\
$u_2$  &-  &-  &-  &-  & \textbf{5}  &\textbf{5}   \\
$u_3$  &-  &\textbf{4}  &-  &\textbf{5}  &-  &-   \\
$u_4$  &-  &-  &\textbf{3}  &-  &-  &\textbf{4}   \\ 
$u_5$  &\textbf{5}  &-  &-  &\textbf{3}  &-  &-   \\ 
$u_6$  &\textbf{4}  &\textbf{2}  &-  &\textbf{4}  &-  &-   \\ 
\bottomrule
%$ f_i$    &3  &3  &1  &3  &1  &2  \\  \hline
\end{tabular}
\caption{User-Movie rating data} \label{tab:example}
\end{table}
% $\mathcal{P}^1= \{ \mathcal{P}_1^1 \{i_1,i_2,i_3\}, \mathcal{P}_2^1:\{i_2,i_3,i_5\}  \}$
 %$\mathcal{P}^2= \{ \mathcal{P}_1^2: \{i_1,i_2,i_3\}, \mathcal{P}_2^2:\{i_2,i_5,i_6\}  \}$
 %$\mathcal{P}^3= \{ \mathcal{P}_1^3: \{i_7,i_8,i_9\}, \mathcal{P}_2^3:\{i_{10},i_{11},i_{12}\}  \}$
\begin{table}[htb]
\centering
\tiny
\begin{tabular}{ccc} 
\toprule
		&$u_1$&$u_2$  \\ \hline 
$\mathcal{P}^1 $ & $\{i_1,i_2,i_3\}$ & $\{i_2,i_3,i_5\} $ \\
$\mathcal{P}^2$ & $\{i_1,i_2,i_3\}$ & $\{i_2,i_5,i_6\} $ \\
$\mathcal{P}^3$ & $\{i_7,i_8,i_9\}$ & $\{i_{10},i_{11},i_{12} \}$ \\
\bottomrule
%$ f_i$    &3  &3  &1  &3  &1  &2  \\  \hline
\end{tabular}
\caption{Top-$\size$ allocations to users.} \label{tab:paretoExamples}
\end{table}
\fi


\iffalse
When considering long-tail items, it is important to consider consumers' willingness  to explore niche or unpopular items and their propensity towards similar items. In particular, they can be characterized by their  {\it risk degree\/} and {\it focusing degree\/}, respectively.  We compute these estimates  based on historical rating information. The following example further describes these notions in the context of movie rating data. 

\begin{example}  
Table~\ref{tab:example} shows rating data for a simplified system with $6$ users, $6$ movies, and $3$ genres. $m_i^{j}$ implies that movie $m_i$ belongs to genre $j$. Note the user-item interaction matrix is sparse. 
  For this setting, we define popular movies as those that have received  three or more ratings; $\{m_1, m_2, m_4\}$ are popular and  $\{m_3, m_5, m_6\}$ are niche movies. We now profile the users according to their risk and focusing degree. E.g., $u_1$ has rated relatively popular movies belonging to the same genre (risk-averse, high focusing degree); $u_2$ has rated niches movies in the same genre (risk-loving, high focusing degree); $u_3$ has rated popular movies in two different genres (risk-averse, low focusing degree), and $u_4$ has rated niches movies in two different genres (risk-loving, low focusing degree). 
\label{ex:running}
\end{example}
\begin{table}[htb]
\centering
\tiny
\begin{tabular}{ccccccc} 
\toprule
			&$m_1^{1}$ &$m_2^{1}$   &$m_3^{2}$    &$m_4^{3}$   &$m_5^{3}$ &$m_6^{3}$  \\ \hline 
$u_1 $ &5  &4  &-  &-  &-  &-   \\
$u_2$  &-  &-  &-  &-  &5  &5   \\
$u_3$  &-  &4  &-  &5  &-  &-   \\
$u_4$  &-  &-  &3  &-  &-  &4   \\ 
$u_5$  &5  &-  &-  &3  &-  &-   \\ 
$u_6$  &4  &2  &-  &4  &-  &-   \\ 
\bottomrule
%$ f_i$    &3  &3  &1  &3  &1  &2  \\  \hline
\end{tabular}
\caption{User-Movie rating data} \label{tab:example}
\end{table}
It is essential to consider these consumer characteristics in designing recommender systems so that they promote long-tail items to the right group of users and spread demand evenly between the hit and niche items.  
\fi
\iffalse
\begin{center}
\begin{figure*}[tp]
%\scalebox{0.5}{%
\resizebox{1\textwidth}{!}{%
%\small%\addtolength{\tabcolsep}{5pt}% below sums to 8
\begin{tabularx}{1.5\textwidth}{>{\hsize=2.5\hsize}X>{\hsize=2.5\hsize}X>{\hsize=0.5\hsize}X>{\hsize=0.5\hsize}X>{\hsize=0.5\hsize}X>{\hsize=0.5\hsize}X>{\hsize=0.5\hsize}X>{\hsize=0.5\hsize}X}
    \multirow{12}{*}{\includegraphics[scale=0.3]{codeForExample/popularity-movie.png}} & \multirow{12}{*}{\includegraphics[scale=0.3]{codeForExample/scatterplot.png}} & & & & & & \\
%   & &               &       &       &       &       &       \\
    & &\multicolumn{1}{l|}{}               &$m_1^{g1}$   	&$m_2^{g1}$    	&$m_3^{g2}$    &$m_4^{g2}$      &$m_5^{g3}$    \\ \cline{3-8}%\hline
    & &\multicolumn{1}{l|}{u1}          &5  &5  &-  &-   &-  \\
    & &\multicolumn{1}{l|}{u2}    		&-  &-  &4  &4  &5  \\
    & &\multicolumn{1}{l|}{u3}   			&1  &2  &1  &-  &-   \\
    & &\multicolumn{1}{l|}{u4}     		&1  &-  &-  &-  &-  \\
    & &               &       &       &       &       &       \\
    & &               &       &       &       &       &       \\
    & &               &       &       &       &       &       \\
    & &               &       &       &       &       &	\\
    \\
\end{tabularx}}
\caption{User-Movie interaction data a) Popularity-Movie histogram b)Movie genres/clusters c) User-Movie rating data} \label{fig:example}
\end{figure*}
\end{center}
\fi



%We propose a novel approach that allows us to  promote long-tail items in a targeted manner, thereby improving the novelty of top-$\size$ sets, the overall item-space coverage across recommendations, while maintaining reasonable levels of accuracy.

%Next, we integrate these learned preferences  in a generic  top-$\size$ recommendation framework to provide customized balance between accuracy and coverage.

%sequentially make recommendations, while adjusting its parameters with regard to the set of top-$\size$ recommendations made so far. However, since  sequential parameter updates  cause  scalability issues, we propose a sampling based algorithm. This variant of our framework, called {\it Ordered Sampling-based Locally Greedy (OSLG)\/},  allows us to  correct for the popularity bias in recommendations with regard to individual user long-tail preferences. 

%ICDE submission
%Our framework differs with  prior work in the following aspects:  unlike~\cite{adomavicius2011maximizing,adomavicius2012improving,zhang2013personalize,ho2014likes},  the long-tail preference personalization in our framework is learned rather than optimized using cross-validation or parameter tuning. In other words, our personalization method is independent of the underlying base  recommendation models.  Moreover, our framework is  generic. This enables us to  plug-in several base recommenders, and evaluate their  effectiveness without requiring  extensive tuning for the accuracy and coverage trade-off. 


%\vspace{-2.8pt}
\begin{itemize}

\item  We examine various measures for estimating user long-tail novelty preference in Section~\ref{sec:lt-pref} and formulate an optimization problem  to directly learn users' preferences for long-tail  items from interaction data in Section~\ref{sec:learning-lt-pref}. %In addition, we introduce several heuristics for measuring the user preference for less common items from historical rating data.% 

\item  We integrate the user preference estimates into GANC %, a generic re-ranking framework that provides customized balance between accuracy, novelty, and coverage 
(Section~\ref{sec:RiskbasedReranking}), and  introduce {\it Ordered Sampling-based Locally Greedy (OSLG)\/}, a scalable algorithm that relies  on user long-tail preferences to correct the popularity bias (Section~\ref{sec:optimizationAlgorithm}).
%We introduce OSLG, a scalable algorithm that relies  on user long-tail preferences to  maximize item space coverage \textcolor{red}{while maintaining acceptable levels of accuracy} (Section~\ref{sec:optimizationAlgorithm}).

\item   We conduct an extensive empirical study and evaluate performance from  accuracy, novelty, and coverage perspectives (Section~\ref{sec:Experiments}).  We use five  datasets with varying density and difficulty levels. %:  Netflix, MovieTweetings, and MovieLens (100K, 1M, 10M). 
  In contrast to most related work,  our evaluation considers realistic settings that include a large number of infrequent  items and users. %This enables us to study the impact of  data density on the performance trade-offs of several  state of the art top-$\size$ recommendation algorithms. %   %,  and use the all-items ranking protocol~\cite{steck2013evaluation,vargas2014improving}, where performance is measured using all items with train data. to evaluate the performance of several  state of the art top-$\size$ recommendation algorithms 
 
\item Our empirical results confirm that the performance of re-ranking models is impacted by the underlying   base recommender and the dataset density. Our generic approach enables us to easily incorporate a suitable base recommender to devise an effective solution for both dense and sparse settings. In dense settings, we use the same base recommender as existing re-ranking approaches, and we outperform them in accuracy and coverage metrics. For sparse settings, we plug-in a more suitable base recommender, and devise an effective solution that is competitive with existing top-$\size$ recommendation methods in accuracy and novelty. 

%Directly estimating the long-tail novelty preferences allows us to customize re-ranking per user, and  devise a generic framework.   
 
\end{itemize}

Section~\ref{sec:related-work} describes related work. Section~\ref{sec:conclusion} concludes.


% \documentclass[main.tex]{subfiles}

% \begin{document}

\chapter{Tangent categories}%
\label{ch:tangent_categories}

Tangent categories are an example of convergent evolution in mathematics, in which two unrelated lines of research with very different aims have arrived at a common endpoint, in this case the same formal setting for abstract differential geometry. The older line of research has its roots in differential geometry proper and, in particular, Weil's algebraic characterization of the tangent bundle of a smooth manifold (\cite{Weil1953}). Weil's work motivated Kock and Lawvere's development of synthetic differential geometry, presented in the book of the same name \cite{Lawvere1979,Kock2006} as well as the Weil functor formalism of \cite{Kolar1993}. The second, more recent line of research has its foundations in theoretical computer science following the publication of \emph{Linear Logic} by \cite{girard1987linear}. Ehrhard and Regnier noticed that some models of linear logic have a notion of the ``Taylor series approximation'' of a proof; this led to the development of differential linear logic by \cite{ehrhard2003differential}. Blute, Cockett, and Seely studied the categorical semantics of models of linear logic equipped with the derivative operation - that is, they identified those categories whose internal language are were models of differential linear logic - developing a categorical theory of differentiation in \cite{blute2006differential,Blute2009}.


Tangent categories arise naturally in each line of research: on the first path with the distillation of synthetic differential geometry into abstract tangent functors in \cite{Rosicky1984}, and more recently when Cockett and Cruttwell refined abstract tangent functors following their investigations into the manifold categories of cartesian differential restriction categories in \cite{MR2861119,Cockett2014}. In a sense, they are categories that axiomatize Weil's characterization of the tangent bundle as an endofunctor in a way that captures the combinatorics of higher-order derivatives when looking at a certain class of internal commutative monoids (\cite{cockett2011faa}), as will be made precise in Chapter \ref{chap:weil-nerve}. Tangent categories also pull Kock and Lawvere's synthetic differential geometry into the framework of enriched category theory, which is explored in Chapter \ref{ch:inf-nerve-and-realization}.

Instances of tangent structure abound throughout mathematics and computer science. For example,  many categories of geometric spaces have natural tangent structure, such as the category of \emph{convenient manifolds} (the category of manifolds modelled locally by \emph{convenient vector spaces} \cite{Kriegl1997}) and the category of schemes (see point (ii) in Example 2 of \cite{Garner2018}). An example from mathematical logic is the category of K\"{o}the sequence spaces( \cite{MR1934421}), and categorical models of the differential lambda-calculus (\cite{MR4037417}). More recently, tangent and differential categories have found applications in differentiable programming and machine learning (\cite{wilson2021reverse}), and to understanding Johnson and McCarthy's \emph{functor calculus} \cite{Bauer2016}.

This thesis studies differential geometric structures using the language of tangent categories, following the tradition of synthetic differential geometry. As such, this chapter will develop tangent categories with a focus on the category of smooth manifolds. Extending the study of these formal structures in the context of novel tangent categories is a significant endeavour and should be treated as a direction for future research. The first section introduces Cartesian differential categories as the categorical axiomatization of multivariable calculus. The second section introduces the category of smooth manifolds and two characterizations of its tangent bundle (kinematic versus operational),  while the third section identifies the structure of the kinematic tangent bundle that characterizes abstract tangent structures. The fourth section presents a pair of structures that allow for ``local-coordinate calculations'' in the tangent category of differential objects and connections. The final section introduces \emph{tangent submersions}. A submersion is a differentiable map between differentiable manifolds whose differential is everywhere surjective; as a preview of the work in Chapters \ref{ch:differential_bundles} and \ref{ch:involution-algebroids}, this section shows that in the category of smooth manifolds, a tangent submersion is precisely a submersion. Section \ref{sec:submersions} first appeared in \cite{MacAdam2021}, and is the only original work in this chapter.


\section{Differential calculus}%
\label{sec:differential-calculus}


As with most treatments of synthetic differential geometry, e.g. \cite{Kock2006}, it makes sense to begin with the differential calculus - in this case, an introduction to the categorical theory of differentiation. Categorical differentiation has recently gained quite a bit of attention due to its relationship with machine learning \cite{Cockett2019}, and applications to homotopy theory \cite{Bauer2016}. This section will just consider the basic structures introduced in \cite{Blute2009}, and the canonical example of a cartesian differential category (the category of finite-dimensional real vector spaces and smooth maps between them).
% Note: This section introduction seems reasonable based on the chapter introduction rewrite

\begin{definition}\label{def:clac}[Definition 1.2.1 \cite{Blute2009}]
    % NOTE: footnote was added to define cartesian category notation, addition & 0 notation was clarified
    A cartesian left additive category is a cartesian category\footnote{We use the standard notation where $1$ is the terminal object, $\x$ is product, and $\pi_i$ is the $i^{th}$ projection.} $\C$ so that:
    \begin{enumerate}[(i)]
        \item Each hom-set $\C(A,B)$ is a commutative monoid with addition $+_{AB}$ and zero map $0_{AB}:A \to B$ (the subscript $AB$ will be suppressed when the context is clear).
        \item The composition operation $\o$ preserves addition \textit{on the left}: \[(g+h) \o f= g \o f + h \o f\]
        \item Projection is an additive map (preserves addition): \[\pi_i \o (f + g) = (\pi_i \o g \o f) + (\pi_i \o g)\]
        Where $\pi_i$ denotes the projection from the $i^{th}$ component of a product or pullback.
j    \end{enumerate}
\end{definition}
There are various examples of cartesian left additive categories - they all fit the same pattern of a category where each object is equipped with a non-natural, but coherent, choice of linear structure:
\begin{example}\label{ex:clacs}
    ~\begin{enumerate}[(i)]
        \item Any category with biproducts is a cartesian left additive category where every map is additive. 
        \item The category of cartesian spaces $\mathsf{CartSp}$, whose objects are finite-dimensional real vector spaces and morphisms are smooth maps between them, is a cartesian left additive category."
        Clearly smooth maps from $A \to B$ are closed under addition, projection is an additive map, and $(g+h) \o f= g \o f + h \o f$.
        \item   The category of topological vector spaces and continuous morphisms is a cartesian left additive category. 
        % We note that each object has chosen commutative monoid structure satisfying the coherences in point (2) of Proposition \ref{prop:clac-defs}.     
    \end{enumerate}
\end{example}
In fact, cartesian left additive categories may equivalently be described as cartesian categories where each object has a coherent choice of commutative monoid structure.
\begin{proposition}[\cite{Blute2009}]%
    \label{prop:clac-defs}
    The following are equivalent:
    \begin{enumerate}[(i)]
        \item $\C$ is a cartesian left additive category
        \item $\C$ is a cartesian category so that each object has a chosen commutative monoid structure $(A,+_A,0_A)$ where the following coherence holds: \input{TikzDrawings/Ch1/clac-coh.tikz} (where $\tau = ((\pi_0\o \pi_0,\pi_0\o \pi_1),(\pi_1\o\pi_0, \pi_1\o\pi_1))$).
        \item There is a category with biproducts $\C^+$ and a bijective-on-objects subcategory inclusion $i: \C^+ \hookrightarrow\C$ that creates products.
    \end{enumerate}
\end{proposition}

Cartesian left additive categories provide an appropriate to define a differentiation operation. Recall that the usual derivative of a map $f: \R \to \R$ from elementary calculus can be written
\[
    \frac{\partial f}{\partial x}: \R \to \R. 
\]
More generally, for a map $f: \R^n \to \R$, one writes the \emph{Jacobian} of $f$ at $x$:
\[
    J[f]: \R^n \to (\R^n \multimap \R^m) :=
    \begin{bmatrix}
        \frac{\partial f_1}{\partial x_1} & \dots & \frac{\partial f_1}{\partial x_n} \\
        \vdots & \ddots & \vdots \\
        \frac{\partial f_1}{\partial x_1} & \dots & \frac{\partial f_1}{\partial x_n} 
    \end{bmatrix}
\]\pagenote{tidied notation, removed jargon}
The Jacobian, however, requires some notion of a ``matrix''\footnote{This would be called an \emph{internal hom} in the 
 categorical logic literature.} representing a linear map from $\mathbb{R}^n$ to $\mathbb{R}^m$---not every category that has a notion of differentiation supports that operation. Instead, the \emph{directional derivative}:
\[
    D[f](x,v) := \lim_{d \to 0} \frac{f(x + t\cdot v)}{t}
\]
gives an appropriately general notion of differentiation that extends to categories where the space of linear maps $A \to B$ is not representable by an object in the category \footnote{In the case of automatic differentiation, it is also worth noticing that computing the directional derivative of a map $\mathbb{R}^n \to \mathbb{R}^m$ has complexity $2\mathcal{O}(f)$, while forming the Jacobian has complexity $n\mathcal{O}(f)$, so the directional derivative is a more appropriate primitive for purely practical computational reasons (see Section 5 of \cite{hoffmann2016hitchhiker} for a discussion of the computational complexity of forward-mode automatic differentiation).}.\pagenote{Spelled out AD.} 
A cartesian differential category axiomatizes the directional derivative as a combinator on a cartesian left-additive category.


\begin{definition}\label{def:cdc}[Definition 2.1.1 in \cite{Blute2009}]
    A cartesian differential category is a cartesian left additive category equipped with a combinator (e.g. a function on hom-sets)\pagenote{
    I have added a precise reference to definition of a CDC, and included a brief definition for what a combinator is.}
    %I have never seen "combinators" formally defined...
    \[
        \infer{A \x A \xrightarrow[{D[f]}]{} B}{A \xrightarrow{f} B}
    \]
    satisfying the following axioms:
  \begin{enumerate}[{[CD.1]}]
      \item Additive:
        \[D[f+g] = D[f] + D[g] \hspace{1cm} D[0] = 0\]
      \item Additive in the second variable:
        \[D[f] \o (g, h+k) = D[f]\o (g,h) + D[f]\o (g,k) \hspace{1cm} D[f] \o (g,0) = 0\]
      \item Projection is linear:
        \[D[\pi_i] =  \pi_i \o \pi_1 \hspace{1cm} D[id] = \pi_1\]
      \item Pairing:
        \[D[(f,g)] = (D[f],D[g])\]
      \item Chain rule:
        \[D[g\o f] = D[g] \o (\pi_0, D[f])\]
      \item Linear in the second variable:
        \[D[D[f]] \o ((a,0),(0,d)) = D[f] \o (a,d)\]
      \item Symmetry of partial differentiation:
        \[D[D[f]] \o ((a,b),(c,d)) = D[D[f]] \o ((a,c),(b,d))\]
  \end{enumerate}
\end{definition}
\begin{example}
    The category of cartesian spaces (Example \ref{ex:clacs}(ii)), $\mathsf{CartSp}$, is the canonical cartesian differential category. Let $f: \mathbb{R}^n \to \mathbb{R}^m $, and consider its Jacobian at $x \in \mathbb{R}^n , J[f](x) \in \mathbb{R}^{n \x m}$.  Define the differential combinator:
	    \[
	        D[f]\o (u,v) = J[f](v) \cdot u = \lim_{t \to 0} \frac{f(x + t\cdot v)}{t}.
	    \]\pagenote{Removed extra examples, since they distracted from the original point.}
	\item In \cite{Bauer2016}, the authors construct a cartesian differential category based on the Abelian functor calculus of \cite{MR1451606}.
	\item In \cite{wilson2021reverse}, the authors consider a cartesian differential category whose objects are $\mathbb{Z}_2$-modules to apply gradient-based methods to learn the parameters of of Boolean circuits. 
	\pagenote{
	We have added to 
	}
	% \pagenote{added some extra examples w/references.}
\end{example}
Every cartesian differential category comes with a notion of linearity.
This notion of linearity is strictly stronger than additivity - there do exist examples of non-linear additive maps.
\begin{definition}[Definition 2.2.1 of \cite{Blute2009}]\label{def:linear-map-in-cdc}
    A map $f:A \to B$ is \textit{linear} whenever $D[f] \o (0_{AB},id) = f$. 
    \pagenote{
        I have added a precise reference to the definition of a linear map, and the $0$ has now been defined in the definition of a cartesian left additive category.
    }
\end{definition}
We denote the category of linear maps in a cartesian differential category $\C$ as $\mathsf{Lin}(\C)$.
The category $\mathsf{Lin}(\C)$ will have biproducts, and will be a cartesian differential subcategory of $\C$.
\begin{lemma}[Corollary 2.2.3 in \cite{Blute2009}]\label{lem:collected-remarks-lin}\pagenote{
    I have added the reference to this lemma, this was originally missing.
}
    Let $\C$ be a cartesian differential category, and denote its category of linear maps as $\mathsf{Lin}(\C)$.
    \begin{enumerate}[(i)]
        \item Linear maps preserve addition.
        \item The category $\mathsf{Lin}(\C)$ is a bijective-on-objects subcategory of $\C$ with biproducts, and the inclusion $\mathsf{Lin}(\C) \hookrightarrow \C$ creates products. 
        \item Every category with biproducts is a cartesian differential category, where
        \[
            D[f] = f \o \pi_1,
        \]
        and this differential structure makes the inclusion $\mathsf{Lin}(\C) \hookrightarrow \C$ preserve the left additive structure and differential combinator (that is, it is a cartesian differential functor). 
    \end{enumerate}
\end{lemma}


\section{The category of smooth manifolds}%
\label{sec:smooth-manifolds}

% {\color{red}
%     This thesis applies abstract methods to structures arising in differential geometry using the tangent bundle.
%     It is important, then, to set a working definition of the category of smooth manifolds and their tangent bundle construction, along with the universal properties satisfied by this construction.
% }

Tangent categories axiomatize a more general structure than differential calculus, one in which spaces are only ``locally linear.''
The category of smooth manifolds gives the historically canonical example, and a good portion of this thesis relates to structures internal to that category,
so it seems worthwhile to set a working definition for that context.
We follow \cite{Tu2011}, and allow for disconnected components of a manifold to have different dimensions.
% The material in this section is classical and may be found in \cite{Kolar1993,Tu2011}.

\begin{definition}%
\label{def:smooth-manifold}[Definitions 5.5--5.7 in \cite{Tu2011}]
    A \emph{chart} on a topological space $M$ is pair $(U_i, \phi_i:U_i \hookrightarrow \R^n)$, where $U_i$ is an open subset $U_i \subseteq M$ and $\phi_i:U_i \to \R^n$ is a local homeomorphism.
    An \emph{atlas} is a collection of charts $\{ (U_i,\phi_i:U_i \to \R^n) | i \in I \}$ (where $n$ is fixed for each connected component of $M$) so that for each $i,j \in I$, the transition function $\psi_{i,j}$ that completes the diagram
    % https://q.uiver.app/?q=WzAsNCxbMCwxLCJVX3tpfSBcXGNhcCBVX2oiXSxbMSwxLCJVX2oiXSxbMCwwLCJcXHBoaV97aX1eey0xfShVX3tpfSBcXGNhcCBVX2opIl0sWzEsMCwiXFxSXm4iXSxbMiwwLCJcXHBoaV9pIl0sWzAsMSwiXFxzdWJzZXRlcSJdLFsxLDMsIlxccGhpX2oiXSxbMiwzLCJcXHBzaV97aSxqfSIsMCx7InN0eWxlIjp7ImJvZHkiOnsibmFtZSI6ImRhc2hlZCJ9fX1dXQ==
    \[\begin{tikzcd}
        {\phi_{i}^{-1}(U_{i} \cap U_j)} & {\R^n} \\
        {U_{i} \cap U_j} & {U_j}
        \arrow["{\phi_i}", from=1-1, to=2-1]
        \arrow["\subseteq", from=2-1, to=2-2]
        \arrow["{\phi_j}", from=2-2, to=1-2]
        \arrow["{\psi_{i,j}}", dashed, from=1-1, to=1-2]
    \end{tikzcd}\]
    is a smooth map.
    \begin{figure}
        \centering
        \input{TikzDrawings/Ch1/chart-diagram.tikz}     
        \caption{Overlapping charts in an atlas (Credit for this tex code belongs to user Cragfelt \url{https://tex.stackexchange.com/a/388493/101171}.)}
        \label{fig:overlapping-charts}
    \end{figure}
    A \emph{smooth manifold} is a topological space equipped with a (maximal\footnote{With respect to subset inclusion.}) atlas. \pagenote{This definition was originally incomplete, and it lacked a reference. I have added the missing details and included a reference.
    } A morphism of smooth manifolds is a topological map $f: M \to N$ that is locally smooth - for each chart pair of charts $(U_i, \phi_i)$ on $M$ and $(V_j, \theta_j)$, see Figure \ref{fig:overlapping-charts} for an illustration. The map $f$ is smooth whenever each map $f_{i,j}$ that completes the diagram is smooth 
    % https://q.uiver.app/?q=WzAsNCxbMCwwLCJVX2kiXSxbMCwxLCJNIl0sWzEsMSwiTiJdLFsxLDAsIlZfaiJdLFswLDFdLFszLDJdLFsxLDIsImYiXSxbMCwzLCJmX3tpLGp9IiwwLHsic3R5bGUiOnsiYm9keSI6eyJuYW1lIjoiZGFzaGVkIn19fV1d
    \[\begin{tikzcd}
        {U_i} & {V_j} \\
        M & N
        \arrow[from=1-1, to=2-1]
        \arrow[from=1-2, to=2-2]
        \arrow["f", from=2-1, to=2-2]
        \arrow["{f_{i,j}}", dashed, from=1-1, to=1-2]
    \end{tikzcd}\]

    The \emph{category of smooth manifolds}, $\mathsf{SMan}$, is the category of smooth manifolds and their morphisms.
\end{definition}
\begin{remark}
    In \Cref{ch:differential_bundles}, some results will implicitly use partition of unity arguments, which require that the underlying topological space for a manifold is \emph{Hausdorff} and has a countable basis (i.e. it is \emph{second-countable}). We will avoid any direct reference to these properties, and so we omit them from the definition of a smooth manifold.
\end{remark}

\begin{example}
    ~\begin{enumerate}[(i)]
        \item Each vector space $\R^n$, for $n\in \N$, has a canonical smooth manifold structure whose atlas is a single chart (the identity map $\mathbb{R}^n \to \mathbb{R}^n$).
        \item Most geometric shapes that do not have any singularities or sharp edges can be equipped with an atlas without any issue.
        For example, consider the circle:  $\{ (\cos(x), \sin(x)) : x \in [-\pi, \pi) \}$
        \[\input{TikzItDrawings/circle-plain.tikz}\]
        For any appropriately small $\epsilon>0$, there are two charts from $I_\epsilon = (-\epsilon, \pi+\epsilon)$; 
        \[
            \phi_0(t) = (\cos(t), \sin(t)), \hspace{0.25cm} \phi_1(t) = (\cos(t-\pi), \sin(t-\pi))
        \]
        making the circle a smooth manifold. \pagenote{
           I have added more details to the chart maps here.
        }
    \end{enumerate}
\end{example}
Category theory has not seen as many applications in differential geometry as it has in topology or algebra, likely because the category of smooth manifolds (Definition \ref{def:smooth-manifold}) is somewhat poorly behaved.
The two main reasons that the category of smooth manifolds is ``inconvenient'' are as follows:
\begin{itemize}
    \item The set of smooth maps between two manifolds $M, N$ fails, in general, to form a smooth manifold; thus, the category is not cartesian closed.
    \item The category of manifolds does not have quotients or arbitrary fibre products.
    \pagenote{
       As per Kristine's recommendation, we give more specific limits and colimits that fail to exist in the category of smooth manifolds.
    }
\end{itemize}
However, the category of smooth manifolds admits some limits, for example finite products.
\begin{proposition}[1.12 \cite{Kolar1993}]
    The category \emph{SMan} of smooth manifolds has finite products.
\end{proposition}
\begin{proof}
    Given two manifolds $M, N$, take the product of their underlying topological spaces together with the product charts
    \[
        (\phi_i \x \psi_j): U_i \x V_j \to \R^n \x \R^m.
    \]
\end{proof}
The category of smooth manifolds---using our definition where disconnected components of a manifold may have different dimensions---does have a class of (co)limits used throughout this paper, namely idempotent splittings.\pagenote{
   I have added the definition of an idempotent splitting.
}
\begin{definition}[\cite{MR850528}]
    An \emph{idempotent} is an endomorphism $e:E \to E$ so that $e \o e = e$. The \emph{splitting} of an idempotent is given by a pair of maps $e = s\o r$ so that $r \o s = id$. The existence of a splitting of an idempotent $e$ is equivalent to asking that the following pair of parallel arrows has a (co)equalizer:
    % https://q.uiver.app/?q=WzAsMixbMCwwLCJFIl0sWzEsMCwiRSJdLFswLDEsImUiLDAseyJvZmZzZXQiOi0xfV0sWzAsMSwiIiwyLHsib2Zmc2V0IjoxLCJsZXZlbCI6Miwic3R5bGUiOnsiaGVhZCI6eyJuYW1lIjoibm9uZSJ9fX1dXQ==
    \[\begin{tikzcd}
        E & E
        \arrow["e", shift left=1, from=1-1, to=1-2]
        \arrow[shift right=1, Rightarrow, no head, from=1-1, to=1-2]
    \end{tikzcd}\]
    The idempotent splitting $\C$ of a category (also known as the \emph{Cauchy completion} of the category), is the full subcategory of presheaves $[\C^{op}, \s]$ that are retracts of representable functors.
    Any functor into a category with idempotent splittings will factor through this inclusion of categories, so it is the \emph{free cocompletion} of $\C$ under idempotent splittings.
\end{definition}
\begin{proposition}[\cite{MR1003203}]
    The category of smooth manifolds is the idempotent splitting of the category whose objects are open subsets of Cartesian spaces and whose morphisms are smooth maps $f: U \to V$.
\end{proposition}
% A natural benefit to this result is that the category of smooth manifolds is closed to idempotent splittings.
An idempotent in the idempotent splitting of a category is also an idempotent in the base category, and thus admits a splitting.
\begin{corollary}
    The category of smooth manifolds is closed to idempotent splittings: for every map $e:M \to M$ so that $e = e \o e$ there exists a pair of maps $r:Q \to M, s:M \to Q$ so that $e = s\o r, r \o s = id$.\pagenote{
       I have added the coherences on $r,s$, so that $r \o s = id, s \o r = e$.    }
\end{corollary}

The main construction of interest on the category of smooth manifolds, for tangent categories at least, is the \emph{tangent bundle}. Given  a smooth map $f:M \to N$, restrict it to a morphism between coordinate patches, so it may be regarded as a map $f|_U: U \to V$, where $U \subseteq \R^m, V \subseteq \R^n$. This gives a local derivative operation (remembering that $\pi_i$ denotes the $i^{th}$ projection from a product $\prod^n A_i$)\footnote{
The wording here was originally muddled, it has since been corrected.
}
% Recall that the derivative of a map $f: U \to V$, for $U \subseteq \R^m, V \subset \R^n$ is given by:
\begin{gather*}
    \infer{(f|_U \o \pi_0, D[f|_U]):U \x \R^m \to V \x \R^n.}{ D[f|_U]: U \x \R^m \to \R^n,}
\end{gather*}
% By the definition of a smooth map, for each $m \in M$ there is a local homeomorphism $m \in U_m \hookrightarrow \R^m$, and $f(m) \in V_{f(m)} \hookrightarrow \R^n$. Thus there is a derivative map:
% \[
%     T(f|_m) = (f|_m \o \pi_0, D[f]): U \x \R^m \to V \x \R^m
% \]
The tangent bundle makes this construction global; that is, there is a functor $T:\mathsf{SMan \to SMan}$ giving an assignment \[T.M \xrightarrow{T.f} T.N\] that agrees with the local derivative on coordinate patches of $M, N$.

\begin{definition}[\cite{Kolar1993}]%
    \label{def:tang-vector}
    Write the algebra of smooth functions on a manifold as $C^\infty(M):=\mathsf{SMan}(M,\R)$.
    The set $\mathsf{SMan}(\R, M)/\cong$ of tangent vectors on a smooth manifold $M$ comprises the curves $\R \to M$ subject to the equivalence relation that for a pair of curves $\phi, \theta:\R \to M$, $\phi \cong \theta$ if and only if $\phi(0) = \theta(0)$ and for every $f \in C^\infty(M)$,
    \[
        \frac{\partial f \o \phi}{\partial x}(0) = \frac{\partial f \o \theta}{\partial x}(0).
        % D[f \o \phi]\o (0,id) = D[f \o \theta] \o (0,id))
    \] 
    The set $\mathsf{SMan}(\R,M)/\cong$ has a naturally determined smooth manifold structure which we call the \emph{tangent bundle} over $M$, $TM$.\pagenote{
    The original definition forgot to set the notation $TM$ for the tangent bundle of $M$.}
\end{definition}

\begin{example}
    ~\begin{enumerate}[(i)]
        \item For a vector space, the space of linear paths crossing through a point $v \in V$ is isomorphic to $V$, so $TV \cong V \x V$. 
        \item The tangent bundle above the circle is diffeomorphic to the cylinder. This is follows from the classical result that a tangent vector on the circle must be perpendicular to its position vector.
    \end{enumerate}
\end{example}

The tangent bundle lifts the ``local derivative'' into a globally defined construction, so the tangent bundle construction is functorial.\pagenote{
   This proof has been cleaned up, it was originally quite messy.
}
\begin{proposition}
    The tangent bundle is a product-preserving endofunctor on the category of smooth manifolds.
\end{proposition}
\begin{proof}
    Functoriality follows by showing that a morphism of smooth manifolds preserves the equivalence relation on curves that defines a tangent vector: 
    \[
        \forall g \in C^\infty(M),  \frac{\partial g \o \phi}{\partial x}(0) = \frac{\partial g \o \theta}{\partial x}(0)
    \]
    Note that if $f:N \to M \in C^\infty(N)$, so that $g \o f \in C^\infty(M)$, the chain rule ensures that
    \[
        \forall g \in C^\infty(N), %D[(f \o g) \o \phi] \cong   D[(f \o g) \o \theta]
        \frac{\partial f\o g \o \phi}{\partial x}(0) = \frac{\partial f\o g \o \theta}{\partial x}(0)
    \]
    % Given $\theta \cong \phi$, look at an open chart $U$ containing $\theta(0) = \phi(0)$, $(f|_U \o \theta)'(0) = (f|_U \o \phi)'(0)$ follows by the chain rule.

    To show that $T$ is product-preserving, it suffices to show that the equivalence classes of curves are stable under pairing. First, note that for any $M$ and $\phi \cong \theta:\R \to M$, 
    \[
        (\phi,id) \cong (\theta,id): \R \to M\x\R
    \] 
    Given a pair of curves $\theta_M, \psi_M:\R \to M$, where $\theta_M \cong \psi_M$ and similarly $\theta_N \cong \psi_N$ for $N$, this implies that
    \pagenote{Cleaned up proof per Kristine's comments.}
    \begin{gather*}
        f \o (\phi_M, \phi_N) \\
        = f \o (\phi_M \x id) \o (id, \phi_N) \\
        = f \o (\phi_M \x id) \o (id, \theta_N) \\
        = f \o (id, \theta_N) \o (\phi_M \x id) \\
        = f \o (id, \theta_N) \o (\theta_M \x id) \\
        = f \o (\theta_M \x \theta_N)
    \end{gather*}
    so $(\phi_M, \phi_N) \cong (\theta_M, \theta_N)$.
    % \[
    %     \left(\forall f_M \in C^\infty(M), \frac{\partial f_M \o \phi_M}{\partial x}(0) = \frac{\partial f_M \o \theta_M}{\partial x}(0) \right)
    %     %D[f_M \o \theta_M] = D[f_M \o \phi_M]\right)
    %     \wedge
    %     \left(\forall f_N \in C^\infty(N),\frac{\partial f_N \o \phi_N}{\partial x}(0) = \frac{\partial f_N \o \theta_N}{\partial x}(0) \right)
    %         %\forall f_N \in C^\infty(N), D[f_N \o \theta_N] = D[f_N \o \phi_N]\right)
    % \]
    % Suppose there exists a function $f \in C^\infty(M \x N)$ so that 
    % \[
    %     \frac{\partial f \o (\phi_M,\phi_N)}{\partial x}(0) = \frac{\partial f \o (\theta_M,\theta_N)}{\partial x}(0)
    %     % D[f \o (\theta_M, \theta_N)](0,(x,y)) \not= D[f \o (\phi_M, \phi_N)](0,(x,y))
    % \] 
    % But this would mean that $f \o (id, ) 
    % then for $x=0$, in particular, this means:
    % \[
    %     D[f \o (\theta_M, \theta_N)](0,(0,y)) \not= D[f \o (\phi_M, \phi_N)](0,(0,y))
    % \] 
    % This is the partial derivative of $f \o (\theta_M, \theta_N)$ in the second variable at $(0,0)$, which may be rewritten:
    % \[
    %     D[f \o (\theta_M\o 0, \theta_N)] \o (0,y) \not= D[f\o (\phi_M\o 0, \phi_N)] \o (0,y)
    % \]
    % so that:
    % \[
    %     D[f \o (m, id) \o \theta_N]\o (0,id) \not= D[f \o (m, id) \o \phi_N] \o (0,id)
    % \]
    % then, $f' := f \o (m,id) \in C^\infty(M)$ would violate the original hypothesis that $\theta_N \cong \phi_N$. Therefore, the equivalence relation is stable under pairing.
\end{proof}


The scalar action by $\R$ on tangent vectors and a partially defined addition additionally give the tangent bundle the structure of a fibered $\R$-module (that is, an $\R$-module in the slice category $\mathsf{SMan}/M$ whose objects are morphisms into $M$, $f:X \to M$, and morphisms are commuting triangles).
\begin{proposition}
    The tangent bundle over $M$ is an $\R$-module in $\mathsf{SMan}/M$, as follows:
    let $\gamma, \omega$ be tangent vectors on $M$, and define
    \begin{itemize}
        \item $p: TM \to M; p(\gamma) = \gamma(0)$.
        \item $0: M \to TM; 0(m) =  [r \mapsto m]$ (the constant map $\R \to M$ sending all $r\in \R$ to $m\in M$)
        \item $\cdot_p: TM \x \R \to TM; \gamma \cdot_p r = [x \mapsto \gamma(r \cdot x)]$
        \item $+: TM \ts{p}{p} TM \to TM := [\gamma],[\omega] \mapsto [\gamma + \omega]$ (where addition around $\gamma(0)=\omega(0)$ is defined using local coordinates). \pagenote{The notation in this proposition has been changed to be more clear, and it was clarified that addition is defined using local coordinate charts around points in $TM$.}
    \end{itemize}
\end{proposition}

The second derivative is involved in the more nuanced axioms for a cartesian differential category, namely linearity in the vector argument and the symmetry of mixed partial derivatives.  First, set $f|_U$ to be the restriction of $f:M \to N$ to a map between local coordinate patches $U \subseteq M, V \subseteq N$, and then define
\[
  f_0 = f|_U \o \pi_0 \o \pi_1, \hspace{0.25cm},
\]
and
\[
   f_1 = D[f] \o (\pi_0 \o \pi_0, \pi_1 \o \pi_0), \hspace{0.25cm}
   f_2 = D[f] \o (\pi_0 \o \pi_0, \pi_0 \o \pi_1) \hspace{0.25cm}.
\]
The axioms $[CDC.6],[CDC.7]$ then give:\pagenote{
    I have added the definition of the maps ahead of the diagrams to avoid any confusion, and made a more direct reference to the CDC axioms.
}
% (setting \[f_0 = f|_U\o\pi_0\o\pi_0, f_1 = D[f|_U]\o \pi_0, f_2 = D[f|_U]\o (\pi_0\o\pi_0, \pi_0 \o \pi_1)\] so the diagrams fit on a single page)
\begin{center}
    \input{TikzDrawings/Ch1/local-coord-lift-diagram.tikz}
    \input{TikzDrawings/Ch1/local-coord-flip-diagram.tikz}
\end{center}
% Similarly, a tangent vector in $T^3M$ is then:
% \begin{align*}
%     \phi \cong \theta :\R^3 \to M
%     \iff   &  \left(\phi(0,0,0) = \theta(0,0,0) \right) \\
%     \wedge &  \forall f \in C^\infty(M), \frac{\partial f \o \phi}{\partial x_i}(0,0,0) = \frac{\partial f \o \phi}{\partial x_i}(0,0,0), i = 0,1,2
% \end{align*}
% A tangent vector in $T^3M$ is then:
% \begin{align*}
%     \phi \cong \theta :\R^3 \to M
%     \iff   &  \left(\phi(0,0) = \theta(0,0)\right) \\
%     % \wedge &  \left(\forall f \in C^\infty(M), D[f \o \phi] \o (0, x) = D[f \o theta] \o (0,x) \right) \\
%     \wedge &  \left(\forall f \in C^\infty(M), D[D[f \o \phi]] \o ((0,x),(y,z)) = D[D[f \o theta]] \o ((0,x),(y,z)) \right) \\
% \end{align*}

% A tangent vector in $T^3M$ is a map $\R^3 \to M$ satisfying similar coherences.
The two natural transformations $\ell$ and $c$---the vertical lift and canonical flip---capture these coherences.
Locally, $\ell$ is the map inserting zeros into the second and third coordinates, while $c$ flips the second and third arguments, leading to the coherences established in the next proposition.
To capture these coherences on the tangent bundle, first note that a tangent vector on $TM$ is equivalent to an equivalence class of surfaces on $M$,\pagenote{I have clarified the equivalence relation that defines the second tangent bundle by using normal calculus notation.}
\begin{align*}
    \phi \cong \theta :\R^2 \to M
    \iff   &  \phi(0,0) = \theta(0,0) \\
    % \wedge &  \left(\forall f \in C^\infty(M), D[f \o \phi] \o (0, x) = D[f \o theta] \o (0,x) \right) \\
    \text{and } &  \forall f \in C^\infty(M), \frac{\partial f \o \phi}{\partial x_i}(0,0) = \frac{\partial f \o \theta}{\partial x_i}(0,0), i = 0,1
    % \wedge &  \forall f \in C^\infty(M), D^2[f \o \phi] = D^2[f \o \theta] \big) \\
\end{align*}

\begin{proposition}[\cite{Cockett2014}]
    There are two natural transformations\pagenote{I have fixed the notation in this proposition.}
    \[
        \ell: TM \to T^2M; \ell([\gamma]) = [\gamma \o (\pi_0 \cdot_\R \pi_1)] \hspace{0.5cm}
        c: T^2M \to T^2M; c([\gamma]) = [\gamma \o (\pi_1, \pi_0)]
    \]
    satisfying the following coherences:
    \begin{enumerate}[(i)]
        \item $\ell.T \o \ell = T.\ell \o \ell$\footnote{Recall that we are using the 2-categorical notation described in the front-matter}
        \item The following maps are morphisms of fibred $\R$-modules. 
        % https://q.uiver.app/?q=WzAsNCxbMCwwLCJUTSJdLFswLDEsIk0iXSxbMSwwLCJUXjJNIl0sWzEsMSwiVE0iXSxbMCwyLCJcXGVsbCJdLFswLDEsInAiXSxbMiwzLCJwLlQiXSxbMSwzLCIwIl1d
        \[\begin{tikzcd}
            TM & {T^2M} \\
            M & TM
            \arrow["\ell", from=1-1, to=1-2]
            \arrow["p", from=1-1, to=2-1]
            \arrow["{p.T}", from=1-2, to=2-2]
            \arrow["0", from=2-1, to=2-2]
        \end{tikzcd}% https://q.uiver.app/?q=WzAsNCxbMCwwLCJUTSJdLFswLDEsIk0iXSxbMSwwLCJUXjJNIl0sWzEsMSwiVE0iXSxbMCwyLCJcXGVsbCJdLFswLDEsInAiXSxbMiwzLCJULnAiXSxbMSwzLCIwIl1d
        \begin{tikzcd}
            TM & {T^2M} \\
            M & TM
            \arrow["\ell", from=1-1, to=1-2]
            \arrow["p", from=1-1, to=2-1]
            \arrow["{T.p}", from=1-2, to=2-2]
            \arrow["0", from=2-1, to=2-2]
        \end{tikzcd}\]
        \item $c \o c = id$
        \item $T.c \o c.T \o T.c = c.T \o T.c \o c.T$
        \item $c \o \ell = \ell$
        \item $T.c \o c \o T.\ell = \ell.T \o c$.
    \end{enumerate}
\end{proposition}
\pagenote{Removed the proofs - they are in the literature and they proved to be a distraction.}
% \begin{proof}
%     We sketch some of the proofs.
%     \begin{itemize}
%         \item[(i)] This is a consequence of the associativity of addition - we have:
%         \[
%             \ell.T \o \ell [\gamma] = \ell.T() =  [\gamma(\pi_0 \cdot (\pi_1\cdot \pi_2) )] = T.\ell \o \ell [\gamma]
%         \]
%         \item[(iv)] Note that $T.c \o ([\gamma]) = [\gamma(\pi_0, \pi_2,\pi_1)]$ and $c.T \o ([\gamma]) = [\gamma(\pi_1, \pi_0,\pi_2)]$
%         Each side of the equation becomes $\gamma(\pi_2, \pi_1, \pi_0)$, so the equation holds.
%     \end{itemize}
% \end{proof}
\begin{observation}
\label{obs:yb-eq}
    Equation (iv) is known as the \emph{Yang--Baxter} equation. 
    It is one of the coherences for a symmetric monoidal category, and states that the twisting operation between two variables is coherent.
    We may regard the category of endofunctors on a category as a strict monoidal category and use string diagram notation (see e.g. \cite{selinger2010survey}).
    Interpreting the map $c$ as twisting two strings, the coherence becomes \pagenote{I added a reference to string calculus.}
    \begin{center}
        \input{TikzDrawings/Ch1/YangBaxterString.tikz}
    \end{center}
\end{observation}

The projection $p:TM \to M$ is \emph{locally trivial}: for each connected component of $M$ that is modeled on $\R^n$, each point $m$ lies in an open subset $U_m$ so that $p^{-1}(U_m) \cong U_m \x \R^n$. 
This local triviality property leads to the following universality condition.
\begin{proposition}\label{prop:ell-universal}
    The following diagram is an equalizer:
    % https://q.uiver.app/?q=WzAsMyxbMCwwLCJUTSJdLFsxLDAsIlReMk0iXSxbMiwwLCJUTSJdLFsxLDIsInAiLDFdLFsxLDIsIlQucCIsMCx7Im9mZnNldCI6LTJ9XSxbMSwyLCIwXFxvIHAgXFxvIHAiLDIseyJvZmZzZXQiOjJ9XSxbMCwxLCJcXGVsbCIsMl1d
    \[\begin{tikzcd}
        TM & {T^2M} & TM
        \arrow["p"{description}, from=1-2, to=1-3]
        \arrow["{T.p}", shift left=2, from=1-2, to=1-3]
        \arrow["{0\o p \o p}"', shift right=2, from=1-2, to=1-3]
        \arrow["\ell"', from=1-1, to=1-2]
    \end{tikzcd}\]
\end{proposition}
Therefore the diagram in the following corollary is a pullback: \pagenote{In the original draft it was unclear whether there was a missing diagram or I was referring to the diagram in the following lemma. This change addresses that ambiguity.}
\begin{corollary}\label{cor:mu-universal}
    Write the map $\mu:TM \ts{p}{p} TM \to T^2M$ to be $T.+ \o (\ell \x 0)$. The following diagram is a pullback:
    % https://q.uiver.app/?q=WzAsNCxbMCwwLCJUXzJNIl0sWzEsMCwiVF4yTSJdLFswLDEsIk0iXSxbMSwxLCJUTSJdLFsyLDMsIjAiLDJdLFsxLDMsInAiXSxbMCwxLCJcXG11Il0sWzAsMl0sWzAsMywiIiwxLHsic3R5bGUiOnsibmFtZSI6ImNvcm5lciJ9fV1d&macro_url=https%3A%2F%2Fraw.githubusercontent.com%2Fbenjamin-macadam%2Ftex-preamble%2Fmain%2Fpreamble.sty
    \[\begin{tikzcd}
        {TM \ts{p}{p} TM} & {T^2M} \\
        M & TM
        \arrow["0"', from=2-1, to=2-2]
        \arrow["T.p", from=1-2, to=2-2]
        \arrow["\mu", from=1-1, to=1-2]
        \arrow[from=1-1, to=2-1]
        \arrow["\lrcorner"{anchor=center, pos=0.125}, draw=none, from=1-1, to=2-2]
    \end{tikzcd}\]
\end{corollary}

%NOTE: To fix later comments re: setting notation for Lie algebroids, it makes sense to better introduce the operational tangent bundle at this point.
\pagenote{I now introduce the definition of the operational tangent bundle at this point addresses confusions about notation that comes up later in the chapter/thesis regarding the $C^\infty(M)$-module structure on $\chi(M)$}
The five maps $(p, 0, +, c, \ell)$, along with their coherences and universal properties, characterize the \emph{kinematic} tangent bundle, axiomatized as a tangent structure in the next section. However, there is an equivalent characterization of the tangent bundle for a finite-dimensional smooth manifold that will be important throughout this thesis: the \emph{operational} tangent bundle. We first need to define the module of vector fields on a manifold, where a vector field is essentially an ordinary differential equation defined on a manifold rather than a cartesian space. 
\begin{definition}[3.1,3.3 of \cite{Kolar1993}]\label{def:operational-tang}
    A \emph{vector field} on a manifold $M$ is a section $X:M \to TM$ of the projection $p:TM \to M$ so that $p \o X = id_M$. 
    The set of vector fields on a manifold $M$ is written $\chi(M)$ and carries a $C^\infty(M)$-module structure using the fibered $\R$-module structure on $p:TM \to M$:
    \[
        X +_{\chi(M)} Y := +.M \o (X,Y), \hspace{0.5cm} 0_{\chi(M)} := 0.M,\hspace{0.5cm} f \cdot_{\chi(M)} X (m) := f(m) \cdot_{TM} X(m)
    \]
    where $X,Y \in \chi(M), f \in C^\infty(M)$.
\end{definition}
The module $\chi(M)$ has an important universal property as a $C^\infty(M)$-module---it is precisely the module of \emph{derivations} of $C^\infty(M)$:
 \[
        \chi(M) = \{ X: C^\infty(M) \to C^\infty(M) : \forall f, g \in C^\infty(M), X(f\cdot g) = X(f)\cdot g + f \cdot X(g) \}
\]
\begin{proposition}[3.4 of \cite{Kolar1993}]\label{prop:derivations-tangent-bundle}
    There is an isomorphism of $C^\infty(M)$ modules:
    \[
        \mathsf{Der}(C^\infty(M)) \cong \chi(M).  
    \]
\end{proposition}
Finally, we observe that there is a $C^\infty(M)$-Lie algebra structure on $\chi(M)$, with two equivalent definitions. First, there is the kinematic definition of the bracket, which is induced using the universality of the vertical lift. Given vector fields $X,Y$ on $M$, note that
\[
   X = p.T \o (T.Y \o X -_{TM} c \o T.X \o Y), \hspace{0.25cm}
   0 = T.p \o (T.Y \o X -_{TM} c \o T.X \o Y)
\]
so by Corollary \ref{cor:mu-universal}, there is a unique map $[X,Y]:M \to TM$ so that
\begin{equation}\label{eq:hard}
    (T.Y \o X -_{TM} c \o T.X \o Y) = \mu([X,Y], X).
\end{equation}
Similarly, there is a bracket defined on $\mathsf{Der}(M)$ using the \emph{anticommutator} of derivations:
\begin{equation}\label{eq:anticom}
    [X,Y](f) = X(Y(f)) - Y(X(f)).
\end{equation}
These brackets are equivalent for finite-dimensional smooth manifolds.
\begin{proposition}\label{prop:anti-commm-lie}[\cite{Mackenzie2013}]
    Recall that a Lie algebra over a ring $R$ is an $R$-module $A$ equipped with a bilinear map
    \[
        [-.-]: A \ox A \to A   
    \]%fill this in
    that is alternating and satisfies the Jacobi identity:
    \[
        [X,[Y,Z]] + [Z,[X,Y]] + [Y, [Z,X]] = 0.  
    \]
    The two brackets on $\chi(M)$ (viewed as a Lie algebra) from Equations \ref{eq:hard} and \ref{eq:anticom} coincide.
\end{proposition}
% \begin{observation}%
%     \label{obs:operational-tangent-bundle}
    % {\color{red}
    %     This observation must be merged back into the main section
    %     \begin{itemize}
    %         \item A \emph{vector field} is a section of tangent projection (is an ODE).
    %         \item Vector fields have an evident $\R$-module structure, but can be extended to a $C^\infty(M)$-module structure
    %         \item Prop: eq. char. of $\chi(M)$ as derivations (operational tangent bundle)
    %         \item Note that there is a Lie bracket on the $\R$-module of sections - this can be defined the ``hard way'' using the canonical flip (b.c. the Jacobi identity is a truly difficult technical proof) or the easy way (derivations) - note that both definitions give same Lie algebra.
    %     \end{itemize}
    % }
    % In the category of \emph{finite-dimensional} smooth manifolds, there is an equivalent characterization of the tangent bundle, called the \emph{operational} tangent bundle (this is in contrast to the \emph{kinematic} tangent bundle described in this section, where kinematics refers to the evolution of a physical system as a path through a configuration space that is represented as a manifold). 
    % This construction of the tangent bundle is more algebraic in nature - the primitive structure is the ring of functionals $C^\infty(M)$. In this case, a vector field on a manifold $M$ is a derivation on the ring of functionals on $M$:
    % \[
    %     \chi(M) \{ X: C^\infty(M) \to C^\infty(M) : \forall f, g \in C^\infty(M): X(f\cdot g) = X(f)\cdot g + f \cdot X(g) \}
    % \]
    % % The derivative of $f$ along $X$, then, is written $X(f)$. When $X$ is regarded as a map $X:M \to TM$, this is exactly:
    % % % This equips functions $f:M \to \R$ with a notion of directional derivative:
    % % \[
    % %     X(f) : M \xrightarrow[]{X} TM \xrightarrow[]{T.f} T\R \xrightarrow[]{\phat} \R  
    % % \]
    % And there is a \emph{Lie bracket} on the space of vector fields defined using the anti-commutator:
    % \[
    %     [X,Y] = X\o Y - Y \o X  
    % \]
    % and this bracket satisfies the Liebniz law:
    % \[
    %     [X \cdot f,Y]  = X \cdot Y(f) + [X,Y]\cdot f
    % \]
% \end{observation}

\section{Tangent structures}%
\label{sec:tangent-structure}
This section develops the categorical framework to study more general categories of smooth manifolds by axiomatizing the tangent bundle, tangent categories, which first appeared in \cite{Rosicky1984}. An arbitrary tangent category is significantly more general than smooth manifolds and captures examples of categories with ``tangent bundle'' from computer science and logic, as developed in \cite{Cockett2014}. First, observe that tangent categories forget the base ring from the previous section and only consider the fibered commutative monoid structure of the tangent bundle.
\begin{definition}\label{def:add-bun}
    An additive bundle in a category $\C$ is a triple $E \xrightarrow{q} M, +:E\ts{q}{q}E \to E, \xi:M \to E$ which gives $(q,+,\xi)$ the structure of a commutative monoid in the slice category $\C/M$.
    If $(q,\xi,+), (q',\xi',+')$ are both additive bundles, a bundle morphism
    \[
        \begin{tikzcd}
            E \dar[swap]{q} \rar{f} & E' \dar{q} \\
            M \rar{f_0} & M'
        \end{tikzcd}
    \]
    is \textit{additive} if $f \o + = +' \o (f\o \pi_0,f\o \pi_1)$ and $f \o \xi = \xi' \o f_0$. The category of additive bundles in a a category $\C$ is given by additive bundles in $\C$ and additive bundle morphisms.
\end{definition}
We will often write pullback powers of an additive bundle $E \ts{q}{q} \dots \ts{q}{q} E$ as $E_n$, and use infix notation to write addition so $+ \o (a,b)$ becomes $a + b$.
In the category of smooth manifolds, the tangent bundle functor gives a well-behaved functorial vector bundle. 
A tangent category has a functorial additive bundle and axiomatizes the coherences and universal properties of the tangent bundle from the category of smooth manifolds.
\begin{definition}[\cite{Rosicky1984,Cockett2014}]\label{def:tangent-cat}
 A tangent structure consists of a functor $T:\C \to \C$ equipped with natural transformations
  \begin{gather*}
    p:T \Rightarrow id, \hspace{0.15cm} 0:id \Rightarrow T, +: T\ts{p}{p} T \Rightarrow T \\
    \ell: T \Rightarrow T.T \hspace{0.30cm} c:T.T \Rightarrow T.T
  \end{gather*}
  satisfying the following axioms; we call a category equipped with a tangent structure a \emph{tangent category}.\pagenote{I have added "."'s to the diagrams to keep my notation consistent.}
  \begin{enumerate}[{[TC.1]}]
  \item Additive bundle axioms:
    \begin{enumerate}[(i)]
        \item Pullback powers of $p$ exist and are preserved by $T$; write these $T_n$.
        \item Each triple $(p.M: TM \to M, 0.M:M \to TM, +.M:T_2M \to TM)$ is an additive bundle.
        \begin{equation}
            \label{eq:abun-axioms}
            \input{TikzDrawings/Ch1/additivity-axioms.tikz}
        \end{equation}
        % \item $(\ell,0):p \to p.T$ is additive
        % \item $(c,1): p_T \to T.p$ is additive
    \end{enumerate}
  \item Symmetry axioms:
    \begin{enumerate}[(i)]
        \item Involution: 
        \begin{equation}\label{eq:tc-2-1}
            \input{TikzDrawings/Ch1/sym-1.tikz}
        \end{equation}
        \item Yang--Baxter: $c.T \o T.c \o c.T = T.c \o c.T \o T.c$
        \begin{equation}\label{eq:tc-2-2}
            \input{TikzDrawings/Ch1/sym-2.tikz}
        \end{equation}
        \item Naturality equations:
        \begin{equation}\label{eq:tc-2-3}
            \input{TikzDrawings/Ch1/sym-3.tikz}
        \end{equation}
    \end{enumerate}
  \item Lift axioms:
    \begin{enumerate}[(i)]
        \item Naturality with addition:
        \begin{equation}
            \label{eq:tc-3-0}
            \input{TikzDrawings/Ch1/ell-0.tikz}
        \end{equation}
        \item Coassociativity:
        \begin{equation}
            \label{eq:tc-3-1}
            \input{TikzDrawings/Ch1/ell-1.tikz}
        \end{equation}
        \item Symmetric co-multiplication:
        \begin{equation}
            \label{eq:tc-3-2}
            \input{TikzDrawings/Ch1/ell-2.tikz}
        \end{equation}
        \item Universality: for $\mu:= T.+ \o (0\o \pi_0, \ell \o  \pi_1)$, the following diagram is a pullback for all $X$:
        \[\input{TikzDrawings/Ch1/univ-lift.tikz}\]
    \end{enumerate}
  \end{enumerate}
\end{definition}
%NOTE: Pulled out definition of cartesian tangent category to make it clear that it was defined, added a note re: notation to handle any confusion
\begin{definition}[\cite{Cockett2014}]
    \pagenote{I pulled out the definition of a cartesian tangent category to make it clear that it had been defined, and added a note regarding notation to handle any confusion.}
    A tangent category is \textit{monoidal} whenever $\C$ is a monoidal category, $T$ is a monoidal functor, and $+,p,c,\ell$ are monoidal natural transformations. A tangent category is \textit{cartesian} whenever $\C$ is cartesian and is a strict monoidal tangent category for products.  
\end{definition}
\begin{notation}
Throughout this thesis, two key pieces of notation apply:
    \begin{itemize}
        \item $T_n$ denotes pullback powers of $p:T \Rightarrow id$ (and more generally $E_n$ for $q:E \to M$); iterated powers of $T$ are written $T^n$.
        \item There will often be long strings of $T_n$ pullbacks and functors $F:A \to B$, so a 2-categorical notation where functor composition is written with a period, $T_n.F.T'_m$, will often be adopted. While the natural transformation $c$ at $T_n.T_m.M$ under the image of the functor $T$ would often be written $T(c_{T_n.T_m.M})$, in this thesis it will be written as
        \[
            T.c.T_n.T_m.M:T.T^2.T_n.T_m \Rightarrow T.T^2.T_n.T_m
        \]  
        while the composition of 2-cells will be written using $\o$ in applicative order, rather than the diagrammatic order typically used in the tangent category literature. That is, the composition
        \[
            A \xrightarrow[]{g} B \xrightarrow[]{f} C
        \]
        would be written $f \o g$ rather than $gf$.
    \end{itemize}
\end{notation}
\begin{example}\label{example:tangcat-sman}
    Applying the results from \cref{sec:smooth-manifolds}, the category of smooth manifolds is a tangent category. 
    Recall that if $M$ is a smooth manifold, there is a coordinate patch $m \in U \hookrightarrow M$ around each point $m \in M$ so that $U \cong U' \subseteq \mathbb{R}^n$.
    fibre above $U$, $p^{-1}(U)$, is locally isomorphic to $U' \x \mathbb{R}^n$ and similarly $(p \o p)^{-1}(U) \cong U' \x (\R^n)^3$, so that:
    \begin{center}
        \begin{tabular}{|l|l|}
            \hline
            $p:U \x \R^n \to U$  & $(m,x) \mapsto m$ \\ \hline
            $0: U \to U \x \R^n$ & $m \mapsto (m,0)$ \\ \hline
            $+:U \x \R^n \x R^n \to U \x \R^n$                         & $(m,x,y) \mapsto (m, x+y)$      \\ \hline
            $\ell:U \x \R^n \to U \x \R^n \x \R^n\x \R^n$              & $(m,x) \mapsto (m,0,0,x)$       \\ \hline
            $c: U \x \R^n \x \R^n\x \R^n \to U \x \R^n \x \R^n\x \R^n$ & $(m, x,y,z) \mapsto (m, y,x,z)$ \\ \hline
        \end{tabular}
    \end{center}
\end{example}
\pagenote{removed the example of Lex - there are some coherence issues to sort out, and that would further distract from the main point of this chapter (introducing the tangent categories and smooth manifolds).}

% \begin{example}
%     There is a folklore example of tangent structure, on the category of categories with finite limits and finite-limit-preserving functors, $\mathsf{Lex}$, due to Quillen\footnote{Finding a precise reference has been challenging}. The ``tangent category'' to $\C$ is the category of abelian group bundles in a category $\C$ (they are often called Beck modules, and first appear in \cite{Beck2003}). Several mathematicians have observed that Quillen's tangent category behaves like the tangent bundle on $\mathsf{Lex}$, the  - see, for example, \cite{Frankland2010, Stel2013}.

%     The endofunctor $T$ sends a category $\c$ to Beck modules in $\c$. The projection, then, sends an additive bundle to its base space, and the zero maps send an object to the trivial Beck module $(id,id,id)$. The category $T_2\c$, then, is the category of pairs of Beck modules over a fixed space, and the addition map sends a pair of Beck modules to its biproduct:
%     \input{TikzDrawings/Ch1/BeckModules/addition.tikz}
%     The canonical flip sends a double Beck module to its ``flipped'' bundle
%     \input{TikzDrawings/Ch1/BeckModules/flip.tikz}
%     Furthermore, the vertical lift sends an additive bundle its ``vertical'' bundle:
%     \input{TikzDrawings/Ch1/BeckModules/lift.tikz}
%     Note that for any finitely-complete additive category $\a$ in $\mathsf{Lex}$, any Beck module splits as:
%     \[
%         A \xrightarrow[]{q} B \cong B \oplus A' 
%     \]
%     where $A' = \mathsf{Ker}(q)$ - so there is an equivalence of categories $\a \x \a \cong T\a$. This equivalence of categories proves the universality of the vertical lift - the pullback:
%     \input{TikzDrawings/Ch1/BeckModules/univ-v-lift.tikz}
%     is the category of Beck modules of the form:
%     \input{TikzDrawings/Ch1/BeckModules/ob-in-univ.tikz}
%     This forces $D \to A$ to be a Beck module in the category of Beck modules over $M$, which is an additive category with finite limits, so $D \cong A \x V$, and double Beck module is isomorphic to one of the form:
%     \input{TikzDrawings/Ch1/BeckModules/splitting-of-vlift-ob.tikz}
%     This subcategory is precisely the image of $A\to M, V \to M$ under the map $\mu:T_2\c \to T^2\c$. Thus, this pullback is a limit up to an equivalence of categories. The coherences on addition only holds up to a natural isomorphism (e.g. the associator map), and the universality of the lift holds only up to equivalence of categories, so this is a \emph{2-tangent category} in some appropriate sense.
% \end{example}

The study of tangent categories is closely related to Lawvere's \emph{synthetic differential geometry}, first introduced in \cite{Lawvere1979} and later developed in \cite{Kock2006, Lavendhomme1996, Moerdijk1991}. The setting of synthetic differential geometry is a topos $\e$ (for our purposes, we need only a complete, cartesian closed category) equipped with a chosen ring object $R$ that satisfies the \emph{Kock-Lawvere axiom}: given the object of nilpotent elements in $R$, $D = [d : R | d^2 = 0]$, the following map is an isomorphism:
\[
    \alpha: R \x R \to [D,R]; \hspace{0.25cm} \alpha(a,b) = (d \mapsto a + db).  
\]
% The full subcategory of $R$-modules in $\e$ satisfying the Kock-Lawvere axiom:
% \[
%    \alpha_E: E \x E \to [D,E];  \hspace{0.25cm} \alpha_E(a,b) = d \mapsto a + db 
% \]
% satisfies the axioms of a cartesian differential category (when using the full subcategory of maps in $\C$). 
One can find a class of objects that form a tangent category: the infinitesimally linear objects.\pagenote{
   This passage has been edited substantially to clear up potential confusion about the definition of the Kock-Lawvere axiom (it originally seemed like it was defined twice), and the notions of infinitesimal linearity (the language of ``perceiving'' a diagram as a colimit is a needlessly confusing bit of jargon in synthetic differential geometry that doesn't need to be in this thesis). 
}
\begin{definition}%
\label{def:inf-linear}
    Let $\C$ be a model of synthetic differential geometry where the ring object is $(R, \cdot, 1,+, 0)$.
    An object $M$ in a model of synthetic differential geometry is \emph{infinitesimally linear} when it satisfies the following axioms.
    \begin{enumerate}[(i)]
        \item The following natural morphism must be an isomorphism: 
            \[
                [D(2),M] \cong [D,M]\ts{0^*}{0^*} [D,M];
                \text{ where } 
                D(2) := [(d_0,d_1) \in D^2 : d_i\cdot d_j = 0].
            \]
        \item $M$ satisfies \emph{property W} (credited by  \cite{Kock2006} to Gavin Wraith), namely that the following diagram is a ternary equalizer:
            \input{TikzDrawings/Ch1/InfObj/triple-eq.tikz}
        (We will often use $M$ as shorthand for $id_M$ in diagrams).
    \end{enumerate}
\end{definition}
\pagenote{The treatment of the operational tangent bundle at the end of the last section made the observation regarding the Lie bracket here unnecessary.}
% \begin{observation}%
%     \label{obs:the-lie-bracket-in-a-tang-cat}
%     Recall the ``operational'' description of the tangent bundle at the end of Section \ref{sec:smooth-manifolds}. In a tangent category with negatives, there is a Lie bracket on the set of vector fields:
%     \[
%           \ell \o [X,Y] = (c \o T.X \o Y -_{p.T} T.Y \o X) -_{T.p}  0 \o X
%     \]
%     \cite{Mackenzie2013} showed this bracket is equivalent to the Lie bracket of vector fields in the category of smooth manifolds from Observation \ref{obs:operational-tangent-bundle}, and gave a direct proof that this bracket that it satisfies the Jacobi identity. In \cite{Cockett2015}, the authors provided a proof that the Jacobi identity holds for this Lie bracket in any tangent category - this proof used a string calculus (in some ways anticipating \cite{Leung2017}). In a tangent category with a base ring (e.g. Definition \ref{def:inf-linear}), \cite{Lavendhomme1996} proved that the Liebniz law holds.
% \end{observation}

An \emph{infinitesimal object} generalizes the object $D$ in the category of infinitesimally linear objects in a model of synthetic differential geometry. The definition adopted in this thesis is a strict generalization of the definition given in \cite{Cockett2014}: here, we work with a symmetric monoidal closed category rather than a cartesian closed category in order to capture some examples from logic.\pagenote{I have expanded on this definition to fix  ambiguities in the original draft.}
\begin{definition}[Definition 5.6 \cite{Cockett2014}]
    \label{def:inf-object}
    An \emph{infinitesimal object} in a symmetric monoidal category \[(\C, \ox, I, \alpha, \rho, \sigma)\] is a tuple $(\odot:D \ox D \to D, 0:I \to D, \epsilon: D \to I, \delta: D \to D(2))$ (where $D(n)$ denotes pushout powers of $0$) so that:
    \begin{enumerate}[{[{I}O.1]}]
        \item Pushout powers $D(n)$ of $\hspace{0.05cm} 0:I \to D$ exist, and $\epsilon \o 0 = id_I$.
        \item $\odot$ is a commutative semigroup with zero, so that the following diagrams commute:
        \input{TikzDrawings/Ch1/InfSemiGroupCoh.tikz}
        The third diagram shows that $0$ is an absorbing element rather than a unit (think of $0$ as being in the commutative semigroup on the set $[0,1)$ given by multiplication).
        \item The map $\delta: D \to D(2)$ makes $(0:I \to D, \delta, \epsilon)$ into a commutative comonoid in the coslice $I/\C$:
        \input{TikzDrawings/Ch1/inf-co-add-coh.tikz}
        \item The following diagram commutes ($\odot$ is coadditive):
        \input{TikzDrawings/Ch1/InfObj/odot-coadditive.tikz}
        The notation $(f | g):A + B \to C$ for pushouts/coproducts is dual to pairing $(f',g'):C \to A \x B$ for pullbacks/products, just as $\iota_i$ is dual to projection. Therefore, $(f | g) \o \iota_0 = f$ just as $\pi_0 \o (f,g) = f$.
        \item The following diagram is a coequalizer:
        \input{TikzDrawings/Ch1/InfObj/coeq.tikz}
    \end{enumerate} 
\end{definition}
There are two tangent structures associated to an infinitesimal object in a symmetric monoidal category $\C$. The first relies on the \emph{exponentiability} of the infinitesimal object, while the second tangent structure is on the opposite category of $\C$. The enriched perspective on tangent structure in Section \ref{sec:tang-cats-enrichment} will clarify the relationship between these two tangent structures.
\begin{proposition}%
    \label{prop:inf-object-tangent-structures}
    Let $(\C, \ox, I, \alpha, \rho, \sigma)$ be a symmetric monoidal category, and \[\odot:D \ox D \to D, 0:I \to D, \epsilon: D \to I, \delta: D \to D(2)\] define an infinitesimal object in $\C$.\pagenote{
        The original proposition was unclear, so this passage has been changed to fix some ambiguities. The full structure of the symmetric monoidal category was added, the order of (i) and (ii) are swapped, the wording in part (ii) has been clarified, as has the proof itself.
    }
    \begin{enumerate}[(i)]
        \item There is a tangent structure on $\C^{op}$, where $T = D \ox (-)$.
        \item If $\C$ is also a symmetric monoidal closed category, then it is a tangent category with $T = [D, -]$.
    \end{enumerate}
\end{proposition}
\begin{proof}
   ~\begin{enumerate}[(i)]
        \item The opposite category of $\C$ is a symmetric monoidal category:
        \[
            (\C^{op}, \ox, I, \alpha^{-1}, \rho^{-1}, \sigma).
        \]
        The tangent functor is $D \ox (-)$, and the projection is \[p = D \ox (-) \xrightarrow[]{0^{op}} I \ox (-) \xrightarrow[]{\rho^{-1}} (-).\] The zero map is given by
        \[
            (-) \xrightarrow[]{\rho} I \ox (-) \xrightarrow[]{\epsilon^{op}} D \ox (-).
        \]
        Addition in $\C^{op}$ is given by co-addition in $\C$:
        \[
            D(2) \ox (-) \xrightarrow[]{\delta^{op} \ox (-)} D \ox (-). 
        \]
        The lift is given by the semigroup structure (along with the monoidal coherences):
        \[
            D \ox (-) \xrightarrow[]{\odot \ox (-)} (D \ox D) \ox (-) \xrightarrow[]{\alpha^{-1}} D \ox (D \ox (-)).
        \]
        The flip is also given by monoidal coherences, combined with the symmetry on the monoidal category:
        \[
            D \ox (D \ox (-))  \xrightarrow[]{\alpha} (D \ox D) \o (-) \xrightarrow[]{\sigma \ox (-)}
            (D \ox D) \ox (-) \xrightarrow[]{\alpha^{-1}} D \ox (D \ox (-)).
        \]
      \item First, we identify the following natural isomorphisms:
        \[
            u: id \Rightarrow [I,-],\hspace{0.5cm}
            b: [A,[B,-]] \Rightarrow [A \ox B, -].
        \]
        \begin{itemize}
            \item The tangent functor is $[D,-]$, and the triple $(0:I \to D, \delta:D \to D(2), \linebreak \epsilon:D \to I)$ gives the additive bundle structure
            \begin{gather*}
                p:[D,-] \xrightarrow{0^*} [I,-] \xrightarrow{u} id, \hspace{1cm}
                0: id \xrightarrow{u} [I,-] \xrightarrow{[\epsilon, -]} [D,-] \\
                +:[D(2),-] \xrightarrow{\delta^*} [D,-].
            \end{gather*}
    
            Note that by the continuity of $[-, M]: \C^{op} \to \C$, we have $[D(2),M] \cong [D,M] \ts{p}{p} [D,M]$.
            \item The lift is given by
                \[
                    \ell: [D, -] \xrightarrow{\odot^*} [D \ox D, -] \xrightarrow{b^{-1}} [D,[D,-]]. 
                \]
            \item The canonical flip is given by
                \[
                    c: [D,[D,-]] \xrightarrow{b} [D\ox D, -] \xrightarrow{\sigma^*} [D\ox D, -]\xrightarrow{b^{-1}} [D,[D,-]].
                \]
        \end{itemize}
        The coherences and couniversality properties of an infinitesimal object, along with the continuity of 
        \[
            [-,M]: \C^{op} \to \C 
        \]
        then induce the coherences and universality properties for the tangent bundle. This completes the proof.
    \end{enumerate}
\end{proof}
The tangent structure on $\C^{op}$ induced by the infinitesimal object is the \emph{dual} tangent structure on $\C$. This will be revisited in Chapters \ref{chap:weil-nerve} and \ref{ch:inf-nerve-and-realization} when looking at tangent categories from the enriched perspective.
% \begin{observation}
%     The definition of a infinitesimal object does not strictly necessitate the monoidal category being symmetric, as the object can carry a symmetry $\sigma: D \ox D \to D\ox D$ that is coherent with the other structure maps of an infinitesimal object. Given this, we see that an infinitesimal object in a left-closed monoidal category will give rise to a tangent structure, but chasing through the coherences for a complete proof is best left for future work. 
    % Similarly, we can require that each of the endofunctors \[ \ox^k D(n_i) \ox \_: \C \to \C  \] have left adjoints (e.g. they are exponentiable), so that this result applies to general monoidal categories.
% \end{observation}

Returning to synthetic differential geometry, the co-universality conditions on an infinitesimal object corresponds to infinitesimal linearity (Definition \ref{def:inf-linear}). In some sense, the category of infinitesimally linear objects is the largest subcategory of $\e$ for which $D$ is an infinitesimal object. 
\begin{corollary}
    In a model of synthetic differential geometry $(\e, R)$, the object $D = [d \in R | d^2 = 0]$ is an infinitesimal object in the category of infinitesimally linear objects in $\e$.
\end{corollary}

This thesis makes use of the 2-category of tangent categories (Section 2.3 of \cite{Cockett2014}), which formalizes the notion of a morphism of tangent structure and 2-cells between them. This 2-categorical framework is a departure from the classical theory of synthetic differential geometry, where the literature only really addresses morphisms of tangent structure in the form of fully faithful embeddings $\mathsf{SMan} \hookrightarrow \mathsf{Microl}(\e)$.\pagenote{Removed ``the formal theory of tangent categories, added Microl to $\e$.''}

\begin{definition}\label{def:tang-functor}
    Let $(\C, \mathbb{T}), (\D, \mathbb{T}')$ be a pair of tangent categories. A pair $(F: \C \to \D, \alpha:F.T \Rightarrow T'.F)$ is a \emph{tangent functor} if the following diagrams commute:\pagenote{I have fixed up the notation in this definition by adding a "." to $F.T$, and by setting the notation $(F,\alpha)$ for tangent functors. }
    \begin{center}
        \input{TikzDrawings/Ch1/TangFunctor.tikz}
    \end{center}
    A tangent functor is \emph{strong} whenever $\tnat$ is a natural isomorphism; for a sub-(tangent category) inclusion, $\alpha = id$.  A tangent functor $(F, \alpha)$ between cartesian tangent categories is a \emph{cartesian tangent functor} if $F$ is an isomonoidal functor and $\tnat$ is a monoidal natural transformation. 
\end{definition}
\begin{example}\label{ex:composition-of-tangent-functors}
    ~\begin{enumerate}[(i)]
        \item The coherences on the canonical flip $c$ guarantee that $(T:\C \to \C,c:T.T \Rightarrow T.T)$ is a strong tangent endofunctor on any tangent category. 
        \item Given a pair of tangent functors $(A,\alpha):\C \to \D, (B, \beta): \D \to \mathbb{E}$, the composition 
        \[
            (B.A:\C \to \mathbb{E}, \beta.A \o B.\alpha: % https://q.uiver.app/?q=WzAsMyxbMCwwLCJCLkEuVCJdLFsxLDAsIkIuVC5BIl0sWzIsMCwiQi5BLlQiXSxbMCwxLCJCLlxcYWxwaGEiLDAseyJsZXZlbCI6Mn1dLFsxLDIsIlxcYmV0YS5BIiwwLHsibGV2ZWwiOjJ9XV0=
            \begin{tikzcd}
                {B.A.T^C} & {B.T^D.A} & {T^E.B.A}
                \arrow["{B.\alpha}", Rightarrow, from=1-1, to=1-2]
                \arrow["{\beta.A}", Rightarrow, from=1-2, to=1-3]
            \end{tikzcd})
        \]
        is a tangent functor.
        \item A model of synthetic differential geometry $(\e,R)$ is \emph{well-adapted} whenever there is a fully faithful, strict tangent functor from $\mathsf{SMan}$ to the category of microlinear spaces of $\e$,  $\mathsf{SMan} \hookrightarrow \mathsf{Microl}(\e)$. The original development of well-adapted models for synthetic differential geometry may be found in \cite{Dubuc1981}, and the reader may check \cite{Bunge2018} for a recent account of the construction of such models, or section 3 of \cite{Kock2006}.\pagenote{Clarified that the inclusion into $\mathsf{Microl}(\e)$ is a tangent functor.}
    \end{enumerate}
\end{example}


\begin{definition}\label{def:tang-nat}
    A \emph{tangent natural transformation} $\gamma$ between tangent functors $(F,\alpha), (G,\beta)$ is a natural transformation so that the following diagram commutes:
    \[
        \input{TikzDrawings/Ch1/tang-nat.tikz}
    \] 
    If $F, G$ are cartesian tangent functors, then $\gamma$ is cartesian whenever it is an isomonoidal natural transformation.
\end{definition}
\begin{definition}\label{def:tang-2cat}
    We will call the 2-category of tangent categories, tangent functors, and tangent natural transformations $\mathsf{TangCat}$. The 2-category of cartesian tangent categories is $\mathsf{CartTangCat}$.
\end{definition}


%


\section{Local coordinates in a tangent category}%
\label{sec:diff-and-tang-struct}

This section develops some structures to facilitate reasoning about higher tangent bundles, using ``local coordinates.'' In the case of a cartesian differential category,  $T^n(A) = \prod_{2^n} A$, whereas for an open subset $U \subseteq \R^m$ the tangent bundle decomposes as $T^nU \cong U \x (\R^m)^{2^n - 1}$. This section introduces two structures that allow for these arguments in an arbitrary tangent category: differential objects and connections.\pagenote{
   When I originally wrote this thesis, differential objects figured into the story more prominently. I have cut a few results here that were messily written and were no longer used in subsequent chapters.
}

\begin{definition}[\cite{Cockett2018}]
    \label{def:differential-object}
    A\pagenote{
        This definition was very unclear/inconsistent. I have since removed the term ``linear'', and fixed the notation on the commutative monoid structure on $A$, which addresses the main issued raised. 
    } \emph{ differential object}\footnote{Not to be confused with 4.1 from \cite{Barr2002}.} in a cartesian tangent category is a commutative monoid $(A, +_A,0_A)$ such that there is a section-retract pair $A \xrightarrow{\lambda} TA \xrightarrow{\hat{p}} A$ which exhibits $T(A)$ as a biproduct in the category of commutative monoids:
        \[A \oplus A \cong TA.\]
    % For convenience, denote the isomorphism $T(A) \cong A \times A$ as $\nabla$. 
    Concretely, a differential object is a commutative monoid equipped with $\lambda:A \to TA, \phat:TA \to A, \phat \o \lambda = id$ so that the following axioms hold:
    \begin{enumerate}[DO.1]
    \item Coherence between $+$ and $\lambda, \phat$:
        \[\input{TikzDrawings/Ch1/DiffOb-axioms/addition-coh.tikz}\]
    \item Coherence between $0_A$ and $0.A$:
      \[ \input{TikzDrawings/Ch1/DiffOb-axioms/zero-coh.tikz} \]    
    \item Coherence between $+_A$ and $+.A$:
        \[\input{TikzDrawings/Ch1/DiffOb-axioms/addition-coh-2.tikz}\]
    \item Coherence between $\lambda$ and $\phat$ with $\ell$:
      \[\input{TikzDrawings/Ch1/DiffOb-axioms/conn.tikz}\]
  \end{enumerate}
  A map $f:A \to B$ between differential objects $(A,\lambda_A, \phat_A, +_A,0_B) \to (B,\lambda_B,\phat_B, +_B, 0_B)$ is \emph{linear} whenever $f$ preserves the lifts and projections
  \[
      (T.f \o \lambda_A = \lambda_B \o f) \text{ and } (\phat_B \o T.f = f \o \phat_A).
  \]
\end{definition}
 Following the work in Section 3 of \cite{Cockett2018}, it is sufficient to check that $f$ preserves $\lambda$ or $\phat$ (each condition implies the other).
\begin{example}\label{ex:diffob-sman}
    ~\begin{enumerate}[(i)]
        \item In the category of smooth manifolds, the real numbers are a differential object, as $T.\R = \R[x]/x^2$, so the lift map in this case is
        \[
            \lambda(a) = 0 + a\cdot x,  \hspace{1cm} \phat(a + b\cdot x) = b. 
        \]
        More generally, finite-dimensional real vector spaces, with their canonical smooth manifold structure, are exactly differential objects in the category of smooth manifolds. The lift map is defined using $T\R \cong \R[x]/x^2$, so that
        \[ V \xrightarrow[]{(0,1^\R)} T(V \x \R) \xrightarrow[]{T\cdot_V} TV. \]
        This is equivalent to the isomorphism $T.V = R[x]/x^2 \ox V$.
        % \item Convenient vector spaces are differential objects in the category of convenient manifolds.
        %NOTE: Removed convenient manifolds because I had not introduced that example tangent category
        \pagenote{I have removed the convenient manifolds example because I had not introduced that example tangent category.}
        \item Every object in a cartesian differential category has a canonical differential object structure, as $TA := A \x A$.
        % \item In $\mathsf{Lex}$, a differential object is exactly a finitely-complete additive category. This follows from Beck's result that in an additive category, every additive bundle splits as $E \to M \cong A \x M$, and so $T\a \cong \a \x \a$. The map $\lambda$ sends $A$ to the Beck module $!:A \to 1$, and the map $\lambda$
    \end{enumerate}
\end{example}

Classically, the tangent space above each point of a smooth manifold is a vector space. 
We have a similar result for differential objects from \cite{Cockett2018}.
\begin{lemma}\label{lem:tang-space-diff-ob}
    Suppose we have the following pullback in a cartesian tangent category $\C$, and all powers of $T$ preserve it.
    \[
        \begin{tikzcd}
            E \rar{\iota} \dar{!} & TM \dar{p} \\
            1 \rar{m} & M
        \end{tikzcd}
    \]
    There is a unique differential object structure $(E,\lambda,\phat)$ so that $\ell \o \iota = T.\iota\o \lambda$.\footnote{Some diagrammatic notation had creeped into the original draft here, I have fixed it.}
\end{lemma}
% We now collect some facts about differential objects in a tangent category.
% In particular, we show that differential objects are closed under products and the tangent functor.
% % Furthermore, if a map $f$ preserves $\lambda$ it preserves $\phat$, and $Tf$ splits as a biproduct.
% \begin{lemma}[\cite{Cockett2018}]\label{lem:preserve-lambda-iff-phat}
%     Let $(A,\lambda,\phat),(B,\lambda',\phat')$ be differential objects in a cartesian tangent category $\C$.
%     \begin{enumerate}[(i)]
%         \item $f:A \to B$ preserves $\lambda$ if and only if it preserves $\phat$.
%         % \item $f:A \to B$ preserves $\lambda$ if and only if $(p,\phat)^{-1} \o T(f) (p \o \phat) = f \oplus f$.
%         % \item $(A \x B, \lambda \x \lambda', \phat \x \phat')$ is a differential object.
%         \item $(T(A), c\o T.\lambda, T.\phat \o c)$ is a differential object.
%         % \item A cartesian tangent functor $(F,\tnat,m)$ will preserve differential objects. 
%     \end{enumerate}
% \end{lemma}

There are two natural classes of morphisms between differential objects, linear and ``smooth'' (that is, arbitrary morphisms).
\begin{definition}[\cite{Cockett2018}]\label{def:diff-dlin}
    Let $\C$ be a cartesian tangent category. 
    % A morphism of differential objects is \textit{linear} if it preserves the lift (or, equivalently, $\phat$).
    We define the following categories:
    \begin{enumerate}[(i)]
        \item $\Diff(\C)$ is the category of differential objects and arbitrary morphisms, so for any differential objects $A,B$, we have $\Diff(\C)(A,B) := \C(A,B)$.
        \item $\Dlin(\C)$ is the category of differential objects and linear morphisms.
    \end{enumerate}
\end{definition}

The category of differential objects and smooth maps is a cartesian differential category, exhibiting differential calculus as a specialized logic in a tangent category. 
\begin{proposition}[Section 3.5 of \cite{Cockett2018}]\label{prop:diff-obs-cdc}
    Let $\C$ be a cartesian tangent category.
    Then:
    \begin{enumerate}[(i)]
        \item $\Diff(\C)$ is a cartesian differential category, where we define the differential combinator $D$ to be
        \[
            \infer{A \x A \xrightarrow{\lambda \x 0} T(A \x A) \xrightarrow[]{T.+_A} T(A) \xrightarrow{T(f)} T(B) \xrightarrow{\phat_B} B}{A \xrightarrow{f} B}
        \]
        (where $\lambda_A, +_A$ are the lift and addition for the differential object structure on $A$, and $\phat_B$ is the projection map on the differential object structure on $B$).
        \item There is an equality of categories, $\mathsf{Lin}(\Diff(\C)) = \Dlin(\C)$, meaning that a morphism between differential objects in the cartesian tangent category $\C$ is linear if and only if it is linear in the cartesian differential category $\Diff(\C)$.
    \end{enumerate}
\end{proposition}
% \begin{remark}
%     In the original proof of Proposition \ref{prop:diff-obs-cdc}, Cockett and Cruttwell used the notion of a \emph{coherent differential structure} on objects with a differential object structure associated to them. The phrasing of this result is slightly different, but allows for that subtlety to be ignored as on can show that $\Diff(\C)$ is a cartesian tangent category 
% \end{remark}
Recall that if $M$ is an open subset of $\R^n$, the second tangent bundle of an open subset $U \subseteq \R^n$ splits as
\[
    T^2(U) = T(U \x \R^n) = (U \x \R^n) \x (\R^n \x \R^n) = T_3U.
\]
Every smooth manifold admits such a decomposition on its second tangent bundle; these are known as a \emph{affine connections}\footnote{The prefix ``affine'' differentiates these from more general connections on differential bundles, which are introduced in Section \ref{sec:connections-on-a-differential-bundle}.} and provide a way to reason about an object as though it has local coordinates in an arbitrary tangent category.\pagenote{Added a footnote to explain the prefix ``affine''.}
\pagenote{slightly changed the passage between differential objects -> affine connections}
% However, this is not the case in a general tangent category - not every object has a decomposition $T_3M \cong T^2M$.
%  While every smooth manifold has such a decomposition, it is not natural. These decompositions are known as \emph{connections}

%NOTE: slightly changed the passage between differential objects -> affine connections

\begin{definition}[\cite{Cockett2017}]
    In a tangent category $\C$, define the following:
    \begin{enumerate}[(i)]
        \item An affine vertical connection is a map $\kappa:T^2M \to TM$ so that 
        \begin{enumerate}
            \item $\kappa$ is a \emph{vertical descent}, namely a section of the vertical lift $\ell:TM \to T^2M$, so $\kappa \o \ell = id$;
            \item $\kappa$ is compatible with both lifts on $T^2M$: $T.\kappa \o \ell.T = T.\kappa \o T.\ell = \ell\o \kappa$.
        \end{enumerate}
        \item An affine horizontal connection is a map $\nabla:T_2 \to T^2M$ so that
        \begin{enumerate}
            \item $\nabla$ is a \emph{horizontal lift}, namely a section to the horizontal descent $(p.T, T.p): T^2 \Rightarrow T_2$, so that $(p.T,T.p) \o \nabla = id$;
            \item $\nabla$ is compatible with the linear structures on $T^2, T_2$:$ T.\nabla\o (\ell\x 0) \linebreak = \ell \o \nabla and T.\nabla(0\x \ell) = T.\ell \o \nabla$.
        \end{enumerate}
        \item An \emph{affine connection} is a pair $(\kappa, \nabla)$ comprising a vertical and horizontal connection on $M$ satisfying the compatibility conditions: 
        \begin{enumerate}
            \item $T.+ \o (+.T \o (\ell \o \kappa, T.0 \o p.T), \nabla(p.T, T.p)) = id_{T^2M}$, 
            \item $\kappa \o \nabla = 0 \o p \o \pi_i$.
        \end{enumerate}
        An affine connection is \emph{torsion-free} if $\kappa \o c = \kappa$.
    \end{enumerate}
\end{definition}

% Note that the full data of an affine connection is equivalent to a universality condition on the vertical part of the connection.
% \begin{proposition}
%     An affine vertical connection $\kappa:T^2M \to TM$ determines a full connection on $M$ if and only if the following diagram is a limit.
%     % https://q.uiver.app/?q=WzAsNSxbMCwxLCJUXjJNIl0sWzEsMCwiVE0iXSxbMSwyLCJUTSJdLFsxLDEsIlRNIl0sWzIsMSwiTSJdLFswLDEsIlQucCJdLFswLDIsInAuVCIsMl0sWzAsMywiXFxrYXBwYSIsMV0sWzEsNCwicCJdLFszLDQsInAiLDFdLFsyLDQsInAiLDJdXQ==
%     \[\begin{tikzcd}
%         & TM \\
%         {T^2M} & TM & M \\
%         & TM
%         \arrow["{T.p}", from=2-1, to=1-2]
%         \arrow["{p.T}"', from=2-1, to=3-2]
%         \arrow["\kappa"{description}, from=2-1, to=2-2]
%         \arrow["p", from=1-2, to=2-3]
%         \arrow["p"{description}, from=2-2, to=2-3]
%         \arrow["p"', from=3-2, to=2-3]
%     \end{tikzcd}\]
% \end{proposition}
% It can also be shown that a morphism $f:M \to N$ preserves an affine connection if and only if it preserves the vertical part of 

The data of a vertical connection is sufficient to define a full connection, as observed in \cite{LucyshynWright2018}.
\begin{lemma}[\cite{LucyshynWright2018}]
    A full connection is equivalent to a vertical connection in which the following diagram is a fiber product:
    \input{TikzDrawings/Ch1/full-con.tikz}
\end{lemma}



\begin{example}
    ~\begin{enumerate}[(i)]
        \item Every differential object in a tangent category has a canonical vertical connection given by $(p \o p.T, \phat \o T.\phat)$. A morphism of differential objects will preserve this vertical connection.
        \pagenote{Fixed the notation for the affine vertical connection on a differential object induced by a differential object's $\phat$ map. }
        \item Every smooth manifold has a non-natural choice of Riemannian metric. By the fundamental theorem of Riemannian geometry, there is a torsion-free connection associated with the metric. (See any standard reference on Riemannian geometry, e.g. \cite{Carmo1992}.)
        % \item The category of commutative rings is has a full connection in $\mathsf{Lex}$. An observation due to Quillen states that $TC_{ring} \cong Mod$, where $Mod$ is the category given by:
        % \begin{itemize}
        %     \item $(S, N)$ where $S$ is a commutative ring and $N$ an $S$-module.
        %     \item $(R,M) \to (S,N)$ is a pair $\phi:R \to M$ and $f: M \to \phi^*N$.
        % \end{itemize}
        % Implicit here is the result that a Beck module over $S$ is an $S$-module.
    
        % A double Beck module in the category of commutative rings is then a Beck module in $Mod$.
        % Note that a pullback of modules of $(\phi,f): (S,Q) \to (R,N)$ is given by:
        % \[
        %     (R \ts{\phi}{\phi} R, Q \ts{f}{f} Q) \cong (R_2, Q_2)
        % \]
        % % So $Mod$ is itself the category of models for a multi-sorted algebraic theory.
    
        % \begin{itemize}
        %     \item The ring part of the Beck module is exactly a Beck module in the category of rings, so it splits as $S \cong R \oplus N$.
        %     The commutative ring $S$, then, is the square-zero extension of $R$ by $N$: 
        %     \[
        %         (R \oplus N) \x (R \oplus N) \to (R\oplus N); \hspace{0.15cm}
        %         (r_1,n_1),(r_2,n_2) \mapsto (r_1r_2, r_1n_2 + r_2n_1)
        %     \]
        %     so the map $\phi$ is actually $\pi_0: R \oplus N \to R$.
        %     \item The module part $Q$, then, is a module over the square-zero extension of $R$ by $N$. Moreover, the module $\phi^*(M)$ has the action:
        %     \[
        %         (R \oplus N) \x M \to M; \hspace{0.15cm} (r,n), m \mapsto rm
        %     \]
        %     so that we have 
        %     \[
        %         f((r,n)\cdot m) = r\cdot f(m)
        %     \]
        %     meaning that $f$ is in fact an $R$-module $f:\iota^*(Q) \to M$, and this is a Beck module in the category of $R$-modules and of course splits as $Q \cong N \oplus P$.
        % \end{itemize}
        % Therefore, that there is a natural isomorphism $id \Rightarrow \mu(p.T,\kappa) + \nabla(p.T, T.p)$ on the category of commutative rings.
        % Therefore, the core connection on the category of commutative rings is a full connection.
    \end{enumerate}
\end{example}

Connections allow for arguments on higher powers of the tangent bundle to be pushed down to pullback powers of $T$.
\begin{observation}
    Suppose $M, N$ each have connections $(\kappa_{-}, \nabla_{-})$. 
    The map $T^2.f: T^2M \to T^2N$ can be written using local coordinates $\widehat{T^2.f}: T_3M \to T_3N$ as
    % https://q.uiver.app/?q=WzAsNCxbMCwxLCJUXzNNIl0sWzEsMSwiVF8zTiJdLFswLDAsIlReMk0iXSxbMSwwLCJUXjJOIl0sWzAsMiwiXFx0aGV0YV9NIl0sWzIsMywiVF4yLmYiXSxbMywxLCIocC5ULCBULnAsIFxca2FwcGFfTikiXSxbMCwxLCIiLDIseyJzdHlsZSI6eyJib2R5Ijp7Im5hbWUiOiJkYXNoZWQifX19XV0=
    \[\begin{tikzcd}
        {T^2M} & {T^2N} \\
        {T_3M} & {T_3N}
        \arrow["{T.+ \o (+.T \o (\ell \o \pi_2, T.0 \o \pi_0), \nabla(\pi_0, \pi_1))}", from=2-1, to=1-1]
        \arrow["{T^2.f}", from=1-1, to=1-2]
        \arrow["{(p.T, T.p, \kappa_N)}", from=1-2, to=2-2]
        \arrow[dashed, from=2-1, to=2-2]
    \end{tikzcd}\]
    %T.+ \o (+.T \o (\ell \o \kappa, T.0 \o p.T), \nabla(p.T, T.p))
    where $\widehat{T^2.f}$ is given by
    \[
        (T.f\o\pi_0, T.f \o\pi_1, T.f \o \pi_2 +_N \nabla[f](\pi_0,\pi_1))
    \]
    with $\nabla[f] := \kappa_N \o T^2.f \o \nabla_M$.
\end{observation}
Note that in the case that $f$ preserves the connections, $\nabla[f] = 0$, as 
\[
    \kappa_N \o T^2.f \o \nabla_M = Tf \o \kappa_M \o \nabla_M = Tf \o 0 \o p = 0 \o f \o p.
\]

% The tangent functor preserves connections:
% \begin{lemma}
%     Let $\kappa$ be a vertical connection on $M$. Then $\kappa_T := c \o T.\kappa \o c.T \o T.c$ is a connection on $TM$.
%     If $\kappa$ is universal then so to is $\kappa_T$.
% \end{lemma}
% \begin{proof}
%     Observe that
%     \[
%         c \o T.\kappa \o c.T \o T.c \o \ell = c \o T.\kappa \o T.\ell \o c = c\o c = id
%     \]
%     Linearity follows by composition, and universality follows as $T$ preserves the universal property.
% \end{proof}

\section{Submersions}\label{sec:submersions}

The category of smooth manifolds is incomplete: there are cospans\footnote{Following a general convention in category theory, where the prefix "co"-X means an X in the opposite category, a cospan in $\C$ is a span in the dual category of $\C$.} $X \xrightarrow[]{f} M \xleftarrow[]{g} Y$ for which the pullback fails to exist. Following \cite{thom1954}, the pullback of a cospan exists and is preserved by $T$ (i.e. it is a \emph{T-limit}) whenever for each point $f(x) = g(y)$, the direct sum of the images of $T_xf$ and $T_yg$ is the full vector space $T_{f(x)}M$, such cospans are called \emph{transverse}. Submersions, then, form a convenient class of maps, as any cospan where one map is a submersion will be transverse. More precisely:
\begin{definition}
    \label{def:submersion-sman}
    If $A$ and $B$ are smooth manifolds, a smooth function $f:\linebreak A \rightarrow B$ is a \emph{submersion} if and only if the derivative $Df|_a$ of $f$ at every point $a\in A$ is a surjective linear map.
\end{definition}
With this definition we have the following result:
\begin{proposition}%
    \label{prop:submersion-properties}
    In the category of smooth manifolds, let the class of submersions be denoted by $\sh$.
    \begin{enumerate}[(i)]
        \item Submersions are closed under the tangent functor: $f \in \sh \Rightarrow T.f \in \sh$.
        \item Submersions are closed to pullback along arbitrary maps:
        \input{TikzDrawings/Ch1/submersion-closure.tikz} 
        This will often be referred to as \emph{$T$-stability under reindexing}, as it induces a functor between slice categories: 
        \[g^*: \mathsf{Submersions}/M \to \mathsf{Submersions}/N.\]
    \end{enumerate}
    \pagenote{ Based on a comment from Michael, it seemed appropriate to move the T-prefix to stability, as the reindexing operation is unchanged.}
    %NOTE: Based on a comment from Michael, it seemed appropriate to move the T-prefix to stability, as the reindexing operation is unchanged.
\end{proposition}
The properties of the class of submersions in the category of smooth manifolds were studied in \cite{Cockett2018} and axiomatized as a \emph{tangent display system}.

\begin{definition}\label{def:display-system}
    A \emph{tangent display system} in a tangent category $\C$ is a class of maps $\mathcal{D}$ in $\C$ that is
    \begin{itemize}
        \item stable under the tangent functor, $d \in \d \Rightarrow T.d \in \d$,
        \item $T$-stable under reindexing (as in Proposition \ref{prop:submersion-properties}).
    \end{itemize}
    We call any tangent display system that is closed to retracts in the arrow category a \textit{retractive} display system. If for all $M$, $p_M \in \mathcal{D}$, we call $\mathcal{D}$ a proper (retractive)\pagenote{Brackets are a common notation for optional prefixes, e.g. (co)limits.} display system. 
\end{definition}
This section will show that the submersions in the category of smooth manifolds give a retractive display system, yielding a general construction of retractive display systems from display systems.


The definition of a submersion may be rephrased as follows: $f$ is a submersion if and only if for all $a\in A$ and all $v\in T(B)$ such that $fa = pv$, there exists a $w\in T(A)$ such that $T.f \o w = v$.
This is a \textit{weakly} universal cone over $A \xrightarrow{f} B \xleftarrow{p} TB$: there exists \textit{at least} one morphism into it for any other cone over the diagram. 
\begin{definition}\label{def:weak-pullback}
    A commuting square is a \textit{weak pullback} if for any $x:X \to A$ and $y:X \to B$ so that $fx = gy$, there exists a map $X \to W$ making the following diagram commute:
    \[
        \begin{tikzcd}
        X \ar[bend left]{rrd}{x} \ar[bend right]{ddr}[swap]{y} \ar[dashed]{rd}{\exists} \\
            & W \rar{a} \dar[swap]{b} & A \dar{f} \\
            & B \rar[swap]{g} & C
        \end{tikzcd}
    \]
    If the above diagram is a weak pullback for each $T^n$, then it is a \emph{weak $T$-pullback}.
\end{definition}
\begin{lemma}\label{lem:pb-retract}
    Should the pullback of $A \xrightarrow{f} C \xleftarrow{g} B$ exist, Definition \ref{def:weak-pullback} is equivalent to asking that the induced map $(a, b):W \to A \ts{f}{g} B$ be a split epimorphism.
\end{lemma}
\begin{proof}
    Let $r$ be a retract of $(a,b):W \to A\ts{f}{g} B$. For any $X \xrightarrow[]{(x,y)}  A\ts{f}{g} B$, the map $ r \o (x,y)$ exhibits the diagram as a weak pullback. For the converse, the unique map $(a,b):W \to A \ts{f}{g} B$ will be a section of any map $A \ts{f}{g} B \to W$ induced by weak univerality.\pagenote{
    This result was originally stated without proof.
    }
\end{proof}
We now restate the submersion property for a map $f$ using global elements (for all $a\in A$ and all $v\in T(B)$ such that $fa = pv$, there exists a $w\in T(A)$ such that $T(f)w = v$) using generalized elements.
\begin{definition}\label{def:tangent-submersion}
    An arrow $f:A \rightarrow B$ in a tangent category is a \emph{tangent submersion} if and only if the naturality diagram
    \[
        \begin{tikzcd}
            TA \rar{Tf} \dar[swap]{p} & TB \dar{p} \\
            A \rar[swap]{f} & B
        \end{tikzcd}
    \] 
    is a weak $T$-pullback. 
\end{definition}{}
Following Lemma \ref{lem:pb-retract}, in the case that the pullback exists this is equivalent to asking for a section $h: A \ts{f}{p} TB \to TA$ of the horizontal descent $(p,Tf): TA \to A \ts{f}{p} TB$ (this section is sometimes called a \textit{horizontal lift} in differential geometry literature \cite{Cordero1989}). 
In smooth manifolds, the $T$-pullback along the projection $p:T \Rightarrow id$ always exists, so to prove that every submersion is a tangent submersion it suffices to show the existence of a horizontal lift.
\begin{proposition}
    In the category of smooth manifolds, the tangent submersions are precisely the submersions (Definition \ref{def:submersion-sman})
\end{proposition}
\begin{proof}
    There is an explicit construction of a horizontal lift for a classical smooth submersion in VII.1 of \cite{Cushman2015}.
\end{proof}
It is possible to show that the $T$-stability properties for submersions in the category of smooth manifolds follow from the general theory of weak pullbacks.
We begin by showing that weak pullbacks satisfy a weakened version of the pullback lemma and then show that the retract of a weak pullback is a weak pullback (the second lemma is Lemma 2.1 of \cite{Adamek2010}).
\begin{lemma}[Pullback lemma]\label{lem:weak-pullback}
    Consider the diagram
    \[
        \begin{tikzcd}
            \bullet \rar \dar \ar[rd, phantom, "(A)"] & \bullet \rar{f}\dar{g} \ar[rd, phantom, "(B)"] & \bullet\dar \\
            \bullet \rar & \bullet \rar & \bullet 
        \end{tikzcd}
    \]
    \begin{enumerate}[(i)]
        \item If $f,g$ are jointly monic and $(A)+(B)$ is a weak pullback, then $(A)$ is a weak pullback.
        % \item If $(B)$ is a pullback and $(A)$ is a weak pullback, then $(A+B)$ is a weak pullback.
        \item If $(A),(B)$ are weak pullbacks then $(A)+(B)$ is a weak pullback.
    \end{enumerate}
\end{lemma}    
\begin{proof}
    ~\begin{enumerate}[(i)]
        \item If $(A)+(B)$ is a weak pullback, a map can be induced for a cone over $(A)$ by concatenating it with $(B)$; the jointly monic condition on $f,g$ guarantees that the map induced for $(A)+(B)$ will commute for $(A)$. \pagenote{caught a broken sentence here.}
        % \item Given a cone for $(A+B)$, induce the unique map for $(B)
        \item Given a cone for $(A)+(B)$ induce a map for $(B)$, which then induces a cone for $(A)$.
    \end{enumerate}
\end{proof}

\begin{lemma}\label{lem:wpb-retract}
    (Weak) pullbacks are closed to retracts.
\end{lemma}
\begin{proof}
    Suppose that $S'$ is a weak pullback, and $S$ is a retract of it in the category of commuting squares. Consider the following diagram (suppressing the subscripts for $s,r$):
    \input{TikzDrawings/Ch1/submersion-cube-diagram.tikz}
    Given a cone for $S$, there is a corresponding cone for $S'$ which induces a map $Z \to A'$ and postcomposition with $r_A$ gives the desired map into $A$.
\end{proof}
Using these lemmas, it is straightforward to prove that the following $T$-stability properties hold for tangent submersions.
\begin{lemma}\label{lem:submersion}
    In any tangent category $\mathbb{X}$,
    \begin{enumerate}[(a)]
        \item tangent submersions are closed to composition;
        \item tangent submersions are closed to retracts; 
        \item any $T$-pullback of a tangent submersion is a tangent submersion.
    \end{enumerate}
\end{lemma}
\begin{proof}
    (a) follows from Lemma \ref{lem:weak-pullback} while (b) follows from Lemma \ref{lem:wpb-retract}. It remains to prove (c).     Consider a $T$-pullback, where $u$ is a tangent submersion:
    \[
    \begin{tikzcd}
        A \rar{f} \dar{v} & M \dar{u} \\
        B \rar{g} & N
    \end{tikzcd}
    \]
    By naturality, the outer paths of the following two diagrams are equal: 
    \pagenote{The equality of the two diagrams was originally asserted without explaining why they were equal.}
    %NOTE: clarified a point raised by Kristine
    \[
        \begin{tikzcd}
            TA \rar{Tf} \dar{Tv} & TM \rar{p} \dar{Tu} & M \dar{u} \\
            TB \rar{Tg} & TN \rar{p} & N 
        \end{tikzcd}
        =
        \begin{tikzcd}
            TA \rar{p} \dar{Tv} & A \rar{f} \dar{v} & M \dar{u} \\
            TB \rar{p} & B \rar{g} & N
        \end{tikzcd}
    \]
   Note that the left diagram is a weak pullback by composition.  Therefore the outer perimeter of the right diagram is a weak pullback, and the right square is a pullback, so the left square is a weak pullback by Lemma \ref{lem:weak-pullback}, as desired.
\end{proof}
The class of tangent submersions is closed to retracts in the arrow category and is conditionally $T$-stable under reindexing (if the $T$-pullback of a tangent submersion exists, it is a tangent submersion). This stability property leads to the following result:
\begin{proposition}\label{prop:display-submersions-are-r-display}
    Let $\mathbb{X}$ be a tangent category that allows for reindexing of the class of tangent submersions $\mathcal{R}$. Then the class of tangent submersions is a display system.
\end{proposition}
\begin{proof}
    Any class of maps that is closed to reindexing is a tangent display system, and the class of submersions is closed to retracts in the arrow category.
\end{proof}
% In the category of smooth manifolds, where the class of smooth submersions is the canonical example of a proper tangent display system, this gives the following:
\begin{corollary}\label{cor:sman-r-display}
    The class of submersions in the category of smooth manifolds is a  proper retractive display system.
\end{corollary}



\chapter{Differential bundles}%
\label{ch:differential_bundles}

Our principal aim in this thesis is to provide an abstract tangent-categorical axiomatization for Lie algebroids.  To accomplish this, we must provide an axiomatization for Lie algebroids which is essentially algebraic (in the sense of \cite{Freyd1972}).  However, in the category of smooth manifolds, Lie algebroids are defined in terms of vector bundles and these are prima facie a highly non-algebraic notion.  

In addition to algebraic axioms which make it an $\R$-module in the slice over its base $M$, a vector bundle $q:E \to M$ satisfies a crucial topological requirement:  it must be locally trivial.  This means that the projection $q : E \to M$ must be locally isomorphic to a projection $\pi_0:U \x \R^n \to U$ for some open subset $U$ of $M$ and natural number $n$.  It is this property that permits calculations using local coordinates, an approach deeply enshrined in the culture of differential geometry.

\cite{Cockett2017} introduced the algebraic notion of a differential bundle.  Evidence that differential bundles are the appropriate generalization of vector bundles was provided by showing how classical results for vector bundles could be generalized to differential bundles in any tangent category  \cite{Cockett2017,Cockett2018}.  However, the precise relationship in the category of smooth manifolds between vector bundles and differential bundles was left open. The main result of this chapter (see \cite{MacAdam2021}) is that vector bundles and differential bundles coincide in the category of smooth manifolds.

The axiomatization of differential bundles focuses on another important property of vector bundles: given a vector bundle $q:E \to M$ and a vector $v$ in the fibre $E_x$ above $x \in M$, the tangent space $T_v(E_x)$ can be naturally identified with $E_x$.    This gives a lift map $\lambda:E \to TE$ which can be axiomatized.   While the lift map had long been noted in the differential geometry literature in the guise of the Euler vector field (see 6.11 of \cite{Kolar1993} and also Section 1 of \cite{Michor1996} which explicitly uses the term "lift"), it had not been adopted as the basis of an abstract axiomatization.  

More recently, in the differential geometry literature, \cite{Grabowski2009} and \cite{Bursztyn2016} realized that the multiplicative $\R^+$-action on the total space $E$ determines the vector bundle structure of $q:E \to M$, and conversely such a multiplicative action determines a vector bundle precisely when its Euler vector field (Definition \ref{def:evf}) satisfies an additional ``non-singular'' property.  This chapter extends these more recent observations on vector bundles to differential bundles.

The chapter begins by reviewing vector bundles, describing the Euler vector field construction that sends a vector bundle $q:E \to M$ to a "lift" map $\lambda:E \to TE$ from \cite{Grabowski2009}, whereas the rest of the chapter contains new results developed in collaboration with Matthew Burke. The second section establishes that these lifts are associative coalgebras for the weak comonad $(T,\ell)$, and that there is a fully faithful functor from vector bundles into the category of lifts for smooth manifolds. The third section identifies the universal property satisfied by the lift (or equivalently Euler vector field), while the fourth section shows that a non-singular lift corresponds precisely to a differential bundle. The fifth section proves the main theorem of the chapter: vector bundles are precisely differential bundles for smooth manifolds. The final section contains some remarks on extending affine connections to arbitrary differential bundles, which will be useful in Chapter \ref{ch:involution-algebroids}. 

\pagenote{Each chapter has been given a new introduction.}


\section{Vector bundles}%
\label{sec:vector-bundles}

A vector bundle over a manifold $M$ axiomatizes the notion of a smoothly varying family of vector spaces indexed by the points $m \in M$.
The driving example is that of the tangent bundle over a smooth manifold $M$, where the fibre above each point $m \in M$ is the tangent space $T_mM$.
The manifold structure guarantees that the projection is \emph{locally trivial}: given a chart $U \hookrightarrow M$, the next pullback splits as a product:
\[\input{TikzDrawings/Ch2/splitting-lem.tikz}\]
The local triviality of the tangent bundle is essential for various constructions and is part of the definition of a vector bundle.

\begin{definition}\label{def:vector-bundle}
  A \emph{vector bundle} is a tuple \[(q:E \to M, \xi:M \to E, +: E \ts{q}{q} E \to E, \cdot: \R \x E \to E)\] of morphisms in $\mathsf{SMan}$ so that
  \begin{enumerate}[(i)]
    \item the tuple $(q,\xi,+, \cdot)$ defines an $\R$-module in $\mathsf{SMan}/M$;
    \item the map $q:E \to M$ is locally trivial.
  \end{enumerate}
  The fibred $\R$-module structure means $E$ is a family of vector spaces indexed by $M$, $\{ E_m | m \in M \}$.
\end{definition}
It is important to note that the local triviality axiom guarantees that the projection of a vector bundle is a submersion (Definition \ref{def:submersion-sman}); thus pullback powers of $q:E \to M$ exist and are preserved by the tangent functor.\pagenote{This addresses a concern Kristine had raised about the preservation of $E_2$ by $T$.}
\begin{example}
    Consider the cylinder, defined as  the subset of $\R^3$ spanned by $C = \{ (x,y,z) | x^2 + y^2 = 1, z \in \R \}$:
    \begin{figure}[H]
\centering
\begin{tikzpicture}[scale = 1.0, yscale = 2.5, xscale = 4.0]
\draw[domain=-1.002:1.002, variable=\x, samples=1000, smooth, very thick, dotted] plot ({\x}, {0.1+0.2*((1-\x*\x)^2)^0.25});
\draw[domain=-1.002:1.002, variable=\x, samples=1000, smooth, very thick] plot ({\x}, {0.1-0.2*((1-\x*\x)^2)^0.25});
\draw[domain=-1.002:1.002, variable=\x, samples=1000, smooth, ultra thick, color3] plot ({\x}, {2+0.2*((1-\x*\x)^2)^0.25});
\draw[domain=-1.002:1.002, variable=\x, samples=1000, smooth, ultra thick, color3] plot ({\x}, {2-0.2*((1-\x*\x)^2)^0.25});

\draw[domain=0.1:2, variable=\y, samples=20, smooth, very thick] plot ({-1}, {\y});
\draw[domain=0.1:2, variable=\y, samples=20, smooth, very thick] plot ({1}, {\y});


\draw[domain=-1:1, variable=\x, color1, samples=200, smooth, very thick] plot ({\x}, {2*(1-0.4*exp(-0.076587*(asin(\x)/180*pi+pi/2))) - 0.2*(1-\x*\x)^0.5});
\draw[domain=-1:1, variable=\x, color1, samples=200, smooth, very thick, dotted] plot ({\x}, {2*(1-0.4*((5^0.5-1)/2)^0.5*exp(-0.076587*(asin(-\x)/180*pi+pi/2))) + 0.2*(1-\x*\x)^0.5});

\draw[domain=-1:1, variable=\x, color1, samples=200, smooth, very thick] plot ({\x}, {2*(1-0.4*((5^0.5-1)/2)^1.0*exp(-0.076587*(asin(\x)/180*pi+pi/2))) - 0.2*(1-\x*\x)^0.5});
\draw[domain=-1:1, variable=\x, color1, samples=200, smooth, very thick, dotted] plot ({\x}, {2*(1-0.4*((5^0.5-1)/2)^1.5*exp(-0.076587*(asin(-\x)/180*pi+pi/2))) + 0.2*(1-\x*\x)^0.5});

\draw[domain=-1:1, variable=\x, color1, samples=200, smooth, very thick] plot ({\x}, {2*(1-0.4*((5^0.5-1)/2)^2.0*exp(-0.076587*(asin(\x)/180*pi+pi/2))) - 0.2*(1-\x*\x)^0.5});
\draw[domain=-1:1, variable=\x, color1, samples=200, smooth, very thick, dotted] plot ({\x}, {2*(1-0.4*((5^0.5-1)/2)^2.5*exp(-0.076587*(asin(-\x)/180*pi+pi/2))) + 0.2*(1-\x*\x)^0.5});

\draw[domain=-1:1, variable=\x, color1, samples=200, smooth, very thick] plot ({\x}, {2*(1-0.4*((5^0.5-1)/2)^3.0*exp(-0.076587*(asin(\x)/180*pi+pi/2))) - 0.2*(1-\x*\x)^0.5});
\draw[domain=-1:1, variable=\x, color1, samples=200, smooth, very thick, dotted] plot ({\x}, {2*(1-0.4*((5^0.5-1)/2)^3.5*exp(-0.076587*(asin(-\x)/180*pi+pi/2))) + 0.2*(1-\x*\x)^0.5});

\draw[domain=-1:1, variable=\x, color1, samples=200, smooth, very thick] plot ({\x}, {2*(1-0.4*((5^0.5-1)/2)^4.0*exp(-0.076587*(asin(\x)/180*pi+pi/2))) - 0.2*(1-\x*\x)^0.5});
\draw[domain=-1:1, variable=\x, color1, samples=200, smooth, very thick, dotted] plot ({\x}, {2*(1-0.4*((5^0.5-1)/2)^4.5*exp(-0.076587*(asin(-\x)/180*pi+pi/2))) + 0.2*(1-\x*\x)^0.5});


\draw[domain=-0.29:0.29, variable=\x, color2, samples=200, smooth, ultra thick] plot ({\x}, {2*(1-0.4*exp(-0.076587*(asin(\x)/180*pi+pi/2))) - 0.2*(1-\x*\x)^0.5});


\draw[domain=-0.02:0.02, variable=\y, color2, samples=20, smooth, very thick] plot ({-0.29}, {\y+2*(1-0.4*exp(-0.076587*(asin(-0.29)/180*pi+pi/2))) - 0.2*(1-(-0.29)*(-0.29))^0.5});

\draw[domain=-0.02:0.02, variable=\y, color2, samples=20, smooth, very thick] plot ({-0.10}, {\y+2*(1-0.4*exp(-0.076587*(asin(-0.10)/180*pi+pi/2))) - 0.2*(1-(-0.10)*(-0.10))^0.5});

\draw[domain=-0.02:0.02, variable=\y, color2, samples=20, smooth, very thick] plot ({0.10}, {\y+2*(1-0.4*exp(-0.076587*(asin(0.10)/180*pi+pi/2))) - 0.2*(1-(0.10)*(0.10))^0.5});

\draw[domain=-0.02:0.02, variable=\y, color2, samples=20, smooth, very thick] plot ({0.29}, {\y+2*(1-0.4*exp(-0.076587*(asin(0.29)/180*pi+pi/2))) - 0.2*(1-(0.29)*(0.29))^0.5});


\draw[domain=-0.188:0.188, variable=\x, color2, samples=200, smooth, ultra thick] plot ({\x}, {2*(1-0.4*((5^0.5-1)/2)^2.0*exp(-0.076587*(asin(\x)/180*pi+pi/2))) - 0.2*(1-\x*\x)^0.5});


\draw[domain=-0.02:0.02, variable=\y, color2, samples=20, smooth, very thick] plot ({-0.188}, {\y + 2*(1-0.4*((5^0.5-1)/2)^2.0*exp(-0.076587*(asin(-0.188)/180*pi+pi/2))) - 0.2*(1-(-0.188)*(-0.188))^0.5});

\draw[domain=-0.02:0.02, variable=\y, color2, samples=20, smooth, very thick] plot ({-0.114}, {\y + 2*(1-0.4*((5^0.5-1)/2)^2.0*exp(-0.076587*(asin(-0.114)/180*pi+pi/2))) - 0.2*(1-(-0.114)*(-0.114))^0.5});

\draw[domain=-0.02:0.02, variable=\y, color2, samples=20, smooth, very thick] plot ({-0.0382}, {\y + 2*(1-0.4*((5^0.5-1)/2)^2.0*exp(-0.076587*(asin(-0.0382)/180*pi+pi/2))) - 0.2*(1-(-0.0382)*(-0.0382))^0.5});

\draw[domain=-0.02:0.02, variable=\y, color2, samples=20, smooth, very thick] plot ({0.0382}, {\y + 2*(1-0.4*((5^0.5-1)/2)^2.0*exp(-0.076587*(asin(0.0382)/180*pi+pi/2))) - 0.2*(1-(0.0382)*(0.0382))^0.5});

\draw[domain=-0.02:0.02, variable=\y, color2, samples=20, smooth, very thick] plot ({0.114}, {\y + 2*(1-0.4*((5^0.5-1)/2)^2.0*exp(-0.076587*(asin(0.114)/180*pi+pi/2))) - 0.2*(1-(0.114)*(0.114))^0.5});

\draw[domain=-0.02:0.02, variable=\y, color2, samples=20, smooth, very thick] plot ({0.188}, {\y + 2*(1-0.4*((5^0.5-1)/2)^2.0*exp(-0.076587*(asin(0.188)/180*pi+pi/2))) - 0.2*(1-(0.188)*(0.188))^0.5});


\draw [very thick, ->] (1.2, 2) -- (1.2, 0.1);

\node [color3] at (-0.9, 2.2) {Voters};
\node [color1] at (-0.5, 1.25) {Critical Candidates};
\node [color2] at (0, 1) {Lower-Layer Candidates};
\node [color2] at (0, 1.45) {Higher-Layer Candidates};
\node at (1.45, 1.4) {Decreasing};
\node at (1.45, 1.28) {Preferences};

\node at (1.40, 2.0) {Rank: $0$};
\node at (1.40, 0.1) {Rank: $1$};

\draw [decorate, very thick, color2, decoration={brace,amplitude=5pt}] (-1, {2-0.8}) -- (-1, {2-0.8*(5^0.5-1)/2});
\draw [decorate, very thick, color2, decoration={brace,amplitude=5pt}] (-1, {2-0.8*(5^0.5-1)/2}) -- (-1, {2-0.8*((5^0.5-1)/2)^2});

\node [color2] at (-1.25, {(2-0.8+2-0.8*(5^0.5-1)/2)/2}) {$1^{\text{st}}$ Layer};
\node [color2] at (-1.25, {(2-0.8*(5^0.5-1)/2+2-0.8*((5^0.5-1)/2)^2)/2}) {$2^{\text{nd}}$ Layer};



\end{tikzpicture}
\caption{Construction of the Bad Instance for \g{}}
\label{fig:cylinder}
\end{figure}

    Above each point $i \in S^1 = \{ (x,y) |  x^2 + y^2 = 1\}$ the fibre over $i$ is $\R$. For each point $i$, we can choose a sufficiently small $\epsilon$ and take the open set \[U_i = \{ (x,y) \in S^1 |  (i_x - x)^2 + (i_y - y)^2 \le \epsilon \},\] which may be flattened to $(-(1+\epsilon), 1+\epsilon) \x \R$.
    % \input{TikzDrawings/Ch2/flattened-cylinder-neighbourhood.tikz}
    % So the projection $C \to S^1$ is a vector bundle.
\end{example}
\pagenote{Carried forward the change in chapter 1 re:notation for $\chi(M)$, then defined the $C^\infty$ module structure for the sections of a vector bundle projection (this is a more appropriate place to introduce this notation than the Lie algebroids section)}
%NOTE: Carried forward the change in chapter 1 re:notation for \chi(M), then defined the C^\infty module structure for the sections of a vector bundle projection (this is a more appropriate place to introduce this notation than the Lie algebroids section)
The sections of a vector bundle also give rise to a $C^\infty(M)$-module, generalizing that aspect of the tangent bundle's fibred $\R$-module structure.
\begin{lemma}\label{lem:Cinfty-module-vbun}
  Given a vector bundle $q:E \to M$, write the set of sections of $q$ as $\Gamma(q)$; as, for example, $\Gamma(p.M) = \chi(M)$ (recall the notation from Definition \ref{def:operational-tang}). The set $\Gamma(q)$ has a $C^\infty(M)$-module structure in much the same way as $\chi(M)$:
  \[
      X +_{\Gamma(q)} Y := + \o (X,Y), \hspace{0.25cm}
      0_{\Gamma(q)} := \xi, \hspace{0.25cm}
      (f \cdot_{\Gamma(q)} X) (m) := f(m) \cdot X(m)
  \]
\end{lemma}
There are also a variety of general constructions that yield vector bundles.
\begin{example}
  ~\begin{enumerate}[(i)]
    \item The tangent bundle is a vector bundle: the construction in \Cref{sec:smooth-manifolds} makes it clear that the projection $p:TM \to M$ is a locally trivial, fibred $\R$-module over the base space $M$.
    \item A \emph{trivial} vector bundle over $M$ with fibres in $V$ is the product $M \x V$. In particular, every vector space is a trivial vector bundle above the one-point space $\{*\}$.
    \item Each $T_kM$ will be locally trivial; locally it looks like the $k$-fold product of the tangent space $p_k^{-1}(U) \cong U \x (\R^n)^k$ for an $n$-dimensional manifold $M$. More generally, one can take the fibrewise pullback $E_k = E \ts{q}{q} E \ts{q}{q} \dots \ts{q}{q} E$ and discover a vector bundle over $M$.
    \item The \emph{cotangent bundle} of $M$, $T^*M$,  has the \emph{dual} vector space of $T_mM$ above each point $M$: $T^*_m(M) = (T_mM)^*$. This space can be appropriately topologized to be smooth, and a set of sections of $\Gamma(T^*M)$ is isomorphic to the set of morphisms $TM \to \R$ that are linear in each fibre. This construction may be applied to any vector bundle and is called the dual vector bundle.
    \item Consider the space $\Lambda^n(E)$, the alternating tensor product of $E^*$. The set of sections of this vector bundle is equivalent to the alternating $n$-linear morphism $E_n \to \R$; when restricted to the tangent bundle, this is the space of differential $n$-forms.
  \end{enumerate}
\end{example}
There are two constructions on vector bundles that will be necessary to prove the main theorem of this section.
\begin{proposition}%
  \label{prop:retracts-reindexing-of-vbuns}
  Let $(q:E \to M, \xi, +_q, \cdot_q)$ be a vector bundle. 
  ~\begin{enumerate}[(i)]
    \item For any map $f:N \to M$, the $T$-reindexing of $q$ by $f$ is a vector bundle:
    \begin{equation}%
      \label{eq:reindex-vbun}
      \input{TikzDrawings/Ch2/reindex-vbun.tikz}
    \end{equation}
    % and the morphism $(\bar{f},f)$ is linear.
    \item Any retract of $q$ in the space of arrows is a vector bundle; that is, given
    \begin{equation}%
      \label{eq:ret-of-idemp}
      \input{TikzDrawings/Ch2/ret-of-idemp.tikz}
    \end{equation}
    if there is a vector bundle structure on $q$, then there is a vector bundle structure on $\pi$.
  \end{enumerate}
\end{proposition}

The category of vector bundles has ``locally linear'' bundle morphisms as its maps.
\begin{definition}
  A \emph{morphism of vector bundles} between $q:E \to M$ and $\pi:F \to N$ is a commuting square
  \input{TikzDrawings/Ch2/vbun-morphism.tikz}
  that is \emph{fibrewise linear}, so that above each fibre \[  f|_m: E_m \to F_{v(m)} \] is a linear morphism of vector spaces. This may equivalently be stated as a morphism of fibred $\R$-modules, so that the following diagrams commute:
  \[\input{TikzDrawings/Ch2/vbun-morphism-coh.tikz}\]
\end{definition}
\begin{example}
  ~\begin{enumerate}[(i)]
    \item For the pullback vector bundle in Diagram \ref{eq:reindex-vbun}, the pair $(\bar{f},f)$ is a linear bundle morphism.
    \item For the section/retract vector bundle structure from Diagram \ref{eq:ret-of-idemp}, the section and retract are linear morphisms. 
    Note that this is exactly the splitting of a linear idempotent on $q:E \to M$.
  \end{enumerate}
\end{example}

The lift on the tangent bundle was defined in Section \ref{sec:smooth-manifolds} as
\[
  [\gamma]_\sim \mapsto [\gamma \o \cdot_\R]_\sim. 
\]
Instead, consider the action of $\R$ on a tangent vector:
\[
  ([\gamma]_\sim, r) \mapsto [\gamma \o (r \cdot x)]_\sim.
\]
Note that $T.\cdot$ gives the equation
\[
  T.\cdot \o ([\omega \o (x,y)], [(a,b) \mapsto a + b\cdot x]) = (\omega \o (a \cdot x, a \cdot b\cdot y));
\]
so the lift map $\ell$ can be rederived as follows:
\[\input{TikzDrawings/Ch2/VBun/derive-lift-on-tm.tikz}\]
This general construction is known as the \emph{Euler vector field} of a multiplicative action by $\R^+$.
\begin{definition}%
  \label{def:evf}
  Consider a multiplicative monoid action $a:\R^+ \x E \to E$. The \emph{Euler vector field}\footnote{Somewhat confusingly, the Euler vector field is almost never a vector field.} of the action is the morphism $\lambda:E \to TE$ constructed as follows:
  \[
    \lambda:= E \xrightarrow[]{(id, 1^\R \o !)} E \x \R \xrightarrow{0 \x \lambda} TE \x T\R \cong T(E \x \R) \xrightarrow{T.a} TE.
  \]
  \pagenote{This definition has been tidied up so that is clear that $\lambda$ is defined by the diagram in the definition.}
\end{definition}
Local triviality for the tangent bundle is encoded by the universality of the vertical lift condition. A similar universality condition holds for vector bundles.
\begin{proposition}%
  \label{prop:ros-for-vbun}
  Let $q:E \to M$ be a vector bundle with corresponding Euler vector field $\lambda$.
  Then the following diagram is a $T$-pullback:
  \begin{equation*}
    \input{TikzDrawings/Ch2/ros-universal.tikz}
  \end{equation*}
\end{proposition}
Exploiting the fact that fibred $\R$-modules have subtraction, the following result holds.
\begin{corollary}%
  \label{cor:two-pullbacks-from-ros}
  The following two diagrams are T-equalizers:
  \[
    \input{TikzDrawings/Ch2/strong-cockett.tikz}
  \]
  (recall that $E_2$ is the pullback of a submersion along a submersion and is therefore guaranteed to exist and be preserved by the tangent functor).
\end{corollary}
\begin{proof}
  Given $v: X \to TE$ so that
  \[ T.q \o v = T.q \o 0 \o p \o v \]
  then 
  \[ p \o (v -_{T.q} 0 \o p \o v) =  p \o v -_q p \o v = \xi \o q \o v. \]
  So there is a unique $v'$ so that
  \[
    \lambda \o v' = (v -_{T.q} 0 \o p \o v)
  \]
  meaning that
  \[
    v = 0 \o p \o v  +_{T.q} \lambda \o v'  = \mu(0 \o p \o v, v')
  \] as required. The projection $q$ is a submersion, so the pullback $E_2$ is preserved by the tangent functor, as is the pullback in Proposition \ref{prop:ros-for-vbun}, and the same calculation may be applied for each $T^n$. The proof for $\nu$ follows by the same argument.
\end{proof}
Recall that the class of submersions forms a retractive display system in the category of smooth manifolds (Definition \ref{def:display-system}), so they are stable under reindexing and closed to retracts.\pagenote{
   I had originally used the fact that the pullback $E \ts{q}{q} E$ was preserved without explaining why it was preserved, so I have added a reference to the section on submersions and pointed out that vector bundle projections are submersions (as they are local projections).
}
We may now infer the following:
\begin{corollary}
  The projection for a vector bundle is a submersion.
\end{corollary}
\begin{proof}
  This follows from the fact that $\pi:TE \to E$ is a submersion, so that $q \o \pi_0: E \ts{q}{p} TM \to M$ is a submersion, so the map $q:E \to M$ is a retract of the projection $p.E:TE \to E$ in the arrow category.
\end{proof}
Preservation of the Euler vector field is also sufficient to guarantee that a morphism $f:E \to F$ determines a vector bundle morphism.
\begin{proposition}%
  \label{prop:evf-is-ff}
  Let $q: E \to M, \pi: F \to N$ be a pair of vector bundles with Euler vector fields $\lambda^E,\lambda^F$.
  Then a bundle morphism $(f,v):q \to \pi$ is a vector bundle morphism if and only if \[\lambda^F \o f = T.f \o \lambda^E\]
\end{proposition}
\begin{proof}
  Note that the $\nu^F$ map from Corollary \ref{cor:two-pullbacks-from-ros} is monic, and if $f$ preserves the lift, it preserves $\nu$:
  \[
    \nu^F \o (T.v, f) = + \o (T.\zeta \o T.v, \lambda^F \o f) = + \o (T.f \o T.xi, T.f \o \lambda^E) = T.f \o \nu^E. 
  \]
  Next, observe that $T.v \x f$ is the unique map making the following diagram commute:
  \[\input{TikzDrawings/Ch2/VBun/pres-nu-pres-lambda.tikz}\] 
  Now $\nu$ is a vector bundle morphism and monic, and $T.f$ is a vector bundle morphism, so it follows that $T.v \x f$ is a vector bundle morphism and hence $f$ is also a vector bundle morphism. The reverse implication is immediate.
\end{proof}



\section{Lifts for the tangent weak comonad}%
\label{sec:lifts}
%NOTE: Included some material on weak/semicomonads to clarify some confusion.
The lift $\ell: T \Rightarrow T^2$ gives rise to a \emph{weak comonad}.
% ---an endofunctor equipped with an associative comultiplication map:
% \[
%   \input{TikzDrawings/Ch2/ell-comonad.tikz}
% \]
Weak comonads were introduced in \cite{Wisbauer2013} and have a natural notion of an associative algebra\footnote{
This is not \emph{strictly} true. Wisbauer has a more nuanced hierarchy of almost-monads, and in his language $(T,\ell)$ would be an endofunctor with an associative product.
}. An associative coalgebra of the weak comonad $(T,\ell)$ is called a \emph{lift}; this chapter will demonstrate that lifts provide the essential structure necessary to formulate vector bundles.\pagenote{
Included some material on weak comonads to clear up confusion, as the map $p$ does have the right type for a counit it should be noted that it is not a counit in general.
}
\begin{definition}
  A weak comonad on a category $\C$ is an endofunctor $S: \C \to \C$ equipped with a coassociative map $\delta:S \Rightarrow S.S$:
  % https://q.uiver.app/?q=WzAsNCxbMCwwLCJGIl0sWzEsMCwiRi5GIl0sWzAsMSwiRi5GIl0sWzEsMSwiRi5GLkYiXSxbMCwxLCJcXGRlbHRhIiwxXSxbMCwyLCJcXGRlbHRhIiwxXSxbMSwzLCJGLlxcZGVsdGEiLDFdLFsyLDMsIlxcZGVsdGEuRiIsMV1d
  \[\begin{tikzcd}
    S & {S.S} \\
    {S.S} & {S.S.S}
    \arrow["\delta"{description}, from=1-1, to=1-2]
    \arrow["\delta"{description}, from=1-1, to=2-1]
    \arrow["{S.\delta}"{description}, from=1-2, to=2-2]
    \arrow["{\delta.S}"{description}, from=2-1, to=2-2]
  \end{tikzcd}\]
  An \emph{associative algebra} of a weak comonad is an object $E$ equipped with a map $\lambda:E \to SE$ so that
  % https://q.uiver.app/?q=WzAsNCxbMCwwLCJFIl0sWzEsMCwiRi5FIl0sWzAsMSwiRi5FIl0sWzEsMSwiRi5GLkYiXSxbMCwxLCJcXGxhbWJkYSIsMV0sWzAsMiwiXFxsYW1iZGEiLDFdLFsxLDMsIlxcZGVsdGEuRSIsMV0sWzIsMywiRi5cXGxhbWJkYSIsMV1d
  \[\begin{tikzcd}
    E & {S.E} \\
    {S.E} & {S.S.E}
    \arrow["\lambda"{description}, from=1-1, to=1-2]
    \arrow["\lambda"{description}, from=1-1, to=2-1]
    \arrow["{\delta.E}"{description}, from=1-2, to=2-2]
    \arrow["{S.\lambda}"{description}, from=2-1, to=2-2]
  \end{tikzcd}\]
  A morphism of these algebras is a map $f:(E,\lambda) \Rightarrow (D,\gamma)$ so that
  % https://q.uiver.app/?q=WzAsNCxbMCwwLCJFIl0sWzAsMSwiRkUiXSxbMSwwLCJEIl0sWzEsMSwiRkQiXSxbMCwyLCJmIiwxXSxbMSwzLCJGLmYiLDFdLFswLDEsIlxcbGFtYmRhIiwxXSxbMiwzLCJcXGdhbW1hIiwxXV0=
  \[\begin{tikzcd}
    E & D \\
    S.E & S.D
    \arrow["f"{description}, from=1-1, to=1-2]
    \arrow["{S.f}"{description}, from=2-1, to=2-2]
    \arrow["\lambda"{description}, from=1-1, to=2-1]
    \arrow["\gamma"{description}, from=1-2, to=2-2]
  \end{tikzcd}\]
\end{definition}
Recall that for a full (co)monad, there is an adjunction between the base category and the category of (co)algebras. That result is weakened in this case: 
\begin{lemma}
  For every weak comonad on a category $\C$, there is a free coalgebra functor
  \[
      F: \C \to \mathsf{CoAlg}(\C); E \mapsto (S.E, \delta:S.E \to S.S.E)
  \]  
  an underlying object functor
  \[
      U: \mathsf{CoAlg}(\C) \to \C; (E, \lambda) \mapsto E
  \]
  and a natural transformation
  \[
      \lambda:id \Rightarrow F.U; % https://q.uiver.app/?q=WzAsNCxbMCwwLCJFIl0sWzAsMSwiRkUiXSxbMSwwLCJEIl0sWzEsMSwiRkQiXSxbMCwyLCJmIiwxXSxbMSwzLCJGLmYiLDFdLFswLDEsIlxcbGFtYmRhIiwxXSxbMiwzLCJcXGdhbW1hIiwxXV0=
      \begin{tikzcd}
        E & S.E \\
        S.E & S.S.E
        \arrow["\lambda"{description}, from=1-1, to=1-2]
        \arrow["{S.\lambda}"{description}, from=2-1, to=2-2]
        \arrow["\lambda"{description}, from=1-1, to=2-1]
        \arrow["\delta"{description}, from=1-2, to=2-2]
      \end{tikzcd}
  \]
\end{lemma}
\begin{definition}\label{def:lift}
  A \emph{lift} in a tangent category $\C$ is an associative coalgebra of $(T,\ell)$, namely, a pair $(E, \lambda: E \to TE)$ so that the following diagram commutes:
  \[
    \input{TikzDrawings/Ch2/lift-diag.tikz}
  \]
  A morphism of lifts is a coalgebra morphism. The category of lifts and lift morphisms in a tangent category $\C$ is written $\mathsf{Lift}(\C)$.
\end{definition}
Note that the tangent bundle is \emph{not}, in general, a comonad: while the tangent projection has the correct type for a counit, $p:T \Rightarrow id$, it does not satisfy $p \o \ell = id$. In fact, if $p$ were a counit, this would force $id = p \o \ell = 0 \o p$ so that $0$ = $p^{-1}$, thus if $(T,\ell,p)$ is a comonad then $T$ is naturally isomorphic to the identity functor.
\begin{example}
  \label{ex:lift-examples}
  ~\begin{enumerate}[(i)]
    \item For every object $M$ in a tangent category $\C$, the pair $(TM, \ell:TM \to T^2M)$ is a lift, called the \emph{free lift} on $M$.
    \item Every object $M$ in a tangent category has a \emph{trivial} lift, $0:M \to TM$, where $T.0 \o 0 = \ell \o 0.$
    \item Every differential object has a lift $\lambda$; the coherence is equivalent to axiom $[D0.3]$ in Definition \ref{def:differential-object}. \pagenote{
        I have added the map $\lambda:E \to TE$ from the definition of a differential object, and the exact differential object axiom that makes it a lift.
    }
    \item %NOTE: 
    \pagenote{I moved the Euler vector field into the examples, fixed notation, and tried to clarify some points Kristine brought up. I make explicit use of the Kock-Lawvere axiom here, so that $T\R$ is $R[x]/x^2$ and $T^2\R$ is $R[x]/(x^2,y^2)$.}
    The Euler vector field of a multiplicative $\R^+$-action $h:R^+ \x E \to E$ in $\mathsf{SMan}$ is a lift.
    % \end{proposition}
    % \begin{proof}
      Recall that the \textit{Euler vector field} over the scalar action $s_M: TM \x \R \to \R$ induces the vertical lift on a manifold:
      \[
          \ell = T.s_M \o (0, \lambda^\R \o 1^{\R} \o !).
      \]
      In the category of smooth manifolds, $T\R \cong \R[x]/x^2$, where $\lambda(r) = [x \mapsto r\cdot x]$ corresponds to the map $\lambda'(r) = 0 + r\cdot x$. Similarly, there is an isomorphism $\R[x,y]/(x^2,y^2)$ so that for the maps $0.T and T.0$,
      \[
          0.T(a + r\cdot x) \cong a + r\cdot x + 0y + 0xy,\hspace{0.5cm}
          T.0(a + r\cdot x) \cong a + 0x + r\cdot y + 0xy.
      \]
      Since we know that $(0 + x)(0 + y) = (0 + xy)$ in $\R[x,y]/(x^2,y^2)$ (following \ref{ex:diffob-sman}), we can use these isomorphisms to see that
      % In the category of smooth manifolds, the scalar ring $\R$ has the following universal property on its lift:
      \[
          \ell \o \lambda^\R \o 1^\R \o != 
          (0.T \o \lambda^\R \o 1^\R \o !) \cdot_{T^2.\R} (T.0\o \lambda^\R\o 1^\R \o !).
      \]
      % following from the fact that $T^2\R$ is isomorphic as a ring to $R[x,y]/(x^2,y^2)$ and $(0 + x)(0 + y) = (0 + xy)$.
      % \footnote{This is in fact one of the axioms of synthetic differenta}
  
      Consider a monoid action $(\R^+, h)$ on a manifold $E$. The Euler vector field of this action, 
      \[
          \lambda: E \xrightarrow[]{(id, 1^\R \o !)} E \x \R \xrightarrow{(0,\lambda^\R)} TE \x T\R \xrightarrow{T.h} TE
      \]
      will define an algebra if the induced scalar action on $TE$ commutes with the natural scalar action:
      \begin{align*}
          T.\lambda \o \lambda
          &= T^2.h \o (T.0 \o\lambda, T.\lambda \o 0\o 1^\R \o !) \\
          &= T^2.h \o (T.0 \o T.h \o (0, \lambda \o 1^\R \o !), 0 \o \lambda \o 1^\R \o !) \\
          &= T^2.h \o (T^2.h\o (T0\o 0, T0 \o \lambda \o 1^\R \o !), 0 \o \lambda \o 1^\R \o !) \\
          &= T^2.h \o (T.0 \o 0, (T.0 \o \lambda \o 1^\R \o !) \cdot_{T^2.\R}( 0.T \o  \lambda \o 1^\R \o !) ) \\
          &= T^2.h \o (\ell \o 0, \ell \o \lambda \o 1^\R \o !)  \\
          &= \ell \o T.h \o  (0, \lambda \o 1^\R \o !) \\
          &= \ell \o \lambda.
      \end{align*}
      % so then any splitting of $p \o \lambda$ induces a pre-differential bundle (which corresponds to the image of $h \o (id, 0^\R)$). 
    % \end{proof}
    % \item Returning to the category $\mathsf{Lex}$, where $TE$ is the category of Beck modules in $E$ and $\ell$ sends a Beck module to the double Beck module:
    % \input{TikzDrawings/Ch2/dbl-Beck.tikz}
    % Now, a coalgebra is a map that sends an object in $E$ to a Beck module, so that $\ell \o \lambda = T.\lambda \o \lambda$:
    % \input{TikzDrawings/Ch2/beck-module-example.tikz}
    % It follows that $\lambda (X_1) = \lambda(X)$, and $\lambda(X_0) = 0(X_0)$. 
    % This means that a coalgebra sends an object $X$ to a Beck module $X_1 \to X_0$, so that the choice for $X_0$ is the identity bundle and the choice for $X_1$ is this bundle (e.g. it is the same as the original $X$).
  \end{enumerate}
\end{example}
\begin{observation}\label{obs:evf-is-ff}
  Recall that by Proposition \ref{prop:evf-is-ff}, morphisms preserve a monoid action if and only if they preserve the associated Euler vector field of the action. This means the Euler vector field construction gives a fully faithful functor from monoid actions to lifts in the category of smooth manifolds, and therefore from the category of vector bundles to the category of lifts in $\mathsf{SMan}$.
\end{observation}

The following proposition gives a pair of constructions on lifts---closure under the tangent functor and finite $T$-limits---that will be useful in this section.
\begin{lemma}%
  \label{lem:T-limits-of-lifts}
  Let $\C$ be a tangent category.
  ~\begin{enumerate}[(i)]
    \item The tangent functor lifts to an endofunctor on the category of lifts in $\C$.
    % The tangent functor preserves lift and lift morphisms: given a lift $\lambda:E \to TE$, then $c \o T.\lambda:TE \to TTE$ is a lift.
    \item Given a diagram \pagenote{
       I have clarified the statement and make the proof more concrete, I have also moved this result to a lemma and made the next result a proposition rather than corollary.
    }
    % Observe that for any diagram into the category of lifts:
    \[ D: \d \to \mathsf{Lift}(\C)\]
    in the category of lifts of $\C$, if the $T$-limit of $U.D$ exists in $\C$, then $\lim U.D$ has a natural lift $\lambda'$ associated to it so that $(\lim U.D, \lambda')$ is the limit of $D$ in $\mathsf{Lifts}(\C)$. (That is, $T$-limits of lifts are computed pointwise in the base category.)
    \pagenote{
       The proof for the 
    }
    % There will be a natural transformation:
    % \[\input{TikzDrawings/Ch2/nat-trans.tikz}\]
    % This induces a morphism:
    % \[
    %     \lim\lambda: \lim (U^\ell.D) \to \lim (T.U^\ell.D) \cong T.\lim(U^\ell.D)
    % \] that gives the induced lift on the limit of $\lim (U^\ell.D)$ - this lift is the limit $\lim D$. 
    % Finite limits of lifts are computed point-wise on the objects.
  \end{enumerate}
\end{lemma}
\begin{proof}
  ~\begin{enumerate}[(i)]
    \item Simply check that 
    \begin{gather*}
      T.(c \o T.\lambda) \o c \o T.\lambda = T.c \o T^2.\lambda \o c \o T.\lambda  = T.c \o c.T \o T^2.\lambda  \\= T.c \o c.T \o T.\ell \o T.\lambda = \ell.T \o c \o T.\lambda. 
    \end{gather*}
    \item 
    Concretely, a tangent terminal object will have a lift:
    \[
        (1, 1 \xrightarrow[]{0} T.1 \cong 1).
    \] 
    Given $(E,\lambda)$ and $(F,l)$, if the tangent product $E \x F$ exists there is a lift
    \[
      (E \x F, E \x F \xrightarrow[]{\lambda \x l} TE \x TF \cong T(E \x F)).
    \]
    Given the $T$-equalizer of a fork $f,g:(E,\lambda) \to (F,l)$, the equalizer has a lift induced as follows:
    \[
      \input{TikzDrawings/Ch2/eq-of-lifts.tikz}
    \]
    % And given a span of lifts: $(E,\lambda) \xrightarrow[]{f} (C,k) \xleftarrow[]{g} (F,l)$, if the $T$-pullback $E \ts{f}{g} F$ exists then the induced lift is 
    % \[
    %   (E \ts{f}{g} F, \lambda \x l:E \ts{f}{g} F \to T(E \ts{f}{g} F))
    % \]
  \end{enumerate}
\end{proof}
\begin{proposition}\label{prop:lifts-is-tangent}
  The category of lifts is a tangent category.
\end{proposition}
\begin{proof}
  The tangent functor sends 
  \[
    \infer{T.f: (TE, c \o T.\lambda) \to (TF, c \o T.l)}{f:(E,\lambda) \to (F,l)}.
  \]
  To see that this is still an algebra morphism, compute
  \[
    c \o T.l \o T.f = c \o T^2.f \o T.\lambda = T^2.f \o c \o T.\lambda
  \]
  The structure maps are the structure maps on the underlying object of the lift; the universality conditions follow by Proposition \ref{lem:T-limits-of-lifts}.
\end{proof}

% \begin{example}%
%   \label{ex:lex-dbun}
%   In the tangent category $\mathsf{Lex}$, where $TE$ is the category of Beck modules in $E$, the lift $\ell$ sends a Beck module to the double Beck module:
%   \input{TikzDrawings/Ch2/dbl-Beck.tikz}
%   Now, a coalgebra is a map that sends an object in $E$ to a Beck module, so that $\ell \o \lambda = T.\lambda \o \lambda$:
%   \input{TikzDrawings/Ch2/beck-module-example.tikz}
%   This diagram forces $\lambda (X_1) = \lambda(X)$, and $\lambda(X_0) = 0(X_0)$. Therefore, a coalgebra sends an object $X$ to a Beck module $X_1 \to X_0$, so that the choice for $X_0$ is the identity bundle and the choice for $X_1$ is this bundle (e.g. it is the same as the original $X$).
% \end{example}

The following idempotent is key in the theory of lifts and will be used in defining non-singular lifts (Definition \ref{def:non-singular-lift}), and its splitting will present the projection and zero-section of a vector bundle (Definition \ref{def:pdb}).

\begin{proposition}%
  \label{prop:idempotent-natural}
  The category of lifts in a tangent category $\C$ has a natural idempotent:
  \[
      e: id \Rightarrow id; e_{(E,\lambda)}: (E,\lambda) \xrightarrow[]{p \o \lambda} (E,\lambda).
  \]\pagenote{
      This proposition originally included some ambiguities, these have been handled by earlier changes that make the notiong of lift and lift-morphism more concrete. 
  }
\end{proposition}
\begin{proof}
  First, we see that $e = p \o \lambda$ is an idempotent:
  \[ p \o \lambda \o p \o \lambda
  = p\o p.T \o T.\lambda \o \lambda \\
  = p\o p.T \o\ell\o \lambda \\
  = p \o 0 \o p\o \lambda \\
  = p \o \lambda.\]
  Moreover, every $f:(E,\lambda) \to (F,l)$ preserves the idempotent:
  \[
    f \o p \o \lambda = p \o T.f \o \lambda = p \o l \o f.
  \]
  Finally, note that the idempotent is a lift morphism.:
  \[
    T.\lambda \o \lambda = \ell \o \lambda = c \o \ell \o \lambda = c \o T.\lambda \o \lambda
  \]
  which implies that 
  \[
    \lambda \o e = \lambda \o p \o \lambda = p \o T.\lambda \o \lambda = p \o c \o T.\lambda \o \lambda = T.p \o T.\lambda \o \lambda = T.e \o \lambda.
  \]
\end{proof}

% This idempotent is essential in characterizing Grabowski's non-singularity condition in Definition \ref{def:non-singular-r-module} in an arbitrary tangent category.
% While it is not immediate that this is equivalent to the notion of a non-singular $R$-algebra, this will be proved later in the chapter.

\section{Non-singular lifts}%
\label{sec:non-singular-lifts}
\cite{Grabowski2009} introduced the notion of a non-singular lift as a means to axiomatize the Euler vector field of a vector bundle's multiplicative $\R^+$-action. While it is not immediately clear that our definition is the same as Grabowski's, the results of Section \ref{sec:iso-vbun-dbun} will justify the use of this language as they are necessarily the same.
\begin{definition}%
  \label{def:non-singular-lift}
  A lift $(E,\lambda)$ in a tangent category $\C$ is \emph{non-singular} whenever the following diagram is a $T$-equalizer:
  \[
    \input{TikzDrawings/Ch2/reg-lift-eq.tikz}
  \]
  where $e.E = p \o \ell$ (the idempotent associated to the free lift on $E$) and $T.e = T.p \o T.\lambda$ (the image of the idempotent associated to $(E,\lambda)$ under the tangent functor). The category of non-singular lifts is written $\mathsf{NonSing}(\C)$.\pagenote{
  I have added comments clarifying what maps $e.E$ and $T.e$ are. 
  }
\end{definition}
The most prominent class of examples is given by the Euler vector field of the $\R$-action on a vector bundle.
\begin{proposition}%
  \label{prop:evf-vbun-is-nonsingular}
  The Euler vector field of a vector bundle is a non-singular lift.
\end{proposition}
\begin{proof}
  %NOTE: Attempted to clarify some points of confusion raised by Kristine
  \pagenote{I made it clear that the diagram being discussed is a $T$-limit, as it is not actually an equalizer diagram. I have tried to clarify the wording that ``one diagram is universal if and only if the other is''.}
  Let $(q:E \to M, +, \xi, \cdot)$ be a vector bundle with Euler vector field $\lambda$. By Proposition \ref{prop:ros-for-vbun}, the diagram \[\input{TikzDrawings/Ch2/ros-eq.tikz}\] is a $T$-limit. The $T$-universality of this diagram will hold if and only if the diagram is universal after each parallel pair of arrows is post-composed by a $T$-monic. A section is a $T$-monic, so the previous diagram is $T$-universal if and only if the following diagram is $T$-universal:
  % Using the fact that equalizers may be post-composed with $T$-monics, the following diagram is equivalent to the others' universality.
  \input{TikzDrawings/Ch2/ros-monic-postcompose.tikz}
  Now simplify this diagram using the fact that $T.(\xi \o q) = T.e, 0 \o p = e.E$:
  \input{TikzDrawings/Ch2/ros-new-double-eq.tikz}
  Note that the the two pairs of parallel arrows have a common arrow, implying that they may be pulled together into a single ternary equalizer. All that remains to check, then, is that for any $x:X \to TE$,
  \[
  (e.e \o x = T.e \o x = e.E \o x) \iff 
  (T.e \o x = e.E \o x). 
  \]
  The forward implication is trivial, so it remains to prove the reverse. 
  Suppose $T.e \o x = e.E \o x$; then
  \[
    e.e \o x = e.E \o T.e \o x = e.E \o e.E \o x = e.E \o x
  \]
  giving the result, namely that the diagram \[\input{TikzDrawings/Ch2/reg-lift-eq.tikz}\] is a $T$-equalizer.
\end{proof}
Every map $f:E \to F$ gives a map of free coalgebras $T.f: (TE,\ell) \to (TF,\ell)$ and the idempotent $e$ is a coalgebra morphism by Proposition \ref{prop:idempotent-natural}, so the following is immediate:
\begin{proposition}%
  \label{def:non-singular-lift-eq-of-lifts}
  A non-singular lift is an equalizer in the category of lifts:
  \[
    \input{TikzDrawings/Ch2/reg-lift-eq-lin.tikz}
  \]
\end{proposition}
% \begin{observation}%
%   \label{obs:fork-is-nat}
%   Note that the coalgebra definition means there is a natural transformation:
%   \[
%     \lambda: id \Rightarrow F.U    
%   \]
%   $U$ sends a lift to its underlying object in $\C$, $U(E,\lambda) = E$ and $F$ sends an object to its free lift $F(E) = (TE,\ell)$. By definition, 
%   \[
%     \input{TikzDrawings/Ch2/lift-diag.tikz}
%   \]
% exihibits $\lambda:(E,\lambda) \to (TE,\ell)$ as a coalgebra morphism.
%   Therefore, the fork defining non-singularity is natural (using that $F.U = id.F.U = F.U.id$):
%   \[
%     \input{TikzDrawings/Ch2/nat-fork-reg.tikz}
%   \]
% \end{observation}

Observe that the category of non-singular lifts is closed under finite limits in the category of lifts.
\begin{proposition}%
  \label{prop:nonsing-closed-under-t-limits}
  The category of non-singular lifts in $\C$ is closed under $T$-limits:
  \begin{enumerate}[(i)]
    \item The tangent functor on lifts preserves non-singular lifts, so that if $(E,\lambda)$ is non-singular then $(TE,c\o T.\lambda)$ is non-singular. \pagenote{
        I clarified that this was in fact referring to the tangent functor \emph{for the category of lifts}.
    }
    \item The trivial lift on an object, $0:M \to TM$, is non-singular. 
    % \item If $\C$ has a tangent terminal object is a non-singular lifts with $(1, 0:1 \cong T.1)$
    \item $T$-products of non-singular lifts are non-singular lifts.
    \item $T$-equalizers of non-singular lifts are non-singular lifts.
  \end{enumerate}
\end{proposition}
\begin{proof}
  ~\begin{enumerate}[(i)]
    \item This follows from the fact that the non-singularity condition is a $T$-limit.
    \item The zero map splits the idempotent $0 \o p$, so it is the equalizer of $0 \o p, p \o 0 = id$.
    \item This follows by stability of limits under products.
    \item The following diagram commutes by naturality:
    \[\input{TikzDrawings/Ch2/eq-pres-ros.tikz}\]
    Each horizontal diagram is a $T$-equalizer, and the two columns on the right are $T$-equalizers, so the column on the left is a $T$-equalizer.
  \end{enumerate}
\end{proof}


Finally, when a tangent category has certain $T$-equalizers, there is an idempotent monad on the category of lifts, whose algebras are non-singular lifts:
\begin{theorem}%
  \label{thm:idemp-monad-nonsingular}
  Let $\C$ be a tangent category with chosen $T$-equalizers of idempotents. Then the following equalizer determines a left-exact idempotent monad on the category of lifts, whose algebras are non-singular lifts.\pagenote{
      I have reworded this proposition to make it clear that the functor is defined by sending a lift to the lift defined by the equalizer in the following diagram.
  }
% https://q.uiver.app/?q=WzAsMyxbMCwwLCIoQyxsKSJdLFsxLDAsIihULkUsXFxlbGwpIl0sWzIsMCwiKFQuRSwgXFxlbGwpIl0sWzAsMV0sWzEsMiwiZS5FIiwwLHsib2Zmc2V0IjotMX1dLFsxLDIsIlQuZSIsMix7Im9mZnNldCI6MX1dXQ==
\[\begin{tikzcd}
	{(F,l)} & {(T.E,\ell)} & {(T.E, \ell)}
	\arrow[from=1-1, to=1-2]
	\arrow["{e.E}", shift left=1, from=1-2, to=1-3]
	\arrow["{T.e}"', shift right=1, from=1-2, to=1-3]
\end{tikzcd}\] 
\end{theorem}
\begin{proof}
  First, take the equalizer in $\mathsf{Lift}(\C)$; the functor sends $(E,\lambda)$ to the chosen limit $(F,l)$.The unit of the monad is the unique morphism from $(E,\lambda)$ to the equalizer $(F,l)$ induced by universality:
  \[\input{TikzDrawings/Ch2/idem-monad-unit.tikz}\]
  Note that non-singular lifts are closed under finite limits, so $(F,l)$ is a non-singular lift. If $(E,\lambda)$ is a nonsingular lift then $\lambda: (E,\lambda) \to (TE, \ell)$ equalizes the diagram, so there is a unique isomorphism $(F,l) \cong (E,\lambda)$, making the multiplication of the monad a natural isomorphism (and thus yielding an idempotent monad).
  The functor is defined as a $T$-limit and therefore preserves all $T$-limits of lifts, so it is left-exact.
\end{proof}
% \begin{remark}
%   It is unclear whether or not this equalizer always exists for the category of smooth manifolds.%something about transversality? Maybe this works!
% \end{remark}
In the category of smooth manifolds, $\ell$ is the Euler vector field of an $\R^+$-action, this guarantees that every $\lambda$ is the Euler vector field of a multiplicative $\R^+$-action.  We note the following corollary.
\begin{corollary}%
  \label{cor:non-singular-implies-ract}
  In the category of multiplicative $\mathbb{R}^+$ actions in $\mathsf{SMan}$, multiplication by $0$ is equivalent to the natural idempotent $e$ in the fully faithful functor sending an $\mathbb{R}^+$-action to its Euler vector field. By non-singularity, the following diagram is an equalizer, giving $E$ a multiplicative action by $\mathbb{R}^+$ whose Euler vector field is $\lambda$: 
  \[
    \input{TikzDrawings/Ch2/reg-induces-action.tikz}
  \]
  Moreover, $\lambda$ is the Euler vector field of this lift, and the following diagram commutes:
  \[
    \input{TikzDrawings/Ch2/induced-action-is-lift.tikz}
  \]
\end{corollary}





\section{Differential bundles}%
\label{sec:lifts-pdbs-dbs}

This section introduces (pre-)differential bundles, which provided the rest of the data for a vector bundle: namely the projection, the zero section, and the addition map. The zero section and projection data arise by splitting the natural idempotent $e: id \Rightarrow id$, and non-singularity will induce the addition map. Every differential bundle satisfies a pair of universality diagrams, linking this presentation of differential bundles to the original definition in \cite{Cockett2018}.

\begin{definition}%
  \label{def:pdb}
  ~\begin{enumerate}[(i)]
    \item A \emph{pre-differential bundle} is a lift $\lambda:E \to TE$ equipped with a chosen splitting of the natural idempotent $e = p \o \lambda$ from Proposition \ref{prop:idempotent-natural}. Pre-differential bundles are formally written $(q:E \to M, \xi, \lambda)$, where $q:E \to M$ is the retract, $\xi:M \to E$ the section, and $\lambda:E \to TE$ the lift (the types are only necessary for the projection as the rest may be inferred, and will generally be suppressed to save space).
    \item A \emph{differential bundle} is a pre-differential bundle $(q:E \to M, \xi, \lambda)$ with the properties that $\lambda$ is non-singular and $T$-pullback powers of $q$ exist.
  \end{enumerate}
  Morphisms of (pre-)differential bundles are exactly morphisms of their underlying lifts. The categories of (pre-)differential bundles are exactly (pre-)differential bundles and lift morphisms, and are written $\mathsf{Pre}(\C), \mathsf{DBun}(\C)$ respectively.\pagenote{
      The definition of differential bundles has been expanded so that it is clear that $\xi,q$ are and idempotent splitting of $p \o \lambda$, and the category names $\mathsf{Pre}(\C), \mathsf{DBun}(\C)$ are defined explicitly to help draw attention to the definition of bundle morphisms.
  }
\end{definition}

Recall that as $e$ is a natural idempotent in the category of lifts, any lift morphism will preserve $e$ and will consequently preserve its idempotent splitting. Preserving the idempotent means that every differential bundle morphism is a bundle morphism, where the base map is given by \[\input{TikzDrawings/Ch2/pdb-bundle-morphism.tikz}\]
We now look at the limits of (pre-)differential bundles.
\begin{observation}%
  \label{obs:T-limits-pdbs}
  ~\begin{enumerate}[(i)]
    \item The limit for a diagram of pre-differential bundles is the limit of the underlying lifts equipped with a chosen splitting of $p \o \lambda$ due to basic properties about idempotent splittings.
    \item The limit for a diagram of differential bundles is the limit in the category of pre-differential bundles (the lift will be universal by Proposition \ref{prop:nonsing-closed-under-t-limits}), so long as $T$-pullback powers of the resulting projection exist.
  \end{enumerate}
\end{observation}
Because there is a projection associated to a (pre-)differential bundle, the $T$-reindexing operation described in Proposition \ref{prop:submersion-properties} can now be applied. This gives a pullback differential bundle in a similar way to Proposition \ref*{prop:retracts-reindexing-of-vbuns}.
\begin{lemma}[\cite{Cockett2018}]%
  \label{lem:reindex-db}
  Let $(q:E \to M, \xi, \lambda)$ be a (pre-) differential bundle in a tangent category $\C$, and consider a $T$-pullback in $\C$:
  \[\input{TikzDrawings/Ch2/T-reind-dbun.tikz}\]
  Then the induced triple maps
  \[\input{TikzDrawings/Ch2/T-induce-maps.tikz}\]
  induce a (pre-)differential bundle $(u^*q: u^*E \to N, u^*\xi, u^*\lambda)$. If $\lambda$ is non-singular and $T$-pullback powers of $u^*q$ exist, then $(u^*q, u^*\xi, u^*\lambda)$ is a differential bundle.
\end{lemma}
\begin{proof}
  Note that a pre-differential bundle $(q:E \to M, \xi, \lambda)$ in $\C$ may also be regarded as a pre-differential bundle $(q: (E,\lambda) \to (M, 0), \xi, \lambda)$ in $\mathsf{Lift}(\C)$. Take the following pullback in $\mathsf{Lift}(\C)$:
  \[\input{TikzDrawings/Ch2/pullback-pdb.tikz}\]
  It follows by construction that $\iota \o u^*\lambda = \lambda \o \iota$. Thus the result holds.
\end{proof}

\begin{proposition}%
  \label{prop:induce-abun}
  Let $(q:E \to M, \xi, \lambda)$ be a non-singular pre-differential bundle in a tangent category $\C$, so that $T$-pullback powers of $q$ exist.
  Then there is an additive bundle structure $(q,\xi,+_q)$ so that the differential bundle morphisms are additive:
  \[
    (\lambda,\xi): (q,\xi,+_q) \to (p,0,+) \hspace{0.5cm}
    (\lambda,0): (q,\xi,+_q) \to (T.q,T.\xi,T.+_q).
  \]
  Furthermore, every differential bundle morphism preserves addition.
\end{proposition}
\begin{proof}
  Non-singularity forces the existence of an addition map: 
  \[
    \input{TikzDrawings/Ch2/regularity-induces-addition.tikz}
  \]
  Note that this diagram commutes because $T_2E$ is a pre-differential bundle whose lift is $\ell \x \ell$, and also $+$ is a linear morphism, so it commutes with the addition map $e\o a + e\o b = e \o (a\x b)$. Post-composition with $\lambda$ ensures that $\xi$ is the unit and that associativity holds. A differential bundle morphism will induce a morphism of the equalizer diagrams that induce each addition map to preserve addition.
\end{proof}

Recall that by Proposition \ref{prop:evf-vbun-is-nonsingular}, the Euler vector field for every vector bundle is a non-singular lift.\pagenote{
    The reference to Proposition \ref{prop:evf-vbun-is-nonsingular} has been put into the middle of the sentence to help. Also, rather than ``addition induced by singularity'', we give an explicit reference to the above proposition that constructs the additive bundle structure.
} If $TM \ts{q}{q} E$ exists, the map
\[
  \nu^E: TM \ts{p}{q} E \xrightarrow[]{T.\xi \x \lambda} T_2E \xrightarrow[]{+.E} TE
\] may be formed. Similarly, using the additive bundle structure from Proposition \ref{prop:induce-abun}, the $\mu$ map may be formed: 
\[
  \mu^E: E \ts{q}{q} E \xrightarrow[]{0 \x \lambda} T(E\ts{q}{q}E) \xrightarrow[]{T.+_q} TE
\]
Note that any differential bundle morphism will preserve $\mu$ and $\nu$.
\begin{lemma}%
  \label{lem:pre-mu-nu}
  Differential bundle maps preserve $\mu(x,y) := 0 \o x +_{T.q} \lambda \o y$ and $\nu(v,y) := T.\xi \o v +_p \lambda \o y$.\pagenote{
     The word linear had crept into this sentence in the original draft, it has been correct to ``differential bundle morphism''.
  }
\end{lemma}
\begin{proof}
  The following diagram demonstrates that lift maps preserve $\nu$:
  \[\input{TikzDrawings/Ch2/lin-pres-nu.tikz}\]
  Similarly, this diagram demonstrates that lift maps preserve $\mu$:
  \[\input{TikzDrawings/Ch2/lin-pres-mu.tikz}\]
\end{proof}

\begin{lemma}%
  \label{lem:univ-props}
  Consider the full subcategory of differential bundles in $\C$ whose objects are differential bundles $(E,\lambda)$ so that the forks
  \begin{equation}%
    \label{eq:universality}
    \input{TikzDrawings/Ch2/strong-cockett.tikz}
  \end{equation}
  are $T$-equalizers. This subcategory is closed under $T$-equalizers.\pagenote{
     The original wording for this lemma did not properly quantify the objects over which the theorem was stated for.
  }
\end{lemma}
\begin{proof} 
  Start with the $T$-equalizer of lifts:
  \[\input{TikzDrawings/Ch2/coeq-for-numu.tikz}\]
  Observe that $k$ is a lift map, so by Lemma \ref{lem:pre-mu-nu}, the following diagram commutes:
  \input{TikzDrawings/Ch2/lin-pres-nu-2.tikz}
  Next, since $k$ is a morphism of pre-differential bundles, by Lemma \ref{lem:pre-mu-nu} the following diagram commutes:
  \input{TikzDrawings/Ch2/eq-pres-mu.tikz}
  The top row follows because maps satisfying Rosicky's universality condition preserve $\mu$, and the bottom row by the naturality of $e.C$ and the fact that linear maps preserve the natural idempotent. Thus, if $E,F$ satisfy the universality diagrams in Diagram \ref{eq:universality}, the equalizer $C$ will as well, because $T$-equalizers are closed to $T$-limits in the category of fork diagrams.
\end{proof}
\begin{theorem}%
  \label{thm:universal-prop-differential-bundles}
  For every differential bundle $(q:E \to M, \xi,\lambda)$ in a tangent category, the diagram \[\input{TikzDrawings/Ch2/cc-equalizer.tikz}\] is a $T$-equalizer, and if $TM \ts{p}{q} E$ exists, then \[\input{TikzDrawings/Ch2/strong-eq.tikz}\] is a $T$-equalizer.
\end{theorem}
\begin{proof}
  The tangent bundle satisfies both universality conditions, and by Lemma \ref{lem:univ-props} every differential bundle will satisfy these conditions by the non-singularity of the lift.
\end{proof}
%   Recall that
%   \[
%     \nu^E: TM \ts{p}{q} E \xrightarrow[]{T.\xi \x \lambda} T_2E \xrightarrow[]{+.E} TE, \hspace{0.15cm}
%     \mu^E: E \ts{q}{q} E \xrightarrow[]{0 \x \lambda} T(E\ts{q}{q}E) \xrightarrow[]{T.+_q} TE
%   \]
\begin{corollary}%
  \label{cor:idemp-dbun}
  In a complete tangent category, the category of differential bundles is precisely the category of algebras of the monad in Theorem \ref{thm:idemp-monad-nonsingular} on pre-differential bundles.
\end{corollary}
\begin{remark}
  The universality conditions in Theorem \ref{thm:universal-prop-differential-bundles} demonstrate that this definition of differential bundle agrees with that of \cite{Cockett2018}.%, when combined with some results in \cite{MacAdam2021}.
  \pagenote{I have removed a reference to the published version of this chapter, it was unnecessary.}
\end{remark}
%NOTE: 
% This section bridges the gap between non-singular lifts and differential bundles and shows that these are the same. Every differential bundle has an underlying regular lift, which uniquely determines the rest of the structure of the differential bundle. Pre-differential bundles provide the first step towards reconciling differential bundles - a pre-differential bundle is to a differential bundle what a lift is to a non-singular lift.

\section{The isomorphism of categories}%
\label{sec:iso-vbun-dbun}
It is now straightforward to show that the main theorem of this chapter holds. First, observe that for every differential bundle $(q:E \to M, \xi, \lambda)$ in the category of smooth manifolds, the $T$-pullback $E \ts{p}{q} TM$ exists\pagenote{I have switched the order of the pullback for consistency}
, as $p$ is a submersion. Note that the universality of the two diagrams is equivalent:
\[
  \input{TikzDrawings/Ch2/strong-eq.tikz} 
  \hspace{0.15cm}
  \input{TikzDrawings/Ch2/strong-universality.tikz}
\]%NOTE 
Thus the following holds.
\begin{theorem}%
  \label{iso-of-cats-dbun-sman}
  There is an isomorphism of categories between vector bundles and differential bundles in smooth manifolds. 
\end{theorem}
\begin{proof}
  Note that Proposition \ref{prop:evf-vbun-is-nonsingular} gives a fully faithful functor from the category of vector bundles to differential bundles of smooth manifolds, as the lift associated to the vector bundle is non-singular and the projection and zero section give the rest of the structure of a differential bundle: as remarked after Definition \ref{def:vector-bundle}, local triviality guarantees that the projection is a submersion, so pullback powers of the projection exist, yielding a differential bundle.

  To see there is an isomorphism on objects, recall that every differential bundle is a fibred $\R$-module by Corollary \ref{cor:non-singular-implies-ract}, and that this identification is a bijective mapping (it recovers the original $\R$-action from the Euler vector field of the $\R$-action, and vice versa). Last, note that the universality condition
  \[\input{TikzDrawings/Ch2/strong-universality.tikz}\]
  along with Proposition \ref{prop:retracts-reindexing-of-vbuns} ensures the local triviality of $q$, so that the unique fibred $\R$-module structure associated to a differential bundle is indeed a vector bundle.
  \pagenote{I have added details to this proof to clarify that it does give a bijection on the classes of objects.}
  %NOTE: Changed the wording to make it more clear there is a bijection on objects between vector bundles and differential bundles 
\end{proof}
% And the promised relationship between non-singular lifts and non-singular $R$-algebras.
% \begin{corollary}
%   There is an isomorphism of categories between non-singular lifts and non-singular $R$-algebras.
% \end{corollary}
% \begin{proof}
%   By Corollary \ref{cor:non-singular-implies-ract}, each non-singular lift $(E,\lambda)$ has an $R$-algebra structure, and every splitting of $p\o \lambda$ is a differential/vector bundle, so the $R$-algebra must be non-singular.
% \end{proof}

\section{Connections on a differential bundle}%
\label{sec:connections-on-a-differential-bundle}

The connections discussed in this section generalize the notion of an affine connection to a differential bundle, giving a ``local coordinates'' presentation for $TE$ similar to the presentation of $T^2M$ as $T_3M$ induced by an affine connection. Chapter \ref{ch:involution-algebroids} makes extensive use of connections on vector bundles, so it is useful to set out the basic definitions before embarking on algebroid theory.  
\begin{definition}[\cite{Cockett2017}]%%
\label{def:lin-connection}
  Let $(q:E \to M, \xi, \lambda)$ be a differential bundle in a tangent category $\C$. 
  \begin{itemize}
    \item A vertical connection is a map $\kappa:TE \to E$ so that 
    \begin{enumerate}[(i)]      
      \item $\kappa$ is a vertical descent, and hence a retract of the lift; thus, $\kappa \o \lambda = id$;
      \item $\kappa$ is compatible with both differential bundle structures on $TE$, so that the maps
      \begin{gather*}
        \kappa: (TE,\ell) \to (E,\lambda) \\ \kappa:(TE, c\o T.\lambda) \to (E,\lambda)
      \end{gather*}
      are lift morphisms.
    \end{enumerate}
    \item A horizontal connection is a map $\nabla: E \ts{q}{p} TM \to TE$ so that
    \begin{enumerate}[(i)]
      \item $\nabla$ is a horizontal lift, and so is a retract of $(p.E, T.q):TE \to E \ts{q}{p} TM$\footnote{
        That $(p.E,T.q)$ land in the pullback $E \ts{q}{p} TM$ is a consequence of naturality, as $p \o T.q = q \o p$.
      }; thus, $(p, T.q)\o \nabla = id$;
      \item $\nabla$ is compatible with each pair of lifts, so that the maps
      \begin{gather*}
        \nabla: ( E \ts{q}{p} TM, \lambda \x \xi) \to (TE, c \o T.\lambda) \\
        \nabla: ( E \ts{q}{p} TM, 0 \x \ell) \to (TE, \ell)
      \end{gather*}
       are lift morphisms.
    \end{enumerate}
    \item A full connection is a pair $(\kappa,\nabla)$ that satisfies the following compatibility relations:\pagenote{
        I have extended the exposition on the compatibility relation between $\kappa, \nabla$ to make it clear that it is about constructing an isomorphism $TE \cong E_2\x TM$.
    }
    \begin{enumerate}[(i)]
      \item $\kappa \o \nabla = \xi \o q \o \pi_0$,
      \item $\nabla(p, T.q) +_{T.q} \mu(p, \kappa) = id$, so that there is an isomorphism $E \ts{q}{p} TM \ts{p}{q} E \cong TE$. \pagenote{caught a typo}
    \end{enumerate}
    
  \end{itemize}
\end{definition}
A notion that will be useful when dealing with classical differential geometry is that of a \emph{covariant derivative}, whose definition is equivalent to that of a vertical connection in the category of smooth manifolds.
\begin{definition}%
  \label{def:covariant-derivative}
  Let $(q:E \to M, \xi, \lambda, \kappa)$ be a vertical connection.\footnote{
     The definition of a covariant derivative only uses a vertical connection.
  }
  The \emph{covariant derivative} associated to $(\kappa,\nabla)$ is the map
  \[
      \nabla_{(-)}[=]:\Gamma(\pi) \x \Gamma(p) \to \Gamma(\pi); (A,X) \mapsto (\kappa \o TA \o X).
  \]
\end{definition}
\cite{LucyshynWright2018} drastically simplified the notion of a full connection by showing that it is exactly a vertical connection satisfying a universal property.
\begin{proposition}[\cite{LucyshynWright2018}]\label{prop:rory-connection}
  A connection on a differential bundle is equivalently specified by a vertical connection \[\kappa:TE \to E\] so that the following diagram exhibits $TE$ as a biproduct in the category of differential bundles over $M$:
  \input{TikzDrawings/Ch2/vcon.tikz}
\end{proposition}
% \begin{corollary}
%   The category of connections is equivalent to differential bundles equipped with an effective vertical connection and vertical connection-preserving linear maps.
% \end{corollary}
There is also a notion of flatness for connections that extends to vertical connections on a differential bundle.
\begin{definition}
  A connection $\kappa$ on a differential bundle $(q:E \to M, \xi, \lambda)$ is \emph{flat} whenever
  \[ 
    \kappa \o T.\kappa \o c = \kappa \o T.\kappa.
  \]
\end{definition}
We can see that connections are closed under similar constructions to differential bundles, in particular idempotent splittings and the reindexing construction from Lemma \ref{lem:reindex-db}.
\begin{lemma}%
  \label{lem:conn-const}
  Let $(q:E \to M, \xi, \lambda)$ be a differential bundle equipped with a vertical connection.\pagenote{
      I have reworded the second part of this lemma so that it no longer refers to effective vertical connections. The proof has also been streamlined.
  }
  \begin{enumerate}[(i)]
    \item Any linear retract of $(q,\xi,\lambda)$ will have a vertical connection. 
    \item The pullback differential bundle induced by pulling back $q$ along $f:N \to M$ will have a vertical connection.
  \end{enumerate}
  If the vertical connection is flat or is part of a full connection (that is, it satisfies the universality condition in Proposition \ref{prop:rory-connection}), then the induced connection will be as well.
\end{lemma}
\begin{proof}
  The universal property induces the vertical connection in each case. The construction will preserve flatness as it is an equational condition, and it preserves effectiveness by the commutativity of limits.
\end{proof}
There is no reason for every differential bundle in a tangent category to have a connection (for example, the tangent bundle in the free tangent category $\wone$ in Chapter \ref{chap:weil-nerve} is a differential bundle that does not have a connection). However, if the total space of a differential bundle has an affine connection, this induces a compatible connection on the differential bundle and its base space.
\begin{theorem}
  \label{thm:linear-connection-from-total-space}
  Let $(q:E \to M, \xi, \lambda)$ be a differential bundle in a tangent category in $\C$, where $E$ has a (flat) vertical connection. Then the total space $M$ and the differential bundle $(q,\xi,\lambda)$ each have a (flat) vertical connection. If the connection is full (so there is a compatible horizontal connection), then the induced connections are likewise full.
\end{theorem}
\begin{proof}
  By the strong universality condition for differential bundles, the differential bundle $(q\o\pi_0: E \ts{q}{p} TM \to M, (\xi, 0), \lambda \x \ell)$ is the pullback differential bundle of $p:TE \to E$ along $\xi:M \to E$. 
  This gives $(q\o\pi_0, (\xi,0), (\lambda,\ell))$, a (flat, effective) vertical connection by Lemma \ref{lem:conn-const}, and so it yields $(q:E \to M, \xi, \lambda)$ and $(p:TM \to M, 0, \ell)$ as (flat, effective) vertical connections by the idempotent splitting property.
\end{proof}
Every smooth manifold has an affine connection, thus inducing a connection on any vector bundle.
\begin{corollary}
  Every vector bundle has a connection.
\end{corollary}
\pagenote{Kristine mentioned it's a bit weird to end a chapter on a piece of notation, which I makes sense once it has been pointed out to me.}
% \begin{example}
%   Return to the category $\mathsf{Lex}$, and recall that every object has a (not-necessarily-effective) flat affine connection. Thus every differential bundle in the category has a (not-necessarily-effective) flat, affine connection. The vertical connection, of course, corresponds to the domain map $TE \to E$ that sends a Beck module to its total space (as the lift $\lambda$ must send $X$ to a Beck module with total space $X$). 
% \end{example}

% The following notation will be useful when working with linear morphisms between differential bundles equipped with a connection.
% \begin{definition}%
%   \label{def:nabla-notation}
%   Let $(q:E \to M, \xi, \lambda), (\pi:F \to N, \zeta, l)$ be a pair of differential bundles with connections $(\kappa,\nabla), (\kappa',\nabla')$. Given a linear bundle morphism, write the map:
%   \[
%     \nabla[f]: TM \ts{p}{q} E \xrightarrow[]{\nabla} TE \xrightarrow[]{T.f} TF \xrightarrow[]{\kappa'} F
%   \]
% \end{definition}
%NOTE: As Kristine mentioned, it's a bit weird to end a chapter on a piece of notation - I will re-introduce it later when necessary.




% \documentclass[main.tex]{subfiles}

% \begin{document}

\chapter{Involution algebroids}%
\label{ch:involution-algebroids}

This chapter accomplishes the first major goal of this thesis by providing a tangent-categorical axiomatization of Lie algebroids, namely involution algebroids. Much like vector bundles, Lie algebroids are a highly non-algebraic notion (in the sense of \cite{Freyd1972}), being vector bundles equipped with a Lie algebra structure on the set of sections of the projection. Furthermore, the bracket on sections must satisfy a product rule with respect to $\mathbb{R}$-valued functions on the base space (the \emph{Leibniz law}), introducing another piece of non-algebraic structure to the definition. The tangent-categorical definition of Lie algebroids will treat the tangent bundle as the ``prototypical Lie algebroid'' in which the vertical lift $\ell:T \Rightarrow T^2$ identifies the vector bundle structure and the canonical flip $c:T^2 \Rightarrow T^2$ plays the role of the Lie bracket.

Lie algebroids are the natural many-object analogue to Lie algebras, in the same way that Lie groupoids are the many-object analogue of Lie groups. In the single-object case, a Lie group is classically thought of as a space of symmetries for some smooth manifold (one often identifies a group action $G \x M \to M$), and a Lie algebra may similarly be thought of as a space of \emph{derivations} (often identified as a sub-Lie-algebra of $\chi(M)$ for a manifold $M$).  The extension of groups to groupoids is natural; in fact, Brandt's introduction of groupoids in \cite{brandt1927verallgemeinerung} predates MacLane and Eilenberg's invention of category theory in \cite{eilenberg1945general} by nearly two decades.  The translation of Lie algebras to the many-object case is not as straightforward. The first step is to replace the vector space underlying a Lie algebra with a vector bundle $(\pi:A \to M, \xi,\lambda)$. The idea is to axiomatize this vector bundle so that each section in $\Gamma(\pi)$ corresponds to a derivation on $C^\infty(M)$. The \emph{anti-commutator} operation on derivations from Proposition \ref{prop:anti-commm-lie} suggests there should be a Lie bracket $[-,-]:\Gamma(\pi)\ox\Gamma(\pi) \to \Gamma(\pi)$ (similar to the partially-defined multiplication for a groupoid), while the correspondence with derivations on $C^\infty(M)$ suggests there be a vector bundle morphism $\anc:A \to TM$ satisfying the \emph{Leibniz law}:
\begin{equation}\label{eq:bracket-f}
    [X, f\cdot Y] = f\cdot [X, Y] + [X,f]\cdot Y; \hspace{0.5cm} [X,f] := \phat \o T.f \o \anc \o X\footnote{Recall the notation from Lemma \ref{lem:Cinfty-module-vbun}.}
\end{equation}
(the full definition of Lie algebroids may be found in \ref{def:lie-algd}). This ``operational'' definition of Lie algebroids makes it difficult to describe their morphisms, and furthermore it essentially fails to be an algebraic structure in the classical sense, as it axiomatizes structure on the \emph{set of sections} of a map rather than a morphism in the category itself.

Involution algebroids were introduced to provide a tangent-categorical presentation of Lie algebroids, similar to the relationship between differential bundles and vector bundles. Chapter \ref{ch:differential_bundles} focused on the Euler vector field construction on a vector bundle, showing that this induced a fully-faithful functor from vector bundles to associative coalgebras (lifts) of the weak comonad $(T,\ell)$, and identified vector bundles with a subcategory of $\mathsf{Lift}(\mathsf{SMan})$ satisfying a universal property. The corresponding construction for Lie algebroids, then, is the \emph{canonical involution}, which was identified by Eduardo Martinez and his collaborators (a clearly written exposition may be found in Section 4 of \cite{de2005lagrangian}). Given a Lie algebroid $(\pi:A \to M, \anc:A \to TM, [-,-]:\Gamma(\pi)\ox\Gamma(\pi) \to \Gamma(\pi))$, its canonical involution is a map
\[
    \sigma: \prolong \to \prolong.
\]
Using this $\sigma$ map, there is a straightforward characterization of Lie algebroid morphisms: a Lie algebroid morphism is precisely a vector bundle morphism $(f,m):A \to B$ that preserves the anchor and involution maps:
\[% https://q.uiver.app/?q=WzAsOCxbMCwxLCJUTSJdLFswLDAsIkEiXSxbMSwwLCJCIl0sWzEsMSwiVE4iXSxbMiwwLCJcXHByb2xvbmciXSxbMiwxLCJcXHByb2xvbmciXSxbMywwLCJCXFx0c3tcXGFuY15CfXtULlxccGleQn1UQiJdLFszLDEsIkJcXHRze1xcYW5jXkJ9e1QuXFxwaV5CfVRCIl0sWzEsMCwiXFxhbmNeQSIsMl0sWzIsMywiXFxhbmNeQiJdLFsxLDIsImYiXSxbMCwzLCJULm0iLDJdLFs2LDcsIlxcc2lnbWFeQiJdLFs0LDYsImYgXFx4IFQuZiJdLFs0LDUsIlxcc2lnbWFeQSIsMl0sWzUsNywiZlxceCBULmYiLDJdXQ==
\begin{tikzcd}
	A & B & \prolong & {B\ts{\anc^B}{T.\pi^B}TB} \\
	TM & TN & \prolong & {B\ts{\anc^B}{T.\pi^B}TB}
	\arrow["{\anc^A}"', from=1-1, to=2-1]
	\arrow["{\anc^B}", from=1-2, to=2-2]
	\arrow["f", from=1-1, to=1-2]
	\arrow["{T.m}"', from=2-1, to=2-2]
	\arrow["{\sigma^B}", from=1-4, to=2-4]
	\arrow["{f \x T.f}", from=1-3, to=1-4]
	\arrow["{\sigma^A}"', from=1-3, to=2-3]
	\arrow["{f\x T.f}"', from=2-3, to=2-4]
\end{tikzcd}\]
Furthermore, it is implicit in Martinez's work (\cite{Martinez2001}) that $\sigma$ satisfies axioms corresponding to the Lie algebroid axioms. Thus, involutivity corresponds to antisymmetry of the Lie bracket
\[
    \sigma \o \sigma = id \iff [X,Y] + [Y,X] = 0, 
\]
while the Leibniz law holds if and only if the $\anc$ map sends the algebroid involution to the canonical flip on $M$,
\[
    T.\anc \o \pi_1 \o \sigma = c \o T.\pi \o \anc \iff 
    \forall f \in C^\infty(M), [X, f\cdot Y] = f\cdot [X, Y] + [X,f]\cdot Y
\]
(using the same definition as before for $[X,f]$).

The idea of an involution algebroid, then, is to axiomatize the canonical involution directly, just as differential bundles axiomatize the Euler vector field of a differential bundle. An involution algebroid is a differential bundle equipped with a pair of structure maps
\[
    \anc:A \to TM, \hspace{0.15cm} \sigma: \prolong \to \prolong
\]
satisfying a collection of axioms. Some of them are straightforward translations of the structure equations for Lie algebroids given in \cite{Martinez2001}, for instance
\[
  T.\anc \o \lambda = \ell \o \anc, \hspace{0.15cm}
  \sigma \o \sigma = id,\hspace{0.15cm}
  T.\anc \o \pi_1 \o \sigma = c \o T.\anc \o \pi_1.
\]
However, this requires a new coherence between the Euler vector field of the underlying vector bundle and the involution map:
\[
    \sigma \o (\xi\o\pi,\lambda) = (\xi\o\pi,\lambda).
\]
The most striking new fact about this coherence is that the Jacobi identity on the bracket $[-,-]$ corresponds to the \emph{Yang--Baxter} equation on $\sigma$:
% https://q.uiver.app/?q=WzAsNixbMCwwLCJcXHByb2xvbmcgXFx0c3tULlxcYW5jfXtUXjIuXFxwaX0gVF4yQSJdLFsxLDAsIlxccHJvbG9uZyBcXHRze1QuXFxhbmN9e1ReMi5cXHBpfSBUXjJBIl0sWzEsMSwiXFxwcm9sb25nIFxcdHN7VC5cXGFuY317VF4yLlxccGl9IFReMkEiXSxbMSwyLCJcXHByb2xvbmcgXFx0c3tULlxcYW5jfXtUXjIuXFxwaX0gVF4yQSJdLFswLDEsIlxccHJvbG9uZyBcXHRze1QuXFxhbmN9e1ReMi5cXHBpfSBUXjJBIl0sWzAsMiwiXFxwcm9sb25nIFxcdHN7VC5cXGFuY317VF4yLlxccGl9IFReMkEiXSxbMCwxLCJcXHNpZ21hIFxceCBjLkEiLDJdLFsxLDJdLFsyLDMsIlxcc2lnbWEgXFx4IGMuQSIsMl0sWzAsNCwiMSBcXHggVC5cXHNpZ21hIl0sWzQsNSwiXFxzaWdtYSBcXHggYy5BIl0sWzUsMywiMSBcXHggVC5cXHNpZ21hIl1d
\[\begin{tikzcd}
	{\prolong \ts{T.\anc}{T^2.\pi} T^2A} & {\prolong \ts{T.\anc}{T^2.\pi} T^2A} \\
	{\prolong \ts{T.\anc}{T^2.\pi} T^2A} & {\prolong \ts{T.\anc}{T^2.\pi} T^2A} \\
	{\prolong \ts{T.\anc}{T^2.\pi} T^2A} & {\prolong \ts{T.\anc}{T^2.\pi} T^2A}
	\arrow["{\sigma \x c.A}"', from=1-1, to=1-2]
	\arrow[from=1-2, to=2-2]
	\arrow["{\sigma \x c.A}"', from=2-2, to=3-2]
	\arrow["{1 \x T.\sigma}", from=1-1, to=2-1]
	\arrow["{\sigma \x c.A}", from=2-1, to=3-1]
	\arrow["{1 \x T.\sigma}", from=3-1, to=3-2]
\end{tikzcd}\]
This is both surprising (it is a new characterization of a central object of study in differential geometry and mathematical physics) and yet in a way expected (the work in \cite{Cockett2015}, \cite{Mackenzie2013} indicates that the Lie algebra structure on the set of vector fields over a manifold follows from the Yang--Baxter equation on $c$).  This vector-field-free presentation of the Jacobi identity allows for a structural approach to Lie algebroids that drives the work presented in Chapters \ref{chap:weil-nerve} and \ref{ch:inf-nerve-and-realization}.

As with \Cref{ch:differential_bundles}, the first section is expository, and is concerned with introducing the category of Lie algebroids. 
The second section introduces anchored bundles, together with the space of prolongations of an anchored bundle. The relationship between anchored bundles and involution algebroids is equivalent to that between reflexive graphs and groupoids (the subject of Chapter \ref{ch:inf-nerve-and-realization}), the space of prolongations of an anchored bundle being equivalent to the set of composable arrows for a groupoid. This section is mostly a translation of Martinez's prolongation construction to a general tangent category. The rest of the chapter contains new results, developed in collaboration with Matthew Burke and Richard Garner.

Section 3 introduces involution algebroids, which are anchored bundles equipped with an involution map on their space of prolongations. Section 4 considers an anchored bundle in a tangent category with negatives that is equipped with a connection. The connection gives an involution algebroid a ``local coordinates'' presentation (in the sense of Section \ref{sec:diff-and-tang-struct}) that is equivalent to the local characterization of Lie algebroids from Section 1. The final section of this chapter establishes the main result: the category of Lie algebroids is isomorphic to that of involution algebroids in smooth manifolds.\pagenote{
   The introduction has been substantially expanded to better explain what is happening in this chapter and why we are doing it. 
}

% The Lie functor from Lie groups to Lie algebras has a natural extension to the many-object case. 
% The Lie algebra of a Lie group $G$ is the space of the \emph{left-invariant} vector fields on $G$.
% \[
%    ---
% \]
% It can be show that the space of left-invariant vector fields is isomorphic to the tangent space on $G$ over the identity element $e:1 \to G$. 
% Left-invariance of a vector field on the space of arrows $G$ of a groupoid, the, corresponds to %cite UofT notes
% \[
%     dfd
% \]
% This corresponds to the space of tangent vectors above the embedded submanifold $M \hookrightarrow G$, that are \emph{source-constant}
% \[
%   dd    
% \]
% This is equivalent to the pullback:
% \[
%  ff    
% \]




\section{Lie algebroids}\label{sec:Lie_algebroids}

This section reviews the basic theory of Lie algebroids: their definition and that of their morphisms, along with some introductory examples. The classical definition will not appear elsewhere in this chapter, however, as we quickly introduce Martinez's \emph{structure equations} for a Lie algebroid (\cite{Martinez2001}), then translate them into tangent-categorical terms using a connection. \pagenote{
   The substantial revisions to the introduction made the original preamble to this section redundant. There have also been substantial revisions to the thesis up until this point, and those changes have percolated into this section. In particular, the notation for the $C^\infty(M)$-module of sections $\Gamma(\pi)$ was set in the previous chapter, and we stick to those conventions.
}

\begin{definition}\label{def:lie-algd}
    A \emph{Lie algebroid} is a vector bundle $\pi:A \rightarrow M$ equipped with an anchor $\anc:A \to TM$ and a bracket $[-,-]:\Gamma(\pi)\ox\Gamma(\pi) \rightarrow \Gamma(\pi)$ satisfying the following axioms:
    \begin{itemize}
        \item bilinear: $[aX_1+bX_2, Y] = a[X_1, Y]+ b[X_2, Y]$ and $[X, aY_1+bY_2] = a[X, Y_1]+b[X, Y_2]$
        % \item alternating: $[X, X] = 0$
        \item anti-symmetric: $[X, Y]+[Y, X] = 0$
        \item Jacobi: $[X, [Y, Z]] = [[X, Y], Z] + [Y, [X, Z]]$
        \item Leibniz: $[X, f\cdot Y] = f\cdot [X, Y] + [X,f]\cdot Y$\pagenote{Changed the notation of the Lie derivative to this bracket version.}
    \end{itemize}
    (where $[X,f]$ is defined as in Equation \ref{eq:bracket-f}).
\end{definition}

\begin{example}%
    \label{ex:lie-algebroids}
    ~\begin{enumerate}[(i)]
        \item The canonical example of a Lie algebroid is, of course, the tangent bundle using the operational tangent bundle from Definition \ref{def:operational-tang}.
        % The tangent bundle is the linear approximation of the pair groupoid:
        \item A \emph{Lie algebra} is a Lie algebroid over the terminal object: for a group $G$, the bundle of source-constant tangent vectors is the usual Lie functor from Lie groups to Lie algebras, because a groupoid is a one-object group.
        \item The bundle of \emph{source-constant} tangent vectors $s,t: G \to M$ of a Lie groupoid forms a Lie algebroid. This bundle is defined by the pullback
        \input{TikzDrawings/Ch3/Sec1/Lie-Functor-Of-G.tikz}
        where the projection is $\pi$ and the target is given by $\anc$ in the diagram.
        There is an injective $\R$-module morphism from sections of $\pi$, $\Gamma(\pi)$ to vector fields on $G$, $\chi(G)$, and the Lie bracket on $G$ is closed over the image of this lift, putting a Lie bracket on $\Gamma(\pi)$. In particular, we can see that $TM$ is the bundle of source-constant tangent vectors for the pair groupoid on a manifold $M$:
        \input{TikzDrawings/Ch3/Sec1/LieFuncPairGpd.tikz}
        Given $(u,v):X \to T(M \x M)$ above $(m,m):X \to TM$ with $u = 0 \o m$, it follows that $u = 0 \o p \o u$, so $v$ is the unique map induced into $TM$.
        \item Every group is a groupoid over a single object. The Lie algebroid associated with a group $G$, then, is the usual Lie algebra.
        \item From the Hamiltonian formalism of mechanics, every \emph{Poisson manifold} has an associated Lie algebroid.
        A Poisson manifold is a manifold $M$ equipped with a \emph{Poisson algebra} structure on $C^\infty(M)$, namely a Lie algebra that is also a derivation:
        \[
            [f\cdot g, h] = f\cdot [g,h] + [f,h]\cdot g
        \]
        where $\cdot$ is the multiplication in the algebra $C^\infty(M)$ as in Lemma \ref{lem:Cinfty-module-vbun}. The cotangent bundle over a Poisson manifold $M$ is canonically a Lie algebroid, called a Poisson Lie algebroid \cite{Courant1994}. 
        \item   Any Lie algebra bundle---that is, a vector bundle equipped with a Lie bracket on its space of sections---is a Lie algebroid, with anchor map $\xi \o \pi$.     
    \end{enumerate}
\end{example}
Morphisms of Lie algebroids are notoriously difficult to work with, and have an involved definition.
\begin{definition}\label{def:lie-algd-morphism}
    Let $A, B$ be a pair of anchored bundles over $M,N$, and $\Phi: A \to B$ an anchored bundle morphism over a map $\phi:M \to N$. A $\Phi$-decomposition of $X \in \Gamma(\pi_A)$ is a set of $X_i \in \Gamma(\pi_B)$ and $f_i \in C^\infty(M)$ so that
    \[
        \Phi \o X = \sum_i f_i \cdot X_i \o \phi.
    \]
    An anchor-preserving vector bundle morphism $\Phi$ is a Lie algebroid morphism if and only if for any $X,Y \in \Gamma(\pi_A)$ and $\Phi$-decompositions $\{X_i, f_i\}, \{Y_j, g_j\}$ of $X,Y$, the following equation holds:
    \[        
        \Phi \o [X,Y] 
        = 
        \sum f_i\cdot g_j \cdot ([X_i,Y_j] \o \phi)
        + \sum [X, f_i] \cdot (X_i \o \phi)
        - \sum [Y, g_j](Y_j \o \phi).
    \]
\end{definition}
The equation defining a Lie algebroid morphism holds independently of the choice of $\Phi$-decomposition (see \cite{Higgins1990} for a proof).\pagenote{Here I have tidied up the definition by explaining what a $\Phi$-decomposition is (in particular, quantifying over the $X_i, f_i$, and also clarified some wording.}
% When working over a fixed base, this definition is tractable:
% \begin{example}\label{ex:la-morphism}
%     If $A, B$ are Lie algebroids over a base manifold $M$, $f: A \to B$ is a Lie algebroid morphism if and only if it preserves the anchor and Lie bracket.
%     In particular, the anchor of a Lie algebroid $\an c: A \to TM$ is a Lie algebroid morphism.
% \end{example}

\begin{example}
    ~\begin{enumerate}[(i)]
        \item If $A, B$ are Lie algebroids over a base manifold $M$, $\Phi: A \to B$ is a Lie algebroid morphism if and only if it preserves the anchor and Lie bracket.\pagenote{I removed the remark about tangent Lie algebroids, as it refers to De Rham cohomologies which have disappeared from this thesis.}
        % \item Every map $f: M \to N$ gives rise to a morphism of the tangent Lie algebroids, as this induces a cochain morphism between their De Rham cohomologies.
        \item Given two Lie algebroid morphisms $f: A \to B, g: B \to C$, their composition $g \o f$ is also a Lie algebroid morphism.
        \item A \emph{Poisson Sigma model} (see \cite{Bojowald2005}) is a morphism of Lie algebroids 
        \[
            \phi: T\Sigma \to T^*M
        \]
        for which $\Sigma$ is a 2-dimensional manifold and $T\Sigma$ denotes its tangent Lie algebroid, while $M$ is a Poisson manifold and $T^*M$ denotes the Lie algebroid structure on its cotangent bundle .
    \end{enumerate}
\end{example}

% % This section moves away from bracket-based notation for Lie algebroids - the first step is using Martinez's local-coordinates presentation of the Lie bracket. 
% This bracket-based presentation of Lie algebroids is ill-suited for functoria
Lie algebroids are a natural generalization of Lie algebras to the ``multi-object'' setting, but they are ill-suited for a functorial presentation of the theory. A step in this direction is to consider the coordinate-based presentation of the Lie bracket and its coherences due to  \cite{Martinez2001}.
Let $A$ be an anchored bundle over $M$ equipped with a bilinear bracket on its space of sections, and choose a pair of bases for $\Gamma(\pi)$ and $\chi(M)$: for $\Gamma(\pi)$ write $\{ e_\alpha\}$, and for $\chi(M)$ write $\{ \frac{\partial}{\partial x^i}\}$. The anchor and bracket then have a presentation in local coordinates: 
\[
    \anc(e_\alpha) = \sum_i \anc^i_\alpha \frac{\partial}{\partial x^i}
    \hspace{0.5cm}
    [e_\alpha, e_\beta] = \sum_\gamma C^{\gamma}_{\alpha\beta}e_\gamma
\]
(from here on out, we use Einstein summation notation to simplify our calculations, so instead write
\[
    \anc(e_\alpha) = \anc^i_\alpha \frac{\partial}{\partial x^i}
    \hspace{0.5cm}
    [e_\alpha, e_\beta] = C^{\gamma}_{\alpha\beta}e_\gamma
\]
with $\sum$ suppressed).
The following characterization of the Lie algebroid axioms uses Martinez's \emph{structure equations}.\pagenote{This is better called the "structure equations" for a Lie algebroid.}
\begin{proposition}\label{prop:la-iff-structure-morphisms}[\cite{Martinez2001}]
    An anchored bundle $A$ over $M$ equipped with a bracket is a Lie algebroid if and only if  $\anc$ and $[-,-]$ satisfy the following structure equations:
    \begin{enumerate}[(i)]
        \item Alternating: \[C^\nu_{\alpha\beta} + C^\nu_{\beta\alpha} = 0\]
        \item Leibniz: \[\anc^j_\alpha \frac{\partial \anc^i_\beta}{\partial x^j} 
            = \anc^i_\gamma C^\gamma_{\alpha\beta} + \anc^j_\beta \frac{\partial \anc^i_\alpha}{\partial x^j}\]
        \item Bianchi: \[ 0
            = \anc^i_\alpha \frac{\partial C^{\nu}_{\beta\gamma}}{\partial x^i}
            + \anc^i_\beta \frac{\partial C^{\nu}_{\gamma\alpha}}{\partial x^i}
            + \anc^i_\gamma \frac{\partial C^{\nu}_{\alpha\beta}}{\partial x^i}
            + C^{\mu}_{\beta\gamma}C^{\nu}_{\alpha\mu}
            + C^{\mu}_{\gamma\alpha}C^{\nu}_{\beta\mu}
            + C^{\mu}_{\alpha\beta}C^{\nu}_{\gamma\mu}\]
    \end{enumerate}
\end{proposition}
This proposition is a straightforward translation of the Lie algebroid axioms into local coordinates using a covariant derivative.
First, recall that by the smooth Serre--Swan theorem (11.33 \cite{Nestruev2003}), the bilinearity of the bracket \pagenote{Clarified where this new bracket comes from/what category it lives in.}
\[
    [-,-]:\Gamma(\pi) \x \Gamma(\pi) \to \Gamma(\pi)  
\] as a morphism of $C^\infty(M)$-modules guarantees that it corresponds to a bilinear morphism $A_2 \to A$ of vector bundles, meaning that there exists a globally defined bilinear map $A_2 \to A$ that is equal to the Lie bracket when applied to sections of the projection. We record this as a lemma: \pagenote{Clarified the point that this theorem gives a global definition of the Lie bracket, extending from from $\Gamma(\pi)$ to general elements in the category}
\begin{lemma}\label{lem:ang-brack-lie}
    For every Lie algebroid $\a$, there is a bilinear morphism
    \[
       \langle -,-\rangle: A_2 \to A 
    \]
    so that for any sections $X,Y\in \Gamma(\pi)$, $\langle X,Y\rangle = [X,Y]$.
\end{lemma}

There are two structure maps derived from $\< -, -\>$ that encode the coherences of a Lie algebroid. The first map measures the extent to which the anchor maps fail to preserve the chosen connections on the vector bundle $\pi:A \to M$ and the tangent bundle $p:TM \to M$.\pagenote{Added some explanation of what the curly bracket map does.}
\begin{definition}\label{def:curly-bracket}
    Let $A$ be a Lie algebroid, and for a chosen horizontal connection $\nabla$ on $A$ and vertical connection $\kappa'$ on $TM$, set
    \[
        \{ v, x\}_{\kappa',\nabla} := TM \ts{p}{q} A \xrightarrow[]{\nabla} TA \xrightarrow[]{T.\anc} T^2M \xrightarrow[]{\kappa'} TM.
    \]
    When the choice of connection is evident by context, we will suppress the subscript.
\end{definition}
\begin{observation}
    The above parentheses bracket corresponds to the symbol
    \[
        \{-,-\} = \anc^j_\beta \frac{\partial \anc^i_\alpha}{\partial x^j}.  
    \]
\end{observation}
The Leibniz coherence may be rewritten as follows:
\begin{lemma}\label{lem:curly-bracket-coh}
    Let $A$ be an anchored bundle with a bilinear bracket (inducing an involution $\sigma$).
    Choose connections $(\kappa, \nabla), (\kappa', \nabla')$ on $A$ and $TM$ respectively.
    The bracket and anchor map satisfy the Leibniz law if and only if
    \[
        \anc \o \< x, y\>_{(\kappa, \nabla)} 
        + \{ \anc x, y\}_{(\kappa', \nabla)}
        = \{ \anc y, x\}_{(\kappa', \nabla)}.
    \]
\end{lemma}
\begin{proof}
    The condition is equivalent to the identity
    \[\anc^j_\alpha \frac{\partial \anc^i_\beta}{\partial x^j} 
            = \anc^i_\gamma C^\gamma_{\alpha\beta} + \anc^j_\beta \frac{\partial \anc^i_\alpha}{\partial x^j}.\]
\end{proof}
The Bianchi axiom measures the failure of the Jacobi identity in local coordinates, and states that it must be corrected for by the curvature of the brackets. 
We see that
\[
    C^{\nu}_{\alpha\mu} C^{\mu}_{\beta\gamma} = [e_\alpha, [e_\beta, e_\gamma]]
\]
while 
\[
    \frac{\partial C^{\nu}_{\beta\gamma}}{\partial x^i} e_\nu
    = \kappa \o T(\langle -,-\rangle_{(\kappa, \nabla)})\o \nabla^{A_2} \o ( \frac{\partial}{\partial x^i}, e_\beta, e_\gamma)
\]
determines a trilinear map
\[
    \{\frac{\partial}{\partial x^i}, e_\beta, e_\gamma\}_{(\kappa, \nabla)} 
    := \kappa \o T(\langle -,-\rangle_{(\kappa, \nabla)})\o \nabla^{A_2} \o ( \frac{\partial}{\partial x^i}, e_\beta, e_\gamma).
\]
This is the second derived map used in the structure equations for a Lie algebroid.
\begin{definition}
    Let $(\pi:A \to M, \xi, \lambda, \anc)$ be an anchored bundle equipped with a bilinear map
    \[
        \langle -, - \rangle: A_2 \to A.
    \]
    The derived ternary bracket $\{ -, -, - \}: TM \ts{p}{\pi} A \ts{\pi}{\pi} A \to A$ is defined as
    \[
        \{v,x,y\}_{(\kappa, \nabla)} := 
        TM\ts{p}{\pi} A \ts{\pi}{\pi} A \xrightarrow[]{\nabla[A2]} TA_2 \xrightarrow[]{T.\<-,-\>} TA \xrightarrow{\kappa} A
        % \kappa \o T(\langle -,-\rangle_{(\kappa, \nabla)})\o \nabla^{A_2} \o (v,x,y)
    \]
    where $\nabla[A2]$ is the pairing $(\nabla(\pi_0,\pi_1), \nabla(\pi_0,\pi_2))$.
\end{definition}
\begin{observation}
    The ternary bracket corresponds to the following symbol:
    \[
        \{-,-,-\}_{(\kappa, \nabla)} :=
        \anc^i_\alpha \frac{\partial C^{\nu}_{\beta\gamma}}{\partial x^i}.
    \]
\end{observation}
\begin{lemma}\label{lem:bianchi-connection}
    Let $A$ be an anchored bundle over $M$ with a bilinear bracket and denote the induced involution by $\sigma$.
    Choose a pair of connections and write the derived maps $\langle-,-\rangle, \{-,-,-\}$.
    Then
    \begin{enumerate}[(i)]
        \item the bracket is antisymmetric if and only if its globalization is; that is, $\langle e_\alpha, e_\beta\rangle + \langle e_\beta, e_\alpha\rangle = 0$;
        \item the bracket satisfies the Jacobi identity if and only if it is alternating in the last two arguments and
            \[
                0
                = \sum_{\gamma \in \mathsf{Cy}(3)} \< x_{\gamma_0}, \< x_{\gamma_1}, x_{\gamma_2}\> \>
                + \sum_{\gamma \in \mathsf{Cy}(3)} \{ \anc x_{\gamma_0}, x_{\gamma_1}, x_{\gamma_2}\}.
            \]
    \end{enumerate}
\end{lemma}
\begin{proof}~
    \begin{enumerate}[(i)]
        \item This is equivalent to $C^\nu_{\alpha\beta} + C^\nu_{\beta\alpha} = 0$.
        \item This is equivalent to \[ 0
            =   C^{\mu}_{\beta\gamma}C^{\nu}_{\alpha\mu}
            + C^{\mu}_{\gamma\alpha}C^{\nu}_{\beta\mu}
            + C^{\mu}_{\alpha\beta}C^{\nu}_{\gamma\mu}
            + \anc^i_\alpha \frac{\partial C^{\nu}_{\beta\gamma}}{\partial x^i}
            + \anc^i_\beta \frac{\partial C^{\nu}_{\gamma\alpha}}{\partial x^i}
            + \anc^i_\gamma \frac{\partial C^{\nu}_{\alpha\beta}}{\partial x^i}. \] 
    \end{enumerate}
\end{proof}

\begin{proposition}%
    \label{prop:lie-alg-bil-defn}
    A Lie algebroid is exactly an anchored vector bundle $(\pi:A \to M, \xi, \lambda, \anc)$ with a bilinear, alternating map
    \[
        \langle -, -\rangle: A_2 \to A; \hspace{0.5cm} \< x, y \> + \< y, x\> = 0
    \]
    so that for any connection $(\nabla,\kappa)$ on $A$, the maps 
    \begin{equation}
        \label{eq:lie-algd-structure-maps}
        \{ v, x\}_{(\kappa, \nabla)} := \kappa' \o T.\anc \o \nabla(v, x), \hspace{0.5cm}
        \{v,x,y\}_{(\kappa, \nabla)} := \kappa \o T(\langle -,-\rangle_{(\kappa, \nabla)})\o \nabla^{A_2} \o (v,x,y)
    \end{equation}
    satisfy the equations
    \begin{enumerate}[(i)]
        \item $ \anc \o \< x, y\> 
        + \{ \anc x, y\}_{(\kappa', \nabla)}
        = \{ \anc y, x\}_{(\kappa', \nabla)}$,
        \item $\sum_{\gamma \in \mathsf{Cy}(3)} \< x_{\gamma_0}, \< x_{\gamma_1}, x_{\gamma_2}\> \>
        + \sum_{\gamma \in \mathsf{Cy}(3)} \{ \anc x_{\gamma_0}, x_{\gamma_1}, x_{\gamma_2}\}_{(\kappa, \nabla)} = 0$.
    \end{enumerate}
\end{proposition}

There is also a local coordinates presentation of morphisms as in Section 2 of \cite{Martinez2018}. An anchored bundle morphism $A \to B$ is a Lie algebroid morphism whenever
\[
    f^\beta_\gamma A^\gamma_{\alpha \delta} + \anc^i_\delta \frac{\partial f^\beta_\alpha}{\partial x^i}
    = B^\beta_{\theta\sigma}f^\theta_{\alpha}f^{\sigma}_{\delta} + \anc^i_\alpha \frac{\partial f^\beta_\delta}{\partial x^i}.
\]
The $A$ and $B$ arguments are understood as the brackets, so this condition can be rewritten as
\[
    f^\beta_\gamma A^\gamma_{\alpha\delta} = f \o \< \alpha, \delta \>, \hspace{0.25cm}
    B^\beta_{\theta\sigma}f^\theta_{\alpha}f^{\sigma}_{\delta} = \< f \o \alpha, f \o \delta \>.
\]
Set the following notation for maps between vector bundles with connection:\pagenote{Reintroduce the $\nabla$ notation here after removing it.}
\begin{equation}\label{eq:nabla-notation}
    \infer{\nabla[f]:A_2 \to B := \kappa^B \o T.f \o \nabla^A}{f:A \to B & (\kappa^A,\nabla^A,A, \lambda^A) & (\kappa^B,\nabla^B, B, \lambda^B)}
    % 
\end{equation}
The $\anc$ terms are understood to be the torsion, so that
\begin{gather*}
        \anc^i_\delta \frac{\partial f^\beta_\alpha}{\partial x^i} 
    = \kappa \o T.f \o \nabla(\anc e_\delta, e_\alpha) = \nabla[f](\anc e_\delta, e_\alpha),\\
    \anc^i_\alpha \frac{\partial f^\beta_\delta}{\partial x^i} 
    = \kappa \o T.f \o \nabla(\anc \alpha, \delta) = \nabla[f](\anc e_\alpha, e_\delta),
\end{gather*}
using the notation set up in Equation \ref{eq:nabla-notation}.
The notion of a Lie algebroid morphism, then, has the following presentation:
\begin{proposition}[\cite{Martinez2018}]%
    \label{prop:lie-algd-morphism-defn}
    Let $(\pi:A \to M,\anc^A, \<-,-\>^A), (q:B \to N,\anc^B,\<-,-\>^B)$ be a pair of Lie algebroids with chosen connections $(\kappa^{-},\nabla^{-})$.
    An anchor-preserving vector bundle morphism $f:A \to B$ is a Lie algebroid morphism if and only if
    \[
        f \o \< e_\alpha, e_\delta \> + \nabla[f] \o (\anc e_\delta, e_\alpha)
        = \< f \o \alpha, f \o \delta \> + \nabla[f] \o(\anc e_\alpha, e_\delta).
    \]
\end{proposition}

\section{Anchored bundles}%
\label{sec:anchored-bundles}
Anchored bundles are to Lie algebroids what reflexive graphs are to groupoids. Each theory has (mostly) the same structure, but while a reflexive graph is missing a groupoid's composition operation, an anchored bundle lacks a Lie algebroid's bracket operation. This section reviews the basic theory of anchored bundles and their prolongations (see \cite{Mackenzie2005} for more details).
\begin{definition}\label{def:anchored_bundles}
    An anchored bundle in a tangent category is a differential bundle $(A \xrightarrow{\pi} M, \xi, \lambda)$ equipped with a linear morphism 
    \begin{equation*}
        \input{TikzDrawings/Ch3/Sec4/ancbun.tikz}
    \end{equation*}
    A morphism of anchored bundles is a linear bundle morphism $(f,v)$ that preserves the anchors
    \[\input{TikzDrawings/Ch3/Sec4/ancbun-mor.tikz}\]
    The category of anchored bundles and anchored bundle morphisms in a tangent category $\C$ is written $\mathsf{Anc}(\C)$, and a generic anchored bundle is written $(A \xrightarrow{\pi} M, \xi, \lambda,\anc)$.
    % Furthermore, the first and second \textit{prolongations} must exist:
    % 
\end{definition}
There are two pullbacks that are associated with every anchored bundle.
These play the role of the spaces of composable arrows $G_2 := G \ts{t}{s} G, G_3 := G \ts{t}{s} G \ts{t}{s} G$ for a reflexive graph $s,t:G \to M$.
\begin{definition}\label{def:prolongations}
    Let $(A \xrightarrow{\pi} M, \xi, \lambda, \anc)$ be an anchored bundle. Its \emph{first and second prolongations} are given by the limits
    \[\input{TikzDrawings/Ch3/Sec4/prols.tikz}\]
    (The notation for the fibre product is slightly non-standard, as it is not technically a pullback.)
    Throughout this chapter, it will always be assumed that the first and second prolongations of an anchored bundle exist (although no choice of prolongation is explicitly made).\pagenote{I have explicitly defined the category of anchored bundles, and split the definition of prolongations from the definition of an anchored bundle. Also clarified that we are simply assuming these first two $T$-limits always exist, but we do not make a specific choice of prolongation.}
\end{definition}


\begin{remark}
    It is not strictly necessary that the prolongations of an anchored bundle exist; this condition is primarily a matter of convenience when discussing involution algebroids. Every result in this section, that does not explicitly mention prolongations, holds for an anchored bundle independently of their existence.
\end{remark}
\begin{example}%
    \label{ex:anchored-bundles}\pagenote{
       This set of examples has been substantially revised to fix some clumsy wording in the original version. It made sense to included the full result about $\C$ be a reflective subcategory of $\mathsf{Anc}(\C)$ and the coreflection from the category of differential bundles to anchored bundles rather than having them as separate results, as they each describe classes of anchored bundles.
    }
    ~\begin{enumerate}[(i)]
        \item For any object $M$, $id: TM \to TM$ is an anchor for the tangent differential bundle, and every $f: M \to N$ yields a morphism of anchored bundles. Moreover, for any anchored bundle over $M$ the anchor is itself a morphism of anchored bundles $(A,\pi,\xi,\lambda,\anc) \to (TM, p, 0, \ell, id)$. This induces a fully faithful functor:
        \[
            \C \hookrightarrow \mathsf{Anc}(\C).
        \]
        This inclusion has a left adjoint, which sends an anchored bundle $(\pi:A \to M, \xi,\lambda,\anc)$ to its base space $M$ (the unit is the anchor map $\anc$), so that $\C$ is a reflective subcategory of $\mathsf{Anc}(\C)$.
        \item For any differential bundle $(A, \pi, \xi, \lambda)$, the map $0\o\pi:A \to TM$ is an anchor and every morphism $f: A \to B$ of differential bundles again yields a morphism of anchored bundles. The naturality of $0$ ensures that every differential bundle morphism will preserve this trivial anchor map, giving a fully faithful functor
        \[
            \mathsf{DBun}(\C) \hookrightarrow \mathsf{Anc}(\C).
        \]
        This functor has a right adjoint that replaces the anchor map with the trivial anchor map
        \[
            (\pi:A \to M, \xi, \lambda, \anc) \mapsto (\pi:A \to M, \xi, \lambda, 0 \o \pi)  
        \]
        where the counit is given by the natural idempotent $e = \xi \o \pi:(A,\lambda) \to (A,\lambda)$. It is trivial to check that the anchor map is preserved by the bundle morphism $(e,id)$:
        \[
            \anc \o \xi \o \pi = 0 \o T.id \o \pi 
        \]
        and so the following diagram commutes:\pagenote{I took the opportunity to clarify a point brought up in my defense about this relationship.}
        % https://q.uiver.app/?q=WzAsNCxbMCwxLCJBIl0sWzEsMSwiQSJdLFswLDAsIlRNIl0sWzEsMCwiVE0iXSxbMiwzLCIiLDAseyJsZXZlbCI6Miwic3R5bGUiOnsiaGVhZCI6eyJuYW1lIjoibm9uZSJ9fX1dLFswLDIsIjAgXFxvIFxccGkiXSxbMCwxLCJcXHhpIFxcbyBcXGxhbWJkYSIsMl0sWzEsMywiXFxhbmMiLDJdXQ==
        \[\begin{tikzcd}
            TM & TM \\
            A & A
            \arrow[Rightarrow, no head, from=1-1, to=1-2]
            \arrow["{0 \o \pi}", from=2-1, to=1-1]
            \arrow["{p \o \lambda}"', from=2-1, to=2-2]
            \arrow["\anc"', from=2-2, to=1-2]
        \end{tikzcd}\]
        This means that  differential bundles are a \emph{coreflective} subcategory of anchored bundles.
        \item Any reflexive graph $(s,t: C \to M, e:M \to C)$ in a tangent category has an anchored bundle (when sufficient limits exist), constructed as
        \[\input{TikzDrawings/Ch3/Sec9/eq-of-idem.tikz}\]
        (where $e^s:C \to C = e \o s$). Construct a lift on $C^\partial$:
        \[\input{TikzDrawings/Ch3/Sec9/lift-of-lin-approx.tikz}\]    
        This lift will be non-singular by the commutativity of $T$-limits. The pre-differential bundle data is given by the projection
        \[ C^\partial \hookrightarrow T.C \xrightarrow[]{p.e^s} M.\]
        The section is induced by
        \[
            M \xrightarrow[]{(0.e)}  T.C
        \]
        while post-composition with $T.t$ gives the anchor map:
        \[
            C^\partial \hookrightarrow T.C \xrightarrow[]{T.t} T.C. 
        \]
        The diagram is a pullback by composition, and the outer perimeter defines the pullback $\prol(A)$. Note that any reflexive graph morphism will give rise to an anchored bundle morphism by naturality, making the construction of an anchored bundle from a reflexive graph functorial.\pagenote{This was originally in the examples of prolongations, which was not correct.}
        \item For any object $M$ in a tangent category, $c_M: T^2M \to T^2M$ is an anchor on $(T.p, T.0, c \o T.\ell)$, and every map $f: M \to N$ gives a morphism of anchored bundles.
        \item For any anchored bundle $(q:E \to M, \xi, \lambda, \anc)$, the differential bundle $(T.q, T.\xi, \newline c \o T.\lambda)$ has an anchor 
        \[\anc_T: TE \xrightarrow[]{T.\anc} T^2M \xrightarrow[]{c}T^2M.\]
    \end{enumerate} 
\end{example}
The first prolongation of an anchored bundle $\prol(A)$ behaves similarly to the second tangent bundle, except that it does not have a canonical flip. In the definition of an affine connection, the tangent bundle played a similar role to the ``arities'' of a theory. There is a lift map that makes this connection stronger:
\begin{definition}\label{def:lhat}
    Let $(\pi:A \to M, \xi, \lambda, \anc)$ be an anchored bundle in a tangent category $\C$.
    We define a \emph{generalized lift}:
    \[
        \hat{\lambda}: A \to \prolong := (\xi \o \pi, \lambda).   
    \]
\end{definition}
This generalized lift satisfies the same coherences as the lift on the second tangent bundle:
\begin{proposition}\label{prop:lift-axioms-anchor}
    Let $(\pi:A \to M, \xi, \lambda, \anc)$ be an anchored bundle in a tangent category $\C$.
    It follows that
    \begin{enumerate}[(i)]
        \item (Coassociativity of $\hat{\lambda}$) $(\hat{\lambda}\x\ell)\o\hat{\lambda} = (id \x T.\hat{\lambda}) \o \hat{\lambda}$;
        \item (Universality) the following diagram is a $T$-pullback:
        \[\input{TikzDrawings/Ch3/Sec5/inv-algd-universality.tikz}\]
        where $\hat \mu := (\xi \o \pi \o \pi_0, \mu)$.
    \end{enumerate}
\end{proposition}
\begin{proof}
    ~\begin{enumerate}[(i)]
        \item Compute
            \begin{align*}
                (\hat{\lambda}\x \ell)\o \hat{\lambda} 
                &= (\xi\o\pi\o\xi\o\pi, \lambda\o\xi\o\pi, \ell\o\lambda) \\
                &= (\xi\o\pi, T.(\xi\o\pi)\o\lambda, T.\lambda \o \lambda) \\
                &= (\pi_0, T.(\xi\o\pi)\o\pi_1, T.\lambda \o \pi_1)\o (\xi\o\pi, \lambda) \\
                &= (id \x T(\hat{\lambda}))\o\hat{\lambda}.
            \end{align*}
        \item  Use the pullback lemma to observe that the following diagram is universal for any anchored bundle:
            % https://q.uiver.app/?q=WzAsNixbMCwwLCJBXzIiXSxbMCwxLCJNIl0sWzEsMCwiXFxwcm9sKEEpIl0sWzEsMSwiQSJdLFsyLDAsIlRBIl0sWzIsMSwiVE0iXSxbMSwzLCJcXHhpIl0sWzMsNSwiXFxhbmMiXSxbMSw1LCIwIiwyLHsiY3VydmUiOjJ9XSxbNCw1LCJUXFxwaSJdLFsyLDMsIlxccGlfMCJdLFswLDEsIlxccGlcXG9cXHBpXzEiLDJdLFswLDIsIlxcaGF0e1xcbXV9IiwyXSxbMiw0LCJcXHBpXzEiLDJdLFswLDQsIlxcbXUiLDAseyJjdXJ2ZSI6LTJ9XV0=
            \input{TikzDrawings/Ch3/Sec5/inv-algd-mu-universal-proof.tikz}
            The top triangle of the diagram commutes by definition\pagenote{Removed a redundancy pointed out by Kristine.}. 
            The right square and outer perimeter are pullbacks by definition, and the bottom triangle also commutes by definition. The pullback lemma ensures that the left square is a pullback, so for every anchored bundle, the general lift is universal for $\prol(A)$. Now post-compose with the involution:
            \input{TikzDrawings/Ch3/Sec5/inv-algd-nu-universal.tikz}
            It suffices to check that the top triangle commutes, so $\sigma \o \hat{\mu} = \nu$:
            \[
                \sigma \o \hat{\mu}\o (a,b) = \sigma \o   ((\xi\o \pi,0)\o a +_{\pi_0} (\xi\o \pi,\lambda)\o b) = 
                 (id, T.\xi \o \anc \o a) +_{p\pi_1} (\xi\o \pi,\lambda)\o b.
            \]
            Thus, the lift $(\xi\pi,\lambda)$ involution algebroid is universal for $\prol(A)$.
    \end{enumerate}
\end{proof}


% A similar notion of anchored bundle connection exists using the prolongation as an argument in a similar manner.
\begin{example}%
    \label{ex:prolongations}
    ~\begin{enumerate}[(i)]
        \item For $T(M) = T(M)$, the space of prolongations is $T(M) \ts{id}{Tp} T^2M = T^2M$, and the second prolongation is given by $T^2M$.
        \item   In a tangent category with a tangent display system, if $\pi \in \d$ then the prolongations for $(\pi, \xi, \lambda, \anc)$ automatically exist.
        In particular, for every anchored bundle in the category of smooth manifolds, all prolongations exist because the projection is a submersion (see \Cref{sec:submersions}).
        \item   For any differential bundle with the anchor $0\o \pi$, it follows that $\prol(A) \cong A\ts{\pi}{\pi\o\pi_i} (A_2) \cong A_3$. 
        %   Consider $(a_0,b):X \to A \ts{0\o \pi}{T\pi} TA$. 
            The universality of the vertical lift factors $b$ into $(a_1,a_2)$, as the following diagram is a pullback:
            \input{TikzDrawings/Ch3/Sec4/zero-anc-prol.tikz}
        \pagenote{The functor from reflexive graphs to anchored bundles was here for some reason, rather than in the examples of anchored bundles piece.}
        \item Returning to the anchored bundle constructed from a graph, the space of prolongations $\prol(A)$ embeds into the second tangent bundle of the space of composable arrows:
        \[ \prolong \hookrightarrow T^2(G \ts{t}{s} G).\]
        % \item   For $c: T.p \to p.T$, the space of prolongations is $T^2M \ts{c}{Tp} T^3M$, which is a ``twisting'' of the third tangent bundle.
    \end{enumerate}
\end{example}
\pagenote{It seemed cleaner to include this proposition in the above examples, since this is really just a remark about }
% First, note that for any tangent category, $\C$ is a reflective subcategory of the category of anchored bundles in $\C$ - the functor sends an anchored bundle to the base object.
% \begin{proposition}%
%     \label{prop:c-refl-in-anc}
%     The category $\C$ is a reflective subcategory of $\mathsf{Anc}(\C)$.
%     \[
%         S: (\pi: A \to M, \xi, \lambda, \anc) \mapsto (p:TM \to M, 0,\ell, id)  
%     \]
% \end{proposition}
% \begin{proof}
%     The endofunctor $S: \mathsf{Anc}(\C) \to \mathsf{Anc}(\C)$ is strictly idempotent $S.S = S$, and $\anc: id \Rightarrow S$ is a unit as $S.\anc = \anc.S = id$. 
% \end{proof}



The category of anchored bundles is, in a sense, ``tangent monadic'' over the category of differential bundles: the forgetful functor from anchored bundles to differential bundles ``creates'' $T$-limits and the tangent structure (this is all made precise in \Cref{ch:inf-nerve-and-realization} using an enriched perspective on tangent categories).
\begin{observation}%
    \label{obs:t-limits-of-anc}
    A limit of anchored bundles is the limit of the underlying differential bundles. Because the anchor is preserved by every map in the diagram, this induces a natural transformation for any $D: \d \to \mathsf{Anc}(C)$:
    \[\input{TikzDrawings/Ch3/limit-of-anc.tikz}\]
    Thus, $\lim U.S.D$ computes the limit of the underlying objects, while $\lim U.D$ computes the limit of the underlying differential bundles in the diagram. The anchor/unit map, then, induces a differential bundle map:
    \[
        \lim \anc_i: (\lim A_i, \lim \lambda_i) \to (T.(\lim_i M_i), \ell.(\lim_i M_i)).
    \]
    This is the limit in the category of anchored bundles (so long as pullback powers of the limit projection and the two prolongations of the limit anchor bundle exist).
    % \[\lim \anc_i: \lim A_i \to \lim T.M_i \cong T.(\lim M_i)\]
\end{observation}
The tangent structure on lifts defined in Proposition \ref{prop:lifts-is-tangent} lifts to a tangent structure on anchored bundles. 
\begin{lemma}\label{lem:tangent-anchored-bundle}
    The category of anchored bundles in a tangent category has a tangent structure that maps objects as follows:
    \[
        (\pi:A \to M, \xi, \lambda, \anc) \mapsto (T.\pi:TA \to TM, T.\xi, c \o T.\lambda, c \o T.\anc)  
    \]
    where the structure maps are all defined using the pointwise structure maps in $\C$. 
    \pagenote{There exists multiplie tangent-like structures on anchored bundles and involution algebroids, so I have added more detail in the statements of those results.}
\end{lemma}
\begin{proof}
    Given an anchored bundle $(q:A \to M,\xi, \lambda, \anc)$, there is an anchor on the differential bundle $(Tq, T\xi, c \o T\lambda)$ given by $c \o T\anc$; the diagram commutes by naturality and the coherences on $c, \ell$.
    \[\input{TikzDrawings/Ch3/Sec4/T-pres-anc.tikz}\]
    The tangent structure maps and universality properties all follow from the forgetful property of the functor from anchored bundles to differential bundles as a consequence of Observation \ref{obs:t-limits-of-anc}.
\end{proof}

For any anchored bundle $A$, there are two differential bundles associated to $\prol(A)$. The first is the usual pullback differential bundle given by pulling back $T.\pi$ along $\anc$ as in Lemma \ref{lem:reindex-db}:
\begin{equation*}
    \input{TikzDrawings/Ch3/Sec4/pullback-prol-dbun.tikz}
\end{equation*}
This gives the differential bundle structure 
\[
    (\prolong \xrightarrow{\pi_0} A, A \xrightarrow{(id, T.\xi \o \anc)} \prolong, \prolong \xrightarrow{(0, c \o T.\lambda)} T(\prolong)).
\]
Taking the fibre product in the category of anchored bundles as in Observation \ref{obs:t-limits-of-anc} yields the second differential bundle structure:
\[
    (A, \pi, \xi, \lambda) \xrightarrow{(\anc, id)} (TM, p, 0, \ell) \xleftarrow{(T\pi, \pi)} (TA, p, 0, \ell).
\]
The two lifts behave similarly to the pair $(T.\ell, \ell.T)$ on the second tangent bundle, and $(\ell,\lambda_T)$ on the tangent bundle of a pre-differential bundle.
\begin{lemma}\label{prop:anc-prol-fun}
% \pagenote{
%     This lemma has been restructured to address possible ambiguities. The two pullback structures on $\prol(A)$ are derived by taking two different pullback in the category of differential bundles. Also there was a tiz
% }
    Let $\C$ be a tangent category and $\mathsf{Anc}(\C)$ the category of anchored bundles in $\C$. If $T$-pullback powers of $p \o \pi_0, \pi_1: \prolong \to A$ exist, then there are two differential bundles on $\prol(A)$, with lifts induced as
    \[\input{TikzDrawings/Ch3/pullbacks-of-lifts-for-prol.tikz}\]
    with structure maps
    \begin{enumerate}
        \item $(\prol(A) \xrightarrow{p\o\pi_1} A, A \xrightarrow{(\xi\pi,0)} \prol(A), \prol(A) \xrightarrow{\lambda \x \ell} T \o \prol(A))$,
        \item $(\prol(A) \xrightarrow{\pi_0} A, A \xrightarrow{(id, T.\xi\o\anc)} \prol(A), \prol(A) \xrightarrow{0 \x c \o T.\lambda} T.\prol(A))$.
    \end{enumerate}
    % \begin{gather*}
    %     (\prol(A) \xrightarrow{p\o\pi_1} A, A \xrightarrow{(\xi\pi,0)} \prol(A), \prol(A) \xrightarrow{\lambda \x \ell} T \o \prol(A)) \\
    %     (\prol(A) \xrightarrow{\pi_0} A, A \xrightarrow{(id, T.\xi\o\anc)} \prol(A), \prol(A) \xrightarrow{0 \x c \o T.\lambda} T.\prol(A))
    % \end{gather*}
    Furthermore, the two lifts $\lambda_\prol$ and $\lambda_\anc$ commute: 
    \[c \o T.(\lambda \x \ell) \o ( 0 \x c \o T.\lambda ) = T.( 0 \x c \o T.\lambda ) \o (\lambda \x \ell).\]
\end{lemma}
\begin{proof} 
    % Consider the following diagram:
    % \input{TikzDrawings/Ch3/Sec4/two-lifts-on-prol.tikz}
    % Regarding the diagram as a pullback in the category of differential bundles gives the first differential bundles structure - look at Proposition \ref{prop:T-limits-of-lifts} and Observation \ref{obs:T-limits-pdbs} to see how limits of differential bundles are computed. 
    % Note that the pullback along a differential bundle projection has the canonical pullback differential bundle structure associated to it by Lemma \ref{lem:reindex-db}, which gives the second differential bundle structure.
    The two differential bundles exist as a consequence of Observation \ref{obs:T-limits-pdbs} and Lemma \ref{lem:reindex-db}, respectively. The commutativity of limits follows by postcomposition, as both $c \o T.\lambda, \ell$ and $\lambda, 0$ commute by the differential bundle axioms.
    % To check that the lifts commute, compute:
    % \begin{gather*}
    %     T.(0\o \pi_0, c \o T.\lambda \o \pi_1) \o (\lambda\o \pi_0, \ell\o \pi_1)
    %     = (T.0 \o \lambda \o \pi_0, T.c \o T.\lambda \o \ell \o \pi_1)\\
    %     = (c \o T.\lambda \o \pi_0, T.c \o \ell \o T.\lambda \o \pi_1) 
    %     = (c \o T.\lambda \o \pi_0, c \o T.\ell \o c \o T.\lambda \o \pi_1) 
    % \end{gather*}
\end{proof}
The above lemma determines a functor, which we denote $\prol$, that sends an anchored bundle in $\C$ to an anchored bundle in $\mathsf{Lift}(\C)$ (equipped with the tangent structure from Proposition \ref{prop:lifts-is-tangent}).
\begin{proposition}%
    \label{cor:prol-functor}
    There is a functor $\prol$ from anchored bundles in $\C$ to anchored bundles in the category of lifts\pagenote{added more context for what $\mathsf{Lift}(\C)$ is} $\mathsf{Lift}(\C)$, that sends an anchored bundle $(\pi:A \to M, \xi, \lambda, \anc)$ to the tuple in $\mathsf{Lift}(\C)$
    \begin{align*}
        &\pi^\prol:= (\prol(A), 0 \x c \o T.\lambda) \xrightarrow[]{p \o \pi_1} (A, 0),\\
        &\xi^\prol := (A,0) \xrightarrow[]{(\xi\o\pi,0)} (\prol(A),0 \x c \o T.\lambda)\\ 
        &\lambda^\prol := (\prol(A), 0\x c\o T.\lambda) \xrightarrow[]{(\lambda \x \ell)} (T(\prol(A)), c \o T(0 \x c \o T.\lambda), \\
        &\anc^\prol := (\prol(A), 0 \x c \o T.\lambda) \xrightarrow[]{\pi_1} (TA, c \o T.\lambda) 
    \end{align*}
    (note that this functor lands in anchored bundles of \emph{non-singular} lifts; see Definition \ref{def:non-singular-lift}).
    Morphisms of anchored bundles \[(f,m):(\pi:A \to M, \xi, \lambda, \anc) \to (q:E \to N, \zeta, l, \delta)\] are sent to
    \[
        \prol(f,m) := f \ts{Tm}{Tm} Tf: \prolong \to E \ts{\delta}{Tq} TE
    \]
\end{proposition}
\begin{proof}
    First, note that the tuple
    \[
        (\pi^\prol: \prol(A) \to A, \xi^\prol, \lambda^\prol, \anc^\prol)  
    \]
    is an anchored bundle, and that each morphism is a lift morphism:
    \begin{itemize}
        \item $\pi^\prol: (\prol(A), 0 \x c \o T.\lambda) \to (A,0)$ follows because
        \begin{align*}
            T.p \o T.\pi_1 \o (\lambda \x \ell) = T.p \o \ell \o \pi_1 = 0 \o p \o \pi_1
        \end{align*}
        \item $\xi^\prol: (A,0) \to (\prol(A), 0 \x c \o T.\lambda)$ follows since
        \[
           ( T.\xi \o T.\pi, T.0) \o 0 = (0 \o \xi \o \pi, T.0 \o 0) = (0 \o \xi \o \pi, c \o T.\lambda \o 0) = (0 \x c \o T.\lambda) \o (\xi \o \pi, 0)
        \]
        \item $\lambda^\prol: (\prol(A), 0 \x c \o T.\lambda) \to (T.\prol(A), 0 \x c \o T.(c \o T.\lambda))$
        follows by the commutativity of $0 \x c \o T.\lambda)$ and $\lambda \x \ell$, and
        \item $\anc^\prol: (\prol(A), 0 \x c \o T.\lambda) \to (TA, c \o T.\lambda)$ is a lift by definition of $\anc^\prol = \pi_1$.
    \end{itemize}
    This gives an anchored bundle in the category of non-singular lifts in $\C$.
    
    Next, check that the mapping is functorial. To see that $\prol(f,m)$ preserves the lifts, note that $(f,m)$ gives a morphism of diagrams for each pullback in $\mathsf{Lift}(\C)$ defining the two lifts, so the induced map $\prol(f,m)$\pagenote{There was some ambiguity in map names here, so $(f,v)$ was switched to $(f,m)$.} preserves each lift. To see that $\prol(f,m)$ preserves the anchor, check that
    \[
        \anc^{\prol B} \o \prol(f,m) = \pi_1 \o (f \x T.f)  = T.f \o \pi_1 = T.f \o \anc^{\prol A}.  
    \] 
\end{proof}
\begin{corollary}%
    \label{cor:prol-endofunctor}
    In a tangent category $\C$ where pullbacks along differential bundle projections exist, such as a tangent category equipped with a proper retractive display system (Definition \ref{def:display-system}) like the category of smooth manifolds, there is an endofunctor on the category of anchored bundles in $\C$,  \[\prol': \mathsf{Anc}(\C) \to \mathsf{Anc}(\C) = U^{\mathsf{Lift}}.\prol.\]
\end{corollary}
 \begin{observation}%
    \label{obs:computation-of-lim-prol}
     $T$-Limits in $\mathsf{Anc}(\mathsf{Lift}(\C))$ are computed as pointwise limits in $\C$ by Observations \ref{obs:t-limits-of-anc} and \ref{obs:T-limits-pdbs}, and $\prol$ is constructed as a limit, so it follows that $\prol$ preserves $T$-limits.
 \end{observation}



\begin{proposition}
    \label{prop:anc-almost-tang-cat}
        The prolongation endofunctor $\prol': \mathsf{Anc}(\C) \to \mathsf{Anc}(\C)$ has natural transformations
    \begin{itemize}
        \item $p': \prol' \Rightarrow id$
        \item $0': id \Rightarrow \prol'$
        \item $+': \prol' \ts{p'}{p'} \prol' \Rightarrow \prol'$
        \item $\ell': \prol' \Rightarrow \prol'.\prol'$
    \end{itemize}
    satisfying all of the axioms of a tangent category that do not incvolve the canonical flip (Definition \ref{def:tangent-cat}).
\end{proposition}
\begin{proof}[(Sketch)]
    Note that the full argument is given in \Cref{sec:weil-nerve}, but it is not difficult to sketch it out here. For the projection, zero map, and addition, use the differential bundle structure induced by Corollary \ref{cor:prol-functor}. To see the lift axiom, note that up to a choice of pullback, we have:
    \[
        \prol'.\prol'(\pi:A \to M, \xi, \lambda, \anc) = 
        \begin{cases}
            \quad \quad \quad \quad (p \o \pi_1, p \o \pi_2):& \prol^2(A) \to \prol(A) \\
            (\xi \o \pi \o \pi_0, 0 \o \pi_1, 0 \o \pi_2):& \prol(A) \to \prol^2(A) \\
            \quad \quad \quad \quad \quad \quad (\lambda \x \ell \x \ell):&\prol^2(A) \to T.\prol^2(A) \\
            \quad \quad \quad \quad \quad \quad \quad (\pi_1, \pi_2):& \prol^2(A) \to T.\prol(A)
        \end{cases}  
    \]
    The ``lift map'' is then
    \[
        \ell': \prol' \Rightarrow \prol'.\prol'; \hspace{0.15cm}
        \prol(A) \xrightarrow[]{A \x T.\hat\lambda} \prol^2(A) 
    \]
    (where we recall that $\hat\lambda = (\xi\o\pi,\lambda):A \to \prol(A)$).
\end{proof}

\begin{remark}
    The structure described in Proposition \ref{cor:prol-functor} leads to the theory of \emph{double vector bundles}, developed by MacKenzie and his collaborators \cite{flari2019warps,Mackenzie1992}. A double vector bundle is a commuting square
    \[\input{TikzDrawings/Ch3/dvb.tikz}\]
    where each projection is a vector bundle projection, and ``vertical'' and ``horizontal'' orientations of the square are each vector bundle homomorphisms. It was observed by \cite{Grabowski2009} that this is equivalent to a pair of commuting $\R^+$-actions on the total space $E$; following the development in \Cref{ch:differential_bundles}, this is a \emph{commuting} pair of non-singular lifts on $E$,
    \[
        \lambda^A,\lambda^B:E \to TE, \hspace{0.15cm}
        T.\lambda^B \o \lambda^A = c \o T.\lambda^A \o \lambda^B.  
    \]
    A proper exposition of the so-called ``Ehresmann doubles'' (\cite{Mackenzie2011}) for the structures in Lie theory would substantially expand the scope of this thesis, and so it has been relegated to the margins.
\end{remark}


Finally, observe that a connection (see Section \ref{sec:connections-on-a-differential-bundle}) on an anchored bundle's underlying differential bundle behaves similarly to an affine connection.
\begin{lemma}\label{lem:anchored-bundle-conn}
    Let $(\pi:A \to M, \xi, \lambda, \anc)$ be an anchored bundle with $\prol(A)$ existing in a tangent category $\C$.\pagenote{The original draft had a typo - it was meant to be ``anchored bundle equipped with a connection''.}
    If $(\pi, \xi, \lambda)$ has a connection, then define \[\hat{\kappa} := \kappa \pi_1, \quad \hat{\nabla} := \nabla(\pi_0, \anc\pi_1)\] and note that
    \begin{enumerate}[(i)]
        \item $\hat{\kappa}: \prol(A) \to A$ is a retract of $(\xi\pi, \lambda)$, and $\hat{\kappa}: (\prol(A),l) \to (A, \lambda)$ is linear for both $l = (\lambda \x \ell), (0 \x c \o T.\lambda)$;
        \item $\hat{\nabla}:A_2 \to \prol(A)$ is a section of $(\pi_0, p \o \pi_1)$ and is bilinear;
        \item there is an isomorphism $\prol(A) \cong A_3$:\pagenote{added the explicit isomorphism}
        \[
            \nabla(p \o \pi_1, \anc \o \pi_0) +_{(\pi, T.\pi)} \hat{\mu}(p \o \pi_1, \hat{\kappa}) = id. 
        \]
    \end{enumerate}
\end{lemma}
\begin{example}
    Every vector bundle in the category of smooth manifolds has a connection; it follows that  Lemma \ref{lem:anchored-bundle-conn} holds for every anchored bundle in the category of smooth manifolds.
\end{example}

\begin{remark}
    The category of anchored bundles in a tangent category is almost a tangent category, except that it lacks a symmetry map. The ``differential objects'' in such a category will act like a cartesian differential category, except that the symmetry of mixed partial derivatives fails. There has been some interest in settings for differentiable programming where the symmetry of mixed partial derivatives need not hold (see Definition 3.4 along with the discussion at the end of Section 6 in \cite{cruttwell2021categorical});\pagenote{I have included an actual reference to quantify some interest.} the category of anchored bundles in a tangent category appears to be a source of examples.
\end{remark}
\section{Involution algebroids}%
\label{sec:involution-algebroids}

% In  \Cref{sec:Lie_algebroids}, a formalism for describing Lie algebroids and Lie algebroid morphisms that would work in an arbitrary tangent category (Definition \ref{def:tangent-cat}). However, this presentation relies on the existence of a connection on the differential bundle and base space of the Lie algebroid, and there are objects in tangent categories that do not have affine connections (see, for example, the category $\wone$ outlined in \cref{sec:weil-algebras-tangent-structure}), and this would break the intuition that a Lie algebroid is a generalized tangent bundle. 

Involution algebroids are a tangent-categorical axiomatization of Lie algebroids. Recall that a Lie algebroid has almost all of the structure of the operational tangent bundle, in particular a Lie bracket on sections that satisfies the Leibniz law so that it gives a directional derivative. % (see Observation \ref{obs:operational-tangent-bundle} for discussion of the operational tangent bundle). 
In Proposition \ref{prop:anc-almost-tang-cat}, it was demonstrated that the category of anchored bundles have almost all of the same maps as the tangent bundle, the only structure map missing being the canonical flip
\[
    c:T^2M \Rightarrow T^2M
\]
which, we may recall from the chapter introduction (and \cite{Cockett2015}, \cite{Mackenzie2013}), is used in constructing the Lie bracket of vector fields in a tangent category with negatives. Thus, a natural next step is to add an involution map to an anchored bundle and require that it satisfies the same coherences as $c$ from the tangent bundle.


\begin{definition}\label{def:involution-algd}
    An involution algebroid is an anchored bundle $(A, \pi,\xi,\lambda, \anc)$ equipped with a map $\sigma: \prol(A) \to \prol(A)$ satisfying the following axioms:
    \begin{enumerate}[(i)]
        \item Involution: \[\input{TikzDrawings/Ch3/inv-is-inv.tikz}\]
        \item Double linearity:\[% https://q.uiver.app/?q=WzAsNCxbMCwxLCJcXHByb2woQSkiXSxbMSwxLCJcXHByb2woQSkiXSxbMCwwLCJULlxccHJvbChBKSJdLFsxLDAsIlQuXFxwcm9sKEEpIl0sWzAsMSwiXFxzaWdtYSJdLFsxLDMsIjAgXFx4IGMgXFxvIFQuXFxsYW1iZGEiLDJdLFsyLDMsIlQuXFxzaWdtYSJdLFswLDIsIlxcbGFtYmRhIFxceCBcXGVsbCJdXQ==
        \begin{tikzcd}
            {T.\prol(A)} & {T.\prol(A)} \\
            {\prol(A)} & {\prol(A)}
            \arrow["\sigma", from=2-1, to=2-2]
            \arrow["{0 \x c \o T.\lambda}"', from=2-2, to=1-2]
            \arrow["{T.\sigma}", from=1-1, to=1-2]
            \arrow["{\lambda \x \ell}", from=2-1, to=1-1]
        \end{tikzcd}\]
        \item Symmetry of lift: \[\input{TikzDrawings/Ch3/symmety-sig-l.tikz}\]
        \item Target: \[\input{TikzDrawings/Ch3/anc-is-inva-morphism.tikz}\]
        \item Yang--Baxter: \[\input{TikzDrawings/Ch3/yb.tikz}\]
    \end{enumerate}
    $A$ is an \emph{almost}-involution algebroid if the involution does not satisfy the Yang--Baxter equation.
    A morphism of involution algebroids is a morphism of anchored bundles, so that $\prol(f)\o \sigma_A = \sigma_B\o\prol(f)$.
    (Note that because $\sigma$ is an isomorphism and $\sigma = \sigma^{-1}$, 
    \[\sigma: (\prol(A), 0 \x c\o T.\lambda) \to (\prol(A),\lambda\x \ell)\]
    is linear as well.) Write the category of involution algebroids and involution algebroid morphisms in $\C$ is written $\mathsf{Inv}(\C)$.\pagenote{
    The category $\mathsf{Inv}(\C)$ has been explicitly defined - this addresses a few later remarks in this section.
}
\end{definition}

\begin{observation}
    It is not immediately clear that the Yang--Baxter equation is well-typed. This follows from the target axiom (iv) and the double linearity axiom (ii). Starting with $(u,v,w): \prolong \ts{T.\anc}{T^2.\pi} T^2A$, we see that $\sigma \x c$ is well-typed if and only if
    \[
       T.\anc \o \pi_1 \o \sigma(u,v) = T^2.\pi \o c \o w = c \o T^2.\pi \o w = c \o T.\anc \o v.
    \]
    Similarly, $1 \x T.\sigma$ is well-typed if $T.\anc \o u = T.\pi \o \pi_0 \o T.\sigma(v,w)$; then use the double linearity axiom to compute
    \begin{gather*}
        T.\pi \o \pi_0 \o T.\sigma(v,w) = T.\pi \o T.p \o T.\pi_1 (v,w) = T.\pi \o T.p \o w 
        \\ = T.p \o T^2.\pi \o w = T.p \o T.\anc \o v = T.\pi \o v.
    \end{gather*}
\end{observation}
This perspective on Lie algebroids has already appearad in the work of Martinez and his collaborators in \cite{Leon2005}, where a ``canonical involution'' was derived on space of prolongations of a Lie algebroid using the formula
\[
    \sigma: \prol(A) \to \prol(A); \sigma(x,y,z) = \sigma(y,x,z + \< x, y\>).
\]
The structure of this map has been largely unexplored; helpfully, involution algebroids succeed in reverse-engineering axioms for an involution map that will induce a Lie bracket on the sections of the projection map. \pagenote{I removed the remark about the Yang-Baxter equation, as that is discussed more thoroughly in the introduction of this chapter. Identifying axioms on the involution to identify Lie brackets is original, I think I may have muddied the waters with my original remark.} 
The bracket from the original Lie algebroid is induced using the same formula as for the bracket of vector fields on the tangent bundle:
\[
    \lambda \o [X,Y]^* 
    =
    \left( 
        (\pi_1 \o \sigma \o (id, T.X \o \anc) \o Y -_p T.Y \o X \o \anc) -_{T.\pi} 0Y
    \right).
\]
% Choosing a connection on $A$,  this map is equivalent to (see equation (4.1) in \cite{Leon2005})
% \[
%    (u,v,w):A_3 \mapsto (v,u,w + \< u,v \>):A_3
% \]
Furthermore, a morphism of anchored bundles is a Lie algebroid morphism if and only if it preserves the derived involution map.

\begin{example}\label{ex:inv-algds}
    \pagenote{I have updated this example following to clarify the relationships between $\C, \mathsf{Anc}(\C), \mathsf{Inv}(\C)$.}
    ~\begin{enumerate}[(i)]
        \item For any $M$ in $\C$, $(TM, p, 0, id, c)$ is an involution algebroid. Furthermore, for any involution algebroid anchored on $M$, $(\anc, id)$ is a morphism of involution algebroids (by the target axiom). This defines a fully faithful functor $\C \hookrightarrow \mathsf{Inv}(\C)$. The same construction as the anchored bundle case in Example \ref{ex:anchored-bundles}(i) exhibits $\C$ as a reflective subcategory of $\mathsf{Inv}(\C)$.
        \item For any differential bundle, the trivial anchored bundle $(\pi:A \to M,\xi,\lambda, 0\o\pi)$ has an involution using the isomorphism $\prol(A) \cong A_3$, given by $\sigma := (\pi_1,\pi_0,\pi_2)$ (proving this map satisfies the involution algebroid axioms is just an exercise in combinatorics). It follows that every differential bundle morphism gives rise to a morphism of these \emph{trivial} involution algebroids. The same construction from the anchored bundle case (Example \ref{ex:anchored-bundles}(ii)) exhibits $\mathsf{Diff}(\C)$ as a \emph{coreflective} subcategory of involution algebroids.
        \item Consider a groupoid 
        \[s,t:G \to M, \hspace{0.15cm} e: M \to G, \hspace{0.15cm} (-)^{-1}:G \to G, \hspace{0.15cm} m:G_2 \to G.\] 
        The underlying reflexive graph has an associated anchored bundle, constructed in Example \ref{ex:anchored-bundles}, and the space of prolongations of this anchored bundle includes into $(T^2.G_2)$. Note that there is a well-formed involution map:
        \[
            \sigma(u,v) = 
            c \o ((0 \o p \o v)^{-1}; (T.0 \o u);v)
        \] 
        The direct proof of this involves a more conceptual construction, which is the focus of Section \ref{sec:inf-nerve-of-a-gpd}.
    \end{enumerate}
\end{example}

An involution algebroid resembles a generalized tangent bundle, and so the lift $(\xi\o\pi, \lambda):A \to \prol(A)$ satisfies the same universality conditions as $\ell:T \Rightarrow T^2$. The double linearity condition is equivalent to the naturality condition for $c, \ell$ in Definition \ref{def:tangent-cat}:
\begin{equation}
    \label{eq:symmetry-of-lift}
    \input{TikzDrawings/Ch3/Sec5/twist-axiom-prol.tikz}
\end{equation}
which, using string diagrams for monoidal categories (\cite{selinger2010survey}), is the equation
\begin{equation*}
    % \label{eq:symmety-of-lift}
    \input{TikzItDrawings/twist-axiom.tikz}
\end{equation*}
where the circle denotes the lift and $c$ is the crossing of two lines. 
\begin{proposition}%
    \label{prop:nat-of-sigma-ell}
    Let $(\pi:A \to M, \xi, \lambda, \anc)$ be an anchored bundle, and suppose that
    \[
        \sigma: \prol(A) \to \prol(A)  
    \]
    satisfies the involution axiom (i) and the symmetry of lift axiom (iii), and furthermore that the involution ``exchanges'' the idempotents associated to the two lifts $(\lambda \x \ell)$ and $(0 \x c \o T.\lambda)$ on $\prol(A)$ (Proposition \ref{prop:idempotent-natural})
    \[
        \sigma \o ((p \o \lambda) \x (p \o \ell)) = (p \o 0) \x (p \o c \o T.\lambda) = id \x (T.p \o T.\lambda)
    \]
    Then the double linearity axiom is equivalent to the left-hand commuting diagram in Diagram \ref{eq:symmetry-of-lift}:
    \[
      (\hat{\lambda} \x \ell) \o \sigma = (id \x T.\sigma) \o (\sigma \x c) \o (id \x T.\hat{\lambda}).  
    \]
\end{proposition}
\begin{proof}
    Starting with the left-hand side of the equation, 
    \begin{align*}
        &\quad (id \x T.\sigma) \o (\sigma \x c) \o (id \x T.\hat\lambda) \o (u,v) \\
        &= (id \x T.\sigma) \o (\sigma \x c) \o (u, T.\xi \o T.\pi \o v, T.\lambda v) \\
        &= (id \x T.\sigma) \o (\sigma \o (u, T.\xi \o \anc \o u), T.\lambda v) \\
        &= (id \x T.\sigma) \o (\xi \o \pi \o u, 0 \o \anc \o u, T.\lambda v) \\
        &= (id \x T.\sigma) \o (\xi \o \pi \o u, 0 \o T.\pi \o v, T.\lambda v) \\
        &= (\xi \o \pi \o u, T.\sigma \o \hat \lambda \o v) .
    \end{align*}
    For the right-hand side:
    \begin{align*}
        &\quad (\hat \lambda, \ell) \o \sigma \o (u,v) \\
        &= (\xi \o \pi \o \pi_0 \o \sigma\o (u,v), (\lambda \x \ell) \o \sigma\o (u,v)) \\
        &= (\xi \o \pi \o p \o v, (\lambda \x \ell) \o \sigma\o (u,v)) \\
        &= (\xi \o p \o T.\pi \o v, (\lambda \x \ell) \o \sigma\o (u,v)) \\
        &= (\xi \o p \o \anc \o u, (\lambda \x \ell) \o \sigma\o (u,v)) \\
        &= (\xi \o \pi \o u, (\lambda \x \ell) \o \sigma\o (u,v)) \\
        &= (\xi \o \pi \o u, T.\sigma \o (0 \x c \o T.\lambda) \o (u,v)).
    \end{align*}\pagenote{Added a step to clarify the calculation.}
    So the naturality equation \ref*{eq:symmetry-of-lift} is equivalent to
    \[
        T.\sigma \o (0 \x c \o T.\lambda) = (\lambda \x \ell) \o \sigma.
    \]
\end{proof}



The category of involution algebroids is ``tangent monadic'' over the category of anchored bundles, in the same sense that anchored bundles are tangent monadic over the category of differential bundles, or internal categories over reflexive graphs in a category. The ``tangent monadicity'' leads to a similar observation about $T$-limits of involution algebroids as in Observation \ref{obs:t-limits-of-anc}.
\begin{observation}%
    \label{obs:inv-tmonad-anc}
    The forgetful functor from involution algebroids to anchored bundles creates limits; that is to say, the $T$-limits of the underlying anchored bundles give the limits of involution algebroids. Recall that by Observation \ref{obs:computation-of-lim-prol}, the limit $\prol(\lim A_i) = \lim \prol(A_i)$ in the category of anchored bundles, and this induces a map between objects in $\C$
    \[
        \lim \sigma_i: \lim \prol(A_i) \to \lim \prol(A_i).  
    \]
    The axioms for an involution algebroid are induced by universality. 
\end{observation}
Note that a point-wise tangent structure may be defined following the anchored bundles example from Lemma \ref{lem:tangent-anchored-bundle}:\pagenote{Added some preamble to this result to help differentiate it from the prolongation tangent structure.}
\begin{proposition}%
    \label{prop:pointwise-tangent-structure-inv}
    For a tangent category $\C$, the category of involution algebroids has a ``point-wise'' tangent structure that maps objects as follows:
    \[
        (\pi:A \to M, \xi, \lambda, \anc, \sigma) \mapsto
        (T.\pi, T.\xi, c \o T.\lambda, T.\anc, \sigma_T)  
    \]
    where $\sigma_T$ is defined as
    \begin{equation}\label{eq:sigmaTdeff}
        \sigma_T := (1 \x_c c)\o T.\sigma\o (1 \x_c c): \prol(A_T) \to \prol(A_T).
    \end{equation}
\end{proposition}
\begin{proof}    
    The tangent structure on involution algebroids is inherited from the functor $\mathsf{Inv}(\C) \to \mathsf{Anc}(\C)$. Thus, it suffices to give the involution map for the tangent involution algebroid.

    Note that given $(\pi:A \to M, \xi, \lambda, \anc, \sigma)$, the space of prolongations on $(T.\pi, T.\xi, \newline c\o T.\lambda, c\o T.\anc)$ is $TA \ts{c\o T.\anc}{T^2\pi} T^2A$. We can construct an isomorphism between the objects $\prol(TA)$ and $T(\prol(A))$ in $\C$ using the cospan isomorphism
    \input{TikzDrawings/Ch3/Sec5/prol-via-cospan.tikz}
    thus inducing a map
    \[
       TA \ts{c\o T.\anc}{T^2\pi} T^2A
       \xrightarrow[]{id \x c} T(\prolong)
       \xrightarrow[]{T.\sigma} T(\prolong)
       \xrightarrow[]{id \x c} TA \ts{c\o T.\anc}{T^2\pi} T^2A
    \]
    which we call $\sigma_T$. The linearity and involution axioms follow by construction.
    
    Now check the rest of the axioms. For the unit:
    \begin{gather*}
        (1 \x_c c)\o T.\sigma\o (1 \x_c c) \o (T.(\xi\o\pi), c \o T.\lambda)\\
        = (1 \x_c c) \o T.\sigma \o (T(\xi\pi), T\lambda) 
        = (1 \x_c c) \o (T.(\xi\o\pi), T.\lambda) = (T.(\xi\o\pi), c \o T.\lambda).
    \end{gather*}
    For the anchor:
    \begin{gather*}
        T.\anc_T \o \pi_1 \o \sigma_T = T(c\o T\anc) \o \pi_1 \o (1 \x_c c)\o T.\sigma\o (1 \x_c c)\\
        = T.c\o c \o  T^2.\anc \o T.\pi_1 \o T.\sigma\o (1 \x_c c)
        = T.c \o c \o T.c \o T^2.\anc \o T.\pi_1 \o (1 \x_c c)\\
        = c \o T.c \o c \o T^2\anc \o c \o \pi_1  = c \o Tc \o T^2.\anc \o \pi_1
        = c \o T.\anc_T \o \pi_1.
    \end{gather*}
    The Yang--Baxter equation is straightforward to check.
\end{proof}
The tangent bundle is a canonical involution algebroid on every object in a tangent category, and the anchor induces a morphism from an involution algebroid to the tangent involution algebroid on its base space.
The anchor acts as a reflector from involution algebroids in $\C$ to $\C$ itself.
\begin{proposition}%
    \label{prop:c-refl-in-inv}
    Any tangent category $\C$ is a reflective subcategory of the category of involution algebroids in $\C$.
\end{proposition}
\begin{proof}
    First, observe that the inclusion of $\C$ into $\mathsf{Inv}(\C)$ (Definition \ref{def:involution-algd})\pagenote{The notation $\mathsf{Inv}(\C)$ had not yet been introduced at this point, it has been added to the definition of involution algebroids and a reference has been made to that definition.} is fully faithful because the anchor on the tangent involution algebroid is $id: TM \to TM$. This means that the only involution algebroid morphisms $TA \to TB$ are pairs $(Tf,f)$, $f: A \to B$.
    Now consider the functor $\mathsf{Inv}(\C) \to \C$ that sends $(\pi:A \to M, \xi, \lambda, \anc, \sigma)$ to $M$: this gives an endofunctor $S: \mathsf{Inv}(\C) \to \mathsf{Inv}(\C)$ along with a natural transformation $\anc: id \Rightarrow S$, so that $S.\anc = \anc.S = id$, given by the anchor map.
    Thus,  the category $\C$ is the category of algebras for an idempotent monad on $\mathsf{Inv}(\C)$.
\end{proof}
\begin{corollary}
    Let \[ \widehat{A} = (\pi:A \to M, \xi, \lambda, \anc, \sigma) \]   be an involution algebroid in a tangent category $\C$.
    \begin{enumerate}[(i)]
        % \item The category of involution algebroids is a tangent category, so that given an involution algebroid:
        % \[ \widehat{}{A} = (\pi:A \to M, \xi, \lambda, \anc, \sigma) \]   
        \item The morphism
        \[(T.\pi,\pi): TA \to TM\]
        is an involution algebroid morphism from the tangent involution algebroid on $A$ to the tangent involution algebroid on $M$.
        \item The morphism
        \[(\anc,id):\widehat{A} \to TM\]  
        is an involution algebroid morphism.
    \end{enumerate}
    If pullback powers of $p \o \pi_1: \prol(A) \to A$ exist, then
    \[
        (p\o \pi_1: \prol(A) \to A, (\xi \o \pi, 0), (\lambda \x \ell))
    \] is a differential bundle\pagenote{A ``differentiable bundle'' slipped in.}; note that $\pi_0$ acts as an anchor. If the prolongations exist, then 
    \[
        (p \o \pi_1:\prol(A) \to A, (\xi \o \pi, 0), (\lambda \x \ell), \pi_0, 
        \sigma')
    \] is an involution algebroid; this follows from computing the pullback in the category of involution algebroids ($\sigma'$ is induced as in Observation \ref{obs:t-limits-of-anc}).
\end{corollary}
The above corollary puts an involution algebroid structure on the differential bundle $(\prol(A), \lambda \x \ell)$. Note that the map $\sigma$ gives an isomorphism of differential bundles
\[
  (\prol(A),\lambda \x \ell) \to (\prol(A),0 \x c \o T.\lambda).
\]
Martinez observed that the canonical involution $\sigma$ puts a unique Lie algebroid structure on $(\prol(A),0 \x c \o T.\lambda)$, and $\sigma$ is a uniquely determined isomorphism of involution algebroids (see \cite{Leon2005}):
\begin{corollary}\label{cor:second-lie-algd-unique-map}
    For an involution algebroid $(\pi:A \to M, \xi, \lambda, \anc, \sigma)$, the isomorphism of differential bundles
    \[
        \sigma: (\prol(A), 0 \x c \o T.\lambda) \to (\prol(A), \lambda \x \ell)
    \]
    induces a second involution algebroid structure on $\prol(A)$.
\end{corollary}

Recall that Proposition \ref{prop:anc-almost-tang-cat} sketched out a proof that the category of anchored bundles in $\C$ has an endofunctor $\prol'$ and natural transformations $p', 0', +', \ell'$ satisfying the axioms of a tangent structure; this endofunctor and the natural transformations all lift to $\mathsf{Inv}(\C)$. The involution map $\sigma$ is the missing piece that gives a tangent structure on $\mathsf{Inv}(\C)$. The construction may be spelled out here at a big-picture level, but the actual proof brings up tricky coherence issues that make up the bulk of Chapter \ref{chap:weil-nerve}.
\begin{proposition}%
    \label{prop:second-tangent-structure-inv-algds}
    The category of involution algebroids in a tangent category $\C$ has a second tangent structure, where the tangent functor is given by\pagenote{Changed wording to help differentiate this structure against the pointwise tangent structure defined beforehand.}
    \[
        \prol': \mathsf{Inv}(\C) \to \mathsf{Inv}(\C)  
    \]
    and the tangent natural transformations are given as in Proposition \ref{prop:anc-almost-tang-cat}, with the canonical flip
    \[
        \sigma': \prol'.\prol' \to \prol'.\prol' := \prol^2(A) \xrightarrow[]{1 \x T.\sigma} \prol^2(A).  
    \]
    Starting with an involution algebroid, $\prol'(A)$ and $\prol'.\prol'(A)$ are given by
    \[
       \prol'(A) 
       \begin{cases}
            \pi':& \prol(A) \xrightarrow[]{p \o \pi_1} A \\
            \xi':& A \xrightarrow[]{(\xi \o \pi, 0)} \prol(A) \\
            \lambda':& \prol(A) \xrightarrow[]{\lambda \x \ell} T.\prol(A) \\
            \anc':& \prol(A) \xrightarrow[]{\pi_1} TA \\
            \sigma':& \prol^2(A) \xrightarrow{\sigma \x c} \prol^2(A)
        \end{cases}  
        \hspace{0.15cm}
        \prol'.\prol'(A)
        \begin{cases}
            \pi''& \prol^2(A) \xrightarrow[]{(p \o \pi_1, p \o \pi_2):} \prol(A) \\
            \xi'':& \prol(A) \xrightarrow[]{(\xi \o \pi \o \pi_0, 0 \o \pi_1, 0 \o \pi_2)} \prol^2(A) \\
            \lambda'':&\prol^2(A) \xrightarrow[]{(\lambda \x \ell \x \ell)} T.\prol^2(A) \\
            \anc'':& \prol^2(A) \xrightarrow[]{(\pi_1, \pi_2)} T.\prol(A) \\
            \sigma'':& \prol^3(A) \xrightarrow[]{(\sigma \x c \x c)}\prol^3(A) 
        \end{cases}  
    \]
    (where $\prol^3(A) = \prolong \ts{T.\anc}{T^2.\pi} T.(\prolong)$).
    The tangent natural transformations are given by
    \begin{itemize}
        \item $p: \prol' \Rightarrow id; \hspace{0,15cm} \prol(A) \xrightarrow[]{\pi_0} A$ 
        \item $0: id \Rightarrow \prol'; \hspace{0,15cm} A \xrightarrow[]{(id,T.\xi \o T.\pi)} \prol'(A)$
        \item $+: \prol'_2 \Rightarrow \prol'; \hspace{0,15cm} A \ts{\anc}{T.\pi \o \pi_i} T_2A \xrightarrow[]{id \x +} \prol(A)$
        \item $\ell:\prol'(A) \Rightarrow \prol'.\prol'; \hspace{0,15cm} \prol(A) \xrightarrow[]{1 \x T.(\xi\o\pi, \lambda)} \prol^2(A)$
        \item $\anc': \prol' \Rightarrow T$
        \item $c:\prol'.\prol'(A) \Rightarrow \prol'.\prol';\hspace{0,15cm}  \prol^2(A) \xrightarrow[]{1 \x T.\sigma} \prol^2(A)$.
    \end{itemize}
\end{proposition}
\begin{proof}
    Deferred to \Cref{sec:prol_tang_struct}.
\end{proof}
% \begin{remark}
%     This construction of pulling back a map $(T.f,f):TN \to TM$ along the anchor map $\anc: A \to TM$ in the category of involution algebroids is called the \emph{prolongation of $f$ along $A$} in the Lie algebroid literature (see, for example, \cite{Leon2005,Martinez2018}).
% \end{remark}

% \begin{lemma}
%     Let $(\pi:A \to M,\xi,\lambda,\anc,\xi)$ be an involution algebroid in a tangent category $\C$, and $f:N \to M$ a map.
%     If the prolongation $\prol_f(A)$ exists, it is an involution algebroid over $N$.
% \end{lemma}
% \begin{proof}
%     By Observation \ref{obs:inv-tmonad-anc}, \cref{eq:prol-by-f} induces a pullback in the category of involution algebroids where the base square is the pullback of $f$ along $id$.
% \end{proof}


% \begin{definition}
%     Let $(\pi:A \to M,\xi,\lambda,\anc,\xi)$ be an involution algebroid in a tangent category $\C$, and $f:N \to M$ a map.
%     The prolongation of $f$ along $A$ is the $T$-pullback:
%     % https://q.uiver.app/?q=WzAsNCxbMSwwLCJUTiJdLFsxLDEsIlRNIl0sWzAsMSwiQSJdLFswLDAsIlxccHJvbF9mQSJdLFsyLDEsIlxcYW5jIiwyXSxbMCwxLCJUZiJdLFszLDIsIlxccGlfMCIsMl0sWzMsMCwiXFxwaV8xIl1d
%     \begin{equation}\label{eq:prol-by-f}
%         \input{TikzDrawings/Ch3/Sec5/prol-by-f.tikz}
%     \end{equation}
% \end{definition}
% Using this lemma and the fact that $\anc$ is an involution algebroid morphism, there is an involution algebroid structure on $\prol(A)$.
% \begin{proposition}
%     Let $(\pi:A \to M,\xi,\lambda,\anc,\xi)$ be an involution algebroid in a tangent category $\C$ so that $\pi$ is $T$-display.
%     Then the anchored bundle $(p\o\pi_1:\prol(A) \to A, (\xi\pi,0),(\lambda, \ell),\pi_0)$ has an involution algebroid structure.
% \end{proposition}

\section{Connections on an involution algebroid}%
\label{sec:connections_on_an_involution_algebroid}
In this section we take an involution algebroid with a chosen linear connection on its underlying anchored bundle (Definition \ref{def:involution-algd}, \ref{def:lin-connection}), and rederive Martinez's structure equations for a Lie algebroid (Proposition \ref{prop:la-iff-structure-morphisms}). 
% This section relates the coherences of an involution algebroid (Definition \ref{def:involution-algd}) with Martinez's \emph{structure equations} for a Lie algebroid  Recall that in Section \ref{sec:Lie_algebroids} the structure morphisms and equations were translated into tangent-categorical syntax using an explicit choice of connection on the underlying anchored bundle. By making an explicit choice of a connection on an \emph{involution} algebroid, we may rederive the structure equations in a more general setting.
% This section relati
% Martinez's structure equations for a Lie algebroid (--) rely upon the fact that every vector bundle in the category of smooth manifolds admits a connection. 
% In the category of smooth manifolds, every vector bundle has a connection, so the space of prolongations for an anchored bundle admits a decomposition $\prol(A) \cong A_3$.  In practice, this means the simplest way to define a geometric structure is  multilinear operations on an anchored bundle and a covariant derivative are . 
% Thus, in comparing involution algebroids to Lie algebroids, it will be helpful to characterize involution algebroids in terms of multilinear operations on $A$ using a connection. We work in in an arbitrary tangent category with negatives, where appropriate connections exist when stated.

In a tangent category $\C$  with negatives, there is a natural transformation\pagenote{Clarifying notation for this section as there is substantially more fibered linear algebra, including the use of substraction, than is present in other sections/chapters.}
\[
    n:T \Rightarrow T  
\]
making each fibred commutative monoid $(p:TM \Rightarrow M, 0, +, n)$ a fibred abelian group. Because the additive bundle structure on differential bundles is induced via universality (Proposition \ref{prop:induce-abun}), in a tangent category with negatives the additive bundle structure for differential bundles will likewise have negatives.
We adopt the following notation for the ``fibred linear algebra'' used in this section, as there are a significant number of computations done on bundles with multiple choices of additive bundle structure (e.g. the second tangent bundle of a differential bundle has three additive bundles structures).
\begin{notation}
    Let $E$ be an object with multiple differential bundle structures $(\pi^i:E \to M^i, \xi^i, \lambda^i)$ in a tangent category with negatives. Addition over a specific differential bundle is written using infix notation, where a subscript is added to the symbol denoting the projection of the differential bundle. Letting $x,y:X \to E$ denote a pair of generalized elements for which the $\pi^i$-addition operation is well-defined; we set
    \[
        x +_{\pi[i]} y = X \xrightarrow[]{(x,y)} E \ts{\pi[i]}{\pi[i]} E \xrightarrow[]{+^i} E.
    \]
    Similar notation is used for subtraction:
    \[
        x -_{\pi[i]} y  = X 
        \xrightarrow[]{(x,y)} E \ts{\pi[i]}{\pi[i]} E 
        \xrightarrow[]{id \x n[i]} E \ts{\pi[i]}{\pi[i]} E 
        \xrightarrow[]{+^i} E.
    \]
\end{notation}

Throughout this section, we work in a tangent category $\C$ with negatives\pagenote{I had  originally forgotten to add the ``with negatives'' caveat} and a chosen anchored bundle $(\pi:A \to M, \xi, \lambda, \anc)$, equipped with a connection $(\kappa,\nabla)$, whose base object has a torsion-free connection $(\kappa',\nabla')$ and a morphism
\[
    \sigma: \prol(A) \to \prol(A)
\]
that exchanges the two projection maps $p \o \pi_1, \pi_0$, meaning that the following diagram commutes:
\begin{equation*}
    \input{TikzDrawings/Ch3/Sec6/connection-notation.tikz}
\end{equation*}
The following notation will be useful when working with local coordinates.
\begin{notation}
    First, recall the $\nabla$-notation for morphisms of differential bundle where each morphism has a choice of connection from Equation \ref{eq:nabla-notation}:
    \[
        \infer{\nabla[f] := \kappa^B \o T.f \o \nabla^A:A_2 \to B}{f:A \to B & (\kappa^A,\nabla^A,A, \lambda^A) & (\kappa^B,\nabla^B, B, \lambda^B)}
        % 
    \]
    Let \[(\pi:A \to M, \xi, \lambda, \anc), (q:B \to N, \zeta, l, \rho)\] be a pair of anchored bundles equipped with connections.
    ``Hatting'' a map $f:\prol(A) \to \prol(B)$ refers to the map:
    \[
        \widehat{f}:A_3 \xrightarrow[]{\hat\nu(\pi_0,\pi_2) + \hat\nabla(\pi_0,\pi_1)} \prol(A) \xrightarrow[]{f} \prol(B) \xrightarrow[]{(\pi_0,p \o \pi_1, \kappa \o \pi_1)} B_3.  
    \]
    Similarly, for $f:TA \to TB$,
    \[
        \widehat{f}: TM \ts{p}{\pi} A \ts{\pi}{\pi} A \xrightarrow[]{\nu(\pi_0,\pi_2) + \nabla(\pi_0,\pi_1)} TA \xrightarrow[]{f} TB \xrightarrow[]{(\pi_0,p \o \pi_1, \kappa \o \pi_1)} TM \ts{p}{q} B \ts{\pi}{q} B  
    \]
    Clearly, $\widehat{f \o g} = \widehat{f} \o \widehat{g}$. Similarly, a map $A_3 \to B_3$ may be ``barred'' to form a map $\prol(A) \to \prol(B)$:
    \[
        \overline{g}:\prol(A) \xrightarrow[]{(\pi_0,p \o \pi_1, \kappa \o \pi_1)} A_3  \xrightarrow[]{g} B_3 \xrightarrow[]{\hat\nu(\pi_0,\pi_2) + \hat\nabla(\pi_0,\pi_1)} \prol(B).
    \]
    It is straightforward to see that $\overline{\widehat{f}} = f, \widehat{\overline{g}} = g$.
\end{notation}



\begin{lemma}\label{lem:intertwining-induces-bracket}
    $\sigma: \prol(A) \to \prol(A)$ induces a bracket on $\Gamma(\pi)$:
    \[
        \lambda \o [X,Y]
        = (\sigma \o (id, TX \o \anc) \o Y -_{T\pi} (id, TY \o \anc) \o X) - 0\o Y.
    \]    
\end{lemma}
\begin{proof}
    Let $X,Y \in \Gamma(\pi)$ and compute:
    \begin{align*}
         &\quad p \o \pi_1 \o (\sigma \o (id, TX \o \anc) \o Y -_{\pi_0} (id, TY \o \anc) \o X) \\
        &= p \o (\pi_1 \o \sigma \o (id, TX \o \anc) \o Y -_{T\pi} TY \o \anc \o X) \\
        &= p \o \pi_1 \o \sigma \o (id, TX \o \anc) \o Y -_{\pi} p\o TY \o \anc \o X \\
        &= \pi_0 \o (id, TX \o \anc) \o Y - Y \o p \o \anc \o X \\
        &= Y - Y = \xi, \\
         & \pi_0 \o ( \sigma \o (id, TX \o \anc) \o Y -_{\pi_0} (id, TY \o \anc) \o X) \\
        &= \pi_0 \o \sigma \o (id, TX \o \anc) \o Y -_{\pi} \pi_0 \o (id, TY \o \anc) \o X \\
        &= p \o \pi_1 \o (id, TX \o \anc) \o Y -_{\pi} X \\
        &= p \o TX \o \anc \o Y -_{\pi} X = X -_{\pi} X = \xi. 
    \end{align*}
    The universality of the lift induces a new section $[X,Y]$ so that
    \[
        \lambda \o [X,Y]
        = (\sigma \o (id, TX \o \anc) \o Y -_{T\pi} (id, TY \o \anc) \o X) - 0Y.
    \]    
\end{proof}

\begin{definition}\label{def:linear-intertwining}
    The map $\sigma:\prol(A) \to \prol(A)$ is \emph{linear} whenever the two bundle morphisms
    \[
        \sigma: (\prol(A),\lambda \x \ell) \to (\prol(A),0 \x c \o T\lambda)
        \hspace{0.25cm}
        \sigma: (\prol(A),0 \x c \o T\lambda) \to (\prol(A),\lambda \x \ell)
    \]
    are linear, and \emph{cosymmetric} if \[\sigma \o \widehat{\lambda} = \widehat{\lambda} = (\xi\pi,\lambda).\]
    Note that whenever $\sigma$ is linear and cosymmetric, $\sigma \o \hat\mu (u,v) = \hat\nu (u,v)$.
\end{definition}
Linearity and unit axioms, along with the connection on the differential bundle, force the existence of a bilinear bracket $\<-,-\>$ as in the definition of a Lie algebroid in Proposition \ref{prop:lie-alg-bil-defn}.
\begin{proposition}\label{prop:intertwining-linear-iff-bracket-linear}
    For an anchored bundle $(\pi:A \to M, \xi, \lambda, \anc)$ with connection $(\kappa, \nabla)$, a cosymmetric and bilinear $\sigma$ is equivalent to a bilinear $\<-,-\>:A_2 \to A$, with the correspondence given by
    \[
        \infer{\sigma: \prol(A)\xrightarrow[\overline{(\pi_1,\pi_0, \pi_2 + \< \pi_0,\pi_1 \>)}]{} \prol(A)}
        {\< -,-\>: \kappa \o \sigma \o \nabla -_\pi \kappa \o \nabla}
    \]
\end{proposition}
\begin{proof}
    Derive
    \begin{align*}
    \widehat{\sigma} \o (x,y,z)
        &= (y,x, \kappa \o \sigma \o (\nabla(\anc x,y) +_{p\o\pi_1} \mu(y, z)))\\
        &= (y,x, \kappa \o \sigma \o \nabla(\anc x, y) +_\pi \kappa \o \sigma \o \mu (y,z))\\
        &= (y,x, \kappa \o \sigma \o \nabla(\anc x, y) +_\pi \kappa \o \nu \o (y,z))\\
        &= (y,x, \kappa \o \sigma \o \nabla(\anc x, y) +_\pi z)\\
        &=: (y,x, \langle x, y\rangle +_\pi z )
    \end{align*}
    where $\<-,-\>$ is certainly bilinear. The converse is immediate: take
    \[
        \widehat{\sigma}(u,v,w) := (v,u,w + \< u,v\>).
    \]
    It is easy to see that $\sigma$ is linear and cosymmetric.    
\end{proof}
The linear bracket must be involutive for the bilinear bracket to be alternating.
\begin{proposition}\label{prop:intertwining-involutive}
    If $\sigma$ is cosymmetric and linear, then the bilinear bracket $\< -,-\>$ is alternating if and only if $\sigma\o \sigma = id$.
\end{proposition}
\begin{proof}
    First, note that $\sigma \o \sigma = id$ if and only if $\widehat{\sigma} \o \widehat{\sigma} = id$.
    Then check that
    \[
        \widehat{\sigma} \o \widehat{\sigma} (u,v,w) = \widehat{\sigma}(v,u,w+\<u,v\>) = (u,v,w + \< u,v \> + \< v, u\>).
    \]
    By the cancellativity of addition on $A$, 
    \[
    w = w+ \< u,v \> + \< v,u \> \iff 0 = \< u,v \> + \< v,u \>.
    \]
\end{proof}
\begin{observation}
    A bilinear $\<-,-\>$ on an anchored bundle with a connection induces the maps
    \[
        \{ v, x\}_{(\kappa, \nabla)} := \kappa' \o T.\anc \o \nabla(v, x), \quad
        \{v,x,y\}_{(\kappa, \nabla)} := \kappa \o T(\langle -,-\rangle_{(\kappa, \nabla)})\o \nabla^{A_2} \o (v,x,y)
    \]
    from \Cref{eq:lie-algd-structure-maps} in Proposition \ref{prop:lie-alg-bil-defn}.
\end{observation}
\begin{proposition}\label{prop:cosymmetric-leibniz}
    Let $\sigma$ be cosymmetric and bilinear, with the associated bracket $\<-,-\>$. Then
    \[
        T\anc \o \pi_1 \o \sigma = c \o T\anc \o \pi_1
    \] 
    if and only if the Leibniz equation is satisfied:
    \begin{equation}
        \label{eq:liebniz-local}
        \anc \o \langle u,v \rangle + \{ \anc v, u\} = \{ \anc u, v.\}
    \end{equation}
\end{proposition}
\begin{proof}
    Following the given notation and using the hypothesis that the connection is torsion-free on $M$, 
    \[
        \widehat{T\anc \o \pi_1}(u,v,w) = (\anc u, \anc v, w + \{u,v\})
        \hspace{0.5cm}
        \widehat{c}(x,y,z) = (y,x,z). 
    \]
    Computing each side,
    \begin{align*}
        \widehat{T\anc} \o \widehat{\pi_1} \o \widehat{\sigma} \o (u,v,w)
        &= \widehat{T\anc} \o \widehat{\pi_1} \o (v,u,w + \langle u,v\rangle) \\
        &= \widehat{T\anc} \o (\anc v,u,w + \langle u,v\rangle) \\
        &= (\anc v,\anc u, \anc w + \anc \langle u,v\rangle + D[\anc](\anc v, u)) \\
        &= (\anc v,\anc u, \anc w + \anc \langle u,v\rangle + \{ v, u\} ),\\
        \widehat{c} \o \widehat{T\anc} \o \widehat{\pi_1}(u,v,w)
        &= \widehat{c} (\anc u, \anc v, \anc w + \{u, v\}) \\
        &= (\anc v, \anc u, \anc w + \{ u, v\}),
    \end{align*}
    so it follows that the two terms are equal if and only if the desired equality holds.
\end{proof}

\begin{lemma}\label{lem:tang-of-sigma}
    Let $\sigma$ be linear and cosymmetric.
    % \[
    %     \{-,-,-\}: TM \ts{p}{\pi \o\pi_i} A_2 \to A; \{a, u_1, u_2\} := \kappa \o T(\<-,-\>) \o (\nabla(a,u_1), \nabla(a,u_2))
    % \]
    Then
    \begin{enumerate}[(i)]
        \item $T(\<-,-\>):TA_2 \to TA$ satisfies 
        \begin{align*}
            &\quad \widehat{T(\<-,-\>)}(a_x, u_y, u_z,u_{xy}, u_{xz})\\
            &= (a_x, \< u_y, u_z \>, \{a_x, u_y, y_z\} + \<u_y, u_{xz}\> + \langle u_{xy}, u_{z}\rangle); 
        \end{align*}
        \item $T.\sigma$ satisfies
        \begin{align*}
                &\quad \widehat{(id \x T(\widehat{\sigma}))}(u_{x}, u_{y}, u_{xy}, u_{z}, u_{xz}, u_{yz}, u_{xyz})\\
                &= (u_x, u_z, u_{xz}, u_y, u_{xy},  u_{yz} + \< u_y, u_z\>, \\&\quad u_{xyz} + \{a_x, u_y, y_z\} + \<u_y, u_{xz}\> 
                + \langle u_{xy}, u_{z}\rangle).
        \end{align*}
    \end{enumerate}
\end{lemma}
\begin{proof}~
    \begin{enumerate}[(i)]
        \item By the universality of the vertical lift and bilinearity of $\<-,-\>$, the outer squares below are pullbacks:
            \input{TikzDrawings/Ch3/Sec6/t-of-pair.tikz}
            so that 
               \[T(\langle,\rangle)(0,u_y, u_z, u_{xy},u_{xz}) 
               = \mu^A(\< u_y,u_z \>, \<u_y, u_{xz}\> + \langle u_{xy}, u_{z}\rangle).\]
               Now we compute
               \begin{align*}
                   &\quad T(\langle,\rangle)(\mu^{A_2}((u_y,y_z), (u_{xy}, u_{xz})) +_p \nabla^{A_2}(a_x, (u_y,u_z))) \\
                   &= T(\langle,\rangle)\mu^{A_2}((u_y, u_z),(u_{xy}, u_{xz})) +_p T(\langle,\rangle)\o\nabla(a_x, (u_y,u_z))\\
                   &= \mu^A(\< u_y,u_z \>, \<u_y, u_{xz}\> + \langle u_{xy}, u_{z}\rangle) 
                   +_p T(\langle,\rangle)\o\nabla(a_x, (u_y,u_z)) 
%                   T(\langle,\rangle)( \mu^{A_2}((u_x, u_y), (u_{xz}, u_{yz})) +_p \nabla^{A_2}(u_x, u_y, a_z))
%                   &= T(\langle,\rangle)\o \mu^{A_2}((u_x, u_y), (u_{xz}, u_{yz})) +_p T(\langle,\rangle)\o\nabla^{A_2}(u_x, u_y, a_z)\\
%                   &= \mu^{A}(\langle u_x, u_y\rangle, \langle u_{xz}, u_{yz}\rangle) +_p T(\langle,\rangle)\o\nabla^{A_2}(u_x, u_y, a_z)\\
               \end{align*}                   
               and then postcompose this with $(T\pi, p, \kappa)$ to obtain
               \begin{align*}
                   &\quad (0, \< u_y,u_z \>, \ \<u_y, u_{xz}\> + \langle u_{xy}, u_{z}\rangle) 
                   +_p (a_x, \< u_y, u_z \>, \{a_x, u_y, y_z\})\\
                   &=  
                   (a_x, \< u_y, u_z \>, \{a_x, u_y, y_z\} +  \<u_y, u_{xz}\> + \< u_{xy}, u_{z}\>). 
               \end{align*}
        \item Consider the following diagram:
               \input{TikzDrawings/Ch3/Sec6/rewriting-t-sigma.tikz}
            We want to find $\widehat{T\widehat{\sigma}} = \widehat{T(\pi_1, \pi_0, \pi_2 + \< \pi_0, \pi_1 \> )}$.
            Note that
            \[
                T(\pi_1, \pi_0, \pi_2 +_\pi \< \pi_0, \pi_1 \>)
                = T(\pi_1, \pi_0, \pi_2) +_{T\pi} T(\xi\pi\pi_i, \pi_0, \< \pi_0, \pi_1\>).
            \]
            The left term is straightforward:
            \begin{align*}
                &\widehat{T(\pi_0, \pi_1, \pi_2)}(a_x, (u_y, u_{xy}), (u_z, u_{xz}), (u_{yz}, u_{xyz}))\\
                &=(a_x,  (u_z, u_{xz}), (u_y, u_{xy}),(u_{yz}, u_{xyz}))
            \end{align*}
            and for the right term, use part (i) of this lemma:
            \begin{align*}
                &\quad \widehat{T\<-,-\>}\o \widehat{T(\pi_0, \pi_1))}(a_x, (u_y, u_{xy}), (u_z, u_{xz}), (u_{yz}, u_{xyz})) \\
                &= 
                \widehat{T\<-,-\>}(a_x, u_y, u_{xy}, u_z, u_{xyz}) \\
                &= (a_x, \< u_y, u_z \>, 
                \{a_x, u_y, y_z\} + \<u_y, u_{xz}\> + \< u_{xy}, u_{z}\>) 
            \end{align*}

            Then compute
            \begin{align*}
                &\quad \widehat{T\widehat{\sigma}}(a_x, u_y, u_{xy}, u_z, u_{xz}, u_{yz}, u_{xyz}) \\
                &= (a_x, u_z, u_{xz}, u_y, u_{xy}, u_{yz} + \< u_y, u_z\>, u_{xyz} + q)
            \end{align*}
            where $q = \{a_x, u_y, y_z\} + \<u_y, u_{xz}\> + \langle u_{xy}, u_{z}\rangle$, 
            giving the desired equation.
    \end{enumerate}
\end{proof}

\begin{proposition}\label{prop:yang-baxter-iff-bianchi}
    % Let $(\pi:A \to M, \xi, \lambda, \anc, \sigma)$ be an almost involution algebroid in a tangent category $\C$ with negatives, with anchored connection $(\nabla,\kappa)$ on $A$ and torsion-free affine connection $(\nabla',\kappa')$ on $M$.
    Let $\sigma$ be cosymmetric, doubly linear, and involutive, and satisfy the target axiom.
    Then $\sigma$ satisfies the Yang--Baxter equation if and only if $\<-,-\>$ and $\{-,-,-\}$ satisfy the Bianchi identity: 
    \begin{equation}
        \label{eq:bianchi-identity}
        0
        =\sum_{\gamma \in \mathsf{Cy}(3)} \langle x_{\gamma_0}, \langle x_{\gamma_1}, x_{\gamma_2}\rangle\rangle 
        + \sum_{\gamma \in \mathsf{Cy}(3)} \{ \anc x_{\gamma_0}, x_{\gamma_1}, x_{\gamma_2}.\}
    \end{equation}
\end{proposition}
\begin{proof}
    We expand $\widehat{id \x T\sigma}$ and  $\widehat{\sigma \x c}$.
    Start with $T(id \x \sigma)$, which was derived in Lemma \ref{lem:tang-of-sigma}:
    \begin{align*}
        \sigma_1(u) &= \widehat{(id \x T(\widehat{\sigma}))}(u_{x}, u_{y}, u_{xy}, u_{z}, u_{xz}, u_{yz}, u_{xyz})\\
        &= (u_x, u_z, u_{xz}, u_y, u_{xy}
        ,  u_{yz} + \< u_y, u_z\>, \\
        &\quad u_{xyz} + \langle u_y, u_{xz} \rangle  + \langle u_{xy}, u_z \rangle  + \{  \anc \o u_x, u_y, u_z \}).
    \end{align*}
    Using the fact that $\kappa'$ is torsion free, so 
    $\hat{c}(u_{xz}, u_{yz}, u_{xzy}) = (u_{yz}, u_{xz}, u_{xyz})$, it follows that
    \[
       \sigma_2(u_x, u_y, u_{xy}, u_z, u_{xz}, u_{yz}, u_{xyz})
        = (u_y, u_x, u_{xy}+ \< u_x, u_y \>, u_z, u_{yz}, u_{xz}, u_{xyz}).
    \]
    Then compute
    \begin{align*}
         &\quad \sigma_2\sigma_1\sigma_2(u) \\
        &= \big( u_z
                , u_y
                , u_{yz} + \langle u_y, u_z \rangle 
                , u_x
                , u_{xz} + \langle u_x, u_z \rangle 
                , u_{xy} + \langle u_x, u_y \rangle , z_1 \big),
%               (u_{xyz} + \langle u_y, u_{xy} \rangle  + \langle u_z, u_{xz} \rangle  + \{  \anc \o u_x, u_y, u_z \} + \langle u_x, u_{xz} + \langle u_x, u_z \rangle  \rangle  
%               \\& + \langle u_y, u_{yz} + \langle u_y, u_z \rangle  \rangle  + \{  \anc \o u_z, u_x, u_y \}) \
                \\
        &\quad  \sigma_1\sigma_2\sigma_1(u) \\
        &= \big( u_z
              , u_y
              , u_{yz} + \langle u_y, u_z \rangle 
              , u_x
              , u_{xz} + \langle u_x, u_z \rangle 
              , u_{xy} + \langle u_x, u_y \rangle , z_2\big).
%             u_{xyz} + \langle u_y, u_{xz} \rangle  + \langle u_{xy}, u_z \rangle  + \{  \anc \o u_x, u_y, u_z \} + \langle u_x, u_{yz}+ \langle u_y, u_z \rangle  \rangle \\
%             &  + \langle u_{xz} + \langle u_x, u_z \rangle , u_y \rangle  + \{  \anc \o u_z, u_x, u_y \}      
    \end{align*}
   Note that the first five terms are equal, so it suffices to check $z_1 = z_2$ for $z_1 = \pi_6\sigma_1\sigma_2\sigma_1(u), z_2 = \pi_6\sigma_2\sigma_1\sigma_2(u)$.
    \begin{align*}
    z_1 &= u_{xyz} + \langle u_y, u_{xz} \rangle  + \langle u_{xy}, u_z \rangle  
    +    \{  \anc \o u_x, u_y, u_z \} + \{  \anc \o u_z, u_x, u_y \}  \\
    &\quad +\langle u_x, u_{yz} + \langle u_y, u_z \rangle  \rangle  
    + \langle u_{xz} + \langle u_x, u_z \rangle , u_y \rangle \\ 
    &= u_{xyz} + \langle u_y, u_{xz} \rangle  + \langle u_{xy}, u_z \rangle  + \{  \anc \o u_x, u_y, u_z \} + \{  \anc \o u_z, u_x, u_y \}  \\
    &\quad +\langle u_x, u_{yz} \rangle + \langle u_x, \langle u_y, u_z \rangle  \rangle
    + \langle u_{xz}, u_y \rangle + \langle \langle u_x, u_z \rangle , u_y \rangle \\ 
    &= u_{xyz} + \langle u_{xy}, u_z \rangle  + \{  \anc \o u_x, u_y, u_z \} + \{  \anc \o u_z, u_x, u_y \}  \\
    &\quad +\langle u_x, u_{yz} \rangle + \langle u_x, \langle u_y, u_z \rangle  \rangle
    + \langle \langle u_x, u_z \rangle , u_y \rangle, \\ 
    z_2 &= u_{xyz} 
    + \langle u_x, u_{yz} \rangle + \{  \anc \o u_y, u_x, u_z \} + \langle u_{xy} + \langle u_x, u_y \rangle , u_z \rangle \\
    &= u_{xyz} + \langle u_x, u_{yz} \rangle + \{  \anc \o u_y, u_x, u_z \}    + \langle u_{xy} , u_z \rangle + \langle \langle u_x, u_y \rangle , u_z \rangle .
    \end{align*}    
    So $z_1 = z_2$ is equivalent to requiring
    \begin{align*}
     &\quad  u_{xyz} + \langle u_{xy}, u_z \rangle + \{  \anc \o u_x, u_y, u_z \} + \{  \anc \o u_z, u_x, u_y \}  \\
     &\quad +\langle u_x, u_{yz} \rangle + \langle u_x, \langle u_y, u_z \rangle  \rangle
    + \langle \langle u_x, u_z \rangle , u_y \rangle \\ 
    &= u_{xyz} + \langle u_x, u_{yz} \rangle + \{  \anc \o u_y, u_x, u_z \}     + \langle u_{xy} , u_z \rangle + \langle \langle u_x, u_y \rangle , u_z \rangle;
    \end{align*}
    cancelling alike terms, this is equivalent to
%   \[
%   \langle u_x, \langle u_y, u_z \rangle  \rangle + \langle \langle u_x, u_z \rangle , u_y \rangle + \{  \anc \o u_x, u_y, u_z \} 
%   + \{  \anc \o u_z, u_x, u_y \}
%   \]
    \begin{align*}
     &\quad \langle u_x, \langle u_y, u_z \rangle  \rangle + \langle \langle u_x, u_z \rangle , u_y \rangle + \{  \anc \o u_x, u_y, u_z \} 
    + \{  \anc \o u_z, u_x, u_y \} \\
    &=  \{  \anc \o u_y, u_x, u_z \}   +  \langle \langle u_x, u_y \rangle , u_z \rangle.
    \end{align*}
    Using the fact that $\< -, -\>$ and $\{-,-,-\}$ are alternating in the last two arguments, this is equivalent to
    \begin{align*}
        0&=\langle u_x, \langle u_y, u_z \rangle  \rangle + \langle \langle u_x, u_z \rangle , u_y \rangle +  \langle u_z , \langle u_x, u_y \rangle \rangle \\
         &\quad + \{  \anc \o u_x, u_y, u_z \} + \{  \anc \o u_z, u_x, u_y \} + \{  \anc \o u_y, u_z, u_x \} \\
         &=\langle u_x, \langle u_y, u_z \rangle  \rangle + \langle u_y, \langle u_z, u_x \rangle\rangle +  \langle u_z , \langle u_x, u_y \rangle \rangle \\
         &\quad + \{  \anc \o u_x, u_y, u_z \} + \{  \anc \o u_z, u_x, u_y \} + \{  \anc \o u_y, u_z, u_x \} 
    \end{align*}
%    \[
%       0 =
%       \langle u_x, \langle u_y, u_z \rangle  \rangle + \langle \langle u_x, u_z \rangle , u_y \rangle +  \langle \langle u_x, u_y \rangle , u_z \rangle
%       + \{  \anc \o u_x, u_y, u_z \} + \{  \anc \o u_z, u_x, u_y \} + \{  \anc \o u_y, u_z, u_x \}
%    \]
    giving the desired identity.
\end{proof}

\begin{corollary}\label{cor:inv-con-def}
    Let $(\pi:A \to M, \xi, \lambda, \anc)$ be an anchored bundle in a tangent category with negatives, with anchored connection $(\nabla,\kappa)$ on $A$ and torsion-free affine connection $(\nabla',\kappa')$ on $M$.
    An involution algebroid structure on $A$ is equivalent to a bilinear map
    \[
        \<-,-\>:A_2 \to A 
    \]
    with derived maps 
    \begin{gather*}
        \{-,-\}: A \ts{\pi}{p} TM \to TM := \kappa' \o T\anc \o (\pi_0,\pi_1), \\
        \{-,-,-\}: TM \ts{p}{\pi \o\pi_i} A_2 \to A; \{a, u_1, u_2\} := \kappa \o T(\<-,-\>) \o (\nabla(a,u_1), \nabla(a,u_2))
    \end{gather*}
    satisfying 
    \begin{enumerate}[(i)]
        \item $\< -, -\>$ is linear and cosymmetric,
        \item $\< -, -\>$ is alternating,
        \item $\< -, - \>$ and $\{-,-\}$ satisfy the Leibniz equation, Equation \ref{eq:liebniz-local}.
        \item $\< -, -\>, \{-,-\},$ and $\{-,-,-\}$ satisfy the Bianchi identity, \Cref{eq:bianchi-identity}.
    \end{enumerate}
\end{corollary}
Morphisms of involution algebroids may also be characterized by preservation of the tensor.
\begin{proposition}%
    \label{prop:inv-algd-mor-conn}
    Let $A, B$ be a pair of involution algebroids with chosen connections in a tangent category with negatives.
    Then an anchored bundle morphism $f: A \to B$ is an involution algebroid morphism if and only if (recalling the notation from Equation \ref{eq:nabla-notation})
    \[
        \nabla[f](x,y) + \< f\o x, f\o y\> = \nabla[f](y,x) + f \o \< x, y\>.
    \]
\end{proposition}
\begin{proof}
    Note that
    \begin{gather*}
        (\sigma \o \prol(f) = \prol(f) \o \sigma) \\
        \iff (\pi_1,\pi_0, \pi_2 + \< \pi_0, \pi_1 \> ) \o (f\o x, f\o y, f \o z + \nabla[f](x,y))
        \\
        = (f, f, f \o \pi_2 + \nabla[f](\pi_0,\pi_1)) \o (y,x, z + \< x, y \> ) 
    \end{gather*}
    while the second condition reduces to
    \[
        \nabla[f](x,y) + \< f\o x, f\o y\> = \nabla[f](y,x) + f \o \< x, y\>.
    \]
\end{proof}


\section{The isomorphism of Lie and involution algebroid categories}
\label{sec:the-isomorphism-of-categories}

Sections \ref{sec:Lie_algebroids} and \ref{sec:connections_on_an_involution_algebroid} have made the relationship between involution algebroids and Lie algebroids clear. It is important to note that while the proofs used connections as a tool to identify the local coherences satisfied by involution and Lie algebroids, the construction of a Lie algebroid from an involution algebroid (or vice versa) is independent of the choice of connection.

\begin{theorem}%
    \label{thm:iso-of-cats-Lie}
    There is an isomorphism of categories between Lie algebroids and involution algebroids in smooth manifolds.
\end{theorem}
\begin{proof}
    For the equivalence of categories, note that by Propositions \ref{prop:lie-alg-bil-defn} and \ref{prop:lie-algd-morphism-defn}, Corollary \ref{cor:inv-con-def}, and Proposition \ref{prop:inv-algd-mor-conn} there is an isomorphism of categories between involution algebroids with a choice of connection and Lie algebroids with a choice of connection (morphisms are \emph{not} restricted to connection preserving morphisms). This allows us to chain together isomorphisms
    \[
        \mathsf{Inv}(\mathsf{SMan}) 
        \cong \mathsf{Inv}(\mathsf{SMan})_{\mathsf{ChosenConn}}
        \cong \mathsf{LieAlgd}_{\mathsf{ChosenConn}}
        \cong \mathsf{LieAlgd}.
    \]
    \pagenote{One thing that was unclear in the original proof of this statement is that while connections are helpful for the nuts and bolts of this proof, the actual constructions do not rely on connections.}

    
    To complete the proof, we must show that the assignment that sends an involution algebroid to a Lie algebroid whose bracket is given by
    \begin{equation}\label{eq:inv-to-lie}
                \lambda \o [X,Y]^* = 
        \left( (\pi_1 \o \sigma \o (id, T.X \o \anc) \o Y -_p T.Y \o \anc \o) -_{T.\pi} 0 \o Y
        \right),
    \end{equation}
    is a bijection on objects, which brings up some subtleties. First, while an involution map 
    \[
        \sigma:\prolong \to \prolong
    \]
    is defined with respect to a particular choice of pullback $\prolong$, the category of involution algebroids \emph{does not} distinguish between different choices of this pullback (and therefore different representations of the map $\sigma$), and it is not part of the data of an involution algebroid. It is immediate by universality that the left-hand-side of Equation \ref{eq:inv-to-lie} is independent of the choice of pullback $\prolong$.
    
    Now, recall that the canonical involution of a Lie algebroid is uniquely by Theorem 4.7 of \cite{de2005lagrangian} (this was also mentioned in Corollary \ref{cor:second-lie-algd-unique-map}). Once we make a choice of prolongation $\prolong$, we have made a choice of pullback $\prol(A)$ in $\mathsf{LieAlgd}$, which uniquely determines the canonical involution
    \[
    \sigma: \prolong \to \prolong.
    \]
    While the exact map $\sigma$ depends on the choice of pullback $\prolong$, they all determine the same involution algebroid, thus proving the bijection correspondence of objects.
\end{proof}






% \documentclass[main.tex]{subfiles}

% \begin{document}

% Here talk about how the prolongation acts like a tangent functor. 
% 

\chapter{The Weil nerve of an algebroid}%
\label{chap:weil-nerve}\pagenote{
   This chapter has been given a new introduction to help with its exposition.
   Various typos have been fixed, but the most substantial changes are in Section \ref{sec:weil-nerve}, where the proof has been restructured to address some concerns brought up in Michael's comments.
}

The first three chapters of this thesis demonstrated that tangent categories allow for an essentially algebraic description of Lie algebroids by axiomatizing the behaviour of the tangent bundle, and showing that a Lie algebroid over a manifold $M$ is a ``generalized tangent bundle'', namely an \emph{involution algebroid}. This chapter will make precise the sense in which an involution algebroid is a generalized tangent bundle, by showing that the category of involution algebroids in a tangent category $\C$ is equivalent to a certain tangent-functor category from the free tangent category over a single object to $\C$, or more generally that there is a fully faithful functor
\[
	\mathsf{Inv}(\C) \hookrightarrow \mathsf{Tang_{Lax}}(\mathsf{FreeTangCat}(*),\C).
\]
This functor, the \emph{Weil nerve} of an involution algebroid, builds a functor from the free tangent category over a single object to $\C$ using a span construction. This chapter primarily builds on two pieces of work: Leung's construction of the free tangent category $\wone$ (\cite{Leung2017}) , and Grothendieck's original nerve construction (first published in \cite{Segal1974}).

To understand Leung's construction of the free tangent category, and more generally his actegory-theoretic presentation of tangent categories (Section \ref{sec:tang-struct-as-wone}), we first look at Weil's original insight relating the kinematic and operational descriptions of the tangent bundle in $\mathsf{SMan}$. The definition of a tangent vector on a manifold $M$ as an equivalence class of curves (Definition \ref{def:tang-vector})  puts a bijective correspondence between tangent vectors and $\R$-algebra homomorphisms from the ring of smooth functions $C^\infty(M)$ to the ring of dual numbers:
\[
	C^\infty(M) \to \R[x]/x^2.
\] The hom-set $\R\mathsf{Alg}(C^\infty(M),\R[x]/x^2)$ is precisely the set of \emph{derivations} on $C^\infty(M)$, which defines the operational tangent bundle discussed in Definition \ref{def:operational-tang}: there is a natural smooth manifold structure on this set. The Weil functor formalism, most notably developed in \cite{Kolar1993}, extends this observation to a general class of endofunctors on $\mathsf{SMan}$. For example, the fibre product  $T_2M$ corresponds to $\R$-algebra morphisms,
\[
C^\infty(M) \to R[x,y]/(x^2, y^2, xy), 
\] while the second tangent bundle corresponds to $\R$-algebra morphisms into the tensor product,
\[
C^\infty(M) \to R[x]/(x^2) \ox R[y]/(y^2) = R[x,y]/(x^2,y^2).
\] By applying Milnor's exercise (Problem 1-C \cite{Milnor1974}), which states that the $C^\infty$ functor
\[
    \mathsf{SMan} \to \R\mathsf{Alg^{op}};\hspace{0.15cm} M \mapsto C^\infty(M)=\mathsf{SMan}(M,\R)
\] is fully faithful, the structure maps occur as natural transformations. For example, the tangent projection is induced by the morphism
\[
	{p}: \R[x]/x^2 \xrightarrow[]{a + bx \mapsto a} \R,  
\] 
so that 
\[
    TM \xrightarrow[]{p} M = [C^\infty(M),\R[x]/x^2] \xrightarrow[]{(p)_*} [C^\infty(M), \R].
\]
The zero map and addition are similarly induced by
\[
	{0}: \R \xrightarrow[]{a \mapsto a + 0x} R[x]/x^2 \text{ and } +:R[x,y]/(x^2,y^2,xy) \xrightarrow[\mapsto a + (b+c)x]{a + bx + cy} R[x]/x^2,
\] 
respectively, while the lift and flip are induced by the morphisms
\[
	{\ell}: \R[x]/x^2 \xrightarrow[\mapsto a + bxy]{a+bx} \R[x,y]/(x^2,y^2) 
\]
and
\[
    {c}: \R[x,y]/(x^2,y^2) \xrightarrow[\mapsto a + cx + by + dxy]{a + bx + cy + dxy} \R[x,y]/(x^2,y^2).
\] 
More generally, there is a monoidal category of \emph{Weil algebras} (Definition \ref{def:weil-algebra-and-prol}) which has a monoidal action on the category of smooth manifolds. The Weil functor formalism, then, studies differential geometric structures from the perspective of the endofunctors and natural transformations induced by this action. Leung's insight is that there is an analogous category of commutative rigs\footnote{A rig is a ri\emph{n}g without \emph{n}egatives, i.e. a commutative monoid equipped with a bilinear multiplication.} built by replacing $\R[x]/x^2$ with $\N[x]/x^2$, called $\wone$ (Definition \ref{def:Weil-algebra}); a tangent structure is precisely a monoidal action by $\wone$ satisfying some universal properties. In particular, this category $\wone$ is precisely the free tangent category over a point, $\mathsf{FreeTang}(*)$, so that every object $A$ in a tangent category $\C$ determines a strict tangent functor
\[
    T^{(-)}A: \wone \to \C; V \mapsto T^VA
\]
and morphisms $f:A \to B$ are in bijective correspondence with tangent-natural transformations $T^{(-)}A \Rightarrow T^{(-)}(B)$.

The axioms of an involution algebroid in a tangent category $\C$ correspond bijectively with those of the tangent bundle - this suggests that an involution algebroid should determine a tangent functor from $\wone$ to $\C$. A first guess would lead one to think that $p:\N[x]/x^2 \to \N$ is sent to $\pi:A \to M$, $0$ to $\xi$, and $+$ to $+_A$. As the space of prolongations $\prol(A) = \prolong$ plays the role of the second tangent bundle, we can see that
\[
  \ell:N[x]/x^2 \to N[x,y]/(x^2,y^2) \mapsto (\xi\o\pi,\lambda): A \to \prol(A), 
\]
and 
\[
  c:N[x,y]/(x^2,y^2) \to N[x,y]/(x^2,y^2) \mapsto \sigma \prol(A) \to \prol(A).
\]
This pattern may be neatly summed up using span composition - we will construct a functor that sends $\N[x]/x^2$ to the span
% https://q.uiver.app/?q=WzAsMyxbMCwxLCJNIl0sWzEsMCwiQSJdLFsyLDEsIlRNIl0sWzEsMiwiXFxhbmMiLDFdLFsxLDAsIlxccGkiLDFdXQ==
\[\begin{tikzcd}
	& A \\
	M && TM
	\arrow["\anc"{description}, from=1-2, to=2-3]
	\arrow["\pi"{description}, from=1-2, to=2-1]
\end{tikzcd}\]
and the tensor product $\N[x]/x^2 \ox \N[x]/x^2$ to the span composition (e.g. the pullback)
\[\begin{tikzcd}
	&& {\prol(A)} \\
	& A && TA \\
	M && TM && {T^2M}
	\arrow["\anc"{description}, from=2-2, to=3-3]
	\arrow["\pi"{description}, from=2-2, to=3-1]
	\arrow["{T.\pi}"{description}, from=2-4, to=3-3]
	\arrow["{T.\anc}"{description}, from=2-4, to=3-5]
	\arrow[from=1-3, to=2-2]
	\arrow[from=1-3, to=2-4]
	\arrow["\lrcorner"{anchor=center, pos=0.125, rotate=-45}, draw=none, from=1-3, to=3-3]
\end{tikzcd}\]
which is the space of prolongations. This leads to the first major result of this chapter, the \emph{Weil Nerve} (Theorem \ref{thm:weil-nerve}), which states there is a fully faithful embedding
\[
    N_{\mathsf{Weil}}:\mathsf{Inv}(\C) \hookrightarrow [\wone, \C].
\]
This bears a strong similarity to Grothendieck's original nerve theorem, which takes an internal category $s,t:C \to M$ and constructs a functor $\Delta^{op} \to \C$ (where $\Delta^{op}$ is the monoidal theory of an internal monoid) by sending tensor to span composition, and the composition and unit maps given span morphisms
% https://q.uiver.app/?q=WzAsOCxbMCwxLCJNIl0sWzEsMCwiQ18yIl0sWzIsMSwiTSJdLFsxLDIsIkMiXSxbMywxLCJNIl0sWzQsMiwiQyJdLFs1LDEsIk0iXSxbNCwwLCJNIl0sWzEsMCwicyBcXG8gXFxwaV8wIiwxXSxbMSwyLCJ0IFxcbyBcXHBpXzEiLDFdLFszLDAsInMiLDFdLFszLDIsInQiLDFdLFsxLDMsIm0iLDFdLFs3LDQsIiIsMSx7ImxldmVsIjoyLCJzdHlsZSI6eyJoZWFkIjp7Im5hbWUiOiJub25lIn19fV0sWzcsNiwiIiwxLHsibGV2ZWwiOjIsInN0eWxlIjp7ImhlYWQiOnsibmFtZSI6Im5vbmUifX19XSxbNSw0LCJzIiwxXSxbNSw2LCJ0IiwxXSxbNyw1LCJlIiwxXV0=
\[\begin{tikzcd}
	& {C_2} &&& M \\
	M && M & M && M \\
	& C &&& C
	\arrow["{s \o \pi_0}"{description}, from=1-2, to=2-1]
	\arrow["{t \o \pi_1}"{description}, from=1-2, to=2-3]
	\arrow["s"{description}, from=3-2, to=2-1]
	\arrow["t"{description}, from=3-2, to=2-3]
	\arrow["m"{description}, from=1-2, to=3-2]
	\arrow[Rightarrow, no head, from=1-5, to=2-4]
	\arrow[Rightarrow, no head, from=1-5, to=2-6]
	\arrow["s"{description}, from=3-5, to=2-4]
	\arrow["t"{description}, from=3-5, to=2-6]
	\arrow["e"{description}, from=1-5, to=3-5]
\end{tikzcd}\]
while the unit and associativity axioms for a category are exactly the unit and associativity laws for a monoid in this setting. The \emph{Segal conditions} identify exactly the functors $C:\Delta^{op} \to \C$ that lie in the image of the nerve functor $N$ as those whose $[n]^{th}$ object is sent to the wide pullback $C([n]) = C[2] \ts{t}{s} C[2]\dots \ts{t}{s} C[2]$. The corresponding result for involution algebroids is found in Theorem \ref{thm:iso-of-cats-inv-emcs}, which states that a tangent functor $(A,\alpha):\wone \to \C$ is the nerve of an involution algebroid if and only if $A$ preserves tangent limits and $\alpha$ is a $T$-cartesian natural transformation (this forces $A.(W^{\ox n}) = A \ts{\anc}{T.\pi} TA \dots \ts{T^{n-1}\anc}{T^n\pi} T^nA$). The similarity between the Weil nerve and Grothendieck/Segal's nerve runs deep, and in Chapter \ref{ch:inf-nerve-and-realization} we demonstrate that the \emph{enriched} perspective on tangent categories puts these both into the same formal framework.
% There is a natural analogy between involution algebroids and the tangent bundle, as there is a bijective correspondence between the structure maps and the axioms they satisfy, and so it is a natural candidate for the ``classifying tangent category'' of an involution algebroid. However, the construction of a non-monoidal tangent functor $(A,\alpha):\wone \to \C$ from an involution algebroid in $\C$ poses a technical challenge. Luckily, inspiration may be drawn from a cornerstone result in category theory, Grothendieck's \emph{nerve theorem}, which shows that the internal categories in $\C$ embed into the category of simplicial objects in $\C$ $[\Delta^{op}, \C]$. The simplex category $\Delta$, a full subcategory of $\mathsf{Cat}$ whose objects are preorders $[n] = 1 < \dots < n$ (preorder morphisms are exactly functors), and its opposite category is the monoidal theory of a monoid; that is to say, any monoid in a monoidal category is given by a monoidal functor
% \[
%     (\Delta^{op}, +, [0]) \to (\C, \ox, I).
% \]
% Given an internal category $C$ in a finitely complete category $\C$, there is a base span given by the underlying graph:
% % https://q.uiver.app/?q=WzAsMyxbMCwxLCJNIl0sWzEsMCwiQyJdLFsyLDEsIk0iXSxbMSwwLCJzIiwxXSxbMSwyLCJ0IiwxXV0=
% \[\begin{tikzcd}
% 	& C \\
% 	M && M
% 	\arrow["s"{description}, from=1-2, to=2-1]
% 	\arrow["t"{description}, from=1-2, to=2-3]
% \end{tikzcd}\]
% The tensor product $[1]+[1]$ is sent to the \emph{span composition} (e.g. the pullback)
% % https://q.uiver.app/?q=WzAsNixbMCwyLCJNIl0sWzEsMSwiQyJdLFsyLDIsIk0iXSxbMywxLCJDIl0sWzQsMiwiTSJdLFsyLDAsIkNfMiJdLFsxLDAsInMiLDFdLFsxLDIsInQiLDFdLFszLDIsInMiLDFdLFszLDQsInQiLDFdLFs1LDFdLFs1LDNdLFs1LDAsIiIsMSx7ImN1cnZlIjoxfV0sWzUsNCwiIiwxLHsiY3VydmUiOi0xfV0sWzUsMiwiIiwxLHsic3R5bGUiOnsibmFtZSI6ImNvcm5lciJ9fV1d
% \[\begin{tikzcd}
% 	&& {C_2} \\
% 	& C && C \\
% 	M && M && M
% 	\arrow["s"{description}, from=2-2, to=3-1]
% 	\arrow["t"{description}, from=2-2, to=3-3]
% 	\arrow["s"{description}, from=2-4, to=3-3]
% 	\arrow["t"{description}, from=2-4, to=3-5]
% 	\arrow[from=1-3, to=2-2]
% 	\arrow[from=1-3, to=2-4]
% 	\arrow[curve={height=10pt}, from=1-3, to=3-1]
% 	\arrow[curve={height=-10pt}, from=1-3, to=3-5]
% 	\arrow["\lrcorner"{anchor=center, pos=0.125, rotate=-45}, draw=none, from=1-3, to=3-3]
% \end{tikzcd}\]
% where the monoid composition is sent to the internal category's composition map, and the identity map is sent by the unit map for the category:

% For the Weil nerve construction of an anchored bundle construction, we will generalize the prolongations of the underlying anchored bundle. An anchored bundle $(\pi:A \to M, \xi, \lambda, \anc)$ determines a span, where the object $\mathbb{N}[x]/x^2$ is sent to

% The tensor product of commutative rigs $\mathbb{N}[x]/x^2 \ox \mathbb{N}[x]/x^2$ will be sent to the following span composition:
% % https://q.uiver.app/?q=WzAsNixbMCwyLCJNIl0sWzEsMSwiQSJdLFsyLDIsIlRNIl0sWzMsMSwiVEEiXSxbNCwyLCJUXjJNIl0sWzIsMCwiXFxwcm9sKEEpIl0sWzEsMiwiXFxhbmMiLDFdLFsxLDAsIlxccGkiLDFdLFszLDIsIlQuXFxwaSIsMV0sWzMsNCwiVC5cXGFuYyIsMV0sWzUsMV0sWzUsM10sWzUsMiwiIiwxLHsic3R5bGUiOnsibmFtZSI6ImNvcm5lciJ9fV1d
% \[\begin{tikzcd}
% 	&& {\prol(A)} \\
% 	& A && TA \\
% 	M && TM && {T^2M}
% 	\arrow["\anc"{description}, from=2-2, to=3-3]
% 	\arrow["\pi"{description}, from=2-2, to=3-1]
% 	\arrow["{T.\pi}"{description}, from=2-4, to=3-3]
% 	\arrow["{T.\anc}"{description}, from=2-4, to=3-5]
% 	\arrow[from=1-3, to=2-2]
% 	\arrow[from=1-3, to=2-4]
% 	\arrow["\lrcorner"{anchor=center, pos=0.125, rotate=-45}, draw=none, from=1-3, to=3-3]
% \end{tikzcd}\]
% The morphisms in $\wone$ are then constructed using the structure maps for the involution algebroid.

Sections \ref{sec:weil-algebras-tangent-structure} and \ref{sec:tang-struct-as-wone} give a detailed introduction to the Weil functor formalism (\cite{Kolar1993},\cite{Bertram2014a}) and Leung's unification of Weil functors with tangent categories \cite{Leung2017}. The rest of the chapter contains contains new results developed by the author. Section \ref{sec:weil-nerve} proves the embedding part of the Weil nerve theorem, that the category of involution algebroids embeds into the category of tangent functors and tangent natural transformations $[\wone, \C]$. Section \ref{sec:identifying-involution-algebroids} identifies exactly those tangent functors $(A,\alpha):\wone \to \C$ that are the nerve of an involution algebroid, completing the proof of the Weil nerve theorem. Section \ref{sec:prol_tang_struct} uses the Weil nerve to develop a novel tangent structure on the category of involution algebroids in a tangent category (in particular, the category of Lie algebroids will have this novel tangent structure).


\section{Weil algebras and tangent structure}%
\label{sec:weil-algebras-tangent-structure}

This section gives a more thorough introduction to the Weil functor formalism of \cite{Kolar1993}, and in particular how the structure maps of a tangent category may be teased out of it. We begin by introducing  Weil algebras, and the \emph{prolongation} of a smooth manifold by a Weil algebra. (The relationship with prolongations of involution algebroids from Definition \ref{def:anchored_bundles} will be made clear in Section \ref{sec:weil-nerve}.)

\begin{definition}%
    \label{def:weil-algebra-and-prol}
    An $\R$-Weil algebra\footnote{Not to be confused with the normal usage of ``Weil algebra'' in Lie theory, e.g. \cite{Meinrenken2019}.} is a finite-dimensional $\R$-algebra $V$ so that $V = \R \oplus \dot{V}$ as $\R$-modules and $\dot{V}$ is a nilpotent ideal. The category $\R\mathsf{Weil}$ is the full subcategory of $\R\mathsf{Alg}$ spanned by the $\R$-Weil algebras. The \emph{prolongation} of a manifold by a Weil algebra $V$ is given by the manifold
    \[
        T^VM := \R\mathsf{Alg}(C^\infty(M,\R), V).
    \]
\end{definition}
\cite{Eck1986} showed that every product-preserving endofunctor on the category of smooth manifolds is constructed as the Weil prolongation by some Weil algebra. Consequently, $\R$-algebra homomorphisms induce natural transformations between these product-preserving  endofunctors on the category of smooth manifolds.
\begin{example}\label{ex:weil-algebras-and-maps}
    Consider the following $\R$-Weil algebras and their associated prolongation functors.
    \begin{enumerate}[(i)]
        \item Prolongation by $\R$ induces the identity functor, and the tangent bundle is given by $\R[x]/x^2$. The tangent projection, then, is equivalent to the $\R$-algebra morphism
        \[
            p: \R[x]/x^2 \to \R; \hspace{0.2cm}
            p(a + bx) = a
        \]
        while the 0-map induces the zero vector field:
        \[
            0: \R \to \R[x]/x^2; \hspace{0.2cm}
            0(a) = a + 0x.
        \]
        \item The algebra $\R[x_i]_{1 \le i \le n}/(x_ix_j)_{1 \le i \le j \le n} = (\R[x]/x^2)^n$ is the wide pullback $T_nM = TM \ts{p}{p} TM \dots \ts{p}{p} TM$. 
        In particular, prolongation by $\R[x,y]/(x^2,y^2,xy)$ gives the bundle $T_2M = TM \ts{p}{p} TM$. The $\R$-algebra morphism
        \[
            +:\R[x,y]/(x^2,y^2,xy) \to R[x]/x^2; \hspace{0.2cm}
            +(a_0 + a_1x + a_2y) = a_0 + (a_1 + a_2)x
        \]corresponds to the addition of tangent vectors.
        \item The algebra $\R[x,y]/(x^2,y^2) = (\R[x]/x^2)\ox (\R[x]/x^2)$ is the second tangent bundle $T^2M = TTM$. The vertical lift $T \to T^2$ is induced by the morphism 
        \[
            \ell: \R[x]/x^2 \to \R[x,y]/(x^2,y^2);\hspace{0.2cm}
            \ell(a + bx) = a + bxy.
        \]
        \item The monoidal symmetry map induces $c: T^2 \Rightarrow T^2$, as follows: 
        \begin{gather*}
                        c: (\R[x]/x^2)\ox (\R[y]/y^2) \to (\R[y]/y^2)\ox (\R[x]/x^2);\\ c(a + b_1x + b_2y + b_3xy) = a + b_2x + b_1y + b_3xy.        
        \end{gather*}
        %by exchanging the $x,y$ variables.
        \item For $n \ge 2$, the algebra $\R[x]/x^n$ gives the $n$-jet bundle.
        Note that this is the equalizer of $\ox^n \R[x]/x^2$ by the symmetry actions of $S_n$.
    \end{enumerate}
\end{example}
Further examples may be found in the monograph \cite{Kolar1993}. Tangent categories bridge the gap between the Weil functor approach to studying the differential geometry of smooth manifolds and the synthetic differential geometry approach of axiomatizing a tangent bundle using nilpotent infinitesimals. The main structure axiomatized here is that of \emph{monoidal action} of a symmetric monoidal category on a category $M \x \C \to \C$, or equivalently, a lift from a category to the category of complexes $\C \to [M,\C]$, which involves translating a bit of classical category theory to the 2-categorical setting.

Example  \ref{ex:weil-algebras-and-maps} leads to the classical theorem that the category of smooth manifolds has an action by the category of $\R$-Weil algebras that preserves all connected limits that exist. These ``natural'' universal properties (in the sense of \cite{Kolar1993}) is foundational to synthetic differential geometry; see, for example, Chapter Two of \cite{Lavendhomme1996}. Unfortunately, Weil algebras are not an ideal syntactic presentation: they are not a finitely presentable category, and it is not immediately clear when a diagram is a connected limit.\footnote{It should be noted that \cite{Nishimura2007} made progress applying techniques from computer algebra to latter problem.} Moving from $\R$-algebras to commutative rigs and restricting to an appropriate subcategory solves this problem.\pagenote{Clarified a point raised by Michael}

\begin{definition}[Definition 3.1 \cite{Leung2017}]
    \label{def:Weil-algebra}
    The category $\wone$ is defined to be the full subcategory of commutative rigs, $\mathsf{CRig}$, generated by the rig of dual numbers $W := \N[x]/x^2$, constructed as follows:
    \begin{enumerate}
        \item Start with finite product powers of $W$ in $\mathsf{CRig}$, and make a strict choice of presentation:
        \[
            W_0 = \N, \hspace{0.15cm} W_n := \N[x_i]/(x_ix_j)_{i \le j}, {0 \le i < n}.
        \]
        \item Then take the closure of $W_n$ under coproduct of commutative rigs, written $\ox$. Again, make a strict choice of presentation:
        \[
            W_{n(0)} \ox \dots \ox W_{n(m-1)} :=
            \N[x_{i,j}]/(x_{ij}x_{ik})_{j \le k}, 0 < i < m, 0 < j < n(i).
        \]
    \end{enumerate}
    Note that we will often suppress the tensor product $\ox$ and simply write\pagenote{explained notation used throughout this chapter}
    \[
        UV := U\ox V.
    \]
\end{definition}
\begin{proposition}[Definition 3.3 \cite{Leung2017} ]
    ~\begin{enumerate}[(i)]
        \item The category $\wone$ is a symmetric strict monoidal category with unit $\N$ and coproduct $\ox$.
        \item $\N$ is a terminal object in $\wone$.
    \end{enumerate}
\end{proposition} 
Note that there is a forgetful functor
\[
    \wone \to (\mathsf{CMon}/\N) \to \mathsf{CMon}    
\]
that reflects connected limits. This gives the following class of limits, identified in \cite{Leung2017}.
\begin{definition}%
    \label{def:transverse-limit}
    We say the following pullback diagrams in $\wone$ are \emph{transverse}:\pagenote{switched superscript to subscript in second diagram}
    \input{TikzDrawings/Ch4/Sec2/transverse-limits.tikz}
    where $\mu(a + a_1x + a_2y) = a + a_1x + a_2xy$. The $\ox$-closure of these three pullbacks is the set of \emph{transverse squares}, and they are also pullback squares by \cite{Leung2017}.
\end{definition}
To see that each transverse square in the $\ox$-closure is a pullback diagram, take the two non-identity squares and rewrite them in $\mathsf{CMon}$:
\[\input{TikzDrawings/Ch4/cmon-pb.tikz}\]
The coproduct of Weil algebras is the tensor product of the underlying commutative monoids, which are finite-dimensional and free, so these limits are closed under $\ox$.
\begin{proposition}[\cite{Leung2017} Proposition 4.1]
    The category $\wone$ is a tangent category, where the tangent functor is
    \[
        T := W \ox (\_): \wone \to \wone  
    \]
    and the natural transformations are given by 
    \begin{gather*}
        p: W \ox (\_) \xrightarrow[]{p \ox (\_)} (\_), \hspace{0.25cm}
        0: (\_) \xrightarrow[]{0 \ox (\_)} W \ox (\_), \hspace{0.25cm}
        +: W_2 \ox (\_) \xrightarrow[]{+ \ox (\_)} W \ox (\_), \\
        \ell: W \ox (\_) \xrightarrow[]{\ell \ox (\_)} W \ox W \ox (\_), \hspace{0.25cm}
        c: W \ox W \ox (\_) \xrightarrow[]{p \ox (\_)} W \ox W \ox (\_)).
    \end{gather*}
\end{proposition}

The category $\wone$ is, in some sense, a finitely presented theory. It is precisely the free tangent category on a single object:
\begin{proposition}[Proposition 9.5, \cite{Leung2017}]
    \label{thm:leung}
    The category $\wone$ is generated by the maps $\{p, 0, +, \ell, c\}$ from Example  \ref{ex:weil-algebras-and-maps}, closed under composition, tensor, and maps induced by transverse limits.
\end{proposition}
\begin{corollary}
    The category $\wone$ is the \emph{free} tangent category over a single object: every object $C$ in a tangent category $\C$ determines a strict tangent functor $T_{-}.C: \wone \to \C$, mapping
    \[
        V = W^{n{1}} \ox \dots \ox W^{n(k)}  
        \mapsto 
        T_{n(1)}.(\dots).T_{n(k)}.C = T^VC
    \]
    so that there is an isomorphism of categories between $\C$ and the category of strict tangent functors $\wone \to \C$ with tangent natural transformations as morphisms.\pagenote{Original statement was incomplete.}
\end{corollary}
\begin{notation}
    Throughout this section, the notation $T^VC$ will refer to the action of the Weil algebra $V$ on an object $C$ in a tangent category.
    In particular, we will make use of the isomorphism $T^U.T^VC = T^{UV}C$.
\end{notation}
% In some sense, then, a tangent structure is somehow a functorial choice of a ``tangent complex'' $\C \mapsto [\wone,\C]$.


\section{Tangent structures as monoidal actions}%
\label{sec:tang-struct-as-wone}

The presentation of $\wone$ as the free tangent category situates the formal theory of tangent categories as an instance of more general categorical machinery, namely monoidal actions. Recall that in a symmetric monoidal category $(\C, \ox, I)$, an internal monoid $(C, \bullet, i)$ determines a monad:
\[
    (C \ox \_: \C \to \C, \mu: \C \ox (\C \ox \_) \xrightarrow{\bullet \ox \_} \C \ox \_, 
    \eta: \_ \xrightarrow{\rho} I \ox \_ \xrightarrow{e \ox \_} C \ox \_).
\]
The category of algebras for this monad is exactly the category of $C$-modules, objects with an associative and unital action by $C$. A morphism will be a map on the base object that preserves the action:
\input{TikzDrawings/Ch4/Sec2/monoidal-action.tikz}
Strict actegories are the 2-categorical generalization of modules over a monoid. The coherences for a 2-monad follow from the coherences from a strict monoidal category in the 2-category of categories. The following proposition relies on a few facts from enriched category theory (treating the cartesian closed category $\mathsf{Cat}$ as a $\mathsf{Cat}$-enriched category, per \cite{Kelly2005}) but a more general treatment of non-strict actegories may be found in \cite{janelidze2001note}:\pagenote{Since I'm not referencing any sort of weakness, this is just enriched category theory so I think Kelly is a fair reference. However, I have included the reference to Janelidze and Kelly's work on actegories.} 
\begin{itemize}
    \item A 2-functor and 2-natural transformations are exactly a functor and natural transformations that satisfy extra coherences. These coherences follow for free by constructing the monad and comonad $\m \x \_, [\m, \_]$.
    \item An algebra of the underlying 1-monad is exactly an algebra of the 2-monad (the same result holds for comonads).
\end{itemize}
When working with algebras of a 2-monad, four different notions of morphisms can come into play (\cite{Lack2009}). These arise through using the 2-categorical data to weaken the notion of a morphism:
\begin{enumerate}[(i)]
    \item Strict: this is exactly a morphism of the underlying algebras. Write the 2-category of strict $\m$-actegories.
    \item Strong: the morphisms preserve the action to an isomorphism:
    \[\input{TikzDrawings/Ch4/Sec2/act-mor-strong.tikz}\]
    \item Lax: the 2-cell is no longer an isomorphism:
    \[\input{TikzDrawings/Ch4/Sec2/act-mor-lax.tikz}\]
    \item Oplax: the 2-cell travels in the opposite direction (these will not figure into this account).
\end{enumerate}
2-cells between actegory morphisms must satisfy a coherence between the natural transformation parts of the actegory morphisms.
\begin{definition}\label{def:actegory-natural}
    In the case of strict, strong, and lax tangent functors, the same notion of a 2-cell applies: a natural transformation $\gamma:F \Rightarrow G$ satisfying the following coherences with the natural transformations $\alpha$ and $\beta$:
% https://q.uiver.app/?q=WzAsOCxbMCwwLCJcXG1cXHhcXEMiXSxbMSwwLCJcXG1cXHhcXEQiXSxbMSwxLCJcXEQiXSxbMCwxLCJcXEMiXSxbMiwwLCJcXG1cXHhcXEMiXSxbMywwLCJcXG1cXHhcXEQiXSxbMiwxLCJcXEMiXSxbMywxLCJcXEQiXSxbMCwxLCJcXG1cXHggRiJdLFsxLDIsIlxccHJvcHRvXlxcRCJdLFswLDMsIlxccHJvcHRvXlxcQyIsMl0sWzMsMiwiRiIsMV0sWzEsMywiXFxhbHBoYSIsMSx7ImxldmVsIjoyfV0sWzMsMiwiRyIsMix7ImN1cnZlIjozfV0sWzQsNSwiXFxtXFx4IEciLDIseyJsYWJlbF9wb3NpdGlvbiI6MjB9XSxbNCw1LCJcXG1cXHggRiIsMCx7ImN1cnZlIjotM31dLFs0LDYsIlxccHJvcHRvXlxcQyIsMl0sWzUsNywiXFxwcm9wdG9eXFxEIl0sWzYsNywiRyIsMl0sWzUsNiwiXFxiZXRhIiwwLHsibGV2ZWwiOjJ9XSxbMTEsMTMsIlxcZ2FtbWEiLDAseyJzaG9ydGVuIjp7InNvdXJjZSI6MjAsInRhcmdldCI6MjB9fV0sWzE1LDE0LCJcXG1cXHggXFxnYW1tYSIsMix7Im9mZnNldCI6LTIsInNob3J0ZW4iOnsic291cmNlIjoyMCwidGFyZ2V0IjoyMH19XSxbOSwxNiwiPSIsMSx7InNob3J0ZW4iOnsic291cmNlIjoyMCwidGFyZ2V0IjoyMH0sInN0eWxlIjp7ImJvZHkiOnsibmFtZSI6Im5vbmUifSwiaGVhZCI6eyJuYW1lIjoibm9uZSJ9fX1dXQ==
\[\begin{tikzcd}
	\m\x\C & \m\x\D & \m\x\C & \m\x\D \\
	\C & \D & \C & \D
	\arrow["{\m\x F}", from=1-1, to=1-2]
	\arrow[""{name=0, anchor=center, inner sep=0}, "{\propto^\D}", from=1-2, to=2-2]
	\arrow["{\propto^\C}"', from=1-1, to=2-1]
	\arrow[""{name=1, anchor=center, inner sep=0}, "F"{description}, from=2-1, to=2-2]
	\arrow["\alpha"{description}, Rightarrow, from=1-2, to=2-1]
	\arrow[""{name=2, anchor=center, inner sep=0}, "G"', curve={height=18pt}, from=2-1, to=2-2]
	\arrow[""{name=3, anchor=center, inner sep=0}, "{\m\x G}"'{pos=0.2}, from=1-3, to=1-4]
	\arrow[""{name=4, anchor=center, inner sep=0}, "{\m\x F}", curve={height=-18pt}, from=1-3, to=1-4]
	\arrow[""{name=5, anchor=center, inner sep=0}, "{\propto^\C}"', from=1-3, to=2-3]
	\arrow["{\propto^\D}", from=1-4, to=2-4]
	\arrow["G"', from=2-3, to=2-4]
	\arrow["\beta", Rightarrow, from=1-4, to=2-3]
	\arrow["\gamma", shorten <=2pt, shorten >=2pt, Rightarrow, from=1, to=2]
	\arrow["{\m\x \gamma}"', shift left=2, shorten <=2pt, shorten >=2pt, Rightarrow, from=4, to=3]
	\arrow["{=}"{description}, Rightarrow, draw=none, from=0, to=5]
\end{tikzcd}\]
We call these \emph{actegory natural transformations}. 
\end{definition}
Note that for strict actegory morphisms, this condition holds for any natural transformation $\gamma:F \Rightarrow G$. Now consider the following three 2-categories.\pagenote{Added the actual definition of the 2-categories referenced later on. I also added the coherence for actegory-natural transformation}
\begin{definition}\label{def:2-categories}
    Let $(\m,\ox,I)$ be a strict monoidal category. Define the following three 2-categories.
    \begin{enumerate}
        \item $\m\mathsf{Act}_{\mathsf{strict}}$: the 2-category of strict $\m$-actegories, strict actegory morphisms, and natural transformations.
        \item $\m\mathsf{Act}_{\mathsf{strong}}$: the 2-category of strict $\m$-actegories, strong actegory morphisms, and actegory natural transformations.
        \item $\m\mathsf{Act}_{\mathsf{lax}}$: the 2-category of strict $\m$-actegories, lax actegory morphisms, and actegory natural transformations.
    \end{enumerate}
    Note that the inclusions of these 2-categories are \emph{locally} fully faithful, so only the 1-cells differ.
\end{definition}

The case where the action preserves certain limits in the monoidal category is of particular interest. A small category equipped with a class of chosen limits is known as a \emph{sketch}. The previous correspondence restricts to the class of limit-preserving actions in this case.
\begin{definition}
    A \emph{sketch} is a small category with a class of chosen limits, and a sketch morphism is a functor sending chosen limits to the chosen limits in the domain (up to isomorphism). The category of models of a sketch $\c$ in a category $\C$, $\mathsf{Mod}(\c, \C)$, is the full subcategory $[\c, \C]$ whose functors preserve the chosen limits.
    A \emph{monoidal sketch}, then, is a sketch $(\c, \prol)$ equipped with a symmetric monoidal category structure on $\c$ so that $\_ \ox \_$ preserves limits in each argument.
\end{definition}

Now use the fact that the category $\wone$ is a monoidal sketch, since it is a small, strict monoidal category equipped with a class of limits stable under the tensor product.
\begin{theorem}[Theorem 14.1, \cite{Leung2017}]
    Let $\C$ be a category. The following are equivalent:
    \begin{enumerate}[(i)]
        \item A tangent structure on $\C$,
        \item A sketch action $\propto: \wone \x \C \to \C$.
        % \item A sketch lift $\phi: \C \to \mathsf{Mod}(\wone, \C)$.
    \end{enumerate}
\end{theorem}

\begin{observation}%
    \label{obs:cofree-tangent-cat}
    There is a \emph{coalgebraic} perspective on tangent categories, coming from the equivalence between algebras of the 2-monad $(\wone \x (\_), \ox, I:1 \to \wone)$ and the 2-comonad $([\wone,\_], [\ox,\_], [I,\_])$.
    For any  category $\C$, there is a free tangent category given by
    \[
        \wone \x \C
    \] and this agrees with the free $\wone$-actegory. However, for the \emph{cofree} tangent category, take
    \[
        \mathsf{Mod}(\wone, \C),
    \]
    the category of transverse-limit-preserving functors $\wone \to \C$.
\end{observation}

We can use Leung's theorem to induce a monoidal functor $\wone \to \C$ when a tangent structure is induced by a single object.
\begin{corollary}\label{cor:using-leung-thm}
    Let $(\C,\ox,I)$ be a strict monoidal category.\pagenote{I've added this intermediary result to make the Weil nerve a bit more explicit.}
    If an additive bundle $(p:A \to I, +:A_2 \to A, 0:I \to A)$ equipped with morphisms
    \[
       A \ox A \xrightarrow[]{c} A \ox A \hspace{0.5cm} A \xrightarrow[]{\ell} A \ox A
    \]
    determines a tangent structure on $\C$ using the endofunctor $A \ox (-)$, then $A$ determines a strict, transverse-limit-preserving, monoidal functor 
    \[
       A(-):\wone \to \C; W_{n[1]}\ox \dots \ox W_{n[k]} \mapsto A_{n[1]}\ox \dots \ox A_{n[k]} = T_{n[1]}\dots T_{n[k]}.I
    \]
\end{corollary}
Note that this allows for a more conceptual description of representable tangent structure.
\begin{proposition}\label{prop:monoidal-functor-inf-obj}
    In a symmetric monoidal closed category, an infinitesimal object is exactly a strict symmetric monoidal functor $D:\wone \to \C$.\pagenote{The other statement was probably more general than necessary - this statement is more to the point.}
\end{proposition}
This presentation of an infinitesimal object makes it tautological that $\C^{op}$ has a tangent structure.
\begin{corollary}%
    \label{cor:dual-tangent-structure}
    Given a strict symmetric monoidal functor
    \[
        D:\wone \to \C^{op}
    \]
    there is a strict action of $\wone$ on $\C^{op}$ given by
    \[
        \wone \x \C^{op} \xrightarrow[]{D^{op} \ox \C} \C^{op} \x \C^{op} \xrightarrow[]{\otimes} \C^{op}. 
    \]
\end{corollary}


There is a clear correspondence between the notions of a (strict, strong, lax) tangent functor and a (strict, strong, lax) actegory morphism. This proposition extends to the following equivalence of 2-categories.
\begin{corollary}[Theorem 14.1 \cite{Leung2017}]
    The following pairs of 2-categories are equivalent.
    \begin{enumerate}[(i)]
        \item The 2-category of tangent categories and strict tangent functors is the full sub-2-category of $\wone\mathsf{Act_{Strict}}$ spanned by sketch actions.
        \item The 2-category of tangent categories and strong tangent functors is the full sub-2-category of $\wone\mathsf{Act_{strong}}$ spanned by sketch actions.
        \item The 2-category of tangent categories and lax tangent functors is the full sub-2-category of $\wone\mathsf{Act_{Lax}}$ spanned by sketch actions.
    \end{enumerate}
\end{corollary}



% Make sure you use 
\section{The Weil nerve of an involution algebroid}
\label{sec:weil-nerve}

The construction in this section is analagous to the nerve of an internal category---hence the ``Weil nerve'' construction---and deals with similar technical issues. In particular, the construction in this section will mimic the nerve construction for internal categories by replacing the tensor product of $\wone$ with span composition in the domain category. Recall that every anchored bundle or internal category has a canonical span associated with it:
\[\input{TikzDrawings/Ch4/Sec3/anc-bundle.tikz}\]
In any category $\C$, there is a category of spans in $\C$ as well as span composition.
\begin{definition}%
    \label{def:span-stuff}
    A \emph{span from $A$ to $B$} in a category $\C$ is a diagram of the form 
    \[\input{TikzDrawings/Ch4/Sec3/span.tikz}\]
    There is a notion of \emph{span composition}, so given a span $X:A \to B$ and $Y:B \to C$, then the composition of $X$ and $Y$ is the pullback (if it exists):
    \[\input{TikzDrawings/Ch4/Sec3/span-hor-comp.tikz}\]
    A morphism of spans is a commuting diagram of the form
    \[\input{TikzDrawings/Ch4/Sec3/span-morp.tikz}\]
    Note that if $f$ and $g$ are span morphisms with $f_r = g_l$, then the horizontal composition may be formed if each respective span composition exists:
    \[\input{TikzDrawings/Ch4/Sec3/span-hor-comp-morp.tikz}\]
    When discussing span composition in a tangent category, it is assumed that the pullback is a $T$-pullback.
\end{definition}
\begin{observation}%
    \label{obs:lim-of-spans}
    Note that the category of spans in $\C$ is a functor category, so that limits are computed pointwise in $\C$. This also means that the horizontal composition operation, when it exists, preserves limits in either argument.
\end{observation}
These span constructions can be helpful in constructing functors from a monoidal category into a non-monoidal category $\C$, by forming a monoidal category from $\C$ using spans. In the case of an internal category over $M$, one takes the slice category $\C/(M \x M)$ where the tensor product is span composition. An internal category $s,t:C \to M$ is a monoid in this category of spans over $M$, so that it determines a monoidal functor
\[
    C: \Delta^{op} \to \C/(M \x M),    
\] 
remembering that $\Delta^{op}$ is the monoidal theory for monoid (every monoid in a monoidal category $\C$ determines a monoidal functor $\Delta \to \C$). The construction of the corresponding monoidal category for spans is more nuanced, as the category $\wone$ is not $\N$-indexed. Observe that the prolongation of an anchored bundle is constructed as a span composition:
\input{TikzDrawings/Ch4/Sec3/prol-span-comp.tikz}
The third prolongation is given by span composition as well:
\[\input{TikzDrawings/Ch4/Sec3/prol-span-comp-2.tikz}\]
This horizontal composition will play the same role as the tensor product in $\C/(M\x M)$.
\begin{definition}%
    \label{def:boxtimes-span}
    In a tangent category $\C$, consider a pair of spans
    \[
        X:M \to T^UM, \hspace{0.15cm} Y:M \to T^VM.
    \]
    Define $X \boxtimes Y$ to be the horizontal composition (when it exists):
    \[\input{TikzDrawings/Ch4/boxtimes-span.tikz}\]
    (recall that we will often suppress the $\ox$ in $\wone$ to save space).\pagenote{clarifying notation}
    So the span composition is
    \[
        M \xrightarrow[]{X} T^UM \xrightarrow[]{T^U.Y} T^U.T^VM    
    \]
    \begin{equation}
        \label{eq:span-form}
        \input{TikzDrawings/Ch4/span-form.tikz}
    \end{equation}
    The horizontal composition $f \boxtimes g$ is defined as $f \x_{\theta.M} \theta.g$:
    \[\input{TikzDrawings/Ch4/boxtimes-diag.tikz}\]
\end{definition}
% \begin{lemma}
%     Consider an object $M$ in a tangent category $\C$, and define the span:
%     \[
%         p^U := M \xleftarrow[]{p^U} T^UM = T^UM  
%     \]
%     This object behaves like the unit
% \end{lemma}
% \begin{observation}
In any tangent category with a tangent display system (Definition\pagenote{there was no reason to pull this out as an observation} \ref{def:display-system}), the category of spans on $M$ whose maps are of the form given by Equation \ref{eq:span-form} with $l \in \d$ is a monoidal category.
% \end{observation}
Any anchored bundle in a tangent category gives rise to a monoidal category after a strict choice of $T$-pullbacks (assuming those $T$-pullbacks exist).
\begin{definition}\label{def:monoidal-category}
    Let $(\pi:A \to M,\xi,\lambda,\anc)$ be an anchored bundle in a tangent category $\C$. Write the span
    \[
        \widehat{A}.W_n := (M \xleftarrow[]{\pi \o \pi_i} A_n \xrightarrow[]{\anc^n} T_nM),
        \hspace{0.5cm}
        \widehat{A}.\N := (M = M = M).
    \]
    A \emph{choice of prolongations} for $(\pi,\xi,\lambda,\anc)$ is a strict choice of horizontal composition for each $V \in \wone$:
    \[
        \widehat{A}.V = \widehat{A}.(W_{n[1]}\dots W_{n[k]}) :=
        \widehat{A}.W_{n[1]} \boxtimes \dots \boxtimes \widehat{A}.W_{n[k]}.
    \]
    We will write the span as follows:
    % https://q.uiver.app/?q=WzAsMyxbMCwxLCJNIl0sWzEsMCwiQS5WIl0sWzIsMSwiVF5WLk0iXSxbMSwwLCJcXHBpXlYiLDJdLFsxLDIsIlxcYW5jXlYiXV0=
\[\begin{tikzcd}
	& {A.V} \\
	M && {T^V.M}
	\arrow["{\pi^V}"', from=1-2, to=2-1]
	\arrow["{\anc^V}", from=1-2, to=2-3]
\end{tikzcd}\]
    (notice that the apex is not hatted).
    Given a choice of prolongations for an anchored bundle $(\pi,\xi,\lambda,\anc)$, the category $\mathsf{Span}(\pi,\xi,\lambda,\anc)$ is defined as follows:
    \begin{itemize}
        \item Objects are $\widehat{A}.V$ for $V \in \wone$.
        \item Morphisms are given by pairs
        \[
            (f,\phi):\widehat{A}.V \to \widehat{A}.U
        \]
        where $f:A.V \to A.U$ and $\phi:V \to U$ determine a span morphism of the form
        % https://q.uiver.app/?q=WzAsNixbMCwwLCJNIl0sWzEsMCwiQS5WIl0sWzIsMCwiVF5WLk0iXSxbMiwxLCJUXlUuTSJdLFswLDEsIk0iXSxbMSwxLCJBLlYiXSxbMSwwLCJcXHBpXlYiLDJdLFsxLDIsIlxcYW5jXlYiXSxbMiwzLCJcXHBoaS5NIl0sWzAsNCwiIiwyLHsibGV2ZWwiOjIsInN0eWxlIjp7ImhlYWQiOnsibmFtZSI6Im5vbmUifX19XSxbNSw0LCJcXHBpXlYiXSxbNSwzLCJcXGFuY15VIiwyXSxbMSw1LCJmIl1d
\[\begin{tikzcd}
	M & {A.V} & {T^V.M} \\
	M & {A.U} & {T^U.M}
	\arrow["{\pi^V}"', from=1-2, to=1-1]
	\arrow["{\anc^V}", from=1-2, to=1-3]
	\arrow["{\phi.M}", from=1-3, to=2-3]
	\arrow[Rightarrow, no head, from=1-1, to=2-1]
	\arrow["{\pi^V}", from=2-2, to=2-1]
	\arrow["{\anc^U}"', from=2-2, to=2-3]
	\arrow["f", from=1-2, to=2-2]
\end{tikzcd}\]
        as discussed in Definition \ref{def:span-stuff}.
        \item Tensor structure: The tensor product is defined using the horizontal composition $\boxtimes$ as defined in Definition \ref{def:span-stuff}.
    \end{itemize}
\end{definition}
The idea is to show that an involution algebroid structure on an anchored bundle induces a tangent structure on the monoidal category of prolongations, and then to apply Leung's theorem. 
The following two lemmas will simplify this proof.
\begin{lemma}\label{lemma:pullback-part-of-theorem}
    Let $(\pi:A \to M, \xi, \lambda, \anc)$ be an anchored bundle with chosen prolongations in a tangent category $\C$, and identify the monoidal category $\mathsf{Span}(\pi,\xi,\lambda,\anc)$.
    \begin{enumerate}[(i)]
        \item There is a functor 
        \[
           U^\anc: \mathsf{Span}(\pi,\xi,\lambda,\anc) \to \C
        \]
        constructed by sending a span morphism to the morphism between the objects at its apex:
% https://q.uiver.app/?q=WzAsOCxbMCwwLCJNIl0sWzEsMCwiQS5WIl0sWzIsMCwiVF5WLk0iXSxbMiwxLCJUXlUuTSJdLFswLDEsIk0iXSxbMSwxLCJBLlUiXSxbMywwLCJBLlYiXSxbMywxLCJBLlUiXSxbMSwwLCJcXHBpXlYiLDJdLFsxLDIsIlxcYW5jXlYiXSxbMiwzLCJcXHBoaS5NIiwyXSxbMCw0LCIiLDIseyJsZXZlbCI6Miwic3R5bGUiOnsiaGVhZCI6eyJuYW1lIjoibm9uZSJ9fX1dLFs1LDQsIlxccGleViJdLFs1LDMsIlxcYW5jXlUiLDJdLFsxLDUsImYiXSxbNiw3LCJmIl0sWzEwLDE1LCIiLDIseyJzaG9ydGVuIjp7InNvdXJjZSI6NDAsInRhcmdldCI6NDB9LCJsZXZlbCI6MSwic3R5bGUiOnsidGFpbCI6eyJuYW1lIjoibWFwcyB0byJ9fX1dXQ==
\[\begin{tikzcd}
	M & {A.V} & {T^V.M} & {A.V} \\
	M & {A.U} & {T^U.M} & {A.U}
	\arrow["{\pi^V}"', from=1-2, to=1-1]
	\arrow["{\anc^V}", from=1-2, to=1-3]
	\arrow[""{name=0, anchor=center, inner sep=0}, "{\phi.M}"', from=1-3, to=2-3]
	\arrow[Rightarrow, no head, from=1-1, to=2-1]
	\arrow["{\pi^V}", from=2-2, to=2-1]
	\arrow["{\anc^U}"', from=2-2, to=2-3]
	\arrow["f", from=1-2, to=2-2]
	\arrow[""{name=1, anchor=center, inner sep=0}, "f", from=1-4, to=2-4]
	\arrow[shorten <=14pt, shorten >=14pt, maps to, from=0, to=1]
\end{tikzcd}\]
        \item Suppose we have a square
% https://q.uiver.app/?q=WzAsNCxbMCwxLCJcXHdpZGVoYXR7QX0uWCJdLFsxLDEsIlxcd2lkZWhhdHtBfS5aIl0sWzEsMCwiXFx3aWRlaGF0e0F9LlkiXSxbMCwwLCJcXHdpZGVoYXR7QX0uVSJdLFswLDEsIihmLFxccGhpKSIsMl0sWzIsMSwiKGcsXFxwc2kpIl0sWzMsMCwiKGwsXFxhbHBoYSkiLDJdLFszLDIsIihyLFxcYmV0YSkiXV0=
\[\begin{tikzcd}
	{\widehat{A}.U} & {\widehat{A}.Y} \\
	{\widehat{A}.X} & {\widehat{A}.Z}
	\arrow["{(f,\phi)}"', from=2-1, to=2-2]
	\arrow["{(g,\psi)}", from=1-2, to=2-2]
	\arrow["{(l,\alpha)}"', from=1-1, to=2-1]
	\arrow["{(r,\beta)}", from=1-1, to=1-2]
\end{tikzcd}\]
        whose image under $U^\anc$ is a $T$-pullback in $\C$, and so that the square in $\wone$ is a transverse $T$-pullback:
% https://q.uiver.app/?q=WzAsOCxbMCwxLCJ7QX0uWCJdLFsxLDEsIntBfS5aIl0sWzEsMCwie0F9LlkiXSxbMCwwLCJ7QX0uVSJdLFsyLDAsIlUiXSxbMywwLCJZIl0sWzIsMSwiWCJdLFszLDEsIloiXSxbMCwxLCJmIiwyXSxbMiwxLCJnIl0sWzMsMCwibCIsMl0sWzMsMiwiciJdLFs2LDcsIlxccGhpIiwyXSxbNSw3LCJcXHBzaSJdLFs0LDUsIlxcYmV0YSJdLFs0LDYsIlxcYWxwaGEiLDJdLFszLDEsIiIsMSx7InN0eWxlIjp7Im5hbWUiOiJjb3JuZXIifX1dLFs0LDcsIiIsMSx7InN0eWxlIjp7Im5hbWUiOiJjb3JuZXIifX1dXQ==
\[\begin{tikzcd}
	{{A}.U} & {{A}.Y} & U & Y \\
	{{A}.X} & {{A}.Z} & X & Z
	\arrow["f"', from=2-1, to=2-2]
	\arrow["g", from=1-2, to=2-2]
	\arrow["l"', from=1-1, to=2-1]
	\arrow["r", from=1-1, to=1-2]
	\arrow["\phi"', from=2-3, to=2-4]
	\arrow["\psi", from=1-4, to=2-4]
	\arrow["\beta", from=1-3, to=1-4]
	\arrow["\alpha"', from=1-3, to=2-3]
	\arrow["\lrcorner"{anchor=center, pos=0.125}, draw=none, from=1-1, to=2-2]
	\arrow["\lrcorner"{anchor=center, pos=0.125}, draw=none, from=1-3, to=2-4]
\end{tikzcd}\]
      Then $U^\anc$ reflects the limit; that is, the original square in $\mathsf{Span}(\pi,\xi,\lambda,\anc)$ is a $T$-pullback.
      \item $T$-pullbacks of the form described in (ii) are closed under $\boxtimes$.
    \end{enumerate}
\end{lemma}
\begin{proof}
    The functor in in (i) is straightforward to construct, as it simply forgets the left and right legs of the spans. For (ii), note that because the $\wone$ part of the diagram is a transverse $T$-pullback, then given a pair of maps
    % https://q.uiver.app/?q=WzAsNSxbMSwyLCJcXHdpZGVoYXR7QX0uWCJdLFsyLDIsIlxcd2lkZWhhdHtBfS5aIl0sWzIsMSwiXFx3aWRlaGF0e0F9LlkiXSxbMSwxLCJcXHdpZGVoYXR7QX0uVSJdLFswLDAsIlxcd2lkZWhhdHtBfS5WIl0sWzAsMSwiKGYsXFxwaGkpIiwyXSxbMiwxLCIoZyxcXHBzaSkiXSxbMywwLCIobCxcXGFscGhhKSIsMl0sWzMsMiwiKHIsXFxiZXRhKSJdLFs0LDAsIih4LFxcb21lZ2EpIiwyLHsiY3VydmUiOjJ9XSxbNCwyLCIoeSxcXGdhbW1hKSIsMCx7ImN1cnZlIjotMn1dXQ==
\[\begin{tikzcd}
	{\widehat{A}.V} \\
	& {\widehat{A}.U} & {\widehat{A}.Y} \\
	& {\widehat{A}.X} & {\widehat{A}.Z}
	\arrow["{(f,\phi)}"', from=3-2, to=3-3]
	\arrow["{(g,\psi)}", from=2-3, to=3-3]
	\arrow["{(l,\alpha)}"', from=2-2, to=3-2]
	\arrow["{(r,\beta)}", from=2-2, to=2-3]
	\arrow["{(x,\omega)}"', curve={height=12pt}, from=1-1, to=3-2]
	\arrow["{(y,\gamma)}", curve={height=-12pt}, from=1-1, to=2-3]
\end{tikzcd}\]
    a unique span morphism $\widehat{A}.V \to \widehat{A}.U$ may be induced using the apex map from $\C$, and the unique map induced in $\wone$ by the universality of transverse squares (this square is also universal in $\C$), so the span morphism diagram will commute by universality.
    
    For (iii), $T$-pullback squares of the form in (ii) are closed under $\boxtimes$ as transverse squares in $\wone$ are closed under $\ox$, so the result follows by the commutativity of limits and by applying part (ii) of this lemma. 
\end{proof}
% \begin{definition}%
% \label{def:prolongation-of-anchored-bundle}
%     Let $(\pi:A \to M, \xi, \lambda, \anc)$ be an anchored bundle in a tangent category $\C$. 
%     This anchored bundle defines a mapping on objects:
%     \[
%         \hat A: \mathsf{Ob}(\wone) \to \mathsf{Ob}(\C); V \mapsto \hat A .V
%     \]  
%     defined inductively using a choice of span compositions from Definition \ref{def:boxtimes-span}.

%     For $W_n$, the object is $A.W_n$. There is a canonical span:
%     \[
%       A_n: M \to T_nM := M \xleftarrow[]{\pi \o \pi_i} A_n \xrightarrow[]{\anc_n} T_n.M    
%     \]
%     The \emph{prolongation of $A$ by $V \in \wone$} is the span composition:
%     \[
%       V = W_{n(1)} \ox \dots W_{n(k)} \mapsto
%       \hat A. W_{n(1)} \boxtimes \dots \boxtimes\hat A. W_{n(k)} =: \hat . V
%     \]
%     Write the span maps:
%     \[
%         \hat A.V : M \to T^VM :=
%         M \xleftarrow{\pi^V} \hat A.V  \xrightarrow{\anc^V} T^VM  
%     \]
%     An anchored bundle has \emph{chosen prolongations} when all $\hat A. V$ exist and for each $U,V$ the equation:
%     \[
%         \hat A. U \boxtimes \hat A. V = \hat A. UV   
%     \]
%     holds. The $V$-prolongation of $(\pi, \xi, \lambda, \anc)$ is the apex of the $V$-span. 
% \end{definition}

% Whenever an anchored bundle has chosen prolongations, it is possible to identify the following monoidal category.
%TODO fill in this definition
%TODO fill in this proposition
% The tensor product 
% The monoidal category to be construction for an involution algebroid, then, takes the span determined by its underlying anchored bundle, and uses span composition to construct a span for each Weil algebra $V \in \wone$. The construction is inductive, and is outlined below:
% A ``higher order'' prolongation may be constructed for each Weil algebra $U \in \wone$, following the metaphor that the space of prolongations $\prol(A)$ is like the second tangent bundle, and $\prol^2(A)$ the third tangent bundle.



\begin{observation}%
    \label{obs:concrete-desc-zigzag}
    It will be useful to have a ``flat'' presentation of the prolongation $A.W_{n(1)}\dots W_{n(k)}$. 
    % Inductively, look at the prolongation $\prol(W_mW_nV,A)$:
    % \[\input{TikzDrawings/Ch4/prolong-ind-case.tikz}\]
    % But apply the same construction for $T_m.\prol(W_nV,A)$ to get the double pullback:
    % \[\input{TikzDrawings/Ch4/prolong-ind-case-2.tikz}\]
    The higher prolongations of an anchored bundle may be concretely described as the $T$-pullback of the zig-zag below:
    \[\input{TikzDrawings/Ch4/Sec3/prol-zigzag.tikz}\]
    so that the prolongation $A.W^{n[1]}\dots W^{n[k]}$ may be written concretely as
    \[
      (u_1,\dots, u_{k}) : A_{n[1]} \ts{\anc'}{T.\pi'}  T_{n[1]}.A_{n[2]} \ts{T.\anc}{T^2.\pi} \dots \ts{\anc'}{T.\pi'} T_{n[1]\dots n[k-1]}A_{n[k]}.
    \]
    Furthermore, the choice of prolongation identifies the following limits:
    \[\input{TikzDrawings/Ch4/prol-abuse.tikz}\input{TikzDrawings/Ch4/prol-abuse-2.tikz}\]
    so that
    \[
        A. UV
        = A.U \boxtimes A.V
        = A.U \ \boxtimes id_M \boxtimes A.V
    \]
    where $id_M$ is the span $M = M = M$.
\end{observation}

% \begin{example}
%     ~\begin{enumerate}[(i)]
%         \item For the tangent bundle, $\widehat{TM}.V = T^VM$. 
%         \item For a trivial bundle, $(\pi:A \to M, \xi, \lambda, 0 \o \pi)$, then $A_V = A_{|V|}$ where $|V| \in \N$ denotes the dimension of the underlying free commutative monoid of $V$.
%     \end{enumerate}
% \end{example}

%
% \begin{definition}
%     \label{def:boxtimes}
%     Let $(\pi:A \to M, \xi, \lambda, \anc)$ be an anchored bundle with chosen prolongations. 
%     For any pair of span morphisms $f:U \to U', g:V \to V'$ of the form:
%     \begin{equation}
%         \label{eq:span-form}
%         \input{TikzDrawings/Ch4/span-form.tikz}
%     \end{equation}
%     The horizontal composition $f \boxtimes g$ is defined:
%     \[\input{TikzDrawings/Ch4/boxtimes-diag.tikz}\]
% \end{definition}
% \begin{lemma}
%     Let $(\pi:A \to M, \xi, \lambda, \anc)$ be an anchored bundle with chosen prolongations.
%     Consider a commuting cospan in $\mathsf{Span}(\C)$ of the form:

%     Now the 
% \end{lemma}
% Given an anchored bundle $(\pi:A \to M, \xi,\lambda, \anc)$, a choice of prolongations allows for the construction of a strict monoidal category with a bijective correspondence on objects with the category of Weil algebras.
                   
Note that a $\anc$ sends the involution algebroid structure map to its corresponding tangent structure map. Each of the structure maps, then, gives a span morphism where $\boxtimes$ is well defined:
\begin{definition}%
    \label{def:generators-for-wone-in-anc}
    Let $(\pi:A \to M, \xi, +_q, \lambda, \anc, \sigma)$ be an involution algebroid in $\C$ with chosen prolongations.
    Then define the following maps in $\mathsf{Span}(\pi,\xi,\lambda,\anc)$:
    \begin{itemize}
        \item The projection $p: \hat{A}.W \to \hat{A}.\N$,
        \[\input{TikzDrawings/Ch4/Sec4/proj.tikz}\]
        \item The zero map $0: \hat{A}.\N \to \hat{A}.W$,
        \[\input{TikzDrawings/Ch4/Sec4/zero.tikz}\]
        \item The addition map $+:  \hat{A}.W_2 \to \hat{A}.W$,
        \[\input{TikzDrawings/Ch4/Sec4/add.tikz}\]
        \item The lift map $\ell: \hat{A}.W \to \hat{A}.WW$,
        \[\input{TikzDrawings/Ch4/Sec4/lift.tikz}\]
        \item The flip map $c: \hat{A}.WW \to \hat{A}.WW$,
        \[\input{TikzDrawings/Ch4/Sec4/flip.tikz}\]
    \end{itemize}
\end{definition}
The idea is to show that the monoidal category of chosen prolongations $\mathsf{Span}(\pi,\xi,\lambda,\anc)$ for an involution algebroid has a tangent structure generated by the structure maps in Definition \ref{def:generators-for-wone-in-anc} and the endofunctor $\widehat{A}.W \boxtimes (-)$. Using the flat presentation, we can then show that $U^\anc$ will determine a tangent functor in $\C$. The following lemma about $\anc$ will be useful in constructing the natural transformation part of a tangent functor.
% The idea is to prove that the generators define a functor $\wone \to \C$. 
% First, following table associates maps in $\wone$ to the structure maps of an involution algebroid:
% \begin{center}
%     \begin{tabular}{|l|l|l|l|l|l|l|}
%     \hline
%     Tangent bundle & $T^VM$ & $p$   & $0$   & $+$     & $\ell$    & $c$      \\ \hline
%     Involution algebroid & $\hat{A}.V$     & $\pi$ & $\xi$ & $+_\pi$ & $\hat\lambda$ & $\sigma$ \\ \hline
%     \end{tabular}
% \end{center}
% Having a flat presentation of these maps will prove useful.
% \begin{proposition}\label{def:higher-morphisms}
%     Let $(\pi:A \to M, \xi, \lambda, \sigma)$ be an involution algebroid with a chosen prolongations. By the earlier observation, identify $\prolong$ with $A \ts{\anc}{id} TM \ts{id}{T\pi} TA$. For any pair of Weil algebras $U,V$, define the following morphisms (note that $\anc^V_\N$ is the original $\anc^V$ map)
%     \begin{enumerate}[(i)]
%         \item \input{TikzDrawings/Ch4/LocMaps/proj.tikz}
%         \item \input{TikzDrawings/Ch4/LocMaps/zero.tikz}
%         \item \input{TikzDrawings/Ch4/LocMaps/plus.tikz}
%         \item \input{TikzDrawings/Ch4/LocMaps/lift.tikz}
%         \item \input{TikzDrawings/Ch4/LocMaps/flip.tikz}
%         % \item \input{TikzDrawings/Ch4/LocMaps/anc.tikz}
%     \end{enumerate}
% \end{proposition}
\begin{definition}\label{def:anc-nat}
    Let $(\pi:A \to M, \xi, \lambda, \anc)$ be an anchored bundle with chosen prolongations. 
    Recall that by Definition \ref{def:monoidal-category}, the right leg of $A.U$ is written $\anc$, so it induces a span map:\[\input{TikzDrawings/Ch4/anc-uv.tikz}\]
    This map has a flat presentation as
    \[
        \input{TikzDrawings/Ch4/LocMaps/anc.tikz}        
    \]
    We write the map
    \[
        \anc^U.V := \anc^U \boxtimes (\widehat A.V)
    \]
    which corresponds to the following span morphism:
    % https://q.uiver.app/?q=WzAsOSxbMCwyLCJNIl0sWzEsMSwiQS5VIl0sWzIsMiwiVF5VTSJdLFszLDEsIlReVS5BLlYiXSxbNCwyLCJUXntVVn1NIl0sWzEsMywiVF5VTSJdLFszLDMsIlReVS5BLlYiXSxbMiwwLCJBLlVWIl0sWzIsNCwiVF5VLkEuViJdLFsxLDUsIlxcYW5jXlUiXSxbMyw2LCIiLDAseyJsZXZlbCI6Miwic3R5bGUiOnsiaGVhZCI6eyJuYW1lIjoibm9uZSJ9fX1dLFszLDQsIlReVS5cXGFuY15WIiwxXSxbMywyLCJUXlUuXFxwaV5WIiwxXSxbMSwyLCJcXGFuY15VIiwxXSxbMSwwLCJcXHBpXlUiLDFdLFs4LDVdLFs4LDYsIiIsMSx7ImxldmVsIjoyLCJzdHlsZSI6eyJoZWFkIjp7Im5hbWUiOiJub25lIn19fV0sWzcsMV0sWzcsM10sWzcsMiwiIiwxLHsic3R5bGUiOnsibmFtZSI6ImNvcm5lciJ9fV0sWzUsMCwicF5VLk0iLDFdLFs1LDIsIiIsMSx7ImxldmVsIjoyLCJzdHlsZSI6eyJoZWFkIjp7Im5hbWUiOiJub25lIn19fV0sWzYsMiwiVF5VLlxccGleViIsMV0sWzYsNF1d
\[\begin{tikzcd}
	&& {A.UV} \\
	& {A.U} && {T^U.A.V} \\
	M && {T^UM} && {T^{UV}M} \\
	& {T^UM} && {T^U.A.V} \\
	&& {T^U.A.V}
	\arrow["{\anc^U}", from=2-2, to=4-2]
	\arrow[Rightarrow, no head, from=2-4, to=4-4]
	\arrow["{T^U.\anc^V}"{description}, from=2-4, to=3-5]
	\arrow["{T^U.\pi^V}"{description}, from=2-4, to=3-3]
	\arrow["{\anc^U}"{description}, from=2-2, to=3-3]
	\arrow["{\pi^U}"{description}, from=2-2, to=3-1]
	\arrow[from=5-3, to=4-2]
	\arrow[Rightarrow, no head, from=5-3, to=4-4]
	\arrow[from=1-3, to=2-2]
	\arrow[from=1-3, to=2-4]
	\arrow["\lrcorner"{anchor=center, pos=0.125, rotate=-45}, draw=none, from=1-3, to=3-3]
	\arrow["{p^U.M}"{description}, from=4-2, to=3-1]
	\arrow[Rightarrow, no head, from=4-2, to=3-3]
	\arrow["{T^U.\pi^V}"{description}, from=4-4, to=3-3]
	\arrow[from=4-4, to=3-5]
\end{tikzcd}\]
\end{definition}

% \begin{proposition}
%     Let $(\pi:A \to M, \xi, \lambda, \sigma)$ be an involution algebroid with a chosen prolongations.
%     The following coherences hold:
%     \begin{enumerate}
%         \item $\anc^{U}_V \o (\theta \boxtimes \phi) = 
%     \end{enumerate}
% \end{proposition}

% The following proposition establishes some coherences and universality conditions for the tangent bundle that are not directly axiomatized Definition \ref{def:involution-algd} but are necessary to prove Theorem  \ref{thm:weil-nerve}.
% \begin{proposition}\label{prop:higher-tangent-bundle-construction}
%     Let $(\pi:A \to M, \xi, \lambda, \anc, \sigma)$ be an involution algebroid in a tangent category $\C$.
%     It follows that:
%     \begin{enumerate}[(i)]
%         \item Coassociativity of $\hat{\lambda}$: $(\hat{\lambda}\x\ell)\o\hat{\lambda} = (id \x T.\hat{\lambda}) \o \hat{\lambda}$
%         \item Coherence between $\sigma$ and $\hat{\lambda}$, $(\sigma \x c) \o (1\x T\sigma) \o (\hat{\lambda},\ell) = (1 \x T\hat{\lambda}) \o \sigma$
%         \item Both of the diagrams are $T$-pullbacks:
%         \input{TikzDrawings/Ch3/Sec5/inv-algd-universality.tikz}
%     \end{enumerate}
% \end{proposition}
% \begin{proof}
%     ~\begin{enumerate}[(i)]
%         \item Compute:
%             \begin{align*}
%                 (\hat{\lambda}\x \ell)\o \hat{\lambda} 
%                 &= (\xi\o\pi\o\xi\o\pi, \lambda\o\xi\o\pi, \ell\o\lambda) \\
%                 &= (\xi\o\pi, T.(\xi\o\pi)\o\lambda, T.\lambda \o \lambda) \\
%                 &= (\pi_0, T.(\xi\o\pi)\o\pi_1, T.\lambda \o \pi_1)\o (\xi\o\pi, \lambda) \\
%                 &= (id \x T(\hat{\lambda}))\o\hat{\lambda}
%             \end{align*}
%         \item Compute:
%             \begin{align*}
%                 (1\x T.\sigma) \o (\sigma \x c) \o (1 \x T.\hat{\lambda}) 
%                 &= (1 \x T.\sigma) \o (\sigma \o (\pi_0, T.(\xi\o\pi)\o\pi_1), c \o T.\lambda) \\
%                 &= (1 \x T.\sigma) \o (\xi\o\pi\o\pi_0, 0\o\pi_0, c \o T.\lambda\o\pi_1) \\
%                 &= (\xi\o\pi\o\pi_0, T.\sigma\o(0\o\pi_0,c\o T.\lambda \o \pi_1) \\
%                 &= (\xi\o\pi\o\pi_0, \lambda\pi_0, \ell\o\pi_1) \o \sigma = (\hat{\lambda} \x \ell) \o \sigma
%             \end{align*}
%         \item  Use the pullback lemma to observe that the following diagram is universal for any anchored bundle
%             % https://q.uiver.app/?q=WzAsNixbMCwwLCJBXzIiXSxbMCwxLCJNIl0sWzEsMCwiXFxwcm9sKEEpIl0sWzEsMSwiQSJdLFsyLDAsIlRBIl0sWzIsMSwiVE0iXSxbMSwzLCJcXHhpIl0sWzMsNSwiXFxhbmMiXSxbMSw1LCIwIiwyLHsiY3VydmUiOjJ9XSxbNCw1LCJUXFxwaSJdLFsyLDMsIlxccGlfMCJdLFswLDEsIlxccGlcXG9cXHBpXzEiLDJdLFswLDIsIlxcaGF0e1xcbXV9IiwyXSxbMiw0LCJcXHBpXzEiLDJdLFswLDQsIlxcbXUiLDAseyJjdXJ2ZSI6LTJ9XV0=
%             \input{TikzDrawings/Ch3/Sec5/inv-algd-mu-universal-proof.tikz}
%             Define the map $\hat{\mu}(a,b) := (\xi\o \pi a, \lambda \o a) +_{\pi_0} (\xi\o \pi\o  b, 0\o b)$ so the top triangle of the diagram commutes. The right square and outer perimeter are pullbacks by definition, and the bottom triangle also commutes by definition. The pullback lemma ensures that the left square is a pullback - this means that for every anchored bundle, the general lift is universal for $\prol(A)$. Now post-compose with the involution:
%             \input{TikzDrawings/Ch3/Sec5/inv-algd-nu-universal.tikz}
%             It suffices to check that the top triangle commutes, so $\sigma \o \hat{\mu} = \nu$:
%             \[
%                 \sigma \o \hat{\mu}\o (a,b) = \sigma \o   ((\xi\o \pi,0)\o a +_{\pi_0} (\xi\o \pi,\lambda)\o b) = 
%                  (id, T.\xi \o \anc \o a) +_{p\pi_1} (\xi\o \pi,\lambda)\o b
%             \]
%             The lift $(\xi\pi,\lambda)$ involution algebroid is universal for $\prol(A)$.
%     \end{enumerate}
% \end{proof}


% \begin{proposition}%
%     \label{prop:wone-anc-strict-moncat}
%     The category $\wone^\anc$ is a strict monoidal category, where the tensor product on objects $\prol(U,A),\prol(V,A)$ is the chosen span prolongation $\prol(UV,A)$ and the unit is $\prol(\N, A)$. On morphisms, the tensor product:
%     \[
%         \infer{
%             (f,\theta)\ox (g,\phi) = (f \ts{\anc^U}{\theta.\pi^V} T^U.g, f.g.M )
%         }{
%             (f,\theta):U \to V & (g,\phi): X \to Y
%         }  
%     \]
%     induced by the span composition:
%     \[\input{TikzDrawings/Ch4/Sec3/tensor.tikz}\]
%     And the unit is the identity span.
% \end{proposition}
% \begin{observation}%
%     \label{obs:apex-functor}
%     % The category $\wone^\anc$ is a subcategory of the category of spans in $\C$. Suppose a diagram $D: \d \to \wone^\anc$
%     There is a functor $U^\anc: \wone^\anc \to \C$ so that projects out the apex map, sending:
%     \[
%         (f,\theta): \prol(A,U) \to \prol(A,V) \in \wone^\anc  
%     \] 
%     to 
%     \[
%         f: \prol(A,U) \to \prol(A,V) \in \C
%     \]
%     Similarly, there is a functor from $U^\C: \wone^\anc \to \wone$ that sends:
%     \[
%         (f,\theta): \prol(A,U) \to \prol(A,V) \in \wone^\anc  
%     \]
%     to 
%     \[
%         \theta:U \to V  
%     \]
% \end{observation}
% \begin{proposition}%
%     \label{prop:limits-in-wanc}
%     Let $(\pi:A \to M, \xi, \lambda, \anc)$ be anchored bundle with chosen prolongations in a tangent category $\C$. Consider a commuting square $D$ 
%     \[\input{TikzDrawings/Ch4/pb-wanc.tikz}\]
%     so that $U^\C(D) \in \wone$ is a transverse limit, and $U^\anc(D)\in \C$ is a $T$ limit.
%     Then this square is a pullback in $\wone^\anc$ that is sent to a $T$-limit by $U^\anc$.
% \end{proposition}
% \begin{proof}
%     Start with a pair:
%     \[\input{TikzDrawings/Ch4/pb-wanc-2.tikz}\]
%     This square is a pullback in the category of spans by Observation \ref{obs:lim-of-spans}, so it suffices to prove that the induced map is in $\wone^\anc$. Consider the diagram:
%     \[\input{TikzDrawings/Ch4/pb-wanc-3.tikz}\]
%     The map $(\alpha,\beta):Q \to U \in \wone$ by transversality, and the diagram commutes because it is the limit in $\mathsf{Span}(\C)$. 
% \end{proof}
% \begin{corollary}
%     The tensor product in $\wone$ is continuous in each variable for limits of the form \Cref*{prop:limits-in-wanc}.
% \end{corollary}
% \begin{corollary}%
%     \label{cor:tang-limits-in-wone}
%     For any anchored bundle with chosen prolongations $(\pi:A \to M, \xi, \lambda, \anc)$ in a tangent category $\C$, the following diagrams are limits in $\wone^\anc$ and are sent to $T$-limits in $\anc$.
%     \begin{equation}
%         \label{eq:univ-lift}
%         \input{TikzDrawings/Ch4/univ-pb.tikz}
%     \end{equation}
%     \begin{equation}
%         \label{eq:span-limit}
%         \input{TikzDrawings/Ch4/span-pullback-diag.tikz}
%     \end{equation}
% \end{corollary}
% \begin{proof}
%     The base pullback in \Cref{eq:univ-lift} was proved to be a $T$-limit in part (iii) of Proposition  \ref{prop:higher-tangent-bundle-construction}, whereas \Cref{eq:span-limit} is part of the axioms of a differential bundle. Note that \Cref{eq:univ-lift} is mapped to the universality of the vertical lift in $\wone$, and \Cref{eq:span-limit} is sent to the pullback powers defining $W_n$.
% \end{proof}


% \begin{remark}
%     The simplicial object of an internal category requires a choice of $n$-fold span compositions $M \xleftarrow[]{s} C \xrightarrow[]{t} M$. The construction in Definition \ref{def:weil-anc-cat}, then, takes the full subcategory of $\C/(M \x M)$ and notes that the category of span compositions  $M \xleftarrow[]{s} C \xrightarrow[]{t} M$ is still a monoidal category.
% \end{remark}

% The construction of the structure maps builds on the prolongation construction in Definition \ref{def:prolongation-of-anchored-bundle}. First, following table associates maps in $\wone$ to the structure maps of an involution algebroid:
% \begin{center}
%     \begin{tabular}{|l|l|l|l|l|l|l|}
%     \hline
%     Tangent bundle & $T^2M,TM,M$ & $p$   & $0$   & $+$     & $\ell$    & $c$      \\ \hline
%     Involution algebroid & $\prol(A), A, M$     & $\pi$ & $\xi$ & $+_\pi$ & $\lambda$ & $\sigma$ \\ \hline
%     \end{tabular}
% \end{center}

% Rather that constructing an internal monoid in the span category, the Weil nerve induces a tangent structure on the monoidal category.
% \begin{proposition}%
%     \label{prop:wone-is-tang-cat}
%     Given an involution algebroid $(\pi:A \to M, \xi, \lambda, \anc, \sigma)$, the monoidal category $\wone^\anc$ has a tangent structure given by:
%     \begin{itemize}
%         \item $T := \prol(W,A) \ox (-), p := (\pi,p), 0:= (\xi,0), +:= (+_q,+)$.
%         \item $\ell := (\xi\o\pi,\lambda), c := (\sigma, c)$
%     \end{itemize}
% \end{proposition}
% \begin{proof}
%     Functoriality of $T$ and the naturality of $(p,0,+,\ell,c)$ are by construction. Now check each of the involution algebroid axioms:
%     \begin{enumerate}[{[TC.1]}]
%         \item Additive bundle axioms:
%         \begin{enumerate}[(i)]
%             \item By Corollary  \ref{cor:tang-limits-in-wone}, pullback powers of $p$ exist and are preserved by $T$.
%             \item The additive bundle structure comes from the addition induced 
%         \end{enumerate} 
%         \item Symmetry axioms:
%         \begin{enumerate}[(i)]
%             \item $c \o c$ follows from the involution axiom.
%             \item For Yang-Baxter, note that: 
%             \[c.T \o T.c \o c.T\] 
%             is equivalent to the Yang-Baxter equation on an involution algebroid:
%             \[
%               (\sigma \x c)  \o (id \x T.\sigma) \o (\sigma \x c) =
%               (id \x T.\sigma) \o (\sigma \x c)\o (id \x T.\sigma)
%             \]
%             and 
%             \[\sigma \x c = (c,\anc) \ox \prol(W,A) = c.T,
%             \hspace{0.15cm}
%             id \x T.\sigma = \prol(W,A) \ox (\sigma,c) = T.c\]
%             \item For the naturality conditions:
%             \begin{enumerate}[(a)]
%                 \item The interchange of $+,0,p$ all follow from the fact that $\sigma:(\prol(A),\lambda \x \ell) \to (\prol(A),id \x c \o T.\lambda)$ is linear, so it is an additive bundle morphism.
%                 \item The axiom 
%                 \[\ell.T \o c = T.c \o c.T \o T.\ell\] 
%                 is equivalent to the equation:
%                 \[
%                     (\sigma \x c) \o (1\x T.\sigma) \o (\hat{\lambda} \x \ell) = (1 \x T\hat{\lambda}) \o \sigma
%                 \]
%                 proved in (ii) of Proposition  \ref{prop:higher-tangent-bundle-construction} (which requires that $\sigma$ be bilinear).
%             \end{enumerate}
%         \end{enumerate}
%         \item The lift axioms:
%             \begin{enumerate}[(i)]
%                 \item The additive bundle equations are a consequence of $\lambda$ being a lift, and $+$ being the addition induced by the non-singularity of $\lambda$.
%                 \item The coassociativity axiom \[\ell.T \o \ell = T.\ell \o \ell\] 
%                 is equivalent to 
%                 \[(\hat{\lambda}\x\ell)\o\hat{\lambda} = (id \x T.\hat{\lambda}) \o \hat{\lambda}\] proved in (i) of Proposition  \ref{prop:higher-tangent-bundle-construction}.
%                 \item The symmetry of comultiplication, $c \o \ell = \ell$ is given by the unique equation for an involution algebroid, so that $\sigma \o (\xi\o\pi,\lambda) = (\xi\o\pi,\lambda)$.
%                 \item The universality of the lift follows from part (iii) of Proposition  \ref{prop:higher-tangent-bundle-construction} and the coherence on limits in Corollary  \ref{cor:tang-limits-in-wone}.
%             \end{enumerate}
%     \end{enumerate}
%     Observe that just as in the case for the equivalence between involution algebroids and Lie algebroids proved in Theorem  \ref{thm:iso-of-cats-Lie} and \Cref{sec:connections_on_an_involution_algebroid}, this used each equation and universality condition on an involution algebroid (the anchor is used in the construction of $\wone^\anc$).
%     % For [TC.1], that the pullback powers of $p$ exist and are preserved by $T$ follows by Corollary  \ref{cor:tang-limits-in-wone}, the additive bundle coherences follow from the involution algebroid axioms. The coherences in [TC.2] follow by the involution algebroid axioms and parts (i) and (ii) Proposition  \ref{prop:higher-tangent-bundle-construction}. The universality of the vertical lift axiom [TC.3] follows by Corollary  \ref{cor:tang-limits-in-wone}, so all of the coherences of a tangent structure are satisfied.
%     % To see that pullback powers of $p$ exist and are preserved by $T$, note that by the commutativity of pullbacks each $\prol(UW_nV,A)$ is the $n$-fold pullback of: 
%     % \[\input{TikzDrawings/Ch4/pullback-diag.tikz}\]
%     % This is also a pullback in the category $\wone^\anc$ - given a family of maps:
%     % \[
%     %     (f_i, \theta_i): \prol(X,A) \to   \prol(UWV,A), \hspace{.5cm}
%     %     (id_U \ox (\pi,p) \ox id_V) \o (f_i, \theta_i) = (f,\theta)
%     % \]
%     % then the following diagram commutes: 
%     % \begin{equation}
%     %     \label{eq:span-limit}
%     %     \input{TikzDrawings/Ch4/span-pullback-diag.tikz}
%     % \end{equation}
%     % The commutativity of limits in $\C$ ensures that this pullback is preserved by $\ox$, and consequently by $T^\anc$. Thus, this gives the additive bundle coherence [TC.1].
%     % The rest of the coherences in [TC.2] follow by the involution algebroid axioms and parts (i) and (ii) Proposition  \ref{prop:higher-tangent-bundle-construction}.
%     % For the universality of the vertical lift, the diagram:
%     % \[\input{TikzDrawings/Ch1/univ-lift.tikz}\]
%     % is a pullback by part (iii) of Proposition  \ref{prop:higher-tangent-bundle-construction}. 
%     % Then observe that by the commutativity of $T$-limits, the following diagram is a pullback in $\C$:
%     % \begin{equation}
%     %     \label{eq:univ-lift}
%     %     \input{TikzDrawings/Ch4/univ-pb.tikz}
%     % \end{equation}
%     % By the same argument as in \Cref{eq:span-limit}, the uniquely induced map
%     % \[\input{TikzDrawings/Ch4/univ-pb.tikz}\]
%     % will be a morphism in $\wone^\anc$ - exhibiting \Cref{eq:univ-lift} as a pullback in $\wone^\anc$ - this induces 
% \end{proof}

\begin{theorem}[The Weil Nerve]
    \label{thm:weil-nerve}
    There is a fully faithful functor
    \[
        \mathsf{N}_{\weil}: \mathsf{Inv}(\C) \to [\wone, \C]  
    \]
    that sends an involution algebroid to the transverse-limit-preserving tangent functor:
    \[
        (\widehat{A},\alpha): \wone \to \C 
    \]
\end{theorem}
\begin{proof}
    % Starting with objects, note that $V \mapsto \prol(V)$. %
    % \begin{center}
    %     \begin{tabular}{|l|l|}
    %         \hline
    %         $\wone$           & $\widehat{A}:\wone \to \C$                    \\ \hline
    %         $W_n$             & $\hat{A}.W_n$                                \\ \hline
    %         $U \ox V$         & $\hat{A}.U \boxtimes\hat{A}.V$             \\ \hline
    %         $+:T_2 \to T$     & $\hat{A}.+: \hat{A}.W_2 \to  \hat{A}.W$ \\ \hline
    %         $p:T \to \N$      & $\hat{A}.p: \hat{A}.W \to \hat{A}.\N$  \\ \hline
    %         $0: I \to T$      & $\hat{A}.0: \hat{A}.\N \to  \hat{A}.W)$  \\ \hline
    %         $\ell:T \to T^2$  & $\hat{A}.\ell:  \hat{A}.W \to  \hat{A}.WW$ \\ \hline
    %         $c:T^2 \to T^2$   & $\hat{A}.WW \to  \hat{A}.WW$                  \\ \hline
    %         $\phi \ox \theta$ & $ \hat{A}.\phi \boxtimes  \hat{A}.\theta$        \\ \hline
    %     \end{tabular}    
    % \end{center}
    For the first step of this proof, we show that an involution algebroid structure on an anchored bundle $(\pi:A \to M, \xi, \lambda, \anc)$ determines a tangent category structure on the monoidal category $\mathsf{Span}(\pi:A \to M, \xi, \lambda, \anc)$.
    
    We check that the endofunctor $\hat{A} \boxtimes (-)$ determines a tangent structure, with the structure maps given by Definition \ref{def:generators-for-wone-in-anc}:
    
    % The $\boxtimes$ operation preserves equations on the left and right argument and is continuous. Therefore, it suffices to check that equations hold on the basic generators of $\wone$, and that the three generating transverse limits (identity, $W_2 = W \ts{p}{p} W$, and universality of the lift) are sent to $T$-limits.
    \begin{enumerate}[{[TC.1]}]
        \item Additive bundle axioms:
        \begin{enumerate}[(i)]
            \item Use Lemma \ref{lemma:pullback-part-of-theorem} to see that \[ \hat{A}.W_2 = \hat{A}.W \ts{p}{p} \hat{A}.W  A \ts{\pi}{\pi} A;\] this is preserved by $\hat A.V \boxtimes (-)$.
            \item The triple $(\hat{A}.+,\hat{A}.p,\hat{A}.0) = (+_q, \pi, \xi)$ is an additive bundle induced by Proposition  \ref{prop:induce-abun}, and $\boxtimes$ preserves pullbacks (and therefore additive bundles), so the additive bundle axioms hold.
        \end{enumerate} 
        \item Symmetry axioms:
        \begin{enumerate}[(i)]
            \item $\hat{A}.c \o \hat{A}.c = id$ follows from the involution axiom $\sigma \o \sigma = id$.
            \item For Yang--Baxter, note that 
            \[ 
                (\hat{A} \boxtimes c) \o (c \boxtimes \hat{A}) \o (\hat{A} \boxtimes c)  = 
                (c \boxtimes \hat{A})  \o (\hat{A} \boxtimes c)  \o (c \boxtimes \hat{A}) 
            \] 
            follows from the Yang--Baxter equation on an involution algebroid
            \[
              (\sigma \x c)  \o (id \x T.\sigma) \o (\sigma \x c) =
              (id \x T.\sigma) \o (\sigma \x c)\o (id \x T.\sigma),
            \]
            since 
            \[\sigma \x c.A = (\hat{A}.c) \boxtimes (\hat{A}.W )
                \text{ and }
            id \x T.\sigma = (\hat{A}.W ) \boxtimes c. \]
            % Note that $\boxtims$ preserves equations, so this holds for $U \boxtimes (W \boxtimes c) \boxtimes V$ and $U \boxtimes (c \boxtimes W) \boxtimes V$.
            \item For the naturality conditions:
            \begin{enumerate}[(a)]
                \item The interchanges of $+,0,p$ all follow from the fact that 
                \[\sigma:(A.WW,\lambda \x \ell) \to (A.WW,id \x c \o T.\lambda)\] 
                is linear, and so is an additive bundle morphism.
                \item The axiom 
                \[\ell.T \o c = T.c \o c.T \o T.\ell\] 
                is equivalent to the equation
                \[
                    (\sigma \x c) \o (1\x T.\sigma) \o (\hat{\lambda} \x \ell) = (1 \x T\hat{\lambda}) \o \sigma
                \]
                which is equivalent to the double linearity axiom on $\sigma$ by Proposition  \ref{prop:nat-of-sigma-ell}.
            \end{enumerate}
        \end{enumerate}
        \item The lift axioms:
            \begin{enumerate}[(i)]
                \item The additive bundle equations are a consequence of $\lambda$ being a lift and $+$ being the addition induced by the non-singularity of $\lambda$.
                \item The coassociativity axiom 
                \[\ell.T \o \ell = T.\ell \o \ell\] 
                is equivalent to 
                \[(\hat{\lambda}\x\ell)\o\hat{\lambda} = (id \x T.\hat{\lambda}) \o \hat{\lambda}\] 
                proved in (i) of Proposition  \ref{prop:lift-axioms-anchor}.
                \item The symmetry of comultiplication, $c \o \ell = \ell$, is given by the unique equation for an involution algebroid, so that $\sigma \o (\xi\o\pi,\lambda) = (\xi\o\pi,\lambda)$.
                \item The universality of the lift follows from part (ii) of Proposition  \ref{prop:lift-axioms-anchor}; Lemma \ref{lemma:pullback-part-of-theorem} ensures that for any $V \in \wone$, $\widehat{A}.V \boxtimes \mu$ and $\mu \boxtimes \widehat{A}.V$ are universal.
            \end{enumerate}
    \end{enumerate}
    This lemma puts a tangent structure on $\mathsf{Span}(\pi,\xi,\lambda,\anc)$. Now consider the functor sending spans to the apex map, 
    \[
        U^\anc: \mathsf{Span}(\pi,\xi,\lambda,\anc) \to \C.
    \]
    The family of maps
    \[
        \{\anc^U.V: \anc^U \boxtimes \widehat A.V | U,V \in \wone\}
    \]
    gives a family of natural transformations
    \[
        \anc^{U}:A.T^U \Rightarrow T^U.A,
    \]
    so that the following pair constitute a tangent functor
    \[
      (U^\anc,\anc): \mathsf{Span}(\pi,\xi,\lambda,\anc) \to \C.
    \]
    Because the universality conditions on $\mathsf{Span}(\pi,\xi,\lambda,\anc)$ followed by reflecting limits in $\C$ using Lemma \ref{lemma:pullback-part-of-theorem}, it follows that $(U^\anc,\anc)$ will preserve the tangent-natural limits in $\mathsf{Span}(\pi,\xi,\lambda,\anc)$ corresponding to transverse limits in $\wone$.
    
    By Leung's Theorem  \ref{thm:leung} (by way of Corollary \ref{cor:using-leung-thm}), the tangent structure on $\mathsf{Span}(\pi,\xi,\lambda,\anc)$ induces a strict, monoidal, transverse-limit-preserving functor
    \[
        \bar{A}: \wone \to \mathsf{Span}(\pi,\xi,\lambda,\anc)    
    \]
    that sends the tensor product $\ox$ to the span composition $\boxtimes$. By composing the strict tangent functor $(\bar{A},id)$ and $(U^\anc,\anc)$, we have a lax, transverse-limit-preserving, tangent functor:
    \[
        (A,\anc):\wone \to \C; V \mapsto A.V
    \]
    

    % The natural transformation is defined \[\alpha: \hat{A}.UV \Rightarrow T^U.\hat{A}.V = \anc^U \boxtimes \hat{A}.V \] using the definition of $\anc$ given in Definition \ref{def:anc-nat}. First look at $\anc^U \boxtimes \prol(V,A) \o (f \boxtimes g)$:
    % \[\input{TikzDrawings/Ch4/Sec4/anc-is-nat.tikz}\]
    % Then look at $\theta.g \o (\anc^U \boxtimes \prol(V,A))$:
    % \[\input{TikzDrawings/Ch4/Sec4/anc-is-nat-2.tikz}\]
    % The two maps induced from $\hat{A}.UV \to T^{U'}.\hat{A}.V'$ are equal, so $\alpha$ is natural.
    % The composite span is given by $\theta.g $
    % The assignment on objects was induced by Corollary  \ref{cor:actual-nerve}, note that a strict tangent functor $\wone \to \wone^\anc$ will preserve all transverse limits, and that $U^\anc:\wone^\anc \to \C$ will preserve tangent limits (as tangent limits in $\wone^\anc$ are computed ``pointwise'' in $\C$), all that remains is the bijection on morphisms.

    Now, check the bijection on morphisms. Starting with an involution algebroid morphism $(f,m):A \to B$, note that this gives a span morphism $\hat f$:
    % https://q.uiver.app/?q=WzAsNixbMCwwLCJNIl0sWzEsMCwiQSJdLFsyLDAsIlRNIl0sWzAsMSwiTiJdLFsxLDEsIkIiXSxbMiwxLCJUTiJdLFsyLDUsIlQubSJdLFswLDMsIm0iXSxbMSwwXSxbMSwyXSxbNCw1XSxbNCwzXSxbMSw0LCJmIiwxXV0=
\[\begin{tikzcd}[ampersand replacement=\&]
	M \& A \& TM \\
	N \& B \& TN
	\arrow["{T.m}", from=1-3, to=2-3]
	\arrow["m", from=1-1, to=2-1]
	\arrow[from=1-2, to=1-1]
	\arrow[from=1-2, to=1-3]
	\arrow[from=2-2, to=2-3]
	\arrow[from=2-2, to=2-1]
	\arrow["f", from=1-2, to=2-2]
\end{tikzcd}\]
    This gives a natural definition of $\hat{f}.V$ using the horizontal composition of span morphism, so that 
    \begin{equation}\label{eq:boxtimes-def-of-hat-f}
        \hat{f}.(UV) = \hat{f}.U \boxtimes \hat{f}.V \text{ and } \hat{f}.\N = m,
    \end{equation}
    giving a family of maps $\{\hat{f}_{U}: U \in \mathsf{objects}(\wone)\}$. Because $f$ will commute with the structure maps $\{ \pi,\xi,+,(\xi\o\pi,\lambda),\sigma\}$, it follow immediately that $\hat{f}$ is a natural transformation, because the following calculation holds for each $\theta:X \to Y \in \{ p,0,+,\ell,c\}$:
    \begin{align*}
        & \quad \hat{f}.UYV \o (\hat{A}.U \boxtimes \theta \boxtimes \hat{A}.V) \\
        &= (\hat{f}.U \boxtimes \hat{f}.Y \boxtimes \hat{f}.V) \o  (\hat{A}.U \boxtimes \theta \boxtimes \hat{A}.V) \\
        &= \hat{f}.U \boxtimes( \hat{f}.Y \o \theta)\boxtimes \hat{f}.V \\
        &= \hat{f}.U \boxtimes( \theta \o \hat{f}.X) \boxtimes \hat{f}.V \\
        &= (\hat{A}.U \boxtimes \theta \boxtimes \hat{A}.V) \o \hat{f}.UXV
        % =& (\prol(U,B) \ts{}{} T^U.\theta \ts{}{} T^{UY}.\prol(U,B)) \o (\prol(U,f) \ts{}{} T^U.\prol(X,f) \ts{}{} T^{UY}.\prol(V,f)) \\
        % =& (\prol(U,B) \boxtimes \theta \boxtimes \prol(U,B)) \o\prol(UXV,f) 
    \end{align*}
    Tangent naturality will follow by the preservation of the anchor map by $f$. 
    The equality, for any Weil algebra $U$, of the diagrams
% https://q.uiver.app/?q=WzAsMTgsWzAsMCwiTSJdLFsxLDAsIkEuVSJdLFsyLDAsIlReVU0iXSxbMCwxLCJOIl0sWzEsMSwiQi5VIl0sWzIsMSwiVF5VTiJdLFswLDIsIk4iXSxbMSwyLCJUXlVOIl0sWzIsMiwiVF5VTiJdLFs0LDEsIlReVU0iXSxbNCwwLCJBLlUiXSxbMywwLCJNIl0sWzUsMCwiVF5VTSJdLFszLDEsIk0iXSxbNSwxLCJUXlVNIl0sWzMsMiwiTiJdLFs0LDIsIlReVk4iXSxbNSwyLCJUXlZOIl0sWzIsNSwiVF5VLm0iXSxbMCwzLCJtIl0sWzEsMCwiXFxwaV5VIiwyXSxbMSwyLCJcXGFuY15VIl0sWzQsNSwiXFxhbmNeVSJdLFs0LDNdLFsxLDQsIlxcaGF0e2Z9LlUiLDFdLFs0LDcsIlxcYW5jXlUiXSxbNyw2LCJwXlUiXSxbMyw2LCIiLDAseyJsZXZlbCI6Miwic3R5bGUiOnsiaGVhZCI6eyJuYW1lIjoibm9uZSJ9fX1dLFs1LDgsIiIsMCx7ImxldmVsIjoyLCJzdHlsZSI6eyJoZWFkIjp7Im5hbWUiOiJub25lIn19fV0sWzcsOCwiIiwwLHsibGV2ZWwiOjIsInN0eWxlIjp7ImhlYWQiOnsibmFtZSI6Im5vbmUifX19XSxbMTAsMTEsIlxccGleVSIsMl0sWzksMTMsInBeVSIsMl0sWzEwLDEyLCJcXGFuY15VIiwyXSxbOSwxNCwiIiwyLHsibGV2ZWwiOjIsInN0eWxlIjp7ImhlYWQiOnsibmFtZSI6Im5vbmUifX19XSxbMTYsMTcsIiIsMix7ImxldmVsIjoyLCJzdHlsZSI6eyJoZWFkIjp7Im5hbWUiOiJub25lIn19fV0sWzE2LDE1LCJwXlUiXSxbMTIsMTQsIiIsMSx7ImxldmVsIjoyLCJzdHlsZSI6eyJoZWFkIjp7Im5hbWUiOiJub25lIn19fV0sWzE0LDE3LCJUXlUubSJdLFs5LDE2LCJUXlUubSJdLFsxMywxNSwibSIsMl0sWzExLDEzLCIiLDIseyJsZXZlbCI6Miwic3R5bGUiOnsiaGVhZCI6eyJuYW1lIjoibm9uZSJ9fX1dLFsxMCw5LCJcXGFuY15VIl0sWzUsMTMsIj0iLDEseyJzdHlsZSI6eyJib2R5Ijp7Im5hbWUiOiJub25lIn0sImhlYWQiOnsibmFtZSI6Im5vbmUifX19XV0=
\[\begin{tikzcd}[ampersand replacement=\&]
	M \& {A.U} \& {T^UM} \& M \& {A.U} \& {T^UM} \\
	N \& {B.U} \& {T^UN} \& M \& {T^UM} \& {T^UM} \\
	N \& {T^UN} \& {T^UN} \& N \& {T^VN} \& {T^VN}
	\arrow["{T^U.m}", from=1-3, to=2-3]
	\arrow["m", from=1-1, to=2-1]
	\arrow["{\pi^U}"', from=1-2, to=1-1]
	\arrow["{\anc^U}", from=1-2, to=1-3]
	\arrow["{\anc^U}", from=2-2, to=2-3]
	\arrow[from=2-2, to=2-1]
	\arrow["{\hat{f}.U}"{description}, from=1-2, to=2-2]
	\arrow["{\anc^U}", from=2-2, to=3-2]
	\arrow["{p^U}", from=3-2, to=3-1]
	\arrow[Rightarrow, no head, from=2-1, to=3-1]
	\arrow[Rightarrow, no head, from=2-3, to=3-3]
	\arrow[Rightarrow, no head, from=3-2, to=3-3]
	\arrow["{\pi^U}"', from=1-5, to=1-4]
	\arrow["{p^U}"', from=2-5, to=2-4]
	\arrow["{\anc^U}"', from=1-5, to=1-6]
	\arrow[Rightarrow, no head, from=2-5, to=2-6]
	\arrow[Rightarrow, no head, from=3-5, to=3-6]
	\arrow["{p^U}", from=3-5, to=3-4]
	\arrow[Rightarrow, no head, from=1-6, to=2-6]
	\arrow["{T^U.m}", from=2-6, to=3-6]
	\arrow["{T^U.m}", from=2-5, to=3-5]
	\arrow["m"', from=2-4, to=3-4]
	\arrow[Rightarrow, no head, from=1-4, to=2-4]
	\arrow["{\anc^U}", from=1-5, to=2-5]
	\arrow["{=}"{description}, draw=none, from=2-3, to=2-4]
\end{tikzcd}\]
is precisely the tangent-naturality condition from Definitions \ref{def:tang-nat}, \ref{def:actegory-natural}.
    
    For the inverse of this mapping, consider a tangent natural transformation (Definition \ref{def:tang-nat})
    \[
        \gamma: ({A},\alpha) \to ({B}, \beta),  \hspace{0.15cm}
            \input{TikzDrawings/Ch1/tang-nat-AB.tikz}
    \]
    where $(A,\alpha)$ and $(B,\beta)$ are tangent functors $\wone \to \C$ built out of involution algebroids with chosen prolongations. For any $U,V$, the map $\gamma.UV$ decomposes as $\gamma.U \boxtimes \gamma.V$:
    \[
        \input{TikzDrawings/Ch4/cube-map-nat.tikz}   
    \]
    Applying this relationship inductively, it is clear that the base maps $\gamma.W$ and $\gamma.\N$ determine the entire morphism $\gamma.V$:
% https://q.uiver.app/?q=WzAsMTYsWzAsMiwiTSJdLFsyLDIsIlRfe25bMV19Lk0iXSxbMywyLCJcXGRvdHMiXSxbNCwyLCJUX3tuWzFdfVxcZG90cyBUX3tuW2tdfS5NIl0sWzQsMywiVF97blsxXX1cXGRvdHMgVF97bltrXX0uTiJdLFsyLDMsIlRfe25bMV19Lk0iXSxbMCwzLCJOIl0sWzEsMiwiQV97blsxXX0iXSxbMSwzLCJCX3tuWzFdfSJdLFszLDMsIlxcZG90cyJdLFsxLDAsIk0iXSxbMSwxLCJOIl0sWzMsMCwiVF5VTSJdLFszLDEsIlReVU4iXSxbMiwwLCJBLlUiXSxbMiwxLCJCLlUiXSxbMyw0LCJUX3tuWzFdfVxcZG90cyBUX3tuW2tdfS5cXGdhbW1hLlxcTiJdLFswLDYsIlxcZ2FtbWEuXFxOIl0sWzcsMSwiXFxhbmNfe25bMV19Il0sWzcsMCwiXFxwaVxcb1xccGlfaSIsMl0sWzgsNiwiXFxwaVxcb1xccGlfaSJdLFs4LDUsIlxcYW5jX3tuWzFdfSIsMl0sWzcsOCwiXFxnYW1tYS5XX3tuWzFdfSJdLFsxLDUsIlRfe25bMV19LlxcZ2FtbWEuXFxOIl0sWzIsMSwiVF97blsxXX0uKFxccGkgXFxvIFxccGlfaSkiLDJdLFsyLDNdLFs5LDUsIlRfe25bMV19LihcXHBpIFxcbyBcXHBpX2kpIl0sWzksNF0sWzIsOSwiXFxkb3RzIiwxLHsic3R5bGUiOnsiYm9keSI6eyJuYW1lIjoibm9uZSJ9LCJoZWFkIjp7Im5hbWUiOiJub25lIn19fV0sWzEyLDEzLCJUXlUuXFxnYW1tYS5cXE4iLDJdLFsxNCwxMCwiXFxwaV5VIiwyXSxbMTUsMTEsIlxccGleVSJdLFsxNSwxMywiXFxhbmNeVSIsMl0sWzE0LDEyLCJcXGFuY15VIl0sWzEwLDExLCJcXGdhbW1hLlxcTiIsMl0sWzE0LDE1LCJcXGdhbW1hLlUiLDJdLFszNSwyMywiPSIsMSx7InNob3J0ZW4iOnsic291cmNlIjoyMCwidGFyZ2V0IjoyMH0sInN0eWxlIjp7ImJvZHkiOnsibmFtZSI6Im5vbmUifSwiaGVhZCI6eyJuYW1lIjoibm9uZSJ9fX1dXQ==
\[\begin{tikzcd}
	& M & {A.U} & {T^UM} \\
	& N & {B.U} & {T^UN} \\
	M & {A_{n[1]}} & {T_{n[1]}.M} & \dots & {T_{n[1]}\dots T_{n[k]}.M} \\
	N & {B_{n[1]}} & {T_{n[1]}.M} & \dots & {T_{n[1]}\dots T_{n[k]}.N}
	\arrow["{T_{n[1]}\dots T_{n[k]}.\gamma.\N}", from=3-5, to=4-5]
	\arrow["{\gamma.\N}", from=3-1, to=4-1]
	\arrow["{\anc_{n[1]}}", from=3-2, to=3-3]
	\arrow["{\pi\o\pi_i}"', from=3-2, to=3-1]
	\arrow["{\pi\o\pi_i}", from=4-2, to=4-1]
	\arrow["{\anc_{n[1]}}"', from=4-2, to=4-3]
	\arrow["{\gamma.W_{n[1]}}", from=3-2, to=4-2]
	\arrow[""{name=0, anchor=center, inner sep=0}, "{T_{n[1]}.\gamma.\N}", from=3-3, to=4-3]
	\arrow["{T_{n[1]}.(\pi \o \pi_i)}"', from=3-4, to=3-3]
	\arrow[from=3-4, to=3-5]
	\arrow["{T_{n[1]}.(\pi \o \pi_i)}", from=4-4, to=4-3]
	\arrow[from=4-4, to=4-5]
	\arrow["\dots"{description}, draw=none, from=3-4, to=4-4]
	\arrow["{T^U.\gamma.\N}"', from=1-4, to=2-4]
	\arrow["{\pi^U}"', from=1-3, to=1-2]
	\arrow["{\pi^U}", from=2-3, to=2-2]
	\arrow["{\anc^U}"', from=2-3, to=2-4]
	\arrow["{\anc^U}", from=1-3, to=1-4]
	\arrow["{\gamma.\N}"', from=1-2, to=2-2]
	\arrow[""{name=1, anchor=center, inner sep=0}, "{\gamma.U}"', from=1-3, to=2-3]
	\arrow["{=}"{description}, Rightarrow, draw=none, from=1, to=0]
\end{tikzcd}\]
    Thus, every tangent-natural transformation is constructed out of a pair
    \[
        (\gamma.\N: M \to N, \gamma.W:A \to B)
    \]
    using the $\boxtimes$ construction from Equation \ref{eq:boxtimes-def-of-hat-f}.
    All that remains to show is that this pair is an involution algebroid morphism. 

    Tangent naturality gives the following two coherences:
    \[
        \anc^B \o \gamma.W = T.\gamma.\N \o \anc^A \text{ and } 
        \sigma^B \o \gamma.WW = \gamma.WW \o \sigma^A
    \]
    since $\anc^B = \beta.W, \anc^A = \alpha.W, \sigma^B = B.c$, and $\sigma^A = A.c$ by construction. The following diagram proves that $\gamma.W$ preserves the lifts, so that $(\gamma.W, \gamma.\N)$ is an involution algebroid morphism:
    \[\input{TikzDrawings/Ch4/Sec4/tang-nat-pres-lifts.tikz}\]
    % and anchors:
    % \input{TikzDrawings/Ch4/Sec4/tang-nat-pres-anchors.tikz}
    % And the $\gamma$ commutes with the involution by naturality. 
    Thus, a tangent natural transformation $(\widehat{A},\alpha) \to (\widehat{B}, \beta)$ is exactly a morphism of involution algebroids $A \to B$, proving the theorem.
    
    % The bijection follows from the fact that by tangent naturality, every
    % \[
    %     f.V: \widehat{A}.V \to \widehat{B}.V  
    % \]
    % decomposes as a pullback power of the base maps $f.\N, f.W$, which are exactly involution algebroid morphisms.
\end{proof} 
Now, the projection for a Lie algebroid is a submersion, as we may make a choice of prolongations for each $U \in \wone$. These prolongations lead to a new observation about Lie algebroids: they embed into a category of functors into smooth manifolds.
\begin{corollary}%
    \label{cor:SMan-embedding}
    Using the Weil nerve construction, the category of Lie algebroids embeds into the tangent-functor category:
    \[
        \mathsf{LieAlgd} \hookrightarrow [\wone, \mathsf{SMan}].  
    \]
\end{corollary}





\section{Identifying involution algebroids}%
\label{sec:identifying-involution-algebroids}

This section identifies those tangent functors
\[
    (A,\alpha): \wone \to \C  
\] 
that are involution algebroids as precisely those where $A$ preserves transverse limits and $\alpha$ is a \emph{$T$-cartesian} natural transformation (Definition \ref{def:cart-nat}). 
These conditions will force each $A.V$ to be the $V$-prolongation of the underlying anchored pre-differential bundle:
\[
    (A.p: A.T \to A,\;\ A.0: A \to A.T,\;\ A.T \xrightarrow[]{A.\ell} A.T.T \xrightarrow[]{\alpha.T} T.A.T,\;\ \alpha:A.T \to T.A )  
\]
(these conditions also ensure that this tuple is an anchored differential bundle). 

Initially, it is only clear that $\alpha$ is $T$-cartesian for the projection $p$. Indeed, recall that the prolongation $A.UV$ is defined to be the $T$-pullback of the cospan:
\[ \widehat A .U \xrightarrow[]{\alpha^U} T^U.A.\N \xleftarrow[]{T^U.A.p^V} T^U.\widehat{ A}.V\]
Then consider the following diagram:
\[\input{TikzDrawings/Ch4/anc-cart-p.tikz}\]
This means that every naturality square of $\alpha$ for $p$ is a $T$-pullback; natural transformations satisfying this property for every map in the domain category are called $T$-cartesian.
\begin{definition}%
    \label{def:cart-nat}
    A natural transformation $\gamma: F \Rightarrow G$ is \emph{cartesian} whenever each naturality square
    \[\input{TikzDrawings/Ch4/Sec4/equifibred.tikz}\]
    is a pullback. A natural transformation between functors into a tangent category is \emph{$T$-cartesian} whenever each component square is a $T$-pullback (we will generally suppress the $T$ when the context is clear). 
\end{definition}
% \begin{example}
%     For the Weil nerve of an involution algebroid, $\alpha$ is $T$-cartesian for
% \end{example}

Now, recall that the Weil complex determined by an involution algebroid has $A.U.V$ determined by the following $T$-pullback squares:
\[\input{TikzDrawings/Ch4/Sec5/weil-nerve-pb.tikz}\]
Then it is not difficult to show that the $T$-cartesian condition on a Weil complex forces it to be an involution algebroid.  We first need:
\begin{definition}
    A $T$-cartesian Weil complex in $\C$ is a tangent functor
    \[
        (A,\alpha): \wone \to \C  
    \]
    for which $A$ sends transverse limits to $T$-limits and $\alpha$ is a $T$-cartesian natural transformation.
\end{definition}
The first condition to check is that a $T$-cartesian Weil complex gives a natural anchored bundle $\hat{A}$ whose Weil prolongations coincide with the functor assignments on objects.
\begin{proposition}%
    \label{prop:anc-bun-cw-complex}
    Let $(A,\alpha)$ be a $T$-cartesian Weil complex. Then we have an anchored bundle
    \[
        (M  := A.\N, \hspace{0.15cm}
        \hat{A} := A.W , \hspace{0.15cm}
        \pi := A.\pi, \hspace*{0.15cm}
        \xi := A.\xi, \hspace*{0.15cm}
        \lambda := \alpha.T \o A.\ell).
    \]
    Furthermore, 
    \[
        \prol(\hat{A}) = A.WW, \hspace*{0.30cm}
        \prol^2(\hat{A}) = A.WWW. 
    \]
\end{proposition}
\begin{proof}
    Suppose we have a tangent functor $(F,\alpha): \C \to \D$ and a differential bundle $(\pi,\xi,\lambda)$ in $\C$. If $F$ preserves $T$-pullbacks of $\pi$, it preserves the additive bundle structure on $(\pi,\xi,+)$, so to show $(F.\pi, F.\xi, \alpha \o F.\lambda)$ is universal it suffices to show that the following diagram is a $T$-pullback in $\D$:
    \input{TikzDrawings/Ch4/Sec5/mu-anc-lambda.tikz}
    Expand this to
    \input{TikzDrawings/Ch4/Sec5/mu-anc-lambda-expanded.tikz}
    In this case, it restricts to the diagram
    \input{TikzDrawings/Ch4/Sec5/mu-anc-lambda-restricted-diagram.tikz}
    Each square is a $T$-pullback by hypothesis, so the universality of the lift follows by the $T$-pullback lemma. Because the complex is $T$-cartesian, the assignment $A.V$ gives a coherent choice of prolongations by the $T$-pullback
    \input{TikzDrawings/Ch4/Sec5/mu-anc-equifibered-prol.tikz}
\end{proof}
There is, of course, a natural candidate for the involution map.
\begin{corollary}
    Let $(\pi:A \to M, \xi, \lambda, \anc)$ be the anchored bundle induced by a $T$-cartesian Weil complex in a tangent category $\C$. Then we have an involution map
    \[
        \sigma: \prol(A) \xrightarrow{A.c} \prol(A).
    \]
\end{corollary}
The equations for an involution algebroid should follow immediately by functoriality; one need only ensure that the maps take the correct form.
\begin{lemma}\label{lem:cwm-map-structure}
    Let $A$ be a $T$-cartesian Weil complex in a tangent category $\C$, with $(\pi:A \to M, \xi, \lambda, \sigma)$ its underlying anchored bundle.
    Then we have:
    \begin{enumerate}[(i)]
        \item $A.c.T = \sigma \x c$,
        \item $A.T.c = 1 \x T.\sigma$,
        \item $A.\ell.T = \hat{\lambda} \x \ell.A$,
        \item $A.T.\ell = id \x T.\hat{\lambda}$.
    \end{enumerate}
\end{lemma}
\begin{proof}
    ~\begin{enumerate}[(i)]
        \item Consider the diagram
        \input{TikzDrawings/Ch4/Sec5/rewrite-act.tikz}
        Observe that this forces $A.c.T = A.c \x c.A.T = \sigma \x c$.
        \item Likewise, the diagram
        \input{TikzDrawings/Ch4/Sec5/rewrite-atc.tikz}
        forces $A.T.c = id \x T.A.c = id \x T.\sigma$.
        \item The diagram
        \input{TikzDrawings/Ch4/Sec5/rewrite-alt.tikz}
        forces $A.\ell.T = A.\ell \x \ell.A = \hat{\lambda} \x \ell$.
        \item As $\alpha$ is $T$-cartesian, the following diagram is a $T$-pullback:
        \input{TikzDrawings/Ch4/Sec5/rewrite-atl-1.tikz}
        Using previous results, this means that $A.\ell.T$ is the unique map making the following diagram commute:
        \input{TikzDrawings/Ch4/Sec5/rewrite-atl-2.tikz}
        which we can see is $id \x \hat\lambda$.
    \end{enumerate}
\end{proof}

Pulling together this lemma and the previous proposition, the following is now clear:
\begin{proposition}
    A $T$-cartesian Weil complex determines an involution algebroid.
\end{proposition}
However, we have not yet exhibited an isomorphism of categories between the image of the Weil nerve functor and $T$-cartesian Weil complexes. 
At first glance, the Weil nerve construction only gives a Weil complex that is $T$-cartesian for the tangent projection $p \in \wone$. Being $T$-cartesian for $p$ is, however, sufficient: a Weil complex that is $T$-cartesian for tangent projections will be $T$-cartesian for every map in $\wone$ (a similar result appears in the context of differentiable programming languages; see \cite{Cruttwell2019}).

\begin{proposition}\label{prop:mod-is-cart-if-p}
    A lax tranverse-limit-preserving tangent functor $(F,\alpha):\wone \to \C$ for which $F$ preserves pullback powers of each $T^U.p$ is $T$-cartesian if and only if each
    \begin{equation*}
        \input{TikzDrawings/Ch4/Sec4/equifibred-iff-p.tikz}
    \end{equation*}
    is a $T$-pullback.
\end{proposition}
\begin{proof}
    We only check the converse since the forward implication is trivial. We make use of the $T$-pullback lemma.
    \begin{enumerate}[(i)]
        \item $c$ is an isomorphism, so its naturality square is a $T$-pullback.
        \item For projections $T_2 \to T$,  the retract of a $T$-pullback diagram is a $T$-pullback, so the following diagram is universal:
        \input{TikzDrawings/Ch4/Sec5/pcartiff-proj.tikz}
        \item For $0$, observe that the following two diagrams are equal:
        \input{TikzDrawings/Ch4/Sec5/pcartiff-zero.tikz}
        The right diagram is a $T$-pullback, and the right square of the left diagram is a $T$-pullback by hypothesis.
        By the $T$-pullback lemma, the left square of the left diagram is a $T$-pullback.
        \item For $\ell$, observe that
        \input{TikzDrawings/Ch4/Sec5/pcart-iff-ell.tikz}
        The outer perimeter of the right diagram is a $T$-pullback (left square by hypothesis, right square by (ii)), as is the right square of the left diagram (by hypothesis). 
        By the $T$-pullback lemma, the left square of the left diagram is a $T$-pullback.
        \item For $+$, observe that
        \input{TikzDrawings/Ch4/Sec5/pcart-iff-add.tikz}
        The outer diagram on the right is a $T$-pullback by composition, and the right square on the left diagram is a $T$-pullback by hypothesis, so the result follows. 
    \end{enumerate}
    To check that the naturality square is a $T$-pullback for \emph{every} map in $\wone$, we once again use Leung's characterization of maps in $\wone$ from Proposition \ref{thm:leung}. Inductively, the set of maps generated by $\{p,0,+,\ell,c\}$ closed under $\ox$ and $\o$ follows as $T$-pullback squares are closed to composition. For maps induced by a tranverse limit in $\wone$, $F$ preserves transverse limits so this follows by the commutativity of limits.
\end{proof}

\begin{theorem}\label{thm:iso-of-cats-inv-emcs}
    For any tangent category $\C$, the replete image of the Weil nerve functor
    \[
        \mathsf{Inv}(\C) \hookrightarrow [\wone, \C]  
    \]
    is precisely the category of $T$-cartesian Weil complexes.
\end{theorem}
\begin{corollary}\label{cor:the-prolongation-description}
    That $\alpha:A.T \Rightarrow T.A$ is $T$-cartesian is equivalent to requiring that the tangent functor
    \[
        (A,\alpha): \wone \to \C  
    \]
    restricts to an anchored bundle
    \[(\pi: A.T \xrightarrow[]{A.p} A.\N,\;\ \xi: A \xrightarrow[]{A.0} A.T,\;\ \lambda: A.T \xrightarrow[]{A.\ell} A.TT \xrightarrow[]{\alpha.T}T.A.T,\;\ \anc:A.T \xrightarrow[]{\alpha} T.A)\]
    and each $A.T^V$ is the $V$-prolongation of this anchor bundle.
\end{corollary}
\begin{remark}
    The condition in Corollary  \ref{cor:the-prolongation-description} is analogous to the Segal conditions identifying those simplicial complexes
    \[\Delta \to \C \]
    that are internal categories. Note that every simplicial object has an underlying reflexive graph
    \[
        \mathsf{tr}_1(X) := (s,t:X([1]) \to X([0]), i:X([0]) \to X([1]))
    \]
    where $X([n])$ is isomorphic to the object of $n$-composable arrows for the underlying reflexive graph.
\end{remark}
\begin{remark}
    Notably, being $T$-cartesian for $p$ is enough to force that a natural transformation is $T$-cartesian for the other tangent-structural natural transformations. This has consequences when one uses partial maps to combine \emph{topological} notions with tangent categories.
    In this context, a partial map $N \to X$ with domain $M \hookrightarrow N$  is a span
    \[\input{TikzDrawings/Ch4/Sec4/span-remark.tikz}\]
    whose right leg is monic. The intuition is that the map $f$ is defined on the subobject $M$ of $N$, which introduces a new problem: what is the proper notion of a \emph{subobject} in a tangent category?
    Such a notion should give rise to a \emph{stable class of monics}: one that is closed under horizontal span composition.
    One answer is the notion of etale monics: a morphism is \emph{etale} whenever the naturality square for $p$ is a $T$-pullbacks:
    \[\input{TikzDrawings/Ch4/Sec4/etale-cond.tikz}\]
    Geometrically, this means that the morphism is a local diffeomorphism; for example, an etale subobject of $\R^n$ in the Dubuc topos is precisely an open subset in the usual sense.
    An endofunctor lifts to the partial map category whenever it preserves the class of monics. A natural transformation lifts to endofunctors on the partial map category whenever it is $T$-cartesian for the class of monics, and the same proof will show that this property holds for etale monics \cite{Cruttwell2019}. 
\end{remark}
  


\section{The prolongation tangent structure}
\label{sec:prol_tang_struct}

One of the most important consequences of the Weil Nerve Theorem  \ref{thm:weil-nerve} is that the category of involution algebroids (with chosen prolongations) may be equipped with two tangent structures. The first tangent structure is the pointwise tangent structure described in Proposition \ref{prop:pointwise-tangent-structure-inv}.
%TODO add reference to the tangent structure on involution algebroids in chapter 3
The tangent functor sends
\[
    (A,\alpha) \mapsto (T.A:\wone \to \C, c.A \o T.\alpha: T.A.T \Rightarrow T.T.A)  
\]
(recall the composition of tangent functors given in Example \ref{ex:composition-of-tangent-functors} (ii)). The structure morphisms will be given by whiskering, so in this case $\theta.A, \theta \in \{ p,0,+,\ell,c\}$. The restriction to tangent functors that preserve transverse limits along with the fact that the natural part $\alpha$ is $T$-cartesian, however, ensures that precomposition with the tangent functor
\[
    (A,\alpha) \mapsto (A.T: \wone \to \C, T.\alpha \o A.c: A.T.T \Rightarrow T.A.T)  
\]
returns an involution algebroid. The structure maps are once again given by whiskering, with the pre-composition tangent structure $A.\theta, \theta \in \{ p,0,+,\ell,c\}$. Preservation of transverse limits guarantees that this tangent structure will satisfy the necessary universality conditions. 

\begin{proposition}[Proposition \ref{prop:second-tangent-structure-inv-algds}]
\label{prop:second-tangent-structure-inv-algds-2}
    The category of involution algebroids with chosen prolongations in a tangent category $\C$ has a second tangent structure, where the action by $\wone$ is given by
    \[
        (A,\alpha) \mapsto ( A.T: \wone \to \C, \alpha.T \o \hat A.c: \hat A.T.T \Rightarrow T.\hat A.T).  
    \]
\end{proposition}
\begin{proof}
    The proposition statement means that the structure morphisms for this new involution algebroid are given by
    \[
        ( A.T, \alpha.T \o  A.c) \cong
        \begin{cases}
            \alpha.T \o  A.c = \anc':& \prol(A) \xrightarrow[]{\pi_1} TA \\
             \quad \;\; A.T.p = \pi':& \prol(A) \xrightarrow[]{p \o \pi_1} A \\
             \quad \;\;\; A.T.0 = \xi':& A \xrightarrow[]{(\xi \o \pi, 0)} \prol(A) \\
            \anc' \o  A.T.\ell = \lambda':& \prol(A) \xrightarrow[]{\lambda \x \ell} T.\prol(A) \\
             \quad \;\ A.T.c = \sigma':& \prol^2(A) \xrightarrow{\sigma \x c} \prol^2(A)
        \end{cases}  
    \]
    Similarly, we can see that
    \begin{gather*}
        ( A.T.T, \alpha.T.T \o  A.c.T \o  A.T.c) \\ = 
        \begin{cases}
            \alpha.T.T \o  A.c.T \o  A.T.c = \anc'':& \prol^2(A) \xrightarrow[]{(\pi_1, \pi_2)} T.\prol(A) \\
             \hspace{2.5cm} A.T.p = \pi'':& \prol^2(A) \xrightarrow[]{(p \o \pi_1, p \o \pi_2):} \prol(A) \\
             \hspace{2.6cm} A.T.0 = \xi'':& \prol(A) \xrightarrow[]{(\xi \o \pi \o \pi_0, 0 \o \pi_1, 0 \o \pi_2)} \prol^2(A) \\
            \hspace{1.85cm} \anc''\o  A.T.\ell = \lambda'':&\prol^2(A) \xrightarrow[]{(\lambda \x \ell \x \ell)} T.\prol^2(A) \\
             \hspace{2.15cm} A.T.T.c = \sigma'':& \prol^3(A) \xrightarrow[]{(\sigma \x c \x c)}\prol^3(A) 
        \end{cases}  
    \end{gather*}
    These coincide with the involution algebroids $\prol'(A), \prol'.\prol'(A)$ in Proposition \ref{prop:second-tangent-structure-inv-algds}: the second tangent structure follows from the fact that the natural transformations for the tangent structure there are given by
    \[
        A.\phi: A.U \Rightarrow A.V, \phi:U \to V \in \{p,0,+, \ell, c\}.  
    \]
    The result follows as a corollary of Theorem \ref{thm:weil-nerve}.
\end{proof}



% This section exposits the second tangent structure on the category of involution algebroids (with chosen prolongations) in some tangent category $\C$. This tangent structure is a natural consequence of the Weil Nerve from \Cref*{thm:weil-nerve}, but is worth giving a more concrete presentation. The Jacobi identity on the sections of an involution algebroid follows as a consequence of this tangent structure.

% The category of cartesian tangent functors $[\wone, \C]$ into any tangent category $\C$ has two actions by $\wone$. The first is post-composition of the tangent functor in $\C$:
% \[
%     \mathsf{CartTang}[\wone, \C] \x \wone \to \mathsf{CartTang}[\wone, \C] ;   \hspace{0.15cm} ((A,\alpha), V) \mapsto (T^V.A, T^V.\alpha)  
% \]
% sends transverse limits in $\wone$ to limits in the functor category $\mathsf{CartTang}(\wone, \C)$ - this tangent structure was described in \Cref{obs:inv-tmonad-anc}. %TODO add the reference
% However, the restriction to transverse-limit-preserving cartesian tangent functors also guarantees that the \emph{pre-composition} action:
% \[
%     \mathsf{CartTang}[\wone, \C] \x \wone \to \mathsf{CartTang}[\wone, \C] ;   \hspace{0.15cm} ((A,\alpha), V) \mapsto (A.T^V, \alpha.T^V)  
% \]
% sends all the transverse limits in $\wone$ to tangent limits in the tangent category $\C$. This will induce a second tangent structure on the category, where each of the structure maps are tangent natural transformations with respect to the pointwise tangent structure.


% One of the original sticking points in the development of abstract tangent structure is quite surprising:  the proof of the Jacobi identity for the Lie bracket of vector fields (see section 3.4 of \cite{Cockett2014} for the history). The eventual proof made use of a string calculus, and a collection of identities \cite{Cockett2015}.  It should come as no surprise, then, that showing the bracket on the sections of an involution algebroid satisfies the Jacobi identity is complex. This section describes the second tangent structure on the category of    involution algebroids, corresponding to the prolongation functor.
% When a tangent category has negatives, there is a natural bracket associated with the abelian group of vector fields on a given $M$, discussed in Lemma  \ref{lem:intertwining-induces-bracket}. Proving that the Jacobi identity holds for this bracket - a reasonably trivial result in the category of smooth manifolds - proves to be a significant challenge (see \cite{Cockett2014} for the history). The eventual proof of this result in \cite{Cockett2015} developed a string calculus to facilitate calculations.

% It should not be surprising, then, that proving the Jacobi identity holds for an involution algebroid also proves to be a challenge. Using the new characterization of involution algebroids as a functor category $[\wone, \C]$, Cockett and Cruttwell's result may be applied directly to an involution algebroid.

% \begin{definition}\label{def:two-tang-on-inv}
%     Let $\C$ be a tangent category. The category of complete involution algebroids in $\C$ has two tangent structures:
%     \begin{enumerate}[(i)]
%         \item Post-composition: This is given by $T_{post}(A)(V) = T.A(V)$ - this is, the cartesian Weil complex is post-composed by $T$. We write this simply as $T$. This tangent structure was expounded on in %TODO look up tangent structure for involution algebroids in chapter three
%         \item Pre-composition: Pre-composition with the tangent functor preserves Cartesian models of $\wone$. This tangent functor will be written as $\prol$. 
%     \end{enumerate}
% \end{definition}
%IDEA: make this more concrete by building the second tangent structure on the category of involution algebroids, where the functor is a tangent functor each of the structure maps will be a tangent natural transformation. This gives a tangent category in the category of tangent categories, e.g. the action by $\wone$ is itself a tangent functor.
% The second tangent structure on involution algebroids can be confusing (proving that the prolongation operation is a tangent bundle on the category of involution algebroids is equivalent to the Weil Nerve Theorem  \ref{thm:weil-nerve}). It is instructive to have a concrete description of $\prol.\hat A, \prol.\prol.\hat A$. 
% \begin{example}%
%     \label{ex:prol-inv-algds}
%     Let $A = (\pi:A \to M, \xi, \lambda, \anc, \sigma)$ be an involution algebroid with chosen prolongations in a tangent category $\C$ (note that this induces an addition map $+_q$). It is important to realize here that the anchored bundle comes with a choice of prolongations - this gives a coherent choice for the prolongations of the anchored bundles (see -).%TODO add reference to the exact tangent structure.
    
%     % The anchored bundle $\prol \hat{A}$ is defined to be:
%     % \begin{gather}
%     %     p \o \pi_1: \prol(A) \to A, (\xi\o \pi, 0):A \to \prol(A), \lambda \x \ell: \prol(A) \to T.\prol(A), \pi_1: \prol(A) \to TA \\
%     %     % p \o (\pi_1,\pi_2): \prol^2(A) \to \prol(A), (\xi\o\pi \o \pi_1, 0 \o \pi_1, 0 \o \pi_2): \prol(A) \to \prol^2(A), \lambda \x \ell \x \ell: \prol^2(A) \to T.\prol^2(A), (\pi_1,\pi_2): \prol^2(A) \to T.\prol(A)
%     % \end{gather}%TODO format
%     Recall that by -, pullback of anchored bundles are computed as pullbacks of the underlying differential bundle, and the anchor is induced universality. %TODO cross reference limits of anchored bundles
%     The structure maps are then given by the following diagram:
%     \[\input{TikzDrawings/Ch4/dbun-pullback-abun.tikz}\]
%     It can be show that the prolongation of $\prol \hat{A}$ is $\prol^2(A)$ using the following diagram:
%     \[\input{TikzDrawings/Ch4/prol-2-pp.tikz}\]
%     The diagram commutes as:
%     \begin{gather*}
%         \anc \o g_1 = T.\pi \o g_2 = T.\pi \o T.p \o f_1 = T.p \o T^2.\pi \o f_1 \\
%         = T.p \o T.\anc \o f_2 = T.\pi \o f_2
%     \end{gather*}
%     This gives the induced involution:
%     \[\input{TikzDrawings/Ch4/induced-involution.tikz}\]
%     as state in -. %TODO fill in cross reference.
%     This gives the pullback involution algebroid:
%     \begin{gather*}
%         p \o \pi_1: \prol(A) \to A, (\xi\o \pi, 0):A \to \prol(A), \lambda \x \ell: \prol(A) \to T.\prol(A), \\
%          \pi_1: \prol(A) \to TA, \sigma \x c: \prol^2(A) \to \prol^2(A)
%         % p \o (\pi_1,\pi_2): \prol^2(A) \to \prol(A), (\xi\o\pi \o \pi_1, 0 \o \pi_1, 0 \o \pi_2): \prol(A) \to \prol^2(A), \lambda \x \ell \x \ell: \prol^2(A) \to T.\prol^2(A), (\pi_1,\pi_2): \prol^2(A) \to T.\prol(A)
%     \end{gather*}%TODO format
%     The involution algebroid axioms all follow by universality, but we check the axioms on this involution map (Definition \ref{def:involution-algd})
%     \begin{itemize}
%         \item The two differential bundle structures on $\prol^2(A)$ are then $\lambda \x \ell \x \ell.T, 0 \x c \o T.\lambda \x T.\ell$, so bilinearity follows.
%         \item The involution axiom is straightforward to check:
%         \[( \sigma \x c )\o (\sigma \x c) = (\sigma \o \sigma) \x (c \o c)  = id\]
%         \item The target axiom follows:
%         \[
%             \pi_1 \o (\pi_1,\pi_2) \o (\sigma \x c) = c \o \pi_2
%         \]
%         \item The unit axiom follows by:  
%         \[
%             (\sigma \x c) \o (\xi\o \pi \o x, \lambda x, \ell \o y) = 
%             (\sigma \o (\xi \o \pi \o x, \lambda x), c \o \ell \o y) =   (\xi\o \pi \o x, \lambda x, \ell \o y)
%         \]          
%         \item In this case, the Yang-Baxter equation becomes:
%         \begin{gather*}
%             (\sigma \x c \x c) \o (A \x T.\sigma \x T.c) \o (\sigma \x c \x c) \\
%             = 
%             (A \x T.\sigma \x T.c) \o (\sigma \x c \x c)\o (A \x T.\sigma \x T.c)
%         \end{gather*}
%         And it follows by parallel application of the Yang-Baxter equation:
%         \begin{itemize}
%             \item $ (\sigma \x c ) \o (A \x T.\sigma) \o (\sigma \x c) 
%             = 
%             (A \x T.\sigma ) \o (\sigma \x c) \o (A \x T.\sigma)$
%             \item $c.T \o T.c \o c.T = T.c \o c.T \o T.c$
%         \end{itemize}            
%     \end{itemize}
%     Applying an identical argument to this anchored bundle show that the involution algebroid $\prol^2 \hat{A}$ is given by the tuple:
%     \begin{gather*}
%         \pi \x p \x p: \prol^2(A) \to M \ts{id}{\pi}\prolong, \hspace{0.15cm}
%         \xi \x 0 \x 0: M \ts{id}{\pi} \prolong \to \prol^2(A), \\
%         \lambda \x \ell: \prolong \to T.(\prolong), \hspace{0.15cm}
%         (\pi_1,\pi_2): \prol^2(A) \to T.\prol(A) \\
%         \sigma \x c \x c: \prol^3(A) \to \prol^3(A)
%     \end{gather*}


%     % Similarly, the anchored bundle $\prol.\prol \hat{A}$ is induced by the pullback of anchored bundles:

%     % Giving it the structure maps:
%     % %TODO put in the itemized list

%     % The chosen prolongations for each $\prol(A), \prol^2(A)$ is given by:
%     % \begin{gather*}
%     %     \prol(U,\prol.\hat A) = \prol(UW,A) \\
%     %     \prol(U,\prol.\prol. \hat A) = \prol(UWW,A)
%     % \end{gather*}
%     % Where $\prol(U,-)$ refers the total object of the differential bundle.
%     % The induced involution maps are given by:
%     % %TODO insert span diagrams

%     % Then Yang-Baxter equation for the prolongation anchored bundles follows by parallel applications of the basic Yang-Baxter equation for $\hat{A}$. 
%     %TODO insert diagrams
%     % \begin{enumerate}
%     %     \item     The ``prolongation'' Lie algebroid, $\prol(A)$, is given by $f \boxtimes W$, for $f \in \{ \pi,\xi,+_q,\hat \lambda, \sigma\}$. It has been discussed in different pieces throughout \Cref*{ch:involution-algebroids} - the anchored bundle structure in -, the tangent involution algebroid structure in -, and then pullback involution algebroid structure in -. %TODO insert references
%     %     Here it is all pulled together and clarified based on insights provided by the Weil nerve construction for convenience. 
%     %     \begin{itemize}
%     %         \item The projection map is given by:
%     %         \[ \infer{p \o \pi_1: \prol(A) \to A}{\pi \x p: \prolong \to M \ts{id}{\pi} A}\]
%     %         Note that this means the $n$-fold pullback powers of the projection are given by $A_n \ts{\anc_n}{T_n.\pi} T_nA$.
%     %         \item The zero map is given by:
%     %         \[
%     %             \infer{(\xi \o \pi, 0): A \to \prol(A)}{\xi \x 0: M \ts{id}{\pi} \to \prolong}
%     %         \]
%     %         \item The addition map is given:
%     %         \[
%     %            ( +_q \x +.A ): A_2 \ts{\anc_2}{T_2.\pi} T_2A \to \prolong
%     %         \]
%     %         \item The (differential bundle) lift map is given:
%     %         \[
%     %             (\lambda \x \ell):\prolong \to T.(\prolong)
%     %         \]
%     %         \item The anchor map is given by:
%     %         \[
%     %             \infer{\pi_1: \prol(A) \to TA}{\anc \x id: \prolong \to TA}  
%     %         \]
%     %         \item The involution map is given by:
%     %         \[
%     %             \sigma \x c: \prolong \ts{T.\anc}{T^2.\pi} T^2A \to \prolong \ts{T.\anc}{T^2.\pi} T^2A 
%     %         \]
%     %     \end{itemize}
%     %     The involution proves to be the most difficult part of this proof. This involution algebroid is induced by the pullbac of the cospan of involution algeboids:
%     %     \[
%     %         A \xrightarrow[]{\anc} TM \xleftarrow[]{T.\pi} TA
%     %     \]
%     %     where $A$ is the involution algebroid, and $TM, TA$ are the tangent involution algebroids on $M, A$. This induces an involution map:
%     %     \[
%     %         (\prolong) \ts{\pi_1}{T.p \o \pi_1} (\prolong)
%     %         \to (\prolong) \ts{\pi_1}{T.p \o \pi_1} (\prolong)
%     %     \]
%     %     Upon identifying 
%     %     \[
%     %         (\prolong) \ts{\pi_1}{T.p \o \pi_1} (\prolong) \cong  \prolong \ts{T.\anc}{T^2.\pi} T^2A
%     %     \]
%     %     Then one can instead check that the involution $\sigma \x c$ satisfies the universal property of the induced map 
%     %     \[
%     %         \prol^2(A) \to \prol^2(A)  
%     %     \]

%     %     \item The second prolongation's involution algebroid structure follows by the same argument. \dots
%     % \end{enumerate}
% \end{example}
% %TODO this paragraph is messy as hell
% Using the Weil nerve (Theorem  \ref{thm:weil-nerve}), involution algebroids have a natural tangent structure given by pre-composing an involution algebroid $\hat A: \wone \to \C$ with the tangent functor in $\wone$ - that is, sending $\hat A \mapsto \hat A . T$. 
% The Weil nerve statement, then, identifies the strict tangent functors:
% \[
%     \wone \to \mathsf{Inv}^*(\C)  
% \]
% and identifies morphisms with tangent natural transformations. It is worth giving a concrete description of each of the tangent structure morphisms.

% \begin{lemma}
%     The prolongation and the anchor $(\prol, anc)$ is a tangent endofunctor on the category of involution algebroids.
% \end{lemma}
% \begin{proof}
%     The assignment on objects was demonstrated in Example  \ref{ex:prol-inv-algds}. On morphisms, note that an involution algebroid morphism $(f,u):\hat{A} \to \hat{B}$ gives a linear morphism:
%     \[
%         f \x T.f: (A \ts{\anc}{T.\pi} TB, \lambda \x \ell) \to (B \ts{\anc}{T.\pi} TB, \lambda \x \ell)   
%     \]
%     as $f$ is linear for the $\lambda$ part and $T.f$ is linear for the $\ell$ part. The base part of this map is $f$, and the anchor map is preserved as the anchor maps are $\pi_1$, so:
%     \[
%         \pi_1 \o (f \x T.f ) = T.f  
%     \]
%     Similarly, the involution is preserved as:
%     \begin{gather*}
%         (f \x T.f \x T^2.f) \o (\sigma \x c) = ((f \x T.f) \o \sigma, T^2.f \o c)  \\= ((\sigma \o (f \x T.f), c \o T.f)) = (\sigma \x c) \o (f \x T.f \x T^2.f)
%     \end{gather*}
%     Thus the map $(f,v)$ maps to $(f \x T.f, f)$, giving functoriality (composition is straightforward to check).

%     To see that this is a tangent functor with respect to the pointwise tangent structure on involution algebroids, first observe that .%TODO finish this proof
% \end{proof}
% Now given the functor part of the tangent structure, check each of the structure maps:
% \begin{lemma}
%     Regard the category of involution algebroids with chosen prolongations in $\C$ as a tangent category, using the pointwise tangent structure. 
%     Then there are tangent natural transformations:
%     \begin{enumerate}[(i)]
%         \item Projection:
%         \[
%             \hat p : (\prol,\anc) \Rightarrow id  
%         \]
%         where $\hat p = (\pi_1, \pi): \prol \hat{A} \to A$.  Note that the pullback powers of $\hat p$ has the underlying differential bundle:
%         \begin{gather*}
%             A \ts{\anc}{T.\pi \o T.\pi_i} T.A_n \xrightarrow[]{p \o \pi_1} A_n, \hspace{0.15cm}
%             A_N \xrightarrow[]{(\xi \o \pi \o \pi_i, 0)} A \ts{\anc}{T.\pi \o T.\pi_i} T.A_n \\
%             A \ts{\anc}{T.\pi \o T.\pi_i} T.A_n \xrightarrow[]{\lambda \x \ell.A_n}  T.(A \ts{\anc}{T.\pi \o T.\pi_i} T.A_n)
%         \end{gather*}
%         \item Addition:
%         \[
%             \hat +: (\prol_2, \anc_2) \Rightarrow (\prol, \anc); \hspace{0.15cm}
%             A \ts{\anc}{T.\pi \o T.\pi_i} T.A_2 \xrightarrow[]{id \x T.+_q} \prolong
%         \]
%         \item Zero:
%         \[
%             \hat 0: id \Rightarrow (\prol,\anc); \hspace{0.15cm}
%             \mathbf{A} \xrightarrow[]{(id, T.(\xi \o \pi))} \mathbf{A}
%         \]
%         \item Lift: 
%         \[
%             \hat \ell: (\prol, \anc) \Rightarrow (\prol, \anc)^2; \hspace{0.15cm}
%             {\hat \ell: \prol.A \xrightarrow[]{id \x (T.\xi \o T.\pi, T.\lambda)} \prol.\prol(A)}
%         \]
%         \item Flip:
%         \[
%             \hat c: (\prol,\anc)^2 \Rightarrow (\prol, \anc)^2; \hspace{0.15cm}
%             \prol^2(A) \xrightarrow[]{(1 \x T.\sigma)} \prol^2(A)
%         \]
%     \end{enumerate}
% \end{lemma}
% \begin{proof}
%     ~\begin{enumerate}[(i)]
%         \item 
%     \end{enumerate}
% \end{proof}

% \begin{observation}
%     Relating this to the $\boxtimes$ operation (Definition \ref{def:boxtimes-span}), the involution algebroid maps for $\prol(V,A)$ are given by $\theta \boxtimes \prol(V,A)$, where as the structure maps for the tangent category of involution algebroids is given by $\prol(V,A) \boxtimes \theta$.
% \end{observation}

% \begin{theorem}
%     There is a tangent structure on the category of involution algebroids with chosen prolongations in $\C$, given by
%     \[
%         (\prol, \hat p, \hat +, \hat 0, \hat \ell, \hat c)  
%     \]
% \end{theorem}





% \begin{theorem}%
%     \label{thm:second-tangent-structure}
%     For any tangent category $\C$, the prolongation endofunctor determines a tangent structure on the category of involution algebroids with chosen prolongations in $\C$. The tangent natural transformations are given by:
%     \begin{itemize}
%         \item The projection $p$ is given by:
%         \[
%             \hat p: 
%             \prol.\mathbf{A} \xrightarrow[]{\pi_0} \mathbf{A}   
%         \]
%         Note that the pullback powers of $\hat p$ has the underlying differential bundle:
%         \begin{gather*}
%             A \ts{\anc}{T.\pi \o T.\pi_i} T.A_n \xrightarrow[]{p \o \pi_1} A_n, \hspace{0.15cm}
%             A_N \xrightarrow[]{(\xi \o \pi \o \pi_i, 0)} A \ts{\anc}{T.\pi \o T.\pi_i} T.A_n \\
%             A \ts{\anc}{T.\pi \o T.\pi_i} T.A_n \xrightarrow[]{\lambda \x \ell.A_n}  T.(A \ts{\anc}{T.\pi \o T.\pi_i} T.A_n)
%         \end{gather*}
%         \item The zero map is givem by:
%         \[
%             \hat 0: 
%             \mathbf{A} \xrightarrow[]{(id, T.(\xi \o \pi))} \mathbf{A}              
%         \]
%         \item The addition map is given by:
%         \[
%             \hat +:A \ts{\anc}{T.\pi \o T.\pi_i} T.A_2 \xrightarrow[]{id \x T.+_q} \prolong
%         \]
%         \item The lift map is given by:
%         \[
%             \infer{A \ts{\anc}{id} TM \ts{id}{T.\pi} TA \xrightarrow[]{id \x T.\xi \x T.\lambda} \prolong \ts{T.\anc}{T^2.\pi} T^2A}
%             {\hat \ell: \prol.A \xrightarrow[]{id \x (T.\xi \o T.\pi, T.\lambda)} \prol.\prol(A)}
%         \]
%         \item The involution map is given by:
%         \[
%             \hat c: \prolong \ts{T.\anc}{T^2.\pi} T^2A
%             \xrightarrow[]{A \x T.\sigma}
%             \prolong \ts{T.\anc}{T^2.\pi} T^2A
%         \]
%     \end{itemize}
% \end{theorem}
% \begin{proof}
    
% \end{proof}
% The ``tangent involution'' algebroid of 
% \[\bar{A} = (\pi:A \to M, \xi, \lambda, \anc, \sigma)\] 
% is the prolongation involution algeboid 
% \begin{gather*}
%     (\pi\x p): \prol(A) \to M \ts{id}{\pi} A, \hspace{0.15cm}
%     (\xi \x 0): TM \ts{id}{\pi} A \to \prol(A), \hspace{0.15cm}
%     (\lambda \x \ell):\prol(A) \to T.\prol(A), \\
%     \pi_1: \prol(A) \to T.A,\hspace{0.15cm}
%     \sigma \x c: \prol^2(A) \to \prol^2(A)
% \end{gather*}
% One of the main sticking point in providing a direct proof that 
% \[\prol: \mathsf{Inv}(\C) \to \mathsf{Inv}(\C)\] 
% is a tangent structure is showing coherences like:
% \[
%     \prol(\prol(\bar{A})) = \prol^2(A)  
% \] and keeping the isomorphism maps straight as one checks the involution algebroid axioms. The coherence theorem Theorem  \ref{thm:weil-nerve} allows us to sidestep this difficulty and describe the involution $\prol(A)$ directly as:
% \begin{gather*}
%     \sigma \x c: \prol^2(A) \to \prol^2(A)
% \end{gather*}
% In this case, the Yang-Baxter equation becomes:
% \[
%     (\sigma \x c \x c) \o (A \x T.\sigma \x T.c) \o (\sigma \x c \x c) 
%     = 
%     (A \x T.\sigma \x T.c) \o (\sigma \x c \x c)\o (A \x T.\sigma \x T.c)
% \]
% And it follows by parallel application of the Yang-Baxter equation:
% \begin{itemize}
%     \item $ (\sigma \x c ) \o (A \x T.\sigma) \o (\sigma \x c) 
%     = 
%     (A \x T.\sigma ) \o (\sigma \x c) \o (A \x T.\sigma)$
%     \item $c.T \o T.c \o c.T = T.c \o c.T \o T.c$
% \end{itemize}
% The ``tangent differential bundle'' structure maps are then the second differential bundle structure on $\prol(A)$ from Proposition  \ref{prop:anc-prol-fun}
% \[
%     \pi_0: \prol(A) \to A    
% \]
% The zero map is given by:
% \[
%     id \x T.\xi \o \anc : A \ts{\anc}{id} TM \to \prol(A)  
% \]
% The addition map by:
% \[
%     (+_q \x +.A): A_2 \ts{\anc_2}{T_2.(\pi)} T_2A \to \prolong
% \]
% The second-prolongation involution algebroid may is:
% \begin{gather*}
%     (\pi \x p): \prol^2(A) \to M \ts{id}{\pi} \prol(A),\hspace{0.15cm}
%     (\xi \x 0): M \ts{id}{\pi} \prol(A) \to \prol^2(A),\hspace{0.15cm}
%     (\lambda \x \ell): \prol(A) \to T.\prol(A), \\
%     (\pi_1,\pi_2): \prol^2(A) \to T.\prol(A) \hspace{0.15cm}
%     (\sigma \x c \x c): \prol^3(A) \to \prol^3(A)
% \end{gather*}

% The lift map, then, is given by:
% \[
%     (id \x T.\xi \x T.\lambda): A \ts{\anc}{id} TM \ts{id}{T.\pi} TA \to \prolong \ts{T.\anc}{T^2.\pi} T^2A  
% \]
% and the involution is given by:
% \[
%     (id \x T.\sigma):
%     \prolong \ts{T.\anc}{T^2.\pi} T^2A \to \prolong \ts{T.\anc}{T^2.\pi} T^2A 
% \]
% Note that in this case, the Yang-Baxter equation reduces to:
% \begin{gather*}
%     (A \x T.\sigma \x T.c \x T.c) \o (A \x TA \x T^2.\sigma \x T^2.c) \o (A \x T.\sigma \x T.c \x T.c)
%     \\ = 
%     (A \x TA \x T^2.\sigma \x T^2.c) \o (A \x T.\sigma \x T.c \x T.c) \o (A \x TA \x T^2.\sigma \x T^2.c)
% \end{gather*}
% which, once again, boils down parallel applications of the Yang-Baxter equation:
% \begin{itemize}
%     \item $id \o id \o id = id = id \o id \o id$,
%     \item $T.((\sigma \x c) \o (1 \x T.\sigma) \o (\sigma \x c))  =  T.((1 \x T.\sigma) \o (\sigma \x c)\o (1 \x T.\sigma))$
%     \item $T.(c.T \o T.c \o c.T) = T.(T.c \o c.T \o T.c)$
% \end{itemize}
% Relating this back to the tangent structure on $\wone^\anc$ induced by an involution algebroid - the structure maps for the anchored bundle are given by tensoring by the identity on the \emph{right}, while the structure maps for the tangent structure on the category of involution algebroids is given by applying the tensor product to the \emph{left}.



\subsection*{The Jacobi identity for involution algebroids}
Classically, the theory of Lie algebroids uses the algebra of sections $\Gamma(\pi)$.  One key observation is that when using the Lie tangent structure $(\mathsf{Inv}(\C), \prol)$, sections of $\pi$ are in bijective correspondence with $\chi_\prol(A)$.
This observation allows for different statements about Lie algebroids to be translated into formal statements about the tangent bundle in $(\mathsf{Inv}(\C), \prol)$.
\begin{proposition}\label{prop:section-morphism}
    Let $A$ be an involution algebroid in $\C$.
    There is a bijection between the sections of\, $\pi$ in $\C$ and the vector fields on $A$ in $\mathsf{Inv}(\C)$:
    \[
        X \in \Gamma(\pi) \mapsto ((id, TX \o \anc), X): A \to T_L(A);
        \hspace{.2cm}
        \hat{X} \in \chi_\prol(A) \mapsto \hat{X}_R: A.R \to A.T.
    \]
\end{proposition}
\begin{proof}
    Recall the coherence for tangent natural transformations $\gamma: (H,\phi) \Rightarrow (G,\psi)$:
    \begin{equation*}
        \input{TikzDrawings/Ch4/Sec5/tangnat.tikz}
    \end{equation*}
    We specify this to a morphism $\hat{X}: (A, \alpha) \Rightarrow T_L(A,\alpha) = (A.T, \alpha.T)$ at $\N, W$:
    \begin{equation*}
        \input{TikzDrawings/Ch4/Sec5/hat-thing.tikz}
    \end{equation*}
    and so  infer that, if we set $X := X_R$, we have $\pi_1 \o X.T = T(X) \o \anc$.
    Furthermore, the condition that $p_L \o X = id$ forces $id = \pi_0 \o X.T$; thus, we can see that every section $X$ of $p_L$ is given by a morphism of the form $((id, TX \o \anc), X))$ on the underlying involution algebroids, where $\pi \o X = id$. 
    
    We now show that every $X \in \Gamma(\pi)$ gives rise to a section of $\pi_L$.
    Observe that the following is a morphism of involution algebroids:
    \begin{equation*}
        \input{TikzDrawings/Ch4/Sec5/inv-algd-mor.tikz}
    \end{equation*}
    Note that it is well typed, as $T.\pi \o T.X \o \anc = \anc \o id$. Check that it is a bundle morphism:
    \[
        p \o \pi_1 \o (id, T.X \o \anc) = p \o T.X \o \anc = X \o p \o \anc = X \o \pi
    \]
    and that it is linear:
    \begin{gather*}
        (\lambda \o \pi_0, \ell\o\pi_1)\o(id, T.X\o \anc) = (\lambda, \ell \o T.X \o \anc) \\= (\lambda, T^2.X \o \ell \o \anc) = (\lambda, T^2.X \o \lambda) = T(id, TX)\o\lambda. 
    \end{gather*}
    Then check that it preserves the anchor:
    \[
        \pi_1 \o (id, TX \o \anc) = TX \o \anc
    \]
    and the involution:
    \begin{gather*}
        (\pi_0, \pi_1, T^2X \o T\anc \pi_1)\o \sigma = (\sigma, T^2X \o T\anc \o \pi_1 \sigma)\\ 
        = (\sigma, T^2X \o c T\anc\o \pi_1)
        = (\sigma(\pi_0, \pi_1), c\pi_3) \o (id, T^2X \o T\anc \o \pi_1).
    \end{gather*}

    Thus we have that $(id, TX \o \anc \pi_1)$ is a morphism of involution algebroids, inducing a morphism of $T$-cartesian Weil complexes.
    Lastly, we check that it is a section of $p^L_A$, but this is clear, since
    \[
        \pi_0 \o (id, TX \o \anc) = id;
    \]
    thus we have the desired bijection.
\end{proof}

Recall that given an involution on an anchored bundle, there is a bracket on its set of sections (see the explicit construction in Section \ref{sec:connections_on_an_involution_algebroid}). 
Given an $X,Y \in \Gamma(\pi)$, there is a bracket defined as follows:
\[
    \hat{\lambda} \o [X,Y]_A +_1 (\xi\pi,0)\o Y = ((\sigma \o (id, TY \o \anc) \o X -_2 (id, TX \o \anc)\o Y)).
\]
A direct proof of the Jacobi identity is a detailed calculation (see the original preprint on involution algebroids \cite{Burke2019}) and still relies on Cockett and Cruttwell's result for an arbitrary tangent category with negatives. As a result of Proposition  \ref{prop:section-morphism}, we can instead use Cockett and Cruttwell's result directly:
\begin{corollary}\label{cor:lie-bracket}
    Let $A$ be a complete involution algebroid in a tangent category $\C$ with negatives. 
    There is a Lie bracket defined on $\Gamma(\pi)$, $[-,-]$ induced by
    \[
        \hat{\lambda} \o [X,Y]_A +_1 (\xi\pi,0)\o Y = ((\sigma \o (id, TY \o \anc) \o X -_2 (id, TX \o \anc)\o Y)).
    \]
\end{corollary}
\begin{proof}
    The bracket is induced by Rosicky's universality diagram, as
    \begin{align*}
        0&=p \o  ((\sigma \o (id, TY \o \anc) \o X - (id, TX \o \anc)\o Y)) - 0Y \\
        &= Tp \o  ((\sigma \o (id, TY \o \anc) \o X - (id, TX \o \anc)\o Y)) - 0Y. 
    \end{align*}
    We look at the Lie tangent structure for $\mathsf{Inv}^*(A)$; this is precisely the vector field induced by
    \[
        ev_R( [\hat{X}, \hat{Y}] ).
    \]
    We complete the proof by using the result that for any $A$ in a tangent category with negatives, the bracket on $\chi(A)$ that is defined by
    \[
        \ell \o [X,Y] = (c\o T.X \o Y - T.Y \o X) - 0X
    \]
    satisfies the Jacobi identity.
\end{proof}

\subsection*{Identifying categories of involution algebroids}%
\label{sub:identifying-cats-of-inv}

Section \ref{sec:identifying-involution-algebroids} identified whenever a functor $\wone \to \C$ is an involution algebroid, whereas this section identifies tangent categories $\C$ that embed into the category of involution algebroids in some tangent category $\C$. We call this structure an \emph{abstract category} of involution algebroids. This notion involves some 2-category theory, using a modified notion of \emph{codescent} (see \cite{Bourke2010} for a development of codescent).

Recall that for any tangent category $\C$, the category of involution algebroids has $\C$ as a reflective subcategory. Furthermore, because limits of involution algebroids are computed pointwise, this reflector is left-exact. This left-exact reflection is the main structure we axiomatize.
\begin{definition}
    An \emph{abstract category of involution algebroids} is a tangent category $\C$ with a left-exact $T$-cartesian tangent localization $(Z,\anc): \C \to \D$, where $L$ satisfies a \emph{codescent} condition:
    \[
        \mathsf{TangCat_{Strict}}(\wone, \C)  \hookrightarrow \mathsf{TangCat_{Lax}}(\wone, \C) \xrightarrow{L_*} \mathsf{TangCat_{Lax}}(\wone, \D)
    \]
    (where $L_*$ denotes post-composition by $L$) is fully faithful. 
\end{definition}
\begin{example}
    The category of involution algebroids in any tangent category $\C$ is an abstract category of involution algebroids using $(\mathsf{Inv}(C), \prol)$. The reflector is the functor sending an involution algebroid to its base space; the $T$-cartesian natural transformation is the anchor map. Any tangent subcategory of $\mathsf{Inv}(\C)$ that contains $\C$ as a full subcategory will give rise to an abstract category of involution algebroids.
\end{example}

\begin{proposition}
    Let $\Z \hookrightarrow \C$ be an abstract category of involution algebroids.
    Then there is an embedding $\C \hookrightarrow \mathsf{Inv}(\Z)$.
\end{proposition}
\begin{proof}
    The proof follows by treating objects in $\C$ as strict tangent functors $\wone \to \C$ and morphisms as tangent natural transformations.
    \[\input{TikzDrawings/Ch4/Sec5/coherence-thing.tikz}\]
    The natural part of $Z$ is $T$-cartesian, and the functor part preserves limits, so $Z.\prol(-,A) =: Z[A]$ determines an involution algebroid in $\Z$, and $f$ a morphism of involution algebroids.
    The embedding is guaranteed by the codescent condition so that the post-composition functor is fully faithful.
\end{proof}
\begin{corollary}
    An abstract category of involution algebroids $\Z \hookrightarrow \C$ is exactly a full subcategory $\Z \hookrightarrow \C \hookrightarrow \mathsf{Inv}(\Z)$.
\end{corollary}







\chapter{The infinitesimal nerve and its realization}%
\label{ch:inf-nerve-and-realization}

The main thrust of Chapters \ref{ch:differential_bundles}, \ref{ch:involution-algebroids}, and \ref{chap:weil-nerve} has been that the tangent categories framework allows for Lie algebroids to be regarded as tangent functors
\[
    \wone \to \mathsf{SMan} 
\] which satisfy certain universality conditions. 
This chapter, which is more experimental than the previous four chapters and represents work still in progress, puts Lie algebroids into the framework of \emph{enriched functorial semantics}. This new perspective on algebroids uses Garner's enriched perspective on tangent categories (\cite{Garner2018}) and the enriched theories paradigm from \cite{Bourke2019}. The functorial-semantics presentation of the Lie functor will generalize the Cartan--Lie theorem (that the category of Lie algebras is a coreflective subcategory of Lie groups) into a statement within the general theory of functorial semantics. 

The goal is to show that the infinitesimal approximation of a groupoid, as discussed in Example \ref{ex:lie-algebroids}, may be constructed as a \emph{nerve}, just like Kan's original simplical approximation of a topological space. The nerve of a functor $K:\a \to \C$ approximates objects and morphisms in $\C$ by $\a$-presheaves, so it sends an object in $\C$ to the $\a$-presheaf
\[
    N_K: \a \to \C; C \mapsto \C(K-, C). 
\]
Thus there will be an infinitesimal object,
\[
    \partial: \wone^{op} \to \mathsf{Gpd}(\w)   
\] where $\mathsf{Gpd}(\w)$ denotes groupoids in the category $\W$ of Weil spaces (formally defined in Section \ref{sec:tang-cats-enrichment}). The nerve of $\partial$ has a left adjoint, the \emph{Lie realization}, given by the left Kan extension (just as Kan's geometric approximation of a simplicial set is, in \cite{Kan1958}):
\begin{equation}\label{eq:lan-lie}
    % https://q.uiver.app/?q=WzAsMyxbMCwwLCJcXHdvbmVee29wfSJdLFswLDEsIlxcd2lkZWhhdHtcXHdvbmVee29wfX0iXSxbMSwwLCJcXG1hdGhzZntHcGR9KFxcdykiXSxbMCwxLCJcXHlvbiIsMl0sWzAsMiwiXFxwYXJ0aWFsIl0sWzEsMiwiTGFuX1xceW9uXFxwYXJ0aWFsIiwyLHsic3R5bGUiOnsiYm9keSI6eyJuYW1lIjoiZGFzaGVkIn19fV1d
    \begin{tikzcd}
        {\wone^{op}} & {\mathsf{Gpd}(\w)} \\
        {[\wone,\w]}
        \arrow["\yon"', from=1-1, to=2-1]
        \arrow["\partial", from=1-1, to=1-2]
        \arrow["{Lan_\yon\partial}"', dashed, from=2-1, to=1-2]
    \end{tikzcd}
\end{equation}

\begin{restatable*}[The Lie Realization]{theorem}{lie}
    \label{thm:lie-realization}
    There is a tangent adjunction between the category of involution algebroids and groupoids in $\w$, where each functor preserves products and the base spaces.
\end{restatable*}
\[% https://q.uiver.app/?q=WzAsMixbMCwwLCJcXG1hdGhzZntHcGR9KFxcQykiXSxbMSwwLCJcXG1hdGhzZntJbnZ9KFxcQykiXSxbMCwxLCJOX1xccGFydGlhbCIsMCx7ImN1cnZlIjotMn1dLFsxLDAsInwtfF9cXHBhcnRpYWwiLDAseyJjdXJ2ZSI6LTJ9XSxbMiwzLCIiLDAseyJsZXZlbCI6MSwic3R5bGUiOnsibmFtZSI6ImFkanVuY3Rpb24ifX1dXQ==
\begin{tikzcd}
    {\mathsf{Gpd}(\w)} & {\mathsf{Inv}(\w)}
    \arrow[""{name=0, anchor=center, inner sep=0}, "{N_\partial}", curve={height=-12pt}, from=1-1, to=1-2]
    \arrow[""{name=1, anchor=center, inner sep=0}, "{|-|_\partial}", curve={height=-12pt}, from=1-2, to=1-1]
    \arrow["\dashv"{anchor=center, rotate=-90}, draw=none, from=0, to=1]
\end{tikzcd}\]

Note, however, that the left Kan extension in Equation \ref{eq:lan-lie} does not immediately give the desired adjunction of Theorem \ref{thm:lie-realization}, as there is no guarantee that the nerve functor $N_\partial$ lands in the category of algebroids. To prove this we must revisit the work in Section \ref{sec:weil-nerve} presenting involution algebroids as those functors $A:\wone \to \C$ for which each $A(V)$ is the prolongation of its underlying anchored bundle; this leads naturally to the formalism for enriched theories developed in \cite{Bourke2019}.

The presentation of algebroids as models of an enriched theory requires situating the categories of differential bundles, anchored bundles, and involution algebroids in $\w$ as full subcategories of $\w$-presheaf categories on small $\w$-categories. 
% Remove enum here!
% \begin{enumerate}
Section \ref{sec:tang-cats-enrichment} reviews the work in \cite{Garner2018} characterizing tangent categories as categories enriched in $\w = \mathsf{Mod}(\wone, \s)$ (the cofree tangent category on $\s$, by Observation \ref{obs:cofree-tangent-cat}).
    % \item 

Section \ref{sec:enriched-structures} reconfigures the content of Chapter \ref{ch:differential_bundles} using the enriched perspective, so that lifts are $\w$-functors from the $\w$-monoid $D + 1$, while differential bundles are a reflective subcategory of functors from its idempotent splitting. Similarly, the category of anchored bundles are a reflective subcategory of the category $[\wone^1,\C]$, where $\wone^1$ is the category of 1-truncated Weil algebras (Definition \ref{def:truncated-wone}).

Section \ref{sec:enriched-nerve-constructions} reviews the basic idea of an enriched nerve/approximation. A particularly important example is the linear approximation of a reflexive graph, the functor introduced in Example \ref{ex:prolongations}, which is the nerve of a functor
\[
    \partial: (\wone^1)^{op} \to \mathsf{Gph}(\w).
\]

Section \ref{sec:enriched-theories} applies the enriched theories framework introduced in \cite{Bourke2019}, where a \emph{dense} subcategory of a locally presentable $\a \hookrightarrow \C$ forms the ``arities'', and a bijective-on-objects functor $\a \to \th$ is the theory. The category of models is the pullback of (enriched) categories:
% https://q.uiver.app/?q=WzAsNCxbMCwxLCJcXEMiXSxbMSwxLCJbXFxhXntvcH0sIFxcdl0iXSxbMSwwLCJbXFx0aCwgXFx2XSJdLFswLDAsIlxcQ157XFx0aH0iXSxbMCwxLCIiLDEseyJzdHlsZSI6eyJ0YWlsIjp7Im5hbWUiOiJob29rIiwic2lkZSI6InRvcCJ9fX1dLFsyLDFdLFszLDBdLFszLDJdXQ==
\[\begin{tikzcd}
	{\C^{\th}} & {[\th, \vv]} \\
	\C & {[\a^{op}, \vv]}
	\arrow[hook, from=2-1, to=2-2]
	\arrow[from=1-2, to=2-2]
	\arrow[from=1-1, to=2-1]
	\arrow[from=1-1, to=1-2]
\end{tikzcd}\]
The first step is to freely complete the $\w$-category $\wone^1$ of truncated Weil algebras (Definition \ref{def:truncated-wone}) so that its base anchored bundle has all prolongations; call this $\w$-category $\prol$. Then, in every tangent category $\C$, the category of anchored bundles with chosen prolongations (Definition \ref{def:monoidal-category}) in $\C$ embeds fully and faithfully into the functor category:
\[
    \mathsf{Anc}^\prol(\C) \hookrightarrow [\prol, \C]  
\]
(here, $\mathsf{Anc}^\prol(\C)$ denotes the category of anchored bundles with chosen prolongations).
In particular, the tangent bundle on $\N$ in $\wone$ determines a bijective-on-objects functor
\[
    \prol \to \wone 
\]
so that the category of involution algebroids in any tangent category $\C$ is the pullback in $\w$Cat:
\[\begin{tikzcd}
    {\mathsf{Inv}^*(\C)} & {[\wone,\C]} \\
    {\mathsf{Anc}^*(\C)} & {[{\prol}^{op},\C]}
    \arrow[hook, from=2-1, to=2-2]
    \arrow[from=1-2, to=2-2]
    \arrow[from=1-1, to=2-1]
    \arrow[hook, from=1-1, to=1-2]
    \arrow["\lrcorner"{anchor=center, pos=0.125}, draw=none, from=1-1, to=2-2]
\end{tikzcd}\]
This means that the category of involution algebroids in $\w$ is monadic over the category of anchored bundles in $\w$ using the monad-theory correspondence from \cite{Bourke2019}.

The final section looks at the category of $\w$-groupoids. Essentially, the free groupoid over the linear approximation of a graph will now give an infinitesimal object $\wone \to \mathsf{Gpd}(\w)$. The nerve of  $\partial: \wone^{op} \to \mathsf{Gpd}(\w)$---the \emph{infinitesimal approximation}---has a right adjoint via the \emph{realization} of the nerve functor from Definition \ref{def:realization}. This is used to prove the culminating Theorem \ref{thm:lie-realization}.

The individual pieces of categorical machinery used in this chapter are not new (the enriched perspective on tangent categories, enriched nerve constructions, enriched theories). However, all of the results dealing with the application of enriched nerve constructions and enriched theories to tangent categories is original work of the author.

\section{Tangent categories via enrichment}%
\label{sec:tang-cats-enrichment}

This section gives a quick introduction to Garner's enriched perspective on tangent categories. The enriched approach to tangent categories first appeared in \cite{Garner2018} and builds on the category perspective on tangent categories introduced in \cite{Leung2017}. Garner was able to exhibit some of the major results from synthetic differential geometry as pieces of enriched category theory; for example, the Yoneda lemma implies the existence of a well-adapted model of synthetic differential geometry.

The category of Weil spaces is the site of enrichment for tangent categories and is closely related to Dubuc's \emph{Weil topos} from his original work on models of synthetic differential geometry \cite{Dubuc1981}; a deeper study of this topos may be found in \cite{Bertram2014}. Recall that the category $\wone$ is the free tangent category over a single object. The category of Weil spaces is the \emph{cofree} tangent category over $\s$, which is the category of transverse-limit-preserving functors $\wone \to \s$ by Observation \ref{obs:cofree-tangent-cat}. Call this the category of \emph{Weil spaces}, and write it $\w$. Just as a simplicial set $S: \Delta \to \s$ is a gadget recording homotopical data, a Weil space records \emph{infinitesimal} data.

\begin{definition}%
    \label{def:weil-space}
    A \emph{Weil space} is a functor $\wone \to \s$ that preserves transverse limits (Definition \ref{def:transverse-limit}): that is, the $\ox$-closure of the set of limits
    \[
        \left\{
        \input{TikzDrawings/Ch5/wone-pb.tikz}
        ,
        \input{TikzDrawings/Ch5/wone-univ-lift.tikz}
        ,
        \input{TikzDrawings/Ch5/wone-id.tikz}
        \right\}
    \]
    A morphism of Weil spaces is a natural transformation. Write the category of Weil spaces as $\w$.
\end{definition}
\begin{example}
    ~\begin{enumerate}[(i)]
        \item Every commutative monoid may be regarded as a Weil space canonically. Observe that for every $V \in \wone$ and commutative monoid $M$, one has the free $V$-module structure on $M$ given by $|V| \ox_{\mathsf{CMon}} M$ (here $|V|$ is the underlying commutative monoid of $V$). The commutative monoid $|V|$ is exactly $\N^{\mathsf{\dim V}}$, so that
        \[
            |V| \ox_{\mathsf{CMon}} M \cong \oplus^{\dim V} M.
        \]
        This agrees with the usual tangent structure on a category with biproducts.
        % (this is the \emph{copower} of $M$ by $V$, )
        % The infinitesimal linearity of $V \mapsto |V| \ox M$ follows from fundamental properties of finite biproducts.
        \item Following (i), any tangent category $\C$ that is \emph{concrete}---that is, admitting a faithful functor $U: \C \to \s$---will have a natural functor into Weil spaces (copresheaves on $\wone$). Every object $A$ will have an underlying Weil space $V \mapsto U_{\s}(T^V(A))$, and whenever $U$ preserves connected limits (such as the forgetful functor from commutative monoids to sets), each of the underlying copresheaves will be a Weil space.
        \item Consider a symmetric monoidal category with an infinitesimal object, which by Proposition \ref{prop:monoidal-functor-inf-obj} is a transverse-colimit-preserving symmetric monoidal functor $D:\wone \to \C$. Then for every object $X$, the nerve (Definition \ref{def:nerve-of-a-functor}) $N_D(X): \C(D-,X):\wone \to \s$ is a Weil space.
        \item For any pair of objects $A,B$ in a tangent category, $\C(A,T^{(-)}B):\wone \to \s$ is a Weil space by the continuity of $\C(B,-):\C \to \s$.
    \end{enumerate}
\end{example}
Unlike the category of simplicial sets, the category of Weil spaces is not a topos.
The category of Weil spaces does, however, inherit some nice properties from the topos of copresheaves on $\wone$ by applying results from \Cref{sec:enriched-nerve-constructions}, as it is \emph{locally presentable}. The basics of locally presentable categories can be found in the Appendix \ref{appendix}. Roughly speaking, a cocomplete category $\C$ has a subcategory of finitely presentable objects $\C_{fp}$, those $C$ so that
\[
    \C(C,-): \C \to \s  
\]
preserves \emph{filtered} colimits (i.e. those colimit diagrams that commute with all finite limits in $\s$). $\C$ is locally finitely presentable whenever every object is given by the coend
\[
    C \cong \int^{X \in \C_{fp}} \C(X, C) \cdot X
\]
where $S\cdot X$ is the (possibly infinite) product $X^{|S|}$ with $|S|$ the cardinality of the set $S$. This means that $\C = \mathsf{Lex}(\C_{fp}^{op}, \s)$, where $\mathsf{Lex}$ means the category of finite-limit-preserving functors. 

The third point of Corollary \ref{cor:properties-of-w} below, that $\w$ is locally finitely presentable as a cartesian monoidal category, means that the category $\w_{fp}$ is closed under products. This implies that $\w$ is locally presentable \emph{as a} $\w$-category (this ends up being an important technical condition whereby locally presentable $\w$-categories make sense). Whenever we discuss an arbitrary $\vv$-category, we will assume that $\vv$ is locally presentable as a monoidal category.

\begin{proposition}[\cite{Garner2018}]
    The category of Weil spaces is a cartesian-monoidal reflective subcategory of\, $[\wone,\s]$.
\end{proposition}
\begin{corollary}%
    \label{cor:properties-of-w}
    The category of Weil spaces is
    \begin{enumerate}[(i)]
        \item a cartesian closed category;
        \item a representable tangent category, where the infinitesimal object is given by the restricted Yoneda embedding $\yon: \wone^{op} \to \w$;
        \item Locally finitely presentable as a cartesian monoidal category.
    \end{enumerate}
\end{corollary}
% \begin{observation}%
% \label{obs:v-cat-stuff}
% \end{observation}
The cofree tangent structure on $\w$ is given by precomposition, that is:
\[
    T^U.M.(V) = M.(U \ox V) = M.T^U.V.
\]
This tangent coincides with the representable tangent structure induced by the Yoneda embedding. The proof is an application of the Yoneda lemma. Observe that
\[
    [D,M](V) \cong [\wone,\s](D \x \yon(V), M) \cong [\wone, \s](\yon(WV), M) \cong M(WV)
\]
where $D \x \yon(V) = \yon(WV)$ follows because the tensor product in $\wone$ is cocartesian and the reflector is cartesian monoidal.
\begin{notation}
    The tangent category $\w$ is a representable tangent category, where $D = \yon W$ (Definition \ref{def:inf-object}). 
    We will write the Yoneda functor $\yon: \wone^{op} \to \w$ as $D(-)$, so it is closer to the usual notation used in representable tangent categories or synthetic differential geometry (a single $D$ may be used as shorthand for $D(W)$, $D(n)$ for $D(W_n)$, etc.).
    Note that
    \[
        D(V) = D(\ox^k W_{n(i)}) = \prod^K D(n_i)  
    \]
    and that $D(\ell) = \otimes, D(c) = (\pi_1, \pi_0)$, $D(+) = \delta$, and so on.
\end{notation}

At this point, we are ready to move to the enriched perspective on tangent categories. The basics of enriched category theory may be found in the Appendix \ref{appendix}, but one definition in particular is important to include here.
\begin{definition}\label{def:power}
    Let\, $\C$ be a $\vv$-category for $\vv$ a closed symmetric monoidal category. For $J \in \vv$, the \emph{power} by $J$ of an object $C \in \C$ is an object $C^J$ so that the following is an isomorphism:
    \[
        \forall D: \vv(J, \C(D,C)) \cong \C(D,C^J)
    \]
    whereas the \emph{copower} is given by
    \[
        \forall D: \vv(J, \C(C,D)) \cong \C(J \bullet C, D).
    \]
    Let $\j \hookleftarrow \vv$ be a full monoidal subcategory of $\vv$. A $\vv$-category\, $\C$ has \emph{coherently chosen powers} by $\j$ if there is a choice of $\j$-powers\, $(-)^J$ so that
    \[
        (C^J)^K = C^{J \ox K}.
    \]
    Likewise, \emph{coherently chosen copowers} are a choice of $\j$-copowers so that
    \[
        K \bullet (J \bullet C) = (K \ox J) \bullet C.
    \]
    The sub-2-categories of $\vv$-categories equipped with coherently chosen powers and copowers are $\vv\cat^\j$ and $\vv\cat_\j$, respectively.
\end{definition}

\cite{Wood1978} proved that the 2-category of actegories over a monoidal $\C$ is equivalent to the 2-category of $[\C,\s]$-enriched categories with powers by representable functors (Definition \ref{def:power}), using the monoidal structure on $[\C, \s]$ induced by Day convolution\footnote{
    The \emph{Day convolution} tensor product of presheaves $X,Y: \C^{op} \to \s$ is given by $X \widehat\ox Y := \mathsf{Lan}_{\ox}(X \boxtimes Y)$, where $(X \x Y)(V) =X(V) \x Y(V)$ (\cite{Day1970}).
    }. 
Moreover, \cite{Garner2018} showed that a monoidal reflective subcategory $\vv \hookrightarrow \hat \a$ exhibits the 2-category of $\vv$-categories as a reflective sub-2-category of $\a$-categories; this proves that tangent categories are equivalent to a particular class of enriched category.
\begin{proposition}[\cite{Garner2018}]
    A tangent category is exactly a $\w$-category with powers by representables.
\end{proposition}
\begin{proof}
    For every $A, B \in \C_0$, the Weil space is defined as
    \[
        \underline{\C}(A,B) := U \mapsto \C(A, T^{U}B).
    \]
    The functor $\C(A,-)$ is continuous and $T^{-}B$ is an infinitesimally linear functor, so this is a Weil space. The following diagram gives the composition map :
    \input{TikzDrawings/Ch5/enrichment-in-W-comp.tikz}
    Note that it is natural in $U$ and $V$.

    By the Yoneda lemma (as the internal hom in $\w$ is the internal hom of copresheaves on $\wone$), \[\C(A, T^VB) =  (U \mapsto \C(A, B)(V \ox U)) = [D(V), \C(A,B)]\]
    so this category has coherently chosen powers by representable functors.
\end{proof}
% Enriched category theory introduces a new family of limits, \emph{powers}\footnote{Recall that the entire class of weighted limits is equivalent to conical (ordinary) limits and powers.}. Powers are essential in tangent category theory and satisfy a coherence due to \cite{LucyshynWright2016}.

% Now, returning to the specific enrichment in $\w$, remember that the hom-object between $A, B$ is given by:
% \[
%     V \mapsto \C(A,T^V B)
% \]
% For a representable functor $\yon U$, use the fact that the representable tangent structure induced by $\yon$ coincides with the cofree tangent structure and find:
% \[
%     \w(\yon U, V \mapsto \C(A, T^VB)) \cong V \mapsto \C(A, T^V.T^U(B))
% \]
% Then, a tangent category does not simply have enrichment in $\w$; it also has coherently chosen powers by representable functors (this applies to any actegory - the original insight bridging actegories and presheaf-enriched categories \cite{Wood1978}). $\w$-categories with coherently chosen powers by representables are \emph{exactly} tangent categories. Thus we have the following:
% \begin{proposition}[\cite{Garner2018}]
%     A tangent category is exactly an $\w$-category with coherently chosen powers by representables.
% \end{proposition}
Now the original notions of (lax, strong, strict) tangent functors can be shown to correspond to power-preservation properties of $\w$-functors between $\w$-categories with coherently chosen powers:
\begin{theorem}[\cite{Garner2018}]\label{thm:2-cat-tangcat}
    We have the following equivalences of 2-categories:
    \begin{enumerate}[(i)]
        \item the 2-category of $\w$-categories with coherently chosen powers and $\mathsf{TangCat}_{\mathsf{Lax}}$;
        \item the 2-category of $\w$-categories with coherently chosen powers and power-preserving $\w$ functors  and $\mathsf{TangCat}_{\mathsf{Strong}}$;
        \item The 2-category of $\w$-categories with coherently chosen powers and chosen-power-preserving $\w$ functors and $\mathsf{TangCat}_{\mathsf{Strict}}$.
    \end{enumerate}
\end{theorem}
Note that for a lax tangent functor $(F,\alpha): \C \to \D$, the map
\[
    \alpha.X: F.T.X \to T.F.X
\] can be seen as the unique morphism induced by universality. Conversely, for a strong tangent functor, the natural isomorphism $\alpha$ is the isomorphism from \emph{a} power $F.T.X$ to the \emph{coherently chosen} power $T.F.X$. In contrast, a strict tangent functor preserves the coherent choice of powers.

The ability to work with $\w$-categories that do not have powers by representables allows for significant flexibility. It is useful to observe that there are $\w$-categories which are not tangent categories.
\begin{example}%
    \label{ex:w-cats}
    ~\begin{enumerate}[(i)]
        \item Every monoid $(M, m, e)$ in $\w$ gives rise to a one-object $\w$-category whose hom-object is $M$, composition is $m$, and unit is $e$.
        \item Given a tangent category\, $\C$, it is possible to take the full\, $\w$-category over some set of objects\, $\D$, even though $D$ may not be closed under iterated applications of the tangent functor.
        \item The dual of a\, $\w$-category is a\, $\w$ category, where
        \[
            \C^{op}(A,B) = \C(B,A): \wone \to \s.    
        \]
        Dually, in the case that\, $\C$ is a tangent category, $\C^{op}$ will have coherent \emph{copowers} by representables.
        \item If a cartesian category\, $\C$ has an infinitesimal object $D$ (Definition \ref{def:inf-object}), it has a natural $\w$-category structure (with coherent \emph{copowers} by representables)
        \[
            \C(A,B):\wone \to \s := \underline{\C}(A \x D(-), B).
        \]
        Using the dual tangent structure on $\C^{op}$ from Proposition \ref{prop:inf-object-tangent-structures}, the enrichment on $\C$ is exactly the enrichment found by regarding $\C$ as the dual $\w$-category of the tangent category\, $\C^{op}$.
    \end{enumerate}
\end{example}

As a final remark, note that the Yoneda lemma applies to $\w$-categories, so there is an embedding
\[
    \C \hookrightarrow [\C^{op},\s].
\]
The powers and copowers by representables are computed pointwise in a presheaf category, so they inherit the coherent choice. Thus the following holds:
\begin{corollary}[\cite{Garner2018}]
    Every tangent category embeds into a $\w$-cocomplete representable tangent category.
\end{corollary}
(In fact, showing that the Yoneda embedding applies to tangent categories is the main theorem of \cite{Garner2018}.)


\section{Differential and anchored bundles as enriched structures}%
\label{sec:enriched-structures}

This section gives an enriched-categorical reinterpretation of the work in Chapter \ref{ch:differential_bundles} regarding differential bundles and Chapters \ref{ch:involution-algebroids} and \ref{chap:weil-nerve} regarding anchored bundles.

\subsection*{Differential bundles as enriched structures}%
\label{sub:differential-bundles-as-enriched-structures}

As a first case study using the enriched perspective for tangent categories, consider differential bundles. Most of the work in \Cref{ch:differential_bundles} uses the intuition that differential bundles are some sort of tangent-categorical algebraic theory; this section will make that intuition concrete. Recall that a \emph{lift} (Definition \ref{def:lift}) is a map $\lambda:E \to TE$. Using the enriched perspective and treating $TE$ as a power (recall Definition \ref{def:power}), this gives the following correspondence:
\[
    \infer{\hat{\lambda}:D \to \C(E,E)}{\lambda: 1 \to \C(E, E^D)}
\]
The commutativity condition for $\hat{\lambda}$, then, is translated as follows:
\[\input{TikzDrawings/Ch5/dbun-commutativity.tikz}\]
That is, $\hat{\lambda}$ is a semigroup morphism $D \to \C(E,E)$. In any cartesian closed category with coproducts, a semigroup may be freely lifted to a monoid using the "exception monad" $(-) + 1$ from functional programming (see, for example, \cite{Seal2013}).
\begin{definition}%
    \label{def:lambda-monoid}
    Regard the following monoid as the one-object $\w$-category $\Lambda$:
    \[
        \infer{m: (D+1) \x (D+1) \to D+1}{D \x D + D + D + 1 \xrightarrow{(\iota_L\o m | \iota_L | \iota_L |\iota_R)} D + 1}
    \]
\end{definition}
Thus, a lift $\lambda$ is exactly a functor $\hat{\lambda}:\Lambda \to \C$. Now check that morphisms are tangent natural transformations. Note that the semigroup $D$ is commutative, so $\Lambda = \Lambda^{op}$; the choice of using $\Lambda^{op}$ in the next lemma is to be consistent with conventions used in \Cref{sec:enriched-nerve-constructions}.

\begin{lemma}%
    \label{lem:cat-of-lifts-iso}
    The category of lifts in a tangent category $\C$ is isomorphic to the category of $\w$-functors and $\w$-natural transformations\, $\Lambda^{op} \to \C$.
\end{lemma}
\begin{proof}
    Check that a $\w$-natural transformation is exactly a morphism of lifts $f: \lambda \to \lambda'$. Start with the $\w$-naturality square:
    \[\input{TikzDrawings/Ch5/w-nat.tikz}\]
    Now, rewriting $D+1 \to \C(A,B)$ as a semigroup map $D \to \C(A,B)$, we have:
    \[
        \infer{
            \infer{
                1 \xrightarrow{Tf \o \lambda'}\C(A,TB)
            }{
                1 \xrightarrow{(\lambda',f)} \C(A,TA) \x \C(A,B) \xrightarrow{1 \x T} \C(A,TA) \x \C(TA,TB) \xrightarrow{m_{A,TA,TB}} \C(A,TB)
            }}{
            D \xrightarrow{(\lambda,f\o !)} \C(A,A) \x \C(A,B) \xrightarrow{m_{AAB}} \C(A,B)
        }
    \]
    Similarly, the other path is exactly $\lambda' \o f$. Thus, a $\w$-natural transformation is exactly a morphism of lifts. %TODO make sure the change l \Rightarrow \lambda' makes sense
\end{proof}

It is a classical result in synthetic differential geometry that the object $D$ has only one point. In the case of $\w$, this follows from the Yoneda lemma (regarding $\w$ as a $\s$-category):
\[
    \w(1,D) = \wone(\N[x]/x^2, \N) = \{ !: \N[x]/x^2 \to \N \}
\]
Note that the natural idempotent $e:id \Rightarrow id$ from Proposition \ref{prop:idempotent-natural}, then, must be the point $0:1 \to D$. Note that this idempotent is an absorbing element of the monoid $D+1$, so for any $f:X \to D+1$, it follows that $m(f,0\o !) = m(0\o !, f) = 0\o !$.  A pre-differential bundle is exactly a lift with a chosen splitting of the natural idempotent $p\o\lambda$.
\begin{lemma}%
    \label{lem:splitting-of-idemp-is-lambdaplus}
    For every lift $\bar{\lambda}:\Lambda^{op} \to \C$, the natural idempotent $e = p \o \lambda$ is exactly
    \[1 \xrightarrow[]{0} D \xrightarrow[]{\iota_L} D+1 \to \C(E,E).\]
\end{lemma}
Now that the natural idempotent is understood as a map in $\Lambda$, that idempotent splits to give the theory of a pre-differential object.
\begin{definition}%
    \label{def:lambda-plus}
    The $\w$-category $\Lambda^+$ is given by the set of objects $\{0,1\}$ with hom-Weil-spaces. Specifically,
    \begin{itemize}
        \item The hom-spaces are $\Lambda^+(1,1) = D+1$, otherwise $\Lambda^+(i,j) = 1$.
        \item Composition (writing the original composition from $\Lambda$ as $m$) is given by
        \begin{gather*}
            m_{111}: (D+1) \x (D+1) \xrightarrow[]{m} (D+1) \\
            m_{101}: 1 \x 1 \xrightarrow[]{\iota_R} (D + 1) \\
            \text{otherwise: } m_{ijk} = ! 
        \end{gather*}  
    \end{itemize}
\end{definition}
% The following lemma lets simplifies the composition in $\Lambda^+$.
% \begin{lemma}
%   Associativity of composition in $\Lambda^+$ can be written by embedding each hom-space into $D+1$
%   $\w$-category, where we use inclusions:
%   \begin{gather*}
%       \Lambda^+(0,0) = 1 \xrightarrow{\iota_L \o 0 \o !} D+1  \hspace{0.25cm}
%       \Lambda^+(0,1) = 1 \xrightarrow{\iota_L \o 0 \o !} D+1  \\
%       \Lambda^+(1,0) = 1 \xrightarrow{\iota_L \o 0 \o !} D+1  \hspace{0.25cm}
%       \Lambda^+(1,1) = D+1 \xrightarrow{id} D+1  
%   \end{gather*}
%     \[\input{TikzDrawings/Ch5/assoc.tikz}\]
% \end{lemma}
Idempotent splittings are absolute (co)limits, and are preserved by all functors; as this is a limit completion, we have the following.
\begin{lemma}%
    \label{lem:lambda-plus-is-pdb}
    The category of pre-differential bundles is exactly the category of $\w$-functors $\Lambda^+\to \C$ (that is, $\C$-valued presheaves).
\end{lemma}
% ($(\Lambda^+)^{op} = \Lambda^+$, so this is purely conventional).
It is straightforward to exhibit the category of differential bundles as a reflective subcategory of pre-differential bundles in $[\Lambda^+, \C]$ (so long as $\C$ has equalizers).
\begin{proposition}%
    \label{prop:Lambda-is-refl-subcat}
    The category of differential bundles in a tangent category\, $\C$ with $T$-equalizers and $T$-pullbacks is a reflective subcategory of\, $[\Lambda^+, \C]$.
\end{proposition}
\begin{proof}
    The category of pre-differential bundles in $\C$ is isomorphic to $[\Lambda^+, \C]$. By Corollary \ref{cor:idemp-dbun}, the category of differential bundles is the category of algebras for an idempotent monad on the category of pre-differential bundles in $\C$. The reflector sends a pre-differential bundle to the $T$-equalizer: \input{TikzDrawings/Ch5/reflector.tikz} This equalizer will always exist if $\C$ has equalizers, and pullbacks of the projection will exist if $\C$ has pullbacks, so the pullback is a differential bundle. This reflection gives a left-exact idempotent monad on $[\Lambda^+, \C]$ whose algebras are differential bundles.
\end{proof}
Now, in the case that $\C$ is a \emph{locally presentable} tangent category (such as $\w$), $[\Lambda^+, \C]$ is locally presentable and so is the reflective subcategory of differential bundles; thus the following holds.
\begin{corollary}%
    \label{cor:Lambda-dense}
    If\, $\C$ is locally presentable, then $\mathsf{DBun}(\C)$ is a locally presentable category.
\end{corollary}

\subsection*{Anchored bundles}

There are two ways to think about anchored bundles:
\begin{enumerate}[(i)]
    \item an anchored bundle is a differential bundle with an anchor $A \to TM$, or
    \item an anchored bundle is an involution algebroid without an involution.
\end{enumerate} 
These two perspectives can be unified by regarding $A$ as a cylinder for the weighted limit $TM$, so that the anchor is induced by the unique map $A.T.0 \to T.A.0$:
\[\input{TikzDrawings/Ch5/anc-as-nat.tikz}\]
That is, the syntactic category for anchored bundles is constructed as a full $\w$-category of $\wone$ that doesn't include the map $c$. This may be found by taking the full subcategory of $\wone$ whose objects are constructed out of $W_n, n\in \mathbb{N}$.
\begin{definition}%
    \label{def:truncated-wone}
    A Weil algebra has \emph{width} $k \in \N$ if it can be written
    \[
        V = \ox^{0 \le i < k} W_{n(i)},\; n(i) \in \N
    \]
    The category of $k$-truncated Weil algebras, written\, $\wone^k$, is the full $\w$-subcategory of\, $\wone$ of Weil algebras of width $k$ or less.

    The full subcategory whose objects are $\{ \N, W \}$ will be given the special notation $\wone^*$.
\end{definition}
Note that for each $V$ in $\wone^k$, the enrichment is given by
\[
    \wone^k(U,V) := (X \mapsto \underline{\wone}(U, X \ox V) ).
\]
The maps in $\wone^1$---that is, the full subcategory of $\wone$ whose objects are
\[
    \{ W_n | n \in \mathbb{N} \}
\]
---have the useful property that they may be written without the flip $c$. This makes $\wone^1$ a natural candidate for the syntactic category of anchored bundles. 
\begin{lemma}\label{lem:writing-maps-in-wone}
    Every morphism
    \[
        W_n \to V \in \wone
    \]
    may be written without $c$; that is, it is generated by the set of maps $\{p,+,0,\ell\}$ closed under tensor, composition, and maps induced by transverse limits.
\end{lemma}
\begin{proof}
    Every map $W_n \to V$ in $\wone$ is the finite sum
    \[
        v \mapsto \sum_{x} (A_x v)\bullet x
    \]
    where $x \in \mathsf{var}(V)$, and $A_x \in \N^n$ so that $A_n v$ is the ordinary dot product.
    Each term can be written without $c$, and the whole term is then constructed by adding each component using the appropriate $U.+.V$-symbols. \footnote{This may also be regarded as a consequence of the graphical notation for maps in $\wone$ in Table 1 on page 308 of \cite{Leung2017}.}
\end{proof}
 It is possible, then, to show that the category of anchored bundles in $\C$ is a full sub-tangent-category of functors $\wone^1 \to \C$; note that $W$ acts as a cone for $(-)^{D}$, so this induces a map
\[
    \anc: F(W) \to T.F(\N).
\]
\begin{proposition}
    \label{prop:nerve-anc-work}
    Every anchored differential bundle in $\C$ determines a functor $\wone^1 \to \C$; an anchored bundle morphism is exactly an enriched natural transformation.
\end{proposition}
\begin{proof}
    Start with an anchored bundle $(q:E \to M, \xi, \lambda, \anc)$; then for hom-objects with domain $\N$,
    \[
        \wone^1(\N,\N) = 1 \mapsto id_M, \hspace{0.5cm} \wone^1(\N,T) = 1 \mapsto \xi.
    \]
    For the hom-objects with domain $T$, the problem is slightly more difficult, as $\wone^1(T,\N)(T^V) = \wone(T, T^V)$ and $\wone^1(T, T)(T) = \wone(T, T.T^V)$. This part of the  proof amounts to constructing maps
    \[
        \wone(T, T^V) \to \C(E, T^V.M), \hspace{0.5cm}  \wone(T,T^V.T) \to \C(E, T^V.E)/
    \]
    The first mapping is straightforward: send $\theta$ to $\theta.M \o \anc$. For the second map, observe that the following diagrams commute:
    \[
        \input{TikzDrawings/Ch5/anc-atlas.tikz}
    \]
    The idea is to take an anchored bundle morphism $f$ and rewrite it as a string of compositions that does not include $c$, switching out every occurrence of
    \[
        T^V.\theta.M, \theta \in \{p,0,+,\ell\}
    \]
    and replacing it with $T^V.(\theta')$, where $\theta'$ is the corresponding map in $\{q,\xi,+_q,\lambda\}$. This induces a map \[\wone^1(W,W) \to \C(E,E) \in \w\] and the $\anc$ is exactly the unique $\alpha: F.W \to T.F$ induced by universality.

    For morphisms, the inclusion $(\Lambda^+)^{op} \to \wone$ ensures that any tangent natural transformation will be a linear morphism on the underlying differential bundle, and the tangent natural transformations coherences ensure that a tangent natural transformation will preserve the anchor. Conversely, an anchored bundle morphism will preserve each of the constructed morphisms $E \to T^V.E, E \to T^V.M$ (as it preserves each of $\{ q, +_q, \xi, \lambda\}$), giving a natural tangent transformation. Thus there is a faithful embedding $\mathsf{Anc}(\C) \hookrightarrow [\wone^1, \C]$.
\end{proof}
The converse, identifying those functors $\wone^1 \to \C$ that determine anchored bundles, is immediate.
% It is possible to restrict attention to $\wone^*$ from.
\begin{corollary}%
    \label{cor:anchored-bundle-as-nerve}
    The category of anchored bundles comprises precisely the $\w$-functors and $\w$-natural transformations
    $
        A: \wone^* \to \C    
    $ (Definition \ref{def:truncated-wone})
    so that the precomposition $\Lambda^+ \to \wone^* \to \C$ determines a differential bundle.
    That is to say, the category of anchored bundles in $\C$ is the following pullback in $\w$Cat:
    % https://q.uiver.app/?q=WzAsNCxbMCwwLCJcXG1hdGhzZntBbmN9KFxcQykiXSxbMSwwLCJbXFx3b25lXjEsIFxcQ10iXSxbMSwxLCJbXFxMYW1iZGFeKywgXFxDXSJdLFswLDEsIlxcbWF0aHNme0RCdW59KFxcQykiXSxbMCwxLCIiLDAseyJzdHlsZSI6eyJ0YWlsIjp7Im5hbWUiOiJob29rIiwic2lkZSI6InRvcCJ9fX1dLFsxLDJdLFszLDIsIiIsMix7InN0eWxlIjp7InRhaWwiOnsibmFtZSI6Imhvb2siLCJzaWRlIjoidG9wIn19fV0sWzAsM10sWzAsMiwiIiwxLHsic3R5bGUiOnsibmFtZSI6ImNvcm5lciJ9fV1d
    \[\begin{tikzcd}
        {\mathsf{Anc}(\C)} & {[\wone^1, \C]} \\
        {\mathsf{DBun}(\C)} & {[\Lambda^+, \C]}
        \arrow[hook, from=1-1, to=1-2]
        \arrow[from=1-2, to=2-2]
        \arrow[hook, from=2-1, to=2-2]
        \arrow[from=1-1, to=2-1]
        \arrow["\lrcorner"{anchor=center, pos=0.125}, draw=none, from=1-1, to=2-2]
    \end{tikzcd}\]
\end{corollary}

%  
\section{Enriched nerve constructions}%
\label{sec:enriched-nerve-constructions}

Nerve constructions present a powerful generalization of the Yoneda functor that sends an object $C \in \C$ to the representable presheaf $\C(-, C) \in[\C^{op},\vv]$ (for a general reference on nerve constructions and their realizations, see Chapter 3 of \cite{Loregian2015}). 
The Yoneda lemma states that this functor is an \emph{embedding} (that is, it is fully faithful), so no information is lost when embedding a category into its category of presheaves. 
Nerve constructions, then, move this towards an \emph{approximation} of the original category $\C$ by some subcategory $\D \hookrightarrow \C$, or more generally by some functor $K:\a \to \C$.
In Section \ref{sec:inf-nerve-of-a-gpd}, the ``infinitesimal'' approximation of a Lie groupoid as a Lie algebroid will be exhibited as approximation by a $\w$-functor $\partial: \wone^{op} \to \mathsf{Gpd}(\w)$.

We work with a $\vv$ that is locally presentable as a monoidal category, such as $\w, \s$, or the category of commutative monoids.
\begin{definition}%
    \label{def:nerve-of-a-functor}
    The \emph{nerve} of a $\vv$-functor $K: \a \to \C$ is the functor
    \[
        N_K: \C \to [\a^{op}, \vv]; \hspace{0.5cm} C \mapsto \C(K-, C)
    \]
    Any presheaf $A:\a \to \vv$ that is in the image of $N_K$ is a \emph{$K$-nerve}.
\end{definition}
\begin{remark}
    The ``$K$-nerve'' terminology seems to go back to Grothendieck/Segals's original intuition for the nerve construction of a category\footnote{Segal published the result, but seems to have credited the theorem to Grothendieck.} (\cite{Segal1974}) and appears in, for example, \cite{Berger2012} and \cite{Bourke2019}. However, the ``approximation'' of a category by a functor $K:\a \to \C$ originally used in topology seems to be a more intuitive description of the functor $N_K$.
\end{remark}


\begin{example}%
    \label{ex:nerve-functors}
    ~\begin{enumerate}[(i)]
        \item The nerve of the identity functor $\C = \C$ is the usual Yoneda embedding $\C \hookrightarrow[\C^{op}, \w]$.
        \item 
        The first example of a nerve construction in the mathematical literature is the simplicial approximation of a topological space by \cite{Kan1958}. Recall the original construction of the simplicial nerve of a topological space $X$, where $X_n = \mathsf{Top}(\Delta_n,X)$. This is exactly the nerve of the functor $\Delta \to \mathsf{Top}$ that sends $n$ to the $n$-simplex
        \[
            \left\{
                x \in \R^n | \sum x_i = 1
            \right\}.
        \]
        \item The following example figures into Segal's original nerve construction for a category, and is revisited in Section \ref{sec:enriched-theories}. Define a reflexive graph to be a presheaf over the full subcategory of $\mathsf{Cat}$ whose objects are the two preorders
        \[
            [0] = 0, \hspace{0.2cm} [1] =  0 < 1.    
        \]
        This is equivalent to the free category with two parallel arrows and a common retract,
        % https://q.uiver.app/?q=WzAsMixbMCwwLCIxIl0sWzEsMCwiMCJdLFsxLDAsInQiLDIseyJvZmZzZXQiOjJ9XSxbMSwwLCJzIiwwLHsib2Zmc2V0IjotMn1dLFswLDEsImUiLDFdXQ==
        \[\begin{tikzcd}
            {[1]} & {[0]}
            \arrow["t"', shift right=2, from=1-2, to=1-1]
            \arrow["s", shift left=2, from=1-2, to=1-1]
            \arrow["e"{description}, from=1-1, to=1-2]
        \end{tikzcd}\]
        so that a graph in a category has an object-of-vertices $V$ and an object-of-edges $E$, along with source and target maps $s,t:E \to V$; morphisms of graphs are maps $(f_E,f_V)$ that commute with the source and target maps.

        By Corollary \ref{cor:dense-ff-result}, any full subcategory containing the representables will be dense.
        Define the category of \emph{paths}, $\mathsf{Pth}$, to be full subcategory of graphs generated by
        \[ \input{TikzDrawings/Ch5/dense-in-graphs.tikz}, n \in \mathbb{N} \]
        so that $\mathsf{Gph}([n],G)$ picks out the set of paths of length $n$ in a graph $G$. We call this subcategory $P:\mathsf{Pth} \to \mathsf{Gph}$, and see that $N_P$ sends a graph to its $\mathsf{Pth}$-presheaf of composable paths:
        % https://q.uiver.app/?q=WzAsNyxbMCwxLCJFIl0sWzIsMSwiXFxkb3RzIl0sWzEsMiwiViJdLFszLDIsIlYiXSxbNCwxLCJFIl0sWzIsMiwiXFxkb3RzIl0sWzIsMCwiRV9uIl0sWzAsMiwidCJdLFsxLDIsInMiLDJdLFsxLDMsInQiXSxbNCwzLCJzIiwyXSxbNiwwXSxbNiw0XSxbNiwxLCIiLDAseyJzdHlsZSI6eyJuYW1lIjoiY29ybmVyIn19XV0=
        \[\begin{tikzcd}
            && {E_n} \\
            E && \dots && E \\
            & V & {} & V
            \arrow["t", from=2-1, to=3-2]
            \arrow["s"', from=2-3, to=3-2]
            \arrow["t", from=2-3, to=3-4]
            \arrow["s"', from=2-5, to=3-4]
            \arrow[from=1-3, to=2-1]
            \arrow[from=1-3, to=2-5]
            \arrow["\lrcorner"{anchor=center, pos=0.125, rotate=-45}, draw=none, from=1-3, to=2-3]
        \end{tikzcd}\]
        The pushout $[n]$ is precisely the graph
        \[
            0 \xrightarrow[]{} 1 \xrightarrow[]{}\dots \xrightarrow[]{} (n-1) \xrightarrow[]{} n    
        \]
        where the reflexive map $e$ at each vertex is suppressed.
    \end{enumerate}
\end{example}
Recall the functor sending reflexive graphs to anchored bundles from Example \ref{ex:prolongations}(iv).
This may be restated as a nerve construction.
\begin{proposition}%
    \label{prop:lin-approx-gph}
    There is a functor
    \[
        \partial: \wone^* \to \mathsf{Gph}(\w)  
    \]
    so that $N_\partial: \mathsf{Gpd}(\w) \to [\wone^*, \w]$ is the linear approximation of a reflexive graph as described in Example \ref{ex:prolongations} (iv).
\end{proposition}
\begin{proof}
    Let $s,t:C \to M, e:M \to C$ be a reflexive graph, and recall that the anchored bundle $C^\partial \to C$ is induced by the equalizer
    \[\input{TikzDrawings/Ch3/Sec9/eq-of-idem.tikz}\]
    The category $\mathsf{Gph}$ is a representable tangent category, where $D$ is the discrete graph $D = D$. By the Yoneda lemma, we have that
    \[
        [\mathsf{Gph}, \w](\yon 1, G) = G_1,
        [\mathsf{Gph}, \w](\yon 0, G) = G_0,
        [\mathsf{Gph}, \w](D \x \yon(i), G) = T.G_i,
    \]
    so the limit diagram defining the infinitesimal approximation of a graph becomes
    % https://q.uiver.app/?q=WzAsMyxbMSwwLCJcXG1hdGhzZntHcGh9KEQgXFx4IEksIEcpIl0sWzIsMCwiXFxtYXRoc2Z7R3BofShEIFxceCBJLCBHKSJdLFswLDAsIlxcbWF0aHNme0dwaH0oXFxwYXJ0aWFsLCBHKSJdLFsyLDBdLFswLDEsIihlIFxceCBJKV4qIiwwLHsib2Zmc2V0IjotMX1dLFswLDEsIihJIFxceCBlXnMpXioiLDIseyJvZmZzZXQiOjF9XV0=
    \[\begin{tikzcd}
        {C^\partial} & {\mathsf{Gph}(D \x I, C)} & {\mathsf{Gph}(D \x I, C).}
        \arrow[from=1-1, to=1-2]
        \arrow["{(e \x I)^*}", shift left=1, from=1-2, to=1-3]
        \arrow["{(D \x e^-)^*}"', shift right=1, from=1-2, to=1-3]
    \end{tikzcd}\]
    Now observe that $\mathsf{Gph}(\w)^{op}$ with the dual tangent structure from Proposition \ref{prop:inf-object-tangent-structures} has a reflexive graph object \[s,t:1 \to I, !:I \to 1.\] Construct its linear approximation as in Example \ref{ex:prolongations}(iv) (e.g. take the coequalizer of $\w$-graphs):
    % https://q.uiver.app/?q=WzAsMyxbMCwwLCJEXFx4IEkiXSxbMSwwLCJEXFx4IEkiXSxbMiwwLCJcXHBhcnRpYWwiXSxbMCwxLCJlIFxceCBJIiwwLHsib2Zmc2V0IjotMX1dLFswLDEsIkQgXFx4IGVecyIsMix7Im9mZnNldCI6MX1dLFsxLDJdXQ==
    \[\begin{tikzcd}
        {D\x I} & {D\x I} & \partial.
        \arrow["{e \x I}", shift left=1, from=1-1, to=1-2]
        \arrow["{D \x e^s}"', shift right=1, from=1-1, to=1-2]
        \arrow[from=1-2, to=1-3]
    \end{tikzcd}\]
    By Proposition \ref{prop:nerve-anc-work}, this determines a functor
    \[
        \partial: \wone^1 \to \mathsf{Gph}(\w)^{op}. 
    \]
    By the continuity of the hom-functor, the nerve $N_\partial: \mathsf{Gph}(\w) \to [\wone^*, \w]$ lands in the category of anchored bundles. The continuity of the hom-functor ensures that this is indeed the linear approximation from Example \ref{ex:prolongations} (iv).
\end{proof}


The simplicial localization in \cite{Kan1958} has a left adjoint, the \emph{geometric realization}, that constructs a topological space using the data of a simplicial set. This realization may be constructed using a left Kan extension.
\begin{definition}%
    \label{def:realization}
    Let $K: \a \to \C$ be a $\vv$-functor for a cocomplete $\,\C$.
    The \emph{realization} of $K$ is the left Kan extension
    \[
        \input{TikzDrawings/Ch5/realization.tikz}
        \hspace{0.5cm}
        |C|_K := \int^A \C(KA, C) \bullet KA
    \]
    A \emph{nerve/realization context} is a $\vv$-functor $K:\a \to \C$ from a small $A$ to a cocomplete $\C$.   
\end{definition}
The adjunction between simplicial sets and topological spaces follows from general categorical machinery as a nerve/realization context:
\begin{lemma}
    For every nerve/realization context $K: \a \to \C$, the realization of $K$ is left adjoint to the nerve of $K$.
\end{lemma}


\section{Nervous monads and algebroids}%
\label{sec:enriched-theories}

Transposing the characterization of algebroids from Theorem \ref{thm:weil-nerve} to the enriched perspective introduces the challenge of finding an appropriate framework to describe algebroids as enriched structures. Kelly's theory of enriched sketches (see Chapter 6 of \cite{Kelly2005}), small $\vv$-categories equipped with a chosen class of limits cones, seems to be a natural candidate. However, when regarding a tangent functor as a $\w$-functor, the natural part 
\[
    \alpha: F.T \Rightarrow T.F 
\]
is the unique morphism induced by the universality of $T$ as a weighted limit, so it becomes unclear how to translate the condition that $\alpha$ is a cartesian natural transformation (Definition \ref{def:cart-nat}).

A clue for how to proceed may be found in \cite{Kapranov2007}, which proved that Lie algebroids are monadic over anchored bundles (when allowing for infinite-dimensional vector bundles). 
Recent work in \cite{Bourke2019} and \cite{Berger2012} has developed the appropriate notion of (enriched) theories that correspond to monads over general locally presentable categories. % using the notion of \emph{dense} functors $K: \a \to \C$. 
A critical insight is that for a filtered-colimit-preserving monad $\mathbb{T}$ on $\s$, the opposite category of the Lawvere theory $\th$ is precisely the full subcategory of the Kleisli category whose objects are $[n] = \coprod_n 1$, and the \emph{nerve} of the inclusion
\[
    \th^{op} \hookrightarrow \s^{\mathbb{T}}
\]
is fully faithful; that is, when the functor $K$ is dense.
Formally, a functor is \emph{dense} whenever its nerve behaves like the Yoneda embedding. The theory of dense functors is developed in Chapter 5 of \cite{Kelly2005}. We continue to assume that an arbitrary site of enrichment $\vv$ is locally presentable as a $\vv$-category.
\begin{definition}%
    \label{def:dense}
    A functor $K: \a \to \C$ is \emph{dense} whenever the nerve of $K$ is fully faithful, and a subcategory inclusion that is dense will be called a \emph{dense subcategory}.
    % \begin{itemize}
    %   \item The nerve of $K$ is fully faithful.
    %   \item The identity functor $id:\C \to \C$ is the left Kan extension of $K$ along itself, so: $Lan_K K = id$.
    %   \item Every object is given as the weighted colimit $C = N_K(C) \star K$.
    % \end{itemize}
    % Note that a locally finitely presentable category is precisely\footnote{Assuming that Vopenka's principle holds \cite{Adamek1994}.}
    %  a cocomplete category $\C$ with a dense functor $K: \a \to \C$.
\end{definition}

Dense functors are poorly behaved under composition, but there is a useful cancellativity result from Section 5.2 of \cite{Kelly2005}.
\begin{proposition}
    Consider a diagram of $\vv$-categories
    \[\input{TikzDrawings/Ch5/lan-result.tikz}\]
    where $\alpha$ is a natural isomorphism and $K$ is dense.
    If this diagram exhibits $(\alpha, J)$ as the left Kan extension of $K$ along $F$, then $J$ is also dense.
\end{proposition}
\begin{corollary}%
    \label{cor:dense-ff-result}
    If $K$ is dense, $F$ is fully faithful, and  $J.F = K$, then $F$ is a dense subcategory.
\end{corollary}
Generally speaking, we will often refer to dense \emph{subcategories} rather than dense \emph{functors}. 
This is achieved by factoring the dense functor
\[
    K = \a \xrightarrow[]{K'} \mathsf{im}(K) \hookrightarrow \C
\]
where $\mathsf{im}(K)$ is the $\vv$-category whose objects are those of $\a$ and whose hom-objects are given by $\mathsf{im}(K)(A,B) = \C(KA, KB)$,  so that the inclusion of $\mathsf{im}(K)$ into $\C$ is fully faithful and therefore dense by Corollary \ref{cor:dense-ff-result}.

\begin{example}\label{ex:fin-card-def}
    ~\begin{enumerate}[(i)]
        \item The category of finite cardinals $\Sigma$ is the full subcategory of $\,\s$ whose objects are given by finite coproducts of the terminal object, $[n] = \coprod_n 1$. This is a skeleton of the category of finite sets, and a dense subcategory of $\,\s$ (see e.g. \cite{Bourke2019}).
        \item Recall that the category of $\vv$-presheaves on $\a$, $[\a^{op}, \vv]$ is the free colimit completion of $\a$ (see e.g. \cite{Kelly2005}). 
        If $\C$ is cocomplete, and $K:\a \to \C$ is dense, then the realization $|-|_K$ exhibits $\C$ as a reflective subcategory of $[\a^{op}, \vv]$, and is this a locally presentable category\footnote{In fact, an equivalent definition of a locally presentable category is as a cocomplete category with a dense subcategory.}. Conversely, if $\,\C$ is a reflective subcategory of $\,[\a^{op}, \vv]$ that contains the representable functors, then $\a$ is a dense subcategory of $\,\C$. 
        \item By Corollary \ref{cor:Lambda-dense}, the category of differential bundles in $\w$ is a reflective subcategory of $[{\Lambda^+}{op}, \vv]$, so $\yon: \Lambda^+ \hookrightarrow [{\Lambda^+}{op}, \vv]$ is a dense subcategory of differential bundles in $\w$ following the argument in the above example.
    \end{enumerate}
\end{example}

Nervous theories (\cite{Bourke2019}) are generalizations of Lawvere theories that extend to arbitrary locally finitely presentable $\vv$-categories. Recall that a classical Lawvere theory is a bijective-on-objects, product-preserving functor
\[
    t: \Sigma \to \th   
\]
where $\th$ is a cartesian category. Nervous theories replace $\Sigma$ with a dense subcategory of some locally presentable $\vv$-category, and the product preservation condition with conditions on the nerve of the theory map.
\begin{definition}%
    \label{def:nerve-theory}
    Let $K:\a \to \C$ be a dense $\vv$-subcategory of a locally finitely presentable $\,\C$. We call the replete image of $N_K$ in $[\a^{op}, \vv]$ the category of $K$-nerves. An \emph{$\a$-theory} is a bijective-on-objects $\vv$-functor $J: \a \to \th$, where each
    \[
        \th(J-,a): \a \to \vv
    \] is a $K$-nerve.
    The category of models for an $\a$-theory is the pullback in $\vv\mathsf{CAT}$:
    \[\input{TikzDrawings/Ch5/conc-models.tikz}\]
    (These are called \emph{concrete} models in \cite{Bourke2019}.)
\end{definition}
\begin{remark}
    The category of models for a theory is monadic. The core of the argument is due to Weber's \emph{nerve theorem}, found in \cite{Weber2007}, but an exposition on that result is beyond the scope of this thesis. 
\end{remark}
\begin{example}%
    \label{ex:theories}
    ~\begin{enumerate}[(i)]
        \item A functor $t:\Sigma \hookrightarrow \th$ is a $\Sigma$-theory if and only if $\th$ is a Lawvere theory, where  we use the fact that $\Sigma$ (Example \ref{ex:fin-card-def}) a  skeletal subcategory of finite sets. The nerve conditions in this case identify the models of the Lawvere theory, as the nerve of $\,\Sigma \hookrightarrow \s$ sends a set to the strict product-preserving functor $[n] \mapsto A^n$.
        \item As shown in \cite{Berger2012} and \cite{Bourke2019}, the original nerve construction from \cite{Segal1974} may be restated as saying that small categories arise as models of a $\mathsf{Pth}$-theory (Example \ref{ex:nerve-functors}). The set $\mathsf{Gph}([n], G)$ is the set of paths of through the graph $G$ that have $n$ non-identity elements, as
        \[
            G \ts{t}{s} G \cong G \ts{t}{id} M \ts{id}{s} G \cong G \ts{t}{s \o e \o s} G \ts{t \o e \o t}{s} G.     
        \]

        Next, recall that the category $\Delta$ may be regarded as follows:
        \begin{itemize}
            \item Objects: A strict order $[n] = 0 < 1 < \dots < n$, for $n \ge 0$, regarded as a category.
            \item Maps: Functors.
        \end{itemize}
        There is a bijective-on-objects functor from $\mathsf{Pth} \to \Delta$ that sends the graph $[n]$ to the pre-order $[n]$, as every graph homomorphism between paths will be order-preserving. Now observe that a model is precisely a simplicial set, where 
        \[
            X([0]) = M, 
            X([1]) = C,
            X([2]) = C \ts{t}{s} C,
            X([3]) = C \ts{t}{s} C \ts{t}{s} C.  
        \]
        Furthermore, the map
        % https://q.uiver.app/?q=WzAsNSxbMCwxLCIwIl0sWzEsMCwiMCJdLFswLDMsIjEiXSxbMSw0LCIyIl0sWzEsMiwiMSJdLFswLDEsIiIsMCx7InN0eWxlIjp7InRhaWwiOnsibmFtZSI6Im1hcHMgdG8ifX19XSxbMiwzLCIiLDAseyJzdHlsZSI6eyJ0YWlsIjp7Im5hbWUiOiJtYXBzIHRvIn19fV0sWzAsMl0sWzEsNF0sWzQsM11d
        \[\begin{tikzcd}[row sep = tiny]
            & 0 \\
            0 \\
            & 1 \\
            1 \\
            & 2
            \arrow[maps to, from=2-1, to=1-2]
            \arrow[maps to, from=4-1, to=5-2]
            \arrow[from=2-1, to=4-1]
            \arrow[from=1-2, to=3-2]
            \arrow[from=3-2, to=5-2]
        \end{tikzcd}\]
        in $\Delta$ becomes a composition map: 
        \[ 
            \infer{C \ts{t}{s} C \to C}{X([2]) \to X([1])}
        \] 
        Associativity and unitality (for the section $e: M \to C$) follow by functoriality. Thus, a model of $\mathsf{Pth} \to \Delta$ is a small category, and a morphism is exactly a functor.
    \end{enumerate}
\end{example}
Corollary \ref{cor:anchored-bundle-as-nerve}, then, gives a monadicity result for $\w$-anchored bundles over $\w$-differential bundles.
\begin{proposition}
    $\w$-anchored bundles are models of the theory $(\Lambda^+)^{op} \to (\wone^*)^{op}$ (see Section \ref{sec:enriched-structures}).
\end{proposition}
\begin{proof}
    Note that the inclusion $\Lambda^+ \to \wone^1$ is a differential bundle, so it is a nerve. Then the diagram:
    % https://q.uiver.app/?q=WzAsNCxbMCwwLCJcXG1hdGhzZntBbmN9KFxcQykiXSxbMSwwLCJbXFx3b25lXiosXFxDXSJdLFsxLDEsIltcXExhbWJkYV4rLCBcXENdIl0sWzAsMSwiXFxtYXRoc2Z7REJ1bn0oXFxDKSJdLFsxLDJdLFszLDIsIiIsMix7InN0eWxlIjp7InRhaWwiOnsibmFtZSI6Imhvb2siLCJzaWRlIjoidG9wIn19fV0sWzAsM10sWzAsMV0sWzAsMiwiIiwxLHsic3R5bGUiOnsibmFtZSI6ImNvcm5lciJ9fV1d
    \[\begin{tikzcd}
        {\mathsf{Anc}(\w)} & {[\wone^*,\w]} \\
        {\mathsf{DBun}(\w)} & {[\Lambda^+, \w]}
        \arrow[from=1-2, to=2-2]
        \arrow[hook, from=2-1, to=2-2]
        \arrow[from=1-1, to=2-1]
        \arrow[from=1-1, to=1-2]
        \arrow["\lrcorner"{anchor=center, pos=0.125}, draw=none, from=1-1, to=2-2]
    \end{tikzcd}\]
    exhibits $\w$-anchored bundles as models of a $(\Lambda)^{op}$-theory.
\end{proof}
As a corollary, we see that the category of anchored bundles is locally presentable.
\begin{corollary}
    The category of anchored bundles is locally presentable, and $(\wone^*)^{op} \hookrightarrow \mathsf{Anc}(\w)$ is dense.
\end{corollary}
In \cite{Bourke2019}, the authors identify exactly those monads on a locally presentable $\vv$-category that correspond to the models of a theory.
Recall the notation that the category of algebras for a monad is $\C^T$ and the category of free coalgebras is $\C_T$. For a dense subcategory $\a \hookrightarrow \C$, use $\a_T$ for the category of free algebras over objects in $\a$. Recall that the Lawvere theory \[K:\mathsf{FinSet} \to \th\] for a filtered-colimit-preserving monad $\mathbb{T}$ on $\s$ may be re-derived as the full subcategory of free algebras over finite sets, $K_T: \mathsf{FinSet}_{\mathbb{T}} \hookrightarrow \s_{\mathbb{T}} \hookrightarrow \s^{\mathbb{T}}$. Nervous monads abstract this property.
\begin{definition}
    Let $\C$ be a locally presentable $\vv$-category, with $K:\a \to \C$ a dense sub-$\vv$-category. 
    A $\vv$-monad $\mathbb{T}$ over $\C$ is \emph{$K$-nervous} if
    \begin{enumerate}
        \item the inclusion $K_T: \a_T \hookrightarrow \C^T$ is dense;
        \item the following diagram is a pullback in $\vv$CAT:
        % https://q.uiver.app/?q=WzAsNCxbMCwxLCJcXEMiXSxbMSwxLCJbXFxhXntvcH0sIFxcdl0iXSxbMSwwLCJbXFxhX1Ree29wfSwgXFx2XSJdLFswLDAsIlxcQ15UIl0sWzAsMSwiIiwxLHsic3R5bGUiOnsidGFpbCI6eyJuYW1lIjoiaG9vayIsInNpZGUiOiJ0b3AifX19XSxbMiwxXSxbMywwXSxbMywyXV0=
        \[\begin{tikzcd}
            {\C^T} & {[\a_T^{op}, \vv]} \\
            \C & {[\a^{op}, \vv]}
            \arrow[hook, from=2-1, to=2-2]
            \arrow[from=1-2, to=2-2]
            \arrow[from=1-1, to=2-1]
            \arrow[from=1-1, to=1-2]
        \end{tikzcd}\]
    \end{enumerate}
\end{definition}

\begin{theorem}[\cite{Bourke2019}]
    Let $K:\a \hookrightarrow \mathbb{C}$ be a dense sub-$\vv$-category of a cocomplete $\vv$-category $\,\C$.
    There is an equivalence of categories between $\a$-theories and $\a$-nervous monads on $\,\C$.
\end{theorem}

The first step in showing that algebroids are monadic over anchored bundles is to construct a dense subcategory of $\mathsf{Anc}(\w)$ that plays the role of the $V$-prolongations from Definition \ref{def:monoidal-category}. The enriched framework makes this straightforward: prolongations are weighted limits, with which we may freely complete $\wone^*$.

\begin{definition}
    \label{def:prol}
    Consider the category $\mathsf{Anc}(\w)$ as a representable tangent category, and take the dual tangent structure on $\mathsf{Anc}(\w)^{op}$.
    The category $\prol$ is defined as the full subcategory of $\mathsf{Anc}(\w)^{op}$ whose objects are generated by the prolongations of the Yoneda embedding $\yon:\wone^* \to \mathsf{Anc}(\w)$.
\end{definition}

For any small $\w$-category $\C$, the free completion of $\C$ is the opposite $\w$-category of the category of copresheaves on $\C$.\footnote{This is the dual statement of the classical theorem that the category of presheaves is the free cocompletion of a small category $\C$.} Thus, $\prol$ is the free completion of $\wone^1$ with prolongations. Therefore, a choice of prolongations on an anchored bundle $A$ determines a unique functor $\prol \to \C$ given by right Kan extension (see e.g. \cite{Kelly1982}). The continuity of 
\[
    \mathsf{Anc}(\w)(-, A): \mathsf{Anc}(\w)^{op} \to \w 
\]
ensures that $\mathsf{Anc}(\w)(L_V, A)$ is $A_V$. 

By Theorem \ref{thm:weil-nerve}, the category of involution algebroids in $\C$ is a full sub-$\w$-category of $[\wone, \C]$ that has a forgetful functor down to the category of anchored bundles. A consequence of Theorem \ref{thm:iso-of-cats-inv-emcs} is that a functor $\wone \to \C$ is an involution algebroid if and only if the precomposition \[\wone^* \hookrightarrow \wone \to \C\] restricts to an anchored bundle, with each $V$ sent to the $V$-prolongation of this anchored bundle. In other words, algebroids are models of the following enriched theorem (as in Definition \ref{def:nerve-theory}).

\begin{definition}
    \label{def:weil-theory}
    Define the $\prol^{op}$-theory of algebroids as the functor
    \[
        a: \prol \to \wone.
    \]
    This functor is bijective-on-objects by definition, and satisfies the nerve condition because the $V$-prolongation of the tangent bundle is $T^V$, so each presheaf
    \[
        \wone(a-, V):\prol \to \w   
    \] 
    is an $\prol$-nerve.
\end{definition}

The Weil nerve, then, translates to the following characterization of the category of involution algebroids in a tangent category $\C$. 
\begin{theorem}%
    \label{thm:pullback-in-cat-of-cats-inv-algd}
    The category of involution algebroids with chosen prolongations in a tangent category $\C$ is precisely the pullback in $\w\mathsf{CAT}$:
    \begin{equation}\label{eq:prol2}
        \input{TikzDrawings/Ch5/prol2.tikz}
    \end{equation}
    % where $w$ is the reflector from $\mathsf{Anc}(\w)$ to $\w$ restricted to the full subcategory $\prol$. 
    Consequently, in $\w$ involution algebroids are monadic over anchored bundles.
\end{theorem}
\begin{proof}
    Recall the correspondence
    \[
        \infer{(\hat A, \alpha): \wone \to \C \in \mathsf{TangCat}_{lax}}
        {\bar A: \wone \to \C \in \w\cat}
    \]
    If
    \[
        \wone^* \to \prol \to \wone \to \C  
    \]
    determines an anchored bundle $(\pi:A \to M, \xi, \lambda, \anc)$ whose $V$-coprolongation is $\hat A.V$, so that $\hat A$ is the nerve of an involution algebroid by Corollary \ref{cor:the-prolongation-description}.
\end{proof}

Thus, we may characterize $\w$-involution algebroids as algebras for a $\prol$-nervous monad on the category of $\w$-anchored bundles.
\begin{corollary}
    The category of involution algebroids in $\w$ is equivalent to the category of algebras for the $\prol$-nervous monad on $\mathsf{Anc}(\w)$ generated by the theory
    \[
        a: \prol \to \wone
    \]
    from Definition \ref{def:weil-theory}, using Theorem 19 from \cite{Bourke2019}.
\end{corollary}

\begin{remark}
    The construction of involution algebroids as the category of models for a $\prol$-theory is remarkably similar to the original nerve construction for categories in \cite{Segal1974}, with the symmetric nerve construction for groupoids replacing $\Delta$ with the category of finite sets $\Sigma$ (see Example 44 (iv) of \cite{Bourke2019}). Each construction truncates the original category to two objects and builds a new category with a bijective set of objects by freely adding limits to the two-object truncation.
\end{remark}


\section{The infinitesimal approximation of a groupoid}%
\label{sec:inf-nerve-of-a-gpd}

This section contains the main theorem of the chapter, namely that there is an adjunction
% https://q.uiver.app/?q=WzAsMixbMCwwLCJcXG1hdGhzZntHcGR9KFxcdykiXSxbMSwwLCJcXG1hdGhzZntJbnZ9KFxcdykiXSxbMCwxLCIiLDAseyJjdXJ2ZSI6LTJ9XSxbMSwwLCIiLDAseyJjdXJ2ZSI6LTJ9XSxbMiwzLCIiLDAseyJsZXZlbCI6MSwic3R5bGUiOnsibmFtZSI6ImFkanVuY3Rpb24ifX1dXQ==
\[\begin{tikzcd}
    {\mathsf{Gpd}(\w)} & {\mathsf{Inv}(\w)}
    \arrow[""{name=0, anchor=center, inner sep=0}, curve={height=-12pt}, from=1-1, to=1-2]
    \arrow[""{name=1, anchor=center, inner sep=0}, curve={height=-12pt}, from=1-2, to=1-1]
    \arrow["\dashv"{anchor=center, rotate=-90}, draw=none, from=0, to=1]
\end{tikzcd}\]
This adjunction follows by inducing a nerve/realization context on the category of groupoids in $\w$:
\[
    \partial: \wone^{op} \to \mathsf{Gpd}(\w).   
\]
A simplified version of this functor appeared in Proposition \ref{prop:lin-approx-gph}, the linear approximation of a reflexive graph. 
In fact, $\partial$ will be exactly the free groupoid over the graph $\partial$ from Proposition \ref{prop:lin-approx-gph} (if we had worked in the category of reflexive graphs equipped with an involution).
Recall that groupoids and Weil spaces  are both models of a sketch (recall Section \ref{sec:tang-cats-enrichment}). 
The symmetry of the theories gives two equivalent presentations of groupoids in $\w$:
\begin{definition}%
    \label{def:w-gpd}
    The tangent category of $\w$-groupoids is equivalently
    \begin{enumerate}[(i)]
        \item the tangent category of internal groupoids in $\w$, where the tangent structure is computed pointwise;
        \item the tangent category of transverse-limit-preserving functors $\wone \to \mathsf{Gpd}(\s)$ with the cofree tangent structure from Observation \ref{obs:cofree-tangent-cat}.
    \end{enumerate}
\end{definition}
It is important to note that the tangent structure on $\w$-groupoids is representable, as it is a cartesian closed category with an infinitesimal object given by $\wone \hookrightarrow \w \hookrightarrow \mathsf{Gpd}(\w)$ (a proof that $\mathsf{Gpd}(\C)$ is cartesian closed for a cartesian closed category $\C$ with sufficient limits a may be found in Section B2.3 of \cite{Johnstone2002}).

There are two classes of ``trivial'' $\w$-groupoids that will be useful.
\begin{example}
    ~\begin{enumerate}[(i)]
        \item Every small groupoid $G$ may be regarded as a groupoid in $\w$ as the constant functor $\wone \to \mathsf{Gpd}(\s)$ sending $V$ to the groupoid $G$ preserves transverse limits, we may then apply the symmetry of theories to find a groupoid in the category of Weil spaces. 
        \item Every object $M$ in a category has a cofree groupoid given by $s,t:M = M$, this is the \emph{discrete} groupoid whose only morphisms are the identity maps. Thus, every Weil space $M \in \w$ has a corresponding discrete groupoid (this will often be written $M = M$ to denote the discrete groupoid over $M$).
    \end{enumerate}
\end{example}

Using the cartesian closure of $\w$ and the full subcategories of trivial and discrete groupoids, we have the following:
\begin{observation}
    The category of $\w$-groupoids is both
    \begin{enumerate}[(i)]
        \item a $\w$-category, with powers by $\w$. The power of a $\w$-groupoid $\mathcal{G} = s,t:G \to M$ by a Weil space $E$ is given by the $\w$-groupoid
        \[
            [E, \mathcal{G}] = \{[E,s], [E, t]: [E,G] \to [E, M], \hspace{0.15cm} [E,e]: [E, M] \to [E, G]  \};
        \]
        \item a $\mathsf{Gpd}$-enriched category with powers by small groupoids. The power of a $\w$-groupoid $\mathcal{G}:\mathsf{Gpd} \to \w$ by a groupoid $\mathcal{H}$ is the $\w$-groupoid
        \[
            [\mathcal{H}, \mathcal{G}](V) = [\mathcal{H}, \mathcal{G}(V)]_{\mathsf{Gpd}}.
        \]
    \end{enumerate}
\end{observation}

The following two $\w$-groupoids form the basic building block for the main result.
\begin{example}
    ~\begin{enumerate}[(i)]
        \item Set $D := \yon W$ in $\w$. The discrete groupoid $D = D$ represents the tangent functor internally, and the copresheaf $\mathsf{Gpd}(D, -): \mathsf{Gpd}(\w) \to \w$ sends a groupoid $s,t:G \to M$ to $TM$.
        \item The arrow groupoid is the free groupoid generated by the graph:
        \[\bullet \to \bullet\]
        Write the trivial $\w$-groupoid on this groupoid as $I$. Note that the power by $I$ will send a groupoid to its ``arrow groupoid'' $\mathcal{G}^\to$, the groupoid whose objects are arrows in $\mathcal{G}$, with a map $u \to v$ being a commuting square.
        It follows that $\mathsf{Gpd}(I, \mathcal{G})$ is the space of arrows of the groupoid, $\mathsf{Gpd}(I \x I, \mathcal{G})$ the space of commuting squares, and so on.
    \end{enumerate}
\end{example}

The next proposition gives the necessary properties of $I$ to construct the Lie derivative.
\begin{lemma}%
    \label{lem:arrow-gpd-facts}
    ~\begin{enumerate}[(i)]
        \item For every groupoid $\mathcal{G} = s,t:G \to M$, there is an isomorphism
        \[
            G \ts{t}{t} G \ts{s}{t} G \cong \mathsf{Gpd}(\w)(I \x I, \mathcal{G});
        \]
        this corresponds to the unique filler for the diagram
        % https://q.uiver.app/?q=WzAsNCxbMCwxLCJcXGJ1bGxldCJdLFswLDAsIlxcYnVsbGV0Il0sWzEsMCwiXFxidWxsZXQiXSxbMSwxLCJcXGJ1bGxldCJdLFswLDEsInUiXSxbMSwyLCJ2Il0sWzMsMiwidyIsMl0sWzAsMywiXFxleGlzdHMhIHdeey0xfSBcXG8gdiBcXG8gdSIsMix7InN0eWxlIjp7ImJvZHkiOnsibmFtZSI6ImRhc2hlZCJ9fX1dXQ==
        \[\begin{tikzcd}
            \bullet & \bullet \\
            \bullet & \bullet
            \arrow["u", from=2-1, to=1-1]
            \arrow["v", from=1-1, to=1-2]
            \arrow["w"', from=2-2, to=1-2]
            \arrow["{\exists! w^{-1} \o v \o u}"', dashed, from=2-1, to=2-2]
        \end{tikzcd}\]
        \item The arrow groupoid has a semigroup with a zero structure (like an infinitesimal object $D$), and the multiplication is the coequalizer
        % https://q.uiver.app/?q=WzAsNSxbMCwxLCJJIFxceCBJIl0sWzIsMSwiSSBcXHggSSJdLFszLDEsIkkiXSxbMSwwLCIxIFxceCBJIl0sWzEsMiwiSSBcXHggMSJdLFswLDMsIiEgXFx4IEkiXSxbMywxLCJzIFxceCBJIl0sWzAsNCwiSSBcXHggISIsMl0sWzQsMSwiSSBcXHggcyIsMl0sWzEsMiwibSJdXQ==
        \[\begin{tikzcd}
            & {1 \x I} \\
            {I \x I} && {I \x I} & I \\
            & {I \x 1}
            \arrow["{! \x I}", from=2-1, to=1-2]
            \arrow["{s \x I}", from=1-2, to=2-3]
            \arrow["{I \x !}"', from=2-1, to=3-2]
            \arrow["{I \x s}"', from=3-2, to=2-3]
            \arrow["m", from=2-3, to=2-4]
        \end{tikzcd}\]

    \end{enumerate}
\end{lemma}
The dual result of part $(ii)$ is somewhat more obvious to see, as the following fork is always an equalizer in $\w$ for a groupoid $G$:
% https://q.uiver.app/?q=WzAsMyxbMSwwLCJHXlxcc3F1YXJlIl0sWzIsMCwiR15cXHNxdWFyZSJdLFswLDAsIkciXSxbMCwxLCJHXntlXnMgXFx4IEl9IiwwLHsib2Zmc2V0IjotMn1dLFswLDEsIkdee0kgXFx4IGVec30iLDIseyJvZmZzZXQiOjF9XSxbMiwwLCJHXntcXGJveGRvdH0iXV0=
\[\begin{tikzcd}
    G & {G^\square} & {G^\square}
    \arrow["{G^{e[s] \x I}}", shift left=2, from=1-2, to=1-3]
    \arrow["{G^{I \x e[s]}}"', shift right=1, from=1-2, to=1-3]
    \arrow["{G^{m}}", from=1-1, to=1-2]
\end{tikzcd}\]
(where we write $G$ as the object of arrows and $\mathsf{Gpd}(\w)(I\x I, G)$ as $G^\square$). 
Note that $G^{e[s] \x I}$ and $G^{I \x e[s]}$ correspond to the idempotents on $G^\square$:
% https://q.uiver.app/?q=WzAsMTIsWzIsMCwiXFxidWxsZXQiXSxbMiwxLCJcXGJ1bGxldCJdLFszLDAsIlxcYnVsbGV0Il0sWzMsMSwiXFxidWxsZXQiXSxbNCwwLCJcXGJ1bGxldCJdLFs1LDAsIlxcYnVsbGV0Il0sWzQsMSwiXFxidWxsZXQiXSxbNSwxLCJcXGJ1bGxldCJdLFswLDAsIlxcYnVsbGV0Il0sWzAsMSwiXFxidWxsZXQiXSxbMSwwLCJcXGJ1bGxldCJdLFsxLDEsIlxcYnVsbGV0Il0sWzAsMSwicSIsMl0sWzAsMiwidSJdLFsyLDMsInYiXSxbMSwzLCJ3IiwyXSxbNCw1LCJ1Il0sWzQsNiwiIiwwLHsibGV2ZWwiOjIsInN0eWxlIjp7ImhlYWQiOnsibmFtZSI6Im5vbmUifX19XSxbNSw3LCIiLDIseyJsZXZlbCI6Miwic3R5bGUiOnsiaGVhZCI6eyJuYW1lIjoibm9uZSJ9fX1dLFs2LDcsInUiLDJdLFs4LDksInEiLDJdLFsxMCwxMSwicSJdLFs4LDEwLCIiLDEseyJsZXZlbCI6Miwic3R5bGUiOnsiaGVhZCI6eyJuYW1lIjoibm9uZSJ9fX1dLFs5LDExLCIiLDEseyJsZXZlbCI6Miwic3R5bGUiOnsiaGVhZCI6eyJuYW1lIjoibm9uZSJ9fX1dLFsxNCwxNywiR157SSBcXHggZV5zfSIsMCx7InNob3J0ZW4iOnsic291cmNlIjozMCwidGFyZ2V0IjozMH0sInN0eWxlIjp7InRhaWwiOnsibmFtZSI6Im1hcHMgdG8ifX19XSxbMTIsMjEsIkdee2VecyBcXHggSX0iLDIseyJzaG9ydGVuIjp7InNvdXJjZSI6MzAsInRhcmdldCI6MzB9LCJzdHlsZSI6eyJ0YWlsIjp7Im5hbWUiOiJtYXBzIHRvIn19fV1d
\[\begin{tikzcd}
    \bullet & \bullet & \bullet & \bullet & \bullet & \bullet \\
    \bullet & \bullet & \bullet & \bullet & \bullet & \bullet
    \arrow[""{name=0, anchor=center, inner sep=0}, "q"', from=1-3, to=2-3]
    \arrow["u", from=1-3, to=1-4]
    \arrow[""{name=1, anchor=center, inner sep=0}, "v", from=1-4, to=2-4]
    \arrow["w"', from=2-3, to=2-4]
    \arrow["u", from=1-5, to=1-6]
    \arrow[""{name=2, anchor=center, inner sep=0}, Rightarrow, no head, from=1-5, to=2-5]
    \arrow[Rightarrow, no head, from=1-6, to=2-6]
    \arrow["u"', from=2-5, to=2-6]
    \arrow["q"', from=1-1, to=2-1]
    \arrow[""{name=3, anchor=center, inner sep=0}, "q", from=1-2, to=2-2]
    \arrow[Rightarrow, no head, from=1-1, to=1-2]
    \arrow[Rightarrow, no head, from=2-1, to=2-2]
    \arrow["{G^{I \x e[s]}}", shorten <=10pt, shorten >=10pt, Rightarrow, maps to, from=1, to=2]
    \arrow["{G^{e[s] \x I}}"', shorten <=10pt, shorten >=10pt, Rightarrow, maps to, from=0, to=3]
\end{tikzcd}\]
Also note that a commuting square is equalized by these two maps if and only if $u = q = id$, which forces $w = v$; these commuting squares are exactly the image of $G^m$.

The semigroup structure on $I$ represents the map sending
% https://q.uiver.app/?q=WzAsNixbMCwxLCJYIl0sWzAsMCwiWSJdLFsxLDAsIlgiXSxbMiwwLCJZIl0sWzIsMSwiWCJdLFsxLDEsIlgiXSxbMCwxLCJ1IiwxXSxbMiwzLCJ1IiwxXSxbNCwzLCJ1IiwxXSxbMiw1LCIiLDAseyJsZXZlbCI6Miwic3R5bGUiOnsiaGVhZCI6eyJuYW1lIjoibm9uZSJ9fX1dLFs1LDQsIiIsMCx7ImxldmVsIjoyLCJzdHlsZSI6eyJoZWFkIjp7Im5hbWUiOiJub25lIn19fV0sWzYsOSwiIiwwLHsic2hvcnRlbiI6eyJzb3VyY2UiOjQwLCJ0YXJnZXQiOjMwfSwic3R5bGUiOnsidGFpbCI6eyJuYW1lIjoibWFwcyB0byJ9fX1dXQ==
\[\begin{tikzcd}
    Y & X & Y \\
    X & X & X
    \arrow[""{name=0, anchor=center, inner sep=0}, "u"{description}, from=2-1, to=1-1]
    \arrow["u"{description}, from=1-2, to=1-3]
    \arrow["u"{description}, from=2-3, to=1-3]
    \arrow[""{name=1, anchor=center, inner sep=0}, Rightarrow, no head, from=1-2, to=2-2]
    \arrow[Rightarrow, no head, from=2-2, to=2-3]
    \arrow[shorten <=13pt, shorten >=10pt, Rightarrow, maps to, from=0, to=1]
\end{tikzcd}\]
% Multiplication by the zero $id_0$ corresponds to the map sending a morphism to the identity on its source.
% \[
%   G \xrightarrow[]{s} M \xrightarrow[]{e} G   
% \]
% This makes the morphisms defining the Lie derivative of a groupoid representable:
% % https://q.uiver.app/?q=WzAsNyxbMSwwLCJURyJdLFsyLDAsIlRHIl0sWzAsMCwiQSJdLFszLDBdLFswLDEsIkEiXSxbMSwxLCJcXG1hdGhzZntHcGR9KFxcdykoRCBcXHggSSwgRykiXSxbMiwxLCJcXG1hdGhzZntHcGR9KFxcdykoRCBcXHggSSwgRykiXSxbMCwxLCJULihpIFxcbyBzKSIsMCx7Im9mZnNldCI6LTF9XSxbMCwxLCIoMCBcXG8gcCkuRyIsMix7Im9mZnNldCI6MX1dLFsyLDBdLFs1LDYsIihEIFxceCAoXFxpb3RhIFxcbyBzKSleKiIsMCx7Im9mZnNldCI6LTF9XSxbNSw2LCIoKFxceGkgXFxvICEpIFxceCBJKV4qIiwyLHsib2Zmc2V0IjoxfV0sWzQsNV1d
% \[\begin{tikzcd}
%   A & TG & TG & {} \\
%   A & {\mathsf{Gpd}(\w)(D \x I, G)} & {\mathsf{Gpd}(\w)(D \x I, G)}
%   \arrow["{T.(i \o s)}", shift left=1, from=1-2, to=1-3]
%   \arrow["{(0 \o p).G}"', shift right=1, from=1-2, to=1-3]
%   \arrow[from=1-1, to=1-2]
%   \arrow["{(D \x (\iota \o s))^*}", shift left=1, from=2-2, to=2-3]
%   \arrow["{((\xi \o !) \x I)^*}"', shift right=1, from=2-2, to=2-3]
%   \arrow[from=2-1, to=2-2]
% \end{tikzcd}\]
% The category $\mathsf{Gpd}(\w)$ is cocomplete, so it is possible to represent this functor.
Now look at the $\partial$ defined in Proposition \ref{prop:lin-approx-gph}, and take the same colimit in $\mathsf{Gpd}(\w)$. 
Groupoids are monadic over reflexive graphs with an involution, so this is essentially the free groupoid over that graph (as the free functor is a left adjoint and therefore preserves colimits).
\begin{definition}%
    \label{def:partial-gpd}
    Define the $\w$-groupoid $\partial$ to be the $\w$-coequalizer
    % https://q.uiver.app/?q=WzAsMyxbMCwwLCJEXFx4IEkiXSxbMSwwLCJEXFx4IEkiXSxbMiwwLCJcXHBhcnRpYWwiXSxbMCwxLCJlIFxceCBJIiwwLHsib2Zmc2V0IjotMX1dLFswLDEsIkQgXFx4IGVecyIsMix7Im9mZnNldCI6MX1dLFsxLDJdXQ==
    \[\begin{tikzcd}
        {D\x I} & {D\x I} & \partial
        \arrow["{e \x I}", shift left=1, from=1-1, to=1-2]
        \arrow["{D \x e[s]}"', shift right=1, from=1-1, to=1-2]
        \arrow[from=1-2, to=1-3]
    \end{tikzcd}\]
    where $e = 0 \o !, e[s] = i \o s$. Note that just as in Proposition \ref{prop:lin-approx-gph}, $\partial$ is an anchored bundle.
\end{definition}

The properties of the arrow groupoid make it possible to prove the following theorem.

\begin{theorem}%
    \label{thm:inf-nerve}
    The object $\partial$ determines a cartesian $\w$-functor
    \[
        \partial(-): \wone^{op} \to \mathsf{Gpd}(\w)    
    \]
    that is both an infinitesimal object in $\w$-groupoids and an involution algebroid in $\mathsf{Gpd}(\w)^{op}$.
\end{theorem}
\begin{proof}
    Write the pushout powers of $\xi:1 \to \partial$ as $\partial(n)$, and for any Weil algebra $V = W_{n(1)} \ox \dots \ox W_{n(k)}$, write $\partial(V) = \partial(n_1) \x \dots \x \partial(n_k)$. Composition for internal categories will be written in the diagramattic order, writing composition as an infix semicolon ``;'' to keep it distinct from composition of morphisms in $\w$. 
    
    The proof has two main steps:
    \begin{enumerate}
        \item For every Weil algebra $V$, there is an isomorphism
        \[
            \partial_V \cong \partial(V)    
        \]
        where $\partial_V$ is the $V$-prolongation of the anchored bundle in $\mathsf{Gpd}(\w)^{op}$.
        \item The universal lift for the anchored bundle, given as a coequalizer
        % https://q.uiver.app/?q=WzAsMyxbMCwwLCJcXHBhcnRpYWwgXFx4IFxccGFydGlhbCAiXSxbMSwwLCJcXHBhcnRpYWxcXHggXFxwYXJ0aWFsIl0sWzIsMCwiXFxwYXJ0aWFsIl0sWzAsMSwiZV5cXHBhcnRpYWwgXFx4IFxccGFydGlhbCIsMCx7Im9mZnNldCI6LTJ9XSxbMCwxLCJlXlxccGFydGlhbCBcXHggXFxwYXJ0aWFsICIsMix7Im9mZnNldCI6Mn1dLFsxLDIsIlxcYm94ZG90Il1d
        \[\begin{tikzcd}
            {\partial \x \partial } & {\partial\x \partial} & \partial
            \arrow["{e^\partial \x \partial}", shift left=2, from=1-1, to=1-2]
            \arrow["{e^\partial \x \partial }"', shift right=2, from=1-1, to=1-2]
            \arrow["\boxdot", from=1-2, to=1-3]
        \end{tikzcd}\]
        is a semigroup.
    \end{enumerate}
    From (1) and (2), we may infer that the object $\partial$ is an infinitesimal object. The semigroup map $\boxdot$ is commutative and has a zero by (2), and satisfies all of the couniversality axioms by (1). 
    \begin{enumerate}
        \item   The proof of this step follows by induction on the width (Definition \ref{def:truncated-wone}) of the Weil algebra $V$, where the cases $0, 1$ both hold by definition.
        In the case $n=2$, we need only check this holds for $\partial \x \partial \cong \partial_{WW}$:
        %   For $L_2$ and $\partial \x \partial$, first note there is the injection:
        \[\input{TikzDrawings/Ch5/induce-map-from-coprol.tikz}\]
        For any $G$, a map $X: \partial \x \partial \to G$ corresponds to a commuting square in $T^2G$ of the form
        \[\input{TikzDrawings/Ch5/comm-square-T2G.tikz}\]
        so that $T.e \o v = id$ and $e.T \o u = id$.
        It follows that $u = (e.T \o v)^{-1};(T.e \o u);v$ (Lemma \ref{lem:arrow-gpd-facts} (i)), and that precomposition with the uniquely induced map determines $(\bar{u},v):X \to G^{L(2)}$, where
        \[ T.0 \o \bar{u} = u.\]
        Observe that any pair $(\bar{u}, v):X \to  G^{L(2)}$ determines a square
        \[
            \input{TikzDrawings/Ch5/inverse-of-prol.tikz}
        \]
        where $T.e \o v = id$ and $T.e \o \bar{u} = id$.
        Note that $T.0 \o T.p \o T.0 \o \bar{u} = T.0 \o \bar{u}$, and that the bottom horn is $u = (e.T \o v)^{-1};(T.0 \o \bar{u});v$; now check
        \begin{align*}
            T.e \o u
            % &= T.e \o \((e.T \o v)^{-1};(T.0 \o \bar{u});v \) \\
                & = (T.e \o e.T \o v)^{-1};(T.e \o T.0 \o \bar{u});(T.e \o v) \\
                & = id; T.0 \o \bar{u}; T.e \o v = T.0 \o \bar{u}
        \end{align*}
        and
        \begin{align*}
            e.T \o u
                & = e.T \o (e.T \o v)^{-1};(T.0 \o \bar{u});v                 \\
                & = (e.T \o e.T \o v)^{-1};(e.T \o T.0 \o \bar{u});(e.T \o v) \\
                & = (e.T \o v)^{-1};id;(e.T \o v) = id
        \end{align*}
        thus determining a map $X \to G^{\partial \x \partial}$.
        The two maps are inverse to each other, giving an isomorphism.
    
        For the inductive case, look at the prolongations of anchored bundles and recall that
        \[
            A_{UV} \boxtimes_M A_Z  =
            A_{UV} \boxtimes_{\prol(U,A)} A_{UZ}
        \]
        Where $A_{UV}$ and $A_{UZ}$ are treated as spans over $A_U$, so that
        \begin{gather*}
            A_U \xleftarrow[]{id \boxtimes \pi^V} A_{UV} \xrightarrow[]{anc^U \boxtimes A_V} T^U.A_V\\
            A_U \xleftarrow[]{id \boxtimes \pi^Z} A_{UZ} \xrightarrow[]{anc^U \boxtimes A_Z} T^U.A_Z
        \end{gather*}
        and observe that their span composition is
        \[
            \input{TikzDrawings/Ch5/pullback-uvz.tikz}
        \]
        % Now, assume that the hypothesis holds for any two of $U, V, Z \in \wone$ and use the co-continuity of $X \x (-)$ in a cartesian closed category.
        % \begin{align*}
        %   L_{UVZ} & = L_{UV} \boxplus_U L_{UZ}
        %   % \\
        %   % &= (\partial(U) \x \partial(V)) \po_{(id)}{\pi_0} (D(V) \x \partial(U) \x \partial(Z)) \\
        %   % &= \partial(U) \x (\partial(U) \po_{}{} \partial(Z)) \\
        %   % &= \partial(V) \x \partial(U)\x \partial(Z)
        % \end{align*}
        So it now suffices to prove that the diagram
        \[\input{TikzDrawings/Ch5/partial-u-po.tikz}\]
        is a pushout. But by the inductive hypothesis,
        \[\input{TikzDrawings/Ch5/partial-no-u-po.tikz}\]
        so the result follows by the cocontinuity of $\partial_U \x (-)$, and $\mathsf{Gpd}(\w)$ is a cartesian closed category (so $X \x (-)$ is cocontinuous).
        \item Recall that the fork
        % https://q.uiver.app/?q=WzAsMyxbMCwwLCJJIFxceCBJIl0sWzEsMCwiSSBcXHggSSJdLFsyLDAsIkkiXSxbMSwyLCJtIl0sWzAsMSwiZV5zIFxceCBJIiwwLHsib2Zmc2V0IjotMX1dLFswLDEsIkkgXFx4IGVecyIsMix7Im9mZnNldCI6MX1dXQ==
        \[\begin{tikzcd}
            {I \x I} & {I \x I} & I
            \arrow["m", from=1-2, to=1-3]
            \arrow["{e[s] \x I}", shift left=1, from=1-1, to=1-2]
            \arrow["{I \x e[s]}"', shift right=1, from=1-1, to=1-2]
        \end{tikzcd}\]
        is a coequalizer. 
        Also note that $e^\partial$ is the coequalizer of $e, e[s]$. The commutativity of colimits then ensures that there is a multiplication map induced by the following diagram:
        % https://q.uiver.app/?q=WzAsOSxbMCwwLCIoRCBcXHggSSleMiJdLFsxLDAsIihEIFxceCBJKV4yIl0sWzIsMCwiXFxwYXJ0aWFsXjIiXSxbMCwxLCIoRCBcXHggSSleMiJdLFsxLDEsIihEIFxceCBJKV4yIl0sWzIsMSwiXFxwYXJ0aWFsXjIiXSxbMiwyLCJcXHBhcnRpYWwiXSxbMSwyLCJEXFx4IEkiXSxbMCwyLCJEXFx4IEkiXSxbMCwxLCIiLDAseyJvZmZzZXQiOi0xfV0sWzAsMSwiIiwyLHsib2Zmc2V0IjoxfV0sWzEsMl0sWzAsMywiIiwyLHsib2Zmc2V0IjoxfV0sWzAsMywiIiwyLHsib2Zmc2V0IjotMX1dLFsxLDQsIiIsMix7Im9mZnNldCI6MX1dLFsxLDQsIiIsMix7Im9mZnNldCI6LTF9XSxbMiw1LCIiLDIseyJvZmZzZXQiOjF9XSxbMiw1LCIiLDIseyJvZmZzZXQiOi0xfV0sWzMsNCwiIiwxLHsib2Zmc2V0IjotMX1dLFszLDQsIiIsMSx7Im9mZnNldCI6MX1dLFs0LDVdLFs1LDYsIiIsMSx7InN0eWxlIjp7ImJvZHkiOnsibmFtZSI6ImRhc2hlZCJ9fX1dLFs0LDddLFs3LDZdLFszLDhdLFs4LDcsIiIsMSx7Im9mZnNldCI6MX1dLFs4LDcsIiIsMSx7Im9mZnNldCI6LTF9XV0=
        \[\begin{tikzcd}
            {(D \x I)^2} & {(D \x I)^2} & {\partial^2} \\
            {(D \x I)^2} & {(D \x I)^2} & {\partial^2} \\
            {D\x I} & {D\x I} & \partial
            \arrow[shift left=1, from=1-1, to=1-2]
            \arrow[shift right=1, from=1-1, to=1-2]
            \arrow[from=1-2, to=1-3]
            \arrow[shift right=1, from=1-1, to=2-1]
            \arrow[shift left=1, from=1-1, to=2-1]
            \arrow[shift right=1, from=1-2, to=2-2]
            \arrow[shift left=1, from=1-2, to=2-2]
            \arrow[shift right=1, from=1-3, to=2-3]
            \arrow[shift left=1, from=1-3, to=2-3]
            \arrow[shift left=1, from=2-1, to=2-2]
            \arrow[shift right=1, from=2-1, to=2-2]
            \arrow[from=2-2, to=2-3]
            \arrow[dashed, from=2-3, to=3-3]
            \arrow[from=2-2, to=3-2]
            \arrow[from=3-2, to=3-3]
            \arrow[from=2-1, to=3-1]
            \arrow[shift right=1, from=3-1, to=3-2]
            \arrow[shift left=1, from=3-1, to=3-2]
        \end{tikzcd}\]
        Thus the multplication is associative, is commutative, and has a zero given by $1 \xrightarrow[]{0 \x s} D \x I$.
    \end{enumerate}
\end{proof}

Aan immediate corollary of Theorem \ref{thm:inf-nerve} is that the $\partial$ determines a nerve/ realization where the nerve factors through the category of involution algebroids.
This puts the Lie functor into a nerve/realization context (recall Definition \ref{def:realization}).
\begin{definition}%
    \label{def:lie-realization}
    The \emph{Lie realization} is the left Kan extension
    \[
        |-|_\partial = Lan_\partial: \mathsf{Inv}(\w) \to \mathsf{Gpd}(\w).  
    \]
\end{definition}

This functor is well behaved. First, ote that it preserved products:
\begin{lemma}
    The realization functor preserves products.
\end{lemma}
\begin{proof}
    Product preservation is a consequence of $|-|_\partial$ being the left Kan extension of a cartesian functor along a cartesian functor (\cite{Day1995}).
    Note that this implies $\int^v\partial(v) = 1$. 
\end{proof}
Next, we see that the realization of an involution algebroid has the same base space.

\begin{lemma}\label{lem:base-of-groupoid}
    The base space of the groupoid $\partial$ is $D^v$.
\end{lemma}
\begin{proof}
    When constructing the colimit of groupoids, the colimit's base space is the ordinary colimit for the diagram of the base spaces. The reflector from simplicial objects to groupoids preserves products, so it suffices to check that the base space of $\partial(n)$ is $D(n)$.

    The base space of $I$ is $1+1$, and the map $e^s_0$ is given by $\delta^- \o !$.
    Since $D$ is a discrete cubical object, its base space is $D \x 1$.
    Thus the coequalizer defining $\partial$ is
    \[\input{TikzDrawings/Ch5/coeq-def-partial.tikz}.\]
    A map $\gamma: D + D \to M$ is a pair of maps $\gamma_0, \gamma_1: D \to M$.
    We can see that $\gamma_0$ and $\gamma_1$ agree at $0$ (they are both $\gamma_0(0)$), and $\gamma(0)$ is a constant tangent vector. It follows that $\gamma_1 \o (id | id) = \gamma$.
\end{proof}

Now recall the co-Yoneda lemma: for any $\vv$-presheaf $A: \C^{op} \to \vv$,
\[
    F(C) = \int^{C' \in \C} \C(C,C') \ox F(C').
\]
In particular, for an involution algebroid in $\w$ (or any $\w$-presheaf on $\wone^{op}$),
\[
    A(U) = \int^{V \in \wone} \wone^{op}(U,V) \x A(V) = \int^{V \in \wone} \wone(V,U) \x A(V).
\]
\begin{lemma}\label{lem:as-a-presheaf}
    For any involution algebroid $A$,
    \[ A(R) = \int^{V \in \wone} A(V) \x D^V. \]
\end{lemma}
\begin{proof}
    Use the tangent structure on $\wone$ to regard it as a $\w$-enriched category:
    \begin{gather*}
        D^V = \yon(V) = \wone(V,-) = (U \mapsto \wone(V, U))  \\= (U \mapsto \wone(V, U\ox R)) = \wone(V,R).
    \end{gather*}
    The following computation gives the result:
    \[
        A(R) = \int^{V} \wone(V,R) \x A(V) = \int^{V} D^V \x A(V).
    \]
\end{proof}
We can now see that the base space of an involution algebroid is isomorphic to the base space of its realization. That is to say, the realization sends an involution algebroid over a Weil space $M$ to a groupoid over the Weil space $M$ (up to isomorphism).

\begin{proposition}\label{prop:lie-int-first-part}
    Let $A$ be an involution algebroid in $\w$. Then $|A|([0]) = A(R)$.
\end{proposition}
\begin{proof}
    Use the Yoneda lemma, and the fact that $\yon[0] = 1$ is a small projective so that $\mathsf{Gpd}(1,-)$ is $\w$-cocontinuous. Now apply Lemma \ref{lem:base-of-groupoid} and Lemma \ref{lem:as-a-presheaf}:
    \begin{align*}
        |A|([0]) & = \mathsf{Gpd}(1, |A|)                                                   \\
                 & = \mathsf{Gpd}\left(1, \int^{v \in \wone} A(v)\bullet \partial v \right) \\
                 & = \int^{v} A(v)\x \mathsf{Gpd}(1,  \partial v )                          \\
                 & = \int^{v}  A(v)\x D^v = A(R).
    \end{align*}
\end{proof}

Thus, as a final result, we have achieved an adjunction between the category of involution algebroids and groupoids in $\w$ that is product-preserving and stable over the base spaces.
\pagebreak 
\lie
\[% https://q.uiver.app/?q=WzAsMixbMCwwLCJcXG1hdGhzZntHcGR9KFxcQykiXSxbMSwwLCJcXG1hdGhzZntJbnZ9KFxcQykiXSxbMCwxLCJOX1xccGFydGlhbCIsMCx7ImN1cnZlIjotMn1dLFsxLDAsInwtfF9cXHBhcnRpYWwiLDAseyJjdXJ2ZSI6LTJ9XSxbMiwzLCIiLDAseyJsZXZlbCI6MSwic3R5bGUiOnsibmFtZSI6ImFkanVuY3Rpb24ifX1dXQ==
\begin{tikzcd}
    {\mathsf{Gpd}(\w)} & {\mathsf{Inv}(\w)}
    \arrow[""{name=0, anchor=center, inner sep=0}, "{N_\partial}", curve={height=-12pt}, from=1-1, to=1-2]
    \arrow[""{name=1, anchor=center, inner sep=0}, "{|-|_\partial}", curve={height=-12pt}, from=1-2, to=1-1]
    \arrow["\dashv"{anchor=center, rotate=-90}, draw=none, from=0, to=1]
\end{tikzcd}\]

\begin{remark}
    Just as the introduction to this thesis begins with the work of Charles Ehresmann, we should take a moment to see how the Lie realization relates to his original research into sketch theory. Rather than sketches, we use their closely related cousins, \emph{essentially algebraic theories}, which in this case are small, finitely complete $\w$-categories. We may regard $\mathsf{Gpd}(\w)$ and $\mathsf{Inv}(\w)$ as $\w$-functor categories
    \[
        \mathsf{Lex}(\th_{\mathsf{Gpd}}, \w), \hspace{0.5cm}\mathsf{Inv}(\th_{\mathsf{Gpd}}, \w),
    \]
    respectively (where $\mathsf{Lex}$ means finite-limit-preserving $\w$-functors). The functor $\partial:\wone^{op} \to \mathsf{Gpd}(\w)$ induces a left-exact $\w$-functor
    \[
        \hat{\partial}: \th_{\mathsf{Inv}} \to \th_{\mathsf{Gpd}}.
    \]
    This means the functor from Lie groupoids to Lie algebroids is induced by a morphisms of essentially algebraic theories, and may thus be presented as a morphism of sketches.  
\end{remark}


% \documentclass[main.tex]{subfiles} 

% \begin{document}
\chapter{Conclusions and future work}

\section{Conclusions}

We now take stock of what this thesis has accomplished.

\paragraph{Enriched essentially algebraic presentations of geometric structures}

This thesis has been concerned with categories of vector bundles and Lie algebroids in smooth manifolds.
While these are not models of a limit sketch in the category of smooth manifolds, classical results such as the Serre--Swann theorem or Vaintrob's presentation of Lie algebroids indicate that these categories do have some algebraic description.
By combining the results in Chapters \ref{ch:differential_bundles} and \ref{ch:involution-algebroids}, we can see that vector bundles and Lie algebroids are characterized by tangent categorical gadgets, differential bundles (Theorem \ref{iso-of-cats-dbun-sman}) and involution algebroids (Theorem \ref{thm:iso-of-cats-Lie}).
By combining these observations with the enriched perspective on tangent categories, the results in Section \ref{sec:enriched-structures} exhibit vector bundles and Lie algebroids as models of \emph{enriched sketches} (Proposition \ref{prop:Lambda-is-refl-subcat} and Theorem \ref{thm:pullback-in-cat-of-cats-inv-algd}), as can also be found in Chapter 6 of \cite{Kelly2005}.
Note that from this perspective, the construction of the Weil nerve of an involution algebroid is mostly important as an intermediate step.

\paragraph{New tangent structures for Lie algebroids and groupoids}

It is well known that the categories of Lie algebroids and groupoids have tangent structures induced by post-composition with the tangent functor, yielding the tangent algebroid and tangent groupoid, respectively.
The key result in Chapter \ref{chap:weil-nerve} exhibiting the category of Lie algebroids as transverse-limit-preserving cartesian-tangent functors $\wone \to \mathsf{SMan}$, then, gives the category of Lie algebroids a second tangent structure that corresponds to Martinez's \emph{prolongation} Lie algebroid. 
Similarly, the construction of the infinitesimal nerve of a local groupoid restricts to a new tangent structure on the category of Lie groupoids; in this case, the base space is the Lie algebroid of the groupoid $G$, and the total space is the Lie algebroid of the arrow groupoid $G^2$. 

While this may seem like a piece of formal category theory, it is closely related to symmetry reduction in classical mechanics using Lie theory. This goes back to Poincar\'{e}'s celebrated note \cite{poincare1901}, which introduced the \emph{Euler-Poincare} formulation of Lagrangian mechanics on a manifold equipped with a Lie group action, where the tangent bundle is replaced with a Lie algebra action (see \cite{marle2013} for a modern exposition). Poincar\'{e}'s formalism has been extended to Lie groupoids and Lie algebroids in \cite{Weinstein1996} and \cite{Martinez2001}, and the thesis \cite{fusca2018} investigates fluid mechanics through this lens. This work may be interpreted as using the novel tangent structures on Lie groupoids and Lie algebroids given in Chapters \ref{chap:weil-nerve} and \ref{ch:inf-nerve-and-realization}, and warrants further investigation.

\paragraph{Functorial semantics of Lie theory}
The work in Section \ref{sec:inf-nerve-of-a-gpd} puts the Lie derivative on a new formal grounding:
\begin{enumerate}[(i)]
    \item In the enriching category $\w$, the Lie derivative functor now arises as a nerve/realization context. This guarantees the existence of a left adjoint---the \emph{realization}---that constructs a Lie groupoid from an algebroid. The realization preserves products and is stable over the base space.
    \item It is only a small extension of the work in Section \ref{sec:inf-nerve-of-a-gpd} to show that the Lie derivative of a groupoid in a tangent category may generally be regarded as a weighted limit, where $A_V = \{\partial(V), G\}$ for each Weil prolongation of the involution algebroid of the groupoid.
\end{enumerate}
These appear to be new results, although in a nerve/realization approach they had previously appeared in Lie theory in the form of Sullivan's construction (\cite{Sullivan1977}), which takes the differential graded algebra of a Lie algebroid and constructs a simplicial set:
\[
    \mathsf{DGA}^{op}(\Omega(\Delta_n), A)
\]
where $\Omega(\Delta_n)$ is the de Rham cohomology of the $n$-simplex in cartesian space.

\section{Future work}

This section outlines various lines of research that were either cut from the thesis while writing due to time and/or space constraints, or new lines of research that the thesis-writing process has motivated but which have not yet been pursued.

\paragraph{Enriched sketches and Mackenzie theory}

Following \cite{Voronov2012}, Mackenzie theory refers to the body of research developed by Kirill Mackenzie and his collaborators into Lie theory, particularly structures like double Lie algebroids (\cite{Mackenzie1992}), VB-Lie algebroids (\cite{Bursztyn2016}), double Lie groupoids and so on.
Intuitively, a double Lie algebroid is a Lie algebroid in the category of Lie algebroids, while a VB-Lie algebroid is a vector bundle in the category of Lie algebroids.
These structures have an intuitive relationship with tensor products of sketches; that is, for limit sketches $A,B$ there is a chain of isomorphisms of categories:
\[
    \mathsf{Mod}(A, \mathsf{Mod}(B,\s)) \cong \mathsf{Mod}(A \ox B, \s) \cong \mathsf{Mod}(B, \mathsf{Mod}(A, \s)).
\]
We have already demonstrated that the Lie algebroids and vector bundles are $\w$-sketchable, so the natural next step is to revisit Mackenzie theory via this lens. Certain results such as the symmetry of partial Lie derivatives from \cite{Mackenzie1992} (given a double Lie groupoid, the order in which the Lie functor is applied doesn't matter) should be immediate.

\paragraph{A tangent categorical formalization of mechanics}

Several papers in synthetic differential geometry have translated aspects of Lagrangian and Hamiltonian mechanics into the synthetic setting (\cite{Bunge1984,Nishimura1997a}), and to a degree this has made it difficult to build enthusiasm for a tangent categorical presentation of mechanics. These approaches generally emphasize the ability to construct function spaces, or the use of a topos-theoretic internal language. The novel tangent structures for Lie algebroids and groupoids presented here, however, provide a new line of inquiry: to develop a unified framework for Lagrangian or Hamiltonian mechanics in a tangent category that agrees with the algebroid and groupoid approaches to mechanics.

% \paragraph{}
% \paragraph{The enriched nerve/realization paradigm and differential linear logic}
% A modest extension to the Bourke-Garner framework for nervous theories \cite{Bourke2019} allows for a \emph{monoidal} dense subcategory to act as a system of arities for a commutative algebraic theory. A monoidally dense functor is a strict monoidal functor:
% \[
%     m: \a^\ox \hookrightarrow \C^{\boxtimes}
% \]
% where $(-)\boxtimes(=)$ is cocontinuous in each argument,  and $N_m$ a fully faithful monoidal functor into $\vv$-presheaves on $\a$ with the Day convolution monoidal structure:
% \[
%     \C^{\boxtimes} \hookrightarrow \widehat{\a}^{\ox[Day]}
% \]
% (note that this means the reflection to $\C^\boxtimes$ is monoidal).
% Now, observe that the finite coproduct closure of infinitesimals in $\w$ is a monoidal dense subcategory, $D^+$, and furthermore differential object may be regarded as the concrete models of the $D^+$-theory:
% \begin{itemize}
%     \item Objects:$i \in D^+$,
%     \item Homs: $\d(i,j) = \N^{i \x j}$
%     \item 
% \end{itemize}




% \paragraph{The relationship between local Lie groupoids and local groupoids}

% Throughout \cref{ch:inf-nerve-and-realization} we identified cubical objects in $\w$ satisfying a certain universal property as \emph{local groupoids}.
% The definition of a local groupoid can apply to any tangent category, following the observation at the end of \Cref{sec:linear-approx-of-a-cubical-object}.
% It would useful to know if this does capture the notion of a local Lie groupoid, which is an involutive, reflexive graph equipped with an open set $U \subset G \ts{t}{s} G$ where multiplication is well defined. This would involve checking the local Lie groupoids admit a ``cubical nerve'' construction, and that the resulting cubical object satisfies the Segal conditions for local groupoids.


% \paragraph{Sharpening the Lie functor} The adjunction between $\w$-groupoids and involution algebroids should be compared with other adjunctions constructed w

% \paragraph{Lie integration}
% As stated in \cref{sec:Lie_algebroids}, the theory of Lie groupoids and algebroids is a generalization of the theory of Lie groups and Lie algebras.
% Two of the major theorems for Lie groups and algebras, called Lie II and Lie III (although Lie III is properly credited as the Cartan-Lie theorem), state that the category of finite-dimensional real Lie algebras is a coreflective subcategory of Lie groups.
% However, this result famously fails to generalize to Lie groupoids and algebroids, and exact conditions for when a Lie groupoid \emph{integrating} a Lie algebroid $A$ were provided in \cite{Crainic2003}.
% The original motivation for the development in \cref{chap:microlinear-nerve,ch:inf-nerve-and-realization} was to exhibit Lie integration via pure abstract nonsense - while that construction does warrant further development, at this point the idea of inducing a coreflection via purely abstract results seems to be a dead end.
% However, the cubical approach used in \cref{sec:inf-nerve-of-a-gpd} seems like a good fit with the Crainic and Fernandes approach of building a cubical set:
% \[
%     \mathsf{LieAlgd}((TI)(-), A):\mathsf{LieAlgd} \to \widehat{\square}
% \]
% and quotienting out a groupoid.
% ($I$ in this case is the unit interval, and $TI$ is the tangent Lie algebroid above it).
% Thus, revisiting the Weinstein construction combined with the enriched categorical machinery developed in \cref{ch:inf-nerve-and-realization} seems to be a viable path forwards.


% \end{document}

% % \documentclass[main.tex]{subfiles}

% \begin{document}

% Here talk about how the prolongation acts like a tangent functor. 
% 

\chapter{The Weil nerve of an algebroid}%
\label{chap:weil-nerve}\pagenote{
   This chapter has been given a new introduction to help with its exposition.
   Various typos have been fixed, but the most substantial changes are in Section \ref{sec:weil-nerve}, where the proof has been restructured to address some concerns brought up in Michael's comments.
}

The first three chapters of this thesis demonstrated that tangent categories allow for an essentially algebraic description of Lie algebroids by axiomatizing the behaviour of the tangent bundle, and showing that a Lie algebroid over a manifold $M$ is a ``generalized tangent bundle'', namely an \emph{involution algebroid}. This chapter will make precise the sense in which an involution algebroid is a generalized tangent bundle, by showing that the category of involution algebroids in a tangent category $\C$ is equivalent to a certain tangent-functor category from the free tangent category over a single object to $\C$, or more generally that there is a fully faithful functor
\[
	\mathsf{Inv}(\C) \hookrightarrow \mathsf{Tang_{Lax}}(\mathsf{FreeTangCat}(*),\C).
\]
This functor, the \emph{Weil nerve} of an involution algebroid, builds a functor from the free tangent category over a single object to $\C$ using a span construction. This chapter primarily builds on two pieces of work: Leung's construction of the free tangent category $\wone$ (\cite{Leung2017}) , and Grothendieck's original nerve construction (first published in \cite{Segal1974}).

To understand Leung's construction of the free tangent category, and more generally his actegory-theoretic presentation of tangent categories (Section \ref{sec:tang-struct-as-wone}), we first look at Weil's original insight relating the kinematic and operational descriptions of the tangent bundle in $\mathsf{SMan}$. The definition of a tangent vector on a manifold $M$ as an equivalence class of curves (Definition \ref{def:tang-vector})  puts a bijective correspondence between tangent vectors and $\R$-algebra homomorphisms from the ring of smooth functions $C^\infty(M)$ to the ring of dual numbers:
\[
	C^\infty(M) \to \R[x]/x^2.
\] The hom-set $\R\mathsf{Alg}(C^\infty(M),\R[x]/x^2)$ is precisely the set of \emph{derivations} on $C^\infty(M)$, which defines the operational tangent bundle discussed in Definition \ref{def:operational-tang}: there is a natural smooth manifold structure on this set. The Weil functor formalism, most notably developed in \cite{Kolar1993}, extends this observation to a general class of endofunctors on $\mathsf{SMan}$. For example, the fibre product  $T_2M$ corresponds to $\R$-algebra morphisms,
\[
C^\infty(M) \to R[x,y]/(x^2, y^2, xy), 
\] while the second tangent bundle corresponds to $\R$-algebra morphisms into the tensor product,
\[
C^\infty(M) \to R[x]/(x^2) \ox R[y]/(y^2) = R[x,y]/(x^2,y^2).
\] By applying Milnor's exercise (Problem 1-C \cite{Milnor1974}), which states that the $C^\infty$ functor
\[
    \mathsf{SMan} \to \R\mathsf{Alg^{op}};\hspace{0.15cm} M \mapsto C^\infty(M)=\mathsf{SMan}(M,\R)
\] is fully faithful, the structure maps occur as natural transformations. For example, the tangent projection is induced by the morphism
\[
	{p}: \R[x]/x^2 \xrightarrow[]{a + bx \mapsto a} \R,  
\] 
so that 
\[
    TM \xrightarrow[]{p} M = [C^\infty(M),\R[x]/x^2] \xrightarrow[]{(p)_*} [C^\infty(M), \R].
\]
The zero map and addition are similarly induced by
\[
	{0}: \R \xrightarrow[]{a \mapsto a + 0x} R[x]/x^2 \text{ and } +:R[x,y]/(x^2,y^2,xy) \xrightarrow[\mapsto a + (b+c)x]{a + bx + cy} R[x]/x^2,
\] 
respectively, while the lift and flip are induced by the morphisms
\[
	{\ell}: \R[x]/x^2 \xrightarrow[\mapsto a + bxy]{a+bx} \R[x,y]/(x^2,y^2) 
\]
and
\[
    {c}: \R[x,y]/(x^2,y^2) \xrightarrow[\mapsto a + cx + by + dxy]{a + bx + cy + dxy} \R[x,y]/(x^2,y^2).
\] 
More generally, there is a monoidal category of \emph{Weil algebras} (Definition \ref{def:weil-algebra-and-prol}) which has a monoidal action on the category of smooth manifolds. The Weil functor formalism, then, studies differential geometric structures from the perspective of the endofunctors and natural transformations induced by this action. Leung's insight is that there is an analogous category of commutative rigs\footnote{A rig is a ri\emph{n}g without \emph{n}egatives, i.e. a commutative monoid equipped with a bilinear multiplication.} built by replacing $\R[x]/x^2$ with $\N[x]/x^2$, called $\wone$ (Definition \ref{def:Weil-algebra}); a tangent structure is precisely a monoidal action by $\wone$ satisfying some universal properties. In particular, this category $\wone$ is precisely the free tangent category over a point, $\mathsf{FreeTang}(*)$, so that every object $A$ in a tangent category $\C$ determines a strict tangent functor
\[
    T^{(-)}A: \wone \to \C; V \mapsto T^VA
\]
and morphisms $f:A \to B$ are in bijective correspondence with tangent-natural transformations $T^{(-)}A \Rightarrow T^{(-)}(B)$.

The axioms of an involution algebroid in a tangent category $\C$ correspond bijectively with those of the tangent bundle - this suggests that an involution algebroid should determine a tangent functor from $\wone$ to $\C$. A first guess would lead one to think that $p:\N[x]/x^2 \to \N$ is sent to $\pi:A \to M$, $0$ to $\xi$, and $+$ to $+_A$. As the space of prolongations $\prol(A) = \prolong$ plays the role of the second tangent bundle, we can see that
\[
  \ell:N[x]/x^2 \to N[x,y]/(x^2,y^2) \mapsto (\xi\o\pi,\lambda): A \to \prol(A), 
\]
and 
\[
  c:N[x,y]/(x^2,y^2) \to N[x,y]/(x^2,y^2) \mapsto \sigma \prol(A) \to \prol(A).
\]
This pattern may be neatly summed up using span composition - we will construct a functor that sends $\N[x]/x^2$ to the span
% https://q.uiver.app/?q=WzAsMyxbMCwxLCJNIl0sWzEsMCwiQSJdLFsyLDEsIlRNIl0sWzEsMiwiXFxhbmMiLDFdLFsxLDAsIlxccGkiLDFdXQ==
\[\begin{tikzcd}
	& A \\
	M && TM
	\arrow["\anc"{description}, from=1-2, to=2-3]
	\arrow["\pi"{description}, from=1-2, to=2-1]
\end{tikzcd}\]
and the tensor product $\N[x]/x^2 \ox \N[x]/x^2$ to the span composition (e.g. the pullback)
\[\begin{tikzcd}
	&& {\prol(A)} \\
	& A && TA \\
	M && TM && {T^2M}
	\arrow["\anc"{description}, from=2-2, to=3-3]
	\arrow["\pi"{description}, from=2-2, to=3-1]
	\arrow["{T.\pi}"{description}, from=2-4, to=3-3]
	\arrow["{T.\anc}"{description}, from=2-4, to=3-5]
	\arrow[from=1-3, to=2-2]
	\arrow[from=1-3, to=2-4]
	\arrow["\lrcorner"{anchor=center, pos=0.125, rotate=-45}, draw=none, from=1-3, to=3-3]
\end{tikzcd}\]
which is the space of prolongations. This leads to the first major result of this chapter, the \emph{Weil Nerve} (Theorem \ref{thm:weil-nerve}), which states there is a fully faithful embedding
\[
    N_{\mathsf{Weil}}:\mathsf{Inv}(\C) \hookrightarrow [\wone, \C].
\]
This bears a strong similarity to Grothendieck's original nerve theorem, which takes an internal category $s,t:C \to M$ and constructs a functor $\Delta^{op} \to \C$ (where $\Delta^{op}$ is the monoidal theory of an internal monoid) by sending tensor to span composition, and the composition and unit maps given span morphisms
% https://q.uiver.app/?q=WzAsOCxbMCwxLCJNIl0sWzEsMCwiQ18yIl0sWzIsMSwiTSJdLFsxLDIsIkMiXSxbMywxLCJNIl0sWzQsMiwiQyJdLFs1LDEsIk0iXSxbNCwwLCJNIl0sWzEsMCwicyBcXG8gXFxwaV8wIiwxXSxbMSwyLCJ0IFxcbyBcXHBpXzEiLDFdLFszLDAsInMiLDFdLFszLDIsInQiLDFdLFsxLDMsIm0iLDFdLFs3LDQsIiIsMSx7ImxldmVsIjoyLCJzdHlsZSI6eyJoZWFkIjp7Im5hbWUiOiJub25lIn19fV0sWzcsNiwiIiwxLHsibGV2ZWwiOjIsInN0eWxlIjp7ImhlYWQiOnsibmFtZSI6Im5vbmUifX19XSxbNSw0LCJzIiwxXSxbNSw2LCJ0IiwxXSxbNyw1LCJlIiwxXV0=
\[\begin{tikzcd}
	& {C_2} &&& M \\
	M && M & M && M \\
	& C &&& C
	\arrow["{s \o \pi_0}"{description}, from=1-2, to=2-1]
	\arrow["{t \o \pi_1}"{description}, from=1-2, to=2-3]
	\arrow["s"{description}, from=3-2, to=2-1]
	\arrow["t"{description}, from=3-2, to=2-3]
	\arrow["m"{description}, from=1-2, to=3-2]
	\arrow[Rightarrow, no head, from=1-5, to=2-4]
	\arrow[Rightarrow, no head, from=1-5, to=2-6]
	\arrow["s"{description}, from=3-5, to=2-4]
	\arrow["t"{description}, from=3-5, to=2-6]
	\arrow["e"{description}, from=1-5, to=3-5]
\end{tikzcd}\]
while the unit and associativity axioms for a category are exactly the unit and associativity laws for a monoid in this setting. The \emph{Segal conditions} identify exactly the functors $C:\Delta^{op} \to \C$ that lie in the image of the nerve functor $N$ as those whose $[n]^{th}$ object is sent to the wide pullback $C([n]) = C[2] \ts{t}{s} C[2]\dots \ts{t}{s} C[2]$. The corresponding result for involution algebroids is found in Theorem \ref{thm:iso-of-cats-inv-emcs}, which states that a tangent functor $(A,\alpha):\wone \to \C$ is the nerve of an involution algebroid if and only if $A$ preserves tangent limits and $\alpha$ is a $T$-cartesian natural transformation (this forces $A.(W^{\ox n}) = A \ts{\anc}{T.\pi} TA \dots \ts{T^{n-1}\anc}{T^n\pi} T^nA$). The similarity between the Weil nerve and Grothendieck/Segal's nerve runs deep, and in Chapter \ref{ch:inf-nerve-and-realization} we demonstrate that the \emph{enriched} perspective on tangent categories puts these both into the same formal framework.
% There is a natural analogy between involution algebroids and the tangent bundle, as there is a bijective correspondence between the structure maps and the axioms they satisfy, and so it is a natural candidate for the ``classifying tangent category'' of an involution algebroid. However, the construction of a non-monoidal tangent functor $(A,\alpha):\wone \to \C$ from an involution algebroid in $\C$ poses a technical challenge. Luckily, inspiration may be drawn from a cornerstone result in category theory, Grothendieck's \emph{nerve theorem}, which shows that the internal categories in $\C$ embed into the category of simplicial objects in $\C$ $[\Delta^{op}, \C]$. The simplex category $\Delta$, a full subcategory of $\mathsf{Cat}$ whose objects are preorders $[n] = 1 < \dots < n$ (preorder morphisms are exactly functors), and its opposite category is the monoidal theory of a monoid; that is to say, any monoid in a monoidal category is given by a monoidal functor
% \[
%     (\Delta^{op}, +, [0]) \to (\C, \ox, I).
% \]
% Given an internal category $C$ in a finitely complete category $\C$, there is a base span given by the underlying graph:
% % https://q.uiver.app/?q=WzAsMyxbMCwxLCJNIl0sWzEsMCwiQyJdLFsyLDEsIk0iXSxbMSwwLCJzIiwxXSxbMSwyLCJ0IiwxXV0=
% \[\begin{tikzcd}
% 	& C \\
% 	M && M
% 	\arrow["s"{description}, from=1-2, to=2-1]
% 	\arrow["t"{description}, from=1-2, to=2-3]
% \end{tikzcd}\]
% The tensor product $[1]+[1]$ is sent to the \emph{span composition} (e.g. the pullback)
% % https://q.uiver.app/?q=WzAsNixbMCwyLCJNIl0sWzEsMSwiQyJdLFsyLDIsIk0iXSxbMywxLCJDIl0sWzQsMiwiTSJdLFsyLDAsIkNfMiJdLFsxLDAsInMiLDFdLFsxLDIsInQiLDFdLFszLDIsInMiLDFdLFszLDQsInQiLDFdLFs1LDFdLFs1LDNdLFs1LDAsIiIsMSx7ImN1cnZlIjoxfV0sWzUsNCwiIiwxLHsiY3VydmUiOi0xfV0sWzUsMiwiIiwxLHsic3R5bGUiOnsibmFtZSI6ImNvcm5lciJ9fV1d
% \[\begin{tikzcd}
% 	&& {C_2} \\
% 	& C && C \\
% 	M && M && M
% 	\arrow["s"{description}, from=2-2, to=3-1]
% 	\arrow["t"{description}, from=2-2, to=3-3]
% 	\arrow["s"{description}, from=2-4, to=3-3]
% 	\arrow["t"{description}, from=2-4, to=3-5]
% 	\arrow[from=1-3, to=2-2]
% 	\arrow[from=1-3, to=2-4]
% 	\arrow[curve={height=10pt}, from=1-3, to=3-1]
% 	\arrow[curve={height=-10pt}, from=1-3, to=3-5]
% 	\arrow["\lrcorner"{anchor=center, pos=0.125, rotate=-45}, draw=none, from=1-3, to=3-3]
% \end{tikzcd}\]
% where the monoid composition is sent to the internal category's composition map, and the identity map is sent by the unit map for the category:

% For the Weil nerve construction of an anchored bundle construction, we will generalize the prolongations of the underlying anchored bundle. An anchored bundle $(\pi:A \to M, \xi, \lambda, \anc)$ determines a span, where the object $\mathbb{N}[x]/x^2$ is sent to

% The tensor product of commutative rigs $\mathbb{N}[x]/x^2 \ox \mathbb{N}[x]/x^2$ will be sent to the following span composition:
% % https://q.uiver.app/?q=WzAsNixbMCwyLCJNIl0sWzEsMSwiQSJdLFsyLDIsIlRNIl0sWzMsMSwiVEEiXSxbNCwyLCJUXjJNIl0sWzIsMCwiXFxwcm9sKEEpIl0sWzEsMiwiXFxhbmMiLDFdLFsxLDAsIlxccGkiLDFdLFszLDIsIlQuXFxwaSIsMV0sWzMsNCwiVC5cXGFuYyIsMV0sWzUsMV0sWzUsM10sWzUsMiwiIiwxLHsic3R5bGUiOnsibmFtZSI6ImNvcm5lciJ9fV1d
% \[\begin{tikzcd}
% 	&& {\prol(A)} \\
% 	& A && TA \\
% 	M && TM && {T^2M}
% 	\arrow["\anc"{description}, from=2-2, to=3-3]
% 	\arrow["\pi"{description}, from=2-2, to=3-1]
% 	\arrow["{T.\pi}"{description}, from=2-4, to=3-3]
% 	\arrow["{T.\anc}"{description}, from=2-4, to=3-5]
% 	\arrow[from=1-3, to=2-2]
% 	\arrow[from=1-3, to=2-4]
% 	\arrow["\lrcorner"{anchor=center, pos=0.125, rotate=-45}, draw=none, from=1-3, to=3-3]
% \end{tikzcd}\]
% The morphisms in $\wone$ are then constructed using the structure maps for the involution algebroid.

Sections \ref{sec:weil-algebras-tangent-structure} and \ref{sec:tang-struct-as-wone} give a detailed introduction to the Weil functor formalism (\cite{Kolar1993},\cite{Bertram2014a}) and Leung's unification of Weil functors with tangent categories \cite{Leung2017}. The rest of the chapter contains contains new results developed by the author. Section \ref{sec:weil-nerve} proves the embedding part of the Weil nerve theorem, that the category of involution algebroids embeds into the category of tangent functors and tangent natural transformations $[\wone, \C]$. Section \ref{sec:identifying-involution-algebroids} identifies exactly those tangent functors $(A,\alpha):\wone \to \C$ that are the nerve of an involution algebroid, completing the proof of the Weil nerve theorem. Section \ref{sec:prol_tang_struct} uses the Weil nerve to develop a novel tangent structure on the category of involution algebroids in a tangent category (in particular, the category of Lie algebroids will have this novel tangent structure).


\section{Weil algebras and tangent structure}%
\label{sec:weil-algebras-tangent-structure}

This section gives a more thorough introduction to the Weil functor formalism of \cite{Kolar1993}, and in particular how the structure maps of a tangent category may be teased out of it. We begin by introducing  Weil algebras, and the \emph{prolongation} of a smooth manifold by a Weil algebra. (The relationship with prolongations of involution algebroids from Definition \ref{def:anchored_bundles} will be made clear in Section \ref{sec:weil-nerve}.)

\begin{definition}%
    \label{def:weil-algebra-and-prol}
    An $\R$-Weil algebra\footnote{Not to be confused with the normal usage of ``Weil algebra'' in Lie theory, e.g. \cite{Meinrenken2019}.} is a finite-dimensional $\R$-algebra $V$ so that $V = \R \oplus \dot{V}$ as $\R$-modules and $\dot{V}$ is a nilpotent ideal. The category $\R\mathsf{Weil}$ is the full subcategory of $\R\mathsf{Alg}$ spanned by the $\R$-Weil algebras. The \emph{prolongation} of a manifold by a Weil algebra $V$ is given by the manifold
    \[
        T^VM := \R\mathsf{Alg}(C^\infty(M,\R), V).
    \]
\end{definition}
\cite{Eck1986} showed that every product-preserving endofunctor on the category of smooth manifolds is constructed as the Weil prolongation by some Weil algebra. Consequently, $\R$-algebra homomorphisms induce natural transformations between these product-preserving  endofunctors on the category of smooth manifolds.
\begin{example}\label{ex:weil-algebras-and-maps}
    Consider the following $\R$-Weil algebras and their associated prolongation functors.
    \begin{enumerate}[(i)]
        \item Prolongation by $\R$ induces the identity functor, and the tangent bundle is given by $\R[x]/x^2$. The tangent projection, then, is equivalent to the $\R$-algebra morphism
        \[
            p: \R[x]/x^2 \to \R; \hspace{0.2cm}
            p(a + bx) = a
        \]
        while the 0-map induces the zero vector field:
        \[
            0: \R \to \R[x]/x^2; \hspace{0.2cm}
            0(a) = a + 0x.
        \]
        \item The algebra $\R[x_i]_{1 \le i \le n}/(x_ix_j)_{1 \le i \le j \le n} = (\R[x]/x^2)^n$ is the wide pullback $T_nM = TM \ts{p}{p} TM \dots \ts{p}{p} TM$. 
        In particular, prolongation by $\R[x,y]/(x^2,y^2,xy)$ gives the bundle $T_2M = TM \ts{p}{p} TM$. The $\R$-algebra morphism
        \[
            +:\R[x,y]/(x^2,y^2,xy) \to R[x]/x^2; \hspace{0.2cm}
            +(a_0 + a_1x + a_2y) = a_0 + (a_1 + a_2)x
        \]corresponds to the addition of tangent vectors.
        \item The algebra $\R[x,y]/(x^2,y^2) = (\R[x]/x^2)\ox (\R[x]/x^2)$ is the second tangent bundle $T^2M = TTM$. The vertical lift $T \to T^2$ is induced by the morphism 
        \[
            \ell: \R[x]/x^2 \to \R[x,y]/(x^2,y^2);\hspace{0.2cm}
            \ell(a + bx) = a + bxy.
        \]
        \item The monoidal symmetry map induces $c: T^2 \Rightarrow T^2$, as follows: 
        \begin{gather*}
                        c: (\R[x]/x^2)\ox (\R[y]/y^2) \to (\R[y]/y^2)\ox (\R[x]/x^2);\\ c(a + b_1x + b_2y + b_3xy) = a + b_2x + b_1y + b_3xy.        
        \end{gather*}
        %by exchanging the $x,y$ variables.
        \item For $n \ge 2$, the algebra $\R[x]/x^n$ gives the $n$-jet bundle.
        Note that this is the equalizer of $\ox^n \R[x]/x^2$ by the symmetry actions of $S_n$.
    \end{enumerate}
\end{example}
Further examples may be found in the monograph \cite{Kolar1993}. Tangent categories bridge the gap between the Weil functor approach to studying the differential geometry of smooth manifolds and the synthetic differential geometry approach of axiomatizing a tangent bundle using nilpotent infinitesimals. The main structure axiomatized here is that of \emph{monoidal action} of a symmetric monoidal category on a category $M \x \C \to \C$, or equivalently, a lift from a category to the category of complexes $\C \to [M,\C]$, which involves translating a bit of classical category theory to the 2-categorical setting.

Example  \ref{ex:weil-algebras-and-maps} leads to the classical theorem that the category of smooth manifolds has an action by the category of $\R$-Weil algebras that preserves all connected limits that exist. These ``natural'' universal properties (in the sense of \cite{Kolar1993}) is foundational to synthetic differential geometry; see, for example, Chapter Two of \cite{Lavendhomme1996}. Unfortunately, Weil algebras are not an ideal syntactic presentation: they are not a finitely presentable category, and it is not immediately clear when a diagram is a connected limit.\footnote{It should be noted that \cite{Nishimura2007} made progress applying techniques from computer algebra to latter problem.} Moving from $\R$-algebras to commutative rigs and restricting to an appropriate subcategory solves this problem.\pagenote{Clarified a point raised by Michael}

\begin{definition}[Definition 3.1 \cite{Leung2017}]
    \label{def:Weil-algebra}
    The category $\wone$ is defined to be the full subcategory of commutative rigs, $\mathsf{CRig}$, generated by the rig of dual numbers $W := \N[x]/x^2$, constructed as follows:
    \begin{enumerate}
        \item Start with finite product powers of $W$ in $\mathsf{CRig}$, and make a strict choice of presentation:
        \[
            W_0 = \N, \hspace{0.15cm} W_n := \N[x_i]/(x_ix_j)_{i \le j}, {0 \le i < n}.
        \]
        \item Then take the closure of $W_n$ under coproduct of commutative rigs, written $\ox$. Again, make a strict choice of presentation:
        \[
            W_{n(0)} \ox \dots \ox W_{n(m-1)} :=
            \N[x_{i,j}]/(x_{ij}x_{ik})_{j \le k}, 0 < i < m, 0 < j < n(i).
        \]
    \end{enumerate}
    Note that we will often suppress the tensor product $\ox$ and simply write\pagenote{explained notation used throughout this chapter}
    \[
        UV := U\ox V.
    \]
\end{definition}
\begin{proposition}[Definition 3.3 \cite{Leung2017} ]
    ~\begin{enumerate}[(i)]
        \item The category $\wone$ is a symmetric strict monoidal category with unit $\N$ and coproduct $\ox$.
        \item $\N$ is a terminal object in $\wone$.
    \end{enumerate}
\end{proposition} 
Note that there is a forgetful functor
\[
    \wone \to (\mathsf{CMon}/\N) \to \mathsf{CMon}    
\]
that reflects connected limits. This gives the following class of limits, identified in \cite{Leung2017}.
\begin{definition}%
    \label{def:transverse-limit}
    We say the following pullback diagrams in $\wone$ are \emph{transverse}:\pagenote{switched superscript to subscript in second diagram}
    \input{TikzDrawings/Ch4/Sec2/transverse-limits.tikz}
    where $\mu(a + a_1x + a_2y) = a + a_1x + a_2xy$. The $\ox$-closure of these three pullbacks is the set of \emph{transverse squares}, and they are also pullback squares by \cite{Leung2017}.
\end{definition}
To see that each transverse square in the $\ox$-closure is a pullback diagram, take the two non-identity squares and rewrite them in $\mathsf{CMon}$:
\[\input{TikzDrawings/Ch4/cmon-pb.tikz}\]
The coproduct of Weil algebras is the tensor product of the underlying commutative monoids, which are finite-dimensional and free, so these limits are closed under $\ox$.
\begin{proposition}[\cite{Leung2017} Proposition 4.1]
    The category $\wone$ is a tangent category, where the tangent functor is
    \[
        T := W \ox (\_): \wone \to \wone  
    \]
    and the natural transformations are given by 
    \begin{gather*}
        p: W \ox (\_) \xrightarrow[]{p \ox (\_)} (\_), \hspace{0.25cm}
        0: (\_) \xrightarrow[]{0 \ox (\_)} W \ox (\_), \hspace{0.25cm}
        +: W_2 \ox (\_) \xrightarrow[]{+ \ox (\_)} W \ox (\_), \\
        \ell: W \ox (\_) \xrightarrow[]{\ell \ox (\_)} W \ox W \ox (\_), \hspace{0.25cm}
        c: W \ox W \ox (\_) \xrightarrow[]{p \ox (\_)} W \ox W \ox (\_)).
    \end{gather*}
\end{proposition}

The category $\wone$ is, in some sense, a finitely presented theory. It is precisely the free tangent category on a single object:
\begin{proposition}[Proposition 9.5, \cite{Leung2017}]
    \label{thm:leung}
    The category $\wone$ is generated by the maps $\{p, 0, +, \ell, c\}$ from Example  \ref{ex:weil-algebras-and-maps}, closed under composition, tensor, and maps induced by transverse limits.
\end{proposition}
\begin{corollary}
    The category $\wone$ is the \emph{free} tangent category over a single object: every object $C$ in a tangent category $\C$ determines a strict tangent functor $T_{-}.C: \wone \to \C$, mapping
    \[
        V = W^{n{1}} \ox \dots \ox W^{n(k)}  
        \mapsto 
        T_{n(1)}.(\dots).T_{n(k)}.C = T^VC
    \]
    so that there is an isomorphism of categories between $\C$ and the category of strict tangent functors $\wone \to \C$ with tangent natural transformations as morphisms.\pagenote{Original statement was incomplete.}
\end{corollary}
\begin{notation}
    Throughout this section, the notation $T^VC$ will refer to the action of the Weil algebra $V$ on an object $C$ in a tangent category.
    In particular, we will make use of the isomorphism $T^U.T^VC = T^{UV}C$.
\end{notation}
% In some sense, then, a tangent structure is somehow a functorial choice of a ``tangent complex'' $\C \mapsto [\wone,\C]$.


\section{Tangent structures as monoidal actions}%
\label{sec:tang-struct-as-wone}

The presentation of $\wone$ as the free tangent category situates the formal theory of tangent categories as an instance of more general categorical machinery, namely monoidal actions. Recall that in a symmetric monoidal category $(\C, \ox, I)$, an internal monoid $(C, \bullet, i)$ determines a monad:
\[
    (C \ox \_: \C \to \C, \mu: \C \ox (\C \ox \_) \xrightarrow{\bullet \ox \_} \C \ox \_, 
    \eta: \_ \xrightarrow{\rho} I \ox \_ \xrightarrow{e \ox \_} C \ox \_).
\]
The category of algebras for this monad is exactly the category of $C$-modules, objects with an associative and unital action by $C$. A morphism will be a map on the base object that preserves the action:
\input{TikzDrawings/Ch4/Sec2/monoidal-action.tikz}
Strict actegories are the 2-categorical generalization of modules over a monoid. The coherences for a 2-monad follow from the coherences from a strict monoidal category in the 2-category of categories. The following proposition relies on a few facts from enriched category theory (treating the cartesian closed category $\mathsf{Cat}$ as a $\mathsf{Cat}$-enriched category, per \cite{Kelly2005}) but a more general treatment of non-strict actegories may be found in \cite{janelidze2001note}:\pagenote{Since I'm not referencing any sort of weakness, this is just enriched category theory so I think Kelly is a fair reference. However, I have included the reference to Janelidze and Kelly's work on actegories.} 
\begin{itemize}
    \item A 2-functor and 2-natural transformations are exactly a functor and natural transformations that satisfy extra coherences. These coherences follow for free by constructing the monad and comonad $\m \x \_, [\m, \_]$.
    \item An algebra of the underlying 1-monad is exactly an algebra of the 2-monad (the same result holds for comonads).
\end{itemize}
When working with algebras of a 2-monad, four different notions of morphisms can come into play (\cite{Lack2009}). These arise through using the 2-categorical data to weaken the notion of a morphism:
\begin{enumerate}[(i)]
    \item Strict: this is exactly a morphism of the underlying algebras. Write the 2-category of strict $\m$-actegories.
    \item Strong: the morphisms preserve the action to an isomorphism:
    \[\input{TikzDrawings/Ch4/Sec2/act-mor-strong.tikz}\]
    \item Lax: the 2-cell is no longer an isomorphism:
    \[\input{TikzDrawings/Ch4/Sec2/act-mor-lax.tikz}\]
    \item Oplax: the 2-cell travels in the opposite direction (these will not figure into this account).
\end{enumerate}
2-cells between actegory morphisms must satisfy a coherence between the natural transformation parts of the actegory morphisms.
\begin{definition}\label{def:actegory-natural}
    In the case of strict, strong, and lax tangent functors, the same notion of a 2-cell applies: a natural transformation $\gamma:F \Rightarrow G$ satisfying the following coherences with the natural transformations $\alpha$ and $\beta$:
% https://q.uiver.app/?q=WzAsOCxbMCwwLCJcXG1cXHhcXEMiXSxbMSwwLCJcXG1cXHhcXEQiXSxbMSwxLCJcXEQiXSxbMCwxLCJcXEMiXSxbMiwwLCJcXG1cXHhcXEMiXSxbMywwLCJcXG1cXHhcXEQiXSxbMiwxLCJcXEMiXSxbMywxLCJcXEQiXSxbMCwxLCJcXG1cXHggRiJdLFsxLDIsIlxccHJvcHRvXlxcRCJdLFswLDMsIlxccHJvcHRvXlxcQyIsMl0sWzMsMiwiRiIsMV0sWzEsMywiXFxhbHBoYSIsMSx7ImxldmVsIjoyfV0sWzMsMiwiRyIsMix7ImN1cnZlIjozfV0sWzQsNSwiXFxtXFx4IEciLDIseyJsYWJlbF9wb3NpdGlvbiI6MjB9XSxbNCw1LCJcXG1cXHggRiIsMCx7ImN1cnZlIjotM31dLFs0LDYsIlxccHJvcHRvXlxcQyIsMl0sWzUsNywiXFxwcm9wdG9eXFxEIl0sWzYsNywiRyIsMl0sWzUsNiwiXFxiZXRhIiwwLHsibGV2ZWwiOjJ9XSxbMTEsMTMsIlxcZ2FtbWEiLDAseyJzaG9ydGVuIjp7InNvdXJjZSI6MjAsInRhcmdldCI6MjB9fV0sWzE1LDE0LCJcXG1cXHggXFxnYW1tYSIsMix7Im9mZnNldCI6LTIsInNob3J0ZW4iOnsic291cmNlIjoyMCwidGFyZ2V0IjoyMH19XSxbOSwxNiwiPSIsMSx7InNob3J0ZW4iOnsic291cmNlIjoyMCwidGFyZ2V0IjoyMH0sInN0eWxlIjp7ImJvZHkiOnsibmFtZSI6Im5vbmUifSwiaGVhZCI6eyJuYW1lIjoibm9uZSJ9fX1dXQ==
\[\begin{tikzcd}
	\m\x\C & \m\x\D & \m\x\C & \m\x\D \\
	\C & \D & \C & \D
	\arrow["{\m\x F}", from=1-1, to=1-2]
	\arrow[""{name=0, anchor=center, inner sep=0}, "{\propto^\D}", from=1-2, to=2-2]
	\arrow["{\propto^\C}"', from=1-1, to=2-1]
	\arrow[""{name=1, anchor=center, inner sep=0}, "F"{description}, from=2-1, to=2-2]
	\arrow["\alpha"{description}, Rightarrow, from=1-2, to=2-1]
	\arrow[""{name=2, anchor=center, inner sep=0}, "G"', curve={height=18pt}, from=2-1, to=2-2]
	\arrow[""{name=3, anchor=center, inner sep=0}, "{\m\x G}"'{pos=0.2}, from=1-3, to=1-4]
	\arrow[""{name=4, anchor=center, inner sep=0}, "{\m\x F}", curve={height=-18pt}, from=1-3, to=1-4]
	\arrow[""{name=5, anchor=center, inner sep=0}, "{\propto^\C}"', from=1-3, to=2-3]
	\arrow["{\propto^\D}", from=1-4, to=2-4]
	\arrow["G"', from=2-3, to=2-4]
	\arrow["\beta", Rightarrow, from=1-4, to=2-3]
	\arrow["\gamma", shorten <=2pt, shorten >=2pt, Rightarrow, from=1, to=2]
	\arrow["{\m\x \gamma}"', shift left=2, shorten <=2pt, shorten >=2pt, Rightarrow, from=4, to=3]
	\arrow["{=}"{description}, Rightarrow, draw=none, from=0, to=5]
\end{tikzcd}\]
We call these \emph{actegory natural transformations}. 
\end{definition}
Note that for strict actegory morphisms, this condition holds for any natural transformation $\gamma:F \Rightarrow G$. Now consider the following three 2-categories.\pagenote{Added the actual definition of the 2-categories referenced later on. I also added the coherence for actegory-natural transformation}
\begin{definition}\label{def:2-categories}
    Let $(\m,\ox,I)$ be a strict monoidal category. Define the following three 2-categories.
    \begin{enumerate}
        \item $\m\mathsf{Act}_{\mathsf{strict}}$: the 2-category of strict $\m$-actegories, strict actegory morphisms, and natural transformations.
        \item $\m\mathsf{Act}_{\mathsf{strong}}$: the 2-category of strict $\m$-actegories, strong actegory morphisms, and actegory natural transformations.
        \item $\m\mathsf{Act}_{\mathsf{lax}}$: the 2-category of strict $\m$-actegories, lax actegory morphisms, and actegory natural transformations.
    \end{enumerate}
    Note that the inclusions of these 2-categories are \emph{locally} fully faithful, so only the 1-cells differ.
\end{definition}

The case where the action preserves certain limits in the monoidal category is of particular interest. A small category equipped with a class of chosen limits is known as a \emph{sketch}. The previous correspondence restricts to the class of limit-preserving actions in this case.
\begin{definition}
    A \emph{sketch} is a small category with a class of chosen limits, and a sketch morphism is a functor sending chosen limits to the chosen limits in the domain (up to isomorphism). The category of models of a sketch $\c$ in a category $\C$, $\mathsf{Mod}(\c, \C)$, is the full subcategory $[\c, \C]$ whose functors preserve the chosen limits.
    A \emph{monoidal sketch}, then, is a sketch $(\c, \prol)$ equipped with a symmetric monoidal category structure on $\c$ so that $\_ \ox \_$ preserves limits in each argument.
\end{definition}

Now use the fact that the category $\wone$ is a monoidal sketch, since it is a small, strict monoidal category equipped with a class of limits stable under the tensor product.
\begin{theorem}[Theorem 14.1, \cite{Leung2017}]
    Let $\C$ be a category. The following are equivalent:
    \begin{enumerate}[(i)]
        \item A tangent structure on $\C$,
        \item A sketch action $\propto: \wone \x \C \to \C$.
        % \item A sketch lift $\phi: \C \to \mathsf{Mod}(\wone, \C)$.
    \end{enumerate}
\end{theorem}

\begin{observation}%
    \label{obs:cofree-tangent-cat}
    There is a \emph{coalgebraic} perspective on tangent categories, coming from the equivalence between algebras of the 2-monad $(\wone \x (\_), \ox, I:1 \to \wone)$ and the 2-comonad $([\wone,\_], [\ox,\_], [I,\_])$.
    For any  category $\C$, there is a free tangent category given by
    \[
        \wone \x \C
    \] and this agrees with the free $\wone$-actegory. However, for the \emph{cofree} tangent category, take
    \[
        \mathsf{Mod}(\wone, \C),
    \]
    the category of transverse-limit-preserving functors $\wone \to \C$.
\end{observation}

We can use Leung's theorem to induce a monoidal functor $\wone \to \C$ when a tangent structure is induced by a single object.
\begin{corollary}\label{cor:using-leung-thm}
    Let $(\C,\ox,I)$ be a strict monoidal category.\pagenote{I've added this intermediary result to make the Weil nerve a bit more explicit.}
    If an additive bundle $(p:A \to I, +:A_2 \to A, 0:I \to A)$ equipped with morphisms
    \[
       A \ox A \xrightarrow[]{c} A \ox A \hspace{0.5cm} A \xrightarrow[]{\ell} A \ox A
    \]
    determines a tangent structure on $\C$ using the endofunctor $A \ox (-)$, then $A$ determines a strict, transverse-limit-preserving, monoidal functor 
    \[
       A(-):\wone \to \C; W_{n[1]}\ox \dots \ox W_{n[k]} \mapsto A_{n[1]}\ox \dots \ox A_{n[k]} = T_{n[1]}\dots T_{n[k]}.I
    \]
\end{corollary}
Note that this allows for a more conceptual description of representable tangent structure.
\begin{proposition}\label{prop:monoidal-functor-inf-obj}
    In a symmetric monoidal closed category, an infinitesimal object is exactly a strict symmetric monoidal functor $D:\wone \to \C$.\pagenote{The other statement was probably more general than necessary - this statement is more to the point.}
\end{proposition}
This presentation of an infinitesimal object makes it tautological that $\C^{op}$ has a tangent structure.
\begin{corollary}%
    \label{cor:dual-tangent-structure}
    Given a strict symmetric monoidal functor
    \[
        D:\wone \to \C^{op}
    \]
    there is a strict action of $\wone$ on $\C^{op}$ given by
    \[
        \wone \x \C^{op} \xrightarrow[]{D^{op} \ox \C} \C^{op} \x \C^{op} \xrightarrow[]{\otimes} \C^{op}. 
    \]
\end{corollary}


There is a clear correspondence between the notions of a (strict, strong, lax) tangent functor and a (strict, strong, lax) actegory morphism. This proposition extends to the following equivalence of 2-categories.
\begin{corollary}[Theorem 14.1 \cite{Leung2017}]
    The following pairs of 2-categories are equivalent.
    \begin{enumerate}[(i)]
        \item The 2-category of tangent categories and strict tangent functors is the full sub-2-category of $\wone\mathsf{Act_{Strict}}$ spanned by sketch actions.
        \item The 2-category of tangent categories and strong tangent functors is the full sub-2-category of $\wone\mathsf{Act_{strong}}$ spanned by sketch actions.
        \item The 2-category of tangent categories and lax tangent functors is the full sub-2-category of $\wone\mathsf{Act_{Lax}}$ spanned by sketch actions.
    \end{enumerate}
\end{corollary}



% Make sure you use 
\section{The Weil nerve of an involution algebroid}
\label{sec:weil-nerve}

The construction in this section is analagous to the nerve of an internal category---hence the ``Weil nerve'' construction---and deals with similar technical issues. In particular, the construction in this section will mimic the nerve construction for internal categories by replacing the tensor product of $\wone$ with span composition in the domain category. Recall that every anchored bundle or internal category has a canonical span associated with it:
\[\input{TikzDrawings/Ch4/Sec3/anc-bundle.tikz}\]
In any category $\C$, there is a category of spans in $\C$ as well as span composition.
\begin{definition}%
    \label{def:span-stuff}
    A \emph{span from $A$ to $B$} in a category $\C$ is a diagram of the form 
    \[\input{TikzDrawings/Ch4/Sec3/span.tikz}\]
    There is a notion of \emph{span composition}, so given a span $X:A \to B$ and $Y:B \to C$, then the composition of $X$ and $Y$ is the pullback (if it exists):
    \[\input{TikzDrawings/Ch4/Sec3/span-hor-comp.tikz}\]
    A morphism of spans is a commuting diagram of the form
    \[\input{TikzDrawings/Ch4/Sec3/span-morp.tikz}\]
    Note that if $f$ and $g$ are span morphisms with $f_r = g_l$, then the horizontal composition may be formed if each respective span composition exists:
    \[\input{TikzDrawings/Ch4/Sec3/span-hor-comp-morp.tikz}\]
    When discussing span composition in a tangent category, it is assumed that the pullback is a $T$-pullback.
\end{definition}
\begin{observation}%
    \label{obs:lim-of-spans}
    Note that the category of spans in $\C$ is a functor category, so that limits are computed pointwise in $\C$. This also means that the horizontal composition operation, when it exists, preserves limits in either argument.
\end{observation}
These span constructions can be helpful in constructing functors from a monoidal category into a non-monoidal category $\C$, by forming a monoidal category from $\C$ using spans. In the case of an internal category over $M$, one takes the slice category $\C/(M \x M)$ where the tensor product is span composition. An internal category $s,t:C \to M$ is a monoid in this category of spans over $M$, so that it determines a monoidal functor
\[
    C: \Delta^{op} \to \C/(M \x M),    
\] 
remembering that $\Delta^{op}$ is the monoidal theory for monoid (every monoid in a monoidal category $\C$ determines a monoidal functor $\Delta \to \C$). The construction of the corresponding monoidal category for spans is more nuanced, as the category $\wone$ is not $\N$-indexed. Observe that the prolongation of an anchored bundle is constructed as a span composition:
\input{TikzDrawings/Ch4/Sec3/prol-span-comp.tikz}
The third prolongation is given by span composition as well:
\[\input{TikzDrawings/Ch4/Sec3/prol-span-comp-2.tikz}\]
This horizontal composition will play the same role as the tensor product in $\C/(M\x M)$.
\begin{definition}%
    \label{def:boxtimes-span}
    In a tangent category $\C$, consider a pair of spans
    \[
        X:M \to T^UM, \hspace{0.15cm} Y:M \to T^VM.
    \]
    Define $X \boxtimes Y$ to be the horizontal composition (when it exists):
    \[\input{TikzDrawings/Ch4/boxtimes-span.tikz}\]
    (recall that we will often suppress the $\ox$ in $\wone$ to save space).\pagenote{clarifying notation}
    So the span composition is
    \[
        M \xrightarrow[]{X} T^UM \xrightarrow[]{T^U.Y} T^U.T^VM    
    \]
    \begin{equation}
        \label{eq:span-form}
        \input{TikzDrawings/Ch4/span-form.tikz}
    \end{equation}
    The horizontal composition $f \boxtimes g$ is defined as $f \x_{\theta.M} \theta.g$:
    \[\input{TikzDrawings/Ch4/boxtimes-diag.tikz}\]
\end{definition}
% \begin{lemma}
%     Consider an object $M$ in a tangent category $\C$, and define the span:
%     \[
%         p^U := M \xleftarrow[]{p^U} T^UM = T^UM  
%     \]
%     This object behaves like the unit
% \end{lemma}
% \begin{observation}
In any tangent category with a tangent display system (Definition\pagenote{there was no reason to pull this out as an observation} \ref{def:display-system}), the category of spans on $M$ whose maps are of the form given by Equation \ref{eq:span-form} with $l \in \d$ is a monoidal category.
% \end{observation}
Any anchored bundle in a tangent category gives rise to a monoidal category after a strict choice of $T$-pullbacks (assuming those $T$-pullbacks exist).
\begin{definition}\label{def:monoidal-category}
    Let $(\pi:A \to M,\xi,\lambda,\anc)$ be an anchored bundle in a tangent category $\C$. Write the span
    \[
        \widehat{A}.W_n := (M \xleftarrow[]{\pi \o \pi_i} A_n \xrightarrow[]{\anc^n} T_nM),
        \hspace{0.5cm}
        \widehat{A}.\N := (M = M = M).
    \]
    A \emph{choice of prolongations} for $(\pi,\xi,\lambda,\anc)$ is a strict choice of horizontal composition for each $V \in \wone$:
    \[
        \widehat{A}.V = \widehat{A}.(W_{n[1]}\dots W_{n[k]}) :=
        \widehat{A}.W_{n[1]} \boxtimes \dots \boxtimes \widehat{A}.W_{n[k]}.
    \]
    We will write the span as follows:
    % https://q.uiver.app/?q=WzAsMyxbMCwxLCJNIl0sWzEsMCwiQS5WIl0sWzIsMSwiVF5WLk0iXSxbMSwwLCJcXHBpXlYiLDJdLFsxLDIsIlxcYW5jXlYiXV0=
\[\begin{tikzcd}
	& {A.V} \\
	M && {T^V.M}
	\arrow["{\pi^V}"', from=1-2, to=2-1]
	\arrow["{\anc^V}", from=1-2, to=2-3]
\end{tikzcd}\]
    (notice that the apex is not hatted).
    Given a choice of prolongations for an anchored bundle $(\pi,\xi,\lambda,\anc)$, the category $\mathsf{Span}(\pi,\xi,\lambda,\anc)$ is defined as follows:
    \begin{itemize}
        \item Objects are $\widehat{A}.V$ for $V \in \wone$.
        \item Morphisms are given by pairs
        \[
            (f,\phi):\widehat{A}.V \to \widehat{A}.U
        \]
        where $f:A.V \to A.U$ and $\phi:V \to U$ determine a span morphism of the form
        % https://q.uiver.app/?q=WzAsNixbMCwwLCJNIl0sWzEsMCwiQS5WIl0sWzIsMCwiVF5WLk0iXSxbMiwxLCJUXlUuTSJdLFswLDEsIk0iXSxbMSwxLCJBLlYiXSxbMSwwLCJcXHBpXlYiLDJdLFsxLDIsIlxcYW5jXlYiXSxbMiwzLCJcXHBoaS5NIl0sWzAsNCwiIiwyLHsibGV2ZWwiOjIsInN0eWxlIjp7ImhlYWQiOnsibmFtZSI6Im5vbmUifX19XSxbNSw0LCJcXHBpXlYiXSxbNSwzLCJcXGFuY15VIiwyXSxbMSw1LCJmIl1d
\[\begin{tikzcd}
	M & {A.V} & {T^V.M} \\
	M & {A.U} & {T^U.M}
	\arrow["{\pi^V}"', from=1-2, to=1-1]
	\arrow["{\anc^V}", from=1-2, to=1-3]
	\arrow["{\phi.M}", from=1-3, to=2-3]
	\arrow[Rightarrow, no head, from=1-1, to=2-1]
	\arrow["{\pi^V}", from=2-2, to=2-1]
	\arrow["{\anc^U}"', from=2-2, to=2-3]
	\arrow["f", from=1-2, to=2-2]
\end{tikzcd}\]
        as discussed in Definition \ref{def:span-stuff}.
        \item Tensor structure: The tensor product is defined using the horizontal composition $\boxtimes$ as defined in Definition \ref{def:span-stuff}.
    \end{itemize}
\end{definition}
The idea is to show that an involution algebroid structure on an anchored bundle induces a tangent structure on the monoidal category of prolongations, and then to apply Leung's theorem. 
The following two lemmas will simplify this proof.
\begin{lemma}\label{lemma:pullback-part-of-theorem}
    Let $(\pi:A \to M, \xi, \lambda, \anc)$ be an anchored bundle with chosen prolongations in a tangent category $\C$, and identify the monoidal category $\mathsf{Span}(\pi,\xi,\lambda,\anc)$.
    \begin{enumerate}[(i)]
        \item There is a functor 
        \[
           U^\anc: \mathsf{Span}(\pi,\xi,\lambda,\anc) \to \C
        \]
        constructed by sending a span morphism to the morphism between the objects at its apex:
% https://q.uiver.app/?q=WzAsOCxbMCwwLCJNIl0sWzEsMCwiQS5WIl0sWzIsMCwiVF5WLk0iXSxbMiwxLCJUXlUuTSJdLFswLDEsIk0iXSxbMSwxLCJBLlUiXSxbMywwLCJBLlYiXSxbMywxLCJBLlUiXSxbMSwwLCJcXHBpXlYiLDJdLFsxLDIsIlxcYW5jXlYiXSxbMiwzLCJcXHBoaS5NIiwyXSxbMCw0LCIiLDIseyJsZXZlbCI6Miwic3R5bGUiOnsiaGVhZCI6eyJuYW1lIjoibm9uZSJ9fX1dLFs1LDQsIlxccGleViJdLFs1LDMsIlxcYW5jXlUiLDJdLFsxLDUsImYiXSxbNiw3LCJmIl0sWzEwLDE1LCIiLDIseyJzaG9ydGVuIjp7InNvdXJjZSI6NDAsInRhcmdldCI6NDB9LCJsZXZlbCI6MSwic3R5bGUiOnsidGFpbCI6eyJuYW1lIjoibWFwcyB0byJ9fX1dXQ==
\[\begin{tikzcd}
	M & {A.V} & {T^V.M} & {A.V} \\
	M & {A.U} & {T^U.M} & {A.U}
	\arrow["{\pi^V}"', from=1-2, to=1-1]
	\arrow["{\anc^V}", from=1-2, to=1-3]
	\arrow[""{name=0, anchor=center, inner sep=0}, "{\phi.M}"', from=1-3, to=2-3]
	\arrow[Rightarrow, no head, from=1-1, to=2-1]
	\arrow["{\pi^V}", from=2-2, to=2-1]
	\arrow["{\anc^U}"', from=2-2, to=2-3]
	\arrow["f", from=1-2, to=2-2]
	\arrow[""{name=1, anchor=center, inner sep=0}, "f", from=1-4, to=2-4]
	\arrow[shorten <=14pt, shorten >=14pt, maps to, from=0, to=1]
\end{tikzcd}\]
        \item Suppose we have a square
% https://q.uiver.app/?q=WzAsNCxbMCwxLCJcXHdpZGVoYXR7QX0uWCJdLFsxLDEsIlxcd2lkZWhhdHtBfS5aIl0sWzEsMCwiXFx3aWRlaGF0e0F9LlkiXSxbMCwwLCJcXHdpZGVoYXR7QX0uVSJdLFswLDEsIihmLFxccGhpKSIsMl0sWzIsMSwiKGcsXFxwc2kpIl0sWzMsMCwiKGwsXFxhbHBoYSkiLDJdLFszLDIsIihyLFxcYmV0YSkiXV0=
\[\begin{tikzcd}
	{\widehat{A}.U} & {\widehat{A}.Y} \\
	{\widehat{A}.X} & {\widehat{A}.Z}
	\arrow["{(f,\phi)}"', from=2-1, to=2-2]
	\arrow["{(g,\psi)}", from=1-2, to=2-2]
	\arrow["{(l,\alpha)}"', from=1-1, to=2-1]
	\arrow["{(r,\beta)}", from=1-1, to=1-2]
\end{tikzcd}\]
        whose image under $U^\anc$ is a $T$-pullback in $\C$, and so that the square in $\wone$ is a transverse $T$-pullback:
% https://q.uiver.app/?q=WzAsOCxbMCwxLCJ7QX0uWCJdLFsxLDEsIntBfS5aIl0sWzEsMCwie0F9LlkiXSxbMCwwLCJ7QX0uVSJdLFsyLDAsIlUiXSxbMywwLCJZIl0sWzIsMSwiWCJdLFszLDEsIloiXSxbMCwxLCJmIiwyXSxbMiwxLCJnIl0sWzMsMCwibCIsMl0sWzMsMiwiciJdLFs2LDcsIlxccGhpIiwyXSxbNSw3LCJcXHBzaSJdLFs0LDUsIlxcYmV0YSJdLFs0LDYsIlxcYWxwaGEiLDJdLFszLDEsIiIsMSx7InN0eWxlIjp7Im5hbWUiOiJjb3JuZXIifX1dLFs0LDcsIiIsMSx7InN0eWxlIjp7Im5hbWUiOiJjb3JuZXIifX1dXQ==
\[\begin{tikzcd}
	{{A}.U} & {{A}.Y} & U & Y \\
	{{A}.X} & {{A}.Z} & X & Z
	\arrow["f"', from=2-1, to=2-2]
	\arrow["g", from=1-2, to=2-2]
	\arrow["l"', from=1-1, to=2-1]
	\arrow["r", from=1-1, to=1-2]
	\arrow["\phi"', from=2-3, to=2-4]
	\arrow["\psi", from=1-4, to=2-4]
	\arrow["\beta", from=1-3, to=1-4]
	\arrow["\alpha"', from=1-3, to=2-3]
	\arrow["\lrcorner"{anchor=center, pos=0.125}, draw=none, from=1-1, to=2-2]
	\arrow["\lrcorner"{anchor=center, pos=0.125}, draw=none, from=1-3, to=2-4]
\end{tikzcd}\]
      Then $U^\anc$ reflects the limit; that is, the original square in $\mathsf{Span}(\pi,\xi,\lambda,\anc)$ is a $T$-pullback.
      \item $T$-pullbacks of the form described in (ii) are closed under $\boxtimes$.
    \end{enumerate}
\end{lemma}
\begin{proof}
    The functor in in (i) is straightforward to construct, as it simply forgets the left and right legs of the spans. For (ii), note that because the $\wone$ part of the diagram is a transverse $T$-pullback, then given a pair of maps
    % https://q.uiver.app/?q=WzAsNSxbMSwyLCJcXHdpZGVoYXR7QX0uWCJdLFsyLDIsIlxcd2lkZWhhdHtBfS5aIl0sWzIsMSwiXFx3aWRlaGF0e0F9LlkiXSxbMSwxLCJcXHdpZGVoYXR7QX0uVSJdLFswLDAsIlxcd2lkZWhhdHtBfS5WIl0sWzAsMSwiKGYsXFxwaGkpIiwyXSxbMiwxLCIoZyxcXHBzaSkiXSxbMywwLCIobCxcXGFscGhhKSIsMl0sWzMsMiwiKHIsXFxiZXRhKSJdLFs0LDAsIih4LFxcb21lZ2EpIiwyLHsiY3VydmUiOjJ9XSxbNCwyLCIoeSxcXGdhbW1hKSIsMCx7ImN1cnZlIjotMn1dXQ==
\[\begin{tikzcd}
	{\widehat{A}.V} \\
	& {\widehat{A}.U} & {\widehat{A}.Y} \\
	& {\widehat{A}.X} & {\widehat{A}.Z}
	\arrow["{(f,\phi)}"', from=3-2, to=3-3]
	\arrow["{(g,\psi)}", from=2-3, to=3-3]
	\arrow["{(l,\alpha)}"', from=2-2, to=3-2]
	\arrow["{(r,\beta)}", from=2-2, to=2-3]
	\arrow["{(x,\omega)}"', curve={height=12pt}, from=1-1, to=3-2]
	\arrow["{(y,\gamma)}", curve={height=-12pt}, from=1-1, to=2-3]
\end{tikzcd}\]
    a unique span morphism $\widehat{A}.V \to \widehat{A}.U$ may be induced using the apex map from $\C$, and the unique map induced in $\wone$ by the universality of transverse squares (this square is also universal in $\C$), so the span morphism diagram will commute by universality.
    
    For (iii), $T$-pullback squares of the form in (ii) are closed under $\boxtimes$ as transverse squares in $\wone$ are closed under $\ox$, so the result follows by the commutativity of limits and by applying part (ii) of this lemma. 
\end{proof}
% \begin{definition}%
% \label{def:prolongation-of-anchored-bundle}
%     Let $(\pi:A \to M, \xi, \lambda, \anc)$ be an anchored bundle in a tangent category $\C$. 
%     This anchored bundle defines a mapping on objects:
%     \[
%         \hat A: \mathsf{Ob}(\wone) \to \mathsf{Ob}(\C); V \mapsto \hat A .V
%     \]  
%     defined inductively using a choice of span compositions from Definition \ref{def:boxtimes-span}.

%     For $W_n$, the object is $A.W_n$. There is a canonical span:
%     \[
%       A_n: M \to T_nM := M \xleftarrow[]{\pi \o \pi_i} A_n \xrightarrow[]{\anc_n} T_n.M    
%     \]
%     The \emph{prolongation of $A$ by $V \in \wone$} is the span composition:
%     \[
%       V = W_{n(1)} \ox \dots W_{n(k)} \mapsto
%       \hat A. W_{n(1)} \boxtimes \dots \boxtimes\hat A. W_{n(k)} =: \hat . V
%     \]
%     Write the span maps:
%     \[
%         \hat A.V : M \to T^VM :=
%         M \xleftarrow{\pi^V} \hat A.V  \xrightarrow{\anc^V} T^VM  
%     \]
%     An anchored bundle has \emph{chosen prolongations} when all $\hat A. V$ exist and for each $U,V$ the equation:
%     \[
%         \hat A. U \boxtimes \hat A. V = \hat A. UV   
%     \]
%     holds. The $V$-prolongation of $(\pi, \xi, \lambda, \anc)$ is the apex of the $V$-span. 
% \end{definition}

% Whenever an anchored bundle has chosen prolongations, it is possible to identify the following monoidal category.
%TODO fill in this definition
%TODO fill in this proposition
% The tensor product 
% The monoidal category to be construction for an involution algebroid, then, takes the span determined by its underlying anchored bundle, and uses span composition to construct a span for each Weil algebra $V \in \wone$. The construction is inductive, and is outlined below:
% A ``higher order'' prolongation may be constructed for each Weil algebra $U \in \wone$, following the metaphor that the space of prolongations $\prol(A)$ is like the second tangent bundle, and $\prol^2(A)$ the third tangent bundle.



\begin{observation}%
    \label{obs:concrete-desc-zigzag}
    It will be useful to have a ``flat'' presentation of the prolongation $A.W_{n(1)}\dots W_{n(k)}$. 
    % Inductively, look at the prolongation $\prol(W_mW_nV,A)$:
    % \[\input{TikzDrawings/Ch4/prolong-ind-case.tikz}\]
    % But apply the same construction for $T_m.\prol(W_nV,A)$ to get the double pullback:
    % \[\input{TikzDrawings/Ch4/prolong-ind-case-2.tikz}\]
    The higher prolongations of an anchored bundle may be concretely described as the $T$-pullback of the zig-zag below:
    \[\input{TikzDrawings/Ch4/Sec3/prol-zigzag.tikz}\]
    so that the prolongation $A.W^{n[1]}\dots W^{n[k]}$ may be written concretely as
    \[
      (u_1,\dots, u_{k}) : A_{n[1]} \ts{\anc'}{T.\pi'}  T_{n[1]}.A_{n[2]} \ts{T.\anc}{T^2.\pi} \dots \ts{\anc'}{T.\pi'} T_{n[1]\dots n[k-1]}A_{n[k]}.
    \]
    Furthermore, the choice of prolongation identifies the following limits:
    \[\input{TikzDrawings/Ch4/prol-abuse.tikz}\input{TikzDrawings/Ch4/prol-abuse-2.tikz}\]
    so that
    \[
        A. UV
        = A.U \boxtimes A.V
        = A.U \ \boxtimes id_M \boxtimes A.V
    \]
    where $id_M$ is the span $M = M = M$.
\end{observation}

% \begin{example}
%     ~\begin{enumerate}[(i)]
%         \item For the tangent bundle, $\widehat{TM}.V = T^VM$. 
%         \item For a trivial bundle, $(\pi:A \to M, \xi, \lambda, 0 \o \pi)$, then $A_V = A_{|V|}$ where $|V| \in \N$ denotes the dimension of the underlying free commutative monoid of $V$.
%     \end{enumerate}
% \end{example}

%
% \begin{definition}
%     \label{def:boxtimes}
%     Let $(\pi:A \to M, \xi, \lambda, \anc)$ be an anchored bundle with chosen prolongations. 
%     For any pair of span morphisms $f:U \to U', g:V \to V'$ of the form:
%     \begin{equation}
%         \label{eq:span-form}
%         \input{TikzDrawings/Ch4/span-form.tikz}
%     \end{equation}
%     The horizontal composition $f \boxtimes g$ is defined:
%     \[\input{TikzDrawings/Ch4/boxtimes-diag.tikz}\]
% \end{definition}
% \begin{lemma}
%     Let $(\pi:A \to M, \xi, \lambda, \anc)$ be an anchored bundle with chosen prolongations.
%     Consider a commuting cospan in $\mathsf{Span}(\C)$ of the form:

%     Now the 
% \end{lemma}
% Given an anchored bundle $(\pi:A \to M, \xi,\lambda, \anc)$, a choice of prolongations allows for the construction of a strict monoidal category with a bijective correspondence on objects with the category of Weil algebras.
                   
Note that a $\anc$ sends the involution algebroid structure map to its corresponding tangent structure map. Each of the structure maps, then, gives a span morphism where $\boxtimes$ is well defined:
\begin{definition}%
    \label{def:generators-for-wone-in-anc}
    Let $(\pi:A \to M, \xi, +_q, \lambda, \anc, \sigma)$ be an involution algebroid in $\C$ with chosen prolongations.
    Then define the following maps in $\mathsf{Span}(\pi,\xi,\lambda,\anc)$:
    \begin{itemize}
        \item The projection $p: \hat{A}.W \to \hat{A}.\N$,
        \[\input{TikzDrawings/Ch4/Sec4/proj.tikz}\]
        \item The zero map $0: \hat{A}.\N \to \hat{A}.W$,
        \[\input{TikzDrawings/Ch4/Sec4/zero.tikz}\]
        \item The addition map $+:  \hat{A}.W_2 \to \hat{A}.W$,
        \[\input{TikzDrawings/Ch4/Sec4/add.tikz}\]
        \item The lift map $\ell: \hat{A}.W \to \hat{A}.WW$,
        \[\input{TikzDrawings/Ch4/Sec4/lift.tikz}\]
        \item The flip map $c: \hat{A}.WW \to \hat{A}.WW$,
        \[\input{TikzDrawings/Ch4/Sec4/flip.tikz}\]
    \end{itemize}
\end{definition}
The idea is to show that the monoidal category of chosen prolongations $\mathsf{Span}(\pi,\xi,\lambda,\anc)$ for an involution algebroid has a tangent structure generated by the structure maps in Definition \ref{def:generators-for-wone-in-anc} and the endofunctor $\widehat{A}.W \boxtimes (-)$. Using the flat presentation, we can then show that $U^\anc$ will determine a tangent functor in $\C$. The following lemma about $\anc$ will be useful in constructing the natural transformation part of a tangent functor.
% The idea is to prove that the generators define a functor $\wone \to \C$. 
% First, following table associates maps in $\wone$ to the structure maps of an involution algebroid:
% \begin{center}
%     \begin{tabular}{|l|l|l|l|l|l|l|}
%     \hline
%     Tangent bundle & $T^VM$ & $p$   & $0$   & $+$     & $\ell$    & $c$      \\ \hline
%     Involution algebroid & $\hat{A}.V$     & $\pi$ & $\xi$ & $+_\pi$ & $\hat\lambda$ & $\sigma$ \\ \hline
%     \end{tabular}
% \end{center}
% Having a flat presentation of these maps will prove useful.
% \begin{proposition}\label{def:higher-morphisms}
%     Let $(\pi:A \to M, \xi, \lambda, \sigma)$ be an involution algebroid with a chosen prolongations. By the earlier observation, identify $\prolong$ with $A \ts{\anc}{id} TM \ts{id}{T\pi} TA$. For any pair of Weil algebras $U,V$, define the following morphisms (note that $\anc^V_\N$ is the original $\anc^V$ map)
%     \begin{enumerate}[(i)]
%         \item \input{TikzDrawings/Ch4/LocMaps/proj.tikz}
%         \item \input{TikzDrawings/Ch4/LocMaps/zero.tikz}
%         \item \input{TikzDrawings/Ch4/LocMaps/plus.tikz}
%         \item \input{TikzDrawings/Ch4/LocMaps/lift.tikz}
%         \item \input{TikzDrawings/Ch4/LocMaps/flip.tikz}
%         % \item \input{TikzDrawings/Ch4/LocMaps/anc.tikz}
%     \end{enumerate}
% \end{proposition}
\begin{definition}\label{def:anc-nat}
    Let $(\pi:A \to M, \xi, \lambda, \anc)$ be an anchored bundle with chosen prolongations. 
    Recall that by Definition \ref{def:monoidal-category}, the right leg of $A.U$ is written $\anc$, so it induces a span map:\[\input{TikzDrawings/Ch4/anc-uv.tikz}\]
    This map has a flat presentation as
    \[
        \input{TikzDrawings/Ch4/LocMaps/anc.tikz}        
    \]
    We write the map
    \[
        \anc^U.V := \anc^U \boxtimes (\widehat A.V)
    \]
    which corresponds to the following span morphism:
    % https://q.uiver.app/?q=WzAsOSxbMCwyLCJNIl0sWzEsMSwiQS5VIl0sWzIsMiwiVF5VTSJdLFszLDEsIlReVS5BLlYiXSxbNCwyLCJUXntVVn1NIl0sWzEsMywiVF5VTSJdLFszLDMsIlReVS5BLlYiXSxbMiwwLCJBLlVWIl0sWzIsNCwiVF5VLkEuViJdLFsxLDUsIlxcYW5jXlUiXSxbMyw2LCIiLDAseyJsZXZlbCI6Miwic3R5bGUiOnsiaGVhZCI6eyJuYW1lIjoibm9uZSJ9fX1dLFszLDQsIlReVS5cXGFuY15WIiwxXSxbMywyLCJUXlUuXFxwaV5WIiwxXSxbMSwyLCJcXGFuY15VIiwxXSxbMSwwLCJcXHBpXlUiLDFdLFs4LDVdLFs4LDYsIiIsMSx7ImxldmVsIjoyLCJzdHlsZSI6eyJoZWFkIjp7Im5hbWUiOiJub25lIn19fV0sWzcsMV0sWzcsM10sWzcsMiwiIiwxLHsic3R5bGUiOnsibmFtZSI6ImNvcm5lciJ9fV0sWzUsMCwicF5VLk0iLDFdLFs1LDIsIiIsMSx7ImxldmVsIjoyLCJzdHlsZSI6eyJoZWFkIjp7Im5hbWUiOiJub25lIn19fV0sWzYsMiwiVF5VLlxccGleViIsMV0sWzYsNF1d
\[\begin{tikzcd}
	&& {A.UV} \\
	& {A.U} && {T^U.A.V} \\
	M && {T^UM} && {T^{UV}M} \\
	& {T^UM} && {T^U.A.V} \\
	&& {T^U.A.V}
	\arrow["{\anc^U}", from=2-2, to=4-2]
	\arrow[Rightarrow, no head, from=2-4, to=4-4]
	\arrow["{T^U.\anc^V}"{description}, from=2-4, to=3-5]
	\arrow["{T^U.\pi^V}"{description}, from=2-4, to=3-3]
	\arrow["{\anc^U}"{description}, from=2-2, to=3-3]
	\arrow["{\pi^U}"{description}, from=2-2, to=3-1]
	\arrow[from=5-3, to=4-2]
	\arrow[Rightarrow, no head, from=5-3, to=4-4]
	\arrow[from=1-3, to=2-2]
	\arrow[from=1-3, to=2-4]
	\arrow["\lrcorner"{anchor=center, pos=0.125, rotate=-45}, draw=none, from=1-3, to=3-3]
	\arrow["{p^U.M}"{description}, from=4-2, to=3-1]
	\arrow[Rightarrow, no head, from=4-2, to=3-3]
	\arrow["{T^U.\pi^V}"{description}, from=4-4, to=3-3]
	\arrow[from=4-4, to=3-5]
\end{tikzcd}\]
\end{definition}

% \begin{proposition}
%     Let $(\pi:A \to M, \xi, \lambda, \sigma)$ be an involution algebroid with a chosen prolongations.
%     The following coherences hold:
%     \begin{enumerate}
%         \item $\anc^{U}_V \o (\theta \boxtimes \phi) = 
%     \end{enumerate}
% \end{proposition}

% The following proposition establishes some coherences and universality conditions for the tangent bundle that are not directly axiomatized Definition \ref{def:involution-algd} but are necessary to prove Theorem  \ref{thm:weil-nerve}.
% \begin{proposition}\label{prop:higher-tangent-bundle-construction}
%     Let $(\pi:A \to M, \xi, \lambda, \anc, \sigma)$ be an involution algebroid in a tangent category $\C$.
%     It follows that:
%     \begin{enumerate}[(i)]
%         \item Coassociativity of $\hat{\lambda}$: $(\hat{\lambda}\x\ell)\o\hat{\lambda} = (id \x T.\hat{\lambda}) \o \hat{\lambda}$
%         \item Coherence between $\sigma$ and $\hat{\lambda}$, $(\sigma \x c) \o (1\x T\sigma) \o (\hat{\lambda},\ell) = (1 \x T\hat{\lambda}) \o \sigma$
%         \item Both of the diagrams are $T$-pullbacks:
%         \input{TikzDrawings/Ch3/Sec5/inv-algd-universality.tikz}
%     \end{enumerate}
% \end{proposition}
% \begin{proof}
%     ~\begin{enumerate}[(i)]
%         \item Compute:
%             \begin{align*}
%                 (\hat{\lambda}\x \ell)\o \hat{\lambda} 
%                 &= (\xi\o\pi\o\xi\o\pi, \lambda\o\xi\o\pi, \ell\o\lambda) \\
%                 &= (\xi\o\pi, T.(\xi\o\pi)\o\lambda, T.\lambda \o \lambda) \\
%                 &= (\pi_0, T.(\xi\o\pi)\o\pi_1, T.\lambda \o \pi_1)\o (\xi\o\pi, \lambda) \\
%                 &= (id \x T(\hat{\lambda}))\o\hat{\lambda}
%             \end{align*}
%         \item Compute:
%             \begin{align*}
%                 (1\x T.\sigma) \o (\sigma \x c) \o (1 \x T.\hat{\lambda}) 
%                 &= (1 \x T.\sigma) \o (\sigma \o (\pi_0, T.(\xi\o\pi)\o\pi_1), c \o T.\lambda) \\
%                 &= (1 \x T.\sigma) \o (\xi\o\pi\o\pi_0, 0\o\pi_0, c \o T.\lambda\o\pi_1) \\
%                 &= (\xi\o\pi\o\pi_0, T.\sigma\o(0\o\pi_0,c\o T.\lambda \o \pi_1) \\
%                 &= (\xi\o\pi\o\pi_0, \lambda\pi_0, \ell\o\pi_1) \o \sigma = (\hat{\lambda} \x \ell) \o \sigma
%             \end{align*}
%         \item  Use the pullback lemma to observe that the following diagram is universal for any anchored bundle
%             % https://q.uiver.app/?q=WzAsNixbMCwwLCJBXzIiXSxbMCwxLCJNIl0sWzEsMCwiXFxwcm9sKEEpIl0sWzEsMSwiQSJdLFsyLDAsIlRBIl0sWzIsMSwiVE0iXSxbMSwzLCJcXHhpIl0sWzMsNSwiXFxhbmMiXSxbMSw1LCIwIiwyLHsiY3VydmUiOjJ9XSxbNCw1LCJUXFxwaSJdLFsyLDMsIlxccGlfMCJdLFswLDEsIlxccGlcXG9cXHBpXzEiLDJdLFswLDIsIlxcaGF0e1xcbXV9IiwyXSxbMiw0LCJcXHBpXzEiLDJdLFswLDQsIlxcbXUiLDAseyJjdXJ2ZSI6LTJ9XV0=
%             \input{TikzDrawings/Ch3/Sec5/inv-algd-mu-universal-proof.tikz}
%             Define the map $\hat{\mu}(a,b) := (\xi\o \pi a, \lambda \o a) +_{\pi_0} (\xi\o \pi\o  b, 0\o b)$ so the top triangle of the diagram commutes. The right square and outer perimeter are pullbacks by definition, and the bottom triangle also commutes by definition. The pullback lemma ensures that the left square is a pullback - this means that for every anchored bundle, the general lift is universal for $\prol(A)$. Now post-compose with the involution:
%             \input{TikzDrawings/Ch3/Sec5/inv-algd-nu-universal.tikz}
%             It suffices to check that the top triangle commutes, so $\sigma \o \hat{\mu} = \nu$:
%             \[
%                 \sigma \o \hat{\mu}\o (a,b) = \sigma \o   ((\xi\o \pi,0)\o a +_{\pi_0} (\xi\o \pi,\lambda)\o b) = 
%                  (id, T.\xi \o \anc \o a) +_{p\pi_1} (\xi\o \pi,\lambda)\o b
%             \]
%             The lift $(\xi\pi,\lambda)$ involution algebroid is universal for $\prol(A)$.
%     \end{enumerate}
% \end{proof}


% \begin{proposition}%
%     \label{prop:wone-anc-strict-moncat}
%     The category $\wone^\anc$ is a strict monoidal category, where the tensor product on objects $\prol(U,A),\prol(V,A)$ is the chosen span prolongation $\prol(UV,A)$ and the unit is $\prol(\N, A)$. On morphisms, the tensor product:
%     \[
%         \infer{
%             (f,\theta)\ox (g,\phi) = (f \ts{\anc^U}{\theta.\pi^V} T^U.g, f.g.M )
%         }{
%             (f,\theta):U \to V & (g,\phi): X \to Y
%         }  
%     \]
%     induced by the span composition:
%     \[\input{TikzDrawings/Ch4/Sec3/tensor.tikz}\]
%     And the unit is the identity span.
% \end{proposition}
% \begin{observation}%
%     \label{obs:apex-functor}
%     % The category $\wone^\anc$ is a subcategory of the category of spans in $\C$. Suppose a diagram $D: \d \to \wone^\anc$
%     There is a functor $U^\anc: \wone^\anc \to \C$ so that projects out the apex map, sending:
%     \[
%         (f,\theta): \prol(A,U) \to \prol(A,V) \in \wone^\anc  
%     \] 
%     to 
%     \[
%         f: \prol(A,U) \to \prol(A,V) \in \C
%     \]
%     Similarly, there is a functor from $U^\C: \wone^\anc \to \wone$ that sends:
%     \[
%         (f,\theta): \prol(A,U) \to \prol(A,V) \in \wone^\anc  
%     \]
%     to 
%     \[
%         \theta:U \to V  
%     \]
% \end{observation}
% \begin{proposition}%
%     \label{prop:limits-in-wanc}
%     Let $(\pi:A \to M, \xi, \lambda, \anc)$ be anchored bundle with chosen prolongations in a tangent category $\C$. Consider a commuting square $D$ 
%     \[\input{TikzDrawings/Ch4/pb-wanc.tikz}\]
%     so that $U^\C(D) \in \wone$ is a transverse limit, and $U^\anc(D)\in \C$ is a $T$ limit.
%     Then this square is a pullback in $\wone^\anc$ that is sent to a $T$-limit by $U^\anc$.
% \end{proposition}
% \begin{proof}
%     Start with a pair:
%     \[\input{TikzDrawings/Ch4/pb-wanc-2.tikz}\]
%     This square is a pullback in the category of spans by Observation \ref{obs:lim-of-spans}, so it suffices to prove that the induced map is in $\wone^\anc$. Consider the diagram:
%     \[\input{TikzDrawings/Ch4/pb-wanc-3.tikz}\]
%     The map $(\alpha,\beta):Q \to U \in \wone$ by transversality, and the diagram commutes because it is the limit in $\mathsf{Span}(\C)$. 
% \end{proof}
% \begin{corollary}
%     The tensor product in $\wone$ is continuous in each variable for limits of the form \Cref*{prop:limits-in-wanc}.
% \end{corollary}
% \begin{corollary}%
%     \label{cor:tang-limits-in-wone}
%     For any anchored bundle with chosen prolongations $(\pi:A \to M, \xi, \lambda, \anc)$ in a tangent category $\C$, the following diagrams are limits in $\wone^\anc$ and are sent to $T$-limits in $\anc$.
%     \begin{equation}
%         \label{eq:univ-lift}
%         \input{TikzDrawings/Ch4/univ-pb.tikz}
%     \end{equation}
%     \begin{equation}
%         \label{eq:span-limit}
%         \input{TikzDrawings/Ch4/span-pullback-diag.tikz}
%     \end{equation}
% \end{corollary}
% \begin{proof}
%     The base pullback in \Cref{eq:univ-lift} was proved to be a $T$-limit in part (iii) of Proposition  \ref{prop:higher-tangent-bundle-construction}, whereas \Cref{eq:span-limit} is part of the axioms of a differential bundle. Note that \Cref{eq:univ-lift} is mapped to the universality of the vertical lift in $\wone$, and \Cref{eq:span-limit} is sent to the pullback powers defining $W_n$.
% \end{proof}


% \begin{remark}
%     The simplicial object of an internal category requires a choice of $n$-fold span compositions $M \xleftarrow[]{s} C \xrightarrow[]{t} M$. The construction in Definition \ref{def:weil-anc-cat}, then, takes the full subcategory of $\C/(M \x M)$ and notes that the category of span compositions  $M \xleftarrow[]{s} C \xrightarrow[]{t} M$ is still a monoidal category.
% \end{remark}

% The construction of the structure maps builds on the prolongation construction in Definition \ref{def:prolongation-of-anchored-bundle}. First, following table associates maps in $\wone$ to the structure maps of an involution algebroid:
% \begin{center}
%     \begin{tabular}{|l|l|l|l|l|l|l|}
%     \hline
%     Tangent bundle & $T^2M,TM,M$ & $p$   & $0$   & $+$     & $\ell$    & $c$      \\ \hline
%     Involution algebroid & $\prol(A), A, M$     & $\pi$ & $\xi$ & $+_\pi$ & $\lambda$ & $\sigma$ \\ \hline
%     \end{tabular}
% \end{center}

% Rather that constructing an internal monoid in the span category, the Weil nerve induces a tangent structure on the monoidal category.
% \begin{proposition}%
%     \label{prop:wone-is-tang-cat}
%     Given an involution algebroid $(\pi:A \to M, \xi, \lambda, \anc, \sigma)$, the monoidal category $\wone^\anc$ has a tangent structure given by:
%     \begin{itemize}
%         \item $T := \prol(W,A) \ox (-), p := (\pi,p), 0:= (\xi,0), +:= (+_q,+)$.
%         \item $\ell := (\xi\o\pi,\lambda), c := (\sigma, c)$
%     \end{itemize}
% \end{proposition}
% \begin{proof}
%     Functoriality of $T$ and the naturality of $(p,0,+,\ell,c)$ are by construction. Now check each of the involution algebroid axioms:
%     \begin{enumerate}[{[TC.1]}]
%         \item Additive bundle axioms:
%         \begin{enumerate}[(i)]
%             \item By Corollary  \ref{cor:tang-limits-in-wone}, pullback powers of $p$ exist and are preserved by $T$.
%             \item The additive bundle structure comes from the addition induced 
%         \end{enumerate} 
%         \item Symmetry axioms:
%         \begin{enumerate}[(i)]
%             \item $c \o c$ follows from the involution axiom.
%             \item For Yang-Baxter, note that: 
%             \[c.T \o T.c \o c.T\] 
%             is equivalent to the Yang-Baxter equation on an involution algebroid:
%             \[
%               (\sigma \x c)  \o (id \x T.\sigma) \o (\sigma \x c) =
%               (id \x T.\sigma) \o (\sigma \x c)\o (id \x T.\sigma)
%             \]
%             and 
%             \[\sigma \x c = (c,\anc) \ox \prol(W,A) = c.T,
%             \hspace{0.15cm}
%             id \x T.\sigma = \prol(W,A) \ox (\sigma,c) = T.c\]
%             \item For the naturality conditions:
%             \begin{enumerate}[(a)]
%                 \item The interchange of $+,0,p$ all follow from the fact that $\sigma:(\prol(A),\lambda \x \ell) \to (\prol(A),id \x c \o T.\lambda)$ is linear, so it is an additive bundle morphism.
%                 \item The axiom 
%                 \[\ell.T \o c = T.c \o c.T \o T.\ell\] 
%                 is equivalent to the equation:
%                 \[
%                     (\sigma \x c) \o (1\x T.\sigma) \o (\hat{\lambda} \x \ell) = (1 \x T\hat{\lambda}) \o \sigma
%                 \]
%                 proved in (ii) of Proposition  \ref{prop:higher-tangent-bundle-construction} (which requires that $\sigma$ be bilinear).
%             \end{enumerate}
%         \end{enumerate}
%         \item The lift axioms:
%             \begin{enumerate}[(i)]
%                 \item The additive bundle equations are a consequence of $\lambda$ being a lift, and $+$ being the addition induced by the non-singularity of $\lambda$.
%                 \item The coassociativity axiom \[\ell.T \o \ell = T.\ell \o \ell\] 
%                 is equivalent to 
%                 \[(\hat{\lambda}\x\ell)\o\hat{\lambda} = (id \x T.\hat{\lambda}) \o \hat{\lambda}\] proved in (i) of Proposition  \ref{prop:higher-tangent-bundle-construction}.
%                 \item The symmetry of comultiplication, $c \o \ell = \ell$ is given by the unique equation for an involution algebroid, so that $\sigma \o (\xi\o\pi,\lambda) = (\xi\o\pi,\lambda)$.
%                 \item The universality of the lift follows from part (iii) of Proposition  \ref{prop:higher-tangent-bundle-construction} and the coherence on limits in Corollary  \ref{cor:tang-limits-in-wone}.
%             \end{enumerate}
%     \end{enumerate}
%     Observe that just as in the case for the equivalence between involution algebroids and Lie algebroids proved in Theorem  \ref{thm:iso-of-cats-Lie} and \Cref{sec:connections_on_an_involution_algebroid}, this used each equation and universality condition on an involution algebroid (the anchor is used in the construction of $\wone^\anc$).
%     % For [TC.1], that the pullback powers of $p$ exist and are preserved by $T$ follows by Corollary  \ref{cor:tang-limits-in-wone}, the additive bundle coherences follow from the involution algebroid axioms. The coherences in [TC.2] follow by the involution algebroid axioms and parts (i) and (ii) Proposition  \ref{prop:higher-tangent-bundle-construction}. The universality of the vertical lift axiom [TC.3] follows by Corollary  \ref{cor:tang-limits-in-wone}, so all of the coherences of a tangent structure are satisfied.
%     % To see that pullback powers of $p$ exist and are preserved by $T$, note that by the commutativity of pullbacks each $\prol(UW_nV,A)$ is the $n$-fold pullback of: 
%     % \[\input{TikzDrawings/Ch4/pullback-diag.tikz}\]
%     % This is also a pullback in the category $\wone^\anc$ - given a family of maps:
%     % \[
%     %     (f_i, \theta_i): \prol(X,A) \to   \prol(UWV,A), \hspace{.5cm}
%     %     (id_U \ox (\pi,p) \ox id_V) \o (f_i, \theta_i) = (f,\theta)
%     % \]
%     % then the following diagram commutes: 
%     % \begin{equation}
%     %     \label{eq:span-limit}
%     %     \input{TikzDrawings/Ch4/span-pullback-diag.tikz}
%     % \end{equation}
%     % The commutativity of limits in $\C$ ensures that this pullback is preserved by $\ox$, and consequently by $T^\anc$. Thus, this gives the additive bundle coherence [TC.1].
%     % The rest of the coherences in [TC.2] follow by the involution algebroid axioms and parts (i) and (ii) Proposition  \ref{prop:higher-tangent-bundle-construction}.
%     % For the universality of the vertical lift, the diagram:
%     % \[\input{TikzDrawings/Ch1/univ-lift.tikz}\]
%     % is a pullback by part (iii) of Proposition  \ref{prop:higher-tangent-bundle-construction}. 
%     % Then observe that by the commutativity of $T$-limits, the following diagram is a pullback in $\C$:
%     % \begin{equation}
%     %     \label{eq:univ-lift}
%     %     \input{TikzDrawings/Ch4/univ-pb.tikz}
%     % \end{equation}
%     % By the same argument as in \Cref{eq:span-limit}, the uniquely induced map
%     % \[\input{TikzDrawings/Ch4/univ-pb.tikz}\]
%     % will be a morphism in $\wone^\anc$ - exhibiting \Cref{eq:univ-lift} as a pullback in $\wone^\anc$ - this induces 
% \end{proof}

\begin{theorem}[The Weil Nerve]
    \label{thm:weil-nerve}
    There is a fully faithful functor
    \[
        \mathsf{N}_{\weil}: \mathsf{Inv}(\C) \to [\wone, \C]  
    \]
    that sends an involution algebroid to the transverse-limit-preserving tangent functor:
    \[
        (\widehat{A},\alpha): \wone \to \C 
    \]
\end{theorem}
\begin{proof}
    % Starting with objects, note that $V \mapsto \prol(V)$. %
    % \begin{center}
    %     \begin{tabular}{|l|l|}
    %         \hline
    %         $\wone$           & $\widehat{A}:\wone \to \C$                    \\ \hline
    %         $W_n$             & $\hat{A}.W_n$                                \\ \hline
    %         $U \ox V$         & $\hat{A}.U \boxtimes\hat{A}.V$             \\ \hline
    %         $+:T_2 \to T$     & $\hat{A}.+: \hat{A}.W_2 \to  \hat{A}.W$ \\ \hline
    %         $p:T \to \N$      & $\hat{A}.p: \hat{A}.W \to \hat{A}.\N$  \\ \hline
    %         $0: I \to T$      & $\hat{A}.0: \hat{A}.\N \to  \hat{A}.W)$  \\ \hline
    %         $\ell:T \to T^2$  & $\hat{A}.\ell:  \hat{A}.W \to  \hat{A}.WW$ \\ \hline
    %         $c:T^2 \to T^2$   & $\hat{A}.WW \to  \hat{A}.WW$                  \\ \hline
    %         $\phi \ox \theta$ & $ \hat{A}.\phi \boxtimes  \hat{A}.\theta$        \\ \hline
    %     \end{tabular}    
    % \end{center}
    For the first step of this proof, we show that an involution algebroid structure on an anchored bundle $(\pi:A \to M, \xi, \lambda, \anc)$ determines a tangent category structure on the monoidal category $\mathsf{Span}(\pi:A \to M, \xi, \lambda, \anc)$.
    
    We check that the endofunctor $\hat{A} \boxtimes (-)$ determines a tangent structure, with the structure maps given by Definition \ref{def:generators-for-wone-in-anc}:
    
    % The $\boxtimes$ operation preserves equations on the left and right argument and is continuous. Therefore, it suffices to check that equations hold on the basic generators of $\wone$, and that the three generating transverse limits (identity, $W_2 = W \ts{p}{p} W$, and universality of the lift) are sent to $T$-limits.
    \begin{enumerate}[{[TC.1]}]
        \item Additive bundle axioms:
        \begin{enumerate}[(i)]
            \item Use Lemma \ref{lemma:pullback-part-of-theorem} to see that \[ \hat{A}.W_2 = \hat{A}.W \ts{p}{p} \hat{A}.W  A \ts{\pi}{\pi} A;\] this is preserved by $\hat A.V \boxtimes (-)$.
            \item The triple $(\hat{A}.+,\hat{A}.p,\hat{A}.0) = (+_q, \pi, \xi)$ is an additive bundle induced by Proposition  \ref{prop:induce-abun}, and $\boxtimes$ preserves pullbacks (and therefore additive bundles), so the additive bundle axioms hold.
        \end{enumerate} 
        \item Symmetry axioms:
        \begin{enumerate}[(i)]
            \item $\hat{A}.c \o \hat{A}.c = id$ follows from the involution axiom $\sigma \o \sigma = id$.
            \item For Yang--Baxter, note that 
            \[ 
                (\hat{A} \boxtimes c) \o (c \boxtimes \hat{A}) \o (\hat{A} \boxtimes c)  = 
                (c \boxtimes \hat{A})  \o (\hat{A} \boxtimes c)  \o (c \boxtimes \hat{A}) 
            \] 
            follows from the Yang--Baxter equation on an involution algebroid
            \[
              (\sigma \x c)  \o (id \x T.\sigma) \o (\sigma \x c) =
              (id \x T.\sigma) \o (\sigma \x c)\o (id \x T.\sigma),
            \]
            since 
            \[\sigma \x c.A = (\hat{A}.c) \boxtimes (\hat{A}.W )
                \text{ and }
            id \x T.\sigma = (\hat{A}.W ) \boxtimes c. \]
            % Note that $\boxtims$ preserves equations, so this holds for $U \boxtimes (W \boxtimes c) \boxtimes V$ and $U \boxtimes (c \boxtimes W) \boxtimes V$.
            \item For the naturality conditions:
            \begin{enumerate}[(a)]
                \item The interchanges of $+,0,p$ all follow from the fact that 
                \[\sigma:(A.WW,\lambda \x \ell) \to (A.WW,id \x c \o T.\lambda)\] 
                is linear, and so is an additive bundle morphism.
                \item The axiom 
                \[\ell.T \o c = T.c \o c.T \o T.\ell\] 
                is equivalent to the equation
                \[
                    (\sigma \x c) \o (1\x T.\sigma) \o (\hat{\lambda} \x \ell) = (1 \x T\hat{\lambda}) \o \sigma
                \]
                which is equivalent to the double linearity axiom on $\sigma$ by Proposition  \ref{prop:nat-of-sigma-ell}.
            \end{enumerate}
        \end{enumerate}
        \item The lift axioms:
            \begin{enumerate}[(i)]
                \item The additive bundle equations are a consequence of $\lambda$ being a lift and $+$ being the addition induced by the non-singularity of $\lambda$.
                \item The coassociativity axiom 
                \[\ell.T \o \ell = T.\ell \o \ell\] 
                is equivalent to 
                \[(\hat{\lambda}\x\ell)\o\hat{\lambda} = (id \x T.\hat{\lambda}) \o \hat{\lambda}\] 
                proved in (i) of Proposition  \ref{prop:lift-axioms-anchor}.
                \item The symmetry of comultiplication, $c \o \ell = \ell$, is given by the unique equation for an involution algebroid, so that $\sigma \o (\xi\o\pi,\lambda) = (\xi\o\pi,\lambda)$.
                \item The universality of the lift follows from part (ii) of Proposition  \ref{prop:lift-axioms-anchor}; Lemma \ref{lemma:pullback-part-of-theorem} ensures that for any $V \in \wone$, $\widehat{A}.V \boxtimes \mu$ and $\mu \boxtimes \widehat{A}.V$ are universal.
            \end{enumerate}
    \end{enumerate}
    This lemma puts a tangent structure on $\mathsf{Span}(\pi,\xi,\lambda,\anc)$. Now consider the functor sending spans to the apex map, 
    \[
        U^\anc: \mathsf{Span}(\pi,\xi,\lambda,\anc) \to \C.
    \]
    The family of maps
    \[
        \{\anc^U.V: \anc^U \boxtimes \widehat A.V | U,V \in \wone\}
    \]
    gives a family of natural transformations
    \[
        \anc^{U}:A.T^U \Rightarrow T^U.A,
    \]
    so that the following pair constitute a tangent functor
    \[
      (U^\anc,\anc): \mathsf{Span}(\pi,\xi,\lambda,\anc) \to \C.
    \]
    Because the universality conditions on $\mathsf{Span}(\pi,\xi,\lambda,\anc)$ followed by reflecting limits in $\C$ using Lemma \ref{lemma:pullback-part-of-theorem}, it follows that $(U^\anc,\anc)$ will preserve the tangent-natural limits in $\mathsf{Span}(\pi,\xi,\lambda,\anc)$ corresponding to transverse limits in $\wone$.
    
    By Leung's Theorem  \ref{thm:leung} (by way of Corollary \ref{cor:using-leung-thm}), the tangent structure on $\mathsf{Span}(\pi,\xi,\lambda,\anc)$ induces a strict, monoidal, transverse-limit-preserving functor
    \[
        \bar{A}: \wone \to \mathsf{Span}(\pi,\xi,\lambda,\anc)    
    \]
    that sends the tensor product $\ox$ to the span composition $\boxtimes$. By composing the strict tangent functor $(\bar{A},id)$ and $(U^\anc,\anc)$, we have a lax, transverse-limit-preserving, tangent functor:
    \[
        (A,\anc):\wone \to \C; V \mapsto A.V
    \]
    

    % The natural transformation is defined \[\alpha: \hat{A}.UV \Rightarrow T^U.\hat{A}.V = \anc^U \boxtimes \hat{A}.V \] using the definition of $\anc$ given in Definition \ref{def:anc-nat}. First look at $\anc^U \boxtimes \prol(V,A) \o (f \boxtimes g)$:
    % \[\input{TikzDrawings/Ch4/Sec4/anc-is-nat.tikz}\]
    % Then look at $\theta.g \o (\anc^U \boxtimes \prol(V,A))$:
    % \[\input{TikzDrawings/Ch4/Sec4/anc-is-nat-2.tikz}\]
    % The two maps induced from $\hat{A}.UV \to T^{U'}.\hat{A}.V'$ are equal, so $\alpha$ is natural.
    % The composite span is given by $\theta.g $
    % The assignment on objects was induced by Corollary  \ref{cor:actual-nerve}, note that a strict tangent functor $\wone \to \wone^\anc$ will preserve all transverse limits, and that $U^\anc:\wone^\anc \to \C$ will preserve tangent limits (as tangent limits in $\wone^\anc$ are computed ``pointwise'' in $\C$), all that remains is the bijection on morphisms.

    Now, check the bijection on morphisms. Starting with an involution algebroid morphism $(f,m):A \to B$, note that this gives a span morphism $\hat f$:
    % https://q.uiver.app/?q=WzAsNixbMCwwLCJNIl0sWzEsMCwiQSJdLFsyLDAsIlRNIl0sWzAsMSwiTiJdLFsxLDEsIkIiXSxbMiwxLCJUTiJdLFsyLDUsIlQubSJdLFswLDMsIm0iXSxbMSwwXSxbMSwyXSxbNCw1XSxbNCwzXSxbMSw0LCJmIiwxXV0=
\[\begin{tikzcd}[ampersand replacement=\&]
	M \& A \& TM \\
	N \& B \& TN
	\arrow["{T.m}", from=1-3, to=2-3]
	\arrow["m", from=1-1, to=2-1]
	\arrow[from=1-2, to=1-1]
	\arrow[from=1-2, to=1-3]
	\arrow[from=2-2, to=2-3]
	\arrow[from=2-2, to=2-1]
	\arrow["f", from=1-2, to=2-2]
\end{tikzcd}\]
    This gives a natural definition of $\hat{f}.V$ using the horizontal composition of span morphism, so that 
    \begin{equation}\label{eq:boxtimes-def-of-hat-f}
        \hat{f}.(UV) = \hat{f}.U \boxtimes \hat{f}.V \text{ and } \hat{f}.\N = m,
    \end{equation}
    giving a family of maps $\{\hat{f}_{U}: U \in \mathsf{objects}(\wone)\}$. Because $f$ will commute with the structure maps $\{ \pi,\xi,+,(\xi\o\pi,\lambda),\sigma\}$, it follow immediately that $\hat{f}$ is a natural transformation, because the following calculation holds for each $\theta:X \to Y \in \{ p,0,+,\ell,c\}$:
    \begin{align*}
        & \quad \hat{f}.UYV \o (\hat{A}.U \boxtimes \theta \boxtimes \hat{A}.V) \\
        &= (\hat{f}.U \boxtimes \hat{f}.Y \boxtimes \hat{f}.V) \o  (\hat{A}.U \boxtimes \theta \boxtimes \hat{A}.V) \\
        &= \hat{f}.U \boxtimes( \hat{f}.Y \o \theta)\boxtimes \hat{f}.V \\
        &= \hat{f}.U \boxtimes( \theta \o \hat{f}.X) \boxtimes \hat{f}.V \\
        &= (\hat{A}.U \boxtimes \theta \boxtimes \hat{A}.V) \o \hat{f}.UXV
        % =& (\prol(U,B) \ts{}{} T^U.\theta \ts{}{} T^{UY}.\prol(U,B)) \o (\prol(U,f) \ts{}{} T^U.\prol(X,f) \ts{}{} T^{UY}.\prol(V,f)) \\
        % =& (\prol(U,B) \boxtimes \theta \boxtimes \prol(U,B)) \o\prol(UXV,f) 
    \end{align*}
    Tangent naturality will follow by the preservation of the anchor map by $f$. 
    The equality, for any Weil algebra $U$, of the diagrams
% https://q.uiver.app/?q=WzAsMTgsWzAsMCwiTSJdLFsxLDAsIkEuVSJdLFsyLDAsIlReVU0iXSxbMCwxLCJOIl0sWzEsMSwiQi5VIl0sWzIsMSwiVF5VTiJdLFswLDIsIk4iXSxbMSwyLCJUXlVOIl0sWzIsMiwiVF5VTiJdLFs0LDEsIlReVU0iXSxbNCwwLCJBLlUiXSxbMywwLCJNIl0sWzUsMCwiVF5VTSJdLFszLDEsIk0iXSxbNSwxLCJUXlVNIl0sWzMsMiwiTiJdLFs0LDIsIlReVk4iXSxbNSwyLCJUXlZOIl0sWzIsNSwiVF5VLm0iXSxbMCwzLCJtIl0sWzEsMCwiXFxwaV5VIiwyXSxbMSwyLCJcXGFuY15VIl0sWzQsNSwiXFxhbmNeVSJdLFs0LDNdLFsxLDQsIlxcaGF0e2Z9LlUiLDFdLFs0LDcsIlxcYW5jXlUiXSxbNyw2LCJwXlUiXSxbMyw2LCIiLDAseyJsZXZlbCI6Miwic3R5bGUiOnsiaGVhZCI6eyJuYW1lIjoibm9uZSJ9fX1dLFs1LDgsIiIsMCx7ImxldmVsIjoyLCJzdHlsZSI6eyJoZWFkIjp7Im5hbWUiOiJub25lIn19fV0sWzcsOCwiIiwwLHsibGV2ZWwiOjIsInN0eWxlIjp7ImhlYWQiOnsibmFtZSI6Im5vbmUifX19XSxbMTAsMTEsIlxccGleVSIsMl0sWzksMTMsInBeVSIsMl0sWzEwLDEyLCJcXGFuY15VIiwyXSxbOSwxNCwiIiwyLHsibGV2ZWwiOjIsInN0eWxlIjp7ImhlYWQiOnsibmFtZSI6Im5vbmUifX19XSxbMTYsMTcsIiIsMix7ImxldmVsIjoyLCJzdHlsZSI6eyJoZWFkIjp7Im5hbWUiOiJub25lIn19fV0sWzE2LDE1LCJwXlUiXSxbMTIsMTQsIiIsMSx7ImxldmVsIjoyLCJzdHlsZSI6eyJoZWFkIjp7Im5hbWUiOiJub25lIn19fV0sWzE0LDE3LCJUXlUubSJdLFs5LDE2LCJUXlUubSJdLFsxMywxNSwibSIsMl0sWzExLDEzLCIiLDIseyJsZXZlbCI6Miwic3R5bGUiOnsiaGVhZCI6eyJuYW1lIjoibm9uZSJ9fX1dLFsxMCw5LCJcXGFuY15VIl0sWzUsMTMsIj0iLDEseyJzdHlsZSI6eyJib2R5Ijp7Im5hbWUiOiJub25lIn0sImhlYWQiOnsibmFtZSI6Im5vbmUifX19XV0=
\[\begin{tikzcd}[ampersand replacement=\&]
	M \& {A.U} \& {T^UM} \& M \& {A.U} \& {T^UM} \\
	N \& {B.U} \& {T^UN} \& M \& {T^UM} \& {T^UM} \\
	N \& {T^UN} \& {T^UN} \& N \& {T^VN} \& {T^VN}
	\arrow["{T^U.m}", from=1-3, to=2-3]
	\arrow["m", from=1-1, to=2-1]
	\arrow["{\pi^U}"', from=1-2, to=1-1]
	\arrow["{\anc^U}", from=1-2, to=1-3]
	\arrow["{\anc^U}", from=2-2, to=2-3]
	\arrow[from=2-2, to=2-1]
	\arrow["{\hat{f}.U}"{description}, from=1-2, to=2-2]
	\arrow["{\anc^U}", from=2-2, to=3-2]
	\arrow["{p^U}", from=3-2, to=3-1]
	\arrow[Rightarrow, no head, from=2-1, to=3-1]
	\arrow[Rightarrow, no head, from=2-3, to=3-3]
	\arrow[Rightarrow, no head, from=3-2, to=3-3]
	\arrow["{\pi^U}"', from=1-5, to=1-4]
	\arrow["{p^U}"', from=2-5, to=2-4]
	\arrow["{\anc^U}"', from=1-5, to=1-6]
	\arrow[Rightarrow, no head, from=2-5, to=2-6]
	\arrow[Rightarrow, no head, from=3-5, to=3-6]
	\arrow["{p^U}", from=3-5, to=3-4]
	\arrow[Rightarrow, no head, from=1-6, to=2-6]
	\arrow["{T^U.m}", from=2-6, to=3-6]
	\arrow["{T^U.m}", from=2-5, to=3-5]
	\arrow["m"', from=2-4, to=3-4]
	\arrow[Rightarrow, no head, from=1-4, to=2-4]
	\arrow["{\anc^U}", from=1-5, to=2-5]
	\arrow["{=}"{description}, draw=none, from=2-3, to=2-4]
\end{tikzcd}\]
is precisely the tangent-naturality condition from Definitions \ref{def:tang-nat}, \ref{def:actegory-natural}.
    
    For the inverse of this mapping, consider a tangent natural transformation (Definition \ref{def:tang-nat})
    \[
        \gamma: ({A},\alpha) \to ({B}, \beta),  \hspace{0.15cm}
            \input{TikzDrawings/Ch1/tang-nat-AB.tikz}
    \]
    where $(A,\alpha)$ and $(B,\beta)$ are tangent functors $\wone \to \C$ built out of involution algebroids with chosen prolongations. For any $U,V$, the map $\gamma.UV$ decomposes as $\gamma.U \boxtimes \gamma.V$:
    \[
        \input{TikzDrawings/Ch4/cube-map-nat.tikz}   
    \]
    Applying this relationship inductively, it is clear that the base maps $\gamma.W$ and $\gamma.\N$ determine the entire morphism $\gamma.V$:
% https://q.uiver.app/?q=WzAsMTYsWzAsMiwiTSJdLFsyLDIsIlRfe25bMV19Lk0iXSxbMywyLCJcXGRvdHMiXSxbNCwyLCJUX3tuWzFdfVxcZG90cyBUX3tuW2tdfS5NIl0sWzQsMywiVF97blsxXX1cXGRvdHMgVF97bltrXX0uTiJdLFsyLDMsIlRfe25bMV19Lk0iXSxbMCwzLCJOIl0sWzEsMiwiQV97blsxXX0iXSxbMSwzLCJCX3tuWzFdfSJdLFszLDMsIlxcZG90cyJdLFsxLDAsIk0iXSxbMSwxLCJOIl0sWzMsMCwiVF5VTSJdLFszLDEsIlReVU4iXSxbMiwwLCJBLlUiXSxbMiwxLCJCLlUiXSxbMyw0LCJUX3tuWzFdfVxcZG90cyBUX3tuW2tdfS5cXGdhbW1hLlxcTiJdLFswLDYsIlxcZ2FtbWEuXFxOIl0sWzcsMSwiXFxhbmNfe25bMV19Il0sWzcsMCwiXFxwaVxcb1xccGlfaSIsMl0sWzgsNiwiXFxwaVxcb1xccGlfaSJdLFs4LDUsIlxcYW5jX3tuWzFdfSIsMl0sWzcsOCwiXFxnYW1tYS5XX3tuWzFdfSJdLFsxLDUsIlRfe25bMV19LlxcZ2FtbWEuXFxOIl0sWzIsMSwiVF97blsxXX0uKFxccGkgXFxvIFxccGlfaSkiLDJdLFsyLDNdLFs5LDUsIlRfe25bMV19LihcXHBpIFxcbyBcXHBpX2kpIl0sWzksNF0sWzIsOSwiXFxkb3RzIiwxLHsic3R5bGUiOnsiYm9keSI6eyJuYW1lIjoibm9uZSJ9LCJoZWFkIjp7Im5hbWUiOiJub25lIn19fV0sWzEyLDEzLCJUXlUuXFxnYW1tYS5cXE4iLDJdLFsxNCwxMCwiXFxwaV5VIiwyXSxbMTUsMTEsIlxccGleVSJdLFsxNSwxMywiXFxhbmNeVSIsMl0sWzE0LDEyLCJcXGFuY15VIl0sWzEwLDExLCJcXGdhbW1hLlxcTiIsMl0sWzE0LDE1LCJcXGdhbW1hLlUiLDJdLFszNSwyMywiPSIsMSx7InNob3J0ZW4iOnsic291cmNlIjoyMCwidGFyZ2V0IjoyMH0sInN0eWxlIjp7ImJvZHkiOnsibmFtZSI6Im5vbmUifSwiaGVhZCI6eyJuYW1lIjoibm9uZSJ9fX1dXQ==
\[\begin{tikzcd}
	& M & {A.U} & {T^UM} \\
	& N & {B.U} & {T^UN} \\
	M & {A_{n[1]}} & {T_{n[1]}.M} & \dots & {T_{n[1]}\dots T_{n[k]}.M} \\
	N & {B_{n[1]}} & {T_{n[1]}.M} & \dots & {T_{n[1]}\dots T_{n[k]}.N}
	\arrow["{T_{n[1]}\dots T_{n[k]}.\gamma.\N}", from=3-5, to=4-5]
	\arrow["{\gamma.\N}", from=3-1, to=4-1]
	\arrow["{\anc_{n[1]}}", from=3-2, to=3-3]
	\arrow["{\pi\o\pi_i}"', from=3-2, to=3-1]
	\arrow["{\pi\o\pi_i}", from=4-2, to=4-1]
	\arrow["{\anc_{n[1]}}"', from=4-2, to=4-3]
	\arrow["{\gamma.W_{n[1]}}", from=3-2, to=4-2]
	\arrow[""{name=0, anchor=center, inner sep=0}, "{T_{n[1]}.\gamma.\N}", from=3-3, to=4-3]
	\arrow["{T_{n[1]}.(\pi \o \pi_i)}"', from=3-4, to=3-3]
	\arrow[from=3-4, to=3-5]
	\arrow["{T_{n[1]}.(\pi \o \pi_i)}", from=4-4, to=4-3]
	\arrow[from=4-4, to=4-5]
	\arrow["\dots"{description}, draw=none, from=3-4, to=4-4]
	\arrow["{T^U.\gamma.\N}"', from=1-4, to=2-4]
	\arrow["{\pi^U}"', from=1-3, to=1-2]
	\arrow["{\pi^U}", from=2-3, to=2-2]
	\arrow["{\anc^U}"', from=2-3, to=2-4]
	\arrow["{\anc^U}", from=1-3, to=1-4]
	\arrow["{\gamma.\N}"', from=1-2, to=2-2]
	\arrow[""{name=1, anchor=center, inner sep=0}, "{\gamma.U}"', from=1-3, to=2-3]
	\arrow["{=}"{description}, Rightarrow, draw=none, from=1, to=0]
\end{tikzcd}\]
    Thus, every tangent-natural transformation is constructed out of a pair
    \[
        (\gamma.\N: M \to N, \gamma.W:A \to B)
    \]
    using the $\boxtimes$ construction from Equation \ref{eq:boxtimes-def-of-hat-f}.
    All that remains to show is that this pair is an involution algebroid morphism. 

    Tangent naturality gives the following two coherences:
    \[
        \anc^B \o \gamma.W = T.\gamma.\N \o \anc^A \text{ and } 
        \sigma^B \o \gamma.WW = \gamma.WW \o \sigma^A
    \]
    since $\anc^B = \beta.W, \anc^A = \alpha.W, \sigma^B = B.c$, and $\sigma^A = A.c$ by construction. The following diagram proves that $\gamma.W$ preserves the lifts, so that $(\gamma.W, \gamma.\N)$ is an involution algebroid morphism:
    \[\input{TikzDrawings/Ch4/Sec4/tang-nat-pres-lifts.tikz}\]
    % and anchors:
    % \input{TikzDrawings/Ch4/Sec4/tang-nat-pres-anchors.tikz}
    % And the $\gamma$ commutes with the involution by naturality. 
    Thus, a tangent natural transformation $(\widehat{A},\alpha) \to (\widehat{B}, \beta)$ is exactly a morphism of involution algebroids $A \to B$, proving the theorem.
    
    % The bijection follows from the fact that by tangent naturality, every
    % \[
    %     f.V: \widehat{A}.V \to \widehat{B}.V  
    % \]
    % decomposes as a pullback power of the base maps $f.\N, f.W$, which are exactly involution algebroid morphisms.
\end{proof} 
Now, the projection for a Lie algebroid is a submersion, as we may make a choice of prolongations for each $U \in \wone$. These prolongations lead to a new observation about Lie algebroids: they embed into a category of functors into smooth manifolds.
\begin{corollary}%
    \label{cor:SMan-embedding}
    Using the Weil nerve construction, the category of Lie algebroids embeds into the tangent-functor category:
    \[
        \mathsf{LieAlgd} \hookrightarrow [\wone, \mathsf{SMan}].  
    \]
\end{corollary}





\section{Identifying involution algebroids}%
\label{sec:identifying-involution-algebroids}

This section identifies those tangent functors
\[
    (A,\alpha): \wone \to \C  
\] 
that are involution algebroids as precisely those where $A$ preserves transverse limits and $\alpha$ is a \emph{$T$-cartesian} natural transformation (Definition \ref{def:cart-nat}). 
These conditions will force each $A.V$ to be the $V$-prolongation of the underlying anchored pre-differential bundle:
\[
    (A.p: A.T \to A,\;\ A.0: A \to A.T,\;\ A.T \xrightarrow[]{A.\ell} A.T.T \xrightarrow[]{\alpha.T} T.A.T,\;\ \alpha:A.T \to T.A )  
\]
(these conditions also ensure that this tuple is an anchored differential bundle). 

Initially, it is only clear that $\alpha$ is $T$-cartesian for the projection $p$. Indeed, recall that the prolongation $A.UV$ is defined to be the $T$-pullback of the cospan:
\[ \widehat A .U \xrightarrow[]{\alpha^U} T^U.A.\N \xleftarrow[]{T^U.A.p^V} T^U.\widehat{ A}.V\]
Then consider the following diagram:
\[\input{TikzDrawings/Ch4/anc-cart-p.tikz}\]
This means that every naturality square of $\alpha$ for $p$ is a $T$-pullback; natural transformations satisfying this property for every map in the domain category are called $T$-cartesian.
\begin{definition}%
    \label{def:cart-nat}
    A natural transformation $\gamma: F \Rightarrow G$ is \emph{cartesian} whenever each naturality square
    \[\input{TikzDrawings/Ch4/Sec4/equifibred.tikz}\]
    is a pullback. A natural transformation between functors into a tangent category is \emph{$T$-cartesian} whenever each component square is a $T$-pullback (we will generally suppress the $T$ when the context is clear). 
\end{definition}
% \begin{example}
%     For the Weil nerve of an involution algebroid, $\alpha$ is $T$-cartesian for
% \end{example}

Now, recall that the Weil complex determined by an involution algebroid has $A.U.V$ determined by the following $T$-pullback squares:
\[\input{TikzDrawings/Ch4/Sec5/weil-nerve-pb.tikz}\]
Then it is not difficult to show that the $T$-cartesian condition on a Weil complex forces it to be an involution algebroid.  We first need:
\begin{definition}
    A $T$-cartesian Weil complex in $\C$ is a tangent functor
    \[
        (A,\alpha): \wone \to \C  
    \]
    for which $A$ sends transverse limits to $T$-limits and $\alpha$ is a $T$-cartesian natural transformation.
\end{definition}
The first condition to check is that a $T$-cartesian Weil complex gives a natural anchored bundle $\hat{A}$ whose Weil prolongations coincide with the functor assignments on objects.
\begin{proposition}%
    \label{prop:anc-bun-cw-complex}
    Let $(A,\alpha)$ be a $T$-cartesian Weil complex. Then we have an anchored bundle
    \[
        (M  := A.\N, \hspace{0.15cm}
        \hat{A} := A.W , \hspace{0.15cm}
        \pi := A.\pi, \hspace*{0.15cm}
        \xi := A.\xi, \hspace*{0.15cm}
        \lambda := \alpha.T \o A.\ell).
    \]
    Furthermore, 
    \[
        \prol(\hat{A}) = A.WW, \hspace*{0.30cm}
        \prol^2(\hat{A}) = A.WWW. 
    \]
\end{proposition}
\begin{proof}
    Suppose we have a tangent functor $(F,\alpha): \C \to \D$ and a differential bundle $(\pi,\xi,\lambda)$ in $\C$. If $F$ preserves $T$-pullbacks of $\pi$, it preserves the additive bundle structure on $(\pi,\xi,+)$, so to show $(F.\pi, F.\xi, \alpha \o F.\lambda)$ is universal it suffices to show that the following diagram is a $T$-pullback in $\D$:
    \input{TikzDrawings/Ch4/Sec5/mu-anc-lambda.tikz}
    Expand this to
    \input{TikzDrawings/Ch4/Sec5/mu-anc-lambda-expanded.tikz}
    In this case, it restricts to the diagram
    \input{TikzDrawings/Ch4/Sec5/mu-anc-lambda-restricted-diagram.tikz}
    Each square is a $T$-pullback by hypothesis, so the universality of the lift follows by the $T$-pullback lemma. Because the complex is $T$-cartesian, the assignment $A.V$ gives a coherent choice of prolongations by the $T$-pullback
    \input{TikzDrawings/Ch4/Sec5/mu-anc-equifibered-prol.tikz}
\end{proof}
There is, of course, a natural candidate for the involution map.
\begin{corollary}
    Let $(\pi:A \to M, \xi, \lambda, \anc)$ be the anchored bundle induced by a $T$-cartesian Weil complex in a tangent category $\C$. Then we have an involution map
    \[
        \sigma: \prol(A) \xrightarrow{A.c} \prol(A).
    \]
\end{corollary}
The equations for an involution algebroid should follow immediately by functoriality; one need only ensure that the maps take the correct form.
\begin{lemma}\label{lem:cwm-map-structure}
    Let $A$ be a $T$-cartesian Weil complex in a tangent category $\C$, with $(\pi:A \to M, \xi, \lambda, \sigma)$ its underlying anchored bundle.
    Then we have:
    \begin{enumerate}[(i)]
        \item $A.c.T = \sigma \x c$,
        \item $A.T.c = 1 \x T.\sigma$,
        \item $A.\ell.T = \hat{\lambda} \x \ell.A$,
        \item $A.T.\ell = id \x T.\hat{\lambda}$.
    \end{enumerate}
\end{lemma}
\begin{proof}
    ~\begin{enumerate}[(i)]
        \item Consider the diagram
        \input{TikzDrawings/Ch4/Sec5/rewrite-act.tikz}
        Observe that this forces $A.c.T = A.c \x c.A.T = \sigma \x c$.
        \item Likewise, the diagram
        \input{TikzDrawings/Ch4/Sec5/rewrite-atc.tikz}
        forces $A.T.c = id \x T.A.c = id \x T.\sigma$.
        \item The diagram
        \input{TikzDrawings/Ch4/Sec5/rewrite-alt.tikz}
        forces $A.\ell.T = A.\ell \x \ell.A = \hat{\lambda} \x \ell$.
        \item As $\alpha$ is $T$-cartesian, the following diagram is a $T$-pullback:
        \input{TikzDrawings/Ch4/Sec5/rewrite-atl-1.tikz}
        Using previous results, this means that $A.\ell.T$ is the unique map making the following diagram commute:
        \input{TikzDrawings/Ch4/Sec5/rewrite-atl-2.tikz}
        which we can see is $id \x \hat\lambda$.
    \end{enumerate}
\end{proof}

Pulling together this lemma and the previous proposition, the following is now clear:
\begin{proposition}
    A $T$-cartesian Weil complex determines an involution algebroid.
\end{proposition}
However, we have not yet exhibited an isomorphism of categories between the image of the Weil nerve functor and $T$-cartesian Weil complexes. 
At first glance, the Weil nerve construction only gives a Weil complex that is $T$-cartesian for the tangent projection $p \in \wone$. Being $T$-cartesian for $p$ is, however, sufficient: a Weil complex that is $T$-cartesian for tangent projections will be $T$-cartesian for every map in $\wone$ (a similar result appears in the context of differentiable programming languages; see \cite{Cruttwell2019}).

\begin{proposition}\label{prop:mod-is-cart-if-p}
    A lax tranverse-limit-preserving tangent functor $(F,\alpha):\wone \to \C$ for which $F$ preserves pullback powers of each $T^U.p$ is $T$-cartesian if and only if each
    \begin{equation*}
        \input{TikzDrawings/Ch4/Sec4/equifibred-iff-p.tikz}
    \end{equation*}
    is a $T$-pullback.
\end{proposition}
\begin{proof}
    We only check the converse since the forward implication is trivial. We make use of the $T$-pullback lemma.
    \begin{enumerate}[(i)]
        \item $c$ is an isomorphism, so its naturality square is a $T$-pullback.
        \item For projections $T_2 \to T$,  the retract of a $T$-pullback diagram is a $T$-pullback, so the following diagram is universal:
        \input{TikzDrawings/Ch4/Sec5/pcartiff-proj.tikz}
        \item For $0$, observe that the following two diagrams are equal:
        \input{TikzDrawings/Ch4/Sec5/pcartiff-zero.tikz}
        The right diagram is a $T$-pullback, and the right square of the left diagram is a $T$-pullback by hypothesis.
        By the $T$-pullback lemma, the left square of the left diagram is a $T$-pullback.
        \item For $\ell$, observe that
        \input{TikzDrawings/Ch4/Sec5/pcart-iff-ell.tikz}
        The outer perimeter of the right diagram is a $T$-pullback (left square by hypothesis, right square by (ii)), as is the right square of the left diagram (by hypothesis). 
        By the $T$-pullback lemma, the left square of the left diagram is a $T$-pullback.
        \item For $+$, observe that
        \input{TikzDrawings/Ch4/Sec5/pcart-iff-add.tikz}
        The outer diagram on the right is a $T$-pullback by composition, and the right square on the left diagram is a $T$-pullback by hypothesis, so the result follows. 
    \end{enumerate}
    To check that the naturality square is a $T$-pullback for \emph{every} map in $\wone$, we once again use Leung's characterization of maps in $\wone$ from Proposition \ref{thm:leung}. Inductively, the set of maps generated by $\{p,0,+,\ell,c\}$ closed under $\ox$ and $\o$ follows as $T$-pullback squares are closed to composition. For maps induced by a tranverse limit in $\wone$, $F$ preserves transverse limits so this follows by the commutativity of limits.
\end{proof}

\begin{theorem}\label{thm:iso-of-cats-inv-emcs}
    For any tangent category $\C$, the replete image of the Weil nerve functor
    \[
        \mathsf{Inv}(\C) \hookrightarrow [\wone, \C]  
    \]
    is precisely the category of $T$-cartesian Weil complexes.
\end{theorem}
\begin{corollary}\label{cor:the-prolongation-description}
    That $\alpha:A.T \Rightarrow T.A$ is $T$-cartesian is equivalent to requiring that the tangent functor
    \[
        (A,\alpha): \wone \to \C  
    \]
    restricts to an anchored bundle
    \[(\pi: A.T \xrightarrow[]{A.p} A.\N,\;\ \xi: A \xrightarrow[]{A.0} A.T,\;\ \lambda: A.T \xrightarrow[]{A.\ell} A.TT \xrightarrow[]{\alpha.T}T.A.T,\;\ \anc:A.T \xrightarrow[]{\alpha} T.A)\]
    and each $A.T^V$ is the $V$-prolongation of this anchor bundle.
\end{corollary}
\begin{remark}
    The condition in Corollary  \ref{cor:the-prolongation-description} is analogous to the Segal conditions identifying those simplicial complexes
    \[\Delta \to \C \]
    that are internal categories. Note that every simplicial object has an underlying reflexive graph
    \[
        \mathsf{tr}_1(X) := (s,t:X([1]) \to X([0]), i:X([0]) \to X([1]))
    \]
    where $X([n])$ is isomorphic to the object of $n$-composable arrows for the underlying reflexive graph.
\end{remark}
\begin{remark}
    Notably, being $T$-cartesian for $p$ is enough to force that a natural transformation is $T$-cartesian for the other tangent-structural natural transformations. This has consequences when one uses partial maps to combine \emph{topological} notions with tangent categories.
    In this context, a partial map $N \to X$ with domain $M \hookrightarrow N$  is a span
    \[\input{TikzDrawings/Ch4/Sec4/span-remark.tikz}\]
    whose right leg is monic. The intuition is that the map $f$ is defined on the subobject $M$ of $N$, which introduces a new problem: what is the proper notion of a \emph{subobject} in a tangent category?
    Such a notion should give rise to a \emph{stable class of monics}: one that is closed under horizontal span composition.
    One answer is the notion of etale monics: a morphism is \emph{etale} whenever the naturality square for $p$ is a $T$-pullbacks:
    \[\input{TikzDrawings/Ch4/Sec4/etale-cond.tikz}\]
    Geometrically, this means that the morphism is a local diffeomorphism; for example, an etale subobject of $\R^n$ in the Dubuc topos is precisely an open subset in the usual sense.
    An endofunctor lifts to the partial map category whenever it preserves the class of monics. A natural transformation lifts to endofunctors on the partial map category whenever it is $T$-cartesian for the class of monics, and the same proof will show that this property holds for etale monics \cite{Cruttwell2019}. 
\end{remark}
  


\section{The prolongation tangent structure}
\label{sec:prol_tang_struct}

One of the most important consequences of the Weil Nerve Theorem  \ref{thm:weil-nerve} is that the category of involution algebroids (with chosen prolongations) may be equipped with two tangent structures. The first tangent structure is the pointwise tangent structure described in Proposition \ref{prop:pointwise-tangent-structure-inv}.
%TODO add reference to the tangent structure on involution algebroids in chapter 3
The tangent functor sends
\[
    (A,\alpha) \mapsto (T.A:\wone \to \C, c.A \o T.\alpha: T.A.T \Rightarrow T.T.A)  
\]
(recall the composition of tangent functors given in Example \ref{ex:composition-of-tangent-functors} (ii)). The structure morphisms will be given by whiskering, so in this case $\theta.A, \theta \in \{ p,0,+,\ell,c\}$. The restriction to tangent functors that preserve transverse limits along with the fact that the natural part $\alpha$ is $T$-cartesian, however, ensures that precomposition with the tangent functor
\[
    (A,\alpha) \mapsto (A.T: \wone \to \C, T.\alpha \o A.c: A.T.T \Rightarrow T.A.T)  
\]
returns an involution algebroid. The structure maps are once again given by whiskering, with the pre-composition tangent structure $A.\theta, \theta \in \{ p,0,+,\ell,c\}$. Preservation of transverse limits guarantees that this tangent structure will satisfy the necessary universality conditions. 

\begin{proposition}[Proposition \ref{prop:second-tangent-structure-inv-algds}]
\label{prop:second-tangent-structure-inv-algds-2}
    The category of involution algebroids with chosen prolongations in a tangent category $\C$ has a second tangent structure, where the action by $\wone$ is given by
    \[
        (A,\alpha) \mapsto ( A.T: \wone \to \C, \alpha.T \o \hat A.c: \hat A.T.T \Rightarrow T.\hat A.T).  
    \]
\end{proposition}
\begin{proof}
    The proposition statement means that the structure morphisms for this new involution algebroid are given by
    \[
        ( A.T, \alpha.T \o  A.c) \cong
        \begin{cases}
            \alpha.T \o  A.c = \anc':& \prol(A) \xrightarrow[]{\pi_1} TA \\
             \quad \;\; A.T.p = \pi':& \prol(A) \xrightarrow[]{p \o \pi_1} A \\
             \quad \;\;\; A.T.0 = \xi':& A \xrightarrow[]{(\xi \o \pi, 0)} \prol(A) \\
            \anc' \o  A.T.\ell = \lambda':& \prol(A) \xrightarrow[]{\lambda \x \ell} T.\prol(A) \\
             \quad \;\ A.T.c = \sigma':& \prol^2(A) \xrightarrow{\sigma \x c} \prol^2(A)
        \end{cases}  
    \]
    Similarly, we can see that
    \begin{gather*}
        ( A.T.T, \alpha.T.T \o  A.c.T \o  A.T.c) \\ = 
        \begin{cases}
            \alpha.T.T \o  A.c.T \o  A.T.c = \anc'':& \prol^2(A) \xrightarrow[]{(\pi_1, \pi_2)} T.\prol(A) \\
             \hspace{2.5cm} A.T.p = \pi'':& \prol^2(A) \xrightarrow[]{(p \o \pi_1, p \o \pi_2):} \prol(A) \\
             \hspace{2.6cm} A.T.0 = \xi'':& \prol(A) \xrightarrow[]{(\xi \o \pi \o \pi_0, 0 \o \pi_1, 0 \o \pi_2)} \prol^2(A) \\
            \hspace{1.85cm} \anc''\o  A.T.\ell = \lambda'':&\prol^2(A) \xrightarrow[]{(\lambda \x \ell \x \ell)} T.\prol^2(A) \\
             \hspace{2.15cm} A.T.T.c = \sigma'':& \prol^3(A) \xrightarrow[]{(\sigma \x c \x c)}\prol^3(A) 
        \end{cases}  
    \end{gather*}
    These coincide with the involution algebroids $\prol'(A), \prol'.\prol'(A)$ in Proposition \ref{prop:second-tangent-structure-inv-algds}: the second tangent structure follows from the fact that the natural transformations for the tangent structure there are given by
    \[
        A.\phi: A.U \Rightarrow A.V, \phi:U \to V \in \{p,0,+, \ell, c\}.  
    \]
    The result follows as a corollary of Theorem \ref{thm:weil-nerve}.
\end{proof}



% This section exposits the second tangent structure on the category of involution algebroids (with chosen prolongations) in some tangent category $\C$. This tangent structure is a natural consequence of the Weil Nerve from \Cref*{thm:weil-nerve}, but is worth giving a more concrete presentation. The Jacobi identity on the sections of an involution algebroid follows as a consequence of this tangent structure.

% The category of cartesian tangent functors $[\wone, \C]$ into any tangent category $\C$ has two actions by $\wone$. The first is post-composition of the tangent functor in $\C$:
% \[
%     \mathsf{CartTang}[\wone, \C] \x \wone \to \mathsf{CartTang}[\wone, \C] ;   \hspace{0.15cm} ((A,\alpha), V) \mapsto (T^V.A, T^V.\alpha)  
% \]
% sends transverse limits in $\wone$ to limits in the functor category $\mathsf{CartTang}(\wone, \C)$ - this tangent structure was described in \Cref{obs:inv-tmonad-anc}. %TODO add the reference
% However, the restriction to transverse-limit-preserving cartesian tangent functors also guarantees that the \emph{pre-composition} action:
% \[
%     \mathsf{CartTang}[\wone, \C] \x \wone \to \mathsf{CartTang}[\wone, \C] ;   \hspace{0.15cm} ((A,\alpha), V) \mapsto (A.T^V, \alpha.T^V)  
% \]
% sends all the transverse limits in $\wone$ to tangent limits in the tangent category $\C$. This will induce a second tangent structure on the category, where each of the structure maps are tangent natural transformations with respect to the pointwise tangent structure.


% One of the original sticking points in the development of abstract tangent structure is quite surprising:  the proof of the Jacobi identity for the Lie bracket of vector fields (see section 3.4 of \cite{Cockett2014} for the history). The eventual proof made use of a string calculus, and a collection of identities \cite{Cockett2015}.  It should come as no surprise, then, that showing the bracket on the sections of an involution algebroid satisfies the Jacobi identity is complex. This section describes the second tangent structure on the category of    involution algebroids, corresponding to the prolongation functor.
% When a tangent category has negatives, there is a natural bracket associated with the abelian group of vector fields on a given $M$, discussed in Lemma  \ref{lem:intertwining-induces-bracket}. Proving that the Jacobi identity holds for this bracket - a reasonably trivial result in the category of smooth manifolds - proves to be a significant challenge (see \cite{Cockett2014} for the history). The eventual proof of this result in \cite{Cockett2015} developed a string calculus to facilitate calculations.

% It should not be surprising, then, that proving the Jacobi identity holds for an involution algebroid also proves to be a challenge. Using the new characterization of involution algebroids as a functor category $[\wone, \C]$, Cockett and Cruttwell's result may be applied directly to an involution algebroid.

% \begin{definition}\label{def:two-tang-on-inv}
%     Let $\C$ be a tangent category. The category of complete involution algebroids in $\C$ has two tangent structures:
%     \begin{enumerate}[(i)]
%         \item Post-composition: This is given by $T_{post}(A)(V) = T.A(V)$ - this is, the cartesian Weil complex is post-composed by $T$. We write this simply as $T$. This tangent structure was expounded on in %TODO look up tangent structure for involution algebroids in chapter three
%         \item Pre-composition: Pre-composition with the tangent functor preserves Cartesian models of $\wone$. This tangent functor will be written as $\prol$. 
%     \end{enumerate}
% \end{definition}
%IDEA: make this more concrete by building the second tangent structure on the category of involution algebroids, where the functor is a tangent functor each of the structure maps will be a tangent natural transformation. This gives a tangent category in the category of tangent categories, e.g. the action by $\wone$ is itself a tangent functor.
% The second tangent structure on involution algebroids can be confusing (proving that the prolongation operation is a tangent bundle on the category of involution algebroids is equivalent to the Weil Nerve Theorem  \ref{thm:weil-nerve}). It is instructive to have a concrete description of $\prol.\hat A, \prol.\prol.\hat A$. 
% \begin{example}%
%     \label{ex:prol-inv-algds}
%     Let $A = (\pi:A \to M, \xi, \lambda, \anc, \sigma)$ be an involution algebroid with chosen prolongations in a tangent category $\C$ (note that this induces an addition map $+_q$). It is important to realize here that the anchored bundle comes with a choice of prolongations - this gives a coherent choice for the prolongations of the anchored bundles (see -).%TODO add reference to the exact tangent structure.
    
%     % The anchored bundle $\prol \hat{A}$ is defined to be:
%     % \begin{gather}
%     %     p \o \pi_1: \prol(A) \to A, (\xi\o \pi, 0):A \to \prol(A), \lambda \x \ell: \prol(A) \to T.\prol(A), \pi_1: \prol(A) \to TA \\
%     %     % p \o (\pi_1,\pi_2): \prol^2(A) \to \prol(A), (\xi\o\pi \o \pi_1, 0 \o \pi_1, 0 \o \pi_2): \prol(A) \to \prol^2(A), \lambda \x \ell \x \ell: \prol^2(A) \to T.\prol^2(A), (\pi_1,\pi_2): \prol^2(A) \to T.\prol(A)
%     % \end{gather}%TODO format
%     Recall that by -, pullback of anchored bundles are computed as pullbacks of the underlying differential bundle, and the anchor is induced universality. %TODO cross reference limits of anchored bundles
%     The structure maps are then given by the following diagram:
%     \[\input{TikzDrawings/Ch4/dbun-pullback-abun.tikz}\]
%     It can be show that the prolongation of $\prol \hat{A}$ is $\prol^2(A)$ using the following diagram:
%     \[\input{TikzDrawings/Ch4/prol-2-pp.tikz}\]
%     The diagram commutes as:
%     \begin{gather*}
%         \anc \o g_1 = T.\pi \o g_2 = T.\pi \o T.p \o f_1 = T.p \o T^2.\pi \o f_1 \\
%         = T.p \o T.\anc \o f_2 = T.\pi \o f_2
%     \end{gather*}
%     This gives the induced involution:
%     \[\input{TikzDrawings/Ch4/induced-involution.tikz}\]
%     as state in -. %TODO fill in cross reference.
%     This gives the pullback involution algebroid:
%     \begin{gather*}
%         p \o \pi_1: \prol(A) \to A, (\xi\o \pi, 0):A \to \prol(A), \lambda \x \ell: \prol(A) \to T.\prol(A), \\
%          \pi_1: \prol(A) \to TA, \sigma \x c: \prol^2(A) \to \prol^2(A)
%         % p \o (\pi_1,\pi_2): \prol^2(A) \to \prol(A), (\xi\o\pi \o \pi_1, 0 \o \pi_1, 0 \o \pi_2): \prol(A) \to \prol^2(A), \lambda \x \ell \x \ell: \prol^2(A) \to T.\prol^2(A), (\pi_1,\pi_2): \prol^2(A) \to T.\prol(A)
%     \end{gather*}%TODO format
%     The involution algebroid axioms all follow by universality, but we check the axioms on this involution map (Definition \ref{def:involution-algd})
%     \begin{itemize}
%         \item The two differential bundle structures on $\prol^2(A)$ are then $\lambda \x \ell \x \ell.T, 0 \x c \o T.\lambda \x T.\ell$, so bilinearity follows.
%         \item The involution axiom is straightforward to check:
%         \[( \sigma \x c )\o (\sigma \x c) = (\sigma \o \sigma) \x (c \o c)  = id\]
%         \item The target axiom follows:
%         \[
%             \pi_1 \o (\pi_1,\pi_2) \o (\sigma \x c) = c \o \pi_2
%         \]
%         \item The unit axiom follows by:  
%         \[
%             (\sigma \x c) \o (\xi\o \pi \o x, \lambda x, \ell \o y) = 
%             (\sigma \o (\xi \o \pi \o x, \lambda x), c \o \ell \o y) =   (\xi\o \pi \o x, \lambda x, \ell \o y)
%         \]          
%         \item In this case, the Yang-Baxter equation becomes:
%         \begin{gather*}
%             (\sigma \x c \x c) \o (A \x T.\sigma \x T.c) \o (\sigma \x c \x c) \\
%             = 
%             (A \x T.\sigma \x T.c) \o (\sigma \x c \x c)\o (A \x T.\sigma \x T.c)
%         \end{gather*}
%         And it follows by parallel application of the Yang-Baxter equation:
%         \begin{itemize}
%             \item $ (\sigma \x c ) \o (A \x T.\sigma) \o (\sigma \x c) 
%             = 
%             (A \x T.\sigma ) \o (\sigma \x c) \o (A \x T.\sigma)$
%             \item $c.T \o T.c \o c.T = T.c \o c.T \o T.c$
%         \end{itemize}            
%     \end{itemize}
%     Applying an identical argument to this anchored bundle show that the involution algebroid $\prol^2 \hat{A}$ is given by the tuple:
%     \begin{gather*}
%         \pi \x p \x p: \prol^2(A) \to M \ts{id}{\pi}\prolong, \hspace{0.15cm}
%         \xi \x 0 \x 0: M \ts{id}{\pi} \prolong \to \prol^2(A), \\
%         \lambda \x \ell: \prolong \to T.(\prolong), \hspace{0.15cm}
%         (\pi_1,\pi_2): \prol^2(A) \to T.\prol(A) \\
%         \sigma \x c \x c: \prol^3(A) \to \prol^3(A)
%     \end{gather*}


%     % Similarly, the anchored bundle $\prol.\prol \hat{A}$ is induced by the pullback of anchored bundles:

%     % Giving it the structure maps:
%     % %TODO put in the itemized list

%     % The chosen prolongations for each $\prol(A), \prol^2(A)$ is given by:
%     % \begin{gather*}
%     %     \prol(U,\prol.\hat A) = \prol(UW,A) \\
%     %     \prol(U,\prol.\prol. \hat A) = \prol(UWW,A)
%     % \end{gather*}
%     % Where $\prol(U,-)$ refers the total object of the differential bundle.
%     % The induced involution maps are given by:
%     % %TODO insert span diagrams

%     % Then Yang-Baxter equation for the prolongation anchored bundles follows by parallel applications of the basic Yang-Baxter equation for $\hat{A}$. 
%     %TODO insert diagrams
%     % \begin{enumerate}
%     %     \item     The ``prolongation'' Lie algebroid, $\prol(A)$, is given by $f \boxtimes W$, for $f \in \{ \pi,\xi,+_q,\hat \lambda, \sigma\}$. It has been discussed in different pieces throughout \Cref*{ch:involution-algebroids} - the anchored bundle structure in -, the tangent involution algebroid structure in -, and then pullback involution algebroid structure in -. %TODO insert references
%     %     Here it is all pulled together and clarified based on insights provided by the Weil nerve construction for convenience. 
%     %     \begin{itemize}
%     %         \item The projection map is given by:
%     %         \[ \infer{p \o \pi_1: \prol(A) \to A}{\pi \x p: \prolong \to M \ts{id}{\pi} A}\]
%     %         Note that this means the $n$-fold pullback powers of the projection are given by $A_n \ts{\anc_n}{T_n.\pi} T_nA$.
%     %         \item The zero map is given by:
%     %         \[
%     %             \infer{(\xi \o \pi, 0): A \to \prol(A)}{\xi \x 0: M \ts{id}{\pi} \to \prolong}
%     %         \]
%     %         \item The addition map is given:
%     %         \[
%     %            ( +_q \x +.A ): A_2 \ts{\anc_2}{T_2.\pi} T_2A \to \prolong
%     %         \]
%     %         \item The (differential bundle) lift map is given:
%     %         \[
%     %             (\lambda \x \ell):\prolong \to T.(\prolong)
%     %         \]
%     %         \item The anchor map is given by:
%     %         \[
%     %             \infer{\pi_1: \prol(A) \to TA}{\anc \x id: \prolong \to TA}  
%     %         \]
%     %         \item The involution map is given by:
%     %         \[
%     %             \sigma \x c: \prolong \ts{T.\anc}{T^2.\pi} T^2A \to \prolong \ts{T.\anc}{T^2.\pi} T^2A 
%     %         \]
%     %     \end{itemize}
%     %     The involution proves to be the most difficult part of this proof. This involution algebroid is induced by the pullbac of the cospan of involution algeboids:
%     %     \[
%     %         A \xrightarrow[]{\anc} TM \xleftarrow[]{T.\pi} TA
%     %     \]
%     %     where $A$ is the involution algebroid, and $TM, TA$ are the tangent involution algebroids on $M, A$. This induces an involution map:
%     %     \[
%     %         (\prolong) \ts{\pi_1}{T.p \o \pi_1} (\prolong)
%     %         \to (\prolong) \ts{\pi_1}{T.p \o \pi_1} (\prolong)
%     %     \]
%     %     Upon identifying 
%     %     \[
%     %         (\prolong) \ts{\pi_1}{T.p \o \pi_1} (\prolong) \cong  \prolong \ts{T.\anc}{T^2.\pi} T^2A
%     %     \]
%     %     Then one can instead check that the involution $\sigma \x c$ satisfies the universal property of the induced map 
%     %     \[
%     %         \prol^2(A) \to \prol^2(A)  
%     %     \]

%     %     \item The second prolongation's involution algebroid structure follows by the same argument. \dots
%     % \end{enumerate}
% \end{example}
% %TODO this paragraph is messy as hell
% Using the Weil nerve (Theorem  \ref{thm:weil-nerve}), involution algebroids have a natural tangent structure given by pre-composing an involution algebroid $\hat A: \wone \to \C$ with the tangent functor in $\wone$ - that is, sending $\hat A \mapsto \hat A . T$. 
% The Weil nerve statement, then, identifies the strict tangent functors:
% \[
%     \wone \to \mathsf{Inv}^*(\C)  
% \]
% and identifies morphisms with tangent natural transformations. It is worth giving a concrete description of each of the tangent structure morphisms.

% \begin{lemma}
%     The prolongation and the anchor $(\prol, anc)$ is a tangent endofunctor on the category of involution algebroids.
% \end{lemma}
% \begin{proof}
%     The assignment on objects was demonstrated in Example  \ref{ex:prol-inv-algds}. On morphisms, note that an involution algebroid morphism $(f,u):\hat{A} \to \hat{B}$ gives a linear morphism:
%     \[
%         f \x T.f: (A \ts{\anc}{T.\pi} TB, \lambda \x \ell) \to (B \ts{\anc}{T.\pi} TB, \lambda \x \ell)   
%     \]
%     as $f$ is linear for the $\lambda$ part and $T.f$ is linear for the $\ell$ part. The base part of this map is $f$, and the anchor map is preserved as the anchor maps are $\pi_1$, so:
%     \[
%         \pi_1 \o (f \x T.f ) = T.f  
%     \]
%     Similarly, the involution is preserved as:
%     \begin{gather*}
%         (f \x T.f \x T^2.f) \o (\sigma \x c) = ((f \x T.f) \o \sigma, T^2.f \o c)  \\= ((\sigma \o (f \x T.f), c \o T.f)) = (\sigma \x c) \o (f \x T.f \x T^2.f)
%     \end{gather*}
%     Thus the map $(f,v)$ maps to $(f \x T.f, f)$, giving functoriality (composition is straightforward to check).

%     To see that this is a tangent functor with respect to the pointwise tangent structure on involution algebroids, first observe that .%TODO finish this proof
% \end{proof}
% Now given the functor part of the tangent structure, check each of the structure maps:
% \begin{lemma}
%     Regard the category of involution algebroids with chosen prolongations in $\C$ as a tangent category, using the pointwise tangent structure. 
%     Then there are tangent natural transformations:
%     \begin{enumerate}[(i)]
%         \item Projection:
%         \[
%             \hat p : (\prol,\anc) \Rightarrow id  
%         \]
%         where $\hat p = (\pi_1, \pi): \prol \hat{A} \to A$.  Note that the pullback powers of $\hat p$ has the underlying differential bundle:
%         \begin{gather*}
%             A \ts{\anc}{T.\pi \o T.\pi_i} T.A_n \xrightarrow[]{p \o \pi_1} A_n, \hspace{0.15cm}
%             A_N \xrightarrow[]{(\xi \o \pi \o \pi_i, 0)} A \ts{\anc}{T.\pi \o T.\pi_i} T.A_n \\
%             A \ts{\anc}{T.\pi \o T.\pi_i} T.A_n \xrightarrow[]{\lambda \x \ell.A_n}  T.(A \ts{\anc}{T.\pi \o T.\pi_i} T.A_n)
%         \end{gather*}
%         \item Addition:
%         \[
%             \hat +: (\prol_2, \anc_2) \Rightarrow (\prol, \anc); \hspace{0.15cm}
%             A \ts{\anc}{T.\pi \o T.\pi_i} T.A_2 \xrightarrow[]{id \x T.+_q} \prolong
%         \]
%         \item Zero:
%         \[
%             \hat 0: id \Rightarrow (\prol,\anc); \hspace{0.15cm}
%             \mathbf{A} \xrightarrow[]{(id, T.(\xi \o \pi))} \mathbf{A}
%         \]
%         \item Lift: 
%         \[
%             \hat \ell: (\prol, \anc) \Rightarrow (\prol, \anc)^2; \hspace{0.15cm}
%             {\hat \ell: \prol.A \xrightarrow[]{id \x (T.\xi \o T.\pi, T.\lambda)} \prol.\prol(A)}
%         \]
%         \item Flip:
%         \[
%             \hat c: (\prol,\anc)^2 \Rightarrow (\prol, \anc)^2; \hspace{0.15cm}
%             \prol^2(A) \xrightarrow[]{(1 \x T.\sigma)} \prol^2(A)
%         \]
%     \end{enumerate}
% \end{lemma}
% \begin{proof}
%     ~\begin{enumerate}[(i)]
%         \item 
%     \end{enumerate}
% \end{proof}

% \begin{observation}
%     Relating this to the $\boxtimes$ operation (Definition \ref{def:boxtimes-span}), the involution algebroid maps for $\prol(V,A)$ are given by $\theta \boxtimes \prol(V,A)$, where as the structure maps for the tangent category of involution algebroids is given by $\prol(V,A) \boxtimes \theta$.
% \end{observation}

% \begin{theorem}
%     There is a tangent structure on the category of involution algebroids with chosen prolongations in $\C$, given by
%     \[
%         (\prol, \hat p, \hat +, \hat 0, \hat \ell, \hat c)  
%     \]
% \end{theorem}





% \begin{theorem}%
%     \label{thm:second-tangent-structure}
%     For any tangent category $\C$, the prolongation endofunctor determines a tangent structure on the category of involution algebroids with chosen prolongations in $\C$. The tangent natural transformations are given by:
%     \begin{itemize}
%         \item The projection $p$ is given by:
%         \[
%             \hat p: 
%             \prol.\mathbf{A} \xrightarrow[]{\pi_0} \mathbf{A}   
%         \]
%         Note that the pullback powers of $\hat p$ has the underlying differential bundle:
%         \begin{gather*}
%             A \ts{\anc}{T.\pi \o T.\pi_i} T.A_n \xrightarrow[]{p \o \pi_1} A_n, \hspace{0.15cm}
%             A_N \xrightarrow[]{(\xi \o \pi \o \pi_i, 0)} A \ts{\anc}{T.\pi \o T.\pi_i} T.A_n \\
%             A \ts{\anc}{T.\pi \o T.\pi_i} T.A_n \xrightarrow[]{\lambda \x \ell.A_n}  T.(A \ts{\anc}{T.\pi \o T.\pi_i} T.A_n)
%         \end{gather*}
%         \item The zero map is givem by:
%         \[
%             \hat 0: 
%             \mathbf{A} \xrightarrow[]{(id, T.(\xi \o \pi))} \mathbf{A}              
%         \]
%         \item The addition map is given by:
%         \[
%             \hat +:A \ts{\anc}{T.\pi \o T.\pi_i} T.A_2 \xrightarrow[]{id \x T.+_q} \prolong
%         \]
%         \item The lift map is given by:
%         \[
%             \infer{A \ts{\anc}{id} TM \ts{id}{T.\pi} TA \xrightarrow[]{id \x T.\xi \x T.\lambda} \prolong \ts{T.\anc}{T^2.\pi} T^2A}
%             {\hat \ell: \prol.A \xrightarrow[]{id \x (T.\xi \o T.\pi, T.\lambda)} \prol.\prol(A)}
%         \]
%         \item The involution map is given by:
%         \[
%             \hat c: \prolong \ts{T.\anc}{T^2.\pi} T^2A
%             \xrightarrow[]{A \x T.\sigma}
%             \prolong \ts{T.\anc}{T^2.\pi} T^2A
%         \]
%     \end{itemize}
% \end{theorem}
% \begin{proof}
    
% \end{proof}
% The ``tangent involution'' algebroid of 
% \[\bar{A} = (\pi:A \to M, \xi, \lambda, \anc, \sigma)\] 
% is the prolongation involution algeboid 
% \begin{gather*}
%     (\pi\x p): \prol(A) \to M \ts{id}{\pi} A, \hspace{0.15cm}
%     (\xi \x 0): TM \ts{id}{\pi} A \to \prol(A), \hspace{0.15cm}
%     (\lambda \x \ell):\prol(A) \to T.\prol(A), \\
%     \pi_1: \prol(A) \to T.A,\hspace{0.15cm}
%     \sigma \x c: \prol^2(A) \to \prol^2(A)
% \end{gather*}
% One of the main sticking point in providing a direct proof that 
% \[\prol: \mathsf{Inv}(\C) \to \mathsf{Inv}(\C)\] 
% is a tangent structure is showing coherences like:
% \[
%     \prol(\prol(\bar{A})) = \prol^2(A)  
% \] and keeping the isomorphism maps straight as one checks the involution algebroid axioms. The coherence theorem Theorem  \ref{thm:weil-nerve} allows us to sidestep this difficulty and describe the involution $\prol(A)$ directly as:
% \begin{gather*}
%     \sigma \x c: \prol^2(A) \to \prol^2(A)
% \end{gather*}
% In this case, the Yang-Baxter equation becomes:
% \[
%     (\sigma \x c \x c) \o (A \x T.\sigma \x T.c) \o (\sigma \x c \x c) 
%     = 
%     (A \x T.\sigma \x T.c) \o (\sigma \x c \x c)\o (A \x T.\sigma \x T.c)
% \]
% And it follows by parallel application of the Yang-Baxter equation:
% \begin{itemize}
%     \item $ (\sigma \x c ) \o (A \x T.\sigma) \o (\sigma \x c) 
%     = 
%     (A \x T.\sigma ) \o (\sigma \x c) \o (A \x T.\sigma)$
%     \item $c.T \o T.c \o c.T = T.c \o c.T \o T.c$
% \end{itemize}
% The ``tangent differential bundle'' structure maps are then the second differential bundle structure on $\prol(A)$ from Proposition  \ref{prop:anc-prol-fun}
% \[
%     \pi_0: \prol(A) \to A    
% \]
% The zero map is given by:
% \[
%     id \x T.\xi \o \anc : A \ts{\anc}{id} TM \to \prol(A)  
% \]
% The addition map by:
% \[
%     (+_q \x +.A): A_2 \ts{\anc_2}{T_2.(\pi)} T_2A \to \prolong
% \]
% The second-prolongation involution algebroid may is:
% \begin{gather*}
%     (\pi \x p): \prol^2(A) \to M \ts{id}{\pi} \prol(A),\hspace{0.15cm}
%     (\xi \x 0): M \ts{id}{\pi} \prol(A) \to \prol^2(A),\hspace{0.15cm}
%     (\lambda \x \ell): \prol(A) \to T.\prol(A), \\
%     (\pi_1,\pi_2): \prol^2(A) \to T.\prol(A) \hspace{0.15cm}
%     (\sigma \x c \x c): \prol^3(A) \to \prol^3(A)
% \end{gather*}

% The lift map, then, is given by:
% \[
%     (id \x T.\xi \x T.\lambda): A \ts{\anc}{id} TM \ts{id}{T.\pi} TA \to \prolong \ts{T.\anc}{T^2.\pi} T^2A  
% \]
% and the involution is given by:
% \[
%     (id \x T.\sigma):
%     \prolong \ts{T.\anc}{T^2.\pi} T^2A \to \prolong \ts{T.\anc}{T^2.\pi} T^2A 
% \]
% Note that in this case, the Yang-Baxter equation reduces to:
% \begin{gather*}
%     (A \x T.\sigma \x T.c \x T.c) \o (A \x TA \x T^2.\sigma \x T^2.c) \o (A \x T.\sigma \x T.c \x T.c)
%     \\ = 
%     (A \x TA \x T^2.\sigma \x T^2.c) \o (A \x T.\sigma \x T.c \x T.c) \o (A \x TA \x T^2.\sigma \x T^2.c)
% \end{gather*}
% which, once again, boils down parallel applications of the Yang-Baxter equation:
% \begin{itemize}
%     \item $id \o id \o id = id = id \o id \o id$,
%     \item $T.((\sigma \x c) \o (1 \x T.\sigma) \o (\sigma \x c))  =  T.((1 \x T.\sigma) \o (\sigma \x c)\o (1 \x T.\sigma))$
%     \item $T.(c.T \o T.c \o c.T) = T.(T.c \o c.T \o T.c)$
% \end{itemize}
% Relating this back to the tangent structure on $\wone^\anc$ induced by an involution algebroid - the structure maps for the anchored bundle are given by tensoring by the identity on the \emph{right}, while the structure maps for the tangent structure on the category of involution algebroids is given by applying the tensor product to the \emph{left}.



\subsection*{The Jacobi identity for involution algebroids}
Classically, the theory of Lie algebroids uses the algebra of sections $\Gamma(\pi)$.  One key observation is that when using the Lie tangent structure $(\mathsf{Inv}(\C), \prol)$, sections of $\pi$ are in bijective correspondence with $\chi_\prol(A)$.
This observation allows for different statements about Lie algebroids to be translated into formal statements about the tangent bundle in $(\mathsf{Inv}(\C), \prol)$.
\begin{proposition}\label{prop:section-morphism}
    Let $A$ be an involution algebroid in $\C$.
    There is a bijection between the sections of\, $\pi$ in $\C$ and the vector fields on $A$ in $\mathsf{Inv}(\C)$:
    \[
        X \in \Gamma(\pi) \mapsto ((id, TX \o \anc), X): A \to T_L(A);
        \hspace{.2cm}
        \hat{X} \in \chi_\prol(A) \mapsto \hat{X}_R: A.R \to A.T.
    \]
\end{proposition}
\begin{proof}
    Recall the coherence for tangent natural transformations $\gamma: (H,\phi) \Rightarrow (G,\psi)$:
    \begin{equation*}
        \input{TikzDrawings/Ch4/Sec5/tangnat.tikz}
    \end{equation*}
    We specify this to a morphism $\hat{X}: (A, \alpha) \Rightarrow T_L(A,\alpha) = (A.T, \alpha.T)$ at $\N, W$:
    \begin{equation*}
        \input{TikzDrawings/Ch4/Sec5/hat-thing.tikz}
    \end{equation*}
    and so  infer that, if we set $X := X_R$, we have $\pi_1 \o X.T = T(X) \o \anc$.
    Furthermore, the condition that $p_L \o X = id$ forces $id = \pi_0 \o X.T$; thus, we can see that every section $X$ of $p_L$ is given by a morphism of the form $((id, TX \o \anc), X))$ on the underlying involution algebroids, where $\pi \o X = id$. 
    
    We now show that every $X \in \Gamma(\pi)$ gives rise to a section of $\pi_L$.
    Observe that the following is a morphism of involution algebroids:
    \begin{equation*}
        \input{TikzDrawings/Ch4/Sec5/inv-algd-mor.tikz}
    \end{equation*}
    Note that it is well typed, as $T.\pi \o T.X \o \anc = \anc \o id$. Check that it is a bundle morphism:
    \[
        p \o \pi_1 \o (id, T.X \o \anc) = p \o T.X \o \anc = X \o p \o \anc = X \o \pi
    \]
    and that it is linear:
    \begin{gather*}
        (\lambda \o \pi_0, \ell\o\pi_1)\o(id, T.X\o \anc) = (\lambda, \ell \o T.X \o \anc) \\= (\lambda, T^2.X \o \ell \o \anc) = (\lambda, T^2.X \o \lambda) = T(id, TX)\o\lambda. 
    \end{gather*}
    Then check that it preserves the anchor:
    \[
        \pi_1 \o (id, TX \o \anc) = TX \o \anc
    \]
    and the involution:
    \begin{gather*}
        (\pi_0, \pi_1, T^2X \o T\anc \pi_1)\o \sigma = (\sigma, T^2X \o T\anc \o \pi_1 \sigma)\\ 
        = (\sigma, T^2X \o c T\anc\o \pi_1)
        = (\sigma(\pi_0, \pi_1), c\pi_3) \o (id, T^2X \o T\anc \o \pi_1).
    \end{gather*}

    Thus we have that $(id, TX \o \anc \pi_1)$ is a morphism of involution algebroids, inducing a morphism of $T$-cartesian Weil complexes.
    Lastly, we check that it is a section of $p^L_A$, but this is clear, since
    \[
        \pi_0 \o (id, TX \o \anc) = id;
    \]
    thus we have the desired bijection.
\end{proof}

Recall that given an involution on an anchored bundle, there is a bracket on its set of sections (see the explicit construction in Section \ref{sec:connections_on_an_involution_algebroid}). 
Given an $X,Y \in \Gamma(\pi)$, there is a bracket defined as follows:
\[
    \hat{\lambda} \o [X,Y]_A +_1 (\xi\pi,0)\o Y = ((\sigma \o (id, TY \o \anc) \o X -_2 (id, TX \o \anc)\o Y)).
\]
A direct proof of the Jacobi identity is a detailed calculation (see the original preprint on involution algebroids \cite{Burke2019}) and still relies on Cockett and Cruttwell's result for an arbitrary tangent category with negatives. As a result of Proposition  \ref{prop:section-morphism}, we can instead use Cockett and Cruttwell's result directly:
\begin{corollary}\label{cor:lie-bracket}
    Let $A$ be a complete involution algebroid in a tangent category $\C$ with negatives. 
    There is a Lie bracket defined on $\Gamma(\pi)$, $[-,-]$ induced by
    \[
        \hat{\lambda} \o [X,Y]_A +_1 (\xi\pi,0)\o Y = ((\sigma \o (id, TY \o \anc) \o X -_2 (id, TX \o \anc)\o Y)).
    \]
\end{corollary}
\begin{proof}
    The bracket is induced by Rosicky's universality diagram, as
    \begin{align*}
        0&=p \o  ((\sigma \o (id, TY \o \anc) \o X - (id, TX \o \anc)\o Y)) - 0Y \\
        &= Tp \o  ((\sigma \o (id, TY \o \anc) \o X - (id, TX \o \anc)\o Y)) - 0Y. 
    \end{align*}
    We look at the Lie tangent structure for $\mathsf{Inv}^*(A)$; this is precisely the vector field induced by
    \[
        ev_R( [\hat{X}, \hat{Y}] ).
    \]
    We complete the proof by using the result that for any $A$ in a tangent category with negatives, the bracket on $\chi(A)$ that is defined by
    \[
        \ell \o [X,Y] = (c\o T.X \o Y - T.Y \o X) - 0X
    \]
    satisfies the Jacobi identity.
\end{proof}

\subsection*{Identifying categories of involution algebroids}%
\label{sub:identifying-cats-of-inv}

Section \ref{sec:identifying-involution-algebroids} identified whenever a functor $\wone \to \C$ is an involution algebroid, whereas this section identifies tangent categories $\C$ that embed into the category of involution algebroids in some tangent category $\C$. We call this structure an \emph{abstract category} of involution algebroids. This notion involves some 2-category theory, using a modified notion of \emph{codescent} (see \cite{Bourke2010} for a development of codescent).

Recall that for any tangent category $\C$, the category of involution algebroids has $\C$ as a reflective subcategory. Furthermore, because limits of involution algebroids are computed pointwise, this reflector is left-exact. This left-exact reflection is the main structure we axiomatize.
\begin{definition}
    An \emph{abstract category of involution algebroids} is a tangent category $\C$ with a left-exact $T$-cartesian tangent localization $(Z,\anc): \C \to \D$, where $L$ satisfies a \emph{codescent} condition:
    \[
        \mathsf{TangCat_{Strict}}(\wone, \C)  \hookrightarrow \mathsf{TangCat_{Lax}}(\wone, \C) \xrightarrow{L_*} \mathsf{TangCat_{Lax}}(\wone, \D)
    \]
    (where $L_*$ denotes post-composition by $L$) is fully faithful. 
\end{definition}
\begin{example}
    The category of involution algebroids in any tangent category $\C$ is an abstract category of involution algebroids using $(\mathsf{Inv}(C), \prol)$. The reflector is the functor sending an involution algebroid to its base space; the $T$-cartesian natural transformation is the anchor map. Any tangent subcategory of $\mathsf{Inv}(\C)$ that contains $\C$ as a full subcategory will give rise to an abstract category of involution algebroids.
\end{example}

\begin{proposition}
    Let $\Z \hookrightarrow \C$ be an abstract category of involution algebroids.
    Then there is an embedding $\C \hookrightarrow \mathsf{Inv}(\Z)$.
\end{proposition}
\begin{proof}
    The proof follows by treating objects in $\C$ as strict tangent functors $\wone \to \C$ and morphisms as tangent natural transformations.
    \[\input{TikzDrawings/Ch4/Sec5/coherence-thing.tikz}\]
    The natural part of $Z$ is $T$-cartesian, and the functor part preserves limits, so $Z.\prol(-,A) =: Z[A]$ determines an involution algebroid in $\Z$, and $f$ a morphism of involution algebroids.
    The embedding is guaranteed by the codescent condition so that the post-composition functor is fully faithful.
\end{proof}
\begin{corollary}
    An abstract category of involution algebroids $\Z \hookrightarrow \C$ is exactly a full subcategory $\Z \hookrightarrow \C \hookrightarrow \mathsf{Inv}(\Z)$.
\end{corollary}






% The back matter includes commands to produce endnotes, index,
% glossary, bibiliography, and the like.

\onecolumn


% \tableofcontents{}

% \newpage

\section*{Supplementary Material}
\addcontentsline{toc}{section}{Supplementary Material}


Throughout this discussion, 
we will make frequently use 
of the following standard results
concerning the exponential concentration 
of random variables:

\begin{lemma}[Hoeffding's inequality for independent RVs~\citep{hoeffding1994probability}] Let $Z_1, Z_2, \ldots, Z_n$ be independent bounded random variables with $Z_i \in [a,b]$ for all $i$, then 
    \begin{align*}
        \prob\left( \frac{1}{n} \sum_{i=1}^n (Z_i - \Expo{Z_i}) \ge t \right) \le \exp{\left( -\frac{2nt^2}{(b-a)^2} \right) }
    \end{align*} 
    and 
    \begin{align*}
        \prob\left( \frac{1}{n} \sum_{i=1}^n (Z_i - \Expo{Z_i}) \le -t \right) \le \exp{\left( -\frac{2nt^2}{(b-a)^2} \right) }
    \end{align*} 
    for all $t \ge 0$. 
\end{lemma}

\begin{lemma}[Hoeffding's inequality for sampling with replacement~\citep{hoeffding1994probability}] \label{lem:hoeffding_sampling} Let $\calZ = (Z_1, Z_2, \ldots, Z_N)$ be a finite population of $N$ points with $Z_i \in [a.b]$ for all $i$. Let $X_1, X_2, \ldots X_n$ be a random sample drawn without replacement from $\calZ$. Then for all $t \ge 0$, we have 
    \begin{align*}
        \prob\left( \frac{1}{n} \sum_{i=1}^n (X_i - \mu ) \ge t \right) \le \exp{\left( -\frac{2nt^2}{(b-a)^2} \right) }
    \end{align*} 
    and 
    \begin{align*}
        \prob\left( \frac{1}{n} \sum_{i=1}^n (X_i - \mu ) \le -t \right) \le \exp{\left( -\frac{2nt^2}{(b-a)^2} \right) } \,,
    \end{align*} 
    where $\mu = \frac{1}{N} \sum_{i=1}^{N} Z_i$. 
\end{lemma}

We now discuss one condition that generalizes the exponential concentration to dependent random variables.
\begin{condition}[Bounded difference inequality] \label{cond:BDC} Let $\calZ$ be some set and $\phi: \calZ^n \to \Real$. We say that $\phi$ satisfies the bounded difference assumption if 
there exists $c_1, c_2, \ldots c_n \ge 0$ s.t. for all $i$, we have 
\begin{align*}
    \sup_{Z_1,Z_2, \ldots,Z_n, Z_i^\prime \in \calZ^{n+1} } \abs{\phi (Z_1, \ldots, Z_i, \ldots, Z_n ) - \phi (Z_1, \ldots, Z_i^\prime, \ldots, Z_n ) } \le c_i \,.
\end{align*} 
\end{condition}

\begin{lemma}[McDiarmid’s inequality~\citep{mcdiarmid1989}] \label{lem:McDiarmid} Let $Z_1, Z_2, \ldots, Z_n$ be independent random variables on set $\calZ$ and $\phi : \calZ^n \to \Real$ satisfy bounded difference inequality (\codref{cond:BDC}). Then for all $t>0$, we have 
    \begin{align*}
        \prob\left( \phi(Z_1, Z_2, \ldots, Z_n) - \Expo{\phi(Z_1, Z_2, \ldots, Z_n)} \ge t \right) \le \exp{\left( -\frac{2t^2}{\sum_{i=1}^n c_i^2} \right) } 
    \end{align*} 
    and 
    \begin{align*}
        \prob\left( \phi(Z_1, Z_2, \ldots, Z_n) - \Expo{\phi(Z_1, Z_2, \ldots, Z_n)} \le -t \right) \le \exp{\left( -\frac{2t^2}{\sum_{i=1}^n c_i^2} \right) } \,.
    \end{align*} 
\end{lemma}


\section{Proofs from \secref{sec:ERM_training}}\label{app:proof_erm}

\textbf{Additional notation {} {}} Let $m_1$ be the number of mislabeled points ($\wt S_M$) and $m_2$ be the number of correctly labeled points ($\wt S_C$). Note $m_1 + m_2 = m$. 


\subsection{Proof of \thmref{thm:error_ERM}}


\begin{proof}[Proof of \lemref{lem:fit_mislabeled}] 
    The main idea of our proof is to regard 
    the clean portion of the data 
    ($S \cup \wt S_C$) as fixed.   
    Then, there exists an (unknown) classifier $f^*$ 
    that minimizes the expected risk
    calculated on the (fixed) clean data
    and (random draws of) the mislabeled data $\wt S_M$. 
    % 
    % 
    Formally, 
    \begin{align}
    f^* \defeq \argmin_{f \in \calF} \error_{\widecheck {\calD}} (f) \,, \label{eq:modified_ERM}
    \end{align}
    where $$\widecheck \calD = \frac{n}{m+n} \calS + \frac{m_2}{m+n} \wt \calS_C  + \frac{m_1}{m+n}\calDm \,.$$ 
    Note here that $\widecheck \calD$ is a combination 
    of the \emph{empirical distribution} 
    over correctly labeled data $S \cup \wt S_C$
    and the (population) distribution 
    over mislabeled data $\calDm$.
    Recall that 
    \begin{align}
    \wh f \defeq \argmin_{f \in \calF} \error_{\calS \cup \wt S} (f) \,. \label{eq:orig_ERM}
    \end{align}
    % 
    % 
    Since, $\widehat f$ minimizes 0-1 error 
    on $S \cup \wt S$, using ERM optimality on \eqref{eq:orig_ERM},  
    we have 
    \begin{align}
        \error_{\calS \cup \wt \calS}(\widehat f) \le \error_{
            \calS \cup \wt \calS}(f^*) \,.    \label{eq:step1}
    \end{align}
    Moreover, since $f^*$ is independent of $\wt S_M$, using Hoeffding's bound,
    % \footnote{For a fully rigorous argument,
    % refer to the complete proof in App.~\ref{app:proof_erm}.} 
    we have with probability at least $1-\delta$ that
    \begin{align}
      \error_{\wt \calS_M}(f^*) \le \error_{ \calDm}(f^*) +  \sqrt{\frac{\log(1/\delta)}{2 m_1}} \,. \label{eq:step2} 
    \end{align}
    %$ 
    %for some constant $c_1\le 1/2$. 
    Finally, since $f^*$ is the optimal classifier on $\widecheck \calD$, 
    we have 
    \begin{align}
        \error_{\widecheck \calD}(f^*) \le \error_{\widecheck \calD}(\widehat f) \,. \label{eq:step3}
    \end{align}
    Now to relate \eqref{eq:step1} and \eqref{eq:step3}, we multiply \eqref{eq:step2} by $\frac{m_1}{m+n}$ and add $\frac{n}{m+n} \error_{\calS} (f)  + \frac{m_2}{m+n} \error_{\wt \calS_C} (f)$ both the sides. Hence, 
    we can rewrite \eqref{eq:step2} as follows: 
    \begin{align}
        \error_{\calS \cup \wt\calS}(f^*) \le \error_{ \widecheck \calD}(f^*) +  \frac{m_1}{m+n}\sqrt{\frac{\log(1/\delta)}{2 m_1}} \,. \label{eq:step4} 
    \end{align}
    Now we combine equations \eqref{eq:step1}, \eqref{eq:step4}, and \eqref{eq:step3}, to get 
    \begin{align}
        \error_{\calS \cup \wt \calS}(\wh f) \le \error_{\widecheck \calD}(\wh f) +  \frac{m_1}{m+n}\sqrt{\frac{\log(1/\delta)}{2 m_1}} \,, 
    \end{align}
    which implies 
    \begin{align}
        \error_{ \wt \calS_M}(\wh f) \le \error_{\calDm}(\wh f) + \sqrt{\frac{\log(1/\delta)}{2 m_1}} \,. \label{eq:lemma1_final}
    \end{align}
    Since $\wt S$ is obtained by randomly labeling an unlabeled dataset, we assume $2m_1 \approx m$ \footnote{Formally, with probability at least $1-\delta$, we have  $(m - 2m_1)\le \sqrt{m\log(1/\delta)/2}$.}. Moreover, using $\error_{\calDm} = 1 - \error_{\calD}$ we obtain the desired result.   
    % Combining the above steps and using the fact 
    % that $\error_\calD = 1- \error_{\calDm} $, 
    % we obtain the desired result.
\end{proof}

\begin{proof}[Proof of \lemref{lem:mislabeled_error}]
    Recall $\error_{\wt S} (f) = \frac{m_1}{m} \error_{\wt S_M}(f) + \frac{m_2}{m} \error_{\wt S_C}(f)$. Hence, we have 
    \begin{align}
        2\error_{\wt S}(f) - \error_{\wt S_M}(f) - \error_{\wt S_C}(f) &= \left(\frac{2m_1}{m} \error_{\wt S_M}(f) - \error_{\wt S_M}(f)\right) + \left(\frac{2m_2}{m} \error_{\wt S_C}(f) - \error_{\wt S_C}(f)\right) \\ &= \left(\frac{2m_1}{m} - 1\right) \error_{\wt S_M}(f) + \left(\frac{2m_2}{m} - 1 \right)\error_{\wt S_C} (f) \,.
    \end{align} 
    Since the dataset is labeled uniformly at random, with probability at least $1-\delta$, we have  $\left(\frac{2m_1}{m} - 1\right) \le \sqrt{\frac{\log(1/\delta)}{2m}}$. Similarly, we have with probability at least $1-\delta$, $\left(\frac{2m_2}{m} - 1\right) \le \sqrt{\frac{\log(1/\delta)}{2m}}$. Using union bound, with probability at least $1-\delta$, we have
    % \begin{align}
    %     2\error_{\wt S} - \error_{\wt S_M}(f) - \error_{\wt S_C}(f) \le \sqrt{\frac{\log(2/\delta)}{2m}} \left(\error_{\wt S_M}(f) + \error_{\wt S_C}(f) \right) \le 2\sqrt{\frac{\log(2/\delta)}{2m}} \,. \label{eq:lemma2_final}
    % \end{align}
    \begin{align}
        2\error_{\wt S} - \error_{\wt S_M}(f) - \error_{\wt S_C}(f) \le \sqrt{\frac{\log(2/\delta)}{2m}} \left(\error_{\wt S_M}(f) + \error_{\wt S_C}(f) \right) \,. \label{eq:lemma2_prefinal}
    \end{align}
    With re-arranging $\error_{\wt S_M}(f) + \error_{\wt S_C}(f)$ and using the inequality $ 1- a\le \frac{1}{1+a} $, we have  
    \begin{align}
        2\error_{\wt S} - \error_{\wt S_M}(f) - \error_{\wt S_C}(f) \le 2\error_{\wt \calS} \sqrt{\frac{\log(2/\delta)}{2m}}  \,. \label{eq:lemma2_final}
    \end{align}

    % We obtain the desired result by using 
\end{proof}

\begin{proof}[Proof of \lemref{lem:clear_error}]
% Recall 0-1 error on each point  $(x,y) \in S \cup \wt S$ is given by $\I{ f(x)\ne y}$.
In the set of correctly labeled points $S \cup \wt S_C$, we have $S$ as a random subset of $S \cup \wt S_C$. Hence, using Hoeffding's inequality for sampling without replacement (\lemref{lem:hoeffding_sampling}), we have with probability at least $1-\delta$
\begin{align}
    \error_{\wt \calS_C} (\wh f)- \error_{\calS \cup \wt \calS_C}( \wh f) \le  \sqrt{\frac{\log(1/\delta)}{2m_2}} \,.
\end{align}
Re-writing $\error_{\calS \cup \wt \calS_C}( \wh f)$ as $\frac{m_2}{m_2 + n} \error_{\wt \calS_C }(\wh f) + \frac{n}{m_2 + n} \error_{\calS }(\wh f)$, we have with probability at least $1-\delta$
\begin{align}
   \left(\frac{n}{n+m_2}\right) \left(\error_{\wt \calS_C} (\wh f)- \error_{\calS}( \wh f) \right) \le  \sqrt{\frac{\log(1/\delta)}{2m_2}} \,.
\end{align}
As before, assuming $2m_2 \approx m$, we have with probability at least $1-\delta$ 
\begin{align}
    \error_{\wt \calS_C} (\wh f)- \error_{\calS}( \wh f) \le \left(1+\frac{m_2}{n}\right)  \sqrt{\frac{\log(1/\delta)}{m}} \le \left(1 + \frac{m}{2n}\right) \sqrt{\frac{\log(1/\delta)}{m}} \,. \label{eq:lemma3_final}
\end{align} 
\end{proof}

\begin{proof}[Proof of \thmref{thm:error_ERM}] 
    Having established these core intermediate results, we can now combine above three lemmas to prove the main result. 
    In particular, we bound the population error on clean data ($\error_\calD(\wh f)$) as follows:  
    \begin{enumerate}[(i)]
        \item First, use \eqref{eq:lemma1_final}, to obtain an upper bound on the population error on clean data, i.e., with probability at least $1-\delta/4$, we have
        \begin{align}
            \error_{ \calD} (\wh f) \le 1 - \error_{ \wt \calS_M}(\wh f) + \sqrt{\frac{\log(4/\delta)}{m}} \,. 
        \end{align}
        \item  Second, use \eqref{eq:lemma2_final}, to relate the error on the mislabeled fraction with error on clean portion of randomly labeled data and error on whole randomly labeled dataset, i.e., with probability at least $1-\delta/2$, we have 
        \begin{align}
            - \error_{\wt S_M}(f) \le \error_{\wt S_C}(f) - 2\error_{\wt S}  + 2\error_{\wt S} \sqrt{\frac{\log(4/\delta)}{2m}}  \,. 
        \end{align} 
        \item Finally, use \eqref{eq:lemma3_final} to relate the error on the clean portion of randomly labeled data and error on clean training data, i.e., with probability $1-\delta/4$, we have 
        \begin{align}
            \error_{\wt \calS_C} (\wh f)\le - \error_{\calS}( \wh f) + \left(1 + \frac{m}{2n} \right) \sqrt{\frac{\log(4/\delta)}{m}} \,. 
        \end{align} 
    \end{enumerate}

    Using union bound on the above three steps, we have with probability at least $1-\delta$: 
    \begin{align}
        \error_\calD (\wh f) \le \error_{\calS}(\wh f)   + 1 - 2\error_{\wt \calS}(\wh f)   + \left(\sqrt{2} \error_{\wt S} + 2 + \frac{m}{2n}\right)  \sqrt{\frac{\log(4/\delta)}{m}} \,.
    \end{align}
    % Note that $(1/\sqrt{2} + 2.5)$ is a loose constant. In experiments, we use the ratio $\frac{m}{n}$
    %  the exact error $\error_{\wt \calS}(\wh f)$ 
    % to evaluate R.H.S.    
\end{proof}

\subsection{Proof of \propref{prop:rademacher}}

\begin{proof}[Proof of \propref{prop:rademacher}]
    For a classifier $ f: \calX \to \{-1, 1\}$, we have $1 - 2\,\indict{ f(x) \ne y} = y \cdot f(x)$. Hence, by definition of $\error$, we have 
    \begin{align}
        1 -2\error_{\wt \calS}(f) = \frac{1}{m}\sum_{i=1}^m y_i \cdot f(x_i) \le \sup_{f \in \calF} \, \frac{1}{m} \sum_{i=1}^m y_i \cdot f(x_i)  \,. \label{eq:error_rademacher}
    \end{align}
    Note that for fixed inputs $(x_1, x_2, \ldots, x_m)$ in $\wt S$, $(y_1, y_2, \ldots y_m)$ are random labels. Define $\phi_1 (y_1, y_2, \ldots, y_m) \defeq \sup_{f \in \calF} \, \frac{1}{m} \sum_{i=1}^m y_i \cdot f(x_i)$. We have the following bounded difference condition on $\phi_1$. For all i, 
    \begin{align}
        \sup_{y_1, \ldots y_m, y_i^\prime \in \{-1, 1\}^{m+1} } \abs{ \phi_1 (y_1,\ldots, y_i, \ldots, y_m) - \phi_1 (y_1,\ldots, y_i^\prime, \ldots, y_m)  } \le 1/m \,. \label{cond1_rademacher}
    \end{align} 
    
    Similarly, we define $\phi_2 (x_1, x_2, \ldots, x_m) \defeq \Expt{ y_i \sim_U \{-1, 1\}  }{ \sup_{f \in \calF} \, \frac{1}{m}  \sum_{i=1}^m y_i \cdot f(x_i)}$. We have the following bounded difference condition on $\phi_2$. 
    For all i,
    \begin{align}
        \sup_{x_1, \ldots x_m, x_i^\prime \in \calX^{m+1} } \abs{ \phi_2 (x_1,\ldots, x_i, \ldots, x_m) - \phi_1 (x_1,\ldots, x_i^\prime, \ldots, x_m)  } \le 1/m \,. \label{cond2_rademacher}
    \end{align}
    Using McDiarmid’s inequality (\lemref{lem:McDiarmid}) twice 
    with Condition \eqref{cond1_rademacher} and \eqref{cond2_rademacher}, 
    with probability at least $1-\delta$, we have
    \begin{align}
        \sup_{f \in \calF} \, \frac{1}{m} \sum_{i=1}^m y_i \cdot f(x_i)  - \Expt{x,y}{\sup_{f \in \calF} \, \frac{1}{m} \sum_{i=1}^m y_i \cdot f(x_i) } \le \sqrt{\frac{2\log(2/\delta)}{m}} \,. \label{eq:final_rademacher}
    \end{align} 
    Combining \eqref{eq:error_rademacher} and \eqref{eq:final_rademacher}, we obtain the desired result. 
\end{proof}


\subsection{Proof of \thmref{thm:error_regularized_ERM}}

Proof of \thmref{thm:error_regularized_ERM} follows similar to the proof of \thmref{thm:error_ERM}. Note that the same results in \lemref{lem:fit_mislabeled}, \lemref{lem:mislabeled_error}, and \lemref{lem:clear_error} hold in the regularized ERM case. However, the arguments in the proof of \lemref{lem:fit_mislabeled} change slightly. Hence, we state the lemma for regularized ERM and prove it here for completeness. 

\begin{lemma} \label{lem:lemma1_reg}
    Assume the same setup as \thmref{thm:error_regularized_ERM}. 
    Then for any $\delta >0$, with probability at least  $1-\delta$ 
    over the random draws of mislabeled data $\wt S_M$, we have 
    \begin{align}
        \error_\calD(\widehat f)  \le 1 -\error_{\wt \calS_M}(\widehat f) + \sqrt{\frac{\log(1/\delta)}{m}}\,. 
    \end{align} 
\end{lemma}
\begin{proof}
    The main idea of the proof remains the same, i.e. regard 
    the clean portion of the data 
    ($S \cup \wt S_C$) as fixed.   
    Then, there exists a classifier $f^*$ 
    that is optimal over draws 
    of the mislabeled data $\wt S_M$. 

    
    Formally, 
    \begin{align}
    f^* \defeq \argmin_{f \in \calF} \error_{\widecheck {\calD}} (f)  + \lambda R(f) \,, \label{eq:modified_ERM_reg}
    \end{align}
    where $$\widecheck \calD = \frac{n}{m+n} \calS + \frac{m_1}{m+n} \wt \calS_C  + \frac{m_2}{m+n}\calDm \,.$$ That is, $\widecheck \calD$ a combination of 
    the \emph{empirical distribution} 
    over correctly labeled data $S \cup \wt S_C$
    % in $S\cup \wt S$ 
    and the (population) distribution 
    over mislabeled data $\calDm$.
    Recall that 
    \begin{align}
    \wh f \defeq \argmin_{f \in \calF} \error_{\calS \cup \wt S} (f) + \lambda R(f) \,. \label{eq:orig_ERM_reg}
    \end{align}
    % 
    % 
    Since, $\widehat f$ minimizes 0-1 error 
    on $S \cup \wt S$, using ERM optimality on \eqref{eq:orig_ERM},  
    we have 
    \begin{align}
        \error_{\calS \cup \wt \calS}(\widehat f) + \lambda R(\wh f) \le \error_{
            \calS \cup \wt \calS}(f^*) + \lambda R(f^*) \,.    \label{eq:step1_reg}
    \end{align}
    Moreover, since $f^*$ is independent of $\wt S_M$, using Hoeffding's bound,
    % \footnote{For a fully rigorous argument,
    % refer to the complete proof in App.~\ref{app:proof_erm}.} 
    we have with probability at least $1-\delta$ that
    \begin{align}
      \error_{\wt \calS_M}(f^*) \le \error_{ \calDm}(f^*) +  \sqrt{\frac{\log(1/\delta)}{2 m_1}} \,. \label{eq:step2_reg} 
    \end{align}
    %$ 
    %for some constant $c_1\le 1/2$. 
    Finally, since $f^*$ is the optimal classifier on $\widecheck \calD$, 
    we have 
    \begin{align}
        \error_{\widecheck \calD}(f^*) + \lambda R(f^*) \le \error_{\widecheck \calD}(\widehat f) + \lambda R(\wh f) \,. \label{eq:step3_reg}
    \end{align}
     Now to relate \eqref{eq:step1_reg} and \eqref{eq:step3_reg}, we can re-write the \eqref{eq:step2_reg} as follows: 
    \begin{align}
        \error_{\calS \cup \wt\calS}(f^*) \le \error_{ \widecheck \calD}(f^*) +  \frac{m_1}{m+n}\sqrt{\frac{\log(1/\delta)}{2 m_1}} \,. \label{eq:step4_reg} 
    \end{align}
    After adding $\lambda R(f^*)$ on both sides in \eqref{eq:step4_reg}, we combine equations \eqref{eq:step1_reg}, \eqref{eq:step4_reg}, and \eqref{eq:step3_reg}, to get 
    \begin{align}
        \error_{\calS \cup \wt \calS}(\wh f) \le \error_{\widecheck \calD}(\wh f) +  \frac{m_1}{m+n}\sqrt{\frac{\log(1/\delta)}{2 m_1}} \,, 
    \end{align}
    which implies 
    \begin{align}
        \error_{ \wt \calS_M}(\wh f) \le \error_{\calDm}(\wh f) + \sqrt{\frac{\log(1/\delta)}{2 m_1}} \,. \label{eq:lemma_reg_final}
    \end{align}
    Similar as before, since $\wt S$ is obtained by randomly labeling an unlabeled dataset, we assume 
    $2m_1 \approx m$. Moreover, using $\error_{\calDm} = 1 - \error_{\calD}$ we obtain the desired result. 
\end{proof}
% \begin{proof}[Proof of ]
    
% \end{proof}

\subsection{Proof of \thmref{thm:multiclass_ERM}}

To prove our results in the multiclass case,
we first state and prove lemmas
parallel to those
% We first state and prove lemmas 
% parallel 
% to the three lemmas 
used in the proof of balanced binary case. 
We then combine these results 
% in the three lemmas 
to obtain the result in \thmref{thm:multiclass_ERM}. 

Before stating the result, 
we define mislabeled distribution $\calDm$ for any $\calD$.
While $\calDm$ and $\calD$ share 
the same marginal distribution over inputs $\calX$,
the conditional distribution over labels $y$ 
given an input $x\sim \calD_\calX$ is changed as follows:
For any $x$, the Probability Mass Function (PMF) over $y$ is defined as:  
$p_{\calDm} (\cdot \vert x) \defeq \frac{1 - p_{\calD}(\cdot \vert x)}{k - 1}$, where $ p_{\calD}(\cdot \vert x)$ is the PMF over $y$ for the distribution $\calD$. 

\begin{lemma} \label{lem:fit_mislabeled_multi}
    Assume the same setup as \thmref{thm:multiclass_ERM}. 
    Then for any $\delta >0$, with probability at least  $1-\delta$ 
    over the random draws of mislabeled data $\wt S_M$, we have 
    \begin{align}
        \error_\calD(\widehat f)  \le (k-1)\left(1 -\error_{\wt \calS_M}(\widehat f)\right) + (k-1)\sqrt{\frac{\log(1/\delta)}{m}}\,. \label{eq:lemma1_multi}
    \end{align}   
\end{lemma} 

\begin{proof}
   
    The main idea of the proof remains the same.
    We begin by regarding the clean portion of the data 
    ($S \cup \wt S_C$) as fixed. 
    Then, there exists a classifier $f^*$ 
    that is optimal over draws 
    of the mislabeled data $\wt S_M$. 
    
    However, in the multiclass case,
    we cannot as easily relate the population error on mislabeled data 
    to the population accuracy on clean data.   
    While for binary classification, 
    % we could upper bound $\error_{\wt \calS_M}$ 
    % with $1-\error_\calD$ 
    we could lower bound the population accuracy $1-\error_\calD$
    with the empirical error on mislabeled data $\error_{\wt \calS_M}$ 
    (in the proof of \lemref{lem:fit_mislabeled}), 
    for multiclass classification, 
    error on the mislabeled data 
    and accuracy on the clean data 
    in the population 
    are not so directly related.  
    To establish \eqref{eq:lemma1_multi},
    we break the error on the 
    (unknown) mislabeled data 
    into two parts: one term corresponds 
    to predicting the true label on mislabeled data, 
    and the other corresponds to predicting 
    neither the true label 
    nor the assigned (mis-)label.  
    Finally, we relate these errors to their
    population counterparts to establish \eqref{eq:lemma1_multi}. 
    
    Formally, 
    \begin{align}
    f^* \defeq \argmin_{f \in \calF} \error_{\widecheck {\calD}} (f)  + \lambda R(f) \,, \label{eq:modified_ERM_reg2}
    \end{align}
    where $$\widecheck \calD = \frac{n}{m+n} \calS + \frac{m_1}{m+n} \wt \calS_C  + \frac{m_2}{m+n}\calDm \,.$$ 
    That is, $\widecheck \calD$ is a combination 
    of the \emph{empirical distribution} 
    over correctly labeled data $S \cup \wt S_C$
    % in $S\cup \wt S$ 
    and the (population) distribution 
    over mislabeled data $\calDm$.
    Recall that 
    \begin{align}
    \wh f \defeq \argmin_{f \in \calF} \error_{\calS \cup \wt S} (f) + \lambda R(f) \,. \label{eq:orig_ERM_reg2}
    \end{align}
    % 
    % 
    Following the exact steps from the proof of \lemref{lem:lemma1_reg}, 
    with probability at least $1-\delta$, we have  
    \begin{align}
        \error_{ \wt \calS_M}(\wh f) \le \error_{\calDm}(\wh f) + \sqrt{\frac{\log(1/\delta)}{2 m_1}} \,. \label{eq:lemma1_final_multi_prev}
    \end{align}
    Similar to before, since $\wt S$ is obtained 
    by randomly labeling an unlabeled dataset, 
    we assume 
    $\frac{k}{k-1} m_1 \approx m$. 
    
    Now we will relate $\error_{\calDm} (\wh f)$ with $\error_{\calD}(\wh f)$. 
    Let $y^T$ denote the (unknown) true label 
    for a mislabeled point $(x, y)$ 
    (i.e., label before replacing it with a mislabel). 
    \begin{align*}    
         \Expt{(x, y) \in \sim \calDm}{\indict{ \wh f(x) \ne y }}  &= \underbrace{\Expt{(x, y) \in \sim \calDm}{\indict{ \wh f(x) \ne y \land \wh f(x) \ne y^T}}}_{\RN{1}} \\ &\qquad \qquad + \underbrace{\Expt{(x, y) \in \sim \calDm}{\indict{ \wh f(x) \ne y \land \wh f(x) = y^T}}}_{\RN{2}} \,. \numberthis \label{eq:excess_term}
    \end{align*}
    Clearly, term 2 is one minus the accuracy 
    on the clean unseen data, i.e.,
    \begin{align}
        \RN{2} = 1 - \Expt{{x,y} \sim \calD}{ \indict{ \wh f(x) \ne y}} = 1- \error_{\calD}(\wh f) \,. \label{eq:term1}    
    \end{align}
    Next, we relate term 1 with the error on the unseen clean data. 
    We show that term 1 is equal to the error on the unseen clean data 
    scaled by $\frac{k-2}{k-1}$,
    where $k$ is the number of labels.
    Using the definition of mislabeled distribution $\calDm$,  
    we have 
    \begin{align}
        \RN{1} = \frac{1}{k-1} \left( \Expt{(x, y) \in \sim \calD}{ \sum_{i \in \calY \land i\ne y}  \indict{ \wh f(x) \ne i \land \wh f(x) \ne y}} \right) = \frac{k-2}{k-1} \error_{\calD}(\wh f) \,.\label{eq:term2}
    \end{align}    

    Combining the result in \eqref{eq:term1}, \eqref{eq:term2} and \eqref{eq:excess_term}, we have 
    \begin{align}
        \error_{\calDm}(\wh f) = 1- \frac{1}{k-1} \error_{\calD}(\wh f) \,.\label{eq:combine_terms}
    \end{align}
    Finally, combining the result in \eqref{eq:combine_terms} 
    with equation \eqref{eq:lemma1_final_multi_prev}, 
    we have with probability $1-\delta$, 
    \begin{align}
      \error_{\calD}(\wh f) \le  (k-1) \left( 1- \error_{ \wt \calS_M}(\wh f) \right)  + (k-1) \sqrt{\frac{k \log(1/\delta)}{ 2(k-1)m}} \,. \label{eq:lemma1_final_multi}
    \end{align}
\end{proof}

\begin{lemma} \label{lem:mislabeled_error_multi}
    Assume the same setup as \thmref{thm:multiclass_ERM}. 
    Then for any $\delta >0$, 
    with probability at least $1-\delta$ 
    over the random draws of $\wt S$, we have  
    % \begin{align}
        $$\abs{k\error_{\wt \calS}(\widehat f) - \error_{\wt \calS_C}(\widehat f) -  (k-1)\error_{\wt \calS_M}(\widehat f) } \le  2k\sqrt{\frac{\log(4/\delta)}{2m}}\,. $$ % \label{eq:lemma2}
    % \end{align}   
    %  for some constant $c_3 \le 1.0\,$.
\end{lemma} 


\begin{proof}
    Recall $\error_{\wt S} (f) = \frac{m_1}{m} \error_{\wt S_M}(f) + \frac{m_2}{m} \error_{\wt S_C}(f)$. Hence, we have 
    \begin{align*}
        k\error_{\wt S}(f) - (k-1)\error_{\wt S_M}(f) - \error_{\wt S_C}(f) &= (k-1)\left(\frac{k m_1}{(k-1) m} \error_{\wt S_M}(f) - \error_{\wt S_M}(f)\right) \\ & \qquad \qquad + \left(\frac{km_2}{m} \error_{\wt S_C}(f) - \error_{\wt S_C}(f)\right) \\ &= k \left[ \left(\frac{m_1}{m} - \frac{k-1}{k}\right) \error_{\wt S_M}(f) + \left(\frac{m_2}{m} - \frac{1}{k} \right) \error_{\wt S_C} (f) \right] \,.
    \end{align*} 
    Since the dataset is randomly labeled, 
    we have with probability at least $1-\delta$, 
    $\left(\frac{m_1}{m} - \frac{k-1}{k}\right) \le \sqrt{\frac{\log(1/\delta)}{2m}}$. 
    Similarly, we have with probability at least $1-\delta$, 
    $\left(\frac{m_2}{m} - \frac{1}{k}\right) \le \sqrt{\frac{\log(1/\delta)}{2m}}$. 
    Using union bound, we have with probability at least $1-\delta$
    % \begin{align}
    %     2\error_{\wt S} - \error_{\wt S_M}(f) - \error_{\wt S_C}(f) \le \sqrt{\frac{\log(2/\delta)}{2m}} \left(\error_{\wt S_M}(f) + \error_{\wt S_C}(f) \right) \le 2\sqrt{\frac{\log(2/\delta)}{2m}} \,. \label{eq:lemma2_final}
    % \end{align}
    \begin{align}
        k\error_{\wt S}(f) - (k-1)\error_{\wt S_M}(f) - \error_{\wt S_C}(f)  \le k \sqrt{\frac{\log(2/\delta)}{2m}} \left(\error_{\wt S_M}(f) + \error_{\wt S_C}(f) \right) \,. \label{eq:lemma2_final_multi}
    \end{align}

    % We obtain the desired result by using 
\end{proof}

\begin{lemma} \label{lem:clear_error_multi}
    Assume the same setup as \thmref{thm:multiclass_ERM}. 
    Then for any $\delta >0$, with probability at least $1-\delta$ 
    over the random draws of $\wt S_C$ and $S$, we have 
    % \begin{align}
        $$\abs{\error_{\wt \calS_C}(\widehat f) - \error_{\calS}(\widehat f) } \le 1.5 \sqrt{\frac{k\log(2/\delta)}{2m}}\,.$$ %\label{eq:lemma3}
    % \end{align}   
    % for some constant $c_2 \le 1.2\,$.
\end{lemma} 
\begin{proof}
    % Recall 0-1 error on each point  $(x,y) \in S \cup \wt S$ is given by $\I{ f(x)\ne y}$.
    In the set of correctly labeled points $S \cup \wt S_C$,
    we have $S$ as a random subset of $S \cup \wt S_C$. 
    Hence, using Hoeffding's inequality 
    for sampling without replacement 
    (\lemref{lem:hoeffding_sampling}), 
    we have with probability at least $1-\delta$
    \begin{align}
        \error_{\wt \calS_c} (\wh f)- \error_{\calS \cup \wt \calS_C}( \wh f) \le  \sqrt{\frac{\log(1/\delta)}{2m_2}} \,.
    \end{align}
    Re-writing $\error_{\calS \cup \wt \calS_C}( \wh f)$ 
    as $\frac{m_2}{m_2 + n} \error_{\wt \calS_C }(\wh f) + \frac{n}{m_2 + n} \error_{\calS }(\wh f)$, 
    we have with probability at least $1-\delta$
    \begin{align}
       \left(\frac{n}{n+m_2}\right) \left(\error_{\wt \calS_c} (\wh f)- \error_{\calS}( \wh f) \right) \le  \sqrt{\frac{\log(1/\delta)}{2m_2}} \,.
    \end{align}
    As before, assuming $km_2 \approx m$, 
    we have with probability at least $1-\delta$ 
    \begin{align}
        \error_{\wt \calS_c} (\wh f)- \error_{\calS}( \wh f) \le \left(1+\frac{m_2}{n}\right)  \sqrt{\frac{k\log(1/\delta)}{2m}} \le \left( 1 + \frac{1}{k}\right) \sqrt{\frac{k\log(1/\delta)}{2m}} \,. \label{eq:lemma3_final_multi}
    \end{align} 
\end{proof}

\begin{proof}[Proof of \thmref{thm:multiclass_ERM}] 
    Having established these core intermediate results, 
    we can now combine above three lemmas. 
    In particular, we bound the population error 
    on clean data ($\error_\calD(\wh f)$) as follows:  
    \begin{enumerate}[(i)]
        \item First, use \eqref{eq:lemma1_final_multi}, 
        to obtain an upper bound on the population error on clean data, 
        i.e., with probability at least $1-\delta/4$, we have
        \begin{align}
            \error_{ \calD} (\wh f) \le (k-1)\left(1 - \error_{ \wt \calS_M}(\wh f) \right) + (k-1) \sqrt{\frac{k\log(4/\delta)}{2(k-1)m}} \,. 
        \end{align}
        \item  Second, use \eqref{eq:lemma2_final_multi}
        to relate the error on the mislabeled fraction 
        with error on clean portion of randomly labeled data 
        and error on whole randomly labeled dataset, 
        i.e., with probability at least $1-\delta/2$, we have 
        \begin{align}
            - (k-1)\error_{\wt S_M}(f) \le \error_{\wt S_C}(f) - k\error_{\wt S}  + k\sqrt{\frac{\log(4/\delta)}{2m}}  \,. 
        \end{align} 
        \item Finally, use \eqref{eq:lemma3_final_multi} 
        to relate the error on the clean portion of randomly labeled data 
        and error on clean training data, 
        i.e., with probability $1-\delta/4$, we have 
        \begin{align}
            \error_{\wt \calS_C} (\wh f)\le - \error_{\calS}( \wh f) + \left(1 + \frac{m}{kn} \right) \sqrt{\frac{k\log(4/\delta)}{2m}} \,. 
        \end{align} 
    \end{enumerate}

    Using union bound on the above three steps, 
    we have with probability at least $1-\delta$: 
    \begin{align}
        \error_\calD (\wh f) \le \error_{\calS}(\wh f) + (k-1) - k\error_{\wt \calS}(\wh f)   + (\sqrt{k(k-1)} + k + \sqrt{k} + \frac{m}{n\sqrt{k}})  \sqrt{\frac{\log(4/\delta)}{2m}} \,.\label{eq:multiclass_ERM_final}
    \end{align}
    Simplifying the term in RHS of \eqref{eq:multiclass_ERM_final}, 
    we get the desired result. 
    % Note that since $\frac{m}{n\sqrt{k}}$ 
    % is much smaller than the sum of the other terms
    % the other terms in summation, 
    % we ignore $\frac{m}{n\sqrt{k}}$  
    % Z: ??? --- great
    % that 
    % them
    in the final bound. 
    % we ignore that in the final bound. 
    % Note that $(1/\sqrt{2} + 2.5)$ is a loose constant. In experiments, we use the ratio $\frac{m}{n}$
    %  the exact error $\error_{\wt \calS}(\wh f)$ 
    % to evaluate R.H.S.    
\end{proof}

\newpage
\section{Proofs from \secref{sec:linear_models}}\label{app:proof_gd}
We suppose that the parameters of the linear function 
are obtained via gradient descent on 
the following $L_2$ regularized problem: 
\begin{align}
    % n in denominator is avoided deliberately
    \calL_S(w; \lambda) \defeq \sum_{i=1}^n{(w^Tx_i - y_i)^2} + \lambda \norm{w}{2}^2 \,, \label{eq:l2_MSE_app}   
\end{align}
where $\lambda\ge0$ is a regularization parameter. 
We assume access to a clean dataset 
$S = \{(x_i, y_i)\}_{i=1}^n \sim \calD^n$ 
and randomly labeled dataset 
$\wt S = \{(x_i, y_i)\}_{i=n+1}^{n+m} \sim \wt \calD^m$. 
Let $\bX = [x_1, x_2, \cdots, x_{m+n}]$ 
and $\by = [y_1, y_2, \cdots, y_{m+n}]$. 
Fix a positive learning rate $\eta$ such that 
$\eta \le 1/\left(\norm{\bX^T\bX}{\text{op}} + \lambda^2\right)$ 
and an initialization $w_0 = 0$. 
% \todos{Assumption made for simplicty}. 
Consider the following gradient descent iterates 
to minimize objective \eqref{eq:l2_MSE_app} on $S \cup \wt S$:
\begin{align}
w_t = w_{t-1} - \eta \grad_w \calL_{S \cup \wt S} (w_{t-1}; \lambda) \quad \forall t=1,2,\ldots \label{eq:GD_iterates_app}
\end{align} 
Then we have $\{ w_t\}$ converge to the limiting solution 
$\wh w = \left( \bX^T\bX+\lambda \boldsymbol{I}\right)^{-1}\bX^T\by$. Define $\widehat f (x) \defeq f(x ; \wh w) $.  

% \subsection{\textcolor{red}{Errata}}

% We wish to correct the following error in the body:
% \codref{cond:error_stability} is not enough 
% to guarantee the result in \thmref{thm:linear}. 
% We now present a slightly stronger condition 
% called \emph{hypothesis stability} 
% under which we obtain a result 
% similar to \thmref{thm:linear}. 

% This error doesn't change the main arguments of the proof,
% where we show that the empirical train error 
% is less than or equal to the leave-one-out error.
% We need a stronger condition to relate leave-one-out error 
% with the population error of the original classifier. 
% Specifically, while \codref{cond:error_stability} 
% relates the average population error of leave-one-out classifiers 
% with the population error of the original classifier, 
% we need the new condition to show the concentration 
% of the empirical leave-one-out error 
% and average population error of leave-one-out classifiers. 
% main takeaway 

% Note that the new condition, 
% while being stronger than the previous one, 
% still doesn't imply generalization \citep{bousquet2002stability,elisseeff2003leave,abou2019exponential}. 
% Overall, the main results in \secref{sec:ERM_training} 
% and takeaways of the paper remain unaffected by the error.  

% We now present the new condition 
% and a corrected statement of \thmref{thm:linear}. 
% Recall, for a given training set $S \sim \calD^n $, 
% we use $S_{(i)}$ to denote the training set $S$ 
% with the $i^{\text{th}}$ point removed.

% \begin{condition}[Hypothesis Stability] 
%     \label{cond:hypothesis_stability}
%     We have $\beta$ hypothesis stability 
%     if our training algorithm $\calA$ satisfies the following: 
%     \begin{align*}
%     % ${\sum_{i=1}^n \frac{\error_{\calD}( f(\calA, S_{(i)}))}{n} - \error_\calD(f(\calA, S))} \le \beta\,$.
%     \forall i \in \{1,2,\ldots, n\}, \quad  \Expt{\calS, (x,y) \in \calD}{ \abs{\error\left( f(x) ,y  \right) - \error\left( f_{(i)}(x), y \right) }} \le \frac{\beta}{n} \,,
%     \end{align*}
%     where $f_{(i)} \defeq f(\calA, S_{(i)})$ and $ f \defeq f(\calA, S)$.
% \end{condition}

% \begin{theorem}[Correct statement of \thmref{thm:linear}] \label{thm:new_linear}
%     Assume that this gradient descent algorithm satisfies \codref{cond:hypothesis_stability}
%     with $\beta=\calO(1)$.  
%     Then for any $\delta >0$, with probability at least $1-\delta$ 
%     over the random draws of datasets $\wt S$ and $S$, we have:
%     \begin{align}
%         \error_\calD(\widehat f) \le \error_\calS(\widehat f) + 1 - 2 \error_{\wt\calS}(\widehat f) + \left(\frac{1}{\sqrt{2}} + 1.5 \right) \sqrt{\frac{\log(4/\delta)}{m}} + \sqrt{\frac{4}{\delta}\left(\frac{1}{m} +\frac{3\beta}{m+n} \right)}  \,. \label{eq:gd_error}
%     \end{align} 
%     % for some constant $c\le 3.2$.
% \end{theorem}

\subsection{Proof of \thmref{thm:linear}}
We use a standard result from linear algebra, 
namely the Shermann-Morrison formula 
\citep{sherman1950adjustment} for matrix inversion:  

\begin{lemma}[\citet{sherman1950adjustment}] \label{lem:sherman}
    Suppose $\bA \in \Real^{n \times n}$ 
    is an invertible square matrix 
    and $u,v \in \Real^n$ are column vectors. 
    Then $\bA + uv^T$ is invertible iff $1 + v^T \bA u \ne 0$ 
    and in particular
    \begin{align}
        (\bA + u v^T)^{-1} = \bA^{-1}  - \frac{\bA^{-1} uv^T \bA^{-1} }{ 1 + v^T \bA^{-1} u} \,.
    \end{align}   
\end{lemma}
\newcommand\byy[1]{\by_{\left(#1\right)}}
\newcommand\bXX[1]{\bX_{\left(#1\right)}}
\newcommand\ff[1]{\wh f_{\left(#1\right)}}

For a given training set $S \cup \wt S_C$, 
define leave-one-out error 
on mislabeled points in the training data 
as $$\error_{\text{LOO}(\wt S_M) } = \frac{\sum_{(x_i, y_i) \in \wt S_M} \error( f_{(i)}( x_i), y_i)}{ \abs{\wt S_M }} \,, $$
where $f_{(i)} \defeq f(\calA, (S \cup \wt S)_{(i)})$. 
To relate empirical leave-one-out error and population error 
with hypothesis stability condition, 
we use the following lemma:   

\begin{lemma}[\citet{bousquet2002stability}] \label{lem:stability_error}
    For the leave-one-out error, we have
    \begin{align}
        \Expo{ \left( \error_{\calDm}(\wh f) -\error_{\text{LOO}(\wt S_M) } \right)^2 } \le \frac{1}{2m_1}+  \frac{3\beta}{n + m}\,.
    \end{align}   
    % where $ f \defeq f(\calA, S \cup \wt S) $.
\end{lemma}

Proof of the above lemma is similar 
to the proof of Lemma 9 in \citet{bousquet2002stability} 
and can be found in \appref{app:proof_lem_error}. 
% 
% Before presenting the result, we introduce some notation. 
Before presenting the proof of \thmref{thm:linear}, 
we introduce some more notation. 
Let $\bX_{(i)}$ denote the matrix of covariates 
with the $i^{\text{th}}$ point removed. 
Similarly, let $\by_{(i)}$ be the array of responses 
with the $i^{\text{th}}$ point removed. 
Define the corresponding regularized GD solution 
as $\wh w_{(i)} = \left( \bXX{i}^T\bXX{i}+\lambda \boldsymbol{I}\right)^{-1}\bXX{i}^T\byy{i}$. 
Define $\ff{i}(x) \defeq f(x ; \wh w_{(i)}) $.

\begin{proof}[Proof of \thmref{thm:linear}]
    Because squared loss minimization does not imply 0-1 error minimization, 
    we cannot use arguments from \lemref{lem:fit_mislabeled}. 
    This is the main technical difficulty. 
    To compare the 0-1 error at a train point with an unseen point, 
    we use the closed-form expression for $\widehat{w}$ 
    and Shermann-Morrison formula 
    to upper bound training error 
    with leave-one-out cross validation error. 
    
    The proof is divided into three parts: 
    In part one, we show that 0-1 error 
    on mislabeled points in the training set 
    is lower than the error obtained 
    by leave-one-out error at those points. 
    In part two, we relate this leave-one-out error 
    with the population error on mislabeled distribution
    using \codref{cond:hypothesis_stability}.
    While the empirical leave-one-out error is an unbiased estimator 
    of the average population error of leave-one-out classifiers, 
    we need hypothesis stability 
    to control the variance 
    of empirical leave-one-out error. 
    Finally, in part three, we show 
    that the error on the mislabeled training points 
    can be estimated with just the randomly labeled 
    and clean training data (as in proof of \thmref{thm:error_ERM}).  

    \textbf{Part 1 {} {}} First we relate training error with leave-one-out error.        
    For any training point $(x_i, y_i)$ in $\wt S \cup S$, we have 
    \begin{align}
        \error(\wh f(x_i), y_i ) &= \indict{ y_i \cdot x_i^T \wh w < 0 } = \indict{ y_i \cdot x_i^T \left( \bX^T\bX+\lambda \boldsymbol{I}\right)^{-1}\bX^T\by < 0 } \\
        &= \indict{ y_i \cdot x_i^T \underbrace{\left( \bXX{i}^T\bXX{i} + x_i ^T x_i +\lambda \boldsymbol{I}\right)^{-1}}_{\RN{1}} (\bXX{i}^T\byy{i} + y_i \cdot x_i) < 0 } \,.
    \end{align}
    Letting $\bA = \left(\bXX{i}^T\bXX{i} +\lambda \boldsymbol{I}\right)$ 
    and using \lemref{lem:sherman} on term 1, we have 
    \begin{align}
        \error(\wh f(x_i), y_i ) &= \indict{ y_i \cdot x_i^T \left[\bA^{-1} -  \frac{\bA^{-1} x_i x_i^T \bA^{-1}}{ 1 + x_i ^T \bA^{-1} x_i } \right] (\bXX{i}^T\byy{i} + y_i \cdot x_i) < 0 } \\
        &= \indict{ y_i \cdot\left[ \frac{ x_i^T \bA^{-1} ( 1 + x_i ^T \bA^{-1} x_i ) -  x_i^T \bA^{-1} x_i x_i^T \bA^{-1}}{ 1 + x_i ^T \bA ^{-1}x_i } \right] (\bXX{i}^T\byy{i} + y_i \cdot x_i) < 0 } \\
        &= \indict{ y_i \cdot\left[ \frac{ x_i^T \bA^{-1}}{ 1 + x_i ^T \bA ^{-1}x_i } \right] (\bXX{i}^T\byy{i} + y_i \cdot x_i) < 0 } \,.
    \end{align}

    Since $1 + x_i^T \bA^{-1} x_i > 0$, we have 
    \begin{align}
        \error(\wh f(x_i), y_i ) &= \indict{ y_i \cdot x_i^T \bA^{-1} (\bXX{i}^T\byy{i} + y_i \cdot x_i) < 0 } \\
        &= \indict{ x_i^T \bA^{-1} x_i +  y_i \cdot x_i^T \bA^{-1} (\bXX{i}^T\byy{i}) < 0 } \\
        &\le \indict{ y_i \cdot x_i^T \bA^{-1} (\bXX{i}^T\byy{i}) < 0 } = \error(\ff{i}(x_i), y_i ) \,.\label{eq:LOO_error}
    \end{align}

    Using \eqref{eq:LOO_error}, we have 
    \begin{align}
        \error_{\wt \calS_M } (\wh f) \le \error_{\text{LOO} (\wt S_M)} \defeq \frac{\sum_{(x_i, y_i) \in \wt S_M} \error(\ff{i}(x_i), y_i ) }{\abs{\wt \calS_M}}\label{eq:LOO_error_final} \,.
    \end{align}
    \textbf{Part 2 {}{}} We now relate RHS in \eqref{eq:LOO_error_final} 
    with the population error on mislabeled distribution. 
    To do this, we leverage \codref{cond:hypothesis_stability} 
    and \lemref{lem:stability_error}. 
    In particular, we have 

    \begin{align}
        \Expt{\calS \cup \wt \calS_M }{ \left(\error_{\calDm}(\wh f) - \error_{\text{LOO} (\wt S_M)}\right)^2 } \le \frac{1}{2m_1} + \frac{3\beta}{m+n} \,.
    \end{align}

    Using Chebyshev's inequality, with probability at least $1-\delta$, we have 
    \begin{align}
        \error_{\text{LOO} (\wt S_M)} \le  \error_{\calDm}(\wh f)   + \sqrt{\frac{1}{\delta}\left(\frac{1}{2m_1} +\frac{3\beta}{m+n} \right)} \,. \label{eq:final_mislabeled_linear}
    \end{align}
    

    \textbf{Part 3 {}{}} Combining \eqref{eq:final_mislabeled_linear} and \eqref{eq:LOO_error_final}, we have 

    \begin{align}
        \error_{\wt \calS_M } (\wh f) \le \error_{\calDm}(\wh f)   + \sqrt{\frac{1}{\delta}\left(\frac{1}{2m_1} +\frac{3\beta}{m+n} \right)} \,. \label{eq:linear_parallel_lem1}
    \end{align}

    Compare \eqref{eq:linear_parallel_lem1} with \eqref{eq:lemma1_final} 
    in the proof of \lemref{lem:fit_mislabeled}. 
    We obtain a similar relationship 
    between $\error_{\wt \calS_M }$ and $\error_{\calDm}$ 
    but with a polynomial concentration 
    instead of exponential concentration. 
    In addition, since we just use concentration arguments 
    to relate mislabeled error to the errors
    on the clean and unlabeled portions 
    of the randomly labeled data, 
    we can directly use the results 
    in \lemref{lem:mislabeled_error} and \lemref{lem:clear_error}. 
    Therefore, combining results in \lemref{lem:mislabeled_error}, \lemref{lem:clear_error}, and \eqref{eq:linear_parallel_lem1} with union bound, 
    we have with probability at least $1-\delta$
    \begin{align}
        \error_\calD(\widehat f) \le \error_\calS(\widehat f) + 1 - 2 \error_{\wt\calS}(\widehat f) + \left(\sqrt{2}\error_{\wt\calS}(\widehat f) + 1 + \frac{m}{2n} \right) \sqrt{\frac{\log(4/\delta)}{m}} + \sqrt{\frac{4}{\delta}\left(\frac{1}{m} +\frac{3\beta}{m+n} \right)}  \,.
    \end{align}
    

       
\end{proof}

\subsection{Extension to multiclass classification} \label{app:multiclass_linear}
For multiclass problems with squared loss minimization, as standard practice, we consider one-hot encoding for the underlying label, i.e., a class label $c \in [k]$ is treated as $(0, \cdot, 0,1,0, \cdot, 0) \in \Real^k$ (with $c$-th coordinate being 1).  As before, we suppose that the parameters of the linear function 
are obtained via gradient descent on the following $L_2$ regularized problem: 
\begin{align}
    % n in denominator is avoided deliberately
    \calL_S(w; \lambda) \defeq \sum_{i=1}^n\norm{w^Tx_i - y_i}{2}^2 + \lambda \sum_{j=1}^k \norm{w_j}{2}^2 \,, \label{eq:l2_multiclass_MSE_app}   
\end{align}
where $\lambda\ge0$ is a regularization parameter. 
We assume access to a clean dataset 
$S = \{(x_i, y_i)\}_{i=1}^n \sim \calD^n$ 
and randomly labeled dataset 
$\wt S = \{(x_i, y_i)\}_{i=n+1}^{n+m} \sim \wt \calD^m$. 
Let $\bX = [x_1, x_2, \cdots, x_{m+n}]$ 
and $\by = [e_{y_1}, e_{y_2}, \cdots, e_{y_{m+n}}]$. 
Fix a positive learning rate $\eta$ such that 
$\eta \le 1/\left(\norm{\bX^T\bX}{\text{op}} + \lambda^2\right)$ 
and an initialization $w_0 = 0$. 
% \todos{Assumption made for simplicty}. 
Consider the following gradient descent iterates 
to minimize objective \eqref{eq:l2_MSE_app} on $S \cup \wt S$:
\begin{align}
{w_j}^t = {w_j}^{t-1} - \eta \grad_{w_j} \calL_{S \cup \wt S} (w^{t-1}; \lambda) \quad \forall t=1,2,\ldots \text{ and } j=1,2,\ldots,k  \,. \label{eq:GD_multi_iterates_app}
\end{align} 
Then we have $\{ {w_j}^t\}$ for all $j =1,2,\cdots, k$ converge to the limiting solution 
$\wh w_j = \left( \bX^T\bX+\lambda \boldsymbol{I}\right)^{-1}\bX^T\by_j$. Define $\widehat f (x) \defeq f(x ; \wh w) $.  

\begin{theorem}\label{thm:multi_linear}
    Assume that this gradient descent algorithm satisfies \codref{cond:hypothesis_stability}
    with $\beta=\calO(1)$.  
    Then for a multiclass classification problem wth $k$ classes, for any $\delta >0$, with probability at least $1-\delta$, we have:
    \begin{align*}
        \error_\calD(\widehat f) \le \error_\calS(\widehat f) &+ (k-1)\left(1 - \frac{k}{k-1} \error_{\wt\calS}(\widehat f) \right) \\ &+ \left(k + \sqrt{k} + \frac{m}{n\sqrt{k}} \right) \sqrt{\frac{\log(4/\delta)}{2m}} + \sqrt{k(k-1)} \sqrt{\frac{4}{\delta}\left(\frac{1}{m} +\frac{3\beta}{m+n} \right)}  \,. \numberthis \label{eq:gd_multi_error}
    \end{align*} 
    % for some constant $c\le 3.2$.
\end{theorem}
\begin{proof}
    The proof of this theorem is divided into two parts. In the first part, we relate the error on the mislabeled samples with the population error on the mislabeled data. Similar to the proof of \thmref{thm:linear}, we use Shermann-Morrison formula to upper bound training error with leave-one-out error on each $\wh w^j$. Second part of the proof follows entirely from the proof of \thmref{thm:multiclass_ERM}. In essence, the first part derives an equivalent of \eqref{eq:lemma1_final_multi_prev} for GD training with squared loss and then the second part follows from the proof  of \thmref{thm:multiclass_ERM}. 
    
    \textbf{Part-1:} Consider a training point $(x_i,y_i)$ in $\wt S \cup S $. For simplicity, we use $c_i$ to denote the class of $i$-th point and use $y_i$ as the corresponding one-hot embedding. Recall error in multiclass point is given by $\error(\wh f(x_i), y_i ) = \indict{ c_i \not \in \argmax x_i^T \wh w }$. Thus, there exists a $j \ne c_i \in [k]$, such that we have
     \begin{align}
        \error(\wh f(x_i), y_i ) &= \indict{ c_i \not \in \argmax x_i^T \wh w } = \indict{ x_i^T \wh w_{c_i} < x_i^T \wh w_{j}  } \\ &= \indict{ x_i^T \left( \bX^T\bX+\lambda \boldsymbol{I}\right)^{-1}\bX^T\by_{c_i} < x_i^T \left( \bX^T\bX+\lambda \boldsymbol{I}\right)^{-1}\bX^T\by_{j} } \\
        &= \indict{ x_i^T \underbrace{\left( \bXX{i}^T\bXX{i} + x_i ^T x_i +\lambda \boldsymbol{I}\right)^{-1}}_{\RN{1}} \left(\bXX{i}^T{\by_{c_i}}_{(i)} + x_i - \bXX{i}^T{\by_{j}}_{(i)}\right) < 0 } \,.
    \end{align}
    Letting $\bA = \left(\bXX{i}^T\bXX{i} +\lambda \boldsymbol{I}\right)$ 
    and using \lemref{lem:sherman} on term 1, we have 
    \begin{align}
        \error(\wh f(x_i), y_i ) &= \indict{ x_i^T \left[\bA^{-1} -  \frac{\bA^{-1} x_i x_i^T \bA^{-1}}{ 1 + x_i ^T \bA^{-1} x_i } \right]  \left(\bXX{i}^T{\by_{c_i}}_{(i)} + x_i - \bXX{i}^T{\by_{j}}_{(i)}\right) < 0 } \\
        &= \indict{ \left[ \frac{ x_i^T \bA^{-1} ( 1 + x_i ^T \bA^{-1} x_i ) -  x_i^T \bA^{-1} x_i x_i^T \bA^{-1}}{ 1 + x_i ^T \bA ^{-1}x_i } \right]  \left(\bXX{i}^T{\by_{c_i}}_{(i)} + x_i - \bXX{i}^T{\by_{j}}_{(i)}\right) < 0 } \\
        &= \indict{ \left[ \frac{ x_i^T \bA^{-1}}{ 1 + x_i ^T \bA ^{-1}x_i } \right]  \left(\bXX{i}^T{\by_{c_i}}_{(i)} + x_i - \bXX{i}^T{\by_{j}}_{(i)}\right) < 0} \,.
    \end{align}
    Since $1 + x_i^T \bA^{-1} x_i > 0$, we have 
    \begin{align}
        \error(\wh f(x_i), y_i ) &= \indict{ x_i^T \bA^{-1}  \left(\bXX{i}^T{\by_{c_i}}_{(i)} + x_i - \bXX{i}^T{\by_{j}}_{(i)}\right) < 0 } \\
        &= \indict{ x_i^T \bA^{-1} x_i +  x_i^T \bA^{-1}  \bXX{i}^T{\by_{c_i}}_{(i)}  - x_i^T\bA^{-1}  \bXX{i}^T{\by_{j}}_{(i)} < 0 } \\
        &\le \indict{  x_i^T \bA^{-1}  \bXX{i}^T{\by_{c_i}}_{(i)}  - x_i^T\bA^{-1}  \bXX{i}^T{\by_{j}}_{(i)} < 0  } = \error(\ff{i}(x_i), y_i ) \,.\label{eq:LOO_error_multi}
    \end{align}
    Using \eqref{eq:LOO_error_multi}, we have 
    \begin{align}
        \error_{\wt \calS_M } (\wh f) \le \error_{\text{LOO} (\wt S_M)} \defeq \frac{\sum_{(x_i, y_i) \in \wt S_M} \error(\ff{i}(x_i), y_i ) }{\abs{\wt \calS_M}}\label{eq:LOO_error_multi_final} \,.
    \end{align}
    
    We now relate RHS in \eqref{eq:LOO_error_final} 
    with the population error on mislabeled distribution. 
    Similar as before, to do this, we leverage \codref{cond:hypothesis_stability} 
    and \lemref{lem:stability_error}. Using  \eqref{eq:final_mislabeled_linear} and \eqref{eq:LOO_error_multi_final}, we have 
    \begin{align}
        \error_{\wt \calS_M } (\wh f) \le \error_{\calDm}(\wh f)   + \sqrt{\frac{1}{\delta}\left(\frac{1}{2m_1} +\frac{3\beta}{m+n} \right)} \,. \label{eq:linear_multi_parallel_lem1}
    \end{align}
    
    We have now derived a parallel to \eqref{eq:lemma1_final_multi_prev}. Using the same arguments in the proof of \lemref{lem:fit_mislabeled_multi}, we have 
    \begin{align}
      \error_{\calD}(\wh f) \le  (k-1) \left( 1- \error_{ \wt \calS_M}(\wh f) \right)  + (k-1)\sqrt{\frac{k}{\delta(k-1)}\left(\frac{1}{2m_1} +\frac{3\beta}{m+n} \right)}  \,. \label{eq:lemma1_linear_final_multi}
    \end{align}
    
    \textbf{Part-2:} We now combine the results in \lemref{lem:mislabeled_error_multi} and \lemref{lem:clear_error_multi} to obtain the final inequality in terms of quantities that can be computed from just the randomly labeled and clean data. Similar to the binary case, we obtained a polynomial concentration instead of exponential concentration. Combining \eqref{eq:lemma1_linear_final_multi} with \lemref{lem:mislabeled_error_multi} and \lemref{lem:clear_error_multi}, we have with probability at least $1-\delta$
    \begin{align*}
        \error_\calD(\widehat f) \le \error_\calS(\widehat f) &+ (k-1)\left(1 - \frac{k}{k-1} \error_{\wt\calS}(\widehat f) \right) \\ &+ \left(k + \sqrt{k} + \frac{m}{n\sqrt{k}} \right) \sqrt{\frac{\log(4/\delta)}{2m}} + \sqrt{k(k-1)} \sqrt{\frac{4}{\delta}\left(\frac{1}{m} +\frac{3\beta}{m+n} \right)}  \,. \numberthis \label{eq:gd_multi_error_proof}
    \end{align*} 
\end{proof}

\subsection{Discussion on \codref{cond:hypothesis_stability}} \label{app:discuss_cond1}
The quantity in LHS of \codref{cond:hypothesis_stability} 
measures how much the function learned by the algorithm 
(in terms of error on unseen point) will change 
when one point in the training set is removed. 
% Discussion on exponential concentration and stronger condition. 
% Notice that hypothesis stability implies error stability, i.e., \codref{cond:error_stability} \citep{bousquet2002stability}.  
% In summary, while error stability allowed us 
% to relate the average population error 
% of the leave-one-out classifiers 
% with the population error of the original classifier, 
We need hypothesis stability condition 
to control the variance of the empirical leave-one-out error to show concentration of average leave-one-error with the population error. 

Additionally, we note that while the dominating term in the RHS of \thmref{thm:linear} matches with the dominating term in ERM bound in \thmref{thm:error_ERM}, there is a polynomial concentration term 
(dependence on $1/\delta$ instead of $\log(\sqrt{1/\delta})$) 
in \thmref{thm:linear}. 
Since with hypothesis stability, 
we just bound the variance, 
the polynomial concentration is due 
to the use of Chebyshev's inequality 
instead of an exponential tail inequality
(as in \lemref{lem:fit_mislabeled}).
Recent works have highlighted that 
a slightly stronger condition than hypothesis stability 
can be used to obtain an exponential concentration 
for leave-one-out error \citep{abou2019exponential},
but we leave this for future work for now. 
% We leave 
% However, the constants 

% we also want to highlight  

\subsection{Formal statement and proof of \propref{prop:early_stop}} \label{app:formal_early_stop}

Before formally presenting the result, 
we will introduce some notation.  
By $\calL_{S}(w)$, we denote 
the objective in \eqref{eq:l2_MSE_app} with $\lambda=0$. 
Assume Singular Value Decomposition (SVD) of $\bX$
as $\sqrt{n} \bU \bS^{1/2} \bV^T$. 
Hence $\bX^T \bX = \bV \bS \bV^T$.
Consider the GD iterates defined in \eqref{eq:GD_iterates_app}. 
% 
We now derive closed form expression 
for the $t^\text{th}$ iterate of gradient descent:  
% 
\begin{align}
    w_t = w_{t-1} + \eta \cdot \bX^T (\by - \bX w_{t-1}) = (\bI - \eta \bV \bS \bV^T )w_{k-1} + \eta \bX^T \by \,.
\end{align}
Rotating by $\bV^T$, we get 
\begin{align}
    \wt w_t = (\bI - \eta\bS )\wt w_{k-1} + \eta \wt \by \label{eq:GD_recur},
\end{align}
where $\wt w_t = \bV^T w_t $ and $\wt \by = \bV^T \bX^T \by$. 
Assuming the initial point $w_0 = 0$ 
and applying the recursion in \eqref{eq:GD_recur}, we get
\begin{align}
    \wt w_t = \bS ^{-1} ( \bI - (\bI - \eta \bS)^k ) \wt \by \,, 
\end{align} 
Projecting solution back to the original space, we have 
\begin{align}
     w_t = \bV \bS ^{-1} ( \bI - (\bI - \eta \bS)^k ) \bV^T \bX^T \by \,. 
\end{align} 
% We will work with this GD solution at any iterate $t$ in the next proposition. 
Define $f_t(x) \defeq f(x;w_t)$ 
as the solution at the $t^{\text{th}}$ iterate. 
Let $\wt w_{\lambda} = \argmin_{w} \calL_\calS (w;\lambda) = (\bX^T \bX + \lambda \bI)^{-1} \bX^T \by = \bV (\bS + \lambda \bI )^{-1} \bV^T \bX^T \by $. 
% ) \,,$ for all $t=1,2,\ldots\,.$ 
and define $\wt f_\lambda(x) \defeq f(x;\wt w_\lambda)$ as the regularized solution. 
Assume $\kappa$ be the condition number 
of the population covariance matrix 
and let $s_\text{min}$ be the minimum positive 
singular value of the empirical covariance matrix. 
Our proof idea is inspired from recent work 
on relating gradient flow solution 
and regularized solution 
for regression problems \citep{ali2018continuous}. 
We will use the following lemma in the proof: 
\begin{lemma} \label{lem:ineq_soln}
    For all $x \in [0,1]$ and for all $ k \in \mathbb{N}$, 
    we have (a) $ \frac{kx}{1+kx} \le 1- (1-x)^k$ 
    and (b) $ 1- (1-x)^k \le 2 \cdot \frac{kx}{kx+1} $.
    %  where $g(c)$ is a constant dependent on $c$. For $c = 1$, $g(c) = 2.0$.   
\end{lemma}
\begin{proof}
    % [Proof of \lemref{lem:ineq_soln}]
    % Part (a) is easy. 
    Using $ (1-x)^k \le \frac{1}{1+kx}$, we have part (a). 
    For part (b), we numerically maximize 
    $\frac{ (1+kx ) (1 - (1-x)^k) }{kx}$ 
    for all $k\ge 1$ and for all $x \in [0, 1]$.  
\end{proof}

% 
% Next, 

\begin{prop}[Formal statement of \propref{prop:early_stop}] \label{prop:formal_early_stop}
Let $\lambda = \frac{1}{t\eta}$. 
For a training point $x$, we have 
\begin{align*}
    \Expt{x \sim \calS}{(f_t(x) - \wt f_\lambda(x))^2} &\le c(t,\eta) \cdot \Expt{x \sim \calS}{f_t(x)^2} \,, %\label{eq:early_stop}
\end{align*}
where $c(t, \eta) \defeq \min( 0.25, \frac{1}{s_\text{min}^2 t^2 \eta^2})$. 
Similarly for a test point, we have 
\begin{align*}
    \Expt{x \sim \calD_\calX}{(f_t(x) - \wt f_\lambda(x))^2} &\le \kappa \cdot c(t,\eta) \cdot \Expt{x \sim \calD_\calX}{f_t(x)^2} \,. %\label{eq:early_stop}
\end{align*}
\end{prop} 

\begin{proof}
    %%%%%%%%%%%%% 
    We want to analyze the expected squared difference output 
    of regularized linear regression 
    with regularization constant $\lambda = \frac{1}{\eta t}$ 
    and the gradient descent solution at the $t^\text{th}$ iterate. 
    We separately expand the algebraic expression 
    for squared difference at a training point and a test point. 
    % We start by considering the difference  
    Then the main step is to show that 
    $\left[ \bS ^{-1} ( \bI - (\bI - \eta \bS)^k )  - (\bS + \lambda \bI )^{-1}\right] \preceq c(\eta, t) \cdot \bS ^{-1} ( \bI - (\bI - \eta \bS)^k ) $.

    %%%%%%%%%%%%%
    
   \textbf{Part 1 {} {}} 
    First, we will analyze the squared difference 
    of the output at a training point 
    (for simplicity, we refer to $S \cup \wt S$ as $S$), i.e., 
    \begin{align}
        \Expt{ x \sim \calS }{\left(f_t(x) - \wt f_\lambda (x)\right)^2} &= \norm{\bX w_t - \bX \wt w_\lambda}{2}^2\\ &=   \norm{\bX \bV \bS ^{-1} ( \bI - (\bI - \eta \bS)^t ) \bV^T \bX^T \by - \bX \bV (\bS + \lambda \bI )^{-1} \bV^T \bX^T \by }{2}^2 \\
        &= \norm{\bX \bV \left(\bS ^{-1} ( \bI - (\bI - \eta \bS)^t ) - (\bS + \lambda \bI )^{-1} \right) \bV^T \bX^T \by  }{2} \\
        &=  \by^T \bV \bX \left( \underbrace{\bS ^{-1} ( \bI - (\bI - \eta \bS)^t ) - (\bS + \lambda \bI )^{-1}}_{\RN{1}} \right)^2 \bS \bV^T \bX^T \by \label{eq:train_GD_rel} \,.
        %  (\bX \bV \bS ^{-1} ( \bI - (\bI - \eta \bS)^k ) \bV^T \bX^T \by)^T \bX \bV \bS ^{-1} ( \bI - (\bI - \eta \bS)^k ) \bV^T \bX^T \by
    \end{align}
    We now separately consider term 1. 
    Substituting $\lambda = \frac{1}{t \eta}$, 
    we get
    \begin{align}
        \bS ^{-1} ( \bI - (\bI - \eta \bS)^t ) - (\bS + \lambda \bI )^{-1} &= \bS^{-1} \left( ( \bI - (\bI - \eta \bS)^t ) - (\bI + \bS^{-1} \lambda )^{-1}\right) \\
        &= \underbrace{\bS^{-1} \left( ( \bI - (\bI - \eta \bS)^t ) - (\bI + ( \bS t \eta)^{-1}  )^{-1}\right)}_{\bA} \,.
    \end{align}

    We now separately bound the diagonal entries in matrix $\bA$. 
    With $s_i$, we denote $i^{\text{th}}$ diagonal entry of $\bS$.
    Note that since $ \eta\le 1/\norm{S}{\text{op}}$, 
    for all $i$, $\eta s_i  \le 1$.  
    Consider $i^{\text{th}}$ diagonal term (which is non-zero) 
    of the diagonal matrix $\bA$, we have 
    \begin{align}
        \bA_{ii} = \frac{1}{s_i} \left(  1 - (1 - s_i \eta)^t - \frac{t \eta s_i}{1 + t \eta s_i } \right) &=  \frac{1 - (1 - s_i \eta)^t}{s_i} \left( \underbrace{ 1 - \frac{t \eta s_i}{(1 + t \eta s_i)(1 - (1 - s_i \eta)^t)}}_{\RN{2}} \right) \\ 
         &\le \frac{1}{2}\left[ \frac{1 - (1 - s_i \eta)^t}{ s_i} \right] \tag*{(Using \lemref{lem:ineq_soln} (b))} \,.
    \end{align} 
    Additionally, we can also show the following upper bound on term 2: 
    \begin{align}
         1 - \frac{t \eta s_i}{(1 + t \eta s_i)(1 - (1 - s_i \eta)^t)} &= \frac{(1 + t \eta s_i)(1 - (1 - s_i \eta)^t) - t \eta s_i }{(1 + t \eta s_i)(1 - (1 - s_i \eta)^t)} \\
         & \le  \frac{ 1 -  (1 - s_i \eta)^t - t \eta s_i (1 - s_i \eta)^t}{(1 + t \eta s_i)(1 - (1 - s_i \eta)^t)} \\
         & \le \frac{1}{t\eta s_i} \,. \tag{Using \lemref{lem:ineq_soln} (a)}
        %  &\le \frac{1}{2}\left[ \frac{1 - (1 - s_i \eta)^t}{ s_i} \right] \tag*{(Using \lemref{lem:ineq_soln})} \,.
    \end{align} 

    Combining both the upper bounds 
    on each diagonal entry $\bA_{ii}$, we have 
    \begin{align}
    \bA \preceq c_1(\eta, t) \cdot \bS^{-1} ( \bI - (\bI - \eta \bS)^t ) \,, \label{eq:upperbound_diagonal}
    \end{align}
    where $c_1(\eta, t ) = \min(0.5, \frac{1}{t s_i \eta })$. Plugging this into \eqref{eq:train_GD_rel}, we have 
    \begin{align}
        \Expt{ x \sim \calS }{\left(f_t(x) - \wt f_\lambda (x)\right)^2} &\le c(\eta, t) \cdot \by^T \bV \bX  \left( \bS^{-1} ( \bI - (\bI - \eta \bS)^t ) \right)^2 \bS \bV^T \bX^T \by \\
        &=   c(\eta, t) \cdot \by^T \bV \bX  \left( \bS^{-1} ( \bI - (\bI - \eta \bS)^t ) \right) \bS \left( \bS^{-1} ( \bI - (\bI - \eta \bS)^t ) \right) \bV^T \bX^T \by \\
        & =  c(\eta, t) \cdot \norm{\bX w_t}{2}^2 \\
        &= c(\eta, t) \cdot  \Expt{ x \sim \calS }{\left(f_t(x) \right)^2} \,,
    \end{align}
    where $c(\eta, t ) = \min(0.25, \frac{1}{t^2 s^2_i \eta^2 })$.

    \textbf{Part 2 {} {}} With $\bSigma$, 
    we denote the underlying true covariance matrix. 
    We now consider the squared difference of output at an unseen point: 
    \begin{align}
        \Expt{ x \sim \calD_{\calX} }{\left(f_t(x) - \wt f_\lambda (x)\right)^2} &= \Expt{x \sim \calD_{\calX}}{\norm{x^T w_t - x^T \wt w_\lambda}{2}} \\
        &=   \norm{x^T \bV \bS ^{-1} ( \bI - (\bI - \eta \bS)^t ) \bV^T \bX^T \by - x^T \bV (\bS + \lambda \bI )^{-1} \bV^T \bX^T \by }{2} \\
        &= \norm{x^T \bV \left(\bS ^{-1} ( \bI - (\bI - \eta \bS)^t ) - (\bS + \lambda \bI )^{-1} \right) \bV^T \bX^T \by  }{2} \\
        &= \by^T \bV \bX \left( \bS ^{-1} ( \bI - (\bI - \eta \bS)^t ) - (\bS + \lambda \bI )^{-1} \right) \bV^T \bSigma \bV \\ &\qquad \qquad \qquad \qquad \qquad \left( (\bI - (\bI - \eta \bS)^t ) - (\bS + \lambda \bI )^{-1} \right) \bV^T \bX^T \by \\
        &\le \sigma_{\text{max}} \cdot \by^T \bV \bX \left( \underbrace{\bS ^{-1} ( \bI - (\bI - \eta \bS)^t ) - (\bS + \lambda \bI )^{-1}}_{\RN{1}} \right)^2 \bV^T \bX^T \by \,, \label{eq:test_GD_rel}
        %  (\bX \bV \bS ^{-1} ( \bI - (\bI - \eta \bS)^k ) \bV^T \bX^T \by)^T \bX \bV \bS ^{-1} ( \bI - (\bI - \eta \bS)^k ) \bV^T \bX^T \by
    \end{align}
    where $\sigma_{\text{max}}$ is the maximum eigenvalue 
    of the underlying covariance matrix $\bSigma$. 
    Using the upper bound on term 1 in \eqref{eq:upperbound_diagonal}, 
    we have 
    \begin{align}
        \Expt{ x \sim \calD_{\calX} }{\left(f_t(x) - \wt f_\lambda (x)\right)^2} &\le \sigma_{\text{max}} \cdot c(\eta, t) \cdot \by^T \bV \bX  \left( \bS^{-1} ( \bI - (\bI - \eta \bS)^t ) \right)^2 \bV^T \bX^T \by \\
        &=   \kappa \cdot c(\eta, t) \cdot \sigma_{\text{min}}\cdot \norm{\bV \left( \bS^{-1} ( \bI - (\bI - \eta \bS)^t ) \right) \bV^T \bX^T \by}{2}^2 \\
        &\le \kappa \cdot c(\eta, t) \cdot \left[ \bV \left( \bS^{-1} ( \bI - (\bI - \eta \bS)^t ) \right) \bV^T \bX^T \right]^T \bSigma \\
        &\qquad \qquad \qquad \qquad \qquad \left[ \bV \left( \bS^{-1} ( \bI - (\bI - \eta \bS)^t ) \right) \bV^T \bX^T \right] \by \\
        & = \kappa \cdot c(\eta, t) \cdot \Expt{x \sim \calD_{\calX}}{\norm{x^T w_t}{2}} \,.
    \end{align}
% 
% 
    % Since $ \eta\le 1/\norm{S}{\text{op}}$, invoking \lemref{lem:ineq_soln} to upper bound term 1 with
\end{proof}

\subsection{Extension to deep learning} \label{appsubsec:ext_DL}
Under \asmpref{appsubsec:justifying_assumption1}, we present the formal result parallel to \thmref{thm:multiclass_ERM}. 
\begin{theorem} \label{thm:multiclass_ERM_algoA}
    Consider a multiclass classification problem 
    with $k$ classes. Under \asmpref{asmp:deep_models}, 
    for any $\delta >0$, with probability at least $1-\delta$,
    we have
    \vspace{-10pt}
    \begin{align*}
        \error_\calD(\widehat f)  \le \error_\calS(\widehat f) + (k-1) \left(1 - \tfrac{k}{k-1} \error_{\wt\calS}(\widehat f)\right) + c\sqrt{\frac{\log(\frac{4}{\delta})}{2m}} \,,\numberthis \label{eq:multiclass_ERM_deep}
    % \vspace{-20pt}
    \end{align*}
    for some constant $c \le ((c+1) k+\sqrt{k} + \frac{m}{n\sqrt{k}})$.
\end{theorem}

The proof follows exactly as in step (i) to (iii) in \thmref{thm:multiclass_ERM}.  

\subsection{Justifying~\asmpref{asmp:deep_models}} \label{appsubsec:justifying_assumption1}

Motivated by the analysis on linear models, we now discuss alternate (and weaker) conditions that imply \asmpref{asmp:deep_models}. 
We need hypothesis stability (\codref{cond:hypothesis_stability}) and the following assumption relating training error and leave-one-error: 

\begin{assumption} \label{asmp:loo_error}
Let $\wh f$ be a model obtained by training with algorithm $\calA$ on a mixture of clean $S$ and randomly labeled data $\wt S$. Then we assume we have 
\begin{align*}
    \error_{\wt \calS_M} (\wh f) \le  \error_{\text{LOO} (\wt S_M)} \,, 
\end{align*}
for all $(x_i, y_i) \in  \wt S_M$ where $\wh f_{(i)} \defeq f(\calA, S \cup {{}\wt S_M}_{(i)})$ and  $\error_{\text{LOO} (\wt S_M)} \defeq  \frac{\sum_{(x_i, y_i) \in \wt S_M} \error(\ff{i}(x_i), y_i ) }{\abs{\wt \calS_M}}$.  
\end{assumption}

% we assume this to extend our result (parallel to \thmref{thm:multi_linear}) for deep models. 
Intuitively, this assumption states that the error on a (mislabeled) datum $(x,y)$ included in the training set is less than the error on that datum $(x,y)$ obtained by a model trained on the training set $S - \{(x,y)\}$. We proved this for linear models trained with GD in the proof of \thmref{thm:multi_linear}. 
% 
\codref{cond:hypothesis_stability} with $\beta = \calO(1)$ and \asmpref{asmp:loo_error} together with \lemref{lem:stability_error} implies \asmpref{asmp:deep_models} with a polynomial residual term (instead of logarithmic in $1/\delta$): 
\begin{align}
     \error_{\calS_M} (\wh f) \le  \error_{\calDm}(\wh f)   + \sqrt{\frac{1}{\delta}\left(\frac{1}{m} +\frac{3\beta}{m+n} \right)} \,.
\end{align}
% Note that this  

\newpage 
\section{Additional experiments and details}\label{app:exp}
\newcommand\tab[1][1cm]{\hspace*{#1}}

\subsection{Datasets} \label{sec:app_dataset}

\textbf{Toy Dataset {} {}} Assume fixed constants $\mu$ and $\sigma$. For a given label $y$, we simulate features $x$ in our toy classification setup as follows: 
\begin{align*}
    x \defeq \texttt{concat} \left[ x_1, x_2\right] \quad \text{where} \quad  x_1 \sim  \calN( y \cdot \mu, \sigma^2 I_{d \times d}) \ \  \text{and} \ \  x_1 \sim  \calN( 0, \sigma^2 I_{d \times d}) \,.
\end{align*}  
% where $y$ is the true label and $x$ is the corresponding feature vector. 
In experiements throughout the paper, we fix dimention $d=100$, $\mu = 1.0 $, and $\sigma = \sqrt{d}$. Intuitively, $x_1$ carries the information about the underlying label and $x_2$ is additional noise independent of the underlying label. 

\textbf{CV datasets {} {}} We use MNIST~\citep{lecun1998mnist} and CIFAR10~\cite{krizhevsky2009learning}. 
% For binary tasks, 
We produce a binary variant from the multiclass classification problem by mapping classes $\{0,1,2,3,4\}$ to label $1$ and $\{ 5,6,7,8,9\}$ to label $-1$. For CIFAR dataset, we also use the standard data augementation of random crop and horizontal flip. PyTorch code is as follows: 

\texttt{(transforms.RandomCrop(32, padding=4),\\
\tab transforms.RandomHorizontalFlip())}

\textbf{NLP dataset {} {}} We use IMDb Sentiment analysis~\citep{maas2011learning} corpus.  

\subsection{Architecture Details} 

All experiments were run on NVIDIA GeForce RTX 2080 Ti GPUs. We used PyTorch~\citep{NEURIPS2019a9015} and Keras with Tensorflow~\citep{abadi2016tensorflow} backend for experiments. 
% , ELMo embeddings~\citep{Peters:2018}, and Hugging Face Transformers~\citep{wolf-etal-2020-transformers}. 

\textbf{Linear model {} {}} For the toy dataset, we simulate a linear model with scalar output and the same number of parameters as the number of dimensions.   

\textbf{Wide nets {} {}} To simulate the NTK regime, we experiment with $2-$layered wide nets. The PyTorch code for 2-layer wide MLP is as follows: 


\texttt{ nn.Sequential( \\
\tab     nn.Flatten(),\\
\tab    nn.Linear(input\_dims, 200000, bias=True),\\
\tab    nn.ReLU(),\\
\tab    nn.Linear(200000, 1, bias=True)\\
\tab     )}


We experiment both (i) with the second layer fixed at random initialization; (ii)  and updating both layers' weights.     

\textbf{Deep nets for CV tasks {} {}} We consider a 4-layered MLP. The PyTorch code for 4-layer MLP is as follows: 

\texttt{ nn.Sequential(nn.Flatten(), \\
\tab        nn.Linear(input\_dim, 5000, bias=True),\\
\tab        nn.ReLU(),\\
\tab        nn.Linear(5000, 5000, bias=True),\\
\tab        nn.ReLU(),\\
\tab        nn.Linear(5000, 5000, bias=True),\\
\tab        nn.ReLU(),\\
% \tab        nn.Linear(5000, 5000, bias=True),\\
% \tab        nn.ReLU(),\\
\tab        nn.Linear(1024, num\_label, bias=True)\\
\tab        )}

For MNIST, we use $1000$ nodes instead of $5000$ nodes in the hidden layer. 
% 
We also experiment with convolutional nets. In particular, we use ResNet18 \citep{he2016deep}. Implementation adapted from:  \url{https://github.com/kuangliu/pytorch-cifar.git}. 

\textbf{Deep nets for NLP {} {}} We use a simple LSTM model with embeddings intialized with ELMo embeddings~\citep{Peters:2018}. Code adapted from: \url{https://github.com/kamujun/elmo_experiments/blob/master/elmo_experiment/notebooks/elmo_text_classification_on_imdb.ipynb} 

We also evaluate our bounds with a BERT model. In particular, we fine-tune an off-the-shelf uncased BERT model~\citep{devlin2018bert}. Code adapted from Hugging Face Transformers~\citep{wolf-etal-2020-transformers}: \url{https://huggingface.co/transformers/v3.1.0/custom_datasets.html}. 


\subsection{Additonal experiments}

\textbf{Results with SGD on underparameterized linear models {} {}} 

\begin{figure*}[h]
    \centering 
    % \vspace{-15pt}
    % \includegraphics[width=0.9\linewidth]{example-image-a}
    \includegraphics[width=0.3\linewidth]{figures/lowdim-Gaussian-SGD.pdf}
    % \includegraphics[width=0.9\linewidth]{figures/{CIFAR10_rn=0.1_lr=0.2_wd=0.005}.png}
    \vspace{-5pt}
    \caption{ 
    % Predicted lower bound 
    % on different
    We plot the accuracy and corresponding bound 
    (RHS in \eqref{eq:erm}) at $\delta = 0.1$
    for toy binary classification task. 
    Results aggregated over $3$ seeds. 
    % i.e., $1-\error$ where $\error$ is the term in the RHS of \eqref{eq:erm}
    Accuracy vs fraction of unlabeled data (w.r.t clean data) 
    in the toy setup with a linear model trained with SGD. Results parallel to \figref{fig:error_binary}(a) with SGD.  }
    \label{fig:error_binary_linear}
    \vspace{-5pt}
\end{figure*}

\textbf{Results with wide nets on binary MNIST {} {}}

\begin{figure*}[h]
    \centering 
    % \vspace{-15pt}
    % \includegraphics[width=0.9\linewidth]{example-image-a}
    \subfigure[GD with MSE loss]{\includegraphics[width=0.3\linewidth]{figures/MNIST-GD_MSE.pdf}} \hfil
    \subfigure[SGD with CE loss]{\includegraphics[width=0.3\linewidth]{figures/MNIST-SGD_CE.pdf}}
    \subfigure[SGD with MSE loss]{\includegraphics[width=0.3\linewidth]{figures/MNIST-SGD_MSE-first-layer.pdf}}
    % \includegraphics[width=0.9\linewidth]{figures/{CIFAR10_rn=0.1_lr=0.2_wd=0.005}.png}
    \vspace{-5pt}
    \caption{ 
    % Predicted lower bound 
    % on different
    We plot the accuracy and corresponding bound 
    (RHS in \eqref{eq:erm}) at $\delta = 0.1$ 
    for binary MNIST classification. 
    Results aggregated over $3$ seeds. 
    % i.e., $1-\error$ where $\error$ is the term in the RHS of \eqref{eq:erm}
    Accuracy vs fraction of unlabeled data 
    for a 2-layer wide network on binary MNIST with both the layers training in (a,b) and only first layer training in (c). 
    Results parallel to \figref{fig:error_binary}(b) .  }
    \label{fig:error_binary_MNIST}
    \vspace{-5pt}
\end{figure*}

% \begin{figure*}[h]
%     \centering 
%     % \vspace{-15pt}
%     % \includegraphics[width=0.9\linewidth]{example-image-a}
%     \subfigure[GD with MSE loss]{\includegraphics[width=0.3\linewidth]{figures/MNIST.pdf}} \hfil
    
%     \subfigure[SGD with CE loss]{\includegraphics[width=0.3\linewidth]{figures/MNIST.pdf}}
%     % \includegraphics[width=0.9\linewidth]{figures/{CIFAR10_rn=0.1_lr=0.2_wd=0.005}.png}
%     \vspace{-5pt}
%     \caption{ 
%     % Predicted lower bound 
%     % on different
%     We plot the accuracy and corresponding bound 
%     (RHS in \eqref{eq:erm}) at $\delta = 0.1$
%     for binary MNIST classification. 
%     Results aggregated over $3$ seeds. 
%     % i.e., $1-\error$ where $\error$ is the term in the RHS of \eqref{eq:erm}
%     Accuracy vs fraction of unlabeled data 
%     for a 2-layer wide network on binary MNIST with just the first layer training. 
%     Results parallel to \figref{fig:error_binary}(b) with only the first layer training.  }
%     \label{fig:error_binary_MNIST}
%     \vspace{-5pt}
% \end{figure*}

\textbf{Results on CIFAR 10 and MNIST {} {}} 
% 
We plot epoch wise error curve for results in \tabref{table:multiclass}(\figref{fig:error_epoch_CIFAR10} and \figref{fig:error_epoch_MNIST}). We observe the same trend as in \figref{fig:error_CIFAR10}. Additionally, we plot an \emph{oracle bound} obtained by tracking the error on mislabeled data which nevertheless were predicted as true label. To obtain an exact emprical value of the oracle bound, we need underlying true labels for the randomly labeled data. 
% Note that our bound in \thmref{thm:multiclass_ERM}, lower bounds the accuracy as predicted by the oracle bound. 
While with just access to extra unlabeled data we cannot calculate oracle bound, we note that the oracle bound is very tight and never violated in practice underscoring an importamt aspect of generalization in multiclass problems. This highlight that even a stronger conjecture may hold in multiclass classification, i.e., error on mislabeled data (where nevertheless true label was predicted) lower bounds the population error on the distribution of mislabeled data and hence, the error on (a specific) mislabeled portion predicts the population accuracy on clean data. 
% 
On the other hand, the dominating term of in \thmref{thm:multiclass_ERM} is loose when compared with the oracle bound. The main reason, we believe is the pessimistic upper bound in \eqref{eq:lemma1_final_multi_prev} in the proof of \lemref{lem:fit_mislabeled_multi}. We leave an investigation on this gap for future. 
% of fit 

% However, oracle bound highlights two . One,  



\begin{figure}[h]
    \centering 
    % \vspace{-15pt}
    % \includegraphics[width=0.9\linewidth]{example-image-a}
    \subfigure[MLP]{\includegraphics[width=0.3\linewidth]{figures/CIFAR10-FNN.pdf}} \hfil
    \subfigure[ResNet]{\includegraphics[width=0.3\linewidth]{figures/CIFAR10-Resnet.pdf}}
    % \includegraphics[width=0.9\linewidth]{figures/{CIFAR10_rn=0.1_lr=0.2_wd=0.005}.png}
    % \vspace{-10pt}
    \caption{ Per epoch curves for CIFAR10 corresponding results in \tabref{table:multiclass}. As before, we just plot the dominating term in the RHS of \eqref{eq:multiclass_ERM} as predicted bound. Additionally, we also plot the predicted lower bound by the error on mislabeled data which nevertheless were predicted as true label. We refer to this as ``Oracle bound''. See text for more details. 
    % 
    % except for the stopping point. 
    % The bound predicted by RATT (RHS in \eqref{eq:multiclass_ERM}) is vacuous. 
    }\label{fig:error_epoch_CIFAR10}
    % \vspace{-15pt}
\end{figure}


\begin{figure}[h]
    \centering 
    % \vspace{-15pt}
    % \includegraphics[width=0.9\linewidth]{example-image-a}
    \subfigure[MLP]{\includegraphics[width=0.3\linewidth]{figures/MNIST-FNN.pdf}} \hfil
    \subfigure[ResNet]{\includegraphics[width=0.3\linewidth]{figures/MNIST-Resnet.pdf}}
    % \includegraphics[width=0.9\linewidth]{figures/{CIFAR10_rn=0.1_lr=0.2_wd=0.005}.png}
    % \vspace{-10pt}
    \caption{ Per epoch curves for MNIST corresponding results in \tabref{table:multiclass}. As before, we just plot the dominating term in the RHS of \eqref{eq:multiclass_ERM} as predicted bound. Additionally, we also plot the predicted lower bound by the error on mislabeled data which nevertheless were predicted as true label. We refer to this as ``Oracle bound''. See text for more details. 
    % 
    % except for the stopping point. 
    % The bound predicted by RATT (RHS in \eqref{eq:multiclass_ERM}) is vacuous. 
    }\label{fig:error_epoch_MNIST}
    % \vspace{-15pt}
\end{figure}

\textbf{Results on CIFAR 100 {} {}} 
% 
On CIFAR100, our bound in \eqref{eq:multiclass_ERM} yields vacous bounds. However, the oracle bound as explained above yields tight guarantees in the initial phase of the learning (i.e., when learning rate is less than $0.1$) (\figref{fig:error_CIFAR100}).  

\begin{figure}[h]
    \centering 
    % \vspace{-15pt}
    % \includegraphics[width=0.9\linewidth]{example-image-a}
    \includegraphics[width=0.3\linewidth]{figures/CIFAR100-Resnet.pdf}
    % \includegraphics[width=0.9\linewidth]{figures/{CIFAR10_rn=0.1_lr=0.2_wd=0.005}.png}
    % \vspace{-10pt}
    \caption{ Predicted lower bound by the error on mislabeled data which nevertheless were predicted as true label with ResNet18 on CIFAR100. We refer to this as ``Oracle bound''. See text for more details. 
    % 
    % except for the stopping point. 
    The bound predicted by RATT (RHS in \eqref{eq:multiclass_ERM}) is vacuous. 
    }\label{fig:error_CIFAR100}
    % \vspace{-15pt}
\end{figure}


% \paragraph{Experiments on CIFAR100} 


% \subsection{Model Selection using RATT}


\subsection{Hyperparameter Details}


\textbf{\figref{fig:error_CIFAR10} {} {}} We use clean training dataset of size $40,000$. We fix the amount of unlabeled data at $20\%$ of the clean size, i.e. we include additional $8,000$ points with randomly assigned labels. We use test set of $10,000$ points. For both MLP and ResNet, we use SGD with an initial learning rate of $0.1$ and momentum $0.9$. We fix the weight decay parameter at $5\times 10^{-4}$. After $100$ epochs, we decay the learning rate to $0.01$. We use SGD batch size of $100$. 

\textbf{\figref{fig:error_binary} (a) {} {}} We obtain a toy dataset according to the process described in \secref{sec:app_dataset}. We fix $d=100$ and create a dataset of $50,000$ points with balanced classes. Moreover, we sample additional covariates with the same procedure to create randomly labeled dataset. For both SGD and GD training, we use a fixed learning rate $0.1$.    

\textbf{\figref{fig:error_binary} (b) {} {}} Similar to binary CIFAR, we use clean training dataset of size $40,000$ and fix the amount of unlabeled data at $20\%$ of the clean dataset size. To train wide nets, we use a fixed learning of $0.001$ with GD and SGD. We decide the weight decay parameter and the early stopping point that maximizes our generalization bound (i.e. without peeking at unseen data ).  We use SGD batch size of $100$. 

\textbf{\figref{fig:error_binary} (c) {} {}} With IMDb dataset, we use a clean dataset of size $20,000$ and as before, fix the amount of unlabeled data at $20\%$ of the clean data. To train ELMo model, we use Adam optimizer with a fixed learning rate $0.01$ and weight decay $10^{-6}$ to minimize cross entropy loss. We train with batch size $32$ for 3 epochs. To fine-tune BERT model, we use Adam optimizer with learning rate $5\times 10^{-5}$ to minimize cross entropy loss. We train with a batch size of $16$ for 1 epoch.    

\textbf{\tabref{table:multiclass} {} {}} For multiclass datasets, we train both MLP and ResNet with the same hyperparameters as described before. We sample a clean training dataset of size $40,000$ and fix the amount of unlabeled data at $20\%$ of the clean size. We use SGD with an initial learning rate of $0.1$ and momentum $0.9$. We fix the weight decay parameter at $5\times 10^{-4}$. After $30$ epochs for ResNet and after $50$ epochs for MLP, we decay the learning rate to $0.01$.  We use SGD with batch size $100$. 
For \figref{fig:error_CIFAR100}, we use the same hyperparameters as 
CIFAR10 training, except we now decay learning rate after $100$ epochs. 


In all experiments, to identify the best possible accuracy on just the clean data, we use the exact same set of hyperparamters except the stopping point. We choose a stopping point that maximizes test performance. 

\subsection{Summary of experiments }

\begin{center}
    \begin{table}[H] 
        \centering
        \begin{tabular}{|c|c|c|c|} 
        \hline
        Classification type & Model category & Model & Dataset  \\ [0.5ex] 
        \hline
        \hline
        \multirow{10}{*}{Binary} & Low dimensional & Linear model & Toy Gaussain dataset  \\
                        \cline{2-4}
                         & Overparameterized 
                        %  & Linear model & Toy Gaussain dataset \\
                        %  \cline{3-4}
                        %  & & 2-layer wide net& Toy Gaussain dataset \\
                        %  \cline{3-4}
                         & \multirow{2}{*}{2-layer wide net} & \multirow{2}{*}{Binary MNIST} \\
                         & linear nets & &  
                         \\
                         \cline{2-4}                 
                         & \multirow{6}{*}{Deep nets} & \multirow{2}{*}{MLP} & Binary MNIST \\
                         \cline{4-4}
                         & &  & Binary CIFAR \\
                         \cline{3-4}
                         &  & \multirow{2}{*}{ResNet} & Binary MNIST \\
                         \cline{4-4}
                         & &  & Binary CIFAR \\
                         \cline{3-4}
                         &  & ELMo-LSTM model & IMDb Sentiment Analysis \\
                         \cline{3-4}
                         & & BERT pre-trained model & IMDb Sentiment Analysis \\
        \hline
        \multirow{5}{*}{Multiclass} & \multirow{5}{*}{Deep nets} & \multirow{2}{*}{MLP} & MNIST \\
                        \cline{4-4} 
                        & & & CIFAR10 \\                   
                        \cline{3-4}
                         &   & \multirow{3}{*}{ResNet} & MNIST \\
                         \cline{4-4}
                         &   & & CIFAR10 \\
                         \cline{4-4}
                         &   & & CIFAR100 \\
        \hline
        \end{tabular}
        % \caption{Summary of experiments performed} \label{table:experiments}
    \end{table}    
    % \footnotetext[6]{We use both MSE loss and cross-entropy loss.}
    % \footnotetext[6]{We try 2 variants: one with a fixed first layer and the other with both layers trainable.}
\end{center}

\newpage
\section{Proof of \lemref{lem:stability_error}} \label{app:proof_lem_error}

\begin{proof}[Proof of \lemref{lem:stability_error}]
    Recall, we have a training set $S \cup \wt S_C$. We defined leave-one-out error on mislabeled points as $$\error_{\text{LOO}(\wt S_M) } = \frac{\sum_{(x_i, y_i) \in \wt S_M} \error( f_{(i)}( x_i), y_i)}{ \abs{\wt S_M }} \,, $$
    where $f_{(i)} \defeq f(\calA, (S \cup \wt S)_{(i)})$. Define $S^\prime \defeq S \cup \wt S$. Assume $(x,y)$ and $(x^\prime,y^\prime)$ as i.i.d. samples from ${\calDm}$. 
    Using Lemma 25 in \citet{bousquet2002stability}, we have
    \begin{align*}
        \Expo{ \left( \error_{\calDm}(\wh f) -\error_{\text{LOO}(\wt S_M) } \right)^2 } \le & \Expt{ S^\prime, (x,y), (x^\prime,y^\prime) }{ \error(\wh f(x), y ) \error(\wh f(x^\prime), y^\prime )} - 2 \Expt{ S^\prime, (x,y) }{ \error(\wh f(x), y ) \error(f_{(i)}(x_i), y_i )} \\
        & + \frac{m_1-1}{m_1}\Expt{ S^\prime }{  \error(f_{(i)}(x_i), y_i )  \error(f_{(j)}(x_j), y_j )} + \frac{1}{m_1} \Expt{ S^\prime }{  \error(f_{(i)}(x_i), y_i ) } \,. \numberthis \label{eq:main_reln}
    \end{align*}
    We can rewrite the equation above as : 
    \begin{align*}
        \Expo{ \left( \error_{\calDm}(\wh f) -\error_{\text{LOO}(\wt S_M) } \right)^2 } \le &  \, \underbrace{\Expt{ S^\prime, (x,y), (x^\prime,y^\prime) }{ \error(\wh f(x), y ) \error(\wh f(x^\prime), y^\prime ) - \error(\wh f(x), y ) \error(f_{(i)}(x_i), y_i )}}_{\RN{1}} \\
        & + \underbrace{\Expt{ S^\prime }{  \error(f_{(i)}(x_i), y_i )  \error(f_{(j)}(x_j), y_j ) -  \error(\wh f(x), y ) \error(f_{(i)}(x_i), y_i )}}_{\RN{2}} \\ &+ \underbrace{\frac{1}{m_1} \Expt{ S^\prime }{  \error(f_{(i)}(x_i), y_i ) - \error(f_{(i)}(x_i), y_i )  \error(f_{(j)}(x_j), y_j ) }}_{\RN{3}} \,. \numberthis \label{eq:main_reln2}
    \end{align*}
    
    We will now bound term $\RN{3}$.  Using Cauchy-Schwarz's inequality, we have
    
    \begin{align}
        \Expt{ S^\prime }{  \error(f_{(i)}(x_i), y_i ) - \error(f_{(i)}(x_i), y_i )  \error(f_{(j)}(x_j), y_j ) }^2 &\le  \Expt{ S^\prime }{  \error(f_{(i)}(x_i), y_i ) }^2 \Expt{S^\prime}{1 -   \error(f_{(j)}(x_j), y_j ) }^2 \\
        &\le \frac{1}{4} \,.\label{eq:term1_lem12}
    \end{align}
    
    Note that since $(x_i,y_i)$, $(x_j ,y_j )$, $(x,y)$, and $(x^\prime, y^\prime)$ are all from same distribution $\calDm$, we directly incorporate the bounds on term $\RN{1}$ and $\RN{2}$ from the proof of Lemma 9 in \citet{bousquet2002stability}. Combining that with \eqref{eq:term1_lem12} and our definition of hypothesis stability in \codref{cond:hypothesis_stability}, we have the required claim. 
    
    
    % We now re-write term $\RN{1}$ as
    % \begin{align*}
    %         &\Expt{S^\prime, (x,y), (x^\prime,y^\prime) }{ \error(\wh f(x), y ) \error(\wh f(x^\prime), y^\prime ) - \error(\wh f(x), y ) \error(f_{(i)}(x_i), y_i )} \\ & \qquad = \Expt{ S^\prime, (x,y), (x^\prime,y^\prime) }{ \error(\wh f(x), y ) \error(\wh f  (x^\prime), y^\prime ) - \error(\wh f ^\prime(x), y ) \error(f_{(i)}(x^\prime), y^\prime )} \tag{Exchanging $(x_i, y_i)$ with $(x^\prime, y^\prime)$ in the second term} \\
    %         & \qquad = \Expt{ S^\prime, (x,y), (x^\prime,y^\prime) }{  \left(\error(\wh f(x), y )-  \error(f_{(i)}(x), y ) \right) \error(\wh f  (x^\prime), y^\prime )  } \\
    %         & \qquad  + \Expt{ S^\prime, (x,y), (x^\prime,y^\prime) }{  \left(\error(f_{(i)}(x), y ) -\error(\wh f ^\prime(x), y ) \right) \error(\wh f  (x^\prime), y^\prime )}  \\
    %         & \qquad +\Expt{ S^\prime, (x,y), (x^\prime,y^\prime) }{  \left( \error(\wh f  (x^\prime), y^\prime ) -  \error(f_{(i)}(x^\prime), y^\prime ) \right) \error(\wh f ^\prime(x), y ) }  \,, \numberthis \label{eq:term1_final}
    % \end{align*}
    % where $\wh f^\prime$ is the classifier obtained by training on $ S^\prime_{(i)} \cup \{ (x^\prime, y^\prime) \} $. Similarly we can re-write term $\RN{2}$ as 
    % \begin{align*}
    %     & \Expt{ S^\prime }{  \error(f_{(i)}(x_i), y_i )  \error(f_{(j)}(x_j), y_j ) -  \error(\wh f(x), y ) \error(f_{(i)}(x_i), y_i )} \\
    %     &\quad  = \Expt{ S^\prime, (x,y), (x^\prime,y^\prime)}{  \error(f^{\prime\prime}_{(i)}(x), y )  \error(f_{(j)}^{\prime}(x^\prime), y^\prime ) -  \error(\wh f(x), y ) \error(f_{(i)}(x_i), y_i )} \tag{Exchanging $(x_i, y_i)$ with $(x, y)$ and $(x_j, y_j)$ with $(x^\prime, y^\prime)$ in the first term}\\
    %     &\quad = \Expt{ S^\prime, (x,y), (x^\prime,y^\prime)}{  \error(f^{\prime\prime}_{(j)}(x), y )  \error(f_{(i)}^{\prime}(x^\prime), y^\prime ) -  \error(\wh f^\prime (x), y ) \error(f^\prime_{(j)}(x^\prime), y^\prime )} \tag{Exchanging $(x_i, y_i)$ and $(x_j, y_j)$ and then replacing $(x_j, y_j)$ with $(x^\prime, y^\prime)$ in the second term} \\
    %     & \quad = \Expt{ S^\prime, (x,y), (x^\prime,y^\prime) }{  \left( \error(f_{(i)}^{\prime}(x^\prime), y^\prime )   -  \error(\wh f^{\prime\prime}  (x^\prime), y^\prime ) \right)  \error(f^{\prime\prime}_{(j)}(x), y )   } \\
    %     & \quad  + \Expt{ S^\prime, (x,y), (x^\prime,y^\prime) }{  \left( \error(f^{\prime\prime}_{(j)}(x), y )  -\error(\wh f ^\prime(x), y ) \right) \error(\wh f^{\prime\prime}  (x^\prime), y^\prime )  }  \\
    %     & \quad+ \Expt{ S^\prime, (x,y), (x^\prime,y^\prime) }{  \left( \error(\wh f^{\prime\prime}  (x^\prime), y^\prime )  -  \error(f^\prime_{(j)}(x^\prime), y^\prime ) \right)  \error(\wh f^\prime (x), y ) }   \\
    %     & \quad = \Expt{ S^\prime, (x,y), (x^\prime,y^\prime) }{  \left( \error(f_{(i)}^{\prime}(x^\prime), y^\prime )   -  \error(\wh f (x^\prime), y^\prime ) \right)  \error(f_{(i)}(x_j), y_j )   } \\
    %     & \quad  + \Expt{ S^\prime, (x,y), (x^\prime,y^\prime) }{  \left( \error(f^{\prime\prime}_{(j)}(x), y )  -\error(\wh f (x), y ) \right) \error(\wh f^{\prime\prime}  (x_j), y_j )  }  \\
    %     & \quad+ \Expt{ S^\prime, (x,y), (x^\prime,y^\prime) }{  \left( \error(\wh f^{\prime\prime}  (x^\prime), y^\prime )  -  \error(f^\prime_{(j)}(x^\prime), y^\prime ) \right)  \error(\wh f^\prime (x^\prime), y^\prime ) }  \,, \numberthis \label{eq:term2_final}
    % \end{align*}
    % where $f^{\prime\prime}_{(j)}$ is trained on $S^\prime_{(j,i)} \cup {(x,y)}$, $f^{\prime}_{(i)}$ is trained on $S^\prime_{(j,i)} \cup {(x^\prime,y^\prime)}$, and $\wh f^{\prime\prime} $ is trained on $S^\prime_{(j)} \cup {(x,y)}$. Note in the last line we replaced $(x,y)$ by $(x_j, y_j)$ in the first term, replaced $(x^\prime,y^\prime)$ by $(x_j, y_j)$ in the second term and exchanged $(x_i,y_i)$ with $(x_j,y_j)$ and also $(x,y)$ and $(x^\prime, y^\prime)$
    
    
\end{proof}


% 
% 16th Century Version Control 
% 

% \onecolumn

% \section*{Supplementary Material}
% We will be using the following standard results
% on exponential concentration of random variables 
% all throughout the discussion:

% \begin{lemma}[Hoeffding's inequality for independent RVs~\citep{hoeffding1994probability}] Let $Z_1, Z_2, \ldots, Z_n$ be independent bounded random variables with $Z_i \in [a,b]$ for all $i$, then 
%     \begin{align*}
%         \prob\left( \frac{1}{n} \sum_{i=1}^n (Z_i - \Expo{Z_i}) \ge t \right) \le \exp{\left( -\frac{2nt^2}{(b-a)^2} \right) }
%     \end{align*} 
%     and 
%     \begin{align*}
%         \prob\left( \frac{1}{n} \sum_{i=1}^n (Z_i - \Expo{Z_i}) \le -t \right) \le \exp{\left( -\frac{2nt^2}{(b-a)^2} \right) }
%     \end{align*} 
%     for all $t \ge 0$. 
% \end{lemma}

% \begin{lemma}[Hoeffding's inequality for sampling with replacement~\citep{hoeffding1994probability}] \label{lem:hoeffding_sampling} Let $\calZ = (Z_1, Z_2, \ldots, Z_N)$ be a finite population of $N$ points with $Z_i \in [a.b]$ for all $i$. Let $X_1, X_2, \ldots X_n$ be a random sample drawn without replacement from $\calZ$. Then for all $t \ge 0$, we have 
%     \begin{align*}
%         \prob\left( \frac{1}{n} \sum_{i=1}^n (X_i - \mu ) \ge t \right) \le \exp{\left( -\frac{2nt^2}{(b-a)^2} \right) }
%     \end{align*} 
%     and 
%     \begin{align*}
%         \prob\left( \frac{1}{n} \sum_{i=1}^n (X_i - \mu ) \le -t \right) \le \exp{\left( -\frac{2nt^2}{(b-a)^2} \right) } \,,
%     \end{align*} 
%     where $\mu = \frac{1}{N} \sum_{i=1}^{N} Z_i$. 
% \end{lemma}

% We now discuss one condition that generalizes the exponential concentration to dependent random variables.
% \begin{condition}[Bounded difference inequality] \label{cond:BDC} Let $\calZ$ be some set and $\phi: \calZ^n \to \Real$. We say that $\phi$ satisfies the bounded difference assumption if 
% there exists $c_1, c_2, \ldots c_n \ge 0$ s.t. for all $i$, we have 
% \begin{align*}
%     \sup_{Z_1,Z_2, \ldots,Z_n, Z_i^\prime in \calZ^{n+1} } \abs{\phi (Z_1, \ldots, Z_i, \ldots, Z_n ) - \phi (Z_1, \ldots, Z_i^\prime, \ldots, Z_n ) } \le c_i \,.
% \end{align*} 
% \end{condition}

% \begin{lemma}[McDiarmid’s inequality~\citep{mcdiarmid1989}] \label{lem:McDiarmid} Let $Z_1, Z_2, \ldots, Z_n$ be independent random variables on set $\calZ$ and $\phi : \calZ^n \to \Real$ satisfy bounded difference assumption (\codref{cond:BDC}). Then for all $t>0$, we have 
%     \begin{align*}
%         \prob\left( \phi(Z_1, Z_2, \ldots, Z_n) - \Expo{\phi(Z_1, Z_2, \ldots, Z_n)} \ge t \right) \le \exp{\left( -\frac{2t^2}{\sum_{i=1}^n c_i^2} \right) } 
%     \end{align*} 
%     and 
%     \begin{align*}
%         \prob\left( \phi(Z_1, Z_2, \ldots, Z_n) - \Expo{\phi(Z_1, Z_2, \ldots, Z_n)} \le -t \right) \le \exp{\left( -\frac{2t^2}{\sum_{i=1}^n c_i^2} \right) } \,
%     \end{align*} 
% \end{lemma}


% \section{Proofs from \secref{sec:ERM_training}}\label{app:proof_erm}

% \textbf{Additional notation {} {}} Let $m_1$ be the number of mislabeled points ($\wt S_M$) and $m_2$ be the number of correctly labeled points ($\wt S_C$). Note $m_1 + m_2 = m$. 


% \subsection{Proof of \thmref{thm:error_ERM}}


% \begin{proof}[Proof of \lemref{lem:fit_mislabeled}] 
%     The main idea of our proof is to regard 
%     the clean portion of the data 
%     ($S \cup \wt S_C$) as fixed.   
%     Then, there exists a classifier $f^*$ 
%     that is optimal over draws 
%     of the mislabeled data $\wt S_M$. 
% % 
%     % 
%     Formally, 
%     \begin{align}
%     f^* \defeq \argmin_{f \in \calF} \error_{\widecheck {\calD}} (f) \,, \label{eq:modified_ERM}
%     \end{align}
%     where $$\widecheck \calD = \frac{n}{m+n} \calS + \frac{m_1}{m+n} \wt \calS_C  + \frac{m_2}{m+n}\calDm \,.$$ That is, $\widecheck \calD$ a combination of 
%     the \emph{empirical distribution} 
%     over correctly labeled data $S \cup \wt S_C$
%     % in $S\cup \wt S$ 
%     and the (population) distribution 
%     over mislabeled data $\calDm$.
%     Recall that 
%     \begin{align}
%     \wh f \defeq \argmin_{f \in \calF} \error_{\calS \cup \wt S} (f) \,. \label{eq:orig_ERM}
%     \end{align}
%     % 
%     % 
%     Since, $\widehat f$ minimizes 0-1 error 
%     on $S \cup \wt S$, using ERM optimality on \eqref{eq:orig_ERM},  
%     we have 
%     \begin{align}
%         \error_{\calS \cup \wt \calS}(\widehat f) \le \error_{
%             \calS \cup \wt \calS}(f^*) \,.    \label{eq:step1}
%     \end{align}
%     Moreover, since $f^*$ is independent of $\wt S_M$, using Hoeffding's bound,
%     % \footnote{For a fully rigorous argument,
%     % refer to the complete proof in App.~\ref{app:proof_erm}.} 
%     we have with probability at least $1-\delta$ that
%     \begin{align}
%       \error_{\wt \calS_M}(f^*) \le \error_{ \calDm}(f^*) +  \sqrt{\frac{\log(1/\delta)}{2 m_1}} \,. \label{eq:step2} 
%     \end{align}
%     %$ 
%     %for some constant $c_1\le 1/2$. 
%     Finally, since $f^*$ is the optimal classifier on $\widecheck \calD$, 
%     we have 
%     \begin{align}
%         \error_{\widecheck \calD}(f^*) \le \error_{\widecheck \calD}(\widehat f) \label{eq:step3}
%     \end{align}
%      Now to relate \eqref{eq:step1} and \eqref{eq:step3}, we can re-write the \eqref{eq:step2} as follows: 
%     \begin{align}
%         \error_{\calS \cup \wt\calS}(f^*) \le \error_{ \widecheck \calD}(f^*) +  \frac{m_1}{m+n}\sqrt{\frac{\log(1/\delta)}{2 m_1}} \,. \label{eq:step4} 
%     \end{align}
%     Now we combine equations \eqref{eq:step1}, \eqref{eq:step4}, and \eqref{eq:step3}, to get 
%     \begin{align}
%         \error_{\calS \cup \wt \calS}(\wh f) \le \error_{\widecheck \calD}(\wh f) +  \frac{m_1}{m+n}\sqrt{\frac{\log(1/\delta)}{2 m_1}} \,, 
%     \end{align}
%     which implies 
%     \begin{align}
%         \error_{ \wt \calS_M}(\wh f) \le \error_{\calDm}(\wh f) + \sqrt{\frac{\log(1/\delta)}{2 m_1}} \,. \label{eq:lemma1_final}
%     \end{align}
%     Since $\wt S$ is obtained by randomly labeling an unlabeled dataset, we assume $2m_1 \approx m$ \footnote{Formally, with probability at least $1-\delta$, we have  $(m - 2m_1)\le \sqrt{m\log(1/\delta)/2}$ }. Moreover, using $\error_{\calDm} = 1 - \error_{\calD}$ we obtain the desired result.   
%     % Combining the above steps and using the fact 
%     % that $\error_\calD = 1- \error_{\calDm} $, 
%     % we obtain the desired result.
% \end{proof}

% \begin{proof}[Proof of \lemref{lem:mislabeled_error}]
%     Recall $\error_{\wt S} (f) = \frac{m_1}{m} \error_{\wt S_M}(f) + \frac{m_2}{m} \error_{\wt S_C}(f)$. Hence, we have 
%     \begin{align}
%         2\error_{\wt S}(f) - \error_{\wt S_M}(f) - \error_{\wt S_C}(f) &= \left(\frac{2m_1}{m} \error_{\wt S_M}(f) - \error_{\wt S_M}(f)\right) + \left(\frac{2m_2}{m} \error_{\wt S_C}(f) - \error_{\wt S_C}(f)\right) \\ &= \left(\frac{2m_1}{m} - 1\right) \error_{\wt S_M}(f) + \left(\frac{2m_2}{m} - 1 \right)\error_{\wt S_C} (f) \,.
%     \end{align} 
%     Since the dataset is randomly labeled, with probability at least $1-\delta$, we have  $\left(\frac{2m_1}{m} - 1\right) \le \sqrt{\frac{\log(1/\delta)}{2m}}$. Similarly, we have with probability at least $1-\delta$, $\left(\frac{2m_2}{m} - 1\right) \le \sqrt{\frac{\log(1/\delta)}{2m}}$. Using union bound, we have with probability at least $1-\delta$
%     % \begin{align}
%     %     2\error_{\wt S} - \error_{\wt S_M}(f) - \error_{\wt S_C}(f) \le \sqrt{\frac{\log(2/\delta)}{2m}} \left(\error_{\wt S_M}(f) + \error_{\wt S_C}(f) \right) \le 2\sqrt{\frac{\log(2/\delta)}{2m}} \,. \label{eq:lemma2_final}
%     % \end{align}
%     \begin{align}
%         2\error_{\wt S} - \error_{\wt S_M}(f) - \error_{\wt S_C}(f) \le \sqrt{\frac{\log(2/\delta)}{2m}} \left(\error_{\wt S_M}(f) + \error_{\wt S_C}(f) \right) \,. \label{eq:lemma2_prefinal}
%     \end{align}
%     With re-arranging $\error_{\wt S_M}(f) + \error_{\wt S_C}(f)$ and using the inequality $ 1- a\le \frac{1}{1+a} $, we have  
%     \begin{align}
%         2\error_{\wt S} - \error_{\wt S_M}(f) - \error_{\wt S_C}(f) \le 2\error_{\wt \calS} \sqrt{\frac{\log(2/\delta)}{2m}}  \,. \label{eq:lemma2_final}
%     \end{align}

%     % We obtain the desired result by using 
% \end{proof}

% \begin{proof}[Proof of \lemref{lem:clear_error}]
% % Recall 0-1 error on each point  $(x,y) \in S \cup \wt S$ is given by $\I{ f(x)\ne y}$.
% In the set of correctly labeled points $S \cup \wt S_C$, we have $S$ as a random subset of $S \cup \wt S_C$. Hence, using Hoeffding's inequality for sampling without replacement (\lemref{lem:hoeffding_sampling}), we have with probability at least $1-\delta$
% \begin{align}
%     \error_{\wt \calS_c} (\wh f)- \error_{\calS \cup \wt \calS_C}( \wh f) \le  \sqrt{\frac{\log(1/\delta)}{2m_2}} \,.
% \end{align}
% Re-writing $\error_{\calS \cup \wt \calS_C}( \wh f)$ as $\frac{m_2}{m_2 + n} \error_{\wt \calS_C }(\wh f) + \frac{n}{m_2 + n} \error_{\calS }(\wh f)$, we have with probability at least $1-\delta$
% \begin{align}
%   \left(\frac{n}{n+m_2}\right) \left(\error_{\wt \calS_c} (\wh f)- \error_{\calS}( \wh f) \right) \le  \sqrt{\frac{\log(1/\delta)}{2m_2}} \,.
% \end{align}
% As before, assuming $2m_2 \approx m$, we have with probability at least $1-\delta$ 
% \begin{align}
%     \error_{\wt \calS_c} (\wh f)- \error_{\calS}( \wh f) \le \left(1+\frac{m_2}{n}\right)  \sqrt{\frac{\log(1/\delta)}{m}} \le 1.5 \sqrt{\frac{\log(1/\delta)}{m}} \,. \label{eq:lemma3_final}
% \end{align} 
% \end{proof}

% \begin{proof}[Proof of \thmref{thm:error_ERM}] 
%     Having established these core intermediate results, we can now combine above three lemmas to prove the main result. 
%     In particular, we bound the population error on clean data ($\error_\calD(\wh f)$) as follows:  
%     \begin{enumerate}[(i)]
%         \item First, use \eqref{eq:lemma1_final}, to obtain an upper bound on the population error on clean data, i.e., with probability at least $1-\delta/4$, we have
%         \begin{align}
%             \error_{ \calD} (\wh f) \le 1 - \error_{ \wt \calS_M}(\wh f) + \sqrt{\frac{\log(4/\delta)}{m}} \,. 
%         \end{align}
%         \item  Second, use \eqref{eq:lemma2_final}, to relate the error on the mislabeled fraction with error on clean portion of randomly labeled data and error on whole randomly labeled dataset, i.e., with probability at least $1-\delta/2$, we have 
%         \begin{align}
%             - \error_{\wt S_M}(f) \le \error_{\wt S_C}(f) - 2\error_{\wt S}  + \sqrt{\frac{\log(4/\delta)}{2m}}  \,. 
%         \end{align} 
%         \item Finally, use \eqref{eq:lemma3_final} to relate the error on the clean portion of randomly labeled data and error on clean training data, i.e., with probability $1-\delta/4$, we have 
%         \begin{align}
%             \error_{\wt \calS_C} (\wh f)\le - \error_{\calS}( \wh f) + \left(1 + \frac{m}{2n} \right) \sqrt{\frac{\log(4/\delta)}{m}} \,. 
%         \end{align} 
%     \end{enumerate}

%     Using union bound on the above three steps, we have with probability at least $1-\delta$: 
%     \begin{align}
%         \error_\calD (\wh f) \le \error_{\calS}(\wh f)   + 1 - 2\error_{\wt \calS}(\wh f)   + (1/\sqrt{2} + 2.5)  \sqrt{\frac{\log(4/\delta)}{m}} \,.
%     \end{align}
%     Note that $(1/\sqrt{2} + 2.5)$ is a loose constant. In experiments, we use the ratio $\frac{m}{n}$
%     %  the exact error $\error_{\wt \calS}(\wh f)$ 
%     to evaluate R.H.S.    
% \end{proof}

% \subsection{Proof of \propref{prop:rademacher}}

% \begin{proof}[Proof of \propref{prop:rademacher}]
%     For a classifier $ f: \calX \to \{-1, 1\}$, we have $1 - 2\,\indict{ f(x) \ne y} = y \cdot f(x)$. Hence, by definition of $\error$, we have 
%     \begin{align}
%         1 -2\error_{\wt \calS}(f) = \frac{1}{m}\sum_{i=1}^m y_i \cdot f(x_i) \le \sup_{f \in \calF} \, \frac{1}{m} \sum_{i=1}^m y_i \cdot f(x_i)  \,. \label{eq:error_rademacher}
%     \end{align}
%     Note that for fixed inputs $(x_1, x_2, \ldots, x_m)$ in $\wt S$, $(y_1, y_2, \ldots y_m)$ are random labels. Define $\phi_1 (y_1, y_2, \ldots, y_m) \defeq \sup_{f \in \calF} \, \frac{1}{m} \sum_{i=1}^m y_i \cdot f(x_i)$. We have the following bounded difference condition on $\phi_1$. For all i, 
%     \begin{align}
%         \sup_{y_1, \ldots y_m, y_i^\prime \in \{-1, 1\}^{m+1} } \abs{ \phi_1 (y_1,\ldots, y_i, \ldots, y_m) - \phi_1 (y_1,\ldots, y_i^\prime, \ldots, y_m)  } \le 1/m \,. \label{cond1_rademacher}
%     \end{align} 
    
%     Similarly define $\phi_2 (x_1, x_2, \ldots, x_m) \defeq \Expt{ y_i \sim_U \{-1, 1\}  }{ \sup_{f \in \calF} \, \frac{1}{m}  \sum_{i=1}^m y_i \cdot f(x_i)}$. We have the following bounded difference condition on $\phi_2$. For all i,
%     \begin{align}
%         \sup_{x_1, \ldots x_m, x_i^\prime \in \calX^{m+1} } \abs{ \phi_2 (x_1,\ldots, x_i, \ldots, x_m) - \phi_1 (x_1,\ldots, x_i^\prime, \ldots, x_m)  } \le 1/m \,. \label{cond2_rademacher}
%     \end{align}
%     Using McDiarmid’s inequality (\lemref{lem:McDiarmid}) twice with Condition \eqref{cond1_rademacher} and \eqref{cond2_rademacher}, with probability at least $1-\delta$, we have
%     \begin{align}
%         \sup_{f \in \calF} \, \frac{1}{m} \sum_{i=1}^m y_i \cdot f(x_i)  - \Expt{x,y}{\sup_{f \in \calF} \, \frac{1}{m} \sum_{i=1}^m y_i \cdot f(x_i) } \le \sqrt{\frac{2\log(2/\delta)}{m}} \label{eq:final_rademacher}
%     \end{align} 
%     Combining \eqref{eq:error_rademacher} and \eqref{eq:final_rademacher}, we obtain the desired result. 
% \end{proof}


% \subsection{Proof of \thmref{thm:error_regularized_ERM}}

% Proof of \thmref{thm:error_regularized_ERM} follows similar to the proof of \thmref{thm:error_ERM}. Note that the same results in \lemref{lem:fit_mislabeled}, \lemref{lem:mislabeled_error}, and \lemref{lem:clear_error} hold in the regularized ERM case. However, the arguments in the proof of \lemref{lem:fit_mislabeled} changes slightly. Hence, we state and prove a lemma parallel to \lemref{lem:fit_mislabeled} for completeness. 

% \begin{lemma} \label{lem:lemma1_reg}
%     Assume the same setup as \thmref{thm:error_regularized_ERM}. 
%     Then for any $\delta >0$, with probability at least  $1-\delta$ 
%     over the random draws of mislabeled data $\wt S_M$, we have 
%     \begin{align}
%         \error_\calD(\widehat f)  \le 1 -\error_{\wt \calS_M}(\widehat f) + \sqrt{\frac{\log(1/\delta)}{m}}\,. 
%     \end{align} 
% \end{lemma}
% \begin{proof}
%     The main idea of the proof remains the same, i.e. regard 
%     the clean portion of the data 
%     ($S \cup \wt S_C$) as fixed.   
%     Then, there exists a classifier $f^*$ 
%     that is optimal over draws 
%     of the mislabeled data $\wt S_M$. 

    
%     Formally, 
%     \begin{align}
%     f^* \defeq \argmin_{f \in \calF} \error_{\widecheck {\calD}} (f)  + \lambda R(f) \,, \label{eq:modified_ERM_reg}
%     \end{align}
%     where $$\widecheck \calD = \frac{n}{m+n} \calS + \frac{m_1}{m+n} \wt \calS_C  + \frac{m_2}{m+n}\calDm \,.$$ That is, $\widecheck \calD$ a combination of 
%     the \emph{empirical distribution} 
%     over correctly labeled data $S \cup \wt S_C$
%     % in $S\cup \wt S$ 
%     and the (population) distribution 
%     over mislabeled data $\calDm$.
%     Recall that 
%     \begin{align}
%     \wh f \defeq \argmin_{f \in \calF} \error_{\calS \cup \wt S} (f) + \lambda R(f) \,. \label{eq:orig_ERM_reg}
%     \end{align}
%     % 
%     % 
%     Since, $\widehat f$ minimizes 0-1 error 
%     on $S \cup \wt S$, using ERM optimality on \eqref{eq:orig_ERM},  
%     we have 
%     \begin{align}
%         \error_{\calS \cup \wt \calS}(\widehat f) + \lambda R(\wh f) \le \error_{
%             \calS \cup \wt \calS}(f^*) + \lambda R(f^*) \,.    \label{eq:step1_reg}
%     \end{align}
%     Moreover, since $f^*$ is independent of $\wt S_M$, using Hoeffding's bound,
%     % \footnote{For a fully rigorous argument,
%     % refer to the complete proof in App.~\ref{app:proof_erm}.} 
%     we have with probability at least $1-\delta$ that
%     \begin{align}
%       \error_{\wt \calS_M}(f^*) \le \error_{ \calDm}(f^*) +  \sqrt{\frac{\log(1/\delta)}{2 m_1}} \,. \label{eq:step2_reg} 
%     \end{align}
%     %$ 
%     %for some constant $c_1\le 1/2$. 
%     Finally, since $f^*$ is the optimal classifier on $\widecheck \calD$, 
%     we have 
%     \begin{align}
%         \error_{\widecheck \calD}(f^*) + \lambda R(f^*) \le \error_{\widecheck \calD}(\widehat f) + \lambda R(\wh f) \label{eq:step3_reg}
%     \end{align}
%      Now to relate \eqref{eq:step1_reg} and \eqref{eq:step3_reg}, we can re-write the \eqref{eq:step2_reg} as follows: 
%     \begin{align}
%         \error_{\calS \cup \wt\calS}(f^*) \le \error_{ \widecheck \calD}(f^*) +  \frac{m_1}{m+n}\sqrt{\frac{\log(1/\delta)}{2 m_1}} \,. \label{eq:step4_reg} 
%     \end{align}
%     After adding $\lambda R(f^*)$ on both sides in \eqref{eq:step4_reg}, we combine equations \eqref{eq:step1_reg}, \eqref{eq:step4_reg}, and \eqref{eq:step3_reg}, to get 
%     \begin{align}
%         \error_{\calS \cup \wt \calS}(\wh f) \le \error_{\widecheck \calD}(\wh f) +  \frac{m_1}{m+n}\sqrt{\frac{\log(1/\delta)}{2 m_1}} \,, 
%     \end{align}
%     which implies 
%     \begin{align}
%         \error_{ \wt \calS_M}(\wh f) \le \error_{\calDm}(\wh f) + \sqrt{\frac{\log(1/\delta)}{2 m_1}} \,. \label{eq:lemma_reg_final}
%     \end{align}
%     Similar as before, since $\wt S$ is obtained by randomly labeling an unlabeled dataset, we assume 
%     $2m_1 \approx m$. Moreover, using $\error_{\calDm} = 1 - \error_{\calD}$ we obtain the desired result. 
% \end{proof}
% % \begin{proof}[Proof of ]
    
% % \end{proof}

% \subsection{Proof of \thmref{thm:multiclass_ERM}}

% We first state and prove lemmas parallel to three lemmas used in the proof of balanced binary case. Then we combine the results in the three lemmas to obtain the result in \thmref{thm:multiclass_ERM}. 

% Before stating the result, we define mislabeled distribution $\calDm$ for any $\calD$. While $\calDm$ and $\calD$ share 
% the same marginal distribution over $\calX$, 
% the distribution over labels $y$ 
% given an input $x\sim \calD_\calX$ is changed.
% In particular, for any $x$, the pdf over $y$ is changed to:  
% $p_{\calDm} (\cdot \vert x) \defeq \frac{1 - p_{\calD}(\cdot \vert x)}{k - 1}$.

% \begin{lemma} \label{lem:fit_mislabeled_multi}
%     Assume the same setup as \thmref{thm:multiclass_ERM}. 
%     Then for any $\delta >0$, with probability at least  $1-\delta$ 
%     over the random draws of mislabeled data $\wt S_M$, we have 
%     \begin{align}
%         \error_\calD(\widehat f)  \le (k-1)\left(1 -\error_{\wt \calS_M}(\widehat f)\right) + (k-1)\sqrt{\frac{\log(1/\delta)}{m}}\,. \label{eq:lemma1_multi}
%     \end{align}   
% \end{lemma} 

% \begin{proof}
%     The main idea of the proof remains the same, i.e. regard 
%     the clean portion of the data 
%     ($S \cup \wt S_C$) as fixed. 
%     Then, there exists a classifier $f^*$ 
%     that is optimal over draws 
%     of the mislabeled data $\wt S_M$. 
    
%     However, we need to be careful while relating population error on mislabeled data with population accuracy on clean data.   
%     While for binary classification,  we could upper bound $\error_{\wt \calS_M}$ 
%     with $1-\error_\calD$  (in the proof of \lemref{lem:fit_mislabeled}), 
%     for multiclass classification, 
%     error on the mislabeled data 
%     and accuracy on the clean data 
%     in the population 
%     are not so directly related.  
%     To establish \eqref{eq:lemma1_multi},
%     we break the error on the 
%     (unknown) mislabeled data 
%     into two parts: one term corresponds 
%     to predicting the true label on mislabeled data, 
%     and the other corresponds to predicting 
%     neither the true label 
%     nor the assigned (mis-)label.  
%     Finally, we relate these errors to their
%     population counterparts to establish \eqref{eq:lemma1_multi}. 
    
%     Formally, 
%     \begin{align}
%     f^* \defeq \argmin_{f \in \calF} \error_{\widecheck {\calD}} (f)  + \lambda R(f) \,, \label{eq:modified_ERM_reg2}
%     \end{align}
%     where $$\widecheck \calD = \frac{n}{m+n} \calS + \frac{m_1}{m+n} \wt \calS_C  + \frac{m_2}{m+n}\calDm \,.$$ That is, $\widecheck \calD$ a combination of 
%     the \emph{empirical distribution} 
%     over correctly labeled data $S \cup \wt S_C$
%     % in $S\cup \wt S$ 
%     and the (population) distribution 
%     over mislabeled data $\calDm$.
%     Recall that 
%     \begin{align}
%     \wh f \defeq \argmin_{f \in \calF} \error_{\calS \cup \wt S} (f) + \lambda R(f) \,. \label{eq:orig_ERM_reg2}
%     \end{align}
%     % 
%     % 
%     Following the exact steps from the proof of \lemref{lem:lemma1_reg}, with probability at least $1-\delta$, we have  
%     \begin{align}
%         \error_{ \wt \calS_M}(\wh f) \le \error_{\calDm}(\wh f) + \sqrt{\frac{\log(1/\delta)}{2 m_1}} \,. \label{eq:lemma1_final_multi_prev}
%     \end{align}
%     Similar to before, since $\wt S$ is obtained by randomly labeling an unlabeled dataset, we assume 
%     $\frac{k}{k-1} m_1 \approx m$. 
    
%     Now we will relate $\error_\calDm (\wh f)$ with $\error_{\calD}(\wh f)$. Let $y^T$ denote the (unknown) true label for a mislabeled point $(x, y)$ (i.e., label before replacing it with a mislabel). 
%     \begin{align}    
%          \Expt{(x, y) \in \sim \calDm}{\indict{ \wh f(x) \ne y }}  &= \underbrace{\Expt{(x, y) \in \sim \calDm}{\indict{ \wh f(x) \ne y \land \wh f(x) \ne y^T}}}_{\RN{1}} + \underbrace{\Expt{(x, y) \in \sim \calDm}{\indict{ \wh f(x) \ne y \land \wh f(x) = y^T}}}_{\RN{2}} \,. \label{eq:excess_term}
%     \end{align}
%     Clearly, term 2 is one minus the accuracy on the clean unseen data, i.e. 
%     \begin{align}
%         \RN{2} = 1 - \Expt{{x,y} \sim \calD}{ \indict{ \wh f(x) \ne y}} = 1- \error_{\calD}(\wh f) \,. \label{eq:term1}    
%     \end{align}
%     Next, we  relate term 1 with the error on the unseen clean data. We show that term 1 is equal to the error on the unseen clean data scaled by $\frac{k-2}{k-1}$ where $k$ is the number of labels. Using the definition of mislabeled distribution $\calDm$,  we have 
%     \begin{align}
%         \RN{1} = \frac{1}{k-1} \left( \Expt{(x, y) \in \sim \calD}{ \sum_{i \in \calY \land i\ne y}  \indict{ \wh f(x) \ne i \land \wh f(x) \ne y}} \right) = \frac{k-2}{k-1} \error_{\calD}(\wh f) \,.\label{eq:term2}
%     \end{align}    

%     Combining the result in \eqref{eq:term1}, \eqref{eq:term2} and \eqref{eq:excess_term}, we have 
%     \begin{align}
%         \error_{\calDm}(\wh f) = 1- \frac{1}{k-1} \error_{\calD}(\wh f) \,.\label{eq:combine_terms}
%     \end{align}
%     Finally, combining the result in \eqref{eq:combine_terms} with equation \eqref{eq:lemma1_final_multi_prev}, we have with probability $1-\delta$, 
%     \begin{align}
%       \error_{\calD}(\wh f) \le  (k-1) \left( 1- \error_{ \wt \calS_M}(\wh f) \right)  + (k-1) \sqrt{\frac{k \log(1/\delta)}{ 2(k-1)m}} \,. \label{eq:lemma1_final_multi}
%     \end{align}
% \end{proof}

% \begin{lemma} \label{lem:mislabeled_error_multi}
%     Assume the same setup as \thmref{thm:multiclass_ERM}.  Then for any $\delta >0$, with probability at least $1-\delta$ over the random draws of $\wt S$, we have  
%     % \begin{align}
%         $$\abs{k\error_{\wt \calS}(\widehat f) - \error_{\wt \calS_C}(\widehat f) -  (k-1)\error_{\wt \calS_M}(\widehat f) } \le  2k\sqrt{\frac{\log(4/\delta)}{2m}}\,. $$ % \label{eq:lemma2}
%     % \end{align}   
%     %  for some constant $c_3 \le 1.0\,$.
% \end{lemma} 


% \begin{proof}
%     Recall $\error_{\wt S} (f) = \frac{m_1}{m} \error_{\wt S_M}(f) + \frac{m_2}{m} \error_{\wt S_C}(f)$. Hence, we have 
%     \begin{align}
%         k\error_{\wt S}(f) - (k-1)\error_{\wt S_M}(f) - \error_{\wt S_C}(f) &= (k-1)\left(\frac{k m_1}{(k-1) m} \error_{\wt S_M}(f) - \error_{\wt S_M}(f)\right) + \left(\frac{km_2}{m} \error_{\wt S_C}(f) - \error_{\wt S_C}(f)\right) \\ &= k \left[ \left(\frac{m_1}{m} - \frac{k-1}{k}\right) \error_{\wt S_M}(f) + \left(\frac{m_2}{m} - \frac{1}{k} \right) \error_{\wt S_C} (f) \right] \,.
%     \end{align} 
%     Since the dataset is randomly labeled, we have with probability at least $1-\delta$, $\left(\frac{m_1}{m} - \frac{k-1}{k}\right) \le \sqrt{\frac{\log(1/\delta)}{2m}}$. Similarly, we have with probability at least $1-\delta$, $\left(\frac{m_2}{m} - \frac{1}{k}\right) \le \sqrt{\frac{\log(1/\delta)}{2m}}$. Using union bound, we have with probability at least $1-\delta$
%     % \begin{align}
%     %     2\error_{\wt S} - \error_{\wt S_M}(f) - \error_{\wt S_C}(f) \le \sqrt{\frac{\log(2/\delta)}{2m}} \left(\error_{\wt S_M}(f) + \error_{\wt S_C}(f) \right) \le 2\sqrt{\frac{\log(2/\delta)}{2m}} \,. \label{eq:lemma2_final}
%     % \end{align}
%     \begin{align}
%         k\error_{\wt S}(f) - (k-1)\error_{\wt S_M}(f) - \error_{\wt S_C}(f)  \le k \sqrt{\frac{\log(2/\delta)}{2m}} \left(\error_{\wt S_M}(f) + \error_{\wt S_C}(f) \right) \,. \label{eq:lemma2_final_multi}
%     \end{align}

%     % We obtain the desired result by using 
% \end{proof}

% \begin{lemma} \label{lem:clear_error_multi}
%     Assume the same setup as \thmref{thm:multiclass_ERM}. 
%     Then for any $\delta >0$, with probability at least $1-\delta$ 
%     over the random draws of $\wt S_C$ and $S$, we have 
%     % \begin{align}
%         $$\abs{\error_{\wt \calS_C}(\widehat f) - \error_{\calS}(\widehat f) } \le 1.5 \sqrt{\frac{k\log(2/\delta)}{2m}}\,.$$ %\label{eq:lemma3}
%     % \end{align}   
%     % for some constant $c_2 \le 1.2\,$.
% \end{lemma} 
% \begin{proof}
%     % Recall 0-1 error on each point  $(x,y) \in S \cup \wt S$ is given by $\I{ f(x)\ne y}$.
%     In the set of correctly labeled points $S \cup \wt S_C$, we have $S$ as a random subset of $S \cup \wt S_C$. Hence, using Hoeffding's inequality for sampling without replacement (\lemref{lem:hoeffding_sampling}), we have with probability at least $1-\delta$
%     \begin{align}
%         \error_{\wt \calS_c} (\wh f)- \error_{\calS \cup \wt \calS_C}( \wh f) \le  \sqrt{\frac{\log(1/\delta)}{2m_2}} \,.
%     \end{align}
%     Re-writing $\error_{\calS \cup \wt \calS_C}( \wh f)$ as $\frac{m_2}{m_2 + n} \error_{\wt \calS_C }(\wh f) + \frac{n}{m_2 + n} \error_{\calS }(\wh f)$, we have with probability at least $1-\delta$
%     \begin{align}
%       \left(\frac{n}{n+m_2}\right) \left(\error_{\wt \calS_c} (\wh f)- \error_{\calS}( \wh f) \right) \le  \sqrt{\frac{\log(1/\delta)}{2m_2}} \,.
%     \end{align}
%     As before, assuming $km_2 \approx m$, we have with probability at least $1-\delta$ 
%     \begin{align}
%         \error_{\wt \calS_c} (\wh f)- \error_{\calS}( \wh f) \le \left(1+\frac{m_2}{n}\right)  \sqrt{\frac{k\log(1/\delta)}{2m}} \le \left( 1 + \frac{1}{k}\right) \sqrt{\frac{k\log(1/\delta)}{2m}} \,. \label{eq:lemma3_final_multi}
%     \end{align} 
% \end{proof}

% \begin{proof}[Proof of \thmref{thm:multiclass_ERM}] 
%     Having established these core intermediate results, we can now combine above three lemmas. 
%     In particular, we bound the population error on clean data ($\error_\calD(\wh f)$) as follows:  
%     \begin{enumerate}[(i)]
%         \item First, use \eqref{eq:lemma1_final_multi}, to obtain an upper bound on the population error on clean data, i.e., with probability at least $1-\delta/4$, we have
%         \begin{align}
%             \error_{ \calD} (\wh f) \le (k-1)\left(1 - \error_{ \wt \calS_M}(\wh f) \right) + (k-1) \sqrt{\frac{k\log(4/\delta)}{2(k-1)m}} \,. 
%         \end{align}
%         \item  Second, use \eqref{eq:lemma2_final_multi}, to relate the error on the mislabeled fraction with error on clean portion of randomly labeled data and error on whole randomly labeled dataset, i.e., with probability at least $1-\delta/2$, we have 
%         \begin{align}
%             - (k-1)\error_{\wt S_M}(f) \le \error_{\wt S_C}(f) - k\error_{\wt S}  + k\sqrt{\frac{\log(4/\delta)}{2m}}  \,. 
%         \end{align} 
%         \item Finally, use \eqref{eq:lemma3_final_multi} to relate the error on the clean portion of randomly labeled data and error on clean training data, i.e., with probability $1-\delta/4$, we have 
%         \begin{align}
%             \error_{\wt \calS_C} (\wh f)\le - \error_{\calS}( \wh f) + \left(1 + \frac{m}{kn} \right) \sqrt{\frac{k\log(4/\delta)}{2m}} \,. 
%         \end{align} 
%     \end{enumerate}

%     Using union bound on the above three steps, we have with probability at least $1-\delta$: 
%     \begin{align}
%         \error_\calD (\wh f) \le \error_{\calS}(\wh f) + (k-1) - k\error_{\wt \calS}(\wh f)   + (\sqrt{k(k-1)} + k + \sqrt{k} + \frac{m}{n\sqrt{k}})  \sqrt{\frac{\log(4/\delta)}{2m}} \,.
%     \end{align}
%     % Note that $\frac{m}{n\sqrt{k}}$ is much smaller than the other terms in addition. Hence, we ignore this in the final bound. 
%     % Note that $(1/\sqrt{2} + 2.5)$ is a loose constant. In experiments, we use the ratio $\frac{m}{n}$
%     %  the exact error $\error_{\wt \calS}(\wh f)$ 
%     % to evaluate R.H.S.    
% \end{proof}

% \newpage
% \section{Proofs from \secref{sec:linear_models}}\label{app:proof_gd}

% We suppose that the parameters of the linear function 
% are obtained via gradient descent on 
% the following $L_2$ regularized problem: 
% \begin{align}
%     % n in denominator is avoided deliberately
%     \calL_S(w; \lambda) \defeq \sum_{i=1}^n{(w^Tx_i - y_i)^2} + \lambda \norm{w}{2}^2 \,, \label{eq:l2_MSE_app}   
% \end{align}
% where $\lambda\ge0$ is a regularization parameter. 
% We assume access to a clean dataset 
% $S = \{(x_i, y_i)\}_{i=1}^n \sim \calD^n$ 
% and randomly labeled dataset 
% $\wt S = \{(x_i, y_i)\}_{i=n+1}^{n+m} \sim \wt \calD^m$. 
% Let $\bX = [x_1, x_2, \cdots, x_{m+n}]$ 
% and $\by = [y_1, y_2, \cdots, y_{m+n}]$. 
% Fix a positive learning rate $\eta$ such that 
% $\eta \le 1/\left(\norm{\bX^T\bX}{\text{op}} + \lambda^2\right)$ 
% and an initialization $w_0 = 0$. 
% % \todos{Assumption made for simplicty}. 
% Consider the following gradient descent iterates 
% to minimize objective \eqref{eq:l2_MSE_app} on $S \cup \wt S$:
% \begin{align}
% w_t = w_{t-1} - \eta \grad_w \calL_{S \cup \wt S} (w_{t-1}; \lambda) \quad \forall t=1,2,\ldots \label{eq:GD_iterates_app}
% \end{align} 
% Then we have $\{ w_t\}$ converge to the limiting solution 
% $\wh w = \left( \bX^T\bX+\lambda \boldsymbol{I}\right)^{-1}\bX^T\by$. Define $\widehat f (x) \defeq f(x ; \wh w) $.  

% \subsection{\textcolor{red}{Errata}}

% We wish to correct the following error in the body: \codref{cond:error_stability} is not enough to guarantee the result in \thmref{thm:linear}. We now present a slightly stronger condition called \emph{hypothesis stability} under which we obtain a result similar to \thmref{thm:linear}. 

% This error doesn't change the main arguments of the proof where we show that the empirical train error is less than or equal to the leave-one-out error. We need a stronger condition to relate leave-one-out error with the population error of the original classifier. Specifically, while \codref{cond:error_stability} relates the average population error of leave-one-out classifiers with the population error of the original classifier, we need the new condition to show the concentration of the empirical leave-one-out error and  average population error of leave-one-out classifiers. 
% % main takeaway 

% Note that the new condition, while being stronger than the previous one, still doesn't imply generalization~\cite{bousquet2002stability,elisseeff2003leave,abou2019exponential}. Overall, the main results in \secref{sec:ERM_training} and takeaways of the paper remain unaffected by the error.  

% We now present the new condition and a corrected statement of \thmref{thm:linear}. Recall, for a given training set $S \sim \calD^n $, 
% we use $S_{(i)}$ to denote the training set $S$ 
% with the $i^{\text{th}}$ point removed.

% \begin{condition}[Hypothesis Stability] 
%     \label{cond:hypothesis_stability}
%     We have $\beta$ hypothesis stability 
%     if our training algorithm $\calA$ satisfies the following: 
%     \begin{align*}
%     % ${\sum_{i=1}^n \frac{\error_{\calD}( f(\calA, S_{(i)}))}{n} - \error_\calD(f(\calA, S))} \le \beta\,$.
%     \forall i \in \{1,2,\ldots, n\}, \quad  \Expt{\calS, (x,y) \in \calD}{ \abs{\error\left( f(x) ,y  \right) - \error\left( f_{(i)}(x), y \right) }} \le \frac{\beta}{n} \,,
%     \end{align*}
%     where $f_{(i)} \defeq f(\calA, S_{(i)})$ and $ f \defeq f(\calA, S)$.
% \end{condition}

% \begin{theorem}[Correct statement of \thmref{thm:linear}] \label{thm:new_linear}
%     Assume that this gradient descent algorithm satisfies \codref{cond:hypothesis_stability}
%     with $\beta=\calO(1)$.  
%     Then for any $\delta >0$, with probability at least $1-\delta$ 
%     over the random draws of datasets $\wt S$ and $S$, we have:
%     \begin{align}
%         \error_\calD(\widehat f) \le \error_\calS(\widehat f) + 1 - 2 \error_{\wt\calS}(\widehat f) + \left(\frac{1}{\sqrt{2}} + 1.5 \right) \sqrt{\frac{\log(4/\delta)}{m}} + \sqrt{\frac{4}{\delta}\left(\frac{1}{m} +\frac{3\beta}{m+n} \right)}  \,. \label{eq:gd_error}
%     \end{align} 
%     % for some constant $c\le 3.2$.
% \end{theorem}

% \subsection{Proof of \thmref{thm:new_linear}}
% We use a standard result from linear algebra, namely Shermann-Morrison formula~\citep{sherman1950adjustment} for matrix inversion:  

% \begin{lemma}[\citet{sherman1950adjustment}] \label{lem:sherman}
%     Suppose $\bA \in \Real^{n \times n}$ is an invertible square matrix and $u,v \in \Real^n$ are column vectors. Then $\bA + uv^T$ is invertible iff $1 + v^T \bA u \ne 0$ and in particular
%     \begin{align}
%         (\bA + u v^T)^{-1} = \bA^{-1}  - \frac{\bA^{-1} uv^T \bA^{-1} }{ 1 + v^T \bA^{-1} u} \,.
%     \end{align}   
% \end{lemma}
% \newcommand\byy[1]{\by_{\left(#1\right)}}
% \newcommand\bXX[1]{\bX_{\left(#1\right)}}
% \newcommand\ff[1]{\wh f_{\left(#1\right)}}

% For a given training set $S \cup \wt S_C$, define leave-one-out error on mislabeled points in the training data as $$\error_{\text{LOO}(\wt S_M) } = \frac{\sum_{(x_i, y_i) \in \wt S_M} \error( f_{(i)}( x_i), y_i)}{ \abs{\wt S_M }} \,, $$
% where $f_{(i)} \defeq f(\calA, (S \cup \wt S)_{(i)})$. To relate empirical leave-one-out error and population error with hypothesis stability condition, we use the following lemma:   

% \begin{lemma}[\citet{bousquet2002stability}] \label{lem:stability_error}
%     For the leave-one-out error, we have
%     \begin{align}
%         \Expo{ \left( \error_{\calDm}(\wh f) -\error_{\text{LOO}(\wt S_M) } \right)^2 } \le \frac{1}{2m_1}+  \frac{3\beta}{n + m}\,.
%     \end{align}   
%     % where $ f \defeq f(\calA, S \cup \wt S) $.
% \end{lemma}

% Proof of the above lemma is similar to the proof of  Lemma 9 in \citet{bousquet2002stability} and can be found in \appref{app:proof_lem_error}. 
% % 
% % Before presenting the result, we introduce some notation. 
% Before presenting the proof of \thmref{thm:new_linear}, we introduce some more notation. Let $\bX_{(i)}$ denote the matrix of covariates with $i^{\text{th}}$ point removed. Similarly let $\by_{(i)}$ be the array of responses with $i^{\text{th}}$ point removed. Define the corresponding regularized GD solution as $\wh w_{(i)} = \left( \bXX{i}^T\bXX{i}+\lambda \boldsymbol{I}\right)^{-1}\bXX{i}^T\byy{i}$. Define $\ff{i}(x) \defeq f(x ; \wh w_{(i)}) $.

% \begin{proof}[Proof of \thmref{thm:new_linear}]
%     Because squared loss minimization does not imply 0-1 error minimization, we cannot use arguments from \lemref{lem:fit_mislabeled}. This is the main technical difficulty. To compare the 0-1 error at a train point with an unseen point, 
%     we use the closed-form expression for $\widehat{w}$ and Shermann-Morrison formula to upper bound training error with leave-one-out cross validation error. 
    
%     The proof is divided into three parts: In part one, we show that 0-1 error on mislabeled points in the training set is lower than the error obtained by leave-one-out error at those points. In part two, we relate this leave-one-out error with the population error on mislabeled distribution using \codref{cond:hypothesis_stability}. While the empirical leave-one-out error is unbiased estimator of the average population error of leave-one-out classifiers, we need hypothesis stability to control the variance of empirical leave-one-out error. Finally in part three, we show that the error on the mislabeled training points can be estimated with just the randomly labeled and  clean training data (as in proof of \thmref{thm:error_ERM}).  

%     \textbf{Part 1 {} {}} First we relate training error with leave-one-out error.        
%     For any 
%     training point $(x_i, y_i)$ in $\wt S \cup S$, we have 
%     \begin{align}
%         \error(\wh f(x_i), y_i ) &= \indict{ y_i \cdot x_i^T \wh w < 0 } = \indict{ y_i \cdot x_i^T \left( \bX^T\bX+\lambda \boldsymbol{I}\right)^{-1}\bX^T\by < 0 } \\
%         &= \indict{ y_i \cdot x_i^T \underbrace{\left( \bXX{i}^T\bXX{i} + x_i ^T x_i +\lambda \boldsymbol{I}\right)^{-1}}_{\RN{1}} (\bXX{i}^T\byy{i} + y \cdot x_i) < 0 }
%     \end{align}
%     Letting $\bA = \left(\bXX{i}^T\bXX{i} +\lambda \boldsymbol{I}\right)$ and using \lemref{lem:sherman} on term 1, we have 
%     \begin{align}
%         \error(\wh f(x_i), y_i ) &= \indict{ y_i \cdot x_i^T \left[\bA^{-1} -  \frac{\bA^{-1} x_i x_i^T \bA^{-1}}{ 1 + x_i ^T \bA^{-1} x_i } \right] (\bXX{i}^T\byy{i} + y \cdot x_i) < 0 } \\
%         &= \indict{ y_i \cdot\left[ \frac{ x_i^T \bA^{-1} ( 1 + x_i ^T \bA^{-1} x_i ) -  x_i^T \bA^{-1} x_i x_i^T \bA^{-1}}{ 1 + x_i ^T \bA ^{-1}x_i } \right] (\bXX{i}^T\byy{i} + y \cdot x_i) < 0 } \\
%         &= \indict{ y_i \cdot\left[ \frac{ x_i^T \bA^{-1}}{ 1 + x_i ^T \bA ^{-1}x_i } \right] (\bXX{i}^T\byy{i} + y \cdot x_i) < 0 } \,.
%     \end{align}

%     Since $1 + x_i^T \bA^{-1} x_i > 0$, we have 
%     \begin{align}
%         \error(\wh f(x_i), y_i ) &= \indict{ y_i \cdot x_i^T \bA^{-1} (\bXX{i}^T\byy{i} + y \cdot x_i) < 0 } \\
%         &= \indict{ x_i^T \bA^{-1} x_i +  y_i \cdot x_i^T \bA^{-1} (\bXX{i}^T\byy{i}) < 0 } \\
%         &\le \indict{ y_i \cdot x_i^T \bA^{-1} (\bXX{i}^T\byy{i}) < 0 } = \error(\ff{i}(x_i), y_i ) \,.\label{eq:LOO_error}
%     \end{align}

%     Using \eqref{eq:LOO_error}, we have 
%     \begin{align}
%         \error_{\wt \calS_M } (\wh f) \le \error_{\text{LOO} (S_M)} \defeq \frac{\sum_{(x_i, y_i) \in \wt S_M} \error(\ff{i}(x_i), y_i ) }{\abs{\wt \calS_M}}\label{eq:LOO_error_final}
%     \end{align}
%     \textbf{Part 2 {}{}} We now relate RHS in \eqref{eq:LOO_error_final} with the population error on mislabeled distribution. To do this, we leverage \codref{cond:hypothesis_stability} and \lemref{lem:stability_error}. In particular, we have 

%     \begin{align}
%         \Expt{\calS \cup \wt \calS_M }{ \left(\error_{\calDm}(\wh f) - \error_{\text{LOO} (S_M)}\right)^2 } \le \frac{1}{2m_1} + \frac{3\beta}{m+n} \,.
%     \end{align}

%     Using Chebyshev's inequality, with probability at least $1-\delta$, we have 
%     \begin{align}
%         \error_{\text{LOO} (S_M)} \le  \error_{\calDm}(\wh f)   + \sqrt{\frac{1}{\delta}\left(\frac{1}{2m_1} +\frac{3\beta}{m+n} \right)} \,. \label{eq:final_mislabeled_linear}
%     \end{align}
    

%     \textbf{Part 3 {}{}} Combining \eqref{eq:final_mislabeled_linear} and \eqref{eq:LOO_error_final}, we have 

%     \begin{align}
%         \error_{\wt \calS_M } (\wh f) \le \error_{\calDm}(\wh f)   + \sqrt{\frac{1}{\delta}\left(\frac{1}{2m_1} +\frac{3\beta}{m+n} \right)} \,. \label{eq:linear_parallel_lem1}
%     \end{align}

%     Compare \eqref{eq:linear_parallel_lem1}, with \eqref{eq:lemma1_final} in the proof of \lemref{lem:fit_mislabeled}. We obtain a similar relationship between $\error_{\wt \calS_M }$ and $\error_{\calDm}$ but with a polynomial concentration instead of exponential concentration. 
%     In addition, since we just use concentration arguments to relate mislabeled error with the error on clean portion and unlabeled portion, we can directly use the results in \lemref{lem:mislabeled_error} and \lemref{lem:clear_error}. Therefore, combining results in \lemref{lem:mislabeled_error}, \lemref{lem:clear_error}, and \eqref{eq:linear_parallel_lem1} with union bound, we have with probability at least $1-\delta$

%     \begin{align}
%         \error_\calD(\widehat f) \le \error_\calS(\widehat f) + 1 - 2 \error_{\wt\calS}(\widehat f) + \left(\frac{1}{\sqrt{2}} + 1.5 \right) \sqrt{\frac{\log(4/\delta)}{m}} + \sqrt{\frac{4}{\delta}\left(\frac{1}{m} +\frac{3\beta}{m+n} \right)}  \,.
%     \end{align}
    

       
% \end{proof}

% \subsection{Discussion on \codref{cond:hypothesis_stability}}

% The quantity in LHS of \codref{cond:hypothesis_stability} measures how much the function learned by the algorithm (in terms of error on unseen point) will change when one point in the training set is removed. 
% % Discussion on exponential concentration and stronger condition. 
% Notice that hypothesis stability implies error stability, i.e., \codref{cond:error_stability} ~\cite{bousquet2002stability}.  In summary, while error stability allowed us to relate the average population error of the leave-one-out classifiers with the population error of the original classifier, we need hypothesis stability condition to control the variance of the empirical leave-one-out error. 

% Additionally, we note that while the dominating term in the RHS of \thmref{thm:new_linear} matches with the dominating term in ERM bound in \thmref{thm:error_ERM}, there is a polynomial concentration term (dependence on $1/\delta$ instead of $\log(\sqrt{1/\delta})$) in  \thmref{thm:new_linear}. 
% Since with hypothesis stability, we just bound the variance,  the polynomial concentration is due to the use of Chebyshev's inequality instead of an exponential tail inequality (as in \lemref{lem:fit_mislabeled}).
% Recent works have highlighted that slightly stronger condition than hypothesis stability can be used to obtained an exponential concentration for leave-one-out error~\citep{abou2019exponential}, but we leave this for future work for now. 
% % We leave 
% % However, the constants 

% % we also want to highlight  

% \subsection{Formal statement and proof of  of \propref{prop:early_stop}}

% Before formally presenting the result, we will introduce some notation.  By $\calL_{S}(w)$, we denote 
% the objective in \eqref{eq:l2_MSE_app} with $\lambda=0$. 
% Assume Singular Value Decomposition (SVD) of $\bX$  as $\sqrt{n} \bU \bS^{1/2} \bV^T$. Hence $\bX^T \bX = \bV \bS \bV^T$.
% Consider the GD iterates defined in \eqref{eq:GD_iterates_app}. 
% % 
% We now derive closed form expression for the $t^\text{th}$ iterate of gradient descent:  
% % 
% \begin{align}
%     w_t = w_{t-1} + \eta \cdot \bX^T (\by - \bX w_{t-1}) = (\bI - \eta \bV \bS \bV^T )w_{k-1} + \eta \bX^T \by \,.
% \end{align}
% Rotating by $\bV^T$, we get 
% \begin{align}
%     \wt w_t = (\bI - \eta\bS )\wt w_{k-1} + \eta \wt \by \,, \label{eq:GD_recur}
% \end{align}
% where $\wt w_t = \bV^T w_t $ and $\wt \by = \bV^T \bX^T \by$. Assuming the initial point $w_0 = 0$ and applying the recursion in \eqref{eq:GD_recur}, we get
% \begin{align}
%     \wt w_t = \bS ^{-1} ( \bI - (\bI - \eta \bS)^k ) \wt \by \,, 
% \end{align} 
% Projecting solution back to the original space, we have 
% \begin{align}
%      w_t = \bV \bS ^{-1} ( \bI - (\bI - \eta \bS)^k ) \bV^T \bX^T \by \,, 
% \end{align} 
% % We will work with this GD solution at any iterate $t$ in the next proposition. 
% Define $f_t(x) \defeq f(x;w_t)$ as the solution at the $t^{\text{th}}$ iterate. 
% Let $\wt w_{\lambda} = \argmin_{w} \calL_\calS (w;\lambda) = (\bX^T \bX + \lambda \bI)^{-1} \bX^T \by = \bV (\bS + \lambda \bI )^{-1} \bV^T \bX^T \by $. 
% % ) \,,$ for all $t=1,2,\ldots\,.$ 
% and define $\wt f_\lambda(x) \defeq f(x;\wt w_\lambda)$ as the regularized solution. 
% Assume $\kappa$ be the condition number of the population covariance matrix 
% and 
% let $s_\text{min}$ be the minimum positive singular value of the empirical covariance matrix. Our proof idea is inspired from recent work on relating gradient flow solution and regularized solution for regression problems \citep{ali2018continuous}. We will use the following lemma in the proof: 
% \begin{lemma} \label{lem:ineq_soln}
%     For all $x \in [0,1]$ and for all $ k \in \mathbb{N}$, we have (a) $ \frac{kx}{1+kx} \le 1- (1-x)^k$ and (b) $ 1- (1-x)^k \le 2 \cdot \frac{kx}{kx+1} $.
%     %  where $g(c)$ is a constant dependent on $c$. For $c = 1$, $g(c) = 2.0$.   
% \end{lemma}
% \begin{proof}
%     % [Proof of \lemref{lem:ineq_soln}]
%     % Part (a) is easy. 
%     Using $ (1-x)^k \le \frac{1}{1+kx}$, we have part (a). For part (b), we numerically maximize $\frac{ (1+kx ) (1 - (1-x)^k) }{kx}$ for all $k\ge 1$ and for all $x \in [0, 1]$.  
% \end{proof}

% % 
% % Next, 

% \begin{prop}[Formal statement of \propref{prop:early_stop}] \label{prop:formal_early_stop}
% Let $\lambda = \frac{1}{t\eta}$. For a training point $x$, we have 
% \begin{align*}
%     \Expt{x \sim \calS}{(f_t(x) - \wt f_\lambda(x))^2} &\le c(t,\eta) \cdot \Expt{x \sim \calS}{f_t(x)^2} \,, %\label{eq:early_stop}
% \end{align*}
% where $c(t, \eta) \defeq \min( 0.25, \frac{1}{s_\text{min}^2 t^2 \eta^2})$. Similarly for a test point, we have 
% \begin{align*}
%     \Expt{x \sim \calD_\calX}{(f_t(x) - \wt f_\lambda(x))^2} &\le \kappa \cdot c(t,\eta) \cdot \Expt{x \sim \calD_\calX}{f_t(x)^2} \,. %\label{eq:early_stop}
% \end{align*}
% \end{prop} 

% \begin{proof}
%     %%%%%%%%%%%%% 
%     We want to analyze the expected squared difference output of regularized linear regression with regularization constant $\lambda = \frac{1}{\eta t}$ and gradient descent solution at $t^\text{th}$ iterate. We separately expand the algebraic expression for squared difference at a training point and a test point. 
%     % We start by considering the difference  
%     Then the main step is to show that  $\left[ \bS ^{-1} ( \bI - (\bI - \eta \bS)^k )  - (\bS + \lambda \bI )^{-1}\right] \preceq c(\eta, t) \cdot \bS ^{-1} ( \bI - (\bI - \eta \bS)^k ) $.

%     %%%%%%%%%%%%%
    
%   \textbf{Part 1 {} {}} 
%     First, we will analyze the squared difference of output at a training point (for simplicity, we refer to $S \cup \wt S$ as $S$), i.e. 
%     \begin{align}
%         \Expt{ x \sim \calS }{\left(f_t(x) - \wt f_\lambda (x)\right)^2} &= \norm{\bX w_t - \bX \wt w_\lambda}{2}^2 =   \norm{\bX \bV \bS ^{-1} ( \bI - (\bI - \eta \bS)^t ) \bV^T \bX^T \by - \bX \bV (\bS + \lambda \bI )^{-1} \bV^T \bX^T \by }{2}^2 \\
%         &= \norm{\bX \bV \left(\bS ^{-1} ( \bI - (\bI - \eta \bS)^t ) - (\bS + \lambda \bI )^{-1} \right) \bV^T \bX^T \by  }{2} \\
%         &=  \by^T \bV \bX \left( \underbrace{\bS ^{-1} ( \bI - (\bI - \eta \bS)^t ) - (\bS + \lambda \bI )^{-1}}_{\RN{1}} \right)^2 \bS \bV^T \bX^T \by \label{eq:train_GD_rel}
%         %  (\bX \bV \bS ^{-1} ( \bI - (\bI - \eta \bS)^k ) \bV^T \bX^T \by)^T \bX \bV \bS ^{-1} ( \bI - (\bI - \eta \bS)^k ) \bV^T \bX^T \by
%     \end{align}
%     We now separately consider term 1. Substituting $\lambda = \frac{1}{t \eta}$, we get
%     \begin{align}
%         \bS ^{-1} ( \bI - (\bI - \eta \bS)^t ) - (\bS + \lambda \bI )^{-1} &= \bS^{-1} \left( ( \bI - (\bI - \eta \bS)^t ) - (\bI + \bS^{-1} \lambda )^{-1}\right) \\
%         &= \underbrace{\bS^{-1} \left( ( \bI - (\bI - \eta \bS)^t ) - (\bI + ( \bS t \eta)^{-1}  )^{-1}\right)}_{\bA}
%     \end{align}

%     We now separately bound the diagonal entries in matrix $\bA$. 
%     With $s_i$, we denote $i^{\text{th}}$ diagonal entry of $\bS$. Note that since $ \eta\le 1/\norm{S}{\text{op}}$, for all $i$, $\eta s_i  \le 1$.  Consider $i^{\text{th}}$ diagonal term (which is non-zero) of the diagonal matrix $\bA$, we have 
%     \begin{align}
%         \bA_{ii} = \frac{1}{s_i} \left(  1 - (1 - s_i \eta)^t - \frac{t \eta s_i}{1 + t \eta s_i } \right) &=  \frac{1 - (1 - s_i \eta)^t}{s_i} \left( \underbrace{ 1 - \frac{t \eta s_i}{(1 + t \eta s_i)(1 - (1 - s_i \eta)^t)}}_{\RN{2}} \right) \\ 
%          &\le \frac{1}{2}\left[ \frac{1 - (1 - s_i \eta)^t}{ s_i} \right] \tag*{(Using \lemref{lem:ineq_soln} (b))} \,.
%     \end{align} 
%     Additionally, we can also show the following upper bound on term 2: 
%     \begin{align}
%          1 - \frac{t \eta s_i}{(1 + t \eta s_i)(1 - (1 - s_i \eta)^t)} &= \frac{(1 + t \eta s_i)(1 - (1 - s_i \eta)^t) - t \eta s_i }{(1 + t \eta s_i)(1 - (1 - s_i \eta)^t)} \\
%          & \le  \frac{ 1 -  (1 - s_i \eta)^t - t \eta s_i (1 - s_i \eta)^t}{(1 + t \eta s_i)(1 - (1 - s_i \eta)^t)} \\
%          & \le \frac{1}{t\eta s_i} \,. \tag{Using \lemref{lem:ineq_soln} (a)}
%         %  &\le \frac{1}{2}\left[ \frac{1 - (1 - s_i \eta)^t}{ s_i} \right] \tag*{(Using \lemref{lem:ineq_soln})} \,.
%     \end{align} 

%     Combining both the upper bounds on each diagonal entry $\bA_{ii}$, we have 
%     \begin{align}
%     \bA \preceq c_1(\eta, t) \cdot \bS^{-1} ( \bI - (\bI - \eta \bS)^t ) \,, \label{eq:upperbound_diagonal}
%     \end{align}
%     where $c_1(\eta, t ) = \min(0.5, \frac{1}{t s_i \eta })$. Plugging this into \eqref{eq:train_GD_rel}, we have 
%     \begin{align}
%         \Expt{ x \sim \calS }{\left(f_t(x) - \wt f_\lambda (x)\right)^2} &\le c(\eta, t) \cdot \by^T \bV \bX  \left( \bS^{-1} ( \bI - (\bI - \eta \bS)^t ) \right)^2 \bS \bV^T \bX^T \by \\
%         &=   c(\eta, t) \cdot \by^T \bV \bX  \left( \bS^{-1} ( \bI - (\bI - \eta \bS)^t ) \right) \bS \left( \bS^{-1} ( \bI - (\bI - \eta \bS)^t ) \right) \bV^T \bX^T \by \\
%         & =  c(\eta, t) \cdot \norm{\bX w_t}{2}^2 \\
%         &= c(\eta, t) \cdot  \Expt{ x \sim \calS }{\left(f_t(x) \right)^2} \,,
%     \end{align}
%     where $c(\eta, t ) = \min(0.25, \frac{1}{t^2 s^2_i \eta^2 })$.

%     \textbf{Part 2 {} {}} With $\bSigma$, we denote the underlying true covariance matrix. We now consider the squared difference of output at an unseen point: 
%     \begin{align}
%         \Expt{ x \sim \calD_{\calX} }{\left(f_t(x) - \wt f_\lambda (x)\right)^2} &= \Expt{x \sim \calD_{\calX}}{\norm{x^T w_t - x^T \wt w_\lambda}{2}} \\
%         &=   \norm{x^T \bV \bS ^{-1} ( \bI - (\bI - \eta \bS)^t ) \bV^T \bX^T \by - x^T \bV (\bS + \lambda \bI )^{-1} \bV^T \bX^T \by }{2} \\
%         &= \norm{x^T \bV \left(\bS ^{-1} ( \bI - (\bI - \eta \bS)^t ) - (\bS + \lambda \bI )^{-1} \right) \bV^T \bX^T \by  }{2} \\
%         &= \by^T \bV \bX \left( \bS ^{-1} ( \bI - (\bI - \eta \bS)^t ) - (\bS + \lambda \bI )^{-1} \right) \bV^T \bSigma \bV \\ &\qquad \qquad \qquad \qquad \qquad \left( (\bI - (\bI - \eta \bS)^t ) - (\bS + \lambda \bI )^{-1} \right) \bV^T \bX^T \by \\
%         &\le \sigma_{\text{max}} \cdot \by^T \bV \bX \left( \underbrace{\bS ^{-1} ( \bI - (\bI - \eta \bS)^t ) - (\bS + \lambda \bI )^{-1}}_{\RN{1}} \right)^2 \bV^T \bX^T \by \,, \label{eq:test_GD_rel}
%         %  (\bX \bV \bS ^{-1} ( \bI - (\bI - \eta \bS)^k ) \bV^T \bX^T \by)^T \bX \bV \bS ^{-1} ( \bI - (\bI - \eta \bS)^k ) \bV^T \bX^T \by
%     \end{align}
%     where $\sigma_{\text{max}}$ is the maximum eigenvalue of the underlying covariance matrix $\bSigma$. Using the upper bound on term 1 in \eqref{eq:upperbound_diagonal}, we have 
%     \begin{align}
%         \Expt{ x \sim \calD_{\calX} }{\left(f_t(x) - \wt f_\lambda (x)\right)^2} &\le \sigma_{\text{max}} \cdot c(\eta, t) \cdot \by^T \bV \bX  \left( \bS^{-1} ( \bI - (\bI - \eta \bS)^t ) \right)^2 \bV^T \bX^T \by \\
%         &=   \kappa \cdot c(\eta, t) \cdot \sigma_{\text{min}}\cdot \norm{\bV \left( \bS^{-1} ( \bI - (\bI - \eta \bS)^t ) \right) \bV^T \bX^T \by}{2}^2 \\
%         &\le \kappa \cdot c(\eta, t) \cdot \left[ \bV \left( \bS^{-1} ( \bI - (\bI - \eta \bS)^t ) \right) \bV^T \bX^T \right]^T \bSigma \\
%         &\qquad \qquad \qquad \qquad \qquad \left[ \bV \left( \bS^{-1} ( \bI - (\bI - \eta \bS)^t ) \right) \bV^T \bX^T \right] \by \\
%         & = \kappa \cdot c(\eta, t) \cdot \Expt{x \sim \calD_{\calX}}{\norm{x^T w_t}{2}} \,.
%     \end{align}
% % 
% % 
%     % Since $ \eta\le 1/\norm{S}{\text{op}}$, invoking \lemref{lem:ineq_soln} to upper bound term 1 with
% \end{proof}


% \newpage
% \section{Additional experiments and details}\label{app:exp}
% \newcommand\tab[1][1cm]{\hspace*{#1}}

% \subsection{Datasets} \label{sec:app_dataset}

% \textbf{Toy Dataset {} {}} Assume fixed constants $\mu$ and $\sigma$. For a given label $y$, we simulate features $x$ in our toy classification setup as follows: 
% \begin{align*}
%     x \defeq \texttt{concat} \left[ x_1, x_2\right] \quad \text{where} \quad  x_1 \sim  \calN( y \cdot \mu, \sigma^2 I_{d \times d}) \ \  \text{and} \ \  x_1 \sim  \calN( 0, \sigma^2 I_{d \times d}) \,.
% \end{align*}  
% % where $y$ is the true label and $x$ is the corresponding feature vector. 
% In experiements throughout the paper, we fix dimention $d=100$, $\mu = 1.0 $, and $\sigma = \sqrt{d}$. Intuitively, $x_1$ carries the information about the underlying label and $x_2$ is additional noise independent of the underlying label. 

% \textbf{CV datasets {} {}} We use MNIST~\citep{lecun1998mnist} and CIFAR10~\cite{krizhevsky2009learning}. 
% % For binary tasks, 
% We produce a binary variant from the multiclass classification problem by mapping classes $\{0,1,2,3,4\}$ to label $1$ and $\{ 5,6,7,8,9\}$ to label $-1$. For CIFAR dataset, we also use the standard data augementation of random crop and horizontal flip. PyTorch code is as follows: 

% \texttt{(transforms.RandomCrop(32, padding=4),\\
% \tab transforms.RandomHorizontalFlip())}

% \textbf{NLP dataset {} {}} We use IMDb Sentiment analysis~\citep{maas2011learning} corpus.  

% \subsection{Architecture Details} 

% All experiments were run on NVIDIA GeForce RTX 2080 Ti GPUs. We used PyTorch~\citep{NEURIPS2019a9015} and Keras with Tensorflow~\citep{abadi2016tensorflow} backend for experiments. 
% % , ELMo embeddings~\citep{Peters:2018}, and Hugging Face Transformers~\citep{wolf-etal-2020-transformers}. 

% \textbf{Linear model {} {}} For the toy dataset, we simulate a linear model with scalar output and the same number of parameters as the number of dimensions.   

% \textbf{Wide nets {} {}} To simulate the NTK regime, we experiment with $2-$layered wide nets. The PyTorch code for 2-layer wide MLP is as follows: 


% \texttt{ nn.Sequential( \\
% \tab     nn.Flatten(),\\
% \tab    nn.Linear(input\_dims, 200000, bias=True),\\
% \tab    nn.ReLU(),\\
% \tab    nn.Linear(200000, 1, bias=True)\\
% \tab     )}


% We experiment both (i) with the first layer fixed at random initialization; (ii)  and updating both layers' weights.     

% \textbf{Deep nets for CV tasks {} {}} We consider a 4-layered MLP. The PyTorch code for 4-layer MLP is as follows: 

% \texttt{ nn.Sequential(nn.Flatten(), \\
% \tab        nn.Linear(input\_dim, 5000, bias=True),\\
% \tab        nn.ReLU(),\\
% \tab        nn.Linear(5000, 5000, bias=True),\\
% \tab        nn.ReLU(),\\
% \tab        nn.Linear(5000, 5000, bias=True),\\
% \tab        nn.ReLU(),\\
% % \tab        nn.Linear(5000, 5000, bias=True),\\
% % \tab        nn.ReLU(),\\
% \tab        nn.Linear(1024, num\_label, bias=True)\\
% \tab        )}

% For MNIST, we use $1000$ nodes instead of $5000$ nodes in the hidden layer. 
% % 
% We also experiment with convolutional nets. In particular, we use ResNet18 \citep{he2016deep}. Implementation adapted from:  \url{https://github.com/kuangliu/pytorch-cifar.git}. 

% \textbf{Deep nets for NLP {} {}} We use a simple LSTM model with embeddings intialized with ELMo embeddings~\citep{Peters:2018}. Code adapted from: \url{https://github.com/kamujun/elmo_experiments/blob/master/elmo_experiment/notebooks/elmo_text_classification_on_imdb.ipynb} 

% We also evaluate our bounds with a BERT model. In particular, we fine-tune an off-the-shelf uncased BERT model~\citep{devlin2018bert}. Code adapted from Hugging Face Transformers~\citep{wolf-etal-2020-transformers}: \url{https://huggingface.co/transformers/v3.1.0/custom_datasets.html}. 


% \subsection{Additonal experiments}

% 1. SGD with linear models on cross entropy and MSE loss. 

% 2. CE loss and SGD. GD with MSE loss 

% 3. Binary MNIST with MLP. multiclass MNIST  

% \textbf{Results on CIFAR 10 {} {}} 
% % 
% We plot epoch wise error curve for results in \tabref{table:multiclass}. We observe the same trend as in \figref{fig:error_CIFAR10}. Additionally, we plot an \emph{oracle bound} obtained by tracking the error on mislabeled data which nevertheless were predicted as true label. To obtain an exact emprical value of the oracle bound, we need underlying true labels for the randomly labeled data. 
% % Note that our bound in \thmref{thm:multiclass_ERM}, lower bounds the accuracy as predicted by the oracle bound. 
% While with just access to extra unlabeled data we cannot calculate oracle bound, we note that the oracle bound is very tight and never violated in practice underscoring an importamt aspect of generalization in multiclass problems. This highlight that even a stronger conjecture may hold in multiclass classification, i.e., error on mislabeled data (where nevertheless true label was predicted) lower bounds the population error on the distribution of mislabeled data and hence, the error on (a specific) mislabeled portion predicts the population accuracy on clean data. 
% % 
% On the other hand, the dominating term of in \thmref{thm:multiclass_ERM} is loose when compared with the oracle bound. The main reason, we believe is the pessimistic upper bound in \eqref{eq:lemma1_final_multi_prev} in the proof of \lemref{lem:fit_mislabeled_multi}. We leave an investigation on this gap for future. 
% % of fit 

% % However, oracle bound highlights two . One,  



% \begin{figure}[h]
%     \centering 
%     % \vspace{-15pt}
%     % \includegraphics[width=0.9\linewidth]{example-image-a}
%     \includegraphics[width=0.4\linewidth]{figures/CIFAR10-FNN.pdf} \hfil
%     \includegraphics[width=0.4\linewidth]{figures/CIFAR10-Resnet.pdf}
%     % \includegraphics[width=0.9\linewidth]{figures/{CIFAR10_rn=0.1_lr=0.2_wd=0.005}.png}
%     % \vspace{-10pt}
%     \caption{ Per epoch curves for CIFAR10 corresponding results in \tabref{table:multiclass}. As before, we just plot the dominating term in the RHS of \eqref{eq:multiclass_ERM} as predicted bound. Additionally, we also plot the predicted lower bound by the error on mislabeled data which nevertheless were predicted as true label. We refer to this as ``Oracle bound''. See text for more details. 
%     % 
%     % except for the stopping point. 
%     % The bound predicted by RATT (RHS in \eqref{eq:multiclass_ERM}) is vacuous. 
%     }\label{fig:error_epoch_CIFAR10}
%     % \vspace{-15pt}
% \end{figure}


% \textbf{Results on CIFAR 100 {} {}} 
% % 
% On CIFAR100, our bound in \eqref{eq:multiclass_ERM} yields vacous bounds. However, the oracle bound as explained above yields tight guarantees in the initial phase of the learning (i.e., when learning rate is less than $0.1$). 

% \begin{figure}[h]
%     \centering 
%     % \vspace{-15pt}
%     % \includegraphics[width=0.9\linewidth]{example-image-a}
%     \includegraphics[width=0.4\linewidth]{figures/CIFAR100-Resnet.pdf}
%     % \includegraphics[width=0.9\linewidth]{figures/{CIFAR10_rn=0.1_lr=0.2_wd=0.005}.png}
%     % \vspace{-10pt}
%     \caption{ Predicted lower bound by the error on mislabeled data which nevertheless were predicted as true label with ResNet18 on CIFAR100. We refer to this as ``Oracle bound''. See text for more details. 
%     % 
%     % except for the stopping point. 
%     The bound predicted by RATT (RHS in \eqref{eq:multiclass_ERM}) is vacuous. 
%     }\label{fig:error_CIFAR100}
%     % \vspace{-15pt}
% \end{figure}


% % \paragraph{Experiments on CIFAR100} 



% \subsection{Hyperparameter Details}


% \textbf{\figref{fig:error_CIFAR10} {} {}} We use clean training dataset of size $40,000$. We fix the amount of unlabeled data at $20\%$ of the clean size, i.e. we include additional $8,000$ points with randomly assigned labels. We use test set of $10,000$ points. For both MLP and ResNet, we use SGD with an initial learning rate of $0.1$ and momentum $0.9$. We fix the weight decay parameter at $5\times 10^{-4}$. After $100$ epochs, we decay the learning rate to $0.01$. We use SGD batch size of $100$. 

% \textbf{\figref{fig:error_binary} (a) {} {}} We obtain a toy dataset according to the process described in \secref{sec:app_dataset}. We fix $d=100$ and create a dataset of $50,000$ points with balanced classes. Moreover, we sample additional covariates with the same procedure to create randomly labeled dataset. For both SGD and GD training, we use a fixed learning rate $0.1$.    

% \textbf{\figref{fig:error_binary} (b) {} {}} Similar to binary CIFAR, we use clean training dataset of size $40,000$ and fix the amount of unlabeled data at $20\%$ of the clean dataset size. To train wide nets, we use a fixed learning of $0.001$ with GD and SGD. We decide the weight decay parameter and the early stopping point that maximizes our generalization bound (i.e. without peeking at unseen data ).  We use SGD batch size of $100$. 

% \textbf{\figref{fig:error_binary} (c) {} {}} With IMDb dataset, we use a clean dataset of size $20,000$ and as before, fix the amount of unlabeled data at $20\%$ of the clean data. To train ELMo model, we use Adam optimizer with a fixed learning rate $0.01$ and weight decay $10^{-6}$ to minimize cross entropy loss. We train with batch size $32$ for 3 epochs. To fine-tune BERT model, we use Adam optimizer with learning rate $5\times 10^{-5}$ to minimize cross entropy loss. We train with a batch size of $16$ for 1 epoch.    

% \textbf{\tabref{table:multiclass} {} {}} For multiclass datasets, we train both MLP and ResNet with the same hyperparameters as described before. We sample a clean training dataset of size $40,000$ and fix the amount of unlabeled data at $20\%$ of the clean size. We use SGD with an initial learning rate of $0.1$ and momentum $0.9$. We fix the weight decay parameter at $5\times 10^{-4}$. After $30$ epochs for ResNet and after $50$ epochs for MLP, we decay the learning rate to $0.01$.  We use SGD with batch size $100$. 
% For \figref{fig:error_CIFAR100}, we use the same hyperparameters as 
% CIFAR10 training, except we now decay learning rate after $100$ epochs. 


% In all experiments, to identify the best possible accuracy on just the clean data, we use the exact same set of hyperparamters except the stopping point. We choose a stopping point that maximizes test performance. 

% \subsection{Summary of experiments }

% \begin{center}
%     \begin{table}[H] 
%         \centering
%         \begin{tabular}{|c|c|c|c|} 
%         \hline
%         Classification type & Model category & Model & Dataset  \\ [0.5ex] 
%         \hline
%         \hline
%         \multirow{9}{*}{Binary} & Low dimensional & Linear model & Toy Gaussain dataset  \\
%                         \cline{2-4}
%                          & \multirow{1}{*}{Overparameterized linear nets} 
%                         %  & Linear model & Toy Gaussain dataset \\
%                         %  \cline{3-4}
%                         %  & & 2-layer wide net& Toy Gaussain dataset \\
%                         %  \cline{3-4}
%                          & 2-layer wide net & Binary MNIST \\
%                          \cline{2-4}                 
%                          & \multirow{6}{*}{Deep nets} & \multirow{2}{*}{MLP} & Binary MNIST \\
%                          \cline{4-4}
%                          & &  & Binary CIFAR \\
%                          \cline{3-4}
%                          &  & \multirow{2}{*}{ResNet} & Binary MNIST \\
%                          \cline{4-4}
%                          & &  & Binary CIFAR \\
%                          \cline{3-4}
%                          &  & ELMo-LSTM model & IMDb Sentiment Analysis \\
%                          \cline{3-4}
%                          & & BERT pre-trained model & IMDb Sentiment Analysis \\
%         \hline
%         \multirow{5}{*}{Multiclass} & \multirow{5}{*}{Deep nets} & \multirow{2}{*}{MLP} & MNIST \\
%                         \cline{4-4} 
%                         & & & CIFAR10 \\                   
%                         \cline{3-4}
%                          &   & \multirow{3}{*}{ResNet} & MNIST \\
%                          \cline{4-4}
%                          &   & & CIFAR10 \\
%                          \cline{4-4}
%                          &   & & CIFAR100 \\
%         \hline
%         \end{tabular}
%         % \caption{Summary of experiments performed} \label{table:experiments}
%     \end{table}    
%     % \footnotetext[6]{We use both MSE loss and cross-entropy loss.}
%     % \footnotetext[6]{We try 2 variants: one with a fixed first layer and the other with both layers trainable.}
% \end{center}

% \newpage
% \section{Proof of \lemref{lem:stability_error}} \label{app:proof_lem_error}

% \begin{proof}[Proof of \lemref{lem:stability_error}]
%     Recall, we have a training set $S \cup \wt S_C$. We defined leave-one-out error on mislabeled points as $$\error_{\text{LOO}(\wt S_M) } = \frac{\sum_{(x_i, y_i) \in \wt S_M} \error( f_{(i)}( x_i), y_i)}{ \abs{\wt S_M }} \,, $$
%     where $f_{(i)} \defeq f(\calA, (S \cup \wt S)_{(i)})$. Define $S^\prime \defeq S \cup \wt S$. Assume $(x,y)$ and $(x^\prime,y^\prime)$ as i.i.d. samples from ${\calDm}$. 
%     Using Lemma 25 in \citet{bousquet2002stability}, we have
%     \begin{align*}
%         \Expo{ \left( \error_{\calDm}(\wh f) -\error_{\text{LOO}(\wt S_M) } \right)^2 } \le & \Expt{ S^\prime, (x,y), (x^\prime,y^\prime) }{ \error(\wh f(x), y ) \error(\wh f(x^\prime), y^\prime )} - 2 \Expt{ S^\prime, (x,y) }{ \error(\wh f(x), y ) \error(f_{(i)}(x_i), y_i )} \\
%         & + \frac{m_1-1}{m_1}\Expt{ S^\prime }{  \error(f_{(i)}(x_i), y_i )  \error(f_{(j)}(x_j), y_j )} + \frac{1}{m_1} \Expt{ S^\prime }{  \error(f_{(i)}(x_i), y_i ) } \,. \numberthis \label{eq:main_reln}
%     \end{align*}
%     We can rewrite the equation above as : 
%     \begin{align*}
%         \Expo{ \left( \error_{\calDm}(\wh f) -\error_{\text{LOO}(\wt S_M) } \right)^2 } \le &  \, \underbrace{\Expt{ S^\prime, (x,y), (x^\prime,y^\prime) }{ \error(\wh f(x), y ) \error(\wh f(x^\prime), y^\prime ) - \error(\wh f(x), y ) \error(f_{(i)}(x_i), y_i )}}_{\RN{1}} \\
%         & + \underbrace{\Expt{ S^\prime }{  \error(f_{(i)}(x_i), y_i )  \error(f_{(j)}(x_j), y_j ) -  \error(\wh f(x), y ) \error(f_{(i)}(x_i), y_i )}}_{\RN{2}} \\ &+ \underbrace{\frac{1}{m_1} \Expt{ S^\prime }{  \error(f_{(i)}(x_i), y_i ) - \error(f_{(i)}(x_i), y_i )  \error(f_{(j)}(x_j), y_j ) }}_{\RN{3}} \,. \numberthis \label{eq:main_reln2}
%     \end{align*}
    
%     We will now bound term $\RN{3}$.  Using Schwarz's inequality, we have
    
%     \begin{align}
%         \Expt{ S^\prime }{  \error(f_{(i)}(x_i), y_i ) - \error(f_{(i)}(x_i), y_i )  \error(f_{(j)}(x_j), y_j ) }^2 &\le  \Expt{ S^\prime }{  \error(f_{(i)}(x_i), y_i ) }^2 \Expt{S^\prime}{1 -   \error(f_{(j)}(x_j), y_j ) }^2 \\
%         &\le \frac{1}{4} \label{eq:term1_lem12}
%     \end{align}
    
%     Note that since $(x_i,y_i)$, $(x_j ,y_j )$, $(x,y)$, and $(x^\prime, y^\prime)$ are all from same distribution $\calDm$, we directly incorporate the bounds on term $\RN{1}$ and $\RN{2}$ from proof of Lemma 9 in \citet{bousquet2002stability}. Combining that with \eqref{eq:term1_lem12} and our definition of hypothesis stability in \codref{cond:hypothesis_stability}, we have the required claim. 
    
    
%     % We now re-write term $\RN{1}$ as
%     % \begin{align*}
%     %         &\Expt{S^\prime, (x,y), (x^\prime,y^\prime) }{ \error(\wh f(x), y ) \error(\wh f(x^\prime), y^\prime ) - \error(\wh f(x), y ) \error(f_{(i)}(x_i), y_i )} \\ & \qquad = \Expt{ S^\prime, (x,y), (x^\prime,y^\prime) }{ \error(\wh f(x), y ) \error(\wh f  (x^\prime), y^\prime ) - \error(\wh f ^\prime(x), y ) \error(f_{(i)}(x^\prime), y^\prime )} \tag{Exchanging $(x_i, y_i)$ with $(x^\prime, y^\prime)$ in the second term} \\
%     %         & \qquad = \Expt{ S^\prime, (x,y), (x^\prime,y^\prime) }{  \left(\error(\wh f(x), y )-  \error(f_{(i)}(x), y ) \right) \error(\wh f  (x^\prime), y^\prime )  } \\
%     %         & \qquad  + \Expt{ S^\prime, (x,y), (x^\prime,y^\prime) }{  \left(\error(f_{(i)}(x), y ) -\error(\wh f ^\prime(x), y ) \right) \error(\wh f  (x^\prime), y^\prime )}  \\
%     %         & \qquad +\Expt{ S^\prime, (x,y), (x^\prime,y^\prime) }{  \left( \error(\wh f  (x^\prime), y^\prime ) -  \error(f_{(i)}(x^\prime), y^\prime ) \right) \error(\wh f ^\prime(x), y ) }  \,, \numberthis \label{eq:term1_final}
%     % \end{align*}
%     % where $\wh f^\prime$ is the classifier obtained by training on $ S^\prime_{(i)} \cup \{ (x^\prime, y^\prime) \} $. Similarly we can re-write term $\RN{2}$ as 
%     % \begin{align*}
%     %     & \Expt{ S^\prime }{  \error(f_{(i)}(x_i), y_i )  \error(f_{(j)}(x_j), y_j ) -  \error(\wh f(x), y ) \error(f_{(i)}(x_i), y_i )} \\
%     %     &\quad  = \Expt{ S^\prime, (x,y), (x^\prime,y^\prime)}{  \error(f^{\prime\prime}_{(i)}(x), y )  \error(f_{(j)}^{\prime}(x^\prime), y^\prime ) -  \error(\wh f(x), y ) \error(f_{(i)}(x_i), y_i )} \tag{Exchanging $(x_i, y_i)$ with $(x, y)$ and $(x_j, y_j)$ with $(x^\prime, y^\prime)$ in the first term}\\
%     %     &\quad = \Expt{ S^\prime, (x,y), (x^\prime,y^\prime)}{  \error(f^{\prime\prime}_{(j)}(x), y )  \error(f_{(i)}^{\prime}(x^\prime), y^\prime ) -  \error(\wh f^\prime (x), y ) \error(f^\prime_{(j)}(x^\prime), y^\prime )} \tag{Exchanging $(x_i, y_i)$ and $(x_j, y_j)$ and then replacing $(x_j, y_j)$ with $(x^\prime, y^\prime)$ in the second term} \\
%     %     & \quad = \Expt{ S^\prime, (x,y), (x^\prime,y^\prime) }{  \left( \error(f_{(i)}^{\prime}(x^\prime), y^\prime )   -  \error(\wh f^{\prime\prime}  (x^\prime), y^\prime ) \right)  \error(f^{\prime\prime}_{(j)}(x), y )   } \\
%     %     & \quad  + \Expt{ S^\prime, (x,y), (x^\prime,y^\prime) }{  \left( \error(f^{\prime\prime}_{(j)}(x), y )  -\error(\wh f ^\prime(x), y ) \right) \error(\wh f^{\prime\prime}  (x^\prime), y^\prime )  }  \\
%     %     & \quad+ \Expt{ S^\prime, (x,y), (x^\prime,y^\prime) }{  \left( \error(\wh f^{\prime\prime}  (x^\prime), y^\prime )  -  \error(f^\prime_{(j)}(x^\prime), y^\prime ) \right)  \error(\wh f^\prime (x), y ) }   \\
%     %     & \quad = \Expt{ S^\prime, (x,y), (x^\prime,y^\prime) }{  \left( \error(f_{(i)}^{\prime}(x^\prime), y^\prime )   -  \error(\wh f (x^\prime), y^\prime ) \right)  \error(f_{(i)}(x_j), y_j )   } \\
%     %     & \quad  + \Expt{ S^\prime, (x,y), (x^\prime,y^\prime) }{  \left( \error(f^{\prime\prime}_{(j)}(x), y )  -\error(\wh f (x), y ) \right) \error(\wh f^{\prime\prime}  (x_j), y_j )  }  \\
%     %     & \quad+ \Expt{ S^\prime, (x,y), (x^\prime,y^\prime) }{  \left( \error(\wh f^{\prime\prime}  (x^\prime), y^\prime )  -  \error(f^\prime_{(j)}(x^\prime), y^\prime ) \right)  \error(\wh f^\prime (x^\prime), y^\prime ) }  \,, \numberthis \label{eq:term2_final}
%     % \end{align*}
%     % where $f^{\prime\prime}_{(j)}$ is trained on $S^\prime_{(j,i)} \cup {(x,y)}$, $f^{\prime}_{(i)}$ is trained on $S^\prime_{(j,i)} \cup {(x^\prime,y^\prime)}$, and $\wh f^{\prime\prime} $ is trained on $S^\prime_{(j)} \cup {(x,y)}$. Note in the last line we replaced $(x,y)$ by $(x_j, y_j)$ in the first term, replaced $(x^\prime,y^\prime)$ by $(x_j, y_j)$ in the second term and exchanged $(x_i,y_i)$ with $(x_j,y_j)$ and also $(x,y)$ and $(x^\prime, y^\prime)$
    
    
% \end{proof}
\backmatter

\subsection{Funding:}
\noindent The work carried out in Montréal was supported by Natural Science and Engineering Research Council of Canada (Discovery, SPG, and CRD Grants), Canada Research Chairs, Canada Foundation for Innovation, Mitacs, PRIMA Québec, Defence Canada (Innovation for Defence Excellence and Security, IDEaS), the European Union's Horizon Europe research and innovation programme under grant agreement No 101070700 (MIRAQLS), and the US Army Research Office Grant No. W911NF-22-1-0277. 
\subsection{Acknowledgement:} The authors would like to thank Joel Bouchard for technical support.



% The appendix contains material that would clutter up the main text,
% such as program code, survey instruments, or interview transcripts.
% Remove it if you don't have an appendix.

% \appendix

% \onecolumn


% \tableofcontents{}

% \newpage

\section*{Supplementary Material}
\addcontentsline{toc}{section}{Supplementary Material}


Throughout this discussion, 
we will make frequently use 
of the following standard results
concerning the exponential concentration 
of random variables:

\begin{lemma}[Hoeffding's inequality for independent RVs~\citep{hoeffding1994probability}] Let $Z_1, Z_2, \ldots, Z_n$ be independent bounded random variables with $Z_i \in [a,b]$ for all $i$, then 
    \begin{align*}
        \prob\left( \frac{1}{n} \sum_{i=1}^n (Z_i - \Expo{Z_i}) \ge t \right) \le \exp{\left( -\frac{2nt^2}{(b-a)^2} \right) }
    \end{align*} 
    and 
    \begin{align*}
        \prob\left( \frac{1}{n} \sum_{i=1}^n (Z_i - \Expo{Z_i}) \le -t \right) \le \exp{\left( -\frac{2nt^2}{(b-a)^2} \right) }
    \end{align*} 
    for all $t \ge 0$. 
\end{lemma}

\begin{lemma}[Hoeffding's inequality for sampling with replacement~\citep{hoeffding1994probability}] \label{lem:hoeffding_sampling} Let $\calZ = (Z_1, Z_2, \ldots, Z_N)$ be a finite population of $N$ points with $Z_i \in [a.b]$ for all $i$. Let $X_1, X_2, \ldots X_n$ be a random sample drawn without replacement from $\calZ$. Then for all $t \ge 0$, we have 
    \begin{align*}
        \prob\left( \frac{1}{n} \sum_{i=1}^n (X_i - \mu ) \ge t \right) \le \exp{\left( -\frac{2nt^2}{(b-a)^2} \right) }
    \end{align*} 
    and 
    \begin{align*}
        \prob\left( \frac{1}{n} \sum_{i=1}^n (X_i - \mu ) \le -t \right) \le \exp{\left( -\frac{2nt^2}{(b-a)^2} \right) } \,,
    \end{align*} 
    where $\mu = \frac{1}{N} \sum_{i=1}^{N} Z_i$. 
\end{lemma}

We now discuss one condition that generalizes the exponential concentration to dependent random variables.
\begin{condition}[Bounded difference inequality] \label{cond:BDC} Let $\calZ$ be some set and $\phi: \calZ^n \to \Real$. We say that $\phi$ satisfies the bounded difference assumption if 
there exists $c_1, c_2, \ldots c_n \ge 0$ s.t. for all $i$, we have 
\begin{align*}
    \sup_{Z_1,Z_2, \ldots,Z_n, Z_i^\prime \in \calZ^{n+1} } \abs{\phi (Z_1, \ldots, Z_i, \ldots, Z_n ) - \phi (Z_1, \ldots, Z_i^\prime, \ldots, Z_n ) } \le c_i \,.
\end{align*} 
\end{condition}

\begin{lemma}[McDiarmid’s inequality~\citep{mcdiarmid1989}] \label{lem:McDiarmid} Let $Z_1, Z_2, \ldots, Z_n$ be independent random variables on set $\calZ$ and $\phi : \calZ^n \to \Real$ satisfy bounded difference inequality (\codref{cond:BDC}). Then for all $t>0$, we have 
    \begin{align*}
        \prob\left( \phi(Z_1, Z_2, \ldots, Z_n) - \Expo{\phi(Z_1, Z_2, \ldots, Z_n)} \ge t \right) \le \exp{\left( -\frac{2t^2}{\sum_{i=1}^n c_i^2} \right) } 
    \end{align*} 
    and 
    \begin{align*}
        \prob\left( \phi(Z_1, Z_2, \ldots, Z_n) - \Expo{\phi(Z_1, Z_2, \ldots, Z_n)} \le -t \right) \le \exp{\left( -\frac{2t^2}{\sum_{i=1}^n c_i^2} \right) } \,.
    \end{align*} 
\end{lemma}


\section{Proofs from \secref{sec:ERM_training}}\label{app:proof_erm}

\textbf{Additional notation {} {}} Let $m_1$ be the number of mislabeled points ($\wt S_M$) and $m_2$ be the number of correctly labeled points ($\wt S_C$). Note $m_1 + m_2 = m$. 


\subsection{Proof of \thmref{thm:error_ERM}}


\begin{proof}[Proof of \lemref{lem:fit_mislabeled}] 
    The main idea of our proof is to regard 
    the clean portion of the data 
    ($S \cup \wt S_C$) as fixed.   
    Then, there exists an (unknown) classifier $f^*$ 
    that minimizes the expected risk
    calculated on the (fixed) clean data
    and (random draws of) the mislabeled data $\wt S_M$. 
    % 
    % 
    Formally, 
    \begin{align}
    f^* \defeq \argmin_{f \in \calF} \error_{\widecheck {\calD}} (f) \,, \label{eq:modified_ERM}
    \end{align}
    where $$\widecheck \calD = \frac{n}{m+n} \calS + \frac{m_2}{m+n} \wt \calS_C  + \frac{m_1}{m+n}\calDm \,.$$ 
    Note here that $\widecheck \calD$ is a combination 
    of the \emph{empirical distribution} 
    over correctly labeled data $S \cup \wt S_C$
    and the (population) distribution 
    over mislabeled data $\calDm$.
    Recall that 
    \begin{align}
    \wh f \defeq \argmin_{f \in \calF} \error_{\calS \cup \wt S} (f) \,. \label{eq:orig_ERM}
    \end{align}
    % 
    % 
    Since, $\widehat f$ minimizes 0-1 error 
    on $S \cup \wt S$, using ERM optimality on \eqref{eq:orig_ERM},  
    we have 
    \begin{align}
        \error_{\calS \cup \wt \calS}(\widehat f) \le \error_{
            \calS \cup \wt \calS}(f^*) \,.    \label{eq:step1}
    \end{align}
    Moreover, since $f^*$ is independent of $\wt S_M$, using Hoeffding's bound,
    % \footnote{For a fully rigorous argument,
    % refer to the complete proof in App.~\ref{app:proof_erm}.} 
    we have with probability at least $1-\delta$ that
    \begin{align}
      \error_{\wt \calS_M}(f^*) \le \error_{ \calDm}(f^*) +  \sqrt{\frac{\log(1/\delta)}{2 m_1}} \,. \label{eq:step2} 
    \end{align}
    %$ 
    %for some constant $c_1\le 1/2$. 
    Finally, since $f^*$ is the optimal classifier on $\widecheck \calD$, 
    we have 
    \begin{align}
        \error_{\widecheck \calD}(f^*) \le \error_{\widecheck \calD}(\widehat f) \,. \label{eq:step3}
    \end{align}
    Now to relate \eqref{eq:step1} and \eqref{eq:step3}, we multiply \eqref{eq:step2} by $\frac{m_1}{m+n}$ and add $\frac{n}{m+n} \error_{\calS} (f)  + \frac{m_2}{m+n} \error_{\wt \calS_C} (f)$ both the sides. Hence, 
    we can rewrite \eqref{eq:step2} as follows: 
    \begin{align}
        \error_{\calS \cup \wt\calS}(f^*) \le \error_{ \widecheck \calD}(f^*) +  \frac{m_1}{m+n}\sqrt{\frac{\log(1/\delta)}{2 m_1}} \,. \label{eq:step4} 
    \end{align}
    Now we combine equations \eqref{eq:step1}, \eqref{eq:step4}, and \eqref{eq:step3}, to get 
    \begin{align}
        \error_{\calS \cup \wt \calS}(\wh f) \le \error_{\widecheck \calD}(\wh f) +  \frac{m_1}{m+n}\sqrt{\frac{\log(1/\delta)}{2 m_1}} \,, 
    \end{align}
    which implies 
    \begin{align}
        \error_{ \wt \calS_M}(\wh f) \le \error_{\calDm}(\wh f) + \sqrt{\frac{\log(1/\delta)}{2 m_1}} \,. \label{eq:lemma1_final}
    \end{align}
    Since $\wt S$ is obtained by randomly labeling an unlabeled dataset, we assume $2m_1 \approx m$ \footnote{Formally, with probability at least $1-\delta$, we have  $(m - 2m_1)\le \sqrt{m\log(1/\delta)/2}$.}. Moreover, using $\error_{\calDm} = 1 - \error_{\calD}$ we obtain the desired result.   
    % Combining the above steps and using the fact 
    % that $\error_\calD = 1- \error_{\calDm} $, 
    % we obtain the desired result.
\end{proof}

\begin{proof}[Proof of \lemref{lem:mislabeled_error}]
    Recall $\error_{\wt S} (f) = \frac{m_1}{m} \error_{\wt S_M}(f) + \frac{m_2}{m} \error_{\wt S_C}(f)$. Hence, we have 
    \begin{align}
        2\error_{\wt S}(f) - \error_{\wt S_M}(f) - \error_{\wt S_C}(f) &= \left(\frac{2m_1}{m} \error_{\wt S_M}(f) - \error_{\wt S_M}(f)\right) + \left(\frac{2m_2}{m} \error_{\wt S_C}(f) - \error_{\wt S_C}(f)\right) \\ &= \left(\frac{2m_1}{m} - 1\right) \error_{\wt S_M}(f) + \left(\frac{2m_2}{m} - 1 \right)\error_{\wt S_C} (f) \,.
    \end{align} 
    Since the dataset is labeled uniformly at random, with probability at least $1-\delta$, we have  $\left(\frac{2m_1}{m} - 1\right) \le \sqrt{\frac{\log(1/\delta)}{2m}}$. Similarly, we have with probability at least $1-\delta$, $\left(\frac{2m_2}{m} - 1\right) \le \sqrt{\frac{\log(1/\delta)}{2m}}$. Using union bound, with probability at least $1-\delta$, we have
    % \begin{align}
    %     2\error_{\wt S} - \error_{\wt S_M}(f) - \error_{\wt S_C}(f) \le \sqrt{\frac{\log(2/\delta)}{2m}} \left(\error_{\wt S_M}(f) + \error_{\wt S_C}(f) \right) \le 2\sqrt{\frac{\log(2/\delta)}{2m}} \,. \label{eq:lemma2_final}
    % \end{align}
    \begin{align}
        2\error_{\wt S} - \error_{\wt S_M}(f) - \error_{\wt S_C}(f) \le \sqrt{\frac{\log(2/\delta)}{2m}} \left(\error_{\wt S_M}(f) + \error_{\wt S_C}(f) \right) \,. \label{eq:lemma2_prefinal}
    \end{align}
    With re-arranging $\error_{\wt S_M}(f) + \error_{\wt S_C}(f)$ and using the inequality $ 1- a\le \frac{1}{1+a} $, we have  
    \begin{align}
        2\error_{\wt S} - \error_{\wt S_M}(f) - \error_{\wt S_C}(f) \le 2\error_{\wt \calS} \sqrt{\frac{\log(2/\delta)}{2m}}  \,. \label{eq:lemma2_final}
    \end{align}

    % We obtain the desired result by using 
\end{proof}

\begin{proof}[Proof of \lemref{lem:clear_error}]
% Recall 0-1 error on each point  $(x,y) \in S \cup \wt S$ is given by $\I{ f(x)\ne y}$.
In the set of correctly labeled points $S \cup \wt S_C$, we have $S$ as a random subset of $S \cup \wt S_C$. Hence, using Hoeffding's inequality for sampling without replacement (\lemref{lem:hoeffding_sampling}), we have with probability at least $1-\delta$
\begin{align}
    \error_{\wt \calS_C} (\wh f)- \error_{\calS \cup \wt \calS_C}( \wh f) \le  \sqrt{\frac{\log(1/\delta)}{2m_2}} \,.
\end{align}
Re-writing $\error_{\calS \cup \wt \calS_C}( \wh f)$ as $\frac{m_2}{m_2 + n} \error_{\wt \calS_C }(\wh f) + \frac{n}{m_2 + n} \error_{\calS }(\wh f)$, we have with probability at least $1-\delta$
\begin{align}
   \left(\frac{n}{n+m_2}\right) \left(\error_{\wt \calS_C} (\wh f)- \error_{\calS}( \wh f) \right) \le  \sqrt{\frac{\log(1/\delta)}{2m_2}} \,.
\end{align}
As before, assuming $2m_2 \approx m$, we have with probability at least $1-\delta$ 
\begin{align}
    \error_{\wt \calS_C} (\wh f)- \error_{\calS}( \wh f) \le \left(1+\frac{m_2}{n}\right)  \sqrt{\frac{\log(1/\delta)}{m}} \le \left(1 + \frac{m}{2n}\right) \sqrt{\frac{\log(1/\delta)}{m}} \,. \label{eq:lemma3_final}
\end{align} 
\end{proof}

\begin{proof}[Proof of \thmref{thm:error_ERM}] 
    Having established these core intermediate results, we can now combine above three lemmas to prove the main result. 
    In particular, we bound the population error on clean data ($\error_\calD(\wh f)$) as follows:  
    \begin{enumerate}[(i)]
        \item First, use \eqref{eq:lemma1_final}, to obtain an upper bound on the population error on clean data, i.e., with probability at least $1-\delta/4$, we have
        \begin{align}
            \error_{ \calD} (\wh f) \le 1 - \error_{ \wt \calS_M}(\wh f) + \sqrt{\frac{\log(4/\delta)}{m}} \,. 
        \end{align}
        \item  Second, use \eqref{eq:lemma2_final}, to relate the error on the mislabeled fraction with error on clean portion of randomly labeled data and error on whole randomly labeled dataset, i.e., with probability at least $1-\delta/2$, we have 
        \begin{align}
            - \error_{\wt S_M}(f) \le \error_{\wt S_C}(f) - 2\error_{\wt S}  + 2\error_{\wt S} \sqrt{\frac{\log(4/\delta)}{2m}}  \,. 
        \end{align} 
        \item Finally, use \eqref{eq:lemma3_final} to relate the error on the clean portion of randomly labeled data and error on clean training data, i.e., with probability $1-\delta/4$, we have 
        \begin{align}
            \error_{\wt \calS_C} (\wh f)\le - \error_{\calS}( \wh f) + \left(1 + \frac{m}{2n} \right) \sqrt{\frac{\log(4/\delta)}{m}} \,. 
        \end{align} 
    \end{enumerate}

    Using union bound on the above three steps, we have with probability at least $1-\delta$: 
    \begin{align}
        \error_\calD (\wh f) \le \error_{\calS}(\wh f)   + 1 - 2\error_{\wt \calS}(\wh f)   + \left(\sqrt{2} \error_{\wt S} + 2 + \frac{m}{2n}\right)  \sqrt{\frac{\log(4/\delta)}{m}} \,.
    \end{align}
    % Note that $(1/\sqrt{2} + 2.5)$ is a loose constant. In experiments, we use the ratio $\frac{m}{n}$
    %  the exact error $\error_{\wt \calS}(\wh f)$ 
    % to evaluate R.H.S.    
\end{proof}

\subsection{Proof of \propref{prop:rademacher}}

\begin{proof}[Proof of \propref{prop:rademacher}]
    For a classifier $ f: \calX \to \{-1, 1\}$, we have $1 - 2\,\indict{ f(x) \ne y} = y \cdot f(x)$. Hence, by definition of $\error$, we have 
    \begin{align}
        1 -2\error_{\wt \calS}(f) = \frac{1}{m}\sum_{i=1}^m y_i \cdot f(x_i) \le \sup_{f \in \calF} \, \frac{1}{m} \sum_{i=1}^m y_i \cdot f(x_i)  \,. \label{eq:error_rademacher}
    \end{align}
    Note that for fixed inputs $(x_1, x_2, \ldots, x_m)$ in $\wt S$, $(y_1, y_2, \ldots y_m)$ are random labels. Define $\phi_1 (y_1, y_2, \ldots, y_m) \defeq \sup_{f \in \calF} \, \frac{1}{m} \sum_{i=1}^m y_i \cdot f(x_i)$. We have the following bounded difference condition on $\phi_1$. For all i, 
    \begin{align}
        \sup_{y_1, \ldots y_m, y_i^\prime \in \{-1, 1\}^{m+1} } \abs{ \phi_1 (y_1,\ldots, y_i, \ldots, y_m) - \phi_1 (y_1,\ldots, y_i^\prime, \ldots, y_m)  } \le 1/m \,. \label{cond1_rademacher}
    \end{align} 
    
    Similarly, we define $\phi_2 (x_1, x_2, \ldots, x_m) \defeq \Expt{ y_i \sim_U \{-1, 1\}  }{ \sup_{f \in \calF} \, \frac{1}{m}  \sum_{i=1}^m y_i \cdot f(x_i)}$. We have the following bounded difference condition on $\phi_2$. 
    For all i,
    \begin{align}
        \sup_{x_1, \ldots x_m, x_i^\prime \in \calX^{m+1} } \abs{ \phi_2 (x_1,\ldots, x_i, \ldots, x_m) - \phi_1 (x_1,\ldots, x_i^\prime, \ldots, x_m)  } \le 1/m \,. \label{cond2_rademacher}
    \end{align}
    Using McDiarmid’s inequality (\lemref{lem:McDiarmid}) twice 
    with Condition \eqref{cond1_rademacher} and \eqref{cond2_rademacher}, 
    with probability at least $1-\delta$, we have
    \begin{align}
        \sup_{f \in \calF} \, \frac{1}{m} \sum_{i=1}^m y_i \cdot f(x_i)  - \Expt{x,y}{\sup_{f \in \calF} \, \frac{1}{m} \sum_{i=1}^m y_i \cdot f(x_i) } \le \sqrt{\frac{2\log(2/\delta)}{m}} \,. \label{eq:final_rademacher}
    \end{align} 
    Combining \eqref{eq:error_rademacher} and \eqref{eq:final_rademacher}, we obtain the desired result. 
\end{proof}


\subsection{Proof of \thmref{thm:error_regularized_ERM}}

Proof of \thmref{thm:error_regularized_ERM} follows similar to the proof of \thmref{thm:error_ERM}. Note that the same results in \lemref{lem:fit_mislabeled}, \lemref{lem:mislabeled_error}, and \lemref{lem:clear_error} hold in the regularized ERM case. However, the arguments in the proof of \lemref{lem:fit_mislabeled} change slightly. Hence, we state the lemma for regularized ERM and prove it here for completeness. 

\begin{lemma} \label{lem:lemma1_reg}
    Assume the same setup as \thmref{thm:error_regularized_ERM}. 
    Then for any $\delta >0$, with probability at least  $1-\delta$ 
    over the random draws of mislabeled data $\wt S_M$, we have 
    \begin{align}
        \error_\calD(\widehat f)  \le 1 -\error_{\wt \calS_M}(\widehat f) + \sqrt{\frac{\log(1/\delta)}{m}}\,. 
    \end{align} 
\end{lemma}
\begin{proof}
    The main idea of the proof remains the same, i.e. regard 
    the clean portion of the data 
    ($S \cup \wt S_C$) as fixed.   
    Then, there exists a classifier $f^*$ 
    that is optimal over draws 
    of the mislabeled data $\wt S_M$. 

    
    Formally, 
    \begin{align}
    f^* \defeq \argmin_{f \in \calF} \error_{\widecheck {\calD}} (f)  + \lambda R(f) \,, \label{eq:modified_ERM_reg}
    \end{align}
    where $$\widecheck \calD = \frac{n}{m+n} \calS + \frac{m_1}{m+n} \wt \calS_C  + \frac{m_2}{m+n}\calDm \,.$$ That is, $\widecheck \calD$ a combination of 
    the \emph{empirical distribution} 
    over correctly labeled data $S \cup \wt S_C$
    % in $S\cup \wt S$ 
    and the (population) distribution 
    over mislabeled data $\calDm$.
    Recall that 
    \begin{align}
    \wh f \defeq \argmin_{f \in \calF} \error_{\calS \cup \wt S} (f) + \lambda R(f) \,. \label{eq:orig_ERM_reg}
    \end{align}
    % 
    % 
    Since, $\widehat f$ minimizes 0-1 error 
    on $S \cup \wt S$, using ERM optimality on \eqref{eq:orig_ERM},  
    we have 
    \begin{align}
        \error_{\calS \cup \wt \calS}(\widehat f) + \lambda R(\wh f) \le \error_{
            \calS \cup \wt \calS}(f^*) + \lambda R(f^*) \,.    \label{eq:step1_reg}
    \end{align}
    Moreover, since $f^*$ is independent of $\wt S_M$, using Hoeffding's bound,
    % \footnote{For a fully rigorous argument,
    % refer to the complete proof in App.~\ref{app:proof_erm}.} 
    we have with probability at least $1-\delta$ that
    \begin{align}
      \error_{\wt \calS_M}(f^*) \le \error_{ \calDm}(f^*) +  \sqrt{\frac{\log(1/\delta)}{2 m_1}} \,. \label{eq:step2_reg} 
    \end{align}
    %$ 
    %for some constant $c_1\le 1/2$. 
    Finally, since $f^*$ is the optimal classifier on $\widecheck \calD$, 
    we have 
    \begin{align}
        \error_{\widecheck \calD}(f^*) + \lambda R(f^*) \le \error_{\widecheck \calD}(\widehat f) + \lambda R(\wh f) \,. \label{eq:step3_reg}
    \end{align}
     Now to relate \eqref{eq:step1_reg} and \eqref{eq:step3_reg}, we can re-write the \eqref{eq:step2_reg} as follows: 
    \begin{align}
        \error_{\calS \cup \wt\calS}(f^*) \le \error_{ \widecheck \calD}(f^*) +  \frac{m_1}{m+n}\sqrt{\frac{\log(1/\delta)}{2 m_1}} \,. \label{eq:step4_reg} 
    \end{align}
    After adding $\lambda R(f^*)$ on both sides in \eqref{eq:step4_reg}, we combine equations \eqref{eq:step1_reg}, \eqref{eq:step4_reg}, and \eqref{eq:step3_reg}, to get 
    \begin{align}
        \error_{\calS \cup \wt \calS}(\wh f) \le \error_{\widecheck \calD}(\wh f) +  \frac{m_1}{m+n}\sqrt{\frac{\log(1/\delta)}{2 m_1}} \,, 
    \end{align}
    which implies 
    \begin{align}
        \error_{ \wt \calS_M}(\wh f) \le \error_{\calDm}(\wh f) + \sqrt{\frac{\log(1/\delta)}{2 m_1}} \,. \label{eq:lemma_reg_final}
    \end{align}
    Similar as before, since $\wt S$ is obtained by randomly labeling an unlabeled dataset, we assume 
    $2m_1 \approx m$. Moreover, using $\error_{\calDm} = 1 - \error_{\calD}$ we obtain the desired result. 
\end{proof}
% \begin{proof}[Proof of ]
    
% \end{proof}

\subsection{Proof of \thmref{thm:multiclass_ERM}}

To prove our results in the multiclass case,
we first state and prove lemmas
parallel to those
% We first state and prove lemmas 
% parallel 
% to the three lemmas 
used in the proof of balanced binary case. 
We then combine these results 
% in the three lemmas 
to obtain the result in \thmref{thm:multiclass_ERM}. 

Before stating the result, 
we define mislabeled distribution $\calDm$ for any $\calD$.
While $\calDm$ and $\calD$ share 
the same marginal distribution over inputs $\calX$,
the conditional distribution over labels $y$ 
given an input $x\sim \calD_\calX$ is changed as follows:
For any $x$, the Probability Mass Function (PMF) over $y$ is defined as:  
$p_{\calDm} (\cdot \vert x) \defeq \frac{1 - p_{\calD}(\cdot \vert x)}{k - 1}$, where $ p_{\calD}(\cdot \vert x)$ is the PMF over $y$ for the distribution $\calD$. 

\begin{lemma} \label{lem:fit_mislabeled_multi}
    Assume the same setup as \thmref{thm:multiclass_ERM}. 
    Then for any $\delta >0$, with probability at least  $1-\delta$ 
    over the random draws of mislabeled data $\wt S_M$, we have 
    \begin{align}
        \error_\calD(\widehat f)  \le (k-1)\left(1 -\error_{\wt \calS_M}(\widehat f)\right) + (k-1)\sqrt{\frac{\log(1/\delta)}{m}}\,. \label{eq:lemma1_multi}
    \end{align}   
\end{lemma} 

\begin{proof}
   
    The main idea of the proof remains the same.
    We begin by regarding the clean portion of the data 
    ($S \cup \wt S_C$) as fixed. 
    Then, there exists a classifier $f^*$ 
    that is optimal over draws 
    of the mislabeled data $\wt S_M$. 
    
    However, in the multiclass case,
    we cannot as easily relate the population error on mislabeled data 
    to the population accuracy on clean data.   
    While for binary classification, 
    % we could upper bound $\error_{\wt \calS_M}$ 
    % with $1-\error_\calD$ 
    we could lower bound the population accuracy $1-\error_\calD$
    with the empirical error on mislabeled data $\error_{\wt \calS_M}$ 
    (in the proof of \lemref{lem:fit_mislabeled}), 
    for multiclass classification, 
    error on the mislabeled data 
    and accuracy on the clean data 
    in the population 
    are not so directly related.  
    To establish \eqref{eq:lemma1_multi},
    we break the error on the 
    (unknown) mislabeled data 
    into two parts: one term corresponds 
    to predicting the true label on mislabeled data, 
    and the other corresponds to predicting 
    neither the true label 
    nor the assigned (mis-)label.  
    Finally, we relate these errors to their
    population counterparts to establish \eqref{eq:lemma1_multi}. 
    
    Formally, 
    \begin{align}
    f^* \defeq \argmin_{f \in \calF} \error_{\widecheck {\calD}} (f)  + \lambda R(f) \,, \label{eq:modified_ERM_reg2}
    \end{align}
    where $$\widecheck \calD = \frac{n}{m+n} \calS + \frac{m_1}{m+n} \wt \calS_C  + \frac{m_2}{m+n}\calDm \,.$$ 
    That is, $\widecheck \calD$ is a combination 
    of the \emph{empirical distribution} 
    over correctly labeled data $S \cup \wt S_C$
    % in $S\cup \wt S$ 
    and the (population) distribution 
    over mislabeled data $\calDm$.
    Recall that 
    \begin{align}
    \wh f \defeq \argmin_{f \in \calF} \error_{\calS \cup \wt S} (f) + \lambda R(f) \,. \label{eq:orig_ERM_reg2}
    \end{align}
    % 
    % 
    Following the exact steps from the proof of \lemref{lem:lemma1_reg}, 
    with probability at least $1-\delta$, we have  
    \begin{align}
        \error_{ \wt \calS_M}(\wh f) \le \error_{\calDm}(\wh f) + \sqrt{\frac{\log(1/\delta)}{2 m_1}} \,. \label{eq:lemma1_final_multi_prev}
    \end{align}
    Similar to before, since $\wt S$ is obtained 
    by randomly labeling an unlabeled dataset, 
    we assume 
    $\frac{k}{k-1} m_1 \approx m$. 
    
    Now we will relate $\error_{\calDm} (\wh f)$ with $\error_{\calD}(\wh f)$. 
    Let $y^T$ denote the (unknown) true label 
    for a mislabeled point $(x, y)$ 
    (i.e., label before replacing it with a mislabel). 
    \begin{align*}    
         \Expt{(x, y) \in \sim \calDm}{\indict{ \wh f(x) \ne y }}  &= \underbrace{\Expt{(x, y) \in \sim \calDm}{\indict{ \wh f(x) \ne y \land \wh f(x) \ne y^T}}}_{\RN{1}} \\ &\qquad \qquad + \underbrace{\Expt{(x, y) \in \sim \calDm}{\indict{ \wh f(x) \ne y \land \wh f(x) = y^T}}}_{\RN{2}} \,. \numberthis \label{eq:excess_term}
    \end{align*}
    Clearly, term 2 is one minus the accuracy 
    on the clean unseen data, i.e.,
    \begin{align}
        \RN{2} = 1 - \Expt{{x,y} \sim \calD}{ \indict{ \wh f(x) \ne y}} = 1- \error_{\calD}(\wh f) \,. \label{eq:term1}    
    \end{align}
    Next, we relate term 1 with the error on the unseen clean data. 
    We show that term 1 is equal to the error on the unseen clean data 
    scaled by $\frac{k-2}{k-1}$,
    where $k$ is the number of labels.
    Using the definition of mislabeled distribution $\calDm$,  
    we have 
    \begin{align}
        \RN{1} = \frac{1}{k-1} \left( \Expt{(x, y) \in \sim \calD}{ \sum_{i \in \calY \land i\ne y}  \indict{ \wh f(x) \ne i \land \wh f(x) \ne y}} \right) = \frac{k-2}{k-1} \error_{\calD}(\wh f) \,.\label{eq:term2}
    \end{align}    

    Combining the result in \eqref{eq:term1}, \eqref{eq:term2} and \eqref{eq:excess_term}, we have 
    \begin{align}
        \error_{\calDm}(\wh f) = 1- \frac{1}{k-1} \error_{\calD}(\wh f) \,.\label{eq:combine_terms}
    \end{align}
    Finally, combining the result in \eqref{eq:combine_terms} 
    with equation \eqref{eq:lemma1_final_multi_prev}, 
    we have with probability $1-\delta$, 
    \begin{align}
      \error_{\calD}(\wh f) \le  (k-1) \left( 1- \error_{ \wt \calS_M}(\wh f) \right)  + (k-1) \sqrt{\frac{k \log(1/\delta)}{ 2(k-1)m}} \,. \label{eq:lemma1_final_multi}
    \end{align}
\end{proof}

\begin{lemma} \label{lem:mislabeled_error_multi}
    Assume the same setup as \thmref{thm:multiclass_ERM}. 
    Then for any $\delta >0$, 
    with probability at least $1-\delta$ 
    over the random draws of $\wt S$, we have  
    % \begin{align}
        $$\abs{k\error_{\wt \calS}(\widehat f) - \error_{\wt \calS_C}(\widehat f) -  (k-1)\error_{\wt \calS_M}(\widehat f) } \le  2k\sqrt{\frac{\log(4/\delta)}{2m}}\,. $$ % \label{eq:lemma2}
    % \end{align}   
    %  for some constant $c_3 \le 1.0\,$.
\end{lemma} 


\begin{proof}
    Recall $\error_{\wt S} (f) = \frac{m_1}{m} \error_{\wt S_M}(f) + \frac{m_2}{m} \error_{\wt S_C}(f)$. Hence, we have 
    \begin{align*}
        k\error_{\wt S}(f) - (k-1)\error_{\wt S_M}(f) - \error_{\wt S_C}(f) &= (k-1)\left(\frac{k m_1}{(k-1) m} \error_{\wt S_M}(f) - \error_{\wt S_M}(f)\right) \\ & \qquad \qquad + \left(\frac{km_2}{m} \error_{\wt S_C}(f) - \error_{\wt S_C}(f)\right) \\ &= k \left[ \left(\frac{m_1}{m} - \frac{k-1}{k}\right) \error_{\wt S_M}(f) + \left(\frac{m_2}{m} - \frac{1}{k} \right) \error_{\wt S_C} (f) \right] \,.
    \end{align*} 
    Since the dataset is randomly labeled, 
    we have with probability at least $1-\delta$, 
    $\left(\frac{m_1}{m} - \frac{k-1}{k}\right) \le \sqrt{\frac{\log(1/\delta)}{2m}}$. 
    Similarly, we have with probability at least $1-\delta$, 
    $\left(\frac{m_2}{m} - \frac{1}{k}\right) \le \sqrt{\frac{\log(1/\delta)}{2m}}$. 
    Using union bound, we have with probability at least $1-\delta$
    % \begin{align}
    %     2\error_{\wt S} - \error_{\wt S_M}(f) - \error_{\wt S_C}(f) \le \sqrt{\frac{\log(2/\delta)}{2m}} \left(\error_{\wt S_M}(f) + \error_{\wt S_C}(f) \right) \le 2\sqrt{\frac{\log(2/\delta)}{2m}} \,. \label{eq:lemma2_final}
    % \end{align}
    \begin{align}
        k\error_{\wt S}(f) - (k-1)\error_{\wt S_M}(f) - \error_{\wt S_C}(f)  \le k \sqrt{\frac{\log(2/\delta)}{2m}} \left(\error_{\wt S_M}(f) + \error_{\wt S_C}(f) \right) \,. \label{eq:lemma2_final_multi}
    \end{align}

    % We obtain the desired result by using 
\end{proof}

\begin{lemma} \label{lem:clear_error_multi}
    Assume the same setup as \thmref{thm:multiclass_ERM}. 
    Then for any $\delta >0$, with probability at least $1-\delta$ 
    over the random draws of $\wt S_C$ and $S$, we have 
    % \begin{align}
        $$\abs{\error_{\wt \calS_C}(\widehat f) - \error_{\calS}(\widehat f) } \le 1.5 \sqrt{\frac{k\log(2/\delta)}{2m}}\,.$$ %\label{eq:lemma3}
    % \end{align}   
    % for some constant $c_2 \le 1.2\,$.
\end{lemma} 
\begin{proof}
    % Recall 0-1 error on each point  $(x,y) \in S \cup \wt S$ is given by $\I{ f(x)\ne y}$.
    In the set of correctly labeled points $S \cup \wt S_C$,
    we have $S$ as a random subset of $S \cup \wt S_C$. 
    Hence, using Hoeffding's inequality 
    for sampling without replacement 
    (\lemref{lem:hoeffding_sampling}), 
    we have with probability at least $1-\delta$
    \begin{align}
        \error_{\wt \calS_c} (\wh f)- \error_{\calS \cup \wt \calS_C}( \wh f) \le  \sqrt{\frac{\log(1/\delta)}{2m_2}} \,.
    \end{align}
    Re-writing $\error_{\calS \cup \wt \calS_C}( \wh f)$ 
    as $\frac{m_2}{m_2 + n} \error_{\wt \calS_C }(\wh f) + \frac{n}{m_2 + n} \error_{\calS }(\wh f)$, 
    we have with probability at least $1-\delta$
    \begin{align}
       \left(\frac{n}{n+m_2}\right) \left(\error_{\wt \calS_c} (\wh f)- \error_{\calS}( \wh f) \right) \le  \sqrt{\frac{\log(1/\delta)}{2m_2}} \,.
    \end{align}
    As before, assuming $km_2 \approx m$, 
    we have with probability at least $1-\delta$ 
    \begin{align}
        \error_{\wt \calS_c} (\wh f)- \error_{\calS}( \wh f) \le \left(1+\frac{m_2}{n}\right)  \sqrt{\frac{k\log(1/\delta)}{2m}} \le \left( 1 + \frac{1}{k}\right) \sqrt{\frac{k\log(1/\delta)}{2m}} \,. \label{eq:lemma3_final_multi}
    \end{align} 
\end{proof}

\begin{proof}[Proof of \thmref{thm:multiclass_ERM}] 
    Having established these core intermediate results, 
    we can now combine above three lemmas. 
    In particular, we bound the population error 
    on clean data ($\error_\calD(\wh f)$) as follows:  
    \begin{enumerate}[(i)]
        \item First, use \eqref{eq:lemma1_final_multi}, 
        to obtain an upper bound on the population error on clean data, 
        i.e., with probability at least $1-\delta/4$, we have
        \begin{align}
            \error_{ \calD} (\wh f) \le (k-1)\left(1 - \error_{ \wt \calS_M}(\wh f) \right) + (k-1) \sqrt{\frac{k\log(4/\delta)}{2(k-1)m}} \,. 
        \end{align}
        \item  Second, use \eqref{eq:lemma2_final_multi}
        to relate the error on the mislabeled fraction 
        with error on clean portion of randomly labeled data 
        and error on whole randomly labeled dataset, 
        i.e., with probability at least $1-\delta/2$, we have 
        \begin{align}
            - (k-1)\error_{\wt S_M}(f) \le \error_{\wt S_C}(f) - k\error_{\wt S}  + k\sqrt{\frac{\log(4/\delta)}{2m}}  \,. 
        \end{align} 
        \item Finally, use \eqref{eq:lemma3_final_multi} 
        to relate the error on the clean portion of randomly labeled data 
        and error on clean training data, 
        i.e., with probability $1-\delta/4$, we have 
        \begin{align}
            \error_{\wt \calS_C} (\wh f)\le - \error_{\calS}( \wh f) + \left(1 + \frac{m}{kn} \right) \sqrt{\frac{k\log(4/\delta)}{2m}} \,. 
        \end{align} 
    \end{enumerate}

    Using union bound on the above three steps, 
    we have with probability at least $1-\delta$: 
    \begin{align}
        \error_\calD (\wh f) \le \error_{\calS}(\wh f) + (k-1) - k\error_{\wt \calS}(\wh f)   + (\sqrt{k(k-1)} + k + \sqrt{k} + \frac{m}{n\sqrt{k}})  \sqrt{\frac{\log(4/\delta)}{2m}} \,.\label{eq:multiclass_ERM_final}
    \end{align}
    Simplifying the term in RHS of \eqref{eq:multiclass_ERM_final}, 
    we get the desired result. 
    % Note that since $\frac{m}{n\sqrt{k}}$ 
    % is much smaller than the sum of the other terms
    % the other terms in summation, 
    % we ignore $\frac{m}{n\sqrt{k}}$  
    % Z: ??? --- great
    % that 
    % them
    in the final bound. 
    % we ignore that in the final bound. 
    % Note that $(1/\sqrt{2} + 2.5)$ is a loose constant. In experiments, we use the ratio $\frac{m}{n}$
    %  the exact error $\error_{\wt \calS}(\wh f)$ 
    % to evaluate R.H.S.    
\end{proof}

\newpage
\section{Proofs from \secref{sec:linear_models}}\label{app:proof_gd}
We suppose that the parameters of the linear function 
are obtained via gradient descent on 
the following $L_2$ regularized problem: 
\begin{align}
    % n in denominator is avoided deliberately
    \calL_S(w; \lambda) \defeq \sum_{i=1}^n{(w^Tx_i - y_i)^2} + \lambda \norm{w}{2}^2 \,, \label{eq:l2_MSE_app}   
\end{align}
where $\lambda\ge0$ is a regularization parameter. 
We assume access to a clean dataset 
$S = \{(x_i, y_i)\}_{i=1}^n \sim \calD^n$ 
and randomly labeled dataset 
$\wt S = \{(x_i, y_i)\}_{i=n+1}^{n+m} \sim \wt \calD^m$. 
Let $\bX = [x_1, x_2, \cdots, x_{m+n}]$ 
and $\by = [y_1, y_2, \cdots, y_{m+n}]$. 
Fix a positive learning rate $\eta$ such that 
$\eta \le 1/\left(\norm{\bX^T\bX}{\text{op}} + \lambda^2\right)$ 
and an initialization $w_0 = 0$. 
% \todos{Assumption made for simplicty}. 
Consider the following gradient descent iterates 
to minimize objective \eqref{eq:l2_MSE_app} on $S \cup \wt S$:
\begin{align}
w_t = w_{t-1} - \eta \grad_w \calL_{S \cup \wt S} (w_{t-1}; \lambda) \quad \forall t=1,2,\ldots \label{eq:GD_iterates_app}
\end{align} 
Then we have $\{ w_t\}$ converge to the limiting solution 
$\wh w = \left( \bX^T\bX+\lambda \boldsymbol{I}\right)^{-1}\bX^T\by$. Define $\widehat f (x) \defeq f(x ; \wh w) $.  

% \subsection{\textcolor{red}{Errata}}

% We wish to correct the following error in the body:
% \codref{cond:error_stability} is not enough 
% to guarantee the result in \thmref{thm:linear}. 
% We now present a slightly stronger condition 
% called \emph{hypothesis stability} 
% under which we obtain a result 
% similar to \thmref{thm:linear}. 

% This error doesn't change the main arguments of the proof,
% where we show that the empirical train error 
% is less than or equal to the leave-one-out error.
% We need a stronger condition to relate leave-one-out error 
% with the population error of the original classifier. 
% Specifically, while \codref{cond:error_stability} 
% relates the average population error of leave-one-out classifiers 
% with the population error of the original classifier, 
% we need the new condition to show the concentration 
% of the empirical leave-one-out error 
% and average population error of leave-one-out classifiers. 
% main takeaway 

% Note that the new condition, 
% while being stronger than the previous one, 
% still doesn't imply generalization \citep{bousquet2002stability,elisseeff2003leave,abou2019exponential}. 
% Overall, the main results in \secref{sec:ERM_training} 
% and takeaways of the paper remain unaffected by the error.  

% We now present the new condition 
% and a corrected statement of \thmref{thm:linear}. 
% Recall, for a given training set $S \sim \calD^n $, 
% we use $S_{(i)}$ to denote the training set $S$ 
% with the $i^{\text{th}}$ point removed.

% \begin{condition}[Hypothesis Stability] 
%     \label{cond:hypothesis_stability}
%     We have $\beta$ hypothesis stability 
%     if our training algorithm $\calA$ satisfies the following: 
%     \begin{align*}
%     % ${\sum_{i=1}^n \frac{\error_{\calD}( f(\calA, S_{(i)}))}{n} - \error_\calD(f(\calA, S))} \le \beta\,$.
%     \forall i \in \{1,2,\ldots, n\}, \quad  \Expt{\calS, (x,y) \in \calD}{ \abs{\error\left( f(x) ,y  \right) - \error\left( f_{(i)}(x), y \right) }} \le \frac{\beta}{n} \,,
%     \end{align*}
%     where $f_{(i)} \defeq f(\calA, S_{(i)})$ and $ f \defeq f(\calA, S)$.
% \end{condition}

% \begin{theorem}[Correct statement of \thmref{thm:linear}] \label{thm:new_linear}
%     Assume that this gradient descent algorithm satisfies \codref{cond:hypothesis_stability}
%     with $\beta=\calO(1)$.  
%     Then for any $\delta >0$, with probability at least $1-\delta$ 
%     over the random draws of datasets $\wt S$ and $S$, we have:
%     \begin{align}
%         \error_\calD(\widehat f) \le \error_\calS(\widehat f) + 1 - 2 \error_{\wt\calS}(\widehat f) + \left(\frac{1}{\sqrt{2}} + 1.5 \right) \sqrt{\frac{\log(4/\delta)}{m}} + \sqrt{\frac{4}{\delta}\left(\frac{1}{m} +\frac{3\beta}{m+n} \right)}  \,. \label{eq:gd_error}
%     \end{align} 
%     % for some constant $c\le 3.2$.
% \end{theorem}

\subsection{Proof of \thmref{thm:linear}}
We use a standard result from linear algebra, 
namely the Shermann-Morrison formula 
\citep{sherman1950adjustment} for matrix inversion:  

\begin{lemma}[\citet{sherman1950adjustment}] \label{lem:sherman}
    Suppose $\bA \in \Real^{n \times n}$ 
    is an invertible square matrix 
    and $u,v \in \Real^n$ are column vectors. 
    Then $\bA + uv^T$ is invertible iff $1 + v^T \bA u \ne 0$ 
    and in particular
    \begin{align}
        (\bA + u v^T)^{-1} = \bA^{-1}  - \frac{\bA^{-1} uv^T \bA^{-1} }{ 1 + v^T \bA^{-1} u} \,.
    \end{align}   
\end{lemma}
\newcommand\byy[1]{\by_{\left(#1\right)}}
\newcommand\bXX[1]{\bX_{\left(#1\right)}}
\newcommand\ff[1]{\wh f_{\left(#1\right)}}

For a given training set $S \cup \wt S_C$, 
define leave-one-out error 
on mislabeled points in the training data 
as $$\error_{\text{LOO}(\wt S_M) } = \frac{\sum_{(x_i, y_i) \in \wt S_M} \error( f_{(i)}( x_i), y_i)}{ \abs{\wt S_M }} \,, $$
where $f_{(i)} \defeq f(\calA, (S \cup \wt S)_{(i)})$. 
To relate empirical leave-one-out error and population error 
with hypothesis stability condition, 
we use the following lemma:   

\begin{lemma}[\citet{bousquet2002stability}] \label{lem:stability_error}
    For the leave-one-out error, we have
    \begin{align}
        \Expo{ \left( \error_{\calDm}(\wh f) -\error_{\text{LOO}(\wt S_M) } \right)^2 } \le \frac{1}{2m_1}+  \frac{3\beta}{n + m}\,.
    \end{align}   
    % where $ f \defeq f(\calA, S \cup \wt S) $.
\end{lemma}

Proof of the above lemma is similar 
to the proof of Lemma 9 in \citet{bousquet2002stability} 
and can be found in \appref{app:proof_lem_error}. 
% 
% Before presenting the result, we introduce some notation. 
Before presenting the proof of \thmref{thm:linear}, 
we introduce some more notation. 
Let $\bX_{(i)}$ denote the matrix of covariates 
with the $i^{\text{th}}$ point removed. 
Similarly, let $\by_{(i)}$ be the array of responses 
with the $i^{\text{th}}$ point removed. 
Define the corresponding regularized GD solution 
as $\wh w_{(i)} = \left( \bXX{i}^T\bXX{i}+\lambda \boldsymbol{I}\right)^{-1}\bXX{i}^T\byy{i}$. 
Define $\ff{i}(x) \defeq f(x ; \wh w_{(i)}) $.

\begin{proof}[Proof of \thmref{thm:linear}]
    Because squared loss minimization does not imply 0-1 error minimization, 
    we cannot use arguments from \lemref{lem:fit_mislabeled}. 
    This is the main technical difficulty. 
    To compare the 0-1 error at a train point with an unseen point, 
    we use the closed-form expression for $\widehat{w}$ 
    and Shermann-Morrison formula 
    to upper bound training error 
    with leave-one-out cross validation error. 
    
    The proof is divided into three parts: 
    In part one, we show that 0-1 error 
    on mislabeled points in the training set 
    is lower than the error obtained 
    by leave-one-out error at those points. 
    In part two, we relate this leave-one-out error 
    with the population error on mislabeled distribution
    using \codref{cond:hypothesis_stability}.
    While the empirical leave-one-out error is an unbiased estimator 
    of the average population error of leave-one-out classifiers, 
    we need hypothesis stability 
    to control the variance 
    of empirical leave-one-out error. 
    Finally, in part three, we show 
    that the error on the mislabeled training points 
    can be estimated with just the randomly labeled 
    and clean training data (as in proof of \thmref{thm:error_ERM}).  

    \textbf{Part 1 {} {}} First we relate training error with leave-one-out error.        
    For any training point $(x_i, y_i)$ in $\wt S \cup S$, we have 
    \begin{align}
        \error(\wh f(x_i), y_i ) &= \indict{ y_i \cdot x_i^T \wh w < 0 } = \indict{ y_i \cdot x_i^T \left( \bX^T\bX+\lambda \boldsymbol{I}\right)^{-1}\bX^T\by < 0 } \\
        &= \indict{ y_i \cdot x_i^T \underbrace{\left( \bXX{i}^T\bXX{i} + x_i ^T x_i +\lambda \boldsymbol{I}\right)^{-1}}_{\RN{1}} (\bXX{i}^T\byy{i} + y_i \cdot x_i) < 0 } \,.
    \end{align}
    Letting $\bA = \left(\bXX{i}^T\bXX{i} +\lambda \boldsymbol{I}\right)$ 
    and using \lemref{lem:sherman} on term 1, we have 
    \begin{align}
        \error(\wh f(x_i), y_i ) &= \indict{ y_i \cdot x_i^T \left[\bA^{-1} -  \frac{\bA^{-1} x_i x_i^T \bA^{-1}}{ 1 + x_i ^T \bA^{-1} x_i } \right] (\bXX{i}^T\byy{i} + y_i \cdot x_i) < 0 } \\
        &= \indict{ y_i \cdot\left[ \frac{ x_i^T \bA^{-1} ( 1 + x_i ^T \bA^{-1} x_i ) -  x_i^T \bA^{-1} x_i x_i^T \bA^{-1}}{ 1 + x_i ^T \bA ^{-1}x_i } \right] (\bXX{i}^T\byy{i} + y_i \cdot x_i) < 0 } \\
        &= \indict{ y_i \cdot\left[ \frac{ x_i^T \bA^{-1}}{ 1 + x_i ^T \bA ^{-1}x_i } \right] (\bXX{i}^T\byy{i} + y_i \cdot x_i) < 0 } \,.
    \end{align}

    Since $1 + x_i^T \bA^{-1} x_i > 0$, we have 
    \begin{align}
        \error(\wh f(x_i), y_i ) &= \indict{ y_i \cdot x_i^T \bA^{-1} (\bXX{i}^T\byy{i} + y_i \cdot x_i) < 0 } \\
        &= \indict{ x_i^T \bA^{-1} x_i +  y_i \cdot x_i^T \bA^{-1} (\bXX{i}^T\byy{i}) < 0 } \\
        &\le \indict{ y_i \cdot x_i^T \bA^{-1} (\bXX{i}^T\byy{i}) < 0 } = \error(\ff{i}(x_i), y_i ) \,.\label{eq:LOO_error}
    \end{align}

    Using \eqref{eq:LOO_error}, we have 
    \begin{align}
        \error_{\wt \calS_M } (\wh f) \le \error_{\text{LOO} (\wt S_M)} \defeq \frac{\sum_{(x_i, y_i) \in \wt S_M} \error(\ff{i}(x_i), y_i ) }{\abs{\wt \calS_M}}\label{eq:LOO_error_final} \,.
    \end{align}
    \textbf{Part 2 {}{}} We now relate RHS in \eqref{eq:LOO_error_final} 
    with the population error on mislabeled distribution. 
    To do this, we leverage \codref{cond:hypothesis_stability} 
    and \lemref{lem:stability_error}. 
    In particular, we have 

    \begin{align}
        \Expt{\calS \cup \wt \calS_M }{ \left(\error_{\calDm}(\wh f) - \error_{\text{LOO} (\wt S_M)}\right)^2 } \le \frac{1}{2m_1} + \frac{3\beta}{m+n} \,.
    \end{align}

    Using Chebyshev's inequality, with probability at least $1-\delta$, we have 
    \begin{align}
        \error_{\text{LOO} (\wt S_M)} \le  \error_{\calDm}(\wh f)   + \sqrt{\frac{1}{\delta}\left(\frac{1}{2m_1} +\frac{3\beta}{m+n} \right)} \,. \label{eq:final_mislabeled_linear}
    \end{align}
    

    \textbf{Part 3 {}{}} Combining \eqref{eq:final_mislabeled_linear} and \eqref{eq:LOO_error_final}, we have 

    \begin{align}
        \error_{\wt \calS_M } (\wh f) \le \error_{\calDm}(\wh f)   + \sqrt{\frac{1}{\delta}\left(\frac{1}{2m_1} +\frac{3\beta}{m+n} \right)} \,. \label{eq:linear_parallel_lem1}
    \end{align}

    Compare \eqref{eq:linear_parallel_lem1} with \eqref{eq:lemma1_final} 
    in the proof of \lemref{lem:fit_mislabeled}. 
    We obtain a similar relationship 
    between $\error_{\wt \calS_M }$ and $\error_{\calDm}$ 
    but with a polynomial concentration 
    instead of exponential concentration. 
    In addition, since we just use concentration arguments 
    to relate mislabeled error to the errors
    on the clean and unlabeled portions 
    of the randomly labeled data, 
    we can directly use the results 
    in \lemref{lem:mislabeled_error} and \lemref{lem:clear_error}. 
    Therefore, combining results in \lemref{lem:mislabeled_error}, \lemref{lem:clear_error}, and \eqref{eq:linear_parallel_lem1} with union bound, 
    we have with probability at least $1-\delta$
    \begin{align}
        \error_\calD(\widehat f) \le \error_\calS(\widehat f) + 1 - 2 \error_{\wt\calS}(\widehat f) + \left(\sqrt{2}\error_{\wt\calS}(\widehat f) + 1 + \frac{m}{2n} \right) \sqrt{\frac{\log(4/\delta)}{m}} + \sqrt{\frac{4}{\delta}\left(\frac{1}{m} +\frac{3\beta}{m+n} \right)}  \,.
    \end{align}
    

       
\end{proof}

\subsection{Extension to multiclass classification} \label{app:multiclass_linear}
For multiclass problems with squared loss minimization, as standard practice, we consider one-hot encoding for the underlying label, i.e., a class label $c \in [k]$ is treated as $(0, \cdot, 0,1,0, \cdot, 0) \in \Real^k$ (with $c$-th coordinate being 1).  As before, we suppose that the parameters of the linear function 
are obtained via gradient descent on the following $L_2$ regularized problem: 
\begin{align}
    % n in denominator is avoided deliberately
    \calL_S(w; \lambda) \defeq \sum_{i=1}^n\norm{w^Tx_i - y_i}{2}^2 + \lambda \sum_{j=1}^k \norm{w_j}{2}^2 \,, \label{eq:l2_multiclass_MSE_app}   
\end{align}
where $\lambda\ge0$ is a regularization parameter. 
We assume access to a clean dataset 
$S = \{(x_i, y_i)\}_{i=1}^n \sim \calD^n$ 
and randomly labeled dataset 
$\wt S = \{(x_i, y_i)\}_{i=n+1}^{n+m} \sim \wt \calD^m$. 
Let $\bX = [x_1, x_2, \cdots, x_{m+n}]$ 
and $\by = [e_{y_1}, e_{y_2}, \cdots, e_{y_{m+n}}]$. 
Fix a positive learning rate $\eta$ such that 
$\eta \le 1/\left(\norm{\bX^T\bX}{\text{op}} + \lambda^2\right)$ 
and an initialization $w_0 = 0$. 
% \todos{Assumption made for simplicty}. 
Consider the following gradient descent iterates 
to minimize objective \eqref{eq:l2_MSE_app} on $S \cup \wt S$:
\begin{align}
{w_j}^t = {w_j}^{t-1} - \eta \grad_{w_j} \calL_{S \cup \wt S} (w^{t-1}; \lambda) \quad \forall t=1,2,\ldots \text{ and } j=1,2,\ldots,k  \,. \label{eq:GD_multi_iterates_app}
\end{align} 
Then we have $\{ {w_j}^t\}$ for all $j =1,2,\cdots, k$ converge to the limiting solution 
$\wh w_j = \left( \bX^T\bX+\lambda \boldsymbol{I}\right)^{-1}\bX^T\by_j$. Define $\widehat f (x) \defeq f(x ; \wh w) $.  

\begin{theorem}\label{thm:multi_linear}
    Assume that this gradient descent algorithm satisfies \codref{cond:hypothesis_stability}
    with $\beta=\calO(1)$.  
    Then for a multiclass classification problem wth $k$ classes, for any $\delta >0$, with probability at least $1-\delta$, we have:
    \begin{align*}
        \error_\calD(\widehat f) \le \error_\calS(\widehat f) &+ (k-1)\left(1 - \frac{k}{k-1} \error_{\wt\calS}(\widehat f) \right) \\ &+ \left(k + \sqrt{k} + \frac{m}{n\sqrt{k}} \right) \sqrt{\frac{\log(4/\delta)}{2m}} + \sqrt{k(k-1)} \sqrt{\frac{4}{\delta}\left(\frac{1}{m} +\frac{3\beta}{m+n} \right)}  \,. \numberthis \label{eq:gd_multi_error}
    \end{align*} 
    % for some constant $c\le 3.2$.
\end{theorem}
\begin{proof}
    The proof of this theorem is divided into two parts. In the first part, we relate the error on the mislabeled samples with the population error on the mislabeled data. Similar to the proof of \thmref{thm:linear}, we use Shermann-Morrison formula to upper bound training error with leave-one-out error on each $\wh w^j$. Second part of the proof follows entirely from the proof of \thmref{thm:multiclass_ERM}. In essence, the first part derives an equivalent of \eqref{eq:lemma1_final_multi_prev} for GD training with squared loss and then the second part follows from the proof  of \thmref{thm:multiclass_ERM}. 
    
    \textbf{Part-1:} Consider a training point $(x_i,y_i)$ in $\wt S \cup S $. For simplicity, we use $c_i$ to denote the class of $i$-th point and use $y_i$ as the corresponding one-hot embedding. Recall error in multiclass point is given by $\error(\wh f(x_i), y_i ) = \indict{ c_i \not \in \argmax x_i^T \wh w }$. Thus, there exists a $j \ne c_i \in [k]$, such that we have
     \begin{align}
        \error(\wh f(x_i), y_i ) &= \indict{ c_i \not \in \argmax x_i^T \wh w } = \indict{ x_i^T \wh w_{c_i} < x_i^T \wh w_{j}  } \\ &= \indict{ x_i^T \left( \bX^T\bX+\lambda \boldsymbol{I}\right)^{-1}\bX^T\by_{c_i} < x_i^T \left( \bX^T\bX+\lambda \boldsymbol{I}\right)^{-1}\bX^T\by_{j} } \\
        &= \indict{ x_i^T \underbrace{\left( \bXX{i}^T\bXX{i} + x_i ^T x_i +\lambda \boldsymbol{I}\right)^{-1}}_{\RN{1}} \left(\bXX{i}^T{\by_{c_i}}_{(i)} + x_i - \bXX{i}^T{\by_{j}}_{(i)}\right) < 0 } \,.
    \end{align}
    Letting $\bA = \left(\bXX{i}^T\bXX{i} +\lambda \boldsymbol{I}\right)$ 
    and using \lemref{lem:sherman} on term 1, we have 
    \begin{align}
        \error(\wh f(x_i), y_i ) &= \indict{ x_i^T \left[\bA^{-1} -  \frac{\bA^{-1} x_i x_i^T \bA^{-1}}{ 1 + x_i ^T \bA^{-1} x_i } \right]  \left(\bXX{i}^T{\by_{c_i}}_{(i)} + x_i - \bXX{i}^T{\by_{j}}_{(i)}\right) < 0 } \\
        &= \indict{ \left[ \frac{ x_i^T \bA^{-1} ( 1 + x_i ^T \bA^{-1} x_i ) -  x_i^T \bA^{-1} x_i x_i^T \bA^{-1}}{ 1 + x_i ^T \bA ^{-1}x_i } \right]  \left(\bXX{i}^T{\by_{c_i}}_{(i)} + x_i - \bXX{i}^T{\by_{j}}_{(i)}\right) < 0 } \\
        &= \indict{ \left[ \frac{ x_i^T \bA^{-1}}{ 1 + x_i ^T \bA ^{-1}x_i } \right]  \left(\bXX{i}^T{\by_{c_i}}_{(i)} + x_i - \bXX{i}^T{\by_{j}}_{(i)}\right) < 0} \,.
    \end{align}
    Since $1 + x_i^T \bA^{-1} x_i > 0$, we have 
    \begin{align}
        \error(\wh f(x_i), y_i ) &= \indict{ x_i^T \bA^{-1}  \left(\bXX{i}^T{\by_{c_i}}_{(i)} + x_i - \bXX{i}^T{\by_{j}}_{(i)}\right) < 0 } \\
        &= \indict{ x_i^T \bA^{-1} x_i +  x_i^T \bA^{-1}  \bXX{i}^T{\by_{c_i}}_{(i)}  - x_i^T\bA^{-1}  \bXX{i}^T{\by_{j}}_{(i)} < 0 } \\
        &\le \indict{  x_i^T \bA^{-1}  \bXX{i}^T{\by_{c_i}}_{(i)}  - x_i^T\bA^{-1}  \bXX{i}^T{\by_{j}}_{(i)} < 0  } = \error(\ff{i}(x_i), y_i ) \,.\label{eq:LOO_error_multi}
    \end{align}
    Using \eqref{eq:LOO_error_multi}, we have 
    \begin{align}
        \error_{\wt \calS_M } (\wh f) \le \error_{\text{LOO} (\wt S_M)} \defeq \frac{\sum_{(x_i, y_i) \in \wt S_M} \error(\ff{i}(x_i), y_i ) }{\abs{\wt \calS_M}}\label{eq:LOO_error_multi_final} \,.
    \end{align}
    
    We now relate RHS in \eqref{eq:LOO_error_final} 
    with the population error on mislabeled distribution. 
    Similar as before, to do this, we leverage \codref{cond:hypothesis_stability} 
    and \lemref{lem:stability_error}. Using  \eqref{eq:final_mislabeled_linear} and \eqref{eq:LOO_error_multi_final}, we have 
    \begin{align}
        \error_{\wt \calS_M } (\wh f) \le \error_{\calDm}(\wh f)   + \sqrt{\frac{1}{\delta}\left(\frac{1}{2m_1} +\frac{3\beta}{m+n} \right)} \,. \label{eq:linear_multi_parallel_lem1}
    \end{align}
    
    We have now derived a parallel to \eqref{eq:lemma1_final_multi_prev}. Using the same arguments in the proof of \lemref{lem:fit_mislabeled_multi}, we have 
    \begin{align}
      \error_{\calD}(\wh f) \le  (k-1) \left( 1- \error_{ \wt \calS_M}(\wh f) \right)  + (k-1)\sqrt{\frac{k}{\delta(k-1)}\left(\frac{1}{2m_1} +\frac{3\beta}{m+n} \right)}  \,. \label{eq:lemma1_linear_final_multi}
    \end{align}
    
    \textbf{Part-2:} We now combine the results in \lemref{lem:mislabeled_error_multi} and \lemref{lem:clear_error_multi} to obtain the final inequality in terms of quantities that can be computed from just the randomly labeled and clean data. Similar to the binary case, we obtained a polynomial concentration instead of exponential concentration. Combining \eqref{eq:lemma1_linear_final_multi} with \lemref{lem:mislabeled_error_multi} and \lemref{lem:clear_error_multi}, we have with probability at least $1-\delta$
    \begin{align*}
        \error_\calD(\widehat f) \le \error_\calS(\widehat f) &+ (k-1)\left(1 - \frac{k}{k-1} \error_{\wt\calS}(\widehat f) \right) \\ &+ \left(k + \sqrt{k} + \frac{m}{n\sqrt{k}} \right) \sqrt{\frac{\log(4/\delta)}{2m}} + \sqrt{k(k-1)} \sqrt{\frac{4}{\delta}\left(\frac{1}{m} +\frac{3\beta}{m+n} \right)}  \,. \numberthis \label{eq:gd_multi_error_proof}
    \end{align*} 
\end{proof}

\subsection{Discussion on \codref{cond:hypothesis_stability}} \label{app:discuss_cond1}
The quantity in LHS of \codref{cond:hypothesis_stability} 
measures how much the function learned by the algorithm 
(in terms of error on unseen point) will change 
when one point in the training set is removed. 
% Discussion on exponential concentration and stronger condition. 
% Notice that hypothesis stability implies error stability, i.e., \codref{cond:error_stability} \citep{bousquet2002stability}.  
% In summary, while error stability allowed us 
% to relate the average population error 
% of the leave-one-out classifiers 
% with the population error of the original classifier, 
We need hypothesis stability condition 
to control the variance of the empirical leave-one-out error to show concentration of average leave-one-error with the population error. 

Additionally, we note that while the dominating term in the RHS of \thmref{thm:linear} matches with the dominating term in ERM bound in \thmref{thm:error_ERM}, there is a polynomial concentration term 
(dependence on $1/\delta$ instead of $\log(\sqrt{1/\delta})$) 
in \thmref{thm:linear}. 
Since with hypothesis stability, 
we just bound the variance, 
the polynomial concentration is due 
to the use of Chebyshev's inequality 
instead of an exponential tail inequality
(as in \lemref{lem:fit_mislabeled}).
Recent works have highlighted that 
a slightly stronger condition than hypothesis stability 
can be used to obtain an exponential concentration 
for leave-one-out error \citep{abou2019exponential},
but we leave this for future work for now. 
% We leave 
% However, the constants 

% we also want to highlight  

\subsection{Formal statement and proof of \propref{prop:early_stop}} \label{app:formal_early_stop}

Before formally presenting the result, 
we will introduce some notation.  
By $\calL_{S}(w)$, we denote 
the objective in \eqref{eq:l2_MSE_app} with $\lambda=0$. 
Assume Singular Value Decomposition (SVD) of $\bX$
as $\sqrt{n} \bU \bS^{1/2} \bV^T$. 
Hence $\bX^T \bX = \bV \bS \bV^T$.
Consider the GD iterates defined in \eqref{eq:GD_iterates_app}. 
% 
We now derive closed form expression 
for the $t^\text{th}$ iterate of gradient descent:  
% 
\begin{align}
    w_t = w_{t-1} + \eta \cdot \bX^T (\by - \bX w_{t-1}) = (\bI - \eta \bV \bS \bV^T )w_{k-1} + \eta \bX^T \by \,.
\end{align}
Rotating by $\bV^T$, we get 
\begin{align}
    \wt w_t = (\bI - \eta\bS )\wt w_{k-1} + \eta \wt \by \label{eq:GD_recur},
\end{align}
where $\wt w_t = \bV^T w_t $ and $\wt \by = \bV^T \bX^T \by$. 
Assuming the initial point $w_0 = 0$ 
and applying the recursion in \eqref{eq:GD_recur}, we get
\begin{align}
    \wt w_t = \bS ^{-1} ( \bI - (\bI - \eta \bS)^k ) \wt \by \,, 
\end{align} 
Projecting solution back to the original space, we have 
\begin{align}
     w_t = \bV \bS ^{-1} ( \bI - (\bI - \eta \bS)^k ) \bV^T \bX^T \by \,. 
\end{align} 
% We will work with this GD solution at any iterate $t$ in the next proposition. 
Define $f_t(x) \defeq f(x;w_t)$ 
as the solution at the $t^{\text{th}}$ iterate. 
Let $\wt w_{\lambda} = \argmin_{w} \calL_\calS (w;\lambda) = (\bX^T \bX + \lambda \bI)^{-1} \bX^T \by = \bV (\bS + \lambda \bI )^{-1} \bV^T \bX^T \by $. 
% ) \,,$ for all $t=1,2,\ldots\,.$ 
and define $\wt f_\lambda(x) \defeq f(x;\wt w_\lambda)$ as the regularized solution. 
Assume $\kappa$ be the condition number 
of the population covariance matrix 
and let $s_\text{min}$ be the minimum positive 
singular value of the empirical covariance matrix. 
Our proof idea is inspired from recent work 
on relating gradient flow solution 
and regularized solution 
for regression problems \citep{ali2018continuous}. 
We will use the following lemma in the proof: 
\begin{lemma} \label{lem:ineq_soln}
    For all $x \in [0,1]$ and for all $ k \in \mathbb{N}$, 
    we have (a) $ \frac{kx}{1+kx} \le 1- (1-x)^k$ 
    and (b) $ 1- (1-x)^k \le 2 \cdot \frac{kx}{kx+1} $.
    %  where $g(c)$ is a constant dependent on $c$. For $c = 1$, $g(c) = 2.0$.   
\end{lemma}
\begin{proof}
    % [Proof of \lemref{lem:ineq_soln}]
    % Part (a) is easy. 
    Using $ (1-x)^k \le \frac{1}{1+kx}$, we have part (a). 
    For part (b), we numerically maximize 
    $\frac{ (1+kx ) (1 - (1-x)^k) }{kx}$ 
    for all $k\ge 1$ and for all $x \in [0, 1]$.  
\end{proof}

% 
% Next, 

\begin{prop}[Formal statement of \propref{prop:early_stop}] \label{prop:formal_early_stop}
Let $\lambda = \frac{1}{t\eta}$. 
For a training point $x$, we have 
\begin{align*}
    \Expt{x \sim \calS}{(f_t(x) - \wt f_\lambda(x))^2} &\le c(t,\eta) \cdot \Expt{x \sim \calS}{f_t(x)^2} \,, %\label{eq:early_stop}
\end{align*}
where $c(t, \eta) \defeq \min( 0.25, \frac{1}{s_\text{min}^2 t^2 \eta^2})$. 
Similarly for a test point, we have 
\begin{align*}
    \Expt{x \sim \calD_\calX}{(f_t(x) - \wt f_\lambda(x))^2} &\le \kappa \cdot c(t,\eta) \cdot \Expt{x \sim \calD_\calX}{f_t(x)^2} \,. %\label{eq:early_stop}
\end{align*}
\end{prop} 

\begin{proof}
    %%%%%%%%%%%%% 
    We want to analyze the expected squared difference output 
    of regularized linear regression 
    with regularization constant $\lambda = \frac{1}{\eta t}$ 
    and the gradient descent solution at the $t^\text{th}$ iterate. 
    We separately expand the algebraic expression 
    for squared difference at a training point and a test point. 
    % We start by considering the difference  
    Then the main step is to show that 
    $\left[ \bS ^{-1} ( \bI - (\bI - \eta \bS)^k )  - (\bS + \lambda \bI )^{-1}\right] \preceq c(\eta, t) \cdot \bS ^{-1} ( \bI - (\bI - \eta \bS)^k ) $.

    %%%%%%%%%%%%%
    
   \textbf{Part 1 {} {}} 
    First, we will analyze the squared difference 
    of the output at a training point 
    (for simplicity, we refer to $S \cup \wt S$ as $S$), i.e., 
    \begin{align}
        \Expt{ x \sim \calS }{\left(f_t(x) - \wt f_\lambda (x)\right)^2} &= \norm{\bX w_t - \bX \wt w_\lambda}{2}^2\\ &=   \norm{\bX \bV \bS ^{-1} ( \bI - (\bI - \eta \bS)^t ) \bV^T \bX^T \by - \bX \bV (\bS + \lambda \bI )^{-1} \bV^T \bX^T \by }{2}^2 \\
        &= \norm{\bX \bV \left(\bS ^{-1} ( \bI - (\bI - \eta \bS)^t ) - (\bS + \lambda \bI )^{-1} \right) \bV^T \bX^T \by  }{2} \\
        &=  \by^T \bV \bX \left( \underbrace{\bS ^{-1} ( \bI - (\bI - \eta \bS)^t ) - (\bS + \lambda \bI )^{-1}}_{\RN{1}} \right)^2 \bS \bV^T \bX^T \by \label{eq:train_GD_rel} \,.
        %  (\bX \bV \bS ^{-1} ( \bI - (\bI - \eta \bS)^k ) \bV^T \bX^T \by)^T \bX \bV \bS ^{-1} ( \bI - (\bI - \eta \bS)^k ) \bV^T \bX^T \by
    \end{align}
    We now separately consider term 1. 
    Substituting $\lambda = \frac{1}{t \eta}$, 
    we get
    \begin{align}
        \bS ^{-1} ( \bI - (\bI - \eta \bS)^t ) - (\bS + \lambda \bI )^{-1} &= \bS^{-1} \left( ( \bI - (\bI - \eta \bS)^t ) - (\bI + \bS^{-1} \lambda )^{-1}\right) \\
        &= \underbrace{\bS^{-1} \left( ( \bI - (\bI - \eta \bS)^t ) - (\bI + ( \bS t \eta)^{-1}  )^{-1}\right)}_{\bA} \,.
    \end{align}

    We now separately bound the diagonal entries in matrix $\bA$. 
    With $s_i$, we denote $i^{\text{th}}$ diagonal entry of $\bS$.
    Note that since $ \eta\le 1/\norm{S}{\text{op}}$, 
    for all $i$, $\eta s_i  \le 1$.  
    Consider $i^{\text{th}}$ diagonal term (which is non-zero) 
    of the diagonal matrix $\bA$, we have 
    \begin{align}
        \bA_{ii} = \frac{1}{s_i} \left(  1 - (1 - s_i \eta)^t - \frac{t \eta s_i}{1 + t \eta s_i } \right) &=  \frac{1 - (1 - s_i \eta)^t}{s_i} \left( \underbrace{ 1 - \frac{t \eta s_i}{(1 + t \eta s_i)(1 - (1 - s_i \eta)^t)}}_{\RN{2}} \right) \\ 
         &\le \frac{1}{2}\left[ \frac{1 - (1 - s_i \eta)^t}{ s_i} \right] \tag*{(Using \lemref{lem:ineq_soln} (b))} \,.
    \end{align} 
    Additionally, we can also show the following upper bound on term 2: 
    \begin{align}
         1 - \frac{t \eta s_i}{(1 + t \eta s_i)(1 - (1 - s_i \eta)^t)} &= \frac{(1 + t \eta s_i)(1 - (1 - s_i \eta)^t) - t \eta s_i }{(1 + t \eta s_i)(1 - (1 - s_i \eta)^t)} \\
         & \le  \frac{ 1 -  (1 - s_i \eta)^t - t \eta s_i (1 - s_i \eta)^t}{(1 + t \eta s_i)(1 - (1 - s_i \eta)^t)} \\
         & \le \frac{1}{t\eta s_i} \,. \tag{Using \lemref{lem:ineq_soln} (a)}
        %  &\le \frac{1}{2}\left[ \frac{1 - (1 - s_i \eta)^t}{ s_i} \right] \tag*{(Using \lemref{lem:ineq_soln})} \,.
    \end{align} 

    Combining both the upper bounds 
    on each diagonal entry $\bA_{ii}$, we have 
    \begin{align}
    \bA \preceq c_1(\eta, t) \cdot \bS^{-1} ( \bI - (\bI - \eta \bS)^t ) \,, \label{eq:upperbound_diagonal}
    \end{align}
    where $c_1(\eta, t ) = \min(0.5, \frac{1}{t s_i \eta })$. Plugging this into \eqref{eq:train_GD_rel}, we have 
    \begin{align}
        \Expt{ x \sim \calS }{\left(f_t(x) - \wt f_\lambda (x)\right)^2} &\le c(\eta, t) \cdot \by^T \bV \bX  \left( \bS^{-1} ( \bI - (\bI - \eta \bS)^t ) \right)^2 \bS \bV^T \bX^T \by \\
        &=   c(\eta, t) \cdot \by^T \bV \bX  \left( \bS^{-1} ( \bI - (\bI - \eta \bS)^t ) \right) \bS \left( \bS^{-1} ( \bI - (\bI - \eta \bS)^t ) \right) \bV^T \bX^T \by \\
        & =  c(\eta, t) \cdot \norm{\bX w_t}{2}^2 \\
        &= c(\eta, t) \cdot  \Expt{ x \sim \calS }{\left(f_t(x) \right)^2} \,,
    \end{align}
    where $c(\eta, t ) = \min(0.25, \frac{1}{t^2 s^2_i \eta^2 })$.

    \textbf{Part 2 {} {}} With $\bSigma$, 
    we denote the underlying true covariance matrix. 
    We now consider the squared difference of output at an unseen point: 
    \begin{align}
        \Expt{ x \sim \calD_{\calX} }{\left(f_t(x) - \wt f_\lambda (x)\right)^2} &= \Expt{x \sim \calD_{\calX}}{\norm{x^T w_t - x^T \wt w_\lambda}{2}} \\
        &=   \norm{x^T \bV \bS ^{-1} ( \bI - (\bI - \eta \bS)^t ) \bV^T \bX^T \by - x^T \bV (\bS + \lambda \bI )^{-1} \bV^T \bX^T \by }{2} \\
        &= \norm{x^T \bV \left(\bS ^{-1} ( \bI - (\bI - \eta \bS)^t ) - (\bS + \lambda \bI )^{-1} \right) \bV^T \bX^T \by  }{2} \\
        &= \by^T \bV \bX \left( \bS ^{-1} ( \bI - (\bI - \eta \bS)^t ) - (\bS + \lambda \bI )^{-1} \right) \bV^T \bSigma \bV \\ &\qquad \qquad \qquad \qquad \qquad \left( (\bI - (\bI - \eta \bS)^t ) - (\bS + \lambda \bI )^{-1} \right) \bV^T \bX^T \by \\
        &\le \sigma_{\text{max}} \cdot \by^T \bV \bX \left( \underbrace{\bS ^{-1} ( \bI - (\bI - \eta \bS)^t ) - (\bS + \lambda \bI )^{-1}}_{\RN{1}} \right)^2 \bV^T \bX^T \by \,, \label{eq:test_GD_rel}
        %  (\bX \bV \bS ^{-1} ( \bI - (\bI - \eta \bS)^k ) \bV^T \bX^T \by)^T \bX \bV \bS ^{-1} ( \bI - (\bI - \eta \bS)^k ) \bV^T \bX^T \by
    \end{align}
    where $\sigma_{\text{max}}$ is the maximum eigenvalue 
    of the underlying covariance matrix $\bSigma$. 
    Using the upper bound on term 1 in \eqref{eq:upperbound_diagonal}, 
    we have 
    \begin{align}
        \Expt{ x \sim \calD_{\calX} }{\left(f_t(x) - \wt f_\lambda (x)\right)^2} &\le \sigma_{\text{max}} \cdot c(\eta, t) \cdot \by^T \bV \bX  \left( \bS^{-1} ( \bI - (\bI - \eta \bS)^t ) \right)^2 \bV^T \bX^T \by \\
        &=   \kappa \cdot c(\eta, t) \cdot \sigma_{\text{min}}\cdot \norm{\bV \left( \bS^{-1} ( \bI - (\bI - \eta \bS)^t ) \right) \bV^T \bX^T \by}{2}^2 \\
        &\le \kappa \cdot c(\eta, t) \cdot \left[ \bV \left( \bS^{-1} ( \bI - (\bI - \eta \bS)^t ) \right) \bV^T \bX^T \right]^T \bSigma \\
        &\qquad \qquad \qquad \qquad \qquad \left[ \bV \left( \bS^{-1} ( \bI - (\bI - \eta \bS)^t ) \right) \bV^T \bX^T \right] \by \\
        & = \kappa \cdot c(\eta, t) \cdot \Expt{x \sim \calD_{\calX}}{\norm{x^T w_t}{2}} \,.
    \end{align}
% 
% 
    % Since $ \eta\le 1/\norm{S}{\text{op}}$, invoking \lemref{lem:ineq_soln} to upper bound term 1 with
\end{proof}

\subsection{Extension to deep learning} \label{appsubsec:ext_DL}
Under \asmpref{appsubsec:justifying_assumption1}, we present the formal result parallel to \thmref{thm:multiclass_ERM}. 
\begin{theorem} \label{thm:multiclass_ERM_algoA}
    Consider a multiclass classification problem 
    with $k$ classes. Under \asmpref{asmp:deep_models}, 
    for any $\delta >0$, with probability at least $1-\delta$,
    we have
    \vspace{-10pt}
    \begin{align*}
        \error_\calD(\widehat f)  \le \error_\calS(\widehat f) + (k-1) \left(1 - \tfrac{k}{k-1} \error_{\wt\calS}(\widehat f)\right) + c\sqrt{\frac{\log(\frac{4}{\delta})}{2m}} \,,\numberthis \label{eq:multiclass_ERM_deep}
    % \vspace{-20pt}
    \end{align*}
    for some constant $c \le ((c+1) k+\sqrt{k} + \frac{m}{n\sqrt{k}})$.
\end{theorem}

The proof follows exactly as in step (i) to (iii) in \thmref{thm:multiclass_ERM}.  

\subsection{Justifying~\asmpref{asmp:deep_models}} \label{appsubsec:justifying_assumption1}

Motivated by the analysis on linear models, we now discuss alternate (and weaker) conditions that imply \asmpref{asmp:deep_models}. 
We need hypothesis stability (\codref{cond:hypothesis_stability}) and the following assumption relating training error and leave-one-error: 

\begin{assumption} \label{asmp:loo_error}
Let $\wh f$ be a model obtained by training with algorithm $\calA$ on a mixture of clean $S$ and randomly labeled data $\wt S$. Then we assume we have 
\begin{align*}
    \error_{\wt \calS_M} (\wh f) \le  \error_{\text{LOO} (\wt S_M)} \,, 
\end{align*}
for all $(x_i, y_i) \in  \wt S_M$ where $\wh f_{(i)} \defeq f(\calA, S \cup {{}\wt S_M}_{(i)})$ and  $\error_{\text{LOO} (\wt S_M)} \defeq  \frac{\sum_{(x_i, y_i) \in \wt S_M} \error(\ff{i}(x_i), y_i ) }{\abs{\wt \calS_M}}$.  
\end{assumption}

% we assume this to extend our result (parallel to \thmref{thm:multi_linear}) for deep models. 
Intuitively, this assumption states that the error on a (mislabeled) datum $(x,y)$ included in the training set is less than the error on that datum $(x,y)$ obtained by a model trained on the training set $S - \{(x,y)\}$. We proved this for linear models trained with GD in the proof of \thmref{thm:multi_linear}. 
% 
\codref{cond:hypothesis_stability} with $\beta = \calO(1)$ and \asmpref{asmp:loo_error} together with \lemref{lem:stability_error} implies \asmpref{asmp:deep_models} with a polynomial residual term (instead of logarithmic in $1/\delta$): 
\begin{align}
     \error_{\calS_M} (\wh f) \le  \error_{\calDm}(\wh f)   + \sqrt{\frac{1}{\delta}\left(\frac{1}{m} +\frac{3\beta}{m+n} \right)} \,.
\end{align}
% Note that this  

\newpage 
\section{Additional experiments and details}\label{app:exp}
\newcommand\tab[1][1cm]{\hspace*{#1}}

\subsection{Datasets} \label{sec:app_dataset}

\textbf{Toy Dataset {} {}} Assume fixed constants $\mu$ and $\sigma$. For a given label $y$, we simulate features $x$ in our toy classification setup as follows: 
\begin{align*}
    x \defeq \texttt{concat} \left[ x_1, x_2\right] \quad \text{where} \quad  x_1 \sim  \calN( y \cdot \mu, \sigma^2 I_{d \times d}) \ \  \text{and} \ \  x_1 \sim  \calN( 0, \sigma^2 I_{d \times d}) \,.
\end{align*}  
% where $y$ is the true label and $x$ is the corresponding feature vector. 
In experiements throughout the paper, we fix dimention $d=100$, $\mu = 1.0 $, and $\sigma = \sqrt{d}$. Intuitively, $x_1$ carries the information about the underlying label and $x_2$ is additional noise independent of the underlying label. 

\textbf{CV datasets {} {}} We use MNIST~\citep{lecun1998mnist} and CIFAR10~\cite{krizhevsky2009learning}. 
% For binary tasks, 
We produce a binary variant from the multiclass classification problem by mapping classes $\{0,1,2,3,4\}$ to label $1$ and $\{ 5,6,7,8,9\}$ to label $-1$. For CIFAR dataset, we also use the standard data augementation of random crop and horizontal flip. PyTorch code is as follows: 

\texttt{(transforms.RandomCrop(32, padding=4),\\
\tab transforms.RandomHorizontalFlip())}

\textbf{NLP dataset {} {}} We use IMDb Sentiment analysis~\citep{maas2011learning} corpus.  

\subsection{Architecture Details} 

All experiments were run on NVIDIA GeForce RTX 2080 Ti GPUs. We used PyTorch~\citep{NEURIPS2019a9015} and Keras with Tensorflow~\citep{abadi2016tensorflow} backend for experiments. 
% , ELMo embeddings~\citep{Peters:2018}, and Hugging Face Transformers~\citep{wolf-etal-2020-transformers}. 

\textbf{Linear model {} {}} For the toy dataset, we simulate a linear model with scalar output and the same number of parameters as the number of dimensions.   

\textbf{Wide nets {} {}} To simulate the NTK regime, we experiment with $2-$layered wide nets. The PyTorch code for 2-layer wide MLP is as follows: 


\texttt{ nn.Sequential( \\
\tab     nn.Flatten(),\\
\tab    nn.Linear(input\_dims, 200000, bias=True),\\
\tab    nn.ReLU(),\\
\tab    nn.Linear(200000, 1, bias=True)\\
\tab     )}


We experiment both (i) with the second layer fixed at random initialization; (ii)  and updating both layers' weights.     

\textbf{Deep nets for CV tasks {} {}} We consider a 4-layered MLP. The PyTorch code for 4-layer MLP is as follows: 

\texttt{ nn.Sequential(nn.Flatten(), \\
\tab        nn.Linear(input\_dim, 5000, bias=True),\\
\tab        nn.ReLU(),\\
\tab        nn.Linear(5000, 5000, bias=True),\\
\tab        nn.ReLU(),\\
\tab        nn.Linear(5000, 5000, bias=True),\\
\tab        nn.ReLU(),\\
% \tab        nn.Linear(5000, 5000, bias=True),\\
% \tab        nn.ReLU(),\\
\tab        nn.Linear(1024, num\_label, bias=True)\\
\tab        )}

For MNIST, we use $1000$ nodes instead of $5000$ nodes in the hidden layer. 
% 
We also experiment with convolutional nets. In particular, we use ResNet18 \citep{he2016deep}. Implementation adapted from:  \url{https://github.com/kuangliu/pytorch-cifar.git}. 

\textbf{Deep nets for NLP {} {}} We use a simple LSTM model with embeddings intialized with ELMo embeddings~\citep{Peters:2018}. Code adapted from: \url{https://github.com/kamujun/elmo_experiments/blob/master/elmo_experiment/notebooks/elmo_text_classification_on_imdb.ipynb} 

We also evaluate our bounds with a BERT model. In particular, we fine-tune an off-the-shelf uncased BERT model~\citep{devlin2018bert}. Code adapted from Hugging Face Transformers~\citep{wolf-etal-2020-transformers}: \url{https://huggingface.co/transformers/v3.1.0/custom_datasets.html}. 


\subsection{Additonal experiments}

\textbf{Results with SGD on underparameterized linear models {} {}} 

\begin{figure*}[h]
    \centering 
    % \vspace{-15pt}
    % \includegraphics[width=0.9\linewidth]{example-image-a}
    \includegraphics[width=0.3\linewidth]{figures/lowdim-Gaussian-SGD.pdf}
    % \includegraphics[width=0.9\linewidth]{figures/{CIFAR10_rn=0.1_lr=0.2_wd=0.005}.png}
    \vspace{-5pt}
    \caption{ 
    % Predicted lower bound 
    % on different
    We plot the accuracy and corresponding bound 
    (RHS in \eqref{eq:erm}) at $\delta = 0.1$
    for toy binary classification task. 
    Results aggregated over $3$ seeds. 
    % i.e., $1-\error$ where $\error$ is the term in the RHS of \eqref{eq:erm}
    Accuracy vs fraction of unlabeled data (w.r.t clean data) 
    in the toy setup with a linear model trained with SGD. Results parallel to \figref{fig:error_binary}(a) with SGD.  }
    \label{fig:error_binary_linear}
    \vspace{-5pt}
\end{figure*}

\textbf{Results with wide nets on binary MNIST {} {}}

\begin{figure*}[h]
    \centering 
    % \vspace{-15pt}
    % \includegraphics[width=0.9\linewidth]{example-image-a}
    \subfigure[GD with MSE loss]{\includegraphics[width=0.3\linewidth]{figures/MNIST-GD_MSE.pdf}} \hfil
    \subfigure[SGD with CE loss]{\includegraphics[width=0.3\linewidth]{figures/MNIST-SGD_CE.pdf}}
    \subfigure[SGD with MSE loss]{\includegraphics[width=0.3\linewidth]{figures/MNIST-SGD_MSE-first-layer.pdf}}
    % \includegraphics[width=0.9\linewidth]{figures/{CIFAR10_rn=0.1_lr=0.2_wd=0.005}.png}
    \vspace{-5pt}
    \caption{ 
    % Predicted lower bound 
    % on different
    We plot the accuracy and corresponding bound 
    (RHS in \eqref{eq:erm}) at $\delta = 0.1$ 
    for binary MNIST classification. 
    Results aggregated over $3$ seeds. 
    % i.e., $1-\error$ where $\error$ is the term in the RHS of \eqref{eq:erm}
    Accuracy vs fraction of unlabeled data 
    for a 2-layer wide network on binary MNIST with both the layers training in (a,b) and only first layer training in (c). 
    Results parallel to \figref{fig:error_binary}(b) .  }
    \label{fig:error_binary_MNIST}
    \vspace{-5pt}
\end{figure*}

% \begin{figure*}[h]
%     \centering 
%     % \vspace{-15pt}
%     % \includegraphics[width=0.9\linewidth]{example-image-a}
%     \subfigure[GD with MSE loss]{\includegraphics[width=0.3\linewidth]{figures/MNIST.pdf}} \hfil
    
%     \subfigure[SGD with CE loss]{\includegraphics[width=0.3\linewidth]{figures/MNIST.pdf}}
%     % \includegraphics[width=0.9\linewidth]{figures/{CIFAR10_rn=0.1_lr=0.2_wd=0.005}.png}
%     \vspace{-5pt}
%     \caption{ 
%     % Predicted lower bound 
%     % on different
%     We plot the accuracy and corresponding bound 
%     (RHS in \eqref{eq:erm}) at $\delta = 0.1$
%     for binary MNIST classification. 
%     Results aggregated over $3$ seeds. 
%     % i.e., $1-\error$ where $\error$ is the term in the RHS of \eqref{eq:erm}
%     Accuracy vs fraction of unlabeled data 
%     for a 2-layer wide network on binary MNIST with just the first layer training. 
%     Results parallel to \figref{fig:error_binary}(b) with only the first layer training.  }
%     \label{fig:error_binary_MNIST}
%     \vspace{-5pt}
% \end{figure*}

\textbf{Results on CIFAR 10 and MNIST {} {}} 
% 
We plot epoch wise error curve for results in \tabref{table:multiclass}(\figref{fig:error_epoch_CIFAR10} and \figref{fig:error_epoch_MNIST}). We observe the same trend as in \figref{fig:error_CIFAR10}. Additionally, we plot an \emph{oracle bound} obtained by tracking the error on mislabeled data which nevertheless were predicted as true label. To obtain an exact emprical value of the oracle bound, we need underlying true labels for the randomly labeled data. 
% Note that our bound in \thmref{thm:multiclass_ERM}, lower bounds the accuracy as predicted by the oracle bound. 
While with just access to extra unlabeled data we cannot calculate oracle bound, we note that the oracle bound is very tight and never violated in practice underscoring an importamt aspect of generalization in multiclass problems. This highlight that even a stronger conjecture may hold in multiclass classification, i.e., error on mislabeled data (where nevertheless true label was predicted) lower bounds the population error on the distribution of mislabeled data and hence, the error on (a specific) mislabeled portion predicts the population accuracy on clean data. 
% 
On the other hand, the dominating term of in \thmref{thm:multiclass_ERM} is loose when compared with the oracle bound. The main reason, we believe is the pessimistic upper bound in \eqref{eq:lemma1_final_multi_prev} in the proof of \lemref{lem:fit_mislabeled_multi}. We leave an investigation on this gap for future. 
% of fit 

% However, oracle bound highlights two . One,  



\begin{figure}[h]
    \centering 
    % \vspace{-15pt}
    % \includegraphics[width=0.9\linewidth]{example-image-a}
    \subfigure[MLP]{\includegraphics[width=0.3\linewidth]{figures/CIFAR10-FNN.pdf}} \hfil
    \subfigure[ResNet]{\includegraphics[width=0.3\linewidth]{figures/CIFAR10-Resnet.pdf}}
    % \includegraphics[width=0.9\linewidth]{figures/{CIFAR10_rn=0.1_lr=0.2_wd=0.005}.png}
    % \vspace{-10pt}
    \caption{ Per epoch curves for CIFAR10 corresponding results in \tabref{table:multiclass}. As before, we just plot the dominating term in the RHS of \eqref{eq:multiclass_ERM} as predicted bound. Additionally, we also plot the predicted lower bound by the error on mislabeled data which nevertheless were predicted as true label. We refer to this as ``Oracle bound''. See text for more details. 
    % 
    % except for the stopping point. 
    % The bound predicted by RATT (RHS in \eqref{eq:multiclass_ERM}) is vacuous. 
    }\label{fig:error_epoch_CIFAR10}
    % \vspace{-15pt}
\end{figure}


\begin{figure}[h]
    \centering 
    % \vspace{-15pt}
    % \includegraphics[width=0.9\linewidth]{example-image-a}
    \subfigure[MLP]{\includegraphics[width=0.3\linewidth]{figures/MNIST-FNN.pdf}} \hfil
    \subfigure[ResNet]{\includegraphics[width=0.3\linewidth]{figures/MNIST-Resnet.pdf}}
    % \includegraphics[width=0.9\linewidth]{figures/{CIFAR10_rn=0.1_lr=0.2_wd=0.005}.png}
    % \vspace{-10pt}
    \caption{ Per epoch curves for MNIST corresponding results in \tabref{table:multiclass}. As before, we just plot the dominating term in the RHS of \eqref{eq:multiclass_ERM} as predicted bound. Additionally, we also plot the predicted lower bound by the error on mislabeled data which nevertheless were predicted as true label. We refer to this as ``Oracle bound''. See text for more details. 
    % 
    % except for the stopping point. 
    % The bound predicted by RATT (RHS in \eqref{eq:multiclass_ERM}) is vacuous. 
    }\label{fig:error_epoch_MNIST}
    % \vspace{-15pt}
\end{figure}

\textbf{Results on CIFAR 100 {} {}} 
% 
On CIFAR100, our bound in \eqref{eq:multiclass_ERM} yields vacous bounds. However, the oracle bound as explained above yields tight guarantees in the initial phase of the learning (i.e., when learning rate is less than $0.1$) (\figref{fig:error_CIFAR100}).  

\begin{figure}[h]
    \centering 
    % \vspace{-15pt}
    % \includegraphics[width=0.9\linewidth]{example-image-a}
    \includegraphics[width=0.3\linewidth]{figures/CIFAR100-Resnet.pdf}
    % \includegraphics[width=0.9\linewidth]{figures/{CIFAR10_rn=0.1_lr=0.2_wd=0.005}.png}
    % \vspace{-10pt}
    \caption{ Predicted lower bound by the error on mislabeled data which nevertheless were predicted as true label with ResNet18 on CIFAR100. We refer to this as ``Oracle bound''. See text for more details. 
    % 
    % except for the stopping point. 
    The bound predicted by RATT (RHS in \eqref{eq:multiclass_ERM}) is vacuous. 
    }\label{fig:error_CIFAR100}
    % \vspace{-15pt}
\end{figure}


% \paragraph{Experiments on CIFAR100} 


% \subsection{Model Selection using RATT}


\subsection{Hyperparameter Details}


\textbf{\figref{fig:error_CIFAR10} {} {}} We use clean training dataset of size $40,000$. We fix the amount of unlabeled data at $20\%$ of the clean size, i.e. we include additional $8,000$ points with randomly assigned labels. We use test set of $10,000$ points. For both MLP and ResNet, we use SGD with an initial learning rate of $0.1$ and momentum $0.9$. We fix the weight decay parameter at $5\times 10^{-4}$. After $100$ epochs, we decay the learning rate to $0.01$. We use SGD batch size of $100$. 

\textbf{\figref{fig:error_binary} (a) {} {}} We obtain a toy dataset according to the process described in \secref{sec:app_dataset}. We fix $d=100$ and create a dataset of $50,000$ points with balanced classes. Moreover, we sample additional covariates with the same procedure to create randomly labeled dataset. For both SGD and GD training, we use a fixed learning rate $0.1$.    

\textbf{\figref{fig:error_binary} (b) {} {}} Similar to binary CIFAR, we use clean training dataset of size $40,000$ and fix the amount of unlabeled data at $20\%$ of the clean dataset size. To train wide nets, we use a fixed learning of $0.001$ with GD and SGD. We decide the weight decay parameter and the early stopping point that maximizes our generalization bound (i.e. without peeking at unseen data ).  We use SGD batch size of $100$. 

\textbf{\figref{fig:error_binary} (c) {} {}} With IMDb dataset, we use a clean dataset of size $20,000$ and as before, fix the amount of unlabeled data at $20\%$ of the clean data. To train ELMo model, we use Adam optimizer with a fixed learning rate $0.01$ and weight decay $10^{-6}$ to minimize cross entropy loss. We train with batch size $32$ for 3 epochs. To fine-tune BERT model, we use Adam optimizer with learning rate $5\times 10^{-5}$ to minimize cross entropy loss. We train with a batch size of $16$ for 1 epoch.    

\textbf{\tabref{table:multiclass} {} {}} For multiclass datasets, we train both MLP and ResNet with the same hyperparameters as described before. We sample a clean training dataset of size $40,000$ and fix the amount of unlabeled data at $20\%$ of the clean size. We use SGD with an initial learning rate of $0.1$ and momentum $0.9$. We fix the weight decay parameter at $5\times 10^{-4}$. After $30$ epochs for ResNet and after $50$ epochs for MLP, we decay the learning rate to $0.01$.  We use SGD with batch size $100$. 
For \figref{fig:error_CIFAR100}, we use the same hyperparameters as 
CIFAR10 training, except we now decay learning rate after $100$ epochs. 


In all experiments, to identify the best possible accuracy on just the clean data, we use the exact same set of hyperparamters except the stopping point. We choose a stopping point that maximizes test performance. 

\subsection{Summary of experiments }

\begin{center}
    \begin{table}[H] 
        \centering
        \begin{tabular}{|c|c|c|c|} 
        \hline
        Classification type & Model category & Model & Dataset  \\ [0.5ex] 
        \hline
        \hline
        \multirow{10}{*}{Binary} & Low dimensional & Linear model & Toy Gaussain dataset  \\
                        \cline{2-4}
                         & Overparameterized 
                        %  & Linear model & Toy Gaussain dataset \\
                        %  \cline{3-4}
                        %  & & 2-layer wide net& Toy Gaussain dataset \\
                        %  \cline{3-4}
                         & \multirow{2}{*}{2-layer wide net} & \multirow{2}{*}{Binary MNIST} \\
                         & linear nets & &  
                         \\
                         \cline{2-4}                 
                         & \multirow{6}{*}{Deep nets} & \multirow{2}{*}{MLP} & Binary MNIST \\
                         \cline{4-4}
                         & &  & Binary CIFAR \\
                         \cline{3-4}
                         &  & \multirow{2}{*}{ResNet} & Binary MNIST \\
                         \cline{4-4}
                         & &  & Binary CIFAR \\
                         \cline{3-4}
                         &  & ELMo-LSTM model & IMDb Sentiment Analysis \\
                         \cline{3-4}
                         & & BERT pre-trained model & IMDb Sentiment Analysis \\
        \hline
        \multirow{5}{*}{Multiclass} & \multirow{5}{*}{Deep nets} & \multirow{2}{*}{MLP} & MNIST \\
                        \cline{4-4} 
                        & & & CIFAR10 \\                   
                        \cline{3-4}
                         &   & \multirow{3}{*}{ResNet} & MNIST \\
                         \cline{4-4}
                         &   & & CIFAR10 \\
                         \cline{4-4}
                         &   & & CIFAR100 \\
        \hline
        \end{tabular}
        % \caption{Summary of experiments performed} \label{table:experiments}
    \end{table}    
    % \footnotetext[6]{We use both MSE loss and cross-entropy loss.}
    % \footnotetext[6]{We try 2 variants: one with a fixed first layer and the other with both layers trainable.}
\end{center}

\newpage
\section{Proof of \lemref{lem:stability_error}} \label{app:proof_lem_error}

\begin{proof}[Proof of \lemref{lem:stability_error}]
    Recall, we have a training set $S \cup \wt S_C$. We defined leave-one-out error on mislabeled points as $$\error_{\text{LOO}(\wt S_M) } = \frac{\sum_{(x_i, y_i) \in \wt S_M} \error( f_{(i)}( x_i), y_i)}{ \abs{\wt S_M }} \,, $$
    where $f_{(i)} \defeq f(\calA, (S \cup \wt S)_{(i)})$. Define $S^\prime \defeq S \cup \wt S$. Assume $(x,y)$ and $(x^\prime,y^\prime)$ as i.i.d. samples from ${\calDm}$. 
    Using Lemma 25 in \citet{bousquet2002stability}, we have
    \begin{align*}
        \Expo{ \left( \error_{\calDm}(\wh f) -\error_{\text{LOO}(\wt S_M) } \right)^2 } \le & \Expt{ S^\prime, (x,y), (x^\prime,y^\prime) }{ \error(\wh f(x), y ) \error(\wh f(x^\prime), y^\prime )} - 2 \Expt{ S^\prime, (x,y) }{ \error(\wh f(x), y ) \error(f_{(i)}(x_i), y_i )} \\
        & + \frac{m_1-1}{m_1}\Expt{ S^\prime }{  \error(f_{(i)}(x_i), y_i )  \error(f_{(j)}(x_j), y_j )} + \frac{1}{m_1} \Expt{ S^\prime }{  \error(f_{(i)}(x_i), y_i ) } \,. \numberthis \label{eq:main_reln}
    \end{align*}
    We can rewrite the equation above as : 
    \begin{align*}
        \Expo{ \left( \error_{\calDm}(\wh f) -\error_{\text{LOO}(\wt S_M) } \right)^2 } \le &  \, \underbrace{\Expt{ S^\prime, (x,y), (x^\prime,y^\prime) }{ \error(\wh f(x), y ) \error(\wh f(x^\prime), y^\prime ) - \error(\wh f(x), y ) \error(f_{(i)}(x_i), y_i )}}_{\RN{1}} \\
        & + \underbrace{\Expt{ S^\prime }{  \error(f_{(i)}(x_i), y_i )  \error(f_{(j)}(x_j), y_j ) -  \error(\wh f(x), y ) \error(f_{(i)}(x_i), y_i )}}_{\RN{2}} \\ &+ \underbrace{\frac{1}{m_1} \Expt{ S^\prime }{  \error(f_{(i)}(x_i), y_i ) - \error(f_{(i)}(x_i), y_i )  \error(f_{(j)}(x_j), y_j ) }}_{\RN{3}} \,. \numberthis \label{eq:main_reln2}
    \end{align*}
    
    We will now bound term $\RN{3}$.  Using Cauchy-Schwarz's inequality, we have
    
    \begin{align}
        \Expt{ S^\prime }{  \error(f_{(i)}(x_i), y_i ) - \error(f_{(i)}(x_i), y_i )  \error(f_{(j)}(x_j), y_j ) }^2 &\le  \Expt{ S^\prime }{  \error(f_{(i)}(x_i), y_i ) }^2 \Expt{S^\prime}{1 -   \error(f_{(j)}(x_j), y_j ) }^2 \\
        &\le \frac{1}{4} \,.\label{eq:term1_lem12}
    \end{align}
    
    Note that since $(x_i,y_i)$, $(x_j ,y_j )$, $(x,y)$, and $(x^\prime, y^\prime)$ are all from same distribution $\calDm$, we directly incorporate the bounds on term $\RN{1}$ and $\RN{2}$ from the proof of Lemma 9 in \citet{bousquet2002stability}. Combining that with \eqref{eq:term1_lem12} and our definition of hypothesis stability in \codref{cond:hypothesis_stability}, we have the required claim. 
    
    
    % We now re-write term $\RN{1}$ as
    % \begin{align*}
    %         &\Expt{S^\prime, (x,y), (x^\prime,y^\prime) }{ \error(\wh f(x), y ) \error(\wh f(x^\prime), y^\prime ) - \error(\wh f(x), y ) \error(f_{(i)}(x_i), y_i )} \\ & \qquad = \Expt{ S^\prime, (x,y), (x^\prime,y^\prime) }{ \error(\wh f(x), y ) \error(\wh f  (x^\prime), y^\prime ) - \error(\wh f ^\prime(x), y ) \error(f_{(i)}(x^\prime), y^\prime )} \tag{Exchanging $(x_i, y_i)$ with $(x^\prime, y^\prime)$ in the second term} \\
    %         & \qquad = \Expt{ S^\prime, (x,y), (x^\prime,y^\prime) }{  \left(\error(\wh f(x), y )-  \error(f_{(i)}(x), y ) \right) \error(\wh f  (x^\prime), y^\prime )  } \\
    %         & \qquad  + \Expt{ S^\prime, (x,y), (x^\prime,y^\prime) }{  \left(\error(f_{(i)}(x), y ) -\error(\wh f ^\prime(x), y ) \right) \error(\wh f  (x^\prime), y^\prime )}  \\
    %         & \qquad +\Expt{ S^\prime, (x,y), (x^\prime,y^\prime) }{  \left( \error(\wh f  (x^\prime), y^\prime ) -  \error(f_{(i)}(x^\prime), y^\prime ) \right) \error(\wh f ^\prime(x), y ) }  \,, \numberthis \label{eq:term1_final}
    % \end{align*}
    % where $\wh f^\prime$ is the classifier obtained by training on $ S^\prime_{(i)} \cup \{ (x^\prime, y^\prime) \} $. Similarly we can re-write term $\RN{2}$ as 
    % \begin{align*}
    %     & \Expt{ S^\prime }{  \error(f_{(i)}(x_i), y_i )  \error(f_{(j)}(x_j), y_j ) -  \error(\wh f(x), y ) \error(f_{(i)}(x_i), y_i )} \\
    %     &\quad  = \Expt{ S^\prime, (x,y), (x^\prime,y^\prime)}{  \error(f^{\prime\prime}_{(i)}(x), y )  \error(f_{(j)}^{\prime}(x^\prime), y^\prime ) -  \error(\wh f(x), y ) \error(f_{(i)}(x_i), y_i )} \tag{Exchanging $(x_i, y_i)$ with $(x, y)$ and $(x_j, y_j)$ with $(x^\prime, y^\prime)$ in the first term}\\
    %     &\quad = \Expt{ S^\prime, (x,y), (x^\prime,y^\prime)}{  \error(f^{\prime\prime}_{(j)}(x), y )  \error(f_{(i)}^{\prime}(x^\prime), y^\prime ) -  \error(\wh f^\prime (x), y ) \error(f^\prime_{(j)}(x^\prime), y^\prime )} \tag{Exchanging $(x_i, y_i)$ and $(x_j, y_j)$ and then replacing $(x_j, y_j)$ with $(x^\prime, y^\prime)$ in the second term} \\
    %     & \quad = \Expt{ S^\prime, (x,y), (x^\prime,y^\prime) }{  \left( \error(f_{(i)}^{\prime}(x^\prime), y^\prime )   -  \error(\wh f^{\prime\prime}  (x^\prime), y^\prime ) \right)  \error(f^{\prime\prime}_{(j)}(x), y )   } \\
    %     & \quad  + \Expt{ S^\prime, (x,y), (x^\prime,y^\prime) }{  \left( \error(f^{\prime\prime}_{(j)}(x), y )  -\error(\wh f ^\prime(x), y ) \right) \error(\wh f^{\prime\prime}  (x^\prime), y^\prime )  }  \\
    %     & \quad+ \Expt{ S^\prime, (x,y), (x^\prime,y^\prime) }{  \left( \error(\wh f^{\prime\prime}  (x^\prime), y^\prime )  -  \error(f^\prime_{(j)}(x^\prime), y^\prime ) \right)  \error(\wh f^\prime (x), y ) }   \\
    %     & \quad = \Expt{ S^\prime, (x,y), (x^\prime,y^\prime) }{  \left( \error(f_{(i)}^{\prime}(x^\prime), y^\prime )   -  \error(\wh f (x^\prime), y^\prime ) \right)  \error(f_{(i)}(x_j), y_j )   } \\
    %     & \quad  + \Expt{ S^\prime, (x,y), (x^\prime,y^\prime) }{  \left( \error(f^{\prime\prime}_{(j)}(x), y )  -\error(\wh f (x), y ) \right) \error(\wh f^{\prime\prime}  (x_j), y_j )  }  \\
    %     & \quad+ \Expt{ S^\prime, (x,y), (x^\prime,y^\prime) }{  \left( \error(\wh f^{\prime\prime}  (x^\prime), y^\prime )  -  \error(f^\prime_{(j)}(x^\prime), y^\prime ) \right)  \error(\wh f^\prime (x^\prime), y^\prime ) }  \,, \numberthis \label{eq:term2_final}
    % \end{align*}
    % where $f^{\prime\prime}_{(j)}$ is trained on $S^\prime_{(j,i)} \cup {(x,y)}$, $f^{\prime}_{(i)}$ is trained on $S^\prime_{(j,i)} \cup {(x^\prime,y^\prime)}$, and $\wh f^{\prime\prime} $ is trained on $S^\prime_{(j)} \cup {(x,y)}$. Note in the last line we replaced $(x,y)$ by $(x_j, y_j)$ in the first term, replaced $(x^\prime,y^\prime)$ by $(x_j, y_j)$ in the second term and exchanged $(x_i,y_i)$ with $(x_j,y_j)$ and also $(x,y)$ and $(x^\prime, y^\prime)$
    
    
\end{proof}


% 
% 16th Century Version Control 
% 

% \onecolumn

% \section*{Supplementary Material}
% We will be using the following standard results
% on exponential concentration of random variables 
% all throughout the discussion:

% \begin{lemma}[Hoeffding's inequality for independent RVs~\citep{hoeffding1994probability}] Let $Z_1, Z_2, \ldots, Z_n$ be independent bounded random variables with $Z_i \in [a,b]$ for all $i$, then 
%     \begin{align*}
%         \prob\left( \frac{1}{n} \sum_{i=1}^n (Z_i - \Expo{Z_i}) \ge t \right) \le \exp{\left( -\frac{2nt^2}{(b-a)^2} \right) }
%     \end{align*} 
%     and 
%     \begin{align*}
%         \prob\left( \frac{1}{n} \sum_{i=1}^n (Z_i - \Expo{Z_i}) \le -t \right) \le \exp{\left( -\frac{2nt^2}{(b-a)^2} \right) }
%     \end{align*} 
%     for all $t \ge 0$. 
% \end{lemma}

% \begin{lemma}[Hoeffding's inequality for sampling with replacement~\citep{hoeffding1994probability}] \label{lem:hoeffding_sampling} Let $\calZ = (Z_1, Z_2, \ldots, Z_N)$ be a finite population of $N$ points with $Z_i \in [a.b]$ for all $i$. Let $X_1, X_2, \ldots X_n$ be a random sample drawn without replacement from $\calZ$. Then for all $t \ge 0$, we have 
%     \begin{align*}
%         \prob\left( \frac{1}{n} \sum_{i=1}^n (X_i - \mu ) \ge t \right) \le \exp{\left( -\frac{2nt^2}{(b-a)^2} \right) }
%     \end{align*} 
%     and 
%     \begin{align*}
%         \prob\left( \frac{1}{n} \sum_{i=1}^n (X_i - \mu ) \le -t \right) \le \exp{\left( -\frac{2nt^2}{(b-a)^2} \right) } \,,
%     \end{align*} 
%     where $\mu = \frac{1}{N} \sum_{i=1}^{N} Z_i$. 
% \end{lemma}

% We now discuss one condition that generalizes the exponential concentration to dependent random variables.
% \begin{condition}[Bounded difference inequality] \label{cond:BDC} Let $\calZ$ be some set and $\phi: \calZ^n \to \Real$. We say that $\phi$ satisfies the bounded difference assumption if 
% there exists $c_1, c_2, \ldots c_n \ge 0$ s.t. for all $i$, we have 
% \begin{align*}
%     \sup_{Z_1,Z_2, \ldots,Z_n, Z_i^\prime in \calZ^{n+1} } \abs{\phi (Z_1, \ldots, Z_i, \ldots, Z_n ) - \phi (Z_1, \ldots, Z_i^\prime, \ldots, Z_n ) } \le c_i \,.
% \end{align*} 
% \end{condition}

% \begin{lemma}[McDiarmid’s inequality~\citep{mcdiarmid1989}] \label{lem:McDiarmid} Let $Z_1, Z_2, \ldots, Z_n$ be independent random variables on set $\calZ$ and $\phi : \calZ^n \to \Real$ satisfy bounded difference assumption (\codref{cond:BDC}). Then for all $t>0$, we have 
%     \begin{align*}
%         \prob\left( \phi(Z_1, Z_2, \ldots, Z_n) - \Expo{\phi(Z_1, Z_2, \ldots, Z_n)} \ge t \right) \le \exp{\left( -\frac{2t^2}{\sum_{i=1}^n c_i^2} \right) } 
%     \end{align*} 
%     and 
%     \begin{align*}
%         \prob\left( \phi(Z_1, Z_2, \ldots, Z_n) - \Expo{\phi(Z_1, Z_2, \ldots, Z_n)} \le -t \right) \le \exp{\left( -\frac{2t^2}{\sum_{i=1}^n c_i^2} \right) } \,
%     \end{align*} 
% \end{lemma}


% \section{Proofs from \secref{sec:ERM_training}}\label{app:proof_erm}

% \textbf{Additional notation {} {}} Let $m_1$ be the number of mislabeled points ($\wt S_M$) and $m_2$ be the number of correctly labeled points ($\wt S_C$). Note $m_1 + m_2 = m$. 


% \subsection{Proof of \thmref{thm:error_ERM}}


% \begin{proof}[Proof of \lemref{lem:fit_mislabeled}] 
%     The main idea of our proof is to regard 
%     the clean portion of the data 
%     ($S \cup \wt S_C$) as fixed.   
%     Then, there exists a classifier $f^*$ 
%     that is optimal over draws 
%     of the mislabeled data $\wt S_M$. 
% % 
%     % 
%     Formally, 
%     \begin{align}
%     f^* \defeq \argmin_{f \in \calF} \error_{\widecheck {\calD}} (f) \,, \label{eq:modified_ERM}
%     \end{align}
%     where $$\widecheck \calD = \frac{n}{m+n} \calS + \frac{m_1}{m+n} \wt \calS_C  + \frac{m_2}{m+n}\calDm \,.$$ That is, $\widecheck \calD$ a combination of 
%     the \emph{empirical distribution} 
%     over correctly labeled data $S \cup \wt S_C$
%     % in $S\cup \wt S$ 
%     and the (population) distribution 
%     over mislabeled data $\calDm$.
%     Recall that 
%     \begin{align}
%     \wh f \defeq \argmin_{f \in \calF} \error_{\calS \cup \wt S} (f) \,. \label{eq:orig_ERM}
%     \end{align}
%     % 
%     % 
%     Since, $\widehat f$ minimizes 0-1 error 
%     on $S \cup \wt S$, using ERM optimality on \eqref{eq:orig_ERM},  
%     we have 
%     \begin{align}
%         \error_{\calS \cup \wt \calS}(\widehat f) \le \error_{
%             \calS \cup \wt \calS}(f^*) \,.    \label{eq:step1}
%     \end{align}
%     Moreover, since $f^*$ is independent of $\wt S_M$, using Hoeffding's bound,
%     % \footnote{For a fully rigorous argument,
%     % refer to the complete proof in App.~\ref{app:proof_erm}.} 
%     we have with probability at least $1-\delta$ that
%     \begin{align}
%       \error_{\wt \calS_M}(f^*) \le \error_{ \calDm}(f^*) +  \sqrt{\frac{\log(1/\delta)}{2 m_1}} \,. \label{eq:step2} 
%     \end{align}
%     %$ 
%     %for some constant $c_1\le 1/2$. 
%     Finally, since $f^*$ is the optimal classifier on $\widecheck \calD$, 
%     we have 
%     \begin{align}
%         \error_{\widecheck \calD}(f^*) \le \error_{\widecheck \calD}(\widehat f) \label{eq:step3}
%     \end{align}
%      Now to relate \eqref{eq:step1} and \eqref{eq:step3}, we can re-write the \eqref{eq:step2} as follows: 
%     \begin{align}
%         \error_{\calS \cup \wt\calS}(f^*) \le \error_{ \widecheck \calD}(f^*) +  \frac{m_1}{m+n}\sqrt{\frac{\log(1/\delta)}{2 m_1}} \,. \label{eq:step4} 
%     \end{align}
%     Now we combine equations \eqref{eq:step1}, \eqref{eq:step4}, and \eqref{eq:step3}, to get 
%     \begin{align}
%         \error_{\calS \cup \wt \calS}(\wh f) \le \error_{\widecheck \calD}(\wh f) +  \frac{m_1}{m+n}\sqrt{\frac{\log(1/\delta)}{2 m_1}} \,, 
%     \end{align}
%     which implies 
%     \begin{align}
%         \error_{ \wt \calS_M}(\wh f) \le \error_{\calDm}(\wh f) + \sqrt{\frac{\log(1/\delta)}{2 m_1}} \,. \label{eq:lemma1_final}
%     \end{align}
%     Since $\wt S$ is obtained by randomly labeling an unlabeled dataset, we assume $2m_1 \approx m$ \footnote{Formally, with probability at least $1-\delta$, we have  $(m - 2m_1)\le \sqrt{m\log(1/\delta)/2}$ }. Moreover, using $\error_{\calDm} = 1 - \error_{\calD}$ we obtain the desired result.   
%     % Combining the above steps and using the fact 
%     % that $\error_\calD = 1- \error_{\calDm} $, 
%     % we obtain the desired result.
% \end{proof}

% \begin{proof}[Proof of \lemref{lem:mislabeled_error}]
%     Recall $\error_{\wt S} (f) = \frac{m_1}{m} \error_{\wt S_M}(f) + \frac{m_2}{m} \error_{\wt S_C}(f)$. Hence, we have 
%     \begin{align}
%         2\error_{\wt S}(f) - \error_{\wt S_M}(f) - \error_{\wt S_C}(f) &= \left(\frac{2m_1}{m} \error_{\wt S_M}(f) - \error_{\wt S_M}(f)\right) + \left(\frac{2m_2}{m} \error_{\wt S_C}(f) - \error_{\wt S_C}(f)\right) \\ &= \left(\frac{2m_1}{m} - 1\right) \error_{\wt S_M}(f) + \left(\frac{2m_2}{m} - 1 \right)\error_{\wt S_C} (f) \,.
%     \end{align} 
%     Since the dataset is randomly labeled, with probability at least $1-\delta$, we have  $\left(\frac{2m_1}{m} - 1\right) \le \sqrt{\frac{\log(1/\delta)}{2m}}$. Similarly, we have with probability at least $1-\delta$, $\left(\frac{2m_2}{m} - 1\right) \le \sqrt{\frac{\log(1/\delta)}{2m}}$. Using union bound, we have with probability at least $1-\delta$
%     % \begin{align}
%     %     2\error_{\wt S} - \error_{\wt S_M}(f) - \error_{\wt S_C}(f) \le \sqrt{\frac{\log(2/\delta)}{2m}} \left(\error_{\wt S_M}(f) + \error_{\wt S_C}(f) \right) \le 2\sqrt{\frac{\log(2/\delta)}{2m}} \,. \label{eq:lemma2_final}
%     % \end{align}
%     \begin{align}
%         2\error_{\wt S} - \error_{\wt S_M}(f) - \error_{\wt S_C}(f) \le \sqrt{\frac{\log(2/\delta)}{2m}} \left(\error_{\wt S_M}(f) + \error_{\wt S_C}(f) \right) \,. \label{eq:lemma2_prefinal}
%     \end{align}
%     With re-arranging $\error_{\wt S_M}(f) + \error_{\wt S_C}(f)$ and using the inequality $ 1- a\le \frac{1}{1+a} $, we have  
%     \begin{align}
%         2\error_{\wt S} - \error_{\wt S_M}(f) - \error_{\wt S_C}(f) \le 2\error_{\wt \calS} \sqrt{\frac{\log(2/\delta)}{2m}}  \,. \label{eq:lemma2_final}
%     \end{align}

%     % We obtain the desired result by using 
% \end{proof}

% \begin{proof}[Proof of \lemref{lem:clear_error}]
% % Recall 0-1 error on each point  $(x,y) \in S \cup \wt S$ is given by $\I{ f(x)\ne y}$.
% In the set of correctly labeled points $S \cup \wt S_C$, we have $S$ as a random subset of $S \cup \wt S_C$. Hence, using Hoeffding's inequality for sampling without replacement (\lemref{lem:hoeffding_sampling}), we have with probability at least $1-\delta$
% \begin{align}
%     \error_{\wt \calS_c} (\wh f)- \error_{\calS \cup \wt \calS_C}( \wh f) \le  \sqrt{\frac{\log(1/\delta)}{2m_2}} \,.
% \end{align}
% Re-writing $\error_{\calS \cup \wt \calS_C}( \wh f)$ as $\frac{m_2}{m_2 + n} \error_{\wt \calS_C }(\wh f) + \frac{n}{m_2 + n} \error_{\calS }(\wh f)$, we have with probability at least $1-\delta$
% \begin{align}
%   \left(\frac{n}{n+m_2}\right) \left(\error_{\wt \calS_c} (\wh f)- \error_{\calS}( \wh f) \right) \le  \sqrt{\frac{\log(1/\delta)}{2m_2}} \,.
% \end{align}
% As before, assuming $2m_2 \approx m$, we have with probability at least $1-\delta$ 
% \begin{align}
%     \error_{\wt \calS_c} (\wh f)- \error_{\calS}( \wh f) \le \left(1+\frac{m_2}{n}\right)  \sqrt{\frac{\log(1/\delta)}{m}} \le 1.5 \sqrt{\frac{\log(1/\delta)}{m}} \,. \label{eq:lemma3_final}
% \end{align} 
% \end{proof}

% \begin{proof}[Proof of \thmref{thm:error_ERM}] 
%     Having established these core intermediate results, we can now combine above three lemmas to prove the main result. 
%     In particular, we bound the population error on clean data ($\error_\calD(\wh f)$) as follows:  
%     \begin{enumerate}[(i)]
%         \item First, use \eqref{eq:lemma1_final}, to obtain an upper bound on the population error on clean data, i.e., with probability at least $1-\delta/4$, we have
%         \begin{align}
%             \error_{ \calD} (\wh f) \le 1 - \error_{ \wt \calS_M}(\wh f) + \sqrt{\frac{\log(4/\delta)}{m}} \,. 
%         \end{align}
%         \item  Second, use \eqref{eq:lemma2_final}, to relate the error on the mislabeled fraction with error on clean portion of randomly labeled data and error on whole randomly labeled dataset, i.e., with probability at least $1-\delta/2$, we have 
%         \begin{align}
%             - \error_{\wt S_M}(f) \le \error_{\wt S_C}(f) - 2\error_{\wt S}  + \sqrt{\frac{\log(4/\delta)}{2m}}  \,. 
%         \end{align} 
%         \item Finally, use \eqref{eq:lemma3_final} to relate the error on the clean portion of randomly labeled data and error on clean training data, i.e., with probability $1-\delta/4$, we have 
%         \begin{align}
%             \error_{\wt \calS_C} (\wh f)\le - \error_{\calS}( \wh f) + \left(1 + \frac{m}{2n} \right) \sqrt{\frac{\log(4/\delta)}{m}} \,. 
%         \end{align} 
%     \end{enumerate}

%     Using union bound on the above three steps, we have with probability at least $1-\delta$: 
%     \begin{align}
%         \error_\calD (\wh f) \le \error_{\calS}(\wh f)   + 1 - 2\error_{\wt \calS}(\wh f)   + (1/\sqrt{2} + 2.5)  \sqrt{\frac{\log(4/\delta)}{m}} \,.
%     \end{align}
%     Note that $(1/\sqrt{2} + 2.5)$ is a loose constant. In experiments, we use the ratio $\frac{m}{n}$
%     %  the exact error $\error_{\wt \calS}(\wh f)$ 
%     to evaluate R.H.S.    
% \end{proof}

% \subsection{Proof of \propref{prop:rademacher}}

% \begin{proof}[Proof of \propref{prop:rademacher}]
%     For a classifier $ f: \calX \to \{-1, 1\}$, we have $1 - 2\,\indict{ f(x) \ne y} = y \cdot f(x)$. Hence, by definition of $\error$, we have 
%     \begin{align}
%         1 -2\error_{\wt \calS}(f) = \frac{1}{m}\sum_{i=1}^m y_i \cdot f(x_i) \le \sup_{f \in \calF} \, \frac{1}{m} \sum_{i=1}^m y_i \cdot f(x_i)  \,. \label{eq:error_rademacher}
%     \end{align}
%     Note that for fixed inputs $(x_1, x_2, \ldots, x_m)$ in $\wt S$, $(y_1, y_2, \ldots y_m)$ are random labels. Define $\phi_1 (y_1, y_2, \ldots, y_m) \defeq \sup_{f \in \calF} \, \frac{1}{m} \sum_{i=1}^m y_i \cdot f(x_i)$. We have the following bounded difference condition on $\phi_1$. For all i, 
%     \begin{align}
%         \sup_{y_1, \ldots y_m, y_i^\prime \in \{-1, 1\}^{m+1} } \abs{ \phi_1 (y_1,\ldots, y_i, \ldots, y_m) - \phi_1 (y_1,\ldots, y_i^\prime, \ldots, y_m)  } \le 1/m \,. \label{cond1_rademacher}
%     \end{align} 
    
%     Similarly define $\phi_2 (x_1, x_2, \ldots, x_m) \defeq \Expt{ y_i \sim_U \{-1, 1\}  }{ \sup_{f \in \calF} \, \frac{1}{m}  \sum_{i=1}^m y_i \cdot f(x_i)}$. We have the following bounded difference condition on $\phi_2$. For all i,
%     \begin{align}
%         \sup_{x_1, \ldots x_m, x_i^\prime \in \calX^{m+1} } \abs{ \phi_2 (x_1,\ldots, x_i, \ldots, x_m) - \phi_1 (x_1,\ldots, x_i^\prime, \ldots, x_m)  } \le 1/m \,. \label{cond2_rademacher}
%     \end{align}
%     Using McDiarmid’s inequality (\lemref{lem:McDiarmid}) twice with Condition \eqref{cond1_rademacher} and \eqref{cond2_rademacher}, with probability at least $1-\delta$, we have
%     \begin{align}
%         \sup_{f \in \calF} \, \frac{1}{m} \sum_{i=1}^m y_i \cdot f(x_i)  - \Expt{x,y}{\sup_{f \in \calF} \, \frac{1}{m} \sum_{i=1}^m y_i \cdot f(x_i) } \le \sqrt{\frac{2\log(2/\delta)}{m}} \label{eq:final_rademacher}
%     \end{align} 
%     Combining \eqref{eq:error_rademacher} and \eqref{eq:final_rademacher}, we obtain the desired result. 
% \end{proof}


% \subsection{Proof of \thmref{thm:error_regularized_ERM}}

% Proof of \thmref{thm:error_regularized_ERM} follows similar to the proof of \thmref{thm:error_ERM}. Note that the same results in \lemref{lem:fit_mislabeled}, \lemref{lem:mislabeled_error}, and \lemref{lem:clear_error} hold in the regularized ERM case. However, the arguments in the proof of \lemref{lem:fit_mislabeled} changes slightly. Hence, we state and prove a lemma parallel to \lemref{lem:fit_mislabeled} for completeness. 

% \begin{lemma} \label{lem:lemma1_reg}
%     Assume the same setup as \thmref{thm:error_regularized_ERM}. 
%     Then for any $\delta >0$, with probability at least  $1-\delta$ 
%     over the random draws of mislabeled data $\wt S_M$, we have 
%     \begin{align}
%         \error_\calD(\widehat f)  \le 1 -\error_{\wt \calS_M}(\widehat f) + \sqrt{\frac{\log(1/\delta)}{m}}\,. 
%     \end{align} 
% \end{lemma}
% \begin{proof}
%     The main idea of the proof remains the same, i.e. regard 
%     the clean portion of the data 
%     ($S \cup \wt S_C$) as fixed.   
%     Then, there exists a classifier $f^*$ 
%     that is optimal over draws 
%     of the mislabeled data $\wt S_M$. 

    
%     Formally, 
%     \begin{align}
%     f^* \defeq \argmin_{f \in \calF} \error_{\widecheck {\calD}} (f)  + \lambda R(f) \,, \label{eq:modified_ERM_reg}
%     \end{align}
%     where $$\widecheck \calD = \frac{n}{m+n} \calS + \frac{m_1}{m+n} \wt \calS_C  + \frac{m_2}{m+n}\calDm \,.$$ That is, $\widecheck \calD$ a combination of 
%     the \emph{empirical distribution} 
%     over correctly labeled data $S \cup \wt S_C$
%     % in $S\cup \wt S$ 
%     and the (population) distribution 
%     over mislabeled data $\calDm$.
%     Recall that 
%     \begin{align}
%     \wh f \defeq \argmin_{f \in \calF} \error_{\calS \cup \wt S} (f) + \lambda R(f) \,. \label{eq:orig_ERM_reg}
%     \end{align}
%     % 
%     % 
%     Since, $\widehat f$ minimizes 0-1 error 
%     on $S \cup \wt S$, using ERM optimality on \eqref{eq:orig_ERM},  
%     we have 
%     \begin{align}
%         \error_{\calS \cup \wt \calS}(\widehat f) + \lambda R(\wh f) \le \error_{
%             \calS \cup \wt \calS}(f^*) + \lambda R(f^*) \,.    \label{eq:step1_reg}
%     \end{align}
%     Moreover, since $f^*$ is independent of $\wt S_M$, using Hoeffding's bound,
%     % \footnote{For a fully rigorous argument,
%     % refer to the complete proof in App.~\ref{app:proof_erm}.} 
%     we have with probability at least $1-\delta$ that
%     \begin{align}
%       \error_{\wt \calS_M}(f^*) \le \error_{ \calDm}(f^*) +  \sqrt{\frac{\log(1/\delta)}{2 m_1}} \,. \label{eq:step2_reg} 
%     \end{align}
%     %$ 
%     %for some constant $c_1\le 1/2$. 
%     Finally, since $f^*$ is the optimal classifier on $\widecheck \calD$, 
%     we have 
%     \begin{align}
%         \error_{\widecheck \calD}(f^*) + \lambda R(f^*) \le \error_{\widecheck \calD}(\widehat f) + \lambda R(\wh f) \label{eq:step3_reg}
%     \end{align}
%      Now to relate \eqref{eq:step1_reg} and \eqref{eq:step3_reg}, we can re-write the \eqref{eq:step2_reg} as follows: 
%     \begin{align}
%         \error_{\calS \cup \wt\calS}(f^*) \le \error_{ \widecheck \calD}(f^*) +  \frac{m_1}{m+n}\sqrt{\frac{\log(1/\delta)}{2 m_1}} \,. \label{eq:step4_reg} 
%     \end{align}
%     After adding $\lambda R(f^*)$ on both sides in \eqref{eq:step4_reg}, we combine equations \eqref{eq:step1_reg}, \eqref{eq:step4_reg}, and \eqref{eq:step3_reg}, to get 
%     \begin{align}
%         \error_{\calS \cup \wt \calS}(\wh f) \le \error_{\widecheck \calD}(\wh f) +  \frac{m_1}{m+n}\sqrt{\frac{\log(1/\delta)}{2 m_1}} \,, 
%     \end{align}
%     which implies 
%     \begin{align}
%         \error_{ \wt \calS_M}(\wh f) \le \error_{\calDm}(\wh f) + \sqrt{\frac{\log(1/\delta)}{2 m_1}} \,. \label{eq:lemma_reg_final}
%     \end{align}
%     Similar as before, since $\wt S$ is obtained by randomly labeling an unlabeled dataset, we assume 
%     $2m_1 \approx m$. Moreover, using $\error_{\calDm} = 1 - \error_{\calD}$ we obtain the desired result. 
% \end{proof}
% % \begin{proof}[Proof of ]
    
% % \end{proof}

% \subsection{Proof of \thmref{thm:multiclass_ERM}}

% We first state and prove lemmas parallel to three lemmas used in the proof of balanced binary case. Then we combine the results in the three lemmas to obtain the result in \thmref{thm:multiclass_ERM}. 

% Before stating the result, we define mislabeled distribution $\calDm$ for any $\calD$. While $\calDm$ and $\calD$ share 
% the same marginal distribution over $\calX$, 
% the distribution over labels $y$ 
% given an input $x\sim \calD_\calX$ is changed.
% In particular, for any $x$, the pdf over $y$ is changed to:  
% $p_{\calDm} (\cdot \vert x) \defeq \frac{1 - p_{\calD}(\cdot \vert x)}{k - 1}$.

% \begin{lemma} \label{lem:fit_mislabeled_multi}
%     Assume the same setup as \thmref{thm:multiclass_ERM}. 
%     Then for any $\delta >0$, with probability at least  $1-\delta$ 
%     over the random draws of mislabeled data $\wt S_M$, we have 
%     \begin{align}
%         \error_\calD(\widehat f)  \le (k-1)\left(1 -\error_{\wt \calS_M}(\widehat f)\right) + (k-1)\sqrt{\frac{\log(1/\delta)}{m}}\,. \label{eq:lemma1_multi}
%     \end{align}   
% \end{lemma} 

% \begin{proof}
%     The main idea of the proof remains the same, i.e. regard 
%     the clean portion of the data 
%     ($S \cup \wt S_C$) as fixed. 
%     Then, there exists a classifier $f^*$ 
%     that is optimal over draws 
%     of the mislabeled data $\wt S_M$. 
    
%     However, we need to be careful while relating population error on mislabeled data with population accuracy on clean data.   
%     While for binary classification,  we could upper bound $\error_{\wt \calS_M}$ 
%     with $1-\error_\calD$  (in the proof of \lemref{lem:fit_mislabeled}), 
%     for multiclass classification, 
%     error on the mislabeled data 
%     and accuracy on the clean data 
%     in the population 
%     are not so directly related.  
%     To establish \eqref{eq:lemma1_multi},
%     we break the error on the 
%     (unknown) mislabeled data 
%     into two parts: one term corresponds 
%     to predicting the true label on mislabeled data, 
%     and the other corresponds to predicting 
%     neither the true label 
%     nor the assigned (mis-)label.  
%     Finally, we relate these errors to their
%     population counterparts to establish \eqref{eq:lemma1_multi}. 
    
%     Formally, 
%     \begin{align}
%     f^* \defeq \argmin_{f \in \calF} \error_{\widecheck {\calD}} (f)  + \lambda R(f) \,, \label{eq:modified_ERM_reg2}
%     \end{align}
%     where $$\widecheck \calD = \frac{n}{m+n} \calS + \frac{m_1}{m+n} \wt \calS_C  + \frac{m_2}{m+n}\calDm \,.$$ That is, $\widecheck \calD$ a combination of 
%     the \emph{empirical distribution} 
%     over correctly labeled data $S \cup \wt S_C$
%     % in $S\cup \wt S$ 
%     and the (population) distribution 
%     over mislabeled data $\calDm$.
%     Recall that 
%     \begin{align}
%     \wh f \defeq \argmin_{f \in \calF} \error_{\calS \cup \wt S} (f) + \lambda R(f) \,. \label{eq:orig_ERM_reg2}
%     \end{align}
%     % 
%     % 
%     Following the exact steps from the proof of \lemref{lem:lemma1_reg}, with probability at least $1-\delta$, we have  
%     \begin{align}
%         \error_{ \wt \calS_M}(\wh f) \le \error_{\calDm}(\wh f) + \sqrt{\frac{\log(1/\delta)}{2 m_1}} \,. \label{eq:lemma1_final_multi_prev}
%     \end{align}
%     Similar to before, since $\wt S$ is obtained by randomly labeling an unlabeled dataset, we assume 
%     $\frac{k}{k-1} m_1 \approx m$. 
    
%     Now we will relate $\error_\calDm (\wh f)$ with $\error_{\calD}(\wh f)$. Let $y^T$ denote the (unknown) true label for a mislabeled point $(x, y)$ (i.e., label before replacing it with a mislabel). 
%     \begin{align}    
%          \Expt{(x, y) \in \sim \calDm}{\indict{ \wh f(x) \ne y }}  &= \underbrace{\Expt{(x, y) \in \sim \calDm}{\indict{ \wh f(x) \ne y \land \wh f(x) \ne y^T}}}_{\RN{1}} + \underbrace{\Expt{(x, y) \in \sim \calDm}{\indict{ \wh f(x) \ne y \land \wh f(x) = y^T}}}_{\RN{2}} \,. \label{eq:excess_term}
%     \end{align}
%     Clearly, term 2 is one minus the accuracy on the clean unseen data, i.e. 
%     \begin{align}
%         \RN{2} = 1 - \Expt{{x,y} \sim \calD}{ \indict{ \wh f(x) \ne y}} = 1- \error_{\calD}(\wh f) \,. \label{eq:term1}    
%     \end{align}
%     Next, we  relate term 1 with the error on the unseen clean data. We show that term 1 is equal to the error on the unseen clean data scaled by $\frac{k-2}{k-1}$ where $k$ is the number of labels. Using the definition of mislabeled distribution $\calDm$,  we have 
%     \begin{align}
%         \RN{1} = \frac{1}{k-1} \left( \Expt{(x, y) \in \sim \calD}{ \sum_{i \in \calY \land i\ne y}  \indict{ \wh f(x) \ne i \land \wh f(x) \ne y}} \right) = \frac{k-2}{k-1} \error_{\calD}(\wh f) \,.\label{eq:term2}
%     \end{align}    

%     Combining the result in \eqref{eq:term1}, \eqref{eq:term2} and \eqref{eq:excess_term}, we have 
%     \begin{align}
%         \error_{\calDm}(\wh f) = 1- \frac{1}{k-1} \error_{\calD}(\wh f) \,.\label{eq:combine_terms}
%     \end{align}
%     Finally, combining the result in \eqref{eq:combine_terms} with equation \eqref{eq:lemma1_final_multi_prev}, we have with probability $1-\delta$, 
%     \begin{align}
%       \error_{\calD}(\wh f) \le  (k-1) \left( 1- \error_{ \wt \calS_M}(\wh f) \right)  + (k-1) \sqrt{\frac{k \log(1/\delta)}{ 2(k-1)m}} \,. \label{eq:lemma1_final_multi}
%     \end{align}
% \end{proof}

% \begin{lemma} \label{lem:mislabeled_error_multi}
%     Assume the same setup as \thmref{thm:multiclass_ERM}.  Then for any $\delta >0$, with probability at least $1-\delta$ over the random draws of $\wt S$, we have  
%     % \begin{align}
%         $$\abs{k\error_{\wt \calS}(\widehat f) - \error_{\wt \calS_C}(\widehat f) -  (k-1)\error_{\wt \calS_M}(\widehat f) } \le  2k\sqrt{\frac{\log(4/\delta)}{2m}}\,. $$ % \label{eq:lemma2}
%     % \end{align}   
%     %  for some constant $c_3 \le 1.0\,$.
% \end{lemma} 


% \begin{proof}
%     Recall $\error_{\wt S} (f) = \frac{m_1}{m} \error_{\wt S_M}(f) + \frac{m_2}{m} \error_{\wt S_C}(f)$. Hence, we have 
%     \begin{align}
%         k\error_{\wt S}(f) - (k-1)\error_{\wt S_M}(f) - \error_{\wt S_C}(f) &= (k-1)\left(\frac{k m_1}{(k-1) m} \error_{\wt S_M}(f) - \error_{\wt S_M}(f)\right) + \left(\frac{km_2}{m} \error_{\wt S_C}(f) - \error_{\wt S_C}(f)\right) \\ &= k \left[ \left(\frac{m_1}{m} - \frac{k-1}{k}\right) \error_{\wt S_M}(f) + \left(\frac{m_2}{m} - \frac{1}{k} \right) \error_{\wt S_C} (f) \right] \,.
%     \end{align} 
%     Since the dataset is randomly labeled, we have with probability at least $1-\delta$, $\left(\frac{m_1}{m} - \frac{k-1}{k}\right) \le \sqrt{\frac{\log(1/\delta)}{2m}}$. Similarly, we have with probability at least $1-\delta$, $\left(\frac{m_2}{m} - \frac{1}{k}\right) \le \sqrt{\frac{\log(1/\delta)}{2m}}$. Using union bound, we have with probability at least $1-\delta$
%     % \begin{align}
%     %     2\error_{\wt S} - \error_{\wt S_M}(f) - \error_{\wt S_C}(f) \le \sqrt{\frac{\log(2/\delta)}{2m}} \left(\error_{\wt S_M}(f) + \error_{\wt S_C}(f) \right) \le 2\sqrt{\frac{\log(2/\delta)}{2m}} \,. \label{eq:lemma2_final}
%     % \end{align}
%     \begin{align}
%         k\error_{\wt S}(f) - (k-1)\error_{\wt S_M}(f) - \error_{\wt S_C}(f)  \le k \sqrt{\frac{\log(2/\delta)}{2m}} \left(\error_{\wt S_M}(f) + \error_{\wt S_C}(f) \right) \,. \label{eq:lemma2_final_multi}
%     \end{align}

%     % We obtain the desired result by using 
% \end{proof}

% \begin{lemma} \label{lem:clear_error_multi}
%     Assume the same setup as \thmref{thm:multiclass_ERM}. 
%     Then for any $\delta >0$, with probability at least $1-\delta$ 
%     over the random draws of $\wt S_C$ and $S$, we have 
%     % \begin{align}
%         $$\abs{\error_{\wt \calS_C}(\widehat f) - \error_{\calS}(\widehat f) } \le 1.5 \sqrt{\frac{k\log(2/\delta)}{2m}}\,.$$ %\label{eq:lemma3}
%     % \end{align}   
%     % for some constant $c_2 \le 1.2\,$.
% \end{lemma} 
% \begin{proof}
%     % Recall 0-1 error on each point  $(x,y) \in S \cup \wt S$ is given by $\I{ f(x)\ne y}$.
%     In the set of correctly labeled points $S \cup \wt S_C$, we have $S$ as a random subset of $S \cup \wt S_C$. Hence, using Hoeffding's inequality for sampling without replacement (\lemref{lem:hoeffding_sampling}), we have with probability at least $1-\delta$
%     \begin{align}
%         \error_{\wt \calS_c} (\wh f)- \error_{\calS \cup \wt \calS_C}( \wh f) \le  \sqrt{\frac{\log(1/\delta)}{2m_2}} \,.
%     \end{align}
%     Re-writing $\error_{\calS \cup \wt \calS_C}( \wh f)$ as $\frac{m_2}{m_2 + n} \error_{\wt \calS_C }(\wh f) + \frac{n}{m_2 + n} \error_{\calS }(\wh f)$, we have with probability at least $1-\delta$
%     \begin{align}
%       \left(\frac{n}{n+m_2}\right) \left(\error_{\wt \calS_c} (\wh f)- \error_{\calS}( \wh f) \right) \le  \sqrt{\frac{\log(1/\delta)}{2m_2}} \,.
%     \end{align}
%     As before, assuming $km_2 \approx m$, we have with probability at least $1-\delta$ 
%     \begin{align}
%         \error_{\wt \calS_c} (\wh f)- \error_{\calS}( \wh f) \le \left(1+\frac{m_2}{n}\right)  \sqrt{\frac{k\log(1/\delta)}{2m}} \le \left( 1 + \frac{1}{k}\right) \sqrt{\frac{k\log(1/\delta)}{2m}} \,. \label{eq:lemma3_final_multi}
%     \end{align} 
% \end{proof}

% \begin{proof}[Proof of \thmref{thm:multiclass_ERM}] 
%     Having established these core intermediate results, we can now combine above three lemmas. 
%     In particular, we bound the population error on clean data ($\error_\calD(\wh f)$) as follows:  
%     \begin{enumerate}[(i)]
%         \item First, use \eqref{eq:lemma1_final_multi}, to obtain an upper bound on the population error on clean data, i.e., with probability at least $1-\delta/4$, we have
%         \begin{align}
%             \error_{ \calD} (\wh f) \le (k-1)\left(1 - \error_{ \wt \calS_M}(\wh f) \right) + (k-1) \sqrt{\frac{k\log(4/\delta)}{2(k-1)m}} \,. 
%         \end{align}
%         \item  Second, use \eqref{eq:lemma2_final_multi}, to relate the error on the mislabeled fraction with error on clean portion of randomly labeled data and error on whole randomly labeled dataset, i.e., with probability at least $1-\delta/2$, we have 
%         \begin{align}
%             - (k-1)\error_{\wt S_M}(f) \le \error_{\wt S_C}(f) - k\error_{\wt S}  + k\sqrt{\frac{\log(4/\delta)}{2m}}  \,. 
%         \end{align} 
%         \item Finally, use \eqref{eq:lemma3_final_multi} to relate the error on the clean portion of randomly labeled data and error on clean training data, i.e., with probability $1-\delta/4$, we have 
%         \begin{align}
%             \error_{\wt \calS_C} (\wh f)\le - \error_{\calS}( \wh f) + \left(1 + \frac{m}{kn} \right) \sqrt{\frac{k\log(4/\delta)}{2m}} \,. 
%         \end{align} 
%     \end{enumerate}

%     Using union bound on the above three steps, we have with probability at least $1-\delta$: 
%     \begin{align}
%         \error_\calD (\wh f) \le \error_{\calS}(\wh f) + (k-1) - k\error_{\wt \calS}(\wh f)   + (\sqrt{k(k-1)} + k + \sqrt{k} + \frac{m}{n\sqrt{k}})  \sqrt{\frac{\log(4/\delta)}{2m}} \,.
%     \end{align}
%     % Note that $\frac{m}{n\sqrt{k}}$ is much smaller than the other terms in addition. Hence, we ignore this in the final bound. 
%     % Note that $(1/\sqrt{2} + 2.5)$ is a loose constant. In experiments, we use the ratio $\frac{m}{n}$
%     %  the exact error $\error_{\wt \calS}(\wh f)$ 
%     % to evaluate R.H.S.    
% \end{proof}

% \newpage
% \section{Proofs from \secref{sec:linear_models}}\label{app:proof_gd}

% We suppose that the parameters of the linear function 
% are obtained via gradient descent on 
% the following $L_2$ regularized problem: 
% \begin{align}
%     % n in denominator is avoided deliberately
%     \calL_S(w; \lambda) \defeq \sum_{i=1}^n{(w^Tx_i - y_i)^2} + \lambda \norm{w}{2}^2 \,, \label{eq:l2_MSE_app}   
% \end{align}
% where $\lambda\ge0$ is a regularization parameter. 
% We assume access to a clean dataset 
% $S = \{(x_i, y_i)\}_{i=1}^n \sim \calD^n$ 
% and randomly labeled dataset 
% $\wt S = \{(x_i, y_i)\}_{i=n+1}^{n+m} \sim \wt \calD^m$. 
% Let $\bX = [x_1, x_2, \cdots, x_{m+n}]$ 
% and $\by = [y_1, y_2, \cdots, y_{m+n}]$. 
% Fix a positive learning rate $\eta$ such that 
% $\eta \le 1/\left(\norm{\bX^T\bX}{\text{op}} + \lambda^2\right)$ 
% and an initialization $w_0 = 0$. 
% % \todos{Assumption made for simplicty}. 
% Consider the following gradient descent iterates 
% to minimize objective \eqref{eq:l2_MSE_app} on $S \cup \wt S$:
% \begin{align}
% w_t = w_{t-1} - \eta \grad_w \calL_{S \cup \wt S} (w_{t-1}; \lambda) \quad \forall t=1,2,\ldots \label{eq:GD_iterates_app}
% \end{align} 
% Then we have $\{ w_t\}$ converge to the limiting solution 
% $\wh w = \left( \bX^T\bX+\lambda \boldsymbol{I}\right)^{-1}\bX^T\by$. Define $\widehat f (x) \defeq f(x ; \wh w) $.  

% \subsection{\textcolor{red}{Errata}}

% We wish to correct the following error in the body: \codref{cond:error_stability} is not enough to guarantee the result in \thmref{thm:linear}. We now present a slightly stronger condition called \emph{hypothesis stability} under which we obtain a result similar to \thmref{thm:linear}. 

% This error doesn't change the main arguments of the proof where we show that the empirical train error is less than or equal to the leave-one-out error. We need a stronger condition to relate leave-one-out error with the population error of the original classifier. Specifically, while \codref{cond:error_stability} relates the average population error of leave-one-out classifiers with the population error of the original classifier, we need the new condition to show the concentration of the empirical leave-one-out error and  average population error of leave-one-out classifiers. 
% % main takeaway 

% Note that the new condition, while being stronger than the previous one, still doesn't imply generalization~\cite{bousquet2002stability,elisseeff2003leave,abou2019exponential}. Overall, the main results in \secref{sec:ERM_training} and takeaways of the paper remain unaffected by the error.  

% We now present the new condition and a corrected statement of \thmref{thm:linear}. Recall, for a given training set $S \sim \calD^n $, 
% we use $S_{(i)}$ to denote the training set $S$ 
% with the $i^{\text{th}}$ point removed.

% \begin{condition}[Hypothesis Stability] 
%     \label{cond:hypothesis_stability}
%     We have $\beta$ hypothesis stability 
%     if our training algorithm $\calA$ satisfies the following: 
%     \begin{align*}
%     % ${\sum_{i=1}^n \frac{\error_{\calD}( f(\calA, S_{(i)}))}{n} - \error_\calD(f(\calA, S))} \le \beta\,$.
%     \forall i \in \{1,2,\ldots, n\}, \quad  \Expt{\calS, (x,y) \in \calD}{ \abs{\error\left( f(x) ,y  \right) - \error\left( f_{(i)}(x), y \right) }} \le \frac{\beta}{n} \,,
%     \end{align*}
%     where $f_{(i)} \defeq f(\calA, S_{(i)})$ and $ f \defeq f(\calA, S)$.
% \end{condition}

% \begin{theorem}[Correct statement of \thmref{thm:linear}] \label{thm:new_linear}
%     Assume that this gradient descent algorithm satisfies \codref{cond:hypothesis_stability}
%     with $\beta=\calO(1)$.  
%     Then for any $\delta >0$, with probability at least $1-\delta$ 
%     over the random draws of datasets $\wt S$ and $S$, we have:
%     \begin{align}
%         \error_\calD(\widehat f) \le \error_\calS(\widehat f) + 1 - 2 \error_{\wt\calS}(\widehat f) + \left(\frac{1}{\sqrt{2}} + 1.5 \right) \sqrt{\frac{\log(4/\delta)}{m}} + \sqrt{\frac{4}{\delta}\left(\frac{1}{m} +\frac{3\beta}{m+n} \right)}  \,. \label{eq:gd_error}
%     \end{align} 
%     % for some constant $c\le 3.2$.
% \end{theorem}

% \subsection{Proof of \thmref{thm:new_linear}}
% We use a standard result from linear algebra, namely Shermann-Morrison formula~\citep{sherman1950adjustment} for matrix inversion:  

% \begin{lemma}[\citet{sherman1950adjustment}] \label{lem:sherman}
%     Suppose $\bA \in \Real^{n \times n}$ is an invertible square matrix and $u,v \in \Real^n$ are column vectors. Then $\bA + uv^T$ is invertible iff $1 + v^T \bA u \ne 0$ and in particular
%     \begin{align}
%         (\bA + u v^T)^{-1} = \bA^{-1}  - \frac{\bA^{-1} uv^T \bA^{-1} }{ 1 + v^T \bA^{-1} u} \,.
%     \end{align}   
% \end{lemma}
% \newcommand\byy[1]{\by_{\left(#1\right)}}
% \newcommand\bXX[1]{\bX_{\left(#1\right)}}
% \newcommand\ff[1]{\wh f_{\left(#1\right)}}

% For a given training set $S \cup \wt S_C$, define leave-one-out error on mislabeled points in the training data as $$\error_{\text{LOO}(\wt S_M) } = \frac{\sum_{(x_i, y_i) \in \wt S_M} \error( f_{(i)}( x_i), y_i)}{ \abs{\wt S_M }} \,, $$
% where $f_{(i)} \defeq f(\calA, (S \cup \wt S)_{(i)})$. To relate empirical leave-one-out error and population error with hypothesis stability condition, we use the following lemma:   

% \begin{lemma}[\citet{bousquet2002stability}] \label{lem:stability_error}
%     For the leave-one-out error, we have
%     \begin{align}
%         \Expo{ \left( \error_{\calDm}(\wh f) -\error_{\text{LOO}(\wt S_M) } \right)^2 } \le \frac{1}{2m_1}+  \frac{3\beta}{n + m}\,.
%     \end{align}   
%     % where $ f \defeq f(\calA, S \cup \wt S) $.
% \end{lemma}

% Proof of the above lemma is similar to the proof of  Lemma 9 in \citet{bousquet2002stability} and can be found in \appref{app:proof_lem_error}. 
% % 
% % Before presenting the result, we introduce some notation. 
% Before presenting the proof of \thmref{thm:new_linear}, we introduce some more notation. Let $\bX_{(i)}$ denote the matrix of covariates with $i^{\text{th}}$ point removed. Similarly let $\by_{(i)}$ be the array of responses with $i^{\text{th}}$ point removed. Define the corresponding regularized GD solution as $\wh w_{(i)} = \left( \bXX{i}^T\bXX{i}+\lambda \boldsymbol{I}\right)^{-1}\bXX{i}^T\byy{i}$. Define $\ff{i}(x) \defeq f(x ; \wh w_{(i)}) $.

% \begin{proof}[Proof of \thmref{thm:new_linear}]
%     Because squared loss minimization does not imply 0-1 error minimization, we cannot use arguments from \lemref{lem:fit_mislabeled}. This is the main technical difficulty. To compare the 0-1 error at a train point with an unseen point, 
%     we use the closed-form expression for $\widehat{w}$ and Shermann-Morrison formula to upper bound training error with leave-one-out cross validation error. 
    
%     The proof is divided into three parts: In part one, we show that 0-1 error on mislabeled points in the training set is lower than the error obtained by leave-one-out error at those points. In part two, we relate this leave-one-out error with the population error on mislabeled distribution using \codref{cond:hypothesis_stability}. While the empirical leave-one-out error is unbiased estimator of the average population error of leave-one-out classifiers, we need hypothesis stability to control the variance of empirical leave-one-out error. Finally in part three, we show that the error on the mislabeled training points can be estimated with just the randomly labeled and  clean training data (as in proof of \thmref{thm:error_ERM}).  

%     \textbf{Part 1 {} {}} First we relate training error with leave-one-out error.        
%     For any 
%     training point $(x_i, y_i)$ in $\wt S \cup S$, we have 
%     \begin{align}
%         \error(\wh f(x_i), y_i ) &= \indict{ y_i \cdot x_i^T \wh w < 0 } = \indict{ y_i \cdot x_i^T \left( \bX^T\bX+\lambda \boldsymbol{I}\right)^{-1}\bX^T\by < 0 } \\
%         &= \indict{ y_i \cdot x_i^T \underbrace{\left( \bXX{i}^T\bXX{i} + x_i ^T x_i +\lambda \boldsymbol{I}\right)^{-1}}_{\RN{1}} (\bXX{i}^T\byy{i} + y \cdot x_i) < 0 }
%     \end{align}
%     Letting $\bA = \left(\bXX{i}^T\bXX{i} +\lambda \boldsymbol{I}\right)$ and using \lemref{lem:sherman} on term 1, we have 
%     \begin{align}
%         \error(\wh f(x_i), y_i ) &= \indict{ y_i \cdot x_i^T \left[\bA^{-1} -  \frac{\bA^{-1} x_i x_i^T \bA^{-1}}{ 1 + x_i ^T \bA^{-1} x_i } \right] (\bXX{i}^T\byy{i} + y \cdot x_i) < 0 } \\
%         &= \indict{ y_i \cdot\left[ \frac{ x_i^T \bA^{-1} ( 1 + x_i ^T \bA^{-1} x_i ) -  x_i^T \bA^{-1} x_i x_i^T \bA^{-1}}{ 1 + x_i ^T \bA ^{-1}x_i } \right] (\bXX{i}^T\byy{i} + y \cdot x_i) < 0 } \\
%         &= \indict{ y_i \cdot\left[ \frac{ x_i^T \bA^{-1}}{ 1 + x_i ^T \bA ^{-1}x_i } \right] (\bXX{i}^T\byy{i} + y \cdot x_i) < 0 } \,.
%     \end{align}

%     Since $1 + x_i^T \bA^{-1} x_i > 0$, we have 
%     \begin{align}
%         \error(\wh f(x_i), y_i ) &= \indict{ y_i \cdot x_i^T \bA^{-1} (\bXX{i}^T\byy{i} + y \cdot x_i) < 0 } \\
%         &= \indict{ x_i^T \bA^{-1} x_i +  y_i \cdot x_i^T \bA^{-1} (\bXX{i}^T\byy{i}) < 0 } \\
%         &\le \indict{ y_i \cdot x_i^T \bA^{-1} (\bXX{i}^T\byy{i}) < 0 } = \error(\ff{i}(x_i), y_i ) \,.\label{eq:LOO_error}
%     \end{align}

%     Using \eqref{eq:LOO_error}, we have 
%     \begin{align}
%         \error_{\wt \calS_M } (\wh f) \le \error_{\text{LOO} (S_M)} \defeq \frac{\sum_{(x_i, y_i) \in \wt S_M} \error(\ff{i}(x_i), y_i ) }{\abs{\wt \calS_M}}\label{eq:LOO_error_final}
%     \end{align}
%     \textbf{Part 2 {}{}} We now relate RHS in \eqref{eq:LOO_error_final} with the population error on mislabeled distribution. To do this, we leverage \codref{cond:hypothesis_stability} and \lemref{lem:stability_error}. In particular, we have 

%     \begin{align}
%         \Expt{\calS \cup \wt \calS_M }{ \left(\error_{\calDm}(\wh f) - \error_{\text{LOO} (S_M)}\right)^2 } \le \frac{1}{2m_1} + \frac{3\beta}{m+n} \,.
%     \end{align}

%     Using Chebyshev's inequality, with probability at least $1-\delta$, we have 
%     \begin{align}
%         \error_{\text{LOO} (S_M)} \le  \error_{\calDm}(\wh f)   + \sqrt{\frac{1}{\delta}\left(\frac{1}{2m_1} +\frac{3\beta}{m+n} \right)} \,. \label{eq:final_mislabeled_linear}
%     \end{align}
    

%     \textbf{Part 3 {}{}} Combining \eqref{eq:final_mislabeled_linear} and \eqref{eq:LOO_error_final}, we have 

%     \begin{align}
%         \error_{\wt \calS_M } (\wh f) \le \error_{\calDm}(\wh f)   + \sqrt{\frac{1}{\delta}\left(\frac{1}{2m_1} +\frac{3\beta}{m+n} \right)} \,. \label{eq:linear_parallel_lem1}
%     \end{align}

%     Compare \eqref{eq:linear_parallel_lem1}, with \eqref{eq:lemma1_final} in the proof of \lemref{lem:fit_mislabeled}. We obtain a similar relationship between $\error_{\wt \calS_M }$ and $\error_{\calDm}$ but with a polynomial concentration instead of exponential concentration. 
%     In addition, since we just use concentration arguments to relate mislabeled error with the error on clean portion and unlabeled portion, we can directly use the results in \lemref{lem:mislabeled_error} and \lemref{lem:clear_error}. Therefore, combining results in \lemref{lem:mislabeled_error}, \lemref{lem:clear_error}, and \eqref{eq:linear_parallel_lem1} with union bound, we have with probability at least $1-\delta$

%     \begin{align}
%         \error_\calD(\widehat f) \le \error_\calS(\widehat f) + 1 - 2 \error_{\wt\calS}(\widehat f) + \left(\frac{1}{\sqrt{2}} + 1.5 \right) \sqrt{\frac{\log(4/\delta)}{m}} + \sqrt{\frac{4}{\delta}\left(\frac{1}{m} +\frac{3\beta}{m+n} \right)}  \,.
%     \end{align}
    

       
% \end{proof}

% \subsection{Discussion on \codref{cond:hypothesis_stability}}

% The quantity in LHS of \codref{cond:hypothesis_stability} measures how much the function learned by the algorithm (in terms of error on unseen point) will change when one point in the training set is removed. 
% % Discussion on exponential concentration and stronger condition. 
% Notice that hypothesis stability implies error stability, i.e., \codref{cond:error_stability} ~\cite{bousquet2002stability}.  In summary, while error stability allowed us to relate the average population error of the leave-one-out classifiers with the population error of the original classifier, we need hypothesis stability condition to control the variance of the empirical leave-one-out error. 

% Additionally, we note that while the dominating term in the RHS of \thmref{thm:new_linear} matches with the dominating term in ERM bound in \thmref{thm:error_ERM}, there is a polynomial concentration term (dependence on $1/\delta$ instead of $\log(\sqrt{1/\delta})$) in  \thmref{thm:new_linear}. 
% Since with hypothesis stability, we just bound the variance,  the polynomial concentration is due to the use of Chebyshev's inequality instead of an exponential tail inequality (as in \lemref{lem:fit_mislabeled}).
% Recent works have highlighted that slightly stronger condition than hypothesis stability can be used to obtained an exponential concentration for leave-one-out error~\citep{abou2019exponential}, but we leave this for future work for now. 
% % We leave 
% % However, the constants 

% % we also want to highlight  

% \subsection{Formal statement and proof of  of \propref{prop:early_stop}}

% Before formally presenting the result, we will introduce some notation.  By $\calL_{S}(w)$, we denote 
% the objective in \eqref{eq:l2_MSE_app} with $\lambda=0$. 
% Assume Singular Value Decomposition (SVD) of $\bX$  as $\sqrt{n} \bU \bS^{1/2} \bV^T$. Hence $\bX^T \bX = \bV \bS \bV^T$.
% Consider the GD iterates defined in \eqref{eq:GD_iterates_app}. 
% % 
% We now derive closed form expression for the $t^\text{th}$ iterate of gradient descent:  
% % 
% \begin{align}
%     w_t = w_{t-1} + \eta \cdot \bX^T (\by - \bX w_{t-1}) = (\bI - \eta \bV \bS \bV^T )w_{k-1} + \eta \bX^T \by \,.
% \end{align}
% Rotating by $\bV^T$, we get 
% \begin{align}
%     \wt w_t = (\bI - \eta\bS )\wt w_{k-1} + \eta \wt \by \,, \label{eq:GD_recur}
% \end{align}
% where $\wt w_t = \bV^T w_t $ and $\wt \by = \bV^T \bX^T \by$. Assuming the initial point $w_0 = 0$ and applying the recursion in \eqref{eq:GD_recur}, we get
% \begin{align}
%     \wt w_t = \bS ^{-1} ( \bI - (\bI - \eta \bS)^k ) \wt \by \,, 
% \end{align} 
% Projecting solution back to the original space, we have 
% \begin{align}
%      w_t = \bV \bS ^{-1} ( \bI - (\bI - \eta \bS)^k ) \bV^T \bX^T \by \,, 
% \end{align} 
% % We will work with this GD solution at any iterate $t$ in the next proposition. 
% Define $f_t(x) \defeq f(x;w_t)$ as the solution at the $t^{\text{th}}$ iterate. 
% Let $\wt w_{\lambda} = \argmin_{w} \calL_\calS (w;\lambda) = (\bX^T \bX + \lambda \bI)^{-1} \bX^T \by = \bV (\bS + \lambda \bI )^{-1} \bV^T \bX^T \by $. 
% % ) \,,$ for all $t=1,2,\ldots\,.$ 
% and define $\wt f_\lambda(x) \defeq f(x;\wt w_\lambda)$ as the regularized solution. 
% Assume $\kappa$ be the condition number of the population covariance matrix 
% and 
% let $s_\text{min}$ be the minimum positive singular value of the empirical covariance matrix. Our proof idea is inspired from recent work on relating gradient flow solution and regularized solution for regression problems \citep{ali2018continuous}. We will use the following lemma in the proof: 
% \begin{lemma} \label{lem:ineq_soln}
%     For all $x \in [0,1]$ and for all $ k \in \mathbb{N}$, we have (a) $ \frac{kx}{1+kx} \le 1- (1-x)^k$ and (b) $ 1- (1-x)^k \le 2 \cdot \frac{kx}{kx+1} $.
%     %  where $g(c)$ is a constant dependent on $c$. For $c = 1$, $g(c) = 2.0$.   
% \end{lemma}
% \begin{proof}
%     % [Proof of \lemref{lem:ineq_soln}]
%     % Part (a) is easy. 
%     Using $ (1-x)^k \le \frac{1}{1+kx}$, we have part (a). For part (b), we numerically maximize $\frac{ (1+kx ) (1 - (1-x)^k) }{kx}$ for all $k\ge 1$ and for all $x \in [0, 1]$.  
% \end{proof}

% % 
% % Next, 

% \begin{prop}[Formal statement of \propref{prop:early_stop}] \label{prop:formal_early_stop}
% Let $\lambda = \frac{1}{t\eta}$. For a training point $x$, we have 
% \begin{align*}
%     \Expt{x \sim \calS}{(f_t(x) - \wt f_\lambda(x))^2} &\le c(t,\eta) \cdot \Expt{x \sim \calS}{f_t(x)^2} \,, %\label{eq:early_stop}
% \end{align*}
% where $c(t, \eta) \defeq \min( 0.25, \frac{1}{s_\text{min}^2 t^2 \eta^2})$. Similarly for a test point, we have 
% \begin{align*}
%     \Expt{x \sim \calD_\calX}{(f_t(x) - \wt f_\lambda(x))^2} &\le \kappa \cdot c(t,\eta) \cdot \Expt{x \sim \calD_\calX}{f_t(x)^2} \,. %\label{eq:early_stop}
% \end{align*}
% \end{prop} 

% \begin{proof}
%     %%%%%%%%%%%%% 
%     We want to analyze the expected squared difference output of regularized linear regression with regularization constant $\lambda = \frac{1}{\eta t}$ and gradient descent solution at $t^\text{th}$ iterate. We separately expand the algebraic expression for squared difference at a training point and a test point. 
%     % We start by considering the difference  
%     Then the main step is to show that  $\left[ \bS ^{-1} ( \bI - (\bI - \eta \bS)^k )  - (\bS + \lambda \bI )^{-1}\right] \preceq c(\eta, t) \cdot \bS ^{-1} ( \bI - (\bI - \eta \bS)^k ) $.

%     %%%%%%%%%%%%%
    
%   \textbf{Part 1 {} {}} 
%     First, we will analyze the squared difference of output at a training point (for simplicity, we refer to $S \cup \wt S$ as $S$), i.e. 
%     \begin{align}
%         \Expt{ x \sim \calS }{\left(f_t(x) - \wt f_\lambda (x)\right)^2} &= \norm{\bX w_t - \bX \wt w_\lambda}{2}^2 =   \norm{\bX \bV \bS ^{-1} ( \bI - (\bI - \eta \bS)^t ) \bV^T \bX^T \by - \bX \bV (\bS + \lambda \bI )^{-1} \bV^T \bX^T \by }{2}^2 \\
%         &= \norm{\bX \bV \left(\bS ^{-1} ( \bI - (\bI - \eta \bS)^t ) - (\bS + \lambda \bI )^{-1} \right) \bV^T \bX^T \by  }{2} \\
%         &=  \by^T \bV \bX \left( \underbrace{\bS ^{-1} ( \bI - (\bI - \eta \bS)^t ) - (\bS + \lambda \bI )^{-1}}_{\RN{1}} \right)^2 \bS \bV^T \bX^T \by \label{eq:train_GD_rel}
%         %  (\bX \bV \bS ^{-1} ( \bI - (\bI - \eta \bS)^k ) \bV^T \bX^T \by)^T \bX \bV \bS ^{-1} ( \bI - (\bI - \eta \bS)^k ) \bV^T \bX^T \by
%     \end{align}
%     We now separately consider term 1. Substituting $\lambda = \frac{1}{t \eta}$, we get
%     \begin{align}
%         \bS ^{-1} ( \bI - (\bI - \eta \bS)^t ) - (\bS + \lambda \bI )^{-1} &= \bS^{-1} \left( ( \bI - (\bI - \eta \bS)^t ) - (\bI + \bS^{-1} \lambda )^{-1}\right) \\
%         &= \underbrace{\bS^{-1} \left( ( \bI - (\bI - \eta \bS)^t ) - (\bI + ( \bS t \eta)^{-1}  )^{-1}\right)}_{\bA}
%     \end{align}

%     We now separately bound the diagonal entries in matrix $\bA$. 
%     With $s_i$, we denote $i^{\text{th}}$ diagonal entry of $\bS$. Note that since $ \eta\le 1/\norm{S}{\text{op}}$, for all $i$, $\eta s_i  \le 1$.  Consider $i^{\text{th}}$ diagonal term (which is non-zero) of the diagonal matrix $\bA$, we have 
%     \begin{align}
%         \bA_{ii} = \frac{1}{s_i} \left(  1 - (1 - s_i \eta)^t - \frac{t \eta s_i}{1 + t \eta s_i } \right) &=  \frac{1 - (1 - s_i \eta)^t}{s_i} \left( \underbrace{ 1 - \frac{t \eta s_i}{(1 + t \eta s_i)(1 - (1 - s_i \eta)^t)}}_{\RN{2}} \right) \\ 
%          &\le \frac{1}{2}\left[ \frac{1 - (1 - s_i \eta)^t}{ s_i} \right] \tag*{(Using \lemref{lem:ineq_soln} (b))} \,.
%     \end{align} 
%     Additionally, we can also show the following upper bound on term 2: 
%     \begin{align}
%          1 - \frac{t \eta s_i}{(1 + t \eta s_i)(1 - (1 - s_i \eta)^t)} &= \frac{(1 + t \eta s_i)(1 - (1 - s_i \eta)^t) - t \eta s_i }{(1 + t \eta s_i)(1 - (1 - s_i \eta)^t)} \\
%          & \le  \frac{ 1 -  (1 - s_i \eta)^t - t \eta s_i (1 - s_i \eta)^t}{(1 + t \eta s_i)(1 - (1 - s_i \eta)^t)} \\
%          & \le \frac{1}{t\eta s_i} \,. \tag{Using \lemref{lem:ineq_soln} (a)}
%         %  &\le \frac{1}{2}\left[ \frac{1 - (1 - s_i \eta)^t}{ s_i} \right] \tag*{(Using \lemref{lem:ineq_soln})} \,.
%     \end{align} 

%     Combining both the upper bounds on each diagonal entry $\bA_{ii}$, we have 
%     \begin{align}
%     \bA \preceq c_1(\eta, t) \cdot \bS^{-1} ( \bI - (\bI - \eta \bS)^t ) \,, \label{eq:upperbound_diagonal}
%     \end{align}
%     where $c_1(\eta, t ) = \min(0.5, \frac{1}{t s_i \eta })$. Plugging this into \eqref{eq:train_GD_rel}, we have 
%     \begin{align}
%         \Expt{ x \sim \calS }{\left(f_t(x) - \wt f_\lambda (x)\right)^2} &\le c(\eta, t) \cdot \by^T \bV \bX  \left( \bS^{-1} ( \bI - (\bI - \eta \bS)^t ) \right)^2 \bS \bV^T \bX^T \by \\
%         &=   c(\eta, t) \cdot \by^T \bV \bX  \left( \bS^{-1} ( \bI - (\bI - \eta \bS)^t ) \right) \bS \left( \bS^{-1} ( \bI - (\bI - \eta \bS)^t ) \right) \bV^T \bX^T \by \\
%         & =  c(\eta, t) \cdot \norm{\bX w_t}{2}^2 \\
%         &= c(\eta, t) \cdot  \Expt{ x \sim \calS }{\left(f_t(x) \right)^2} \,,
%     \end{align}
%     where $c(\eta, t ) = \min(0.25, \frac{1}{t^2 s^2_i \eta^2 })$.

%     \textbf{Part 2 {} {}} With $\bSigma$, we denote the underlying true covariance matrix. We now consider the squared difference of output at an unseen point: 
%     \begin{align}
%         \Expt{ x \sim \calD_{\calX} }{\left(f_t(x) - \wt f_\lambda (x)\right)^2} &= \Expt{x \sim \calD_{\calX}}{\norm{x^T w_t - x^T \wt w_\lambda}{2}} \\
%         &=   \norm{x^T \bV \bS ^{-1} ( \bI - (\bI - \eta \bS)^t ) \bV^T \bX^T \by - x^T \bV (\bS + \lambda \bI )^{-1} \bV^T \bX^T \by }{2} \\
%         &= \norm{x^T \bV \left(\bS ^{-1} ( \bI - (\bI - \eta \bS)^t ) - (\bS + \lambda \bI )^{-1} \right) \bV^T \bX^T \by  }{2} \\
%         &= \by^T \bV \bX \left( \bS ^{-1} ( \bI - (\bI - \eta \bS)^t ) - (\bS + \lambda \bI )^{-1} \right) \bV^T \bSigma \bV \\ &\qquad \qquad \qquad \qquad \qquad \left( (\bI - (\bI - \eta \bS)^t ) - (\bS + \lambda \bI )^{-1} \right) \bV^T \bX^T \by \\
%         &\le \sigma_{\text{max}} \cdot \by^T \bV \bX \left( \underbrace{\bS ^{-1} ( \bI - (\bI - \eta \bS)^t ) - (\bS + \lambda \bI )^{-1}}_{\RN{1}} \right)^2 \bV^T \bX^T \by \,, \label{eq:test_GD_rel}
%         %  (\bX \bV \bS ^{-1} ( \bI - (\bI - \eta \bS)^k ) \bV^T \bX^T \by)^T \bX \bV \bS ^{-1} ( \bI - (\bI - \eta \bS)^k ) \bV^T \bX^T \by
%     \end{align}
%     where $\sigma_{\text{max}}$ is the maximum eigenvalue of the underlying covariance matrix $\bSigma$. Using the upper bound on term 1 in \eqref{eq:upperbound_diagonal}, we have 
%     \begin{align}
%         \Expt{ x \sim \calD_{\calX} }{\left(f_t(x) - \wt f_\lambda (x)\right)^2} &\le \sigma_{\text{max}} \cdot c(\eta, t) \cdot \by^T \bV \bX  \left( \bS^{-1} ( \bI - (\bI - \eta \bS)^t ) \right)^2 \bV^T \bX^T \by \\
%         &=   \kappa \cdot c(\eta, t) \cdot \sigma_{\text{min}}\cdot \norm{\bV \left( \bS^{-1} ( \bI - (\bI - \eta \bS)^t ) \right) \bV^T \bX^T \by}{2}^2 \\
%         &\le \kappa \cdot c(\eta, t) \cdot \left[ \bV \left( \bS^{-1} ( \bI - (\bI - \eta \bS)^t ) \right) \bV^T \bX^T \right]^T \bSigma \\
%         &\qquad \qquad \qquad \qquad \qquad \left[ \bV \left( \bS^{-1} ( \bI - (\bI - \eta \bS)^t ) \right) \bV^T \bX^T \right] \by \\
%         & = \kappa \cdot c(\eta, t) \cdot \Expt{x \sim \calD_{\calX}}{\norm{x^T w_t}{2}} \,.
%     \end{align}
% % 
% % 
%     % Since $ \eta\le 1/\norm{S}{\text{op}}$, invoking \lemref{lem:ineq_soln} to upper bound term 1 with
% \end{proof}


% \newpage
% \section{Additional experiments and details}\label{app:exp}
% \newcommand\tab[1][1cm]{\hspace*{#1}}

% \subsection{Datasets} \label{sec:app_dataset}

% \textbf{Toy Dataset {} {}} Assume fixed constants $\mu$ and $\sigma$. For a given label $y$, we simulate features $x$ in our toy classification setup as follows: 
% \begin{align*}
%     x \defeq \texttt{concat} \left[ x_1, x_2\right] \quad \text{where} \quad  x_1 \sim  \calN( y \cdot \mu, \sigma^2 I_{d \times d}) \ \  \text{and} \ \  x_1 \sim  \calN( 0, \sigma^2 I_{d \times d}) \,.
% \end{align*}  
% % where $y$ is the true label and $x$ is the corresponding feature vector. 
% In experiements throughout the paper, we fix dimention $d=100$, $\mu = 1.0 $, and $\sigma = \sqrt{d}$. Intuitively, $x_1$ carries the information about the underlying label and $x_2$ is additional noise independent of the underlying label. 

% \textbf{CV datasets {} {}} We use MNIST~\citep{lecun1998mnist} and CIFAR10~\cite{krizhevsky2009learning}. 
% % For binary tasks, 
% We produce a binary variant from the multiclass classification problem by mapping classes $\{0,1,2,3,4\}$ to label $1$ and $\{ 5,6,7,8,9\}$ to label $-1$. For CIFAR dataset, we also use the standard data augementation of random crop and horizontal flip. PyTorch code is as follows: 

% \texttt{(transforms.RandomCrop(32, padding=4),\\
% \tab transforms.RandomHorizontalFlip())}

% \textbf{NLP dataset {} {}} We use IMDb Sentiment analysis~\citep{maas2011learning} corpus.  

% \subsection{Architecture Details} 

% All experiments were run on NVIDIA GeForce RTX 2080 Ti GPUs. We used PyTorch~\citep{NEURIPS2019a9015} and Keras with Tensorflow~\citep{abadi2016tensorflow} backend for experiments. 
% % , ELMo embeddings~\citep{Peters:2018}, and Hugging Face Transformers~\citep{wolf-etal-2020-transformers}. 

% \textbf{Linear model {} {}} For the toy dataset, we simulate a linear model with scalar output and the same number of parameters as the number of dimensions.   

% \textbf{Wide nets {} {}} To simulate the NTK regime, we experiment with $2-$layered wide nets. The PyTorch code for 2-layer wide MLP is as follows: 


% \texttt{ nn.Sequential( \\
% \tab     nn.Flatten(),\\
% \tab    nn.Linear(input\_dims, 200000, bias=True),\\
% \tab    nn.ReLU(),\\
% \tab    nn.Linear(200000, 1, bias=True)\\
% \tab     )}


% We experiment both (i) with the first layer fixed at random initialization; (ii)  and updating both layers' weights.     

% \textbf{Deep nets for CV tasks {} {}} We consider a 4-layered MLP. The PyTorch code for 4-layer MLP is as follows: 

% \texttt{ nn.Sequential(nn.Flatten(), \\
% \tab        nn.Linear(input\_dim, 5000, bias=True),\\
% \tab        nn.ReLU(),\\
% \tab        nn.Linear(5000, 5000, bias=True),\\
% \tab        nn.ReLU(),\\
% \tab        nn.Linear(5000, 5000, bias=True),\\
% \tab        nn.ReLU(),\\
% % \tab        nn.Linear(5000, 5000, bias=True),\\
% % \tab        nn.ReLU(),\\
% \tab        nn.Linear(1024, num\_label, bias=True)\\
% \tab        )}

% For MNIST, we use $1000$ nodes instead of $5000$ nodes in the hidden layer. 
% % 
% We also experiment with convolutional nets. In particular, we use ResNet18 \citep{he2016deep}. Implementation adapted from:  \url{https://github.com/kuangliu/pytorch-cifar.git}. 

% \textbf{Deep nets for NLP {} {}} We use a simple LSTM model with embeddings intialized with ELMo embeddings~\citep{Peters:2018}. Code adapted from: \url{https://github.com/kamujun/elmo_experiments/blob/master/elmo_experiment/notebooks/elmo_text_classification_on_imdb.ipynb} 

% We also evaluate our bounds with a BERT model. In particular, we fine-tune an off-the-shelf uncased BERT model~\citep{devlin2018bert}. Code adapted from Hugging Face Transformers~\citep{wolf-etal-2020-transformers}: \url{https://huggingface.co/transformers/v3.1.0/custom_datasets.html}. 


% \subsection{Additonal experiments}

% 1. SGD with linear models on cross entropy and MSE loss. 

% 2. CE loss and SGD. GD with MSE loss 

% 3. Binary MNIST with MLP. multiclass MNIST  

% \textbf{Results on CIFAR 10 {} {}} 
% % 
% We plot epoch wise error curve for results in \tabref{table:multiclass}. We observe the same trend as in \figref{fig:error_CIFAR10}. Additionally, we plot an \emph{oracle bound} obtained by tracking the error on mislabeled data which nevertheless were predicted as true label. To obtain an exact emprical value of the oracle bound, we need underlying true labels for the randomly labeled data. 
% % Note that our bound in \thmref{thm:multiclass_ERM}, lower bounds the accuracy as predicted by the oracle bound. 
% While with just access to extra unlabeled data we cannot calculate oracle bound, we note that the oracle bound is very tight and never violated in practice underscoring an importamt aspect of generalization in multiclass problems. This highlight that even a stronger conjecture may hold in multiclass classification, i.e., error on mislabeled data (where nevertheless true label was predicted) lower bounds the population error on the distribution of mislabeled data and hence, the error on (a specific) mislabeled portion predicts the population accuracy on clean data. 
% % 
% On the other hand, the dominating term of in \thmref{thm:multiclass_ERM} is loose when compared with the oracle bound. The main reason, we believe is the pessimistic upper bound in \eqref{eq:lemma1_final_multi_prev} in the proof of \lemref{lem:fit_mislabeled_multi}. We leave an investigation on this gap for future. 
% % of fit 

% % However, oracle bound highlights two . One,  



% \begin{figure}[h]
%     \centering 
%     % \vspace{-15pt}
%     % \includegraphics[width=0.9\linewidth]{example-image-a}
%     \includegraphics[width=0.4\linewidth]{figures/CIFAR10-FNN.pdf} \hfil
%     \includegraphics[width=0.4\linewidth]{figures/CIFAR10-Resnet.pdf}
%     % \includegraphics[width=0.9\linewidth]{figures/{CIFAR10_rn=0.1_lr=0.2_wd=0.005}.png}
%     % \vspace{-10pt}
%     \caption{ Per epoch curves for CIFAR10 corresponding results in \tabref{table:multiclass}. As before, we just plot the dominating term in the RHS of \eqref{eq:multiclass_ERM} as predicted bound. Additionally, we also plot the predicted lower bound by the error on mislabeled data which nevertheless were predicted as true label. We refer to this as ``Oracle bound''. See text for more details. 
%     % 
%     % except for the stopping point. 
%     % The bound predicted by RATT (RHS in \eqref{eq:multiclass_ERM}) is vacuous. 
%     }\label{fig:error_epoch_CIFAR10}
%     % \vspace{-15pt}
% \end{figure}


% \textbf{Results on CIFAR 100 {} {}} 
% % 
% On CIFAR100, our bound in \eqref{eq:multiclass_ERM} yields vacous bounds. However, the oracle bound as explained above yields tight guarantees in the initial phase of the learning (i.e., when learning rate is less than $0.1$). 

% \begin{figure}[h]
%     \centering 
%     % \vspace{-15pt}
%     % \includegraphics[width=0.9\linewidth]{example-image-a}
%     \includegraphics[width=0.4\linewidth]{figures/CIFAR100-Resnet.pdf}
%     % \includegraphics[width=0.9\linewidth]{figures/{CIFAR10_rn=0.1_lr=0.2_wd=0.005}.png}
%     % \vspace{-10pt}
%     \caption{ Predicted lower bound by the error on mislabeled data which nevertheless were predicted as true label with ResNet18 on CIFAR100. We refer to this as ``Oracle bound''. See text for more details. 
%     % 
%     % except for the stopping point. 
%     The bound predicted by RATT (RHS in \eqref{eq:multiclass_ERM}) is vacuous. 
%     }\label{fig:error_CIFAR100}
%     % \vspace{-15pt}
% \end{figure}


% % \paragraph{Experiments on CIFAR100} 



% \subsection{Hyperparameter Details}


% \textbf{\figref{fig:error_CIFAR10} {} {}} We use clean training dataset of size $40,000$. We fix the amount of unlabeled data at $20\%$ of the clean size, i.e. we include additional $8,000$ points with randomly assigned labels. We use test set of $10,000$ points. For both MLP and ResNet, we use SGD with an initial learning rate of $0.1$ and momentum $0.9$. We fix the weight decay parameter at $5\times 10^{-4}$. After $100$ epochs, we decay the learning rate to $0.01$. We use SGD batch size of $100$. 

% \textbf{\figref{fig:error_binary} (a) {} {}} We obtain a toy dataset according to the process described in \secref{sec:app_dataset}. We fix $d=100$ and create a dataset of $50,000$ points with balanced classes. Moreover, we sample additional covariates with the same procedure to create randomly labeled dataset. For both SGD and GD training, we use a fixed learning rate $0.1$.    

% \textbf{\figref{fig:error_binary} (b) {} {}} Similar to binary CIFAR, we use clean training dataset of size $40,000$ and fix the amount of unlabeled data at $20\%$ of the clean dataset size. To train wide nets, we use a fixed learning of $0.001$ with GD and SGD. We decide the weight decay parameter and the early stopping point that maximizes our generalization bound (i.e. without peeking at unseen data ).  We use SGD batch size of $100$. 

% \textbf{\figref{fig:error_binary} (c) {} {}} With IMDb dataset, we use a clean dataset of size $20,000$ and as before, fix the amount of unlabeled data at $20\%$ of the clean data. To train ELMo model, we use Adam optimizer with a fixed learning rate $0.01$ and weight decay $10^{-6}$ to minimize cross entropy loss. We train with batch size $32$ for 3 epochs. To fine-tune BERT model, we use Adam optimizer with learning rate $5\times 10^{-5}$ to minimize cross entropy loss. We train with a batch size of $16$ for 1 epoch.    

% \textbf{\tabref{table:multiclass} {} {}} For multiclass datasets, we train both MLP and ResNet with the same hyperparameters as described before. We sample a clean training dataset of size $40,000$ and fix the amount of unlabeled data at $20\%$ of the clean size. We use SGD with an initial learning rate of $0.1$ and momentum $0.9$. We fix the weight decay parameter at $5\times 10^{-4}$. After $30$ epochs for ResNet and after $50$ epochs for MLP, we decay the learning rate to $0.01$.  We use SGD with batch size $100$. 
% For \figref{fig:error_CIFAR100}, we use the same hyperparameters as 
% CIFAR10 training, except we now decay learning rate after $100$ epochs. 


% In all experiments, to identify the best possible accuracy on just the clean data, we use the exact same set of hyperparamters except the stopping point. We choose a stopping point that maximizes test performance. 

% \subsection{Summary of experiments }

% \begin{center}
%     \begin{table}[H] 
%         \centering
%         \begin{tabular}{|c|c|c|c|} 
%         \hline
%         Classification type & Model category & Model & Dataset  \\ [0.5ex] 
%         \hline
%         \hline
%         \multirow{9}{*}{Binary} & Low dimensional & Linear model & Toy Gaussain dataset  \\
%                         \cline{2-4}
%                          & \multirow{1}{*}{Overparameterized linear nets} 
%                         %  & Linear model & Toy Gaussain dataset \\
%                         %  \cline{3-4}
%                         %  & & 2-layer wide net& Toy Gaussain dataset \\
%                         %  \cline{3-4}
%                          & 2-layer wide net & Binary MNIST \\
%                          \cline{2-4}                 
%                          & \multirow{6}{*}{Deep nets} & \multirow{2}{*}{MLP} & Binary MNIST \\
%                          \cline{4-4}
%                          & &  & Binary CIFAR \\
%                          \cline{3-4}
%                          &  & \multirow{2}{*}{ResNet} & Binary MNIST \\
%                          \cline{4-4}
%                          & &  & Binary CIFAR \\
%                          \cline{3-4}
%                          &  & ELMo-LSTM model & IMDb Sentiment Analysis \\
%                          \cline{3-4}
%                          & & BERT pre-trained model & IMDb Sentiment Analysis \\
%         \hline
%         \multirow{5}{*}{Multiclass} & \multirow{5}{*}{Deep nets} & \multirow{2}{*}{MLP} & MNIST \\
%                         \cline{4-4} 
%                         & & & CIFAR10 \\                   
%                         \cline{3-4}
%                          &   & \multirow{3}{*}{ResNet} & MNIST \\
%                          \cline{4-4}
%                          &   & & CIFAR10 \\
%                          \cline{4-4}
%                          &   & & CIFAR100 \\
%         \hline
%         \end{tabular}
%         % \caption{Summary of experiments performed} \label{table:experiments}
%     \end{table}    
%     % \footnotetext[6]{We use both MSE loss and cross-entropy loss.}
%     % \footnotetext[6]{We try 2 variants: one with a fixed first layer and the other with both layers trainable.}
% \end{center}

% \newpage
% \section{Proof of \lemref{lem:stability_error}} \label{app:proof_lem_error}

% \begin{proof}[Proof of \lemref{lem:stability_error}]
%     Recall, we have a training set $S \cup \wt S_C$. We defined leave-one-out error on mislabeled points as $$\error_{\text{LOO}(\wt S_M) } = \frac{\sum_{(x_i, y_i) \in \wt S_M} \error( f_{(i)}( x_i), y_i)}{ \abs{\wt S_M }} \,, $$
%     where $f_{(i)} \defeq f(\calA, (S \cup \wt S)_{(i)})$. Define $S^\prime \defeq S \cup \wt S$. Assume $(x,y)$ and $(x^\prime,y^\prime)$ as i.i.d. samples from ${\calDm}$. 
%     Using Lemma 25 in \citet{bousquet2002stability}, we have
%     \begin{align*}
%         \Expo{ \left( \error_{\calDm}(\wh f) -\error_{\text{LOO}(\wt S_M) } \right)^2 } \le & \Expt{ S^\prime, (x,y), (x^\prime,y^\prime) }{ \error(\wh f(x), y ) \error(\wh f(x^\prime), y^\prime )} - 2 \Expt{ S^\prime, (x,y) }{ \error(\wh f(x), y ) \error(f_{(i)}(x_i), y_i )} \\
%         & + \frac{m_1-1}{m_1}\Expt{ S^\prime }{  \error(f_{(i)}(x_i), y_i )  \error(f_{(j)}(x_j), y_j )} + \frac{1}{m_1} \Expt{ S^\prime }{  \error(f_{(i)}(x_i), y_i ) } \,. \numberthis \label{eq:main_reln}
%     \end{align*}
%     We can rewrite the equation above as : 
%     \begin{align*}
%         \Expo{ \left( \error_{\calDm}(\wh f) -\error_{\text{LOO}(\wt S_M) } \right)^2 } \le &  \, \underbrace{\Expt{ S^\prime, (x,y), (x^\prime,y^\prime) }{ \error(\wh f(x), y ) \error(\wh f(x^\prime), y^\prime ) - \error(\wh f(x), y ) \error(f_{(i)}(x_i), y_i )}}_{\RN{1}} \\
%         & + \underbrace{\Expt{ S^\prime }{  \error(f_{(i)}(x_i), y_i )  \error(f_{(j)}(x_j), y_j ) -  \error(\wh f(x), y ) \error(f_{(i)}(x_i), y_i )}}_{\RN{2}} \\ &+ \underbrace{\frac{1}{m_1} \Expt{ S^\prime }{  \error(f_{(i)}(x_i), y_i ) - \error(f_{(i)}(x_i), y_i )  \error(f_{(j)}(x_j), y_j ) }}_{\RN{3}} \,. \numberthis \label{eq:main_reln2}
%     \end{align*}
    
%     We will now bound term $\RN{3}$.  Using Schwarz's inequality, we have
    
%     \begin{align}
%         \Expt{ S^\prime }{  \error(f_{(i)}(x_i), y_i ) - \error(f_{(i)}(x_i), y_i )  \error(f_{(j)}(x_j), y_j ) }^2 &\le  \Expt{ S^\prime }{  \error(f_{(i)}(x_i), y_i ) }^2 \Expt{S^\prime}{1 -   \error(f_{(j)}(x_j), y_j ) }^2 \\
%         &\le \frac{1}{4} \label{eq:term1_lem12}
%     \end{align}
    
%     Note that since $(x_i,y_i)$, $(x_j ,y_j )$, $(x,y)$, and $(x^\prime, y^\prime)$ are all from same distribution $\calDm$, we directly incorporate the bounds on term $\RN{1}$ and $\RN{2}$ from proof of Lemma 9 in \citet{bousquet2002stability}. Combining that with \eqref{eq:term1_lem12} and our definition of hypothesis stability in \codref{cond:hypothesis_stability}, we have the required claim. 
    
    
%     % We now re-write term $\RN{1}$ as
%     % \begin{align*}
%     %         &\Expt{S^\prime, (x,y), (x^\prime,y^\prime) }{ \error(\wh f(x), y ) \error(\wh f(x^\prime), y^\prime ) - \error(\wh f(x), y ) \error(f_{(i)}(x_i), y_i )} \\ & \qquad = \Expt{ S^\prime, (x,y), (x^\prime,y^\prime) }{ \error(\wh f(x), y ) \error(\wh f  (x^\prime), y^\prime ) - \error(\wh f ^\prime(x), y ) \error(f_{(i)}(x^\prime), y^\prime )} \tag{Exchanging $(x_i, y_i)$ with $(x^\prime, y^\prime)$ in the second term} \\
%     %         & \qquad = \Expt{ S^\prime, (x,y), (x^\prime,y^\prime) }{  \left(\error(\wh f(x), y )-  \error(f_{(i)}(x), y ) \right) \error(\wh f  (x^\prime), y^\prime )  } \\
%     %         & \qquad  + \Expt{ S^\prime, (x,y), (x^\prime,y^\prime) }{  \left(\error(f_{(i)}(x), y ) -\error(\wh f ^\prime(x), y ) \right) \error(\wh f  (x^\prime), y^\prime )}  \\
%     %         & \qquad +\Expt{ S^\prime, (x,y), (x^\prime,y^\prime) }{  \left( \error(\wh f  (x^\prime), y^\prime ) -  \error(f_{(i)}(x^\prime), y^\prime ) \right) \error(\wh f ^\prime(x), y ) }  \,, \numberthis \label{eq:term1_final}
%     % \end{align*}
%     % where $\wh f^\prime$ is the classifier obtained by training on $ S^\prime_{(i)} \cup \{ (x^\prime, y^\prime) \} $. Similarly we can re-write term $\RN{2}$ as 
%     % \begin{align*}
%     %     & \Expt{ S^\prime }{  \error(f_{(i)}(x_i), y_i )  \error(f_{(j)}(x_j), y_j ) -  \error(\wh f(x), y ) \error(f_{(i)}(x_i), y_i )} \\
%     %     &\quad  = \Expt{ S^\prime, (x,y), (x^\prime,y^\prime)}{  \error(f^{\prime\prime}_{(i)}(x), y )  \error(f_{(j)}^{\prime}(x^\prime), y^\prime ) -  \error(\wh f(x), y ) \error(f_{(i)}(x_i), y_i )} \tag{Exchanging $(x_i, y_i)$ with $(x, y)$ and $(x_j, y_j)$ with $(x^\prime, y^\prime)$ in the first term}\\
%     %     &\quad = \Expt{ S^\prime, (x,y), (x^\prime,y^\prime)}{  \error(f^{\prime\prime}_{(j)}(x), y )  \error(f_{(i)}^{\prime}(x^\prime), y^\prime ) -  \error(\wh f^\prime (x), y ) \error(f^\prime_{(j)}(x^\prime), y^\prime )} \tag{Exchanging $(x_i, y_i)$ and $(x_j, y_j)$ and then replacing $(x_j, y_j)$ with $(x^\prime, y^\prime)$ in the second term} \\
%     %     & \quad = \Expt{ S^\prime, (x,y), (x^\prime,y^\prime) }{  \left( \error(f_{(i)}^{\prime}(x^\prime), y^\prime )   -  \error(\wh f^{\prime\prime}  (x^\prime), y^\prime ) \right)  \error(f^{\prime\prime}_{(j)}(x), y )   } \\
%     %     & \quad  + \Expt{ S^\prime, (x,y), (x^\prime,y^\prime) }{  \left( \error(f^{\prime\prime}_{(j)}(x), y )  -\error(\wh f ^\prime(x), y ) \right) \error(\wh f^{\prime\prime}  (x^\prime), y^\prime )  }  \\
%     %     & \quad+ \Expt{ S^\prime, (x,y), (x^\prime,y^\prime) }{  \left( \error(\wh f^{\prime\prime}  (x^\prime), y^\prime )  -  \error(f^\prime_{(j)}(x^\prime), y^\prime ) \right)  \error(\wh f^\prime (x), y ) }   \\
%     %     & \quad = \Expt{ S^\prime, (x,y), (x^\prime,y^\prime) }{  \left( \error(f_{(i)}^{\prime}(x^\prime), y^\prime )   -  \error(\wh f (x^\prime), y^\prime ) \right)  \error(f_{(i)}(x_j), y_j )   } \\
%     %     & \quad  + \Expt{ S^\prime, (x,y), (x^\prime,y^\prime) }{  \left( \error(f^{\prime\prime}_{(j)}(x), y )  -\error(\wh f (x), y ) \right) \error(\wh f^{\prime\prime}  (x_j), y_j )  }  \\
%     %     & \quad+ \Expt{ S^\prime, (x,y), (x^\prime,y^\prime) }{  \left( \error(\wh f^{\prime\prime}  (x^\prime), y^\prime )  -  \error(f^\prime_{(j)}(x^\prime), y^\prime ) \right)  \error(\wh f^\prime (x^\prime), y^\prime ) }  \,, \numberthis \label{eq:term2_final}
%     % \end{align*}
%     % where $f^{\prime\prime}_{(j)}$ is trained on $S^\prime_{(j,i)} \cup {(x,y)}$, $f^{\prime}_{(i)}$ is trained on $S^\prime_{(j,i)} \cup {(x^\prime,y^\prime)}$, and $\wh f^{\prime\prime} $ is trained on $S^\prime_{(j)} \cup {(x,y)}$. Note in the last line we replaced $(x,y)$ by $(x_j, y_j)$ in the first term, replaced $(x^\prime,y^\prime)$ by $(x_j, y_j)$ in the second term and exchanged $(x_i,y_i)$ with $(x_j,y_j)$ and also $(x,y)$ and $(x^\prime, y^\prime)$
    
    
% \end{proof}

\end{document}
