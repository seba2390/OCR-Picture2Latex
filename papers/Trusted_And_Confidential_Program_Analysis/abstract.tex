%%
%% The abstract is a short summary of the work to be presented in the
%% article.
\begin{abstract}

% Bill
We develop the concept of Trusted and Confidential Program Analysis (\tcpa) which enables 
program certification to be used where previously there was insufficient trust. 
Imagine a scenario where a producer may not be trusted to certify its own software (perhaps by a foreign regulator), 
and the producer is unwilling to release its sources and detailed design to any external body. 
We present a protocol that can, using trusted computing based on encrypted sources, create 
certification via which all can trust the delivered object code without revealing the 
unencrypted sources to any party. Furthermore, we describe 
a realization of \tcpa with trusted execution environments (\tee) that enables general and efficient 
computation. We have implemented the \tcpa protocol in a system called \tool for web assembly 
architectures. In our evaluation with \neval benchmark cases, \tool managed to finish the 
analysis with relatively slight overheads.

% Previous
%The import and export of IT products (\eg, software, hardware, services \etc) require 
%that the product itself is compliant with local and international regulations. More importantly, 
%such regulatory processes need to be enforced in a both trusted and confidential way, \ie, form 
%a consensus on the compliance of a product among relevant parties without leaking further 
%information to any of them. While there are many techniques in the literature designed to 
%check systems against a specific set of properties, little is studied to fulfill the 
%aforementioned demand. To address the problem, we highlight and formalize the technique 
%of \emph{Trusted and Confidential Program Analysis} (\tcpa) for software. Specifically, 
%\tcpa introduces a protocol to verify whether a piece of imported or exported software (with a given 
%encrypted source code) proposed by $A$ is compliant with a group of 
%public regulatory properties required by $B$; this compliance status can be 
%trusted by any third party $C$ without new knowledge of the software. Furthermore, we describe 
%a realization of \tcpa with trusted execution environments (\tee) that enables general and efficient 
%computation. We have implemented the \tcpa protocol in a system called \tool for web assembly 
%architectures. In our evaluation with \neval benchmark cases, \tool managed to finish the 
%analysis with relatively slight overheads: \overheadtime and \overheadmemory for time and memory usage, respectively.
%\tool is currently accessible via \toolUrl.



\end{abstract}

%%
%% The code below is generated by the tool at http://dl.acm.org/ccs.cfm.
%% Please copy and paste the code instead of the example below.
%%
\begin{CCSXML}
	<ccs2012>
	<concept>
	<concept_id>10002978.10003022.10003028</concept_id>
	<concept_desc>Security and privacy~Domain-specific security and privacy architectures</concept_desc>
	<concept_significance>300</concept_significance>
	</concept>
	</ccs2012>
\end{CCSXML}

\ccsdesc[300]{Security and privacy~Domain-specific security and privacy architectures}

%%
%% Keywords. The author(s) should pick words that accurately describe
%% the work being presented. Separate the keywords with commas.
\keywords{Program analysis, regulatory property, trusted execution environment}

%% A "teaser" image appears between the author and affiliation
%% information and the body of the document, and typically spans the
%% page.
%\begin{teaserfigure}
%  \includegraphics[width=\textwidth]{sampleteaser}
%  \caption{Seattle Mariners at Spring Training, 2010.}
%  \Description{Enjoying the baseball game from the third-base
%  seats. Ichiro Suzuki preparing to bat.}
%  \label{fig:teaser}
%\end{teaserfigure}

%\copyrightyear{2020}
%\acmYear{2020}
%%\setcopyright{rightsretained}
%\acmConference[ICSE '20 Companion]{42nd International Conference on Software Engineering Companion}{May 23--29, 2020}{Seoul, Republic of Korea}
%\acmBooktitle{42nd International Conference on Software Engineering Companion (ICSE '20 Companion), May 23--29, 2020, Seoul, Republic of Korea}
%\acmDOI{10.1145/3377812.3382157}
%\acmISBN{978-1-4503-7122-3/20/05}

%%
%% This command processes the author and affiliation and title
%% information and builds the first part of the formatted document.
\maketitle