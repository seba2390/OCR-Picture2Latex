This paper offers a tentative solution to the following problem: vendor Alice wishes to sell executable software E to clients such as Bob. Bob does not trust Alice and wants to examine the high-level design or sources behind E, but Alice is unwilling to do this to protect her intellectual property. 
In particular, the rapid growth of the international IT market, where IT products (\eg, software, hardware, services, \etc) 
are widely exported from one country to another has led to many such issues. Unfortunately, IT 
products are notoriously prone to bugs and security risks no matter in what 
forms they are deployed and in what environments they are running. Any vulnerable component 
in an IT product may cause a great loss to customers, \eg, national security 
threats, abuse of personal data, manipulation on digital asset, \etc In practice, the 
import and export of IT products are commonly required to be strictly compliant with 
local and international regulations. More importantly, such regulatory processes have to 
take place in a both \emph{trusted} and \emph{confidential} way, \ie, form a consensus 
on the regulatory compliance of an IT product among a group of relevant parties without 
leaking further sensitive information, \eg, source code, hardcoded values, \etc Figure~\ref{fig:illustrative} 
shows an illustrative example for further explanation.

\begin{figure}[h]
\centering
\includegraphics[width=.8\linewidth]{illustrative.pdf}
\caption{\label{fig:illustrative}Multiparty trusted and confidential regulation.}
\end{figure}



\myparagraph{Illustrative Example}
We consider the case in which \emph{B} and \emph{C} have jointly developed software and exported it to \emph{A}. 
As a service provider, \emph{A} imported the software to set up a public service which further reached 
\emph{D} as a general user. Although this case is designed for explanation, it actually 
abstracts a typical scenario where nowadays IT products are deployed and used worldwide. 
Considering that the four participants may be from different countries, 
they are required to follow various regulations. 
For \emph{B} and \emph{C} as software exporters, they need to be compliant with local export regulations. 
For \emph{A} as a technology importer, his or her obligation is to check that the imported software introduces no 
regulatory risks, \eg, national security issues. From the perspective of \emph{D}, his or her country might 
only allow the service to reach domestic users if it poses no threats to the public good, \eg, user 
privacy. In practice, it is hard to enforce the regulatory processes among these 
four parties due to the fact that the source code of the software is strictly confidential 
and therefore cannot be directly shared in a straightforward manner. Consequently, 
no single party in this case is able to believe that the software indeed delivers the 
required regulatory compliance across different countries.


\myparagraph{Trusted and Confidential Program Analysis}
To address this problem, we proposed a novel protocol in this paper to enable trusted and confidential 
program analysis (\tcpa) for checking regulatory compliance of software across multiple parties without mutual 
trust. More specifically, the proposed protocol guarantees that a) imported or exported software $E$ 
is indeed built from a given piece of secret source code $C$, b) both $E$ and $C$ are compliant with a 
set of mutually agreed regulatory properties $P=\{ p_1, p_2, \cdots, p_n \}$, \ie, $E, S \vDash P$ and 
c) the compliance of $E$ and $S$ is verifiable without revealing sensitive information of $S$. 
Furthermore, we described a realization of \tcpa called \tool using 
trusted execution environments (\tee{}s) and applied the system for analysis of web assembly (\wasm) programs. In the preliminary evaluation with \neval 
benchmark files, \tool finished the analysis tasks with slight overheads of \overheadtime and 
\overheadmemory in terms of time and memory usage, respectively; the baseline system executes the same analysis tasks without relying on TEEs. These overheads seem entirely acceptable to us given the added guarantees in terms of trust and confidentiality that the use of TEEs provides. 


We summarize our main contributions below.
\begin{itemize}[leftmargin=*]
\item We describe the problem of trusted and confidential program analysis and present a 
formalization of it to guide our subsequent research.

\item We propose the first protocol to enable \tcpa in practice, which is consistent with 
popular trusted computing technologies such as \tee{}s. 

\item We realize the protocol using AMD's SEV \tee implementation. We develop the system \tool for verifying web assembly programs 
via \tcpa; the first of its kind, to the best of the authors' knowledge.

\item We have conducted a large-scale evaluation of \tool and here report empirical evidence to 
demonstrate the feasibility of applying \tcpa in practice.
\end{itemize}

\myparagraph{Paper Organization}
The rest of this paper is organized as follows. \S\ref{sec:bg} introduces background information. 
\S\ref{sec:method} presents an in-depth explanation of the \tcpa protocol. \S\ref{sec:design} 
describes the system design of \tool. \S\ref{sec:eval} summarizes empirical results of the 
evaluation and \S\ref{sec:rw} discusses related works. \S\ref{sec:conclusion} concludes the paper.


%\begin{figure*}
%	\centering
%	\includegraphics[width=.85\linewidth]{res/blockeye-workflow.pdf}
%	\caption{\label{fig:framework}%
%		The general workflow of \tool.}
%\end{figure*}






