\begin{abstract}

Deep learning is a key approach for object detection in the environment perception function of Cooperative Intelligent Transportation Systems (C-ITS) with autonomous vehicles and a smart traffic infrastructure. %
The performance of the object detection strongly depends on the data volume in the model training, which is usually not sufficient due to the limited data collected by a typically small fleet of test vehicles. %
In today's C-ITS, smart traffic participants are capable to timely generate and transmit a large amount of data. However, these data can not be used for model training directly due to privacy constraints. %
In this paper, we introduce a federated learning framework coping with Hierarchical Heterogeneity (\emph{H2Fed}), which can notably enhance the conventional %OEM 
pre-trained deep learning model. The framework exploits data from connected public traffic agents in vehicular networks without affecting the user data privacy. 
Through coordinating existing traffic infrastructure, including roadside units and road traffic clouds, the model parameters are efficiently disseminated by vehicular communications and hierarchically aggregated.
Considering the individual heterogeneity of data distribution, computational and communication capabilities across traffic agents and roadside units, we employ a novel method that addresses the fairness problem at different aggregation layers of the framework architecture, i.e., aggregation in layers of roadside units and cloud. %
%Finally, the experiment results 
The simulation results indicate that the model is enhanced up to xx\% by applying federated learning in our framework. Even when 80\,\% of the agents are timely disconnected, the learning stability compared to the baseline methods is still increased by up to xx\,\% and learning speed is raised up to xx\,\%.
%\Crs{Finally, we show the model enhancement in real traffic scenarios by running deep learning methods in a federated fashion and demonstrate our proposal of federated object detection with traffic image data. }
%The simulation results show that the perception accuracy is improved on average by around \Crs{xx\%} after only \Crs{xx} communication rounds.
\end{abstract}

\begin{IEEEkeywords}
federated learning, deep learning model enhancement, cooperative ITS, road environmental perception, image processing
\end{IEEEkeywords}