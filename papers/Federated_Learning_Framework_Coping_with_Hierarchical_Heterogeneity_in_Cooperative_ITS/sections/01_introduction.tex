\section{Introduction}
\label{sec:introduction}

\begin{figure*}[ht]
\centering
%\includegraphics[trim=0 0 0 0,clip,width=0.99\linewidth]
\includegraphics[width=0.9\linewidth]
{figures/InfoExchange_V5.png}
\vspace{3mm}
\caption{Proposed framework for federated learning in a Cooperative Intelligent Transportation System (C-ITS): Blue, green and purple arrows indicate I2N, V2I and N2V communication, respectively. The gray line represents OEM-internal communication.}
\label{fig:sys}
\end{figure*}

Artificial intelligence, especially deep learning, %approaches, 
is one of the most important technologies in today's \emph{Cooperative Intelligent Transportation System} (C-ITS) development. %
Its application requires a huge amount of data collected on traffic roads. However, the deep learning models, i.e., \emph{Deep Neural Network} (DNN) models, are usually trained by an insufficient amount of data because automotive manufacturers (Original Equipment Manufacturers, OEMs) train their DNN models for autonomous driving only with data from a very limited number of test vehicles. The resulting bias in the data distribution can lead to errors in autonomous driving functions and threaten traffic safety.

In most regions, the acquisition of raw data from public road traffic is challenged due to privacy constraints. Fortunately, federated learning~\cite{pmlr-v54-mcmahan17a} has been proposed as a distributed learning approach that does not affect the privacy of user data. %
It provides C-ITS a great opportunity to gain data from public road traffic and enhance the DNN models, which can be centralized pre-trained on internal or open datasets~\cite{cress2022a9}.  \emph{Vehicle-to-Everything} (V2X) communication~\cite{Festag-11p-to-5G}
%\cite{Festag-Commag} 
enables the data exchange among vehicles, roadside infrastructure, backends and cloud systems, and facilitates federated learning in C-ITS. Based on abundant \emph{Roadside Units} (RSUs), digital twins~\cite{Cress2021_ITSReview} can provide accurate labels for the learning process.

As many other federated learning application scenarios, \mbox{C-ITS} faces heterogeneity problems. % that must be solved. % 
These problems exist due to the lack of full access and control to traffic agents. In C-ITS, heterogeneity derives mainly from \emph{(i)}~various available computational capabilities, %\Caf{What is this?} \Crs{Consider onboard computer unit in some vehicles is doing some other tasks in parallel, then the computation ability for FL is reduced.} 
\emph{(ii)} timely varied communication quality, and \emph{(iii)} individual local training datasets, known as not-independent and identically distributed (Non-IID) data. It can lead to inconsistency in the training process across the traffic agents and cause low convergence speed and instability in federated learning in C-ITS.

To guarantee the success of federated learning and enhance the pre-trained DNN models in C-ITS, we develop a novel federated learning framework as shown in Fig.~\ref{fig:sys}. The framework allows all connected vehicles to collaboratively train a common DNN model via hybrid vehicular communications, i.e., \emph{Vehicle-to-Infrastructure} (V2I) and \emph{Vehicle-to-Network} (V2N). Through sharing both global and roadside federated learning models in our framework, the effect of hierarchical heterogeneity on the learning process can be beneficially reduced, and hence the performance of pre-trained DNN models is powered by data acquired from public traffic.

\noindent
The key contributions of the paper are:

\begin{itemize}
    \item Formulate the problem with Non-IID data across RSUs, individual communication and computation capabilities at the connected traffic agents, and analyze the impact on the federated learning performance in vehicular networks,
    \item Address individual heterogeneity in different layers of the hierarchical federated learning system and develop multiple proximal terms in the learning algorithm,
    \item Propose a general and empirical federated forward training framework for enhancing deep learning tasks in C-ITS by means of V2X networks without affecting the data privacy, 
    \item Evaluate our methods by federated training a DNN model on benchmark datasets considering the Non-IID data properties and heterogeneous communication qualities in C-ITS,
    \item Provide comprehensive experiment results and in-depth analysis of parameters in our framework with heterogeneously connected traffic agents in public road traffic,
    \item Compare our framework with other federated learning approaches and address the trade-off between stability and accuracy in federated learning application scenarios. The software implementation of our framework is publicly available as open source at GitHub.\footnote{\url{https://github.com/rruisong/H2-Fed}}.
\end{itemize}

The remainder of this paper is structured as follows. Section~\ref{sec:related_work} summarizes related work in federated learning and its application in C-ITS. The problem with individual hierarchical heterogeneity in vehicular networks and federated learning is formulated in Section~\ref{sec:problem_formulation}. The federated learning methods are introduced in Section~\ref{sec:method} and the proposed {\myHFed} framework is presented in Section~\ref{sec:system}. The system is comprehensively evaluated in Section~\ref{sec:evaluation} considering heterogeneity in individual layers of the framework architecture with respect to C-ITS scenarios. Section~\ref{sec:conclusion} provides a summary and concludes. 



