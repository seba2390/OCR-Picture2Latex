\section{Related Work}
\label{sec:related_work}
\vspace{-1mm}
Federated learning is a particular distributed learning framework, which was proposed in~\cite{pmlr-v54-mcmahan17a} as \emph{FedAvg} and can be employed for training models without accessing the user's data in massively distributed systems. To achieve joint learning in a federated fashion, especially wireless communication networks for sharing the locally training results across infrastructure are required. Specifically, deployment of C-V2X networks~\cite{9128410, Festag2021,delooz2022analysis} according to corresponding standards~\cite{Festag-Commag, SongFISITA2021} facilities federated learning in C-ITS, where many traffic services and applications need to be developed based on DNN models, e.g., object classifier~\cite{ResNet}, quality monitors~\cite{zhou2021semml,song2022edge}, cooperative detectors~\cite{xu2021opv2v,xu2022v2xvit,chen2022model,zimmer2022real}, etc..
%for traffic object recognition and detection functions. 
In this context, \cite{Posner2021_FL_V2X,FL_VehEdgeCloud,Elbir2020_FL_Veh} indicate the great potential benefits of federated learning in C-ITS.

As with many other federated learning application scenarios, heterogeneity problems exist while implementing federated as well. Research in~\cite{li2020federated, Elbir2020_FL_Veh, 9521822} and others address heterogeneity problems and provide general federated learning algorithms (e.g., \emph{FedProx}) by adjusting the training objective to reduce the influence of straggling devices. However, depending on specific federated learning applications and associated communication infrastructure, the heterogeneity can be caused by various %miscellaneous 
reasons and needs to be optimized individually. 


%Conventional non-hierarchical federated learning are established by two layers of nodes, i.e. clients and server. In vehicular networks, non-hierarchical federated learning, the traffic agents need to share their models with cloud server using up- and downlink and ideally the latency of it can be $140-200$ ms. However, the sidelink communication aspects in managed and unmanaged modes have been studied separately and the mean end-to-end message latencies in LTE-sidelink (resulting from factors such as synchronization, resource allocation and configuration) are reported to be $60-80$ ms as studied in []. According the to ETSI[], 5G-NR is expected to deliver a promising E2E latency of $20-30$ ms. Therefore, we believe the hierarchical learning architecture with additional roadside unit layer for pre-aggregation can speed up the federated learning. 

\begin{table*}[ht]
\begin{threeparttable}
\caption{\centering Overview of heterogeneity metrics while hierarchically applying federated learning system in C-ITS}
\label{table:comparison}
\begin{tabular}{M{3.2cm}M{1cm}M{4.8cm}M{7.1cm}}
   \toprule %%%%%%%%%%%%%%%%%%
    \centering\textbf{Heterogeneity metric} & \centering\textbf{Notation} & \centering\textbf{Main cause}  & \textbf{Effect on Federated Learning}\\
    \midrule %%%%%%%%%%%%%%%%%%
   \vspace{1mm}
    Local Aggregation Round  & LAR & Diverse Quality of Service (QoS) & The pre-aggregation\,\footnote{1} round will be done in each RSU \\
   \vspace{1mm}
    Connection Success Ratio  & CSR & Varied communication quality & The number of agents connected for federated learning\\
   \vspace{1mm}
    Stable Connection Duration & SCD & Varied communication quality & Stable participating time of one agent, once it is connected \\
    %\vspace{1mm}
    Full-task Success Ratio\,\footnote{2} & FSR & Individual available computation resource & Finished training epochs in each agent\\
    %
    \bottomrule %%%%%%%%%%%%%%%%%%
\end{tabular}
\small
\begin{tablenotes}
    \scriptsize 
    \item[1] Aggregation in an RSU before the global aggregation in the cloud.
    \item[2] FSR is not the focus in this paper and has %sometimes 
    a similar effect as CSR.
\end{tablenotes}
\end{threeparttable}
\end{table*}

Moreover, to utilize the hierarchical infrastructure for federated learning in C-ITS, multi-layer aggregation can be used. % in federated learning. 
~\cite{9207469} provides a federated learning framework in hierarchical clustering steps and~\cite{9148862} proposes hierarchical federated learning (\emph{HierFAVG}) in a client-edge-cloud system, which can be regarded as similar to C-ITS considered in this paper. However, heterogeneity is not considered. % there. 
\cite{9054634} focuses on federated learning in cellular networks but do not take the possible direct communication among vehicles and with RSUs via sidelink into account.  %$60-80$ ms in LTE-V2X sidelink~\cite{Hegde2020-VNC} and $20-30$ ms in 5G-NR V2X~\cite{5GNRV2X-tutorial}, resulting from factors such as synchronization, resource allocation, and configuration), which 
Sidelink is regarded as a typical communication method in V2X networks and is considerably faster than up- and downlink~\cite{Hegde2020-VNC,5GNRV2X-tutorial}. %($140-200$\,ms). 
Meanwhile, due to the various V2X messaging services, especially for traffic safety, as evaluated in~\cite{Kuehlmorgen2020}, the priority of data transmission for federated learning in V2X networks can be low compared to safety-related messages, which causes additional time-variant heterogeneity problems across traffic agents.

In this paper, we develop a hierarchical federated learning framework specifically for hybrid V2X networks in C-ITS with consideration of heterogeneity problems. Differentiating from other existing research work, we analyze the heterogeneity caused by V2X communication and available computational capabilities in traffic agents. By employing multiple proximal terms in the federated learning algorithm, the individual heterogeneity in different layers of the framework can be coped for the successes of model training.


