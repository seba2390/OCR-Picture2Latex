% ****** Start of file apssamp.tex ******
%
%   This file is part of the APS files in the REVTeX 4 distribution.
%   Version 4.0 of REVTeX, August 2001
%
%   Copyright (c) 2001 The American Physical Society.
%
%   See the REVTeX 4 README file for restrictions and more information.
%
% TeX'ing this file requires that you have AMS-LaTeX 2.0 installed
% as well as the rest of the prerequisites for REVTeX 4.0
%
% See the REVTeX 4 README file
% It also requires running BibTeX. The commands are as follows:
%
%  1)  latex apssamp.tex
%  2)  bibtex apssamp
%  3)  latex apssamp.tex
%  4)  latex apssamp.tex
%
%\documentclass[twocolumn,showpacs,preprintnumbers,amsmath,amssymb]{revtex4}
%\documentclass[preprint,showpacs,preprintnumbers,amsmath,amssymb]{revtex4}
\documentclass[twocolumn, aps, superscriptaddress, amsfonts,floatfix]{revtex4}% For Physical Review B (Shozoku Betsu)

% Some other (several out of many) possibilities
%\documentclass[preprint,aps]{revtex4}
%\documentclass[preprint,aps,draft]{revtex4}
%\documentclass[prb]{revtex4}% Physical Review B

\usepackage{graphicx}% Include figure files
\usepackage{dcolumn}% Align table columns on decimal point
\usepackage{bm}% bold math
\usepackage{color}
\usepackage{amssymb}
\usepackage{here}
%\nofiles

\newcommand{\correct}[1]{\textcolor{magenta}{#1}}
\newcommand{\Correct}[1]{\textcolor{cyan}{#1}}

\begin{document}

\preprint{APS/123-QED}

\title{Wing structure in the phase diagram of  the Ising Ferromagnet  URhGe close to its tricritical point investigated by angle-resolved  magnetization measurements}% Force line breaks with \\

\author{Shota~Nakamura}
\email{sna@issp.u-tokyo.ac.jp}
\affiliation{Institute for Solid State Physics, The University of Tokyo,  Kashiwa 277-8581, Japan}
\author{Toshiro~Sakakibara}
\affiliation{Institute for Solid State Physics, The University of Tokyo,  Kashiwa 277-8581, Japan}
\author{Yusei~Shimizu}
\affiliation{Institute for Solid State Physics, The University of Tokyo,  Kashiwa 277-8581, Japan}
\affiliation{Institute for Materials Research, Tohoku University, Oarai 311-1313, Japan}
\author{Shunichiro~Kittaka}
\affiliation{Institute for Solid State Physics, The University of Tokyo,  Kashiwa 277-8581, Japan}
\author{Yohei~Kono}
\affiliation{Institute for Solid State Physics, The University of Tokyo,  Kashiwa 277-8581, Japan}
\author{Yoshinori~Haga}
\affiliation{Japan Atomic Energy Agency (JAEA), Tokai 319-1106, Japan}
\author{Ji\v{r}\'{i}~Posp\'{i}\v{s}il}
\affiliation{Japan Atomic Energy Agency (JAEA), Tokai 319-1106, Japan}
\affiliation{Charles University in Prague, Faculty of Mathematics and Physics, DCMP, Ke Karlovu 5, 121 16 Prague 2, Czech Republic}

\author{Etsuji~Yamamoto}%
\affiliation{Japan Atomic Energy Agency (JAEA), Tokai 319-1106, Japan}
%\author{Charlie Author}
% \homepage{http://www.Second.institution.edu/~Charlie.Author}
%\affiliation{
%Second institution and/or address\\
%This line break forced% with \\
%}%

\date{\today}% It is always \today, today,
             %  but any date may be explicitly specified



%%%%%%%%%%%%%%%%%%%%%%%%%%%%%%%%%%%%%%%%%%%%%%%%%%%%%%%%%%%%%%%%%%%%%%%%%%%%%%%%%%%%%%%%%%%%%%%%%%%%%%%%%%%%%%%%%%%%
\begin{abstract}
%%%%
High-precision angle-resolved dc magnetization and magnetic torque studies were performed on a single-crystalline sample of URhGe, an orthorhombic Ising ferromagnet with the $c$ axis being the magnetization easy axis, in order to investigate the phase diagram around the ferromagnetic (FM) reorientation transition in a magnetic field near the $b$ axis.
We have clearly detected first-order transition  in both the magnetization and the magnetic torque at low temperatures, and determined detailed profiles of the wing structure of the three-dimensional $T$-$H_{b}$-$H_{c}$ phase diagram, where $H_{c}$ and $H_{b}$ denotes the field components along the $c$ and the $b$ axes, respectively.
The quantum wing critical points are  located at $\mu_0H_c\sim\pm$1.1~T and $\mu_0H_b\sim$13.5~T.
Two second-order transition lines at the boundaries of the wing planes rapidly tend to approach with each other with increasing temperature up to $\sim 3$~K. Just at the zero conjugate field ($H_c=0$), however, a signature of the first-order transition can still be seen in the field derivative of the magnetization at $\sim 4$~K, indicating that the tricritical point exists in a rather high temperature region above 4~K.
This feature of the wing plane structure is consistent with the theoretical expectation that three second-order transition lines merge tangentially at the triciritical point.

%%%%
%%%%%%%
\end{abstract}
%%%%%%%%%%%%%%%%%%%%%%%%%%%%%%%%%%%%%%%%%%%%%%%%%%%%%%%%%%%%%%%%%%%%%%%%%%%%%%%%%%%%%%%%%%%%%%%%%%%%%%%%%%%%%%%%%%%%



\pacs{Valid PACS appear here}% PACS, the Physics and Astronomy
                             % Classification Scheme.
%\keywords{Suggested keywords}%Use showkeys class option if keyword
                              %display desired
\maketitle

%%%%%%%%%%%%%%%%%%%%%%%%%%%%%%%%%%%%%%%%%%%%%%%%%%%%%%%%%%%%%%%%%%%%%%%%%%%%%%%%%%%%
\section{Introduction}
%%%%%%%%%%%%%%%%%%%%%%%%%%%%%%%%%%%%%%%%%%%%%%%%%%%%%%%%%%%%%%%%%%%%%%%%%%%%%%%%%%%%

%
%%%%%%
%It has been considered that the superconducting excludes ferromagnetism in general.
% In the above three materials, ferromagnetism and superconductivity are thought to coexist at the micro level  \cite{hattori2012superconductivity, ohta2010microscopic, PhysRevLett.102.167003}.
%%%%%%
% Surprisingly, re-entrant superconductivity (RSC) and the huge upper critical field of superconductivity $\mu_{\rm 0}$$H_{\rm c2}$ were observed in URhGe [ref] and UCoGe [ref], respectively.
%%%%%%
% In particular, the ferromagnetic superconductor URhGe has attracted considerable interest because it exhibits the re-entrant superconductivity (RSC) \cite{levy2005magnetic,hardy2011transverse, aoki2014superconductivity,levy2007acute, tokunaga2015reentrant}. 
%%%%%%
%The RSC is observed when a magnetic field between 8 T and 13.5 T is applied along the  $b$ axis (the magnetization hard axis), which is perpendicular to the easy $c$ axis, and, very surprisingly, the superconducting transition temperature is most enhanced at around 12 T.

%%%%%%
%The magnitude of the ordered moment at U site is $\sim 0.4$ $\mu_{\rm B}$ for URhGe, and the moment is oriented along $c$ axis.
%%%%%%
% Superconductivity (SC) emerges below $T_{\rm SC} \sim$ 0.25 K at zero field and the upper critical field $\mu_{\rm 0}$$H_{\rm c2}$  is $\sim 2$ T. RSC appears at higher fields near $\mu_{\rm 0}$$H \sim 12$ T (between $\sim 8$ T and $\sim 13.5$ T) with higher $T_{\rm SC} \sim 0.42$ K at about 12 T \cite{tokunaga2015reentrant, levy2007acute}.
%%%%%%
 %It is interesting that $T_{\rm SC}$ of RSC is higher than that of SC in low field. 


The discoveries of uranium-based ferromagnetic (FM) superconductors, as represented by UGe$_2$ \cite{saxena2000superconductivity}, URhGe \cite{aoki2001coexistence}, and UCoGe \cite{huy2007superconductivity}, have had a great impact, because superconductivity and  ferromagnetism had been thought to compete with each other. 
%%%%%%
 The superconducting properties in the above three uranium-based
 superconductors are extremely unusual, such as the microscopic coexistence of superconductivity and ferromagnetism  \cite{hattori2012superconductivity, ohta2010microscopic, PhysRevLett.102.167003, kotegawaJPSJ.74.705coexist}, 
possible occurrence of an odd-parity pairing \cite{saxena2000superconductivity, aoki2001coexistence, huy2007superconductivity}, 
the huge enhancement of $H_{\rm c2}$ exceeding the Pauli-limiting field in UCoGe \cite{huy2007superconductivity, 200915989}, and  re-entrant superconductivity (RSC) in URhGe \cite{levy2009coexistence}.
%%%%%%
These anomalous behavior are observed around FM quantum phase transition, and hence 
 magnetic quantum fluctuations are considered to be responsible for the emergence of  such unusual superconducting states \cite{tokunaga2015reentrant, hattori2012superconductivity,taufour1742-6596-273-1-012017}.
%%%%%%

%%%%%%
We focus in this paper on the magnetic behavior of URhGe,
 which crystallizes in the orthorhombic TiNiSi structure with the space group $Pnma$ having a zig-zag chain of uranium atoms  along the $a$ axis \cite{TRAN199881}.
%URhGe crystallizes in the orthorhombic structure with the space group $Pnma$ \cite{Troc1988389, chevalier1990ferromagnetic}. 
%%%%%%
%The uranium zig-zag chain with the distance of the next nearest neighbor $d_{\rm U-U} \sim 3.5$ \AA\quad is formed along the $b$ axis. 
%%%%%%
URhGe is known to be an itinerant ferromagnet  in which a magnetic moment  $M$ of $\sim 0.4$ $\mu_{\rm B}$/U aligns along the $c$ axis below $T_{\rm C} \sim 9.5$ K \cite{levy2005magnetic,hardy2011transverse}.
Magnetic anisotropy is very strong in URhGe, with the $c$ axis being the magnetization easy axis.
%%%%%%
Owing to its strong anisotropy,  $T_{\rm C}$  of this compound can be tuned to zero by applying  a magnetic field $H$ along the $b$ axis, perpendicular to the spontaneous moment~\cite{hardy2011transverse}.
The situation is analogous to an Ising ferromagnet in a transverse magnetic field \cite{PFEUTY197079}, for which a quantum phase transition (QPT) accompanying a reorientation of the magnetic moment into a state with $M \parallel H$ can be expected at a finite critical field $H_{\rm R}$. 
%%%%%%
 Previous studies by transport ($T\ge 0.5$~K), magnetic torque ($T\ge 0.1$~K) and magnetization ($T\ge 2$~K) measurements indicate that the transition occurs at $\mu_{0}$$H_{\rm R}\sim 12$ T in URhGe for $H \parallel b$ and becomes first order at low temperature \cite{levy2009coexistence,levy2005magnetic,hardy2011transverse,aoki2014superconductivity}.
%%%%%%
% The NMR measurements \cite{tokunaga2015reentrant} have revealed that longitudinal magnetic fluctuations are significantly enhanced for $H \parallel b$.
%%%%%%
Because the critical field $H_{\rm R}$ is very close to the field in which the RSC emerges, it has been argued that the magnetic fluctuations associated with the moment reorientation play an essential role of RSC~\cite{levy2005magnetic,levy2009coexistence,tokunaga2015reentrant}. 
%%%%%%


%%%%%%
Recently, a FM QPT in clean metals has attracted much interest because a first-order QPT is commonly observed~\cite{RevModPhys.88.025006}.
If $T_{\rm C}$ is decreased by some tuning parameter such as pressure, the nature of the transition changes from second order to first order  at a tricritical point TCP,  and by application of a small field parallel to the spontaneous moment surfaces or ``wings'' of first-order transition emerge~\cite{belitz1999first, PhysRevLett.94.247205}.
%%%%%%
The edges of the wing planes are second-order transition lines, terminating at $T=0$ in quantum wing critical points (QWCPs)~\cite{PhysRevLett.115.020402}.
%%%%%%
This type of ``$T$-$p$-$H$ phase diagram'' with pressure $p$ as a tuning parameter has been studied in itinerant FM compounds, such as UGe$\rm_2$ \cite{taufour2010tricritical, pfleiderer2002pressure, kotegawa2011evolution}, ZrZn$\rm_2$ \cite{uhlarz2004quantum}, URhAl \cite{PhysRevB.91.125115}, UCoGa \cite{mivsek2017pressure}, and an itinerant metamagnet UCoAl \cite{aoki2011ferromagnetic}.
%%%%%%
In these systems, however, either high pressure (UGe$_2$, ZrZn$_2$, URhAl, UCoGa) or negative pressure (UCoAl) is required to tune $T_{\rm C}$ to zero, making it difficult to examine the magnetization behavior near QPT.
In contrast, as a magnetic field $H_b$ parallel to the $b$ axis being the tuning parameter, URhGe provides a good opportunity to investigate the whole FM phase diagram by various means.
%%%%%%
Indeed, when a magnetic field is slightly tilted from the $b$ axis towards $c$ axis, the FM wing structure has been observed below TCP in the  ``$T$-$H_{b}$-$H_{c}$ phase diagram'' in URhGe~\cite{levy2007acute, levy2009coexistence}, where $H_{c}$ denotes the $c$-axis component of the magnetic field, conjugate to the order parameter (OP).
%%%%%%
The location of TCP has been reported to be $\sim$ 2~K and  $>4$~K by thermoelectric power~\cite{gourgout2016collapse} and nuclear magnetic resonance (NMR)~\cite{kotegawa201573ge} experiments, respectively.
%%%%%%
Remarkably, the zero resistivity region of the RSC at 50~mK exactly overlaps the wing QPT region in the $H_{b}$-$H_{c}$ plane~\cite{levy2007acute, levy2009coexistence}, further evidencing the close connection between RSC and the FM QPT.

It should be noticed that the RSC apparently emerges around the  first-order FM transition region. 
A possible explanation of this unusual behavior is that URhGe might be close to a quantum TCP~\cite{levy2007acute}.
There has been, however, a controversy regarding the location of TCP in URhGe~\cite{gourgout2016collapse,kotegawa201573ge}.
Further investigation is thus needed to clarify the FM QPT in URhGe.
Up to present, no direct magnetization measurement has been performed in URhGe in the QPT region; field variation of the magnetization in a $T=0$ limit has  been obtained by a $T^2$ extrapolation of the $M$ vs. $T$ data measured above 2 K for  $H \parallel b$~\cite{hardy2011transverse}. 
In the present paper, we have performed low-temperature angle-resolved dc magnetization measurements on URhGe in order to investigate the magnetization behavior near the wing structure of the FM QPT.
We obtained the $T$-$H_{b}$-$H_{c}$ phase diagram and determined the detailed profiles of the wing structure as well as the location of TCP.

%The purpose of the present study is to examine thermodynamic properties on the phase diagram of URhGe with $H$ near the $b$ axis by magnetization measurements and to clarify relationship between re-entrant superconductivity and quantum fluctuations. In this paper, we focused on the TCP and the  changes of magnetization behavior around the TCP or  at lower temperature. Here, we report the results of the DC magnetization and the torque measurements for URhGe. The temperature and the angle dependences of the magnetization and the magnetic torque curves are measured for  $H \parallel b$ and for the $bc$-plane near the $b$-axis, respectively.
%%%%%%
%%%%%%%%%%%%%%%%%%%%%%%%%%%%%%%%%%%%%%%%%%%%%%%%%%%%%%%%%%%%%%%%%%%%%%%%%%%%%%%%%%%%
%%%%%%%%%%%%%%%%%%%%%%%%%%%%%%%%%%%%%%%%%%%%%%%%%%%%%%%%%%%%%%%%%%%%%%%%%%%%%%%%%%%%

%%%%%%%%%%%%%%%%%%%%%%%%%%%%%%%%%%%%%%%%%%
\begin{figure}[t]
\begin{center}
\includegraphics[width=1\linewidth]{Capa_G80_URhGe_0p21K_c}
\caption{ 
(color online). Example of the raw capacitance ($C$) data (solid circles) obtained at 0.21 K in magnetic fields tilted by 1.4$^\circ$  from  the $b$ axis in the $bc$ plane, with the field gradient $G = 0$ and 8  T/m. Taking a difference of these two yields the magnetization curve   (solid squares).
\label{Capa_G80_URhGe_0p2K_c}}
\end{center}
\end{figure}

%%%%%%%%%%%%%%%%%%%%%%%%%%%%%%%%%%%%%%%%%%
%%%%%%%%%%%%%%%%%%%%%%%%%%%%%%%%%%%%%%%%%%%%%%%%%%%%%%%%%%%%%%%%%%%%%%%%%%%%%%%%%%%%
\section{Experimental Procedures}
%%%%%%%%%%%%%%%%%%%%%%%%%%%%%%%%%%%%%%%%%%%%%%%%%%%%%%%%%%%%%%%%%%%%%%%%%%%%%%%%%%%%
%%%%%%%%%%%%%%%%%%%%%%%%%%%%%%%%%%%%%%%%%%
\begin{figure}[t]
\begin{center}
\includegraphics[width=1\linewidth]{Cell_APS}
\caption{ 
(color online). Schematic view of the two-axis rotation  device. $\theta$ and $\phi$ rotations are performed by a home-made tilting stage and  a piezo-stepper-driven goniometer, respectively. The angle of the tilting stage is controlled from the top of the insert by a screw rod that is thermally isolated.
\label{Cell_APS}}
\end{center}
\end{figure}

%%%%%%%%%%%%%%%%%%%%%%%%%%%%%%%%%%%%%%%%%%

Single-crystalline URhGe was grown at JAEA, and cut into a rectangular shape with the 4.4 mg mass.
%%%%%% 
The present sample does not show superconductivity. It has been recognized that superconductivity as well as RSC only appears in stoichiometric samples of URhGe with a very small residual resistivity~\cite{doi:10.1143/JPSJ.77.094709}. 
 By contrast, the FM transition  is much more robust and does not change much even in doped systems  URh$_{0.9}$Co$_{0.1}$Ge and URh$_{x}$Ir$_{1-x}$Ge~\cite{tokunaga2015reentrant,PhysRevB.95.155138}. 
It should be noticed, however, that the sample quality might influence the magnetization behavior, in particular at the vicinity of QCPs.

DC magnetization  measurements were performed by means of a capacitively-detected Faraday magnetometer \cite{sakakibara1994faraday}.
%%%%%%
 In this method, we detect a magnetic force ($M_{z}dH_z/dz$) proportional to the magnetization of the sample situated in an inhomogeneous field  as a capacitance change of a capacitive transducer. 
 Here, $z$ denotes the vertical axis, along which the magnetic fields up to 14.5~T were generated by a superconducting solenoid.
The capacitance transducer consists of a fixed plate and a mobile plate that is suspended by thin phosphor-bronz wires and can move  in proportion to an applied force.
We applied the field gradient of $G (=dH_z/dz) =8$~T/m in this experiment.
The sample was mounted on the capacitor transducer with varnish (GE7031) so that its $b$ axis is oriented close to the $z$ direction. 
In this situation, the $b$-axis component of the magnetization, $M_b$, is mainly detected. However, a huge magnetic torque component (${\bm M}\times {\bm H}$) is superposed on the output of the capacitor transducer due to the strong magnetic anisotropy. In order to eliminate the torque contribution, we measure the torque back ground with $G$ switched off ($G=0$), and subtract it from the data with $G$ switched on. 

Figure \ref{Capa_G80_URhGe_0p2K_c} shows an example of the data processing, in which the raw capacitance ($C$) data of URhGe obtained at 0.21 K in magnetic fields tilted by 1.4$^\circ$ from the $b$ axis in the $bc$ plane, are shown (solid circles) with two different field gradient values $G = 8$ T/m and $G = 0$.  The magnetization curve (solid squares) is obtained by taking a difference of the two data.
Further details of the data processing are given in Ref. \cite{sakakibara1994faraday}.
{Just at $\theta=0^\circ$, however, we had a difficulty in subtracting the torque background, as discussed later. 
The $G=0$ data is also useful to qualitatively estimate the $c$-axis component of the magnetization $M_c$, i.e., the OP of the FM state, under the fields near the $b$ axis.
%%%%%%


%%%%%%

%%%%%%
%%%%%%
 A $^3$He-$^4$He dilution refrigerator was used to cool the sample in the temperature range of 0.25 K $\leq T \leq$ 6 K.
%%%%%%
The orientation of the URhGe crystal was precisely controlled in the $bc$ and $ab$ planes within an accuracy of less than 0.1$^\circ$ using a piezo-stepper-driven goniometer ($\phi$ rotation) combined with a home-made tilting stage ($\theta$ rotation)~\cite{PhysRevB.90.220502}, where $\phi$ ($-3^\circ \leq \phi \leq 3^\circ$) and $\theta$ ($-7^\circ \leq \theta \leq 7^\circ$) are the rotation angle in the $ab$ and the $bc$ planes, respectively.
%%%%%%
Figure \ref{Cell_APS} shows a schematic view of the two-axis rotation device.
%%%%%%
%%%%%%
The two rotation axes, orthogonal to each other, intersect  the sample position. 
The angle of the tilting stage is varied by a screw, which is rotated from the top of the insert with a shaft that is thermally isolated.
The full details of the two-axis rotation device will be published elsewhere.
%%%%%%
 In the present study, we measured the $\theta$ dependence of the magnetic responses in the $bc$ plane.
% (see Fig.~\ref{3D_v2_c}(c)). 
%%%%%%
%We also checked that the temperature of $T_{\rm C} \sim 9.5$ K is the same as the present study \cite{aoki2011properties} by using a commercial magnetic properties measurement system (MPMS).
%Figure \ref{fig_adjust} shows angle dependence of magnetic torque curves in the $bc$-plane, from which we adjust the field direction along $b$ axis. In this figure, $\theta$. = 0$^\circ$ means that the field direction is along $b$-axis.

%Usually, we can obtain the magnetization curve from the difference of the black and red curve. However, in this present study, it was impossible to obtain magnetization curve simply by taking the difference between these curves, owing to very large anisotropy of the sample.

%%%%%%%%%%%%%%%%%%%%%%%%%%%%%%%%%%%%%%%%%%%%%%%%%%%%%%%%%%%%%%%%%%%%%%%%%%%%%%%%%%%%
\section{Results}
%%%%%%%%%%%%%%%%%%%%%%%%%%%%%%%%%%%%%%%%%%%%%%%%%%%%%%%%%%%%%%%%%%%%%%%%%%%%%%%%
%%%%%%%%%%%%%%%%%%%%%%%%%%%%%%%%%%%%%%%%%%%%%%%%%%%%%%%%%%%%%%%%%%%%%%%%%%%%%%%%%%%%
\subsection{Torque component}
%%%%%%%%%%%%%%%%%%%%%%%%%%%%%%%%%%%%%%%%%%%%%%%%%%%%%%%%%%%%%%%%%%%%%%%%%%%%%%%%
%%%%%%%%%%%%%%%%%%%%%%%%%%%%%%%%%%%%%%%%%%%%%%%%%%%%%%%%%%%%%%%%%%%%%%%%%%%%%%%%%%%%
\begin{figure}[t]
\begin{flushright}
%\includegraphics[scale=0.30, , angle=90]{deg_Torque-H_c}
\includegraphics[width=0.95\linewidth]{deg_Torque-H_c}
\end{flushright}
\caption{(color online).  The raw capacitance data $C$ near $H_{\rm R}(\theta)$ with zero field gradient, measured at 0.5~K for several $\theta$ values, where $\theta$ is the angle between $H$ and the $b$ axis in the $bc$ plane. The black dotted line is zero torque state. These data were collected in a different run from the one in Fig.~\ref{Capa_G80_URhGe_0p2K_c}.
\label{deg_Torque-H_c}}
\end{figure}
%%%%%%%%%%%%%%%%%%%%%%%%%%%%%%%%%%%%%%%%%%%%%%%%%%%%%%%%%%%%%%%%%%%%%%%%%%%%%%%%%%%%
%%%%%%%%%%%%%%%%%%%%%%%%%%%%%%%%%%%%%%%%%%%%%%%%%%%%%%%%%%%%%%%%%%%%%%%%%%%%%%%%%%%%
\begin{figure}[t]

\begin{flushright}
%\includegraphics[scale=0.30, , angle=90]{Capa-deg_0p6K_c}
\includegraphics[width=0.95\linewidth]{Capa-deg_0p6K_c}
\end{flushright}
\caption{(color online).  The angular $\theta$ dependence of the capacitance $C$ with two different field gradient values, $G = 0$ and 8  T/m, obtained at 0.5 K in 10.5 T below $H_{\rm R}$. In the yellow hatched region, the torque component changes so dramatically with $\theta$ that the precise evaluation of the magnetization becomes difficult. The inset schematically shows the $\theta$-evolution of the FM domains with positive and negative components of $M_c$ (solid arrows) in a magnetic field $H$ (open arrows). There is no bulk magnetization $M_c$ at $\theta=0^\circ$, and the zero-torque ($\tau=0$) state persists irrespective of the magnitude of $H_b$.
\label{Capa-deg_0p6K_c}}
\end{figure}
%%%%%%%%%%%%%%%%%%%%%%%%%%%%%%%%%%%%%%%%%%%%%%%%%%%%%%%%%%%%%%%%%%%%%%%%%%%%%%%%%%%%

%%%%%%%%%%%%%%%%%%%%%%%%%%%%%%%%%%%%%%%%%%%%%%%%%%%%%%%%%%%%%%%%%%%%%%%%%%%%%%%%%%%%
\begin{figure*}[t]
\begin{center}
%\includegraphics[scale=0.30, , angle=90]{Torque-theta-T_c}
\includegraphics[width=0.95\linewidth]{Torque-theta-T_c}
\caption{(color online). Magnetic torque divided by field, $\Delta C_{\rm\tau}/H$, of URhGe measured at (a) 0.25, (b) 3, and  (c) 6 K in fields near $H_{\rm R}(\theta)$ with $\theta = 0$$^\circ$, 0.79$^\circ$, 1.65$^\circ$, 3.64$^\circ$, and 5.64$^\circ$,  together with the differential curves $d(\Delta$$C_{\rm \tau})$$/dH$ ((d)-(f)).
\label{Torque-theta-T_c}}
\end{center}
\end{figure*}
%%%%%%%%%%%%%%%%%%%%%%%%%%%%%%%%%%%%%%%%%%%%%%%%%%%%%%%%%%%%%%%%%%%%%%%%%%%%%%%%%%%%
%%%%%%%%%%%%%%%%%%%%%%%%%%%%%%%%%%%%%%%%%%%%%%%%%%%%%%%%%%%%%%%%%%%%%%%%%%%%%%%%%%%%
\begin{figure}[t]
\begin{flushright}
%\includegraphics[scale=0.30, , angle=90]{theta_zero_MH}
\includegraphics[width=0.9\linewidth]{theta_zero_MH}
\end{flushright}
\caption{(color online).  (a) $\tilde{M}(H)$  of URhGe measured in the magnetic field along the $b$ axis.
 (b) The field derivative $d\tilde{M}/dH$ of the magnetization curves in Fig. \ref{theta_zero_MH}(a). In these figures, only the down-sweep traces are plotted for simplicity. The inset of Fig.~\ref{theta_zero_MH}(b) shows  $d\tilde{M}/dH$ at $T=0.25$~K and $\theta=0^\circ$ for both up- and down-field sweeps.
\label{theta_zero_MH}}
\end{figure}
%%%%%%%%%%%%%%%%%%%%%%%%%%%%%%%%%%%%%%%%%%%%%%%%%%%%%%%%%%%%%%%%%%%%%%%%%%%%%%%%%%%%
%%%%%%%%%%%%%%%%%%%%%%%%%%%%%%%%%%%%%%%%%%%%%%%%%%%%%%%%%%%%%%%%%%%%%%%%%%%%%%%%%%%%
\begin{figure}[t]
\begin{flushright}
%\includegraphics[scale=0.30, , angle=90]{theta_zero_TCP_HR}
\includegraphics[width=0.9\linewidth]{theta_zero_TCP_HR}
\end{flushright}
\caption{(color online).  (a)  Temperature dependence of  $H_{\rm R}(\theta)$ obtained from the present measurements at $\theta = 0$$^\circ$, 1.65$^\circ$, and 3.64$^\circ$,
 together with the results of the previous study (dashed line)~\cite{hardy2011transverse}. The inset is an expanded plot for $\theta=0^\circ$, indicating $H_{\rm R}(T)$ defined at up-sweep (open squares) and down-sweep (open circles) fields.
 (b) Temperature evolution of the peak amplitude of $d\tilde{M}/dH$ at $H_{\rm R}$, obtained at several $\theta$. (c) Temperature evolution of the transition width, obtained at several $\theta$.
\label{theta_zero_TCP_HR}}
\end{figure}
%%%%%%%%%%%%%%%%%%%%%%%%%%%%%%%%%%%%%%%%%%%%%%%%%%%%%%%%%%%%%%%%%%%%%%%%%%%%%%%%%%%%
%%%%%%%%%%%%%%%%%%%%%%%%%%%%%%%%%%%%%%%%%%%%%%%%%%%%%%%%%%%%%%%%%%%%%%%%%%%%%%%%%%%%
\begin{figure*}[t]
\begin{center}
%\includegraphics[scale=0.30, , angle=90]{MH_deg_raw}
\includegraphics[width=0.95\linewidth]{MH_deg_raw}
\caption{(color online). The  magnetization curves $M(H)$ of URhGe near $H_{\rm R}(\theta)$ ($\theta = 0.79$$^\circ$, 1.65$^\circ$, 3.64$^\circ$, and 5.64$^\circ$), measured at (a) 0.25, (b) 3, and  (c) 6 K, together with their differential curves $dM/dH$ for (d) 0.25, (e) 3, and  (f) 6 K.
 For comparison, $\tilde{M}(H)$ and $d\tilde{M}/dH$ at $\theta = 0$$^\circ$ are also plotted. The insets in Figs. \ref{MH_deg_raw}(a) and \ref{MH_deg_raw} (b) show the angular variation of the magnetization jump $\Delta M$ ($\theta\ge 0.79^\circ$, solid squares) measured at 0.25 and 3 K, respectively. The solid circles are the linear extrapolation of  $\Delta M$ to $\theta=0$$^\circ$, which are a factor of 0.7 smaller than $\Delta \tilde{M}$ measured at $\theta=0$$^\circ$.
\label{MH_deg_raw}}
\end{center}
\end{figure*}
%%%%%%%%%%%%%%%%%%%%%%%%%%%%%%%%%%%%%%%%%%%%%%%%%%%%%%%%%%%%%%%%%%%%%%%%%%%%%%%%%%%%%%%%
%%%%
%The gradient magnetic field makes the field angle slightly tilted by 0.05 degree from the b axis; the tilt of 0.05 degree is corresponding to transverse field ~ 70 Oe, which easily destroys the zero torque state.
%%%%

%\color{blue}
Figure \ref{deg_Torque-H_c} shows the raw capacitance ($C$) data near $H_{\rm R}$ with zero field gradient ($G = 0$) measured at $T=0.5$~K for several $\theta$ values, where $\theta$ is the angle between $H$ and the $b$ axis in the $bc$ plane. 
The dashed line at $C_0=1.683$~pF indicates the capacitance value at $H=0$. The capacitance difference $\Delta C_{\tau}$ $= C - C_{\rm 0}$ is thus proportional to the torque component $\tau = {\bm M} \times {\bm H}$.
%%%%
There is a huge torque contribution below the reorientation field $\mu_0H_{\rm R}(\theta)\sim$ 12-13 T for $|\theta|\ge0.14^\circ$ coming from the large $M_c$ component.
%%%%
Interestingly, the $G = 0$ data at $\theta = 0^\circ$, in which the magnetic field direction is precisely adjusted to the $b$ axis, show virtually no torque contribution.
%%%%
This is because a perfect alignment of the magnetic field along the $b$ axis ($H_c=0$) yields an equal population of the FM  domains with $M_c$ pointing along $+c$ and $-c$ directions (the inset of Fig. \ref{Capa-deg_0p6K_c}), resulting in the zero-torque state even below $H_{\rm R}$.
%%%%
The degree of the domain alignment changes with $\theta$, and saturates above 0.79$^\circ$. Note that the torque changes its sign for a negative $\theta$ value, as expected.
Above $H_{\rm R}$, the torque component almost vanishes for $\theta=0.14^\circ$, indicating that the  magnetic moment becomes almost parallel to the field direction. 
For  $\theta>0.79^\circ$, on the other hand, a finite torque remains even well above $H_{\rm R}(\theta)$. This is due to an intrinsic magnetic anisotropy of the system.
%%%%
In Fig.~\ref{Capa-deg_0p6K_c},  we plot the $C$ value at $\mu_0H=10.5$~T and $T=0.5$~K with $G=0$ (open circles) and $G=8$~T/m (open triangles) for several $\theta$ values. The $G=0$ data represents the domain alignment as a function of the $c$-axis field $\mu_0H_c~[T]=10.5\sin\theta$. 
The data clearly show that the single domain state is reached at $\theta\sim 0.8^\circ$, or $\mu_0H_c\sim 0.15$~T. 
%%%%


%All of the magnetization curves obtained by this measurement contain the change of the field which caused by the $G$, but the change can be neglected in the magnetization curves at $\theta \neq 0$$^\circ$ because the effect becomes much smaller than the magnetization signal at $\theta > 0.1$$^\circ$.
%%%%





Figure \ref{Torque-theta-T_c} shows the field variation of $\Delta C_{\tau}/H$, the quantity proportional to $M_c$, measured at (a) 0.25, (b) 3, and  (c) 6 K in a field range near $H_{\rm R}$ for the angles $\theta = 0$$^\circ$, 0.79$^\circ$, 1.65$^\circ$, 3.64$^\circ$, and 5.64$^\circ$. 
The differential curves $d(\Delta$$C_{\tau})$$/dH$ are also shown in Figs. \ref{Torque-theta-T_c}(d)-\ref{Torque-theta-T_c}(f).
%%%%%%%%
For $\theta = 0$$^\circ$, the domain state with zero magnetic torque persists up to $\mu_0H_{\rm R}(0^\circ )=11.2$~T, where a small kink appears upon the moment reorientation.
%%%%
 For $\theta =0.79^\circ$ and $T=0.25$~K, the sudden collapse of $\Delta C_{\tau}/H$ seen at $H_{\rm R}(0.79^\circ)=11.7$~T    indicates a first-order transition. This transition becomes broader and shifts to the higher field side with increasing $\theta$ and decreasing $T$.
%%%%%
These features are more clearly seen in the differential data [Figs. \ref{Torque-theta-T_c}(d)-(f)].
%%%%%
%%%%%

These torque data thus directly probe the behavior of the OP across the transition, and can be used to construct the wing structure phase diagram for $\theta > 0.8$$^\circ$.
%%%%
As mentioned above, however, the torque component is not  so sensitive to the phase transition very close to $\theta = 0$$^\circ$ because of the domain formation.
%%%%
In order to explore the phase transition for $\theta \approx 0$$^\circ$, in particular the TCP, we evaluate the field variation of the magnetization  in the following.
%%%%

%\color{blue}
%\color{blue}
%%%%


%Figure 3 shows that the raw capacitance data C of URhGe obtained at 0.25 K with two different field gradient value $G = 0$ and 8 T/m , in the magnetic field along the $b$ axis.
\color{black}


%%%%%%%%%%%%%%%%%%%%%%%%%%%%%%%%%%%%%%%%%%%%%%%%%%%%%%%%%%%%%%%%%%%%%%%%%%%%%%%%%%%%
\subsection{Magnetization}
%%%%%%%%%%%%%%%%%%%%%%%%%%%%%%%%%%%%%%%%%%%%%%%%%%%%%%%%%%%%%%%%%%%%%%%%%%%%%%%%

%%%%
 
Magnetization curves for various field angle $\theta$ near the $b$ axis can be obtained from the capacitance data with $G=8$~T/m by subtracting a torque background ($G=0$ data), on the basis of the assumption that the torque contribution is the same for $G=0$ and 8~T/m. In most cases, this condition holds with  good accuracy. As explained later, however, we found  it  difficult to fulfill this condition at $|\theta|\lesssim 0.1^\circ$ for a technical problem, and as a consequence some residual torque contribution remains in the magnetization curve at $H\lesssim H_{\rm R}$ for $\theta\approx 0^\circ$. For this reason, we denote the magnetization curve at $\theta=0^\circ$ by $\tilde{M}(H)$, and distinguish it from the $M(H)$ data for $\theta>0.1^\circ$ for which the torque component is properly subtracted.
 
%%%%%
Figures \ref{theta_zero_MH}(a) and (b) show $\tilde{M}(H)$ of URhGe and the differential curve $d\tilde{M}/dH$, respectively,  obtained at 0.25, 1, 2, 3, 4.2, and 6 K. 
%%%%%
At 0.25 K, a magnetization jump  is observed at $\mu_{\rm 0}$$H_{\rm R}(0)=11.2$ T.
$\tilde{M}(H)$  reaches $\sim$0.46 $\mu_{\rm B}$/U above $H_{\rm R}$, in agreement with the previous result~\cite{hardy2011transverse}. This magnetization value is very close to the spontaneous magnetization $M_c$ at $H=0$, in accordance with a simple picture of the moment reorientation
% and the magnetization value $m$, which is parallel to the $b$ axis, almost correspond to those for the $c$ axis (0.45 $\mu_{\rm B}$/f.u.) in fields $H > H_{\rm R}$.
%%%%%
 from the easy $c$ axis to the $b$ axis at $H_{\rm R}$~\cite{tokunaga2015reentrant,levy2009coexistence,hardy2011transverse}. 
%%%%%
Just above $H_{\rm R}$, there is a small shoulder-like anomaly, which is also seen in $d\tilde{M}/dH$ as a small hump.
%%%%%
As shown in the inset of Fig.~\ref{theta_zero_MH}(b), a small hysteresis is observed in the transition at 0.25~K, implying the transition to be of first order.
With increasing temperature, the magnetization jump becomes broader and weaker, and the critical field shifts to the lower field side.  %the magnitude of the magnetization jump reduces almost by half.
%%%%%
This change of the transition behavior becomes prominent above 2~K. 
%%%%%
Surprisingly, however, the peak feature in $d\tilde{M}/dH$ can be seen even at 6 K.


 Figure \ref{theta_zero_TCP_HR}(a) shows temperature dependence of $H_{\rm R}(\theta,T)$ obtained from the present torque and magnetization measurements at $\theta = 0$$^\circ$, 1.65$^\circ$, and 3.64$^\circ$, together with the previous results (dotted line) \cite{aoki2011ferromagnetic}.
%%%%
Here $H_{\rm R}(\theta,T)$ is determined by the position of the peak of $d(\Delta$$C_{\tau})$$/dH$ and $d\tilde{M}/dH$ in Figs. \ref{Torque-theta-T_c} and \ref{theta_zero_MH}.
%%%%
The results at $\theta=0$ is qualitatively the same as the previous reports, and the line of $H_{\rm R}(\theta,T)$ shifts to the higher field side with increasing the angle $\theta$.
%%%%
More precisely, the critical field of $\mu_0H_{\rm R}(0, T\!<\!1~{\rm K})=11.2$~T obtained here is slightly lower than the previously reported values~\cite{levy2005magnetic,levy2007acute,aoki2011ferromagnetic,hardy2011transverse}. 
%%%% 
The inset of Fig.~\ref{theta_zero_TCP_HR}(a) compares the critical field defined in the ascending (open squares) and descending (open circles) fields. 
A small but distinct hysteresis appears and grows in amplitude on cooling below 1.5~K.

Figures \ref{theta_zero_TCP_HR}(b) and \ref{theta_zero_TCP_HR}(c) show the temperature evolution of the amplitude of the peak in $d\tilde{M}/dH$  at $H_{\rm R}(\theta)$ and the transition width defined by the full width at the half maximum, respectively, measured at several $\theta$.
%The background is inclination of magnetization curve obtained in a magnetic field, in which the peak in $dm/dH$ becomes sufficiently  small, and it is almost constant $\sim 0.03$ $\mu_{\rm B}$.
%%%%%
A remarkable weakening of the transition is evident above  2 K, the  temperature which is close to $T_{\rm TCP}$ reported previously~\cite{gourgout2016collapse}.
The transition width for $\theta\geq 0.79$$^\circ$ shows a significant broadening above 2~K.
At $\theta=0$, by contrast, the transition remains relatively sharp even near 6~K, suggesting that the first-order-like behavior persists up to this temperature.
%%%%%
%Nevertheless, weak first-order-like behavior can still be seen at 6 K for $\theta=0^\circ$, and the width is still narrow.
%%%%%


%%%%%
Figure \ref{MH_deg_raw} shows the magnetization curves $M(H)$ of URhGe near $H_{\rm R}(\theta)$ for $\theta$ values from 0.79$^\circ$ to 5.64$^\circ$  measured at (a) 0.25, (b) 3, and  (c) 6 K, together with their differential curves $dM/dH$ for (d) 0.25, (e) 3, and  (f) 6 K.
%in applied magnetic field from  9 to 14.5 T at $\theta = 0.79$$^\circ$, 1.65$^\circ$, 3.64$^\circ$, and 5.64$^\circ$. 
%%%%%
Note that the torque component is properly subtracted for the results at $\theta > 0.1^\circ$.
%%%%%
Comparing Figs.~\ref{MH_deg_raw}(d)-\ref{MH_deg_raw}(f) with Figs.~\ref{Torque-theta-T_c}(d-f), one can see that $dM/dH$ shows qualitatively the same behavior with $-d(\Delta$$C_{\tau}/H)$$/dH$; the jump in $M_b$  is correlated to the negative jump in $M_c$. 
In particular, both data yield the same critical field $H_{\rm R}(\theta,T)$.


For comparison, $\tilde{M}(H)$ and $d\tilde{M}/dH$ for $\theta = 0$$^\circ$ are also plotted in these figures.
%%%%%
%%%%%
%%%%%
Whereas $\tilde{M}(H)$ agrees with $M(H)$ for $\theta\ge 0.79^\circ$ at fields above $H_{\rm R}$, 
an apparent disparity is evident below $H_{\rm R}$;
 $\tilde{M}(H)$ appears to be underestimated.
%%%%%
%The difference between $m$ and $M$ is caused by the torque contribution which is explained in Fig. \ref{Capa-deg_0p6K_c}. 
%%%%%
We attribute this problem to an incomplete subtraction of the torque component in the $\theta=0^\circ$ condition. 
A difficulty is that the vertical field gradient $G$ produces a small $c$-axis component of the magnetic field on the sample~\cite{note} and slightly deflects the field angle, accordingly. We estimate the angle shift to be $\sim -0.05^\circ$ at $G=8$~T/m. Even such a tiny change in $\theta$, however, causes a significant effect at $\theta\approx0$$^\circ$ because the torque component for $H<H_{\rm R}$ shows very strong $\theta$ variation there (Fig.~\ref{Capa-deg_0p6K_c}).
As a consequence, the condition of the torque component being independent of $G$ fails, resulting in an incomplete torque subtraction for $H<H_{\rm R}$ and an overestimate of the magnetization jump. 

In order to get a reliable estimation of the magnetization jump for $\theta=0$$^\circ$, we plot 
the metamagnetic jump $\Delta M$ for the field direction $\theta = 0.79$$^\circ$, 1.65$^\circ$ and 3.64$^\circ$ in the inset of Figs. \ref{MH_deg_raw}(a) and \ref{MH_deg_raw}(b). 
%%%%%
In both plots, $\Delta M$ shows a gradual angular variation, and its linear extrapolation to $\theta=0$$^\circ$ gives 
$\Delta M\sim 0.09$ $\mu_{\rm B}$/f.u. at 0.25 K and $\sim0.07$ $\mu_{\rm B}$/f.u. at 3 K.
One sees that $\Delta \tilde{M}$ for $\theta = 0$$^\circ$ (open circles) is about 1.5 times overestimated for both $T=0.25$  and 3 K. 
In what follows, accordingly, we reduce the peak value of $d\tilde{M}/dH$ by a factor 0.7 in the discussion of the angular variation of the transition. 
We note that thus corrected amplitude of the magnetization jump at 0.25~K for $\theta=0$$^\circ$ is in good agreement with the previous estimate by $T\rightarrow 0$ extrapolation of the $M_b(T)$ data~\cite{hardy2011transverse}.
%%%%%
%The inset in Fig. \ref{MH_deg_raw}(d), (e), and (f) shows the angle $\theta$ dependence of the differential curve of torque $d\rm\tau/dH_{\parallel c}$  $vs$ $H_{\parallel c}$, measured at 0.25, 3, and 6 K, respectively.
%%%%%
%The differential curve of torque $d\rm\tau$$/dH$ reflects the qualitative information of $dM_{\parallel c}/dH$.
%%%%%
%It is important to discuss $M_{\parallel c}$, because an effect of $M_{\parallel c}$ is very large in transverse Ising model.

%%%%%
%The angle dependence of $dM/dH$ and $d\rm\tau/dH_{\parallel c}$ qualitatively almost the same at any temperature.
%%%%%
%We note that the peak of $d\rm\tau$$/dH$ apparently looks like sharp because the horizontal axis is not $\mu_{\rm 0}H$ but $\mu_{\rm 0}H_{\parallel c}$, which is corresponding to $\mu_{\rm 0}Hsin\theta$. (On the other hand, $\mu_{\rm 0}H_{\parallel b}$ is corresponding to $\mu_{\rm 0}Hcos\theta$, and almost the same as $\mu_{\rm 0}H$ near $\theta = 0$$^{\circ}$.)
%%%%%
%With increasing the angle $\theta$, the process of the transition weakening, is different from temperature to temperature, and it gets weak faster above 3 K.
%%%%%
%This suggests that not only metamagnetic transition but also the wing structure are weaken above 3 K.
%%%%%
%At $\theta $ = 5.64$^{\circ}$, the peak amplitude of $dM/dH$ at $\mu_{\rm 0}$$H_{\rm R}$ does not change between 0.25 and 6 K.
%%%%%
%The fact shows that $\theta $ = 5.64$^{\circ}$ may be outside of wing structure.
%%%%%



%%%%Figure 1 shows the magnetization isotherms measured at 0.25, 2, 3, 4.2 and 6 K in magnetic fields precisely oriented along the b axis. 




%  It is interesting that $T_{TCP}$ is very high in comparison with $T_{RSC}$ $\sim$ 0.5 K. Below 2 K (lower temperature than $T_{TCP}$), the width becomes almost constant and the value of the width remains quite. Generally, there is a close relationship between SC and quantum fluctuations, and the fluctuations are large near critical points i.e. TCP or QCP.  It is naturally presumed that the fact that $T_{RSC}$ ($H$ $\sim$ 13 T) is higher than $T_{SC}$ ($H$ $\sim$ 0 T)  $\sim$ 0.25 K in URhGe  caused by quantum fluctuations which remain at lowest temperature in this measurements.

%Generally, second order metamagnetic transition is caused by the spins reorientation in classical transverse Ising model \cite{nagai1987prb}. But, in this case, the transition by the spins reorientation becomes first order like below $T_{TCP}$. When $T_{TCP}$ wing phase diagram can see, the transition caused by the spins reorientation also becomes first order like below $T_{TCP}$. The appearance of $T_{TCP}$ and first order transition below $T_{TCP}$ may be due to crystal structure i.e. uranium zig-zag structure.  For example UCoAl and UGe$_{2}$ which have the zig-zag structure show first order transition below  $T_{TCP}$.
\color{black}


%\color{blue}



%Interestingly, the magnetization value at 9 T changes remarkably when the direction of magnetic field deviates from the $b$ axis.
%%%%
%In particular, the magnetization value for the $b$ axis ($\theta = 0$$^{\circ}$)  is considerably smaller, compared to $\theta = 0.79 $$^{\circ}$  to $\theta = 5.64$$^{\circ}$.
%%%%%
%The  value from $\theta = 0.79 $$^{\circ}$  to $\theta = 5.64$$^{\circ}$ can be explained by its magnetization  component parallel to the applied magnetic fields, however, the smallness of the value for the $b$ axis  cannot be simply understood.
%%%%%%
%\color{magenta}
%We have checked that there is no difference between increasing and decreasing processes, 
% suggesting that this small magnetization value may  not  come from   effect of ferromagnetic domains.   
% (is it OK ?)
%%%%%
%(There may exist  some effect of itinerant magentism ???)
%%%%%
%\color{black}

%%%%%%%%%%%%%%%%%%%%%%%%%%%%%%%%%%%%%%%%%%%%%%%%%%%%%%%%%%%%%%%%%%%%%%%%%%%%%%%%%%%%%%%%%%%
%\subsection{Contour plot}
%%%%%%%%%%%%%%%%%%%%%%%%%%%%%%%%%%%%%%%%%%%%%%%%%%%%%%%%%%%%%%%%%%%%%%%%%%%%%%%%%%%%%%%%%%%%
%%%%%%%%%%%%%%%%%%%%%%%%%%%%%%%%%%%%%%%%%%%%%%%%%%%%%%
%\begin{figure*}[h]
%\begin{minipage}{0.95\columnwidth}
%\begin{center}
%\includegraphics[width=75mm]{0p25K_c.eps}
%\end{center}

%\label{fig:11}
%\end{minipage}
%\begin{minipage}{0.95\columnwidth}
%\begin{center}
%\includegraphics[width=75mm]{3K_c.eps}
%\end{center}
%\label{fig:12}
%\end{minipage}
%%%%%%%%%%%%%%%%%%%%%%%%%%%%%%%%%%%%%%%%%%%%%%%%%%%%%%
%\begin{minipage}{0.95\columnwidth}
%\begin{center}
%\includegraphics[width=75mm]{1K_c.eps}
%\end{center}
%%\label{fig:21}
%\end{minipage}
%\begin{minipage}{0.95\columnwidth}
%\begin{center}
%\includegraphics[width=75mm]{4p2K_c.eps}
%\end{center}
%\label{fig:22}
%\end{minipage}
%%%%%%%%%%%%%%%%%%%%%%%%%%%%%%%%%%%%%%%%%%%%%%%%%%%%%%
%\begin{minipage}{0.95\columnwidth}
%\begin{center}
%\includegraphics[width=75mm]{2K_c.eps}
%\end{center}
%\label{fig:31}
%\end{minipage}
%\begin{minipage}{0.95\columnwidth}
%\begin{center}
%\includegraphics[width=75mm]{6K_c.eps}
%\end{center}
%\label{fig:32}
%\end{minipage}
%\caption{The contour plot of the differential of the magnetization, $dM/dH$,  around the ferromagnetic wing structure of  URhGe,
% obtained from the data, measured at 0.25, 1, 2, 3, 4.2, and 6 K.
%These plots are mirrored by $H \parallel b$ axis line.}
%\end{figure*}
%%%%%%%%%%%%%%%%%%%%%%%%%%%%%%%%%%%%%%%%%%%%%%%%%%%%%%
%%%%%%%%%%%%%%%%%%%%%%%%%%%%%%%%%%%%%%%%%%%%%%%%%%%%%%%%%%%%%%%%%%%%%%%%%%%%%%%%%%%%
\section{Discussion}
%%%%%%%%%%%%%%%%%%%%%%%%%%%%%%%%%%%%%%%%%%%%%%%%%%%%%%%%%%%%%%%%%%%%%%%%%%%%%%%%


We employ these data of $dM/dH$ and $d(\Delta$$C_{\tau})$$/dH$ for construction of the URhGe wing structure phase diagram. As $H(\theta)$ passes through the first-order wing plane at a fixed $T$ in the $T-H_b-H_c$ phase diagram (see Fig.~\ref{3D_v2_c}(c)), $dM/dH$ as well as $|d(\Delta$$C_{\tau})$$/dH|$ exhibit a peak. Mapping those peak positions to the $T-H_b-H_c$ space then provides the wing phase diagram. 
 %%%%% 
%We shall see the ferromagnetic wing of URhGe, mapping the  differential of magnetization, $dM/dH$, on the contour plot at typical temperatures.
%%%%
Figure \ref{Wing-H_c_Mag} shows the contour plot of $dM/dH$ around the FM wing structure of  URhGe in the $H_b$-$H_c$ plane  at various temperatures.
Dotted lines indicate the traces of the field sweep at fixed angles $\theta$, along which the magnetization data were obtained. 
%%%%
%%%%
%\color{blue}
Green dots on the contour plots represent the peak position of $dM/dH$ measured at $\theta = 0$$^\circ$, 0.79$^\circ$, 1.65$^\circ$, 3.64$^\circ$, and 5.64$^\circ$, and solid lines are guides to the eye.
%%%%
%The peak amplitude of the $dM/dH$ is plotted pm this contour plot around the FM wings. 
%%%%
One can see that the bright arc in  Fig.~\ref{Wing-H_c_Mag}, i.e., the first-order transition region, becomes narrower as $T$ increases. Above 4.2~K, the bright spot can only be seen at $H_c=0$, indicating the first-order transition is confined to the narrow region. Similar plots can also be obtained from the $d(\Delta$$C_{\tau})$$/dH$ data and the results are shown in Fig.~\ref{Wing-H_c_torque}. 
Note that the data points are absent at $H_c=0$ in Fig.~\ref{3D_v2_c}(b) because the torque component vanishes there due to a ferromagnetic domain formation. 
These plots thus represent cuts of the wing plane at various temperatures (see Fig.~\ref{3D_v2_c}(c)), indicating that the wing planes are slightly warped. 

%%%%%%%%%%%%%%%%%%%%%%%%%%%%%%%%%%%%%%%%%%%%%%%%%%%%%%%%%%%%%%%%%%%%%%%%%%%%%%%%%%%%
\begin{figure}[t]
\begin{center}
\includegraphics[width=0.95\linewidth]{Wing-H_c_Mag}
\caption{ 
(color online). The contour plot of $dM/dH$ near the FM wing structure of  URhGe in the $H_b$-$H_c$ plane for various temperatures 0.25, 1, 2, 3, 4.2, and 6 K. Green dots represent the peak position of $dM/dH$ obtained at $\theta$ = 0$^\circ$, 0.79$^\circ$, 1.65$^\circ$, 3.64$^\circ$, and 5.64$^\circ$, and the green solid lines are guide to the eye. Mirrored copy data are plotted for $\theta<0$$^\circ$.
\label{Wing-H_c_Mag}}
\end{center}
\end{figure}
%%%%%%%%%%%%%%%%%%%%%%%%%%%%%%%%%%%%%%%%%%%%%%%%%%%%%%%%%%%%%%%%%%%%%%%%%%%%%%%%%%%%
%%%%%%%%%%%%%%%%%%%%%%%%%%%%%%%%%%%%%%%%%%%%%%%%%%%%%%%%%%%%%%%%%%%%%%%%%%%%%%%%%%%%
\begin{figure}[t]
\begin{center}
\includegraphics[width=0.95\linewidth]{Wing-H_c_torque}
\caption{ 
(color online). The contour plot of $|d(\Delta$$C_{\tau})$$/dH|$ near the FM wing structure of URhGe in the $H_b$-$H_c$ plane for various temperatures 0.25, 1, 2, 3, 4.2, and 6 K. Green dots represent the peak position of $|d(\Delta$$C_{\tau})$$/dH|$ obtained at $\theta$ = 0$^\circ$, 0.79$^\circ$, 1.65$^\circ$, 3.64$^\circ$, and 5.64$^\circ$, and the green solid lines are guide to the eye. Mirrored copy data are plotted for $\theta<0$$^\circ$.
\label{Wing-H_c_torque}}
\end{center}
\end{figure}
%%%%%%%%%%%%%%%%%%%%%%%%%%%%%%%%%%%%%%%%%%%%%%%%%%%%%%%%%%%%%%%%%%%%%%%%%%%%%%%%%%%%
%%%%%%%%%%%%%%%%%%%%%%%%%%%%%%%%%%%%%%%%%%%%%%%%%%%%%%%%%%%%%%%%%%%%%%%%%%%%%%%%%%%%
\begin{figure}[t]
\begin{center}
%\includegraphics[scale=0.40, , angle=90]{3D_v2_c}
\includegraphics[width=0.95\linewidth]{3D_v2_c}
\caption{ 
(color online). The color contour plot of the peak amplitude of (a) $dM/dH$ and (b) $|d(\Delta$$C_{\tau})$$/dH|$, projected on the $T-H_c$ plane. These plots  give imaging of the wing plane, viewed from the $H_b$ axis.   These plots are constructed from the data obtained at $T = 0.5$, 1.5, 2.5, 3.5, and 5 K (not shown), in addition to those given in Figs. \ref{Wing-H_c_Mag} and \ref{Wing-H_c_torque}.  The white dots in these figures show the data points from which the color mappings are generated.  A schematic $T-H_b-H_c$ phase diagram is given in (c).
%%%%%%
\label{3D_v2_c}}
\end{center}
\end{figure}
%%%%%%%%%%%%%%%%%%%%%%%%%%%%%%%%%%%%%%%%%%%%%%%%%%%%%%%%%%%%%%%%%%%%%%%%%%%%%%%%%%%%


Figure \ref{3D_v2_c} shows the color contour plot of the peak amplitude of (a) $dM/dH$ and (b) $|d(\Delta$$C_{\tau})$$/dH|$, projected on the $T-H_c$ plane. 
These plots  provide imaging of the wing plane viewed from the $H_b$ axis. 
Overall, the wing plane is bell shaped and steeply extends to higher temperatures above $\sim$4~K  at $H_c=0$. This feature of the wing plane is in agreement with the phenomenological analysis that three second-order transition lines meet at TCP tangentially~\cite{PhysRevB.94.060410}.
%%%%%%
%%%%
%These figures clearly shows that the metamagnetic transition continuously becomes broaden with increasing the angle $\theta$ at low temperature.
%%%%%

In a prototypical ferromagnet,  a first-order transition changes into a second-order one at the edge of the wing plane. One therefore expects that $|d(\Delta$$C_{\tau})$$/dH(\theta)|$, the field derivative of the OP, becomes divergent on the line connecting TCP and QWCP \cite{PhysRevB.86.024428}. 
Unlike the expectation, however, the peak amplitude of $|d(\Delta$$C_{\tau})$$/dH(\theta)|$ of URhGe decreases progressively as $\theta$ increases, making it somewhat difficult to define the wing edge from these data. This observation, along with the smallness of the hysteresis in $H_{\rm R}$, demonstrate the weak nature of the first-order transition in this compound.
Nevertheless, from Fig.~\ref{3D_v2_c}(b) we may judge that the wing plane extends to $\mu_0H_c\sim$1.1~T at $T\rightarrow 0$, because outside this range the landscape of $|d(\Delta$$C_{\tau})$$/dH(\theta)|$ becomes suddenly flat and low. Thus the location of  QWCP is estimated to be $\mu_0H_c\sim$1.1~T and $\mu_0H_b\sim$13.5~T.


Similar difficulty exists in the determination of TCP. Since $|d(\Delta$$C_{\tau})$$/dH(\theta)|$ has poor sensitivity to detect TCP at $\theta=0$ ($H_c=0$) because of the ferromagnetic domain issue, we inspect the $dM_b/dH$ data. One should keep in mind that $M_b$ is not the OP of the phase transition under consideration. We therefore need some theoretical inputs to discuss the phase transition by the $dM/dH$ data near $\theta=0$$^\circ$.
Up to now, no established microscopic theory is at hand for the field-induced phase transition in URhGe. We thus rely on the phenomenological model~\cite{PhysRevB.91.014506} that treats the phase transition of an Ising ferromagnet in a magnetic field perpendicular to the spontaneous magnetization. According to the theory, $M_b$ can be expressed in terms of $M_c$ as
\begin{equation}
M_b=\frac{H_b}{2(\alpha+\beta M_c^2)},
\end{equation}
where $\alpha$ and $\beta$ are the coefficients of the $M_b^2$ and the $M_c^2M_b^2$ terms in the Landau free energy expansion, respectively. 
A first-order spin reorientation transition is predicted by this model when $\beta$ exceeds a certain critical value~\cite{PhysRevB.91.014506}.
As described in the next paragraph, we can see from this equation how $dM_b/dH_b$ at the transition evolves with $T$ in the $T-H_b$ plane; $dM_b/dH_b$ diverges at the transition for $T\leq T_{\rm TCP}$, whereas it does not for $T> T_{\rm TCP}$.

At a second-order transition point above TCP, $M_c$ on the $H_c=0$ plane develops as  $M_c\propto \sqrt{T_{\rm C}(H_b)-T}$, where $T_{\rm C}(H_b)$ is given by
\begin{equation}
T_{\rm C}(H_b)=T_{\rm C}(0)-A\beta H_b^2,
\end{equation}
with $A$ being a constant~\cite{PhysRevB.91.014506}. Even though $M_c$ shows an infinite change of slope at  $T_{\rm C}(H_b)$, $dM_b/dH_b$ does not exhibit a strong singularity; 
from Eqs. (1) and (2), $M_b$ would only exhibit a finite change of slope as a function of $T$ or $H_b$. 
This feature of $M_b(T)$ can indeed be seen in the magnetization data measured in various fields $H_b$~\cite{hardy2011transverse}.
%when the second-order transition line is crossed in the $H_c=0$ plane. 
%.
%TCP\UTF{00E4}\UTF{00BB}(J\(B\UTF{00E4}\UTF{00B8}\UTF{008B}\UTF{00E3}\UTF{0081}$B!x(B\UTF{00E3}\UTF{0082}\UTF{0082}1\UTF{00E6}$B"L(B\UTF{00A1}\UTF{00E7}\UTF{009A}\UTF{0084}\UTF{00E6}\UTF{008C}\UTF{00AF}\UTF{00E3}\UTF{0082}\UTF{008B}\UTF{00E8}\UTF{0088}\UTF{009E}\UTF{00E3}\UTF{0081}\UTF{0084}\UTF{00E3}\UTF{0081}\UTF{00BF}\UTF{00E3}\UTF{0081}\UTF{0088}\UTF{00E3}\UTF{0081}\UTF{00AA}\UTF{00E3}\UTF{0081}\UTF{0084} 
By contrast, just at $T=T_{\rm TCP}$, $M_c \propto  (T_{\rm C}(H_b)-T)^{1/4}$ because the $M_c^4$ term in the renormalized free energy vanishes. In this case, $dM_b/dH_b$ would diverge as $(T_{\rm C}(H_b)-T)^{-1/2}$ because a square-root singularity remains in $M_b$.
%A first-order spin reorientation transition is predicted when $\beta$ exceeds a certain critical value.
Below TCP, $dM_b/dH_b$ diverges as well, reflecting a finite jump of $M_c$ at the first-order transition. 


%%%%%
By looking at the $dM/dH$ contour plot in Fig.~\ref{3D_v2_c}(a), we find that
the peak amplitude for $\theta=0$$^\circ$ becomes progressively smaller with increasing $T$ above 2~K.  It can be seen, however, that $dM_b/dH_b$ in Fig.~\ref{MH_deg_raw}(f) still exhibits a rather sharp peak, i.e. divergent behavior, at 6~K. This fact suggests that the first-order nature of the transition persists up to this temperature.
We note that this important feature of the transition is observable only in a very narrow angular window of $|\theta|<0.8^\circ$.
%%%%%%
% Naturally, this phenomenon also appears in the wing structure
%As increasing temperature, the wing structure suddenly starts to be closed around 2 K, and it is almost closed when the temperature reaches to 4.2 K.
%%%%%%



%%%%%%%%%%%%%%%%%%%%%%%%%%%%%%%%%%%%%%%%%%%%%%%%%%%%%%%%%%%%%%%%%%%%%%%%%%%%%%%%%%%%%%%%%%%
%\subsection{schematic view}
%%%%%%%%%%%%%%%%%%%%%%%%%%%%%%%%%%%%%%%%%%%%%%%%%%%%%%%%%%%%%%%%%%%%%%%%%%%%%%%%%%%%%%%%%%%%

Up to now, there have been a few reports regarding the location of TCP in URhGe.
In the $^{73}$Ge NMR spectra study performed in a field of 12~T applied parallel to the $b$ axis, a phase separation of the FM and the paramagnetic states, the fingerprint of the first-order transition,  can be seen at least up to 4.3~K, giving rather strong evidence that $T_{\rm TCP}$ is well above this temperature~\cite{kotegawa201573ge}. 
 %%%%%
By contrast, the thermoelectric power experiment claims much lower TCP temperature of 2~K~\cite{gourgout2016collapse}.
The wing structure phase diagram (Fig.~\ref{3D_v2_c}) obtained in the present experiment is consistent with the NMR results. It should be noticed that a misalignment of the magnetic field by $\sim 1^\circ$ from the $b$ axis would yield an incorrect estimate  of $T_{\rm TCP}\lesssim 3$~K.

Finally, some remarks are made regarding RSC in URhGe. The RSC in this system emerges not only near the quantum wing  critical point, but also along the first-order quantum phase transition line of the wing structure at $T=0$. Indeed, the zero-resistivity state of RSC at 50~mK occurs along the first-order transition line in the $H_b-H_c$ plane, terminating at QWCP~\cite{levy2005magnetic}.
A possible origin of this unusual phenomena has been attributed to longitudinal ($\parallel b$) magnetic fluctuations, and discussed in relation to a quantum TCP that can be expected when $T_{\rm TCP}$ is very low~\cite{levy2005magnetic,levy2009coexistence,tokunaga2015reentrant}.
The present results reveal, however, that there is a large disparity between $T_{\rm TCP}>4$~K and $T_{\rm RSC}\approx 0.42$~K; $T_{\rm TCP}/T_{\rm RSC}\gtrsim 10$, indicating that the system is not close to a quantum TCP. In this regard, we point out that the first-order transition in this system is very weak in nature, as evidenced by a smallness in the hysteresis of the critical field as well as a rapid broadening of the transition with $T$. Such a weakness of the first-order transition might host substantial fluctuations even at low temperatures $T\ll T_{\rm TCP}$.

%%%%%%%%%%%%%%%%%%%%%%%%%%%%%%%%%%%%%%%%%%%%%%%%%%%%%%%%%%%%%%%%%%%%%%%%%%%%%%%%%%%%%%%%%%%
\section{Conclusion}
%%%%%%%%%%%%%%%%%%%%%%%%%%%%%%%%%%%%%%%%%%%%%%%%%%%%%%%%%%%%%%%%%%%%%%%%%%%%%%%%%%%%%%%%%%%%
 
We have investigated the quantum phase transition of an Ising ferromagnet URhGe by means of 
 high-precision angle-resolved dc magnetization measurements in magnetic fields applied near the $b$ axis.
%%%%%
A first-order spin reorientation transition has been observed at low temperatures, accompanied by a small hysteresis in the critical field.
The temperature and angular variations of the transition observed in the magnetization as well as in the magnetic torque allow us to construct the three-dimensional  $T-H_c-H_b$ phase diagram, where $H_c$ ($\parallel c$) is the conjugate field parallel to the order parameter and $H_b$ is the $b$-axis component of the field that tunes $T_{\rm C}$ down to zero.
The tricritical point $T_{\rm TCP}$ is estimated to be located above 4~K in the $H_c=0$ plane.
On cooling below $T_{\rm TCP}$, a wing structure develops by increasing $|H_c|$. We have succeeded in directly determining the detailed profiles of the wing structure. The quantum wing critical points exist at $H_c=\pm 1.1$~T and $H_b=13.5$~T. Three second-order transition lines meet at $T_{\rm TCP}$ tangentially, so that a precise tuning of $H$ along the $b$ axis within 0.8$^\circ$ is needed to correctly determine the position of TCP.
The reentrant superconductivity in this system is not due to a quantum TCP~\cite{levy2007acute}, but is rather related to unusually weak nature of the first-order transition represented by a smallness of the hysteresis and a broadness of the transition. 


\begin{acknowledgments}
The present work was supported in part by a Grant-in-Aid for Scientific Research on Innovative Areas ``J-Physics'' (15H05883)  and KAKENHI (15H03682) from MEXT.

\end{acknowledgments}

 
 

 

%\subsection{\label{app:subsec}A subsection in an appendix}

 


%They turn out to be Eqs.~(\ref{appa}), (\ref{appb}), and (\ref{appc}).
%\newpage %Just because of unusual number of tables stacked at end
\bibliography{apssamp.bib}% Produces the bibliography via BibTeX.

\end{document}
%
% ****** End of file apssamp.tex ******




