\documentclass[aps, prb, twocolumn, amssymb, amsmath, showpacs, superscriptaddress]{revtex4-1}

\usepackage{bm}
\usepackage{times}
\usepackage{graphicx}
\usepackage{color}
\usepackage{dcolumn}
\usepackage[colorlinks=true, letterpaper=true, pdfstartview=FitV, linkcolor=blue, citecolor=blue, urlcolor=blue]{hyperref}
\usepackage{appendix}
\usepackage[normalem]{ulem}

\setlength{\parskip}{1em}


\newcommand{\JPHc}[1]{\textbf{\color{red} [JPH: #1]}}
\newcommand{\JPH}[1]{\textbf{\color{red} #1}}
\newcommand{\JPHrm}[1]{\sout{\textbf{\color{red} #1}}}

\begin{document}
%\title{Anisotropic spin splitting induced gapless superconducting state and mirage gap in altermagnet-superconducting hybrid systems}
\title{Gapless superconducting state and mirage gap in altermagnets}
\author{Miaomiao Wei}
\affiliation{College of Physics and Optoelectronic Engineering, Shenzhen University, Shenzhen 518060, China}
\author{Longjun Xiang}
\affiliation{College of Physics and Optoelectronic Engineering, Shenzhen University, Shenzhen 518060, China}
\author{Fuming Xu}
%\email[]{xufuming@szu.edu.cn}
\affiliation{College of Physics and Optoelectronic Engineering, Shenzhen University, Shenzhen 518060, China}
\author{Lei Zhang}
\email[]{zhanglei@sxu.edu.cn}
\affiliation{State Key Laboratory of Quantum Optics and Quantum Optics Devices, Institute of Laser Spectroscopy,
Shanxi University, Taiyuan 030006, China}
\affiliation{Collaborative Innovation Center of Extreme Optics, Shanxi University, Taiyuan 030006, China}
%\author{Bin Wang}
%\email[]{bwang@szu.edu.cn}
%\affiliation{College of Physics and Optoelectronic Engineering, Shenzhen University, Shenzhen 518060, China}
\author{Gaomin Tang}
\email[]{gmtang@gscaep.ac.cn}
\affiliation{Graduate School of China Academy of Engineering Physics, Beijing 100193, China}
\author{Jian Wang}
\email[]{jianwang@hku.hk}
\affiliation{College of Physics and Optoelectronic Engineering, Shenzhen University, Shenzhen 518060, China}
\affiliation{Department of Physics, University of Hong Kong, Pokfulam Road, Hong Kong, China}

\begin{abstract}
Interplay between Rashba spin orbit interaction (SOI) and superconductivity can give rise to many interesting effects where an in-plane magnetic field is essential. For instance, for a 2D system with strong Rashba SOI proximity coupled to a s-wave superconductor, the in-plane magnetic field can drive the system into a gapless superconducting state while it can also induce a mirage gap at finite energies for an Ising superconductor while keeping the main gap at Fermi level intact. We show that when a s-wave superconductor proximitized to an altermagnet in the absence of SOI and in-plane magnetic field, the gapless superconducting state with mirage gap can emerge showing d-wave signature, due to the anisotropic spin splitting of the altermagnet. When the Rashba SOI is added, the system can turn into a gapped superconductor with mirage gap. Pairing mechanism and transport properties of mirage gap are investigated. Our result suggests that altermagnet is an ideal platform for studying gapless superconducting state and mirage gap.
\end{abstract}

\maketitle

\noindent{\it Introduction} --- The interplay of magnetism and superconductivity is an important research arena in condensed matter physics\cite{RMP1, RMP2, RMP3}. While the magnetism can hamper the conventional superconducting pairing and superconductivity ceases to exist if the magnetic field exceed Pauli limit\cite{Clogston},
%the magnetism can also enable the finite momentum pairing for unconventional superconductivity.
the magnetism can enable the finite momentum and/or triplet pairing for unconventional superconductivity, giving rise to interesting physics. For instance, an in-plane magnetic field can partially destroy the pairing
for a 2D system with strong spin orbit interaction (SOI) proximity coupled with a s-wave superconductor. This in turn leads to a segmented Fermi surface that can be used to create Majorana bound states, reveal information on spin textures of electron Fermi surface in the normal state, and characterize Fulde-Ferrell-Larkin-Ovchinnikov state in unconventional superconductors\cite{L-Fu1, L-Fu3, L-Fu2}. Recently, this gapless superconducting state has been observed experimentally\cite{L-Fu2}.
Note that besides the gapless superconducting states discussed here,
there are two other gapless superconducting states:
the first one features a Bogoliubov Fermi surface,
in which the gapless superconducting states are due to the form factor in excitation spectrum,
like that in the p-wave and d-wave superconductors \cite{Agterberg, Brydon, Sim}
and the second one is also created by applying an external magnetic field (under Pauli limit) to the superconductor,
but in which the gap is fully closed along the whole Fermi surface, as observed in Ref.\onlinecite{J-Lee}.
In addition, for an Ising superconductor\cite{KT-Law1, J-Lu, X-Xi, Y-Saito, QF-Sun1, QF-Sun2} with in-plane magnetic field, the presence of equal-spin triplet pairing at finite energy leads to a mirage gap that coexists with quasi-particle density of states\cite{GM-Tang,GM-Tang1}.
%The mirage gap was also found in $j=3/2$ superconductors with strong SOI\cite{M-Bahari}.
The interplay of finite momentum and finite energy superconducting pairing was investigated\cite{Chakraborty}.

Recently, in addition to ferromagnetic phase and antiferromagnetic phase, a third magnetic phase dubbed altermagnetic phase has been identified\cite{AM0, AM1, AM2, AM3, AM4, AM5, C-Song1, Nitta, C-Song2}. The altermagnet (AM) has a collinear antiferromagnetic structure with a large non-relativistic anisotropic spin splitting (ASS), which leads to a number of interesting physics unique to AM, including giant and tunneling magnetoresistance\cite{AM1}, anomalous spin Hall effect\cite{AM4, AM6, AM7, NC1}, spin splitting torque and T-odd spin Hall effect\cite{AM5, C-Song1, Nitta, C-Song2}, pronounced thermal transport\cite{Y-Yao1}, and the spin Seebeck and spin Nernst effect of magnon in the absence of Berry curvature as a result of the giant spin splitting of magnonic band\cite{Sinova1, T-Yu}. Moreover, there are abundant materials that exhibit AM phase such as RuO$_2$, MnTe, CrO, and CrSb ranging from insulator, semiconductor, semimetal to metallic systems\cite{AM3} making it an ideal platform for material engineering\cite{Q-Liu, Y-Yao, Lovesey, Sattigeri}.

When an AM is sandwiched between two superconducting leads, $0$-$\pi$ oscillation was predicted due to the finite momentum pairing\cite{Brataas1, SB-Zhang, Beenakker}. Andreev reflection from the interface of AM and superconductor was studied to explore its dependence on the orientation of AM relative to the interface, impurity disorder, and tunneling barrier\cite{Papaj, Brataas2}. In addition, it was shown that the first and second order topological superconductivity in 2D AM metals can emerge\cite{D-Zhu,Hughes}.

For a 2D system proximitized to a s-wave superconductor, it normally require in-plane magnetic field and effective SOI to achieve gapless superconducting state and mirage gap. Since the in-plane magnetic field may destroy the proximitized superconducting state before creating the gapless superconducting state, there is a very narrow window that the in-plane magnetic field can maneuver, making it difficult to control and manipulate the gapless state.
Our work shows that the use of in-plane magnetic field is not necessary. By tuning anisotropic spin splitting (ASS) the AM proximitized to a s-wave superconductor (AM-SC) can change from s-wave superconductor to a gapless superconductor with a d-wave like segmented Fermi surface. At the same time, the mirage gap emerges due to the finite energy pairing, which can be identified by the quantized Andreev reflection coefficient. Turning on SOI destroys the gapless superconducting state but enriches the physics of mirage gap. For instance varying strength of SOI can lead to a transition from a d-wave AM-SC state to a s-wave AM-SC state while the mirage gap can become anisotropic with $C_4$ symmetry.

\bigskip

\noindent{\it Hamiltonian} ---
The Hamiltonian of altermagnet (AM) is given by ($\hbar=e =2m=1$)
\begin{eqnarray}
H_0 = {\bf k}^2 + t_J (k_x^2 -k_y^2)\sigma_z + \lambda (k_x \sigma_y - k_y \sigma_x) - \mu\nonumber
\end{eqnarray}
where $\mu$ is the chemical potential and $t_J$ is a coupling constant responsible for the anisotropic spin spitting. Since this Hamiltonian has $C_4$ symmetry, we can also rotate one of the principal axes by an angle $\theta$
\begin{eqnarray}
k_x &=& k'_x \cos\theta + k'_y \sin\theta \nonumber \\
k_y &=& -k'_x \sin\theta + k'_y \cos\theta \nonumber
\end{eqnarray}
to find
\begin{eqnarray}
H_0(\theta) &=& k^2 + t_1 (k_x^2 -k_y^2)\sigma_z + t_2 k_x k_y \sigma_z \nonumber\\
&+& \lambda_1 (k_x \sigma_y - k_y \sigma_x) + \lambda_2 (k_x \sigma_x + k_y \sigma_y)-\mu\label{ham1}
\end{eqnarray}
where $t_1 = t_J \cos2\theta$, $t_2 = t_J \sin\theta \cos\theta$, $\lambda_1 = \lambda \cos\theta$, and $\lambda_2 = \lambda \sin\theta$. In the following calculation, our energy unit is eV.

If AM is proximitized to a s-wave superconductor with a gap function $\Delta$, the Hamiltonian of this AM-SC becomes\cite{SC-Zhang, GM-Tang}
\begin{equation}
H = \left( \begin{matrix}
   {{H}_{0}}( k ) & \Delta i{{\sigma }_{y}}  \\
   -\Delta i{{\sigma }_{y}} & -H_{0}^{*}( -k )  \\
\end{matrix} \right)
\label{ham}
\end{equation}
It is easy to show that the operator ${\sigma}_z$ commutes with $H$\cite{note2}.

\bigskip

\noindent{\it Pairing mechanism} --- Defining the general pairing correlation function\cite{Gorkov, Sigrist, GM-Tang}
\begin{eqnarray}
{\cal F}({\bf k}, \epsilon) = \Delta (F_0 \sigma_0 + {\bf F} \cdot {\boldsymbol{\sigma}})  i\sigma_y
\end{eqnarray}
where $F_0$ and ${\bf F}$ denote the singlet and triplet pairing correlations. For instance, the triplet pairing wave function corresponding to $F_z$ is $|\psi\rangle = F_z (|\uparrow \downarrow\rangle + |\downarrow\uparrow\rangle )$. From the Gorkov equation\cite{Gorkov, Sigrist, book1, GM-Tang},
\begin{equation}
\left[ \begin{matrix}
   \varepsilon -{{H}_{0}}( k ) & -\Delta i{{\sigma }_{y}}  \\
   \Delta i{{\sigma }_{y}} & \varepsilon +H_{0}^{*}( -k )  \\
\end{matrix} \right]\left[ \begin{matrix}
   {\cal F}( k,\varepsilon  )  \\
   \bar{G}( k,\varepsilon )  \\
\end{matrix} \right]=\left[ \begin{matrix}
   0  \\
   1  \\
\end{matrix} \right]
\end{equation}
where $\varepsilon$ is energy, ${\cal F}$ and ${\bar G}$ are anomalous and regular Green's functions, respectively.
${\cal F}$ is determined by the following equation,
\begin{equation}
[ {{\Delta }^{2}}i{{\sigma }_{y}}-i( \varepsilon +H_{0}^{*}( -k ) ){{\sigma }_{y}}( \varepsilon -{{H}_{0}}( k ) )]{\cal F}=\Delta.
\label{F1}
\end{equation}
%The gap function of the AM-SC is obtained by $\Delta({\bf k}) = \int {\cal F}({\bf k}, \epsilon) \tanh(\epsilon)/kT) d\epsilon$.
Assuming $\lambda=0$ and $a \equiv t_1 (k_x^2 -k_y^2) + t_2 k_x k_y $, the Hamiltonian is expressed as $H_0 = k^2\sigma_0 + a\sigma_z$. In this case, ${\cal F}$ is solved from Eq.(\ref{F1}),
\begin{equation}
\begin{split}
{{F_{0}( \textbf{k},\varepsilon  )}}&={\left( {{\varepsilon }^{2}}-{{\Delta }^{2}}-{\bf k}^{4} +a^2 \right)}/{M( \textbf{k},\varepsilon  )}\;,\\
{{F_{z}( \textbf{k},\varepsilon  )}}&={2\varepsilon a}/{M( \textbf{k},\varepsilon  )}\;,
\end{split}
\end{equation}
with
\begin{align}
M( \textbf{k},\varepsilon  )=4{{\varepsilon }^{2}}a^2 -{{( {{\varepsilon }^{2}}-{{\Delta }^{2}}- {{\bf k}^{4}} +a^2 )}^{2}}.
\end{align}
Hence both singlet and triplet pairing are present at finite energy while at $\varepsilon=0$ where only singlet pairing survives. The pairing at finite energy leads to a pseudo-gap which was termed as mirage gap\cite{GM-Tang}. To find the location and width of mirage gap, we diagonalize the Hamiltonian Eq.(\ref{ham}) and obtain four eigenvalues $E_{- \pm} = - a \pm \sqrt{\Delta^2 +k^4}$ and $E_{+\pm} =  a \pm \sqrt{\Delta^2 +k^4}$.

The main gap is determined by $E_{-+} - E_{+-} = -2a + 2\sqrt{\Delta^2 +k^4}$. Hence $\sqrt{\Delta^2 +k^4}=a$ gives the condition for closing of the main gap at particular (${\bf k}, \theta$), giving rise to the segmented Fermi surface (the graphical solution is shown in Fig.\ref{FIG1}b for $\theta=0$).
The evolution of band structure of mirage gap ($E$ versus $k_x$ for $k_y=0$) is shown in Fig.\ref{FIG1}a, from which we see that the main gap at $E_F=0$ is opened for small $t_J$. At a critical value of $t_J$, e.g., $t_J=0.36$ for $k_y=0$, we have $E_{+-}=E_{-+}=0$ and the main gap is closed at $k_y=0$. However, it does not mean that the system becomes a normal state. In Fig.\ref{FIG1}b, we plot the Fermi surface at $E_F=0$ with chemical potential $\mu=0.05$, which clearly shows that it is a gapless superconducting state with segmented Fermi surface\cite{L-Fu1}.
Upon further increasing $t_J$, the mirage gap is formed while the main gap remains closed, suggesting that the existence of main gap and mirage gap are mutual exclusive at fixed $k_y$. When $t_J>2.0$ the system turns into a normal state and the Fermi surface becomes a circle. Hence during the increasing of $t_J$, the system changes from the s-wave superconducting states to a d-wave like gapless superconducting state and finally becomes a normal state.
This
%is similar to the mirage gap induced by in-plane magnetic field for a system with SOI discussed in Ref.\onlinecite{L-Fu1} and
is different from the mirage gap investigated in Ref.\onlinecite{GM-Tang} where both main gap and mirage gap can be present at the same time.

The width of mirage gap is
\begin{equation}
\delta= E_{++} - E_{+-} = 2\sqrt{\Delta^2 +k^4}\label{width}
\end{equation}
which is independent of $t_J$ while the location of the mirage gap (mid point of the gap) is $(E_{++} + E_{+-})/2 = a$ which is linearly proportional to $t_J$. Since the system has $C_4$ symmetry we expect that the mirage gap enjoys the same symmetry as well.
If we turn on the SOI, spin is not a good quantum number anymore and additional features emerge. It is easy to show that all four components of the pairing correlation function are nonzero. Moreover, mirage gap and main gap can open up at the same time similar to the case of Ising superconductor with in-plane magnetic field\cite{GM-Tang}.
As will be seen below that these conclusions agree with the quantum transport calculation. To reveal the nature of mirage gap, in the following we perform quantum transport calculation for AM system with one normal lead and one AM-superconducting lead.

\begin{figure}[ht!]
\centering
\includegraphics[width=0.90\columnwidth]{FIG1}
\caption{
\label{FIG1}
(a). Energy for different $t_J$. Expressions of $E_{\pm\pm}$ are given in text main text. Here we only show the energy band in the region $k_x=(-\pi,0)$ while fixing $k_y=0$. Red and blue curves denote spin up and down, respectively. Note that the band structure is symmetric when $k_x$ changes to $-k_x$. (b). Segmented Fermi surface for $t_J=0.45$ showing d-wave signature. (c). The Andreev reflection coefficient $T^A = T^A_\uparrow + T^A_\downarrow$ versus $t_J$ at $E_F=0$.  (d). The spin resolved Andreev reflection $T^A_\sigma$ and quasi-particle transmission coefficient $T^Q_\sigma$ versus Fermi energy at a fixed $t_J=0.4$ where the mirage gap emerges. In (c) and (d), we set $\lambda=0$ and $\theta=0$.
}
\end{figure}

\bigskip

\noindent{\it Quantum transport formalism} ---
The schematic plot of the system we considered is shown in Fig.\ref{FIG4}a where an AM-nanoribbon is in contact with an AM-superconducting lead.

In the presence of mirage gap, the transmission coefficient in general consists of the Andreev reflection coefficient $T^A$ and quasi-particle transmission coefficient $T^Q$ which can be calculated using the nonequilibrium Green's function. In the Nambu representation ($e\uparrow ,e\downarrow ,h\uparrow ,h\downarrow $), the Andreev reflection and quasi-particle transmission coefficients are defined as (assuming $E_F \ge 0$)
\begin{eqnarray}
{{T}^{A}} &=&{\rm Tr}[ {{\Gamma }_{L\text{e}}}{{G}^{r}}{{\Gamma }_{L\text{h}}}{{G}^{a}}],\nonumber\\
{{T}^{Q}} &=&{\rm Tr}[ {{\Gamma }_{L\text{e}}}{{G}^{r}}{{\Gamma }_{R\text{e}}}{{G}^{a}}],\nonumber
\end{eqnarray}
where e and h denote the electron and hole. In addition,
$T_{\sigma}^{A}={\rm Tr}{[ {{\Gamma }_{L\text{e}}}{{G}^{r}}{{\Gamma }_{L\text{h}}}{{G}^{a}} ]}_{\sigma\sigma}$ and $T_{\sigma }^{Q}={\rm Tr}{{[{{\Gamma }_{L\text{e}}}{{G}^{r}}{{\Gamma }_{R\text{e}}}{{G}^{a}}]}_{\sigma\sigma}}$ are the Andreev reflection and quasi-particle transmission coefficients with spin $\sigma=\uparrow,\downarrow$\cite{note1}.
The linewidth function is defined as ${{\Gamma }_{L/R}}=i\left[ \Sigma _{L/R}^{r}-\Sigma _{L/R}^{a} \right]$ where
$\Sigma _{L/R}^{r}$ is the retarded self-energy describing the coupling between the left/right lead and the central scattering region. Here
${{G}^{r}}=[ E_F-{H}-\Sigma _{L}^{r}-\Sigma _{R}^{r} ]$ is the retarded Green's function, where $E_F$ is the Fermi energy and $H$ is the Hamiltonian of the central scattering region.
The advanced Green's function is given by ${{G}^{a}}={{\left[ {{G}^{r}} \right]}^{\dagger }}$.
%With these definitions, the charge conductance is defined as $T^\alpha_{\uparrow} + T^\alpha_{\downarrow}$ while the spin conductance is $(T^\alpha_{\uparrow} - T^\alpha_{\downarrow})/2$ where $\alpha= A, Q$ stand for Andreev reflection and quasi-particle transmission, respectively.
In the numerical calculation, we discretize the Hamiltonian in a $20\times 20$ mesh and set $\mu=0.05$ and $\Delta=0.001$.

\noindent{\it Numerical results} --- We first discuss the case of $\lambda=0$. Note that in Eq.(\ref{ham1}) the principal axis makes an angle $\theta$ with normal of the normal metal-superconductor interface. We first give an example of the Andreev reflection at $\theta=0$ and establish the fact that the spin resolved Andreev reflection coefficient $T^A_\sigma$ is an integer within the gap (note that the main gap and mirage gap are mutual exclusive at a particular angle). In Fig.\ref{FIG1}c, we plot the Andreev reflection coefficient versus $t_J$. Typical values of $t_J$ with the corresponding band structures is shown in Fig.1a. As long as the main gap is not closed, i.e., $t_J < 0.36$, we find $T^A_\sigma=1$ within the gap while away from the gap $T^A_\sigma$ decays to zero.

Fig.\ref{FIG1}d depicts $T^A_\sigma$ and $T^Q_\sigma$ versus $E_F$ at $t_J=0.4$ for the mirage gap. It shows that, by increasing $t_J$, $T^A$ at the main gap splits into two spin resolved $T^A_\sigma$ below and above $E_F=0$ with $T^A_\sigma=1$ within the mirage gap. Therefore the energy dependence of $T^A_\sigma$ for the main gap and the mirage gap have the same behavior.
However, if we plot the total Andreev reflection coefficient, $T^A$ is not a constant value with the mirage gap since $T^A_\sigma$ is nonzero outside of the mirage gap. Similar behavior is found at $E_F=0$. When the main gap is closed, $T^A$ at $E_F=0$ is not equal to $2$ and its nonzero value is contributed from Andreev reflection of the mirage gap at $E_F \ne 0$.
From Fig.\ref{FIG1}d, we also see that both charge and spin Andreev reflections are nonzero, confirming the existence of singlet and triplet ($F_z$) pairing because if there was only singlet pairing the spin Andreev reflection would not be allowed.
Note that in the presence of mirage gap, quasi-particle transmission is allowed as seen from Fig.\ref{FIG1}d since there is no global gap. It is easily confirmed that $\sum_\sigma (T^A_\sigma + T^Q_\sigma)=2$ from Fig.\ref{FIG1}d. Therefore perfect quasi-particle transmission occurs when Andreev reflection coefficient vanishes.
\begin{figure}[ht!]
\centering
\includegraphics[width=0.90\columnwidth]{FIG2}
\caption{
\label{FIG2}
(a) and (b): angular dependence of $T^A$ at different $t_J=0.3, 0.37, 0.5, 1$ for $E_F=0, 2\Delta$, respectively.
(c) and (d): Andreev reflection $T^A$ versus $E_F$ for different $\theta$ and $t_J$, respectively. We set $t_J=1.0$ in Fig.\ref{FIG2}c, $\theta=0$ in Fig.\ref{FIG2}d and $\lambda=0$ in Fig.\ref{FIG2}.
}
\end{figure}

Now we study the angular dependence of Andreev reflection coefficients which are plotted in Fig.\ref{FIG2}a,b for different $t_J$ at $E_F=0$ and $E_F=2\Delta$, respectively. We see that the Andreev reflection from the main gap is isotropic (s-wave superconducting AM) for small $t_J$ until $t_J > 0.36$ where $T^A$ versus $\theta$ becomes anisotropic with d-wave signature, indicating formation of the mirage gap. Therefore a critical value for $t_J$ exists, separating s-wave and d-wave behaviors of superconducting AM.
Even for $t_J=1.0$, the main gap still exists for certain range of angles. We note that the both the main gap and mirage gap show $C_4$ symmetry with principal axis at $\theta=\pi/4$ and $\pi/2$, respectively, which confirms that the main gap and mirage gap are mutually exclusive at particular angle.

In Fig.\ref{FIG2}c, we display $T^A$ versus $E_F$ for different $\theta$ while fixing $t_J=1.0$. At $\theta=0$, there are two mirage gaps with spin resolved $T^A=1$ within individual gap indicating that the mirage gap is spin resolved. As we increase the angle $\theta$ to $\pi/16$ (not shown in the figure) or $\pi/8$, the mirage gaps moves towards $E_F=0$ while an additional pair of mirage gaps appear with a much narrow width. At $\theta = 3\pi/16$, there is only one pair of mirage gap left and at $\theta=\pi/4$ the mirage gap disappears and the main gap opens up with $T^A=2$ within the gap.
In Fig.\ref{FIG2}d, we depict Andreev reflection coefficient versus $E_F$ for different $t_J$ while fixing $\theta=0$. At $t_J=0$, we have the main gap with $T^A=2$ and $t_J=0.4$ the Andreev reflection coefficient is obtained by adding two spin resolved $T^A_\sigma$ in Fig.\ref{FIG1}d. At $t_J=1.0$ we find that the mirage gap is moving away from $E_F=0$ with the center of the gap shifting linearly in $t_J$ while the width of the mirage gap is independent of $t_J$ which agrees with the analytic analysis in Eq.(\ref{width}).

\begin{figure}[ht!]
\centering
\includegraphics[width=0.90\columnwidth]{FIG3}
\caption{
\label{FIG3}
(a) Energy evolution as one changes $t_J$ for $\lambda=0.07$ and $k_x=(-\pi,0)$ for $k_y=0$ at $\theta=0$. (b) Energy for $t_J=1.0$, $\lambda=0.07$, and $\theta = 40^o, 60^o, 80^o$.
(c) Angular dependence of $T^A$ at different $\lambda=0.0, 0.04, 0.07, 0.1$ for $E_F=0$.
(d) Andreev reflection $T^A$ versus $E_F$ for $t_J$, where $\theta=0$ and $\lambda=0.07$.
}
\end{figure}

Next we investigate the effect of SOI on the main gap, mirage gap, and Andreev reflection. In the presence of SOI, the spin is not a good quantum number and we use total Andreev reflection coefficient instead of spin resolved one. Once again, we show numerical results for $\theta=0$ unless specified otherwise. In Fig.\ref{FIG3}a, we show the evolution of band structure for different $t_J$ at $\lambda=0.07$. Several observations are in order. (1). The presence of SOI will shift the band horizontally and therefore there are two main gaps at different momenta.
The mirage gap opens up for small $t_J$ and can coexist with the main gap in the presence of SOI. We see that the mirage gap and main gaps are located at different momenta as well. At this stage, the system shows s-wave superconducting character.
(2). As we increase $t_J$ the width of main gap and the position of the mirage gap remain almost the same while the width of mirage gap increases slowly. When $t_J=0.6$, the second pair of mirage gap occurs at a larger energy $|E_F|$ with a much narrow width (we only show one of them here). At this point, the number of transmission channel is two. We also notice that along with the occurrence of the mirage gap there is also a huge insulating local gap $\Delta_{\rm ins}$ marked in Fig.\ref{FIG3}a whose width decreases with increasing of $t_J$.
(3). When $t_J$ is increase further, the second pair of mirage gap moves towards $E_F=0$ and the insulating gap is closed at a critical value of $t_J$. When $t_J$ is larger than the critical value, the gap is reopened forming a pair of superconducting main gap adjacent to the original main gap along $k_x$-axis. At the same time, the maximum number of transmission channel can be three. Note that the number of transmission channel depends on $\mu$, $t_J$, $\lambda$, and $E_F$.
Fig.\ref{FIG3}b depicts the energy band at $t_J=1.0$ and $\lambda=0.07$ for other angles, showing that this new gap is highly anisotropic with $C_4$ symmetry. The mirage gap also exhibits anisotropy similar to Fig.\ref{FIG2}b. As will be discussed below that the angular dependence of this new gap is the same as $T^A(\theta)$ shown in Fig.\ref{FIG3}b. In this sense, the system displays d-wave character for the main gap at $E_F=0$.

In Fig.\ref{FIG3}c, we show the angular dependence of Andreev reflection coefficient $T^A(\theta)$ for different $\lambda$ and fixing $t_J=1.0$. It shows that for both zero and small SOI, $T^A(\theta)$ displays a d-wave like character. However, as soon as the SOI is turned on the symmetry axis of the d-wave is rotated by $\pi/4$. For larger SOI, for example, $\lambda=0.1$, $T^A(\theta)$ changes from d-wave like to s-wave like. At $t_J=1.0$, the maximum number of transmission channel reaches three as long as $T^A(\theta)$ is d-wave like.
Fig.\ref{FIG3}d plots the Andreev reflection versus $E_F$ for different $t_J$ and fixed $\lambda=0.07$. Due to the existence of two different widths of main gap, the Andreev reflection close to $E_F=0$ is three and when $E_F$ is outside of narrow main gap but within the wide main gap, we have $T^A=2$ and $T^Q=1$. Similar situation occurs for the mirage gap since there are also two pairs of mirage gap that overlap with each other near $E_F=0.012$ (see also Fig.\ref{FIG3}a). While near $E_F=0.018$ there is only one pair of mirage gap and therefore $T^A=1$ within the gap. We also see that the width of the wide main gap and position of the first mirage gap remain the same for different $t_J$ while the width of the first mirage gap increases with $t_J$ in agreement with the observation made in Fig.\ref{FIG3}a.
Interestingly, although $T^A$ remains symmetric when $E_F$ changes sign, for $\lambda \ne 0$ the quasi-particle transmission coefficient is not a symmetric function any more because the number of transmission channel across $E_F = (-0.02, 0.02)$ can vary from two to three.

Now we show that the gapless superconducting state studied in Ref.\onlinecite{L-Fu1, L-Fu3}, the mirage gap occurs as well. The Hamiltonian is the 2D surface of a topological insulator with an in-plane magnetic field or Zeeman energy ${\bf V}$ defined as
\begin{eqnarray}
H_0 = v_F (k_x \sigma_y - k_y \sigma_x) - \mu +{\bf V} \cdot \boldsymbol{\sigma} \label{Fu}
\end{eqnarray}
which is proximity coupled with a s-wave superconductor with the full Hamiltonian given by Eq.(\ref{ham}).
In Fig.\ref{FIG4}b, we depict the band evolution of this model which clearly shows that the superconducting gap is closed at $\theta=\pi/2$, i.e., along $k_y$-axis. In Fig.\ref{FIG4}c, the angular dependence of $T^A$ is plotted for different $V$, from which we see that is a s-wave like superconducting state at $V=0$ and changes to p-wave like gapless superconducting state which has been studied in details in Ref.\onlinecite{L-Fu1}. In Fig.\ref{FIG4}d, the mirage gap is manifested in the integer Andreev transmission coefficient similar to what we just discussed for AM-superconductor.


\begin{figure}[ht!]
\centering
\includegraphics[width=0.90\columnwidth]{FIG4}
\caption{
\label{FIG4}
(a) Schematic plot of the system for quantum transport calculation.  (b) Energy band evolution for the model Eq.(\ref{Fu}) for different $V$. (c) Angular dependence of $T^A$ at different $V=0, 1.2\Delta, 2.0\Delta$ for $E_F=0$.
(d) Andreev reflection $T^A$ versus $E_F$ for different $V$ with $\theta=\pi/2$, i.e., ${\bf V}= V {\bf {\hat e}}_y$. We set $v_F=0.1$, $\mu=0.02$, $\Delta=0.001$.
}
\end{figure}

%JIAN how to tune ASS?

\noindent{\it Conclusion} --- In summary, we show that in the absence of SOI the AM-superconductor can exhibit d-wave gapless superconducting state by tuning ASS and the singlet and triplet pairing occurs at finite energy at the same time, leading to the mirage gap which is quantified by integer Andreev reflection coefficient with a nonzero quasi-particle transmission coefficient. When SOI is present, both main gap and mirage gap are nonzero and the system changes from a d-wave superconducting state to a  s-wave superconducting state when the strength of SOI exceeds a critical value.

\bigskip

\noindent{\it Acknowledgments} ---
This work was supported by the National Natural Science Foundation of China (Grant No. 12034014, 12074230, and 12174262).


\begin{thebibliography}{00}
\bibitem{RMP1} A. I. Buzdin, Rev. Mod. Phys. 77, 935 (2005).
\bibitem{RMP2} F. S. Bergeret, A. F. Volkov, and K. B. Efetov Rev. Mod. Phys. 77, 1321 (2005).
\bibitem{RMP3} F. S. Bergeret, M. Silaev, P. Virtanen, and T. T. Heikkila, Rev. Mod. Phys. 90, 041001 (2018).
\bibitem{Clogston}
A. M. Clogston, Phys. Rev. Lett. 9, 266 (1962). %Pauli limit
\bibitem{L-Fu1}
N. F. Q. Yuan and L. Fu, Phys. Rev. B 97, 115139 (2018).
\bibitem{L-Fu3}
M. Papaj and L. Fu, Nat. Commun. 12, 577 (2021).
\bibitem{L-Fu2}
Z. Zhu, M. Papaj, X.A. Nie, H.K. Xu, Y.S. Gu, X. Yang, D.D. Guan, S.Y. Wang, Y.Y. Li, C.H. Liu, J.L. Luo, Z.A. Xu, H. Zheng, L. Fu, J.F. Jia, Science 374, 1381 (2021).
%\bibitem{note0} There are three scenarios for the gapless superconducting states:
%(i) the first one is the so-called Bogoliubov Fermi surface, in which the gapless SC state is formed due to the form factor in excitation spectrum, like that in the p-wave and d-wave SC\cite{Agterberg, Brydon, Sim}.
%(ii) the second one is created by applying an external magentic field (under Pauli limit) to the SC, in which the SC gap is fully closed, as discussed in Ref.\onlinecite{J-Lee}. (iii) The third one the so-called segmented Fermi surface (There is no a gap along a segment of Fermi surface), which is created by in-plane magnetic field\cite{L-Fu1, Phan, Banerjee}. In this work, the gapless SC states refers to the last senario.
\bibitem{Agterberg} D. F. Agterberg, P. M. R. Brydon, and C. Timm, Phys. Rev. Lett. 118, 127001 (2017).
\bibitem{Brydon} P. M. R. Brydon, D. F. Agterberg, H. Menke, and C. Timm, Phys. Rev. B 98, 224509 (2018).
%\bibitem{Link} J. M. Link, I. Boettcher, and I. F. Herbut, Phys. Rev. B 101, 184503 (2020).
\bibitem{Sim} G. Sim1, M. J. Park, and S.B. Lee, Commun. Phys. 5, 220 (2022).
%\bibitem{Phan} D. Phan, J. Senior, A. Ghazaryan, M. Hatefipour, W. M. Strickland, J. Shabani, M. Serbyn, and A. P. Higginbotham, Phys. Rev. Lett. 128, 107701 (2022).
%\bibitem{Banerjee} S. Banerjee, S. Ikegaya, and A. P. Schnyder, Phys. Rev. Res. 4, L042049 (2022).
\bibitem{J-Lee} J. E. Lee \textit{et al.}, Nat. Communi. 14, 2737(2023).
\bibitem{J-Lu}
J. M. Lu, O. Zheliuk, I. Leermakers, N. F. Q. Yuan, U.
Zeitler, K. T. Law, and J. T. Ye, Science 350, 1353 (2015). %Ising supercond
\bibitem{X-Xi}
X. Xi, Z. Wang, W. Zhao, J.-H. Park, K. T. Law, H. Berger,
L. Forro, J. Shan, and K. F. Mak, Nat. Phys. 12, 139 (2016). %Ising supercond
\bibitem{KT-Law1}
B. T. Zhou, N. F. Q. Yuan, H.-L. Jiang, and K. T. Law, Phys. Rev. B 93, 180501(R) (2016). %Ising Supercond and MF
\bibitem{Y-Saito}
Y. Saito, Y. Nakamura, M. S. Bahramy, Y. Kohama, J. Ye, Y.
Kasahara, Y. Nakagawa, M. Onga, M. Tokunaga, T. Nojima,
Y. Yanase, and Y. Iwasa, Nat. Phys. 12, 144 (2016). %Ising supercond
\bibitem{QF-Sun1}
P. Lv, Y.-F. Zhou, N.-X. Yang, and Q.-F. Sun, Phys. Rev. B 97, 144501 (2018). %quantum transport in Ising supercond
\bibitem{QF-Sun2}
Q. Cheng and Q.-F. Sun, Phys. Rev. B 99, 184507 (2019). %quantum transport in Ising supercond
\bibitem{GM-Tang}
G.M. Tang, C. Bruder, and W. Belzig, Phys. Rev. Lett. 126, 237001 (2021).
\bibitem{GM-Tang1}
S. Patil, W. Belzig, and G.M. Tang, arXiv: 2307.03456.
%\bibitem{M-Bahari}
%M. Bahari, S.B. Zhang, and B. Trauzettel, Phys. Rev. Res. 4, L012017 (2022).
\bibitem{Chakraborty}
D. Chakraborty and A. M. Black-Schaffer, Phys. Rev. B 106, 024511 (2022).
\bibitem{AM0}
I. I. Mazin, K. Koepernik, M. D. Johannes, R. Gonzalez-Hernandez, and L. Smejkal, Proceedings of the National Academy of Sciences 118, e2108924118 (2021).
\bibitem{AM1}
L. Smejkal, A. B. Hellenes, R. Gonzalez-Hernandez, J. Sinova, and T. Jungwirth, Phys. Rev. X 12, 011028 (2022).
\bibitem{AM2}
L. Smejkal, J. Sinova, and T. Jungwirth, Phys. Rev. X 12, 031042 (2022).
\bibitem{AM3}
L. Smejkal, J. Sinova, and T. Jungwirth, Phys. Rev. X 12, 040501 (2022).
\bibitem{AM4}
Z.X. Feng, X.R. Zhou, L. Smejkal, L. Wu, Z.W. Zhu, H.X. Guo, R. Gonzalez-Hernandez, X.N. Wang, H. Yan, P.X. Qin, X. Zhang, H.J. Wu, H.Y. Chen, Z. Meng, L. Liu, Z.C. Xia, J. Sinova, T. Jungwirth, and Z.Q. Liu, Nature Electronics 5, 735C743 (2022).
\bibitem{AM5}
R. Gonzalez-Hernandez, L. Smejkal, K. Vyborn, Y. Yahagi, J. Sinova,  T. Jungwirth, and J. Zelezn, Phys. Rev. Lett. 126, 127701 (2021).
\bibitem{C-Song1}
H. Bai, L. Han, X. Y. Feng, Y. J. Zhou, R. X. Su, Q. Wang, L. Y. Liao, W. X. Zhu, X. Z. Chen, F. Pan, X. L. Fan, and C. Song, Phys. Rev. Lett. 128, 197202 (2022)
\bibitem{Nitta}
S. Karube, T. Tanaka, D. Sugawara, N. Kadoguchi, M. Kohda. and J. Nitta, Phys. Rev. Lett. 129, 137201 (2022).
\bibitem{C-Song2}
H. Bai, Y. C. Zhang, Y. J. Zhou, P. Chen, C. H. Wan, L. Han, W. X. Zhu, S. X. Liang, Y. C. Su, X. F. Han, F. Pan, and C. Song, Phys. Rev. Lett. 130, 216701 (2023).
\bibitem{Y-Yao1}
X.D. Zhou, W.X. Feng, R.W. Zhang, L. Smejkal, J. Sinova, Y. Mokrousov, and Y.G. Yao, arXiv: 2305.01410.
\bibitem{AM6}
L. Smejkal, R. Gonzalez-Hernandez, T. Jungwirth, J. Sinova, Sci. Adv. 6, eaaz8809 (2020).
\bibitem{AM7}
L. Smejkal, A. H. MacDonald, J. Sinova, S. Nakatsuji, and T. Jungwirth, Nat. Rev. Mat. 7, 482 (2022).
\bibitem{NC1}
N. J. Ghimire, A.S. Botana, J.S. Jiang, J.J. Zhang, Y.-S. Chen, and J.F. Mitchell, Nat. Commun. 9, 3280 (2018).
\bibitem{Sinova1}
L. Smejkal, A. Marmodoro, K.H. Ahn, R. Gonzalez-Hernandez, I. Turek, S. Mankovsky, H. Ebert, S. W. D'Souza, O. Sipr, J. Sinova, and T. Jungwirth, arXiv: 2211.13806.
\bibitem{T-Yu}
Q.R. Cui, B.W. Zeng, T. Yu, H.X. Yang, and P. Cui, arXiv: 2306.08976.
\bibitem{Q-Liu}
P.F. Liu, J.Y. Li, J.Z. Han, X.G. Wan, and Q.H. Liu, Phys. Rev. X 12, 021016 (2022).
\bibitem{Y-Yao} R.W. Zhang, C.X. Cui, R.Z. Li, J.Y. Duan, L. Li, Z.M. Yu, and Y.G. Yao, arXiv: 2306.08902.
\bibitem{Lovesey} S. W. Lovesey, D. D. Khalyavin, and G. van der Laan, arXiv: 2306.12130.
\bibitem{Sattigeri} R. M Sattigeri, G. Cuono, and C. Autieri, arXiv: 2307.10146.
\bibitem{Brataas1}
J. A. Ouassou, A. Brataas, and J. Linder, arXiv: 2301.03603.
\bibitem{SB-Zhang}
S.B. Zhang, L.H. Hu, and T. Neupert, arXiv: 2302.13185.
\bibitem{Beenakker}
C. W. J. Beenakker and T. Vakhtel, arXiv: 2306.16300.
\bibitem{Papaj}
M. Papaj, arXiv: 2305.03856.
\bibitem{Brataas2}
C. Sun, A. Brataas, and J. Linder, arXiv: 2303.14236.
\bibitem{D-Zhu}
D. Zhu, Z.Y. Zhuang, Z.G. Wu, and Z.B. Yan, arXiv: 2305.10479.
\bibitem{Hughes}
S. A. A. Ghorashi, T. L. Hughes, and J. Cano, arXiv: 2306.09413.





\bibitem{SC-Zhang}
S.B. Chung, X.L. Qi, J. Maciejko, and S.C. Zhang, Phys. Rev. B 83, 100512(R) (2011).
\bibitem{Gorkov}
L. P. Gorkov and E. I. Rashba, Phys. Rev. Lett. 87, 037004 (2001).
\bibitem{note1}
Note that $T_{\uparrow}^{A}={\rm Tr}{[ {{\Gamma }_{L\text{e}}}{{G}^{r}}{{\Gamma }_{L\text{h}}}{{G}^{a}} ]}_{\uparrow\uparrow}$ consists of two processes: ${\rm Tr}[ \Gamma_{L\text{e}\uparrow} G^r_{\uparrow\downarrow} \Gamma_{L\text{h}\downarrow} G^{a}_{\downarrow\uparrow}]$ and ${\rm Tr}[ \Gamma_{L\text{e}\uparrow} G^r_{\uparrow\uparrow} \Gamma_{L\text{h}\uparrow} G^{a}_{\uparrow\uparrow}]$ which correspond to, respectively, the spin singlet and triplet pairing ($F_z \ne 0$).
\bibitem{note2}
In a different Nambu basis, the Hamiltonian can be written as\cite{Beenakker}
\begin{equation}
H = \left( \begin{matrix}
   {{H}_{0}}( k ) & \Delta   \\
   \Delta  & -\sigma_y H_{0}^{*}( -k )\sigma_y  \\
\end{matrix} \right)
\nonumber
\end{equation}
and it is not difficult to see that $\sigma_z$ is a good quantum number.
\bibitem{Sigrist}
P. A. Frigeri, D. F. Agterberg, A. Koga, and M. Sigrist, Phys. Rev. Lett. 92, 097001 (2004).
\bibitem{book1}
N. Kopnin, Theory of Nonequilibrium Superconductivity (Oxford University Press, 2001).



\end{thebibliography}


\end{document}
