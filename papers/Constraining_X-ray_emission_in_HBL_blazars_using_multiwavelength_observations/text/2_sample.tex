\section{The sample} \label{sample}
52 HBL type blazars have been detected in the TeV energy range (TeVCat\footnote{\url{http://tevcat.uchicago.edu/}}).
Most of them were observed with \xrt\ with PC or WT mode of the instrument \citep[see for detail][]{Wierzcholska2016}.
In this work, we aim to characterize emission observed from infrared to X-ray range for extreme HBL type blazars. 
We note here that extreme blazars are described by extremely energetic synchrotron emission, and the inverse Compton bump extends to the very high energy $\gamma$-ray regime. 
For such sources, the X-ray spectrum is located at energies below the peak of the low energy bump in SED.
For further studies, we select sources, TeV $\gamma$-ray emitting blazars, which fulfil the following criteria:
\begin{itemize}
 \item the source belongs to HBL type blazars with the X-ray spectrum located at the growing part of the synchrotron bump in the SED,
  \item multiwavelength data covering low energy bump in the SED are available;
 \item archival observations of the global SED (e.g., taken from ASDC\footnote{\url{http://www.asdc.asi.it/}})  revealed prominent emission visible in the \itu\ range indicating that host galaxy emission is significantly stronger than emission originating from all other components.
\end{itemize}
As a result the selected sample consists of 5 sources: 1ES\,0229+200, PKS\,0548-322, RX\,J1136+6737, 1ES\,1741+196, 1ES\,2344+514 as targets for further studies.\footnote{We note here also that during the review process of the paper, the MAGIC Collaboration reported the discovery of a new sample of hard-TeV extreme blazars \citep{MAGIC_extreme}. These AGNs are also possible targets: TXS\,0210+515, RBS\,0723, 1ES\,1426+428, 1ES\,2037+521, and RGB\,J2042+244, for similar studies as presented in the paper.}

Two potential sources: Mrk\,421, and Mrk\,501 are not included in these studies due to significant X-ray variability seen in the X-ray observations and the fact that most \xrt\ data taken for these blazars are in WT mode. 
We limit our analysis to PC mode data only since data  taken in the WT mode are affected by charge redistribution problems inherently related to how the CCD of the instrument is read \citep[also noted by, e.g.][]{Massaro2008, Wierzcholska2016}. This causes that in the case of the longterm integrated X-ray spectra, wavy features can be present. 


\section{Data and  analysis methods} \label{data}

In order to ensure good data coverage of the low energy bump in the SEDs of  sources studied: 1ES\,0229+200, PKS\,0548-322, RX\,J1136+6737, 1ES\,1741+196, 1ES\,2344+514
multiwavelength data are used. In the case of all blazars, WISE, 2MASS, ATOM, \uvot, \xrt\ data are used. In the case of 1ES\,0229+200 also \nus\ observations are studied. 


\subsection{Data analysis}


\subsection*{NuSTAR data}

The Nuclear Spectroscopic Telescope Array (NuSTAR) is a satellite instrument dedicated to observations in the hard X-ray regime of 3-79 keV \citep{Harrison2013}.
Out of the five targets selected for these studies, only \one\ was observed with \nus. Details on three pointings performed are summarized in Table\,\ref{table:nustar}.
All observations were performed in the \verb|SCIENCE| mode.

The raw data were processed with \nus\ Data Analysis Software package
(\verb|NuSTARDAS|, 
released as part of
\verb|HEASOFT|~6.25)
using standard 
\verb|nupipeline|
task. 
Instrumental response matrices and effective area files were produced with
\verb|nuproducts|
procedure. 
The spectral analysis was performed for the channels corresponding to the energy band of 3-79\,keV. 
We mention here that \nus\ observations of \one\ have been previously reported by \cite{Bhatta_0229} and \cite{Pandey_0229}.
However, \cite{Bhatta_0229} have discussed only one \nus\ observation of \one, while \cite{Pandey_0229} focused on temporal variability.








\subsection*{\textit{Swift}-XRT data}
The X-ray data used in this work are the ones collected with Swift \citep{Gehrels}, which are exactly the same sets as studied by \cite{Wierzcholska2016}, reanalyzed with a more recent version of the HEASoft software (6.25) with the recent CALDB. %v.\,20170505. 
All events were cleaned and calibrated using \verb|xrtpipeline| task, and the PC mode data in the energy range of 0.3-10\,keV with grades 0-12 were analyzed.
The reanalyzed spectra are entirely consistent with the previous ones.

 \subsection*{\textit{Swift}-UVOT data}
The UVOT instrument onboard \textit{Swift}  measures the ultraviolet and optical emission simultaneously with the X-ray telescope.
The observations are taken in the UV and optical  bands with 
the central wavelengths of UVW2 (188 nm), UVM2 (217 nm), UVW1 (251 nm), U (345 nm), B
(439 nm), and V (544 nm). 
The instrumental magnitudes were calculated using \verb|uvotsource| including all photons from a circular region with radius 5''.
The background was determined from a circular region with a radius of 10'' near the source region, not contaminated with the signal from nearby sources. 
The flux conversion factors as provided by \cite{Poole08} were used. 
All data were corrected for the dust absorption using the reddening $E(B-V)$   as provided by \cite{Schlafly} and summarized in Table\,\ref{table:datasummary}. 
The ratios of the extinction to reddening, $A_{\lambda} / E(B-V)$, for each filter were provided by \cite{Giommi06}.
The reddening $E(B-V$) and atomic hydrogen column density (\nh) are related with the following formula \nh = 8.3 $\cdot 10^{21}E(B-V)$ \citep{Liszt2014}.

\noindent

\subsection*{ATOM data}
ATOM (Automatic Telescope for Optical Monitoring) is a 75\,cm optical telescope located in Namibia near the H.E.S.S. site \citep{Hauser2004}.
The instrument measures optical magnitudes in four optical filters: B (440 nm), V (550 nm), R (640 nm), and I (790 nm).

For all sources, the data collected in the period starting from 2007 to 2015 have been analyzed using an aperture of 4'' radius and differential photometry. The observations have been corrected for dust absorption.

ATOM observations are available for the following sources from the sample, namely: 1ES\,0229+200,  PKS\,0548-322, and 1ES\,1741+196. In the case of each of the sources, there is no significant variability detected in the ATOM bands. 

\subsection*{2MASS data}
The 2MASS data are taken from the 2MASS All-Sky Point Source Catalog (PSC) \citep{TwoMassPSC}. 
The data were corrected for the dust absorption using the same $E(B-V)$ factors as for the optical data mentioned previously.

\subsection*{WISE data}
Wide-field Infrared Survey Explorer (WISE) is a space telescope which performs observations in the infrared energy band at four wavelengths: 3.4\,$\mu$m (W1), 4.6\,$\mu$m (W2), 12\,$\mu$m (W3) and 22\,$\mu$m (W4). The spectral data were taken from the AllWISE Source Catalog and the light curve from the AllWISE Multiepoch Photometry Table\footnote{\url{http://wise2.ipac.caltech.edu/docs/release/allwise/}}. The magnitudes were converted to flux by applying the standard procedure \citep{Wright_2010}. For W1 and W2, we applied the colour correction of a power-law with index zero, as suggested for  Galactic emission. Since W3 and W4 are widely dominated by the non-thermal radiation and show similar flux densities, we used the colour correction for a power-law of index -2.
For none of the blazars studied, the infrared light curves does not show any  variability within uncertainties of the measurements.

\subsection{Simultaneity of the data}

Blazars are known for their variability observed in all energy regimes. 
Thus, simultaneous multiwavelength observations are essential to study broadband emission observed from blazars.
Unfortunately, single \xrt\ observations usually do not allow to constrain spectral parameters with small uncertainties.
In order to find a sustainable solution for this, here, we study  integrated X-ray spectra of five blazars.
However, only PC mode data is used, since high state observations are usually performed with the WT mode of \xrt.
In order to check if the effect of variability is significant, we quantified it by fitting a constant function to the light curve data points. 
For all cases, such a fit yields $\chi_{red}^2$ of 0.8-1.4,  showing no indication for variability in the X-ray regime for any of the targets.


Furthermore, thanks to simultaneous observations with \uvot, optical, and ultraviolet data collected with this instrument are averaged over the same period as X-ray data. 
Optical observations collected with ATOM are averaged over a similar period as X-ray monitoring data. 
We note here, that in the case of X-ray spectra obtained with \nus, exactly simultaneous \xrt\ and \uvot\ data are used. 








\subsection{Spectral analysis and fitting procedure} \label{fitting}
In order to characterize X-ray spectra of five sources from the sample two different models, the absorbed power-law (\texttt{wabs\,$*$\,powerlaw} in \verb|XSPEC|) and absorbed logparabola (\texttt{wabs\,$*$\,logparabola} in \verb|XSPEC|) were tested.

A single power-law is  defined as:
\begin{equation}
 \frac{dN}{dE}=N_p  \left( \frac{E}{E_0}\right)^{-{\gamma}},
\end{equation}
with the normalization $N_p$, and the photon index $\gamma$;
while a logparabola model is defined as:
\begin{equation}
 \frac{dN}{dE}=N_l  \left( \frac{E}{E_0}\right)^{-({\alpha+\beta \log (E/E_0)})},
\end{equation}
 with the normalization $N_l$, the spectral parameter $\alpha$ and the curvature parameter $\beta$. 
 Galactic absorption is a function of exponent:  $ e^{-(N_{H}^{S}+N_{H}^{A})\sigma (E)}$, where  N$_{H}^{S}$ is the hydrogen column density value taken for a given survey,  N$_{H}^{A}$ is the additional, intrinsic column density  and $\sigma$(E) is the non-Thomson  energy dependent photoelectric  cross section \citep{Morrison83}.
 For the case when there is an assumption of lack of an additional absorption, only N$_{H}^{S}$ is taken into account, and N$_{H}^{A}$ is fixed to zero. 

We note here that except for absorbing model described with \texttt{wabs}, we also tested \texttt{phabs} and \texttt{tbabs} \citep{wabsy}.
This resulted in similar spectral parameters, as presented in the paper and can be found in the online material to the paper. 



\subsection{Fitting of different energy bands}
The fitting of the X-ray spectrum for \xrt\ observations is performed in two different energy bands: 0.3-10\,keV and 2.0-10\,keV. 
The energy band of 0.3-10\,keV includes complete spectral information from \xrt\ observations.
In this case, different corrections for the Galactic absorption are tested. 
The 2.0-10\,keV band has limited coverage and is slightly less affected by absorption. 
This is simply because the Galactic absorption plays a substantial role from low energies up to about 2\,keV \citep[e.g.][]{Campana2014} and thus by limiting the fitting range the influence of the Galactic absorption is marginal.



\subsection{Fitting of host galaxy}
The fitting procedure of a host galaxy profile to the observational data is the following:
\begin{enumerate}
\item The preferred X-ray spectral model, according to X-ray data analysis as presented by \cite{Wierzcholska2016}, is chosen for a  given object, as a starting point. 
\item The X-ray spectrum is then extrapolated to the lower energies, up to the IR regime, and added to the synthetic host galaxy profile.
\item The total profile is then fitted to the observational data points.
\end{enumerate}



The synthetic host galaxy profiles are generated with GRASIL \citep[for more details see][]{Silva98, Silvathesis, grasil}, a code to compute spectral evolution of stellar systems.
For this work,  the simulations of the standard model of an elliptical galaxy based on the model described in the manual pages\footnote{\url{http://adlibitum.oats.inaf.it/silva/grasil/modelling/modlib.html}} is used. 
We generated $11 \times 26$ synthetic spectra for a set of 11 different infall masses, $M_\textrm{inf}$: $10^7$, $10^{7.5}$, ... , $10^{12}$ $M_\odot$, and 26 different galaxy ages, $t_\textrm{gal}$:  $0.1$, $0.6$, ..., $13.1$ Gyr.


The fitting method is as follows:
\begin{enumerate}
 \item The age of the galaxy is set to a given value.
 \item The $\texttt{scipy.interpolate.interp2d}$ function \citep{SciPy} is used to interpolate generated spectra (points) in two dimensions of infall mass and frequency in order to have a well-covered grid. 
 This returns a function $S_\textrm{syn}(\nu,M_\textrm{inf})$ which computes the spectral points for a given frequency and infall mass.
 \item the $S_\textrm{syn}$  (with added extrapolated X-ray spectrum) is fitted to the observational data using  $\texttt{scipy.optimize.curve\_fit}$ function.
 \item The procedure is repeated for each galaxy age.
 \item The best fit is selected according to the $\chi^2$ value.
\end{enumerate}

\noindent





\begin{table} 
\centering 
\begin{tabular}{c|c|c}
\hline
\hline
 Source &  Swift obsIDs  & $E(B-V)$   \\
(1) &  (2)& (3)     \\
\hline
1ES\,0229+200 & 00031249001-00031249050   &  0.1172  \\
PKS\,0548-322 & 00044002001-00044002065  &  0.0304   \\
RXJ\,1136+6737 & 00037135001-00040562010  &  0.0074   \\
1ES\,1741+196  & 00030950001-00040639018   &  0.0759   \\
1ES\,2344+514 & 00035031001-00035031121  &  0.1819   \\
\hline
\end{tabular}
\caption[]{Details on data studied. The following columns present: (1) name of the source; (2) Swift (XRT and UVOT) observation IDs used in the paper; (3) $E(B-V)$ reddening coefficient.}
  \label{table:datasummary}
\end{table}







 


