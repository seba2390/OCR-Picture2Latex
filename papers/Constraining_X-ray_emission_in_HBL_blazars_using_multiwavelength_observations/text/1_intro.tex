\section{Introduction}\label{intro}
The class of blazars is characterized by a polarized and highly variable non-thermal continuum emission and composed of BL Lacertae (BL Lac) type sources and flat spectrum radio quasars (FSRQs).
The distinction has  historically been  defined based on the equivalent width of the optical emission lines.
According to the unified model \citep[e.g.][]{Urry95}, blazars are  observed at small angles of the observer's line of sight to the relativistic jet axis \citep[e.g.][]{begelman84}.
The electromagnetic radiation that is emitted by blazars is observed in the full energy range, starting from radio frequencies up to very high energy $\gamma$ rays \citep[e.g.][]{Wagner2009, Abramowski2014}.
Temporal variability, observed in all energy regimes on different time scales from minutes up to years is a characteristic feature of these sources \citep[e.g.][]{2155flare, Wierzcholska_0048, Liao15, Wierzcholskas5}.
Furthermore, temporal flux changes in blazars are often associated with spectral variability \citep[e.g.][]{Xue06, Bottcher10, Wierzcholska2016, Siejkowski_2017}.

A typical spectral energy distribution (SED) of blazars  is described by two broad emission components. 
The  low-energy peak in the SED is usually explained by synchrotron radiation of relativistic electrons from the jet, 
while the  high-energy component can be explained in leptonic or hadronic scenarios \citep[see, e.g.][]{Dermer92, Sikora94, Mucke13, Bottcher13}.
In the case of leptonic models, the high-energy peak can be either caused by inverse Compton scattering of relativistic electrons from the jet (synchrotron-self-Compton models, SSC) or by photon fields external to the jets (external Compton models, EC). 
The external field of photons can be caused by emission observed from  a dusty torus or broad line region. 

The position of two peaks allows us to distinguish three subgroups of BL Lac type blazars, namely: high-, intermediate- and low-energy  peaked BL Lac objects: HBL, IBL, LBL, respectively \citep[see, e.g.][]{padovani95, fossati98, Abdo2010}.
In the case of HBL type blazars the synchrotron peak is located in the X-ray domain ($\nu_{s}>10^{15}$\,Hz) \citep[][]{Abdo2010}.



  
Two functions are commonly used to describe the synchrotron spectrum of blazars. 
These are a single power-law model and a curved power-law (also known as logparabola). 
The first one is characterized by a photon index $\gamma$, which is related to the index of input particles as $q = 2\gamma +1$ \citep{Rybicki}.
Similarly, a curved spectrum can be produced by logparabolic particle distribution \citep{Paggi09} and is an indication for statistical acceleration process \citep[e.g.][]{Massaro2004, Massaro2006, Massaro2008, Tramacere2007}.


The emission observed in the soft X-ray range is affected by absorption of the interstellar medium,  along line of sight, in our Galaxy. 
The effect of the absorption  extends up to about 10\,keV with the most substantial influence up to 2\,keV
and it is calculated using a neutral hydrogen column density.
The hydrogen column density observations have been reported in different surveys. 
The  Leiden Argentine Bonn Survey \citep[LAB, ][]{Kalberla2005}  includes only N$_{HI}$ measure. 

%here



\cite{Willingale13} have proposed a  measure of N$_{H,tot}$, which includes both the atomic gas column density N$_{HI}$ and molecular hydrogen column density N$_{H_2}$.
The N$_{HI}$ value is taken from the LAB survey, while  N$_{H_2}$ is estimated based on maps of infrared dust emission and dust-gas ratio as provided by \cite{Schlegel98} and \cite{Dame01}, respectively.
As an alternative, spectra of blazars can be fitted with a free value of column density, which allows  the derivation of the total X-ray absorption \citep[e.g.][]{Furniss2013, Wierzcholska2016, Gaur18}.

The X-ray spectra alone cover too small range in energy to decompile the effect of absorption from the range of different effects that can cause spectral curvature. 
Thus a more in-depth look into multifrequency observations of a source is needed. 


In this paper, we investigate X-ray observations of HBL type blazars together with multiwavelength data  of synchrotron part of the spectrum and a host galaxy template to study properties of the X-ray spectra and host galaxy emission observed in blazars selected. 
The paper is organized as follows: Sect.~\ref{sample} introduces the sample of targets studied, Sect.~\ref{data} presents detail on data presented in the paper and data analysis detail, Sect.~\ref{results} shows the results.
The work is summarized in Sect.~\ref{summary}. 

\noindent

