\begin{abstract}
The X-ray spectrum of extreme HBL type blazars is located in the synchrotron branch of the broadband spectral energy distribution (SED), at energies below the peak. 
A joint fit of the extrapolated X-ray spectra together with a host galaxy template allows characterizing the synchrotron branch in the SED. 
 The X-ray spectrum is usually characterized either with a  pure or a curved power-law model. 
In the latter case, however,  it is hard to distinguish an intrinsic curvature from excess absorption. 
In this paper, we focus on five well-observed  blazars: 1ES\,0229+200, PKS\,0548-322, RX\,J1136+6737, 1ES\,1741+196, 1ES\,2344+514. 
We constrain the infrared-to-X-ray emission of these five blazars using a model that is characterized by the host galaxy, spectral curvature, absorption, and ultraviolet excess to separate these spectral features. 
 In the case of four sources: 1ES\,0229+200, PKS\,0548-322,  1ES\,1741+196, 1ES\,2344+514 the spectral fit with the atomic neutral hydrogen from the Leiden Argentina Bonn Survey result in a significant UV excess present in the broadband spectral energy distribution. 
Such excess can be interpreted as an additional component, for example, a blue bump.
However, in order to describe spectra of these blazars without such excess, additional absorption to the atomic neutral hydrogen from the Leiden Argentina Bonn Survey is needed. 



\end{abstract}