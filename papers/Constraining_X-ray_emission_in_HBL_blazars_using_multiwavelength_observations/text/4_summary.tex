\section{Summary} 
\label{summary}
The X-ray spectra of blazars in the energy range of 0.3-10\,keV are commonly well explained using a  power-law or curved power-law model. 
Both these forms can well describe the particle distribution responsible for the emission observed in a given energy range \citep[e.g.][]{Tramacere2007,  Massaro2008}. 
The curvature parameter reported for X-ray spectra of blazar, ranging between 0.1-0.9,  illustrates a variety of spectral properties of blazars'  X-ray spectra \citep [see, e.g.][]{Massaro2004, Tramacere2007, Wierzcholska2016}.
The curved \po\ model is successfully used for all subclasses of blazars, including FSRQs, LBLs, IBLs, and HBLs \citep[e.g.,][]{Wierzcholska2016}.

However, a concave curvature seen in the soft X-ray range can be either an intrinsic feature of the blazar or can be caused by the absorption that is not included in the spectral model.
Usually, using only X-ray observations of the blazars, it is not possible to judge either intrinsic curvature or a  need for an additional absorbing component is a more plausible scenario that can explain the spectral shape. 
One way to verify the origin of spectral curvature is dedicated observations that confirm the existence of gas that could be intrinsic to the blazar or located somewhere between the source and an observer.
Absorption features in the X-ray regime, around 0.5-0.6keV  have been detected in the case of few blazars including: PKS\,2155-304 \citep{Canizares_2155, Madejski_2155}, H\,1426-428 \citep{Sambruna_1426}, 3C\,273 \citep{Grandi_273}, PKS 1034-293 \citep{Sambruna}, PKS\,0548-322 \cite{Sambruna_1998}.
Also, absorption features have been reported in regime of 0.15-0.20\,keV for PKS\,2155-304 \citep{Koenigl_2155} and Mrk\,421\cite{Kartje_421}.

Alternatively, simultaneous multiwavelength data covering a significant part of the broadband spectral energy distribution can be used to get more constraints on the X-ray spectrum and absorption in the soft X-ray range. 


We use the models characterized with host galaxy, curvature of the X-ray spectrum, absorption and UV excess in order to describe broadband emission observed from infrared frequencies up to the X-ray band of five extreme HBLs: 1ES\,0229+200, PKS\,0548-322, RX\,J1136+6737, 1ES\,1741+196, 1ES\,2344+514.


We investigate whether curvature that is seen in the X-ray spectra in the energy range of 0.3-10\,keV is an intrinsic feature or caused by the absorption.

Our findings are summarized as follows:

\begin{itemize}

 \item  In the case of four blazars: 1ES\,0229+200, PKS\,0548-322,  1ES\,1741+196, 1ES\,2344+514 prefered spectral characteristic in the X-ray range is a \lp\ model, while for RX\,J1136+6737 a \po\ model is favourable. For all four blazars described with a \lp\ model, spectral fits with the LAB \nh\ value are characterized with a UV excess. 
 If such excess is a real feature of the broadband spectral energy distribution, it can be interpreted either as an additional component such as a blue bump or by unaccounted thermal emission from the AGN.  The latter case, however, seen to be less plausible since it then should be at the order of one or even two magnitudes above the host galaxy template.  
 
 
  \item In several works focusing on X-ray spectra authors have discussed also problem of proper \nh\ correction \citep[e.g.][]{Acciari_wcom, Wierzcholska2016}.
By using multiwavelength data, we could constrain the X-ray spectrum with all its aspects, including proper correction for hydrogen column absorption. 
 We then conclude that in the case of four sources mentioned, additional absorption,  than the values quoted in by \cite{Kalberla2005}, is needed to explain the spectral properties of these targets in the energy range of 0.3-10\,keV. In the case of 1ES\,0229+200, PKS\,0548-322, 1ES\,1741+196 \nh\ values proposed in this work are higher than previously reported. 
 This column density absorption can be either intrinsic to the source or caused by Galactic absorption in addition to the atomic neutral hydrogen from \cite{Kalberla2005}.
 In the case of PKS\,0548-322, \cite{Sambruna_1998} have reported the possibility of a presence of circumnuclear ionized gas that could explain the need for an additional absorption.
The upper limit of the intergalactic gas luminosity is about 15$\%$ of the BL Lac luminosity.

  
 
 \item The \lp\ model is commonly used to describe the X-ray spectra of blazars in the energy regime of 0.3-10\,keV. 
 By using extrapolated X-ray data together with host galaxy template and multiwavelength observations of blazars,  these preferred spectral shapes were confirmed. 
Albeit, curvature seen in the spectra of 1ES\,1741+196, 1ES\,2344+514 is negligible within the uncertainties.
 We then conclude that only in the case of the blazar PKS\,0548-322 the intrinsic spectral curvature is confirmed. 
 This suggests that for PKS\,0548-322 particle population responsible for the synchrotron emission observed should be assumed to be curved as well. 

 
  
 \item  For four blazars with the UV excess in the broadband SED while using the LAB value of \nh. The excess is no longer present in the spectrum, in the case when a higher amount of the absorption is used. 

 The component needed to explain UV excess  is consistent with the Galactic column density value from the survey by \cite{Willingale13} (5$\%$-10$\%$ difference only) and up to 60$\%$ higher than the amount quoted by \cite{Kalberla2005}.
 The uncertainties of the LAB survey are estimated as 2$\%$. 
 
 
 \item  In the case of \one\ existence of the cut-off in the UV regime is not confirmed. 
 Such a feature has been previously reported by \cite{Kaufmann2011}. 
 However, the authors constrained the synchrotron peak in the SED of \one\ with poorer multiwavelength coverage, as presented in this work. 
\end{itemize}





 
 
  



 

 
 
 





  












