\section{Results} \label{results}

\subsubsection*{1ES\,0229+200}
The studies of \one\ performed by \cite{Kaufmann2011} revealed unusual spectral characteristics of the X-ray emission detected up to about 100\,keV without any significant cut-off in the energy range studied. 
The model preferred by the authors has been a single \po, however, with an indication of excess absorption above the Galactic value. 
\cite{Kaufmann2011} has also shown that a cut-off characterizes the SED of \one\ in the low energy part of the synchrotron emission, located in the UV regime. 
In more recent studies, \cite{Wierzcholska2016} have demonstrated that the longterm integrated X-ray spectrum in the energy range of 0.3-10\,keV of \one\ could be described both with a single \po\ or a \lp\ model.
The authors have also shown that according to \ch\ values of the fits, the preferred \nh\ amount is the one from the LAB survey.
Within the uncertainties, this result is consistent with the work by \cite{Kaufmann2011}.

In the first step, all X-ray observations (as analysed by \cite{Wierzcholska2016}) collected with \xrt\  during the period of monitoring in the energy range of 0.3-10\.keV were fitted with a single \po\ and a \lp\ model, in both cases with the Galactic absorption (\nh$=8.06$\,$\cdot$10$^{20}$cm$^{-2}$, \citealt{Kalberla2005}).
Fits parameters of both models are given in Table~\ref{table:results}. 
The figures illustrating fits with the data points are included in the online material. 
Both spectra are then used together with the host galaxy templates, as described in Sect.~\ref{fitting},  to fit infrared-to-ultraviolet observations of \one. 
The results of the fits are presented in Fig.~\ref{figure:spectra_global}. 
In the case of the \po\ spectral model used, a  fit to \itu\ data is characterized  with a significant excess  observed in the UV regime. 
 We note here, that, the UV excess is defined as a difference between model and fluxes observed in the UV range.


In order to check how the excess observed depends on the Galactic absorption, the longterm integrated X-ray spectrum has been fitted again with the \po\ model in two different ways. 
\begin{itemize}
 \item In the first approach, the \nh\ value by \cite{Kalberla2005} is used, but the energy range is limited to 2.0-10\,keV. 
The fitting procedure used in this case is the same as described before, and the same multiwavelength data set is used. 
The resulting fit parameters are collected in the Table~\ref{table:results}, and the plot is presented in Fig.~\ref{figure:spectra_limited}.

\item 
In the second approach, the X-ray energy range used for the fitting is the entire \xrt\ energy range, but the fixed \nh\ value is higher than the one mentioned above.
 \cite{Willingale13} have suggested that in order to determine appropriately the influence of the Galactic absorption N$_{H,tot}$ value, which is a sum of the atomic gas column density N$_{HI}$ and the molecular hydrogen column density N$_{H_2}$ should be used.
Here, N$_{HI}$ is taken form the LAB survey \citep{Kalberla2005}, while N$_{H_2}$ is estimated using maps of dust infrared emission by \cite{Schlegel98} and the dust-gas ratio by \cite{Dame01}.
For the case of \one, N$_{H,tot}$  is equal to 11.80 $\cdot$ 10$^{20}$\,cm$^{-2}$. 
The fit to the data with the \po\ model plus the Galactic hydrogen absorption as provided by \cite{Willingale13} is presented in Fig.~\ref{figure:spectra_limited}
and fit parameters are collected in Table~\ref{table:results}.
\end{itemize}



In the case of the spectrum fitted to the limited energy range, the host galaxy is well described, and a UV excess is not required. 
The fit with the Galactic column density  value as provided by \cite{Willingale13} also does not require a UV excess and more significant discrepancy in the observed and modelled infrared emission in the low energy part of the WISE range. 

Previous works dedicated to X-ray studies of \one\ also revealed additional absorption observed in the case of \one. 
\cite{Kaufmann2011} have found that X-ray spectra of \one\ are well described with a single \po\ model with the additional absorption that can be either intrinsic to the blazar or in the line of sight to the observer or in the Milky Way. 
 
We note here also, that the evidence for UV excess in the X-ray spectrum of 1ES\,0229+200 has been reported by \cite{Costamante2018}.
The authors suggested that it can be either due to an additional emitting component or could be explained by thermal emission from the AGN. 
While no explanation could be identified, the shape of the X-ray spectrum required an excess emission in the UV range or excess absorption.



\subsubsection*{NuSTAR observations of \one}
In the case of one target from the sample, \one, hard X-ray data collected with \nus\ are available. 
This allows testing stability of the results obtained with \xrt, and check if additional X-ray observations ranging to 79\,keV improve the fit. 

In total, three \nus\ observations of \one\ were performed, all in October 2013.
During the \nus\ campaign, the source was also observed with \xrt\ and \uvot, providing almost a strictly simultaneous overlap of the broadband spectral energy distribution. 
Table~\ref{table:nustar} presents details on data that have been gathered with \nus\ and \xrt\ during \nus\ observation period. 

In order to test whether the \nus\ X-ray spectra of \one\ are better constrained with a single \po\ model or a \lp, all three \nus\ observations were fitted with  both models.
Table~\ref{table:results_0229} summarizes parametres of spectral fits and Figure~\ref{figure:nus_spec} shows three \nus\ spectra. 
We note here that in the case of the  \nus\ observations of \one\ X-ray spectrum is here constrained in the energy range of 3-50\,keV. 
Similar $\chi_{red}^2$ values for both \po\ and \lp\ fits indicate that the hard X-ray spectrum suggests that the spectra can be described with the same precision while using one of these models. 
However, using spectral parameters for the \po\ model of the \nus\ spectra, together with the host galaxy template, we are not able to reproduce the \itu\ SED of \one. 
The extrapolated fit was above the datapoints. 
Thus, we conclude that the proper description of the X-ray spectrum of \one\ in the energy range above 3\,keV can be done only with a curved model. 
As the \po\ model was ruled out, we used the logparbolic description of the X-ray spectra together with the host galaxy templates. 
The resulting SEDs are presented in the first column in Figure~\ref{figure:nu}. 
For all ObsID, none of the marginal UV excesses is visible. 


Simultaneous \xrt-\nus\ observations of \one\ allow for joint spectral fitting in the energy range of 0.3-50\,keV. 
Figure~\ref{figure:nu} (the second and third column) presents broadband SEDs of \one\ being the result of \lp\ spectral model extrapolated together with the host galaxy template. 
Two different values of \nh\ are tested: the one provided by \cite{Kalberla2005} and by \cite{Willingale13}.
The UV excess is present in the SEDs obtained while using \nh\ by \cite{Kalberla2005}, and it is now visible in the case of SEDs obtained with \nh\ by e.g. \cite[][]{Willingale13}.

The analysis of \nus\ and \xrt-\nus\ data allows us concluding that while using  X-ray data above  10\,keV, we received as good information about synchrotron emission of \one\ as in the case of \xrt\ data only. 




\begin{figure}
 \centering{\includegraphics[width=0.5\textwidth]{figures2/nustar_all.pdf}}
 \caption[]{The spectral energy distribution of \one\ as seen with \nus. Three different colors correspond to different ObsIds: nu60002047002, nu600020470042, and nu60002047006. For each ObsId, the data are fitted with a single \po\ model.  }
 \label{figure:nus_spec}
\end{figure}

















 

\subsubsection*{PKS\,0548-322}
\cite{Wierzcholska2016}  have shown that the longterm integrated X-ray spectrum of PKS\,0548-322 in the energy range of 0.3-10\,keV has been well described with logparabola model with Galactic column density value found by \cite{Kalberla2005}. 
The X-ray spectrum fit parameters provided by \cite{Wierzcholska2016} together with the galaxy template reproduce the optical and IR observations well, but significant excess in the UV band is observed.
In order to check the influence of the Galactic absorption on X-ray spectral parameters, the X-ray spectrum has been again fitted in the energy range of 2.0-10\,keV. 
The spectrum has been extrapolated together with the template of the host galaxy up to the \itu\ range. The resulting fit parameters are only slightly different from the 0.3-10\,keV fit, but they allow to reproduce broadband emission of    PKS\,0548-322 without significant UV excess visible. 
From this fit, the host galaxy of the blazar is described with a mass of $(4.7\pm0.3) \cdot 10^{9} M_{\astrosun}$. 
Fit parameters of the X-ray spectrum for both cases are collected in Table~\ref{table:results} and plots are demonstrated in Figure~\ref{figure:spectra_other}.
The studies performed have shown that additional absorption is needed to reproduce the SED of the blazar without significant UV excess. 


PKS\,0548-322 was a frequent target of X-ray observations. 
The analysis reported by \cite{Costamante_0548} indicates the evidence of the intrinsic curvature of the X-ray spectrum of the blazar, even in the presence of an additional absorption. 
\cite{Perri_0548} also have reported a logparabola as a preferred model describing the X-ray spectrum of the blazar as seen with \xrt\ and Beppo-SAX.
\cite{hess_0548} have described the X-ray spectrum of PKS\,0548-322 in the energy range using a single \po\ and broken \po\ model without significant improvement while using the later model. However, the authors have mentioned a possible presence of an absorption feature near 0.7\,keV.




\subsubsection*{RX\,J1136+6737}
\cite{Wierzcholska2016} have found the power-law model as a good description of the X-ray spectrum in the energy range of 0.3-10\,keV. The authors have not  found any need for an additional absorption component in the spectral description. The preferred value of the Galactic column density has been the one provided by  \cite{Willingale13}.
Using extrapolated power-law fit together with a template of the host galaxy fits well the multiwavelength data without any excess in the UV range.  
A mass of host galaxy is estimated to have $(2.23 \pm 0.14)\cdot 10^{10}\,M_{\astrosun}$.
Fit parameters of the X-ray spectrum is presented in Table~\ref{table:results} and SED is shown in Figure~\ref{figure:spectra_other}.



\subsubsection*{1ES\,1741+196}
\cite{Wierzcholska2016} have shown that the X-ray spectrum of 1ES\,1741+196 is well described using a logparabola model with Galactic column density value provided by \cite{Willingale13}.
These spectral parameters together with the host galaxy template  were used and fitted to multiwavelength data collected for 1ES\,1741+196 to reproduce the low energy bump of the SED since significant UV excess is evident. 
The second fit to the data using the X-ray spectrum in the energy of 2.0-10\,keV reproduced data well without any excess visible in the UV range. 
According to this fit, the estimated value of the host galaxy mass is  $(1.37 \pm 0.08)\cdot 10^{10} \,M_{\astrosun}$.
The fit parameters of the X-ray spectrum is presented in Table~\ref{table:results} and SED is shown in Figure~\ref{figure:spectra_other}.
The studies performed have shown that additional absorption is needed to reproduce the SED of the blazar without significant UV excess. 

\cite{MAGIC_1741} have found that the X-ray spectrum of 1ES\,1741+196 is well described with a single \po\ without any indication of absorption excess. 
Contrary to these studies, \cite{VERITAS_1741} have concluded analysis of \xrt\ spectra of 1ES\,1741+196 that curvature in the spectrum is needed in order to explain the spectral shape in this range.


\subsubsection*{1ES\,2344+514}
In the studies of longterm X-ray spectra of blazars \cite{Wierzcholska2016} have shown that the spectrum of 1ES\,2344+514 is well described with the logparabola model with Galactic column density taken from the survey by \cite{Kalberla2005}.
This X-ray spectral fit extrapolated to the \itu\ range, together with a host galaxy template, results in large UV excess. 
Fit to the same X-ray data, but in the energy range of 2.0-10\,keV, together with the host galaxy, gives appropriate fit without UV excess. 
For the second fit estimated value of  host galaxy mass is $(19.2\pm00.5)\cdot 10^{10} \,M_{\astrosun}$.
Results of fits for both cases can be found in Fig.~\ref{figure:spectra_other}, and parameters of all fits are presented in Table~\ref{table:results}. 
The studies performed have shown that additional absorption is needed to reproduce the SED of the blazar without significant UV excess. 

No need for a curved model describing spectrum or additional absorbing component has been reported by \cite{MAGIC_2344}.
\cite{Kapanadze_2344} also have  prefered \po\ model to describe the X-ray spectra of this blazar. 
 
  We note here also, that for the case of 1ES\,2344+514 due to the low Galactic latitude of the blazar the \nh\ values taken from different catalogues are affected by large uncertainties.
 
 
\begin{figure*}
\centering{\includegraphics[width=0.495\textwidth]{figures2/sed_time2_polab.pdf}}
\centering{\includegraphics[width=0.495\textwidth]{figures2/sed_time2_lplab.pdf}}
\caption[]{Broadband SED of 1ES\,0229+200. Left: modelling with X-ray spectrum fitted with the power-law model in the energy range of 0.3-10\,keV with \nh\ value taken from \cite{Kalberla2005}; right: same as left but X-ray spectrum fitted with the logparabola model. 
Red points present WISE data, light blue 2MASS data, dark blue ATOM data, and green Swift-UVOT.}
\label{figure:spectra_global}
\end{figure*}

\begin{figure*}
\centering{\includegraphics[width=0.495\textwidth]{figures2/sed_time2_po210.pdf}}
\centering{\includegraphics[width=0.495\textwidth]{figures2/sed_time2_powill.pdf}}

\caption[]{Broadband SED of 1ES\,0229+200. Left: modelling with X-ray spectrum fitted with the power-law model in the energy range of 2.0-10\,keV with \nh\ value taken from \cite{Kalberla2005}; right: modelling with X-ray spectrum fitted with the power-law model in the energy range of 0.3-10\,keV with \nh\ value taken from \cite{Willingale13}.
Points colour coding same as in Fig.\ref{figure:spectra_global}.}
\label{figure:spectra_limited}
\end{figure*}



\subsection{\nh\  calculation}
 The spectral parameters of the fit in the energy range of  2-10\,keV were used as frozen for the fit to data in the energy range of 0.3-10\,keV. 
The only free fit parameter here was \nh, and thus, we were able to calculate \nh\ values for all  targets studied. 



 
 
 
 \begin{itemize}
 \item
 For 1ES\,0229+200 the calculated value of N$_H^{tot}$ is (13.00$\pm$0.37)$\cdot$10$^{20}$cm$^{-2}$. This value is larger than the one from the LAB survey (8.06$\cdot$10$^{20}$cm$^{-2}$) and consistent with the one provided by \cite{Willingale13} (10.80$\cdot$10$^{20}$cm$^{-2}$). 
 The value is also higher than free \nh = (10.6-10.8)$\cdot$10$^{20}$cm$^{-2}$ calculeted by \cite{Kaufmann2011}.
 \item
 For PKS\,0548-322 the calculated value of N$_H^{tot}$ is (3.10$\pm$0.10)$\cdot$10$^{20}$cm$^{-2}$. This value is larger than the one from the LAB survey (2.58$\cdot$10$^{20}$cm$^{-2}$)  and consistent with the one  provided by \cite{Willingale13} (2.87$\cdot$10$^{20}$cm$^{-2}$).
 However, \cite{Sambruna_1998} have found significatly higher value of \nh\ of 1.03$\cdot$10$^{21}$cm$^{-2}$ that is needed to explain X-ray properties of PKS\,0548-322. 
  \item
 For 1ES\,1741+196 the calculated value of N$_H^{tot}$ is (10.47$\pm$0.41)$\cdot$10$^{20}$cm$^{-2}$. This value is larger than the one from the LAB survey (7.32$\cdot$10$^{20}$cm$^{-2}$)  and consistent with the one  provided by \cite{Willingale13} (9.58$\cdot$10$^{20}$cm$^{-2}$).
   \item
 For 1ES\,2344+514 calculated value of N$_H^{tot}$ is (17.34$\pm$0.15)$\cdot$10$^{20}$cm$^{-2}$. This value is larger than the one from the LAB survey (15.10$\cdot$10$^{20}$cm$^{-2}$), however it is smaller than  the one provided by \cite{Willingale13} (24.60$\cdot$10$^{20}$cm$^{-2}$).
\end{itemize}    


We note here  that the reddening $E(B-V)$ and the atomic hydrogen column density (\nh) are related.
 According to recent work, \cite{Liszt2014} has described this dependence by the following formula \nh = $8.3 \cdot$ 10$^{21}E(B-V)$.
The formula gives \nh\ values of 9.7$\cdot$ 10$^{20}$, 2.5$\cdot$ 10$^{20}$, 0.6$\cdot$ 10$^{20}$, 6.3$\cdot$ 10$^{20}$, and 15$\cdot$ 10$^{20}$ for  1ES\,0229+200, PKS\,0548-322, RX\,J1136+6737, 1ES\,1741+196, 1ES\,2344+514, respectively. 
The quoted values are consistent with the ones from the LAB survey and do not indicate a need for an additional absorption.
All values of \nh\ values, including the catalogue ones, are collected in Table\,\ref{table:nhvalues}. 


