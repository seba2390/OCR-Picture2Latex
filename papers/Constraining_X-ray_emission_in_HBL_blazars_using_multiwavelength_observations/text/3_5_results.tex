\subsection{Constrains on the model parameters}
The model used to describe the infrared to X-ray emission of the blazars is characterized by: host galaxy, curvature seen in the X-ray spectrum, total absorption (which includes both Galactic component and possibly additional ones), and ultraviolet excess. 
The host galaxy, understood as the entire component described with the template by \cite{Silvathesis},  is well constrained with the model used. Still, the remaining three parameters cannot be disentangled using only information from the X-ray spectral fitting. 
The limitations on three entangled parameters are the following:

\begin{itemize}
 \item \textbf{$\beta$}: 
 For all of the blazars studied, the convex curvature of the X-ray spectrum is assumed. This implies that negative values of $\beta$ are not accepted. 
 Furthermore, \cite{Wierzcholska2016} have demonstrated that in the case of a fixed value of absorption (e.g., from the LAB survey), X-ray spectra can be characterized with $\beta$ from 0.1 up to 0.9. 
Albeit, in the case of the free value of \nh, the value of $\beta$ ranges from 0.04 up to 0.4. 
In our studies, spectral curvatures of extreme HBL blazars range between 0.05 and 0.37, which is consistent with the work by \cite{Wierzcholska2016} and also with other studies focusing on spectral curvatures seen in the X-ray range \citep[][]{Massaro2004, Tramacere2007}.
 \item \textbf{\nh}: According to the LAB survey, values of the Galactic absorption range between 10$^{20}$-10$^{22}$.
For the blazars in the sample, \nh\ values do not exceed 10$^{21}$, and additional value required to explain UV excess is of about 60$\%$ higher than the survey ones, which is still in the range of reasonable value. 
 \item \textbf{UV excess}: The value of UV excess cannot be negative. 
\end{itemize}




To check the impact of a UV excess  on other parameters, for each source, we performed a study on parameters variability in the following steps:
\begin{itemize}
 \item  Range of possible \nh\ values is selected. The starting value is the lowest one from the surveys.
 \item  Spectral fitting of the X-ray spectrum in the energy range of 0.3-10\,keV with the frozen value of \nh\ is performed. 
 \item  Fitting to the host galaxy template as described in \ref{fitting}.
 \item  Calculation of the UV excess as a distance of the model to UVW2 datapoint.
\end{itemize}


Figure\,\ref{figure:parameters}   presents comparison of $\alpha$,   $\beta$, and UV excess as a function of  \nh\  for 1ES\,0229+200, PKS\,0548-322, 1ES\,1741+196, and 1ES\,2344+514.
The figures illustrate the strong dependence of $\alpha$,   $\beta$, and UV excess parameters for all blazars. 
The $\alpha$ and $\beta$ parameters of the model change significantly within changes of the \nh\ value, and this also generates different UV excesses.  

Accoring to the limitations of the values of $\beta$ and UV excess, we can conclude that the values of \nh\ must be limited to:
 13.3$\cdot$10$^{20}$\,cm$^{-2}$, 3.1$\cdot$10$^{20}$\,cm$^{-2}$, 10.5$\cdot$10$^{20}$\,cm$^{-2}$, and 17.4$\cdot$10$^{20}$\,cm$^{-2}$ for 1ES\,0229+200, PKS\,0548-322, 1ES\,1741+196, and 1ES\,2344+514, respectively. 

 

\begin{figure*}
 \centering{\includegraphics[width=0.4\textwidth]{figures2/1es0229.png}}
  \centering{\includegraphics[width=0.4\textwidth]{figures2/pks0548.png}} \\
   \centering{\includegraphics[width=0.4\textwidth]{figures2/1es1741.png}}
  \centering{\includegraphics[width=0.4\textwidth]{figures2/1es2344.png}} \\
\caption[]{Comparison of model parameters: $\alpha$ (top panel), $\beta$ (middle panel), and UV excess (bottom panel) as a function of \nh\ value for 
1ES\,0229+200, PKS\,0548-322, 1ES\,1741+196, 1ES\,2344+514. }
\label{figure:parameters}
\end{figure*}




\noindent
