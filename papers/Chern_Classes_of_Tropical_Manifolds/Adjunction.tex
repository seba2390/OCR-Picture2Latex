
 \section{Adjunction formula}\label{sec:adjunction}
 
In this section we give a product formula for CSM cycles of matroids, and use it to prove a tropical analogue of the adjunction formula for smooth algebraic varieties. The tropical adjunction formula will require us to restrict to codimension-1 tropical submanifolds in the sense of Definition \ref{def:submanifold}. 

\subsection{Product formula for CSM cycles of matroids}
We start with the following result about intersections of CSM cycles of matroids. If $M$ is a loopless matroid on the ground set $\{0,\dots,n\}$ and $\Sigma_M$ is its corresponding matroidal tropical cycle in $\R^{n+1} / \, \mathbb R \cdot \mathbf{1}$, we consider the formal sum of tropical cycles
$$\csm(\Sigma_M) := \csm(M) := \sum_{i=0}^n \csm_i(M).$$

\begin{prop}\label{prop:intersectionMatroid}
Let $M$ and $M'$ be two loopless matroids on the ground set $\{0,\dots,n\}$, with matroidal tropical cycles $\Sigma_M$ and $\Sigma_{M'}$ in $\R^{n+1} / \, \mathbb R \cdot \mathbf{1} \cong \R^n$. Then
$$\csm(\Sigma_M) \cdot \csm(\Sigma_{M'})= \csm(\Sigma_{M} \cdot \Sigma_{M'}),$$
where the products denote tropical (stable) intersection in $\R^n$.
\end{prop}

\begin{proof}
Any matroidal tropical cycle can be obtained as the intersection of divisors corresponding to codimension-many tropical rational functions \cite{ShawInt}, \cite[Corollary 3.11]{FrancoisRau}. 
In other words, if $c$ and $c'$ are the codimensions of $\Sigma_M$ and $\Sigma_{M'}$ in $\R^n$, respectively, there are tropical rational functions $f_1, \dots, f_c$ and $g_1, \dots, g_{c'}$ on $\R^n$ such that
$$[\Sigma_M] = f_1 \cdot f_2 \cdot \dots \cdot f_c \cdot [\R^n] \quad \text{and} \quad  [\Sigma_{M'}] = g_1 \cdot g_2 \cdot \dots \cdot g_{c'} \cdot [\R^n].$$
This implies that $$[\Sigma_M \cdot \Sigma_{M'}] = f_1 \cdot \dots \cdot f_c \cdot g_1 \cdot \dots \cdot g_{c'} \cdot [\R^n].$$
Using the formula for CSM cycles of matroids in \cite[Proposition 2.3]{RauHopf},
we have 
$$\csm (\Sigma_M) = \prod_{i = 1}^c \frac{1}{1+ f_i} \cdot [\Sigma_M] \quad \text{and} \quad  \csm (\Sigma_{M'}) = \prod_{i = 1}^{c'} \frac{1}{1+ g_i} \cdot [\Sigma_{M'}],$$
and thus
$$\csm (\Sigma_M) \cdot \csm(\Sigma_{M'}) =  \prod_{i = 1}^c \frac{1}{1+ f_i} \prod_{i = 1}^{c'} \frac{1}{1+ g_i} \cdot [\Sigma_M \cdot \Sigma_{M'}] = \csm (\Sigma_M \cdot \Sigma_{M'}),$$
as claimed.
\end{proof}


\subsection{Adjunction in tropical manifolds}

The following theorem relates the CSM cycles of a codimension-$1$ tropical submanifold $D$ to the CSM cycles of the ambient tropical manifold $X$ using the tropical intersection theory in $X$.  
Notice that we do not require that $X$ be compact. 
To prove the theorem we restrict to codimension one tropical  submanifolds from Definition \ref{def:submanifold}.
Recall that Example \ref{ex:singline} presented tropical manifolds (lines) which are also codimension one  tropical subvarieties of tropical manifolds  which do  not satisfy the following theorem, due to the fact that they do not satisfy the compatibility condition to be a tropical submanifold. 


\begin{thm}[Adjunction formula]\label{thm:Adjunction}
Let $X$ be a tropical manifold  of dimension $d$ and $D \subset X$ a tropical submanifold of codimension $1$ in $X$. 
Then 
$$\csm_{d-2}(D) = (\csm_{d-1}(X) -  D )\cdot D,$$
where $\cdot$ denotes the tropical intersection product in $X$. 
\end{thm}


\begin{proof}
We assume the submanifold $D$ is connected, otherwise the statement can be proved on each connected component. 
We denote by $D^o$ (respectively, $X^o$) the strata of $D$ (respectively, $X$) of sedentarity $0$.
We first prove the case when $D$ is not a boundary divisor of $X$, in other words, $D$ is the closure in $X$ of $D^o$. 


The cycle $D$ is a Cartier divisor in $X$, meaning that there is a collection of tropical rational functions $\{f_{\alpha}\}$ such that $\varphi_\alpha(D \cap U_\alpha) = \divis_{X_{\alpha}}(f_{\alpha})$ for every chart $\varphi_\alpha$. Moreover, in the chart indexed by $\alpha$, the manifold $X$ itself is cut out by a collection of functions $g^{\alpha}_1, \dots, g^{\alpha}_{c_{\alpha}}$. Therefore, in each chart there are collections of functions  $g^{\alpha}_1, \dots, g^{\alpha}_{c_{\alpha}}$ cutting out  $X^o_{\alpha}$, and $f_{\alpha}, g^{\alpha}_1, \dots, g^{\alpha}_{c_{\alpha}}$ cutting out $D_{\alpha} := D \cap U_\alpha$.
Following the proof of Proposition \ref{prop:intersectionMatroid} and working locally in charts we have 
that
\begin{align*}
\csm(D^o_{\alpha}) &= \frac{1}{1+f_{\alpha}} \prod_{i= 1}^{c_\alpha}\frac{1}{1+g^{\alpha}_i} \cdot [D^o_{\alpha}]\\
&= (1-f_{\alpha}+f^2_{\alpha}- \dotsb ) \prod_{i= 1}^{c_\alpha}(1-g^{\alpha}_i+(g^{\alpha}_i)^2- \dotsb )  \cdot [D^o_{\alpha}].
\end{align*}
Looking only at the dimension-$(d-2)$ part of this equation we obtain, 
\begin{align*}
\csm_{d-2}(D^o_{\alpha}) &  = -(f_{\alpha} + g^{\alpha}_1+  \dots +  g^{\alpha}_{c_{\alpha}}) \cdot [D^o_{\alpha}] \\
& = -f_{\alpha} \cdot [D^o_{\alpha}] - (g^{\alpha}_1+  \dots +  g^{\alpha}_{c_{\alpha}}) \cdot [D^o_{\alpha}] \\
& = - (D_{\alpha}^o)^2  + \csm_{d-1}(X^o_{\alpha}) \cdot D^o_{\alpha}.  
\end{align*}
Therefore we can conclude that 
$$\csm_{d-2}(D^o)   = (\csm_{d-1}(X^o) -D^o)\cdot D^o.$$


Notice that, by definition of the boundary of tropical manifolds in terms of sedentarity, when $D$ is of sedentarity $0$ in $X$ we have $\partial D = \partial X \cap D$. 
Since $D$ is a submanifold of $X$ of sedentarity $0$, it is transverse to the  boundary $\partial X$ of $X$.
Moreover, $\partial X$ is a codimension-$1$ tropical cycle when equipped with weights equal to $1$,  
thus as tropical cycles we have 
$\partial D =  \partial X \cdot D$. 
We thus get
\begin{align*}
\csm_{d-2}(D) &= \csm_{d-2}(D^o) + \partial D \\
&= \csm_{d-2}(D^o) + \partial X  \cdot  D \\
&= (\csm_{d-1}(X^o) -D^o) \cdot D^o + \partial X  \cdot  D \\
&= (\csm_{d-1}(X^o) + \partial X)  \cdot  D - (D^o)^2 \\
&= \csm_{d-1}(X)  \cdot  D - D^2.
\end{align*}
The second to last equality holds since $\csm_{d-1}(X^o) \cdot D^o = \csm_{d-1}(X^o) \cdot D$ and $(D^o)^2 = D^2$ due to the transversality of $D$ with the boundary of $X$. This finishes the case when $D$ is not a  boundary divisor of $X$. 

Let $\partial X = \bigcup D_i$ be the decomposition of the boundary of $X$ into irreducible divisors. If $D$ is a boundary divisor of $X$ then
$$\csm_{d-1}(X) = \csm_{d-1}(X^o) + \partial X = \csm_{d-1}(X^o) + D + \sum_{D \neq D_i} D_i.$$   
Therefore
$$\csm_{d-1}(X) \cdot  D - D^2 = \csm_{d-1}(X^o) \cdot D + D \cdot   \sum_{D \neq D_i} D_i.$$
By Lemma \ref{lem:csmtransbdy} we have  
$\csm_{d-1}(X^o) \cdot D = \csm_{d-2}(D^o)$,  
where $D^o=D\setminus\bigcup_{D \neq D_i}D_i$. 
Moreover, the intersection 
$D\cdot \sum_{D \neq D_i} D_i =  \csm_{d-2}(\partial D) = \partial D$.
Combining all this we obtain
$$\csm_{d-1}(X) \cdot  D - D^2 = \csm_{d-2}(D^o) + \csm_{d-2}(\partial D) = \csm_{d-2}(D),$$
which proves the formula. 
\end{proof}