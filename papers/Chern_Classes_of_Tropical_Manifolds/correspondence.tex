
\section{Correspondence theorems} 
\label{sec:correspondence}

In this section we establish Theorem \ref{thm:Eulercharacteristic} relating the Euler characteristic of complex algebraic varieties to the degree of $\csm_0$ of their tropicalisation, in the case this tropicalisation is a tropical manifold. 

For $X$ a tropical manifold and $A$ a $0$-dimensional tropical cycle in $X$, let $m_x(A)$ denote the multiplicity of the point $x \in X$ in $A$. 
We denote $$\deg(A) := \sum_{x \in A} m_x(A).$$  
Note that in the case where $A = \csm_0(X)$, 
the sign of the multiplicity $m_x(\csm_0(X))$ depends only on the sedentarity of $x$, 
and thus if $X$ is of dimension $d$ and has only points of sedentarity $0$ 
then $(-1)^d\deg(\csm_0(X)) > 0$.  

 
Suppose  $\mathcal{X}$ is  a meromorphic family of subvarieties of a complex toric variety $\mathcal{Y}$ defined over the punctured disc $ \mathcal{D}^*$. 
 Let $Y$ denote the tropicalization of the toric variety $\mathcal{Y}$ and let $X = \trop(\mathcal{X})$ be the extended tropicalisation of $\mathcal{X} $ in $Y$  in the sense of Payne \cite{Payne}. 
Consider the fibre $\mathcal{X}_t := f^{-1}(t)$ for a generic $t \in \mathcal{D}^*$. 
The following theorem relates the Euler characteristic of $\mathcal{X}_t$ to $\csm_0(X)$ when $X$ is a tropical manifold. 


\begin{thm}
\label{thm:Eulercharacteristic}
 Let $\mathcal{X}$ be a meromorphic family of subvarieties of a complex toric variety $\mathcal{Y}$ over the punctured disk $\mathcal{D}^*$. Suppose that $X := \trop(\mathcal{X})$ is a tropical submanifold of $Y := \trop(\mathcal{Y})$, then 
  $$\chi(\mathcal{X}_t) = \deg(\csm_0(X)).$$
\end{thm}
 
\begin{proof}
Consider first the case when  $\mathcal{X}$ is a family of very affine subvarieties, namely is contained in $(\C^*)^N \times \mathcal{D^*}$. 
 Then $\trop(\mathcal{X}) = X \subset \R^N$. Let   ${\Vertices}(\hat{X})$ denote the vertex set of $X$ when equipped with some polyhedral structure $\hat{X}$. 
 By \cite[Corollary 1.4]{KatzStapledon} we have
 $$\chi(\mathcal{X}_t)  = \sum_{v \in {\Vertices}(\hat{X})} \chi(V_{\C}(\In_v \mathcal{X})),$$
 where $\In_v \mathcal{X}$ denotes the initial ideal of $\mathcal X$ at the vertex $v $ of $X$ and $V_{\C}(\In_v \mathcal{X})$ its variety over $\mathbb{C}$.
 
By \cite[Proposition 4.2]{KatzPayne}, since $X$ is a tropical manifold, the initial ideal $\In_v \mathcal{X} \subseteq \mathbb{C}[x_1^{\pm}, \dots, x_N^{\pm}]$ must be generated by linear forms up to toric coordinate change. This means that $V_{\C}(\In_v \mathcal{X})$ is the image under an invertible monomial map of the complement of a hyperplane arrangement. A monomial automorphism of $(\mathbb{C}^*)^n$ is a homeomorphism, and thus the cohomology groups and hence Euler characteristic of the complex points of $V_{\C}(\In_v \mathcal{X})$ are determined by the underlying matroid $M_v$; see  \cite{OrlikTerao}. 
 Namely, the Euler characteristic is up to sign the beta invariant of $M_v$. Therefore we obtain   
  $$\chi(\mathcal{X}_t)  = (-1)^d \sum_{v \in {\Vertices}(\hat X)} \beta(M_v) = 
  \sum_{v \in {\Vertices}(\hat X)} m_v(\csm_0(X)) = \sum_{x \in X} m_x(\csm_0(X)).$$ 
  The last equality holds since $m_x(\csm_0(X)) = 0$ if $x$ lies in a positive-dimensional face of $\hat X$. 
 This proves the statement when $\mathcal{Y}$ is a torus. 
  
Consider now the case where $\mathcal{X}$ is a meromorphic family in 
$\mathcal{Y} \times \mathcal{D}^*$ 
where $\mathcal{Y}$ is some complex toric variety. 
  Let $\Sigma$ be the fan defining $\mathcal{Y}$.  We will apply the above proof for subvarieties of $(\C^*)^N$ to the open strata $\mathcal{X}_{\sigma} := \mathcal{X} \cap (\mathcal{Y}_{\sigma} \times \mathcal{D}^*)$, where $\mathcal{Y}_{\sigma}$ is the open toric strata of $\mathcal{Y}$ corresponding to the cone  $\sigma $ of $\Sigma$. Let $X_{\sigma}$ denote the tropicalisation of $\mathcal{X}_{\sigma}$ in open toric strata
  $\mathcal{Y}_{\sigma}$ corresponding to $\sigma$. 
  By our assumption that $X = \trop(\mathcal{X})$ is a tropical submanifold of $Y = \trop(\mathcal{Y})$, our previous case implies that $\chi( \mathcal{X}_{\sigma,t}) = \deg(\csm_0(X_{\sigma}))$.
  
Compactly supported Euler characteristics are  additive over strata, namely 
$$\chi^c(\mathcal{X}_t) =   \sum_{\sigma \in \Sigma} \chi^c( \mathcal{X}_{\sigma,t}).$$
However, since each stratum is non-singular, Poincar\'e duality relates the usual and compactly supported cohomology groups. Therefore we obtain 
$\chi(V) = (-1)^{\dim_{\mathbb{R}} (V)} \chi^c(V) = \chi^c(V),$ since the real dimensions of the complex varieties are even. 
Hence we have  
$$\chi (\mathcal{X}_t) = \sum_{\sigma \in \Sigma}   \chi( \mathcal{X}_{\sigma,t}) = \sum_{x \in X}  m_x({\csm_0(X)}) = \deg(\csm_0(X)),$$
and the proof is complete. 
 \end{proof}
 
 
 We now recall the notion of recession cycle of a tropical cycle, from \cite[Section 5]{AHR}. 
 
 \begin{defi}\label{def:reccycle}
 Let $A$ be a $k$-dimensional tropical cycle in $\R^N$, with weight function $w_A$ on its top-dimensional faces. Its {\bf recession cycle} $\rec(A)$ is a $k$-dimensional fan tropical cycle in $\R^N$ whose support is the recession fan of $A$. The weight function on the $k$-dimensional cones of $\rec(A)$ is given by
 \begin{equation}\label{recweight}
 w_{\rec(A)}(\sigma) = \sum_{\substack{ \sigma' \in A \\ \rec(\sigma')  = \sigma}} w_A(\sigma').
 \end{equation}
 \end{defi} 
 By \cite[Theorem 5.3]{AHR}, the weighted fan $\rec(A)$ is a balanced tropical cycle, and it is tropically rationally equivalent to $A$.

 \begin{rem}\label{rem:EulerChar}
 Under the same notation as in the proof of Theorem \ref{thm:Eulercharacteristic}, if $X$ is a tropical manifold transverse to the boundary of $Y$, for $\sigma $ a cone of $\Sigma$ of dimension $k$ we have
  $$\chi( \mathcal{X}_{\sigma,t}) = \deg(\csm_0(X_{\sigma})) = \sum_{\substack{\alpha \text{ face of } \hat X \\ \dim \alpha = k \\ \rec(\alpha)  = \sigma} } m_{\alpha}(\csm_k(X)).$$
 Indeed, the first equality follows directly from Theorem \ref{thm:Eulercharacteristic} applied to the subvariety $\mathcal{X}_{\sigma}$. 
 Every vertex of $X_{\sigma}$ is contained in the closure of a unique face of $X$ of sedentarity $0$ and dimension $k$ whose recession fan is $\sigma$. 
Moreover, this provides a  bijection between the vertices in $X_{\sigma}$ and such faces of $X$.  
It follows from Lemma \ref{lem:csmtransbdy} that, under this bijection, the weight in $\csm_k(X)$ of  such a $k$-dimensional face is equal to the weight in $\csm_0(X_{\sigma})$ of the corresponding vertex. This proves the above equality. 
 \end{rem}

 


The next theorem recovers and generalises the correspondence theorem for CSM cycles of matroids and wonderful compactifications proved in  \cite[Theorem 3.1]{LdMRS}. 
Here we consider a subvariety of a toric variety and prove that the CSM class of the subvariety intersected with the big open torus, considered in the Chow ring of the ambient toric variety, is determined by the tropical CSM class. 
Our proof relies heavily on the results and set-up previously established in \cite{Esterov} for tropicalisations of constant families. 


The cohomology ring of a non-singular complete toric variety $\mathcal{Y}$ defined by a fan $\Sigma$ is isomorphic to the ring of Minkowski weights on $\Sigma$.  Let $N$ be the dimension of $\mathcal{Y}$, and let $\MW_{k}(\Sigma)$ denote the group of $k$-dimensional Minkowski weights supported on $\Sigma$. 
Then by \cite{FultonSturmfels}, we have $ \MW_{k}(\Sigma)\cong \Ac^{N-k}(\mathcal{Y})$, where $\Ac^{N-k}(\mathcal{Y})$ is the $(N-k)$-th graded piece of the Chow cohomology  ring of $\mathcal{Y}$. 
Since $\mathcal{Y}$ is non-singular and complete, by Poincar\'e duality the cap product with the fundamental class $[\mathcal{Y}]$ provides an isomorphism between Chow homology and cohomology: 
$$\frown [\mathcal{Y}] \colon \MW_k(\Sigma) \xrightarrow{\,\cong\,} \Ac_k(\mathcal{Y}).$$
 
Let   $\mathcal{V}$ be a subvariety of the torus $(\C^*)^N$ and $\mathcal{Y}$ a toric compactification of $(\C^*)^N $ such that  $\overline{\mathcal{V}}$ is transverse to all of the toric  orbits of $\mathcal{Y}$.
Esterov showed that the tropical characteristic classes defined using Euler characteristics recover the Poincar\'e duals of the CSM classes of $\mathcal{V}$ \cite[Theorem 2.39]{Esterov}.
We generalise this result to the tropicalisation of families in the next theorem.  
   
\begin{thm}
\label{thm:corrToric}
Suppose $\mathcal{X}$ is a meromorphic family of subvarieties  of $(\C^*)^N$.   Let $\mathcal{Y}$ be a non-singular projective toric variety 
with defining fan $\Sigma$ in $\R^N$ and $\overline{\mathcal{X}}$ be the closure of $\mathcal{X}$ in $\mathcal{Y}$. 
Suppose that  the closure $\overline{X} = Trop(\overline{\mathcal{X}})$ is  a tropical manifold transverse to the boundary of $Trop(\mathcal{Y})$. Then
$$\csm_k( \mathbbm{1}_{\mathcal{X}_t}) = \rec(\csm_{k}(X)) \frown [\mathcal{Y}] 
 \in \Ac_{k}(\mathcal{Y}).$$
\end{thm}


\begin{proof}
Since $X$ is a tropical manifold and is transverse to the boundary of $\trop(\mathcal{Y})$, all of the open strata  $ \mathcal{X}_{\sigma,t}$ are  sch\"on by \cite[Proposition 7.10]{KatzStapledon}. 
Remark \ref{rem:EulerChar} and  Equation \ref{recweight} imply that for $\sigma$ a $k$-dimensional cone in the recession fan of $X$,  the weight of $\sigma $ in $\rec(\csm_{k}(X))$ is equal to  the Euler characteristic of $ \mathcal{X}_{\sigma,t}$.
 Therefore, the statement follows after applying \cite[Theorem 2.39 (1)]{Esterov}. 
\end{proof}

