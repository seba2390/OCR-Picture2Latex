% !TEX root = main.tex


\section{Tropical Noether's Formula}\label{sec:Noether}

In this section we restrict to compact tropical manifolds of dimension $2$, which we call compact tropical surfaces. 
In order to prove Noether's formula for a compact tropical surface $X$,
we will assume the existence of a cellular structure $\hat{X}$ on $X$ satisfying the following properties. 

\begin{defi}\label{def:facestructure}
Let $X$ be a tropical manifold. A cellular structure $\hat{X}$ supported on $X$ is a {\bf face structure} on $X$ if 
for each cell $\sigma$ of $\hat{X}$ there exists 
a chart $\varphi: U_\alpha \to \T^{r_\alpha} \times \R^{n_\alpha}$ of $X$ such that $U_\alpha \supset \sigma$ and $\varphi(\sigma)$ is the closure of a polyhedron in $\R^{r_\alpha} \times \R^{n_\alpha}$ which is transverse to the boundary of $\T^{r_\alpha} \times \R^{n_\alpha}$. 
The face structure $\hat{X}$ is called {\bf rational} if all these polyhedra are $\RR$-rational polyhedra, i.e., their facets have normal vectors with entries in $\QQ$. 

A rational face structure $\hat X$ on a $2$-dimensional tropical manifold is called {\bf Delzant} if for any vertex $v$ of $\hat{X}$ of sedentarity 0 and any $2$-dimensional face $F$ of $\hat{X}$ with $v \in F$ we have that, in any chart of $X$ containing $v$, the primitive integer vectors $\bfw_v(e)$ and $\bfw_v(e')$ of the two edges $e, e'$ of $F$ containing $v$ can be completed to a basis of the ambient lattice. 
\end{defi}

\begin{rem}
The notion of a face structure on a tropical manifold has previously appeared in \cite{JRS}. 
We do not know exactly which tropical manifolds admit face structures, rational face structures, or Delzant face structures. 
However, given a compact tropical surface with a rational face structure $\hat{X}$,  we can use tropical wave front propagation from \cite{MikShk} on each $2$-dimensional  face of $\hat X$ to pass to a finer face structure. This face structure is not necessarily Delzant, but it has the property that for each $2$-dimensional face $F$ and each vertex $v$ of $F$, the outgoing primitive normal vectors to the two edges adjacent to $v$ in $F$ span a triangle with no interior lattice points; see  \cite{MikShk}. 
\end{rem}

We denote the topological Euler characteristic of a tropical manifold $X$ by $\chi(X)$.
Our main goal in this section is to prove the following theorem.  

\begin{thm}[Tropical Noether's Formula]
\label{thm:Noether}
Let $X$ be a compact tropical surface admitting a Delzant face structure.
Then 
$$\chi(X) = \frac{\deg (\csm_0(X) + \csm_1(X)^2) }{12}.$$
\end{thm}




In order to prove Noether's Formula, we first recall formulas for the intersection numbers of $1$-cycles in a compact tropical surface. In a  compact tropical surface, there are  $1$-cycles of sedentarity $0$ and boundary $1$-cycles. A boundary $1$-cycle  is a linear combination of boundary divisors of $X$; see Subsection \ref{sec:boundary}. 


\begin{prop}\cite[Section 3.5]{Shaw:Surf}\label{prop:intersections}
Let $X$ be a compact tropical surface and $A, B$ two $1$-cycles in $X$. In the following three cases, the intersection of $A$ and $B$ is defined on the cycle level, as described below. 
\begin{enumerate}
\item If  $A, B$ are both of sedentarity $0$ and transverse to the boundary of $X$, then the intersection $A \cdot B$ is supported on points of sedentarity $0$. Namely, we can write $A\cdot B =  \sum_{x \in (A \cap B)_{0}} m_x(A \cdot B) \, x $. 
\item If $A$ is of sedentarity $0$ and transverse to the boundary of $X$, and $B$
 is an irreducible boundary divisor then $A\cdot B = \sum_{ x\in A \cap B} m_x(A \cdot B) \,x$, where the multiplicity $m_x$ is equal to the weight $w_A(e)$ of the the unique edge $e$ of $A$ containing $x$ in its boundary. 
\item If $A$ and $B$ are distinct irreducible boundary divisors of $X$ then
$A \cdot B = \sum_{x \in A \cap B} x$. 
\end{enumerate}
\end{prop}

We now fix some notation for the rest of this section.
Suppose $X$ is a tropical manifold of dimension $d$. 
For $i = 0,1,\dots, d$, let $X_i$ be the subset of $X$ consisting of points of sedentarity $i$. We let $\hat{X}_i$ denote the subset of cells of $\hat{X}$ whose relative interiors are of sedentarity $i$. 
A cell $\sigma$ of $\hat{X}_0$ is called {\bf bounded} if its closure $\bar{\sigma}$ is contained in $X_0$, and otherwise it is called {\bf unbounded}.

For any polyhedral complex $Y$, we will denote its set of vertices, edges, and $2$-dimensional faces by $\Vertices(Y)$, $\Edges(Y)$, and $\Faces(Y)$, respectively.
Moreover, if $v$ a vertex of $Y$, we denote by $\Edges(v,Y)$ the set of edges of $Y$ containing $v$, and by $\Faces(v,Y)$ the set of $2$-dimensional faces of $Y$ containing $v$.
Similarly, if $e$ is an edge of $Y$, we denote by $\Faces(e,Y)$ the set of $2$-dimensional faces of $Y$ containing $e$.


Let $X$ be a compact tropical surface admitting a Delzant face structure $\hat X$.
Suppose $e$ is an edge of $\hat X$ of sedentarity $0$ and $v$ is a vertex of $e$ of sedentarity $0$.
Fix a chart of $X$ containing the edge $e$.
Since $X$ satisfies the balancing condition with weights equal to $1$ around the edge $e$, there exists an integer $\sigma_v(e)$ such that
\begin{equation}\label{eqn:sigma} 
- \sigma_v(e) \, \bfw_{v}(e) = \sum_{\substack{e' \in {\Edges}(v, \hat X) \\ e' \neq e \text{ are in some face } F \in {\Faces}(v, \hat X)}} \bfw_{v}(e'),
\end{equation}
where $\bfw_{v}(e)$ denotes the primitive integer vector in the direction of $e$ pointing outwards from the vertex $v$. 
Note that the integer $\sigma_v(e)$ is independent of the chart chosen for $X$.


Our strategy for proving Noether's Formula begins by analysing the tropical $0$-cycle 
$\csm_0(X) + \csm_1(X)^2 - \sum_{D \in \partial X} D^2$.
We use the intersection multiplicities in Proposition \ref{prop:intersections} 
to obtain the following presentation of it, supported only at the vertices of $\hat X$.

\begin{lemma}\label{lem:localcontribution}
Let $X$ be a compact tropical surface with a Delzant face structure $\hat{X}$. Let $D_1, \dots, D_k$ denote the irreducible boundary divisors of $X$. Then for any vertex $v \in \hat{X}$ we have 
$$\csm_0(X) + \csm_1(X)^2 - \sum_{i = 1}^k D_i^2 = \sum_{v \in \Vertices(\hat X)} m_v \cdot v$$
with
$$
m_v = 12 - 6|{\Edges}(v, \hat{X})| + 3|{\Faces}(v, \hat{X})| - \delta(v) \cdot \sum_{e \in {\Edges}(v, \hat{X})} \sigma_v(e),
$$
where $\delta(v) = 1$ if $v$ is of sedentarity $0$, and $\delta(v)= 0$ otherwise.
\end{lemma}



\begin{proof}
The cycle $\csm_0(X)$ is supported on the vertices of $\hat{X}$. 
The cycle $\csm_1(X)$ can be written as 
$\csm_1(X) = \csm_1(X_0) + \csm_1(X_1) = \csm_1(X_0) + \sum_{i=1}^k D_i.$ 
Therefore, we have 
\begin{equation}\label{eqn:csm1sq}
\csm_1(X)^2 -  \sum_{i=0}^k D_i^2 = \csm_1(X_0)^2 + 2 \csm_1(X_0) \csm_1(X_1)  + 2\sum_{i < j} D_i D_j. 
\end{equation}

First suppose $v$ is a vertex in $\hat{X}_2$. 
In this case, the multiplicity of $v$ in $\csm_0(X)$ is equal to 1. 
Among the terms on the right-hand side of Equation \eqref{eqn:csm1sq}, the vertex  $v$ only appears with non-zero multiplicity in $2\sum_{i < j} D_i D_j$. Since $v$ is the intersection of two unique boundary divisors $D_i$ and $D_j$ of $X$, we thus have 
$m_v = 3$. 
Moreover, $|{\Edges}(v, \hat{X})| = 2$  and $|{\Faces}(v, \hat{X})|  = 1$. Therefore, in this case we have 
$m_v = 3 = 12 - 6|{\Edges}(v, \hat{X})| + 3|{\Faces}(v, \hat{X})|$,
and the claim holds.

Suppose now $v$ is a vertex in $\hat{X}_1$. The multiplicity of $v$ in $\csm_0(X)$
is equal to $2 - |{\Faces}(v, \hat{X})|$.
The only term on the right-hand side of 
Equation \eqref{eqn:csm1sq} with non-zero multiplicity in $v$ is $2 \csm_1(X_0) \csm_1(X_1)$. Moreover, the cycle  $ \csm_1(X_1)$ is of weight 1 everywhere, and the weight of $ \csm_1(X_0)$ along the unique edge $e$ of $\hat{X}_0$ whose boundary is $v$  is
$2 - |{\Faces}(v, \hat{X})|$ by Lemma \ref{lem:csmtransbdy}.  
Therefore, in this case $m_v = 3(2 - |{\Faces}(v, \hat{X})|)$.
Since each vertex $v \in \hat{X}_1$ is adjacent to exactly one edge in $X_0$ and satisfies $|{\Edges}(v,\hat{X})| = |{\Faces}(v,\hat{X})| + 1$, we can rewrite this as 
$m_v = 3(2 - |{\Faces}(v, \hat{X})|) = 12 -6|{\Edges}(v,\hat{X})| + 3|{\Faces}(v,\hat{X})|$,
and the claim holds.

Lastly, suppose $v \in \hat{X}_0$. The only term on the right-hand side of Equation \eqref{eqn:csm1sq} with non-zero multiplicity in $v$ is
$\csm_1(X_0)^2$. 
Using \cite[Proposition 3.18]{Shaw:Surf}, the term $\csm_1(X_0)^2$ can be expressed as a cycle supported only on the vertices of $\hat X_0$, where a vertex $v \in \hat X_0$ has multiplicity
\begin{equation*}
m_v(\csm_1(X_0)^2) = 10 + N_v - 5|{\Edges}(v, \hat{X})| + 2|{\Faces}(v, \hat{X}) | + \sum_{ e \in {\Edges}(v, \hat{X})} -\sigma_v(e)
\end{equation*}
with $N_v$ denoting the dimension of the affine span of the local fan $\starr_{\hat X}(v)$.

For a vertex $v$ of $\hat{X}_0$, denote by $\Fs_p(v)$ the vector space $\Fs_p( \starr_{\hat X}(v))$ from Definition \ref{def:Fp}.  
By Lemma \ref{lem:sameweights}, the multiplicity of a vertex $v \in \hat X_0$ in the term $\csm_0(X_0)$ is 
$$m_v(\csm_0(X_0)) = \dim \Fs_0(v) - \dim \Fs_1(v) + \dim \Fs_2(v).$$
We have $\dim \Fs_0(v) = 1$ and $\dim \Fs_1(v) = N_v$.
The fan $\Sigma := \starr_{\hat X}(v)$ is matroidal, thus by \cite[Theorem 5.3]{JRS},  it satisfies Poincar\'e duality for tropical homology with $\mathbb{Z}$ and hence $\mathbb{Q}$ coefficients. We refer the reader to \cite[Section 5]{JRS} and Equations 5.1 and 5.3 there for the definitions of the chain complexes computing these groups.  Tropical Poincar\'e duality implies that  
 $H^q(\Sigma; \Fs^p) \cong H^{BM}_{2 - q}(\Sigma; \Fs_{2 - p})$
for all $0 \leq p, q \leq 2$, where the left-hand side denotes tropical cohomology and the right-hand side tropical Borel-Moore homology.  
From their definitions, we have $H^q(\Sigma; \Fs^p) = 0$ for $q \neq 0$ and $\dim H^0(\Sigma; \Fs^p) = \dim \Fs_p(v)$. 
Since  $H^{BM}_{2 - q}(\Sigma; \Fs_{0}) = 0 $ for $q \neq 0$, we can compute its dimension using Euler characteristics.  Notice also that $H^{BM}_{2 }(\Sigma; \Fs_{0}) =H^{BM}_{2 }(\Sigma; \R)$ and that $\dim C^{BM}_q(\Sigma, \R)$ is equal to the number of $q$-dimensional faces of the fan $\Sigma$. 
It follows that 
\begin{align*}
\dim \Fs_2(v) &= \dim H^{BM}_2(\Sigma; \Fs_0) \\
 & = \dim C^{BM}_0(\Sigma, \R)  - \dim C^{BM}_1(\Sigma, \R) + \dim C^{BM}_2(\Sigma, \R) \\
  & = 1 - |\Edges(v, \hat{X})| +  |\Faces(v, \hat{X})|.
\end{align*}
We conclude that the multiplicity of the vertex $v \in \hat X_0$ in the cycle $\csm_0(X_0)$ is equal to  
$$m_v(\csm_0(X))  = 2 - N_v - |{\Edges}(v,\hat{X})| + |{\Faces}(v,\hat{X})|.$$ 
Combining this with the contribution of $m_v(\csm_1(X_0)^2)$ from above, we obtain
$$m_v = 12 - 6|{\Edges}(v, \hat{X})| + 3|{\Faces}(v, \hat{X})| - 
\sum_{e \in {\Edges}(v, \hat{X})} \sigma_v(e),$$
and the claim is proved for all vertices of $\hat{X}$. 
\end{proof}


Following  Lemma \ref{lem:localcontribution}, it remains to take the self-intersection of the boundary divisors of $X$ into account. 
If $D$ is an irreducible  boundary divisor of a  tropical surface $X$, then the self-intersection $D^2$ is only defined up to  rational, homological, or numerical equivalence. 
We are only interested in the self-intersection number, so any of these equivalences will do upon taking the degree. We abuse notation and let  $D^2$ denote the degree of the self-intersection. This degree can be computed in a variety of ways. Here we will do this by finding a section of the normal bundle of $D$ in $X$ and computing its degree \cite{Shaw:Surf}. The section we will use will be {\bf Delzant} in the sense of Definition \ref{def:admissible}. This will not only help us to compute the degree of the normal bundle, but it will be used to construct smooth toric surfaces. 

We briefly review the theory of line bundles, sections, and divisors on tropical curves. 
A tropical curve  $C$ is a tropical manifold of dimension $1$. Here we will assume that $C$ is compact. 
Each boundary divisor $D$ of a compact tropical surface $X$ is a compact tropical curve; an atlas for $D$ is obtained by restricting the atlas for $X$. Moreover, any face structure $\hat X$ on $X$ induces a face structure on $D$.

A line bundle on a tropical curve $C$ is a $2$-dimensional tropical surface $L$ together with a map $\pi : L \to C$ such that for any point $p \in C$ we have $\pi^{-1}(p) = \T$ and furthermore, there exist local trivialisations: for any point $p \in C$ there exists a neighbourhood $U_p$ of $p$ such that  $\pi^{-1}(U_{p}) = U_{p} \times \T$.  If $\{U_{\alpha}, \phi_{\alpha}\}$ is a covering such that $\pi^{-1}(U_{\alpha}) = U_{\alpha} \times \T$ then the line bundle $L$ can be specified via integer affine transition functions $f_{\alpha \beta} : U_{\alpha} \cap U_{\beta} \to \R$ which obey the cocycle condition. 
Let $\mathcal{O}^*$ denote the sheaf of tropical invertible regular functions on $C$. Then line bundles are in correspondence with $H^1(C ; \mathcal{O}^*)$ \cite{MikZha:Jac}. 
The normal bundle $N_X(D)$ of a boundary divisor $D$ in a tropical surface $X$ is the tropical line bundle on $D$ where the transition functions $f_{\alpha \beta} : U_{\alpha} \cap U_{\beta} \to \R$ are inherited from the transition functions from the charts of $X$ in a neighbourhood of $D$ \cite[Section 3.5.4]{Shaw:Surf}.

A section of a line bundle is a continuous function $s : C \to L$ such that $\pi(s(p)) = p$ for all $p \in C$ and in every trivialisation $U \subset C$ with $\pi^{-1}(U) = U \times \T$ the function $s|_U$ is a tropical rational function $U \to \T$ which is bounded. In particular, in a neighbourhood of  each point $p$ of $C$, the section $s$ is given by a piecewise integer affine  function. 
For each $p \in C$ of sedentarity $0$,  we can consider the collection of outgoing primitive integer tangent vectors of $C$ at $p$, call these $\bar{{\bf w}}_p(e_1), \dots, \bar{{\bf w}}_p(e_{{\rm val}(p)})$, where the $e_i$ denote the edges of $C$ adjacent to $p$ and  ${\rm val}(p)$ denotes the valency of $p$ in the underlying graph of $C$. 
By the balancing condition for $C$ at $p$ we have  that $\sum_{i = 1}^{{\rm val}(p)} \bar{{\bf w}}_p(e_i) = 0$. 
Let $\frac{\partial s}{\partial  \bar{{\bf w}}_p(e_i)}(p)$ denote the slope of the integer affine function at $p$ in the direction of $ \bar{{\bf w}}_p(e_i)$. The order of vanishing $m_p(s)$ of $s$ along $p$ is the order of vanishing of the tropical rational function determining $s$ in a local trivialisation at $p$; more specifically, $$m_p(s) = \sum_{i = 1}^{{\rm val}(p)}\frac{\partial s}{\partial \bar{{\bf w}}_p(e_i)} (p).$$
The {\bf  divisor associated to the section} $s$ is $(s) := \sum_{p \in C} m_p(s) p$. 

At a point $p$ of $C$ we will also consider the primitive outgoing tangent vectors to the graph of the section $s$ at the point $s(p)$; namely, the vectors ${{\bf t}}_p(e_1), \dots, {{\bf t}}_p(e_{\rm val(p)})$
given by 
\begin{equation}\label{eq:tangentvectors}
{{\bf t}}_p(e_i) = \left( \bar{{\bf w}}_p(e_i),  \frac{\partial s}{\partial  \bar{{\bf w}}_p(e_i)}(p)\right).
\end{equation}

There is an equivalence between divisors on a tropical curve $C$ and tropical line bundles on $C$ together with a section (up to adding scalars) \cite[Proposition 4.6]{MikZha:Jac}.
Two sections  of the same line bundle that differ only by adding a constant yield the same divisor.

A global tropical rational function on a curve $C$ is a function $h : C \to \T$ such that in each chart of $C$ the function $h$ can be expressed as the difference of two tropical polynomials. We can think of a tropical rational function $h$ as a section of the trivial line bundle $h :C \to \T$.  A divisor is {\bf principal} if it arises from a global tropical rational function $h: C \to \T$ considered as such a section. Rationally equivalent divisors differ by a principal divisor and thus have the same degree. 

If $D$ is a divisor in a compact tropical surface $X$ and $s_D$ is a section of its normal bundle, then by \cite[Section 3.5.4]{Shaw:Surf} we have 
\begin{equation}\label{eq:boundaryselfintersection}
D^2 = \deg((s_D)) = \sum_{p \in D} m_p(s_D).
\end{equation}
To obtain the degree of the self-intersection of a boundary divisor $D$, any section $s_D$ of the normal bundle will do. However, for our proof of Noether's Formula, we will require the notion of a Delzant section. 

\begin{defi}\label{def:admissible}
Let $C$ be a tropical curve with  a face structure $\hat{C}$, and let $L$ be a tropical line bundle on $C$. A section $s : C \to L $ is  {\bf Delzant} if its associated divisor $Q = (s)$ is  supported on the bounded edges of $\hat C$, and for every bounded edge $e$ of $\hat{C}$ there is at most one point $p \in {\rm relint}(e) $ such that $p \in {\rm Supp}(Q)$ and if this point exists we have  $m_p(Q) = 1$. 
\end{defi}


We call such sections Delzant as for every bounded edge $e$ of $\hat C$ and every point $p \in {\rm relint}(e)$, the outgoing tangent vectors to the graph of $s$ are either parallel (in the case $m_p(Q) = 0$) or they are unimodular (in the case $m_p(Q) = 1$). 
 In particular, if $s$ is a Delzant  section, any points with negative multiplicity in the associated divisor $Q = (s)$ must be vertices of $\hat{C}$. 


\begin{lemma}\label{lem:admissiblesection} 
For any tropical curve $C$ with face structure $\hat{C}$ and any line bundle $L$ on $C$ with local trivialisations over the edges of $\hat{C}$ there exists a Delzant section $s : C \to L$. 
\end{lemma}

\begin{proof}
By \cite[Proposition 4.6]{MikZha:Jac}, there exists a section $\tilde{s}: C \to L$ of $L$. Denote its associated divisor by $\tilde{Q}$. 
Subtracting any principal divisor  $P$ from $\tilde{Q}$ provides a linearly equivalent divisor and thus another section of $L$. 

We now construct for every edge $e$ of $\hat C$ a certain tropical rational function $h_e : C \to \T$ which is constant outside of $e$. 
For $e$ an unbounded edge of $C$, let $v$ denote the unique vertex of $\hat{C}$ which is of sedentarity $0$. Then we define  $h_e(p) = - \tilde{s}(p) $  if $p $ is a point of $e$, and $h_e(p) = -\tilde{s}(v) $ otherwise. Then $h_e : C \to \T$ is a tropical rational function,  and its associated divisor $(h_e)$ satisfies $m_p((h_e)) = - m_p(\tilde{Q})$ if $p \in {\rm relint}(e)$ or if $p$ is the vertex of $e$ of sedentarity 1, and $m_p((h_e)) = 0$ for all $p \notin e$. 

For $e$ a bounded edge of $\hat{C}$, let $v_1$ and $v_2$ denote its two vertices. 
 If $\tilde{s}(v_1) = \tilde{s}(v_2)$, we   proceed as above and define $h_e(p) = - \tilde{s}(p) $  if $p $ is a point of $e$ and $h_e(p) = -\tilde{s}(v_i)$ otherwise. 
Then $h_e: C \to \T$ is a tropical rational function satisfying $m_p((h_e)) = - m_p(\tilde{Q})$ for all $p \in {\rm relint}(e)$, and $m_p((h_e)) = 0$ for all $p \notin e$. 

Suppose that   $\tilde{s}(v_1) \neq  \tilde{s}(v_2)$. 
Upon taking a chart we can suppose that $v_1, v_2 \in \R$ and that $v_1 < v_2$. Let $$m = \Bigl \lfloor \frac{\tilde{s}(v_2) -\tilde{s}(v_1)}{v_2-v_1} \Bigr\rfloor \quad \text{and} \quad m+1 = -\Bigl\lceil \frac{\tilde{s}(v_2) -\tilde{s}(v_1)}{v_2-v_1} \Bigr\rceil.$$
Let $f_e : \R \to \R$ be the  tropical polynomial given by the piecewise integer affine  function 
$$f_e(x) = \max \{ m(x- v_1) + \tilde{s}(v_1), (m+1)(x-v_2) + \tilde{s}(v_2)\}. $$ 
It can be checked that there is a single zero $p_0$ of $f_e$ in $\R$ and it satisfies $v_1 < p_0 <v_2$. Moreover, the tropical polynomial $f_e$ satisfies $f_e(v_i) = \tilde{s}(v_i)$. Then we can cook up a tropical rational function 
$h_e : C \to L$ by setting $h_e(p) = - \tilde{s}(p) + f_e(p)$ for $p \in e$ and $h_e(p) = 0$ otherwise. Notice that $h_e$ is a tropical rational function and that $m_p((h_e)) = - m_p(\tilde{Q})$ for all $p  \in {\rm relint}(e)$ except $p = p_0$, and $m_p((h_e)) = 0$ for all $p \notin e$.  

By construction, the divisor of the  section  $s = \tilde{s} + \sum_{e \in \Edges(C)} h_e$ is Delzant  in the sense of Definition \ref{def:admissible}, and the lemma is proven. 
\end{proof}



Returning to our proof of Noether's Formula, for each irreducible boundary divisor $D$ of $X$ we fix a Delzant section $s_D$.
Using the Delzant face structure $\hat X$ on $X$ and the choice of Delzant sections $s_D$, 
we will associate to each face $F$ of $\hat X$ a $2$-dimensional rational complete unimodular fan $\Sigma_F$, called the {\rm face fan} of $F$.
  
First, suppose $P$ is a $2$-dimensional $\RR$-rational convex polyhedron in $\R^n$ equipped with an orientation. Let $T(P) \cong \R^2$ denote its tangent space. We define the {\bf edge fan}  $\Sigma_P$ of $P$ to be the  rational polyhedral fan in $T(P)$ whose rays are generated by the (oriented) primitive integer directions $\bfw_P(e)$ of the edges $e$ of $P$ and whose $2$-dimensional cones correspond to the vertices of $P$. See Figure \ref{fig:boundedSigmaF}. 
Notice that choosing another orientation of $P$ simply produces the negative fan to ${\Sigma}_P$. The edge fan ${\Sigma}_P$ is complete if and only if $P$ is compact, and ${\Sigma}_P$ is unimodular if and only if $P$ is Delzant. 

\begin{figure}
\includegraphics[scale=0.7]{boundedSigmaF} 
\caption{An bounded face $F$ of $\hat{X}$ and its corresponding fan $\Sigma_F$.}
\label{fig:boundedSigmaF}
\end{figure}

\begin{defi}\label{def:facefan}
Let $X$ be a compact tropical surface with a Delzant face structure $\hat{X}$, 
and fix a Delzant  section $s_D$ for each boundary divisor $D$ of $X$; see Lemma \ref{lem:admissiblesection}. 
For every $2$-dimensional face $F$ of $\hat{X}$, we define the {\bf face fan} $\Sigma_F$ of $F$,
which is a $2$-dimensional rational complete unimodular fan. As in Definition \ref{def:facestructure}, choose a chart $\varphi_\alpha$ under which $F$ is the closure of a $2$-dimensional polyhedron $P_F$ in $\R^{n_\alpha} \times \R^{r_\alpha}$ transverse to the boundary of $\T^{n_\alpha} \times \R^{r_\alpha}$, and fix an orientation of $P_F$.

\begin{itemize}
\item If $F$ contains only points of sedentarity 0, then the face fan $\Sigma_F$ is equal to the edge fan $\Sigma_{P_F}$. See Figure \ref{fig:boundedSigmaF}.

\item If $F$ contains points of sedentarity 1 and no points of sedentarity 2,  
the face $F$ contains a unique edge $e$ of sedentarity $1$. The edge $e$ is in some boundary divisor $D$, and has endpoints $v_1, v_2 \in \Vertices(\hat X_1)$.  Suppose the orientation of $F$ induces an orientation from $v_1 $ to $v_2$ on $e$. 
Fix a local trivialisation of the normal bundle to $D$ over the edge $e$.
Recall that ${\bf t}_{v_i}(e)$ denotes the primitive outward tangent vector to the graph of $s_D$ over $e$ at $v_i$.
We complete the edge fan $\Sigma_{P_F}$ to the face fan $\Sigma_F$ by adding the rays generated by the 
tangent vectors ${\bf t}_{v_1}(e)$ and $- {\bf t}_{v_2}(e)$, 
and adding the missing $2$-dimensional cones spanned by adjacent rays. 
If ${\bf t}_{v_1}(e) = -{\bf t}_{v_2}(e)$, i.e., if the section
$s_D$ has no zero on $e$, then only one ray is added. 
The fact that $s_D$ is a Delzant section implies that the fan $\Sigma_F$  is unimodular.
See Figure~\ref{fig:sed1SigmaF}.
\begin{figure}
\includegraphics[scale=0.7]{sed1SigmaF} 
\caption{An unbounded face $F$ of $\hat{X}$ with no points of sedentarity 2, the corresponding part of the Delzant section $s_D$ of the boundary divisor $D$, and the resulting fan $\Sigma_F$ associated to $F$. }
\label{fig:sed1SigmaF}
\end{figure}

\item If $F$ contains a point of sedentarity 2, the support of the edge fan ${\Sigma}_{P_F}$ is 
equal to the cone spanned by the (oriented) primitive integer directions of the two unbounded rays of $P_F$.
We construct the face fan $\Sigma_F$ from $\Sigma_{P_F}$ by adding the rays spanned by the negative of these two integer directions, and adding the three $2$-dimensional cones spanned by adjacent rays.
Again, $\Sigma_F$ is a unimodular fan.
See Figure \ref{fig:sed2SigmaF}.
\begin{figure}[b]
\includegraphics[scale=0.7]{sed2SigmaF} 
\caption{An unbounded face $F$ of $\hat{X}$ with a point of sedentarity 2 and its corresponding fan $\Sigma_F$.}
\label{fig:sed2SigmaF}
\end{figure}

\end{itemize}
Notice that $\Sigma_F$ is defined up to a choice of orientation and chart for $F$. However,  different choices yield a face fan which is ${\rm GL}_2(\Z)$-equivalent. 
\end{defi}



We now relate the geometry of the face fans $\Sigma_F$ to the integers $\sigma_v(e)$ defined in Equation \eqref{eqn:sigma}. Suppose $e$ is an edge of $\hat{X}$ of sedentarity $0$ and $F$ a $2$-dimensional face of $\hat X$ containing $e$. 
Let $\rho_e$ be the ray in the face fan $\Sigma_F$ generated by the (oriented) primitive integer vector $\bfw_F(e)$ of $e$. 
If $\rho$ and $\rho'$ are the two rays in $\Sigma_F$ adjacent to $\rho_e$, denote by ${\bf w}_{\rho'}$ and ${\bf w}_{\rho''}$ their primitive integer vectors. 
Define the integer $\tau_F(e)$ by 
 \begin{equation} \label{eqn:tau} 
 - \tau_F(e) {\bf w}_F(e) = {\bf w}_{\rho'} + {\bf w}_{\rho'}.
\end{equation}
Such an equation always holds for a unique integer $ \tau_F(e)$ since the cones adjacent to the ray $\rho_e$ in $\Sigma_F$ are unimodular \cite[Section 2.5]{Fulton}.


\begin{lemma}\label{lem:ToricInt}
Let $X$ be a compact tropical surface with a Delzant face structure $\hat{X}$. 
If $e$ is a bounded edge of $\hat{X}$ with vertices $v_1, v_2 \in \Vertices(\hat X_0)$, we have 
$$\sigma_{v_1}(e) + \sigma_{v_2}(e)  = - \sum_{\substack{F \in {\Faces}(\hat{X}) \\ F \supset e}} \tau_F(e).$$
If $e$ is an unbounded edge of $\hat{X}$ with vertex $v$ of sedentarity $0$ and vertex $\bar{v}$ 
of sedentarity $1$, we have  
$$\sigma_{v}(e) - m_{\bar{v}}(s_D)  = - \sum_{\substack{F \in {\Faces}(\hat{X}) \\ F \supset e}} \tau_F(e),$$ 
where $D$ is the boundary divisor of $X$ containing $\bar v$ and $s_D$ is the corresponding Delzant section used to construct the face fans in Definition \ref{def:facefan}. 
\end{lemma}

\begin{proof} 
We begin with the case of a bounded edge $e$ with vertices $v_1, v_2 \in \Vertices(\hat X_0)$.
For any (possibly unbounded) $2$-dimensional face $F$ of $\hat X$ adjacent to $e$,
denote by $e_F^1$ and $e_F^2$ the edges of $F$ adjacent to $e$ at the vertices $v_1$ and $v_2$, respectively. 
Without loss of generality, suppose the orientation of $F$ 
induces the orientation on $e$ from $v_1$ to $v_2$.
The rays of $\Sigma_F$ corresponding to $e_F^1$ and $e_F^2$ are then 
$-\bfw_{v_1}(e_F^1)$ and $\bfw_{v_2}(e_F^2)$, respectively, as the orientation induces the inward pointing direction at $v_1$ and the outward pointing direction at $v_2$. See Figure \ref{fig:boundedSigmaF}. Then by Equation \eqref{eqn:tau} we have 
$$- \tau_F(e)\bfw_{v_1}(e)  = -\bfw_{v_1}(e_F^1) + \bfw_{v_2}(e_F^2).$$
Summing over all faces $F$ containing $e$ we get
$$
 - \sum_{\substack{F \in {\Faces}(\hat{X}) \\ F \supset e}} \tau_F(e) \bfw_{v_1}(e)  = 
\sum_{\substack{ e' \in {\Edges}(v_1, \hat{X}) \\ e'\neq e \text{ are in a common face}}} -\bfw_{v_1}(e') + \sum_{\substack{ e'' \in {\Edges}(v_2, \hat{X}) \\e''\neq e \text{ are in a common face}}}  \bfw_{v_2}(e'').
$$
Applying the definition of $\sigma_{v_i}(e)$ from Equation \eqref{eqn:sigma} we obtain 
$$- \sum_{\substack{F \in {\Faces}(\hat{X}) \\ F \supset e}} \tau_F(e) \bfw_{v_1}(e)  = \sigma_{v_1}(e) \bfw_{v_1}(e)  - \sigma_{v_2}(e) \bfw_{v_2}(e).$$ 
Since $\bfw_{v_1}(e)$ is the outward pointing vector of $e$ from $v_1$ and $\bfw_{v_2}(e)$ is the outward pointing vector of $e$ from $v_2$, we have $\bfw_{v_1}(e)  = - \bfw_{v_2}(e)$, 
and so
$$  - \sum_{\substack{F \in {\Faces}(\hat{X}) \\ F \supset e}} \tau_F(e) \bfw_{v_1}(e)  = [\sigma_{v_1}(e) + \sigma_{v_2}(e)] \, \bfw_{v_1}(e),$$
which implies the desired result.


Suppose now $e$ is an unbounded edge of $\hat X$ with vertex $v$ of sedentarity $0$ and vertex $\bar{v}$ of sedentarity $1$.
Consider the Delzant section $s_D$ corresponding to the divisor $D$ containing $\bar v$,
and the local trivialisations of the normal bundle to $D$ used to construct the fans $\Sigma_F$. 
For each edge $\bar e$ of $\hat{D}$ containing $\bar{v}$, recall that ${\bf t}_{\bar{v}}(\bar e)$ denotes the primitive tangent vector to the graph of $s_D$ at $\bar v$ in the local trivialisation over $\bar e$, and that
$$- m_{\bar{v}}(s_D){\bf w}_{v}(e)=  \sum_{\bar{e} \in \Edges(\bar{v}, \hat{D})} {\bf t}_{\bar{v}}(\bar{e})$$
Using this together with Equation \eqref{eqn:sigma}, we can write
$$[ \sigma_v(e) - m_{\bar{v}}(s_D)]\, {\bf w}_{v}(e) = -  \sum_{\substack{e' \in {\Edges}(v, X) \\ e' \neq e \text{ are in some face } F \in {\Faces}(v, X)}} \bfw_{v}(e') + \sum_{\bar{e} \in \Edges(\bar{v}, \hat{D})} {\bf t}_{\bar{v}}(\bar{e}).$$
For any $2$-dimensional face $F$ of $\hat X$ adjacent to $e$ there is a unique edge $e'$ of $F$ of sedentarity $0$ adjacent to $e$ at $v$, and a unique edge $\bar{e}$ of $F$ of sedentarity $1$ adjacent to $e$ at $\bar v$.
We can assume that $F$ was oriented so that the induced orientation on $e$ is from $v$ to $\bar{v}$. 
Then the vector ${\bf t}_{\bar{v}}(\bar{e})$ is exactly the one used to construct the face fan $\Sigma_F$, whereas due to our choice of orientation we have ${\bf w}_F(e') = -\bfw_{v}(e')$. 
By Equation \eqref{eqn:tau} we have 
$$- \tau_F(e) {\bf w}_v(e)  = -{\bf w}_v(e')  + {\bf t}_{\bar{v}}(\bar{e}).$$ 
Adding this over all faces $F$ containing $e$ and comparing with the previous equation proves the claim. 
\end{proof}


The next lemma provides an expression for the number of $2$-dimensional faces of $\hat X$
arising from the Noether's formulas satisfied by the toric surfaces defined by the face fans $\Sigma_F$.


\begin{lemma}\label{NoetherToric}
Let $X$ be a compact tropical surface with a Delzant face structure $\hat{X}$.
Let $D_1, \dots, D_k$ denote the irreducible boundary divisors of $X$.
Then 
$$12|{\Faces}(\hat{X})| = \sum_{v \in {\Vertices}(\hat{X})}  3|{\Faces}(v, \hat{X})| - \sum_{v \in {\Vertices}(\hat{X}_0)}\sum_{e \in {\Edges}(v, \hat{X})} \sigma_v(e) + \sum_{i=1}^k D_i^2.$$
\end{lemma}

\begin{proof}
To prove the lemma, we provide a Noether-type formula for every face $F$ of $\hat X$.
We do this by cases.

$\bullet$ {\em Case $F$ is bounded.} Suppose $F$ is a bounded $2$-dimensional face of $\hat X$. 
As $\hat X$ is a Delzant face structure, the toric variety associated to the fan 
$\Sigma_F$ is a non-singular toric surface. 
Noether's Formula for this toric surface (see \cite[Section 2.5]{Fulton}) says that
\begin{equation}\label{eq:Noetherforpolygon}
12 =  3 |{\Faces}(\Sigma_F)|  +  \sum_{\rho \in {\rm Rays}(\Sigma_F)} D_\rho^2,
\end{equation}
where ${\Faces}(\Sigma_F)$ is the set of $2$-dimensional cones of $\Sigma_F$ and  $D_{\rho}$ is the divisor in the toric surface corresponding to the ray $\rho$ of $\Sigma_F$.  
Since the fan $\Sigma_F$ is unimodular, the self-intersection $D_{\rho}^2$ is equal to $\tau_F(e)$ where $e \in {\Edges}(F)$ is the edge of $F$ corresponding to the ray $\rho$ of $\Sigma_F$ \cite[Section 2.5]{Fulton}. 
Taking into account that all edges of $F$ are of sedentarity $0$, we obtain
\begin{equation*}\label{eq:noethersed0}
12 = 3 |{\Vertices}(F)|  + \sum_{\substack{e \in {\Edges}(F)\\ \sed(e)=0}} \tau_F(e).
\end{equation*}

$\bullet$ {\em Case $F$ is unbounded and contains no vertex of sedentarity 2.} Suppose $F$ is a $2$-dimensional face of $\hat X$ containing points of sedentarity 1 but no points of sedentarity $2$. 
Let $e$ be the unique edge of $F$ of sedentarity 1 and $D_e$ be the boundary divisor containing $e$.
The fan $\Sigma_F$ is unimodular, so the corresponding toric surface satisfies Equation \eqref{eq:Noetherforpolygon}. 
The number of $2$-dimensional cones of $\Sigma_F$ is equal to
$|\Vertices(F)| + 1$ if the section $s_{D_e}$ has a zero on $e$ 
(in which case two extra rays were added to the edge fan of $F$ to obtain $\Sigma_F$), 
and equal to $|\Vertices(F)|$ if $s_{D_e}$ has no zero on $e$ (in which case only one ray was added to the edge fan of $F$).
Moreover, the self-intersections $D_{\rho}^2$ are equal to $\tau_F(e)$ if $\rho$ is a ray of $\Sigma_F$ corresponding to an edge $e \in {\Edges}(F)$ of sedentarity 0, and equal to $-1$ if $\rho$ is one of the extra rays added to the edge fan of $F$.
In all cases, we have
\begin{equation*}\label{eq:noethersed1}
12 = 3 |{\Vertices}(F)| + m_e(s_{D_e}) + \sum_{\substack{e \in {\Edges}(F)\\ \sed(e)=0}} \tau_F(e),
\end{equation*}
where $m_e(s_{D_e}) = 1$ if $s_{D_e}$ has a zero on $e$ and $m_e(s_{D_e}) = 0$ otherwise.

$\bullet$ {\em Case $F$ is unbounded and contains a vertex of sedentarity 2.}
Suppose $F$ is a $2$-dimensional face of $\hat X$ containing a point of sedentarity $2$. 
Again, the fan $\Sigma_F$ is unimodular, and so the corresponding toric surface satisfies Equation \eqref{eq:Noetherforpolygon}. We have $|\Faces(\Sigma_F)|=|\Vertices(F)|$. 
The self-intersections $D_{\rho}^2$ are equal to $\tau_F(e)$ if $\rho$ is a ray of $\Sigma_F$ corresponding to an edge $e \in {\Edges}(F)$ of sedentarity 0, 
and equal to $0$ if $\rho$ is one of the two extra rays added to the edge fan of $F$.
We thus get
\begin{equation*}\label{eq:noethersed2}
12 = 3 |{\Vertices}(F)|  + \sum_{\substack{e \in {\Edges}(F)\\ \sed(e)=0}} \tau_F(e).
\end{equation*}

Now, we add all these equalities over all faces $F$ of $\hat X$ to obtain
$$12|{\Faces}(\hat{X})| =  \sum_{F \in {\Faces}(\hat{X})}  3|{\Vertices}(F)| + \sum_{\substack{e \in \Edges(\hat{X})  \\ \sed(e) = 1}} m_e(s_{D_e}) + \sum_{\substack{F \in \Faces(\hat{X})}} \sum_{\substack{e \in \Edges(F) \\ \sed(e) = 0}}   \tau_{F}(e).$$
Using that $\sum_{F \in {\Faces}(\hat{X})}  |{\Vertices}(F)| = \sum_{v \in {\Vertices}(\hat{X})}  |{\Faces}(v, \hat{X})|$ and changing the order of the summation, we get
$$12|{\Faces}(\hat{X})| =  \sum_{v \in {\Vertices}(\hat{X})}  3|{\Faces}(v, \hat{X})| +  \sum_{\substack{e \in \Edges(\hat{X})  \\ \sed(e) = 1}} m_e(s_{D_e}) + \sum_{\substack{e \in \Edges(\hat{X}) \\ \sed(e) = 0}} \sum_{\substack{F \in \Faces(\hat{X}) \\ F \supset e}}  \tau_{F}(e).$$
Applying Lemma \ref{lem:ToricInt}, we further obtain
$$12|{\Faces}(\hat{X})| =  \sum_{v \in {\Vertices}(\hat{X})}  3|{\Faces}(v, \hat{X})|+   \sum_{\substack{e \in \Edges(\hat{X})  \\ \sed(e) = 1}} m_e(s_{D_e})  + \sum_{\substack{\bar{v} \in \Vertices(\hat{X}) \\ \sed(\bar{v}) = 1}} m_{\bar{v}}(s_{D_{\bar v}}) - \sum_{v \in {\Vertices}(\hat{X}_0)}\sum_{e \in {\Edges}(v, \hat{X})} \sigma_v(e),$$
where $D_{\bar v}$ denotes the boundary divisor containing the vertex $\bar v$.
Finally, Equation \eqref{eq:boundaryselfintersection} gives
$$\sum_{i = 1}^k D_i^2 = \sum_{\substack{e \in \Edges(\hat{X})  \\ \sed(e) = 1}} m_{e}(s_{D_e})+  \sum_{\substack{\bar{v} \in \Vertices(\hat{X}) \\ \sed(\bar{v}) = 1}} m_{\bar{v}}(s_{D_{\bar v}}),$$
which proves the lemma.  
\end{proof}



We now combine these results together to complete the proof of Noether's Formula.

\begin{proof}[Proof of Theorem \ref{thm:Noether}]
From Lemma \ref{lem:localcontribution} we obtain
\begin{multline}\label{ChernSimp}
\deg \left( \csm_0(X) + \csm_1(X)^2 - \sum_{i = 0}^k D_i^2 \right) = 
\sum_{v \in {\Vertices}(\hat{X})}  \left( 12 - 6|{\Edges}(v, \hat{X})| + 3|{\Faces}(v, \hat{X})| \right) \\ 
- \sum_{v \in {\Vertices}(\hat{X}_0)}\sum_{e \in {\Edges}(v, \hat{X})} \sigma_v(e).
\end{multline}
Adding $\deg(\sum_{i = 0}^k D_i^2)$ to both sides, and given that each edge of $\hat X$ is adjacent to exactly two vertices, we obtain
\begin{multline*}
\deg \left(\csm_1(X)^2 + \csm_0(X) \right) =   12 |{\Vertices}(\hat{X}) | - 12|{\Edges}(\hat{X})| + \sum_{v \in {\Vertices}(\hat{X})}  3|{\Faces}(v, \hat{X})|  \\ 
- \sum_{v \in {\Vertices}(\hat{X}_0)}\sum_{e \in {\Edges}(v, \hat{X})} \sigma_v(e) + \sum_{i = 1}^k D_i^2. 
\end{multline*}
Finally, by applying Lemma \ref{NoetherToric}, we get
$$\deg(\csm_1(X)^2 + \csm_0(X)) =   12 |{\Vertices}(\hat{X}) | - 12|{\Edges}(\hat{X})| + 12|\Faces(\hat{X})| = 12 \chi(X),$$
and the proof is complete. 
\end{proof}



We next present a corollary that proves Noether's Formula for toric compactifications of tropical surfaces in $\R^n$. 
To do this we consider the {\bf combinatorial stratification} $X^c$ of $X$ from \cite[Section 1.5]{MikZha:Jac}. 
Suppose that a tropical surface $X \subset \R^n$ has lineality space of dimension $0$. 
Since
the local fans around all vertices of the combinatorial stratification $X^c$ of $X$ are matroidal fans, the cell structure $X^c$ gives a polyhedral face structure of $X$ in $\R^n$.
The collection of unbounded directions of $X$ can also be stratified using the combinatorial stratification. We denote by $\rec{(X^c)}$ this stratified set; the next corollary assumes that this is a unimodular fan.

\begin{cor}\label{cor:surfaceToricvariety}
Let $X \subset \RR^n$ be a tropical submanifold of dimension 2 with lineality space of dimension $0$ such that $\Sigma = \rec{(X^c)}$ is a unimodular fan. Then the closure $\overline{X} \subset \T\Sigma$ of $X$ in the tropical toric variety $\T\Sigma$ is a compact tropical surface satisfying Noether's Formula. 
\end{cor}

\begin{proof}
Since $\Sigma$ is unimodular, the tropical toric variety  $ \T\Sigma$ is non-singular. Moreover, the closure $\overline{X}$ is compact and the closure of each 2-dimensional face of the combinatorial stratification of $X$ is either bounded or, up to extended  integer linear transformation, equal to the compactification in $\T \times \R^{n-1}$ or $\T^2 \times \R^{n-2}$ of a polyhedron transverse to the boundary. 
Therefore the combinatorial stratification of $\overline{X}$ is a Delzant face structure, in the sense of Definition \ref{def:facestructure}. We can thus apply Theorem \ref{thm:Noether} to obtain the statement. 
\end{proof}


Corollary \ref{cor:surfaceToricvariety} can also be extended to a larger class of 2-dimensional tropical submanifolds of toric varieties by applying tropical blowup operations to the surfaces described above and the fact that Noether's Formula is preserved under these operations by \cite[Theorem 5.1]{Shaw:Surf}.

We conclude with an example showing the meaning of Noether's Formula in the particular case where $X$ is a hypersurface of a tropical toric 3-fold.

\begin{exa}
Suppose that $X_0$ is a non-singular tropical hypersurface in $\R^3$ defined by a tropical polynomial $f$. Let $\Delta$ be the Newton polytope of $f$ and let $\Sigma$ be the normal fan of $\Delta$. Consider the compact tropical surface $X$ obtained as the compactification of $X_0$ in the tropical toric variety $\TT\Sigma$, as in Example \ref{ex:hyper}. 

Topologically, $X$ is a wedge of $|{\rm Int}(\Delta) \cap \Z^3|$ spheres, so that 
$$\chi(X) = 1 + |{\rm Int}(\Delta) \cap \Z^3|.$$
The tropical hypersurface $X$ has a face structure $\hat{X}$ dual to the subdivision of its Newton polytope. This face structure is Delzant. 
We can use Lemma \ref{lem:localcontribution} to determine the local contribution of each vertex to $ \deg(\csm_0(X) + \csm_1(X)^2 - \sum_{i = 1}^k D_i^2).$ We have

$$
m_v(\csm_0(X) + \csm_1(X)^2 - \sum_{i = 1}^k D_i^2)= \begin{cases}
  2 & \text{if } \sed(v) = 0, \\
  - 3 &  \text{if } \sed(v) = 1, \\
3 & \text{if } \sed(v) = 2.  
\end{cases}$$

To compute the self-intersection of the boundary divisors $D_i$ we notice that the boundary divisors are in bijection with the facets of the polytope $\Delta$. Each facet has a normal vector ${\mathbf n}_F$, and two facets which intersect do so along an edge. 
It can be shown that the self-intersection number $D_i^2$ is equal to $\tau_i$, given by
$$- \tau_i \, {\mathbf n}_{F_i} = \sum_{\substack{F' \in {\F}(\Delta) \\ F \neq F'}} L(F \cap F') \, {\mathbf n}_{F'}, $$ where $L(F \cap F') $ denotes the lattice length of the intersection of the faces $F$ and $F'$. In particular, the lattice length is $0 $ if $F \cap F' = \emptyset$ or if $F \cap F'$ is a point.

Using the duality between $\hat{X}$ and the unimodular subdivision of $\Delta$, Noether's formula translates to a formula for the number of interior lattice points of a polytope in terms of its lattice volume $ {\rm Vol}(\Delta) $, the total lattice area of its $2$-dimensional faces ${\rm Area} (\Delta)$, the total lattice length of its edges  ${\rm Peri} (\Delta)$, and the numbers $\tau_i$:

$$12(1 + |{\rm Int}(\Delta) \cap \Z^3|) = 2 {\rm Vol}(\Delta) - 3 {\rm Area} (\Delta) + 3 {\rm Peri} (\Delta)   + \sum_{i = 1}^k \tau_i.$$
\end{exa}
We conjecture a generalization of Theorem \ref{thm:Noether} for compact tropical manifolds of any dimension using the tropical Todd class. 
The Todd class of a complex vector bundle $F$ is a formal power series in its Chern classes. The first few terms of the Todd class are 
\[ \Todd =  1 + \frac{c_1}{2} + \frac{c_1^2  + c_2}{12} + \frac{c_1c_2}{24} + \dots.\]
We define the Todd class of a tropical manifold $X$ of dimension $n$ to be obtained from the formal power series above with the substitution 
$c_k = \csm_{n-k}(X)$, which accounts for the indexing of Chern classes by codimension and the CSM classes by dimension. Let $\Todd_d$ denote the degree $d$ part of $\Todd$ above. 

\begin{conj}
For a compact tropical manifold $X$ of dimension $n$ we have 
$$\chi(X) = \deg \Todd_n(X). $$
\end{conj}

\vspace{2mm}