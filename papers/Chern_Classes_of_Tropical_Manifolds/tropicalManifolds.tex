% !TEX root = main.tex



\section{Introduction}

The Chern-Schwartz-MacPherson (CSM) cycles of a matroid $M$ are a collection of tropical cycles introduced in \cite{LdMRS}. 
When the matroid is realisable by a hyperplane arrangement in $\CC^n$, 
its CSM cycles encode the CSM classes of the complement of the arrangement inside its 
wonderful compactification.

CSM cycles of matroids geometrically encode many properties of the matroid. 
Their degrees, for example, are the coefficients of the shifted reduced characteristic
polynomial; 
this was proved via deletion-contraction in \cite{LdMRS}, and later by \cite{AshrafBackman} using a refined basis activities expansion of the Tutte polynomial. 
The fact that the coefficients of the shifted reduced characteristic polynomial of a matroid have this tropical-geometric interpretation was used in 
\cite{ADH} to prove the log-concavity 
of this sequence.
CSM cycles of matroids have also been related to tautological classes of matroids in \cite{BEST}.

In this paper we extend the definition of CSM cycles of matroids to (smooth) tropical manifolds,
and we give evidence that they behave like Chern classes of tangent bundles.
Tropical manifolds are topological spaces 
equipped with an atlas of charts to Bergman fans of matroids 
and transition maps that are integer affine maps; 
see Definition \ref{def:tropmanifold} for more details. 
Examples of tropical manifolds include tropical curves, tropical linear spaces, 
non-singular tropical hypersurfaces, and integral affine manifolds; see Section \ref{sec:examples}.


The CSM cycles of a matroid $M$ are Minkowski weights supported on the different skeleta of the Bergman fan of $M$. 
The weights assigned to the faces of the skeleta are given by products of beta invariants of matroid minors of $M$; see Definition \ref{def:chernweight}. 
A priori, this means that the CSM cycles of $M$ depend heavily on $M$. 
There are examples of non-isomorphic matroids whose corresponding Bergman fans have the same support up to integer linear maps; see, for example, Proposition \ref{prop:parallelBergman}.  
As a first step in extending the definition of CSM cycles to tropical manifolds, we prove in Proposition  \ref{thm:csmisomorphism} that the definition of CSM cycles of matroids 
is ${\rm GL}_n(\Z)$ invariant. 
For this purpose, we show that the weights of the CSM cycles can be equivalently determined from the cosheaves arising in tropical homology \cite{IKMZ}. 

In Section \ref{sec:correspondence}, we provide results that relate the CSM classes of algebraic varieties with the CSM cycles of their tropicalisations.
We first show that the $0$-th CSM cycle of the tropicalisation of a family of subvarieties of a toric variety recovers the topological Euler characteristic of a general member of the family.

\newtheorem*{thm:Eulercharacteristic}{Theorem \ref{thm:Eulercharacteristic}}
\begin{thm:Eulercharacteristic}
Let $\mathcal{X}$ be a meromorphic family of subvarieties of a toric variety $\mathcal{Y}$ over the punctured disk $\mathcal{D}^*$, with general member $\XX_t$. Suppose that $\trop(\mathcal{X}) = X$ is a tropical submanifold in $Y = \trop(\mathcal{Y})$. Then 
$$\chi(\mathcal{X}_t) = \deg(\csm_0(X)),$$
where $\chi(\mathcal{X}_t)$ denotes the topological Euler characteristic. 
\end{thm:Eulercharacteristic}

To prove the  above theorem, we combine the correspondence  results for CSM cycles of matroid fans from \cite{LdMRS} and 
the description of the tropical motivic nearby fibre of Katz and Stapledon \cite{KatzStapledon}. 


In the case $\mathcal Y$ is a non-singular projective toric variety, we relate the CSM cycles of $X = \trop(\mathcal{X})$ to the CSM classes of a general member $\XX_t$ of the family $\XX$.
To do this, we combine the above theorem and the work of Esterov \cite{Esterov}, and relate the CSM class of $\XX_t$ in the Chow ring of $\mathcal{Y}$ to the CSM cycles of the tropicalisation of $\XX$, as we now explain. 

The Chow cohomology ring of a non-singular projective toric variety $\mathcal Y$ defined by a fan $\Sigma$ is isomorphic to the ring of Minkowski weights supported on $\Sigma$ \cite{FultonSturmfels}. We denote this graded ring  by $\MW_{\ast}(\Sigma)$. 
The group of Minkowski weights of dimension $k$ is denoted $\MW_k(\Sigma)$, and is isomorphic to $\Hom ( \A_{n-k}(\mathcal{Y}) , \Z)$, where $n$ is the dimension of $\mathcal Y$ and $\A_{n-k}(\mathcal{Y})$ is the $(n-k)$-graded piece of the Chow group of $\mathcal{Y}$. 
By Poincar\'e duality, taking the cap product with the fundamental class of $\mathcal{Y}$ provides an isomorphism
$$\cdot \frown  [\mathcal{Y}] \colon \MW_{k}(\Sigma) \xrightarrow{\,\,\cong\,\,} \A_{k}(\mathcal{Y}).$$
If $\mathcal{X}$ is transverse to the toric boundary of $\mathcal{Y}$ then the recession fan of $X$ is supported on the fan $\Sigma$. Following \cite{AHR}, by taking the recession cycles $\rec(\csm_k(X))$ of $\csm_k(X)$ (see Definition \ref{def:reccycle}), we thus obtain a collection of Minkowski weights on $\Sigma$.

\newtheorem*{thm:corrToric}{Theorem \ref{thm:corrToric}}
\begin{thm:corrToric}
Suppose $\mathcal{X}$ is a meromorphic family of subvarieties of $(\C^*)^N$ over the punctured disk $\mathcal{D}^*$, with general member $\XX_t$. 
Let $\mathcal{Y}$ be a non-singular projective toric variety 
with defining fan $\Sigma$, and let $\overline{\mathcal{X}}$ be the closure of $\mathcal{X}$ in $\mathcal{Y} \times \mathcal{D}^*$. 
Suppose that $\overline{X} := \trop(\overline{\mathcal{X}})$ is  a tropical manifold transverse to $\trop(\mathcal{Y})$. Then
$$\csm_k( \mathbbm{1}_{\mathcal{X}_t}) \, = \, \rec(\csm_{k}(X)) \frown [\mathcal{Y}] 
\,\, \in \A_{k}(\mathcal{Y}).$$
\end{thm:corrToric}

Theorems \ref{thm:Eulercharacteristic} and \ref{thm:corrToric} generalize \cite[Theorem 3.1]{LdMRS}, which relates the CSM class of the complement of a hyperplane arrangement in the Chow ring of its wonderful compactification 
to the CSM cycles of the corresponding matroid. 

In previous work, 
Bertrand and Bihan equipped the skeleta of tropical complete intersections in $\R^n$ with integer weights 
to produce balanced tropical cycles \cite{BertrandBihan}.  
In their construction, their weights are, up to  sign, related to the Euler characteristic of a non-degenerate complete intersection 
in $(\CC^*)^n$ \cite[Theorem 5.9]{BertrandBihan}. 
The situation they consider overlaps with our own when the tropical complete intersections are also tropical manifolds. In the hypersurface case, this is equivalent to the dual subdivision being unimodular. 

In Section \ref{sec:adjunction}, we give a product formula for CSM cycles of matroids, and use it to prove an adjunction-like formula relating the codimension-1 CSM cycles of a tropical manifold and a submanifold of codimension 1. This generalises the adjunction formula for curves in tropical surfaces proved in \cite[Theorem 6]{Shaw:Surf}.

\newtheorem*{thm:Adjunction}{Theorem \ref{thm:Adjunction}}
\begin{thm:Adjunction}[Adjunction formula]
Let $X$ be a tropical manifold  of dimension $d$ and $D \subset X$ a tropical submanifold of codimension 1 in $X$. 
Then 
$$\csm_{d-2}(D) = (\csm_{d-1}(X) -  D )\cdot D,$$
where $\cdot$ denotes the tropical intersection product in $X$. 
\end{thm:Adjunction}

The notion of a tropical submanifold is more subtle than just being a tropical manifold contained in a tropical manifold; see Definition \ref{def:submanifold}. In Example \ref{ex:singline} we recall a situation where being a tropical manifold of contained in $X$ and having codimension 1 is not sufficient for Theorem \ref{thm:Adjunction}. 
 
 
In the case $X$ is a 2-dimensional compact tropical manifold, we prove a tropical version of Noether's Formula expressing its topological Euler characteristic $\chi(X)$ in terms of its CSM cycles and their intersections. This is proven under the assumption that $X$ admits a Delzant face structure, i.e., a cell structure where every cell locally looks like a polytope in which the primitive vectors of the edges adjacent to any vertex can be extended to a basis of the ambient lattice; see Definition \ref{def:facestructure}.
 
\newtheorem*{thm:Noether}{Theorem \ref{thm:Noether}}
\begin{thm:Noether}[Tropical Noether's Formula]
Let $X$ be a compact tropical surface admitting a Delzant face structure.
Then 
$$\chi(X) = \frac{\deg( \csm_0(X) + \csm_1(X)^2) }{12}.$$
\end{thm:Noether}
Noether's formula had previously been proven for tropical surfaces arising from tropical toric surfaces via the operations of tropical modifications and summation \cite{Shaw:Surf}. 

 
\subsection*{Acknowledgements}
We would like to give our thanks to Alex Esterov, Kyle Huang, Grigory Mikhalkin, Dmitry Mineyev, Johannes Rau, and Paco Santos for very helpful conversations.  

L.~L.d.M.~was funded by ECOS NORD 298995 and PAPIIT-IN105123. 
F.~R.~was partially supported by EPSRC grant EP/T031042/1.
K.~S.~is supported in part by the Trond Mohn Foundation project ``Algebraic and topological cycles in complex and tropical geometry". 
This collaboration was also carried out under the  Center for Advanced Study Young Fellows Project ``Real Structures in Discrete, Algebraic, Symplectic, and Tropical Geometries", as well as during the workshops Tropical Methods in Geometry in the MFO in Oberwolfach and Algebraic Aspects of Matroid Theory in BIRS in Banff. We thank all  institutions for the their support and excellent working conditions.
 





\section{Preliminaries}
 
\subsection{Tropical manifolds}
To define tropical manifolds we first introduce a partial compactification of $\R^n$ used in tropical geometry. Let $\mathbb{T} := \R \cup \{-\infty\}$ and equip it with the topology whose basis consists of intervals $(a, b)$ and $[-\infty, b)$ where $a < b$. The real numbers $\R$ will be equipped with the Euclidean topology. The spaces $\T^r$ and $\T^r \times \R^n$ are equipped with the product topologies. All subsets of $\T^r \times \R^n$ are equipped with the subspace topologies. 

The space $\T^r$ is a stratified,  where the strata are 
$$\T^r_I := \{ (x_1, \dots , x_r)  \colon x_i = -\infty \text{ iff } i \in I\}$$ for 
$I \subset [n]$.
Given a stratum $\T^r_I$, we call $I$ its \textbf{sedentarity}, and we say  a point $x \in \T^r$ is of  sedentarity $I$ if and only if it is in $\T^r_I$.  Notice that we have 
$\T^r = \bigsqcup_{I \subset [r]} \T^r_I$ 
and $\T^r_I$ can be identified with $\R^{r -|I|}$.   

We assume the reader is familiar with the definitions of tropical cycles in $\R^n$, including weight functions and the balancing condition, and refer to \cite{MaclaganSturmfels} and \cite{MikRau} for more background. 
A \textbf{tropical cycle of sedentarity $I$} in $\T^r$ is the topological closure of a tropical cycle in $\T^r_I \cong \R^{r-|I|}$. 
Notice that we do not ask for a cycle in $\T^r$ of sedentarity $I$ to satisfy a balancing condition on any new codimension-1 faces in the closure.
 A tropical cycle in $\T^r$ is a formal sum of tropical cycles of different  sedentarities. 

We can also extend the notion of  sedentarity  to spaces of the form $\mathbb{T}^{r} \times \R^n$. A tropical cycle of sedentarity $I$ is the closure  in $\mathbb{T}^{r} \times \R^n$ of a cycle in $\mathbb{T}^r_I \times \R^n$.   A $k$-dimensional  tropical cycle in $\mathbb{T}^r_I$ is again a formal sum of $k$-dimensional cycles of different sedentarities. 


\begin{exa}[{\em Matroidal tropical cycles.}]
Denote by $\{e_0, e_1, \dotsc, e_n\}$ the standard basis of the lattice $\ZZ^{n+1} \subseteq \R^{n+1}$. 
For any subset $S \subseteq \{0,\dotsc,n\}$, let $e_S \coloneqq \sum_{i\in S} e_i \in \ZZ^{n+1}$.
We denote $\mathbf 1 := (1,1,\dots,1) \in \RR^{n+1}$.

Suppose $M$ is a loopless matroid of rank $d+1$ on the ground set $\{0,1,\dots,n\}$. 
The {affine Bergman fan} $\hat{\B}(M)$ of $M$ 
is the pure $(d+1)$-dimensional rational polyhedral fan in $\R^{n+1}$ consisting of the collection of cones of the form 
\[\sigma_\F \coloneqq  -\cone(e_{F_1},e_{F_2},\dotsc,e_{F_k}) + \RR \!\cdot\! \mathbf 1\]
where 
$\F = \{\emptyset \subsetneq F_1 \subsetneq F_2 \subsetneq \dotsb \subsetneq F_k \subsetneq \{0,\dotsc, n\}\}$ is a chain of flats in the lattice of flats $\L(M)$ of $M$. 
If $M$ is a matroid with loops then we define $\hat{\B}(M) = \emptyset$.
The {(projective) Bergman fan} $\B(M)$ of $M$
is the pure $d$-dimensional rational polyhedral fan obtained as the image of $\hat{\B}(M)$ 
in the quotient vector space $\R^{n+1} / \,\RR \!\cdot\! \mathbf 1$.

The {\bf matroidal tropical cycle} $Z_M$ associated to $M$ is the tropical cycle in the vector space $\R^{n+1} / \,\RR \!\cdot\! \mathbf 1 \cong \R^n$ whose support is the projective Bergman fan 
$\B(M)$ and all multiplicities on the maximal cones are equal to $1$. 
Note that the product $ X = \T^r \times Z_M$ is a tropical cycle in $\mathbb{T}^r \times \R^n$ of sedentarity $\emptyset$ --- indeed, $X$ is the closure of the matroidal tropical cycle $Z_{C_r \oplus M} \subseteq \R^r \times \R^n$, where $C_r$ is the free matroid on $r$ elements (i.e., having no circuits).
\end{exa}

Spaces of the form $\T^r \times Z$ for $Z$ a matroidal cycle in $\R^n$ will be the local building blocks of tropical manifolds. Before we give the precise definition we must also define the notion of {extended integer affine maps}. Recall that an integer affine map $F: \R^{r'} \to \R^r$ is the composition of an integer linear map with a translation defined over $\R$, that is, a map of the form $F(x) = Ax + b$ with $A \in \ZZ^{r \times r'}$ and $b \in \RR^r$.



\begin{defi}
\label{def:integeraffmap}
Let $F\colon  \R^{r'} \rightarrow \R^r$ be an integer affine map and let $A \in \ZZ^{r \times r'}$ denote the integer matrix representing the linear part of $F$. Let $I$ be the set of 
$i \in [r']$ such that the $i$-th column of $A$ has only non-negative
entries. 
Then $F$ can be extended to a map 
$$\hat F\colon \left( \bigcup \limits_{J \subset I} \T^{r'}_J \right)\rightarrow \TT^r$$ 
by continuity. 
The restriction of such an $\hat F$ to an open subset $U' \subset \TT^{r'}$ is called an \textbf{extended integer affine map}. Note that this only makes sense if we have $\text{sed}(x) \subset I$ for all $x \in U'$. 
\end{defi}

We are now ready to define tropical manifolds. These will be topological spaces with an atlas of charts to the supports of matroid fans and extended integer affine linear  transition maps. 
We follow the definitions from  \cite[Section 7]{MikRau} for abstract tropical cycles and polyhedral spaces,
and collect the necessary properties from \cite[Definition 7.4.1]{MikRau} in the definition below. 

\begin{defi} \label{def:tropmanifold}
A $d$-dimensional \textbf{tropical manifold} is a Hausdorff topological space $X$ equipped with an atlas of charts  $\{\varphi_{\alpha} \colon U_{\alpha} \rightarrow \Omega_{\alpha} \subset X_{\alpha}\}_{{\alpha} \in \mathcal I}$ where $\mathcal I$ is a finite set 
and 
\begin{enumerate}
\item $X_{\alpha} = \mathbb{T}^{r_{\alpha}} \times Y_{\alpha}$ with $Y_{\alpha}$ a matroidal tropical cycle of dimension $d- r_{\alpha}$ in $\R^{n_{\alpha}}$ and $\Omega_\alpha$ is  an open subset of $X_{\alpha}$ for every $\alpha \in \mathcal I$, 
\item for every $\alpha \in \mathcal I$ the map $\varphi_{\alpha} \colon U_{\alpha} \rightarrow \Omega_{\alpha}$ is a homeomorphism, 
\item the transition functions $\varphi_{\beta} \circ \varphi_{\alpha}^{-1}$ are extended integer affine maps,
\item \label{closure} for each $\alpha \in \mathcal I$ there exists an extension $\varphi_{\alpha}' \colon U_{\alpha}' \rightarrow \Omega_{\alpha}' \subset X_{\alpha}$ of $\varphi_\alpha$ such that $\overline{\Omega}_{\alpha} \subset  \Omega_{\alpha}' $. 
\end{enumerate}
\end{defi}



\subsection{The boundary of a tropical manifold}
\label{sec:boundary}

Points of the stratified space $\T^r \times \R^n$ have an {order of sedentarity} defined by
$$s(x) := |\{ x_i : x_i = -\infty\}|.$$
The order of sedentarity is preserved under invertible extended integer affine maps and hence this notion extends to tropical manifolds, as long as we assume that the chart chosen contains points of sedentarity $0$ in their image. 

\begin{defi}
For a point $x \in  X$ of a tropical manifold, the {\bf order of sedentarity} of $x$ is $s(x) = |s(\phi_{\alpha}(x))|$, where $\phi_{\alpha}: U_{\alpha} \to  X_{\alpha}$ is any  chart satisfying $x \in U_{\alpha}$ and with $\phi_{\alpha}(U_{\alpha})$ containing points of sedentarity $0$.
The {\bf boundary} of a tropical manifold $X$ is $$\partial X = \{ x \in X : s(x) > 0 \}.$$ 
A {\bf boundary divisor} of a tropical manifold $X$ is 
the closure of a connected component of the set $\{ x \in X : s(x) = 1\}$. 
Every boundary divisor is of codimension 1 in $X$ and is itself a tropical manifold; \cite[Proposition 1.2.8]{Shaw:Surf}. Call the set of boundary divisors $\A = \{ D_1, \dots, D_k\}$ the \textbf{arrangement of boundary divisors} of $X$. 
\end{defi}

For a tropical manifold $X$ we denote the complement of its boundary by $X_0 = X \backslash \partial X$. This is the collection of points of $X$ with order of sedentarity  $0$.
Note that $X_0$ is also a tropical manifold and that $X$ is the closure of $X_0$. 
Each connected component of $\{ x \in X : s(x) = k\}$ is a tropical manifold of dimension $\dim X - k$, and so is its closure in $X$. 




\subsection{Examples of tropical manifolds}
\label{sec:examples}


\begin{exa}[{\em Abstract tropical curves.}] An abstract tropical curve is a  $1$-dimensional finite simplicial complex equipped with a metric on the complement of its 1-valent vertices. Abstract tropical curves are exactly $1$-dimensional tropical manifolds, since in dimension $1$ the metric is equivalent to an integer affine structure. Indeed, each point of a tropical curve comes equipped with a chart to a matroidal fan in $\R^n$ of a loopless matroid of rank $2$ or to a neighbourhood of $-\infty \in \T$.
\end{exa}

\begin{exa}[{\em Integral affine manifolds.}]
An integral affine manifold is a manifold equipped with an atlas of charts such that the transition functions are affine transformations with linear part defined by an integral matrix. 
An integer affine manifold is thus a tropical manifold where the tropical charts from Definition \ref{def:tropmanifold} satisfy $X_{\alpha} = \R^d$ for all $\alpha$. In dimension 2 the topological  type of a compact integer affine manifold without boundary is either $S^1 \times S^1$ or the Klein bottle. 

Tropical Abelian varieties are integer affine manifolds obtained as quotients  $\R^n / \Lambda$, where $\Lambda \subseteq \R^n$ is a full-rank sublattice satisfying certain conditions \cite{MikZha:Jac}.  An $n$-dimensional tropical Abelian variety is homeomorphic to $(S^1)^n$. 
\end{exa}


\begin{exa}[{\em Tropical toric varieties.}]\label{ex:toricman}
Let $\Sigma$ be a rational polyhedral fan in $\R^N$.
As a topological space, the associated tropical toric variety $\T\Sigma$ can be described as a quotient of the disjoint union of tropical tori $U_{\sigma} = \T^{\dim(\sigma)} \times \R^{N - \dim(\sigma)},$ ranging over all cones $\sigma \in \Sigma$ \cite[Definition 3.2.3]{MikRau}. 
From this description we obtain an atlas of charts $\{U_{\alpha}, \phi_{\alpha}, X_{\alpha} \}_{\alpha \in I}$ where $I$ is in bijection with the top dimensional cones of $\Sigma$ and the charts $\phi_{\alpha}: U_{\alpha} \to \Omega_{\alpha} \subset X_{\alpha}$ satisfy $X_{\alpha} = \T^N$ for all $\alpha \in I$. 
The tropical toric variety $\T\Sigma$ is a tropical manifold if and only if  $\Sigma \subset \R^N$ is a  unimodular fan, see \cite[Section 3.2]{MikRau}.  The tropical toric variety $\T\Sigma$ is compact if and only if $\Sigma$ is complete.   
\end{exa}


\begin{exa}[{\em Tropical projective space.}]
Projective space is a toric variety and 
tropical projective space can be constructed using the same fan as projective space over a field. It can also be described as the following quotient:
$$\TP^{n} = \frac{\R^{n+1} \backslash ( -\infty, \dots, -\infty)}{(x_0: \dots :x_n) \sim (a + x_0: \dots :a +x_n)  },$$
where $a \in \T \backslash \{-\infty\}$.
From the quotient description we obtain tropical homogeneous coordinates on $\TP^n$ which we write as $[x_0 : \dots : x_n]$.   
\end{exa}



\begin{exa}\label{ex:submanifolds}
[\emph{Tropical manifolds in  tropical toric varieties}]
Let $\hat{X}$ be a pure dimensional polyhedral complex in $\R^n$. The support $X = |\hat{X}|$ is a tropical manifold if for every face $\tau \in \hat X$, the star fan ${\rm star}_\tau(\Sigma)$ of $X$ at  $\tau$ is the support of a Bergman fan of some matroid $M_{\tau}$ up to an integer affine transformation of $\R^n$. 

Suppose in addition that the collection of recession cones of all the cones $\tau \in \hat X$ forms a fan $\Sigma$, called the recession fan of $\hat X$.
Let $\overline{X}$ denote the closure of $X$ in the tropical toric variety $\T \Sigma$.
Then $\overline{X}$ is a compact space and is the canonical compactification of $X$ from \cite{KastnerShawWinz}, \cite{AminiPiquerez}.  Moreover, the space $\overline{X}$ is a tropical manifold in the sense of Definition \ref{def:tropmanifold}. 
\end{exa}



\begin{exa}[{\em Tropical linear spaces.}]
A {tropical Pl\"ucker vector} is a vector $p \in \Rbar^{\binom{[n+1]}{d+1}}$
satisfying the tropical Pl\"ucker relations: 
For every $A \in \binom{[n+1]}{d+2}$ and $B \in \binom{[n+1]}{d}$, the maximum $\max_{i \in A \setminus B} (p_{A \setminus i} + p_{B \cup i})$ is attained twice. 
Any tropical Pl\"ucker vector $p$ gives rise to a {tropical linear space} $\textstyle L(p)$ defined as
\[
\textstyle L(p) := \{x \in \R^{n+1} / \,\RR \!\cdot\! \mathbf 1 : \max_{i \in S} (p_{S-i} + x_i) \text{ is achieved twice for any } S \in \binom{[n+1]}{d+2}\}.
\]
Tropical linear spaces are $d$-dimensional tropical manifolds: around any $x \in \textstyle L(p)$,
the tropical linear space $\textstyle L(p)$ looks like a matroidal fan \cite{Speyer}.

We can consider the closure of a tropical linear space in different tropical toric varieties. For example, the closure of $L(p)$ in $ \TP^n$ is a tropical projective subspace; however, the closure $\overline{L(p)}$ does not necessarily intersect the boundary of $\TP^n$ transversely, and $\overline{L(p)}$ will not be a tropical manifold in the sense of Definition \ref{def:tropmanifold}.
Nonetheless, the closure of $L(p)$ in the tropical toric variety of $\Sigma = \rec(L)$ where $\rec(L)$ denotes the recession fan of $L$ is a tropical manifold in the sense of Definition \ref{def:tropmanifold}.
\end{exa}
 


\begin{exa}[{\em Non-singular tropical hypersurfaces.}]\label{ex:hyper}
A tropical hypersurface $X_f \subset \R^N$ is the divisor $\text{div}_{\R^N}(f)$ of a tropical regular  function  $f:\R^N  \to \R$, see \cite{MaclaganSturmfels}, \cite{MikICM}. It is a weighted  polyhedral complex dual to a regular subdivision of the Newton polytope of $f$.  If the dual subdivision is unimodular, meaning each polytope in the subdivision is a simplex with normalised volume equal to $1$, the hypersurface is called non-singular. Non-singular hypersurfaces locally look like matroidal tropical cycles of corank-$1$ matroids, and are thus tropical manifolds.

Let $\Sigma$ be the dual fan of the Newton polytope of $f$. 
The recession fan of $X_f$ is the codimension-1 skeleton of $\Sigma$. As in Example \ref{ex:submanifolds}, we can compactify $X_f$ in $\T \Sigma$. If the  dual fan $\Sigma$ is unimodular then $\T \Sigma$ and $\overline{X_f}$ are both compact tropical manifolds in the sense of Definition \ref{def:tropmanifold}.
\end{exa}



\subsection{Cycles in tropical manifolds}
Here we recall the definitions of cycles in tropical manifolds from \cite{ShawInt} and  \cite{MikRau}. 
As previously stated, a tropical cycle $A$ in $\R^n \times \T^r$ is a formal sum of cycles $A = \sum_{I \subset [r]} A_I$ where $A_I$ is the closure in $\R^n \times \T^r$ of a tropical cycle in 
$\R^n \times \TT^r_I = \R^n \times \RR^{r - |I|}$.  

\begin{defi}\label{defi:tropicalCycles}
A \textbf{tropical $k$-cycle} $Z$ in a tropical manifold $X$ is a subset $Z \subset X$ equipped with a weight function $w: \Omega \to \Z$ on an open dense subset $\Omega \subset Z$ such that, for all charts $\varphi_{\alpha} \colon U_{\alpha} \to Y_{\alpha} \times \T^{r_{\alpha}}$ of $X$, the image $\varphi_{\alpha}(Z \cap U_{\alpha})$ is a tropical $k$-cycle in $\R^{n_{\alpha}} \times \T^{r_{\alpha}}$ with the weight function induced from the weight function on $Z \cap U_{\alpha}$. 
\end{defi}


\begin{defi}\label{def:transverse}
A tropical $k$-cycle $A$ of a tropical manifold $X$ is {\bf transverse to the boundary} of $X$ if for every boundary stratum  $D_{I} := \bigcap_{i \in I} D_i$ of $X$ and for every chart $\varphi_{\alpha} \colon U_{\alpha} \to Y_{\alpha} \times \T^{r_{\alpha}}$, the image $\varphi_{\alpha}(U_\alpha \cap A \cap D_I)$ is of codimension $|I|$ in $\varphi_{\alpha}(U_\alpha \cap A)$  or it is empty.
\end{defi}





The adjunction formula for tropical CSM cycles in Section  \ref{sec:adjunction} concerns the intersection of divisors in a tropical manifold. 
Similarly, Noether's Formula in Section \ref{sec:Noether} talks about the 
intersection of $1$-cycles in tropical surfaces. 
There are various approaches to intersection theory in tropical manifolds \cite{ShawThesis}, \cite{FrancoisRau}, all of them equivalent for our purposes. 
For completeness, we now recall the theory of intersecting with tropical Cartier divisors. 

A {\bf tropical Cartier divisor} on a tropical manifold $X$ equipped with charts $\varphi_{\alpha} \colon U_{\alpha} \to X_{\alpha} \subset \RR^{n_\alpha} \times \T^{r_{\alpha}}$ is a collection of tropical rational functions $\{f_{\alpha}\}$ where 
$f_{\alpha} \colon \R^{n_{\alpha}} \times \T^{r_{\alpha}} \to \TP ^1 = [-\infty, \infty]$ and such that 
on the overlaps $U_{\alpha} \cap U_{\beta} $ the difference $f_{\alpha} - f_{\beta} $ is a 
bounded integer affine function on $X_{\alpha}$, implying in particular that $f_{\alpha} - f_{\beta} $ does not attain the values $\pm \infty$.
Every codimension-$1$ tropical cycle $D$ in a tropical manifold is a Cartier divisor \cite[Lemma 2.23]{ShawInt}, 
meaning that 
there exists a tropical Cartier divisor $\{f_{\alpha}\}$ such that $\varphi_\alpha (D \cap U_{\alpha}) = \Div_{X_{\alpha}}(f_{\alpha})$ for all $\alpha$.  

Suppose  $X$ is a tropical manifold without boundary. 
Then for $D$ a codimension-$1$ tropical cycle in $X$ and $A$ a tropical $k$-cycle in $X$, we can define the intersection of $D$ and $A$ by first expressing $D$ as a Cartier divisor $f = \{ f_{\alpha}\}$, and then setting
$$D \cdot A = \Div_A( f ),$$
where $\Div_A(f)$ is the tropical cycle which in each chart $U_{\alpha}$ is equal to $\Div_{A \cap U_{\alpha}}(f_{\alpha})$. 
The above recipe also works in the case $X$ has a boundary and $D$ does not contain any boundary divisors. 


When the codimension-$1$ cycle $D$ contains components which are boundary divisors it is possible that some of the functions $\{f_{\alpha}\}$ are identically equal to $\pm \infty$  on $A$. This is the case, for example, when we wish to consider $D^2$ for a boundary divisor $D$. In this situation we use the theory of tropical line bundles from \cite{JRS}. The charts of $X$ restricted to a neighbourhood of $D$ in $X$ defines a line bundle on $D$ \cite{ShawThesis}. 
The Cartier divisor $\{f_{\alpha}\}$ defines a tropical line bundle $L \in \Pic(X) = H^1(X, \Aff_{\mathbb{Z}})$ and 
all Cartier divisors rationally equivalent to $f$ arise from tropical rational sections of $L$. By \cite[Proposition 4.6]{JRS}, every tropical line bundle admits a non-zero section $s$. A choice of section produces another codimension-$1$ cycle $D'$ in $X$ which is rationally equivalent to $D$ and is a sedentarity $0$ cycle. This cycle can then be intersected with $A$, so we define $D \cdot A := D' \cdot A$.
Notice that in this case the intersection product is only well defined up to rational equivalence, since $D'$ depends on the choice of section of $L$. 
 



\subsection{Tropical submanifolds}
\label{sec:submanifolds}
For a tropical subvariety to behave as a submanifold of a tropical manifold we require a compatibility condition on the corresponding charts, as described below.

\begin{defi}\label{def:submanifold}
Let $X$ be a tropical manifold and let $W$ be a tropical cycle in $X$ of dimension $k$ which has constant weight function equal to $1$. 
Then $W$ is a {\bf tropical submanifold} of $X$ if 
there is an atlas of charts $\{U_{\alpha}, \varphi_{\alpha}\}$ for $X$ 
such that $\{(U_{\alpha} \cap W), \varphi_{\alpha}|_W\}$ is also an atlas of charts for $W$. 
\end{defi}

Note that, in particular, if the atlas for $X$ has charts $\varphi_{\alpha} \colon U_{\alpha} \to \Omega_{\alpha} \subseteq Y_{\alpha} \times \T^{r_\alpha}$ where $Y_{\alpha}$ are matroid fans, then $\varphi_{\alpha}|_W \colon U_{\alpha} \cap W \to W_{\alpha} \times \T^{r'_\alpha}$, where $W_{\alpha}$ are matroid fans such that $W_{\alpha} \subseteq Y_{\alpha}$  and $r'_\alpha \leq r_\alpha$.  This implies that the matroid of $W_{\alpha}$ must be a matroid quotient of the matroid of $Y_{\alpha}$  \cite[Lemma 2.21]{ShawInt}, \cite[Proposition 3.3]{FrancoisRau}. Example \ref{ex:singline} shows a pair of tropical manifolds $W$ and $X$ with $W \subset X$ which violates this condition on the compatibility of the charts and hence $W$ is not a submanifold of $X$ in the sense of Definition \ref{def:submanifold}.


We say that $W$ is a sedentarity-$0$ submanifold of $X$ if $r'_\alpha=r_\alpha$ in all charts $\varphi_\alpha$ of $X$. By the condition imposed on the compatibility of the charts, a sedentarity-$0$ submanifold of $X$  is necessarily transverse to the boundary of $X$ in the sense of Definition \ref{def:transverse} when considered as a tropical cycle of $X$. 

\begin{exa*}\label{ex:singline}
As we have mentioned before, being a tropical submanifold depends on a compatibility condition on the charts. 
For instance, there exist tropical surfaces $X$ in $\R^3$ of degree $\geq 3$ which contain tropical lines $L$ not satisfying Definition \ref{def:submanifold}.
These examples were discovered by Vigeland \cite{Vig1} as infinite families of lines on tropical hypersurfaces.
Both the surface $X$ and the line $L$ are tropical manifolds;
however, the atlases for $X$ and $L$ are not compatible with each other. Moreover, these lines in tropical surfaces do not satisfy the adjunction-like formula that we present in Theorem \ref{thm:Adjunction};  see \cite[Section 7.3]{BrugalleShaw}. 
\end{exa*}







\setcounter{thm}{0}


\section{CSM cycles of tropical manifolds}

In this section we show that CSM cycles of matroids are invariant under invertible integer affine transformations of the underlying Bergman fan, which allows us to extend the definition of CSM cycles to tropical manifolds; see Definition \ref{def:csmManifold}. 
We start by recalling the definition from \cite{LdMRS}.


\begin{defi}\label{def:chernweight}
Suppose $M$ is a loopless rank $d+1$ matroid on the ground set $\{0,1,\dots,n\}$.
For $0 \leq k \leq d$, the $k$-dimensional {\bf Chern-Schwartz-MacPherson (CSM) cycle}
of $M$, denoted $\csm_k(M)$, is the tropical cycle in $\R^{n+1} / \, \mathbb R \cdot \mathbf{1} \cong \RR^n$ supported on the $k$-dimensional skeleton 
of the Bergman fan $\B(M)$ in which the weight of the top-dimensional cone $\sigma_\F$ corresponding to a flag of flats $\F \coloneqq  \{\emptyset = F_0
\subsetneq F_1 \subsetneq \dotsb \subsetneq F_{k} \subsetneq F_{k+1} =
\{0,\dotsc,n\}\}$ is 
\[ w_{\csm_k(M)}(\sigma_\F) \coloneqq  (-1)^{d-k} \prod_{i=0}^{k} \beta(M|F_{i+1}/F_i),\]
where $M|F_{i+1}/F_i$ denotes the minor of $M$ obtained by restricting to $F_{i+1}$ and contracting $F_i$.
If $M$ is a matroid with loops then we define $\csm_k(M) \coloneqq  \emptyset$ for all $k$. If $\Sigma_M$ is the matroidal fan of $M$, we will also denote $\csm_k(\Sigma_M) := \csm_k(M)$.
\end{defi}




The next definition extends the notion of CSM cycles to spaces of the form $X =  \T^r \times \Sigma_M$.
The space $X$ is the closure of the matroidal fan of the matroid $C_r \oplus M$, where $C_r$ is the  free matroid on $r$ elements (i.e., the matroid with no circuits).
Note that for any $I \subset [r]$, the intersection $X \cap (\T^r_I \times \R^n)$ is again the support of a matroid fan in  $\R^{r+n -|I|}$ where the underlying  matroid is $C_{r-|I|} \oplus M$. 

\begin{defi}\label{def:csmInTn}
Let $\Sigma_M$ be a matroid fan in $\R^n$.  The $k$-th Chern-Schwartz-MacPherson cycle of 
$\T^r \times \Sigma_M$  is  
$$\csm_k( \T^r \times \Sigma_M )  = \sum_{\emptyset \subset I \subset [r]} \csm_k(\T^r_I \times \Sigma_M ).$$
\end{defi}


The following 
proposition, whose proof we provide later in this subsection, shows that our definition of CSM cycles is well-behaved under extended integer affine  maps. This allows us to define CSM cycles of general tropical manifolds in Definition \ref{def:csmManifold}. 
 
\begin{prop}\label{thm:csmisomorphism}
Let $Y'$ and $Y$ be matroidal cycles in $\R^{n'}$ and $\R^n$ respectively, and suppose that there is an invertible map $\varphi: Y' \to Y$ which is induced by  an affine  linear map $\varphi \colon \R^{n'} \to \R^{n}$.
Then $$\varphi( \csm_k(Y')) =  \csm_k(Y)$$ as tropical cycles. 
\end{prop}

Since tropical manifolds are defined as spaces with an atlas of charts in which the transition functions are invertible integer affine maps, Proposition \ref{thm:csmisomorphism} ensures the following notion of CSM cycles for tropical manifolds is well defined.

\begin{defi}\label{def:csmManifold}
The $k$-th \textbf{Chern-Schwartz-MacPherson cycle} $\csm_k(X)$ of a tropical manifold $X$ is the tropical cycle supported on $X$ such that, in each chart $\varphi_{\alpha} \colon U_{\alpha} \to X_{\alpha}$, the image of $\csm_k(X)$ is $\csm_k(X_{\alpha}) \cap \varphi_{\alpha} (U_{\alpha})$. 
\end{defi}

\begin{rem}
We remark that the CSM cycles of products $\Sigma_M \times \T^{r}$, where $\Sigma_{M} \subset \R^n$ is a matroid fan, are transverse to the boundary of $\R^{n} \times \T^{r}$, and hence CSM cycles of a tropical manifold $X$ are transverse to its boundary. 
\end{rem}

Before proving Proposition \ref{thm:csmisomorphism}, we provide an interesting class of examples of non-isomorphic matroids whose matroidal cycles are related by invertible integer linear maps.

\begin{defi}\label{defi:parallel}
Let $M_1$ and $M_2$ be two loopless matroids on the ground sets $E_1$ and $E_2$, respectively, and let $p_1 \in E_1$ and $p_2 \in E_2$. The {\bf parallel connection} $P(M_1,M_2)$ of $M_1$ and $M_2$ at the basepoints $p_1$ and $p_2$ is a matroid on the ground set $E := (E_1 - p_1) \sqcup (E_2 - p_2) \sqcup \{p\}$. If we make the identification $p_1 = p_2 = p$, the flats of $P(M_1,M_2)$ are those $F \subseteq E$ such that $F \cap E_1$ is a flat of $M_1$ and $F \cap E_2$ is a flat of $M_2$. Equivalently, the circuits of $P(M_1,M_2)$ are the subsets of $C \subseteq E$ such that $C$ is a circuit of $M_1$ or $M_2$, or $C = I_1 \sqcup I_2$ with $I_i \sqcup \{p\}$ a circuit of $M_i$; see \cite[Proposition 7.6.6]{White1}.
\end{defi}

In general, the isomorphism class of the parallel connection of two matroids $M_1$ and $M_2$ depends on the choice of basepoints. It follows from the next proposition that the matroidal cycles of any two parallel connections of $M_1$ and $M_2$ are related by an invertible integer affine map, even though their underlying matroids might not be isomorphic. 


\begin{prop}\label{prop:parallelBergman}
If $M$ is the parallel connection of the matroids $M_1$ and $M_2$ (at any basepoints) then there exists an invertible map
$\varphi: Z_M \to Z_{M_1} \times Z_{M_2}$, which is the restriction of an invertible integer linear map between the ambient vector spaces.  
\end{prop}
\begin{proof}
Let $E_1$ and $E_2$ be the ground sets of $M_1$ and $M_2$, respectively. Suppose $M$ is the parallel connection of $M_1$ and $M_2$ at the base points $p_1 \in E_1$ and $p_2 \in E_2$, and let $E := (E_1 - p_1) \sqcup (E_2 - p_2) \sqcup \{p\}$ be the ground set of $M$. Consider the vector spaces $V_1 := \RR^{E_1}$, $V_2 := \RR^{E_2}$, and $V := \RR^{E}$, and let $\Delta_1 = (e_a)_{a \in E_1}$, $\Delta_2 = (e_b)_{b \in E_2}$, and $\Delta = (e_c)_{c \in E}$ denote their standard bases, respectively. Let $\varphi: V/ \RR \cdot e_{E} \to (V_1/ \RR \cdot e_{E_1}) \oplus (V_2/ \RR \cdot e_{E_2})$ be the map defined by $\varphi(\bar e_x) := \bar e_x$ if $x \in E - p$ and $\varphi(\bar e_p) = \bar e_{p_1} + \bar e_{p_2}$. The function $\varphi$ is an isomorphism of vector spaces; in fact, its inverse is given by $\varphi^{-1}(\bar e_x) = \bar e_x$ if $x \in (E_1 - p_1) \cup (E_2 - p_2)$, and $\varphi^{-1}(\bar e_{p_i}) = -\bar e_{E_i-p_i}$ for $i=1,2$. 

The support of the matroidal cycle
$Z_M$ of any matroid $M$  can be described as
\[\textstyle \bigl|Z_M\bigr| = \bigl\{\sum \alpha_i \bfe_i : 
\forall \text{ circuits $C$ of $M$, } \min \{\alpha_i : i \in C\} \text{ is achieved at least twice} \bigr\},\] see \cite[Theorem 4.2.6]{MaclaganSturmfels}. 
Using the description of the circuits of $M$ from Definition \ref{defi:parallel}, it is easy to check that for any $x \in V/ \RR \cdot e_{E}$ we have $x \in |Z_M|$ if and only if $\varphi(x) \in |Z_{M_1} \times Z_{M_2}|$, as desired.
\end{proof}



\begin{conj}\label{conj:iso}
If $M$ and $M'$ are non-isomorphic matroids and there exists an invertible map  $\phi: Z_M \to Z_{M'}$  which is the restriction of an invertible integer linear map between their ambient spaces, then $M$ and $M'$ are parallel connections.   
\end{conj} 


We will prove Proposition \ref{thm:csmisomorphism} by 
giving an alternate description of the weights of the CSM cycles; see Lemma \ref{lem:sameweights}. 
It will be clear that this description is invariant under invertible integer affine maps. 
For this purpose, we make use of the sheaves arising from tropical homology \cite{IKMZ}.


\begin{defi}\label{def:Fp}
Let $\f$ be a pure $d$-dimensional polyhedral fan in an $\RR$-vector space $V$. 
For $0\leq p\leq d$, define 
\[\Fs_p(\f) := \langle v_1 \wedge \dots \wedge v_p : v_1, \dotsc , v_p \text{ are in a common cone } \sigma \in \f \rangle \, \subseteq  \, \bigwedge\nolimits^p V.\]
By convention, if $\f \neq \emptyset$ then $F_0(\f)$ is the one-dimensional vector space $\bigwedge\nolimits^0 V$, and if $\f = \emptyset$ then $F_0(\f) = 0$. 
\end{defi}

\begin{lemma}\label{lem:invarianceFp}
Let $\f$ be a polyhedral fan in an $\RR$-vector space $V$. Then the vector spaces $\Fs_p(\f)$ from Definition \ref{def:Fp} are invariants of the support of $\Sigma$. 
Moreover, if $\f'$ is a polyhedral fan in an $\RR$-vector space $V'$ and $\phi: \f \to \f'$
is an invertible map induced by a linear map $\hat{\phi}: V \to V'$ then $\dim \Fs_p(\f) = \dim \Fs_p(\f')$ for all $p$. 
\end{lemma}

\begin{proof}
To prove that $\Fs_p(\f)$ is an invariant of the support of $\f$, it suffices to show that $\Fs_p(\f) = \Fs_p(\f')$ for $\f'$ a refinement of $\f$. Given a cone $\sigma \in \f$, let $L(\sigma)$ denote its linear span. Note that
$$\Fs_p(\f) = \big \langle \bigwedge  ^p L(\sigma) : \sigma \in \f \rangle.$$
By definition of a refinement, every cone of $\f^{'}$ is contained in a cone of $\f$ and thus we have an immediate inclusion $\Fs_p(\f') \subset \Fs_p(\f)$. 
On the other hand, given any cone $\sigma \in \f$ there must exist a cone $\sigma' \in \f^{'}$ contained in $\sigma$ such that $\dim(\sigma') = \dim(\sigma)$ and thus $L(\sigma') = L(\sigma)$.
This provides the inclusion $\Fs_p(\f) \subset \Fs_p(\f')$. 

For the second part of the statement, 
if $\f$ and $\f'$ are linearly isomorphic as described, the isomorphism $\hat{\phi}: V \to V'$ induces an isomorphism between $\Fs_p(\f) $ and $\Fs_p(\f')$. In particular, these vector spaces have the same dimension for each $p$, as claimed. 
\end{proof}

\begin{prop}\label{prop:charOS}
If $\f = \B(M)$
is the Bergman fan of a loopless rank-$(d+1)$ matroid $M$ then the polynomial
\[{\psi}_{\f}(\lambda) := \sum_{i=0}^d \, (-1)^{i} \dim \Fs_{i}(\f) \, \lambda^{d-i}\]
is equal to the reduced characteristic polynomial $\bar \chi_M(\lambda)$ of $M$.
\end{prop}

\begin{proof}
Consider the dual vector spaces $\Fs^p(\f) := \Fs_p(\f)^*$. Together, the $\Fs^p(\f)$ form a graded algebra with the product induced by the wedge product \cite[Lemma 2]{Zharkov:Bergman}. Moreover, this algebra is naturally isomorphic to the Orlik-Solomon algebra of the matroid $M$ \cite[Theorem 4]{Zharkov:Bergman}. It follows that the dimensions of the graded pieces of this algebra are the coefficients of the reduced characteristic polynomial of $M$, as claimed.  
\end{proof}

It follows from the above proposition that we can recover the weight of the $0$-th dimensional CSM cycle of $M$ from the fan $\f = B(M)$ by setting $\lambda = 1$ in the polynomial ${\psi}_{\f}(\lambda)$. 
Our goal is to recover the weights of the faces in $\csm_k$ in a similar fashion. 

If $\f$ is a polyhedral fan in a vector space $V$ and $\tau$ is a cone of $\f$, the \textbf{star} of 
$\tau$ in $\f$ is the polyhedral fan $\starr_\tau(\f)$ consisting of all cones of the form
\[ \tilde{\sigma} := \{ \lambda(x - z) : \lambda \geq 0, x \in \sigma, \text{and } z \in \tau\}\]
for any cone $\sigma$ of $\f$ containing $\tau$ as a face.
The \textbf{lineality space} of $\f$ is the maximal linear subspace $L$ such that 
$x + L \in |\f|$ for all  $x \in |\f|$. 



\begin{lemma}\label{lem:sameweights}
Let $\f = \B(M)$ be the Bergman fan of a loopless matroid $M$.
For any $k$-dimensional cone $\sigma \in \f$ the polynomial $\psi_{{\starr_{\sigma}(\f)}}(\lambda)$
is divisible by $(\lambda -1)^{k-1}$ and
\begin{equation}\label{eqn:sameweights}
 w_{\csm_k(M)}(\sigma)
= \left.\frac{\psi_{{\starr_{\sigma}(\f)}}(\lambda)}{(\lambda -1)^{k-1}}\right|_{\lambda = 1}.
 \end{equation}
\end{lemma}

\begin{proof}
Suppose $\sigma$ is a $k$-dimensional face of $\B(M)$ corresponding to the chain of flats
$\F = \{\emptyset \subsetneq F_1 \subsetneq F_2 \subsetneq \dotsb \subsetneq F_k \subsetneq \{0,\dotsc, n\}\}$ of $M$. 
The fan $\starr_{\sigma}(\f)$ is the Bergman fan of the matroid 
$M_\sigma := \bigoplus_{i=0}^k M|F_{i+1}/F_i$; see for example \cite[Corollary 4.4.8]{MaclaganSturmfels}.
By Proposition \ref{prop:charOS}, the polynomial $\psi_{\starr_{\sigma}(\f)}(\lambda)$ is the reduced characteristic polynomial of the matroid $M_\sigma$.
The characteristic polynomial of $M_\sigma$ is the product of the characteristic polynomial of its components $M_i := M|F_{i+1}/F_i$, hence, 
$$\psi_{\starr_{\sigma}(\f)}(\lambda) = (\lambda -1)^{-1}\prod_{i = 0}^k \chi_{M_i}(\lambda) = (\lambda -1)^{k-1}\prod_{i = 0}^k \bar{\chi}_{M_i}(\lambda).$$ 
As $\bar{\chi}_{M_i}(1) = (-1)^{r(M_i)-1} \beta(M_i)$, in view of Definition \ref{def:chernweight} this directly implies the desired result. 
\end{proof}

We have now all the pieces for the proof of Proposition \ref{thm:csmisomorphism}.

\begin{proof}[Proof of Proposition \ref{thm:csmisomorphism}]
The result follows from Lemma \ref{lem:sameweights} together with the second part of Lemma \ref{lem:invarianceFp}, as the dimensions of the vector spaces $\Fs_p(\f)$ are invariant under invertible linear maps. 
\end{proof}




We present the following lemma which relates the CSM cycles of the different strata of  $\Sigma_M \times \T^r$ and will be useful in future sections. 

\begin{lemma}\label{lem:csmtransbdy}
Let $\sigma$ be a face of a matroid fan $\Sigma_M$ in $\R^n$, and consider 
the corresponding face $\sigma \times \T^r_I$ in $X := \Sigma_M \times \T^r$. Then for subsets $I \subseteq J$ we have 
\[w_{\csm_k(X)}(\sigma \times \T^r_I) = w_ {\csm_{k + |I| -|J|} (X)}(\sigma \times \T^r_J).\] 
\end{lemma}

\begin{proof}
We may suppose that $\dim(\sigma \times \T^r_I) = k$ otherwise the weights on both sides are $0$.
Let $N$ be the matroid corresponding to the matroid fan $\starr_{\sigma}(\Sigma_M)$.
The underlying matroid for the fan $\starr_{\sigma \times \T^r_I}(\Sigma_M \times \T^r_I)$ is $N \oplus C_{r-|I|}$, where $C_{r-|I|}$ is the matroid which is a direct sum of $r - |I|$ coloops. 
The (non-reduced) characteristic polynomials of matroids  are multiplicative under direct summation, so we obtain
$$\chi_{N \oplus C_{r-|I|}} (\lambda) = (\lambda-1)^{r-|I|} \chi_N(\lambda)  = (\lambda-1)^{|J|-|I|} \chi_{N \oplus C_{r-|J|}} (\lambda),$$
since the characteristic polynomial of a single coloop is $\lambda-1$. 
Now the formula for the weights of the CSM cycles follows 
from their description in Lemma \ref{lem:sameweights} and Proposition \ref{prop:charOS}. 
\end{proof}



One could hope that the recipe for the weights of CSM cycles using the dimensions of the vector spaces $\Fs_p(\f)$ produces balanced cycles for any balanced rational polyhedral fan $\Sigma$. However, this is not the case, as the next examples show. 
  
\begin{exa}\label{ex:unbalancedHypersurface}
Consider the fan tropical hypersurface $X \subset \RR^3$ dual to the polytope 
$$P = \text{ConvexHull}\{ (0, 0, 0), (1, 0, 0), (1, 1, 0), (0, 1, 0), (0, 0, 1)\}.$$
The fan $X$ is a $2$-dimensional fan consisting of $5$ rays, with primitive
integer directions $$(0, 0, -1), (-1, 0, 0), (0, -1, 0), (1, 0, 1), (0, 1,
1),$$ and eight $2$-dimensional faces. 
Using the definition of weights from Equation (\ref{eqn:sameweights}) on the $1$-skeleton of $X$ does not produce a balanced tropical cycle. 
If $\sigma$ is any of the $5$ rays of $X$ listed above then 
$$\psi_{{\starr_{\sigma}(\f)}}(\lambda) = \lambda^2 - 3\lambda + 2 = (\lambda-1) (\lambda-2),$$
therefore,  Equation (\ref{eqn:sameweights}) produces $w_{\csm_1(X)}(\sigma) = -1$.
From the above list of directions of these rays
we see that constant weights do not produce a balanced $1$-dimensional fan. 

However, we remark that since $X$ is a tropical hypersurface, \cite{BertrandBihan} give canonical weights on the $1$-skeleton of $X$ that make it balanced. 
\end{exa}
 
\begin{exa}
Consider the $2$-dimensional fan $X$ in $\R^4$ defined in \cite[Section 5.6]{BabaeeHuh}. 
This fan has $14$ rays, in directions
$$ e_1,  \pm e_2, \pm e_3, \pm  e_4, f_1, \pm f_2, \pm f_3, \pm f_4,$$
where $f_1, f_2, f_3, f_4$ are the rows of the matrix, 
$$\begin{array}{rrrr}
0 & 1 & 1 & 1 \\ 
1 & 0 & -1 & 1 \\ 
1 & 1 & 0 & -1 \\ 
1 & -1 & 1 & 0.
\end{array} 
$$
See \cite[Section 5.6]{BabaeeHuh} for the description of the $2$-dimensional faces of $X$. 
Upon calculating the polynomials $\psi_{{\starr_{\sigma}(\f)}}(\lambda)$, we find that  a ray $\sigma$ in a direction 
$e_1, \dots, e_4, f_1, \dots, \text{ or } f_4$ has
$w_{\csm_1(X)}(\sigma) = -1$ whereas a  ray $\sigma$ in direction 
$-e_2, -e_3, -e_4, -f_2, -f_3, \text{ or } -f_4$ has $w_{\csm_1(X)}(\sigma) = 0$. It can be checked that these weights the $1$-skeleton do not satisfy the balancing condition. 
\end{exa}



 
\begin{exa}
A naive approach to extend the definition of CSM cycles to more general balanced polyhedral fans would be to insist that the additivity property of classical 
CSM classes should also hold in the tropical situation.
Concretely, if the indicator function of a balanced polyhedral fan $\Sigma$ can be expressed as an integer linear combination of indicator functions of matroidal fans, we could try to define the CSM cycles of $\Sigma$ as the corresponding linear combination of the CSM cycles of the matroidal fans. 
However, this does not always produce well-defined tropical cycles, as the sum of the corresponding CSM cycles might depend on the chosen decomposition of $\Sigma$.

As an example, consider the $1$-dimensional fan $\Sigma$ in $\R^2$ with 1 vertex, 6 rays, and support $|\Sigma|=\bigcup_{i=0}^2\spann_\R(\bfe_i)$, 
where $\bfe_1=(1,0)$, $\bfe_2=(0,1)$, and $\bfe_0=(-1,-1)$.
The fan $\Sigma$ is balanced when equipped with weights equal to $1$ on all top dimensional faces.  
The indicator function of $\Sigma$ can be decomposed as a signed sum of indicator functions of matroidal fans in two different ways. 
On the one hand, it decomposes as the sum of $\B(U_{2, 3})$ and $\text{crem}(\B(U_{2, 3}))$ minus the origin $\mathbf{p}$, where $\text{crem} \colon \R^2 \to \R^2$ is the linear map negating both coordinates. 
On the other, it decomposes as the sum of the three lines $\B(M_i) := \spann_{\R}(\bfe _i)$
for $i = 0, 1, 2$, minus two times the origin $\mathbf{p}$. 
\begin{figure}[ht]
\begin{center}
\includegraphics[scale=0.54]{differentdecompositions}
\caption{Two distinct decompositions of a balanced fan as a (signed) sum of matroidal fans.}
\label{figdecompositions}
\end{center}
\end{figure}
However, the first decomposition yields $\csm_0(U_{2, 3}) + \text{crem}(\csm_0(U_{2, 3})) - \csm_0(U_{1,3}) = (-1-1-1) \mathbf{p} = -3 \mathbf{p}$,
while the second yields $\sum_{i=0}^2 \csm_0(M_i) - 2 \csm_0(U_{1,3}) = (0 + 0 + 0 - 2)\mathbf{p} = -2 \mathbf{p}$.
\end{exa}