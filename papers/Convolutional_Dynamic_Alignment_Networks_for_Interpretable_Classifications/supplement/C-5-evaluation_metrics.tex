In section~4.2, we evaluated the attribution methods against 2 quantitative 
metrics: 
\colornum{(1)} the adapted \emph{pointing game}~\citesupp{zhang2018top} and
\colornum{(2)} the prediction stability under removing the \emph{least important pixels} as in~\citesupp{srinivas2019full}. In section~\ref{sec:additional_quantitative}, we further show results for removing the \emph{most important pixels} first.

For \colornum{(1)}, we constructed 500 (250) $3\times3$ multi-images for CIFAR-10 (TinyImagenet); for an example with $2\times2$, see Fig.~7 in the main paper.
In each of these multi-images, every class occurred at most once. As stated in section~4.2,
    we measured the fraction of positive contributions falling inside the correct mini-image. Further, the images were sorted according to their confidence for each of the classes. For every multi-image, a random set of classes was sampled. For each of the classes, we included the most confidently classified class image in the multi-image that had not been used yet in previous multi-images.

For \colornum{(2)}, we followed~\citesupp{srinivas2019full} and successively replaced one pixel at a time by [0, 0, 0, 0, 0, 0],
    until up to 25\% of the image were removed. The pixels were removed in order, sorted by their assigned importance.