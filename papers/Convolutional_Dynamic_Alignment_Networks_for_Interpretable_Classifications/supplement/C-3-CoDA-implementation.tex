In Algorithm \ref{alg:dlcconv2d}, we present the implementation of the convolutional DAUs (CoDAUs).
As can be seen, a Convolutional Dynamic Alignment Layer applies dynamic (input-dependent) filters to each of the patches extracted at different spatial locations. In detail, the Dynamic Alignment Units are implemented as two consecutive convolutions (lines 11 and 15), which are equivalent to first applying matrix $\mat B$ (line 24) and then $\mat A$ to each patch and adding a bias term $\vec b$ (line 29).
After applying the non-linearity (line 33), we obtain the dynamic weight vectors for CoDAUs as described in eq.~(1) in the main paper.
In particular, for every patch $\vec p_{hw}$ extracted at the spatial positions $hw$ in the input, we obtain the dynamic weight $\vec w_j(\vec p_{hw})$ for the $j$-th DAU as
\begin{align}
    \vec w_j(\vec p_{hw}) = g(\mat{a}_j\mat{b}\vec p_{hw} + \vec b_j)\; ;
\end{align}
note that the projection matrices $\mat b$ are thus shared between the DAUs.
These weights are then applied to the respective locations (line 41) to yield the outputs of the DAUs per spatial location.
As becomes apparent in line 41, the outputs are linear transformations (weighted sums) of the input and can be written as
\begin{align}
    \vec a_{l+1}(\vec a_l) = \mat W(\vec a_l) \vec a_l\quad ,
\end{align}
with $\vec a_l\in \mathbb R^{d}$ the vectorised input to layer $l$ and $\mat w\in\mathbb R^{f\times d}$ and $f$ the number of filters (DAUs).
The rows in matrix $\mat w$ correspond to exactly one filter (DAU) applied to exactly one patch $\vec p_{hw}$ and are non-zero only at those positions that correspond to this specific patch in the input.
