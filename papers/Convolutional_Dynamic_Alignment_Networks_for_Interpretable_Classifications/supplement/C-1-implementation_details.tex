\section{Implementation details}
\label{sec:training}
\subsection{Training and architecture details}
\myparagraph{Architectures.} The architectures used for the experiments in section 4 are given in Table~\ref{tbl:archs}.
All activation maps are padded with $(k - 1)/2$ zeros on each side, such that the spatial dimensions are only reduced by 
    the strides; here, $k$ refers to the kernel size.
As can be seen, the activation maps thus still have a spatial resolution after the last layer, which we further reduce with a global sum-pooling layer.
Note that global sum-pooling is just a linear layer with no trainable parameters and therefore still allows for linear decomposition.
The input itself consists of 6 channels, the image and its negative, as explained in section~3.4;
    hence, the first layer takes an input with 6 channels per pixel.

\myparagraph{Training details.} We use the pytorch library~\citesupp{pytorch} and optimise all networks with the Adam optimiser~\citesupp{kingma2014adam} with default values.
As for the loss function, we use the binary cross entropy loss to optimise class probabilities individually (as `one-vs-all' classifiers).
For all networks, we used a base learning rate of $2.5 \times10^{-4}$; for the Imagenet experiment, we employed learning rate warm-up and linearly increased the learning rate from $2.5 \times10^{-4}$ to $1 \times10^{-3}$ over the first 15 epochs.
Further, we trained for 200 epochs on CIFAR-10, for 100 epochs on TinyImagenet, and for 60 epochs on the Imagenet subset; we decreased the learning rate by a factor of 2 after every 60/30/20 epochs on CIFAR-10/TinyImagenet/Imagenet.
We used a batch size of 16, 128, and 64 for CIFAR-10, TinyImagenet, and Imagenet respectively. For the Imagenet subset, we additionally used RandAugment~\citesupp{cubuk2019randaugment} with parameters $n=2$ and $m=9$; for this, we relied on the publicly available implementation at \url{https://github.com/ildoonet/pytorch-randaugment} and followed their augmentation scheme. The qualitatively evaluated model (see Figs.~5,~8, and~A1-A4) for the Imagenet subset was trained with $T=1e5$ and achieved a top-1 accuracy of 76.5\%. For comparison, we trained several ResNet-50 models (taken from the pytorch library~\citesupp{pytorch}) with the exact same training procedure, i.e., batch size, learning rate, optimiser, augmentation, etc.). The best ResNet-50 out of 4 runs achieved 79.16\% top-1 accuracy\footnote{The best test accuracies per run are given by 79.16\%, 79.04\%, 78.86\%, and 78.7\% respectively.}, which outperforms the CoDA-Net but is nevertheless comparable. While it is surely possible to achieve better accuracies for both models, long training times for the CoDA-Nets have thus far prevented us from properly optimising the architectures both on the 100 classes subset, as well as on the full Imagenet dataset. In order to scale the CoDA-Net models to larger datasets, we believe it is important to first improve the model efficiency in future work.
Lastly, when regularising the matrix entries of $\mat W_{0\rightarrow L}$, see eq.~(11),
    we regularised the absolute values for the true class $c$, $\left[\mat M_{0\rightarrow L}\right]_c$,
    and a randomly sampled incorrect class per image.
    
    
% 
% 
\subsection{Convolutional Dynamic Alignment Units}
\label{subsec:CoDAUs}
In Algorithm \ref{alg:dlcconv2d}, we present the implementation of the convolutional DAUs (CoDAUs).
As can be seen, a Convolutional Dynamic Alignment Layer applies dynamic (input-dependent) filters to each of the patches extracted at different spatial locations. In detail, the Dynamic Alignment Units are implemented as two consecutive convolutions (lines 11 and 15), which are equivalent to first applying matrix $\mat B$ (line 24) and then $\mat A$ to each patch and adding a bias term $\vec b$ (line 29).
After applying the non-linearity (line 33), we obtain the dynamic weight vectors for CoDAUs as described in eq.~(1) in the main paper.
In particular, for every patch $\vec p_{hw}$ extracted at the spatial positions $hw$ in the input, we obtain the dynamic weight $\vec w_j(\vec p_{hw})$ for the $j$-th DAU as
\begin{align}
    \vec w_j(\vec p_{hw}) = g(\mat{a}_j\mat{b}\vec p_{hw} + \vec b_j)\; ;
\end{align}
note that the projection matrices $\mat b$ are thus shared between the DAUs.
These weights are then applied to the respective locations (line 41) to yield the outputs of the DAUs per spatial location.
As becomes apparent in line 41, the outputs are linear transformations (weighted sums) of the input and can be written as
\begin{align}
    \vec a_{l+1}(\vec a_l) = \mat W(\vec a_l) \vec a_l\quad ,
\end{align}
with $\vec a_l\in \mathbb R^{d}$ the vectorised input to layer $l$ and $\mat w\in\mathbb R^{f\times d}$ and $f$ the number of filters (DAUs).
The rows in matrix $\mat w$ correspond to exactly one filter (DAU) applied to exactly one patch $\vec p_{hw}$ and are non-zero only at those positions that correspond to this specific patch in the input.

\subsection{Attribution methods}
\label{subsec:att_details}
In section~4.2, we compare the model-inherent contribution maps of the CoDA-Nets to those of the following methods for importance attribution:
 the gradient of the class logits with respect to the input image~\citesupp{baehrens2010explain} (Grad), `Input$\times$Gradient' (IxG, cf.~\citesupp{adebayo2018sanity}), GradCam~\citesupp{selvaraju2017grad} (GCam), Integrated Gradients~\citesupp{sundararajan2017axiomatic} (IntG), DeepLIFT~\citesupp{shrikumar2017deeplift},
    several occlusion sensitivities (Occ-K, with K the size of the occlusion patch)~\citesupp{zeiler2014visualizing},
    RISE~\citesupp{petsiuk2018rise}, and LIME~\citesupp{lime}.

For RISE and LIME, we relied on the official implementations available at \url{https://github.com/eclique/RISE} and 
\url{https://github.com/marcotcr/lime} respectively. For RISE, we generated 6000 masks with parameters $s=6$ and $p_1=0.1$.
For LIME, we evaluated on 256 samples per image and used the top 3 features for the attribution;
    for the segmentation, we also used the default parameters, namely `quickshift' with $max\_dist=200$, $ratio=0.2$, and a kernel size of 4.

For Grad, GCam, IxG, IntG, DeepLIFT, and the occlusion sensitivities,
    we relied on the publicly available pytorch library `captum' (\url{https://github.com/pytorch/captum}).
GCam was used on the last activation map before global sum-pooling.
The occlusion sensitivities were used with 
    strides of 2 on CIFAR-10 and strides of 4 for TinyImagenet.
Finally, for IntG we used 20 steps for the integral approximation.

% 
% 
\subsection{Evaluation metrics}
\label{subsec:metric_details}
In section~4.2, we evaluated the attribution methods against 2 quantitative 
metrics: 
\colornum{(1)} the adapted \emph{pointing game}~\citesupp{zhang2018top} and
\colornum{(2)} the prediction stability under removing the \emph{least important pixels} as in~\citesupp{srinivas2019full}. In section~\ref{sec:additional_quantitative}, we further show results for removing the \emph{most important pixels} first.

For \colornum{(1)}, we constructed 500 (250) $3\times3$ multi-images for CIFAR-10 (TinyImagenet); for an example with $2\times2$, see Fig.~7 in the main paper.
In each of these multi-images, every class occurred at most once. As stated in section~4.2,
    we measured the fraction of positive contributions falling inside the correct mini-image. Further, the images were sorted according to their confidence for each of the classes. For every multi-image, a random set of classes was sampled. For each of the classes, we included the most confidently classified class image in the multi-image that had not been used yet in previous multi-images.

For \colornum{(2)}, we followed~\citesupp{srinivas2019full} and successively replaced one pixel at a time by [0, 0, 0, 0, 0, 0],
    until up to 25\% of the image were removed. The pixels were removed in order, sorted by their assigned importance.
\begin{table*}
    \newcommand{\rot}[1]{%
\rotatebox[origin=c]{90}{#1}}
\footnotesize
\centering
    \begin{tabular}{c|c|c|c|c|c|c}
         \textbf{Network} &
         \textbf{Input dimensions} &
         \textbf{Layer} &
         \textbf{Number of DAUs} &
         \textbf{Rank of $\mat{AB}$} &
         \textbf{Kernel size} &
         \textbf{Stride} 
        %  \textbf{Dilation} 
         \\[.5em]
         \hline
         & & & & & &   \\[-.5em]
         \multirow{9}{*}{\rot{S-CoDA}}&
         \multirow{9}{*}{{$6\times 32\times 32$}}
         & 1 & 16 &  32 & 3 & 1 \\
         && 2 & 16 &  32 & 3 & 1 \\
         && 3 & 32 &  64 & 3 & 2 \\
         && 4 & 32 &  64 & 3 & 1 \\
         && 5 & 32 &  64 & 3 & 1 \\
         && 6 & 64 &  {64} & 3 & 2 \\
         && 7 & 64 &  {64} & 3 & 1 \\
         && 8 & 64 &  {64} & 3 & 1 \\
         && 9 & 10 &  {64} & 1 & 1 \\[.5em]
         \hline
         & & & & & &  \\[-.5em]
         \multirow{9}{*}{\rot{M-CoDA}}&
         \multirow{9}{*}{{$6\times 32\times 32$}}
         & 1 & 16 &  64 & 3 & 1 \\
         && 2 & 16 &  64 & 3 & 1 \\
         && 3 & 32 &  128 & 3 & 2 \\
         && 4 & 32 &  128 & 3 & 1 \\
         && 5 & 32 &  128 & 3 & 1 \\
         && 6 & 64 & {256} & 3 & 2 \\
         && 7 & 64 & {256} & 3 & 1 \\
         && 8 & 64 & {256} & 3 & 1 \\
         && 9 & 10 & {256} & 1 & 1 \\[.5em]
         \hline
         & & & & & &  \\[-.5em]
         \multirow{9}{*}{\rot{L-CoDA}}&
         \multirow{9}{*}{$6\times 240\times 240$}
         & 1 & 16 &   {64} & 7 & 3 \\
         && 2 & 32 &   {64} & 3 & 1 \\
         && 3 & 32 &  {64} & 3 & 1 \\
         && 4 & 64 &  {128} & 3 & 2 \\
         && 5 & 64 &  {128} & 3 & 1 \\
         && 6 & 64 &  {128} & 3 & 1 \\
         && 7 & 64 &  {256} & 3 & 2 \\
         && 8 & 64 &  {256} & 3 & 1 \\
         && 9 & 100 &  {256} & 3 & 1 \\[.5em]
         \hline
         & & & & & &  \\[-.5em]
         \multirow{9}{*}{\rot{XL-CoDA}}&
         \multirow{9}{*}{$6\times 64\times 64$}
         & 1 & 16 &   {64} & 5 & 1 \\
         && 2 & 32 &   {64} & 3 & 1 \\
         && 3 & 32 &  {128} & 3 & 2 \\
         && 4 & 64 &  {128} & 3 & 1 \\
         && 5 & 64 &  {128} & 3 & 1 \\
         && 6 & 64 &  {256} & 3 & 2 \\
         && 7 & 64 &  {256} & 3 & 1 \\
         && 8 & 64 &  {256} & 3 & 1 \\
         && 9 & 200 &  {256} & 3 & 2 \\
    \end{tabular}
\normalsize
    \vspace{1em}
    \caption{Architecture details for the experiments described in section 4.}
    \label{tbl:archs}
\end{table*}
{\centering
\RestyleAlgo{ruled}\LinesNumbered\setlength{\algomargin}{1.5em}
\begin{algorithm*}[htpb]
\caption{Implementation of a Convolutional Dynamic Alignment Layer}
  \label{alg:dlcconv2d}
    \setstretch{1.25}
  \DontPrintSemicolon
  \newcommand{\mycomment}[1]{{\color[RGB]{112, 128, 144}\textit{\# #1}}\;}
  \newcommand{\self}{{\bf\color[RGB]{0,24,128}{self}}}
  \newcommand{\pykey}[1]{{\bf\color[RGB]{0,128,24}{#1}}}
  \newcommand{\pyword}[1]{{\bf\color[RGB]{197,117,50}{#1}}}
  \SetKwFunction{FMain}{DAUConv2d(nn.Module)}
  \SetKwFunction{Finit}{\_\_init\_\_}
  \SetKwFunction{Ffwd}{forward}
  \SetKwProg{Fn}{\pykey{class}}{:}{}
  \SetKwProg{Imp}{}{}{}
  \SetKwProg{Df}{\pykey{def}}{:}{}
  \Imp{\normalfont \pykey{from} torch \pykey{import} nn}{}\vspace{-.25em}
  \Imp{\normalfont \pykey{import} torch.nn.functional \pykey{as} F}{}\vspace{-.6em}\;\vspace{-.25em}
  \Fn{\FMain}{\;
    \Df{\Finit{\textit{\self, in\_channels, out\_channels, rank, kernel\_size, stride, padding, act\_func}}}{
        \mycomment{act\_func: non-linearity for scaling the weights. E.g., L2 or SQ.}
        \mycomment{out\_channels: Number of convolutional DAUs for this layer.}
        \mycomment{rank: Rank of the matrix $\mat{AB}$.}
        %
        \mycomment{`dim\_reduction' applies matrix $\mat b$.}
        \self.dim\_reduction = nn.Conv2d(\textit{in\_channels, rank, kernel\_size, stride, padding}, bias=\pyword{False})\;
        \mycomment{Total dimensionality of a single patch}
        \self.patch\_dim = in\_channels $\ast$ kernel\_size $\ast$ kernel\_size\;
        \mycomment{`weightings' applies matrix $\mat a$ and adds bias $\vec b$.}
        \self.weightings = nn.Conv2d(\textit{rank, out\_channels $\ast$ \self.patch\_dim , kernel\_size=1, bias=\pyword{True}})\;
        \self.act\_func = act\_func\;
        \self.out\_channels = out\_channels\;
        \self.kernel\_size = kernel\_size\;
        \self.stride = stride\;
        \self.padding = padding\;
        }\;
        \Df{\Ffwd{\textit{\self, in\_tensor}}}{
        \mycomment{Project to lower dimensional representation, i.e., apply matrix $\mat B$. This yields $\mat B\vec p$ for every patch $\vec p$.}
        reduced = \self.dim\_reduction(in\_tensor)\;
        \mycomment{Get new spatial size height h and width w}
        h, w = reduced.shape[--2:]\;
        batch\_size = in\_tensor.shape[0]\;
        \mycomment{Apply matrix $\mat A$ and add bias $\vec b$,
        yielding $\mat a\mat b\vec p + \vec b$ for every patch $\vec p$.}
        weights = \self.weightings(reduced)\;
        \mycomment{Reshape for every location to size \textit{patch\_dim}$\times$out\_channels}
        weights = weights.view(batch\_size, \self.patch\_dim, out\_channels, h, w)\;
        \mycomment{Apply non-linearity to the weight vectors, yielding $\vec w(\vec p) = g(\mat{ab} \vec p + \vec b$) as in eq.~(1) for every patch $\vec p$.}
        weights = \self.act\_func(weights, dim=1)\;
        \mycomment{Extract patches from the input to apply dynamic weights to patches.}
        patches = F.unfold(in\_tensor, \self.kernel\_size, padding=\self.padding, stride=\self.stride)\;
        \mycomment{Reshape for applying weights.}
        patches = patches.view(batch\_size, \self.patch\_dim, 1, h, w)\;
        \mycomment{Apply the weights to the patches.}
        \mycomment{As can be seen, the output is just a weighted combination of the input, i.e., a linear transformation.}
        \mycomment{The output can thus be written as $\vec o = \mat W(\vec x)\vec x$.}
        \KwRet\ (patches $\ast$ weights).sum(1)}
        }\end{algorithm*}
}