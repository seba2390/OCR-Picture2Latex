\section{Discussion and conclusion}
\label{sec:discussion}
In this work, we presented a new family of neural networks, the CoDA-Nets, 
and show that they are performant classifiers with a high degree of interpretability.
For this, we first introduced the Dynamic Alignment Units, which model their output as a dynamic linear transformation of their input and have a structural bias towards alignment maximisation.
    Using the DAUs to model filters in a convolutional network, we obtain the Convolutional Dynamic Alignment Networks (CoDA-Nets).
The successive linear mappings by means of the DAUs within the network make it possible to linearly decompose the output into contributions from individual input dimensions. 
In order to assess the quality of these contribution maps,
    see eq.~\eqref{eq:contrib}, we compare against other attribution methods.
    We find that the CoDA-Net contribution maps consistently perform well under commonly used quantitative metrics.
    Beyond their \emph{interpretability},
        the CoDA-Nets constitute performant classifiers: their accuracy on CIFAR-10 and the TinyImagenet dataset are on par to the commonly employed VGG and ResNet models.
