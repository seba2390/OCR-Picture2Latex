\subsection{Setup and model performance}
\label{subsec:accuracy}
\begin{table}
    \centering
    {\setlength{\tabcolsep}{0.25em}\setlength\extrarowheight{-1pt}
    \begin{tabular}{>{\centering\arraybackslash} >{\centering\arraybackslash}p{2.85cm} | >{\centering\arraybackslash}p{.75cm} >{\centering\arraybackslash}p{.1cm}  >{\centering\arraybackslash}p{2.85cm} | >{\centering\arraybackslash}p{.75cm} }
    \footnotesize \textbf{Model} & \footnotesize \textbf{C10}
    && \footnotesize \textbf{Model} & \footnotesize \textbf{T-IM}\\[.2em]
    \cline{1-2}
    \cline{4-5}
    & \footnotesize&& & \footnotesize\\[-.8em]
    % CIFAR10 
    \footnotesize SENNs~\cite{melis2018towards} & \footnotesize 78.5\% &
    % TinyImagenet
    & \footnotesize ResNet-34~\cite{resnet_tiny} & \footnotesize 52.0\%
    \\
    % CIFAR10
    \footnotesize VGG-19~\cite{maxgain} & \footnotesize 91.5\%&
    % TinyImagenet
    & \footnotesize VGG~16~\cite{vgg_tiny} & \footnotesize 52.2\%
    \\
    % CIFAR10
    \footnotesize DE-CapsNet~\cite{jia2020capsnet} & \footnotesize 93.0\%&
    % TinyImagenet
    & \footnotesize VGG~16  + aug \cite{vgg_tiny} & \footnotesize 56.4\%
    \\
    % CIFAR10
    \footnotesize ResNet-56~\cite{he2016deep} & \footnotesize 93.6\%&
    % TinyImagenet
    & \footnotesize IRRCNN~\cite{irrcnn_alom} & \footnotesize 52.2\%
    \\
    % CIFAR10
    \footnotesize WRN-28-2~\cite{he2016deep} & \footnotesize 94.9\%&
    % TinyImagenet
    & \footnotesize ResNet-110~\cite{rn110_tiny} & \footnotesize 56.6\%
    \\
    % CIFAR10
    \footnotesize WRN-28-2 + aug~\cite{cubuk2019randaugment} & \footnotesize 95.8\%&
    % TinyImagenet
    & \footnotesize WRN-40-20~\cite{hendrycks_tiny} & \footnotesize 63.8\%
    \\[.2em]
    \cline{1-2}\cline{4-5}
    &&&&\\[-.75em]
    % CIFAR10
    \footnotesize S-CoDA-SQ ($\lambda$)& \footnotesize 93.8\%&
    % TinyImagenet
    & \footnotesize XL-CoDA-SQ ($T$)& \footnotesize 54.4\%
    \\
    % CIFAR10
    \footnotesize S-CoDA-L2 ($\lambda$)& \footnotesize 92.6\%&
    % TinyImagenet
    & \footnotesize XL-CoDA-SQ + aug ($T$)& \footnotesize 58.4\%
    \\%\cline{3-4}
    \footnotesize S-CoDA-SQ ($T$)& \footnotesize 93.2\%&
    % TinyImagenet
    &&\\
    % CIFAR10
    \footnotesize S-CoDA-L2 ($T$)& \footnotesize 93.0\%
    % Imagenet
    &&
    \\%\cline{3-4}
    % CIFAR10
    \footnotesize M-CoDA-SQ + aug ($\lambda$)& \footnotesize 96.5\%&
    & &
    \end{tabular}}
    \caption{\small CIFAR-10 (\textbf{C10}) and TinyImagenet (\textbf{T-IM}) 
    classification accuracies. Results taken from specified references. The prefix of the CoDAs indicates model size, the suffix the non-linearity used (eq.~\eqref{eq:nonlin}). With ($\lambda$) and ($T$) we denote if models were trained with regularisation or increased temperature $T$, see eq.~\eqref{eq:loss}.}
    \label{tbl:result_table}
\end{table}

\myparagraph[-.25]{Datasets.} We evaluate and compare the accuracies of the CoDA-Nets to other work on the CIFAR-10~\cite{krizhevsky2009cifar10} and the TinyImagenet~\cite{tinyimagenet} datasets. We use the same datasets for the quantitative evaluations of the model-inherent contribution maps. Additionally, we qualitatively show high-resolution examples from a CoDA-Net trained on the first 100 classes of the Imagenet dataset. 

\myparagraph[-.25]{Models.} 
We evaluate models of four different sizes
denoted by (S/M/L/XL)-CoDA on CIFAR-10 (S and M), Imagenet-100 (L), and TinyImagenet (XL); these models have 8M (S), 28M (M), 48M (L), and 62M (XL) parameters respectively; see the supplement for an evaluation of the impact of model size on accuracy.
All models feature 9 convolutional DAU layers and a final sum-pooling layer, and mainly vary in the number of features, the rank $r$ of the DAUs,  and the convolutional strides for reducing the spatial dimensions. 
No additional methods such as residual connections, dropout, or batch normalisation are used. This 9-layer architecture was initially optimised for the CIFAR-10 dataset and subsequently adapted to the TinyImagenet and Imagenet-100 datasets. 
Further, we investigate the effect that the temperature $T$, the regularisation $\lambda$, and the non-linearities (\text{L2}, \text{SQ}, see eq.~\eqref{eq:nonlin}) have on the CoDA-Nets. Given the computational cost of the regularisation (two additional passes to extract and regularise $\mat w_{0\rightarrow L}$), evaluate the regularisation on models trained on CIFAR-10. Lastly, models marked with $T$ ($\lambda$) in Table \ref{tbl:result_table} were trained with $\lambda$$=$$0$ ($T$$=$$64$, equiv.~to `average pooling'). Details on architectures and training procedure are included in the supplement.

\myparagraph{Classification performance.} In Table \ref{tbl:result_table} we compare the performances of our CoDA-Nets to several other published results. Note that the referenced numbers are meant to be used as a gauge for assessing the CoDA-Net performance and do not exhaustively represent the state of the art. In particular, we would like to highlight that the CoDA-Net performance is on par to models of the VGG~\cite{vgg} and ResNet~\cite{he2016deep} model families on both datasets. Moreover, under the same data augmentation (RandAugment~\cite{cubuk2019randaugment}), it achieves similar results as the WideResNet-28-2~\cite{zagoruyko2016wide} on CIFAR-10.
Additionally, we list the reported results of the SENNs~\cite{melis2018towards} and the DE-CapsNet~\cite{jia2020capsnet} architectures for CIFAR-10. Similar to our CoDA-Nets, the SENNs were designed to improve network interpretability and are also based on the idea of explicitly modelling the output as a dynamic linear transformation of the input. On the other hand, the CoDA-Nets share similarities to capsule networks, which we discuss in the supplement; to the best of our knowledge, the \mbox{DE-CapsNet} currently achieves the state of the art in the field of capsule networks on CIFAR-10. 
Overall, we observed that the CoDA-Nets deliver competitive performances that are fairly robust to the non-linearity (\text{L2}, \text{SQ}), the temperature ($T$), and the regularisation strength ($\lambda$). We note that on average SQ performed better than L2, which we ascribe to the fact that SQ avoids up-scaling vectors with low norm ($||\vec v||<1$, see eq.~\eqref{eq:nonlin}).

\myparagraph[-.3]{Efficiency considerations.} 
The CoDa-Nets achieve good accuracies on the presented datasets, exhibit training behaviour that is robust over a wide range of hyperparameters, and are as fast as a typical ResNet at inference time.
However, under the current formulation and without highly optimised GPU implementations for the DAUs, training times are significantly longer for the CoDA-Nets.
While we are currently working on an improved and optimised version of CoDA-Nets, we were not yet able to generate results for the full ImageNet dataset.
On the 100 classes subset, however, the evaluated L-CoDA-SQ network achieved competitive performance (76.5\% accuracy, for details see supplement) and offers highly detailed explanations for its predictions, as we show in Figs.~\ref{fig:quality} and \ref{fig:comparison}.
