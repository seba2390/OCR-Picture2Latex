\section{Introduction}
\label{sec:Introduction}
Cyber-physical systems are composed of different components, including physical as well as computational parts. One important component of such systems is a software that glues everything together and aims to control the interactions and performs needed processing.
These systems are used in very important mission- and safety-critical domains such as military, medical, transportation, and agriculture.
Therefore, it is crucial to make sure that these software have been correctly implemented and they behave as they are expected. In particular, it is important to investigate this issue before using the system in practice.
Undoubtedly, one way to find a knowledge on what the software does is to read the code and find an insight into the implemented functionality. However, even an expert programmer cannot easily sit in a session and read the code of a software and find a clear idea on the software's functionality. More importantly, most of these systems are delivered in the format of binary codes. Hence, one needs to de-compile the binaries first, and generate its source code to be reviewed. However, through the de-compilation process, the generated source code will be obfuscated, meaning that variable names, method names, and even in some cases class names are changed to meaningless and strange names which makes it dramatically hard to follow the code.

Cyber-physical system designers use tools such as Simulink~\citeme{}, Matlab~\citeme{}, and OpenModelica~\citeme{} to design models and simulate their behavior in various situations. Then, the designed model and its corresponding simulation can be exported by these tools as a standardized portable package called \textit{Functional Mockup Unit (FMU)} that is transportable from one simulator to the other. An FMU contains binary files (e.g., compiled C files) that implement the functionality of the cyber-physical system, alongside some text-based description files which provide information about the modeled cyber-physical system (e.g., information on inputs, outputs, and other internal variables used in the system). 
In this paper, we consider the FMU of a cyber-physical system as the input to our approach. There are three main reasons behind this decision listed below:
\begin{itemize}
    \item FMU is a well-respected and standardized container for exchanging and co-simulating dynamical models.
    \item There are a vast number of modeling tools that support FMU exportation including Matlab, Simulink, Dymola, OpenModelica, SimulationX, and more.
    \item An FMU container can include both source code as well as binaries targeting different platforms.
\end{itemize}

We define the problem of this paper as: \textbf{\textit{Given an FMU of a cyber-physical system, we aim to develop an approach that generates mathematical representations of the given system. The synthesized mathematical representation shall follow the syntax and semantic of Modelica language~\citeme{}, which is a multi-domain modeling language for component-oriented modeling of complex systems and is easily interpretable by subject matter experts (SME) that are not necessarily familiar with programming languages, but are experts in understanding the behavior of cyber-physical systems.}} The synthesized mathematical representations shall be intuitive such that even a person with a non-advanced mathematical knowledge can understand the behavior of the cyber-physical system in generating the output(s) from the given input(s).

To address these challenges, in this paper, we report our efforts on automatically synthesizing a high-level mathematical representation of the behavior of the software component of cyber-physical systems, especially the control part. The ultimate goal in this project is to support non-programmer Subject Domain Experts (SMEs) to be able to easily and effortlessly understand the behavior of the given cyber-physical system to investigate whether the software works as it is expected or not. This can facilitate identification of possible drifts from the intended specification of the system and avoid unwanted behaviors (e.g., a cyber-physical system is malicious or weaponized). 


