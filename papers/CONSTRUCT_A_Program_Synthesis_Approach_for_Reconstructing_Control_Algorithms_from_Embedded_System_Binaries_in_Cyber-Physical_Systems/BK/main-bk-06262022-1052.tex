%%
%% This is file `sample-sigconf.tex',
%% generated with the docstrip utility.
%%
%% The original source files were:
%%
%% samples.dtx  (with options: `sigconf')
%% 
%% IMPORTANT NOTICE:
%% 
%% For the copyright see the source file.
%% 
%% Any modified versions of this file must be renamed
%% with new filenames distinct from sample-sigconf.tex.
%% 
%% For distribution of the original source see the terms
%% for copying and modification in the file samples.dtx.
%% 
%% This generated file may be distributed as long as the
%% original source files, as listed above, are part of the
%% same distribution. (The sources need not necessarily be
%% in the same archive or directory.)
%%
%%
%% Commands for TeXCount
%TC:macro \cite [option:text,text]
%TC:macro \citep [option:text,text]
%TC:macro \citet [option:text,text]
%TC:envir table 0 1
%TC:envir table* 0 1
%TC:envir tabular [ignore] word
%TC:envir displaymath 0 word
%TC:envir math 0 word
%TC:envir comment 0 0
%%
%%
%% The first command in your LaTeX source must be the \documentclass command.
\documentclass[sigconf,review]{acmart}
%\usepackage{subcaption}
\usepackage{color,soul}
%%
%% \BibTeX command to typeset BibTeX logo in the docs
\AtBeginDocument{%
  \providecommand\BibTeX{{%
    \normalfont B\kern-0.5em{\scshape i\kern-0.25em b}\kern-0.8em\TeX}}}

%% Rights management information.  This information is sent to you
%% when you complete the rights form.  These commands have SAMPLE
%% values in them; it is your responsibility as an author to replace
%% the commands and values with those provided to you when you
%% complete the rights form.
\setcopyright{acmcopyright}
\copyrightyear{2021}
\acmYear{2021}
\acmDOI{10.1145/nnnnnnn.nnnnnnn}

%% These commands are for a PROCEEDINGS abstract or paper.
\acmConference[ASE 2022]{The 37th IEEE/ACM International Conference on Automated Software Engineering}{October 10–14, 2022}{Ann Arbor, Michigan, United States}

\acmPrice{15.00}
\acmISBN{978-x-xxxx-xxxx-x/YY/MM}

\definecolor{codeColor}{RGB}{139,26,26}

% Define your commands here
% \newcommand{\ali}[1]{\textcolor{blue}{{\it Ali:#1}}}
\newcommand{\ali}[1]{\textcolor{blue}{{\it Ali:#1}}}
\newcommand{\shantanu}[1]{\textcolor{red Shantanu:#1}{{\it}}}
\newcommand{\code}[1]{\textcolor{codeColor}{\textsf{#1}}}
\newcommand{\concept}[1]{\textbf{\textsf{#1}}}
\newcommand{\figref}[1]{Figure~\ref{#1}}
\newcommand{\secref}[1]{Section~\ref{#1}}
\newcommand{\listref}[1]{Listing~\ref{#1}}
\newcommand{\equationref}[1]{Equation~\eqref{#1}}
\newcommand{\tableref}[1]{Table~\ref{#1}}

\renewcommand{\hl}[1]{#1}

\newcommand{\citeme}[1]{%
  \begingroup
  \definecolor{hlcolor}{RGB}{255, 226, 176}\sethlcolor{hlcolor}%
  [\textcolor{orange}{\hl{\textbf{CITE}}}]%
  \endgroup
}

\newcommand{\subsubsubsection}[1]{\paragraph{\textbf{\textit{#1:}}}}

\usepackage{tikz}
\newcommand*\circled[1]{\tikz[baseline=(char.base)]{
% \node[shape=circle,draw,inner sep=0.8pt] (char) {#1};}}
\node[shape=circle,font=\bfseries,thick,draw=black,fill=black,text=white,inner sep=0.8pt] (char) {#1};}}

\newcommand*\squared[1]{\tikz[baseline=(char.base)]{
% \node[shape=circle,draw,inner sep=0.8pt] (char) {#1};}}
\node[shape=rectangle,font=\bfseries,thin,draw=black,fill=yellow,text=black,inner sep=1pt] (char) {#1};}}


\graphicspath{{figures/}}


%%
%% Submission ID.
%% Use this when submitting an article to a sponsored event. You'll
%% receive a unique submission ID from the organizers
%% of the event, and this ID should be used as the parameter to this command.
%%\acmSubmissionID{123-A56-BU3}

%%
%% The majority of ACM publications use numbered citations and
%% references.  The command \citestyle{authoryear} switches to the
%% "author year" style.
%%
%% If you are preparing content for an event
%% sponsored by ACM SIGGRAPH, you must use the "author year" style of
%% citations and references.
%% Uncommenting
%% the next command will enable that style.
%%\citestyle{acmauthoryear}

%%
%% end of the preamble, start of the body of the document source.
\begin{document}

%%
%% The "title" command has an optional parameter,
%% allowing the author to define a "short title" to be used in page headers.
\title[A Program Synthesis Approach for Reconstructing Control Algorithms from Embedded System Binaries.]{A Program Synthesis Approach for Reconstructing Control Algorithms from Embedded System Binaries in Cyber-Physical Systems}
%%
%% The "author" command and its associated commands are used to define
%% the authors and their affiliations.
%% Of note is the shared affiliation of the first two authors, and the
%% "authornote" and "authornotemark" commands
%% used to denote shared contribution to the research.


\author{$^\dagger$Ali Shokri, $^\beta$Alexandre Perez, $^\alpha$Souma Chowdhury, $^\alpha$Chen Zeng, $^\beta$Gerald Kaloor, $^\beta$Ion Matei, $^\beta$Peter-Patel Schneider, $^\beta$Akshith Gunasekaran, $^\beta$Shantanu Rane}
\affiliation{\institution{$^\dagger$Rochester Institute of Technology, NY, USA, as8308@rit.edu} \country{}}
\affiliation{\institution{$^\beta$Palo Alto Research Center (PARC), CA, USA, \{aperez, imatei, pfps, agunasekar, srane\}@parc.com} \country{}}
\affiliation{\institution{$^\alpha$University at Buffalo, NY, USA, \{soumacho, czeng2, geraldka\}@buffalo.edu} \country{}}

\renewcommand{\shortauthors}{A. Shokri et al.}


%%
%% By default, the full list of authors will be used in the page
%% headers. Often, this list is too long, and will overlap
%% other information printed in the page headers. This command allows
%% the author to define a more concise list
%% of authors' names for this purpose.
%\renewcommand{\shortauthors}{Trovato and Tobin, et al.}

%%
%% The abstract is a short summary of the work to be presented in the
%% article.

\begin{abstract}
Cyber-physical systems are composed of physical and computational components integrated and controlled by computer-based algorithms implemented as a controller software. Due to their vast usage in mission- and safety-critical systems, it is crucial for subject matter experts (SME) to obtain a clear understanding of the behavior of the controller software to accurately assess the safety of the system, before deployment. Current approaches for supporting SMEs require skilled programmers to review the code and manually create a high-level representation of the underlying control algorithms. This is a time-consuming and error-prone task.
To address this challenge, we introduce a novel approach that synthesizes high-level mathematical representations of the controller software in cyber-physical systems, given the embedded system binary. The output of this approach can be used by SMEs for assessment of the system's compliance with expected behavior, and also for forensic applications. 
%Our approach first performs static analysis on the decompiled binary files of the controller to create the sketch of mathematical representations and then, performs an evolutionary-based search to find the correct semantic for the created representation. 
We demonstrate the effectiveness of the introduced approach in practice via three case studies conducted on real-life cyber-physical systems. 

\end{abstract}



%%
%% The code below is generated by the tool at http://dl.acm.org/ccs.cfm.
%% Please copy and paste the code instead of the example below.
%%
\begin{CCSXML}
<ccs2012>
<concept>
<concept_id>10011007.10011074</concept_id>
<concept_desc>Software and its engineering~Software creation and management</concept_desc>
<concept_significance>500</concept_significance>
</concept>
</ccs2012>
\end{CCSXML}

\ccsdesc[500]{Software and its engineering~Software creation and management}

%%
%% Keywords. The author(s) should pick words that accurately describe
%% the work being presented. Separate the keywords with commas.

% Note that keywords are not normally used for peerreview papers.
\keywords{Program Synthesis, Cyber-physical Systems, Symbolic Equations}

\settopmatter{printacmref=false}



%%
%% This command processes the author and affiliation and title
%% information and builds the first part of the formatted document.
\maketitle

% \section{Introduction}
\label{sec:Introduction}


The goal in top-$\size$ recommendation is to recommend to each
consumer a small set of $\size$ items from a large collection of
items~\cite{cremonesi2010performance}.  For example, Netflix may want
to recommend $\size$ appealing movies to each consumer.  Collaborative
Filtering (CF)~\cite{herlocker2002empirical,lee2012comparative} is a
common top-$\size$ recommendation method.  CF infers user interests by
analyzing partially observed user-item interaction data, such as user
ratings on movies or historical purchase
logs~\cite{kanagal2012supercharging}. The main assumption in CF is that
users with similar interaction patterns have similar interests.


Standard CF methods for top-$\size$ recommendation focus on making  suggestions  that accurately reflect the user's preference history. However, as  observed in previous work,  CF recommendations are generally biased toward  popular items, leading to a rich get richer effect~\cite{vargas2014improving,steck2011item}.  The major reasons for this are \textit{popularity bias} and \textit{sparsity} of CF interaction data (detailed in Section~\ref{sec:related-work}). In a nutshell, to maintain  accuracy, recommendations are generated from the dense regions of the data,  where the popular items lie.  

However,  accurately suggesting popular items, may not be satisfactory for the consumers. For example, in Netflix, an accuracy-focused movie recommender may recommend ``Star Wars: The Force Awakens'' to users who have seen ``Star Wars: Rogue One''.  But, those users are probably already aware of ``The Force Awakens''. Considering additional factors, such as novelty of recommendations,  can lead to more effective suggestions~\cite{cremonesi2010performance,Castells2015,zhang2008avoiding,ziegler2005improving,zhang2012auralist}. 
%Second, accuracy-focused models typically achieve a   overall item-space coverage across their recommendations,  whereas high item-space coverage helps providers of the items increase revenue
%, users satisfaction since they are  likely already aware of or can find these items on their own.  

Focusing on popular items also adversely affects the satisfaction of  the providers of the items. This is because  accuracy-focused models typically achieve a  low overall item space coverage across their recommendations, whereas   high item space coverage helps providers of the items increase their revenue~\cite{vargas2014improving,Castells2015,adomavicius2011maximizing,anderson2006thelongtail, yin2012challenging,adomavicius2012improving}.
%accuracy-focused models typically achieve a

In contrast to the relatively small number of popular items, there are copious  {\it long-tail\/} items that have fewer observations (e.g., ratings) available. More precisely,  using the Pareto  principle (i.e.,~the $80/20$ rule),  long-tail items can be defined as items that generate the lower $20\%$ of observations~\cite{yin2012challenging}. Experimentally we found that these items correspond to almost $85\%$ of the items in several datasets (Sections~\ref{sec:Notation} and \ref{sec:Experiments}). %Table~\ref{tab:DatasetStatsticsSmall})


As previously shown, one way to improve the novelty of top-$\size$ sets is to recommend interesting long-tail items~\cite{cremonesi2010performance,ge2010beyond}.  The intuition  is that since they have fewer observations available,  they are more likely to be unseen~\cite{Kaminskas:2016:DSN:3028254.2926720}.  
 %For example, in online commerce,  newly added items are long-tail items that are yet to be discovered.  
Moreover, long-tail item promotion also results in higher overall coverage of the item space%, which increases profits for providers of the items
~\cite{vargas2014improving,Castells2015,zhang2008avoiding,zhang2012auralist,adomavicius2011maximizing,anderson2006thelongtail,yin2012challenging,jambor2010optimizing}. Because long-tail promotion reduces accuracy~\cite{steck2011item}, there are trade-offs to be explored.


%original submitted to ICDE
%This work studies three aspects of top-$\size$ recommendation: accuracy, novelty, and item-space coverage, and examines their trade-offs. In most previous work, predictions of a base recommendation system are re-ranked to handle their trade-offs~\cite{adomavicius2012improving,jambor2010optimizing,zhang2013personalize,wang2009portfolio}. Due to performance considerations, however, these techniques are not customized per user. For example,  parameters that balance the trade-off between novelty and accuracy are cross-validated at a global level.  This can be detrimental since users have varying preferences for  objectives such as long-tail novelty. We explore how to  automatically infer  user  preference for long-tail novelty, and how to leverage  it to correct  the popularity bias in standard recommender models. Our work does not rely on any additional contextual data, although such data, if available, can help promote newly-added long-tail items~\cite{agarwal2009regression,Saveski:2014:ICR:2645710.2645751}.

This work studies three aspects of top-$\size$ recommendation: accuracy, novelty, and item space coverage, and examines their trade-offs. In most previous work, predictions of a base recommendation algorithm are \textit{re-ranked} to handle these trade-offs~\cite{adomavicius2012improving,jambor2010optimizing,zhang2013personalize,wang2009portfolio}. The re-ranking models are computationally efficient but suffer from two drawbacks. First, due to performance considerations,  parameters that balance the trade-off between novelty and accuracy  are not customized per user. Instead they are cross-validated at a global level.  This can be detrimental since users have varying preferences for  objectives such as long-tail novelty. Second,  the re-ranking methods are often limited to a specific base recommender  that may be sensitive to dataset density. 
As a result, the datasets are pruned and the problem is studied in dense settings~\cite{adomavicius2012improving,ho2014likes}; but real world  scenarios are often sparse~\cite{kanagal2012supercharging,liu2017experimental}.   
% Because  dataset density can impact the performance of most base recommenders (like R-SVD), which in turn affects the performance of the re-ranking model, 

\iffalse
We address these limitations by directly inferring  user  preference for long-tail novelty  from interaction data.  This  allows us to customize the re-ranking  per user, and design a \textit{generic} framework, which resolves the second problem. In particular, since the long-tail novelty preferences are estimated independently of any base  recommender model, we can  plug-in an appropriate base recommender w.r.t. the dataset sparsity.% including ones that are more suitable for sparse settings.  

Modelling  user  preference for  long-tail novelty using only item popularity statistics, e.g., the average popularity of rated items as in~\cite{jugovac2017efficient}, disregards additional information like whether the user found the item interesting and the long-tail preferences of other users  of the items. \iffalse To incorporate them, we introduce the notion of  \emph{item long-tail importance}. Both  user long-tail preferences and item long-tail importance are dependent:  a user has high preference for discovering long-tail items if she is interested in important long-tail items, and an item that is associated with many of these kinds of users is likely to be more important.  We propose a joint optimization framework to directly learn,  from interaction data, both the users' long-tail preferences and the  items' long-tail importance. \fi
We propose an optimization approach that  incorporates  this information and  directly learns,  from interaction data, the users' long-tail novelty preferences.

Next, we use these learned preferences  to design a  top-$\size$ recommendation framework thats is generic, and provides customized balance between accuracy, novelty, and coverage. We refer to it as framework as GANC.  Using GANC, we design a novel algorithm, {\it Ordered Sampling-based Locally Greedy (OSLG)\/}, that relies on the learned long-tail novelty preferences  to scalably correct for popularity bias. Our work does not rely on any additional contextual data, although such data, if available, can help promote newly-added long-tail items~\cite{agarwal2009regression,Saveski:2014:ICR:2645710.2645751}. In summary:
\fi

We address the first limitation by directly inferring  user  preference for long-tail novelty  from interaction data.   Estimating these  preferences  using only item popularity statistics, e.g., the average popularity of rated items as in~\cite{jugovac2017efficient}, disregards additional information, like whether the user found the item interesting or the long-tail preferences of other users  of the items. We propose an approach that  incorporates  this information and  learns the users' long-tail novelty preferences from interaction data.

This approach allows us to customize the re-ranking  per user, and  design a \textit{generic} re-ranking framework, which resolves the second limitation of prior work. In particular, since the long-tail novelty preferences are estimated independently of any base recommender, we can  plug-in an appropriate one w.r.t. different factors, such as the dataset sparsity.

Our top-$\size$ recommendation framework, \textbf{GANC}, is \textbf{G}eneric, and provides customized balance between \textbf{A}ccuracy, \textbf{N}ovelty, and \textbf{C}overage. % Moreover, based on the learned long-tail novelty preferences, we also design a novel algorithm, {\it Ordered Sampling-based Locally Greedy (OSLG)\/}, that relies on the learned long-tail novelty preferences  to scalably correct for popularity bias. 
Our work does not rely on any additional contextual data, although such data, if available, can help promote newly-added long-tail items~\cite{agarwal2009regression,Saveski:2014:ICR:2645710.2645751}. In summary:

%Consider  the following toy example:
\vspace{-0.2cm}
\begin{table}[htb]
\centering
\scriptsize
%\small
\begin{tabular}{ccccccc} 
%\toprule
%&\multirow{2}{*}{}&\multicolumn{7}{c}{Ratings}\\
& & \cellcolor{blue!35}$w_1$ &\cellcolor{blue!18} $w_2$ & $\dots$ &\cellcolor{blue!8} $w_{89}$  &\cellcolor{blue!8} $w_{99}$   
\\
&   &$i_1$&$i_2$&$\dots$&$i_{89}$&$i_{90}$\\ 
\cmidrule(r){3-7} 	 
%\midrule
\cellcolor{red!35}$\theta_1$  &$u_1 $   &5 &   & $\dots$ &  &   \\
\cellcolor{red!28}$\theta_2$  &$u_2$     &5 &    & $\dots$ &  &  \\
 $\theta_3=?$  &$\bf u_3$  &5 &  &   $\dots$ &  &  \\
\cellcolor{red!10}$\theta_4$ & $u_4$  &  &5   & $\dots$ & &\\ 
\cellcolor{red!10}$\theta_5$ & $u_5$  &  & 5  & $\dots$ & &\\ 
$\theta_6=?$  & $\bf u_6$ & &5  &      $\dots$& &  \\ 
 & & $\hdots$  &$\hdots$   &$\hdots$   &$\hdots$   &$\hdots$  \\
%\midrule 
\cmidrule(r){3-7} 	 
\multicolumn{2}{c}{item pop.}  & 3  & 3  & $\dots$ &50&60\\  
%\bottomrule
%$ f_i$    &3  &3  &1  &3  &1  &2  \\  \hline
\end{tabular}
%#.
\caption{Simplified user-item interaction data. The user long-tail novelty preference ($\theta_u$), item long-tail importance weight ($w_i$) are highlighted. Darker colors indicate larger values. } \label{tab:example}
\end{table} 
\vspace{-0.2cm}
\begin{example}  
In Table~\ref{tab:example}, we are interested in estimating $\theta_3$ and $\theta_6$,  the long-tail preference of users $u_3$ and $u_6$ who have each rated a single movie. Additional ratings for other users  are not included here.  Considering only rating information, we observe $i_1$ and $i_2$ are  equally popular $|\mathcal{U}_{i_1}^{\trainset}| = |\mathcal{U}_{i_2}^{\trainset}|=3$, and $r_{31}=5$ and $r_{62}=5$. Using Eq.~\ref{eq:tfidf-risk}  we have $\theta_3 = \theta_6$. However, if we were given the long-tail preferences of the each item's user set, specifically that $u_1$ and $u_2$ have high long-tail preference (darker red), while $u_4$ and $u_5$ have lower long-tail preference (lighter red), we could conclude $i_1$ is a more important long-tail item compared to $i_2$ (indicated by a darker blue shade for $w_1$), and we expect  $\theta_3 \geq \theta_6$.

% On the other hand, if we knew that $u_4$ and $u_5$ have lower long-tail preference, we could conclude $i_2$ is a  less significant long-tail item. Therefore, However, if we  consider the long-tail preferences of other users, we may reason differently.    We need another variable $w_i$ which captures this information. 
%we would conclude that $u_3$ has higher long-tail preference compared to $u_6$, since the users $i_1$ is a more prominent long-tail item. 

% Relying only  on item popularity information, we would  conclude   $u_3$ and $u_6$ have equal long-tail preference, since $i_1$ and $i_2$ are  equally popular. However, considering  the second column,  long-tail preference of users,  long-tail importance for each item,  which captures the long-tail preference of its users. Since  that  both users of $i_1$ have high long-tail preference while  the users of $i_2$ have lower preference,  we may conclude $i_1$ is a more important long-tail item compared to $i_2$. Therefore, $u_3$'s long-tail preference should be at least as large as $u_6$'s preference. Specifically, consider two  items $i_1$ and $i_2$, with the following rating data: $i_1=\{u_1:5, u_2:5, u_3:5 \}$, $i_2=\{u_4:5, u_5:5, u_6:5\}$.  

%Table~\ref{tab:example} shows  simplified rating data. We want an estimate of the long-tail preference of $u_3$ and $u_6$, who have each  rated a single movie.  Relying only  on movie popularity information, we would  conclude   $u_3$ and $u_6$ have similar long-tail preference, since $m_1$ and $m_2$ are  equally popular. However, considering the long-tail preferences of other users of those movies, we may reason differently: since $u_1$ and $u_2$ have high long-tail preference, and $u_4$ and $u_5$ have low long-tail preference, $m_1$ is a more prominent long-tail item compared to $m_2$. Therefore, it is likely that $u_3$ has higher long-tail preference compared to $u_6$.considering the long-tail preferences of other users of those movies, we may reason differently.  For example, 
\label{ex:running}
\end{example}



%------------------------------

\iffalse
\begin{example}
Table~\ref{tab:example} shows rating data for a simplified system. %Note the user-item interaction matrix is sparse.
For this example, we define popular movies as those that have received  three or more ratings; $\{m_1, m_2, m_4\}$ are popular and  $\{m_3, m_5, m_6\}$ are niche movies. We observe $u_1$ and $u_3$  have rated relatively popular movies (risk-averse) while $u_2$ and $u_4$ have rated niche movies (risk-loving). 
\label{ex:running}
\end{example}

\begin{table}[htb]
\centering
\scriptsize
\begin{tabular}{ccccccc} 
\toprule
			&$m_1$ &$m_2$   &$m_3$    &$m_4$   &$m_5$ &$m_6$  \\ \hline 
$u_1 $ &5  &4  & - &-  &-  &-   \\
$u_2$  &-  &-  &-  &-  &5  &5   \\
$u_3$  &-  &4  &-  &5  &-  &-   \\
$u_4$  &-  &-  &3  &-  &-  &4   \\ 
$u_5$  &5  &-  &-  &3  &-  &-   \\ 
$u_6$  &4  &2  &-  &4  &-  &-   \\ 
\bottomrule
%$ f_i$    &3  &3  &1  &3  &1  &2  \\  \hline
\end{tabular}
\caption{User-Movie rating data} \label{tab:example}
\end{table}

It is essential to consider consumer characteristics in designing recommender systems so that they promote long-tail items to the right group of users and spread demand evenly between hit and niche items.  

\fi





%------------------------------
\iffalse
\begin{table}[htb]
\centering
\scriptsize
\begin{tabular}{ccccccc} 
\toprule
			&$m_1$ &$m_2$   &$m_3$    &$m_4$   &$m_5$ &$m_6$  \\ \hline 
$u_1 $ &\textbf{5}  & \textbf{4}  &\textcolor{gray}{ 1.2} &-  &-  &-   \\
$u_2$  &-  &-  &-  &-  & \textbf{5}  &\textbf{5}   \\
$u_3$  &-  &\textbf{4}  &-  &\textbf{5}  &-  &-   \\
$u_4$  &-  &-  &\textbf{3}  &-  &-  &\textbf{4}   \\ 
$u_5$  &\textbf{5}  &-  &-  &\textbf{3}  &-  &-   \\ 
$u_6$  &\textbf{4}  &\textbf{2}  &-  &\textbf{4}  &-  &-   \\ 
\bottomrule
%$ f_i$    &3  &3  &1  &3  &1  &2  \\  \hline
\end{tabular}
\caption{User-Movie rating data} \label{tab:example}
\end{table}
% $\mathcal{P}^1= \{ \mathcal{P}_1^1 \{i_1,i_2,i_3\}, \mathcal{P}_2^1:\{i_2,i_3,i_5\}  \}$
 %$\mathcal{P}^2= \{ \mathcal{P}_1^2: \{i_1,i_2,i_3\}, \mathcal{P}_2^2:\{i_2,i_5,i_6\}  \}$
 %$\mathcal{P}^3= \{ \mathcal{P}_1^3: \{i_7,i_8,i_9\}, \mathcal{P}_2^3:\{i_{10},i_{11},i_{12}\}  \}$
\begin{table}[htb]
\centering
\tiny
\begin{tabular}{ccc} 
\toprule
		&$u_1$&$u_2$  \\ \hline 
$\mathcal{P}^1 $ & $\{i_1,i_2,i_3\}$ & $\{i_2,i_3,i_5\} $ \\
$\mathcal{P}^2$ & $\{i_1,i_2,i_3\}$ & $\{i_2,i_5,i_6\} $ \\
$\mathcal{P}^3$ & $\{i_7,i_8,i_9\}$ & $\{i_{10},i_{11},i_{12} \}$ \\
\bottomrule
%$ f_i$    &3  &3  &1  &3  &1  &2  \\  \hline
\end{tabular}
\caption{Top-$\size$ allocations to users.} \label{tab:paretoExamples}
\end{table}
\fi


\iffalse
When considering long-tail items, it is important to consider consumers' willingness  to explore niche or unpopular items and their propensity towards similar items. In particular, they can be characterized by their  {\it risk degree\/} and {\it focusing degree\/}, respectively.  We compute these estimates  based on historical rating information. The following example further describes these notions in the context of movie rating data. 

\begin{example}  
Table~\ref{tab:example} shows rating data for a simplified system with $6$ users, $6$ movies, and $3$ genres. $m_i^{j}$ implies that movie $m_i$ belongs to genre $j$. Note the user-item interaction matrix is sparse. 
  For this setting, we define popular movies as those that have received  three or more ratings; $\{m_1, m_2, m_4\}$ are popular and  $\{m_3, m_5, m_6\}$ are niche movies. We now profile the users according to their risk and focusing degree. E.g., $u_1$ has rated relatively popular movies belonging to the same genre (risk-averse, high focusing degree); $u_2$ has rated niches movies in the same genre (risk-loving, high focusing degree); $u_3$ has rated popular movies in two different genres (risk-averse, low focusing degree), and $u_4$ has rated niches movies in two different genres (risk-loving, low focusing degree). 
\label{ex:running}
\end{example}
\begin{table}[htb]
\centering
\tiny
\begin{tabular}{ccccccc} 
\toprule
			&$m_1^{1}$ &$m_2^{1}$   &$m_3^{2}$    &$m_4^{3}$   &$m_5^{3}$ &$m_6^{3}$  \\ \hline 
$u_1 $ &5  &4  &-  &-  &-  &-   \\
$u_2$  &-  &-  &-  &-  &5  &5   \\
$u_3$  &-  &4  &-  &5  &-  &-   \\
$u_4$  &-  &-  &3  &-  &-  &4   \\ 
$u_5$  &5  &-  &-  &3  &-  &-   \\ 
$u_6$  &4  &2  &-  &4  &-  &-   \\ 
\bottomrule
%$ f_i$    &3  &3  &1  &3  &1  &2  \\  \hline
\end{tabular}
\caption{User-Movie rating data} \label{tab:example}
\end{table}
It is essential to consider these consumer characteristics in designing recommender systems so that they promote long-tail items to the right group of users and spread demand evenly between the hit and niche items.  
\fi
\iffalse
\begin{center}
\begin{figure*}[tp]
%\scalebox{0.5}{%
\resizebox{1\textwidth}{!}{%
%\small%\addtolength{\tabcolsep}{5pt}% below sums to 8
\begin{tabularx}{1.5\textwidth}{>{\hsize=2.5\hsize}X>{\hsize=2.5\hsize}X>{\hsize=0.5\hsize}X>{\hsize=0.5\hsize}X>{\hsize=0.5\hsize}X>{\hsize=0.5\hsize}X>{\hsize=0.5\hsize}X>{\hsize=0.5\hsize}X}
    \multirow{12}{*}{\includegraphics[scale=0.3]{codeForExample/popularity-movie.png}} & \multirow{12}{*}{\includegraphics[scale=0.3]{codeForExample/scatterplot.png}} & & & & & & \\
%   & &               &       &       &       &       &       \\
    & &\multicolumn{1}{l|}{}               &$m_1^{g1}$   	&$m_2^{g1}$    	&$m_3^{g2}$    &$m_4^{g2}$      &$m_5^{g3}$    \\ \cline{3-8}%\hline
    & &\multicolumn{1}{l|}{u1}          &5  &5  &-  &-   &-  \\
    & &\multicolumn{1}{l|}{u2}    		&-  &-  &4  &4  &5  \\
    & &\multicolumn{1}{l|}{u3}   			&1  &2  &1  &-  &-   \\
    & &\multicolumn{1}{l|}{u4}     		&1  &-  &-  &-  &-  \\
    & &               &       &       &       &       &       \\
    & &               &       &       &       &       &       \\
    & &               &       &       &       &       &       \\
    & &               &       &       &       &       &	\\
    \\
\end{tabularx}}
\caption{User-Movie interaction data a) Popularity-Movie histogram b)Movie genres/clusters c) User-Movie rating data} \label{fig:example}
\end{figure*}
\end{center}
\fi



%We propose a novel approach that allows us to  promote long-tail items in a targeted manner, thereby improving the novelty of top-$\size$ sets, the overall item-space coverage across recommendations, while maintaining reasonable levels of accuracy.

%Next, we integrate these learned preferences  in a generic  top-$\size$ recommendation framework to provide customized balance between accuracy and coverage.

%sequentially make recommendations, while adjusting its parameters with regard to the set of top-$\size$ recommendations made so far. However, since  sequential parameter updates  cause  scalability issues, we propose a sampling based algorithm. This variant of our framework, called {\it Ordered Sampling-based Locally Greedy (OSLG)\/},  allows us to  correct for the popularity bias in recommendations with regard to individual user long-tail preferences. 

%ICDE submission
%Our framework differs with  prior work in the following aspects:  unlike~\cite{adomavicius2011maximizing,adomavicius2012improving,zhang2013personalize,ho2014likes},  the long-tail preference personalization in our framework is learned rather than optimized using cross-validation or parameter tuning. In other words, our personalization method is independent of the underlying base  recommendation models.  Moreover, our framework is  generic. This enables us to  plug-in several base recommenders, and evaluate their  effectiveness without requiring  extensive tuning for the accuracy and coverage trade-off. 


%\vspace{-2.8pt}
\begin{itemize}

\item  We examine various measures for estimating user long-tail novelty preference in Section~\ref{sec:lt-pref} and formulate an optimization problem  to directly learn users' preferences for long-tail  items from interaction data in Section~\ref{sec:learning-lt-pref}. %In addition, we introduce several heuristics for measuring the user preference for less common items from historical rating data.% 

\item  We integrate the user preference estimates into GANC %, a generic re-ranking framework that provides customized balance between accuracy, novelty, and coverage 
(Section~\ref{sec:RiskbasedReranking}), and  introduce {\it Ordered Sampling-based Locally Greedy (OSLG)\/}, a scalable algorithm that relies  on user long-tail preferences to correct the popularity bias (Section~\ref{sec:optimizationAlgorithm}).
%We introduce OSLG, a scalable algorithm that relies  on user long-tail preferences to  maximize item space coverage \textcolor{red}{while maintaining acceptable levels of accuracy} (Section~\ref{sec:optimizationAlgorithm}).

\item   We conduct an extensive empirical study and evaluate performance from  accuracy, novelty, and coverage perspectives (Section~\ref{sec:Experiments}).  We use five  datasets with varying density and difficulty levels. %:  Netflix, MovieTweetings, and MovieLens (100K, 1M, 10M). 
  In contrast to most related work,  our evaluation considers realistic settings that include a large number of infrequent  items and users. %This enables us to study the impact of  data density on the performance trade-offs of several  state of the art top-$\size$ recommendation algorithms. %   %,  and use the all-items ranking protocol~\cite{steck2013evaluation,vargas2014improving}, where performance is measured using all items with train data. to evaluate the performance of several  state of the art top-$\size$ recommendation algorithms 
 
\item Our empirical results confirm that the performance of re-ranking models is impacted by the underlying   base recommender and the dataset density. Our generic approach enables us to easily incorporate a suitable base recommender to devise an effective solution for both dense and sparse settings. In dense settings, we use the same base recommender as existing re-ranking approaches, and we outperform them in accuracy and coverage metrics. For sparse settings, we plug-in a more suitable base recommender, and devise an effective solution that is competitive with existing top-$\size$ recommendation methods in accuracy and novelty. 

%Directly estimating the long-tail novelty preferences allows us to customize re-ranking per user, and  devise a generic framework.   
 
\end{itemize}

Section~\ref{sec:related-work} describes related work. Section~\ref{sec:conclusion} concludes.

% \section{CONSTRUCT}
\label{sec:Construct}
To address the mentioned challenges in understanding the behavior of cyber-physical systems, in this paper, we introduce our novel \textbf{Code-based MOdel SyNthesiS PlaTform foR re-ConstrUcting Control AlgoriThms} \textsc{\textbf{(CONSTRUCT)}} which automatically constructs mathematical representations of controller parts of cyber-physical systems. Figure~\ref{fig:CONSTRUCT} shows an overview of this approach. 
In summary, \textsc{CONSTRUCT} takes-in an FMU file which contains binaries and some descriptions about the cyber-physical model. 
The main idea is to create the structure of the mathematical model from the decompiled codes through AST translation, and then find the correct mappings between symbolic names and the actual variable names by leveraging an evolutionary search technique, Genetic Algorithm.
From the description files inside the FMU, \textsc{CONSTRUCT} identifies the variable names that should be used in the final mathematical model and some attributes of those variables. Then, it decompiles the binary files inside the FMU and localizes the mathematical primitives that are commonly used in controller parts of Cyber-physical systems. Next, it creates the AST of the decompiled sources and translates sub-ASTs that correspond to mathematical primitives to mathematical-formed ASTs. Note that through the decompilation process, the original variable names will be replaced with non-real symbolic names. We use Genetic Algorithm as an evolutionary search-based approach to find a correct mapping between the generated symbolic names of variables in the created mathematical representation and the original variable names retrieved from the description file inside the FMU. The output of \textsc{CONSTRUCT} is a mathematical representation that complies with syntax and semantic of Modelica modeling language. The Modelica language is vastly used for mathematical modeling purposes and provides opportunity for simulating mathematical models. 
We provide details on each step of \textsc{CONSTRUCT} as follows.

\begin{figure}[ht]
  \centering
  \includegraphics[width=.25\textwidth]{figures/CONSTRUCT_new.pdf}
  \caption{An overview of the \textsc{CONSTRUCT} approach.}
  \label{fig:CONSTRUCT}
\end{figure}

\subsection{Step 1: Binary Decompilation}
\ali{To be added by Shantanu/Peter.}

\subsection{Step 2: Mathematical Primitive Isolation \ali{Localization?}}
\ali{To be added by Shantanu/Peter.}

\subsection{Step 3: Transforming Code-level ASTs to Model-level ASTs}
\ali{To be added by Shantanu/Peter.}

\subsection{Step 4: Modelica Model Synthesis}
Since the original FMU is given as an input to \textsc{CONSTRUCT}, we can provide inputs to the FMU and receive its corresponding outputs. Although we have not a clear idea on the behavior of the FMU at this point, we can still rely on the binaries inside the FMU as an important source of information that we use while synthesizing the mathematical model. We basically use this FMU as an oracle to make sure that the synthesized mathematical model correctly represents the behavior of the cyber-physical system.
From the previous steps, we were able to create the structure of the mathematical representation, including equations and symbolic variable names. Moreover, from the description file of the FMU, we were able to collect the required information (e.g., name, type, etc.) on the variables that should be used in the equations. 

In this step, we aim to find a correct mapping between variable names and symbolic variables such that by providing the same input to the given FMU and the synthesized mathematical model, their output also be the same. Note that some of the variables in equations can be independent variables (e.g., they are not derivations of other variables), some might be derivations of the independent variables or other type of calculated variables, and the rest can be parameters i.e., constant values that their value remains unchanged regardless of the value of the input or output or other variables.

\subsubsection{Correct by Testing}
%\subsubsubsection{Problem Formulation}

In order to solve the mapping problem through genetic algorithm (GA), we need to first translate the original mapping problem as a GA problem, find a proper solution, and then, translate back the solution to a solution for the mapping problem. In a GA problem, one should represent possible solutions in the form of a \textit{chromosome}, which basically is a sequence of so-called \textit{genes}. Figure~\ref{fig:GA_Problem_Formulation} shows how we formulate our mapping problem as a GA problem.   

\begin{figure*}[h]
  \centering
  \includegraphics[width=.9\textwidth]{figures/GA_Problem_Formulation.pdf}
  \caption{The genetic algorithm problem formulation. }
  \label{fig:GA_Problem_Formulation}
\end{figure*}

\begin{itemize}
    \item \textbf{Solution representation:} 
 In our approach, we consider the length of a chromosome as the same as the number of symbolic variables that are present in the mathematical model, and each gene represents a variable number from the list of the variables that we extracted from the description file of the FMU (figure~\ref{fig:GA_Problem_Formulation}(a)). For instance, if the third gene in the chromosome equals number 5, it means that we assign the fifth variable in the variable list to the third symbol in the symbol list. Following this solution representation, even in the simple equation demonstrated in \secref{sec:Motivation} that has 3 equations, 12 symbols, and 13 variables, the number of possible solutions to explore is $12!$, around $480M$. Please note that upon finding a solution (i.e., a mapping between symbols and variables), one has to generate the equation, create an FMU based on the equation, and simulate the created FMU by providing inputs to receive its corresponding outputs. This process is a mandatory part of the approach to make sure about the correctness of the synthesized mathematical representation. Due to the mentioned over head in this process, it would not be feasible to explore all the possible solutions to find the best one. In fact, we use the power of GA to avoid such exploration.     
    
    \item \textbf{First population generation:} We randomly generate the first population, meaning that random numbers of variables will be assigned to random symbols (figure~\ref{fig:GA_Problem_Formulation}(b)). However, we make sure that no two genes in a chromosome have the same variable number. This ensures that each symbol will be assigned a unique variable in the synthesized mathematical representation. Moreover, at the end of this process, all the symbols will have one variable assigned to them. 

    \item \textbf{Mutation operation:}
    In order to mutate a chromosome, we randomly select two genes in the chromosome, and swap their value. Figure~\ref{fig:GA_Problem_Formulation}(c) demonstrates this process. 
    
    \item \textbf{Cross-over operation:}
    Given two chromosomes, we randomly select a crossing point and concatenate the first part of first parent to the second part of second parent, and also the first part of second parent to the second part of the first parent to create new children from the given two parents (figure~\ref{fig:GA_Problem_Formulation}(d)). \ali{Any guarantee on not appearance of a variable more than once in the children?}
    
    \item \textbf{Fitness:} In our approach, the fitness is a little bit different from the common sense of the fitness score which is the higher the fitness is, the better is the solution. In our case, the fitness is basically the distance between the output of the generated model with the output of the given FMU. Therefore, the lower the fitness is, we consider the found solution as more valuable. 

\end{itemize}

Following the explained approach, we expect to achieve better solutions (i.e., solutions with lower errors) through next generations. In other words, starting from the early generations, we might find symbol-variable mappings such that the output of their created FMUs have a considerable distance to the output of the given FMU. However, moving forward, we would like to see that the found mappings result in mathematical representations with lower distance between their output and the output of the original FMU. Finally, we would like to see that at some point, the output of the synthesized mathematical representation completely matches the output of the given FMU. However, as will explain further, this is not always the case.

\subsubsection{Correct by Construction}

In order to calculate the fitness of the generated chromosomes, we use the \textit{OpenModelica} to simulate the synthesized mathematical model and compare the output of the simulation against the simulation output of the reference FMU. To be simulatable, the synthesized mathematical model must follow certain syntax and semantic rules of \textit{Modelica} language . Otherwise, OpenModelica throws an exception and does not return an output. Nevertheless, during the individual (i.e., chromosome) generation by the baseline \textit{Correct by Testing} method, we observed that a considerable number of created mathematical representations do not comply with the expected syntax or semantic of Modelica language. In particular, we identified the following issues that can result in creation of non-simulatable mathematical representations. To mitigate these challenge, we followed the \textit{Correct by Construction} paradigm in which we make sure that all the operations in our GA approach generate simulatable solutions. 

\begin{itemize}
    \item \textit{Different types of variables:} We observed that it is quite possible that the symbolic, as well as actual variables in the equations might be from different types (e.g. Real, Boolean, or String). In such cases, we can not arbitrarily assign a variable to a symbol without considering the type compliance constraint. For example, whenever we find a symbolic statements such as the following, we infer that the type of the variable that will be assigned to $Symbol_2$ should be of type \textit{Boolean}:
    $Symbol_1 = if(Symbol_2) Symbol_3 else Symbol_4;$
    \item \textit{At least one independent variable per equation:} To be able to simulate the generated mathematical model, the OpenModelica requires all the equations to at least include one independent variable. Otherwise, the equation would be considered as too trivial and the simulation will not be carried out. 
    \item \textit{Equal number of the independent variables and the equations:} Another constraint that OpenModelica puts on the generated mathematical models is that the total number of independent variables used in the model should be equal to the number of the equations. Otherwise, the equation system would be considered as under- (in case that the number of variables is less than the number of equations) or over-estimation (in case that the number of variables is greater than the number of equations) and the simulation will not happen.
    \item \textit{Usage of I/O variables in the equations:} Another important constraint is that all the input and output variables should be used in the equations.  
\end{itemize}

By generating chromosomes that syntactically and semantically comply with Modelica language, we basically changed the paradigm of our synthesizer from \textbf{\textit{Correct-by-Testing}} to \textbf{\textit{Correct-by-Construct}}. In other words, instead of synthesizing models that we are not sure about their soundness, we changed our synthesis approach such that it only generates mathematical models that are \textbf{sound}, i.e., the model is simulatable by the simulator. 
% \section{Case Study}
\label{sec:ExperimentalStudy}
In order to showcase the performance of the introduced approach, in this section, we conduct three case studies on real-world cyber-physical systems. In each case, we first provide information on the system, and then, we share the results of the conducted experiment. Figure~\ref{fig:correct_by_construct_result} summarizes the result of this study.   

\subsection{Case \#1: DC Motor}
\ali{Needs more information and characteristics of the system. }
The first case shown in figure~\ref{fig:correct_by_construct_result} is for a model with 8 equations, 18 variables, and 23 symbols. The orange curve represents the performance of the Correct-by-Testing approach, and the blue curve is shows the result of performing the mentioned changes and shifting the paradigm to Correct-by-Construct. Clearly, even in this case that all the variables and symbols are from the same type (Real), the Correct-by-Construct approach is able to find better mathematical representations by end of the run of the approach.

\subsection{Case \#2: PID}
However, by adding to the complexity of the models, the true power of the performed changes would be visible. \ali{Needs more information and characteristics of the system.}
In this case, there were 6 equations, 37 variables, and 20 symbols, and more than one type in the equations in the Case \#2 (PID model). In this case, the Correct-by-Testing approach was not able to generate even one single chromosome that can be simulated using the simulator. However, not only the Correct-by-Construct approach is able to generate such models, but also it finds solutions with relatively good scores.


\subsection{Case \#3: PID Lim}
\ali{Needs more information and characteristics of the system. }
Finally, Case \#3 shows a more complex system compared to Case \#1 and \#2. However, the Correct-by-Construct approach can generate simulatable solutions for this case while the Correct-by-Testing approach cannot. 


\begin{figure*}[ht]
  \centering
  \includegraphics[width=.8\textwidth]{correct_by_construct_result.pdf}
  \caption{The result of changing the synthesis paradigm from Correct-by-Testing to Correct-by-Construct.}
  \label{fig:correct_by_construct_result}
\end{figure*}






% Our experiments are focused on natural language understanding tasks. We recognize that adapting our SubChar tokenization to language generation tasks might require additional efforts, for example, we may want to avoid cases of predicting sub-character tokens that do not form complete characters. Also, evaluating the robustness of language generation models on real-world input noises may require additional benchmarks beyond those used in this paper. We leave such exploration as an interesting direction for future work. 

Another limitation is that our method is designed specifically for the Chinese language. While we hypothesize that our method can also bring benefits to other languages with ideographic symbols, such as Kanji in Japanese, we leave such investigation to future work. 


% \section{Related Work}
\label{sec: Related Work}
In this section, we introduce two representative diversity estimators and discuss the difficulties they meet when handling MMOPs. Subsequently, some existing multi-modal multi-objective optimization algorithms are reviewed.
\subsection{Review of diversity estimators}
\subsubsection{Density in SPEA2}
In SPEA2 \cite{SPEA2}, each solution is assigned a density value which is used to calculate its fitness value. Eq. (\ref{eq: Density in SPEA2}) gives the density of a solution $\boldsymbol{x}$.
\begin{equation}
	\textit{Density} (\boldsymbol{x}) = \frac{1}{\sigma_k(\boldsymbol{x}) + 2},
	\label{eq: Density in SPEA2}
\end{equation}
where $\sigma_k(\boldsymbol{x})$ is the distance from $\boldsymbol{x}$ to its $k$-th nearest neighbor in the objective space. In SPEA2, $k$ is set to the square root of the total number of solutions in the current population as a general parameter setting.

Notice that in SPEA2, higher density means worse diversity in the objective space.

\subsubsection{Crowding distance}
Crowding distance is proposed along with the NSGA-II algorithm\cite{NSGAII} to preserve the diversity of the population in the objective space. The crowding distance of a solution $\boldsymbol{x}$ is given by the average side length of the hypercube constructed by its left and right neighbors in each objective. More precisely, for each objective, the left and right neighbors of $\boldsymbol{x}$ are the solutions at the left and right positions of $\boldsymbol{x}$ for that objective (i.e., in the list obtained by sorting the population in an increasing order of the objective values of that objective). The crowding distance of all boundary solutions (i.e., best solutions in any objectives) are set to $\infty$ to ensure that they are always selected. In NSGA-II, larger crowding distance values indicate better diversity. Formally, Eq. (\ref{eq: crowding distance}) calculates the crowding distance for a solution $\boldsymbol{x}$.
\begin{equation}
	\textit{Crowding-Distance} (\boldsymbol{x}) =
	\begin{cases}
		\infty                                                                     & ,\boldsymbol{x} \text{ is a boundary solution} \\
		\frac{1}{M}\sum_{m=1}^M[f_m(\boldsymbol{x}_{rm})-f_m(\boldsymbol{x}_{lm})] & ,\text{otherwise}
	\end{cases},
	\label{eq: crowding distance}
\end{equation}
where $M$ refers to the number of objectives, and $\boldsymbol{x}_{lm}$ and $\boldsymbol{x}_{rm}$ are the left and right neighbors of solution $\boldsymbol{x}$ regarding the $m$-th objective, respectively.
\subsection{Difficulties when handling MMOPs}
In most diversity estimators in MOEAs, the solution distribution in the decision space is out of consideration, which makes them inefficient on MMOPs. As we have discussed in Section \ref{sec: Introduction}, in MMOPs, equivalent solutions have the same or almost the same objective values. Consequently, they are usually not preferable in terms of diversity (in the objective space). For this reason, diversity estimators used in MOEAs are often responsible for the loss of equivalent solutions when tackling MMOPs. Fig. \ref{fig: Difficulty when handling MMOPs} gives an example when a diversity estimator such as crowding distance produces undesirable effects. In Fig. \ref{fig: Difficulty when handling MMOPs}, $A$ and $B$ are two Pareto optimal solutions on different (but equivalent) Pareto subsets (i.e., the upper and lower dash lines in (a)). Although $A$ and $B$ have similar objective values, the decision maker may want to keep both of them since they represent different implementations (i.e., they are different in the decision space). However, a diversity estimator tends to assign bad diversity values to them due to the small difference between their objective values. As a result, some of them are likely to be removed. From this example, we can see that solutions in different regions in the decision space should be considered separately when estimating solution diversity for MMOPs. Following this idea, we propose a niching diversity estimation method in Section \ref{sec: Proposed method}.

\begin{figure}
	\centering
	\includegraphics[width=.75\textwidth]{figures/RelatedWork/Difficulties}
	\caption{Explanation of the diversity loss in the decision space caused by diversity estimators when handling an MMOP. The dash lines in (a) and (b) denote the Pareto set and Pareto front, respectively.}
	\label{fig: Difficulty when handling MMOPs}
\end{figure}

\subsection{Multi-modal multi-objective optimization algorithms}
\label{sec: Existing multi-modal multi-objective optimization algorithms}
In most state-of-the-art multi-modal multi-objective evolutionary algorithms (MMEAs), the diversity in the decision space is maintained by niching strategies. Some MMEAs extend existing niching strategies in MOEAs to enable them to maintain the diversity in the objective space as well as in the decision space. For example, in \cite{OmniOptimizer}, Deb and Tiwari proposed one of the first MMEA called Omni-optimizer which modifies the crowding distance to measure the diversity in the decision space and the objective space simultaneously. Yue et. al. proposed a particle swarm optimizer named MO\_Ring\_PSO\_SCD \cite{MO_Ring_PSO_SCD} which adopts a similar modified crowding distance and a ring topology to create a niche structure. The DNEA algorithm \cite{DNEA} applies the fitness sharing \cite{Sharing} to both decision and objective spaces and combines them into a single sharing function. Some MMEAs are proposed with dedicated niching strategies in the decision space. Tanabe et. al. proposed a decomposition-based MMEA called MOEA/D-AD\cite{MOEAD_AD} where multiple solutions can be assigned to a weight vector, and a newly generated solution only competes with other solutions which are assigned to the same weight vector and neighboring to that solution in the decision space. In our previous study \cite{MOEAD_MM}, we proposed another decomposition-based MMEA which utilizes a clearing strategy in the decision space. Some MMEAs such as the algorithms proposed in \cite{DBSCAN_MMEA} and \cite{MMOEADC} use clustering approaches to maintain the niching structure in the decision space.
% \section{Conclusion}
\label{sec:conclusion}
This paper presents a generic top-$\size$ recommendation framework for  trading-off accuracy, novelty, and coverage. To achieve this, we profile the users according to their preference for long-tail novelty. We examine various measures, and formulate an optimization problem to learn these user preferences from interaction data.  We then integrate the user preference estimates in our generic framework, GANC.  Extensive experiments on several datasets confirm that there are trade-offs between accuracy, coverage, and novelty. Almost all re-ranking models increase coverage and novelty at the cost of accuracy. However, existing re-ranking models typically rely on rating prediction models, and are hence more effective in dense settings. Using a generic approach, we can easily incorporate a suitable base accuracy recommender to devise an effective solution for both sparse and dense settings.  %Our results  also indicate there is no single method that outperforms other methods in all metrics. However our techniques obtain a significant improvement in coverage, while  . 
Although we integrated the  long-tail novelty preference estimates into a re-ranking framework, their use-case is not limited to these frameworks. In  the future, we intend to explore the temporal and topical dynamics of long-tail novelty preference, particularly in settings where contextual information is  available.  
%We achieve these objectives without using any additional contextual information.


\iffalse
While we focused on promoting long-tail items across users, we did not consider diversity of individual top-$\size$ recommendations, a factor that has been shown to positively affect consumer satisfaction. This is one direction for future work. Moreover, the sequential setting  in our work, creates a dependency between different batches, where,  the items recommended to a batch of users, depends on those recommended to previous batches. This dependency is created through the parameter $\mathbf{f}$, that is updated every time a top-$\size$ set  is allocated to a batch of users. A future direction for our work is to estimate a distribution over $\mathbf{f}$ that allows us to independently solve the problem for each user, leading to improvements across all performance metrics, including recommendation time. 

We design algorithms that take advantage of the structure in the value functions to obtain both efficient and scalable solutions. 
We design an algorithm that takes advantage of the structure in the value functions to obtain both efficient and scalable solutions. 

\textcolor{red}{Our  sequential  algorithms can be applied for batch recommendation contexts,~e.g., personalized email marketing, where based on prior interaction data between users and items,  a new round of recommendations must be sent to all users in the system.  However, the independent coverage algorithms lift the sequential setting restrictions and allow it be applied for re-ranking the output of base recommender in any setting. }A future direction for our work is to incorporate explicit diversity metrics in the framework. 
\fi


%We have a presented a submodular maximization framework to systematically trade-off relevance and diversity in recommendations to individual users and coverage across the item-space. This ensures both consumer and producer satisfaction. We model users according to their risk and focusing degrees and promote long-tail items to the right group of consumers. Consequently, we obtain a significant improvement in coverage while maintaining reasonable levels of user satisfaction. Furthermore, our methods are able to achieve a more balanced distribution across the set of recommended items. In the future, we plan to investigate the effect of using alternative base recommender systems. 

%Future Work
%However most of these methods assume that the ratings are missing at random (MAR). Since our method of generating recommendations is based on the completed matrix, assuming MAR might introduce additional bias, we will use methods which assume that the ratings at missing not at random (MNAR),explored in~\cite{steck2010training, icml2014c2_hernandez-lobatob14}. 	 
%Long Tail %Recently, authors in~\cite{cremonesi2010performance} conducted extensive experiments to evaluate the performances of various matrix factorization-based algorithms and neighborhood models on the task of recommending long tail items. Their experimental results show that long tail recommendation leads to a decrease in accuracy for all algorithms. They also showed that for this task, SVD outperforms other state-of-the-art algorithms. 


\begin{figure*}[h]
  \centering
  \includegraphics[width=.97\textwidth]{figures/CONSTRUCT_4.pdf}
    \vspace{-6pt}

  \caption{An overview of the \textsc{CONSTRUCT} approach.}
  \label{fig:CONSTRUCT}
          \vspace{-6pt}
\end{figure*}

\section{Background and the Problem}
\label{sec:Introduction}
Cyber-physical systems (CPS) consist of different components, including physical as well as computational parts. A key component of such systems is the embedded software that orchestrates interactions between different parts, performs needed processing, and implements the controlling algorithms prescribed by the system designer.
%
CPS designers use different tools, including Simulink~\cite{documentationsimulation} and Modelica~\cite{fritzson2006openmodelica} to design the system and perform corresponding simulations. They then export the designed model as a standardized portable container called a  \textit{Functional Mockup Unit (FMU)}~\cite{blochwitz2011functional} which is transportable between different tools. An FMU contains binary files (e.g., compiled C files) that implement the functionality of the CPS (e.g., the controller part), alongside some text-based description files which provide information about the inputs, outputs, and other variables used in the binary files. 
%
Since a CPS can be used in mission- and safety-critical domains such as military, medical, transportation, and agriculture \cite{serpanos2018cyber}, it is crucial for subject matter experts (SME) to gain a clear and accurate understanding of the system's behavior to better assess the safety of the system before deployment.
%and prevent any damaging threats in the run-time (e.g., the system is malicious or weaponized). 
%
However, not only the SMEs are often unfamiliar with low-level programming languages, in forensic applications, only the system's binary is available \cite{ming2016straighttaint}. Therefore, it is necessary to derive an interpretable model from the binaries which accurately represents the functionality of the underlying code~\cite{shbita2022automated}. 
%For instance, this functionality could be a widely used proportional–integral–derivative (PID) controller~\citeme{} that is a control loop mechanism leveraging feedback.     
%
There have been approaches developed by researchers to create such representations of the code to assist SMEs, however, these techniques heavily rely on intelligent experts (e.g., professional programmers) to manually review either the source code or the decompiled binaries and create an initial abstraction of the software~\citeme{}. This is a non-trivial, time-consuming, and error-prone task. 
%In addition, the controller part of a cyber-physical system in the FMU is mostly in form of a binary file. Hence, intelligent experts need to de-compile the binaries to be able to manually review the code. However, through the de-compilation process, the generated source code will be obfuscated, meaning that variable names, method names, and even in some cases class names are changed to meaningless and strange names which makes it dramatically hard to follow and understand the code.

To address these challenges, we introduce a novel approach for automatically synthesizing a high-level mathematical representation of controller software in a CPS. This representation enables SMEs to easily identify possible drifts from the expected behavior of CPS and avoid damaging threats (e.g., malicious or weaponized software). The synthesized mathematical representation follows the syntax and semantic of Modelica language~\cite{fritzson1998modelica}, which is a multi-domain modeling language and is easily interpretable by subject matter experts (SME). 
\vspace{-15pt}
%
\section{Approach}
\label{sec:Construct}
Figure~\ref{fig:CONSTRUCT} shows an overview of our approach, called \textbf{Code-based MOdel SyNthesiS PlaTform foR re-ConstrUcting Control AlgoriThms} \textsc{\textbf{(CONSTRUCT)}}, which automatically constructs mathematical representations of controller parts of CPS. \textsc{CONSTRUCT} takes in an FMU file which contains binaries and some descriptions about the CPS. 
The main idea is to create the \textbf{structure} of the mathematical representation by translating the symbolic ASTs of decompiled C binary codes to symbolic ASTs in Modelica language. Then, leverage an evolutionary search technique to find correct actual variable names for each symbolic names and add the \textbf{semantic} to the created structure. In the following, we provide more details on each step.

    \squared{1} \textbf{Decompile Binaries:} From the description files inside the FMU, \textsc{CONSTRUCT} identifies the variable names and their attributes (e.g., type, initial value, etc.) that should be used in the final mathematical model. It also decompiles the binary files inside the FMU using Ghidra~\cite{Ghidra} as an off-the-shelf popular decompiler. The decompiled code is contains symbolic variable names. 
    
    \squared{2} \textbf{Isolate Mathematical Primitives:} Next, based on its rule-based engine, CONSTRUCT localizes the mathematical primitives that are commonly used in controller parts of CPS. Some mathematical primitives might differ in the decompiled code comparing to the original source code. CONSTRUCT is able to identify these primitives as well. 
    
    \squared{3} \textbf{Code-level AST to Model-level AST:} In this step, CONSTRUCT creates ASTs of the decompiled binaries (C files) and translates the sub-ASTs that correspond to localized mathematical primitives (previous step) to mathematical-formed ASTs. The mathematical-formed ASTs follow syntax of Modelica language. Note that due to the decompilation, the created mathematical-formed ASTs contain non-real symbolic names. %Therefore, we need to find the correct mapping between symbolic names and the actual variable names collected through the first step.
    
    \squared{4} \textbf{Modelica Model Synthesis} We use Genetic Algorithm (GA) as an evolutionary search-based approach to find a correct mapping between the generated symbolic variable names in the created mathematical representation and the original variable names retrieved from the description file inside the FMU. We follow the \textbf{correct-by-construction} paradigm throughout this process meaning that we carefully design GA operators (i.e., first population generation, mutation, and cross-over) such that every generated chromosome (a solution candidate) by the GA approach complies with syntax and semantic of Modelica language. This significantly prunes the search space and reduces try and errors. For instance, if the type of variable $v_1$ is \textit{Real} but the type of symbol $s_1$ is \textit{Boolean}, then, our correct-by-construction approach does not consider mapping $v_1$ to $s_1$. In contrast, the baseline GA approach (\textit{correct-by-testing}) does not take these constraints into account and might generate faulty solutions that result in crashes while simulating the synthesized model on simulators. %The case study result (section~\ref{sec:PreliminaryResult}) shows how our correct-by-construction approach dramatically improves the quality of the solutions generated by GA in comparison with the base-line \textit{correct-by-testing} approach.
 
The output of \textsc{CONSTRUCT} is a mathematical representation of the controller part of CPS in form of  Modelica language. 

\section{Case Study}
\label{sec:PreliminaryResult}
In order to showcase the performance of the introduced approach, we present case studies conducted on three different controllers used in two real-world CPS, a \textit{Turtlebot Waffle Pi}~\citeme{} which is an inherently stable wheeled robot, and a \textit{PX4 Quadcopter}~\citeme{} which is statically unstable robot. The complexity of controllers increases by the number of cases i.e., the synthesized mathematical representation for Case 1 consists of 8 equations, 18 variables, and 23 symbols, and the mathematical representations for Case 2 and Case 3 contain 6 equations, 37 variables, and 20 symbols and 13 equations, 80 variables, and 39 symbols respectively. Also, the type of all variables and symbols in Case 1 are of type Real, whilst the types of variables and symbols in Case 2 and Case 3 are of different types, making the correct mapping finding problem much harder in the later two cases. In our experiments, the population size for the GA was considered as 400 and the maximum number of generations was 10.
%
\begin{figure}[ht]
  \centering
  \includegraphics[width=.47\textwidth]{Case_Study_4.pdf}
  \caption{The result of synthesizing mathematical models based on the Correct-by-Testing (base-line) and our Correct-by-Construction approach.}
  \label{fig:correct_by_construct_result}
\end{figure}
%
 Figure~\ref{fig:correct_by_construct_result} summarizes the result of this study. Even in the simplest Case (Case 1) that all the variables and symbols are from the same type (i.e., Real), the correct-by-construction approach is able to find better mathematical representations compared to the base-line (correct-by-testing) approach. Interestingly, while in complex cases (cases 2 and 3) the correct-by-testing approach is not able to generate even one single representation runnable by the simulator, the correct-by-construction approach can generate simulatable models that comply with Modelica syntax and semantic and also have relatively good fitness scores.


\section{Conclusion and Future Work}
\label{sec:Conclusion}
In this paper we introduced CONSTRUCT, a novel program synthesis approach that automatically creates mathematical representations of control algorithms in cyber-physical systems. This approach leverages a genetic algorithm for injecting the semantic into the created representation. In future work, we are interested in evaluating the utility of constraint-based solvers, such as SMT solvers, for making the model synthesis more efficient.

%%
%% The next two lines define the bibliography style to be used, and
%% the bibliography file.
\bibliographystyle{ACM-Reference-Format}
\bibliography{bibliography}

\end{document}
\endinput
%%
%% End of file `sample-sigconf.tex'.
