%%
%% This is file `sample-sigconf.tex',
%% generated with the docstrip utility.
%%
%% The original source files were:
%%
%% samples.dtx  (with options: `sigconf')
%% 
%% IMPORTANT NOTICE:
%% 
%% For the copyright see the source file.
%% 
%% Any modified versions of this file must be renamed
%% with new filenames distinct from sample-sigconf.tex.
%% 
%% For distribution of the original source see the terms
%% for copying and modification in the file samples.dtx.
%% 
%% This generated file may be distributed as long as the
%% original source files, as listed above, are part of the
%% same distribution. (The sources need not necessarily be
%% in the same archive or directory.)
%%
%%
%% Commands for TeXCount
%TC:macro \cite [option:text,text]
%TC:macro \citep [option:text,text]
%TC:macro \citet [option:text,text]
%TC:envir table 0 1
%TC:envir table* 0 1
%TC:envir tabular [ignore] word
%TC:envir displaymath 0 word
%TC:envir math 0 word
%TC:envir comment 0 0
%%
%%
%% The first command in your LaTeX source must be the \documentclass command.
\documentclass[sigconf,review]{acmart}
%\usepackage{subcaption}
\usepackage{color,soul}
%%
%% \BibTeX command to typeset BibTeX logo in the docs
\AtBeginDocument{%
  \providecommand\BibTeX{{%
    \normalfont B\kern-0.5em{\scshape i\kern-0.25em b}\kern-0.8em\TeX}}}

%% Rights management information.  This information is sent to you
%% when you complete the rights form.  These commands have SAMPLE
%% values in them; it is your responsibility as an author to replace
%% the commands and values with those provided to you when you
%% complete the rights form.
\setcopyright{acmcopyright}
\copyrightyear{2021}
\acmYear{2021}
\acmDOI{10.1145/nnnnnnn.nnnnnnn}

%% These commands are for a PROCEEDINGS abstract or paper.
\acmConference[ASE 2022]{The 37th IEEE/ACM International Conference on Automated Software Engineering}{October 10–14, 2022}{Ann Arbor, Michigan, United States}

\acmPrice{15.00}
\acmISBN{978-x-xxxx-xxxx-x/YY/MM}

\definecolor{codeColor}{RGB}{139,26,26}

% Define your commands here
% \newcommand{\ali}[1]{\textcolor{blue}{{\it Ali:#1}}}
\newcommand{\ali}[1]{\textcolor{blue}{{\it Ali:#1}}}
\newcommand{\shantanu}[1]{\textcolor{red Shantanu:#1}{{\it}}}
\newcommand{\code}[1]{\textcolor{codeColor}{\textsf{#1}}}
\newcommand{\concept}[1]{\textbf{\textsf{#1}}}
\newcommand{\figref}[1]{Figure~\ref{#1}}
\newcommand{\secref}[1]{Section~\ref{#1}}
\newcommand{\listref}[1]{Listing~\ref{#1}}
\newcommand{\equationref}[1]{Equation~\eqref{#1}}
\newcommand{\tableref}[1]{Table~\ref{#1}}

\renewcommand{\hl}[1]{#1}

\newcommand{\citeme}[1]{%
  \begingroup
  \definecolor{hlcolor}{RGB}{255, 226, 176}\sethlcolor{hlcolor}%
  [\textcolor{orange}{\hl{\textbf{CITE}}}]%
  \endgroup
}

\newcommand{\subsubsubsection}[1]{\paragraph{\textbf{\textit{#1:}}}}

\usepackage{tikz}
\newcommand*\circled[1]{\tikz[baseline=(char.base)]{
% \node[shape=circle,draw,inner sep=0.8pt] (char) {#1};}}
\node[shape=circle,font=\bfseries,thick,draw=black,fill=black,text=white,inner sep=0.8pt] (char) {#1};}}

\newcommand*\squared[1]{\tikz[baseline=(char.base)]{
% \node[shape=circle,draw,inner sep=0.8pt] (char) {#1};}}
\node[shape=rectangle,font=\bfseries,thin,draw=black,fill=yellow,text=black,inner sep=1pt] (char) {#1};}}


\graphicspath{{figures/}}


%%
%% Submission ID.
%% Use this when submitting an article to a sponsored event. You'll
%% receive a unique submission ID from the organizers
%% of the event, and this ID should be used as the parameter to this command.
%%\acmSubmissionID{123-A56-BU3}

%%
%% The majority of ACM publications use numbered citations and
%% references.  The command \citestyle{authoryear} switches to the
%% "author year" style.
%%
%% If you are preparing content for an event
%% sponsored by ACM SIGGRAPH, you must use the "author year" style of
%% citations and references.
%% Uncommenting
%% the next command will enable that style.
%%\citestyle{acmauthoryear}

%%
%% end of the preamble, start of the body of the document source.
\begin{document}

%%
%% The "title" command has an optional parameter,
%% allowing the author to define a "short title" to be used in page headers.
\title[A Program Synthesis Approach for Reconstructing Control Algorithms from Embedded System Binaries.]{A Program Synthesis Approach for Reconstructing Control Algorithms from Embedded System Binaries in Cyber-Physical Systems}
%%
%% The "author" command and its associated commands are used to define
%% the authors and their affiliations.
%% Of note is the shared affiliation of the first two authors, and the
%% "authornote" and "authornotemark" commands
%% used to denote shared contribution to the research.


\author{$^\dagger$Ali Shokri, $^\beta$Alexandre Perez, $^\alpha$Souma Chowdhury, $^\alpha$Chen Zeng, $^\beta$Gerald Kaloor, $^\beta$Ion Matei, $^\beta$Peter-Patel Schneider, $^\beta$Akshith Gunasekaran, $^\beta$Shantanu Rane}
\affiliation{\institution{$^\dagger$Rochester Institute of Technology, NY, USA, as8308@rit.edu} \country{}}
\affiliation{\institution{$^\beta$Palo Alto Research Center (PARC), CA, USA, \{aperez, imatei, pfps, agunasekar, srane\}@parc.com} \country{}}
\affiliation{\institution{$^\alpha$University at Buffalo, NY, USA, \{soumacho, czeng2, geraldka\}@buffalo.edu} \country{}}

\renewcommand{\shortauthors}{A. Shokri et al.}


%%
%% By default, the full list of authors will be used in the page
%% headers. Often, this list is too long, and will overlap
%% other information printed in the page headers. This command allows
%% the author to define a more concise list
%% of authors' names for this purpose.
%\renewcommand{\shortauthors}{Trovato and Tobin, et al.}

%%
%% The abstract is a short summary of the work to be presented in the
%% article.

\begin{abstract}
Cyber-physical systems are composed of physical and computational components integrated and controlled by computer-based algorithms implemented as a controller software. Due to their vast usage in mission- and safety-critical systems, it is crucial for subject matter experts (SME) to obtain a clear understanding of the behavior of the controller software to accurately assess the safety of the system, before deployment. Current approaches for supporting SMEs require skilled programmers to review the code and manually create a high-level representation of the underlying control algorithms. This is a time-consuming and error-prone task.
To address this challenge, we introduce a novel approach that synthesizes high-level mathematical representations of the controller software in cyber-physical systems, given the embedded system binary. The output of this approach can be used by SMEs for assessment of the system's compliance with expected behavior, and also for forensic applications. 
%Our approach first performs static analysis on the decompiled binary files of the controller to create the sketch of mathematical representations and then, performs an evolutionary-based search to find the correct semantic for the created representation. 
We demonstrate the effectiveness of the introduced approach in practice via three case studies conducted on real-life cyber-physical systems. 

\end{abstract}



%%
%% The code below is generated by the tool at http://dl.acm.org/ccs.cfm.
%% Please copy and paste the code instead of the example below.
%%
\begin{CCSXML}
<ccs2012>
<concept>
<concept_id>10011007.10011074</concept_id>
<concept_desc>Software and its engineering~Software creation and management</concept_desc>
<concept_significance>500</concept_significance>
</concept>
</ccs2012>
\end{CCSXML}

\ccsdesc[500]{Software and its engineering~Software creation and management}

%%
%% Keywords. The author(s) should pick words that accurately describe
%% the work being presented. Separate the keywords with commas.

% Note that keywords are not normally used for peerreview papers.
\keywords{Program Synthesis, Cyber-physical Systems, Symbolic Equations}

\settopmatter{printacmref=false}



%%
%% This command processes the author and affiliation and title
%% information and builds the first part of the formatted document.
\maketitle

% \IEEEraisesectionheading{\section{Introduction}}

\IEEEPARstart{V}{ision} system is studied in orthogonal disciplines spanning from neurophysiology and psychophysics to computer science all with uniform objective: understand the vision system and develop it into an integrated theory of vision. In general, vision or visual perception is the ability of information acquisition from environment, and it's interpretation. According to Gestalt theory, visual elements are perceived as patterns of wholes rather than the sum of constituent parts~\cite{koffka2013principles}. The Gestalt theory through \textit{emergence}, \textit{invariance}, \textit{multistability}, and \textit{reification} properties (aka Gestalt principles), describes how vision recognizes an object as a \textit{whole} from constituent parts. There is an increasing interested to model the cognitive aptitude of visual perception; however, the process is challenging. In the following, a challenge (as an example) per object and motion perception is discussed. 



\subsection{Why do things look as they do?}
In addition to Gestalt principles, an object is characterized with its spatial parameters and material properties. Despite of the novel approaches proposed for material recognition (e.g.,~\cite{sharan2013recognizing}), objects tend to get the attention. Leveraging on an object's spatial properties, material, illumination, and background; the mapping from real world 3D patterns (distal stimulus) to 2D patterns onto retina (proximal stimulus) is many-to-one non-uniquely-invertible mapping~\cite{dicarlo2007untangling,horn1986robot}. There have been novel biology-driven studies for constructing computational models to emulate anatomy and physiology of the brain for real world object recognition (e.g.,~\cite{lowe2004distinctive,serre2007robust,zhang2006svm}), and some studies lead to impressive accuracy. For instance, testing such computational models on gold standard controlled shape sets such as Caltech101 and Caltech256, some methods resulted $<$60\% true-positives~\cite{zhang2006svm,lazebnik2006beyond,mutch2006multiclass,wang2006using}. However, Pinto et al.~\cite{pinto2008real} raised a caution against the pervasiveness of such shape sets by highlighting the unsystematic variations in objects features such as spatial aspects, both between and within object categories. For instance, using a V1-like model (a neuroscientist's null model) with two categories of systematically variant objects, a rapid derogate of performance to 50\% (chance level) is observed~\cite{zhang2006svm}. This observation accentuates the challenges that the infinite number of 2D shapes casted on retina from 3D objects introduces to object recognition. 

Material recognition of an object requires in-depth features to be determined. A mineralogist may describe the luster (i.e., optical quality of the surface) with a vocabulary like greasy, pearly, vitreous, resinous or submetallic; he may describe rocks and minerals with their typical forms such as acicular, dendritic, porous, nodular, or oolitic. We perceive materials from early age even though many of us lack such a rich visual vocabulary as formalized as the mineralogists~\cite{adelson2001seeing}. However, methodizing material perception can be far from trivial. For instance, consider a chrome sphere with every pixel having a correspondence in the environment; hence, the material of the sphere is hidden and shall be inferred implicitly~\cite{shafer2000color,adelson2001seeing}. Therefore, considering object material, object recognition requires surface reflectance, various light sources, and observer's point-of-view to be taken into consideration.


\subsection{What went where?}
Motion is an important aspect in interpreting the interaction with subjects, making the visual perception of movement a critical cognitive ability that helps us with complex tasks such as discriminating moving objects from background, or depth perception by motion parallax. Cognitive susceptibility enables the inference of 2D/3D motion from a sequence of 2D shapes (e.g., movies~\cite{niyogi1994analyzing,little1998recognizing,hayfron2003automatic}), or from a single image frame (e.g., the pose of an athlete runner~\cite{wang2013learning,ramanan2006learning}). However, its challenging to model the susceptibility because of many-to-one relation between distal and proximal stimulus, which makes the local measurements of proximal stimulus inadequate to reason the proper global interpretation. One of the various challenges is called \textit{motion correspondence problem}~\cite{attneave1974apparent,ullman1979interpretation,ramachandran1986perception,dawson1991and}, which refers to recognition of any individual component of proximal stimulus in frame-1 and another component in frame-2 as constituting different glimpses of the same moving component. If one-to-one mapping is intended, $n!$ correspondence matches between $n$ components of two frames exist, which is increased to $2^n$  for one-to-any mappings. To address the challenge, Ullman~\cite{ullman1979interpretation} proposed a method based on nearest neighbor principle, and Dawson~\cite{dawson1991and} introduced an auto associative network model. Dawson's network model~\cite{dawson1991and} iteratively modifies the activation pattern of local measurements to achieve a stable global interpretation. In general, his model applies three constraints as it follows
\begin{inlinelist}
	\item \textit{nearest neighbor principle} (shorter motion correspondence matches are assigned lower costs)
	\item \textit{relative velocity principle} (differences between two motion correspondence matches)
	\item \textit{element integrity principle} (physical coherence of surfaces)
\end{inlinelist}.
According to experimental evaluations (e.g.,~\cite{ullman1979interpretation,ramachandran1986perception,cutting1982minimum}), these three constraints are the aspects of how human visual system solves the motion correspondence problem. Eom et al.~\cite{eom2012heuristic} tackled the motion correspondence problem by considering the relative velocity and the element integrity principles. They studied one-to-any mapping between elements of corresponding fuzzy clusters of two consecutive frames. They have obtained a ranked list of all possible mappings by performing a state-space search. 



\subsection{How a stimuli is recognized in the environment?}

Human subjects are often able to recognize a 3D object from its 2D projections in different orientations~\cite{bartoshuk1960mental}. A common hypothesis for this \textit{spatial ability} is that, an object is represented in memory in its canonical orientation, and a \textit{mental rotation} transformation is applied on the input image, and the transformed image is compared with the object in its canonical orientation~\cite{bartoshuk1960mental}. The time to determine whether two projections portray the same 3D object
\begin{inlinelist}
	\item increase linearly with respect to the angular disparity~\cite{bartoshuk1960mental,cooperau1973time,cooper1976demonstration}
	\item is independent from the complexity of the 3D object~\cite{cooper1973chronometric}
\end{inlinelist}.
Shepard and Metzler~\cite{shepard1971mental} interpreted this finding as it follows: \textit{human subjects mentally rotate one portray at a constant speed until it is aligned with the other portray.}



\subsection{State of the Art}

The linear mapping transformation determination between two objects is generalized as determining optimal linear transformation matrix for a set of observed vectors, which is first proposed by Grace Wahba in 1965~\cite{wahba1965least} as it follows. 
\textit{Given two sets of $n$ points $\{v_1, v_2, \dots v_n\}$, and $\{v_1^*, v_2^* \dots v_n^*\}$, where $n \geq 2$, find the rotation matrix $M$ (i.e., the orthogonal matrix with determinant +1) which brings the first set into the best least squares coincidence with the second. That is, find $M$ matrix which minimizes}
\begin{equation}
	\sum_{j=1}^{n} \vert v_j^* - Mv_j \vert^2
\end{equation}

Multiple solutions for the \textit{Wahba's problem} have been published, such as Paul Davenport's q-method. Some notable algorithms after Davenport's q-method were published; of that QUaternion ESTimator (QU\-EST)~\cite{shuster2012three}, Fast Optimal Attitude Matrix \-(FOAM)~\cite{markley1993attitude} and Slower Optimal Matrix Algorithm (SOMA)~\cite{markley1993attitude}, and singular value decomposition (SVD) based algorithms, such as Markley’s SVD-based method~\cite{markley1988attitude}. 

In statistical shape analysis, the linear mapping transformation determination challenge is studied as Procrustes problem. Procrustes analysis finds a transformation matrix that maps two input shapes closest possible on each other. Solutions for Procrustes problem are reviewed in~\cite{gower2004procrustes,viklands2006algorithms}. For orthogonal Procrustes problem, Wolfgang Kabsch proposed a SVD-based method~\cite{kabsch1976solution} by minimizing the root mean squared deviation of two input sets when the determinant of rotation matrix is $1$. In addition to Kabsch’s partial Procrustes superimposition (covers translation and rotation), other full Procrustes superimpositions (covers translation, uniform scaling, rotation/reflection) have been proposed~\cite{gower2004procrustes,viklands2006algorithms}. The determination of optimal linear mapping transformation matrix using different approaches of Procrustes analysis has wide range of applications, spanning from forging human hand mimics in anthropomorphic robotic hand~\cite{xu2012design}, to the assessment of two-dimensional perimeter spread models such as fire~\cite{duff2012procrustes}, and the analysis of MRI scans in brain morphology studies~\cite{martin2013correlation}.

\subsection{Our Contribution}

The present study methodizes the aforementioned mentioned cognitive susceptibilities into a cognitive-driven linear mapping transformation determination algorithm. The method leverages on mental rotation cognitive stages~\cite{johnson1990speed} which are defined as it follows
\begin{inlinelist}
	\item a mental image of the object is created
	\item object is mentally rotated until a comparison is made
	\item objects are assessed whether they are the same
	\item the decision is reported
\end{inlinelist}.
Accordingly, the proposed method creates hierarchical abstractions of shapes~\cite{greene2009briefest} with increasing level of details~\cite{konkle2010scene}. The abstractions are presented in a vector space. A graph of linear transformations is created by circular-shift permutations (i.e., rotation superimposition) of vectors. The graph is then hierarchically traversed for closest mapping linear transformation determination. 

Despite of numerous novel algorithms to calculate linear mapping transformation, such as those proposed for Procrustes analysis, the novelty of the presented method is being a cognitive-driven approach. This method augments promising discoveries on motion/object perception into a linear mapping transformation determination algorithm.



% \section{CONSTRUCT}
\label{sec:Construct}
To address the mentioned challenges in understanding the behavior of cyber-physical systems, in this paper, we introduce our novel \textbf{Code-based MOdel SyNthesiS PlaTform foR re-ConstrUcting Control AlgoriThms} \textsc{\textbf{(CONSTRUCT)}} which automatically constructs mathematical representations of controller parts of cyber-physical systems. Figure~\ref{fig:CONSTRUCT} shows an overview of this approach. 
In summary, \textsc{CONSTRUCT} takes-in an FMU file which contains binaries and some descriptions about the cyber-physical model. 
The main idea is to create the structure of the mathematical model from the decompiled codes through AST translation, and then find the correct mappings between symbolic names and the actual variable names by leveraging an evolutionary search technique, Genetic Algorithm.
From the description files inside the FMU, \textsc{CONSTRUCT} identifies the variable names that should be used in the final mathematical model and some attributes of those variables. Then, it decompiles the binary files inside the FMU and localizes the mathematical primitives that are commonly used in controller parts of Cyber-physical systems. Next, it creates the AST of the decompiled sources and translates sub-ASTs that correspond to mathematical primitives to mathematical-formed ASTs. Note that through the decompilation process, the original variable names will be replaced with non-real symbolic names. We use Genetic Algorithm as an evolutionary search-based approach to find a correct mapping between the generated symbolic names of variables in the created mathematical representation and the original variable names retrieved from the description file inside the FMU. The output of \textsc{CONSTRUCT} is a mathematical representation that complies with syntax and semantic of Modelica modeling language. The Modelica language is vastly used for mathematical modeling purposes and provides opportunity for simulating mathematical models. 
We provide details on each step of \textsc{CONSTRUCT} as follows.

\begin{figure}[ht]
  \centering
  \includegraphics[width=.25\textwidth]{figures/CONSTRUCT_new.pdf}
  \caption{An overview of the \textsc{CONSTRUCT} approach.}
  \label{fig:CONSTRUCT}
\end{figure}

\subsection{Step 1: Binary Decompilation}
\ali{To be added by Shantanu/Peter.}

\subsection{Step 2: Mathematical Primitive Isolation \ali{Localization?}}
\ali{To be added by Shantanu/Peter.}

\subsection{Step 3: Transforming Code-level ASTs to Model-level ASTs}
\ali{To be added by Shantanu/Peter.}

\subsection{Step 4: Modelica Model Synthesis}
Since the original FMU is given as an input to \textsc{CONSTRUCT}, we can provide inputs to the FMU and receive its corresponding outputs. Although we have not a clear idea on the behavior of the FMU at this point, we can still rely on the binaries inside the FMU as an important source of information that we use while synthesizing the mathematical model. We basically use this FMU as an oracle to make sure that the synthesized mathematical model correctly represents the behavior of the cyber-physical system.
From the previous steps, we were able to create the structure of the mathematical representation, including equations and symbolic variable names. Moreover, from the description file of the FMU, we were able to collect the required information (e.g., name, type, etc.) on the variables that should be used in the equations. 

In this step, we aim to find a correct mapping between variable names and symbolic variables such that by providing the same input to the given FMU and the synthesized mathematical model, their output also be the same. Note that some of the variables in equations can be independent variables (e.g., they are not derivations of other variables), some might be derivations of the independent variables or other type of calculated variables, and the rest can be parameters i.e., constant values that their value remains unchanged regardless of the value of the input or output or other variables.

\subsubsection{Correct by Testing}
%\subsubsubsection{Problem Formulation}

In order to solve the mapping problem through genetic algorithm (GA), we need to first translate the original mapping problem as a GA problem, find a proper solution, and then, translate back the solution to a solution for the mapping problem. In a GA problem, one should represent possible solutions in the form of a \textit{chromosome}, which basically is a sequence of so-called \textit{genes}. Figure~\ref{fig:GA_Problem_Formulation} shows how we formulate our mapping problem as a GA problem.   

\begin{figure*}[h]
  \centering
  \includegraphics[width=.9\textwidth]{figures/GA_Problem_Formulation.pdf}
  \caption{The genetic algorithm problem formulation. }
  \label{fig:GA_Problem_Formulation}
\end{figure*}

\begin{itemize}
    \item \textbf{Solution representation:} 
 In our approach, we consider the length of a chromosome as the same as the number of symbolic variables that are present in the mathematical model, and each gene represents a variable number from the list of the variables that we extracted from the description file of the FMU (figure~\ref{fig:GA_Problem_Formulation}(a)). For instance, if the third gene in the chromosome equals number 5, it means that we assign the fifth variable in the variable list to the third symbol in the symbol list. Following this solution representation, even in the simple equation demonstrated in \secref{sec:Motivation} that has 3 equations, 12 symbols, and 13 variables, the number of possible solutions to explore is $12!$, around $480M$. Please note that upon finding a solution (i.e., a mapping between symbols and variables), one has to generate the equation, create an FMU based on the equation, and simulate the created FMU by providing inputs to receive its corresponding outputs. This process is a mandatory part of the approach to make sure about the correctness of the synthesized mathematical representation. Due to the mentioned over head in this process, it would not be feasible to explore all the possible solutions to find the best one. In fact, we use the power of GA to avoid such exploration.     
    
    \item \textbf{First population generation:} We randomly generate the first population, meaning that random numbers of variables will be assigned to random symbols (figure~\ref{fig:GA_Problem_Formulation}(b)). However, we make sure that no two genes in a chromosome have the same variable number. This ensures that each symbol will be assigned a unique variable in the synthesized mathematical representation. Moreover, at the end of this process, all the symbols will have one variable assigned to them. 

    \item \textbf{Mutation operation:}
    In order to mutate a chromosome, we randomly select two genes in the chromosome, and swap their value. Figure~\ref{fig:GA_Problem_Formulation}(c) demonstrates this process. 
    
    \item \textbf{Cross-over operation:}
    Given two chromosomes, we randomly select a crossing point and concatenate the first part of first parent to the second part of second parent, and also the first part of second parent to the second part of the first parent to create new children from the given two parents (figure~\ref{fig:GA_Problem_Formulation}(d)). \ali{Any guarantee on not appearance of a variable more than once in the children?}
    
    \item \textbf{Fitness:} In our approach, the fitness is a little bit different from the common sense of the fitness score which is the higher the fitness is, the better is the solution. In our case, the fitness is basically the distance between the output of the generated model with the output of the given FMU. Therefore, the lower the fitness is, we consider the found solution as more valuable. 

\end{itemize}

Following the explained approach, we expect to achieve better solutions (i.e., solutions with lower errors) through next generations. In other words, starting from the early generations, we might find symbol-variable mappings such that the output of their created FMUs have a considerable distance to the output of the given FMU. However, moving forward, we would like to see that the found mappings result in mathematical representations with lower distance between their output and the output of the original FMU. Finally, we would like to see that at some point, the output of the synthesized mathematical representation completely matches the output of the given FMU. However, as will explain further, this is not always the case.

\subsubsection{Correct by Construction}

In order to calculate the fitness of the generated chromosomes, we use the \textit{OpenModelica} to simulate the synthesized mathematical model and compare the output of the simulation against the simulation output of the reference FMU. To be simulatable, the synthesized mathematical model must follow certain syntax and semantic rules of \textit{Modelica} language . Otherwise, OpenModelica throws an exception and does not return an output. Nevertheless, during the individual (i.e., chromosome) generation by the baseline \textit{Correct by Testing} method, we observed that a considerable number of created mathematical representations do not comply with the expected syntax or semantic of Modelica language. In particular, we identified the following issues that can result in creation of non-simulatable mathematical representations. To mitigate these challenge, we followed the \textit{Correct by Construction} paradigm in which we make sure that all the operations in our GA approach generate simulatable solutions. 

\begin{itemize}
    \item \textit{Different types of variables:} We observed that it is quite possible that the symbolic, as well as actual variables in the equations might be from different types (e.g. Real, Boolean, or String). In such cases, we can not arbitrarily assign a variable to a symbol without considering the type compliance constraint. For example, whenever we find a symbolic statements such as the following, we infer that the type of the variable that will be assigned to $Symbol_2$ should be of type \textit{Boolean}:
    $Symbol_1 = if(Symbol_2) Symbol_3 else Symbol_4;$
    \item \textit{At least one independent variable per equation:} To be able to simulate the generated mathematical model, the OpenModelica requires all the equations to at least include one independent variable. Otherwise, the equation would be considered as too trivial and the simulation will not be carried out. 
    \item \textit{Equal number of the independent variables and the equations:} Another constraint that OpenModelica puts on the generated mathematical models is that the total number of independent variables used in the model should be equal to the number of the equations. Otherwise, the equation system would be considered as under- (in case that the number of variables is less than the number of equations) or over-estimation (in case that the number of variables is greater than the number of equations) and the simulation will not happen.
    \item \textit{Usage of I/O variables in the equations:} Another important constraint is that all the input and output variables should be used in the equations.  
\end{itemize}

By generating chromosomes that syntactically and semantically comply with Modelica language, we basically changed the paradigm of our synthesizer from \textbf{\textit{Correct-by-Testing}} to \textbf{\textit{Correct-by-Construct}}. In other words, instead of synthesizing models that we are not sure about their soundness, we changed our synthesis approach such that it only generates mathematical models that are \textbf{sound}, i.e., the model is simulatable by the simulator. 
% \section{Case Study}
\label{sec:ExperimentalStudy}
In order to showcase the performance of the introduced approach, in this section, we conduct three case studies on real-world cyber-physical systems. In each case, we first provide information on the system, and then, we share the results of the conducted experiment. Figure~\ref{fig:correct_by_construct_result} summarizes the result of this study.   

\subsection{Case \#1: DC Motor}
\ali{Needs more information and characteristics of the system. }
The first case shown in figure~\ref{fig:correct_by_construct_result} is for a model with 8 equations, 18 variables, and 23 symbols. The orange curve represents the performance of the Correct-by-Testing approach, and the blue curve is shows the result of performing the mentioned changes and shifting the paradigm to Correct-by-Construct. Clearly, even in this case that all the variables and symbols are from the same type (Real), the Correct-by-Construct approach is able to find better mathematical representations by end of the run of the approach.

\subsection{Case \#2: PID}
However, by adding to the complexity of the models, the true power of the performed changes would be visible. \ali{Needs more information and characteristics of the system.}
In this case, there were 6 equations, 37 variables, and 20 symbols, and more than one type in the equations in the Case \#2 (PID model). In this case, the Correct-by-Testing approach was not able to generate even one single chromosome that can be simulated using the simulator. However, not only the Correct-by-Construct approach is able to generate such models, but also it finds solutions with relatively good scores.


\subsection{Case \#3: PID Lim}
\ali{Needs more information and characteristics of the system. }
Finally, Case \#3 shows a more complex system compared to Case \#1 and \#2. However, the Correct-by-Construct approach can generate simulatable solutions for this case while the Correct-by-Testing approach cannot. 


\begin{figure*}[ht]
  \centering
  \includegraphics[width=.8\textwidth]{correct_by_construct_result.pdf}
  \caption{The result of changing the synthesis paradigm from Correct-by-Testing to Correct-by-Construct.}
  \label{fig:correct_by_construct_result}
\end{figure*}






% Our experiments are focused on natural language understanding tasks. We recognize that adapting our SubChar tokenization to language generation tasks might require additional efforts, for example, we may want to avoid cases of predicting sub-character tokens that do not form complete characters. Also, evaluating the robustness of language generation models on real-world input noises may require additional benchmarks beyond those used in this paper. We leave such exploration as an interesting direction for future work. 

Another limitation is that our method is designed specifically for the Chinese language. While we hypothesize that our method can also bring benefits to other languages with ideographic symbols, such as Kanji in Japanese, we leave such investigation to future work. 


% \section{Related work}\label{sect:related}

\paragraph{{Recovery}} {The works most closely most closely related to ours are those based on the \emph{recovery} notion, that is, the type system of Gordon et al. \cite{GordonEtAl12} and the Pony language  \cite{ClebschEtAl15}.} Indeed, the capsule property has many variants in the literature, such as \emph{isolated} \cite{GordonEtAl12}, \emph{uniqueness} \cite{Boyland10} and \emph{external uniqueness}~\cite{ClarkeWrigstad03}, \emph{balloon} \cite{Almeida97,ServettoEtAl13a}, \emph{island} \cite{DietlEtAl07}. 
%The fact that aliasing can be controlled by using \emph{lent} (\emph{borrowed}) references is well-known~\cite{Boyland01,NadenEtAl12}.
However, before the work of Gordon et al. \cite{GordonEtAl12}, the capsule property was only ensured in simple situations, such as using a primitive deep clone operator, or composing subexpressions with the same property.

The important novelty of the type system of Gordon et al. \cite{GordonEtAl12} has been \emph{recovery}, that is, the ability to ensure properties (e.g., capsule or immutability) by keeping into account not only the expression itself but the way the surrounding context is used. {Notably,} an expression which does not use external mutable references is recognized to be a capsule. 
{In the Pony language \cite{ClebschEtAl15}  the ideas of Gordon et al. \cite{GordonEtAl12} are extended to a richer set of reference immutability permissions. In their terminology \texttt{value} is immutable, \texttt{ref} is mutable, \texttt{box} is similar to \emph{readonly} as often found in literature, different from our $\readable$ since it can be aliased. An ephemeral isolated reference \lstinline{iso^} is similar to a $\capsule$ reference in our calculus, whereas non ephemeral \texttt{iso} references offer destructive reads and are more
similar to isolated fields \cite{GordonEtAl12}. Finally, \texttt{tag} only allows object identity checks and \texttt{trn} (transition) is a subtype of \texttt{box} that can be converted to \texttt{value}, providing a way to create values without using isolated references. The last two qualifiers have no equivalent in our
work or in  \cite{GordonEtAl12}.}

Our {type system greatly enhances the recovery mechanism used in such previous work \cite{GordonEtAl12,ClebschEtAl15} by using lent references, and rules \rn{t-swap} and \rn{t-unrst}.} For instance, the examples in \refToFigure{TypingOne} and \refToFigure{TypingTwo} would be ill-typed in \cite{GordonEtAl12}. 

{A minor difference with the type systems of Gordon et al. \cite{GordonEtAl12} and Pony \cite{GordonEtAl12,ClebschEtAl15} is that we only allow fields to be $\mutable$ or $\imm$.
Allowing \emph{readonly} fields means holding a reference that is useful for observing but non making remote modifications. However, our type system supports the $\readable$ modifier rather than the \emph{readonly}, and the $\readable$ qualifier includes the $\lent$ restriction. Since something which is $\lent$ cannot be saved as part of a $\mutable$ object, $\lent$ fields are not compatible with the current design where objects are born $\mutable$. The motivation for supporting $\readable$ rather than \emph{readonly} is discussed in a specific point later.
Allowing $\capsule$ fields means that programs can store an externally unique object graph into the heap and decide later whether to unpack
 permanently or freeze the reachable objects.  This can be useful, but, as for $\readable$ versus \emph{readonly}, our opinion is that this power is hard to use for good, since
it requires destructive reads, as discussed in a specific point later. 
In most cases, the same expressive power can be achieved by having the
field as $\mutable$ and recovering the $\capsule$ property for the outer object.}

\paragraph{Capabilities}
 {In other proposals \cite{HallerOdersky10,CastegrenWrigstad16} types are compositions of one or more \emph{capabilities}. The modes of the capabilities in a type control how resources of that
type can be aliased. The compositional aspect of capabilities is an important difference
from type qualifiers, as accessing different parts of an object through different capabilities in the same type gives different properties. 
By using capabilities it is possible to obtain an expressivity which looks similar to our type system, even though with different sharing notions and syntactic constructs. For instance, the \emph{full encapsulation} notion in \cite{HallerOdersky10}\footnote{{This paper includes a very good survey of work in this area, notably explaining the difference between \emph{external uniqueness}~\cite{ClarkeWrigstad03} and \emph{full encapsulation}.}}, apart from the fact that sharing of immutable objects is not allowed, is equivalent to the guarantee of our $\capsule$ qualifier, while
our $\lent$ and their \Q|@transient| achieve similar results in different ways.}
Their model has a higher syntactic/logic overhead to explicitly  track regions.
As for all work before~\cite{GordonEtAl12}, objects need to be born \Q|@unique| and the type system 
permits to manipulate data preserving their uniqueness. With recovery~\cite{GordonEtAl12},
instead, we can forget about uniqueness, use normal code designed to work on conventional shared data, and then
recover the aliasing encapsulation property.

\paragraph{Destructive reads} Uniqueness can be enforced by destructive reads, i.e., assigning a copy of 
the unique reference to a variable an destroying the original reference, see
\cite{GordonEtAl12,Boyland10}. Traditionally, borrowing/fractional permissions~\cite{NadenEtAl12} are related to uniqueness  in the opposite way: a unique reference can be borrowed,
it is possible to track when all borrowed aliases are buried~\cite{Boyland01}, and then uniqueness can be recovered.
These techniques offers a sophisticate alternative to destructive reads. 
We also wish to avoid destructive reads. In our work, we ensure uniqueness by linearity, that is, by allowing at most
one use of a $\capsule$ reference.

In our opinion, programming with destructive reads is involved and hurts the correctness of the program, since it leads to the style of programming outlined below, where \Q@a.f@ is a unique/isolated field with destructive read.
\begin{lstlisting}
a.f=c.doStuff(a.f)//style suggested by destructive reads
\end{lstlisting}
The object referenced by \lstinline{a}{} has an \emph{unique/isolated} field \lstinline{f} containing an object \lstinline{b}.
This object \lstinline{b}{} is passed to a client \lstinline{c}{}, which can use (potentially modifying) it. A typical pattern is that the result of such computation is a reference to \lstinline{b}{}, which \lstinline{a}{} can then recover. This approach allows \emph{isolated} fields, as shown above, but has  a serious drawback:
an \emph{isolated} field can become unexpectedly not available (in the example, during execution of \lstinline{doStuff}{}), hence any object contract
involving such field can be broken.
{This can cause {subtle} bugs if \Q@a@ is in the reachable object graph of \Q@c@.}

In our approach, the  $\capsule$ qualifier cannot be applied to fields. Indeed, the ``only once'' use of capsule variables 
makes no sense on fields.
{However, we support the same level of control of the reachable object graph by passing mutable objects to clients as $\lent$, in order to control aliasing behaviour.
That is, the previous code can be rewritten} as follows:
\begin{lstlisting}
c.doStuff(a.f())//our suggested style
\end{lstlisting}
{where \Q@a.f()@ is a getter returning the field as $\lent$.
Note how, during the execution of \Q@doStuff@, \Q@a.f()@ is still available, and,} after the execution of \Q@doStuff@, the aliasing relation {for this field is the same as it was
before \Q@doStuff@ was called.}

\paragraph{Ownership} A {closely related} stream of research is that on \emph{ownership} (see an overview in~\cite{ClarkeEtAl13}) which, however, offers an {opposite} approach. In the ownership approach, it is provided a way to express and prove the ownership invariant\footnote{{Ownership invariant (owner-as-dominator):
Object $o_1$ is owned by object  $o_2$ if in the object graph $o_2$
is a dominator node for $o_1$;
that is, all paths from the roots of the graph (the stack variables)
to $o_1$ pass throw $o_2$.
Ownership invariant (owner-as-modifier):
Object $o_1$ is owned by object  $o_2$ if any field update over $o_1$
is initiated by $o_2$, that is, a call of a method of $o_2$ is present
in the stack trace.}}, which, however, is expected to be guaranteed by defensive cloning, as explained below. In our approach, instead, the capsule concept models an efficient \emph{ownership transfer}. In other words, when an object $\x$ is ``owned'' by another object $\y$, it remains always true that $\y$ can be only accessed only through $\x$, whereas the capsule notion is more dynamic: a capsule can be ``opened'', that is, assigned to a standard reference and modified, and then we can recover the original capsule guarantee. 

For example, assuming a graph with a list of nodes, and a constructor taking in input such list,
the following code establishes the ownership invariant using $\capsule$, and ensures that it cannot be violated using $\lent$.
\begin{lstlisting}
class Graph{
  private final NodeList nodes;
  private Graph(NodeList nodes){this.nodes=nodes; }

  static Graph factory(capsule NodeList nodes){
    return new Graph(nodes);
    }
  
  lent NodeList borrowNodes(mut){return nodes;}
}
\end{lstlisting}
Requiring the parameter of the \lstinline{factory}{} method to be a $\capsule$ guarantees that the list of nodes provided as argument is not referred from the external environment. 
The factory \emph{moves} an isolated portion of store as local store of the newly created object. 
Cloning, if needed, becomes responsibility of the client which provides the list of nodes to the graph. The getter tailors the exposure level of the private store. 

Without aliasing control ($\capsule$ qualifier),  in order to ensure ownership of its list of nodes, the {factory method} should clone the argument, since it comes from an external client environment.
This solution, called  \label{cloning} \emph{defensive cloning}~\cite{Bloch08}, is very popular in the Java community, but inefficient,
since it requires to duplicate the reachable object
graph of the parameter, until immutable nodes are
reached.\footnote{{In most languages, for owner-as-modifier defensive cloning is needed
only when new data is saved inside of an object, while for owner-as-dominator it is needed also when internal data are exposed.}}
Indeed, many programmers prefer to write {unsafe}
 code instead of using defensive cloning for efficiency reasons.
However, this unsafe approach is only possible when programmers have control of the client code, that is, they are not 
working in a library setting.
Indeed many important Java libraries (including the standard Java libraries) today
use defensive cloning to ensure ownership of their internal state.

As mentioned above, our approach is the opposite of the one offered by many ownership approaches, which provide a formal way to express  and prove the ownership invariant that, however, are expected to be guaranteed by defensive cloning. 
We, instead, model an efficient \emph{ownership transfer} through the capsule concept, then, 
duplication of memory, if needed, is performed on the client side\footnote{
Other work in literature supports ownership transfer, see for example~\cite{MullerRudich07, ClarkeWrigstad03}.
In literature it is however applied to uniquess/external uniqueness, thus not {the whole} reachable object graph is transfered.
}.

Moreover, depending on how we expose the owned data, we can closely model
both \emph{owners-as-dominators} (by providing no getter)
and \emph{owners-as-qualifiers} (by providing a \Q@read@ getter). In the example, the method \lstinline{borrowNodes}{} is an example of a $\lent$ getter, a third variant besides the two described on page \pageref{exposer}.  This variant is particularly useful in the case of a field which is owned, indeed, \Q@Graph@ instances can release the mutation control of their nodes without permanently {losing} the aliasing control.

In our approach all properties are deep. On the opposite side, most ownership approaches allows one to distinguish
subparts of the reachable object graph that are referred but not logically owned. This viewpoint has many advantages, for example the Rust language\footnote{\texttt{rust-lang.org}} leverages on ownership to control object deallocation without a garbage collector.
Rust employs a form of uniqueness that can be seen as a restricted ``owners-as-dominators" discipline.  
Rust lifetime parameters behave like additional ownership parameters~\cite{OstlundEtAl08}.

However, in most ownership based approaches 
it is not trivial to encode the concept of full encapsulation, while supporting (open) sub-typing and avoiding defensive cloning.
This depends on how any specific ownership approach entangles subtyping with 
 gaining extra ownership parameters
and extra references to global ownership domains.

\paragraph{Readable notion} Our $\readable$ qualifier is different from \emph{readonly} as used, e.g., in \cite{BirkaErnst04}.
 An object cannot be modified through a readable/readonly reference. However, 
$\readable$ also prevents aliasing.
As discussed in \cite{Boyland06}, readonly semantics can be easily misunderstood by
programmers. Indeed, some wrongly believe it means immutable, whereas the object denoted by a readonly reference can be modified through other references, while others do not realize that readonly data can still be saved in fields, and thus used as a secondary window to observe the change in the object state.
Our proposal addresses both problems, since we explicitly support the $\imm$ qualifier, hence it is more difficult for programmers to confuse the two concepts, and our $\readable$ (readonly + lent) data  cannot be saved in client's fields.

Javari~\cite{TschantzErnst05} also supports the \emph{readonly} type qualifier, and makes a huge design effort to support \emph{assignable} and \emph{mutable} fields, to have fine-grained readonly constraints.  The need of such flexibility is motivated by performance reasons.
In our design philosophy, we do not offer any way of breaking the invariants enforced by the type system. Since our invariants are very strong, we expect compilers to be able to perform optimization, thus recovering most of the efficiency lost to properly use immutable and readable objects.



% \section{Conclusion}
We have presented a neural performance rendering system to generate high-quality geometry and photo-realistic textures of human-object interaction activities in novel views using sparse RGB cameras only. 
%
Our layer-wise scene decoupling strategy enables explicit disentanglement of human and object for robust reconstruction and photo-realistic rendering under challenging occlusion caused by interactions. 
%
Specifically, the proposed implicit human-object capture scheme with occlusion-aware human implicit regression and human-aware object tracking enables consistent 4D human-object dynamic geometry reconstruction.
%
Additionally, our layer-wise human-object rendering scheme encodes the occlusion information and human motion priors to provide high-resolution and photo-realistic texture results of interaction activities in the novel views.
%
Extensive experimental results demonstrate the effectiveness of our approach for compelling performance capture and rendering in various challenging scenarios with human-object interactions under the sparse setting.
%
We believe that it is a critical step for dynamic reconstruction under human-object interactions and neural human performance analysis, with many potential applications in VR/AR, entertainment,  human behavior analysis and immersive telepresence.





\begin{figure*}[h]
  \centering
  \includegraphics[width=.97\textwidth]{figures/CONSTRUCT_4.pdf}
    \vspace{-6pt}

  \caption{An overview of the \textsc{CONSTRUCT} approach.}
  \label{fig:CONSTRUCT}
          \vspace{-6pt}
\end{figure*}

\section{Background and the Problem}
\label{sec:Introduction}
Cyber-physical systems (CPS) consist of different components, including physical as well as computational parts. A key component of such systems is the embedded software that orchestrates interactions between different parts, performs needed processing, and implements the controlling algorithms prescribed by the system designer.
%
CPS designers use different tools, including Simulink~\cite{documentationsimulation} and Modelica~\cite{fritzson2006openmodelica} to design the system and perform corresponding simulations. They then export the designed model as a standardized portable container called a  \textit{Functional Mockup Unit (FMU)}~\cite{blochwitz2011functional} which is transportable between different tools. An FMU contains binary files (e.g., compiled C files) that implement the functionality of the CPS (e.g., the controller part), alongside some text-based description files which provide information about the inputs, outputs, and other variables used in the binary files. 
%
Since a CPS can be used in mission- and safety-critical domains such as military, medical, transportation, and agriculture \cite{serpanos2018cyber}, it is crucial for subject matter experts (SME) to gain a clear and accurate understanding of the system's behavior to better assess the safety of the system before deployment.
%and prevent any damaging threats in the run-time (e.g., the system is malicious or weaponized). 
%
However, not only the SMEs are often unfamiliar with low-level programming languages, in forensic applications, only the system's binary is available \cite{ming2016straighttaint}. Therefore, it is necessary to derive an interpretable model from the binaries which accurately represents the functionality of the underlying code~\cite{shbita2022automated}. 
%For instance, this functionality could be a widely used proportional–integral–derivative (PID) controller~\citeme{} that is a control loop mechanism leveraging feedback.     
%
There have been approaches developed by researchers to create such representations of the code to assist SMEs, however, these techniques heavily rely on intelligent experts (e.g., professional programmers) to manually review either the source code or the decompiled binaries and create an initial abstraction of the software~\citeme{}. This is a non-trivial, time-consuming, and error-prone task. 
%In addition, the controller part of a cyber-physical system in the FMU is mostly in form of a binary file. Hence, intelligent experts need to de-compile the binaries to be able to manually review the code. However, through the de-compilation process, the generated source code will be obfuscated, meaning that variable names, method names, and even in some cases class names are changed to meaningless and strange names which makes it dramatically hard to follow and understand the code.

To address these challenges, we introduce a novel approach for automatically synthesizing a high-level mathematical representation of controller software in a CPS. This representation enables SMEs to easily identify possible drifts from the expected behavior of CPS and avoid damaging threats (e.g., malicious or weaponized software). The synthesized mathematical representation follows the syntax and semantic of Modelica language~\cite{fritzson1998modelica}, which is a multi-domain modeling language and is easily interpretable by subject matter experts (SME). 
\vspace{-15pt}
%
\section{Approach}
\label{sec:Construct}
Figure~\ref{fig:CONSTRUCT} shows an overview of our approach, called \textbf{Code-based MOdel SyNthesiS PlaTform foR re-ConstrUcting Control AlgoriThms} \textsc{\textbf{(CONSTRUCT)}}, which automatically constructs mathematical representations of controller parts of CPS. \textsc{CONSTRUCT} takes in an FMU file which contains binaries and some descriptions about the CPS. 
The main idea is to create the \textbf{structure} of the mathematical representation by translating the symbolic ASTs of decompiled C binary codes to symbolic ASTs in Modelica language. Then, leverage an evolutionary search technique to find correct actual variable names for each symbolic names and add the \textbf{semantic} to the created structure. In the following, we provide more details on each step.

    \squared{1} \textbf{Decompile Binaries:} From the description files inside the FMU, \textsc{CONSTRUCT} identifies the variable names and their attributes (e.g., type, initial value, etc.) that should be used in the final mathematical model. It also decompiles the binary files inside the FMU using Ghidra~\cite{Ghidra} as an off-the-shelf popular decompiler. The decompiled code is contains symbolic variable names. 
    
    \squared{2} \textbf{Isolate Mathematical Primitives:} Next, based on its rule-based engine, CONSTRUCT localizes the mathematical primitives that are commonly used in controller parts of CPS. Some mathematical primitives might differ in the decompiled code comparing to the original source code. CONSTRUCT is able to identify these primitives as well. 
    
    \squared{3} \textbf{Code-level AST to Model-level AST:} In this step, CONSTRUCT creates ASTs of the decompiled binaries (C files) and translates the sub-ASTs that correspond to localized mathematical primitives (previous step) to mathematical-formed ASTs. The mathematical-formed ASTs follow syntax of Modelica language. Note that due to the decompilation, the created mathematical-formed ASTs contain non-real symbolic names. %Therefore, we need to find the correct mapping between symbolic names and the actual variable names collected through the first step.
    
    \squared{4} \textbf{Modelica Model Synthesis} We use Genetic Algorithm (GA) as an evolutionary search-based approach to find a correct mapping between the generated symbolic variable names in the created mathematical representation and the original variable names retrieved from the description file inside the FMU. We follow the \textbf{correct-by-construction} paradigm throughout this process meaning that we carefully design GA operators (i.e., first population generation, mutation, and cross-over) such that every generated chromosome (a solution candidate) by the GA approach complies with syntax and semantic of Modelica language. This significantly prunes the search space and reduces try and errors. For instance, if the type of variable $v_1$ is \textit{Real} but the type of symbol $s_1$ is \textit{Boolean}, then, our correct-by-construction approach does not consider mapping $v_1$ to $s_1$. In contrast, the baseline GA approach (\textit{correct-by-testing}) does not take these constraints into account and might generate faulty solutions that result in crashes while simulating the synthesized model on simulators. %The case study result (section~\ref{sec:PreliminaryResult}) shows how our correct-by-construction approach dramatically improves the quality of the solutions generated by GA in comparison with the base-line \textit{correct-by-testing} approach.
 
The output of \textsc{CONSTRUCT} is a mathematical representation of the controller part of CPS in form of  Modelica language. 

\section{Case Study}
\label{sec:PreliminaryResult}
In order to showcase the performance of the introduced approach, we present case studies conducted on three different controllers used in two real-world CPS, a \textit{Turtlebot Waffle Pi}~\citeme{} which is an inherently stable wheeled robot, and a \textit{PX4 Quadcopter}~\citeme{} which is statically unstable robot. The complexity of controllers increases by the number of cases i.e., the synthesized mathematical representation for Case 1 consists of 8 equations, 18 variables, and 23 symbols, and the mathematical representations for Case 2 and Case 3 contain 6 equations, 37 variables, and 20 symbols and 13 equations, 80 variables, and 39 symbols respectively. Also, the type of all variables and symbols in Case 1 are of type Real, whilst the types of variables and symbols in Case 2 and Case 3 are of different types, making the correct mapping finding problem much harder in the later two cases. In our experiments, the population size for the GA was considered as 400 and the maximum number of generations was 10.
%
\begin{figure}[ht]
  \centering
  \includegraphics[width=.47\textwidth]{Case_Study_4.pdf}
  \caption{The result of synthesizing mathematical models based on the Correct-by-Testing (base-line) and our Correct-by-Construction approach.}
  \label{fig:correct_by_construct_result}
\end{figure}
%
 Figure~\ref{fig:correct_by_construct_result} summarizes the result of this study. Even in the simplest Case (Case 1) that all the variables and symbols are from the same type (i.e., Real), the correct-by-construction approach is able to find better mathematical representations compared to the base-line (correct-by-testing) approach. Interestingly, while in complex cases (cases 2 and 3) the correct-by-testing approach is not able to generate even one single representation runnable by the simulator, the correct-by-construction approach can generate simulatable models that comply with Modelica syntax and semantic and also have relatively good fitness scores.


\section{Conclusion and Future Work}
\label{sec:Conclusion}
In this paper we introduced CONSTRUCT, a novel program synthesis approach that automatically creates mathematical representations of control algorithms in cyber-physical systems. This approach leverages a genetic algorithm for injecting the semantic into the created representation. In future work, we are interested in evaluating the utility of constraint-based solvers, such as SMT solvers, for making the model synthesis more efficient.

%%
%% The next two lines define the bibliography style to be used, and
%% the bibliography file.
\bibliographystyle{ACM-Reference-Format}
\bibliography{bibliography}

\end{document}
\endinput
%%
%% End of file `sample-sigconf.tex'.
