\section{Case Study}
\label{sec:ExperimentalStudy}
In order to showcase the performance of the introduced approach, in this section, we conduct three case studies on real-world cyber-physical systems. In each case, we first provide information on the system, and then, we share the results of the conducted experiment. Figure~\ref{fig:correct_by_construct_result} summarizes the result of this study.   

\subsection{Case \#1: DC Motor}
\ali{Needs more information and characteristics of the system. }
The first case shown in figure~\ref{fig:correct_by_construct_result} is for a model with 8 equations, 18 variables, and 23 symbols. The orange curve represents the performance of the Correct-by-Testing approach, and the blue curve is shows the result of performing the mentioned changes and shifting the paradigm to Correct-by-Construct. Clearly, even in this case that all the variables and symbols are from the same type (Real), the Correct-by-Construct approach is able to find better mathematical representations by end of the run of the approach.

\subsection{Case \#2: PID}
However, by adding to the complexity of the models, the true power of the performed changes would be visible. \ali{Needs more information and characteristics of the system.}
In this case, there were 6 equations, 37 variables, and 20 symbols, and more than one type in the equations in the Case \#2 (PID model). In this case, the Correct-by-Testing approach was not able to generate even one single chromosome that can be simulated using the simulator. However, not only the Correct-by-Construct approach is able to generate such models, but also it finds solutions with relatively good scores.


\subsection{Case \#3: PID Lim}
\ali{Needs more information and characteristics of the system. }
Finally, Case \#3 shows a more complex system compared to Case \#1 and \#2. However, the Correct-by-Construct approach can generate simulatable solutions for this case while the Correct-by-Testing approach cannot. 


\begin{figure*}[ht]
  \centering
  \includegraphics[width=.8\textwidth]{correct_by_construct_result.pdf}
  \caption{The result of changing the synthesis paradigm from Correct-by-Testing to Correct-by-Construct.}
  \label{fig:correct_by_construct_result}
\end{figure*}





