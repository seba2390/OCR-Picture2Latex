\section{Background \& Problem Definition}
\label{sec:Background}



Cyber-physical system designers use tools such as Simulink~\citeme{}, Matlab~\citeme{}, and OpenModelica~\citeme{} to design models and simulate their behavior in various situations. Then, the designed model and its corresponding simulation can be exported by these tools as a standardized portable package called \textit{Functional Mockup Unit (FMU)} that is transportable from one simulator to the other. An FMU contains binary files (e.g., compiled C files) that implement the functionality of the cyber-physical system, alongside some text-based description files which provide information about the modeled cyber-physical system (e.g., information on inputs, outputs, and other internal variables used in the system). In this paper, we consider the FMU of a cyber-physical system as the input to our approach. There are three main reasons behind this decision listed below:
\begin{itemize}
    \item FMU is a well-respected and standardized container for exchanging and co-simulating dynamical models.
    \item There are a vast number of modeling tools that support FMU exportation including Matlab, Simulink, Dymola, OpenModelica, SimulationX, and more.
    \item An FMU container can include both source code as well as binaries targeting different platforms.
\end{itemize}

\begin{figure}[ht]
  \centering
  \includegraphics[width=.49\textwidth]{figures/Problem_Definition.pdf}
  \caption{Synthesizing mathematical representations of a given cynber-physical system in form of an FMU.}
  \label{fig:Problem_Definition}
\end{figure}

Figure~\ref{fig:Problem_Definition} shows the expected input and output of the proposed approach in this paper. We define the problem of this paper as: \textbf{\textit{Given an FMU of a cyber-physical system, we aim to develop an approach that generates mathematical representations of the given system. The synthesized mathematical representation shall follow the syntax and semantic of Modelica language~\citeme{}, which is a multi-domain modeling language for component-oriented modeling of complex systems and is easily interpretable by subject matter experts (SME) that are not necessarily familiar with programming languages, but are experts in understanding the behavior of cyber-physical systems.}} The synthesized mathematical representations shall be intuitive such that even a person with a non-advanced mathematical knowledge can understand the behavior of the cyber-physical system in generating the output(s) from the given input(s).



\subsection{Motivating Example}
\label{sec:Motivation}
\ali{To be added.}