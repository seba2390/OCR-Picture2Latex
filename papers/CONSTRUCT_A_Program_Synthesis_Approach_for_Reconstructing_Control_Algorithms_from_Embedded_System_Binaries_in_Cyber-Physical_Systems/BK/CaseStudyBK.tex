\section{Case Study}
\label{sec:PreliminaryResult}
In order to showcase the performance of the introduced approach, in this section, we present case studies conducted on three different controllers used in two real-world cyber-physical systems, a \textit{Turtlebot Waffle Pi}~\citeme{} which is a inherently stable wheeled robot, and a \textit{PX4 Quadcopter}~\citeme{} which is statically unstable robot. In each case, we first provide information on the controller, and then, we share the results of the conducted experiment. Figure~\ref{fig:correct_by_construct_result} summarizes the result of this study. In our experiments, the population size was considered as 400 and the maximum number of generations was 10. 
%
\begin{figure}[ht]
%   \vspace{-2pt}
  \centering
  \includegraphics[width=.49\textwidth]{Case_Study.pdf}
  \caption{The result of synthesizing mathematical models based on the Correct-by-Testing (base-line) and our Correct-by-Construct approach.}
  \label{fig:correct_by_construct_result}
%   \vspace{-10pt}
\end{figure}
%
\subsection{Case \#1: PI Controller}
\ali{Needs more information and characteristics of the system. }
The first case shown in figure~\ref{fig:correct_by_construct_result} corresponds to a model with 8 equations, 18 variables, and 23 symbols. The orange curve represents the performance of the correct-by-testing approach, and the blue curve  shows the result of correct-by-construction approach. Clearly, even in this case that all the variables and symbols are from the same type (i.e., Real), the correct-by-construct approach is able to find better mathematical representations.

\subsection{Case \#2: PID Controller}
However, by adding to the complexity of the models, the true power of correct-by-construction would be visible. \ali{Needs more information and characteristics of the system.}
In this case, there were 6 equations, 37 variables, and 20 symbols, and more than one type in the equations. As shown in figure~\ref{fig:correct_by_construct_result}.b, the correct-by-testing approach was not able to generate even one single chromosome that can be simulated using the simulator. However, not only the correct-by-construct approach can generate such models, it also finds solutions with relatively good scores.

\subsection{Case \#3: LimPID Controller}
\ali{Needs more information and characteristics of the system. }
Finally, Case \#3 shows a more complex system compared to Cases \#1 and \#2. Even in this case, the correct-by-construct approach can generate simulatable solutions while the correct-by-testing approach cannot. 