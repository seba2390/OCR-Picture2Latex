\section{Numerical results}\label{sec:numerics}
In this section we show numerical results for simulated fMRI data and real DCE-MRI data. 

\subsection{Methods}
In order to evaluate the potential of the proposed method and to illustrate how each of the terms of our model contributes to the result, we consider the following reconstruction approaches:
\begin{itemize}
\item Least squares (LS) reconstruction of the dynamic sequence: 
\begin{alignat}{4} \label{ls}
\min_{u_t} ~  \| \Kcal_t u_t - f_t \|_{\C^{M_t}}^2   
,\quad t= 1,\ldots,T.
\end{alignat}
The solution of (\ref{ls}) corresponds to a conventional, non-regularized reconstruction.

\item Reconstruction of the dynamic sequence using spatial TV regularization (TV):
\begin{alignat}{4} \label{tv}
\min_{u_t} ~ \frac{\alpha_t}{2} \| \Kcal_t u_t - f_t \|_{\C^{M_t}}^2 + \TV(u_t)  
,\quad t= 1,\ldots,T
\end{alignat}
The solution of (\ref{tv}) gives a gradient sparsity promoting reconstruction without time correlation between the frames $u_t$ in the image sequence.

\item Reconstruction of the dynamic sequence using temporal (smoothness) regularization, but no spatial regularization (temp):
\begin{alignat}{4} \label{l2t}
\min_{\ubold} ~ \sum_{t=1}^T \frac{\alpha_t}{2} \| \Kcal_t u_t - f_t \|_{\C^{M_t}}^2  + \sum_{t=1}^{T-1} \frac{\gamma_t}{2} \|u_{t+1} - u_t \|_{\C_N}^2 
\end{alignat}

\item Reconstruction of the dynamic sequence using spatial TV regularization and temporal regularization (temp + TV):
\begin{alignat}{4} \label{tvt}
\min_{\ubold} ~ \sum_{t=1}^T \frac{\alpha_t}{2} \| \Kcal_t u_t - f_t \|_{\C^{M_t}}^2 + \sum_{t=1}^T \TV(u_t)  + \sum_{t=1}^{T-1} \frac{\gamma_t}{2} \|u_{t+1} - u_t \|_{\C_N}^2 
\end{alignat}
The solution of (\ref{tvt}) is similar to the spatio-temporal regularization techniques in \cite{adluru2009,wundrak2016} and provides a reference
of spatio-temporal regularization without the structural prior.

\item The proposed approach using temporal regularization and $\ICBTV$ regularization for the utilization of the structural prior information:
\begin{alignat}{4} \label{icbtv}
\min_{\ubold} ~  &\sum_{t=1}^T \frac{\alpha_t}{2} \| \Kcal_t u_t - f_t \|_{\C^{M_t}}^2 \notag\\
+ &\sum_{t=1}^T \left(w_t \TV(u_t) + (1-w_t) \ICBTV^{p_0}(u_t,u_0) \right)  \notag\\
+ &\sum_{t=1}^{T-1} \frac{\gamma_t}{2} \|u_{t+1} - u_t \|_{\C_N}^2 
\end{alignat}
\end{itemize}
In the following we show results for two test cases on simulated and experimental dynamic MRI data, respectively.

\subsection{Simulated data}
\label{subsec:artifical data}
In this paragraph, we first describe the generation of a synthetic test data set from a human brain phantom in the style of fMRI measurements. Afterwards we give various illustrations of the corresponding results obtained with the aforementioned reconstruction methods.  

\subsubsection{Generation of a simulated data set}
As described before, in fMRI data is typically acquired in two stages: after a high-resolution scan that serves as anatomical reference the actual dynamic sequence is sampled. 
While conventionally the former is measured in T$1$ contrast, the changes in the cerebral blod flow are detectable in a T$2^*$ weighted MR signal such that the dynamic sequence needs to be acquired with this contrast. 
In particular, one has to face different image contrasts in the anatomical prior and the dynamic sequence.

To mimic such fMRI measurements, we used a slice of a realistically simulated MR phantom from the BrainWeb database \cite{cocosco:brainweb}, consisting of T$1$ and corresponding T$2$ axial MR images of the human brain of a size of $109 \times 91$ pixels each, with pixel intensities in $[0,1]$. 
To create a dynamic ground truth sequence, we duplicated the T$2$ image to obtain a sequence of 60 time steps $t$. 
We then chose a small area of gray matter (see the red box in Figure \ref{fig:results_synth}) and changed the intensity in this area according to a canonical hemodynamic response function (HRF) from the SPM software package\footnote{\texttt{http://www.fil.ion.ucl.ac.uk/spm/}}.
The HRF (see ``ground truth" in Figure \ref{fig:timeplots_synth} and \ref{fig:timeplots_single}) consists of a linear combination of two Gamma functions, where we used the default parameters of the package.
The amplitude of the HRS function has been scaled to approximately $0.1$, which corresponds to $10$ per cent of the overall image scale. 
It is worth noticing that this change is very small in comparison to some of the image intensities. 
This can e.g. be observed in the last line of Figure \ref{fig:results_zoom}, which shows a zoom into the red box in Figure \ref{fig:results_synth}.

To obtain a data set, we applied the forward operator \eqref{eq:forward_op_art} to this synthetic ground truth.
For the anatomical T$1$ prior, we set $\Pcal_0$ to be the identity leading to a full Cartesian sampling of $109 \times 91$ Fourier coefficients in $k$-space. 
For the dynamic T$2$ sequence, we chose $\Pcal_t$ such that at each time step frequencies located on 5 discrete spokes through the center of the $k$-space are measured (see first line in Figure \ref{fig:results_synth}). 
The spokes angles are chosen according to golden angle sampling (cf. e.g. \cite{winkelmann2007}).
This sampling scheme amounts to measuring less than $5 \%$ of the $109 \times 91$ Fourier coefficients of the full Cartesian sampling at each time step.
We finally added noise of approximately $5$ per cent of the energy of the whole signal to the sampled Fourier frequencies.
 
\subsubsection{Reconstructions}
\begin{figure}[ht!]
	\includegraphics[width=0.95\textwidth]{result_synth}
    \caption{Reconstructions using the different methods (LS), (temp), (temp + TV) and the proposed method at five different time points.
    The first line shows the respective $k$-space samplings and the anatomical prior used for the proposed method.
    The red box marks the activated region of interest, where the signal changes are observable.
    Figure \ref{fig:results_zoom} provides a zoom into this region.}
    \label{fig:results_synth}
\end{figure}
%
Figure \ref{fig:results_synth} shows results for the different reconstruction methods for five different times $t= 4,8,13,21,35$, together with the $k$-space sampling and the anatomical prior. 
The prior has been reconstructed using \eqref{tv} and has then been used to create an artificial subgradient $q_0^\eta$ as in \eqref{eq:subgrad_eta} with threshold $\eta = 0.05$.
Note that we did not include single-frame TV reconstructions \eqref{tv}, since the image quality did not increase significantly over the quality of a least squares reconstruction \eqref{ls}.
The presented times $t$ have been chosen to show different states of activation of the region of interest (ROI), which is marked by a red box in the bottom line of Figure \ref{fig:results_synth}. 
Figure \ref{fig:results_zoom} provides a close-up into the ROI.
For $t=4$ there is no activation, $t=8$ shows half of the increase of the signal, $t=13$ shows maximum activation, $t=21$ shows half of the decrease of the signal and $t=35$ a slight undershoot after the signal has decayed.
These times are also marked in Figures \ref{fig:timeplots_synth} and \ref{fig:timeplots_single}. 
The former shows the average signal over time over the activated region in the ROI in comparison to a non-activated control region for different choices of regularization parameters $\alpha$ and $\gamma$.
The latter figure shows the average signal over time over the activated region in the ROI for the different methods \eqref{ls}, \eqref{l2t}, \eqref{tvt} and \eqref{icbtv}, together with the signal from four individual adjacent pixels from the activated region.   
The parameters used to obtain the results are $\gamma_t = \alpha_t = 1$ for (temp) , $\alpha_t = \gamma_t = 500$ for (temp + TV) and $\alpha_t = 50$, $\gamma_t = 25$ for the proposed method. 
Note that we keep the parameters constant for all $t = 1, \dots, T$.

In Figure \ref{fig:results_synth} we observe that while a simple single-frame least squares reconstruction \eqref{ls} does not yield satisfactory results, a temporal regularization \eqref{l2t}, resulting in a pixelwise regularity over time, can improve image quality by far. 
However, while the structures of the brain are clearly visible and also the activation in the ROI, the quality of the reconstructions is degraded by heavy noise. 
This is also confirmed by the plots in Figure \ref{fig:timeplots_single}(a) and \ref{fig:timeplots_single}(d). 
While the reconstruction of the average signal over the activated region is decent (\ref{fig:timeplots_single}(d)), single pixels from the activated region show a lot of variance (\ref{fig:timeplots_single}(d)). 
The main reason for this is the noise in the data and the lack of any spatial regularity.

Adding a spatial TV regularization to the temporal regularization, we obtain method \eqref{tvt}, which leads to an overall reduction of noise and more spatial regularity in the reconstructed time frames, but also blurs out some of the structures.  
The activated region is as well visible, and the averaged signal over the ROI gives a similar result than before, which can be observed in Figure \ref{fig:timeplots_single}(e). 
However, we as well observe a greater coherence among the adjacent single pixels from the ROI depicted in Figure \ref{fig:timeplots_single}(b). 

The reconstructions using the proposed method including the anatomical prior show the most convincing results. 
In particular, the use of the anatomical prior leads to a very high spatial regularity with distinct edges such that most of the features of the brain are clearly visible. 
Furthermore, both the average over the activated region {\it and} the single pixels now show almost the same signal, which comes closest to our simulated ground truth (see Figure \ref{fig:timeplots_single}(c) and (f)). 
We like to point out that the reconstruction also shows the slight undershoot after the decay of the signal at $t=35$, which can be seen in Figure \ref{fig:timeplots_synth}.

We also comment on a few imperfections. 
All of the reconstructed signals show a slight bias in the sense that the amplitude of the signal has been slightly decreased. 
This is due to the (necessary) temporal regularization. 
The problem is that the amplitude of the signal varies over time, which would actually require a lower time regularity where the change between consecutive time frames is high, and a higher regularity where the difference is small. 
This can be observed in Figure \ref{fig:timeplots_synth}, where a higher $\gamma$ decreases the amplitude of the signal while enforcing less oscillations of the signal in the flat region after $t=35$.
Hence the choice of $\gamma$ always results in a trade-off between temporal smoothness and amplitude of the signal. 
If one has a rough estimate on when the signal develops, the proposed method offers a remedy by choosing a smaller $\gamma_t$ while it develops, and a larger $\gamma_t$ in the flat region at the end instead of a global $\gamma$.
This could further increase the quality of the results. 

\begin{figure}[ht!]
	\includegraphics[width=\textwidth]{result_zoom}
    \caption{Zoom into the region of interest marked with the red box in Figure \ref{fig:results_synth} for the different methods (LS), (temp), (temp + TV) and the proposed method.
    The ground truth on the bottom line shows the true activation of the gray matter.}
    \label{fig:results_zoom}
\end{figure}

\begin{figure}[ht!]
	\includegraphics[width=\textwidth]{timeplots_single}
    \caption{Ground truth and reconstructions of the (simulated) hemodynamic response over 60 time frames for the different methods (temp), (temp + TV) and the proposed method: (a)-(c) signal from four adjacent single pixels from the region of interest over time, (d)-(f) average over the entire activated region over time. 
    }
    \label{fig:timeplots_single}
\end{figure}

\begin{figure}[t]
	\includegraphics[width=\textwidth]{timeplots_synth}
    \caption{Ground truth and reconstructions with the proposed method of the (simulated) hemodynamic response over 60 time frames for different amounts of temporal regularization.
    (a) Average of the reconstructed hemodynamic response over the whole activated region.
    (b) Average over an inactive region.}
    \label{fig:timeplots_synth}
\end{figure}



\subsection{Experimental data from small animal imaging}
\label{experes}
Subsequently, we first describe the preparation and execution of an experiment to generate real small animal DCE-MRI data and give details on the acquisition protocol.
Afterwards, we show various visualizations of the respective results obtained with the aforementioned reconstruction methods.

\subsubsection{Animal preparation}
All animal experiments were approved by the Animal Health Welfare and Ethics Committee of University of Eastern Finland. 
1x106 C6 (ECACC 92090409) rat glioma cells (Sigma) were implanted into the brain of a 200g female Wistar rat under ketamin/medetomidine hydrochloride anesthesia. 
Tumor imaging was performed 10 days post-implantation. 
During the experiments, the animal was anesthetized with isoflurane (5\% induction, 1-2\% upkeep) and kept in fixed position in a holder which is inserted into the magnet. 
A needle was placed into the tail vein of the animal for the injection of the contrast agent.

\subsubsection{Acquisition of the data}
All MR data were collected using a 9.4 T horizontal magnet interfaced to Agilent imaging console and a volume coil transmit/quadrature surface coil receive pair (Rapid Biomed, Rimpar, Germany). 
For the dynamical data a gradient-echo based radial pulse sequence 
with repetition time 38.5 ms, echo time 9 ms, flip angle 30 degrees, field-of view 32x32 mm, slice thickness 1.5 mm, number of points in each spoke 128, 610 spokes collected in sequential order with an golden angle interval of 111.246 degrees before repeating the same cycle of spokes 
for 25 times, leading to overall measurement sequence of 15250 spokes of data. 
Measurement time for a full cycle of 610 spokes was 
$610 \cdot 38.5 {\rm ms} = 23.46 {\rm s}$. 
Gadovist (1mmol/kg) was injected i.v. after one minute from the beginning of the dynamic scan over a period of 3s. 

Anatomical reference images were acquired from the same slice before and after the dynamical experiment using a gradient-echo pulse sequence with repetition time of 1s and echo time of 2.8ms
with Cartesian sampling of 128x128 points of k-space data.

\subsubsection{Reconstructions}

\begin{figure}[ht!]
\centerline{\includegraphics[width=0.8\textwidth]{expefig5.pdf}}
\caption{DCE-MRI using radial golden angle data from a glioma model rat specimen. Top half: Anatomical reference image before the dynamical experiment and reconstructed frame
from the initial base line measurement at $t=200$. 
Top row shows the reference, the LS estimate and the estimate with the TV regularization. The second row shows the reconstructions using temporal regularization, TV combined with temporal regularization and the $\ICBTV$ combined with temporal regularization.
The bottom half of image shows in respective order the anatomical reference after the experiment and the last frame of the reconstructed time series.
The number of radial spokes for each time frame was three.}
\label{efig1}
\end{figure}

\begin{figure}[ht!]
 \centerline{\includegraphics[width=0.85\textwidth]{expefig6.pdf}}
\caption{DCE-MRI using radial golden angle data from a glioma model rat specimen. Top: Anatomical reference image before the dynamical experiment. The squares mark the
location of the pixel of interest for each column. 
Rows two to six show time series of the image magnitude in the selected pixels using the LS, TV regularization, temporal  regularization, TV regularization combined with temporal  regularization and $\ICBTV$ combined with temporal regularization, respectively. 
The dashed vertical lines denote the time points of the slices shown in Figure \ref{efig1}.}
\label{efig2}
\end{figure}

\begin{figure}[ht!]
\centerline{\includegraphics[width=0.82\textwidth]{expefig7.pdf}}
\caption{DCE-MRI using radial golden angle data from a glioma model rat specimen. The time series of pixel values in a vertical ROI line through the brain and the glioma. The 
location of the ROI line is indicated with a yellow line in the reconstructions of the last frame on the left. The time series of the pixel values are shown in the right column such that horizontal axis corresponds to time and vertical to location along the ROI line. Rows from top to bottom: 
LS, TV regularization, temporal regularization, TV combined with temporal regularization and $\ICBTV$ combined with  temporal regularization, respectively.}
\label{efig3}
\end{figure}

The prior image $u_0 \in \C^{65536}$ for the structural regularization was 
obtained as solution of (\ref{tvu0}) using the Cartesian reference data measured before the dynamical experiment. 
The value of the weighting parameter in (\ref{tvu0}) was $\alpha_0 = 2$. 
Note that due to the Cartesian sampling in $k$-space, the prior could also be obtained using an inverse Fourier transform, however yielding a $128 \times 128$ resolution in this case. 
Using method \eqref{tvu0} and a finer grid of $256 \times 256$ enables us to ``lift'' the resolution of the prior, which can then also be transferred to the dynamic sequence.
The magnitude of the prior image $u_0$ is shown in the first image of the top row in Figure \ref{efig1}. 

For the dynamical reconstructions, the dynamical data was divided into 
$T=5083$ time steps, 
each set consisting of three consecutive spokes, leading to measurement data $f_t \in \C^{384}$ for each frame in the time series.  
As for the prior, the unknown images were 
represented using a $256 \times 256$ pixel grid for the reconstructions, leading to unknown $u_t \in \C^{65536}$ for each time instant $t$. 
The parameters for the reconstructions were selected manually by pilot runs. The value  
$\alpha_t = 5$ was used in all the regularized solutions (\ref{tv}-\ref{icbtv}) and the value of the temporal regularization parameter was $\gamma_t = 20$ in the models (\ref{l2t}-\ref{icbtv}) containing temporal regularization. The value of the weighting parameter for the structural regularization in (\ref{icbtv}) was selected as $w_t = 0.1$. 

The results using the experimental data are shown in Figures \ref{efig1}-\ref{efig3}. The top half of Figure \ref{efig1} shows the anatomical reference before the experiment and magnitude of reconstructed frames $u_t$ at $t=200$ from the initial baseline measurement before the injection of the gadolinium bolus. 
The first row shows the anatomical reference, the LS reconstruction (\ref{ls}) and the reconstruction with the TV regularization (\ref{tv}). 
The second row shows the reconstruction (\ref{l2t}) using temporal regularization, (\ref{tvt}) with combination of temporal regularization and TV, and the proposed approach (\ref{icbtv}). 
The bottom half of Figure \ref{efig1} shows in the same order the anatomical reference after the experiment and the last frames of the reconstructed time series.    
Notice that the muscle tissues are present more clearly in the reference images than in the dynamical images due to the different echo and repetition times in the dynamical and reference measurements. 

Figure \ref{efig2} shows the time-series of the magnitude of reconstructed pixel value at three selected pixels. Each column shows the data for a single pixel, which is indicated by a red dot mark in the anatomical reference shown in the top row. 
The pixel time series in rows two to five are LS (\ref{ls}), TV regularization (\ref{tv}), temporal regularization (\ref{l2t}), temporal regularization and TV (\ref{tvt}), and the proposed method (\ref{icbtv}).  
The dashed vertical lines in the time series plots mark the time locations of the reconstructed frames in Figure \ref{efig1}.
Figure \ref{efig3} shows time series of reconstructed pixel values in a vertical ROI 
line through the brain and the glioma.
The estimates are from top to bottom in the same order as in Figure \ref{efig2}.
The vertical ROI line is indicated in the reconstructions
of the last time frame on the left.
The time series of the reconstructed ROI values are shown as intensity images on the right such that the horizontal axis corresponds to frame number and vertical axis to pixel location along the vertical ROI line.

The findings from the results with the experimental DCE data are very similar to those with the simulated fMRI data. First, we observe that the single frame reconstructions, namely the LS and TV, do not yield satisfactory results; the images are spatially of poor quality and the recovered time series are very noisy, rendering reliable visual or quantitative analysis of DCE parameters infeasible.
Using the temporal regularization (\ref{l2t}), improves the results clearly over the uncorrelated reconstructions. 
The images show well the structure of the brain and also the dynamics caused by the gadolinium contrast agent are clearly visually detectable, but the results are still contaminated with a high level of noise. 

Using the combination of temporal and TV regularization (\ref{tvt}), improves the results significantly, leading to a clear reduction of noise in the reconstructed images and time series. 
However, similarly as in the simulated test case, this comes with the cost of some blurring of the images, especially on the tissue interfaces.
The proposed method (\ref{icbtv}), which combines the time regularization and structural prior, produces again the most convincing results. 
While the reconstructed pixel time series are visually not much different from the results with (\ref{tvt}), the spatial features of the reconstructions are sharper.



