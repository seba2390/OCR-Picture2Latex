\documentclass[12pt,a4paper]{article}

%% Language and font encodings
\usepackage[english]{babel}
\usepackage[utf8x]{inputenc}
\usepackage[T1]{fontenc}

\usepackage{geometry}
\geometry{
  left=2.5cm,
  right=2.5cm,
  top=2cm,
  bottom=4cm,
  bindingoffset=5mm
}

%% Useful packages
\usepackage{amsmath,amssymb,amsthm}
\usepackage{graphicx}
\graphicspath{{images/}}
\usepackage[colorinlistoftodos]{todonotes}
\usepackage[colorlinks=true, allcolors=blue]{hyperref}

\usepackage[numbered,framed]{matlab-prettifier}

\usepackage{todonotes}

% Algorithms
\usepackage{algorithm}
\usepackage{algpseudocode}
\renewcommand{\algorithmicrequire}{\textbf{Input:}}
\renewcommand{\algorithmicensure}{\textbf{Initialization:}}

\usepackage{color}
\newcommand{\ville}[1]{{\textcolor{red}{#1}}}
\newcommand{\eva}[1]{{\textcolor{blue}{#1}}}

% Headers
\usepackage{fancyhdr}
\pagestyle{fancy}
\fancyhf{}

%\chead[fMRI joint reconstruction]{fMRI joint reconstruction}
\chead{Dynamic MRI Reconstruction with Structural Prior}
\cfoot{\thepage}

\newcommand{\mat}[1]{\lstinline[style=Matlab-editor]{#1}}

\newcommand{\TV}{\mathrm{TV}}
\newcommand{\TVd}{\mathrm{TV}_d}
\newcommand{\C}{\mathbb{C}}
\newcommand{\R}{\mathbb{R}}
\newcommand{\Rplus}{\mathbb{R}_+}
\newcommand{\ICBTV}{\mathrm{ICB}_\TV}

\newcommand{\dx}{\,\mathrm{d}x}

\newcommand{\x}{\mathbf{x}}
\newcommand{\fbold}{\mathbf{f}}
\newcommand{\ubold}{\mathbf{u}}
\newcommand{\vbold}{\mathbf{v}}
\newcommand{\ybold}{\mathbf{y}}
\newcommand{\zbold}{\mathbf{z}}
\newcommand{\Tbold}{\mathbf{T}}

\newcommand{\pa}{p_0^\eta}
\newcommand{\qa}{q_0^\eta}

\newcommand{\diverg}{\mathrm{div}}

\newcommand{\Kcal}{\mathcal{K}}
\newcommand{\Lcal}{\mathcal{L}}
\newcommand{\Fcal}{\mathcal{F}}
\newcommand{\Scal}{\mathcal{S}}
\newcommand{\Pcal}{\mathcal{P}}

\newcommand{\Real}{\mathrm{Re}}
\newcommand{\Imag}{\mathrm{Im}}

\newcommand{\qt}{\tilde{q}}

\newcommand{\prox}{\mathrm{prox}}
\newcommand{\proj}{\mathrm{proj}}

% Spaces
\newcommand{\BVR}{\mathrm{BV}(\Omega;\mathbb{R})}
\newcommand{\BVC}{\mathrm{BV}(\Omega;\mathbb{C})}

% Environments
\newtheorem{mydef}{Definition}
\newtheorem{mythm}{Theorem}
\newtheorem{myprop}{Proposition}
\newtheorem{mylem}{Lemma}
\newtheorem{myrem}{Remark}

\numberwithin{equation}{section}

\begin{document}
\title{Dynamic MRI Reconstruction from Undersampled Data with an Anatomical Prescan}
\author{Julian Rasch\thanks{Applied Mathematics M\"unster: Institute for Analysis and Computational Mathematics, 
Westf\"alische Wilhelms-Universit\"at (WWU) M\"unster. Einsteinstr. 62, 48149 M\"unster, Germany. e-mail: julian.rasch@wwu.de}, 
Ville Kolehmainen\thanks{Department of Applied Physics, University of Eastern Finland, POB1627, 70211 Kuopio, Finland},
Riikka Nivaj\"arvi\thanks{Kuopio Biomedical Imaging Unit,
A. I. Virtanen Insititute for Molecular Sciences, 
University of Eastern Finland, POB1627, 70211 Kuopio, Finland}, 
Mikko Kettunen\footnotemark[3], \\
Olli Gr\"ohn\footnotemark[3], 
Martin Burger\footnotemark[1] \ and 
Eva-Maria Brinkmann\footnotemark[1]}

\maketitle
\abstract{
The goal of dynamic magnetic resonance imaging (dynamic MRI) is to visualize tissue properties and their local changes over time that are traceable in the MR signal.
We propose a new variational approach for the reconstruction of subsampled dynamic MR data, which combines smooth, temporal regularization with spatial total variation regularization. In particular, it furthermore uses the infimal convolution of two total variation Bregman distances to incorporate structural a-priori information from an anatomical MRI prescan into the reconstruction of the dynamic image sequence. 
The method promotes the reconstructed image sequence to have a high structural similarity to the anatomical prior, while still allowing for local intensity changes which are smooth in time.
The approach is evaluated using artificial data simulating functional magnetic resonance imaging (fMRI), and experimental dynamic contrast-enhanced magnetic resonance data from small animal imaging using radial golden angle sampling of the $k$-space. \\

\noindent {\bf Keywords: } Dynamic magnetic resonance imaging, spatio-temporal regularization, structural prior, infimal convolution of Bregman distances, total variation, golden angle subsampling
}

Reinforcement learning has achieved great success in areas such as Game-playing \citep{silver2018general,vinyals2019grandmaster}, robotics \cite{kober2013reinforcement}, large language models \citep{ouyang2022training}, etc.
However, due to safety concerns or physical limitations, in some real-world reinforcement learning problems, we must consider additional constraints that may influence the optimal policy and the learning process \citep{garcia2015comprehensive}.
% For example, a robotic arm must not take actions that may cause harm to itself or the environments.
A standard framework to handle such cases is the constrained Markov Decision Process (CMDP) \citep{altman1999constrained}.
Within the CMDP framework, the agent has to maximize
the expected cumulative reward while
obeying a finite number of constraints, which are usually in the form of expected cumulative cost criteria.

However, we are sometimes concerned with the problem with a continuum of constraints.
For example,
the constraints we meet might be time-evolving or subject to uncertain parameters, which
cannot be formulated as an ordinary CMDP
(see Examples \ref{Example_Time_Evolving} and  \ref{Example_Uncertain}).
In this paper we would study a generalized CMDP  
to address the above problem.  Because the constraints are not only infinite-number but also lie
in a continuous set,
the generalization is not trivial. Fortunately, we find that we can borrow the idea behind semi-infinite programming (SIP) \citep{remez1934determination, hettich1993semi} to deal with the semi-infinite constraints.
Accordingly, we propose \emph{semi-infinitely constrained Markov decision processes} (SICMDPs)
as a novel complement to the ordinary CMDP framework.
%More specifically,  an SICMDP model %, we consider 
%contains a continuum of constraints whereas an ordinary CMDP contains a finite number of constraints. 

%This generalization is natural but not trivial. However, we can brows the idea  
%The idea is quite natural and can be backtracked
%to the practice of extending linear programming to linear semi-infinite programming (LSIP) %\cite{remez1934determination, GobernaLSIO1998}.
%In addition, 
%As a complementary approach to the ordinary CMDP framework, 
%SICMDP can be used to model these problems  which cannot be described by a finite number of constraints
%that are not covered by .
%For example,
%the restrictions we consider can be time-evolving or subject to uncertain parameters
%, thus
%cannot be described by a finite number of constraints but a continuum of constraints 
%(see Examples \ref{Example_Time_Evolving} and  \ref{Example_Uncertain}).

We also present two reinforcement learning algorithms to solve SICMDPs called SI-CRL and SI-CPO, respectively.
SI-CRL is a model-based reinforcement learning algorithm designed for tabular cases, and SI-CPO is a policy optimization algorithm for non-tabular cases.
% and analyze its performance both theoretically and empirically.
The main challenge is that we need to deal with a continuum of constraints, thus reinforcement learning algorithms for ordinary CMDPs do not work anymore.
In SI-CRL, we tackle this difficulty by first transforming the reinforcement learning problem to an equivalent LSIP problem, which can then be solved using methods in the LSIP literature like the dual exchange methods \citep{Hu1990,reemtsen1998numerical}.
In SI-CPO, we resort to the idea of cooperative stochastic approximation developed in \cite{lan2020algorithms, wei2020comirror}.
As far as we know, we are the first to introduce tools from semi-infinitely programming (SIP) into the reinforcement learning community for solving constrained reinforcement learning problems.

% To the best of our knowledge, we are the first to apply tools from semi-infinitely programming (SIP) to solve reinforcement learning problems.
Furthermore, we give theoretical analysis for both SI-CRL and SI-CPO.
We decompose the error of SI-CRL into two parts: the statistical error from approximating the true SICMDP with an offline dataset and the optimization error due to the fact that the solution of the LSIP problem obtained by the dual exchange method is inexact.
On the optimization side, we show that the iteration complexity of SI-CRL is $O\left(\left\{\mathrm{diam}(Y)L\sqrt{|\gS|^2|\gA|m}/\left[(1-\gamma)\epsilon\right]\right\}^m\right)$.
On the statistical side, we show that the sample complexity of SI-CRL is $\widetilde O\left(\frac{|S|^2|A|^2}{\epsilon^2(1-\gamma)^3}\right)$ if the offline dataset is generated by a generative model, and $\widetilde O\left(\frac{|S||A|}{\nu_{\min} \epsilon^2(1-\gamma)^3}\right)$ if the dataset is generated by a probability measure $\nu$ as considered in \cite{chen2019information}.
Here $\widetilde O$ means that all logarithm terms are discarded.
For SI-CPO, things become a little more complicated because other than the statistical error and the optimization error, we also need to consider the function approximation error, which comes from imperfect policy parametrizations.
It is shown if the function approximation error can be controlled to $O(\epsilon)$ order, the iteration complexity of SI-CPO is $\widetilde{O}\left(\frac{1}{\epsilon^2(1-\gamma)^6}\right)$ and the sample complexity of SI-CPO is $\widetilde{O}(\frac{1}{\epsilon^4(1-\gamma)^{10}})$.
Here our iteration complexity bound is equivalent to a typical $\widetilde O(1/\sqrt{T})$ global convergence rate.

We perform a set of numerical experiments to illustrate the SICMDP model and validate our proposed algorithms.
Specifically, we examine two numerical examples, namely the discharge of sewage and ship route planning.
Through the discharge of sewage example, we show the advantage of the SICMDP framework over the CMDP baseline obtained by naive discretization in modeling realistic sequential decision-making problems.
Moreover, we demonstrate the effectiveness of the SI-CRL and SI-CPO algorithms in such tabular environments. 
In the ship route planning example, we illustrate the benefits of the SICMDP framework and the ability of the SI-CPO algorithm to address complex continuous control tasks involving continuous state spaces with modern deep reinforcement learning techniques.

% In summary, our contributions are listed as follows.
% First, we present the SICMDP model, which can be viewed as a generalization of the ordinary CMDP model.
% Second, we propose an algorithm to perform reinforcement learning for SICMDPs, which is called SI-CRL, and we believe that we are the first to apply tools from SIP
% to solve reinforcement learning problems.
% Third, we give a theoretical analysis of SI-CRL and identify both its sample complexity and iteration complexity.
% In addition, we perform numerical experiments to illustrate the SICMDP model and validate the SI-CRL algorithm.
% \{This paragraph can be removed!!! \}





\section{Modelling}
\label{sec:modelling}

The linear coefficients in (1) are tailored to assessing only linear mediating dependencies. To overcome this
limitation, this work considers a non-linear model by introducing a set of node dependent nonlinear functions $\{f_i\}_{i=1}^{N}$. 
Previous works on nonlinear topology identification \cite{shen2019nonlinear,tank2017interpretable,money2021online} estimate nonlinear multivariate models without necessarily assuming linear dependencies in an underlying space; rather, they directly estimate non-linear functions from and into the real measurement space without assuming an underlying structure. In our work, we assume that the multivariate data can be explained as the nonlinear output of a set of observation functions $\{f_i\}_{i=1}^{N}$ with a VAR process as an input. Each function $f_i$ represents a different non-linear distortion at the $i$-th node.

Given data time series, the task is to jointly learn the non-linearities together with a VAR topology in a feature space which is linear in nature, where the outputs of the functions $\{f_i\}_{i=1}^{N}$ belong to. Such functions are required to be invertible, so that sensor measurements can be mapped into the latent feature space, where the linear topology (coefficients) can be used to generate predictions, which can be taken back to the real space through $\{f_i\}_{i=1}^{N}$. In our model, prediction involves the composition of several functions, which can be modeled as neural networks. The nonlinear observation function at each node can be parameterized by a NN that is in turn a universal function approximator \cite{cybenko1989approximation}. 
%\textcolor{red}{ADD REFERENCE TO UNIVERSAL REPRESENTATION THEOREM}. 
Consequently, the topology and non-linear per-node transformations can be seen in aggregation as a DNN, and its parameters can be estimated using appropriate deep learning techniques.

\begin{figure}[h]
\vspace{-0.6cm}
\centering
\includegraphics[width=0.8\textwidth]{figures/lowres/circle_only.jpg}
\vspace{-0.5cm}
\caption{{Causal dependencies among a set of time series are linear in the latent space represented by the green circle. However, the variables in the latent space are not available, only nonlinear observations (output of the functions $f_i$) are available.}\vspace{-0.3cm}} 
\label{fig:swn3}
\end{figure}


The idea is illustrated in Figure \ref{fig:swn3}. The green circle represents the underlying latent vector space. The exterior of the circle is the space where the sensor measurements lie, which need not be a vector space. The blue lines show the linear dependency between the time series inside the latent space. The red line from each time series shows the transformation to the measurement space. Each sensor is associated with a different nonlinear function. Specifically, if $y_i[t]$ denotes the $i$-th time series in the latent space, the measurement (observation) is modeled as $z_{i}[t]=f_{i}\left(y_{i}[t]\right)$. The function $f_i$  is parameterized as a neural network layer with $M$ units, expressed as follows:
\begin{align} 
\label{eq:f_model}
    f_{i}\left(y_{i}\right)=\sum_{j=1}^{M} \alpha_{ij} \sigma\left(w_{ij}y_{i}-k_{ij}\right)+b_{i}
\end{align}

For the function $f_i$ to be monotonically increasing (which guarantees invertibility), it suffices to ensure that $\alpha_{ij}$ and  $w_{ji}$ are positive $\forall j$. The pre-image of $f_i$ is the whole set of real numbers, but the image is an interval $(\ubar{z}_i, \bar{z}_i)$, which is in accordance ro the fact that sensor data are usually restricted to a dynamic range. If the range is not available a priori but sufficient data is available, bounds for the operation interval can be easily inferred.

Let us remark three important advantages in the proposed model: 
\begin{itemize}
    \item It is substantially more expressive than the linear model, while capturing non-linear dependencies with lower complexity than other non-linear models. 
    \item It allows to predict with longer time horizons ahead within the linear latent space. Under a generic non-linear model, the variance of a long-term prediction explodes with the time horizon.
    \item Each non-linear nodal mapping can also adapt and capture any possible drift or irregularity in the sensor measurement, thus, it can directly incorporate imperfections in the sensor measurement itself due to, e.g. lack of calibration.    
\end{itemize}

\subsection{Prediction}

\label{sec:prediction}

Given accurate estimates of the nonlinear functions $\{f_i\}_{i=1}^N$, their inverses, and the parameters of the VAR model, future measurements can be easily predicted. Numerical evaluation of the inverse of $f_i$ as defined in \eqref{eq:f_model} can easily be done with a bisection algorithm. 

Let us define $g_i = f_i^{-1}$. Then, the prediction consists of three steps, the first one being mapping the previous samples back into the latent vector space:
\begin{subequations}
    \begin{equation} \label{eq:forward_g}
        \tilde{y}_{i}[t-p] =g_{i}\left(z_{i}[t-p]\right)
    \end{equation}
Then, the VAR model parameters are used to predict the signal value at time t (also in the latent space):
    \begin{equation} \label{eq:forward_A}
        \hat{y}_{i}[t] =\sum_{p=1}^{p}\sum_{j=1}^{n}  a_{i j}^{(p)} \tilde{y}_{j} [t-p] 
    \end{equation}
Finally, the predicted measurement at each node is obtained by applying $f_i$ to the latent prediction:
    \begin{equation} \label{eq:forward_f}
        \hat{z}_{i}[t] =f_{i}\left(\hat{y}_{i}[t]\right) 
    \end{equation}
\end{subequations}

\begin{figure}[t]
\vspace{-1.7cm}
\hspace{-1.2cm}
\vspace{-0cm}
\includegraphics[width=1.2\textwidth]{figures/network.png}
\vspace{-2.8cm}
\caption{{ Schematic for modeling Granger causality for a toy example with 2 sensors. }} 
\label{fig:example_2sensors}
\end{figure}

These prediction steps can be intuitively visualized as a neural network. The next section formulates an optimization problem intended to learn the parameters of such a neural network. For a simple example with 2 sensors, the network structure is shown in Figure \ref{fig:example_2sensors}.

\section{Problem formulation}

The functional optimization problem consists in minimizing $\|{z}[t]-\hat z\|_{2}^{2}$ (where $z[t]$ is a vector collecting the measurements for all N sensors at time $t$), subject to the constraint of $f_i$ being invertible $\forall i$, and the image of $f_i$ being $(\ubar{z}_i, \bar{z}_i)$. The saturating values can be obtained from the nominal range of the corresponding sensors, or can be inferred from data. 
 
 Incorporating equation (1),  the optimization problem can be written as:
\begin{subequations}
\label{eq:optimization_problem}
\begin{align}
    \min _{f, A} \;\;& \| {z}[t]-f\Big(\sum_{p=1}^{p} A^{(p)}\big[g(z[t-p])\big]\Big) \|_{2}^{2} 
    \\
    \textrm{s. to:}\;\  
    & \sum_{j=1}^{M}\alpha_{j i} 
      = \bar{z}_i %max(z_{i}[t-p])
        - \ubar{z}_i %min(z_{i}[t-p]) 
        \; \forall i
        \label{eq:constraint_alpharange}
    \\
    & b_{i} = \bar{z}_i%min(z_{i}[t-p]) 
        \; \forall i
        \label{eq:constraint_barz}
    \\
    & \alpha_{j i} \geq0 
        \; \forall i, j
        \label{eq:constraint_alphapos}
        \\
    & w_{j i} \geq0 
        ; \forall i, j
        \label{eq:constraint_wpos}
\end{align}
 \end{subequations}
% \textcolor{red}{comment about the max and min saying that the function f saturates at z bar and  below if you know a priori you impose  directly or else infer from the data}     
         
The functional optimization over $f_i$ is tantamount to optimizing over $\alpha_{j i}$,$w_{j i}$,  $k_{j i}$ and $b_{i}$. The main challenge to solve this problem is that there is no closed form for the inverse function $g_i$. This is addressed in the ensuing section.

\iffalse
\begin{lstlisting}
// Algorithm for function g implementation
def g(x,i):
  niter = 10000
  vy = 0
  for j in range(niter):
    vy = vy - (ginverse(vy,i)-x)/gprime(vy,i)  
  return vy
def gprime(x,i): 
  a=0.01
  for j in range(m):
            a = a + alpha[j][i] * sigmoid(x-k[j][i]) * (1-sigmoid(x-k[j][i])) 
  return a
def ginverse(x,i): 
  a = 0   
  for j in range(m):
      a = a + alpha[j][i] * sigmoid(x-k[j][i]) + b[i][0]
  return a
\end{lstlisting}
\fi



%\subsection{Implementation as a neural network}



\section{Learning algorithm}

Without a closed form for $g$, we cannot directly obtaining gradients with automatic differentiation such as Pytorch,%\footnote{One could try to differentiate through all iterations of the Newton method that numerically inverts $f_i$, but there is no guarantee of obtaining a stable gradient estimate, and the computational cost is probably high.}
as is typically done in deep learning with a stochastic gradient-based optimization algorithm. Fortunately, once $\{g_i(\cdot)\}$ is numerically evaluated, the gradient at that point can be calculated with a relatively simple algorithm, derived via implicit differentiation in Sec. \ref{ss:backprop}. Once that gradient is available, the rest of the steps of the backpropagation algorithm are rather standard.

\subsection{Forward equations}
The forward propagation equations are given by the same steps that are used to predict next values of the time series $z$:

\begin{subequations}
    \begin{equation} \label{eq:forward_g}
        \tilde{y}_{i}[t-p] =g_{i}\left(z_{i}[t-p], \theta_{i}\right)
    \end{equation}
    \begin{equation} \label{eq:forward_A}
        \hat{y}_{i}[t] =\sum_{p=1}^{p}\sum_{j=1}^{n}  a_{i j}^{(p)} \tilde{y}_{j} [t-p] 
    \end{equation}
    \begin{equation} \label{eq:forward_f}
        \hat{z}_{i}[t] =f_{i}\left(\hat{y}_{i}[t], \theta_{i}\right) 
    \end{equation}
    \begin{equation} \label{eq:forward_cost}
        C[t] =\sum_{n=1}^{N}\left(z_{n}[t]-\hat{z}_{n}[t]\right)^{2}_\cdot
    \end{equation}
\end{subequations}
Here, the dependency of the nonlinear functions with the neural network parameters is made explicit, where $$\theta_{i}=\left[\begin{array}{l}\alpha_{ i} \\  w_i\\ k_{ i} \\ b_{i}\end{array}\right] \text{ and } 
\alpha_{i}=\left[\begin{array}{c}
\alpha_{i 1} \\
\alpha_{i 2} \\
\vdots \\
\alpha_{i M}
\end{array}\right],w_{i}=\left[\begin{array}{c}
k_{i 1} \\
k_{i 2} \\
\vdots \\
k_{i M}
\end{array}\right], k_{i}=\left[\begin{array}{c}
k_{i 1} \\
k_{i 2} \\
\vdots \\
k_{i M}
\end{array}\right]_. $$

\subsection{Backpropagation equations}
\label{ss:backprop}
The goal of backpropagation is to  calculate the gradient of the cost function with respect to the VAR parameters and the node dependent function parameters $\theta_i$.

%The dimension of the parameter vector $\theta_{i}$ is for the i th sensor  will be (2M+1). 
The gradient of the cost is obtained by applying the chain rule as following:
%\setcounter{equation}{9}
\begin{equation} \label{chainrule1}
\begin{array}{c}

\frac{d C[t]}{d \theta_{i}}=\sum_{n=1}^{N} \frac{\partial C}{\partial \hat{z}_{n}[t]}  \frac{\hat{z}_{n}[t]}{\partial \theta_{i}} \\
\text { where } \frac{\partial C}{\partial \hat{z}_{n}[t]}=2(\hat{z}_n[t]-z_n[t]) = S_n
\end{array}
\end{equation}


\begin{equation} \label{chainrule2}
\frac{\partial \hat{z}_{n}[t]}{\partial \theta_{i}}=\frac{\partial f_{n}}{\partial \hat{y}_{n}}  \frac{\partial \hat{y}_{n}}{\partial \theta_{i}}+\frac{\partial f_{n}}{\partial \theta_{n}}  \frac{\partial \theta_{n}}{\partial \theta_{i}} \\
\quad 
\end{equation}
$$
\text { where } \frac{\partial \theta_{n}}{\partial \theta_{i}}=\left\{
\begin{array}{l}
I, n = i \\
0, n  \neq i
\end{array}\right.
$$
Substituting equation \eqref{chainrule1} into \eqref{chainrule2} yields
\begin{equation} \label{derivativecost1}
\frac{d C[t]}{d \theta_{i}}=\sum_{n=1}^{N} S_{n}\left(\frac{\partial f_{n}}{\partial \hat{y}_{n}}  \frac{\partial \hat{y}_{n}}{\partial \theta_{i}}+\frac{\partial f_{n}}{\partial \theta_{n}}  \frac{\partial \theta_{n}}{\partial \theta_{i}}\right)_\cdot
\end{equation}
Equation\eqref{derivativecost1} can be simplified as:

\begin{equation} \label{derivativecost2}
    \frac{d C[t]}{d \theta_{i}}
    =
    S_{i}
    \frac{\partial f_{i}}{\partial \theta_{i}}
    +\sum_{n=1}^{N} S_{n}\frac{\partial f_{n}}{\partial \hat{y}_{n}}  \frac{\partial \hat{y_n}}{\partial \theta_{i}}.
\end{equation}


$\text { The next step is to derive } \frac{\partial \hat{y}_{n}}{\partial \theta_{i}} \text { and } \frac{\partial f_{i}}{\partial \theta_{i}}$ of  equation \eqref{derivativecost2}:


\begin{equation} \label{gradientyhat}
    \frac{\partial \hat{y}_{n}[t]}{\partial \theta_i}
    =
    \sum_{p=1}^{P}\sum_{j=1}^{N}  
        a_{n j}^{(p)} 
        \frac{\partial}{\partial \theta_{j}} \tilde{y}_{j}[t-p]
        \frac{\partial \theta_{j}}{\partial \theta_{i}} .
\end{equation}

With $f_{i}^{\prime}\left(z\right)= 
\frac{\partial f_i\left(z, \theta_{i}\right)}{\partial\left(z\right)}, $  
expanding $\tilde{y}_j[t-p]$ in equation \eqref{gradientyhat} changes \eqref{derivativecost2} to:

%\begin{equation} \label{derivativecost3}
%\text { So } \frac{d C[t]}{d \theta_{i}}=S_{i}\left(\frac{\partial f_{i}}{\partial \theta_{i}}\right)+\sum_{p=1}^{P} \sum_{n=1}^{N} S_{n} f_{n}^{\prime}(\hat{y}_n[t])  a_{n i}^{(p)} \frac{\partial}{\partial \theta_{i}} g_{i}\left(z_{i}[t-p],\theta_{i}\right)_\cdot
%\end{equation}

%equation \eqref{derivativecost3} becomes:

\begin{equation} \label{derivativecost4}
 \frac{d C[t]}{d \theta_{i}}=S_{i}\left(\frac{\partial f_{i}}{\partial \theta_{i}}\right)+\sum_{n=1}^{N} S_{n}\left(f_{n}^{\prime}(\hat{y}_{n}[t]) \sum_{p=1}^{P} a_{n i}^{(p)} \frac{\partial}{\partial \theta_{i}} g_{i}\left(z_{i}[t-p],\theta_{i}\right)\right)_\cdot
\end{equation}

Here, the vector 
$$\frac{\partial f_i\left(z, \theta_{i}\right)}{\partial \theta_{i}} %\text{ in equation } \eqref{derivativecost4} 
= \left[
\frac{\partial f_i\left(z, \theta_{i}\right)}{\partial \alpha_{i}} 
\frac{\partial f_i\left(z, \theta_{i}\right)}{\partial w_{i}} 
\frac{\partial f_i\left(z, \theta_{i}\right)}{\partial k_{i}} 
\frac{\partial f_i\left(z, \theta_{i}\right)}{\partial b_{i}} \right]$$ can be obtained by standard or automated differentiation via, e.g., Pytorch \cite{NEURIPS2019_9015}.


However, \eqref{derivativecost4} involves the calculation of $\frac{\partial g_i(z, \theta_{i})}{\partial \theta_{i}}$, which is not straightforward to obtain. Since $g_i(z)$ can be computed numerically, the derivative can be obtained by implicit differentiation, realizing that the composition of $f_i$ and $g_i$ remains invariant, so that its total derivative is zero:

\begin{equation} \label{dfwrttheta1}
\frac{d}{d \theta_{i}}\left[f_i\left(g_i\left(z, \theta_{i}\right), \theta_{i}\right)\right]=0
\end{equation}

\begin{equation} \label{dfwrttheta2}
\Rightarrow \frac{\partial f_i\left(g_i\left(z, \theta_{i}\right), \theta_{i}\right)}{\partial g\left(z, \theta_{i}\right)} \frac{\partial g\left(z, \theta_{i}\right)}{\partial \theta_{i}}+\frac{\partial f_i\left(z, \theta_{i}\right)}{\partial \theta_{i}}=0
\end{equation}

\begin{equation} \label{dfwrttheta3}
\Rightarrow {f^{\prime}_i(g_i(z,\theta_{i}))} \frac{\partial g\left(z, \theta_{i}\right)}{\partial \theta_{i}}+\frac{\partial f_i\left(z, \theta_{i}\right)}{\partial \theta_{i}}=0
\end{equation}


\begin{equation} \label{dgwrttheta1}
\text { Hence } \frac{\partial g_i\left(z, \theta_{i}\right)}{\partial \theta_{i}}=
-\big\{f^{\prime}_i(g_i(z,\theta_{i}))\big\}^{-1}{\left(\frac{\partial f_i\left(z, \theta_{i}\right)}{\partial \theta_{i}}\right)}_\cdot 
\end{equation}
%
%$$\text{where }\frac{\partial g_i\left(z, \theta_{i}\right)}{\partial \theta_{i}} =  \left[\begin{array}{l}
%\frac{\partial g_i\left(z, \theta_{i}\right)}{\partial \alpha_{i}} \\
%\frac{\left.\partial g_i\left(z\right, \theta_{i}\right)}{\partial k_{i}} \\
%\frac{\partial g_i(z,\theta_{i})}{\partial b_{i}}
%\end{array}\right]_\cdot$$
%

The gradient of $C_T$ w.r.t. the VAR coefficient $a^{(p)}_{ij}$ is calculated as follows: 

\begin{equation} \label{dCwrtthetaa}
\frac{d C[t]}{d a^{(p)}_{i j}}=\sum_{n=1}^{N} S_{n}  \frac{\partial f_{n}}{\partial \hat y_{n}}  \frac{\partial \hat{y}_{n}}{\partial a_{i j}^{(p)}}
\end{equation}

\begin{equation} \label{dCwrtthetaa1} \nonumber
\frac{\partial \hat{y}_{n}[t]}{\partial a_{i j}^{(p)}}=\frac{\partial}{\partial a_{i j}^{(p)}} \sum_{p^\prime=1}^{P} \sum_{q=1}^{N} a_{n q}^{(p^\prime)} \tilde{y}_{q}[t-p]
\end{equation}

\begin{equation} \label{dCwrtthetaa2}
\begin{array}{c}
\text { where } 
\frac{\partial a_{n q}^{(p^\prime)}}{\partial a_{i j}^{(p)}}=\left\{\begin{array}{l}
1, n=i, p = p^\prime, \text { and } q=j \\
0, \text {otherwise}
\end{array}\right.
\end{array}
\end{equation}

%\begin{equation} \label{dCwrtthetaa3}
%\frac{\partial \hat{y}_{n}[t]}{\partial a_{i j}^{(p)}}=\left\{\begin{array}{c}
%\sum_{p=1}^{P} \tilde{y}_{j}[t-p], i=n \\
%0, \quad i \neq n
%\end{array}\right.
%\end{equation}

\begin{equation} \label{dCwrtthetaa4}
\frac{d C[t]}{d a_{i j}^{(p)}}=S_i f_{i}^{\prime}\left(\hat{y}_{i}[t]\right) \tilde{y}_{j}[t-p]_\cdot 
\end{equation}

Even though the backpropagation cannot be done in a fully automated way, it can be realized by implementing equations \eqref{dgwrttheta1} and \eqref{derivativecost4} after automatically obtaining the necessary expressions.

\subsection{Parameter optimization}

The elements in $\{A^{(p)}\}_{p=1}^P,$ and $\{\theta_i\}_{i=1}^{N}$ can be seen as the parameters of a NN. Recall from Fig. \ref{fig:example_2sensors} that the prediction procedure resembles a typical feedforward NN as it interleaves component-wise nonlinearities with multidimensional linear mappings. The only difference is that one of the layers computes the inverse of a given function, and its backward step has been derived. Moreover, the cost function in \eqref{eq:optimization_problem} is the mean squared error (MSE).

The aforementioned facts support the strategy of learning the parameters using state-of-the-art NN training techniques. A first implementation has been developed using stochastic gradient descent (SGD) and its adaptive-moment variant Adam \cite{kingma2014adam}. Constraints \eqref{eq:constraint_alpharange}-\eqref{eq:constraint_wpos} are imposed by projecting the output of the optimizer into the feasible set at each iteration.

The approach is flexible enough to be extended with neural training regularization techniques such as dropout %\cite{dropout} 
or adding a penalty based on the L1 or L2 norm of the coefficients, to address the issue of over-fitting and/or promote sparsity. The batch normalization technique can be proposed to improve the training speed and stability. %Last but not least, future developments include improving the interpretability by imposing sparsity. 

\iffalse
\begin{lstlisting}
// Algorithm for back propagation implementation
def backward_propagation(H,alpha,b,A,k,X_train, z_pred):
for i in range(n):
          for p in range(n):
              a1=0
              for j in range(len(X_train)):
                  a1  = a1-2*(X_train[j][p] - z_pred[j][p])*(g(X_train[j][i],i))*(f(H[j][p],p))
              dA[i][p] =a1
    for i in range(n): 
        for p in range(n):                        
            for j in range(len(X_train)):
                dalpha[j][i] = dalpha[j][i] -2*(X_train[j][i] - z_pred[j][i])  *(f(H[j][p],p)*A[i][p]*dalphag(X_train[j][i],i)) -2*(X_train[j][i] - z_pred[j][i])*sigmoid(H[j][i]-k[j][i])
    for i in range(n): 
        for p in range(n):                       
            for j in range(len(X_train)):
                dk[j][i] = dk[j][i] -2*(X_train[j][i] - z_pred[j][i]) * (f(H[j][p],p)*A[i][p]*dkg(X_train[j][i],i)) -2*(X_train[j][i] - z_pred[j][i])*alpha[j][i]*sigmoid(H[j][i]-k[j][i])
                *(1-sigmoid(H[j][i]-k[j][i]))
    for i in range(n): 
        a1 = 0
        for p in range(n):                        
            for j in range(len(X_train)):
                a1  = a1 -2*(X_train[j][i] - z_pred[j][i])  *(f(H[j][p],p)*A[i][p]*dbg(X_train[j][i],i)) -2*(X_train[j][i] - z_pred[j][i])*alpha[j][i]*sigmoid(H[j][i]-k[j][i])
        a1 = db[i][0]            
    return 0

\end{lstlisting}
\fi
% !TEX root = main.tex

\begin{figure*}[t!]
  \centering
\includegraphics[width=0.325\textwidth]{./figs/0_simp_c1c2.pdf}
\includegraphics[width=0.315\textwidth]{./figs/0_simp_c3c4.pdf}
\includegraphics[width=0.26\textwidth]{./figs/0_simp_quad_bnds.pdf}
  \caption{We verify that the performative risk bounds in Assumption~\ref{ass:exist_V} are satisfied in the example discussed in Section~\ref{sec:example}. (a) As a function of $r$ (the radius of the domain where the inequalities hold), we show the tightest constants $c_1$ and $c_2$ for the bound. We also plot $\sqrt{c_1/c_2}r$, which is the radius of a neighborhood of $x=0$ to which Theorem~\ref{th:perturb1} can be applied. (b) As a function of $r$, we show the tightest constants for $c_3$ and $c_4$. (c) Choosing the $c_1$ and $c_2$ constants for $r = 0.5$, we visualize how the quadratic bounds hold for the performative risk locally.}
  \label{fig:simp_ex_demo}
\end{figure*}
In this section, we revisit the models introduced in Section~\ref{sec:example}. We demonstrate how the results of Sections~\ref{sec:analysis_prm} and~\ref{sec:analysis_RGD} can be applied. First, we show that the example satisfies Assumption~\ref{ass:exist_V} and we calculate its corresponding constants. Second, we apply Theorem~\ref{th:perturb1} and show the theoretical convergence rates match simulated trajectories. 
Finally, we also apply Theorem~\ref{th:perf_align} to the example from Section~\ref{sec:simple_ex} and characterize the class of distribution shifts satisfy the performative alignment condition.

\subsection{Checking the curvature of the performative risk and region of convergence}

Recall the example from Section~\ref{sec:simple_ex}, where $x$ was a scalar, the loss function was the squared error, and the decision-dependent distribution was a Bernoulli random variable whose distribution was determined by $p(\cdot)$. In this section, we consider the specific decision-dependent distribution shift $p = \varphi$, which is defined in Equation~\eqref{eq:varphi_def}.

When we consider this example, we can see that the bounds on Assumption~\ref{ass:exist_V} cannot hold globally, which matches our previous observation that there are multiple isolated performative risk minimizers. However, these bounds may hold locally: we can view the constants $(c_i)_{i=1}^4$ from Assumption~\ref{ass:exist_V} as a function of the size of the domain $r$.

For concreteness, let us focus on the equilibrium point $x = 0$. Recall that Assumption~\ref{ass:exist_V} must hold locally, on the domain $\{ x : |x-x^*| \le r\}$. As we increase $r$, the constants will worsen; we visualize this in Figure~\ref{fig:simp_ex_demo}(a)--(b). Note that these bounds only have to hold locally around the equilibria, as visualized in Figure~\ref{fig:simp_ex_demo}(c). Furthermore, the gradient bounds in Assumption~\ref{ass:exist_V} cannot hold beyond $r > 0.40$, since $\nabla PR(x) = 0$ at that point.

Recall that the convergence results of Theorem~\ref{th:perturb1} can only apply to all initial conditions satisfying $|x_0 - x^*| < \sqrt{c_1/c_2}r$; we visualize this as well in Figure~\ref{fig:simp_ex_demo}(a). 
On the set $(0,0.40]$, we can see the quantity $\sqrt{c_1/c_2}r$ is the largest at $r = 0.4$, with constants $c_1 = 0.50$ and $c_2 = 1.78$. 
Thus, around the equilibrium $x = 0$, the theorem can be applied to all points in the set $\{ x : |x| \le 0.21 \}$, with $\delta = 0$. Thus, our theorem shows that all points in this neighborhood of $x = 0$ will converge. This under-approximates the true region of attraction, which we numerically saw to be $\{ x : x < 0.23 \}$.



\subsection{Performative alignment with squared error and Bernoulli distributions}
\label{sec:perf_align_ex}

We again consider the example from Section~\ref{sec:simple_ex}. However, in this section, we consider a general decision-dependent distribution shift $p(\cdot)$. 
% Recall the example from Section~\ref{sec:simple_ex}, where $x$ was a scalar, the loss function was the squared error, and the decision-dependent distribution was a Bernoulli random variable whose distribution was determined by $p(\cdot)$.  
We suppose that $p(0) = 0$ and $p(1) = 1$, so we have two performative risk minimizers as in our previous example. We have $\nabla_{x_1}R(x,x) = x - p(x)$ and $\nabla_{x_2}R(x,x) = (1/2 - x) p'(x)$. 
The performative alignment condition becomes:
\begin{equation}
    \label{eq:perf_align_ex}
    |1/2 - x|^2 |p'(x)|^2 \le (p(x)-x)(1/2 - x)p'(x)
\end{equation}
Theorem~\ref{th:perf_align} states that if this condition holds for all $x \in (0,c)$, then any initial conditions $x_0 \in (0,c)$ will converge to $x = 0$. Similarly, if this condition holds for all $x \in (c,1)$, then all initial conditions in $(c,1)$ will converge to $x = 1$. Theorem~\ref{th:perf_align} also implies that this condition cannot be satisfied for all $x \in (0,1)$, as then these initial conditions would converge to \textit{both} $x = 0$ and $x = 1$.

If we suppose that $p(\cdot)$ is monotonic on $(0,1)$, i.e. $p'(x) \ge 0$, we can also interpret the performative alignment condition as follows. For $x \in (1/2,1)$, the performative alignment condition becomes $p(x) - x \ge (1/2 - x)p'(x)$. In this regime, $(1/2 - x)p'(x) \le 0$. In this setting, if $p(x) - x$ is too negative, the RGD flow will push $x$ away from the nearby minimizer $x = 1$. Similarly, for $x \in (0,1/2)$, the condition becomes $p(x) - x \le (1/2 - x)p'(x)$. In this regime, $(1/2 - x)p'(x) \ge 0$, and the condition states that $p(x) - x$ cannot be too large, or the RGD flow will push $x$ away from the minimizer $x = 0$.

In this section, we used Theorem~\ref{th:perf_align} to identify conditions on the decision-dependent distribution shift $p(\cdot)$ which ensure that the performative risk does not increase even when the dynamics follow repeated gradient descent.
For this example, the condition is that $p$ satisfies Equation~\eqref{eq:perf_align_ex} for all $x \in (0,c)$. 
More generally, the performative alignment condition allow us to specify a class of distribution shifts which behave well with respect to performative risk minimization.


\begin{table}[t!]
\centering
\caption{Voice conversion \& F0 manipulation results. MOS results are reported with 95\% confidence interval. VDE, and FFE are reported for F0 manipulation while PER, WER, EER, and MOS are reported for voice conversion. Notice, for VDE, and FFE higher is the better since F0 was flattened.}
\label{tab:conv}

\resizebox{1\columnwidth}{!}{
\begin{tabular}{c@{~} | c@{~} | c@{~}c@{~} | c@{~} | c@{~} ||  c@{~}c@{~} }
\toprule
\multirow{2}{*}{Dataset} & \multirow{2}{*}{Method} & \multicolumn{4}{c||}{Voice Conversion} & \multicolumn{2}{c}{F0 Manipulation} \\
\cmidrule{3-8}
& & PER~$\downarrow$ & WER~$\downarrow$ & EER~$\downarrow$ & MOS~$\uparrow$ & VDE~$\uparrow$ & FFE~$\uparrow$ \\
\midrule
VCTK & GT  & 17.16 & 4.32 & 3.25 & 4.11$\pm$0.29 & -- & -- \\
\midrule 
\multirow{3}{*}{LJ}
% & ASR-TTS   & 50.74  & --     & 66.08 & 32.96 & 1.46 \\
& CPC       & 22.22 	& 16.11 		& 0.46 		& 3.57$\pm$0.15 		& \bf 46.68 & \bf 48.71\\
& HuBERT    & \bf 19.09 & \bf 12.23 & \bf 0.31  & \bf 3.71$\pm$0.24 & 39.20 		& 48.42\\
& VQ-VAE    & 40.88 	& 36.96 		& 9.65 		& 2.90$\pm$0.17 		& 10.54 	& 12.08 \\
\midrule 
\multirow{3}{*}{VCTK} 
% & ASR-TTS   & 68.88  & --    & 41.77 & 13.55 & 6.48 \\
& CPC       &  23.58 		& 15.98 		& \bf 4.83  &  3.42 $\pm$ 0.24 		& \bf 25.29 & \bf 26.97 \\
& HuBERT    &  \bf 20.85 	& \bf 12.72 & 6.01  		& \bf  3.58 $\pm$ 0.28 	& 23.46 	& 26.67 \\
& VQ-VAE    & 36.88  		& 29.44 		& 11.56 		& 3.08 $\pm$ 0.34 		& 7.03  	& 7.80  \\
\bottomrule
\end{tabular}}
\vspace{-0.4cm}
\end{table}

\vspace{-0.1cm}
\section{Results}
\vspace{-0.1cm}
Our results cover
% We report results for 
three different settings: (i) speech reconstruction experiments; (ii) speaker conversion and F0 manipulation; (iii) bitrate analysis with subjective tests for speech codec evaluation. We employ two datasets: LJ~\cite{ljspeech17} single speaker dataset and VCTK~\cite{vctk} multi-speaker dataset. All datasets were resampled to a 16kHz sample rate.

% \paragraph*{Implementation Details.}
% \smallskip
\noindent{\bf Implementation Details\quad} 
\label{sec:impl}
We follow the same setup as in~\cite{lakhotia2021generative}. For CPC, we used the model from~\cite{Riviere2020}, which was trained on a ``clean'' 6k hour sub-sample of the LibriLight dataset~\cite{Kahn2020,Riviere2020}. We extract a downsampled representation from an intermediate layer with a 256-dimensional embedding and a hop size of 160 audio samples. For HuBERT we used a \textsc{Base} 12 transformer-layer model trained for two iterations~\cite{hsu2020hubert} on 960 hours of LibriSpeech corpus~\cite{Panayotov2015}. 
% This model encodes every 320 raw audio samples into a 768-dimensional vector. 
This model downsamples the raw audio $\times320$ into a sequence of 768-dimensional vectors. Similarly to~\cite{lakhotia2021generative}, activations were extracted from the sixth layer.

%CPC: We use a dictionary of 100 units, leading to a bitrate of 700bps.
%HuBERT: A dictionary of 100 units is used, leading to a bitrate of 350bps. 
%VQVE: The VQ-VAE discrete code operates at a bitrate of 800bps.
% For both CPC and HuBERT, the k-means algorithm is applied to convert continuous frames to discrete codes, using the LibriSpeech clean-100h~\cite{Panayotov2015} dataset. 
For CPC and HuBERT, the k-means algorithm is trained on LibriSpeech clean-100h~\cite{Panayotov2015} dataset to convert continuous frames to discrete codes. We quantize both learned representations with $K=100$ centroids. Leading to a bitrate of 700bps for CPC and 350bps for HuBERT.

% VQ-VAE
Similarly to CPC models, we trained the VQ-VAE content encoder model on the ``clean'' 6K hours subset from the LibriLight dataset. We use an encoder operating on the raw signal to extract discrete units, similar to~\cite{jukebox}. In addition, ``random restarts'' were performed when the mean usage of a codebook vector fell below a predetermined threshold. Finally, we used HiFiGAN (architecture and objective) as the decoder instead of a simple convolutional decoder, as it improved the overall audio quality. This model encodes the raw audio into a sequence of discrete tokens from 256 possible tokens~\cite{garbacea2019low} with a hop size of 160 raw audio samples. The VQ-VAE discrete code operates at a bitrate of 800bps. We additionally experimented with 100 discrete units for VQ-VAE, however results were the best for 256. This finding is consistent with~\cite{garbacea2019low}.

% verification model
The speaker verification network uses the architecture proposed in~\cite{heigold2016end}. It was trained on the VoxCeleb2~\cite{voxceleb2} dataset, achieving a 7.4\% Equal Error Rate (EER) for speaker verification on the test split of the VoxCeleb1~\cite{Nagrani17} dataset.

% pitch
Only a single F0 representation is considered across all evaluated models, trained on the VCTK dataset.
% The F0 is extracted from the raw audio using YAAPT~\cite{yaapt} algorithm, using a window size of 20ms and a 5ms hop. 
The F0 is extracted from the raw audio using a window size of 20ms and a 5ms hop. 
As a result, the F0 sequence is sampled at 200Hz. 
% We apply the quantization described at Sec.~\ref{sec:method}, using a pitch codebook of $K'=20$ tokens and an encoder that downsamples the pitch by $\times16$. 
The quantization described at Sec.~\ref{sec:method}, is applied using an F0 codebook of $K'=20$ tokens and an encoder that downsamples the signal by $\times16$. Hence, the discrete F0 representation is sampled at 12.5Hz, leading to a bitrate of 65bps. The final bitrate of the evaluated codecs is the sum of the pitch code bitrate with the content code bitrate.

% \paragraph*{Evaluation Metrics}
% \smallskip
\noindent{\bf Evaluation Metrics\quad} 
We consider both subjective and objective evaluation metrics. For subjective tests, we report the Mean Opinion Scores (MOS). In which human evaluators rate the naturalness of audio samples on a scale of 1--5. Each experiment, included 50 randomly selected samples rated by 30 raters. For objective evaluation, we consider: (i) Equal Error Rate~(EER) as an automatic speaker verification metric obtained using a pre-trained speaker verification network. We report EER between test utterances and enrolled speakers; (ii) Voicing Decision Error (VDE)~\cite{nakatani2008method}, which measures the portion of frames with voicing decision error; (iii) F0 Frame Error (FFE)~\cite{chu2009reducing}, measures the percentage of frames that contain a deviation of more than 20\% in pitch value or have a voicing decision error; (iv) Word Error Rate (WER) and Phoneme Error Rate (PER), proxy metrics to the intelligibility of the generated audio. We used a pre-trained ASR network~\cite{baevski2020wav2vec} on both reconstructed and converted samples to calculate both metrics. %To generate target phonemes, the g2p-en~\cite{g2pE2019} Grapheme2Phoneme module was used.

% \vspace{-0.1cm}
% \smallskip
\noindent{\bf Reconstruction \& Conversion}
% \vspace{-0.1cm}
We start by reporting the reconstruction performance. Results are summarized in Table~\ref{tab:recon}. When considering the intelligibility of the reconstructed signal HuBERT reaches the lowest PER and WER scores across all models, where both CPC and HuBERT are superior to VQ-VAE. However, when considering F0 reconstruction VQ-VAE outperforms both HuBERT and CPC by a significant margin. This results are somewhat intuitive, bearing in mind VQ-VAE objective is to fully reconstruct the input signal. In terms of subjective evaluation, all models reach similar MOS scores, with one exception of CPC on LJ. 

%Notice, since the same F0 units are used for each method, this result implies the VQ-VAE units contain some information about the F0 of the signal, enabling better reconstruction. Regarding speaker information, the CPC gets the lowest EER. 

To better evaluate the disentanglement properties of each method with respect to speaker identity and F0, we conducted an additional set of experiments aiming at speaker conversion and F0 manipulation. For voice conversion, we converted each test utterance into five random target speakers. Next, we employed a speaker verification network, which extracts \emph{d-vector} representation to evaluate speaker-converted utterances' similarity to real speaker utterances (low error-rate indicates good conversion), providing measurement to the speaker identity's disentanglement from the evaluated coding method. The error-rate is reported between converted test utterances and enrolled speakers. For the LJ speech single speaker dataset, we converted samples from the VCTK dataset to the single speaker and enrolled all VCTK speakers together with the single speaker. Results are summarized in Table~\ref{tab:conv} (left). Unlike resynthesis results, on voice conversion CPC and HuBERT outperform VQ-VAE on both LJ and VCTK datasets, indicating VQ-VAE contains more information about the speaker in the encoded units, hence producing more artifacts. Notice, this also affects WER, PER, and the overall subjective quality (MOS). 

Next, to evaluate the presence of F0 in the discrete units, we flattened the F0 units before synthesizing the signal and calculated VDE and FFE with respect to the original F0 values. F0 flattening was done by setting the speakers' mean F0 value across all voiced frames. In this experiment, we expected units that contain F0 information to be better at F0 reconstruction over disentangled units. Results are summarized in Table~\ref{tab:conv} (right). Notice VQ-VAE can still reconstruct the F0 almost at the same level as when using the original F0 as conditioning (5.2 vs 7.03, and 5.59 vs 7.8), in contrast to CPC and HuBERT.

\begin{figure}[t!]
\centering
\includegraphics[width=0.65\columnwidth, trim={50 20 70 20}]{figures/codec_2.pdf}
% \caption{MUSHRA subjective listening test results as a function of bitrate per second for various methods. Purple dots denote the baseline methods, and green dots the proposed SSL based method.} 
\caption{MUSHRA subjective quality results as a function of bitrate per second. Purple dots denote the baseline methods, and green dots the proposed SSL based method.} 
\label{fig:codec}
\vspace{-0.5cm}
\end{figure}

% \vspace{-0.1cm}
% \smallskip
\noindent{\bf Speech Codec}
Our final experiment evaluates the obtained speech units as a low bitrate speech codec. 
% Therefore, we evaluate how the performance varies as a function of the number of discrete units. Changing the number of units is equivalent to varying the bitrate of the encoded signal. 
We use a subjective MUSHRA-type listening test~\cite{series2014method} to measure the perceived quality of the proposed speech codec with regard to its bitrate constraints. In MUSHRA evaluations, listeners are presented with a labeled uncompressed signal for reference, a set of test samples to rate, a copy of the uncompressed reference, and a low-quality anchor. Listeners are asked to rate each test utterance and the copy of the uncompressed reference with respect to the labeled reference in a scale of 1-100.

The experiment is performed on the VCTK dataset~\cite{vctk}. For evaluation, we used 20 utterances from 5 speakers. The set of speakers in the test data is disjoint with those in the training data. For this experiment, HuBERT models with 50, 100, and 200 units were trained as described in Sec.~\ref{sec:impl}. For comparison, we included other speech codecs in our evaluation: Opus~\cite{valin2012definition} wideband at 9 kbps VBR, Codec2~\cite{rowe2011codec} at 2.4 kbps and LPCNet~\cite{valin2019real} operating at 1.6 kbps. The LPCNet model was trained from scratch on the VCTK dataset following the experimental setup in~\cite{valin2019real}. The VQ-VAE model employs the HiFiGAN decoder trained on the LibriLight dataset to match the amount of data reported in~\cite{garbacea2019low}. We compressed the anchor sample with Speex~\cite{valin2016speex} at 4 kbps as a low anchor. Fig.~\ref{fig:codec} depicts the results. HuBERT with 50 units reaches the best MUSHRA score while its bitrate is only 365bps, which is significantly lower than the baseline methods.
% 
\begin{comment}
\begin{figure}
\includegraphics[width=\linewidth]{figs/beyond_tss_lesion.pdf}
\caption[]{End-to-End runtime lesion study of the entire MNIST dataset and the FMA featurized music dataset. Each of DROP's contributions provides a runtime improvement.}
\label{fig:beyond_lesion}
\end{figure}
\end{comment}



\section{Conclusion}
\label{sec:conclusion}

Advanced data analytics techniques must scale to rising data volumes. 
DR techniques offer a powerful toolkit when processing these datasets, with PCA frequently outperforming popular techniques in exchange for high computational cost. 
In response, we propose DROP, a new dimensionality reduction optimizer. 
DROP combines progressive sampling, progress estimation, and online aggregation to identify high quality low dimensional bases via PCA without processing the entire dataset by balancing the runtime of downstream tasks and achieved dimensionality. 
Thus, DROP provides a first step in bridging the gap between quality and efficiency in end-to-end DR for downstream \red{analytics}. 

%We revisit canonical operators for time series dimensionality reduction and the measurement study of~\cite{keogh-study}, and show that PCA is more effective than popular alternatives in the data mining literature often by a margin of over $2\times$ on average on gold-standard time series benchmark data sets with respect to output data dimension. More surprisingly, we empirically demonstrate that a small number of samples are sufficient to accurately characterize directions of maximum variance and obtain a high-quality low-dimensional transformation.



\section{Conclusion and outlook}
\label{sec:outlook}
In the previous sections we presented a novel variational model for the reconstruction of highly subsampled dynamic MRI data where an anatomical scan (at high spatial resolution) has been acquired prior to the dynamic sequence. 
Combining radial golden angle sampling with a suitable time regularization, spatial TV regularization and with the infimal convolution of TV Bregman distances allowing to incorporate the structural information of the anatomical prior, we obtained spatially highly resolved reconstructions at a high temporal resolution. 

Summing up the results of tests on a simulated data set based on fMRI as well as on experimental small animal DCE-MRI data, we draw the following conclusions:
naturally, a simple least squares (LS) reconstruction of each individual frame could not provide meaningful results due to the severe undersampling. 
Adding a spatial TV regularization did not significantly increase the quality of the reconstructed images. 
As expected, this approach yielded piecewise constant images, but the ratio of sampled Fourier coefficients in $k$-space in comparison to the desired spatial resolution of the reconstructions was too small to obtain reasonable results. 
Remarkably, adding only Tikhonov regularization on the time derivative without any additional spatial regularization already resulted in by far more meaningful reconstructions than the LS approach, while the obtained images were still corrupted by heavy noise.
Integrating spatial TV regularization to the aforementioned model removed most of the noise and indeed provided high quality reconstructions. 
Incorporating structural information from the anatomical prior, we could then obtain very detailed results despite the severe subsampling enabling high temporal resolution.

In view of these promising results, we state some open questions and sketch additional ideas whose detailed study is left to future research.

We used the infimal convolution of TV Bregman distances to incorporate structural information from the anatomical prescan. 
Naturally, this gives rise to the question whether alternative means of incorporating structural prior information such as the concepts of weighted total variation (wTV) or directional total variation (dTV), respectively, both proposed in \cite{Ehrhardt2016}, yield significantly different results. 
In any case, it would be interesting to see how such a modified approach compares to the method proposed in this paper concerning quality of the reconstructed images, but also regarding computational complexity of solving the respective minimization problem. 

Moreover, the temporal coupling of time frames serves as a further starting point for future research. 
Here, we decided to apply Tikhonov regularization of the time derivative, however, one could also argue in favor of other concepts: 
since in the areas of application we considered in this paper the dynamic changes happen to take place in only a small portion of the entire image domain, decomposition of the dynamic sequence into a low rank part $L$ and a part $S$ which is sparse in some transform domain \cite{Tremoulheac:lowRankPlusSparsePrior,Otazo:lowRankPlusSparseMatrixDecomposition} could be an interesting alternative. 
Assuming that the dynamic changes mainly are contained in $S$, while $L$ ideally comprises the part staying constant over time, it seems particularly reasonable that the structures of the constant part of every time frame bear close resemblance to the structure of the anatomical prior image. 
Hence it would stand to reason to apply the infimal convolution of TV Bregman distances only to the low rank part leaving the sparse part untouched. 
However, against the backdrop of different dimensions of the low rank part of the dynamic sequence $L$ and the anatomical prior image $u_0$ it is not yet clear what would be the most suitable way of solving the corresponding optimization problem. 

Finally, with respect to experimental data, a more careful correction of artifacts due to different acquisition protocols between anatomical prescan and the dynamic sequence might be an interesting aspect.


\section*{Acknowledgements}

This work has been supported  ERC via Grant EU FP 7 - ERC Consolidator Grant 615216 LifeInverse, by the German Ministery for Science and Education (BMBF) through the project MED4D, by Academy of Finland (Finnish Programme for Center of Excellence in Research 2012-2017, project 250215) and Jane and Aatos Erkko Foundation. 
The authors would like to thank the Isaac Newton Institute for   Mathematical Sciences, Cambridge, for support and hospitality during  the programme Variational Methods for Imaging and Vision, where work on this paper was  undertaken, supported by EPSRC grant no EP/K032208/1.  

\bibliographystyle{abbrv}
\bibliography{fmri_ref}

\end{document}