\documentclass[12pt,a4paper]{article}

%% Language and font encodings
\usepackage[english]{babel}
\usepackage[utf8x]{inputenc}
\usepackage[T1]{fontenc}

\usepackage{geometry}
\geometry{
  left=2.5cm,
  right=2.5cm,
  top=2cm,
  bottom=4cm,
  bindingoffset=5mm
}

%% Useful packages
\usepackage{amsmath,amssymb,amsthm}
\usepackage{graphicx}
\graphicspath{{images/}}
\usepackage[colorinlistoftodos]{todonotes}
\usepackage[colorlinks=true, allcolors=blue]{hyperref}

\usepackage[numbered,framed]{matlab-prettifier}

\usepackage{todonotes}

% Algorithms
\usepackage{algorithm}
\usepackage{algpseudocode}
\renewcommand{\algorithmicrequire}{\textbf{Input:}}
\renewcommand{\algorithmicensure}{\textbf{Initialization:}}

\usepackage{color}
\newcommand{\ville}[1]{{\textcolor{red}{#1}}}
\newcommand{\eva}[1]{{\textcolor{blue}{#1}}}

% Headers
\usepackage{fancyhdr}
\pagestyle{fancy}
\fancyhf{}

%\chead[fMRI joint reconstruction]{fMRI joint reconstruction}
\chead{Dynamic MRI Reconstruction with Structural Prior}
\cfoot{\thepage}

\newcommand{\mat}[1]{\lstinline[style=Matlab-editor]{#1}}

\newcommand{\TV}{\mathrm{TV}}
\newcommand{\TVd}{\mathrm{TV}_d}
\newcommand{\C}{\mathbb{C}}
\newcommand{\R}{\mathbb{R}}
\newcommand{\Rplus}{\mathbb{R}_+}
\newcommand{\ICBTV}{\mathrm{ICB}_\TV}

\newcommand{\dx}{\,\mathrm{d}x}

\newcommand{\x}{\mathbf{x}}
\newcommand{\fbold}{\mathbf{f}}
\newcommand{\ubold}{\mathbf{u}}
\newcommand{\vbold}{\mathbf{v}}
\newcommand{\ybold}{\mathbf{y}}
\newcommand{\zbold}{\mathbf{z}}
\newcommand{\Tbold}{\mathbf{T}}

\newcommand{\pa}{p_0^\eta}
\newcommand{\qa}{q_0^\eta}

\newcommand{\diverg}{\mathrm{div}}

\newcommand{\Kcal}{\mathcal{K}}
\newcommand{\Lcal}{\mathcal{L}}
\newcommand{\Fcal}{\mathcal{F}}
\newcommand{\Scal}{\mathcal{S}}
\newcommand{\Pcal}{\mathcal{P}}

\newcommand{\Real}{\mathrm{Re}}
\newcommand{\Imag}{\mathrm{Im}}

\newcommand{\qt}{\tilde{q}}

\newcommand{\prox}{\mathrm{prox}}
\newcommand{\proj}{\mathrm{proj}}

% Spaces
\newcommand{\BVR}{\mathrm{BV}(\Omega;\mathbb{R})}
\newcommand{\BVC}{\mathrm{BV}(\Omega;\mathbb{C})}

% Environments
\newtheorem{mydef}{Definition}
\newtheorem{mythm}{Theorem}
\newtheorem{myprop}{Proposition}
\newtheorem{mylem}{Lemma}
\newtheorem{myrem}{Remark}

\numberwithin{equation}{section}

\begin{document}
\title{Dynamic MRI Reconstruction from Undersampled Data with an Anatomical Prescan}
\author{Julian Rasch\thanks{Applied Mathematics M\"unster: Institute for Analysis and Computational Mathematics, 
Westf\"alische Wilhelms-Universit\"at (WWU) M\"unster. Einsteinstr. 62, 48149 M\"unster, Germany. e-mail: julian.rasch@wwu.de}, 
Ville Kolehmainen\thanks{Department of Applied Physics, University of Eastern Finland, POB1627, 70211 Kuopio, Finland},
Riikka Nivaj\"arvi\thanks{Kuopio Biomedical Imaging Unit,
A. I. Virtanen Insititute for Molecular Sciences, 
University of Eastern Finland, POB1627, 70211 Kuopio, Finland}, 
Mikko Kettunen\footnotemark[3], \\
Olli Gr\"ohn\footnotemark[3], 
Martin Burger\footnotemark[1] \ and 
Eva-Maria Brinkmann\footnotemark[1]}

\maketitle
\abstract{
The goal of dynamic magnetic resonance imaging (dynamic MRI) is to visualize tissue properties and their local changes over time that are traceable in the MR signal.
We propose a new variational approach for the reconstruction of subsampled dynamic MR data, which combines smooth, temporal regularization with spatial total variation regularization. In particular, it furthermore uses the infimal convolution of two total variation Bregman distances to incorporate structural a-priori information from an anatomical MRI prescan into the reconstruction of the dynamic image sequence. 
The method promotes the reconstructed image sequence to have a high structural similarity to the anatomical prior, while still allowing for local intensity changes which are smooth in time.
The approach is evaluated using artificial data simulating functional magnetic resonance imaging (fMRI), and experimental dynamic contrast-enhanced magnetic resonance data from small animal imaging using radial golden angle sampling of the $k$-space. \\

\noindent {\bf Keywords: } Dynamic magnetic resonance imaging, spatio-temporal regularization, structural prior, infimal convolution of Bregman distances, total variation, golden angle subsampling
}

% !TEX root = ../arxiv.tex

Unsupervised domain adaptation (UDA) is a variant of semi-supervised learning \cite{blum1998combining}, where the available unlabelled data comes from a different distribution than the annotated dataset \cite{Ben-DavidBCP06}.
A case in point is to exploit synthetic data, where annotation is more accessible compared to the costly labelling of real-world images \cite{RichterVRK16,RosSMVL16}.
Along with some success in addressing UDA for semantic segmentation \cite{TsaiHSS0C18,VuJBCP19,0001S20,ZouYKW18}, the developed methods are growing increasingly sophisticated and often combine style transfer networks, adversarial training or network ensembles \cite{KimB20a,LiYV19,TsaiSSC19,Yang_2020_ECCV}.
This increase in model complexity impedes reproducibility, potentially slowing further progress.

In this work, we propose a UDA framework reaching state-of-the-art segmentation accuracy (measured by the Intersection-over-Union, IoU) without incurring substantial training efforts.
Toward this goal, we adopt a simple semi-supervised approach, \emph{self-training} \cite{ChenWB11,lee2013pseudo,ZouYKW18}, used in recent works only in conjunction with adversarial training or network ensembles \cite{ChoiKK19,KimB20a,Mei_2020_ECCV,Wang_2020_ECCV,0001S20,Zheng_2020_IJCV,ZhengY20}.
By contrast, we use self-training \emph{standalone}.
Compared to previous self-training methods \cite{ChenLCCCZAS20,Li_2020_ECCV,subhani2020learning,ZouYKW18,ZouYLKW19}, our approach also sidesteps the inconvenience of multiple training rounds, as they often require expert intervention between consecutive rounds.
We train our model using co-evolving pseudo labels end-to-end without such need.

\begin{figure}[t]%
    \centering
    \def\svgwidth{\linewidth}
    \input{figures/preview/bars.pdf_tex}
    \caption{\textbf{Results preview.} Unlike much recent work that combines multiple training paradigms, such as adversarial training and style transfer, our approach retains the modest single-round training complexity of self-training, yet improves the state of the art for adapting semantic segmentation by a significant margin.}
    \label{fig:preview}
\end{figure}

Our method leverages the ubiquitous \emph{data augmentation} techniques from fully supervised learning \cite{deeplabv3plus2018,ZhaoSQWJ17}: photometric jitter, flipping and multi-scale cropping.
We enforce \emph{consistency} of the semantic maps produced by the model across these image perturbations.
The following assumption formalises the key premise:

\myparagraph{Assumption 1.}
Let $f: \mathcal{I} \rightarrow \mathcal{M}$ represent a pixelwise mapping from images $\mathcal{I}$ to semantic output $\mathcal{M}$.
Denote $\rho_{\bm{\epsilon}}: \mathcal{I} \rightarrow \mathcal{I}$ a photometric image transform and, similarly, $\tau_{\bm{\epsilon}'}: \mathcal{I} \rightarrow \mathcal{I}$ a spatial similarity transformation, where $\bm{\epsilon},\bm{\epsilon}'\sim p(\cdot)$ are control variables following some pre-defined density (\eg, $p \equiv \mathcal{N}(0, 1)$).
Then, for any image $I \in \mathcal{I}$, $f$ is \emph{invariant} under $\rho_{\bm{\epsilon}}$ and \emph{equivariant} under $\tau_{\bm{\epsilon}'}$, \ie~$f(\rho_{\bm{\epsilon}}(I)) = f(I)$ and $f(\tau_{\bm{\epsilon}'}(I)) = \tau_{\bm{\epsilon}'}(f(I))$.

\smallskip
\noindent Next, we introduce a training framework using a \emph{momentum network} -- a slowly advancing copy of the original model.
The momentum network provides stable, yet recent targets for model updates, as opposed to the fixed supervision in model distillation \cite{Chen0G18,Zheng_2020_IJCV,ZhengY20}.
We also re-visit the problem of long-tail recognition in the context of generating pseudo labels for self-supervision.
In particular, we maintain an \emph{exponentially moving class prior} used to discount the confidence thresholds for those classes with few samples and increase their relative contribution to the training loss.
Our framework is simple to train, adds moderate computational overhead compared to a fully supervised setup, yet sets a new state of the art on established benchmarks (\cf \cref{fig:preview}).

\section{Application-Level Resilience \\ Modeling}
\label{sec:modeling}
This section describes our modeling methodology. %in details. 
We start with a classification of the application-level fault masking,
and then introduce a metric and investigate how to use it
to quantify the application resilience. %based on the classification. 

\subsection{General Description}
\label{sec:general_bg}
%Application-level fault masking can be manifested in different representations. 
Application-level fault masking has various representations.  
Listing~\ref{fig:general_desc} gives an example to illustrate the application-level fault masking. 
In this example, we focus on a data object, $par\_A$, which is a sparse matrix with 1$K$ non-zero data elements. We study \textit{fault masking happened in this data object}. $par\_A$ is involved in 4 statements (Lines 6, 7, 9 and 13). 
%To make this example easy to describe, we assume that the fault propagated to the application 
%is a single bit-flip in the least significant bit of . %of mantissa,
%but the application-level fault masking can happen to various faults. 
%the other faults can be tolerated by the application-level fault masking as well. 


\begin{comment}
\begin{figure}
	\begin{center}
		%\includegraphics[height=0.4\textheight,keepaspectratio]{general_desc.PNG} 
		\includegraphics[width=0.35\textheight,keepaspectratio]{general_desc.pdf} 
		\vspace{-8pt}
		\caption{An example code to show application-level fault masking}
		\label{fig:general_desc}
		\vspace{-20pt}
	\end{center}
\end{figure}
\end{comment}


In this example, the statement at Line 6 has a fault masking event:
% a data -> the data by anzheng
if a fault happened at a data element $par\_A[0].data$ of the target data object ($par\_A$), the fault can be overwritten by an assignment operation.
%The second statement has one fault masking event: the value of $c$ is determined by $b$ (not $2*a[2]$), because $b$ is significant bigger than $a[2]$.
The statement at Line 7 has no explicit fault masking event happened in the target data
%a->the by anzheng
object, but if a fault at a data element $par\_A[2].data$ occurs, the fault is propagated to $c$ by multiplication and assignment operations.
%and then indirectly masked at the statement of Line 8 by an addition operation.
At Line 9, assuming that the value of $c$ is much smaller than the value of a variable $GIANT$, 
%the operation result is determined by a variable $GIANT$ whose value is much bigger than $c$, 
the impact of the corrupted $c$ on the application outcome is ignorable.
%tolerating the fault in the second operation. 
Hence, the fault propagated from Line 7 to Line 9 can be indirectly masked.

\begin{lstlisting}[label={fig:general_desc}, caption={An example code to show application-level fault masking}]
void func (Matrix *par_A, Vector *par_b, Vector *par_x) {
	// the data object par_A has 1K data elements;
    float c=0.0;
    
    // pre-processing par_A
    par_A[0].data=sqrt(initInfo);
    c=par_A[2].data*2;
    if (c>THR) {
    	par_A[4].data=c+GIANT; // GIANT >> c
    }
    
    // using the algebraic multi-grid solve
    AMG_Sover(par_A, par_b, par_x);
}
\end{lstlisting}


At Line 9, there is also an explicit fault masking event (i.e., fault overwritten by an assignment operation) for $par\_A[4].data$ if a fault happens in $par\_A[4].data$. This fault masking is similar to the one at Line 6.
At Line 13, there is an invocation of an algebraic multi-grid solver (AMG)
%whose fault masking is quantified based on the algorithm-level analysis (see below).
that can tolerate faults in the matrix because of the algorithm-level semantics of AMG (particularly, AMG's iterative, multilevel structure~\cite{mg_ics12}).

This example reveals many interesting facts.
In essence, a program can be regarded as a combination of data objects and
operations performed on the data objects.
An operation refers to the arithmetic computation, assignment, logical and comparison operations,  
%occurred in a basic block ~\footnote{A basic block is a single entrance, single exit sequence of instructions.}
or an invocation of an algorithm implementation (e.g.,  a multigrid solver, a conjugate gradient method, or a Monte Carlo simulation).  %a conjugate gradient method,
%The operation can be a statement within a basic block or a routine implementing an algorithm.
%The operation can cause a data object to interact with other data objects by reading/writing the data objects, and at last impact the application outcome.
%(\textbf{TODO: Dong: add a sentence here to correlate the operation with LLVM IR}).
An operation may inherently come with fault masking effects, exemplified at Line 6 (fault overwritten);
An operation may propagate faults, exemplified at Line 7. 
%which affects the interaction between the target data objects and other data objects.
Different operations have different fault masking effects, and hence
impact the application outcome differently.
Different applications can have different operations because of
algorithm implementation and compiler optimization, hence the
applications can have different application-level resilience.
%application-level resilience, because 
%the applications have different program constructs, algorithm implementations, and 
%compiler optimizations, which impact operations.


Based on the above discussion, we classify application-level fault masking 
%commonly found in applications 
into three classes.

(1) \textbf{Operation-level fault masking.} At individual operations, a fault happened in a data object is masked because of the semantics of the operations. Line 6 in
%Figure->Listing by anzheng
Listing~\ref{fig:general_desc} is an example.

(2) \textbf{Fault masking due to fault propagation.} 
Some fault masking events are implicit and have to be identified beyond a single operation. %within a larger application context.
In particular, a corrupted bit in a data object is not masked in the current operation (e.g., Line 7 in 
%figure->listing by anzheng
Listing~\ref{fig:general_desc}),
but the fault is propagated to another data object and masked in another operation (e.g., Line 9).
Note that simply relying on the operation-level analysis without the fault propagation analysis is not sufficient to recognize these fault masking events.

(3) \textbf{Algorithm-level fault masking.}
Identification of some fault masking events happened in a data object must include algorithm-level information.
The identification of those events is beyond the first two classes.
Examples of such events include %some fault tolerant algorithms, such as 
the multigrid solver~\cite{mg_ics12}, some iterative methods~\cite{2-shantharam2011characterizing}, and certain sorting algorithm~\cite{prdc13:sharma}.  
Furthermore, some application domains, such as image processing and machine learning~\cite{isca07:li}, can also tolerate faults because of less 
strict requirements on the correctness of data values. 

In general, the first two classes are caused by program constructs, and the third class is caused by algorithm semantics. Due to the random nature, 
the traditional random fault injection may omit some fault masking events, or capture them multiple times.
%Hence, the traditional fault injection can be inaccurate.
%and have to rely on massive number of tests to generate sufficient coverage.
Relying on analytical modeling, we can avoid or control the randomness of the fault injection, hence greatly improve resilience evaluation. %accuracy and repeatability. 

Our resilience modeling is analytical, and %on the application-level resilience 
relies on the quantification of the above application-level fault masking events happened on data objects.
We create a new metric to quantify the application-level resilience at \textit{data objects}, and introduce methods to measure the metric based on the above classification of fault masking events.
%Why do we need a metric? How to identify fault masking? Give a general introduction here.
%(\textbf{Dong: add an example to show why random fault masking does not work??. Add a figure here?})

\vspace{-10pt}
\subsection{aDVF: An Application-Level Resilience \\ Metric}
\label{sec:metric}
To quantify the resilience of a data object due to fault masking events, we could simply count the number of fault masking events
that happen to the target data object. 
%However, the number of fault masking events is related to the code size and
%the access intensity of the target data objects. 
However, a direct resilience comparison between data objects in terms of the number of fault masking events cannot provide meaningful quantification of the resilience of data objects. 
%\textcolor{green}{Because the fault masking ratio varies across fault masking events on different data objects.} 
For example, a data object %referenced in most of basic blocks 
may be involved in a lot of fault masking events, 
but this does not necessarily mean this data object is more resilient to faults
than other data objects with fewer fault masking events, because the fault masking events of this data object 
%can spread throughout the application execution and only happen sporadically given a time frame.
%that happens sporadically can be accumulated throughout the program.
can come from a few repeated operations, and the number of fault masking events is accumulated throughout application execution;
This data object could be not resilient, if most of other operations for this data object do not have fault masking. 
%quantify->quantifying by anzheng
Hence, the key to quantifying the resilience of a data object is
to quantify \textit{how often} fault masking happens to the data object.
%the operations that happen to the data object have fault masking.
We introduce a new metric, \textit{aDVF} (i.e., the application-level Data Vulnerability Factor), 
to quantify application-inherent resilience due to fault masking in data objects. aDVF is defined as follows. %For an operation, 
%if an fault following a specific pattern (represented with $ep$) 
%happens at the target data object before this operation, 
%Before an operation happens, 

For an operation performed on a data element of a data object, we reason that if a fault happens at the data element in this operation, 
the application outcome could or could not remain correct in terms of the outcome value and application semantics. %after the operation.
%Here, the operation is defined at the application statement level.
%It can be a memory reference (load or store), an arithmetic operation, 
If the fault does not cause an incorrect application outcome,
then a fault masking event happens to the target data object.
A single operation can operate on one or more data elements of the target data object. 
For a specific operation, aDVF of the target data object is defined as the total number of fault masking events divided by the total number of data elements of the target data object operated on by the operation.

For example, an assignment operation $a[1] = w$ 
%has three operations happened to a data object, the array $a$. 
%The three operations are two reads and one addition.
happens to a data object, the array $a$.
This operation involves one data element ($a[1]$) of the data object $a$.
%An fault can happen at any operand involved in the operation. 
%Use the example $a[1]+C$ again. 
We calculate aDVF for the target data object $a$ in this operation as follows.
%Assuming that a fault happens to $a[1]$ and $C$ is always significantly larger than the erroneous $a[1]$, we reason that 
If a fault happens to $a[1]$, we deduce that 
the erroneous $a[1]$ does not impact application correctness and the fault in $a[1]$ is always masked. Hence, the number of fault masking events for
the target data object $a$ in this operation is 1. Also, the total number of data elements involved in the operation is 1.
Hence, the aDVF value for the target data object in this addition operation is $1/1=1$.

%The normalization binds the aDVF value to [0, 1], 
%such that we can establish analysis semantics consistent with that of the algorithm-level analysis (see below).
%for the convenience of integration with the algorithm-level analysis (see below). %also the effect the data object size
Based on the above discussion, the definition of aDVF for a data object $X$ in an operation ($aDVF^{X}_{op}$)
is formulated in Equation~\ref{eq:dvf}, where 
$x_i$ is a data element of the target data object $X$, and $m$ is the number of data elements operated on by the operation;
%changed a lot by anzheng
$f$ is a function to count fault masking events happened on a data element. %the i-$th$ data element $x_i$.
\begin{comment}
$f(i)$ %$f(i,ep)$ 
is a function to count fault masking events
happened to a data element $i$ of the target data object operated on by the operation. There are $m$ data elements of the target data object operated on by the operation.
\end{comment}
\vspace{-1pt}
\begin{equation} 
\label{eq:dvf}
%\scriptsize
\footnotesize
	aDVF^{X}_{op} = \sum_{i=0}^{m-1}f(x_i)/m
\end{equation}
\vspace{-5pt}
%We calculate $aDVF_{op}$ for each operation performed on the target data object. %or other interacting data objects,
%aDVF of the target data object for a code region is the arithmetic mean of aDVF of all related operations in the region.
%We calculate $aDVF$ for a code region as follows

The calculation of aDVF for a code segment is similar to the above for an operation, except that
$m$ is the total number of $x$
%add dynamic here by anzheng
involved in all %\textsl{dynamic} 
operations of the code segment. 
To further explain it, we use as an example
a code segment from LU benchmark in SNU\_NPB benchmark suite 1.0.3 (a C-based implementation of the Fortran-based NPB) shown in 
Listing~\ref{fig:advf_example}.
%all Figure -> Listing by anzheng

%\begin{minipage}{\linewidth}
\begin{lstlisting}[label={fig:advf_example}, caption={A code segment from LU.}]
void l2norm(int ldx, int ldy, int ldz, int nx0, \
	int ny0, int nz0, int ist, int iend, int jst, \
    int jend, double v[][ldy/2*2+1][ldx/2*2+1][5], \
    double sum[5])
{
	int i, j, k, m;
    for (m=0;m<5;m++) //the first loop
    	sum[m]=0.0;  //Statement A
    
    for (k=1;k<nz0-1;k++){  //the second loop
    	for (j=jst;j<jend;j++){
        	for (i=ist;i<iend;i++){
            	for (m=0;m<5,m++){
            		sum[m]=sum[m]+v[k][j][i][m]  \
                    	*v[k][j][i][m]; //Statement B
                }
            }
        }
    }
    
    for (m=0;m<5;m++){  //the third loop
    	sum[m]=sqrt(sum[m]/((nx0-2)*  \
        	(ny0-2)*(nz0-2))); //Statement C
    }
} 
\end{lstlisting}
%\end{minipage}


\textbf{An example from LU.} We calculate aDVF for the array $sum[]$. 
%the statement->Statement by Anzheng
Statement $A$ has an assignment operation involving one data element ($sum[m]$) and one fault masking event (i.e., if a fault happens to $sum[m]$, the fault is overwritten by the assignment). Considering that there are five iterations in the first loop ($iter_{num1} = 5$), there are 5 fault masking events happened in 5 data elements of $sum[]$

Statement B has two operations related to $sum[]$ (i.e., an assignment and an addition). The assignment operation involves one data element ($sum[m]$) and one fault masking; the addition operation involves one data element ($sum[m]$) and one potential fault masking (i.e., certain corruptions in $sum[m]$ can be ignored, if ($v[k][j][i][m]*v[k][j][i][m]$) is significantly larger than $sum[m]$). This potential fault masking is counted as $r^\prime$ ($0 \leq r^\prime \leq 1$), depending on where a corruption happens in $sum[m]$ and fault propagation analysis result (see Sections~\ref{sec:statement_analysis} and~\ref{sec:impl} for further discussion). 
Considering the loop structure, there are ($(1+r^\prime) * iter_{num2}$) fault masking events happened in ($2 * iter_{num2}$) elements of $sum[]$, where ``1'' and ``$r^\prime$'' come from the assignment and addition operations respectively. $iter_{num2}$ is the number of iterations in the second loop, which is equal to ($(nz0-2)*(jend-jst)*(iend-jst)*5$).

Statement C has two operations 
%with->to by anzheng
related to $sum[]$ (i.e., an assignment and a division), but only the assignment operation has fault masking.
Considering that there are 5 iterations in the third loop ($iter_{num3} = 5$), there are 5 fault masking events happened on 5 data elements of the target data object in the third loop. In 
%add the  by anzheng
summary, the aDVF calculation for $sum[]$ is shown in Figure~\ref{fig:advf_cal}.  

\begin{comment}
\begin{figure}[t]
	\centering
	\vspace{-10pt}
	\includegraphics[height=0.45\textheight, width=0.48\textwidth]{advf_example.pdf}
	\vspace{-15pt}
	\caption{A code segment from LU. }
	\label{fig:advf_example}
	\vspace{-10pt}
\end{figure}
\end{comment}

\begin{figure}
	\centering
	\includegraphics[height=0.15\textheight, width=0.48\textwidth]{advf_lu_calculation.pdf}
	\vspace{-8pt}
	\caption{Calculating aDVF for a target data object, the array \textit{sum}[] in a code segment from LU. In the figure, $iter_{num1}=5, iter_{num3}=5$ and $iter_{num2} = (nz0-2)*(jend-jst)*(iend-ist)*5.$}
	\label{fig:advf_cal}
	\vspace{-15pt}
\end{figure}

To calculate aDVF for a data object, we must rely on effective identification and counting of fault masking events (i.e., the function $f$).
In Sections~\ref{sec:statement_analysis},~\ref{sec:fault_propagation_analysis} and ~\ref{sec:algo_analysis}, 
we introduce a series of counting methods based on the classification of fault masking events. %(see Section~\ref{sec:general_bg}). 

\subsection{Operation-Level Analysis}
\label{sec:statement_analysis}
To identify fault masking events at the operation level, we analyze 
all possible operations. %performed on any data object.
In particular, we analyze 
architecture-independent, LLVM instructions %code representation
%(see Section~\ref{sec:impl} for implementation details),
and characterize them based on the instruction result sensitivity to corrupted operands. We classify the operation-level fault masking as follows. 

%\begin{itemize}
(1) \textbf{Value overwriting}.  
An operation writes a new value into the target data object, 
and the fault in the target data object is masked. 
For example, the store operation overwrites the fault in the store destination. 
%the add operation overwrites the fault %pre-existing fault in the result variable. 
We also include \textit{trunc} and bit-shifting operations into this category, because the fault can be truncated or shifted away in those operations.

(2) \textbf{Logical and comparison operations}.
If a fault in the target data object does not
change the correctness of logical and comparison operations, the fault is masked.  
Examples of such operations 
include logical \textit{AND} and the predicate expression in a \textit{switch} statement.
%We also attribute bit-shifting operations to this category, because
%these operations are often involved in the logical and comparison operations.    

(3) \textbf{Value shadowing}.
If the corrupted data value in an operand of an operation 
is shadowed by other correct operands involved in the operation,
then the corrupted data has an 
%add an by anzheng
ignorable impact on the correctness of the operation.
%and the incorrect data value is shadowed by other correct operands involved in the operation.
The addition operation at Line 9 in Figure~\ref{fig:general_desc} is such an example. 
We can find many other examples, such as arithmetic multiplication. %and square root.
%, and trunc operation for type casting.
The effectiveness of value shadowing is coupled with the application semantics.  
An operation of $1000+0.0012$ can be treated as equal to $1000+0.0011$ without impacting the execution correctness of application,
while such tiny difference in the two data values may be intolerable in a different application. We will discuss how to identify value shadowing in details in Section~\ref{sec:impl}.
%\end{itemize}
%(\textbf{Dong: add more to describe how we choose a threshold to determine value shadowing.})

Since we focus on the \textit{application}-level resilience modeling,
we do not consider those LLVM instructions that do not have
directly corresponding operations at the application statement level for fault masking analysis. 
Examples of those instructions 
include \textit{getelementptr} (getting the address of a sub-element of an aggregate data structure)
and \textit{phi} (implementing the $\phi$ node in the SSA graph~\cite{llvm_lrm}).

The effectiveness of the operation-level fault masking heavily relies on the fault pattern.
The fault pattern is defined by how fault bits are distributed within
a faulty data element (e.g., single-bit vs. spatial multiple-bit, least significant bit vs. most significant bit, mantissa vs. exponent).
To account for the effects of various fault patterns, an ideal method to count fault masking events %on a particular platform
would be to collect fault patterns in a production environment during a sufficiently long time period, and then use the realistic fault patterns to guide fault masking analysis. 
However, this method is not always practical. 
In the practice of our resilience modeling, we enumerate possible fault patterns for a given operation, %and a data element of the target data object, 
and derive the existence of fault masking for each fault pattern.
Suppose there are $n$ fault patterns, and $m$ ($0 \leq m \leq n$) of which have fault masking happened.
Then, the number of fault masking events is calculated as $m$/$n$,
which is a statistical quantification of possible fault masking.
Using this statistical quantification means that the number of fault masking events can be non-integer. 
%from the probability perspective.
We employ the above enumeration analysis to model fault masking for single-bit faults in our evaluation section, but the method of the enumeration analysis can be applied to analyze all fault patterns.
\vspace{-10pt}

\subsection{Fault Propagation Analysis}
\label{sec:fault_propagation_analysis}
At an operation performed on the target data object, 
if a fault happened in the target data object cannot be masked at the current operation, 
then we use the fault propagation analysis to track whether the corrupted data
is propagated to other data object(s) and the faults (including the original one and the new ones propagated to other data object(s)) 
are masked in the successor operations.
If all of the faults are masked, then we claim that the original fault happened in the target data object is masked.   

For the fault propagation analysis, a big challenge is to 
track all contaminated data which can quickly increase as the fault propagates. 
\begin{comment}
%handle fault explosion.
We use an example shown in Figure~\ref{fig:fault_propagation_code}
as a running example for the fault propagation analysis. %in this section.
This example is from the CG benchmark in SNU\_NPB.
%Figure~\ref{fig:fault_propagation} shows an example of the fault propagation analysis for NPB CG benchmark.
The target data object in this example is the array $r$, and we calculate aDVF for
the assignment operation in the statement $A$. 
If $r[j]$ has a fault at the statement $A$, the fault cannot be masked.
Instead, within the successor four statements (B-E), the fault quickly propagates to four data objects ($rho, d, alpha$ and $z[]$).
%The number of contaminated data objects grows at least linearly with the number of statements. 
\end{comment}
Tracking a large number of contaminated data objects largely increases
analysis time and memory usage. %code complexity. 
%The symbols in the blocks of the figure are a notation for fault masking analysis.
%In particular, the symbol $st_{x}:op_{y}:d_{z}$ refers to an fault occurred in an operation $y$ 
%of an statement $x$, and the fault happens in the data $z$ before the operation.
%For the example of the figure,  we analyze the fault masking of xxx data.
%Within xxx statements, we have xxx different data tainted by the corrupted data at the statement xxx.
To handle the above fault propagation problem, we avoid tracking fault propagation along a long chain of operations
to accelerate the analysis.
%This can address the fault explosion problem, and accelerate the analysis. 
%We introduce two optimization techniques to avoid long tracking.
We introduce an optimization technique to avoid long tracking.

\begin{comment}
\textbf{Optimization 1: leveraging intermediate states.}
%Avoiding tracking fault propagation along a long chain of operations is an effective way to address the fault explosion problem. 
During the fault propagation analysis, if we know the valid data values (i.e., the intermediate state) of some data objects, 
then we can compare the valid intermediate states with the data values in the fault propagation analysis.
A mismatch between the two indicates that the faults occurred in previous operations
will not be masked in the future.
%the pending data states do not have fault masking; otherwise, there is fault masking.
Hence, the valid intermediate state works as ``analysis shortcut'' that allows us to deduce fault masking without tracking fault propagation to the end of the application execution. 

\begin{figure}[h]
	\begin{center}
		\includegraphics[height=0.3\textheight,keepaspectratio]{error_propagation_code.PNG}
		\caption{A code excerpt from the SNU\_NPB CG benchmark to show the fault propagation from the statement $A$ at an iteration $j$.  The target data object is the array $r[]$. A fault in $r[j]$ is propagated to four data objects ($rho$, $d$, $alpha$ and $z$[]) in the statements B-E.}
		\label{fig:fault_propagation_code}
		\vspace{-20pt}
	\end{center}
\end{figure}
\begin{figure}[h]
	\begin{center}
		\includegraphics[height=0.3\textheight,keepaspectratio]{error_propagation_ddg.pdf}
		\caption{The data dependency graph to show the fault propagation for Figure~\ref{fig:fault_propagation_code}.}
		\label{fig:fault_propagation_ddg}
	\end{center}
	\vspace{-20pt}
\end{figure}

To further explain the idea, we use the example in Figure~\ref{fig:fault_propagation_code} again.
We use a dynamic dependency graph (Figure~\ref{fig:fault_propagation_ddg}) of the example to explain the idea.
%shows an example for using the intermediate state. %helps us immediately identify fault masking at the statement C.
%without reaching the statement xxx.
The dynamic dependency graph (DDG) captures the dynamic dependencies among data objects in the course of program execution~\cite{prdc05:Pattabiraman, tdsc11:pattabiraman}. 
%the values produced  in the course of program execution. 
A node in DDG represents a value of a data object produced in the program, and the node is associated with a dynamic operation that produced the value.
An edge in DDG represents an operation.
The source node of the outgoing edge corresponds to an operation operand,
and the destination node corresponds to the value produced by the operation.
The same memory location can be mapped onto
multiple nodes in DDG (e.g., the data object $rho$ in Figure~\ref{fig:fault_propagation_ddg}), just as a memory
location can have multiple value instances during the execution.
%DDG can be generated based on LLVM instrumentation.
%Function calls and returns are represented in the DDG. 
%The reason why DDG  --> (1) fanout; (2) fault propagation;

Tracking the edges in Figure~\ref{fig:fault_propagation_ddg},  we can know how a fault is propagated when $r[j]$ has the fault.
Assume that we know the valid value range\footnote{A \textit{valid} value always results in acceptable application outcomes. The \textit{valid} used in here and in the rest of the paper is defined in terms of application outcomes.} of the data object $alpha$ at the statement $D$.
If a fault that happens in $r[j]$ at the statement $A$ is propagated to $alpha$ and we find that the faulty $alpha$ is not in the valid value range,
then we can deduce that the original fault will not be masked, and 
we avoid the fault tracking after the statement $D$. 

To collect valid intermediate states to accelerate the fault propagation analysis,  
%we instrument the memory references to the target data object, and 
we record the values of some variables (e.g., the values of $alpha$ at the statement $D$) with fault-free execution. %after some operations. 
%In particular, the values of critical data objects are recorded immediately after an operation,
%if the operation has more than x\% of successor operations involving critical data objects 
%or the target data object in the next $y$ operations ($x=40, y=50$ in our tests). 
%involving critical data objects or the target data object in the next $y$ operations ($x=40, y=50$ in our tests). 
We record valid values of a variable if a value of the variable in DDG has at least 10 predecessor nodes.
%In Figure~\ref{fig:fault_propagation_ddg}, the node $alpha$ has 10 predecessor nodes. 
%We use such method to record variable values, because having a large number of predecessor nodes indicates that the recorded value is potentially helpful to address many fault prorogation analysis.
Recording those values is useful, because it is potentially helpful to resolve many pending fault propagation analysis.
%indicates how many nodes are directly impacted by an fault in that no
\end{comment}

\begin{comment}
We choose those operations to output the values, because 
those operations maximize the fanout of data corruption, 
and hence have big potential to result in the above fault propagation problem.
%Those operations heavily involve the target data object, and hence maximum the fanout of the orginia data corruption. 
%Those operations have big potential to lead the fault explosion problem. 
(\textbf{Dong: explain more what is fanout}).

\textbf{Fanout: the fanout of a node is the set of all immediate successors of the node in DDG. The fanout of a node indicates how many nodes are directly impacted by an fault in that node.}
\end{comment}


\textbf{Optimization: bounding propagation path.} 
%Another method to avoid long tracking of fault propagation is to bound the fault propagation path.
We take a sample of the whole fault propagation path.
In particular, we only track the first $k$ operations. 
%sample the whole fault propagation path with the first $k$ operations.
%If the original fault happened in the target data object 
If the original fault and the new faults propagated to other data object(s)
cannot be masked within the first $k$ operations, then we conclude that 
%the original fault 
all of the faults will not be masked after the $k$ operations.
%in the following operations. 

This method, as an analysis approximation, could introduce analysis inaccuracy because of the sampling nature of the method. 
%The effectiveness of this optimization varies from one operation to another.
However, for a fault that propagates to a large amount of data objects, 
%through successor operations, 
bounding the fault propagation path does not cause inaccurate analysis, because given a large amount of corrupted data, it is highly unlikely that all faults are masked, and
%In Bounding the propagation path and immediately 
making a conclusion of no fault masking is correct in most cases.
In the evaluation section, we explore the sensitivity of analysis correctness to the length of the fault propagation path (i.e., $k$). We find that setting the propagation path to 10 is
good to achieve accurate resilience modeling in most cases (87.5\% of all cases). Setting it to 50 is good for all cases.

\vspace{-10pt}
\subsection{Algorithm-Level Analysis}
\label{sec:algo_analysis}
Identifying the algorithm-level fault masking demands domain and algorithm knowledge.  
In our resilience modeling, we want to minimize the usage of domain and algorithm knowledge, such that
the modeling methodology can be general across different domains.

%Recognizing algorithm-level fault masking is challenging, because the domain knowledge is often demanded 
%to determine if an application outcome with the data corruption occurred in an operation is valid.
%The requirement of the domain knowledge imposes a challenge to make the tool portable and generalizable across different domains.
%From the perspective of a programmer who improves program reliability and fault tolerance mechanisms, we want to minimize
%the introduction of domain knowledge.

We use the following strategy to identify the algorithm-level fault masking (see the next paragraph). 
Furthermore, the user can optionally provide a threshold to indicate a satisfiable solution quality.
For example, for an iterative solver such as conjugate gradient and successive over relaxation,
this threshold can be the threshold that governs the convergence of the algorithms.
For the support vector machine algorithm (an artificial intelligence algorithm), this threshold can be a percentage (e.g., 5\%)
of result difference after the fault corruption.
Working hand-in-hand, the strategy (see the next paragraph) and user-defined threshold treat the algorithm as a black box without
%on->of by anzheng
requiring detailed knowledge of the algorithm internal mechanisms and semantics. 
We explain the strategy as follows.

\textbf{A practical strategy for algorithm-level analysis: deterministic fault injection.}
The traditional random fault injection treats the program as a black-box. 
Hence, using the traditional random fault injection could be an effective tool to identify the algorithm-level fault masking.
However, to avoid the limitation of the traditional random fault injection (i.e., randomness), %and high cost),
we use the operation-level analysis and fault propagation analysis to guide fault injection, 
without blindly enforcing fault injection as the traditional method.
%In particular, when determining fault masking for an operation $x$ by the fault propagation analysis and reaching the boundary of the fault propagation analysis,
In particular, when we cannot determine whether a fault masking can happen in the target data object for an operation $op$ because of fault propagation,  we track fault propagation until 
reaching the boundary of the fault propagation analysis.
%(see Optimization 2 in Section~\ref{sec:fault_propagation_analysis}),
If we still cannot determine fault masking at the boundary, then  
we inject a fault into the target data object in $op$, %right after the boundary, 
and then run the application to completion. 
%If the algorithm result is the same as the one without fault injection or does not go beyond the user-provided threshold, 
If the application result is different from the fault-free result, %without fault injection,
but does not go beyond the user-defined threshold, we claim that the algorithm-level fault masking takes effect. %for the operation $x$.

\begin{comment}
As described above, our fault injection has a deterministic plan on when and where to inject faults. Also, the operation-level and fault propagation analysis is complementary to our fault injection. Hence we avoid fault injection if possible.c
\end{comment}

\textbf{Discussion: coupling between fault propagation and algorithm level analysis.}
The fault propagation analysis and algorithm-level analysis are tightly coupled.
If we reach the boundary of the fault propagation analysis and cannot determine fault masking, we use the algorithm-level analysis. %(i.e., the guided fault injection).
%the fault masking attributed to the algorithm-level fault masking may actually come from the fault propagation-based fault masking. 
However, by doing this, 
some of the fault masking events due to the fault propagation and operation-level fault masking after the boundary may be accounted as algorithm-level fault masking.
Although this mis-counting will not impact the correctness of aDVF value, it would overestimate the algorithm-level fault masking.
%we are at a risk of losing accuracy for counting those fault masking events at the level of fault propagation.
%A correct application execution after the deterministic fault injection 
%may be because of fault masking during fault propagation, not because of the algorithm-level fault masking.
\begin{comment}
However, the application execution deemed to be correct at the end of the execution 
may be a result of both the algorithm-level fault masking and the fault propagation-based fault masking.
We cannot count those fault propagation-based fault masking, because we set an upper bound on the number of operations 
for the fault propagation analysis.
Simply speaking, we may lose accuracy trading for simplification of the fault propagation analysis.
\end{comment}

The fundamental reason for the above overestimation is that we bound the boundary of the fault propagation analysis.
However, our study (Section~\ref{sec:eval_sen}) reveals that we can have very good modeling
%in->on by anzheng
accuracy on our count of the algorithm level fault masking,
even if we set the boundary of the fault propagation analysis.
The reason is as follows. %because of the fault explosion:
After the boundary of the fault propagation analysis, the fault is widely propagated, and
the chance to mask all propagated faults by the operation-level fault masking is extremely low.
In fact, in our tests, we found that even if we use a longer fault propagation path to identify fault masking, we are not able to find more fault masking based on the fault propagation analysis.
Hence, as long as the threshold is sufficiently large (e.g., 10), 
we do not overestimate the algorithm-level fault masking. %for identifying fault masking. 
\vspace{-10pt}
\section{Numerical implementation and solution}
In this section, we explain how to implement and solve the minimization problem \eqref{eq:dynamic_recon} numerically which, depending on the amount of time steps $T$, can be challenging. 
We derive a general primal-dual algorithm for its solution, before we line out some strategies to reduce computational costs and speed up the implementation at the end of this section. 

\subsection{Gradients and sampling operators}
\label{subsec:implementation of operators}

In order to use the discrete total variation already defined in Section \ref{subsubsec:TV}, we need a discrete gradient operator that maps an image $u \in \C^N$ to its gradient $\nabla u \in \C^{N \times 2}$. 
Following \cite{ChambollePock}, we implement the gradient by standard forward differences. Moreover, we  will use its discrete adjoint, the negative divergence $-\diverg$, defined by the identity $\langle \nabla u, w \rangle_{\C^{N \times 2}} = - \langle u, \diverg(w) \rangle_{\C^N}$.
The inner product on the gradient space $\C^{N\times 2}$ is defined in a straightforward way as 
\begin{align*}
	\langle v,w \rangle_{\C^{N \times 2}} = \Real(v_1^* w_1) + \Real(v_2^* w_2),
\end{align*}
%
for $v,w \in \C^{N \times 2}$.
For $v \in \C^{N \times 2}$, the (isotropic) 1-norm is defined by 
\begin{align*}
	\|v \|_1 := \sum_{i=1}^N \sqrt{|(v_i)_1|^2 + |(v_i)_2|^2},
\end{align*}
and accordingly the dual $\infty$-norm for $w \in \C^{N \times 2}$ is given by 
\begin{align*}
	\| w \|_\infty := \max_{i=1,\cdots,N} |w_i| = \max_{i=1,\cdots,N} ~ \sqrt{|(w_i)_1|^2 + |(w_i)_2|^2}.
\end{align*}

The sampling operators $\Kcal_t \colon \C^N \to \C^{M_t}$ (and analogously for $\Kcal_0 \colon \C^{N_0} \to \C^{M_0}$) we consider are either a standard fast Fourier transform (FFT) on a Cartesian grid, followed by a projection onto the sampled frequencies, or a non-uniform fast Fourier transform (NUFFT), in case the sampled frequencies are not located on a Cartesian grid \cite{Fessler:NUFFT}. \\

\noindent {\it Fourier transform on a Cartesian grid - the simulated data case}\\
\noindent
In the numerical study on artificial data we use a simple version of the Fourier transform and sampling operator (the same can also be found in e.g. \cite{Ehrhardt2016,Rasch2017}). 
We discretize the image domain on the unit square using an (equi-spaced) Cartesian grid with $N_1 \times N_2$ pixels such that the discrete grid points are given by 
\begin{align*}
 \Omega_N = \left\{ \left(\frac{n_1}{N_1-1}, \frac{n_2}{N_2-1} \right) ~\Big|~ n_1 = 0, \dots, N_1-1, \; n_2 = 0, \dots N_2-1 \right\}.
\end{align*}
We proceed analogously with the $k$-space, i.e. the location of the $(m_1,m_2)$-th Fourier coefficient is given by $(m_1/(N_1-1), m_2/(N_2-1))$. Then, we arrive at the following formula for the (standard) Fourier transform $\Fcal$ applied to $u \in \C^{N_1 \times N_2}$:
\begin{align*}
 (\Fcal u)_{m_1,m_2} = \frac{1}{N_1 N_2} \sum_{n_1=0}^{N_1-1} \sum_{n_2=0}^{N_2-1} u_{n_1,n_2} e^{-2\pi i \left(\frac{n_1 m_1}{N_1} + \frac{n_2 m_2}{N_2} \right)},
\end{align*}
where $ m_1 = 0, \dots, N_1-1, m_2 = 0, \dots, N_2 -1$.
For simplicity, we use a vectorized version such that $\Fcal \colon \C^N \to \C^N$ with $N = N_1 \cdot N_2$.
We then employ a simple sampling operator $\Scal_t \colon \C^N \to \C^{M_t}$ which discards all Fourier frequencies which are not located on the desired sampling geometry at time $t$ (i.e. the chosen spokes). 
More precisely, following \cite{Ehrhardt2016}, if we let $\Pcal_t \colon \{1,\dots,M_t \} \to \{1,\dots,N\}$ be an injective mapping which chooses $M_t$ Fourier coefficients from the $N$ coefficients available, we can define the sampling operator $\Scal$ applied to $f \in \C^N$ as
\begin{align*}
 \Scal_t \colon \C^N \to \C^{M_t}, \quad (\Scal_t f)_k = f_{\Pcal_t(k)}.
\end{align*}
The full forward operator $\Kcal_t$ can hence be expressed as 
\begin{align}\label{eq:forward_op_art}
 \Kcal_t \colon \C^N \xrightarrow{\Fcal} \C^N \xrightarrow{\Scal_t} \C^{M_t}.
\end{align}
The corresponding adjoint operator of $\Kcal_t$ is given by  
\begin{align*}
 \Kcal_t^* \colon \C^{M_t} \xrightarrow{\Scal_t^*} \C^N \xrightarrow{\Fcal^{-1}} \C^N,
\end{align*}
where $\Fcal^{-1}$ denotes the standard inverse Fourier transform and $\Scal_t^*$ `fills' the missing frequencies with zeros, i.e. 
\begin{align*}
 (\Scal_t^*z)_l = \sum_{k=1}^{M_t} z_k \delta_{l,\Pcal_t(k)}, \qquad \text{for } l = 1, \dots, N.
\end{align*}
For the prior $u_0$, we choose a full Cartesian sampling, which corresponds to $\Pcal_0$ being the identity. For the subsequent dynamic scan, we set up $\Pcal_t$ such that it chooses the frequencies located on (discrete) spokes through the center of the $k$-space.     
It is important to notice that this implies that the locations of the (discretized) spokes are still located on a Cartesian grid, which allows to employ a standard fast Fourier transform (FFT) followed by the above projection onto the desired frequencies. 
This is not the case for the operators we use for real data. \\

\noindent {\it Non-uniform Fourier transform - the real data case}\\ 
\noindent 
In contrast to the above (simplified) setup for artificial data, in many real world application the measured $k$-space frequencies $\xi_m$ in \eqref{eq:fourier_transform} are {\it not} located on a Cartesian grid. 
While this is not a problem with respect to the formula itself, it however excludes the possibility to employ a fast Fourier transform, 
%numerically
which usually reduces the computational costs of an $N$-point Fourier transform from an order of $O(N^2)$ to $O(N \log N)$.
To get to a similar order of convergence also for non-Cartesian samplings, it is necessary to employ the concept of non-uniform fast Fourier transforms (NUFFT) \cite{Fessler:NUFFT,Fessler:code,Matej2004,Nguyen:1999,Strohmer2000}. 
We only give a quick intuition here and for further information we refer the reader to the literature listed above. 
The main idea is to use a (weighted) and oversampled standard Cartesian $K$-point FFT $\Fcal$, $K \geq N$ followed by an interpolation $\Scal$ in $k$-space onto the desired frequencies $\xi_m$. 
Note that the oversampling takes place in $k$-space.
The operator $\Kcal_t$ for time $t$ can hence again be expressed as a concatenation of a $K$-point FFT and a sampling operator 
\textbf{\begin{align*}
 \Kcal_t \colon \C^N \xrightarrow{\Fcal} \C^N \xrightarrow{\Scal_t} \C^{M_t}.
\end{align*}}
For our numerical experiments with the experimental DCE-MRI data, the sampling operator $\Scal_t$ and its adjoint were taken from the NUFFT package \cite{Fessler:code}.

\subsection{Numerical solution}
%
Due to the nondifferentiablity and the involved operators we apply a primal-dual method \cite{ChambollePock} to solve the minimization problem \eqref{eq:dynamic_recon}. 
We first line out how to solve the (simple) TV-regularized problem for the prior (\ref{tvu0}) and then extend the approach to the dynamic problem. 
Interestingly, the problem for the prior already provides all the ingredients needed for the numerical solution of the dynamic problem, which can then be done in a very straightforward way. 
We consider the problem 
\begin{equation} \label{tvu0}
	\min_{u_0} ~ \frac{\alpha_0}{2} \| \Kcal_0 u_0 - f_0 \|_{\C^{M_0}}^2 + \| \nabla u_0 \|_1, 
\end{equation}
with $u_0 \in \C^{N_0}$.
Dualizing both terms leads to its primal-dual formulation 
\begin{align}\label{eq:tv_pd}
	\min_{u_0} \max_{y_1,y_2} ~ \langle y_1, \Kcal_0 u_0 - f_0 \rangle_{C^{M_0}} - \frac{1}{2 \alpha_0} \|y_1 \|_{\C^{M_0}}^2 + \langle y_2, \nabla u_0 \rangle_{\C^{N_0 \times 2}} + \chi_{C}(y_2),
\end{align}
where $y_1 \in \C^{M_0}$ and $\chi_{C}$ denotes the characteristic function of the set  
\begin{align*}
	C := \{ y \in \C^{N_0 \times 2} ~|~ \|y \|_\infty \leq 1 \}.
\end{align*}
% 
The primal-dual algorithm in \cite{ChambollePock} now essentially consists in performing a proximal gradient descent on the primal variable $u_0$ and a proximal gradient ascent on the dual variables $y_1$ and $y_2$, where the gradients are taken with respect to the linear part, the proximum with respect to the nonlinear part. 
We hence need to compute the proximal operators for the nonlinear parts in \eqref{eq:tv_pd} to obtain the update steps for $u_0$ and $y_1,y_2$. 
It is easy to see that the proximal operator for $\phi(y_1) = \frac{1}{2 \alpha} \|y_1\|_{\C^{M_0}}^2$ is given by 
\begin{align}\label{eq:prox_dual_l2}
	y_1 = \prox_{\sigma \phi} (r) \Leftrightarrow y_1 = \frac{\alpha r}{\alpha + \sigma}.
\end{align}
The proximal operators for the update of $y_2$ are given by a simple projection onto the set $C$, i.e. 
\begin{align}\label{eq:prox_proj}
	y_2 = \proj_C (r) \Leftrightarrow (y_2)_i = r_i / \max (|r_i|,1) \quad \text{for all } i.
\end{align}
Putting everything together leads to Algorithm \ref{alg:prior}.
\begin{algorithm}[t!] 
\caption{\textbf{Reconstruction of the prior}}
{
\begin{algorithmic}[1]
\Require step sizes $\tau,\sigma > 0$, data $f_0$, parameter $\alpha_0$
\Ensure $u_0^0 = \bar{u}_0^0 = \Kcal_0^*f_0, ~ y_1^0 = y_2^0 = 0$
	\While{$\sim$ stop crit}
    	\State {\it Dual updates}
        \State $y_1^{k+1} = (\alpha_0 \left[ y_1^k + \sigma (\Kcal_0 \bar{u}_0^k - f_0)\right]) / (\alpha_0 + \sigma)$
          \State $y_2^{k+1} = \proj_C \left(y_2^k + \sigma \nabla \bar{u}_0^k\right)$
          \State {\it Primal updates}
          \State $u_0^{k+1} =  u_0^k - \tau \left[ \Kcal_0^* y_1^{k+1} - \diverg(y_2^{k+1}) \right]$
          \State {\it Overrelaxation}
          \State $\bar{u}_0^{k+1}= 2 u_0^{k+1} - u_0^k$
	\EndWhile\\
\Return $u_0 = u_0^k$
\end{algorithmic}
}
\label{alg:prior}
\end{algorithm}
\ \\

The numerical realization of the dynamic problem is now straightforward.
In order to deal with the infimal convolution, we use its definition and introduce an additional auxiliary variable yielding 
\begin{alignat*}{4}
	&\min_{\ubold}&& ~ &&\sum_{t=1}^T \frac{\alpha_t}{2} \| \Kcal_t u_t - f_t \|_{\C^{M_t}}^2 + \sum_{t=1}^{T-1} \frac{\gamma_t}{2} \|u_{t+1} - u_t \|_{\C_N}^2 + \sum_{t=1}^T w_t \TV(u_t) \\
	& && + && \sum_{t=1}^T (1-w_t) \ICBTV^{p_0}(u_t,u_0) \\    
   = &\min_{\ubold,\zbold}&& ~ &&\sum_{t=1}^T \frac{\alpha_t}{2} \| \Kcal_t u_t - f_t \|_{\C^{M_t}}^2 + \sum_{t=1}^{T-1} \frac{\gamma_t}{2} \|u_{t+1} - u_t \|_{\C_N}^2 +\sum_{t=1}^T w_t \| \nabla u_t \|_1\\
   & && + &&\sum_{t=1}^T (1-w_t) \left[ \| \nabla (u_t - z_t) \|_1 + \| \nabla z_t \|_1  - \langle p_0,u_t \rangle_{\C^N} + \langle 2 p_0, z_t \rangle_{\C^N} \right]
\end{alignat*}
where $\ubold = [u_1, \dots, u_T] \in \C^{N \times T}$ and $\zbold = [z_1, \dots, z_T] \in \C^{N \times T}$.
Introducing a dual variable $\ybold$ for all the terms containing an operator, leads to the primal-dual formulation 
\begin{alignat}{4}
\label{eq:primal_dual}
	&\min_{\ubold,\zbold} \max_{\ybold} && ~ && \sum_{t=1}^T \left(\langle y_{t,1}, \Kcal_t u_t - f_t \rangle_{\C^{M_t}} - \frac{1}{2 \alpha_t} \|y_{t,1} \|_{\C^M}^2 \right) + \sum_{t=1}^{T-1} \frac{\gamma_t}{2} \| u_{t+1} - u_t \|_{\C^N}^2 \notag \\
    & && + && \sum_{t=1}^T \left(\langle y_{t,2}, \nabla u_t \rangle_{\C^{N \times 2}} + \langle y_{t,3}, \nabla (u_t - z_t) _{\C^{N \times 2}} + \langle y_{t,4}, \nabla z_t \rangle_{\C^{N \times 2}}\right) \notag \\
    & && - && \sum_{t=1}^T \langle (1-w_t)p_0,u_t \rangle_{\C^N} + \sum_{t=1}^T \langle 2(1-w_t)p_0,z_t \rangle_{\C^N}  \notag \\
    & && + && \sum_{t=1}^T \left(\chi_{C_{t,2}}(y_{t,2}) + \chi_{C_{t,3}}(y_{t,3}) + \chi_{C_{t,4}}(y_{t,4})\right)
\end{alignat}
where $\ybold = [\ybold_1, \dots, \ybold_T]$, $\ybold_t = [y_{t,1}, \dots, y_{t,4}]$, and for all $t = 1, \dots, T$, $u_{t} \in \C^{M_t}$ and
\begin{align*}
	&C_{t,2} := \{ y \in \C^{M \times 2} ~|~ \|y \|_{\infty} \leq w_t \}, \\
    &C_{t,3} := \{ y \in \C^{M \times 2} ~|~ \|y \|_{\infty} \leq (1-w_t) \}, \\
    &C_{t,4} := \{ y \in \C^{M \times 2} ~|~ \|y \|_{\infty} \leq (1-w_t) \}. \\
\end{align*}
%
\begin{algorithm}[t!]
\caption{\textbf{Dynamic reconstruction with structural prior}}
{
\begin{algorithmic}[1]
\Require step sizes $\tau,\sigma > 0$, subgradient $p_0$, for all $t=1,\dots,T$: data $f_t$, parameters $\alpha_t$, $w_t$, $\gamma_t$
\Ensure for all $t=1,\dots,T$: $u_t^0 = \bar{u}_t^0 = \Kcal_t^*f_t, ~ z_t^0 = \bar{z}_t^0 = 0, ~ y_{t,1}^0 = y_{t,2}^0 = y_{t,3}^0 = y_{t,4}^0 = 0$
	\While{$\sim$ stop crit}
    	\For{t=1,\dots,T} 
          \State {\it Dual updates}
          \State $y_{t,1}^{k+1} = \frac{\alpha_t \left[ y_{t,1} + \sigma (\Kcal_t \bar{u}_t^k - f_t)\right]}{\alpha_t + \sigma}$
          \State $y_{t,2}^{k+1} = \proj_{C_2}\left(y_{t,2}^k + \sigma \nabla \bar{u}_t^k\right)$
          \State $y_{t,3}^{k+1} = \proj_{C_3}\left(y_{t,3}^k + \sigma \nabla (\bar{u}_t^k - \bar{z}_t^k) \right)$
          \State $y_{t,4}^{k+1} = \proj_{C_4}\left(y_{t,4}^k + \sigma \nabla \bar{z}_t^k \right)$
          \State {\it Primal updates}
          \State $u_t^{k+1} =  \frac{u_t^k - \tau \left[ \Kcal_t^* y_{t,1}^{k+1} - \diverg(y_{t,2}^{k+1}) - \diverg(y_{t,3}^{k+1}) - (1-w_t)p_0 \right] + \tau \gamma_t u_{t+1}^k + \tau \gamma_{t-1} u_{t-1}^k}{\tau (\gamma_t + \gamma_{t+1}) +1}$
          \State $z_t^{k+1} - \tau \left[ 2(1-w_t) p_0 + \diverg(y_{t,3}^{k+1}) - \diverg(y_{t,4}^{k+1}) \right]$
          \State {\it Overrelaxation}
          \State $(\bar{u}_t^{k+1}, \bar{z}_t^{k+1}) = 2 (u_t^{k+1}, z_t^{k+1}) - (u_t^k,z_t^k)$
    	\EndFor
	\EndWhile\\
\Return for all $t = 1,\dots,T$: $u_t = u_t^k$
\end{algorithmic}
}
\label{alg:fmri}
\end{algorithm}
%
To solve the problem, we again perform a proximal gradient descent on the primal variables $\ubold$ and $\zbold$, and a proximal gradient ascent on the dual variables $\ybold$, where the gradients are taken with respect to the linear parts, the proximum with respect to the nonlinear parts. 
We hence need to compute the proximal operators for the nonlinear parts in \eqref{eq:primal_dual} to obtain the update steps for $u_t,z_t$ and $\ybold_t$ for every $t = 1, \dots, T$. 
The proximal operators for $\phi_t(y_{t,1}) = \frac{1}{2 \alpha_t} \| y_{t,1} \|_{\C^{M_t}}^2$ can be computed exactly as in \eqref{eq:prox_dual_l2}.
The proximal operators for the updates of $y_{t,j}$, $j = 2,3,4$, are given by projections onto the sets $C_{t,j}$ similar to \eqref{eq:prox_proj}.
For the squared norm related to the time regularization, we notice that for every $1< t < T$, $u_t$ only interacts with the previous and the following time step, i.e. $u_{t-1}$ and  $u_{t+1}$. 
Hence, analogously to $\phi_t$, the proximum for 
\begin{align*}
	\psi_t(u_t) = \frac{\gamma_{t-1}}{2} \|u_t - u_{t-1}\|_{\C^N}^2 + \frac{\gamma_t}{2} \|u_{t+1} - u_t\|_{\C^N}^2
\end{align*}
is given by 
\begin{align*}
	u_t = \prox_{\tau \psi_t} (r) \Leftrightarrow u_t = \frac{r + \tau \gamma_t u_{t+1} + \tau \gamma_{t-1} u_{t-1}}{\tau (\gamma_t + \gamma_{t-1}) + 1}. 
\end{align*}
The two odd updates for $t = 1$ and $t = T$ can be obtained by the same formula by simply setting $\gamma_0 = 0$ and $\gamma_T = 0$, respectively.
Putting everything together, we obtain Algorithm \ref{alg:fmri}.\\

\subsection{Step sizes and stopping criteria}
We quickly discuss the choice of the step sizes $\tau, \sigma$ and stopping criteria for Algorithm \ref{alg:fmri}. 
In most standard applications it stands to reason to choose the step sizes according to the condition $\tau \sigma \| L \|^2 < 1$ ($L$ denotes the collection of all operators) such that convergence of the algorithm is guaranteed \cite{ChambollePock}.
However, depending on $T$, i.e. the number of time frames we consider, the norm of the operator $L$ 
can be very costly to compute, or too large such that the condition $\tau \sigma \| L \|^2 < 1$ only permits extremely small step sizes. 
For practical use, we instead simply choose $\tau$ and $\sigma$ reasonably ''small`` and track both the energy of the problem and the primal-dual residual \cite{Goldstein:Adaptive} to monitor convergence. 
For the sake of brevity, we do not write down the primal-dual residual for Algorithm \ref{alg:fmri} and instead refer the reader to \cite{Goldstein:Adaptive} for its definition. 
The implementation is then straightforward.
We hence stop the algorithm if both, the relative change in energy between consecutive iterates and the primal-dual residual, have dropped below a certain threshold.

\subsection{Practical considerations}
It is clear that for a large number of time frames $T$ Algorithm \ref{alg:fmri} starts to require an increasing amount of time to return reliable results and for reasonably ''large`` step sizes $\tau$ and $\sigma$ it is even doubtful whether we can obtain convergence. 
In practice, it is hence necessary to divide the time series $\Tbold = \{1, \dots, T\}$ into $l$ smaller bits of consecutive time frames. More precisely, choose numbers $1 \leq T_1 < \dots < T_l = T$ such that $\Tbold = \Tbold_1 \cup \Tbold_2 \cup \dots \cup \Tbold_l$ with $\Tbold = \{1, \dots, T_1, T_1+1, \dots, T_2, \dots, T_{l-1}+1, \dots, T_l\}$.
We can then perform the reconstruction separately for all $\Tbold_i$. 
In order to keep the ''continuity`` between $\Tbold_i$ and $\Tbold_{i+1}$, we can include the last frame of $\Tbold_i$ into the reconstruction of $\Tbold_{i+1}$ by letting $\gamma_{T_i} \neq 0$ and choosing $u_{T_i}$ as the respective last frame of $\Tbold_i$.
This divides the overall problem into smaller and easier subproblems, which can be solved faster.
In practice, we observed that a size of five to ten frames per subset $\Tbold_i$ is a reasonable choice, which essentially gives very similar results as doing a reconstruction for the entire time series $\Tbold$.







%!TEX ROOT = ../../centralized_vs_distributed.tex

\section{{\titlecap{the centralized-distributed trade-off}}}\label{sec:numerical-results}

\revision{In the previous sections we formulated the optimal control problem for a given controller architecture
(\ie the number of links) parametrized by $ n $
and showed how to compute minimum-variance objective function and the corresponding constraints.
In this section, we present our main result:
%\red{for a ring topology with multiple options for the parameter $ n $},
we solve the optimal control problem for each $ n $ and compare the best achievable closed-loop performance with different control architectures.\footnote{
\revision{Recall that small (large) values of $ n $ mean sparse (dense) architectures.}}
For delays that increase linearly with $n$,
\ie $ f(n) \propto n $, 
we demonstrate that distributed controllers with} {few communication links outperform controllers with larger number of communication links.}

\textcolor{subsectioncolor}{Figure~\ref{fig:cont-time-single-int-opt-var}} shows the steady-state variances
obtained with single-integrator dynamics~\eqref{eq:cont-time-single-int-variance-minimization}
%where we compare the standard multi-parameter design 
%with a simplified version \tcb{that utilizes spatially-constant feedback gains
and the quadratic approximation~\eqref{eq:quadratic-approximation} for \revision{ring topology}
with $ N = 50 $ nodes. % and $ n\in\{1,\dots,10\} $.
%with $ N = 50 $, $ f(n) = n $ and $ \tau_{\textit{min}} = 0.1 $.
%\autoref{fig:cont-time-single-int-err} shows the relative error, defined as
%\begin{equation}\label{eq:relative-error}
%	e \doteq \dfrac{\optvarx-\optvar}{\optvar}
%\end{equation}
%where $ \optvar $ and $ \optvarx $ denote the the optimal and sub-optimal scalar variances, respectively.
%The performance gap is small
%and becomes negligible for large $ n $.
{The best performance is achieved for a sparse architecture with  $ n = 2 $ 
in which each agent communicates with the two closest pairs of neighboring nodes. 
This should be compared and contrasted to nearest-neighbor and all-to-all 
communication topologies which induce higher closed-loop variances. 
Thus, 
the advantage of introducing additional communication links diminishes 
beyond}
{a certain threshold because of communication delays.}

%For a linear increase in the delay,
\textcolor{subsectioncolor}{Figure~\ref{fig:cont-time-double-int-opt-var}} shows that the use of approximation~\eqref{eq:cont-time-double-int-min-var-simplified} with $ \tilde{\gvel}^* = 70 $
identifies nearest-neighbor information exchange as the {near-optimal} architecture for a double-integrator model
with ring topology. 
This can be explained by noting that the variance of the process noise $ n(t) $
in the reduced model~\eqref{eq:x-dynamics-1st-order-approximation}
is proportional to $ \nicefrac{1}{\gvel} $ and thereby to $ \taun $,
according to~\eqref{eq:substitutions-4-normalization},
making the variance scale with the delay.

%\mjmargin{i feel that we need to comment about different results that we obtained for CT and DT double-intergrator dynamics (monotonic deterioration of performance for the former and oscillations for the latter)}
\revision{\textcolor{subsectioncolor}{Figures~\ref{fig:disc-time-single-int-opt-var}--\ref{fig:disc-time-double-int-opt-var}}
show the results obtained by solving the optimal control problem for discrete-time dynamics.
%which exhibit similar trade-offs.
The oscillations about the minimum in~\autoref{fig:disc-time-double-int-opt-var}
are compatible with the investigated \tradeoff~\eqref{eq:trade-off}:
in general, 
the sum of two monotone functions does not have a unique local minimum.
Details about discrete-time systems are deferred to~\autoref{sec:disc-time}.
Interestingly,
double integrators with continuous- (\autoref{fig:cont-time-double-int-opt-var}) ad discrete-time (\autoref{fig:disc-time-double-int-opt-var}) dynamics
exhibits very different trade-off curves,
whereby performance monotonically deteriorates for the former and oscillates for the latter.
While a clear interpretation is difficult because there is no explicit expression of the variance as a function of $ n $,
one possible explanation might be the first-order approximation used to compute gains in the continuous-time case.
%which reinforce our thesis exposed in~\autoref{sec:contribution}.

%\begin{figure}
%	\centering
%	\includegraphics[width=.6\linewidth]{cont-time-double-int-opt-var-n}
%	\caption{Steady-state scalar variance for continuous-time double integrators with $ \taun = 0.1n $.
%		Here, the \tradeoff is optimized by nearest-neighbor interaction.
%	}
%	\label{fig:cont-time-double-int-opt-var-lin}
%\end{figure}
}

\begin{figure}
	\centering
	\begin{minipage}[l]{.5\linewidth}
		\centering
		\includegraphics[width=\linewidth]{random-graph}
	\end{minipage}%
	\begin{minipage}[r]{.5\linewidth}
		\centering
		\includegraphics[width=\linewidth]{disc-time-single-int-random-graph-opt-var}
	\end{minipage}
	\caption{Network topology and its optimal {closed-loop} variance.}
	\label{fig:general-graph}
\end{figure}

Finally,
\autoref{fig:general-graph} shows the optimization results for a random graph topology with discrete-time single integrator agents. % with a linear increase in the delay, $ \taun = n $.
Here, $ n $ denotes the number of communication hops in the ``original" network, shown in~\autoref{fig:general-graph}:
as $ n $ increases, each agent can first communicate with its nearest neighbors,
then with its neighbors' neighbors, and so on. For a control architecture that utilizes different feedback gains for each communication link
	(\ie we only require $ K = K^\top $) we demonstrate that, in this case, two communication hops provide optimal closed-loop performance. % of the system.}

Additional computational experiments performed with different rates $ f(\cdot) $ show that the optimal number of links increases for slower rates: 
for example, 
the optimal number of links is larger for $ f(n) = \sqrt{n} $ than for $ f(n) = n $. 
\revision{These results are not reported because of space limitations.}
% % \vspace{-0.5em}
\section{Conclusion}
% \vspace{-0.5em}
Recent advances in multimodal single-cell technology have enabled the simultaneous profiling of the transcriptome alongside other cellular modalities, leading to an increase in the availability of multimodal single-cell data. In this paper, we present \method{}, a multimodal transformer model for single-cell surface protein abundance from gene expression measurements. We combined the data with prior biological interaction knowledge from the STRING database into a richly connected heterogeneous graph and leveraged the transformer architectures to learn an accurate mapping between gene expression and surface protein abundance. Remarkably, \method{} achieves superior and more stable performance than other baselines on both 2021 and 2022 NeurIPS single-cell datasets.

\noindent\textbf{Future Work.}
% Our work is an extension of the model we implemented in the NeurIPS 2022 competition. 
Our framework of multimodal transformers with the cross-modality heterogeneous graph goes far beyond the specific downstream task of modality prediction, and there are lots of potentials to be further explored. Our graph contains three types of nodes. While the cell embeddings are used for predictions, the remaining protein embeddings and gene embeddings may be further interpreted for other tasks. The similarities between proteins may show data-specific protein-protein relationships, while the attention matrix of the gene transformer may help to identify marker genes of each cell type. Additionally, we may achieve gene interaction prediction using the attention mechanism.
% under adequate regulations. 
% We expect \method{} to be capable of much more than just modality prediction. Note that currently, we fuse information from different transformers with message-passing GNNs. 
To extend more on transformers, a potential next step is implementing cross-attention cross-modalities. Ideally, all three types of nodes, namely genes, proteins, and cells, would be jointly modeled using a large transformer that includes specific regulations for each modality. 

% insight of protein and gene embedding (diff task)

% all in one transformer

% \noindent\textbf{Limitations and future work}
% Despite the noticeable performance improvement by utilizing transformers with the cross-modality heterogeneous graph, there are still bottlenecks in the current settings. To begin with, we noticed that the performance variations of all methods are consistently higher in the ``CITE'' dataset compared to the ``GEX2ADT'' dataset. We hypothesized that the increased variability in ``CITE'' was due to both less number of training samples (43k vs. 66k cells) and a significantly more number of testing samples used (28k vs. 1k cells). One straightforward solution to alleviate the high variation issue is to include more training samples, which is not always possible given the training data availability. Nevertheless, publicly available single-cell datasets have been accumulated over the past decades and are still being collected on an ever-increasing scale. Taking advantage of these large-scale atlases is the key to a more stable and well-performing model, as some of the intra-cell variations could be common across different datasets. For example, reference-based methods are commonly used to identify the cell identity of a single cell, or cell-type compositions of a mixture of cells. (other examples for pretrained, e.g., scbert)


%\noindent\textbf{Future work.}
% Our work is an extension of the model we implemented in the NeurIPS 2022 competition. Now our framework of multimodal transformers with the cross-modality heterogeneous graph goes far beyond the specific downstream task of modality prediction, and there are lots of potentials to be further explored. Our graph contains three types of nodes. while the cell embeddings are used for predictions, the remaining protein embeddings and gene embeddings may be further interpreted for other tasks. The similarities between proteins may show data-specific protein-protein relationships, while the attention matrix of the gene transformer may help to identify marker genes of each cell type. Additionally, we may achieve gene interaction prediction using the attention mechanism under adequate regulations. We expect \method{} to be capable of much more than just modality prediction. Note that currently, we fuse information from different transformers with message-passing GNNs. To extend more on transformers, a potential next step is implementing cross-attention cross-modalities. Ideally, all three types of nodes, namely genes, proteins, and cells, would be jointly modeled using a large transformer that includes specific regulations for each modality. The self-attention within each modality would reconstruct the prior interaction network, while the cross-attention between modalities would be supervised by the data observations. Then, The attention matrix will provide insights into all the internal interactions and cross-relationships. With the linearized transformer, this idea would be both practical and versatile.

% \begin{acks}
% This research is supported by the National Science Foundation (NSF) and Johnson \& Johnson.
% \end{acks}

A new family of Langevin samplers was introduced in this paper. These new SDE samplers consist of perturbations of the underdamped Langevin dynamics (that is known to be ergodic with respect to the canonical measure), where auxiliary drift terms in the equations for both the position and the momentum are added, in a way that the perturbed family of dynamics is ergodic with respect to the same (canonical) distribution. These new Langevin samplers were studied in detail for Gaussian target distributions where it was shown, using tools from spectral theory for differential operators, that an appropriate choice of the perturbations in the equations for the position and momentum can improve the performance of the Langvin sampler, at least in terms of reducing the asymptotic variance. The performance of the perturbed Langevin sampler to non-Gaussian target densities was tested numerically on the problem of diffusion bridge sampling.

The work presented in this paper can be improved and extended in several directions. First, a rigorous analysis of the new family of Langevin samplers for non-Gaussian target densities is needed. The analytical tools developed in~\cite{duncan2016variance} can be used as a starting point. Furthermore, the study of the actual computational cost and its minimization by an appropriate choice of the numerical scheme and of the perturbations in position and momentum would be of interest to practitioners. In addition, the analysis of our proposed samplers can be facilitated by using tools from symplectic and differential geometry. Finally, combining the new Langevin samplers with existing variance reduction techniques such as zero variance MCMC, preconditioning/Riemannian manifold MCMC can lead to sampling schemes that can be of interest to practitioners, in particular in molecular dynamics simulations. All these topics are currently under investigation.


\begin{comment}
\notate{There needs to be a conclusion to the paper}
\subsection{'Symplectic manifold MCMC'}

The generator of the unperturbed Langevin dynamics (\ref{eq:langevin})
given by 
\[
\mathcal{L}_{0}=M^{-1}p\cdot\nabla_{q}-\nabla V(q)\cdot\nabla_{p}-\Gamma M^{-1}p\cdot\nabla_{p}+\nabla\Gamma\nabla
\]
is often written as 
\[
\mathcal{L}_{0}=\{H,\cdot\}-\Gamma M^{-1}p\cdot\nabla_{p}+\nabla\Gamma\nabla,
\]
where the Hamiltonian is expressed in terms of the \emph{Hamiltonian
	\[
	H(q,p)=V(q)+\frac{1}{2}p^{T}M^{-1}p
	\]
}and the \emph{Poisson bracket
	\[
	\{A,B\}=\big((\nabla_{q}A)^{T}(\nabla_{p}A)^{T}\Pi_{0}\left(\begin{array}{c}
	\nabla_{q}B\\
	\nabla_{p}B
	\end{array}\right).
	\]
}Here $\Pi_{0}$ denotes the $2d\times2d$-matrix 
\[
\Pi_{0}=\left(\begin{array}{cc}
\boldsymbol{0} & -I\\
I & \boldsymbol{0}
\end{array}\right)
\]
and $A,B:\mathbb{R}^{2d}\rightarrow\mathbb{R}^{2d}$ are sufficiently
regular functions. Now observe that the generator (\ref{eq:generator})
can be expressed in the same way if $\Pi_{0}$ is replaced by 
\[
\tilde{\Pi}=\left(\begin{array}{cc}
\mu J_{1} & -I\\
I & \nu J_{2}
\end{array}\right).
\]
Therefore, the perturbations under investigation in this paper can
be interpreted more abstractly as a change of Poisson structure. In
this framework, the unperturbed Langevin dynamics (\ref{eq:langevin})
should be thought of as evolving in $\mathbb{R}^{2d}$ equipped with
the canonical symplectic structure associated to $\Pi_{0}$. The perturbed
dynamics (\ref{eq:perturbed_underdamped}) then represent
a process in $\mathbb{R}^{2d}$ equipped with the symplectic structure
given rise to $\tilde{\Pi}$. This alternative viewpoint has multiple
advantages. For instance, the underlying symplectic structure suggests
efficient numerical integrators for the perturbed dynamics. Moreover,
this formulation naturally allows for the possibility of introducing
point-dependent Poisson structures connented to perturbations $J_{1}(q,p)$
and $J_{2}(q,p)$ that depend on $q$ and $p$. In this way, it might
be possible to extend the results of this paper to target measures
with locally different covariance structures or targets that are far
away from Gaussian. Lastly, this formulation interacts nicely with
\emph{Riemannian manifold Monte Carlo }approaches (see \cite{GirolamiCalderhead2011}),
where the metric structure of the underlying manifold instead of the
symplectic structure is changed. All of those connections will be
explored further in a subsequent publication.


\end{comment}


\section*{Acknowledgements}

This work has been supported  ERC via Grant EU FP 7 - ERC Consolidator Grant 615216 LifeInverse, by the German Ministery for Science and Education (BMBF) through the project MED4D, by Academy of Finland (Finnish Programme for Center of Excellence in Research 2012-2017, project 250215) and Jane and Aatos Erkko Foundation. 
The authors would like to thank the Isaac Newton Institute for   Mathematical Sciences, Cambridge, for support and hospitality during  the programme Variational Methods for Imaging and Vision, where work on this paper was  undertaken, supported by EPSRC grant no EP/K032208/1.  

\bibliographystyle{abbrv}
\bibliography{fmri_ref}

\end{document}