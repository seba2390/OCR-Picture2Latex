\section{Introduction}
\label{sec:intro}
Since its invention in the 1970s, magnetic resonance imaging (MRI) has developed into a very versatile non-invasive medical imaging technique, which can provide information about a rich variety of tissue properties with a high spatial resolution and good soft tissue contrast. 
Initially, however, this high image quality came along with long acquisition times rendering the imaging of rapid physiological changes impossible. 
Hence, since the early days of MRI, the speed-up of the data acquisition process has been a major direction of research. 
In the meantime significant progress in hardware and development of fast imaging protocols has paved the way for modern dynamic imaging techniques such as functional MRI (fMRI) and dynamic contrast-enhanced MRI (DCE-MRI). 

In this paper, we propose a novel variational approach for the reconstruction of highly subsampled dynamic MR data. 
It combines temporal smoothing with spatial total variation (TV) regularization and uses an $\ICBTV$ regularization functional to incorporate structural prior information from an anatomical MR scan acquired prior to the dynamic sequence. 
In the subsequent paragraphs, we briefly introduce the concept of fMRI and DCE-MRI and explain the idea of our approach in more detail afterwards.\\

\noindent {\it Functional magnetic resonance imaging}\\
\noindent
The first fMRI images depicting physiological function inside the brain were presented in the early 1990's \cite{ogawa90,ogawa92}. 
Since then, fMRI has developed into a widely used imaging technique in basic brain research and serves as a tool for the identification and characterization of brain diseases such as Alzheimer's disease.
The fMRI technique relies on the idea that neuronal activity leads to an increase in metabolism and hence to a local change in the cerebral blood flow that can be detected in T$2^*$ weighted MR signals. 
Consequently, during an fMRI experiment a series of fast T$2^*$ weighted MR scans is taken over time to obtain spatial information about the hemodynamic changes in the brain. For a general reference on fMRI, see for example \cite{fmribook}.\\

\noindent {\it Dynamic contrast-enhanced magnetic resonance imaging}\\
\noindent
The technique of DCE-MRI also dates back to the early 1990s \cite{tofts91}. 
In contrast to fMRI, this imaging method is based on injection of an exogenous bolus of gadolinium-based contrast agent.  
Aiming at the detection of cerebral ischemia, brain tumor or infections, it relies on the fact that a characteristic property of normal and healthy tissue in the brain is an intact blood-brain barrier. 
This blood-brain barrier prevents the contrast agent from passing the cell membranes and hence causes the contrast agent to quickly re-enter the vessels instead of accumulating in the tissue. 
However, in tissue that has been damaged by a tumor or an infection the blood-brain barrier is disrupted, and hence the contrast agent accumulates.
In order to detect this change of contrast, a time series of fast MR scans is acquired.
The scans can be based on either a short or long echo time depending on the application. 
While imaging of tumors is usually based on a short echo time, aiming at good T$1$ contrast, the brain ischemia studies usually use T$2^\ast$ weighted imaging with short echo times for detection of the fast perfusion related changes. For further details on DCE-MRI, see for example
the review papers \cite{padhani2002,choyke2003,tofts1999}.
\\ 

\noindent {\it Compressed sensing in magnetic resonance imaging}\\
\noindent
Since the strength of dynamic MRI methods lies in their ability to measure spatial contrast or physiological changes with rapidly switching dynamics, a critical point is to guarantee a sufficiently high temporal resolution in the measured data. 
In this context, the evolution of the theory of compressed sensing (CS) and its application to MRI, e.g. \cite{Candes:Robust,Donoho:CompressedSensing,Lustig:Sparse,Huang:CSinMR}, allowed for a significant acceleration of the data acquisition process. 
Building on the observation that images are likely to have a sparse representation in some transform domain, it led to a boost in the development of new mathematical reconstruction techniques.
These are capable of producing reconstructions of comparable quality from a substantially reduced amount of sampled $k$-space coefficients, provided that the remaining coefficients are chosen in a suitable way. 
In this context, the subsampling scheme is regarded as being suitable, if the resulting aliasing artifacts in the images obtained by linear reconstruction methods are incoherent, or in other words noise-like, in the respective transform domain.
Appropriate nonlinear reconstruction methods promoting sparsity in the particular transform domain and consistency with the acquired data can then facilitate high quality reconstructions despite the comparably small amount of measured data. 
In the following, we will therefore assume that at each timestep of the respective dynamic MRI acquisition process only a very sparse subset of the entire $k$-space is sampled. \\

\noindent {\it The proposed approach}\\
\noindent
Gathering the ideas of fMRI, DCE-MRI and compressed sensing described above, we introduce the three buidling blocks of our variational model for the reconstruction of dynamic MR data. 

The first is an appropriate data fidelity term (cf. Sections \ref{subsec:operator modeling} and \ref{subsec:dynamic imaging}) combined with a (spatial) total variation (TV) regularization \cite{Rudin:ROF} to ensure that the reconstructed images to a certain degree exhibit sparsity in the gradient domain (cf. Section \ref{subsubsec:TV}). 
Provided a sufficiently high temporal resolution, we moreover expect the intensity values to change only in a small part of the entire image domain between two consecutive timesteps (since only a small area is activated). 
Note that this is a common assumption for dynamic MRI applications, especially when motion is negligible, see for example \cite{feng2013,feng2014,wundrak2016}.
Hence, we also suppose a temporal redundancy of the dynamic data that we want to derive benefit from. 
Against this background, it seems reasonable to sample the $k$-space for successive timesteps in a complementary manner, meaning that the subsampling scheme should not be identical among timesteps that are temporally close to each other, but rather such that it adds to the information gained at neighboring points in time. 
One such sampling scheme, which is popular in DCE-MRI, is the radial golden angle sampling where the $k$-space is sampled continuously along radial spokes with a uniform angular stepping ($111.25^\circ$).
It leads to a rather regular spatial coverage of the $k$-space over time with high temporal incoherence and offers great flexibility with respect to the selection of the time resolution in the sense that one can afterwards choose how many radial spokes per frame are used in the reconstruction of the data \cite{winkelmann2007,feng2014}.
The second building block is hence a temporal regularization (cf. Section \ref{subsec:temporal regularization}), which couples the sampled data across the dynamic sequence such that the information acquired at neighboring time points can contribute to the reconstruction of the respective frame.

Finally, we seek to take advantage of the specific experimental setups:
in both fMRI and DCE-MRI, the experimental protocol includes an anatomical high-resolution MRI scan, which is carried out prior to the acquisition of the actual dynamic MRI data. 
While this data contains high resolution anatomical information of the target, it is mostly used for visualization purposes only. 
One of the goals of our approach is to utilize the structural information contained in the anatomical image as prior information in the reconstruction of the dynamic image sequence.
In order to increase quality and hence obtain a higher spatial resolution in the reconstruction of the dynamic MRI data series, we incorporate the edge information of the anatomical high-resolution scan encoded in the subgradient into the reconstruction framework (cf. Section \ref{subsubsec:structural prior}). 
Mathematically, this is realized by the $\ICBTV$ regularization that has first been introduced for color image processing in \cite{Moeller:ColorBregmanTV} and discussed in more detail and applied to PET-MRI joint reconstruction in \cite{Rasch2017}.\\ 

\noindent {\it Related work}\\
\noindent
Our approach certainly stands in the tradition of other CS-related variational models for the reconstruction of dynamic MRI data that combine suitable undersampling strategies with spatio-temporal regularization \cite{Lustig:ktSPARSE,Adluru2007_2,Jung:ktFOCUSS,Tremoulheac:lowRankPlusSparsePrior,Otazo:lowRankPlusSparseMatrixDecomposition,Yao:nuclearNormdMRI,feng2013,feng2014,wundrak2016}. 
Note that the list of related work given here is only a small excerpt of methods proposed in literature. 
We believe that the method of Adluru and coworkers \cite{Adluru2007_2} is closest to our new approach.
However, to the best of our knowledge, no method has yet been proposed which combines the incorporation of high-resolution structural a-priori information from the anatomical prescan with spatio-temporal regularization allowing for both simultaneously, a surprisingly high temporal and spatial resolution.\\

\noindent {\it Organization of the paper}\\
\noindent
The remainder of the paper is organized as follows: In Section 2 we describe the mathematical forward model for MRI and introduce the proposed reconstruction method.  
In Section 3, we first comment on the implementation of all operators and then explain how to solve the minimization problem by a first-order primal-dual optimization method. 
The numerical results are given in Section 4. 
First we evaluate the method using a test case based on fMRI with simulated measurement data, then the approach is evaluated using experimental small animal DCE-MRI data from a glioma model rat specimen, where in both cases the data results from golden angle radial sampling. 
We complete the paper with a brief conclusion and give an outlook on future directions of research in this area. 