\section{Conclusion and outlook}
\label{sec:outlook}
In the previous sections we presented a novel variational model for the reconstruction of highly subsampled dynamic MRI data where an anatomical scan (at high spatial resolution) has been acquired prior to the dynamic sequence. 
Combining radial golden angle sampling with a suitable time regularization, spatial TV regularization and with the infimal convolution of TV Bregman distances allowing to incorporate the structural information of the anatomical prior, we obtained spatially highly resolved reconstructions at a high temporal resolution. 

Summing up the results of tests on a simulated data set based on fMRI as well as on experimental small animal DCE-MRI data, we draw the following conclusions:
naturally, a simple least squares (LS) reconstruction of each individual frame could not provide meaningful results due to the severe undersampling. 
Adding a spatial TV regularization did not significantly increase the quality of the reconstructed images. 
As expected, this approach yielded piecewise constant images, but the ratio of sampled Fourier coefficients in $k$-space in comparison to the desired spatial resolution of the reconstructions was too small to obtain reasonable results. 
Remarkably, adding only Tikhonov regularization on the time derivative without any additional spatial regularization already resulted in by far more meaningful reconstructions than the LS approach, while the obtained images were still corrupted by heavy noise.
Integrating spatial TV regularization to the aforementioned model removed most of the noise and indeed provided high quality reconstructions. 
Incorporating structural information from the anatomical prior, we could then obtain very detailed results despite the severe subsampling enabling high temporal resolution.

In view of these promising results, we state some open questions and sketch additional ideas whose detailed study is left to future research.

We used the infimal convolution of TV Bregman distances to incorporate structural information from the anatomical prescan. 
Naturally, this gives rise to the question whether alternative means of incorporating structural prior information such as the concepts of weighted total variation (wTV) or directional total variation (dTV), respectively, both proposed in \cite{Ehrhardt2016}, yield significantly different results. 
In any case, it would be interesting to see how such a modified approach compares to the method proposed in this paper concerning quality of the reconstructed images, but also regarding computational complexity of solving the respective minimization problem. 

Moreover, the temporal coupling of time frames serves as a further starting point for future research. 
Here, we decided to apply Tikhonov regularization of the time derivative, however, one could also argue in favor of other concepts: 
since in the areas of application we considered in this paper the dynamic changes happen to take place in only a small portion of the entire image domain, decomposition of the dynamic sequence into a low rank part $L$ and a part $S$ which is sparse in some transform domain \cite{Tremoulheac:lowRankPlusSparsePrior,Otazo:lowRankPlusSparseMatrixDecomposition} could be an interesting alternative. 
Assuming that the dynamic changes mainly are contained in $S$, while $L$ ideally comprises the part staying constant over time, it seems particularly reasonable that the structures of the constant part of every time frame bear close resemblance to the structure of the anatomical prior image. 
Hence it would stand to reason to apply the infimal convolution of TV Bregman distances only to the low rank part leaving the sparse part untouched. 
However, against the backdrop of different dimensions of the low rank part of the dynamic sequence $L$ and the anatomical prior image $u_0$ it is not yet clear what would be the most suitable way of solving the corresponding optimization problem. 

Finally, with respect to experimental data, a more careful correction of artifacts due to different acquisition protocols between anatomical prescan and the dynamic sequence might be an interesting aspect.
