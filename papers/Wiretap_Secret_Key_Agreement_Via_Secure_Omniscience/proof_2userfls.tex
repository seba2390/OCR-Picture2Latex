
Before presenting the proof, we introduce some notation that will be used in the achievability part of our argument. It is known, from the proof of Lemma~5.2 of \cite{chan18zero}, that a finite linear source $(\RX, \RY)$ can be decomposed as
\begin{align}
        \RX &= (\RX',\RC), \label{eq:X_X'C}\\
        \RY &= (\RY',\RC), \label{eq:Y_Y'C}
\end{align}
where  $\RX'$ (resp. $\RY'$) is a linear function of $\RX$ (resp. $\RY$) and $\RC=\op{mcf}(\RX, \RY)$ is a linear function of each of $\RX$ and $\RY$; altogether, they satisfy the independence relation
\begin{align}
    H(\RX',\RC,\RY') = H(\RX')+H(\RC)+H(\RY'). \label{eq:XCY_independence}
\end{align}
We use the notation $\RX \setminus \RY$ (resp. $\RY \setminus \RX$) to denote $\RX'$ (resp. $\RY'$).

\begin{proof}
\emph{Converse part.} Note that $\RG$ satisfies the Markov condition $\RG-\RZ_{\opw}-\RZ_V$ because $\RG$ is a function of $\RZ_{\opw}$ whether it is chosen to be $\RG_1$, $\RG_2$ or both. By \eqref{eq:RL:lb}, we have
\begin{align*}
\rl &\geq H(\RZ_V|\RZ_{\opw}) - \wskc(\RZ_V\|\RZ_{\opw})\\
&\utag{a}\geq H(\RZ_V|\RZ_{\opw}) - \pkc(\RZ_V|\RG)\\
&\utag{b}=H(\RZ_V|\RZ_{\opw}) - I(\RZ_1\wedge \RZ_2 | \RG)
\end{align*}
where \uref{a} is because for $\RW-\RZ_{\opw}-\RZ_V$, $\wskc(\RZ_V\|\RZ_{\opw}) \leq \pkc(\RZ_V|\RW)$ \cite[Theorem~4]{csiszar04} and $\RG$ forms the Markov condition $\RG-\RZ_{\opw}-\RZ_V$, and we have used the fact that $\pkc(\RZ_V|\RG)=I(\RZ_1\wedge \RZ_2 | \RG)$ \cite[Theorem~2]{csiszar04} in \uref{b} .\\

\emph{Achievability part.}
It suffices to prove the reverse inequality for $\RG=\RG_1$, i.e.,
\begin{align}
    \rl \leq H(\RZ_V|\RZ_{\opw}) - \underbrace{I(\RZ_1\wedge \RZ_2|\RG_1)}_{`(1)} \label{eq:fls:ub}
\end{align}
because then the reverse inequality will also hold for $\RG=\RG_2$ by symmetry, and for $\RG=(\RG_1,\RG_2)$ since 
\[I(\RZ_1\wedge \RZ_2|\RG_1,\RG_2)\leq I(\RZ_1\wedge \RZ_2,\RG_2 \mid \RG_1) = I(\RZ_1\wedge \RZ_2|\RG_1)\]
by the assumption that $\RG_2$ is a function of $\RZ_2$. 



The desired reverse inequality~\eqref{eq:fls:ub} will follow from the following upper bound with an appropriate choice of public discussion $\RF'$ of block length $1$, i.e.,
\begin{align*}
    \rl &\leq \underbrace{\rco(\RZ_V|\RF')}_{=H(\RZ_V|\RF')-I(\RZ_1\wedge \RZ_2|\RF')} + \underbrace{I(\RZ_V\wedge \RF'|\RZ_{\opw})}_{=H(\RZ_V|\RZ_{\opw})-H(\RZ_V|\RZ_{\opw},\RF')}\\
    &= H(\RZ_V|\RZ_{\opw})+ \underbrace{I(\RZ_V\wedge \RZ_{\opw}|\RF')}_{`(2)} -\underbrace{I(\RZ_1\wedge \RZ_2|\RF')}_{`(3)}.
\end{align*}
The idea behind this upper bound involves splitting of the leakage rate into two components after a discussion $\RF'$: one component is the leakage rate due to  $\RF'$, and the other one is the residual leakage rate for subsequent omniscience, which is upper bounded by $\rco(\RZ_V|\RF')$. Note that the equality $\rco(\RZ_V|\RF')=H(\RZ_V|\RF')-I(\RZ_1\wedge \RZ_2|\RF')$ follows from the characterization of the conditional $\rco$ given in \cite[Proposition~1]{csiszar04}.
It suffices to give a feasible $\RF'$ with $`(2)=0$ and $`(3)=`(1)$. We will construct this $\RF'$ by decomposing the source $(\RZ_V, \RZ_{\opw})$.

For the source $(\RX, \RY)$ with $\RX=(\RZ_1, \RG_1)$ and $\RY=(\RZ_2, \RG_1)$, the decomposition is as follows: 
\begin{align}
        (\RZ_1, \RG_1) &:= (\RX_a ,\RX_c), \label{eq:Z'1}\\
        (\RZ_2, \RG_1) &:= (\RX_b ,\RX_c),\label{eq:Z'2}
\end{align}
where $\RX_a$, $\RX_b$, and $\RX_c = \op{mcf}((\RZ_1, \RG_1), (\RZ_2, \RG_1))$ are uniformly random row vectors over some finite field, say $\bbF_q$, satisfying the independence relation
\begin{align}
      H(\RX_a,\RX_b,\RX_c) &= H(\RX_a)+H(\RX_b)+H(\RX_c)\label{eq:XG}.
\end{align}

Observe that $\RG_1$ is a linear common function of $(\RZ_1, \RG_1)$ and $(\RZ_2, \RG_1)$. Using the decomposition \eqref{eq:Z'1} and \eqref{eq:Z'2}, we can write $\RG_1= \RX_a\MM_a+\RX_c\MM_c =\RX_b\MM_b+\RX_c\tMM_c$ for some matrices $\MM_a,\MM_b,\MM_c$ and $\tMM_c$. Therefore, we have $\RX_a\MM_a-\RX_b\MM_b +\RX_c(\MM_c -\tMM_c)=0$. But, $\RX_a$, $\RX_b$, and $\RX_c$ are mutually independent, which implies (for finite linear sources) that $\MM_a=\MM_b=\MM_c -\tMM_c=0$ and $\RG_1= \RX_c\MM_c$. This shows that $\RG_1$ is a linear function of $\RX_c$.
% Since any linear common function is a linear function of the m.c.f. of an FLS, we have that $\RG_1$ is a linear function of $\RX_c$.
% \begin{align}
%      H(\RG_1|\RX_c)=0. \label{eq:g1_xc}
% \end{align}
Let $\RX'_c := \RX_c \setminus \RG_1$. So, we can write $\RX_c=(\RX'_c, \RG_1)$, where $\RX'_c$ is independent of $\RG_1$, and both are linear functions of $\RX_c$. Therefore, we can further decompose the source in \eqref{eq:Z'1} and \eqref{eq:Z'2} as
\begin{align}
        (\RZ_1, \RG_1) &= (\RX_a ,\RX'_c,\RG_1), \label{eq:Z'1_decomp}\\
        (\RZ_2, \RG_1) &= (\RX_b ,\RX'_c, \RG_1),\label{eq:Z'2_decomp}
\end{align}
where $\RX_a$, $\RX_b$, and $\RX'_c$ are uniformly random row vectors such that 
\begin{align}
      H(\RX_a,\RX_b,\RX'_c,\RG_1) &= H(\RX_a)+H(\RX_b)+H(\RX'_c)+H(\RG_1)\label{eq:XG_decomp}.
\end{align}
Note that $(\RX_b ,\RX'_c)=\RZ_2 \setminus \RG_1$ which is a linear function of $\RZ_2$.

Now consider the decomposition of the form \eqref{eq:X_X'C} and \eqref{eq:Y_Y'C} for the source  $(\RZ_V,\RZ_{\opw})$: 
\begin{align}
        \RZ_V &:= (\RZ'_V, \RG_{\opw}), \label{eq:ZV_decomp}\\
        \RZ_{\opw} &:= (\RZ'_{\opw} ,\RG_{\opw}),\label{eq:ZW_decomp}
\end{align}
where $\RG_{\opw}$ is the m.c.f. of $\RZ_V$ and $\RZ_{\opw}$. As the components $(\RZ'_V, \RG_{\opw}, \RZ'_{\opw})$ are mutually independent by \eqref{eq:XCY_independence}, we have 
\begin{align}
    I(\RZ_V \wedge \RZ_{\opw}|\RG_{\opw}) =0.\label{eq:Z'Zw'}
\end{align}
Moreover, using the fact that the m.c.f. $\RG_{\opw}$ is a linear function of $\RZ_V$, and $(\RX_a,\RX_b,\RX'_c,\RG_1)$ is an invertible linear transformation of $\RZ_V$ (by \eqref{eq:Z'1_decomp} and \eqref{eq:Z'2_decomp}),  we can  write $\RG_{\opw}$ as
\begin{align}
    \RG_{\opw} = \RX_a \tMA + \RX_b \tMB + \RX'_c \tMC +\RG_1\tMD \label{eq:Gw}
\end{align}
for some deterministic matrices $\tMA$, $\tMB$, $\tMC$ and $\tMD$ over $\bbF_q$ such that $[\tMA^T \; \tMB^T \; \tMC^T \; \tMD^T]^T$ is a full column-rank matrix.  Since $\RG_1$ is a m.c.f. of $\RZ_1$ and $\RZ_{\opw}$, it is a linear function of $\RG_{\opw}$, which can also be argued along the same lines as the proof of $\RG_1$ is a linear function of $\RX_c$. So we can write \eqref{eq:Gw} as 
\begin{align}
    \RG_{\opw} = (\RX_a \MA + \RX_b \MB + \RX'_c \MC, \RG_1) \label{eq:Gw_decompose}
\end{align}
for some deterministic matrices $\MA$, $\MB$, and $\MC$ over $\bbF_q$ such that $\RX_a \MA + \RX_b \MB + \RX'_c \MC = \RG_{\opw} \setminus \RG_1$. 

Finally, by \eqref{eq:Z'1_decomp}, \eqref{eq:Z'2_decomp}, \eqref{eq:ZW_decomp}, \eqref{eq:Z'Zw'} and \eqref{eq:Gw_decompose}, we can write the decomposition of the source $(\RZ_V, \RZ_{\opw})$ as
\begin{align}
    \RZ_1 &= (\RX_a, \RX'_c,\RG_1),\label{eq:final_z1}\\
    (\RZ_2, \RG_1) &= (\RX_b ,\RX'_c, \RG_1)\label{eq:final_z2}\\
    \RZ_{\opw} &= (\RZ'_{\opw} ,\RX_a \MA + \RX_b \MB + \RX'_c \MC, \RG_1),\label{eq:final_zw}
\end{align}
where the components $\RX_a,\RX_b,\RX'_c,\RG_1,\RZ'_{\opw}$ and $\RX_a \MA + \RX_b \MB + \RX'_c \MC)$ satisfy the following independence relations:
\begin{enumerate}
    \item \eqref{eq:XG_decomp} holds, i.e., $\RX_a,\RX_b,\RX'_c$ and $\RG_1$ are mutually independent;
    \item $\RX_a,\RX'_c,\RG_1,\RZ'_{\opw}$ and $\RX_a \MA + \RX_b \MB + \RX'_c \MC$ are mutually independent.
\end{enumerate}
To verify the second independence relation above, it is enough to show that $I(\RX_a,\RX'_c\wedge \RG_1, \RZ'_{\opw}, \RX_a \MA + \RX_b \MB + \RX'_c \MC)=0$ because of \eqref{eq:XG_decomp},\eqref{eq:ZW_decomp}, and \eqref{eq:Gw_decompose}, which is equivalent to showing  $I(\RX_a,\RX'_c\wedge  \RZ'_{\opw}, \RX_a \MA + \RX_b \MB + \RX'_c \MC \mid \RG_1)=0$ by \eqref{eq:XG_decomp}. Note that  by \eqref{eq:Z'Zw'},  $0=I(\RZ_1\wedge \RZ_{\opw}|\RG_1)=I(\RZ_1, \RG_1\wedge \RZ_{\opw}|\RG_1)= I(\RX_a,\RX'_c,\RG_1\wedge \RG_1, \RZ'_{\opw}, \RX_a \MA + \RX_b \MB + \RX'_c \MC \mid \RG_1) = I(\RX_a,\RX'_c \wedge \RZ'_{\opw}, \RX_a \MA + \RX_b \MB + \RX'_c \MC \mid \RG_1)$.

%  Therefore, we have 
% \begin{align}
%     I(\RX_a, \RX'_c \wedge \RZ_{\opw})=0.
% \end{align}


Let us construct a linear communication using the components from the above decomposition. User $1$ transmits $\RF'_1 := (\RX_a \MA, \RG_1)$ using his source $\RZ_1= (\RX_a, \RX'_c,\RG_1)$. User $2$ communicates $\RF'_2 := \RX_b \MB + \RX'_c \MC$ using the source $(\RX_b ,\RX'_c)$ which is a function of $\RZ_2$. Define $\RF':=(\RF'_1, \RF'_2)$,  a valid  discussion of block length $n=1$. 


By \eqref{eq:Z'Zw'}, we have  $0= I(\RZ_V \wedge \RZ_{\opw}|\RG_{\opw}) = I(\RZ_V, \RF' \wedge \RZ_{\opw}|\RG_{\opw})= I(\RF' \wedge \RZ_{\opw}|\RG_{\opw})+I(\RZ_V \wedge \RZ_{\opw}|\RG_{\opw}, \RF')=I(\RZ_V \wedge \RZ_{\opw}| \RF')$, where the last equality follows from $I(\RF' \wedge \RZ_{\opw}|\RG_{\opw})\leq I(\RZ_V \wedge \RZ_{\opw}|\RG_{\opw}) \stackrel{\eqref{eq:Z'Zw'}}{=}0$, and $H(\RG_{\opw}|\RF')=0$.  Hence we conclude that 
\begin{align}
    `(2) = I(\RZ_V \wedge \RZ_{\opw} \mid \RF')=0 \label{eq:Z'wF'}
\end{align}


Let us show the remaining inequality $`(3)=`(1)$. By the independence relation 1), we evidently have
\begin{align}
I(\RX_a\MA \wedge \RX_b, \RX'_c \mid \RG_1) &= 0 \label{eq:aux1}
    % &I(\RX_a\MA \wedge \RX_b, \RX'_c|\RG_1) \notag\\
    % &\mkern 30mu= H(\RX_a\MA|\RG_1)- H(\RX_a\MA|\RG_1, \RX_b, \RX'_c) \notag \\ &\mkern 30mu= H(\RX_a\MA)-H(\RX_a\MA)=0, \label{eq:aux1}
\end{align}
Using the independence condition 2), we also obtain
\begin{align}
    &I(\RX_a, \RX'_c \wedge \RX_b \MB + \RX'_c \MC \mid \RG_1, \RX_a\MA) \notag\\ &=I(\RX_a, \RX'_c \wedge \RX_a\MA+\RX_b \MB + \RX'_c \MC \mid \RG_1, \RX_a\MA)\notag\\
    &=H(\RX_a\MA+\RX_b \MB + \RX'_c \MC \mid \RG_1, \RX_a\MA)%\notag\\ &\mkern 30mu 
    - H(\RX_a\MA+\RX_b \MB + \RX'_c \MC \mid \RG_1, \RX_a\MA,\RX_a, \RX'_c)\notag\\
    &=H(\RX_a\MA+\RX_b \MB + \RX'_c \MC \mid \RG_1, \RX_a\MA)%\notag\\ &\mkern 30mu  
    - H(\RX_a\MA+\RX_b \MB + \RX'_c \MC \mid \RG_1,\RX_a, \RX'_c)\notag\\
    &=H(\RX_a\MA+\RX_b \MB + \RX'_c \MC)%\notag\\ &\mkern 30mu 
    \ \textcolor{blue}{ - }\ H(\RX_a\MA+\RX_b \MB + \RX'_c \MC)\notag\\
    &=0.\label{eq:aux2}
\end{align}
It follows from \eqref{eq:final_z1} and \eqref{eq:final_z2} that
\begin{align*}
    `(1)&=I(\RZ_1\wedge \RZ_2|\RG_1)\\
        &=I(\RZ_1 \wedge \RZ_2, \RG_1 \mid \RG_1)\\
        &=I(\RX_a, \RX'_c, \RG_1 \wedge \RX_b, \RX'_c, \RG_1 \mid \RG_1)\\
        &=I(\RX_a, \RX'_c \wedge \RX_b, \RX'_c \mid \RG_1)\\
        &=I(\RX_a, \RX'_c, \RX_a\MA \wedge \RX_b, \RX'_c \mid \RG_1)\\
        &=I(\RX_a\MA \wedge \RX_b, \RX'_c \mid \RG_1)+I(\RX_a, \RX'_c \wedge \RX_b, \RX'_c \mid \RG_1, \RX_a\MA)\\
        &\stackrel{\eqref{eq:aux1}}{=}I(\RX_a, \RX'_c \wedge \RX_b, \RX'_c \mid \RG_1, \RX_a\MA)\\
        &=I(\RX_a, \RX'_c \wedge \RX_b, \RX'_c,\RX_b \MB + \RX'_c \MC \mid \RG_1, \RX_a\MA)\\
        &=I(\RX_a, \RX'_c \wedge \RX_b \MB + \RX'_c \MC \mid \RG_1, \RX_a\MA)\\
        &\mkern 30mu+I(\RX_a, \RX'_c \wedge \RX_b, \RX'_c\mid \RG_1, \RX_a\MA, \RX_b \MB + \RX'_c \MC)\\
        &\stackrel{\eqref{eq:aux2}}{=}I(\RX_a, \RX'_c \wedge \RX_b, \RX'_c \mid \RG_1, \RX_a\MA, \RX_b \MB + \RX'_c \MC)\\
        &=I(\RZ_1\wedge \RZ_2 \mid \RF')=`(3)
\end{align*}
 This completes the proof.

\end{proof}

