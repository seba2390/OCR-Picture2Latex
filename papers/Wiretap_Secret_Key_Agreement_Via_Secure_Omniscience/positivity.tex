In this section, we will use the inequality $H(\RZ_V|\RZ_{\opw}) - \wskc \leq \rl$ to establish an equivalent condition for the positivity of $\wskc$. This result extends the two-user result of \cite[Theorem~4]{amin2020} to the multiuser case. In the two-user setting \cite{amin2020}, Gohari, G\"{u}nl\"{u} and Kramer studied the positivity of $\wskc$, and gave an equivalent characterization in terms of R\'enyi divergence by using hypothesis testing and a coding scheme that involves repetition with block-swapping. This coding idea remains one of the main ingredients of our proof.

For two distributions $P_{\RX}$ and $P_{\tRX}$ on a common alphabet $\mc{X}$, the \emph{R\'enyi divergence of order $1/2$} between $P_{\RX}$ and $P_{\tRX}$ is given by $D_{\frac{1}{2}}(P_{\RX} \| P_{\tRX}):=-2\log\left( \sum_{x \in \mc{X}} \sqrt{P_{\RX}(x)P_{\tRX}(x)}\right),$
and the total variation (TV) distance between  $P_{\RX}$ and $P_{\tRX}$ is given by ${\|P_{\RX}-P_{\tRX}\|}_{\mathrm{TV}}:=\frac{1}{2}\sum_{x \in \mc{X}}|P_{\RX}(x)-P_{\tRX}(x)|$.
To state the theorem, let us define $\Delta(\RZ_V \| \RZ_{\opw}) := \inf {\|P_{\RK_1,\ldots, \RK_m,\RF^{(n)},\RZ^n_{\opw}}-\dfrac{1}{2}\mathbf{1}_{\RK_1=\dots=\RK_m}\cdot P_{\RF^{(n)},\RZ^n_{\opw}}\|}_{\mathrm{TV}}$ where the infimum is over all communication schemes and possible binary keys (see \cite[Def.~8]{amin2020}). 

\begin{theorem}\label{thm:multivariate_positivity}
 For a source $(\RZ_1,\ldots,\RZ_m,\RZ_{\opw})$ with distribution $P_{\RZ_1 \ldots \RZ_m \RZ_{\opw}}$ and $m \geq 2$, the following statements are equivalent:
 \begin{enumerate}
  \item There is an integer $r$ and non-empty disjoint sets $\mc{A}_{11},\mc{A}_{12} \subset \mc{Z}_1^r$, $\mc{A}_{21},\mc{A}_{22} \subset \mc{Z}_2^r$, $\ldots, \mc{A}_{m1},\mc{A}_{m2} \subset \mc{Z}_m^r$ such that
  \begin{align*}
       &D_{\frac{1}{2}} \left(P_{\RZ_{\opw}^r}(\cdot \mid \mc{E}_{1, 1, \ldots, 1})  
       %\right. \\& \mkern 100mu \left.
       \| P_{\RZ_{\opw}^r}(\cdot \mid \mc{E}_{2, 2, \ldots, 2}) \right)%\\ & 
       < \log \left(\frac{\Pr(\mc{E}_{1, \ldots, 1})\Pr(\mc{E}_{2, \ldots, 2})}{\sum \limits_{\substack{(j_1,\ldots,j_m)\\ \not\in \{(1,\ldots,1),(2,\ldots,2)\}}}\dfrac{\Pr(\mc{E}_{j_1, \ldots, j_m})\Pr(\mc{E}_{3-j_1,\ldots, 3-j_m})}{2}}\right)
 \end{align*}
 where  $\mc{E}_{j_1, \ldots, j_m}$ denotes the event $\RZ_1^r \in \mc{A}_{1j_1}, \ldots, \RZ_m^r \in \mc{A}_{mj_m}$ for $(j_1,\ldots,j_m) \in \{1, 2\}^m$.
 
%  \item There is an integer $s$, and a non-empty set $\mc{A}_{1}\times\ldots\times\mc{A}_{m}\subset \mc{Z}^s_{1}\times\ldots\times\mc{Z}^s_{m}$ such that 
%  \begin{align}
%      &H(\RZ^s_V|\RZ^s_{\opw},\RZ^s_V\in \mc{A}_{1}\times\ldots\times\mc{A}_{m})
%      \notag \\& \mkern 100mu  > \rco(\RZ^s_V|\RZ^s_V\in \mc{A}_{1}\times\ldots\times\mc{A}_{m})
%  \end{align}

% \item $\rl(\RZ_V||\RZ_{\opw}) < H(\RZ_V|\RZ_{\opw})$.
 \item $\wskc(\RZ_V \| \RZ_{\opw}) > 0$.
 \item $\Delta(\RZ_V \| \RZ_{\opw}) = 0$.
 \item $\Delta(\RZ_V \| \RZ_{\opw}) < \delta_1$ where $\delta_1$ is the smallest root of the equation $16\delta^2-(8+4\sqrt{2^{m-1}-1})\delta+1=0$. (It can be seen that $\delta_1$ is strictly positive for any $m\geq 2$.)
\end{enumerate}
\end{theorem}
% \begin{proof}
%  See Appendix~\ref{app:multivariate_positivity}.
% \end{proof}

{\color{blue}
\begin{IEEEproof}[Proof sketch] Most of the steps in the proof are analogous to the proof for the two-user case \cite[Theorem~4]{amin2020}. We highlight here the fact that we can use our lower bound \eqref{eq:RL:lb} on $\rl$ to prove the statement that 1) implies 2). The idea is to show that if 1) holds then for a transformed source  $(\tRZ_V,\tRZ_{\opw})$ obtained using the sets given in 1),
  \begin{align*} \label{eq:rco_nonmaximal}
     H(\tRZ^n_V \mid \tRZ^n_{\opw},\tRZ^n_V\in \mc{A}_{1}\times\cdots\times\mc{A}_{m})
    %\notag \\& \mkern 100mu  
      -\rco(\tRZ^n_V \mid \tRZ^n_V\in \mc{A}_{1}\times\cdots\times\mc{A}_{m})>0
 \end{align*}
 for some integer $n$, and a positive-probability event $\mc{E}=\big\{\tRZ^n_V\in \mc{A}_{1}\times\cdots\times\mc{A}_{m}\big\}$, where $\mc{A}_{i} \subset \{1,2\}^n$ for all $i \in V$. This step can be argued using the repetition coding with block swapping idea from \cite[Theorem~4]{amin2020}. By using the bound $\rco \geq \rl$ and  the lower bound \eqref{eq:RL:lb} on $\wskc$, we have
 \begin{align*}
    n \, \wskc(\tRZ_V\|\tRZ_{\opw}) & =\wskc(\tRZ^n_V\|\tRZ^n_{\opw})\\ & 
    \utag{a}{\geq} \, \Pr(\mc{E})\, \wskc(\tRZ^n_V\parallel\tRZ^n_{\opw},\tRZ^n_V\in \mc{A}_{1}\times\cdots\times\mc{A}_{m})\\
    &\geq\Pr(\mc{E})\left[H(\tRZ^n_V\mid\tRZ^n_{\opw},\tRZ^n_V\in \mc{A}_{1}\times\cdots\times\mc{A}_{m})
          -\rl(\tRZ^n_V\parallel\tRZ^n_{\opw},\tRZ^n_V\in \mc{A}_{1}\times\cdots\times\mc{A}_{m})\right]\\
          & \geq \Pr(\mc{E}) \left[H(\tRZ^n_V\mid\tRZ^n_{\opw},\tRZ^n_V\in \mc{A}_{1}\times\cdots\times\mc{A}_{m})
    %\notag \\& \mkern 100mu  
      -\rco(\tRZ^n_V \mid\tRZ^n_V\in \mc{A}_{1}\times\cdots\times\mc{A}_{m})\right]\\
      &>0,
 \end{align*}
 where \uref{a} follows from applying \cite[Lemma~3]{maurer93} to the event $\mc{E}$. The source transformation is done in such a way that $\wskc(\tRZ_V\|\tRZ_{\opw})>0$ implies $\wskc(\RZ_V\|\RZ_{\opw}) > 0$, hence condition 2) follows. The remaining steps are analogous to the two-user case, hence omitted.
\end{IEEEproof}

\medskip

We remark here that we presented the above multiuser extension of the two-user result of Gohari, G\"unl\"u and Kramer \cite[Theorem 4]{amin2020} only to illustrate the point that our inequality $H(\RZ_V|\RZ_{\opw}) - \wskc \leq \rl$ could have its uses beyond the equality case. We do not claim that this is the only way to establish the implication 1) $\Longrightarrow$ 2); indeed, other proofs may well be possible which do not use the notion of $\rl$. It would also be of interest to explore the use of Theorem~\ref{thm:multivariate_positivity} in establishing the positivity of $\wskc$ for some class of multiterminal sources $(\RZ_V, \RZ_{\opw})$, as was done in \cite{amin2020} for the case of erasure sources. However, such an exploration would take us well away from the theme of this paper, so we leave it for future work.

One of the ingredients used in the proof of Theorem~\ref{thm:multivariate_positivity} is a result of Maurer, \cite[Lemma~3]{maurer93}, 
that relates $\wskc$ of a source $(\RZ_V,\RZ_{\opw})$ to the source $(\hat{\RZ}_V,\RZ_{\opw})$ whose distribution is obtained by conditioning the distribution of $(\RZ_V,\RZ_{\opw})$ by a certain event. Formally, given some non-empty sets $\mc{A}_1 \subseteq \mc{Z}_1, \ldots, \mc{A}_m \subseteq \mc{Z}_m$, let $\mc{E}$ denote the event that $\RZ_V \in \mc{A}_{1}\times\cdots\times\mc{A}_{m}$ . Define a new source $(\hat{\RZ}_V,\RZ_{\opw})$ taking values in the same alphabets $\mc{Z}_1, \ldots, \mc{Z}_m$ and $\mc{Z}_{\opw}$ with the probability distribution
\begin{align}\label{eq:hatsource}
    P_{\hat{\RZ}_1 \ldots \hat{\RZ}_m \RZ_{\opw}}(z_1, \ldots, z_m, z_{\opw}) := \dfrac{P_{\RZ_1 \ldots \RZ_m, \RZ_{\opw}}(z_1, \ldots, z_m, z_{\opw})}{\Pr(\mc{E})}
\end{align}
if $(z_1, \ldots, z_m) \in \mc{A}_{1}\times\cdots\times\mc{A}_{m}$, and $P_{\hat{\RZ}_1 \ldots \hat{\RZ}_m \RZ_{\opw}}(z_1, \ldots, z_m, z_{\opw}) := 0$ otherwise.
It was shown in \cite[Lemma~3]{maurer93} that  $\wskc(\RZ_V\|\RZ_{\opw})\geq\Pr(\mc{E})\wskc(\hat{\RZ}_V\|\RZ_{\opw})$. The minimum leakage rate for omniscience $\rl$ also satisfies a similar relation.

\begin{lemma}\label{lem:hatsource}
For sources $(\RZ_V,\RZ_{\opw})$ and $(\hat{\RZ}_V,\RZ_{\opw})$, which is defined as above for some event $\mc{E}:=\{\RZ_V \in \mc{A}_{1}\times\cdots\times\mc{A}_{m}\}$, we have 
\begin{align}
    &H(\RZ_V|\RZ_{\opw})-\rl(\RZ_V\|\RZ_{\opw}) %\notag \\& \mkern 100mu 
    \geq \Pr(\mc{E})[H(\hat{\RZ}_V|\RZ_{\opw})-\rl(\hat{\RZ}_V\|\RZ_{\opw})]
\end{align}
\end{lemma}
\begin{proof}
Let $\hat{\RF}^{(n)}$ be an omniscience scheme for the source $\hat{\RZ}_V$ that achieves $\rl(\hat{\RZ}_V\|\RZ_{\opw})$. We will construct an omniscience scheme for the source $\RZ_V$ with the leakage rate $H(\RZ_V|\RZ_{\opw})-\Pr(\mc{E})[H(\hat{\RZ}_V|\RZ_{\opw})-\rl(\hat{\RZ}_V\|\RZ_{\opw})]$, which proves the lemma. Fix a large enough $n$ and consider $n$ i.i.d. realizations of the source  $(\RZ_V,\RZ_{\opw})$. In the first phase of communication, each user reveals publicly the indices of those realizations that fall in their corresponding set $\mcA_i$ to the other users. For instance, user $i$ transmits $\RF_{1,i}(\RZ_i^n):= (b_{ij} : b_{ij} = \mathbf{1}_{\mcA_i}(\RZ_{ij}), 1\leq j\leq n)$, which is a sequence indicating the locations where $\RZ_{ij} \in \mcA_i$. This communication involves $m$ message transmissions. At the end of the first phase, through $(\RF_{1,i}: i \in V)$, every user knows the indices where the event $\mcE$ has occurred, i.e., $\mcE$ occurs at an index $j$ if $b_{ij}=1$ for all $i \in V$.  

In the second phase of communication, users discuss interactively based on the first phase of communication. Let $\mcJ$ denote the set of indices for which $\mcE$ occurs. On the indices in $\mcJ^c$, users reveal their complete observations. For example, user $i$ communicates $\RF_{2,i}(\RZ_i^n, \RF_{1,1}, \ldots, \RF_{1,m}):= \RZ_{i,\mcJ^c}$. And, for the block corresponding to $\mcJ$, they communicate according to $\hat{\RF}^{(\mcJ)}$, which is in general interactive. And, the corresponding communication is $\RF_3:=\hat{\RF}^{(\mcJ)}(\RZ^n_V)$, which acts only on the block corresponding to $\mcJ$. Note that conditioning on a realization $\mcJ=J \subset [n]$, the distribution of  $(\RZ_{V,J},\RZ_{\opw,J})$ is the same as that of $|J|$ i.i.d. realizations of $(\hat{\RZ}_V,\RZ_{\opw})$. 

Let $(\RC_j:1\leq j\leq n)$ be a random sequence where $\RC_j = \mathbf{1}_{\mc{A}_{1}\times\cdots\times\mc{A}_{m}}(\RZ_{V,j})$, and observe that this is an i.i.d. sequence. Using the strong typicality of this sequence, it is easy to verify that the communication $\RF^{(n)}:= (\RF_{1,1}, \ldots, \RF_{1,m},\RF_{2,1}, \ldots, \RF_{2,m},\RF_3)$ satisfies the recoverability condition \eqref{eq:omn:recoverability} for omniscience.  The leakage rate is 
\begin{align*}
    \frac{1}{n} I(\RZ_V^{n} \wedge \RF^{(n)} \mid \RZ_{\opw}^n) = H(\RZ_V \mid \RZ_{\opw}) - \frac{1}{n}H(\RZ_V^{n} \mid \RF^{(n)}, \RZ_{\opw}^n).
\end{align*}

Consider the term $\frac{1}{n}H(\RZ_V^{n} \mid \RF^{(n)}, \RZ_{\opw}^n)$ which, from the form of the communication $\RF^{(n)}$, can be seen to be equal to $\frac{1}{n}H(\RZ_{V,\mcJ} \mid \mcJ, \hat{\RF}^{(\mcJ)}, \RZ_{\opw,\mcJ})$, which we write as $\frac{1}{n}[H(\RZ_{V,\mcJ} \mid \mcJ, \RZ_{\opw,\mcJ})-I(\RZ_{V,\mcJ} \wedge  \hat{\RF}^{(\mcJ)}\mid \mcJ, \RZ_{\opw,\mcJ})]$. The first of these terms can be readily single-letterized:
\begin{align*}
    \frac{1}{n}H(\RZ_{V,\mcJ} \mid \mcJ, \RZ_{\opw,\mcJ}) &= \frac{1}{n}\sum_{J \subseteq [n]} \Pr(\mcJ = J) H(\RZ_{V,J} | \mcJ = J, \RZ_{\opw,J}) \\
    &= \frac{1}{n}\sum_{J \subseteq [n]} \Pr(\mcJ = J) \, |J| \, H(\hat{\RZ}_V|\RZ_{\opw}) \\
    &= \mathbb{E}\left[\frac{|\mcJ|}{n}\right] \, H(\hat{\RZ}_V|\RZ_{\opw})  \ = \ \Pr(\mc{E}) \, H(\hat{\RZ}_V|\RZ_{\opw}).
\end{align*}
It can be easily shown that $\liminf_{n\to \infty}\frac{1}{n}H(\RZ_{V,\mcJ} \mid \mcJ, \hat{\RF}^{(\mcJ)}, \RZ_{\opw,\mcJ})  \geq  \Pr(\mc{E})[H(\hat{\RZ}_V|\RZ_{\opw})-\rl(\hat{\RZ}_V|\RZ_{\opw})]$ for an $\rl(\hat{\RZ}_V\|\RZ_{\opw})$-optimal omniscience scheme  $\hat{\RF}^{(n)}$.
Thus we have 
\begin{align*}
    \rl(\RZ_V\|\RZ_{\opw}) &\leq \limsup_{n \to \infty} \frac{1}{n} I(\RZ_V^{n} \wedge \RF^{(n)} \mid \RZ_{\opw}^n) \\ & \leq H(\RZ_V|\RZ_{\opw})%\\& \mkern 50mu
    -\Pr(\mc{E})[H(\hat{\RZ}_V|\RZ_{\opw})-\rl(\hat{\RZ}_V\|\RZ_{\opw})].
\end{align*}
This completes the proof.
\end{proof} 
}






