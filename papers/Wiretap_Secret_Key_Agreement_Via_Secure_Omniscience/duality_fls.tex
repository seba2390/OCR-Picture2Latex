In this section, we  consider  a  broad  class  of  sources,  namely, finite linear sources, for which we believe the duality between secure omniscience and wiretap secret key agreement must hold. 

\begin{definition}[Finite linear source \cite{chan11itw}]
A source $(\RZ_V, \RZ_{\opw})$ is said to be a \textit{finite linear source (FLS)} if we can express $\RZ_V$ and $\RZ_{\opw}$ as
$$\bM \RZ_V & \RZ_{\opw}\eM=\bM \RZ_1 & \cdots& \RZ_m & \RZ_{\opw}\eM=\RX \bM \MM_1\;\cdots\;\MM_m \; \MW \eM,$$
where  $\RX$ is a random row vector of some length $l$ that is uniformly distributed over a field $\Fq^l$, and $\MM_1, \ldots,\MM_m, \MW$ are some matrices over $\Fq$ with dimensions $l \times l_1, \ldots ,l \times l_m, l \times l_w$, respectively. Each terminal observes a collection of linear combinations of the entries in $\RX$. 
\end{definition} 

In the context of FLS models, we say a communication scheme $\RF^{(n)}$ is \emph{linear} if each user's communication is a linear function of its observations and the previous communication on the channel. Without loss of generality \cite[Sec.~II]{chan19oneshot}, linear communication can be assumed to be non-interactive.  In the rest of the paper, we consider only matrices over $\Fq$ unless otherwise specified.

The following notions related to G\'{a}cs-K\"{o}rner common information will play an important role in proving some of our subsequent results. The \emph{ G\'{a}cs-K\"{o}rner common information} of  $\RX$ and $\RY$ with joint distribution $P_{\RX,\RY}$ is defined as
\begin{align}\label{eq:gk}
 J_{\op{GK}}(\RX \wedge \RY) := \max \left\lbrace H(\RG) : H(\RG|\RX)=H(\RG|\RY) =0 \right\rbrace
\end{align}
A $\RG$  that satisfies the constraint in \eqref{eq:gk} is called a common function (c.f.) of $\RX$ and $\RY$. An optimal $\RG$ in \eqref{eq:gk} is called a \emph{maximal common function} (m.c.f.) of $\RX$ and $\RY$, and is denoted by $\op{mcf}(\RX, \RY)$. Similarly, for $m$ random variables, $\RX_1, \RX_2, \ldots, \RX_m$,  we can extend these definitions by replacing the condition in \eqref{eq:gk} with $H(\RG|\RX_1)=H(\RG|\RX_2)=\ldots=H(\RG|\RX_n)=0$. For a two-user FLS $(\RZ_1, \RZ_2)$, i.e., $\RZ_1 = \RX \MM_1$ and $\RZ_2=\RX \MM_2$ for some matrices $\MM_1$ and $\MM_2$ where $\RX$ is a $1 \times l$ row vector   uniformly distributed on $\Fq^l$, it was shown in \cite{chan18zero} that the $\op{mcf}(\RZ_1, \RZ_2)$ is a linear function of each of $\RZ_1$ and $\RZ_2$. This means that there exists some matrices $\MM_{z_1}$ and $\MM_{z_2}$ such that $\op{mcf}(\RZ_1, \RZ_2) = \RZ_1 \MM_{z_1}=\RZ_2 \MM_{z_2}$. One can infer from this relation that if $\RZ_1$ and $\RZ_2$ are independent, then $\op{mcf}(\RZ_1, \RZ_2)$ is identically $0$.

We prove results in this and the next section favoring the following conjecture.
%we address the question: For what sources, secure omniscience achieves wiretap secret key capacity? In other words, does $$\rl = H(\RZ_V|\RZ_{\opw}) - \wskc $$ hold for a large enough class of sources?\\
\begin{Conjecture}\label{conj:duality:fls}
$\rl = H(\RZ_V|\RZ_{\opw}) - \wskc$ holds for finite linear sources.
\end{Conjecture}

The reason to believe Conjecture~\ref{conj:duality:fls} comes from the following two theorems. Since the source is linear, it is reasonable to conjecture that linear schemes are optimal. Theorem~\ref{thm:linearscheme} below states that if a linear \emph{perfect SKA scheme} is optimal in terms of $\wskc$, then secure omniscience achieves wiretap secret key capacity. Here, we call an SKA scheme \emph{perfect} if there exists a sequence of communication-key pairs $(\RF^{(n)}, \RK^{(n)})_{n\geq1}$ such that 
$H(\RK^{(n)}|\RF^{(n)},\RZ_i^n) = 0$ for all users $i \in V$ (perfect key recoverability condition), and $\log |\mc{K}^{(n)}| = H(\RK^{(n)}| \RF^{(n)},\RZ_{\opw}^n)$ (perfect secrecy condition).

\begin{theorem}\label{thm:linearscheme}
  For an FLS $(\RZ_V,\RZ_{\opw})$, if a linear perfect SKA scheme achieves $\wskc$, then we have
  $$\rl = H(\RZ_V|\RZ_{\opw}) - \wskc.$$
\end{theorem}
\begin{proof}
See Appendix~\ref{app:thm:proof_linearscheme}.
\end{proof}


The next theorem shows the duality between secure omniscience and wiretap secret key agreement for two-user FLS without any restriction to linear schemes. It also provides single-letter expressions for $\rl$ and $\wskc$.

\begin{theorem}[Two-user FLS]
  \label{thm:fls}
  For secure omniscience with $V=\Set{1,2}$ and FLS $(\RZ_V,\RZ_{\opw})$, we have
  \begin{align}
    R_{\opL} &= H(\RZ_1,\RZ_2|\RZ_{\opw}) - \wskc,\\
    \wskc& = I(\RZ_1\wedge \RZ_2|\RG),\label{eq:fls}
  \end{align}
  where $\RG$ can be chosen to be $\RG_1$, $\RG_2$, or $(\RG_1,\RG_2)$, with $\RG_i$ being the solution to 
  \begin{align}
    J_{\op{GK}}(\RZ_{\opw}\wedge \RZ_i) := \max_{\RG_i: H(\RG_i|\RZ_{\opw})=H(\RG_i|\RZ_i)=0} H(\RG_i) \label{eq:JGK}
  \end{align}
  for $i\in V$.
\end{theorem}
\begin{proof}
See Appendix~\ref{app:thm:fls}.
\end{proof}

The following example compares the result of Theorem~\ref{thm:fls} with that of Csisz\'ar and Narayan~\cite{csiszar04} for the case of  two-user FLSs. Recall from Section~\ref{sec:wska:def} that for general sources, $\skc(\RZ_V) = H(\RZ_V) - \rco(\RZ_V)$ and $\pkc(\RZ_V | \RZ_{\opw}) = H(\RZ_V|\RZ_{\opw}) - \rco(\RZ_V|\RZ_{\opw})$; it also holds that $\wskc(\RZ_V \| \RZ_{\opw}) \leq \min\{\skc(\RZ_V),\pkc(\RZ_V | \RZ_{\opw})\}$, and $\rco(\RZ_V|\RZ_{\opw})\leq \rl(\RZ_V \| \RZ_{\opw}) \leq \rco(\RZ_V)$. For the source considered in the next example, we see that all these inequalities are strict.
\begin{example}
Let $V=\{1, 2\}$, and let $\RX= (\RX_{a}, \RX_{b}, \RX_{c}, \RX_{d})$ be a random vector uniformly distributed over $\mathbb{F}_2^4$. Consider the two-user FLS  $(\RZ_V, \RZ_{\opw})$ with
\begin{gather*}
   \RZ_1= (\RX_a, \RX_b, \RX_c)  \quad \RZ_2= (\RX_b, \RX_c, \RX_d) \quad \RZ_{\opw}= (\RX_b+\RX_c, \RX_a+\RX_d).
\end{gather*}
For this source, $\RG_1=\RG_2 = \RG=\RX_b+\RX_c$.
It follows from Theorem~\ref{thm:fls} that $\wskc(\RZ_V \| \RZ_{\opw})= I(\RZ_1 \wedge \RZ_2|\RG)= H(\RZ_1|\RG)-H(\RZ_1|\RG, \RZ_2)= H(\RX_a, \RX_b, \RX_c|\RX_b+\RX_c)-H(\RX_a, \RX_b, \RX_c|\RX_b+\RX_c, \RX_b, \RX_c, \RX_d)= 2-1= 1 \text{ bit}$. On the other hand, the secret key capacity of this source is $\skc(\RZ_V)=I(\RZ_1 \wedge \RZ_2)=H(\RZ_1)-H(\RZ_1|\RZ_2)=3-1=2 \text{ bits}$; and the private key capacity is $\pkc(\RZ_V|\RZ_{\opw})=I(\RZ_1 \wedge \RZ_2|\RZ_{\opw})=H(\RZ_1|\RZ_{\opw})-H(\RZ_1|\RZ_2,\RZ_{\opw})=2-0=2 \text{ bits}$. Observe that $$1=\wskc(\RZ_V \| \RZ_{\opw})< \min \{\skc(\RZ_V), \pkc(\RZ_V|\RZ_{\opw})\}=2.$$

Similarly, using the duality in Theorem~\ref{thm:fls} and the results of \cite{csiszar04}, we get $\rl (\RZ_V\|\RZ_{\opw})= H(\RZ_1,\RZ_2|\RZ_{\opw}) - \wskc(\RZ_V\|\RZ_{\opw})=2-1=1\text{ bit}$, $\rco(\RZ_V) = H(\RZ_1,\RZ_2) - \skc(\RZ_V)=4-2=2\text{ bits}$, and $\rco(\RZ_V|\RZ_{\opw}) = H(\RZ_1,\RZ_2|\RZ_{\opw}) - \pkc(\RZ_V|\RZ_{\opw}) = 2-2 =0 $. Thus, we have $\rco(\RZ_V|\RZ_{\opw})< \rl(\RZ_V\|\RZ_{\opw}) < \rco(\RZ_V)$.


\end{example}

In the next section, we prove the duality between secure omniscience and wiretap secret key agreement for tree-PIN sources with linear wiretapper, a sub-class of FLSs. We also give single-letter expressions for $\rl$ and $\wskc$ for this model.
