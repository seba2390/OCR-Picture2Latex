It suffices to show that $\wskc$ can be achieved through omniscience because then
\begin{align*}
    nH(\RZ_V|\RZ_{\opw})&\geq I(\RK^{(n)},\RF^{(n)} \wedge \RZ_V^n \mid \RZ_{\opw}^n)\\
                        & = I(\RF^{(n)} \wedge \RZ_V^n \mid \RZ_{\opw}^n)+I(\RK^{(n)} \wedge \RZ_V^n \mid \RZ_{\opw}^n,\RF^{(n)})\\
                        & \geq I(\RF^{(n)} \wedge \RZ_V^n \mid \RZ_{\opw}^n) + n(\wskc-\delta_n) 
\end{align*}
for some $\delta_n \to 0$, where the last inequality follows from the fact an optimal key is recoverable from $\RZ_V^n$. By taking limsup on both side of the above inequality after normalizing by $n$, we get $ H(\RZ_V|\RZ_{\opw})\geq \limsup_{n \to \infty}\frac{1}{n}I(\RF^{(n)} \wedge \RZ_V^n \mid \RZ_{\opw}^n)  - \wskc \geq \rl- \wskc$. Therefore, $\rl \leq H(\RZ_V|\RZ_{\opw}) -\wskc$.



Let $(\RF^{(n)}, \RK^{(n)})$ be a communication-key pair of a linear perfect SKA scheme that achieves $\wskc$, but $\RF^{(n)}$ need not achieve omniscience. By  \cite[Theorem~1]{chan19oneshot}, we can assume that $\RF^{(n)}$ is a linear function of $\RZ_V^n$ alone (additional randomization by any user is not needed) and the key is also a linear function of $\RZ_V^n$. 

If $\RF^{(n)}$ already attains omniscience, then we are done. If not, for some $i,j \in V$, $i \ne j$, we have a component $\RX \in \Fq$ of random vector $\RZ_i^{n}$ such that
\begin{align*}
    H(\RX \mid \RF^{(n)}, \RZ_j^n) &\neq 0.
\end{align*}
We will show that there exists an additional discussion $\RF'^{(n)}$ such that
\begin{align}
    H(\RX \mid \RF^{(n)},\RF'^{(n)}, \RZ_j^n) &= 0 \label{eq:additional_recov}
\end{align}  and 
\begin{align}
 I(\RK^{(n)} \wedge \RF^{(n)},\RF'^{(n)}, \RZ_{\opw}^n) = 0.\label{eq:additional_secrecy}
\end{align}
If $(\RF^{(n)},\RF'^{(n)})$ achieves omniscience, we are done; else, we repeat the construction in our argument till we obtain the desired omniscience-achieving communication.

So, consider the non-trivial case where $H(\RX\mid \RF^{(n)}, \RZ_j^n) \neq 0$ and $I(\RK^{(n)} \wedge \RF^{(n)}, \RX, \RZ_{\opw}^n) \neq 0$. (If $I(\RK^{(n)} \wedge \RF^{(n)}, \RX, \RZ_{\opw}^n)=0$, then user $i$ transmits $\RF'^{(n)}:= \RX$ which satisfies \eqref{eq:additional_recov} and \eqref{eq:additional_secrecy}.)
%(Since $\RK^{(n)}$ must be recoverable from $(\RF^{(n)}, \RZ_i^n)$, the alternative assumption $I(\RK^{(n)} \wedge \RF^{(n)}, \RZ_i^{n}, \RZ_{\opw}^n) = 0$ yields $H(\RK^{(n)}) = 0.$) 
Let $\RL^{(n)}$ be a common linear function, not identically $0$, of $\RK^{(n)}$ and $(\RF^{(n)}, \RX, \RZ_{\opw}^n))$ taking values in $\Fq$. Such a function exists since $I(\RK^{(n)} \wedge \RF^{(n)}, \RX, \RZ_{\opw}^n) \neq 0$. So, we can write \begin{align}
    \RL^{(n)}=\RK^{(n)}\MM_K= a\RX + \RF^{(n)}\MM_{F}+\RZ_{\opw}^n \MM_{\opw} \label{eq:mcf_lin_perfect}
\end{align}
for some non-zero element $a \in \Fq$, and some  column vectors $\MM_K \neq \M0, \MM_{F},$ and $\MM_{\opw}$ over $\Fq$. (Here, $\RL^{(n)},\RK^{(n)},\RF^{(n)}$ and $\RZ_{\opw}^n$ are the random row vectors with entries uniformly distributed over $\Fq$.) Note  the coefficient $a$ in the above linear combination must be a non-zero element in $\Fq$. If not, then $\RL^{(n)}(=\RK^{(n)}\MM_K= \RF^{(n)}\MM_{F}+\RZ_{\opw}^n \MM_{\opw})$ is a non-constant common function of $\RK^{(n)}$ and $(\RF^{(n)}, \RZ_{\opw}^n)$. This contradicts the secrecy condition $I(\RK^{(n)} \wedge \RF^{(n)},  \RZ_{\opw}^n) = 0$.

% Then, $\RG^{(n)}:= \op{mcf}(\RK^{(n)}, (\RF^{(n)}, \RX, \RZ_{\opw}^n))$ is a non-constant linear function.  Without of loss of generality, we can write \begin{align}
%     \RG^{(n)}=\RK^{(n)}\MM_K= a\RX + \RF^{(n)}\MM_{F}+\RZ_{\opw}^n \MM_{\opw} \label{eq:mcf_lin_perfect}
% \end{align}
% for some non-zero element $a \in \Fq$, and some  column vectors $\MM_K \neq \M0, \MM_{F},$ and $\MM_{\opw}$ over $\Fq$. (Here, $\RG^{(n)},\RK^{(n)},\RF^{(n)}$ and $\RZ_{\opw}^n$ are the random row vectors with entries uniformly distributed over $\Fq$.) It means that $\RG^{(n)}$ has only one component. Suppose that it has more than two components (independent), then  $\RG^{(n)}=\RK^{(n)}\tMM_K= \RX \Ma + \RF^{(n)}\tMM_{F}+\RZ_{\opw}^n \tMM_{\opw}$ for some some full column-rank matrices $\tMM_K$ and $\bM \Ma^T & \tMM_{F}^T & \tMM_{\opw}^T\eM^T$ over $\Fq$.
% Let $\RG^{(n)}(l) = \RK^{(n)}\tMM_K(l)= \Ma(l) \RX + \RF^{(n)}\tMM_{F}(l)+\RZ_{\opw}^n \tMM_{\opw}(l)$ denote the  entry corresponding to location $l$ in the random vector $\RG^{(n)}$. Here, $\tMM_K(l), \Ma(l),  \tMM_{F}(l)$ and $\tMM_{\opw}(l)$ denote the column vectors at location $l$ in their corresponding matrices. Firstly, note for any $l$, the coefficient $\Ma(l) \in \Fq $ in the above linear combination must be non-zero. If not, $\RG^{(n)}(l)(=\RK^{(n)}\tMM_K(l)= \RF^{(n)}\tMM_{F}(l)+\RZ_{\opw}^n \tMM_{\opw}(l))$ is a non-constant common function of $\RK^{(n)}$ and $(\RF^{(n)}, \RZ_{\opw}^n)$. This contradicts the independence between $\RK^{(n)}$ and $(\RF^{(n)}, \RZ_{\opw}^n)$ (or the secrecy condition of the key , i.e., $I(\RK^{(n)} \wedge \RF^{(n)},  \RZ_{\opw}^n) = 0$). Secondly, consider the entries $\RG^{(n)}(l)$ and $\RG^{(n)}(k)$ for $l \neq k$. Since they are independent, $a(l)^{-1}\RG^{(n)}(l)- a(k)^{-1}\RG^{(n)}(k)= \RK^{(n)} [a(l)^{-1}\tMM_K(l)- a(k)^{-1}\tMM_K(k)]= \RF^{(n)}[a(l)^{-1}\tMM_{F}(l)- a(k)^{-1}\tMM_{F}(k)]+\RZ_{\opw}^n [a(l)^{-1}\tMM_{\opw}(l)- a(k)^{-1}\tMM_{\opw}(k)]$ is a non-constant function. This is also a common function of $\RK^{(n)}$ and $(\RF^{(n)}, \RZ_{\opw}^n)$, contradicting the secrecy condition. Therefore,  $\RG^{(n)}$ has only one component  justifying the form \eqref{eq:mcf_lin_perfect}.

% Let $\RZ'_i^{(n)}:= \RZ_i^n\MM_i$. There exists a $\RZ''_i^{(n)}$, possibly identically $0$, which is a linear function of $\RZ_i^n$ such that 
% \begin{align}
%     I(\RZ'_i^{(n)} \wedge \RZ''_i^{(n)}) = 0 \ \text{ and } \ H(\RZ_i^n\mid\RZ'_i^{(n)},\RZ''_i^{(n)})=0. \label{eq:additional_lineardecomp}
% \end{align}
% In addition, $\RZ''_i^{(n)}$ satisfies the independence relation
% \begin{align}
%     I(\RK^{(n)} \wedge \RZ''_i^{(n)},\RF^{(n)}, \RZ_{\opw}^n)=0,\label{eq:additional_lineardecomp_keyind}
% \end{align}
% since otherwise the maximality of the m.c.f. $\RG^{(n)}$ is violated, as a consequence of \eqref{eq:additional_lineardecomp} and $I(\RK^{(n)} \wedge \RF^{(n)}, \RZ_{\opw}^n) = 0$ (secrecy condition for perfect SKA). Suppose $I(\RK^{(n)} \wedge \RZ''_i^{(n)},\RF^{(n)}, \RZ_{\opw}^n)\neq 0$, then $\RG'^{(n)}:= \op{mcf}(\RK^{(n)},(\RZ''_i^{(n)},\RF^{(n)}, \RZ_{\opw}^n))$ is a non-constant function. Observe that $\RG'^{(n)}$ is a common function of $\RK^{(n)}$ and $(\RF^{(n)}, \RZ_i^n, \RZ_{\opw}^n)$. By the fact that any common function is a function of m.c.f., we have $\RG'^{(n)}=\RG^{(n)}\MA$ for some matrix $\MA$. Let us show that $\RG'^{(n)}$ is a only a function of $(\RF^{(n)}, \RZ_{\opw}^n)$ . This implies that $I(\RG'^{(n)} \wedge \RF^{(n)}, \RZ_{\opw}^n) = 0$ (contradiction to the assumption that $\RG'^{(n)}$ is a non-constant function) because $\RG'^{(n)}$ is a function of $\RK^{(n)}$, and $I(\RK^{(n)} \wedge \RF^{(n)}, \RZ_{\opw}^n) = 0$. 


Define $\RF'^{(n)}:= \RK^{(n)}\MM_K - a\RX$. User $i$ can compute $\RF'^{(n)}$, as it is a function of  $\RK^{(n)}$ and $\RZ_i^n$, and transmit it publicly. Let us verify that $\RF'^{(n)}$ satisfies \eqref{eq:additional_recov} and \eqref{eq:additional_secrecy}. For \eqref{eq:additional_recov}, observe that  $H(\RX\mid \RF^{(n)},\RF'^{(n)}, \RZ_j^n) \leq H(\RX\mid \RF'^{(n)}, \RK^{(n)})=0$, the inequality following from $H(\RK^{(n)}\mid\RF^{(n)},\RZ_j^n)=0$, and the  equality from the fact that $\RX$ is recoverable from $(\RF'^{(n)}, \RK^{(n)})$. For \eqref{eq:additional_secrecy}, $I(\RK^{(n)}\wedge \RF^{(n)},\RF'^{(n)}, \RZ_{\opw}^n)= I(\RK^{(n)}\wedge \RF^{(n)}, \RZ_{\opw}^n) = 0$,  the first equality being a consequence of $\RF'^{(n)}$ also being expressible as $\RF^{(n)}\MM_{F}+\RZ_{\opw}^n \MM_{\opw}$, and the last equality from the secrecy condition of the key, i.e., $I(\RK^{(n)} \wedge \RF^{(n)},  \RZ_{\opw}^n) = 0$. This completes the proof.



% Consider the non-trivial case where $\RF^{(n)}$ does not attain omniscience. There is then a linear function of the source corresponding to a user $i$ that is not recoverable by a user $j \neq i$ through  $\RF^{(n)}$. In other words, for some $i \in V$, there exists a linear function, $\tRZ_i^{(n)}$, of $\RZ_i^n$ such that 
% \begin{align*}
%     H(\tRZ_i^{(n)}\mid \RF^{(n)}, \RZ_j^n) &\neq 0 \text{ for some } j \neq i.
% \end{align*}
% We will show that there exists an additional discussion $\RF'^{(n)}$ such that
% \begin{align}
%     H(\tRZ_i^{(n)}\mid \RF^{(n)},\RF'^{(n)}, \RZ_j^n) &= 0 \text{ for } j \in V \label{eq:additional_recov}
% \end{align}  and 
% \begin{align}
%  I(\RK^{(n)} \wedge \RF^{(n)},\RF'^{(n)}, \RZ_{\opw}^n) = 0.\label{eq:additional_secrecy}
% \end{align}
% Suppose $I(\RK^{(n)} \wedge \RF^{(n)}, \tRZ_i^{(n)}, \RZ_{\opw}^n) = 0$, then we can set $\RF'^{(n)}:=\tRZ_i^{(n)}$, which is communicated by the user $i$. Note that this choice of $\RF'^{(n)}$ satisfies \eqref{eq:additional_recov} and \eqref{eq:additional_secrecy}. Now consider the non-trivial case where $I(\RK^{(n)} \wedge \RF^{(n)}, \tRZ_i^{(n)}, \RZ_{\opw}^n) \neq 0$, i.e., $\RK^{(n)}$ is not independent of $(\RF^{(n)}, \tRZ_i^{(n)}, \RZ_{\opw}^n)$. Then, $\RG^{(n)}:= \op{mcf}(\RK^{(n)}, (\RF^{(n)}, \tRZ_i^{(n)}, \RZ_{\opw}^n))$ is a non-constant linear function.  So, we can write $\RG^{(n)}=\RK^{(n)}\MM_K= \tRZ_i^{(n)}\MM_{\tilde{Z}}+ \RF^{(n)}\MM_{F}+\RZ_{\opw}^n \MM_{Z_w}$  for some full column-rank matrices $\MM_K$ and $[\MM_{\tilde{Z}}^T \; \MM_{F}^T \; \MM_{Z_w}^T]^T$ over $\Fq^n$. Let $\RZ'_i^{(n)}:= \tRZ_i^{(n)}\MM_{\tilde{Z}}$. There exists a $\RZ''_i^{(n)}$, which may be identically $0$, which is a linear function of $\tRZ_i^{(n)}$ such that 
% \begin{align}
%     I(\RZ'_i^{(n)} \wedge \RZ''_i^{(n)})=H(\tRZ_i^{(n)}\mid\RZ'_i^{(n)},\RZ''_i^{(n)})=0. \label{eq:additional_lineardecomp}
% \end{align}
% In addition, $\RZ''_i^{(n)}$ satisfies the independence relation
% \begin{align}
%     I(\RK^{(n)} \wedge \RZ''_i^{(n)},\RF^{(n)}, \RZ_{\opw}^n)=0,\label{eq:additional_lineardecomp_keyind}
% \end{align}
% since otherwise $\RG^{(n)}$ violates the maximality of an m.c.f., as a consequence of \eqref{eq:additional_lineardecomp} and $I(\RK^{(n)} \wedge \RF^{(n)}, \RZ_{\opw}^n) = 0$ (secrecy condition for the perfect SKA).
