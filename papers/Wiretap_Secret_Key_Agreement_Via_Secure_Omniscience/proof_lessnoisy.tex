 First let us show that $\RZ$ is more capable than $\RY$ iff $\epsilon \leq h(p)$. Let $P_{\RX}=(P_{\RX}(0), P_{\RX}(1)):=(q, 1-q)$, and $I(\RX\wedge\RZ) - I(\RX\wedge\RY) = (1-\epsilon)h(q) - h(p*q) +h(p) \triangleq f(q)$ where $q \in [0,1]$ and $p*q=p(1-q)+(1-p)q$. If $\epsilon > h(p)$, then $f(\frac{1}{2})= h(p) - \epsilon < 0$ which implies that $\RZ$ is not more capable than $\RY$. For the other direction, we can use the convexity \cite[Lemma~2]{wynergerber} of the function $h(p*h^{-1}(x))$, $0\leq x \leq 1$. For $x_1=0$, $x_2=1$ and $\lambda= 1- h(q)$, we have $(1-h(q))h(p)+h(q)=\lambda h(p*h^{-1}(x_1))+(1-\lambda)h(p*h^{-1}(x_2))\geq h(p*h^{-1}(\lambda x_1 +(1-\lambda)x_2)) =  h(p*h^{-1}(h(q)))=h(p*\min\{q, 1-q\})= h(p*q)$.  Hence if $\epsilon \leq h(p)$ then $f(q) \geq (1-h(p))h(q) - h(p*q) +h(p) \geq 0$, where the last inequality uses the convexity result. This proves that $\RZ$ is more capable than $\RY$.
 
 Next we will show that $\RZ$ is less noisy than $\RY$ iff $\epsilon \leq 4p(1-p)$. Proving that $\RZ$ is less noisy than $\RY$ is equivalent to proving the concavity of  $I(\RX\wedge\RZ) - I(\RX\wedge\RY)$ with respect to the input distribution $P_{\RX}$ \cite[p.~126]{elgamalbook}. This follows from the observation that for $\RU-\RX-(\RY, \RZ)$,   $I(\RU\wedge\RZ) - I(\RU\wedge\RY) = I(\RX\wedge\RZ) - I(\RX\wedge\RY) - \big[I(\RX\wedge\RZ|\RU) - I(\RX\wedge\RY| \RU)\big]$. So it is enough to prove that $f(q), 0 \leq q \leq 1,$ is concave iff $\epsilon \leq 4p(1-p)$. The first and second derivatives are $f'(q)=(1-\epsilon)h'(q) - (1-2p)h'(p*q)$ and $f''(q)=(1-\epsilon)h''(q) - (1-2p)^2h''(p*q)$ for $0 \leq q \leq 1$. If $\epsilon \leq 4p(1-p)$ then $(1-\epsilon) \geq (1-2p)^2$ and $f''(q)\leq (1-\epsilon)[h''(q) - h''(p*q)]$. The function $h''(x)= \frac{1}{\ln 2}\left(\frac{-1}{x(1-x)}\right)$ is strictly increasing in the interval $(0,\frac{1}{2}]$, and is strictly decreasing in the interval $[\frac{1}{2}, 1)$. Since $p \in(0, \frac{1}{2})$, $q < p*q < \frac{1}{2}$ if $q < \frac{1}{2}$, $  \frac{1}{2}< p*q < q$ if $q > \frac{1}{2}$, and $p*q = q$  if $q=\frac{1}{2}$. In all these cases, $h''(q) - h''(p*q)\leq 0$ which implies that $f(q)$ is concave.  If $\epsilon > 4p(1-p)$ then $f''(\frac{1}{2})= \frac{1}{\ln 2}[-(1-\epsilon)+(1-2p)^2] >0 $, and hence $f(q)$ is not concave. 
 Since $f'(\frac{1}{2}) = 0$ and $f''(\frac{1}{2})>0$, $f(q)$, which is a symmetric function around $q=\frac{1}{2}$, is also strictly convex in a neighbourhood of $q=\frac{1}{2}$. Therefore, we have $f(\frac{1}{2})< \frac{1}{2}f(\frac{1}{2}+\delta)+\frac{1}{2}f(\frac{1}{2}-\delta)$ for some $\delta>0$. Let $\RU$ be a binary random variable with distribution Ber$(\frac{1}{2})$ and $\RX$ is obtained by passing $\RU$ through BSC$(\frac{1}{2}-\delta)$. For this choice of $P^*_{\RU,\RX}$, $I(\RU\wedge\RZ) - I(\RU\wedge\RY)=I(\RX\wedge\RZ) - I(\RX\wedge\RY) - \big[I(\RX\wedge\RZ|\RU) - I(\RX\wedge\RY| \RU)\big] = f(\frac{1}{2})- \big[\frac{1}{2}f(\frac{1}{2}+\delta)+\frac{1}{2}f(\frac{1}{2}-\delta)\big]< 0$. 

