In this section, we address the question of whether there is always a duality between secure omniscience and wiretap secret key agreement for any multiterminal source model with wiretapper. We study this by considering a necessary condition for duality, which is $\wskc > 0$ iff $\rl < H(\RZ_V|\RZ_{\opw})$. One direction, namely, that $\rl < H(\RZ_V|\RZ_{\opw})$ implies $\wskc > 0$ holds for any  source follows from \eqref{eq:RL:lb}. For the other direction, intuitively, if the users can generate a secret key that is independent of the wiretapper's side information, then they can use this advantage to protect some information during an omniscience scheme. However, we will prove that this need not be the case if we limit the number of messages exchanged between the users. 

%Now we will address the question: Is the ability to generate a positive secret key rate equivalent to that the users can achieve omniscience by strictly protecting a part of the source? More precisely, does the statement $\wskc > 0$ iff $\rl < H(\RZ_V|\RZ_{\opw})$ hold? One direction that $\rl < H(\RZ_V|\RZ_{\opw})$ implies $\wskc > 0$ follows from the lower bound \eqref{eq:RL:lb} which uses the idea of privacy amplification of the recovered source.  

To illustrate this result, let us consider a two-user setting ($m=2$) with source distribution $P_{\RZ_1\RZ_2\RZ_{\opw}}$. Let $r$ be the number of messages exchanged between the users, and let $\wskc^{(r)}$ and $\rl^{(r)}$ denote the wiretap secret key capacity and the minimum leakage rate for omniscience, respectively, when we allow at most $r$ messages to be exchanged among the users. Note that we can ensure omniscience only if we allow $r \geq 2$ because omniscience is not guaranteed with one message transmission. Moreover, omniscience can be obtained using a non-interactive communication that involves only $2$ messages.  Here $\rl^{(r)}<H(\RZ_1,\RZ_2|\RZ_{\opw})$ implies $\wskc^{(r)}>0$, because if the users can achieve omniscience using $r$ messages such that $\rl^{(r)}<H(\RZ_1,\RZ_2|\RZ_{\opw})$, then they can apply privacy amplification to recover a key with positive rate implying $\wskc^{(r)} > 0$. For the other direction, we show that $\wskc^{(r)}>0$ does not imply $\rl^{(r)}<H(\RZ_1,\RZ_2|\RZ_{\opw})$ if $r=2$. This is stated in the following proposition.

\begin{proposition} \label{prop:positivity} 

If $r=2$, then for any source $P_{\RZ_1\RZ_2\RZ_{\opw}}$,
    $$\rl^{(r)}<H(\RZ_1,\RZ_2|\RZ_{\opw}) \Longrightarrow \wskc^{(r)}>0.$$
    However, the converse need not hold. %In particular, for the source given in Lemma~\ref{lem:twowaycounter}, $\wskc^{(r)}> 0$ but  $\rl^{(r)}=H(\RZ_1,\RZ_2|\RZ_{\opw})$.

    % \item If $r\geq 3$, then for any source $P_{\RZ_1\RZ_2\RZ_{\opw}}$,
    % $$\rl^{(r)}<H(\RZ_1,\RZ_2|\RZ_{\opw}) \iff \wskc^{(r)}>0.$$

\end{proposition}

%The above proposition hints that this should be the case even with an arbitrary but fixed number of messages and the unlimited number of messages. 

%The following theorem summarizes this result for two user case on the interplay between the positivity of secret key capacity and the non-maximality of minimum leakage rate for omniscience.

For the converse part, we first derive an upper bound on $\rl^{(2)}$ using the results from the one-way communication setting. We then give a source in Lemma~\ref{lem:twowaycounter} that serves as a counterexample to illustrate that the converse does not hold in general. In the rest of this section, we denote $\RZ_1,\RZ_2$ and $\RZ_{\opw}$ by $\RX, \RY$, and $\RZ$, respectively. The random variables $\RX, \RY$, and $\RZ$ take values in  finite sets $\mc{X}$, $\mc{Y}$, and $\mc{Z}$, respectively.

\subsection{One-way communication, i.e., $r=1$}
Before we address the problem completely, first, we consider a model with only one message allowed. Since omniscience requires a minimum of two messages between users, we slightly modify the setup by letting only one of the users recover the other user's observations\textemdash see Fig. \ref{fig:oneway}. We define the minimum leakage rate for recovery of $\RX$ by user $2$ as 
$$\rl^{\text{ow}}:= \inf  \biggl\lbrace \limsup_{n \to \infty} \frac{1}{n}I(\RF_1^{(n)} \wedge \RX^n|\RZ^n) \biggr\rbrace,$$ where the infimum is over all one-way communication schemes that allow user $2$ to recover $\RX$. 
% Furthermore, the definition of one-way wiretap secret key capacity, denoted by $\wskc^{\text{ow}}$,  is the same as \eqref{eq:wskc} with the exception that the supremum is taken over all one-way SKA schemes. 


\begin{figure}[h]
\centering
\resizebox{0.9\width}{!}{\tikzstyle{block}=[rectangle, draw, thick, minimum width=2em, minimum height=2em]

\begin{tikzpicture}[node distance=4cm,auto,>=latex']

    \node (f) {};
    \node [block] (a) [left of = f, node distance=1.5cm] {1};
    \node (x) [above of = a, node distance=1.25cm] {$\RX^n$};
    \node [block] (b) [right of=f, node distance=1.5cm] {2};
    \node (y) [above of = b, node distance=1.25cm] {$\RY^n$};
    \node [block] (w) [below of = f, node distance=1.5cm] {W};
    \node (z) [left of = w, node distance=1.25cm] {$\RZ^n$};
    \node (e1) [left of = a, node distance=1.5cm] {$\RE_1^{(n)}$};
    \node (e2) [right of = b, node distance=1.5cm] {$\RE_2^{(n)}$};

    \draw[->, thick] (x) --  (a);
    \draw[->, thick] (a) --  (e1);
    \draw[->, thick] (y) --  (b);
    \draw[->, thick] (b) --  (e2);
   \draw[->, thick] (a) -- (b) node [midway, above] {$\RF_1$};
%    \draw[->] (f.center) --  (w);
    \draw[->,thick] (z) -- (w);
\end{tikzpicture}}
\caption{Only one message transfer is allowed.  Since omniscience is, in general, not possible within this setup, we only require user $2$ to recover user $1$'s observations, i.e., $\RE_1^{(n)}$ is  constant and $\RE_2^{(n)}= \hat{\RX}^{(n)}$.}
\label{fig:oneway}
 \end{figure}
% Ahlswede and Csisz\'ar, in \cite{ahlswedeCRpart1}, studied the one-way wiretap secret key agreement, and gave a single-letter expression \cite[Theorem~1]{ahlswedeCRpart1} for secret key capacity: 
% \begin{align}
%     \wskc^{\text{ow}} = \max_{\RV-\RU-\RX-(\RY, \RZ)}\Big[I(\RU \wedge \RY | \RV)-I(\RU \wedge \RZ | \RV)\Big]. \label{eq:oneway_wskc}
% \end{align}
% In the above optimization, it is enough to consider random variables $\RU$ and $\RV$ (taking values in sets $\mc{U}$ and $\mc{V}$, respectively) such that $|\mcU| \leq |\mcX|^2$ and $|\mcV| \leq |\mcX|$.
On the other hand, the problem of one-way leakage rate was studied in \cite{vinod07}, but with a measure of leakage $\rl^{\text{ow}}+I(\RX \wedge \RZ)$. A single-letter expression obtained, in \cite[Theorem~1]{vinod07}, for $\rl^{\text{ow}}+I(\RX \wedge \RZ)$ is $\min\limits_{\RS-\RX-(\RY, \RZ)}\left[I(\RS,\RZ \wedge \RX)\right.\linebreak\left.+\,H(\RX | \RS, \RY)\right]$.
Therefore, we have
\begin{align}\label{eq:oneway_rl}
    \rl^{\text{ow}} = \min\limits_{\RS-\RX-(\RY, \RZ)}\left[I(\RS \wedge \RX | \RZ)+H(\RX | \RS, \RY)\right],
\end{align}
where the minimization is over random variables $\RS$ taking values in a set $\mc{S}$ such that $|\mcS|\leq |\mcX|$.

% On the other hand, the problem of one-way leakage rate was studied in \cite{vinod07}, but with a measure of leakage that only differs from $\rl^{\text{ow}}$ by $I(\RX \wedge \RZ)$. They gave a single-letter characterization \cite[Theorem~1]{vinod07} for the minimum leakage rate for recovering $\RX$:
% \begin{align}\label{eq:oneway_rl}
%     \rl^{\text{ow}} = \min\limits_{\RS-\RX-(\RY, \RZ)}\left[I(\RS \wedge \RX | \RZ)+H(\RX | \RS, \RY)\right],
% \end{align}
% where the minimization is over random variable $\RS$ taking values in a set $\mc{S}$ such that $|\mcS|\leq |\mcX|$.

% We will make use of the following standard result on broadcast channels to construct a source $P_{\RX\RY\RZ}$ with $\wskc^{\text{ow}}>0$ and $\rl^{\text{ow}}=H(\RX | \RZ)$. Let $h(q)$ denote the binary entropy function, i.e, $h(q)= -q \log_2q-(1-q) \log_2(1-q)$, for $q \in (0,1)$.
% % The motivation for considering broadcast channels comes from the structure of the expressions.

% \begin{lemma}[{{\cite[p.~121]{elgamalbook}}}] \label{lem:bc_oneway}
%     Consider a discrete memoryless broadcast channel $P_{\RY\RZ | \RX}$ with $\mc{X} \in \{0,1\}$, $\mc{Y} \in \{0,1\}$ and $\mc{Z} \in \{0,1,\Delta\}$, where the channel from $\RX$ to $\RY$ is BSC($p$), $p \in (0,\frac{1}{2})$, and the channel from $\RX$ to 
%     $\RZ$ is BEC($\epsilon$),  $\epsilon \in (0,1)$.  Then, for $4p(1-p) < \epsilon \leq h(p)$,
%     \begin{enumerate}
%         \item $\RZ$ is more capable than $\RY$, i.e., for every input distribution $P_{\RX}$, $$I(\RX\wedge\RZ) \geq I(\RX\wedge\RY),$$
%         \item $\RZ$ is not less noisy than $\RY$, i.e., there exists a joint distribution $P^*_{\RU\RX}$ where $P_{\RU\RX\RY\RZ}=P^*_{\RU\RX}P_{\RY\RZ | \RX}$  such that $$I(\RU\wedge\RZ) < I(\RU\wedge\RY).$$
%         In fact,  $P^*_{\RU\RX}$ that satisfies the above condition is obtained by passing $\RU \sim \text{Ber}(\frac{1}{2})$ through $ \text{BSC}\left(\frac{1}{2}-\delta\right)$ with output $\RX$, where $\delta>0$ is small enough, and depends on $\epsilon$ and $p$. 
%     \end{enumerate}
% \end{lemma}

% %Though the proof is standard, we give it in Appendix~\ref{app:proofbc} for the sake of completeness. 
% Note that, for the distribution $P^*_{\RU\RX}$ in the above lemma, the marginal distribution of $\RX$ is $\text{Ber}(\frac{1}{2})$. 

% %The next lemma indeed shows that there are sources in the one-way communication case with the desired properties. 

% \begin{lemma}\label{lem:onewaycounter}
%  There exists a source $P_{\RX\RY\RZ}$ such that $\wskc^{\text{ow}}>0$ but $\rl^{\text{ow}}=H(\RX | \RZ)$.
% \end{lemma}
% \begin{proof}
%  Consider the source $P_{\RX\RY\RZ}=P_{\RX}P_{\RY | \RX}P_{\RZ | \RX}$ where $\RX \sim \text{Ber}(\frac{1}{2})$, the channel from $\RX$ to $\RY$ is BSC($p$) and the channel from $\RX$ to $\RZ$ is BEC($\epsilon$) such that $4p(1-p) < \epsilon \leq h(p)$. According to Lemma~\ref{lem:bc_oneway},  $\RZ$ is not less noisy than $\RY$. Therefore, $I(\RU\wedge\RZ) < I(\RU\wedge\RY)$ for some  joint distribution $P^*_{\RU\RX}=P^*_{\RU|\RX}P_X$ where $\RX \sim \text{Ber}(\frac{1}{2})$. The joint distribution $P_{\RU\RX\RY\RZ}:=P^*_{\RU|\RX}P_{ \RX\RY\RZ}=P^*_{\RU\RX}P_{\RY\RZ | \RX}$ satisfies the Markov chain $\RU-\RX-(\RY,\RZ)$. It follows that
% \begin{align*}
%     \wskc^{\text{ow}} & = \max_{\RV-\RU-\RX-(\RY, \RZ)}\Big[I(\RU \wedge \RY | \RV)-I(\RU \wedge \RZ | \RV)\Big]\\
%     & \utag{a}{\geq} I(\RU\wedge\RY) - I(\RU\wedge\RZ)> 0
% \end{align*}
% where (a) is obtained  by setting $\RV$ to a constant. This proves that wiretap secret key capacity is strictly positive.\\
% The minimum leakage rate for one-way communication, 
% \begin{align*}
%     \rl^{\text{ow}} &= \min\limits_{\RS-\RX-(\RY, \RZ)}\big[I(\RS \wedge \RX | \RZ)+H(\RX | \RS, \RY)\big]\\& =\min\limits_{\RS-\RX-(\RY, \RZ)}\big[H(\RX | \RZ)+H(\RX | \RS, \RY)-H(\RX | \RS, \RZ)\big]
% \end{align*}
% is upper bounded by $H(\RX | \RZ)$, which is obtained by setting $\RS:=\RX$. For $H(\RX | \RZ)\leq\rl^{\text{ow}}$, it is enough to prove that for any $\RS-\RX-(\RY, \RZ)$, $H(\RX | \RS, \RY)-H(\RX | \RS, \RZ) = I(\RX \wedge \RZ | \RS) - I(\RX \wedge \RY | \RS) \geq 0$. Observe that
% \begin{multline*}
%         I(\RX \wedge \RZ | \RS) - I(\RX \wedge \RY | \RS)= \sum P_{\RS}(s)\left[I(\RX \wedge \RZ | \RS=s) - I(\RX \wedge \RY | \RS=s)\right].
% \end{multline*}


% For an $s$ with $P_{\RS}(s)>0$, the term $I(\RX \wedge \RZ | \RS=s) - I(\RX \wedge \RY | \RS=s)$ is evaluated with respect to $P_{\RX,\RY,\RZ | \RS=s} = P_{\RX | \RS=s}P_{\RY,\RZ | \RX}= P_{\RX | \RS=s}P_{\RY | \RX}P_{\RZ | \RX}$. So this term is equal to $I(\RX_s \wedge \RZ) - I(\RX_s \wedge \RY)$, where $\RX_s \sim P_{\RX | \RS=s}$, and $\RY$ (resp. $\RZ$) is obtained by passing $\RX_s$ through BSC($p$) (resp. BEC($\epsilon$)). Since $\RZ$ is more capable than $\RY$, $I(\RX_s \wedge \RZ) - I(\RX_s \wedge \RY)\geq 0$ for every $s$. As a result, we have $I(\RX \wedge \RZ | \RS) - I(\RX \wedge \RY | \RS) \geq 0$, which completes the proof.
% \end{proof}

% To prove Lemma~\ref{lem:onewaycounter}, it is enough to consider any source $P_{\RX\RY\RZ}=P_{\RX}P_{\RY\RZ| \RX}$ such that for the channel $P_{\RY\RZ| \RX}$, $\RZ$ is more capable than $\RY$, and $\RZ$ is not less noisy than $\RY$. Moreover, $P_{\RX}$ is the marginal distribution of $P^*_{\RU\RX}$, the distribution for which the less noisy condition fails.

\subsection{Two messages are allowed, i.e., $r=2$ }
If we allow the users to exchange two messages interactively (Fig.~\ref{fig:twoway}), then omniscience is possible, as users 1 and 2 can communicate non-interactively at any rate larger than $H(\RX|\RY)+H(\RY|\RX)$ to recover each other's source. Let $\wskc^{(r)}$ and $\rl^{(r)}$ be defined as in \eqref{eq:wskc} and \eqref{eq:rl} but with a restriction to communication schemes involving only  $r=2$ interactive messages. Here we do not impose the condition that a particular user must transmit the first message. So any user can initiate the protocol, but we allow at most two messages to be exchanged. In this case, we can ask the same question: Does $\wskc^{(2)}>0$ imply that $\rl^{(2)}< H(\RX, \RY|\RZ)\/$? 

 \begin{figure}[h]
\centering
\resizebox{0.9\width}{!}{\tikzstyle{block}=[rectangle, draw, thick, minimum width=2em, minimum height=2em]

\begin{tikzpicture}[node distance=4cm,auto,>=latex']

    \node (f) {};
    \node [block] (a) [left of = f, node distance=1.5cm] {1};
    \node (x) [above of = a, node distance=1.25cm] {$\RX^n$};
    \node [block] (b) [right of=f, node distance=1.5cm] {2};
    \node (y) [above of = b, node distance=1.25cm] {$\RY^n$};
    \node [block] (w) [below of = f, node distance=1.5cm] {W};
    \node (z) [left of = w, node distance=1.25cm] {$\RZ^n$};
    \node (e1) [left of = a, node distance=1.5cm] {$\RE_1^{(n)}$};
    \node (e2) [right of = b, node distance=1.5cm] {$\RE_2^{(n)}$};

    \draw[->, thick] (x) --  (a);
    \draw[->, thick] (a) --  (e1);
    \draw[->, thick] (y) --  (b);
    \draw[->, thick] (b) --  (e2);
   \draw[->, thick] ([yshift=0.15 cm] a.east) -- ([yshift=0.15 cm]b.west) node [midway, above] {$\RF$};
   \draw[->, thick] ([yshift=-0.15 cm]b.west) -- ([yshift=-0.15 cm]a.east) ;
  %    \draw[->] (f.center) --  (w);
    \draw[->,thick] (z) -- (w);
\end{tikzpicture}}
\caption{Two messages are allowed. Here omniscience is feasible. If user $1$ initiates the communication, then $\RF=(\RF_1, \RF_2)$ where $\RF_2$, the communication by user $2$, depends on $\RF_1$ . Similarly, if user $2$ starts the communication, then $\RF=(\RF_2, \RF_1)$ and $\RF_1$, the communication made by user $1$, depends on $\RF_1$.}
\label{fig:twoway}
 \end{figure}
 
It turns out that with two messages, the ability to generate a positive secret key rate does not imply that the minimum leakage rate for omniscience is  strictly  less than $H(\RX, \RY | \RZ)$. To show this, we will use the results from the one-way communication setting. Let $\rl^{\text{ow}}(1 \rightarrow 2)$ (resp. $\rl^{\text{ow}}(2 \rightarrow 1))$  denote the minimum leakage rate for recovery of $\RX$ by user $2$ when user $1$ is the transmitter (resp. recovery of $\RY$ by user $1$ when user $2$ is the transmitter). By \eqref{eq:oneway_rl}, we have 
\begin{align*}
    \rl^{\text{ow}} (1 \rightarrow 2) &= \min\limits_{\RS-\RX-(\RY, \RZ)}\left[I(\RS \wedge \RX | \RZ)+H(\RX | \RS, \RY)\right]
    % \wskc^{\text{ow}}(1 \rightarrow 2) & = \max_{\RV-\RU-\RX-(\RY, \RZ)}\left[I(\RU \wedge \RY | \RV)-I(\RU \wedge \RZ | \RV)\right],
\end{align*}
and 
\begin{align*}
    \rl^{\text{ow}} (2 \rightarrow 1) &= \min\limits_{\RS-\RY-(\RX, \RZ)}\left[I(\RS \wedge \RY | \RZ)+H(\RY | \RS, \RX)\right].
    % \wskc^{\text{ow}}(2 \rightarrow 1) & = \max_{\RV-\RU-\RY-(\RX, \RZ)}\left[I(\RU \wedge \RX | \RV)-I(\RU \wedge \RZ | \RV)\right].
\end{align*}

% Since any one-way SKA scheme is also a valid SKA scheme in the $r=2$ case, 
% \begin{align}
%     \wskc^{(2)}\geq \max \left\lbrace \wskc^{\text{ow}}(1 \rightarrow 2), \wskc^{\text{ow}}(2 \rightarrow 1)\right \rbrace.\label{eq:twowskc}
% \end{align}
We next prove the following lower bound on the minimum leakage rate: 
\begin{align}
         \rl^{(2)}\geq \min & \left\lbrace \rl^{\text{ow}} (1 \rightarrow 2)+ H(\RY | \RZ, \RX),  
         %\right. \notag \\ &\mkern 30mu\left.
         \rl^{\text{ow}} (2 \rightarrow 1)+ H(\RX | \RZ, \RY)\right\rbrace,\label{eq:tworl}
\end{align}
% \begin{align*}
%          \rl^{(2)}\geq \min & \left\lbrace \min_{\RU-\RX-\RY,\RZ} I(\RU \wedge \RX | \RZ)+ H(\RX | \RY, \RU)+ H(\RY | \RZ, \RX), \right.  \notag \\ &\mkern -30mu\left. \min_{\RU-\RY-\RX,\RZ} I(\RU \wedge \RY | \RZ)+ H(\RY | \RX, \RU)+ H(\RX | \RZ, \RY)\right\rbrace,\label{eq:tworl}
% \end{align*}
where each term corresponds to a lower bound on the leakage rate when a particular user transmits first.  This bound may not be tight in general but will be enough for our purpose of constructing a counterexample. To prove \eqref{eq:tworl}, first we will show that $\rl^{(2)}\geq \rl^{\text{ow}} (1 \rightarrow 2)+ H(\RY | \RZ, \RX)$ when user $1$ starts the communication. Note that for any omniscience scheme $(\RF_1^{(n)},\RF_2^{(n)})$, we have $I(\RF_1^{(n)},\RF_2^{(n)} \wedge \RX^n,\RY^n|\RZ^n) \geq I(\RF_1^{(n)} \wedge \RX^n|\RZ^n) + I(\RF_2^{(n)} \wedge \RY^n|\RZ^n,\RX^n) \geq I(\RF_1^{(n)} \wedge \RX^n|\RZ^n) + H(\RY^n|\RZ^n,\RX^n)- n\delta_n$, where the last equality follows from Fano's inequality and the recoverability condition of $\RY^n$ from $\RF_2^{(n)}$ and $\RX^n$. Here, $\delta_n \to 0$ as $n \to \infty$. Therefore, we have 
\begin{align*}
\limsup_{n \to \infty} \frac{1}{n} I(\RF_1^{(n)},\RF_2^{(n)} \wedge \RX^n,\RY^n|\RZ^n) & \geq \limsup_{n \to \infty}\frac{1}{n} I(\RF_1^{(n)} \wedge \RX^n|\RZ^n) + H(\RY|\RZ,\RX)\\
&\geq \rl^{\text{ow}} (1 \rightarrow 2)+ H(\RY | \RZ, \RX).
\end{align*}
Since the above inequality holds for any omniscience scheme where user $1$ initiates the communication, we can conclude that $\rl^{(2)}\geq \rl^{\text{ow}} (1 \rightarrow 2)+ H(\RY | \RZ, \RX)$. Similarly, for  omniscience schemes with user $2$ starting the communication, we have that $\rl^{(2)}\geq \rl^{\text{ow}} (2 \rightarrow 1)+ H(\RX | \RZ, \RY)$. This completes the proof of \eqref{eq:tworl}.

We will make use of the following standard result on broadcast channels to construct a source $P_{\RX\RY\RZ}$ with $\wskc^{(2)}>0$ and $\rl^{(2)}=H(\RX, \RY | \RZ)$. Let $h(q)$ denote the binary entropy function, i.e, $h(q)= -q \log_2q-(1-q) \log_2(1-q)$, for $q \in (0,1)$.
 
% The motivation for considering broadcast channels comes from the structure of the expressions.

\begin{lemma}[{{\cite[p.~121]{elgamalbook}}}] \label{lem:bc_oneway}
    Consider a discrete memoryless broadcast channel $P_{\RY\RZ | \RX}$ with $\mc{X} \in \{0,1\}$, $\mc{Y} \in \{0,1\}$ and $\mc{Z} \in \{0,1,\Delta\}$, where the channel from $\RX$ to $\RY$ is BSC($p$), $p \in (0,\frac{1}{2})$, and the channel from $\RX$ to 
    $\RZ$ is BEC($\epsilon$),  $\epsilon \in (0,1)$.  Then, for $\epsilon \leq h(p)$,  $\RZ$ is more capable than $\RY$, i.e., for every input distribution $P_{\RX}$, $I(\RX\wedge\RZ) \geq I(\RX\wedge\RY).$
\end{lemma}

        

For a source distribution  $P_{\RX\RY\RZ}=P_{\RX}P_{\RY\RZ|\RX}=P_{\RY}P_{\RX\RZ|\RY} $, if $\RZ$ is more capable than $\RY$ for the channel $P_{\RY\RZ|\RX}$, then $\min\limits_{\RS-\RX-(\RY, \RZ)}\left[I(\RX \wedge \RZ | \RS) - I(\RX \wedge \RY | \RS)\right] =\sum_{s \in \mcS} P_{\RS}(s)\left[I(\RX \wedge \RZ | \RS=s) - I(\RX \wedge \RY | \RS=s)\right]\geq 0$. This is because for an $s \in \mcS$ with $P_{\RS}(s)>0$, the term $I(\RX \wedge \RZ | \RS=s) - I(\RX \wedge \RY | \RS=s)$ is evaluated with respect to $P_{\RX,\RY,\RZ | \RS=s} = P_{\RX | \RS=s}P_{\RY,\RZ | \RX}= P_{\RX | \RS=s}P_{\RY | \RX}P_{\RZ | \RX}$. So this term is equal to $I(\RX_s \wedge \RZ) - I(\RX_s \wedge \RY)$, where $\RX_s \sim P_{\RX | \RS=s}$, and $\RY$ (resp. $\RZ$) is obtained by passing $\RX_s$ through $P_{\RY|\RX}$ (resp. $P_{\RZ|\RX}$). Since $\RZ$ is more capable than $\RY$, $I(\RX_s \wedge \RZ) - I(\RX_s \wedge \RY)\geq 0$ for every $s$. As a result, we have $I(\RX \wedge \RZ | \RS) - I(\RX \wedge \RY | \RS) \geq 0$. Therefore, we have
\begin{align*}
    \rl^{\text{ow}} (1 \rightarrow 2)+ H(\RY | \RZ, \RX) &= \min\limits_{\RS-\RX-(\RY, \RZ)}\left[I(\RS \wedge \RX | \RZ)+H(\RX | \RS, \RY)\right]+ H(\RY | \RZ, \RX)\\
    &= H(\RX,\RY | \RZ) + \min\limits_{\RS-\RX-(\RY, \RZ)}\left[I(\RX \wedge \RZ | \RS) - I(\RX \wedge \RY | \RS)\right]\\
    &\geq H(\RX,\RY | \RZ).
\end{align*}
Similarly, for the channel $P_{\RX\RZ|\RY}$, if $\RZ$ is more capable than $\RX$, then we have $ \rl^{\text{ow}} (2 \rightarrow 1)\geq H(\RX,\RY | \RZ)$. Thus $\rl^{(2)} = H(\RX, \RY | \RZ)$, which follows from \eqref{eq:RL:lb} and \eqref{eq:tworl}. 

A source $(\RX, \RY, \RZ)$ is called a \emph{DSBE$(p,\epsilon)$ source} if $(\RX, \RY)$ is a doubly  symmetric binary source with parameter $p$, and $\RZ\in \{0,1\}^2 \cup \{\Delta\}$ is obtained by passing $(\RX, \RY)$ through an erasure channel with erasure probability $\epsilon$. It means that for a DSBE$(p,\epsilon)$ source $(\RX, \RY, \RZ)$, $\RX \sim \text{Ber}(\frac{1}{2})$, the channel from $\RX$ to $\RY$ is a BSC($p$), and the channel from $(\RX, \RY)$ to $\RZ$ is 
 $$P_{\RZ | \RX,\RY}\left(z|x,y\right)= \left\{\begin{array}{ll} 1-\epsilon, & \text{if } z=(x,y),\\
        \epsilon, & \text{if } z=\Delta,\\
        0, & \text{otherwise},
        \end{array}\right.$$ 
for every $(x,y)\in \{0,1\}^2$.
\begin{lemma}\label{lem:twowaycounter}
For a DSBE$(p,\epsilon)$ source with $p$ and $\epsilon$ chosen so that $ \frac{\min\{p,1-p\}}{\max\{p,1-p\}} < \epsilon \leq h(p)$, we have $\wskc^{(2)}>0$ but $\rl^{(2)}=H(\RX,\RY|\RZ)$.
\end{lemma}
\begin{proof}
 First note that $\wskc^{(2)}>0$ is equivalent to the condition that $\wskc>0$ (with no restriction on the number of communications), as was shown by Orlitsky and Wigderson in \cite{orlitsky1993secrecy} (and reproduced in Theorem~1 of \cite{amin2020}). For a DSBE$(p,\epsilon)$, it follows from Equation~(69) of \cite{amin2020} that $\wskc>0$ if and only if $\epsilon > \frac{\min\{p,1-p\}}{\max\{p,1-p\}}$. As a result, we have $\wskc^{(2)}>0$ for the chosen parameters.


Let us now argue that $\rl^{(2)}=H(\RX,\RY|\RZ)$. Since a DSBE$(p,\epsilon)$ source is symmetrical in $\RX$ and $\RY$, it is enough to show that the more capable condition hold for one user. In other words, it is sufficient to show that for the channel $P_{\RY\RZ | \RX}$, $\RZ$ is more capable than $\RY$. For any binary input distribution $P_{\tRX}=(P_{\tRX}(0), P_{\tRX}(1)):=(q, 1-q), 0 \leq q \leq 1$, to the channel $P_{\RY \RZ | \RX}$,  $I(\tRX\wedge\RY) =  h(p*q) - h(p) $, where $p*q=p(1-q)+(1-p)q$. Let $f(q):= (1-\epsilon)h(q) - h(p*q) +h(p)= I(\tRX\wedge\RZ) - I(\tRX\wedge\RY)$. Note that this difference is the same as that of the source considered in Lemma~\ref{lem:bc_oneway}. The proof of that lemma involves showing that for $\epsilon \leq h(p)$, $f(q)$ is a non-negative function, % and moreover, $f(q)$ is strictly convex around $q=\frac{1}{2}$, 
which is equivalent to the more capable condition. %and not less noisy conditions, respectively. 
Making use of this property of $f(q)$, we can also conclude that  for $\epsilon \leq h(p)$, $\RZ$ is more capable than $\RY$ for $P_{\RY\RZ | \RX}$. 

Thus, the minimum leakage rate $\rl^{(2)}=H(\RX,\RY|\RZ)$ because $\RZ$ is more capable than $\RY$ for the channel $P_{\RY\RZ | \RX}$, and $\RZ$ is more capable than $\RX$ for the channel $P_{\RX\RZ | \RY}$.
% Since $\RZ$ is not less noisy than $\RY$ for $P_{\RY\RZ | \RX}$, $\wskc^{\text{ow}}(1 \rightarrow 2)$ is positive, and hence we have $\wskc^{(2)}>0$ by \eqref{eq:twowskc}. 
% And, the minimum leakage rate $\rl^{(2)}=H(\RX,\RY|\RZ)$ because $\RZ$ is more capable than $\RY$ for the channel $P_{\RY\RZ | \RX}$, and $\RZ$ is more capable than $\RX$ for the channel $P_{\RX\RZ | \RY}$.
\end{proof}

For the source given in the above lemma, no user can gain an advantage in terms of $\rl^{(2)}$ over the other by starting the communication. This completes the proof of Proposition~\ref{prop:positivity}. 

 This result seems to indicate that duality does not always hold. We conjecture that for the DSBE  source considered in the above lemma, $\wskc^{(r)}>0$ need not imply $\rl^{(r)} < H(\RZ_V|\RZ_{\opw})$, $r\geq 2$. We additionally conjecture that, even with no restriction on the number of communications, $\wskc>0$ need not imply $\rl < H(\RZ_V|\RZ_{\opw})$.   

% \subsection{$r\geq 3$ messages are allowed}
% The interplay between  the positivity of $\wskc$ and non-maximality of $\rl$ becomes more evident if there is no restriction on the number of messages. Unlike as in the case of one or two messages, our intuition actually works with multiple messages, in particular, with three messages.
