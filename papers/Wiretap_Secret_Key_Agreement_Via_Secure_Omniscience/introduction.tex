In the setting of the multiterminal source model for secure computation, users who privately observe correlated random variables from a source try to compute functions of these private observations through interactive public discussion. The goal of the users is to keep these computed functions secure from a wiretapper who has some side information (a random variable possibly correlated with the source), and noiseless access to the public discussion. A well-studied problem within this model is that of secret key agreement, where users try to agree on a key that is kept secure from the wiretapper. In other words, users try to compute a common function that is independent of the public discussion and the wiretapper's side information. 

The secret key agreement problem was first studied for two users by Maurer \cite{maurer93}, and Ahlswede and Csisz\'ar \cite{ahlswedeCRpart1}. These works attempted to characterize the \emph{wiretap secret key capacity} $\wskc$, which is defined as the maximum secret key rate possible with unlimited public discussion. They were able to do this in certain special cases, for instance, in the case when only one user is allowed to communicate \cite[Theorem~1]{ahlswedeCRpart1}, and in the case when the wiretapper's side information is conditionally independent of one user's private information, given that of the other user \cite[Theorems~2 and 3]{maurer93}. In particular, when the wiretapper has no side information, $\wskc$ was shown to be equal to the mutual information between the random variables observed by the two users. But, for the two-user setting without additional assumptions, only upper and lower bounds on $\wskc$ were given. Subsequently, there have been multiple efforts, notably \cite{renner_wolf03,csiszar04,aminsource}, to strengthen and extend these bounds to the general setting of two or more users, but finding a single-letter expression remains a fundamental open problem in this domain.

In the course of extending the earlier results to the setting of multiple users, Csisz{\'a}r and Narayan \cite{csiszar04} gave a single-letter expression for the secret key capacity in the case when the wiretapper has no side information. They did this by establishing an equivalence or ``duality'' between the secret key agreement problem and the source coding problem of communication for \emph{omniscience}, which is attained when each user is able to recover (with high probability) the private observations of all the other users. They observed that a secret key of maximum rate can be extracted from a protocol that involves public discussion at the minimum rate required to attain omniscience. They were thus able to relate the secret key capacity to $\rco$, the minimum rate of communication required for omniscience, which can be obtained as the solution to a relatively simple linear program. 

Subsequently, Gohari and Anantharam \cite{aminsource} succeeded in establishing a similar duality in the more general setting of a wiretapper having side information. They showed an equivalence between the wiretap secret key agreement problem (in the presence of a wiretapper having side information) and a problem of communication for omniscience at a neutral observer. In the latter problem, there is (in addition to the users and the wiretapper) a neutral observer who is given access to the wiretapper's side information. The goal here is for the users to communicate in public to create a shared random variable which when provided to the neutral observer, allows the observer to reconstruct all the users' private observations. Theorem~3 of \cite{aminsource} relates $\wskc$ to the minimum rate of public communication required for omniscience at the neutral observer. However, this does not lead to a single-letter expression for $\wskc$, as it is not known how to compute the minimum rate of communication for omniscience at the neutral observer.

Motivated in part by the results of \cite{csiszar04} and \cite{aminsource}, we explore the possibility of an alternative duality existing between the wiretap secret key problem and a certain \emph{secure omniscience} problem, in the hope of obtaining additional insight on $\wskc$, potentially leading to its evaluation in settings where it still remains unknown. In the secure omniscience problem we consider, we stay within the original setting of the multiterminal source model with a wiretapper having side information. The users communicate interactively in public so as to attain omniscience, but now the aim is not necessarily to minimize the rate of communication needed for this. Instead, the goal is to minimize the rate at which the communication for omniscience leaks information about the source to the wiretapper. We give the formal definition of $\rl$, the minimum information leakage rate of any communication for omniscience, in Section~\ref{sec:problem}. 

%We try to make inroads into this problem by connecting secret key generation with secure source coding namely secure omniscience. In the secure omniscience problem, every user tries to attain omniscience by communicating interactively using their private observations from a correlated source; however, the goal here is to minimize the information leakage rate $\rl$ to a wiretapper who has side information about the source. This is motivated by the work of Csisz\'{a}r and Narayan \cite{csiszar04} on the multiterminal setting with no wiretapper side information, where they gave a single-letter expression for the secret key capacity by connecting it with source coding problem of communication for omniscience. They observed that it is enough for the users to attain omniscience by communicating at the minimum rate possible to extract a secret key of maximum rate. This kind of a duality has been further explored in the case with wiretapper side information by Gohari and Anantharam, in \cite{aminsource}. They proved a duality between secret key agreement and the problem of communication for omniscience by a neutral observer, where the neutral observer attains omniscience given the common function agreed by the users and the wiretapper side information. We would like to explore an alternate duality without the neutral observer by relating $\wskc$ and $\rl$. If we are able to establish a duality between these two problems, then we can use an optimum secure omniscience protocol to achieve the wiretap secret key capacity.

\subsection{Main Contributions}

The starting point of our paper is an inequality that relates the wiretap secret key capacity and the minimum leakage rate for omniscience for a source $(\RZ_V,\RZ_{\opw})$. Here, $V:=\{1,\ldots, m\}$ denotes the set of users, $\RZ_V:=(\RZ_i \mid i \in V)$ is the collection of user observations, and $\RZ_{\opw}$ denotes the wiretapper's side information. We then have 
     \begin{equation}
      H(\RZ_V|\RZ_{\opw}) - \wskc \leq \rl ,
      \label{basic_ineq}
        \end{equation}
%The minimum leakage rate for omniscience is upper bounded by $H(\RZ_V|\RZ_{\opw})$.
  where $\rl$ is the information leakage rate, which we formally define in Section~\ref{sec:problem} as $\limsup_n\frac{1}{n}I(\RF^{(n)} \wedge \RZ_V^n|\RZ_{\opw}^n)$, minimized over all communications $\RF^{(n)}$ for omniscience.
The inequality follows from a standard argument: once the users attain omniscience via a communication protocol that achieves the minimum leakage rate $\rl$, they can extract a secret key of rate $H(\RZ_V | \RZ_{\opw}) - \rl$ from the reconstruction of $\RZ_V$ available to each of them. 
%We give the simple proof of this inequality in Section~\ref{sec:problem}.

If the inequality in \eqref{basic_ineq} holds with equality, then we refer to it as a duality between secure omniscience and wiretap secret key agreement.  Essentially, whenever this duality holds, a secret key of maximum rate can be extracted from a communication for omniscience protocol that minimizes the leakage rate. Note that equality in \eqref{basic_ineq} yields an expression for $\wskc$ in terms of $\rl$, but its utility towards computing $\wskc$ is unclear, as it is not known whether $\rl$ admits a single-letter expression.

We first address the question of whether there is always a duality between secure omniscience and wiretap secret key agreement for any multiterminal source model with wiretapper. Note that if equality holds in \eqref{basic_ineq}, then it must be the case that $\wskc = 0$ iff $\rl = H(\RZ_V | \RZ_{\opw})$. Now, it is easily shown that, for any multiterminal source model, $\wskc = 0$ implies $\rl = H(\RZ_V | \RZ_{\opw})$. This follows directly from \eqref{basic_ineq} and the upper bound $\rl \le H(\RZ_V | \RZ_{\opw})$, which always holds, as is easily seen from the definition of $\rl$ --- see Theorem~\ref{thm:RL:lb} in Section~\ref{sec:problem}. It is not so clear whether the converse is also true, namely, that $\rl = H(\RZ_V | \RZ_{\opw})$ implies $\wskc = 0$. We conjecture that  the converse does not always hold, i.e., there are sources for which $\rl = H(\RZ_V | \RZ_{\opw})$, yet $\wskc > 0$. We make partial progress in this direction by showing that this is the case if we restrict ourselves to omniscience protocols in which at most two communications are allowed. We give an example of a two-user source model for which $\wskc > 0$, but the leakage rate equals $H(\RZ_V | \RZ_{\opw})$ for any omniscience protocol involving at most two messages. While our example does not definitively resolve the issue of duality between secure omniscience and wiretap secret key agreement, it seems to indicate that this duality may not always hold.

Next, we consider a broad class of sources, namely, \emph{finite linear sources}, for which we believe the duality must hold. In a finite linear source (FLS) model, each user's observations, as well as the wiretapper's side information, is given by a linear transformation of an underlying random vector consisting of finitely many i.i.d.\ uniform random variables. This class of sources has received some prior attention \cite{chan11itw,chan11delay,chan19oneshot}. We prove that \eqref{basic_ineq} holds with equality for FLS models in which the wiretap secret key capacity $\wskc$ is achieved by a \emph{perfect} key agreement protocol involving public communications that are linear functions of the users' observations. It is an open question as to whether $\wskc$ can always be achieved through linear communication protocols for any FLS model, but it is reasonable to expect that this is the case. We also give two unconditionally positive results: duality holds in the case of two-user FLS models, and in the case of pairwise independent network (PIN) models on trees \cite{sirinpin, sirinperfect} in which the wiretapper's side information is a linear function of the source. In both these cases, we obtain explicit expressions for $\rl$ and $\wskc$. In fact, in the case of tree-PIN models with a linear wiretapper, we are able to explicitly determine the maximum secret key rate achievable when the total rate of public communication is constrained to be at most $R$.

{ Finally, we consider the problem of secure function computation, where users try to compute a given function $\RG=g(\RZ_1,\ldots, \RZ_m)$ without revealing much information about the computed value to the wiretapper. For this problem, we generalize the inequality \eqref{basic_ineq} to $H(\RG|\RZ_{\opw}) - \wskc \leq \rl^{\RG}$, where $\rl^{\RG}$ is the minimum information leakage rate for computing the function $\RG$, which we define as $\limsup_n\frac{1}{n}I(\RF^{(n)} \wedge \RG^n|\RZ_{\opw}^n)$, minimized over all communications $\RF^{(n)}$ for computing $\RG$. We also give the conditions under which the previous inequality holds with equality. Furthermore, for finite linear sources, we prove a result that says that any information leakage rate in computing a linear function can also be achieved by an omniscience scheme.}

% Finally, we show that the inequality in \eqref{basic_ineq} can be useful on its own. We use it to extend to the multi-user setting a recent result of Gohari, G\"{u}nl\"{u} and Kramer \cite{amin2020} that gives several equivalent conditions for the positivity of $\wskc$ in a two-user source model. 

%The main focus of our work is to explore the role of secure omniscience in the wiretap secret key generation. Firstly by using the inequality  $H(\RZ_V|\RZ_{\opw}) - \wskc \leq \rl$, we find necessary and sufficient conditions for the positivity of $\wskc$. In other words, we extend the two user result of Gohari, G\"{u}nl\"{u} and Kramer \cite{amin2020} to multiuser setting using the relation between $\rl$ and $\wskc$. Secondly, we address the question: Can the above inequality be strict in general? In other words, are there sources for which secure omniscience is not optimal for wiretap secret key generation? We partially answer this question by proving that if we limit the number of messages exchanged to two then there are sources for which $\rl=H(\RZ_V|\RZ_{\opw})$ and $\wskc>0$, thereby proving the strictness at least in a restrictive setting. We conjecture that the inequality is strict even without such restrictions. 
%
%Finally, we present some evidence supporting the conjecture that inequality holds with equality for finite linear sources (FLS) where we prove it in some specific instances like tree-PIN sources with linear wiretapper, two-user FLS, and in the cases where linear communication schemes are optimal in terms of $\wskc$. Also we show that secure omniscience indeed allows us to completely characterize the rate region of tree-PIN  sources with linear wiretapper.

\subsection{Related Work}

Our work is closely related to that of Prabhakaran and Ramchandran \cite{vinod07}. In their work, they considered the problem of secure source coding in a two-user model with a wiretapper where only one user is allowed to communicate to the other. This kind of communication is commonly referred to as \emph{one-way communication}. The goal here is to communicate in such a way that the receiving user recovers the observations of the transmitting user while minimizing the rate of information leaked to the wiretapper about the transmitting user's source. In this case, they  obtained a single-letter characterization of the minimum leakage rate for recovering one terminal's observation by the other terminal by using conventional information-theoretic techniques. Moreover, they used this quantity to lower bound the wiretap secret key capacity. Our work, in fact, generalizes this result by considering the minimum leakage rate for omniscience instead in the multi-user setting where interactive communication is allowed.


The secure source coding problem considered in \cite{vinod07},  has been generalized and studied extensively in the direction of characterizing the minimum rate of leakage of transmitter's source \cite{gunduz08, villard13} by incorporating various constraints. For instance, Villard and Piantanida \cite{villard13} considered a similar model as in \cite{vinod07}, but the receiving user observes  coded side information from a third party. Since uncoded side information is a special case of coded side information, this framework subsumes the model of \cite{vinod07}. For this model, they studied the problem in a broad generality by considering a lossy recovery of the transmitter's observations at the receiving terminal in the presence of a wiretapper. They gave a characterization of the rate-distortion-leakage rate region which is the set of all achievable tuples of communication rate, distortion and leakage rate.

Recently, in \cite{wenwentu19},  Tu and Lai have considered the same model but studied the problem of lossy function computation by the receiving terminal, which is a further generalization of the model of \cite{villard13}. They considered even the privacy aspect (leakage of the transmitting user's source to the receiving user) and studied it along with the rate-distortion-leakage rate region. They were able to give an explicit characterization of the entire achievable rate region.


This problem falls in the class of source coding for distributed function computation; see, for e.g., \cite{han_dichotomy, alon_roche, ma11, tyagi11, wenwentu19}. In this problem, each user has access to a private random variable, and they wish to compute functions of these private random variables by communicating in public, possibly interactively or/and in the presence of a wiretapper. For instance, in \cite{ma11},  Ma and Ishwar have considered a two-user model without a wiretapper, where users, after observing private random variables, interactively communicate to compute functions of these private random variables. They studied the interactive communication rates needed for the computation of functions and completely characterized the rate region. Subsequently, this work has been extended  by \cite{yassaee15} for randomized function computation in the two-user case. Recently, \cite{data_interactive_securefunction} has studied the randomized function computation even by including privacy constraints on the users' observation.

One work that studies the function computation in the context of  multi-user source model with a wiretapper is \cite{tyagi11}. In their work, Tyagi, Narayan, and Gupta assumed that the wiretapper has no side information and addressed the question: when can a common function be computed securely? Here we say a function is \emph{securely computable} if it is kept asymptotically independent of the communication that is needed to compute this function. It means that the wiretapper can gain almost no knowledge of the function output even with access to the communication. They answered this question by 
relating it with the secret key capacity of the source model. The precise result is that a common function is securely computable by all the terminals if and only if  the entropy of the function is less than the secret key capacity. 

Secure omniscience is also a problem of source coding for distributed function computation. Here, all the users try to recover the users' source, and the quantity of interest is the minimum rate of information about the source that gets leaked to the wiretapper through the communication. A problem that is closely related to secure omniscience is the coded cooperative data exchange (CCDE) problem with a secrecy constraint; see, for e.g., \cite{sprinston13, courtade16}. In the problem of CCDE, we consider a hypergraphical source and study one-shot omniscience. The hypergraphical model generalizes the PIN model within the class of FLSs. \cite{courtade16} studied the secret key agreement in the CCDE context and characterized the number of transmissions required versus the number of SKs generated. On the other hand, \cite{sprinston13} considered the same model but with wiretapper side information and explored the leakage aspect of an omniscience protocol. However, the security notion considered therein does not allow the eavesdropper to recover even one hyperedge of the source from the communication except what is already available. However, the communication scheme can still reveal information about the source. In this paper, we are interested in minimizing the rate of information leakage to the wiretapper. Though we consider the asymptotic notion, the  designed optimal communication scheme uses only a finite number of realizations of the source. Hence our scheme can find applications even in CCDE problems. 

The role of omniscience in the multi-user secret key agreement (with wiretapper side information) was highlighted in the work of Csisz{\'a}r and Narayan \cite{csiszar04}. They showed that a maximum key rate could be achieved  by communicating at a minimum rate for omniscience. This led to the question of whether the omniscience is optimal even in terms of the minimum communication rate needed to achieve secret key capacity. The works \cite{chan18, mukherjee16} have addressed this question by giving sufficient conditions for general sources and equivalent conditions for hypergraphical sources. 

Though the characterization of secret key capacity (without wiretapper side information) is known, and its connection with omniscience is well studied, the characterization of wiretap secret key capacity is still an open problem. Results are known only in special sources \cite{ahlswedeCRpart1, maurer93}. However, there has been some progress in this direction in recent times.  For instance,  Gohari, G\"{u}nl\"{u} and Kramer, in \cite{amin2020}, sought for the characterization of the class of two-user sources for which wiretap secret key capacity is positive. They were able to find an equivalent characterization in terms of R\'enyi divergence. Its usefulness has been demonstrated on  sources with an erasure model on the wiretapper side information   by deriving a sufficient condition for the positivity of $\wskc$. In the direction of characterizing $\wskc$,  Poostindouz and Safavi-Naini, in \cite{alireza19}, have made an effort in the case of some special source models. In particular, they considered tree-PIN models with a  wiretapper side information containing  noisy versions of the edge random variables. They obtained a characterization of $\wskc$ in terms of the conditional minimum rate of communication for omniscience which is a solution to a certain linear program.

% Despite all these efforts,  to our knowledge, there is no work that addresses the relationship between secure omniscience and wiretap secret key agreement for multiterminal setting with interactive communication allowed.




\subsection{Organization}
This paper is organized as follows. In Section~\ref{sec:problem}, we introduce the problem and notations. In this section, we also establish an inequality relating the minimum leakage rate for omniscience and wiretap secret key capacity for general source models. Section~\ref{sec:counterexamaple_duality} contains an example showing that the duality does not hold between secure omniscience and secret key agreement in the case of limited interaction (with two messages allowed). This result suggests that the duality need not hold in the general case. In Section~\ref{sec:duality_fls}, we first formally define the finite linear source models and prove a duality result concerning linear protocols. Furthermore, we establish an unconditional result in the two-user FLS. In Section~\ref{sec:treepin}, we prove the duality in the case of the tree-PIN model with linear wiretapper. Moreover, for this model,  we determine the rate region containing all achievable secret key rate and total communication rate pairs. In fact, we use a secure omniscience scheme for a part of the source to obtain this result. 
%In Section~\ref{sec:positivity}, we obtain some equivalent conditions for the positivity of $\wskc$ for multi-user case using \eqref{basic_ineq}. This generalizes the two-user result of \cite{amin2020}. 
{ In Section~\ref{sec:funcomp}, we introduce the problem of secure function computation and derive an inequality between the minimum leakage rate for computing a function and the wiretap secret key capacity. Furthermore, in this section, we explore the conditions under which this inequality holds with equality.} Finally, we discuss the open problems and challenges in establishing duality in Section~\ref{sec:discussion}.


% The main focus of our work is to find the conditions under which these two inequalities hold with equality. For instance, if $\wskc=0$, then $\rl=H(\RZ_V|\RZ_{\opw})$, in which case the second inequality holds trivially with equality. We believe that if $\wskc>0$, then the second inequality is strict. But we could establish this in the case of two users using the work of Gohari, G\"{u}nl\"{u} and Kramer \cite{amin2020}. Moreover, we showed that this need not happen if we limit the number the rounds of communication to one, i.e., two users are allowed to exchange two messages interactively. 

% For the first inequality, if the equality holds, then we refer to it as a duality between secure omniscience and wiretap secret key agreement. Currently, we do not have enough evidence favoring the duality for any general source. But we do have some evidence backing the conjecture that it holds for finite linear sources (FLS). We were able to prove it in some specific instances like tree-PIN sources with linear wiretapper, two-user FLS, and in the cases where linear communication schemes are optimal in terms of $\wskc$.





% to understand the optimality of these two inequalities on $\rl$ by finding some interesting connections to the wiretap secret key capacity. This will help us identify the role played by  secure omniscience in secret key generation and vice versa. 

% Clearly, if $\wskc=0$, then $\rl=H(\RZ_V|\RZ_{\opw})$, in which case both these inequalities hold trivially with equality. On the other hand, if positive rate secret key generation is feasible, i.e., $\wskc>0$, it is unclear about the tightness of either of them. We will explore this question in this paper. In connection to the second inequality, we argue the strictness of it in the two terminal case using a recent work on coding for positive secret key rate \cite{amin2020}, and additionally, we show that this need not happen if we limit the number the rounds of communication to one, i.e., two users are allowed to exchange two messages interactively. 

% Similarly, for the first inequality, we conjecture that it ought to hold with equality for an important class of sources called finite linear sources (FLS).  As a preliminary evidence to this conjecture, we show  a result on linear communication schemes on FLS which says that duality is implied by the optimality of linear schemes. Finally, we confirm the validity of this conjecture on non-trivial sub-classes of FLS, namely two-user FLS and pairwise independent network (PIN)  source model defined on trees with linear wiretapper. The proof involves a key idea of constructing a secure omniscience communication scheme that aligns perfectly with the wiretapper's side information. Finally we prove that for tree-PIN source with linear wiretapper secure omniscience is indeed optimal even in terms of constrained wiretap secret key capacity.











