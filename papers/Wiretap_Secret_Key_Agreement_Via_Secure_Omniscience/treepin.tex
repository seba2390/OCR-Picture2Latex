 A source $\RZ_V$ is said to be \emph{tree-PIN} if there exists a tree $T=(V,E,\xi)$ and for each edge $e \in E$, there is a non-negative integer $n_e$ and a random vector $\RY_e = \left( \RX_{e,1}, \ldots, \RX_{e,n_e} \right)$. We assume that the collection of random variables $\RX :=(\RX_{e,k}: e\in E, k \in [n_e])$ are i.i.d. and each component is  uniformly distributed over a finite field, say $\Fq$. For $i \in V$,
 \begin{align*}
  \RZ_i = \left( \RY_e : i \in \xi (e) \right).
 \end{align*} 

 The linear wiretapper's side information $\RZ_{\opw}$ is defined as 
\begin{align*}
 \RZ_{\opw} = \RX \MW,
\end{align*}
where $\RX$ is a $1 \times (\sum_{e \in E}n_e)$ vector and $\MW$ is a $(\sum_{e \in E}n_e) \times n_w$ full column-rank matrix over $\Fq$. We sometimes refer to $\RX$ as the base vector. We refer to the pair $(\RZ_V, \RZ_{\opw})$ defined as above as a \emph{tree-PIN source with linear wiretapper}. This is a special case of an FLS. 

\begin{example}
Consider the tree $T$ in Fig.~\ref{fig:tree_pin_def} defined on  $V=\{1, \ldots,5\}$ with $E=\{a,b,c,d\}$. Let $\RY_a = (\RX_{a1}, \RX_{a2})$, $\RY_b = \RX_{b1}$, $\RY_c =\RX_{c1}$, and $\RY_d =\RX_{d1}$, where the base vector $\RX= (\RX_{a1}, \RX_{a2}, \RX_{b1}, \RX_{c1}, \RX_{d1})$ is uniformly distributed over $\mathbb{F}_2^5$. The corresponding source $\RZ_V$ is given by
\begin{gather*}
   \RZ_1= (\RX_{a1}, \RX_{a2}), \quad \RZ_2= (\RX_{a1}, \RX_{a2}, \RX_{b1}, \RX_{c1}), \quad \RZ_3=\RX_{c1}\\
    \RZ_4= (\RX_{b1}, \RX_{d1}), \quad \RZ_5= \RX_{d1}.
\end{gather*}


\begin{figure}[h]
    \centering
        \resizebox{0.85\width}{!}{\begin{tikzpicture}[-,>=stealth, line width=5pt,thick, auto]
\tikzstyle{vertex}=[circle, fill=black,inner sep=0pt, minimum size=5pt];

\node[vertex]      (2)        [label= above:{$2$}] {};
\node[vertex]      (1)        [above left = 3 em and 4 em of 2,label= above:{$1$}]  {};
\node[vertex]      (3)       [below left = 3 em and 4 em of 2,label= below:{$3$}] {};
\node[vertex]      (4)      [right = 4 em of 2,label= above:{$4$}]  {};
\node[vertex]      (5)      [right = 4 em of 4,label= above:{$5$}]  {};


\draw (1) -- (2) node (a) [midway, above] {$a$} ;
\draw (2) -- (4) node (b) [midway, above] {$b$} ;
\draw (2) -- (3) node (c) [midway, below] {$c$} ;
\draw (4) -- (5) node (d) [midway, above] {$d$} ;


\end{tikzpicture}}
        \caption{Tree $T$ corresponding to a tree-PIN model}\label{fig:tree_pin_def}
\end{figure}
The observations of a wiretapper are $\RZ_{\opw}=(\RX_{a1}+\RX_{a2}, \RX_{a2}+\RX_{b1}+\RX_{d1})$.
This source $(\RZ_V, \RZ_{\opw})$ is an example of a tree-PIN source with a linear wiretapper. This is a special case of an FLS, as we can express it as $\bM \RZ_1 & \cdots& \RZ_m & \RZ_{\opw}\eM=\RX \bM \MM_1\;\cdots\;\MM_m \; \MW \eM$ for some matrices $\MM_1,\dots, \MM_m, \MW$. For example, we can write  $$\RZ_2= (\RX_{a1}, \RX_{a2}, \RX_{b1}, \RX_{c1})=\RX \underbrace{\bM 1&0&0&0\\0 & 1&0&0\\ 0 & 0&1&0\\0&0&0&1\\ 0 & 0&0&0 \eM}_{\MM_2},$$ 
$$\RZ_{\opw}=(\RX_{a1}+\RX_{a2}, \RX_{a2}+\RX_{b1}+\RX_{d1})=\RX \underbrace{\bM 1&0\\1 & 1\\ 0 & 1\\0&0\\ 0 & 1 \eM}_{\MW}.$$
\end{example}

\subsection{Motivating example} The following example of a tree-PIN source with linear wiretapper appeared in our earlier work \cite{chan20secure}, where we constructed an optimal secure omniscience scheme. Let $V=\{1,2,3,4\}$ and  
     \begin{align}
        \RZ_{\opw} &= \RX_a+\RX_b+\RX_c, \\
        \RZ_1 &= \RX_a, \; \RZ_2 = (\RX_a, \RX_b),\; \RZ_3 = (\RX_b, \RX_c),\; \RZ_4 =  \RX_c,
      \end{align}
  where $\RX_a$, $\RX_b$ and $\RX_c$ are uniformly random and independent bits. The tree here is a path of  length $3$ (Fig.~\ref{fig:exampletree}) and the wiretapper observes a linear combination of all the edge random variables. For secure omniscience, terminals 2 and 3, using $n=2$ i.i.d. realizations of the source, communicate linear combinations of their observations. The communication is of the form $\RF^{(2)} =(\tRF_2^{(2)},\tRF_3^{(2)})$, where  $\tRF_2^{(2)} =\RX^2_a+\MM \RX^2_b$ and $\tRF_3^{(2)}=(\MM + \MI) \RX_b^2 +\RX_c^2$ with $\MM:=\bM 1 & 1\\ 1 & 0\eM$.  Since the matrices $\MM$ and $\MM+\MI$ are invertible, all the terminals can recover $\RZ_V^2$ using this communication. For example, user 1 can first  recover $\RX_b^2$ from $(\RX_a^2, \tRF_2^{(2)})$ as $\RX_b^2 = (\MM+\MI)(\RX_a^2+ \tRF_2^{(2)})$, then $\RX_b^2$ can be used along with $\tRF_3^{(2)}$ to recover $\RX_c^2$ as $\RX_c^2 = (\MM+\MI)\RX_b^2+ \tRF_3^{(2)}$.  More interestingly, this communication is ``aligned" with the eavesdropper's observations, since $\RZ^2_{\opw} = \tRF_2^{(2)}+\tRF_3^{(2)}$. This scheme achieves $R_L$, which is 1 bit. 
  
  For minimizing leakage, this kind of alignment must happen. For example, if $\RZ^2_{\opw}$ were not contained in the span of $\tRF_2^{(2)}$ and $\tRF_3^{(2)}$, then the wiretapper could infer a lot more from the communication.  Ideally, if one wants zero leakage, then $\RF^{(n)}$ must be within the span of $\RZ^n_{\opw}$, which is not feasible in many cases because, with that condition, the communication might not achieve omniscience in the first place. Therefore keeping this in mind, it is reasonable to assume that there can be components of $\RF^{(n)}$ outside the span of $\RZ^n_{\opw}$. But we look for communication schemes that span as much of $\RZ_{\opw}$ as possible. Such an alignment condition is used to control the leakage. In this particular example, it turned out that an omniscience communication that achieves $\rco$ can be made to align with the wiretapper side information completely, i.e., $H(\RZ^n_{\opw}|\RF^{(n)})=0$.  Motivated by this example, we show that it is always possible for some omniscience communication to achieve complete alignment with the wiretapper's observations within the class of tree-PIN sources with linear wiretapper.
  
  
\begin{theorem}\label{thm:cwsk:red} 
For a tree-PIN source $\RZ_V$ with linear wiretapper observing $\RZ_{\opw}$,
\begin{align*}
\wskc &= \min_{e \in E} H(\RY_e|\op{mcf}(\RY_e,\RZ_{\opw})),  \\ 
\rl &=\left(\sum_{e \in E}n_e -n_w\right)\log_2q -\wskc \text{ bits}.
\end{align*}
In fact, a linear non-interactive scheme  is  sufficient to achieve both $\wskc$ and $\rl$ simultaneously.
\end{theorem}

% The above theorem shows that the intrinsic upper bound on $\wskc$ holds with equality. In the multiterminal setting, the intrinsic bound that follows from \cite[Theorem 4]{csiszar04} is given by 
% \begin{align*}
%  \wskc(\RZ_V \|\RZ_{\opw}) \leq \min_{\RJ-\RZ_{\opw}-\RZ_V}\pkc(\RZ_V|\RJ).
% \end{align*}
%  This is analogous to the intrinsic bound for the two-terminal case \cite{maurer99intrinsic}. 
% For the class of tree-PIN sources with linear wiretapper,  when $\RJ^*= \left( \op{mcf}(\RY_e,\RZ_{\opw}) \right)_{e \in E}$, it can be shown that  $\pkc(\RZ_V|\RJ^*)= \min_{e \in E} H(\RY_e|\op{mcf}(\RY_e,\RZ_{\opw})) $. This can be derived using the characterization  in \cite{csiszar04} of the conditional minimum rate of communication for omniscience, $\rco(\RZ_V|\RJ^*)$. In fact, the same derivation can also be found in \cite{alireza19} for a  $\RJ$ that is obtained by passing the edge random variables through independent channels. In particular, $\RJ^{*}$ is a function of edge random variables $(\RY_e)_{e \in E}$ because $\op{mcf}(\RY_e, \RZ_{\opw})$ is a function of $\RY_e$. Therefore, we can see that $\pkc(\RZ_V|\RJ^*)$, which is an upper bound on $ \min_{\RJ-\RZ_{\opw}-\RZ_V}\pkc(\RZ_V|\RJ)$, matches with the $\wskc$ obtained from Theorem~\ref{thm:cwsk:red}.

 The theorem guarantees that we can achieve the wiretap secret key capacity in the tree-PIN case with linear wiretapper through a linear secure omniscience scheme, which establishes the duality between the two problems. This illustrates that omniscience can be helpful even beyond the case when there is no wiretapper side information.

 Our proof of Theorem~\ref{thm:cwsk:red} is through a reduction to the particular subclass of \emph{irreducible} sources, which we defined next.
 
\begin{definition} \label{def:irreducible}
 A tree-PIN source with linear wiretapper is said to be \emph{irreducible} if $\op{mcf}(\RY_e, \RZ_{\opw}) $ is a constant function for every edge $e \in E$ .
\end{definition}

Whenever there is an edge $e$ such that $\RG_e:=\op{mcf}(\RY_e, \RZ_{\opw}) $ is a non-constant function, the user corresponding to a vertex incident on $e$ can reveal $\RG_e$ to the other users. This communication does not leak any additional information to the wiretapper because $\RG_e$ is a function of $\RZ_{\opw}$. Intuitively, for further communication, $\RG_e$ is not useful and hence can be removed from the source. After the reduction, the m.c.f. corresponding to $e$ becomes a constant function. In fact, we can carry out the reduction until the source becomes irreducible. This idea of reduction is illustrated in the following example.

\begin{Example}
Let us consider a source $\RZ_V$ defined on a path of length 3, which is shown in Fig.~\ref{fig:exampletree}. Let $\RY_a = (\RX_{a1}, \RX_{a2})$, $\RY_b = \RX_{b1}$ and $\RY_c =\RX_{c1}$, where $\RX_{a1}$, $\RX_{a2}$, $\RX_{b1}$ and $\RX_{c1}$ are uniformly random and independent bits. 
\begin{figure}[h]
\centering
\resizebox{\width}{1cm}{

\section{An example}
\label{sec:6}
\begin{figure}[H]

\includegraphics[scale=.6]{schematic}
\centering
\caption{ A sketch of the example constructed in this section. The manifold decomposes into three pieces, left and right caps and a middle region. The loop drawn on the boundary of the left cap only bounds surfaces of high topological complexity that are at least partly contained in the right cap. This ensures the isoperimetric ratio of that loop is very small.}
\label{fig:example_sketch}
\end{figure}

The aim of this section is to show that the first positive eigenvalue of the 1-form Laplacian can vanish exponentially fast in relation to volume. This contrasts the behaviour of the first positive eigenvalue of the Laplacian on functions.

Our construction is similar to that in \cite{BD}. Essentially, we choose a hyperbolic 3-manifold with totally geodesic boundary and glue it to itself using a particular psuedoAnosov with several useful properties. By \cite{BMNS}, this family has geometry that up to bounded error can be understood in terms of a simple model family. Using this model family, we show that one can find curves with uniformly bounded length whose stable commutator length grows exponentially in the volume. We then use the spectral gap upper bound in Theorem A to conclude the first positive eigenvalue vanishes exponentially fast.

Throughout this section, we need to compare geodesic lengths in different submanifolds of a given manifold $M$. Let $|\cdot|_{X}$ denote the geodesic length of a homotopy class of curves relative endpoints in a manifold $X$ and $\text{length}(\cdot)$ be the length in $M$ of the curve. Similarly, when we compute stable commutator length for the fundamental group of a manifold $X$, which may or may not be a submanifold of $M$, we denote it $\scl_X$.

We will need that for certain curves, $\scl$ is comparable to length. We begin with a simple but essential technical lemma.
\begin{figure}[H]
\labellist
\small\hair 2pt
 \pinlabel {$a_0$} [ ] at 830 1100
 \pinlabel {$a_1$} [ ] at 1300 1100
 \pinlabel {$b_0$} [ ] at 830 1250
 \pinlabel {$b_1$} [ ] at 1300 1250
 \pinlabel {$t_1$} [ ] at 380 1290
 \pinlabel {$t_0$} [ ] at 380 1120
\endlabellist
\centering
\includegraphics[width = 15cm]{lemdiagram}
\caption{ Illustrating Lemma \ref{lem:6.1} , the rectangular base of the figure is part of the totally geodesic surface $S$ and the box is the corresponding part of the tubular neighborhood $N_\e(S)$ foliated by surfaces $S_t$ parallel to $S$. Drawn in the box is the surface $\Sigma$, which is transverse the foliation except at isolated points. The multicurve $a_0\cup a_1$ is part of a single level set $S_{t_0} \cap \Sigma$, but only $a_0$ is part of the curve $c_{t_0}$ described in the lemma, whereas the multicurve $b_0\cup b_1$ forms the multicurve $c_{t_1}$ in the lemma.}
\label{fig:box}
\end{figure}


\begin{lem} \label{lem:6.1} Let $M$ be a compact hyperbolic 3-manifold with totally geodesic boundary $ \d M = S$. Let $\e$ be smaller than the injectivity radius of $M$ and such that $N_\e(S)$ is an embedded tubular neighborhood. Let $\{S_t\}$ be the leaves of the foliation of $N_\e(S)$ by surfaces equidistant from $S$.  Let $\Sigma$ be a smooth incompressible proper not necessarily immersed surface in $M$ that is transverse to the foliation $\{S_t\}$ except at isolated points. Let $c = \d \Sigma$. By transversality, for generic $t$ the multicurve $c_t$ given by the part of $S_t\cap \Sigma$ that cobounds a subsurface of $\Sigma$ with $c = c_0$ is a smooth multicurve.  Let $T$ be the set (of full measure) of all $t\in[0,\e)$ such that $c_t$ is a smooth multicurve. Since $S$ is totally geodesic, each multicurve $c_t$ is homotopic to a possibly degenerate geodesic multicurve $\gamma_t$ in $S$. Let $\Sigma_\e$ be the part of $\Sigma$ contained in $N_\e(S)$. Then for $C = 1/\e$, we have that $$\inf\limits_{t\in T} |\gamma_t|_{S} \leq C \emph{Area}(\Sigma_\e).$$
\end{lem}

\begin{proof} The coarea formula implies the inequality $\inf\limits_{t\in T} |c_t|_{S_t} \leq C \text{Area}(\Sigma_\e)$.  Since $S$ is totally geodesic, for all $t\in T$, we have $|\gamma_t|_S\leq |c_t|_{S_t}.$  \end{proof}
Note that in the previous lemma, when $\inf\limits_{t\in T} |\gamma_t|_S$ is zero, because $\Sigma$ is incompressible and any loop with length less than $\inj(M)$ bounds a disk, every component of $\Sigma$ can either be homotoped to be disjoint from $N_\e(S)$ or be contained in $S$.

The next proposition requires a notion of geometric complexity for homology classes. For any compact Riemannian manifold $M$ one can define the stable norm on the first homology of $M$ (see \cite{Gromovmetric} Section 4C). The mass of a Lipschitz 1-chain $\alpha = \sum_i t_i\alpha_i$ in $M$ is defined to be $\text{mass}(\alpha) = \sum_i |t_i|\text{length}(\alpha_i).$ The mass of a class $a\in H_1(M)$ is then the infimal value of the mass of a chain $\alpha$ representing $a$.
For a class $a\in H_1(M)$, the stable norm of $a$ is then given by $$||a||_{s,M} = \inf\limits_{m>0}\frac{\text{mass}(m a)}{m}.$$

Stable commutator length can also be generalized to geodesic multicurves (see Section 2.6 of \cite{Calegari}), which can naturally be viewed as Lipschitz chains. Suppose $\gamma_i\in \pi_1 M$ and $ \sum_i n [\gamma_i] = 0$ in $H_1(M)$. Let $\gamma$ be the geodesic multicurve, which is not necessarily simple, consisting of the geodesic loops determined by $\gamma_i$. Say a surface $f:S\to M$ is admissible of degree $n(S)$ if it has no closed components and $\d S$ is a union of circles $S^1_i$ with $f|_{S^1_i}$ a degree $n(S)$ cover of $\gamma_i$.
Then we define stable commutator length of $\gamma$ to be $$\scl(\gamma) = \inf\limits_{S \text{ admissible}} \frac{\chi_-(S)}{2n(S)}.$$ When $\gamma$ is a single loop, this definition agrees with the usual definition of stable commutator length.

 \begin{prop} \label{prop:6.2} Let $M$ be a compact oriented hyperbolic 3-manifold with totally geodesic boundary $\d M = S$. Let $\gamma$ be a geodesic multicurve in $S$ that is rationally nullhomologous in $M$. Then there is a constant $D> 0$ depending only on $M$ such that $$||[\gamma]||_{s,S}\leq D\scl_M(\gamma).$$ \end{prop}

 \begin{proof} If $\gamma$ is nullhomologous in $S$, then the left hand side is zero and the inequality holds. Assume now that $[\gamma]\neq 0\in H_1(S)$. Fix $\delta>0$. Let $\Sigma_m$ be an incompressible admissible surface for $\gamma$ of degree $m = n(S)$ such that $\chi_-(\Sigma_m)/2m -\scl(\gamma) < \delta$. We can triangulate $\Sigma_m$ so that there is a single vertex on each boundary component. This triangulation has $4g +3b - 4$ faces, where $g$ is the genus of $\Sigma_m$ and $b$ the number of boundary components. We can then straighten this triangulation to obtain a piecewise totally geodesic triangulated surface. Replace $\Sigma_m$ with this surface. Since every face of this triangulation of $\Sigma_m$ is geodesic, every face has area at most $\pi$. Since there are $4g+3b - 4$ faces and $\chi_-(\Sigma_m) = 2g-2 + b$, we can estimate  $$\text{Area}(\Sigma_m) \leq 3\pi\chi_-(\Sigma_m).$$

We can perturb $\Sigma_m$ to obtain a smooth surface $\Sigma’_m$ that it is transverse the foliation of $N_\e(S)$ except at isolated points and in doing so increase the area by less than $\delta$. Let $\gamma_t$ be the family of multicurves in Lemma \ref{lem:6.1}  applied to $\Sigma’_m$. Since each curve $\gamma_t$ cobounds a surface in $S$ with $\d \Sigma_m$, they are homologous, thus $||[\gamma_t]||_{s,S} = m||[\gamma]||_{s,S}$. Since $||[\gamma_t]||_{s,S}\leq |\gamma_t|_S$,
Lemma \ref{lem:6.1}  implies that $$m||[\gamma]||_{s,S}\leq C\text{Area}(\Sigma_m’) \leq C \text{Area}(\Sigma_m) + C\delta \leq 3C\pi\chi_-(\Sigma_m) + C\delta.$$
From this we get  $$||[\gamma]||_{s,S}\leq 6C\pi\chi_-(\Sigma_m)/2m + C\delta/m\leq 6C\pi\scl_M(\gamma) + 6C\pi\delta +C\delta/m.$$
Since the stable commutator length of a nontrivial rational commutator is bounded away from zero by a constant only depending on $M$, by Theorem 3.9 in \cite{Calegari}, we can replace $C$ with a larger constant $D$ such that $$||[\gamma]||_{s,S}\leq D\scl_M(\gamma),$$ as desired. \end{proof}

 We now introduce the family of manifolds that we use in our construction. The family $\{W_n\}$ of manifolds we study are easily understood using the model manifold theory of \cite{BMNS}. In particular, there is a $K$-biLipschitz map between $W_n$ and a model manifold $M_n$, where $K$ is independent of $n$. The base of the construction is Thurston’s tripus manifold $W$ (see \cite{thurstonbook}, Section 3.3.12), a hyperbolic manifold with totally geodesic boundary, and a psuedoAnosov homeomorphism $f$ of the boundary surface $\d W$. The model manifold $ M_n$ is a degree $n$ cyclic cover of the mapping torus $M_{f}$ cut open along a fiber with two oppositely oriented copies of $W$, denoted $W^+$ and $W^-$ glued as described in \cite{BMNS} Section 2.15 to the two boundary components of the cut open mapping torus. This decomposes $W_n$ into three pieces, a product region $S\times [0, n]$ and the caps $W^+$ and $W^-$ in a metrically controlled way. It will be convenient to set $M^+ = W^+\subset M_n$ and $M^- = W^-\subset M_n$ when talking about the caps of the model manifold $M_n$ for fixed $n$, and to let $W^+$ and $W^-$ denote the images of these spaces under the natural inclusion into $W_n$.

 \vspace{1cm}
\begin{figure}[H]
\labellist
\small\hair 2pt
 \pinlabel {$M^+$} [ ] at 950 1100
 \pinlabel {$M^-$} [ ] at 2000 1100
 \pinlabel {$S\times[0,n]$} [ ] at 1500 1600
\endlabellist
\centering
\includegraphics[scale=.15]{modelscheme}
\caption{ A schematic picture of the model manifold $M_n$ with caps $M^+$ and $M^-$ two oppositely oriented copies of the tripus manifold.}
\label{fig:caps_schematic}
\end{figure}

Given a multicurve $c$ in $M^{\pm}$, we say $c$ \textbf{bounds on both sides} if there are incompressible surfaces $S^+$ in $M^+$ and $S^-$ in $M^-$ both with boundary homotopic to $c$.

We encode the construction and its essential properties in the following proposition.

\begin{prop}  \label{prop:6.3}There is a family $\{W_n\}$ of closed hyperbolic 3-manifolds with injectivity radius uniformly bounded below and volume growing linearly in $n$ constructed from the tripus and a pseudoAnosov $f$ as described above. Each manifold $W_n$ is $K$-biLipschitz equivalent to the model manifold $M_n$ for some constant $K$ independent of $n$. Any homologically nontrivial loop in $H_1(\d W^{\pm})$ that bounds a surface in $M^{\pm}$ cannot bound on both sides. The pseudoAnosov $f$ is such that for any nonzero class $a\in H_1(\d W^+)$, the stable norm of $f_*^n(a)$ grows exponentially.
\end{prop}

\begin{proof}
Let $W$ be Thurston’s tripus manifold, a compact hyperbolic 3-manifold with totally geodesic boundary a genus 2 surface for which the inclusion map $H_1(\d W;\Z)\to H_1(W;\Z)$ is onto.
The homology of the boundary $\d W$ decomposes as the direct sum of rank 2 submodules $U$ and $V$, where $V\subset H_1(\d W)$ is the image of the boundary map $\d : H_2(W,\d W;\Z)\to H_1(\d W;\Z)$ (which is also the kernel of the inclusion $H_1(\d W)\to H_1(W)$) and $U$ is a compliment of $V$ (note that the inclusion map $H_1(\d W)\to H_1(W)$ restricted to $U$ is an isomorphism).
Let $S$ be a genus 2 surface, which we will use to mark the boundaries of $W^{+}$ and $W^-$. Assume $H_1(S;\Z)$ is generated by $e_1,~e_2,~e_3,~e_4$. Choose a marking $S\to \d W^+$ so in $W^+$ one has $U = \langle e_1,e_2\rangle $ and $V = \langle e_3, e_4\rangle$.
Similarly, choose a marking $S\to \d W^-$ so that in $W^-$ one has $V = \langle e_1,e_2\rangle $ and $U = \langle e_3, e_4\rangle$. We then define $$W_n = W^+\cup_{f^n}W^-$$ where $f:S\to S$ is a pseudo-Anosov that acts on $H_1(S)$ by the symplectic matrix\[ F = \begin{pmatrix}
 2 &  1 & 0 & 0 \\
 1 & 1 & 0 & 0 \\
 0 & 0 & 1 & -1\\
 0 & 0 & -1 & 2
\end{pmatrix} \] For the existence of such a pseudoAnosov mapping class, see the proof Lemma 7.1 in \cite{BD}. This matrix preserves the subspace decomposition above, and so ensures that every curve in $\d W^{\pm}$ that is not nullhomologous in $\d W^{\pm}$ but bounds a surface in $M^{\pm}$ cannot bound on both sides.

The mapping class $f$ acts as an Anosov matrix on $U$ and $V$. This ensures the standard Euclidean $\ell^2$-norm $||F^n(a)||_E$ of an element $a\in H_1(S)$ grows exponentially in $n$ (indeed, for our choice of $F$, it grows like $(\frac{3+\sqrt{5}}{2})^n$). Since norms on finite dimensional real vector spaces are comparable, there is a constant comparing the stable norm induced by the metric inherited from $W$ to the standard Euclidean $\ell^2$-norm on $H_1(S)$.

Lemma 7.3 in \cite{BD} explains how Theorem 8.1 in \cite{BMNS} implies that for large $n$ the manifolds $W_n$ admit a $K$-biLipschitz diffeomorphism $\mu$ from the model manifold $ M_n$ as described above. After increasing $K$, we can drop the large $n$ condition. This then also implies the linear volume growth and injectivity radius bounds.
\end{proof}

\begin{remark}
Using the model manifold, one can easily estimate the Cheeger constant of $W_n$, which will decay like $1/n$.
\end{remark}


\begin{mainthm}  \label{thm:C} The family $W_n$ of closed hyperbolic 3-manifolds from Proposition \ref{prop:6.3}  has 1-form Laplacian spectral gap that vanishes exponentially fast in relation to volume:
$$\sqrt{\lambda(W_n)}\leq B\vol(W_n)e^{-r\vol(W_n)},$$
where $r$ and $B$ are positive positive constants and $\lambda(W_n)$ is the first positive eigenvalue of the 1-form Laplacian on $W_n$.
\end{mainthm}


\begin{proof}
We continue using notation introduced in the previous propositions. Take $\gamma$ in $\d M^+\subset M_n$ to be an embedded geodesic loop representing the class $e_1 \in U \subset H_1(\d W^+)$. Recall from the proof of Proposition \ref{prop:6.3} that $\gamma\subset \d M^+$ does not bound a surface in $M^+$ but that $f^n(\gamma)\subset \d M^-$ bounds a surface in $M^-$. Let $\alpha_n = f^n(\gamma)\subset \d M^- \subset M_n$. Note that $\alpha_n$ and $\gamma$ are isotopic in $M_n$.

Fix $\delta>0$. Consider some positive integer $m$ and  incompressible surface $\Sigma_m$ bounding $\alpha_n^m$ in $M_n$ with $\chi_-(\Sigma_m)/2m - \scl_{M_n}(\gamma) < \delta$ and which minimizes $\chi_-$ among surfaces with boundary $\alpha_n^m$. We can then replace $\Sigma_m$ with a homotopic surface that pushes the boundary of $\Sigma_m$ into the interior of $M^-$ and which intersects $\d M^-$ transversely and essentially in both $\d M^-$ and $\Sigma_m$.
We can then attach an annulus to $\Sigma_m$ cobounding $\alpha_n^m$ and the boundary of the modified surface $\Sigma_m$. This new $\Sigma_m$ bounds $\alpha_n^m$ with a collar neighborhood of the boundary contained entirely in $M^-$ and intersects $\d M^-$ transversely in a union of loops essential in both $\Sigma_m$ and $\d M^-$.

We focus on the portion of $\Sigma_m$ that lies in $M^-$. Define $\Sigma^-_m =\Sigma_m\cap M^-$. If $\Sigma_m$ is contained in $M^-$, then Proposition \ref{prop:6.2}  applied to $\alpha_n^m$ in $M^-$ implies that
$$||[\alpha_n^m]||_{s,\d M^-} = m||[\alpha_n]||_{s,\d M^-} \leq m\scl_{M^-}(\alpha_n)\leq D\chi_-(\Sigma_m^-) = D\chi_-(\Sigma_m) ,$$
where $||\cdot||_{s, \d M^-}$ is the stable norm of $H_1(\d M^-)$. Since $\chi_-(\Sigma_m)/m- \scl_{M_n}(\gamma)\leq \delta$, we conclude $$||[\alpha_n]||_{s,\d M^-}\leq D\scl_{M_n}(\gamma) + D\delta.$$

Our goal now is to get this same estimate for the other possible ways $\Sigma_m$ sits in $M_n$.
\vspace{1cm}
\begin{figure}[H]
\labellist
\small\hair 2pt
 \pinlabel {$M^+$} [ ] at 900 1560
 \pinlabel {$S\times[0,n]$} [ ] at 1480 1650
 \pinlabel {$M^-$} [ ] at 2000 1560
 \pinlabel {$\alpha$} [ ] at 1750 1050
\endlabellist
\centering
\includegraphics[scale=.2]{schematic5}
\caption{ A schematic picture of the simplest case of a surface bounding $\alpha$ in $M^-$.}
\label{fig:alpha_bounds}
\end{figure}

Consider the case that $\Sigma_m$ does not lie entirely in $M^-$. There are two possibilities. The first involves the surface $\Sigma_m$ passing into the product region but not intersecting $M^+$. In this case the surface can be homotoped to lie in $M^-$, so that Proposition \ref{prop:6.2} applies, giving the desired estimate as in the previous case.
\vspace{2cm}
\begin{figure}[H]
\labellist
\small\hair 2pt
 \pinlabel {$M^+$} [ ] at 900 1560
 \pinlabel {$S\times[0,n]$} [ ] at 1480 1650
 \pinlabel {$M^-$} [ ] at 2000 1560
 \pinlabel {$\alpha$} [ ] at 1750 850
\endlabellist
\centering

\includegraphics[scale=.2]{schematic3}
\caption{ A schematic picture of a surface bounding $\alpha$ that passes back into $M^-$ but does not pass into $M^+$.}
\label{fig:pass_back_schematic}
\end{figure}

\vspace{2cm}
\begin{figure}[H]
\labellist
\small\hair 2pt
 \pinlabel {$M^+$} [ ] at 900 1560
 \pinlabel {$S\times[0,n]$} [ ] at 1480 1650
 \pinlabel {$M^-$} [ ] at 2000 1560
 \pinlabel {$\alpha$} [ ] at 1750 850
 \pinlabel {$c_1$} [ ] at 1750 1285
 \pinlabel {$c_0$} [ ] at 1750 1070

\endlabellist
\centering

\includegraphics[scale=.2]{schematic6}
\caption{ A schematic picture of a surface bounding $\alpha$ that passes back into $M^+$. Notice the multicurve $c^- = c_0\cup c_1$ bounds surfaces in $M^+$ and $M^-$, so is homologically trivial. }
\label{fig:bounds_both_sides}
\end{figure}



The second possibility concerns the surface $\Sigma_m$ crossing through the product region into $M^+$ with an essential intersection with $\d \Sigma^+$. In this case, we will see that the surface $\Sigma_m^-$ has boundary homologous to $\alpha_n^m$, which will allow us to apply Proposition \ref{prop:6.2} to obtain the desired estimate. By construction, a sufficiently small collar $C$ of the boundary $\d \Sigma_m$ in $\Sigma_m$ maps into $M^-$, so in particular, a subsurface of $\Sigma_m^-$ has some boundary component that maps to $\alpha_n^m$. That boundary component can be closed by attaching a surface $S^-$ that bounds $\alpha_n^m$ in $M^-$ to $\Sigma_m$.
From this, we see that the multicurve $c^- = \d \Sigma^-_m -\alpha_n $ bounds surfaces in $M^+$ and $M^-$.
Thus by Proposition \ref{prop:6.3}, $c^-$ must be homologically trivial in $\d M^{-}$. Let $x = \d\Sigma_m^- = c^- + \alpha_n^m$. Since $c^-$ is nullhomologous, $||[x]||_{s,\d M^-} = ||[\alpha_n^m]||_{s,\d M^-} = m||[\alpha_n]||_{s,\d M^-}.$
By Proposition \ref{prop:6.2} , $||[x]||_{s,\d M^-}\leq D\scl_{M^-}(x).$ Since $x$ is essential in $\Sigma_m$, we get that $\chi_-(\Sigma_m^-) \leq \chi_-(\Sigma_m)$, then using that $\chi_-(\Sigma_m)/2m - \delta \leq \scl_{M_n}(\alpha_n)$,
we obtain $\scl_{M^-}(x) \leq \chi_-(\Sigma_m^-)/2 \leq m\scl_{M_n}(\alpha_n) + \delta m$.
Putting this all together and dividing by $m$, we get that $$||[\alpha_n]||_{s,\d M^-}\leq D\scl_{M_n}(\alpha_n) + D\delta.$$

We therefore have in each case that there is a constant $D$ independent of $n$ such that $$||[\alpha_n]||_{s,\d M^-}\leq D\scl_{M_n}(\alpha_n) + D\delta.$$ By Proposition \ref{prop:6.3} , $||[\alpha_n]||_{s,\d M^-} = ||[f^n(\gamma)]||_{s,\d M^+}$ grows exponentially in $n$.
Thus for some constants $B>0$ and $r >0$, we have $$Be^{rn}\leq D\scl_{M_n}(\gamma) + D\delta,$$ where we use that $\gamma$ and $\alpha_n$ are homotopic in $M_n$. Using the injectivity radius lower bound and Theorem 3.9 of \cite{Calegari}, we can increase $D$ and drop the additive constant in this inequality. By Proposition \ref{prop:6.3} , the volume growth of the $W_n$ is proportional to $n$ so there is a constant $C$ such that $\vol(W_n)\leq Cn$.
Additionally, using the $K$-biLipschitz comparison of Proposition \ref{prop:6.3} , the length of $\gamma$ in $W_n$ is bounded from above by $2K|\gamma|_{W}$, where $W$ is the tripus. As a result, Theorem A implies that the spectral gap for the 1-form Laplacian of the manifolds $W_n$ vanishes exponentially fast in $n$.
In particular, we have \[\sqrt{\lambda(W_n)} \leq A\vol(W_n)\frac{|\gamma|_{W_n}}{\scl_{W_n}(\gamma)} \leq 2KACB^{-1}D|\gamma|_Wne^{-rn},\] so the result holds after redefining $B$ to be $2KACB^{-1}D|\gamma|_W.$

\end{proof}
}
\caption{A path of length 3}
\label{fig:exampletree}
 \end{figure}
 If $\RZ_{\opw}=\RX_{b1}+\RX_{c1}$, then the source is irreducible because $\op{mcf}(\RY_e, \RZ_{\opw})$ is a constant function for all $e \in \{a,b,c\}$. 
 
 However if  $\RZ_{\opw}=(\RX_{a1}+\RX_{a2},  \RX_{b1}+\RX_{c1})$, then the source is not irreducible, as $\op{mcf}(\RY_a, \RZ_{\opw}) =\RX_{a1}+\RX_{a2}$, which is a non-constant function. An equivalent representation of the source is
 $\RY_a = (\RX_{a1}, \RG_{a})$, $\RY_b = \RX_{b1}$, $\RY_c =\RX_{c1}$ and $\RZ_{\opw}=(\RG_{a}, \RX_{b1}+\RX_{c1})$, where $\RG_{a}=\RX_{a1}+\RX_{a2}$, which is also a uniform bit independent of $(\RX_{a1}, \RX_{b1}, \RX_{c1})$. So, for omniscience, user 2 initially can reveal $\RG_{a}$ without affecting the information leakage as it is completely aligned to $\RZ_{\opw}$. Since everyone has $\RG_a$, users can just communicate according to the omniscience scheme corresponding to the source without $\RG_a$. Note that this new source is irreducible.
\end{Example}

The next lemma shows that the kind of reduction to an irreducible source used in the above example is indeed optimal in terms of $R_L$ and $\wskc$ for all tree-PIN sources with linear wiretapper.
\begin{lemma}\label{lem:irred} 
 If a tree-PIN source with linear wiretapper $(\RZ_V,\RZ_{\opw})$ is not irreducible then there exists an irreducible source $(\tRZ_V, \tRZ_{\opw})$ such that 
 \begin{align*}
\wskc(\RZ_V\| \RZ_{\opw}) = \wskc(\tRZ_V\|\tRZ_{\opw}),\\ \rl(\RZ_V\|\RZ_{\opw}) = \rl(\tRZ_V\|\tRZ_{\opw}),\\
H(\RY_e|\op{mcf}(\RY_e,\RZ_{\opw})) = H(\tRY_e)
\end{align*}
for all $e \in E$.
\end{lemma}

\begin{proof}
See Appendix~\ref{lem:proof:irred}.
\end{proof}
Note that, in the above lemma, the scheme that achieves $\rl(\RZ_V\|\RZ_{\opw})$  involves revealing the reduced m.c.f. components first and then communicating according to the scheme that achieves $\rl(\tRZ_V\|\tRZ_{\opw})$. As a consequence of Lemma~\ref{lem:irred}, to prove Theorem~\ref{thm:cwsk:red}, it suffices to consider only irreducible sources. For ease of reference, we re-state the theorem for irreducible sources below.

\begin{theorem}\label{thm:cwsk:irred} 
If a tree-PIN source $\RZ_V$ with linear wiretapper $\RZ_{\opw}$ is irreducible then 
\begin{align*}
\wskc &= \min_{e \in E} H(\RY_e)=\skc, \\
\rl &=\left(\sum_{e \in E}n_e -n_w\right)\log_2q- \wskc\text{ bits},
\end{align*}
where $\skc$ is the secret key capacity of Tree-PIN source without the wiretapper side information~\cite{csiszar04}.
\end{theorem}

\begin{IEEEproof}[Proof outline]
The main component of the proof is the achievability part, which involves a construction of an omniscience scheme with leakage rate $ \left(\sum_{e \in E}n_e -n_w\right)\log_2q - \min_{e \in E} H(\RY_e)$. Since $\wskc$ is upper bounded by $\skc = \min_{e \in E} H(\RY_e)$, we have $ \rl \leq \left(\sum_{e \in E}n_e -n_w\right)\log_2q - \min_{e \in E} H(\RY_e) \leq \left(\sum_{e \in E}n_e -n_w\right)\log_2q- \wskc$.  This upper bound, together with the $\rl$ lower bound in Theorem~\ref{thm:RL:lb}, proves the theorem.

In the construction of this scheme, we start by assuming a certain linear structure for the communication $\RF^{(n)}$, with the coefficients of the linear combinations involved in the communication taken to be variables (termed ``communication coefficients'') whose values need to be determined. We then argue that the desired leakage rate can be achieved if we additionally impose two conditions on $\RF^{(n)}$, namely, perfect omniscience and perfect alignment. These conditions translate to certain linear-algebraic constraints on the communication coefficients in $\RF^{(n)}$. So it is enough to find an $\RF^{(n)}$ with its communication coefficients satisfying these constraints. To that end, we use the probabilistic method to argue that if the blocklength $n$ of the source is large enough, then there indeed exists such a realization of $\RF^{(n)}$. 

For clarity of exposition, we first execute this proof in the case of tree-PIN sources in which the underlying tree is a simple path, before generalizing it to the case of trees. The details are given in Appendix~\ref{thm:proof:cwsk:irred}.
\end{IEEEproof}

% \begin{proof}
% See Appendix~\ref{thm:proof:cwsk:irred}.
% \end{proof}

Theorem~\ref{thm:cwsk:irred} shows that, for irreducible sources, even when the wiretapper has side information, the users can still extract a key at rate $\skc$. In terms of secret key generation, the users are not really at a disadvantage if the wiretapper has linear observations.

\subsection{Constrained wiretap secret key capacity of tree-PIN source with linear wiretapper}
Secure omniscience in fact plays a role even in achieving the constrained wiretap secret key capacity of tree-PIN source with linear wiretapper. The constrained wiretap secret key capacity, denoted by $\wskc(R)$, is defined as in \eqref{eq:wskc} but with the supremum over all SKA schemes with $\limsup\frac{1}{n}\log|\mc{F}^{(n)}|<R$ where $\mc{F}^{(n)}$ is the alphabet of $\RF^{(n)}$. The following theorem gives a single-letter expression for the constrained wiretap secret key capacity, whose form is reminiscent of the constrained secret key capacity, \cite[Theorem~4.2]{chan19}. The proof involves a construction of a secure omniscience communication scheme for a part of the source.

\begin{theorem}\label{thm:rateconstrained_treepin}
Given a tree-PIN source $\RZ_V$ with a linear wiretapper $\RZ_{\opw}$, we have
\begin{align*}
\wskc(R) = \min \left\{\frac{R}{|E|-1}, \wskc\right\},
\end{align*}
where $R$ is the total discussion rate and $\wskc =\min_{e \in E}H(Y_e|\op{mcf}(Y_e,Z_{\opw}))$, which is the unconstrained wiretap secret key capacity.
\end{theorem}


\begin{IEEEproof}[Proof outline]
As in Theorem~\ref{thm:cwsk:red}, this proof is also divided into two parts. In the first part, which is 
Theorem~\ref{lem:irred:rate}, we argue that removing an edge random variable that is also available with the wiretapper does not affect the constrained wiretap secret key capacity $\wskc(R)$. The argument involves showing that any valid SKA scheme on the original model can be converted into a valid SKA scheme on the reduced model, and vice versa.

In the second part, which is Theorem~\ref{thm:rate:irreducible}, we prove the statement of Theorem~\ref{thm:rateconstrained_treepin} for irreducible sources. The converse part follows from the constrained secret key capacity result in the case of tree-PIN sources without wiretapper side information, \cite[Theorem~4.2]{chan19}. The achievability part pivots on the argument that the rate pair $((|E|-1)\wskc,\wskc)$, where $\wskc=\min_{e \in E}H(Y_e)$, is achievable, as the rest of the curve follows from a time sharing argument.
% Note that, in the proof of Theorem~\ref{thm:cwsk:irred}, we showed that the key rate $\min_{e \in E}H(Y_e)$ is achievable using a  communication of rate $\left(\sum_{e \in E}n_e\right) \log_2q  - \min_{e \in E} H(\RY_e) \geq (|E|-1) \min_{e \in E} H(\RY_e)$. 
To show the achievability of this rate pair, we ignore some of the edge components of the source $(\RZ_V, \RZ_{\opw})$ to obtain a new source $(\RZ'_V, \RZ'_{\opw})$, which will be used for the purpose of secret key generation. In fact, we retain only $s=\min_{e \in E}n_e$ components for each edge. We then argue that for the new source $(\RZ'_V, \RZ'_{\opw})$, we can generate a key of rate $\wskc$ with a communication of rate $(|E|-1)\wskc$, if we use the key generation scheme used in the proof of Theorem~\ref{thm:cwsk:irred}.

Theorem~\ref{thm:rateconstrained_treepin} follows from combining both these parts. The complete details are given in Appendix~\ref{app:thm:proof_rateconstrained_treepin}.
\end{IEEEproof}

%  \begin{proof}
% See Appendix~\ref{app:thm:proof_rateconstrained_treepin}.
% \end{proof}
