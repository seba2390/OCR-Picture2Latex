In this section, we describe two different scenarios, namely wiretap secret key agreement and secure omniscience, in the context of the multiterminal source model. In this model,  the terminals communicate publicly using their correlated observations to compute functions securely from the eavesdropper, who has access to the public communication along with some side information.  More precisely, let $V=[m]:=\left\lbrace1, \ldots, m\right\rbrace$ be the set of users,  and let $\opw$ denote the wiretapper.  Let  $\RZ_1,\ldots, \RZ_m$ and $\RZ_{\opw}$ be the random variables  taking values in finite alphabets $\mc{Z}_1,\ldots, \mc{Z}_m$ and $\mc{Z}_{\opw}$ respectively, and their joint distribution is given by $P_{\RZ_1 \ldots \RZ_m \RZ_{\opw}}$. Let $\RZ_V := (\RZ_i: i \in V)$ and $\RZ_i^n$ denote the $n$ i.i.d. realizations  of $\RZ_i$. For $i \in V$, user $i$ has access to the random variable $\RZ_i$, and the wiretapper observes $\RZ_{\opw}$. Upon observing $n$ i.i.d. realizations, the users communicate interactively using their observations, and possibly independent private randomness, on the noiseless and authenticated channel. In other words, the communication made by a user in any round depends on all the previous rounds' communications and  the user's own observations.  Let $\RF^{(n)}$ denotes this interactive communication. We say $\RF^{(n)}$ is \emph{non-interactive}, if it is of the form $(\tRF_i^{(n)}: i \in V)$, where $\tRF_i^{(n)}$ depends only on $\RZ_i^n$ and the  private randomness of user $i$. Note that the eavesdropper has access to the pair $(\RF^{(n)}, \RZ_{\opw}^n)$. At the end of the communication, each user outputs a value in a finite set using its observations and $\RF^{(n)}$. For example, user $i$ outputs $\RE_i^{(n)}$ using $(\RF^{(n)}, \RZ_i^n)$ and its private randomness { $\RS_i$, i.e., $\RE_i^{(n)} = \psi_i(\RF^{(n)}, \RZ_i^n, \RS_i)$ for some function $\psi_i$}. See Fig.~\ref{fig:system}.
\begin{figure}[h]
\centering
\resizebox{0.85\width}{!}{\input{Figures/sys.tikz}}
\caption{Multiterminal source model with wiretapper side information. The terminals interactively discuss over a public channel using their observations from a correlated source to  compute their respective functions.}
\label{fig:system}
 \end{figure}
\subsection{Secure Omniscience}\label{subsec:omniscience}
In the secure omniscience scenario, each user tries to recover the observations of all the users other than the wiretapper. We say that $(\RF^{(n)}, \RE_1^{(n)}, \ldots, \RE_m^{(n)})_ {n \geq 1}$  is an \emph{omniscience scheme} if it satisfies the recoverability condition for omniscience:
\begin{align}\label{eq:omn:recoverability}
\liminf_{n \to \infty} \Pr(\RE_1^{(n)} = \dots =\RE_m^{(n)} = \RZ_V^n) = 1.
\end{align}
The communication $\RF^{(n)}$ in an omniscience scheme is called an \emph{omniscience communication}.

The \emph{minimum leakage rate for omniscience} is defined as 
\begin{align}
\begin{split}
 \rl&:= \inf  \biggl\lbrace \limsup_{n \to \infty} \frac{1}{n}I(\RF^{(n)} \wedge \RZ_V^n|\RZ_{\opw}^n) \biggr\rbrace, \label{eq:rl}
 \end{split}
\end{align}
where the infimum is over all omniscience schemes. We sometimes use $\rl(\RZ_V\|\RZ_{\opw})$ instead of $\rl$ to make the source explicit. The \emph{minimum rate of communication for omniscience} $\rco(\RZ_V)$, or simply $\rco$, is defined as \cite{csiszar04}
  \begin{align}
    \begin{split}
 \rco&:= \inf  \biggl\lbrace \limsup_{n \to \infty} \frac{1}{n}\log|\mc{F}^{(n)}| \biggr\rbrace, \label{eq:rco_def}
 \end{split}
\end{align}
where $\mc{F}^{(n)}$ is the range of $\RF^{(n)}$, and the infimum is over all omniscience schemes. It is known \cite[Proposition~1]{csiszar04} that $\rco$ is given by the solution to the following linear program:
\begin{align*}
    \rco = \min\left\{\sum \limits_{i \in V} R_i \Bigm\vert \sum \limits_{i \in B}R_i \geq H(\RZ_B|\RZ_{B^c}), \: \forall B\subsetneq V  \right\}.
    \label{rco_expression}
\end{align*}

The \emph{conditional} minimum rate of communication for omniscience, $\rco(\RZ_V|\RJ)$, is used in situations where all the users have access to a common random variable $\RJ^n$ along with  their private observations. This means that user $i$ observes $(\RJ^n, \RZ_i^n)$. 

Observe that for any source, we have
\begin{align}\label{eq:rl_rco}
    \rl  \leq \rco,
\end{align}
which follows easily from \eqref{eq:rl} and \eqref{eq:rco_def} as $I(\RF^{(n)} \wedge \RZ_V^n|\RZ_{\opw}^n)\leq H(\RF^{(n)}) \leq \log|\mc{F}^{(n)}|$, where $\RF^{(n)}$ is an omniscience communication taking values in the set $\mc{F}^{(n)}$. 

To get a sense of how $\rl(\RZ_V\|\RZ_{\opw})$ behaves with respect to the correlation between $\RZ_{\opw}$ and $\RZ_V$, consider another wiretapper side information $\tRZ_{\opw}$ that is less correlated with $\RZ_{V}$ than  $\RZ_{\opw}$, in the sense that they form the Markov chain $\tRZ_{\opw} \textrm{ -- } \RZ_{\opw} \textrm{ -- } \RZ_V$. Using the data processing inequality and the fact that any omniscience communication $\RF^{(n)}$ is a function only of $\RZ_V^n$ and private randomness (which is independent of $\RZ_V^n, \RZ_{\opw}^n, \tRZ_{\opw}^n$), we can infer that $I(\RF^{(n)} \wedge \RZ_V^n|\RZ_{\opw}^n)\leq I(\RF^{(n)} \wedge \RZ_V^n|\tRZ_{\opw}^n)$. We conclude, via \eqref{eq:rl} and \eqref{eq:rl_rco}, that
\begin{align*}
    \rl(\RZ_V\|\RZ_{\opw})\leq \rl(\RZ_V\|\tRZ_{\opw}) \leq \rco(\RZ_V).
\end{align*}
%
At the end of the next subsection, we will show that when $\RZ_{\opw}$ is independent of $\RZ_V$, \eqref{eq:rl_rco} holds with equality.


\subsection{Wiretap Secret Key Agreement}\label{sec:wska:def}
In the wiretap secret key agreement, each user tries to compute a common function, which is called a \emph{key}, that is kept secure from the wiretapper. Specifically, we say that $(\RF^{(n)}, \RE_1^{(n)}, \ldots, \RE_m^{(n)})_ {n \geq 1}$  is a \emph{wiretap secret key agreement (SKA) scheme} if there exists a sequence $(\RK^{(n)})_{n \geq 1}$  such that
\begin{subequations}
\label{eq:sk:constraints}
\begin{align}
\liminf_{n \to \infty} \Pr(\RE_1^{(n)} = \dots =\RE_m^{(n)} = \RK^{(n)}) = 1 \label{eq:sk:recoverability},\\
\limsup_{n \to \infty}\left[\log |\mc{K}^{(n)}| - H(\RK^{(n)}| \RF^{(n)},\RZ_{\opw}^n)\right] =0\label{eq:sk:secrecy},
\end{align}
where $|\mc{K}^{(n)}|$ denotes the cardinality of the range of $\RK^{(n)}$. Conditions \eqref{eq:sk:recoverability} and \eqref{eq:sk:secrecy} are referred to as the key recoverability condition and the secrecy condition of the key, respectively. 
\end{subequations}
The \emph{wiretap secret key capacity}  is defined as
\begin{align}
 \wskc:= \sup \left\lbrace \liminf_{n \to \infty} \frac{1}{n} \log |\mc{K}^{(n)}| \label{eq:wskc}\right\rbrace
\end{align}
where the supremum is over all SKA schemes. The quantity $\wskc$ is also sometimes written as $\wskc(\RZ_V\|\RZ_{\opw})$. In \eqref{eq:wskc}, we use $\skc$ instead of $\wskc$, when the wiretap side information is set to a constant.   Similarly, we use $\pkc(\RZ_V| \RJ)$  in the case when wiretap side information is  $\RZ_{\opw}= \RJ$ and all the users have the shared random variable $\RJ$ along with  their private observations $\RZ_i$. The quantities $\skc$ and $\pkc(\RZ_V|\RJ)$ are referred to  as \emph{secret key capacity} of  $\RZ_V$, and \emph{private key capacity} of $\RZ_V$ with compromised-helper side information $\RJ$ respectively. 

The following theorem gives lower and upper bounds on the minimum leakage rate for omniscience for a general source $(\RZ_V,\RZ_{\opw})$. 

\begin{theorem}\label{thm:RL:lb}
    For a general source $(\RZ_V,\RZ_{\opw})$,
      \begin{align}
      H(\RZ_V|\RZ_{\opw}) - \wskc \leq \rl \leq H(\RZ_V|\RZ_{\opw}).\label{eq:RL:lb}
        \end{align}
\end{theorem}
 \begin{IEEEproof}[Proof sketch]
The upper bound on $\rl$ follows from \eqref{eq:rl}, upon noting that  $\frac{1}{n}I(\RF^{(n)} \wedge \RZ_V^n|\RZ_{\opw}^n)\leq H(\RZ_V|\RZ_{\opw})$. 
%
For the lower bound, the underlying idea of the proof is that given a discussion scheme that achieves $\rl$, one can apply privacy amplification to extract a secret key of rate $H(\RZ_V|\RZ_{\opw})-\rl$ from the recovered source. The details of this argument are given in Appendix~\ref{app:thm:proof_rl_bound}.
\end{IEEEproof}


% Given a discussion scheme that achieves $\rl$, one can apply privacy amplification~\cite[Lemma~B.2]{csiszar04} to extract a secret key of rate $H(\RZ_V|\RZ_{\opw})-\rl$ from the recovered source. Since the secret key rate thus achieved is bounded above by $\wskc$, we obtain the lower bound on $\rl$. The upper bound on $\rl$ follows from \eqref{eq:rl}, upon noting that  $\frac{1}{n}I(\RF^{(n)} \wedge \RZ_V^n|\RZ_{\opw}^n)\leq H(\RZ_V|\RZ_{\opw})$.

\begin{remark}\renewcommand{\qed}{}
Note that the achievable key rate is, intuitively, the total amount of randomness in the recovered source $\RZ_V$ that is not in the wiretapper's side information $\RZ_{\opw}$ nor revealed in public. 
\end{remark}

% Therefore, we have 
% \begin{align*}
%     \rl  \leq {\min} \{\rco, H(\RZ_V|\RZ_{\opw})\}.
% \end{align*}

In Theorems~1-3 of \cite{csiszar04}, Csisz\'ar and Narayan showed that $\skc(\RZ_V)=H(\RZ_V)-\rco(\RZ_V)$ and $ \pkc(\RZ_V | \RZ_{\opw}) = H(\RZ_V|\RZ_{\opw}) - \rco(\RZ_V|\RZ_{\opw})$. They also proved, in Theorem~4 of \cite{csiszar04}, that $\wskc(\RZ_V \| \RZ_{\opw}) \leq \min\{\skc(\RZ_V),\pkc(\RZ_V | \RZ_{\opw})\}$, which implies that $\rl(\RZ_V \| \RZ_{\opw}) \stackrel{\eqref{eq:RL:lb}}{\geq} H(\RZ_V|\RZ_{\opw}) - \wskc (\RZ_V \| \RZ_{\opw}) \geq H(\RZ_V|\RZ_{\opw}) - \pkc (\RZ_V | \RZ_{\opw})=\rco(\RZ_V|\RZ_{\opw})$. By combining this inequality and \eqref{eq:rl_rco}, we get
$\rco(\RZ_V|\RZ_{\opw})\leq \rl(\RZ_V \| \RZ_{\opw}) \leq \rco(\RZ_V)$.



When $\RZ_{\opw}$ is independent of $\RZ_V$, as a straightforward consequence of the definition of $\wskc(\RZ_V\|\RZ_{\opw})$, we have $\wskc(\RZ_V\|\RZ_{\opw})=\skc(\RZ_V)$. Therefore, we see that $$\rco(\RZ_V) = H(\RZ_V)-\skc(\RZ_V)=H(\RZ_V|\RZ_{\opw}) - \wskc(\RZ_V\|\RZ_{\opw}) \stackrel{\eqref{eq:RL:lb}}{\leq} \rl(\RZ_V\|\RZ_{\opw}) \stackrel{\eqref{eq:rl_rco}}{\le} \rco(\RZ_V).$$ Thus, the upper bound in \eqref{eq:rl_rco} and the lower bound in \eqref{eq:RL:lb} in fact hold with equality in this case.
