  In this paper, we explore the connection between secret key agreement and secure omniscience within the setting of the multiterminal source model with a wiretapper who has side information. While the secret key agreement problem considers the generation of a maximum-rate secret key through public discussion, the secure omniscience problem is concerned with communication protocols for omniscience that minimize the rate of information leakage to the wiretapper. The starting point of our work is a lower bound on the minimum leakage rate for omniscience, $\rl$, in terms of the wiretap secret key capacity, $\wskc$. Our interest is in identifying broad classes of sources for which this lower bound is met with equality, in which case we say that there is a duality between secure omniscience and secret key agreement. We show that this duality holds in the case of certain finite linear source (FLS) models, such as two-terminal FLS models and pairwise independent network models on trees with a linear wiretapper. Duality also holds for any FLS model in which $\wskc$ is achieved by a perfect linear secret key agreement scheme. We conjecture that the duality in fact holds unconditionally for any FLS model. On the negative side, we give an example of a (non-FLS) source model for which duality does not hold if we limit ourselves to communication-for-omniscience protocols with at most two (interactive) communications.  We also address the secure function computation problem and explore the connection between the minimum leakage rate for computing a function and the wiretap secret key capacity.
  
%   Finally, we demonstrate the usefulness of our lower bound on $\rl$ by using it to derive equivalent conditions for the positivity of $\wskc$ in the multiterminal model. This extends a recent result of Gohari, G\"{u}nl\"{u} and Kramer (2020) obtained for the two-user setting.
  
   
%   In this paper, we study the problem of secret key generation through an omniscience achieving communication that minimizes the 
%   leakage rate to a wiretapper who has side information in the setting of multiterminal source model.  We explore this problem by deriving a lower bound on the wiretap secret key capacity $\wskc$ in terms of the minimum leakage rate for omniscience, $\rl$. 
%   %The former quantity is defined to be the maximum secret key rate achievable, and the latter one is defined as the minimum possible leakage rate about the source through an omniscience scheme to a wiretapper. 
%   The main focus of our work is the characterization of the sources for which the lower bound holds with equality \textemdash it is referred to as a duality between secure omniscience and wiretap secret key agreement. For general source models, we show that duality need not hold if we limit to the communication protocols with at most two (interactive) communications. In the case when there is no restriction on the number of communications, whether the duality holds or not is still unknown. However, we resolve this question affirmatively for two-user finite linear sources (FLS) and pairwise independent networks (PIN) defined on trees, a subclass of FLS. Moreover, for these sources, we give a single-letter expression for $\wskc$. Furthermore, in the direction of proving the conjecture that duality holds for all FLS, we show that if $\wskc$ is achieved by a \emph{perfect} secret key agreement scheme for FLS then the duality must hold. All these results mount up the evidence in favor of the conjecture on FLS. Moreover, we demonstrate the usefulness of our lower bound on $\wskc$ in terms of $\rl$ by deriving some equivalent conditions on the positivity of secret key capacity for multiterminal source model. Our result indeed extends the work of Gohari, G\"{u}nl\"{u} and Kramer in two-user case.