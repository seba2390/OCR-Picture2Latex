Fix $\epsilon>0$. Let $\left(\RF^{(N)}\right)_{N\geq 1}$ be a communication scheme allowing users to achieve omniscience such that
\begin{align}
    \limsup_{N \to \infty} \frac{1}{N}I\left(\RF^{(N)} \wedge \RZ_V^N \big| \RZ_{\opw}^N\right) < \rl + \epsilon/2
\end{align}
and 
\begin{align}
\liminf_{N \to \infty} \Pr(\RE_1^{(N)} = \dots =\RE_m^{(N)} = \RZ_V^N) >1-\epsilon/2.
\end{align}
%
Then, for a large enough integer $N_0$,  $\frac{1}{N}I\left(\RF^{(N)} \wedge \RZ_V^N \big| \RZ_{\opw}^N\right) < \rl + \epsilon$ and $\Pr(\RE_1^{(N)} = \dots =\RE_m^{(N)} = \RZ_V^N) >1-\epsilon$  for all  $N \geq N_0$. We will use the specific communication scheme $\RF^{(N_0)}$ to construct a SKA scheme that achieves a secret key rate close to $H(\RZ_V|\RZ_{\opw})-\rl$. It suffices to construct this protocol only for $n$'s that are integer multiples of $N_0$ (cf. Theorem 3.2 of \cite{csiszar08}).


% Let $(\tRF^{(n)})_{n\geq 1}$ and $(\RK^{(n)})_{n\geq 1}$ be the  communications and keys, respectively, associated to the resulting WSKA scheme. In fact, it is enough to construct this protocol only for $n$'s that are integer multiples of $N_0$ with the desired properties; because for $kN_0 \leq n < (k+1)N_0$, we can use only the first $kN_0$ realizations of the source in the key generation protocol, i.e., set  $\tRF^{(n)}:= \tRF^{(kN_0)}$ and $\RK^{(n)}:=\RK^{(kN_0)}$. Under this construction, it is easy to show that the desired properties of the subsequence in $k$ will carry over to the whole sequence in $n$.

The constructed omniscience communication $\tRF^{(kN_0)}$ comprises two parts. The first part is of the form $(\RF^{(N_0)})^k$, which is obtained by applying the function corresponding to $\RF^{(N_0)}$ individually on the $k$ sub-blocks of size $N_0$ of the $kN_0$ source realizations. The second part contains an extra Slepian-Wolf communication $f_k(\RZ_V^{kN_0})$ (\cite[Lemma~3.1]{csiszar08}) that ensures the recoverability of $\RZ_V^{kN_0}$ at each user with high probability. The form of  $f_k(\RZ_V^{kN_0})$ is specified next.

Let  $(\RS_i)_{i \in V}$ be the private randomness that can potentially be used by the terminals, which satisfies $P_{\RZ^{N_0}_V\RZ_{\opw}^{N_0}\RS_V}=P_{\RZ^{N_0}_V\RZ_{\opw}^{N_0}}\prod_{i \in V}P_{\RS_i}$. Fix a user $j \in  V$, and let $i \neq j$ be another user. By applying Lemma~3.1 of \cite{csiszar08} with $\RU=\RZ_i^{N_0}$ and $\RV=({\RF}^{(N_0)}, \RZ_j^{N_0}, \RS_j)$, we can conclude that there exists a function $g_{ij}$ of $\RZ_i^{kN_0}$ such that $\RZ_i^{kN_0}$ is recoverable from $\tilde{\RF}^{(kN_0)}$, $\RZ_j^{kN}$, $\RS^k_j$ and $g_{ij}(\RZ_i^{kN_0})$ with probability going to one as $k \to \infty$, and $H(g_{ij}(\RZ_i^{kN_0}))< k(\epsilon N_0 \log|\mc{Z}_i|+ h_2(\epsilon))$. This means that user $i$ can transmit $g_{ij}(\RZ_i^{kN_0})$ to user $j$ so that $\RZ_i^{kN_0}$ is recoverable with high probability. Similarly, every user can communicate in this way to each of the other users. By using the union bound, we can conclude that this communication allows every user to recover the complete source $\RZ_V^{kN_0}$ with probability going to one as $k \to \infty$. The extra communication $f_k(\RZ_V^{kN_0})$ is of the form $\{g_{ij}(\RZ_i^{kN_0}):i \in V, j \in  V\setminus \{i\}\}$, for which $R_f:=\frac{1}{k}H(f_k(\RZ_V^{kN_0})) <(|V|-1)\left[\epsilon N_0 \log|\mc{Z}_V|+h_2(\epsilon)\right]$.  The users now can apply a function on the recovered source to extract a key $\RK^{(kN_0)}$, which satisfies the recoverability condition \eqref{eq:sk:recoverability}.

To obtain a function that makes this key satisfy the secrecy condition \eqref{eq:sk:secrecy}, we rely on the 
balanced coloring lemma~\cite[Lemma~B.3]{csiszar04}. We will apply this lemma  with $\RU=\RZ_V^{N_0}$, $\RV=(\RF^{(N_0)}, \RZ^{N_0}_{\opw})$ and $f(\RU^k)= f_k(\RZ_V^{kN_0})$, which satisfy the inequality 
\begin{align*}
    H(\RU|\RV)-R_f&=H(\RZ_V^{N_0}|\RF^{(N_0)}, \RZ^{N_0}_{\opw})-R_f \\
&> H(\RZ_V^{N_0}|\RF^{(N_0)}, \RZ^{N_0}_{\opw})-(|V|-1)\left[\epsilon N_0 \log|\mc{Z}_V|+h_2(\epsilon)\right].
\end{align*}
The balanced coloring lemma guarantees the existence of a key function $g:\mc{Z}_V^{kN_0} \to \{1, \ldots, 2^{kR_g}\}$ with $$R_g= H(\RZ_V^{N_0}|\RF^{(N_0)}, \RZ^{N_0}_{\opw})-(|V|-1)\left[\epsilon N_0 \log|\mc{Z}_V|+h_2(\epsilon)\right]$$ such that  $kR_g-H(g(\RU^k))+I(g(\RU^k) \wedge f(\RU^k), \RV^k) \longrightarrow 0$ exponentially quickly in $k$. Then, $\RK^{(kN_0)}=g(\RZ_V^{kN_0})$ has rate  $\frac{1}{kN_0}\log|\mc{K}^{(kN_0)}|=\frac{1}{kN_0}kR_g= \frac{1}{N_0} H(\RZ_V^{N_0}|\RF^{(N_0)}, \RZ^{N_0}_{\opw})-(|V|-1)\left[\epsilon \log|\mc{Z}_V|+\frac{h_2(\epsilon)}{N_0}\right]$, and by virtue of the balanced coloring lemma, it satisfies the secrecy condition. Therefore,
\begin{align*}
    \wskc &\geq \frac{1}{N_0} H(\RZ_V^{N_0}|\RF^{(N_0)}, \RZ^{N_0}_{\opw})-(|V|-1)\left[\epsilon \log|\mc{Z}_V|+\frac{h_2(\epsilon)}{N_0}\right]\\
    & = H(\RZ_V| \RZ_{\opw})- \frac{1}{N_0}I(\RZ_V^{N_0}\wedge \RF^{(N_0)}| \RZ^{N_0}_{\opw})-(|V|-1)\left[\epsilon \log|\mc{Z}_V|+\frac{h_2(\epsilon)}{N_0}\right]\\
    & >H(\RZ_V| \RZ_{\opw})- \rl -\epsilon-(|V|-1)\left[\epsilon \log|\mc{Z}_V|+\frac{h_2(\epsilon)}{N_0}\right].
\end{align*}
Since, for a fixed $\epsilon$, $N_0$ can be made arbitrarily large, and $\epsilon$ is also arbitrary, we have $\wskc \geq H(\RZ_V| \RZ_{\opw})- \rl$.