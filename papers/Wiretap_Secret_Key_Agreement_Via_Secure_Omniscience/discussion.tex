In this paper, we have explored the possibility of a duality between the wiretap secret key agreement problem and the secure omniscience problem. Though the problem of characterizing the class of sources for which these two problems are dual to each other is far from being solved completely, we made some progress in the case of limited interaction (with at most two communications allowed), and for the class of finite linear sources. Furthermore, we have made use of \eqref{basic_ineq} to identify several equivalent conditions for the positivity of $\wskc$ in the multi-user case, which is an extension of a recent two-user result of \cite{amin2020}. 



By limiting the number of messages to two, we showed that for the source in Lemma~\ref{lem:twowaycounter}, the duality does not hold. This result seems to indicate that the duality does not always hold. 
In particular, we believe that for the DSBE source  considered in Lemma~\ref{lem:twowaycounter}, the duality does not hold even if we relax the restriction on the number of messages (Conjecture~\ref{conj:duality:msg_converse}). To prove this result, we actually  need a single-letter lower bound on $\rl$ that strictly improves our current bound $H(\RZ_V|\RZ_{\opw}) - \wskc$.  However, it has turned out to be challenging to find a better lower bound on $\rl$. 

\newtheorem*{C1}{Conjecture~\ref{conj:duality:fls}}
\begin{C1}
$\rl = H(\RZ_V|\RZ_{\opw}) - \wskc$ holds for finite linear sources.
\end{C1}

\begin{Conjecture}\label{conj:duality:msg_converse}
For $r\geq m$, $\wskc^{(r)}>0$ need not imply $\rl^{(r)} < H(\RZ_V|\RZ_{\opw})$. Moreover, with no restriction on the number of messages, $\wskc>0$ need not imply $\rl < H(\RZ_V|\RZ_{\opw})$.
\end{Conjecture}



In our attempt to resolve the duality for finite linear sources (Conjecture~\ref{conj:duality:fls}), we were able to prove it in the case of two-user FLS models and in the case of tree-PIN models. The proof construction mainly relies on the idea of aligning the communication with the wiretapper side information. Specifically, in the case of tree-PIN models, we used a reduction to obtain an irreducible source on which we constructed an $\rco$-achieving omniscience scheme that aligns perfectly with the wiretapper side information. In fact, we have shown that this construction is $\rl$-achieving. 

However, for more general PIN sources, this proof strategy fails. The notion of irreducibility in Definition~\ref{def:irreducible} can certainly be extended to general PIN sources. However, it turns out that this definition of irreducibility is not good enough. There are irreducible PIN sources on graphs with cycles whose $\rl$ is not achieved by an omniscience protocol of rate $\rco$ that is perfectly aligned with the wiretapper side information. So, proving the duality conjecture for sources beyond the tree-PIN model could be interesting as it will require new communication strategies other than the ones we used in the proof of the tree-PIN model with a linear wiretapper.  