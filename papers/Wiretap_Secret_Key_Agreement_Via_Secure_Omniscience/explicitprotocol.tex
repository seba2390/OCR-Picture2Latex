In the proof of Theorem \ref{thm:cwsk:irred}, we have fixed the communication matrix structure and argued, using the probabilistic method, that if $n > \log_q sL$ then there exist communication coefficients  that achieve $\rl$. To that end, we first showed the existence of a realization of an $\MS$ such that $\MS\MW^{(n)}=0$ and $\MS_i$'s are invertible. Since $\MS$ and communication coefficients are recoverable from each other, the desired existence follows. However, in the case when $n_e=1$ for all $e \in E$, we give an explicit way to find these coefficients and the sufficient $n$ to do this. Here $\MS$ is just a row vector with entries from $\bb{F}_{q^n}$. Our goal is to find a vector with non-zero entries from $\bb{F}_{q^n}$  for some $n$ such that it satisfies $\MS\MW^{(n)} =\M0$.  Note that $\MW^{(n)}$ is a $\left(\sum_{e \in E} n_e\right) \times n_w$ matrix over $\bb{F}_{q^n}$ with entries $\MW^{(n)}(k,l) = \MW(k,l) \in  \bb{F}_{q}$; since $\bb{F}_{q} \subseteq \bb{F}_{q^n}$, $\MW^{(n)}(k,l) \in \bb{F}_{q^n}$. In the proof of the  following lemma, we actually show how to choose $\MS$.
 \begin{lemma}
  Let $\MW$ be an $(m+k) \times m$ matrix over $\Fq$ and $k,m \geq 1$. Assume that the columns of $\MW$ are linearly independent. If the span of the columns of $\MW$ does not contain any vector that is a scalar multiple of any standard basis vector, then there exists an $1 \times  (m+k)$ vector $\MS$ whose entries belong to $\mathbb{F}_{q^k}^{\times} := \mathbb{F}_{q^k} \backslash \{0\}$ such that $\MS\MW^{(k)}=0$.
 \end{lemma}
 \begin{proof}
  Since the columns of $\MW$ are linearly independent, we can apply elementary column operations and row swappings on the matrix $\MW$ to reduce into the form $\tMW=[\MI_{m\times m} \mid \MA_{m \times k}]^T$, for some matrix $\MA_{m \times k}$. It means that $\tMW=\MP\MW\MC$ for some permutation matrix $\MP$ and an invetible matrix $\MC$ corresponding to the column operations. Furthermore, the matrix $\MA_{m \times k}$ has no zero rows because if there were a zero row in $\MA$ then the corresponding column of $\tMW$ is a standard basis vector which means that the columns of $\MW$ span a standard basis vector contradicting the hypothesis.
  
  Now consider the field $\mathbb{F}_{q^k}$. The condition $\MS\MW^{(k)} =0$ can be written as  $\tMS\tMW^{(k)} = 0$ where $\tMW^{(k)}=\MP\MW^{(k)}\MC$ and $\tMS=\MS\MP^{-1}$. Since $\mathbb{F}_{q^k}$ is a vector space over $\Fq$, there exists a basis  $\{\beta_1, \beta_2, \ldots, \beta_k\} \subset \mathbb{F}_{q^k}$. We will use this basis to construct $\tMS$ and hence $\MS$. For $\MA=[a_{ij}]_{i \in [m], j \in [k]}$, set $\tMS_{m+i} = \beta_i\neq 0$ for $i \in [k]$ and $\tMS_i = - \sum_{j=1} ^{k} a_{ij}\beta_j \neq 0$ for $i \in [m]$. So all entries of $\tMS$  are non-zero entries which follows from the fact that $\beta_j$'s are linearly independent and for a fixed $i$, $a_{ij}$'s are not all zero. Therefore we found an $\tMS$ such that $\tMS\tMW^{(k)} = \M0$. This in turns gives $\MS$, which is obtained by permuting the columns of $\tMS$, such that $\MS\MW^{(k)} = \M0$.
 \end{proof}
 
 In the case when $n_e=1$ and the source is irreducible, the wiretapper matrix satisfies the conditions in the hypothesis of the above lemma. Therefore, we can use the construction given in that lemma to find an $\MS$  such that $\MS\MW^{(n)}=0$ where $n = |E|-n_w$. From $\MS$, we can recover back the communication coefficients $\MA_{i,e} \in \mathbb{F}_{q^k}$ because  given all $\MS_e$ along the unique path from $i$ to the root node, we can recursively compute all $\MA_{i,e}$ along that path. 
 
 We could not extend these ideas beyond this case but it is worth finding such simple and explicit constructions in the arbitrary $n_e$ case. Another interesting question is, for a given tree-PIN source with linear wiretapper, what is the minimum $n$ required to achieve perfect omniscience and perfect alignment using a linear communication? Note that the $n$ required in our protocol is $|E|-n_w$ whereas the  probabilistic method guarantees a scheme if $n > \log_q|E|$. So we clearly see that  $n = |E|-n_w$  is not optimal in some cases. 
 