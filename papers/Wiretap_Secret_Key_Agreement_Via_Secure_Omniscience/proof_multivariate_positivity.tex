Some of the steps in the proof are analogous to the proof for the two-user case, \cite[Theorem~4]{amin2020}.  The new component of the theorem is the identification of a connection between the positivity of $\wskc$ and the non-maximality of $\rl$ of a transformed source.
Since most of the essential ideas of the two-user setting work even in the multiuser case, we only give proof sketches for these analogous steps. However, the new arguments are described in detail.


% we only provide new key arguments  and give proof sketches for the analogous arguments.

The statement 3) implies 4) is trivial. So it is enough to show that 1) implies 2), 2) implies 3), and 4) implies 1).

\textit{1) implies 2):} We prove this by following an approach that is similar to that of the two-user case. First, using the sets given in 1), we construct a new source $(\tRZ_V,\tRZ_{\opw})$ by applying some functions to the user random variables of the source $(\RZ^r_V,\RZ^r_{\opw})$. Then, we show that $\wskc(\tRZ_V||\tRZ_{\opw})>0$ which in turn implies that $\wskc(\RZ_V||\RZ_{\opw})>0$ because any SKA scheme on the source $(\tRZ_V,\tRZ_{\opw})$ is an SKA scheme on $(\RZ_V,\RZ_{\opw})$. To prove $\wskc(\tRZ_V||\tRZ_{\opw})>0$ using condition 1), we use the lower bound of Theorem~\ref{thm:RL:lb} and Lemma~\ref{lem:hatsource} for the new source.


Let $(\tRZ_1,\ldots,\tRZ_m,\tRZ_{\opw})$ be a function of $(\RZ^r_1,\ldots,\RZ^r_m,\RZ^r_{\opw})$ obtained by setting $\tRZ_i = 1$ if $\RZ_i^r \in \mc{A}_{i1}$, $\tRZ_i = 2$ if $\RZ_i^r \in \mc{A}_{i2}$ and  $\tRZ_i = 3$ if $\RZ_i^r \not\in \mc{A}_{i1}\cup \mc{A}_{i2}$ for $1\leq i\leq m$, and $\tRZ_{\opw} = \RZ_{\opw}^r$. Let $p_{j_1\ldots j_m} :=\Pr(\tRZ_1=j_1, \ldots, \tRZ_m=j_m)= \Pr(\RZ^r_1 \in \mcA_{1j_1}, \ldots, \tRZ_m\in \mcA_{mj_m})$ for all $(j_1, \ldots, j_m) \in \{1,2\}^m$. The condition in 1) is equivalent  to the condition 
\begin{align*}
       &D_{\frac{1}{2}} \left(P_{\tRZ_{\opw}}(.|\tRZ_1=1, \ldots, \tRZ_m=1)  
       %\right. \\& \mkern 100mu \left.
       ||P_{\tRZ_{\opw}}(.|\tRZ_1=2, \ldots, \tRZ_m=2) \right)
       < \log \left(\frac{p_{1,1,\ldots,1}p_{2,2,\ldots,2}}{\sum \limits_{\substack{(j_1,\ldots,j_m)\\ \not\in \{(1,\ldots,1),\\\mkern 30mu (2,\ldots,2)\}}}\dfrac{p_{j_1, \ldots, j_m}p_{3-j_1,\ldots, 3-j_m}}{2}}\right).
 \end{align*}
 
 
 We will show that the above condition implies 
  \begin{align} \label{eq:rco_nonmaximal}
     &H(\tRZ^n_V|\tRZ^n_{\opw},\tRZ^n_V\in \mc{A}_{1}\times\cdots\times\mc{A}_{m})
    %\notag \\& \mkern 100mu  
     > \rco(\tRZ^n_V|\tRZ^n_V\in \mc{A}_{1}\times\cdots\times\mc{A}_{m})
 \end{align}
 for some integer $n$, and a non-empty set $\mc{A}_{1}\times\cdots\times\mc{A}_{m}$, where $\mc{A}_{i} \subset \{1,2\}^n$ for all $i \in V$. Because of the following argument, inequality \eqref{eq:rco_nonmaximal} implies that $\wskc(\tRZ_V||\tRZ_{\opw})>0$ which further implies that $\wskc(\RZ_V||\RZ_{\opw})>0$. Suppose that there is an integer $n$, and a non-empty set $\mc{A}_{1}\times\cdots\times\mc{A}_{m}\subset \{1,2\}^n\times\cdots\times \{1,2\}^n$ such that \eqref{eq:rco_nonmaximal} holds. Let $(\widehat{\tRZ_V^n},\tRZ^n_{\opw})$ be the source as defined in \eqref{eq:hatsource} using the source $(\tRZ_V^n,\tRZ^n_{\opw})$ and the set $\mc{A}_{1}\times\cdots\times\mc{A}_{m}$.  Condition \eqref{eq:rco_nonmaximal} can be written as $H(\widehat{\tRZ_V^n}|\tRZ^n_{\opw})> \rco(\widehat{\tRZ_V^n})$. For the new source, it follows from \eqref{eq:rl_rco} that 
 \begin{align*}
     \rl(\widehat{\tRZ_V^n}||\tRZ^n_{\opw}) \leq \rco(\widehat{\tRZ_V^n}).
 \end{align*}
Therefore, we have  $H(\widehat{\tRZ_V^n}|\tRZ^n_{\opw})-\rl(\widehat{\tRZ_V^n}||\tRZ^n_{\opw}) \geq H(\widehat{\tRZ_V^n}|\tRZ^n_{\opw})- \rco(\widehat{\tRZ_V^n})>0$. By combining this with Lemma~\ref{lem:hatsource}, we get 
$H(\tRZ_V^n|\tRZ^n_{\opw})-\rl(\tRZ_V^n||\tRZ^n_{\opw}) \geq \Pr(\mcE) [H(\widehat{\tRZ_V^n}|\tRZ^n_{\opw})-\rl(\widehat{\tRZ_V^n}||\tRZ^n_{\opw})]>0$. So we conclude that $H(\tRZ_V|\tRZ_{\opw})-\rl(\RZ_V||\tRZ_{\opw})=\dfrac{1}{n}[H(\tRZ_V^n|\tRZ^n_{\opw})-\rl(\tRZ_V^n||\tRZ^n_{\opw})]>0$. Hence, it follows from the lower bound of Theorem~\ref{thm:RL:lb} that if $\rl(\tRZ_V||\tRZ_{\opw})< H(\tRZ_V|\tRZ_{\opw})$ then $\wskc(\tRZ_V||\tRZ_{\opw})>0$. Since the source $(\tRZ_V, \tRZ_{\opw})$ is obtained by processing (deterministically) each user observations of the source $(\RZ^r_V, \RZ^r_{\opw})$, any positive rate secret key on $(\tRZ_V, \tRZ_{\opw})$ is also a positive rate secret key on $(\RZ^r_V, \RZ^r_{\opw})$. Thus $\wskc(\tRZ_V||\tRZ_{\opw})>0$ implies that $\wskc(\RZ_V||\RZ_{\opw})= \frac{1}{r}\wskc(\RZ^r_V|| \RZ^r_{\opw})>0$.\\


Now we will show  that condition 1) implies \eqref{eq:rco_nonmaximal}.
Consider the following repetition coding with block swapping: for an even integer $n$, let
 \begin{align*}
  \mathbf{1}&:=\underbrace{1\ldots 1}_{n/2}\underbrace{ 2\ldots 2}_{n/2}\\
  \mathbf{2}&:=2\ldots 2 1\ldots 1
 \end{align*}
 and $\mc{A}_{i}:=\{\mathbf{1},\mathbf{2}\}$, for $1\leq i\leq m$. Let us define $\mc{A}_{V}:=\mc{A}_{1}\times\cdots\times\mc{A}_{m}$. It is enough to show that for large enough $n$, 
\begin{align}\label{eq:non_maximal_1}
     H(\tRZ^n_V|\tRZ^n_{\opw},\tRZ^n_V\in \mc{A}_{V}) > \rco(\tRZ^n_V|\tRZ^n_V\in \mc{A}_{V}).
 \end{align}
 Let $\mc{B}:=2^V \backslash \{\emptyset, V\}$, $\lambda^{(n)}$ be a fractional partition of $\mc{B}$, i.e., $\lambda^{(n)}:\mc{B} \to \mathbb{R}^+$ is such that $\sum_{i\in B}\lambda^{(n)}(B)=1$ for every $i \in V$; and let $\Lambda^{(n)}$ be the set of all fractional partitions. The minimum rate of communication for omniscience \cite[Sec. V]{csiszar04} is given by
 \begin{align*} 
     &\rco(\tRZ^n_V|\tRZ^n_V\in \mc{A}_{V})%\\ & \mkern 70mu
     = \max \limits_{\lambda^{(n)} \in \Lambda^{(n)}} \sum \limits_{B \in \mc{B}} \lambda^{(n)}_B H(\tRZ^n_B|\tRZ^n_{B^c},\tRZ^n_V\in \mc{A}_{V}).
 \end{align*}
Though the optimal fractional partition seems to depend on $n$, because of the repetitive structure of the coding this dependence disappears. We can upper bound $\rco$ as 
 \begin{align}\label{eq:non_maximal_2}
  \rco(\tRZ^n_V|\tRZ^n_V\in \mc{A}_{V})\notag%\\&\mkern 70mu
  &= \max \limits_{\lambda^{(n)} \in \Lambda^{(n)}} \sum \limits_{B \in \mc{B}} \lambda^{(n)}_B H(\tRZ^n_B|\tRZ^n_{B^c},\tRZ^n_V\in \mc{A}_{V})\notag\\
  &= \sum \limits_{B \in \mc{B}} \lambda^{*(n)}_B H(\tRZ^n_B|\tRZ^n_{B^c},\tRZ^n_V\in \mc{A}_{V})\notag\\
  &\leq \sum \limits_{B \in \mc{B}}H(\tRZ^n_B|\tRZ^n_{B^c},\tRZ^n_V\in \mc{A}_{V})\notag\\
  &\leq (2^{m}-2)\max \limits_{i \in V} H(\tRZ^n_{V \backslash i}|\tRZ^n_{i},\tRZ^n_V\in \mc{A}_{V}),
 \end{align}
 where we used in the first inequality that the optimal fractional partition $\lambda^{*(n)}_B$ is bounded above by $1$ for all $B \in \mc{B}$, and in the last inequality that  $H(\tRZ^n_B|\tRZ^n_{B^c},\tRZ^n_V\in \mc{A}_{V}) \leq \max \limits_{i \in V} H(\tRZ^n_{V \backslash i}|\tRZ^n_{i},\tRZ^n_V\in \mc{A}_{V})$ for all $B \in \mc{B}$.\\
 
 
% Since $\rco(\tRZ^n_V|\tRZ^n_V\in \mc{A}_{V})$ is upper bounded by $(2^{m}-2)\max \limits_{i \in V} H(\tRZ^n_{V \backslash i}|\tRZ^n_{i},\tRZ^n_V\in \mc{A}_{V})$,  it suffices to prove \begin{align*}
%     &H(\tRZ^n_V|\tRZ^n_{\opw},\tRZ^n_V\in \mc{A}_{V})
%     \\& \mkern 70mu > (2^{m}-2)\max \limits_{i \in V} H(\tRZ^n_{V \backslash i}|\tRZ^n_{i},\tRZ^n_V\in \mc{A}_{V}).
% \end{align*}
Let us further upper bound $(2^{m}-2)\max \limits_{i \in V} H(\tRZ^n_{V \backslash i}|\tRZ^n_{i},\tRZ^n_V\in \mc{A}_{V})$ as follows. Consider the term $H(\tRZ^n_{V \backslash i_0}|\tRZ^n_{i_0},\tRZ^n_V\in \mc{A}_{V})$ for some $i_0 \in V$. We know that 
 \begin{align*}
  \Pr[\tRZ^n_1= \mathbf{k}_1,\ldots,\tRZ^n_m=\mathbf{k}_m]= p_{k_1\ldots k_m}^{n/2}p_{3-k_1\ldots 3-k_m}^{n/2}
 \end{align*}
for all $(\mathbf{k}_1, \ldots, \mathbf{k}_m) \in \{\mathbf{1},\mathbf{2}\}^m$, and for $i \in V$, $k_i$ denotes the first symbol in the sequence $\mathbf{k}_i$. Therefore, we get
\begin{align*}
  \Pr[\tRZ^n_1=\mathbf{k}_1,\ldots,\tRZ^n_m=\mathbf{k}_m|\tRZ^n_V\in \mc{A}_{V}]%\\&\mkern 70mu
  = \dfrac{p_{k_1\ldots k_m}^{n/2}p_{3-k_1\ldots 3-k_m}^{n/2}}{\sum \limits_{\substack{(j_1,\ldots,j_m)\in \{1,2\}^m}}p_{j_1\ldots j_m}^{n/2}p_{3-j_1\ldots 3-j_m}^{n/2}}.
 \end{align*}
For $i_0 \in V$ and $(\mathbf{k}_1, \ldots, \mathbf{k}_m) \in \{\mathbf{1},\mathbf{2}\}^m$, we have 
 \begin{align*}
  \Pr[\tRZ^n_{V \backslash i_0}=\mathbf{k}_{V \backslash i_0} |\tRZ^n_{i_0}=\mathbf{k}_{i_0}, \tRZ^n_V\in \mc{A}_{V}]
  %& = \dfrac{p_{k_1\ldots k_{i_0}\ldots k_m}^{n/2}p_{3-k_1\ldots 3-k_{j_0}\ldots 3-k_m}^{n/2}}{\sum \limits_{\substack{(j_1,\ldots,k_{i_0},\ldots,j_m)\in \{1,2\}^m}}p_{j_1\ldots k_{i_0}\ldots j_m}^{n/2}p_{3-j_1\ldots 3-k_{i_0}\ldots 3-j_m}^{n/2}}\\
  %\\&\mkern 70mu 
  = \dfrac{p_{k_1\ldots k_{i_0}\ldots k_m}^{n/2}p_{3-k_1\ldots 3-k_{j_0}\ldots 3-k_m}^{n/2}}{\dfrac{1}{2}\sum \limits_{\substack{(j_1,\ldots,j_m)\in \{1,2\}^m}}p_{j_1\ldots j_m}^{n/2}p_{3-j_1\ldots 3-j_m}^{n/2}}
 \end{align*}
 where the equality follows from the symmetry in the probabilities of the sequences $(j_1,\ldots,k_{i_0},\ldots,j_m)$ and $(3-j_1,\ldots,3-k_{i_0},\ldots,3-j_m)$. To compute the entropies, we make use of the grouping property of the entropy: For a probability vector $(q_1, q_2, \ldots, q_s)$, $H(q_1, q_2,.....,q_s)= H\left(q_1, q_2+ \cdots+q_s\right)+ (q_2+ \cdots+q_s)H \Big(\dfrac{q_2}{q_2+ \cdots+q_s},\ldots,\linebreak \dfrac{q_s}{q_2+ \cdots+q_s}\Big)$. If $q_1 \in [0.5, 1]$, then $h(q_1)\leq - 2(1-q_1)\log_2 (1-q_1)$ which implies $H(q_1, q_2,.....,q_s)\leq h(q_1)+(1-q_1)\log_2 (s-1) \leq (1-q_1)\log_2 (s-1) - 2(1-q_1)\log_2 (1-q_1)$. Note that because R\'enyi divergence is non-negative, the inequality in 1) implies that
\begin{align*}
    p_{1\ldots 1}p_{2\ldots 2}& >  \dfrac{1}{2}\sum \limits_{\substack{(j_1,\ldots,j_m)\\ \not\in \{(1,\ldots,1),(2,\ldots,2)\}}}p_{j_1, \ldots, j_m}p_{3-j_1,\ldots, 3-j_m}. %\\ & \geq \dfrac{1}{2}p_{j_1, \ldots, j_m}p_{3-j_1,\ldots, 3-j_m}.
\end{align*}
So, by setting 
 \begin{align*}
     q^{(n)}_1 &:= \dfrac{p_{1\ldots 1}^{n/2}p_{2\ldots 2}^{n/2}}{\dfrac{1}{2}\sum \limits_{\substack{(j_1,\ldots,j_m)\in \{1,2\}^m}}p_{j_1\ldots j_m}^{n/2}p_{3-j_1\ldots 3-j_m}^{n/2}}= \dfrac{p_{1\ldots 1}^{n/2}p_{2\ldots 2}^{n/2}}{p_{1\ldots 1}^{n/2}p_{2\ldots 2}^{n/2}+\dfrac{1}{2}\sum \limits_{\substack{(j_1,\ldots,j_m)\\ \not\in \{(1,\ldots,1),(2,\ldots,2)\}}}p_{j_1\ldots j_m}^{n/2}p_{3-j_1\ldots 3-j_m}^{n/2}},
 \end{align*} 
 which is greater than $1/2$, we have for any $i_0 \in V$, 
 \begin{align}\label{eq:non_maximal_3}
    &H(\tRZ^n_{V \backslash i_0}|\tRZ^n_{i_0},\tRZ^n_V\in \mc{A}_{V})%\notag \\& \mkern 30mu 
    \leq (1-q^{(n)}_1)(\log_2 (2^{m-1}-1) - 2\log_2 (1-q^{(n)}_1)).
\end{align}
where we replaced $s$ by $2^{m-1}$. Notice that the bound is independent of $i_0$. Therefore, from \eqref{eq:non_maximal_2} and \eqref{eq:non_maximal_3}, we have 
\begin{align*}
 &\rco(\tRZ^n_V|\tRZ^n_V\in \mc{A}_{V})%\\& 
 \leq (2^{m}-2)[(1-q^{(n)}_1)(\log_2 (2^{m-1}-1) - 2\log_2 (1-q^{(n)}_1))].\notag
\end{align*}
Because of the above inequality, it is enough prove that condition 1) implies 
\begin{align}\label{eq:final}
 &H(\tRZ^n_V|\tRZ^n_{\opw},\tRZ^n_V\in \mc{A}_{V})%\\& 
 > (2^{m}-2)[(1-q^{(n)}_1)(\log_2 (2^{m-1}-1) - 2\log_2 (1-q^{(n)}_1))].%\notag
\end{align}

Now let us argue that condition 1) implies \eqref{eq:final}. Since 
\begin{align*}
 q^{(n)}_1&= \dfrac{p_{1\ldots 1}^{n/2}p_{2\ldots 2}^{n/2}}{p_{1\ldots 1}^{n/2}p_{2\ldots 2}^{n/2}+\dfrac{1}{2}\sum \limits_{\substack{(k_1,\ldots,k_m) \not\in\\  \{(1,\ldots,1),(2,\ldots,2)\}}}p_{k_1\ldots k_m}^{n/2}p_{3-k_1\ldots 3-k_m}^{n/2}}\\ &\geq \dfrac{p_{1\ldots 1}^{n/2}p_{2\ldots 2}^{n/2}}{p_{1\ldots 1}^{n/2}p_{2\ldots 2}^{n/2}+\left(\dfrac{1}{2} \sum \limits_{\substack{(k_1,\ldots,k_m)\not\in \\  \{(1,\ldots,1),(2,\ldots,2)\}}}p_{k_1\ldots k_m}p_{3-k_1\ldots 3-k_m}\right)^{\frac{n}{2}}},
\end{align*}
 we have
$$\lim \limits_{n \to \infty}(1-q^{(n)}_1)^{\frac{1}{n}} \leq \sqrt{\dfrac{\dfrac{1}{2} \sum \limits_{\substack{(k_1,\ldots,k_m)\\ \not\in \{(1,\ldots,1),(2,\ldots,2)\}}}p_{k_1\ldots k_m}p_{3-k_1\ldots 3-k_m}}{p_{1\ldots 1}p_{2\ldots 2}}} $$
and 
% \begin{align}\label{eq:left_limit}
%     &\lim \limits_{n \to \infty} (2^{m}-2)^{1/n}[(1-q^{(n)}_1)(\log_2 (2^{m-1}-1)  \nonumber \\ & \mkern 150mu - 2\log_2 (1-q^{(n)}_1))]^{1/n} \nonumber \\&\leq \sqrt{\dfrac{\dfrac{1}{2} \sum \limits_{\substack{(k_1,\ldots,k_m)  \\ \not\in \{(1,\ldots,1),(2,\ldots,2)\}}}p_{k_1\ldots k_m}p_{3-k_1\ldots 3-k_m}}{p_{1\ldots 1}p_{2\ldots 2}}}.
% \end{align}
\begin{align}\label{eq:left_limit}
   & \lim \limits_{n \to \infty} (2^{m}-2)^{1/n}[(1-q^{(n)}_1)(\log_2 (2^{m-1}-1) - 2\log_2 (1-q^{(n)}_1))]^{1/n} \nonumber \\ &\mkern 300mu\leq \sqrt{\dfrac{\dfrac{1}{2} \sum \limits_{\substack{(k_1,\ldots,k_m)  \\ \not\in \{(1,\ldots,1),(2,\ldots,2)\}}}p_{k_1\ldots k_m}p_{3-k_1\ldots 3-k_m}}{p_{1\ldots 1}p_{2\ldots 2}}}.
\end{align}
For the asymptotics of the conditional entropy term, we can use the same idea of hypothesis testing at the wiretapper side with $u_1=(\mathbf{1}, \ldots, \mathbf{1})$ and $u_2=(\mathbf{2}, \ldots, \mathbf{2})$ used in \cite[Lemma~2]{amin2020} to get
%  \begin{align}\label{eq:right_limit}
%  &\liminf_{n \to \infty} H(\tRZ^n_V|\tRZ^n_{\opw},\tRZ^n_V\in \mc{A}_{V})^{\frac{1}{n}} \nonumber\\ &\geq  \exp \Big(-\frac{1}{2}D_{\frac{1}{2}} \big(P_{\tRZ_{\opw}}(.|\tRZ_1=1, \ldots, \tRZ_m=1)\nonumber \\& \mkern 150mu  ||P_{\tRZ_{\opw}}(.|\tRZ_1=2, \ldots, \tRZ_m=2) \big)\Big).
%  \end{align}
 \begin{align}\label{eq:right_limit}
 &\liminf_{n \to \infty} H(\tRZ^n_V|\tRZ^n_{\opw},\tRZ^n_V\in \mc{A}_{V})^{\frac{1}{n}} \geq  \exp \Big(-\frac{1}{2}D_{\frac{1}{2}} \big(P_{\tRZ_{\opw}}(.|\tRZ_1=1, \ldots, \tRZ_m=1)  ||P_{\tRZ_{\opw}}(.|\tRZ_1=2, \ldots, \tRZ_m=2) \big)\Big).
 \end{align}


 

Since 
\begin{align*}
       &D_{\frac{1}{2}} \left(P_{\tRZ_{\opw}}(.|\tRZ_1=1, \ldots, \tRZ_m=1)  
       %\right. \\& \mkern 100mu \left.
       ||P_{\tRZ_{\opw}}(.|\tRZ_1=2, \ldots, \tRZ_m=2) \right)
       < \log \left(\frac{p_{1,1,\ldots,1}p_{2,2,\ldots,2}}{\sum \limits_{\substack{(j_1,\ldots,j_m)\\ \not\in \{(1,\ldots,1),\\\mkern 30mu (2,\ldots,2)\}}}\dfrac{p_{j_1, \ldots, j_m}p_{3-j_1,\ldots, 3-j_m}}{2}}\right),
 \end{align*}
 we can conclude from \eqref{eq:left_limit} and \eqref{eq:right_limit} that 
for large enough $n$, \eqref{eq:final} holds. This completes the proof of 1) implies 2).
 
\textit{2) implies 3):} Since the proof follows the same argument as in two user case, we omit most of the details and give only those steps that involve different constants. Following are the multivariate analogues of \cite[eq.~(118) and eq.~(124)]{amin2020}: 
\begin{align*}
    I(\RK_1,\ldots,\RK_m \wedge \RZ^n_{\opw}, \RF^{(n)}) \leq (m+1)\delta+h(\delta)
\end{align*}
and 
\begin{align*}
    &||P_{\RK_1,\ldots, \RK_m,\RF^{(n)},\RZ^n_{\opw}}-P_{\RK_1,\ldots, \RK_m}.P_{\RF^{(n)},\RZ^n_{\opw}}||_{TV}  \leq \sqrt{\frac{(m+1)\delta + h(\delta)}{2}}.
\end{align*}
Using the above inequalities, we get 
\begin{align*}
    &||P_{\RK_1,\ldots, \RK_m,\RF^{(n)},\RZ^n_{\opw}}-\dfrac{1}{2}\mathbf{1}_{\RK_1=\ldots=\RK_m}.P_{\RF^{(n)},\RZ^n_{\opw}}||_{TV}  \leq
    \sqrt{\frac{(m+1)\delta + h(\delta)}{2}}+2\delta
\end{align*}
As $\delta$ can be made arbitrarily close to 0, the condition 3) follows.

\textit{4) implies 1):} To prove this, we need the multivariate analogue of \cite[Lemma~3]{amin2020}. It says that for some sets  $\mc{A}_{11},\mc{A}_{12} \subset \mc{Z}_1^r$, $\mc{A}_{21},\mc{A}_{22} \subset \mc{Z}_2^r$, $\ldots, \mc{A}_{m1},\mc{A}_{m2} \subset \mc{Z}_m^r$,

\begin{align*}
       &\frac{1}{2}D_{\frac{1}{2}} \left(P_{\RZ_{\opw}^r}(.|\mc{E}_{1, 1, \ldots, 1})  
       %\right. \\& \mkern 100mu \left.
       ||P_{\RZ_{\opw}^r}(.|\mc{E}_{2, 2, \ldots, 2}) \right) \leq -\log(1-4\delta),
 \end{align*}
 
\begin{align*} 
& \frac{1}{2}\log \left(\frac{\Pr(\mc{E}_{1, \ldots, 1})\Pr(\mc{E}_{2, \ldots, 2})}{\sum \limits_{\substack{(j_1,\ldots,j_m)\\ \not\in \{(1,\ldots,1),(2,\ldots,2)\}}}\dfrac{\Pr(\mc{E}_{j_1, \ldots, j_m})\Pr(\mc{E}_{3-j_1,\ldots, 3-j_m})}{2}}\right)>\log\left(\dfrac{(\frac{1}{2}-2\delta)}{2\delta\sqrt{2^{m-1}-1}}\right)
 \end{align*}
 where  $\mc{E}_{j_1, \ldots, j_m}$ denotes the event $\RZ_1^r \in \mc{A}_{1j_1}, \ldots, \RZ_m^r \in \mc{A}_{mj_m}$ for $(j_1,\ldots,j_m) \in \{1, 2\}^m$ and $\delta:=||P_{\RK_1,\ldots, \RK_m,\RF^{(r)},\RZ^r_{\opw}}-\dfrac{1}{2}\mathbf{1}_{\RK_1=\dots=\RK_m}.P_{\RF^{(r)},\RZ^r_{\opw}}||_{TV}$. The proof is similar to the two-user case with the following sets $\mc{A}_{ij}=\{z^r_i:\RK_i(z^r_i,f)=j\}$ for all $1\leq i\leq m$ and $j\in\{1,2\}$ where $F^{(r)}=f$ is a realization of the public discussion such that $\delta>||P_{\RK_1,\ldots, \RK_m,\RZ^r_{\opw}|\RF^{(r)}=f}-\dfrac{1}{2}\mathbf{1}_{\RK_1=\ldots=\RK_m}.P_{\RZ^r_{\opw}|\RF^{(r)}=f}||_{TV}$.
 
 Since the condition in 4) implies that any $\delta \leq \delta_1$ satisfies 
\begin{align*}
    -\log(1-4\delta) < \log\left(\dfrac{(\frac{1}{2}-2\delta)}{2\delta\sqrt{2^{m-1}-1}}\right),
\end{align*}
the condition 1) follows. 

