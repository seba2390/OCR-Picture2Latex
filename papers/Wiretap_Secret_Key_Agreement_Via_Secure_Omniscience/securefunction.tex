{We can also consider the problem of secure function computation in the context of the multiterminal source model. Given a function of the users' observations, the goal of the users is to interactively communicate over a public channel to compute this function without revealing much information about the computed value to the wiretapper. 

Formally, let $g(z_1,\ldots,z_m)$ be a function over $\mc{Z}_1\times \cdots \times \mc{Z}_m$. For $i \in V$, user $i$ observes $\RZ_i^n:=(\RZ_{i1},\ldots,\RZ_{in})$, $n$ i.i.d. realizations of the component $\RZ_i$ of the source, and the wiretapper observes $\RZ_{\opw}^n$. Using their observations, the users interactively communicate $\RF^{(n)}$ over a public channel. The wiretapper only listens to this public communication.  Finally, user $i$ computes $\RG_i^{(n)}$ for $i \in V$ using the public communication $\RF^{(n)}$, $\RZ_i^n$, and possible local randomness. We say that the users can compute $\RG= g(\RZ_1,\ldots,\RZ_m)$ if there exists an interactive communication $\RF^{(n)}$ satisfying
    \begin{align}
\liminf_{n \to \infty} \Pr(\RG_1^{(n)} = \ldots =\RG_m^{(n)} = \RG^n) = 1 \label{eq:func:recoverability},
\end{align}
where $\RG^n:=(g(\RZ_{11},\ldots,\RZ_{m1}),\ldots, g(\RZ_{1n},\ldots,\RZ_{mn}))$.% which is $n$ copies of the function $g$ evaluated for each independent realization of the source. 

In this set-up, the minimum leakage rate $\rl^{\RG}$ achievable for computation of $\RG$ is defined as

\begin{align}
\begin{split}
 \rl^{\RG}&:= \inf  \biggl\lbrace \limsup_{n \to \infty} \frac{1}{n}I(\RF^{(n)} \wedge \RG^n|\RZ_{\opw}^n) \biggr\rbrace, \label{eq:rl:func}
 \end{split}
\end{align}
where the infimum is over all communications $\RF^{(n)}$ that allow users to compute $\RG$. When $\RG=\RZ_V$, the secure function computation problem becomes the secure omniscience problem (Sec.~\ref{subsec:omniscience}).

We say that a function $\RG$ is \emph{securely computable} if $\rl^{\RG}=0$. This means that $\RG$ can be computed without revealing any information to the wiretapper other than what is already known through the side information. The communication that achieves $\rl^{\RG}=0$ allows users to compute $\RG$ securely. Henceforth, with an abuse of notation, we write $\rl$ instead $\rl^{\RG}$ when the computed function is clear from the context. The work of \cite{tyagi11} considered the case where wiretapper has no side information with an aim of characterizing the securely computable functions, i.e., $\rl=0$. Remarkably, \cite{tyagi11} gave this characterization in terms of the wiretap secret key capacity: If $\RG$ is securely computable then $H(\RG)\leq \skc$ and conversely, if $\RG$ satisfies $H(\RG)< \skc$, then $\RG$ is securely computable. 



\begin{example}
Let $\RX_i$,  $i \in \{1,2,3,4\}$, be a collection of i.i.d. Ber($\frac{1}{2}$) random variables. A two user source $(m=2)$ is defined as follows.  
\begin{align*}
    \RZ_1&:=(\RX_1,\RX_2,\RX_3)\\
    \RZ_2&:=(\RX_2,\RX_3,\RX_4)
\end{align*}
% User $1$'s observations are $\RZ_1:=(\RX_1,\RX_2,\RX_3)$ and User $2$'s observations are $\RZ_2:=(\RX_3,\RX_4,\RX_5)$. The wiretapper's observations are $\RZ_{\opw}:=(\RX_1, \RX_5)$.
The wiretap secret key capacity is given by $   \skc=I(\RZ_1 \wedge \RZ_2) = H(\RX_2,\RX_3)=2$.
\begin{enumerate}[label=\alph*)]
    \item Let $g_1(x_1,\ldots,x_4):=x_1+x_4$, where $+$ is over $\mathbb{F}_2$. The function $\RG_1=g_1(\RX_1,\ldots,\RX_4)=\RX_1+\RX_4$ is securely computable by the users because $H(\RG_1)=1 < 2=\skc$.
    \item Let $g_2(x_1,\ldots,x_4):=(x_2, x_3, x_4)$. The function $\RG_2=g_2(\RX_1,\ldots,\RX_4)=(\RX_2, \RX_3, \RX_4)$ is not securely computable because $\skc=2<3=H(\RG_2)$.
\end{enumerate}
\end{example}

The following lemma addresses the secure function computation problem when the wiretapper side information is non-trivial.


\begin{lemma}\label{thm:func:necessary}
For a general source $(\RZ_V,\RZ_{\opw})$, any computable function $\RG$ must satisfy 
\begin{align}\label{eq:func_comp:lb}
    H(\RG|\RZ_{\opw})-\rl \leq \wskc.
\end{align}    
\end{lemma}
\begin{proof} We can  prove this result along the lines of the proof of Theorem~\ref{thm:RL:lb}. The idea is that given a discussion scheme that computes the function $\RG$ with leakage rate $\rl$, one can apply privacy amplification to extract a secret key of rate $H(\RG|\RZ_{\opw})-\rl$ from the recovered source.
\end{proof}

 It is clear from \eqref{eq:func_comp:lb} that if $\RG$ is securely computable (i.e., $\rl=0$), then  $H(\RG|\RZ_{\opw})\leq \wskc$. 
 \begin{example}
Let $\RX_i$,  $i \in \{1,2,\ldots,5\}$, be a collection of i.i.d. Ber($\frac{1}{2}$) random variables. A two user source $(m=2)$ is defined as follows.  
\begin{align*}
    \RZ_1&:=(\RX_1,\RX_2,\RX_3)\\
    \RZ_2&:=(\RX_3,\RX_4,\RX_5)\\
    \RZ_{\opw}&:=(\RX_1, \RX_5)
\end{align*}
% User $1$'s observations are $\RZ_1:=(\RX_1,\RX_2,\RX_3)$ and User $2$'s observations are $\RZ_2:=(\RX_3,\RX_4,\RX_5)$. The wiretapper's observations are $\RZ_{\opw}:=(\RX_1, \RX_5)$.
It follows from Theorem~\ref{thm:fls} that the wiretap secret key capacity is
\begin{align*}
    \wskc&=I(\RZ_1 \wedge \RZ_2 | \RG) \\
&= I(\RX_1,\RX_2,\RX_3 \wedge \RX_3,\RX_4,\RX_5 | \RX_1)\\
&= H(\RX_1,\RX_2,\RX_3 | \RX_1)-H(\RX_1,\RX_2,\RX_3 | \RX_3,\RX_4,\RX_5, \RX_1)\\
&= 2-1=1 \text{ bit },
\end{align*}
where $\RG=\RX_1$, which is the maximal common function of $\RZ_1$ and $\RZ_{\opw}$. Let $g_1(x_1,\ldots,x_5):=(x_2, x_3, x_4)$. The function $\RG_1=g_1(\RX_1,\ldots,\RX_5)=(\RX_2, \RX_3, \RX_4)$ is not securely computable because $\wskc=1<3=H(\RG_1|\RZ_{\opw})$ violates the condition in Lemma~\ref{thm:func:necessary}.

\end{example}

 
 Establishing the reverse direction, namely, that $H(\RG|\RZ_{\opw})< \wskc$ implies secure computability of $\RG$, is an interesting open problem. The following lemma addresses the minimum leakage rate of those source models for which this reverse direction holds. 

\begin{lemma}\label{lemma:sc}
If every function $\RS$ with $H(\RS|\RZ_{\opw}) < \wskc$ is securely computable via omniscience, then the minimum leakage rate for computing a given function $\RG$ is
     \begin{align}\label{eq:func:11}
         \rl = |H(\RG|\RZ_{\opw}) - \wskc|^{+}.
     \end{align} 
Here $|u|^{+}\triangleq \max\{u,0\}$.
\end{lemma}
\begin{proof}
    Assume first that $H(\RG|\RZ_{\opw}) < \wskc$. By the hypothesis of the lemma, $\RG$ is securely computable, so that the minimum leakage rate for computing $\RG$ is $\rl=0$. So, it suffices to consider the case when $H(\RG|\RZ_{\opw})\geq \wskc$. By Theorem~\ref{thm:func:necessary}, we have $\rl \geq H(\RG|\RZ_{\opw})- \wskc$. We will now show the reverse inequality  $\rl \leq H(\RG|\RZ_{\opw})- \wskc$, which will imply that $\rl=H(\RG|\RZ_{\opw})- \wskc$.

    Fix $\epsilon > 0$. For $m$ large enough, consider the source $(\RZ_V^m, \RZ_{\opw}^m)$, which corresponds to $m$ i.i.d. realizations of the source $(\RZ_V, \RZ_{\opw})$. By the random binning argument, there exists a function $\tRG$ of $\RG^m$ such that 
    \begin{align}\label{eq:func:1}
\wskc(\RZ_V^m\|\RZ_{\opw}^m) \geq H(\tRG|\RZ_{\opw}^m)> \wskc(\RZ_V^m\|\RZ_{\opw}^m) -\epsilon,
    \end{align}
    where $\wskc(\RZ_V^m\|\RZ_{\opw}^m)=m\wskc(\RZ_V\|\RZ_{\opw})$.

Note that the hypothesis also holds for the $m$-letter source. With $\RZ_V, \RZ_{\opw}$, and $\RG$ replaced by $\RZ_V^m, \RZ^m_{\opw}$ , and $\tRG$, respectively, there exists an omniscience scheme $\tRF$ such that

\begin{align}
 &\frac{1}{n}I(\tRG^n \wedge \tRF|\RZ_{\opw}^{mn}) \leq \delta_n^{(m)}\label{eq:func:2}\\
 &\liminf_{n \to \infty} \Pr(\RE_1^{(mn)} = \ldots =\RE_m^{(mn)} = \RZ_V^{mn}) = 1 
\end{align}
for some $\delta_n^{(m)} \to 0$ as $n \to \infty$.
Observe that  $\RG^m$ can be computed by all the users through $\tRF$ because $\tRF$ is an omniscience scheme. Furthermore, the leakage rate for computing $\RG^m$ is 
  \begin{align*}
        \frac{1}{mn} I(\RG^{mn}\wedge \tRF | \RZ^{mn}_{\opw}) & \stackrel{(a)}{=} \frac{1}{mn} I(\tRG^n, \RG^{mn}\wedge \tRF | \RZ^{mn}_{\opw})\\
        &= \frac{1}{mn} I(\tRG^n\wedge \tRF | \RZ^{mn}_{\opw}) + \frac{1}{mn} I(\RG^{mn}\wedge \tRF | \RZ^{mn}_{\opw}, \tRG^n)\\
        & \stackrel{(b)}{\leq}  \frac{1}{m}\delta_n^{(m)} + \frac{1}{mn} I(\RG^{mn}\wedge \tRF | \RZ^{mn}_{\opw}, \tRG^n)\\
        & \leq  \frac{1}{m}\delta_n^{(m)} + \frac{1}{mn} H(\RG^{mn} | \RZ^{mn}_{\opw}, \tRG^n)\\
        & \stackrel{(c)}{\leq}  \frac{1}{m}\delta_n^{(m)} + \frac{1}{mn} \left[ H(\RG^{mn} | \RZ^{mn}_{\opw}) -  H(\tRG^{n} | \RZ^{mn}_{\opw})\right]\\
        & =  \frac{1}{m}\delta_n^{(m)} +  H(\RG | \RZ_{\opw}) -  \frac{1}{m} H(\tRG | \RZ^{m}_{\opw})\\
        & \stackrel{(d)}{\leq} \frac{1}{m}\delta_n^{(m)} +  H(\RG | \RZ_{\opw}) -  mn \wskc +\epsilon,
    \end{align*}
where $(a)$ and $(c)$ are due to the fact the $\tRG$ is a function of $\RG^m$, $(b)$ and $(d)$ follow respectively from \eqref{eq:func:2} and \eqref{eq:func:1}. By taking limit in the above chain of inequalities, we get 
$$\rl \leq \limsup_{n \to \infty} \frac{1}{mn} I(\RG^{mn}\wedge \tRF | \RZ^{mn}_{\opw}) = H(\RG | \RZ_{\opw}) - \wskc +\epsilon.$$
As $\epsilon$ is arbitrary, we get $\rl \leq H(\RG | \RZ_{\opw}) - \wskc$, completing the proof.
\end{proof}



Next we will consider the secure function computation problem for finite linear sources.

\begin{theorem}\label{thm:func:linearscheme}
  For a finite linear source $(\RZ_V, \RZ_{\opw})$, any leakage rate in computing a linear function $\RG$ via a perfect linear communication scheme is also achievable by a linear omniscience scheme.
\end{theorem}
\begin{proof} The proof is similar to the proof of Theorem~\ref{thm:linearscheme}. Let $\RF^{(n)}$ be a perfect linear communication scheme that computes $\RG$, but $\RF^{(n)}$ need not achieve omniscience. By  \cite[Theorem~1]{chan19oneshot}, we can assume that $\RF^{(n)}$ is a linear function of $\RZ_V^n$ alone (additional randomization by any user is not needed) and $\RG^n$ is also a linear function of $\RZ_V^n$. 

If $\RF^{(n)}$ already attains omniscience, then we are done. If not, for some $i,j \in V$, $i \ne j$, we have a component $\RX \in \Fq$ of random vector $\RZ_i^{n}$ such that
\begin{align*}
    H(\RX \mid \RF^{(n)}, \RZ_j^n) &\neq 0.
\end{align*}
We will show that there exists an additional discussion $\RF'^{(n)}$ such that
\begin{align}
    H(\RX \mid \RF^{(n)},\RF'^{(n)}, \RZ_j^n) &= 0 \label{eq:func:additional_recov}
\end{align}  and 
\begin{align}
 I(\RG^{(n)} \wedge \RF^{(n)},\RF'^{(n)}| \RZ_{\opw}^n) = I(\RG^{(n)} \wedge \RF^{(n)}| \RZ_{\opw}^n).\label{eq:func:additional_secrecy}
\end{align}
If $(\RF^{(n)},\RF'^{(n)})$ achieves omniscience, we are done; else, we repeat the construction in our argument till we obtain the desired omniscience-achieving communication.

So, consider the non-trivial case where $H(\RX\mid \RF^{(n)}, \RZ_j^n) \neq 0$ and $I(\RG^n \wedge \RF^{(n)}, \RX| \RZ_{\opw}^n) \neq 0$. (If $I(\RG^n \wedge \RF^{(n)}, \RX| \RZ_{\opw}^n)=0$, then user $i$ transmits $\RF'^{(n)}:= \RX$ which satisfies \eqref{eq:func:additional_recov} and \eqref{eq:func:additional_secrecy}.)
%(Since $\RG^n$ must be recoverable from $(\RF^{(n)}, \RZ_i^n)$, the alternative assumption $I(\RG^n \wedge \RF^{(n)}, \RZ_i^{n}, \RZ_{\opw}^n) = 0$ yields $H(\RG^n) = 0.$) 
Let $\RL^{(n)}$ be a common linear function, not identically $0$, of $\RG^n$ and $(\RF^{(n)}, \RX, \RZ_{\opw}^n))$ taking values in $\Fq$. Such a function exists since $I(\RG^n \wedge \RF^{(n)}, \RX, \RZ_{\opw}^n) \neq 0$. So, we can write \begin{align}
    \RL^{(n)}=\RG^n\MM_K= a\RX + \RF^{(n)}\MM_{F}+\RZ_{\opw}^n \MM_{\opw} \label{eq:func:mcf_lin_perfect}
\end{align}
for some non-zero element $a \in \Fq$, and some  column vectors $\MM_K \neq \M0, \MM_{F},$ and $\MM_{\opw}$ over $\Fq$. (Here, $\RL^{(n)},\RG^n,\RF^{(n)}$ and $\RZ_{\opw}^n$ are the random row vectors with entries uniformly distributed over $\Fq$.) Note if the coefficient $a$ in the above linear combination is zero element in $\Fq$, then \eqref{eq:func:additional_secrecy} is satisfied for any $\RF'^{(n)}$. In particular, we can choose $\RF'^{(n)}$ to be an omniscience scheme, which satisfies \eqref{eq:func:additional_recov}.

Now consider the case of non-zero $a$. Define $\RF'^{(n)}:= \RG^n\MM_K - a\RX$. User $i$ can compute $\RF'^{(n)}$, as it is a function of  $\RG^n$ and $\RZ_i^n$, and transmit it publicly. Let us verify that $\RF'^{(n)}$ satisfies \eqref{eq:func:additional_recov} and \eqref{eq:func:additional_secrecy}. For \eqref{eq:func:additional_recov}, observe that  $H(\RX\mid \RF^{(n)},\RF'^{(n)}, \RZ_j^n) \leq H(\RX\mid \RF'^{(n)}, \RG^n)=0$, the inequality following from $H(\RG^n\mid\RF^{(n)},\RZ_j^n)=0$, and the  equality from the fact that $\RX$ is recoverable from $(\RF'^{(n)}, \RG^n)$. For \eqref{eq:func:additional_secrecy}, $I(\RG^{(n)} \wedge \RF^{(n)},\RF'^{(n)}| \RZ_{\opw}^n) - I(\RG^{(n)} \wedge \RF^{(n)}| \RZ_{\opw}^n)=I(\RG^n\wedge \RF'^{(n)}|\RF^{(n)}, \RZ_{\opw}^n)\leq H(\RF'^{(n)}|\RF^{(n)}, \RZ_{\opw}^n) = 0$,  which follows from \eqref{eq:func:mcf_lin_perfect} and the definition of $\RF'^{(n)}$. This completes the proof.
\end{proof}

% \begin{theorem}
%     For finite linear source $(\RZ_V,\RZ_{\opw})$, if a linear discussion scheme achieves $\wskc$, then $\wskc$ can be achieved via omniscience. Furthermore, the minimum leakage for secure omniscience 
%     $$\rl = H(\RZ_V|\RZ_{\opw}) - \wskc.$$
% \end{theorem}

% \begin{proof}
% \begin{align*}
%     nH(\RZ_V|\RZ_{\opw})&\geq I(\RK^{(n)},\RF^{(n)} \wedge \RZ_V^n \mid \RZ_{\opw}^n)\\
%                         & = I(\RF^{(n)} \wedge \RZ_V^n \mid \RZ_{\opw}^n)+I(\RK^{(n)} \wedge \RZ_V^n \mid \RZ_{\opw}^n,\RF^{(n)})\\
%                         & \geq I(\RF^{(n)} \wedge \RZ_V^n \mid \RZ_{\opw}^n) + n(\wskc-\delta_n) 
% \end{align*}
% for some $\delta_n \to 0$, where the last inequality follows from the fact an optimal key is recoverable from $\RZ_V^n$. By taking limsup on both side of the above inequality after normalizing by $n$, we get $ H(\RZ_V|\RZ_{\opw})\geq \limsup_{n \to \infty}\frac{1}{n}I(\RF^{(n)} \wedge \RZ_V^n \mid \RZ_{\opw}^n)  - \wskc \geq \rl- \wskc$. Therefore, $\rl \leq H(\RZ_V|\RZ_{\opw}) -\wskc$.

% \end{proof}



% \begin{theorem}\label{thm:fls:func:nec:suff}
%     For a finite linear source $(\RZ_V, \RZ_{\opw})$, if  a perfect linear discussion achieves $\wskc$, then the minimum leakage rate for computing a linear function $\RG$ is 
%     $$\rl = |H(\RG|\RZ_{\opw}) - \wskc|^{+},$$
%     which can be achieved via omniscience.
% \end{theorem}
% \begin{proof}
%  First note that for finite linear sources, if we can show the secure computability condition in Lemma~\ref{lemma:sc} for linear functions, then \eqref{eq:func:11} holds for all linear function. The proof of this result is analogous to the proof of Lemma~\ref{lemma:sc}. It suffices to consider linear functions with $H(\RG|\RZ_{\opw})< \wskc$ and show that they are securely computable, i.e., $\rl=0$.

% Fix a linear function $\RG$ with $H(\RG|\RZ_{\opw})< \wskc$. Let $\wskc- H(\RG|\RZ_{\opw}):=\frac{\epsilon}{2}$.
% As a perfect linear discussion scheme achieves 
% $\wskc$, by the proof of Theorem~\ref{thm:linearscheme}, there exists a linear wiretap secret key agreement scheme $(\RK^{(n)}, \RF^{(n)})$ that attains $\wskc$ via an linear omniscience scheme $\RF^{(n)}$.  As $\RK^{(n)}$ is optimal, we choose $n$ such that $\frac{1}{n}H(\RK^{(n)}) >\wskc -\frac{\epsilon}{2}$. From $\RK^{(n)}$,  we can extract a linear key $\tRK^{(n)}$ of rate $I(\RG^n \wedge \RF | \RZ_{\opw}^n)$ from $\RK^{(n)}$ for large enough $n$. This is possible because 
%     \begin{align*}
%         H(\RK^{(n)})&\geq H(\RK^{(n)}|\RG^n, \RF^{(n)}, \RZ_{\opw}^n)\\
%         &\geq H(\RK^{(n)}|\RF^{(n)}, \RZ_{\opw}^n) - H(\RG^n| \RF^{(n)}, \RZ_{\opw}^n)\\
%         &\stackrel{(a)}{=}H(\RK^{(n)}) - H(\RG^n| \RF^{(n)}, \RZ_{\opw}^n)\\
%         & > n\wskc -\frac{n\epsilon}{2}- H(\RG^n| \RF^{(n)}, \RZ_{\opw}^n)\\
%         &= n\wskc -\frac{\epsilon}{2} - H(\RG^n|\RZ_{\opw}^n)+I(\RG^n \wedge \RF^{(n)} | \RZ_{\opw}^n)\\
%         &\stackrel{(b)}{>}I(\RG^n \wedge \RF^{(n)} | \RZ_{\opw}^n),      
%     \end{align*}
% where $(a)$ follows from the secrecy condition of key and $(b)$ follows from $\wskc- H(\RG|\RZ_{\opw})=\frac{\epsilon}{2}$.

% Without loss of generality, consider $H(\RZ_{\opw}^n|\RZ_V^n)=0$ because any component of $\RZ_V^n$ that is not spanned by $\RZ_{\opw}^n$ can be reduced without changing $\rl$ or $\wskc$. We use the notation $\RU \in  \ll\RV\gg$ to denote that the random vector $\RU$ lies in the linear span of the random vector $\RV$, i.e., $\RU = \RV \MM$ for some matrix over $\Fq$. Also, we use the notation $\RS \in  \ll\RT\gg \cap \ll\RU\gg \setminus \ll\RV\gg$ to denote that $\RS$ lies in the linear span of $\RT$ and $\RU$ but not in the linear span of $\RV$, i.e., there exist matrices $\MM_T$ and $\MM_U$ such that $\RS = \RT \MM_T$ and $\RS = \RU \MM_U$, and there exists no matrix $\MM_V$ such that $\RS = \RV \MM_V$. With this notation, we can decompose $\RG^n$ as $\bM \RG_1^n & \RG_2^n &\RG_3^n \eM$, where $\RG_1^n,\RG_2^n,$and $\RG_3^n$ are mutually independent, $\RG_1^n \in \ll\RZ_{\opw}^n\gg$, and  $\RG_2^n \in \ll\RG^n \gg \cap \ll\RF^{(n)}\gg \setminus \ll \RZ_{\opw}^n\gg$. Moreover, these components are maximal in the sense there is no component of $\RG_3^n$ lies in the linear span of $(\RZ_{\opw}^n, \RF^{(n)})$.
    
% We can write the finite linear source $\RZ_i^n$ as 
% $$\RZ_i^n = \RX^n\MM_i$$
% for $i \in V$, where $\RX^n$ can be decomposed as $\bM \RX_1^n & \RX_2^n &\RX_3^n \eM$ such that $\RX_1^n, \RX_2^n$ and $\RX_3^n$ are mutual independent, and $ \RX_1^n \in \ll\RZ_{\opw}^n\gg, \RX_2^n \in  \ll \RG_2^n \gg, \RG_2^n \in  \ll \RX_2^n \gg$ with no component of $\RX_3^n$ lying in the linear span of $\RZ_{\opw}^n$. 


% Now we will specify a protocol for the secure computation of $\RG$. Assume that a key $\RK$ is available with all the users such that $H(\RK)=H(\RX_2)$ and $\tRK$ is independent of $\RZ_V^{(n)}$ and $\RZ_{\opw}^n$. Users consider the following new observations (which they locally compute using $\RK$)
%     \begin{align*}
%         \tRZ_i^n&=\bM\RX_1^n & \RX_2^n+\RK & \RX_3^n\eM\MM_i\\
%         &= \bM\RX_1^n & \RX_2^n & \RX_3^n\eM\MM_i + \bM0 & \RK & 0\eM\MM_i \\
%         &= \RZ_i^n + \bM0 & \RK & 0\eM\MM_i
%     \end{align*}
%     for $i \in V$
%     and discuss according to the function $\RF^{(n)}$ on $\tRZ_V^n$ instead of $\RZ_V^n$. Since $\tRF=\RF^{(n)}(\tRZ_V^n)$ is an omniscience scheme, users can recover $\tRZ_V^n$ from which they recover $\RZ_V^n$ using $\RK$. From the recovered source $\RZ_V^n$, users compute $\RG^n$. We will now show that this scheme has zero leakage rate.
%     \begin{align*}
%         I(\RG^n \wedge \tRF|\RZ_{\opw}^n) & = I(\RG_1^n,\RG_2^n, \RG_3^n \wedge \tRF|\RZ_{\opw}^n) \\
%         & \stackrel{(a)}{=} I(\RG_2^n, \RG_3^n \wedge \tRF|\RZ_{\opw}^n) \\
%         &\stackrel{(b)}{=}I(\RG_2^n \wedge \tRF|\RZ_{\opw}^n)\\
%         &\leq I(\RG_2^n \wedge \RX_1^n, \RX_2^n+\RK, \RX_3^n|\RZ_{\opw}^n)\\
%         &\stackrel{(c)}{=} I(\RG_2^n \wedge  \RX_2^n+\RK|\RZ_{\opw}^n)\\
%         &\stackrel{(d)}{=} I(\RX_2^n \wedge  \RX_2^n+\RK|\RZ_{\opw}^n)\\
%         &\stackrel{(e)}{=}0
%     \end{align*}
%     where $(a)$ follows from the condition $\RG_1^n \in \ll\RZ_{\opw}^n\gg$;  $(b)$ follows from combining the condition that $\RG_3^n$ is independent of $\RF^{(n)}(\RZ_V^n)$, $\RZ_{\opw}^n$ and $\RK$ with the fact that $\RF^{(n)}(\tRZ_V^n)$ is a linear function of $\RF^{(n)}(\RZ_V^n)$ and $\RK$; $(c)$ uses the condition $\RX_1^n \in \ll\RZ_{\opw}^n\gg$  and the condition that $\RX_3^n$ is independent of $\RX_2^n$, $\RZ_{\opw}^n$ and $\RK$; $(d)$ is because $\RX_2^n \in  \ll \RG_2^n \gg, \RG_2^n \in  \ll \RX_2^n \gg$; and $(e)$ is because $\RX_2^n$ is independent of $\RZ_{\opw}^n$ and $\RK$.


% To obtain the desired key $\RK$, we consider $m$ blocks of $\RZ_V^n$. On the first block, users run the secret key generation protocol $(\tRK^{(n)}(\RZ_{V,1}^n), \RF^{(n)}(\RZ_{V,1}^n))$, where $\RF^{(n)}$ is an omniscience scheme. On the second block, users use the key $\RK:=\tRK^{(n)}(\RZ_{V,1}^n)$ to get the padded source $\tRZ_{V, n+1}^{2n}$ and  they communicate $\RF^{(n)}(\tRZ_{V, n+1}^{2n})$ as described above. The users then use $\RX_{2,n+1}^{2n}$ for the third block and so on. Finally, the communication for $m$ blocks is of the form $\RF'=(\RF^{(n)}(\RZ_{V,1}^n)), \RF^{(n)}(\tRZ_{V, n+1}^{2n}), \ldots, \RF^{(n)}(\tRZ_{V, mn-n+1}^{mn}))$, which achieves omniscience from which users recover $\RG^{mn}$. The leakage rate for this scheme is 
% \begin{align*}
%     \frac{1}{nm}I(\RG^{nm} \wedge \RF'| \RZ_{\opw}^{nm})&= \frac{1}{mn}\left[H(\RG^{nm}| \RZ_{\opw}^{nm})) - H(\RG^{nm}|\RF', \RZ_{\opw}^{nm})) \right]\\
%     &=\frac{1}{mn}\left[\sum_{i=1}^n H(\RG_{(i-1)n+1}^{in}| \RZ_{\opw,(i-1)n+1}^{in}) - H(\RG^{nm}|\RF', \RZ_{\opw}^{nm})) \right]\\
    
%     +\frac{1}{m}\sum_{i=2}^mI(\RG_{(i-1)n+1}^{in} \wedge \RF'| \RZ_{\opw}^{nm}, \RG_{(i-2)n+1}^{(i-1)n})\\
%     &\to 0
% \end{align*}
%     This completes the proof.
% \end{proof}

% \begin{corollary}
% For a tree-PIN source, the minimum leakage rate for computing a linear function $\RG$ is 
%     $$\rl = |H(\RG|\RZ_{\opw}) - \wskc|^{+},$$
%     which can be achieved via omniscience.
% \end{corollary}
% \begin{proof}
%     It follows from Theorem~\ref{thm:cwsk:red} and its proof that $\wskc$ is achieved by a perfect linear discussion scheme. Therefore, we can apply Theorem~\ref{thm:fls:func:nec:suff} to obtain this result.
% \end{proof}

% \begin{example}
% Let $V=\{1, 2, 3\}$, and let $\RX= (\RX_{a}, \RX_{b}, \RX_{c}, \RX_{d})$ be a random vector uniformly distributed over $\mathbb{F}_2^4$. Consider the three-user FLS  $(\RZ_V, \RZ_{\opw})$ with
% \begin{gather*}
%    \RZ_1= (\RX_a, \RX_b, \RX_c)  \quad \RZ_2= (\RX_b, \RX_c, \RX_d) \quad \RZ_{\opw}= (\RX_b+\RX_c, \RX_a+\RX_d).
% \end{gather*}
% For this source, $\RG_1=\RG_2 = \RG=\RX_b+\RX_c$.
% It follows from Theorem~\ref{thm:fls} that $\wskc(\RZ_V \| \RZ_{\opw})= I(\RZ_1 \wedge \RZ_2|\RG)= H(\RZ_1|\RG)-H(\RZ_1|\RG, \RZ_2)= H(\RX_a, \RX_b, \RX_c|\RX_b+\RX_c)-H(\RX_a, \RX_b, \RX_c|\RX_b+\RX_c, \RX_b, \RX_c, \RX_d)= 2-1= 1 \text{ bit}$. On the other hand, the secret key capacity of this source is $\skc(\RZ_V)=I(\RZ_1 \wedge \RZ_2)=H(\RZ_1)-H(\RZ_1|\RZ_2)=3-1=2 \text{ bits}$; and the private key capacity is $\pkc(\RZ_V|\RZ_{\opw})=I(\RZ_1 \wedge \RZ_2|\RZ_{\opw})=H(\RZ_1|\RZ_{\opw})-H(\RZ_1|\RZ_2,\RZ_{\opw})=2-0=2 \text{ bits}$. Observe that $$1=\wskc(\RZ_V \| \RZ_{\opw})< \min \{\skc(\RZ_V), \pkc(\RZ_V|\RZ_{\opw})\}=2.$$

% Similarly, using the duality in Theorem~\ref{thm:fls} and the results of \cite{csiszar04}, we get $\rl (\RZ_V\|\RZ_{\opw})= H(\RZ_1,\RZ_2|\RZ_{\opw}) - \wskc(\RZ_V\|\RZ_{\opw})=2-1=1\text{ bit}$, $\rco(\RZ_V) = H(\RZ_1,\RZ_2) - \skc(\RZ_V)=4-2=2\text{ bits}$, and $\rco(\RZ_V|\RZ_{\opw}) = H(\RZ_1,\RZ_2|\RZ_{\opw}) - \pkc(\RZ_V|\RZ_{\opw}) = 2-2 =0 $. Thus, we have $\rco(\RZ_V|\RZ_{\opw})< \rl(\RZ_V\|\RZ_{\opw}) < \rco(\RZ_V)$.

% \end{example}



}