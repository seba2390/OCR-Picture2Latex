\begin{abstract}
    Pre-training models on Imagenet or other massive datasets of real images has led to major advances in computer vision, albeit accompanied with shortcomings related to curation cost, privacy, usage rights, and ethical issues. In this paper, for the first time, we study the transferability of pre-trained models based on synthetic data generated by graphics simulators to downstream tasks from very different domains.
    In using such synthetic data for pre-training, we find that downstream performance on different tasks are favored by different configurations of simulation parameters (\eg lighting, object pose, backgrounds, etc.), and that there is no one-size-fits-all solution. It is thus better to tailor synthetic pre-training data to a specific downstream task, for best performance. We introduce \ours, a unified model mapping downstream task representations to optimal simulation parameters to generate synthetic pre-training data for them. \ours learns this mapping by training to find the set of best parameters on a set of ``seen'' tasks. Once trained, it can then be used to predict best simulation parameters for novel ``unseen'' tasks in one shot, without requiring additional training. Given a budget in number of images per class, our extensive experiments with 20 diverse downstream tasks show \ours's task-adaptive pre-training data results in significantly better downstream performance than non-adaptively choosing simulation parameters on both seen and unseen tasks. It is even competitive with pre-training on real images from Imagenet.  
    
    
    
\end{abstract}