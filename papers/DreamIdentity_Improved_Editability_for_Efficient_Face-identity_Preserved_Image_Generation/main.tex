\documentclass{article}


% if you need to pass options to natbib, use, e.g.:
%     \PassOptionsToPackage{numbers, compress}{natbib}
% before loading neurips_2023
\usepackage{natbib}
\setcitestyle{numbers,square}

% ready for submission
% \usepackage{neurips_2023}


% to compile a preprint version, e.g., for submission to arXiv, add add the
% [preprint] option:
\usepackage[preprint]{neurips_2023}


% to compile a camera-ready version, add the [final] option, e.g.:
%     \usepackage[final]{neurips_2023}


% to avoid loading the natbib package, add option nonatbib:
%    \usepackage[nonatbib]{neurips_2023}

\pdfoutput=1
\usepackage[utf8]{inputenc} % allow utf-8 input
\usepackage[T1]{fontenc}    % use 8-bit T1 fonts
\usepackage[hidelinks]{hyperref}       % hyperlinks
\usepackage{url}            % simple URL typesetting
\usepackage{booktabs}       % professional-quality tables
\usepackage{amsfonts}       % blackboard math symbols
\usepackage{nicefrac}       % compact symbols for 1/2, etc.
\usepackage{microtype}      % microtypography
\usepackage{xcolor}         % colors
\usepackage{floatrow}
\floatsetup[table]{capposition=top}
\floatsetup[figure]{capposition=bottom}
\usepackage{amsmath,amssymb,amsfonts}
\usepackage{mathtools}
%\usepackage{amsthm}
% \usepackage[pdfpagelabels=true,linktocpage]{hyperref}
\usepackage[usenames,dvipsnames]{color}
\usepackage{xcolor}
\usepackage{xspace}
\usepackage{microtype}
\usepackage{todonotes}
\usepackage{colonequals}
%\usepackage{wasysym}
 \usepackage{booktabs}



\usepackage{multirow}
\usepackage{multicol}

\usepackage{paralist}

\usepackage{subfigure}
\usepackage{url}
\usepackage{appendix}
\usepackage{graphicx}
%\usepackage{float}
%\usepackage[pdfpagelabels=true]{hyperref}
\usepackage{algorithm}
\usepackage{listings}
%\usepackage{newalg}
\usepackage{nicefrac}
\usepackage{tikz} 
\usepackage{pgfplots}
%\usepgfplotslibrary{external}
%\pgfplotsset{compat=1.8}
%\tikzexternalize[prefix=TikzPictures/]
%\usepackage{marvosym}
%\usepackage{wrapfig}




\lstset{
	basicstyle=\ttfamily,
    keywordstyle=\bfseries,
    showstringspaces=false,
    numbers=left,
    numberstyle=\tiny,
    morekeywords={}
}

\usetikzlibrary{arrows,decorations.pathmorphing,positioning,fit,trees,shapes,shadows,automata,calc} 
%\usetikzlibrary{patterns,arrows,arrows.meta,calc,shapes,shadows,decorations.pathmorphing,decorations.pathreplacing,automata,shapes.multipart,positioning,shapes.geometric,fit,circuits,trees,shapes.gates.logic.US,fit,automata,snakes,shapes.geometric}

\pagestyle{plain}

\tikzset{outline/.style args={#1}{%
  draw=#1,thick,fill=#1!50}}

\tikzset{
  dot hidden/.style={},
  line hidden/.style={},
  dice hidden/.style={},
  dot color/.style={dot hidden/.append style={color=#1}},
  dot color/.default=black,
  line color/.style={line hidden/.append style={color=#1}},
  line color/.default=black,
  dice color/.style={dice hidden/.append style={color=#1,fill}},
  dice color/.default=white
}\def\dotsize{0.1}
\newcommand{\drawdie}[2][]{%
\begin{tikzpicture}[x=1em,y=1em,#1]
  \draw 	[thick, rounded corners=0.5,line hidden,dice hidden] (0,0) rectangle (1,1);
  \ifodd#2
    \fill[dot hidden] (0.5,0.5) circle (\dotsize);
  \fi
  \ifnum#2>1
  \fill[dot hidden] (0.25,0.25) circle (\dotsize);
  \fill[dot hidden] (0.75,0.75) circle (\dotsize);
  \ifnum#2>3
    \fill[dot hidden] (0.25,0.75) circle (\dotsize);
    \fill[dot hidden] (0.75,0.25) circle (\dotsize);
    \ifnum#2>5
      \fill[dot hidden] (0.75,0.5) circle (\dotsize);
      \fill[dot hidden] (0.25,0.5) circle (\dotsize);
    \fi
  \fi
\fi
\end{tikzpicture}
}



\newcommand{\figref}[1]{Fig.~\ref{#1}}
\newcommand{\tblref}[1]{Table~\ref{#1}}
\newcommand{\secref}[1]{Section~\ref{#1}}
\renewcommand{\eqref}[1]{Equation~(\ref{#1})}

\def\availableat{\url{url-published-on-acceptance}}

\newcommand{\todo}[1]{{\color{red} TODO: {#1}}}
\newcommand{\newstuff}[1]{{\color{red} CHECK: {#1}}}
%\newcommand{\todo}[1]{{}}

\newcommand{\ckp}[2]{$CK_{#1}P_{#2}$}
\newcommand{\cext}{\ckp{8}{16}$ext$}
\newcommand{\cfin}{$F_{CK_{X}P_{Y}}$}
\newcommand{\cray}{\ckp{8}{8}$ray$}
\newcommand{\csin}{$SK_{8}P_{8}$}
\newcommand{\casin}{$SK_{combined}$}
%\renewcommand{\cext}{$SK_{8}K_{8}P_{8}$}
\newcommand{\ckpnl}[2]{$CK_{#1}P_{#2}nl$}


\makeatletter
\newcommand{\Spvek}[2][r]{%
	\gdef\@VORNE{1}
	\left(\hskip-\arraycolsep%
	\begin{array}{#1}\vekSp@lten{#2}\end{array}%
	\hskip-\arraycolsep\right)}

\def\vekSp@lten#1{\xvekSp@lten#1;vekL@stLine;}
\def\vekL@stLine{vekL@stLine}
\def\xvekSp@lten#1;{\def\temp{#1}%
	\ifx\temp\vekL@stLine
	\else
	\ifnum\@VORNE=1\gdef\@VORNE{0}
	\else\@arraycr\fi%
	#1%
	\expandafter\xvekSp@lten
	\fi}
\makeatother
\usepackage{color}


\title{\ours: 
Improved Editability for Efficient Face-identity Preserved Image Generation}

\author{
    Zhuowei Chen\textsuperscript{\rm 1,2}\quad 
    Shancheng Fang\textsuperscript{\rm 2}\quad  
    Wei Liu\textsuperscript{\rm 2} \quad Qian He\textsuperscript{\rm 2} \quad
    Mengqi Huang\textsuperscript{\rm 1}\\ 
    \textbf{Yongdong Zhang\textsuperscript{\rm 1}\quad
    Zhendong Mao\textsuperscript{\rm 1}\thanks{Corresponding authors} }\\
    \textsuperscript{\rm 1}~University of Science and Technology of China \quad 
    \textsuperscript{\rm 2}~ByteDance Inc.\quad  \\
    \{chenzw01, huangmq\}@mail.ustc.edu.cn \quad \{zdmao, zyd73\}@ustc.edu.cn \\
    \{fangshancheng.lh, liuwei.jikun, heqian\}@bytedance.com\\
\textcolor{red}{\url{https://dreamidentity.github.io/}}
}

\begin{document}


\maketitle
%!TEX root = ms.tex
\begin{abstract}
%Recently, the problem of detecting and categorizing semantic relationship mentions from a given context has received a significant amount of attention.
% Extracting entity relationships for types of interests from text is important for understanding massive text corpora.
Relation extraction is a fundamental task in information extraction.
% Most existing systems for relation extraction heavily rely on manual labeling by human experts to create training data---a process that is costly, non-scalable, and hardly portable across different corpora.
% Most existing principles heavily rely on annotations given by human experts, which is costly and time-consuming.
Most existing methods have heavy reliance on annotations labeled by human experts, which are costly and time-consuming.
%To break the bottleneck of labeled data, knowledge bases like Freebase have been utilized to provide \ds. However, for many domain specific corpora, \ds is either non-existent or insufficient, while other related information like domain specific patterns is available and could be used. In this paper, we combined different types of information to provide \hs and perform relation extraction.
% To break this bottleneck, knowledge bases have been utilized to generate noisy annotations and provide \ds, while for many domain specific corpora, it's either non-existent or insufficient. 
% Therefore, we proposed a novel framework to supervise relation extraction model with knowledge bases and more,   combined different types of information to provide \hs and perform relation extraction.
% To reduce human labeling effort, two kinds of methods are studied by prior work: 
% (1) taking a small set of human-crafted patterns (instead of fully-annotated sentences) as ``\textit{weak}" supervision; and (2) leveraging freely available relation information from external knowledge bases as ``\textit{distant}" supervision. 
% However, both methodologies encounter challenges in dealing with domain-specific corpora. 
% On one hand, 
% Weak supervision, guided by domain knowledge, can generate high-quality but \textit{limited} amount of labeled data, due to its dependence on substantial human effort, 
% On the other hand, 
% whereas distant supervision can automatically produce a large amount of labeled data but the labels so generated may not be ``\textit{perfect}" for individual mentions.
% To overcome this drawback, we conduct relation extractor learning under annotations generated by heterogeneous information (e.g., knowledge base and domain heuristics), which are noisy but require l, and is referred as \hs. 
To overcome this drawback, we propose a novel framework, \our, to conduct relation extractor learning using annotations from heterogeneous information source, e.g., knowledge base and domain heuristics.
% In this paper, we study how to leverage \textit{heterogeneous supervision} (\ie, combining weak supervision and distant supervision) to perform relation extraction in an \textit{effecdtive} way. 
These annotations, referred as \hs, often conflict with each other, which brings a new challenge
 to the original relation extraction task: how to infer the true label from noisy labels for a given instance.
% And the challenges here are relation extraction task itself and more, resolving the conflicts among \hs.
% A key challenge is how to integrate the two kinds of sources while eliminating the conflicts among them in a trustworthy manner.
% In this paper, We propose a novel framework, \our, to jointly conducts relation extractor learning and context-aware truth discovery.
% Specifically, to resolve conflicts among \hs, true label discovery is conducted in a context-aware manner, while context information, or text feature, also serves as the backbone of relation extraction.
Identifying context information as the backbone of both relation extraction and true label discovery, we adopt embedding techniques to learn the distributed representations of context, which bridges all components with mutual enhancement in an iterative fashion.
% and allows them to enhance each other. 
%Specifically, we adopted an embedding method to learn distributed representations of text features, relation types and supervisions, which bridges the context-aware truth discovery module and the relation extraction module.
Extensive experimental results demonstrate the superiority of \our over the state-of-the-art.
\end{abstract} 
Reinforcement learning has achieved great success in areas such as Game-playing \citep{silver2018general,vinyals2019grandmaster}, robotics \cite{kober2013reinforcement}, large language models \citep{ouyang2022training}, etc.
However, due to safety concerns or physical limitations, in some real-world reinforcement learning problems, we must consider additional constraints that may influence the optimal policy and the learning process \citep{garcia2015comprehensive}.
% For example, a robotic arm must not take actions that may cause harm to itself or the environments.
A standard framework to handle such cases is the constrained Markov Decision Process (CMDP) \citep{altman1999constrained}.
Within the CMDP framework, the agent has to maximize
the expected cumulative reward while
obeying a finite number of constraints, which are usually in the form of expected cumulative cost criteria.

However, we are sometimes concerned with the problem with a continuum of constraints.
For example,
the constraints we meet might be time-evolving or subject to uncertain parameters, which
cannot be formulated as an ordinary CMDP
(see Examples \ref{Example_Time_Evolving} and  \ref{Example_Uncertain}).
In this paper we would study a generalized CMDP  
to address the above problem.  Because the constraints are not only infinite-number but also lie
in a continuous set,
the generalization is not trivial. Fortunately, we find that we can borrow the idea behind semi-infinite programming (SIP) \citep{remez1934determination, hettich1993semi} to deal with the semi-infinite constraints.
Accordingly, we propose \emph{semi-infinitely constrained Markov decision processes} (SICMDPs)
as a novel complement to the ordinary CMDP framework.
%More specifically,  an SICMDP model %, we consider 
%contains a continuum of constraints whereas an ordinary CMDP contains a finite number of constraints. 

%This generalization is natural but not trivial. However, we can brows the idea  
%The idea is quite natural and can be backtracked
%to the practice of extending linear programming to linear semi-infinite programming (LSIP) %\cite{remez1934determination, GobernaLSIO1998}.
%In addition, 
%As a complementary approach to the ordinary CMDP framework, 
%SICMDP can be used to model these problems  which cannot be described by a finite number of constraints
%that are not covered by .
%For example,
%the restrictions we consider can be time-evolving or subject to uncertain parameters
%, thus
%cannot be described by a finite number of constraints but a continuum of constraints 
%(see Examples \ref{Example_Time_Evolving} and  \ref{Example_Uncertain}).

We also present two reinforcement learning algorithms to solve SICMDPs called SI-CRL and SI-CPO, respectively.
SI-CRL is a model-based reinforcement learning algorithm designed for tabular cases, and SI-CPO is a policy optimization algorithm for non-tabular cases.
% and analyze its performance both theoretically and empirically.
The main challenge is that we need to deal with a continuum of constraints, thus reinforcement learning algorithms for ordinary CMDPs do not work anymore.
In SI-CRL, we tackle this difficulty by first transforming the reinforcement learning problem to an equivalent LSIP problem, which can then be solved using methods in the LSIP literature like the dual exchange methods \citep{Hu1990,reemtsen1998numerical}.
In SI-CPO, we resort to the idea of cooperative stochastic approximation developed in \cite{lan2020algorithms, wei2020comirror}.
As far as we know, we are the first to introduce tools from semi-infinitely programming (SIP) into the reinforcement learning community for solving constrained reinforcement learning problems.

% To the best of our knowledge, we are the first to apply tools from semi-infinitely programming (SIP) to solve reinforcement learning problems.
Furthermore, we give theoretical analysis for both SI-CRL and SI-CPO.
We decompose the error of SI-CRL into two parts: the statistical error from approximating the true SICMDP with an offline dataset and the optimization error due to the fact that the solution of the LSIP problem obtained by the dual exchange method is inexact.
On the optimization side, we show that the iteration complexity of SI-CRL is $O\left(\left\{\mathrm{diam}(Y)L\sqrt{|\gS|^2|\gA|m}/\left[(1-\gamma)\epsilon\right]\right\}^m\right)$.
On the statistical side, we show that the sample complexity of SI-CRL is $\widetilde O\left(\frac{|S|^2|A|^2}{\epsilon^2(1-\gamma)^3}\right)$ if the offline dataset is generated by a generative model, and $\widetilde O\left(\frac{|S||A|}{\nu_{\min} \epsilon^2(1-\gamma)^3}\right)$ if the dataset is generated by a probability measure $\nu$ as considered in \cite{chen2019information}.
Here $\widetilde O$ means that all logarithm terms are discarded.
For SI-CPO, things become a little more complicated because other than the statistical error and the optimization error, we also need to consider the function approximation error, which comes from imperfect policy parametrizations.
It is shown if the function approximation error can be controlled to $O(\epsilon)$ order, the iteration complexity of SI-CPO is $\widetilde{O}\left(\frac{1}{\epsilon^2(1-\gamma)^6}\right)$ and the sample complexity of SI-CPO is $\widetilde{O}(\frac{1}{\epsilon^4(1-\gamma)^{10}})$.
Here our iteration complexity bound is equivalent to a typical $\widetilde O(1/\sqrt{T})$ global convergence rate.

We perform a set of numerical experiments to illustrate the SICMDP model and validate our proposed algorithms.
Specifically, we examine two numerical examples, namely the discharge of sewage and ship route planning.
Through the discharge of sewage example, we show the advantage of the SICMDP framework over the CMDP baseline obtained by naive discretization in modeling realistic sequential decision-making problems.
Moreover, we demonstrate the effectiveness of the SI-CRL and SI-CPO algorithms in such tabular environments. 
In the ship route planning example, we illustrate the benefits of the SICMDP framework and the ability of the SI-CPO algorithm to address complex continuous control tasks involving continuous state spaces with modern deep reinforcement learning techniques.

% In summary, our contributions are listed as follows.
% First, we present the SICMDP model, which can be viewed as a generalization of the ordinary CMDP model.
% Second, we propose an algorithm to perform reinforcement learning for SICMDPs, which is called SI-CRL, and we believe that we are the first to apply tools from SIP
% to solve reinforcement learning problems.
% Third, we give a theoretical analysis of SI-CRL and identify both its sample complexity and iteration complexity.
% In addition, we perform numerical experiments to illustrate the SICMDP model and validate the SI-CRL algorithm.
% \{This paragraph can be removed!!! \}





\vspace{-0.3cm}
\textbf{Related work}:
% Object detection related datasets/algo in non-medical domain
% Locally labeled CXR dataset
A few CXR datasets have localized abnormality annotations \cite{shih2019augmenting,filice2020crowdsourcing,jaeger2014two} that are curated manually. These are high quality gold standard ground truth datasets but tend to be smaller in scale (< 30,000 images) and have a narrow coverage, with typically only 1-2 labels. In addition, since most labeling efforts only have abnormality semantics attached, no direct relationships with the affected anatomical locations are available. 

%MEHDI: repeated concepts from above. I am removing the following: 

%The lack of anatomic semantics in the annotation is a limitation for complex multi-modal clinical reasoning work, e.g., differential diagnosis, since clinicians often integrate information along anatomical lines, and for downstream report generation tasks, which often requires describing not only the abnormality but also correctly communicate the location of the abnormalities (and medical devices) to the receiving clinicians. 

Two recent CXR datasets have labels for anatomies described in the reports. In \cite{datta2020dataset}, a small manually annotated dataset (2000 reports) included 10 abnormalities that are individually associated with 29 unique spatial locations (anatomies) at the report level. Another CXR dataset has automatically extracted abnormality and anatomy labels as disconnected concepts that are only correlated at the study level from  160,000 reports using a supervised NLP algorithm \cite{bustos2020padchest}. This was trained on a smaller set of manually annotated data. Neither datasets contain localized annotations for the associated CXR images, nor any comparison relation annotations between sequential exams, both of which are available in the Chest ImaGenome dataset. In Table \ref{tab:related}, we present a comparison of our Chest ImagGenome dataset with other datasets available in the literature.

% Table -- Kashyap

% MEdical imaging datasets to go here: Discussed that we will only focus on cxr datasets that are available for this paper. 
% \caption{\color{red} Kashyap, feel free to continue with the table. We should remove the questionmarks and add a line for our dataset (since all others are not graph). For longer text, using abbreviations and explaining them in the caption often works better. If fill in the values is not possible, it is better to remove the table altogether.}


\begin{table}[t!]
\caption{Summary of existing chest X-ray datasets}
\resizebox{\textwidth}{!}{%
\begin{tabular}{@{}lllllllll@{}}
\toprule
\textbf{Dataset} & \textbf{Annotation Level} & \textbf{Annotation Method} & \textbf{Num Labels} & \textbf{Anatomy Labeled} & \textbf{Graph} & \textbf{Dataset Size} & \textbf{Temporal Labels} & \textbf{Reports} \\ \midrule
SIIM-ACR Pneumothorax Segmentation \cite{filice2020crowdsourcing} & Segmentation & Manual + augmented & 1 & No & No & 12,047 & No & No \\
RSNA Pneumonia Detection Challenge   \cite{shih2019augmenting} & Bounding Boxes & Manual & 1 & No & No & 30,000 & No & No \\
Indiana University Chest X-ray collection \cite{demner2016preparing} & Global & Automated & 10 & No & No & 3,813 & No & Yes \\
NIH CXR dataset \cite{wang2017chestx} & Global & Automated & 14 & No & No & 112,120 & No & No \\
PLCO \cite{team2000prostate} & Global & Automated & 24 & Yes & No & 236,000 & Yes & No \\
Stanford CheXpert \cite{irvin2019chexpert} & Global & Automated & 14 & No & No & 224,316 & No & No \\
MIMIC-CXR \cite{johnson2019mimic} & Global & Automated & 14 & No & No & 377,110 & No & Yes \\
Dutta \cite{datta2020dataset} & Global & Manual & 10 & Yes & Yes & 2,000 & No & Yes \\
PadChest \cite{bustos2020padchest} & Global & Manual + automated & 297 & Yes & No & 160,868 & No & Yes \\
Montgomery County Chest X-ray   \cite{jaeger2014two} & Segmentation & Manual & 1 & Yes & No & 138 & No & No \\
Shenzen Hospital Chest X-ray   \cite{jaeger2014two} & Segmentation & Manual & 1 & Yes & No & 662 & No & No \\  \hline \hline
\textbf{Chest ImaGenome} & Bounding Boxes & Automated & 131 & Yes & Yes & 242,072 & Yes & Yes \\
\bottomrule
\end{tabular}%
}
\label{tab:related}
\vspace{-0.4cm}
\end{table}
% removed (Derived from MIMIC-CXR \cite{johnson2019mimic}) % makes table really small

\section{Methods}\label{sec:methods}

Given a single facial image of an individual, our objective is to endow the pre-trained T2I model with the ability to efficiently re-contextualize this unique identity under various textual prompts. These prompts may include variations in clothing, accessories, styles, or backgrounds.


The overall framework is shown in Fig.\ref{fig:pipeline}, given a pre-trained T2I model, 
to achieve fast and identity-preserved image generation, we first directly encode the target identity into the word embedding  space (represented as the pseudo word $S*$) with the proposed $M^2$ ID encoder. Afterward,
$S*$ is integrated with the input template prompt for 
generating the text-guided image. To empower the editability for the target identity, a novel \emph{self-augmented editability learning} is further introduced to train the $M^2$ ID encoder with the editability objective.


In the following parts, we first briefly introduce the pre-trained diffusion-based text-to-image model used in our work, then describe our proposed  $M^2$ ID encoder and self-augmented editability learning in detail, respectively.

\subsection{Preliminary}
In this work, we adopt the open-sourced Stable Diffusion 2.1-base (SD) as our text-to-image model, which has been trained on billions of images and shows amazing image generation quality and prompt understanding. 

SD is a kind of Latent Diffusion Model (LDM) \cite{rombach2022high}. LDM firstly represents the input image $x$ in a lower resolution latent space $z$ via a Variational Auto-Encoder (VAE) \cite{kingma2013auto}. Then a text-conditioned diffusion model is trained to generate the latent code of the target image from text input $c$. The loss function of this diffusion model can be formulated as:
\begin{equation}
    \mathcal{L}_{diffusion} = \mathbb{E}_{\epsilon,z,c,t}[\lVert{\epsilon - \epsilon_{\theta}(z_t,c,t)}\rVert_2^2],
\end{equation}
where $\epsilon_{\theta}$ is the noise predicted by the model with learnable parameters $\theta$, $\epsilon$ is noise sampled from standard normal distribution, $t$ is the time step, and $z_t$ is noisy latent at the time step $t$.

During inference, the image is generated by two stages: latent code is first generated by the diffusion model, then the decoder is employed to map the latent code to image space. 

\subsection{$M^{2}$ ID Encoder}

To accurately represent the input face identity, we propose a novel Multi-word Multi-scale embedding ID encoder ($M^2$ ID encoder) for an accurate mapping, which is achieved by the multi-scale ID features extracted from a dedicated backbone then followed by multiple word embedding projection.




\myparagraph{Backbone.} We argue that an accurate representation of the face identity is crucial, while common image encoder CLIP (which is adopted by \emph{all} existing works) fails for that purpose since it can not capture the identity feature as accurately as the face ID encoder which has been trained for face identification tasks on the large-scale face dataset. As \cite{bhat2023face} shows, the current best CLIP VIT-L/14 model is still much worse than the face recognition model on top-1 face identification accuracy ($80.95\%$ vs $87.61\%$). Therefore, we employ a ViT backbone \cite{dosovitskiy2020image} pre-trained on a large-scale face recognition dataset to faithfully extract ID-aware features for input face.


\myparagraph{Multi-scale Feature.}  However, naively mapping the final layer's output identity vector $v_{final}$ could only bring sub-optimal identity preservation. The reason lies in that $v_{final}$ mainly contains the high-level semantics which is suitable for discriminative tasks (\eg, face identification) rather than generative tasks. For example, the same identity with different expressions should share similar representation under the face recognition training loss, while the generation requests more detailed information like facial expressions. Hence, only mapping the last layer representation could become an information bottleneck for the image generation task. To deal with the above problem, we propose to utilize multi-scale features from the face encoder to represent an identity more faithfully. Specifically, the identity vector is augmented by four CLS embeddings ($v_3$, $v_6$, $v_{12}$, $v_{12}$) from the 3rd, 6th, 9th, and 12th layer, respectively. Formally, the multi-scale feature from the ID encoder is depicted as follows:
\begin{equation}
% \setlength\abovedisplayskip{1pt}
% \setlength\belowdisplayskip{1pt}
V = [v_3, v_6, v_9, v_{12}, v_{final}].
\end{equation}

\myparagraph{Multi-word Embeddings.} The multi-scale feature is further projected into the word embedding domain. To maintain the original large-scale T2I model's generalization and editability, we leave all its parameters and structure unchanged. As a result, it raises the problem that a single word embedding is hard to faithfully represent the face's identity. Therefore, we further propose a multi-word projection mechanism to represent a face with multi-word embedding:
\begin{equation}
\begin{aligned}
s_{i} = MLP_i(V), \text{for } i = 1, ..., k,
\end{aligned}
\end{equation}
where $k$ is the number of embeddings . Experimentally, we set $k=2$ as depicted in Fig.\ref{fig:pipeline}.  Following \cite{gal2023designing}, $l_2$ regularization is further adapted to constrain the output embedding:

\begin{equation}
    \mathcal{L}_{reg} = \sum_{i=1}^k{\lVert{s_{i}}\rVert}.
\end{equation}

Benefiting from the above-dedicated ID feature, we can facilitate highly identity-preservation control in the embedding space only, without sacrificing pre-trained T2I model's editability caused by feature injection. 

\subsection{Self-Augmented Editability  
 Learning}
Current efficient methods are trained under the reconstruction paradigm, which is given an input face image $I$, the objective to learn a unique word $S*$ such that the $S*$ can reconstruct $I$. However, in real-world applications, we wish to generate a set of new images, such as "watercolor style of $S*$ face", "$S*$ as a police". As a result, there exists a huge inconsistency between training and testing. We hope we can rely on the inductive bias in the word embedding space to achieve editability, but in reality, as Fig.\ref{fig:exp_self_aug} shows, the generated image doesn't always follow the text prompt if we only train encoder under the reconstruction objective. 

To deal with the inconsistency between training and testing, in this paper, we propose a novel \emph{self-augmented editability learning} to take the editing task into the training phase. However, collecting such pair data for the editing task is difficult. Experimentally,  we notice that the current state-of-the-art general text-to-image models can generate celebrity (\eg, Boris Johnson, Emma Watson) in different contexts with good identity preservation and text-coherence. With this insight, The \emph{self-augmented editability learning} utilizes the pre-trained model itself to construct a self-augmented dataset by generating various celebrity faces along with the target edited celebrity images, which will be used to train the $M^2$ ID encoder with the editability objective. Formally, the construction of the dataset includes the following four steps:


\myparagraph {Step 1: Celebrity List Generation.} Firstly we collect a candidate celebrity list. The large language model (\ie, ChatGPT) is used to generate the most famous 400 names in four fields (\ie, sports players, singers, actors, and politicians). After filtering duplicate ones, we finally get 1015 celebrity names.

\myparagraph {Step 2: Celebrity Face Generation.} We propose to use generated face images rather than real images because the model has its own understanding of celebrity. Specifically, the celebrities who appeared less frequently in the Stable Diffusion training dataset are not very similar to the real person while these generated faces maintain a high level of identity resemblance. We use the prompt template "<celebrity-name> face, looking at the camera" to produce the source images, then followed by face crop and alignment operation to get face-only images. A face-only image is kept if its short size is larger than 128 pixels. 

\myparagraph {Step 3: Edit Prompts and Edited Images Generation.} We manually design a variety of prompts that contain images of celebrities in different jobs, styles, and accessories (\eg, "<celebrity-name> as a chef", "oil painting style, <celebrity-name> face"). Then these prompts are transformed to images by the T2I model as edited images, and the <celebrity-name> in prompts is replaced by the pseudo word $S^*$ as Editing Prompts.

\myparagraph {Step 4: Data Cleaning.} After the above procedures, we can now get the initial self-augmented dataset consisting of a set of triplets, <identity face, editing prompt, edited images>. Due to the instability of the current diffusion model, the edited images don't always follow the edit instructions. 
Therefore, we need to filter out the noise data in the self-augmented dataset. We employ ID Loss and CLIP score which reflect identity similarity and text-image consistency as the metrics to rank the edited images for every prompt, then the top $25\%$ triplets at kept as the final training set. 

Finally, we construct a high-quality self-augmented dataset from the pre-trained T2I model itself, which is then used for edit-oriented training.


\subsection{Training}
We combine the FFHQ \cite{karras2019style} and the self-augmented dataset to train our proposed $M^2$ ID encoder. The total loss consists of noise prediction loss of diffusion and the embedding regularization loss:  
\begin{equation}
\mathcal{L}_{total} = \mathcal{L}_{diffusion} + \lambda \mathcal{L}_{reg} ,
\end{equation} 
where $\lambda$ is the embedding regularization weight.



\begin{table*}[t]
\centering
  \caption{Quantitative comparisons with the optimization-based and efficient methods. Encoding time means the time cost to obtain the unique/pseudo embedding. Our method achieves optimal results in terms of text-alignment, face similarity, and encoding time.}
  \label{tab:main_result}
  \begin{tabular}{cccc}
    \toprule
    Methods & Text-alignment $\uparrow$ & Face similarity $\uparrow$ & Encoding Time $\downarrow$ \\
    \midrule
    Textual Inversion \cite{gal2022image} & 0.213 & 0.326 & 20 min \\ 
    Dreambooth \cite{ruiz2022dreambooth} & 0.217 & 0.425 & 4 min  \\ 
    E4T \cite{gal2023designing} & 0.220 & 0.420 & 20 s \\ 
    Elite \cite{wei2023elite} & 0.196 & 0.450 & 0.05 s\\ 
    \midrule
    Ours & \textbf{0.228} & \textbf{0.467} & \textbf{0.04 s}\\
    \bottomrule
  \end{tabular}
\end{table*}


\section{Experiments}\label{sec:exp}


\subsection{Experiment Settings}
\myparagraph{Dataset.} Our experiments are conducted on the widely used FFHQ dataset \cite{karras2019style}, which contains $70000$ high-resolution human face images. We resize the images to 512x512 for training. The test set consists of 100 faces from \cite{liu2015faceattributes}. We make certain that there is no intersection between the test set and the self-augmented celebrity set to maintain the integrity of the experiment.

\myparagraph{Metrics.} We evaluate our method on Text-alignment and Face-similarity.  Text-alignment is used to indicate whether the generated image reflects editing prompts, which is calculated by the cosine distance in the CLIP text-image embedding space.  Face-similarity is used to measure whether the face ID is preserved. We use the ID feature from arcface \cite{deng2018arcface}, a model pre-trained on face recognition tasks, to represent the face identity. Then ID-similarity is measured by the cosine distance of ID features between the input face and the face cropped from the edited image. For each editing prompt and face identity, four images are generated.


\myparagraph{Implementation Details.} 
We choose Stable Diffusion 2.1-base as our base text-to-image model. The learning rate and batch size are set to $5e-5$ and $64$. The encoder is trained for 60,000 iterations. The embedding regularization weight $\lambda$ is set to $1e-4$. Our experiments are trained on a server with eight A100-80G GPUs, which takes about 1 day to complete each experiment. During inference, we use the DDIM \cite{song2020denoising} sampler with 30 steps. The guidance scale is set to 7.5.

\begin{table*}[t]
\centering
  \caption{Quantitative comparisons with the optimization-based and efficient methods. Encoding time means the time cost to obtain the unique/pseudo embedding. Our method achieves optimal results in terms of text-alignment, face similarity, and encoding time.}
  \label{tab:main_result}
  \begin{tabular}{cccc}
    \toprule
    Methods & Text-alignment $\uparrow$ & Face similarity $\uparrow$ & Encoding Time $\downarrow$ \\
    \midrule
    Textual Inversion \cite{gal2022image} & 0.213 & 0.326 & 20 min \\ 
    Dreambooth \cite{ruiz2022dreambooth} & 0.217 & 0.425 & 4 min  \\ 
    E4T \cite{gal2023designing} & 0.220 & 0.420 & 20 s \\ 
    Elite \cite{wei2023elite} & 0.196 & 0.450 & 0.05 s\\ 
    \midrule
    Ours & \textbf{0.228} & \textbf{0.467} & \textbf{0.04 s}\\
    \bottomrule
  \end{tabular}
\end{table*}
\begin{table}[t]
\centering
  \caption{Ablation study on $M^2$ ID Encoder. ID encoder with multi-scale feature (MS Feat) and multiple word embeddings (Multi Embedding) achieves best Face-similarity while maintaining a comparable result  Text-alignment metric.}
  \label{tab:id_feat_ablation}
  \begin{tabular}{ccccc}
    \toprule
    ID Encoder & MS Feat & Multi Embedding & Text-alignment $\uparrow$ & Face-similarity $\uparrow$  \\
    \midrule
               &          &             &    \textbf{0.229}   &   0.266   \\
     \checkmark &          &             &  0.228 & 0.302 \\
     \checkmark &  \checkmark &       & \textbf{0.229} & 0.412 \\
      \checkmark &  \checkmark &  \checkmark     & 0.228 & \textbf{0.467} \\
    \bottomrule
  \end{tabular}
\end{table}

\begin{figure}[t]
  \centering
  \includegraphics[width=\linewidth]{figs/exps/id_encoder_fig.pdf}
  \caption{Qualitative comparisons between ID Encoder and the multi-scale features. The editing prompt is "S* as a chef, looking at the camera". We could conclude that both ID Encoder and the multi-scale features greatly improve the ID preservation (\ie, face-similarity).}
  \label{fig:id_encoder}
  \vspace{-2mm}
\end{figure}
\subsection{Comparison to SOTA Methods}
In this section, we compare our method with fine-tuning based methods: Textual Inversion \cite{gal2022image}, DreamBooth \cite{ruiz2022dreambooth} and concurrent works on efficient personalized model: E4T \cite{gal2023designing} which requires finetuning for around 15 iterations for each face, and ELITE \cite{wei2023elite}, a fine-tuning free work. We adopt the widely-used open-sourced Diffusers codebase for Textual Inversion, DreamBooth, and re-implemented  E4T and  ELITE. To ensure a fair comparison, all experiments are conducted with a single face image input. 

\myparagraph{Quantitative and Qualitative Results.} As demonstrated in Tab.\ref{tab:main_result}, our work \ours \ outperforms recent methods across all the metrics, demonstrating superior performance in terms of \editb, \Imetric, and encoding speed. 
We show that \ours \ improves the text-alignment by $7\%$ compared to the second-best E4T \cite{gal2023designing}. Meanwhile,  \ours \ surpasses the second-best model \cite{wei2023elite} on \Imetric \  by $3.7\%$, while enjoying better editability. Benefiting from the direct encoding rather than optimization for unique embeddings, the additional computation cost is only 0.04 s, which can be negligible compared to the time cost (seconds-level) for a standard diffusion-based text-to-image process. The conclusion is further validated by the qualitative results in Fig.\ref{fig:main_result}.

\begin{figure}[t]
\CenterFloatBoxes
\begin{floatrow}
\ttabbox
{\begin{tabular}{cccc}
    \toprule
     {\scriptsize Recon} & {\scriptsize self-aug} & {\scriptsize Text-alignment $\uparrow$} & {\scriptsize Face similarity $\uparrow$ }  \\
    \midrule
    \checkmark &             &  0.213 & 0.380 \\
      &  \checkmark &   0.216 & 0.348 \\
      \checkmark &  \checkmark & \textbf{0.228} & \textbf{0.467} \\
    \bottomrule
  \end{tabular}
  }
  {\caption{Ablation study on self-augmented \editb \  learning. Recon denotes reconstruction training. self-aug denotes self-augmented \editb \  learning, the \editb \ gets improved after applying self-aug.}
  \label{tab:gen_data_ablation}
  }
\killfloatstyle
\ttabbox
{\begin{tabular}{ccc}
    \toprule
     {\scriptsize Emb Num} & {\scriptsize Text-alignment $\uparrow$} & {\scriptsize Face similarity $\uparrow$}  \\
    \midrule
    1           &    \textbf{0.229}   &   0.412   \\
     2           &  0.228 & \textbf{0.462} \\
     3    & 0.188 & 0.472 \\ 
    \bottomrule
  \end{tabular}
  }
  {\caption{Ablation study on the number of word embeddings (Emb Num). Single word embedding could limit the face-similarity while excessive ones may hinder text-alignment.}
  \label{tab:multi_token_ablation}
  }

\end{floatrow}
\end{figure}
\begin{figure}[t]
  \centering
  \includegraphics[width=\linewidth]{figs/exps/gen_fig.pdf}
  \caption{Qualitative comparisons on the self-augmented dataset for \editb \  learning. The editing prompt is "S* as a police, looking at the camera". "w/o edit" and "w/o recon" denote for the encoder is trained without \editb \ learning objective and without reconstruction learning, respectively. We show that the generated images can not follow the prompt properly without the \editb \ learning. Meanwhile, the face similarity will be lower without the reconstruction learning on FFHQ.}
  \label{fig:exp_self_aug}
\end{figure}
\begin{figure}[htb]
  \centering
  \includegraphics[width=\linewidth]{figs/exps/multi_token_fig.pdf}
  \caption{Qualitative comparisons of multiple word embeddings. The editing prompt is "S* as a police, looking at the camera", and "NUM" denotes the number of embeddings.}
  \label{fig:multi_token}
\end{figure}


\subsection{Ablation Studies}
\vspace{-0.2cm}
In this section, we conduct ablation studies to verify the effectiveness of our proposed  $M^2$ ID feature and self-augmented editability learning.
 
\myparagraph{$M^{2}$ ID encoder.} 
We adopt CLIP encoder as our baseline, which is commonly used by concurrent encoder-based methods. Following \cite{wei2023elite, shi2023instantbooth}, we use the last layer CLS feature from CLIP encoder to predict a word embedding. As Fig.\ref{fig:id_encoder} shows, this baseline generally failed to capture the core identity information in the input image, and in some cases, it doesn't even capture the gender information.  
Upon switching from the CLIP encoder to the face-specific ID encoder, the \Imetric \ is improved from $0.266$ to $0.302$, as shown in Tab.\ref{tab:id_feat_ablation}. Integrating the multi-scale features further boosts the \Imetric \ to $0.412$.
Multi-word embeddings are  further utilized to enhance ID-preservation. As shown in Tab.\ref{tab:multi_token_ablation} and Fig.\ref{fig:multi_token}, when we increase the number of embedding  to 2, the Face-similarity is improved by $12\%$ with marginal change of $0.4\%$ on text-alignment. However, when we further increase the number of word embedding,  text-alignment is dropped by $17\%$. We argue that excessive word embeddings may include more information beyond the ID feature such that hinder the editability. Therefore, we choose the embedding number as $2$ to avoid degraded editability.  

\myparagraph{Self-Augmented Editability Learning.} Next, we study the effectiveness of self-augmented editability learning. Fig.\ref{fig:exp_self_aug} indicates that if the model is only trained under the reconstruction objective, the editability \ of embeddings will be limited. To be specific, the model trained without the editability learning objective fails to edit the input identity to a police. Besides, if we only use the limited generated editing dataset, face similarity will be degraded in that there are only around 1000 face IDs in the self-augmented dataset. Combining the reconstruction data (i.e., FFHQ) and generated self-augmented dataset is a better choice to preserve face similarity while following the textual instruction. The quantitative results in Tab.\ref{tab:gen_data_ablation} further confirm our conclusion.




\subsection{Application}
\myparagraph{ID-preserved Scene Switch.} As illustrated in Fig.\ref{fig:anything}, given the input face ID and its location in the canvas indicated by the gaze location, we can generate a series of different scene images which share the same identity information and head location with the help of ControlNet \cite{zhang2023adding}. The scene is specified by the text description and can encompass different accessories, hair style, backgrounds, and styles. With this method, we may achieve the effect of "everything and everywhere all at once". 

\begin{figure}[H]
  \centering
  \includegraphics[width=\linewidth]{figs/app/anything_fig.pdf}
  \caption{Given a face identify and its gaze location on the canvas, our method can generate a series of images that maintain the same identity while following the editing prompts in the same location.}
  \label{fig:anything}
\end{figure}








\section{Limitation}
\setParDis
While our method offers an efficient approach to recreate a human image given one face image, there are several limitations should be noticed. (1) Our model is trained on the high-quality realistic face image dataset, so when the input is a poor-quality face or out-of-domain image, such as a partially obstructed image, the edited image quality is often limited. (2) The \editb \  is undermined when we ask the model to generate a novel scene that may not be satiable for the gender.
\setParDef
% 
\begin{comment}
\begin{figure}
\includegraphics[width=\linewidth]{figs/beyond_tss_lesion.pdf}
\caption[]{End-to-End runtime lesion study of the entire MNIST dataset and the FMA featurized music dataset. Each of DROP's contributions provides a runtime improvement.}
\label{fig:beyond_lesion}
\end{figure}
\end{comment}



\section{Conclusion}
\label{sec:conclusion}

Advanced data analytics techniques must scale to rising data volumes. 
DR techniques offer a powerful toolkit when processing these datasets, with PCA frequently outperforming popular techniques in exchange for high computational cost. 
In response, we propose DROP, a new dimensionality reduction optimizer. 
DROP combines progressive sampling, progress estimation, and online aggregation to identify high quality low dimensional bases via PCA without processing the entire dataset by balancing the runtime of downstream tasks and achieved dimensionality. 
Thus, DROP provides a first step in bridging the gap between quality and efficiency in end-to-end DR for downstream \red{analytics}. 

%We revisit canonical operators for time series dimensionality reduction and the measurement study of~\cite{keogh-study}, and show that PCA is more effective than popular alternatives in the data mining literature often by a margin of over $2\times$ on average on gold-standard time series benchmark data sets with respect to output data dimension. More surprisingly, we empirically demonstrate that a small number of samples are sufficient to accurately characterize directions of maximum variance and obtain a high-quality low-dimensional transformation.




\begin{comment}
\begin{figure}
\includegraphics[width=\linewidth]{figs/beyond_tss_lesion.pdf}
\caption[]{End-to-End runtime lesion study of the entire MNIST dataset and the FMA featurized music dataset. Each of DROP's contributions provides a runtime improvement.}
\label{fig:beyond_lesion}
\end{figure}
\end{comment}



\section{Conclusion}
\label{sec:conclusion}

Advanced data analytics techniques must scale to rising data volumes. 
DR techniques offer a powerful toolkit when processing these datasets, with PCA frequently outperforming popular techniques in exchange for high computational cost. 
In response, we propose DROP, a new dimensionality reduction optimizer. 
DROP combines progressive sampling, progress estimation, and online aggregation to identify high quality low dimensional bases via PCA without processing the entire dataset by balancing the runtime of downstream tasks and achieved dimensionality. 
Thus, DROP provides a first step in bridging the gap between quality and efficiency in end-to-end DR for downstream \red{analytics}. 

%We revisit canonical operators for time series dimensionality reduction and the measurement study of~\cite{keogh-study}, and show that PCA is more effective than popular alternatives in the data mining literature often by a margin of over $2\times$ on average on gold-standard time series benchmark data sets with respect to output data dimension. More surprisingly, we empirically demonstrate that a small number of samples are sufficient to accurately characterize directions of maximum variance and obtain a high-quality low-dimensional transformation.




% \clearpage
\bibliographystyle{plain}
\bibliography{ref}
\clearpage


\section*{Supplementary}
% \renewcommand\thesection{\Alph{section}}

In this supplementary file, in Section.\ref{Self-Aug}, we will provide the details of constructing the self-augmented dataset. 
% In Section.\ref{multi_token}, we will show a qualitative ablation study on multiple word embeddings.
In Section.\ref{InstructPix2Pix}, we will compare our method with the recently proposed general editing method InstructPix2Pix \cite{brooks2022instructpix2pix}. In Section.\ref{scene}, we will compare our method with InstructPix2Pix and the baseline that doesn't use identity information on the scene switch application.

\begin{figure}[htb]
  \centering
  \includegraphics[width=\linewidth]{figs/exps/self-aug_fig.png}
  \caption{Self-augmented dataset }
  \label{fig:self_aug}
\end{figure}

\begin{figure}[htb]
  \centering
  \includegraphics[width=\linewidth]{figs/exps/instruct-p2p_fig.pdf}
  \caption{Qualitative comparisons with InstructPix2Pix\cite{brooks2022instructpix2pix}.}
  \label{fig:instruct-p2p}
\end{figure}
\begin{figure}[htb]
  \centering
  \includegraphics[width=\linewidth]{figs/app/anything_ablation_fig.pdf}
  \caption{Qualitative comparisons with InstructPix2Pix and w/o ID (without identity information, achieved by replacing $S^{*}$ with "a person"). }
  \label{fig:anything_ablation}
\end{figure}
\appendix
\renewcommand\thesection{\Alph{section}}

\section{Self-Augmented Dataset}\label{Self-Aug}
\myparagraph {Editing Prompts. } The editing prompt list:
\begin{itemize}
\item  Oil painting style, S* face
\item  Watercolor style, S* face
\item  Pencil art style, S* face
\item  Fauvism painting, S* face
\item  S* as a wizard, looking at the camera
\item  S* as a wizard, looking at the camera
\item  S* wearing a hat, looking at the camera
\item  S* as a chef, looking at the camera
\item  S* as a nurse, looking at the camera
\end{itemize}

\myparagraph {Celebrity List. } The celebrity list is in the additional supplementary file, celebrity\_list.txt 

\myparagraph {Training examples. } 
We show the representative training samples in Figure.\ref{fig:self_aug}. 

% \section{Qualitative Results on Multiple Word Embeddings}\label{multi_token}
% The results is shown in Figure.\ref{fig:multi_token}. It can be observed
% that as the number of embeddings increases to three, the generated images fail to depict the "police"
% concept, and using a single embedding results in lower face-similarity. To achieve better trade-off, we choose the number of embeddings as 2.


\section{Qualitative comparisons with InstructPix2Pix\cite{brooks2022instructpix2pix}} \label{InstructPix2Pix}
The results is demonstrated in Figure.\ref{fig:instruct-p2p}. In general, InstructPix2Pix faces challenges when the editing
prompt requries modification of the original image's layout.


\section{Scene Switch}\label{scene}


As depicted in Figure.\ref{fig:anything_ablation}, InstructPix2Pix\cite{brooks2022instructpix2pix} struggles to keep the original identity information in some editing prompts (\eg, "At the Great Wall"). When we only use gaze information, the output images fail to reflect the reference image identity. After adopting our ID encoder to provide ID information, the generated outputs show better identity similarity.

\end{document}
