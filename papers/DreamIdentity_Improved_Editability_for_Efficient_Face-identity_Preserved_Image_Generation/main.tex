\documentclass{article}


% if you need to pass options to natbib, use, e.g.:
%     \PassOptionsToPackage{numbers, compress}{natbib}
% before loading neurips_2023
\usepackage{natbib}
\setcitestyle{numbers,square}

% ready for submission
% \usepackage{neurips_2023}


% to compile a preprint version, e.g., for submission to arXiv, add add the
% [preprint] option:
\usepackage[preprint]{neurips_2023}


% to compile a camera-ready version, add the [final] option, e.g.:
%     \usepackage[final]{neurips_2023}


% to avoid loading the natbib package, add option nonatbib:
%    \usepackage[nonatbib]{neurips_2023}

\pdfoutput=1
\usepackage[utf8]{inputenc} % allow utf-8 input
\usepackage[T1]{fontenc}    % use 8-bit T1 fonts
\usepackage[hidelinks]{hyperref}       % hyperlinks
\usepackage{url}            % simple URL typesetting
\usepackage{booktabs}       % professional-quality tables
\usepackage{amsfonts}       % blackboard math symbols
\usepackage{nicefrac}       % compact symbols for 1/2, etc.
\usepackage{microtype}      % microtypography
\usepackage{xcolor}         % colors
\usepackage{floatrow}
\floatsetup[table]{capposition=top}
\floatsetup[figure]{capposition=bottom}
% ====================================================================
% Packages
% ====================================================================
\usepackage[utf8]{inputenc} % allow utf-8 input
\usepackage[T1]{fontenc}    % use 8-bit T1 fonts
\usepackage{hyperref}       % hyperlinks
\usepackage{url}    % simple URL typesetting
\usepackage{booktabs}       % professional-quality tables
\usepackage{amsfonts}       % blackboard math symbols
\usepackage{nicefrac}       % compact symbols for 1/2, etc.
\usepackage{microtype}      % microtypography
\usepackage{lineno}
\usepackage{natbib}
\usepackage{pifont}
\usepackage{fleqn}
\usepackage{amsmath,amssymb}
\usepackage{graphicx}
\usepackage{multicol,graphicx,xcolor}
\usepackage{tikz}
\usepackage{verbatim}
\usepackage{bm}
\usetikzlibrary{shapes.misc, positioning}
\usetikzlibrary{matrix, calc, positioning, arrows,shapes,backgrounds, arrows.meta, quotes}
\usepackage{gensymb,stmaryrd,mathtools,textcomp,xspace,grffile}
\usetikzlibrary{shapes.geometric,fit,matrix,positioning,shapes.multipart}
\usetikzlibrary{decorations.markings}
\usetikzlibrary{shadows,decorations.pathreplacing,fadings}
\pgfdeclarelayer{background}
\pgfsetlayers{background,main}
\usetikzlibrary{topaths, calc,3d}
\usetikzlibrary{fit}
\usepackage{color}

\usepackage{bm}		% Bold maths symbols, including upright Greek
\usepackage{pdflscape}	% Landscape pages
\usepackage{ae,aecompl}
%\usepackage{newtxtext}
%\usepackage{newtxmath}
\usepackage[section]{placeins}
\usepackage{float}
\usepackage[normalem]{ulem}
\usepackage[toc,page]{appendix}
\usepackage{longtable}

\usepackage[normalem]{ulem}

\usepackage{tabularx}
    \newcolumntype{L}{>{\raggedright\arraybackslash}X}

\usepackage{caption}
\usepackage{subcaption}
%TODO
\newcommand{\todo}[1]{{\color{red}{\bf [TODO]:~{#1}}}}

%THEOREMS
\newtheorem{theorem}{Theorem}
\newtheorem{corollary}{Corollary}
\newtheorem{lemma}{Lemma}
\newtheorem{proposition}{Proposition}
\newtheorem{problem}{Problem}
\newtheorem{definition}{Definition}
\newtheorem{remark}{Remark}
\newtheorem{example}{Example}
\newtheorem{assumption}{Assumption}

%HANS' CONVENIENCES
\newcommand{\define}[1]{\textit{#1}}
\newcommand{\join}{\vee}
\newcommand{\meet}{\wedge}
\newcommand{\bigjoin}{\bigvee}
\newcommand{\bigmeet}{\bigwedge}
\newcommand{\jointimes}{\boxplus}
\newcommand{\meettimes}{\boxplus'}
\newcommand{\bigjoinplus}{\bigjoin}
\newcommand{\bigmeetplus}{\bigmeet}
\newcommand{\joinplus}{\join}
\newcommand{\meetplus}{\meet}
\newcommand{\lattice}[1]{\mathbf{#1}}
\newcommand{\semimod}{\mathcal{S}}
\newcommand{\graph}{\mathcal{G}}
\newcommand{\nodes}{\mathcal{V}}
\newcommand{\agents}{\{1,2,\dots,N\}}
\newcommand{\edges}{\mathcal{E}}
\newcommand{\neighbors}{\mathcal{N}}
\newcommand{\Weights}{\mathcal{A}}
\renewcommand{\leq}{\leqslant}
\renewcommand{\geq}{\geqslant}
\renewcommand{\preceq}{\preccurlyeq}
\renewcommand{\succeq}{\succcurlyeq}
\newcommand{\Rmax}{\mathbb{R}_{\mathrm{max}}}
\newcommand{\Rmin}{\mathbb{R}_{\mathrm{min}}}
\newcommand{\Rext}{\overline{\mathbb{R}}}
\newcommand{\R}{\mathbb{R}}
\newcommand{\N}{\mathbb{N}}
\newcommand{\A}{\mathbf{A}}
\newcommand{\B}{\mathbf{B}}
\newcommand{\x}{\mathbf{x}}
\newcommand{\e}{\mathbf{e}}
\newcommand{\X}{\mathbf{X}}
\newcommand{\W}{\mathbf{W}}
\newcommand{\weights}{\mathcal{W}}
\newcommand{\alternatives}{\mathcal{X}}
\newcommand{\xsol}{\bar{\mathbf{x}}}
\newcommand{\y}{\mathbf{y}}
\newcommand{\Y}{\mathbf{Y}}
\newcommand{\z}{\mathbf{z}}
\newcommand{\Z}{\mathbf{Z}}
\renewcommand{\a}{\mathbf{a}}
\renewcommand{\b}{\mathbf{b}}
\newcommand{\I}{\mathbf{I}}
\DeclareMathOperator{\supp}{supp}
\newcommand{\Par}[2]{\mathcal{P}_{{#1} \to {#2}}}
\newcommand{\Laplacian}{\mathcal{L}}
\newcommand{\F}{\mathcal{F}}
\newcommand{\inv}[1]{{#1}^{\sharp}}
\newcommand{\energy}{Q}
\newcommand{\err}{\mathrm{err}}
\newcommand{\argmin}{\mathrm{argmin}}
\newcommand{\argmax}{\mathrm{argmax}}
\usepackage{color}


\title{\ours: 
Improved Editability for Efficient Face-identity Preserved Image Generation}

\author{
    Zhuowei Chen\textsuperscript{\rm 1,2}\quad 
    Shancheng Fang\textsuperscript{\rm 2}\quad  
    Wei Liu\textsuperscript{\rm 2} \quad Qian He\textsuperscript{\rm 2} \quad
    Mengqi Huang\textsuperscript{\rm 1}\\ 
    \textbf{Yongdong Zhang\textsuperscript{\rm 1}\quad
    Zhendong Mao\textsuperscript{\rm 1}\thanks{Corresponding authors} }\\
    \textsuperscript{\rm 1}~University of Science and Technology of China \quad 
    \textsuperscript{\rm 2}~ByteDance Inc.\quad  \\
    \{chenzw01, huangmq\}@mail.ustc.edu.cn \quad \{zdmao, zyd73\}@ustc.edu.cn \\
    \{fangshancheng.lh, liuwei.jikun, heqian\}@bytedance.com\\
\textcolor{red}{\url{https://dreamidentity.github.io/}}
}

\begin{document}


\maketitle
\begin{abstract}
\noindent\textbf{Abstract:} In this paper, we present \textit{JADE}, a targeted linguistic fuzzing platform which strengthens the linguistic complexity of seed questions to simultaneously and consistently break a wide range of widely-used LLMs categorized in three groups: eight open-sourced Chinese, six commercial Chinese and four commercial English LLMs. JADE generates three safety benchmarks for the three groups of LLMs, which contain unsafe questions that are highly threatening: the questions simultaneously trigger harmful generation of multiple LLMs, with an average unsafe generation ratio of \textbf{$70\%$} (please see the table below), while are still natural questions, fluent and preserving the core unsafe semantics. We release the benchmark demos generated for commercial English LLMs and open-sourced Chinese LLMs in the following link: \url{https://github.com/whitzard-ai/jade-db}. For readers who are interested in evaluating on more questions generated by JADE, please contact us.


% This results in a safety benchmark of natural questions which simultaneously trigger harmful generation of a wide range of widely-used LLMs below, in over $70\%$ test cases.



% Table generated by Excel2LaTeX from sheet 'Sheet2'
\begin{center}
\scalebox{0.65}{
    \begin{tabular}{lccccccc}
    \toprule
    \multirow{2}[3]{*}{\textbf{Group}} & \multicolumn{4}{c}{\multirow{2}[3]{*}{\textbf{Model Name}}} & \multicolumn{3}{c}{\textbf{Unsafe Generation Ratio}} \\
\cmidrule{6-8}          & \multicolumn{4}{c}{}          & \textbf{Average} & \textbf{Least} & \textbf{Most} \\
    \midrule
    \multirow{2}[2]{*}{\textbf{Open-sourced LLM (Chinese)}} & ChatGLM & ChatGLM2 & InternLM & Ziya  & \multirow{2}[2]{*}{74.13\%} & \multirow{2}[2]{*}{49.00\%} & \multirow{2}[2]{*}{93.50\%} \\
          & Baichuan & BELLE & MOSS  & ChatYuan2 &       &       &  \\
    \midrule
    \textbf{Commercial LLM (English)} & ChatGPT & Claude & PaLM2 & LLaMA2 & 74.38\% & 35.00\% & 91.25\% \\
    \midrule
    \multirow{2}[2]{*}{\textbf{Commercial LLM (Chinese)}} & Doubao & Wenxin Yiyan & ChatGLM & SenseChat & \multirow{2}[2]{*}{77.5\%} & \multirow{2}[2]{*}{56.00\%} & \multirow{2}[2]{*}{90.00\%} \\
          & Baichuan & ABAB  & \multicolumn{2}{c}{\footnotesize{(For the detailed info., please refer to Table 2)}} &       &       &  \\
    \bottomrule
    \end{tabular}}%
\end{center}





\textit{JADE} is based on Noam Chomsky's seminal theory of transformational-generative grammar. Given a seed question with unsafe intention, \textit{JADE} invokes a sequence of generative and transformational rules to increment the complexity of the syntactic structure of the original question, until the safety guardrail is broken. Our key insight is: Due to the complexity of human language, most of the current best LLMs can hardly recognize the invariant evil from the infinite number of different syntactic structures which form an unbound example space that can never be fully covered. Technically, the generative/transformative rules are constructed by native speakers of the languages, and, once developed, can be used to automatically grow and transform the parse tree of a given question, until the guardrail is broken. Besides, \textit{JADE} also incorporates an active learning algorithm to incrementally improve the LLM-based evaluation module, which
iteratively optimizes the prompts for evaluation with a small amount of annotated data, to effectively strengthen the alignment with the judgement made by human experts. For more evaluation results and demo, please check our website: \url{https://whitzard-ai.github.io/jade.html}.

\noindent\pxd{{\footnotesize[\textbf{Content Warning: This paper contains examples of harmful language.}]}}
\end{abstract}





% Featured by OpenAI's ChatGPT, the rise of aligned large language models (LLM) is recognized as a milestone in the history of AI, and catalyzes wild imagination on the arrival of \textit{Artificial General Intelligence} (AGI). To achieve harmless generation, many approaches are proposed to align the AI generation contents with human values, or called \textit{AI alignment}. This equips pretrained large language models with the ability of generating safe responses under unsafe requests. However, we find human language is more complex than the current best LLM can handle. To validate this point, 
% !TEX root = ../arxiv.tex

Unsupervised domain adaptation (UDA) is a variant of semi-supervised learning \cite{blum1998combining}, where the available unlabelled data comes from a different distribution than the annotated dataset \cite{Ben-DavidBCP06}.
A case in point is to exploit synthetic data, where annotation is more accessible compared to the costly labelling of real-world images \cite{RichterVRK16,RosSMVL16}.
Along with some success in addressing UDA for semantic segmentation \cite{TsaiHSS0C18,VuJBCP19,0001S20,ZouYKW18}, the developed methods are growing increasingly sophisticated and often combine style transfer networks, adversarial training or network ensembles \cite{KimB20a,LiYV19,TsaiSSC19,Yang_2020_ECCV}.
This increase in model complexity impedes reproducibility, potentially slowing further progress.

In this work, we propose a UDA framework reaching state-of-the-art segmentation accuracy (measured by the Intersection-over-Union, IoU) without incurring substantial training efforts.
Toward this goal, we adopt a simple semi-supervised approach, \emph{self-training} \cite{ChenWB11,lee2013pseudo,ZouYKW18}, used in recent works only in conjunction with adversarial training or network ensembles \cite{ChoiKK19,KimB20a,Mei_2020_ECCV,Wang_2020_ECCV,0001S20,Zheng_2020_IJCV,ZhengY20}.
By contrast, we use self-training \emph{standalone}.
Compared to previous self-training methods \cite{ChenLCCCZAS20,Li_2020_ECCV,subhani2020learning,ZouYKW18,ZouYLKW19}, our approach also sidesteps the inconvenience of multiple training rounds, as they often require expert intervention between consecutive rounds.
We train our model using co-evolving pseudo labels end-to-end without such need.

\begin{figure}[t]%
    \centering
    \def\svgwidth{\linewidth}
    \input{figures/preview/bars.pdf_tex}
    \caption{\textbf{Results preview.} Unlike much recent work that combines multiple training paradigms, such as adversarial training and style transfer, our approach retains the modest single-round training complexity of self-training, yet improves the state of the art for adapting semantic segmentation by a significant margin.}
    \label{fig:preview}
\end{figure}

Our method leverages the ubiquitous \emph{data augmentation} techniques from fully supervised learning \cite{deeplabv3plus2018,ZhaoSQWJ17}: photometric jitter, flipping and multi-scale cropping.
We enforce \emph{consistency} of the semantic maps produced by the model across these image perturbations.
The following assumption formalises the key premise:

\myparagraph{Assumption 1.}
Let $f: \mathcal{I} \rightarrow \mathcal{M}$ represent a pixelwise mapping from images $\mathcal{I}$ to semantic output $\mathcal{M}$.
Denote $\rho_{\bm{\epsilon}}: \mathcal{I} \rightarrow \mathcal{I}$ a photometric image transform and, similarly, $\tau_{\bm{\epsilon}'}: \mathcal{I} \rightarrow \mathcal{I}$ a spatial similarity transformation, where $\bm{\epsilon},\bm{\epsilon}'\sim p(\cdot)$ are control variables following some pre-defined density (\eg, $p \equiv \mathcal{N}(0, 1)$).
Then, for any image $I \in \mathcal{I}$, $f$ is \emph{invariant} under $\rho_{\bm{\epsilon}}$ and \emph{equivariant} under $\tau_{\bm{\epsilon}'}$, \ie~$f(\rho_{\bm{\epsilon}}(I)) = f(I)$ and $f(\tau_{\bm{\epsilon}'}(I)) = \tau_{\bm{\epsilon}'}(f(I))$.

\smallskip
\noindent Next, we introduce a training framework using a \emph{momentum network} -- a slowly advancing copy of the original model.
The momentum network provides stable, yet recent targets for model updates, as opposed to the fixed supervision in model distillation \cite{Chen0G18,Zheng_2020_IJCV,ZhengY20}.
We also re-visit the problem of long-tail recognition in the context of generating pseudo labels for self-supervision.
In particular, we maintain an \emph{exponentially moving class prior} used to discount the confidence thresholds for those classes with few samples and increase their relative contribution to the training loss.
Our framework is simple to train, adds moderate computational overhead compared to a fully supervised setup, yet sets a new state of the art on established benchmarks (\cf \cref{fig:preview}).

\vspace{-0.3cm}
\section{Related Work}\label{sec:related}
 
The authors in \cite{humphreys2007noncontact} showed that it is possible to extract the PPG signal from the video using a complementary metal-oxide semiconductor camera by illuminating a region of tissue using through external light-emitting diodes at dual-wavelength (760nm and 880nm).  Further, the authors of  \cite{verkruysse2008remote} demonstrated that the PPG signal can be estimated by just using ambient light as a source of illumination along with a simple digital camera.  Further in \cite{poh2011advancements}, the PPG waveform was estimated from the videos recorded using a low-cost webcam. The red, green, and blue channels of the images were decomposed into independent sources using independent component analysis. One of the independent sources was selected to estimate PPG and further calculate HR, and HRV. All these works showed the possibility of extracting PPG signals from the videos and proved the similarity of this signal with the one obtained using a contact device. Further, the authors in \cite{10.1109/CVPR.2013.440} showed that heart rate can be extracted from features from the head as well by capturing the subtle head movements that happen due to blood flow.

%
The authors of \cite{kumar2015distanceppg} proposed a methodology that overcomes a challenge in extracting PPG for people with darker skin tones. The challenge due to slight movement and low lighting conditions during recording a video was also addressed. They implemented the method where PPG signal is extracted from different regions of the face and signal from each region is combined using their weighted average making weights different for different people depending on their skin color. 
%

There are other attempts where authors of \cite{6523142,6909939, 7410772, 7412627} have introduced different methodologies to make algorithms for estimating pulse rate robust to illumination variation and motion of the subjects. The paper \cite{6523142} introduces a chrominance-based method to reduce the effect of motion in estimating pulse rate. The authors of \cite{6909939} used a technique in which face tracking and normalized least square adaptive filtering is used to counter the effects of variations due to illumination and subject movement. 
The paper \cite{7410772} resolves the issue of subject movement by choosing the rectangular ROI's on the face relative to the facial landmarks and facial landmarks are tracked in the video using pose-free facial landmark fitting tracker discussed in \cite{yu2016face} followed by the removal of noise due to illumination to extract noise-free PPG signal for estimating pulse rate. 

Recently, the use of machine learning in the prediction of health parameters have gained attention. The paper \cite{osman2015supervised} used a supervised learning methodology to predict the pulse rate from the videos taken from any off-the-shelf camera. Their model showed the possibility of using machine learning methods to estimate the pulse rate. However, our method outperforms their results when the root mean squared error of the predicted pulse rate is compared. The authors in \cite{hsu2017deep} proposed a deep learning methodology to predict the pulse rate from the facial videos. The researchers trained a convolutional neural network (CNN) on the images generated using Short-Time Fourier Transform (STFT) applied on the R, G, \& B channels from the facial region of interests.
The authors of \cite{osman2015supervised, hsu2017deep} only predicted pulse rate, and we extended our work in predicting variance in the pulse rate measurements as well.

All the related work discussed above utilizes filtering and digital signal processing to extract PPG signals from the video which is further used to estimate the PR and PRV.  %
The method proposed in \cite{kumar2015distanceppg} is person dependent since the weights will be different for people with different skin tone. In contrast, we propose a deep learning model to predict the PR which is independent of the person who is being trained. Thus, the model would work even if there is no prior training model built for that individual and hence, making our model robust. 

%
\section{SYSTEM OVERVIEW}
\begin{figure}
\centering

\def\picScale{0.08}    % define variable for scaling all pictures evenly
\def\colWidth{0.5\linewidth}

\begin{tikzpicture}
\matrix [row sep=0.25cm, column sep=0cm, style={align=center}] (my matrix) at (0,0) %(2,1)
{
\node[style={anchor=center}] (FREEhand) {\includegraphics[width=0.85\linewidth]{figures/FREEhand.pdf}}; %\fill[blue] (0,0) circle (2pt);
\\
\node[style={anchor=center}] (rigid_v_soft) {\includegraphics[width=0.75\linewidth]{figures/FREE_vs_rigid-v8.pdf}}; %\fill[blue] (0,0) circle (2pt);
\\
};
\node[above] (FREEhand) at ($ (FREEhand.south west)  !0.05! (FREEhand.south east) + (0, 0.1)$) {(a)};
\node[below] (a) at ($ (rigid_v_soft.south west) !0.20! (rigid_v_soft.south east) $) {(b)};
\node[below] (b) at ($ (rigid_v_soft.south west) !0.75! (rigid_v_soft.south east) $) {(c)};
\end{tikzpicture}


% \begin{tikzpicture} %[every node/.style={draw=black}]
% % \draw[help lines] (0,0) grid (4,2);
% \matrix [row sep=0cm, column sep=0cm, style={align=center}] (my matrix) at (0,0) %(2,1)
% {
% \node[style={anchor=center}] {\includegraphics[width=\colWidth]{figures/photos/labFREEs3.jpg}}; %\fill[blue] (0,0) circle (2pt)
% &
% \node[style={anchor=center}] {\includegraphics[width=\colWidth, height=160pt]{figures/stewartRender.png}}; %\fill[blue] (0,0) circle (2pt);
% \\
% };

% %\node[style={anchor=center}] at (0,-5) (FREEstate) {\includegraphics[width=0.7\linewidth]{figures/FREEstate_noLabels2.pdf}};

% \end{tikzpicture}

\caption{\revcomment{2.3}{(a) A fiber-reinforced elastomerc enclosure (FREE) is a soft fluid-driven actuator composed of an elastomer tube with fibers wound around it to impose specific deformations under an increase in volume, such as extension and torsion. (b) A linear actuator and motor combined in \emph{series} has the ability to generate 2 dimensional forces at the end effector (shown in red), but is constrained to motions only in the directions of these forces. (b) Three FREEs combined in \emph{parallel} can generate the same 2 dimensional forces at the end effector (shown in red), without imposing kinematic constraints that prohibit motion in other directions (shown in blue).}}

% \caption{A fiber-reinforced elastomeric enclosure (FREE) (top) is a soft fluid-driven actuator composed of an elastomer tube with fibers wound around it to impose deformation in specific directions upon pressurization, such as extension and torsion. \revcomment{2.3}{In this paper we explore the potential of combining multiple FREEs in parallel to generate fully controllable multi-dimensional spacial forces}, such as in a parallel arrangement around a flexible spine element (bottom-left), or a Stewart Platform arrangement (bottom-right).}

\label{fig:overview}
\end{figure}


We now give an overview of our learning framework as illustrated in Figure~\ref{fig:overview}. Our framework splits athletic jumps into two phases: a run-up phase and a jump phase. The {\em take-off state} marks the transition between these two phases, and consists of a time instant midway through the last support phase before becoming airborne. The take-off state is key to our exploration strategy, as it is a strong determinant of the resulting jump strategy. We characterize the take-off state by a feature vector that captures key aspects of the state, such as the net angular velocity and body orientation. This defines a low-dimensional take-off feature space that we can sample in order to explore and evaluate a variety of motion strategies. While random sampling of take-off state features is straightforward, it is computationally impractical as evaluating one sample involves an expensive DRL learning process that takes hours even on modern machines. Therefore, we introduce a sample-efficient Bayesian Diversity Search (BDS) algorithm as a key part of our Stage~1 optimization process.

Given a specific sampled take-off state, we then need to produce an optimized run-up controller and a jump controller that result in the best possible corresponding jumps. This process has several steps. We first train a {\em }run-up controller, using deep reinforcement learning, that imitates a single generic run-up motion capture clip while also targeting the desired take-off state. For simplicity, the run-up controller and its training are not shown in Figure~\ref{fig:overview}. These are discussed in Section~\ref{sec:Experiments-Runup}. The main challenge lies with the synthesis of the actual jump controller which governs the remainder of the motion, and for which we wish to discover strategies without any recourse to known solutions.

The jump controller begins from the take-off state and needs to control the body during take-off, over the bar, and to prepare for landing. This poses a challenging learning problem because of the demanding nature of the task, the sparse fail/success rewards, and the difficulty of also achieving natural human-like movement. We apply two key insights to make this task learnable using deep reinforcement learning. First, we employ an action space defined by a subspace of natural human poses as modeled with a Pose Variational Autoencoder (P-VAE). Given an action parameterized as a target body pose, individual joint torques are then realized using PD-controllers. We additionally allow for regularized {\em offset} PD-targets that are added to the P-VAE targets to enable strong takeoff forces. Second, we employ a curriculum that progressively increases the task difficulty, i.e., the height of the bar, based on current performance.

A diverse set of strategies can already emerge after the Stage 1 BDS optimization. To achieve further strategy variations, we reuse the take-off states of the existing discovered strategies for another stage of optimization. The diversity is explicitly incentivized during this Stage 2 optimization via a novelty reward, which is focused specifically on features of the body pose at the peak height of the jump. As shown in Figure~\ref{fig:overview}, Stage~2 makes use of the same overall DRL learning procedure as in Stage~1, albeit with a slightly different reward structure.



\begin{table*}[t]
\centering
  \caption{Quantitative comparisons with the optimization-based and efficient methods. Encoding time means the time cost to obtain the unique/pseudo embedding. Our method achieves optimal results in terms of text-alignment, face similarity, and encoding time.}
  \label{tab:main_result}
  \begin{tabular}{cccc}
    \toprule
    Methods & Text-alignment $\uparrow$ & Face similarity $\uparrow$ & Encoding Time $\downarrow$ \\
    \midrule
    Textual Inversion \cite{gal2022image} & 0.213 & 0.326 & 20 min \\ 
    Dreambooth \cite{ruiz2022dreambooth} & 0.217 & 0.425 & 4 min  \\ 
    E4T \cite{gal2023designing} & 0.220 & 0.420 & 20 s \\ 
    Elite \cite{wei2023elite} & 0.196 & 0.450 & 0.05 s\\ 
    \midrule
    Ours & \textbf{0.228} & \textbf{0.467} & \textbf{0.04 s}\\
    \bottomrule
  \end{tabular}
\end{table*}


\section{Experiments}\label{sec:exp}


\subsection{Experiment Settings}
\myparagraph{Dataset.} Our experiments are conducted on the widely used FFHQ dataset \cite{karras2019style}, which contains $70000$ high-resolution human face images. We resize the images to 512x512 for training. The test set consists of 100 faces from \cite{liu2015faceattributes}. We make certain that there is no intersection between the test set and the self-augmented celebrity set to maintain the integrity of the experiment.

\myparagraph{Metrics.} We evaluate our method on Text-alignment and Face-similarity.  Text-alignment is used to indicate whether the generated image reflects editing prompts, which is calculated by the cosine distance in the CLIP text-image embedding space.  Face-similarity is used to measure whether the face ID is preserved. We use the ID feature from arcface \cite{deng2018arcface}, a model pre-trained on face recognition tasks, to represent the face identity. Then ID-similarity is measured by the cosine distance of ID features between the input face and the face cropped from the edited image. For each editing prompt and face identity, four images are generated.


\myparagraph{Implementation Details.} 
We choose Stable Diffusion 2.1-base as our base text-to-image model. The learning rate and batch size are set to $5e-5$ and $64$. The encoder is trained for 60,000 iterations. The embedding regularization weight $\lambda$ is set to $1e-4$. Our experiments are trained on a server with eight A100-80G GPUs, which takes about 1 day to complete each experiment. During inference, we use the DDIM \cite{song2020denoising} sampler with 30 steps. The guidance scale is set to 7.5.

\begin{table*}[t]
\centering
  \caption{Quantitative comparisons with the optimization-based and efficient methods. Encoding time means the time cost to obtain the unique/pseudo embedding. Our method achieves optimal results in terms of text-alignment, face similarity, and encoding time.}
  \label{tab:main_result}
  \begin{tabular}{cccc}
    \toprule
    Methods & Text-alignment $\uparrow$ & Face similarity $\uparrow$ & Encoding Time $\downarrow$ \\
    \midrule
    Textual Inversion \cite{gal2022image} & 0.213 & 0.326 & 20 min \\ 
    Dreambooth \cite{ruiz2022dreambooth} & 0.217 & 0.425 & 4 min  \\ 
    E4T \cite{gal2023designing} & 0.220 & 0.420 & 20 s \\ 
    Elite \cite{wei2023elite} & 0.196 & 0.450 & 0.05 s\\ 
    \midrule
    Ours & \textbf{0.228} & \textbf{0.467} & \textbf{0.04 s}\\
    \bottomrule
  \end{tabular}
\end{table*}
\begin{table}[t]
\centering
  \caption{Ablation study on $M^2$ ID Encoder. ID encoder with multi-scale feature (MS Feat) and multiple word embeddings (Multi Embedding) achieves best Face-similarity while maintaining a comparable result  Text-alignment metric.}
  \label{tab:id_feat_ablation}
  \begin{tabular}{ccccc}
    \toprule
    ID Encoder & MS Feat & Multi Embedding & Text-alignment $\uparrow$ & Face-similarity $\uparrow$  \\
    \midrule
               &          &             &    \textbf{0.229}   &   0.266   \\
     \checkmark &          &             &  0.228 & 0.302 \\
     \checkmark &  \checkmark &       & \textbf{0.229} & 0.412 \\
      \checkmark &  \checkmark &  \checkmark     & 0.228 & \textbf{0.467} \\
    \bottomrule
  \end{tabular}
\end{table}

\begin{figure}[t]
  \centering
  \includegraphics[width=\linewidth]{figs/exps/id_encoder_fig.pdf}
  \caption{Qualitative comparisons between ID Encoder and the multi-scale features. The editing prompt is "S* as a chef, looking at the camera". We could conclude that both ID Encoder and the multi-scale features greatly improve the ID preservation (\ie, face-similarity).}
  \label{fig:id_encoder}
  \vspace{-2mm}
\end{figure}
\subsection{Comparison to SOTA Methods}
In this section, we compare our method with fine-tuning based methods: Textual Inversion \cite{gal2022image}, DreamBooth \cite{ruiz2022dreambooth} and concurrent works on efficient personalized model: E4T \cite{gal2023designing} which requires finetuning for around 15 iterations for each face, and ELITE \cite{wei2023elite}, a fine-tuning free work. We adopt the widely-used open-sourced Diffusers codebase for Textual Inversion, DreamBooth, and re-implemented  E4T and  ELITE. To ensure a fair comparison, all experiments are conducted with a single face image input. 

\myparagraph{Quantitative and Qualitative Results.} As demonstrated in Tab.\ref{tab:main_result}, our work \ours \ outperforms recent methods across all the metrics, demonstrating superior performance in terms of \editb, \Imetric, and encoding speed. 
We show that \ours \ improves the text-alignment by $7\%$ compared to the second-best E4T \cite{gal2023designing}. Meanwhile,  \ours \ surpasses the second-best model \cite{wei2023elite} on \Imetric \  by $3.7\%$, while enjoying better editability. Benefiting from the direct encoding rather than optimization for unique embeddings, the additional computation cost is only 0.04 s, which can be negligible compared to the time cost (seconds-level) for a standard diffusion-based text-to-image process. The conclusion is further validated by the qualitative results in Fig.\ref{fig:main_result}.

\begin{figure}[t]
\CenterFloatBoxes
\begin{floatrow}
\ttabbox
{\begin{tabular}{cccc}
    \toprule
     {\scriptsize Recon} & {\scriptsize self-aug} & {\scriptsize Text-alignment $\uparrow$} & {\scriptsize Face similarity $\uparrow$ }  \\
    \midrule
    \checkmark &             &  0.213 & 0.380 \\
      &  \checkmark &   0.216 & 0.348 \\
      \checkmark &  \checkmark & \textbf{0.228} & \textbf{0.467} \\
    \bottomrule
  \end{tabular}
  }
  {\caption{Ablation study on self-augmented \editb \  learning. Recon denotes reconstruction training. self-aug denotes self-augmented \editb \  learning, the \editb \ gets improved after applying self-aug.}
  \label{tab:gen_data_ablation}
  }
\killfloatstyle
\ttabbox
{\begin{tabular}{ccc}
    \toprule
     {\scriptsize Emb Num} & {\scriptsize Text-alignment $\uparrow$} & {\scriptsize Face similarity $\uparrow$}  \\
    \midrule
    1           &    \textbf{0.229}   &   0.412   \\
     2           &  0.228 & \textbf{0.462} \\
     3    & 0.188 & 0.472 \\ 
    \bottomrule
  \end{tabular}
  }
  {\caption{Ablation study on the number of word embeddings (Emb Num). Single word embedding could limit the face-similarity while excessive ones may hinder text-alignment.}
  \label{tab:multi_token_ablation}
  }

\end{floatrow}
\end{figure}
\begin{figure}[t]
  \centering
  \includegraphics[width=\linewidth]{figs/exps/gen_fig.pdf}
  \caption{Qualitative comparisons on the self-augmented dataset for \editb \  learning. The editing prompt is "S* as a police, looking at the camera". "w/o edit" and "w/o recon" denote for the encoder is trained without \editb \ learning objective and without reconstruction learning, respectively. We show that the generated images can not follow the prompt properly without the \editb \ learning. Meanwhile, the face similarity will be lower without the reconstruction learning on FFHQ.}
  \label{fig:exp_self_aug}
\end{figure}
\begin{figure}[htb]
  \centering
  \includegraphics[width=\linewidth]{figs/exps/multi_token_fig.pdf}
  \caption{Qualitative comparisons of multiple word embeddings. The editing prompt is "S* as a police, looking at the camera", and "NUM" denotes the number of embeddings.}
  \label{fig:multi_token}
\end{figure}


\subsection{Ablation Studies}
\vspace{-0.2cm}
In this section, we conduct ablation studies to verify the effectiveness of our proposed  $M^2$ ID feature and self-augmented editability learning.
 
\myparagraph{$M^{2}$ ID encoder.} 
We adopt CLIP encoder as our baseline, which is commonly used by concurrent encoder-based methods. Following \cite{wei2023elite, shi2023instantbooth}, we use the last layer CLS feature from CLIP encoder to predict a word embedding. As Fig.\ref{fig:id_encoder} shows, this baseline generally failed to capture the core identity information in the input image, and in some cases, it doesn't even capture the gender information.  
Upon switching from the CLIP encoder to the face-specific ID encoder, the \Imetric \ is improved from $0.266$ to $0.302$, as shown in Tab.\ref{tab:id_feat_ablation}. Integrating the multi-scale features further boosts the \Imetric \ to $0.412$.
Multi-word embeddings are  further utilized to enhance ID-preservation. As shown in Tab.\ref{tab:multi_token_ablation} and Fig.\ref{fig:multi_token}, when we increase the number of embedding  to 2, the Face-similarity is improved by $12\%$ with marginal change of $0.4\%$ on text-alignment. However, when we further increase the number of word embedding,  text-alignment is dropped by $17\%$. We argue that excessive word embeddings may include more information beyond the ID feature such that hinder the editability. Therefore, we choose the embedding number as $2$ to avoid degraded editability.  

\myparagraph{Self-Augmented Editability Learning.} Next, we study the effectiveness of self-augmented editability learning. Fig.\ref{fig:exp_self_aug} indicates that if the model is only trained under the reconstruction objective, the editability \ of embeddings will be limited. To be specific, the model trained without the editability learning objective fails to edit the input identity to a police. Besides, if we only use the limited generated editing dataset, face similarity will be degraded in that there are only around 1000 face IDs in the self-augmented dataset. Combining the reconstruction data (i.e., FFHQ) and generated self-augmented dataset is a better choice to preserve face similarity while following the textual instruction. The quantitative results in Tab.\ref{tab:gen_data_ablation} further confirm our conclusion.




\subsection{Application}
\myparagraph{ID-preserved Scene Switch.} As illustrated in Fig.\ref{fig:anything}, given the input face ID and its location in the canvas indicated by the gaze location, we can generate a series of different scene images which share the same identity information and head location with the help of ControlNet \cite{zhang2023adding}. The scene is specified by the text description and can encompass different accessories, hair style, backgrounds, and styles. With this method, we may achieve the effect of "everything and everywhere all at once". 

\begin{figure}[H]
  \centering
  \includegraphics[width=\linewidth]{figs/app/anything_fig.pdf}
  \caption{Given a face identify and its gaze location on the canvas, our method can generate a series of images that maintain the same identity while following the editing prompts in the same location.}
  \label{fig:anything}
\end{figure}








\section{Limitation}
\setParDis
While our method offers an efficient approach to recreate a human image given one face image, there are several limitations should be noticed. (1) Our model is trained on the high-quality realistic face image dataset, so when the input is a poor-quality face or out-of-domain image, such as a partially obstructed image, the edited image quality is often limited. (2) The \editb \  is undermined when we ask the model to generate a novel scene that may not be satiable for the gender.
\setParDef
% % \vspace{-0.5em}
\section{Conclusion}
% \vspace{-0.5em}
Recent advances in multimodal single-cell technology have enabled the simultaneous profiling of the transcriptome alongside other cellular modalities, leading to an increase in the availability of multimodal single-cell data. In this paper, we present \method{}, a multimodal transformer model for single-cell surface protein abundance from gene expression measurements. We combined the data with prior biological interaction knowledge from the STRING database into a richly connected heterogeneous graph and leveraged the transformer architectures to learn an accurate mapping between gene expression and surface protein abundance. Remarkably, \method{} achieves superior and more stable performance than other baselines on both 2021 and 2022 NeurIPS single-cell datasets.

\noindent\textbf{Future Work.}
% Our work is an extension of the model we implemented in the NeurIPS 2022 competition. 
Our framework of multimodal transformers with the cross-modality heterogeneous graph goes far beyond the specific downstream task of modality prediction, and there are lots of potentials to be further explored. Our graph contains three types of nodes. While the cell embeddings are used for predictions, the remaining protein embeddings and gene embeddings may be further interpreted for other tasks. The similarities between proteins may show data-specific protein-protein relationships, while the attention matrix of the gene transformer may help to identify marker genes of each cell type. Additionally, we may achieve gene interaction prediction using the attention mechanism.
% under adequate regulations. 
% We expect \method{} to be capable of much more than just modality prediction. Note that currently, we fuse information from different transformers with message-passing GNNs. 
To extend more on transformers, a potential next step is implementing cross-attention cross-modalities. Ideally, all three types of nodes, namely genes, proteins, and cells, would be jointly modeled using a large transformer that includes specific regulations for each modality. 

% insight of protein and gene embedding (diff task)

% all in one transformer

% \noindent\textbf{Limitations and future work}
% Despite the noticeable performance improvement by utilizing transformers with the cross-modality heterogeneous graph, there are still bottlenecks in the current settings. To begin with, we noticed that the performance variations of all methods are consistently higher in the ``CITE'' dataset compared to the ``GEX2ADT'' dataset. We hypothesized that the increased variability in ``CITE'' was due to both less number of training samples (43k vs. 66k cells) and a significantly more number of testing samples used (28k vs. 1k cells). One straightforward solution to alleviate the high variation issue is to include more training samples, which is not always possible given the training data availability. Nevertheless, publicly available single-cell datasets have been accumulated over the past decades and are still being collected on an ever-increasing scale. Taking advantage of these large-scale atlases is the key to a more stable and well-performing model, as some of the intra-cell variations could be common across different datasets. For example, reference-based methods are commonly used to identify the cell identity of a single cell, or cell-type compositions of a mixture of cells. (other examples for pretrained, e.g., scbert)


%\noindent\textbf{Future work.}
% Our work is an extension of the model we implemented in the NeurIPS 2022 competition. Now our framework of multimodal transformers with the cross-modality heterogeneous graph goes far beyond the specific downstream task of modality prediction, and there are lots of potentials to be further explored. Our graph contains three types of nodes. while the cell embeddings are used for predictions, the remaining protein embeddings and gene embeddings may be further interpreted for other tasks. The similarities between proteins may show data-specific protein-protein relationships, while the attention matrix of the gene transformer may help to identify marker genes of each cell type. Additionally, we may achieve gene interaction prediction using the attention mechanism under adequate regulations. We expect \method{} to be capable of much more than just modality prediction. Note that currently, we fuse information from different transformers with message-passing GNNs. To extend more on transformers, a potential next step is implementing cross-attention cross-modalities. Ideally, all three types of nodes, namely genes, proteins, and cells, would be jointly modeled using a large transformer that includes specific regulations for each modality. The self-attention within each modality would reconstruct the prior interaction network, while the cross-attention between modalities would be supervised by the data observations. Then, The attention matrix will provide insights into all the internal interactions and cross-relationships. With the linearized transformer, this idea would be both practical and versatile.

% \begin{acks}
% This research is supported by the National Science Foundation (NSF) and Johnson \& Johnson.
% \end{acks}
% \vspace{-0.5em}
\section{Conclusion}
% \vspace{-0.5em}
Recent advances in multimodal single-cell technology have enabled the simultaneous profiling of the transcriptome alongside other cellular modalities, leading to an increase in the availability of multimodal single-cell data. In this paper, we present \method{}, a multimodal transformer model for single-cell surface protein abundance from gene expression measurements. We combined the data with prior biological interaction knowledge from the STRING database into a richly connected heterogeneous graph and leveraged the transformer architectures to learn an accurate mapping between gene expression and surface protein abundance. Remarkably, \method{} achieves superior and more stable performance than other baselines on both 2021 and 2022 NeurIPS single-cell datasets.

\noindent\textbf{Future Work.}
% Our work is an extension of the model we implemented in the NeurIPS 2022 competition. 
Our framework of multimodal transformers with the cross-modality heterogeneous graph goes far beyond the specific downstream task of modality prediction, and there are lots of potentials to be further explored. Our graph contains three types of nodes. While the cell embeddings are used for predictions, the remaining protein embeddings and gene embeddings may be further interpreted for other tasks. The similarities between proteins may show data-specific protein-protein relationships, while the attention matrix of the gene transformer may help to identify marker genes of each cell type. Additionally, we may achieve gene interaction prediction using the attention mechanism.
% under adequate regulations. 
% We expect \method{} to be capable of much more than just modality prediction. Note that currently, we fuse information from different transformers with message-passing GNNs. 
To extend more on transformers, a potential next step is implementing cross-attention cross-modalities. Ideally, all three types of nodes, namely genes, proteins, and cells, would be jointly modeled using a large transformer that includes specific regulations for each modality. 

% insight of protein and gene embedding (diff task)

% all in one transformer

% \noindent\textbf{Limitations and future work}
% Despite the noticeable performance improvement by utilizing transformers with the cross-modality heterogeneous graph, there are still bottlenecks in the current settings. To begin with, we noticed that the performance variations of all methods are consistently higher in the ``CITE'' dataset compared to the ``GEX2ADT'' dataset. We hypothesized that the increased variability in ``CITE'' was due to both less number of training samples (43k vs. 66k cells) and a significantly more number of testing samples used (28k vs. 1k cells). One straightforward solution to alleviate the high variation issue is to include more training samples, which is not always possible given the training data availability. Nevertheless, publicly available single-cell datasets have been accumulated over the past decades and are still being collected on an ever-increasing scale. Taking advantage of these large-scale atlases is the key to a more stable and well-performing model, as some of the intra-cell variations could be common across different datasets. For example, reference-based methods are commonly used to identify the cell identity of a single cell, or cell-type compositions of a mixture of cells. (other examples for pretrained, e.g., scbert)


%\noindent\textbf{Future work.}
% Our work is an extension of the model we implemented in the NeurIPS 2022 competition. Now our framework of multimodal transformers with the cross-modality heterogeneous graph goes far beyond the specific downstream task of modality prediction, and there are lots of potentials to be further explored. Our graph contains three types of nodes. while the cell embeddings are used for predictions, the remaining protein embeddings and gene embeddings may be further interpreted for other tasks. The similarities between proteins may show data-specific protein-protein relationships, while the attention matrix of the gene transformer may help to identify marker genes of each cell type. Additionally, we may achieve gene interaction prediction using the attention mechanism under adequate regulations. We expect \method{} to be capable of much more than just modality prediction. Note that currently, we fuse information from different transformers with message-passing GNNs. To extend more on transformers, a potential next step is implementing cross-attention cross-modalities. Ideally, all three types of nodes, namely genes, proteins, and cells, would be jointly modeled using a large transformer that includes specific regulations for each modality. The self-attention within each modality would reconstruct the prior interaction network, while the cross-attention between modalities would be supervised by the data observations. Then, The attention matrix will provide insights into all the internal interactions and cross-relationships. With the linearized transformer, this idea would be both practical and versatile.

% \begin{acks}
% This research is supported by the National Science Foundation (NSF) and Johnson \& Johnson.
% \end{acks}

% \clearpage
\bibliographystyle{plain}
\bibliography{ref}
\clearpage


\section*{Supplementary}
% \renewcommand\thesection{\Alph{section}}

In this supplementary file, in Section.\ref{Self-Aug}, we will provide the details of constructing the self-augmented dataset. 
% In Section.\ref{multi_token}, we will show a qualitative ablation study on multiple word embeddings.
In Section.\ref{InstructPix2Pix}, we will compare our method with the recently proposed general editing method InstructPix2Pix \cite{brooks2022instructpix2pix}. In Section.\ref{scene}, we will compare our method with InstructPix2Pix and the baseline that doesn't use identity information on the scene switch application.

\begin{figure}[htb]
  \centering
  \includegraphics[width=\linewidth]{figs/exps/self-aug_fig.png}
  \caption{Self-augmented dataset }
  \label{fig:self_aug}
\end{figure}

\begin{figure}[htb]
  \centering
  \includegraphics[width=\linewidth]{figs/exps/instruct-p2p_fig.pdf}
  \caption{Qualitative comparisons with InstructPix2Pix\cite{brooks2022instructpix2pix}.}
  \label{fig:instruct-p2p}
\end{figure}
\begin{figure}[htb]
  \centering
  \includegraphics[width=\linewidth]{figs/app/anything_ablation_fig.pdf}
  \caption{Qualitative comparisons with InstructPix2Pix and w/o ID (without identity information, achieved by replacing $S^{*}$ with "a person"). }
  \label{fig:anything_ablation}
\end{figure}
\appendix
\renewcommand\thesection{\Alph{section}}

\section{Self-Augmented Dataset}\label{Self-Aug}
\myparagraph {Editing Prompts. } The editing prompt list:
\begin{itemize}
\item  Oil painting style, S* face
\item  Watercolor style, S* face
\item  Pencil art style, S* face
\item  Fauvism painting, S* face
\item  S* as a wizard, looking at the camera
\item  S* as a wizard, looking at the camera
\item  S* wearing a hat, looking at the camera
\item  S* as a chef, looking at the camera
\item  S* as a nurse, looking at the camera
\end{itemize}

\myparagraph {Celebrity List. } The celebrity list is in the additional supplementary file, celebrity\_list.txt 

\myparagraph {Training examples. } 
We show the representative training samples in Figure.\ref{fig:self_aug}. 

% \section{Qualitative Results on Multiple Word Embeddings}\label{multi_token}
% The results is shown in Figure.\ref{fig:multi_token}. It can be observed
% that as the number of embeddings increases to three, the generated images fail to depict the "police"
% concept, and using a single embedding results in lower face-similarity. To achieve better trade-off, we choose the number of embeddings as 2.


\section{Qualitative comparisons with InstructPix2Pix\cite{brooks2022instructpix2pix}} \label{InstructPix2Pix}
The results is demonstrated in Figure.\ref{fig:instruct-p2p}. In general, InstructPix2Pix faces challenges when the editing
prompt requries modification of the original image's layout.


\section{Scene Switch}\label{scene}


As depicted in Figure.\ref{fig:anything_ablation}, InstructPix2Pix\cite{brooks2022instructpix2pix} struggles to keep the original identity information in some editing prompts (\eg, "At the Great Wall"). When we only use gaze information, the output images fail to reflect the reference image identity. After adopting our ID encoder to provide ID information, the generated outputs show better identity similarity.

\end{document}
