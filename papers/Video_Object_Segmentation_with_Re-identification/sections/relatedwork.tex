\section{Related Work}

The realm of object segmentation witnesses drastic progress these days, including the marriage of deep learning and graphical models~\cite{zheng2015conditional, liu2015semantic} and the efforts to enable real-time inference on high-res images~\cite{li2017not, zhao2017icnet}.
Since most visual sensory data are videos, it is crucial to extend object segmentation from image to video.
Existing video segmentation methods~\cite{liu2016deep, Perazzi2017} rely on temporal continuity to establish spatio-temporal correlation.
However, real-life videos exhibit severe deformation and occlusion, rendering such assumption to suffer from issues like drifting and inability to handle large object  displacement.
In this work, we propose a novel method known as Video Object Segmentation with Re-identification (VS-ReID) to overcome these issues.

% The realm of object segmentation witnesses drastic progress these days, including the marriage of deep learning and graphical models~\cite{}, the integration of multi-level information~\cite{} and the efforts to enable real-time inference~\cite{}.
% As the results of semantic image segmentation gradually become accurate, there are two notable research trends emerging, both of which constitute the building elements for DAVIS video segmentation challenge:

% \noindent
% \textbf{Video Segmentation.}
% %
% Most of visual sensory data are videos, thus it is crucial to extend object segmentation from image to video.
% However, since real-life videos observe lots of deformation and occlusion, which are not well handled by the existing methods.

% \noindent
% \textbf{Instance Segmentation.}
% %
% Instance segmentation aims at segmenting out each instance in the image, either class-aware or class-agnostic.
% It can be viewed as the combination of object detection and semantic segmentation.
