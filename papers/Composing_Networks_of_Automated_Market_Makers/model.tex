We use the following notation and terminology.
Vectors are in bold face $(\bx)$ and scalars in italics ($x$).
Variables, scalar or vector,
are usually taken from the end of the alphabet ($x,y,z$),
and constants from the beginning ($a,b,c$).
If $\ba = (a_1, \ldots, a_n)$ and
$\bb = (b_1, \ldots, b_m)$,
then $(\ba,\bb)$ is the vector $(a_1, \ldots, a_n, b_1, \ldots, b_m)$.
Vector comparisons are component-wise:
for vectors of the same dimension,
$\ba \leq \bb$ means that each $a_i \leq b_i$.
Constant vectors
$\bone = (1, \ldots, 1)$ and $\bzero = (0, \ldots, 0)$ have dimension that will be clear from context.
$\PosReals = \set{x \in \Reals | x > 0}$,
and $\NegReals = \set{x \in \Reals | x < 0}$.
We use $\bx \lneqq \bx'$ to mean that $\bx \leq \bx'$ but $\bx \neq \bx'$,
so at least one inequality is strict.

A function $f: \Reals^n \to \Reals$ is \emph{strictly convex}
if for all $t \in (0,1)$ and distinct $\bx,\bx' \in \Reals^n$,
$f(t \bx + (1-t) \bx') < tf(\bx) + (1-t)f(\bx')$.
A convex function's tangent line or plane lies below the function's curve or surface.
The function is \emph{strictly concave} if the inequality is reversed.
A set $X$ is \emph{strictly convex} if for all distinct
$x,x' \in X$ and $t \in (0,1)$,
$t x + (1-t) x'$ lies in the interior of $X$.

For $A: \PosReals^n \to \Reals$,
the set $\set{\bx \in \PosReals^n | A(\bx) = c}$ is called a \emph{level set} at $c$,
and the set $\set{\bx \in \PosReals^n | A(\bx) \geq c}$
is called the \emph{upper contour set} at $c$.

We use L1 and L2 norms: if $\bv = (v_1, \ldots, v_n)$ is a vector,
then $\|\bv\|_1 = \sum_{i=1}^n |v_i|$ and 
$\|\bv\|_2 = \sqrt{\sum_{i=1}^n v_i^2}$.

\subsection{AMM State Spaces are Manifolds}
An \emph{asset} might be a cryptocurrency, a token,
an electronic deed to property, and so on.
Assets can fluctuate in value,
and participants may want to trade assets,
perhapsto respond to past price changes,
or to anticipate future price changes.

An \emph{$n$-dimensional} AMM trades across assets $X_1, \ldots, X_n$.
Each AMM state has the form $\bx = (x_1,\ldots,x_n) \in \PosReals^n$,
where each $x_i$ is the amount of units of asset $X_i$ in the AMM's custody.
AMM states are points in a (twice) differentiable manifold,
a higher-dimensional generalization of a surface or curve.
A \emph{trade} moves the AMM from state
$\bx = (x_1,\ldots,x_n)$ to state $\bx' = (x_1',\ldots,x_n')$.
For each $i$ where $x_i' > x_i$ ,
the trader pays $x_i'-x_i$ units of $X_i$ to the AMM,
while if $x_i' < x_i$,
the AMM pays $x_i-x_i'$ units of $X_i$ to the trader.
We call $\bx-\bx'$ a \emph{profit-loss vector}.
Requiring the AMM state space to be a differentiable manifold ensures that
both prices and slippage change gradually rather than abruptly,
although we will see in \secref{definitions} that not every manifold makes sense as an AMM state space.

AMMs typically charge per-transaction fees.
For example, Uniswap v1 charges a 0.3\% fee on trades,
and the sums collected are added to the AMM's assets.
For ease of exposition, we focus initially on AMMs that do not divert
fees to their own liquidity pools
(instead, fees might be paid to a separate account).
In \secref{fees},
we show how an AMM with a Uniswap-style fee structure can be modeled as
the sequential composition of a no-fee AMM with a specialized ``linear'' AMM.

\subsection{System Model}
There are two kinds of participants in decentralized finance.
(1) \emph{Traders} transfer assets to AMMs, and receive assets back.
Traders can compose AMMs into networks in complicated ways.
(2) \emph{Liquidity providers} (or ``providers'') fund the AMMs by
lending assets, and receiving shares, fees, or other profits.
Traders and providers play a kind of alternating game:
traders modify AMM states by adding and removing assets,
and providers can respond by adding or removing assets,
reinvesting fees, or adjusting other AMM properties.

Today, AMMs are usually implemented by smart contracts on blockchains.
For our purposes,
a \emph{blockchain} is a highly-available, tamper-proof distributed ledger
that records which participants own various assets.
A \emph{smart contract} is a public program that controls how assets are
recorded and transferred on a blockchain.
Smart contracts typically support \emph{atomic transactions} that
allow traders to execute atomic sequences of trades on multiple AMMs.
The analysis given in this paper is largely independent of the
the particular technology used to implement blockchains and smart contracts.
