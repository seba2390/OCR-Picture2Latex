It is useful to be able to reduce an AMM's dimension,
perhaps by ignoring some assets,
or by creating ``baskets'' of distinct assets that can
be treated as a unit.
In this section we introduce two tools for reducing dimensionality:
projection, and asset virtualization.

\subsection{Projection}
An AMM may provide the ability to trade across a variety of asset types,
but traders may choose to restrict their attention to a subset,
ignoring the rest.
Perhaps the ignored assets are too volatile, or not volatile enough,
or there are regulatory barriers to owning them.

Mathematically,
the \emph{projection} operator acts on an AMM by fixing some
state coordinates to constant values and letting the rest vary.
We will show that projecting an AMM in this way yields another AMM
of lower dimension.
Informally, traders are free to ignore uninteresting assets.
\begin{definition}
  Let $\bx = (x_1, \ldots, x_n)$, $\by = (y_1, \ldots, y_m)$,
  and a constant $\ba = (a_1, \ldots, a_n)$.
  The projection of $A$ onto $\ba$ is given by
  $A_{\ba}(\by) = A(\ba, \by) = 0$
\end{definition}
\begin{lemma}
  Given an $(n+m)$-dimensional AMM $A(\bx,\by) = 0$ and $\ba \in \PosReals^n$,
  the projection $A_{\ba}(\by)$ is an $m$-dimensional AMM.
\end{lemma}
\begin{proof}
  It is enough to check that $A_{\ba}$ is twice-differentiable,
  strictly increasing, and $\upper(A)$ is strictly convex.
  Because $A(\bx,\by)$ is twice-differentiable,
  so is $A(\ba,\by) = A_{\ba}(\by)$.
  To show that $A_{\ba}$ is strictly increasing,
  let $\bx' \gneqq \bx$.
  \begin{align*}
    A_{\ba}(\bx)
    &= A(\ba,\bx)\\
    &< A(\ba,\bx')\\
    &= A_{\ba}(\bx').
  \end{align*}
  To show that $\upper(A_{\ba})$ is strictly convex,
  pick distinct $\bx$ and $\bx'$ in $\upper(A_{\ba})$.
  Namely $A(\ba,\bx) = A_{\ba}(\bx) \geq 0$ and $ A(\ba,\bx^{'}) = A_{\ba}(\bx') \geq 0$.
  For $t \in (0,1)$:
  \begin{align*}
    A_{\ba}(t \bx + (1-t) \bx')
    &= A(t \ba + (1-t) \ba, t \bx + (1-t) \bx')\\
    &= A(t(\ba,\bx) + (1-t)(\ba,\bx^{'})) > 0
  \end{align*}
  by the strict convexity of $\upper(A)$.
\end{proof}

\begin{lemma}
For index set $I$,
let $\bv = (v_i | i \in I)$ be a valuation for
a sequence of asset types $X = (X_i | i \in I)$.
For $J \subset I$,
$\bv' = (\frac{v_j}{1 - \sum_{k \not \in J}v_k} | j \in J)$
is a valuation for $X' = (X_j | j \in J) \subset X$.
\end{lemma}
We say that $X'$ \emph{inherits} $\bv'$ from valuation $\bv$ of $X$.

The next lemma states that stable points persist under projection.
\begin{lemma}
\lemmalabel{project-stable}
Let $A(\bx,\by) = 0$ be an $(n+m)$-dimensional AMM,
$\bv \in \int(\Delta^{n+m})$ a valuation on $(\bx,\by)$.
If $(\ba,\bb)$ is the stable point for $\bv$ in $A(\bx,\by)$
then $\bb$ is the stable point for the inherited valuation $\bv^{'}$.
\end{lemma}
\begin{proof}
  Suppose the stable point for $\bv^{'}$ is $\bb' \neq \bb$, namely $\bv^{'} \cdot \bb^{'} < \bv^{'} \cdot \bb$.
  Scaling both sides by $1 - \sum_{i = 1}^{n} v_i$ yields
  $(v_{n+1},\ldots,v_{n+m}) \cdot \bb' \leq (v_{n+1},\ldots,v_{n+m}) \cdot \bb$.
  Adding $(v_1, \ldots, v_n) \cdot \ba$ to both sides yields
  $v \cdot (\ba,\bb') \leq v \cdot (\ba,\bb)$,
  contradicting the assumption that $(\ba,\bb)$ is the stable point for $\bv$.
\end{proof}

\subsection{Virtualization}
\seclabel{virtualization}
It is sometimes convenient to create a ``virtual asset''
from a linear combination of assets.
Here we show that replacing a set of assets
traded by an AMM with a single virtual asset
is also an AMM.
This construction works for any linear combination,
although the most sensible combination
is usually the assets' current market valuation.

Here is a simple example of asset virtualization.
Consider an AMM that trades across three asset types,
$X,Y,Z$, defined by the constant-product formula 
\begin{equation*}
    A(x,y,z) = x y z - 8 = 0,
\end{equation*}
initialized in state $(2,2,2)$.
A trader believes that 2 units of $Y$ are always worth 1 unit of $Z$,
and that it makes sense to link them in that ratio
by creating a virtual asset $W$
worth 2/3 units of $Y$ and 1/3 unit of $Z$,
and to trade in a single denomination of $W$ instead of
individual denominations of $Y$ and $Z$.

Formally, the trader defines $W$ in terms of the valuation
$\bv = (\frac{2}{3},\frac{1}{3})$ on $Y,Z$.
The virtualized AMM $A|\bv$ is defined by
\begin{align*}
  (A|\bv)(x,w) 
  &= A(x,\frac{2w}{3},\frac{w}{3}+1)\\
  &= x \frac{2w}{3}(\frac{w}{3}+1) - 8 \\
  &= 0,
\end{align*}
with initial state $(2,3)$.
(The ``+1'' in the $Z$ co-ordinate appears
because 2 $Y$ and 2 $Z$ units are not evenly divisible
into $W$ units.)

Let $\bx = (x_1, \ldots, x_n)$ and $\by = (y_1, \ldots, y_m)$.
Let $A(\bx,\by) := A(x_1, \ldots, x_n,y_1,\ldots,y_m) = 0$ be an
$(n+m)$-dimensional AMM with initial state
$(\ba,\bb) = (a_1, \ldots, a_n, b_1, \ldots, b_m)$.
Let us create a virtual asset $Z$ from
$y_1, \ldots, y_m$,
using the valuation $\bv = (v_1, \ldots, v_m)$.

Let $c \in \PosReals$ be the largest value such that
$\bb - c \bv \geq \bzero$.
The value $c$ is the number of $Z$ assets in $\bb$,
and $\br =\bb - c \bv$ is the vector of residues
if $\bb$ is not evenly divisible into $Z$ units.
The virtualized AMM is given by
\begin{align*}
(A|\bv)(\bx,z) 
&= A(\bx, v_1 z + r_1, \ldots, v_m z + r_m) \\
&= 0,
\end{align*}
with initial state $(a_1, \ldots, a_n, c)$.

The next lemma says that in any AMM state,
it is always possible to virtualize any set of assets.
\begin{lemma}
  Let $A$ be an $(m+n)$-dimensional AMM in state $(\ba,\bb)$,
  where $\ba \in \PosReals^m$, $\bb \in \PosReals^n$,
  and valuation $\bv \in \int(\Delta^n)$.
  We claim that for any $\ba' \in \PosReals^m$,
  there is a unique $t \in \Reals$ such that $(\ba',\bb + t \bv)$
  is a state of $A$.
\end{lemma}

\begin{proof}
We seek $t \in \Reals$ such that $A(\ba',\bb + t \bv) = 0$.
There are several cases.
If $A(\ba',\bb) = 0$, then $t=0$ and we are done.
Suppose $A(\ba',\bb) < 0$.
If $A(\ba',\bb + \bv) = 0$, then $t=1$ and we are done.
If $A(\ba',\bb + \bv) < 0$,
pick a vector $\bc = (c_1, \ldots, c_n) \in \PosReals^n$
such that $A(\ba',\bb+\bc) = 0$.
Let $\epsilon > 0$,
\begin{equation*}
s_i = \frac{c_i + \epsilon}{v_i}, 0 \leq i \leq n,
\end{equation*}
and $s = \max_{0 \leq i \leq n}s_i$.
It follows that $s \bv \geq \bc + \epsilon \bone$,
and $\bb + s \bv \geq \bb + \bc + \epsilon \bone$.
Since $A_{\ba^{'}}$ is strictly increasing,
$A(\ba',\bb + s \bv) \geq A_{\ba'}(\bb+\bc) = 0$.
Define $\alpha(t): [0,1] \to \Reals$ by $\alpha(t) = A_{\ba'}(\bb + \bv + t(s -1)\bv)$.
Because $\alpha$ is continuous,
the intermediate value theorem guarantees a unique
$t^{*} \in (0,1)$ such that $\alpha(t^{*}) = 0$.
Taking $t = (1 + t^{*}(s-1))$ establishes the claim.
If $A(\ba',\bb + \bv) > 0$,
let
\begin{equation*}
s_i = \frac{c_i - \epsilon}{v_i}, 0 \leq i \leq n,
\end{equation*}
and $s = \min_{0 \leq i \leq n} s_i$,
and the claim follows from a symmetric argument.

Suppose $A(\ba',\bb) > 0$.
Pick a vector $\bc = (c_1, \ldots, c_n) \in \PosReals^n$
such that $A(\ba',\bb-\bc) = 0$.
Let $\epsilon > 0$,
$s_i = \frac{c_i + \epsilon}{v_i}, 0 \leq i \leq n$.
and $s = \max_{0 \leq i \leq n}s_i$.
It follows that $s \bv \geq \bc + \epsilon \bone$,
and $\bb - s \bv \leq \bb - \bc - \epsilon \bone$,
so $A(\ba',\bb - s \bv) \leq A(\ba',\bb-\bc) = 0$.
Let $\alpha(t): [0,1] \to \Reals$ be
$\alpha(t) = A(\ba',\bb + (t-1)s\bv)$.
As before, the intermediate value theorem guarantees a unique
$t^{*} \in (0,1)$ such that $\alpha(t^{*}) = 0$.
Taking $t = (1-t^{*})s$ establishes the claim.
\end{proof}

\begin{theorem}
\thmlabel{virtual-amm}
Given an $(n+m)$-dimensional AMM $A(\bx,\by) = 0$,
and a valuation $\bv \in \int(\Delta^m)$,
the virtualized $(A|\bv)(\bx,z)$ is an $(n+1)$-dimensional AMM.
\end{theorem}

\begin{proof}\sloppy
  It is enough to check that $A|\bv$ is twice-differentiable,
  strictly increasing, and $\upper(A|\bv)$ is strictly convex.
  $(A|\bv)$ is twice-differentiable because $A$ is twice-differentiable.

  To show that $A|\bv$ is strictly increasing,
  let $\bx' \geq \bx$ and $z' \gneqq z$.
  \begin{equation*}
    \begin{aligned}
    (A|\bv)&(x_1,\ldots,x_n,z)\\
    &= A(x_1,\ldots,x_n, v_1 z + r_1, \ldots, v_m z + r_m)\\
    &< A(x_1',\ldots,x_n', v_1 z' + r_1, \ldots, v_m z' + r_m)\\
    &= (A|\bv)(x_1',\ldots,x_n',z').
    \end{aligned}
  \end{equation*}
  
  To show that $\upper(A|\bv)$ is strictly convex,
  pick distinct $(\bx,z)$ and $(\bx',z')$ on the manifold:
  \begin{equation*}
    (A|\bv)(\bx,z) = (A|\bv)(\bx',z') = 0.
  \end{equation*}
  For $t \in (0,1)$,
  \begin{equation*}
    \begin{aligned}
    (A|\bv)&(t \bx + (1-t) \bx', t z + (1-t)z')\\
    &= A(t \bx + (1-t) \bx',\bv (tz + (1-t)z') + \br) \\
    &=  A(t \bx + (1-t) \bx',t(\bv z + r)  + (1-t)(\bv z' + r)) \\
    &= A(t (\bx,\bv z+ r) + (1-t) (\bx',\bv z' + r)) > 0
    \end{aligned}
  \end{equation*}
  by the strict convexity of $\upper(A)$.
\end{proof}

Stable points are well-behaved under virtualization.
Let $A(\bx,\by) = 0$ be an $(n+m)$-dimensional AMM
and $\bv \in \int(\Delta^n)$ a valuation.
If $A$ is an $(n+m)$-dimensional AMM in state $(\ba,\bb)$,
and $\bv \in \int(\Delta^n)$ a valuation,
then $(A|\bv)(\ba,c) = A(\ba,c\bv + \br) = A(\ba,\bb) = 0$.
For any state in the virtualized AMM,
$(A|\bv)(\bx,t) = A(\bx,\bb + (t-c) \bv) = 0$.
This expression depends on $\bb$ and $\bv$,
where $c$ is a constant determined by $\bb$ and $\bv$.
Since $(A|\bv)(\bx,t)$ is an $(n+1)$-dimensional AMM,
we can write $t = f(\bx)$ for some $f: \PosReals^n \to \Reals$.
The virtualized AMM can be expressed as $(\bx,f(\bx))$
where $A(\bx,\bb + (f(\bx) - c)\bv) = 0$.

\begin{lemma}
\lemmalabel{virtual-stable}
\sloppy
    If $(\ba^{*},\bb^{*})$ is the stable point on AMM $A(\bx,\by)$ for valuation $(\bv,\bw)$,
    then $(\ba^{*},f(\ba^{*}))$ is the stable point on the virtualized AMM $(A| \bw)(\bx,t)$
    for the valuation $(\bv,\|\bw\|^2_2) / \| (\bv,\|\bw\|^2_2) \|_1$.
\end{lemma}

\begin{proof}
\sloppy
    Suppose $(\ba^{*},f(\ba^{*}))$ is not a stable point for $(\bv,\|\bw\|^2_2)$:
    there is a distinct point $(\ba,f(\ba)) \in A|\bw$ where $\bv \cdot \ba + \|\bw\|^2_2 f(\ba) < \bv \cdot \ba^{*} + \|\bw\|^2_2 f(\ba^{*})$.
    Now define $\bb = \bb^{*} + \bw(f(\ba) - f(\ba^{*}))$, which by the virtualization construction we have $(\ba,\bb) \in A$.
    Then
    \begin{align*}
        &\bv \cdot \ba + \bw \cdot \bb = \bv \cdot \ba + \bw \cdot \bb^{*} + \bw \cdot \bw (f(\ba) - f(\ba^{*})) \\
        &= \bv \cdot \ba + \|\bw\|^2_2 f(\ba) - \|\bw\|^2_2 f(\ba^{*}) + \bw \cdot \bb^{*} \\
        &<  \bv \cdot \ba^{*} + \|\bw\|^2_2 f(\ba^{*}) - \|\bw\|^2_2 f(\ba^{*}) + \bw \cdot \bb^{*}\\
        &= \bv \cdot \ba^{*}  + \bw \cdot \bb^{*}
    \end{align*}
    This is a contradiction since $(\ba^{*},\bb^{*})$ is the stable point for $(\bv,\bw)$.
\end{proof}
