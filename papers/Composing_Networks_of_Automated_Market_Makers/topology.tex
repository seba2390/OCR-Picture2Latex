How many truly distinct AMMs of a given dimension are there?
Considered as a mathematical object,
much of an AMM's structure is captured by the link between valuations and their stable points.
We say that two AMMs are \emph{topologically equivalent}
if there is a stable-point preserving homeomorphism between their manifolds.
By itself, a homeomorphism between manifolds conveys little information,
but a homeomorphism that preserves stable points preserves the AMMs' common underlying structure.

In this section we show that \emph{all} AMMs over the same set of assets,
if they satisfy our axioms, are topologically equivalent.
More precisely,
for any two AMMs $A(x_1,\ldots,x_n)$ and $B(x_1,\ldots,x_n)$ over asset types
$X_1, \ldots, X_n$, satisfying our axioms,
there is a homeomorphism $\mu: A \to B$
such that $\bx$ and $\mu(\bx)$ are the stable states for the same valuation.
This proof relies on the uniqueness of stable points:
for example it would not hold if AMM functions were convex
instead of strictly convex.

Although topological equivalence implies a common mathematical structure,
two topologically equivalent AMMs may differ substantially
with respect to price slippage, fees,
or how expensive it is to move from one valuation's stable state to another's.

Recall from \lemmaref{stable-unique} and \lemmaref{stable-for-some}
that there is a unique function $\phi: \int(\Delta^n) \to A$
carrying each valuation to its unique stable point.

\begin{lemma}%L17
\lemmalabel{stable-state-biject}
\sloppy
For $A$, an AMM,
the stable point map
\begin{equation*}
    \phi:~\int(\Delta^n)~\to~A
\end{equation*}
is a continuous bijection.
\end{lemma}

\begin{proof}
The map $\phi$ is surjective by \lemmaref{stable-for-some},
and injective by \lemmaref{stable-unique}.
To show continuity,
consider the sequence
$\set{\bv_n}_{n=1}^{\infty} \subset \int(\Delta^n)$ where $\lim_{n \to \infty}\bv_n = \bv$.
Let $\tilde{\bx} = \lim_{n \to \infty} \phi(\bv_n)$ and $\bx^{*} = \phi(\bv)$.
  Suppose $\tilde{\bx} \neq \bx^{*}$.
  Note that $\bv \cdot \bx^{*} < \bv \cdot \tilde{\bx}$ by definition of stable point.
  Letting $\overline{\bx} = \frac{\bx^{*} + \tilde{\bx}}{2}$ by strict convexity we know $\overline{\bx} \in \int(\upper(A))$.
  We also have that $\bv \cdot \bx^{*} < \bv \cdot \overline{\bx} < \bv \cdot \tilde{\bx}$.
  Notice now that $\bv_n \cdot \overline{\bx} > \bv_n \cdot \phi(\bv_n)$ by definition so taking limits we get
  $\bv \cdot \overline{\bx} > \bv \cdot \tilde{\bx}$, a contradiction.
  Thus $\lim_{n \to \infty} \phi(\bv_n) = \phi(\bv)$.
\end{proof}

\begin{lemma}
\lemmalabel{stable-state-homeo}
For $A$, an AMM,
the stable point map
\begin{equation*}
    \phi:~\int(\Delta^n)~\to~A
\end{equation*}
is a homeomorphism.
\end{lemma}

\begin{proof}
From \lemmaref{stable-state-biject}, $\phi$ is both bijective and continuous,
so it is enough to show $\phi^{-1}$ is continuous.
For any $\bv$ with stable point $\bx$,
the first-order conditions imply that $\bv = \lambda \nabla A(\bx)$
for some non-zero (Lagrange multiplier) $\lambda \in \Reals$.
Since $A$ is strictly increasing, $\lambda > 0$.
Thus we can think of $\bv$ as a function of $\bx$,
written $\bv(\bx) = \lambda(\bx) \nabla A(\bx)$.
Because $A$ is continuously differentiable, $\nabla A(\bx)$ is continuous,
so it is enough to check $\lambda(\bx)$ is continuous.
Because $\bv(\bx)$ is a convex combination, and $\lambda(\bx)$ is unique,
\begin{equation*}
\lambda(\bx) = \frac{1}{\|\nabla A(\bx)\|_1},
\end{equation*}
which is continuous because each $\frac{\partial A(\bx)}{\partial x_i}$ is continuous.
It follows that $\phi^{-1}$ is continuous.
\end{proof}

\begin{theorem}
Let $A$ and $B$ be AMMs over the same set of assets.
There is a homeomorphism $\mu: A \to B$
that preserves stable points:
if $\bx$ is the stable point for valuation $\bv$ in $A$,
then $\mu(\bx)$ is the stable point for $\bv$ in $B$.
\end{theorem}

\begin{proof}
  By \lemmaref{stable-state-homeo},
  there exist homeomorphisms
  \begin{align*}
  \phi: \int(\Delta^n) &\rightarrow A,\\
  \phi': \int(\Delta^n) & \rightarrow B
  \end{align*}
  Their composition $\mu = \phi' \circ \phi^{-1}$ is also a homeomorphism.
  For $\bv \in \int(\Delta^n)$ with stable points $\bx \in A, \bx' \in B$,
  $\phi(\bv) = \bx$ and $\phi^{'}(\bv) = \bx^{'}$,
  so $\mu(\bx) = \phi^{'}(\phi^{-1}(\bx)) = \phi^{'}(\bv) = \bx^{'}$,
  implying that $\mu$ preserves stable points.
\end{proof}
Note that this result requires that $A$ have continuous first derivatives.