In practice, each AMM trade incurs a \emph{fee}:
each trader's deposit includes a small additional fee
added directly to the AMM's capitalization to benefit the liquidity providers.
So far, we have neglected fees because the amounts involved
are expected to be small in relation to AMM capitalization.
Nevertheless, in this section,
we show that AMM fees can be modeled as the sequential composition
of a no-fee AMM with a simple \emph{linear AMM}.
For brevity, we restrict our attention to AMMs that manage two assets.

Here is how fees work for an AMM such as Uniswap v1.
Let $A := (x,g(x))$, currently at state $(a,g(a))$,
and let $\gamma, 0 < \gamma < 1$, be the fee parameter.
\begin{itemize}
\item
  The trader sends $\delta$ units of $X$ to $A$.
\item
  $(1-\gamma)\delta$ units of $X$ are traded for $g(a) - g(a +(1-\gamma)\delta)$ of $Y$.
\item
  A fee of $\gamma \delta$ units of $X$ is deposited directly in $A$'s pool.
\end{itemize}
$A$'s final state is $(a + \delta, g(a + (1-\gamma)\delta))$.

A \emph{linear AMM} exchanges assets at a constant rate,
governed by a constraint function:
\begin{equation*}
  \lambda \cdot \bx = c,
\end{equation*}
where $\lambda$ is a constant vector, and $c > 0$ a constant.
A linear AMM \emph{does not} satisfy all our common-sense axioms:
because its exchange rate is fixed, it is not expressive,
and although its curve is convex, it is not strictly convex,
so it does not have unique stable points.
A stand-alone linear AMM $L$ would be a poor investment for providers,
because if the market rate diverges from the AMM's fixed rate,
perhaps so that $L$ overprices $X$ and underprices $Y$,
then arbitrage traders will exchange $X$ for $Y$ until $L$'s $Y$ reserve is depleted.

Even if stand-alone linear AMMs are not useful in practice,
they provide a convenient formal device for modeling AMMs with fees.
For example,
consider the 2-dimensional AMM $L := (x,f(x))$ that trades between $X$ and $X$,
where
\begin{equation*}
f(x) = (1-\gamma)(a + \delta - x),  
\end{equation*}
for $\gamma, 0 < \gamma < 1$.
Note that
\begin{equation}
\eqnlabel{fee-f}
f(a) = (1-\gamma) \delta \qquad \text{and} \qquad f(a+\delta) = 0. 
\end{equation}
Suppose $A:=(x,g(x))$ starts in state $(a,g(a))$,
and linear $L:=(x,f(x))$ starts in state $(a,f(a))$.
As in \secref{sequential},
the sequential composition $L \otimes A$ is given by
\begin{equation}
\eqnlabel{fee-h}
h(x) = g(a + f(a) - f(a+x)).
\end{equation}
Sending $\delta$ units of $X$ to $L \otimes A$ returns
\begin{align*}
\eqnlabel{fee-delta}
  h(\delta)
  &= g(a + f(a) - f(a + \delta))\\
  &= g(a + (1-\gamma) \delta)
\end{align*}
The trade leaves $L \otimes A$ in state $(a + \delta, g(a + (1-\gamma)\delta))$,
precisely the behavior of $A$ augmented by a fee $\gamma$ levied on incoming assets.
An alternative structure where a fee is levied on outgoing assets
can be modeled by composing the AMMs in the reverse order,
with a linear AMM deducting $\delta \gamma$ units of the trade's output to its pool.

