Angeris and Chitra~\cite{AngerisC2020}
introduce a \emph{constant function market maker} model
and consider conditions that ensure that agents who
interact with AMMs correctly report asset prices.
Our work, based on a similar but not identical AMM model,
focuses on properties such as defining AMM composition,
AMM topology, and the role of stable points.

\emph{Uniswap}~\cite{uniswap,zhang2018}
is a family of constant-product AMMs that originally traded between
ERC-20 tokens~\cite{erc20} and ether cryptocurrency.
Trading between ERC-20 assets requires sequential composition
of the kind analyzed in \secref{sequential}.
Uniswap v2~\cite{uniswapv2} added direct trading between
selected pairs of ERC-20 tokens,
and Uniswap v3 ~\cite{uniswapv2} allows liquidity providers
to restrict the range of prices in which their asset participate,
giving rise to a form of parallel composition of the kind analyzed in \secref{parallel}.

\emph{Bancor}~\cite{bancor} AMMs permit more flexible pricing schemes.
The state space manifold is parameterized by a \emph{weight},
where different weights yield different curves.
Bancor AMMs trade between ERC-20 assets and Bancor-issued BNT tokens.
Prices are a function of assets held and BNT tokens in circulation. 
Later versions~\cite{bancorv2} include integration with external
``price oracles'' to keep prices in line with market conditions.

\emph{Balancer}~\cite{balancer} AMMs trade across more than two assets.
Instead of constant product,
their state space is given by \emph{constant mean} formula
$c = \Pi_i^n x_i^{w_i}$,
where $c$ is constant, $x_i$ is amount of asset $X_i$ held by the contract,
and the $w_i$ are adjustable weights that form a valuation.

\emph{Curve}~\cite{curve} uses a custom curve specialized for trading across
multiple \emph{stablecoins},
digital assets whose values are tied to fiat currencies such as the
dollar, and likely to trade at near-parity.

There are many are more examples of AMMs:
see Pourpouneh \emph{et al.}~\cite{pourpouneh} for a survey.

Before there were AMMs for decentralized finance,
there were AMMs for \emph{event prediction markets},
where parties trade securities that pay a premium if
and only if some event occurs within a specified time.
A community of researchers has focused on prediction-market AMMs
\cite{AbernethyYV2011,ChenW2010,ChenP2007,Hanson2003,Hanson2007}.
Despite superficial similarities,
event prediction AMMs and security AMMs differ from DEFI AMMs
is important ways:
pricing models are different because
prediction outcome spaces are discrete rather than continuous,
prediction securities have finite lifetimes,
and composition of AMMs is not a concern.

We also note that the mathematical structure of AMMs resembles that of a consumer utility curve from classical economics ~\cite{micro}, where assets are replaced with goods.
The optimal arbitrage problem is not new.
A consumer choosing an optimal bundle of goods for a fixed set of prices is the same as an arbitrageur choosing an optimal point on an AMM with respect to a market valuation.
This is known as the \emph{expenditure minimization problem}~\cite{micro}.
While AMMs and consumer indifference surfaces are mathematically similar, they are different in application.
In particular,
traders interact with AMMs via composition,
an issue that does not arise in the consumer model.
