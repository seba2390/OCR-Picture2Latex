\subsection{Common-Sense Axioms}
\seclabel{commonsense}
Although any automaton that trades assets can be regarded as an AMM,
only those AMMs that satisfy certain properties
make sense in practice.
To make this presentation self-contained,
we list some informal, common-sense axioms any practical AMM should satisfy,
and then restate these informal requirements as more precise mathematical properties.
These properties mirror prior proposals~\cite{AngerisC2020,Berenzon2020,krishnamachari2021dynamic},
with some adjustments to encompass higher-dimensional AMMs 
(those that trade more than two kinds of assets),
and to facilitate later introduction of composition operators.

\begin{informal}[Continuity]
Every AMM state should define precise rates of exchange
between every pair of assets,
and trades should change these rates gradually rather than abruptly.
\end{informal}

\begin{informal}[Expressivity]
An AMM must be able to adapt to any market conditions.
\end{informal}
If an AMM is unable to adapt to market conditions that
cause one asset to be undervalued with respect to the others,
then traders will drain all of the undervalued asset from the AMM
at the expense of the providers.

\begin{informal}[Stability]
Every AMM state should be the appropriate response to some possible market condition.
\end{informal}
If no market condition justifies entering a particular state,
then that state is superfluous.

\begin{informal}[Convexity]
Slippage should work to the disadvantage of the trader.
Buying more of asset $X$ should make $X$ more expensive, not less.
\end{informal}

Otherwise, a runaway effect can occur where traders are motivated
to buy more and more of an asset until the AMM's supply is exhausted.

The constant product AMM $A:=(x,c/x)$ is an example of an AMM
that conforms to these axioms.
By contrast,
consider the \emph{constant-sum} AMM $C:=a x + b y = c$,
which trades between assets $X$ and $Y$ at a fixed exchange rate.
Constant-sum AMMs fail to satisfy expressivity:
as long as the exchange rate matches the market rate,
a constant-sum AMM trades without slippage,
but as soon as the market rate departs from the AMM's exchange rate,
arbitrage traders will exhaust the AMM's supply of the undervalued asset,
to the detriment of the liquidity providers.
For this reason, with very few exceptions~\cite{Mstable},
constant-sum AMMs are not used in practice.
(Note, however, that the continuity axiom implies
that any AMM's behavior \emph{approximates} the behavior of
a constant-sum market maker when trades are sufficiently small.)
Henceforth, except when explicitly noted,
we use ``AMM'' as shorthand for ``AMM that satisfies these axioms''.

\subsection{AMMs and Valuations}
Formally, an $n$-dimensional AMM is given by a function 
$A:\PosReals^n \to \Reals$ such that:
\begin{itemize}
\item
  For all $c \geq 0$,
  the \emph{upper contour set} $A(\bx) \geq c$ is closed and strictly convex.
\item
  $A(\bx)$ is strictly increasing in each coordinate, and
\item
  $A$ is twice-differentiable.
\end{itemize}
We adopt the convention that the AMM's state space is the level set $A(\bx) = 0$, though sometimes we replace 0 with another constant.
For brevity, when there is no danger of confusion,
we use $A$ to refer to the AMM's function,
its state space, and the AMM itself.
We use $\upper(A)$ for $A$'s upper contour set at $0$.
We sometimes define an AMM by saying $A := F(x) = 0$
to mean the AMM $A$'s states lie on the curve $F(x) = 0$

We remark that restricting the domain of an AMM's function to all-positive coordinates is a ``without loss of generality'' convention,
since an AMM's trading behavior is unaffected
by any linear change of variables.

It is often convenient to express an AMM in an alternative form.
The implicit function theorem ~\cite{implicit} implies that for any point on the manifold,
all but one coordinate can be chosen freely,
and the remaining coordinate is a twice-differentiable 
convex function $f_i$ of the rest:
\begin{multline*}
  A(x_1, \ldots, x_{i-1}, f_i(x_1, \ldots, x_{i-1}, x_{i+1}, \ldots x_n), x_{i+1}, \ldots x_n) = 0.
\end{multline*}

An opinion on the relative values of assets $X_1, \ldots, X_n$ is
captured by a \emph{valuation} $\bv = (v_1, \ldots, v_n)$,
where\footnote{For simplicity, we assume valuations never assign an asset relative value 0 or 1.} $0 < v_i < 1$ and $\sum_i v_i = 1$.
A trader who moves an AMM from state $\bx$ to state $\bx'$
makes a profit if the dot product $\bv \cdot (\bx - \bx')$ is positive,
and otherwise incurs a loss.
The current \emph{market value} is a valuation accepted by most participants.

The \emph{standard simplex} $\Delta^n \in \PosReals^n$ is the set of points
$(v_1,\ldots,v_n)$ where for $i=1,\ldots,n$, $0 < v_i < 1$, and $\sum^n_{i=1} v_i = 1$.
Each valuation forms the barycentric coordinates of a point
in the interior of the standard $n$-simplex\footnote{This notation is non-standard but convenient; others define $\Delta^n$ to be a subset of $\PosReals^{n+1}$.}: $\int(\Delta^n)$.

A \emph{stable point} for an AMM $A$ and valuation $\bv$
is a point $\bx \in A$ that minimizes the dot product $\bv \cdot \bx$.
If $\bv$ is the market valuation,
then any trader can make an arbitrage profit
by moving the AMM from any state to a stable point,
and no trader can make a profit by moving the AMM out of a stable point.

Of course, this model is idealized in several ways.
Asset pools are not continuous variables: they assume discrete values.
Computation is not infinite-precision:
round-off errors and numerical instability are concerns.
Popular AMM such as Uniswap v2~\cite{uniswapv2} and v3~\cite{uniswapv3} perform trades under a more complicated and dynamic model than the one
considered here.
Nevertheless, the problem of AMM composition remains largely unaddressed,
at least in a formal way,
and we believe our model and definitions capture enough of the
essential properties of AMMs to make useful progress.

We assume that the functions defining an AMM do not change over time.
In practice, an AMM's defining function can change over time.
For example, AMMs charge fees in a variety of ways,
usually adding those fees to the assets managed by the AMM.
Nevertheless, an AMM's defining function does not
change in the course of a single transaction,
the duration for which AMM composition is meaningful.
We will further discuss the effects of fees in \secref{fees}.

\subsection{Formal Axioms}
We are now able to restate the common-sense axioms of
\secref{commonsense} in more precise terms.

\begin{axiom}[Continuity]
For every AMM $A$, the function $A:\PosReals^n \to \mathbb{R}$
is twice-differentiable.
\end{axiom}
As far as we know,
existing AMMs use smooth (infinitely differentiable) functions.

\begin{axiom}[Convexity]
For every AMM $A$, $\upper(A)$ is strictly convex.
\end{axiom}

The following is a standard result from convex analysis.

\begin{lemma}
    \lemmalabel{epigraph}
    A function $f: \PosReals^n \to \PosReals$ is a strictly convex function if and only if its epigraph \\
    $\epi(f) = \set{(x,a) \in \PosReals^n \times \PosReals: f(x) \leq a}$ is a strictly convex set.
\end{lemma}

\begin{axiom}[Expressivity]
Every valuation has a unique stable point in $A$.
\end{axiom}

\begin{lemma}
  \lemmalabel{stable-upper}
For any AMM $A$,
every valuation $\bv \in \int(\Delta^n)$ has a stable point in $\upper(A)$.
\end{lemma}

\begin{proof}
    Pick any $\bx$ in $\upper(A)$.
    The set 
    \begin{equation*}
        S = \set{\bx^{'} \in \NNegReals^n:\bv \cdot \bx^{'} \leq \bv \cdot \bx}
    \end{equation*}
    is compact.
    Since $\upper(A)$ is closed, $\tilde{S} = S \cap \upper(A)$ is compact.
    A stable point solves the optimization problem
    \begin{equation*}
        \min_{\bx^{'} \in \tilde{S}} \bv \cdot \bx^{'}.
    \end{equation*}
    This minimum exists since $\bv \cdot \bx^{'}$
    is a continuous function on the compact set $\tilde{S}$.
\end{proof}

\begin{lemma}%L10
  \lemmalabel{stable-unique}
For any AMM $A$ and any valuation $\bv$,
the stable point for $\bv$ in $\upper(A)$ is unique.
\end{lemma}

\begin{proof}
  Fix valuation $\bv \in \int(\Delta^{n})$ and
  let $\bx,\bx^{'} \in \upper(A)$ where $\bx \neq \bx^{'}$
  and $w = \bv \cdot \bx = \bv \cdot \bx^{'}$.
  Because $\upper(A)$ is strictly convex,
  for all $t \in (0,1)$,
  $\tilde{\bx} = t \bx + (1 - t)\bx^{'}$ is in $\int(\upper(A))$.
  Since $\int(\upper(A))$ is open,
  there is some $\epsilon > 0$ such that the open
  $\epsilon$-ball $B_{\tilde{\bx}}(\epsilon) \subset \int(\upper(A))$.
  Now choose $\bx^{*}$ in $B_{\tilde{\bx}}(\epsilon)$
  such that $\bx^{*} < \tilde{\bx}$.
  Then we have
  \begin{align*}
    \bv \cdot \bx^{*}
    &< \bv \cdot \tilde{\bx}\\
    &= t \bv \cdot \bx + (1 - t)\bv \cdot \bx^{'}\\
    &= t w + (1-t) w = w,
\end{align*}
  a contradiction.
\end{proof}

For example,
for the 2-dimensional constant-product AMM given by $(x,c/x)$,
the valuation $(v,1-v)$ has the unique stable point
\begin{equation*}
    \left(\sqrt{\frac{c(1-v)}{v}},\sqrt{\frac{v}{c(1-v)}}\right).
\end{equation*}
More generally,
for the 2-dimensional constant-product AMM given by $(x,f(x))$,
the valuation $(v,1-v)$ has the unique stable point
\begin{equation*}
\left(f^{\prime -1}\left(-\frac{v}{1-v}\right), f\left(f^{\prime -1}
\left(-\frac{v}{1-v}\right)\right)\right).
\end{equation*}
\begin{lemma}%L11
  \lemmalabel{stable-bdry}
  For all valuations $\bv$,
  the stable point for $\bv$ in $\upper(A)$ lies on the level set $A$.
\end{lemma}

\begin{proof}
    Fix valuation $\bv$ and let $\bx^{*}$ be its stable point.
    Suppose that $\bx^{*} \not \in A$ but $\bx^{*} \in \int(\upper(A))$.
    As in \lemmaref{stable-unique},
    we can find $\epsilon > 0$ and an open ball $B_{\bx^{*}}(\epsilon)$ in $\int(\upper(A))$.
    Choosing $\bx \in B_{\bx^{*}}(\epsilon)$ such that $\bx< \bx^{*}$,
    we have $\bv \cdot \bx < \bv \cdot \bx^{*}$,
    a contradiction.
\end{proof}

\begin{corollary}
  Every AMM satisfies expressivity.
\end{corollary}

\begin{axiom}[Stability]
Every $\bx \in A$ is the stable point for some valuation.
\end{axiom}

For example,
for the 2-dimensional constant-product AMM $(x,c/x)$,
the point $(x,c/x)$ is the stable point for the
valuation $(\frac{c}{c+x^2},1-\frac{c}{c+x^2})$.
More generally,
for the 2-dimensional constant-product AMM $(x,f(x))$,
the point $(x,f(x))$ is the stable point for the
valuation $(\frac{f'(x)}{f'(x)-1},\frac{1}{1-f'(x)})$.

\begin{lemma}%Lemma 12
\lemmalabel{every-y-some-w}
  For every $\bx \in A$,
  there is some $\bw \in \NegReals^n$ such that $\bw \cdot \bx > \bw \cdot \by$ for all $\by \in \upper(A)$, $\by \neq \bx$. 
\end{lemma}

\begin{proof}
    Pick $\bx \in A$. 
    Since $\upper(A)$ is strictly convex,
    the supporting hyperplane theorem implies there is a non-zero
    $\bw \in \Reals^{n}$ such that
    $\bw \cdot \bx > \bw \cdot \by$ for all $\by \in upper(A)$, $\by \neq \bx$.
    We need to show that $\bw \in \NegReals^n$.
    Say $\bw \in \PosReals^n$.
    Choose any $\epsilon > 0$ and let $\tilde{\bx} = \bx + \epsilon \bw$
    so that $\bw \cdot \tilde{\bx} = \bw \cdot \bx + \epsilon\|\bw\|^2 > \bw \cdot \bx$.
    By monotonicity, $\tilde{\bx} \in \upper(A)$, a contradiction.
    Now consider the case where $\bw \in \Reals^n \setminus (\NegReals^n \cup \PosReals^n)$,
    namely $\bw$ cannot have all strictly positive or all strictly negative entries.
    We construct $\tilde{\bw}$ orthogonal to $\bw$.
    Replace all of the non-negative entries of $\bw$ in $\tilde{\bw}$
    with the sum of the absolute values of the negative coordinates.
    Replace all the negative entries of $\bw$ in $\tilde{\bw}$
    with the sum of all of the non-negative entries of $\bw$.
    These replacements guarantee that all coordinates of $\tilde{\bw}$ are positive,
    and $\tilde{\bw} \cdot \bw = 0$.
    Pick $\epsilon > 0$ and let
    $\tilde{\bx} = \bx + \epsilon \tilde{\bw}$
    so
    \begin{align*}
      \bw \cdot \tilde{\bx} 
      &= \bw \cdot \bx + \epsilon \bw \cdot \tilde{\bw} \\
      &= \bw \cdot \bx
    \end{align*}
and yet $\tilde{\bx} \in \upper(A)$ since $A$ is strictly increasing.
By contradiction, $\bw \in \NegReals^n$.
\end{proof}

\begin{lemma}
\lemmalabel{stable-for-some}
  For every $\bx \in A$,
  there exists a valuation $\bv$ for which $\bx$ is a stable point.
\end{lemma}

\begin{proof}
    Let $A$ be an $n$-dimensional AMM and let $\bx \in A$.
    Choose $\bw \in \NegReals^n$ as described in the previous lemma for $\bx$.
    We have $\bw \cdot \bx > \bw \cdot \by$ for $\by \in A$, $\by \neq \bx$.
    Negating $\bw$ and re-scaling the result so the elements sum to $1$ yields
    $\bv \in \int(\Delta^n)$.
    Thus we have $\bv \cdot \bx < \bv \cdot \by$ for all $\by \in A$ where $\by \neq \bx$,
    implying $\bx$ is a stable point for $\bv$.
\end{proof}

\begin{corollary}
  Every AMM satisfies stability.
\end{corollary}

In short, the goal of this paper is to balance axioms and composition operators so that the class of AMMs satisfying
these axioms remains closed under these composition operators.