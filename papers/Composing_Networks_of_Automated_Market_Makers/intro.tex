Decentralized finance (or ``DeFi'') has become a booming area of
distributed computing.
For example, between June 2020 and October 2020,
the total value of assets locked in decentralized finance (DeFi) protocols
surged from \$1 billion to \$7.7 billion~\cite{defipulse}.
An \emph{automated market maker} (``AMM'') is an automaton that has
custody of several pools of assets,
sets prices for those assets according to a mathematical formula,
and is always willing to trade those assets at those prices.
Unlike traditional ``order book'' traders,
AMMs do not need to match up (and wait for) compatible buyers and sellers.
Today, AMMs such as Uniswap~\cite{uniswap}, Bancor~\cite{bancor}, Balancer~\cite{balancer},
and others~\cite{pourpouneh} have become one of the most popular ways to trade electronic assets.

Here is an example of a \emph{constant-product} AMM
loosely based on Uniswap v1~\cite{uniswap}.
The AMM has state $(x,y)$ if it has custody of $x$ units of asset $X$,
and $y$ units of asset $Y$.
The AMM's state is subject to the invariant that the product $xy$ is
constant.
The AMM's states thus lie on the hyperbolic curve $x y = c$, 
for $x,y > 0$ and constant $c > 0$.
If a trader transfers $dx$ units of $X$ to the AMM, 
the AMM will transfer $dy$ units of $Y$ back to the trader,
preserving the invariant $(x+dx) (y - dy) = x y = c$.
The client profits if the value of $dy$ units of $Y$ exceeds
the value of $dx$ units of $X$ in the current market (or at another AMM).

The \emph{price} of asset $Y$ in units of $X$ at state $(x,y)$
is the curve's slope at that point.
Trades move the AMM's state along the curve:
trading $X$ for $Y$ makes $X$ cheaper and $Y$ more expensive,
a phenomenon known as \emph{slippage}.
Usually, an AMM's state reflects current market conditions:
if a constant-product AMM is in state $(x,y)$,
then the market value of $x$ units of $X$ should be the same as $y$ units of $Y$,
because otherwise a trader can make an arbitrage profit by buying
the undervalued asset.
Note that a constant-product AMM can adjust to any
(finite, non-zero)
market rate between $X$ and $Y$,
and every AMM state matches some market valuation.

It is natural to compose AMMs.
If AMM $A$ trades assets $X$ and $Y$,
and $B$ trades assets $Y$ and $Z$,
then a trader can buy $Z$ with $X$ by transferring $X$ to $A$,
feeding $A$'s $Y$ output to $B$, and pocketing $B$'s $Z$ output.
Existing systems of AMMs encourage exactly this kind of composition:
for Uniswap v1, the intermediate asset $Y$ would be ether (ETH),
and for Bancor, it would be Bancor network token (BNT).
Here is the question at the heart of this paper:
can we treat the result of composing these two constant-product AMMs
as a ``black box'' AMM for trading $X$ for $Z$?
Note that this composition is not itself a constant-product AMM,
so the class of constant-product AMMs is not closed under
composition for any reasonable notion of composition.
Can we instead pick a broader AMM definition that does support
common-sense notions of composition?
What constraints on pricing formulas yield AMMs that behave reasonably under composition?
What should composition mean when AMMs trade more than two kinds of assets?
In short, what notions of AMM composition make sense?

This paper makes the following contributions.
\begin{itemize}
\item 
We give an axiomatic characterization for well-behaved AMMs.
These axioms build on prior work,
but require small but critical changes to support
mathematical properties such as composition.

\item
For AMMs satisfying these axioms,
we show there is a duality between asset valuations,
and AMM states where assets are balanced to reflect those valuations.
Computation can be done in whichever domain is more convenient.

\item
The paper's principal contribution is to
propose novel operators for sequential and parallel
composition of AMMs.
Well-structured notions of composition for
``higher-dimensional'' AMMs that manage more than two asset classes
requires novel intermediate projection and virtualization operators.
\end{itemize}
Properly defined,
AMMs are mathematical objects that are closed under sequential and parallel composition.
We hope this paper will encourage further research into how AMMs can be combined
into networks and what such networks can do.

This paper is organized as follows.
\secref{model} describes the problem and the model of computation,
\secref{definitions} presents in axiomatic form the properties a practical AMM should satisfy,
\secref{topology} shows that all such AMMs have a common underlying topological structure,
and \secref{operators} presents basic mathematical operators useful for defining composition.
\secref{sequential} defines \emph{sequential composition},
where the outputs of one trade becomes the inputs to another.
Some careful choices are needed to impose order on composition of ``higher-dimensional''
AMMs that manage more than two assets.
\secref{parallel} defines \emph{parallel composition},
where a trader decides how to split a trade among alternative AMMs.
\secref{fees} shows how AMMs with fees fit into out composition framework,
\secref{related} surveys related work,
and \secref{conclusions} discusses future directions and open problems.
