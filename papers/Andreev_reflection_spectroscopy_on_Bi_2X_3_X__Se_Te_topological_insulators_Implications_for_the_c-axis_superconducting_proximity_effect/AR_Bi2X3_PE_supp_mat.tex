%% ****** Start of file apstemplate.tex ****** %
%%
%%
%%   This file is part of the APS files in the REVTeX 4 distribution.
%%   Version 4.1r of REVTeX, August 2010
%%
%%
%%   Copyright (c) 2001, 2009, 2010 The American Physical Society.
%%
%%   See the REVTeX 4 README file for restrictions and more information.
%%
%
% This is a template for producing manuscripts for use with REVTEX 4.0
% Copy this file to another name and then work on that file.
% That way, you always have this original template file to use.
%
% Group addresses by affiliation; use superscriptaddress for long
% author lists, or if there are many overlapping affiliations.
% For Phys. Rev. appearance, change preprint to twocolumn.
% Choose pra, prb, prc, prd, pre, prl, prstab, prstper, or rmp for journal
%  Add 'draft' option to mark overfull boxes with black boxes
%  Add 'showpacs' option to make PACS codes appear
%  Add 'showkeys' option to make keywords appear
%\documentclass[aps,prb,reprint,groupedaddress]{revtex4-1}
%\documentclass[aps,prl,12pt,onecolumn,notitlepage,superscriptaddress]{revtex4-1}

%\documentclass[aps,prl,12pt,onecolumn,notitlepage,superscriptaddress]{revtex4-1}


\documentclass[aps,prl,onecolumn,notitlepage,reprintnumbers,amsmath,amssymb,superscriptaddress,citeautoscript]{revtex4-1}

\usepackage{graphicx}
\usepackage{dcolumn}% Align table columns on decimal point
\usepackage{bm}% bold math
\usepackage{color,soul}
\usepackage{multirow}
\usepackage{verbatim}
\usepackage[normalem]{ulem}


%\documentclass[aps,prl,preprint,superscriptaddress]{revtex4-1}
%\documentclass[aps,prl,reprint,groupedaddress]{revtex4-1}

% You should use BibTeX and apsrev.bst for references
% Choosing a journal automatically selects the correct APS
% BibTeX style file (bst file), so only uncomment the line
% below if necessary.
%\bibliographystyle{apsrev4-1}

\begin{document}

% Use the \preprint command to place your local institutional report
% number in the upper righthand corner of the title page in preprint mode.
% Multiple \preprint commands are allowed.
% Use the 'preprintnumbers' class option to override journal defaults
% to display numbers if necessary
%\preprint{}

\title{Supplemental material for \\ ``Andreev reflection spectroscopy on Bi$_{2}$X$_{3}$ (X = Se, Te) topological insulators: Implications for the \textit{c}-axis superconducting proximity effect''}


\author{C. R. Granstrom}
\affiliation{Department of Physics, University of Toronto, 60 St. George St., Toronto, Ontario M5S 1A7, Canada}

\author{I. Fridman}
\affiliation{Department of Physics, University of Toronto, 60 St. George St., Toronto, Ontario M5S 1A7, Canada}

\author{H.-C. Lei}
\altaffiliation{Present address: Department of Physics, Renmin University, Beijing 100872, China}
\affiliation{Condensed Matter Physics \& Materials Science Department, Brookhaven National Laboratory, NY 11973, USA}

\author{C. Petrovic}
\affiliation{Condensed Matter Physics \& Materials Science Department, Brookhaven National Laboratory, NY 11973, USA}

\author{J. Y. T. Wei}
\affiliation{Department of Physics, University of Toronto, 60 St. George St., Toronto, Ontario M5S 1A7, Canada}
\affiliation{Canadian Institute for Advanced Research, Toronto, Ontario M5G 1Z8, Canada}



\renewcommand{\thefigure}{S\arabic{figure}}

\setcounter{figure}{0}

\renewcommand{\thetable}{S\arabic{table}} 
 
\setcounter{table}{0}



\pacs{74.45.+c, 73.20.At, 73.63.Rt}


\maketitle

\section{S1. Geometries used in TI/SC PE experiments}

Experimental studies of the PE on Bi$_{2}$X$_{3}$ samples have primarily been based on \textit{s}-wave pairing in two types of heterostructures (Figure \ref{fig:geometries}): TI/SC, where the PE is primarily along the \textit{c}-axis; and SC/TI/SC, where the PE is in the \textit{a}-\textit{b} plane.

\begin{figure}[ht]
\includegraphics[width=0.4\textwidth]{expt_geometries.eps}
\caption{\label{fig:geometries} Two types of heterostructures used in proximity-effect experiments between Bi$_{2}$X$_{3}$ TIs and $s$-wave SCs.}
\end{figure}



\section{S2. Experimental Details}

For PCS measurements, $dI/dV$  vs. bias voltage was measured using a four-terminal geometry with an AC resistance bridge. For STS measurements, standard two-point AC lock-in technique was used. For PCS, the STM approach method was used to gently bring the tip and sample into contact without crashes. Namely, a piezo stepper motor with a $\sim10$ nm step size was used to move the tip into tunneling range of the sample. The STM feedback was then disengaged, the four-terminal wiring engaged, and the tip gently brought into contact with the sample using the STM piezo.   Nb tips of 99.9\% purity were cut, then cleaned \textit{in situ} through high-voltage field emission on a Ag film. A typical $dI/dV$ spectrum taken by STS with these tips is shown in Figure \ref{fig:NbSTS}. Although the superconducting gap of Nb is $\sim1.5$ meV at 0 K, spectral broadening due to finite quasiparticle lifetime and broadening of the Fermi-Dirac function at 4.2 K causes the coherence peaks in STS data to shift to $\sim\pm$ 2.5 mV.

\newpage

\begin{figure}[ht]
\includegraphics[width=0.35\textwidth]{Nb_STS.eps}
\caption{\label{fig:NbSTS} STS data taken on a Ag film with a Nb tip at 4.2 K. Feedback settings: sample bias -18 mV and tunneling current 200 pA.}
\end{figure}


The Bi$_{2}$X$_{3}$ single crystals were grown with the self-flux method \cite{Fisk1989}, using excess Se melt for Bi$_{2}$Se$_{3}$ and excess Te melt for Bi$_{2}$Te$_{3}$. The crystals were cleaved either in air or in a N$_{2}$-filled glovebox. Immediately after cleaving, contacts were made on the freshly exposed $c$-axis surface of the crystals using Ag paint, shortly before being loaded into the STM. Typical atomically-resolved STM topographs of Bi$_{2}$Se$_{3}$ and Bi$_{2}$Te$_{3}$ crystals are shown in the insets of Figures 3 and 4 in the main text, respectively. 

\section{S3. Current flow through BBS}

Figure \ref{fig:current_flow} shows a schematic of the contact geometry used in our experiment. It is important to note that in this geometry, a significant amount of current flows via BBS along the $c$-axis of the crystal. This predominance of BBS is due to the large difference in resistance between TSS and BBS, as detailed below. 

The sheet resistance corresponding to the TSS in our crystals (see Figure \ref{fig:current_flow}) can be estimated as $R_{\Box}=3.69\,k\Omega$ from Ref. [\onlinecite{Checkelsky2011}], which used electrostatic gating and Ca doping of Bi$_{2}$Se$_{3}$ crystals to shift the Fermi level into the bulk band gap. For non-ideal Bi$_{2}$Se$_{3}$ crystals that have dopings similar to ours, the $ab$-plane and $c$-axis resistivities corresponding to the BBS are $\rho_{ab}^{\mathrm{BBS}}=0.25\,\mathrm{m}\Omega\mathrm{cm}$ \cite{Analytis2010} and $\rho_{c}^{\mathrm{BBS}}=1.1\,\mathrm{m}\Omega\mathrm{cm}$ \cite{Gobrecht1969}.

Using the above values, we find that the resistance corresponding to our crystals' BBS is 6 orders of magnitude smaller than that of the TSS: the dimensions of our crystals are $\sim$ 1 x 5 x 5 mm, so the $ab$-plane and $c$-axis BBS resistances for Bi$_{2}$Se$_{3}$ can be estimated as $R_{ab}^{\mathrm{BBS}}=\left(0.25\,\mathrm{m}\Omega\mathrm{cm}\times0.25\,\mathrm{cm}\right)/(0.1\,\mathrm{cm}\times0.5\,\mathrm{cm})=1.25\,\mathrm{m}\Omega$, and $R_{c}^{\mathrm{BBS}}=\left(1.1\,\mathrm{m}\Omega\mathrm{cm}\times0.1\,\mathrm{cm}\times2\right)/(0.1\,\mathrm{cm}\times0.1\,\mathrm{cm})=22\,\mathrm{m}\Omega$. Thus, the BBS shunt much of the current.


\begin{figure}[ht]
\includegraphics[width=0.65\textwidth]{current_flow.eps}
\caption{\label{fig:current_flow} Schematic of our experimental geometry. The BBS shunt much of the current, as indicated by the white arrow.}
\end{figure} 

\section{S4. Background conductance normalization}

Our PCS data can be normalized by dividing out the background conductance, after interpolating the latter from outside $\sim\pm$ 10 mV using a polynomial fit. An example of this procedure is shown in Figure \ref{fig:fitting}, using Bi$_{2}$Te$_{3}$ PCS data from Figure 1 in the main text. The normalization largely removes the spectral asymmetry, as shown in panel (b).



\begin{figure}[ht]
\includegraphics[width=0.4\textwidth]{norm_proc.eps}
\caption{\label{fig:fitting}(a) PCS data from Figure 1 in the main text (open circles) and a polynomial fit (solid line) to the conductance background above $\pm$ 10 mV. (b) Normalized data, obtained by dividing the $dI/dV$ data by the polynomial fit.}
\end{figure}


\section{S5. BTK modeling of $\text{Bi}_{2}\text{X}_{3}/\text{Nb}$ junctions}


\begin{table}[bh]
\begin{center}
\begin{tabular}{|c|c|c|c|c|}
\hline
 & $v_{F}^{c\text{-axis}}$ $\left(10^{5}\,\text{m/s}\right)$ & $v_{F}^{ab\text{-plane}}$ $\left(10^{5}\,\text{m/s}\right)$ & $k_{F}^{c\text{-axis}}$ $\left(\text{nm}^{-1}\right)$ & $k_{F}^{ab\text{-plane}}$ $\left(\text{nm}^{-1}\right)$ \\ \hline
Nb & 2.7 & 2.7 & 4.2 & 4.2 \\ \hline
TSS in Bi$_{2}$Se$_{3}$ & -- & 5.0 & -- & 1.0 \\ \hline
BBS in Bi$_{2}$Se$_{3}$ & 3.2 & 6.7 & 1.5 & 0.7 \\ \hline
TSS in Bi$_{2}$Te$_{3}$ & -- & 4.0 & -- & 1.5 \\ \hline
BBS in Bi$_{2}$Te$_{3}$ & 7.1 & 1.2 & 0.6 & 0.3 \\ 
\hline
\end{tabular}
\end{center}
\caption{\label{tab:Fermi_values} Fermi velocities $v_F$ and wavevectors $k_F$ used in our BTK models. Values are taken from  Refs. [\onlinecite{Analytis2010,Gobrecht1969,Crabtree1987,Xia2009,Chen2009,Koehler1976,Qu2010}]. The Nb values are averages over $k$-space, to account for our polycrystalline Nb tip.}
\end{table}

For the single-interface BTK model in the main text, we apply the formulas in the extended BTK model derived by Mortensen et al. \cite{Mortensen1999}. In this model, the authors derive an angle-dependent $Z(\theta)$:

\begin{align*}
Z(\theta)=\sqrt{\Gamma(\theta)\left(\frac{Z_b}{\cos\theta}\right)^2+\frac{\left(\Gamma(\theta)r_v-1\right)^2}{4\Gamma(\theta)r_v}},
\label{eq:Z}
\end{align*}




where $\Gamma(\theta)\equiv\cos\theta/\sqrt{1-r_{k}^2\sin^2\theta}$, and $r_k=k^N_F / k^{SC}_F$ is the Fermi wavevector matching term mentioned in the main text. This model was used to generate Figure 5(a) in the main text. In our simulation, the $r_v$ and $r_k$ values were fixed by the $c$-axis $v_F$ and $ab$-plane $k_F$ values in Table \ref{tab:Fermi_values}, respectively. The other parameters in the model---$Z_b$, the superconducting energy gap $\Delta$, and a finite quasiparticle lifetime term $\Gamma$---were used as fitting parameters, while $T$ was fixed at 4.2 K. The values for these fitting parameters were 0.3, 1.5 meV, and 1.1 meV, respectively.  As stated in the main text, this BTK model indicates that TSS contribute negligibly to $c$-axis AR, when BBS are also present.

For our multi-interface BTK model, we assume that a Bi$_{2}$X$_{3}$/Nb junction has both $c$-axis and in-plane interfaces, as shown in Figure \ref{fig:BTK_geometries}. First, we apply the single-interface BTK model to each interface, again using the $k_F$ and $v_F$ values from Table \ref{tab:Fermi_values}, but this time using the $ab$-plane $v_F$ for the TSS. $Z_b$, $\Delta$, and $\Gamma$ were used as fitting parameters to resemble the data in Figure 2(a), and had values of 0.35, 1.5 meV, and 1.35 meV, respectively. $T$ was again fixed at 4.2 K.

Next, to weight the BBS' and TSS' contributions to the total conductance spectrum, we estimate the number of Landauer conduction channels $N_c$ for each interface. For the BBS' $c$-axis interface of area $\pi a^2$, $N_c =\left(k_{F}^{2}\pi a^2 \right)/\left(4\pi\right)= \left(k_{F} a \right)^2 /4$. For the TSS' in-plane interface of circumference $2 \pi a$, $N_c = \left(k_{F}2\pi a\right)/\pi =2 k_{F} a $.

Finally, we apply these weightings to the BTK spectra for each interface. While we find double-hump AR characteristics for both the BBS/Nb and TSS/Nb channels, once again, the BBS predominate over the TSS. Namely, $N_c^{\text{BBS}}\approx5,200$ and $N_c^{\text{TSS}}\approx260$, as plotted in Figure 5(b) in the main text.

\begin{figure}[th]
\includegraphics[width=0.35\textwidth]{BTK_multi_int.eps}
\caption{\label{fig:BTK_geometries} Illustration of a Bi$_2$X$_3$/Nb junction used for our multi-interface BTK model. The interface for the BBS is a circle of area $\pi a^2$, while the interface for the TSS is a ring of circumference $2\pi a$.}
\end{figure}

\bibliography{./AR_Bi2X3_PE_supp_mat}



\end{document}
