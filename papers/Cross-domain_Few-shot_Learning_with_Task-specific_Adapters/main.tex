% CVPR 2022 Paper Template
% based on the CVPR template provided by Ming-Ming Cheng (https://github.com/MCG-NKU/CVPR_Template)
% modified and extended by Stefan Roth (stefan.roth@NOSPAMtu-darmstadt.de)

\documentclass[10pt,twocolumn,letterpaper]{article}

%%%%%%%%% PAPER TYPE  - PLEASE UPDATE FOR FINAL VERSION
\usepackage{authblk}
% \usepackage[review]{cvpr}      % To produce the REVIEW version
% \usepackage{cvpr}              % To produce the CAMERA-READY version
\usepackage[pagenumbers]{cvpr} % To force page numbers, e.g. for an arXiv version
\usepackage[accsupp]{axessibility}  % Improves PDF readability for those with disabilities.

% Include other packages here, before hyperref.
% \usepackage{graphicx}
\usepackage{amsmath}
% \usepackage{amssymb}
% \usepackage{booktabs}
% OLD PREAMBLE:

% \usepackage{jsen}
% \usepackage{cite}
% \usepackage{amsmath,amssymb,amsfonts, bbm, mathtools}
% \usepackage{algorithm,algorithmic}
% \usepackage{graphicx}
% \usepackage{textcomp}
% \usepackage{wrapfig}
% \usepackage{xfrac}
% \usepackage{stackengine}
% \usepackage{subfigure}
% \def\delequal{\mathrel{\ensurestackMath{\stackon[1pt]{=}{\scriptstyle\Delta}}}}



% \usepackage{color, soul}
% \newcommand{\hlt}[1]{\hl{#1}}
% \newcommand{\red}[1]{\textcolor{red}{#1}}

% \def\BibTeX{{\rm B\kern-.05em{\sc i\kern-.025em b}\kern-.08em
%     T\kern-.1667em\lower.7ex\hbox{E}\kern-.125emX}}
% \markboth{\journalname, VOL. XX, NO. XX, XXXX 2017}
% {Author \MakeLowercase{\textit{et al.}}: Preparation of Papers for IEEE TRANSACTIONS and JOURNALS (February 2017)}
% \definecolor{abstractbg}{rgb}{0.89804,0.94510,0.83137}
% \setlength{\fboxrule}{0pt}
% \setlength{\fboxsep}{0pt}

% NEW PREAMBLE:


\usepackage{amsmath,amsfonts,amssymb,bbm, amsthm, xfrac}
\usepackage{algorithmic}
\usepackage{algorithm}
\usepackage{array, multirow}
% \usepackage[caption=false,font=normalsize,labelfont=sf,textfont=sf]{subfig}
\usepackage{caption, subcaption}
\usepackage{textcomp}
\usepackage{stfloats}
\usepackage{url}
\usepackage{verbatim}
\usepackage{graphicx}
\usepackage{cite}
\usepackage{caption}
\usepackage{subcaption}
\hyphenation{}

\theoremstyle{plain}
\newtheorem{theorem}{Theorem}

\usepackage{color, soul}
\newcommand{\hlt}[1]{\hl{#1}}
\newcommand{\red}[1]{\textcolor{red}{#1}}


% It is strongly recommended to use hyperref, especially for the review version.
% hyperref with option pagebackref eases the reviewers' job.
% Please disable hyperref *only* if you encounter grave issues, e.g. with the
% file validation for the camera-ready version.
%
% If you comment hyperref and then uncomment it, you should delete
% ReviewTempalte.aux before re-running LaTeX.
% (Or just hit 'q' on the first LaTeX run, let it finish, and you
%  should be clear).
% \usepackage[pagebackref,breaklinks,colorlinks]{hyperref}


% Support for easy cross-referencing
% \usepackage[capitalize]{cleveref}
\crefname{section}{Sec.}{Secs.}
\Crefname{section}{Section}{Sections}
\Crefname{table}{Table}{Tables}
\crefname{table}{Tab.}{Tabs.}
% \newcommand{\whupdate}{\textcolor{blue}}



\begin{document}

%%%%%%%%% TITLE - PLEASE UPDATE
\title{Cross-domain Few-shot Learning with Task-specific Adapters}


\author[]{\vspace{-0.3cm}Wei-Hong Li}
\author[]{Xialei Liu\thanks{Xialei Liu is the corresponding author.}}
\author[]{Hakan Bilen\vspace{-0.25cm}}

\affil[]{VICO Group, University of Edinburgh, United Kingdom\vspace{-0.25cm}}
\affil[]{\small \rurl{github.com/VICO-UoE/URL}\vspace{-0.3cm}}

\maketitle

\begin{abstract}
    In this paper, we look at the problem of cross-domain few-shot classification that aims to learn a classifier from previously unseen classes and domains with few labeled samples. 
    Recent approaches broadly solve this problem by parameterizing their few-shot classifiers with task-agnostic and task-specific weights where the former is typically learned on a large training set and the latter is dynamically predicted through an auxiliary network conditioned on a small support set. 
    In this work, we focus on the estimation of the latter, and propose to learn task-specific weights from scratch directly on a small support set, in contrast to dynamically estimating them.
    In particular, through systematic analysis, we show that task-specific weights through parametric adapters in matrix form with residual connections to multiple intermediate layers of a backbone network significantly improves the performance of the state-of-the-art models in the Meta-Dataset benchmark with minor additional cost.
\end{abstract}

%-------------------------------------------------------------------------
\section{Introduction}\label{sec:intro}
Reinforcement learning has achieved great success in areas such as Game-playing \citep{silver2018general,vinyals2019grandmaster}, robotics \cite{kober2013reinforcement}, large language models \citep{ouyang2022training}, etc.
However, due to safety concerns or physical limitations, in some real-world reinforcement learning problems, we must consider additional constraints that may influence the optimal policy and the learning process \citep{garcia2015comprehensive}.
% For example, a robotic arm must not take actions that may cause harm to itself or the environments.
A standard framework to handle such cases is the constrained Markov Decision Process (CMDP) \citep{altman1999constrained}.
Within the CMDP framework, the agent has to maximize
the expected cumulative reward while
obeying a finite number of constraints, which are usually in the form of expected cumulative cost criteria.

However, we are sometimes concerned with the problem with a continuum of constraints.
For example,
the constraints we meet might be time-evolving or subject to uncertain parameters, which
cannot be formulated as an ordinary CMDP
(see Examples \ref{Example_Time_Evolving} and  \ref{Example_Uncertain}).
In this paper we would study a generalized CMDP  
to address the above problem.  Because the constraints are not only infinite-number but also lie
in a continuous set,
the generalization is not trivial. Fortunately, we find that we can borrow the idea behind semi-infinite programming (SIP) \citep{remez1934determination, hettich1993semi} to deal with the semi-infinite constraints.
Accordingly, we propose \emph{semi-infinitely constrained Markov decision processes} (SICMDPs)
as a novel complement to the ordinary CMDP framework.
%More specifically,  an SICMDP model %, we consider 
%contains a continuum of constraints whereas an ordinary CMDP contains a finite number of constraints. 

%This generalization is natural but not trivial. However, we can brows the idea  
%The idea is quite natural and can be backtracked
%to the practice of extending linear programming to linear semi-infinite programming (LSIP) %\cite{remez1934determination, GobernaLSIO1998}.
%In addition, 
%As a complementary approach to the ordinary CMDP framework, 
%SICMDP can be used to model these problems  which cannot be described by a finite number of constraints
%that are not covered by .
%For example,
%the restrictions we consider can be time-evolving or subject to uncertain parameters
%, thus
%cannot be described by a finite number of constraints but a continuum of constraints 
%(see Examples \ref{Example_Time_Evolving} and  \ref{Example_Uncertain}).

We also present two reinforcement learning algorithms to solve SICMDPs called SI-CRL and SI-CPO, respectively.
SI-CRL is a model-based reinforcement learning algorithm designed for tabular cases, and SI-CPO is a policy optimization algorithm for non-tabular cases.
% and analyze its performance both theoretically and empirically.
The main challenge is that we need to deal with a continuum of constraints, thus reinforcement learning algorithms for ordinary CMDPs do not work anymore.
In SI-CRL, we tackle this difficulty by first transforming the reinforcement learning problem to an equivalent LSIP problem, which can then be solved using methods in the LSIP literature like the dual exchange methods \citep{Hu1990,reemtsen1998numerical}.
In SI-CPO, we resort to the idea of cooperative stochastic approximation developed in \cite{lan2020algorithms, wei2020comirror}.
As far as we know, we are the first to introduce tools from semi-infinitely programming (SIP) into the reinforcement learning community for solving constrained reinforcement learning problems.

% To the best of our knowledge, we are the first to apply tools from semi-infinitely programming (SIP) to solve reinforcement learning problems.
Furthermore, we give theoretical analysis for both SI-CRL and SI-CPO.
We decompose the error of SI-CRL into two parts: the statistical error from approximating the true SICMDP with an offline dataset and the optimization error due to the fact that the solution of the LSIP problem obtained by the dual exchange method is inexact.
On the optimization side, we show that the iteration complexity of SI-CRL is $O\left(\left\{\mathrm{diam}(Y)L\sqrt{|\gS|^2|\gA|m}/\left[(1-\gamma)\epsilon\right]\right\}^m\right)$.
On the statistical side, we show that the sample complexity of SI-CRL is $\widetilde O\left(\frac{|S|^2|A|^2}{\epsilon^2(1-\gamma)^3}\right)$ if the offline dataset is generated by a generative model, and $\widetilde O\left(\frac{|S||A|}{\nu_{\min} \epsilon^2(1-\gamma)^3}\right)$ if the dataset is generated by a probability measure $\nu$ as considered in \cite{chen2019information}.
Here $\widetilde O$ means that all logarithm terms are discarded.
For SI-CPO, things become a little more complicated because other than the statistical error and the optimization error, we also need to consider the function approximation error, which comes from imperfect policy parametrizations.
It is shown if the function approximation error can be controlled to $O(\epsilon)$ order, the iteration complexity of SI-CPO is $\widetilde{O}\left(\frac{1}{\epsilon^2(1-\gamma)^6}\right)$ and the sample complexity of SI-CPO is $\widetilde{O}(\frac{1}{\epsilon^4(1-\gamma)^{10}})$.
Here our iteration complexity bound is equivalent to a typical $\widetilde O(1/\sqrt{T})$ global convergence rate.

We perform a set of numerical experiments to illustrate the SICMDP model and validate our proposed algorithms.
Specifically, we examine two numerical examples, namely the discharge of sewage and ship route planning.
Through the discharge of sewage example, we show the advantage of the SICMDP framework over the CMDP baseline obtained by naive discretization in modeling realistic sequential decision-making problems.
Moreover, we demonstrate the effectiveness of the SI-CRL and SI-CPO algorithms in such tabular environments. 
In the ship route planning example, we illustrate the benefits of the SICMDP framework and the ability of the SI-CPO algorithm to address complex continuous control tasks involving continuous state spaces with modern deep reinforcement learning techniques.

% In summary, our contributions are listed as follows.
% First, we present the SICMDP model, which can be viewed as a generalization of the ordinary CMDP model.
% Second, we propose an algorithm to perform reinforcement learning for SICMDPs, which is called SI-CRL, and we believe that we are the first to apply tools from SIP
% to solve reinforcement learning problems.
% Third, we give a theoretical analysis of SI-CRL and identify both its sample complexity and iteration complexity.
% In addition, we perform numerical experiments to illustrate the SICMDP model and validate the SI-CRL algorithm.
% \{This paragraph can be removed!!! \}






\section{Method}\label{sec:method}
The proposed segmentation-by-detection framework, as depicted in Figure \ref{fig:framework}, consists of a detection module and a segmentation module.
In detection stage, 2D slices (layered box) from the input volume are fed to the RPN. Based on the region proposals obtained from RPN, an attention model (block in orange) is formed. The input volume as well as the attention model are further processed in segmentation stage to get the refined anatomical segmentation. 
\vspace{1em} 

\begin{figure}[t]
\centering
\includegraphics[width=0.95\linewidth]{fig/framework.pdf}
\caption{Schematic representation of the segmentation-by-detection framework. The left part is the detection module while the segmentation module is followed on the right. The blue block denotes the input volume which is 3D ultrasound scan of femoral head. The output segmentation is in red.}
\label{fig:framework}
\end{figure}
% dana could you improve the figure. we can try to think together of better ways 

\noindent\textbf{Detection Module:} 
% dana : here you have to make the clarification that you have ground truth on the boxes (in implementation part)
The detection module follows an RPN architecture, a fully convolutional network which takes image slice as input and outputs object region candidates. 
We use the VGG-16 model as the backbone \cite{simonyan2014very} to learn convolutional features and an $3 \times 3$ spatial window to generate region proposals. At each sliding-window location, 9 anchors are predicted associated with different scales and aspect ratios. The last layer consists of a box-regression (reg) layer and a box-classification (cls) layer in parallel. The reg layer outputs 4 regression offsets, $ t = (t_x,t_y,t_w,t_h)$, denoting a scale-invariant translation as well as log-space height and width shift, where $x,y,w$ and $h$ specify two coordinates of the box center, width and height. The cls layer outputs two scores by softmax, related to probabilities of object and background for each proposal. We assign a positive label (of being object) to candidate which has an Intersection-over-Union (IoU) ratio higher than 0.7 with ground truth box. Note that an image slice may contain multiple object regions or none. 

The loss function of RPN follows the multi-task loss \cite{ren2015faster} which is defined as $L = L_{reg} + L_{cls}$. The regression loss, $L_{reg} = -\log p_{obj}$ is log loss and the classification loss,
\begin{equation} \label{eq:loss}
L_{cls} = \sum_{i \in \{x,y,w,h\}} smooth_{L_1} (t_i - t_i^*)
\end{equation}
is smooth $L_1$ loss where $t_i^*$ denotes the ground truth box for the target object. 
\vspace{1em}

\noindent\textbf{Segmentation Module:}
3D U-Net \cite{cciccek20163d} is utilized in the segmentation module as its outstanding performance in medical image segmentation. The u-shaped architecture consists of two paths: a contracting path, where each layer contains two $3\times3\times3$ convolutions followed by a rectified linear unit (ReLU) and then a max pooling, provides high resolution features. While, the symmetric expanding path for semantically richer features replaces max pooling with a upconvolution $2\times2\times2$ with stride of 2 in each dimension, and then two $3\times3\times3$ convolutions each followed by a ReLU. Skip connections between layers of equal resolution in the contracting path and the expanding path enables context information as well as precise localization.

Different from 3D U-Net, to incorporate the attention model detected by the RPN, our architecture takes as input both the volumetric image data and the candidate RoIs proposed by the RPN, concatenated as 3D volume. 
% dana not sure what you like to say below
% densely annotated
The attention model makes the network to focus on the potential RoIs and can reduce the interference of the surrounding noise.
The anatomical segmentation is then generated from a $1\times1\times1$ convolution which reduces the number of feature maps to the number of labels.  The energy function is computed by a pixel-wise softmax combined with the cross entropy loss.
% dana equation ??

\subsection{System and implementation Details}
The segmentation-by-detection approach adopts a cascade structure with two stages: detection and segmentation. The two networks are trained separately in an end-to-end manner. All the new layers are randomly initialized from zero-mean Gaussian distribution with standard deviations 0.01. Biases are initialized to 0. We use Caffe \cite{jia2014caffe} for the implementation and an NVIDIA Titan X GPU for training.

In the detection stage, we initialize the VGG-16 model by the pre-trained model for ImageNet classification \cite{russakovsky2015imagenet} and further fine-tune the model for our detection task. The input fed to the network are image slices with a fixed size of $184\times96$ and the corresponding ground truth boxes are generated from the annotation in the format of tight bounding boxes surrounding the segmentation contour (as illustrated in Figure \ref{fig:hip} (b), the boundary of white area). To optimize the energy function, stochastic gradient descent (SGD) is used. The global learning rate is set to 0.001, while a momentum of 0.9 and a weight decay of 0.0005 are used. The batch size is set to 256 and each mini-batch only contains the positive anchors for training. The region proposals are obtained from the reg path for each image slice. The attention model is then formed by concatenating all the detected regions, as binary masks, into a volume.

In the segmentation stage, we use the Adam optimizer \cite{kingma2014adam} to learn the network parameters. A global learning rate is set to 0.001 while the two momentum coefficients are set to 0.9 and 0.999 respectively. A batch size of 1 is used due to the memory constraints of the GPU. The network takes the volume data as well as the attention model as input. We train the network for a maximum of 30K iterations and reserve the learned weights with the best performance from every 1K iterations. 
\vspace{1em}

\noindent\textbf{Inference:}
At test time, the 2D slices from an input volume are first fed to the detection module. The attention model is obtained based on the output. Then the volume data as well as the attention model are fed to the segmentation module to get the pixel-wise prediction.




\section{Experiments}\label{sec:exp}

\section{Experiments}\label{sec:experiments}
We validate our approach using multiple datasets containing real-life data from the fields of criminal risk assessment, credit, lending, and college admissions. In each of the datasets we select a binary feature and treat it as the protected attribute (e.g., race or gender), which is the feature we require our trained classifier to behave fairly upon. Our proposed method performs well on all of these datasets, succeeding in removing unfairness almost entirely, at a very modest price in terms of accuracy.


\begin{table*}[h]
\centering
\resizebox{\textwidth}{!}{
\def\arraystretch{1.2}

\begin{tabular}{c c c | c | c | c || c | c | c || c | c | c |}

\cline{4-12}
&&&
\multicolumn{9}{ c| }{\textbf{COMPAS Dataset}}
\\ \cline{4-12}
&&&
\multicolumn{3}{ c|| }{\textbf{FPR Considerations}}&
\multicolumn{3}{ c|| }{\textbf{FNR Considerations}}&
\multicolumn{3}{ c| }{\textbf{Both Considerations}}
\\ \cline{4-12}
&&&
 $\mathbf{Acc.}$ &  $\mathbf{D_{FPR}}$ &  $\mathbf{D_{FNR}}$ &  $\mathbf{Acc.}$ &  $\mathbf{D_{FPR}}$ &  $\mathbf{D_{FNR}}$ &  $\mathbf{Acc.}$ &  $\mathbf{D_{FPR}}$ &  $\mathbf{D_{FNR}}$
\\  \cline{4-12}
\vspace*{-0.5ex}
\\ \cline{1-2} \cline{4-12}
\multicolumn{1}{ |c  }{} &
\multicolumn{1}{ c|  }{  \textbf{Our Method (AVD Penalizers)}}  &&
$\mathbf{0.660}$    &  $\mathbf{0.01}$  &  $0.04$ &
$\mathbf{0.653}$    &  $0.02$   &  $\mathbf{0.04}$ &
$\mathbf{0.654}$    &  $\mathbf{0.02}$  &  $\mathbf{0.04}$
\\ \cline{1-2} \cline{4-12}
\multicolumn{1}{ |c  }{} &
\multicolumn{1}{ c|  }{  \textbf{Our Method (SD Penalizers)}}  &&
$\mathbf{0.664}$    &  $\mathbf{0.02}$  &  $0.09$ &
$\mathbf{0.661}$    &  $0.05$   &  $\mathbf{0.03}$ &
$\mathbf{0.661}$    &  $\mathbf{0.02}$  &  $\mathbf{0.03}$
\\ \cline{1-2} \cline{4-12}
\multicolumn{1}{ |c  }{} &
\multicolumn{1}{ c|  }{  Zafar et al.~(\citeyear{disparatemistreatment})}  &&
$0.660$    &   $0.06$    &   $0.14$  &
$0.662$    &   $0.03$    &   $0.10$  &
$0.661$    &   $0.03$    &   $0.11$
\\ \cline{1-2} \cline{4-12}
\multicolumn{1}{ |c  }{} &
\multicolumn{1}{ c|  }{  Zafar et al. Baseline~(\citeyear{disparatemistreatment})}  &&
$0.643$    &   $0.03$    &   $0.11$  &
$0.660$    &   $0.00$    &   $0.07$  &
$0.660$    &   $0.01$    &   $0.09$
\\ \cline{1-2} \cline{4-12}
\multicolumn{1}{ |c  }{} &
\multicolumn{1}{ c|  }{  Hardt et al.~(\citeyear{hardt})}  &&
$0.659$    &  $0.02$    &   $0.08$  &
$0.653$    &  $0.06$   &    $0.01$  &
$0.645$    &  $0.01$   &    $0.01$
\\ \cline{1-2} \cline{4-12}
\multicolumn{1}{ |c  }{} &
\multicolumn{1}{ c|  }{  \textbf{Vanilla Regularized Logistic Regression}}  &&
$\mathbf{0.672}$    &   $\mathbf{0.20}$    &   $\mathbf{0.30}$  &
$\mathbf{0.672}$    &   $\mathbf{0.20}$    &   $\mathbf{0.30}$  &
$\mathbf{0.672}$    &   $\mathbf{0.20}$    &   $\mathbf{0.30}$
\\ \cline{1-2} \cline{4-12}
\end{tabular}
}
\vspace{3mm}
\caption{Performance comparison on the COMPAS dataset. For the approaches in bold -- Accuracy, FPR difference and FNR difference are evaluated on the test set, averaging over five runs and using a 70-30 training/test split. The performance of the remaining three approaches is stated as reported in Zafar et al.~(\citeyear{disparatemistreatment}).} \label{table:comparison_results}
\end{table*}



\begin{figure*}[b]
  \includegraphics[scale=0.6]{compas0-400.png}
  \caption{COMPAS Dataset. Accuracy, FPR difference ($\mathbf{D_{FPR}}$), and FNR difference ($\mathbf{D_{FNR}}$) (all evaluated on the test set) of the learned classifier, as a function of the weight $c=c_1 = c_2 \geq 0$ placed on the fairness penalizer terms. On the left we use the Absolute Value Difference (AVD) penalizer, and the Squared Difference (SD) penalizer on the right, both as presented in Section~\ref{regularization}. ``Relaxed FPR/FNR Diff.'' plots the value of the relevant penalization term.} %In this particular run, parameters chosen for the absolute value relaxation were: $c=80, q_c=60$, and for the squared relaxation: $c=220, q_c=30$.}
  \label{fig:compas}
\end{figure*}


\subsection{Implementation}
\textbf{Our method} 
%We instantiate our method in the following way: Given dataset $Q$, we split it randomly into a training set $S$ (which we will use for learning) and a test set $T$ (which we will only use for reporting performance). 
For the purpose of comparison with  Zafar et al.~(\citeyear{disparatemistreatment}) and Hardt et al.~\cite{hardt} on the COMPAS data, we use a parameter $c$ to induce three possible combinations of weights on the FPR and FNR penalization terms: $c = c_1$ and $c_2 = 0$; $c_1 = 0$ and $c = c_2$; and $c = c_1 = c_2$. For the other three datasets, we consider only $c = c_1 = c_2$.\footnote{The reason for varying the values of $c$ in the training phase is since we shifted to a proxy problem, in which we rely on the distance from the decision boundary rather the actual classifications. 
%Our hope is that there is no need for a worst-case cross validation between all of the combinations of $c_1, c_2, c_3$, and that the training scheme we propose is sufficient. 
It is possible, of course, that even better results are attainable using our scheme with other combinations of $c_1, c_2$, and $q$.} To explore the accuracy/fairness trade-off curve for the relaxed optimization problem~(\ref{eq:2}), we train for different values of $c$, starting at $c=0$ (which is just standard logistic regression), and growing gradually.



Given a dataset $Q$ and fixing a $d_1, d_2 \in \{0, 1\}$ of interest, we use the following training scheme:
\begin{enumerate}
\item Split $Q$ at random into training set $S$ and test set $T$.
\item For each $c$, perform cross-validation on $S$ to select the corresponding best value $q_c$ for the regularization parameter.
\item For each $(c,q_c)$, let $\theta_c = \argmin\limits_{\theta} \text{Proxy}(\theta;S,c,c,q_c)$.
\item Select $\theta^* \in \argmin\limits_{\theta_c} \text{Objective}(\theta_c;S,d_1,d_2)$.
\item Evaluate performance using $\theta^*$ on test set $T$.
\end{enumerate}
We report the average of five such runs, each with a fresh training-test split.




%We instantiate our method by solving the relaxed optimization problem~(\ref{eq:2}), in place of the original, non-convex problem~(\ref{eq:1}).  
%We test our approach with three different combinations of weights on the penalization terms:
%\katrina{What are the $d$, and how are they related to the $c$s?}
%\begin{enumerate}
%\item FPR considerations only: $d_1 = 1, d_2 = 0$.
%\item FNR considerations only: $d_1 = 0, d_2 = 1$.
%\item Both FPR, FNR considerations, assigned similar significance: $d_1 = 1, d_2 = 1$.
%\end{enumerate}
%One could, of course, pick any other combination of the FPR and FNR penalty weights.

%\katrina{I don't understand how the below is distinct from the list above}
%Learning is done by training the parameters of a logistic regressor to solve~\ref{eq:2}, while picking the value of $c_1, %c_2$ as the following:
%\begin{enumerate}
%\item FPR considerations only: $c_1 = c \geq 0$, $c_2 = 0$.
%\item FNR considerations only: $c_1 = 0$, $c_2 = c \geq 0$.
%\item Both FPR, FNR considerations, assigned similar significance: $c_1 = c_2 = c \geq 0$
%\end{enumerate}



% We then cross-validate to pick the best $c_3$ (the weight on the standard $\ell_2$-regularization term) given $c$.\footnote{The reason for varying the values of $c$ in the training phase is since we shifted to a proxy problem, in which we rely on the distance from the decision boundary rather the actual classifications. 
%Our hope is that there is no need for a worst-case cross validation between all of the combinations of $c_1, c_2, c_3$, and that the training scheme we propose is sufficient. 
%It is possible, of course, that even better results are attainable using our scheme with other combinations of $c_1, c_2, c_3$.} For each such combination, we report results as the averages of multiple \katrina{how many?} different runs, each time splitting data randomly into training and test sets.
%\yahav{We need to shorten this description.}

We solve the relaxed convex optimization problem using the CVXPY solver. Due to stability issues with large training sets, we use a train/test split of 30-70 on the larger datasets, rather than 70-30 as on the COMPAS dataset\footnote{The code implementing our method can be found at https://github.com/jjgold012/lab-project-fairness}.

%
%
%We then report the results (as evaluated on the test set) attained by a regressor $\theta \in \mathbb{R}^d$ that minimizes (on the training set $S$) a weighted combination of the $0$-$1$ loss and the differences in FPR and FNR across populations:
%\begin{equation*}
%\begin{aligned}
%&\underset{\theta}{\text{argmin}}
%& & L_{S}^{0\text{-}1}(\theta) \\
%&&& + d_1|FPR_{A=0}(\theta;S)-FPR_{A=1}(\theta;S)| \\
%&&& + d_2|FNR_{A=0}(\theta;S)-FNR_{A=1}(\theta;S)|
%\end{aligned}
%\end{equation*}
%
%\katrina{What is $d_1$ vs. $c_1$ etc.?}



%For classification, we decided use a standard cut-off threshold of $c=0.5$. There are of course, further possible interactions between the FPR, FNR considerations, and picking a certain cut-off level. These are not straightforward, since  these interactions are data-specific. 



%allows for flexibility in picking the values of $c_1, c_2$, which reflect the significance we wish to place on the objectives of achieving accuracy, equal FPR, and equal FNR. As for $c_3$, we will want to find the value of it that achieves the best results, for any combined objective of accuracy and fairness defined by a specific selection of $c_1,c_2$. Therefore, given a specific selection of $c_1, c_2$, we apply cross-validation to select the value of $c_3$. 




We briefly describe the other algorithmic approaches to which we compare:\\
\textbf{Zafar et al.}~(\citeyear{disparatemistreatment}) performs optimization by considering a proxy for the bias: the covariance between the samples' sensitive attributes and the signed distance between the feature vectors of misclassified users and the classifier decision boundary.\\
\textbf{Zafar et al. Baseline}~(\citeyear{disparatemistreatment}) tries to enforce equal FP/FN rates on the different groups by introducing different penalties for misclassified data points with different sensitive attribute values during the training phase.\\
\textbf{Hardt et al.}~(\citeyear{hardt}) performs post-processing on a standard trained (unfair) logistic regressor, picking different decision thresholds for different groups, and possibly adding randomization.


\subsection{Experimental Results}

In what follows, we use the following notation, given a trained classifier $\hat{Y}$:
\begin{align*}
\mathbf{D_{FPR}}&=\left|FPR_{A=0}(\hat{Y})-FPR_{A=1}(\hat{Y})\right| \\ 
\mathbf{D_{FNR}}&=\left|FNR_{A=0}(\hat{Y})-FNR_{A=1}(\hat{Y})\right|
\end{align*}
The values $FPR_{A=0}(\hat{Y})$, $FPR_{A=1}(\hat{Y})$, $FNR_{A=0}(\hat{Y})$, $FNR_{A=1}(\hat{Y})$ are reported as evaluated on the test set.

\paragraph{The COMPAS Dataset\footnote{https://github.com/propublica/compas-analysis}} The Correctional Offender Management Profiling for Alternative Sanctions (COMPAS) records from Broward County, Florida 2013-2014, made available online by ProPublica, are perhaps the best-studied data in the context of fairness.  The goal in this scenario is to successfully predict recidivism within two years, based on features such as age, gender, race, number of prior offenses, and charge degree. The dataset contains 5,278 samples. The protected attribute in this scenario is race, where $A$ indicates black or white. We filtered the dataset using the same features as Zafar et al.~(\citeyear{disparatemistreatment}), to allow for comparison.

%\begin{table}[h]
%\centering
%\begin{tabularx}{\columnwidth}{c|c|c|c}
%\hline
%  &  Recid. ($y = 1$)        & No Recid.  ($y = 0$)       & Total \\ \hline
%Black &  $ 1661   $ & $ 1514 $ &  $ 3175 $ \\ \hline
%White &  $ 822   $  & $1281  $ &  $ 2103 $ \\ \hline
%Total &  $ 2483  $  & $2795 $ &  $ 5278 $ \\\hline
%\end{tabularx}
%\caption{Statistics of the ProPublica COMPAS data.} \label{table:compas-stats}
%\label{tab:stats}
%\end{table}
%\vspace{-1em}

%\begin{table}[h]
%\centering
%\begin{tabularx}{\columnwidth}{c|c}
%\hline
%Feature  &  Description \\ \hline
%Age Category &  $<25$, between $25$ and $45$, $>45$ \\
%Gender &  Male or Female \\
%Race &  White or Black \\
%Priors Count &  0--37 \\
%Charge Degree &  Misconduct or Felony \\
%\hline
%2-year-recid. & Whether or not the  \\
%(target feature)  & defendant recidivated within two years
%\end{tabularx}
%\caption{Description of features used from ProPublica COMPAS data.} \label{table:compas-features}
%\label{tab:features}
%\end{table}




\begin{table*}[t]
\centering
\caption{A description of the datasets used, along with parameters of the training procedure used for each.}
\label{table:datasets_description}
\begin{adjustbox}{max width=\textwidth}
\begin{tabular}{|l|l|l|l|l|l|l|l|}
\hline
\textbf{Dataset} & \textbf{No. Samples} & \textbf{No. Features} & \textbf{Train/Test Split} & \textbf{No. Repetitions} & \textbf{No. Folds in CV} & \textbf{Protected Feature} & \textbf{Target Variable} \\ \hline
COMPAS           & 5,278                     & 5                          & 70-30                     & 5                        & 5                                 & Race                       & 2-Year-Recidivism        \\ \hline
Adult            & 30,162                    & 10                         & 30-70                     & 5                        & 5                                 & Gender                     & Income Over/Under 50K    \\ \hline
Default          & 30,000                    & 23                         & 30-70                     & 5                        & 3                                 & Gender                     & Defaulting On Payments   \\ \hline
Admissions       & 20,839                    & 17                         & 30-70                     & 5                        & 3                                 & Race                       & Passing Bar Exam         \\ \hline
\end{tabular}
\end{adjustbox}
\end{table*}


\begin{table*}[t]
\centering
\resizebox{\textwidth}{!}{
\def\arraystretch{1.2}

\begin{tabular}{c c c | c | c | c || c | c | c || c | c | c |}

\cline{4-12}
&&&
\multicolumn{3}{ c|| }{\textbf{Adult Dataset}}&
\multicolumn{3}{ c|| }{\textbf{Default Dataset}}&
\multicolumn{3}{ c| }{\textbf{Admissions Dataset}}
\\ \cline{4-12}
%&&&
%\multicolumn{3}{ c|| }{\textbf{Both Considerations}}&
%\multicolumn{3}{ c|| }{\textbf{Both Considerations}}&
%\multicolumn{3}{ c| }{\textbf{Both Considerations}}
%\\ \cline{4-12}
&&&
 $\mathbf{Acc.}$ &  $\mathbf{D_{FPR}}$ &  $\mathbf{D_{FNR}}$ &  $\mathbf{Acc.}$ &  $\mathbf{D_{FPR}}$ &  $\mathbf{D_{FNR}}$ &  $\mathbf{Acc.}$ &  $\mathbf{D_{FPR}}$ &  $\mathbf{D_{FNR}}$
\\  \cline{4-12}
\vspace*{-0.5ex}
\\ \cline{1-2} \cline{4-12}
\multicolumn{1}{ |c  }{} &
\multicolumn{1}{ c|  }{  \textbf{Our Method (AVD Penalizers)}}  &&
$\mathbf{0.776}$    &  $\mathbf{0.00}$  &  $\mathbf{0.04}$ &
$\mathbf{0.807}$    &  $\mathbf{0.00}$   &  $\mathbf{0.01}$ &
$\mathbf{0.950}$    &  $\mathbf{0.01}$  &  $\mathbf{0.00}$
\\ \cline{1-2} \cline{4-12}
\multicolumn{1}{ |c  }{} &
\multicolumn{1}{ c|  }{  \textbf{Our Method (SD Penalizers)}}  &&
$\mathbf{0.783}$    &  $\mathbf{0.00}$  &  $\mathbf{0.09}$ &
$\mathbf{0.806}$    &  $\mathbf{0.01}$   &  $\mathbf{0.02}$ &
$\mathbf{0.950}$    &  $\mathbf{0.00}$  &  $\mathbf{0.00}$
\\ \cline{1-2} \cline{4-12}
\multicolumn{1}{ |c  }{} &
\multicolumn{1}{ c|  }{  \textbf{Vanilla Regularized Logistic Regression}}  &&
$\mathbf{0.800}$    &   $\mathbf{0.08}$    &   $\mathbf{0.39}$  &
$\mathbf{0.807}$    &   $\mathbf{0.01}$    &   $\mathbf{0.05}$  &
$\mathbf{0.951}$    &   $\mathbf{0.16}$    &   $\mathbf{0.02}$
\\ \cline{1-2} \cline{4-12}
\end{tabular}
}
\vspace{3mm}
\caption{Performance on the Adult, Loan Default, and Admissions datasets, penalizing for both FPR and FNR difference. Accuracy, FPR difference and FNR difference are evaluated on the test set, averaging over five runs and using a 30-70 training/test split.} \label{table:comparison_results_rest}
\end{table*}


In Table~\ref{table:comparison_results}, we compare the performance of our approach with that of three other techniques from the literature. Each method was trained based on logistic regression.  As a basis for comparison, we also present the performance of vanilla logistic regression, absent fairness considerations, with the regularization parameter selected via cross-validation.\footnote{Zafar et al.~(\citeyear{disparatemistreatment}) do not incorporate regularization in any of the approaches they report.}
%Results are reported as the averages of 5 different runs \katrina{Is that still correct?}, each time splitting data evenly and randomly into training and test sets. 
Results for Zafar et al., Zafar et al. baseline, and Hardt et al. appear here as reported in Zafar et al.~(\citeyear{disparatemistreatment}).\footnote{Our method selects the classifier based on the training set only and reports its performance over the test set. Results for the three other approaches, reported by Zafar et al.~(\citeyear{disparatemistreatment}), are based on tuning parameters after seeing the trade-off curve over the test set, and reporting according to the best selection of these parameters.}
%\katrina{Perhaps here is the right place for a footnote about the discrepancy with the Zafar baseline}

We find that the vanilla logistic regressor (absent fairness considerations) results in significant unfairness, as $\mathbf{D_{FPR}}=0.20$, and $\mathbf{D_{FNR}}=0.30$. The overall accuracy of this classifier measured on the test set was $0.672$.\footnote{Zafar et al.~(\citeyear{disparatemistreatment}) report a slightly different baseline of: Accuracy = 0.668, $\mathbf{D_{FPR}}=0.18$, $\mathbf{D_{FNR}}=0.30$.} Our SD penalization approach empirically achieves approximately the same accuracy as the Zafar et al.~(\citeyear{disparatemistreatment}) approach, with significantly better fairness. It is difficult to compare fairness-accuracy tradeoffs with the Hardt et al.~(\citeyear{hardt}) approach, since their accuracy is significantly lower than ours. A more direct comparison is possible by noting that our learned classifier can be post-processed to improve its fairness at a direct cost to accuracy. Hence, we can achieve accuracy of $0.659$ with $\mathbf{D_{FPR}} = \mathbf{D_{FNR}} = 0.01$, which compares very favorably with the Hardt et al. accuracy rate of 0.645 given the same FPR and FNR rates.\footnote{For completeness, we note that using a 50-50 training-test split (again not using the test set for parameter selection), our method (SD, both considerations) produces a classifier that provides: Accuracy = 0.659, $\mathbf{D_{FPR}} = 0.01, \mathbf{D_{FNR}} = 0.05$. This classifier can be post-processed to achieve rates of: Accuracy = 0.655, $\mathbf{D_{FPR}} = \mathbf{D_{FNR}} = 0.01$.}

Figure \ref{fig:compas} illustrates the accuracy/fairness trade-offs achievable using our scheme. Increasing the weight $c$ on the proxy fairness penalizers results in reducing their magnitude. The figure also illustrates how our relaxed penalizers succeed in tracking the real FPR and FNR differences. 
%
%
%\katrina{Must rewrite the following paragraph}
%We observe that our method succeeds in eliminating unfairness almost completely on the COMPAS dataset, while retaining most of the accuracy, when compared to the vanilla logistic regression. We achieve very low difference rates when penalizing for achieving each of the FPR and FNR criteria individually, and also for both. We achieve preferable results comparing to Zafar et al. and Zafar et al. baseline in all 3 scenarios, and also comparing to Hardt et al. in the settings of false positive/false negative considerations only. In the setting of both considerations - The Hardt et al. method removes a larger portion of the unfairness, however it results in major accuracy loss as it achieves accuracy rate of 0.645 in comparison to our method which results in accuracy of 0.665, retaining most of the original accuracy rate while removing most of the unfairness.




%The Hardt et al.~\cite{hardt} approach as reported removes a smaller portion of the bias in the different scenarios, however for FP/FN constraints alone, it provides higher accuracy rates. The Zafar et al.~(\citeyear{disparatemistreatment}) approach as reported retains significant bias (in most cases), but in some cases  achieves slightly superior accuracy rates to the methods above. 

%These performance comparisons are incomplete in the sense that each of the compared techniques has the potential to trade off between accuracy and fairness, using some degree of parameter tuning; what we report here is only one point on the achievable trade-off frontier for each algorithm. The ``correct'' trade-off, and, in particular, the best manner in which to weigh unfairness in the FPR against unfairness in the FNR, are matters of opinion. We have chosen to report our method's performance under parameters designed to very aggressively mitigate unfairness, at some cost to the accuracy.

%It would certainly be desirable to evaluate these and other approaches to fair learning on other datasets and on different tasks, particularly on larger datasets, which might afford both greater accuracy and better bias-reduction. The present empirical evaluations, however, suggest that our regularization-based approach provides a new tool worthy of consideration---we succeed in almost entirely eliminating bias on the hold-out set, at a modest price in terms of accuracy.

%Due to the fact that our true objective includes the original non-convex penalization terms, our approach does not carry any formal guarantees. However, the ease of implementation, generality, and empirical results are encouraging. Figure~\ref{fig:test1} illustrates the rate of convergence to a fair, accurate classifier on this dataset.
%In terms of computation costs, given that at each iteration we must calculate the gradient according to the FPR and FNR regularizers, we are required to predict the labels for the entire training set at each step. 
%However, this does not pose a computational burden, as it is already required by the (classic) gradient descent algorithm in our logistic regressor fitting scheme. Furthermore, when given a sufficiently large dataset (one or two orders of magnitude larger than the one currently available for the COMPAS scores data), this could be relaxed to sampling only a mini-batch of samples from the training data set at each iteration (much as is done in stochastic gradient descent).






\subsection{Additional Datasets}


Table~\ref{table:datasets_description} provides summary statistics on each of the datasets on which we tested our approach. We also briefly describe the datasets below. 


{\bf The Adult Dataset}\footnote{http://archive.ics.uci.edu/ml/datasets/Adult} is based on 1994 US Census data. The task we consider is to predict whether the income of each individual is over or under 50K dollars per year, based on features such as occupation, marital status, and education. The protected attribute selected in this task is gender. 

{\bf The Loan Default Dataset}\footnote{{\scriptsize https://archive.ics.uci.edu/ml/datasets/default+of+credit+card+clients}}
contains data regrading Taiwanese credit card users. The task we consider is to predict whether an individual will default on payments, based on features such as history of past payments, age, and the amount of given credit. The protected attribute is gender.

{\bf The Admissions Dataset}\footnote{http://www2.law.ucla.edu/sander/Systemic/Data.htm}
contains records of law school students who went on to take the bar exam. The task we consider is to predict whether a student will pass the exam based on features such as LSAT score, undergraduate GPA, and family income. The protected attribute is set to race.

Table~\ref{table:comparison_results_rest} describes the performance of our approach on these datasets, and Figures~\ref{fig:adult},~\ref{fig:default}, and~\ref{fig:lawschool} illustrate the fairness-accuracy trade-offs we achieve in each context. Overall, we see that unfairness is nearly eliminated while accuracy remains quite high. The dataset on which accuracy suffers most under our approach is the Adult dataset, which is also the dataset on which the vanilla regression is the most unfair.


\begin{figure*}[]
  \includegraphics[scale=0.6]{adult0-800.png}
  \caption{Adult Dataset. Fairness-Accuracy tradeoffs, as in Figure~\ref{fig:compas}.}
  \label{fig:adult}  
\end{figure*}



\begin{figure*}[]
  \includegraphics[scale=0.6]{default0-50.png}
  \caption{Loan Default Dataset. Fairness-Accuracy tradeoffs, as in Figure~\ref{fig:compas}.}
  \label{fig:default}
\end{figure*}



\begin{figure*}[]
  \includegraphics[scale=0.6]{admissions0-400.png}
  \caption{Admissions Dataset. Fairness-Accuracy tradeoffs, as in Figure~\ref{fig:compas}.}
  \label{fig:lawschool}
\end{figure*}




\section{Conclusion and Limitations}\label{sec:con}

\begin{comment}
\begin{figure}
\includegraphics[width=\linewidth]{figs/beyond_tss_lesion.pdf}
\caption[]{End-to-End runtime lesion study of the entire MNIST dataset and the FMA featurized music dataset. Each of DROP's contributions provides a runtime improvement.}
\label{fig:beyond_lesion}
\end{figure}
\end{comment}



\section{Conclusion}
\label{sec:conclusion}

Advanced data analytics techniques must scale to rising data volumes. 
DR techniques offer a powerful toolkit when processing these datasets, with PCA frequently outperforming popular techniques in exchange for high computational cost. 
In response, we propose DROP, a new dimensionality reduction optimizer. 
DROP combines progressive sampling, progress estimation, and online aggregation to identify high quality low dimensional bases via PCA without processing the entire dataset by balancing the runtime of downstream tasks and achieved dimensionality. 
Thus, DROP provides a first step in bridging the gap between quality and efficiency in end-to-end DR for downstream \red{analytics}. 

%We revisit canonical operators for time series dimensionality reduction and the measurement study of~\cite{keogh-study}, and show that PCA is more effective than popular alternatives in the data mining literature often by a margin of over $2\times$ on average on gold-standard time series benchmark data sets with respect to output data dimension. More surprisingly, we empirically demonstrate that a small number of samples are sufficient to accurately characterize directions of maximum variance and obtain a high-quality low-dimensional transformation.




\paragraph{Acknowledgments.} HB is supported by the EPSRC programme grant Visual AI EP/T028572/1.



\bibliographystyle{ieee_fullname}
\bibliography{ref}

\clearpage
\appendix
\setcounter{section}{0} 

\section*{Supplementary Material} In the supplementary material, we first discuss more implementation details in Section~\ref{sec:implem} and present additional experiments in Section~\ref{sec:add_exp}. Then we present some examples that were selected as `hard to verify' in the user study in Section~\ref{sec:user_study_ex}. We present other qualitative examples in Section~\ref{sec:qual_analysis2}. Finally, we discuss societal aspects and potential risks in Section~\ref{sec:risks}.


\section{Implementation Details} \label{sec:implem}
We elaborate on some implementation details of our framework. 
\begin{itemize}

\item \textbf{Sentence representation.}
We preprocessed the crawled captions to remove some artefacts (e.g., HTML tags). When using BERT+LSTM, we used the pre-trained `bert-base-uncased' model, whose dimension is 768. We set a maximum length of 150 tokens for the captions. Items (i.e., query captions, evidence captions, and entities) are padded to the maximum sequence length in this item's batch. When using the sentence transformer model, we used the `paraphrase-mpnet-base-v2' model\footnote{https://huggingface.co/sentence-transformers/paraphrase-mpnet-base-v2}. For both, we used the Hugging Face library\footnote{https://huggingface.co/}. We used the PyTorch framework\footnote{https://pytorch.org/} for all our experiments.

\item \textbf{Memory.}
The items in each memory (images, entities, and captions) are padded to the maximum number of evidence items in this memory's batch.

\item \textbf{CLIP.}
We used the pre-trained ViT-B/32 CLIP model\footnote{https://github.com/openai/CLIP}, where the text length is truncated at 77 tokens.

\item \textbf{Training details.}
When fine-tuning CLIP, we follow the implementation details in~\cite{luo2021newsclippings}, we used a learning rate of 5e-5 for
the linear classifier and 5e-7 for other layers of the CLIP model itself, in addition to using the Adam optimizer~\cite{kingma2014adam}. We used a batch size of 64 and trained the model for 100 epochs. For training \model{}, we used a batch size of 32, the Adam optimizer, and a cyclical learning rate~\cite{smith2017cyclical} with a maximum value of 6e-5. We trained the model for 30 epochs. We used a dropout~\cite{srivastava2014dropout} value of 0.05 to the input representations, 0.25 to domain embeddings, and 0.25 to the memory representations. Experiments were done on one NVIDIA A100 GPU. With precomputing the representations, the training takes roughly 5 hours. When training using BERT without precomputing, training takes roughly 30 hours.

\end{itemize}
\section{Additional Experiments} \label{sec:add_exp}

\paragraph{Evidence-only classification.} We examine whether claims (and consequently, the evidence) are having different characteristics (and thus, unwanted biases or naive give-aways) between pristine and falsified classes. The NewsCLIPpings dataset avoided linguistic biases in creating falsified examples by using real news \textbf{\textcolor{myblue}{captions}} mismatched with real news \textbf{\textcolor{myOrange}{images}}, instead of introducing manipulations in the captions. Also, to avoid text bias, each \textbf{\textcolor{myblue}{caption}} (and consequently, its \textbf{\textcolor{myOrange}{visual evidence}} in our dataset) appears twice (within the same split), once as pristine and once as falsified. Therefore, we hypothesize that the evidence websites for both classes are similar. To confirm, we ran an \textit{evidence-only} model, which achieved 53.4\% (\textit{basically chance level}), showing that \textit{reasoning against the query} is the distinguishing factor.

\begin{table}[!b]
\begin{center}
\vspace{-1mm}
\resizebox{0.65\linewidth}{!}{
\begin{tabular}{cccc}
\toprule
Conc. & Avg-pool & Max-pool & Multiply \\ \midrule
\textbf{83.9} &  82.46 & 82.48 & 77.1 \\ \bottomrule
\end{tabular}}
\end{center}
\vspace{-6mm}
\caption{Accuracy (\%) vs. aggregation strategies.}
\vspace{-3mm}
\label{rebuttal_tab:ablation1}
\end{table}

\paragraph{Additional ablation studies.} We include further experiments related to the fusion of the different components in our model (visual reasoning, textual reasoning, and CLIP). We tried a late fusion by having a separate classifier on top of each branch and aggregating the decision, however, this performed worse than the current intermediate fusion we employ. We also tried other strategies (\autoref{rebuttal_tab:ablation1}) to combine visual and textual memories before concatenating with CLIP, where we found that concatenation had the highest performance. 

Finally, we found that changing the dimension of the penultimate layer had a relatively small effect; e.g., increasing the dimension to 2048 increased the accuracy by 0.3 percentage points.

\section{User Study: `Hard to Verify' Examples} \label{sec:user_study_ex}
In Figure~\ref{tbl:qual_study_appendix}, we show some examples that were selected as `hard to verify' in the user study. This is possibly due to: 1) the \textbf{\textcolor{myblue}{captions}} could contain specific context information (e.g., locations such as \textit{`Denver'} or \textit{`Massachusetts'}) that is hard to verify with the \textbf{\textcolor{myOrange}{image}} alone, 2) the lack of \textbf{\textcolor{myblue}{textual}} evidence returned by the search \vcenteredinclude{figs/icon3.pdf}; the \textbf{\textcolor{myOrange}{images}} were not found by the inverse image search, so there are no \textbf{\textcolor{myblue}{captions/titles}} found. Moreover, the \textbf{\textcolor{myblue}{entities}} are generic descriptions of the \textbf{\textcolor{myOrange}{image}}, or not at all related (the first example). The performance of the model on these examples is possibly dependent on how similar the \textbf{\textcolor{myOrange}{visual}} evidence is to the query \textbf{\textcolor{myOrange}{image}} \vcenteredinclude{figs/icon4.pdf}. 

Another possible reason is having falsified pairs that are highly similar in context to the original ones (and, therefore, to the evidence as well). For instance, the last example shows a `hard to verify' falsified example (that was also misclassified by our model); the \textbf{\textcolor{myOrange}{image}} shows the same people mentioned in the \textbf{\textcolor{myblue}{caption}}, and thus, they also appeared in the \textbf{\textcolor{myOrange}{visual}} evidence. Additionally, the \textbf{\textcolor{myblue}{caption}} mentions the band name \textit{`One Direction'} that is also mentioned in the \textbf{\textcolor{myblue}{textual}} evidence, without strong contradictions. Meanwhile, the actual \textbf{\textcolor{myOrange}{image}} of this \textbf{\textcolor{myblue}{caption}} showed the band performing on a stage, however, this was not clearly emphasized by the \textbf{\textcolor{myblue}{caption}}; that is possibly why the \textbf{\textcolor{myOrange}{visual}} evidence is generic.

\section{Qualitative Examples} \label{sec:qual_analysis2}
In Figure~\ref{tbl:qual_appendix}, we show more qualitative examples. \model{} predicted many examples correctly despite not having a one-to-one matching with the evidence in the case of pristine examples and having close similarity to the evidence in the case of falsified examples. 

For instance, in the first three examples (pristine), we observed that the model highly attended to supporting evidence such as persons' and countries' names, topics, and events. Additionally, in the third example, we observed that the model prioritized the \textbf{\textcolor{myOrange}{image}} that is from the same scene and the evidence \textbf{\textcolor{myblue}{caption}} that contains a subset from the query \textbf{\textcolor{myblue}{caption}} (\textit{`soon to be a Trump International Hotel'}). 

The fourth and fifth examples (falsified) suggest that the model does not simply rely on having any similarity or overlap between the query and evidence in order to identify pristine examples. Despite having the same persons in the evidence, they were correctly predicted as falsified, possibly as they have contradicting location information and different scene details (e.g., lighting, stage setup, or colours), indicating a different context or event. The last falsified example also indicates that both \textbf{\textcolor{myblue}{textual}} and \textbf{\textcolor{myOrange}{visual}} evidence is helpful, as the evidence \textbf{\textcolor{myOrange}{images}} are clearly different from the falsified one (showing a different building and place). 

%As for the last example, it highlights one of the `hard to detect' examples in the dataset, even with the presence of evidence, as the falsified image is also showing a similar object \textit{`iPhone'} without a strongly different context. 
%\clearpage

\section{Limitations and Societal Aspects}  \label{sec:risks}
%Automating fact-checking can be beneficial to fight the spread of misinformation. 
Nowadays, with the spread and reliance on social media to digest and get updated with news, misinformation (e.g., on Twitter) can reach hundreds of millions of users~\cite{vo2020facts}. This crucially motivates the need to fact-check and verify the credibility of online content, especially during critical times such as a pandemic or political instabilities. On the other hand, manual fact-checking is usually time-consuming, needing from less than one hour to many days to verify a claim~\cite{thorne2018automated}. Therefore, automating fact-checking can be extremely beneficial to alleviate the burden upon fact-checkers and journalists. 

However, completely or overly relying on automated tools might give an unwanted sense of security and could have many dangerous consequences. These include the dangers of flagging many true examples as falsified due to the real-life class imbalance, and missing out challenging falsified examples that require more fine-grained and complex reasoning. In addition, a currently active and much-needed research direction in the textual domain shows that fact-verification models might be partially relying on dataset biases without in-depth understanding and reasoning~\cite{schuster2019towards}. They might also be brittle to complex claims that require multi-hop reasoning~\cite{hidey2020deseption}. Additionally, as facts are continuously evolving, we face the danger of relying on old retrieved evidence~\cite{schuster2021get} or even possibly outdated world knowledge that is implicitly stored in pre-trained language models during training~\cite{schuster2019towards}.

In addition to their inherent limitations in reasoning and interpretation, several works have shown that textual verifications models are also vulnerable to adversarial attacks~\cite{thorne2019evaluating}, such as inserting trigger words~\cite{atanasova2020generating}, introducing lexical variations~\cite{hidey2020deseption}, or paraphrasing\cite{thorne2019evaluating}. As we have a multi-modal task, our model might also be vulnerable to image-based adversarial attacks~\cite{goodfellow2014explaining}. %Beyond manipulating claims via adversarial attacks, adversaries can also poison the evidence and introduce items that lead to the required entailment. 
Another potential misuse scenario is using the fact-checking model as an adversarial filter in order to curate hard examples that might be misclassified by fact-checking models in general. 

As a conclusion, we believe that automating fact-checking is strongly beneficial and that there have been many encouraging advancements to improve and harden it in the textual domain and the multi-modal domain, as we propose. However, due to their limitations and vulnerabilities to active attacks and manipulation, they should be used to assist humans and speed up the process, while still keeping them in the loop to avoid such dangers and consequences. In this regard, in our framework, we show that the model can filter and select the most important evidence, which would enable quicker inspection of the evidence items. 


\section{SIMULATION RESULTS}
\label{sec:examples}
This section presents simulation results of the proposed method implemented on the unicycle model example.
Each semidefinite program was prepared using a custom software toolbox and the modeling tool YALMIP \cite{lofberg2004yalmip}.
The programs are run with commercial solver MOSEK on a machine with $1$ TB availabe memory. 

\subsection{FRS Computation}
We computed the FRS for a 3$^\text{rd}$ order Taylor-expanded Dubins car as the low-fidelity model $f_s$.
Trajectories produced by this model were tracked by the unicycle model from Equation \eqref{eq:big_dyn} as the high-fidelity model $f$.
The vehicle's representation as an initial distribution $X_0 \subset X_s$, was a rectangle of length $0.2$ [m] in $x$ and width $0.1$ [m] in $y$, at $0^\circ$ initial heading, and centered at $x=-0.75$ and $y=0$.
This is the same vehicle representation shown in all previous figures.

% The error function $g$, illustrated in Figure \ref{fig:error_dynamics}, was given by:
% \begin{equation}
% \label{eq:g_definition}
% g(t,x_s) = \begin{bmatrix}
% v_\text{err}\cdot(1 - \frac{1}{2}\theta^2)  \\
% v_\text{err}\cdot(\theta - \frac{1}{6}\theta^3) \\
% \dot{\theta}_\text{err}
% \end{bmatrix}
% \end{equation}
% where $v_\text{err} = (t-1)^2$ and $\dot{\theta}_\text{err} = (t-1)^4$.
We chose $\tau_\text{stop} = \tau_\text{plan} = 0.5$ [s], so $T = 1$ [s].
The stopping time can be seen in Figure \ref{fig:error_dynamics}. 
The FRS computation took 79 hours and used a maximum of 150 GB of memory 
%on a server with 1 TB of available memory and 18 processors each running at 1.2 GHz.

\subsection{Set Intersection and Trajectory Planning}

We used the precomputed FRS for safe trajectory planning in $1000$ simulated trials in MATLAB on the aforementioned machine.
For each trial, the vehicle began at the same initial location and heading, surrounded by $1-10$ randomized obstacles and a randomly-located goal to reach.
%If the planning time took more than $\tau_\text{plan}$, the simulation paused until the computation was complete. 
%In practice, if $\tau_\text{plan}$ was exceeded the vehicle could begin braking to ensure safety.
The vehicle's initial speed, and the desired speed to maintain for the duration of the trial, were randomly chosen between $0.25$ and $0.75$ [m/s].
% The trials ran in 12.7 hours.
% Prior to running these trials, several example trials were run on a laptop with a 2.3 GHz processor and 16 GB of RAM.
% The trials run on the server were individually no faster than running on the laptop, because the set intersection optimization is a single-core process that uses very little memory. 
% Therefore, the server did not provide any significant decrease in the implemented planning time.


Obstacles were represented as line segments between $0.1$ and $0.2$[m] in length, with random location and orientation.
The obstacles were always placed between the vehicle and the goal.
We checked for crashes conservatively for each trial, by inspecting if any obstacle was within a circle circumscribing the rectangular vehicle at any point of the vehicle's trajectory. 
Using this method, \emph{no crashes were detected in any trial}.
Out of all the trials, $82\%$ reached the goal, and $15\%$ performed an emergency braking maneuver (by setting $v_\text{des} = 0$). 
The remaining 3\% hit a simulation iteration limit.
Examples of the vehicle's path from a randomly-generated trial and from two constructed emergency braking cases are shown in Figure \ref{fig:example_trial}.


\begin{figure}
\centering
\includegraphics[width=1\columnwidth]{running_examples.pdf}
\caption{The top subplot shows an example result out of the $1000$ trials.
This trial used eight randomly-generated obstacles.
The vehicle begins on the left at $x = -0.75$ and reaches a randomly-generated goal near $(2.5, 0.5)$, plotted as a blue circle.
Every $\tau_\text{plan} = 0.5$[s], the vehicle replans its trajectory, shown by an asterisk plotted on the global trajectory in blue.
The bounding box of the vehicle at each planning step is shown as a grey rectangle. In the bottom-left subplot, an obstacle was constructed between the vehicle and the goal, forcing an emergency braking maneuver. In the bottom-right subplot, an obstacle was constructed with a hole that would allow the vehicle to pass, but the set intersection result is overly conservative, resulting in a braking maneuver.}
\label{fig:example_trial}
\end{figure}

Currently, our implementation cannot consistently achieve $\tau_\text{plan} = 0.5$ [s].
Consequently, instead of replanning and driving simultaneously, we pause time every 0.5 [s] of the simulation to guarantee that the vehicle can finish replanning.
In a physical implementation, if $\tau_\text{plan}$ is exceeded, then the vehicle must emergency brake; recall that a safe braking trajectory is always available.
As shown in Figure \ref{fig:planning_time_vs_Nobs}, $\tau_\text{plan}$ scales linearly with the number of obstacles.
%Methods for reducing the set intersection to meet $\tau_\text{plan}$ will be presented in future work.

\begin{figure}
\centering
\includegraphics[scale=0.45,trim={1cm 6cm 1cm 7cm},clip]{planning_time_vs_Nobs.pdf}
\caption{The mean set intersection time (top) and trajectory optimization time (bottom) versus the number of obstacles. Over the $1000$ trials, each number of obstacles from $1$ to $10$ was used for $100$ trials. Notice that set intersection takes up to $3$[s], and scales with the number of obstacles. On the other hand, the trajectory optimization takes around $80$ [ms] and has low correlation with number of obstacles.}
\label{fig:planning_time_vs_Nobs}
\end{figure}

% \begin{figure}
% \centering
% \includegraphics[scale=0.5,trim={1cm 8cm 1cm 8cm},clip]{example_trial_bluecar.pdf}
% \caption{An example result out of the 1000 trials.
% This trial used eight randomly-generated obstacles.
% The vehicle begins on the left at $x = -0.75$ and reaches a randomly-generated goal near $(2.5, 0.5)$, plotted as a blue circle.
% Every $\tau_\text{plan} = 0.5$ [s], the vehicle replans its trajectory, shown by an asterisk plotted on the global trajectory in blue.
% The bounding box of the vehicle at each planning step is shown as a grey rectangle.}
% \label{fig:example_trial}
% \end{figure}

% \begin{figure}
% \centering
% \includegraphics[scale=0.4,trim={1cm 7cm 1cm 7cm},clip]{example_emergency_brake.pdf}
% \caption{An example of a forced emergency braking situation. The vehicle cannot find a path to the desired location (plotted as a blue circle), so it brakes.}
% \label{fig:example_emergency_brake}
% \end{figure}

% \begin{figure}
% \centering
% \includegraphics[scale=0.4,trim={1cm 7cm 1cm 7cm},clip]{example_overly_conservative.pdf}
% \caption{An example of an unnecessary emergency braking situation. The vehicle cannot find a path to the desired location despite an obviously-safe path existing, because the FRS is overly conservative.}
% \label{fig:example_overly_conservative}
% \end{figure}


\begin{table*}[!t]
\centering
\resizebox{\linewidth}{!}{%
\begin{tabular}{c|c c}
\toprule
\textbf{\textcolor{myOrange}{\large{Image}}}-\textbf{\textcolor{myblue}{\large{caption}}} \large{pair} & \large{\textbf{\textcolor{myblue}{Textual evidence}}} \largericon{figs/icon3.pdf} & \large{\textbf{\textcolor{myOrange}{Visual evidence}}} \largericon{figs/icon4.pdf} \\ \midrule
\makecell{\fcolorbox{ao(english)}{lightgreen}{
\begin{varwidth}{\textwidth} \begin{center}\fcolorbox{myOrange}{white}{\includegraphics[width=5cm,keepaspectratio]{figs/qual2/260/235.jpg}}\end{center} 
\fcolorbox{myblue}{white}{\begin{varwidth}{\textwidth}\normalsize{Hungary has erected a fence on\\its border with Serbia}\end{varwidth} }\end{varwidth}}} & 

\makecell{\fcolorbox{myblue}{white}{\begin{varwidth}{\textwidth} \normalsize{\hlc[light_yellow]{`Hungary'}, \hlc[light_yellow]{`European migrant crisis'},\\\hlc[light_yellow]{`Refugee'}, `Human migration',\\\hlc[light_yellow]{`Immigration'}, `Border', \\`Fence',`Hungarians', `Asylum seeker'\\`Hungary–Serbia border',\\`Hungarian border barrier',\\`International law',`Refugee law',\\`hungary fences refugees'} \end{varwidth}}
\fcolorbox{myblue}{white}{\begin{varwidth}{\textwidth} \normalsize{\hlc[light_yellow]{1- Hungary police recruit}\\\hlc[light_yellow]{border-hunters.}\\2-Migrants and refugees walk\\near razor-wire along a 3-meter-high\\fence secured by Hungarian police\\at the official border crossing\\between Serbia and Hungary.} \end{varwidth}}}
& 
\makecell{ \fcolorbox{myOrange}{light_yellow}{\includegraphics[width=5cm,keepaspectratio]{figs/qual2/260/8.jpg}} \fcolorbox{myOrange}{white}{\includegraphics[width=5cm,keepaspectratio]{figs/qual2/260/5.jpg}}
\fcolorbox{myOrange}{white}{\includegraphics[width=5cm,keepaspectratio]{figs/qual2/260/9.jpg}}}
\\ & \multicolumn{2}{c}{\hspace{-4cm}\large{\textbf{Prediction: \textcolor{ao(english)}{Pristine}}}} \\


\makecell{\fcolorbox{ao(english)}{lightgreen}{
\begin{varwidth}{\textwidth} \begin{center}\fcolorbox{myOrange}{white}{\includegraphics[width=5cm,keepaspectratio]{figs/qual2/5884/417.jpg}}\end{center} 
\fcolorbox{myblue}{white}{\begin{varwidth}{\textwidth}\normalsize{Last year Shinzo Abe said Africa\\would help drive global growth\\in the future
}\end{varwidth} }\end{varwidth}}} & 

\makecell{\fcolorbox{myblue}{white}{\begin{varwidth}{\textwidth} \normalsize{\hlc[light_yellow]{`Shinzo Abe'}, `Akie Abe',\\`Prime Minister of Japan',\\`Japan', `Prime minister',\\\hlc[light_yellow]{`Trinidad and Tobago'}, \\`Dominica Vibes News',\\\hlc[light_yellow]{`United National Congress'},\\`Week', `Businessperson', 'official'\\\hlc[light_yellow]{`Dominica Housing Recovery Project'}} \end{varwidth}}
\fcolorbox{myblue}{white}{\begin{varwidth}{\textwidth} \normalsize{\hlc[light_yellow]{1-Japanese Prime Minister}\\\hlc[light_yellow]{Shinzo Abe and his wife,}\\\hlc[light_yellow]{Akie Abe.}\\2-Japanese Prime Minister\\Shinzo Abe, center, and\\his wife Akie wave as they\\depart for Africa, at Haneda\\Airport in Tokyo Thursday.} \end{varwidth}}}
& 
\makecell{ \fcolorbox{myOrange}{light_yellow}{\includegraphics[width=5cm,keepaspectratio]{figs/qual2/5884/1.jpg}} \fcolorbox{myOrange}{white}{\includegraphics[width=5cm,keepaspectratio]{figs/qual2/5884/8.jpg}}
\fcolorbox{myOrange}{white}{\includegraphics[width=6cm,keepaspectratio]{figs/qual2/5884/5.jpg}}}  
\\ & \multicolumn{2}{c}{\hspace{-4cm}\large{\textbf{Prediction: \textcolor{ao(english)}{Pristine}}}} \\

\makecell{\fcolorbox{ao(english)}{lightgreen}{
\begin{varwidth}{\textwidth} \begin{center}\fcolorbox{myOrange}{white}{\includegraphics[width=5cm,keepaspectratio]{figs/qual2/146/019.jpg}}\end{center} 
\fcolorbox{myblue}{white}{\begin{varwidth}{\textwidth}\normalsize{The GOP candidate at the soontobe\\Trump International Hotel a couple\\of blocks from the White House\\on Pennsylvania Avenue}\end{varwidth} }\end{varwidth}}} & 

\makecell{\fcolorbox{myblue}{white}{\begin{varwidth}{\textwidth} \normalsize{`Judge', \hlc[light_yellow]{`Legal case'}, `official'\\\hlc[light_yellow]{`Superior Court of the District}\\\hlc[light_yellow]{of Columbia'},\\`President-Elect', \hlc[light_yellow]{`Deposition'},\\`Court',`Plea', `Chef',\\\hlc[light_yellow]{`A Washington Law Firm'}} \end{varwidth}}
\fcolorbox{myblue}{white}{\begin{varwidth}{\textwidth} \normalsize{\hlc[light_yellow]{1-Republican presidential candidate}\\\hlc[light_yellow]{Donald Trump speaks during}\\\hlc[light_yellow]{a campaign press conference at}\\\hlc[light_yellow]{the at the Old Post Office Pavilion,}\\\hlc[light_yellow]{soon to be a Trump International Hotel}\\2-Judge rejects Trump plea to avoid\\deposition in José Andrés case} \end{varwidth}}}
& 
\makecell{ \fcolorbox{myOrange}{light_yellow}{\includegraphics[width=5cm,keepaspectratio]{figs/qual2/146/0.jpg}} \fcolorbox{myOrange}{white}{\includegraphics[width=5cm,keepaspectratio]{figs/qual2/146/6.jpg}}
\fcolorbox{myOrange}{white}{\includegraphics[width=5cm,keepaspectratio]{figs/qual2/146/2.jpg}}}  
\\ & \multicolumn{2}{c}{\hspace{-4cm}\large{\textbf{Prediction: \textcolor{ao(english)}{Pristine}}}} \\

\makecell{\fcolorbox{darkred}{lightred}{\begin{varwidth}{\textwidth}   \begin{center} \fcolorbox{myOrange}{white}{\includegraphics[width=5cm,keepaspectratio]{figs/qual2/3283/099.jpg}}\end{center}
\fcolorbox{myblue}{white}{\begin{varwidth}{\textwidth}\normalsize{Hillary Clinton speaks at a campaign\\event at Truckee Meadows Community\\College in Reno Nev Aug 25}\end{varwidth}}\end{varwidth}}} & 

\makecell{\fcolorbox{myblue}{white}{\begin{varwidth}{\textwidth} \normalsize{\hlc[light_yellow]{`United States'},`Commentator',\\\hlc[light_yellow]{`President of the United States'},\\\hlc[light_yellow]{`Clinton Foundation'},\\`Dinesh DSouza', `performance'} \end{varwidth}}   
\fcolorbox{myblue}{white}{\begin{varwidth}{\textwidth} \normalsize{\hlc[light_yellow]{1-Democratic presidential candidate}\\\hlc[light_yellow]{Hillary Clinton speaks at the}\\\hlc[light_yellow]{Iowa Democratic Wing Ding on}\\\hlc[light_yellow]{Friday in Clear Lake, Iowa.}\\2-At Wing Ding dinner, Clinton\\proves she still dominates Iowa} \end{varwidth} }}
& 
\makecell{ \fcolorbox{myOrange}{light_yellow}{\includegraphics[width=5cm,keepaspectratio]{figs/qual2/3283/8.jpg}} \fcolorbox{myOrange}{white}{\includegraphics[width=5cm,keepaspectratio]{figs/qual2/3283/4.jpg}}
\fcolorbox{myOrange}{white}{\includegraphics[width=5cm,keepaspectratio]{figs/qual2/3283/9.jpg}}} \\
&\multicolumn{2}{c}{\large{\hspace{-4cm}\textbf{Prediction: \textcolor{darkred}{Falsified}}}}\\

\makecell{\fcolorbox{darkred}{lightred}{\begin{varwidth}{\textwidth}   \begin{center} \fcolorbox{myOrange}{white}{\includegraphics[width=5cm,keepaspectratio]{figs/qual2/3023/620.jpg}}\end{center}
\fcolorbox{myblue}{white}{\begin{varwidth}{\textwidth}\normalsize{Taylor Swift performs during a concert at\\the Lanxess Arena in Cologne Germany}\end{varwidth}}\end{varwidth}}} & 

\makecell{\fcolorbox{myblue}{white}{\begin{varwidth}{\textwidth} \normalsize{\hlc[light_yellow]{`Taylor Swift'}, `The 1989 World Tour',\\\hlc[light_yellow]{`Grammy Award for Album of the Year'},\\\hlc[light_yellow]{`Grammy Awards'}, `Album', \\`Welcome to New York', `Speak Now',\\`Pop music', `reputation',\\`Apple Music', `Kendrick Lamar',\\`Adele', \hlc[light_yellow]{'taylor swift 1989'}} \end{varwidth}}   
\fcolorbox{myblue}{white}{\begin{varwidth}{\textwidth} \normalsize{\hlc[light_yellow]{1-Taylor Swift performs during}\\\hlc[light_yellow]{her '1989' World Tour Nov. 28,}\\\hlc[light_yellow]{2015, in Sydney, Australia.}\\2-Taylor Swift earned nominations\\in the major categories.\\3-Taylor Swift performs\\during her `1989' tour.} \end{varwidth} }}
& 
\makecell{ \fcolorbox{myOrange}{light_yellow}{\includegraphics[width=5cm,keepaspectratio]{figs/qual2/3023/3.jpg}} \fcolorbox{myOrange}{white}{\includegraphics[width=5cm,keepaspectratio]{figs/qual2/3023/6.jpg}}
\fcolorbox{myOrange}{white}{\includegraphics[width=5cm,keepaspectratio]{figs/qual2/3023/5.jpg}}} \\
&\multicolumn{2}{c}{\hspace{-4cm}\large{\textbf{Prediction: \textcolor{darkred}{Falsified}}}}\\ 

\makecell{\fcolorbox{darkred}{lightred}{\begin{varwidth}{\textwidth}   \begin{center} \fcolorbox{myOrange}{white}{\includegraphics[width=4cm,keepaspectratio]{figs/qual2/415/507.jpg}}\end{center}
\fcolorbox{myblue}{white}{\begin{varwidth}{\textwidth}\normalsize{The Blue House the executive office\\and residence of Korea s president}\end{varwidth}}\end{varwidth}}} & 

\makecell{\fcolorbox{myblue}{white}{\begin{varwidth}{\textwidth} \normalsize{`Parliament Hill',`Parliament of Canada'\\, \hlc[light_yellow]{`2014 shootings at Parliament Hill, Ottawa'},\\`Prime Minister of Canada', \\`Royal Canadian Mounted Police', \hlc[light_yellow]{`Ottawa'},\\`Terrorism', `Prime minister',\\`Michael Zehaf-Bibeau', `Stephen Harper',\\ \hlc[light_yellow]{`Kevin Vickers'}, \hlc[light_yellow]{`Ontario'},\\`Canada', `Ottawa'} \end{varwidth}}   
\fcolorbox{myblue}{white}{\begin{varwidth}{\textwidth} \normalsize{\hlc[light_yellow]{1-Police tape surrounds the Canadian}\\\hlc[light_yellow]{War Memorial in Ottawa after a soldier}\\\hlc[light_yellow]{guarding the monument was shot on}\\\hlc[light_yellow]{Wednesday.}\\2-Shooting Near Canada's Parliament.\\3-Shooting at War Memorial in Canada\\Photos} \end{varwidth} }}
& 
\makecell{ \fcolorbox{myOrange}{light_yellow}{\includegraphics[width=5cm,keepaspectratio]{figs/qual2/415/0.jpg}} \fcolorbox{myOrange}{white}{\includegraphics[width=5cm,keepaspectratio]{figs/qual2/415/7.jpg}}
\fcolorbox{myOrange}{white}{\includegraphics[width=5cm,keepaspectratio]{figs/qual2/415/5.jpg}}}\\ 
&\multicolumn{2}{c}{\hspace{-4cm}\large{\textbf{Prediction: \textcolor{darkred}{Falsified}}}}\\ \bottomrule 

\end{tabular}}
\captionof{figure}{Other qualitative examples. The ground truth is indicated by the pairs' background colour; examples with \hlc[lightgreen]{green background} are pristine, \hlc[lightred]{red background} are falsified. The model's prediction is indicated below each example's set; \textbf{\textcolor{ao(english)}{green}} for predicting pristine and \textbf{\textcolor{darkred}{red}} for predicting falsified. \hlc[light_yellow]{Highlighted items} are the ones with the highest attention.}
\label{tbl:qual_appendix}
\end{table*}


\end{document}
