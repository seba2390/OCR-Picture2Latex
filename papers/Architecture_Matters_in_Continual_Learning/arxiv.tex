\documentclass{article}
\usepackage[utf8]{inputenc}
\usepackage[margin=8em]{geometry}
\usepackage{graphicx}
\usepackage{amsmath}
\usepackage{amsfonts}
\usepackage{amsthm}
\usepackage{enumitem}
\usepackage{booktabs}
\usepackage{amsmath}
\usepackage{caption}
\usepackage{subcaption}
\usepackage{color}
\usepackage{authblk}
\usepackage{float}
% \usepackage{floatrow, tabularx, makecell, array}

\makeatletter
\renewcommand\AB@affilsepx{, \protect\Affilfont}
\makeatother
\newcounter{daggerfootnote}
\newcommand*{\daggerfootnote}[1]{%
    \setcounter{daggerfootnote}{\value{footnote}}%
    \renewcommand*{\thefootnote}{\fnsymbol{footnote}}%
    \footnote[2]{#1}%
    \setcounter{footnote}{\value{daggerfootnote}}%
    \renewcommand*{\thefootnote}{\arabic{footnote}}%
    }
\usepackage{amssymb}

\setlength{\parindent}{0em}
\setlength{\parskip}{1em}
\usepackage[round]{natbib}
\renewcommand{\bibname}{References}
\renewcommand{\bibsection}{\subsubsection*{\bibname}}
\newcommand{\norm}[1]{\left\lVert#1\right\rVert}
\setlength{\affilsep}{2em}   % set the space between author and affiliation

\usepackage{url}
\title{Architecture Matters in Continual Learning}
\author[1]{Seyed Iman Mirzadeh\footnote{Work done during an internship at DeepMind.}}
\author[2]{Arslan Chaudhry} 
\author[2]{Dong Yin}
\author[2]{\authorcr Timothy Nguyen}
\author[2]{Razvan Pascanu}
\author[2]{Dilan Gorur}
\author[2]{Mehrdad Farajtabar\footnote{Correspondence to farajtabar@google.com}}

\affil[1]{Washington State University}
\affil[2]{DeepMind}
\date{}

\usepackage{pifont}
\newcommand{\tick}{{\ding{51}}}

\newcommand\figref[1]{Fig.~\ref{#1}}
\newcommand\secref[1]{Sec.~\ref{#1}}
\newcommand\tabref[1]{Tab.~\ref{#1}}

% style for architectures (could be texttt or anything)
\newcommand{\archstyle}[1]{#1}
% Architecture Templates: e.g., CNN-D
\newcommand\CNNTemp[1]{\archstyle{CNN$\times${#1}}}
\newcommand\MLPTemp[1]{\archstyle{MLP-{#1}}}
\newcommand\ResNetTemp[1]{\archstyle{ResNet-{#1}}}
\newcommand\WRNTemp[2]{\archstyle{WRN-{#1}-{#2}}}
\newcommand\ViTTemp[2]{\archstyle{ViT {#1}/{#2}}}

% Architectures Shortcuts
\newcommand{\CNNArch}{\CNNTemp{N}~}
\newcommand{\MLPArch}{\MLPTemp{N}~}
\newcommand{\ResNetArch}{\ResNetTemp{D}~}
\newcommand{\WRNArch}{\WRNTemp{D}{N}~}
\newcommand{\ViTArch}{\ViTTemp{N}{M}~}

% Architectures names only
\newcommand{\CNN}{\archstyle{CNN}}
\newcommand{\MLP}{\archstyle{MLP}}
\newcommand{\ResNet}{\archstyle{ResNet}}
\newcommand{\WRN}{\archstyle{WRN}}
\newcommand{\ViT}{\archstyle{ViT}}

\newcommand{\negspace}[1]{}
% \newcommand{\negspace}[1]{}



\usepackage{hyperref}

% Attempt to make hyperref and algorithmic work together better:
\newcommand{\theHalgorithm}{\arabic{algorithm}}

% Use the following line for the initial blind version submitted for review:

% define custom commands
\newcommand{\SKIP}[1]{}

\begin{document}
\vspace{-1mm}
\maketitle

\vspace{-2mm}
\begin{abstract}

A large body of research in continual learning is devoted to overcoming the catastrophic forgetting of neural networks by designing new algorithms that are robust to the distribution shifts. However, the majority of these works are strictly focused on the ``algorithmic'' part of continual learning for a ``fixed neural network architecture'', and the implications of using different architectures are mostly neglected. Even the few existing continual learning methods that modify the model assume a fixed architecture and aim to develop an algorithm that efficiently uses the model throughout the learning experience. However, in this work, we show that the choice of architecture can significantly impact the continual learning performance, and different architectures lead to different trade-offs between the ability to remember previous tasks and learning new ones. Moreover, we study the impact of various architectural decisions, and our findings entail best practices and recommendations that can improve the continual learning performance.
\end{abstract}

% \leavevmode
% \\
% \\
% \\
% \\
% \\
\section{Introduction}
\label{introduction}

AutoML is the process by which machine learning models are built automatically for a new dataset. Given a dataset, AutoML systems perform a search over valid data transformations and learners, along with hyper-parameter optimization for each learner~\cite{VolcanoML}. Choosing the transformations and learners over which to search is our focus.
A significant number of systems mine from prior runs of pipelines over a set of datasets to choose transformers and learners that are effective with different types of datasets (e.g. \cite{NEURIPS2018_b59a51a3}, \cite{10.14778/3415478.3415542}, \cite{autosklearn}). Thus, they build a database by actually running different pipelines with a diverse set of datasets to estimate the accuracy of potential pipelines. Hence, they can be used to effectively reduce the search space. A new dataset, based on a set of features (meta-features) is then matched to this database to find the most plausible candidates for both learner selection and hyper-parameter tuning. This process of choosing starting points in the search space is called meta-learning for the cold start problem.  

Other meta-learning approaches include mining existing data science code and their associated datasets to learn from human expertise. The AL~\cite{al} system mined existing Kaggle notebooks using dynamic analysis, i.e., actually running the scripts, and showed that such a system has promise.  However, this meta-learning approach does not scale because it is onerous to execute a large number of pipeline scripts on datasets, preprocessing datasets is never trivial, and older scripts cease to run at all as software evolves. It is not surprising that AL therefore performed dynamic analysis on just nine datasets.

Our system, {\sysname}, provides a scalable meta-learning approach to leverage human expertise, using static analysis to mine pipelines from large repositories of scripts. Static analysis has the advantage of scaling to thousands or millions of scripts \cite{graph4code} easily, but lacks the performance data gathered by dynamic analysis. The {\sysname} meta-learning approach guides the learning process by a scalable dataset similarity search, based on dataset embeddings, to find the most similar datasets and the semantics of ML pipelines applied on them.  Many existing systems, such as Auto-Sklearn \cite{autosklearn} and AL \cite{al}, compute a set of meta-features for each dataset. We developed a deep neural network model to generate embeddings at the granularity of a dataset, e.g., a table or CSV file, to capture similarity at the level of an entire dataset rather than relying on a set of meta-features.
 
Because we use static analysis to capture the semantics of the meta-learning process, we have no mechanism to choose the \textbf{best} pipeline from many seen pipelines, unlike the dynamic execution case where one can rely on runtime to choose the best performing pipeline.  Observing that pipelines are basically workflow graphs, we use graph generator neural models to succinctly capture the statically-observed pipelines for a single dataset. In {\sysname}, we formulate learner selection as a graph generation problem to predict optimized pipelines based on pipelines seen in actual notebooks.

%. This formulation enables {\sysname} for effective pruning of the AutoML search space to predict optimized pipelines based on pipelines seen in actual notebooks.}
%We note that increasingly, state-of-the-art performance in AutoML systems is being generated by more complex pipelines such as Directed Acyclic Graphs (DAGs) \cite{piper} rather than the linear pipelines used in earlier systems.  
 
{\sysname} does learner and transformation selection, and hence is a component of an AutoML systems. To evaluate this component, we integrated it into two existing AutoML systems, FLAML \cite{flaml} and Auto-Sklearn \cite{autosklearn}.  
% We evaluate each system with and without {\sysname}.  
We chose FLAML because it does not yet have any meta-learning component for the cold start problem and instead allows user selection of learners and transformers. The authors of FLAML explicitly pointed to the fact that FLAML might benefit from a meta-learning component and pointed to it as a possibility for future work. For FLAML, if mining historical pipelines provides an advantage, we should improve its performance. We also picked Auto-Sklearn as it does have a learner selection component based on meta-features, as described earlier~\cite{autosklearn2}. For Auto-Sklearn, we should at least match performance if our static mining of pipelines can match their extensive database. For context, we also compared {\sysname} with the recent VolcanoML~\cite{VolcanoML}, which provides an efficient decomposition and execution strategy for the AutoML search space. In contrast, {\sysname} prunes the search space using our meta-learning model to perform hyperparameter optimization only for the most promising candidates. 

The contributions of this paper are the following:
\begin{itemize}
    \item Section ~\ref{sec:mining} defines a scalable meta-learning approach based on representation learning of mined ML pipeline semantics and datasets for over 100 datasets and ~11K Python scripts.  
    \newline
    \item Sections~\ref{sec:kgpipGen} formulates AutoML pipeline generation as a graph generation problem. {\sysname} predicts efficiently an optimized ML pipeline for an unseen dataset based on our meta-learning model.  To the best of our knowledge, {\sysname} is the first approach to formulate  AutoML pipeline generation in such a way.
    \newline
    \item Section~\ref{sec:eval} presents a comprehensive evaluation using a large collection of 121 datasets from major AutoML benchmarks and Kaggle. Our experimental results show that {\sysname} outperforms all existing AutoML systems and achieves state-of-the-art results on the majority of these datasets. {\sysname} significantly improves the performance of both FLAML and Auto-Sklearn in classification and regression tasks. We also outperformed AL in 75 out of 77 datasets and VolcanoML in 75  out of 121 datasets, including 44 datasets used only by VolcanoML~\cite{VolcanoML}.  On average, {\sysname} achieves scores that are statistically better than the means of all other systems. 
\end{itemize}


%This approach does not need to apply cleaning or transformation methods to handle different variances among datasets. Moreover, we do not need to deal with complex analysis, such as dynamic code analysis. Thus, our approach proved to be scalable, as discussed in Sections~\ref{sec:mining}.
%!TEX ROOT = ../centralized_vs_distributed.tex

\section{Problem Setup}\label{sec:setup}

We consider {an undirected network} with $ N $ agents 
{in which the state of the $ i $th agent at time $ t $ is given by $ \xbar{i}{t}\in\Real{} $ with the control input $ \u{i}{t}\in\Real{} $.}
For {notational} convenience,
we {introduce} the aggregate {state of the} system  $ \xbar{}{t} $ and {the}
aggregate control input $ \u{}{t} $ {by stacking states and control inputs of each subsystem} $ \xbar{i}{t} $ and $ \u{i}{t} $, 
respectively.

\iffalse
\begin{rem}[Network topology]
	While in the first part of the paper we focus on circular formations for the sake of analysis and ease of presentation,
	the control design can be readily extended to generic undirected topologies.
	We discuss theoretical guarantees in~\autoref{sec:generic-topology} and observe with computational experiments in~\autoref{sec:numerical-results}
	that the \tradeoff holds regardless of the specific topology. % at hand.
\end{rem}
\fi

\myParagraph{Problem Statement}
The agents aim to reach consensus towards a common state trajectory. 
The $i$th component of the vector $ \x{}{t} \doteq \Omega\xbar{}{t} $ represents 
the mismatch between the state of agent $ i $ and the average network state at time $ t $\revision{\cite{bamjovmitpat12}},
where
\begin{equation}\label{eq:error-matrix}
	\Omega \doteq I_{N}-\consMatrix
\end{equation}
and $ \mathds{1}_N \in\Real{N} $ is the vector of all ones,
such that $ \Omega\mathds{1}_N=0 $.
%\red{The target consensus vector is defined as $ \x{m}{t} \doteq \xbar{}{t} - \x{}{t} $.}

\done{
	\tcb{\myParagraph{Ring Topology}
	We focus on ring topology to obtain analytical insights about 
	optimal control design and fundamental performance trade-offs in the presence of communication delays. 
	While some of our notation is tailored to such topology (\eg see equations~\eqref{eq:meas} and~\eqref{eq:feedback-matrix}), 
	in~\autoref{sec:generic-topology} we discuss extension of the optimal control design to generic undirected networks 
	and complement these developments with computational experiments in~\autoref{sec:numerical-results}.}
}

%\myParagraph{\titlecap{Communication model}}
%The agents communicate through a {shared wireless channel}.
%Data are exchanged through a {shared wireless channel} in a symmetric fashion.
%Agent $ i $ communicates with
%\red{$ n $ pairs of agents,
%both agents in each such pair being at equal distance from $ i $}
%\tcb{its} $ 2n $ closest neighbors \tcb{in ring topology.}
%Also, we make the following assumption
%to address channel constraints.

\begin{ass}[Communication model]\label{ass:hypothesis}
	Data are exchanged through a shared wireless channel in a symmetric fashion.
%	\tcb{its} $ 2n $ closest neighbors \tcb{in ring topology.}
	\revisiontwo{Agent $ i $ receives state measurements from
	all agents within $ n $ communication hops.}
	All measurements are received with delay $ \taun \doteq f(n) $
	where $ f(\cdot) $ is a positive increasing sequence.
	\revisiontwo{In particular,
		in ring topology,
		agent $ i $ receives state measurements from the
		$ 2n $ closest agents,
		that is,
		from the $ n $ pairs of agents at distance $ \ell = 1,\dots,n $,
		with $ 1\le n<\nicefrac{N}{2} $.}\footnote{\revision{
%	where both agents in each such pair are at equal distance $ \ell $ from $ i $. % in the ring topology.
%%	located $ \ell $ positions ahead and behind in the formation,.
%	In what follows,
%	without loss of generality,
%	we assume that such $ n $ agent pairs coincide with the
%	$ 2n $ closest agents in ring topology,
%	and that each pair is at distance $ \ell = 1,\dots,n<\nicefrac{N}{2} $.
	\revisiontwo{For example,
	$ n = 1 $ corresponds to nearest-neighbor interaction in ring topology
	and $ n = \floor{\nicefrac{(N-1)}{2}} $ to all-to-all communication topology.}}}
%	Also, each agent measures its own state instantaneously.
\end{ass}

\revision{\begin{rem}[Architecture parametrization]\label{rem:architecture-param}
	Parameter $ n $ will play a crucial role throughout our discussion. 
	In particular,
	we will use it to (i) evaluate the optimal performance %that can be attained 
	for a given
	budget of links
	\revisiontwo{(see~\cref{prob:variance-minimization})};
	and to (ii) compare optimal performance of different control architectures.
	In the first part of the paper, 
	we examine circular formations and
	$ n $ represents how many neighbor pairs communicate with each agent.
	For \linebreak general undirected networks,
	$ n $ determines the number of communication hops for each agent.
	In general,
	$ n $ characterizes sparsity of a controller architecture:
	sparse controllers correspond to small $n$ while highly connected ones to 
	large $ n $.
\end{rem}}

%\begin{rem}
%	The time $ \delayn $ embeds both the communication delay,
%	due to channel constraints,
%	and the computation delay,
%%	Even though the rate $ f(n) = n $ may seem natural,
%%	other rates are possible, 
%	arising if the agents preprocess the acquired measurements.
%	In practice, $ f(n) $ is to be estimated or learned from data.
%\end{rem}

\myParagraph{\titlecap{Feedback control}}
Agent $ i $ uses the received information to compute the
state mismatches $ \meas{i}{\ell^\pm}{t} $ {relative to its} neighbors,
\begin{equation}\label{eq:meas}
	\meas{i}{\ell^\pm}{t} = 
	\begin{cases}
		\xbar{i}{t} - \xbar{i\pm\ell}{t}, & 0<i\pm\ell\le N\\
		\xbar{i}{t} - \xbar{i\pm\ell\mp N}{t}, & \mbox{otherwise},
	\end{cases}
\end{equation}
%Such mismatches are exploited to compute the 
{and} the proportional control input is {given by}
\begin{equation}\label{eq:prop-control}
	\u{P,i}{t} = -\sum_{\ell=1}^{n}k_\ell\left(\meas{i}{\ell^+}{t-\taun}+\meas{i}{\ell^-}{t-\taun}\right),
\end{equation}
where measurements are delayed according to~\cref{ass:hypothesis}.

For networks with double integrator agents,
the control input $u_i(t)$ may also include a derivative term,
%Depending on the agent dynamics, the control input $ \u{i}{t} $
%may be purely proportional or include a derivative term, such as
\begin{equation}\label{eq:control-input-PD}
	\u{i}{t} = \gvel\u{P,i}{t} - \gvel\dfrac{d\xbar{i}{t}}{dt} = \gvel\u{P,i}{t} -\gvel\dfrac{d\x{i}{t}}{dt}.
\end{equation}
The derivative term in~\eqref{eq:control-input-PD} is delay free
because it only requires measurements coming from the agent itself,
which we assume {to be} available instantaneously. 
%The latter will be defined in due time.
The proportional input can be compactly written as $ \u{P}{t} = -K\xbar{}{t-\taun}=-K\x{}{t-\taun} $.
\revision{With ring topology, the feedback gain matrix is}
\begin{equation}\label{eq:feedback-matrix}
%	\begin{array}{c}
%		K \doteq K_f + K_f^\top \\
		K = \mathrm{circ}
		\left(\sum_{\ell=1}^nk_\ell, -k_1, \dots, -k_n, 0,  \dots, 0, -k_n, \dots, -k_1\right),
%	\end{array}
\end{equation}
where $ \mathrm{circ}\left(a_1,\dots,a_n\right) $ denotes the circulant matrix in $ \Real{n\times n} $
with elements $ a_1,\dots,a_n $ in the first row.

\revisiontwo{For agents with additive stochastic disturbances
	(see Sections \ref{sec:cont-time} and~\ref{sec:disc-time}),
	we consider the following problem for each $ n $.}

\begin{prob}\label{prob:variance-minimization}
	Design the feedback gains in order to minimize the steady-state variance of the consensus error,
%	\marginpar{\tiny Added both problems to highlight the optimization variables in the two cases.}
	\blue{\begin{subequations}\label{eq:problem}
		\begin{equation}\label{eq:variance-minimization-P}
		\mbox{P control:} \qquad \argmin_{K} \; \var(K),
		\end{equation}
		\begin{equation}\label{eq:variance-minimization-PD}
		\mbox{PD control:} \qquad \argmin_{\gvel,K} \; \var(\gvel,K),
		\end{equation}
	\end{subequations}}
	where
	\begin{equation}\label{}
	\var \doteq \lim_{t\rightarrow+\infty} \mathbb{E}\left[\lVert\x{}{t}\rVert^2\right]
	\end{equation}
	and w.l.o.g. we assume $ \mathbb{E}\left[\x{}{\cdot}\right] \equiv \mathbb{E}\left[\x{}{0}\right] = 0 $.
	%	and $ \varx{x} \stackrel{!}{=} $ if the system is mean-square unstable.
\end{prob}

\section{Case Studies}
\label{sec:case_studies}
In this section, we present a case study of Facebook posts from an Australian public page.
The page shifts between early 2020 (\emph{2019-2020 Australian bushfire season}) and late 2020 (\emph{COVID-19 crises}) from being a moderate-right group for discussion around climate change to a far-right extremist group for conspiracy theories.


\begin{figure*}[!tbp]
	\begin{subfigure}{0.21\textwidth}
		\includegraphics[width=\textwidth]{images/facebook1.png}
		\caption{}
		\label{subfig:first-posting}
		\includegraphics[width=0.9\textwidth]{images/facebook3.jpg}
		\caption{}
		\label{subfig:comment-post-1}
	\end{subfigure}
    \begin{subfigure}{0.28\textwidth}
		\includegraphics[width=\textwidth]{images/facebook2.jpg}
		\caption{}
		\label{subfig:second-posting}
	\end{subfigure}
    \begin{subfigure}{0.23\textwidth}
		\includegraphics[width=\textwidth]{images/facebook4.jpg}
		\caption{}
		\label{subfig:comment-post-2a}
	\end{subfigure}
    \begin{subfigure}{0.23\textwidth}
		\includegraphics[width=\textwidth]{images/facebook5.jpg}
		\caption{}
		\label{subfig:comment-post-2b}
	\end{subfigure}
	\caption{
		Examples of postings and comment threads from a public Facebook page from two periods of time early 2020 (a) and late 2020 (b)-(e), which show a shift from climate change debates to extremist and far-right messaging.
	}
	\label{fig:facebook}
\end{figure*}

We focus on a sample of 2 postings and commenting threads from one Australian Facebook page we classified as ``far-right'' based on the content on the page. 
We have anonymized the users in \Cref{fig:facebook} to avoid re-identification.
The first posting and comment thread (see \Cref{subfig:first-posting}) was collected on Jan 10, 2020, and responded to the Australian bushfire crisis that began in late 2019 and was still ongoing in January 2020. It contains an ambivalent text-based provocation that references disputes in the community regarding the validity of climate change and climate science. 

The second posting and comment thread (see \Cref{subfig:second-posting}) was collected from the same page in September 2020, months after the bushfire crisis had abated.
At that time, a new crisis was energizing and connecting the far-right groups in our dataset --- i.e., the COVID-19 pandemic and the government interventions to curb the spread of the virus. 
The post is different in style compared to the first.
It is image-based instead of text-based and highly emotive, with a photo collage bringing together images of prison inmates with iron masks on their faces (top row) juxtaposed to people wearing face masks during COVID-19 (bottom row). 
The image references the public health orders issued during Melbourne's second lockdown and suggests that being ordered to wear masks is an infringement of citizen rights and freedoms, similar to dehumanizing restraints used on prisoners.

To analyze reactions to the posts, two researchers used a deductive analytical approach to separately code and to analyze the commenting threads --- see \Cref{subfig:comment-post-1} for comments of the first posting, and \Cref{subfig:comment-post-2a,subfig:comment-post-2b} for comments on the second posting. 
Conversations were also inductively coded for emerging themes. 
During the analysis, we observed qualitative differences in the types of content users posted, interactions between commenters, tone and language of debate, linked media shared in the commenting thread, and the opinions expressed.
The rest of this section further details these differences.
To ensure this was not a random occurrence, we tested the exemplar threads against field notes collected on the group during the entire study.
We also used Facebook's search function within pages to find a sample of posts from the same period and which dealt with similar topics. 
After this analysis, we can confidently say that key changes occurred in the group between the bushfire crisis and COVID-19, that we detail next.

\subsubsection*{Exemplar 1 --- climate change skepticism.}
To explore this transformation in more depth, we analyzed comments scraped on the first posting --- \cref{sub@subfig:comment-post-1} shows a small sample of these comments.
The language used was similar to comments that we observed on numerous far-right nationalist pages at the time of the bushfires.
These comments are usually text-based, employing emojis to denote emotions, and sometimes being mocking or provocative in tone. 
Noteworthy for this commenting thread is the 50/50 split in the number of members posting in favor of action on climate change (on one side) and those who posted anti-Greens and anti-climate change science posts and memes (on the other side).
The two sides aligned strongly with political partisanship --- either with Liberal/National coalition (climate change deniers) or Labor/Green (climate change believers) parties. 
This is rather unusual for pages classified as far-right. 

We observed trolling practices between the climate change deniers and believers, which often descend into \emph{flame wars} --- i.e., online ``firefights that take place between disembodied combatants on electronic bulletin boards''~\citep{bukatman1994flame}.
The result is a boosted engagement on the post but also the frustration and confusion of community members and lurkers who came to the discussions to become informed or debate rationally on key differences between the two positions.
They often even become targeted, victimized, and baited by trolls on both sides of the partisan divide. 
The opinions expressed by deniers in commenting sections range from skepticism regarding climate change science to plain denial.
Deniers also regard a range of targets as embroiled in a climate change conspiracy to deceive the public, such as The Greens and their environmental policy, in some cases the government, the United Nations, and climate change celebrities like David Attenborough and Greta Thunberg. 
These figures are blamed for either exaggerating risks of climate change or creating a climate change hoax to increase the influence of the UN on domestic governments or to increase domestic governments' social control over citizens. 

Both coders noted that flame wars between these opposing personas contained very few links to external media. 
Where links were added, they often seemed disconnected from the rest of the conversation and were from users whose profiles suggested they believed in more radical conspiracy theories.
One such example is ``geo-engineering'' (see \cref{sub@subfig:comment-post-1}).
Its adherents believe that solar geo-engineering programs designed to combat climate change are secretly used by a global elite to depopulate the world through sterilization or to control and weaponize the weather.

Nonetheless, apart from the random comments that hijack the thread, redirecting users to external ``alternative'' news sites and Twitter, and the trolls who seem to delight in victimizing unsuspecting victims, the discussion was pretty healthy.
There are many questions, rational inquiries, and debates between users of different political persuasion and views on climate change.
This, however, changes in the span of only a couple of months.

\subsubsection*{Exemplar 2 --- posting and commenting thread.}
We observe a shift in the comment section of the post collected during the second wave of the COVID pandemic (\Cref{sub@subfig:second-posting}) --- which coincided with government laws mandating the public to wear masks and stay at home in Victoria, Australia.
There emerges much more extreme far-right content that converges with anti-vaccination opinions and content.
We also note a much higher prevalence of conspiracy theories often implicating racialized targets.
This is exemplified in the comments on the second post (\Cref{sub@subfig:comment-post-2a,sub@subfig:comment-post-2b}) where Islamophobia and antisemitism are confidently asserted alongside anti-mask rhetoric.
These comments consider face masks similar to the religious head coverings worn by some Muslim women, which users describe as ``oppressive'' and ``silencing''. 
In this way, anti-maskers cast women as a distinct, sympathetic marginalized demographic.
However, this is enacted alongside the racialization and demonization of Islam as an oppressive religion. 

Given the extreme racialization of anti-mask rhetoric, some commenters contest these positions, arguing that the page is becoming less an anti-Scott Morrison page (Australia's Prime Minister at the time) and changing into a page that harbors ``far-right dickheads''.
This questioning is actively challenged by far-right commenters and conspiracy theorists on the page, who regarded pro-mask users and the Scott Morrison government as ``puppets'' being manipulated by higher forces (see \Cref{sub@subfig:comment-post-2b}). 

This indicates a significant change on the page's membership towards the extreme-right, who employs more extreme forms of racialized imagery, with more extreme opinion being shared.
Conspiracy theorists become more active and vocal, and they consistently challenge the opinions of both center conservative and left-leaning users. 
This is evident in the final two comments in \Cref{subfig:comment-post-2b}, which reflect QAnon style conspiracy theories and language.
Public health orders to wear masks are being connected to a conspiracy that all of these decisions are directed by a secret network of global Jewish elites, who manipulate the pandemic to increase their power and control. 
This rhetoric intersects with the contemporary ``QAnon'' conspiracy theory, which evolved from the ``Pizzagate'' conspiracy theory.
They also heavily draw on well-established antisemitic blood libel conspiracy theories, which foster beliefs that a powerful global elite is controlling the decisions of organizations such as WHO and are responsible for the vaccine rollout and public health orders related to the pandemic.
The QAnon conspiracy is also influenced by Bill Gates' Microchips conspiracy theory, i.e., the theory that the WHO and the Bill Gates Foundation global vaccine programs are used to inject tracking microchips into people.

These conspiracy theories have, since COVID-19, connected formerly separate communities and discourses, uniting existing anti-vaxxer communities, older demographics who are mistrustful of technology, far-right communities suspicious of global and national left-wing agendas, communities protesting against 5G mobile networks (for fear that they will brainwash, control, or harm people), as well as generating its own followers out of those anxious during the 2020 onset of the COVID-19 pandemic.
We detect and describe some of these opinion dynamics in the next section.

\section{Related Work}
%\mz{We lack a comparison to this paper: https://arxiv.org/abs/2305.14877}
%\anirudh{refine to be more on-topic?}
\iffalse
\paragraph{In-Context Learning} As language models have scaled, the ability to learn in-context, without any weight updates, has emerged. \cite{brown2020language}. While other families of large language models have emerged, in-context learning remains ubiquitous \cite{llama, bloom, gptneo, opt}. Although such as HELM \cite{helm} have arisen for systematic evaluation of \emph{models}, there is no systematic framework to our knowledge for evaluating \emph{prompting methods}, and validating prompt engineering heuristics. The test-suite we propose will ensure that progress in the field of prompt-engineering is structured and objectively evaluated. 

\paragraph{Prompt Engineering Methods} Researchers are interested in the automatic design of high performing instructions for downstream tasks. Some focus on simple heuristics, such as selecting instructions that have the lowest perplexity \cite{lowperplexityprompts}. Other methods try to use large language models to induce an instruction when provided with a few input-output pairs \cite{ape}. Researchers have also used RL objectives to create discrete token sequences that can serve as instructions \cite{rlprompt}. Since the datasets and models used in these works have very little intersection, it is impossible to compare these methods objectively and glean insights. In our work, we evaluate these three methods on a diverse set of tasks and models, and analyze their relative performance. Additionally, we recognize that there are many other interesting angles of prompting that are not covered by instruction engineering \cite{weichain, react, selfconsistency}, but we leave these to future work.

\paragraph{Analysis of Prompting Methods} While most prompt engineering methods focus on accuracy, there are many other interesting dimensions of performance as well. For instance, researchers have found that for most tasks, the selection of demonstrations plays a large role in few-shot accuracy \cite{whatmakesgoodicexamples, selectionmachinetranslation, knnprompting}. Additionally, many researchers have found that even permuting the ordering of a fixed set of demonstrations has a significant effect on downstream accuracy \cite{fantasticallyorderedprompts}. Prompts that are sensitive to the permutation of demonstrations have been shown to also have lower accuracies \cite{relationsensitivityaccuracy}. Especially in low-resource domains, which includes the large public usage of in-context learning, these large swings in accuracy make prompting less dependable. In our test-suite we include sensitivity metrics that go beyond accuracy and allow us to find methods that are not only performant but reliable.

\paragraph{Existing Benchmarks} We recognize that other holistic in-context learning benchmarks exist. BigBench is a large benchmark of 204 tasks that are beyond the capabilities of current LLMs. BigBench seeks to evaluate the few-shot abilities of state of the art large language models, focusing on performance metrics such as accuracy \cite{bigbench}. Similarly, HELM is another benchmark for language model in-context learning ability. Rather than only focusing on performance, HELM branches out and considers many other metrics such as robustness and bias \cite{helm}. Both BigBench and HELM focus on ranking different language model, while fix a generic instruction and prompt format. We instead choose to evaluate instruction induction / selection methods over a fixed set of models. We are the first ever evaluation script that compares different prompt-engineering methods head to head. 
\fi

\paragraph{In-Context Learning and Existing Benchmarks} As language models have scaled, in-context learning has emerged as a popular paradigm and remains ubiquitous among several autoregressive LLM families \cite{brown2020language, llama, bloom, gptneo, opt}. Benchmarks like BigBench \cite{bigbench} and HELM \cite{helm} have been created for the holistic evaluation of these models. BigBench focuses on few-shot abilities of state-of-the-art large language models, while HELM extends to consider metrics like robustness and bias. However, these benchmarks focus on evaluating and ranking \emph{language models}, and do not address the systematic evaluation of \emph{prompting methods}. Although contemporary work by \citet{yang2023improving} also aims to perform a similar systematic analysis of prompting methods, they focus on simple probability-based prompt selection while we evaluate a broader range of methods including trivial instruction baselines, curated manually selected instructions, and sophisticated automated instruction selection.

\paragraph{Automated Prompt Engineering Methods} There has been interest in performing automated prompt-engineering for target downstream tasks within ICL. This has led to the exploration of various prompting methods, ranging from simple heuristics such as selecting instructions with the lowest perplexity \cite{lowperplexityprompts}, inducing instructions from large language models using a few annotated input-output pairs \cite{ape}, to utilizing RL objectives to create discrete token sequences as prompts \cite{rlprompt}. However, these works restrict their evaluation to small sets of models and tasks with little intersection, hindering their objective comparison. %\mz{For paragraphs that only have one work in the last line, try to shorten the paragraph to squeeze in context.}

\paragraph{Understanding in-context learning} There has been much recent work attempting to understand the mechanisms that drive in-context learning. Studies have found that the selection of demonstrations included in prompts significantly impacts few-shot accuracy across most tasks \cite{whatmakesgoodicexamples, selectionmachinetranslation, knnprompting}. Works like \cite{fantasticallyorderedprompts} also show that altering the ordering of a fixed set of demonstrations can affect downstream accuracy. Prompts sensitive to demonstration permutation often exhibit lower accuracies \cite{relationsensitivityaccuracy}, making them less reliable, particularly in low-resource domains.

Our work aims to bridge these gaps by systematically evaluating the efficacy of popular instruction selection approaches over a diverse set of tasks and models, facilitating objective comparison. We evaluate these methods not only on accuracy metrics, but also on sensitivity metrics to glean additional insights. We recognize that other facets of prompting not covered by instruction engineering exist \cite{weichain, react, selfconsistency}, and defer these explorations to future work. 
\mySection{Related Works and Discussion}{}
\label{chap3:sec:discussion}

In this section we briefly discuss the similarities and differences of the model presented in this chapter, comparing it with some related work presented earlier (Chapter \ref{chap1:artifact-centric-bpm}). We will mention a few related studies and discuss directly; a more formal comparative study using qualitative and quantitative metrics should be the subject of future work.

Hull et al. \citeyearpar{hull2009facilitating} provide an interoperation framework in which, data are hosted on central infrastructures named \textit{artifact-centric hubs}. As in the work presented in this chapter, they propose mechanisms (including user views) for controlling access to these data. Compared to choreography-like approach as the one presented in this chapter, their settings has the advantage of providing a conceptual rendezvous point to exchange status information. The same purpose can be replicated in this chapter's approach by introducing a new type of agent called "\textit{monitor}", which will serve as a rendezvous point; the behaviour of the agents will therefore have to be slightly adapted to take into account the monitor and to preserve as much as possible the autonomy of agents.

Lohmann and Wolf \citeyearpar{lohmann2010artifact} abandon the concept of having a single artifact hub \cite{hull2009facilitating} and they introduce the idea of having several agents which operate on artifacts. Some of those artifacts are mobile; thus, the authors provide a systematic approach for modelling artifact location and its impact on the accessibility of actions using a Petri net. Even though we also manipulate mobile artifacts, we do not model artifact location; rather, our agents are equipped with capabilities that allow them to manipulate the artifacts appropriately (taking into account their location). Moreover, our approach considers that artifacts can not be remotely accessed, this increases the autonomy of agents.

The process design approach presented in this chapter, has some conceptual similarities with the concept of \textit{proclets} proposed by Wil M. P. van der Aalst et al. \citeyearpar{van2001proclets, van2009workflow}: they both split the process when designing it. In the model presented in this chapter, the process is split into execution scenarios and its specification consists in the diagramming of each of them. Proclets \cite{van2001proclets, van2009workflow} uses the concept of \textit{proclet-class} to model different levels of granularity and cardinality of processes. Additionally, proclets act like agents and are autonomous enough to decide how to interact with each other.

The model presented in this chapter uses an attributed grammar as its mathematical foundation. This is also the case of the AWGAG model by Badouel et al. \citeyearpar{badouel14, badouel2015active}. However, their model puts stress on modelling process data and users as first class citizens and it is designed for Adaptive Case Management.

To summarise, the proposed approach in this chapter allows the modelling and decentralized execution of administrative processes using autonomous agents. In it, process management is very simply done in two steps. The designer only needs to focus on modelling the artifacts in the form of task trees and the rest is easily deduced. Moreover, we propose a simple but powerful mechanism for securing data based on the notion of accreditation; this mechanism is perfectly composed with that of artifacts. The main strengths of our model are therefore : 
\begin{itemize}
	\item The simplicity of its syntax (process specification language), which moreover (well helped by the accreditation model), is suitable for administrative processes;
	\item The simplicity of its execution model; the latter is very close to the blockchain's execution model \cite{hull2017blockchain, mendling2018blockchains}. On condition of a formal study, the latter could possess the same qualities (fault tolerance, distributivity, security, peer autonomy, etc.) that emanate from the blockchain;
	\item Its formal character, which makes it verifiable using appropriate mathematical tools;
	\item The conformity of its execution model with the agent paradigm and service technology.
\end{itemize}
In view of all these benefits, we can say that the objectives set for this thesis have indeed been achieved. However, the proposed model is perfectible. For example, it can be modified to permit agents to respond incrementally to incoming requests as soon as any prefix of the extension of a bud is produced. This makes it possible to avoid the situation observed on figure \ref{chap3:fig:execution-figure-4} where the associated editor is informed of the evolution of the subtree resulting from $C$ only when this one is closed. All the criticisms we can make of the proposed model in particular, and of this thesis in general, have been introduced in the general conclusion (page \pageref{chap5:general-conclusion}) of this manuscript.





\section*{Acknowledgments}
We would like to thank Huiyi Hu, Alexandre Galashov, and Yee Whye Teh for helpful discussions and feedback on this project. 

\bibliography{refs}
\bibliographystyle{apalike}

\chapter{Supplementary Material}
\label{appendix}

In this appendix, we present supplementary material for the techniques and
experiments presented in the main text.

\section{Baseline Results and Analysis for Informed Sampler}
\label{appendix:chap3}

Here, we give an in-depth
performance analysis of the various samplers and the effect of their
hyperparameters. We choose hyperparameters with the lowest PSRF value
after $10k$ iterations, for each sampler individually. If the
differences between PSRF are not significantly different among
multiple values, we choose the one that has the highest acceptance
rate.

\subsection{Experiment: Estimating Camera Extrinsics}
\label{appendix:chap3:room}

\subsubsection{Parameter Selection}
\paragraph{Metropolis Hastings (\MH)}

Figure~\ref{fig:exp1_MH} shows the median acceptance rates and PSRF
values corresponding to various proposal standard deviations of plain
\MH~sampling. Mixing gets better and the acceptance rate gets worse as
the standard deviation increases. The value $0.3$ is selected standard
deviation for this sampler.

\paragraph{Metropolis Hastings Within Gibbs (\MHWG)}

As mentioned in Section~\ref{sec:room}, the \MHWG~sampler with one-dimensional
updates did not converge for any value of proposal standard deviation.
This problem has high correlation of the camera parameters and is of
multi-modal nature, which this sampler has problems with.

\paragraph{Parallel Tempering (\PT)}

For \PT~sampling, we took the best performing \MH~sampler and used
different temperature chains to improve the mixing of the
sampler. Figure~\ref{fig:exp1_PT} shows the results corresponding to
different combination of temperature levels. The sampler with
temperature levels of $[1,3,27]$ performed best in terms of both
mixing and acceptance rate.

\paragraph{Effect of Mixture Coefficient in Informed Sampling (\MIXLMH)}

Figure~\ref{fig:exp1_alpha} shows the effect of mixture
coefficient ($\alpha$) on the informed sampling
\MIXLMH. Since there is no significant different in PSRF values for
$0 \le \alpha \le 0.7$, we chose $0.7$ due to its high acceptance
rate.


% \end{multicols}

\begin{figure}[h]
\centering
  \subfigure[MH]{%
    \includegraphics[width=.48\textwidth]{figures/supplementary/camPose_MH.pdf} \label{fig:exp1_MH}
  }
  \subfigure[PT]{%
    \includegraphics[width=.48\textwidth]{figures/supplementary/camPose_PT.pdf} \label{fig:exp1_PT}
  }
\\
  \subfigure[INF-MH]{%
    \includegraphics[width=.48\textwidth]{figures/supplementary/camPose_alpha.pdf} \label{fig:exp1_alpha}
  }
  \mycaption{Results of the `Estimating Camera Extrinsics' experiment}{PRSFs and Acceptance rates corresponding to (a) various standard deviations of \MH, (b) various temperature level combinations of \PT sampling and (c) various mixture coefficients of \MIXLMH sampling.}
\end{figure}



\begin{figure}[!t]
\centering
  \subfigure[\MH]{%
    \includegraphics[width=.48\textwidth]{figures/supplementary/occlusionExp_MH.pdf} \label{fig:exp2_MH}
  }
  \subfigure[\BMHWG]{%
    \includegraphics[width=.48\textwidth]{figures/supplementary/occlusionExp_BMHWG.pdf} \label{fig:exp2_BMHWG}
  }
\\
  \subfigure[\MHWG]{%
    \includegraphics[width=.48\textwidth]{figures/supplementary/occlusionExp_MHWG.pdf} \label{fig:exp2_MHWG}
  }
  \subfigure[\PT]{%
    \includegraphics[width=.48\textwidth]{figures/supplementary/occlusionExp_PT.pdf} \label{fig:exp2_PT}
  }
\\
  \subfigure[\INFBMHWG]{%
    \includegraphics[width=.5\textwidth]{figures/supplementary/occlusionExp_alpha.pdf} \label{fig:exp2_alpha}
  }
  \mycaption{Results of the `Occluding Tiles' experiment}{PRSF and
    Acceptance rates corresponding to various standard deviations of
    (a) \MH, (b) \BMHWG, (c) \MHWG, (d) various temperature level
    combinations of \PT~sampling and; (e) various mixture coefficients
    of our informed \INFBMHWG sampling.}
\end{figure}

%\onecolumn\newpage\twocolumn
\subsection{Experiment: Occluding Tiles}
\label{appendix:chap3:tiles}

\subsubsection{Parameter Selection}

\paragraph{Metropolis Hastings (\MH)}

Figure~\ref{fig:exp2_MH} shows the results of
\MH~sampling. Results show the poor convergence for all proposal
standard deviations and rapid decrease of AR with increasing standard
deviation. This is due to the high-dimensional nature of
the problem. We selected a standard deviation of $1.1$.

\paragraph{Blocked Metropolis Hastings Within Gibbs (\BMHWG)}

The results of \BMHWG are shown in Figure~\ref{fig:exp2_BMHWG}. In
this sampler we update only one block of tile variables (of dimension
four) in each sampling step. Results show much better performance
compared to plain \MH. The optimal proposal standard deviation for
this sampler is $0.7$.

\paragraph{Metropolis Hastings Within Gibbs (\MHWG)}

Figure~\ref{fig:exp2_MHWG} shows the result of \MHWG sampling. This
sampler is better than \BMHWG and converges much more quickly. Here
a standard deviation of $0.9$ is found to be best.

\paragraph{Parallel Tempering (\PT)}

Figure~\ref{fig:exp2_PT} shows the results of \PT sampling with various
temperature combinations. Results show no improvement in AR from plain
\MH sampling and again $[1,3,27]$ temperature levels are found to be optimal.

\paragraph{Effect of Mixture Coefficient in Informed Sampling (\INFBMHWG)}

Figure~\ref{fig:exp2_alpha} shows the effect of mixture
coefficient ($\alpha$) on the blocked informed sampling
\INFBMHWG. Since there is no significant different in PSRF values for
$0 \le \alpha \le 0.8$, we chose $0.8$ due to its high acceptance
rate.



\subsection{Experiment: Estimating Body Shape}
\label{appendix:chap3:body}

\subsubsection{Parameter Selection}
\paragraph{Metropolis Hastings (\MH)}

Figure~\ref{fig:exp3_MH} shows the result of \MH~sampling with various
proposal standard deviations. The value of $0.1$ is found to be
best.

\paragraph{Metropolis Hastings Within Gibbs (\MHWG)}

For \MHWG sampling we select $0.3$ proposal standard
deviation. Results are shown in Fig.~\ref{fig:exp3_MHWG}.


\paragraph{Parallel Tempering (\PT)}

As before, results in Fig.~\ref{fig:exp3_PT}, the temperature levels
were selected to be $[1,3,27]$ due its slightly higher AR.

\paragraph{Effect of Mixture Coefficient in Informed Sampling (\MIXLMH)}

Figure~\ref{fig:exp3_alpha} shows the effect of $\alpha$ on PSRF and
AR. Since there is no significant differences in PSRF values for $0 \le
\alpha \le 0.8$, we choose $0.8$.


\begin{figure}[t]
\centering
  \subfigure[\MH]{%
    \includegraphics[width=.48\textwidth]{figures/supplementary/bodyShape_MH.pdf} \label{fig:exp3_MH}
  }
  \subfigure[\MHWG]{%
    \includegraphics[width=.48\textwidth]{figures/supplementary/bodyShape_MHWG.pdf} \label{fig:exp3_MHWG}
  }
\\
  \subfigure[\PT]{%
    \includegraphics[width=.48\textwidth]{figures/supplementary/bodyShape_PT.pdf} \label{fig:exp3_PT}
  }
  \subfigure[\MIXLMH]{%
    \includegraphics[width=.48\textwidth]{figures/supplementary/bodyShape_alpha.pdf} \label{fig:exp3_alpha}
  }
\\
  \mycaption{Results of the `Body Shape Estimation' experiment}{PRSFs and
    Acceptance rates corresponding to various standard deviations of
    (a) \MH, (b) \MHWG; (c) various temperature level combinations
    of \PT sampling and; (d) various mixture coefficients of the
    informed \MIXLMH sampling.}
\end{figure}


\subsection{Results Overview}
Figure~\ref{fig:exp_summary} shows the summary results of the all the three
experimental studies related to informed sampler.
\begin{figure*}[h!]
\centering
  \subfigure[Results for: Estimating Camera Extrinsics]{%
    \includegraphics[width=0.9\textwidth]{figures/supplementary/camPose_ALL.pdf} \label{fig:exp1_all}
  }
  \subfigure[Results for: Occluding Tiles]{%
    \includegraphics[width=0.9\textwidth]{figures/supplementary/occlusionExp_ALL.pdf} \label{fig:exp2_all}
  }
  \subfigure[Results for: Estimating Body Shape]{%
    \includegraphics[width=0.9\textwidth]{figures/supplementary/bodyShape_ALL.pdf} \label{fig:exp3_all}
  }
  \label{fig:exp_summary}
  \mycaption{Summary of the statistics for the three experiments}{Shown are
    for several baseline methods and the informed samplers the
    acceptance rates (left), PSRFs (middle), and RMSE values
    (right). All results are median results over multiple test
    examples.}
\end{figure*}

\subsection{Additional Qualitative Results}

\subsubsection{Occluding Tiles}
In Figure~\ref{fig:exp2_visual_more} more qualitative results of the
occluding tiles experiment are shown. The informed sampling approach
(\INFBMHWG) is better than the best baseline (\MHWG). This still is a
very challenging problem since the parameters for occluded tiles are
flat over a large region. Some of the posterior variance of the
occluded tiles is already captured by the informed sampler.

\begin{figure*}[h!]
\begin{center}
\centerline{\includegraphics[width=0.95\textwidth]{figures/supplementary/occlusionExp_Visual.pdf}}
\mycaption{Additional qualitative results of the occluding tiles experiment}
  {From left to right: (a)
  Given image, (b) Ground truth tiles, (c) OpenCV heuristic and most probable estimates
  from 5000 samples obtained by (d) MHWG sampler (best baseline) and
  (e) our INF-BMHWG sampler. (f) Posterior expectation of the tiles
  boundaries obtained by INF-BMHWG sampling (First 2000 samples are
  discarded as burn-in).}
\label{fig:exp2_visual_more}
\end{center}
\end{figure*}

\subsubsection{Body Shape}
Figure~\ref{fig:exp3_bodyMeshes} shows some more results of 3D mesh
reconstruction using posterior samples obtained by our informed
sampling \MIXLMH.

\begin{figure*}[t]
\begin{center}
\centerline{\includegraphics[width=0.75\textwidth]{figures/supplementary/bodyMeshResults.pdf}}
\mycaption{Qualitative results for the body shape experiment}
  {Shown is the 3D mesh reconstruction results with first 1000 samples obtained
  using the \MIXLMH informed sampling method. (blue indicates small
  values and red indicates high values)}
\label{fig:exp3_bodyMeshes}
\end{center}
\end{figure*}

\clearpage



\section{Additional Results on the Face Problem with CMP}

Figure~\ref{fig:shading-qualitative-multiple-subjects-supp} shows inference results for reflectance maps, normal maps and lights for randomly chosen test images, and Fig.~\ref{fig:shading-qualitative-same-subject-supp} shows reflectance estimation results on multiple images of the same subject produced under different illumination conditions. CMP is able to produce estimates that are closer to the groundtruth across different subjects and illumination conditions.

\begin{figure*}[h]
  \begin{center}
  \centerline{\includegraphics[width=1.0\columnwidth]{figures/face_cmp_visual_results_supp.pdf}}
  \vspace{-1.2cm}
  \end{center}
	\mycaption{A visual comparison of inference results}{(a)~Observed images. (b)~Inferred reflectance maps. \textit{GT} is the photometric stereo groundtruth, \textit{BU} is the Biswas \etal (2009) reflectance estimate and \textit{Forest} is the consensus prediction. (c)~The variance of the inferred reflectance estimate produced by \MTD (normalized across rows).(d)~Visualization of inferred light directions. (e)~Inferred normal maps.}
	\label{fig:shading-qualitative-multiple-subjects-supp}
\end{figure*}


\begin{figure*}[h]
	\centering
	\setlength\fboxsep{0.2mm}
	\setlength\fboxrule{0pt}
	\begin{tikzpicture}

		\matrix at (0, 0) [matrix of nodes, nodes={anchor=east}, column sep=-0.05cm, row sep=-0.2cm]
		{
			\fbox{\includegraphics[width=1cm]{figures/sample_3_4_X.png}} &
			\fbox{\includegraphics[width=1cm]{figures/sample_3_4_GT.png}} &
			\fbox{\includegraphics[width=1cm]{figures/sample_3_4_BISWAS.png}}  &
			\fbox{\includegraphics[width=1cm]{figures/sample_3_4_VMP.png}}  &
			\fbox{\includegraphics[width=1cm]{figures/sample_3_4_FOREST.png}}  &
			\fbox{\includegraphics[width=1cm]{figures/sample_3_4_CMP.png}}  &
			\fbox{\includegraphics[width=1cm]{figures/sample_3_4_CMPVAR.png}}
			 \\

			\fbox{\includegraphics[width=1cm]{figures/sample_3_5_X.png}} &
			\fbox{\includegraphics[width=1cm]{figures/sample_3_5_GT.png}} &
			\fbox{\includegraphics[width=1cm]{figures/sample_3_5_BISWAS.png}}  &
			\fbox{\includegraphics[width=1cm]{figures/sample_3_5_VMP.png}}  &
			\fbox{\includegraphics[width=1cm]{figures/sample_3_5_FOREST.png}}  &
			\fbox{\includegraphics[width=1cm]{figures/sample_3_5_CMP.png}}  &
			\fbox{\includegraphics[width=1cm]{figures/sample_3_5_CMPVAR.png}}
			 \\

			\fbox{\includegraphics[width=1cm]{figures/sample_3_6_X.png}} &
			\fbox{\includegraphics[width=1cm]{figures/sample_3_6_GT.png}} &
			\fbox{\includegraphics[width=1cm]{figures/sample_3_6_BISWAS.png}}  &
			\fbox{\includegraphics[width=1cm]{figures/sample_3_6_VMP.png}}  &
			\fbox{\includegraphics[width=1cm]{figures/sample_3_6_FOREST.png}}  &
			\fbox{\includegraphics[width=1cm]{figures/sample_3_6_CMP.png}}  &
			\fbox{\includegraphics[width=1cm]{figures/sample_3_6_CMPVAR.png}}
			 \\
	     };

       \node at (-3.85, -2.0) {\small Observed};
       \node at (-2.55, -2.0) {\small `GT'};
       \node at (-1.27, -2.0) {\small BU};
       \node at (0.0, -2.0) {\small MP};
       \node at (1.27, -2.0) {\small Forest};
       \node at (2.55, -2.0) {\small \textbf{CMP}};
       \node at (3.85, -2.0) {\small Variance};

	\end{tikzpicture}
	\mycaption{Robustness to varying illumination}{Reflectance estimation on a subject images with varying illumination. Left to right: observed image, photometric stereo estimate (GT)
  which is used as a proxy for groundtruth, bottom-up estimate of \cite{Biswas2009}, VMP result, consensus forest estimate, CMP mean, and CMP variance.}
	\label{fig:shading-qualitative-same-subject-supp}
\end{figure*}

\clearpage

\section{Additional Material for Learning Sparse High Dimensional Filters}
\label{sec:appendix-bnn}

This part of supplementary material contains a more detailed overview of the permutohedral
lattice convolution in Section~\ref{sec:permconv}, more experiments in
Section~\ref{sec:addexps} and additional results with protocols for
the experiments presented in Chapter~\ref{chap:bnn} in Section~\ref{sec:addresults}.

\vspace{-0.2cm}
\subsection{General Permutohedral Convolutions}
\label{sec:permconv}

A core technical contribution of this work is the generalization of the Gaussian permutohedral lattice
convolution proposed in~\cite{adams2010fast} to the full non-separable case with the
ability to perform back-propagation. Although, conceptually, there are minor
differences between Gaussian and general parameterized filters, there are non-trivial practical
differences in terms of the algorithmic implementation. The Gauss filters belong to
the separable class and can thus be decomposed into multiple
sequential one dimensional convolutions. We are interested in the general filter
convolutions, which can not be decomposed. Thus, performing a general permutohedral
convolution at a lattice point requires the computation of the inner product with the
neighboring elements in all the directions in the high-dimensional space.

Here, we give more details of the implementation differences of separable
and non-separable filters. In the following, we will explain the scalar case first.
Recall, that the forward pass of general permutohedral convolution
involves 3 steps: \textit{splatting}, \textit{convolving} and \textit{slicing}.
We follow the same splatting and slicing strategies as in~\cite{adams2010fast}
since these operations do not depend on the filter kernel. The main difference
between our work and the existing implementation of~\cite{adams2010fast} is
the way that the convolution operation is executed. This proceeds by constructing
a \emph{blur neighbor} matrix $K$ that stores for every lattice point all
values of the lattice neighbors that are needed to compute the filter output.

\begin{figure}[t!]
  \centering
    \includegraphics[width=0.6\columnwidth]{figures/supplementary/lattice_construction}
  \mycaption{Illustration of 1D permutohedral lattice construction}
  {A $4\times 4$ $(x,y)$ grid lattice is projected onto the plane defined by the normal
  vector $(1,1)^{\top}$. This grid has $s+1=4$ and $d=2$ $(s+1)^{d}=4^2=16$ elements.
  In the projection, all points of the same color are projected onto the same points in the plane.
  The number of elements of the projected lattice is $t=(s+1)^d-s^d=4^2-3^2=7$, that is
  the $(4\times 4)$ grid minus the size of lattice that is $1$ smaller at each size, in this
  case a $(3\times 3)$ lattice (the upper right $(3\times 3)$ elements).
  }
\label{fig:latticeconstruction}
\end{figure}

The blur neighbor matrix is constructed by traversing through all the populated
lattice points and their neighboring elements.
% For efficiency, we do this matrix construction recursively with shared computations
% since $n^{th}$ neighbourhood elements are $1^{st}$ neighborhood elements of $n-1^{th}$ neighbourhood elements. \pg{do not understand}
This is done recursively to share computations. For any lattice point, the neighbors that are
$n$ hops away are the direct neighbors of the points that are $n-1$ hops away.
The size of a $d$ dimensional spatial filter with width $s+1$ is $(s+1)^{d}$ (\eg, a
$3\times 3$ filter, $s=2$ in $d=2$ has $3^2=9$ elements) and this size grows
exponentially in the number of dimensions $d$. The permutohedral lattice is constructed by
projecting a regular grid onto the plane spanned by the $d$ dimensional normal vector ${(1,\ldots,1)}^{\top}$. See
Fig.~\ref{fig:latticeconstruction} for an illustration of the 1D lattice construction.
Many corners of a grid filter are projected onto the same point, in total $t = {(s+1)}^{d} -
s^{d}$ elements remain in the permutohedral filter with $s$ neighborhood in $d-1$ dimensions.
If the lattice has $m$ populated elements, the
matrix $K$ has size $t\times m$. Note that, since the input signal is typically
sparse, only a few lattice corners are being populated in the \textit{slicing} step.
We use a hash-table to keep track of these points and traverse only through
the populated lattice points for this neighborhood matrix construction.

Once the blur neighbor matrix $K$ is constructed, we can perform the convolution
by the matrix vector multiplication
\begin{equation}
\ell' = BK,
\label{eq:conv}
\end{equation}
where $B$ is the $1 \times t$ filter kernel (whose values we will learn) and $\ell'\in\mathbb{R}^{1\times m}$
is the result of the filtering at the $m$ lattice points. In practice, we found that the
matrix $K$ is sometimes too large to fit into GPU memory and we divided the matrix $K$
into smaller pieces to compute Eq.~\ref{eq:conv} sequentially.

In the general multi-dimensional case, the signal $\ell$ is of $c$ dimensions. Then
the kernel $B$ is of size $c \times t$ and $K$ stores the $c$ dimensional vectors
accordingly. When the input and output points are different, we slice only the
input points and splat only at the output points.


\subsection{Additional Experiments}
\label{sec:addexps}
In this section, we discuss more use-cases for the learned bilateral filters, one
use-case of BNNs and two single filter applications for image and 3D mesh denoising.

\subsubsection{Recognition of subsampled MNIST}\label{sec:app_mnist}

One of the strengths of the proposed filter convolution is that it does not
require the input to lie on a regular grid. The only requirement is to define a distance
between features of the input signal.
We highlight this feature with the following experiment using the
classical MNIST ten class classification problem~\cite{lecun1998mnist}. We sample a
sparse set of $N$ points $(x,y)\in [0,1]\times [0,1]$
uniformly at random in the input image, use their interpolated values
as signal and the \emph{continuous} $(x,y)$ positions as features. This mimics
sub-sampling of a high-dimensional signal. To compare against a spatial convolution,
we interpolate the sparse set of values at the grid positions.

We take a reference implementation of LeNet~\cite{lecun1998gradient} that
is part of the Caffe project~\cite{jia2014caffe} and compare it
against the same architecture but replacing the first convolutional
layer with a bilateral convolution layer (BCL). The filter size
and numbers are adjusted to get a comparable number of parameters
($5\times 5$ for LeNet, $2$-neighborhood for BCL).

The results are shown in Table~\ref{tab:all-results}. We see that training
on the original MNIST data (column Original, LeNet vs. BNN) leads to a slight
decrease in performance of the BNN (99.03\%) compared to LeNet
(99.19\%). The BNN can be trained and evaluated on sparse
signals, and we resample the image as described above for $N=$ 100\%, 60\% and
20\% of the total number of pixels. The methods are also evaluated
on test images that are subsampled in the same way. Note that we can
train and test with different subsampling rates. We introduce an additional
bilinear interpolation layer for the LeNet architecture to train on the same
data. In essence, both models perform a spatial interpolation and thus we
expect them to yield a similar classification accuracy. Once the data is of
higher dimensions, the permutohedral convolution will be faster due to hashing
the sparse input points, as well as less memory demanding in comparison to
naive application of a spatial convolution with interpolated values.

\begin{table}[t]
  \begin{center}
    \footnotesize
    \centering
    \begin{tabular}[t]{lllll}
      \toprule
              &     & \multicolumn{3}{c}{Test Subsampling} \\
       Method  & Original & 100\% & 60\% & 20\%\\
      \midrule
       LeNet &  \textbf{0.9919} & 0.9660 & 0.9348 & \textbf{0.6434} \\
       BNN &  0.9903 & \textbf{0.9844} & \textbf{0.9534} & 0.5767 \\
      \hline
       LeNet 100\% & 0.9856 & 0.9809 & 0.9678 & \textbf{0.7386} \\
       BNN 100\% & \textbf{0.9900} & \textbf{0.9863} & \textbf{0.9699} & 0.6910 \\
      \hline
       LeNet 60\% & 0.9848 & 0.9821 & 0.9740 & 0.8151 \\
       BNN 60\% & \textbf{0.9885} & \textbf{0.9864} & \textbf{0.9771} & \textbf{0.8214}\\
      \hline
       LeNet 20\% & \textbf{0.9763} & \textbf{0.9754} & 0.9695 & 0.8928 \\
       BNN 20\% & 0.9728 & 0.9735 & \textbf{0.9701} & \textbf{0.9042}\\
      \bottomrule
    \end{tabular}
  \end{center}
\vspace{-.2cm}
\caption{Classification accuracy on MNIST. We compare the
    LeNet~\cite{lecun1998gradient} implementation that is part of
    Caffe~\cite{jia2014caffe} to the network with the first layer
    replaced by a bilateral convolution layer (BCL). Both are trained
    on the original image resolution (first two rows). Three more BNN
    and CNN models are trained with randomly subsampled images (100\%,
    60\% and 20\% of the pixels). An additional bilinear interpolation
    layer samples the input signal on a spatial grid for the CNN model.
  }
  \label{tab:all-results}
\vspace{-.5cm}
\end{table}

\subsubsection{Image Denoising}

The main application that inspired the development of the bilateral
filtering operation is image denoising~\cite{aurich1995non}, there
using a single Gaussian kernel. Our development allows to learn this
kernel function from data and we explore how to improve using a \emph{single}
but more general bilateral filter.

We use the Berkeley segmentation dataset
(BSDS500)~\cite{arbelaezi2011bsds500} as a test bed. The color
images in the dataset are converted to gray-scale,
and corrupted with Gaussian noise with a standard deviation of
$\frac {25} {255}$.

We compare the performance of four different filter models on a
denoising task.
The first baseline model (`Spatial' in Table \ref{tab:denoising}, $25$
weights) uses a single spatial filter with a kernel size of
$5$ and predicts the scalar gray-scale value at the center pixel. The next model
(`Gauss Bilateral') applies a bilateral \emph{Gaussian}
filter to the noisy input, using position and intensity features $\f=(x,y,v)^\top$.
The third setup (`Learned Bilateral', $65$ weights)
takes a Gauss kernel as initialization and
fits all filter weights on the train set to minimize the
mean squared error with respect to the clean images.
We run a combination
of spatial and permutohedral convolutions on spatial and bilateral
features (`Spatial + Bilateral (Learned)') to check for a complementary
performance of the two convolutions.

\label{sec:exp:denoising}
\begin{table}[!h]
\begin{center}
  \footnotesize
  \begin{tabular}[t]{lr}
    \toprule
    Method & PSNR \\
    \midrule
    Noisy Input & $20.17$ \\
    Spatial & $26.27$ \\
    Gauss Bilateral & $26.51$ \\
    Learned Bilateral & $26.58$ \\
    Spatial + Bilateral (Learned) & \textbf{$26.65$} \\
    \bottomrule
  \end{tabular}
\end{center}
\vspace{-0.5em}
\caption{PSNR results of a denoising task using the BSDS500
  dataset~\cite{arbelaezi2011bsds500}}
\vspace{-0.5em}
\label{tab:denoising}
\end{table}
\vspace{-0.2em}

The PSNR scores evaluated on full images of the test set are
shown in Table \ref{tab:denoising}. We find that an untrained bilateral
filter already performs better than a trained spatial convolution
($26.27$ to $26.51$). A learned convolution further improve the
performance slightly. We chose this simple one-kernel setup to
validate an advantage of the generalized bilateral filter. A competitive
denoising system would employ RGB color information and also
needs to be properly adjusted in network size. Multi-layer perceptrons
have obtained state-of-the-art denoising results~\cite{burger12cvpr}
and the permutohedral lattice layer can readily be used in such an
architecture, which is intended future work.

\subsection{Additional results}
\label{sec:addresults}

This section contains more qualitative results for the experiments presented in Chapter~\ref{chap:bnn}.

\begin{figure*}[th!]
  \centering
    \includegraphics[width=\columnwidth,trim={5cm 2.5cm 5cm 4.5cm},clip]{figures/supplementary/lattice_viz.pdf}
    \vspace{-0.7cm}
  \mycaption{Visualization of the Permutohedral Lattice}
  {Sample lattice visualizations for different feature spaces. All pixels falling in the same simplex cell are shown with
  the same color. $(x,y)$ features correspond to image pixel positions, and $(r,g,b) \in [0,255]$ correspond
  to the red, green and blue color values.}
\label{fig:latticeviz}
\end{figure*}

\subsubsection{Lattice Visualization}

Figure~\ref{fig:latticeviz} shows sample lattice visualizations for different feature spaces.

\newcolumntype{L}[1]{>{\raggedright\let\newline\\\arraybackslash\hspace{0pt}}b{#1}}
\newcolumntype{C}[1]{>{\centering\let\newline\\\arraybackslash\hspace{0pt}}b{#1}}
\newcolumntype{R}[1]{>{\raggedleft\let\newline\\\arraybackslash\hspace{0pt}}b{#1}}

\subsubsection{Color Upsampling}\label{sec:color_upsampling}
\label{sec:col_upsample_extra}

Some images of the upsampling for the Pascal
VOC12 dataset are shown in Fig.~\ref{fig:Colour_upsample_visuals}. It is
especially the low level image details that are better preserved with
a learned bilateral filter compared to the Gaussian case.

\begin{figure*}[t!]
  \centering
    \subfigure{%
   \raisebox{2.0em}{
    \includegraphics[width=.06\columnwidth]{figures/supplementary/2007_004969.jpg}
   }
  }
  \subfigure{%
    \includegraphics[width=.17\columnwidth]{figures/supplementary/2007_004969_gray.pdf}
  }
  \subfigure{%
    \includegraphics[width=.17\columnwidth]{figures/supplementary/2007_004969_gt.pdf}
  }
  \subfigure{%
    \includegraphics[width=.17\columnwidth]{figures/supplementary/2007_004969_bicubic.pdf}
  }
  \subfigure{%
    \includegraphics[width=.17\columnwidth]{figures/supplementary/2007_004969_gauss.pdf}
  }
  \subfigure{%
    \includegraphics[width=.17\columnwidth]{figures/supplementary/2007_004969_learnt.pdf}
  }\\
    \subfigure{%
   \raisebox{2.0em}{
    \includegraphics[width=.06\columnwidth]{figures/supplementary/2007_003106.jpg}
   }
  }
  \subfigure{%
    \includegraphics[width=.17\columnwidth]{figures/supplementary/2007_003106_gray.pdf}
  }
  \subfigure{%
    \includegraphics[width=.17\columnwidth]{figures/supplementary/2007_003106_gt.pdf}
  }
  \subfigure{%
    \includegraphics[width=.17\columnwidth]{figures/supplementary/2007_003106_bicubic.pdf}
  }
  \subfigure{%
    \includegraphics[width=.17\columnwidth]{figures/supplementary/2007_003106_gauss.pdf}
  }
  \subfigure{%
    \includegraphics[width=.17\columnwidth]{figures/supplementary/2007_003106_learnt.pdf}
  }\\
  \setcounter{subfigure}{0}
  \small{
  \subfigure[Inp.]{%
  \raisebox{2.0em}{
    \includegraphics[width=.06\columnwidth]{figures/supplementary/2007_006837.jpg}
   }
  }
  \subfigure[Guidance]{%
    \includegraphics[width=.17\columnwidth]{figures/supplementary/2007_006837_gray.pdf}
  }
   \subfigure[GT]{%
    \includegraphics[width=.17\columnwidth]{figures/supplementary/2007_006837_gt.pdf}
  }
  \subfigure[Bicubic]{%
    \includegraphics[width=.17\columnwidth]{figures/supplementary/2007_006837_bicubic.pdf}
  }
  \subfigure[Gauss-BF]{%
    \includegraphics[width=.17\columnwidth]{figures/supplementary/2007_006837_gauss.pdf}
  }
  \subfigure[Learned-BF]{%
    \includegraphics[width=.17\columnwidth]{figures/supplementary/2007_006837_learnt.pdf}
  }
  }
  \vspace{-0.5cm}
  \mycaption{Color Upsampling}{Color $8\times$ upsampling results
  using different methods, from left to right, (a)~Low-resolution input color image (Inp.),
  (b)~Gray scale guidance image, (c)~Ground-truth color image; Upsampled color images with
  (d)~Bicubic interpolation, (e) Gauss bilateral upsampling and, (f)~Learned bilateral
  updampgling (best viewed on screen).}

\label{fig:Colour_upsample_visuals}
\end{figure*}

\subsubsection{Depth Upsampling}
\label{sec:depth_upsample_extra}

Figure~\ref{fig:depth_upsample_visuals} presents some more qualitative results comparing bicubic interpolation, Gauss
bilateral and learned bilateral upsampling on NYU depth dataset image~\cite{silberman2012indoor}.

\subsubsection{Character Recognition}\label{sec:app_character}

 Figure~\ref{fig:nnrecognition} shows the schematic of different layers
 of the network architecture for LeNet-7~\cite{lecun1998mnist}
 and DeepCNet(5, 50)~\cite{ciresan2012multi,graham2014spatially}. For the BNN variants, the first layer filters are replaced
 with learned bilateral filters and are learned end-to-end.

\subsubsection{Semantic Segmentation}\label{sec:app_semantic_segmentation}
\label{sec:semantic_bnn_extra}

Some more visual results for semantic segmentation are shown in Figure~\ref{fig:semantic_visuals}.
These include the underlying DeepLab CNN\cite{chen2014semantic} result (DeepLab),
the 2 step mean-field result with Gaussian edge potentials (+2stepMF-GaussCRF)
and also corresponding results with learned edge potentials (+2stepMF-LearnedCRF).
In general, we observe that mean-field in learned CRF leads to slightly dilated
classification regions in comparison to using Gaussian CRF thereby filling-in the
false negative pixels and also correcting some mis-classified regions.

\begin{figure*}[t!]
  \centering
    \subfigure{%
   \raisebox{2.0em}{
    \includegraphics[width=.06\columnwidth]{figures/supplementary/2bicubic}
   }
  }
  \subfigure{%
    \includegraphics[width=.17\columnwidth]{figures/supplementary/2given_image}
  }
  \subfigure{%
    \includegraphics[width=.17\columnwidth]{figures/supplementary/2ground_truth}
  }
  \subfigure{%
    \includegraphics[width=.17\columnwidth]{figures/supplementary/2bicubic}
  }
  \subfigure{%
    \includegraphics[width=.17\columnwidth]{figures/supplementary/2gauss}
  }
  \subfigure{%
    \includegraphics[width=.17\columnwidth]{figures/supplementary/2learnt}
  }\\
    \subfigure{%
   \raisebox{2.0em}{
    \includegraphics[width=.06\columnwidth]{figures/supplementary/32bicubic}
   }
  }
  \subfigure{%
    \includegraphics[width=.17\columnwidth]{figures/supplementary/32given_image}
  }
  \subfigure{%
    \includegraphics[width=.17\columnwidth]{figures/supplementary/32ground_truth}
  }
  \subfigure{%
    \includegraphics[width=.17\columnwidth]{figures/supplementary/32bicubic}
  }
  \subfigure{%
    \includegraphics[width=.17\columnwidth]{figures/supplementary/32gauss}
  }
  \subfigure{%
    \includegraphics[width=.17\columnwidth]{figures/supplementary/32learnt}
  }\\
  \setcounter{subfigure}{0}
  \small{
  \subfigure[Inp.]{%
  \raisebox{2.0em}{
    \includegraphics[width=.06\columnwidth]{figures/supplementary/41bicubic}
   }
  }
  \subfigure[Guidance]{%
    \includegraphics[width=.17\columnwidth]{figures/supplementary/41given_image}
  }
   \subfigure[GT]{%
    \includegraphics[width=.17\columnwidth]{figures/supplementary/41ground_truth}
  }
  \subfigure[Bicubic]{%
    \includegraphics[width=.17\columnwidth]{figures/supplementary/41bicubic}
  }
  \subfigure[Gauss-BF]{%
    \includegraphics[width=.17\columnwidth]{figures/supplementary/41gauss}
  }
  \subfigure[Learned-BF]{%
    \includegraphics[width=.17\columnwidth]{figures/supplementary/41learnt}
  }
  }
  \mycaption{Depth Upsampling}{Depth $8\times$ upsampling results
  using different upsampling strategies, from left to right,
  (a)~Low-resolution input depth image (Inp.),
  (b)~High-resolution guidance image, (c)~Ground-truth depth; Upsampled depth images with
  (d)~Bicubic interpolation, (e) Gauss bilateral upsampling and, (f)~Learned bilateral
  updampgling (best viewed on screen).}

\label{fig:depth_upsample_visuals}
\end{figure*}

\subsubsection{Material Segmentation}\label{sec:app_material_segmentation}
\label{sec:material_bnn_extra}

In Fig.~\ref{fig:material_visuals-app2}, we present visual results comparing 2 step
mean-field inference with Gaussian and learned pairwise CRF potentials. In
general, we observe that the pixels belonging to dominant classes in the
training data are being more accurately classified with learned CRF. This leads to
a significant improvements in overall pixel accuracy. This also results
in a slight decrease of the accuracy from less frequent class pixels thereby
slightly reducing the average class accuracy with learning. We attribute this
to the type of annotation that is available for this dataset, which is not
for the entire image but for some segments in the image. We have very few
images of the infrequent classes to combat this behaviour during training.

\subsubsection{Experiment Protocols}
\label{sec:protocols}

Table~\ref{tbl:parameters} shows experiment protocols of different experiments.

 \begin{figure*}[t!]
  \centering
  \subfigure[LeNet-7]{
    \includegraphics[width=0.7\columnwidth]{figures/supplementary/lenet_cnn_network}
    }\\
    \subfigure[DeepCNet]{
    \includegraphics[width=\columnwidth]{figures/supplementary/deepcnet_cnn_network}
    }
  \mycaption{CNNs for Character Recognition}
  {Schematic of (top) LeNet-7~\cite{lecun1998mnist} and (bottom) DeepCNet(5,50)~\cite{ciresan2012multi,graham2014spatially} architectures used in Assamese
  character recognition experiments.}
\label{fig:nnrecognition}
\end{figure*}

\definecolor{voc_1}{RGB}{0, 0, 0}
\definecolor{voc_2}{RGB}{128, 0, 0}
\definecolor{voc_3}{RGB}{0, 128, 0}
\definecolor{voc_4}{RGB}{128, 128, 0}
\definecolor{voc_5}{RGB}{0, 0, 128}
\definecolor{voc_6}{RGB}{128, 0, 128}
\definecolor{voc_7}{RGB}{0, 128, 128}
\definecolor{voc_8}{RGB}{128, 128, 128}
\definecolor{voc_9}{RGB}{64, 0, 0}
\definecolor{voc_10}{RGB}{192, 0, 0}
\definecolor{voc_11}{RGB}{64, 128, 0}
\definecolor{voc_12}{RGB}{192, 128, 0}
\definecolor{voc_13}{RGB}{64, 0, 128}
\definecolor{voc_14}{RGB}{192, 0, 128}
\definecolor{voc_15}{RGB}{64, 128, 128}
\definecolor{voc_16}{RGB}{192, 128, 128}
\definecolor{voc_17}{RGB}{0, 64, 0}
\definecolor{voc_18}{RGB}{128, 64, 0}
\definecolor{voc_19}{RGB}{0, 192, 0}
\definecolor{voc_20}{RGB}{128, 192, 0}
\definecolor{voc_21}{RGB}{0, 64, 128}
\definecolor{voc_22}{RGB}{128, 64, 128}

\begin{figure*}[t]
  \centering
  \small{
  \fcolorbox{white}{voc_1}{\rule{0pt}{6pt}\rule{6pt}{0pt}} Background~~
  \fcolorbox{white}{voc_2}{\rule{0pt}{6pt}\rule{6pt}{0pt}} Aeroplane~~
  \fcolorbox{white}{voc_3}{\rule{0pt}{6pt}\rule{6pt}{0pt}} Bicycle~~
  \fcolorbox{white}{voc_4}{\rule{0pt}{6pt}\rule{6pt}{0pt}} Bird~~
  \fcolorbox{white}{voc_5}{\rule{0pt}{6pt}\rule{6pt}{0pt}} Boat~~
  \fcolorbox{white}{voc_6}{\rule{0pt}{6pt}\rule{6pt}{0pt}} Bottle~~
  \fcolorbox{white}{voc_7}{\rule{0pt}{6pt}\rule{6pt}{0pt}} Bus~~
  \fcolorbox{white}{voc_8}{\rule{0pt}{6pt}\rule{6pt}{0pt}} Car~~ \\
  \fcolorbox{white}{voc_9}{\rule{0pt}{6pt}\rule{6pt}{0pt}} Cat~~
  \fcolorbox{white}{voc_10}{\rule{0pt}{6pt}\rule{6pt}{0pt}} Chair~~
  \fcolorbox{white}{voc_11}{\rule{0pt}{6pt}\rule{6pt}{0pt}} Cow~~
  \fcolorbox{white}{voc_12}{\rule{0pt}{6pt}\rule{6pt}{0pt}} Dining Table~~
  \fcolorbox{white}{voc_13}{\rule{0pt}{6pt}\rule{6pt}{0pt}} Dog~~
  \fcolorbox{white}{voc_14}{\rule{0pt}{6pt}\rule{6pt}{0pt}} Horse~~
  \fcolorbox{white}{voc_15}{\rule{0pt}{6pt}\rule{6pt}{0pt}} Motorbike~~
  \fcolorbox{white}{voc_16}{\rule{0pt}{6pt}\rule{6pt}{0pt}} Person~~ \\
  \fcolorbox{white}{voc_17}{\rule{0pt}{6pt}\rule{6pt}{0pt}} Potted Plant~~
  \fcolorbox{white}{voc_18}{\rule{0pt}{6pt}\rule{6pt}{0pt}} Sheep~~
  \fcolorbox{white}{voc_19}{\rule{0pt}{6pt}\rule{6pt}{0pt}} Sofa~~
  \fcolorbox{white}{voc_20}{\rule{0pt}{6pt}\rule{6pt}{0pt}} Train~~
  \fcolorbox{white}{voc_21}{\rule{0pt}{6pt}\rule{6pt}{0pt}} TV monitor~~ \\
  }
  \subfigure{%
    \includegraphics[width=.18\columnwidth]{figures/supplementary/2007_001423_given.jpg}
  }
  \subfigure{%
    \includegraphics[width=.18\columnwidth]{figures/supplementary/2007_001423_gt.png}
  }
  \subfigure{%
    \includegraphics[width=.18\columnwidth]{figures/supplementary/2007_001423_cnn.png}
  }
  \subfigure{%
    \includegraphics[width=.18\columnwidth]{figures/supplementary/2007_001423_gauss.png}
  }
  \subfigure{%
    \includegraphics[width=.18\columnwidth]{figures/supplementary/2007_001423_learnt.png}
  }\\
  \subfigure{%
    \includegraphics[width=.18\columnwidth]{figures/supplementary/2007_001430_given.jpg}
  }
  \subfigure{%
    \includegraphics[width=.18\columnwidth]{figures/supplementary/2007_001430_gt.png}
  }
  \subfigure{%
    \includegraphics[width=.18\columnwidth]{figures/supplementary/2007_001430_cnn.png}
  }
  \subfigure{%
    \includegraphics[width=.18\columnwidth]{figures/supplementary/2007_001430_gauss.png}
  }
  \subfigure{%
    \includegraphics[width=.18\columnwidth]{figures/supplementary/2007_001430_learnt.png}
  }\\
    \subfigure{%
    \includegraphics[width=.18\columnwidth]{figures/supplementary/2007_007996_given.jpg}
  }
  \subfigure{%
    \includegraphics[width=.18\columnwidth]{figures/supplementary/2007_007996_gt.png}
  }
  \subfigure{%
    \includegraphics[width=.18\columnwidth]{figures/supplementary/2007_007996_cnn.png}
  }
  \subfigure{%
    \includegraphics[width=.18\columnwidth]{figures/supplementary/2007_007996_gauss.png}
  }
  \subfigure{%
    \includegraphics[width=.18\columnwidth]{figures/supplementary/2007_007996_learnt.png}
  }\\
   \subfigure{%
    \includegraphics[width=.18\columnwidth]{figures/supplementary/2010_002682_given.jpg}
  }
  \subfigure{%
    \includegraphics[width=.18\columnwidth]{figures/supplementary/2010_002682_gt.png}
  }
  \subfigure{%
    \includegraphics[width=.18\columnwidth]{figures/supplementary/2010_002682_cnn.png}
  }
  \subfigure{%
    \includegraphics[width=.18\columnwidth]{figures/supplementary/2010_002682_gauss.png}
  }
  \subfigure{%
    \includegraphics[width=.18\columnwidth]{figures/supplementary/2010_002682_learnt.png}
  }\\
     \subfigure{%
    \includegraphics[width=.18\columnwidth]{figures/supplementary/2010_004789_given.jpg}
  }
  \subfigure{%
    \includegraphics[width=.18\columnwidth]{figures/supplementary/2010_004789_gt.png}
  }
  \subfigure{%
    \includegraphics[width=.18\columnwidth]{figures/supplementary/2010_004789_cnn.png}
  }
  \subfigure{%
    \includegraphics[width=.18\columnwidth]{figures/supplementary/2010_004789_gauss.png}
  }
  \subfigure{%
    \includegraphics[width=.18\columnwidth]{figures/supplementary/2010_004789_learnt.png}
  }\\
       \subfigure{%
    \includegraphics[width=.18\columnwidth]{figures/supplementary/2007_001311_given.jpg}
  }
  \subfigure{%
    \includegraphics[width=.18\columnwidth]{figures/supplementary/2007_001311_gt.png}
  }
  \subfigure{%
    \includegraphics[width=.18\columnwidth]{figures/supplementary/2007_001311_cnn.png}
  }
  \subfigure{%
    \includegraphics[width=.18\columnwidth]{figures/supplementary/2007_001311_gauss.png}
  }
  \subfigure{%
    \includegraphics[width=.18\columnwidth]{figures/supplementary/2007_001311_learnt.png}
  }\\
  \setcounter{subfigure}{0}
  \subfigure[Input]{%
    \includegraphics[width=.18\columnwidth]{figures/supplementary/2010_003531_given.jpg}
  }
  \subfigure[Ground Truth]{%
    \includegraphics[width=.18\columnwidth]{figures/supplementary/2010_003531_gt.png}
  }
  \subfigure[DeepLab]{%
    \includegraphics[width=.18\columnwidth]{figures/supplementary/2010_003531_cnn.png}
  }
  \subfigure[+GaussCRF]{%
    \includegraphics[width=.18\columnwidth]{figures/supplementary/2010_003531_gauss.png}
  }
  \subfigure[+LearnedCRF]{%
    \includegraphics[width=.18\columnwidth]{figures/supplementary/2010_003531_learnt.png}
  }
  \vspace{-0.3cm}
  \mycaption{Semantic Segmentation}{Example results of semantic segmentation.
  (c)~depicts the unary results before application of MF, (d)~after two steps of MF with Gaussian edge CRF potentials, (e)~after
  two steps of MF with learned edge CRF potentials.}
    \label{fig:semantic_visuals}
\end{figure*}


\definecolor{minc_1}{HTML}{771111}
\definecolor{minc_2}{HTML}{CAC690}
\definecolor{minc_3}{HTML}{EEEEEE}
\definecolor{minc_4}{HTML}{7C8FA6}
\definecolor{minc_5}{HTML}{597D31}
\definecolor{minc_6}{HTML}{104410}
\definecolor{minc_7}{HTML}{BB819C}
\definecolor{minc_8}{HTML}{D0CE48}
\definecolor{minc_9}{HTML}{622745}
\definecolor{minc_10}{HTML}{666666}
\definecolor{minc_11}{HTML}{D54A31}
\definecolor{minc_12}{HTML}{101044}
\definecolor{minc_13}{HTML}{444126}
\definecolor{minc_14}{HTML}{75D646}
\definecolor{minc_15}{HTML}{DD4348}
\definecolor{minc_16}{HTML}{5C8577}
\definecolor{minc_17}{HTML}{C78472}
\definecolor{minc_18}{HTML}{75D6D0}
\definecolor{minc_19}{HTML}{5B4586}
\definecolor{minc_20}{HTML}{C04393}
\definecolor{minc_21}{HTML}{D69948}
\definecolor{minc_22}{HTML}{7370D8}
\definecolor{minc_23}{HTML}{7A3622}
\definecolor{minc_24}{HTML}{000000}

\begin{figure*}[t]
  \centering
  \small{
  \fcolorbox{white}{minc_1}{\rule{0pt}{6pt}\rule{6pt}{0pt}} Brick~~
  \fcolorbox{white}{minc_2}{\rule{0pt}{6pt}\rule{6pt}{0pt}} Carpet~~
  \fcolorbox{white}{minc_3}{\rule{0pt}{6pt}\rule{6pt}{0pt}} Ceramic~~
  \fcolorbox{white}{minc_4}{\rule{0pt}{6pt}\rule{6pt}{0pt}} Fabric~~
  \fcolorbox{white}{minc_5}{\rule{0pt}{6pt}\rule{6pt}{0pt}} Foliage~~
  \fcolorbox{white}{minc_6}{\rule{0pt}{6pt}\rule{6pt}{0pt}} Food~~
  \fcolorbox{white}{minc_7}{\rule{0pt}{6pt}\rule{6pt}{0pt}} Glass~~
  \fcolorbox{white}{minc_8}{\rule{0pt}{6pt}\rule{6pt}{0pt}} Hair~~ \\
  \fcolorbox{white}{minc_9}{\rule{0pt}{6pt}\rule{6pt}{0pt}} Leather~~
  \fcolorbox{white}{minc_10}{\rule{0pt}{6pt}\rule{6pt}{0pt}} Metal~~
  \fcolorbox{white}{minc_11}{\rule{0pt}{6pt}\rule{6pt}{0pt}} Mirror~~
  \fcolorbox{white}{minc_12}{\rule{0pt}{6pt}\rule{6pt}{0pt}} Other~~
  \fcolorbox{white}{minc_13}{\rule{0pt}{6pt}\rule{6pt}{0pt}} Painted~~
  \fcolorbox{white}{minc_14}{\rule{0pt}{6pt}\rule{6pt}{0pt}} Paper~~
  \fcolorbox{white}{minc_15}{\rule{0pt}{6pt}\rule{6pt}{0pt}} Plastic~~\\
  \fcolorbox{white}{minc_16}{\rule{0pt}{6pt}\rule{6pt}{0pt}} Polished Stone~~
  \fcolorbox{white}{minc_17}{\rule{0pt}{6pt}\rule{6pt}{0pt}} Skin~~
  \fcolorbox{white}{minc_18}{\rule{0pt}{6pt}\rule{6pt}{0pt}} Sky~~
  \fcolorbox{white}{minc_19}{\rule{0pt}{6pt}\rule{6pt}{0pt}} Stone~~
  \fcolorbox{white}{minc_20}{\rule{0pt}{6pt}\rule{6pt}{0pt}} Tile~~
  \fcolorbox{white}{minc_21}{\rule{0pt}{6pt}\rule{6pt}{0pt}} Wallpaper~~
  \fcolorbox{white}{minc_22}{\rule{0pt}{6pt}\rule{6pt}{0pt}} Water~~
  \fcolorbox{white}{minc_23}{\rule{0pt}{6pt}\rule{6pt}{0pt}} Wood~~ \\
  }
  \subfigure{%
    \includegraphics[width=.18\columnwidth]{figures/supplementary/000010868_given.jpg}
  }
  \subfigure{%
    \includegraphics[width=.18\columnwidth]{figures/supplementary/000010868_gt.png}
  }
  \subfigure{%
    \includegraphics[width=.18\columnwidth]{figures/supplementary/000010868_cnn.png}
  }
  \subfigure{%
    \includegraphics[width=.18\columnwidth]{figures/supplementary/000010868_gauss.png}
  }
  \subfigure{%
    \includegraphics[width=.18\columnwidth]{figures/supplementary/000010868_learnt.png}
  }\\[-2ex]
  \subfigure{%
    \includegraphics[width=.18\columnwidth]{figures/supplementary/000006011_given.jpg}
  }
  \subfigure{%
    \includegraphics[width=.18\columnwidth]{figures/supplementary/000006011_gt.png}
  }
  \subfigure{%
    \includegraphics[width=.18\columnwidth]{figures/supplementary/000006011_cnn.png}
  }
  \subfigure{%
    \includegraphics[width=.18\columnwidth]{figures/supplementary/000006011_gauss.png}
  }
  \subfigure{%
    \includegraphics[width=.18\columnwidth]{figures/supplementary/000006011_learnt.png}
  }\\[-2ex]
    \subfigure{%
    \includegraphics[width=.18\columnwidth]{figures/supplementary/000008553_given.jpg}
  }
  \subfigure{%
    \includegraphics[width=.18\columnwidth]{figures/supplementary/000008553_gt.png}
  }
  \subfigure{%
    \includegraphics[width=.18\columnwidth]{figures/supplementary/000008553_cnn.png}
  }
  \subfigure{%
    \includegraphics[width=.18\columnwidth]{figures/supplementary/000008553_gauss.png}
  }
  \subfigure{%
    \includegraphics[width=.18\columnwidth]{figures/supplementary/000008553_learnt.png}
  }\\[-2ex]
   \subfigure{%
    \includegraphics[width=.18\columnwidth]{figures/supplementary/000009188_given.jpg}
  }
  \subfigure{%
    \includegraphics[width=.18\columnwidth]{figures/supplementary/000009188_gt.png}
  }
  \subfigure{%
    \includegraphics[width=.18\columnwidth]{figures/supplementary/000009188_cnn.png}
  }
  \subfigure{%
    \includegraphics[width=.18\columnwidth]{figures/supplementary/000009188_gauss.png}
  }
  \subfigure{%
    \includegraphics[width=.18\columnwidth]{figures/supplementary/000009188_learnt.png}
  }\\[-2ex]
  \setcounter{subfigure}{0}
  \subfigure[Input]{%
    \includegraphics[width=.18\columnwidth]{figures/supplementary/000023570_given.jpg}
  }
  \subfigure[Ground Truth]{%
    \includegraphics[width=.18\columnwidth]{figures/supplementary/000023570_gt.png}
  }
  \subfigure[DeepLab]{%
    \includegraphics[width=.18\columnwidth]{figures/supplementary/000023570_cnn.png}
  }
  \subfigure[+GaussCRF]{%
    \includegraphics[width=.18\columnwidth]{figures/supplementary/000023570_gauss.png}
  }
  \subfigure[+LearnedCRF]{%
    \includegraphics[width=.18\columnwidth]{figures/supplementary/000023570_learnt.png}
  }
  \mycaption{Material Segmentation}{Example results of material segmentation.
  (c)~depicts the unary results before application of MF, (d)~after two steps of MF with Gaussian edge CRF potentials, (e)~after two steps of MF with learned edge CRF potentials.}
    \label{fig:material_visuals-app2}
\end{figure*}


\begin{table*}[h]
\tiny
  \centering
    \begin{tabular}{L{2.3cm} L{2.25cm} C{1.5cm} C{0.7cm} C{0.6cm} C{0.7cm} C{0.7cm} C{0.7cm} C{1.6cm} C{0.6cm} C{0.6cm} C{0.6cm}}
      \toprule
& & & & & \multicolumn{3}{c}{\textbf{Data Statistics}} & \multicolumn{4}{c}{\textbf{Training Protocol}} \\

\textbf{Experiment} & \textbf{Feature Types} & \textbf{Feature Scales} & \textbf{Filter Size} & \textbf{Filter Nbr.} & \textbf{Train}  & \textbf{Val.} & \textbf{Test} & \textbf{Loss Type} & \textbf{LR} & \textbf{Batch} & \textbf{Epochs} \\
      \midrule
      \multicolumn{2}{c}{\textbf{Single Bilateral Filter Applications}} & & & & & & & & & \\
      \textbf{2$\times$ Color Upsampling} & Position$_{1}$, Intensity (3D) & 0.13, 0.17 & 65 & 2 & 10581 & 1449 & 1456 & MSE & 1e-06 & 200 & 94.5\\
      \textbf{4$\times$ Color Upsampling} & Position$_{1}$, Intensity (3D) & 0.06, 0.17 & 65 & 2 & 10581 & 1449 & 1456 & MSE & 1e-06 & 200 & 94.5\\
      \textbf{8$\times$ Color Upsampling} & Position$_{1}$, Intensity (3D) & 0.03, 0.17 & 65 & 2 & 10581 & 1449 & 1456 & MSE & 1e-06 & 200 & 94.5\\
      \textbf{16$\times$ Color Upsampling} & Position$_{1}$, Intensity (3D) & 0.02, 0.17 & 65 & 2 & 10581 & 1449 & 1456 & MSE & 1e-06 & 200 & 94.5\\
      \textbf{Depth Upsampling} & Position$_{1}$, Color (5D) & 0.05, 0.02 & 665 & 2 & 795 & 100 & 654 & MSE & 1e-07 & 50 & 251.6\\
      \textbf{Mesh Denoising} & Isomap (4D) & 46.00 & 63 & 2 & 1000 & 200 & 500 & MSE & 100 & 10 & 100.0 \\
      \midrule
      \multicolumn{2}{c}{\textbf{DenseCRF Applications}} & & & & & & & & &\\
      \multicolumn{2}{l}{\textbf{Semantic Segmentation}} & & & & & & & & &\\
      \textbf{- 1step MF} & Position$_{1}$, Color (5D); Position$_{1}$ (2D) & 0.01, 0.34; 0.34  & 665; 19  & 2; 2 & 10581 & 1449 & 1456 & Logistic & 0.1 & 5 & 1.4 \\
      \textbf{- 2step MF} & Position$_{1}$, Color (5D); Position$_{1}$ (2D) & 0.01, 0.34; 0.34 & 665; 19 & 2; 2 & 10581 & 1449 & 1456 & Logistic & 0.1 & 5 & 1.4 \\
      \textbf{- \textit{loose} 2step MF} & Position$_{1}$, Color (5D); Position$_{1}$ (2D) & 0.01, 0.34; 0.34 & 665; 19 & 2; 2 &10581 & 1449 & 1456 & Logistic & 0.1 & 5 & +1.9  \\ \\
      \multicolumn{2}{l}{\textbf{Material Segmentation}} & & & & & & & & &\\
      \textbf{- 1step MF} & Position$_{2}$, Lab-Color (5D) & 5.00, 0.05, 0.30  & 665 & 2 & 928 & 150 & 1798 & Weighted Logistic & 1e-04 & 24 & 2.6 \\
      \textbf{- 2step MF} & Position$_{2}$, Lab-Color (5D) & 5.00, 0.05, 0.30 & 665 & 2 & 928 & 150 & 1798 & Weighted Logistic & 1e-04 & 12 & +0.7 \\
      \textbf{- \textit{loose} 2step MF} & Position$_{2}$, Lab-Color (5D) & 5.00, 0.05, 0.30 & 665 & 2 & 928 & 150 & 1798 & Weighted Logistic & 1e-04 & 12 & +0.2\\
      \midrule
      \multicolumn{2}{c}{\textbf{Neural Network Applications}} & & & & & & & & &\\
      \textbf{Tiles: CNN-9$\times$9} & - & - & 81 & 4 & 10000 & 1000 & 1000 & Logistic & 0.01 & 100 & 500.0 \\
      \textbf{Tiles: CNN-13$\times$13} & - & - & 169 & 6 & 10000 & 1000 & 1000 & Logistic & 0.01 & 100 & 500.0 \\
      \textbf{Tiles: CNN-17$\times$17} & - & - & 289 & 8 & 10000 & 1000 & 1000 & Logistic & 0.01 & 100 & 500.0 \\
      \textbf{Tiles: CNN-21$\times$21} & - & - & 441 & 10 & 10000 & 1000 & 1000 & Logistic & 0.01 & 100 & 500.0 \\
      \textbf{Tiles: BNN} & Position$_{1}$, Color (5D) & 0.05, 0.04 & 63 & 1 & 10000 & 1000 & 1000 & Logistic & 0.01 & 100 & 30.0 \\
      \textbf{LeNet} & - & - & 25 & 2 & 5490 & 1098 & 1647 & Logistic & 0.1 & 100 & 182.2 \\
      \textbf{Crop-LeNet} & - & - & 25 & 2 & 5490 & 1098 & 1647 & Logistic & 0.1 & 100 & 182.2 \\
      \textbf{BNN-LeNet} & Position$_{2}$ (2D) & 20.00 & 7 & 1 & 5490 & 1098 & 1647 & Logistic & 0.1 & 100 & 182.2 \\
      \textbf{DeepCNet} & - & - & 9 & 1 & 5490 & 1098 & 1647 & Logistic & 0.1 & 100 & 182.2 \\
      \textbf{Crop-DeepCNet} & - & - & 9 & 1 & 5490 & 1098 & 1647 & Logistic & 0.1 & 100 & 182.2 \\
      \textbf{BNN-DeepCNet} & Position$_{2}$ (2D) & 40.00  & 7 & 1 & 5490 & 1098 & 1647 & Logistic & 0.1 & 100 & 182.2 \\
      \bottomrule
      \\
    \end{tabular}
    \mycaption{Experiment Protocols} {Experiment protocols for the different experiments presented in this work. \textbf{Feature Types}:
    Feature spaces used for the bilateral convolutions. Position$_1$ corresponds to un-normalized pixel positions whereas Position$_2$ corresponds
    to pixel positions normalized to $[0,1]$ with respect to the given image. \textbf{Feature Scales}: Cross-validated scales for the features used.
     \textbf{Filter Size}: Number of elements in the filter that is being learned. \textbf{Filter Nbr.}: Half-width of the filter. \textbf{Train},
     \textbf{Val.} and \textbf{Test} corresponds to the number of train, validation and test images used in the experiment. \textbf{Loss Type}: Type
     of loss used for back-propagation. ``MSE'' corresponds to Euclidean mean squared error loss and ``Logistic'' corresponds to multinomial logistic
     loss. ``Weighted Logistic'' is the class-weighted multinomial logistic loss. We weighted the loss with inverse class probability for material
     segmentation task due to the small availability of training data with class imbalance. \textbf{LR}: Fixed learning rate used in stochastic gradient
     descent. \textbf{Batch}: Number of images used in one parameter update step. \textbf{Epochs}: Number of training epochs. In all the experiments,
     we used fixed momentum of 0.9 and weight decay of 0.0005 for stochastic gradient descent. ```Color Upsampling'' experiments in this Table corresponds
     to those performed on Pascal VOC12 dataset images. For all experiments using Pascal VOC12 images, we use extended
     training segmentation dataset available from~\cite{hariharan2011moredata}, and used standard validation and test splits
     from the main dataset~\cite{voc2012segmentation}.}
  \label{tbl:parameters}
\end{table*}

\clearpage

\section{Parameters and Additional Results for Video Propagation Networks}

In this Section, we present experiment protocols and additional qualitative results for experiments
on video object segmentation, semantic video segmentation and video color
propagation. Table~\ref{tbl:parameters_supp} shows the feature scales and other parameters used in different experiments.
Figures~\ref{fig:video_seg_pos_supp} show some qualitative results on video object segmentation
with some failure cases in Fig.~\ref{fig:video_seg_neg_supp}.
Figure~\ref{fig:semantic_visuals_supp} shows some qualitative results on semantic video segmentation and
Fig.~\ref{fig:color_visuals_supp} shows results on video color propagation.

\newcolumntype{L}[1]{>{\raggedright\let\newline\\\arraybackslash\hspace{0pt}}b{#1}}
\newcolumntype{C}[1]{>{\centering\let\newline\\\arraybackslash\hspace{0pt}}b{#1}}
\newcolumntype{R}[1]{>{\raggedleft\let\newline\\\arraybackslash\hspace{0pt}}b{#1}}

\begin{table*}[h]
\tiny
  \centering
    \begin{tabular}{L{3.0cm} L{2.4cm} L{2.8cm} L{2.8cm} C{0.5cm} C{1.0cm} L{1.2cm}}
      \toprule
\textbf{Experiment} & \textbf{Feature Type} & \textbf{Feature Scale-1, $\Lambda_a$} & \textbf{Feature Scale-2, $\Lambda_b$} & \textbf{$\alpha$} & \textbf{Input Frames} & \textbf{Loss Type} \\
      \midrule
      \textbf{Video Object Segmentation} & ($x,y,Y,Cb,Cr,t$) & (0.02,0.02,0.07,0.4,0.4,0.01) & (0.03,0.03,0.09,0.5,0.5,0.2) & 0.5 & 9 & Logistic\\
      \midrule
      \textbf{Semantic Video Segmentation} & & & & & \\
      \textbf{with CNN1~\cite{yu2015multi}-NoFlow} & ($x,y,R,G,B,t$) & (0.08,0.08,0.2,0.2,0.2,0.04) & (0.11,0.11,0.2,0.2,0.2,0.04) & 0.5 & 3 & Logistic \\
      \textbf{with CNN1~\cite{yu2015multi}-Flow} & ($x+u_x,y+u_y,R,G,B,t$) & (0.11,0.11,0.14,0.14,0.14,0.03) & (0.08,0.08,0.12,0.12,0.12,0.01) & 0.65 & 3 & Logistic\\
      \textbf{with CNN2~\cite{richter2016playing}-Flow} & ($x+u_x,y+u_y,R,G,B,t$) & (0.08,0.08,0.2,0.2,0.2,0.04) & (0.09,0.09,0.25,0.25,0.25,0.03) & 0.5 & 4 & Logistic\\
      \midrule
      \textbf{Video Color Propagation} & ($x,y,I,t$)  & (0.04,0.04,0.2,0.04) & No second kernel & 1 & 4 & MSE\\
      \bottomrule
      \\
    \end{tabular}
    \mycaption{Experiment Protocols} {Experiment protocols for the different experiments presented in this work. \textbf{Feature Types}:
    Feature spaces used for the bilateral convolutions, with position ($x,y$) and color
    ($R,G,B$ or $Y,Cb,Cr$) features $\in [0,255]$. $u_x$, $u_y$ denotes optical flow with respect
    to the present frame and $I$ denotes grayscale intensity.
    \textbf{Feature Scales ($\Lambda_a, \Lambda_b$)}: Cross-validated scales for the features used.
    \textbf{$\alpha$}: Exponential time decay for the input frames.
    \textbf{Input Frames}: Number of input frames for VPN.
    \textbf{Loss Type}: Type
     of loss used for back-propagation. ``MSE'' corresponds to Euclidean mean squared error loss and ``Logistic'' corresponds to multinomial logistic loss.}
  \label{tbl:parameters_supp}
\end{table*}

% \begin{figure}[th!]
% \begin{center}
%   \centerline{\includegraphics[width=\textwidth]{figures/video_seg_visuals_supp_small.pdf}}
%     \mycaption{Video Object Segmentation}
%     {Shown are the different frames in example videos with the corresponding
%     ground truth (GT) masks, predictions from BVS~\cite{marki2016bilateral},
%     OFL~\cite{tsaivideo}, VPN (VPN-Stage2) and VPN-DLab (VPN-DeepLab) models.}
%     \label{fig:video_seg_small_supp}
% \end{center}
% \vspace{-1.0cm}
% \end{figure}

\begin{figure}[th!]
\begin{center}
  \centerline{\includegraphics[width=0.7\textwidth]{figures/video_seg_visuals_supp_positive.pdf}}
    \mycaption{Video Object Segmentation}
    {Shown are the different frames in example videos with the corresponding
    ground truth (GT) masks, predictions from BVS~\cite{marki2016bilateral},
    OFL~\cite{tsaivideo}, VPN (VPN-Stage2) and VPN-DLab (VPN-DeepLab) models.}
    \label{fig:video_seg_pos_supp}
\end{center}
\vspace{-1.0cm}
\end{figure}

\begin{figure}[th!]
\begin{center}
  \centerline{\includegraphics[width=0.7\textwidth]{figures/video_seg_visuals_supp_negative.pdf}}
    \mycaption{Failure Cases for Video Object Segmentation}
    {Shown are the different frames in example videos with the corresponding
    ground truth (GT) masks, predictions from BVS~\cite{marki2016bilateral},
    OFL~\cite{tsaivideo}, VPN (VPN-Stage2) and VPN-DLab (VPN-DeepLab) models.}
    \label{fig:video_seg_neg_supp}
\end{center}
\vspace{-1.0cm}
\end{figure}

\begin{figure}[th!]
\begin{center}
  \centerline{\includegraphics[width=0.9\textwidth]{figures/supp_semantic_visual.pdf}}
    \mycaption{Semantic Video Segmentation}
    {Input video frames and the corresponding ground truth (GT)
    segmentation together with the predictions of CNN~\cite{yu2015multi} and with
    VPN-Flow.}
    \label{fig:semantic_visuals_supp}
\end{center}
\vspace{-0.7cm}
\end{figure}

\begin{figure}[th!]
\begin{center}
  \centerline{\includegraphics[width=\textwidth]{figures/colorization_visuals_supp.pdf}}
  \mycaption{Video Color Propagation}
  {Input grayscale video frames and corresponding ground-truth (GT) color images
  together with color predictions of Levin et al.~\cite{levin2004colorization} and VPN-Stage1 models.}
  \label{fig:color_visuals_supp}
\end{center}
\vspace{-0.7cm}
\end{figure}

\clearpage

\section{Additional Material for Bilateral Inception Networks}
\label{sec:binception-app}

In this section of the Appendix, we first discuss the use of approximate bilateral
filtering in BI modules (Sec.~\ref{sec:lattice}).
Later, we present some qualitative results using different models for the approach presented in
Chapter~\ref{chap:binception} (Sec.~\ref{sec:qualitative-app}).

\subsection{Approximate Bilateral Filtering}
\label{sec:lattice}

The bilateral inception module presented in Chapter~\ref{chap:binception} computes a matrix-vector
product between a Gaussian filter $K$ and a vector of activations $\bz_c$.
Bilateral filtering is an important operation and many algorithmic techniques have been
proposed to speed-up this operation~\cite{paris2006fast,adams2010fast,gastal2011domain}.
In the main paper we opted to implement what can be considered the
brute-force variant of explicitly constructing $K$ and then using BLAS to compute the
matrix-vector product. This resulted in a few millisecond operation.
The explicit way to compute is possible due to the
reduction to super-pixels, e.g., it would not work for DenseCRF variants
that operate on the full image resolution.

Here, we present experiments where we use the fast approximate bilateral filtering
algorithm of~\cite{adams2010fast}, which is also used in Chapter~\ref{chap:bnn}
for learning sparse high dimensional filters. This
choice allows for larger dimensions of matrix-vector multiplication. The reason for choosing
the explicit multiplication in Chapter~\ref{chap:binception} was that it was computationally faster.
For the small sizes of the involved matrices and vectors, the explicit computation is sufficient and we had no
GPU implementation of an approximate technique that matched this runtime. Also it
is conceptually easier and the gradient to the feature transformations ($\Lambda \mathbf{f}$) is
obtained using standard matrix calculus.

\subsubsection{Experiments}

We modified the existing segmentation architectures analogous to those in Chapter~\ref{chap:binception}.
The main difference is that, here, the inception modules use the lattice
approximation~\cite{adams2010fast} to compute the bilateral filtering.
Using the lattice approximation did not allow us to back-propagate through feature transformations ($\Lambda$)
and thus we used hand-specified feature scales as will be explained later.
Specifically, we take CNN architectures from the works
of~\cite{chen2014semantic,zheng2015conditional,bell2015minc} and insert the BI modules between
the spatial FC layers.
We use superpixels from~\cite{DollarICCV13edges}
for all the experiments with the lattice approximation. Experiments are
performed using Caffe neural network framework~\cite{jia2014caffe}.

\begin{table}
  \small
  \centering
  \begin{tabular}{p{5.5cm}>{\raggedright\arraybackslash}p{1.4cm}>{\centering\arraybackslash}p{2.2cm}}
    \toprule
		\textbf{Model} & \emph{IoU} & \emph{Runtime}(ms) \\
    \midrule

    %%%%%%%%%%%% Scores computed by us)%%%%%%%%%%%%
		\deeplablargefov & 68.9 & 145ms\\
    \midrule
    \bi{7}{2}-\bi{8}{10}& \textbf{73.8} & +600 \\
    \midrule
    \deeplablargefovcrf~\cite{chen2014semantic} & 72.7 & +830\\
    \deeplabmsclargefovcrf~\cite{chen2014semantic} & \textbf{73.6} & +880\\
    DeepLab-EdgeNet~\cite{chen2015semantic} & 71.7 & +30\\
    DeepLab-EdgeNet-CRF~\cite{chen2015semantic} & \textbf{73.6} & +860\\
  \bottomrule \\
  \end{tabular}
  \mycaption{Semantic Segmentation using the DeepLab model}
  {IoU scores on the Pascal VOC12 segmentation test dataset
  with different models and our modified inception model.
  Also shown are the corresponding runtimes in milliseconds. Runtimes
  also include superpixel computations (300 ms with Dollar superpixels~\cite{DollarICCV13edges})}
  \label{tab:largefovresults}
\end{table}

\paragraph{Semantic Segmentation}
The experiments in this section use the Pascal VOC12 segmentation dataset~\cite{voc2012segmentation} with 21 object classes and the images have a maximum resolution of 0.25 megapixels.
For all experiments on VOC12, we train using the extended training set of
10581 images collected by~\cite{hariharan2011moredata}.
We modified the \deeplab~network architecture of~\cite{chen2014semantic} and
the CRFasRNN architecture from~\cite{zheng2015conditional} which uses a CNN with
deconvolution layers followed by DenseCRF trained end-to-end.

\paragraph{DeepLab Model}\label{sec:deeplabmodel}
We experimented with the \bi{7}{2}-\bi{8}{10} inception model.
Results using the~\deeplab~model are summarized in Tab.~\ref{tab:largefovresults}.
Although we get similar improvements with inception modules as with the
explicit kernel computation, using lattice approximation is slower.

\begin{table}
  \small
  \centering
  \begin{tabular}{p{6.4cm}>{\raggedright\arraybackslash}p{1.8cm}>{\raggedright\arraybackslash}p{1.8cm}}
    \toprule
    \textbf{Model} & \emph{IoU (Val)} & \emph{IoU (Test)}\\
    \midrule
    %%%%%%%%%%%% Scores computed by us)%%%%%%%%%%%%
    CNN &  67.5 & - \\
    \deconv (CNN+Deconvolutions) & 69.8 & 72.0 \\
    \midrule
    \bi{3}{6}-\bi{4}{6}-\bi{7}{2}-\bi{8}{6}& 71.9 & - \\
    \bi{3}{6}-\bi{4}{6}-\bi{7}{2}-\bi{8}{6}-\gi{6}& 73.6 &  \href{http://host.robots.ox.ac.uk:8080/anonymous/VOTV5E.html}{\textbf{75.2}}\\
    \midrule
    \deconvcrf (CRF-RNN)~\cite{zheng2015conditional} & 73.0 & 74.7\\
    Context-CRF-RNN~\cite{yu2015multi} & ~~ - ~ & \textbf{75.3} \\
    \bottomrule \\
  \end{tabular}
  \mycaption{Semantic Segmentation using the CRFasRNN model}{IoU score corresponding to different models
  on Pascal VOC12 reduced validation / test segmentation dataset. The reduced validation set consists of 346 images
  as used in~\cite{zheng2015conditional} where we adapted the model from.}
  \label{tab:deconvresults-app}
\end{table}

\paragraph{CRFasRNN Model}\label{sec:deepinception}
We add BI modules after score-pool3, score-pool4, \fc{7} and \fc{8} $1\times1$ convolution layers
resulting in the \bi{3}{6}-\bi{4}{6}-\bi{7}{2}-\bi{8}{6}
model and also experimented with another variant where $BI_8$ is followed by another inception
module, G$(6)$, with 6 Gaussian kernels.
Note that here also we discarded both deconvolution and DenseCRF parts of the original model~\cite{zheng2015conditional}
and inserted the BI modules in the base CNN and found similar improvements compared to the inception modules with explicit
kernel computaion. See Tab.~\ref{tab:deconvresults-app} for results on the CRFasRNN model.

\paragraph{Material Segmentation}
Table~\ref{tab:mincresults-app} shows the results on the MINC dataset~\cite{bell2015minc}
obtained by modifying the AlexNet architecture with our inception modules. We observe
similar improvements as with explicit kernel construction.
For this model, we do not provide any learned setup due to very limited segment training
data. The weights to combine outputs in the bilateral inception layer are
found by validation on the validation set.

\begin{table}[t]
  \small
  \centering
  \begin{tabular}{p{3.5cm}>{\centering\arraybackslash}p{4.0cm}}
    \toprule
    \textbf{Model} & Class / Total accuracy\\
    \midrule

    %%%%%%%%%%%% Scores computed by us)%%%%%%%%%%%%
    AlexNet CNN & 55.3 / 58.9 \\
    \midrule
    \bi{7}{2}-\bi{8}{6}& 68.5 / 71.8 \\
    \bi{7}{2}-\bi{8}{6}-G$(6)$& 67.6 / 73.1 \\
    \midrule
    AlexNet-CRF & 65.5 / 71.0 \\
    \bottomrule \\
  \end{tabular}
  \mycaption{Material Segmentation using AlexNet}{Pixel accuracy of different models on
  the MINC material segmentation test dataset~\cite{bell2015minc}.}
  \label{tab:mincresults-app}
\end{table}

\paragraph{Scales of Bilateral Inception Modules}
\label{sec:scales}

Unlike the explicit kernel technique presented in the main text (Chapter~\ref{chap:binception}),
we didn't back-propagate through feature transformation ($\Lambda$)
using the approximate bilateral filter technique.
So, the feature scales are hand-specified and validated, which are as follows.
The optimal scale values for the \bi{7}{2}-\bi{8}{2} model are found by validation for the best performance which are
$\sigma_{xy}$ = (0.1, 0.1) for the spatial (XY) kernel and $\sigma_{rgbxy}$ = (0.1, 0.1, 0.1, 0.01, 0.01) for color and position (RGBXY)  kernel.
Next, as more kernels are added to \bi{8}{2}, we set scales to be $\alpha$*($\sigma_{xy}$, $\sigma_{rgbxy}$).
The value of $\alpha$ is chosen as  1, 0.5, 0.1, 0.05, 0.1, at uniform interval, for the \bi{8}{10} bilateral inception module.


\subsection{Qualitative Results}
\label{sec:qualitative-app}

In this section, we present more qualitative results obtained using the BI module with explicit
kernel computation technique presented in Chapter~\ref{chap:binception}. Results on the Pascal VOC12
dataset~\cite{voc2012segmentation} using the DeepLab-LargeFOV model are shown in Fig.~\ref{fig:semantic_visuals-app},
followed by the results on MINC dataset~\cite{bell2015minc}
in Fig.~\ref{fig:material_visuals-app} and on
Cityscapes dataset~\cite{Cordts2015Cvprw} in Fig.~\ref{fig:street_visuals-app}.


\definecolor{voc_1}{RGB}{0, 0, 0}
\definecolor{voc_2}{RGB}{128, 0, 0}
\definecolor{voc_3}{RGB}{0, 128, 0}
\definecolor{voc_4}{RGB}{128, 128, 0}
\definecolor{voc_5}{RGB}{0, 0, 128}
\definecolor{voc_6}{RGB}{128, 0, 128}
\definecolor{voc_7}{RGB}{0, 128, 128}
\definecolor{voc_8}{RGB}{128, 128, 128}
\definecolor{voc_9}{RGB}{64, 0, 0}
\definecolor{voc_10}{RGB}{192, 0, 0}
\definecolor{voc_11}{RGB}{64, 128, 0}
\definecolor{voc_12}{RGB}{192, 128, 0}
\definecolor{voc_13}{RGB}{64, 0, 128}
\definecolor{voc_14}{RGB}{192, 0, 128}
\definecolor{voc_15}{RGB}{64, 128, 128}
\definecolor{voc_16}{RGB}{192, 128, 128}
\definecolor{voc_17}{RGB}{0, 64, 0}
\definecolor{voc_18}{RGB}{128, 64, 0}
\definecolor{voc_19}{RGB}{0, 192, 0}
\definecolor{voc_20}{RGB}{128, 192, 0}
\definecolor{voc_21}{RGB}{0, 64, 128}
\definecolor{voc_22}{RGB}{128, 64, 128}

\begin{figure*}[!ht]
  \small
  \centering
  \fcolorbox{white}{voc_1}{\rule{0pt}{4pt}\rule{4pt}{0pt}} Background~~
  \fcolorbox{white}{voc_2}{\rule{0pt}{4pt}\rule{4pt}{0pt}} Aeroplane~~
  \fcolorbox{white}{voc_3}{\rule{0pt}{4pt}\rule{4pt}{0pt}} Bicycle~~
  \fcolorbox{white}{voc_4}{\rule{0pt}{4pt}\rule{4pt}{0pt}} Bird~~
  \fcolorbox{white}{voc_5}{\rule{0pt}{4pt}\rule{4pt}{0pt}} Boat~~
  \fcolorbox{white}{voc_6}{\rule{0pt}{4pt}\rule{4pt}{0pt}} Bottle~~
  \fcolorbox{white}{voc_7}{\rule{0pt}{4pt}\rule{4pt}{0pt}} Bus~~
  \fcolorbox{white}{voc_8}{\rule{0pt}{4pt}\rule{4pt}{0pt}} Car~~\\
  \fcolorbox{white}{voc_9}{\rule{0pt}{4pt}\rule{4pt}{0pt}} Cat~~
  \fcolorbox{white}{voc_10}{\rule{0pt}{4pt}\rule{4pt}{0pt}} Chair~~
  \fcolorbox{white}{voc_11}{\rule{0pt}{4pt}\rule{4pt}{0pt}} Cow~~
  \fcolorbox{white}{voc_12}{\rule{0pt}{4pt}\rule{4pt}{0pt}} Dining Table~~
  \fcolorbox{white}{voc_13}{\rule{0pt}{4pt}\rule{4pt}{0pt}} Dog~~
  \fcolorbox{white}{voc_14}{\rule{0pt}{4pt}\rule{4pt}{0pt}} Horse~~
  \fcolorbox{white}{voc_15}{\rule{0pt}{4pt}\rule{4pt}{0pt}} Motorbike~~
  \fcolorbox{white}{voc_16}{\rule{0pt}{4pt}\rule{4pt}{0pt}} Person~~\\
  \fcolorbox{white}{voc_17}{\rule{0pt}{4pt}\rule{4pt}{0pt}} Potted Plant~~
  \fcolorbox{white}{voc_18}{\rule{0pt}{4pt}\rule{4pt}{0pt}} Sheep~~
  \fcolorbox{white}{voc_19}{\rule{0pt}{4pt}\rule{4pt}{0pt}} Sofa~~
  \fcolorbox{white}{voc_20}{\rule{0pt}{4pt}\rule{4pt}{0pt}} Train~~
  \fcolorbox{white}{voc_21}{\rule{0pt}{4pt}\rule{4pt}{0pt}} TV monitor~~\\


  \subfigure{%
    \includegraphics[width=.15\columnwidth]{figures/supplementary/2008_001308_given.png}
  }
  \subfigure{%
    \includegraphics[width=.15\columnwidth]{figures/supplementary/2008_001308_sp.png}
  }
  \subfigure{%
    \includegraphics[width=.15\columnwidth]{figures/supplementary/2008_001308_gt.png}
  }
  \subfigure{%
    \includegraphics[width=.15\columnwidth]{figures/supplementary/2008_001308_cnn.png}
  }
  \subfigure{%
    \includegraphics[width=.15\columnwidth]{figures/supplementary/2008_001308_crf.png}
  }
  \subfigure{%
    \includegraphics[width=.15\columnwidth]{figures/supplementary/2008_001308_ours.png}
  }\\[-2ex]


  \subfigure{%
    \includegraphics[width=.15\columnwidth]{figures/supplementary/2008_001821_given.png}
  }
  \subfigure{%
    \includegraphics[width=.15\columnwidth]{figures/supplementary/2008_001821_sp.png}
  }
  \subfigure{%
    \includegraphics[width=.15\columnwidth]{figures/supplementary/2008_001821_gt.png}
  }
  \subfigure{%
    \includegraphics[width=.15\columnwidth]{figures/supplementary/2008_001821_cnn.png}
  }
  \subfigure{%
    \includegraphics[width=.15\columnwidth]{figures/supplementary/2008_001821_crf.png}
  }
  \subfigure{%
    \includegraphics[width=.15\columnwidth]{figures/supplementary/2008_001821_ours.png}
  }\\[-2ex]



  \subfigure{%
    \includegraphics[width=.15\columnwidth]{figures/supplementary/2008_004612_given.png}
  }
  \subfigure{%
    \includegraphics[width=.15\columnwidth]{figures/supplementary/2008_004612_sp.png}
  }
  \subfigure{%
    \includegraphics[width=.15\columnwidth]{figures/supplementary/2008_004612_gt.png}
  }
  \subfigure{%
    \includegraphics[width=.15\columnwidth]{figures/supplementary/2008_004612_cnn.png}
  }
  \subfigure{%
    \includegraphics[width=.15\columnwidth]{figures/supplementary/2008_004612_crf.png}
  }
  \subfigure{%
    \includegraphics[width=.15\columnwidth]{figures/supplementary/2008_004612_ours.png}
  }\\[-2ex]


  \subfigure{%
    \includegraphics[width=.15\columnwidth]{figures/supplementary/2009_001008_given.png}
  }
  \subfigure{%
    \includegraphics[width=.15\columnwidth]{figures/supplementary/2009_001008_sp.png}
  }
  \subfigure{%
    \includegraphics[width=.15\columnwidth]{figures/supplementary/2009_001008_gt.png}
  }
  \subfigure{%
    \includegraphics[width=.15\columnwidth]{figures/supplementary/2009_001008_cnn.png}
  }
  \subfigure{%
    \includegraphics[width=.15\columnwidth]{figures/supplementary/2009_001008_crf.png}
  }
  \subfigure{%
    \includegraphics[width=.15\columnwidth]{figures/supplementary/2009_001008_ours.png}
  }\\[-2ex]




  \subfigure{%
    \includegraphics[width=.15\columnwidth]{figures/supplementary/2009_004497_given.png}
  }
  \subfigure{%
    \includegraphics[width=.15\columnwidth]{figures/supplementary/2009_004497_sp.png}
  }
  \subfigure{%
    \includegraphics[width=.15\columnwidth]{figures/supplementary/2009_004497_gt.png}
  }
  \subfigure{%
    \includegraphics[width=.15\columnwidth]{figures/supplementary/2009_004497_cnn.png}
  }
  \subfigure{%
    \includegraphics[width=.15\columnwidth]{figures/supplementary/2009_004497_crf.png}
  }
  \subfigure{%
    \includegraphics[width=.15\columnwidth]{figures/supplementary/2009_004497_ours.png}
  }\\[-2ex]



  \setcounter{subfigure}{0}
  \subfigure[\scriptsize Input]{%
    \includegraphics[width=.15\columnwidth]{figures/supplementary/2010_001327_given.png}
  }
  \subfigure[\scriptsize Superpixels]{%
    \includegraphics[width=.15\columnwidth]{figures/supplementary/2010_001327_sp.png}
  }
  \subfigure[\scriptsize GT]{%
    \includegraphics[width=.15\columnwidth]{figures/supplementary/2010_001327_gt.png}
  }
  \subfigure[\scriptsize Deeplab]{%
    \includegraphics[width=.15\columnwidth]{figures/supplementary/2010_001327_cnn.png}
  }
  \subfigure[\scriptsize +DenseCRF]{%
    \includegraphics[width=.15\columnwidth]{figures/supplementary/2010_001327_crf.png}
  }
  \subfigure[\scriptsize Using BI]{%
    \includegraphics[width=.15\columnwidth]{figures/supplementary/2010_001327_ours.png}
  }
  \mycaption{Semantic Segmentation}{Example results of semantic segmentation
  on the Pascal VOC12 dataset.
  (d)~depicts the DeepLab CNN result, (e)~CNN + 10 steps of mean-field inference,
  (f~result obtained with bilateral inception (BI) modules (\bi{6}{2}+\bi{7}{6}) between \fc~layers.}
  \label{fig:semantic_visuals-app}
\end{figure*}


\definecolor{minc_1}{HTML}{771111}
\definecolor{minc_2}{HTML}{CAC690}
\definecolor{minc_3}{HTML}{EEEEEE}
\definecolor{minc_4}{HTML}{7C8FA6}
\definecolor{minc_5}{HTML}{597D31}
\definecolor{minc_6}{HTML}{104410}
\definecolor{minc_7}{HTML}{BB819C}
\definecolor{minc_8}{HTML}{D0CE48}
\definecolor{minc_9}{HTML}{622745}
\definecolor{minc_10}{HTML}{666666}
\definecolor{minc_11}{HTML}{D54A31}
\definecolor{minc_12}{HTML}{101044}
\definecolor{minc_13}{HTML}{444126}
\definecolor{minc_14}{HTML}{75D646}
\definecolor{minc_15}{HTML}{DD4348}
\definecolor{minc_16}{HTML}{5C8577}
\definecolor{minc_17}{HTML}{C78472}
\definecolor{minc_18}{HTML}{75D6D0}
\definecolor{minc_19}{HTML}{5B4586}
\definecolor{minc_20}{HTML}{C04393}
\definecolor{minc_21}{HTML}{D69948}
\definecolor{minc_22}{HTML}{7370D8}
\definecolor{minc_23}{HTML}{7A3622}
\definecolor{minc_24}{HTML}{000000}

\begin{figure*}[!ht]
  \small % scriptsize
  \centering
  \fcolorbox{white}{minc_1}{\rule{0pt}{4pt}\rule{4pt}{0pt}} Brick~~
  \fcolorbox{white}{minc_2}{\rule{0pt}{4pt}\rule{4pt}{0pt}} Carpet~~
  \fcolorbox{white}{minc_3}{\rule{0pt}{4pt}\rule{4pt}{0pt}} Ceramic~~
  \fcolorbox{white}{minc_4}{\rule{0pt}{4pt}\rule{4pt}{0pt}} Fabric~~
  \fcolorbox{white}{minc_5}{\rule{0pt}{4pt}\rule{4pt}{0pt}} Foliage~~
  \fcolorbox{white}{minc_6}{\rule{0pt}{4pt}\rule{4pt}{0pt}} Food~~
  \fcolorbox{white}{minc_7}{\rule{0pt}{4pt}\rule{4pt}{0pt}} Glass~~
  \fcolorbox{white}{minc_8}{\rule{0pt}{4pt}\rule{4pt}{0pt}} Hair~~\\
  \fcolorbox{white}{minc_9}{\rule{0pt}{4pt}\rule{4pt}{0pt}} Leather~~
  \fcolorbox{white}{minc_10}{\rule{0pt}{4pt}\rule{4pt}{0pt}} Metal~~
  \fcolorbox{white}{minc_11}{\rule{0pt}{4pt}\rule{4pt}{0pt}} Mirror~~
  \fcolorbox{white}{minc_12}{\rule{0pt}{4pt}\rule{4pt}{0pt}} Other~~
  \fcolorbox{white}{minc_13}{\rule{0pt}{4pt}\rule{4pt}{0pt}} Painted~~
  \fcolorbox{white}{minc_14}{\rule{0pt}{4pt}\rule{4pt}{0pt}} Paper~~
  \fcolorbox{white}{minc_15}{\rule{0pt}{4pt}\rule{4pt}{0pt}} Plastic~~\\
  \fcolorbox{white}{minc_16}{\rule{0pt}{4pt}\rule{4pt}{0pt}} Polished Stone~~
  \fcolorbox{white}{minc_17}{\rule{0pt}{4pt}\rule{4pt}{0pt}} Skin~~
  \fcolorbox{white}{minc_18}{\rule{0pt}{4pt}\rule{4pt}{0pt}} Sky~~
  \fcolorbox{white}{minc_19}{\rule{0pt}{4pt}\rule{4pt}{0pt}} Stone~~
  \fcolorbox{white}{minc_20}{\rule{0pt}{4pt}\rule{4pt}{0pt}} Tile~~
  \fcolorbox{white}{minc_21}{\rule{0pt}{4pt}\rule{4pt}{0pt}} Wallpaper~~
  \fcolorbox{white}{minc_22}{\rule{0pt}{4pt}\rule{4pt}{0pt}} Water~~
  \fcolorbox{white}{minc_23}{\rule{0pt}{4pt}\rule{4pt}{0pt}} Wood~~\\
  \subfigure{%
    \includegraphics[width=.15\columnwidth]{figures/supplementary/000008468_given.png}
  }
  \subfigure{%
    \includegraphics[width=.15\columnwidth]{figures/supplementary/000008468_sp.png}
  }
  \subfigure{%
    \includegraphics[width=.15\columnwidth]{figures/supplementary/000008468_gt.png}
  }
  \subfigure{%
    \includegraphics[width=.15\columnwidth]{figures/supplementary/000008468_cnn.png}
  }
  \subfigure{%
    \includegraphics[width=.15\columnwidth]{figures/supplementary/000008468_crf.png}
  }
  \subfigure{%
    \includegraphics[width=.15\columnwidth]{figures/supplementary/000008468_ours.png}
  }\\[-2ex]

  \subfigure{%
    \includegraphics[width=.15\columnwidth]{figures/supplementary/000009053_given.png}
  }
  \subfigure{%
    \includegraphics[width=.15\columnwidth]{figures/supplementary/000009053_sp.png}
  }
  \subfigure{%
    \includegraphics[width=.15\columnwidth]{figures/supplementary/000009053_gt.png}
  }
  \subfigure{%
    \includegraphics[width=.15\columnwidth]{figures/supplementary/000009053_cnn.png}
  }
  \subfigure{%
    \includegraphics[width=.15\columnwidth]{figures/supplementary/000009053_crf.png}
  }
  \subfigure{%
    \includegraphics[width=.15\columnwidth]{figures/supplementary/000009053_ours.png}
  }\\[-2ex]




  \subfigure{%
    \includegraphics[width=.15\columnwidth]{figures/supplementary/000014977_given.png}
  }
  \subfigure{%
    \includegraphics[width=.15\columnwidth]{figures/supplementary/000014977_sp.png}
  }
  \subfigure{%
    \includegraphics[width=.15\columnwidth]{figures/supplementary/000014977_gt.png}
  }
  \subfigure{%
    \includegraphics[width=.15\columnwidth]{figures/supplementary/000014977_cnn.png}
  }
  \subfigure{%
    \includegraphics[width=.15\columnwidth]{figures/supplementary/000014977_crf.png}
  }
  \subfigure{%
    \includegraphics[width=.15\columnwidth]{figures/supplementary/000014977_ours.png}
  }\\[-2ex]


  \subfigure{%
    \includegraphics[width=.15\columnwidth]{figures/supplementary/000022922_given.png}
  }
  \subfigure{%
    \includegraphics[width=.15\columnwidth]{figures/supplementary/000022922_sp.png}
  }
  \subfigure{%
    \includegraphics[width=.15\columnwidth]{figures/supplementary/000022922_gt.png}
  }
  \subfigure{%
    \includegraphics[width=.15\columnwidth]{figures/supplementary/000022922_cnn.png}
  }
  \subfigure{%
    \includegraphics[width=.15\columnwidth]{figures/supplementary/000022922_crf.png}
  }
  \subfigure{%
    \includegraphics[width=.15\columnwidth]{figures/supplementary/000022922_ours.png}
  }\\[-2ex]


  \subfigure{%
    \includegraphics[width=.15\columnwidth]{figures/supplementary/000025711_given.png}
  }
  \subfigure{%
    \includegraphics[width=.15\columnwidth]{figures/supplementary/000025711_sp.png}
  }
  \subfigure{%
    \includegraphics[width=.15\columnwidth]{figures/supplementary/000025711_gt.png}
  }
  \subfigure{%
    \includegraphics[width=.15\columnwidth]{figures/supplementary/000025711_cnn.png}
  }
  \subfigure{%
    \includegraphics[width=.15\columnwidth]{figures/supplementary/000025711_crf.png}
  }
  \subfigure{%
    \includegraphics[width=.15\columnwidth]{figures/supplementary/000025711_ours.png}
  }\\[-2ex]


  \subfigure{%
    \includegraphics[width=.15\columnwidth]{figures/supplementary/000034473_given.png}
  }
  \subfigure{%
    \includegraphics[width=.15\columnwidth]{figures/supplementary/000034473_sp.png}
  }
  \subfigure{%
    \includegraphics[width=.15\columnwidth]{figures/supplementary/000034473_gt.png}
  }
  \subfigure{%
    \includegraphics[width=.15\columnwidth]{figures/supplementary/000034473_cnn.png}
  }
  \subfigure{%
    \includegraphics[width=.15\columnwidth]{figures/supplementary/000034473_crf.png}
  }
  \subfigure{%
    \includegraphics[width=.15\columnwidth]{figures/supplementary/000034473_ours.png}
  }\\[-2ex]


  \subfigure{%
    \includegraphics[width=.15\columnwidth]{figures/supplementary/000035463_given.png}
  }
  \subfigure{%
    \includegraphics[width=.15\columnwidth]{figures/supplementary/000035463_sp.png}
  }
  \subfigure{%
    \includegraphics[width=.15\columnwidth]{figures/supplementary/000035463_gt.png}
  }
  \subfigure{%
    \includegraphics[width=.15\columnwidth]{figures/supplementary/000035463_cnn.png}
  }
  \subfigure{%
    \includegraphics[width=.15\columnwidth]{figures/supplementary/000035463_crf.png}
  }
  \subfigure{%
    \includegraphics[width=.15\columnwidth]{figures/supplementary/000035463_ours.png}
  }\\[-2ex]


  \setcounter{subfigure}{0}
  \subfigure[\scriptsize Input]{%
    \includegraphics[width=.15\columnwidth]{figures/supplementary/000035993_given.png}
  }
  \subfigure[\scriptsize Superpixels]{%
    \includegraphics[width=.15\columnwidth]{figures/supplementary/000035993_sp.png}
  }
  \subfigure[\scriptsize GT]{%
    \includegraphics[width=.15\columnwidth]{figures/supplementary/000035993_gt.png}
  }
  \subfigure[\scriptsize AlexNet]{%
    \includegraphics[width=.15\columnwidth]{figures/supplementary/000035993_cnn.png}
  }
  \subfigure[\scriptsize +DenseCRF]{%
    \includegraphics[width=.15\columnwidth]{figures/supplementary/000035993_crf.png}
  }
  \subfigure[\scriptsize Using BI]{%
    \includegraphics[width=.15\columnwidth]{figures/supplementary/000035993_ours.png}
  }
  \mycaption{Material Segmentation}{Example results of material segmentation.
  (d)~depicts the AlexNet CNN result, (e)~CNN + 10 steps of mean-field inference,
  (f)~result obtained with bilateral inception (BI) modules (\bi{7}{2}+\bi{8}{6}) between
  \fc~layers.}
\label{fig:material_visuals-app}
\end{figure*}


\definecolor{city_1}{RGB}{128, 64, 128}
\definecolor{city_2}{RGB}{244, 35, 232}
\definecolor{city_3}{RGB}{70, 70, 70}
\definecolor{city_4}{RGB}{102, 102, 156}
\definecolor{city_5}{RGB}{190, 153, 153}
\definecolor{city_6}{RGB}{153, 153, 153}
\definecolor{city_7}{RGB}{250, 170, 30}
\definecolor{city_8}{RGB}{220, 220, 0}
\definecolor{city_9}{RGB}{107, 142, 35}
\definecolor{city_10}{RGB}{152, 251, 152}
\definecolor{city_11}{RGB}{70, 130, 180}
\definecolor{city_12}{RGB}{220, 20, 60}
\definecolor{city_13}{RGB}{255, 0, 0}
\definecolor{city_14}{RGB}{0, 0, 142}
\definecolor{city_15}{RGB}{0, 0, 70}
\definecolor{city_16}{RGB}{0, 60, 100}
\definecolor{city_17}{RGB}{0, 80, 100}
\definecolor{city_18}{RGB}{0, 0, 230}
\definecolor{city_19}{RGB}{119, 11, 32}
\begin{figure*}[!ht]
  \small % scriptsize
  \centering


  \subfigure{%
    \includegraphics[width=.18\columnwidth]{figures/supplementary/frankfurt00000_016005_given.png}
  }
  \subfigure{%
    \includegraphics[width=.18\columnwidth]{figures/supplementary/frankfurt00000_016005_sp.png}
  }
  \subfigure{%
    \includegraphics[width=.18\columnwidth]{figures/supplementary/frankfurt00000_016005_gt.png}
  }
  \subfigure{%
    \includegraphics[width=.18\columnwidth]{figures/supplementary/frankfurt00000_016005_cnn.png}
  }
  \subfigure{%
    \includegraphics[width=.18\columnwidth]{figures/supplementary/frankfurt00000_016005_ours.png}
  }\\[-2ex]

  \subfigure{%
    \includegraphics[width=.18\columnwidth]{figures/supplementary/frankfurt00000_004617_given.png}
  }
  \subfigure{%
    \includegraphics[width=.18\columnwidth]{figures/supplementary/frankfurt00000_004617_sp.png}
  }
  \subfigure{%
    \includegraphics[width=.18\columnwidth]{figures/supplementary/frankfurt00000_004617_gt.png}
  }
  \subfigure{%
    \includegraphics[width=.18\columnwidth]{figures/supplementary/frankfurt00000_004617_cnn.png}
  }
  \subfigure{%
    \includegraphics[width=.18\columnwidth]{figures/supplementary/frankfurt00000_004617_ours.png}
  }\\[-2ex]

  \subfigure{%
    \includegraphics[width=.18\columnwidth]{figures/supplementary/frankfurt00000_020880_given.png}
  }
  \subfigure{%
    \includegraphics[width=.18\columnwidth]{figures/supplementary/frankfurt00000_020880_sp.png}
  }
  \subfigure{%
    \includegraphics[width=.18\columnwidth]{figures/supplementary/frankfurt00000_020880_gt.png}
  }
  \subfigure{%
    \includegraphics[width=.18\columnwidth]{figures/supplementary/frankfurt00000_020880_cnn.png}
  }
  \subfigure{%
    \includegraphics[width=.18\columnwidth]{figures/supplementary/frankfurt00000_020880_ours.png}
  }\\[-2ex]



  \subfigure{%
    \includegraphics[width=.18\columnwidth]{figures/supplementary/frankfurt00001_007285_given.png}
  }
  \subfigure{%
    \includegraphics[width=.18\columnwidth]{figures/supplementary/frankfurt00001_007285_sp.png}
  }
  \subfigure{%
    \includegraphics[width=.18\columnwidth]{figures/supplementary/frankfurt00001_007285_gt.png}
  }
  \subfigure{%
    \includegraphics[width=.18\columnwidth]{figures/supplementary/frankfurt00001_007285_cnn.png}
  }
  \subfigure{%
    \includegraphics[width=.18\columnwidth]{figures/supplementary/frankfurt00001_007285_ours.png}
  }\\[-2ex]


  \subfigure{%
    \includegraphics[width=.18\columnwidth]{figures/supplementary/frankfurt00001_059789_given.png}
  }
  \subfigure{%
    \includegraphics[width=.18\columnwidth]{figures/supplementary/frankfurt00001_059789_sp.png}
  }
  \subfigure{%
    \includegraphics[width=.18\columnwidth]{figures/supplementary/frankfurt00001_059789_gt.png}
  }
  \subfigure{%
    \includegraphics[width=.18\columnwidth]{figures/supplementary/frankfurt00001_059789_cnn.png}
  }
  \subfigure{%
    \includegraphics[width=.18\columnwidth]{figures/supplementary/frankfurt00001_059789_ours.png}
  }\\[-2ex]


  \subfigure{%
    \includegraphics[width=.18\columnwidth]{figures/supplementary/frankfurt00001_068208_given.png}
  }
  \subfigure{%
    \includegraphics[width=.18\columnwidth]{figures/supplementary/frankfurt00001_068208_sp.png}
  }
  \subfigure{%
    \includegraphics[width=.18\columnwidth]{figures/supplementary/frankfurt00001_068208_gt.png}
  }
  \subfigure{%
    \includegraphics[width=.18\columnwidth]{figures/supplementary/frankfurt00001_068208_cnn.png}
  }
  \subfigure{%
    \includegraphics[width=.18\columnwidth]{figures/supplementary/frankfurt00001_068208_ours.png}
  }\\[-2ex]

  \subfigure{%
    \includegraphics[width=.18\columnwidth]{figures/supplementary/frankfurt00001_082466_given.png}
  }
  \subfigure{%
    \includegraphics[width=.18\columnwidth]{figures/supplementary/frankfurt00001_082466_sp.png}
  }
  \subfigure{%
    \includegraphics[width=.18\columnwidth]{figures/supplementary/frankfurt00001_082466_gt.png}
  }
  \subfigure{%
    \includegraphics[width=.18\columnwidth]{figures/supplementary/frankfurt00001_082466_cnn.png}
  }
  \subfigure{%
    \includegraphics[width=.18\columnwidth]{figures/supplementary/frankfurt00001_082466_ours.png}
  }\\[-2ex]

  \subfigure{%
    \includegraphics[width=.18\columnwidth]{figures/supplementary/lindau00033_000019_given.png}
  }
  \subfigure{%
    \includegraphics[width=.18\columnwidth]{figures/supplementary/lindau00033_000019_sp.png}
  }
  \subfigure{%
    \includegraphics[width=.18\columnwidth]{figures/supplementary/lindau00033_000019_gt.png}
  }
  \subfigure{%
    \includegraphics[width=.18\columnwidth]{figures/supplementary/lindau00033_000019_cnn.png}
  }
  \subfigure{%
    \includegraphics[width=.18\columnwidth]{figures/supplementary/lindau00033_000019_ours.png}
  }\\[-2ex]

  \subfigure{%
    \includegraphics[width=.18\columnwidth]{figures/supplementary/lindau00052_000019_given.png}
  }
  \subfigure{%
    \includegraphics[width=.18\columnwidth]{figures/supplementary/lindau00052_000019_sp.png}
  }
  \subfigure{%
    \includegraphics[width=.18\columnwidth]{figures/supplementary/lindau00052_000019_gt.png}
  }
  \subfigure{%
    \includegraphics[width=.18\columnwidth]{figures/supplementary/lindau00052_000019_cnn.png}
  }
  \subfigure{%
    \includegraphics[width=.18\columnwidth]{figures/supplementary/lindau00052_000019_ours.png}
  }\\[-2ex]




  \subfigure{%
    \includegraphics[width=.18\columnwidth]{figures/supplementary/lindau00027_000019_given.png}
  }
  \subfigure{%
    \includegraphics[width=.18\columnwidth]{figures/supplementary/lindau00027_000019_sp.png}
  }
  \subfigure{%
    \includegraphics[width=.18\columnwidth]{figures/supplementary/lindau00027_000019_gt.png}
  }
  \subfigure{%
    \includegraphics[width=.18\columnwidth]{figures/supplementary/lindau00027_000019_cnn.png}
  }
  \subfigure{%
    \includegraphics[width=.18\columnwidth]{figures/supplementary/lindau00027_000019_ours.png}
  }\\[-2ex]



  \setcounter{subfigure}{0}
  \subfigure[\scriptsize Input]{%
    \includegraphics[width=.18\columnwidth]{figures/supplementary/lindau00029_000019_given.png}
  }
  \subfigure[\scriptsize Superpixels]{%
    \includegraphics[width=.18\columnwidth]{figures/supplementary/lindau00029_000019_sp.png}
  }
  \subfigure[\scriptsize GT]{%
    \includegraphics[width=.18\columnwidth]{figures/supplementary/lindau00029_000019_gt.png}
  }
  \subfigure[\scriptsize Deeplab]{%
    \includegraphics[width=.18\columnwidth]{figures/supplementary/lindau00029_000019_cnn.png}
  }
  \subfigure[\scriptsize Using BI]{%
    \includegraphics[width=.18\columnwidth]{figures/supplementary/lindau00029_000019_ours.png}
  }%\\[-2ex]

  \mycaption{Street Scene Segmentation}{Example results of street scene segmentation.
  (d)~depicts the DeepLab results, (e)~result obtained by adding bilateral inception (BI) modules (\bi{6}{2}+\bi{7}{6}) between \fc~layers.}
\label{fig:street_visuals-app}
\end{figure*}



\end{document}

