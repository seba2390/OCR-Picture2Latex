\documentclass[aps,twocolumn,nofootinbib,longbibliography,prd]{revtex4-1}
\usepackage[babel]{csquotes}
\usepackage{graphicx}
\usepackage{amsmath,amssymb}
\usepackage[colorlinks,citecolor=blue,linkcolor=blue,urlcolor=blue]{hyperref}
\usepackage{mathrsfs}
\usepackage{enumerate}
\def\be{\begin{equation}}
\def\ee{\end{equation}}
\newcommand{\citazione}[1]{\cite{#1}}

\begin{document}

\title{Black holes have more states than those giving the\\ Bekenstein-Hawking entropy: a simple argument.}

\author{Carlo Rovelli}

\affiliation{CPT, Aix-Marseille Universit\'e, Universit\'e de Toulon, CNRS, Case 907, F-13288 Marseille, France.}
\date{\small\today}

\begin{abstract} 
\noindent  It is often assumed that the maximum number of independent states a black hole may contain is $N_{BH}=e^{S_{BH}}$, where $S_{BH}=A/4$ is the Bekenstein-Hawking entropy and $A$ the horizon area in Planck units.  I present a simple and straightforward  argument showing that the number of states that can be distinguished by local observers inside the hole must be greater than this number. \end{abstract} 

\maketitle

There are several arguments supporting the idea that the thermodynamical interaction between a black hole and its surroundings is well described by treating the black hole as a system with $N_{BH}=e^{A/4}$ (orthogonal) states, where $A$ is the horizon area in Planck units $\hbar=G=c=1$.   These arguments are convinging.   However, it has then become fashionable to deduce from this fact that the black hole itself cannot have more than $N_{BH}$ states (see for instance the discussion in \cite{don} and references therein). I present here an argument indicating that this further step is wrong and that the actual number $N$ of independent states of a black hole of area $A$ can be larger than $N_{BH}$. 

The possibility of a distinction between $N$ and $N_{BH}$ is opened by the fact that according to classical general relativity the interaction between a black hole and its surroundings is entirely determined by what happen in the vicinity of the horizon. This may be true in general, and therefore it is possible that $N_{BH}$ counts only states that can be distinguishable from the exterior, which may be called ``surface" states. On the other hand, $N$ counts also states that can be distinguished by local observables \emph{inside} the horizon.  Here I argue that to have more states than $N_{BH}$ is not just a possibility: it follows from elementary considerations of causality. 

\begin{figure}[b]
\includegraphics[height=4cm]{BH2.pdf}
\caption{The (lowest part) of the conformal diagram of a gravitational collapse. The clear grey region is the object, the dotted line is the horizon, the thick upper line is the singularity, the dark upper region is where quantum gravity effect may become relevant (this region play no role in this paper.)  The two Cauchy surfaces used in the paper are the dashed lines.}
\label{uno}
\end{figure}

To show this, consider a gravitationally collapsed object and let $\Sigma_1$ be a Cauchy surface that crosses the horizon but does not hit the singularity, see Figure 1. Let $\Sigma_2$ be a later similar Cauchy surface and $i=1,2$.  Let $A_i$ be the area of the intersection of  $\Sigma_i$ with the horizon. Assume that no positive energy falls into the horizon during the interval between the two surfaces.  Let quantum fields live on this geometry, back-reacting on it \cite{qft}. Finally, let $\Sigma_i^{in}$ be the (open) portions of $\Sigma_i$ inside the horizon. 

Care is required in specifying what is meant here by `horizon', since there are several such notions (event horizon, trapping horizon, apparent horizon, dynamical horizon...) which in this context may give tiny (exponentially small in the mass) differences in location.  For precision, by `horizon' I mean here the event horizon, if this exist. If it doesn't (as for instance in \cite{BHbounce}), I mean the boundary of the past of a late-time spacelike region lying outside the black hole (say outside the trapping region). With this definition, the horizon is light-like. 

Because of the back-reaction of the Hawking radiation, the area of the horizon shrinks and therefore  
\be
            A_2 < A_1. 
            \label{uno}
\ee
Now consider the evolution of the quantum fields from $\Sigma_1$ to $\Sigma_2$. We are in a region far away from the singularity and therefore (assuming the black hole is large) from high curvature.  Therefore we expects conventional quantum field theory to hold here, without  strange quantum gravity effects, at least up to high energy scales.  Since the horizon is light-like, $\Sigma_1^{in}$ is in the causal past of $\Sigma_2^{in}$.  This implies that any local observable on $\Sigma_1^{in}$ is fully determined by observables on  $\Sigma_2^{in}$.  That is, if ${\cal A}_i$ is the local algebra of observables on  $\Sigma_i^{in}$ then ${\cal A}_1$ is a subalgebra of ${\cal A}_2$:
\be
{\cal A}_1\subset {\cal A}_2.
\ee
Therefore any state on ${\cal A}_2 $ is also a state on ${\cal A}_1$ and if two such states can be distinguished by observables in ${\cal A}_1$ they certainly can be distinguished by observables in ${\cal A}_2$ as the first are included in the latest. Therefore the states that can be distinguished by ${\cal A}_1$ ---which is to say: on $\Sigma_1^{in}$--- can also be distinguished by ${\cal A}_2$ ---which is to say: on $\Sigma_2^{in}$.  Therefore the distinguishable states on $\Sigma_1^{in}$ are a subset of those in $\Sigma_2^{in}$.  How many are them? Either there is an infinite number of them, or a finite number due to some high-energy (say Planckian) cut-off. If there is an infinite number of them, then immediately the number of states distinguishable from inside the black hole is larger that $N_{NB}$, which is finite.   If there is a finite number of them, then the number $N_2$ of distinguishable states on $\Sigma_2^{in}$ must be equal or larger than the number $N_2$ of states distinguishable on $\Sigma_1^{in}$, because the second is a subset of the first. That is
\be
            N_2\ge N_1. 
            \label{due}
\ee
Comparing equations \eqref{uno} and \eqref{due} shows immediately that it is impossible that $N_i=e^{A_i/4}$, as the exponential is a monotonic function.  

The conclusion is that the number of states distinguishable from the interior of the black hole must be different from the number $N_{BH}=e^{A/4}$ of the states contributing to the Bekenstein-Hawking entropy.  Since the second is shrinking to zero with the evaporation, the first must overcome the second at some point. Therefore  in the interior of a black hole there are more possible states than $e^{A/4}$.  

The physical interpretation of the conclusion is simple: the thermal behaviour of the black hole described by the Bekenstein-Hawking entropy $S=A/4$ is fully determined by the physics of the vicinity of the horizon. 

In classical general relativity, the effect of a black hole on its surroundings is independent from the black hole interior.  A vivid expression of this fact is in the numerical simulations of black hole merging and radiation emission by oscillating black holes: in writing the numerical code, it is routine to  cut away a region inside the (trapping) horizon: it is irrelevant for whatever happens outside!  This is true in classical general relativity, and there is no compelling reason to suppose it to fail if quantum fields are around.  Therefore a natural interpretation of $S_{BH}$ is to count states of near-surface degrees of freedom, not interior ones.  This is of course not a new idea: it has a long history \cite{1,2,3,4,5,6,7,8,9} and see in particular \cite{LQG-BH} and \cite{Strominger} in support of this idea from two different research camps, loops and strings.   The argument presented here strongly support this idea, by making clear that there are interior states that do not affect the Bekenstein-Hawking entropy. 

This conclusion is not in contrast with the the various arguments leading to identify Bekenstein-Hawking entropy with a counting of states. To the opposite, evidence from it comes from the membrane paradigm \cite{membrane} and from Loop Quantum Gravity \cite{LQG-BH,LQG-BH2}, which both show explicitly that the relevant states are surface states, but also from the string theory counting \cite{StringBH,StringBH2}, because the counting is in a context where the relevant state space is identified with the scattering state space, which could be blind to interior observables.

The consequences of this observation are far reaching for the discussions on the black-hole information paradox \cite{don,thooft}.  The solid version of the paradox is Page's \cite{Page}, which does not require hypotheses on the future of the hole. If there are more states available in a black hole than $e^{A/4}$, then Page argument for the information loss paradox fails. Page argument is based on the fact that if the number of black hole states is determined by the area, then there are no more available state to be entangled with the Hawking radiation when the black hole shrinks. For the radiation to be thermal it must be entangled with something, and the only option is earlier Hawking quanta, and this is in tension with quantum field theory.  But if there can be many states also inside a black hole with small horizon area, then late-time Hawking radiation does not need to be correlated with early time Hawking radiation, because it can simply be correlated with internal black hole states, even when the surface area of the back hole has become small. 

Recall indeed that the interior of an old black hole can have large volume even if its horizon has small area. It was in fact shown in \cite{BHvolume} that at a time $v$ after the collapse, a black hole with mass $m$ has interior volume 
\be
V\sim 3\sqrt{3} \pi\; m^2 v
\ee
for $v\ll m$. See also \cite{BHvolume2, ing,yen,yen2,shao}. This volume may store large number of states.  When the evaporation ends (because the hole has become small, or earlier if the hole is disrupted by non perturbative quantum gravitational effects \cite{BHbounce,BHbounce2}) this information can leak out, possibly slowly, if much of it is in long wavelength modes (see \cite{abhay, met} and references therein).  Therefore information \emph{can} emerge from the hole, before total dissipation, and is not lost. 

I do realize that these observations go against diffused prejudices regarding holography, but I think we should not be blocked by prejudices. The result presented here does not invalidate holographic ideas: it sharpens them by pointing out that what is bound by the area of the boundary of a region is not the number of possible states in the region, but only the number of states distinguishable from observations outside the region.

\centerline{---}

I thank Don Marolf, Tommaso De Lorenzo, Alejandro Perez and Eugenio Bianchi for crucial conversations that have lead to this result. 

\begin{thebibliography}{1}

\bibitem{don} D.~Marolf: The Black Hole information problem: past, present, and future, Rept. Prog. Phys. 80 (2017) no.9, 092001 

\bibitem{qft} R.~M.~Wald: Quantum Field Theory in Curved Spacetime and Black Hole Thermodynamics, Chicago: University of Chicago Press. 1994.

\bibitem{BHbounce} C.~Rovelli, F.~Vidotto: Planck Stars, International Journal of Modern Physics D23 (2014) 12, 1442026. 

\bibitem{membrane}  K.~S.~Thorne, R.~H.~Price, D.~A.~Macdonald (Eds.): Black Holes: the Membrane Paradigm. New Haven: Yale University Press 1986.

\bibitem{LQG-BH} C.~Rovelli: Loop Quantum Gravity and Black Hole Physics, Helvetica Physica Acta 69 (1996) 582; gr-qc/9608032. 

\bibitem{LQG-BH2} C.~Rovelli, Black Hole Entropy from Loop  Quantum Gravity, Physical Review Letter 14 (1996) 3288. 

\bibitem{StringBH}  A.~Strominger, C.~Vafa: Microscopic origin of the Bekenstein-Hawking entropy. Physics Letters B379  (1996) 99�104.

\bibitem{1} J. W. York: Dynamical origin of black-hole radiance, Phys. Rev. D28 (1983) 2929.

\bibitem{2} W. H. Zurek and K. S. Thorne: Statistical Mechanical Origin of the Entropy of a Rotating, Charged Black Hole, Phys. Rev. Lett. 54 (1985) 2171.

\bibitem{3} J.~A.~Wheeler, A Journey into Gravity and Spacetime, New York: Freeman1990.

\bibitem{4} G.~�t~Hooft: The black hole interpretation of string theory, Nucl. Phys. B335 (1990) 138.

\bibitem{5} L.~Susskind, L.~Thorlacius, R.~Uglum, The Stretched Horizon and Black Hole Complementarity, Phys. Rev. D48 (1993) 3743.

\bibitem{6} V.~Frolov, I.~Novikov: Dynamical Origin of the Entropy of a Black Hole, Phys. Rev. D48 (1993) 4545.

\bibitem{7} S.~Carlip: The Statistical Mechanics of the (2+ 1)-Dimensional Black Hole,  Phys. Rev. D51 (1995) 632.

\bibitem{8} M.~Cvetic, A.~Tseytlin: Solitonic strings and BPS saturated dyonic black holes, Phys. Rev. D53 (1996) 5619.

\bibitem{9} F.~Larsen, F.~Wilczek: Internal structure of black holes, Phys. Lett. B375 (1996) 37.

\bibitem{StringBH2} G.~Horowitz, A.~Strominger: Counting states of near-extremal black holes. Physical Review Letters 77  (1996) 2368�2371.  

\bibitem{Strominger} A.~Strominger: Black hole entropy from near-horizon microstates, JHEP 9802 (1998) 009.

\bibitem{thooft} G.~'t Hooft, S.~B.~Giddings, C.~Rovelli, P.~Nicolini, J.~Mureika, M.~Kaminski, M.~Bleicher: The Good, the Bad, and the Ugly of Gravity and Information, panel discussion held at the 2nd Karl Schwarzschild Meeting, Frankfurt am Main, Germany, July 2015; to be published in Springer Proceedings in Physics, arXiv:1609.01725. 

\bibitem{Page} D.~M.~Page: Information in black hole radiation. Physical Review Letters 71  (1993) 3743�3746. 

\bibitem{BHvolume} M.~Christodoulou, C.~Rovelli: How big is a black hole?  Physical Review D 91 (2015) 064046,  arXiv:1411.2854.  

\bibitem{BHvolume2}  T.~De Lorenzo M.~Christodoulou: Volume inside old black holes, Physical Review D94 (2016) 104002, arXiv:1604.07222. 

\bibitem{ing} I.~Bengtsson, E.~Jakobsson, Black holes: Their large interiors, Mod.Phys.Lett. A30 (2015) no.21, 1550103. 

\bibitem{yen} Yen Chin Ong:  Never Judge a Black Hole by Its Area, JCAP 1504 (2015) no.04, 003

\bibitem{yen2} Yen Chin Ong:  The Persistence of the Large Volumes in Black Holes, Gen.Rel.Grav. 47 (2015) no.8, 88. 

\bibitem{shao}  Shao-Jun Wang, Xin-Xuan Guo, Towe Wang: Maximal volume behind horizons without curvature singularity, arXiv:1702.05246. 

\bibitem{BHbounce2} M.~Christodoulou, C.~Rovelli, S.~Speziale, I.~Vilensky: Realistic Observable in Background-Free Quantum Gravity: the Planck-Star Tunnelling-Time, Physical Review D 94 (2016) 084035, arXiv:1605.05268. 

\bibitem{abhay}A.~Ashtekar and M.~Bojowald,: Black hole evaporation: A paradigm,  Classical and Quantum Gravity 22 (2005)  3349--3362. 

\bibitem{met} E.~Bianchi, T.~ De Lorenzo, M.~ Smerlak: Entanglement entropy production in gravitational collapse: covariant regularization and solvable models
Eugenio Bianchi, JHEP 1506 (2015) 180

\end{thebibliography}

\end{document}


\noindent 



\bibliographystyle{/Users/carlorovelli/Documents/utcaps}
\bibliography{/Users/carlorovelli/Documents/library}
\end{document}

\noindent 
%\bibliography{./kmax}

\providecommand{\href}[2]{#2}\begingroup\raggedright\begin{thebibliography}{1}


\end{thebibliography}\endgroup


\end{document}









