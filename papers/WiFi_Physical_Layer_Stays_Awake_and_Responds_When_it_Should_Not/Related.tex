\section{Related Work}
The loopholes we present in this paper are explored using packet injection, in which an attacker sends fake WiFi packets to devices in a secured WiFi network.
Packet injection has been used in the past to perform various types of attacks against WiFi networks
such as denial of service attacks for a particular client device or total disruption of the network~\cite{vanhoef2020protecting, dos, rogue-ap, deauth}. These attacks use different approaches such as beacon stuffing to send false information to WiFi devices~\cite{beacon-stuffing-1, beacon-stuffing-2}, or Traffic Indication Map (TIM) forgery to prevent clients from receiving data ~\cite{bellardo2003802, tim-forgery}. However, all of these attacks focus on spoofing 802.11 MAC-layer management frames to interrupt the normal operation of WiFi networks. 
To provide a countermeasure for some of these attacks, the 802.11w standard~\cite{ieee802.11w} 
introduces a protected management frame that prevents attackers from spoofing 802.11 management frames. 
Instead of spoofing 802.11 MAC frames, we exploit properties of the 802.11 physical layer to force a device to stay awake and respond when it should not. 
These loopholes open the door to multiple research avenues including new security and privacy threats. 


%\textcolor{blue}{Our recent preliminary work has shown that all WiFi devices respond with ACKs to packets received from outside of their network~\cite{abedi2020wifi}. However, this workshop paper does not show how to keep WiFi devices awake and avoid going to sleep mode. Moreover, it does not explore turning WiFi devices into sensitive motion sensors and monitoring people's breathing rates. In contrast,  \name\ shows the first attack which forces a WiFi device to be awake and pushes it to continuously transmit. We show how our technique enables an attacker to monitor the breathing rate of people by analyzing their WiFi signals.}

% The related work can be divided into three categories. The first category is WiFi sensing systems that use WiFi signals to infer some information, such as gesture detection, to enable a useful application for the user (the good). The second category is WiFi attacks which show how an attacker can interfere with the normal operation of WiFi networks (the bad). Finally, some recent studies show new privacy attacks against users by analyzing their WiFi signals (the ugly).


%Over the past decade, there has been a significant amount of research on WiFi sensing where WiFi signals are used to detect human activities~\cite{wifi-sensing-survey} to enable useful applications. These systems target different applications such as tracking and localization\cite{adib2013see, activity-recognition-1}, human detection \cite{gong2016adaptive, gong2015wifi}, gesture recognition \cite{abdelnasser2015wigest, gesture-recognition-1, gesture-recognition-2, gesture-recognition-3} and vital measurement such as respiration rate~\cite{abdelnasser2015ubibreathe, adib2015smart, breathing-rate-1, breathing-rate-2}. However, these systems target applications with social benefits and cannot be easily used by an attacker to create privacy and security threats. This is because either these techniques require cooperation from the target WiFi device or the attacker needs to be very close to the target to use these systems.


%WiFi Sensing is a technique that uses ambient WiFi signals to detect events or human activities~\cite{wifi-sensing-survey}. The motivation behind WiFi sensing is that we can obtain certain information without dedicated sensors. In particular, WiFi sensing techniques analyze changes in WiFi signals to infer different types of information. As mentioned earlier in Section~\ref{sec:csi}, CSI has been shown to be well suited for sensing techniques. For instance, there are applications that can identify the number of people in a closed room and their relative locations\cite{adib2013see, activity-recognition-1}. Applications that  develop wireless device-free human detection \cite{gong2016adaptive, gong2015wifi} are also implemented to determine the presence of human activities. Other than this, subtle movements like gesture recognition \cite{abdelnasser2015wigest, gesture-recognition-1, gesture-recognition-2, gesture-recognition-3} and vital measurement such as respiration rate~\cite{abdelnasser2015ubibreathe, adib2015smart, breathing-rate-1, breathing-rate-2}, can also be measured using wireless sensing. These applications are great tools that bring convenience to people's lives.However, for these techniques to work WiFi devices should cooperate to enable WiFi sensing. Therefore, an attacker cannot use these techniques to perform WiFi sensing since he/she has no access to the target building and the devices inside it.

% \subsection{The Bad}
% \label{sec:stealth}


%\emph{Beacon Injection:} 
%This attack is performed by forging 802.11 beacons and broadcasting them to all devices in a WiFi network~\cite{beacon-stuffing-1, beacon-stuffing-2}. The attacking device pretends to be the actual access point and injects false information in the forged beacons to enable a variety of attacks such as the ``evil twin access point'' attack~\cite{evil-twin}. 
%Since beacons are broadcasted to all devices, this will attack all the devices at the same time. The forged beacon frames can also be sent (i.e., unicast) to a particular device to attack individual devices rather than the entire network.

%\emph{TIM Forgery:}
%Traffic Indication Map (TIM), as mentioned in Section \ref{sec:beacon}, is used in 802.11 beacon frames. It contains information about whether sleeping devices have buffered packets at Access Point (AP) or not. It is suggested that an adversary can manipulate the Time Indication Map (TIM) inside beacons to change the behavior of WiFi devices~\cite{bellardo2003802, tim-forgery}. \name\ builds on these attacks. In particular, \name\ forges the TIM to make a device believe that it has buffered data to receive, so it cannot enter the sleep mode.
%For instance, the adversary can forge the TIM to make a device believe that it has buffered data to receive, so it cannot enter the sleep mode. In contrast, it can also prevent a device from receiving any data by telling it that the AP  has no buffered data for it. 


\textbf{WiFi sensing attack:} Over the past decade, there has been a significant amount of research on WiFi sensing where WiFi signals are used to detect human activities~\cite{iot-wifi-localization, rf-sensing, wifi-sensing-survey,adib2015smart, breathing-rate-1, breathing-rate-2,gesture-recognition-1, gesture-recognition-2, gesture-recognition-3,pu2013whole}. However, these systems target applications with social benefits and cannot be easily used by an attacker to create privacy and security threats. This is because either these techniques require cooperation from the target WiFi device or the attacker needs to be very close to the target to use these systems. A recent study shows that by capturing WiFi signals coming out of a private building, it is possible for an adversary to track user movements inside that building~\cite{zhu2018tu}. However, this attack has a bootstrapping stage which requires the attacker to walk around the target building for a long time to find the location of the WiFi devices. Furthermore, since this work relies on only the normal intermittent WiFi activities, it cannot capture continuous data such as breathing rate.  

\textbf{Battery draining attack:} 
Battery draining attacks date back to 1999 \cite{stajano1999resurrecting} and there have been many studies on such attacks and potential defense mechanisms since then~\cite{caviglione2012energy}.
%Battery draining attack was first introduced by Stajano, F. and Anderson, R. in 1999 \cite{stajano1999resurrecting}, where they prophesied that the battery of a mobile device can be exhausted by a malicious user even via legitimate usage. Extensive studies have been conducted to investigate the attack and defense manners. 
%One possible approach is by constantly launching canonical attacks and forcing defense systems to attempt to defeat them. Running defending software consumes CPU resources which leads to quick drainage of battery energy \cite{caviglione2012energy}. 
Battery discharge models and energy vulnerability due to operating systems have been investigated \cite{zhang2010accurate,jindal2013hypnos}. A more recent study plays multimedia files implicitly to increase power consumption during web browsing \cite{fiore2014multimedia, fiore2017exploiting}. In terms of defending, a monitoring agent that searches for abnormal current draw is discussed in \cite{buennemeyer2008mobile}. In contrast, our attack exploits the loopholes in the 802.11 physical layer protocol and the power-hungry WiFi transmission to quickly drain a target device's battery. We will discuss in Section~\ref{sec:cannot-be-fixed} that stopping our proposed attack is nearly impossible on today's WiFi devices.


This paper is an extension of our previous workshop publication ~\cite{polite-wifi}. The workshop paper shows preliminary results for our finding that WiFi devices respond with ACKs to packets received from outside of their network, and provides a brief discussion on potential privacy and security concerns of this behavior without studying them. We have also explored how the WiFi power saving mechanism can be exploited to keep a target device awake in a localization attack~\cite{wi-peep}. 
In this paper, we provide an in-depth study of these previously discovered loopholes. We also design and perform two privacy and security attacks, based on these loopholes. Finally, we implement these attacks on off-the-shelve WiFi devices and present detailed performance evaluations.


% To the best of our knowledge, \name is the first attack that forces WiFi devices to continuously transmit, enabling an attacker, who does not have access to the network, to estimate the breathing rate of a person from outside of the building using a low-cost WiFi module.

%\name\ combines techniques from WiFi sensing, attacks against WiFi networks to enable a new privacy attack to show for the first time that an attacker can estimate the respiration rate of a person from outside of the building.


%These attacks relies on the fact that wireless signals pass through walls; therefore, an attacker who is outside a building can receive signals coming from inside that building. The attacker can analyze the distortions in WiFi signals, caused by the body of a target person, to infer various types of information from a target building.

%To make the problem worse, it has been shown that WiFi devices send acknowledgments (ACK) when they receive fake packets coming from outside the their network~\cite{abedi2020wifi}. This behavior, called \emph{Polite WiFi}, enlightens the possibility of turning any WiFi device into a secret sensor.  This is because the CSI information can be extract from the ACKs send by a victim device to perform WiFi sensing and potentially obtain some sensitive information.
%\name\ combines techniques from WiFi sensing, attacks against WiFi networks to enable a new privacy attack to show for the first time that an attacker can estimate the respiration rate of a person from outside of the building.
%We also propose potential methodologies to protect against this attack.


% Another privacy attack which uses WiFi to extract the International Mobile Subscriber Identity (IMSI) from mobile phones \cite{o2017mobile}. The leakage of IMSI may results in potential tracking of the devices through link to the target user's other hardware addresses such as WiFi MAC addresses. As a result, the mobile device can be turned into a secret tracker which works for the attacker.


% some features of the techniques mentioned above like \emph{Polite WiFi} and TIM forgery, and it can detect the target people's breathing rate without their notice. The leak of breathing rate may seem to be harmless, but attacker can use this information to further infer people's presence in the house.




