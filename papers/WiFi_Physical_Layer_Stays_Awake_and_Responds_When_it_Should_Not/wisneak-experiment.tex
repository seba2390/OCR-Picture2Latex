\section{Evaluation \& Analysis}

%In order to evaluate this attack, a set of experiments are done. These experiments can be mainly divided into two parts: 

To evaluate the feasibility of \wisneak, we conduct a set of experiments in different scenarios. In the following, we first describe our attack scenarios and experiment setup. We then present evaluation results on accuracy and effective range of \wisneak. 

\subsection{Attack Scenarios}
\label{sec:setup}
We evaluate \wisneak\ in two different scenarios: Indoor and Outdoor. In the indoor scenario, the attacker and the target are placed in the same building but on different floors. The height of one floor in the building is around $2.8$ m. 
This scenario is similar to when the attacker and the target person are in different units of an apartment or townhouse.
In the outdoor scenario, the attacker is outside the target's house. For each scenario, we evaluate the attack for different target and attacker locations, as shown in Figure \ref{fig:floor_plan}. For example, for the indoor experiments, the attacker is at location C, the basement of the house, while the target person is at either locations A or B on the first floor. For the outdoor experiments, the attacker is at location D, the backyard, which is around $3$ m away from the outside wall of the house. The target person is at locations A, B, or C. In all of the experiments, the target WiFi devices are placed $0.5$ to $1.4$ m away from the target person's body. The target person is either watching a movie, typing on a laptop, or surfing the web using his cellphone during the experiments. During the experiments, other people are walking and living normally in the house.


\subsection{Attacker Hardware}
The attacker uses a Linksys AE6000 WiFi card and an ESP32 WiFi module~\cite{esp32} as the attacking device. Both devices are connected to a ThinkPad laptop via USB. The Linksys AE6000 is used to send fake packets and the ESP32 WiFi module is used to receive acknowledgments (ACK) and extract CSI. Although we use two different devices for sending and receiving, one can simply use an ESP32 WiFi module for both purposes.
The use of two separate modules gave us more flexibility in running many experiments.

As for the target device, we use a One Plus 8T smartphone without any software or hardware modifications. We have also tested our attack on an unmodified Lenovo laptop, a Microsoft Surface Pro 4 laptop, and a USB WiFi card as the target device and we obtained similar results. It is worth mentioning that any WiFi device can be a target for \wisneak, without any software or hardware modification. 


\begin{figure}[!t]
    \centering
    \includegraphics[width=0.4\textwidth]{figures/floorplan.pdf}
    \caption{Our evaluation testbed. For our indoor experiments, the attacker is placed at location C in the basement of the house, while the target is tested at both locations A and B on the first floor. For our outdoor experiments, the attacker is placed at location D outside the house while the target device is tested at locations A, B, and C.
    } 
    \vspace{-0.1in}
    \label{fig:floor_plan}
\end{figure}

\subsection{Attacker Software}
We have implemented the CSI collecting script on the ESP32 WiFi module, and the breathing rate estimation algorithm (described in Section~\ref{sec:algo}) on the laptop. The collected CSI data is fed to the algorithm which produces the breathing rate estimation values in real-time. 

Upon running the program, the ACK packets constantly flow into \wisneak. To process this data in real-time, a sliding window (buffer) is used. The size of the window is $30$ s and the stride step is $1$ s. 
$30$ seconds is a large enough window for estimating a stable breathing rate value. Note that an adult breathes around 6 times during such a window. The window is a queue of data points, and it updates every second by including $1$ second of new data points to its head and removing $1$ second of old data points from its tail. \wisneak\  runs the analysis algorithm on the data points inside the window whenever it is updated. The window slides once per second. Hence, \wisneak\  reports an estimation of breathing rate every second. Note that there is a $30$-second delay at the beginning, since the window needs to be filled first.

%\begin{figure}[!t]
%    \centering
%    \includegraphics[width=0.3\textwidth]{figures/sliding.png}
%    \caption{Real-time analysis with sliding window}
%    \label{fig:sliding}
%\end{figure}


\subsection{Ground Truth}
\label{sec:ground_truth}
To evaluate the attack's accuracy of estimating the breathing rate, we need to measure the breathing rate ground truth. %A simple method to measure the ground truth is to count the number of breaths per minute, since this does not require any equipment and it is very accurate for a duration of $1$ min. However, \wisneak\  reports a result every second, but it is very hard for counting method to update a number every second to match our estimations. 
To do so, we use a similar method as introduced in \cite{dafna2015sleep}. In particular, we place a microphone near the target person's nose to record the sound of breathing. We then take an FFT of the sound signal to estimate the breathing rate accurately. However, each peak of the sound signal is noisy. Therefore, an envelope function is applied before taking the FFT such that these noises are removed. Note that the use of a microphone is just to obtain the ground truth data and evaluate the performance of \wisneak.

%\begin{figure}[!t]
%    \centering
%    \includegraphics[width=0.4\textwidth]{figures/audio.pdf}
%    \caption{An audio signal is captured by placing a microphone close to the target person's nose. The signal is used to measure the breathing rate ground truth.}
%    \label{fig:audio}
%\end{figure}


Similar to processing the WiFi data, we also apply the sliding window method to the ground truth data. Thus, the results from both the ground truth and WiFi signal can be compared with each other. In addition, the ground truth data and the results produced by \wisneak need to be synchronized in the time domain. To do so, once the program starts, we start a timer that serves as a coordinator. When the timer reaches one minute, the ground truth recorder starts, which marks the start of the experiment. 
%as shown in Figure \ref{fig:sync}. 
The one-minute synchronizing time also covers the $30$s starting delay of \wisneak, so that the window has already buffered enough data for analysis.

%\begin{figure}[!t]
%     \centering
%        \includegraphics[width=0.3\textwidth]{figures/syncrho.png}
%    \caption{Synchronization of the ground truth measurements and \wisneak's estimations}
%    \label{fig:sync}
%\end{figure}


\subsection{Effectiveness}
\label{sec:results}
%For performance metrics, we designed experiments to evaluate several aspects of the system:
%\begin{itemize}
%    \item \textbf{Sensitivity}: Whether the attack is able to detect a change of breathing rate, and whether it can detect if the target person leaves the device 
   % \item \textbf{Accuracy}: %Whether the estimation is accurate enough compared to the ground truth
%    \item \textbf{Effective Range}: How far can the target person be from the target device
%\end{itemize}

We first evaluate the effectiveness of \wisneak\  in detecting breathing by performing an experiment where a person changes his breathing rate. The goal is to see if \wisneak can correctly detect this change in the breathing rate. The attacker is outside the house at location D and the target person is at location A. This experiment lasts three minutes and the target person breathes slowly during the first minute, fast during the following minute, and back to slow again during the last minute. As shown in Figure \ref{fig:change}, \wisneak\  can accurately capture changes in the breathing rate and it matches the ground truth data. This controlled experiment shows that although a person's breathing rate does not usually change so suddenly, \wisneak\  is sensitive enough to capture even such changes. 

\begin{figure}
    \centering
    \includegraphics[width=0.36\textwidth]{figures/control.png}
    \caption{The capability of \wisneak in detecting changes in the breathing rate of a target person.}
    \label{fig:change}
\end{figure}

Next, we run an experiment to examine the \wisneak's capability in detecting whether there is a target person near the WiFi device or not. In this experiment, the target phone is placed on a desk and the person stays around the device for $30$ seconds, then walks away from the device, and then comes back near the device. Note, in our algorithm, when there is no majority vote during the voting phase, we return $-1$ to indicate no breathing detected. Figure~\ref{fig:absence} shows the results of this experiment. As illustrated in the figure, \wisneak can correctly detect the breathing rate when the person is near the device. In other words, it can detect if there is no one near the target device and refrain from reporting a random value.

\begin{figure}[!t]
    \centering
    \includegraphics[width=0.36\textwidth]{figures/absence.png}
    \caption{The efficacy of \wisneak when there is no target near the WiFi device.}
    \label{fig:absence}
\end{figure}



\begin{table}[!t]
    \centering
    \begin{tabular}{V{2.5}c|c|c|c|c|c|cV{2.5}}
    \clineB{1-7}{2.5}
    Tar & Att & Method & 12 & 15 & 20 & 30 \\

    
        \hline     \clineB{1-7}{2.5}

    \multirow{2}{*}{A} & \multirow{2}{*}{C} & \wisneak & 11.97 & 15.03 & 20.01 & 29.94 \\
                                        \cline{3-7}
                                        & & Ground Truth & 12.05 & 14.91 & 20.06 & 30.17 \\
        \clineB{1-7}{2.5}

    \multirow{2}{*}{B} & \multirow{2}{*}{C} & \wisneak & 11.95 & 15.00 & 19.97 & 29.94 \\
                                        \cline{3-7}
                                        & & Ground Truth & 11.98 & 15.07 & 20.01 & 30.06 \\
    \hline    \clineB{1-7}{2.5}

    \multirow{2}{*}{A} & \multirow{2}{*}{D} & \wisneak & 11.96 & 15.00 & 19.98 & 29.96 \\
                                        \cline{3-7}
                                          & & Ground Truth & 11.97 & 15.12 & 20.05 & 30.05 \\
    \hline    \clineB{1-7}{2.5}

    \multirow{2}{*}{B} & \multirow{2}{*}{D} & \wisneak & 11.99 & 15.00 & 20.00 & 29.97 \\
                                        \cline{3-7}
                                        & & Ground Truth & 12.08 & 15.04 & 20.11 & 30.05 \\
    \hline    \clineB{1-7}{2.5}

    \multirow{2}{*}{C} & \multirow{2}{*}{D} & \wisneak & 12.00 & 14.94 & 19.98 & 29.95 \\
                                        \cline{3-7}
                                       & & Ground Truth & 12.04 & 15.01 & 20.08 & 30.16 \\

    \hline    \clineB{1-7}{2.5}

    \end{tabular}
    \vspace{10pt}
    \caption{Estimated breathing rate using \wisneak\ compared to the ground truth for different locations of Attacker(Att) and Target(Tar) and different breathing rates (ranging from 12 to 30 BPM).}
    \label{tab:exp}
\end{table}

\begin{figure*}[!t]
    \centering
    
    \setkeys{Gin}{width=\linewidth}
    \begin{tabularx}{\linewidth}{XXX}
    
    \includegraphics[width=0.32\textwidth]{figures/error_bar.png}
    \caption{The average accuracy of \wisneak\ in estimating the target person's breathing rate across all experiments.}
    \label{fig:bar}
    &
    \includegraphics[width=0.32\textwidth]{figures/cdf.PNG}
    \caption{The CDF of \wisneak's error in estimating the target person's breathing rate.}
    \label{fig:out_cdf}
    &
    \includegraphics[width=0.32\textwidth]{figures/orientation-vs-accuracy.png}
    \caption{Estimation accuracy at various orientations}
    \label{fig:orientation_vs_accuracy}
 \end{tabularx}    
\end{figure*}

\subsection{Accuracy}
\label{sec:accuracy}
Next, we evaluate the accuracy of \wisneak\ in estimating the breathing rate in both indoor and outdoor scenarios as explained in Section ~\ref{sec:setup}. For each scenario, we evaluate \wisneak's accuracy when the target's breathing rate is 12, 15, 20, and 30 breaths per minute.  Note, that the normal breathing rate for an adult is 12-20 breaths per minute while resting, and higher when exercising. In this experiment, the user is watching a video. To make sure the target person's breathing rate is close to our desired numbers, we place a timer in front of the person, where they can adjust their breathing rate accordingly. We run each experiment for two minutes. During this time, we collect the estimated breathing rate from both audio (used for the ground truth) and \wisneak. Table~\ref{tab:exp} shows the results of these experiments. The results show that \wisneak\ accurately detects the breathing rate of the target person in different scenarios and for various breathing rates. To quantify the accuracy of \wisneak in estimating the breathing rate, we also plot the average accuracy of \wisneak in estimating breathing rate across all experiments in Figure \ref{fig:bar}. The accuracy is calculated as the ratio of estimated breathing rate reported by \wisneak\ over the ground truth breathing rate. The figure shows that the accuracy of \wisneak is over $99$\% in all scenarios. 
Finally, Figure \ref{fig:out_cdf} plots the Cumulative Distribution Function (CDF) of the error in  detecting breathing rate for over $2400$ measurements. The CDF is generated based on the estimated breathing rate reported every second by \wisneak. Therefore, in each $2$ minute experiment there are $120$ estimated values. The figure shows that $78\%$ of the estimated results have no error. The figure also shows that $99\%$ of measurements have less than one breath per minute error which is negligible. 



%\begin{figure}[!t]
%    \centering
%    \includegraphics[width=0.4\textwidth]{figures/cdf.PNG}
%    \caption{The CDF of \wisneak's error in estimating the target person's breathing rate.}
%    \label{fig:out_cdf}
%\end{figure}

%\begin{figure}
%    \centering
%    \includegraphics[width=0.4\textwidth]{figures/orientation-vs-accuracy.png}
%    \caption{Estimation accuracy at various orientations}
%    \label{fig:orientation_vs_accuracy}
%\end{figure}

\subsection{Orientation}
Next, we evaluate the effect of orientation of the target person with respect to the target device (laptop). We run the same attack as before for different orientations (i.e. sitting in front, back, left, and right side of a laptop). The user is 0.5m away from the target device in all cases. Figure \ref{fig:orientation_vs_accuracy} shows the result of this experiment. Each bar shows the average accuracy for 90 measurements. Our result shows that regardless of the orientation of the person with respect to the device, \wisneak is always effective and detects the breathing rate of the person accurately. In particular, even when the person was behind the target device, \wisneak still detects the breathing rate with 99\% accuracy.




\subsection{Effective Range}

\textbf{Distance of target device to the person:} So far, the target device is placed $0.5$-$0.7$ m away from the target person's body in our experiments. Here, we evaluate \wisneak\ performance for different distances between the target device and the target person. In particular, we are interested to find out what the maximum distance between the target device and the person can be while \wisneak\ still detects the person's breathing rate. To do so, we place the attacker device and the target device 5 meters apart in two different rooms with a wall in between. We then run different experiments in which the target person stays at different distances from the target device. In each experiment, we measure the breathing rate for two minutes and calculate the average breathing rate over this time. Finally, we compare the estimated breathing rate to the ground truth and calculate the accuracy as mentioned before. 

\begin{figure}
    \centering
    \includegraphics[width=0.4\textwidth]{figures/dist.png}
    \caption{Accuracy of \wisneak\ in estimating the breathing rate for different distances between the target person and the target WiFi device.}
    \label{fig:dist}
\end{figure}

Figure \ref{fig:dist} shows the results of this experiment.  The figure shows that \wisneak's accuracy is over $99\%$ when the distance between the target device and the target person is less than $60$ cm. Note, in reality, people have their laptops or cellphone very close to themselves most of the time, and $60$ cm is representative of these situations. The figure also shows that the accuracy drops as we increase the distance. However, even when the device is at 1.4 m from the person's body, the attack can still estimate the breathing rate with $80\%$ accuracy. Note, this is the accuracy in finding the absolute breathing rate and the change in the breathing rate can be detected with much higher accuracy. Finally, the figure shows that the accuracy suddenly drops to zero for distance beyond 1.4 m. This is due to the fact that at that distance the power of the peak at the output of the FFT goes bellow noise floor, and hence, the peak is not detectable.

\begin{figure}
    \centering
    \includegraphics[width=0.4\textwidth]{figures/cross-building-cdf.png}
    \caption{The CDF of \wisneak's error in estimating the target person's breathing rate when attacker and target are in different buildings.}
    \label{fig:cross-building}
\end{figure}

\textbf{Distance of the attacker to the target:} So far, we have evaluated our attack in different scenarios where the target and the attacker are in different rooms or floors of the same home, or the attacker is outside of the target home. Here we push this further and examine whether our attack works if the attacker and the target person are in a totally different building.  We place the target device in a building on a university campus on a weekday with people around. A person is sitting around 0.5 m away from the device.  We then place the attacker in another building which is around 20 m away from the target building.  Similar to the previous experiment, we run the attack and compare the estimated breathing rate with the ground truth. Figure \ref{fig:cross-building} shows the CDF of error for 180 measurements in this experiment. Our results show that the attacker successfully estimates the breathing rate. Note, the reason that the attack works even in such a challenging scenario with other people being around is two-fold. First, as mentioned in \ref{sec:algo}, using an FFT helps to filter out the effect of most non-periodic movements and focus on periodic movements and patterns. Second, wireless channels are more sensitive to changes as we get closer to the transmitter \cite{abedi2020witag}, and since in these scenarios, the target person is very close to the target device, their breathing motion has a higher impact on the CSI signal compared to the other mobility in the environment. 



\subsection{Multiple People}
\begin{figure}[!t]
    \centering
    \begin{subfigure}[b]{0.17\textwidth}
        \centering 
        \includegraphics[width=\textwidth]{figures/side-by-side.jpg}
        \caption{Scenario 1}
        \label{fig:scene1}
    \end{subfigure}
    \hfill
    \begin{subfigure}[b]{0.3\textwidth}
        \centering
        \includegraphics[width=\textwidth]{figures/face-to-face.jpg}
        \caption{Scenario 2}
        \label{fig:scene2}
    \end{subfigure}
    \caption{Illustrations of multi-people scenarios}
    \label{fig:two-scenes}
\end{figure}
Last, we evaluate if \wisneak can be used to detect the breathing rate of multiple people simultaneously. We test our attack in three different scenarios. In the first scenario, two people are near the laptop while one is working on the laptop and the other is just sitting next to him, as shown in Figure \ref{fig:scene1}. The attacker targets the laptop and tries to estimate their breathing rate. Note, that the attacker has no prior information about how many people are next to the laptop. In the second scenario, we repeat the same experiment as the first scenario except that the second person is sitting behind the laptop, as shown in Figure \ref{fig:scene2}. In the third scenario, there are two people in the same space but each person is next to a different device. The attacker targets the laptops and tries to estimate their breathing rates. Note, that the attacker does not require any prior information on how many devices are next to people. It uses our technique explained in section \ref{sec:target_selection} to find devices that are next to people and estimate their breathing rate. In these experiments, the target device is 0.5-0.7 m away from the person. 

\begin{figure}
    \centering
    \includegraphics[width=0.4\textwidth]{figures/scenario-vs-accuracy.png}
    \caption{Accuracy under three different scenarios: Scenario 1: two people sit side-by-side in front of the target device; Scenario 2: one person sits in front of the target device, the other one sits behind the target device; Scenario 3: two people sit in front of two target devices, respectively. Attacker attacks one by one.}
    \label{fig:scenarios_vs_accuracy}
\end{figure}

Figure \ref{fig:scenarios_vs_accuracy} shows the results for this evaluation. The blue bars show the result for the first person who is working on the laptop, and the red bars show the results for the second person. Our results show that \wisneak can effectively detect the breathing rate of both people regardless of the scenario and their orientation. However, the accuracy in detecting the breathing rate for the second person is a bit lower than the first person for the first and second scenarios. This is because the second person's distance to the target device is slightly more and hence the accuracy has decreased. 





