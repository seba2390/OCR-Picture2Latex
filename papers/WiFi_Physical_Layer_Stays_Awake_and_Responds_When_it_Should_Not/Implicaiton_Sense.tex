

\begin{figure*}[th!]
    \centering
    \begin{subfigure}[b]{0.329\textwidth}
        \centering 
        \includegraphics[width=\textwidth]{figures/down_sample.png}
        \caption{Raw and filtered data before \\ interpolation}
    \end{subfigure}
    \begin{subfigure}[b]{0.329\textwidth}
        \centering
        \includegraphics[width=\textwidth]{figures/up_sample.png}
        \caption{Raw and filtered data after \\ interpolation}
    \end{subfigure}
        \begin{subfigure}[b]{0.328\textwidth}
        \centering
        \includegraphics[width=\textwidth]{figures/fft_compare.pdf}
        \caption{Standard FFT and a non-uniform FFT of Data}
    \end{subfigure}
    \caption{Steps to extract breathing rate from the CSI.}
    \label{fig:process-step}
\end{figure*}

\section{Privacy Implication: WiFi Sensing Attack}\label{sec:implications}

Recently, there has been a significant amount of work on WiFi sensing technologies that use WiFi signals to detect events such as motion, gesture, and breathing rate. In this section, we show how an adversary can combine WiFi sensing techniques with the above loopholes to monitor people's breathing rate whenever she/he wants from outside buildings despite the WiFi network and devices being completely secured. In particular, an adversary can force our WiFi devices to stay awake and continuously transmit WiFi signals. Then she/he can continuously analyze our signals and extract information such as our breathing rate and presents. Note, since most of the time, we are close to a WiFi device (such as a smartwatch, laptop, or tablet), our body will change the amplitude and phase of the signals which can be easily extracted by the adversary.


\subsection{Attack Design, Scenarios and Setup}

\subsubsection{Attack Design}
The attacker sends fake packets to a WiFi device in the target property and pushes it to transmit ACK packets. In particular, since an adult’s normal breathing rate is around 12 -20
times per minute (i.e., 0.2- 0.33Hz), receiving several ACK packets per second is sufficient for the attacker to estimate the breathing rate, without impacting the performance of the target WiFi network. The attacker then takes the Fourier transform of the CSI information of ACK packets to estimate the breathing rate of the person who is nearby the WiFi device. However, due to the random delays of the WiFi random access protocol and the operating
system’s scheduling protocol, the collected data samples are not uniformly spaced in time. Hence, the attacker cannot simply use standard FFT to estimate the breathing rate. Instead, they need to use a non-uniform Fourier transform, and a voting algorithm to extract the breathing rate. The  Non-Uniform Fast Fourier Transform (NUFFT) algorithm \ref{alg:nfft}  used is shown below.

\begin{algorithm}
\SetAlgoLined
\SetKwFunction{Union}{Union}\SetKwFunction{Interpolation}{Interpolation}
\KwData{Time indices $t$, data samples $x$ of length $n$}
\KwResult{Magnitude of each frequency component}
 $d \leftarrow \min_i({t_i - t_{i - 1}}) \quad i = 1, 2, ..., n.$\;
 \For{$i \leftarrow 1$ \KwTo $n - 1$}{
    $interval \leftarrow t[i] - t[i - 1]$\;
  \If{$interval > d$}{
    $count \leftarrow \lfloor interval / d \rfloor$\;
    \textbf{Interpolation}($t$, $x$, $t[i]$, $t[i-1]$, $count$)\;
   }
 }
 \Return \textbf{FFT}($t$, $x$)
 \caption{Non-uniform FFT}
 \label{alg:nfft}
\end{algorithm}

The algorithm first finds the minimum time gap between any two adjacent data points $d$, then linearly interpolates any interval that is larger than the gap with $\lfloor interval / d \rfloor$\$ samples. Finally, it uses a regular FFT algorithm to find the magnitude of each frequency component. A low-pass filter is applied before feeding data to the FFT analysis to reduce noise (not shown in the algorithm).



Figure~\ref {fig:process-step}(a) and ~\ref{fig:process-step}(b) show the amplitude of CSI before and after interpolation, respectively, when the attacker sends 10 packets per second to a WiFi device that is close to the victim. Each figure shows both the original data (in blue) and the filtered data (in orange). Figure~\ref{fig:process-step}(c) shows the frequency spectrum of the same signals when a standard FFT or our non-uniform FFT is applied. A prominent peak at 0.3Hz is shown in the non-uniform FFT spectrum, indicating a breathing rate of 18 bpm.

WiFi CSI gives us the amplitude of 52 subcarriers per packet. We observed that these subcarriers are not equally sensitive to the motion of the chest. Besides, a subcarrier's sensitivity may vary depending on the surrounding environment. For a more reliable attack, the attacker should identify the most sensitive subcarriers over a sampling window. Previously proposed voting mechanisms for coarse-grained motion detection applications \cite{zhu2018tu,  Arshad17,MoSense,WiGest} cannot be directly applied in this situation, as chest motion during respiration is at a much smaller scale. Instead, we developed a soft voting mechanism, where each subcarrier gives a weighted vote to a breathing rate value. The breathing rate that gets the most votes is reported.  Specifically, We first find the power of the highest peak ($P_{peak}$), and then calculate the average power of the rest bins ($P_{ave}$). The exponent of the Peak-to-Average Ratio (PAR): $e^{\frac{f_{peak}}{f_{ave}}}$ is used as the weight of the corresponding subcarrier. In this way, we guarantee the subcarriers with higher SNR have significantly more votes than the rest of the subcarriers. 


\subsubsection{Attack Scenarios}
We evaluate the WiFi sensing attack in different scenarios, both indoor and outdoor. In the indoor scenario, the attacker and the target are placed in the same building but on different floors. The height of one floor in the building is around $2.8$ m.  This scenario is similar to when the attacker and the target person are in different units of an apartment or townhouse.
In the outdoor scenario, the attacker is outside the target's house. For the outdoor experiments, We place the attacker in another building which is around 20 m away from the target building. In all of the experiments, the target WiFi devices are placed $0.5$ to $1.4$ m away from the person's body. The person is either watching a movie, typing on a laptop, or surfing the web using his cell phone. During the experiments, other people are walking and living normally in the house. Finally, we run the attack and compare the estimated breathing rate with the ground truth.
To obtain the ground truth, we record the target person's breathing sound by attaching a microphone near his/her mouth~\cite{dafna2015sleep}. We then calculate the FFT on the sound signal to measure the breathing frequency. Note that the attack does not need this information and this is just to obtain the ground truth in our experiments.

\subsubsection{Attacker Setup}
\noindent\textbf{Hardware Setup:}
The attacker uses a Linksys AE6000 WiFi card and an ESP32 WiFi module~\cite{esp32} as the attacking device. Both devices are connected to a ThinkPad laptop via USB. The Linksys AE6000 is used to send fake packets and the ESP32 WiFi module is used to receive acknowledgments (ACK) and extract CSI. Although we use two different devices for sending and receiving, one can simply use an ESP32 WiFi module for both purposes. The use of two separate modules gave us more flexibility in running many experiments. As for the target device, we use a One Plus 8T smartphone without any software or hardware modifications. We have also tested our attack on an unmodified Lenovo laptop, a Microsoft Surface Pro 4 laptop, and a USB WiFi card as the target device and we obtained similar results. It is worth mentioning that any WiFi device can be a target without any software or hardware modification. 

\vspace{0.05in}
\noindent\textbf{Software Setup:}
We have implemented the CSI collecting script on the ESP32 WiFi module, and the breathing rate estimation algorithm on the laptop. The collected CSI data is fed to the algorithm which produces the breathing rate estimation values in real-time.  To process this data in real time, a sliding window (buffer) is used. The size of the window is $30$ s and the stride step is $1$ s. 
$30$ seconds is a large enough window for estimating a stable breathing rate value. Note that an adult breathes around 6 times during such a window. The window is a queue of data points, and it updates every second by including $1$ second of new data points to its head and removing $1$ second of old data points from its tail. The breathing rate estimation runs the analysis algorithm on the data points inside the window whenever it is updated. The window slides once per second. Hence, our software reports an estimation of breathing rate every second. Note that there is a $30$-second delay at the beginning since the window needs to be filled first.


\begin{figure*}[t]
    \centering
    
    \setkeys{Gin}{width=\linewidth}
    \begin{tabularx}{\linewidth}{XXX}
    
    \includegraphics[width=0.32\textwidth]{figures/breathing_rate-vs-accuracy.png}
    \caption{The average accuracy of the attack in estimating the target person's breathing rate when he attacker and target device are in the same building.}
    \label{fig:breath_accuracy_bar}
    &
    \includegraphics[width=0.32\textwidth]{figures/legacy-cdf.png}
    \caption{The CDF of the error in estimating the target person's breathing rate when he attacker and target device are in the same building (different floor).}
    \label{fig:breath_accuracy_cdf1}
    &
    \includegraphics[width=0.32\textwidth]{figures/cross-building-cdf.png}
    \caption{The CDF of the error in estimating the target person's breathing rate when he attacker and target device are in different buildings (20m away)}
    \label{fig:breath_accuracy_cdf2}
 \end{tabularx}    
\end{figure*}

\subsection{Results}
We evaluate the effectiveness of the attack in different scenarios such as when the attacker and the target are in the same building or different buildings.
\subsubsection{Accuracy in Detecting Breathing Rate}
\noindent\textbf{Same Building Scenario:} First, we evaluate the accuracy of the attack by estimating the breathing rate in an indoor scenario where the target device and attacker are in the same building. We evaluate the accuracy when the target's breathing rate is 12, 15, 20, and 30 breaths per minute.  Note, that the normal breathing rate for an adult is 12-20 breaths per minute while resting, and higher when exercising. In this experiment, the user is watching a video. To make sure the target person's breathing rate is close to our desired numbers, we place a timer in front of the person, where they can adjust their breathing rate accordingly. 
This is just to better control the breathing rate during the experiment and is not a requirement nor an assumption in this attack.
We run each experiment for two minutes. During this time, we collect the estimated breathing rate from both ground truth and the attack for different locations of the target device. Figure \ref{fig:breath_accuracy_bar} shows the average accuracy in estimating breathing rate across all experiments. The accuracy is calculated as the ratio of the estimated breathing rate reported by the attack over the ground truth breathing rate. The figure shows that the accuracy of estimating the breathing rate is over $99$\% in all scenarios.  Finally, Figure \ref{fig:breath_accuracy_cdf1} plots the Cumulative Distribution Function (CDF) of the error in  detecting breathing rate for over $2400$ measurements. The figure shows that $78\%$ of the estimated results have no error. The figure also shows that $99\%$ of measurements have less than one breath per minute error which is negligible. 


\vspace{0.05in}\noindent\textbf{Different Building Scenario:}
So far, we have evaluated our attack where the target and the attacker are in different rooms or floors of the same building. Here we push this further and examine whether our attack works if the attacker and the target person are in a different building.  We place the target device in a building on a university campus on a weekday with people around. A person is sitting around 0.5 m away from the device.  We then place the attacker in another building which is around 20 m away from the target building.  Similar to the previous experiment, we run the attack and compare the estimated breathing rate with the ground truth. Figure \ref{fig:breath_accuracy_cdf2} shows the CDF of error for 180 measurements in this experiment. Our results show that the attacker successfully estimates the breathing rate. Note, that the reason that the attack works even in such a challenging scenario with other people being around is two-fold. First, using an FFT helps to filter out the effect of most non-periodic movements and focuses on periodic movements and patterns. Second, wireless channels are more sensitive to changes as we get closer to the transmitter \cite{abedi2020witag,dehbashi2021verification}, and since in these scenarios, the target person is very close to the target device, their breathing motion has a higher impact on the CSI signal compared to the other mobility in the environment. 

\subsubsection{Human Presence Detection}
We next evaluate the efficacy of detecting whether there is a target person near the WiFi device or not. In this experiment, the target phone is placed on a desk and the person stays around the device for $30$ seconds, then walks away from the device, and then comes back near the device. Note, in our algorithm, when there is no majority vote during the voting phase, we return $-1$ to indicate no breathing detected. Figure~\ref{fig:absence} shows the results of this experiment. As illustrated in the figure, we can correctly detect the breathing rate when a person is near the device. In other words, the algorithm can detect if there is no one near the target device and refrain from reporting a random value.

\begin{figure}[!t]
    \centering
    \includegraphics[width=0.45\textwidth]{figures/annotated-absence.png}
    \caption{The efficacy of estimating the breathing rate when there is no target near the WiFi device.}
    \label{fig:absence}
\end{figure}

\subsubsection{Effect of Distance and Orientation}
Next, we evaluate the effectiveness of the attack for different orientations of the device with respect to the person. We also evaluate its performance for different distances between the target device and the target person.

\vspace{0.05in}
\noindent\textbf{Orientation:}
We evaluate the effect of orientation of the target person with respect to the target device (laptop). We run the same attack as before for different orientations (i.e. sitting in front, back, left, and right side of a laptop). The user is 0.5m away from the target device in all cases. Figure~\ref{fig:orientation_vs_accuracy} shows the result of this experiment. Each bar shows the average accuracy for 90 measurements. Our result shows that regardless of the orientation of the person with respect to the device, the attack is effective and detects the breathing rate of the person accurately. In particular, even when the person was behind the target device, the attack still detects the breathing rate with 99\% accuracy.

\vspace{0.05in}
\noindent
\textbf{Distance:}
Here,  we are interested to find out what the maximum distance between the target device and the person can be while the attacker still detects the person's breathing rate. To do so, we place the attacker device and the target device 5 meters apart in two different rooms with a wall in between. We then run different experiments in which the target person stays at different distances from the target device. In each experiment, we measure the breathing rate for two minutes and calculate the average breathing rate over this time. Finally, we compare the estimated breathing rate to the ground truth and calculate the accuracy as mentioned before. 



\begin{figure}[t]
    \centering
    \begin{subfigure}[b]{0.32\textwidth}
        \centering 
        \includegraphics[width=\textwidth]{figures/orientation-vs-accuracy.png}
         \caption{various orientations}
        \label{fig:orientation_vs_accuracy}
    \end{subfigure}
    \hfill
    \begin{subfigure}[b]{0.32\textwidth}
        \centering
        \includegraphics[width=\textwidth]{figures/legacy-dist.png}
        \caption{different distances.}
        \label{fig:dist_vs_accuracy}
    \end{subfigure}
    \caption{effectiveness of the attack for different orientation and distance of the targeted WiFi device respect to the person.}
\end{figure}

Figure~\ref{fig:dist_vs_accuracy} shows the results of this experiment.  The accuracy is over $99\%$ when the distance between the target device and the target person is less than $60$ cm. Note, in reality, people have their laptops or cellphone very close to themselves most of the time, and $60$ cm is representative of these situations. The accuracy drops as we increase the distance. However, even when the device is at 1.4 m from the person's body, the attack can still estimate the breathing rate with $80\%$ accuracy. Note, this is the accuracy in finding the absolute breathing rate and the change in the breathing rate can be detected with much higher accuracy. Finally, the figure shows that the accuracy suddenly drops to zero for a distance beyond 1.4 m. This is due to the fact that at that distance the power of the peak at the output of the FFT goes below the noise floor, and hence, the peak is not detectable.


\subsubsection{Effect of Multiple People}


\begin{figure*}[!t]
    \centering
    \begin{subfigure}[b]{0.19\textwidth}
        \centering 
        \includegraphics[width=\textwidth]{figures/side-by-side.jpg}
        \caption{Scenario 1}
        \label{fig:scene1}
    \end{subfigure}
    \hfill
    \begin{subfigure}[b]{0.34\textwidth}
        \centering
        \includegraphics[width=\textwidth]{figures/face-to-face.jpg}
        \caption{Scenario 2}
        \label{fig:scene2}
    \end{subfigure}
        % \hfill
    \begin{subfigure}[b]{0.4\textwidth}
        \centering
        \includegraphics[width=\textwidth]{figures/scenario-vs-accuracy.png}
        \vspace{-18pt}
        \caption{Breathing Rate Estimation of two persons}
        \label{fig:scenarios_vs_accuracy}
    \end{subfigure}
    \caption{Accuracy under three different scenarios: Scenario 1: two people sit side-by-side in front of the target device; Scenario 2: one person sits in front of the target device, the other one sits behind the target device; Scenario 3: two people sit in front of two target devices, respectively. Attacker attacks one by one.}
    \label{fig:two-scenes}
\end{figure*}
Last, we evaluate if the attack can be used to detect the breathing rate of multiple people simultaneously. We test our attack in three different scenarios. In the first scenario, two people are near the laptop while one is working on the laptop and the other is just sitting next to him, as shown in Figure \ref{fig:scene1}. The attacker targets the laptop and tries to estimate their breathing rate. Note, that the attacker has no prior information about how many people are next to the laptop. In the second scenario, we repeat the same experiment as the first scenario except that the second person is sitting behind the laptop, as shown in Figure \ref{fig:scene2}. In the third scenario, there are two people in the same space but each person is next to a different device. The attacker targets the laptops and tries to estimate their breathing rates. In these experiments, the target device is 0.5-0.7 m away from the person. 

Figure \ref{fig:scenarios_vs_accuracy} shows the results for this evaluation. The blue bars show the result for the first person who is working on the laptop, and the red bars show the results for the second person. Our results show that the attack effectively detects the breathing rate of both people regardless of their orientation. However, the accuracy in detecting the breathing rate for the second person is a bit lower than the first person for the first and second scenarios. This is because the second person's distance to the target device is slightly more and hence the accuracy has decreased. 

