\section{WiFi Stays Awake When It Should Not}
We have also found a loophole that allows an unauthorized device to keep a WiFi device awake all the time. One may think that a WiFi device can be kept awake by just sending fake back-to-back packets to it and forcing it to transmit acknowledgment. However, this approach does not work. Most WiFi radios go to sleep mode to save energy during inactive states such as screen lock, during which the attacker is not able to keep them awake by sending back-to-back packets. Figure~\ref{fig:dis} show the results of an experiment where the attacker is continuously transmitting fake packets to a WiFi device. In this figure, we plot the amplitude of CSI over time for the ACK packets received from the WiFi device. As can be seen, the responses are sparse and discontinued even when the attacker sends back-to-back packets to the WiFi device. This is because the WiFi device goes to sleep mode frequently. However, we have found a loophole in the power saving mechanism of WiFi devices which can be used by an unauthorized device to keep any WiFi device awake all the time.


\begin{figure}[!ht]
    \centering
    \begin{subfigure}[b]{0.24\textwidth}
        \centering 
        \includegraphics[width=\textwidth]{figures/disjoint-data.png}
        \caption{Without fake beacon frames}
        \label{fig:dis}
    \end{subfigure}
    \begin{subfigure}[b]{0.24\textwidth}
        \centering
        \includegraphics[width=\textwidth]{figures/continuous-data.png}
        \caption{With fake beacon frames}
        \label{fig:cont}
    \end{subfigure}
    \caption{The CSI amplitude of ACKs responded by the target device when an attacker sends back-to-back fake packets to it in two scenarios. (a) In this scenario, the attacker is not using fake beacon frames. Therefore, the target device goes to sleep mode frequently and does not respond to fake packets. (b) In this scenario, the attacker infrequently sends fake beacon frames to keep the target device awake all the time.}
    \label{fig:time-comp}
\end{figure}


\subsection{How does WiFi power saving mechanism work?}
Wireless tranceivers are very power-hungry. %In fact, receiving consume as much as transmitting in most radios. 
Therefore, WiFi radios spend most of the time in the sleep mode to save power. When a WiFi radio is in sleep mode, it cannot send or receive WiFi packets. To avoid missing any incoming packets, when a WiFi device wants to enter the sleep mode it notifies the WiFi access point so that the access point buffers any incoming packets for this device. WiFi devices, however, wake up periodically to receive beacon frames to find out if packets are waiting for them. In particular,  WiFi access points broadcast beacon frames periodically which includes a Traffic Indication Map (TIM) field that indicates which devices have buffered packets on the access point. For example, if the association ID of a WiFi device is $x$, then the $(x+1)^{th}$ bit of TIM is assigned to that device. Finally, when a device is notified that has some buffered packets on the access point, it stays awake and replies with a \textit{Null-function} packet with a power management bit set to "0". In this way, the WiFi device informs the access point it is awake and ready to receive packets.

\subsection{How can one manipulate power saving?}
We have found that an unauthorized device can use the power-saving mechanism of WiFi devices to force them to stay awake. In particular, an attacker can pretend to be the access point and broadcasts fake beacon frames indicating that the WiFi device has buffered traffic, forcing them to stay awake. However, this requires the attacker to know the MAC address and the SSID of the network's access point, as well as the association ID and MAC address of the targeted device so that it can set the correct bit in TIM. 
The access point MAC address and SSID can be easily discovered by sniffing the WiFi traffic using software such as Wireshark since the MAC address is never encrypted and all nodes send packets to the access point. 
Note that MAC randomization does not cause any problem for this process because the attacker finds the randomized MAC address that is currently being used.
Next, the attacker pretends to be the access point and broadcasts fake beacon frames with TIM set to "0xFF", indicating all client devices have buffered traffic. Then, it enters the sniffing mode to sniff for the \textit{Null-function} packets. The null-function packets contain the ID and MAC addresses of all WiFi devices. To avoid keeping all WiFi devices awake, we find that one can send a fake beacon frame as a unicast packet, instead of the usual broadcast beacons. This way only the target device receives the packet and we do not interfere with the operation of other devices. Interestingly, our experiments show that target devices do not care if they receive beacons as broadcast or unicast frames.

\begin{figure}[!t]
    \centering
    \includegraphics[width = 0.8\columnwidth]{figures/polite-wifi-beacon.png}
    \caption{WiFi devices stay awake on hearing a forged beacon frame with TIM flags set up.}
    \label{fig:polite-wifi-beacon}
\end{figure}

To better understand this behavior, we run an experiment where we use two WiFi devices to act as a victim and an attacker, respectively. The attacker sends fake WiFi packets to the victim. We monitor the real traffic between the attacker and the victim's device.

\vspace{0.05in}
\noindent
\textbf{Setup:} Similar to the experiment described in Section~\ref{sec:polite-wifi}, we use an RTL8812AU USB dongle to inject fake packets to a smartphone held by a person who is watching YouTube on the phone. The distance between the smartphone and the user is about 60 cm. The attacking device and the victim are in two separate rooms. The attacker also uses an ESP32 WiFi module to record the Channel State Information (CSI) of received ACKs. 

\vspace{0.05in}
\noindent
\textbf{Result:}
We find that although sending fake beacon frames keeps the target device awake, sending them very frequently will cause WiFi devices to recognize the suspicious attacker's behavior and disconnect from it. Therefore, to keep the WiFi device awake, instead of just sending beacon frames back-to-back, the attacker can continuously transmit normal fake packets to a WiFi device and periodically sends fake beacon frames to keep it awake. Figure~\ref{fig:cont} shows the result of an experiment where the attacker is continuously transmitting fake packets to a WiFi device and periodically sends fake beacon frames. As it can be seen, the target device is continuously awake and responding to fake packets with ACKs. %Finally, note that in this experiment, the target device was placed close to a person and therefore a periodic breathing pattern can be seen in the CSI amplitude of the acknowledgment packets responded by the target WiFi device. \textcolor{red}{I re-plot figure \ref{fig:cont} with my data. It does not seem very periodic now.} In Section~\ref{sec:implications}, we discuss this attack in more details.



