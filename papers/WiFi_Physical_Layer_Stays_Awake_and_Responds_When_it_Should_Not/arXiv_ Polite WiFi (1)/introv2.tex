\section{Introduciton}

Today's WiFi networks use advanced authentication and encryption mechanisms (such as WPA3) to protect our privacy and security by stopping unauthorized devices from accessing our devices and data. Despite all these mechanisms, WiFi networks remain vulnerable to attacks mainly due to their physical layer behaviors and requirements defined by WiFi standards. In this paper, we find two loopholes in the IEEE 802.11 standard for the first time and show how they can put our privacy and security at risk. 

\textbf{a) WiFi radios respond when they should not.}  In a WiFi network, when a device sends a packet to another device, the receiving device sends an acknowledgment back to the transmitter. In particular, upon receiving a frame, the receiver calculates the cyclic redundancy check (CRC) of the packet in the physical layer to detect possible errors. If it passes CRC, then the receiver sends an Acknowledgment (ACK) to the transmitter to notify the correct reception of the frame. Surprisingly, we have found that all existing WiFi devices send back ACKs to even fake packets received from unauthorized WiFi devices outside of their network. Why should a WiFi device respond to a fake packet from an unauthorized device?! 

\textbf{b) WiFi radios stay awake when they should not.}
WiFi chipsets are mostly in sleep mode to save power. However, to make sure that they do not miss their incoming packets, they notify their WiFi access point before entering sleep mode so that the access point buffers any incoming packets for them. Then, WiFi devices wake up periodically to receive beacon frames sent by the associated access point. In regular operation, only the access point sends beacon frames to notify the devices that have buffered packets. When a device is notified, it stays awake to receive them. However, these beacon frames are not encrypted. Hence, we find that an unauthorized user can forge those beacon frames to keep a specific device awake for receiving the (non-existent) buffered frames. %has packets waiting for it. 
%This keeps the WiFi radio awake and prevents it from going to sleep mode to save power.

We examine these behaviors and loopholes in detail over different WiFi chipsets from different vendors. Our examination of over 5,000 WiFi devices from 186 vendors shows that these are widespread issues. We then study the root cause of these issues and show that, unfortunately, they cannot be fixed by a simple solution such as updating WiFi chipsets firmware.  Finally, we implement and demonstrate two attacks based on these loopholes. In the first attack, we show that by forcing WiFi devices to stay awake and continuously transmit, an adversary can continuously analyze the signal and extract personal information such as the breathing rate of the WiFi users. In the second attack, we show that by forcing WiFi devices to stay awake and continuously transmit, the adversary can quickly drain the battery, and hence disable WiFi devices such as home and office security sensors. These attacks can be performed from outside buildings despite the WiFi network and devices being completely secured. All the attacker needs is a \$10 microcontroller with integrated WiFi (such as ESP32) and a battery bank. The attacker device can easily be carried in a pocket or hidden somewhere near the target building. 

The main contributions of this work are:
%\footnote{We discussed our project and experiments with our institution’s IRB office and they determined that no IRB review nor IRB approval is required.}:
\begin{itemize}
    \item We find that WiFi devices respond to fake 802.11 frames with ACK, even when they are from unauthorized devices. We also find that WiFi radios can be kept awake by sending them fake beacon frames indicating they have packets waiting for them. 
    \item We study these loopholes and their root causes in detail, and have tested more than 5,000 WiFi access points and client devices from more than 186 vendors.  
    
    \item We implement two attacks based on these loopholes using just a 10-dollar off-the-shelf WiFi module and validate them in real-world settings.
    



    
\end{itemize}




