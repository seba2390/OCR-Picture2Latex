\section{Privacy and Security Implications}

% \subsection{Discovering WiFi Devices in a Network}
% \label{sec:filter}

% To discover the connected devices in a WiFi network, the attacker can take the advantage of the power-saving mechanism implemented in all of the 802.11 a/b/g/n/ac/ax standards. Specifically, the power-saving mechanism allows client devices to temporarily turn off their WiFi radio and go to sleep. When a client device is in sleep mode, the access point buffers all the incoming packets and notifies the device with a beacon frame. In each beacon frame, there is a field called Traffic Indication Map (TIM) which indicates which device has buffered traffic at the access point. When the client device receives a beacon frame, it will reply with a \textit{Null-function} packet with a power management bit set to "0". In this way, the client device informs the access point it is awake and ready to receive packets.

% We note that the attacker can leverage this mechanism to discover all the WiFi devices in a secured network. First, the attacker need to obtain the MAC address and the SSID of the network's access point. This can be easily done by sniffing the WiFi traffic using software such as Wireshark since the MAC address is never encrypted and all nodes send packets to the AP. Then, the attacker pretends to be the access point and broadcasts fake beacon frames with TIM set to "0xFF", indicating all client devices have buffered traffic. Then, it enters the sniffing mode to sniff for the \textit{Null-function} packets. The null-function packets contain the MAC addresses of client devices, which will be used in the subsequent attack process.  Now that the attacker discovered the MAC addresses of all WiFi devices, the next question is how an attacker can force a targeted WiFi device to continuously transmit. 


% \subsection{Forcing a WiFi Device to Continuously Transmit} 
% \label{sec:turn}
% Existing WiFi sensing attack \cite{zhu2018tu, Banerjee14} relies on the normal WiFi traffic leaking out of the victim's property. This makes it hard for the attacker to execute WiFi sensing anytime. Moreover, it renders difficulty in monitoring continuous miniature motion such as breathing. To be able to attack anytime and uncover the breathing rate of a person, the attacker at least needs a few measurements per second. However, WiFi devices only transmit when there is traffic. To make things worse, WiFi client devices periodically enter sleep mode to save power, during which they barely send any packet. The intermittent nature of WiFi transmission makes revealing fine-grained personal information extremely difficult. So, How can an attacker force a WiFi device to continuously transmit?

% One may think that the attacker can do this by hacking the network and get control over the WiFi devices and access point. However, this approach is becoming more and more difficult, with the advancement of WiFi Protected Access (WPA). In contrast, we realize that some inherent properties of WiFi protocol can be leveraged by the attacker to overcome these challenges and obtain continuous WiFi packets for sensing without gaining access to the network. 

% We found that WiFi devices reply to even fake data packets out of their network with acknowledgment (ACK) packets. In particular, we examined over 5,000 WiFi devices from 186 vendors, and found that all existing WiFi devices respond  with ACK to fake packets transmitted to them. Hence, an attacker can exploit this property and generate back-to-back fake packet streams to push the WiFi devices to continuously transmit ACK packets. The signal of these ACK packets can be used to perform WiFi sensing attacks. The only needed information to do so is the MAC address of the target device, which has been obtained with the method stated in \ref{sec:filter}. 

% However, there is still one issue: most WiFi devices go to sleep to save energy during inactive states such as screen lock, during which the attacker is not able to push them to transmit by sending back-to-back packets. Figure~\ref{fig:dis} show the results of an experiment where the attacker is continuously transmitting fake packets to a WiFi device. In this figure, we plot the amplitude of CSI over time for the ACK packets received from the WiFi device. As it can be seen, the responses are sparse and discontinued even when the attacker sends back-to-back packets to the WiFi device.



% \begin{figure}[!ht]
%     \centering
%     \begin{subfigure}[b]{0.4\textwidth}
%         \centering 
%         \includegraphics[width=\textwidth]{figures/disjoint_data.png}
%         \caption{Without fake beacon frames}
%         \label{fig:dis}
%     \end{subfigure}
%     \hfill
%     \begin{subfigure}[b]{0.4\textwidth}
%         \centering
%         \includegraphics[width=\textwidth]{figures/continous_data.png}
%         \caption{With fake beacon frames}
%         \label{fig:cont}
%     \end{subfigure}
%     \caption{The CSI amplitude of ACKs responded by the target device when an attacker sends back-to-back fake packets to it in two scenarios. (a) In this scenario, the attacker is not using fake beacon frames. Therefore, the target device goes to sleep mode frequently and does not respond to fake packets. (b) In this scenario, the attacker infrequently sends fake beacon frames to keep the target device awake all the time.}
%     \label{fig:time-comp}
% \end{figure}

% To solve this problem and keep the target device awake, we found that the attacker can send the target device a forged beacon frame that claims buffered packets are waiting for the target device. Note, this can simply be done by alternating bits in TIM as mentioned in Section \ref{sec:beacon}. In this way, \wisneak \  prevents the target device from going to sleep, as it will always be deceived to believe that buffered packets are waiting for it at the AP. Although sending these fake beacon frames wakes the target device up, sending them very frequently will cause WiFi devices to recognize the suspicious attacker's behavior and disconnect from it. Therefore, instead of just sending beacon frames back-to-back, the attacker is continuously transmitting normal packets to a WiFi device and periodically sends fake beacon frames to keep it awake. 

% Figure~\ref{fig:cont} shows the result of an experiment where the attacker is continuously transmitting fake packets to a WiFi device and periodically sends fake beacon frames. As it can be seen, the target device is continuously awake and responding to fake packets with ACKs. Finally, note that in this experiment, the target device was placed close to a person and therefore a periodic breathing pattern can be seen in the CSI amplitude of the acknowledgment packets responded by the target WiFi device. Therefore, \wisneak\ can easily detect the breathing rate of the person by taking a Fourier Transform of the signal.




% \subsection{Choosing a Target Device} 
% \label{sec:target_selection}
% So far, we have explained how \wisneak\  can control the transmission of a WiFi device and potentially analyze its signal to monitor people. The next question is which WiFi device in a network would be a good candidate for the attack. Although in theory, \wisneak\  can make any WiFi device into a sensor, not all devices are good candidates for measuring activities such as breathing rate. For instance, people do not stay close to devices like TV or fridge for a long time, and thus these WiFi devices are not suitable candidates for \wisneak. Devices such as laptops and mobile phones are better options for measuring subtle movements since people are likely to spend a good amount of time near them. 

% To pick the target device, \wisneak\  first sends fake packets to all devices continuously. As mentioned before, all WiFi devices respond to these fake packets with ACK packets. \wisneak\  then analyze the responses of each WiFi device, looking for breathing patterns. Note, as we will show in the following section, there are no complex computations when analyzing received signals. Therefore, the data from all WiFi devices can be processed on a single processing unit such as a typical laptop. Next, \wisneak\ discovers the devices for which their signal includes breathing  patterns. Then it stops monitoring other devices and continues sending fake packets only to those specific WiFi devices. In the following section, we explain how \wisneak\  extracts breathing rate from the responses of a WiFi device using a non-uniform FFT algorithm.


\subsection{Extracting Breathing Rate}
\label{sec:algo}
\textcolor{blue}{An attacker can use Polite WiFi to extract the breathing rate of a victim. An adult's normal breathing rate is around $12$-$20$ times per minute (i.e., $0.2$-$0.33 Hz$). By Nyquist Sampling Theorem, acquire a few packets per second) is more than enough to detect the breathing rate. However, due to the random delays of WiFi random access protocol and operating system's scheduling protocol, our collected data samples are not uniformly spaced in time, which calls for at least $100$ samples per second to detect the breathing rate, according to our experiment. In fact, this is not practical for an attacker, as injecting such a large amount of packets in a network can be suspicious to the network owner.}

\textcolor{blue}{The attacker can overcome this issue by transmitting a fewer number of fake packets and utilizing a Non-Uniform Fast Fourier Transform (NUFFT) algorithm \ref{alg:nfft} to recover the missing information.}
\begin{algorithm}
\SetAlgoLined
\SetKwFunction{Union}{Union}\SetKwFunction{Interpolation}{Interpolation}
\KwData{Time indices $t$, data samples $x$ of length $n$}
\KwResult{Magnitude of each frequency component}
 $d \leftarrow \min_i({t_i - t_{i - 1}}) \quad i = 1, 2, ..., n.$\;
 \For{$i \leftarrow 1$ \KwTo $n - 1$}{
    $interval \leftarrow t[i] - t[i - 1]$\;
  \If{$interval > d$}{
    $count \leftarrow \lfloor interval / d \rfloor$\;
    \textbf{Interpolation}($t$, $x$, $t[i]$, $t[i-1]$, $count$)\;
   }
 }
 \Return \textbf{FFT}($t$, $x$)
 \caption{Non-uniform FFT}
 \label{alg:nfft}
\end{algorithm}

\textcolor{blue}{The algorithm first finds the minimum time gap between any two adjacent data points $d$, then linearly interpolate any interval that is larger than the gap with $\lfloor interval / d \rfloor$\$ samples. Finally, it uses regular FFT algorithm to find the magnitude of each frequency component. A low-pass filter is applied before feeding data to the FFT analysis to reduce noise (not shown in the algorithm). }

\begin{figure*}[!t]
    \centering
    \begin{subfigure}[b]{0.33\textwidth}
        \centering 
        \includegraphics[width=\textwidth]{figures/down_sample.png}
        \caption{Raw and filtered data before \\ interpolation}
    \end{subfigure}
    \begin{subfigure}[b]{0.33\textwidth}
        \centering
        \includegraphics[width=\textwidth]{figures/up_sample.png}
        \caption{Raw and filtered data after \\ interpolation}
    \end{subfigure}
        \begin{subfigure}[b]{0.33\textwidth}
        \centering
        \includegraphics[width=\textwidth]{figures/fft_compare.pdf}
        \caption{Standard FFT and a non-uniform FFT of Data}
    \end{subfigure}
    \caption{Steps to extract breathing rate from the CSI.}
    \label{fig:process-steps}
\end{figure*}

\textcolor{blue}{Figure~\ref {fig:process-steps}(a) and ~\ref{fig:process-steps}(b) show the amplitude of CSI before and after interpolation, respectively, when the attacker sends 30 packets per second to a WiFi device that is close to the victim. Each figure shows both the original data (in blue) and the filtered data (in orange). Figure~\ref{fig:process-steps}(c) shows the frequency spectrum of the same signals when a standard FFT or our non-uniform FFT is applied. A prominent peak at 0.3Hz is shown in the non-uniform FFT spectrum, indicating a breathing rate of 18 bpm.}

\textcolor{blue}{WiFi CSI gives us the amplitude of 52 subcarriers per packet. We observed that these subcarriers are not equally sensitive to the motion of the chest. Besides, a subcarrier's sensitivity may vary depending on the surrounding environment. For more reliably attack, the attacker should identify the most sensitive subcarriers over a sampling window. Previously proposed voting mechanisms for coarse-grained motioned detection applications \cite{alexa,  Wi-chase,MoSense,WiGest} cannot be directly applied in this situation, as chest motion during the respiration is at much small scale. Instead, we developed a soft voting mechanism, where each subcarrier gives weighted vote to a breathing rate value. The breathing rate that gets most votes is reported. }

\textcolor{blue}{Specifically, We first find the power of the highest peak ($P_{peak}$), and then calculate the average power of rest bins ($P_{ave}$). The exponent of the Peak-to-Average Ratio (PAR): $e^{\frac{f_{peak}}{f_{ave}}}$ is used as the weight of the corresponding subcarrier. In this way, we guarantee the subcarriers with higher SNR have significantly more votes than the rest of the subcarriers. }
% Then for any interval between two consecutive data points such that $t_i - t_{i-1} > d$, interpolated points will be added between them. As shown in Algorithm \ref{alg:nfft}, once the interval that needs interpolations is found, the number of points needed is recorded into $count$. Then the function \textbf{Interpolation}($x, t, start, end, n$)  modifies the lists $x$ and $t$ by adding $n$ interpolated points between $start$ and $end$.  Specifically, linear interpolation is used in this algorithm. Finally, to reduce the noise, the attacker applies a low pass filter before feeding the data to the FFT analysis. 
% So far, we have explained how  \wisneak\  can discover the MAC addresses of WiFi devices in a network and push them to transmit packets. However, if \wisneak\  sends too many packets per second, it will always be occupying the channel. This will result in a significant drop in the throughput of the WiFi network.
% As a result, the network owner might suspect that something is going on.
% In this section, we investigate what is the minimum number of packets required to extract the breathing rate without impacting the network performance. 

% An adult's normal breathing rate is around $12$-$20$ times per minute (i.e., $0.2$-$0.33 Hz$). One can imagine that based on the Nyquist rate, the sampling rate of a few times per second (i.e., sending a few packets per second) is more than enough to detect the breathing rate. However, based on our experiments, we found that we require at least $100$ samples per second to detect breathing rate. This is mainly due to the fact that the collected data is not uniformly spaced in time. In particular, when \wisneak\  tries to send back-to-back packets to the target device, there are random delays due to channel access protocol and the operating system, which cause the sample measurements (i.e., ACKs received from the target) to be non-uniformly spaced in time. Therefore, \wisneak\  requires sending packets to the target device at a much higher rate than the Nyquist Rate (e.g., at least $100$ packets per second).  This is not practical for the attacker. Sending a packet and receiving an ACK takes at least a millisecond. Therefore, when \wisneak\  needs to monitor multiple target devices, sending $100$ packets per second to each device is impossible and significantly impacts the throughput of the network which makes the attack detectable for a normal user. In the following, we explain how an attacker can solve this problem by transmitting a fewer number of fake packets and compensating for the fewer number of samples by utilizing the non-uniform FFT.


% \textbf{\emph{Non-uniform FFT:}}
% Let $t_0, t_1, ..., t_n$ be the timestamp of all the data points. The attacker first finds the minimum time gap between these data points $d$ where
% $$
% d = \min({t_i - t_{i - 1}}) \quad i = 1, 2, ..., n.
% $$
% Then for any interval between two consecutive data points such that $t_i - t_{i-1} > d$, interpolated points will be added between them. As shown in Algorithm \ref{alg:nfft}, once the interval that needs interpolations is found, the number of points needed is recorded into $count$. Then the function \textbf{Interpolation}($x, t, start, end, n$)  modifies the lists $x$ and $t$ by adding $n$ interpolated points between $start$ and $end$.  Specifically, linear interpolation is used in this algorithm. Finally, to reduce the noise, the attacker applies a low pass filter before feeding the data to the FFT analysis. 

% \begin{algorithm}
% \SetAlgoLined
% \SetKwFunction{Union}{Union}\SetKwFunction{Interpolation}{Interpolation}
% \KwData{Two lists $t$, $x$ of length $n$ and a number $d$}
% \KwResult{The non-uniform FFT results}
 
%  \For{$i \leftarrow 1$ \KwTo $n - 1$}{
%     $interval \leftarrow t[i] - t[i - 1]$\;
%   \If{$interval > d$}{
%     $count \leftarrow \lfloor interval / d \rfloor$\;
%     \textbf{Interpolation}($t$, $x$, $t[i]$, $t[i-1]$, $count$)\;
%   }
%  }
%  \Return \textbf{FFT}($t$, $x$)
%  \caption{Non-uniform FFT}
%  \label{alg:nfft}
% \end{algorithm}

% Figure~\ref {fig:process-steps}(a) and ~\ref{fig:process-steps}(b) show the amplitude of CSI before and after interpolation, respectively, when the attacker sends 30 packets per second to a WiFi device that is close to a person. Each figure shows both the original data (in blue) and the filtered data (in orange). These figures show that both interpolation and filtering significantly help in extracting the breathing pattern. To better demonstrate the effectiveness of our non-uniform FFT algorithm, we compare the frequency spectrum of the same signals when a standard FFT or our non-uniform FFT is applied. Figure~\ref{fig:process-steps}(c) shows the results of this comparison. 
% The output of the standard FFT does not have any clear peak, making it impossible to detect the breathing rate. On the other hand, the output of the non-uniform FFT has a clear peak at around $0.3$ Hz, which matches the actual breathing rate of $18$ breaths per minute. 
 
%\begin{figure*}[!t]
%    \centering
%    \includegraphics[width=0.4\textwidth]{figures/fft_compare.pdf}
%    \caption{Fourier transform of CSI amplitude, computed using standard FFT and a non-uniform FFT }
%    \label{fig:fft-comp}
%\end{figure*}

% \textbf{\emph{Voting Algorithm:}}
% Since each WiFi packet gives us the amplitude and phase of 52 subcarriers, \wisneak\  can compute the breathing rate by taking the non-uniform FFT over the time-varying amplitude of each subcarrier. However, we found that some subcarriers are more robust than others in detecting the breathing rate, depending on the environment. Therefore, instead of computing the breathing rate using only one of the subcarriers, \wisneak\ applies the non-uniform FFT to all of the subcarriers and uses a soft voting mechanism to calculate the breathing rate. In particular, each subcarrier gives a weighted vote to a breathing rate value. We then calculate which breathing rate value has the highest number of votes. In the following, we explain the voting mechanism.
%The idea is that each subcarrier votes for its own peak, but they all have different weights for their votes. Good subcarriers should have more weight and bad subcarriers should have less.

% Let's assume $P_i$ is the power of each bin in the FFT spectrum, where $i = 1, 2, ..., n$. We first find the power of the highest peak ($P_{peak} = max(P_i)$), and then calculate the average power of other bins ($P_{ave} = \frac{\sum_{i \neq peak} P_i}{n - 1}$). We then calculate the ratio of these two values ($\frac{P_{peak}}{P_{ave}})$ which defines Peak-to-Average Ratio (PAR) of that subcarrier. Finally, we use the exponent of the PAR $w = e^{\frac{f_{peak}}{f_{ave}}}$ as the weights. This guarantees that subcarriers with higher power and SNR have significantly more votes than the rest of the subcarriers. For example, even if there is only one subcarrier that shows the breathing pattern, it can still have a higher weight than the sum of votes from other subcarriers. Finally, It is worth mentioning that body or hand movements would also result in changes in the CSI, however, because such movements are non-periodic, their interference is weak as we take the FFT and filter out the high frequency noise. In contrast, periodic movements due to breathing are enhanced in the FFT operation, which makes the attack even work in non-static scenarios.

% To summarize, \wisneak\ first sends fake packets to a WiFi device in the target property and pushes it to continuously transmit ACK packets. It then uses the CSI information of ACK packets to estimate the breathing rate of the person who is nearby the WiFi device. However, since the packets arrive in non-uniform intervals, \wisneak\ cannot simply use standard FFT to estimate the breathing rate. Instead, it uses a non-uniform Fourier transform, and a voting algorithm to extract the breathing rate.  


\subsection{Attacking Multiple People}
\label{sec:multi}
So far, we have assumed that the attacker is monitoring only a single person. The next question is what would happen if there are multiple people in the space? In the following, we explain different scenarios.

\noindent\textbf{Multiple people around different devices:}
If there are multiple people but each is next to a different WiFi device, \wisneak can easily separate their signal using their MAC addresses. In particular, \wisneak\ attacks devices in a round-robin fashion and extracts the breathing rate of the people one by one by analyzing the signal of the WiFi device next to them. 


\noindent\textbf{Multiple people around a single device:}
If multiple people are next to a single WiFi device, the CSI signal of the ACKs received from that device includes the combination of breathing patterns of all people around it. However, since the chance is very low that two people have exactly the same breathing rate, after taking the Fourier transform of the signal, there will be multiple peaks in the output where each presents the breathing rate of a person. Figure \ref{fig:two_peak_fft} shows an example of Fourier transform output for an experiment where two people were sitting close to a targeted WiFi device. As the figure shows, the signal has two clear peaks where each is presenting the frequency of breathing rate for a person. 

\begin{figure}[!t]
    \centering
    \includegraphics[width=0.4\textwidth]{figures/two-peaks-fft.png}
    \caption{Fourier transform showing two peaks when there are two people breathing with different rates.}
    \label{fig:two_peak_fft}
\end{figure}



%The advantage of this method is that it does not discard information from any subcarrier compared to the best picking method, since human eyes may miss the best subcarrier and tie breaking sometimes can be hard as well. More importantly, the automation of this process makes the real-time analysis of the technique possible. This method always returns a value even when there is no one near the device. Since none of the subcarriers detect a nice peak, it will just pick randomly from them. To avoid this behaviour, a threshold is added for the algorithm. When there is no breathing pattern detected, no major peak should exists in any of the subcarrier, so a threshold $\epsilon$ is set, and the system will report $-1$ if all the weights are smaller than $\epsilon$, meaning there is no breathing detected.



%Now that we explain how the attacker can turn a WiFi device to a breathing sensor. The next question is that which WiFi devices would be a good candidate for the attack. Although in theory the attack can make any WiFi device into a sensor, not all devices are good candidates for measuring breathing rate. For instance, people do not stay close to devices like TV or fridge for a long time, and thus their WiFi devices are not suitable candidates for the attack. Devices such as laptops and mobile phones are better options for measuring subtle movements since people are likely to spend a good amount of time near them. To pick the target device, the attacker first sends fake packets to all devices continuously. As mentioned before, all WiFi devices respond to these fake packets with acknowledgement packets. The attacker then analysis the the responses of each WiFi device, looking for breathing patterns. Note, as we will show in the following section, there is no complex computations during analysing the signal. Therefore, the data from all WiFi devices can be processed on a single processing unit such as a typical laptop. Once the attacker discovers the signal of which WiFi devices includes breathing rate patterns, it stops monitoring the other devices and continues sending fake packets only to those specific WiFi device. In the following we explain how the attacker extract breathing rate from the responses of a WiFi device.



%\subsection{Keep the target device responsive}
%To accurately measure breathing rate, a continuous data flow is required. When the target person is using the device, for example watching a video, the device continuously sends packets to AP, and the normal traffic has enough packets for analysis. However, when the device is not active, it barely sends any packet and we need to force the devices to send more. A simple intuition is to use \emph{Polite WiFi} \cite{abedi2020wifi} which will force devices to send ACK so that there are enough traffic, but the result is not as expected. As shown in Figure \ref{fig:dis}, an attacking device is sending fake packets to the target device using \emph{Polite WiFi}, and the plot is picked from one of the subcarriers. The CSI amplitude is plotted against the time, and it can be seen that the data points are still sparse and discontinued even when the \emph{Polite WiFi} is used. The reason for this is that most WiFi devices will go to sleep to save energy during the inactive state such as screen lock. They will only wake up to receive the beacon frames occasionally, and during the sleeping we are not able to get any information from the devices even using \emph{Polite WiFi}. Knowing this, we can send the target device a forged beacon frame that claims it has buffered packets by alternating bits in TIM mentioned in Section \ref{sec:beacon} to prevent it from going to sleep. However, sending fake beacon frames too frequent will cause devices to have suspicious behaviours such as disconnecting from WiFi. Therefore, combining \emph{polite WiFi} \cite{abedi2020wifi} with fake beacon frames can achieve continuous data flow. The \emph{Polite WiFi} will be used constantly since the fake packets are just normal traffic, but the fake beacon will only be sent periodically, and the rate will be around 20-30 times lower than the fake packets. As shown in Figure \ref{fig:cont}, the data flow is continuous after using the combined technique and since a target person is near the device, a periodic breathing pattern can be seen from this figure. To ensure this method does not disrupt the normal internet traffic and remain stealthy, the data rate for injecting packets must be properly selected. The details of data rate selection will be mentioned in future Section \ref{sec:process}. 