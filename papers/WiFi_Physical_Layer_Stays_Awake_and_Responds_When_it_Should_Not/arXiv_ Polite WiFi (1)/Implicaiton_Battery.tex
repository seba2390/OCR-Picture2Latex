\section{Security Implication: Battery Drain Attack}
 In this section, we show how an adversary can drain the battery of our WiFi devices by using the above loopholes and forcing our WiFi devices to stay awake and continuously transmit WiFi signals. 


\subsection{Attack Design and Setup}


\subsubsection{Attack Design}
\label{BatteryAttackDesign}
The attacker forces the target device to stay awake and continuously transmit WiFi packets by sending it back-to-back fake frames and some periodic fake beacons. However, to maximize the amount of time the target device spends transmitting, we study a few different types of fake query packets that the attacker can send. 
Note, that the power consumption of transmission is typically higher than that of reception.\footnote{For example, ESP8266~\cite{esp8266} and ESP32~\cite{esp32} WiFi modules draw 50 and 100 mA when receiving while they draw 170 and 240 mA when transmitting.  These low-power WiFi modules are very popular for IoT devices~\cite{abedi2019wi}.} Hence, to maximize the battery drain, we want to send a short query packet and receive a long response.

\begin{table}[h]
    \centering
    \begin{tabular}{|l|l|l|l|}
        \hline
         Query & Query size & Response & Response size \\
         \hline
         Null    &   28 bytes & ACK   &   14 bytes\\
         RTS            &  20 bytes & CTS   &  14 bytes\\
         BAR     &  24 bytes & BA &  32 bytes \\
         \hline
    \end{tabular}
    \caption{Different types of fake queries and their  responses. Note, Null is a data packet without any payload. BAR and BA stand for Block ACK Request, and Block ACK, respectivly.}
    \vspace{-10pt}
    \label{tbl:queries}
\end{table}

Table~\ref{tbl:queries} lists some query packets and their corresponding responses. The best choice for a query packet is Block ACK requests since the target will respond with a Block ACK that is larger than other query responses. Another important factor to consider for maximizing the battery drain is the bitrate.  When the bitrate of the query packet increases, the bitrate of the response will also increase as specified in the IEEE 802.11 standard. Hence, at first glance, it may seem that to maximize the battery drain, the attacker must use the fastest bitrate possible to transmit query packets, forcing the target device to transmit as many responses as possible. However, it turns out that this is not the case. The power consumption depends mostly on the amount of time the target device spends transmitting packets. Hence, when a higher rate is used for the query and response packets, the total time the target spends on transmission does not increase. In fact, the total time spent transmitting decreases mainly due to overheads such as channel sensing and backoffs. For example, if we increase the bitrate by 6 times (i.e., from 1 Mbps to 6 Mbps), the number of packets will increase by only 3.3 times. As a result, to maximize the transmission time of the target device, the attacker should use the lowest rate (i.e., 1 Mbps) for the query packet. 





\subsubsection{Attack Setup}
\vspace{0.05in}
\noindent

\noindent\textbf{Attacking device:}
Any WiFi card capable of packet injection can be used as the attacking device. We use a USB WiFi card connected to a laptop running Ubuntu 20.04. The WiFi card has an RTL8812AU chipset~\cite{rtl8812au} that supports IEEE 802.11 a/b/g/n/ac standards. We have installed the aircrack-ng/rtl8812au driver~\cite{aircrack-ng} for this card which enables robust packet injection. We utilize the Scapy~\cite{scapy} library to inject fake WiFi packets to the target device. 
%Scapy is a Python library that enables convenient packet sniffing and injection functionalities. 
Scapy allows defining customized packets and multiple options for packet injection. 
Since we need to inject many packets in this attack, we use the \textit{sendpfast} function to inject packets at high rates. \textit{sendpfast} relies on \textit{tcpreplay}~\cite{tcpreplay} for high performance packet injection. 
%Figure~\ref{fig:wifi-attack-code} shows our high-performance fake packets creation and injection implementation.

% \begin{figure}
% \begin{python}
% from scapy.all import *

% # Defining beacon packet
% beacon_dot11 = Dot11(type=0x00, subtype=0x8,
% addr1=<TARGET>, addr2=<ROUTER>, addr3=<ROUTER>)
% essid=Dot11Elt(ID='SSID', info=<SSID>, 
% len=len(<SSID>))
% beacon_frame = RadioTap()/beacon_dot11/
% Dot11Beacon(cap='ESS+privacy')/essid/
% hex_bytes('0504000100ff')

% # Defining block ACK request
% BAR_dot11 = Dot11(type=0x001, subtype=0xb8,
% FCfield=0x00, addr1= <TARGET>)
% BAR_frame = RadioTap()/BAR_dot11/
% (hex_bytes(<ROUTER>))/hex_bytes('04002090')

% #Creating packet list for injection
% frames = [beacon_frame]
% for i in range(<num BARs>):
%     frames.append(BAR_frame)

% #Injecting packets
% while True:
%         sendpfast(frames, iface=<INTERFACE>, 
%         pps=1300, loop=100000)
% \end{python}
% \vspace{-10pt}
% \caption{Python code for injecting fake beacons and queries.}
% \vspace{-10pt}
% \label{fig:wifi-attack-code}
% \end{figure}


%\vspace{0.05in}
\noindent
\textbf{Target device:}
Any WiFi-based IoT device can be used as a target.  We choose Amazon Ring Spotlight Cam Battery HD Security Camera~\cite{ringcamera}
for our battery drain experiments. The camera is powered by a custom 6040 mAh lithium-ion battery. 
The battery life of this camera is estimated to be between 6 and 12 months under normal usage~\cite{xyz, bat2}.
We leave the camera settings to their defaults which means most power-consuming options are turned off. This assures that our measurements will be an upper bound on the battery life and hence the attack might drain the battery much faster in the real world.
Authors in \cite{vanhoef2020protecting} pointed out the possibility of a battery draining attack by forging beacon frames. However, they did not provide any evaluations to test this idea. Moreover, we show how sending fake packets in addition to fake beacon frames can significantly increase the power consumption on the victim device.

\subsection{Results}
We evaluate the effectiveness of the battery drain attack in terms of range and using different payload configuration. 

\begin{figure}[t]
    \centering
    \begin{subfigure}[b]{0.35\textwidth}
    \includegraphics[width=1\columnwidth]{figures/battery-draining/packets-bar.pdf}
    \caption{}
    \end{subfigure}
    \begin{subfigure}[b]{0.35\textwidth}
    \includegraphics[width=1\columnwidth]{figures/battery-draining/battery-bar-WH.pdf}
    \caption{}
    \end{subfigure}
    \caption{The figure shows (a) Average number of packets sent to and received from the target device. (b) Energy consumption in Watt Hour measured under different configurations (i.e. packet type / bitrate (Mbps) }
    \label{fig:batteryDrain}
\end{figure}

\begin{figure*}[t]
    \centering
    \includegraphics[width=0.8\textwidth]{figures/battery-draining/wifi-range.pdf}
    \vspace{-10pt}
    \caption{Percentage of attacker's query packets responded by the target device for different attacker's locations.}
    \label{fig:battery-range-map}
\end{figure*}


\begin{table*}[t]
    \centering
    \begin{tabular}{|c|c|c|c|c|}
        \hline
        Battery Type & Voltage (V) & Full Capacity (Wh) & 100\% Drain  (min)  & 25\% Drain (min) \\
        \hline
        CR2032 coin & 3.0  &  0.68  &  14  & 3.5  \\
        AAA    & 1.5  &  1.87  &  39  & 10 \\
        AA     & 1.5  &  4.20  &  90  & 22\\
        \hline
    \end{tabular}
    \caption{The time it takes for the attack to drain different types of batteries}
    \label{tab:my_label}
\end{table*}

\subsubsection{Finding the optimal configuration:}
As discussed in~\ref{BatteryAttackDesign}, sending block ACK requests at the lowest bitrate (i.e., 1 Mbps) should maximize the power consumption of the target device. To verify this, we have conducted a series of experiments with different types of query packets and transmission bitrates. In each experiment, we continuously transmit query packets to the Ring security camera. In all experiments, we start with a fully charged battery and the attacker injects query packets as fast as possible.

Figure~\ref{fig:batteryDrain} (a) shows the maximum number of packets the attacker could transmit to the target device, and the number of responses it receives per second. Figure~\ref{fig:batteryDrain} (b) shows the amount of energy drawn from the battery during one hour of the attack. As expected, sending Block ACK Requests (BAR) drains more energy from the battery since the target device spends more time on transmission than receiving. Moreover, the results verify that although increasing the data rate from 1Mbps to 6Mbps (BAR/1 versus BAR/6) increases the number of responses, it decreases the energy drained. As mentioned before, this is because the total time spent transmitting decreases mainly due to overheads such as channel sensing and backoffs. This result confirms that sending block ACK requests (BAR) with the lowest datarate is the best option to drain the battery of the target device. 







% \begin{figure}[t]
%     \centering
%     \includegraphics[width=0.7\columnwidth]{figures/battery-draining/wifi-range-setup.pdf}
%     \caption{The setup of the WiFi battery draining experiment.}
%     \vspace{-10pt}
%     \label{fig:wifi-range-setup}
% \end{figure}




\subsubsection{Battery drain with optimal configurations}
We use the best setting which is a block ACK request (BAR) query transmitted at 1~Mbps to fully drain the battery of the Ring security camera. We are able to drain a fully charged battery in 36 hours. Considering the fact that the typical battery life of this camera is 6 to 12 months, our attack reduces the battery life by 120 to 240 times! It is worth mentioning that since a typical user charges the battery every 6-12 months, on average the batteries are at 40-60\%, and therefore it would take much less for our attack to kill the battery. Moreover, the RING security camera is using a very large battery, most security sensors are using smaller batteries. Table~\ref{tab:my_label} shows the amount of time it takes to drain different batteries. For example, it takes less than 40 mins to kill a fully charged AAA battery which is a common battery in many sensors.  





\subsubsection{Range of WiFi battery draining attack}
A key factor in the effectiveness of the battery draining attack is how far the attacker can be from the victim's device and still be able to carry on the attack. If the attack can be done from far away, it becomes more threatening. 
To evaluate the range of this attack, we design an experiment in which the attacker transmits packets to the target from different distances and we measure what percentage of the attacker's packets are responded to by the target device.
We use a realistic testbed. The Ring security camera is installed in front of a house, and the attacker is placed in a car, parked at different locations on the street. We test the attack at 10 different locations up to 150 meters away from the target device. Figure~\ref{fig:battery-range-map} 
shows these locations and our setup. Each yellow circle represents each of the locations tested at. The numbers inside the circles show the percentage of the attacker's packets responded to by the camera.
Each number is an average of over 60 one-second measurements.
The closest distance is about 5 meters when we park the car in front of the target house. 
In this location 97\% of the attacker's packets are responded to.
We conducted other experiments within 10 meters of the target (not shown here) and we obtained similar results. Our results show that even within a distance of 100 meters, almost all attacker's packets are responded to by the victim's device. In some locations such as the rightmost circle (at 150 meters away), we could still achieve a reply rate as high as 73\%, confirming our attack works even at that distance. The reason for achieving such a long range is that the attacker transmits at a 1 Mbps bitrate which uses extremely robust modulation and coding rate (i.e. BPSK modulation and a 1/11 coding rate). 


