\section{Introduction}

In a WiFi network, when a device sends a frame to another device,
the receiving device sends an acknowledgement back to the transmitter. This mechanism is deployed to deal with error prone wireless channels and to handle retransmissions in the physical and MAC layer. In particular, upon receiving a frame, the receiver calculates the cyclic redundancy check (CRC) of the frame to detect possible errors. If the frame passes CRC, then the receiver sends an Acknowledgment (ACK) to the transmitter to notify the correct reception of the frame.

%A WiFi device needs to associate with an access point to join the network. Security protocols such as WPA2 are typically used to prevent unauthorized devices to join the network.
%As a result, a device that is not associated with a WiFi network should not be able to communicate with the devices inside that network.Moreover, in infrastructure-mode WiFi networks, communications only happen between the access point and client devices. Therefore, a client device should only communicate with the access point.

%A WiFi device needs to associate with an access point before it can join that network.
Since most networks use security protocols (such as WPA2) to prevent unauthorized devices from joining, one may assume that a WiFi device only acknowledges frames received from the associated access point or other devices in the same network. 
However, surprisingly, we have found that all today's WiFi devices acknowledge \textbf{any} frame they receive as long as the destination address matches their MAC address. 
%In particular, not only they acknowledge frames from devices which they are associate with, they also acknowledge the frames sent by devices from outside of their network. 
The physical layer acknowledges all frames even those without any valid payload, although higher layers eventually discard the fake packet.
Consider a scenario where a client device is connected to an access point, as shown in Figure~\ref{fig:idea}. This is a private network secured by protocols such as WPA2.
We have found that if an attacker sends a fake and unencrypted 802.11 frame to the client device (labeled as victim), the client device sends back an acknowledgment! We call this behavior \name because WiFi devices respond to any stranger with an acknowledgement. 


\begin{figure}[t!]
\centering
        \centering
        \includegraphics[width=\linewidth, page=3]{figures/overview.pdf}
        \caption{WiFi devices send an ACK for any frame they receive without checking if the frame is valid.}
        \label{fig:idea}
\end{figure}  


The  \name behavior creates many threats. 
For example, an attacker can send back-to-back fake frames to a victim device. 
Then, it can analyze the signal of the received ACKs to infer some personal information about the victim. Recent studies have shown that by monitoring the WiFi signal, one can infer some information about the environment such as localization~\cite{localization-1, localization-2}, gesture recognition~\cite{gesture-1, gesture-2}, breathing rate estimation~\cite{breathing-1, breathing-2}, and keystroke inference~\cite{windtalker}. Hence, the attacker can exploit these systems and use the signal of ACKs to infer some personal information. 
Another possible threat is the battery-drain attack.
An attacker can force a victim to continuously transmit ACKs by bombarding the victim with fake packets.
This would drain victim's battery very quickly. Our experiments show that this attack can increase the power consumption of a low-power IoT device by an order of 35x. This can be problematic for many sensitive sensor devices and medical devices. 

In this paper, we study the \name behavior in more detail on over 5000 WiFi devices. 
We also explore if this behavior can be avoided, and why today's WiFi devices have such behavior. 
Moreover, we examine the possibility of a few attacks using \name. Finally, as opposed to creating threats, we also show how it can be used as an opportunity to help WiFi sensing techniques to be more practical.


%Initially, we assumed that making the victim to send an ACK requires a lot of efforts such as creating a frame that looks like normal frames it receives from the access point.However, it turns out that little effort is needed to get an acknowledgment.
%We have tested over 5000 WiFi devices and have confirmed that they respond to our fake 802.11 frames.
%We call this vulnerability \textit{\name} because WiFi devices respond to any stranger with an acknowledgement.


% \begin{figure}[t!]
% \centering
%     \begin{subfigure}{\columnwidth}
%         \centering
%         \includegraphics[width=\linewidth, page=3]{figures/figure.pdf}
%         \vspace{-15pt}
%         \caption{Victim sends an ACK for the fake frame}
%         \label{fig:security-existing}
%     \end{subfigure}
%     \vspace{10pt}
    
%     \begin{subfigure}{\columnwidth}
%         \includegraphics[width=\linewidth, page=2]{figures/wireshark.png}
%         \vspace{-15pt}
%         \caption{Frames exchanged between attacker and victim} 
%         \label{fig:security-polit-wifi}
%     \end{subfigure}
%     \vspace{-10pt}
%     \caption{WiFi devices send an ACK for any frame they receive without checking if the frame is valid.}
%     \vspace{-15pt}
%     \label{fig:idea}
% \end{figure}  


%The \name vulnerability creates many important threats and opportunities. In this paper, we study the consequences of this finding in a few domains such as WiFi sensing and battery-drain attack. WiFi sensing systems use WiFi signals to infer some information about the environment such as occupancy detection~\cite{}, gesture detection\cite{}, breathing rate estimation\cite{}, and keystroke inference. We describe how \name can be used to create a threat against the privacy of users. Specifically, we study the impact of \name on keystroke inference techniques. These techniques utilize changes in WiFi signals transmitted from smartphones and tablets to infer the text being typed on these devices~\cite{windtalker, wikey, no-training-keystroke}. These techniques have some requirements that limit their effectiveness.  For example, they require to lure the victim into connecting to a rogue access point. \name completely removes this requirement because we no longer need to establish  a connection to receive packets from the victim device.  Instead, we can inject fake frames to the target device and measure the signal of the ACK transmitted by this device.

%As opposed to creating threats, we also show how \name can help WiFi sensing techniques to be more practical. WiFi sensing systems typically require two WiFi devices to operate. One device transmits WiFi packets and the other device measures WiFi signals.This requires software modifications on two WiFi devices which can limit these systems in practice. Instead by utilizing \name one device can send fake frames to another device, and measure the WiFi signal of the received ACKs. In this approach, no modification is needed on the second WiFi device. As a result, an access point or an IoT hub can run WiFi sensing techniques without anymodification to the other devices.


%Another consequence we examine in this paper is battery-drain attack. Transmitting WiFi packets requires a lot of energy in the order of hundreds of milliwatts. The \name vulnerability can be used to bombard a battery-operated WiFi device with fake 802.11 frames. This forces the device to continuously transmit acknowledgments.We show that this attack can increase the power consumption of a low-power WiFi module by an order of 35x.

The main contributions of this work are:
\begin{itemize}
    \item We find that WiFi devices respond to fake 802.11 frames by acknowledging the reception of these frames.
    We have tested over 5,000 WiFi access points and client devices and have found that all of them are susceptible to the \name behavior.
     \item We demonstrate examples of how \name can be utilized in conjunction with WiFi sensing techniques to create new threats.
    Specifically, we show how \name makes existing keystroke inference systems significantly more dangerous.
    In addition, we show that battery-operated WiFi devices can be attacked by forcing them to transmit ACKs which significantly increases their power consummation by 35x.
    
    \item We demonstrate how \name  creates new opportunities for WiFi sensing techniques by making them more practical.
    We show that sensing techniques can be implemented by making software modifications on only one WiFi device,
    instead of existing two-device implementations.
    
        
\end{itemize}