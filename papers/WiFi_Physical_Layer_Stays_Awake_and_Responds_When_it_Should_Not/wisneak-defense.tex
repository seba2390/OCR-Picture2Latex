\section{Is It Possible to Stop This Attack?}
%\textcolor{red}{(Haofan: I suppose although this attack is hard to stop, it is possible to detect that the attack is happening, since the attack has to pretend to be the access point and send fake beacon packets, the actual access point and listen to these packets and realize these packets are not send by itself.)}

In the previous section, we showed how effective \wisneak\ is in detecting the breathing rate of a target person. A natural question is whether it is possible to stop an attacker from monitoring a person's breathing rate. 

\wisneak\ relies on the fact that WiFi devices respond to the attacker's fake packets. This flaw exists in all WiFi devices and cannot be stopped. This is because the physical layer only has $10 \mu s$ to respond to a received packet and it cannot check the validity of the received packet (e.g., check the encryption of the packet). 
%Target devices respond to the attacker's fake packets because of the Polite WiFi behavior~\cite{abedi2020wifi}. This flaw cannot be stopped because the physical layer only has $10 \mu s$ to respond to a received packet and it cannot check the validity of the received packet (e.g., check the encryption of the packet). 
Even if someone finds out an attack is happening by monitoring and checking for fake packets, the physical layer still sends back ACKs for received packets.
Therefore, the attack cannot be stopped at the root. Despite this limitation, we present a solution that confuses the \wisneak attack and can potentially prevent it from estimating the breathing rate accurately.

As explained before, \wisneak\ relies on the CSI changes of WiFi signals to estimate the breathing rate. Therefore, one possible solution to stop such an attack is to artificially create similar changes in the CSI. For example, if the target WiFi device periodically changes its transmission power, it might impact the CSI measured by the attacker. As a result, it might be possible to prevent the attacker from estimating the target person's breathing rate accurately.

\begin{figure}[!t]
    \centering
    \includegraphics[width=0.4\textwidth]{figures/periodic_defense.png}
    \caption{Changing the transmit power periodically when there is no person near the target device.}
    \label{fig:periodic_defense}
\end{figure}

To verify the effectiveness of such a technique, we perform an experiment, in which we periodically change the transmission power of the target device between $10$ dBm and $18$ dBm every $1$ second, while \wisneak\ tries to estimate the breathing rate. Figure \ref{fig:periodic_defense} shows the results of this experiment. The periodic pattern, caused by changing the transmission power, can clearly be seen in the CSI amplitude of the WiFi packets measured by the attacker.

\begin{figure*}[t!]
    \centering
    \setkeys{Gin}{width=\linewidth}
    \begin{tabularx}{\linewidth}{XXX}
    \includegraphics[width=0.31\textwidth]{{figures/disrupt.png}}
    \caption{The effectiveness of the proposed defence for different interval and transmit power changes. The gray area shows the region where the defense was effective.}
    \label{fig:disrupt}
    &
    \includegraphics[width=0.31\textwidth]{figures/12_15_20M_cdf.png}
    \caption{The impact of the proposed defence technique on network throughput when the application data rate is set to 20 Mbps.}
    \label{fig:throughput_20_cdf}
    &
    \includegraphics[width=0.31\textwidth]{figures/12_15_cdf.png}
    \caption{The impact of the proposed defence technique on network throughput when there is no limit set for the application data rate (saturating the channel).}
    \label{fig:throughput_cdf}
 \end{tabularx}    
\end{figure*}

In this experiment, there is no one near the target device. However, \wisneak\ reports a breathing rate of 30 breaths per minute which is the frequency of changes in the transmission power. Therefore, the target device could successfully confuse \wisneak to think that there is someone near the device with a breathing rate of 30 BPM. Moreover, instead of just changing the transmission power between two steps, one can change it between multiple steps to create multiple fake peaks at the output of the attacker's Fourier transform. 


Although, these results imply the effectiveness of this technique in  stopping \wisneak\ attack, changing the transmission power by as much as $8$ dB every second can significantly impact the throughput of WiFi devices. This is because changing the transmission power impacts the SNR of the link, as a result, it can cause packet drops. Moreover, it can also confuse the rate adaptation algorithm on the target device to believe that the channel changes very dramatically over time which can further impact the throughput.
However, it might be possible to change the transmission power by a small amount to stop the \wisneak\ attack, while the throughput is not impacted significantly.
Next, we try to find the minimum required change in the transmission power to disrupt \wisneak.

Similar to the previous experiment, we run a set of experiments where the target device is changing its transmission power. However, we try different transmission power changes at different intervals. In all experiments, there is a person next to the target device and the attacker tries to estimate the breathing rate. Figure \ref{fig:disrupt} shows the results of this experiment. In particular, the figure shows whether the attacker was able to successfully detect the target person's breathing rate for a given amount of change in the transmission power and a given period of change. The shaded area in the plot shows the configurations under which the defense work and the attacker is not able to detect the breathing rate. 

These results show that if we lower the interval (i.e. increase the frequency of change), the required change in the transmission power to prevent the attack decreases. However, note that lowering the intervals below 0.5 s (i.e. 60 times per minute) will make the defense ineffective. This is due to the fact that an adult's breathing rate is in the range of 12-20 breaths per minute and a baby's breathing rate is in the range of 40-60 breaths per minute~\cite{healthsite}. Therefore, the attacker can easily filter out any changes which are above 60 times per minute. Hence, the optimal way to prevent the attacker from estimating the breathing rate is to make the target device's transmission power change by $3$ dB every $0.5$ s. 





%\end{tabularx}  
 
% \begin{figure}[!t]
%     \centering
%     \setkeys{Gin}{width=\linewidth}
%     \begin{tabularx}{\linewidth}{XXX}
%         \begin{minipage}[b]{0.32\textwidth}
%             \centering
%             \includegraphics[width=\textwidth]{figures/disrupt.png}
%             \caption{The effectiveness of the proposed defence technique for different interval and transmit power changes. The gray area shows the region where the defense was effective.}
%             \label{fig:disrupt}
%         \end{minipage}
%         &
%         % \begin{minipage}[b]{0.64\textwidth}
%         \centering
%         \begin{subfigure}[b]{0.32\textwidth}
%         \includegraphics[width=1\textwidth]{figures/12_15_20M_cdf.png}
%         \caption{Application data rate is set to 20 Mbps}
%         \label{fig:throughput_20_cdf}
%         \end{subfigure}
%         &
%         \begin{subfigure}[b]{0.32\textwidth}
%         \includegraphics[width=1\textwidth]{figures/12_15_cdf.png}
%         \caption{Saturating the wireless channel}
%         \label{fig:throughput_cdf}
%         \end{subfigure}
%         \vspace{-0.1in}
%         \caption{The CDF of network throughput for two different application data rates. In each scenario, we run the experiment under three different conditions: (1) the transmit power is stable at 12~dB, (2) the transmit power is stable at 15~dB, and (3) the transmit power oscillates between 12~dB and 15~dB every 0.5s}
%         % \end{minipage}
%     \end{tabularx}    
% \end{figure}



% \begin{figure}[!t]
%     \centering
%     \includegraphics[width=0.4\textwidth]{figures/disrupt.png}
%      \vspace{-0.1in}
%     \caption{The effectiveness of the proposed defence technique for different interval and transmit power changes. The gray area shows the region where the defense was effective.}
%     \vspace{-0.1in}
%     \label{fig:disrupt}
% \end{figure}

Now, the next question is whether such a change in the transmission power impacts the throughput of WiFi devices. 
To evaluate this hypothesis, we set up two laptops. One is a server acting as an Access Point (AP), and the other one is a client acting as a target device. The server hosts a WiFi network and the target device connects to it. We use iperf~\cite{iperf} to send UDP packets from the client to the server and to monitor the throughput for $10$ minutes. We set the application data rate on the target device to 20~Mbps and perform this experiment in three different scenarios: (1) the Tx power is stable at 12~dB, (2) the Tx power is stable at 15~dB, and (3) the Tx power oscillates between 12~dB and 15~dB every $0.5$ s.


% \begin{figure}[!t]
%     \centering
%     \begin{subfigure}[b]{0.4\textwidth}
%     \includegraphics[width=1\textwidth]{figures/12_15_20M_cdf.png}
%     \caption{Application data rate is set to 20 Mbps}
%     \label{fig:throughput_20_cdf}
%     \end{subfigure}
    
%     \begin{subfigure}[b]{0.4\textwidth}
%     \includegraphics[width=1\textwidth]{figures/12_15_cdf.png}
%     \caption{Saturating the wireless channel}
%     \label{fig:throughput_cdf}
%     \end{subfigure}
%     \vspace{-0.1in}
%     \caption{The CDF of network throughput for two different application data rates. In each scenario, we run the experiment under three different conditions: (1) the transmit power is stable at 12~dB, (2) the transmit power is stable at 15~dB, and (3) the transmit power oscillates between 12~dB and 15~dB every 0.5s}
% \end{figure}

Figure \ref{fig:throughput_20_cdf} shows the CDF of throughput for all three scenarios. The figure shows that the WiFi device achieves roughly the same throughput in all scenarios. In particular, oscillating the transmission power does not impact the throughput of the device.  Although the 20 Mbps data rate used in this experiment is more than what most WiFi devices use, there are applications that require higher bandwidths. Therefore, next, we evaluate if oscillating the transmission power impact the throughput of WiFi devices when the device transmits at the maximum data rate (i.e., saturating the WiFi channel). 

Figure \ref{fig:throughput_cdf} shows the CDF of throughput when the client WiFi device saturates the channel for all three scenarios. As expected, lowering the transmission power from 15~dBm to 12~dBm reduces the throughput. However, the results show that when the transmission power oscillates between 12 and 15~dBm, the throughput is even lower than the case where the transmission power is kept at 12~dBm. The reason for this is that WiFi uses a rate adaptation algorithm that changes the physical-layer bitrate according to the channel conditions. High bitrates are used when the signal is strong, while low bitrates are used in poor channel conditions. Therefore, when we change the transmission power back and forth, the algorithm tries to adapt constantly. However, these repeated artificial changes cause the rate adaptation algorithm to aggressively reduce the bitrate or to choose a bitrate that is too high which results in excessive packet drop. As a result, the throughput suffers significantly.

In summary, although our idea of changing the transmission power prevents an attacker from monitoring the breathing rate, it potentially impacts the throughput especially when high application layer data rates are used.
Therefore, WiFi devices can change their transmission power when they do not need high bandwidth. However, as we approach the limit of the channel this problem becomes a trade-off between privacy and throughput.

 






