\documentclass[sigconf,9pt]{acmart}
\usepackage[utf8]{inputenc}
\usepackage{hyperref}
\hypersetup{pdfstartview=FitH,pdfpagelayout=SinglePage}
\usepackage{xspace}
\usepackage{xcolor}
\usepackage{subcaption}
\usepackage{comment}
\usepackage{graphicx}

\usepackage{tikz}
\usepackage{amsmath}
\usepackage{caption}
\usepackage{multirow}
\usepackage{boldline}
\usepackage[inline]{enumitem}
\usepackage[ruled,vlined]{algorithm2e}
\usepackage{tabularx}
\usepackage{pythonhighlight}

%\toappear{To appear in the awesome conference XXX}

\renewcommand\footnotetextcopyrightpermission[1]{} % removes footnote with conference info
% Copyright
% 
%\setcopyright{none}
\settopmatter{printacmref=false, printccs=false, printfolios=false}


%\permission{© 20xx IEEE. Personal use of this material is permitted. Permission
% from IEEE must be obtained for all other uses, in any current or future
% media, including reprinting/republishing this material for advertising or
% promotional purposes, creating new collective works, for resale or
% redistribution to servers or lists, or reuse of any copyrighted
% component of this work in other works.}
\begin{document}
\title{WiFi Physical Layer Stays Awake and Responds \\When it Should Not}
% \author{Ali Abedi,
% Haofan Lu,
% Alex Chen, 
% Charlie~Liu,
% and~Omid~Abari
% \thanks{Ali Abedi is with the Department of Electrical Engineering, Stanford University, CA 94305, USA (email: abedi@stanford.edu)}
% \thanks{Alex Chen and Charlie Liu are with the School of Computer Science, University of Waterloo, ON N2L 3G1, Canada (email: zihanchen.ca@gmail.com, charlie.liu@uwaterloo.ca)}
% \thanks{Omid Abari and Haofan Lu are with the Department of Computer Science, University of California at Los Angeles, CA 90095, USA (email: omid@cs.ucla.edu, haofan@cs.ucla.edu)}
%}


\author{Ali Abedi}
\affiliation{
\institution{Stanford University}
\country{USA}}
\email{abedi@stanford.edu}

\author{Haofan Lu}
\affiliation{
\institution{UCLA}
\country{USA}}
\email{haofan@cs.ucla.edu}

\author{Alex Chen}
\affiliation{
\institution{University of Waterloo}
\country{Canada}}
\email{zihanchen.ca@gmail.com}

\author{Charlie Liu}
\affiliation{
\institution{University of Waterloo}
\country{Canada}}
\email{charlie.liu@uwaterloo.ca}

\author{Omid Abari}
\affiliation{
\institution{UCLA}
\country{USA}}
\email{omid@cs.ucla.edu}

\begin{abstract}
WiFi communication should be possible only between devices inside the same network. However, we find that all existing WiFi devices send back acknowledgments (ACK) to even fake packets received from unauthorized WiFi devices outside of their network. Moreover, we find that an unauthorized device can manipulate the power-saving mechanism of WiFi radios and keep them continuously awake by sending specific fake beacon frames to them. Our evaluation of over 5,000 devices from 186 vendors confirms that these are widespread issues. We believe these loopholes cannot be prevented, and hence they create privacy and security concerns.  Finally, to show the importance of these issues and their consequences, we implement and demonstrate two attacks where an adversary performs battery drain and WiFi sensing attacks just using a tiny WiFi module which costs less than ten dollars.  
\end{abstract}
\maketitle
\pagestyle{plain} %removing headers

%HotNets Polite WiFi paper
%========================
\section{Introduciton}

Today's WiFi networks use advanced authentication and encryption mechanisms (such as WPA3) to protect our privacy and security by stopping unauthorized devices from accessing our devices and data. Despite all these mechanisms, WiFi networks remain vulnerable to attacks mainly due to their physical layer behaviors and requirements defined by WiFi standards. In this paper, we find two loopholes in the IEEE 802.11 standard for the first time and show how they can put our privacy and security at risk. 

\textbf{a) WiFi radios respond when they should not.}  In a WiFi network, when a device sends a packet to another device, the receiving device sends an acknowledgment back to the transmitter. In particular, upon receiving a frame, the receiver calculates the cyclic redundancy check (CRC) of the packet in the physical layer to detect possible errors. If it passes CRC, then the receiver sends an Acknowledgment (ACK) to the transmitter to notify the correct reception of the frame. Surprisingly, we have found that all existing WiFi devices send back ACKs to even fake packets received from unauthorized WiFi devices outside of their network. Why should a WiFi device respond to a fake packet from an unauthorized device?! 

\textbf{b) WiFi radios stay awake when they should not.}
WiFi chipsets are mostly in sleep mode to save power. However, to make sure that they do not miss their incoming packets, they notify their WiFi access point before entering sleep mode so that the access point buffers any incoming packets for them. Then, WiFi devices wake up periodically to receive beacon frames sent by the associated access point. In regular operation, only the access point sends beacon frames to notify the devices that have buffered packets. When a device is notified, it stays awake to receive them. However, these beacon frames are not encrypted. Hence, we find that an unauthorized user can forge those beacon frames to keep a specific device awake for receiving the (non-existent) buffered frames. %has packets waiting for it. 
%This keeps the WiFi radio awake and prevents it from going to sleep mode to save power.

We examine these behaviors and loopholes in detail over different WiFi chipsets from different vendors. Our examination of over 5,000 WiFi devices from 186 vendors shows that these are widespread issues. We then study the root cause of these issues and show that, unfortunately, they cannot be fixed by a simple solution such as updating WiFi chipsets firmware.  Finally, we implement and demonstrate two attacks based on these loopholes. In the first attack, we show that by forcing WiFi devices to stay awake and continuously transmit, an adversary can continuously analyze the signal and extract personal information such as the breathing rate of the WiFi users. In the second attack, we show that by forcing WiFi devices to stay awake and continuously transmit, the adversary can quickly drain the battery, and hence disable WiFi devices such as home and office security sensors. These attacks can be performed from outside buildings despite the WiFi network and devices being completely secured. All the attacker needs is a \$10 microcontroller with integrated WiFi (such as ESP32) and a battery bank. The attacker device can easily be carried in a pocket or hidden somewhere near the target building. 

The main contributions of this work are:
%\footnote{We discussed our project and experiments with our institution’s IRB office and they determined that no IRB review nor IRB approval is required.}:
\begin{itemize}
    \item We find that WiFi devices respond to fake 802.11 frames with ACK, even when they are from unauthorized devices. We also find that WiFi radios can be kept awake by sending them fake beacon frames indicating they have packets waiting for them. 
    \item We study these loopholes and their root causes in detail, and have tested more than 5,000 WiFi access points and client devices from more than 186 vendors.  
    
    \item We implement two attacks based on these loopholes using just a 10-dollar off-the-shelf WiFi module and validate them in real-world settings.
    



    
\end{itemize}





\section{Related Work}
The loopholes we present in this paper are explored using packet injection, in which an attacker sends fake WiFi packets to devices in a secured WiFi network.
Packet injection has been used in the past to perform various types of attacks against WiFi networks
such as denial of service attacks for a particular client device or total disruption of the network~\cite{vanhoef2020protecting, dos, rogue-ap, deauth}. These attacks use different approaches such as beacon stuffing to send false information to WiFi devices~\cite{beacon-stuffing-1, beacon-stuffing-2}, or Traffic Indication Map (TIM) forgery to prevent clients from receiving data ~\cite{bellardo2003802, tim-forgery}. However, all of these attacks focus on spoofing 802.11 MAC-layer management frames to interrupt the normal operation of WiFi networks. 
To provide a countermeasure for some of these attacks, the 802.11w standard~\cite{ieee802.11w} 
introduces a protected management frame that prevents attackers from spoofing 802.11 management frames. 
Instead of spoofing 802.11 MAC frames, we exploit properties of the 802.11 physical layer to force a device to stay awake and respond when it should not. 
These loopholes open the door to multiple research avenues including new security and privacy threats. 


%\textcolor{blue}{Our recent preliminary work has shown that all WiFi devices respond with ACKs to packets received from outside of their network~\cite{abedi2020wifi}. However, this workshop paper does not show how to keep WiFi devices awake and avoid going to sleep mode. Moreover, it does not explore turning WiFi devices into sensitive motion sensors and monitoring people's breathing rates. In contrast,  \name\ shows the first attack which forces a WiFi device to be awake and pushes it to continuously transmit. We show how our technique enables an attacker to monitor the breathing rate of people by analyzing their WiFi signals.}

% The related work can be divided into three categories. The first category is WiFi sensing systems that use WiFi signals to infer some information, such as gesture detection, to enable a useful application for the user (the good). The second category is WiFi attacks which show how an attacker can interfere with the normal operation of WiFi networks (the bad). Finally, some recent studies show new privacy attacks against users by analyzing their WiFi signals (the ugly).


%Over the past decade, there has been a significant amount of research on WiFi sensing where WiFi signals are used to detect human activities~\cite{wifi-sensing-survey} to enable useful applications. These systems target different applications such as tracking and localization\cite{adib2013see, activity-recognition-1}, human detection \cite{gong2016adaptive, gong2015wifi}, gesture recognition \cite{abdelnasser2015wigest, gesture-recognition-1, gesture-recognition-2, gesture-recognition-3} and vital measurement such as respiration rate~\cite{abdelnasser2015ubibreathe, adib2015smart, breathing-rate-1, breathing-rate-2}. However, these systems target applications with social benefits and cannot be easily used by an attacker to create privacy and security threats. This is because either these techniques require cooperation from the target WiFi device or the attacker needs to be very close to the target to use these systems.


%WiFi Sensing is a technique that uses ambient WiFi signals to detect events or human activities~\cite{wifi-sensing-survey}. The motivation behind WiFi sensing is that we can obtain certain information without dedicated sensors. In particular, WiFi sensing techniques analyze changes in WiFi signals to infer different types of information. As mentioned earlier in Section~\ref{sec:csi}, CSI has been shown to be well suited for sensing techniques. For instance, there are applications that can identify the number of people in a closed room and their relative locations\cite{adib2013see, activity-recognition-1}. Applications that  develop wireless device-free human detection \cite{gong2016adaptive, gong2015wifi} are also implemented to determine the presence of human activities. Other than this, subtle movements like gesture recognition \cite{abdelnasser2015wigest, gesture-recognition-1, gesture-recognition-2, gesture-recognition-3} and vital measurement such as respiration rate~\cite{abdelnasser2015ubibreathe, adib2015smart, breathing-rate-1, breathing-rate-2}, can also be measured using wireless sensing. These applications are great tools that bring convenience to people's lives.However, for these techniques to work WiFi devices should cooperate to enable WiFi sensing. Therefore, an attacker cannot use these techniques to perform WiFi sensing since he/she has no access to the target building and the devices inside it.

% \subsection{The Bad}
% \label{sec:stealth}


%\emph{Beacon Injection:} 
%This attack is performed by forging 802.11 beacons and broadcasting them to all devices in a WiFi network~\cite{beacon-stuffing-1, beacon-stuffing-2}. The attacking device pretends to be the actual access point and injects false information in the forged beacons to enable a variety of attacks such as the ``evil twin access point'' attack~\cite{evil-twin}. 
%Since beacons are broadcasted to all devices, this will attack all the devices at the same time. The forged beacon frames can also be sent (i.e., unicast) to a particular device to attack individual devices rather than the entire network.

%\emph{TIM Forgery:}
%Traffic Indication Map (TIM), as mentioned in Section \ref{sec:beacon}, is used in 802.11 beacon frames. It contains information about whether sleeping devices have buffered packets at Access Point (AP) or not. It is suggested that an adversary can manipulate the Time Indication Map (TIM) inside beacons to change the behavior of WiFi devices~\cite{bellardo2003802, tim-forgery}. \name\ builds on these attacks. In particular, \name\ forges the TIM to make a device believe that it has buffered data to receive, so it cannot enter the sleep mode.
%For instance, the adversary can forge the TIM to make a device believe that it has buffered data to receive, so it cannot enter the sleep mode. In contrast, it can also prevent a device from receiving any data by telling it that the AP  has no buffered data for it. 


\textbf{WiFi sensing attack:} Over the past decade, there has been a significant amount of research on WiFi sensing where WiFi signals are used to detect human activities~\cite{iot-wifi-localization, rf-sensing, wifi-sensing-survey,adib2015smart, breathing-rate-1, breathing-rate-2,gesture-recognition-1, gesture-recognition-2, gesture-recognition-3,pu2013whole}. However, these systems target applications with social benefits and cannot be easily used by an attacker to create privacy and security threats. This is because either these techniques require cooperation from the target WiFi device or the attacker needs to be very close to the target to use these systems. A recent study shows that by capturing WiFi signals coming out of a private building, it is possible for an adversary to track user movements inside that building~\cite{zhu2018tu}. However, this attack has a bootstrapping stage which requires the attacker to walk around the target building for a long time to find the location of the WiFi devices. Furthermore, since this work relies on only the normal intermittent WiFi activities, it cannot capture continuous data such as breathing rate.  

\textbf{Battery draining attack:} 
Battery draining attacks date back to 1999 \cite{stajano1999resurrecting} and there have been many studies on such attacks and potential defense mechanisms since then~\cite{caviglione2012energy}.
%Battery draining attack was first introduced by Stajano, F. and Anderson, R. in 1999 \cite{stajano1999resurrecting}, where they prophesied that the battery of a mobile device can be exhausted by a malicious user even via legitimate usage. Extensive studies have been conducted to investigate the attack and defense manners. 
%One possible approach is by constantly launching canonical attacks and forcing defense systems to attempt to defeat them. Running defending software consumes CPU resources which leads to quick drainage of battery energy \cite{caviglione2012energy}. 
Battery discharge models and energy vulnerability due to operating systems have been investigated \cite{zhang2010accurate,jindal2013hypnos}. A more recent study plays multimedia files implicitly to increase power consumption during web browsing \cite{fiore2014multimedia, fiore2017exploiting}. In terms of defending, a monitoring agent that searches for abnormal current draw is discussed in \cite{buennemeyer2008mobile}. In contrast, our attack exploits the loopholes in the 802.11 physical layer protocol and the power-hungry WiFi transmission to quickly drain a target device's battery. We will discuss in Section~\ref{sec:cannot-be-fixed} that stopping our proposed attack is nearly impossible on today's WiFi devices.


This paper is an extension of our previous workshop publication ~\cite{polite-wifi}. The workshop paper shows preliminary results for our finding that WiFi devices respond with ACKs to packets received from outside of their network, and provides a brief discussion on potential privacy and security concerns of this behavior without studying them. We have also explored how the WiFi power saving mechanism can be exploited to keep a target device awake in a localization attack~\cite{wi-peep}. 
In this paper, we provide an in-depth study of these previously discovered loopholes. We also design and perform two privacy and security attacks, based on these loopholes. Finally, we implement these attacks on off-the-shelve WiFi devices and present detailed performance evaluations.


% To the best of our knowledge, \name is the first attack that forces WiFi devices to continuously transmit, enabling an attacker, who does not have access to the network, to estimate the breathing rate of a person from outside of the building using a low-cost WiFi module.

%\name\ combines techniques from WiFi sensing, attacks against WiFi networks to enable a new privacy attack to show for the first time that an attacker can estimate the respiration rate of a person from outside of the building.


%These attacks relies on the fact that wireless signals pass through walls; therefore, an attacker who is outside a building can receive signals coming from inside that building. The attacker can analyze the distortions in WiFi signals, caused by the body of a target person, to infer various types of information from a target building.

%To make the problem worse, it has been shown that WiFi devices send acknowledgments (ACK) when they receive fake packets coming from outside the their network~\cite{abedi2020wifi}. This behavior, called \emph{Polite WiFi}, enlightens the possibility of turning any WiFi device into a secret sensor.  This is because the CSI information can be extract from the ACKs send by a victim device to perform WiFi sensing and potentially obtain some sensitive information.
%\name\ combines techniques from WiFi sensing, attacks against WiFi networks to enable a new privacy attack to show for the first time that an attacker can estimate the respiration rate of a person from outside of the building.
%We also propose potential methodologies to protect against this attack.


% Another privacy attack which uses WiFi to extract the International Mobile Subscriber Identity (IMSI) from mobile phones \cite{o2017mobile}. The leakage of IMSI may results in potential tracking of the devices through link to the target user's other hardware addresses such as WiFi MAC addresses. As a result, the mobile device can be turned into a secret tracker which works for the attacker.


% some features of the techniques mentioned above like \emph{Polite WiFi} and TIM forgery, and it can detect the target people's breathing rate without their notice. The leak of breathing rate may seem to be harmless, but attacker can use this information to further infer people's presence in the house.





\section{WiFi Responds When It Should Not}\label{sec:polite-wifi}

Most networks use security protocols to prevent unauthorized devices from communicating with their devices. Therefore, one may assume that a WiFi device only acknowledges frames received from the associated access point or other devices in the same network. However, we have found that all today's WiFi devices acknowledge even the frames they receive from an unauthorized device from outside of their network. In particular, as long as the destination address matches their MAC address, their physical layer acknowledges it, even if the frame has no valid payload. In this section, we examine this behavior in more detail, and explain why this problem happens and why it is not preventable.

\begin{figure}[!t]
    \centering
    \includegraphics[width = 0.8\columnwidth]{figures/polite-wifi.png}
    \caption{WiFi devices send an ACK for any frame they receive without checking if the frame is valid.}
    \label{fig:polite-wifi}
\end{figure}

To better understand this behavior, we run an experiment where we use two WiFi devices to act as a victim and an attacker. The attacker sends fake WiFi packets to the victim. We monitor the real traffic between the attacker and the victim's device.

\vspace{0.05in}
\noindent \textbf{Setup:} For the victim, we use a tablet, and for the attacker, we use a USB WiFi dongle that has a Realtek RTL8812AU 802.11ac chipset. This is a \$12 commodity WiFi device. The attacker uses this device to send fake frames to the victim's device. To do so, we develop a python program that uses the Scapy library~\cite{scapy} to create fake frames. Scapy is a python-based framework that can generate arbitrary frames with custom data in the header fields. Note, that the only valid information in the frame is the destination MAC address (i.e., the victim's MAC address). The transmitter MAC address is set to a fake MAC address (i.e., aa:bb:bb:bb:bb:bb), and the frame has no payload (i.e., null frame) and is not encrypted.

\vspace{0.05in}
\noindent
\textbf{Result:} Figure~\ref{fig:wireshark} shows the real traffic between the attacker and the victim device captured using Wireshark packet sniffer~\cite{wireshark}. As can be seen, when the attacker sends a fake frame to the victim, the victim sends back an ACK to the fake MAC address (aa:bb:bb:bb:bb:bb). This experiment confirms that WiFi devices acknowledge  frames without checking their validity. 
\begin{figure}[t!]
        \centering
        \includegraphics[width=0.8\linewidth, page=2]{figures/wireshark.png}
        \caption{Frames exchanged between attacker and victim} 
        \label{fig:wireshark}
\end{figure}  
Finally, to see if this behavior exists on other WiFi devices, we have repeated this test with a variety of devices (such as laptops, smart thermostats, tablets, smartphones, and access points) with different WiFi chipsets from different vendors, as shown in Table~\ref{tbl:controled-devices}. Note, target devices are connected to a private network and the attacker does not have their secret key. After performing the same experiment as before, we found that all of these devices also respond to fake packets received from a device outside of their network.

\begin{table}[t]
\centering
\begin{tabular}{|l|l|l|}
     \hline
     Device & WiFi module & Standard  \\
     \hline
     MSI GE62 laptop&Intel AC 3160 & 11ac\\
     Ecobee3 thermostat& Atheros & 11n\\
     Surface Pro 2017& Marvel 88W8897 & 11ac\\
     Samsung Galaxy S8 & Murata KM5D18098 & 11ac\\
     Google Wifi AP &  Qualcomm IPQ 4019 & 11ac\\
     \hline
\end{tabular}
\caption{List of tested chipsets/devices}
\label{tbl:controled-devices}
\end{table}

\subsection{How widespread is this loophole?}\label{sec:testing}
In the previous section, we examined a few different WiFi devices and showed that they are all responding to fake frames from unauthorized devices. Here, we examine thousands of devices to see how widespread this behavior is. In the following, we explain the setup and results of this experiment. 

\vspace{0.05in}
\noindent
\textbf{Setup:} To examine thousands of devices, we mounted a WiFi dongle on the roof of a vehicle and drove around the city to test all nearby devices. For the WiFi dongle, we use the same  Realtek  RTL8812AU  USB  WiFi dongle, and connect it to a Microsoft Surface, running Ubuntu 18.04. We develop a multi-threaded program using the Scapy library~\cite{scapy} to discover nearby devices, send fake 802.11 frames to the discovered devices, and verify that target devices respond to our fake frames. Specifically, our implementation contains three threads. The first thread discovers nearby devices by sniffing WiFi traffic and adding the MAC address of unseen devices to a target list. The second thread sends fake 802.11 frames to the list of target devices. Finally, the third thread checks to verify that target devices respond with an ACK. 


% Note that 802.11 ACK does not have a transmitter field in the MAC header and it only has a destination address.
% Therefore, when we receive an ACK, we do not know which target device transmitted this ACK.
% To solve this problem, we use a unique transmitter MAC address in the fake frame transmitted to each target device.
% The target device then sends an ACK to this MAC address and we can find out which device responded.
% One might think that the transmitter of an ACK is known because it must be the destination of the last fake frame.
% However, since independent threads perform sending fake frames and receiving ACKs, there is no way to synchronize them.

\begin{table}[t]
    \centering
    \begin{tabular}{|l|c||l|c|}
        \hline
        \multicolumn{2}{|c||}{WiFi Client Device} & \multicolumn{2}{|c|}{WiFi Access Point}\\
        \hline
        Vendor & \# devices &   Vendor & \# devices  \\
        \hline
        Apple&  143	&	Hitron &  723 \\
        Google&  102	&	Sagemcom &  601 \\
        Intel&  66	&	Technicolor&  410 \\
        Hitron &  65	&	eero &  195 \\
        HP &  63	&	Extreme N. &  188 \\
        Samsung&  56	&	Cisco &  156 \\
        Espressif&  47	&	HP &  104 \\
        Hon Hai&  46	&	TP-LINK &  101 \\
        Amazon &  41	&	Google &  80 \\
        Sagemcom &  38	&	D-Link  &  75 \\
        Liteon &  33	&	NETGEAR &  69 \\
        AzureWave &  30	&	ASUSTek  &  51 \\
        Sonos &  30	&	Aruba &  46 \\
        Nest Labs &  27	&	SmartRG, &  44 \\
        Murata  &  24	&	Ubiquiti N.&  35 \\
        Belkin &  20	&	Zebra &  35 \\
        TP-LINK  &  20	&	Pegatron &  28 \\
        Cisco&  16	&	Belkin  &  25 \\
        ecobee &  13	&	Mitsumi &  25 \\
        Microsoft &  13	&	Apple &  19 \\
        Others & 630	&	Others & 789 \\
        \hline\hline	
        Total & 1523	&	Total & 3805  \\
        \hline
    \end{tabular}
    \caption{List of WiFi devices and APs that respond to our fake 802.11 frames.}
    \label{tbl:uncontroled-devices}
\end{table}

\vspace{0.05in}
\noindent
\textbf{Results}: We perform this experiment for one hour while driving around the city. In total, we discovered 5,328 WiFi nodes from 186 vendors. The list includes 1,523 different WiFi client devices from 147 vendors and 3,805 access points from 94 vendors. Table~\ref{tbl:uncontroled-devices} shows the top 20 vendors for WiFi devices and WiFi access points in terms of the number of devices discovered in our experiment. The list includes devices from major smartphone manufacturers (such as Apple, Google, and Samsung) and major IoT vendors (such as Nest, Google, Amazon, and Ecobee). We found that all 5,328 WiFi Access Points and devices responded to our fake 802.11 frames with an acknowledgment, and hence we infer that most probably all of today's WiFi devices and access points respond to fake frames when they should not.

\subsection{Can this loophole be fixed?}\label{sec:cannot-be-fixed}
So far, we have demonstrated that all existing WiFi devices respond to fake packets received from unauthorized WiFi devices outside of their network. Now, the next question is why this behavior exists, and if it can be prevented in future WiFi chipsets.

In a WiFi device, when the physical layer receives a frame, it checks the correctness of the frame using error-checking mechanisms (such as CRC) and transmits an ACK if the frame has no error. However, checking the validity of the content of a frame is performed by the MAC and higher layers. Unfortunately, this separation of responsibilities and the fact that the physical layer does not coordinate with higher layers about sending ACKs seem to be the root cause of the behavior. In particular, we have observed that when some access points receive fake frames, they start sending \emph{deauthentication frames} to the attacker, requesting it to leave the network. %Note this makes no sense since the attacking device has never been part of the network and is not authenticated. 
These access points detect the attacker as a ``malfunctioning'' device and that is why they send deauthentication frames. Surprisingly, although the access points have detected that they are receiving fake frames from a ``malfunctioning'' device, we found that they still acknowledge the fake frames.

An example traffic that demonstrates this behavior is shown in Figure~\ref{fig:deauth-rts-cts}. As can be seen, although the access point has already sent three deauthentication frames to the attacker, it still acknowledges the attacker's fake frame. We then manually blocked the attacker's fake MAC address on the access point. Surprisingly, we observed that the AP still acknowledges the fake frames. These observations verify that sending ACK frames happens automatically in the physical layer without any communication with higher layers. Therefore, the software running on the access points does not prevent the physical layer from sending ACKs to fake frames.

\begin{figure}[t!]
    \centering
    \includegraphics[width=\columnwidth]{figures/polite-wifi-deaut-frame.png}
    \caption{The attacked access point detects that something strange is happening, however it still ACKs fake frames}
    \label{fig:deauth-rts-cts}
\end{figure}


The next question is why the software running on WiFi devices does not prevent this behavior by verifying if the frame is legitimate before sending an ACK. Unfortunately, this is not possible due to the WiFi standard timing requirements. Specifically, in the IEEE 802.11 standard, upon receiving a frame, an ACK must be transmitted by the end of the Short Interframe Space (SIFS)\footnote{The SIFS is used in the 802.11 standard to give the receiver time to go through different procedures before it is ready to send the ACK.
These procedures include Physical-layer and MAC-layer header processing, creating the waveform for the ACK, and switching the RF circuit from receiving to transmitting mode.
} interval which is 10~$\mu$s and 16 $\mu$s for the 2.4 GHz and 5 GHz bands, respectively.
If the transmitter does not receive an ACK by the end of SIFS, it assumes that the frame has
been lost and retransmits the frame. Therefore, the WiFi device nefeds to verify the validity of the received frame in less than 10 $\mu s$. This verification must be done by decoding the frame using the secret shared key. Unfortunately, decoding a frame in such a short period is not possible. In particular, past work has shown that the time required to decode a frame is between 200 to 700 $\mu s$ when the WPA2 security protocol is used~\cite{decoding-time-1, decoding-time-2, decoding-time-3}. This processing time is orders of magnitude longer than SIFS. Hence, existing devices cannot verify the validity of the frame before sending the ACK, and they acknowledge a frame as long as it passes the error detection check. One potential approach to solve this loophole is to implement the security decoder in WiFi hardware instead of software to significantly speed up its delay. Although this may solve the problem in future WiFi chipsets, it will not fix the problem in billions of WiFi chipsets which are already deployed. 




% \textcolor{blue}{I think the following part needs to be removed}
% One potential solution to fix this behaviour is to design a faster security decoder in the WiFi hardware rather than software. Unfortunately, even with a faster security decoder, \name is still unpreventable since the  attacker  can send fake  Request  to Send (RTS) frames instead of fake  data  frames. Wang et. al~\cite{rts-cts} show that if a Request to Send (RTS) frame
% is sent to an unassociated device, it responds with a Clear To Send (CTS) frame. Therefore, if an attacker sends fake RTS frames, the victim responds with CTS frames. The RTS and CTS frames are typically used between WiFi devices in a network to reserve the wireless channel for a certain amount of time. What makes RTS/CTS interesting is that these frames cannot be encrypted in WiFi networks. This is because all nearby devices must receive them to respect channel reservation.
% The lack of encryption for RTS and CTS frames
% makes the \name behavior unpreventable.\footnote{The IEEE 802.11w standard~\cite{ieee802.11w} 
% supports protected management frames to prevent an unauthenticated station from carrying out an attack by injecting fake management frames. However, control frames are still unprotected. Fundamentally, WiFi cannot encrypt control packets because all devices in the vicinity must understand them.}
% For the rest of this paper, we continue using fake frame and ACK for simplicity, although CTS/RTS can be used interchangeably.










% \begin{table}[t]
%     \centering
%     \begin{tabular}{|l|c|}
%     \hline
%     Vendor & \# devices  \\
%     \hline
%     Hitron Technologies. Inc &  723 \\
%     Sagemcom Broadband SAS &  601 \\
%     Technicolor CH USA Inc. &  410 \\
%     eero inc. &  195 \\
%     Extreme Networks, Inc. &  188 \\
%     Cisco Systems, Inc &  132 \\
%     TP-LINK Technologies Co., Ltd. &  101 \\
%     Google, Inc. &  80 \\
%     Hewlett Packard &  79 \\
%     D-Link International &  75 \\
%     NETGEAR &  69 \\
%     ASUSTek COMPUTER INC. &  51 \\
%     Aruba, a Hewlett Packard Company &  46 \\
%     SmartRG, Inc. &  44 \\
%     Ubiquiti Networks Inc. &  35 \\
%     Zebra Technologies Inc &  35 \\
%     PEGATRON CORPORATION &  28 \\
%     Belkin International Inc. &  25 \\
%     MITSUMI Electric Co., Ltd. &  25 \\
%     Hewlett Packard Enterprise &  25 \\
%     Others & 838 \\
%     \hline\hline
%     Total & 3805  \\
%     \hline
%     \end{tabular}
%     \vspace{5pt}
%     \caption{3805 WiFi access points from 94 vendors are verified to have the \name vulnerability.}
%     \label{tab:polite_wifi_ap}
% \end{table}




%\begin{table}[t]
%     \centering
%     \begin{tabular}{|l|c|}
%     \hline
%     Vendor & \# devices  \\
%     \hline
%     Apple, Inc. &  143 \\
%     Google, Inc. &  102 \\
%     Intel Corporate &  66 \\
%     Hitron Technologies. Inc &  65 \\
%     Hewlett Packard &  63 \\
%     Samsung Electronics Co.,Ltd &  56 \\
%     Espressif Inc. &  47 \\
%     Hon Hai Precision Ind. Co.,Ltd. &  46 \\
%     Amazon Technologies Inc. &  41 \\
%     Sagemcom Broadband SAS &  38 \\
%     Liteon Technology Corporation &  33 \\
%     AzureWave Technology Inc. &  30 \\
%     Sonos, Inc. &  30 \\
%     Nest Labs Inc. &  27 \\
%     Murata Manufacturing Co., Ltd. &  24 \\
%     Belkin International Inc. &  20 \\
%     TP-LINK Technologies Co., Ltd. &  20 \\
%     Cisco Systems, Inc &  16 \\
%     ecobee inc &  13 \\
%     Microsoft Corporation &  13 \\
%     Others & 630 \\
%     \hline\hline
%     Total & 1523 \\
%     \hline
%     \end{tabular}
%     \vspace{5pt}
%     \caption{1523 WiFi devices from 147 vendors are verified to have the \name vulnerability.}
%     \label{tab:polite_wifi_client}
% \end{table}




%The existing 802.11 chipsets already use most of SIFS time.
%For example, the Broadcom BCM4339 802.11ac chip~\cite{BCM4339} requires 5 $\mu s$ just for the Tx to Rx switching.




% To answer this question, we need to understand the underlying mechanism for sending an acknowledgment.
% In 802.11 networks, two types of frames require an immediate response from the receiver.
% A data frame must be followed by an ACK if the frame is received correctly, and a Request To Send (RTS) frame
% must be followed by a Clear to Send (CTS) frame if the channel is clear for transmission.
% Ideally there should be no gap between these frame to utilize the channel as much as possible.
% However, it is not possible to immediately send an ACK or CTS after receiving a data frame or RTS.
% This is because the receive needs some time to go through different procedures before it is ready to send transmit the response.




% SIFS = RTT (based on PHY Transmission rate) + FRAME PROCESSING DELAY AT RECEIVER (PHY PROCESSING DELAY + MAC PROCESSING DELAY) + FRAME PROCESSING DELAY (FOR COMPOSING RESPONSE CTS/ACK)+ RF TUNER DELAY (CHANGE FROM RX to TX)

% SIFS=D+M+Rx/Tx

% Where,

% D=(Receiver delay (RF delay) and decoding of physical layer convergence procedure (PLCP) preamble/header)

% M=(MAC processing delay)

% Rx/Tx=(transceiver turnaround time)

% Modern 802.11 ac chip requires 4 micro seconds to swtich from TX to RX
% https://www.cypress.com/file/298796/download

% 5 micro
% %https://www.mouser.com/datasheet/2/100/002-14784_CYW4339_Single_Chip_5G_WiFi_IEEE_802.11a-961626.pdf

% BCM43569 5 micro
% https://media.digikey.com/pdf/Data%20Sheets/Cypress%20PDFs/BCM43569_RevI_Jul1,2016.pdf

% ====
% 0.2 ms delay for wpa2?
% %https://ieeexplore.ieee.org/stamp/stamp.jsp?arnumber=5763598&casa_token=xZlWJCfXARgAAAAA:q9oTITvt7JVJrBEixWnhj0furAXDFFTxD9DtF4vzxt7-4i0e2Q7FLPbBn2zsPLwozqF33OgH&tag=1
% same?
% %https://ieeexplore.ieee.org/stamp/stamp.jsp?tp=&arnumber=5763598


% also around 0.2 RTT
% %https://ieeexplore.ieee.org/stamp/stamp.jsp?tp=&arnumber=7993928

% 0.7
% %https://dl.acm.org/doi/pdf/10.1145/3102304.3102335


\section{WiFi Stays Awake When It Should Not}
We have also found a loophole that allows an unauthorized device to keep a WiFi device awake all the time. One may think that a WiFi device can be kept awake by just sending fake back-to-back packets to it and forcing it to transmit acknowledgment. However, this approach does not work. Most WiFi radios go to sleep mode to save energy during inactive states such as screen lock, during which the attacker is not able to keep them awake by sending back-to-back packets. Figure~\ref{fig:dis} show the results of an experiment where the attacker is continuously transmitting fake packets to a WiFi device. In this figure, we plot the amplitude of CSI over time for the ACK packets received from the WiFi device. As can be seen, the responses are sparse and discontinued even when the attacker sends back-to-back packets to the WiFi device. This is because the WiFi device goes to sleep mode frequently. However, we have found a loophole in the power saving mechanism of WiFi devices which can be used by an unauthorized device to keep any WiFi device awake all the time.


\begin{figure}[!ht]
    \centering
    \begin{subfigure}[b]{0.24\textwidth}
        \centering 
        \includegraphics[width=\textwidth]{figures/disjoint-data.png}
        \caption{Without fake beacon frames}
        \label{fig:dis}
    \end{subfigure}
    \begin{subfigure}[b]{0.24\textwidth}
        \centering
        \includegraphics[width=\textwidth]{figures/continuous-data.png}
        \caption{With fake beacon frames}
        \label{fig:cont}
    \end{subfigure}
    \caption{The CSI amplitude of ACKs responded by the target device when an attacker sends back-to-back fake packets to it in two scenarios. (a) In this scenario, the attacker is not using fake beacon frames. Therefore, the target device goes to sleep mode frequently and does not respond to fake packets. (b) In this scenario, the attacker infrequently sends fake beacon frames to keep the target device awake all the time.}
    \label{fig:time-comp}
\end{figure}


\subsection{How does WiFi power saving mechanism work?}
Wireless tranceivers are very power-hungry. %In fact, receiving consume as much as transmitting in most radios. 
Therefore, WiFi radios spend most of the time in the sleep mode to save power. When a WiFi radio is in sleep mode, it cannot send or receive WiFi packets. To avoid missing any incoming packets, when a WiFi device wants to enter the sleep mode it notifies the WiFi access point so that the access point buffers any incoming packets for this device. WiFi devices, however, wake up periodically to receive beacon frames to find out if packets are waiting for them. In particular,  WiFi access points broadcast beacon frames periodically which includes a Traffic Indication Map (TIM) field that indicates which devices have buffered packets on the access point. For example, if the association ID of a WiFi device is $x$, then the $(x+1)^{th}$ bit of TIM is assigned to that device. Finally, when a device is notified that has some buffered packets on the access point, it stays awake and replies with a \textit{Null-function} packet with a power management bit set to "0". In this way, the WiFi device informs the access point it is awake and ready to receive packets.

\subsection{How can one manipulate power saving?}
We have found that an unauthorized device can use the power-saving mechanism of WiFi devices to force them to stay awake. In particular, an attacker can pretend to be the access point and broadcasts fake beacon frames indicating that the WiFi device has buffered traffic, forcing them to stay awake. However, this requires the attacker to know the MAC address and the SSID of the network's access point, as well as the association ID and MAC address of the targeted device so that it can set the correct bit in TIM. 
The access point MAC address and SSID can be easily discovered by sniffing the WiFi traffic using software such as Wireshark since the MAC address is never encrypted and all nodes send packets to the access point. 
Note that MAC randomization does not cause any problem for this process because the attacker finds the randomized MAC address that is currently being used.
Next, the attacker pretends to be the access point and broadcasts fake beacon frames with TIM set to "0xFF", indicating all client devices have buffered traffic. Then, it enters the sniffing mode to sniff for the \textit{Null-function} packets. The null-function packets contain the ID and MAC addresses of all WiFi devices. To avoid keeping all WiFi devices awake, we find that one can send a fake beacon frame as a unicast packet, instead of the usual broadcast beacons. This way only the target device receives the packet and we do not interfere with the operation of other devices. Interestingly, our experiments show that target devices do not care if they receive beacons as broadcast or unicast frames.

\begin{figure}[!t]
    \centering
    \includegraphics[width = 0.8\columnwidth]{figures/polite-wifi-beacon.png}
    \caption{WiFi devices stay awake on hearing a forged beacon frame with TIM flags set up.}
    \label{fig:polite-wifi-beacon}
\end{figure}

To better understand this behavior, we run an experiment where we use two WiFi devices to act as a victim and an attacker, respectively. The attacker sends fake WiFi packets to the victim. We monitor the real traffic between the attacker and the victim's device.

\vspace{0.05in}
\noindent
\textbf{Setup:} Similar to the experiment described in Section~\ref{sec:polite-wifi}, we use an RTL8812AU USB dongle to inject fake packets to a smartphone held by a person who is watching YouTube on the phone. The distance between the smartphone and the user is about 60 cm. The attacking device and the victim are in two separate rooms. The attacker also uses an ESP32 WiFi module to record the Channel State Information (CSI) of received ACKs. 

\vspace{0.05in}
\noindent
\textbf{Result:}
We find that although sending fake beacon frames keeps the target device awake, sending them very frequently will cause WiFi devices to recognize the suspicious attacker's behavior and disconnect from it. Therefore, to keep the WiFi device awake, instead of just sending beacon frames back-to-back, the attacker can continuously transmit normal fake packets to a WiFi device and periodically sends fake beacon frames to keep it awake. Figure~\ref{fig:cont} shows the result of an experiment where the attacker is continuously transmitting fake packets to a WiFi device and periodically sends fake beacon frames. As it can be seen, the target device is continuously awake and responding to fake packets with ACKs. %Finally, note that in this experiment, the target device was placed close to a person and therefore a periodic breathing pattern can be seen in the CSI amplitude of the acknowledgment packets responded by the target WiFi device. \textcolor{red}{I re-plot figure \ref{fig:cont} with my data. It does not seem very periodic now.} In Section~\ref{sec:implications}, we discuss this attack in more details.






\begin{figure*}[th!]
    \centering
    \begin{subfigure}[b]{0.329\textwidth}
        \centering 
        \includegraphics[width=\textwidth]{figures/down_sample.png}
        \caption{Raw and filtered data before \\ interpolation}
    \end{subfigure}
    \begin{subfigure}[b]{0.329\textwidth}
        \centering
        \includegraphics[width=\textwidth]{figures/up_sample.png}
        \caption{Raw and filtered data after \\ interpolation}
    \end{subfigure}
        \begin{subfigure}[b]{0.328\textwidth}
        \centering
        \includegraphics[width=\textwidth]{figures/fft_compare.pdf}
        \caption{Standard FFT and a non-uniform FFT of Data}
    \end{subfigure}
    \caption{Steps to extract breathing rate from the CSI.}
    \label{fig:process-step}
\end{figure*}

\section{Privacy Implication: WiFi Sensing Attack}\label{sec:implications}

Recently, there has been a significant amount of work on WiFi sensing technologies that use WiFi signals to detect events such as motion, gesture, and breathing rate. In this section, we show how an adversary can combine WiFi sensing techniques with the above loopholes to monitor people's breathing rate whenever she/he wants from outside buildings despite the WiFi network and devices being completely secured. In particular, an adversary can force our WiFi devices to stay awake and continuously transmit WiFi signals. Then she/he can continuously analyze our signals and extract information such as our breathing rate and presents. Note, since most of the time, we are close to a WiFi device (such as a smartwatch, laptop, or tablet), our body will change the amplitude and phase of the signals which can be easily extracted by the adversary.


\subsection{Attack Design, Scenarios and Setup}

\subsubsection{Attack Design}
The attacker sends fake packets to a WiFi device in the target property and pushes it to transmit ACK packets. In particular, since an adult’s normal breathing rate is around 12 -20
times per minute (i.e., 0.2- 0.33Hz), receiving several ACK packets per second is sufficient for the attacker to estimate the breathing rate, without impacting the performance of the target WiFi network. The attacker then takes the Fourier transform of the CSI information of ACK packets to estimate the breathing rate of the person who is nearby the WiFi device. However, due to the random delays of the WiFi random access protocol and the operating
system’s scheduling protocol, the collected data samples are not uniformly spaced in time. Hence, the attacker cannot simply use standard FFT to estimate the breathing rate. Instead, they need to use a non-uniform Fourier transform, and a voting algorithm to extract the breathing rate. The  Non-Uniform Fast Fourier Transform (NUFFT) algorithm \ref{alg:nfft}  used is shown below.

\begin{algorithm}
\SetAlgoLined
\SetKwFunction{Union}{Union}\SetKwFunction{Interpolation}{Interpolation}
\KwData{Time indices $t$, data samples $x$ of length $n$}
\KwResult{Magnitude of each frequency component}
 $d \leftarrow \min_i({t_i - t_{i - 1}}) \quad i = 1, 2, ..., n.$\;
 \For{$i \leftarrow 1$ \KwTo $n - 1$}{
    $interval \leftarrow t[i] - t[i - 1]$\;
  \If{$interval > d$}{
    $count \leftarrow \lfloor interval / d \rfloor$\;
    \textbf{Interpolation}($t$, $x$, $t[i]$, $t[i-1]$, $count$)\;
   }
 }
 \Return \textbf{FFT}($t$, $x$)
 \caption{Non-uniform FFT}
 \label{alg:nfft}
\end{algorithm}

The algorithm first finds the minimum time gap between any two adjacent data points $d$, then linearly interpolates any interval that is larger than the gap with $\lfloor interval / d \rfloor$\$ samples. Finally, it uses a regular FFT algorithm to find the magnitude of each frequency component. A low-pass filter is applied before feeding data to the FFT analysis to reduce noise (not shown in the algorithm).



Figure~\ref {fig:process-step}(a) and ~\ref{fig:process-step}(b) show the amplitude of CSI before and after interpolation, respectively, when the attacker sends 10 packets per second to a WiFi device that is close to the victim. Each figure shows both the original data (in blue) and the filtered data (in orange). Figure~\ref{fig:process-step}(c) shows the frequency spectrum of the same signals when a standard FFT or our non-uniform FFT is applied. A prominent peak at 0.3Hz is shown in the non-uniform FFT spectrum, indicating a breathing rate of 18 bpm.

WiFi CSI gives us the amplitude of 52 subcarriers per packet. We observed that these subcarriers are not equally sensitive to the motion of the chest. Besides, a subcarrier's sensitivity may vary depending on the surrounding environment. For a more reliable attack, the attacker should identify the most sensitive subcarriers over a sampling window. Previously proposed voting mechanisms for coarse-grained motion detection applications \cite{zhu2018tu,  Arshad17,MoSense,WiGest} cannot be directly applied in this situation, as chest motion during respiration is at a much smaller scale. Instead, we developed a soft voting mechanism, where each subcarrier gives a weighted vote to a breathing rate value. The breathing rate that gets the most votes is reported.  Specifically, We first find the power of the highest peak ($P_{peak}$), and then calculate the average power of the rest bins ($P_{ave}$). The exponent of the Peak-to-Average Ratio (PAR): $e^{\frac{f_{peak}}{f_{ave}}}$ is used as the weight of the corresponding subcarrier. In this way, we guarantee the subcarriers with higher SNR have significantly more votes than the rest of the subcarriers. 


\subsubsection{Attack Scenarios}
We evaluate the WiFi sensing attack in different scenarios, both indoor and outdoor. In the indoor scenario, the attacker and the target are placed in the same building but on different floors. The height of one floor in the building is around $2.8$ m.  This scenario is similar to when the attacker and the target person are in different units of an apartment or townhouse.
In the outdoor scenario, the attacker is outside the target's house. For the outdoor experiments, We place the attacker in another building which is around 20 m away from the target building. In all of the experiments, the target WiFi devices are placed $0.5$ to $1.4$ m away from the person's body. The person is either watching a movie, typing on a laptop, or surfing the web using his cell phone. During the experiments, other people are walking and living normally in the house. Finally, we run the attack and compare the estimated breathing rate with the ground truth.
To obtain the ground truth, we record the target person's breathing sound by attaching a microphone near his/her mouth~\cite{dafna2015sleep}. We then calculate the FFT on the sound signal to measure the breathing frequency. Note that the attack does not need this information and this is just to obtain the ground truth in our experiments.

\subsubsection{Attacker Setup}
\noindent\textbf{Hardware Setup:}
The attacker uses a Linksys AE6000 WiFi card and an ESP32 WiFi module~\cite{esp32} as the attacking device. Both devices are connected to a ThinkPad laptop via USB. The Linksys AE6000 is used to send fake packets and the ESP32 WiFi module is used to receive acknowledgments (ACK) and extract CSI. Although we use two different devices for sending and receiving, one can simply use an ESP32 WiFi module for both purposes. The use of two separate modules gave us more flexibility in running many experiments. As for the target device, we use a One Plus 8T smartphone without any software or hardware modifications. We have also tested our attack on an unmodified Lenovo laptop, a Microsoft Surface Pro 4 laptop, and a USB WiFi card as the target device and we obtained similar results. It is worth mentioning that any WiFi device can be a target without any software or hardware modification. 

\vspace{0.05in}
\noindent\textbf{Software Setup:}
We have implemented the CSI collecting script on the ESP32 WiFi module, and the breathing rate estimation algorithm on the laptop. The collected CSI data is fed to the algorithm which produces the breathing rate estimation values in real-time.  To process this data in real time, a sliding window (buffer) is used. The size of the window is $30$ s and the stride step is $1$ s. 
$30$ seconds is a large enough window for estimating a stable breathing rate value. Note that an adult breathes around 6 times during such a window. The window is a queue of data points, and it updates every second by including $1$ second of new data points to its head and removing $1$ second of old data points from its tail. The breathing rate estimation runs the analysis algorithm on the data points inside the window whenever it is updated. The window slides once per second. Hence, our software reports an estimation of breathing rate every second. Note that there is a $30$-second delay at the beginning since the window needs to be filled first.


\begin{figure*}[t]
    \centering
    
    \setkeys{Gin}{width=\linewidth}
    \begin{tabularx}{\linewidth}{XXX}
    
    \includegraphics[width=0.32\textwidth]{figures/breathing_rate-vs-accuracy.png}
    \caption{The average accuracy of the attack in estimating the target person's breathing rate when he attacker and target device are in the same building.}
    \label{fig:breath_accuracy_bar}
    &
    \includegraphics[width=0.32\textwidth]{figures/legacy-cdf.png}
    \caption{The CDF of the error in estimating the target person's breathing rate when he attacker and target device are in the same building (different floor).}
    \label{fig:breath_accuracy_cdf1}
    &
    \includegraphics[width=0.32\textwidth]{figures/cross-building-cdf.png}
    \caption{The CDF of the error in estimating the target person's breathing rate when he attacker and target device are in different buildings (20m away)}
    \label{fig:breath_accuracy_cdf2}
 \end{tabularx}    
\end{figure*}

\subsection{Results}
We evaluate the effectiveness of the attack in different scenarios such as when the attacker and the target are in the same building or different buildings.
\subsubsection{Accuracy in Detecting Breathing Rate}
\noindent\textbf{Same Building Scenario:} First, we evaluate the accuracy of the attack by estimating the breathing rate in an indoor scenario where the target device and attacker are in the same building. We evaluate the accuracy when the target's breathing rate is 12, 15, 20, and 30 breaths per minute.  Note, that the normal breathing rate for an adult is 12-20 breaths per minute while resting, and higher when exercising. In this experiment, the user is watching a video. To make sure the target person's breathing rate is close to our desired numbers, we place a timer in front of the person, where they can adjust their breathing rate accordingly. 
This is just to better control the breathing rate during the experiment and is not a requirement nor an assumption in this attack.
We run each experiment for two minutes. During this time, we collect the estimated breathing rate from both ground truth and the attack for different locations of the target device. Figure \ref{fig:breath_accuracy_bar} shows the average accuracy in estimating breathing rate across all experiments. The accuracy is calculated as the ratio of the estimated breathing rate reported by the attack over the ground truth breathing rate. The figure shows that the accuracy of estimating the breathing rate is over $99$\% in all scenarios.  Finally, Figure \ref{fig:breath_accuracy_cdf1} plots the Cumulative Distribution Function (CDF) of the error in  detecting breathing rate for over $2400$ measurements. The figure shows that $78\%$ of the estimated results have no error. The figure also shows that $99\%$ of measurements have less than one breath per minute error which is negligible. 


\vspace{0.05in}\noindent\textbf{Different Building Scenario:}
So far, we have evaluated our attack where the target and the attacker are in different rooms or floors of the same building. Here we push this further and examine whether our attack works if the attacker and the target person are in a different building.  We place the target device in a building on a university campus on a weekday with people around. A person is sitting around 0.5 m away from the device.  We then place the attacker in another building which is around 20 m away from the target building.  Similar to the previous experiment, we run the attack and compare the estimated breathing rate with the ground truth. Figure \ref{fig:breath_accuracy_cdf2} shows the CDF of error for 180 measurements in this experiment. Our results show that the attacker successfully estimates the breathing rate. Note, that the reason that the attack works even in such a challenging scenario with other people being around is two-fold. First, using an FFT helps to filter out the effect of most non-periodic movements and focuses on periodic movements and patterns. Second, wireless channels are more sensitive to changes as we get closer to the transmitter \cite{abedi2020witag,dehbashi2021verification}, and since in these scenarios, the target person is very close to the target device, their breathing motion has a higher impact on the CSI signal compared to the other mobility in the environment. 

\subsubsection{Human Presence Detection}
We next evaluate the efficacy of detecting whether there is a target person near the WiFi device or not. In this experiment, the target phone is placed on a desk and the person stays around the device for $30$ seconds, then walks away from the device, and then comes back near the device. Note, in our algorithm, when there is no majority vote during the voting phase, we return $-1$ to indicate no breathing detected. Figure~\ref{fig:absence} shows the results of this experiment. As illustrated in the figure, we can correctly detect the breathing rate when a person is near the device. In other words, the algorithm can detect if there is no one near the target device and refrain from reporting a random value.

\begin{figure}[!t]
    \centering
    \includegraphics[width=0.45\textwidth]{figures/annotated-absence.png}
    \caption{The efficacy of estimating the breathing rate when there is no target near the WiFi device.}
    \label{fig:absence}
\end{figure}

\subsubsection{Effect of Distance and Orientation}
Next, we evaluate the effectiveness of the attack for different orientations of the device with respect to the person. We also evaluate its performance for different distances between the target device and the target person.

\vspace{0.05in}
\noindent\textbf{Orientation:}
We evaluate the effect of orientation of the target person with respect to the target device (laptop). We run the same attack as before for different orientations (i.e. sitting in front, back, left, and right side of a laptop). The user is 0.5m away from the target device in all cases. Figure~\ref{fig:orientation_vs_accuracy} shows the result of this experiment. Each bar shows the average accuracy for 90 measurements. Our result shows that regardless of the orientation of the person with respect to the device, the attack is effective and detects the breathing rate of the person accurately. In particular, even when the person was behind the target device, the attack still detects the breathing rate with 99\% accuracy.

\vspace{0.05in}
\noindent
\textbf{Distance:}
Here,  we are interested to find out what the maximum distance between the target device and the person can be while the attacker still detects the person's breathing rate. To do so, we place the attacker device and the target device 5 meters apart in two different rooms with a wall in between. We then run different experiments in which the target person stays at different distances from the target device. In each experiment, we measure the breathing rate for two minutes and calculate the average breathing rate over this time. Finally, we compare the estimated breathing rate to the ground truth and calculate the accuracy as mentioned before. 



\begin{figure}[t]
    \centering
    \begin{subfigure}[b]{0.32\textwidth}
        \centering 
        \includegraphics[width=\textwidth]{figures/orientation-vs-accuracy.png}
         \caption{various orientations}
        \label{fig:orientation_vs_accuracy}
    \end{subfigure}
    \hfill
    \begin{subfigure}[b]{0.32\textwidth}
        \centering
        \includegraphics[width=\textwidth]{figures/legacy-dist.png}
        \caption{different distances.}
        \label{fig:dist_vs_accuracy}
    \end{subfigure}
    \caption{effectiveness of the attack for different orientation and distance of the targeted WiFi device respect to the person.}
\end{figure}

Figure~\ref{fig:dist_vs_accuracy} shows the results of this experiment.  The accuracy is over $99\%$ when the distance between the target device and the target person is less than $60$ cm. Note, in reality, people have their laptops or cellphone very close to themselves most of the time, and $60$ cm is representative of these situations. The accuracy drops as we increase the distance. However, even when the device is at 1.4 m from the person's body, the attack can still estimate the breathing rate with $80\%$ accuracy. Note, this is the accuracy in finding the absolute breathing rate and the change in the breathing rate can be detected with much higher accuracy. Finally, the figure shows that the accuracy suddenly drops to zero for a distance beyond 1.4 m. This is due to the fact that at that distance the power of the peak at the output of the FFT goes below the noise floor, and hence, the peak is not detectable.


\subsubsection{Effect of Multiple People}


\begin{figure*}[!t]
    \centering
    \begin{subfigure}[b]{0.19\textwidth}
        \centering 
        \includegraphics[width=\textwidth]{figures/side-by-side.jpg}
        \caption{Scenario 1}
        \label{fig:scene1}
    \end{subfigure}
    \hfill
    \begin{subfigure}[b]{0.34\textwidth}
        \centering
        \includegraphics[width=\textwidth]{figures/face-to-face.jpg}
        \caption{Scenario 2}
        \label{fig:scene2}
    \end{subfigure}
        % \hfill
    \begin{subfigure}[b]{0.4\textwidth}
        \centering
        \includegraphics[width=\textwidth]{figures/scenario-vs-accuracy.png}
        \vspace{-18pt}
        \caption{Breathing Rate Estimation of two persons}
        \label{fig:scenarios_vs_accuracy}
    \end{subfigure}
    \caption{Accuracy under three different scenarios: Scenario 1: two people sit side-by-side in front of the target device; Scenario 2: one person sits in front of the target device, the other one sits behind the target device; Scenario 3: two people sit in front of two target devices, respectively. Attacker attacks one by one.}
    \label{fig:two-scenes}
\end{figure*}
Last, we evaluate if the attack can be used to detect the breathing rate of multiple people simultaneously. We test our attack in three different scenarios. In the first scenario, two people are near the laptop while one is working on the laptop and the other is just sitting next to him, as shown in Figure \ref{fig:scene1}. The attacker targets the laptop and tries to estimate their breathing rate. Note, that the attacker has no prior information about how many people are next to the laptop. In the second scenario, we repeat the same experiment as the first scenario except that the second person is sitting behind the laptop, as shown in Figure \ref{fig:scene2}. In the third scenario, there are two people in the same space but each person is next to a different device. The attacker targets the laptops and tries to estimate their breathing rates. In these experiments, the target device is 0.5-0.7 m away from the person. 

Figure \ref{fig:scenarios_vs_accuracy} shows the results for this evaluation. The blue bars show the result for the first person who is working on the laptop, and the red bars show the results for the second person. Our results show that the attack effectively detects the breathing rate of both people regardless of their orientation. However, the accuracy in detecting the breathing rate for the second person is a bit lower than the first person for the first and second scenarios. This is because the second person's distance to the target device is slightly more and hence the accuracy has decreased. 


\section{Security Implication: Battery Drain Attack}
 In this section, we show how an adversary can drain the battery of our WiFi devices by using the above loopholes and forcing our WiFi devices to stay awake and continuously transmit WiFi signals. 


\subsection{Attack Design and Setup}


\subsubsection{Attack Design}
\label{BatteryAttackDesign}
The attacker forces the target device to stay awake and continuously transmit WiFi packets by sending it back-to-back fake frames and some periodic fake beacons. However, to maximize the amount of time the target device spends transmitting, we study a few different types of fake query packets that the attacker can send. 
Note, that the power consumption of transmission is typically higher than that of reception.\footnote{For example, ESP8266~\cite{esp8266} and ESP32~\cite{esp32} WiFi modules draw 50 and 100 mA when receiving while they draw 170 and 240 mA when transmitting.  These low-power WiFi modules are very popular for IoT devices~\cite{abedi2019wi}.} Hence, to maximize the battery drain, we want to send a short query packet and receive a long response.

\begin{table}[h]
    \centering
    \begin{tabular}{|l|l|l|l|}
        \hline
         Query & Query size & Response & Response size \\
         \hline
         Null    &   28 bytes & ACK   &   14 bytes\\
         RTS            &  20 bytes & CTS   &  14 bytes\\
         BAR     &  24 bytes & BA &  32 bytes \\
         \hline
    \end{tabular}
    \caption{Different types of fake queries and their  responses. Note, Null is a data packet without any payload. BAR and BA stand for Block ACK Request, and Block ACK, respectivly.}
    \vspace{-10pt}
    \label{tbl:queries}
\end{table}

Table~\ref{tbl:queries} lists some query packets and their corresponding responses. The best choice for a query packet is Block ACK requests since the target will respond with a Block ACK that is larger than other query responses. Another important factor to consider for maximizing the battery drain is the bitrate.  When the bitrate of the query packet increases, the bitrate of the response will also increase as specified in the IEEE 802.11 standard. Hence, at first glance, it may seem that to maximize the battery drain, the attacker must use the fastest bitrate possible to transmit query packets, forcing the target device to transmit as many responses as possible. However, it turns out that this is not the case. The power consumption depends mostly on the amount of time the target device spends transmitting packets. Hence, when a higher rate is used for the query and response packets, the total time the target spends on transmission does not increase. In fact, the total time spent transmitting decreases mainly due to overheads such as channel sensing and backoffs. For example, if we increase the bitrate by 6 times (i.e., from 1 Mbps to 6 Mbps), the number of packets will increase by only 3.3 times. As a result, to maximize the transmission time of the target device, the attacker should use the lowest rate (i.e., 1 Mbps) for the query packet. 





\subsubsection{Attack Setup}
\vspace{0.05in}
\noindent

\noindent\textbf{Attacking device:}
Any WiFi card capable of packet injection can be used as the attacking device. We use a USB WiFi card connected to a laptop running Ubuntu 20.04. The WiFi card has an RTL8812AU chipset~\cite{rtl8812au} that supports IEEE 802.11 a/b/g/n/ac standards. We have installed the aircrack-ng/rtl8812au driver~\cite{aircrack-ng} for this card which enables robust packet injection. We utilize the Scapy~\cite{scapy} library to inject fake WiFi packets to the target device. 
%Scapy is a Python library that enables convenient packet sniffing and injection functionalities. 
Scapy allows defining customized packets and multiple options for packet injection. 
Since we need to inject many packets in this attack, we use the \textit{sendpfast} function to inject packets at high rates. \textit{sendpfast} relies on \textit{tcpreplay}~\cite{tcpreplay} for high performance packet injection. 
%Figure~\ref{fig:wifi-attack-code} shows our high-performance fake packets creation and injection implementation.

% \begin{figure}
% \begin{python}
% from scapy.all import *

% # Defining beacon packet
% beacon_dot11 = Dot11(type=0x00, subtype=0x8,
% addr1=<TARGET>, addr2=<ROUTER>, addr3=<ROUTER>)
% essid=Dot11Elt(ID='SSID', info=<SSID>, 
% len=len(<SSID>))
% beacon_frame = RadioTap()/beacon_dot11/
% Dot11Beacon(cap='ESS+privacy')/essid/
% hex_bytes('0504000100ff')

% # Defining block ACK request
% BAR_dot11 = Dot11(type=0x001, subtype=0xb8,
% FCfield=0x00, addr1= <TARGET>)
% BAR_frame = RadioTap()/BAR_dot11/
% (hex_bytes(<ROUTER>))/hex_bytes('04002090')

% #Creating packet list for injection
% frames = [beacon_frame]
% for i in range(<num BARs>):
%     frames.append(BAR_frame)

% #Injecting packets
% while True:
%         sendpfast(frames, iface=<INTERFACE>, 
%         pps=1300, loop=100000)
% \end{python}
% \vspace{-10pt}
% \caption{Python code for injecting fake beacons and queries.}
% \vspace{-10pt}
% \label{fig:wifi-attack-code}
% \end{figure}


%\vspace{0.05in}
\noindent
\textbf{Target device:}
Any WiFi-based IoT device can be used as a target.  We choose Amazon Ring Spotlight Cam Battery HD Security Camera~\cite{ringcamera}
for our battery drain experiments. The camera is powered by a custom 6040 mAh lithium-ion battery. 
The battery life of this camera is estimated to be between 6 and 12 months under normal usage~\cite{xyz, bat2}.
We leave the camera settings to their defaults which means most power-consuming options are turned off. This assures that our measurements will be an upper bound on the battery life and hence the attack might drain the battery much faster in the real world.
Authors in \cite{vanhoef2020protecting} pointed out the possibility of a battery draining attack by forging beacon frames. However, they did not provide any evaluations to test this idea. Moreover, we show how sending fake packets in addition to fake beacon frames can significantly increase the power consumption on the victim device.

\subsection{Results}
We evaluate the effectiveness of the battery drain attack in terms of range and using different payload configuration. 

\begin{figure}[t]
    \centering
    \begin{subfigure}[b]{0.35\textwidth}
    \includegraphics[width=1\columnwidth]{figures/battery-draining/packets-bar.pdf}
    \caption{}
    \end{subfigure}
    \begin{subfigure}[b]{0.35\textwidth}
    \includegraphics[width=1\columnwidth]{figures/battery-draining/battery-bar-WH.pdf}
    \caption{}
    \end{subfigure}
    \caption{The figure shows (a) Average number of packets sent to and received from the target device. (b) Energy consumption in Watt Hour measured under different configurations (i.e. packet type / bitrate (Mbps) }
    \label{fig:batteryDrain}
\end{figure}

\begin{figure*}[t]
    \centering
    \includegraphics[width=0.8\textwidth]{figures/battery-draining/wifi-range.pdf}
    \vspace{-10pt}
    \caption{Percentage of attacker's query packets responded by the target device for different attacker's locations.}
    \label{fig:battery-range-map}
\end{figure*}


\begin{table*}[t]
    \centering
    \begin{tabular}{|c|c|c|c|c|}
        \hline
        Battery Type & Voltage (V) & Full Capacity (Wh) & 100\% Drain  (min)  & 25\% Drain (min) \\
        \hline
        CR2032 coin & 3.0  &  0.68  &  14  & 3.5  \\
        AAA    & 1.5  &  1.87  &  39  & 10 \\
        AA     & 1.5  &  4.20  &  90  & 22\\
        \hline
    \end{tabular}
    \caption{The time it takes for the attack to drain different types of batteries}
    \label{tab:my_label}
\end{table*}

\subsubsection{Finding the optimal configuration:}
As discussed in~\ref{BatteryAttackDesign}, sending block ACK requests at the lowest bitrate (i.e., 1 Mbps) should maximize the power consumption of the target device. To verify this, we have conducted a series of experiments with different types of query packets and transmission bitrates. In each experiment, we continuously transmit query packets to the Ring security camera. In all experiments, we start with a fully charged battery and the attacker injects query packets as fast as possible.

Figure~\ref{fig:batteryDrain} (a) shows the maximum number of packets the attacker could transmit to the target device, and the number of responses it receives per second. Figure~\ref{fig:batteryDrain} (b) shows the amount of energy drawn from the battery during one hour of the attack. As expected, sending Block ACK Requests (BAR) drains more energy from the battery since the target device spends more time on transmission than receiving. Moreover, the results verify that although increasing the data rate from 1Mbps to 6Mbps (BAR/1 versus BAR/6) increases the number of responses, it decreases the energy drained. As mentioned before, this is because the total time spent transmitting decreases mainly due to overheads such as channel sensing and backoffs. This result confirms that sending block ACK requests (BAR) with the lowest datarate is the best option to drain the battery of the target device. 







% \begin{figure}[t]
%     \centering
%     \includegraphics[width=0.7\columnwidth]{figures/battery-draining/wifi-range-setup.pdf}
%     \caption{The setup of the WiFi battery draining experiment.}
%     \vspace{-10pt}
%     \label{fig:wifi-range-setup}
% \end{figure}




\subsubsection{Battery drain with optimal configurations}
We use the best setting which is a block ACK request (BAR) query transmitted at 1~Mbps to fully drain the battery of the Ring security camera. We are able to drain a fully charged battery in 36 hours. Considering the fact that the typical battery life of this camera is 6 to 12 months, our attack reduces the battery life by 120 to 240 times! It is worth mentioning that since a typical user charges the battery every 6-12 months, on average the batteries are at 40-60\%, and therefore it would take much less for our attack to kill the battery. Moreover, the RING security camera is using a very large battery, most security sensors are using smaller batteries. Table~\ref{tab:my_label} shows the amount of time it takes to drain different batteries. For example, it takes less than 40 mins to kill a fully charged AAA battery which is a common battery in many sensors.  





\subsubsection{Range of WiFi battery draining attack}
A key factor in the effectiveness of the battery draining attack is how far the attacker can be from the victim's device and still be able to carry on the attack. If the attack can be done from far away, it becomes more threatening. 
To evaluate the range of this attack, we design an experiment in which the attacker transmits packets to the target from different distances and we measure what percentage of the attacker's packets are responded to by the target device.
We use a realistic testbed. The Ring security camera is installed in front of a house, and the attacker is placed in a car, parked at different locations on the street. We test the attack at 10 different locations up to 150 meters away from the target device. Figure~\ref{fig:battery-range-map} 
shows these locations and our setup. Each yellow circle represents each of the locations tested at. The numbers inside the circles show the percentage of the attacker's packets responded to by the camera.
Each number is an average of over 60 one-second measurements.
The closest distance is about 5 meters when we park the car in front of the target house. 
In this location 97\% of the attacker's packets are responded to.
We conducted other experiments within 10 meters of the target (not shown here) and we obtained similar results. Our results show that even within a distance of 100 meters, almost all attacker's packets are responded to by the victim's device. In some locations such as the rightmost circle (at 150 meters away), we could still achieve a reply rate as high as 73\%, confirming our attack works even at that distance. The reason for achieving such a long range is that the attacker transmits at a 1 Mbps bitrate which uses extremely robust modulation and coding rate (i.e. BPSK modulation and a 1/11 coding rate). 



\section{Conclusion}
\label{sec:conclusion}
This paper presents a generic top-$\size$ recommendation framework for  trading-off accuracy, novelty, and coverage. To achieve this, we profile the users according to their preference for long-tail novelty. We examine various measures, and formulate an optimization problem to learn these user preferences from interaction data.  We then integrate the user preference estimates in our generic framework, GANC.  Extensive experiments on several datasets confirm that there are trade-offs between accuracy, coverage, and novelty. Almost all re-ranking models increase coverage and novelty at the cost of accuracy. However, existing re-ranking models typically rely on rating prediction models, and are hence more effective in dense settings. Using a generic approach, we can easily incorporate a suitable base accuracy recommender to devise an effective solution for both sparse and dense settings.  %Our results  also indicate there is no single method that outperforms other methods in all metrics. However our techniques obtain a significant improvement in coverage, while  . 
Although we integrated the  long-tail novelty preference estimates into a re-ranking framework, their use-case is not limited to these frameworks. In  the future, we intend to explore the temporal and topical dynamics of long-tail novelty preference, particularly in settings where contextual information is  available.  
%We achieve these objectives without using any additional contextual information.


\iffalse
While we focused on promoting long-tail items across users, we did not consider diversity of individual top-$\size$ recommendations, a factor that has been shown to positively affect consumer satisfaction. This is one direction for future work. Moreover, the sequential setting  in our work, creates a dependency between different batches, where,  the items recommended to a batch of users, depends on those recommended to previous batches. This dependency is created through the parameter $\mathbf{f}$, that is updated every time a top-$\size$ set  is allocated to a batch of users. A future direction for our work is to estimate a distribution over $\mathbf{f}$ that allows us to independently solve the problem for each user, leading to improvements across all performance metrics, including recommendation time. 

We design algorithms that take advantage of the structure in the value functions to obtain both efficient and scalable solutions. 
We design an algorithm that takes advantage of the structure in the value functions to obtain both efficient and scalable solutions. 

\textcolor{red}{Our  sequential  algorithms can be applied for batch recommendation contexts,~e.g., personalized email marketing, where based on prior interaction data between users and items,  a new round of recommendations must be sent to all users in the system.  However, the independent coverage algorithms lift the sequential setting restrictions and allow it be applied for re-ranking the output of base recommender in any setting. }A future direction for our work is to incorporate explicit diversity metrics in the framework. 
\fi


%We have a presented a submodular maximization framework to systematically trade-off relevance and diversity in recommendations to individual users and coverage across the item-space. This ensures both consumer and producer satisfaction. We model users according to their risk and focusing degrees and promote long-tail items to the right group of consumers. Consequently, we obtain a significant improvement in coverage while maintaining reasonable levels of user satisfaction. Furthermore, our methods are able to achieve a more balanced distribution across the set of recommended items. In the future, we plan to investigate the effect of using alternative base recommender systems. 

%Future Work
%However most of these methods assume that the ratings are missing at random (MAR). Since our method of generating recommendations is based on the completed matrix, assuming MAR might introduce additional bias, we will use methods which assume that the ratings at missing not at random (MNAR),explored in~\cite{steck2010training, icml2014c2_hernandez-lobatob14}. 	 
%Long Tail %Recently, authors in~\cite{cremonesi2010performance} conducted extensive experiments to evaluate the performances of various matrix factorization-based algorithms and neighborhood models on the task of recommending long tail items. Their experimental results show that long tail recommendation leads to a decrease in accuracy for all algorithms. They also showed that for this task, SVD outperforms other state-of-the-art algorithms. 

%========================

%WiSneak
% \section{Privacy and Security Implications}

% \subsection{Discovering WiFi Devices in a Network}
% \label{sec:filter}

% To discover the connected devices in a WiFi network, the attacker can take the advantage of the power-saving mechanism implemented in all of the 802.11 a/b/g/n/ac/ax standards. Specifically, the power-saving mechanism allows client devices to temporarily turn off their WiFi radio and go to sleep. When a client device is in sleep mode, the access point buffers all the incoming packets and notifies the device with a beacon frame. In each beacon frame, there is a field called Traffic Indication Map (TIM) which indicates which device has buffered traffic at the access point. When the client device receives a beacon frame, it will reply with a \textit{Null-function} packet with a power management bit set to "0". In this way, the client device informs the access point it is awake and ready to receive packets.

% We note that the attacker can leverage this mechanism to discover all the WiFi devices in a secured network. First, the attacker need to obtain the MAC address and the SSID of the network's access point. This can be easily done by sniffing the WiFi traffic using software such as Wireshark since the MAC address is never encrypted and all nodes send packets to the AP. Then, the attacker pretends to be the access point and broadcasts fake beacon frames with TIM set to "0xFF", indicating all client devices have buffered traffic. Then, it enters the sniffing mode to sniff for the \textit{Null-function} packets. The null-function packets contain the MAC addresses of client devices, which will be used in the subsequent attack process.  Now that the attacker discovered the MAC addresses of all WiFi devices, the next question is how an attacker can force a targeted WiFi device to continuously transmit. 


% \subsection{Forcing a WiFi Device to Continuously Transmit} 
% \label{sec:turn}
% Existing WiFi sensing attack \cite{zhu2018tu, Banerjee14} relies on the normal WiFi traffic leaking out of the victim's property. This makes it hard for the attacker to execute WiFi sensing anytime. Moreover, it renders difficulty in monitoring continuous miniature motion such as breathing. To be able to attack anytime and uncover the breathing rate of a person, the attacker at least needs a few measurements per second. However, WiFi devices only transmit when there is traffic. To make things worse, WiFi client devices periodically enter sleep mode to save power, during which they barely send any packet. The intermittent nature of WiFi transmission makes revealing fine-grained personal information extremely difficult. So, How can an attacker force a WiFi device to continuously transmit?

% One may think that the attacker can do this by hacking the network and get control over the WiFi devices and access point. However, this approach is becoming more and more difficult, with the advancement of WiFi Protected Access (WPA). In contrast, we realize that some inherent properties of WiFi protocol can be leveraged by the attacker to overcome these challenges and obtain continuous WiFi packets for sensing without gaining access to the network. 

% We found that WiFi devices reply to even fake data packets out of their network with acknowledgment (ACK) packets. In particular, we examined over 5,000 WiFi devices from 186 vendors, and found that all existing WiFi devices respond  with ACK to fake packets transmitted to them. Hence, an attacker can exploit this property and generate back-to-back fake packet streams to push the WiFi devices to continuously transmit ACK packets. The signal of these ACK packets can be used to perform WiFi sensing attacks. The only needed information to do so is the MAC address of the target device, which has been obtained with the method stated in \ref{sec:filter}. 

% However, there is still one issue: most WiFi devices go to sleep to save energy during inactive states such as screen lock, during which the attacker is not able to push them to transmit by sending back-to-back packets. Figure~\ref{fig:dis} show the results of an experiment where the attacker is continuously transmitting fake packets to a WiFi device. In this figure, we plot the amplitude of CSI over time for the ACK packets received from the WiFi device. As it can be seen, the responses are sparse and discontinued even when the attacker sends back-to-back packets to the WiFi device.



% \begin{figure}[!ht]
%     \centering
%     \begin{subfigure}[b]{0.4\textwidth}
%         \centering 
%         \includegraphics[width=\textwidth]{figures/disjoint_data.png}
%         \caption{Without fake beacon frames}
%         \label{fig:dis}
%     \end{subfigure}
%     \hfill
%     \begin{subfigure}[b]{0.4\textwidth}
%         \centering
%         \includegraphics[width=\textwidth]{figures/continous_data.png}
%         \caption{With fake beacon frames}
%         \label{fig:cont}
%     \end{subfigure}
%     \caption{The CSI amplitude of ACKs responded by the target device when an attacker sends back-to-back fake packets to it in two scenarios. (a) In this scenario, the attacker is not using fake beacon frames. Therefore, the target device goes to sleep mode frequently and does not respond to fake packets. (b) In this scenario, the attacker infrequently sends fake beacon frames to keep the target device awake all the time.}
%     \label{fig:time-comp}
% \end{figure}

% To solve this problem and keep the target device awake, we found that the attacker can send the target device a forged beacon frame that claims buffered packets are waiting for the target device. Note, this can simply be done by alternating bits in TIM as mentioned in Section \ref{sec:beacon}. In this way, \wisneak \  prevents the target device from going to sleep, as it will always be deceived to believe that buffered packets are waiting for it at the AP. Although sending these fake beacon frames wakes the target device up, sending them very frequently will cause WiFi devices to recognize the suspicious attacker's behavior and disconnect from it. Therefore, instead of just sending beacon frames back-to-back, the attacker is continuously transmitting normal packets to a WiFi device and periodically sends fake beacon frames to keep it awake. 

% Figure~\ref{fig:cont} shows the result of an experiment where the attacker is continuously transmitting fake packets to a WiFi device and periodically sends fake beacon frames. As it can be seen, the target device is continuously awake and responding to fake packets with ACKs. Finally, note that in this experiment, the target device was placed close to a person and therefore a periodic breathing pattern can be seen in the CSI amplitude of the acknowledgment packets responded by the target WiFi device. Therefore, \wisneak\ can easily detect the breathing rate of the person by taking a Fourier Transform of the signal.




% \subsection{Choosing a Target Device} 
% \label{sec:target_selection}
% So far, we have explained how \wisneak\  can control the transmission of a WiFi device and potentially analyze its signal to monitor people. The next question is which WiFi device in a network would be a good candidate for the attack. Although in theory, \wisneak\  can make any WiFi device into a sensor, not all devices are good candidates for measuring activities such as breathing rate. For instance, people do not stay close to devices like TV or fridge for a long time, and thus these WiFi devices are not suitable candidates for \wisneak. Devices such as laptops and mobile phones are better options for measuring subtle movements since people are likely to spend a good amount of time near them. 

% To pick the target device, \wisneak\  first sends fake packets to all devices continuously. As mentioned before, all WiFi devices respond to these fake packets with ACK packets. \wisneak\  then analyze the responses of each WiFi device, looking for breathing patterns. Note, as we will show in the following section, there are no complex computations when analyzing received signals. Therefore, the data from all WiFi devices can be processed on a single processing unit such as a typical laptop. Next, \wisneak\ discovers the devices for which their signal includes breathing  patterns. Then it stops monitoring other devices and continues sending fake packets only to those specific WiFi devices. In the following section, we explain how \wisneak\  extracts breathing rate from the responses of a WiFi device using a non-uniform FFT algorithm.


\subsection{Extracting Breathing Rate}
\label{sec:algo}
\textcolor{blue}{An attacker can use Polite WiFi to extract the breathing rate of a victim. An adult's normal breathing rate is around $12$-$20$ times per minute (i.e., $0.2$-$0.33 Hz$). By Nyquist Sampling Theorem, acquire a few packets per second) is more than enough to detect the breathing rate. However, due to the random delays of WiFi random access protocol and operating system's scheduling protocol, our collected data samples are not uniformly spaced in time, which calls for at least $100$ samples per second to detect the breathing rate, according to our experiment. In fact, this is not practical for an attacker, as injecting such a large amount of packets in a network can be suspicious to the network owner.}

\textcolor{blue}{The attacker can overcome this issue by transmitting a fewer number of fake packets and utilizing a Non-Uniform Fast Fourier Transform (NUFFT) algorithm \ref{alg:nfft} to recover the missing information.}
\begin{algorithm}
\SetAlgoLined
\SetKwFunction{Union}{Union}\SetKwFunction{Interpolation}{Interpolation}
\KwData{Time indices $t$, data samples $x$ of length $n$}
\KwResult{Magnitude of each frequency component}
 $d \leftarrow \min_i({t_i - t_{i - 1}}) \quad i = 1, 2, ..., n.$\;
 \For{$i \leftarrow 1$ \KwTo $n - 1$}{
    $interval \leftarrow t[i] - t[i - 1]$\;
  \If{$interval > d$}{
    $count \leftarrow \lfloor interval / d \rfloor$\;
    \textbf{Interpolation}($t$, $x$, $t[i]$, $t[i-1]$, $count$)\;
   }
 }
 \Return \textbf{FFT}($t$, $x$)
 \caption{Non-uniform FFT}
 \label{alg:nfft}
\end{algorithm}

\textcolor{blue}{The algorithm first finds the minimum time gap between any two adjacent data points $d$, then linearly interpolate any interval that is larger than the gap with $\lfloor interval / d \rfloor$\$ samples. Finally, it uses regular FFT algorithm to find the magnitude of each frequency component. A low-pass filter is applied before feeding data to the FFT analysis to reduce noise (not shown in the algorithm). }

\begin{figure*}[!t]
    \centering
    \begin{subfigure}[b]{0.33\textwidth}
        \centering 
        \includegraphics[width=\textwidth]{figures/down_sample.png}
        \caption{Raw and filtered data before \\ interpolation}
    \end{subfigure}
    \begin{subfigure}[b]{0.33\textwidth}
        \centering
        \includegraphics[width=\textwidth]{figures/up_sample.png}
        \caption{Raw and filtered data after \\ interpolation}
    \end{subfigure}
        \begin{subfigure}[b]{0.33\textwidth}
        \centering
        \includegraphics[width=\textwidth]{figures/fft_compare.pdf}
        \caption{Standard FFT and a non-uniform FFT of Data}
    \end{subfigure}
    \caption{Steps to extract breathing rate from the CSI.}
    \label{fig:process-steps}
\end{figure*}

\textcolor{blue}{Figure~\ref {fig:process-steps}(a) and ~\ref{fig:process-steps}(b) show the amplitude of CSI before and after interpolation, respectively, when the attacker sends 30 packets per second to a WiFi device that is close to the victim. Each figure shows both the original data (in blue) and the filtered data (in orange). Figure~\ref{fig:process-steps}(c) shows the frequency spectrum of the same signals when a standard FFT or our non-uniform FFT is applied. A prominent peak at 0.3Hz is shown in the non-uniform FFT spectrum, indicating a breathing rate of 18 bpm.}

\textcolor{blue}{WiFi CSI gives us the amplitude of 52 subcarriers per packet. We observed that these subcarriers are not equally sensitive to the motion of the chest. Besides, a subcarrier's sensitivity may vary depending on the surrounding environment. For more reliably attack, the attacker should identify the most sensitive subcarriers over a sampling window. Previously proposed voting mechanisms for coarse-grained motioned detection applications \cite{alexa,  Wi-chase,MoSense,WiGest} cannot be directly applied in this situation, as chest motion during the respiration is at much small scale. Instead, we developed a soft voting mechanism, where each subcarrier gives weighted vote to a breathing rate value. The breathing rate that gets most votes is reported. }

\textcolor{blue}{Specifically, We first find the power of the highest peak ($P_{peak}$), and then calculate the average power of rest bins ($P_{ave}$). The exponent of the Peak-to-Average Ratio (PAR): $e^{\frac{f_{peak}}{f_{ave}}}$ is used as the weight of the corresponding subcarrier. In this way, we guarantee the subcarriers with higher SNR have significantly more votes than the rest of the subcarriers. }
% Then for any interval between two consecutive data points such that $t_i - t_{i-1} > d$, interpolated points will be added between them. As shown in Algorithm \ref{alg:nfft}, once the interval that needs interpolations is found, the number of points needed is recorded into $count$. Then the function \textbf{Interpolation}($x, t, start, end, n$)  modifies the lists $x$ and $t$ by adding $n$ interpolated points between $start$ and $end$.  Specifically, linear interpolation is used in this algorithm. Finally, to reduce the noise, the attacker applies a low pass filter before feeding the data to the FFT analysis. 
% So far, we have explained how  \wisneak\  can discover the MAC addresses of WiFi devices in a network and push them to transmit packets. However, if \wisneak\  sends too many packets per second, it will always be occupying the channel. This will result in a significant drop in the throughput of the WiFi network.
% As a result, the network owner might suspect that something is going on.
% In this section, we investigate what is the minimum number of packets required to extract the breathing rate without impacting the network performance. 

% An adult's normal breathing rate is around $12$-$20$ times per minute (i.e., $0.2$-$0.33 Hz$). One can imagine that based on the Nyquist rate, the sampling rate of a few times per second (i.e., sending a few packets per second) is more than enough to detect the breathing rate. However, based on our experiments, we found that we require at least $100$ samples per second to detect breathing rate. This is mainly due to the fact that the collected data is not uniformly spaced in time. In particular, when \wisneak\  tries to send back-to-back packets to the target device, there are random delays due to channel access protocol and the operating system, which cause the sample measurements (i.e., ACKs received from the target) to be non-uniformly spaced in time. Therefore, \wisneak\  requires sending packets to the target device at a much higher rate than the Nyquist Rate (e.g., at least $100$ packets per second).  This is not practical for the attacker. Sending a packet and receiving an ACK takes at least a millisecond. Therefore, when \wisneak\  needs to monitor multiple target devices, sending $100$ packets per second to each device is impossible and significantly impacts the throughput of the network which makes the attack detectable for a normal user. In the following, we explain how an attacker can solve this problem by transmitting a fewer number of fake packets and compensating for the fewer number of samples by utilizing the non-uniform FFT.


% \textbf{\emph{Non-uniform FFT:}}
% Let $t_0, t_1, ..., t_n$ be the timestamp of all the data points. The attacker first finds the minimum time gap between these data points $d$ where
% $$
% d = \min({t_i - t_{i - 1}}) \quad i = 1, 2, ..., n.
% $$
% Then for any interval between two consecutive data points such that $t_i - t_{i-1} > d$, interpolated points will be added between them. As shown in Algorithm \ref{alg:nfft}, once the interval that needs interpolations is found, the number of points needed is recorded into $count$. Then the function \textbf{Interpolation}($x, t, start, end, n$)  modifies the lists $x$ and $t$ by adding $n$ interpolated points between $start$ and $end$.  Specifically, linear interpolation is used in this algorithm. Finally, to reduce the noise, the attacker applies a low pass filter before feeding the data to the FFT analysis. 

% \begin{algorithm}
% \SetAlgoLined
% \SetKwFunction{Union}{Union}\SetKwFunction{Interpolation}{Interpolation}
% \KwData{Two lists $t$, $x$ of length $n$ and a number $d$}
% \KwResult{The non-uniform FFT results}
 
%  \For{$i \leftarrow 1$ \KwTo $n - 1$}{
%     $interval \leftarrow t[i] - t[i - 1]$\;
%   \If{$interval > d$}{
%     $count \leftarrow \lfloor interval / d \rfloor$\;
%     \textbf{Interpolation}($t$, $x$, $t[i]$, $t[i-1]$, $count$)\;
%   }
%  }
%  \Return \textbf{FFT}($t$, $x$)
%  \caption{Non-uniform FFT}
%  \label{alg:nfft}
% \end{algorithm}

% Figure~\ref {fig:process-steps}(a) and ~\ref{fig:process-steps}(b) show the amplitude of CSI before and after interpolation, respectively, when the attacker sends 30 packets per second to a WiFi device that is close to a person. Each figure shows both the original data (in blue) and the filtered data (in orange). These figures show that both interpolation and filtering significantly help in extracting the breathing pattern. To better demonstrate the effectiveness of our non-uniform FFT algorithm, we compare the frequency spectrum of the same signals when a standard FFT or our non-uniform FFT is applied. Figure~\ref{fig:process-steps}(c) shows the results of this comparison. 
% The output of the standard FFT does not have any clear peak, making it impossible to detect the breathing rate. On the other hand, the output of the non-uniform FFT has a clear peak at around $0.3$ Hz, which matches the actual breathing rate of $18$ breaths per minute. 
 
%\begin{figure*}[!t]
%    \centering
%    \includegraphics[width=0.4\textwidth]{figures/fft_compare.pdf}
%    \caption{Fourier transform of CSI amplitude, computed using standard FFT and a non-uniform FFT }
%    \label{fig:fft-comp}
%\end{figure*}

% \textbf{\emph{Voting Algorithm:}}
% Since each WiFi packet gives us the amplitude and phase of 52 subcarriers, \wisneak\  can compute the breathing rate by taking the non-uniform FFT over the time-varying amplitude of each subcarrier. However, we found that some subcarriers are more robust than others in detecting the breathing rate, depending on the environment. Therefore, instead of computing the breathing rate using only one of the subcarriers, \wisneak\ applies the non-uniform FFT to all of the subcarriers and uses a soft voting mechanism to calculate the breathing rate. In particular, each subcarrier gives a weighted vote to a breathing rate value. We then calculate which breathing rate value has the highest number of votes. In the following, we explain the voting mechanism.
%The idea is that each subcarrier votes for its own peak, but they all have different weights for their votes. Good subcarriers should have more weight and bad subcarriers should have less.

% Let's assume $P_i$ is the power of each bin in the FFT spectrum, where $i = 1, 2, ..., n$. We first find the power of the highest peak ($P_{peak} = max(P_i)$), and then calculate the average power of other bins ($P_{ave} = \frac{\sum_{i \neq peak} P_i}{n - 1}$). We then calculate the ratio of these two values ($\frac{P_{peak}}{P_{ave}})$ which defines Peak-to-Average Ratio (PAR) of that subcarrier. Finally, we use the exponent of the PAR $w = e^{\frac{f_{peak}}{f_{ave}}}$ as the weights. This guarantees that subcarriers with higher power and SNR have significantly more votes than the rest of the subcarriers. For example, even if there is only one subcarrier that shows the breathing pattern, it can still have a higher weight than the sum of votes from other subcarriers. Finally, It is worth mentioning that body or hand movements would also result in changes in the CSI, however, because such movements are non-periodic, their interference is weak as we take the FFT and filter out the high frequency noise. In contrast, periodic movements due to breathing are enhanced in the FFT operation, which makes the attack even work in non-static scenarios.

% To summarize, \wisneak\ first sends fake packets to a WiFi device in the target property and pushes it to continuously transmit ACK packets. It then uses the CSI information of ACK packets to estimate the breathing rate of the person who is nearby the WiFi device. However, since the packets arrive in non-uniform intervals, \wisneak\ cannot simply use standard FFT to estimate the breathing rate. Instead, it uses a non-uniform Fourier transform, and a voting algorithm to extract the breathing rate.  


\subsection{Attacking Multiple People}
\label{sec:multi}
So far, we have assumed that the attacker is monitoring only a single person. The next question is what would happen if there are multiple people in the space? In the following, we explain different scenarios.

\noindent\textbf{Multiple people around different devices:}
If there are multiple people but each is next to a different WiFi device, \wisneak can easily separate their signal using their MAC addresses. In particular, \wisneak\ attacks devices in a round-robin fashion and extracts the breathing rate of the people one by one by analyzing the signal of the WiFi device next to them. 


\noindent\textbf{Multiple people around a single device:}
If multiple people are next to a single WiFi device, the CSI signal of the ACKs received from that device includes the combination of breathing patterns of all people around it. However, since the chance is very low that two people have exactly the same breathing rate, after taking the Fourier transform of the signal, there will be multiple peaks in the output where each presents the breathing rate of a person. Figure \ref{fig:two_peak_fft} shows an example of Fourier transform output for an experiment where two people were sitting close to a targeted WiFi device. As the figure shows, the signal has two clear peaks where each is presenting the frequency of breathing rate for a person. 

\begin{figure}[!t]
    \centering
    \includegraphics[width=0.4\textwidth]{figures/two-peaks-fft.png}
    \caption{Fourier transform showing two peaks when there are two people breathing with different rates.}
    \label{fig:two_peak_fft}
\end{figure}



%The advantage of this method is that it does not discard information from any subcarrier compared to the best picking method, since human eyes may miss the best subcarrier and tie breaking sometimes can be hard as well. More importantly, the automation of this process makes the real-time analysis of the technique possible. This method always returns a value even when there is no one near the device. Since none of the subcarriers detect a nice peak, it will just pick randomly from them. To avoid this behaviour, a threshold is added for the algorithm. When there is no breathing pattern detected, no major peak should exists in any of the subcarrier, so a threshold $\epsilon$ is set, and the system will report $-1$ if all the weights are smaller than $\epsilon$, meaning there is no breathing detected.



%Now that we explain how the attacker can turn a WiFi device to a breathing sensor. The next question is that which WiFi devices would be a good candidate for the attack. Although in theory the attack can make any WiFi device into a sensor, not all devices are good candidates for measuring breathing rate. For instance, people do not stay close to devices like TV or fridge for a long time, and thus their WiFi devices are not suitable candidates for the attack. Devices such as laptops and mobile phones are better options for measuring subtle movements since people are likely to spend a good amount of time near them. To pick the target device, the attacker first sends fake packets to all devices continuously. As mentioned before, all WiFi devices respond to these fake packets with acknowledgement packets. The attacker then analysis the the responses of each WiFi device, looking for breathing patterns. Note, as we will show in the following section, there is no complex computations during analysing the signal. Therefore, the data from all WiFi devices can be processed on a single processing unit such as a typical laptop. Once the attacker discovers the signal of which WiFi devices includes breathing rate patterns, it stops monitoring the other devices and continues sending fake packets only to those specific WiFi device. In the following we explain how the attacker extract breathing rate from the responses of a WiFi device.



%\subsection{Keep the target device responsive}
%To accurately measure breathing rate, a continuous data flow is required. When the target person is using the device, for example watching a video, the device continuously sends packets to AP, and the normal traffic has enough packets for analysis. However, when the device is not active, it barely sends any packet and we need to force the devices to send more. A simple intuition is to use \emph{Polite WiFi} \cite{abedi2020wifi} which will force devices to send ACK so that there are enough traffic, but the result is not as expected. As shown in Figure \ref{fig:dis}, an attacking device is sending fake packets to the target device using \emph{Polite WiFi}, and the plot is picked from one of the subcarriers. The CSI amplitude is plotted against the time, and it can be seen that the data points are still sparse and discontinued even when the \emph{Polite WiFi} is used. The reason for this is that most WiFi devices will go to sleep to save energy during the inactive state such as screen lock. They will only wake up to receive the beacon frames occasionally, and during the sleeping we are not able to get any information from the devices even using \emph{Polite WiFi}. Knowing this, we can send the target device a forged beacon frame that claims it has buffered packets by alternating bits in TIM mentioned in Section \ref{sec:beacon} to prevent it from going to sleep. However, sending fake beacon frames too frequent will cause devices to have suspicious behaviours such as disconnecting from WiFi. Therefore, combining \emph{polite WiFi} \cite{abedi2020wifi} with fake beacon frames can achieve continuous data flow. The \emph{Polite WiFi} will be used constantly since the fake packets are just normal traffic, but the fake beacon will only be sent periodically, and the rate will be around 20-30 times lower than the fake packets. As shown in Figure \ref{fig:cont}, the data flow is continuous after using the combined technique and since a target person is near the device, a periodic breathing pattern can be seen from this figure. To ensure this method does not disrupt the normal internet traffic and remain stealthy, the data rate for injecting packets must be properly selected. The details of data rate selection will be mentioned in future Section \ref{sec:process}. 
% \section{Evaluation \& Analysis}

%In order to evaluate this attack, a set of experiments are done. These experiments can be mainly divided into two parts: 

To evaluate the feasibility of \wisneak, we conduct a set of experiments in different scenarios. In the following, we first describe our attack scenarios and experiment setup. We then present evaluation results on accuracy and effective range of \wisneak. 

\subsection{Attack Scenarios}
\label{sec:setup}
We evaluate \wisneak\ in two different scenarios: Indoor and Outdoor. In the indoor scenario, the attacker and the target are placed in the same building but on different floors. The height of one floor in the building is around $2.8$ m. 
This scenario is similar to when the attacker and the target person are in different units of an apartment or townhouse.
In the outdoor scenario, the attacker is outside the target's house. For each scenario, we evaluate the attack for different target and attacker locations, as shown in Figure \ref{fig:floor_plan}. For example, for the indoor experiments, the attacker is at location C, the basement of the house, while the target person is at either locations A or B on the first floor. For the outdoor experiments, the attacker is at location D, the backyard, which is around $3$ m away from the outside wall of the house. The target person is at locations A, B, or C. In all of the experiments, the target WiFi devices are placed $0.5$ to $1.4$ m away from the target person's body. The target person is either watching a movie, typing on a laptop, or surfing the web using his cellphone during the experiments. During the experiments, other people are walking and living normally in the house.


\subsection{Attacker Hardware}
The attacker uses a Linksys AE6000 WiFi card and an ESP32 WiFi module~\cite{esp32} as the attacking device. Both devices are connected to a ThinkPad laptop via USB. The Linksys AE6000 is used to send fake packets and the ESP32 WiFi module is used to receive acknowledgments (ACK) and extract CSI. Although we use two different devices for sending and receiving, one can simply use an ESP32 WiFi module for both purposes.
The use of two separate modules gave us more flexibility in running many experiments.

As for the target device, we use a One Plus 8T smartphone without any software or hardware modifications. We have also tested our attack on an unmodified Lenovo laptop, a Microsoft Surface Pro 4 laptop, and a USB WiFi card as the target device and we obtained similar results. It is worth mentioning that any WiFi device can be a target for \wisneak, without any software or hardware modification. 


\begin{figure}[!t]
    \centering
    \includegraphics[width=0.4\textwidth]{figures/floorplan.pdf}
    \caption{Our evaluation testbed. For our indoor experiments, the attacker is placed at location C in the basement of the house, while the target is tested at both locations A and B on the first floor. For our outdoor experiments, the attacker is placed at location D outside the house while the target device is tested at locations A, B, and C.
    } 
    \vspace{-0.1in}
    \label{fig:floor_plan}
\end{figure}

\subsection{Attacker Software}
We have implemented the CSI collecting script on the ESP32 WiFi module, and the breathing rate estimation algorithm (described in Section~\ref{sec:algo}) on the laptop. The collected CSI data is fed to the algorithm which produces the breathing rate estimation values in real-time. 

Upon running the program, the ACK packets constantly flow into \wisneak. To process this data in real-time, a sliding window (buffer) is used. The size of the window is $30$ s and the stride step is $1$ s. 
$30$ seconds is a large enough window for estimating a stable breathing rate value. Note that an adult breathes around 6 times during such a window. The window is a queue of data points, and it updates every second by including $1$ second of new data points to its head and removing $1$ second of old data points from its tail. \wisneak\  runs the analysis algorithm on the data points inside the window whenever it is updated. The window slides once per second. Hence, \wisneak\  reports an estimation of breathing rate every second. Note that there is a $30$-second delay at the beginning, since the window needs to be filled first.

%\begin{figure}[!t]
%    \centering
%    \includegraphics[width=0.3\textwidth]{figures/sliding.png}
%    \caption{Real-time analysis with sliding window}
%    \label{fig:sliding}
%\end{figure}


\subsection{Ground Truth}
\label{sec:ground_truth}
To evaluate the attack's accuracy of estimating the breathing rate, we need to measure the breathing rate ground truth. %A simple method to measure the ground truth is to count the number of breaths per minute, since this does not require any equipment and it is very accurate for a duration of $1$ min. However, \wisneak\  reports a result every second, but it is very hard for counting method to update a number every second to match our estimations. 
To do so, we use a similar method as introduced in \cite{dafna2015sleep}. In particular, we place a microphone near the target person's nose to record the sound of breathing. We then take an FFT of the sound signal to estimate the breathing rate accurately. However, each peak of the sound signal is noisy. Therefore, an envelope function is applied before taking the FFT such that these noises are removed. Note that the use of a microphone is just to obtain the ground truth data and evaluate the performance of \wisneak.

%\begin{figure}[!t]
%    \centering
%    \includegraphics[width=0.4\textwidth]{figures/audio.pdf}
%    \caption{An audio signal is captured by placing a microphone close to the target person's nose. The signal is used to measure the breathing rate ground truth.}
%    \label{fig:audio}
%\end{figure}


Similar to processing the WiFi data, we also apply the sliding window method to the ground truth data. Thus, the results from both the ground truth and WiFi signal can be compared with each other. In addition, the ground truth data and the results produced by \wisneak need to be synchronized in the time domain. To do so, once the program starts, we start a timer that serves as a coordinator. When the timer reaches one minute, the ground truth recorder starts, which marks the start of the experiment. 
%as shown in Figure \ref{fig:sync}. 
The one-minute synchronizing time also covers the $30$s starting delay of \wisneak, so that the window has already buffered enough data for analysis.

%\begin{figure}[!t]
%     \centering
%        \includegraphics[width=0.3\textwidth]{figures/syncrho.png}
%    \caption{Synchronization of the ground truth measurements and \wisneak's estimations}
%    \label{fig:sync}
%\end{figure}


\subsection{Effectiveness}
\label{sec:results}
%For performance metrics, we designed experiments to evaluate several aspects of the system:
%\begin{itemize}
%    \item \textbf{Sensitivity}: Whether the attack is able to detect a change of breathing rate, and whether it can detect if the target person leaves the device 
   % \item \textbf{Accuracy}: %Whether the estimation is accurate enough compared to the ground truth
%    \item \textbf{Effective Range}: How far can the target person be from the target device
%\end{itemize}

We first evaluate the effectiveness of \wisneak\  in detecting breathing by performing an experiment where a person changes his breathing rate. The goal is to see if \wisneak can correctly detect this change in the breathing rate. The attacker is outside the house at location D and the target person is at location A. This experiment lasts three minutes and the target person breathes slowly during the first minute, fast during the following minute, and back to slow again during the last minute. As shown in Figure \ref{fig:change}, \wisneak\  can accurately capture changes in the breathing rate and it matches the ground truth data. This controlled experiment shows that although a person's breathing rate does not usually change so suddenly, \wisneak\  is sensitive enough to capture even such changes. 

\begin{figure}
    \centering
    \includegraphics[width=0.36\textwidth]{figures/control.png}
    \caption{The capability of \wisneak in detecting changes in the breathing rate of a target person.}
    \label{fig:change}
\end{figure}

Next, we run an experiment to examine the \wisneak's capability in detecting whether there is a target person near the WiFi device or not. In this experiment, the target phone is placed on a desk and the person stays around the device for $30$ seconds, then walks away from the device, and then comes back near the device. Note, in our algorithm, when there is no majority vote during the voting phase, we return $-1$ to indicate no breathing detected. Figure~\ref{fig:absence} shows the results of this experiment. As illustrated in the figure, \wisneak can correctly detect the breathing rate when the person is near the device. In other words, it can detect if there is no one near the target device and refrain from reporting a random value.

\begin{figure}[!t]
    \centering
    \includegraphics[width=0.36\textwidth]{figures/absence.png}
    \caption{The efficacy of \wisneak when there is no target near the WiFi device.}
    \label{fig:absence}
\end{figure}



\begin{table}[!t]
    \centering
    \begin{tabular}{V{2.5}c|c|c|c|c|c|cV{2.5}}
    \clineB{1-7}{2.5}
    Tar & Att & Method & 12 & 15 & 20 & 30 \\

    
        \hline     \clineB{1-7}{2.5}

    \multirow{2}{*}{A} & \multirow{2}{*}{C} & \wisneak & 11.97 & 15.03 & 20.01 & 29.94 \\
                                        \cline{3-7}
                                        & & Ground Truth & 12.05 & 14.91 & 20.06 & 30.17 \\
        \clineB{1-7}{2.5}

    \multirow{2}{*}{B} & \multirow{2}{*}{C} & \wisneak & 11.95 & 15.00 & 19.97 & 29.94 \\
                                        \cline{3-7}
                                        & & Ground Truth & 11.98 & 15.07 & 20.01 & 30.06 \\
    \hline    \clineB{1-7}{2.5}

    \multirow{2}{*}{A} & \multirow{2}{*}{D} & \wisneak & 11.96 & 15.00 & 19.98 & 29.96 \\
                                        \cline{3-7}
                                          & & Ground Truth & 11.97 & 15.12 & 20.05 & 30.05 \\
    \hline    \clineB{1-7}{2.5}

    \multirow{2}{*}{B} & \multirow{2}{*}{D} & \wisneak & 11.99 & 15.00 & 20.00 & 29.97 \\
                                        \cline{3-7}
                                        & & Ground Truth & 12.08 & 15.04 & 20.11 & 30.05 \\
    \hline    \clineB{1-7}{2.5}

    \multirow{2}{*}{C} & \multirow{2}{*}{D} & \wisneak & 12.00 & 14.94 & 19.98 & 29.95 \\
                                        \cline{3-7}
                                       & & Ground Truth & 12.04 & 15.01 & 20.08 & 30.16 \\

    \hline    \clineB{1-7}{2.5}

    \end{tabular}
    \vspace{10pt}
    \caption{Estimated breathing rate using \wisneak\ compared to the ground truth for different locations of Attacker(Att) and Target(Tar) and different breathing rates (ranging from 12 to 30 BPM).}
    \label{tab:exp}
\end{table}

\begin{figure*}[!t]
    \centering
    
    \setkeys{Gin}{width=\linewidth}
    \begin{tabularx}{\linewidth}{XXX}
    
    \includegraphics[width=0.32\textwidth]{figures/error_bar.png}
    \caption{The average accuracy of \wisneak\ in estimating the target person's breathing rate across all experiments.}
    \label{fig:bar}
    &
    \includegraphics[width=0.32\textwidth]{figures/cdf.PNG}
    \caption{The CDF of \wisneak's error in estimating the target person's breathing rate.}
    \label{fig:out_cdf}
    &
    \includegraphics[width=0.32\textwidth]{figures/orientation-vs-accuracy.png}
    \caption{Estimation accuracy at various orientations}
    \label{fig:orientation_vs_accuracy}
 \end{tabularx}    
\end{figure*}

\subsection{Accuracy}
\label{sec:accuracy}
Next, we evaluate the accuracy of \wisneak\ in estimating the breathing rate in both indoor and outdoor scenarios as explained in Section ~\ref{sec:setup}. For each scenario, we evaluate \wisneak's accuracy when the target's breathing rate is 12, 15, 20, and 30 breaths per minute.  Note, that the normal breathing rate for an adult is 12-20 breaths per minute while resting, and higher when exercising. In this experiment, the user is watching a video. To make sure the target person's breathing rate is close to our desired numbers, we place a timer in front of the person, where they can adjust their breathing rate accordingly. We run each experiment for two minutes. During this time, we collect the estimated breathing rate from both audio (used for the ground truth) and \wisneak. Table~\ref{tab:exp} shows the results of these experiments. The results show that \wisneak\ accurately detects the breathing rate of the target person in different scenarios and for various breathing rates. To quantify the accuracy of \wisneak in estimating the breathing rate, we also plot the average accuracy of \wisneak in estimating breathing rate across all experiments in Figure \ref{fig:bar}. The accuracy is calculated as the ratio of estimated breathing rate reported by \wisneak\ over the ground truth breathing rate. The figure shows that the accuracy of \wisneak is over $99$\% in all scenarios. 
Finally, Figure \ref{fig:out_cdf} plots the Cumulative Distribution Function (CDF) of the error in  detecting breathing rate for over $2400$ measurements. The CDF is generated based on the estimated breathing rate reported every second by \wisneak. Therefore, in each $2$ minute experiment there are $120$ estimated values. The figure shows that $78\%$ of the estimated results have no error. The figure also shows that $99\%$ of measurements have less than one breath per minute error which is negligible. 



%\begin{figure}[!t]
%    \centering
%    \includegraphics[width=0.4\textwidth]{figures/cdf.PNG}
%    \caption{The CDF of \wisneak's error in estimating the target person's breathing rate.}
%    \label{fig:out_cdf}
%\end{figure}

%\begin{figure}
%    \centering
%    \includegraphics[width=0.4\textwidth]{figures/orientation-vs-accuracy.png}
%    \caption{Estimation accuracy at various orientations}
%    \label{fig:orientation_vs_accuracy}
%\end{figure}

\subsection{Orientation}
Next, we evaluate the effect of orientation of the target person with respect to the target device (laptop). We run the same attack as before for different orientations (i.e. sitting in front, back, left, and right side of a laptop). The user is 0.5m away from the target device in all cases. Figure \ref{fig:orientation_vs_accuracy} shows the result of this experiment. Each bar shows the average accuracy for 90 measurements. Our result shows that regardless of the orientation of the person with respect to the device, \wisneak is always effective and detects the breathing rate of the person accurately. In particular, even when the person was behind the target device, \wisneak still detects the breathing rate with 99\% accuracy.




\subsection{Effective Range}

\textbf{Distance of target device to the person:} So far, the target device is placed $0.5$-$0.7$ m away from the target person's body in our experiments. Here, we evaluate \wisneak\ performance for different distances between the target device and the target person. In particular, we are interested to find out what the maximum distance between the target device and the person can be while \wisneak\ still detects the person's breathing rate. To do so, we place the attacker device and the target device 5 meters apart in two different rooms with a wall in between. We then run different experiments in which the target person stays at different distances from the target device. In each experiment, we measure the breathing rate for two minutes and calculate the average breathing rate over this time. Finally, we compare the estimated breathing rate to the ground truth and calculate the accuracy as mentioned before. 

\begin{figure}
    \centering
    \includegraphics[width=0.4\textwidth]{figures/dist.png}
    \caption{Accuracy of \wisneak\ in estimating the breathing rate for different distances between the target person and the target WiFi device.}
    \label{fig:dist}
\end{figure}

Figure \ref{fig:dist} shows the results of this experiment.  The figure shows that \wisneak's accuracy is over $99\%$ when the distance between the target device and the target person is less than $60$ cm. Note, in reality, people have their laptops or cellphone very close to themselves most of the time, and $60$ cm is representative of these situations. The figure also shows that the accuracy drops as we increase the distance. However, even when the device is at 1.4 m from the person's body, the attack can still estimate the breathing rate with $80\%$ accuracy. Note, this is the accuracy in finding the absolute breathing rate and the change in the breathing rate can be detected with much higher accuracy. Finally, the figure shows that the accuracy suddenly drops to zero for distance beyond 1.4 m. This is due to the fact that at that distance the power of the peak at the output of the FFT goes bellow noise floor, and hence, the peak is not detectable.

\begin{figure}
    \centering
    \includegraphics[width=0.4\textwidth]{figures/cross-building-cdf.png}
    \caption{The CDF of \wisneak's error in estimating the target person's breathing rate when attacker and target are in different buildings.}
    \label{fig:cross-building}
\end{figure}

\textbf{Distance of the attacker to the target:} So far, we have evaluated our attack in different scenarios where the target and the attacker are in different rooms or floors of the same home, or the attacker is outside of the target home. Here we push this further and examine whether our attack works if the attacker and the target person are in a totally different building.  We place the target device in a building on a university campus on a weekday with people around. A person is sitting around 0.5 m away from the device.  We then place the attacker in another building which is around 20 m away from the target building.  Similar to the previous experiment, we run the attack and compare the estimated breathing rate with the ground truth. Figure \ref{fig:cross-building} shows the CDF of error for 180 measurements in this experiment. Our results show that the attacker successfully estimates the breathing rate. Note, the reason that the attack works even in such a challenging scenario with other people being around is two-fold. First, as mentioned in \ref{sec:algo}, using an FFT helps to filter out the effect of most non-periodic movements and focus on periodic movements and patterns. Second, wireless channels are more sensitive to changes as we get closer to the transmitter \cite{abedi2020witag}, and since in these scenarios, the target person is very close to the target device, their breathing motion has a higher impact on the CSI signal compared to the other mobility in the environment. 



\subsection{Multiple People}
\begin{figure}[!t]
    \centering
    \begin{subfigure}[b]{0.17\textwidth}
        \centering 
        \includegraphics[width=\textwidth]{figures/side-by-side.jpg}
        \caption{Scenario 1}
        \label{fig:scene1}
    \end{subfigure}
    \hfill
    \begin{subfigure}[b]{0.3\textwidth}
        \centering
        \includegraphics[width=\textwidth]{figures/face-to-face.jpg}
        \caption{Scenario 2}
        \label{fig:scene2}
    \end{subfigure}
    \caption{Illustrations of multi-people scenarios}
    \label{fig:two-scenes}
\end{figure}
Last, we evaluate if \wisneak can be used to detect the breathing rate of multiple people simultaneously. We test our attack in three different scenarios. In the first scenario, two people are near the laptop while one is working on the laptop and the other is just sitting next to him, as shown in Figure \ref{fig:scene1}. The attacker targets the laptop and tries to estimate their breathing rate. Note, that the attacker has no prior information about how many people are next to the laptop. In the second scenario, we repeat the same experiment as the first scenario except that the second person is sitting behind the laptop, as shown in Figure \ref{fig:scene2}. In the third scenario, there are two people in the same space but each person is next to a different device. The attacker targets the laptops and tries to estimate their breathing rates. Note, that the attacker does not require any prior information on how many devices are next to people. It uses our technique explained in section \ref{sec:target_selection} to find devices that are next to people and estimate their breathing rate. In these experiments, the target device is 0.5-0.7 m away from the person. 

\begin{figure}
    \centering
    \includegraphics[width=0.4\textwidth]{figures/scenario-vs-accuracy.png}
    \caption{Accuracy under three different scenarios: Scenario 1: two people sit side-by-side in front of the target device; Scenario 2: one person sits in front of the target device, the other one sits behind the target device; Scenario 3: two people sit in front of two target devices, respectively. Attacker attacks one by one.}
    \label{fig:scenarios_vs_accuracy}
\end{figure}

Figure \ref{fig:scenarios_vs_accuracy} shows the results for this evaluation. The blue bars show the result for the first person who is working on the laptop, and the red bars show the results for the second person. Our results show that \wisneak can effectively detect the breathing rate of both people regardless of the scenario and their orientation. However, the accuracy in detecting the breathing rate for the second person is a bit lower than the first person for the first and second scenarios. This is because the second person's distance to the target device is slightly more and hence the accuracy has decreased. 






% \section{Is It Possible to Stop This Attack?}
%\textcolor{red}{(Haofan: I suppose although this attack is hard to stop, it is possible to detect that the attack is happening, since the attack has to pretend to be the access point and send fake beacon packets, the actual access point and listen to these packets and realize these packets are not send by itself.)}

In the previous section, we showed how effective \wisneak\ is in detecting the breathing rate of a target person. A natural question is whether it is possible to stop an attacker from monitoring a person's breathing rate. 

\wisneak\ relies on the fact that WiFi devices respond to the attacker's fake packets. This flaw exists in all WiFi devices and cannot be stopped. This is because the physical layer only has $10 \mu s$ to respond to a received packet and it cannot check the validity of the received packet (e.g., check the encryption of the packet). 
%Target devices respond to the attacker's fake packets because of the Polite WiFi behavior~\cite{abedi2020wifi}. This flaw cannot be stopped because the physical layer only has $10 \mu s$ to respond to a received packet and it cannot check the validity of the received packet (e.g., check the encryption of the packet). 
Even if someone finds out an attack is happening by monitoring and checking for fake packets, the physical layer still sends back ACKs for received packets.
Therefore, the attack cannot be stopped at the root. Despite this limitation, we present a solution that confuses the \wisneak attack and can potentially prevent it from estimating the breathing rate accurately.

As explained before, \wisneak\ relies on the CSI changes of WiFi signals to estimate the breathing rate. Therefore, one possible solution to stop such an attack is to artificially create similar changes in the CSI. For example, if the target WiFi device periodically changes its transmission power, it might impact the CSI measured by the attacker. As a result, it might be possible to prevent the attacker from estimating the target person's breathing rate accurately.

\begin{figure}[!t]
    \centering
    \includegraphics[width=0.4\textwidth]{figures/periodic_defense.png}
    \caption{Changing the transmit power periodically when there is no person near the target device.}
    \label{fig:periodic_defense}
\end{figure}

To verify the effectiveness of such a technique, we perform an experiment, in which we periodically change the transmission power of the target device between $10$ dBm and $18$ dBm every $1$ second, while \wisneak\ tries to estimate the breathing rate. Figure \ref{fig:periodic_defense} shows the results of this experiment. The periodic pattern, caused by changing the transmission power, can clearly be seen in the CSI amplitude of the WiFi packets measured by the attacker.

\begin{figure*}[t!]
    \centering
    \setkeys{Gin}{width=\linewidth}
    \begin{tabularx}{\linewidth}{XXX}
    \includegraphics[width=0.31\textwidth]{{figures/disrupt.png}}
    \caption{The effectiveness of the proposed defence for different interval and transmit power changes. The gray area shows the region where the defense was effective.}
    \label{fig:disrupt}
    &
    \includegraphics[width=0.31\textwidth]{figures/12_15_20M_cdf.png}
    \caption{The impact of the proposed defence technique on network throughput when the application data rate is set to 20 Mbps.}
    \label{fig:throughput_20_cdf}
    &
    \includegraphics[width=0.31\textwidth]{figures/12_15_cdf.png}
    \caption{The impact of the proposed defence technique on network throughput when there is no limit set for the application data rate (saturating the channel).}
    \label{fig:throughput_cdf}
 \end{tabularx}    
\end{figure*}

In this experiment, there is no one near the target device. However, \wisneak\ reports a breathing rate of 30 breaths per minute which is the frequency of changes in the transmission power. Therefore, the target device could successfully confuse \wisneak to think that there is someone near the device with a breathing rate of 30 BPM. Moreover, instead of just changing the transmission power between two steps, one can change it between multiple steps to create multiple fake peaks at the output of the attacker's Fourier transform. 


Although, these results imply the effectiveness of this technique in  stopping \wisneak\ attack, changing the transmission power by as much as $8$ dB every second can significantly impact the throughput of WiFi devices. This is because changing the transmission power impacts the SNR of the link, as a result, it can cause packet drops. Moreover, it can also confuse the rate adaptation algorithm on the target device to believe that the channel changes very dramatically over time which can further impact the throughput.
However, it might be possible to change the transmission power by a small amount to stop the \wisneak\ attack, while the throughput is not impacted significantly.
Next, we try to find the minimum required change in the transmission power to disrupt \wisneak.

Similar to the previous experiment, we run a set of experiments where the target device is changing its transmission power. However, we try different transmission power changes at different intervals. In all experiments, there is a person next to the target device and the attacker tries to estimate the breathing rate. Figure \ref{fig:disrupt} shows the results of this experiment. In particular, the figure shows whether the attacker was able to successfully detect the target person's breathing rate for a given amount of change in the transmission power and a given period of change. The shaded area in the plot shows the configurations under which the defense work and the attacker is not able to detect the breathing rate. 

These results show that if we lower the interval (i.e. increase the frequency of change), the required change in the transmission power to prevent the attack decreases. However, note that lowering the intervals below 0.5 s (i.e. 60 times per minute) will make the defense ineffective. This is due to the fact that an adult's breathing rate is in the range of 12-20 breaths per minute and a baby's breathing rate is in the range of 40-60 breaths per minute~\cite{healthsite}. Therefore, the attacker can easily filter out any changes which are above 60 times per minute. Hence, the optimal way to prevent the attacker from estimating the breathing rate is to make the target device's transmission power change by $3$ dB every $0.5$ s. 





%\end{tabularx}  
 
% \begin{figure}[!t]
%     \centering
%     \setkeys{Gin}{width=\linewidth}
%     \begin{tabularx}{\linewidth}{XXX}
%         \begin{minipage}[b]{0.32\textwidth}
%             \centering
%             \includegraphics[width=\textwidth]{figures/disrupt.png}
%             \caption{The effectiveness of the proposed defence technique for different interval and transmit power changes. The gray area shows the region where the defense was effective.}
%             \label{fig:disrupt}
%         \end{minipage}
%         &
%         % \begin{minipage}[b]{0.64\textwidth}
%         \centering
%         \begin{subfigure}[b]{0.32\textwidth}
%         \includegraphics[width=1\textwidth]{figures/12_15_20M_cdf.png}
%         \caption{Application data rate is set to 20 Mbps}
%         \label{fig:throughput_20_cdf}
%         \end{subfigure}
%         &
%         \begin{subfigure}[b]{0.32\textwidth}
%         \includegraphics[width=1\textwidth]{figures/12_15_cdf.png}
%         \caption{Saturating the wireless channel}
%         \label{fig:throughput_cdf}
%         \end{subfigure}
%         \vspace{-0.1in}
%         \caption{The CDF of network throughput for two different application data rates. In each scenario, we run the experiment under three different conditions: (1) the transmit power is stable at 12~dB, (2) the transmit power is stable at 15~dB, and (3) the transmit power oscillates between 12~dB and 15~dB every 0.5s}
%         % \end{minipage}
%     \end{tabularx}    
% \end{figure}



% \begin{figure}[!t]
%     \centering
%     \includegraphics[width=0.4\textwidth]{figures/disrupt.png}
%      \vspace{-0.1in}
%     \caption{The effectiveness of the proposed defence technique for different interval and transmit power changes. The gray area shows the region where the defense was effective.}
%     \vspace{-0.1in}
%     \label{fig:disrupt}
% \end{figure}

Now, the next question is whether such a change in the transmission power impacts the throughput of WiFi devices. 
To evaluate this hypothesis, we set up two laptops. One is a server acting as an Access Point (AP), and the other one is a client acting as a target device. The server hosts a WiFi network and the target device connects to it. We use iperf~\cite{iperf} to send UDP packets from the client to the server and to monitor the throughput for $10$ minutes. We set the application data rate on the target device to 20~Mbps and perform this experiment in three different scenarios: (1) the Tx power is stable at 12~dB, (2) the Tx power is stable at 15~dB, and (3) the Tx power oscillates between 12~dB and 15~dB every $0.5$ s.


% \begin{figure}[!t]
%     \centering
%     \begin{subfigure}[b]{0.4\textwidth}
%     \includegraphics[width=1\textwidth]{figures/12_15_20M_cdf.png}
%     \caption{Application data rate is set to 20 Mbps}
%     \label{fig:throughput_20_cdf}
%     \end{subfigure}
    
%     \begin{subfigure}[b]{0.4\textwidth}
%     \includegraphics[width=1\textwidth]{figures/12_15_cdf.png}
%     \caption{Saturating the wireless channel}
%     \label{fig:throughput_cdf}
%     \end{subfigure}
%     \vspace{-0.1in}
%     \caption{The CDF of network throughput for two different application data rates. In each scenario, we run the experiment under three different conditions: (1) the transmit power is stable at 12~dB, (2) the transmit power is stable at 15~dB, and (3) the transmit power oscillates between 12~dB and 15~dB every 0.5s}
% \end{figure}

Figure \ref{fig:throughput_20_cdf} shows the CDF of throughput for all three scenarios. The figure shows that the WiFi device achieves roughly the same throughput in all scenarios. In particular, oscillating the transmission power does not impact the throughput of the device.  Although the 20 Mbps data rate used in this experiment is more than what most WiFi devices use, there are applications that require higher bandwidths. Therefore, next, we evaluate if oscillating the transmission power impact the throughput of WiFi devices when the device transmits at the maximum data rate (i.e., saturating the WiFi channel). 

Figure \ref{fig:throughput_cdf} shows the CDF of throughput when the client WiFi device saturates the channel for all three scenarios. As expected, lowering the transmission power from 15~dBm to 12~dBm reduces the throughput. However, the results show that when the transmission power oscillates between 12 and 15~dBm, the throughput is even lower than the case where the transmission power is kept at 12~dBm. The reason for this is that WiFi uses a rate adaptation algorithm that changes the physical-layer bitrate according to the channel conditions. High bitrates are used when the signal is strong, while low bitrates are used in poor channel conditions. Therefore, when we change the transmission power back and forth, the algorithm tries to adapt constantly. However, these repeated artificial changes cause the rate adaptation algorithm to aggressively reduce the bitrate or to choose a bitrate that is too high which results in excessive packet drop. As a result, the throughput suffers significantly.

In summary, although our idea of changing the transmission power prevents an attacker from monitoring the breathing rate, it potentially impacts the throughput especially when high application layer data rates are used.
Therefore, WiFi devices can change their transmission power when they do not need high bandwidth. However, as we approach the limit of the channel this problem becomes a trade-off between privacy and throughput.

 







% 
\section{WiFi Battery Drain Attack} \label{sec:wifi-battery-drain}
When IoT devices require high bandwidth, WiFi is the most lucrative option.
It provides a nice balance between throughput, range, and power consumption for applications that require more than 1~Mbps of bandwidth.
This is why most wireless cameras use WiFi to connect to the Internet. 

As explained in Section~\ref{sec:background}, WiFi devices consume hundreds of milliwatts even in the idle mode. Therefore, they utilize power saving mechanisms to significantly reduce the power consumption. In the following section, we show how an attacker can drain the battery of an IoT device by disabling WiFi power saving on the device which significantly increases its power consumption.



\subsection{Attack Overview}
The WiFi battery draining attack we study in this paper employs two main ideas: 1) disabling power saving on the target device and 2) forcing the target device to continuously transmit WiFi packets. In the following we describe each in more details.

\textbf{1) Disabling power saving:} As described in Section~\ref{sec:background}, WiFi devices that are in the power saving mode periodically wake up to receive beacon frames to see if the access point has buffered any incoming packets for them. 
If there are incoming packets for them, they stay awake to receive said packets. Our attack exploits this behavior by actively preventing a WiFi device from going back to sleep.
To do this, the attacker injects fake beacon frames that indicate the target device has some buffered packets on the access point\footnote{Packet injection is the process of transmitting a raw WiFi packet without being connected to any WiFi network. This feature is supported by many WiFi chipsets~\cite{beacon-stuffing-1, beacon-stuffing-2}.}. 
Upon receiving this fake beacon frame, the target device does not sleep and instead tries to contact the access point. 
Soon after, the target device will find out there are no packets buffered for it and will return to sleep. 
However, we found that if the attacker's device keeps sending these fake beacon frames, it can prevent the victim's device from going back to sleep, keeping it in an awake state.

\textbf{2) Forcing the target device to continuously transmit:} Although preventing a device from going to the sleep mode significantly impacts battery life, an attacker can further increase the power consumption of the target device by forcing it to transmit WiFi packets.
Transmission is typically the most power-hungry operation on WiFi chipsets because it involves amplifying the signal before transmission.
But how can an attacker, that is not part of the target device's network, force the target device to transmit packets?

%lets put Abedi et al for camera ready.

A recent study has shown all existing WiFi devices respond with acknowledgement to fake packets transmitted to them, even when the packet is sent from outside of their network~\cite{polite-wifi}. This is because upon receiving a packet, the device must send back an acknowledgment within a short amount of time which is between 10 to 16 microseconds (i.e., SIFS defined in IEEE 802.11). This short turnaround time does not allow for deep inspection of received packets. During this short time, WiFi devices can only calculate the cyclic redundancy check (CRC) of the frame to detect possible errors. As a result, WiFi devices will even acknowledge fake packets received from an attacker. Therefore, the attacker can keep sending fake packets to the target device, forcing it to continuously transmit acknowledgements back, resulting in a significant increase in its power consumption. 

To summarize, an attacker can combine the two above techniques to keep the radio of a WiFi device on and force it to continuously send packets, further increasing the power consumption of a target device. We next explain this attack in detail and provide an analysis of its performance in different scenarios.

% \begin{figure*}[ht]
%     \centering
%     \includegraphics[width=0.75\textwidth]{figures/battery-draining/wireshark.png}
%     \vspace{-10pt}
%     \caption{Injected WiFi traffic and responses from the target device.}
%     \vspace{-15pt}
%     \label{fig:wifi_traffic}
% \end{figure*}

\subsection{Attack Details}
We now present the technical details of the battery draining attack.
As we have previously described, for this attack the attacker transmits a combination of beacon frames and some other fake packets:

\textbf{Fake beacons: } To keep the radio of the target device on, the attacker keeps sending fake beacon frames that tell the target device that it must stay on to receive its buffered frames.
When a client device associates with a WiFi access point, it receives an association ID. In every beacon frame, there is a Traffic Indication Map (TIM) which is a bitmap that indicates which clients have buffered packets on the access point. If the association ID is $x$, then the $(x+1)^{th}$ bit of TIM is assigned to that station.

Therefore, the attacker needs to find the association ID of the device so that they can then set the correct bit in TIM. This requires the attacker to monitor the ongoing traffic for a long time or potentially be present when the target device associates with the access point to figure out the association ID. 
To solve this problem, we set all bits in the TIM to 1. 
The side effect is that all devices in the target network will think they need to stay awake. To eliminate this side effect, we send the fake beacon frame as a unicast packet, instead of the usual broadcast beacons. This way only the target device receives the packet and we don't interfere with the operation of other devices. Interestingly, our experiments show that target devices do not care if they receive beacons as broadcast or unicast frames.

\textbf{Fake queries: }
We refer to any fake WiFi packet that forces a target device to respond as a \textit{query packet}. The attacker tries to force the target device to transmit WiFi packets by continuously sending it fake queries. To maximize the amount of time the target device spends transmitting, we study a few different types of fake query packets that the attacker can send. 
Note, the power consumption of transmission is typically higher than that of reception. For example, ESP8266~\cite{esp8266} and ESP32~\cite{esp32} WiFi modules draw 50 and 100 mA when receiving while they draw 170 and 240 mA when transmitting. 
These low-power WiFi modules are very popular for IoT devices.

\begin{table}[h]
    \centering
    \begin{tabular}{|l|l|l|l|}
        \hline
         Query & Query size & Response & Response size \\
         \hline
         Data (null)    &   28 bytes & ACK   &   14 bytes\\
         RTS            &  20 bytes & CTS   &  14 bytes\\
         B-ACK Req.     &  24 bytes & B-ACK &  32 bytes \\
         \hline
    \end{tabular}
    \caption{Different types of fake queries and their  responses.}
    \vspace{-10pt}
    \label{tbl:queries}
\end{table}



Ideally, we want to send a short query packet and receive a long response. However, the target device's responses are limited to some control packets.
Table~\ref{tbl:queries} lists some query packets and their corresponding responses.
As you can see, the best choice of for a query packet is Block ACK requests because the target will respond with a Block ACK that is bigger than the query packet. We will empirically evaluate this analysis in Section~\ref{sec:wifi-results}.




Another aspect of the query packets is the transmission rate. 
When the bitrate of the query packet increases, the bitrate of the response will also increase as specified in the IEEE 802.11 standard. When the query packet is a control packet such as RTS or a block ACK request, the bitrate of the response will be the same. The control packets are supposed to be transmitted using a basic rate (i.e., legacy 802.11b/g rates up to 24 Mbps). A similar rule is valid for data packets and ACKs. ACKs must be transmitted at the highest basic rate that is lower than the bitrate of the data packet.



At first glance, it may seem that the attacker must use the fastest 
bitrate possible to transmit query packets. This way it forces the target device to transmit as many responses as possible. However, it turns out that this is not the case. The power consumption depends mostly on the amount of time the target device spends on transmitting packets. When a higher rate is used for the query and response packets the total time the target spends on transmission does not increase. The number of responses increases but the duration of each response decreases too.
In fact, the total time spent transmitting decreases mainly due to overheads such as channel sensing and backoffs. For example, if we increase the bitrate by 6 times (i.e., from 1 Mbps to 6 Mbps), the number of packets will increase by only 3.3 times.
As a result, to maximize the transmission time of the target device, the attacker should use the lowest rate (i.e., 1 Mbps) for the query packet.
In contrast, when a high bitrate is used, the target device needs to process more packets. We believe that the power consumption of packet processing is much less than that of active transmission. Therefore, regardless of packet processing, it seems that 1 Mbps is the best choice for query packets. We verify this analysis in Section~\ref{sec:wifi-results}.

% Figure~\ref{fig:wifi_traffic} demonstrates sample beacons, query and response packets.
% The attacker transmits a fake beacon frame to a target device by spoofing the MAC address of the access point that the target device is actually connected to. Since there is no response for the beacon, the attacker does not need to use the slowest rate. In this example, we used 24 Mbps.
% The attacker then sends 5 block ACK requests which are replied to by 5 block ACKs. 
% The attacker continues this process to drain the battery of the target device.
% Our experiments show that sending a beacon between every 5 query packets keeps our target device awake. However, note the optimal number of block ACKs to send for each beacon may depend upon the target device. 



% WiFi modules are typically power-hungry. (\textbf{it might be good to include some figure here for a typical power consumption for a wifi module}). Consequently, the battery-powered WiFi-enabled IoT devices need some mechanisms to save power from the WiFi module, otherwise, they can easily run out of battery and stop functioning. A common mechanism used by many manufacturers is to reduce activity of WiFi module when the device is not used, and hence, entering a power-saving mode. In power-saving mode, the device may not transmit as many packets in the WiFi network, but it still listens to the Access Point (AP) to enable its IoT features. The device may still sense the environment based on their functionality. When the users try to communicate with the device through the access point or the device sense something that needs to report to the users, the device will return to normal mode and activates its WiFi module. There are other mechanisms for power saving such as introducing a fixed time interval sleep in the low level design. However, these mechanisms are not useful in a home security setting that will be discussed in this paper. 



\subsection{Experiment Setup}
Our testbed includes a wireless security camera as the target device and an attacking device. 

\textbf{Target device: }
The most common IoT devices that utilize WiFi are security cameras.
We choose Amazon Ring Spotlight Cam Battery HD Security Camera~\cite{ring-camera} for our battery drain experiments. However, our attack also works on other wireless security cameras since they all use WiFi for communication. The Ring Spotlight Cam is a battery-powered outdoor security camera that comes with various security features. It captures a short video clip when it senses motion. Users can use the application on their phones to view the captured video. It also supports live streaming and two-way audio communication to enhance its security features. To realize all these features with limited battery, the WiFi card of this camera is in the power saving mode most of the time when there is no motion detected and no interactions from the user.
The camera is powered by a custom 6040 mAh lithium-ion battery. The device also has a second battery slot as a backup. However, the box comes with only one battery which is enough to run the camera. The battery life of this camera is estimated to be between 6 and 12 months under normal usage~\cite{ring-camera-battery1, ring-camera-battery2}.

We leave the camera settings to their defaults which means most power consuming options are turned off. This assures that our measurements will be an upper bound on the battery life and hence the attack might drain the battery much faster in the real world.

\textbf{Attacking device: }
Any WiFi card capable of packet injection can be used as the attacking device.
We use a USB WiFi card connected to a laptop running Ubuntu 20.04. The WiFi card has an RTL8812AU chipset~\cite{rtl8812au} that supports IEEE 802.11 a/b/g/n/ac standards.
We have installed the aircrack-ng/rtl8812au driver~\cite{aircrack-ng} for this card which enables robust packet injection.
We utilize the Scapy~\cite{scapy} library to inject fake WiFi packets to the target device. Scapy is a Python library that enables convenient packet sniffing and injection functionalities. It allows defining customized packets and multiple options for packet injection. 
Since we need to inject many packets in this attack, we use the \textit{sendpfast} function to inject packets at high rates. \textit{sendpfast} relies on \textit{tcpreplay}~\cite{tcpreplay} for high performance packet injection. 

% The fake WiFi frames for packets injection are created by Python's Scapy library and are sent through WT-AC9006 WiFi adapter. 
% % I forgot its name... Is it called WiFi adapter? Or WiFi Card? It is the thing we plugged to the laptop
% For the physical setup, the camera is faced towards the wall and the WiFi adapter's antenna is placed 1 meter away from the camera.
% % We may need to convince the audience that it is kind of a worst-case setup because the camera is never activated by the motion. So it increases battery life and makes it harder for us to drain the battery
% WAITING FOR ALI's SETUP FOR A MORE REALISTIC ATTACK.



\subsection{Results\\} \label{sec:wifi-results}
\vspace{-20pt}
\subsubsection{Finding the optimal configuration:}
We first evaluate our analysis for the best type of query packet and bitrate.
We found that sending block ACK requests at the lowest bitrate (i.e., 1 Mbps) should maximize the power consumption of the target device.
To verify this analysis, we have conducted a series of experiments with different types of query packets and transmission bitrates.
In each experiment, we continuously transmit query packets to the Ring security camera for 6 hours.
We start with a fully charged battery and we record the battery level at the end of each 6-hour experiment.
In all experiments, the attacker injects query packets as fast as possible.

Figure~\ref{fig:inject-time} shows the number of packets the attacker transmits to the target device and the number of responses it receives during 100 seconds of our 6-hour experiment.
In this experiment, the query packet is a null packet and the transmission rate is 1 Mbps.
The number of transmitted packets include beacon frames. There is a beacon frame after every five null packets. Since the target only responds to null packets, the number of target's packets (i.e., ACKs) is about 17\% lower than attacker's packets. 


\begin{figure}
    \centering
    \includegraphics[width=0.9\columnwidth]{figures/battery-draining/null-1-mbps.pdf}
    \vspace{-10pt}
    \caption{Packets sent to and received from the target device.}
    \vspace{-15pt}
    \label{fig:inject-time}
\end{figure}

Our first assumption was that transmitting packets consumes more power than receiving WiFi packets. 
To verify this assumption, we increase the payload of the attacker's queries. Therefore, the target spends more time on receiving than responding. Figure~\ref{fig:inject-bar} shows that when we add a 100 byte payload to the null packets the number of attackers packets and responses drop by about 40\% (denoted by Null+Payload).
Figure~\ref{fig:battery-bar} reveals that after 6 hours of sending regular null packets the battery is drained by 10\%. However, when the same attack is performed with null packets that have a 100 byte payload the battery of the camera drains only 8\%. This small change will make the battery drain attack 20\% longer.
Adding a bigger payload will further reduce the effectiveness of the battery drain attack.
This observation confirms that to speed up battery drain the attacker needs to force the target device to spend more time in the transmitting mode.
Note that after 6-hours of normal use without any attack, the battery of the camera stays at 100\%.

As we showed before, a block ACK is larger than a block ACK request. This should improve the speed of the battery drain. 
Figure~\ref{fig:inject-bar} shows that the number of block ACKs we can receive from the target is slightly less than the number of ACK because of larger frame sizes. We show this experiment by BAR/1 in the figure to represent Block ACK Request.
Figure~\ref{fig:battery-bar} also shows that the battery can be drain 30\% faster using block ACK requests compared to null packets.

\begin{figure}
    \centering
    \includegraphics[width=0.9\columnwidth]{figures/battery-draining/packets-bar.pdf}
    \vspace{-10pt}
    \caption{Average number of packets sent to and received from the target device under different settings.}
    \vspace{-15pt}
    \label{fig:inject-bar}
\end{figure}

\begin{figure}
    \centering
    \includegraphics[width=0.9\columnwidth]{figures/battery-draining/battery-bar.pdf}
    \vspace{-10pt}
    \caption{Battery drain during the 6-hour test.}
    \vspace{-15pt}
    \label{fig:battery-bar}
\end{figure}

\begin{figure*}[t]
    \centering
    \includegraphics[width=0.9\textwidth]{figures/battery-draining/wifi-range.pdf}
    \vspace{-15pt}
    \caption{Percentage of attacker's query packets responded by the target device for different attacker's locations.}
    \label{fig:wifi-range}
\end{figure*}

% \begin{figure}[t]
%     \centering
%     \includegraphics[width=0.7\columnwidth]{figures/battery-draining/wifi-range-setup.pdf}
%     \caption{The setup of the WiFi battery draining experiment.}
%     \vspace{-10pt}
%     \label{fig:wifi-range-setup}
% \end{figure}

Finally, we investigate the impact of transmission bitrate.
Our previous analysis suggested that increasing the bitrate should hurt the speed of draining a battery. 
To validate this analysis, we repeat the block ACK request experiment but instead of using 1~Mbps to transmit the requests, we use 6~Mbps. Due to a higher physical layer, many more query packets (i.e., around 3000 per second) can be transmitted. Note that the responses are also transmitted back to us at 6~Mbps as specified by the IEEE 802.11 standard.
The attacker sends close to 2500 block ACKs per second as shown in Figure~\ref{fig:inject-bar} marked by BAR/6.
Despite the 3x increase in the number of responses, the battery is only drained 5\% which is considerably less than the 13\% when 1 Mbps is used.
Interestingly, although the attacker imposes 3x more packet processing overhead to the target device, the energy consumption is still less than half of when 1 Mbps is used. This result shows that the power consumption of transmitting WiFi packets is at least an order of magnitude more than that of packet processing.

\subsubsection{Battery drain with optimal configurations}
We use the best setting which is a block ACK request query transmitted at 1~Mbps to fully drain the battery of the Ring security camera.
We keep blasting beacon and block ACK requests at the camera and measure how long it takes to kill the battery. We are able to drain a fully charged battery in 36 hours. Considering the fact that the typical battery life of this camera is 6 to 12 months, our attack reduces the battery life by 120 to 240 times! Finally, it is worth mentioning that it takes our attack 36 hours to kill a \textit{fully charged} battery (i.e. at 100\%). However, since a typical user charges the battery every 6-12 months, in most cases the batteries are at 20-60\%, and therefore it would take much less for our attack to kill the battery. 



\subsubsection{Range of WiFi battery draining attack}
A key factor in the effectiveness of the battery draining attack is how far the 
attacker can be from the victim device and still be able to carry on the attack. 
If the attack can be done from far away, it becomes more threatening. 
To evaluate the range of this attack, we design an experiment in which the attacker transmits packets to the target from different distances and we measure what percentage of the attacker's packets are responded by the target device.
We use a realistic testbed. The Ring security camera is installed in front of a house, and the attacker is placed in a car, parked at different locations on the street. We test the attack at 10 different locations up to 150 meters away from the target device. Figure~\ref{fig:wifi-range} 
%and Figure~\ref{fig:wifi-range-setup} 
shows these locations and our setup. Each yellow circle represents each of the locations tested at. The numbers inside the circles show the percentage of the attacker's packets responded to by the camera.
Each number is an average over 60 one-second measurements.
The closest distance is about 5 meters when we park the car in front of the target house. 
In this location 97\% of the attacker's packets are responded.
We conducted other experiments within 10 meters of the target (no shown here) and we obtained similar results.

One interesting finding in this experiment is that, even within a distance of 100 meters, almost all attacker's packets are responded by the victim device. In some locations such as the rightmost circle (at 150 meters away), we could still achieve a reply rate as high as 73\%, confirming our attack works even at that distance. The reason for achieving such a long range is that the attacker transmits at 1 Mbps bitrate which uses extremely robust modulation and coding rate (i.e. BPSK modulation and a 1/11 coding rate). 



\bibliographystyle{ACM-Reference-Format}
\bibliography{main}
\end{document}