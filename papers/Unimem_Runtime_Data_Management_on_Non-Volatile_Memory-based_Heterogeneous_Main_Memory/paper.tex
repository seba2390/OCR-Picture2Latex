% This is "sig-alternate.tex" V2.1 April 2013
% This file should be compiled with V2.5 of "sig-alternate.cls" May 2012
%
% This example file demonstrates the use of the 'sig-alternate.cls'
% V2.5 LaTeX2e document class file. It is for those submitting
% articles to ACM Conference Proceedings WHO DO NOT WISH TO
% STRICTLY ADHERE TO THE SIGS (PUBS-BOARD-ENDORSED) STYLE.
% The 'sig-alternate.cls' file will produce a similar-looking,
% albeit, 'tighter' paper resulting in, invariably, fewer pages.
%
% ----------------------------------------------------------------------------------------------------------------
% This .tex file (and associated .cls V2.5) produces:
%       1) The Permission Statement
%       2) The Conference (location) Info information
%       3) The Copyright Line with ACM data
%       4) NO page numbers
%
% as against the acm_proc_article-sp.cls file which
% DOES NOT produce 1) thru' 3) above.
%
% Using 'sig-alternate.cls' you have control, however, from within
% the source .tex file, over both the CopyrightYear
% (defaulted to 200X) and the ACM Copyright Data
% (defaulted to X-XXXXX-XX-X/XX/XX).
% e.g.
% \CopyrightYear{2007} will cause 2007 to appear in the copyright line.
% \crdata{0-12345-67-8/90/12} will cause 0-12345-67-8/90/12 to appear in the copyright line.
%
% ---------------------------------------------------------------------------------------------------------------
% This .tex source is an example which *does* use
% the .bib file (from which the .bbl file % is produced).
% REMEMBER HOWEVER: After having produced the .bbl file,
% and prior to final submission, you *NEED* to 'insert'
% your .bbl file into your source .tex file so as to provide
% ONE 'self-contained' source file.
%
% ================= IF YOU HAVE QUESTIONS =======================
% Questions regarding the SIGS styles, SIGS policies and
% procedures, Conferences etc. should be sent to
% Adrienne Griscti (griscti@acm.org)
%
% Technical questions _only_ to
% Gerald Murray (murray@hq.acm.org)
% ===============================================================
%
% For tracking purposes - this is V2.0 - May 2012

%\documentclass{sig-alternate-05-2015}
\documentclass[sigconf]{acmart}

\usepackage{xifthen}
\usepackage{multirow}
%\usepackage[cmex10]{amsmath}
\usepackage{verbatim}
%\usepackage{url}
\usepackage{amsmath}
%\usepackage[usenames, dvipsnames]{color}
%\makeatletter
%\patchcmd{\maketitle}{\@copyrightspace}{}{}{}
%\makeatother
%\usepackage[hidelinks]{hyperref} 
%\usepackage{etoolbox}
%\usepackage{array}% for fancier tabular
%\usepackage{mathtools}
%\usepackage{tikz}  % for colored circle

%\apptocmd{\thebibliography}{\scriptsize}{}{}

\makeatletter
%\patchcmd{\maketitle}{\@copyrightspace}{}{}{}
\makeatother

\usepackage{subcaption}
\PassOptionsToPackage{hyphens}{url}
\usepackage[hyphens]{url}
\usepackage{hyperref}
\usepackage{booktabs}
\usepackage{graphicx}
\usepackage{caption}
\usepackage{pdfpages}
\usepackage{setspace}
\usepackage{subfiles}
\usepackage[font=small,labelfont=bf]{caption}

\newcommand{\cL}{{\cal L}}
\date{} 
%\apptocmd{\thebibliography}{\scriptsize}{}{}

\setcopyright{none}
%\acmConference[]{}{}{}
%\acmDOI{}
%\acmISBN{}
%\acmPrice{}
%\acmYear{}
\settopmatter{printacmref=false, printfolios=false}
\renewcommand\footnotetextcopyrightpermission[1]{} % removes footnote with conference information in first column
\pagestyle{plain} % removes running headers


\begin{document}

% Copyright
%\setcopyright{acmcopyright}
%\setcopyright{acmlicensed}
%\setcopyright{rightsretained}
%\setcopyright{usgov}
%\setcopyright{usgovmixed}
%\setcopyright{cagov}
%\setcopyright{cagovmixed}


% DOI
%\doi{10.475/123_4}

% ISBN
%\isbn{123-4567-24-567/08/06}

%Conference
%\conferenceinfo{PLDI '13}{June 16--19, 2013, Seattle, WA, USA}

%\acmPrice{\$15.00}



%
% --- Author Metadata here ---
%\conferenceinfo{WOODSTOCK}{'97 El Paso, Texas USA}
%\CopyrightYear{2007} % Allows default copyright year (20XX) to be over-ridden - IF NEED BE.
%\crdata{0-12345-67-8/90/01}  % Allows default copyright data (0-89791-88-6/97/05) to be over-ridden - IF NEED BE.
% --- End of Author Metadata ---

\title{Unimem: Runtime Data Management on Non-Volatile Memory-based Heterogeneous Main Memory} %for High Performance Computing}

\author{Kai Wu}
\affiliation{%
  \institution{University of California, Merced}
}
\email{kwu42@ucmerced.edu}

\author{Yingchao Huang}
\affiliation{%
  \institution{University of California, Merced}
}
\email{yhuang46@ucmerced.edu}

\author{Dong Li}
\affiliation{%
  \institution{University of California, Merced}
}
\email{dli35@ucmerced.edu}

%
% You need the command \numberofauthors to handle the 'placement
% and alignment' of the authors beneath the title.
%
% For aesthetic reasons, we recommend 'three authors at a time'
% i.e. three 'name/affiliation blocks' be placed beneath the title.
%
% NOTE: You are NOT restricted in how many 'rows' of
% "name/affiliations" may appear. We just ask that you restrict
% the number of 'columns' to three.
%
% Because of the available 'opening page real-estate'
% we ask you to refrain from putting more than six authors
% (two rows with three columns) beneath the article title.
% More than six makes the first-page appear very cluttered indeed.
%
% Use the \alignauthor commands to handle the names
% and affiliations for an 'aesthetic maximum' of six authors.
% Add names, affiliations, addresses for
% the seventh etc. author(s) as the argument for the
% \additionalauthors command.
% These 'additional authors' will be output/set for you
% without further effort on your part as the last section in
% the body of your article BEFORE References or any Appendices.

%\numberofauthors{0} %  in this sample file, there are a *total*
%\numberofauthors{8} %  in this sample file, there are a *total*
% of EIGHT authors. SIX appear on the 'first-page' (for formatting
% reasons) and the remaining two appear in the \additionalauthors section.
%
%\author{
%}
% There's nothing stopping you putting the seventh, eighth, etc.
% author on the opening page (as the 'third row') but we ask,
% for aesthetic reasons that you place these 'additional authors'
% in the \additional authors block, viz.
%\additionalauthors{Additional authors: John Smith (The Th{\o}rv{\"a}ld Group,
%email: {\texttt{jsmith@affiliation.org}}) and Julius P.~Kumquat
%(The Kumquat Consortium, email: {\texttt{jpkumquat@consortium.net}}).}
%\date{30 July 1999}
% Just remember to make sure that the TOTAL number of authors
% is the number that will appear on the first page PLUS the
% number that will appear in the \additionalauthors section.

\begin{abstract}  %150 words limitation for SC'17
Non-volatile memory (NVM) provides a scalable and power-efficient solution to replace DRAM as main memory. However, because of relatively high latency and low bandwidth of NVM, NVM is often paired with DRAM to build a heterogeneous memory system (HMS). As a result, data objects of the application must be carefully placed to NVM and DRAM for best performance. In this paper, we introduce a lightweight runtime solution that automatically and transparently manage data placement on HMS without the requirement of hardware modifications and disruptive change to applications. Leveraging online profiling and performance models, the runtime characterizes memory access patterns associated with data objects, and minimizes unnecessary data movement. Our runtime solution effectively bridges the performance gap between NVM and DRAM. We demonstrate that using NVM to replace the majority of DRAM can be a feasible solution for future HPC systems with the assistance of a software-based data management.

%Non-volatile memory (NVM), such as phase change memory and resistive random-access memory, provides a scalable and power-efficient solution to replace DRAM as main memory for future HPC systems. However, because of relatively high latency and low bandwidth of NVM, NVM is often paired with DRAM to build a heterogeneous memory system (HMS) to gain the benefit of each. As a result, data objects of the application must be carefully placed to NVM and DRAM in HMS for best performance.

%In this paper, we introduce a lightweight runtime solution that automatically and transparently manage data placement on HMS without the requirement of hardware modifications and disruptive change to applications. Leveraging online profiling and performance models, the runtime characterizes memory access patterns associated
%with data objects and minimizes unnecessary data movement.
%To improve runtime performance, we introduce a series of techniques, such as proactive data movement, optimizing initial data placement, partitioning large data objects, and exploring the tradeoff between phase local search and cross-phase global search. 
%Our performance evaluation with representative HPC workloads shows that our runtime solution effectively bridges the performance gap between
%NVM-only and DRAM-only main memories.
%We demonstrate that using NVM to replace the majority of DRAM memory can be a feasible solution for future HPC systems with the assistant of a software-based data management.
%Whether this model for the usage of NVM to replace the majority of DRAM memory
%leads to a large slowdown in their applications?
%It is crucial to answer this question because a large performance impact will be an impediment to the adoption of such systems.
\end{abstract}


\maketitle

\input text/introduction
\input text/background
\input text/sys_design
\input text/eval_methodology
\input text/evaluation
\input text/related_work
\input text/conclusions





%  Use this command to print the description
%
%\printccsdesc

% We no longer use \terms command
%\terms{Theory}

%\keywords{ACM proceedings; \LaTeX; text tagging}


%
% The following two commands are all you need in the
% initial runs of your .tex file to
% produce the bibliography for the citations in your paper.
%\begin{spacing}{0.9}
%\bibliographystyle{ACM-Reference-Format}
\bibliographystyle{abbrvnat}
\bibliography{li}  % sigproc.bib is the name of the Bibliography in this case
% You must have a proper ".bib" file
%  and remember to run:
% latex bibtex latex latex
% to resolve all references
%\end{spacing}

% ACM needs 'a single self-contained file'!
%
%\includepdf[pages=-]{figures/sc17-unimem-artifacts.pdf}

\end{document}
