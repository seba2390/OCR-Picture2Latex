% --------------------------------------------------------------------------- %
% Packages and settings
% --------------------------------------------------------------------------- %
\usepackage[T1]{fontenc}
\usepackage[utf8]{inputenc}
\usepackage{csquotes}
\usepackage[mode=buildmissing]{standalone}

% Math packages
\usepackage{amsmath}
\usepackage{amssymb}
\usepackage{ebproof}
\usepackage{mathtools}
\usepackage{bm}
\usepackage{moresize}
\usepackage[nounderscore]{syntax}

% Package to generate and customize Algorithm as per ACM style
\usepackage[ruled]{algorithm2e}
\usepackage[listings,most]{tcolorbox}
\usepackage{etoolbox}
\renewcommand{\algorithmcfname}{ALGORITHM}
\SetAlFnt{\small}
\SetAlCapFnt{\small}
\SetAlCapNameFnt{\small}
\SetAlCapHSkip{0pt}
\IncMargin{-\parindent}

\usepackage{booktabs}
%\usepackage{caption}

% Package to format and display source code listings
\usepackage{tikz}
  \usetikzlibrary{calc, backgrounds, shapes, positioning}
\usepackage{graphicx}

% --------------------------------------------------------------------------- %
% Listings + Styles
% --------------------------------------------------------------------------- %
\usepackage{listings,lstautogobble,xcolor,beramono}

% Colors for diff visualization
\definecolor{diffstart}{HTML}{808080} % Gray
\definecolor{diffincl}{HTML}{008000}  % Green
\definecolor{diffrem}{HTML}{FF4500}   % OrangeRed

\lstdefinestyle{default}{
  tabsize=2,
  numbers=left,
  breaklines=true,
  postbreak=\raisebox{0ex}[0ex][0ex]{\ensuremath{\color{darkgray}\hookrightarrow\space}},
}
\lstset{autogobble=true, style=default}

\lstdefinelanguage{PPML}{
  language=Fortran,
  morekeywords={%
    fields, particles, initialize, solve, integrate, on, from, to, by, exchange ghosts,%
    print, end, distribution, displaced, foreach, with, position, if, import, external, then, client, rhs%
  },
  morecomment=[l]{!},
  moredelim=[is][keywordstyle]{||}{||}
}

\lstdefinelanguage{PPME}{
  language=Java,
  keywordstyle=\color{blue!50!black}\bfseries,
  morekeywords={foreach, particle, ode, method, on, end},
  stringstyle=\color{green!50!black}\bfseries
}

\lstdefinelanguage{diff}{
  basicstyle=\ttfamily\small,
  morecomment=[f][\color{diffstart}]{@@},
  morecomment=[f][\color{diffincl}]{+\ },
  morecomment=[f][\color{diffincl}]{>\ },
  morecomment=[f][\color{diffrem}]{-\ },
  morecomment=[f][\color{diffrem}]{<\ },
}

\def\inline{\lstinline[basicstyle=\ttfamily\small,mathescape=true,breaklines=false]}
% --------------------------------------------------------------------------- %

\newcommand*\rot{\rotatebox{90}}
\newcommand{\arrowOp}[2]{\ensuremath{#1{\to}#2}}

\newcommand*\circled[1]{\tikz[baseline=(char.base)]{
            \node[shape=circle,draw,inner sep=0.5pt] (char) {#1};}}

\newcommand{\vect}[1]{\ensuremath{\mathbb{V}\langle #1 \rangle}}
\newcommand{\mat}[1]{\ensuremath{\mathbb{M}\langle #1 \rangle}}
%\newcommand{\field}[1]{\ensuremath{\mathcal{E}_F\langle #1 \rangle}}
%\newcommand{\prop}[1]{\ensuremath{\mathcal{E}_P\langle #1 \rangle}}
\newcommand{\prop}[1]{\ensuremath{\mathcal{E}\langle #1 \rangle}}
\newcommand{\integer}{\ensuremath{\mathbb{Z}}}
\newcommand{\bool}{\ensuremath{\mathbb{B}}}
\newcommand{\real}{\ensuremath{\mathbb{R}}}
\newcommand{\particle}{\ensuremath{\mathbb{P}}}

% Presentation of a Herbie diff-file together with its input and output error
% (arg0) input error
% (arg1) output error
% (arg2) body (e.g., diff file, equation, …)
\newcommand{\improvementsFigure}[3]%
  {%
    \centering\footnotesize
    \begin{tabular}{lr}
      \toprule
      \textbf{Input Error} & \textbf{Output Error} \\
      \midrule
      $#1$ & $#2$ \\
      \midrule
      \multicolumn{2}{c}{
        \begin{minipage}{.85\textwidth}
          \footnotesize
          #3
        \end{minipage}
      } \\
      \bottomrule
    \end{tabular}
  }

\newcommand\NoBibDot[1]{}

\makeatletter
\newenvironment{btHighlight}[1][]
{\begingroup\tikzset{bt@Highlight@par/.style={#1}}\begin{lrbox}{\@tempboxa}}
{\end{lrbox}\bt@HL@box[bt@Highlight@par]{\@tempboxa}\endgroup}

\newcommand\btHL[1][]{%
  \begin{btHighlight}[#1]\bgroup\aftergroup\bt@HL@endenv%
}
\def\bt@HL@endenv{%
  \end{btHighlight}%
  \egroup
}
\newcommand{\bt@HL@box}[2][]{%
  \tikz[#1]{%
    \pgfpathrectangle{\pgfpoint{1pt}{0pt}}{\pgfpoint{\wd #2}{\ht #2}}%
    \pgfusepath{use as bounding box}%
    \node[anchor=base west, fill=orange!30,outer sep=0pt,inner xsep=1pt, inner ysep=0pt, rounded corners=3pt, minimum height=\ht\strutbox+1pt,#1]{\raisebox{1pt}{\strut}\strut\usebox{#2}};
  }%
}
\makeatother

%%%% Environment for comments. Set the boolean to false to produce a comment-free version.
\newboolean{showcomments}
\setboolean{showcomments}{false}
\ifthenelse{\boolean{showcomments}}
{ \newcommand{\mynote}[3]{
   \fbox{\bfseries\sffamily\scriptsize#1}
   {\small$\blacktriangleright$\textsf{\emph{\color{#3}{#2}}}$\blacktriangleleft$}}}
{ \newcommand{\mynote}[3]{}}
% One command per author:
\newcommand{\tn}[1]{\mynote{Tobias}{#1}{red}}
\newcommand{\sk}[1]{\mynote{Sven}{#1}{blue}}
\newcommand{\jc}[1]{\mynote{JC}{#1}{green}}

\newcommand{\revii}[1]{#1}

% --------------------------------------------------------------------------- %
% ACM metadata and information
% --------------------------------------------------------------------------- %
% Metadata Information
\acmVolume{V}
\acmNumber{N}
\acmArticle{A}
\acmYear{YYYY}
\acmMonth{11}
\setcopyright{acmlicensed}



