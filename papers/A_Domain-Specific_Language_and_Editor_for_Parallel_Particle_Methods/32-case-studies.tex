% vim: set ts=2 sw=2 sts=2:
% vim: set wrap breakindent:
% !TEX root = master.tex        --- atom

\subsection{Case Studies} % (fold)
\label{sub:case_studies}

To demonstrate the capabilities of PPME, we use the same two simulations as case
studies that were already considered for PPML \cite{Awile:2013a}. The first one, the Gray-Scott reaction-diffusion system as presented in
Section~\ref{sub:a_simple_application_example}, is an example of a simulation of a continuous
deterministic model. The second one, Lennard-Jones molecular dynamics is an example of a simulation of a 
discrete deterministic model. An N-body simulation as a third example further illustrates the initialization of particles from external data. 

\subsubsection{Gray-Scott Reaction-Diffusion System} % (fold)
\label{par:gray_scott_reaction_diffusios_system}
  The PPME program for the Gray-Scott simulation is shown in Fig.~\ref{fig:ppme-gs}. It follows the typical structure of a particle-based simulation, starting
  with the initialization of topology and particles, followed by the simulation loop.
  The notation in PPME is concise and close to the domain idiom.
  The program starts with
  the module definition and the referenced runtime constants. At the beginning of the
  simulation, topology, particles, and neighbor lists are set up. The time steps are contained in the \inline{timeloop} and are solely defined through
  the differential equations to be solved. For the equation block, the developer has to specify the
  particle list the equations are working on, and the time-stepping method. Note that
  the continuous fields $U$ and $V$ are automatically discretized on the particle
  list during code generation. 

  \begin{figure}[tp]
    \centering
    \begin{minipage}{.49\textwidth}
      \includegraphics[scale=1.0]{img/ppme-gs-01.png}
    \end{minipage}
    \begin{minipage}{.50\textwidth}
      \includegraphics[scale=1.0]{img/ppme-gs-02.png}
    \end{minipage}
    \caption{Gray-Scott simulation program in PPME.}
    \label{fig:ppme-gs}
  \end{figure}
% paragraph gray_scott_reaction_diffusios_system (end)

\subsubsection{Lennard-Jones Molecular Dynamics} % (fold)
\label{par:lennard_jones_potential}
  Lennard-Jones is an instance of
  molecular dynamics~\cite{Frenkel2001}, an item-based simulation to study molecular 
  processes. The atoms are directly represented as particles,
  located in continuous space. Pairwise potentials between atoms define the
  continuous forces acting on them. While the basic algorithm for the simulation,
  i.e., computing pairwise interactions of particles and updating their positions
  and properties, remains the same, the exact definition of the forces is specific to
  the application. A classical force definition is given by the Lennard-Jones potential, which is suitable for describing inert gases. 
  %
  The pairwise force between 
  atoms depends on the distance between them ($r$), the depth of the potential well
  ($\varepsilon$), and the fall-off distance ($\sigma$) of the interaction potential. Particle properties such as
  acceleration ($a$) or velocity ($v$) change according to the forces, 
  causing the particles to move. Additionally, a cutoff radius to ignore negligibly small
  long-range interactions is applied.
  
    % JC: Next as separate paragraph - (issue 13) 
  The essential part of simulating the potential is located in the \inline{timeloop}
  depicted in Figure~\ref{fig:ppme-lj-loop}. Therein, the force acting on the
  particles due to pairwise interactions is computed and applied. The loop can be
  divided into four sections (\circled{1}--\circled{4}). First, the particle positions (\arrowOp{p}{pos}) are
  updated based on the values of velocity (\arrowOp{p}{v}) and acceleration
  (\arrowOp{p}{a}) (cf.~\circled{1}). After the particle positions change, the boundary
  condition must be imposed, followed by updating the mappings and neighbor list (cf.~\circled{2}). 
  The block of two nested particle loops implements the actual particle--particle 
  interactions (cf.~\circled{3}). For each particle $p$ the pairwise interaction with all nearby 
  particles $q$, retrieved via \inline{neighbors(p, nlist)}, is computed. The force
  $F=-\nabla E$ acting between two particles and the potential (or energy) $E$ are given by
  %
  \begin{equation} \vec{F}(r) = 24 \varepsilon r \left( 2 \frac{\sigma^{12}}{r^7} -
  \frac{\sigma^6}{r^4}\right) , \qquad E(r) = 4 \varepsilon \left[ {\left(
  \frac{\sigma}{r}\right)}^{12} - {\left( \frac{\sigma}{r} \right)}^{6} \right]
  \end{equation}% 
  %
  This corresponds to the lines with assignments to \inline{dF}
  and \arrowOp{p}{E}, respectively, where \inline{r_s_pq2} corresponds 
  to the squared distance $r^2$ between $p$ and $q$.
  The last update on the particle list modifies the
  velocity as a consequence of the new force (cf.~\circled{4}).
% paragraph lennard_jones_potential (end)

  \begin{figure}[tp]
    \centering
    \begin{minipage}[b]{.51\textwidth}
      \includegraphics[scale=0.95]{img/ppme-lj-timeloop.pdf}
      \caption{The simulation loop body for the Lennard-Jones dynamics.}
      \label{fig:ppme-lj-loop}
    \end{minipage}
    \hfill
    \begin{minipage}[b]{.48\textwidth}
      \centering
      \includegraphics{img/ppme-dimensions-annotations.png}%
      \caption{Dimensions annotated to particle properties in the Lennard-Jones example.}
      \label{fig:dimension_annotation}
      \vspace{6mm}
      \includegraphics{img/ppme-dimensions-declaration.png}
      \caption{Declarations of base dimensions (length, mass, and time) with velocity and acceleration as derived dimensions.}
      \label{fig:dimension_declaration}
     \vspace{6mm}
     \includegraphics{img/ppme-dimensions-error}
      \caption{User notification of an error caused by incompatible
      dimensions.}
       \label{fig:ppme-type_dimension_error}
      \end{minipage}
  \end{figure}

In this example, we also use dimensions to further improve static error detection.
A \emph{dimension declaration} in PPME resides in a special file owned 
by a model, where each declaration contains an identifier \inline{<d>}, an 
optional specification \inline{<spec>}, and an optional suggestive name \inline{<desc>}.
%
Figure~\ref{fig:dimension_declaration} shows the
declaration of fundamental and derived dimensions. From the three fundamental
dimensions length ($l$), time ($t$), and mass ($m$), convenient notations for
velocity ($v$) and acceleration ($a$) are derived. 
%IFS: maybe define a=v/t in order to show that dimensions can also be nested? 
Note that this specification
is not bound to this example and can be reused in any other PPME project.

% JC: Next as separate paragraph - (issue 13)
Figure~\ref{fig:dimension_annotation} shows the annotated particle properties,
which are used by PPME's type system to derive expression types. The  
type system corresponds to the formalization introduced in the previous section.
It enables capturing type and dimension errors right in the editor. Errors are reported 
to the user where they occur, as shown in Figure~\ref{fig:ppme-type_dimension_error}
using the deduction in Figure~\ref{fig:type_deduction}.
The outer addition is highlighted, and the information states that the operation cannot be applied
to operands with given (annotated) types  $[\real, v]$ and $[\real, a \cdot t^2]$. 
    
\subsubsection{N-body Simulation} % (fold)
\label{par:n_body_simulation}
As a third case study, we implement an N-body simulation of two galaxies using 
particle methods. As the model structure of this example is similar to the 
Lennard-Jones example above, we skip the corresponding details
of the code and focus on another important aspect of PPME: its interface with the
underlying Fortran language. 

% JC: Next as separate paragraph - (issue 13) 
\revii{PPME is designed as a standalone DSL, nevertheless unanticipated 
use cases can be supported by inline code statements. In the N-body simulation, 
the initial particle data need to be loaded from an external data source \inline{data.tab}.
As PPME has no built-in functionality to import data from this type of file, it has to 
be specified using custom code.}
%
\begin{figure}
 \centering
    \includegraphics{img/ppme-nb-init.png}
    \caption{Particle initialization using PPML inline code.}
    \label{fig:ppme-nb}
\end{figure}
%
This can be achieved through an \inline{InlineCodeStatement}, which supports 
inlining arbitrary Fortran or PPML code directly into the program. Figure~\ref{fig:ppme-nb} shows how
\inline{data.tab} is read and into PPME's data structures. The PPML code is located 
within a pair of squared brackets, which demarcate it from the rest of the program. 
The code has direct access to the matrix \inline{parts_data} as well as to the declared 
fields \inline{v} and \inline{m}. During this custom initialization, the data are first loaded into 
the matrix (which corresponds to a Fortran array) and afterwards copied to the corresponding
fields. 

Notice that the inlined PPML code is not analyzed by PPME. Errors may therefore be 
introduced by the developers that propagate to later stages in the compile chain. 
However, such code can be conceptualized easily by extending the language and
converting it into an MPS generator, which is one of the central ideas in 
language-oriented programming.  


% paragraph n_body_simulation (end)
% subsection case_studies (end)
