% --------------------------------------------------------------------------- %
% Preface (abstract, classification, author details, etc.)
% --------------------------------------------------------------------------- %

% --------------------------------------------------------------------------- %
% Title, Author, and Page Heads
% --------------------------------------------------------------------------- %
\title{A Domain-Specific Language and Editor for Parallel Particle Methods }

\author{SVEN KAROL$^1$, TOBIAS NETT$^1$, JERONIMO CASTRILLON$^1$ and IVO F.
SBALZARINI$^{1,2}$ 
\affil{$^1$: Technische Universit\"at Dresden, Faculty of Computer Science, Dresden, Germany\\
$^2$: Center for Systems Biology Dresden, Max Planck Institute of Molecular Cell Biology 
and Genetics, Dresden, Germany}}


% --- End of Author Metadata ---

\markboth{S. Karol, T. Nett, J. Castrillon and I. F. Sbalzarini}{A Domain-Specific Language and Editor for Parallel Particle Methods}

%
% At a minimum you need to supply the author names, year and a title.
% IMPORTANT:
% Full first names whenever they are known, surname last, followed by a period.
% In the case of two authors, 'and' is placed between them.
% In the case of three or more authors, the serial comma is used, that is, all author names
% except the last one but including the penultimate author's name are followed by a comma,
% and then 'and' is placed before the final author's name.
% If only first and middle initials are known, then each initial
% is followed by a period and they are separated by a space.
% The remaining information (journal title, volume, article number, date, etc.) is 'auto-generated'.
\acmformat{Sven Karol, Tobias Nett, Jeronimo Castrillon and Ivo F.~Sbalzarini. 2017. A Domain-specific Language and Editor for Parallel Particle Methods}

% --------------------------------------------------------------------------- %
% Abstract
% --------------------------------------------------------------------------- %

\begin{abstract}
% JCM: abstract on 200 chars
Domain-specific languages (DSLs) are of increasing importance in scientific high-performance computing to reduce development costs, raise the level of abstraction and, thus, ease scientific programming. However, designing DSLs is not easy, as it requires knowledge of the application domain and experience in language engineering and compilers. Consequently, many DSLs follow a weak approach using macros or text generators, which lack many of the features that make a DSL comfortable for programmers. Some of these features---e.g., syntax highlighting, type inference, error reporting---are easily provided by language workbenches, which combine language engineering techniques and tools in a common ecosystem. In this paper, we present the Parallel Particle-Mesh Environment (PPME), a DSL and development environment for numerical simulations based on particle methods and hybrid particle-mesh methods. PPME uses the Meta Programming System (MPS), a projectional language workbench. PPME is the successor of the Parallel Particle-Mesh Language, a Fortran-based DSL that uses conventional implementation strategies. We analyze and compare both languages and demonstrate how the programmer's experience is improved using static analyses and projectional editing\revii{, i.e., code-structure editing, constrained by syntax, as opposed to free-text editing}. We present an explicit domain model for particle abstractions and the first formal type system for particle methods.
%%  Domain-specific languages (DSLs) are of increasing importance in scientific high-performance computing to reduce development costs, raise the level of  
%%  abstraction and, thus, ease scientific programming.
%%  However, designing and implementing DSLs is not an easy task, as it requires knowledge of the application domain 
%%  as well as skills and experience in language engineering and compiler construction. Consequently, many  
%%  DSLs follow a weak approach using macros or text generators, which lack many of 
%%  the features that make a DSL a comfortable tool for scientific programmers. Some of these features---e.g., syntax highlighting, instant
%%  analysis, type inference, error reporting, and code completion---are easily provided by language workbenches, which group and combine language
%%  engineering techniques and tools in a common ecosystem.
%%  % 
%%  In this paper, we present the Parallel Particle-Mesh Environment (PPME), a DSL and development environment for numerical simulations based on   
%%  particle methods and hybrid particle-mesh methods. PPME has been created using the meta programming system (MPS), a projectional language workbench developed by JetBrains.
%%  PPME is the successor of the Parallel Particle-Mesh Language (PPML), a Fortran-based DSL that used conventional implementation strategies, such as macro
%%  expansion. We analyze and compare both languages and demonstrate how the programmer's experience can be improved using static
%%  analyses and projectional editing. Furthermore, we present an explicit domain model for particle abstractions and the first formal type system for particle
%%   methods.  
 %
%  We present the Parallel Particle-Mesh Environment (PPME) is a DSL and projectional editor for numerical simulations based on the particle method.
%  PPME implements a generative approach: it generates parallel Fortran code that links with the parallel particle-mesh library (PPM), which is also implemented in Fortran.
%  PPM provides efficient implementations of the particle and mesh abstractions, discrete numerics, as well as an abstraction layer on the underlying HPC hardware.
%
%  The presented DSL supports built-in abstractions such as particles, properties, fields, loops, as well as systems of partial differential equations and differential operators such as the Laplacian.
%  Equations can be written using conventional mathematical notations.
%
%  - analysis features such as, for instance type analysis and dead-code analysis. 
%  - potential optimizations to improve the efficiency of generated code
%  - enhancements to improve user experience
%
%  - extensible, modular
%  - design and implementation of a type system
%  - type system extension with physical units/dimensions
%  - tool integration for static analysis and optimization
%  - evaluation
\end{abstract}

% --------------------------------------------------------------------------- %
% ACM Classification
% --------------------------------------------------------------------------- %
%
% ACM publications are classified according to the ACM Computing Classification Scheme (CCS). CCS codes are used both in the typeset version of the publications and in the metadata in the various databases. Therefore you need to provide both TEX commands and XML metadata with the paper.
%
% The code below should be generated by the tool at
% http://dl.acm.org/ccs.cfm
% Please copy and paste the code instead of the example below. 
%
\begin{CCSXML}
 <ccs2012>
  <concept>
   <concept_id>10011007.10011006.10011066.10011070</concept_id>
   <concept_desc>Software and its engineering~Application specific development environments</concept_desc>
   <concept_significance>500</concept_significance>
  </concept>
<concept>
<concept_id>10010147.10010341.10010349.10010355</concept_id>
<concept_desc>Computing methodologies~Agent / discrete models</concept_desc>
<concept_significance>300</concept_significance>
</concept>
<concept>
<concept_id>10010147.10010341.10010366.10010368</concept_id>
<concept_desc>Computing methodologies~Simulation languages</concept_desc>
<concept_significance>300</concept_significance>
</concept>
<concept>
<concept_id>10002950.10003705.10003707</concept_id>
<concept_desc>Mathematics of computing~Solvers</concept_desc>
<concept_significance>100</concept_significance>
</concept>
  </ccs2012>
\end{CCSXML}

\ccsdesc[500]{Software and its engineering~Application specific development environments}
\ccsdesc[300]{Computing methodologies~Agent / discrete models}
\ccsdesc[300]{Computing methodologies~Simulation languages}
\ccsdesc[100]{Mathematics of computing~Solvers}

%
% The command \terms is obsolete; we no longer use “General Terms” line.
%
%\terms{Design, Languages}

% --------------------------------------------------------------------------- %
% Keywords
% --------------------------------------------------------------------------- %
%
% The “Additional Keywords and Phrases” item on the title page is provided by the \keywords declaration, listed alphabetically.
% There is no prescribed list of “additional keywords;” use any that you want.
\keywords{language workbenches, mathematical software, MPS, particle methods, scientific computing}

\begin{bottomstuff}
\textit{Preprint}. This work is partly supported by the German Research Foundation (DFG) within the Cluster of Excellence “Center for Advancing Electronics Dresden” (EXC 1056). 
\end{bottomstuff}
% --------------------------------------------------------------------------- %
% Generate title
% --------------------------------------------------------------------------- %
\maketitle

