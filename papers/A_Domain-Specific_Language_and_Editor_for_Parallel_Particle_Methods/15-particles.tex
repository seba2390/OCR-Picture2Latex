\section{Particle Methods}
\label{sec:particle_methods}
Particle methods provide a universal approach for numerical simulations in scientific computing. 
In contrast to other simulation frameworks, such as finite element methods (FEM) or Monte-Carlo methods,
particle methods can simulate models of all four kinds: discrete/deterministic, discrete/stochastic, 
continuous/deterministic, continuous/stochastic \cite{Sbalzarini:2013}. In case of continuous models, particles correspond to 
discretization points. When discrete models are simulated, entities in a model are directly represented
by particles. In deterministic simulations, particle positions and properties evolve according to deterministic
interactions between particles, whereas in stochastic models, these interactions are probabilistic.

In general, particles are zero-dimensional point-like objects characterized by a collection of properties 
of arbitrary types and a position in any space given as a vector whose length corresponds to the dimension
of that space. While a particle always has a position, its list of properties may grow or shrink in the course of 
a simulation. 
%
As an example of a discrete particle, consider a car on a street. The car's position may correspond to its 
GPS coordinates on a map while its properties could be velocity, the driver's age, number of passengers,  
or the color of the car. Other examples may be a pixel of an image (i.e., in a discrete space) or a discretization point of a continuous mathematical field, 
where the space is continuous.

Particles can \emph{interact} pairwise with other particles and they can \emph{evolve}. Evolving means
that a particle's position and/or properties change due to its own state and/or the states of other particles in the domain.
The influence of other particles is due to the interactions, which may yield a contribution to the change.
Hence, in pseudo code, the essential ingredients of particle objects can be described as shown in 
Figure~\ref{list:particles_structure}.%
{\addtolength{\abovecaptionskip}{-4mm}%
%\addtolength{\belowcaptionskip}{-1pt}%
\begin{figure}[th]%
\begin{lstlisting}[basicstyle=\footnotesize\ttfamily,numbers=none,
keywords={class,method,struct,vector},belowskip=0pt,aboveskip=0pt,
framesep=0pt,rulesep=0pt,frame=none,mathescape=true]
class PARTICLE {
    vector(space-dimension) :: position $\vec{x}$, positionChange $\Delta \vec{x}$
    struct :: properties $\vec{\omega}$, propertiesChange $\Delta \vec{\omega}$
    
    method [vector $\vec{K}_x$, struct $\vec{K}_\omega$]  interact(PARTICLE $q$)
    method evolve()
}
\end{lstlisting}%
\caption{Particle declaration and properties in pseudo code.}%
\label{list:particles_structure}%
\end{figure}%
}
Using this very basic interface, the evolution of a particle may depend on the position vector $\vec{x}$
and the list of properties  $\vec{\omega}$ of the particle itself, as well as the values of all other particles
in the system. If we assume that all particles influence each other, in the most general form, the 
 changes for any particle $p$ in the system can be described abstractly by Eq.~\ref{eq:pmgeneric}:
%
\begin{equation}
\begin{bmatrix} \Delta \vec{x}_p \\ \Delta \vec{\omega}_p \end{bmatrix} = \sum_{q=1}^{N} \begin{bmatrix} \vec{K}_x \\ \vec{K}_\omega \end{bmatrix} = \sum_{q=1}^{N} \vec{K}(\vec{x}_p, \vec{x}_q, \vec{\omega}_p, \vec{\omega}_q)\, .  \label{eq:pmgeneric}
\end{equation}
%
Here, $N$ is the total number of particles in the system, which may change over time (e.g., depending on a
boundary condition). $\vec{K}$ represents the \emph{interaction kernel} that encapsulates the 
computational model and corresponds to a mathematical representation of the \texttt{interact} method. 
\revii{Applied on position vectors $\vec{x}_p, \vec{x}_q$ and properties $\vec{\omega}_p, \vec{\omega}_q$}, the kernel produces the elementary changes $\vec{K}_x$ and $\vec{K}_\omega$ of the pairwise interactions 
between particles $p$ and $q$. The cumulative change for particle $p$ due to all interactions with other
particles is represented by two deltas, $\Delta \vec{x}_p$ and $\Delta \vec{\omega}_p $, which are used
by \texttt{evolve} to update the property values and position of $p$. 

In numerical simulations, updates of particle properties and positions occur at each time step. Thus, it is important
to evaluate the pairwise interactions efficiently. In the worst case, each particle interacts with each other 
particle, which leads to quadratic time complexity. However, typically, a particle only needs to interact with its
``neighbors'' within a finite range. In such cases, optimized data structures such as cell lists~\cite{Hockney_1988} 
exist, which allow for computing particle interactions in linear time (average complexity if particles are 
uniformly distributed). Nevertheless, the worst-case complexity remains quadratic if all particles are located
within the interaction range or the interaction range is the size of the domain. In these cases, efficient approximation algorithms
are available, e.g., the Barnes-Hut algorithm~\cite{Barnes_1986} and Fast Multipole Methods~\cite{Greengard_1987}.
Another way to address this problem is to use a hybrid particle-mesh approach, where interactions
of finite range are computed using particles, whereas interactions of infinite range are evaluated  
using mesh-based approaches~\cite{Hockney_1988}. 

% JC: breaking paragraph - Only about the kernel and discretizations.
The range of the particle--particle interactions is defined by the support of the interaction kernel $\vec{K}$. 
This kernel is the mathematical representation of the system to be simulated and encapsulates all application-specific details. 
When simulating discrete models, $\vec{K}$ corresponds to the pairwise interaction potential between the entities in the model, e.g., the inter-atomic force fields in a molecular-dynamics simulation. When simulating continuous models, such as partial differential equations (PDEs), $\vec{K}$ contains the discretized continuous or differential operators. In this case, the particles as discretization/colocation points at which the value of the continuous function is sampled. 

% On discretizations
Particle discretizations of differential operators in PDEs (i.e., the kernel $\vec{K}$) can be determined using a variety of classical approaches from numerical analysis ~\cite{Lucy_1977,liu1995reproducing,belytschko_1994,lancaster1981surfaces,broomhead1988radial,Degond_1989,Eldredge_2002} that are generic to arbitrary linear differential operators. They all have in common that the kernel $\vec{K}$ is pre-computed, usually analytically by hand, and then implemented in the discrete form. To free the scientific programmer from this analytical calculation, we here implement \revii{a method known as \emph{Discretization-Corrected Particle Strength Exchange (DC-PSE)}}~\cite{Schrader_2010}. 
DC-PSE is a general particle discretization framework where the discrete kernels are automatically computed at runtime. In addition, DC-PSE also shows superior stability and accuracy properties compared to other mesh-free discretization methods \cite{Schrader_2010,Reboux:2012,Schrader:2012,Bourantas:2016}.

% JCM: Not a friend of this sentence. If at all, then put it at the beginning of next section. 
%The next section introduces a classic reaction-diffusion problem and shows how it can be solved using 
%particle methods in PPM(L).
% SK: Why not, actually?
% JC: overdue answer ;-) ... I like connecting sentences, but rather at the beginning of (large) sections.
%    - With connections all over the place (e.g., beginning and end), the paper gets a tutorial-like feeling
%    - Sometime people jump to the start of a section, there the connection helps. At the end of the section, not so much. 

