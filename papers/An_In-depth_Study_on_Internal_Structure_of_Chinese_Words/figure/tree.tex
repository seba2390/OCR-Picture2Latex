\begin{figure}[tb]
\centering
\subfigure[\citet{zhang-etal-2014-char}: labels mark head positions.]
{
\begin{minipage}[c]{1\linewidth}
    \label{fig:zhangs-tree}
    \centering
    \begin{tikzpicture}[
        scale=0.92,
        level distance=22pt,
        sibling distance=10pt,
        every tree node/.style={align=center, anchor=base},
        frontier/.style={distance from root=50pt},
        postag/.style={text opacity=1, rounded corners=1mm, align=center, font=\scriptsize},
        label/.style={text=white, font=\scriptsize},
        gloss/.style={font=\tiny, anchor=base},
        edge from parent/.style={draw,edge from parent path={(\tikzparentnode.south) {[rounded corners=0.8pt]-- ($(\tikzchildnode |- \tikzparentnode.south) + (0, -5pt)$) -- (\tikzchildnode)}}}
        ]
        \begin{scope}[xshift=-100pt]
            \Tree
            [.\node[postag]{coordinate};
                [. \node[postag]{left}; \node(0_0){想}; \node(0_1){方}; ]
                [. \node[postag]{left}; \node(0_2){设}; \node(0_3){法}; ] ];
            \node[label] at ($(0_0) + (+0.0, +1.1)$) {root};
            \node[gloss] at ($(0_0.base) + (0.0, -0.3)$) {think};
            \node[gloss] at ($(0_1.base) + (0.0, -0.3)$) {plan};
            \node[gloss] at ($(0_2.base) + (0.0, -0.3)$) {design};
            \node[gloss] at ($(0_3.base) + (-0.05, -0.3)$) {method};
        \end{scope}

        \Tree
        [.\node[postag]{right};
            [. \node[postag]{coordinate}; \node(1_0){婚}; \node(1_1){姻}; ]
            \node(1_2){法}; ];
        \node[gloss] at ($(1_0.base) + (0.05, -0.3)$) {marriage};
        \node[gloss] at ($(1_1.base) + (0.0, -0.3)$) {marriage};
        \node[gloss] at ($(1_2.base) + (0.0, -0.3)$) {law};
        
        \begin{scope}[xshift=+57pt]
            \Tree [.\node[postag, text=white]{coordinate}; \edge[draw=none]; [.\node[postag]{coordinate}; \node(2_0){法}; \node(2_1){老}; ] ];
            \node[gloss] at ($(2_0.base)!0.5!(2_1.base) + (0.0, -0.3)$) {Pharaoh};
        \end{scope}
        
        \draw[densely dotted] ($(0_3)!0.5!(1_0) + (0, 0.25)$) to ($(0_3)!0.5!(1_0) + (0, 1.35)$);
        \draw[densely dotted] ($(1_2)!0.5!(2_0) + (0, 0.25)$) to ($(1_2)!0.5!(2_0) + (0, 1.35)$);

    \end{tikzpicture}
    \end{minipage}
}\\
\subfigure[\citet{li-etal-aaai-2018-zhaohai}: labels correspond to POS tag triples.]
{
\begin{minipage}[c]{1\linewidth}
\label{fig:zhaos-tree}
    \centering
    \begin{tikzpicture}[
        scale=0.92,
        level distance=10pt,
        sibling distance=10pt,
        every tree node/.style={align=center,anchor=base},
        frontier/.style={distance from root=20pt},
        postag/.style={fill=none, text=white, rounded corners=1mm, align=center},
        label/.style={font=\scriptsize},
        gloss/.style={font=\tiny, anchor=base},
        edge from parent/.style={draw=none,edge from parent path={(\tikzparentnode.south) {[rounded corners=0.8pt]-- ($(\tikzchildnode |- \tikzparentnode.south) + (0, -5pt)$) -- (\tikzchildnode)}}}
        ]
        \begin{scope}[xshift=-100pt]
            \Tree
            [ \node(0_0){想}; \node(0_1){方}; \node(0_2){设}; \node(0_3){法}; ];
            \draw[<-] (0_1.north) to[out=120, in=60] ($(0_0.north) + (+0.1, 0) $);
            \draw[<-] (0_2.north) to[out=120, in=60] ($(0_0.north) + (+0.1, 0) $);
            \draw[<-] (0_0.north) to ($(0_0) + (0, +0.9)$);
            \draw[<-] (0_3.north) to[out=120, in=60] ($(0_2.north) + (+0.1, 0) $);
            \node[label] at ($(0_0) + (+0.0, +1.1)$) {root};
            \node[label] at ($(0_1) + (+0.2, +0.55)$) {v-v-n};
            \node[label] at ($(0_1) + (+0.0, +0.9)$) {v-v-v};
            \node[label] at ($(0_3) + (-0.48, +0.65)$) {v-v-n};
            \node[gloss] at ($(0_0.base) + (0.0, -0.3)$) {think};
            \node[gloss] at ($(0_1.base) + (0.0, -0.3)$) {plan};
            \node[gloss] at ($(0_2.base) + (0.0, -0.3)$) {design};
            \node[gloss] at ($(0_3.base) + (-0.05, -0.3)$) {method};
        \end{scope}

        \Tree
        [ [\node(1_0){婚}; \node(1_1){姻}; ] \node(1_2){法}; ];
        \draw[<-] (1_0.north) to[out=60, in=120] ($(1_1.north) + (-0.1, 0) $);
        \draw[<-] (1_1.north) to[out=60, in=120] ($(1_2.north) + (-0.1, 0) $);
        \draw[<-] (1_2.north) to ($(1_2) + (0, +0.9)$);
        \node[label] at ($(1_1) + (-0.55, +0.65)$) {n-n-n};
        \node[label] at ($(1_2) + (-0.55, +0.65)$) {n-n-n};
        \node[label] at ($(1_2) + (+0.0, +1.1)$) {root};
        \node[gloss] at ($(1_0.base) + (0.05, -0.3)$) {marriage};
        \node[gloss] at ($(1_1.base) + (0.0, -0.3)$) {marriage};
        \node[gloss] at ($(1_2.base) + (0.0, -0.3)$) {law};
        
        \begin{scope}[xshift=+57pt]
            \Tree [ \node(2_0){法}; \node(2_1){老}; ];
            \draw[<-] (2_0.north) to[out=60, in=120] ($(2_1.north) + (-0.1, 0) $);
            \draw[<-] (2_1.north) to ($(2_1) + (0, +0.9)$);
            \node[label] at ($(2_1) + (-0.55, +0.65)$) {n-n-n};
            \node[label] at ($(2_1) + (+0.0, +1.1)$) {root};
            \node[gloss] at ($(2_0.base)!0.5!(2_1.base) + (0.0, -0.3)$) {Pharaoh}; 
        \end{scope}

        \draw[densely dotted] ($(0_3)!0.5!(1_0) + (0, 0.25)$) to ($(0_3)!0.5!(1_0) + (0, 0.95)$);
        \draw[densely dotted] ($(1_2)!0.5!(2_0) + (0, 0.25)$) to ($(1_2)!0.5!(2_0) + (0, 0.95)$);
        
    \end{tikzpicture}
    \end{minipage}
}\\
\subfigure[Ours: fine-grained structure with 11 labels.]
{
\begin{minipage}[c]{1\linewidth}

\label{fig:ours-tree}
    \centering
    \begin{tikzpicture}[
        scale=0.92,
        level distance=10pt,
        sibling distance=10pt,
        every tree node/.style={align=center,anchor=base},
        frontier/.style={distance from root=20pt},
        postag/.style={fill=none, text=white, rounded corners=1mm, align=center},
        label/.style={font=\scriptsize},
        gloss/.style={font=\tiny, anchor=base},
        edge from parent/.style={draw=none,edge from parent path={(\tikzparentnode.south) {[rounded corners=0.8pt]-- ($(\tikzchildnode |- \tikzparentnode.south) + (0, -5pt)$) -- (\tikzchildnode)}}}
        ]
        \begin{scope}[xshift=-100pt]
            \Tree
            [ \node(0_0){想}; \node(0_1){方}; \node(0_2){设}; \node(0_3){法}; ];
            \draw[<-] (0_1.north) to[out=120, in=60] ($(0_0.north) + (+0.1, 0) $);
            \draw[<-] (0_2.north) to[out=120, in=60] ($(0_0.north) + (+0.1, 0) $);
            \draw[<-] (0_0.north) to ($(0_0) + (0, +0.9)$);
            \draw[<-] (0_3.north) to[out=120, in=60] ($(0_2.north) + (+0.1, 0) $);
            \node[label] at ($(0_0) + (+0.0, +1.1)$) {root};
            \node[label] at ($(0_1) + (+0.1, +0.55)$) {obj};
            \node[label] at ($(0_1) + (+0.0, +0.90)$) {coo};
            \node[label] at ($(0_3) + (-0.48, +0.7)$) {obj};
            \node[gloss] at ($(0_0.base) + (0.0, -0.3)$) {think};
            \node[gloss] at ($(0_1.base) + (0.0, -0.3)$) {plan};
            \node[gloss] at ($(0_2.base) + (0.0, -0.3)$) {design};
            \node[gloss] at ($(0_3.base) + (-0.05, -0.3)$) {method};
        \end{scope}

        \Tree
        [ [ \node(1_0){婚}; \node(1_1){姻}; ] \node(1_2){法}; ];
        \draw[<-] (1_0.north) to[out=60, in=120] ($(1_2.north) + (-0.1, 0) $);
        \draw[<-] (1_1.north) to[out=120, in=60] ($(1_0.north) + (+0.1, 0) $);
        \draw[<-] (1_2.north) to ($(1_2) + (0, +0.9)$);
        \node[label] at ($(1_1) + (+0.1, +0.55)$) {coo};
        \node[label] at ($(1_1) + (-0.05, +0.91)$) {att};
        \node[label] at ($(1_2) + (+0.0, +1.1)$) {root};
        \node[gloss] at ($(1_0.base) + (0.05, -0.3)$) {marriage};
        \node[gloss] at ($(1_1.base) + (0.0, -0.3)$) {marriage};
        \node[gloss] at ($(1_2.base) + (0.0, -0.3)$) {law};

        \begin{scope}[xshift=+57pt]
            \Tree [ \node(2_0){法}; \node(2_1){老}; ];
            \draw[<-] (2_1.north) to[out=120, in=60] ($(2_0.north) + (+0.1, 0) $);
            \draw[<-] (2_0.north) to ($(2_0) + (0, +0.9)$);
            \node[label] at ($(2_1) + (-0.48, +0.70)$) {frag};
            \node[label] at ($(2_0) + (+0.0, +1.1)$) {root};
            \node[gloss] at ($(2_0.base)!0.5!(2_1.base) + (0.0, -0.3)$) {Pharaoh}; 
        \end{scope}
        
        \draw[densely dotted] ($(0_3)!0.5!(1_0) + (0, 0.25)$) to ($(0_3)!0.5!(1_0) + (0, 0.95)$);
        \draw[densely dotted] ($(1_2)!0.5!(2_0) + (0, 0.25)$) to ($(1_2)!0.5!(2_0) + (0, 0.95)$);
        
    \end{tikzpicture}
    \end{minipage}
}
\caption{
Three example words with internal structure under different annotation paradigms. ``想(think of) 方(plan) 设(design) 法(method)'' is a verb and means ``find ways or means to do''. ``婚(marriage) 姻(marriage) 法(law)'' is a noun. ``法老'' is phonetic transliteration of ``Pharaoh''. The three words all contain the character ``法'' under different meanings. 
}
\label{fig:example}

\end{figure}

