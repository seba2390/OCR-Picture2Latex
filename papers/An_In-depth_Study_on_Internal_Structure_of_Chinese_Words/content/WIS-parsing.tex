% \section{Char-level Syntactic Parsing for Chinese Words}\label{char-pasrsing}

\section{Word-internal Structure Parsing}\label{sec:WIS-parsing}

% 【词内部结构parsing叫char-level,句法parsing叫sentence-level好像不合适】

With annotated WIST, we try to address the second question: can we train a model to predict word-internal structure? We adapt the Biaffine parser proposed by \citet{dozat2016deep}, a widely used sentence-level dependency parser, for this purpose, and present results and analysis.  

\subsection{Biaffine Parser}

%To produce syntactic char-level structures for Chinese words, we build a word-internal structure parsing model based on the state-of-the-art biaffine parser of \citet{dozat2016deep} with two modifications to accomodate our task, i.e., replacing the word and POS tag embeddings with the character embeddings for char-level representation, 
% replacing the original non-projective MST algorithm with first-order Eisner algorithm \cite{Eisner2000} for projective decoding.
\begin{figure}[tb]
    \centering
    \begin{tikzpicture}[
        connect/.style={
                rounded corners=4pt,
                semithick,
                draw=black!80
            },
        arrow/.style={
                % >=latex,
                arrows = {-Straight Barb[length=0.5mm]},
                shorten >= 2pt,
                shorten <= 1.5pt,
                thin
            },
        inner arrow/.style={
                arrows = {-Straight Barb[length=0.4mm]},
                shorten >= 2pt,
                shorten <= 2pt,
                thin,
                draw=black!50
            },
        input/.style={
                rectangle,
                rounded corners=1mm,
                thin,
                dashed,
                draw=none,
                minimum width=3.5cm,
                minimum height=0.6cm,
            },
        share/.style={
                minimum height=0.5cm,
                fill={rgb,255:red,230; green,197; blue,180},
                % fill={rgb,255:red,240; green,245; blue,229},
                fill opacity=0.6,
                text opacity=1.0,
                draw=black,
                % thick,
                rounded corners=2mm,
            },
        task2/.style={
                minimum height=0.5cm,
                fill={rgb,255:red,251; green,153; blue,104},
                fill opacity=0.6,
                text opacity=1.0,
                draw=black,
                rounded corners=2mm,
            },
        label/.style={
                inner sep=0.5mm,
                fill=white,
                minimum height=0.5cm,
            },
        task1/.style={
                minimum height=0.5cm,
                fill={rgb,255:red,117; green,178; blue,231},
                draw=black,
                rounded corners=2mm
            },
        inner lstm/.style={
                fill=white,
                rectangle,
                rounded corners=1mm,
                semithick,
                % thin,
                draw=black!50,
                fill opacity=0.8
            },
        cell/.style={
                inner sep=2mm,
                rectangle,
                rounded corners=1mm,
                semithick,
                draw=black!50,
            },
        ocell/.style={
                solid,
                minimum height=0.5cm,
                rectangle,
                rounded corners=1mm,
                thick
            },
        dep arrow/.style={
        arrows = {-Latex[round,open,length=8pt,width=6pt]},
        shorten >= 2pt,
        shorten <= 1.5pt,
        thick
        }
        ]
        \centering
        \node [input, inner sep=1pt] [minimum width=4.8cm] (input) at (0, 0) {$\ldots\;\; \mathbf{x}_i \;\;\ldots\;\; \mathbf{x}_j \;\;\ldots$};
        % Concat
        \node [inner sep=0] (EmbedCat) at ($(input.north)$) {};
    
        %\scriptsize BiLSTM
        \node [share, ocell] [minimum width=7cm, minimum height=0.675cm, anchor=south] (lstm) at ($(input.north) + (0, 0.3cm)$) {\scriptsize $\mathrm{BiLSTM}\, \times \, 3$};
    

        \draw [arrow, connect] ($(EmbedCat.north) + (0cm, -0.15cm)$) -- ($(lstm.south) + (0cm, 0)$);


        % fisrt two mlps
        \node [share, fill={rgb,255:red,245; green,245; blue,245}, semithick, draw=gray] [minimum width=7cm, minimum height=0.775cm, anchor=south] (share-mlp) at ($(lstm.north) + (0cm, 0.455cm)$) {};
        

        \node [task2, ocell] [minimum width=1.625cm, minimum height=0.585cm, anchor=south west, fill opacity=0.85] (mlp-1-1-f) at ($(lstm.north west) + (0.1cm, 0.55cm)$) {};
        \node [task2, ocell] [minimum width=1.625cm, minimum height=0.585cm, anchor=south west, fill opacity=0.85] (mlp-1-1-b) at ($(lstm.north west) + (1.825cm, 0.55cm)$) {};
        \node [anchor=base] at  ($(mlp-1-1-f.south) + (0, 0.2cm)$)  {\scriptsize $\mathrm{MLP}^{h}$};
        \node [anchor=base] at  ($(mlp-1-1-b.south) + (0, 0.2cm)$)  {\scriptsize $\mathrm{MLP}^{d}$};
    
        \node [task2, ocell, fill=none, draw=none] [minimum width=1.75cm, minimum height=0.40cm, anchor=south east, fill opacity=0.5, draw opacity=0.6] (mlp-1-2-b) at ($(mlp-1-1-b.north east) + (0, 0.5cm)$) {};
    
        \node [task2, ocell, anchor=south, fill opacity=0.85, minimum height=1.29cm, minimum width=1.29cm, align=center] (arc-biaffine) at ($(mlp-1-1-f.north)!0.5!(mlp-1-1-b.north) + (0, 0.73cm)$) {\scriptsize $\mathrm{Biaffine}$};
    
        \draw [arrow, connect] ($(mlp-1-1-b.north) + (0, 0.06cm)$) -- ($(mlp-1-1-b.north) + (0, 0.35cm)$) -- ($(arc-biaffine.south) + (0.2, 0)$);
        \draw [arrow, connect] ($(mlp-1-1-f.north) + (0, 0.06cm)$) -- ($(mlp-1-1-f.north) + (0, 0.35cm)$) -- ($(arc-biaffine.south) + (-0.2, 0)$);
    
        % second two mlps
        \node [task1, ocell] [minimum width=1.625cm, minimum height=0.585cm, anchor=south east, fill opacity=0.85] (mlp-2-1-f) at ($(lstm.north east) + (-1.825cm, 0.55cm)$) {};
        \node [task1, ocell] [minimum width=1.625cm, minimum height=0.585cm, anchor=south east, fill opacity=0.85] (mlp-2-1-b) at ($(lstm.north east) + (-0.1cm, 0.55cm)$) {};
        \node [anchor=base] at  ($(mlp-2-1-f.south) + (0, 0.2cm)$)  {\scriptsize $\mathrm{MLP}^{h'}$};
        \node [anchor=base] at  ($(mlp-2-1-b.south) + (0, 0.2cm)$)  {\scriptsize $\mathrm{MLP}^{d'}$};
    
        \node [anchor=south, minimum height=1.29cm, minimum width=1.29cm, draw=none] (label-biaffine-invis) at ($(mlp-2-1-f.north)!0.5!(mlp-2-1-b.north) + (0, 0.73)$) {};
    
        \filldraw[task1, ocell, fill={rgb,255:red,235; green,235; blue,235}, draw=none, rounded corners=0.5mm] ($(mlp-2-1-f.north)!0.5!(mlp-2-1-b.north) + (-0.875, 0.5)$) -- ($(mlp-2-1-f.north)!0.5!(mlp-2-1-b.north) + (0.415, 0.5)$) -- ($(mlp-2-1-f.north)!0.5!(mlp-2-1-b.north) + (0.875, 0.96)$) -- ($(mlp-2-1-f.north)!0.5!(mlp-2-1-b.north) + (0.875, 2.25)$) -- ($(mlp-2-1-f.north)!0.5!(mlp-2-1-b.north) + (-0.415, 2.25)$) -- ($(mlp-2-1-f.north)!0.5!(mlp-2-1-b.north) + (-0.875, 1.79)$) -- cycle;
    
        \draw [arrow, connect] ($(mlp-2-1-b.north) + (0, 0.06cm)$) -- ($(mlp-2-1-b.north) + (0, 0.35cm)$) -- ($(label-biaffine-invis.south) + (0.2, 0)$);
        \draw [arrow, connect] ($(mlp-2-1-f.north) + (0, 0.06cm)$) -- ($(mlp-2-1-f.north) + (0, 0.35cm)$) -- ($(label-biaffine-invis.south) + (-0.2, 0)$);
    
        \filldraw[task1, ocell, fill=none, draw=lightgray, rounded corners=0.5mm] ($(mlp-2-1-f.north)!0.5!(mlp-2-1-b.north) + (-0.875, 0.5)$) -- ($(mlp-2-1-f.north)!0.5!(mlp-2-1-b.north) + (0.415, 0.5)$) -- ($(mlp-2-1-f.north)!0.5!(mlp-2-1-b.north) + (0.875, 0.96)$) -- ($(mlp-2-1-f.north)!0.5!(mlp-2-1-b.north) + (0.875, 2.25)$) -- ($(mlp-2-1-f.north)!0.5!(mlp-2-1-b.north) + (-0.415, 2.25)$) -- ($(mlp-2-1-f.north)!0.5!(mlp-2-1-b.north) + (-0.875, 1.79)$) -- cycle;
    
        \node [anchor=south west, minimum height=1.29cm, minimum width=1.29cm, draw=none] (label-biaffine-f) at ($(mlp-2-1-f.north)!0.5!(mlp-2-1-b.north) + (-0.875, 0.5)$) {};

        \node [anchor=north east, minimum height=1.29cm, minimum width=1.29cm, draw=none] (label-biaffine-b) at ($(mlp-2-1-f.north)!0.5!(mlp-2-1-b.north) + (0.875, 2.25)$) {};  

        \node [fill={rgb,255:red,117; green,178; blue,231}, fill opacity=0.3, minimum height=1.21cm, minimum width=1.21cm, draw=black!80, rounded corners=0.5mm, align=center, inner sep=0.1mm] at ($(label-biaffine-f)!1.0!(label-biaffine-b)$) {};
    
        \node [fill={rgb,255:red,117; green,178; blue,231}, fill opacity=0.4, minimum height=1.21cm, minimum width=1.21cm, draw=black!80, rounded corners=0.5mm, align=center, inner sep=0.1mm] at ($(label-biaffine-f)!0.8!(label-biaffine-b)$) {};
    
        \node [fill={rgb,255:red,117; green,178; blue,231}, fill opacity=0.5, minimum height=1.21cm, minimum width=1.21cm, draw=black!80, rounded corners=0.5mm, align=center, inner sep=0.1mm] at ($(label-biaffine-f)!0.57!(label-biaffine-b)$) {};

        \node [fill={rgb,255:red,117; green,178; blue,231}, fill opacity=0.7, minimum height=1.21cm, minimum width=1.21cm, draw=black!80, rounded corners=0.5mm, align=center, inner sep=0.1mm] at ($(label-biaffine-f)!0.3!(label-biaffine-b)$) {};
        
        \node [fill={rgb,255:red,117; green,178; blue,231}, fill opacity=0.9, minimum height=1.21cm, minimum width=1.21cm, draw=black!80, rounded corners=0.5mm, align=center, inner sep=0.1mm] at ($(label-biaffine-f)!0.0!(label-biaffine-b)$) {\scriptsize $\mathrm{Biaffines}$};
    
        %  LSTM -> MLP
        \draw [arrow, connect, rounded corners=1.2pt, shorten >= 2pt] ($(lstm.north) + (-1.725cm, 0)$) -- ($(share-mlp.south) + (-1.725cm, 0)$);
        \draw [arrow, connect, rounded corners=1.2pt, shorten >= 2pt] ($(lstm.north) + (1.725cm, 0)$) -- ($(share-mlp.south) + (1.725cm, 0)$);
        \node[anchor=base] at ($(share-mlp.south) + (-2cm,-0.32cm) $) {$\mathbf{h}_{i}$};
        \node[anchor=base] at ($(share-mlp.south) + (1.45cm,-0.32cm) $) {$\mathbf{h}_{j}$};
    
    
        \node[anchor=base] at ($(share-mlp.north) + (-3.0cm,0.18cm) $) {$\mathbf{r}_{i}^{h}$};
        \node[anchor=base] at ($(share-mlp.north) + (-0.5cm,0.18cm) $) {$\mathbf{r}_{j}^{d}$};
        \node[anchor=base] at ($(share-mlp.north) + (0.5cm,0.18cm) $) {$\mathbf{r}_{i}^{h'}$};
        \node[anchor=base] at ($(share-mlp.north) + (3.0cm,0.18cm) $) {$\mathbf{r}_{j}^{d'}$};
    
    
        \node[anchor=base] at ($(arc-biaffine.north) + (0cm,0.5cm) $) (arc-biaffine-label) {$\mathrm{score}(i \rightarrow j)$};
        \node[anchor=base] at ($(label-biaffine-invis.north) + (0,0.5cm) $) (label-biaffine-label) {$\mathrm{score}(i \xrightarrow{l} j)$};
    
    
        \draw [arrow, connect, rounded corners=1.2pt, shorten >= 2pt] ($(arc-biaffine-label.south) + (0, -0.25cm) $) -- ($(arc-biaffine-label.south) + (0, 0.15cm) $);
        \draw [arrow, connect, rounded corners=1.2pt, shorten >= 2pt, line cap=round] ($(label-biaffine-label.south) + (0, -0.25cm) $) -- ($(label-biaffine-label.south) + (0, 0.15cm) $);
    
    \end{tikzpicture}
    
    \caption{
      The basic architecture of Biaffine Parser. 
    }
    \label{fig:biaffine-parser}
\end{figure}
We adopt the SuPar implementation released by  \citet{zhang-etal-2020-dep}.\footnote{\url{https://github.com/yzhangcs/parser}}  
As a graph-based parser, Biaffine parser casts a tree parsing task as searching for a maximum-scoring tree from a fully-connected graph, with nodes corresponding to characters in our case. As shown in Figure \ref{fig:biaffine-parser}, it adopts standard encoder-decoder architecture, consisting of the following components.  
%Biaffine parser is a graph-based dependency parser which employs a deep biaffine scoring architecture to compute the score of each dependency and aims to find the highest-scoring dependency tree.
%\subsection{Model Architecture}
% The model architecture is shown in Figure xxx, consisting of following four components. 
% it consists of three major layers. 

\textbf{Input layer.} 
Given an input sequence, 
%consisting of $n$  $c_0c_1...c_n$, where $c_i$ is the $i$-th character and $c_0$ is a pseudo character used as tree root. 
each item is represented as a dense vector $\mathbf{x}_i$. 
For word-internal structure parsing, an item corresponds to a character, and we use char embedding. 
%the char representation output by BERT. 
%using the character embeddings $\mathbf{emb}^{c}$:
\begin{equation}
\mathbf{x}_i = \mathbf{emb}(c_i) 
% \oplus \mathbf{emb}^{bc}(c_{i-1}c_i)
\end{equation}

\textbf{BiLSTM encoder.} 
Then, a three-layer BiLSTM is applied to obtain  context-aware representations. 
We denote the hidden vector of the top-layer BiLSTM for  the i-th position as $\mathbf{h}_i$.

\textbf{Biaffine scorer.} %MLP feature abstraction.} 
Two separate MLPs are applied to each $\mathbf{h}_i$, resulting in 
%fed into two separate MLPs to 
%producing 
two lower-dimensional vectors $ \mathbf{r}_{i}^{h}$ (as head) and $\mathbf{r}_{i}^{d}$ (as dependent).
Then the score of a dependency $i \rightarrow j$ is obtained via a biaffine attention over $\mathbf{r}_{i}^{h}$ and $\mathbf{r}_{j}^{d}$. 
%To save space, we omit the 
Scoring of labeled dependencies such as $i \xrightarrow{l} j$ is analogous. 

%, representing  as a head and a dependent respectively. 
% \begin{equation}
%     \mathbf{r}_{i}^{h} = {\textup{MLP}}^{h} \left(\mathbf{h}_i \right) ;
%     \mathbf{r}_{i}^{d} = {\textup{MLP}}^{d} \left(\mathbf{h}_i \right)
% \end{equation} 
%\textbf{Biaffine classifier.} 
%The biaffine classifier computes 
% \begin{equation} \label{eq:biaffine}
%  \texttt{s}(i \leftarrow j) =  \left[
%  \begin{array}{c}
%     \mathbf{r}_{i}^{d}    \\
%       1 
%  \end{array} 
%  \right]^\mathrm{T}  \mathbf{W}^{biaffine}  \mathbf{r}_{j}^{h} 
% \end{equation} 
% where ${\mathbf{W}^{biaffine}}$ is a biaffine parameter.

% Analogously, the parser uses extra MLPs and a biaffine attention to compute scores for dependency labels. Please refer to \citet{dozat2016deep} for details.

\textbf{Decoder. }
%During decoding, the biaffine parser treats unlabeled dependency tree searching and dependency labeling as two cascaded tasks.
% First, with the scores of all dependencies, the score of a unlabeled dependency tree $\bm{y}$ is:
% \begin{equation}
% \begin{split}
% \texttt{s}(\bm{y})= \sum_{{{i \leftarrow j} \in \bm{y}}}\texttt{s}(i \leftarrow j)
% \end{split}
% \end{equation}
With the scores of all dependencies, 
we adopt the first-order algorithm of \citet{Eisner2000} % for projective decoding 
to find the optimal unlabeled dependency tree, and then independently decide the highest-scoring label for each arc. 
%$\bm{y}^{*}$ with the highest score from all the possible trees.
%Then, the biaffine parser finds the highest-scoring label for each dependency of the optimal tree $\bm{y}^{*}$.


\textbf{Training loss.} %with Local Char-wise Loss}
During training, the parser computes two independent cross-entropy losses for each position, i.e., maximizing the probability of its correct head and
%, and maximizing the probability of 
the correct label between them. 
%$c_j$ and the corresponding correct dependency label $l$ for each character $c_i$ independently.
% \begin{equation}
% \begin{split}
%  \texttt{loss}(i \xleftarrow{l} j) = &-\log{ 
% \frac{e^{\texttt{s}(i \leftarrow j)}}
% {\sum\limits_{0 \le k \le n, k \neq i}e^{\texttt{s}(i \leftarrow k)}} 
% }\\
% & -\log{ 
% \frac{e^{\texttt{s}(i \xleftarrow{l} j)}}
% {\sum\limits_{l' \in \mathcal{L}}e^{\texttt{s}(i \xleftarrow{l'} k)}} 
% }
% \end{split}
% \end{equation}
% where $\texttt{s}(i \xleftarrow{l} j)$ is the score of labeling the dependency $i \leftarrow j$ as $l$. $\mathcal{L}$ is the whole dependency label set.




