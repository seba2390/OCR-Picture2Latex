\section{Introduction}\label{sec:intro}

% 赛豪得试一下:pre-trained char embedding是否有用?

% 分析部分:左弧、右弧的比例
% 单字、两个字、三个字、四个字及以上,在dict中的比例分布;在CTB语料中的比例分布

% 两个字的单词,左弧右弧的比例
% 三个字的单词,左弧右弧的比例
% 四个字及以上的单词,左弧右弧的比例

% 三字及以上,波浪形的比例(厚全已经分析了)

% 统一一下:
% multi-morpheme word; 也有说multi-morphemic  word,但是没有说 single-morphemic word)
% single-morpheme word
% single morpheme. 

% 建模问题

Unlike English, 
% in which characters represent sound directly, 
Chinese adopts a logographic writing system and contains tens of thousands of distinct characters. 
Many characters, especially frequently used ones, have rich and specific meanings. % and have their clear syntactic roles/function.  
% There are about 10K characters 

However, words, instead of characters, are often considered as the basic unit in processing Chinese texts. We believe the reason may be two-fold. First, usually a character may have many meanings and usages. Word formation process greatly reduces such char-level ambiguity. Second, by definition, words are the minimal units that express a complete semantic concept or play a grammatical role independently \cite{ctb-xiafei,yu2003ppd}.\footnote{There 
is still a dispute on the word granularity issue \cite{gong-2017-multi,naacl21-latticebert}. Words are defined as a character sequence that is in tight and steady combination. However, the combination intensity is usually yet vaguely qualified according to co-occurrence frequency. We believe this work may also be potentially useful to %contribute to 
this direction.
%study on word granularity. 
% 可能related work部分没有地方写了
%as discussed in Section \ref{sec:rel-work}.
}
\section{How do generative models work? How do GANs compare to others?}
\label{sec:tree}

We now have some idea of what generative models can do and why it might be
desirable to build one.
Now we can ask: how does a generative model actually work? And in particular,
how does a GAN work, in comparison to other generative models?

\subsection{Maximum likelihood estimation}

To simplify the discussion somewhat, we will focus on generative models
that work via the principle of \newterm{maximum likelihood}.
Not every generative model uses maximum likelihood.
Some generative models do not use maximum likelihood by default, but
can be made to do so (GANs fall into this category).
By ignoring those models that do not use maximum likelihood, and
by focusing on the maximum likelihood version of models that do not
usually use maximum likelihood, we can eliminate some of the more 
distracting differences between different models.

The basic idea of maximum likelihood is to define a model that provides
an estimate of a probability distribution, parameterized by parameters
$\vtheta$.
We then refer to the \newterm{likelihood} as the probability that the model
assigns to the training data: $\prod_{i=1}^m \pmodel\left(\vx^{(i)}; \vtheta \right),$
for a dataset containing $m$ training examples $\vx^{(i)}$.

The principle of maximum likelihood simply says to choose the parameters for the model
that maximize the likelihood of the training data.
This is easiest to do in log space, where we have a sum rather than a product
over examples.
This sum simplifies the algebraic expressions for the derivatives of the likelihood
with respect to the models, and when implemented on a digital computer, is less
prone to numerical problems, such as underflow resulting from multiplying together
several very small probabilities.

\begin{align}
\vtheta^* =& \argmax_\vtheta \prod_{i=1}^m \pmodel\left(\vx^{(i)}; \vtheta \right) \\
  =& \argmax_\vtheta \log \prod_{i=1}^m \pmodel\left(\vx^{(i)}; \vtheta \right) \label{eq:log} \\
          =& \argmax_\vtheta \sum_{i=1}^m \log \pmodel\left(\vx^{(i)}; \vtheta \right).
\end{align}

In \eqref{eq:log}, we have used the property that $\argmax_v f(v) = \argmax_v \log f(v)$ for 
positive $v$, because the logarithm is a function that increases everywhere and does not change
the location of the maximum.

The maximum likelihood process is illustrated in \figref{fig:mle}.

We can also think of maximum likelihood estimation as minimizing the
\newterm{KL divergence} between the data generating distribution and the
model:
\begin{equation}
\vtheta^* = \argmin_\vtheta \KL\left( \pdata(\vx) \Vert \pmodel(\vx ; \vtheta) \right).
\label{eq:kl}
\end{equation}
If we were able to do this precisely, then if $\pdata$ lies within the family of distributions
$\pmodel(\vx ; \vtheta)$, the model would recover $\pdata$ exactly.
In practice, we do not have access to $\pdata$ itself, but only to a training set
consisting of $m$ samples from $\pdata$.
We uses these to define $\ptrain$, an \newterm{empirical distribution} that places mass only
on exactly those $m$ points, approximating $\pdata$.
Minimizing the KL divergence between $\ptrain$ and $\pmodel$ is exactly equivalent to maximizing
the log-likelihood of the training set.

\begin{figure}
\centering
\includegraphics[width=\textwidth]{mle.pdf}
\caption{The maximum likelihood process consists of taking several samples from
  the data generating distribution to form a training set, then pushing up on the
  probability the model assigns to those points, in order to maximize the likelihood
  of the training data.
  This illustration shows how different data points push up on different parts of
  the density function for a Gaussian model applied to 1-D data.
  The fact that the density function must sum to $1$ means that we cannot simply
  assign infinite likelihood to all points; as one point pushes up in one place
  it inevitably pulls down in other places.
  The resulting density function balances out the upward forces from all the data
  points in different locations.
}
\label{fig:mle}
\end{figure}

For more information on maximum likelihood and other statistical estimators,
see chapter 5 of \citet{Goodfellow-et-al-2016}.


\subsection{A taxonomy of deep generative models}

If we restrict our attention to deep generative models that work by maximizing
the likelihood, we can compare several models by contrasting the ways that they
compute either the likelihood and its gradients, or approximations to these
quantities.
As mentioned earlier, many of these models are often used with principles other
than maximum likelihood, but we can examine the maximum likelihood variant of
each of them in order to reduce the amount of distracting differences between
the methods.
Following this approach, we construct the taxonomy shown in \figref{fig:tree}.
Every leaf in this taxonomic tree has some advantages and disadvantages.
GANs were designed to avoid many of the disadvantages present in pre-existing
nodes of the tree, but also introduced some new disadvantages.

\begin{figure}
\centering
\includegraphics[width=\textwidth]{fuck_arxiv_tree.pdf}
\caption{
Deep generative models that can learn via the principle of maximim likelihood
differ with respect to how they represent or approximate the likelihood.
On the left branch of this taxonomic tree, models construct an explicit density,
$\pmodel(\vx; \vtheta)$, and thus an explicit likelihood which can be maximized.
Among these explicit density models, the density may be computationally tractable,
or it may be intractable, meaning that to maximize the likelihood it is necessary
to make either variatioanl approximations or Monte
Carlo approximations (or both).
On the right branch of the tree, the model does not explicitly represent a
probability distribution over the space where the data lies.
Instead, the model provides some way of interacting less directly with this
probability distribution.
Typically the indirect means of interacting with the probability distribution is
the ability to draw samples from it.
Some of these implicit models that offer the ability to sample from the distribution
do so using a Markov Chain; the model defines a way to stochastically transform
an existing sample in order to obtain another sample from the same distribution.
Others are able to generate a sample in a single step, starting without any input.
While the models used for GANs can sometimes be constructed to define an explicit
density, the training algorithm for GANs makes use only of the model's ability to
generate samples.
GANs are thus trained using the strategy from the rightmost leaf of the tree:
using an implicit model that samples directly from the distribution represented
by the model.
}
\label{fig:tree}
\end{figure}

\subsection{Explicit density models}

In the left branch of the taxonomy shown in \figref{fig:tree} are models that define
an explicit density function $\pmodel(\vx ; \vtheta)$.
For these models, maxmimization of the likelihood is straightforward; we simply plug
the model's definition of the density function into the expression for the likelihood,
and follow the gradient uphill.

The main difficulty present in explicit density models is designing a model that can
capture all of the complexity of the data to be generated 
while still maintaining computational tractability.
There are two different strategies used to confront this challenge:
(1) careful construction of models whose structure guarantees their tractability,
as described in \secref{sec:explicit_tractable},
and (2) models that admit tractable approximations to the likelihood and its
gradients, as described in \secref{sec:approx}.

\subsubsection{Tractable explicit models}
\label{sec:explicit_tractable}

In the leftmost leaf of the taxonomic tree of \figref{fig:tree} are the models
that define an explicit density function that is computationally tractable.
There are currently two popular approaches to tractable explicit density models:
fully visible belief networks and nonlinear independent components analysis.

\paragraph{Fully visible belief networks}
\newterm{Fully visible belief networks} \citep{Frey96,Frey98} or FVBNs are models that use the chain
rule of probability to decompose a probability distribution over an $n$-dimensional vector $\vx$
into a product of one-dimensional probability distributions:
\[
\pmodel(\vx) = \prod_{i=1}^n \pmodel\left(\evx_i \mid \evx_1, \dots, \evx_{i-1} \right).
\]
FVBNs are, as of this writing, one of the three most popular
approaches to generative modeling, alongside GANs and variational autoencoders.
They form the basis for sophisticated generative models from DeepMind, such as
WaveNet \citep{aaron-wavenet-2016}. WaveNet is able to generate realistic human speech.
The main drawback of FVBNs is that samples must be generated
one entry at a time: first $\evx_1$, then $\evx_2$, etc., so the cost of generating
a sample is $O(n)$.
In modern FVBNs such as WaveNet, the distribution over each $\evx_i$ is computed by a deep
neural network, so each of these $n$ steps involves a nontrivial amount of computation.
Moreover, these steps cannot be parallelized.
WaveNet thus requires two minutes of computation time to generate one second of audio,
and cannot yet be used for interactive conversations.
GANs were designed to be able to generate all of $\vx$ in parallel, yielding greater
generation speed.

\paragraph{Nonlinear independent components analysis}
Another family of deep generative models with explicit density functions is based on
defining continuous, nonlinear transformations between two different spaces.
For example, if there is a vector of latent variables $\vz$ and a continuous, differentiable,
invertible transformation
$g$ such that $g(\vz)$ yields a sample from the model in $\vx$ space,
then
\begin{equation}
  \label{eq:change-of-variable}
  p_x(\vx) = p_z(g^{-1}(\vx)) \left| \mathrm{det}
  \left( \frac{\partial g^{-1}(\vx)} {\partial \vx}\right) \right|. 
\end{equation}
The density $p_x$ is tractable if the density $p_z$ is tractable
and the determinant of the Jacobian of $g^{-1}$ is tractable.
In other words, a simple distribution over $\vz$ combined with
a transformation $g$ that warps space in complicated ways can yield
a complicated distribution over $\vx$, and if $g$ is carefully designed,
the density is tractable too.
Models with nonlinear $g$ functions date back at least to
~\citet{deco1995higher}.
The latest member of this family is real NVP \citep{dinh2016density}.
See \figref{fig:nvp} for some visualizations of ImageNet samples
generated by real NVP.
The main drawback to nonlinear ICA models is that they impose restrictions
on the choice of the function $g$. In particular, the invertibility requirement
means that the latent variables $\vz$ must have the same dimensionality as $\vx$.
GANs were designed to impose very few requirements on $g$, and, in particular,
admit the use of $\vz$ with larger dimension than $\vx$.

\begin{figure}
\centering
\includegraphics[width=\textwidth]{fig_imnet_64_samples}
\caption{Samples generated by a real NVP model trained on 64x64 ImageNet images.
Figure reproduced from \citet{dinh2016density}.}
\label{fig:nvp}
\end{figure}

For more information about the chain rule of probability used to define FVBNs
or about the effect of deterministic transformations on probability densities
as used to define nonlinear ICA models, see chapter 3 of \citet{Goodfellow-et-al-2016}.

In summary, models that define an explicit, tractable density are highly
effective, because they permit the use of an optimization algorithm directly
on the log-likelihood of the training data.
However, the family of models that provide a tractable density is limited,
with different families having different disadvantages.

\subsubsection{Explicit models requiring approximation}
\label{sec:approx}

To avoid some of the disadvantages imposed by the design requirements of models
with tractable density functions, other models have been developed that still
provide an explicit density function but use one that is intractable, requiring
the use of approximations to maximize the likelihood.
These fall roughly into two categories: those using deterministic approximations,
which almost always means variational methods, and those using stochastic approximations,
meaning Markov chain Monte Carlo methods.

\paragraph{Variational approximations}
Variational methods define a lower bound
\[ \mathcal{L}(\vx; \vtheta) \leq \log \pmodel(\vx; \vtheta). \]
A learning algorithm that maximizes $\mathcal{L}$ is guaranteed to obtain at least
as high a value of the log-likelihood as it does of $\mathcal{L}$.
For many families of models, it is possible to define an $\mathcal{L}$ that is computationally
tractable even when the log-likelihood is not.
Currently, the most popular approach to variational learning in deep generative models
is the \newterm{variational autoencoder} \citep{Kingma-arxiv2013,Rezende-et-al-ICML2014} or VAE.
Variational autoencoders are one of the three approaches to deep generative modeling that are
the most popular as of this writing, along with FVBNs and GANs.
The main drawback of variational methods is that,
when too weak of an approximate posterior distribution or too weak of a prior distribution is used,
\footnote{
  Empirically, VAEs with highly flexible priors or highly flexible approximate posteriors
  can obtain values of $\mathcal{L}$ that are near their own log-likelihood
  \citep{kingma2016improving,chen2016variational}.
  Of course, this is testing the gap between the objective and the bound at the maximum of the bound;
  it would be better, but not feasible, to test the gap at the maximum of the objective.
  VAEs obtain likelihoods that are competitive with other methods, suggesting that they are also
  near the maximum of the objective.
  In personal conversation, L. Dinh and D. Kingma have conjectured that a family of models
  \citep{Dinh-et-al-arxiv2014,rezende2015variational,kingma2016improving,dinh2016density}
  usable as VAE priors or approximate posteriors are universal approximators.
  If this could be proven, it would establish VAEs as being asymptotically consistent.
}
even with a perfect optimization algorithm and infinite training data, the gap
between $\mathcal{L}$ and the true likelihood can result in $\pmodel$ learning something other than
the true $\pdata$.
GANs were designed to be unbiased, in the sense that with a large enough model and infinite data,
the Nash equilibrium for a GAN game corresponds to recovering $\pdata$ exactly.
In practice, variational methods often obtain very good likelihood, but are regarded as producing
lower quality samples.
There is not a good method of quantitatively measuring sample quality, so this is a subjective opinion,
not an empirical fact.
See \figref{fig:vae_samples} for an example of some samples drawn from a VAE.
While it is difficult to point to a single aspect of GAN design and say that it results in
better sample quality, GANs are generally regarded as producing better samples.
Compared to FVBNs, VAEs are regarded as more difficult to optimize, but GANs are not
an improvement in this respect.
For more information about variational approximations, see chapter 19 of
\citet{Goodfellow-et-al-2016}.

\begin{figure}
  \centering
  \includegraphics[width=\textwidth]{cifar_vae}
  \caption{Samples drawn from a VAE trained on the CIFAR-10 dataset.
    Figure reproduced from \citet{kingma2016improving}.
  }
  \label{fig:vae_samples}
\end{figure}

\paragraph{Markov chain approximations}
Most deep learning algorithms make use of some form of stochastic approximation,
at the very least in the form of using a small number of randomly selected training
examples to form a minibatch used to minimize the expected loss.
Usually, sampling-based approximations work reasonably well as long as a fair sample
can be generated quickly (e.g. selecting a single example from the training set
is a cheap operation) and as long as the variance across samples is not too high.
Some models require the generation of more expensive samples, using Markov chain
techniques.
A Markov chain is a process for generating samples by repeatedly drawing a sample
$\vx' \sim q(\vx' \mid \vx).$
By repeatedly updating $\vx$ according to the transition operator $q$, Markov chain
methods can sometimes guarantee that $\vx$ will eventually converge to a sample from
$\pmodel(\vx)$.
Unfortunately, this convergence can be very slow, and there is no clear way to test
whether the chain has converged, so in practice one often uses $\vx$ too early, before
it has truly converged to be a fair sample from $\pmodel$.
In high-dimensional spaces, Markov chains become less efficient.
Boltzmann machines \citep{Fahlman83,Ackley85,Hinton-Boltzmann,Hinton86a} are a
family of generative models that rely on Markov chains both to train the model
or to generate a sample from the model.
Boltzmann machines were an important part of the deep learning renaissance beginning
in 2006 \citep{Hinton06,hinton2007learning} but they are now used only very rarely,
presumably mostly because the underlying Markov chain approximation techniques have
not scaled to problems like ImageNet generation.
Moreover, even if Markov chain methods scaled well enough to be used for training,
the use of a Markov chain to generate samples from a trained model is undesirable
compared to single-step generation methods because the multi-step Markov chain
approach has higher computational cost.
GANs were designed to avoid using Markov chains for these reasons.
For more information about Markov chain Monte Carlo approximations, see chapter 18 of
\citet{Goodfellow-et-al-2016}.
For more information about Boltzmann machines, see chapter 20 of the same book.

Some models use both variational and Markov chain approximations.
For example, deep Boltzmann machines make use of both types of
approximation \citep{SalHinton09}.

\subsection{Implicit density models}

Some models can be trained without even needing to explicitly define a density
functions.
These models instead offer a way to train the model while interacting only
indirectly with $\pmodel$, usually by sampling from it.
These constitute the second branch, on the right side, of our taxonomy of
generative models depicted in \figref{fig:tree}.

Some of these implicit models based on drawing samples from $\pmodel$ define
a Markov chain transition operator that must be run several times to obtain
 a sample from the model.
From this family, the primary example is the \newterm{generative stochastic network}
\citep{Bengio-et-al-ICML-2014}.
As discussed in \secref{sec:approx}, Markov chains often fail to scale to high
dimensional spaces, and impose increased computational costs for using the
generative model. GANs were designed to avoid these problems.

Finally, the rightmost leaf of our taxonomic tree is the family of implicit models
that can generate a sample in a single step.
At the time of their introduction, GANs were the only notable member of this family,
but since then they have been joined by additional models based on
kernelized moment matching \citep{Li-et-al-2015,dziugaite2015training}.

\subsection{Comparing GANs to other generative models}

In summary, GANs were designed to avoid many disadvantages associated with other generative
models:
\begin{itemize}
  \item They can generate samples in parallel, instead of using runtime proportional to the
    dimensionality of $\vx$. This is an advantage relative to FVBNs.
  \item The design of the generator function has very few restrictions. This is an advantage
    relative to Boltzmann machines, for which few probability distributions admit tractable
    Markov chain sampling, and relative to nonlinear ICA, for which the generator must be
    invertible and the latent code $\vz$ must have the same dimension as the samples
    $\vx$.
  \item No Markov chains are needed. This is an advantage relative to Boltzmann machines and GSNs.
  \item No variational bound is needed, and specific model families usable within the GAN
    framework are already known to be universal approximators, so GANs are already known
    to be asymptotically consistent.
    Some VAEs are conjectured to be asymptotically consistent, but this is not yet proven.
  \item GANs are subjectively regarded as producing better samples than other methods.
\end{itemize}
At the same time, GANs have taken on a new disadvantage: training them requires finding
the Nash equilibrium of a game, which is a more difficult problem than optimizing an
objective function.

Roles played by characters in word formation can be divided into three types. 
\textbf{\footnotesize{(1)}} There is a stable and important set of 
%about %【xxxx不要忘了】 
\emph{single-char words}, such as ``你'' (you)'', ``的'' (of), and most punctuation marks. 
\textbf{\footnotesize{(2)}} A character having no specific meaning acts as a \emph{part of a single-morpheme word}, such as ``仿佛'' (like) and %``沙(sh\=a)发(f\=a)'' (sofa, 
``法(f\v{a})老(l\v{a}o)'' (Pharaoh, 
transliteration of foreign words). 
% play phonetic roles for transliteration of foreign words, such as ``沙发'' (sha fa音调也放上?) for sofa. 
\textbf{\footnotesize{(3)}} A character corresponds to a \emph{morpheme}, the smallest meaningful unit in a language, and composes a polysyllabic word with other characters. This work  targets multi-char words, and  is particularly interested in the third type which most characters belong to. 

%Broadly speaking, understanding how multiple characters form a word, i.e., the word-formation process, is crucial for learning Chinese grammar. 
Intuitively, modeling how multiple characters form a word, i.e., the word-formation process, allows us to  more effectively represent the meaning of a word via composing the meanings of characters. 
This is especially helpful for handling rare words,  considering that the vocabulary size of characters is much  smaller than that of words.
%data sparseness is the key challenge for NLP models. % (long tail)
%can provide useful direct  directly help beneficial for both understanding (thus representing) common words and handling rare (new) words.  
% Specifically, it helps 
%people to better understand known words, and better handle 
% This motivates 
In fact, many NLP researchers have tried to utilize char-level word-internal structures for better Chinese understanding. 
Most related to ours, previous studies on syntactic parsing have proposed to annotate word-internal structures to alleviate the data sparseness problem \cite{zhang-etal-2014-char,li-etal-aaai-2018-zhaohai}. However, their annotations mainly consider flat and shallow word-internal structure, as shown in Figure \ref{fig:example}-(a) and (b). 
Meanwhile, researchers try to make use of character information to learn better word embeddings \cite{chen-ijcai2015-joint-char-word-emb,xu-naacl-2016-internal-structure}. 
Without explicitly capturing word-internal structures, these studies have to treat a word as a bag of characters.
%focus on modeling semantic contribution of individual chars for a composed word. 
%2) for to obtain better Chinese word embeddings. 不考虑结构信息,只是扁平的、compose char的
% Word-formation process has drawn tremendous interest from  linguistics, and NLP fields. % In a broad sense, understanding how multiple characters form a word is the basic for learning the Chinese grammar (CITE?).  
% 少一句过渡,汉语的呢?
% Intuitively, understanding how multiple characters form a word is very useful for representing the meaning of 
See Section \ref{sec:related-work} for more discussion. 

This paper presents an in-depth study on char-level internal structure of Chinese  words. We endeavour to address three questions. \textbf{\footnotesize{(1)}} What are 
the word-formation patterns for Chinese words? \textbf{\footnotesize{(2)}} Can we train a model to predict deep word-internal structures? 
\textbf{\footnotesize{(3)}} Is modeling word-internal structures beneficial for word representation learning? 
%Specifically, this work can be decomposed into four parts. 

For the first question, we propose to use labeled dependency trees to represent word-internal structures, and employ 11 labels to distinguish syntactic roles in word formation. We compile annotation guidelines following the famous textbook of \citet{grammar-notes-zhu-1982} on Chinese syntax, and annotate a high-quality word-internal structure treebank (WIST), consisting of 30K words from Penn Chinese Treebank (CTB) \cite{ctb-xiafei}. We conduct detailed analysis on WIST to gain insights on Chinese word-formation patterns.
%via  detailed analysis. 

For the second question, we propose word-internal structure parsing as a new task, and present benchmark experimental results using a competitive open-source dependency parser. 
%应该讲一下为什么研究构词法只需要考虑句法?为什么用依存树?

For the third question, we investigate two simple ways to encode word-internal structure, i.e., LabelCharLSTM and LabelGCN, and show that using the resulting word representation leads to promising gains on the dependency parsing task.
%extra input can substantially boost the dependency parsing performance.
%, both dependency and constituent.
%for two state-of-the-art syntactic parsers. Experiments show that 

We release WIST at \url{https://github.com/SUDA-LA/ACL2021-wist}, and also provide a demo to parse the internal structure of any input word. 

% 汉语构词法的相关研究(看一些文献)。为什么要用句法来刻画构词法呢?汉语构词法与句法关联紧密。汉语词内部和句子的构成是非常相似的。

% TODO: 在unknown/rare word上的性能如何?分析一下。要佐证我们intro中的论点。

% 例子:想方设法 婚姻法 法老(Pharaoh) 
 %沙发? 劳动法
 
% 汉语的形态变化很少,例如suffix prefix,不通过形态变化来体现额外的时态、单复数信息。当然有一些(如们、者、性等后缀词素)

% rare/new word的表示,可以通过charLSTM等获取上下文无关表示,或者ELMo/BERT通过上下文信息来弥补得到上下文相关表示。但是,内部结构是否还可以继续提高呢?

% 词是最小的能够独立活动的有意义的语言成分。
% 根据中华人民共和国国家标准颁布的信息处理用现代汉语分词规范:使用稳定、结合紧密的汉字串被看作是词。和句法结构类似,汉语词也通常有syntactic structure,typical的结构包括定中、并列、主谓、述宾等。然后举个例子:一个词由xxx字组成,这些字之间是什么关系(可以考虑给出示意图)。(MeishanZhang13)
% 然而,很多语料都以词为最小单位.如CTB,pku,msr..``This form of annotation does not
% give character-level syntactic structures for words, a source of linguistic information that is more fundamental and less sparse than atomic words.'' (Zhang13)
% 很多中文NLP工作也都是基于字或词,忽略了词的内部结构,
% 而词内部结构是很重要的:
% 与flat word相比,intra-word structure can be informative在以下几个方面有优势:
% 1.很多词的含义可以通过分析其内部组成成分来获得,从而可以缓解伪OOV现象:比如劳动法在ctb5中是OOV,但是ctb5中有婚姻法、劳动等词,通过分析内部结构,可以使更准确地表示伪OOV(李中国12)。很多词的含义可以通过充分考虑各内部成分的关系和组合其内部成分的含义得到。
% 2.词内部结构比flat词蕴含更丰富的信息。有些特殊的语言现象需要通过词内部结构才能准确刻画:比如大中小学、动植物,如果不分析内部结构,就丢失了大学、中学、小学并列的信息(李中国11和赵海09都提到了)
% 3.此外,中文分词标准不同,导致标注不一致。利用词内部结构信息可以灵活得到不同粒度的词以满足不同应用的需求(Zhang14)【这个要写吗?】
% 因此,基于词内部结构可以建模词的语义表示,作为传统以词为基本单元的词向量表示方法的有效补充。
% 考虑到词内部结构存在以上优点,近年来,有一些工作研究了词内部结构,在xxx任务上验证了有效性【zhao09, li1112,zhang1314,su20】。但是这些工作通常只考虑结构框架和核心字信息,不考虑各个字之间的关系类型。Zhao18用词性组合来表示各个字之间的关系,然而这种表示也并不能直观体现出各个字之间的依存类型。

% 在我们的工作中,我们根据词结构的特点设计了11种依存标签,能直观地反映出词内部各个字直接的依存关系,``provide a more general and natural
% way to reflect character relations within a sequence than word boundary annotations do.''


% word internal structure 还是word-intra structure统一一下

%中文和英文不同之处,词是由字构成的,字本身是有含义的。而英文字母是没有含义的(词根有可能有含义,词根里面的每一个字母又没有含义的。可以加个脚注:有一部分英语单词的含义,可以由切分出来的几个子串的含义猜出来)。但是中文,非常明确的,由几个字构成。每个字的含义很具体,并且通过一定的含义组合composition构成。大部分词都是composition,很少的是atom,这种词通常是外来语等等。

%中文语言理解通常都以词为单位(比如句法分析、语义分析、检索)。
%(BERT是以字为单位去学习表示的)
%但是也有很多工作加入词信当然神经网络下,很多端到端的工作,可以以字为单位
%(机器翻译中以词为单位比较好)。

% As the minimal semantic units that can act independently (can be used alone), words play an important role in the Chinese language. 
% Traditionally, words are defined as the character sequences that combine tightly and occur frequently 【信息处理用现代汉语分词规范】. 
% Similar to the sentences which have word-level syntactic structures, most of the Chinese words have character-level internal syntactic structures. 【As shown in Figure 用图吗?】, typical word internal structures include modifier-noun, coordination, subject-predicate and verb-object structures.
% The meaning of a word can be reflected by the meaning of its constituent characters and the way that characters interact each other. For example, the word ``创业者 (entrepreneur)'' is composed of three characters ``创 (start)'', ``业 (business)'' and ``者 (people)'', where the characters ``创 (start)'' and ``业 (business)'' form a verb-object structure and modify the character ``者 (people)''. This accurately convey the meaning of ``创业者 (entrepreneur)'' is ``the people who start a business''.

% 词的表示是非常核心的问题。word embedding之前,one-hot的表示,相似的词之间的相似度很难刻画。
% word embedding呢,虽然缓解这个问题,但是是靠大规模无标注数据,也就是上下文信息来表示词。而从词本身出发去理解词、表示词,必然能够得到更好的词表示。



% However, many manually annotated data in Chinese, such as CTB【全称,引用】, PKU【】 and MSR【】, takes words as the basic processing unit, and previous approach for Chinese language processing is typically based on characters or words, giving no consideration to the syntactic structures within words.
% Internal word structures can be beneficial from the following perspectives:
% 1)【新词的理解】The issue of Out-of-vocabulary (OOV) words can be alleviated by analyzing their internal structures. As pointed out by \citet{li-2011}, more than 60\% of the OOVs are pseudo OOVs, which means though these words do not exist in the training data, their components or words with similar structures are frequently-occurring words. For example, ``劳动法 (labor law)'' is an OOV of CTB5, but ``劳动 (labor)'', ``法 (law)'' and ``婚姻法 (marriage law)'' appears several times in the CTB5 training data. The meaning of ``劳动法 (labor law)'' can be obtained by fully considering the syntactic relations between its constituents and combining their semantics.
% 2)Word internal structure can be informative with dependency and relation labels between characters. It can represent dependencies between discontinuous characters, while word boundary information is insufficient to handle such language phenomena in Chinese. 
% 3) Words of granularities can be extracted according to the intra-word dependency, and hence can benefit downstream NLP applications with flexible segmentation standards【参考文献】.   

% In fact, 
% % Motivated by above perspectives,
% modeling word internal structures had attracted a lot of research attention, mostly before the pre-DL era (cite三个人的三个文章: 
% 李、张梅山、赵海三个工作)。上面有图对几个工作进行了对比。
% Please refer to Section \ref{xx} for more discussions on previous works. 
% Dataset【】, and verifying the effectiveness of word internal structure in improving performance for various Chinese language processing tasks such as word segmentation \cite{zhao-2009}, parsing \cite{li-zhou-2012,zhang-etal-2013,zhang-etal-2014}, and machine translation【苏20】 by exploring rich embedding features or joint learning. 
%  【下面这段话和annotation的开头重复了,需要考虑把这段话放在哪,或者改一下表达方式】
%  Most of these work mainly focus on the the hierarchical structure or the head character information for each word, without considering the relation types between constituent characters inside the word. 
% Li et al. \cite{DBLP:conf/aaai/LiZJZ18} represent intra-word character-level dependency label by combining the character-level POS tags of the constituent characters. However, the typical syntactic structures such as subject-predicate, verb-object, modifier-noun, coordination, which is quite important for analyzing。。。  can not be directly reflected by such dependency labels.
% To overcome this obstacle, we construct intra-word character dependency treebank of over xxx words in CTB5 \cite{ctb-xiafei} and CoNLL09 \cite{conll09} 
% according to our newly complied annotation guideline, which contains 11 relations to intuitively reflect the internal dependency syntax for Chinese words.
% 。。。。。。。。。。


% 1.词内部结构比flat词蕴含更丰富的信息。convey meaning (zhang13)  interact with each other from both syntax and semantics (AAAI18). provide more general and natural way to reflect character relations within a character sequence than word boundary annotations (Zhao09)
% providing both internal structures to synthetic words and global structure to sentences in a seamless fashion (LREC14)
% 建模中文词内不同字之间的关系,建模词的语义表示,作为传统以词为基本单元的词向量表示方法的有效补充。(苏劲松2020)


% 2.中文分词标准不同,导致标注不一致。利用词内部结构信息可以灵活得到不同粒度的词(Zhang14)。a. 从而满足不同应用的需求。b. 得到子词来缓解OOV(LREC14,IJCNLP15)

% 3.动植物、大中小学这些词难以通过分词准确刻画?而词内部结构可以刻画这种语言现象



