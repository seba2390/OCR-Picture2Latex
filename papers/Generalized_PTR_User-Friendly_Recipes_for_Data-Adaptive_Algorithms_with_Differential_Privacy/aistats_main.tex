\documentclass[twoside]{article}

\usepackage[numbers,square]{natbib}

\usepackage{aistats2023}

\usepackage{amsmath}
\usepackage{subfigure}
\usepackage[toc,page,header]{appendix}
\usepackage{minitoc}
\usepackage[utf8]{inputenc}
\usepackage{amssymb}
\usepackage{mathtools}
\usepackage{comment}
\usepackage{amsthm}
\usepackage{graphicx}   
\usepackage[textsize=tiny]{todonotes}
\usepackage{xcolor}

\usepackage{algorithmic,algorithm}

\usepackage[utf8]{inputenc} % allow utf-8 input
\usepackage[T1]{fontenc}    % use 8-bit T1 fonts
\usepackage[colorlinks,citecolor=blue,urlcolor=blue,linkcolor=blue,linktocpage=true]{hyperref}       % hyperlinks
\usepackage{url}            % simple URL typesetting
\usepackage{booktabs}       % professional-quality tables
\usepackage{amsfonts}       % blackboard math symbols
\usepackage{nicefrac}       % compact symbols for 1/2, etc.
\usepackage{microtype}      % microtypography
\usepackage{xcolor}         % colors

%%%%%%%%%%%%%%%%%%%%%%%%%%%%%%%%
% Notations
\newcommand{\blue}[1]{\textcolor{blue}{#1}}
\newcommand{\purple}[2]{\textcolor{purple}{#1}}
\def\rdp{\mathrm{RDP}}
\def\er{\mathrm{Err}}
\def\dis{\mathrm{Dis}}
\def\pr{\mathrm{Pr}}
\def\E{\mathbb{E}}
\def\P{\mathbb{P}}
\def\argmax{Argmax}
\def\Cov{\mathrm{Cov}}
\def\Var{\mathrm{Var}}
\def\half{\frac{1}{2}}
\def\th{\mathrm{th}}
\def\tr{\mathrm{tr}}
\def\df{\mathrm{df}}
\def\dim{\mathrm{dim}}
\def\col{\mathrm{col}}
\def\row{\mathrm{row}}
\def\nul{\mathrm{null}}
\def\rank{\mathrm{rank}}
\def\nuli{\mathrm{nullity}}
\def\sign{\mathrm{sign}}
\def\supp{\mathrm{supp}}
\def\diag{\mathrm{diag}}
\def\Count{\mathrm{count}}
\def\Gauss{\mathrm{Gauss}}
\def\subsample{\mathsf{PoissonSample}}
\def\sparse{\mathsf{SparseVector}}
\def\Lap{\mathsf{Lap}}
\def\Gau{\mathsf{Gau}}
\def\rr{\mathsf{Random Response}}
\def\lap{\mathsf{Laplace Mechanism}}
\def\expon{\mathsf{Exponential Mechanism}}
\def\aff{\mathrm{aff}}
\def\hy{\hat{y}}
\def\ty{\tilde{y}}
\def\hbeta{\hat{\beta}}
\def\tbeta{\tilde{\beta}}
\def\htheta{\hat{\theta}}
\def\halpha{\hat{\alpha}}
\def\hf{\hat{f}}
\def\lone{1}
\def\ltwo{2}
\def\linf{\infty}
\def\lzero{0}
\def\T{^T}
\def \E{\mathbb{E}}
\def\hp{\hat{h}^{priv}}
\def\R{\mathbb{R}}
\def\cA{\mathcal{A}}
\def\cB{\mathcal{B}}
\def\cD{\mathcal{D}}
\def\cE{\mathcal{E}}
\def\cF{\mathcal{F}}
\def\cG{\mathcal{G}}
\def\cH{\mathcal{H}}
\def\cI{\mathcal{I}}
\def\cJ{\mathcal{J}}
\def\cO{\mathcal{O}}
\def\cL{\mathcal{L}}
\def\cM{\mathcal{M}}
\def\cN{\mathcal{N}}
\def\cP{P}
\def\cp{p} % denote the probability density
\def\cQ{\mathcal{Q}}
\def\cR{\mathcal{R}}
\def\cS{\mathcal{S}}
\def\cT{\mathcal{T}}
\def\cW{\mathcal{W}}
\def\cX{\mathcal{X}}
\def\cY{\mathcal{Y}}
\def\cZ{\mathcal{Z}}
\def\ls{\triangle_{LS}}
\def\TV{\mathrm{TV}}
\def\Int{\int_{-\infty}^{+\infty}}
\def \Intt{\int_{-\infty}^{T}}
\def \Inttc{\int_{-\infty}^{T+\triangle}}
\def \dt{\triangle t}
\newcommand{\diff}{\,\mathrm{d}}
\newcommand{\e}{\mathrm{e}}
\newcommand{\eps}{\varepsilon}
\newcommand{\argmin}{\text{argmin}}
\newcommand{\ttheta}{\tilde{\theta}}




\def\shownotes{1}  %set 1 to show author notes
\ifnum\shownotes=1
\newcommand{\authnote}[2]{$\ll$\textsf{\footnotesize #1 notes: #2}$\gg$}
\else
\newcommand{\authnote}[2]{}
\fi

\newcommand{\yw}[1]{\textcolor{red}{\textbf{[yuxiang: #1]}}}
\newcommand{\yq}[2]{\textcolor{blue}{\textbf{[yuqing: #1]}}}
% If your paper is accepted, change the options for the package
% aistats2023 as follows:
%
% \usepackage[accepted]{aistats2023}


\begin{document}

%%%%%%%%%%%%%%%%%%%%%%%%%%%%%%%%
% THEOREMS
%%%%%%%%%%%%%%%%%%%%%%%%%%%%%%%%
\theoremstyle{plain}
\newtheorem{theorem}{Theorem}[section]
\newtheorem{proposition}[theorem]{Proposition}
\newtheorem{lemma}[theorem]{Lemma}
\newtheorem{corollary}[theorem]{Corollary}
\theoremstyle{definition}
\newtheorem{definition}[theorem]{Definition}
\newtheorem{assumption}[theorem]{Assumption}
%\theoremstyle{remark}
\newtheorem{remark}[theorem]{Remark}
\newtheorem{claim}[theorem]{Claim}
\theoremstyle{plain}
\newtheorem{example}[theorem]{Example}


% If your paper is accepted and the title of your paper is very long,
% the style will print as headings an error message. Use the following
% command to supply a shorter title of your paper so that it can be
% used as headings.
%
%\runningtitle{I use this title instead because the last one was very long}

% If your paper is accepted and the number of authors is large, the
% style will print as headings an error message. Use the following
% command to supply a shorter version of the authors names so that
% they can be used as headings (for example, use only the surnames)
%
%\runningauthor{Surname 1, Surname 2, Surname 3, ...., Surname n}

\twocolumn[

\aistatstitle{Generalized PTR: User-Friendly Recipes for Data-Adaptive Algorithms with Differential Privacy}

\aistatsauthor{ Rachel Redberg \And Yuqing Zhu \And Yu-Xiang Wang }

\aistatsaddress{ UC Santa Barbara \And UC Santa Barbara \And UC Santa Barbara} ]

\begin{abstract}
    The ``Propose-Test-Release'' (PTR) framework \cite{dwork2009differential} is a classic recipe for designing differentially private (DP) algorithms that are data-adaptive, i.e. those that  add less noise when the input dataset is ``nice''. We extend PTR to a more general setting by privately testing \emph{data-dependent privacy losses} rather than \emph{local sensitivity}, hence making it applicable beyond the standard noise-adding mechanisms, e.g. to queries with unbounded or undefined sensitivity. We demonstrate the versatility of generalized PTR using private linear regression as a case study. Additionally, we apply our algorithm to solve an open problem from “Private Aggregation of Teacher Ensembles (PATE)” \citep{papernot2017, papernot2018scalable} --- privately releasing the entire model with a delicate data-dependent analysis.
\end{abstract}

% \leavevmode
% \\
% \\
% \\
% \\
% \\
\section{Introduction}
\label{introduction}

AutoML is the process by which machine learning models are built automatically for a new dataset. Given a dataset, AutoML systems perform a search over valid data transformations and learners, along with hyper-parameter optimization for each learner~\cite{VolcanoML}. Choosing the transformations and learners over which to search is our focus.
A significant number of systems mine from prior runs of pipelines over a set of datasets to choose transformers and learners that are effective with different types of datasets (e.g. \cite{NEURIPS2018_b59a51a3}, \cite{10.14778/3415478.3415542}, \cite{autosklearn}). Thus, they build a database by actually running different pipelines with a diverse set of datasets to estimate the accuracy of potential pipelines. Hence, they can be used to effectively reduce the search space. A new dataset, based on a set of features (meta-features) is then matched to this database to find the most plausible candidates for both learner selection and hyper-parameter tuning. This process of choosing starting points in the search space is called meta-learning for the cold start problem.  

Other meta-learning approaches include mining existing data science code and their associated datasets to learn from human expertise. The AL~\cite{al} system mined existing Kaggle notebooks using dynamic analysis, i.e., actually running the scripts, and showed that such a system has promise.  However, this meta-learning approach does not scale because it is onerous to execute a large number of pipeline scripts on datasets, preprocessing datasets is never trivial, and older scripts cease to run at all as software evolves. It is not surprising that AL therefore performed dynamic analysis on just nine datasets.

Our system, {\sysname}, provides a scalable meta-learning approach to leverage human expertise, using static analysis to mine pipelines from large repositories of scripts. Static analysis has the advantage of scaling to thousands or millions of scripts \cite{graph4code} easily, but lacks the performance data gathered by dynamic analysis. The {\sysname} meta-learning approach guides the learning process by a scalable dataset similarity search, based on dataset embeddings, to find the most similar datasets and the semantics of ML pipelines applied on them.  Many existing systems, such as Auto-Sklearn \cite{autosklearn} and AL \cite{al}, compute a set of meta-features for each dataset. We developed a deep neural network model to generate embeddings at the granularity of a dataset, e.g., a table or CSV file, to capture similarity at the level of an entire dataset rather than relying on a set of meta-features.
 
Because we use static analysis to capture the semantics of the meta-learning process, we have no mechanism to choose the \textbf{best} pipeline from many seen pipelines, unlike the dynamic execution case where one can rely on runtime to choose the best performing pipeline.  Observing that pipelines are basically workflow graphs, we use graph generator neural models to succinctly capture the statically-observed pipelines for a single dataset. In {\sysname}, we formulate learner selection as a graph generation problem to predict optimized pipelines based on pipelines seen in actual notebooks.

%. This formulation enables {\sysname} for effective pruning of the AutoML search space to predict optimized pipelines based on pipelines seen in actual notebooks.}
%We note that increasingly, state-of-the-art performance in AutoML systems is being generated by more complex pipelines such as Directed Acyclic Graphs (DAGs) \cite{piper} rather than the linear pipelines used in earlier systems.  
 
{\sysname} does learner and transformation selection, and hence is a component of an AutoML systems. To evaluate this component, we integrated it into two existing AutoML systems, FLAML \cite{flaml} and Auto-Sklearn \cite{autosklearn}.  
% We evaluate each system with and without {\sysname}.  
We chose FLAML because it does not yet have any meta-learning component for the cold start problem and instead allows user selection of learners and transformers. The authors of FLAML explicitly pointed to the fact that FLAML might benefit from a meta-learning component and pointed to it as a possibility for future work. For FLAML, if mining historical pipelines provides an advantage, we should improve its performance. We also picked Auto-Sklearn as it does have a learner selection component based on meta-features, as described earlier~\cite{autosklearn2}. For Auto-Sklearn, we should at least match performance if our static mining of pipelines can match their extensive database. For context, we also compared {\sysname} with the recent VolcanoML~\cite{VolcanoML}, which provides an efficient decomposition and execution strategy for the AutoML search space. In contrast, {\sysname} prunes the search space using our meta-learning model to perform hyperparameter optimization only for the most promising candidates. 

The contributions of this paper are the following:
\begin{itemize}
    \item Section ~\ref{sec:mining} defines a scalable meta-learning approach based on representation learning of mined ML pipeline semantics and datasets for over 100 datasets and ~11K Python scripts.  
    \newline
    \item Sections~\ref{sec:kgpipGen} formulates AutoML pipeline generation as a graph generation problem. {\sysname} predicts efficiently an optimized ML pipeline for an unseen dataset based on our meta-learning model.  To the best of our knowledge, {\sysname} is the first approach to formulate  AutoML pipeline generation in such a way.
    \newline
    \item Section~\ref{sec:eval} presents a comprehensive evaluation using a large collection of 121 datasets from major AutoML benchmarks and Kaggle. Our experimental results show that {\sysname} outperforms all existing AutoML systems and achieves state-of-the-art results on the majority of these datasets. {\sysname} significantly improves the performance of both FLAML and Auto-Sklearn in classification and regression tasks. We also outperformed AL in 75 out of 77 datasets and VolcanoML in 75  out of 121 datasets, including 44 datasets used only by VolcanoML~\cite{VolcanoML}.  On average, {\sysname} achieves scores that are statistically better than the means of all other systems. 
\end{itemize}


%This approach does not need to apply cleaning or transformation methods to handle different variances among datasets. Moreover, we do not need to deal with complex analysis, such as dynamic code analysis. Thus, our approach proved to be scalable, as discussed in Sections~\ref{sec:mining}.
\label{section:introduction}

\section{Related Work}
\label{sec:related_work}
We now provide a brief overview of related work in the areas of language grounding and transfer for reinforcement learning.
%There has been work on learning to make optimal local decisions for structured prediction problems~\cite{daume2006searn}.
%
%\newcite{branavan2010reading} looked at a similar task of building a partial model of the environment while following instructions. The differences with our work are (1) the text in their case is instructions, while we only have text describing the environment, and (2) their environment is deterministic, hence the transition function can be learned more easily. 
%
%TODO - model-based RL, value iteration, predictron.


\subsection{Grounding Language in Interactive Environments}
In recent years, there has been increasing interest in systems that can utilize textual knowledge to learn control policies. Such applications include interpreting help documentation~\fullcite{branavan2010reading}, instruction following~\fullcite{vogel2010learning,kollar2010toward,artzi2013weakly,matuszek2013learning,Andreas15Instructions} and learning to play computer games~\fullcite{branavan2011nonlinear,branavan2012learning,narasimhan2015language,he2016deep}. In all these applications, the models are trained and tested on the same domain.

Our work represents two departures from prior work on grounding. First, rather than optimizing control performance for a single domain,
we are interested in the multi-domain transfer scenario, where language 
descriptions drive generalization. Second, prior work used text in the form of strategy advice to directly learn the policy. Since the policies are typically optimized for a specific task, they may be harder to transfer across domains. Instead, we utilize text to bootstrap the induction of the environment dynamics, moving beyond task-specific strategies. 

%Previous work has explored the use of text manuals in game playing, %ranging from constructing useful features by mining patterns in %text~\cite{eisenstein2009reading}, learning a semantic interpreter %with access to limited gameplay examples~\cite{goldwasser2014learning} %to learning through reinforcement from in-game %rewards~\cite{branavan2011learning}. These efforts have demonstrated %the usefulness of exploiting domain knowledge encoded in text to learn %effective policies. However, these methods use the text to infer %directly the best strategy to perform a task. In contrast, our goal is %to learn mappings from the text to the dynamics of an environment and %separate out the learning of the strategy/motives. 

Another related line of work consists of systems that learn spatial and topographical maps of the environment for robot navigation using natural language descriptions~\fullcite{walter2013learning,hemachandra2014learning}. These approaches use text mainly containing appearance and positional information, and integrate it with other semantic sources (such as appearance models) to obtain more accurate maps. In contrast, our work uses language describing the dynamics of the environment, such as entity movements and interactions, which 
is complementary to static positional information received through state observations. Further, our goal is to help an agent learn policies that generalize over different stochastic domains, while their works consider a single domain.

%karthik: I don't see the direct relevance
%Another line of work explores using textual interactive %environments~\cite{narasimhan2015language,he2016deep} to ground %language understanding into actions taken by the system in the %environment. In these tasks, understanding of language is crucial, %without which a system would not be able to take reasonable actions. %Our motivation is different -- we take an existing set of tasks and %domains which are amenable to learning through reinforcement, and %demonstrate how to utilize textual knowledge to learn faster and more %optimal policies in both multitask and transfer setups.

\subsection{Transfer in Reinforcement Learning}
Transferring policies across domains is a challenging problem in reinforcement learning~\fullcite{konidaris2006framework,taylor2009transfer}. The main hurdle lies in learning a good mapping between the state and action spaces of different domains to enable effective transfer. Most previous approaches have either explored skill transfer~\fullcite{konidaris2007building,konidaris2012transfer} or value function/policy transfer~\fullcite{liu2006value,taylor2007transfer,taylor2007cross}. There have also been attempts at model-based transfer for RL~\fullcite{taylor2008transferring,nguyen2012transferring,gavsic2013pomdp,wang2015learning,joshi2018cross} but these methods either rely on hand-coded inter-task mappings for state and actions spaces or require significant interactions in the target task to learn an effective mapping. Our approach doesn't use any explicit mappings and can learn to predict the dynamics of a target task using its descriptions.

% Work by \newcite{konidaris2006autonomous} look at knowledge transfer by learning a mapping from sensory signals to reward functions.

A closely related line of work concerns transfer methods for deep reinforcement learning. \citeA{parisotto2016actor}  train a deep network to mimic pre-trained experts on source tasks using policy distillation. The learned parameters are then used to initialize a network on a target task to perform transfer. Rusu et al.~\citeyear{rusu2016progressive} facilitate transfer by freezing parameters learned on source tasks and adding a new set of parameters for every new target task, while using both sets to learn the new policy. Work by Rajendran et al.~\citeyear{rajendran20172t} uses attention networks to selectively transfer from a set of expert policies to a new task. \textcolor{black}{Barreto et al.~\citeyear{barreto2017successor} use features based on successor representations~\fullcite{dayan1993improving} for transfer across tasks in the same domain. Kansky~et~al.~\citeyear{kansky2017schema} learn a generative model of causal physics in order to help zero-shot transfer learning.} Our approach is orthogonal to all these directions since we use text to bootstrap transfer, and can potentially be combined with these methods to achieve more effective transfer. 

\textcolor{black}{There has also been prior work on zero-shot policy generalization for tasks with input goal specifications. \fullciteA{schaul2015universal} learn a universal value function approximator that can generalize across both states and goals. \fullcite{andreas2016modular} use policy sketches, which are annotated sequences of subgoals, in order to learn a hierarchical policy that can generalize to new goals. \fullciteA{oh2017zero} investigate zero-shot transfer for instruction following tasks, aiming to generalize to unseen instructions in the same domain. The main departure of our work compared to these is in the use of environment descriptions for generalization across domains rather than generalizing across text instructions.}

Perhaps closest in spirit to our hypothesis is the recent work by~\fullcite{harrison2017guiding}. Their approach makes use of paired instances of text descriptions and state-action information from human gameplay to learn a machine translation model. This model is incorporated into a policy shaping algorithm to better guide agent exploration. Although the motivation of language-guided control policies is similar to ours, their work considers transfer across tasks in a single domain, and requires human demonstrations to learn a policy.

\textcolor{black}{
\subsection{Using Task Features for Transfer}
The idea of using task features/dictionaries for zero-shot generalization has been explored previously in the context of image classification. \fullciteA{kodirov2015unsupervised} learn a joint feature embedding space between domains and also induce the corresponding projections onto this space from different class labels. 
\fullciteA{kolouri2018joint} learn a joint dictionary across visual features and class attributes using sparse coding techniques. \fullciteA{romera2015embarrassingly} model the relationship between input features, task attributes and classes as a linear model to achieve efficient yet simple zero-shot transfer for classification. \fullciteA{socher2013zero} learn a joint semantic representation space for images and associated words to perform zero-shot transfer.}

\textcolor{black}{
Task descriptors have also been explored in zero-shot generalization for control policies. \fullciteA{sinapov2015learning} use task meta-data as features to learn a mapping between pairs of tasks. This mapping is then used to select the most relevant source task to transfer a policy from. \fullciteA{isele2016using} build on the ELLA framework~\fullcite{ruvolo2013ella,ammar2014online}, and their key idea is to maintain two shared linear bases across tasks -- one for the policy ($L$) and the other for task descriptors ($D$). Once these bases are learned on a set of source tasks, they can be used to predict policy parameters for a new task given its corresponding descriptor. 
% The training scheme is similar to Actor-mimic scheme~\cite{parisotto2016actor} -- for each task, an expert policy is trained separately and then distilled into policy parameters dependent on the shared basis $L$. 
In these lines of work, the task features were either manually engineered or directly taken from the underlying system parameters defining the dynamics of the environment. In contrast, our framework only requires access to crowd-sourced textual descriptions, alleviating the need for expert domain knowledge.}





% A major difference in our work is that we utilize natural language descriptions of different environments to bootstrap transfer, requiring less exploration in the new task.

% using a policy distillation~\cite{parisotto2016actor,rusu2016progressive,yin2017knowledge} or selective attention over expert networks learnt in the source tasks~\cite{rajendran20172t}. Though these approaches provide some benefits, they still suffer from the requirement of efficiently exploring the new environment to learn how to transfer their existing policies. In contrast, we utilize natural language descriptions of different environments to bootstrap transfer, leading to more focused exploration in the target task. 


% Describe amn in detail





\label{section:related_work}

%\vspace{-1em}
\section{Preliminaries}
%\vspace{-1em}
\label{sec:preliminaries}
Datasets $X, X' \in \mathcal{X}$ are neighbors if they differ by no more than one datapoint -- i.e., $X \simeq X'$ if $d(X, X') \leq 1$. We will define $d(\cdot)$ to be the number of coordinates that differ between two datasets of the same size $n$: $d(X, Y) = \#\{i \in [n]: X_i \neq Y_i  \}$.

We use $||\cdot||$ to denote the radius of the smallest Euclidean ball that contains the input set, e.g. $||\mathcal{X}|| = \sup_{x \in \mathcal{X}} ||x||$.

The parameter $\phi$ denotes the privacy parameters associated with a mechanism (e.g. noise level, regularization). $\mathcal{M}_{\phi}$ is a mechanism parameterized by $\phi$.
For mechanisms with continuous output space, we will take $\text{Pr}[\mathcal{M}(X) = y]$ to be the probability density function of $\mathcal{M}(X)$ at $y$.



    \begin{definition}[Differential privacy \citep{dwork2006calibrating}] \label{def:dp}
        Fix $\epsilon, \delta \geq 0$. 
A randomized algorithm $\mathcal{M}: \mathcal{X} \rightarrow \mathcal{S}$ satisfies $(\epsilon, \delta)$-DP if for all neighboring datasets $X \simeq X'$ and for all measurable sets $S \subset \mathcal{S}$, 
            \[\text{Pr}\big[\mathcal{M}(X) \in S\big] \leq e^{\epsilon}\text{Pr}\big[\mathcal{M}(X') \in S\big] + \delta.\]
    \end{definition}
%\vspace{-3mm}
% We now define \emph{data-dependent} differential privacy that conditions on an input dataset $X$. 

% \begin{definition}[Data-dependent privacy\cite{papernot2018scalable}]\vspace{-1mm}
% \label{def:data_dep_dp}
% Suppose we have $\delta > 0$ and a function $\epsilon: \mathcal{X} \rightarrow \mathbb{R}$. We say that mechanism $\mathcal{M}$ satisfies ($\epsilon(X), \delta$) data-dependent DP for dataset $X$ if for all possible output sets $S$ and neighboring datasets $X'$,
% \begin{align*}
%     \text{Pr}\big[\mathcal{M}(X) \in S\big] &\leq e^{\epsilon(X)}\text{Pr}\big[\mathcal{M}(X') \in S\big] + \delta, \\
%       \text{Pr}\big[\mathcal{M}(X') \in S\big] &\leq e^{\epsilon(X)}\text{Pr}\big[\mathcal{M}(X) \in S\big] + \delta.
% \end{align*}
% \end{definition}

%\subsection{Additive Noise Mechanisms}
Suppose we wish to privately release the output of a real-valued function $f: \mathcal{X} \rightarrow \mathcal{R}$. We can do so by calculating the \emph{global sensitivity} $\Delta_{GS}$, calibrating the noise scale to the global sensitivity and then adding sampled noise to the output.



\begin{definition}[Local / Global sensitivity]
The local $\ell_*$-sensitivity of a function $f$ is defined as $\Delta_{LS}(X) = \max\limits_{X \simeq X'} || f(X) - f(X') ||_* $ and the global sensitivity of $f$ is $\Delta_{GS} = \sup_X \Delta_{LS}(X)$.
% The local $\ell_*$-sensitivity of a function $f: \cX \to \mathbb{R}^d$ is defined as $\Delta_{LS}(X) = \max\limits_{X \simeq X'} || f(X) - f(X') ||_* $ and the global sensitivity of $f$ is $\Delta_{GS} = \sup_X \Delta_{LS}(X)$.
\end{definition}
%\vspace{-2mm}
% The choice of $\ell_*$ depends on which kind of noise we use, e.g., $\ell_2$-norm is used for Gaussian noise.

\begin{comment}

\begin{definition}[Global sensitivity]
The global $\ell_*$-sensitivity of a function $f: \mathcal{X} \rightarrow \mathcal{R}^d$ is defined as
\begin{align*}
    \Delta_{GS} &= \max\limits_{X, X' \in \mathcal{X}:X \simeq X'} || f(X) - f(X') ||_*. 
\end{align*}
\end{definition}
\begin{definition}[Laplace mechanism]
The Laplace mechanism $\mathcal{M}: \mathcal{X} \rightarrow \mathbb{R}$ applied to a function $f$ is given as
\begin{align*}
    \mathcal{M}(X) &= f(X) + \text{Lap}\left(b \right).
\end{align*}
\end{definition}
\begin{theorem}
Suppose the function $f: \mathcal{X} \rightarrow \mathbb{R}$ has global $\ell_1$-sensitivity $\Delta_f$. Then the Laplace mechanism satisfies $\epsilon$-differential privacy with noise parameter $b = \Delta_f/\epsilon$.	 \vspace{-2mm}
\end{theorem}

\begin{definition}[Gaussian mechanism]
The Gaussian mechanism $\mathcal{M}: \mathcal{X} \rightarrow \mathbb{R}$ applied to a function $f$ is given as
\begin{align*}
    \mathcal{M}(X) &= f(X) + \mathcal{N}(0, \sigma^2).
\end{align*}
\end{definition}
\begin{theorem}
Suppose the function $f: \mathcal{X} \rightarrow \mathbb{R}$ has global $\ell_2$-sensitivity $\Delta_f$. Then the Gaussian mechanism satisfies $(\epsilon, \delta)$-differential privacy with noise parameter $\sigma = \Delta_f\sqrt{2 \log(1.25/\delta)}/\epsilon$.
\end{theorem}
Both the Laplace and Gaussian mechanisms generalize easily to releasing the output of a $d$-dimensional function $f$ by adding i.i.d. noise to each coordinate.
\begin{definition}[Local sensitivity]
The local $\ell_*$-sensitivity of a function $f: \mathcal{X} \rightarrow \mathbb{R}^d$ is defined as
\begin{align*}
    \Delta_{LS}(X) &= \max\limits_{X \simeq X'} || f(X) - f(X') ||_*. 
\end{align*}
\end{definition}
\end{comment}
% \todo{Define Global sensitivity}
%define global/local sensitivity
%Laplace, Gaussian mech
% explain how related to PTR and its generalization

%\vspace{-0.em}
\subsection{Propose-Test-Release}
%\vspace{-0.5em}
Calibrating the noise level to the local sensitivity $\Delta_{LS}(X)$ of a function would allow us to add less noise and therefore achieve higher utility for releasing private queries. However, the local sensitivity is a data-dependent function and na\"ively calibrating the noise level to $\Delta_{LS}(X)$ will not satisfy DP.

PTR resolves this issue in a three-step procedure: \textbf{propose} a bound on the local sensitivity, privately \textbf{test} that the bound is valid (with high probability), and if so calibrate noise according to the bound and \textbf{release} the output.

% \begin{figure}[t]
% \vspace{-1em}
% \centering
% \resizebox{0.95\columnwidth}{!}{%
% \begin{minipage}{0.50\textwidth}
% \begin{algorithm}[H]
% \caption{Propose-Test-Release \cite{dwork2009differential}}
% \label{alg:classic_ptr}
% \begin{algorithmic}[1]
% \STATE{\textbf{Input}: Dataset $X$; privacy parameters $\epsilon,\delta$; proposed bound $\beta$ on $\Delta_{LS}(X)$; query function $f: \mathcal{X} \rightarrow \mathbb{R}$.}
% \STATE{\textbf{Output}: $f^P(X)$ or $\perp$.}
% \STATE{Compute the distance $\gamma(X)$ to the nearest dataset $X''$ such that $ \Delta_{LS}(X'')> \beta$:
% $\gamma(X) = \min\limits_{X''} \{ \text{dist}(X, X''): \Delta_{LS}(X'')> \beta \}$.}
% \STATE{Privately release $\gamma^P(X) = \gamma(X) + \text{Lap}\left(\frac{1}{\epsilon}\right)$.}
% \IF{$\gamma^P(X) > \dfrac{\log(1/\delta)}{\epsilon}$}\vspace{-1pt}
% \STATE{Release $f^P(X) = f(X) + \text{Lap}\left(\frac{\beta}{\epsilon}\right)$.}\vspace{-2pt}
% \ELSE
% \STATE{Output $\perp$.}\vspace{-1pt}
% \ENDIF
% \end{algorithmic}
% \end{algorithm}
% \end{minipage}
% \quad
% \begin{minipage}{0.46\textwidth}
% \begin{algorithm}[H]
% \caption{Generalized PTR}
% \label{alg:gen_ptr}
% \begin{algorithmic}[1]
% \STATE{{Input}: Proposed~parameter~$\phi$;~privacy~parameters~$\epsilon,  \hat{\epsilon}, \hat{\delta}$; dataset $X$;\blue{an $(\epsilon,\delta)$~DP test $\cT$; ~data-dependent~DP~function~$\epsilon_{\phi}(\cdot, \hat{\delta})$;~mechanism~$\mathcal{M}_{\phi}$.}}
% \STATE{\textbf{Output}: 
% $\mathcal{M}_{\phi}(X)$ or $\perp$.}
% \STATE{Let $\cT$ privately test if $\epsilon_\phi(X,\hat{\delta}) \leq \hat{\epsilon}$.}% with privacy limit $(\hat{\epsilon}, \hat{\delta})$ }.
% \IF{the test $\cT$ passes}
% \vspace{1pt}
% \STATE{Run $\theta = \mathcal{M}_{\phi}(X)$ and output $\theta$.}\vspace{2pt}

% \ELSE \vspace{2pt}

% \STATE{Output $\perp$.}\vspace{2pt}

% \ENDIF
% \end{algorithmic}
% \end{algorithm}
% \end{minipage}
% }
% \vspace{-1em}
% \end{figure}


% \begin{theorem} 
% Algorithm~\ref{alg:classic_ptr} satisfies ($2 \epsilon, \delta$)-DP.
% \cite{dwork2009differential}
% \end{theorem}

PTR privately computes the distance $\cD_{\beta}(X)$ between the input dataset $X$ and the nearest dataset $X''$ whose local sensitivity exceeds the proposed bound $\beta$:
\begin{align*}
    \cD_{\beta}(X) = \min\limits_{X''} \{ d(X, X''): \Delta_{LS}(X'')> \beta \}.
\end{align*}
%\vspace{-.8em}
% The $\epsilon$-DP "test" fails (with probability $\delta$) if PTR decides to release $f^P(X)$ when $\gamma(X) = 0$, i.e. when dataset $X$ has local sensitivity greater than $\beta$.

\begin{figure}[H]
\vspace{-1.4em}
\centering
% \resizebox{0.95\columnwidth}{!}{%
% \begin{minipage}{0.54\textwidth}
\begin{algorithm}[H]
\caption{Propose-Test-Release \citep{dwork2009differential}}
\label{alg:classic_ptr}
\begin{algorithmic}[1]
\STATE{\textbf{Input}: Dataset $X$; privacy parameters $\epsilon,\delta$; proposed bound $\beta$ on $\Delta_{LS}(X)$; query function $f: \mathcal{X} \rightarrow \mathbb{R}$.}
% \STATE{\textbf{Output}: $f^P(X)$ or $\perp$.}
% \STATE{Compute the distance $\gamma(X)$ to the nearest dataset $X''$ such that $ \Delta_{LS}(X'')> \beta$.}
% \STATE{Privately release $\gamma^P = \gamma(X; \beta) + \text{Lap}\left(\frac{1}{\epsilon}\right)$.}

\STATE{\textbf{if} $\cD_{\beta}(X) + \text{Lap}\left(\frac{1}{\epsilon}\right) \leq \frac{\log(1/\delta)}{\epsilon}$ \textbf{then} output $\perp$,}
\STATE{\textbf{else} release $f(X) + \text{Lap}\left(\frac{\beta}{\epsilon}\right)$.}
\end{algorithmic}
\end{algorithm}
\end{figure}
%\vspace{-.8em}
\begin{theorem} 
Algorithm~\ref{alg:classic_ptr} satisfies ($2 \epsilon, \delta$)-DP.
\citep{dwork2009differential}
\end{theorem}
%\vspace{-1em}
Rather than proposing an arbitrary threshold $\beta$, one can also privately release an upper bound of the local sensitivity and calibrate noise according to this upper bound. This was used for node DP in graph statistics \citep{kasiviswanathan2013analyzing}, and for fitting topic models using spectral methods \citep{decarolis2020end}.

%This gives a more efficient alternative and avoids the need to propose $\beta$. This variant is  the local sensitivity itself has a global sensitivity.

%there exist other variants of PTR --- e.g., compute a differentially private upper bound of the local sensitivity and calibrate noise according to this upper bound. This type of PTR requires a global sensitivity of the local sensitivity. We refer readers to the excellent summary of PTR in  section 3 of \citet{vadhan2017complexity}.

% We may mention other types of PTR: propose and release
% There are other variants of PTR... 
% \vspace{-1mm}
% \subsection{Motivation}
% \vspace{-1mm}
% Why do we want to generalize PTR beyond noise-adding mechanisms? For other mechanisms, the local sensitivity either does not exist or is only defined for specific data-dependent quantities (e.g., the sensitivity of the score function in the exponential mechanism) rather than the mechanism's output. We give a concrete example below. 

%In this section, we give a concrete example to demonstrate this limitation and motivate our generalization.
%This section discusses a few limitations of PTR approaches that motivated our work.
%Let us first ask, ``is PTR a general framework applicable to any mechanism with a data-dependent analysis?'' If so, we could explore other less costly approaches to privately test the local sensitivity.

%However, the answer is unfortunately ``no''.  The reasons are twofold. First, the framework above applies only to ``noise-adding'' mechanisms --- where we have a well-defined local sensitivity (of the output), and the noise scale is calibrated according to that. For other non-noise-adding mechanisms, the local sensitivity either does not exist or is only defined for specific data-dependent quantities (e.g., the sensitivity of the score function in the exponential mechanism) rather than the mechanism's output. Consider the difficulties of applying PTR to the following example.


% \begin{example}[Private posterior sampling]\label{exp: posterior}
% Let $\cM: \cX\times \cY \to \Theta $ be a private posterior sampling   mechanism~\citep{minami2016differential,wang2015privacy,gopi2022private} for approximately minimizing $F_{X}(\theta)$. % for linear regression problem, i.e., $\min_{\theta} \frac{1}{2}||y-X\theta||^2 + \lambda ||\theta||^2$. 
% $\cM$ samples $\theta \sim P(\theta)\propto e^{-\gamma(F_X(\theta)+ 0.5\lambda ||\theta||^2)}$ with parameters $\gamma, \lambda$. $\gamma,\lambda$ cannot be appropriately chosen for this mechanism to satisfy DP without going through a sensitivity calculation of $\arg\min F_X(\theta)$. In fact, the global and local sensitivity of the minimizer is unbounded even in linear regression problems, i.e., when $F_X(\theta) = \frac{1}{2}||y-X\theta||^2.$ 
% %The local sensitivity $\Delta:=||P_{X,y}(\theta)-P_{X', y'}(\theta)||$ is not well-defined  for the sampling algorithm, thus the standard PTR is not applicable.  
% \end{example}
% Output perturbation algorithms do work for the above problem when we regularize, but they are known to be suboptimal in theory and in practice \cite{chaudhuri2011differentially}.% do not achieve the level of utility in theory and in practice when comparing to posterior sampling. 
 

% Moreover, even in the cases of noise-adding mechanisms where PTR seems to be applicable, it does not lead to a tight privacy guarantee. Specifically, by an example of privacy amplification by post-processing (Example~\ref{exp: binary_vote} in the appendix), we demonstrate that the local sensitivity does not capture all sufficient statistics for data-dependent privacy analysis and thus is loose.

% Instead of identifying sufficient statistics of each mechanism, we develop a unified framework --- generalized PTR, offering the flexibility for any mechanism to exploit data-dependent quantities.

% \textbf{On data-dependent DP losses.} In addition to the above, there has been an increasing list of empirical DP work that fix the parameters of a randomized algorithm while reporting the resulting data-dependent DP losses $\epsilon(\text{Data})$ after running on a specific dataset \citep{ligett2017accuracy,papernot2018scalable,zhu2020private, feldman2021individual}. The data-dependent DP losses are often smaller than the worst-case DP losses, but technically speaking, these algorithms are not formally DP with DP guarantees any smaller than that of the worst-case. In addition, the data-dependent DP losses themselves are sensitive, and thus cannot be reported. A typical solution is to privately release $\epsilon(\text{Data})$, but it still does not satisfy DP as this would require a prescribed $(\epsilon,\delta)$-DP parameter to be satisfied for all input datasets. Part of our contribution is to resolve this conundrum by showing that a simple post-processing step of the privately released upper bound of $\epsilon(\text{Data})$ gives a formal DP algorithm.
%\yq{Shall we combine this part with the related work section?}

%Instead,  exploits data-dependent quantities by first privately choosing $\gamma, \lambda$ adapted to the dataset and then applying posterior sampling with the sanitized parameters.  

\label{section:preliminary}

%\vspace{-1em}
\section{Generalized PTR}
%\vspace{-1em}
\label{sec:gen_ptr}
This section introduces the generalized PTR framework. We first formalize the notion of \emph{data-dependent} differential privacy that conditions on an input dataset $X$. 

\begin{definition}[Data-dependent privacy]
\label{def:data_dep_dp}
Suppose we have $\delta > 0$ and a function $\epsilon: \mathcal{X} \rightarrow \mathbb{R}$. We say that mechanism $\mathcal{M}$ satisfies ($\epsilon(X), \delta$) data-dependent DP\footnote{We will sometimes write that $\cM(X)$ satisfies $\epsilon(X)$ data-dependent DP with respect to $\delta$.} for dataset $X$ if for all possible output sets $S$ and neighboring datasets $X'$,
\begin{align*}
    \text{Pr}\big[\mathcal{M}(X) \in S\big] &\leq e^{\epsilon(X)}\text{Pr}\big[\mathcal{M}(X') \in S\big] + \delta, \\
       \text{Pr}\big[\mathcal{M}(X') \in S\big] &\leq e^{\epsilon(X)}\text{Pr}\big[\mathcal{M}(X) \in S\big] + \delta.
\end{align*}
\end{definition}

In generalized PTR, we propose a value $\phi$ for the randomized algorithm $\cM$, which could be a noise scale or regularization parameter -- or a set including both. For example, $\phi = (\lambda, \gamma)$ in Example~\ref{exp: posterior}. We then say that $\cM_{\phi}$ is the mechanism $\mathcal{M}$ parameterized by $\phi$, and $\epsilon_{\phi}(X)$ its data-dependent DP.

The following example illustrates how to derive the data-dependent DP for a familiar friend -- the Laplace mechanism.

\begin{example}(\emph{Data-dependent DP of Laplace Mechanism.}) \label{examp:lap_mech}
Given a function $f: \mathcal{X} \rightarrow \mathbb{R}$, we will define
\begin{align*}
    \mathcal{M}_{\phi}(X) = f(X) + \text{Lap}\left(\phi\right).
\end{align*}
We then have
\begin{align*}
    \log \dfrac{\emph{Pr}[\mathcal{M}_{\phi}(X) = y]}{\emph{Pr}[\mathcal{M}_{\phi}(X') = y]} &\leq \dfrac{|f(X) - f(X')|}{\phi}.
\end{align*}
Maximizing the above calculation over all possible outputs $y$ and using Definition~\ref{def:data_dep_dp},
\begin{align*}
    \epsilon_{\phi}(X) = \max\limits_{X': X' \simeq X} \frac{|f(X) - f(X')|}{\phi} = \frac{\Delta_{LS}(X)}{\phi}.
\end{align*}
% We can then verify that choosing $\phi = \beta/\hat{\epsilon}$ and $\hat{\epsilon} = \epsilon$ reduces Algorithm~\ref{alg:no_ls} exactly to Algorithm~\ref{alg:classic_ptr}.
\end{example}


The data-dependent DP $\epsilon_\phi(X)$ is a function of both the dataset $X$ and the parameter $\phi$. Maximizing $\epsilon_\phi(X)$ over $X$ recovers the standard DP guarantee of running $\cM$ with parameter $\phi$.


 %We take the functional view of $\epsilon_\phi(X)$ and use $\epsilon_{\phi, \delta}(X)$ to denote the $\epsilon$ as a function of $\delta$.
  % Add what is data-dependent dp, it's a function representation

 
%  Suppose we are given an $(\epsilon, \delta)$-DP test $\mathcal{T}$ that fails with probability $\Tilde{\delta}$ in order to determine whether or not $\mathcal{M}_{\phi}$ satisfies ($\hat{\epsilon}, \hat{\delta}$)-DP. Based on the outcome of $\mathcal{T}$,  we decide whether to run $\mathcal{M}_{\phi}$ or to halt and return $\perp$.
 
\begin{figure}[H]
%\vspace{-1em}
\centering
% \resizebox{0.95\columnwidth}{!}{%
% \begin{minipage}{0.54\textwidth}
\begin{algorithm}[H]
\caption{Generalized Propose-Test-Release}
\label{alg:gen_ptr}
\begin{algorithmic}[1]
\STATE{\textbf{Input}: Dataset $X$; mechanism $\cM_\phi: \mathcal{X} \rightarrow \cR$ and its privacy budget $\epsilon, \delta$; $(\hat{\epsilon}, \hat{\delta})$-DP test $\mathcal{T}$; false positive rate $\leq \delta'$; data-dependent DP function $\epsilon_\phi(\cdot)$ w.r.t. $\delta$}.
% \STATE{\blue{Run a test $\mathcal{T}$ to determine if $\cM_\phi$ is $(\hat{\epsilon}, \hat{\delta})$-DP at $X$.}}
% \STATE{\textbf{if} the test $\mathcal{T}$ passes \textbf{then} release $\theta = \mathcal{M}_{\phi}(X)$.}\vspace{-1pt}
\STATE{\textbf{if not} $\mathcal{T}(\cX)$ \textbf{then} output $\perp$,}\vspace{-1pt}
\STATE{\textbf{else} release $\theta = \mathcal{M}_{\phi}(X).$}%\vspace{-2pt}
\end{algorithmic}
\end{algorithm}
\end{figure}

 
\begin{theorem}[Privacy guarantee of generalized PTR]
\label{thm:gen_ptr}
Consider a proposal $\phi$ and a data-dependent DP function $\epsilon_{\phi}(X)$ w.r.t. $\delta$. Suppose that we have an ($\hat{\epsilon}, \hat{\delta}$)-DP test $\cT: \cX \rightarrow \{0, 1\}$ such that when $\epsilon_{\phi}(X) > \epsilon$, \end{theorem}
\vspace{-8mm}
\begin{align*}
    \cT(X) =
    \begin{cases}
        0  \text{ \:with probability } 1 - \delta', \\
        1  \text{\: with probability }  \delta'. %\vspace{-2mm}
    \end{cases}
\end{align*}
%\vspace{-2mm}
\textit{Then Algorithm~\ref{alg:gen_ptr} satisfies ($\epsilon + \hat{\epsilon}, \delta +  \hat{\delta} + \delta'$)-DP. }
% \vspace{-2mm}
%  \end{theorem}
 \begin{proof}[Proof sketch]
 There are three main cases to consider:
\begin{enumerate}
    %\vspace{-1em}
    \item We decide not to run $\mathcal{M}_{\phi}$.   % \vspace{-0.5em}
    \item We decide to run $\mathcal{M}_{\phi}$ and $\epsilon_{\phi}(X) > \epsilon$;    %\vspace{-0.5em}
    \item We decide to run $\mathcal{M}_{\phi}$ and $\epsilon_{\phi}(X) \leq \epsilon$.    %\vspace{-0.5em}
\end{enumerate}%\vspace{-2mm}
In the first case, the decision to output $\perp$ is post-processing of an $(\hat{\epsilon}, \hat{\delta})$-DP mechanism and inherits its privacy guarantees. The second case occurs when the $(\hat{\epsilon}, \hat{\delta})$-DP test "fails" (produces a false positive) and occurs with probability at most $\delta'$. The third case is a composition of an $(\hat{\epsilon}, \hat{\delta})$-DP algorithm and an ($\epsilon, \delta$)-DP algorithm.
 \end{proof}
 
% The construction of the test required by Theorem~\ref{thm: gen_ptr}
% can be flexibly adapted to any use case. We outline two  approaches to construct such a test in Sec~\ref{section:applications}. 

% \blue{Theorem~\ref{thm: gen_ptr} requires a private test $\mathcal{T}$ to be clear.   The data-dependent privacy loss might depend on the dataset in different ways, and thus the derivation of the test requires a case-specific study. We outline two approaches to construct such a test in Sec~\ref{section:applications}. }
% Before we reach that point, we first show that the classic PTR (Algorithm~\ref{alg:classic_ptr}) can be viewed under our framework as Algorithm~\ref{alg:gen_ptr} instantiated with a specific type of test.
% \begin{theorem}[Generalized PTR with distance test]\label{exp: dist_ptr}
% Let $\delta, \delta' > 0$ and define an algorithm $\cA: \cR^d \to \{\perp, \cR^d\}$ as follows.
% Let $\gamma(X)$ denote the distance to the nearest dataset $X''$ such that $\epsilon_{\phi, \hat{\delta}}(X'')>\hat{\epsilon}$, i.e.,
% $\gamma(X) = \min_{X''} \{dist(X, X''):\epsilon_{\phi, \hat{\delta}}(X'')>\hat{\epsilon}\}$. Let $\mathcal{T}$ test whether $ \gamma(X)+ Lap(1/\epsilon)>\frac{\log(1/\tilde{\delta})}{\epsilon}$.
%   $\cA$ returns $\cM_\phi(X)$ if $\mathcal{T}$ passes and otherwise returns $\perp$. Then $\cA$ is $(\epsilon+\hat{\epsilon}, \hat{\delta}+ \tilde{\delta})$-DP. 
% \end{theorem}
% \blue{Look at this again! Are the deltas correct?}



Generalized PTR is a \emph{strict} generalization of Propose-Test-Release. For some function $f$, define  $\cM_{\phi}$ and $\cT$ as follows:
\begin{align*}
&\mathcal{M}_{\phi}(X) = f(X) + \text{Lap}(\phi); \\
&\cT(X) = 
\begin{cases}
0 & \text{ if\:\: } \cD_{\beta}(X) + \text{Lap}\left(\frac{1}{\epsilon}\right) > \frac{\log(1/\delta)}{\epsilon},\\
1 & \text{ otherwise.} \\
\end{cases}
\end{align*}
Notice that our choice of parameterization is $\phi = \frac{\beta}{\epsilon}$, where $\phi$ is the scale of the Laplace noise. In other words, we know from Example~\ref{examp:lap_mech} that $\epsilon_{\phi}(X) > \epsilon$ exactly when $\Delta_{LS}(X) > \beta$.

For noise-adding mechanisms such as the Laplace mechanism, the sensitivity is proportional to the privacy loss (in both the global and local sense, i.e. $\Delta_{GS} \propto \epsilon$ and $\Delta_{LS} \propto \epsilon(X)$). Therefore for these mechanisms the only difference between privately testing the local sensitivity (Algorithm~\ref{alg:classic_ptr}) and privately testing the data-dependent DP (Theorem~\ref{thm:gen_ptr}) is a change of parameterization.
%\vspace{-0.5em}
\subsection{Limitations of local sensitivity}
%\vspace{-1em}
Why do we want to generalize PTR beyond noise-adding mechanisms? Compared to classic PTR, the generalized PTR framework allows us to be more flexible in both the type of test conducted and also the type of mechanism whose output we wish to release. For many mechanisms, the local sensitivity either does not exist or is only defined for specific data-dependent quantities (e.g., the sensitivity of the score function in the exponential mechanism) rather than the mechanism's output. 

The following example illustrates this issue.

\begin{example}[Private posterior sampling]\label{exp: posterior}
Let $\cM: \cX\times \cY \to \Theta $ be a private posterior sampling   mechanism~\citep{minami2016differential,wang2015privacy,gopi2022private} for approximately minimizing $F_{X}(\theta)$.

$\cM$ samples $\theta \sim P(\theta)\propto e^{-\gamma(F_X(\theta)+ 0.5\lambda ||\theta||^2)}$ with parameters $\gamma, \lambda$. Note that $\gamma,\lambda$ cannot be appropriately chosen for this mechanism to satisfy DP without going through a sensitivity calculation of $\arg\min F_X(\theta)$. In fact, the global and local sensitivity of the minimizer is unbounded even in linear regression problems, i.e when $F_X(\theta) = \frac{1}{2}||y-X\theta||^2.$ 
%The local sensitivity $\Delta:=||P_{X,y}(\theta)-P_{X', y'}(\theta)||$ is not well-defined  for the sampling algorithm, thus the standard PTR is not applicable.  

\end{example}
Output perturbation algorithms do work for the above problem when we regularize, but they are known to be suboptimal in theory and in practice \citep{chaudhuri2011differentially}. In Section~\ref{subsections:private_linear_regression} we demonstrate how to apply generalized PTR to achieve a data-adaptive posterior sampling mechanism.

Even in the cases of noise-adding mechanisms where PTR seems to be applicable, it does not lead to a tight privacy guarantee. Specifically, by an example of privacy amplification by post-processing (Example~\ref{exp: binary_vote} in the appendix), we demonstrate that the local sensitivity does not capture all sufficient statistics for data-dependent privacy analysis and thus is loose.

% We've just seen an example where the local sensitivity is unbounded. In other cases, the local sensitivity exists and can be tested efficiently -- \emph{but} isn't sufficiently descriptive to make full use of data-dependent properties. Our next example demonstrates that considering only the local sensitivity leads to a loose privacy analysis.

% \begin{example}\label{exp: binary_vote}
% Consider a binary class voting problem: $n$ users vote for a binary class $\{0, 1\}$ and the goal is to output the class that is supported by the majority. Let $n_i$ denote the number of people who vote for the class $i$. We consider the report-noisy-max mechanism: $\cM(X): \text{argmax}_{i \in [0,1]} n_i(X)+ Lap(1/\epsilon)$,where $1/\epsilon$ denotes the scale of Laplace noise.

% For a dataset $X$, the gap between number of votes in each class is $t(X) = |n_0(X) - n_1(X)|$. Observe that if $t(X) > 1$, adding or removing one user will not change the majority class -- in other words, for a neighboring dataset $X'$ we have $\text{argmax}_{i \in [0,1]} n_i(X) = \text{argmax}_{i \in [0,1]} n_i(X')$. The local sensitivity $\Delta_{LS}(X)$ of the report-noisy-max mechanism is therefore $0$ if $t(X) >1$. 
% \end{example}
\label{subsections:local_sensitivity}
%\vspace{-1em}
\subsection{Which $\phi$ to propose}
%\vspace{-0.5em}
% upper bound apporach and hyperparameter tuning
The main limitation of generalized PTR is that one needs to ``propose'' a good guess of parameter $\phi$.  Take the example of $\phi$ being the noise level in a noise-adding mechanism. Choosing too small a $\phi$ will result in a useless output $\perp$, while choosing too large a $\phi$ will add more noise than necessary. Finding this 'Goldilocks' $\phi$ might require trying out many different possibilities -- each of which will consume privacy budget.

% Calculating the data-dependent DP offers a systematic way to find this 'Goldilocks' $\phi$.


\label{subsections:which_phi}

This section introduces a method to jointly tune privacy parameters (e.g., noise scale) along with parameters related only to the utility of an algorithm (e.g., learning rate or batch size in stochastic gradient descent) -- while avoiding the $\perp$ output.

Algorithm~\ref{alg: parameter_ptr} takes a list of parameters as input, runs generalized PTR with each of the parameters, and returns the output with the best utility. We show that the privacy guarantee with respect to $\epsilon$ is independent of the number of $\phi$ that we try.  

Formally, let $\phi_1, ..., \phi_k$ be a set of hyper-parameters and $\tilde{\theta}_i \in\{\perp, \text{Range}(\cM)\}$ denotes the output of running generalized PTR on a private dataset $X$ with $\phi_i$. 
Let $X_{val}$ be a public validation set and $q(\tilde{\theta}_i)$ be the score of evaluating $\tilde{\theta}_i$ with $X_{val}$ (e.g., validation accuracy). The goal is to select a pair $(\tilde{\theta}_i$, $\phi_i)$ such that DP model $\tilde{\theta}_i$ maximizes the validation score.

The generalized PTR framework with privacy calibration is described in Algorithm~\ref{alg: parameter_ptr}. The privacy guarantee of Algorithm~\ref{alg: parameter_ptr} is an application of \citet{liu2019private}.


\begin{algorithm}[H]
	\caption{PTR with hyper-parameter selection}
	\label{alg: parameter_ptr}
	\begin{algorithmic}[1]
	   \STATE {\textbf{Input}:  Privacy budget per PTR algorithm ($\epsilon^*, \delta^*$), cut-off $T$, parameters $\phi_{1:k}$, flipping probability $\tau$ and validation score function $q(\cdot)$. } 
		\STATE {Initialize the set $S=\varnothing$.}
		\STATE{Draw $G$ from a geometric distribution $\cD_\tau$ and let $\hat{T}=\text{min}(T, G)$.}
		\FOR{i = 1 ,..., $\hat{T}$}
		\STATE{ pick a random $\phi_i$ from $\phi_{1:k}$.}
		\STATE{evaluate $\phi_i$: $(\tilde{\theta}_i, q(\tilde{\theta}_i))\gets$ Algorithm~\ref{alg:gen_ptr}($\phi_i, (\epsilon^*, \delta^*)$).}
		\STATE {$S \gets S \cup \{\tilde{\theta}_i, q(\tilde{\theta}_i)\}$.}
		%\STATE{with probability $\gamma$, we output the highest scored candidate from $S$ and halt.}
		\ENDFOR %\vspace{-1mm}
	\STATE{Output the highest scored candidate from $S$.}
	\end{algorithmic}
	%\vspace{-1mm}
\end{algorithm}

\begin{theorem}[ Theorem 3.4 \citet{liu2019private} ]
Fix any $\tau \in [0, 1], \delta_2>0$ and let $T =\frac{1}{\tau} \log \frac{1}{\delta_2}$. If each oracle access to Algorithm~\ref{alg:gen_ptr} is $(\epsilon^*, \delta^*)$-DP, then
Algorithm~\ref{alg: parameter_ptr} is $(3\epsilon^* + 3\sqrt{2\delta^*}, \sqrt{2\delta^*} T +\delta_2 )$-DP.%\yw{What if it does not reach $T$? Also in the algorithm it has $k$ right?  Also, did you define which score it is?}
\end{theorem}
The theorem implies that one can try a random number of $\phi$ while paying a constant $\epsilon$.
In practice, we can roughly set $\tau = \frac{1}{10k}$ so that the algorithm is likely to test all $k$ parameters. We emphasize that the privacy and the utility guarantee (stated in the appendix) is not our contribution. But the idea of applying generalized PTR to enforce a uniform DP guarantee over all choices of parameters with a data-dependent analysis is new, and in our opinion, significantly broadens the applicability to generic hyper-parameter tuning machinery from \citet{liu2019private}.
%\vspace{-0.5em}
\subsection{Construction of the DP test}

Classic PTR uses the Laplace mechanism to construct a differentially private upper bound of $\cD_{\beta}(X)$, the distance from input dataset $X$ to the closest dataset whose local sensitivity exceeds the proposed bound $\beta$. The tail bound of the Laplace distribution then ensures that if $\cD_{\beta}(X) = 0$ (i.e. if $\Delta_{LS}(X) > \beta$), then the output will be released with only a small probability $\delta$.

The following theorem shows that we could instead use a differentially private upper bound of the data-dependent DP $\epsilon_{\phi}(X)$ in order to test whether to run the mechanism $\cM_{\phi}$.

\begin{theorem}[Generalized PTR with private upper bound]\label{exp: upperbound}
Suppose we have a differentially private upper bound of $\epsilon_\phi(X)$ w.r.t. $\delta$ such that with probability at least $1-\delta'$, $\epsilon_{\phi }^P(X)>\epsilon_{\phi}(X)$. Further suppose we have an $(\hat{\epsilon}, \hat{\delta})$-DP test $\cT$ such that
\begin{align*}
    T(X) &= \begin{cases}
    1 & \text{ if } \epsilon_{\phi }^P(X) < \epsilon, \\
    0 & \text{ otherwise}.
    \end{cases}
    %\vspace{-1em}
\end{align*}
%\yw{Why do you need it to hold for all $\tilde{\delta}$? I thought we only need $\hat{\delta}$}

Then Algorithm~\ref{alg:gen_ptr} is $(\epsilon +\hat{\epsilon}, \delta +\hat{\delta} + \delta')$-DP. %\vspace{-0.5em}
\end{theorem}

%\vspace{-0.5em}

In Section~\ref{subsections:pate}, we demonstrate that one can upper bound the data-dependent DP through a modification of the smooth sensitivity framework applied on $\epsilon_\phi(X)$. Moreover, in Section~\ref{subsections:private_linear_regression} we provide a direct application of Theorem~\ref{exp: upperbound} with private linear regression by making use of the per-instance DP technique~\citep{wang2017per}.

The applications in Section~\ref{sections:applications} are illustrative of two distinct approaches to constructing the DP test for generalized PTR:

\begin{enumerate}
%\vspace{-0.5em}
    \item Private sufficient statistics release (used in the private linear regression example of Section~\ref{subsections:private_linear_regression}) specifies the data-dependent DP as a function of the dataset and privately releases each data-dependent component. %\vspace{-0.5em}
    \item The second approach (used in the PATE example of Section~\ref{subsections:pate}) uses the smooth sensitivity framework to privately release the data-dependent DP as a whole, and then construct a high-confidence test using the Gaussian mechanism. %\vspace{-0.5em}
    
\end{enumerate}
%\vspace{-0.5em}
These two approaches cover most of the scenarios arising in data-adaptive analysis. For example, in the appendix we demonstrate the merits of generalized PTR in handling data-adaptive private generalized linear models (GLMs)  using private sufficient statistics release. Moreover, sufficient statistics release together with our private hyper-parameter tuning (Algorithm~\ref{alg: parameter_ptr}) can be used to construct data-adaptive extensions of DP-PCA and Sparse-DP-ERM (see details in the future work section).




\label{subsections:test_construction}

%  Next, we provide another instantiation of generalized PTR through the construction of a high-probability upper bound of $\epsilon_{\phi}(X)$.








% \blue{Compared to  classic PTR, the key idea of the procedure above is that we can construct a ``sanitized'' upper bound of $\epsilon_\phi(X)$ instead of local sensitivity, which always exists.  Moreover, the above framework allows us to be more flexible in both the type of test conducted and also the type of mechanism whose output we wish to release.}
% \blue{Put the above into the 'Unified Framework section?'}


%In particular, our result suggests that one can privately publish any mechanism $\cM$ with a data-dependfent analysis through publishing $\epsilon_\phi(X)$, which captures sufficient statistics of the dataset, and is less costly as it is a real-valued number rather than a $d$-dimensional model $\cM$.
% \begin{figure}[t]
% \vspace{-1em}
% \centering
% \resizebox{0.95\columnwidth}{!}{%
% \begin{minipage}{0.54\textwidth}
% \begin{algorithm}[H]
% \caption{Propose-Test-Release \cite{dwork2009differential}}
% \label{alg:classic_ptr}
% \begin{algorithmic}[1]
% \STATE{\textbf{Input}: Dataset $X$; privacy parameters $\epsilon,\delta$; proposed bound $\beta$ on $\Delta_{LS}(X)$; query function $f: \mathcal{X} \rightarrow \mathbb{R}$.}
% \STATE{\textbf{Output}: $f^P(X)$ or $\perp$.}
% \STATE{Compute the distance $\gamma(X)$ to the nearest dataset $X''$ such that $ \Delta_{LS}(X'')> \beta$:
% $\gamma(X) = \min\limits_{X''} \{ \text{dist}(X, X''): \Delta_{LS}(X'')> \beta \}$.}
% \STATE{Privately release $\gamma^P(X) = \gamma(X) + \text{Lap}\left(\frac{1}{\epsilon}\right)$.}
% \STATE{\textbf{if} $\gamma^P(X) \leq \dfrac{\log(1/\delta)}{\epsilon}$ \textbf{then} output $\perp$}\vspace{-1pt}
% \STATE{\textbf{else} release $f^P(X) = f(X) + \text{Lap}\left(\frac{\beta}{\epsilon}\right)$.}\vspace{-2pt}
% \end{algorithmic}
% \end{algorithm}
% \end{minipage}
% \quad
% \begin{minipage}{0.45\textwidth}
% \begin{algorithm}[H]
% \caption{Generalized PTR}
% \label{alg:gen_ptr}
% \begin{algorithmic}[1]
% \STATE{{Input}: Proposed~parameter~$\phi$;~privacy~parameters~$\epsilon,  \hat{\epsilon}, \hat{\delta}$; dataset $X$;\blue{an -$(\epsilon,\delta)$~DP test $\cT$ with false positive rate at most $\tilde{\delta}$; ~data-dependent~DP~function~$\epsilon_{\phi}(\cdot, \hat{\delta})$;~mechanism~$\mathcal{M}_{\phi}$.}}
% \STATE{\textbf{Output}: 
% $\mathcal{M}_{\phi}(X)$ or $\perp$.}
% \STATE{\blue{Run a test $\cT$ to determine if $\cM_{\phi}$ is $(\hat{\epsilon}, \hat{\delta})$-DP at $X$.}}% with privacy limit $(\hat{\epsilon}, \hat{\delta})$ }.
% \IF{the test $\cT$ passes}
% \vspace{1pt}
% \STATE{Run $\theta = \mathcal{M}_{\phi}(X)$ and output $\theta$.}%\vspace{2pt}
% \ELSE 
% \STATE{Output $\perp$.}%\vspace{2pt}

% \ENDIF
% \end{algorithmic}
% \end{algorithm}
% \end{minipage}
% }
% \vspace{-1em}
% \end{figure}







% \begin{comment}
% \begin{example}(\emph{Laplace Mechanism}) \label{examp:lap_mech}

% Given a function $f: \mathcal{X} \rightarrow \mathbb{R}$, we will define
% \begin{align*}
%     \mathcal{M}_{\phi}(X) = f(X) + \text{Lap}\left(\phi\right).
% \end{align*}
% % We'll now derive the data-dependent DP function $\epsilon_{\phi}(X)$ associated with $\mathcal{M}_{\phi}$:
% A rote calculation with the Laplace distribution tells us that
% \begin{align*}
%     \log \dfrac{\text{Pr}[\mathcal{M}_{\phi}(X) = y]}{\text{Pr}[\mathcal{M}_{\phi}(X') = y]} &\leq \dfrac{|f(X) - f(X')|}{\phi}.
% \end{align*}
% Using the above calculation and  Definition~\ref{def:data_dep_dp},
% \begin{align*}
%     \epsilon_{\phi}(X) = \max\limits_{X': X' \simeq X} \frac{|f(X) - f(X')|}{\phi} = \frac{\Delta_{LS}(X)}{\phi}.
% \end{align*}
% We can then verify that choosing $\phi = \beta/\hat{\epsilon}$ and $\hat{\epsilon} = \epsilon$ reduces Algorithm~\ref{alg:no_ls} exactly to Algorithm~\ref{alg:classic_ptr}.
% \end{example}
% \end{comment}



% \begin{remark}[Remove distance test]
% The distance test allows Example~\ref{exp: dist_ptr} to exploit properties of the local neighborhood of datasets close to the input dataset $X$. If $X$ is far away from the closest dataset whose local sensitivity or data-dependent DP exceeds the proposed bound, then $\gamma^P(X)$ will likely be quite large. The analogous condition in Example~\ref{exp: upperbound} to large distance $\gamma(X)$ is \emph{stable} data-dependent DP $\epsilon_{\phi, \hat{\delta}}(X)$. %Our caveat here is that
% % Example~\ref{exp: upperbound} works best for problems whose data-dependent DP has small sensitivity, or is parameterized by statistics  that have small sensitivity. 
% %-- a basic requirement (without which we would have zero utility) is $\Delta_{\phi} \leq \frac{\epsilon\hat{\epsilon}}{\log(1/\delta)}$. 
% \yw{I still don't understsand this remark. Remove?}
% \end{remark}

% Example~\ref{exp: upperbound} is more cautious than  Example~\ref{exp: dist_ptr} in the sense that if the data-dependent DP is at or just below the budget $\hat{\epsilon}$, then the test is more likely to produce a false negative which will cause the algorithm to output $\perp$. Fortunately, this downside is unlikely to manifest in the average case; the experimental results of \cite{redberg2021privately} indicate that instance-based privacy losses are often orders of magnitude smaller than the worst-case DP guarantee.
% \yw{The above paragraph confuses me more than it explains. If the data-dependent DP is at or just below, then the distance test will fail too?}
 



% In Section~\ref{sec: pate}, we demonstrate that one can upper bound the data-dependent DP through a modification of the smooth sensitivity framework applied on $\epsilon_\phi(X)$. Moreover, we provide a direct application of Example~\ref{exp: upperbound} with private linear regression by making use of the per-instance DP technique~\citep{wang2017per}. 



% \begin{comment}
% Moreover, if $\epsilon_\phi(X)$ admits a global sensitivity $\triangle_\phi$, we can use the tail bound of Laplace distribution for such a construction. \todo{Shall we mention privately release sufficient statistics here?}


% \begin{corollary}
% Assume that $\epsilon_\phi(X)$ has a global $L_1$ sensitivity $\triangle_\phi$. Let $\cT$ privately test if $\epsilon_{\phi, \hat{\delta}}^P(X) \leq \hat{\epsilon}$, where $\epsilon_{\phi,\hat{\delta}}^P(X):= \epsilon_{\phi, \hat{\delta}}(X)+ \text{Lap}(\frac{\triangle_
% \phi}{\epsilon}) + \frac{\triangle_\phi \log(1/\delta)}{\epsilon}$. Then an instantiation of Algorithm~\ref{alg:gen_ptr} with the test $\cT$ satisfies $(\epsilon+\hat{\epsilon}, \delta + \hat{\delta})$-DP.
% \end{corollary}
% \end{comment}



%\begin{theorem}
%Assume the data-dependent DP function of mechanism $\cM$
%satisfies $\epsilon_\phi(X)$ and its global $L_1$ sensitivity is bounded by $\triangle_\phi$. Then Algorithm~\ref{alg:no_dist} is $(\epsilon + \hat{\epsilon},\delta+\hat{\delta} )$-DP.
%\end{theorem}

 





% \begin{theorem}[\citep{wang2017per}]\label{thm: per}
% The algorithm used in Example~\ref{exp: posterior} with  parameter $(\lambda, \gamma)$ obeys $(\epsilon, \delta)$ data-dependent DP for each dataset $(X, \textbf{y})$  with 
% \[\epsilon = \sqrt{\frac{\gamma L^2 \log(2/\delta)}{\lambda + \lambda_{\min}}} + \frac{\gamma L^2}{2(\lambda + \lambda_{\min}+||\cX||^2)}+ \frac{1 + \log(2/\delta)||\cX||^2}{2(\lambda+\lambda_{\min})},
% \] where $L:=||\cX||(||\cX||||\theta_\lambda^*||+||\cY||)$ is the local Lipschitz constant, $\lambda_{\min}$ denotes the smallest eigenvalue of $X^TX$ and $||\theta_\lambda^*||$ is the magnitude of the solution $\theta_\lambda^* = (X^TX +\lambda I )^{-1}X^T \bf{y}$.
% \end{theorem}

% Notice that the data-dependent DP is a function of $(\lambda_{\min}, L, ||\theta_\lambda^*||, \lambda, \gamma)$, where $(\lambda_{\min}, L, ||\theta_\lambda^*||)$ are data-dependent quantities. One can apply the generalized PTR framework as in the following example.
% \begin{example} We demonstrate here how to apply generalized PTR to the one-posterior sample (OPS) algorithm, a differentially private mechanism which outputs one sample from the posterior distribution of a Bayesian model with bounded log-likelihood.
% \label{examp:ops}
% \begin{enumerate}\vspace{-1mm}
%     \item Propose $\phi=(\lambda, \gamma)$.\vspace{-1mm}
%     \item Based on $(\lambda, \gamma)$, differentially privately release $\lambda_{min}, ||\theta_\lambda^*||, L$ 
%     with privacy budget $(\epsilon, \delta/2)$.\vspace{-1mm}
%     \item Condition on a high probability event (with probability at least $1-\delta/2$) of $\lambda_{min}, ||\theta_\lambda^*||, L$, test if $\blue{\epsilon_{\phi}^P(X)}$  is smaller than the predefined privacy budget $(\hat{\epsilon}, \hat{\delta})$, where $\epsilon_\phi^P(X)$ denotes the sanitized data-dependent DP.\vspace{-1mm}
%     \item Based on the outcome of the test, decide whether to release $\theta \propto e^{-\frac{\gamma}{2}||\bf{y}-X\theta||^2 + \lambda||\theta||^2}$.\vspace{-1mm}
% \end{enumerate}
% \begin{theorem}
% The algorithm outlined in Example~\ref{examp:ops} satisfies $(\epsilon+ \hat{\epsilon}, \delta + \hat{\delta})$-DP. 
% \end{theorem}
% \end{example}


% The main idea of the above algorithm boils down to  privately releasing all data-dependent quantities in data-dependent DP, constructing high-probability confidence intervals of these quantities, and then deciding whether to run the mechanism $\cM$ with the proposed parameters. We defer the details of the privacy calibration of data-dependent quantities to the appendix. 

% One may ask why we cannot directly tune privacy parameters ($\lambda, \gamma$) based on the sanitized data-dependent DP. This is because, in many scenarios, data-dependent quantities depend on the choice of privacy parameters, e.g., $||\theta_\lambda^*||$ is a complicated function of $\lambda$. Thus, the optimization on $\lambda$ becomes a circular problem --- to solve $\lambda$, we need to sanitize $||\theta_
% \lambda^*||$, which needs to choose a $\lambda$ to begin with. Alternatively, generalized PTR provides a clear and flexible framework to test the validity of privacy parameters adapted to the dataset. 

% \begin{remark}
% The above ``circular'' issue is even more serious for generalized linear models (GLMs) beyond linear regression. The data-dependent DP there involves a local strong-convexity parameter, a complex function of the regularizer $\lambda$  and we only have zeroth-order access to. In the appendix, we demonstrate how to apply generalized PTR to provide a generic solution to a family of private GLMs where the link function satisfies a self-concordance assumption.
% \end{remark}
% \vspace{-2mm}

%we demonstrate that our generalized PTR provides a generic solution to private GLMs with data-adaptive analysis.
%In the appendix, we demonstrate that our generalized PTR can handle these cases effectively 
% in next selction, we demonstrate how we can combine the idea of privacy section for privacy calibration with the generalized PTR.




% remark it's close to sufficient statistic release. however, the data-dependent quantilties depends on the other parameters .


%We start by privately releasing $\lambda_{min}, L$ and $ ||\theta_\lambda^*||$ (condition on a high probability event). Then, the data-dependent DP is a function of $\phi = (\lambda, \gamma)$ with known global sensitivity $\triangle_\phi$. Next, we can apply Algorithm~\ref{alg:no_dist} to propose and test$(\lambda, \gamma)$ and decide whether to release $\theta \propto e^{-\frac{\gamma}{2}||\bf{y}-X\theta||^2 + \lambda||\theta||^2}$. We defer the full details to the appendix. 
\label{section:genPTR}


\vspace{-1em}
 \section{Applications}\label{sec:applications}

Here we will see many problems \cite{peng2016approximate} that fall into the framework of $p$-extendible systems and thus, we directly have recovery results on their stable instances. Some of the problems might be hard, like Weighted Independent Set, whereas others may be easy (i.e. in $P$), having exact algorithms, however the greedy is extremely simple and fast compared to those. 


 \vspace{-1em}
 \label{section:applications}
\vspace{-0.5em}
% \vspace{-0.5em}
\section{Conclusion}
% \vspace{-0.5em}
Recent advances in multimodal single-cell technology have enabled the simultaneous profiling of the transcriptome alongside other cellular modalities, leading to an increase in the availability of multimodal single-cell data. In this paper, we present \method{}, a multimodal transformer model for single-cell surface protein abundance from gene expression measurements. We combined the data with prior biological interaction knowledge from the STRING database into a richly connected heterogeneous graph and leveraged the transformer architectures to learn an accurate mapping between gene expression and surface protein abundance. Remarkably, \method{} achieves superior and more stable performance than other baselines on both 2021 and 2022 NeurIPS single-cell datasets.

\noindent\textbf{Future Work.}
% Our work is an extension of the model we implemented in the NeurIPS 2022 competition. 
Our framework of multimodal transformers with the cross-modality heterogeneous graph goes far beyond the specific downstream task of modality prediction, and there are lots of potentials to be further explored. Our graph contains three types of nodes. While the cell embeddings are used for predictions, the remaining protein embeddings and gene embeddings may be further interpreted for other tasks. The similarities between proteins may show data-specific protein-protein relationships, while the attention matrix of the gene transformer may help to identify marker genes of each cell type. Additionally, we may achieve gene interaction prediction using the attention mechanism.
% under adequate regulations. 
% We expect \method{} to be capable of much more than just modality prediction. Note that currently, we fuse information from different transformers with message-passing GNNs. 
To extend more on transformers, a potential next step is implementing cross-attention cross-modalities. Ideally, all three types of nodes, namely genes, proteins, and cells, would be jointly modeled using a large transformer that includes specific regulations for each modality. 

% insight of protein and gene embedding (diff task)

% all in one transformer

% \noindent\textbf{Limitations and future work}
% Despite the noticeable performance improvement by utilizing transformers with the cross-modality heterogeneous graph, there are still bottlenecks in the current settings. To begin with, we noticed that the performance variations of all methods are consistently higher in the ``CITE'' dataset compared to the ``GEX2ADT'' dataset. We hypothesized that the increased variability in ``CITE'' was due to both less number of training samples (43k vs. 66k cells) and a significantly more number of testing samples used (28k vs. 1k cells). One straightforward solution to alleviate the high variation issue is to include more training samples, which is not always possible given the training data availability. Nevertheless, publicly available single-cell datasets have been accumulated over the past decades and are still being collected on an ever-increasing scale. Taking advantage of these large-scale atlases is the key to a more stable and well-performing model, as some of the intra-cell variations could be common across different datasets. For example, reference-based methods are commonly used to identify the cell identity of a single cell, or cell-type compositions of a mixture of cells. (other examples for pretrained, e.g., scbert)


%\noindent\textbf{Future work.}
% Our work is an extension of the model we implemented in the NeurIPS 2022 competition. Now our framework of multimodal transformers with the cross-modality heterogeneous graph goes far beyond the specific downstream task of modality prediction, and there are lots of potentials to be further explored. Our graph contains three types of nodes. while the cell embeddings are used for predictions, the remaining protein embeddings and gene embeddings may be further interpreted for other tasks. The similarities between proteins may show data-specific protein-protein relationships, while the attention matrix of the gene transformer may help to identify marker genes of each cell type. Additionally, we may achieve gene interaction prediction using the attention mechanism under adequate regulations. We expect \method{} to be capable of much more than just modality prediction. Note that currently, we fuse information from different transformers with message-passing GNNs. To extend more on transformers, a potential next step is implementing cross-attention cross-modalities. Ideally, all three types of nodes, namely genes, proteins, and cells, would be jointly modeled using a large transformer that includes specific regulations for each modality. The self-attention within each modality would reconstruct the prior interaction network, while the cross-attention between modalities would be supervised by the data observations. Then, The attention matrix will provide insights into all the internal interactions and cross-relationships. With the linearized transformer, this idea would be both practical and versatile.

% \begin{acks}
% This research is supported by the National Science Foundation (NSF) and Johnson \& Johnson.
% \end{acks}
\vspace{-2em}
\label{section:conclusion}



\newpage
\onecolumn
\chapter{Supplementary Material}
\label{appendix}

In this appendix, we present supplementary material for the techniques and
experiments presented in the main text.

\section{Baseline Results and Analysis for Informed Sampler}
\label{appendix:chap3}

Here, we give an in-depth
performance analysis of the various samplers and the effect of their
hyperparameters. We choose hyperparameters with the lowest PSRF value
after $10k$ iterations, for each sampler individually. If the
differences between PSRF are not significantly different among
multiple values, we choose the one that has the highest acceptance
rate.

\subsection{Experiment: Estimating Camera Extrinsics}
\label{appendix:chap3:room}

\subsubsection{Parameter Selection}
\paragraph{Metropolis Hastings (\MH)}

Figure~\ref{fig:exp1_MH} shows the median acceptance rates and PSRF
values corresponding to various proposal standard deviations of plain
\MH~sampling. Mixing gets better and the acceptance rate gets worse as
the standard deviation increases. The value $0.3$ is selected standard
deviation for this sampler.

\paragraph{Metropolis Hastings Within Gibbs (\MHWG)}

As mentioned in Section~\ref{sec:room}, the \MHWG~sampler with one-dimensional
updates did not converge for any value of proposal standard deviation.
This problem has high correlation of the camera parameters and is of
multi-modal nature, which this sampler has problems with.

\paragraph{Parallel Tempering (\PT)}

For \PT~sampling, we took the best performing \MH~sampler and used
different temperature chains to improve the mixing of the
sampler. Figure~\ref{fig:exp1_PT} shows the results corresponding to
different combination of temperature levels. The sampler with
temperature levels of $[1,3,27]$ performed best in terms of both
mixing and acceptance rate.

\paragraph{Effect of Mixture Coefficient in Informed Sampling (\MIXLMH)}

Figure~\ref{fig:exp1_alpha} shows the effect of mixture
coefficient ($\alpha$) on the informed sampling
\MIXLMH. Since there is no significant different in PSRF values for
$0 \le \alpha \le 0.7$, we chose $0.7$ due to its high acceptance
rate.


% \end{multicols}

\begin{figure}[h]
\centering
  \subfigure[MH]{%
    \includegraphics[width=.48\textwidth]{figures/supplementary/camPose_MH.pdf} \label{fig:exp1_MH}
  }
  \subfigure[PT]{%
    \includegraphics[width=.48\textwidth]{figures/supplementary/camPose_PT.pdf} \label{fig:exp1_PT}
  }
\\
  \subfigure[INF-MH]{%
    \includegraphics[width=.48\textwidth]{figures/supplementary/camPose_alpha.pdf} \label{fig:exp1_alpha}
  }
  \mycaption{Results of the `Estimating Camera Extrinsics' experiment}{PRSFs and Acceptance rates corresponding to (a) various standard deviations of \MH, (b) various temperature level combinations of \PT sampling and (c) various mixture coefficients of \MIXLMH sampling.}
\end{figure}



\begin{figure}[!t]
\centering
  \subfigure[\MH]{%
    \includegraphics[width=.48\textwidth]{figures/supplementary/occlusionExp_MH.pdf} \label{fig:exp2_MH}
  }
  \subfigure[\BMHWG]{%
    \includegraphics[width=.48\textwidth]{figures/supplementary/occlusionExp_BMHWG.pdf} \label{fig:exp2_BMHWG}
  }
\\
  \subfigure[\MHWG]{%
    \includegraphics[width=.48\textwidth]{figures/supplementary/occlusionExp_MHWG.pdf} \label{fig:exp2_MHWG}
  }
  \subfigure[\PT]{%
    \includegraphics[width=.48\textwidth]{figures/supplementary/occlusionExp_PT.pdf} \label{fig:exp2_PT}
  }
\\
  \subfigure[\INFBMHWG]{%
    \includegraphics[width=.5\textwidth]{figures/supplementary/occlusionExp_alpha.pdf} \label{fig:exp2_alpha}
  }
  \mycaption{Results of the `Occluding Tiles' experiment}{PRSF and
    Acceptance rates corresponding to various standard deviations of
    (a) \MH, (b) \BMHWG, (c) \MHWG, (d) various temperature level
    combinations of \PT~sampling and; (e) various mixture coefficients
    of our informed \INFBMHWG sampling.}
\end{figure}

%\onecolumn\newpage\twocolumn
\subsection{Experiment: Occluding Tiles}
\label{appendix:chap3:tiles}

\subsubsection{Parameter Selection}

\paragraph{Metropolis Hastings (\MH)}

Figure~\ref{fig:exp2_MH} shows the results of
\MH~sampling. Results show the poor convergence for all proposal
standard deviations and rapid decrease of AR with increasing standard
deviation. This is due to the high-dimensional nature of
the problem. We selected a standard deviation of $1.1$.

\paragraph{Blocked Metropolis Hastings Within Gibbs (\BMHWG)}

The results of \BMHWG are shown in Figure~\ref{fig:exp2_BMHWG}. In
this sampler we update only one block of tile variables (of dimension
four) in each sampling step. Results show much better performance
compared to plain \MH. The optimal proposal standard deviation for
this sampler is $0.7$.

\paragraph{Metropolis Hastings Within Gibbs (\MHWG)}

Figure~\ref{fig:exp2_MHWG} shows the result of \MHWG sampling. This
sampler is better than \BMHWG and converges much more quickly. Here
a standard deviation of $0.9$ is found to be best.

\paragraph{Parallel Tempering (\PT)}

Figure~\ref{fig:exp2_PT} shows the results of \PT sampling with various
temperature combinations. Results show no improvement in AR from plain
\MH sampling and again $[1,3,27]$ temperature levels are found to be optimal.

\paragraph{Effect of Mixture Coefficient in Informed Sampling (\INFBMHWG)}

Figure~\ref{fig:exp2_alpha} shows the effect of mixture
coefficient ($\alpha$) on the blocked informed sampling
\INFBMHWG. Since there is no significant different in PSRF values for
$0 \le \alpha \le 0.8$, we chose $0.8$ due to its high acceptance
rate.



\subsection{Experiment: Estimating Body Shape}
\label{appendix:chap3:body}

\subsubsection{Parameter Selection}
\paragraph{Metropolis Hastings (\MH)}

Figure~\ref{fig:exp3_MH} shows the result of \MH~sampling with various
proposal standard deviations. The value of $0.1$ is found to be
best.

\paragraph{Metropolis Hastings Within Gibbs (\MHWG)}

For \MHWG sampling we select $0.3$ proposal standard
deviation. Results are shown in Fig.~\ref{fig:exp3_MHWG}.


\paragraph{Parallel Tempering (\PT)}

As before, results in Fig.~\ref{fig:exp3_PT}, the temperature levels
were selected to be $[1,3,27]$ due its slightly higher AR.

\paragraph{Effect of Mixture Coefficient in Informed Sampling (\MIXLMH)}

Figure~\ref{fig:exp3_alpha} shows the effect of $\alpha$ on PSRF and
AR. Since there is no significant differences in PSRF values for $0 \le
\alpha \le 0.8$, we choose $0.8$.


\begin{figure}[t]
\centering
  \subfigure[\MH]{%
    \includegraphics[width=.48\textwidth]{figures/supplementary/bodyShape_MH.pdf} \label{fig:exp3_MH}
  }
  \subfigure[\MHWG]{%
    \includegraphics[width=.48\textwidth]{figures/supplementary/bodyShape_MHWG.pdf} \label{fig:exp3_MHWG}
  }
\\
  \subfigure[\PT]{%
    \includegraphics[width=.48\textwidth]{figures/supplementary/bodyShape_PT.pdf} \label{fig:exp3_PT}
  }
  \subfigure[\MIXLMH]{%
    \includegraphics[width=.48\textwidth]{figures/supplementary/bodyShape_alpha.pdf} \label{fig:exp3_alpha}
  }
\\
  \mycaption{Results of the `Body Shape Estimation' experiment}{PRSFs and
    Acceptance rates corresponding to various standard deviations of
    (a) \MH, (b) \MHWG; (c) various temperature level combinations
    of \PT sampling and; (d) various mixture coefficients of the
    informed \MIXLMH sampling.}
\end{figure}


\subsection{Results Overview}
Figure~\ref{fig:exp_summary} shows the summary results of the all the three
experimental studies related to informed sampler.
\begin{figure*}[h!]
\centering
  \subfigure[Results for: Estimating Camera Extrinsics]{%
    \includegraphics[width=0.9\textwidth]{figures/supplementary/camPose_ALL.pdf} \label{fig:exp1_all}
  }
  \subfigure[Results for: Occluding Tiles]{%
    \includegraphics[width=0.9\textwidth]{figures/supplementary/occlusionExp_ALL.pdf} \label{fig:exp2_all}
  }
  \subfigure[Results for: Estimating Body Shape]{%
    \includegraphics[width=0.9\textwidth]{figures/supplementary/bodyShape_ALL.pdf} \label{fig:exp3_all}
  }
  \label{fig:exp_summary}
  \mycaption{Summary of the statistics for the three experiments}{Shown are
    for several baseline methods and the informed samplers the
    acceptance rates (left), PSRFs (middle), and RMSE values
    (right). All results are median results over multiple test
    examples.}
\end{figure*}

\subsection{Additional Qualitative Results}

\subsubsection{Occluding Tiles}
In Figure~\ref{fig:exp2_visual_more} more qualitative results of the
occluding tiles experiment are shown. The informed sampling approach
(\INFBMHWG) is better than the best baseline (\MHWG). This still is a
very challenging problem since the parameters for occluded tiles are
flat over a large region. Some of the posterior variance of the
occluded tiles is already captured by the informed sampler.

\begin{figure*}[h!]
\begin{center}
\centerline{\includegraphics[width=0.95\textwidth]{figures/supplementary/occlusionExp_Visual.pdf}}
\mycaption{Additional qualitative results of the occluding tiles experiment}
  {From left to right: (a)
  Given image, (b) Ground truth tiles, (c) OpenCV heuristic and most probable estimates
  from 5000 samples obtained by (d) MHWG sampler (best baseline) and
  (e) our INF-BMHWG sampler. (f) Posterior expectation of the tiles
  boundaries obtained by INF-BMHWG sampling (First 2000 samples are
  discarded as burn-in).}
\label{fig:exp2_visual_more}
\end{center}
\end{figure*}

\subsubsection{Body Shape}
Figure~\ref{fig:exp3_bodyMeshes} shows some more results of 3D mesh
reconstruction using posterior samples obtained by our informed
sampling \MIXLMH.

\begin{figure*}[t]
\begin{center}
\centerline{\includegraphics[width=0.75\textwidth]{figures/supplementary/bodyMeshResults.pdf}}
\mycaption{Qualitative results for the body shape experiment}
  {Shown is the 3D mesh reconstruction results with first 1000 samples obtained
  using the \MIXLMH informed sampling method. (blue indicates small
  values and red indicates high values)}
\label{fig:exp3_bodyMeshes}
\end{center}
\end{figure*}

\clearpage



\section{Additional Results on the Face Problem with CMP}

Figure~\ref{fig:shading-qualitative-multiple-subjects-supp} shows inference results for reflectance maps, normal maps and lights for randomly chosen test images, and Fig.~\ref{fig:shading-qualitative-same-subject-supp} shows reflectance estimation results on multiple images of the same subject produced under different illumination conditions. CMP is able to produce estimates that are closer to the groundtruth across different subjects and illumination conditions.

\begin{figure*}[h]
  \begin{center}
  \centerline{\includegraphics[width=1.0\columnwidth]{figures/face_cmp_visual_results_supp.pdf}}
  \vspace{-1.2cm}
  \end{center}
	\mycaption{A visual comparison of inference results}{(a)~Observed images. (b)~Inferred reflectance maps. \textit{GT} is the photometric stereo groundtruth, \textit{BU} is the Biswas \etal (2009) reflectance estimate and \textit{Forest} is the consensus prediction. (c)~The variance of the inferred reflectance estimate produced by \MTD (normalized across rows).(d)~Visualization of inferred light directions. (e)~Inferred normal maps.}
	\label{fig:shading-qualitative-multiple-subjects-supp}
\end{figure*}


\begin{figure*}[h]
	\centering
	\setlength\fboxsep{0.2mm}
	\setlength\fboxrule{0pt}
	\begin{tikzpicture}

		\matrix at (0, 0) [matrix of nodes, nodes={anchor=east}, column sep=-0.05cm, row sep=-0.2cm]
		{
			\fbox{\includegraphics[width=1cm]{figures/sample_3_4_X.png}} &
			\fbox{\includegraphics[width=1cm]{figures/sample_3_4_GT.png}} &
			\fbox{\includegraphics[width=1cm]{figures/sample_3_4_BISWAS.png}}  &
			\fbox{\includegraphics[width=1cm]{figures/sample_3_4_VMP.png}}  &
			\fbox{\includegraphics[width=1cm]{figures/sample_3_4_FOREST.png}}  &
			\fbox{\includegraphics[width=1cm]{figures/sample_3_4_CMP.png}}  &
			\fbox{\includegraphics[width=1cm]{figures/sample_3_4_CMPVAR.png}}
			 \\

			\fbox{\includegraphics[width=1cm]{figures/sample_3_5_X.png}} &
			\fbox{\includegraphics[width=1cm]{figures/sample_3_5_GT.png}} &
			\fbox{\includegraphics[width=1cm]{figures/sample_3_5_BISWAS.png}}  &
			\fbox{\includegraphics[width=1cm]{figures/sample_3_5_VMP.png}}  &
			\fbox{\includegraphics[width=1cm]{figures/sample_3_5_FOREST.png}}  &
			\fbox{\includegraphics[width=1cm]{figures/sample_3_5_CMP.png}}  &
			\fbox{\includegraphics[width=1cm]{figures/sample_3_5_CMPVAR.png}}
			 \\

			\fbox{\includegraphics[width=1cm]{figures/sample_3_6_X.png}} &
			\fbox{\includegraphics[width=1cm]{figures/sample_3_6_GT.png}} &
			\fbox{\includegraphics[width=1cm]{figures/sample_3_6_BISWAS.png}}  &
			\fbox{\includegraphics[width=1cm]{figures/sample_3_6_VMP.png}}  &
			\fbox{\includegraphics[width=1cm]{figures/sample_3_6_FOREST.png}}  &
			\fbox{\includegraphics[width=1cm]{figures/sample_3_6_CMP.png}}  &
			\fbox{\includegraphics[width=1cm]{figures/sample_3_6_CMPVAR.png}}
			 \\
	     };

       \node at (-3.85, -2.0) {\small Observed};
       \node at (-2.55, -2.0) {\small `GT'};
       \node at (-1.27, -2.0) {\small BU};
       \node at (0.0, -2.0) {\small MP};
       \node at (1.27, -2.0) {\small Forest};
       \node at (2.55, -2.0) {\small \textbf{CMP}};
       \node at (3.85, -2.0) {\small Variance};

	\end{tikzpicture}
	\mycaption{Robustness to varying illumination}{Reflectance estimation on a subject images with varying illumination. Left to right: observed image, photometric stereo estimate (GT)
  which is used as a proxy for groundtruth, bottom-up estimate of \cite{Biswas2009}, VMP result, consensus forest estimate, CMP mean, and CMP variance.}
	\label{fig:shading-qualitative-same-subject-supp}
\end{figure*}

\clearpage

\section{Additional Material for Learning Sparse High Dimensional Filters}
\label{sec:appendix-bnn}

This part of supplementary material contains a more detailed overview of the permutohedral
lattice convolution in Section~\ref{sec:permconv}, more experiments in
Section~\ref{sec:addexps} and additional results with protocols for
the experiments presented in Chapter~\ref{chap:bnn} in Section~\ref{sec:addresults}.

\vspace{-0.2cm}
\subsection{General Permutohedral Convolutions}
\label{sec:permconv}

A core technical contribution of this work is the generalization of the Gaussian permutohedral lattice
convolution proposed in~\cite{adams2010fast} to the full non-separable case with the
ability to perform back-propagation. Although, conceptually, there are minor
differences between Gaussian and general parameterized filters, there are non-trivial practical
differences in terms of the algorithmic implementation. The Gauss filters belong to
the separable class and can thus be decomposed into multiple
sequential one dimensional convolutions. We are interested in the general filter
convolutions, which can not be decomposed. Thus, performing a general permutohedral
convolution at a lattice point requires the computation of the inner product with the
neighboring elements in all the directions in the high-dimensional space.

Here, we give more details of the implementation differences of separable
and non-separable filters. In the following, we will explain the scalar case first.
Recall, that the forward pass of general permutohedral convolution
involves 3 steps: \textit{splatting}, \textit{convolving} and \textit{slicing}.
We follow the same splatting and slicing strategies as in~\cite{adams2010fast}
since these operations do not depend on the filter kernel. The main difference
between our work and the existing implementation of~\cite{adams2010fast} is
the way that the convolution operation is executed. This proceeds by constructing
a \emph{blur neighbor} matrix $K$ that stores for every lattice point all
values of the lattice neighbors that are needed to compute the filter output.

\begin{figure}[t!]
  \centering
    \includegraphics[width=0.6\columnwidth]{figures/supplementary/lattice_construction}
  \mycaption{Illustration of 1D permutohedral lattice construction}
  {A $4\times 4$ $(x,y)$ grid lattice is projected onto the plane defined by the normal
  vector $(1,1)^{\top}$. This grid has $s+1=4$ and $d=2$ $(s+1)^{d}=4^2=16$ elements.
  In the projection, all points of the same color are projected onto the same points in the plane.
  The number of elements of the projected lattice is $t=(s+1)^d-s^d=4^2-3^2=7$, that is
  the $(4\times 4)$ grid minus the size of lattice that is $1$ smaller at each size, in this
  case a $(3\times 3)$ lattice (the upper right $(3\times 3)$ elements).
  }
\label{fig:latticeconstruction}
\end{figure}

The blur neighbor matrix is constructed by traversing through all the populated
lattice points and their neighboring elements.
% For efficiency, we do this matrix construction recursively with shared computations
% since $n^{th}$ neighbourhood elements are $1^{st}$ neighborhood elements of $n-1^{th}$ neighbourhood elements. \pg{do not understand}
This is done recursively to share computations. For any lattice point, the neighbors that are
$n$ hops away are the direct neighbors of the points that are $n-1$ hops away.
The size of a $d$ dimensional spatial filter with width $s+1$ is $(s+1)^{d}$ (\eg, a
$3\times 3$ filter, $s=2$ in $d=2$ has $3^2=9$ elements) and this size grows
exponentially in the number of dimensions $d$. The permutohedral lattice is constructed by
projecting a regular grid onto the plane spanned by the $d$ dimensional normal vector ${(1,\ldots,1)}^{\top}$. See
Fig.~\ref{fig:latticeconstruction} for an illustration of the 1D lattice construction.
Many corners of a grid filter are projected onto the same point, in total $t = {(s+1)}^{d} -
s^{d}$ elements remain in the permutohedral filter with $s$ neighborhood in $d-1$ dimensions.
If the lattice has $m$ populated elements, the
matrix $K$ has size $t\times m$. Note that, since the input signal is typically
sparse, only a few lattice corners are being populated in the \textit{slicing} step.
We use a hash-table to keep track of these points and traverse only through
the populated lattice points for this neighborhood matrix construction.

Once the blur neighbor matrix $K$ is constructed, we can perform the convolution
by the matrix vector multiplication
\begin{equation}
\ell' = BK,
\label{eq:conv}
\end{equation}
where $B$ is the $1 \times t$ filter kernel (whose values we will learn) and $\ell'\in\mathbb{R}^{1\times m}$
is the result of the filtering at the $m$ lattice points. In practice, we found that the
matrix $K$ is sometimes too large to fit into GPU memory and we divided the matrix $K$
into smaller pieces to compute Eq.~\ref{eq:conv} sequentially.

In the general multi-dimensional case, the signal $\ell$ is of $c$ dimensions. Then
the kernel $B$ is of size $c \times t$ and $K$ stores the $c$ dimensional vectors
accordingly. When the input and output points are different, we slice only the
input points and splat only at the output points.


\subsection{Additional Experiments}
\label{sec:addexps}
In this section, we discuss more use-cases for the learned bilateral filters, one
use-case of BNNs and two single filter applications for image and 3D mesh denoising.

\subsubsection{Recognition of subsampled MNIST}\label{sec:app_mnist}

One of the strengths of the proposed filter convolution is that it does not
require the input to lie on a regular grid. The only requirement is to define a distance
between features of the input signal.
We highlight this feature with the following experiment using the
classical MNIST ten class classification problem~\cite{lecun1998mnist}. We sample a
sparse set of $N$ points $(x,y)\in [0,1]\times [0,1]$
uniformly at random in the input image, use their interpolated values
as signal and the \emph{continuous} $(x,y)$ positions as features. This mimics
sub-sampling of a high-dimensional signal. To compare against a spatial convolution,
we interpolate the sparse set of values at the grid positions.

We take a reference implementation of LeNet~\cite{lecun1998gradient} that
is part of the Caffe project~\cite{jia2014caffe} and compare it
against the same architecture but replacing the first convolutional
layer with a bilateral convolution layer (BCL). The filter size
and numbers are adjusted to get a comparable number of parameters
($5\times 5$ for LeNet, $2$-neighborhood for BCL).

The results are shown in Table~\ref{tab:all-results}. We see that training
on the original MNIST data (column Original, LeNet vs. BNN) leads to a slight
decrease in performance of the BNN (99.03\%) compared to LeNet
(99.19\%). The BNN can be trained and evaluated on sparse
signals, and we resample the image as described above for $N=$ 100\%, 60\% and
20\% of the total number of pixels. The methods are also evaluated
on test images that are subsampled in the same way. Note that we can
train and test with different subsampling rates. We introduce an additional
bilinear interpolation layer for the LeNet architecture to train on the same
data. In essence, both models perform a spatial interpolation and thus we
expect them to yield a similar classification accuracy. Once the data is of
higher dimensions, the permutohedral convolution will be faster due to hashing
the sparse input points, as well as less memory demanding in comparison to
naive application of a spatial convolution with interpolated values.

\begin{table}[t]
  \begin{center}
    \footnotesize
    \centering
    \begin{tabular}[t]{lllll}
      \toprule
              &     & \multicolumn{3}{c}{Test Subsampling} \\
       Method  & Original & 100\% & 60\% & 20\%\\
      \midrule
       LeNet &  \textbf{0.9919} & 0.9660 & 0.9348 & \textbf{0.6434} \\
       BNN &  0.9903 & \textbf{0.9844} & \textbf{0.9534} & 0.5767 \\
      \hline
       LeNet 100\% & 0.9856 & 0.9809 & 0.9678 & \textbf{0.7386} \\
       BNN 100\% & \textbf{0.9900} & \textbf{0.9863} & \textbf{0.9699} & 0.6910 \\
      \hline
       LeNet 60\% & 0.9848 & 0.9821 & 0.9740 & 0.8151 \\
       BNN 60\% & \textbf{0.9885} & \textbf{0.9864} & \textbf{0.9771} & \textbf{0.8214}\\
      \hline
       LeNet 20\% & \textbf{0.9763} & \textbf{0.9754} & 0.9695 & 0.8928 \\
       BNN 20\% & 0.9728 & 0.9735 & \textbf{0.9701} & \textbf{0.9042}\\
      \bottomrule
    \end{tabular}
  \end{center}
\vspace{-.2cm}
\caption{Classification accuracy on MNIST. We compare the
    LeNet~\cite{lecun1998gradient} implementation that is part of
    Caffe~\cite{jia2014caffe} to the network with the first layer
    replaced by a bilateral convolution layer (BCL). Both are trained
    on the original image resolution (first two rows). Three more BNN
    and CNN models are trained with randomly subsampled images (100\%,
    60\% and 20\% of the pixels). An additional bilinear interpolation
    layer samples the input signal on a spatial grid for the CNN model.
  }
  \label{tab:all-results}
\vspace{-.5cm}
\end{table}

\subsubsection{Image Denoising}

The main application that inspired the development of the bilateral
filtering operation is image denoising~\cite{aurich1995non}, there
using a single Gaussian kernel. Our development allows to learn this
kernel function from data and we explore how to improve using a \emph{single}
but more general bilateral filter.

We use the Berkeley segmentation dataset
(BSDS500)~\cite{arbelaezi2011bsds500} as a test bed. The color
images in the dataset are converted to gray-scale,
and corrupted with Gaussian noise with a standard deviation of
$\frac {25} {255}$.

We compare the performance of four different filter models on a
denoising task.
The first baseline model (`Spatial' in Table \ref{tab:denoising}, $25$
weights) uses a single spatial filter with a kernel size of
$5$ and predicts the scalar gray-scale value at the center pixel. The next model
(`Gauss Bilateral') applies a bilateral \emph{Gaussian}
filter to the noisy input, using position and intensity features $\f=(x,y,v)^\top$.
The third setup (`Learned Bilateral', $65$ weights)
takes a Gauss kernel as initialization and
fits all filter weights on the train set to minimize the
mean squared error with respect to the clean images.
We run a combination
of spatial and permutohedral convolutions on spatial and bilateral
features (`Spatial + Bilateral (Learned)') to check for a complementary
performance of the two convolutions.

\label{sec:exp:denoising}
\begin{table}[!h]
\begin{center}
  \footnotesize
  \begin{tabular}[t]{lr}
    \toprule
    Method & PSNR \\
    \midrule
    Noisy Input & $20.17$ \\
    Spatial & $26.27$ \\
    Gauss Bilateral & $26.51$ \\
    Learned Bilateral & $26.58$ \\
    Spatial + Bilateral (Learned) & \textbf{$26.65$} \\
    \bottomrule
  \end{tabular}
\end{center}
\vspace{-0.5em}
\caption{PSNR results of a denoising task using the BSDS500
  dataset~\cite{arbelaezi2011bsds500}}
\vspace{-0.5em}
\label{tab:denoising}
\end{table}
\vspace{-0.2em}

The PSNR scores evaluated on full images of the test set are
shown in Table \ref{tab:denoising}. We find that an untrained bilateral
filter already performs better than a trained spatial convolution
($26.27$ to $26.51$). A learned convolution further improve the
performance slightly. We chose this simple one-kernel setup to
validate an advantage of the generalized bilateral filter. A competitive
denoising system would employ RGB color information and also
needs to be properly adjusted in network size. Multi-layer perceptrons
have obtained state-of-the-art denoising results~\cite{burger12cvpr}
and the permutohedral lattice layer can readily be used in such an
architecture, which is intended future work.

\subsection{Additional results}
\label{sec:addresults}

This section contains more qualitative results for the experiments presented in Chapter~\ref{chap:bnn}.

\begin{figure*}[th!]
  \centering
    \includegraphics[width=\columnwidth,trim={5cm 2.5cm 5cm 4.5cm},clip]{figures/supplementary/lattice_viz.pdf}
    \vspace{-0.7cm}
  \mycaption{Visualization of the Permutohedral Lattice}
  {Sample lattice visualizations for different feature spaces. All pixels falling in the same simplex cell are shown with
  the same color. $(x,y)$ features correspond to image pixel positions, and $(r,g,b) \in [0,255]$ correspond
  to the red, green and blue color values.}
\label{fig:latticeviz}
\end{figure*}

\subsubsection{Lattice Visualization}

Figure~\ref{fig:latticeviz} shows sample lattice visualizations for different feature spaces.

\newcolumntype{L}[1]{>{\raggedright\let\newline\\\arraybackslash\hspace{0pt}}b{#1}}
\newcolumntype{C}[1]{>{\centering\let\newline\\\arraybackslash\hspace{0pt}}b{#1}}
\newcolumntype{R}[1]{>{\raggedleft\let\newline\\\arraybackslash\hspace{0pt}}b{#1}}

\subsubsection{Color Upsampling}\label{sec:color_upsampling}
\label{sec:col_upsample_extra}

Some images of the upsampling for the Pascal
VOC12 dataset are shown in Fig.~\ref{fig:Colour_upsample_visuals}. It is
especially the low level image details that are better preserved with
a learned bilateral filter compared to the Gaussian case.

\begin{figure*}[t!]
  \centering
    \subfigure{%
   \raisebox{2.0em}{
    \includegraphics[width=.06\columnwidth]{figures/supplementary/2007_004969.jpg}
   }
  }
  \subfigure{%
    \includegraphics[width=.17\columnwidth]{figures/supplementary/2007_004969_gray.pdf}
  }
  \subfigure{%
    \includegraphics[width=.17\columnwidth]{figures/supplementary/2007_004969_gt.pdf}
  }
  \subfigure{%
    \includegraphics[width=.17\columnwidth]{figures/supplementary/2007_004969_bicubic.pdf}
  }
  \subfigure{%
    \includegraphics[width=.17\columnwidth]{figures/supplementary/2007_004969_gauss.pdf}
  }
  \subfigure{%
    \includegraphics[width=.17\columnwidth]{figures/supplementary/2007_004969_learnt.pdf}
  }\\
    \subfigure{%
   \raisebox{2.0em}{
    \includegraphics[width=.06\columnwidth]{figures/supplementary/2007_003106.jpg}
   }
  }
  \subfigure{%
    \includegraphics[width=.17\columnwidth]{figures/supplementary/2007_003106_gray.pdf}
  }
  \subfigure{%
    \includegraphics[width=.17\columnwidth]{figures/supplementary/2007_003106_gt.pdf}
  }
  \subfigure{%
    \includegraphics[width=.17\columnwidth]{figures/supplementary/2007_003106_bicubic.pdf}
  }
  \subfigure{%
    \includegraphics[width=.17\columnwidth]{figures/supplementary/2007_003106_gauss.pdf}
  }
  \subfigure{%
    \includegraphics[width=.17\columnwidth]{figures/supplementary/2007_003106_learnt.pdf}
  }\\
  \setcounter{subfigure}{0}
  \small{
  \subfigure[Inp.]{%
  \raisebox{2.0em}{
    \includegraphics[width=.06\columnwidth]{figures/supplementary/2007_006837.jpg}
   }
  }
  \subfigure[Guidance]{%
    \includegraphics[width=.17\columnwidth]{figures/supplementary/2007_006837_gray.pdf}
  }
   \subfigure[GT]{%
    \includegraphics[width=.17\columnwidth]{figures/supplementary/2007_006837_gt.pdf}
  }
  \subfigure[Bicubic]{%
    \includegraphics[width=.17\columnwidth]{figures/supplementary/2007_006837_bicubic.pdf}
  }
  \subfigure[Gauss-BF]{%
    \includegraphics[width=.17\columnwidth]{figures/supplementary/2007_006837_gauss.pdf}
  }
  \subfigure[Learned-BF]{%
    \includegraphics[width=.17\columnwidth]{figures/supplementary/2007_006837_learnt.pdf}
  }
  }
  \vspace{-0.5cm}
  \mycaption{Color Upsampling}{Color $8\times$ upsampling results
  using different methods, from left to right, (a)~Low-resolution input color image (Inp.),
  (b)~Gray scale guidance image, (c)~Ground-truth color image; Upsampled color images with
  (d)~Bicubic interpolation, (e) Gauss bilateral upsampling and, (f)~Learned bilateral
  updampgling (best viewed on screen).}

\label{fig:Colour_upsample_visuals}
\end{figure*}

\subsubsection{Depth Upsampling}
\label{sec:depth_upsample_extra}

Figure~\ref{fig:depth_upsample_visuals} presents some more qualitative results comparing bicubic interpolation, Gauss
bilateral and learned bilateral upsampling on NYU depth dataset image~\cite{silberman2012indoor}.

\subsubsection{Character Recognition}\label{sec:app_character}

 Figure~\ref{fig:nnrecognition} shows the schematic of different layers
 of the network architecture for LeNet-7~\cite{lecun1998mnist}
 and DeepCNet(5, 50)~\cite{ciresan2012multi,graham2014spatially}. For the BNN variants, the first layer filters are replaced
 with learned bilateral filters and are learned end-to-end.

\subsubsection{Semantic Segmentation}\label{sec:app_semantic_segmentation}
\label{sec:semantic_bnn_extra}

Some more visual results for semantic segmentation are shown in Figure~\ref{fig:semantic_visuals}.
These include the underlying DeepLab CNN\cite{chen2014semantic} result (DeepLab),
the 2 step mean-field result with Gaussian edge potentials (+2stepMF-GaussCRF)
and also corresponding results with learned edge potentials (+2stepMF-LearnedCRF).
In general, we observe that mean-field in learned CRF leads to slightly dilated
classification regions in comparison to using Gaussian CRF thereby filling-in the
false negative pixels and also correcting some mis-classified regions.

\begin{figure*}[t!]
  \centering
    \subfigure{%
   \raisebox{2.0em}{
    \includegraphics[width=.06\columnwidth]{figures/supplementary/2bicubic}
   }
  }
  \subfigure{%
    \includegraphics[width=.17\columnwidth]{figures/supplementary/2given_image}
  }
  \subfigure{%
    \includegraphics[width=.17\columnwidth]{figures/supplementary/2ground_truth}
  }
  \subfigure{%
    \includegraphics[width=.17\columnwidth]{figures/supplementary/2bicubic}
  }
  \subfigure{%
    \includegraphics[width=.17\columnwidth]{figures/supplementary/2gauss}
  }
  \subfigure{%
    \includegraphics[width=.17\columnwidth]{figures/supplementary/2learnt}
  }\\
    \subfigure{%
   \raisebox{2.0em}{
    \includegraphics[width=.06\columnwidth]{figures/supplementary/32bicubic}
   }
  }
  \subfigure{%
    \includegraphics[width=.17\columnwidth]{figures/supplementary/32given_image}
  }
  \subfigure{%
    \includegraphics[width=.17\columnwidth]{figures/supplementary/32ground_truth}
  }
  \subfigure{%
    \includegraphics[width=.17\columnwidth]{figures/supplementary/32bicubic}
  }
  \subfigure{%
    \includegraphics[width=.17\columnwidth]{figures/supplementary/32gauss}
  }
  \subfigure{%
    \includegraphics[width=.17\columnwidth]{figures/supplementary/32learnt}
  }\\
  \setcounter{subfigure}{0}
  \small{
  \subfigure[Inp.]{%
  \raisebox{2.0em}{
    \includegraphics[width=.06\columnwidth]{figures/supplementary/41bicubic}
   }
  }
  \subfigure[Guidance]{%
    \includegraphics[width=.17\columnwidth]{figures/supplementary/41given_image}
  }
   \subfigure[GT]{%
    \includegraphics[width=.17\columnwidth]{figures/supplementary/41ground_truth}
  }
  \subfigure[Bicubic]{%
    \includegraphics[width=.17\columnwidth]{figures/supplementary/41bicubic}
  }
  \subfigure[Gauss-BF]{%
    \includegraphics[width=.17\columnwidth]{figures/supplementary/41gauss}
  }
  \subfigure[Learned-BF]{%
    \includegraphics[width=.17\columnwidth]{figures/supplementary/41learnt}
  }
  }
  \mycaption{Depth Upsampling}{Depth $8\times$ upsampling results
  using different upsampling strategies, from left to right,
  (a)~Low-resolution input depth image (Inp.),
  (b)~High-resolution guidance image, (c)~Ground-truth depth; Upsampled depth images with
  (d)~Bicubic interpolation, (e) Gauss bilateral upsampling and, (f)~Learned bilateral
  updampgling (best viewed on screen).}

\label{fig:depth_upsample_visuals}
\end{figure*}

\subsubsection{Material Segmentation}\label{sec:app_material_segmentation}
\label{sec:material_bnn_extra}

In Fig.~\ref{fig:material_visuals-app2}, we present visual results comparing 2 step
mean-field inference with Gaussian and learned pairwise CRF potentials. In
general, we observe that the pixels belonging to dominant classes in the
training data are being more accurately classified with learned CRF. This leads to
a significant improvements in overall pixel accuracy. This also results
in a slight decrease of the accuracy from less frequent class pixels thereby
slightly reducing the average class accuracy with learning. We attribute this
to the type of annotation that is available for this dataset, which is not
for the entire image but for some segments in the image. We have very few
images of the infrequent classes to combat this behaviour during training.

\subsubsection{Experiment Protocols}
\label{sec:protocols}

Table~\ref{tbl:parameters} shows experiment protocols of different experiments.

 \begin{figure*}[t!]
  \centering
  \subfigure[LeNet-7]{
    \includegraphics[width=0.7\columnwidth]{figures/supplementary/lenet_cnn_network}
    }\\
    \subfigure[DeepCNet]{
    \includegraphics[width=\columnwidth]{figures/supplementary/deepcnet_cnn_network}
    }
  \mycaption{CNNs for Character Recognition}
  {Schematic of (top) LeNet-7~\cite{lecun1998mnist} and (bottom) DeepCNet(5,50)~\cite{ciresan2012multi,graham2014spatially} architectures used in Assamese
  character recognition experiments.}
\label{fig:nnrecognition}
\end{figure*}

\definecolor{voc_1}{RGB}{0, 0, 0}
\definecolor{voc_2}{RGB}{128, 0, 0}
\definecolor{voc_3}{RGB}{0, 128, 0}
\definecolor{voc_4}{RGB}{128, 128, 0}
\definecolor{voc_5}{RGB}{0, 0, 128}
\definecolor{voc_6}{RGB}{128, 0, 128}
\definecolor{voc_7}{RGB}{0, 128, 128}
\definecolor{voc_8}{RGB}{128, 128, 128}
\definecolor{voc_9}{RGB}{64, 0, 0}
\definecolor{voc_10}{RGB}{192, 0, 0}
\definecolor{voc_11}{RGB}{64, 128, 0}
\definecolor{voc_12}{RGB}{192, 128, 0}
\definecolor{voc_13}{RGB}{64, 0, 128}
\definecolor{voc_14}{RGB}{192, 0, 128}
\definecolor{voc_15}{RGB}{64, 128, 128}
\definecolor{voc_16}{RGB}{192, 128, 128}
\definecolor{voc_17}{RGB}{0, 64, 0}
\definecolor{voc_18}{RGB}{128, 64, 0}
\definecolor{voc_19}{RGB}{0, 192, 0}
\definecolor{voc_20}{RGB}{128, 192, 0}
\definecolor{voc_21}{RGB}{0, 64, 128}
\definecolor{voc_22}{RGB}{128, 64, 128}

\begin{figure*}[t]
  \centering
  \small{
  \fcolorbox{white}{voc_1}{\rule{0pt}{6pt}\rule{6pt}{0pt}} Background~~
  \fcolorbox{white}{voc_2}{\rule{0pt}{6pt}\rule{6pt}{0pt}} Aeroplane~~
  \fcolorbox{white}{voc_3}{\rule{0pt}{6pt}\rule{6pt}{0pt}} Bicycle~~
  \fcolorbox{white}{voc_4}{\rule{0pt}{6pt}\rule{6pt}{0pt}} Bird~~
  \fcolorbox{white}{voc_5}{\rule{0pt}{6pt}\rule{6pt}{0pt}} Boat~~
  \fcolorbox{white}{voc_6}{\rule{0pt}{6pt}\rule{6pt}{0pt}} Bottle~~
  \fcolorbox{white}{voc_7}{\rule{0pt}{6pt}\rule{6pt}{0pt}} Bus~~
  \fcolorbox{white}{voc_8}{\rule{0pt}{6pt}\rule{6pt}{0pt}} Car~~ \\
  \fcolorbox{white}{voc_9}{\rule{0pt}{6pt}\rule{6pt}{0pt}} Cat~~
  \fcolorbox{white}{voc_10}{\rule{0pt}{6pt}\rule{6pt}{0pt}} Chair~~
  \fcolorbox{white}{voc_11}{\rule{0pt}{6pt}\rule{6pt}{0pt}} Cow~~
  \fcolorbox{white}{voc_12}{\rule{0pt}{6pt}\rule{6pt}{0pt}} Dining Table~~
  \fcolorbox{white}{voc_13}{\rule{0pt}{6pt}\rule{6pt}{0pt}} Dog~~
  \fcolorbox{white}{voc_14}{\rule{0pt}{6pt}\rule{6pt}{0pt}} Horse~~
  \fcolorbox{white}{voc_15}{\rule{0pt}{6pt}\rule{6pt}{0pt}} Motorbike~~
  \fcolorbox{white}{voc_16}{\rule{0pt}{6pt}\rule{6pt}{0pt}} Person~~ \\
  \fcolorbox{white}{voc_17}{\rule{0pt}{6pt}\rule{6pt}{0pt}} Potted Plant~~
  \fcolorbox{white}{voc_18}{\rule{0pt}{6pt}\rule{6pt}{0pt}} Sheep~~
  \fcolorbox{white}{voc_19}{\rule{0pt}{6pt}\rule{6pt}{0pt}} Sofa~~
  \fcolorbox{white}{voc_20}{\rule{0pt}{6pt}\rule{6pt}{0pt}} Train~~
  \fcolorbox{white}{voc_21}{\rule{0pt}{6pt}\rule{6pt}{0pt}} TV monitor~~ \\
  }
  \subfigure{%
    \includegraphics[width=.18\columnwidth]{figures/supplementary/2007_001423_given.jpg}
  }
  \subfigure{%
    \includegraphics[width=.18\columnwidth]{figures/supplementary/2007_001423_gt.png}
  }
  \subfigure{%
    \includegraphics[width=.18\columnwidth]{figures/supplementary/2007_001423_cnn.png}
  }
  \subfigure{%
    \includegraphics[width=.18\columnwidth]{figures/supplementary/2007_001423_gauss.png}
  }
  \subfigure{%
    \includegraphics[width=.18\columnwidth]{figures/supplementary/2007_001423_learnt.png}
  }\\
  \subfigure{%
    \includegraphics[width=.18\columnwidth]{figures/supplementary/2007_001430_given.jpg}
  }
  \subfigure{%
    \includegraphics[width=.18\columnwidth]{figures/supplementary/2007_001430_gt.png}
  }
  \subfigure{%
    \includegraphics[width=.18\columnwidth]{figures/supplementary/2007_001430_cnn.png}
  }
  \subfigure{%
    \includegraphics[width=.18\columnwidth]{figures/supplementary/2007_001430_gauss.png}
  }
  \subfigure{%
    \includegraphics[width=.18\columnwidth]{figures/supplementary/2007_001430_learnt.png}
  }\\
    \subfigure{%
    \includegraphics[width=.18\columnwidth]{figures/supplementary/2007_007996_given.jpg}
  }
  \subfigure{%
    \includegraphics[width=.18\columnwidth]{figures/supplementary/2007_007996_gt.png}
  }
  \subfigure{%
    \includegraphics[width=.18\columnwidth]{figures/supplementary/2007_007996_cnn.png}
  }
  \subfigure{%
    \includegraphics[width=.18\columnwidth]{figures/supplementary/2007_007996_gauss.png}
  }
  \subfigure{%
    \includegraphics[width=.18\columnwidth]{figures/supplementary/2007_007996_learnt.png}
  }\\
   \subfigure{%
    \includegraphics[width=.18\columnwidth]{figures/supplementary/2010_002682_given.jpg}
  }
  \subfigure{%
    \includegraphics[width=.18\columnwidth]{figures/supplementary/2010_002682_gt.png}
  }
  \subfigure{%
    \includegraphics[width=.18\columnwidth]{figures/supplementary/2010_002682_cnn.png}
  }
  \subfigure{%
    \includegraphics[width=.18\columnwidth]{figures/supplementary/2010_002682_gauss.png}
  }
  \subfigure{%
    \includegraphics[width=.18\columnwidth]{figures/supplementary/2010_002682_learnt.png}
  }\\
     \subfigure{%
    \includegraphics[width=.18\columnwidth]{figures/supplementary/2010_004789_given.jpg}
  }
  \subfigure{%
    \includegraphics[width=.18\columnwidth]{figures/supplementary/2010_004789_gt.png}
  }
  \subfigure{%
    \includegraphics[width=.18\columnwidth]{figures/supplementary/2010_004789_cnn.png}
  }
  \subfigure{%
    \includegraphics[width=.18\columnwidth]{figures/supplementary/2010_004789_gauss.png}
  }
  \subfigure{%
    \includegraphics[width=.18\columnwidth]{figures/supplementary/2010_004789_learnt.png}
  }\\
       \subfigure{%
    \includegraphics[width=.18\columnwidth]{figures/supplementary/2007_001311_given.jpg}
  }
  \subfigure{%
    \includegraphics[width=.18\columnwidth]{figures/supplementary/2007_001311_gt.png}
  }
  \subfigure{%
    \includegraphics[width=.18\columnwidth]{figures/supplementary/2007_001311_cnn.png}
  }
  \subfigure{%
    \includegraphics[width=.18\columnwidth]{figures/supplementary/2007_001311_gauss.png}
  }
  \subfigure{%
    \includegraphics[width=.18\columnwidth]{figures/supplementary/2007_001311_learnt.png}
  }\\
  \setcounter{subfigure}{0}
  \subfigure[Input]{%
    \includegraphics[width=.18\columnwidth]{figures/supplementary/2010_003531_given.jpg}
  }
  \subfigure[Ground Truth]{%
    \includegraphics[width=.18\columnwidth]{figures/supplementary/2010_003531_gt.png}
  }
  \subfigure[DeepLab]{%
    \includegraphics[width=.18\columnwidth]{figures/supplementary/2010_003531_cnn.png}
  }
  \subfigure[+GaussCRF]{%
    \includegraphics[width=.18\columnwidth]{figures/supplementary/2010_003531_gauss.png}
  }
  \subfigure[+LearnedCRF]{%
    \includegraphics[width=.18\columnwidth]{figures/supplementary/2010_003531_learnt.png}
  }
  \vspace{-0.3cm}
  \mycaption{Semantic Segmentation}{Example results of semantic segmentation.
  (c)~depicts the unary results before application of MF, (d)~after two steps of MF with Gaussian edge CRF potentials, (e)~after
  two steps of MF with learned edge CRF potentials.}
    \label{fig:semantic_visuals}
\end{figure*}


\definecolor{minc_1}{HTML}{771111}
\definecolor{minc_2}{HTML}{CAC690}
\definecolor{minc_3}{HTML}{EEEEEE}
\definecolor{minc_4}{HTML}{7C8FA6}
\definecolor{minc_5}{HTML}{597D31}
\definecolor{minc_6}{HTML}{104410}
\definecolor{minc_7}{HTML}{BB819C}
\definecolor{minc_8}{HTML}{D0CE48}
\definecolor{minc_9}{HTML}{622745}
\definecolor{minc_10}{HTML}{666666}
\definecolor{minc_11}{HTML}{D54A31}
\definecolor{minc_12}{HTML}{101044}
\definecolor{minc_13}{HTML}{444126}
\definecolor{minc_14}{HTML}{75D646}
\definecolor{minc_15}{HTML}{DD4348}
\definecolor{minc_16}{HTML}{5C8577}
\definecolor{minc_17}{HTML}{C78472}
\definecolor{minc_18}{HTML}{75D6D0}
\definecolor{minc_19}{HTML}{5B4586}
\definecolor{minc_20}{HTML}{C04393}
\definecolor{minc_21}{HTML}{D69948}
\definecolor{minc_22}{HTML}{7370D8}
\definecolor{minc_23}{HTML}{7A3622}
\definecolor{minc_24}{HTML}{000000}

\begin{figure*}[t]
  \centering
  \small{
  \fcolorbox{white}{minc_1}{\rule{0pt}{6pt}\rule{6pt}{0pt}} Brick~~
  \fcolorbox{white}{minc_2}{\rule{0pt}{6pt}\rule{6pt}{0pt}} Carpet~~
  \fcolorbox{white}{minc_3}{\rule{0pt}{6pt}\rule{6pt}{0pt}} Ceramic~~
  \fcolorbox{white}{minc_4}{\rule{0pt}{6pt}\rule{6pt}{0pt}} Fabric~~
  \fcolorbox{white}{minc_5}{\rule{0pt}{6pt}\rule{6pt}{0pt}} Foliage~~
  \fcolorbox{white}{minc_6}{\rule{0pt}{6pt}\rule{6pt}{0pt}} Food~~
  \fcolorbox{white}{minc_7}{\rule{0pt}{6pt}\rule{6pt}{0pt}} Glass~~
  \fcolorbox{white}{minc_8}{\rule{0pt}{6pt}\rule{6pt}{0pt}} Hair~~ \\
  \fcolorbox{white}{minc_9}{\rule{0pt}{6pt}\rule{6pt}{0pt}} Leather~~
  \fcolorbox{white}{minc_10}{\rule{0pt}{6pt}\rule{6pt}{0pt}} Metal~~
  \fcolorbox{white}{minc_11}{\rule{0pt}{6pt}\rule{6pt}{0pt}} Mirror~~
  \fcolorbox{white}{minc_12}{\rule{0pt}{6pt}\rule{6pt}{0pt}} Other~~
  \fcolorbox{white}{minc_13}{\rule{0pt}{6pt}\rule{6pt}{0pt}} Painted~~
  \fcolorbox{white}{minc_14}{\rule{0pt}{6pt}\rule{6pt}{0pt}} Paper~~
  \fcolorbox{white}{minc_15}{\rule{0pt}{6pt}\rule{6pt}{0pt}} Plastic~~\\
  \fcolorbox{white}{minc_16}{\rule{0pt}{6pt}\rule{6pt}{0pt}} Polished Stone~~
  \fcolorbox{white}{minc_17}{\rule{0pt}{6pt}\rule{6pt}{0pt}} Skin~~
  \fcolorbox{white}{minc_18}{\rule{0pt}{6pt}\rule{6pt}{0pt}} Sky~~
  \fcolorbox{white}{minc_19}{\rule{0pt}{6pt}\rule{6pt}{0pt}} Stone~~
  \fcolorbox{white}{minc_20}{\rule{0pt}{6pt}\rule{6pt}{0pt}} Tile~~
  \fcolorbox{white}{minc_21}{\rule{0pt}{6pt}\rule{6pt}{0pt}} Wallpaper~~
  \fcolorbox{white}{minc_22}{\rule{0pt}{6pt}\rule{6pt}{0pt}} Water~~
  \fcolorbox{white}{minc_23}{\rule{0pt}{6pt}\rule{6pt}{0pt}} Wood~~ \\
  }
  \subfigure{%
    \includegraphics[width=.18\columnwidth]{figures/supplementary/000010868_given.jpg}
  }
  \subfigure{%
    \includegraphics[width=.18\columnwidth]{figures/supplementary/000010868_gt.png}
  }
  \subfigure{%
    \includegraphics[width=.18\columnwidth]{figures/supplementary/000010868_cnn.png}
  }
  \subfigure{%
    \includegraphics[width=.18\columnwidth]{figures/supplementary/000010868_gauss.png}
  }
  \subfigure{%
    \includegraphics[width=.18\columnwidth]{figures/supplementary/000010868_learnt.png}
  }\\[-2ex]
  \subfigure{%
    \includegraphics[width=.18\columnwidth]{figures/supplementary/000006011_given.jpg}
  }
  \subfigure{%
    \includegraphics[width=.18\columnwidth]{figures/supplementary/000006011_gt.png}
  }
  \subfigure{%
    \includegraphics[width=.18\columnwidth]{figures/supplementary/000006011_cnn.png}
  }
  \subfigure{%
    \includegraphics[width=.18\columnwidth]{figures/supplementary/000006011_gauss.png}
  }
  \subfigure{%
    \includegraphics[width=.18\columnwidth]{figures/supplementary/000006011_learnt.png}
  }\\[-2ex]
    \subfigure{%
    \includegraphics[width=.18\columnwidth]{figures/supplementary/000008553_given.jpg}
  }
  \subfigure{%
    \includegraphics[width=.18\columnwidth]{figures/supplementary/000008553_gt.png}
  }
  \subfigure{%
    \includegraphics[width=.18\columnwidth]{figures/supplementary/000008553_cnn.png}
  }
  \subfigure{%
    \includegraphics[width=.18\columnwidth]{figures/supplementary/000008553_gauss.png}
  }
  \subfigure{%
    \includegraphics[width=.18\columnwidth]{figures/supplementary/000008553_learnt.png}
  }\\[-2ex]
   \subfigure{%
    \includegraphics[width=.18\columnwidth]{figures/supplementary/000009188_given.jpg}
  }
  \subfigure{%
    \includegraphics[width=.18\columnwidth]{figures/supplementary/000009188_gt.png}
  }
  \subfigure{%
    \includegraphics[width=.18\columnwidth]{figures/supplementary/000009188_cnn.png}
  }
  \subfigure{%
    \includegraphics[width=.18\columnwidth]{figures/supplementary/000009188_gauss.png}
  }
  \subfigure{%
    \includegraphics[width=.18\columnwidth]{figures/supplementary/000009188_learnt.png}
  }\\[-2ex]
  \setcounter{subfigure}{0}
  \subfigure[Input]{%
    \includegraphics[width=.18\columnwidth]{figures/supplementary/000023570_given.jpg}
  }
  \subfigure[Ground Truth]{%
    \includegraphics[width=.18\columnwidth]{figures/supplementary/000023570_gt.png}
  }
  \subfigure[DeepLab]{%
    \includegraphics[width=.18\columnwidth]{figures/supplementary/000023570_cnn.png}
  }
  \subfigure[+GaussCRF]{%
    \includegraphics[width=.18\columnwidth]{figures/supplementary/000023570_gauss.png}
  }
  \subfigure[+LearnedCRF]{%
    \includegraphics[width=.18\columnwidth]{figures/supplementary/000023570_learnt.png}
  }
  \mycaption{Material Segmentation}{Example results of material segmentation.
  (c)~depicts the unary results before application of MF, (d)~after two steps of MF with Gaussian edge CRF potentials, (e)~after two steps of MF with learned edge CRF potentials.}
    \label{fig:material_visuals-app2}
\end{figure*}


\begin{table*}[h]
\tiny
  \centering
    \begin{tabular}{L{2.3cm} L{2.25cm} C{1.5cm} C{0.7cm} C{0.6cm} C{0.7cm} C{0.7cm} C{0.7cm} C{1.6cm} C{0.6cm} C{0.6cm} C{0.6cm}}
      \toprule
& & & & & \multicolumn{3}{c}{\textbf{Data Statistics}} & \multicolumn{4}{c}{\textbf{Training Protocol}} \\

\textbf{Experiment} & \textbf{Feature Types} & \textbf{Feature Scales} & \textbf{Filter Size} & \textbf{Filter Nbr.} & \textbf{Train}  & \textbf{Val.} & \textbf{Test} & \textbf{Loss Type} & \textbf{LR} & \textbf{Batch} & \textbf{Epochs} \\
      \midrule
      \multicolumn{2}{c}{\textbf{Single Bilateral Filter Applications}} & & & & & & & & & \\
      \textbf{2$\times$ Color Upsampling} & Position$_{1}$, Intensity (3D) & 0.13, 0.17 & 65 & 2 & 10581 & 1449 & 1456 & MSE & 1e-06 & 200 & 94.5\\
      \textbf{4$\times$ Color Upsampling} & Position$_{1}$, Intensity (3D) & 0.06, 0.17 & 65 & 2 & 10581 & 1449 & 1456 & MSE & 1e-06 & 200 & 94.5\\
      \textbf{8$\times$ Color Upsampling} & Position$_{1}$, Intensity (3D) & 0.03, 0.17 & 65 & 2 & 10581 & 1449 & 1456 & MSE & 1e-06 & 200 & 94.5\\
      \textbf{16$\times$ Color Upsampling} & Position$_{1}$, Intensity (3D) & 0.02, 0.17 & 65 & 2 & 10581 & 1449 & 1456 & MSE & 1e-06 & 200 & 94.5\\
      \textbf{Depth Upsampling} & Position$_{1}$, Color (5D) & 0.05, 0.02 & 665 & 2 & 795 & 100 & 654 & MSE & 1e-07 & 50 & 251.6\\
      \textbf{Mesh Denoising} & Isomap (4D) & 46.00 & 63 & 2 & 1000 & 200 & 500 & MSE & 100 & 10 & 100.0 \\
      \midrule
      \multicolumn{2}{c}{\textbf{DenseCRF Applications}} & & & & & & & & &\\
      \multicolumn{2}{l}{\textbf{Semantic Segmentation}} & & & & & & & & &\\
      \textbf{- 1step MF} & Position$_{1}$, Color (5D); Position$_{1}$ (2D) & 0.01, 0.34; 0.34  & 665; 19  & 2; 2 & 10581 & 1449 & 1456 & Logistic & 0.1 & 5 & 1.4 \\
      \textbf{- 2step MF} & Position$_{1}$, Color (5D); Position$_{1}$ (2D) & 0.01, 0.34; 0.34 & 665; 19 & 2; 2 & 10581 & 1449 & 1456 & Logistic & 0.1 & 5 & 1.4 \\
      \textbf{- \textit{loose} 2step MF} & Position$_{1}$, Color (5D); Position$_{1}$ (2D) & 0.01, 0.34; 0.34 & 665; 19 & 2; 2 &10581 & 1449 & 1456 & Logistic & 0.1 & 5 & +1.9  \\ \\
      \multicolumn{2}{l}{\textbf{Material Segmentation}} & & & & & & & & &\\
      \textbf{- 1step MF} & Position$_{2}$, Lab-Color (5D) & 5.00, 0.05, 0.30  & 665 & 2 & 928 & 150 & 1798 & Weighted Logistic & 1e-04 & 24 & 2.6 \\
      \textbf{- 2step MF} & Position$_{2}$, Lab-Color (5D) & 5.00, 0.05, 0.30 & 665 & 2 & 928 & 150 & 1798 & Weighted Logistic & 1e-04 & 12 & +0.7 \\
      \textbf{- \textit{loose} 2step MF} & Position$_{2}$, Lab-Color (5D) & 5.00, 0.05, 0.30 & 665 & 2 & 928 & 150 & 1798 & Weighted Logistic & 1e-04 & 12 & +0.2\\
      \midrule
      \multicolumn{2}{c}{\textbf{Neural Network Applications}} & & & & & & & & &\\
      \textbf{Tiles: CNN-9$\times$9} & - & - & 81 & 4 & 10000 & 1000 & 1000 & Logistic & 0.01 & 100 & 500.0 \\
      \textbf{Tiles: CNN-13$\times$13} & - & - & 169 & 6 & 10000 & 1000 & 1000 & Logistic & 0.01 & 100 & 500.0 \\
      \textbf{Tiles: CNN-17$\times$17} & - & - & 289 & 8 & 10000 & 1000 & 1000 & Logistic & 0.01 & 100 & 500.0 \\
      \textbf{Tiles: CNN-21$\times$21} & - & - & 441 & 10 & 10000 & 1000 & 1000 & Logistic & 0.01 & 100 & 500.0 \\
      \textbf{Tiles: BNN} & Position$_{1}$, Color (5D) & 0.05, 0.04 & 63 & 1 & 10000 & 1000 & 1000 & Logistic & 0.01 & 100 & 30.0 \\
      \textbf{LeNet} & - & - & 25 & 2 & 5490 & 1098 & 1647 & Logistic & 0.1 & 100 & 182.2 \\
      \textbf{Crop-LeNet} & - & - & 25 & 2 & 5490 & 1098 & 1647 & Logistic & 0.1 & 100 & 182.2 \\
      \textbf{BNN-LeNet} & Position$_{2}$ (2D) & 20.00 & 7 & 1 & 5490 & 1098 & 1647 & Logistic & 0.1 & 100 & 182.2 \\
      \textbf{DeepCNet} & - & - & 9 & 1 & 5490 & 1098 & 1647 & Logistic & 0.1 & 100 & 182.2 \\
      \textbf{Crop-DeepCNet} & - & - & 9 & 1 & 5490 & 1098 & 1647 & Logistic & 0.1 & 100 & 182.2 \\
      \textbf{BNN-DeepCNet} & Position$_{2}$ (2D) & 40.00  & 7 & 1 & 5490 & 1098 & 1647 & Logistic & 0.1 & 100 & 182.2 \\
      \bottomrule
      \\
    \end{tabular}
    \mycaption{Experiment Protocols} {Experiment protocols for the different experiments presented in this work. \textbf{Feature Types}:
    Feature spaces used for the bilateral convolutions. Position$_1$ corresponds to un-normalized pixel positions whereas Position$_2$ corresponds
    to pixel positions normalized to $[0,1]$ with respect to the given image. \textbf{Feature Scales}: Cross-validated scales for the features used.
     \textbf{Filter Size}: Number of elements in the filter that is being learned. \textbf{Filter Nbr.}: Half-width of the filter. \textbf{Train},
     \textbf{Val.} and \textbf{Test} corresponds to the number of train, validation and test images used in the experiment. \textbf{Loss Type}: Type
     of loss used for back-propagation. ``MSE'' corresponds to Euclidean mean squared error loss and ``Logistic'' corresponds to multinomial logistic
     loss. ``Weighted Logistic'' is the class-weighted multinomial logistic loss. We weighted the loss with inverse class probability for material
     segmentation task due to the small availability of training data with class imbalance. \textbf{LR}: Fixed learning rate used in stochastic gradient
     descent. \textbf{Batch}: Number of images used in one parameter update step. \textbf{Epochs}: Number of training epochs. In all the experiments,
     we used fixed momentum of 0.9 and weight decay of 0.0005 for stochastic gradient descent. ```Color Upsampling'' experiments in this Table corresponds
     to those performed on Pascal VOC12 dataset images. For all experiments using Pascal VOC12 images, we use extended
     training segmentation dataset available from~\cite{hariharan2011moredata}, and used standard validation and test splits
     from the main dataset~\cite{voc2012segmentation}.}
  \label{tbl:parameters}
\end{table*}

\clearpage

\section{Parameters and Additional Results for Video Propagation Networks}

In this Section, we present experiment protocols and additional qualitative results for experiments
on video object segmentation, semantic video segmentation and video color
propagation. Table~\ref{tbl:parameters_supp} shows the feature scales and other parameters used in different experiments.
Figures~\ref{fig:video_seg_pos_supp} show some qualitative results on video object segmentation
with some failure cases in Fig.~\ref{fig:video_seg_neg_supp}.
Figure~\ref{fig:semantic_visuals_supp} shows some qualitative results on semantic video segmentation and
Fig.~\ref{fig:color_visuals_supp} shows results on video color propagation.

\newcolumntype{L}[1]{>{\raggedright\let\newline\\\arraybackslash\hspace{0pt}}b{#1}}
\newcolumntype{C}[1]{>{\centering\let\newline\\\arraybackslash\hspace{0pt}}b{#1}}
\newcolumntype{R}[1]{>{\raggedleft\let\newline\\\arraybackslash\hspace{0pt}}b{#1}}

\begin{table*}[h]
\tiny
  \centering
    \begin{tabular}{L{3.0cm} L{2.4cm} L{2.8cm} L{2.8cm} C{0.5cm} C{1.0cm} L{1.2cm}}
      \toprule
\textbf{Experiment} & \textbf{Feature Type} & \textbf{Feature Scale-1, $\Lambda_a$} & \textbf{Feature Scale-2, $\Lambda_b$} & \textbf{$\alpha$} & \textbf{Input Frames} & \textbf{Loss Type} \\
      \midrule
      \textbf{Video Object Segmentation} & ($x,y,Y,Cb,Cr,t$) & (0.02,0.02,0.07,0.4,0.4,0.01) & (0.03,0.03,0.09,0.5,0.5,0.2) & 0.5 & 9 & Logistic\\
      \midrule
      \textbf{Semantic Video Segmentation} & & & & & \\
      \textbf{with CNN1~\cite{yu2015multi}-NoFlow} & ($x,y,R,G,B,t$) & (0.08,0.08,0.2,0.2,0.2,0.04) & (0.11,0.11,0.2,0.2,0.2,0.04) & 0.5 & 3 & Logistic \\
      \textbf{with CNN1~\cite{yu2015multi}-Flow} & ($x+u_x,y+u_y,R,G,B,t$) & (0.11,0.11,0.14,0.14,0.14,0.03) & (0.08,0.08,0.12,0.12,0.12,0.01) & 0.65 & 3 & Logistic\\
      \textbf{with CNN2~\cite{richter2016playing}-Flow} & ($x+u_x,y+u_y,R,G,B,t$) & (0.08,0.08,0.2,0.2,0.2,0.04) & (0.09,0.09,0.25,0.25,0.25,0.03) & 0.5 & 4 & Logistic\\
      \midrule
      \textbf{Video Color Propagation} & ($x,y,I,t$)  & (0.04,0.04,0.2,0.04) & No second kernel & 1 & 4 & MSE\\
      \bottomrule
      \\
    \end{tabular}
    \mycaption{Experiment Protocols} {Experiment protocols for the different experiments presented in this work. \textbf{Feature Types}:
    Feature spaces used for the bilateral convolutions, with position ($x,y$) and color
    ($R,G,B$ or $Y,Cb,Cr$) features $\in [0,255]$. $u_x$, $u_y$ denotes optical flow with respect
    to the present frame and $I$ denotes grayscale intensity.
    \textbf{Feature Scales ($\Lambda_a, \Lambda_b$)}: Cross-validated scales for the features used.
    \textbf{$\alpha$}: Exponential time decay for the input frames.
    \textbf{Input Frames}: Number of input frames for VPN.
    \textbf{Loss Type}: Type
     of loss used for back-propagation. ``MSE'' corresponds to Euclidean mean squared error loss and ``Logistic'' corresponds to multinomial logistic loss.}
  \label{tbl:parameters_supp}
\end{table*}

% \begin{figure}[th!]
% \begin{center}
%   \centerline{\includegraphics[width=\textwidth]{figures/video_seg_visuals_supp_small.pdf}}
%     \mycaption{Video Object Segmentation}
%     {Shown are the different frames in example videos with the corresponding
%     ground truth (GT) masks, predictions from BVS~\cite{marki2016bilateral},
%     OFL~\cite{tsaivideo}, VPN (VPN-Stage2) and VPN-DLab (VPN-DeepLab) models.}
%     \label{fig:video_seg_small_supp}
% \end{center}
% \vspace{-1.0cm}
% \end{figure}

\begin{figure}[th!]
\begin{center}
  \centerline{\includegraphics[width=0.7\textwidth]{figures/video_seg_visuals_supp_positive.pdf}}
    \mycaption{Video Object Segmentation}
    {Shown are the different frames in example videos with the corresponding
    ground truth (GT) masks, predictions from BVS~\cite{marki2016bilateral},
    OFL~\cite{tsaivideo}, VPN (VPN-Stage2) and VPN-DLab (VPN-DeepLab) models.}
    \label{fig:video_seg_pos_supp}
\end{center}
\vspace{-1.0cm}
\end{figure}

\begin{figure}[th!]
\begin{center}
  \centerline{\includegraphics[width=0.7\textwidth]{figures/video_seg_visuals_supp_negative.pdf}}
    \mycaption{Failure Cases for Video Object Segmentation}
    {Shown are the different frames in example videos with the corresponding
    ground truth (GT) masks, predictions from BVS~\cite{marki2016bilateral},
    OFL~\cite{tsaivideo}, VPN (VPN-Stage2) and VPN-DLab (VPN-DeepLab) models.}
    \label{fig:video_seg_neg_supp}
\end{center}
\vspace{-1.0cm}
\end{figure}

\begin{figure}[th!]
\begin{center}
  \centerline{\includegraphics[width=0.9\textwidth]{figures/supp_semantic_visual.pdf}}
    \mycaption{Semantic Video Segmentation}
    {Input video frames and the corresponding ground truth (GT)
    segmentation together with the predictions of CNN~\cite{yu2015multi} and with
    VPN-Flow.}
    \label{fig:semantic_visuals_supp}
\end{center}
\vspace{-0.7cm}
\end{figure}

\begin{figure}[th!]
\begin{center}
  \centerline{\includegraphics[width=\textwidth]{figures/colorization_visuals_supp.pdf}}
  \mycaption{Video Color Propagation}
  {Input grayscale video frames and corresponding ground-truth (GT) color images
  together with color predictions of Levin et al.~\cite{levin2004colorization} and VPN-Stage1 models.}
  \label{fig:color_visuals_supp}
\end{center}
\vspace{-0.7cm}
\end{figure}

\clearpage

\section{Additional Material for Bilateral Inception Networks}
\label{sec:binception-app}

In this section of the Appendix, we first discuss the use of approximate bilateral
filtering in BI modules (Sec.~\ref{sec:lattice}).
Later, we present some qualitative results using different models for the approach presented in
Chapter~\ref{chap:binception} (Sec.~\ref{sec:qualitative-app}).

\subsection{Approximate Bilateral Filtering}
\label{sec:lattice}

The bilateral inception module presented in Chapter~\ref{chap:binception} computes a matrix-vector
product between a Gaussian filter $K$ and a vector of activations $\bz_c$.
Bilateral filtering is an important operation and many algorithmic techniques have been
proposed to speed-up this operation~\cite{paris2006fast,adams2010fast,gastal2011domain}.
In the main paper we opted to implement what can be considered the
brute-force variant of explicitly constructing $K$ and then using BLAS to compute the
matrix-vector product. This resulted in a few millisecond operation.
The explicit way to compute is possible due to the
reduction to super-pixels, e.g., it would not work for DenseCRF variants
that operate on the full image resolution.

Here, we present experiments where we use the fast approximate bilateral filtering
algorithm of~\cite{adams2010fast}, which is also used in Chapter~\ref{chap:bnn}
for learning sparse high dimensional filters. This
choice allows for larger dimensions of matrix-vector multiplication. The reason for choosing
the explicit multiplication in Chapter~\ref{chap:binception} was that it was computationally faster.
For the small sizes of the involved matrices and vectors, the explicit computation is sufficient and we had no
GPU implementation of an approximate technique that matched this runtime. Also it
is conceptually easier and the gradient to the feature transformations ($\Lambda \mathbf{f}$) is
obtained using standard matrix calculus.

\subsubsection{Experiments}

We modified the existing segmentation architectures analogous to those in Chapter~\ref{chap:binception}.
The main difference is that, here, the inception modules use the lattice
approximation~\cite{adams2010fast} to compute the bilateral filtering.
Using the lattice approximation did not allow us to back-propagate through feature transformations ($\Lambda$)
and thus we used hand-specified feature scales as will be explained later.
Specifically, we take CNN architectures from the works
of~\cite{chen2014semantic,zheng2015conditional,bell2015minc} and insert the BI modules between
the spatial FC layers.
We use superpixels from~\cite{DollarICCV13edges}
for all the experiments with the lattice approximation. Experiments are
performed using Caffe neural network framework~\cite{jia2014caffe}.

\begin{table}
  \small
  \centering
  \begin{tabular}{p{5.5cm}>{\raggedright\arraybackslash}p{1.4cm}>{\centering\arraybackslash}p{2.2cm}}
    \toprule
		\textbf{Model} & \emph{IoU} & \emph{Runtime}(ms) \\
    \midrule

    %%%%%%%%%%%% Scores computed by us)%%%%%%%%%%%%
		\deeplablargefov & 68.9 & 145ms\\
    \midrule
    \bi{7}{2}-\bi{8}{10}& \textbf{73.8} & +600 \\
    \midrule
    \deeplablargefovcrf~\cite{chen2014semantic} & 72.7 & +830\\
    \deeplabmsclargefovcrf~\cite{chen2014semantic} & \textbf{73.6} & +880\\
    DeepLab-EdgeNet~\cite{chen2015semantic} & 71.7 & +30\\
    DeepLab-EdgeNet-CRF~\cite{chen2015semantic} & \textbf{73.6} & +860\\
  \bottomrule \\
  \end{tabular}
  \mycaption{Semantic Segmentation using the DeepLab model}
  {IoU scores on the Pascal VOC12 segmentation test dataset
  with different models and our modified inception model.
  Also shown are the corresponding runtimes in milliseconds. Runtimes
  also include superpixel computations (300 ms with Dollar superpixels~\cite{DollarICCV13edges})}
  \label{tab:largefovresults}
\end{table}

\paragraph{Semantic Segmentation}
The experiments in this section use the Pascal VOC12 segmentation dataset~\cite{voc2012segmentation} with 21 object classes and the images have a maximum resolution of 0.25 megapixels.
For all experiments on VOC12, we train using the extended training set of
10581 images collected by~\cite{hariharan2011moredata}.
We modified the \deeplab~network architecture of~\cite{chen2014semantic} and
the CRFasRNN architecture from~\cite{zheng2015conditional} which uses a CNN with
deconvolution layers followed by DenseCRF trained end-to-end.

\paragraph{DeepLab Model}\label{sec:deeplabmodel}
We experimented with the \bi{7}{2}-\bi{8}{10} inception model.
Results using the~\deeplab~model are summarized in Tab.~\ref{tab:largefovresults}.
Although we get similar improvements with inception modules as with the
explicit kernel computation, using lattice approximation is slower.

\begin{table}
  \small
  \centering
  \begin{tabular}{p{6.4cm}>{\raggedright\arraybackslash}p{1.8cm}>{\raggedright\arraybackslash}p{1.8cm}}
    \toprule
    \textbf{Model} & \emph{IoU (Val)} & \emph{IoU (Test)}\\
    \midrule
    %%%%%%%%%%%% Scores computed by us)%%%%%%%%%%%%
    CNN &  67.5 & - \\
    \deconv (CNN+Deconvolutions) & 69.8 & 72.0 \\
    \midrule
    \bi{3}{6}-\bi{4}{6}-\bi{7}{2}-\bi{8}{6}& 71.9 & - \\
    \bi{3}{6}-\bi{4}{6}-\bi{7}{2}-\bi{8}{6}-\gi{6}& 73.6 &  \href{http://host.robots.ox.ac.uk:8080/anonymous/VOTV5E.html}{\textbf{75.2}}\\
    \midrule
    \deconvcrf (CRF-RNN)~\cite{zheng2015conditional} & 73.0 & 74.7\\
    Context-CRF-RNN~\cite{yu2015multi} & ~~ - ~ & \textbf{75.3} \\
    \bottomrule \\
  \end{tabular}
  \mycaption{Semantic Segmentation using the CRFasRNN model}{IoU score corresponding to different models
  on Pascal VOC12 reduced validation / test segmentation dataset. The reduced validation set consists of 346 images
  as used in~\cite{zheng2015conditional} where we adapted the model from.}
  \label{tab:deconvresults-app}
\end{table}

\paragraph{CRFasRNN Model}\label{sec:deepinception}
We add BI modules after score-pool3, score-pool4, \fc{7} and \fc{8} $1\times1$ convolution layers
resulting in the \bi{3}{6}-\bi{4}{6}-\bi{7}{2}-\bi{8}{6}
model and also experimented with another variant where $BI_8$ is followed by another inception
module, G$(6)$, with 6 Gaussian kernels.
Note that here also we discarded both deconvolution and DenseCRF parts of the original model~\cite{zheng2015conditional}
and inserted the BI modules in the base CNN and found similar improvements compared to the inception modules with explicit
kernel computaion. See Tab.~\ref{tab:deconvresults-app} for results on the CRFasRNN model.

\paragraph{Material Segmentation}
Table~\ref{tab:mincresults-app} shows the results on the MINC dataset~\cite{bell2015minc}
obtained by modifying the AlexNet architecture with our inception modules. We observe
similar improvements as with explicit kernel construction.
For this model, we do not provide any learned setup due to very limited segment training
data. The weights to combine outputs in the bilateral inception layer are
found by validation on the validation set.

\begin{table}[t]
  \small
  \centering
  \begin{tabular}{p{3.5cm}>{\centering\arraybackslash}p{4.0cm}}
    \toprule
    \textbf{Model} & Class / Total accuracy\\
    \midrule

    %%%%%%%%%%%% Scores computed by us)%%%%%%%%%%%%
    AlexNet CNN & 55.3 / 58.9 \\
    \midrule
    \bi{7}{2}-\bi{8}{6}& 68.5 / 71.8 \\
    \bi{7}{2}-\bi{8}{6}-G$(6)$& 67.6 / 73.1 \\
    \midrule
    AlexNet-CRF & 65.5 / 71.0 \\
    \bottomrule \\
  \end{tabular}
  \mycaption{Material Segmentation using AlexNet}{Pixel accuracy of different models on
  the MINC material segmentation test dataset~\cite{bell2015minc}.}
  \label{tab:mincresults-app}
\end{table}

\paragraph{Scales of Bilateral Inception Modules}
\label{sec:scales}

Unlike the explicit kernel technique presented in the main text (Chapter~\ref{chap:binception}),
we didn't back-propagate through feature transformation ($\Lambda$)
using the approximate bilateral filter technique.
So, the feature scales are hand-specified and validated, which are as follows.
The optimal scale values for the \bi{7}{2}-\bi{8}{2} model are found by validation for the best performance which are
$\sigma_{xy}$ = (0.1, 0.1) for the spatial (XY) kernel and $\sigma_{rgbxy}$ = (0.1, 0.1, 0.1, 0.01, 0.01) for color and position (RGBXY)  kernel.
Next, as more kernels are added to \bi{8}{2}, we set scales to be $\alpha$*($\sigma_{xy}$, $\sigma_{rgbxy}$).
The value of $\alpha$ is chosen as  1, 0.5, 0.1, 0.05, 0.1, at uniform interval, for the \bi{8}{10} bilateral inception module.


\subsection{Qualitative Results}
\label{sec:qualitative-app}

In this section, we present more qualitative results obtained using the BI module with explicit
kernel computation technique presented in Chapter~\ref{chap:binception}. Results on the Pascal VOC12
dataset~\cite{voc2012segmentation} using the DeepLab-LargeFOV model are shown in Fig.~\ref{fig:semantic_visuals-app},
followed by the results on MINC dataset~\cite{bell2015minc}
in Fig.~\ref{fig:material_visuals-app} and on
Cityscapes dataset~\cite{Cordts2015Cvprw} in Fig.~\ref{fig:street_visuals-app}.


\definecolor{voc_1}{RGB}{0, 0, 0}
\definecolor{voc_2}{RGB}{128, 0, 0}
\definecolor{voc_3}{RGB}{0, 128, 0}
\definecolor{voc_4}{RGB}{128, 128, 0}
\definecolor{voc_5}{RGB}{0, 0, 128}
\definecolor{voc_6}{RGB}{128, 0, 128}
\definecolor{voc_7}{RGB}{0, 128, 128}
\definecolor{voc_8}{RGB}{128, 128, 128}
\definecolor{voc_9}{RGB}{64, 0, 0}
\definecolor{voc_10}{RGB}{192, 0, 0}
\definecolor{voc_11}{RGB}{64, 128, 0}
\definecolor{voc_12}{RGB}{192, 128, 0}
\definecolor{voc_13}{RGB}{64, 0, 128}
\definecolor{voc_14}{RGB}{192, 0, 128}
\definecolor{voc_15}{RGB}{64, 128, 128}
\definecolor{voc_16}{RGB}{192, 128, 128}
\definecolor{voc_17}{RGB}{0, 64, 0}
\definecolor{voc_18}{RGB}{128, 64, 0}
\definecolor{voc_19}{RGB}{0, 192, 0}
\definecolor{voc_20}{RGB}{128, 192, 0}
\definecolor{voc_21}{RGB}{0, 64, 128}
\definecolor{voc_22}{RGB}{128, 64, 128}

\begin{figure*}[!ht]
  \small
  \centering
  \fcolorbox{white}{voc_1}{\rule{0pt}{4pt}\rule{4pt}{0pt}} Background~~
  \fcolorbox{white}{voc_2}{\rule{0pt}{4pt}\rule{4pt}{0pt}} Aeroplane~~
  \fcolorbox{white}{voc_3}{\rule{0pt}{4pt}\rule{4pt}{0pt}} Bicycle~~
  \fcolorbox{white}{voc_4}{\rule{0pt}{4pt}\rule{4pt}{0pt}} Bird~~
  \fcolorbox{white}{voc_5}{\rule{0pt}{4pt}\rule{4pt}{0pt}} Boat~~
  \fcolorbox{white}{voc_6}{\rule{0pt}{4pt}\rule{4pt}{0pt}} Bottle~~
  \fcolorbox{white}{voc_7}{\rule{0pt}{4pt}\rule{4pt}{0pt}} Bus~~
  \fcolorbox{white}{voc_8}{\rule{0pt}{4pt}\rule{4pt}{0pt}} Car~~\\
  \fcolorbox{white}{voc_9}{\rule{0pt}{4pt}\rule{4pt}{0pt}} Cat~~
  \fcolorbox{white}{voc_10}{\rule{0pt}{4pt}\rule{4pt}{0pt}} Chair~~
  \fcolorbox{white}{voc_11}{\rule{0pt}{4pt}\rule{4pt}{0pt}} Cow~~
  \fcolorbox{white}{voc_12}{\rule{0pt}{4pt}\rule{4pt}{0pt}} Dining Table~~
  \fcolorbox{white}{voc_13}{\rule{0pt}{4pt}\rule{4pt}{0pt}} Dog~~
  \fcolorbox{white}{voc_14}{\rule{0pt}{4pt}\rule{4pt}{0pt}} Horse~~
  \fcolorbox{white}{voc_15}{\rule{0pt}{4pt}\rule{4pt}{0pt}} Motorbike~~
  \fcolorbox{white}{voc_16}{\rule{0pt}{4pt}\rule{4pt}{0pt}} Person~~\\
  \fcolorbox{white}{voc_17}{\rule{0pt}{4pt}\rule{4pt}{0pt}} Potted Plant~~
  \fcolorbox{white}{voc_18}{\rule{0pt}{4pt}\rule{4pt}{0pt}} Sheep~~
  \fcolorbox{white}{voc_19}{\rule{0pt}{4pt}\rule{4pt}{0pt}} Sofa~~
  \fcolorbox{white}{voc_20}{\rule{0pt}{4pt}\rule{4pt}{0pt}} Train~~
  \fcolorbox{white}{voc_21}{\rule{0pt}{4pt}\rule{4pt}{0pt}} TV monitor~~\\


  \subfigure{%
    \includegraphics[width=.15\columnwidth]{figures/supplementary/2008_001308_given.png}
  }
  \subfigure{%
    \includegraphics[width=.15\columnwidth]{figures/supplementary/2008_001308_sp.png}
  }
  \subfigure{%
    \includegraphics[width=.15\columnwidth]{figures/supplementary/2008_001308_gt.png}
  }
  \subfigure{%
    \includegraphics[width=.15\columnwidth]{figures/supplementary/2008_001308_cnn.png}
  }
  \subfigure{%
    \includegraphics[width=.15\columnwidth]{figures/supplementary/2008_001308_crf.png}
  }
  \subfigure{%
    \includegraphics[width=.15\columnwidth]{figures/supplementary/2008_001308_ours.png}
  }\\[-2ex]


  \subfigure{%
    \includegraphics[width=.15\columnwidth]{figures/supplementary/2008_001821_given.png}
  }
  \subfigure{%
    \includegraphics[width=.15\columnwidth]{figures/supplementary/2008_001821_sp.png}
  }
  \subfigure{%
    \includegraphics[width=.15\columnwidth]{figures/supplementary/2008_001821_gt.png}
  }
  \subfigure{%
    \includegraphics[width=.15\columnwidth]{figures/supplementary/2008_001821_cnn.png}
  }
  \subfigure{%
    \includegraphics[width=.15\columnwidth]{figures/supplementary/2008_001821_crf.png}
  }
  \subfigure{%
    \includegraphics[width=.15\columnwidth]{figures/supplementary/2008_001821_ours.png}
  }\\[-2ex]



  \subfigure{%
    \includegraphics[width=.15\columnwidth]{figures/supplementary/2008_004612_given.png}
  }
  \subfigure{%
    \includegraphics[width=.15\columnwidth]{figures/supplementary/2008_004612_sp.png}
  }
  \subfigure{%
    \includegraphics[width=.15\columnwidth]{figures/supplementary/2008_004612_gt.png}
  }
  \subfigure{%
    \includegraphics[width=.15\columnwidth]{figures/supplementary/2008_004612_cnn.png}
  }
  \subfigure{%
    \includegraphics[width=.15\columnwidth]{figures/supplementary/2008_004612_crf.png}
  }
  \subfigure{%
    \includegraphics[width=.15\columnwidth]{figures/supplementary/2008_004612_ours.png}
  }\\[-2ex]


  \subfigure{%
    \includegraphics[width=.15\columnwidth]{figures/supplementary/2009_001008_given.png}
  }
  \subfigure{%
    \includegraphics[width=.15\columnwidth]{figures/supplementary/2009_001008_sp.png}
  }
  \subfigure{%
    \includegraphics[width=.15\columnwidth]{figures/supplementary/2009_001008_gt.png}
  }
  \subfigure{%
    \includegraphics[width=.15\columnwidth]{figures/supplementary/2009_001008_cnn.png}
  }
  \subfigure{%
    \includegraphics[width=.15\columnwidth]{figures/supplementary/2009_001008_crf.png}
  }
  \subfigure{%
    \includegraphics[width=.15\columnwidth]{figures/supplementary/2009_001008_ours.png}
  }\\[-2ex]




  \subfigure{%
    \includegraphics[width=.15\columnwidth]{figures/supplementary/2009_004497_given.png}
  }
  \subfigure{%
    \includegraphics[width=.15\columnwidth]{figures/supplementary/2009_004497_sp.png}
  }
  \subfigure{%
    \includegraphics[width=.15\columnwidth]{figures/supplementary/2009_004497_gt.png}
  }
  \subfigure{%
    \includegraphics[width=.15\columnwidth]{figures/supplementary/2009_004497_cnn.png}
  }
  \subfigure{%
    \includegraphics[width=.15\columnwidth]{figures/supplementary/2009_004497_crf.png}
  }
  \subfigure{%
    \includegraphics[width=.15\columnwidth]{figures/supplementary/2009_004497_ours.png}
  }\\[-2ex]



  \setcounter{subfigure}{0}
  \subfigure[\scriptsize Input]{%
    \includegraphics[width=.15\columnwidth]{figures/supplementary/2010_001327_given.png}
  }
  \subfigure[\scriptsize Superpixels]{%
    \includegraphics[width=.15\columnwidth]{figures/supplementary/2010_001327_sp.png}
  }
  \subfigure[\scriptsize GT]{%
    \includegraphics[width=.15\columnwidth]{figures/supplementary/2010_001327_gt.png}
  }
  \subfigure[\scriptsize Deeplab]{%
    \includegraphics[width=.15\columnwidth]{figures/supplementary/2010_001327_cnn.png}
  }
  \subfigure[\scriptsize +DenseCRF]{%
    \includegraphics[width=.15\columnwidth]{figures/supplementary/2010_001327_crf.png}
  }
  \subfigure[\scriptsize Using BI]{%
    \includegraphics[width=.15\columnwidth]{figures/supplementary/2010_001327_ours.png}
  }
  \mycaption{Semantic Segmentation}{Example results of semantic segmentation
  on the Pascal VOC12 dataset.
  (d)~depicts the DeepLab CNN result, (e)~CNN + 10 steps of mean-field inference,
  (f~result obtained with bilateral inception (BI) modules (\bi{6}{2}+\bi{7}{6}) between \fc~layers.}
  \label{fig:semantic_visuals-app}
\end{figure*}


\definecolor{minc_1}{HTML}{771111}
\definecolor{minc_2}{HTML}{CAC690}
\definecolor{minc_3}{HTML}{EEEEEE}
\definecolor{minc_4}{HTML}{7C8FA6}
\definecolor{minc_5}{HTML}{597D31}
\definecolor{minc_6}{HTML}{104410}
\definecolor{minc_7}{HTML}{BB819C}
\definecolor{minc_8}{HTML}{D0CE48}
\definecolor{minc_9}{HTML}{622745}
\definecolor{minc_10}{HTML}{666666}
\definecolor{minc_11}{HTML}{D54A31}
\definecolor{minc_12}{HTML}{101044}
\definecolor{minc_13}{HTML}{444126}
\definecolor{minc_14}{HTML}{75D646}
\definecolor{minc_15}{HTML}{DD4348}
\definecolor{minc_16}{HTML}{5C8577}
\definecolor{minc_17}{HTML}{C78472}
\definecolor{minc_18}{HTML}{75D6D0}
\definecolor{minc_19}{HTML}{5B4586}
\definecolor{minc_20}{HTML}{C04393}
\definecolor{minc_21}{HTML}{D69948}
\definecolor{minc_22}{HTML}{7370D8}
\definecolor{minc_23}{HTML}{7A3622}
\definecolor{minc_24}{HTML}{000000}

\begin{figure*}[!ht]
  \small % scriptsize
  \centering
  \fcolorbox{white}{minc_1}{\rule{0pt}{4pt}\rule{4pt}{0pt}} Brick~~
  \fcolorbox{white}{minc_2}{\rule{0pt}{4pt}\rule{4pt}{0pt}} Carpet~~
  \fcolorbox{white}{minc_3}{\rule{0pt}{4pt}\rule{4pt}{0pt}} Ceramic~~
  \fcolorbox{white}{minc_4}{\rule{0pt}{4pt}\rule{4pt}{0pt}} Fabric~~
  \fcolorbox{white}{minc_5}{\rule{0pt}{4pt}\rule{4pt}{0pt}} Foliage~~
  \fcolorbox{white}{minc_6}{\rule{0pt}{4pt}\rule{4pt}{0pt}} Food~~
  \fcolorbox{white}{minc_7}{\rule{0pt}{4pt}\rule{4pt}{0pt}} Glass~~
  \fcolorbox{white}{minc_8}{\rule{0pt}{4pt}\rule{4pt}{0pt}} Hair~~\\
  \fcolorbox{white}{minc_9}{\rule{0pt}{4pt}\rule{4pt}{0pt}} Leather~~
  \fcolorbox{white}{minc_10}{\rule{0pt}{4pt}\rule{4pt}{0pt}} Metal~~
  \fcolorbox{white}{minc_11}{\rule{0pt}{4pt}\rule{4pt}{0pt}} Mirror~~
  \fcolorbox{white}{minc_12}{\rule{0pt}{4pt}\rule{4pt}{0pt}} Other~~
  \fcolorbox{white}{minc_13}{\rule{0pt}{4pt}\rule{4pt}{0pt}} Painted~~
  \fcolorbox{white}{minc_14}{\rule{0pt}{4pt}\rule{4pt}{0pt}} Paper~~
  \fcolorbox{white}{minc_15}{\rule{0pt}{4pt}\rule{4pt}{0pt}} Plastic~~\\
  \fcolorbox{white}{minc_16}{\rule{0pt}{4pt}\rule{4pt}{0pt}} Polished Stone~~
  \fcolorbox{white}{minc_17}{\rule{0pt}{4pt}\rule{4pt}{0pt}} Skin~~
  \fcolorbox{white}{minc_18}{\rule{0pt}{4pt}\rule{4pt}{0pt}} Sky~~
  \fcolorbox{white}{minc_19}{\rule{0pt}{4pt}\rule{4pt}{0pt}} Stone~~
  \fcolorbox{white}{minc_20}{\rule{0pt}{4pt}\rule{4pt}{0pt}} Tile~~
  \fcolorbox{white}{minc_21}{\rule{0pt}{4pt}\rule{4pt}{0pt}} Wallpaper~~
  \fcolorbox{white}{minc_22}{\rule{0pt}{4pt}\rule{4pt}{0pt}} Water~~
  \fcolorbox{white}{minc_23}{\rule{0pt}{4pt}\rule{4pt}{0pt}} Wood~~\\
  \subfigure{%
    \includegraphics[width=.15\columnwidth]{figures/supplementary/000008468_given.png}
  }
  \subfigure{%
    \includegraphics[width=.15\columnwidth]{figures/supplementary/000008468_sp.png}
  }
  \subfigure{%
    \includegraphics[width=.15\columnwidth]{figures/supplementary/000008468_gt.png}
  }
  \subfigure{%
    \includegraphics[width=.15\columnwidth]{figures/supplementary/000008468_cnn.png}
  }
  \subfigure{%
    \includegraphics[width=.15\columnwidth]{figures/supplementary/000008468_crf.png}
  }
  \subfigure{%
    \includegraphics[width=.15\columnwidth]{figures/supplementary/000008468_ours.png}
  }\\[-2ex]

  \subfigure{%
    \includegraphics[width=.15\columnwidth]{figures/supplementary/000009053_given.png}
  }
  \subfigure{%
    \includegraphics[width=.15\columnwidth]{figures/supplementary/000009053_sp.png}
  }
  \subfigure{%
    \includegraphics[width=.15\columnwidth]{figures/supplementary/000009053_gt.png}
  }
  \subfigure{%
    \includegraphics[width=.15\columnwidth]{figures/supplementary/000009053_cnn.png}
  }
  \subfigure{%
    \includegraphics[width=.15\columnwidth]{figures/supplementary/000009053_crf.png}
  }
  \subfigure{%
    \includegraphics[width=.15\columnwidth]{figures/supplementary/000009053_ours.png}
  }\\[-2ex]




  \subfigure{%
    \includegraphics[width=.15\columnwidth]{figures/supplementary/000014977_given.png}
  }
  \subfigure{%
    \includegraphics[width=.15\columnwidth]{figures/supplementary/000014977_sp.png}
  }
  \subfigure{%
    \includegraphics[width=.15\columnwidth]{figures/supplementary/000014977_gt.png}
  }
  \subfigure{%
    \includegraphics[width=.15\columnwidth]{figures/supplementary/000014977_cnn.png}
  }
  \subfigure{%
    \includegraphics[width=.15\columnwidth]{figures/supplementary/000014977_crf.png}
  }
  \subfigure{%
    \includegraphics[width=.15\columnwidth]{figures/supplementary/000014977_ours.png}
  }\\[-2ex]


  \subfigure{%
    \includegraphics[width=.15\columnwidth]{figures/supplementary/000022922_given.png}
  }
  \subfigure{%
    \includegraphics[width=.15\columnwidth]{figures/supplementary/000022922_sp.png}
  }
  \subfigure{%
    \includegraphics[width=.15\columnwidth]{figures/supplementary/000022922_gt.png}
  }
  \subfigure{%
    \includegraphics[width=.15\columnwidth]{figures/supplementary/000022922_cnn.png}
  }
  \subfigure{%
    \includegraphics[width=.15\columnwidth]{figures/supplementary/000022922_crf.png}
  }
  \subfigure{%
    \includegraphics[width=.15\columnwidth]{figures/supplementary/000022922_ours.png}
  }\\[-2ex]


  \subfigure{%
    \includegraphics[width=.15\columnwidth]{figures/supplementary/000025711_given.png}
  }
  \subfigure{%
    \includegraphics[width=.15\columnwidth]{figures/supplementary/000025711_sp.png}
  }
  \subfigure{%
    \includegraphics[width=.15\columnwidth]{figures/supplementary/000025711_gt.png}
  }
  \subfigure{%
    \includegraphics[width=.15\columnwidth]{figures/supplementary/000025711_cnn.png}
  }
  \subfigure{%
    \includegraphics[width=.15\columnwidth]{figures/supplementary/000025711_crf.png}
  }
  \subfigure{%
    \includegraphics[width=.15\columnwidth]{figures/supplementary/000025711_ours.png}
  }\\[-2ex]


  \subfigure{%
    \includegraphics[width=.15\columnwidth]{figures/supplementary/000034473_given.png}
  }
  \subfigure{%
    \includegraphics[width=.15\columnwidth]{figures/supplementary/000034473_sp.png}
  }
  \subfigure{%
    \includegraphics[width=.15\columnwidth]{figures/supplementary/000034473_gt.png}
  }
  \subfigure{%
    \includegraphics[width=.15\columnwidth]{figures/supplementary/000034473_cnn.png}
  }
  \subfigure{%
    \includegraphics[width=.15\columnwidth]{figures/supplementary/000034473_crf.png}
  }
  \subfigure{%
    \includegraphics[width=.15\columnwidth]{figures/supplementary/000034473_ours.png}
  }\\[-2ex]


  \subfigure{%
    \includegraphics[width=.15\columnwidth]{figures/supplementary/000035463_given.png}
  }
  \subfigure{%
    \includegraphics[width=.15\columnwidth]{figures/supplementary/000035463_sp.png}
  }
  \subfigure{%
    \includegraphics[width=.15\columnwidth]{figures/supplementary/000035463_gt.png}
  }
  \subfigure{%
    \includegraphics[width=.15\columnwidth]{figures/supplementary/000035463_cnn.png}
  }
  \subfigure{%
    \includegraphics[width=.15\columnwidth]{figures/supplementary/000035463_crf.png}
  }
  \subfigure{%
    \includegraphics[width=.15\columnwidth]{figures/supplementary/000035463_ours.png}
  }\\[-2ex]


  \setcounter{subfigure}{0}
  \subfigure[\scriptsize Input]{%
    \includegraphics[width=.15\columnwidth]{figures/supplementary/000035993_given.png}
  }
  \subfigure[\scriptsize Superpixels]{%
    \includegraphics[width=.15\columnwidth]{figures/supplementary/000035993_sp.png}
  }
  \subfigure[\scriptsize GT]{%
    \includegraphics[width=.15\columnwidth]{figures/supplementary/000035993_gt.png}
  }
  \subfigure[\scriptsize AlexNet]{%
    \includegraphics[width=.15\columnwidth]{figures/supplementary/000035993_cnn.png}
  }
  \subfigure[\scriptsize +DenseCRF]{%
    \includegraphics[width=.15\columnwidth]{figures/supplementary/000035993_crf.png}
  }
  \subfigure[\scriptsize Using BI]{%
    \includegraphics[width=.15\columnwidth]{figures/supplementary/000035993_ours.png}
  }
  \mycaption{Material Segmentation}{Example results of material segmentation.
  (d)~depicts the AlexNet CNN result, (e)~CNN + 10 steps of mean-field inference,
  (f)~result obtained with bilateral inception (BI) modules (\bi{7}{2}+\bi{8}{6}) between
  \fc~layers.}
\label{fig:material_visuals-app}
\end{figure*}


\definecolor{city_1}{RGB}{128, 64, 128}
\definecolor{city_2}{RGB}{244, 35, 232}
\definecolor{city_3}{RGB}{70, 70, 70}
\definecolor{city_4}{RGB}{102, 102, 156}
\definecolor{city_5}{RGB}{190, 153, 153}
\definecolor{city_6}{RGB}{153, 153, 153}
\definecolor{city_7}{RGB}{250, 170, 30}
\definecolor{city_8}{RGB}{220, 220, 0}
\definecolor{city_9}{RGB}{107, 142, 35}
\definecolor{city_10}{RGB}{152, 251, 152}
\definecolor{city_11}{RGB}{70, 130, 180}
\definecolor{city_12}{RGB}{220, 20, 60}
\definecolor{city_13}{RGB}{255, 0, 0}
\definecolor{city_14}{RGB}{0, 0, 142}
\definecolor{city_15}{RGB}{0, 0, 70}
\definecolor{city_16}{RGB}{0, 60, 100}
\definecolor{city_17}{RGB}{0, 80, 100}
\definecolor{city_18}{RGB}{0, 0, 230}
\definecolor{city_19}{RGB}{119, 11, 32}
\begin{figure*}[!ht]
  \small % scriptsize
  \centering


  \subfigure{%
    \includegraphics[width=.18\columnwidth]{figures/supplementary/frankfurt00000_016005_given.png}
  }
  \subfigure{%
    \includegraphics[width=.18\columnwidth]{figures/supplementary/frankfurt00000_016005_sp.png}
  }
  \subfigure{%
    \includegraphics[width=.18\columnwidth]{figures/supplementary/frankfurt00000_016005_gt.png}
  }
  \subfigure{%
    \includegraphics[width=.18\columnwidth]{figures/supplementary/frankfurt00000_016005_cnn.png}
  }
  \subfigure{%
    \includegraphics[width=.18\columnwidth]{figures/supplementary/frankfurt00000_016005_ours.png}
  }\\[-2ex]

  \subfigure{%
    \includegraphics[width=.18\columnwidth]{figures/supplementary/frankfurt00000_004617_given.png}
  }
  \subfigure{%
    \includegraphics[width=.18\columnwidth]{figures/supplementary/frankfurt00000_004617_sp.png}
  }
  \subfigure{%
    \includegraphics[width=.18\columnwidth]{figures/supplementary/frankfurt00000_004617_gt.png}
  }
  \subfigure{%
    \includegraphics[width=.18\columnwidth]{figures/supplementary/frankfurt00000_004617_cnn.png}
  }
  \subfigure{%
    \includegraphics[width=.18\columnwidth]{figures/supplementary/frankfurt00000_004617_ours.png}
  }\\[-2ex]

  \subfigure{%
    \includegraphics[width=.18\columnwidth]{figures/supplementary/frankfurt00000_020880_given.png}
  }
  \subfigure{%
    \includegraphics[width=.18\columnwidth]{figures/supplementary/frankfurt00000_020880_sp.png}
  }
  \subfigure{%
    \includegraphics[width=.18\columnwidth]{figures/supplementary/frankfurt00000_020880_gt.png}
  }
  \subfigure{%
    \includegraphics[width=.18\columnwidth]{figures/supplementary/frankfurt00000_020880_cnn.png}
  }
  \subfigure{%
    \includegraphics[width=.18\columnwidth]{figures/supplementary/frankfurt00000_020880_ours.png}
  }\\[-2ex]



  \subfigure{%
    \includegraphics[width=.18\columnwidth]{figures/supplementary/frankfurt00001_007285_given.png}
  }
  \subfigure{%
    \includegraphics[width=.18\columnwidth]{figures/supplementary/frankfurt00001_007285_sp.png}
  }
  \subfigure{%
    \includegraphics[width=.18\columnwidth]{figures/supplementary/frankfurt00001_007285_gt.png}
  }
  \subfigure{%
    \includegraphics[width=.18\columnwidth]{figures/supplementary/frankfurt00001_007285_cnn.png}
  }
  \subfigure{%
    \includegraphics[width=.18\columnwidth]{figures/supplementary/frankfurt00001_007285_ours.png}
  }\\[-2ex]


  \subfigure{%
    \includegraphics[width=.18\columnwidth]{figures/supplementary/frankfurt00001_059789_given.png}
  }
  \subfigure{%
    \includegraphics[width=.18\columnwidth]{figures/supplementary/frankfurt00001_059789_sp.png}
  }
  \subfigure{%
    \includegraphics[width=.18\columnwidth]{figures/supplementary/frankfurt00001_059789_gt.png}
  }
  \subfigure{%
    \includegraphics[width=.18\columnwidth]{figures/supplementary/frankfurt00001_059789_cnn.png}
  }
  \subfigure{%
    \includegraphics[width=.18\columnwidth]{figures/supplementary/frankfurt00001_059789_ours.png}
  }\\[-2ex]


  \subfigure{%
    \includegraphics[width=.18\columnwidth]{figures/supplementary/frankfurt00001_068208_given.png}
  }
  \subfigure{%
    \includegraphics[width=.18\columnwidth]{figures/supplementary/frankfurt00001_068208_sp.png}
  }
  \subfigure{%
    \includegraphics[width=.18\columnwidth]{figures/supplementary/frankfurt00001_068208_gt.png}
  }
  \subfigure{%
    \includegraphics[width=.18\columnwidth]{figures/supplementary/frankfurt00001_068208_cnn.png}
  }
  \subfigure{%
    \includegraphics[width=.18\columnwidth]{figures/supplementary/frankfurt00001_068208_ours.png}
  }\\[-2ex]

  \subfigure{%
    \includegraphics[width=.18\columnwidth]{figures/supplementary/frankfurt00001_082466_given.png}
  }
  \subfigure{%
    \includegraphics[width=.18\columnwidth]{figures/supplementary/frankfurt00001_082466_sp.png}
  }
  \subfigure{%
    \includegraphics[width=.18\columnwidth]{figures/supplementary/frankfurt00001_082466_gt.png}
  }
  \subfigure{%
    \includegraphics[width=.18\columnwidth]{figures/supplementary/frankfurt00001_082466_cnn.png}
  }
  \subfigure{%
    \includegraphics[width=.18\columnwidth]{figures/supplementary/frankfurt00001_082466_ours.png}
  }\\[-2ex]

  \subfigure{%
    \includegraphics[width=.18\columnwidth]{figures/supplementary/lindau00033_000019_given.png}
  }
  \subfigure{%
    \includegraphics[width=.18\columnwidth]{figures/supplementary/lindau00033_000019_sp.png}
  }
  \subfigure{%
    \includegraphics[width=.18\columnwidth]{figures/supplementary/lindau00033_000019_gt.png}
  }
  \subfigure{%
    \includegraphics[width=.18\columnwidth]{figures/supplementary/lindau00033_000019_cnn.png}
  }
  \subfigure{%
    \includegraphics[width=.18\columnwidth]{figures/supplementary/lindau00033_000019_ours.png}
  }\\[-2ex]

  \subfigure{%
    \includegraphics[width=.18\columnwidth]{figures/supplementary/lindau00052_000019_given.png}
  }
  \subfigure{%
    \includegraphics[width=.18\columnwidth]{figures/supplementary/lindau00052_000019_sp.png}
  }
  \subfigure{%
    \includegraphics[width=.18\columnwidth]{figures/supplementary/lindau00052_000019_gt.png}
  }
  \subfigure{%
    \includegraphics[width=.18\columnwidth]{figures/supplementary/lindau00052_000019_cnn.png}
  }
  \subfigure{%
    \includegraphics[width=.18\columnwidth]{figures/supplementary/lindau00052_000019_ours.png}
  }\\[-2ex]




  \subfigure{%
    \includegraphics[width=.18\columnwidth]{figures/supplementary/lindau00027_000019_given.png}
  }
  \subfigure{%
    \includegraphics[width=.18\columnwidth]{figures/supplementary/lindau00027_000019_sp.png}
  }
  \subfigure{%
    \includegraphics[width=.18\columnwidth]{figures/supplementary/lindau00027_000019_gt.png}
  }
  \subfigure{%
    \includegraphics[width=.18\columnwidth]{figures/supplementary/lindau00027_000019_cnn.png}
  }
  \subfigure{%
    \includegraphics[width=.18\columnwidth]{figures/supplementary/lindau00027_000019_ours.png}
  }\\[-2ex]



  \setcounter{subfigure}{0}
  \subfigure[\scriptsize Input]{%
    \includegraphics[width=.18\columnwidth]{figures/supplementary/lindau00029_000019_given.png}
  }
  \subfigure[\scriptsize Superpixels]{%
    \includegraphics[width=.18\columnwidth]{figures/supplementary/lindau00029_000019_sp.png}
  }
  \subfigure[\scriptsize GT]{%
    \includegraphics[width=.18\columnwidth]{figures/supplementary/lindau00029_000019_gt.png}
  }
  \subfigure[\scriptsize Deeplab]{%
    \includegraphics[width=.18\columnwidth]{figures/supplementary/lindau00029_000019_cnn.png}
  }
  \subfigure[\scriptsize Using BI]{%
    \includegraphics[width=.18\columnwidth]{figures/supplementary/lindau00029_000019_ours.png}
  }%\\[-2ex]

  \mycaption{Street Scene Segmentation}{Example results of street scene segmentation.
  (d)~depicts the DeepLab results, (e)~result obtained by adding bilateral inception (BI) modules (\bi{6}{2}+\bi{7}{6}) between \fc~layers.}
\label{fig:street_visuals-app}
\end{figure*}

 \subsection{Details of PATE case study}

\begin{definition}[Renyi DP \citep{mironov2017renyi}]
    We say a randomized algorithm $\cM$ is $(\alpha, \epsilon_\cM(\alpha))$-RDP with order $\alpha \geq 1$ if for neighboring datasets $X, X'$
    \begin{align*}
    &\mathbb{D}_{\alpha}(\cM(X)||   \cM(X')):=\\
    & \frac{1}{\alpha-1}\log \mathbb{E}_{o \sim \cM(X')}\bigg[ \bigg( \frac{\pr[\cM(X)=o]}{\pr[\cM(X')=o]}\bigg)^\alpha \bigg]\leq \epsilon_\cM(\alpha).
    \end{align*}
\end{definition}
At the limit of $\alpha \to \infty$, RDP reduces to $(\epsilon, 0)$-DP. 
We now define the  data-dependent Renyi DP that conditioned on an input dataset $X$.
\begin{definition}[Data-dependent Renyi DP \citep{papernot2018scalable}]
    We say a randomized algorithm $\cM$ is $(\alpha, \epsilon_\cM(\alpha, X))$-RDP with order $\alpha \geq 1$ for dataset $X$ if for neighboring datasets $X'$
    \begin{align*}
    &\mathbb{D}_{\alpha}(\cM(X)||   \cM(X')):=\\
    & \frac{1}{\alpha-1}\log \mathbb{E}_{o \sim \cM(X')}\bigg[ \bigg( \frac{\pr[\cM(X)=o]}{\pr[\cM(X')=o]}\bigg)^\alpha \bigg]\leq \epsilon_\cM(\alpha, X).
    \end{align*}
\end{definition}

%we can take a functional view of RDP and use $\epsilon_\cM(\cdot)$ to denote the RDP as a lambda function of $\alpha$.  

RDP features two useful properties.
\begin{lemma}[Adaptive composition]
    $\epsilon_{(\cM_1, \cM_2)} = \epsilon_{\cM_1}(\cdot) + \epsilon_{\cM_2}(\cdot)$.
\end{lemma}
\begin{lemma}[From RDP to DP] If a randomized algorithm $\cM$ satisfies $(\alpha,\epsilon(\alpha))$-RDP, then $\cM$ also satisfies $(\epsilon(\alpha)+\frac{\log(1/\delta)}{\alpha-1},\delta)$-DP for any $\delta \in (0,1)$. \label{lem: rdp2dp}
\end{lemma}


\begin{definition}[Smooth Sensitivity]\label{def: smooth}
	Given the smoothness parameter $\beta$, a $\beta$-smooth sensitivity of $f(X)$ is defined as 
	\[SS_\beta(X):= \max_{d\geq 0} e^{-\beta d} \cdot \max_{\tilde{X'}: dist(X, \tilde{X'})\leq d} \Delta_{LS}(\tilde{X}')\]
\end{definition}


\begin{lemma}[Private upper bound of data-dependent  RDP, Restatement of Theorem~\ref{lem: upperbound}]]
Given a RDP function $\rdp(\alpha, X)$ and a $\beta$-smooth sensitivity bound $SS(\cdot)$ of $\rdp(\alpha, X)$. Let $\mu$ (defined in Algorithm~\ref{alg: pate_ptr}) denote the private release of $\log(SS_\beta(X))$. Let $(\beta, \sigma_s, \sigma_2)$-GNSS mechanism be 
\[\scriptstyle
\rdp^{\text{upper}}(\alpha):=\rdp(\alpha, X) + SS_\beta(X) \cdot \cN(0, \sigma_s^2) + \sigma_s \sqrt{2\log(\frac{2}{\delta_2}) } e^{\mu} \]
	 Then, the release of $\rdp^{\text{upper}}(X)$ satisfies $(\alpha, \frac{3\alpha +2}{2\sigma_s^2})$-RDP for all $1<\alpha < \frac{1}{2\beta}$; w.p. at least $1-\delta_2$, $\rdp^{\text{upper}}(\alpha)$ is an upper bound of $\rdp(\alpha, X)$.
\vspace{-2mm}
\end{lemma}


\begin{proof}[Proof sketch]
%Let $\epsilon_{\sigma_1}(\alpha)^p$ denotes the output of applying $(\beta,\sigma_s, \sigma_2)$-GNSS on $\epsilon_{\sigma_1}(X)$.
  
We first show that releasing the smooth sensitivity $SS_\beta$ with $e^\mu$ satisfies $(\alpha, \frac{\alpha}{2\sigma_2^2})$-RDP. Notice that the log of $SS_\beta(X)$ has a bounded global sensitivity $\beta$ (Definition~\ref{def: smooth} implies that $|\log SS_\beta(X)-\log SS_\beta(X')|\leq \beta $ for any neighboring dataset $X, X'$). By Gaussian mechanism, scaling noise with $\beta \sigma_2$ to $\log SS_\beta(X)$ is $(\alpha, \frac{\alpha}{2\sigma_2^2})$-RDP.
Therefore, the release of $\rdp(\alpha, X)$ is $(\alpha, \epsilon_s(\alpha)+\frac{\alpha}{2\sigma_2^2})$-RDP. Since the release  of $ f(X) + SS_\beta(X)\cdot \cN(0, \sigma_s^2)$ is $(\alpha, \frac{\alpha+1}{\sigma_s^2})$-RDP (Theorem 23 from \citet{papernot2018scalable}) for $\alpha<\frac{1}{2\beta}$, we have
$\epsilon_s(\alpha)+\frac{\alpha}{2\sigma_2^2}=\frac{3\alpha+2}{2\sigma_s^2}$.


We next prove the second statement. First, notice that with probability at least $1-\delta_2/2$, $e^\mu \geq SS_\beta(X)$ using the standard Gaussian tail bound.  Let $E$ denote the event that $e^{\mu}\geq SS_\beta(X)$. 


%We next prove that with probability at least $1-\delta_2$, $\epsilon_{\sigma_1}^p(\alpha)\geq 
%\epsilon_{\sigma_1}(\alpha, X)$. Let $E$ denote the event that $e^{\mu}\geq SS_\beta(X)$. 


\begin{align*}
   & \pr\bigg[\rdp^{\text{upper}}(\alpha)\leq \rdp(\alpha, X)\bigg] \\
   &=  \pr\bigg[\rdp^{\text{upper}}(\alpha) \leq \rdp(\alpha,X)|E\bigg] + \pr\bigg[\rdp^{\text{upper}}(\alpha)\leq \rdp(\alpha, X)|E^c\bigg]\\
   &\leq \pr\bigg[\rdp^{\text{upper}}(\alpha) \leq\rdp(\alpha, X)|E\bigg] + \delta_2/2\\
   &= \underbrace{\pr\bigg[\cN(0, \sigma_s^2)\cdot SS_{\beta(X)}\geq \sigma_s \cdot \sqrt{2\log(2/\delta_2)}e^{\mu} |E\bigg]}_{\text{denoted by} (*)} + \delta_2/2\\
\end{align*}



Condition on the event $E$, $e^{\mu}$ is a valid upper bound of $SS_\beta(X)$, which implies  \[ (*) \leq  \pr[\cN(0, \sigma_s^2)\cdot SS_\beta(X) \geq \sigma_s \cdot \sqrt{2\log(2/\delta_2)} SS_\beta(X) |E] \leq \delta_2/2\]
Therefore, with probability at least $1- \delta_2$, $\rdp^{\text{upper}}(\alpha) \geq \rdp(\alpha, X)$.
\end{proof}


\begin{theorem}[Restatement of Theorem~\ref{thm: pate_ptr}]
Algorithm~\ref{alg: pate_ptr} satisfies $(\epsilon'+\hat{\epsilon}, \delta)$-DP.
\end{theorem}
%short version

\begin{proof}
The privacy analysis consists of two components --- the privacy cost of releasing an upper bound of data-dependent RDP ($\epsilon_{\text{upper}}(\alpha):=\epsilon_s(\alpha)+\frac{\alpha}{2\sigma_2^2}$ and the valid upper bound $\epsilon_{\sigma_1}^p(\alpha)$.
First, set $\alpha =\frac{2\log(2/\delta)}{\epsilon}+1$ and use RDP to DP conversion with $\delta/2$ ensures that the cost of $\delta/2$ contribution to be roughly $\epsilon/2$ (i.e., $\frac{\log(2/\delta)}{\alpha-1} = \epsilon/2$). Second,  choosing $\sigma_s = \sqrt{\frac{2+3\alpha}{\epsilon}}$ gives us another $\epsilon/2$. 
\end{proof}
%by choosing $\beta = \frac{0.2}{\alpha}$, we can upper bound $\epsilon_{\text{upper}}(\alpha)$ with $\frac{2+3\alpha}{2\sigma_s^2}$ (the first term in $\epsilon_s(\alpha)$ will be the dominant term).Then,

%In the experiments, we consider a tighter DP version of Algorithm~\ref{alg: pate_ptr} by choosing
%$\sigma_s=\sigma_2$ as the input. The choices on other parameters $(\beta, \alpha, \delta_2)$ and the algorithm procedure remain unchanged. Then the algorithm satisfies $(\epsilon_{\sigma_1}^p(\alpha)+ \frac{\epsilon}{2} + \epsilon_s(\alpha)+ \frac{\alpha}{2\sigma_2^2}, \delta)$-DP. The only difference is that, we  use the  exact $\epsilon_s(\alpha)=\frac{\alpha \cdot e^{2\beta}}{\sigma_s^2} + \frac{\beta \alpha - 0.5 \ln(1-2\alpha \beta)}{ \alpha-1}$ without approximation. Then, the privacy cost of releasing the upper bound will be $\epsilon_s(\alpha)+\frac{\alpha}{2\sigma_2^2}$.

\textbf{Experimental details}
 $K=400$ teacher models are trained individually on the disjoint set using AlexNet model. We set $\sigma_2 = \sigma_s = 15.0$.   Our data-dependent RDP calculation and the smooth-sensitivity calculation follow \citet{papernot2018scalable}. Specifically, we use the following theorem (Theorem~6 from~\citet{papernot2018scalable}) to compute the data-dependent RDP of each unlabeled data $x$ from the public domain.

\begin{theorem}[data-dependent RDP ~\citet{papernot2018scalable}]
 Let $\tilde{q}\geq \pr[\cM(X)\neq \argmax_{j\in [C]} n_j(x)]$, i.e., an upper bound of the probability that the noisy label does not match the majority label. Assume $\alpha\leq \mu_1$ and $\tilde{q}\leq e^{(\mu_2 -1)\epsilon_2}/\bigg(\frac{\mu_1}{\mu_1 -1} \cdot \frac{\mu_2}{\mu_2 -1}\bigg)^{\mu_2}$, then we have:
 \[\epsilon_{\cM}(\alpha, X) \leq \frac{1}{\alpha-1}\log \bigg( (1-\tilde{q})\cdot A(\tilde{q}, \mu_2, \epsilon_2)^{\alpha-1} +\tilde{q}\cdot B(\tilde{q}, \mu_1, \epsilon_1)^{\alpha-1}\bigg)  \]
 where $A(\tilde{q}, \mu_2, \epsilon_2):= (1-\tilde{q})/\bigg(1-(\tilde{q}e^{\epsilon_2})^{\frac{\mu_2-1}{\mu_2}}\bigg)$, $B(\tilde{q},\mu_1, \epsilon_1)=e^{\epsilon_1}/\tilde{q}^{\frac{1}{\mu_1 -1}}, \mu_2=\sigma_1 \cdot \sqrt{\log(1/\tilde{q})}, \mu_1 = \mu_2 +1, \epsilon_1 = \mu_1/\sigma_1^2 $ and $\epsilon_2 = \mu_2/\sigma_2^2$.
    
\end{theorem}
 
In the experiments, the non-private data-dependent DP baseline is also based on the above theorem.  Notice that the data-dependent RDP of each query is a function of $\tilde{q}$, where $\tilde{q}$ denotes an upper bound of the probability where the plurality output does not match the noisy output. $\tilde{q}$ is a complex function of both the noisy scale and data and is not monotonically decreasing when $\sigma_1$ is increasing.  


\textbf{Simulation of two distributions.}
The motivation of the experimental design is to compare three approaches under different data distributions. 
Notice that there are $K=400$ teachers, which implies the number of the vote count for each class will be bounded by $400$. In the simulation of high-consensus distribution, we choose $T=200$ unlabeled public data such that the majority vote count will be larger than $150$ (i.e., $\max_{j\in[C]} n_j(x)>150$). For the low-consensus distribution, we choose to select $T$ unlabeled data such that the majority vote count will be smaller than $150$.





\section{Omitted proofs in private GLM}
\subsection{Per-instance DP of GLM}
\begin{theorem}[Per-instance differential privacy guarantee\label{thm: glm}]
	Consider two adjacent data sets $Z$ and $Z' =[Z, (x,y)]$, and denote the smooth part of the loss function $F_s =   \sum_{i=1}^n l(y_i,\langle x_i, \cdot\rangle) + r_s(\cdot)$ (thus $\tilde{F}_s = F_s  +  l(y,\langle x, \cdot \rangle)$.
	Let the local neighborhood be the line segment between $\theta^*$ and $\tilde{\theta}^*$. Assume 
	\begin{enumerate}
		\item the GLM loss function $l$ be convex, three-time continuous differentiable and $R$-generalized-self-concordant w.r.t. $\|\cdot\|_2$,
		\item $F_s$ is locally $\alpha$-strongly convex w.r.t. $\|\cdot\|_2$,
	%	\item Denote the maximum generalized leverage score in the local region $\mu  =  \sup_{\theta \in [\theta^*,\tilde{theta}^*]}  l''(y,x^T\theta) \|x\|_{H_{\theta^{-1}}^2}$
		\item and in addition, denote $L := \sup_{\theta\in [\theta^*,\tilde{\theta}^*]}|l'(y,x^T\theta)|$, $\beta := \sup_{\theta\in [\theta^*,\tilde{\theta}^*]}|l''(y,x^T\theta)|$.	
	%	$l(y,x^T\theta)$ be locally $L$-Lipschitz and $\beta$-smooth (namely, $|l'|\leq L$ and $l'' \leq \beta$).
	\end{enumerate}
	
	Then the algorithm obeys $(\epsilon,\delta)$-pDP for $Z$ and $z=(x,y)$ with any $0<\delta < 2/e$ and
$$
\epsilon \leq \epsilon_0(1+\log(2/\delta))  +  e^{\frac{RL\|x\|_2}{\alpha}} \left[\frac{\gamma L^2\|x\|_{H^{-1}}^2}{2} +  \sqrt{ \gamma L^2\|x\|_{H^{-1}}^2\log(2/\delta) }\right]
$$
%where $\epsilon_0 := (e^{\frac{\|v\|_2}{R}} -1)(1+\log(2/\delta))  +  2\mu_2 + \mu_1\log(2/\delta)$
where 
$\epsilon_0 \leq e^{\frac{RL\|x\|_2}{\alpha}} -1  + 2\beta \|x\|_{H_1^{-1}}^2 +  2\beta\|x\|_{\tilde{H}_2^{-1}}^2.$
If we instead assume that $l$ is $R$-self concordant. Then the same results hold, but with all $e^{\frac{RL\|x\|_2}{\alpha}}$ replaced with $(1-RL\|x\|_{H^{-1}})^2$.

\end{theorem}
	
	Under the stronger three-times continuous differentiable assumption, by mean value theorem, there exists $\xi$ on the line-segment between $\theta^*$ and $\ttheta^*$ such that 
	$$
	H = \left[\int_{t=0}^{1}\nabla^2 F_s((1-t)\theta^* + t\ttheta^*)  dt \right]  =  \nabla^2 F_s(\xi).
	$$
	
	The two distributions of interests are $\cN(\theta^*,  [\gamma \nabla^2 F_s(\theta^*)]^{-1})$ and $\cN(\ttheta^*, [\gamma \nabla^2 F_s(\ttheta^*) + \nabla^2l(y,x^T\ttheta^*)]^{-1}).$
	Denote $[\nabla^2 F_s(\theta^*)]^{-1} =: \Sigma$ and $[\nabla^2 F_s(\ttheta^*) + \nabla^2l(y,x^T\ttheta^*)]^{-1} =: \tilde{\Sigma}$.
	Both the means and the covariance matrices are different, so we cannot use multivariate Gaussian mechanism naively. Instead we will take the tail bound interpretation of $(\epsilon,\delta)$-DP and make use of the per-instance DP framework as internal steps of the proof. 
	
	First, we can write down the privacy loss random variable in analytic form
	\begin{align*}
	\log\frac{|\Sigma|^{-1/2}e^{- \frac{\gamma}{2}\|\theta -\theta^*\|_{\Sigma^{-1}}^2}}{|\tilde{\Sigma}|^{-1/2}e^{- \frac{\gamma}{2}\|\theta -\ttheta^*\|_{\tilde{\Sigma}^{-1}}^2}}
	=\underbrace{\frac{1}{2}\log \left(\frac{|\Sigma^{-1}|}{|\tilde{\Sigma}^{-1}|}\right)}_{(*)} +  \underbrace{\frac{\gamma}{2}\left[\|\theta -\theta^*\|_{\Sigma^{-1}}^2 - \|\theta -\ttheta^*\|_{\tilde{\Sigma}^{-1}}^2\right]}_{(**)}
	\end{align*}
	The general idea of the proof is to simplify the expression above and  upper bounding the two terms separately using self-concordance and matrix inversion lemma, and ultimately show that the privacy loss random variable is dominated by another random variable having an appropriately scaled shifted $\chi$-distribution, therefore admits a Gaussian-like tail bound.
	
	
	To ensure the presentation is readable, we define a few short hands. We will use $H$ and $\tilde{H}$ to denote the Hessian of $F_s$ and $F_s +  f$ respectively and subscript $1$ $2$ indicates whether the Hessian evaluated at at $\theta^*$ or $\ttheta^*$. $H$ without any subscript or superscript represents the Hessian of $F_s$ evaluated at $\xi$ as previously used.
	\begin{align*}
	(*)  = \frac{1}{2} \log  \frac{|H_1|}{ |H| }\frac{|H|}{|H_2|}\frac{|H_2|}{|\tilde{H}_2|}  \leq \frac{1}{2}\left[  	\log\frac{|H_1|}{ |H| }  + \log \frac{|H|}{|H_2|} + \log\frac{|H_2|}{|\tilde{H}_2|}\right]
	\end{align*}
	By the $R$-generalized self-concordance of $F_s$, we can apply Lemma~\ref{lem:selfconcordant-hessian}, 
	$$
-\|\theta^*-\xi\|_2R\leq \log\frac{|H_1|}{ |H| } \leq R\|\theta^*-\xi\|_2, \quad   -R\|\xi - \ttheta^*\|_2\leq \log\frac{|H|}{ |H_2| } \leq R\|\xi - \ttheta^*\|_2.
	$$
	The generalized linear model ensures that the Hessian of $f$ is rank-$1$:
	$$\nabla^2 f(\ttheta^*) =  l''(y,x^T\ttheta^*)  xx^T$$
	and we can apply Lemma~\ref{lem:determinant} in both ways (taking $A=H_2$ and $A=\tilde{H}_2$) and obtain
	$$
	\frac{|H_2|}{|\tilde{H}_2|}   =  \frac{1}{1 + l''(y,x^T\ttheta^*)x^T H_2^{-1}  x}  =  1- l''(y,x^T\ttheta^*)x^T\tilde{H}_2 x
	$$
	Note that $ l''(y,x^T\ttheta^*)x^T\tilde{H}_2^{-1} x$ is the in-sample leverage-score and $ l''(y,x^T\ttheta^*)x^T H_2^{-1}  x$ is the out-of-sample leverage-score of the locally linearized problem at $\ttheta^*$. We denote them by $\mu_2$ and $\mu'_2$ respectively (similarly, for the consistency of notations, we denote the in-sample and out of sample leverage score at $\theta^*$ by $\mu_1$ and $\mu'_1$ ). %Note that $\mu_2'\leq \mu_2 \leq \beta \|x\|_{H_2^{-1}}$ and $\mu_1'\leq \mu_1 \leq\beta \|x\|_{H_1^{-1}}$
	
Combine the above arguments we get
	\begin{align}\label{eq:der_part1}
	   (*)\leq&  R\|\theta^*-\xi\|_2 + R\|\xi - \ttheta^*\|_2  + \log (1 - \mu_2) \leq R\|\theta^*-\ttheta^*\|_2 + \log(1-\mu_2)\\
	   (*) \geq& -R\|\theta^*-\ttheta^*\|_2  - \log(1-\mu_2).
	\end{align}
	
We now move on to deal with the second part, where we would like to express everything in terms of $\|\theta-\theta^*\|_{H_1}$, which we know from the algorithm is $\chi$-distributed.
\begin{align*}
(**)  = \frac{\gamma}{2}\left[ \|\theta -\theta^*\|_{H_1}^2 - \|\theta -\theta^*\|_{H_2}^2  + \|\theta -\theta^*\|_{H_2}^2 - \|\theta -\ttheta^*\|_{H_2}^2+ \|\theta -\ttheta^*\|_{H_2}^2- \|\theta -\ttheta^*\|_{\tilde{H}_2}^2  \right]
\end{align*}
By the generalized self-concordance at $\theta^*$ %\|\theta -\theta^*\|_{H_1}^2 - 
\begin{align*}
e^{-R\|\theta^*-\ttheta^*\|_2}\|\cdot\|_{H_1}^2 \leq \|\cdot\|_{H_2}^2 \leq   e^{R\|\theta^*-\ttheta^*\|_2}\|\cdot\|_{H_1}^2
\end{align*}
This allows us to convert from $\|\cdot\|_{H_2}$ to $\|\cdot\|_{H_1}$, and as a consequence:
$$
\left|\|\theta -\theta^*\|_{H_1}^2 - \|\theta -\theta^*\|_{H_2}^2 \right|  \leq   [e^{R\|\theta^*-\ttheta^*\|_2} - 1]\|\theta -\theta^*\|_{H_1}^2.
$$
%\begin{align*}
%e^{-R\|\theta^*-\ttheta^*\|_2}\|\theta -\theta^*\|_{H_1}^2 \leq \|\theta -\theta^*\|_{H_2}^2 \leq   e^{R\|\theta^*-\ttheta^*\|_2}\|\theta -\theta^*\|_{H_1}^2
%\end{align*}
Also, 
\begin{align*}
 \|\theta -\theta^*\|_{H_2}^2 - \|\theta -\ttheta^*\|_{H_2}^2  &=  \left\langle \ttheta^* -\theta^* ,  2\theta - 2\theta^* + \theta^*-\ttheta^*  \right\rangle_{H_2}  =  2 \langle  \theta-\theta^*, \ttheta^* -\theta^* \rangle_{H_2} -  \|\theta^*-\ttheta^*\|_{H_2}^2
 \end{align*}
 Therefore
 \begin{align*}
 \left|  \|\theta -\theta^*\|_{H_2}^2 - \|\theta -\ttheta^*\|_{H_2}^2\right|  &\leq 2\|\theta - \theta^*\|_{H_2} \|\theta^*-\ttheta^*\|_{H_2}  +  \|\theta^*-\ttheta^*\|_{H_2}^2  \\
 &\leq 2e^{R\|\ttheta^* - \theta^*\|_2}\|\theta - \theta^*\|_{H_1} \|\theta^*-\ttheta^*\|_{H}  + e^{R\|\ttheta^* - \theta^*\|_2}\|\theta^*-\ttheta^*\|_{H}^2.
\end{align*}
Then lastly  we have
\begin{align*}
0\geq \|\theta -\ttheta^*\|_{H_2}^2- \|\theta -\ttheta^*\|_{\tilde{H}_2}^2 &=  -l''(y,x^T\ttheta^*)\left[ \langle x, \theta-\theta^* \rangle + \langle x,\theta^*-\ttheta^*\rangle\right]^2   \\
&\geq -2\beta \|x\|_{H_1^{-1}}^2\|\theta-\theta^*\|_{H_1}^2   -  2\beta \|x\|_{H^{-1}}^2\|\theta^*-\ttheta^*\|_{H}^2
\end{align*}
$$
\left|  \|\theta -\ttheta^*\|_{H_2}^2- \|\theta -\ttheta^*\|_{\tilde{H}_2}^2\right|  \leq 2\beta \|x\|_{H_1^{-1}}^2\|\theta-\theta^*\|_{H_1}^2   +  2\beta \|x\|_{H^{-1}}^2\|\theta^*-\ttheta^*\|_{H}^2
$$

Combine the above derivations, we get 
\begin{align}
\left|(**)\right|  \leq \frac{\gamma}{2}\left[  a \|\theta-\theta^*\|_{H_1}^2 + b \|\theta-\theta^*\|_{H_1}  +c\right] \label{eq:der_part2}
\end{align}
where 
\begin{align*}
a :=& \left[ e^{R\|\theta^*-\ttheta^*\|_2} -1  + 2\beta \|x\|_{H_1^{-1}}^2\right] \\
b:=& 2 e^{R\|\theta^*-\ttheta^*\|_2}   \|\theta^*-\ttheta^*\|_H \\
c:=& (e^{R\|\theta^*-\ttheta^*\|_2} + 2\beta \|x\|_{H^{-1}}^2)\|\theta^*-\ttheta^*\|_H^2
\end{align*}

Lastly, by \eqref{eq:der_part1} and $\eqref{eq:der_part2}$, 
$$
\left|  \log\frac{p(\theta|Z)}{p(\theta|Z')}  \right|  \leq R\|\theta^*-\ttheta^*\|_2  + \log(1-\mu_2)  +  \frac{\gamma}{2} [ a W^2 + bW + c].
$$
where according to the algorithm $W:= \|\theta-\theta^*\|_{H_1}$ follows a half-normal distribution with $\sigma=\gamma^{-1/2}$.

By standard Gaussian tail bound, we have for all $\delta<2/e$.
$$
\P(|W|\leq \gamma^{-1/2} \sqrt{\log(2/\delta)} )  \leq \delta.
$$
This implies that a high probability upper bound of the absolute value of the privacy loss random variable $\log \frac{p(\theta|Z)}{p(\theta|Z')}$ under $p(\theta|Z)$.
By the tail bound to privacy conversion lemma (Lemma~\ref{lem:tailbound2DP}), we get 
that for any set $S\subset \Theta$
$\P(\theta \in S | Z) \leq e^\epsilon \P(\theta \in S | Z') +\delta$
for any $0<\delta<2/e$ and 
$$
\epsilon  = R\|\theta^*-\ttheta^*\|_2  + \log(1-\mu_2)  + \frac{\gamma c}{2}  + \frac{a}{2}  \log(2/\delta)  +  \frac{\gamma^{1/2} b}{2}  \sqrt{\log(2/\delta)}.
$$
Denote $v:=  \theta^*-\ttheta^*$, by strong convexity
$$\|v\|_2\leq \|\nabla l(y,x^T\theta)[\ttheta^*]\|_2/\alpha  = |l'| \|x\|_2 / \alpha \leq L\|x\|_2/\alpha$$
and 
$$
\|v\|_H \leq \|\nabla l(y,x^T\theta)[\ttheta^*]\|_{H^{-1}}  =  |l'| \|x\|_{H^{-1}} \leq L\|x\|_{H^{-1}}.
$$
Also use the fact that $|\log(1-\mu_2)| \leq 2\mu_2$ for $\mu_2<0.5$ and $\mu_2\leq \beta\|x\|_{\tilde{H}_2^{-1}}^2 $, we can then combine similar terms and have a more compact representation.
%$$
%\epsilon=   \frac{\|v\|_2}{R} +  \log(1- \mu_2)  + (\mu_1  + e^{\frac{\|v\|_2}{R}} -1)\log(2/\delta)  +  e^{\frac{\|v\|_2}{R}} \left[\frac{\gamma \|v\|_H^2}{2} +  \sqrt{ \gamma \|v\|_H^2\log(1/\delta) }\right]
%$$
$$
\epsilon \leq \epsilon_0(1+\log(2/\delta))  +  e^{\frac{RL\|x\|_2}{\alpha}} \left[\frac{\gamma L^2\|x\|_{H^{-1}}^2}{2} +  \sqrt{ \gamma L^2\|x\|_{H^{-1}}^2\log(2/\delta) }\right]
$$
%where $\epsilon_0 := (e^{\frac{\|v\|_2}{R}} -1)(1+\log(2/\delta))  +  2\mu_2 + \mu_1\log(2/\delta)$
where 
$$\epsilon_0 \leq e^{\frac{RL\|x\|_2}{\alpha}} -1  + 2\beta \|x\|_{H_1^{-1}}^2 +  2\beta\|x\|_{\tilde{H}_2^{-1}}^2$$ 
is the part of the privacy loss that does not get smaller as $\gamma$ decreases.





\begin{proposition}\label{prop:generalnorm}
	Let $\|\cdot\|$ be a norm and $\|\cdot\|_*$ be its dual norm. Let $F(\theta)$, $f(\theta)$ and $\tilde{F}(\theta) = F(\theta) + f(\theta)$ be proper convex functions and $\theta^*$ and $\tilde{theta}^*$ be their minimizers, i.e., $0\in \partial F(\theta^*)$ and $0\in \partial \tilde{F}(\tilde{theta}^*)$.  If in addition, $F,\tilde{F}$ is $\alpha,\tilde{\alpha}$-strongly convex with respect to $\|\cdot\|$ within the restricted domain 
	$\theta \in \{  t\theta^* + (1-t)\tilde{\theta}^*  \;|\;  t\in[0,1]  \}$. 	Then there exists $g \in \partial f(\theta^*)$ and $\tilde{g}\in \partial f(\tilde{\theta}^*)$ such that
	$$
	\|\theta^*-\tilde{\theta}^*\| \leq\min\left\{\frac{1}{\alpha}\| \tilde{g}\|_*,  \frac{1}{\tilde{\alpha}}\| g\|_*\right\}.
	$$
	 %continuously differentiable with Lipschitz gradient. Let $\theta^*$ and $\ttheta^*$ be such that $0\in \partial F(\theta^*)$ and $0\in \partial \tilde{F}(\ttheta^*)$. If in addition $F(\theta)$ is $\alpha$-strongly convex with respect to $\|\cdot\|$ for all 
	%$\theta \in \{  t\theta^* + (1-t)\ttheta^*  \;|\;  t\in[0,1]  \}$. 
\end{proposition}
\begin{proof}
	Apply the first order condition to $F$ restricted to the line segment between $\tilde{\theta}^*$ and $\theta^*$, there are we get
	\begin{align}
	F(\tilde{\theta}^*) \geq F(\theta^*)  +  \langle \partial F(\theta^*),  \tilde{\theta}^*-\theta^* \rangle  + \frac{\alpha}{2}\|\tilde{\theta}^*-\theta^*\|^2\label{eq:strongcvx1} \\
	F(\theta^*) \geq F(\tilde{\theta}^*)  +  \langle \partial F(\tilde{\theta}^*),  \theta^*-\tilde{\theta}^* \rangle  + \frac{\alpha}{2}\|\tilde{\theta}^*-\theta^*\|^2 \label{eq:strongcvx2}
	\end{align}
	Note by the convexity of $F$ and $f$, $\partial\tilde{F}=  \partial F + \partial f$, where $+$ is the Minkowski Sum. Therefore, $0\in \partial\tilde{F}(\tilde{\theta}^*)$ implies that there exists $\tilde{g}$ such that $\tilde{g}\in \partial f(\tilde{\theta}^*)$ and $-\tilde{g}\in\partial F(\tilde{\theta}^*)$.
	Take $-\tilde{g}\in\partial F(\tilde{\theta}^*)$ in Equation~\ref{eq:strongcvx2} and $0 \in \partial F(\theta^*)$ in Equation~\ref{eq:strongcvx1}  and add the two inequalities, we obtain
	$$
		0\geq \langle -\tilde{g},  \theta^*-\tilde{\theta}^* \rangle  + \alpha \|\tilde{\theta}^* - \theta^*\|^2 \geq - \|\tilde{g}\|_* \|\theta^*-\tilde{\theta}^*\|  +  \alpha\|\tilde{\theta}^* - \theta^*\|^2. 
	$$
	For $\|\tilde{\theta}^* - \theta^*\|=0$ the claim is trivially true, otherwise, we can divide the both sides of the above inequality by $\|\tilde{\theta}^* - \theta^*\|$ and get
	$	\|\theta^*-\tilde{\theta}^*\| \leq \frac{1}{\alpha}\| \tilde{g}\|_*$. 
	
	It remains to show that $\|\theta^*-\tilde{\theta}^*\| \leq \frac{1}{\tilde{\alpha}}\|g\|_*$. This can be obtained by exactly the same arguments above but applying strong convexity to $\tilde{F}$ instead. Note that we can actually get something slightly stronger than the statement because the inequality holds for all $g\in \partial f(\theta^*)$.
\end{proof}



A consequence of (generalized) self-concordance is the spectral (\emph{multiplicative}) stability of Hessian to small perturbations of parameters.
\begin{lemma}[Stability of Hessian{\citep[Theorem~2.1.1]{nesterov1994interior}, \citep[Proposition~1]{bach2010self}}]\label{lem:selfconcordant-hessian}
	Let $H_\theta :=  \nabla^2F_s(\theta)$. If $F_s$ is $R$-self-concordant at $\theta$. Then for any $v$ such that $R \|v\|_{H_\theta} < 1$, we have that
	$$
	(1-R\|v\|_{H_\theta})^2 \nabla^2 F_s(\theta) 	\prec	\nabla^2 F_s(\theta+v) \prec  \frac{1}{(1-R\|v\|_{H_\theta})^2}   \nabla^2 F_s(\theta)  .
	$$
	If instead we assume $F_s$ is $R$-generalized-self-concordant at $\theta$ with respect to norm $\|\cdot\|$, then
	$$
	e^{-R\|v\|} \nabla^2 F_s(\theta) \prec  \nabla^2 F_s(\theta+v)  \prec e^{R\|v\|}  \nabla^2 F_s(\theta) 
	$$
\end{lemma}\label{stability}
The two bounds are almost identical when  $R\|v\|$ and $R\|v\|_{\theta}$ are close to $0$, in particular, for $x\leq 1/2$, $e^{-2x} \leq 1-x \leq e^{-x}$.
\bibliographystyle{plainnat}
\bibliography{gen_ptr}


\end{document}
