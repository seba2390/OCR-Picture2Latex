
\subsection{Ex-post DP}

\begin{theorem}
Algorithm~\ref{alg: ptr_gen_post} is $(\max(\epsilon_1 +\epsilon_2, \epsilon_2 -\epsilon_1 +\nu), 2\delta_2 +e^{\epsilon_2}\cdot \delta)$-DP if with probability at least $1-\delta$, we have $\epsilon(\theta, X)\leq \epsilon_\phi(X, \theta)^p\leq \epsilon_\phi(X,\theta) + \nu$ for any neighboring dataset $X'$.
\end{theorem}

\begin{proof}

Let $\tilde{\cM}$ denote Algorithm~\ref{alg: ptr_gen_post}.
We consider two cases: $\tilde{\cM}(X)=\perp$ and $\tilde{\cM}(X)=\theta$.
The probability of outputting $\perp$ can be written explicitly as follows:
\begin{align*}
&\pr[\tilde{\cM}(X)=\perp] = \pr_{\tilde{\cM}(X)}[\epsilon(\theta, X)^p>\epsilon_1]\\
&= \mathbb{E}_{\theta \sim \cM(X)}\bigg[ \pr_{\cA(X, \theta)}[\epsilon(\theta, X)^p >\epsilon_1|\theta]\bigg]\\
&\leq e^{\epsilon_2} \underbrace{{\mathbb{E}_{\theta \sim \cM(X)}\bigg[ \pr_{\cA(X',\theta)}\bigg[\epsilon(\theta, X')^p >\epsilon_1|\theta\bigg]\bigg]}}_{\text{denoted by} (*) } + \delta_2\\
\end{align*}
The inequality in the last line uses the definition of DP of $\cA(X,\theta)$, which satisfies $(\epsilon_2, \delta_2)$-DP.
Then our goal is to bound $(*)$.  Using the assumption ``w.h.p. $1-\delta$, $\epsilon(\theta, X)\leq \epsilon(X, \theta)^p\leq \epsilon(X,\theta) + \nu$, we have 

\[ (*)\leq \mathbb{E}_{\theta \sim \cM(X)}\bigg[ \pr_{\cA(X',\theta)}\bigg[\epsilon(\theta, X')^p >\epsilon_1
\cap E | \theta \bigg]\bigg] + \delta\]
where $E$ denotes the event of $\epsilon(\theta, X)\leq \epsilon(X, \theta)^p\leq \epsilon(X,\theta) + \nu$.

Note that \[\epsilon(\theta, X')^p \leq \epsilon(\theta, X') +\nu = \log \frac{\pr_{\cM(X')}(\theta)}{\pr_{\cM(X)}(\theta)} + \nu\], which combines with $\epsilon(\theta,X')^p>\epsilon_1$, gives us $\log \frac{\pr_{\cM(X)}(\theta)}{\pr_{\cM(X')}(\theta)} \leq -\epsilon_1 + \nu$. This inequality further implies $\pr_{\cM(X)}(\theta)\leq e^{\nu-\epsilon_1}\cdot \pr_{\cM(X')}(\theta)$.

Substituting the above expression to $(*)$ and we obtain
\begin{align*}
&(*) \leq e^{\nu -\epsilon_1}\cdot \mathbb{E}_{\theta \sim \cM(X')} \bigg[\pr_{\cA(X', \theta)}\bigg[\epsilon(X', \theta)^p >\epsilon \cap E \bigg]\bigg] + \delta\\
&\leq e^{-\epsilon_1 + \nu}\cdot \pr[{\tilde{\cM}(X')=\perp}] + \delta
\end{align*}
Therefore, we have 
\begin{align*}
& \pr[\tilde{\cM}(X)=\perp] \leq e^{\epsilon_2}\bigg(e^{-\epsilon_1 +\nu}\cdot \pr[\tilde{\cM}(X')=\perp ]+\delta\bigg)+ \delta_2\\
&= e^{\epsilon_2 -\epsilon_1 +\nu}\cdot \pr[\tilde{\cM}(X')=\perp] + \delta\cdot \epsilon^{\epsilon_2} +\delta_2
\end{align*}
We next proceed to bound the other case --- $\pr[\tilde{\cM}(X) \in S]$, where $S \subset Range(\cM)$.

 The probability of outputting $S \subset Range(\cM)$ is as follows:
 \begin{align}
 & \pr[\tilde{\cM}(X)\in S)] \\
 &= \underbrace{\mathbb{E}_{\theta \sim \cM(X)}\bigg[ \pr_{\cA(X,\theta)}\bigg[  \epsilon(X, \theta)^p\leq \epsilon_1, E|\theta \bigg]\mathbb{1}(\theta\in S)\bigg]}_{\text{denoted by} (*)} + \\
 & \underbrace{\mathbb{E}_{\theta \sim \cM(X)}\bigg[ \pr_{\cA(X,\theta)}\bigg[  \epsilon(X, \theta)^p\leq \epsilon_1, E^c |\theta\bigg] \mathbb{1}(\theta\in S)\bigg]}_{\text{denoted by} (**)}
\end{align}, where $E$ denotes the event of $\epsilon(\theta, X)\leq \epsilon( \theta,X) +\nu$.
Similarly, we apply the post-processing property on $(*)$ and we have
\begin{align}
&(*)\leq \mathbb{E}_{\theta \sim \cM(X)}\bigg[\bigg( \pr_{\cA(X',\theta)}[\epsilon(\theta, X')\leq \epsilon(\theta, X')^p\leq \epsilon_1|\theta]\cdot \epsilon^{\epsilon_2}+\delta\bigg)
\mathbb{1}(\theta \in S)\bigg]\\
&\leq e^{\epsilon_2}\int_S \pr[\cM(X)=\theta]\pr_{\cA(\theta, X')}[\epsilon(\theta, X')\leq \epsilon(\theta, X')^p \leq \epsilon_1]d\theta + \delta_2\\
&=e^{\epsilon_2}\int_S \int_{\epsilon(\theta, X')}^{\epsilon_1} \pr[\cM(X)=\theta]\pr_{\cA(\theta, X')}[\epsilon(\theta, X')^p]d \epsilon(\theta, X')^p d\theta + \delta \label{ine: change}\\
&\leq e^{\epsilon_2}\int_S \int_{\epsilon(\theta, X')}^{\epsilon_1}e^{\epsilon_1} \pr[\cM(X')=\theta] \pr_{\cA(\theta, X')}
\epsilon(\theta, X')^p d\epsilon(\theta, X')^pd\theta +\delta_2\\
&=e^{\epsilon_1 +\epsilon_2}\mathbb{E}_{\theta\sim \cM(X')}\bigg[ \pr_{\cA(\theta, X')}[\epsilon(\theta,X')\leq \epsilon(\theta, X')^p \leq \epsilon_1|\theta]\cdot \mathbb{1}(\theta \in S)\bigg] + \delta_2\\
&\leq e^{\epsilon_1 + \epsilon_2} \pr_{\theta \sim \cM(X')}[\theta \in S]+\delta_2
\end{align}
In Inequality~\ref{ine: change}, $\epsilon(\theta, X')^p \leq \epsilon_1 $ implies $\pr(\cM(X)=\theta)\leq e^{\epsilon_1}\cdot \pr[\cM(X')=\theta]$, which allows us to replace $\cM(X)=\theta$ with $\cM(X')=\theta$ in the next step.
For $(**)$, given that $\pr_{\cA(\theta, X)}[\epsilon(\theta, X)>\epsilon(\theta,X')^p]\leq \delta$, we have $(**)\leq \delta$. Therefore, we have 
\[\pr[\tilde{\cM}(X)\in S] \leq e^{\epsilon_1 + \epsilon_2 } \pr[\tilde{\cM}(X')\in S] + \delta_2 + \delta
\]
Combine the case of $\tilde{\cM}(X)=\perp$ and $\tilde{\cM}(X)\in \{Range(\cM)\}$, we have $\forall \tilde{S} \subset \{ Range(\cM) \cup \{\perp\}\}$:
\begin{align*}
    &\pr[\tilde{\cM}(X)\in \tilde{S}] = \pr[\tilde{\cM}(X) \in \tilde{S} \cap \perp^c] + \pr[\tilde{\cM}(X) = \perp]\\
    &\leq e^{\epsilon_1 + \epsilon_2}\pr[\tilde{\cM}(X) \in \tilde{S} \cap \perp^c] + e^{\epsilon_2 -\epsilon_1 +\nu}\pr[\tilde{\cM}(X')=\perp] + 2\delta_2 + e^{\epsilon_2} \delta
\end{align*}
\end{proof}