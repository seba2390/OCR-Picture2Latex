% Template article for preprint document class `elsart'
% SP 2006/04/26

%\documentclass{elsart}
\documentclass{article}
\usepackage{setspace}
\doublespacing
\usepackage{amsthm}
\usepackage{rotating}
\theoremstyle{definition}
\newtheorem{definition}{Definition}[section]
%\usepackage{amsart}
\usepackage{longtable}
\usepackage{url}
\usepackage{algorithm}
%\usepackage{algorithmicx}
\usepackage{multirow}
\usepackage{multicol}
\usepackage{algorithmic}
\usepackage{rotating}
% Use the option doublespacing or reviewcopy to obtain double line spacing
% \documentclass[doublespacing]{elsart}

% if you use PostScript figures in your article
% use the graphics package for simple commands
\usepackage{graphics}
% or use the graphicx package for more complicated commands
\usepackage{graphicx}
% or use the epsfig package if you prefer to use the old commands
% \usepackage{epsfig}

% The amssymb package provides various useful mathematical symbols
\usepackage{amssymb}
\usepackage{amsmath}
\usepackage{amsfonts}
\usepackage{calligra}
\usepackage{calrsfs}

% The lineno packages adds line numbers. Start line numbering with
% \begin{linenumbers}, end it with \end{linenumbers}. Or switch it on
% for the whole article with \linenumbers.
% \usepackage{lineno}

% \linenumbers
\begin{document}

%\begin{frontmatter}

% Title, authors and addresses

% use the thanksref command within \title, \author or \address for footnotes;
% use the corauthref command within \author for corresponding author footnotes;
% use the ead command for the email address,
% and the form \ead[url] for the home page:
% \title{Title\thanksref{label1}}
% \thanks[label1]{}
% \author{Name\corauthref{cor1}\thanksref{label2}}
% \ead{email address}
% \ead[url]{home page}
% \thanks[label2]{}
% \corauth[cor1]{}
% \address{Address\thanksref{label3}}
% \thanks[label3]{}
% use optional labels to link authors explicitly to addresses:
% \author[label1,label2]{} 
% \address[label1]{}
% \address[label2]{} 


\title{Learning Weighted Association Rules in Human Phenotype Ontology.}


\author{Pietro Hiram Guzzi, Giuseppe Agapito, Marianna Milano, Mario Cannataro}
\maketitle 
%\address{Department of Medical and Surgical Sciences, University Magna Graecia of Catanzaro, Italy}

\begin{abstract}
The Human Phenotype Ontology (HPO) is a structured repository of concepts (HPO Terms) that are associated to one or more diseases.  The process of association is referred to as annotation.  The relevance and the specificity of both HPO terms and annotations are evaluated by a measure defined as Information Content (IC). The analysis of annotated data is thus an important challenge for bioinformatics. There exist different approaches of analysis. From those, the use of Association Rules (AR) may provide useful knowledge, and it has been used in some applications, e.g. improving the quality of annotations. Nevertheless classical association rules algorithms do not take into account the source of annotation nor the importance yielding to the generation of candidate rules with low IC. This paper presents HPO-Miner (Human Phenotype Ontology-based Weighted Association Rules) a methodology for extracting Weighted Association Rules. HPO-Miner can extract relevant rules from a biological point of view. A case study on using of HPO-Miner on  publicly available HPO annotation datasets is used to demonstrate the effectiveness of our methodology.
\end{abstract}

%\begin{keyword}
%Association Rule Learning \sep Human Phenotype Ontology \sep Annotation Quality \sep Data Mining
% keywords here, in the form: keyword \sep keyword

% PACS codes here, in the form: \PACS code \sep code
%\PACS 
%\end{keyword}
%\end{frontmatter}

% main text
\section{Introduction}
\label{sec:Intro}



In computer science, the term ontology defines a set of representational primitives with which to model a domain of knowledge or discourse \cite{gruber2009ontology}. In particular, ontologies are mainly used in bioinformatics and computational biology.

For instance, the Gene Ontology aims to provide a common language to describe genes product \cite{gene2004gene}. More recently, the annotation efforts have also focused on the description of relation among molecular biology and disease, leading to the introduction of novel ontologies such as Human Phenotype Ontology (HPO) \cite{} and Disease Ontology (DO) \cite{}.

HPO aims to provide a standardized vocabulary of phenotypic abnormalities encountered in human diseases. A generic HPO annotation contains a link between a disease and phenotypic abnormality. A disease is indexed by using a unified identifier known as Online Mendelian Inheritance in Man (OMIM). OMIM is a comprehensive, authoritative compendium of human genes and genetic phenotypes that are freely available and updated daily \cite{hamosh2005online}. The Disease Ontology (DO) has been developed as a standardized ontology for human disease with the purpose of providing strong and sustainable descriptions of human disease terms and phenotype characteristics \cite{schriml2012disease}. 


The amount of annotations available is steadily growing, raising new challenges to face, related to ambiguous or incomplete annotations and ontology terms \cite{flouris2006inconsistencies}. The annotation task is becoming an even harder challenge in the genomic era, which is characterized by an unprecedented growth in the production of genes, gene products, and even other information. To speed-up the updating and maintenance processes of ontologies and annotations, it is required the development of computational approaches that guarantee a remarkable speed, on the current approaches of annotation carried out manually by the curators. 
The literature contains several computational methods developed to aid GO curators to improve GO annotations consistency \cite{yeh2003knowledge}, \cite{10.1371/journal.pone.0040519}, \cite{manda2012cross}. As opposed to GO, in literature, there are only a few automatic methodologies able to aid the HPO curators to improve annotation consistency and retrieve link between terms not explicitly related.  

As demonstrated in some recent works by Faria et al. \cite{faria2012}, by Manda et al. \cite{manda2013interestingness}, and by Agapito et al. \cite{agapito2014improving,agapito2015using},  association rules may be used to improve annotations consistency and highlight relationships among terms did not seem explicitly related. In this work, we present HPO-Miner an improvement of our previous works in which we introduced GO-WAR \cite{agapito2015using}. HPO-Miner is a tool for learning weighted association rules (WAR) to check annotation consistency and to identify hidden relationships between two phenotype abnormalities from HPO.
Traditional association rule approaches are not able to distinguish between items; they are unaware of the relevance of terms yielding to the generation of rules with low specificity. The specificity of each term may be measured by the information content (IC) of a term \cite{harispe2013framework}. The use of IC computed for each HPO term, is a measure of the specificity of a term, yielding to the IC-weighted annotation as conveyed in the following: \textit{OMIM100100: (HP:0000126, 11.18), (HP:0000144, 9.57)}.
HPO-Miner is able to extract weighted association rules starting from an annotated dataset of diseases. The proposed approach is based on the following steps: (i) initially we rearrange the information for each OMIM term to get transactional data; (ii) then, we extract weighted association rules using a modified \textit{FP-Tree} like algorithm able to deal with the dimension of classical biological datasets. We use publicly available HPO annotation data to demonstrate our method. 


The rest of the paper is structured as follows:  Section \ref{sec:methods} discusses HPO-Miner methodology and implementation, Section \ref{sec:results} presents results of the application of HPO-Miner on a biological dataset. Finally Section \ref{sec:conclusion} concludes the paper.



\section{Materials and Methods}
\label{sec:methods}

\subsection{The Human Phenotype Ontology}

 HPO is a structured and controlled vocabulary with more than 10,000 terms able to describe the phenotypic abnormalities in human diseases. HPO provides annotations of more than 7,000 human hereditary syndromes and other phenotypic abnormalities that characterize the diseases, are also available at the website \footnote{http://www.human-phenotype-ontology.org}. 
 HPO consists of three independent sub-ontologies: the \textit{mode of inheritance} i.e. the way in which a specific hereditary attribute is transmitted from a generation to another, \textit{onset and clinical course} i.e. in medicine refers to the first symptoms of a sickness and the medical treatments involved to cure them and finally, the \textit{phenotypic abnormalities} i.e.  the abnormal traits of a living organism that are possible to observe. As other ontologies, terms in HPO are organized in a directed acyclic graph (DAG). The relations among DAG's terms are modelled by means of \textit{is\_a} and \textit{part\_of} edges "relations", in order to distinguish between general or specific terms. Moreover, terms in HPO are arranged in a hierarchical way, where each path respects the \textit{true-path-rule}. To each HPO class is assigned a  stable and unique identifier (e.g. \textit{HP:0001629}), a label and a list of synonyms,  describing a well definite phenotypic abnormality i.e. "\textit{Ventricular Septal Defect}" see Figure \ref{fig:path}. 
 
  
 \begin{figure}[h]
\centering
 \includegraphics[width=4in]{figure1.eps}
   \caption{HPO graph Example}
\label{fig:path}
\end{figure}


% Annotazioni ottenute con HPO
%About 110,301 annotations to HPO terms for 7,354 diseases listed in the OMIM (Online Mendelian Inheritance in Man) database are also available at the website (http://www.human-phenotype-ontology.org). 
Diseases are annotated with terms of the HPO, meaning that HPO terms are used to describe all the signs, symptoms, and other phenotypic manifestations that characterize the disease in question. %Since HPO contains information related to phenotypic abnormalities, the computation of semantic similarities among concepts annotated with HPO terms, may enable database searches for clinical diagnostics or computational analysis of gene expression patterns associated with human diseases \cite{kohler2009clinical, Peng01012013}.
 
The annotations of OMIM entries are a mixture of manual annotations performed by the HPO curators team and automated matching of the OMIM Clinical Synopsis to HPO term labels. In particular HPO is an ontology designed to provide qualitative information and not to capture quantitative information such as body weight or height. Each diseases may be annotated to multiple HPO terms. 
Consequently the need of the introduction of methodologies and tools to support HPO curators to improve annotation consistency and the structure of the ontology arises. %With the goal to promote the interoperability among different researcher fields and in particular to achieve the interoperability between the ontologies belonging to the OBO foundry \cite{Smith2007}. 
%For these reasons, we proposed in the past HPO-Miner \cite{7172489}, a data-mining strategy based on weighted-association rule mining to support GO curators. Furthermore, literature reports different approaches based on associative rules mining (ARM) from annotated data \cite{faria2012, 10.1371/journal.pone.0084475, manda2012cross, Agapito2015113}. 


\subsection{Association Rules}% as tool to improve annotation consistency}

Association Rule (AR) extraction is very popular in data mining, it is used for discovering associations in market basket analysis and unknown relations among features in databases. Historically, was proposed by Agrawal \cite{citeulike:1005421} to discovery associations to support marketing decision. %For instance, discovery regularity in shopping behaviours of the customers, could help the managers to define better marketing strategies. Association rule can be generalized and used in different research fields that are interested in finding hidden pattern between nominal variables in a given dataset \cite{surveychen}. 
 
Formally, the association rules extraction problem may be stated as follows: let $ I=\{i_1,i_2,\ldots,i_{n}\}$ be a set of items %(HPO terms), 
and $D = \{t_1, \ldots, t_m \}$ a transactional database that contains a set of transactions, where %where each transaction represent a HPO class annotated with one or more HPO terms, as depicted in Figure \cite{}. The number of items contained in a transaction is defined as \textit{transaction width}.
 a transaction $t_j$ is a subset of items belonging to $I$. An association rule is an implication of the form $A \rightarrow B$, where $A$ and $B$ are two disjoint sets. %this is, $ A \cap B =\emptyset$. 
 AR are based on two fundamental properties to define the relevance of the mined rules, \textit{Support} and \textit{Confidence}.
The formal Support definition is: 
\begin{definition}
\centering	
	$S(A \rightarrow B) = \frac{  \sigma (A \cup B)}{N} $
\end{definition}
Where $N$ is the total number of transactions contained in $D$ and $\sigma$ is called \textit{support count}, namely, the number of transaction that contain a particular item.
\newline The Confidence is defined as: 
\begin{definition}
\centering	
	$C(A \rightarrow B) = \frac{  \sigma (A \cup B)}{\sigma (A)} $.
\end{definition}

Where $\sigma(A)$ is the number of transactions in \textit{D} containing A and $\sigma(A\cup B)$ is the number of transactions in \textit{D} that contains both items A and B.\newline

A drawback with the use of classical AR approach is that it precludes the derivation of certain rules in which the items have a very different levels of support. In several areas do not make sense to assign equal importance to all items involved in the dataset. For example in the supermarket context, some items like computer, smartphone have much  value than trivial items like ice-cream or butter. Rules involving smartphone or computer have less support than those involving butter or ice-cream but are much more significant in term of profit by the store. In the ontology context, the term HP:0000924 (\textit{An abnormality of the skeletal system}) has a relevance value (IC value) lower than HP:0011803 (\textit{Bifid nose}) although it is much more frequent. Rules involving the term HP:0000924 are less interesting (as it is a more general term) then rules involving the term HP:0011803 (as it is a more specific term) in terms of actionable knowledge. 

This limitation of classical AR approach can be overcome by introducing the weighted association rules (WAR). WAR models the \textit{significance} of a term by means of a \textit{weight} ($\omega$). A weight ($\omega$) is a positive real number that reflect the relevance of a HPO terms,  for which high values represent very significant items as reported in \cite{Wang:2000:EMW:347090.347149, 694360}. In our case, the relevance can be represented by using the information content (\textit{IC}). %Starting from the HPO dataset depicted in Figure \ref{fig:HPO_daataset}, we defined a new HPO dataset by adding for each term the related value of IC as shown in Figure \ref{fig:w_HPO_dataset} and called weighted HPO dataset. The weighted HPO dataset has to be converted in transaction database by using the same methodology described above, producing the weighted transaction HPO dataset as depicted in Figure \ref{fig:WD}.

A generic HPO dataset is a list of \textit{OMIM} identifiers annotated with multiple HPO terms, as conveyed in Figure \ref{fig:HPO_daataset}.

\begin{figure}[h]
\centering
 \includegraphics[width=2.0in]{HPODataset}
   \caption{An example of HPO dataset.}
\label{fig:HPO_daataset}
\end{figure}

In order to extract rules from the HPO dataset, it is necessary to convert it in a format more suitable to represent transaction data. The conversion consists in put together the same OMIM identifiers that became the transaction \textit{identifier} while the HPO terms associated with the current OMIM identifier are the items of the transaction, as depicted in Figure \ref{fig:WD}. %%%

%\begin{figure}[h]
%\centering
% \includegraphics[width=4.0in]{D}
%   \caption{An example of transaction HPO dataset.}
%\label{fig:D}
%\end{figure}


 
%traduzione del weighted hopo dataset.
%\begin{figure}[h]
%\centering
% \includegraphics[width=2.5in]{wHPODataset}
%   \caption{A simple weighted HPO dataset.}
%\label{fig:w_HPO_dataset}
%\end{figure}

\begin{figure}[h]
\centering
 \includegraphics[width=4.0in]{WD}
   \caption{An example of weighted transaction HPO dataset.}
\label{fig:WD}
\end{figure}

%Thus relevance of a term could be modeled by means of a non negative number called \textit{weight} ($\omega$) for which high values represent very significant items as reported in \cite{Wang:2000:EMW:347090.347149, 694360}. %Consequently, AR algorithms may take into account the relevance of items balancing thus relevance and frequency as explained in the fundamental works of Agapito et. al. \cite{7172489, Agapito2015113}, Wang et. al., \cite{Wang:2000:EMW:347090.347149} and Cai et. al. \cite{694360}.  Following a similar approach we here state the problem of extraction of weighted association rules from HPO data, introducing main notions. 


%There are different possible methods to solve the conflict between the frequency and the weight of each item. However, we have chosen to use the sum of the multiplication between weight and frequency because this balances the effects of weight and frequency. 

 

\subsection{Weighting HPO term with Information Content}
Each HP term is associated to IC value. There exist different IC formulations  that fall into two classes, intrinsic and extrinsic methods. Intrinsic method rely on the topology of the GO graph analyzing the positions of terms in a taxonomy. In this way the approaches define information content for each term. Different topological characteristics as ancestors, number of children, depth (see\cite{harispe2013framework} for a complete review) can used in order to estimate the Intrinsic IC calculus. Instead the extrinsic approaches involve annotation data for an considered corpus.
In this work we used the intrinsic method proposed by Sanchez et al. \cite{sanchez2011ontology}, Harispe et al.\cite{harispe2013framework}, Resnick et al. \cite{resnink:simmeasure:879855}, Seco et al.  \cite{14755292}, Zhou et al. \cite{zhou2008new}. 

The measure of Sanchez exploits only the number of leaves and the set of  ancestors of $a$ including itself, \emph{subsumers(a)} and introduce the root node as the number of leaves \emph{max\_leaves} in IC assessment. Leaves are more informative than concepts with many leaves, roots, so the leaves are suited  to describe and to distinguish any concept.
\begin{equation}
IC_{Sanchez\,et\,al.}(a) =-log\left(\frac{\frac{|leaves(a)|}{|subsumers(a)|}+1}{max\_leaves+1} \right)
\end{equation}

Harispe et al., in oder to highlights the specificity of leaves according to their number of ancestors,   consider \emph{leaves(a)} = \emph{a} concept when \emph{a} is a root and evaluating \emph{max\_leaves} as the number of inclusive ancestors of a node revising  the IC assessment  suggested by Sanchez et al.
\begin{equation}
IC_{Harispe\,et\,al.}(a) =-log\left(\frac{\frac{|leaves(a)|}{|subsumers(a)|}}{max\_leaves} \right)
\end{equation}
 
  The  formulation provided from Resnick et al. computes the IC of a concept  evaluating all the top-downs path from a concept $a$ to the reachable leaves, $p(a)$, and then calculates the log  yielding to the formula: \begin{equation}
IC_{Resnik}(a) = -log(p(a)).
 \end{equation}


Seco et al. calculate the IC of a concept by considering the ratio between the number of hyponyms in ontology, for example, the number of descendant with respect to the whole  number of ontological concepts.
\begin{equation}
{IC_{Seco\,et\,al}(a) =\frac{log\left( \frac{hypo(a)+1}{max\_nodes} \right)}{log\left( \frac{1}{max\_nodes} \right)} }
\end{equation}


 Thus Zhou et al. considers the depth of a concept in a taxonomy, \emph{depth(a)}, and the maximum depth of the taxonomy  \emph{max\_depth}.
\begin{equation}
\small{IC_{Zhou\,et\,al.}(a) =k-\left(1- \frac{log(hypo(a)+1)}{log(max\_nodes)} \right)+(1-k)\left(\frac{log(depth(a))}{log(depth\_nodes)} \right)}
\end{equation}
In this formulation K is a factor which enables to weight the contribution of the two evaluated features.








\section{The HPO-Miner Algorithm}
\label{sec:HPO-Miner}

In this section we briefly describe the \textit{HPO-Miner} algorithm, developed to extract weighted association rules form HPO dataset. 

First of all we define the \textit{Weighted Item $x$}, i.e. a weighted HPO item is obtained by multiplying the number of occurrences of item $x$ by the value of its related value of IC (the weight $\omega$). We define as $Weighted Support$, ($\omega S$), obtained by integrating  the classical formulation of the support of an item by its weight. The weighted Support \textit{$\omega S$} of a generic item $x_i$ is defined as: $\omega S(x_i)=w_i*\sigma(x_i)$ where $\omega _i$ is the information content of the \textit{i-th} term and $\sigma(x_i)$ is the number of transaction containing $x_i$. Let $I=\{i_1 \ldots i_m\}$ be a set of weighted items (HPO terms) and let $WD$ be a set of weighted transactions database, where each transaction $t_j$ is a sub-set of weighted items such that $t_j$ belongs to $I$. %, each of which may appear in multiple transactions with the same weight. 
 We defined the \textit{weighted minimum support ($m\omega S$)} as:\\
\begin{definition}
\centering	
$m\omega S =  \left(\frac{\sum_{i=1}^{|WD|} \sigma(x_i)*\omega_i} {|WD|}\right)*p$.
 \end{definition}
 Where, $|WD|$ is the cardinality of the weighted database nominally, the number of transactions into the dataset, $p$ is a threshold value given in input by the user in order to define which items are significant in percentage.  Thus only the items for which the following constraint $\omega S(I) \geq m\omega S$ is verified, are significant and can be used as candidates to generate frequent item-sets and rules.

Algorithm \ref{alg:mapping} is a summary of the main phases of the\textit{HPO-Miner} algorithm.
The first step of HPO-Miner algorithm is the loading of the input HPO dataset ($D$) and its transformation in \textit{weightedTable} $WT$ a data structure suitable to represent weighted transaction data (as reported in Algorithm \ref{alg:mapping} row 2). Concurrently to the loading and conversion phase, are evaluated the occurrences of each HPO term in $D$. Subsequently is possible to obtain a list of frequent weighted items (as stated in Algorithm \ref{alg:mapping} at row 3). We remove from the $\mathcal{FW}ItemsList$ the weighted items for which is not verified the following condition: $\omega S(I) \geq m\omega S$. Frequent weighted items are hence used to build a data structure based on  $FP-Tree$.  Finally, \textit{HPO-Miner} iteratively analyzes the $FP-Tree$ in order to mine and save significant rules. 
%The first step of \textit{HPO-Miner} algorithm is the transformation of input data in a format suitable to mine rules.
%A typical input file is formatted as tabular text file (let see Figure \ref{fig:w_HPO_dataset}, where in the first column there is a OMIM-identifier, in the second row there is the HPO term and in the last row there is the information content \textit{IC} related with this phenotype. Each disease can be annotated with more than a single \textit{HPO} terms as depicted in Figure \ref{fig:w_HPO_dataset}. The use of data merging allow \textit{HPO-Miner} to collect together all the HPO-terms used to annotate the same OMIM-identifier, producing a new data arrangement more suitable to represent \textit{weighted transactions} as conveyd in Figure \ref{fig:WD}. 
%Simultaneously, HPO-Miner computes the value of $m\omega S$ and count the occurrences of each HPO-term to obtain the value of $\omega S$. Weighted transaction are implemented by using a customized hash-table data structure called \textit{weightedTable} with the goal to reduce the memory necessary to represent the input data into the main memory and avoiding to waste time by accessing multiple time to the secondary memory to detect frequent items. Then all the items into the \textit{weightedTable} whose $\omega S$ is greater than the weighted minimum support ($m\omega S_\%$) are inserted into the \textit{frequent items list}. Frequent items are hence mapped onto a path into a data structure based on \textit{FPTree} (Frequent Pattern Tree). In order to build the \textit{FPTree} the algorithm considers each instance that meet the constraint $\omega S \geq m\omega S_\%$. Then it sorts items into the  \textit{weightedTable} in descending order of their $\omega S$. Different transactions that have common subsets allow the tree to remain compact because their paths overlap enabling an efficient processing. The recursive processing of \textit{FPTree} enables the building of large item sets without the generation of candidate items. The building of the \textit{FPTree}, is based on two main operation: \textit{$branch$-$creation$} and \textit{$\omega$S-Update}. When evaluating an item into the \textit{weightedTable}, it may be absent on the current path of the \textit{FPTree} or it may already be mapped. In the first case the function \textit{branch-creation} add to the current node of the \textit{FPTree} a new branch comprising all the elements of the current transaction. In the second case, the weighted support is updated by means of the function \textit{$\omega$S-Update}.
%Finally, \textit{HPO-Miner} iteratively analyze the \textit{FPTree} in order to mine significant rules using a recursive methodology of tree visiting. In particular we adopted an \textit{inverted DFS} scan method to examine the \textit{FPTree}. \textit{Inverted DFS} starts to explore the \textit{FPTree} from the leave nodes (bottom) and goes up to the (root node). The advantage to use \textit{inverted DFS} respect to traditional \textit{DFS} is related with the possibility to automatically prune (remove) the postfix part of a frequent pattern. All frequent patterns of a given item are mined following the links connecting all occurrences of the current item in the \textit{FPTree} and computing the weighted support related with each path (frequent patterns), producing a new tree called $\beta$Tree,  used to mine rules. Starting from the leaf nodes the prefix part of a path can be used to mine rules by gathering all the paths containing a particular suffix node. Any path that ends with this suffix is examined obtaining a $\beta$Tree. The $\beta$Tree is used to determine whether the suffix is frequent. This is done by evaluating the weighted support associated with the node and if the $\omega S$ is greater than or equal to the $m\omega S$ the node is frequent. If the node is not frequent the node is pruned and the analysis ends for this suffix. The process goes ahead until all items of the current $\beta$Tree are analyzed, and/or we reached the root \textit{prefix set} related with the current item is empty.
%Algorithm \ref{alg:mapping} shows the pseudo code of the core \textit{HPO-Miner} algorithm.
\begin{algorithm}
\caption{HPO Weighted Association Rules Miner (HPO-Miner)}\label{alg:mapping}
\label{alg:mapping}
\begin{algorithmic}[1]
   \small \REQUIRE A table of HPO annotation as input dataset $D$
%   \small  \ENSURE 
    \STATE \textbf{\textit{Data Structure initialization: $WT$, $\mathcal{FW}ItemsList$, \textit{FPTree}}}
    \medskip
    \STATE $WT \leftarrow $\textit{getTransactionalData($D$)}
    \medskip
%     \FORALL {$x \in WT$}
%     \STATE {$\omega S(x) \geq m\omega S$}
%   \STATE { $frequentItemsList \leftarrow x$ }
%    \ENDFOR
	\STATE $\mathcal{FW}ItemsList$ $\leftarrow$ retrieve$\mathcal{FW}ItemsList$($WT$)
   \medskip
%       \STATE {$descendingSorting()$}
%       \STATE {$descendingSorting(WT)$}
       \STATE{$FPTree.$create($\mathcal{FW}ItemsList$) }
        \medskip
%    \FORALL {$t \in WT$}
%%        \STATE{map each item of t on FPTree}
%        \IF{($perfectmatch(t))$} \STATE {$(\omega S$-$Update(t))$}
%        \ELSE \STATE {$branch$-$Creation(t)$}
%      \ENDIF
%       \ENDFOR %$(path \in FPTree)$\textbf{\&\&}
%    \STATE {buildFP-Tree()}
%        \medskip
%     \FORALL {$(node \in FPTree)$}
%         \STATE {create $\beta Tree$}
%         \REPEAT {}
%             \IF{$\beta node_{freq} < m\omega S$}\STATE{$remove(n)$}
%             \ENDIF
%         \UNTIL {$\beta Tree = \emptyset $}
%       \ENDFOR
       \medskip
       \STATE {$mineWeightedRules()$}
%       \STATE {$saveWeightedRules()$}
       \STATE \textbf{end.}

\end{algorithmic}
\end{algorithm}


%\subsection{Performance comparison}
%\label{performance}

%This Section presents the performance evaluation of HPO-Miner. The experiments have been done using a machine equipped with processor Intel i7 at 2.3 GHz, 16 GB of RAM, and 512 GB of SSD (solid state drive disk).
%The website of the project stores more detailed example of rules.



%We should point out that the generation of compatible input files for Knime and Weka is not possible required a lot of efforts to make files in a format compatible with the analysis. This step is expensive from a temporal point of view, and make the analysis error prone, especially if it is done by not expert users.
%
%Results obtained by HPO-Miner have been not compared with Arules package of Bioconductor \cite{arule} since, Arule did not find significant rules. It should be noted that the lack of results with Arules is due to the use of the classical implementation of A-priori algorithm. This causes the lack of significant rules.

\section{Results}
\label{sec:results}

%In this section we present and discuss the meaning of the most significant rules extracted by using \textit{HPO-Miner} that is: rules which have high value of weightedSupport and confidence among all those extracted. Data were downloaded from the 
\texttt{HPO} database is freely available online \footnote{\url{http://www.human-phenotype-ontology.org/downloads.html}}, the size of the dataset is about $4.4$ \texttt{MB} on disk. After collecting data, by using all the methods introduced in Section \ref{sec:methods}, we produced $5$ different datasets. %by value of IC obtained applying the methodologies explained in Section \ref{sec:methods}, on which we ran HPO-Miner. 
We tested HPO-Miner using several combinations of values for weightedSupport and confidence. Then we selected the values for the parameters able to ensure the best results in terms of reduced number of mined rules and in the same time with relevant values of weightedSupport and confidence. The best combination of values was weightedSupport equal to $50\%$ and confidence greater than $80\%$. We chose the first top $10$ rules from each dataset, and we manually analyzed the literature to find claims that can prove the validity of the mined rules. %frequent itemsets, requiring $2,582$ \textit{ms} to perform the task.
%The set of rules we mined should be subdivided into two groups: (i) binary rules (i.e. rules having only two items); (ii) n-ary rules (i.e. rules having more than two items). In the second group, we focus on a particular subset that we refer to as \textit{extended rules}.

\subsection{HPO-Miner rules extraction comparison}
The effectiveness of \textit{HPO-Miner} is proved comparing our tool with respect to other well known tools such as: Knime \cite{citeulike:5770121} and Weka \cite{weka}. We chose these tools because both provides an implementation of the FP-Growth algorithm a necessary condition in order to fairly compare HPO-Miner with both tools.  The FP-Growth algorithm implementation in Weka and Knime, is able to handle only binary attributes, making both tools unable to analyze weighted HPO datasets enriched with IC values. A possible way to make weighted HPO enriched dataset compatible with Weka and Knime is to leave for each OMIM entry only two HPO terms, making this solution infeasible because leads to lose a lot of useful information. 
Differently, HPO-Miner is the only tool that comes with a version of FP-Growth able to handle a generic number of attribute for each OMIM entry, making it suitable to analyze HPO dataset enriched with IC values.



\subsection{Analysis of Mined Rules}

%We here present a manual evaluation and discussion of the meaning of the top rules found by using HPO-Miner with their biological meaning and ranked by their weightedSupport and confidence.    

\begin{table}[ht]

\caption{The ten first rules found by HPO-Miner using the Dataset obtained by applying the Resnik measure and ranked by weightedSupport. (IDs are inserted for a better discussion in the following.) }
\label{tab:Resnik}
\begin{tabular}{|p{0.5cm}|l|l|p{0.5cm}|p{0.5cm}|p{2.0cm}|p{2.0cm}|}
\hline 
\textbf{}&\textbf{Term 1}& \textbf{Term 2} & \textbf{WS} & \textbf{C} &\textbf{Function} & \textbf{Function }\\ 
\hline
$1R$ & HP:0200084 & HP:0000007 &$1.00$&$1.00$&Giant cell hepatitis&Autosomal recessive inheritance\\
\hline
$2R$&HP:0200084&HP:0002910&$1.00$&$1.00$&Giant cell hepatitis&Elevated hepatic transaminases\\
\hline
$3R$&HP:0200067&HP:0000006&$1.00$&$1.00$&Recurrent spontaneous abortion&Autosomal dominant inheritance\\
\hline
$4R$&HP:0100818&HP:0000774&$1.00$&$1.00$&Long thorax&Narrow chest\\
\hline
$5R$&HP:0100775&HP:0001537&$1.00$&$1.00$&Dural ectasia&Umbilical hernia\\
\hline
$6R$&HP:0100775&HP:0000006&$1.00$&$1.00$&Dural ectasia&Autosomal dominant inheritance\\
\hline
$7R$&HP:0100775&HP:0000494&$1.00$&$1.00$&Dural ectasia&Downslanted palpebral fissures\\
\hline
$8R$&HP:0100775&HP:0000316&$1.00$&$1.00$&Dural ectasia&Hypertelorism\\
\hline
$9R$&HP:0100626&HP:0001394&$1.00$&$1.00$&Chronic hepatic failur&Cirrhosis\\
\hline
$10R$&HP:0100626&HP:0000007&$1.00$&$1.00$&Chronic hepatic failure&Autosomal recessive inheritance\\
\hline
\end{tabular}
\end{table}

\begin{table}[ht]

\caption{The ten first rules found by HPO-Miner using the Dataset obtained by applying the Sanchez measure and ranked by weightedSupport. (IDs are inserted for a better discussion in the following.) }
\label{tab:Sanchez}
\begin{tabular}{|p{0.5cm}|l|l|p{0.5cm}|p{0.5cm}|p{2.2cm}|p{2.2cm}|}
\hline 
\textbf{}&\textbf{Term 1}& \textbf{Term 2} & \textbf{WS} & \textbf{C} &\textbf{Function} & \textbf{Function }\\ 
\hline
$1S$&HP:0100818&HP:0000774&$0.88$&$1.00$& Long thorax&Narrow chest\\
\hline
$2S$&HP:0030034&HP:0003774&$0.88$&$1.00$&Diffuse glomerular basement membrane lamellation&Stage 5 chronic kidney disease\\
 \hline
$3S$&HP:0012743	&HP:0001773&$0.88$&$1.00$&Abdominal obesity&Short foot\\ 
 \hline
$4S$&HP:0012263&HP:0000007&$0.88$&$1.00$&Immotile cilia&Autosomal recessive inheritance\\
 \hline
$5S$&HP:0012023&HP:0000007&$0.88$&$1.00$&Galactosuria&Autosomal recessive inheritance\\ 
 \hline
$6S$&HP:0011727&HP:0009049&$0.88$&$1.00$&Peroneal muscle weakness&Peroneal muscle atrophy\\
 \hline
$7S$&HP:0010636&HP:0000316&$0.88$&$1.00$&Schizencephaly&Hypertelorism\\
 \hline
$8S$&HP:0009793&HP:0000316&$0.88$&$1.00$& Presacral teratoma&Hypertelorism\\
 \hline
$9S$&HP:0009760&HP:0006443&$0.88$&$1.00$&Antecubital pterygium&Patellar aplasia\\
 \hline
$10S$&HP:0008845&HP:0003067&$0.88$&$1.00$&Mesomelic short stature&Madelung deformity\\
\hline
\end{tabular}
\end{table}

\begin{table}[ht]

\caption{The ten first rules found by HPO-Miner using the Dataset obtained by applying the Harispe measure and ranked by weightedSupport. (IDs are inserted for a better discussion in the following.) }
\label{tab:Harispe}
\begin{tabular}{|p{0.5cm}|l|l|p{0.5cm}|p{0.5cm}|p{2.0cm}|p{2.0cm}|}
\hline 
\textbf{}&\textbf{Term 1}& \textbf{Term 2} & \textbf{WS} & \textbf{C} &\textbf{Function} & \textbf{Function }\\ 
\hline
$	1H	$&	HP:0009577	&	HP:0004220	&$	1.00	$&$	1.00$&	Short middle phalanx of the 2nd finger   &      Short middle phalanx of the 5th finger \\ 
\hline 
$	2H	$&	HP:0010105	&	HP:0010034	&$	1.00	$&$	1.00	$&	Short first metatarsal      &     Short 1st metacarpal\\ 
\hline
$	3H	$&	HP:0000933	&	HP:0001305	&$	1.00	$&$	1.00	$&	Posterior fossa cyst at the fourth ventricle	 &   Dandy-Walker malformation\\ 
\hline
$	4H	$&	HP:0004704	&	HP:0004689	&$	1.00	$&$	1.00	$&	Short fifth metatarsal	&      Short fourth metatarsal\\ 
\hline
$	5H	$&	HP:0001885	&	HP:0004209	&$	1.00	$&$	0.99	$&	Short 2nd toe   &   Clinodactyly of the 5th finger\\ 
\hline
$	6H	$&	HP:0003065	&	HP:0006443	&$	1.00	$&$	1.00	$&	Patellar hypoplasia  &     Patellar aplasia\\ 
\hline
$	7H	$&	HP:0009464	&	HP:0004209	&$	1.00	$&$	1.00	$&	 Ulnar deviation of the 2nd finger   &    Clinodactyly of the 5th finger\\ 
\hline
$	8H	$&	HP:0002834	&	HP:0002857	&$	1.00	$&$	1.00	$&	Flared femoral metaphysis     &    Genu valgum\\ 
\hline
$	9H	$&	HP:0004209	&	HP:0000272	&$	1.00	$&$	0.99	$&	Clinodactyly of the 5th finger &    Malar flattening\\ 
\hline
$	10H	$&	HP:0001773	&	HP:0004279	&$	1.00	$&$	1.00	$&	  Short foot   &  Short palm\\ 
\hline
\end{tabular}
\end{table}

\begin{table}[ht]

\caption{The ten first rules found by HPO-Miner using the Dataset obtained by applying the Seco measure and ranked by weightedSupport. (IDs are inserted for a better discussion in the following.) }
\label{tab:Seco}
\begin{tabular}{|p{0.55cm}|l|l|p{0.5cm}|p{0.5cm}|p{2.0cm}|p{2.0cm}|}
\hline 
\textbf{}&\textbf{Term 1}& \textbf{Term 2} & \textbf{WS} & \textbf{C} &\textbf{Function} & \textbf{Function }\\ 
\hline
$	1Se	$&	HP:0200084	&	HP:0000007	&$	1	.00$&$	1	.00$&	Giant cell hepatitis	&	Autosomal recessevie inheritance\\ 
\hline
$	2Se	$&	HP:0100818	&	HP:0000774	&$	1	.00$&$	1	.00$&	Long thorax	&	Narrow chest\\ 
\hline
$	3Se	$&	HP:0100775	&	HP:0001537	&$	1	.00$&$	1	.00$&	Dural ectasia	&	humbilical hernia\\ 
\hline
$	4Se	$&	HP:0100775	&	HP:0000494	&$	1	.00$&$	1	.00$&	Dural ectasia	&	Downslanted palpebral fissures\\ 
\hline
$	5Se	$&	HP:0100775	&	HP:0000316	&$	1	.00$&$	1	.00$&	Dural ectasia	&	Hypertelorism\\ 
\hline
$	6Se	$&	HP:0100626	&	HP:0000007	&$	1	.00$&$	1	.00$&	Chronic hepatic failure	&	Autosomal recessevie inheritance\\ 
\hline
$	7Se	$&	HP:0030050	&	HP:0002524	&$	1	.00$&$	1	.00$&	Narcolepsy	&	Cataplexy\\ 
\hline
$	8Se	$&	HP:0012240	&	HP:0000007	&$	1	.00$&$	1	.00$&	Increased intramyocellular lipid droplets	&	Autosomal recessevie inheritance\\ 
\hline
$	9Se	$&	HP:0010780	&	HP:0007018	&$	1	.00$&$	1	.00$&	Hyperacusis	&	Attention deficit hyperactivity disorder (ADHD)\\ 
\hline
$	10Se	$&	HP:0010780	&	HP:0000179	&$	1	.00$&$	1	.00$&	Short 3rd metacarpal	&	Hypertelorism\\ 
\hline
\end{tabular}
\end{table}


\begin{table}[!h]

\caption{The ten first rules found by HPO-Miner using the Dataset obtained by applying the Zhou measure and ranked by weightedSupport. (IDs are inserted for a better discussion in the following.) }
\label{tab:Zhou}
\begin{tabular}{|p{0.55cm}|l|l|p{0.5cm}|p{0.5cm}|p{2.0cm}|p{2.0cm}|}
\hline 
\textbf{}&\textbf{Term 1}& \textbf{Term 2} & \textbf{WS} & \textbf{C} &\textbf{Function} & \textbf{Function }\\ 
\hline
$1Z$&	HP:0002335	&	HP:0001305	&$	0.97	$&$	1	$&	Congenital absence of the vermis of cerebellum	&	Dandy Walker malformation\\ 
\hline
$2Z$&	HP:0003031	&	HP:0002986	&$	0.95	$&$	1	$&	Bending of the diaphysis (shaft) of the ulna (Uknar bowing)	&	A bending or abnormal curvature of the radius (Radial bowing)\\ 
\hline
$3Z$&	HP:0000176	&	HP:0000193	&$	0.95	$&$	0.97	$&	submucous clefts Hard-palate	&	Bifid uvula\\ 
\hline
$4Z$&	HP:0001338	&	HP:0002007	&$	0.95	$&$	0.94	$&	Partial agenesis of the corpus callosum	&	Frontal Bossing\\ 
\hline
$5Z$&	HP:0001338	&	HP:0000494	&$	0.95	$&$	0.94	$&	Partial agenesis of the corpus callosum	&	Downslated palpebral fissures\\ 
\hline
$6Z$&	HP:0000308	&	HP:0001305	&$	0.95	$&$	1	$&	Microre~trognathia	&	Dandy Walker malformation\\ 
\hline
$7Z$&	HP:0000269	&	HP:0001305	&$	0.95	$&$	1	$&	Promiment occiput	&	Dandy Walker malformation\\ 
\hline
$8Z$&	HP:0010804	&	HP:0001305	&$	0.95	$&$	1	$&	Tented upper lip vermilion	&	Dandy Walker malformation\\ 
\hline
$9Z$&	HP:0009623	&	HP:0001305	&$	0.95	$&$	1	$&	Proximal placement of the thumb	&	Dandy Walker malformation\\ 
\hline
$10Z$&	HP:0000567	&	HP:0001305	&$	0.95	$&$	1	$&	Chorioretinal coloboma	&	Dandy Walker malformation\\ 
\hline
\end{tabular}
\end{table}

%We begin to discuss the rules contained in Table \ref{tab:Resnik} mined by HPO-Miner from the Resnik dataset.

Let us consider rule (1R): (HP:0200084, HP:0000007) - \textit{Giant cell hepatitis, Autosomal recessive inheritance}. Searching the literature we found some evidences that describe the relationship between this two terms. As stated in \cite{Danks01051977} both terms could be related with defects in the biological mechanisms of the liver. In particular,
\textit{Autosomal recessive inheritance} suggests a biochemical defect that might cause a metabolic disorder in the liver while, \textit{Giant cell hepatitis} is responsible of "thick bile syndrome" in neonatal. Consequently, HPO-Miner was able to found a relation between two apparently unrelated terms into the graph of HPO classes.

Rule (2R) (HP:0200084, HP:0002910) i.e.,  (\textit{Giant cell hepatitis}, \textit{Elevated hepatic transaminases}) consists of two terms involved in the hepatitis process. Analyzing in depth the literature it revealed   the following links between the two terms. In \cite{APA:APA285} is presented a study on three siblings with neonatal jaundice who died before the age of three months. They were shown on autopsy to be suffering from Niemann-Pick disease together with a giant cell transformation of the liver. Clayton et. al. in \cite{ClaytonPaper} including the infant studied in \cite{Kase1985} were able to inferrer, that due to the elevated transaminases most patients develop hepatic fibrosis or cirrhosis due to the presence of \textit{Giant cell hepatitis}. Thus, manually analyzing this rule has been possible to infer that both terms are responsible of the liver disorder in infants and adults.

Rule (3R) involves the following two HPO terms (HP:0200067, HP:0000006) i.e.,\textit{Recurrent spontaneous abortion} and \textit{Autosomal dominant inheritance}. There is a growing literature on the importance of  Autosomal dominant inheritance in pregnancy complications as reported in \cite{byrne1994genetic}. As stated in \cite{Kutteh2006} Thrombophilia is a cause of maternal mortality due to certain inherited thrombophilic factors that activated protein C resistance. In \cite{coumans1999haemostatic} the authors point out the rare familial disorders that are usually inherited as \textit{Autosomal dominant inheritance}. 

Rule (4R) (HP:0100818, HP:0000774) composed by the following phenotypic abnormalities \textit{Long thorax}, \textit{Narrow chest} involved in the syndrome of Jeune and Ellis-Van Creveld syndrome as reported in literature in \cite{AJMG:AJMG1320210304, baujat2007ellis}. Browsing HPO Ontology with its on line browser did not reveal any information that allows the user to associate both abnormalities with the syndrome of Juene and Ellis-Van Creveld. This may suggest to the curator to restructure ontology in order to make easily available this knowledge in order to clarify these associations.

Rule (5R) (HP:0100775, HP:0001537) whose translation is \textit{Dural ectasia}, \textit{Umbilical hernia} at first glance seems that there not exists a connection among the two terms. Analyzing the literature we found the work of Mizuguchi et.al. \cite{mizuguchi2004heterozygous} and Chen et. al., \cite{chen2005lateral}. In Mizuguchi et.al. have been found both abnormalities in a patient affected by the Marfan syndrome in infancy, instead Chen et. al. have found  these abnormalities in patients affected by Lateral meningocele syndrome. These knowledge it is not readily available for the users by using HPO, consequently this may suggest to the curator to add this further knowledge into the HPO. 

Rule (6R) (HP:0012023, HP:0000007) define an association between the   \textit{Galactosuria} and \textit{Autosomal recessive inheritance}. Analyzing the literature looking for evidence on the validity of the association we found the works of Pickering et. al., \cite{pickering1972galactokinase} and Monteleone et. al. \cite{monteleone1971cataracts}, in which in both works, the authors stated that hereditary galactokinase deficiency is characterized by galactosuria. In particular, in this study support the autosomal recessive inheritance of this disorder. This evidence support the validity of the current association found it by using HPO-Miner.

To verify the reliability of Rule (7R) (HP:0100775,HP:0000494) i.e. (\textit{Dural ectasia}, \textit{Downslanted palpebral fissures}) and Rule (8R) (HP:0100775, HP:0000316) i.e., \textit{Dural ectasia}, \textit{Hypertelorism}, we analyzed the literature founding that the terms of both rules are symptoms involved in the Marfan syndrome as stated in \cite{lemaire2007severe, loeys2010revised}. Consequently these association rules may suggest to the curator to add new informative links among HPO terms, making easier for the users to obtain further knowledge.


Rule (9R) (HP:0100626, HP:0001394) refers to \textit{Chronic hepatic failure} and \textit{Cirrhosis}. Analyzing the literature showed that both terms are involved in fat elimination as stated in the work of Druml et. al. \cite{druml1995fat}. This evidence may be suggest to the curator to make this explicit knowledge in implicit, by adding new links among the HPO terms.

 (10R) (HP:0100626, HP:0000007) \textit{Chronic hepatic failure}, \textit{Autosomal recessive inheritance}

Here we discuss the rules contained in Table \ref{tab:Sanchez} that refer to the rules mined by HPO-Miner from the Sanchez dataset.

Rule (1S) (HP:010081, HP:0000774) i.e.,  (\textit{Long thorax}, \textit{Narrow chest}) consists in two terms involved in the Asphyxiating Thoracic Dysplasia (Jeune Syndrome).Jeune syndrome is a congenital disorder with abnormalities of which thoracic hypoplasia is the most prominent. The literature confirms that both phenotype, long thorax and narrow chest are manifestations of Jeune syndrome. In \cite{elejalde1985prenatal} is reported this evidence.

Rule (2S) (HP:0030034, HP:0003774) associates  with\textit{Diffuse glomerular basement membrane lamellation}, \textit{Stage 5 chronic kidney disease}. Searching in the current literature the glomerular basement membrane lamellation is a manifestation in patients after transplantation of kidneys from pediatric cadaveric donors, as \cite{nadasdy1999diffuse} reported. There is not evidence that this phenotype is related to the Stage 5 chronic kidney disease.

About the Rule (3S) (HP:0012743, HP:0001773), \textit{Abdominal obesity}, \textit{Short foot} we didn't find a correlation among \textit{Abdominal obesity}(term 1) and \textit{Short foot} (term 2) despite a depth research in literature was conducted .

Rule (4S) (HP:0012263, HP:0000007) and Rule (5S) (HP:0012023, HP:0000007), associate two pathologic phenotypes, \textit{Immotile cilia} and (\textit{Galactosuria} to \textit{Autosomal recessive inheritance}). In fact, in \cite{afzelius1985immotile} is reported that the immotile cilia syndrome seems to be that of an autosomal recessive disease; as well as   galactosuria due to galactokinase deficiency in a newborn is inherited in an autosomal recessive manner \cite{pickering1972galactokinase}.

HPO-MINER finds the Rule (6S) (HP:0011727, HP:0009049) that associates(\textit{Peroneal muscle weakness} with \textit{Peroneal muscle atrophy}).In fact the peroneal muscle atrophy is characterized
by wasting and flaccid weakness of the intrinsic muscles of the feet and of the muscles innervated by the peroneal nerve \cite{buchthal1977peroneal}.

Rule (7S) (HP:0010636, HP:0000316) relates (\textit{Schizencephaly}, \textit{Hypertelorism}) involved in the same disease, the LEOPARD syndrome. A case study \cite{liang2009schizencephaly} reported patient affect by this disease with open-lip schizencephaly and Ocular hypertelorism pathologic phenotype.

Instead Rule (8S) (HP:0009793, HP:0000316), highlights a link among \textit{Hypertelorism}) with (\textit{Presacral teratoma} in the Schinzel‐Giedion syndrome as reported in \cite{robin1993new}.


In \cite{reichenbach1995hereditary} is discussed a Hereditary Congenital Posterior Dislocation
of Radial Heads which disorder is characterized by The association of nailpatella syndrome
with typical antecubital pterygium as HPO-MINER found in Rule (9S) (HP:0009760, HP:0006443), 

Rule (10S) (HP:0008845, HP:0003067), composed by (\textit{Mesomelic short stature}, \textit{Madelung deformity}). Both phenotype are involved in Madelung deformity of childhood \cite{flanagan2002prevalence}





Here we analyze the rules contained in Table \ref{tab:Seco}.

Rule (1Se) (HP:0200084, HP:0000007) associates \textit{Giant cell hepatitis}and  \textit{Autosomal recessive inheritance}. This evidence is highlighted in a case study reported a patient suffered from a unique form of giant cell hepatitis which condition appears to be an autosomal recessive one\cite{clayton1987familial}

Rule (2Se) (HP:0100818, HP:0000774) \textit{Long thorax}, \textit{Narrow chest} is discussed above.

About the Rule (3Se) (HP:0100775, HP:0001537) i.e. \textit{Dural ectasia}, \textit{humbilical hernia} HPO-MINER find a association that is not confirmed in literature. 

The Rule (4Se) (HP:0100775, HP:0000494) and the Rule (5Se) (HP:0100775, HP:0000316) associate the phenotype\textit{Dural ectasia} with\textit{Downslanted palpebral fissures} and \textit{Hypertelorism}. Carrying out a analysis in the state of art, we found a clinical case which report a patient with lateral meningocele syndrome (LMS) affected by both down slanting palpebral fissures and hyperteloris \cite{chen2005lateral}.

About the Rule (6Se) (HP:0100626, HP:0000007) associates \textit{Chronic hepatic failure} to characteristic \textit{Autosomal recessive inheritance}
\cite{blumberg1969hepatitis}

In the Rule (7Se) (HP:0030050, HP:0002524) are connected two pathologic phenotype \textit{Narcolepsy} and \textit{Cataplexy} that are known as a sleep disorder associated with a centrally mediated hypocretin deficiency\cite{mignot2001complex}.

About Rule (8Se) (HP:0012240, HP:0000007) the evidence that the \textit{Increased intramyocellular lipid droplets} is \textit{Autosomal recessive inheritance}.

The Rule (9Se) (HP:0010780, HP:0007018) associates the symptom  \textit{Hyperacusis} to \textit{Attention deficit hyperactivity disorder (ADHD)} as reported in\cite{einfeld1995issues}

Instead the Rule (10Se) (HP:0000179, HP:0010780) i.e\textit{Short 3rd metacarpal}, \textit{Hyperacusis}  has not evidence in literature. 

Here we interpret the rule mined by HPO-Miner from the dataset Harispe and contained in Table \ref{tab:Harispe}.

Rule (1H) is composed of terms (HP:0009577, HP:0004220)	i.e., (Short middle phalanx of the 2nd finger, Short middle phalanx of the 5th finger). Analyzing the literature, we found that this abnormalities have been observed in the Adams-Oliver Syndrome as reported in the work of Kuster et.al. \cite{kuster1988congenital}.

Rule (2H) contains the terms	(HP:0010105, HP:0010034) i.e., \textit{Short first metatarsal, Short 1st metacarpal} 


Rule (3H) (HP:0000933,HP:0001305) i.e., \textit{Posterior fossa cyst at the fourth ventricle} \textit{Dandy-Walker malformation} involved in abnormality that affects brain development.

Analyzing the literature has not been possible found any evidence on the involvement of the (HP:0004704, HP:0004689)	i.e., \textit{Short fifth metatarsal, Short fourth metatarsalRule}, contained in the rule (4H) found by HPO-Miner.


Rule (5H) is formed by the two terms (HP:0001885, HP:0004209)	i.e., \textit{Short 2nd toe, Clinodactyly of the 5th finger}. Searching into the literature we found that both symptoms occurred in  Carpenter Syndromeas states in the work of Gershoni et.al., \cite{gershoni1990carpenter}.

Rule (6H) involves the following two HPO terms (HP:0003065, HP:0006443) 	i.e., \textit{Patellar hypoplasia,  Patellar aplasia}. The work of Kaariainen et. al. \cite{kaariainen1989rapadilino} that RAPADILINO syndrome involve both symptoms.

Rule (7H) (HP:0009464, HP:0004209) i.e., \textit{Ulnar deviation of the 2nd finger, Clinodactyly of the 5th finger} consists of two terms involved in the KBG syndrome as reported in the work of Sirmaci et. al. \cite{sirmaci2011mutations}.

Rule (8H) is composed of (HP:0002834, HP:0002857)	i.e., \textit{Flared femoral metaphysis, Genu valgum}. Both symptom are observed in the metatropic dwarfism as described into the work of LaRose et. al. \cite{larose1969metatropic}.

The terms contained into the rule (9H) HP:0004209, HP:0000272 i.e.,	\textit{Clinodactyly of the 5th finger, Malar flattening} are involved in \textit{49,XXXXY} syndrome as stated in the work of Peet et. al. \cite{peet199849}.

About Rules (10H) (HP:0001773, HP:0004279) i.e. \textit{Short foot, Short palm} we didn't find any correlation between the terms, despite a depth research in literature it was conducted.

\pagebreak
Here we analyze the rules contained in Table \ref{tab:Zhou}.

The first rule Rule (1Z) (HP:0002335, HP:0001305) associates the \textit{Congenital absence of the vermis of cerebellum} with \textit{Dandy Walker malformation}. This evidence is confirmed in \cite{bordarier1990dandy}, that reported a cases of Dandy-Walker malformation including agenesis cerebellar vermis.



HPO-MINER extracts the Rule (2Z) (HP:0003031, HP:0002986) i.e\textit{Bending of the diaphysis (shaft) of the ulna (Ulnar bowing)}  \textit{A bending or abnormal curvature of the radius (Radial bowing)} and the Rule (3Z) (HP:0000176, HP:0000193), i.e.\textit{submucous clefts Hard-palate	} \textit{Bifid uvula}. Although we conducted a deep analysis of stare of art, these rules are not confirmed in literature.

The Rule (4Z) (HP0001338 HP0002007) and the Rule (5Z) (HP0001338 HP0000494) associate the \textit{Partial agenesis of the corpus callosum} to two abnormal phenotype: \textit{Frontal Bossing} and \textit{Downslated palpebral fissures} as confirmed in \cite{taylor1968nevoid} and\cite{gelman1991further}.



HPO-MINER finds the Rule (6Z) (HP:0000308, HP:0001305), Rule (8Z) (HP:0010804, HP:0001305), Rule (9Z) (HP:0009623, HP:0001305) that associate the phenotypes\textit{Microretrognathia} \textit{Tented upper lip vermilion} in, \textit{Proximal placement of the thumb}  to\textit{Dandy Walker malformation}. Unfortunately we didn't find this evidences in literature.

About Rule (7Z) (HP:0000269, HP:0001305) and the Rule (10Z) (HP:0000567, HP:0001305) \textit{Prominent occiput} (HP:0010636 term) and the  \textit{Chorioretinal colo- ~ boma} (HP:0000567) are the abnormalities related to the \textit{Dandy Walker malformation} as reported in \cite{archer1978enlarged} and \cite{dobyns1989diagnostic}.






\section{Conclusion}
\label{sec:conclusion}
We presented a new methodology based on weighted association rule for HPO data analysis that takes into account the relevance of terms;  the relevance is a weight assigned to a term based on, for example, its specificity to describe a phenotypic abnormality. The relevance of a HPO term, is obtained by computing the IC value related with each term. We presented the outline of an algorithm called HPO-Miner to mine weighted itemsets that have sufficient weighted supports. These itemsets are used in turn to generate association rules that have high weighted support. Finally, the relevance of the mined rules by HPO-Miner, is proved by the evidences found analyzing the literature.

%Future works will regard testing of HPO-Miner on larger datasets for improving annotation consistency.

\newpage

\flushbottom
%\bibliographystyle{plain}
%\bibliography{biblio}
\begin{thebibliography}{10}
\expandafter\ifx\csname url\endcsname\relax
  \def\url#1{\texttt{#1}}\fi
\expandafter\ifx\csname urlprefix\endcsname\relax\def\urlprefix{URL }\fi
\expandafter\ifx\csname href\endcsname\relax
  \def\href#1#2{#2} \def\path#1{#1}\fi

\bibitem{gruber2009ontology}
T.~Gruber, Ontology, Encyclopedia of database systems (2009) 1963--1965.

\bibitem{gene2004gene}
G.~O. Consortium, et~al., The gene ontology (go) database and informatics
  resource, Nucleic acids research 32~(suppl 1) (2004) D258--D261.

\bibitem{hamosh2005online}
A.~Hamosh, A.~F. Scott, J.~S. Amberger, C.~A. Bocchini, V.~A. McKusick, Online
  mendelian inheritance in man (omim), a knowledgebase of human genes and
  genetic disorders, Nucleic acids research 33~(suppl 1) (2005) D514--D517.

\bibitem{schriml2012disease}
L.~M. Schriml, C.~Arze, S.~Nadendla, Y.-W.~W. Chang, M.~Mazaitis, V.~Felix,
  G.~Feng, W.~A. Kibbe, Disease ontology: a backbone for disease semantic
  integration, Nucleic acids research 40~(D1) (2012) D940--D946.

\bibitem{flouris2006inconsistencies}
G.~Flouris, Z.~Huang, J.~Z. Pan, D.~Plexousakis, H.~Wache, Inconsistencies,
  negations and changes in ontologies, in: Proceedings of the National
  Conference on Artificial Intelligence, Vol.~21, Menlo Park, CA; Cambridge,
  MA; London; AAAI Press; MIT Press; 1999, 2006, p. 1295.

\bibitem{yeh2003knowledge}
I.~Yeh, P.~D. Karp, N.~F. Noy, R.~B. Altman, Knowledge acquisition, consistency
  checking and concurrency control for gene ontology (go), Bioinformatics
  19~(2) (2003) 241--248.

\bibitem{10.1371/journal.pone.0040519}
D.~Faria, A.~Schlicker, C.~Pesquita, H.~Bastos, A.~E.~N. Ferreira, M.~Albrecht,
  A.~O. Falc√£o,
  \href{http://dx.doi.org/10.1371%2Fjournal.pone.0040519}{Mining go annotations
  for improving annotation consistency}, PLoS ONE 7~(7) (2012) e40519.
\newblock \href {http://dx.doi.org/10.1371/journal.pone.0040519}
  {\path{doi:10.1371/journal.pone.0040519}}.
\newline\urlprefix\url{http://dx.doi.org/10.1371%2Fjournal.pone.0040519}

\bibitem{manda2012cross}
P.~Manda, S.~Ozkan, H.~Wang, F.~McCarthy, S.~M. Bridges, Cross-ontology
  multi-level association rule mining in the gene ontology, PloS one 7~(10)
  (2012) e47411.

\bibitem{faria2012}
D.~Faria, A.~Schlicker, C.~Pesquita, H.~Bastos, A.~E.~N. Ferreira, M.~Albrecht,
  A.~O. Falcão, Mining go annotations for improving annotation consistency,
  PLoS ONE 7~(7) (2012) e40519.
\newblock \href {http://dx.doi.org/10.1371/journal.pone.0040519}
  {\path{doi:10.1371/journal.pone.0040519}}.

\bibitem{manda2013interestingness}
P.~Manda, F.~McCarthy, S.~M. Bridges, Interestingness measures and strategies
  for mining multi-ontology multi-level association rules from gene ontology
  annotations for the discovery of new go relationships, Journal of biomedical
  informatics 46~(5) (2013) 849--856.

\bibitem{agapito2014improving}
G.~Agapito, M.~Milano, P.~H. Guzzi, M.~Cannataro, Improving annotation quality
  in gene ontology by mining cross-ontology weighted association rules, in:
  Bioinformatics and Biomedicine (BIBM), 2014 IEEE International Conference on,
  IEEE, 2014, pp. 1--8.

\bibitem{agapito2015using}
G.~Agapito, M.~Cannataro, P.~H. Guzzi, M.~Milano, Using go-war for mining
  cross-ontology weighted association rules, Computer methods and programs in
  biomedicine 120~(2) (2015) 113--122.

\bibitem{harispe2013framework}
S.~Harispe, D.~S{\'a}nchez, S.~Ranwez, S.~Janaqi, J.~Montmain, A framework for
  unifying ontology-based semantic similarity measures: A study in the
  biomedical domain, Journal of biomedical informatics.

\bibitem{citeulike:1005421}
R.~Agrawal, T.~Imieli\&\#324;ski, A.~Swami,
  \href{http://dx.doi.org/10.1145/170036.170072}{Mining association rules
  between sets of items in large databases}, SIGMOD Rec. 22~(2) (1993)
  207--216.
\newblock \href {http://dx.doi.org/10.1145/170036.170072}
  {\path{doi:10.1145/170036.170072}}.
\newline\urlprefix\url{http://dx.doi.org/10.1145/170036.170072}

\bibitem{Wang:2000:EMW:347090.347149}
W.~Wang, J.~Yang, P.~S. Yu,
  \href{http://doi.acm.org/10.1145/347090.347149}{Efficient mining of weighted
  association rules (war)}, in: Proceedings of the Sixth ACM SIGKDD
  International Conference on Knowledge Discovery and Data Mining, KDD '00,
  ACM, New York, NY, USA, 2000, pp. 270--274.
\newblock \href {http://dx.doi.org/10.1145/347090.347149}
  {\path{doi:10.1145/347090.347149}}.
\newline\urlprefix\url{http://doi.acm.org/10.1145/347090.347149}

\bibitem{694360}
C.~Cai, A.~Fu, C.~Cheng, W.~Kwong, Mining association rules with weighted
  items, in: Database Engineering and Applications Symposium, 1998.
  Proceedings. IDEAS'98. International, 1998, pp. 68--77.
\newblock \href {http://dx.doi.org/10.1109/IDEAS.1998.694360}
  {\path{doi:10.1109/IDEAS.1998.694360}}.

\bibitem{sanchez2011ontology}
D.~S{\'a}nchez, M.~Batet, D.~Isern, Ontology-based information content
  computation, Knowledge-Based Systems 24~(2) (2011) 297--303.

\bibitem{resnink:simmeasure:879855}
P.~Resnik,
  \href{http://citeseerx.ist.psu.edu/viewdoc/summary?doi=10.1.1.55.5277}{Using
  information content to evaluate semantic similarity in a taxonomy}, in:
  IJCAI, 1995, pp. 448--453.
\newline\urlprefix\url{http://citeseerx.ist.psu.edu/viewdoc/summary?doi=10.1.1.55.5277}

\bibitem{14755292}
H.~Hermjakob, L.~Montecchi-Palazzi, G.~Bader, J.~Wojcik, L.~Salwinski, A.~Ceol,
  S.~Moore, S.~Orchard, U.~Sarkans, C.~von Mering, The hupo psi's molecular
  interaction format - a community standard for the representation of protein
  interaction data, Nat Biotechnol 22 (2004) 177--183.
\newblock \href {http://dx.doi.org/10.1038/nbt926} {\path{doi:10.1038/nbt926}}.

\bibitem{zhou2008new}
Z.~Zhou, Y.~Wang, J.~Gu, A new model of information content for semantic
  similarity in wordnet, in: Future Generation Communication and Networking
  Symposia, 2008. FGCNS'08. Second International Conference on, Vol.~3, IEEE,
  2008, pp. 85--89.

\bibitem{citeulike:5770121}
M.~R. Berthold, N.~Cebron, F.~Dill, T.~R. Gabriel, T.~K\"{o}tter, T.~Meinl,
  P.~Ohl, C.~Sieb, K.~Thiel, B.~Wiswedel,
  \href{http://dx.doi.org/10.1007/978-3-540-78246-9\_38}{{KNIME: The Konstanz
  Information Miner}}, in: C.~Preisach, H.~Burkhardt, L.~Schmidt-Thieme,
  R.~Decker (Eds.), Data Analysis, Machine Learning and Applications, Springer
  Berlin Heidelberg, Berlin, Heidelberg, 2008, Ch.~38, pp. 319--326.
\newblock \href {http://dx.doi.org/10.1007/978-3-540-78246-9\_38}
  {\path{doi:10.1007/978-3-540-78246-9\_38}}.
\newline\urlprefix\url{http://dx.doi.org/10.1007/978-3-540-78246-9\_38}

\bibitem{weka}
M.~Hall, E.~Frank, G.~Holmes, B.~Pfahringer, P.~Reutemann, I.~Witten,
  \href{http://dx.doi.org/10.1145/1656274.1656278}{{The WEKA data mining
  software: an update}}, Special Interest Group on Knowledge Discovery and Data
  Mining Explorer Newsletter 11~(1) (2009) 10--18.
\newblock \href {http://dx.doi.org/10.1145/1656274.1656278}
  {\path{doi:10.1145/1656274.1656278}}.
\newline\urlprefix\url{http://dx.doi.org/10.1145/1656274.1656278}

\bibitem{Danks01051977}
D.~M. Danks, P.~E. Campbell, I.~Jack, J.~Rogers, A.~L. Smith,
  \href{http://adc.bmj.com/content/52/5/360.abstract}{Studies of the aetiology
  of neonatal hepatitis and biliary atresia.}, Archives of Disease in Childhood
  52~(5) (1977) 360--367.
\newblock \href
  {http://arxiv.org/abs/http://adc.bmj.com/content/52/5/360.full.pdf+html}
  {\path{arXiv:http://adc.bmj.com/content/52/5/360.full.pdf+html}}, \href
  {http://dx.doi.org/10.1136/adc.52.5.360} {\path{doi:10.1136/adc.52.5.360}}.
\newline\urlprefix\url{http://adc.bmj.com/content/52/5/360.abstract}

\bibitem{APA:APA285}
A.~ASHKENAZI, R.~YAROM, A.~GUTMAN, A.~ABRAHAMOV, A.~RUSSELL,
  \href{http://dx.doi.org/10.1111/j.1651-2227.1971.tb06658.x}{Niemann-pick
  disease and giant cell transformation of the liver}, Acta PÊdiatrica 60~(3)
  (1971) 285--294.
\newblock \href {http://dx.doi.org/10.1111/j.1651-2227.1971.tb06658.x}
  {\path{doi:10.1111/j.1651-2227.1971.tb06658.x}}.
\newline\urlprefix\url{http://dx.doi.org/10.1111/j.1651-2227.1971.tb06658.x}

\bibitem{ClaytonPaper}
P.~T. Clayton, M.~Casteels, G.~Mieli-Vergani, A.~M. Lawson,
  \href{http://dx.doi.org/10.1203/00006450-199504000-00007}{Familial giant cell
  hepatitis with low bile acid concentrations and increased urinary excretion
  of specific bile alcohols: A new inborn error of bile acid synthesis?},
  Pediatr Res 37~(4) (1995) 424--431.
\newline\urlprefix\url{http://dx.doi.org/10.1203/00006450-199504000-00007}

\bibitem{Kase1985}
W.~J. Byrne, B.~F. Kase, I.~Bjorkhem, P.~Haga, J.~I. Pedersen,
  \href{http://journals.lww.com/jpgn/Fulltext/1985/08000/DEFECTIVE_PEROXISOMAL_CLEAVAGE_OF_THE_C27_STEROID.40.aspx}{Defective
  peroxisomal cleavage of the c27 steroid side chain in the cerebro.}, Journal
  of Pediatric Gastroenterology and Nutrition 4~(4).
\newline\urlprefix\url{http://journals.lww.com/jpgn/Fulltext/1985/08000/DEFECTIVE_PEROXISOMAL_CLEAVAGE_OF_THE_C27_STEROID.40.aspx}

\bibitem{byrne1994genetic}
J.~L. Byrne, K.~Ward, Genetic factors in recurrent abortion., Clinical
  obstetrics and gynecology 37~(3) (1994) 693--704.

\bibitem{Kutteh2006}
W.~H. Kutteh, D.~A. Triplett, Thrombophilias and recurrent pregnancy loss,
  Semin Reprod Med 24~(01) (2006) 054--066.
\newblock \href {http://dx.doi.org/10.1055/s-2006-931801}
  {\path{doi:10.1055/s-2006-931801}}.

\bibitem{coumans1999haemostatic}
A.~Coumans, P.~Huijgens, C.~Jakobs, R.~Schats, J.~De~Vries, M.~Van~Pampus,
  G.~Dekker, Haemostatic and metabolic abnormalities in women with unexplained
  recurrent abortion, Human Reproduction 14~(1) (1999) 211--214.

\bibitem{AJMG:AJMG1320210304}
B.~R. Elejalde, M.~M. De~Elejalde, D.~Pansch, J.~M. Opitz, J.~F. Reynolds,
  \href{http://dx.doi.org/10.1002/ajmg.1320210304}{Prenatal diagnosis of jeune
  syndrome}, American Journal of Medical Genetics 21~(3) (1985) 433--438.
\newblock \href {http://dx.doi.org/10.1002/ajmg.1320210304}
  {\path{doi:10.1002/ajmg.1320210304}}.
\newline\urlprefix\url{http://dx.doi.org/10.1002/ajmg.1320210304}

\bibitem{baujat2007ellis}
G.~Baujat, M.~Le~Merrer, Ellis-van creveld syndrome, Orphanet J Rare Dis 2~(6)
  (2007) 27.

\bibitem{mizuguchi2004heterozygous}
T.~Mizuguchi, G.~Collod-Beroud, T.~Akiyama, M.~Abifadel, N.~Harada,
  T.~Morisaki, D.~Allard, M.~Varret, M.~Claustres, H.~Morisaki, et~al.,
  Heterozygous tgfbr2 mutations in marfan syndrome, Nature genetics 36~(8)
  (2004) 855--860.

\bibitem{chen2005lateral}
K.~M. Chen, L.~Bird, P.~Barnes, R.~Barth, L.~Hudgins, Lateral meningocele
  syndrome: vertical transmission and expansion of the phenotype, American
  Journal of Medical Genetics Part A 133~(2) (2005) 115--121.

\bibitem{pickering1972galactokinase}
W.~R. Pickering, R.~R. Howell, Galactokinase deficiency: clinical and
  biochemical findings in a new kindred, The Journal of pediatrics 81~(1)
  (1972) 50--55.

\bibitem{monteleone1971cataracts}
J.~A. Monteleone, E.~Beutler, P.~L. Monteleone, C.~L. Utz, E.~C. Casey,
  Cataracts, galactosuria and hypergalactosemia due to galactokinase deficiency
  in a child: studies of a kindred, The American journal of medicine 50~(3)
  (1971) 403--407.

\bibitem{lemaire2007severe}
S.~A. LeMaire, H.~Pannu, V.~Tran-Fadulu, S.~A. Carter, J.~S. Coselli, D.~M.
  Milewicz, Severe aortic and arterial aneurysms associated with a tgfbr2
  mutation, Nature Clinical Practice Cardiovascular Medicine 4~(3) (2007)
  167--171.

\bibitem{loeys2010revised}
B.~L. Loeys, H.~C. Dietz, A.~C. Braverman, B.~L. Callewaert, J.~De~Backer,
  R.~B. Devereux, Y.~Hilhorst-Hofstee, G.~Jondeau, L.~Faivre, D.~M. Milewicz,
  et~al., The revised ghent nosology for the marfan syndrome, Journal of
  medical genetics 47~(7) (2010) 476--485.

\bibitem{druml1995fat}
W.~Druml, M.~Fischer, J.~Pidlich, K.~Lenz, Fat elimination in chronic hepatic
  failure: long-chain vs medium-chain triglycerides., The American journal of
  clinical nutrition 61~(4) (1995) 812--817.

\bibitem{elejalde1985prenatal}
B.~R. Elejalde, M.~M. De~Elejalde, D.~Pansch, J.~M. Opitz, J.~F. Reynolds,
  Prenatal diagnosis of jeune syndrome, American journal of medical genetics
  21~(3) (1985) 433--438.

\bibitem{nadasdy1999diffuse}
T.~Nadasdy, R.~Abdi, J.~Pitha, D.~Slakey, L.~Racusen, Diffuse glomerular
  basement membrane lamellation in renal allografts from pediatric donors to
  adult recipients, The American journal of surgical pathology 23~(4) (1999)
  437--442.

\bibitem{afzelius1985immotile}
B.~A. Afzelius, J.~Srurgess, The immotile-cilia syndrome: a
  microtubule-associated defec, CRC critical reviews in biochemistry 19~(1)
  (1985) 63--87.

\bibitem{buchthal1977peroneal}
F.~Buchthal, F.~Behse, Peroneal muscular atrophy (pma) and related disorders,
  Brain 100~(1) (1977) 41--66.

\bibitem{liang2009schizencephaly}
J.-S. Liang, Y.-H. Chien, W.-L. Hwu, S.-J. Yeh, S.-F. Peng, Schizencephaly in
  leopard syndrome, Pediatric neurology 41~(1) (2009) 71--73.

\bibitem{robin1993new}
N.~H. Robin, K.~Grace, T.~G. DeSouza, D.~McDonald-McGinn, E.~H. Zackai, New
  finding of schinzel-giedion syndrome: A case with a malignant sacrococcygeal
  teratoma, American journal of medical genetics 47~(6) (1993) 852--856.

\bibitem{reichenbach1995hereditary}
H.~Reichenbach, D.~H{\"o}rmann, H.~Theile, Hereditary congenital posterior
  dislocation of radial heads, American journal of medical genetics 55~(1)
  (1995) 101--104.

\bibitem{flanagan2002prevalence}
S.~Flanagan, C.~Munns, M.~Hayes, B.~Williams, M.~Berry, D.~Vickers, E.~Rao,
  G.~Rappold, J.~Batch, V.~Hyland, et~al., Prevalence of mutations in the short
  stature homeobox containing gene (shox) in madelung deformity of childhood,
  Journal of medical genetics 39~(10) (2002) 758--763.

\bibitem{clayton1987familial}
P.~Clayton, J.~Leonard, A.~Lawson, K.~Setchell, S.~Andersson, B.~Egestad,
  J.~Sj{\"o}vall, Familial giant cell hepatitis associated with synthesis of 3
  beta, 7 alpha-dihydroxy-and 3 beta, 7 alpha, 12 alpha-trihydroxy-5-cholenoic
  acids., Journal of Clinical Investigation 79~(4) (1987) 1031.

\bibitem{blumberg1969hepatitis}
B.~Blumberg, J.~Friedlaender, A.~Woodside, A.~Sutnick, W.~London, Hepatitis and
  australia antigen: autosomal recessive inheritance of susceptibility to
  infection in humans, Proceedings of the National Academy of Sciences 62~(4)
  (1969) 1108--1115.

\bibitem{mignot2001complex}
E.~Mignot, L.~Lin, W.~Rogers, Y.~Honda, X.~Qiu, X.~Lin, M.~Okun, H.~Hohjoh,
  T.~Miki, S.~H. Hsu, et~al., Complex hla-dr and-dq interactions confer risk of
  narcolepsy-cataplexy in three ethnic groups, The American Journal of Human
  Genetics 68~(3) (2001) 686--699.

\bibitem{einfeld1995issues}
S.~L. Einfeld, M.~Aman, Issues in the taxonomy of psychopathology in mental
  retardation, Journal of Autism and Developmental Disorders 25~(2) (1995)
  143--167.

\bibitem{kuster1988congenital}
W.~K{\"u}ster, W.~Lenz, H.~K{\"a}{\"a}ri{\"a}inen, F.~Majewski, J.~M. Opitz,
  J.~F. Reynolds, Congenital scalp defects with distal limb anomalies
  (adams-oliver syndrome): Report of ten cases and review of the literature,
  American journal of medical genetics 31~(1) (1988) 99--115.

\bibitem{gershoni1990carpenter}
R.~Gershoni-Baruch, Carpenter syndrome: Marked variability of expression to
  include the summitt and goodman syndromes, American journal of medical
  genetics 35~(2) (1990) 236--240.

\bibitem{kaariainen1989rapadilino}
H.~K{\"a}{\"a}ri{\"a}inen, S.~Ry{\"o}ppy, R.~Norio, Rapadilino syndrome with
  radial and patellar aplasia/hypoplasia as main manifestations, American
  journal of medical genetics 33~(3) (1989) 346--351.

\bibitem{sirmaci2011mutations}
A.~Sirmaci, M.~Spiliopoulos, F.~Brancati, E.~Powell, D.~Duman, A.~Abrams,
  G.~Bademci, E.~Agolini, S.~Guo, B.~Konuk, et~al., Mutations in ankrd11 cause
  kbg syndrome, characterized by intellectual disability, skeletal
  malformations, and macrodontia, The American Journal of Human Genetics 89~(2)
  (2011) 289--294.

\bibitem{larose1969metatropic}
J.~H. LAROSE, B.~B. GAY~JR, Metatropic dwarfism, American Journal of
  Roentgenology 106~(1) (1969) 156--161.

\bibitem{peet199849}
J.~Peet, D.~D. Weaver, G.~H. Vance, 49, xxxxy: a distinct phenotype. three new
  cases and review., Journal of medical genetics 35~(5) (1998) 420--424.

\bibitem{bordarier1990dandy}
C.~Bordarier, J.~Aicardi, Dandy-walker syndrome and agenesis of the cerebellar
  vermis: Diagnostic problems and genetic counselling, Developmental Medicine
  \& Child Neurology 32~(4) (1990) 285--294.

\bibitem{taylor1968nevoid}
W.~B. Taylor, D.~E. Anderson, J.~Howell, C.~S. Thurston, The nevoid basal cell
  carcinoma syndrome: autopsy findings, Archives of dermatology 98~(6) (1968)
  612--614.

\bibitem{gelman1991further}
Z.~Gelman-Kohan, J.~Antonelli, H.~Ankori-Cohen, H.~Adar, J.~Chemke, Further
  delineation of the acrocallosal syndrome, European journal of pediatrics
  150~(11) (1991) 797--799.

\bibitem{archer1978enlarged}
C.~R. Archer, H.~Darwish, K.~Smith~Jr, Enlarged cisternae magnae and posterior
  fossa cysts simulating dandy-walker syndrome on computed tomography 1,
  Radiology 127~(3) (1978) 681--686.

\bibitem{dobyns1989diagnostic}
W.~B. Dobyns, R.~A. Pagon, D.~Armstrong, C.~J. Curry, F.~Greenberg, A.~Grix,
  L.~B. Holmes, R.~Laxova, V.~V. Michels, M.~Robinow, et~al., Diagnostic
  criteria for walker-warburg syndrome, American journal of medical genetics
  32~(2) (1989) 195--210.

\end{thebibliography}


\end{document}

