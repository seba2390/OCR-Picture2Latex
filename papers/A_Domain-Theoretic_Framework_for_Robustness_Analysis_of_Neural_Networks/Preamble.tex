%%%%%%%%%%%%%%%%%%%%%%%%%%%%%%%%%%%%%%%%%%%%%%%%%%%%%%%%%
%%%%%%%%%%%%%%%%%%%%%%%%%%%%%%%%%%%%%%%%%%%%%%%%%%%%%%%%%
% Preamble to my latex files.
% Written by Amin Farjudian
% Contains:
%     . usepackages
%     . New commands for abbreviations, math symbols 
%       and other syntactic structures.
%     . New theorem like  environments.
%     . New Environments.
%     . Global style parameters.
%     . New counters.
%%%%%%%%%%%%%%%%%%%%%%%%%%%%%%%%%%%%%%%%%%%%%%%%%%%%%%%%%


%%%%%%%%%%%%%%%%%%%%%%%%%%%%%%%%%%%%%%%%%%%%%%%%%%%%%%%%%
%%%%%%%%%%%%%%%%        Packages          %%%%%%%%%%%%%%%
%%%%%%%%%%%%%%%%%%%%%%%%%%%%%%%%%%%%%%%%%%%%%%%%%%%%%%%%%





%\usepackage[page-ref=plain]{acro}
\usepackage[page-ref=none]{acro}

\usepackage{algorithm}
\usepackage{algorithmic}
\renewcommand{\algorithmicrequire}{\textbf{Input:}}
\renewcommand{\algorithmicensure}{\textbf{Output:}}
\renewcommand{\algorithmiccomment}[1]{\quad // #1}

\usepackage[sumlimits]{amsmath}
\usepackage{amssymb}
\usepackage{amsthm}
%\usepackage[dvipsnames]{xcolor}
%\usepackage[most]{tcolorbox}

\usepackage[backend=bibtex,citestyle=numeric,bibstyle=numeric,maxcitenames=3,maxbibnames=5,firstinits=true]{biblatex}
%\usepackage[backend=bibtex,citestyle=alphabetic,bibstyle=alphabetic,maxcitenames=3,maxbibnames=5,firstinits=true]{biblatex}

\DeclareNameAlias{default}{last-first}

\addbibresource{Biblio.bib}


\usepackage{color}
\usepackage{diagrams}

%\usepackage{enumerate}
\usepackage{enumitem}
%\renewcommand\theenumi{(\roman{enumi})}
\renewcommand\labelenumi{\theenumi}
%\renewcommand\theenumii{(\alph{enumii})}
\renewcommand\labelenumii{\theenumii}

\usepackage[mathscr]{eucal}
\usepackage{fancybox}
\usepackage{float}

% setting the margins
%\usepackage[a4paper, portrait, margin=1.3in]{geometry}
\usepackage[text={6.5in,9.5in},centering]{geometry}


\usepackage{graphics}
\usepackage{graphicx}

\usepackage[hidelinks]{hyperref}

\usepackage{ifpdf}
\usepackage{ifthen}
\usepackage{lineno}
\usepackage{listings}
\lstset{basicstyle=\ttfamily,frame=single,framesep=5pt,aboveskip=3ex,belowskip=2ex}

\usepackage{makecell}

%\usepackage{MnSymbol}


\usepackage{multicol}
\usepackage{multirow}

%\usepackage[square]{natbib}
% Includes for figures
% \usepackage{subcaption} % In lipics theis clashes with subfig

\usepackage{pgfplots}
%\pgfplotsset{compat=1.12} 

%\usepackage{setspace}

\usepackage{stackengine}

\usepackage{stmaryrd}
\usepackage{subfig}

%%LINES ADDED BY ADAM (FOR FIGURES)
\usepackage{tikz}
\usetikzlibrary{arrows}
\usetikzlibrary{fadings}

% \usepackage{tikz}
% \tikzset{help lines/.style={very thin}}
% \tikzset{Amin's grid/.style={ help lines, color = #1!50, xstep=0.4cm, ystep=0.2cm},
%     Amin's grid/.default=magenta}
\usepackage{txfonts}    % coloneqq

\usepackage{todonotes}
\newcommand{\noteAFinline}[1]{\todo[inline,color=green!10]{#1}}
\newcommand{\noteAFmargin}[1]{\todo[color=green!10]{#1}}

\usepackage{url}
%\usepackage{showkeys}
\usepackage{varioref}
\usepackage{xcolor}



%%%%%%%%%%%%%%%%%%%%%%%%%%%%%%%%%%%%%%%%%%%%%%%%%%%%%%%%%%%%%%
% Acronyms

\DeclareAcronym{AMP}{
  short = AMP ,
  long  = asymmetric multicore processor
}

\DeclareAcronym{BVP}{
  short = BVP ,
  long  = boundary value problem
}

\DeclareAcronym{CLEVER}{
  short = \textsc{Clever} ,
  long  = Cross Lipschitz Extreme Value for nEtwork Robustness
}

\DeclareAcronym{CPPO}{
  short = cppo ,
  long  = complete pointed partial order
}

\DeclareAcronym{CPS}{
  short = CPS ,
  long  = cyber-physical system
}

\DeclareAcronym{DAG}{
  short = DAG ,
  long  = directed acyclic graph
}

\DeclareAcronym{DBF}{
  short = DBF ,
  long  = Digital Beam Former
}

\DeclareAcronym{DCPO}{
  short = dcpo ,
  long  = directed-complete partial order
}

\DeclareAcronym{DSL}{
  short = DSL ,
  long  = domain-specific language
}

\DeclareAcronym{EDSL}{
  short = EDSL ,
  long  = embedded domain-specific language
}

\DeclareAcronym{FBP}{
  short = FBP ,
  long  = free boundary problem
}

\DeclareAcronym{FivePs}{
  short = 5Ps ,
  long  = five Ps
}


\DeclareAcronym{HS}{
  short = HS ,
  long  = hybrid system
}

\DeclareAcronym{ISA}{
  short = ISA ,
  long  = instruction set architecture
}

\DeclareAcronym{IVP}{
  short = IVP ,
  long  = initial value problem
}

\DeclareAcronym{ODE}{
  short = ODE ,
  long  = ordinary differential equation
}

\DeclareAcronym{PC}{
  short = PC ,
  long  = Pulse Compression
}

\DeclareAcronym{PCROP}{
  short = PC-ROP ,
  long  = PDE-constrained rearrangement optimization problem
}

\DeclareAcronym{PDCPO}{
  short = pointed dcpo ,
  long  = pointed directed-complete partial order
}


\DeclareAcronym{PDE}{
  short = PDE ,
  long  = partial differential equation
}


\DeclareAcronym{PI}{
  short = PI ,
  long  = principal investigator
}

\DeclareAcronym{POSET}{
  short = poset ,
  long  = partially ordered set
}

\DeclareAcronym{RE}{
  short = r.~e. ,
  long  = recursively enumerable
}

\DeclareAcronym{RNN}{
  short = RNN ,
  long  = recurrent neural network
}

\DeclareAcronym{ROP}{
  short = ROP ,
  long  = rearrangement optimization problem
}

\DeclareAcronym{SAC}{
  short = \textsc{Sac} ,
  long  = Single Assignment C
}

\DeclareAcronym{SBSE}{
  short = SBSE ,
  long  = search-based software engineerign
}

\DeclareAcronym{SOP}{
  short = SOP ,
  long  = shape optimization problem
}

\DeclareAcronym{TTE}{
  short = TTE ,
  long  = Type-II Theory of Effectivity
}

%%%%%%%%%%%%%%%%%%%%%%%%%%%%%%%%%%%%%%%%%




%%%%%%%%%%%%%%%%%%%%%%%%%%%%%%%%%%%%%%%%%%%%%%%
% Taken from macros.tex of the paper with Eugenio

%NEW THEOREMS

\newcommand{\pto}{\rightharpoonup}
%\newcommand{\defeq}{\stackrel{\vartriangle}{=}} % This is Eugenio's
%notation
\newcommand{\defeq}{\coloneqq} % This is Amin's notation

\newcommand{\Bool}{\mathbb{B}}

\renewcommand{\t}{t}
\newcommand{\T}{T}

%%BEGIN macro.tex

%%Reachability MAPS
\newcommand{\Rf}[1]{\mathsf{Rf}_{#1}}%in finite steps
\newcommand{\Rs}[1]{\mathsf{Rs}_{#1}}%safe over-approximation
\newcommand{\RT}[1]{\mathsf{R}_{#1}}%at given time 
\newcommand{\Rt}[1]{\mathsf{Rt}_{#1}}%in finite time


\newcommand{\defiff}{\stackrel{\vartriangle}{\iff}}

\newcommand{\HS}{\mathcal{H}}
\newcommand{\HLat}[1]{\mathbb{H}(#1)}

%%MATHS
%\DeclareMathOperator{\dom}{dom}
\newcommand{\Real}{\mathbb{R}}
\newcommand{\Nat}{\mathbb{N}}
\newcommand{\State}{\mathbb{S}}
\newcommand{\Time}{\mathbb{T}}
\newcommand{\Discrete}[1]{D(#1)}
%\newcommand{\Der}[1]{\dot{#1}}%derivative
\newcommand{\Der}[1]{\ensuremath{{#1}'}}%derivative
\newcommand{\inv}[1]{{#1}^{-1}}%inverse image
\newcommand{\bpca}[1]{#1^\square}%best pca

\newcommand{\Den}[1]{[\![#1]\!]}%denotation
\newcommand{\pair}[1]{\langle#1\rangle}%denotation

%%SYNTAX
\newcommand{\bnf}{\ |\ }
\newcommand{\whiled}[3]{\\mathbb{while $#1\downarrow$ then $#2$ else $#3$}}
\newcommand{\whend}[2]{\text{$#2$ when $#1\downarrow$}}
\newcommand{\whenb}[2]{\text{$#2$ when $#1$}}
\newcommand{\ifp}[3]{\text{ifp $#1$ then $#2$ else $#3$}}
\newcommand{\ifb}[3]{\text{if $#1$ then $#2$ else $#3$}}

%%Amin
\newcommand{\CB}[1]{\ensuremath{\mathscr{CB}{#1}}} % Countable Basis property for a lattice

%%SUBSETS and COMPLETE LATTICES
\newcommand{\PS}{\mathscr{P}}%powerset
\newcommand{\CS}{\mathscr{C}}%closed powerset
\newcommand{\KS}{\mathscr{K}}%compact closed powerset


\newcommand{\pCPPO}[1]{\mathbb{P}(#1)} % CPPO of non-empty subsets of #1

\newcommand{\cCPPO}[1]{\mathbb{C}#1} % CPPO of non-empty closed subsets of #1

\newcommand{\kCPPO}[1]{\mathbb{K}#1} % CPPO of non-empty closed-compact subsets of #1



\newcommand{\pLat}[1]{\hat{\mathbb{P}}(#1)} % Lattice of power subsets of #1
\newcommand{\cLat}[1]{\hat{\mathbb{C}}(#1)} % Lattice of closed subsets of #1
\newcommand{\kLat}[1]{\hat{\mathbb{K}}(#1)} % Lattice of closed-compact subsets of #1

%\newcommand{\arrayoptions}[2]{\setlength{\arraycolsep}{#1}\renewcommand{\arraystretch}{#2}}


%%CATEGORIES
\newcommand{\Set}{\mathbf{Set}}
\newcommand{\Po}{\mathbf{Po}}
\newcommand{\Cpo}{\mathbf{Cpo}}
\newcommand{\Rel}{\mathbf{Rel}}
%\newcommand{\Top}{\mathbf{Top}}
\newcommand{\C}{\mathbb{C}}%some category
\newcommand{\id}{\text{id}}
%%END

%%TOPOLOGY
\newcommand{\cl}[1]{\overline{#1}}%closure
%\newcommand{\closure}[2][3]{{}\mkern#1mu\overline{\mkern-#1mu#2}}
%%END



%%%%%%%%%%%%%%%%%%%%%%%%%%%%%%%%%%%%%%%%%%%%%%%%%%%%%%%%%
%%%%%%%%%%%%%%       Abbreviations           %%%%%%%%%%%%
%%%%%%%%%%%%%%%%%%%%%%%%%%%%%%%%%%%%%%%%%%%%%%%%%%%%%%%%%
\newcommand{\cpo}[1][s]{\ifthenelse{\equal{#1}{s}}{complete partial
order}{Complete partial order}}

\newcommand{\eg}{e.\,g.}
\newcommand{\ie}{i.\,e.}
\newcommand{\wrt}{w.\,r.\,t.}


%%%%%%%%%%%%%%%%%%%%%%%%%%%%%%%%%%%%%%%%%%%%%%%%%%%%%%%%%
%%%%%%%%%%%%%%%%       Notations           %%%%%%%%%%%%%%
%%%%%%%%%%%%%%%%%%%%%%%%%%%%%%%%%%%%%%%%%%%%%%%%%%%%%%%%%

\newcommand{\aAlphabet}{\ensuremath{\mathfrak{A}}} % answer alphabet
\newcommand{\aAlphabetOf}[1]{\ensuremath{{\aAlphabet}_{#1}}}
\newcommand{\absn}[1]{\lvert\thinspace {#1} \thinspace\rvert}


\newcommand{\alphabet}{\ensuremath{\Sigma}} % alphabet

\newcommand{\alg}{\ensuremath{\mathsf{Alg}}} % Maximisation algorithm 
\newcommand{\alga}{\ensuremath{\mathsf{Alg}_{\mathsf{a}}}} % Maximisation algorithm with protocol P_a
\newcommand{\algqa}{\ensuremath{\mathsf{Alg}_{\mathsf{qa}}}} % Maximisation algorithm with protocol P_qa

\newcommand{\algb}{\ensuremath{\mathsf{Alg}_{\mathsf{b}}}} % blind maximisation algorithm
\newcommand{\algf}{\ensuremath{\mathsf{Alg}_{\mathsf{f}}}} % focused maximisation algorithm


\newcommand{\algol}{\textsc{Algol}} % ALGOL, capitalized in SC style.

\newcommand{\basis}{\ensuremath{\mathcal{B}}} % notation for a basis B of, say, the interval domain
\newcommand{\bigparen}[1]{\ensuremath{\left( {#1} \right)}}

\newcommand{\bindepth}{\ensuremath{\mathrm{bd}}} % binary depth
\newcommand{\bindepthOf}[1]{\ensuremath{{\bindepth}(#1)}} % binary depth of
\newcommand{\bindepthOfRelTo}[2]{\ensuremath{{\bindepth}_{#2}(#1)}} % binary depth of relative to
\newcommand{\bisect}{\ensuremath{\mathsf{Bisect}}}
\newcommand{\boxVariable}{\ensuremath{\mathsf{Box}}}

\newcommand{\bra}[1]{#1^\circledcirc}%best robust approximation

\newcommand{\setcomplementOf}[1]{\ensuremath{{#1}^{\mathit{c}}}} % set complement


\newcommand{\clarke}[1]{\ensuremath{\partial #1}} % Clarke gradient of #1
\newcommand{\clarkeHR}[1]{\ensuremath{\hat{\partial} #1}}
% Hyper-rectangular estimate of the Clarke gradient of #1



\newcommand{\concat}[2]{\ensuremath{#1 \mathrel{\concatSymbol} #2}}
\newcommand{\concatSymbol}{\ensuremath{+\negthinspace +}}
\newcommand{\consistent}{\ensuremath{\uparrow}} % consistency as in
                                % domain theory

\newcommand{\Cmplx}{\ensuremath{\mathbb{C}}} % the set of complex numbers

\newcommand{\Cpp}{\ensuremath{\mathtt{C}\negthinspace + \negthickspace +}} 

\newcommand{\D}{\ensuremath{\mathbb{D}}}

\newcommand{\dateND}[1]{\ensuremath{{#1}^{\textsuperscript{nd}}}} % formatting dates such as 2nd, 22nd
\newcommand{\dateRD}[1]{\ensuremath{{#1}^{\textsuperscript{rd}}}} % formatting dates such as 3rd, 23rd
\newcommand{\dateST}[1]{\ensuremath{{#1}^{\textsuperscript{st}}}} % formatting dates such as 1st, 21st
\newcommand{\dateTH}[1]{\ensuremath{{#1}^{\textsuperscript{th}}}} % formatting dates such as 7th, 27th

\newcommand{\dft}{\ensuremath{\mathrm{DFT}}} % Discrete Fourier transform

\newcommand{\depth}{\ensuremath{\mathsf{Depth}}}
\newcommand{\derivOf}[1]{\ensuremath{D(#1)}} % derivative notation
\newcommand{\derivOfDir}[2]{\ensuremath{D_{#2}({#1})}}

\newcommand{\dif}{\mathrm{d}} % differential operator
\newcommand{\dom}{\ensuremath{\mathrm{dom}}}
\newcommand{\dsem}[1]{\ensuremath{\left\llbracket #1 \right\rrbracket}} % denotatonal semantics of


\newcommand{\escher}{\textsc{Escher}} % ESCHER, in SC style

\newcommand{\ext}[1]{\ensuremath{\widehat{\intvalfun[#1]}}} % an extension of a function f to the interval domain

\newcommand{\F}{\ensuremath{\mathbb{F}}} % The set of floating-point numbers

\newcommand{\fft}{\ensuremath{\mathrm{FFT}}} % Fast Fourier transform

\newcommand{\fieldOracle}{\ensuremath{\mathscr{F}}} % Field oracle in
                                % Picard oracle machine

\newcommand{\forth}{\textsc{Forth}} % The language FORTH

\newcommand{\fpoint}[1][s]{\ifthenelse{\equal{#1}{c}}{Floating-point}{floating-point}}

\newcommand{\FtoF}{\ensuremath{{\F}^{\F}}} % The set of floating-point numbers
\newcommand{\fpn}[1]{\ensuremath{\F_{\negmedspace #1}}} % The set of floating-point numbers representable in #1 bits
\newcommand{\fPrimeMax}{\ensuremath{F'_{\mathsf{max}}}} % Maximum of the absolute value of the derivative
\newcommand{\fPrimeMaxIn}[1]{\ensuremath{{\fPrimeMax} \left( #1 \right) }} % Maximum of the absolute value of the derivative in the set #1
\newcommand{\fSecMin}{\ensuremath{F''_{\mathsf{min}}}} % Minimum of absolute value of second derivative
\newcommand{\fSecMax}{\ensuremath{F''_{\mathsf{max}}}} % Maximum of absolute value of second derivative
\newcommand{\funEncl}[2]{\ensuremath{[ #1, #2 ]}} % function
                                % enclosures
\newcommand{\funEnclPoset}{\ensuremath{\mathbb{FE}}} % the poset of
                                % function enclosures under reverse
                                % inclusion of their graphs
\newcommand{\funProcess}{\ensuremath{F}} % The process supplying the function

\newcommand{\funSetLZ}{\ensuremath{\mathcal{F}_{\negthickspace {\lambda}_0}}} % set of functions whose maximum set has Lebesgue measure Zero
\newcommand{\funSetLLa}[1]{\ensuremath{\mathcal{F}_{\negthickspace {\lambda}_{< {#1}}}}} % set of functions whose maximum set has Lebesgue measure less than a given alpha

\newcommand{\funSetMS}{\ensuremath{\mathcal{F}_{\Sigma^1}}} % set of functions whose Maximum set is a Singleton
\newcommand{\funSetSD}{\ensuremath{\mathcal{F}_{\negthickspace \mathit{sd}}}} % set of functions with certain conditions on their second derivative
\newcommand{\funSetStar}{\ensuremath{\mathcal{F}_{\negthickspace \star}}} % The unknown set of functions in the open question

\newcommand{\graph}{\ensuremath{\Gamma}} % graph of a function
\newcommand{\graphOf}[1]{\ensuremath{\graph(#1)}} % graph of a function

\newcommand{\hbClos}[1]{\ensuremath{#1^\Box}} % Hyperbox closure of a set


%\newcommand{\id}{\ensuremath{\mathrm{id}}} % identity function

\newcommand{\imprecise}[1]{\ensuremath{\mathring{#1}}}

\newcommand{\innerChannel}{\ensuremath{\mathcal{C}}} % Main channel between function and Maximiser
\newcommand{\innerQAlphabet}{\ensuremath{\qAlphabetOf{\innerChannel}}} % qAlphabet of the main channel protocol
\newcommand{\innerAAlphabet}{\ensuremath{\aAlphabetOf{\innerChannel}}} % aAlphabet of the main channel protocol
\newcommand{\inRegs}{\ensuremath{\mathsf{inRegs}}}
\newcommand{\interiorOf}[1]{\ensuremath{{#1}^{\circ}}}

\newcommand{\intvalType}{\ensuremath{\mathtt{interval}}} % type of interval [0,1]

\newcommand{\intervalOf}[2]{\ensuremath{\left[ #1, #2 \right]}} % notation for intervals especially with tall symbols
\newcommand{\intervalUnderOver}[1]{\ensuremath{[ \underline{#1}, \overline{#1}]}} % writing the end-points of an interval with underline and overline
%\newcommand{\intvaldom}[1][{\Rinf}]{\ensuremath{{\boldsymbol{\mathbb{I}}}#1}}
\newcommand{\intvaldom}[1][{\Rinf}]{\ensuremath{{\mathbb{I}}#1}}
%\newcommand{\intvaldom}[1][{\Rinf}]{\ensuremath{{\mathbf{I}#1}}}
\newcommand{\intvallat}[1][{\Rinf}]{\ensuremath{{\hat{\mathbf{I}}#1}}}
%\newcommand{\intvaldom}[1][{\Rinf}]{\ensuremath{{\boldsymbol{\mathrm{I}}}#1}}
%\newcommand{\intvalfun}[1][f]{\ensuremath{\mathcal{I}{\negmedspace}#1}}
\newcommand{\intvalfun}[1][f]{\ensuremath{{\boldsymbol{I}}#1}}
\newcommand{\intvalfunWithType}{\ensuremath{{\intvalfun} : {\intvaldom[{[0,1]}]} \to {\intvaldom}}} % handy way of writing the interval function and its type, used quite often in this paper
\newcommand{\invarIdeal}{\ensuremath{\mathcal{U}}} % default symbol for an
                                % invariant ideal

\newcommand{\invarIdealOf}[2]{\ensuremath{#1 \oplus #2}} % smallest
                                % inv ideal over an inv ideal and a function

\newcommand{\kc}[1][s]{\ifthenelse{\equal{#1}{s}}{Kolmogorov
complexity}{Kolmogorov complexities}} % short hand for Kolmogorov complexity, s for singular and otherwise plural                                

\newcommand{\kolcom}{\ensuremath{\mathit{K}}} % Kolmogorov complexity symbol
\newcommand{\kolcomfun}[1]{\ensuremath{\kolcom (#1)}} % Kolmogorov
                                % complexity function


\newcommand{\kolcomc}{\ensuremath{\kolcom_{\mathit{C}}}} % Kolmogorov complexity over continuous functions symbol
\newcommand{\kolcomfunc}[1]{\ensuremath{\kolcomc (#1)}} % Kolmogorov complexity over continuous functions function
\newcommand{\kolcomfuncAt}[2]{\ensuremath{{\kolcomc (#1)}\thinspace (#2)}} % Kolmogorov complexity over continuous functions function at a point
\newcommand{\invkolcomfun}[1]{\ensuremath{\kolcom^{-1} (#1)}} % inverse of the Kolmogorov complexity function
\newcommand{\invkolcomfunc}[1]{\ensuremath{{\kolcom}_{\mathit{C}}^{-1}
(#1)}} % The inverse of Kolmogorov complexity function over continuous functions

\newcommand{\kolcomr}{\ensuremath{\kolcom_{\R}}} % Kolmogorov complexity over real numbers symbol
\newcommand{\kolcomfunr}[1]{\ensuremath{\kolcomr (#1)}} % Kolmogorov complexity over real numbers
\newcommand{\invkolcomfunr}[1]{\ensuremath{{\kolcom}_{\R}^{-1} (#1)}} % The inverse of Kolmogorov complexity function over real numbers

\newcommand{\kolcoms}{\ensuremath{\kappa}} % Kolmogorov complexity
                                % symbol for strings
\newcommand{\kolcomfuns}[1]{\ensuremath{\kolcoms(#1)}} % Kolmogorov
                                % complexity function for strings



\newcommand{\lebMeas}[1]{\ensuremath{{\lvert\thinspace {#1}
\thinspace\rvert}}} % Lebesgue measure

\newcommand{\lengthOf}[1]{\ensuremath{| #1 |}}

\newcommand{\lep}[1][x]{\ensuremath{{\underline{#1}}}} % Lower End-Point.of an interval.
\newcommand{\uep}[1][x]{\ensuremath{{\overline{#1}}}} % Upper End-Point.of an interval.

\newcommand{\lift}{_\bot} % Adding a bottom element to a poset

\newcommand{\glb}{\ensuremath{\bigsqcap}} % greatest lower bound
\newcommand{\lub}{\ensuremath{\bigsqcup}} % least upper bound

\newcommand{\maxBox}{\ensuremath{\mathsf{MaxBox}}}
\newcommand{\maxIntval}{\ensuremath{\mathsf{Max}}}
\newcommand{\maxProcess}{\ensuremath{\mathit{Max}}} % the maximiser process
\newcommand{\maxSet}[1]{\ensuremath{\mathit{MaxSet}(#1)}} % The operator returning the maximum set of a function
\newcommand{\pMaxSet}[2]{\ensuremath{\mathit{MaxSet}(#1, #2)}} % partial maximum set at depth n

\newcommand{\ptime}[1][s]{\ifthenelse{\equal{#1}{c}}{Polynomial-time}{polynomial-time}}
\newcommand{\pspace}[1][s]{\ifthenelse{\equal{#1}{c}}{Polynomial-space}{polynomial-space}}


\newcommand{\matlab}{\textsc{Matlab}} 

\newcommand{\md}{\thinspace \mathrm{d}} % differential symbol
\newcommand{\me}{\mathrm{e}} % exponential symbol
\newcommand{\mi}{\mathrm{i}} % imaginary complex number i

\newcommand{\N}{\ensuremath{\mathbb{N}}}
\newcommand{\Nplus}{\ensuremath{{\N}_{+}}}
\newcommand{\NtoN}{\ensuremath{{\N}^{\N}}} % the set of functions from N to N
\newcommand{\NFL}[1][c]{\ifthenelse{\equal{#1}{s}}{no free lunch}{No
Free Lunch}}

%\newcommand{\norm}[1]{\ensuremath{{\lVert\thinspace {#1} \thinspace\rVert}}} % norm
\newcommand{\norm}[1]{\ensuremath{{\left\lVert\thinspace {#1}
        \thinspace \right\rVert}}} % norm


\newcommand{\normSup}[1]{\ensuremath{{\lVert\thinspace {#1}
\thinspace\rVert}_{\infty}}} % supremum norm
\newcommand{\ntuple}[3]{\ensuremath{\left( {#1}_{#2}, \ldots, {#1}_{#3} \right)}}

\newcommand{\outputChannel}{\ensuremath{\mathcal{O}}} % Output channel
\newcommand{\outRegs}{\ensuremath{\mathsf{outRegs}}}

\newcommand{\paren}[1]{\ensuremath{\left({#1}\right)}}
\newcommand{\pcomp}{\ensuremath{P_{C[0,1]}}} % Polynomial-time computable functions in C[0,1] 
\newcommand{\PeqNP}{\ensuremath{\mathrm{P}\thinspace
{?\negthickspace=} \mathrm{NP}}} % P \?= NP

\newcommand{\finiteSubsets}[1]{\ensuremath{{\cal
P}_{\mathit{\negthinspace fin}}} \left( #1 \right)}
% set of finite subsets of #1

\newcommand{\PNP}{\ensuremath{\mathrm{P}=\mathrm{NP}}} % P = NP

\newcommand{\polyEncl}[2]{\ensuremath{{<}#1, #2{>}}} % polynomial
                                                              % enclosures


\newcommand{\polyEnclPoset}{\ensuremath{\mathbb{PE}}} % the poset of
                                % polynomial enclosures under reverse
                                % inclusion of their graphs


\newcommand{\polyRing}[2]{\ensuremath{#1[#2]}} % ring of polynomials
                                % such as R[X]
\newcommand{\powerSetOf}[1]{\ensuremath{\mathcal{P}(#1)}} % P(X) = Power set of X
\newcommand{\powerTypeOf}[1]{\ensuremath{\mathtt{P}\ #1}} % power type
\newcommand{\principalIdealOf}[1]{\ensuremath{\downarrow\negthinspace #1}} % principal ideal of
\newcommand{\probeSet}{\ensuremath{\mathcal{P}_{\negthickspace \scriptscriptstyle{\id}}}} % The subspace of C(U) spanned by identity, used as a probe
\newcommand{\protocol}{\ensuremath{\mathcal{P}}} % Protocols
\newcommand{\protocolOf}[1]{\ensuremath{{\protocol}_{{\negthinspace}#1}}}

\newcommand{\Q}{\ensuremath{\mathbb{Q}}}
\newcommand{\Qplus}{\ensuremath{{\Q}_{+}}}
\newcommand{\QnonNeg}{\ensuremath{{\Q}_{\geq 0}}}

\newcommand{\qAlphabet}{\ensuremath{\mathfrak{Q}}} % query alphabet
\newcommand{\qAlphabetOf}[1]{\ensuremath{{\qAlphabet}_{#1}}}
\newcommand{\qaaAlphabet}{\ensuremath{\aAlphabetOf{\mathit{qa}}}}
\newcommand{\qaProtocol}{\ensuremath{\protocolOf{\negthinspace \mathit{qa}}}} % The query-answer protocol
\newcommand{\qaqAlphabet}{\ensuremath{\qAlphabetOf{\mathit{qa}}}}


\newcommand{\R}{\ensuremath{\mathbb{R}}}
\newcommand{\Rinf}{\ensuremath{\R_\infty}}
\newcommand{\Rinfn}[1][n]{\ensuremath{\R^{#1}_{\infty}}}

\newcommand{\realType}{\ensuremath{\mathtt{real}}} % type of real numbers R

\newcommand{\aaAlphabet}{\ensuremath{\aAlphabetOf{a}}}
\newcommand{\aqAlphabet}{\ensuremath{\qAlphabetOf{a}}}
\newcommand{\aProtocol}{\ensuremath{\protocolOf{\negthinspace a}}} % The answer protocol (i.e. query-free)

\newcommand{\reg}{\ensuremath{\mathit{Reg}}} % the region around the maximum point
\newcommand{\regs}{\ensuremath{\mathsf{Regs}}}
\newcommand{\regions}{\ensuremath{\mathsf{Regions}}}
\newcommand{\regOfDep}[2]{\ensuremath{{\reg}_{#2} ( #1) }} % region of
                                % at depth
\newcommand{\restrictTo}[2]{\ensuremath{#1 \upharpoonright_{#2}}} % Restriction of a function


\newcommand{\Scott}[1]{\ensuremath{\Sigma ({#1})}} % Scott topology of a given DCPO



% notation for an finite/infinite sequence of numbers
% first argument is optional: i for infinite [default] and f for finite
\newcommand{\sequence}[5][i]{\ifthenelse{\equal{#1}{i}}{\ensuremath{\left< {#2}_{#3}, {#2}_{#4},
\ldots,{#2}_{#5}, \ldots \right>}}{\ensuremath{\left< {#2}_{#3}, {#2}_{#4},
\ldots,{#2}_{#5} \right>}}}

\newcommand{\sequenceAngle}[2]{\ensuremath{ \left< #1 \right>_{#2}}} % angle bracket notation for sequences


\newcommand{\set}[1]{\ensuremath{\left\{{#1}\right\}}}
\newcommand{\setbarTall}[2]{\ensuremath{\left\{{#1}\;\vrule\;
{#2}\right\}}} % set comprehension notation with tall bar and braces
\newcommand{\setbarNormal}[2]{\ensuremath{\{{#1} \mid
{#2}\}}}% set comprehension notation with normal size bar and
              % braces

\newcommand{\shrinkRate}{\ensuremath{c}} % the rate at which intervals shrink to the maximum point
\newcommand{\cardOf}[1]{\ensuremath{\card(#1)}} % cardinality of a set
\newcommand{\slice}{\ensuremath{\mathit{Slice}}} % The slice operator
\newcommand{\sliceBasis}{\ensuremath{\mathcal{S}}} % The basis formed
                                % by the slices
                                % \newcommand{\sliceDepth}[1]{\ensuremath{{\slice}_{#1}}}

\newcommand{\sliceDepthOf}[2]{\ensuremath{{\slice}_{#1}({#2})}}
\newcommand{\sliceToStep}[1]{\ensuremath{\widehat{#1}}} % The function
                                % constructed from a slice using step functions

%\newcommand{\smythOf}[1]{\ensuremath{\mathit{Smyth}(#1)}} % Smyth power domain of
\newcommand{\smythOf}[1]{\ensuremath{{\cal P}^S(#1)}} % Smyth power domain of



\newcommand{\maxSlice}{\ensuremath{\mathsf{MaxSlice}}}
\newcommand{\sliceVar}{\ensuremath{\mathsf{Slice}}}

\newcommand{\solverMachine}{\ensuremath{\mathscr{S}}} % solver machine
                                % in Picard oracle machine


\newcommand{\spiral}{\textsc{Spiral}} 

\newcommand{\supNorm}[1]{\ensuremath{{\Arrowvert #1
\Arrowvert}_{\scriptscriptstyle\mathrm{sup}}}}

\newcommand{\symExpand}{\ensuremath{\oplus}} % Domain theoretic symmetric expansion

\newcommand{\templar}{\textsc{Templar}} % Templar, in SC style

\newcommand{\unionOfClosures}{\ensuremath{\mathrm{UCL}}} % union of
                                % closures (used in a proof in the
                                % Kolmogorov paper)
\newcommand{\unitIntvalDom}{\ensuremath{\intvaldom[{[0,1]}]}} % unit
                                % interval domain
\newcommand{\unitIntvalFunDom}{\ensuremath{\mathbb{IF}}} % the domain
                                % of continuous functions over unit
                                % interval

\newcommand{\unitCircle}{\ensuremath{S^{1}}} 

\newcommand{\U}{\ensuremath{\mathbb{U}}} % U = [0,1], the unit interval

\newcommand{\UpC}[1]{\uparrow\negthickspace{#1}}%closed subsets included in #1


%\newcommand{\wayaboves}[1]{\ensuremath{\Uparrow \negmedspace #1}}
\newcommand{\wayaboves}[1]{\ensuremath{\twoheaduparrow {#1}}}
%\newcommand{\waybelows}[1]{\ensuremath{\Downarrow {#1}}}
\newcommand{\waybelows}[1]{\ensuremath{\twoheaddownarrow {#1}}}


%\newcommand{\widthOf}[1]{\ensuremath{w(#1)}} % width of

\newcommand{\Z}{\ensuremath{\mathbb{Z}}}


%%%%%%%%%%%%%%%%%%%%%%%%%%%%%%%%%%%%%%%%%%%%%%%%%%%%%%%%%
%%%%%%%%%%%%%%%%          Commands        %%%%%%%%%%%%%%%
%%%%%%%%%%%%%%%%%%%%%%%%%%%%%%%%%%%%%%%%%%%%%%%%%%%%%%%%%

\newcommand{\includexfig}[1]
{\ifpdf
 \input{#1.pdftex_t}
 \else
 \input{#1.pstex_t}
 \fi}



% New Theorem like environments
%%%%%%%%%%%%%%%%%%%%%%%%%%%%%%%%%%%%%%

\theoremstyle{plain} % The following theorem-like environments will have the title and number boldface, text italic

% The following are numbered independently
%\newtheorem{algorithm}{Algorithm}
%\newtheorem{assumption}{Assumption}
\newtheorem{definition}{Definition}[section]
%\newtheorem{question}{Question}

% Numbering for these follow that of Definition
\newtheorem{corollary}[definition]{Corollary}
\newtheorem{example}[definition]{Example}
\newtheorem{lemma}[definition]{Lemma}
\newtheorem{notation}[definition]{Notation}
\newtheorem{proposition}[definition]{Proposition}
\newtheorem{remark}[definition]{Remark}
\newtheorem{theorem}[definition]{Theorem}



% The following a normal text rather than bold
%\theoremstyle{definition}
%\newtheorem{conj}{Conjecture}
%\newtheorem{chal}{Challenge}


%New Environments
%%%%%%%%%%%%%%%%%%%%%%%%%%%%%%%%%%%%%%

\newcounter{defenumalph}
\newenvironment{defenumalph}
  {\begin{list}
     {(\alph{defenumalph})}
     {\usecounter{defenumalph} 
      \setlength{\itemsep}{0.4ex plus 0.1ex minus0.1ex}
      \setlength{\labelwidth}{3ex}
      \setlength{\labelsep}{0.5ex}
      \setlength{\leftmargin}{5ex}
      \setlength{\partopsep}{0ex plus0.2ex minus0.1ex}
      \setlength{\topsep}{\itemsep}}}
   {\end{list}}
%
\newcounter{defenum}
\newenvironment{defenum}
  {\begin{list}
     {\arabic{defenum}.}
     {\usecounter{defenum} 
      \setlength{\itemsep}{0.4ex plus 0.1ex minus0.1ex}
      \setlength{\labelwidth}{3ex}
      \setlength{\labelsep}{0.5ex}
      \setlength{\leftmargin}{5ex}
      \setlength{\partopsep}{0ex plus0.2ex minus0.1ex}
      \setlength{\topsep}{\itemsep}}}
   {\end{list}}

%
% \newcounter{defitemize}
\newenvironment{defitemize}
  {\begin{list}
     {\labelitemi}
     {%\usecounter{defenum} 
      \setlength{\itemsep}{0.4ex plus 0.1ex minus0.1ex}
      \setlength{\labelwidth}{3ex}
      \setlength{\labelsep}{0.5ex}
      \setlength{\leftmargin}{5ex}
      \setlength{\partopsep}{0ex plus0.2ex minus0.1ex}
      \setlength{\topsep}{\itemsep}}}
   {\end{list}}



%%%%%%%%%%%%%%%%%%%%%%%%%%%%%%%%%%%%%%%%%%%%%%%%%%%%%%%%%%%%%%%%
% New global style parameters
\floatstyle{ruled} %used with the ``float'' package to style figures
\setlength{\textfloatsep}{1em}
\restylefloat{figure}
%\floatstyle{boxed}
\restylefloat{table}

% horizontal line below float figures
%\newcommand{\topfigrule}{\vspace*{-3pt}\rule{\columnwidth}{0.4pt}\vspace{2.6pt}}
% \newcommand{\topfigrule}{\hrule\kern-0.4pt\relax}


% Commands to deal with undesired counter handling:
%%%%%%%%%%%%%%%%%%%%%%%%%%%%%%%%%%%%%%

%From Kupka's Book, page 190
\newcounter{saveeqn}
\newcommand{\alpheqn}{\setcounter{saveeqn}{\value{equation}}
  \stepcounter{saveeqn}%
  \setcounter{equation}{0}%
  \renewcommand{\theequation}
      {\mbox{\arabic{saveeqn}.\alph{equation}}}}
      
\newcommand{\reseteqn}{\setcounter{equation}{\value{saveeqn}}%
  \renewcommand{\theequation}{\arabic{equation}}}


% Setting array options
\newcommand{\arrayoptions}[2]{\setlength{\arraycolsep}{#1}\renewcommand{\arraystretch}{#2}}

% Counters to use
%%%%%%%%%%%%%%%%%%%%%%%%%%%%%%%%%%%%%%

\newcounter{xLim}
\newcounter{yLim}
\newcounter{xPos}



%%%%%%%%%%%%%%%%%%%%%%%%%%%%%%%%%%%%%%
% New Log-Like Symbols
%%%%%%%%%%%%%%%%%%%%%%%%%%%%%%%%%%%%%%

\DeclareMathOperator{\algA}{\mathcal{A}}
\DeclareMathOperator{\algB}{\mathcal{B}}

\DeclareMathOperator{\atan}{atan}

\DeclareMathOperator{\bigO}{\mathit{O}} % the big O notation


\DeclareMathOperator{\can}{can} % canonical domain-theoretic extension of a classical map 

\DeclareMathOperator{\card}{card} % Cardinality of a set


\DeclareMathOperator{\closure}{cl} % topological closure of a set

\DeclareMathOperator{\commcmpl}{\mathit{CC}} % The communication complexity function

\DeclareMathOperator{\Cons}{Cons} % The consistency relation in domains

\DeclareMathOperator{\convHull}{co} % convex hull

\DeclareMathOperator{\CP}{\mathit{CP}} % composition of polynomials in the Picard paper


\DeclareMathOperator{\cbound}{cb} % chebyshev-bound
\DeclareMathOperator{\ctrunc}{ct} % chebyshev-truncation


\DeclareMathOperator{\degree}{deg} % degree of, e.g. a polynomial

\DeclareMathOperator{\fix}{fix} % fixed-point operator (mainly in domain theory)
\DeclareMathOperator{\fst}{fst}
\DeclareMathOperator{\funComp}{\circ} % function composition

\DeclareMathOperator{\limRat}{\Omega} % limit of ratios
\DeclareMathOperator{\Lip}{Lip} % Lip(a) is the set of Lipschitz
                                % functions, where Lip(1) is the usual
                                % set of Lipschitz functions. See page
                                % 254 of Ko book
\DeclareMathOperator{\lpol}{lpol} % lower polyline

\DeclareMathOperator{\maxP}{max_P} % The operator returning a maximum point of a function
%\DeclareMathOperator{\maxP}{maxp} % The operator returning a maximum point of a function


\DeclareMathOperator{\maxS}{max_S} % The operator returning the maximum set of a function
%\DeclareMathOperator{\maxS}{maxs} % The operator returning the maximum set of a function

\DeclareMathOperator{\maxV}{max_V} % The operator returning the maximum value of a function
%\DeclareMathOperator{\maxV}{maxv} % The operator returning the maximum value of a function

\DeclareMathOperator{\modulus}{\omega} % modulus of a continuous real function


\DeclareMathOperator{\NS}{NS} % operator on functions which returns the set of t where max set of f+ti is Non-Singleton

\DeclareMathOperator{\opt}{\Box} % optimizing operator: could be inf or sup



\DeclareMathOperator{\peak}{\mathit{Peak}}
\DeclareMathOperator{\picop}{\mathit{Pic}} % Picard iteration
\DeclareMathOperator{\proj}{\pi}

\DeclareMathOperator{\ReLU}{\mathrm{ReLU}}

\DeclareMathOperator{\SCons}{SCons} % The strong consistency relation in domains

\DeclareMathOperator{\selQ}{\sigma_{\negthinspace Q}} % The operator which selects an interval function for the quality function Q

\DeclareMathOperator{\symDiff}{\Delta} % symmetric difference of sets

\DeclareMathOperator{\sizeOf}{s}

\DeclareMathOperator{\snd}{snd}

\DeclareMathOperator{\traceOf}{tr} % the trace of a function under a certain algorithm, with a certain length
\DeclareMathOperator{\tracesOf}{Tr} % the set of traces of an alphabet

\DeclareMathOperator{\tbound}{tb} % taylor-bound
\DeclareMathOperator{\ttrunc}{tt} % taylor-truncation


\DeclareMathOperator{\upar}{upar} % uniform partition
\DeclareMathOperator{\upol}{upol} % upper polyline
\DeclareMathOperator{\upper}{\uparrow} % upper set of an element in a poset


\DeclareMathOperator{\widthOf}{w} % width of

%%%%%%%%%%%%%%%%%%%%%%%%%%%%%%%%%%%%%%


\endinput