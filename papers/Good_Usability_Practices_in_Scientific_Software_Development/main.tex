%% bare_conf.tex
%% V1.3
%% 2007/01/11
%% by Michael Shell
%% See:
%% http://www.michaelshell.org/
%% for current contact information.
%%
%% This is a skeleton file demonstrating the use of IEEEtran.cls
%% (requires IEEEtran.cls version 1.7 or later) with an IEEE conference paper.
%%
%% Support sites:
%% http://www.michaelshell.org/tex/ieeetran/
%% http://www.ctan.org/tex-archive/macros/latex/contrib/IEEEtran/
%% and
%% http://www.ieee.org/

%%*************************************************************************
%% Legal Notice:
%% This code is offered as-is without any warranty either expressed or
%% implied; without even the implied warranty of MERCHANTABILITY or
%% FITNESS FOR A PARTICULAR PURPOSE!
%% User assumes all risk.
%% In no event shall IEEE or any contributor to this code be liable for
%% any damages or losses, including, but not limited to, incidental,
%% consequential, or any other damages, resulting from the use or misuse
%% of any information contained here.
%%
%% All comments are the opinions of their respective authors and are not
%% necessarily endorsed by the IEEE.
%%
%% This work is distributed under the LaTeX Project Public License (LPPL)
%% ( http://www.latex-project.org/ ) version 1.3, and may be freely used,
%% distributed and modified. A copy of the LPPL, version 1.3, is included
%% in the base LaTeX documentation of all distributions of LaTeX released
%% 2003/12/01 or later.
%% Retain all contribution notices and credits.
%% ** Modified files should be clearly indicated as such, including  **
%% ** renaming them and changing author support contact information. **
%%
%% File list of work: IEEEtran.cls, IEEEtran_HOWTO.pdf, bare_adv.tex,
%%                    bare_conf.tex, bare_jrnl.tex, bare_jrnl_compsoc.tex
%%*************************************************************************

% *** Authors should verify (and, if needed, correct) their LaTeX system  ***
% *** with the testflow diagnostic prior to trusting their LaTeX platform ***
% *** with production work. IEEE's font choices can trigger bugs that do  ***
% *** not appear when using other class files.                            ***
% The testflow support page is at:
% http://www.michaelshell.org/tex/testflow/



% Note that the a4paper option is mainly intended so that authors in
% countries using A4 can easily print to A4 and see how their papers will
% look in print - the typesetting of the document will not typically be
% affected with changes in paper size (but the bottom and side margins will).
% Use the testflow package mentioned above to verify correct handling of
% both paper sizes by the user's LaTeX system.
%
% Also note that the "draftcls" or "draftclsnofoot", not "draft", option
% should be used if it is desired that the figures are to be displayed in
% draft mode.
%
\documentclass[conference]{IEEEtran}
\usepackage{blindtext, graphicx, todonotes}
\usepackage[utf8]{inputenc}
% Add the compsoc option for Computer Society conferences.
%
% If IEEEtran.cls has not been installed into the LaTeX system files,
% manually specify the path to it like:
% \documentclass[conference]{../sty/IEEEtran}





% Some very useful LaTeX packages include:
% (uncomment the ones you want to load)


% *** MISC UTILITY PACKAGES ***
%
%\usepackage{ifpdf}
% Heiko Oberdiek's ifpdf.sty is very useful if you need conditional
% compilation based on whether the output is pdf or dvi.
% usage:
% \ifpdf
%   % pdf code
% \else
%   % dvi code
% \fi
% The latest version of ifpdf.sty can be obtained from:
% http://www.ctan.org/tex-archive/macros/latex/contrib/oberdiek/
% Also, note that IEEEtran.cls V1.7 and later provides a builtin
% \ifCLASSINFOpdf conditional that works the same way.
% When switching from latex to pdflatex and vice-versa, the compiler may
% have to be run twice to clear warning/error messages.






% *** CITATION PACKAGES ***
%
%\usepackage{cite}
% cite.sty was written by Donald Arseneau
% V1.6 and later of IEEEtran pre-defines the format of the cite.sty package
% \cite{} output to follow that of IEEE. Loading the cite package will
% result in citation numbers being automatically sorted and properly
% "compressed/ranged". e.g., [1], [9], [2], [7], [5], [6] without using
% cite.sty will become [1], [2], [5]--[7], [9] using cite.sty. cite.sty's
% \cite will automatically add leading space, if needed. Use cite.sty's
% noadjust option (cite.sty V3.8 and later) if you want to turn this off.
% cite.sty is already installed on most LaTeX systems. Be sure and use
% version 4.0 (2003-05-27) and later if using hyperref.sty. cite.sty does
% not currently provide for hyperlinked citations.
% The latest version can be obtained at:
% http://www.ctan.org/tex-archive/macros/latex/contrib/cite/
% The documentation is contained in the cite.sty file itself.






% *** GRAPHICS RELATED PACKAGES ***
%
\ifCLASSINFOpdf
  % \usepackage[pdftex]{graphicx}
  % declare the path(s) where your graphic files are
  % \graphicspath{{../pdf/}{../jpeg/}}
  % and their extensions so you won't have to specify these with
  % every instance of \includegraphics
  % \DeclareGraphicsExtensions{.pdf,.jpeg,.png}
\else
  % or other class option (dvipsone, dvipdf, if not using dvips). graphicx
  % will default to the driver specified in the system graphics.cfg if no
  % driver is specified.
  % \usepackage[dvips]{graphicx}
  % declare the path(s) where your graphic files are
  % \graphicspath{{../eps/}}
  % and their extensions so you won't have to specify these with
  % every instance of \includegraphics
  % \DeclareGraphicsExtensions{.eps}
\fi
% graphicx was written by David Carlisle and Sebastian Rahtz. It is
% required if you want graphics, photos, etc. graphicx.sty is already
% installed on most LaTeX systems. The latest version and documentation can
% be obtained at:
% http://www.ctan.org/tex-archive/macros/latex/required/graphics/
% Another good source of documentation is "Using Imported Graphics in
% LaTeX2e" by Keith Reckdahl which can be found as epslatex.ps or
% epslatex.pdf at: http://www.ctan.org/tex-archive/info/
%
% latex, and pdflatex in dvi mode, support graphics in encapsulated
% postscript (.eps) format. pdflatex in pdf mode supports graphics
% in .pdf, .jpeg, .png and .mps (metapost) formats. Users should ensure
% that all non-photo figures use a vector format (.eps, .pdf, .mps) and
% not a bitmapped formats (.jpeg, .png). IEEE frowns on bitmapped formats
% which can result in "jaggedy"/blurry rendering of lines and letters as
% well as large increases in file sizes.
%
% You can find documentation about the pdfTeX application at:
% http://www.tug.org/applications/pdftex





% *** MATH PACKAGES ***
%
%\usepackage[cmex10]{amsmath}
% A popular package from the American Mathematical Society that provides
% many useful and powerful commands for dealing with mathematics. If using
% it, be sure to load this package with the cmex10 option to ensure that
% only type 1 fonts will utilized at all point sizes. Without this option,
% it is possible that some math symbols, particularly those within
% footnotes, will be rendered in bitmap form which will result in a
% document that can not be IEEE Xplore compliant!
%
% Also, note that the amsmath package sets \interdisplaylinepenalty to 10000
% thus preventing page breaks from occurring within multiline equations. Use:
%\interdisplaylinepenalty=2500
% after loading amsmath to restore such page breaks as IEEEtran.cls normally
% does. amsmath.sty is already installed on most LaTeX systems. The latest
% version and documentation can be obtained at:
% http://www.ctan.org/tex-archive/macros/latex/required/amslatex/math/





% *** SPECIALIZED LIST PACKAGES ***
%
%\usepackage{algorithmic}
% algorithmic.sty was written by Peter Williams and Rogerio Brito.
% This package provides an algorithmic environment fo describing algorithms.
% You can use the algorithmic environment in-text or within a figure
% environment to provide for a floating algorithm. Do NOT use the algorithm
% floating environment provided by algorithm.sty (by the same authors) or
% algorithm2e.sty (by Christophe Fiorio) as IEEE does not use dedicated
% algorithm float types and packages that provide these will not provide
% correct IEEE style captions. The latest version and documentation of
% algorithmic.sty can be obtained at:
% http://www.ctan.org/tex-archive/macros/latex/contrib/algorithms/
% There is also a support site at:
% http://algorithms.berlios.de/index.html
% Also of interest may be the (relatively newer and more customizable)
% algorithmicx.sty package by Szasz Janos:
% http://www.ctan.org/tex-archive/macros/latex/contrib/algorithmicx/




% *** ALIGNMENT PACKAGES ***
%
%\usepackage{array}
% Frank Mittelbach's and David Carlisle's array.sty patches and improves
% the standard LaTeX2e array and tabular environments to provide better
% appearance and additional user controls. As the default LaTeX2e table
% generation code is lacking to the point of almost being broken with
% respect to the quality of the end results, all users are strongly
% advised to use an enhanced (at the very least that provided by array.sty)
% set of table tools. array.sty is already installed on most systems. The
% latest version and documentation can be obtained at:
% http://www.ctan.org/tex-archive/macros/latex/required/tools/


%\usepackage{mdwmath}
%\usepackage{mdwtab}
% Also highly recommended is Mark Wooding's extremely powerful MDW tools,
% especially mdwmath.sty and mdwtab.sty which are used to format equations
% and tables, respectively. The MDWtools set is already installed on most
% LaTeX systems. The lastest version and documentation is available at:
% http://www.ctan.org/tex-archive/macros/latex/contrib/mdwtools/


% IEEEtran contains the IEEEeqnarray family of commands that can be used to
% generate multiline equations as well as matrices, tables, etc., of high
% quality.


%\usepackage{eqparbox}
% Also of notable interest is Scott Pakin's eqparbox package for creating
% (automatically sized) equal width boxes - aka "natural width parboxes".
% Available at:
% http://www.ctan.org/tex-archive/macros/latex/contrib/eqparbox/





% *** SUBFIGURE PACKAGES ***
%\usepackage[tight,footnotesize]{subfigure}
% subfigure.sty was written by Steven Douglas Cochran. This package makes it
% easy to put subfigures in your figures. e.g., "Figure 1a and 1b". For IEEE
% work, it is a good idea to load it with the tight package option to reduce
% the amount of white space around the subfigures. subfigure.sty is already
% installed on most LaTeX systems. The latest version and documentation can
% be obtained at:
% http://www.ctan.org/tex-archive/obsolete/macros/latex/contrib/subfigure/
% subfigure.sty has been superceeded by subfig.sty.



%\usepackage[caption=false]{caption}
%\usepackage[font=footnotesize]{subfig}
% subfig.sty, also written by Steven Douglas Cochran, is the modern
% replacement for subfigure.sty. However, subfig.sty requires and
% automatically loads Axel Sommerfeldt's caption.sty which will override
% IEEEtran.cls handling of captions and this will result in nonIEEE style
% figure/table captions. To prevent this problem, be sure and preload
% caption.sty with its "caption=false" package option. This is will preserve
% IEEEtran.cls handing of captions. Version 1.3 (2005/06/28) and later
% (recommended due to many improvements over 1.2) of subfig.sty supports
% the caption=false option directly:
%\usepackage[caption=false,font=footnotesize]{subfig}
%
% The latest version and documentation can be obtained at:
% http://www.ctan.org/tex-archive/macros/latex/contrib/subfig/
% The latest version and documentation of caption.sty can be obtained at:
% http://www.ctan.org/tex-archive/macros/latex/contrib/caption/




% *** FLOAT PACKAGES ***
%
%\usepackage{fixltx2e}
% fixltx2e, the successor to the earlier fix2col.sty, was written by
% Frank Mittelbach and David Carlisle. This package corrects a few problems
% in the LaTeX2e kernel, the most notable of which is that in current
% LaTeX2e releases, the ordering of single and double column floats is not
% guaranteed to be preserved. Thus, an unpatched LaTeX2e can allow a
% single column figure to be placed prior to an earlier double column
% figure. The latest version and documentation can be found at:
% http://www.ctan.org/tex-archive/macros/latex/base/



%\usepackage{stfloats}
% stfloats.sty was written by Sigitas Tolusis. This package gives LaTeX2e
% the ability to do double column floats at the bottom of the page as well
% as the top. (e.g., "\begin{figure*}[!b]" is not normally possible in
% LaTeX2e). It also provides a command:
%\fnbelowfloat
% to enable the placement of footnotes below bottom floats (the standard
% LaTeX2e kernel puts them above bottom floats). This is an invasive package
% which rewrites many portions of the LaTeX2e float routines. It may not work
% with other packages that modify the LaTeX2e float routines. The latest
% version and documentation can be obtained at:
% http://www.ctan.org/tex-archive/macros/latex/contrib/sttools/
% Documentation is contained in the stfloats.sty comments as well as in the
% presfull.pdf file. Do not use the stfloats baselinefloat ability as IEEE
% does not allow \baselineskip to stretch. Authors submitting work to the
% IEEE should note that IEEE rarely uses double column equations and
% that authors should try to avoid such use. Do not be tempted to use the
% cuted.sty or midfloat.sty packages (also by Sigitas Tolusis) as IEEE does
% not format its papers in such ways.





% *** PDF, URL AND HYPERLINK PACKAGES ***
%
%\usepackage{url}
% url.sty was written by Donald Arseneau. It provides better support for
% handling and breaking URLs. url.sty is already installed on most LaTeX
% systems. The latest version can be obtained at:
% http://www.ctan.org/tex-archive/macros/latex/contrib/misc/
% Read the url.sty source comments for usage information. Basically,
% \url{my_url_here}.





% *** Do not adjust lengths that control margins, column widths, etc. ***
% *** Do not use packages that alter fonts (such as pslatex).         ***
% There should be no need to do such things with IEEEtran.cls V1.6 and later.
% (Unless specifically asked to do so by the journal or conference you plan
% to submit to, of course. )


% correct bad hyphenation here
\hyphenation{op-tical net-works semi-conduc-tor}

% cc-license
\usepackage{cclicenses}

% A todo macro
\usepackage{xcolor}
\newcommand{\TODO}[1]{{\color{red}[}{\color{red}TODO:} {\color{blue}#1}{\color{red}]}}

% hyperref
\usepackage{hyperref, cite}

% Balanced Columns on Last Page 
\usepackage{flushend}

\IEEEoverridecommandlockouts
\begin{document}
%
% paper title
% can use linebreaks \\ within to get better formatting as desired
\title{Track 1 Paper: Good Usability Practices \\ in Scientific Software Development}


% author names and affiliations
% use a multiple column layout for up to three different
% affiliations
\author{\IEEEauthorblockN{Francisco Queiroz}
\IEEEauthorblockA{Tecgraf Institute, PUC-Rio\\
  Department of Arts \& Design, PUC-Rio\\
  Rio de Janeiro, Brazil}
\\
\IEEEauthorblockN{Sandor Brockhauser}
\IEEEauthorblockA{
  European XFEL GmbH\\
  Schenefeld, Germany}
\and
\IEEEauthorblockN{Raniere Silva}
\IEEEauthorblockA{Software Sustainability Institute UK\\
  School of Computer Science\\
  University of Manchester\\
  Manchester, United Kingdom}
\and
\IEEEauthorblockN{Jonah Miller}
\IEEEauthorblockA{Perimeter Institute of Theoretical Physics\\
  Waterloo, Canada}
\IEEEauthorblockA{University of Guelph\\
  Guelph, Canada}
\\
\IEEEauthorblockN{Hans Fangohr}
\IEEEauthorblockA{University of Southampton\\
  United Kingdom\\
  European XFEL GmbH\\
  Schenefeld, Germany}

\thanks{Licensed under a \href{https://creativecommons.org/licenses/by/4.0/}{CC-BY-4.0 license}. DOI: 10.6084/m9.figshare.5331814.}

}

% conference papers do not typically use \thanks and this command
% is locked out in conference mode. If really needed, such as for
% the acknowledgment of grants, issue a \IEEEoverridecommandlockouts
% after \documentclass

% for over three affiliations, or if they all won't fit within the width
% of the page, use this alternative format:
%
%\author{\IEEEauthorblockN{Michael Shell\IEEEauthorrefmark{1},
%Homer Simpson\IEEEauthorrefmark{2},
%James Kirk\IEEEauthorrefmark{3},
%Montgomery Scott\IEEEauthorrefmark{3} and
%Eldon Tyrell\IEEEauthorrefmark{4}}
%\IEEEauthorblockA{\IEEEauthorrefmark{1}School of Electrical and Computer Engineering\\
%Georgia Institute of Technology,
%Atlanta, Georgia 30332--0250\\ Email: see http://www.michaelshell.org/contact.html}
%\IEEEauthorblockA{\IEEEauthorrefmark{2}Twentieth Century Fox, Springfield, USA\\
%Email: homer@thesimpsons.com}
%\IEEEauthorblockA{\IEEEauthorrefmark{3}Starfleet Academy, San Francisco, California 96678-2391\\
%Telephone: (800) 555--1212, Fax: (888) 555--1212}
%\IEEEauthorblockA{\IEEEauthorrefmark{4}Tyrell Inc., 123 Replicant Street, Los Angeles, California 90210--4321}}




% use for special paper notices
%\IEEEspecialpapernotice{(Invited Paper)}




% make the title area
\maketitle


\begin{abstract}
%\boldmath
Scientific software often presents very particular requirements regarding 
usability, which is often completely overlooked in this setting.
As computational science has emerged as its own discipline, distinct from theoretical and experimental science, it has put new requirements on future scientific software developments. 
In this paper, we discuss the background of these problems and introduce nine 
aspects of good usability. We also highlight best practices for each aspect 
with an emphasis on applications in computational science. 
\end{abstract}
% IEEEtran.cls defaults to using nonbold math in the Abstract.
% This preserves the distinction between vectors and scalars. However,
% if the journal you are submitting to favors bold math in the abstract,
% then you can use LaTeX's standard command \boldmath at the very start
% of the abstract to achieve this. Many IEEE journals frown on math
% in the abstract anyway.

% Note that keywords are not normally used for peerreview papers.
\begin{IEEEkeywords}
Best Practices, Usability, Scientific Software, Computational Science,
Software for Science.
\end{IEEEkeywords}






% For peer review papers, you can put extra information on the cover
% page as needed:
% \ifCLASSOPTIONpeerreview
% \begin{center} \bfseries EDICS Category: 3-BBND \end{center}
% \fi
%
% For peerreview papers, this IEEEtran command inserts a page break and
% creates the second title. It will be ignored for other modes.
\IEEEpeerreviewmaketitle



\section{Introduction}
Scientific software development is a field of growing importance
but lacks a widespread methodology.
Scientists generally have little or no training
in software engineering but tend to be main developers of
computational science codes. They face a number of challenges including:
quickly changing requirements due to the research nature of the work,
competition between maintainable and performance code, and lack of metrics
that would reward investment into sustainable software~\cite{Segal:2007, Kelly:2007}. 
Of particular detriment is the pressure to rapidly produce scientific
publications~\cite{Wilson:2006, Killcoyne:2009}. 
It may be possible to overcome this publication pressure 
when funding agencies are convinced that it is 
worth investing directly in software software for computationally intensive fields. 
The Science and Technology Facilities Council (STFC) in the UK Collaborative 
Computational Projects (\url{http://www.ccp.ac.uk/about.html}) 
sets a good example.


In this work we focus on \emph{usability}, a particular aspect of
software development and design. Usability  is one of the attributes of
sustainable software and can be defined as
``the extent to which a product can be used by specified users to achieve
specified goals with effectiveness, efficiency, and satisfaction in a specified
context of us''~\cite[p.3]{Venters_WSSSPE}. Without proper usability, a software 
cannot be distributed and applied even within its targeted domain.
More importantly, its unusable software can easily result in non-reproducible 
science and the violation of the FAIR principles~\cite{Wilkinson:2016}. 
Unfortunately, usability is often neglected in scientific software development~\cite{Ahmed:2014},
and is of mixed perceived importance to users and developers~\cite{Nguyen-Hoan:2010, Hucka:2016}.
Scientific software usage and development present many challenges for usability 
design that can be related to development models, user-base needs and 
specialization, professional practices, technical constraints, and scientific 
demands~\cite{Queiroz:2016}. Computational science is therefore 
an idiosyncratic field with unique and, occasionally, counterintuitive usability 
requirements. There are, nevertheless, a significant number of informative 
case studies and guidelines on the subject~\cite{MacLeod:1992, Springmeyer:1993, 
Pancake:1996, Javahery:2004, Schraefel:2004,Letondal:2004,Talbott:2005, 
Macaulay:2009, DeRoure:2009, Keefe:2010, DeMatos:2013,Ahmed:2014, Fangohr:2016, 
Beg:2016,List:2017}. Supported by those references and informed by first-hand
experience, we discuss usability challenges and how to address them. 



\section{Good Practices}

\subsection{Think Beyond Graphical User Interfaces}
\label{sec:beyond:GUIs}
Graphical user interfaces (GUIs) have made software user-friendly and 
arguably fosteed the popularization of software in general. However, scientific 
software might require alternatives that, if not more intuitive, are more 
appropriate and efficient depending on the user's needs --- especially if they 
involve entering a large amounts of data, and reading the data from many files, 
or running on a shared or distributed architectures.
Command-Line Interfaces (CLIs) are popular in computational science because they 
often allow for quick repetition of tasks~\cite{bestprSC} and scriptability.
The analysis of large datasets can be significantly easier and more productive 
when done through command-line input than through visual-based interfaces~\cite{Springmeyer:1993}.\footnote{Of course using a CLI does not guarantee ease of use. One must still follow usability best practices when designing the CLI.}
Moreover, GUIs can be extremely cumbersome on distributed infrastructures 
such as supercomputers.
To implement a GUI for distributed code, a graphical frontend must connect via 
network to a distributed backend. Although many scientific visualization tools 
such as VisIt~\cite{HPV:VisIt} and Paraview~\cite{ahrens2005paraview, ayachit2015paraview} 
have implemented this scheme, full rendering via GUI can be impractical and computationally intensive renders are often performed ``headless'' without the GUI~\cite{HPV:VisIt}.
Because of these difficulties, most distributed scientific codes completely 
lack a graphical frontend or separate computation and visualization into separate and 
subsequent stages in the workflow. Similarly, complex experimental protocols 
combining data analysis and scientific instrumentation control can be designed 
with separate User Interaction points in the complete workflow~\cite{Brockhauser:gm5021}.

Even for software where daily use relies on a GUI, such as text
processors and web browsers, there are times when having a CLI for
some tasks is a time saver. For example, users of LaTeX, or the
LibreOffice or Chrome CLIs can convert a text document into PDF format
from the command line.


\subsection{Keep UI Code Separate From Scientific Calculation}

Simulation (or any scientific calculation) should not be embedded in User Interface (UI) code~\cite{Kelly:2009}.
This rule is particularly true for scientific software primarily because, as previously stated, scientific software should be usable via a number of alternative interfaces, such as GUI and CLI. Moreover, it should be possible to access these interfaces both locally and over a network (e.g., via ssh).
Keeping the scientific calculation code wrapped into functions that are called by the UI should
make reconfiguration and customization more convenient~\cite{Bastos:2013},
make porting the functionalities to another UI easier
and
make integration with other software simpler.

% TODO Add example of scientific software that has the UI separated from the rest of the code.

\subsection{Keep the Configuration in a File} \label{sec:beyond:GUIs}

Some tasks requires researchers to provide a long list of parameters to define their computational problem, 
and the software they are using may not provide default values for the parameters, or the default parameters need to be overriden.
In these situations, it
is very handy to have the ability to store some or all of the parameters
in a configuration file that the software can read
at the begining of every execution. 
Alternatively, if a command line tool asks for input parameters from 
the standard input, which requires continuous user interaction, 
it can be modified to be scriptable.
In a script, the configuration parameters are stored next to the 
execution command itself~\cite{Potterton:hv0002}.

Configuration files have the advantage of being \textit{declarative} and
\textit{automatically verifiable}. 
The file defines a \textit{state} which the program will start from or try to achieve, rather than a procedure which leads to that state. Moreover, a parser can automatically check to see if state is valid.
The former is good for reproducibility because it is (ideally) unambiguous even years later~\cite{DBLP:conf/agp/Lloyd94}. 
The latter is good for accuracy because the code can check if the parameters in the file are sensible~\cite{Beg:2016}. 
This state-based approach can also make parallelism easier to automatically reason about~\cite{chakravarty1997massively} and some parallel runtime environments have made use of this property~\cite{Buss:2010:SST:1815695.1815713,Bauer:2012:LEL:2388996.2389086}.

Domain Specific Languages (DSLs), on the other hand,
provide additional flexibility not present in a plain configuration file.
They allow the user to \textit{programmatically} define new behaviour for the code.
This can be a major advantage since it often enables the code to be extended to
unforeseen use-cases without major rewrites.
DSLs can also allow the user to interact with the code at runtime, which can be
helpful for debugging, prototyping, and visualization~\cite{Beg:2016}.
The syntax and rules of the DSL can also provide the same error checking as a
parameter file.

While defining a new domain specific language (DSL) requires
the development of a parser for the language, 
this extra effort can be avoided by
\emph{embedding} the domain specific language in an existing general
purposes language. This has been demonstrated by a number of projects
recently, and Python is a common choice as the general purpose
language. In this context, the domain specific language is given
through a Python module that the user imports into their generic
Python program, and which provides commands, objects and operations
that are specific to the domain in question. The Python program 
then becomes the (very flexible) configuration file for the computational problem.

However, some care is required when designing such DSLs: the elements
of the DSL must be constructed so that users cannot combine them in
ways that would take the tool outside its range of applicability. This
could be achieved through explicit assert statements in the DSL's
implementation or appropriate (often Object Oriented) design. If the
code author lacks the experience or time available to achieve this, it
is important to document the assumptions made for use of the DSL so
that it is not used incorrectly inadvertently by others in the future. For 
example, the yt project~\cite{2011ApJS..192....9T} is a DSL for scientific 
visualization and data analysis built in Python. If the data fed into yt 
doesn't satisfy the correct assumptions, yt could produce spurious 
visualization artifacts or incorrectly integrate a quantity over the domain. 
Therefore the authors of yt take extreme care to document their API, 
sanitize their inputs and throw informative error messages when incorrect 
data is fed into the tool. A major part of this process is unit testing the 
DSL's functionality

%On the other hand, the flexibility allowed by a DSL may make it possible for the code to behave in unforeseen ways.
%And without careful version control of programs written in the DSL, DSLs can be bad for reproducibility.
Some codes combine both plain configuration files \textit{and} DSLs.
For example, the Einstein Toolkit~\cite{Loffler:2011ay},\footnote{For which one 
of us is a developer.} a code for relativistic astrophysics, uses configuration 
files for day-to-day simulations. However, it also provides a low-level DSL, 
Kranc~\cite{Husa:2004ip}, for defining systems of equations to solve.

\subsection{Design for Small, Incremental Changes}

Making incremental changes is considered a best practice for scientific software 
development~\cite{bestprSC}, and the same principle applies to user interfaces. 
Ideally, \emph{UIs should be planned for extensibility and frequent changes} as 
new requisites emerge. Through incremental changes, software is more likely to 
\emph{stay attuned to users’ needs}, not forcing them to radically change the 
way they work~\cite{DeRoure:2009}. 

Regarding constant updates and addition of 
new functionalities, UI components that can be easily extended might offer 
interesting solutions. This is the case for map3D, a scientific 
visualization software for displaying and editing three-dimensional models and
associated data~\cite{SCI:Map3d}. During development, pop-up menus were
implemented for providing the necessary flexibility, allowing 
developers to add new commands and submenus as the software development 
and requisites evolved~\cite{MacLeod:1992}. It is worth mentioning that
web-based applications might take advantage of the modularity allowed by 
frontend design methodologies such as Atomic Design~\cite{Atomic}, making
it easier to configure user interfaces as the project advances.


The parameter files and DSLs described in section~\ref{sec:beyond:GUIs}
are particularly good for satisfying this design
constraint. For example, the Einstein Toolkit~\cite{Loffler:2011ay}
packages low-level code in modules. Each module must declare which
functionality it adds, which relevant parameters can be set in the
parameter file, and how these parameters depend on other modules. The parameter
file parser then automatically adds these options to the parameter
file at compile time. The yt project~\cite{2011ApJS..192....9T}
provides a DSL for scientific visualization. This DSL interacts with
the low-level code only through function calls and so functionality
can easily be incrementally added by the introduction of new DSL
language features or functions.

\subsection{Facilitate and Register User Activity and Environment}
There are a number of ways through which usability can be enhanced based on \emph{past} user activity.
First, providing access to a list of recent commands and allowing
users to re-execute them can help users save time.
This is a major reason for the popularity of command-line interfaces~\cite{bestprSC}.
A very popular implementation of this concept is the ability to access previously typed
commands by pressing the up arrow key or do a reverse search on the history of executed commands.
Users can also press the right and left arrow keys to navigate through a previous command
and edit it to suit their needs.
Second, it might be a good idea to give users quick access
to frequently used commands~\cite{Julvez:2014}.
In some environments, the tab key is used to roll among frequent used commands
or to auto complete a command.
Third, logging user activity
might help users identify and support research reproducibility~\cite{List:2017}
by exporting the history to a file.

After registering user activity,
developers can go further and log the user environment,
i.e. compiled binary, configuration files, input files and output files,
used when running the program.
This is useful in scientific software since the output of any experiment can be different because of different implementations (or compiler optimizations) of
the Basic Linear Algebra Subprograms (BLAS), LAPACK (Linear Algebra Package) or
any other library used when performing the experiment.
This automatic logging not only provides users
quick access to their exact configuration for debugging purposes
but also allows the computation to be reproduced years after it was run for the first time.
One example framework is Formaline~\cite{formaline}.
For software developers working with Python, we
mention the related packages ReciPy~\cite{ReciPy} and Sumatra~\cite{sumatra2014}.

\subsection{Learn About How Users Work}

Guidelines and case studies often recommend the adoption of a user-centered 
design process that seeks to develop a firm understanding of how scientists do 
their work before developing a piece of software.
This understanding can be acquired by learning the meanderings of scientific work~\cite{Springmeyer:1993}, 
or through a participatory design approach in which users are actively involved in 
the design process~\cite{Letondal:2004,Aragon:2008, Thomer:2016,  Luna2017204}.
It is also important to \emph{analyze the scientific work within the 
environment where it actually takes place}~\cite{Pancake:1996} and evaluate 
existing tools which are already in use~\cite{Javahery:2004}. In this last case
it might be advantageous to adopt preexisting industry standards (e.g.: keyboard
shortcuts for common functionalities, iconography, etc.).  

When designing user interfaces for scientific software, it is a good idea to 
\emph{address specific users or user-bases} rather than aim for a general 
solution~\cite {Javahery:2004, DeRoure:2009}. Ideally, GUIs should be open to 
user customization and adjustable to personal preferences and professional 
specialization~\cite{Gertz:1994, Javahery:2004}. However, users should not be
overwhelmed by an excessive number of customizable parameters --- some 
of which can be unimportant or meaningless to their specific case.
Instead, there should be an additional section for setting advanced 
parameters~\cite{List:2017}. 

As a user base grows, users may have suggestions for improving the UI
or the underlying scientific code. If the code is open source, it can
be extremely advantageous to transform these \textit{users} into
\textit{developers} so that they can bring their user experience and
domain expertise to bear~\cite{Turk:2013:SCH:2484762.2484782}.
Additionally, it is advisable that scientific domain experts are brought into 
the design process for informing \emph{domain best 
practices}~\cite{Schraefel:2004,  DeMatos:2013} and evaluating the 
tool~\cite{Keefe:2010}.

\subsection{Be Minimalistic, but Look Out for Exceptional Needs}

Designers should be attentive to information that is particularly relevant in
scientific software, but that could be eluded otherwise. \emph{Metadata},
for instance, is often required to be readable and easy to
access~\cite{Talbott:2005, Baxter:2006, Macaulay:2009, Keefe:2010, bestprSC, 
Thomer:2016}. 

Also, despite recent trends
favoring flat design over skeuomorphism (i.e.: visual design that imitates the 
appearance of real-world objects), software versions of physical
instruments might benefit from adopting the looks of their real-world
counterparts~\cite{Foster:1998}, making it easier for users to recognize and 
learn about their functioning. An example for that approach is LabViEW's 
set of GUI components mimicking dials, knobs and meters~\cite{LabVIEW}.

Finally, minimalism should \emph{emphasize, rather than
conceal, critical information} such as system malfunctioning~\cite{Morais:2014},
emergency information~\cite{Ferguson:2016}, and situations where awareness and
response under time pressure are essential. For instance, the
Sky software for astronomical visualization reduces users' cognitive load by 
simplifying three-dimensional visualization data as a two-dimensional 
projection~\cite{Aragon:2008}. 


\subsection{Design for Precision}
%The importance of correctness and precision in science demands software that 
%allows for accuracy.
%The first safety net for uses and developers is to have tests and a continuous 
%integration pipeline.
% TODO citation need for the previous statement

In order to achieve satisfying results, scientific work often demands precision
regarding user's input. A possible means for that would be continuously 
constraining users' input and providing feedback on it. The Dynamic Dragging
Interface, for instance, makes user of force-feedback input devices to 
help users selecting sections of 3D brain visualization ~\cite{Keefe:2010}.
Another solution would be not accepting the input when 
the input device, such as a stylus or mouse, moves too fast. 

It can be advantageous to have two input modes for the same action ---
one designed for accuracy,
and another for speed. In map3D, for instance, geometric models could be moved, 
rotated and scaled through dial boxes (accurate, but slower) or, alternatively, 
via mouse (less accurate, but faster)~\cite{MacLeod:1992}. Another approach 
would be to give users a way to switch between fast and precise working modes.
For instance, by activating a 'snap' mode where the mouse cursor snaps to a gridline,
objects or other elements on screen. 

% TODO citation or example need for the previous statement

\subsection{Contextualize User Actions}
Work in scientific software can involve a number of different and/or sequential
tasks to be performed by users. In those cases, it is desirable to contextualize
users' actions, facilitating their access to functions that are relevant to their 
current tasks and preventing their access to functions that are not.

The Petri Net Toolbox for MATLAB, for instance, features a button that toggles 
between \emph{Draw Mode}, for creating and editing models, and \emph{Explore Mode},
for simulation and analysis~\cite{Julvez:2014}. When switching between modes, 
GUI elements are displayed or hidden depending on their relevance to the 
selected mode. This approach is known as a design pattern 
named  \emph{Disabled Irrelevant Things}~\cite{Zeeshan:2011}. 

Another possible approach is the \emph{Window Per Task}~\cite{Zeeshan:2011} design pattern, in which tasks are distributed
across individual screens containing the appropriate commands for that task only. Pharmaceutical
biology software Lipid-Pro makes use of this design pattern by organizing its tasks into separate
panels~\cite{Ahmed:2014}. 



% needed in second column of first page if using \IEEEpubid
%\IEEEpubidadjcol

% An example of a floating figure using the graphicx package.
% Note that \label must occur AFTER (or within) \caption.
% For figures, \caption should occur after the \includegraphics.
% Note that IEEEtran v1.7 and later has special internal code that
% is designed to preserve the operation of \label within \caption
% even when the captionsoff option is in effect. However, because
% of issues like this, it may be the safest practice to put all your
% \label just after \caption rather than within \caption{}.
%
% Reminder: the "draftcls" or "draftclsnofoot", not "draft", class
% option should be used if it is desired that the figures are to be
% displayed while in draft mode.
%
%\begin{figure}[!t]
%\centering
%\includegraphics[width=2.5in]{myfigure}
% where an .eps filename suffix will be assumed under latex,
% and a .pdf suffix will be assumed for pdflatex; or what has been declared
% via \DeclareGraphicsExtensions.
%\caption{Simulation Results}
%\label{fig_sim}
%\end{figure}

% Note that IEEE typically puts floats only at the top, even when this
% results in a large percentage of a column being occupied by floats.


% An example of a double column floating figure using two subfigures.
% (The subfig.sty package must be loaded for this to work.)
% The subfigure \label commands are set within each subfloat command, the
% \label for the overall figure must come after \caption.
% \hfil must be used as a separator to get equal spacing.
% The subfigure.sty package works much the same way, except \subfigure is
% used instead of \subfloat.
%
%\begin{figure*}[!t]
%\centerline{\subfloat[Case I]\includegraphics[width=2.5in]{subfigcase1}%
%\label{fig_first_case}}
%\hfil
%\subfloat[Case II]{\includegraphics[width=2.5in]{subfigcase2}%
%\label{fig_second_case}}}
%\caption{Simulation results}
%\label{fig_sim}
%\end{figure*}
%
% Note that often IEEE papers with subfigures do not employ subfigure
% captions (using the optional argument to \subfloat), but instead will
% reference/describe all of them (a), (b), etc., within the main caption.


% An example of a floating table. Note that, for IEEE style tables, the
% \caption command should come BEFORE the table. Table text will default to
% \footnotesize as IEEE normally uses this smaller font for tables.
% The \label must come after \caption as always.
%
%\begin{table}[!t]
%% increase table row spacing, adjust to taste
%\renewcommand{\arraystretch}{1.3}
% if using array.sty, it might be a good idea to tweak the value of
% \extrarowheight as needed to properly center the text within the cells
%\caption{An Example of a Table}
%\label{table_example}
%\centering
%% Some packages, such as MDW tools, offer better commands for making tables
%% than the plain LaTeX2e tabular which is used here.
%\begin{tabular}{|c||c|}
%\hline
%One & Two\\
%\hline
%Three & Four\\
%\hline
%\end{tabular}
%\end{table}


% Note that IEEE does not put floats in the very first column - or typically
% anywhere on the first page for that matter. Also, in-text middle ("here")
% positioning is not used. Most IEEE journals use top floats exclusively.
% Note that, LaTeX2e, unlike IEEE journals, places footnotes above bottom
% floats. This can be corrected via the \fnbelowfloat command of the
% stfloats package.



\section{Summary}
Throughout the previous section, we have presented a nonexhaustive set of good 
practices in usability for scientific software, taking in consideration 
challenging aspects of scientific software development and use such as the lack 
of attention to software engineering; the need for reproducibility; the handling 
of large amounts of data; the complexity of actions and parameters involved in 
scientific work; frequent changes in requirements; particularities of scientific 
work and its environment; the need for accessing and responding to critical 
information; and the importance of precision.

By adopting the presented practices, developers should be
able to deliver applications that are more usable, robust and more appropriate 
for scientific work.




% if have a single appendix:
%\appendix[Proof of the Zonklar Equations]
% or
%\appendix  % for no appendix heading
% do not use \section anymore after \appendix, only \section*
% is possibly needed

% use appendices with more than one appendix
% then use \section to start each appendix
% you must declare a \section before using any
% \subsection or using \label (\appendices by itself
% starts a section numbered zero.)
%



% use section* for acknowledgement
\section*{Acknowledgment}

This project began during the Fourth Annual Workshop on Sustainable
Software for Science: Practices and Experiences (WSSSPE4), where four of the 
authors established the Working Group on Software best practices for 
undergraduates~\cite{WSSSPE4Report}. The authors would therefore like to thank the
organizers of WSSSPE4 for facilitating this conversation. J. Miller
and F. Queiroz would like to thank travel grants from the 
National Science Foundation of the USA and The  Gordon  and  Betty  Moore  
Foundation, which made attendance possible. J. Miller also acknowledges 
support from the Natural Sciences and Engineering Research Council of 
Canada and from the National Science Foundation of the USA (OCI 0905046, 
PHY 1212401). Research at Perimeter Institute is supported by the Government of
Canada through the Department of Innovation, Science and Economic
Development and by the Province of Ontario through the Ministry of
Research and Innovation. F. Queiroz acknowledges support from PUC-Rio and 
Tecgraf Institute.
H. Fangohr and R. Silva acknowledge support from the Software Sustainability Institute
and Engineering and Physical Sciences Research Council (EPSRC) in the UK.

% cclicense
%\section*{Creative Commons}
%This work is licensed under a
%\href{https://creativecommons.org/licenses/by/4.0/}{Creative
 % Commons Attribution 4.0 International License.}\\
 %\cc \by


% Can use something like this to put references on a page
% by themselves when using endfloat and the captionsoff option.
\ifCLASSOPTIONcaptionsoff
  \newpage
\fi



% trigger a \newpage just before the given reference
% number - used to balance the columns on the last page
% adjust value as needed - may need to be readjusted if
% the document is modified later
%\IEEEtriggeratref{8}
% The "triggered" command can be changed if desired:
%\IEEEtriggercmd{\enlargethispage{-5in}}

% references section

% can use a bibliography generated by BibTeX as a .bbl file
% BibTeX documentation can be easily obtained at:
% http://www.ctan.org/tex-archive/biblio/bibtex/contrib/doc/
% The IEEEtran BibTeX style support page is at:
% http://www.michaelshell.org/tex/ieeetran/bibtex/
%\bibliographystyle{IEEEtran}
% argument is your BibTeX string definitions and bibliography database(s)
%\bibliography{IEEEabrv,../bib/paper}
%
% <OR> manually copy in the resultant .bbl file
% set second argument of \begin to the number of references
% (used to reserve space for the reference number labels box)
\bibliographystyle{IEEEtran}

\bibliography{main}
% biography section
%
% If you have an EPS/PDF photo (graphicx package needed) extra braces are
% needed around the contents of the optional argument to biography to prevent
% the LaTeX parser from getting confused when it sees the complicated
% \includegraphics command within an optional argument. (You could create
% your own custom macro containing the \includegraphics command to make things
% simpler here.)
%\begin{biography}[{\includegraphics[width=1in,height=1.25in,clip,keepaspectratio]{mshell}}]{Michael Shell}
% or if you just want to reserve a space for a photo:

%\begin{IEEEbiography}[{\includegraphics[width=1in,height=1.25in,clip,keepaspectratio]{picture}}]{John Doe}
%\blindtext
%\end{IEEEbiography}

% You can push biographies down or up by placing
% a \vfill before or after them. The appropriate
% use of \vfill depends on what kind of text is
% on the last page and whether or not the columns
% are being equalized.

%\vfill

% Can be used to pull up biographies so that the bottom of the last one
% is flush with the other column.
%\enlargethispage{-5in}


% that's all folks
\end{document}
