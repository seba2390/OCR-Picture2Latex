

%% bare_conf.tex
%% V1.1
%% 2002/08/13u
%% by Michael Shell
%% mshell@ece.gatech.edu
%%
%% NOTE: This text file uses UNIX line feed conventions. When (human)
%% reading this file on other platforms, you may have to use a text
%% editor that can handle lines terminated by the UNIX line feed
%% character (0x0A).
%%
%% This is a skeleton file demonstrating the use of IEEEtran.cls
%% (requires IEEEtran.cls version 1.6 or later) with an IEEE conference paper.
%%
%% Support sites:
%% http://www.ieee.org
%% and/or
%% http://www.ctan.org/tex-archive/macros/latex/contrib/supported/IEEEtran/
%%
%% This code is offered as-is - no warranty - user assumes all risk.
%% Free to use, distribute and modify.

% *** Authors should verify (and, if needed, correct) their LaTeX system  ***
% *** with the testflow diagnostic prior to trusting their LaTeX platform ***
% *** with production work. IEEE's font choices can trigger bugs that do  ***
% *** not appear when using other class files.                            ***
% Testflow can be obtained at:
% http://www.ctan.org/tex-archive/macros/latex/contrib/supported/IEEEtran/testflow


% Note that the a4paper option is mainly intended so that authors in
% countries using A4 can easily print to A4 and see how their papers will
% look in print. Authors are encouraged to use U.S. letter paper when
% submitting to IEEE. Use the testflow package mentioned above to verify
% correct handling of both paper sizes by the author's LaTeX system.
%
% Also note that the "draftcls", not "draft", option should be used if
% it is desired that the figures are to be displayed in draft mode.
%\documentclass[12pt,journal,onecolumn,draftcls]{IEEEtran}
%\documentclass[10pt,onecolumn, conference]{IEEEtran}

\documentclass[11pt,letterpaper]{article}
%\textwidth=6.6in \textheight=9in \topmargin=-0.5in \oddsidemargin=0in
%\usepackage{times}
\usepackage{aaai21}

\usepackage{algorithm}
\usepackage{algorithmic}
\usepackage{booktabs}
\usepackage{textcomp}
\usepackage{tabularx,booktabs}
\usepackage{latexsym}
\usepackage{times}
\usepackage{amssymb}
\usepackage{amsmath}
\usepackage{amsfonts}
\usepackage{graphicx, acronym}
\usepackage{booktabs}
\usepackage{textcomp}
\usepackage{tabularx,booktabs}
\usepackage[normalem]{ulem}
\usepackage{epsfig}
\usepackage{algorithm}
\usepackage{algorithmic}
\usepackage{cite}
%\usepackage{psfig}
\usepackage{multicol}
\usepackage{mathrsfs}
\usepackage{wrapfig}
\usepackage[tight,footnotesize]{subfigure}
\evensidemargin=0in
%\textwidth=6.8in
%\textheight=9.6in \topmargin=-0.85in \oddsidemargin=-0.1in
%\textwidth=6.5in
%\textheight=9.0in \topmargin=-0.60in \oddsidemargin=-0.0in
\textwidth=6.6in
\textheight=9.5in \topmargin=-0.85in \oddsidemargin=-0.09in
%\baselineskip=4.0mm
\usepackage{multirow}
\usepackage{url}
\usepackage{amsmath,amssymb,amsthm,xspace,bm}

\usepackage{mathtools}
\DeclarePairedDelimiter\abs{\lvert}{\rvert}%
\DeclarePairedDelimiter\norm{\lVert}{\rVert}%

\newtheorem{prop}{Proposition}
\makeatletter
\let\qnoise\@undefined
\makeatother
% Math symbol definitions
\newcommand{\csobssymb}{z}
\newcommand{\csobssignal}[1][]{\ensuremath{\mathbf\csobssymb_{#1}}\xspace}
\newcommand{\csmeassymb}{y}
\newcommand{\csmeas}[1][]{\ensuremath{\mathbf\csmeassymb_{#1}}\xspace}
\newcommand{\cssparsevecsymb}{x}

\newcommand{\csksparsesymb}{d}
\newcommand{\csksparse}[1][]{\ensuremath{\mathbf \csksparsesymb_{#1}}\xspace}

\newcommand{\csidenmtxsymb}{I}
\newcommand{\csidenmtx}[1][]{\ensuremath{\mathbf \csidenmtxsymb_{#1}}\xspace}

\newcommand{\csidenerrmtxsymb}{I_e}
\newcommand{\csidenerrmtx}[1][]{\ensuremath{\mathbf \csidenerrmtxsymb_{#1}}\xspace}

\newcommand{\cssparsevec}[1][]{\ensuremath{\mathbf \cssparsevecsymb_{#1}}\xspace}
\newcommand{\cssparsevecvarsymb}{u}
\newcommand{\cssparsevecvar}[1][]{\ensuremath{\mathbf \cssparsevecvarsymb_{#1}}\xspace}
\newcommand{\cssparsevecvaraltsymb}{v}
\newcommand{\cssparsevecvaralt}[1][]{\ensuremath{\mathbf \cssparsevecvaraltsymb_{#1}}\xspace}
\newcommand{\cssparsevecest}[1][]{\ensuremath{\hat{\mathbf \cssparsevecsymb}_{#1}}\xspace}
\newcommand{\cssparsevecshift}[1][]{\ensuremath{\tilde{\mathbf \cssparsevecsymb}_{#1}}\xspace}
\newcommand{\csdict}[1][]{\ensuremath{\bm\Psi_{#1}}\xspace}
\newcommand{\csmeasmtx}[1][]{\ensuremath{\bm\Phi_{#1}}\xspace}
\newcommand{\cssysmtx}[1][]{\ensuremath{\mathbf A_{#1}}\xspace}
\newcommand{\circshift}[3]{\ensuremath{#1_{\circlearrowright(#2,#3)}}}
\newcommand{\circshiftvec}[2]{\ensuremath{#1_{\circlearrowright#2}}}
\newcommand{\diag}{\operatorname{diag}}

\newcommand{\cshmtx}[1][]{\ensuremath{\mathbf H_{#1}}\xspace}
\newcommand{\cserrhmtx}[1][]{\ensuremath{\mathbf{\tilde H}_{#1}}\xspace}
\newcommand{\cshmtxerr}[1][]{\ensuremath{\mathbf E_{#1}}\xspace}
\newcommand{\csdmtx}[1][]{\ensuremath{\mathbf D_{#1}}\xspace}
\newcommand{\cserrmeas}[1][]{\ensuremath{\mathbf{\tilde y}_{#1}}\xspace}
\newcommand{\cserrmeasmtx}[1][]{\ensuremath{\bm{\tilde \Phi}_{#1}}\xspace}

\newcommand{\csherrvec}[1][]{\ensuremath{\mathbf e_{#1}}\xspace}
\newcommand{\cssigmtx}[1][]{\ensuremath{\mathbf Z'_{#1}}\xspace}

\newcommand{\noisesymb}{n}
\newcommand{\noise}[1][]{\ensuremath{\mathbf \noisesymb_{#1}}\xspace}
\newcommand{\qnoisesymb}{q}
\newcommand{\qnoise}[1][]{\ensuremath{\mathbf \qnoisesymb_{#1}}\xspace}
\newcommand{\rnoisesymb}{r}
\newcommand{\rnoise}[1][]{\ensuremath{\mathbf \rnoisesymb_{#1}}\xspace}

\DeclareMathOperator{\E}{\mathbb E}
\makeatletter
\newcommand{\expectation}[1]{%
  \@ifnextchar[{\expectation@i{#1}}{\expectation@j{#1}}%]
}
\def\expectation@i#1[#2]{%
  \ensuremath{\E_{#2}\left[#1\right]}\xspace%
}
\def\expectation@j#1{%
  \ensuremath{\E\left[#1\right]}\xspace%
}
\makeatother
\newcommand{\transpose}{\ensuremath{^\mathrm{T}}}

\newcommand{\csuncorqnsymb}{w}
\newcommand{\csuncorqn}[1][]{\ensuremath{\mathbf \csuncorqnsymb_{#1}}\xspace}
\newcommand{\bpdnsolset}[1]{\ensuremath{\uppercase{#1}}\xspace}
\newcommand{\argmin}{\operatornamewithlimits{argmin}}
\newcommand{\Argmin}{\operatornamewithlimits{Argmin}}

\newcommand{\bq}{\begin{equation}}
\newcommand{\eq}{\end{equation}\vspace{0.02in}}
\newcommand{\bn}{\begin{eqnarray}}
\newcommand{\en}{\end{eqnarray}}
\newcommand{\bnn}{\begin{eqnarray*}}
\newcommand{\enn}{\end{eqnarray*}}
\newcommand{\bp}{\begin{picture}}
\newcommand{\ep}{\end{picture}}
\newcommand{\lb}{\label}
\newcommand{\f}{\frac}
\newcommand{\al}{\alpha}
\newcommand{\lf}{\left}
\newcommand{\rt}{\right}
\newcommand{\non}{\nonumber}
\newenvironment{syntax}{\begin{quotation}\begin{sf}\begin{tabular}{lcll}}{\end{tabular}\end{sf}\end{quotation}}

\usepackage[usenames,dvipsnames]{xcolor}
%\parindent=0pt     %Nilanka change
\newcommand{\hilight}[1]{\colorbox{green}{#1}}
\newcommand{\hilightr}[1]{\colorbox{red}{#1}}
\newcommand{\hilightb}[1]{\colorbox{blue}{#1}}
\newcommand{\hilighto}[1]{\colorbox{magenta}{#1}}
\newcommand{\hilighty}[1]{\colorbox{yellow}{#1}}
\newcommand{\hilightc}[1]{\colorbox{cyan}{#1}}
\newcommand{\hilightu}[1]{\colorbox{Green}{#1}}

\usepackage{graphicx, epsfig, cite, acronym, amssymb, algorithm, algorithmic}
\usepackage[normalem]{ulem}
\newcommand{\rev}[1]{\textcolor{blue}{\uline{#1}}}  %revise the text
%\newcommand{\note}[1]{{\sffamily\itshape\bfseries\uline{#1}}}

%\newcommand{\rev}[1]{\uwave{#1}}  %revise the text

%\newcommand{\rev}[1]{#1}

%\newcommand{\del}[1]{\sout{#1}}  %revise the text

\newcommand{\del}[1]{}

\newcommand{\note}[1]{{\sffamily\itshape\bfseries\uline{#1}}}




% If the IEEEtran.cls has not been installed into the LaTeX system files,
% manually specify the path to it:
% \documentclass[conference]{../sty/IEEEtran}


% some very useful LaTeX packages include:

%\usepackage{cite}      % Written by Donald Arseneau
                        % V1.6 and later of IEEEtran pre-defines the format
                        % of the cite.sty package \cite{} output to follow
                        % that of IEEE. Loading the cite package will
                        % result in citation numbers being automatically
                        % sorted and properly "ranged". i.e.,
                        % [1], [9], [2], [7], [5], [6]
                        % (without using cite.sty)
                        % will become:
                        % [1], [2], [5]--[7], [9] (using cite.sty)
                        % cite.sty's \cite will automatically add leading
                        % space, if needed. Use cite.sty's noadjust option
                        % (cite.sty V3.8 and later) if you want to turn this
                        % off. cite.sty is already installed on most LaTeX
                        % systems. The latest version can be obtained at:
                        % http://www.ctan.org/tex-archive/macros/latex/contrib/supported/cite/

\usepackage{graphicx}  % Written by David Carlisle and Sebastian Rahtz
                        % Required if you want graphics, photos, etc.
                        % graphicx.sty is already installed on most LaTeX
                        % systems. The latest version and documentation can
                        % be obtained at:
                        % http://www.ctan.org/tex-archive/macros/latex/required/graphics/
                        % Another good source of documentation is "Using
                        % Imported Graphics in LaTeX2e" by Keith Reckdahl
                        % which can be found as esplatex.ps and epslatex.pdf
                        % at: http://www.ctan.org/tex-archive/info/

%\usepackage{psfrag}    % Written by Craig Barratt, Michael C. Grant,
                        % and David Carlisle
                        % This package allows you to substitute LaTeX
                        % commands for text in imported EPS graphic files.
                        % In this way, LaTeX symbols can be placed into
                        % graphics that have been generated by other
                        % applications. You must use latex->dvips->ps2pdf
                        % workflow (not direct pdf output from pdflatex) if
                        % you wish to use this capability because it works
                        % via some PostScript tricks. Alternatively, the
                        % graphics could be processed as separate files via
                        % psfrag and dvips, then converted to PDF for
                        % inclusion in the main file which uses pdflatex.
                        % Docs are in "The PSfrag System" by Michael C. Grant
                        % and David Carlisle. There is also some information
                        % about using psfrag in "Using Imported Graphics in
                        % LaTeX2e" by Keith Reckdahl which documents the
                        % graphicx package (see above). The psfrag package
                        % and documentation can be obtained at:
                        % http://www.ctan.org/tex-archive/macros/latex/contrib/supported/psfrag/

%\usepackage{subfigure} % Written by Steven Douglas Cochran
                        % This package makes it easy to put subfigures
                        % in your figures. i.e. "figure 1a and 1b"
                        % Docs are in "Using Imported Graphics in LaTeX2e"
                        % by Keith Reckdahl which also documents the graphicx
                        % package (see above). subfigure.sty is already
                        % installed on most LaTeX systems. The latest version
                        % and documentation can be obtained at:
                        % http://www.ctan.org/tex-archive/macros/latex/contrib/supported/subfigure/

%\usepackage{url}       % Written by Donald Arseneau
                        % Provides better support for handling and breaking
                        % URLs. url.sty is already installed on most LaTeX
                        % systems. The latest version can be obtained at:
                        % http://www.ctan.org/tex-archive/macros/latex/contrib/other/misc/
                        % Read the url.sty source comments for usage information.

%\usepackage{stfloats}  % Written by Sigitas Tolusis
                        % Gives LaTeX2e the ability to do double column
                        % floats at the bottom of the page as well as the top.
                        % (e.g., "\begin{figure*}[!b]" is not normally
                        % possible in LaTeX2e). This is an invasive package
                        % which rewrites many portions of the LaTeX2e output
                        % routines. It may not work with other packages that
                        % modify the LaTeX2e output routine and/or with other
                        % versions of LaTeX. The latest version and
                        % documentation can be obtained at:
                        % http://www.ctan.org/tex-archive/macros/latex/contrib/supported/sttools/
                        % Documentation is contained in the stfloats.sty
                        % comments as well as in the presfull.pdf file.
                        % Do not use the stfloats baselinefloat ability as
                        % IEEE does not allow \baselineskip to stretch.
                        % Authors submitting work to the IEEE should note
                        % that IEEE rarely uses double column equations and
                        % that authors should try to avoid such use.
                        % Do not be tempted to use the cuted.sty or
                        % midfloat.sty package (by the same author) as IEEE
                        % does not format its papers in such ways.

%\usepackage{amsmath}   % From the American Mathematical Society
                        % A popular package that provides many helpful commands
                        % for dealing with mathematics. Note that the AMSmath
                        % package sets \interdisplaylinepenalty to 10000 thus
                        % preventing page breaks from occurring within multiline
                        % equations. Use:
%\interdisplaylinepenalty=2500
                        % after loading amsmath to restore such page breaks
                        % as IEEEtran.cls normally does. amsmath.sty is already
                        % installed on most LaTeX systems. The latest version
                        % and documentation can be obtained at:
                        % http://www.ctan.org/tex-archive/macros/latex/required/amslatex/math/



% Other popular packages for formatting tables and equations include:

% Frank Mittelbach's and David Carlisle's array.sty which improves the
% LaTeX2e array and tabular environments to provide better appearances and
% additional user controls. Array.sty is already installed on most systems.
% The latest version and documentation can be obtained at:
% http://www.ctan.org/tex-archive/macros/latex/required/tools/

% Mark Wooding's extremely powerful MDW tools, especially mdwmath.sty and
% mdwtab.sty which are used to format equations and tables, respectively.
% The MDWtools set is already installed on most LaTeX systems. The lastest
% version and documentation is available at:
% http://www.ctan.org/tex-archive/macros/latex/contrib/supported/mdwtools/

% V1.6 of IEEEtran contains the IEEEeqnarray family of commands that can
% be used to generate multiline equations as well as matrices, tables, etc.


% Also of notable interest:

% Scott Pakin's eqparbox package for creating (automatically sized) equal
% width boxes. Available:
% http://www.ctan.org/tex-archive/macros/latex/contrib/supported/eqparbox/




% *** Do not adjust lengths that control margins, column widths, etc. ***
% *** Do not use packages that alter fonts (such as pslatex).         ***
% There should be no need to do such things with IEEEtran.cls V1.6 and later.


% correct bad hyphenation here
\hyphenation{op-tical net-works semi-conduc-tor IEEEtran}

\begin{document}
\bibliographystyle{IEEE}

% paper title
%\title{Coordinated Resource Management for CDMA-based Radio Access Networks}

%\author{\authorblockN{Sneha K. Kasera, Ramachandran Ramjee, Sandra Thuel and Xin Wang}
%\authorblockA{Bell Laboratories\\
%Lucent Technologies\\
%Holmdel, New Jersey 07733\\
%Email: \{kasera, ramjee, thuel, xwang\}@bell-labs.com}}

%\author{\authorblockN{Sneha K. Kasera}\\
%\authorblockA{School of Computing\\
%University of Utah\\
%Salt Lake City, Utah 84112\\
%Email: kasera@cs.utah.edu}\\
%\and
%\authorblockN{Ramachandran Ramjee, Sandra Thuel}\\
%\authorblockA{Bell Laboratories\\
%Lucent Technologies\\
%Holmdel, New Jersy 07733\\
%Email: \{ramjee, thuel\}@bell-labs.com}\\
%\and
%\authorblockN{Xin Wang}\\
%\authorblockA{Computer Science and Engineering\\
%SUNY at Buffalo\\
%Buffalo, NY 14260\\
%Email: xwang8@cse.buffalo.edu}}

% author names and affiliations
% use a multiple column layout for up to three different
% affiliations
%\author{\authorblockN{Michael Shell}
%\authorblockA{School of Electrical and\\Computer Engineering\\
%Georgia Institute of Technology\\
%Atlanta, Georgia 30332--0250\\
%Email: mshell@ece.gatech.edu}
%\and
%\authorblockN{Homer Simpson}
%\authorblockA{Twentieth Century Fox\\
%Springfield, USA\\
%Email: homer@thesimpsons.com}
%\and
%\authorblockN{James Kirk\\ and Montgomery Scott}
%\authorblockA{Starfleet Academy\\
%San Francisco, California 96678-2391\\
%Telephone: (800) 555--1212\\
%Fax: (888) 555--1212}}
%

% avoiding spaces at the end of the author lines is not a problem with
% conference papers because we don't use \thanks or \IEEEmembership


% for over three affiliations, or if they all won't fit within the width
% of the page, use this alternative format:
%
%\author{\authorblockN{Sneha Kumar Kasera\authorrefmark{1},
%Ramachandran Ramjee\authorrefmark{2},
%Sandra Thuel\authorrefmark{2} and
%Xin Wang\authorrefmark{3}}
%\authorblockA{\authorrefmark{1}School of Computing\\
%University of Utah,
%Salt Lake City, UT 84112\\ Email: kasera@cs.utah.edu}
%\authorblockA{\authorrefmark{2}Bell Laboratories\\
%Lucent Technologies\\
%Homdel, NJ 07733\\ Email: \{ramjee, thuel\}@bell-labs.com}
%\authorblockA{\authorrefmark{3}Computer Science and Engineering\\
%SUNY at Buffalo\\
%Buffalo, NY 14260\\
%Email: xwang8@cse.buffalo.edu}}

%\author{
%{\small
%\begin{tabular}[t]{ccc}
%Sneha K. Kasera & Ramachandran Ramjee, Sandra Thuel & Xin Wang\\
%School of Computing & Bell Laboratories & Computer Science and Engineering\\
%University of Utah & Lucent Technologies & SUNY at Buffalo\\
%Salt Lake City, UT 84112 & Holmdel, NJ 07733 & NY 14260\\
%Email: kasera@cs.utah.edu & \{ramjee, thuel\}@bell-labs.com & xwang8@cse.buffalo.edu
%\end{tabular}
%}
%}

%Telephone: (800) 555--1212, Fax: (888) 555--1212}
%\authorblockA{\authorrefmark{4}Tyrell Inc., 123 Replicant Street, Los Angeles, California 90210--4321}}
%\author{\authorblockN{Michael Shell\authorrefmark{1},
%Homer Simpson\authorrefmark{2},
%James Kirk\authorrefmark{3},
%Montgomery Scott\authorrefmark{3} and
%Eldon Tyrell\authorrefmark{4}}
%\authorblockA{\authorrefmark{1}School of Electrical and Computer Engineering\\
%Georgia Institute of Technology,
%Atlanta, Georgia 30332--0250\\ Email: mshell@ece.gatech.edu}
%\authorblockA{\authorrefmark{2}Twentieth Century Fox, Springfield, USA\\
%Email: homer@thesimpsons.com}
%\authorblockA{\authorrefmark{3}Starfleet Academy, San Francisco, California 96678-2391\\
%Telephone: (800) 555--1212, Fax: (888) 555--1212}
%\authorblockA{\authorrefmark{4}Tyrell Inc., 123 Replicant Street, Los Angeles, California 90210--4321}}



% use only for invited papers
%\specialpapernotice{(Invited Paper)}

% make the title area
%\maketitle

%\begin{abstract}
%The abstract goes here.
%\input abs.tex
%\end{abstract}

% no key words

%\section{Introduction}
% no \PARstart
%This demo file is intended to serve as a ``starter file"
%for IEEE conference papers produced under \LaTeX\ using IEEEtran.cls version
%1.6 and later.


%\input Abstract



\IEEEraisesectionheading{\section{Introduction}}

\IEEEPARstart{V}{ision} system is studied in orthogonal disciplines spanning from neurophysiology and psychophysics to computer science all with uniform objective: understand the vision system and develop it into an integrated theory of vision. In general, vision or visual perception is the ability of information acquisition from environment, and it's interpretation. According to Gestalt theory, visual elements are perceived as patterns of wholes rather than the sum of constituent parts~\cite{koffka2013principles}. The Gestalt theory through \textit{emergence}, \textit{invariance}, \textit{multistability}, and \textit{reification} properties (aka Gestalt principles), describes how vision recognizes an object as a \textit{whole} from constituent parts. There is an increasing interested to model the cognitive aptitude of visual perception; however, the process is challenging. In the following, a challenge (as an example) per object and motion perception is discussed. 



\subsection{Why do things look as they do?}
In addition to Gestalt principles, an object is characterized with its spatial parameters and material properties. Despite of the novel approaches proposed for material recognition (e.g.,~\cite{sharan2013recognizing}), objects tend to get the attention. Leveraging on an object's spatial properties, material, illumination, and background; the mapping from real world 3D patterns (distal stimulus) to 2D patterns onto retina (proximal stimulus) is many-to-one non-uniquely-invertible mapping~\cite{dicarlo2007untangling,horn1986robot}. There have been novel biology-driven studies for constructing computational models to emulate anatomy and physiology of the brain for real world object recognition (e.g.,~\cite{lowe2004distinctive,serre2007robust,zhang2006svm}), and some studies lead to impressive accuracy. For instance, testing such computational models on gold standard controlled shape sets such as Caltech101 and Caltech256, some methods resulted $<$60\% true-positives~\cite{zhang2006svm,lazebnik2006beyond,mutch2006multiclass,wang2006using}. However, Pinto et al.~\cite{pinto2008real} raised a caution against the pervasiveness of such shape sets by highlighting the unsystematic variations in objects features such as spatial aspects, both between and within object categories. For instance, using a V1-like model (a neuroscientist's null model) with two categories of systematically variant objects, a rapid derogate of performance to 50\% (chance level) is observed~\cite{zhang2006svm}. This observation accentuates the challenges that the infinite number of 2D shapes casted on retina from 3D objects introduces to object recognition. 

Material recognition of an object requires in-depth features to be determined. A mineralogist may describe the luster (i.e., optical quality of the surface) with a vocabulary like greasy, pearly, vitreous, resinous or submetallic; he may describe rocks and minerals with their typical forms such as acicular, dendritic, porous, nodular, or oolitic. We perceive materials from early age even though many of us lack such a rich visual vocabulary as formalized as the mineralogists~\cite{adelson2001seeing}. However, methodizing material perception can be far from trivial. For instance, consider a chrome sphere with every pixel having a correspondence in the environment; hence, the material of the sphere is hidden and shall be inferred implicitly~\cite{shafer2000color,adelson2001seeing}. Therefore, considering object material, object recognition requires surface reflectance, various light sources, and observer's point-of-view to be taken into consideration.


\subsection{What went where?}
Motion is an important aspect in interpreting the interaction with subjects, making the visual perception of movement a critical cognitive ability that helps us with complex tasks such as discriminating moving objects from background, or depth perception by motion parallax. Cognitive susceptibility enables the inference of 2D/3D motion from a sequence of 2D shapes (e.g., movies~\cite{niyogi1994analyzing,little1998recognizing,hayfron2003automatic}), or from a single image frame (e.g., the pose of an athlete runner~\cite{wang2013learning,ramanan2006learning}). However, its challenging to model the susceptibility because of many-to-one relation between distal and proximal stimulus, which makes the local measurements of proximal stimulus inadequate to reason the proper global interpretation. One of the various challenges is called \textit{motion correspondence problem}~\cite{attneave1974apparent,ullman1979interpretation,ramachandran1986perception,dawson1991and}, which refers to recognition of any individual component of proximal stimulus in frame-1 and another component in frame-2 as constituting different glimpses of the same moving component. If one-to-one mapping is intended, $n!$ correspondence matches between $n$ components of two frames exist, which is increased to $2^n$  for one-to-any mappings. To address the challenge, Ullman~\cite{ullman1979interpretation} proposed a method based on nearest neighbor principle, and Dawson~\cite{dawson1991and} introduced an auto associative network model. Dawson's network model~\cite{dawson1991and} iteratively modifies the activation pattern of local measurements to achieve a stable global interpretation. In general, his model applies three constraints as it follows
\begin{inlinelist}
	\item \textit{nearest neighbor principle} (shorter motion correspondence matches are assigned lower costs)
	\item \textit{relative velocity principle} (differences between two motion correspondence matches)
	\item \textit{element integrity principle} (physical coherence of surfaces)
\end{inlinelist}.
According to experimental evaluations (e.g.,~\cite{ullman1979interpretation,ramachandran1986perception,cutting1982minimum}), these three constraints are the aspects of how human visual system solves the motion correspondence problem. Eom et al.~\cite{eom2012heuristic} tackled the motion correspondence problem by considering the relative velocity and the element integrity principles. They studied one-to-any mapping between elements of corresponding fuzzy clusters of two consecutive frames. They have obtained a ranked list of all possible mappings by performing a state-space search. 



\subsection{How a stimuli is recognized in the environment?}

Human subjects are often able to recognize a 3D object from its 2D projections in different orientations~\cite{bartoshuk1960mental}. A common hypothesis for this \textit{spatial ability} is that, an object is represented in memory in its canonical orientation, and a \textit{mental rotation} transformation is applied on the input image, and the transformed image is compared with the object in its canonical orientation~\cite{bartoshuk1960mental}. The time to determine whether two projections portray the same 3D object
\begin{inlinelist}
	\item increase linearly with respect to the angular disparity~\cite{bartoshuk1960mental,cooperau1973time,cooper1976demonstration}
	\item is independent from the complexity of the 3D object~\cite{cooper1973chronometric}
\end{inlinelist}.
Shepard and Metzler~\cite{shepard1971mental} interpreted this finding as it follows: \textit{human subjects mentally rotate one portray at a constant speed until it is aligned with the other portray.}



\subsection{State of the Art}

The linear mapping transformation determination between two objects is generalized as determining optimal linear transformation matrix for a set of observed vectors, which is first proposed by Grace Wahba in 1965~\cite{wahba1965least} as it follows. 
\textit{Given two sets of $n$ points $\{v_1, v_2, \dots v_n\}$, and $\{v_1^*, v_2^* \dots v_n^*\}$, where $n \geq 2$, find the rotation matrix $M$ (i.e., the orthogonal matrix with determinant +1) which brings the first set into the best least squares coincidence with the second. That is, find $M$ matrix which minimizes}
\begin{equation}
	\sum_{j=1}^{n} \vert v_j^* - Mv_j \vert^2
\end{equation}

Multiple solutions for the \textit{Wahba's problem} have been published, such as Paul Davenport's q-method. Some notable algorithms after Davenport's q-method were published; of that QUaternion ESTimator (QU\-EST)~\cite{shuster2012three}, Fast Optimal Attitude Matrix \-(FOAM)~\cite{markley1993attitude} and Slower Optimal Matrix Algorithm (SOMA)~\cite{markley1993attitude}, and singular value decomposition (SVD) based algorithms, such as Markley’s SVD-based method~\cite{markley1988attitude}. 

In statistical shape analysis, the linear mapping transformation determination challenge is studied as Procrustes problem. Procrustes analysis finds a transformation matrix that maps two input shapes closest possible on each other. Solutions for Procrustes problem are reviewed in~\cite{gower2004procrustes,viklands2006algorithms}. For orthogonal Procrustes problem, Wolfgang Kabsch proposed a SVD-based method~\cite{kabsch1976solution} by minimizing the root mean squared deviation of two input sets when the determinant of rotation matrix is $1$. In addition to Kabsch’s partial Procrustes superimposition (covers translation and rotation), other full Procrustes superimpositions (covers translation, uniform scaling, rotation/reflection) have been proposed~\cite{gower2004procrustes,viklands2006algorithms}. The determination of optimal linear mapping transformation matrix using different approaches of Procrustes analysis has wide range of applications, spanning from forging human hand mimics in anthropomorphic robotic hand~\cite{xu2012design}, to the assessment of two-dimensional perimeter spread models such as fire~\cite{duff2012procrustes}, and the analysis of MRI scans in brain morphology studies~\cite{martin2013correlation}.

\subsection{Our Contribution}

The present study methodizes the aforementioned mentioned cognitive susceptibilities into a cognitive-driven linear mapping transformation determination algorithm. The method leverages on mental rotation cognitive stages~\cite{johnson1990speed} which are defined as it follows
\begin{inlinelist}
	\item a mental image of the object is created
	\item object is mentally rotated until a comparison is made
	\item objects are assessed whether they are the same
	\item the decision is reported
\end{inlinelist}.
Accordingly, the proposed method creates hierarchical abstractions of shapes~\cite{greene2009briefest} with increasing level of details~\cite{konkle2010scene}. The abstractions are presented in a vector space. A graph of linear transformations is created by circular-shift permutations (i.e., rotation superimposition) of vectors. The graph is then hierarchically traversed for closest mapping linear transformation determination. 

Despite of numerous novel algorithms to calculate linear mapping transformation, such as those proposed for Procrustes analysis, the novelty of the presented method is being a cognitive-driven approach. This method augments promising discoveries on motion/object perception into a linear mapping transformation determination algorithm.



\subsection{Network}
We consider a network, e.g., the whole Internet or a part of it, that consists of $N$ autonomous systems (ASes). We represent each AS as a \textit{single node} that operates as a BGP router; this abstraction that is common in related literature~\cite{Labozitz-Delayed-convergence-CCR-2000,Kotronis-Routing-Centralization-ComNets-2015}, allows to hide the details of the intra-AS structure and functionality, and focus on inter-domain routing. When two ASes are connected (transit, peering, etc., relation), we consider that a link exists between the corresponding routers, over which data traffic and BGP messages can be exchanged. %In reality, an AS can be a very large network composed of hundreds or thousands switches/routers, and extend over a large area. However, the  abstraction of our model allows to hide the details of the intra-AS structure and functionality, whose effects on inter-domain routing is less important.

%We consider a network, e.g., the whole Internet or a part of it, that consists of $N$ autonomous systems (ASes). Since we are interested in the inter-domain routing, we represent each AS as a \textit{single node} that operates as a BGP router, which is common when studying inter-domain routing~\cite{Labozitz-Delayed-convergence-CCR-2000,Kotronis-Routing-Centralization-ComNets-2015}. When two ASes are connected (transit, peering, etc., relation), we consider that a link exists between the corresponding routers, over which data traffic and BGP messages can be exchanged. In reality, an AS can be a very large network composed of hundreds or thousands switches/routers, and extend over a large area. However, the  abstraction of our model allows to hide the details of the intra-AS structure and functionality, whose effects on inter-domain routing is less important.



\subsection{SDN Cluster}
%\subsection{Inter-domain Routing Centralization}

ASes can be ISPs, enterprises, CDNs, IXPs, etc., belong to different administrative entities, and span a wide range of topological, operational, economic, etc., characteristics. As a result, not all ASes should be expected to be willing to cooperate for and/or participate in an inter-domain centralization effort. Routing centralization is envisioned to begin from a group of a few ASes cooperating with each other, e.g., at an IXP location~\cite{Gupta-SDX-CCR-2014,Kotronis-CXP-SOSR-2016}; then, more ASes could be attracted (performance or economics related incentives) to join the group, or form another group.
%a centralized inter-domain routing service

%ASes in the Internet belong to different administrative entities (e.g., ISPs, enterprises, CDNs, IXPs), spanning a wide range of topological, operational, financial, etc., characteristics. Hence, not all ASes should be expected to be willing to cooperate in order to establish a globally centralized inter-domain routing system. Routing centralization is more probable to begin from a group of a few ASes cooperating with each other, e.g., at an IXP location~\cite{}; then, more ASes could be attracted (due to incentives related to performance or financial factors) to join the group, or form another group.


To this end, we assume that $k\in[1,N]$ ASes, i.e., a fraction of the entire network, cooperate in order to centralize their inter-domain routing. In the remainder, we refer to the set of these $k$ ASes, as the \textit{SDN cluster}\footnote{Although we use the term \textit{SDN}, our framework does not require necessarily that routing centralization is implemented on an SDN architecture.}. To avoid delving into system-specific issues of the centralization implementation, we assume the following setup, which captures main characteristics of several proposed solutions(e.g.,~\cite{Kotronis-Routing-Centralization-ComNets-2015,Rothenberg-Revisiting-RCP-HotSDN-2012,fibbing-sigcomm-2015}), and is generic enough to accommodate future solutions: ASes in the SDN cluster exchange routing information with a central entity, which we call \textit{multi-domain SDN controller}. The multi-domain SDN controller might be an SDN controller that directly controls the BGP routers of the ASes (e.g., as in~\cite{Kotronis-Routing-Centralization-ComNets-2015}), or a central server that only provides information or sends BGP messages to the ASes (e.g., similar to~\cite{fibbing-sigcomm-2015}). %\blue{In the latter case, the ASes in the SDN cluster would not need to grant access to their routers, or disclose all their routing policies.}


%there exists a \textit{multi-domain SDN controller}, which is connected to (and controls) the BGP routers of all the ASes in the SDN cluster. \blue{[say that we do not necessarily mean a centralized SDN controller that controls all the routers of the members; such a service could be implemented as a centralized service (e.g., a server running some routing software) that provides information to the participants. E.g., the central server provides information about the routing changes (in a far part of the Internet) to the intra-domain SDN controller / BGP routers / etc. of its member. Then the member, which is controls its own routers, decides based on this information. In this way, no routing policies need to be disclosed etc.]}





\subsection{BGP Updates}
Each node has a routing table (Routing Information Base, RIB), in which each entry contains an IP Prefix, and the corresponding AS-path (i.e., sequence of ASes) through which this prefix can be reached. RIBs are built from the information received by the neighbor ASes: upon a routing change, the ``source'' AS (e.g., the AS that originates a prefix) sends BGP updates to its neighbors to notify them about the change; when an AS receives a BGP update, it calculates the needed updates (if any) for its RIB, and sends BGP updates to its own neighbors. In this way, BGP updates propagate over the entire network, and paths to prefixes are built in a distributed way.

%Each AS/router keeps a routing table (Routing Information Base, RIB) with information about how to route traffic to different IP prefixes. Each entry in the table includes an IP Prefix, and the corresponding AS-path, i.e., the sequence of ASes through which this prefix can be reached. The routing tables are built based on the information received by the neighbor ASes: upon a routing change, the ``source'' AS (e.g., the AS that announced a prefix) sends BGP messages/updates to its neighbors to notify them about the change; when an AS receives a BGP update, it calculates the needed updates (if any) for its RIB, and sends BGP messages to its own neighbors. In this way the BGP updates propagate over the whole network, and (shortest) paths to prefixes are built in a distributed way.


Let us assume that an AS receives a BGP update at time $t_{1}$ and forwards it to a neighbor AS at time $t_{2}$. We call \textit{BGP update time}, and denote $T_{bgp}$, the time between the reception of a BGP update in an AS and its forwarding to a neighbor AS, i.e., $T_{bgp} = t_{2}-t_{1}$. The BGP update times may vary a lot among different ASes and/or connections, since they depend on a number of parameters: routers' hardware/software (e.g., time to process BGP data and update RIB) and/or configuration (e.g., MRAI timers), intra-AS network structure (e.g., number of routers, topology) and/or operation (e.g,. iBGP configuration, intra-AS SDN), etc. 

%Knowing all these parameters and for every AS in the network is not possible. Even if we knew them, the involved complexity would not allow us to perform a tractable analysis for the propagation of the routing information. To this end, we follow a probabilistic approach to model the BGP update times. Specifically, we make the following assumption.

Knowing all these parameters for every AS is not possible, and using (upper) bounds for $T_{bgp}$ would not lead to practical conclusions~\cite{Labozitz-Delayed-convergence-CCR-2000}. Thus, to be able to perform a useful analysis, we follow a probabilistic approach, and model the BGP update times as follows.


\begin{assumption}[BGP updates - renewal process]\label{assumption:t-bgp}
The BGP update times $T_{bgp}$ are independent and identically distributed random variables, drawn from an arbitrary distribution $f_{bgp}(t)$, with $E[T_{bgp}] = \mu_{bgp}$.
%$f_{bgp}:[0,+\infty)\rightarrow [0,1]$, where $\int_{0}^{+\infty} f_{bgp}(x)dx=1$, and $E[T_{bgp}] = \mu_{bgp}$.
\end{assumption}
Under Assumption~\ref{assumption:t-bgp}, BGP update times are given by a renewal process. The model is very generic, since it allows to use any valid function $f_{bgp}(t)$, and thus describe a wide range of scenarios with different parameters. Real measurements can be used to make a realistic selection for the distribution $f_{bgp}(t)$, as we show in Appendix~\ref{sec:distr-t-bgp}; however, a detailed study for fitting the $f_{bgp}(t)$ is beyond the scope of this paper.


%\blue{[say since bgp update times are mostly related to intra-domain characteristics, the independence assumption is not far from reality(???)]}
%\blue{[see comment about autocorrelation - I'd suppose it's not a problem for a single bgp update]}



\subsection{Inter-domain SDN Routing}
Routing information in the SDN cluster propagates in a centralized way, through the multi-domain SDN controller. When an AS in the SDN cluster receives a BGP update from a neighbor AS (not in the SDN cluster), it forwards the update to the SDN controller. The SDN controller, which is aware of the topology in the SDN cluster and the connections/paths to external ASes, informs every AS in the SDN cluster about the needed changes in the routing paths. The ASes that receive the updated routes from the controller, notify their non-SDN neighbors using the standard BGP mechanism. 

%The propagation of the routing information among the ASes in the SDN cluster takes place in a centralized way, through the multi-domain SDN controller. When an AS belonging to the SDN cluster receives a BGP update from an neighbor AS (not in the SDN cluster), forwards the update to the SDN controller. The SDN controller, which is aware of the topology in the SDN cluster and the connections/paths to external ASes, calculates the needed changes in the routing tables and informs every AS in the SDN cluster. The ASes that receive the updated routes from the controller, notify their non-SDN neighbors using the standard BGP mechanism.


Let $t_{1}$ be the time that the first AS belonging to the SDN cluster receives a BGP update from a non-SDN neighbor, and $t_{2}$ the time till \textit{all} ASes in the SDN cluster have been informed (by the controller) for the BGP updates. We denote as $T_{sdn}$ the time needed for all the SDN cluster to be informed after a member has received a BGP update, i.e., $T_{sdn} = t_{2}-t_{1}$. The times $T_{sdn}$ would depend on the system implementation. However, it was shown that system designs can achieve $T_{sdn}\ll T_{bgp}$~\cite{supercharge-me-2015}. Hence, in the remainder -for the sake of presentation- we assume that $T_{sdn}\rightarrow 0$. Nevertheless, our results can be easily modified for arbitrary $T_{sdn}$ (even for cases with $E[T_{sdn}] > E[T_{bgp}]$), without this affecting the main conclusions of the study.

%Thus, we assume -for simplicity- that $T_{sdn}\rightarrow 0$. 


%For instance, in~\cite{Kotronis-Routing-Centralization-ComNets-2015} $T_{sdn}$ is not more than a few seconds, whereas the default value for MRAI timers in Cisco routers is $30sec.$.

%\blue{[Say that (a) T-{sdn} is similarly given by a distribution, (b) in some case it might be much smaller than T-bgp, (c) in the remainder we assume (for the sake of presentation) that T-sdn << T-bgp, (d) however, all our results can be easily modified for arbitrary T-sdn (e.g., even for T-sdn >> T-bgp, (e) maybe cite the TechRep where you mention what are the needed modifications, (f) the main conclusions/insights are not affected by considering Tsdn << T-bgp], (g) remove the argument of MRAI timers}

%This time can be expected to be much lower than the BGP updating process, thus, for simplicity, we assume here that $T_{sdn}=0$.
%This time can be expected to be in the order of few seconds~\cite{}, and much lower than the BGP updating process (cf., the default value for MRAI timers is Cisco routers is $30sec.$), thus, for simplicity, we assume here that $T_{sdn}=0$.

\subsection{Preliminaries and Problem Statement}
\noindent In our analysis, we consider the following setup: 

Every node in the network knows at least one (BGP) path to every other node.

A node initiates a routing change that affects the inter-domain routing (e.g., node $n_{0}$ in Fig.~\ref{fig:sd-path}). This could be an announcement or withdrawal of an IP prefix, an interruption of an AS connection (e.g., a link is down), etc. Here, we consider that a node, which we call the ``source node'', announces a new IP prefix; this routing change affects the entire network, every node will install a path for this prefix in its RIB upon the reception of the BGP update.

Nodes in the SDN cluster, receive route information from the SDN controller, and add an entry in their RIB for the prefix to the source node; even if the path is not established in the node preceding in this path (e.g., in Fig.~\ref{fig:sd-path} node $n_{j}$ might receive the update before node $n_{j-1}$). In this case only the node in the SDN cluster knows how to route traffic to the new prefix, therefore, if the SDN node sends traffic to the new prefix, this would not necessarily reach the source-node. The connectivity will be established when every AS in the path has been informed about the BGP update.

BGP updates do not propagate backwards in the path; this would create loops or longer paths, which are discarded or not preferred by BGP.

We call ``SD-path'' the final path, i.e., the shortest conforming to the routing policies, between the source node (``S'') and another node (``destination'', or ``D'').

In the remainder of the paper we investigate the effects of routing centralization on: (a) the data-plane connectivity between the source node (``S'') and any node (``D'') in the network, i.e., the time needed till all nodes in an SD-path have installed the updated BGP paths after a routing change (Section~\ref{sec:data-plane}); and (b) the control-plane convergence, i.e, the time needed till the entire network has established the final paths corresponding to the routing change (Section~\ref{sec:control-plane}).
 
%\begin{itemize}
%\item every AS in the network knows at least a path to reach every other AS
%\item an AS announces a new prefix
%\item an AS in the SDN cluster when it receives the BGP update, adds an entry in its RIB for the prefix to the source-AS; even if the path is not established in the ASes preceding in this path (in this case the SDN AS knows how to route traffic to the new prefix, although other ASes do not. Hence if the SDN AS sends traffic this would not necessarily reach the source-AS. The connectivity will be established when every AS in the path has been informed about the BGP update.
%\item no AS sends BGP updates to its neighbors backwards in the path, since this would create loops or longer paths (which are discarded or not preferred by BGP)
%\item an SD-path is the final path (i.e., the shortest conforming to the routing policies) between the nodes S and D
%\end{itemize}

For ease of reference, we summarize the notation in Table~\ref{table:important-notation}.
%For ease of reference, we summarize the main notation we use in the paper in Table~\ref{table:important-notation}.
\begin{table}
\centering
\caption{Important Notation}
\label{table:important-notation}
\begin{tabular}{|l|l|l|}
\hline
{$N$}	& {network size (total \# of nodes)}&{}\\
\hline
{$k$}	& {SDN cluster size (total \# of SDN-nodes)}&{}\\
\hline
{$T_{bgp}$}	& {BGP update time}&{}\\
\hline
{$f_{bgp}(t)$}	& {distribution of BGP update times}&{Assumption~\ref{assumption:t-bgp}}\\
\hline
{$d$}	& {path length}&{}\\
\hline
{$k^{'}$}	& {\# of SDN-nodes \textit{on a path}}&{}\\
\hline
{$T_{SD}$}	& {data-plane connectivity time in a SD-path}&{Theorem~\ref{thm:sd-path-d-k}}\\
\hline
{$T_{c}$}	& {BGP convergence time}&{Theorem~\ref{thm:expectation-and-variance-Tc}}\\
\hline
{$T_{\ell}$}	& {$\ell$-partial BGP convergence time}&{Corollary~\ref{thm:l-partial-Tc}}\\
\hline
\end{tabular}
\end{table}
\subsection{Data-driven Learning}

Rather than following a pre-established model, modern machine learning and deep learning methods derive the knowledge directly from data  
%Taking deep supervised learning as an example, it mainly starts from the data itself, sets the corresponding activation function and loss function, and passes the given input and output during the training phase. 
without assuming their distribution format, and can well follow nonlinear data. However, they often suffer from large computational cost, weak interpretability,  and bias when data are unbalanced. In the case of time series, %the Support Vector Regression(SVR) \note{Is SVR for time series data only or general data?} \note{SVR can solve general nonlinear regression problems and it can be extend to solve time series problem}
sequential neural networks, such as Recurrent Neural Network (RNN), Long Short-Term Memory (LSTM) and Gated Recurrent Units (GRU), are often applied. 

%\paragraph{A. Support Vector Regression}

%Support vector machine (SVM) is originally proposed to solve the binary classification problems,  and it follows the VC dimension theory and the principle of minimizing the structural risk.
%It was originally used (support vector machine classification), and 
%It is extended to solve the function approximation problem, and called  support vector regression (SVR). 
%With SVR, \del{a kernel technique can be used to convert an input non-linear sample set into a high-dimensional space to improve the sample separation.}
%With SVR, the input \rev{data samples can be converted into ones in a higher} dimensional space that \rev{a nonlinear regression problem in the original space becomes the linear regression problem} in the new space.  

%\note{It is better to put the remaining in the model design part. My earlier question was because you were presenting it as a classification. I haven't modified this part, and will after your moving to fit in the context} 
%\rev{For a time series, the  relationships between ${y_t}$ and its previous values ${y_{t-1},y_{t-2},...,y_{t-k}}$ is actually a regression problem. However, through the ARIMA-GARCH model we know that this is not a linear regression problem, let alone the relationship between unstable parts. As for the unstable part, represented by $N_t$, the relationships between $N_t$ and ${N_{t-1},N{t-2},...,N_{t-k}}$ are more complicated and chaotic.} \rev{For a data set $S$ with a complex nonlinear regression relationship, it} can be mapped to the high dimensional feature space(Hilbert space) through the nonlinear mapping $\phi(x)$ \rev{and the kernel techniques to make their relationships linear. Therefore, SVR becomes our choice to deal with time series problems. }

%the feature sets $X = {x_1,x_2,..,x_k}$
%For \rev{a} sample set $S$ that is linearly inseparable in the original space, the feature $x$ can be mapped to the high-dimensional feature space (Hilbert space) through the nonlinear mapping $\phi(x)$ \rev{so that $\phi(x)$ is Linearly separable.} \del{in the  the high-dimensional feature space. Therefore, the nonlinear problem of S in the original space is well solved. This is a support vector machine nonlinear regression, called support vector regression (SVR).}



\paragraph{A. Recurrent Neural Network}

RNN \cite{manaswi2018rnn}\cite{tokgoz2018rnn}\cite{poulos2017rnn}includes a neural network to provide a very straight-forward but effective way of handling time series or other sequential data. RNN is recurrent, where the same function is performed for each time stage and the output of the previous time stage is the input of the next stage. RNN can model a sequence of data so that each sample  depends on previous ones. Backward Propagation Algorithm is often used to train RNN. 

%Fig~\ref{RNN} shows a typical RNN structure.  Given the sequence of input x = (x1,...xt), RNN iteratively calculates the hidden layer \rev{vector} \note{multiple vectors? Why do you use "vectors"? You use this plural or single form randomly everywhere. It gives people the bad impression of very lousy writing. } $h^* = (h^*_1;...; h^*_t)$ and the output layers \rev{vector} \note{multiple vectors?} $y = (y_1;...; y_t)$ as follows  for $t = 1$ to $T $:
%\begin{equation}
 %    h^*_t = H(W_{xh^*}x_t + W_{h^*h^*}h^*_{t-1}+b_h^*), \hspace{0.2in}  y_t = W_{h^*y}h^*_t + b_y,
%\end{equation}
%\begin{equation}
%    y_t = W_{hy}h_t + b_y
%\end{equation}
%where  $h^*_t$ is the  hidden vector,  $b_h^*$ is the hidden bias, $H$ is the hidden layer function, $W_{h^*h^*}$ is the weight at previous hidden state and $W_{xh^*}$ is the weight at the current state, $y_t$ is the output state, $W_{h^*y}$ is the weight at the current hidden state and $b_y$ is the output bias. RNN can model a sequence of data so that each sample can be assumed to be dependent on previous ones. 
%Backward Propagation Algorithm is often used to train RNN. 

%\begin{wrapfigure}[10]{r}{2.5in}
  %\vspace{-25pt}
 % \begin{center}
%\includegraphics[width=2.5 in]{images/RNN.png}
  %\end{center}
  %\vspace{-20pt}
  %\caption{\footnotesize RNN Structure}
%\label{RNN}
  %\vspace{-10pt}
%\end{wrapfigure}

\paragraph{B. Long Short-Term Memory}
As the back-propagated error  either explodes or vanishes during the propagation in the training process, RNN cannot process very long time sequence.  To alleviate the problem, 
LSTM \cite{song2020time}\cite{sagheer2019time}
is introduced. It has two transmission states, the  cell state $c_t$  and the hidden state $h^*_t$. $c_t$  changes slowly, while $h^*_t$ changes faster. 
%Fig~\ref{LSTM} shows the structure of LSTM with a single layer cell.
An LSTM time stage consists of many recurrently
memory cells. Each cell contains three multiplicative gate units, named input gate, forget gate and output gate.

%\begin{wrapfigure}[10]{r}{2.5in}
  %\vspace{-45pt}
 % \begin{center}
%\includegraphics[width=2.5 in]{images/LSTM.png}
 % \end{center}
  %\vspace{-20pt}
  %\caption{\footnotesize Basic LSTM Structure}
%\label{formulas}
 % \vspace{-10pt}
%\end{wrapfigure}

%Simply put, LSTM can perform better in longer sequences than normal RNN.
%LSTM has two transmission states, the  cell state $c_t$  and the hidden state $h^*_t$. $c_t$  changes slowly, while $h^*_t$ changes faster. 
%Fig~\ref{LSTM} shows the structure of LSTM with a single layer cell.
%An LSTM \rev{temporal} layer \note{temporal or spatial layer?}consists of many recurrently memory cells and each cell contains three multiplicative gate units, including the input, forget and output.

%The LSTM architecture is implemented through the following composite functions:
%\begin{equation}
   % i_t = \sigma(W_{xi}x_t + W_{h^*i}h^*_t-1 + b_i), \hspace{0.1in}  f_t = %\sigma(W_{xf}x_t + W_{h^*f}h^*_{t-1} + b_f), \hspace{0.1in} \mbox{and}  %\hspace{0.1in} o_t = \sigma(W_{xo}x_t + W_{h^*o}h^*_{t^*-1}+b_o)
%\end{equation}
%\begin{equation}
%    f_t = \sigma(W_{xf}x_t + W_{hf}h_{t-1} + b_f)
%\end{equation}
%\begin{equation}
    %c_t = f_tc_{t-1} + i_ttanh(W_{xc}x_t + W_{h^*c}h^*_{t-1}+b_c), %\hspace{0.1in} \mbox{and}  \hspace{0.1in}  h^*_t = o_ttanh(c_t). 
%\end{equation}
%\begin{equation}
%    o_t = \sigma(W_{xo}x_t + W_{ho}h_{t-1}+b_o)
%\end{equation}
%\begin{equation}
%    h_t = o_ttanh(c_t)
%\end{equation}
%where $\sigma$ is the logistic sigmoid function, and $i$, $f$, $o$ and $c$ stand for input gates, forget gates, output gates and cell vectors. In LSTM, weights are divided into 3 sets, input weights ($W_{xi}$, $W_{xf}$, $W_{xc}$, $W_{xo}$ $\in{R^{N \times M}}$), recurrent weights ($W_{h^*i}$,$W_{h^*f}$ ,$W_{h^*c}$,$W_{h^*o}$ $\in{R^{N \times N}}$) and bias weights ($b_i$,$b_f$,$b_c$,$b_o$ $\in{R^N}$).  $N$ is the number of LSTM cells, and $M$ is the dimension of the input.

%\section{Modeling Time Series Data with Model-driven and Data-driven Components}
\section{Modeling Time Series Data with Mathematical Models and Data-driven Learning}

%\note{ Don't put method, scheme, etc in the title. people knows it is a method, or a scheme. Don't put novel, which only gives people bad impression. You need to convince people it is novel with your description, but not to claim it is novel.  It triggers the anger.}



%\note{You need to pay attention how English sentences are written formally from paper reading. Your writing is too oral and informal. Also comparing my writing and your previous writing to see the difference. See why I reorganize your writing each time to make the logic more natural.}

%Generally, statistical models can well represent stable data with slow variations, but are not good at tracking dynamic data with fast changes. \rev{On the other hand,  Purely relying on machine learning techniques to track data, however, may suffer from long training time and low interpretability. In order to more accurately model the time series data while reducing the complexity and increasing the interpretability, we explore the use of both  model-driven  and data-driven approaches. 
 
 Generally, statistical models can well represent stable data with slow variations, but are not good at tracking dynamic data with fast changes. 
On the other hand,  without assuming the data format, learning directly from data can more flexible represent data. However, if there is no knowledge on the data,  it may suffer from long training time. It is also hard to control the learning process and explain the results.
 %\rev{On the other hand,  without assuming the data format, learning directly from data can more flexible represent data. However, if there is no knowledge on the data,  it may suffer from long training time. It is also hard to control the learning process and explain the results.} 
 In order to more accurately model the (time series) data while reducing the complexity and increasing the interpretability, we explore the use of both  model-driven  and data-driven approaches. 
 
 
 

%\note{You are not decomposing a data set to multiple parts, but each data item to multiple parts, right?} \note{Yes! Not  decomposing a data set but for each data item to multiple parts.}
%\rev{In order to more accurately model the data while reducing the complexity and increasing  interpret-ability, fully explore the information of the data itself and enhance the interpret-ability of the model, some model-driven + data-driven approaches can be considered. In this section, assuming we just have one time series data, our major goal is to boost the current model-driven or data-driven methods just based on the single time series data we have without introducing any other information. Primarily we have two different directions to achieve our goal: The first one is to introduce the data decomposing method. \sout{According to the learning abilities  of statistical time series model and machine learning model}, we decompose the data set into different parts. For each part, we use the specific model with the best learning capacity to train and model. The motivation here is we believe that statistical methods and machine learning methods both have their own advantages and limitations. After the data is decomposed, choosing the optimal modeling method according to the characteristics of different parts can make full use of the advantages of different learners, improve the learning ability of the entire time series data, and make the prediction more accurate. }


We first introduce a decomposition-based method to exploit both types of techniques.  We decompose each data item into multiple parts, and for each part, we choose a specific model or learning format that can best represent its data features. Given the tradeoff between methods that are purely based on models and purely driven by data, if the statistics of data are known, it may help to %speed up the training process and  
better understand the data characteristics. Therefore, as a second method, we propose to extract the statistics of data and incorporate them into the pure data-driven learning framework to improve the performance. 
%For the third method, we will design a mixed model that can concurrently exploit the decomposition-based method and the extraction-based method.

%We first introduce a decomposition-based method to exploit both types of techniques.  We decompose each data item into multiple parts.  For each part, we choose a specific model that can best represent the data based on the features of the part. \rev{The performance of statistic models is often limited by pre-setting the parameters of models. While} neural networks are known to be able to more flexibly model the data \rev{without assuming the} data format, without any knowledge on the data, \rev{it} may take a long time for the training process to converge and it is \rev{also} hard to explain and control the network learnt. If the statistics of the data are known, \rev{it may} help to speed up the training process \rev{and  better understand the} data characteristics. \rev{Therefore,} as a second method,  we propose to extract the statistics of data and incorporate them into the pure neural-network based models to improve the learning capability. For the third method, we will design a mixed model that can concurrently exploit the decomposition-based method and the extraction-based method.



%\note{You may need to compare the onverge time when the error threshold is set up, not the number of epoches is fixed. Also, you may need to compare with the case the model-based method with the model parameters are learning, not obtained through statistics.}  


%Another direction is to use statistical models to mine new valuable information or statistics to improve the learning ability of machine learning models. The usual data-driven supervised learning method mainly lies in learning the mapping relationship between input variables and output variables. While Model-based statistical methods are very concerned about the generation of certain statistics, and use different statistics to measure the distribution of the population or the sample and certain conditions of the sample itself. For example, statistical researchers will use average to assess overall quality, variance to measure fluctuations in things, or mode to judge mainstream opinions, and so on. Therefore, we want to use the statistical model to find valuable statistics first then feed these statistics into the data-driven learner to enhance the learning ability.This is our motivation of second direction. }

%\rev{In this section, section 3.1 is the method of first direction, section 3.2 is the method of second direction and section 3.3 is a boosting method based on the two directions. The performance of different methods will be shown on Section 4.}


\subsection{Method 1: Modeling Time Series Data with Decomposition}


Time series data can be divided into a stable part and an unstable part. Generally, the stable part can be more easily modeled, while it is often hard to  accurately model the unstable part. Further, the stable part can be categorized  into two types, linearly stable  and non-linearly stable. From the model features we introduce earlier, we could choose to model the linearly stable part of the data with ARIMA and the non-linearly stable part with GARCH. In order to capture all possible features in practical data, we can represent the data with  ARIMA-GARCH and Machine Learning techniques (ML) together, so both stable part and and unstable part of the data can be tracked. 
%Here we use "ML" to represent the Machine Learning techniques.   

A few ML approaches have been introduced to represent time series data. 
%\del{From our preliminary studies, adding ML into the model does not always help. An improper learning method may produce negative effect. Thus, the selection of viable ML method is crucial.}
LSTM is a  popular ML method to deal with time series with a recurrent neural network. We use  LSTM to learn the unstable part,
%\del{and SVR can better deal with complex and chaotic time series data due to its use of kernel techniques.} 
%\del{For the learning of unstable part, as it is more chaotic and irregular, SVR may outperform LSTM.} 
%We use  LSTM as the primary machine learning technique to model the unstable part. %\del{compare the results with single used SVR and LSTM, respectively.}\del{but will still compare the results with the use of LSTM.} 
%del{Taking $LSTM$ model as an example, if we include the parameters into the model representation, data can be} 
and represent the overall data with $ARIMA(p,d,q)-GARCH(P,Q)-LSTM$. It is not difficult to see that this complete data representation can reduce to $ARIMA(p,d,q)-LSTM$ when there are only linear stable part and unstable part in the time series, that is ($P=0,Q=0$). The complete $ARIMA-GARCH-LSTM$ model  will reduce to the $GARCH($P$,$Q$)-LSTM$ model if there is not linearly stable part in the time series, that is ($p=0,q=0$). Next we we introduce the detailed formats of these three representations. 

%\rev{However machine learning techniques with the ability to solve time series problems are suitable here. The improper learning method may produce negative effect. Therefore, the selection of viable ML method is a vital point. In this paper, we prefer to use the LSTM and SVR(Support Vector Regression) as the primary machine learning techniques. LSTM is a typical recurrent neural network with good ability to handle time series problem; while SVR has a strong ability to deal with the more complex and chaotic series data due to the kernel techniques it used. Because the unstable part is more chaotic and irregular, SVR may outperform LSTM.  If we include the parameters into the model representation, this data can be represented with $ARIMA(p,d,q)-GARCH(P,Q)-ML$. It is not difficult to see that this complete data representation can reduce to $ARIMA(p,d,q)-LSTM$ when there are only linear stable part and unstable part in the time series, that is ($P=0,Q=0$). The complete ARIMA-GARCH-LSTM model  will reduce to the GARCH($P,Q$)-LSTM model if there is no linear stable part in the time series, that is ($p=0,q=0$). Next we we introduce the detailed formats of these three representations.}


%Finally we compare ARIMA-GARCH-SVR with ARIMA-GARCH-LSTM, ARIMA-GARCH, LSTM and SVR to show the decomposition method with right machine learning algorithm will have the best performance.

%LSTM If we include the parameters into the model representation, this data can be represented with $ARIMA(p,d,q)-GARCH(P,Q)-ML$. Next we we introduce the detailed formats of these three representations. 

%The stable part refers to that can be modeled by the mathematical model like ARIMA and GARCH models; while the unstable part is that can't be modeled by those mathematical model. Further, the stable part can be divided into linear stable part and nonlinear stable part. Linear stable part is that can be explained by ARIMA model while the nonlinear stable part is that can be explained by GARCH model. Genrally, a hybrid ARIMA-GARCH and LSTM method can be used to model the  time series. However, ARIMA(p,d,q)-GARCH(P,Q)-LSTM model will reduce to ARIMA(p,d,q)-LSTM when there are only linear stable part and unstable part in the time series, that is (P=0,Q=0); and ARIMA-GARCH-LSTM model will reduce to GARCH(P,Q)-LSTM model if there is no linear stable part in the time series, that is (p=0,q=0).

\paragraph{A. Complete data representation with ARIMA-GARCH-LSTM}

As a general format, a time series $y_t$ can be represented as:
\begin{eqnarray}
     y_t = S_t + N_t = LS_t + NS_t + N_t
\end{eqnarray}
where $S_t$ is the  stable part, $N_t$ is the unstable part, $LS_t$ is the linearly stable part and $NS_t$ is the non-linearly stable part.  $S_t$ can be modeled by ARIMA-GARCH and $N_t$ can be learnt through $LSTM$. In this case, $y_t$ can be expressed more completely as:
\begin{eqnarray}
&& y_t = \mu_0  + u_t + \widehat{LSTM}, \nonumber \\
&& u_t = \sum_{i=1}^{p} {\alpha_i{u_{t-i}}} + e_t+ \sum_{j=1}^{q} {\theta_j{e_{t-j}}},\nonumber\\
&& e_t = z_t{\sqrt{h_t}}, \quad  z_t\sim N(0,1), \nonumber \\
&& h_t = w+\sum_{j=1}^{P}{\lambda_j{e_{t-j}^2}}+\sum_{i=1}^{Q}{\beta_i{h_{t-i}^2}}
\end{eqnarray}

where $y_t$ is a stationary time series, and $\mu_0$ is a constant. The stable part $u_t$ follows the $ARIMA-GARCH$ model and the unstable part  is learnt through $LSTM$. The random error $e_t$ is a function of $z_t$, which follows the normal distribution,  and $h_t$ is the GARCH term  (i.e. conditional variance). The parameters $\alpha_i$, $\theta_j$, $\beta_i$ and $\lambda_j$ are the weights. 

                    
\paragraph{B. ARIMA-LSTM}

If a time series just includes a linearly stable part and an unstable part, the time series can be represented as
\begin{eqnarray}
     y_t = LS_t +  N_t,
\end{eqnarray}
where $LS_t$ can be modeled by ARIMA model. The equation above can be expressed as
\begin{eqnarray}
&& y_t = \mu_0 + u_t + \widehat{LSTM} \nonumber\\
&& u_t = \sum_{i=1}^{p} {\alpha_i{u_{t-i}}} + e_t+ \sum_{j=1}^{q} {\theta_j{e_{t-j}}}
\end{eqnarray}
%\begin{eqnarray}
%    u_t = \sum_{i=1}^{p} {\alpha_i{u_{t-i}}} + e_t+ \sum_{j=1}^{q} {\theta_j{e_{t-j}}}, e_t\sim N(0,1) 
%\end{eqnarray}
\paragraph{C. GARCH-LSTM}

If a time series just includes a non-linearly stable part and an unstable part, the time series can be represented as
\begin{eqnarray}
     y_t = NS_t +  N_t
\end{eqnarray}
where $NS_t$ can be modeled by the GARCH model. The equation above can be expressed as
\begin{eqnarray}
&& y_t = \mu_0 + u_t + \widehat{LSTM}, u_t = z_t{\sqrt{h_t}},\quad z_t\sim N(0,1),\nonumber \\
&& h_t = w+\sum_{j=1}^{P} {\lambda_j{u_{t-j}^2}}+\sum_{i=1}^{Q}{\beta_i{h_{t-i}^2}} 
\end{eqnarray}

%\begin{eqnarray}
 %     u_t = z_t{\sqrt{h_t}},\quad z_t\sim N(0,1),\nonumber
%\end{eqnarray}
%\begin{eqnarray}
 %     h_t = w+\sum_{j=1}^{P} %{\lambda_j{u_{t-j}^2}}+\sum_{i=1}^{Q}{\beta_i{h_{t-i}^2}}
%\end{eqnarray}
%\rev{If LSTM model is sued to track the residual part, the expression is similar, and} we only need to  replace $\widehat{SVR}$ with $\widehat{LSTM}$ in the equations above \rev{to represent} the the ARIMA-GARCH-LSTM model.
%To investigate the efficiency of our decomposition method, we explore both one-step and multi-step prediction. For one-step prediction, we only make prediction for the next step. While for multi-step prediction, a \del{rolling}\rev{recursive} %\note{ or calling it "recursive"?} 
%procedure is taken, where the predicted value of the next step is used as the new input for the following prediction. %\del{Figure X shows the multi-steps rolling prediction \del{algorithm} based on our decomposition method.} 
%\note{Yes, A rolling forecast is a type of financial model that predicts the future performance of a business over a continuous period, based on historical data.That is, it relies on an add/drop approach to forecasting that drops a month/period as it passes and adds a new month/period automatically.}







\subsection{Method 2: Sequential Neural Network with Statistic Extraction}
 Good statistical values of a time series can help to better understand and capture the characteristics  of data. The average, median, mode and quantile are some example statistics parameters that can reflect the distribution of a data set and are used the most. 
 %However, in time series, if one wants to use some statistics to help us model and predict, it is difficult to get help from these commonly used statistics. An important reason is that they all have only one value.
 However, if these statistics are only drawn from existing data and set to fixed values, they can not well reflect the time varying features of time series data. If the statistics themselves can be modeled as  time series,  they may be fed as additional input variables to help better train the sequential neural network.
 %But if there exits a statistics that changes over time, in another word, if there is a statistics as time series data, then this kind of statistics can be fed as another input variable into the sequential neural network for training. 
 
 Apart from variables commonly used as inputs to the sequential neural network, we add in the statistics extracted from the data series as additional inputs. The GARCH term $h_t$ is a good candidate to choose. As a conditional variance, it can reflect the volatility of the sequence at different time and is essential for describing and predicting the changes of sequence. To explain our design principles, we use LSTM as an example neural network and the GARCH term $h_t$ as the example statistics. 
 %a conditional variance that can be  attracts our attention. 


  We use the past values of both $y_t$ and $h_t$ to train LSTM and predict the future values of $y_t$.  That is, besides past values of $y_t$ (i.e., ${y_{t-1}, y_{t-2},...,y_{t-k}}$), we add past values of the GARCH term $h_t$ (i.e., ${h_{t-1}, h_{t-2},...,h_{t-k}}$) as another input: %\note{For prediction, you may start from $y_{t+1}$} \note{The reason I use $y_t$ here is to be consistent with the previous expression. In time series, it's  normal to express in the way of $y_t$ = ....}
 
% \paragraph{Method I of Multi-variables LSTM based on statistics extraction}
% \rev{Apart from the variable of time series data in the past, the GARCH term $h_t$ will be another variable to train the LSTM. In method I, past values of $y_t$ are the first input and the past values of $h_t$ will be the second input. Specifically, ${y_{t}, y_{t-2},...,y_{t-k1}}$ is the first input and ${h_{t-1}, h_{t-2},...,h_{t-k2}}$ is the second input. Method I is designed by the common sense, where we use both the past information of $y_t$ and $h_t$ to train the LSTM and predict $y_t$ in the future. The equation can be expressed as below:}
 \begin{eqnarray}
    {y_{t}} = LSTM(y_{t-1},y_{t-2},...,y_{t-k};h_{t-1},h_{t-2},...h_{t-k}) + \varepsilon_t
\end{eqnarray}
where $\varepsilon_t$ is the random error. For the convenience of expression, we call the model above as LSTM-GARCH model. 

%\note{we don't do the recursive method in method 2 cause LSTM didn't have a good performance}

%\note{You only used LSRM in method 2.}

%\rev{Like that in method 1, we explore both  one-step  and  multi-step  predictions to show the performance of  method 2.} 

 %\rev{In method II, besides past values of $y_t$ and $h_t$, we will also add the future values of $h_t$ to train the sequential neural network. As a statistic variable, $h_t$ reflects the fluctuation of $y_t$, but is also more stable thus can be better predicted than the raw data.
 %a current to current mapping between $h_t$ and $y_t$ will better capture the current data states. 
 %In this case, to predict ${y_{t+k},y_{t+k-1},...,y_{t}}$,} we will not only use ${y_{t-1},y_{t-2},...,y_{t-k1}}$ and ${h_{t-1},h_{t-2},...h_{t-k2}}$, but also ${h_{t+k},h_{t+k-1},...,h_{t}}$. \rev{Although this design is easy to realize in the training phase, if multi-step prediction is needed,  we don't have the future values of ${h_{t+k},h_{t+k-1},...,h_{t}}$ in the testing phase and practical applications. To address the issue, we apply another LSTM to predict the future values of $h_t$ based on past $h_t$ values.} 
 %we train another multi-steps single-variable LSTM to  predict $h_t$ in the future. Therefore, in method II, we have trained two LSTM. The first one is trained to predict $h_t$; while the second one utilizes  the past values of  $y_t$ and the past to current values of $h_t$ as variables to train to predict $y_t$.  
 %The algorithm of method II is shown below and the structure is shown in Fig. 
 
 

% \paragraph{Method II of Multi-variables LSTM based on statistics extraction}
% \rev{In the Method I above, the past information both of $y_t$ and $h_t$ are used to predict the future $y_t$. But in method II, we want to add the present values of $h_t$ into the past $h_t$ to train our model. The reason to do so is that $h_t$ itself reflects the  fluctuation in current time of $y_t$. A current to current mapping between $h_t$ and $y_t$ will be a more logical relationship. For example, to predict ${y_{t+k},y_{t+k-1},...,y_{t}}$, we use not only  ${y_{t-1},y_{t-2},...,y_{t-k1}}$ and ${h_{t-1},h_{t-2},...h_{t-k2}}$, but ${h_{t+k},h_{t+k-1},...,h_{t}}$. In the training phase, this design is easy to implement but in the testing phase, we don't have the future values of ${h_{t+k},h_{t+k-1},...,h_{t}}$ because we assume we just have the information before t. \sout{However, to predict the future in test stage, the future values of $h_t$ are needed but we have no information of future $h_t$.} In order to solve this problem, we train another multi-steps single-variable LSTM to  predict $h_t$ in the future. Therefore, in method II, we have trained two LSTM. The first one is trained to predict $h_t$; while the second one utilizes  the past values of  $y_t$ and the past to current values of $h_t$ as variables to train to predict $y_t$.  The algorithm of method II is shown below and the structure is shwon in Fig. }
 
 
 
 
 
 
% The mathematical expression of method II is as follows:

 %\begin{equation}
   % {y_t} = LSTM(y_{t-1},y_{t-2},...,y_{t-k};h_t,h_{t-1},h_{t-2},...h_{t-k}) + \varepsilon_t
%\end{equation}
 
 
 
 
 %\note{ I still need some time to determine Method 3}
%\subsection{Method 3: Hybrid  Time Series Model \rev{with  Data Decomposition and Statistic Feature Extraction}}

%With data decomposition,  ARIMA-GARCH is applied to model the stable part of the data, and LSTM is used to track the unstable part of the data. In the method of exploiting statistic feature extraction, LSTM-GARCH takes the statistic-based time series $h_t$ as an additional input to predict $y_t$. To take full advantage of both types of method, we introduce a hybrid  model  Hybrid-LSTM-GARCH to exploit the use of both.   In this model,  a time series $y_t$ is first divided into a stable part $S_t$ and an unstable part $N_t$. Then $S_t$ is modeled by ARIMA-GARCH and  $N_t$ is modeled by LSTM-GARCH. Rather than predicting $y_t$ directly, the statistics $h_t$ is applied with LSTM to predict  $N_t$. In the example case that   LSTM-GARCH is used,  the hybrid model is \rev{briefly} represented as

%\rev{Hybrid  model here takes advantage of both the decomposing and statics extracting method. In section 3.2, the LSTM-GARCH method utilizes the new statistics $h_t$ to help predict $y_t$; and in section 3.1, for the complete version of model, after decomposing the data, ARIMA-GARCH is used to deal with stable part and LSTM is used to deal with unstable part. After carefully thinking about the two methods above, we found a new method that can take full advantage of the  two methods. Take the LSTM-GARCH-I model in section 3.2 for example. First of all, still dividing the time series $y_t$ into stable part $S_t$ and unstable part $N_t$. $S_t$ is molded by ARIMA-GARCH first; then the $N_t$ (not $y_t$) are modeled by LSTM-GARCH.That is, we use the new statistics $h_t$ to help predict  $N_t$ instead of $y_t$. We call this model as Hybrid-LSTM-GARCH model, which is represented as follosw:}

%\begin{equation}
 %   y_t = S_t + N_t
%\end{equation}
%\begin{equation}
 %   N_t = LSTM(N_{t-1},N_{t-2},...,N_{t-k};h_{t-1},h_{t-2},...,h_{t-k})+\varepsilon_t
%\end{equation}

%Where $\varepsilon_t$ is the random error.
%\note{Why here you only predict one step, while in the previous section you predict multiple steps? The paper does not have consistent notations everywhere.}



%\subsection{\sout{\rev{Sequential neural network based on statistical statistics feature extraction II}}}
%\sout{Based on method , a hybrid LSTM-GARCH model will be builded up to predict $y_t$. The difference between method IV and method II is method IV will take the current value of $h_t$ into consideration to predict $y_t$,} \note{Why?} that is,
%\begin{equation}
    %y_t = LSTM(y_{t-1},y_{t-2},...,y_{t-n};h_t,h_{t-1},h_{t-2},...h_{t-n}) + \varepsilon_t
%\end{equation}
%\sout{For the multiple steps prediction, method IV can be expressed as:}
%\begin{equation}
    %{y_t,y_{t+1},...,y_{t+m}} = LSTM(y_{t-1},y_{t-2},...,y_{t-n};h_{t+m},...,h_t,h_{t-1},h_{t-2},...h_{t-n}) + \varepsilon_t
%\end{equation}

%\sout{Take one step prediction for example, GARCH values $h_t$ is a single time series like $y_t$. ARIMA-GARCH model was built up first to $y_t$; then make prediction for $h_t$ by LSTM model. Next, the current GARCH value $h_t$ was added into LSTM-GARCH to predict $y_t$, which is shown in (23).} \note{At the beginning of this section you need to introduce problem and motivation and design consideration.}
\input{Experiment Analysis}
\section{Conclusion}
We have presented a neural performance rendering system to generate high-quality geometry and photo-realistic textures of human-object interaction activities in novel views using sparse RGB cameras only. 
%
Our layer-wise scene decoupling strategy enables explicit disentanglement of human and object for robust reconstruction and photo-realistic rendering under challenging occlusion caused by interactions. 
%
Specifically, the proposed implicit human-object capture scheme with occlusion-aware human implicit regression and human-aware object tracking enables consistent 4D human-object dynamic geometry reconstruction.
%
Additionally, our layer-wise human-object rendering scheme encodes the occlusion information and human motion priors to provide high-resolution and photo-realistic texture results of interaction activities in the novel views.
%
Extensive experimental results demonstrate the effectiveness of our approach for compelling performance capture and rendering in various challenging scenarios with human-object interactions under the sparse setting.
%
We believe that it is a critical step for dynamic reconstruction under human-object interactions and neural human performance analysis, with many potential applications in VR/AR, entertainment,  human behavior analysis and immersive telepresence.




%
% May all your publication endeavors be successful.
%
%\hfill mds
%
%\hfill August 13, 2002


%\subsection{Subsection Heading Here}
%Subsection text here.
%
%\subsubsection{Subsubsection Heading Here}
%Subsubsection text here.
%
% Reminder: the "draftcls", not "draft", class option should be used if
% it is desired that the figures are to be displayed while in draft mode.

% An example of a floating figure using the graphicx package.
% Note that \label must occur AFTER (or within) \caption.
% For figures, \caption should occur after the \includegraphics.
%
%\begin{figure}
%\centering
%\includegraphics[width=2.5in]{myfigure.eps}
%\caption{Simulation Results}
%\label{fig_sim}
%\end{figure}


% An example of a double column floating figure using two subfigures.
%(The subfigure.sty package must be loaded for this to work.)
% The subfigure \label commands are set within each subfigure command, the
% \label for the overall fgure must come after \caption.
% \hfil must be used as a separ ator to get equal spacing
%
%\begin{figure*}
%\centerline{\subfigure[Case I]{\includegraphics[width=2.5in]{subfigcase1.eps}
%\label{fig_first_case}}
%\hfil
%\subfigure[Case II]{\includegraphics[width=2.5in]{subfigcase2.eps}
%\label{fig_second_case}}}
%\caption{Simulation results}
%\label{fig_sim}
%\end{figure*}



% An example of a floating table. Note that, for IEEE style tables, the
% \caption command should come BEFORE the table.Table text will default to
% \footnotesize as IEEE normally uses this smaller font for tables.
% The \label must come after \caption as always.
%
%\begin{table}
%% increase table row spacing, adjust to taste
%\renewcommand{\arraystretch}{1.3}
%\caption{An Example of a Table}
%\label{table_example}
%\begin{center}
%% The array package and the MDW tools package offers better commands
%% for making tables than plain LaTeX2e's tabular which is used here.
%\begin{tabular}{|c||c|}
%\hline
%One & Two\\
%\hline
%Three & Four\\
%\hline
%\end{tabular}
%\end{center}
%\end{table}


%\section{Conclusion}
%The conclusion goes here.
%
% conference papers do not normally have an appendix

% use section* for acknowledgement
%\section*{Acknowledgment}
% optional entry into table of contents (if used)
%\addcontentsline{toc}{section}{Acknowledgment}
%The authors would like to thank...
%The authors thank Tom LaPorta, Srinivas Kadaba, Ganesh
%Sundaram for useful discussions on the topic.


% trigger a \newpage just before the given reference
% number - used to balance the columns on the last page
% adjust value as needed - may need to be readjusted if
% the document is modified later
%\IEEEtriggeratref{8}
% The "triggered" command can be changed if desired:
%\IEEEtriggercmd{\enlargethispage{-5in}}

% references section
% NOTE: BibTeX documentation can be easily obtained at:
% http://www.ctan.org/tex-archive/biblio/bibtex/contrib/doc/

% can use a bibliography generated by BibTeX as a .bbl file
% standard IEEE bibliography style from:
% http://www.ctan.org/tex-archive/macros/latex/contrib/supported/IEEEtran/testflow/bibtex
%\bibliographystyle{IEEEtran.bst}
% argument is your BibTeX string definitions and bibliography database(s)
%\bibliography{IEEEabrv,../bib/paper}
%
% <OR> manually copy in the resultant .bbl file
% set second argument of \begin to the number of references
% (used to reserve space for the reference number labels box)
%\begin{thebibliography}{1}

%\bibitem{IEEEhowto:kopka}
%H.~Kopka and P.~W. Daly, \emph{A Guide to {\LaTeX}}, 3rd~ed.\hskip 1em plus
%  0.5em minus 0.4em\relax Harlow, England: Addison-Wesley, 1999.
%
%\end{thebibliography}

%\bibliography{s,neg,WGRID,latex8}

%\bibliography{SparseSensing,MyPaper,zhou_add_2}
%\bibliography{MCS,falsedata,SparseSensing,MyPaper}
% \bibliography{MCS,falsedata,SparseSensing,MyPaper,references}
%\bibliography{falsedata,SparseSensing,MyPaper,references}
%\bibliography{SparseSensing,MyPaper}
%\bibliography{SparseSensing,MyPaper,zhou_add_2}
% that's all folks
\bibliography{references_all}
\end{document}


