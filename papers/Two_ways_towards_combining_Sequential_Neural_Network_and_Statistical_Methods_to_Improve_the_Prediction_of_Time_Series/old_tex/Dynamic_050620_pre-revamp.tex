%\section{Learning of Dynamics Process }
\section{\textcolor{red}{Learning Out-of-Equilibrium Dynamics} }

\begin{wrapfigure}[9]{r}{2.9in}
  \vspace{-30pt}
  \begin{center}
\includegraphics[width=2.9in]{images/speckle}
  \end{center}
  \vspace{-20pt}
  \caption{\footnotesize Time-series of speckle pattern.}
\label{speckle}
  \vspace{-10pt}
\end{wrapfigure}

If the distribution of matter or disorder within a sample fluctuates in time, the X-ray speckle pattern generated by coherent scattering from the sample will also fluctuate. XPCS relates fluctuations
in the speckle pattern to the dynamics within the sample via the normalized intensity-intensity correlation
function:

\begin{equation}
g_2(\vec{Q},t)=\frac{\mathopen{<}I(\vec{Q},t')I(\vec{Q},t'+t) \mathclose{>}}{\mathopen{<}I(\vec{Q})\mathclose{>}^2}= 1+\beta\mathopen{<}E(\vec{Q},0)E(\vec{Q},t)\mathclose{>}^2/I^2=1+\beta\abs{f(\vec{Q},t)}^2,
\label{g2}
\end{equation}
where $\beta$ is the optical contrast, $f(Q,t)$ is the normalized intermediate scattering function (ISF) that measures the density fluctuations, \textcolor{red}{$t$ is the lag time, and $\vec{Q}$ is the scattering vector (wavevector-transfer)}.
 %and can also provide connections with theoretical calculations of material properties.
% Non-equilibrium dynamics can be probed by using a two-time correlation function $g_2(\vec{Q},t_1, t_2)$.
% ISF is found to be able to follow various forms, such as $e^{-D(Q)Q^2t}$ with $D(Q)$ a length scale dependent diffusion constant, a double-exponential decay of the form $f(Q,t) = (1 - \alpha)e^{−\Gamma_1 t} + \alpha e^{−\Gamma_2 t}$ with $−\Gamma_1$ the short-time decay constant and $−\Gamma_2$ the long-time decay constant, or a stretched exponential of the form $e^{-(\Gamma_1 t)^v}$ where $v$ is a stretching factor.




%Intensity correlation has been widely applied in along with materials scanning with XPCS.
%The dynamic structure factor $g_2$ provides the information about how a material responds to perturbations close to equilibrium.

\begin{wrapfigure}[13]{r}{1.9in}
  \vspace{-30pt}
  \begin{center}
\includegraphics[width=1.9 in]{images/2tc}
  \end{center}
  \vspace{-20pt}
  \caption{\footnotesize 2-time Correlation Function in out-of-equilibrium system}
\label{2tc}
  \vspace{-20pt}
\end{wrapfigure}

%The time auto correlation functions of the speckle fluctuations (Fig.~\ref{speckle}) reveals the  information of the density correlations in the sample and their time dependencies.
Fig.~\ref{speckle} shows the time series of speckle pattern. The normalized intensity-intensity  correlation function $g_2$ is widely applied to quantify the sample dynamics.  When the system is under equilibrium, $g_2$ in Equ.~\ref{g2} depends only on the time difference $t$, but not the sample “age” $t'$. The one-time correlation functions are related to the dynamic structure factor of the system and provide information such as the \sout{time and spatially dependent} relaxation time of the nanoscale dynamics within \sout{ the nano-composite system} 
\textcolor{red}{a nanocomposite at a given age}
.  
\textcolor{blue}{[MF: mentioning ``time- and spatially dependent'' in the preceding sentence is distracting and confusing. Time or age dependence will be covered by the following sentence. Leave the spatial heterogeneity to the following section.]}
When the system is not in equilibrium, however, two-time correlation function (2TCF) $g_2(\vec{Q},t_1, t_2)$ will be used (Fig~\ref{2tc}), which can be evaluated as \sout{aged}
\textcolor{red}{evolving or age-dependent} one-time correlation functions, 
\textcolor{red}{with each} corresponding to a specific time point in the experiment.




 %As the internal structure or the environment of the materials vary, it will also impact the parameters of the model. In addition, the actual measurement data often deviate from the theoretical model.

 Exponential-based models such as Kohlrausch function~\cite{Kohlrausch} and double decay function are commonly applied to represent $g_2$ function\sout{, where the parameters are time independent}. \textcolor{blue}{[MF: Too much detail without context; unclear what ``parameters'' refer to; ``time independent'' applies only to equilibrium cases. \textbf{Bring Eq. 2 up here}.]} \sout{Despite} 
 \textcolor{red}{Although} the models provide a good guidance on the pattern of $g_2$, the actual \textcolor{red}{model-}parameter value \textcolor{red}{(e.g., $\alpha$ in Eq. 2)} could change over time during a \sout{dynamic} 
 \textcolor{red}{non-equilibrium}
 process, and can be impacted by materials type and factors such as temperature and concentration. Simply modeling a process with a fixed form or parameter will be subject to failure \textcolor{blue}{[MF: maybe ``prone to failure'' is more appropriate.]}. In addition, the data values from actual measurements often deviate from the theoretical model. It is crucial to better track the dynamics of materials to more accurately predict the evolution \sout{process}
 \textcolor{red}{behavior}, classify the materials
 \textcolor{red}{dynamics}, and guide for more efficient experiment\textcolor{red}{s}.




%In light of the problems, in this project, we propose to investigate along two research thrusts: 1) Model driven data aware machine learning to accurately track the evolution of the dynamic structure factor, and 2) Automatic experiment with multi-dimensional sparse measurements.

\begin{wrapfigure}[9]{r}{1.6in}
  \vspace{-25pt}
  \begin{center}
\includegraphics[width=1.6 in]{images/Model_Data.png}
  \end{center}
  \vspace{-20pt}
  \caption{\footnotesize Model-driven data-aware learning of dynamics.}
\label{mdal}
  \vspace{-20pt}
\end{wrapfigure}

\textcolor{blue}{[MF: In Fig.~\ref{mdal}, replace ``(e.g. Intensity correlation'' with ``e.g., $g_{2}$ function''.]}

%However, for the non-equilibrium system like Gaseous or liquid systems, everything is changing over time. The inner particles of the samples keep doing the Brownian motions and sometimes fluctuate a lot.
 To better understand the dynamics of materials, we would like to explore a {\em model-driven data-aware machine learning} method to track the evolution of the dynamic structure factor $g_2$. To explain our method,  we have $g_2$  represented with a Kohlrausch function as an example, where  a stretched ($0<\alpha<1$) or compressed ($\alpha>1$) exponential-based model is applied:
%\begin{equation}
%g_2(\vec{Q},t) = 1+ \beta \abs{f(\vec{Q},t)}^2 = 1 + \beta e^{-2{(\Gamma(\vec{Q}) t)}^\alpha}, \hspace{0.01in} \Gamma(\vec{Q}) = D_0 q^2.
%\end{equation}
\begin{equation}
g_2(\vec{Q},t) = 1+ \beta \abs{f(\vec{Q},t)}^2 = 1 + \beta e^{-2{(t/\tau_{0}(\vec{Q}))}^\alpha}, %\hspace{0.01in} \Gamma(\vec{Q}) = D_0 q^2.
\end{equation}
%\begin{equation}
%\Gamma(\vec{Q}) = D_0 q^2
%\end{equation}
%
\textcolor{blue}{[MF: Note Fig. 2 was modified.]}
\textcolor{red}{where $\tau_{0}(\vec{Q})$ is the characteristic time scale of the relevant dynamics (e.g., for Brownian dynamics, $\tau_{0} \sim q^{-2}$ with $q = |\vec{Q}|$),}
\sout{Both the  momentum transfer coefficient $q$}
%and $\alpha$ is a stretched or compressed exponential
\sout{and $\alpha$  can be a function of the time $t$} 
 \textcolor{red}{and $\alpha$ can vary with age $t_{age}$}. ML-based methods have proven to be powerful tools to learn the time-varying parameters. Nikravesh~\cite{Nikravesh1996Model} demonstrates that neural networks in conjunction with recursive least squares can be used to effectively learn the parameter of a model for the nonlinear time variant processes. Applying the update method to the non-isothermal continuous stirred tank reactor with time-varying parameters achieves good performance 
 \textcolor{blue}{[MF: What does non-isothermal...reactor have anything to do with ML/NN here? Good performance in/of what? I think the preceding sentence can be removed.]}. Even with the learning of model parameters,  actual measured data still often deviate from the theoretical model. Rather than only relying on the model, to more accurately represent the $g_2$ function, we propose to explore  a {\em model-driven data-aware}  method by concurrently exploiting the model-based learning and pure data driven learning with a neural network (Fig~\ref{mdal}). More specifically, we can first apply
the exponential-based model with time-varying parameters {\em learnt} to fit the model part. We then compute the difference between the model and the actual data measured for training. If the loss is beyond some preset acceptable threshold, we will introduce a suitable neural network method to learn the difference part. In the actual application, the trained exponential-based model and  network will be applied to quickly predict the dynamic process.
%Finally, combining the two part to get more accurate results.
We expect that our proposed method can fully take advantage of the existing $g_2$ models to approximate the dynamic process, and the emerging ML/DL methods 
\textcolor{blue}{[MF: DL is undefined. Just replace ``ML/DL'' with ``deep learning''?]} to capture the non-linear components of the data. Compared to pure data driven machine learning, the incorporation of the  model part helps better interpret the overall process, and greatly reduce the training time.
%The advantage of this method is it can give full play to the ability of analysis based on the model and the ability to process non-linear information based on the drive of the data. This method is not only improving the fit prediction but increase the interpretation of the model.
This method is general and can be  extended to many application fields.



%As a second method(Fig~\ref{mdal}b), we would like to apply a neural network to find the relationship between the intensity and quantified dynamic fluctuation inside the materials by learning the evolution of the intensity directly,  where the intensity measured at a time slice is a multi-dimensional function itself. The correlation of intensity can be calculated  to find the dynamics of the materials. As the intensity at different time slices will form a time serials, a neural network such as RNN, LSTM and GRU  can be used to capture the change of the intensity and make a good prediction of the dynamic process.
%When data is lacking, this approach can help us get more  predicted intensity, which can be feed into the previous DNN model for training.



%instead of intensity-intensity correlation function to acquire the fluctuations in the speckle pattern, we propose a new method to capture the intensity change and specify the dynamic process of materials. In our proposed method, we use Neural Network to find the relationship between intensity and quantified dynamic fluctuation inside the materials.


