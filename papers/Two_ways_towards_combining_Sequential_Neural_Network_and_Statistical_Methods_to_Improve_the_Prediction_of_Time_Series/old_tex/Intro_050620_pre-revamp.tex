\section{Introduction}
\vspace{-0.15in}

%Long-term monitoring is often needed for surveillance and system management. For example, radar sensors can be placed over rooftops to protect against rogue drones from approaching the surrounding area.
%A network of drones could be applied to monitor a domain to support applications such as coastal patrol and mine protection.

Modern synchrotrons enable in-situ/operando experiments across a wide range of disciplines like physics, chemistry and material science. Commonly the phase space of such an experiment is larger than the phase space that can be simultaneously sampled. While AI and ML based autonomous experiments are drawing attentions for efficiently sampling the phase space, especially in ex-situ experiments (i.e. static samples), 
experiments covering out-of-equilibrium phenomena present an additional challenge, due to the time-dependent nature of the sample. Many out-of-equilibrium processes are irreversible, i.e. each point in time is only available once for a measurement for any given sample and adding a different measurement at the same time point requires re-creating the sample conditions or the sample itself. In such a situation, it is of paramount importance to conduct measurements following an ‘Optimum Data Collection Trajectory’ (ODCT) to maximize the value of data collected during the continuous evolution of the sample.
%, while potentially employing different experimental techniques in a multi-modal approach.

In this seed project, we propose to develop fundamental techniques that can facilitate more efficient and intelligent experiments. We will explore emerging machine learning (ML) and sparse data sampling techniques for accurately modeling out-of-equilibrium processes and finding ODCT. As a specific use-case, we will apply X-ray Photon Correlation Spectroscopy (XPCS) to Additive Manufacturing (‘3D printing’) of polymer nanocomposites. 
\textcolor{blue}{[MF: Cite Lutz' recent papers on XPCS/3D printing/nanocomposites here: doi: 10.1080/08940886.2019.1582285 ; doi: 10.1021/acs.langmuir.9b00766 ; doi: 10.1063/1.5141488 ; doi: 10.1016/j.mtphys.2020.100220 ]}
Our research will fall into two thrusts: 1) Model-driven data-aware machine learning to accurately track the 
evolving dynamics at a given material location, and 2) Active sparse scanning for efficiently probing spatially-varying dynamics that occurs in 3D printing processes.
%Determining ODCT with multi-dimensional sparse measurements.

 Beyond the specific use case, the principle of ODCTs will be applicable to a wide range of experiments, such as spectroscopy imaging with X-ray nanoprobes, and can be generalized to many out-of-equilibrium experiments where the parameter phase space is larger than what can be probed simultaneously.



%X-ray diffraction is an important tool as structural probe for condensed matter on atomic length scales. Since its early beginning, it allows the structural determination of a wide variety of materials. Inherent in conventional X-ray diffraction experiment is an ensemble averaging process over the illuminated volume of the sample. Such an averaging process is often desirable as it allows atomic-scale quantities to be measured over a region of micro- to millimeters with a great statistic precision. However, the incoherent averaging process leads to a loss of information, which is not desirable when samples are non-periodic or evolve over time.

%Understanding and creating new materials increasingly require measurements of not only the time averaged or instantaneous properties of the material but also its kinetic and dynamic behavior. X-ray photon correlation spectroscopy (XPCS) provides such information at the nanoscale and below by characterizing fluctuations in condensed matter across a broad range of length and time scales. The spatiotemporal range provided by XPCS today with the ongoing development of diffraction-limited storage rings will enable studies \cite{doi:10.1146/annurev-matsci-070317-124334} of nucleation and precipitation and growth, coarsening, eutectic solidification and spinodal decomposition, dealloying, dewetting, solute trapping, electro migration, and many other processes that are vital for understanding fundamentals of materials phase transformation, materials processing and synthesis.

%\vspace{0.06in}
%\noindent{{\bf{Problem 1:} } If the distribution of matter or disorder within a sample fluctuates in time, the X-ray speckle pattern generated by coherent scattering from the sample will also fluctuate. XPCS relates fluctuations in the speckle pattern to the dynamics within the sample via the normalized intensity-intensity correlation function:

%\begin{equation}
%g_2(\vec{Q},t)=\frac{\mathopen{<}I(\vec{Q},t')I(\vec{Q},t'+t) \mathclose{>}}{\mathopen{<}I(\vec{Q})\mathclose{>}^2}= 1+\beta\mathopen{<}E(\vec{Q},0)E(\vec{Q},t)\mathclose{>}^2/I^2=1+\beta\abs{f(\vec{Q},t)}^2,
%\label{g2}
%\end{equation}
%where $\beta$ is the optical contrast, $f(Q,t)$ is the normalized intermediate scattering function (ISF). The ISF provides a measure of density fluctuations within a material. Knowledge of the dynamic structure factor $g_2$ provides the information about how a material responds to perturbations close to equilibrium and can also provide connections with theoretical calculations of material properties. Non-equilibrium dynamics can be probed by using a two-time correlation function $g_2(\vec{Q},t_1, t_2)$. ISF is found to be able to follow various forms, such as $e^{-D(Q)Q^2t}$ with $D(Q)$ a length scale dependent diffusion constant, a double-exponential decay of the form $f(Q,t) = (1 - \alpha)e^{−\Gamma_1 t} + \alpha e^{−\Gamma_2 t}$ with $−\Gamma_1$ the short-time decay constant and $−\Gamma_2$ the long-time decay constant, or a stretched exponential of the form $e^{-(\Gamma_1 t)^v}$ where $v$ is a stretching factor.

%For a material, it needs to determine which model follow. Despite the models provide a good guidance for the trend of ISF and $g_2$, the actual parameter value could change over time during a dynamic process, and can be impacted by materials type and factors such as temperature and concentration. Simply modeling a process with a fixed form or parameter will be subject to failure. In addition, the data values from actual measurements often deviate from the theoretical model. It is crucial to better track the dynamics of materials to more accurately predict the evolution process, classify the materials, and guide for more efficient experiment.

%to more accurately follow the evolution process, classify the materials, and better guide the experiment.
%\vspace{0.06in}
%\noindent{{\bf{Problem 2:} } In a dynamic process, the model parameter may evolve over time, and the actual data often deviate from the model. In addition, the structure and properties of the majority of materials for industry and research depend on a large number of material composition and processing parameters. This makes it essential to explore and understand the composition-processing-structure-property relations of materials in their associated multi-dimensional parameter spaces. Each space is a combination of the parameters that affect an experiment, including synthesis and processing conditions, material composition, and environmental conditions during the experiment. Traditionally, the parameters of the experiment are changed  manually and interactively, where the next measurement parameters are determined based on the analysis of the recent and all prior results. This process is not only time and resource consuming, but is also insufficient or even infeasible to explore the vast, high-dimensional parameter spaces that underlies the complex materials.

%In light of the problems, in this project, we propose to investigate along two research thrusts: 1) Model driven data aware machine learning to accurately track the evolution of the dynamic structure factor, and 2) Automatic experiment with multi-dimensional sparse measurements.

%\vspace{0.06in}
%\noindent{{\bf External funding opportunities:}} Our research methods are general. Besides its application in materials scanning with XPCS, it can be extended to apply in many scientific fields that involve the modeling of time evolving system states as well as low cost yet fine grade measurements. We expect the collaboration of SBU and BNL will attract additonal funding from DOE, NSF and DOD.




