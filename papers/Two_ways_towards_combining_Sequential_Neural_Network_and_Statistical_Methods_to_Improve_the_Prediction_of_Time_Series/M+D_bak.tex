\section{Model-driven+Data-driven Learning}
\vspace{-0.10in}
In order to better model the data, fully explore the information of the data itself and enhance the interpret-ability of the model, some model-driven + data-driven approaches can be considered.
\subsection{Hybrid model-driven+data-driven methods I}
\vspace{-0.10in}
For a time series, one can divide into stable part and unstable part. The stable part refers to that can be modeled by the mathematical model like ARIMA and GARCH models; while the unstable part is that can't be modeled by those mathematical model. Further, the stable part can be divided into linear stable part and nonlinear stable part. Linear stable part is that can be explained by ARIMA model while the nonlinear stable part is that can be explained by GARCH model. Genrally, a hybrid ARIMA-GARCH and LSTM method can be used to model the  time series. However, ARIMA(p,d,q)-GARCH(P,Q)-LSTM model will reduce to ARIMA(p,d,q)-LSTM when there are only linear stable part and unstable part in the time series, that is (P=0,Q=0); and ARIMA-GARCH-LSTM model will reduce to GARCH(P,Q)-LSTM model if there is no linear stable part in the time series, that is (p=0,q=0).

\subsubsection{ARIMA-GARCH-LSTM}
\vspace{-0.10in}
A time series $y_t$ can be represented as:
\begin{equation}
     y_t = S_t + N_t = LS_t + NS_t + N_t
\end{equation}
where $S_t$ is the  stable part, $N_t$ is the unstable part, $LS_t$ is the linear stable part and $NS_t$ is the nonlinear stable part.  $S_t$ can be modeled by ARIMA-GARCH and $N_t$ can be modeled by $LSTM$; then $y_t$ can be expressed more formally as:
\begin{equation}
\begin{cases}
    y_t = \mu_0 + \widehat{LSTM} + u_t\\
    u_t = \sum_{i=1}^{p} {\alpha_i{u_{t-i}}} + e_t+ \sum_{j=1}^{q} {\theta_j{e_{t-j}}}\\
    e_t = z_t{\sqrt{h_t}} \quad  z_t\sim N(0,1)\\
    h_t = w+\sum_{j=1}^{P}{\lambda_j{e_{t-j}^2}}+\sum_{i=1}^{Q}{\beta_i{h_{t-i}^2}}
\end{cases}
\end{equation}
where $\mu_0$ is a constant and $\widehat{LSTM}$ represents the unstable part modeled by $LSTM$ method. 
                    
\subsubsection{ARIMA-LSTM}
\vspace{-0.10in}
If a time series just includes linear stable part and unstable part, the time series can be represented as:
\begin{equation}
     y_t = LS_t +  N_t
\end{equation}
where $LS_t$ can be modeled by ARIMA model. The equation above can be expressed as:
\begin{equation}
\begin{cases}
    y_t = \mu_0 + \widehat{LSTM} + u_t\\
    u_t = \sum_{i=1}^{p} {\alpha_i{u_{t-i}}} + e_t+ \sum_{j=1}^{q} {\theta_j{e_{t-j}}}\\
\end{cases}
\end{equation}
\subsubsection{GARCH-LSTM}
\vspace{-0.10in}
If a time series just includes nonlinear stable part and unstable part, the time series can be represented as:
\begin{equation}
     y_t = NS_t +  N_t
\end{equation}
where $NS_t$ can be modeled by GARCH model. The equation above can be expressed as:
\begin{equation}
    \begin{cases}
      y_t = \mu_0 +\widehat{LSTM}+ u_t \\
      u_t = z_t{\sqrt{h_t}}\quad z_t\sim N(0,1)\\
      h_t = w+\sum_{j=1}^{P} {\lambda_j{u_{t-j}^2}}+\sum_{i=1}^{Q}{\beta_i{h_{t-i}^2}}
    \end{cases}
\end{equation}
\subsection{Hybrid model-driven+data-driven method II: LSTM-GARCH}
 GARCH values $h_t$ estimated by ARIMA-GARCH model is an important statistics. $h_t$ reflects the volatility of the sequence at different times and is essential for describing and predicting sequence changes. In the hybrid LSTM-GARCH model, $h_t$ was added into the input of LSTM model, then using LSTM method to train and make prediction, which can be expressed as below:
\begin{equation}
                                                        y_t = LSTM(y_{t-1},y_{t-2},...,y_{t-n};h_{t-1},h_{t-2},...h_{t-n}) + e_t
\end{equation}
The model above is called as Hybrid LSTM-GARCH model.
\subsection{Hybrid model-driven+data-driven method III: Residual LSTM-GARCH}
Residual LSTM-GARCH model seems like the combination of Hybrid model-driven+data-driven method I and II. ARIMA-GARCH-LSTM model and LSTM-GARCH model. Still dividing the time series $y_t$ into stable part $S_t$ and unstalbe part $N_t$. $S_t$ is molded by ARIMA-GARCH first; then the residuals(not $y_t$) are modeled by LSTM-GARCH. The Residual LSTM-GARCH can be represented as follows:
\begin{equation}
    y_t = S_t + N_t
\end{equation}
\begin{equation}
    N_t = LSTM(N_{t-1},N_{t-2},...,N_{t-n};h_{t-1},h_{t-2},...,h_{t-n})+e_t
\end{equation}
\subsection{Hybrid model-driven+data-driven method IV: xxxx}
Based on method II, a hybrid LSTM-GARCH model will be builded up to predict $y_t$. The difference between method IV and method II is method IV will take the current value of $h_t$ into consideration to predict $y_t$,that is,
\begin{equation}
    y_t = LSTM(y_{t-1},y_{t-2},...,y_{t-n};h_t,h_{t-1},h_{t-2},...h_{t-n}) + e_t
\end{equation}
For the multiple steps prediction, method IV can be expressed as:
\begin{equation}
    {y_t,y_{t+1},...,y_{t+m}} = LSTM(y_{t-1},y_{t-2},...,y_{t-n};h_{t+m},...,h_t,h_{t-1},h_{t-2},...h_{t-n}) + e_t
\end{equation}
Take one step prediction for example, GARCH values $h_t$ is a single time series like $y_t$. ARIMA-GARCH model was built up first to $y_t$; then make prediction for $h_t$ by LSTM model. Next, the current GARCH value $h_t$ was added into LSTM-GARCH to predict $y_t$, which is shown in (23). 