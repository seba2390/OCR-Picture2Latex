\section{\protect \py and CKKW-L}
\label{sec:pythia}

There are several reasons for using \py to handle the collinear
resummation rather than \ariadne as was done in
ref.\@  \cite{Andersen:2011zd}. First of all, the handling of initial-state
radiation in \ariadne is somewhat peculiar \cite{Andersson:1988gp} and
does not quite fit into a conventional resummation
scenario. Furthermore, \py has a much more advanced infrastructure for
handling matching and merging. Finally, \py has a very advanced model
for multiple partonic scattering (based on ref.\@  \cite{Sjostrand:1987su}) which is
needed to have a realistic description of the underlying event.

\subsection{The Interleaved Shower in \py}
\label{sec:interl-show-py}

\py implements a transverse-momentum-ordered shower
\cite{Sjostrand:2004ef}, which includes not only initial- 
and
final-state emissions, but also interleaves these with multiple
partonic scatterings. The general philosophy is that emissions (or
sub-scatterings) with high transverse momentum should always be
performed before those with lower transverse momentum.

As in all parton shower algorithms the ordering is used to ensure that
the probability for any emission remains finite, and that the whole
shower process is unitary. Even though the splitting functions for an
emission diverge for small transverse momenta according to
$P(k_\perp^2,z)\propto1/k_\perp^2$, at each step of the shower the
basic splitting probabilities are amended by the probability that no
splittings with larger transverse momenta had happened before.
The probability that the \emph{hardest} emission occurs at the scale 
$k_\perp^2$ with an energy splitting $z$ is given by:

\begin{align}
  \label{eq:hardest}
  \frac{d\mathcal{P}_{\mbox{\scriptsize hardest}}}{dk_\perp^2dz}=&
  P(k_\perp^2,z)
  \exp\left\{-\int_{k_\perp^2}^{{k^2_\perp}_{\max}}dk_\perp^{'2}
    \int dz' P(k_\perp^{'2}, z')\right\} \nonumber \\
    \equiv &
  P(k_\perp^2,z)\Delta({k^2_\perp}_{\max},k_\perp^2). 
\end{align}
Here the Sudakov factor  $\Delta({k^2_\perp}_{\max},k_\perp^2)$ corresponds to the no-emission probability,
ensuring that there
were no other emissions between the maximum scale ${k^2_\perp}_{\max}$ and $k_\perp^2$. 
A lower cutoff,
${k^2_\perp}_\mathrm{cut}$, is still needed but can be taken very small and
still result in probabilities below unity.
Formally, \cref{eq:hardest} resums the leading double-logarithms of ${k^2_\perp}_{\max}/k_\perp^2$
in the soft-collinear limit in the leading colour (large $N_c$) approximation. 
% The choice of ordering variable $k_\perp^2$ in principle only affects terms 
% at subleading logarithmic accuracy. 
It should be noted however that many formally subleading contributions, such as momentum conservation,
which in practice give rise to large effects are also included.


The no-emission probabilities are fairly easily implemented using the
Sudakov veto algorithm \cite{Sjostrand:2006za}, and
has simple factorisation properties if several different types of
emissions are possible, due to the exponential form. The ordering variable, $k_\perp$, used 
in the evolution is not necessarily the actual transverse momentum of an emission in any
Lorentz frame, and it is defined slightly differently depending on the class of emission
in the interleaved shower. For final-state radiation -- \textit{FSR} -- (or time-like splittings) it is
defined as ${k^2_\perp}_\mathrm{FSR}=z(1-z)Q^2$, where $Q^2$
is the invariant mass of the two final-state partons. For initial-state radiation -- \textit{ISR} --
(or space-like splittings) we instead have ${k^2_\perp}_\mathrm{ISR}=(1-z)Q^2$, where now $Q^2$ is the virtuality of
the incoming parton entering the hard sub-system after the emission. Finally for
multi-parton interactions (\textit{MPI}), ${k^2_\perp}_\mathrm{MPI}$ is simply
defined as the transverse momentum in the lab system for the $2\to2$
scattering.

\subsection{Merging \`a la CKKW(-L)}
\label{sec:merging-a-la}

The partonic states generated by a parton shower are exclusive;
in other words, the probability to produce an $n$-parton state in the parton shower is approximately given by the exclusive
cross section for exactly $n$ partons.
This is in contrast to
$n$-parton states generated by a matrix element generator,
where the state is exactly given by the inclusive cross
section for having \emph{at least} $n$ partons. The main principle
of algorithms that merge matrix elements with parton showers is
therefore to take several inclusive samples with different numbers of
partons from a matrix element generator and reweight them with
no-emission probabilities to make them exclusive. 
This allows the
samples to be safely added and subsequently showered 
without any double-counting.

The general idea in this paper is to use \HEJ as a matrix element
generator and add emissions from \py in a consistent way. In doing so
we will use ideas from the CKKW-L merging algorithm
\cite{Lonnblad:2001iq,Lavesson:2005xu,Lonnblad:2011xx}, but with some
important modifications which will be described in \cref{sec:matching}.
Here we shall review the pertinent features of the CKKW-L method. 

Similarly to
merging algorithms such as CKKW \cite{Catani:2001cc} and
MLM \cite{Mangano:2006rw}, the CKKW-L method takes matrix-element-generated states
and tries to reconstruct a sequence of emission scales from which the
no-emission probabilities are calculated. While some merging procedures use
jet clustering algorithms to do this, CKKW-L looks at the partonic states and
tries to answer the question \textit{``How would my parton shower have
  generated this state?''}, and then reconstructs the full kinematics of
the corresponding sequence of emissions in the parton shower. Often
there is more than one sequence of emissions possible, in which case
one sequence is chosen at random with relative weights given by 
the product of the values of the corresponding splitting
functions. The sequence chosen is referred to as the parton shower history
and will comprise of a complete set of intermediate states,
$\{S_0,\ldots,S_n\}$ (where $S_0$ is the lowest multiplicity state and
$S_i$ has $i$ additional partons) and a series of $n$ parton shower
emissions. Each emission $i$ is characterised by an ordering scale
${k^2_\perp}_i$, a splitting fraction $z_i$, and an azimuthal angle,
$\phi_i$. This procedure differs from the standard
  CKKW algorithm, where the intermediate states are not needed and
  instead only the emission scales are calculated by the
  $k_\perp$ jet-clustering algorithm. Formally this difference only affects
  sub-leading logarithms.

The no-emission probabilities are then
calculated by generating trial emissions from each intermediate state in turn, starting at
$S_0$. The emission generated from $S_i$ will have a maximum
scale given by ${k^2_\perp}_i$. The probability that this emission has a
scale above ${k^2_\perp}_{i+1}$ is exactly the no-emission probability
$\Delta_i({k^2_\perp}_i,{k^2_\perp}_{i+1})$. Giving the
matrix-element-generated state a weight zero if a trial
emission from a given state $S_i$ has a scale above ${k^2_\perp}_{i+1}$
is therefore equivalent to reweighting the cross
section by the no-emission probability:
\begin{equation}
  \label{eq:noem}
  \prod_{i=0}^n\Delta_i({k^2_\perp}_i,{k^2_\perp}_{i+1}).
\end{equation}
Here ${k^2_\perp}_0 $ is the maximum possible
scale and corresponds to the scale of the Born level process; ${k^2_\perp}_{n+1}\equiv {k^2_\perp}_M$ is the merging
scale which is given by the cut used in the matrix element
generator, and is used to isolate the region of soft and collinear divergences
where the parton shower is assumed to give a better description.

We can now freely add more partons below the merging scale with our
parton shower.  For the case that $n=N$ is the maximum multiplicity of
the matrix element samples to be merged, trial emissions from $S_N$
are not performed and the last factor
$\Delta_N({k^2_\perp}_\mathrm{N},{k^2_\perp}_{N+1})$ is omitted. (This
is because there is no possibility of double-counting with states of
higher multiplicity, $n>N$.)  Consequently the shower is instead
started from ${k^2_\perp}_N$.

In addition to the reweighting of the cross section by the no-emission probability,
there is also a reweighting of the value for \as\ used in the matrix element,
typically evaluated at some fixed renormalisation scale $\mu_R$ characteristic of the Born level process. For a parton
shower resummation, however, it can be shown that true collinear
logarithms are better reproduced if \as\ is evaluated at the
scale of the individual shower splittings. The states
are therefore reweighted by the factor: 
\begin{equation}
\frac{1}{\as^n(\mu_R^2)}\prod_{i=1}^n \as({k^2_\perp}_i).
\label{eq:alpharew}
\end{equation}
Additionally there is a reweighting with PDFs,
related to the fact that the no-emission probabilities contains PDF
ratios, as explained in more detail in \cite{Lavesson:2005xu}.
The end result is that the merged event sample will be constructed by
exclusive partonic states where the $N$ hardest emissions above the
merging scale are given by the full tree-level matrix element, and all
softer emissions are given by the shower.

So far we have only considered initial- and final-state showers, but
to get a realistic description we also need to consider the underlying
event. This cannot be described by a tree-level matrix element, but it
may be accounted for in an interleaved parton shower using MPI.
This means that we also want to incorporate the MPI ``emissions''
from the shower in the merged event sample. 
The underlying event may actually contain hard
jets, and it is impossible to separate these from the jets given by
the matrix element generator. Therefore we cannot blindly include MPI
emissions in the CKKW-L no-emission probabilities above, but the
procedure is modified \cite{Lonnblad:2011xx} as follows.

As before we reconstruct a parton shower history for every matrix
element state. We make trial emissions \textit{including MPI} 
from each intermediate state $S_i$ for $i<n$,
giving the event a weight zero if the emission scale is above ${k^2_\perp}_{i+1}$.
The last state $S_n$ is treated separately. 
When an emission is generated above ${k^2_\perp}_M$, if it corresponds to either
ISR or FSR we still give the event a weight zero; if
however an MPI is generated we will accept the generated state and
continue the shower below the emission scale rather than below the merging
scale.
The end result is thus changed such that the merged event sample will now
consist of exclusive partonic states where the $N$ hardest emissions
above the merging scale \emph{that are not from an MPI} are given by
the full tree-level matrix element, and all softer emissions are given by
the shower. 

  We note that merging procedures such as CKKW-L are not necessarily
  unitary, in that the inclusive lowest multiplicity Born level cross
  section is not preserved, as is the case in parton showers. This is
  because of the mismatch between the ratio of full matrix element
  describing the addition of a parton and the splitting function used
  in the no-emission probabilities. This is in contrast to
  \emph{matching} procedures (see \eg~\cite{Bengtsson:1987rw}) where
  it is the matrix element ratios that are exponentiated in the
  no-emission probabilities.

%
% \todo[inline]{Somewhere, explain that merging methods *do not*
%   preserve the cross section (that's matching), but rather are
%   focussed on getting the amount of real radiation correct.  Indeed,
%   virtual corrections are missing.}
%
% \todo[color=green,inline]{Well, they do intend to preserve the
%   \textit{Born-level} crosssection, and that we should maybe comment on. Leif to describe.}

%%% Local Variables:
%%% mode: latex
%%% TeX-master: "main"
%%% End:
