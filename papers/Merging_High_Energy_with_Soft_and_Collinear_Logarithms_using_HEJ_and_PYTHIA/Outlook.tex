\section{Summary and Outlook}
\label{sec:outlook}

We have introduced a new CKKW-L-inspired merging algorithm for combining the
all-order summation of high energy logarithms in \HEJ with a parton shower.  
For the first time
\HEJ events have been fully evolved down to particle level using the modern
parton shower, hadronisation and modelling of MPI in \py. The merging
algorithm systematically combines the dominant perturbative
corrections due to hierarchies of transverse momenta (i.e.~of soft and
collinear origin) from the parton shower with those due to large invariant
masses between jets of similar transverse momenta, as implemented in \HEJ or
BFKL.

The performance of the merging algorithm was assessed by considering observables which measure the additional radiation in the rapidity interval between two tagging jets. 
Many of the observables measured so far have (intentionally or not) a hierarchy of transverse scales
induced, and so require a systematic resummation of the
logarithms from the parton shower in order to arrive at a satisfactory
description. For such observables we find that 
the description of \HEJpy is consistent with standalone \pyt and data.
The improvement upon \HEJ in such distributions is notable. 
In addition, an investigation of related observables but with more inclusive cuts demonstrated that 
the jet multiplicity for large rapidity intervals is increased relative to
\pyt through merging. A measurement of such clean observables can serve as
test of high energy evolution.
These results demonstrate that we have combined effects originating from both parton shower and from \HEJ, providing a proof of concept for this method.

% For the average number of jets as a function of the rapidity separation of the tagging jets, where a $p_T$ hierarchy is induced by the disparity of cuts on jet transverse momentum,
% both high energy and soft-collinear logarithms become important. Though neither \HEJ nor \pyt could fully describe the data, \HEJ+\pyt was consistent with data.
% Furthermore, an investigation of related observables but with more inclusive cuts revealed the jet multiplicity for large rapidity intervals is increased relative to \pyt through merging. 
% These results demonstrate that we have combined effects originating from both parton shower and from \HEJ, and indicate that we have ensured there is no double-counting of emissions. 
% This work therefore provides a proof of concept for this method.

Notwithstanding what has been so far achieved, what has been presented constitutes a first attempt at merging \HEJ with a parton shower. We envisage numerous 
refinements that can be made. There is a need to implement a prescription for
incorporating full fixed-order matching into the merging procedure and the
inclusions of sub-leading partonic channels (non-FKL) to the \HEJ resummation\cite{Andersen:2017kfc}. 
In particular this will have an impact upon which states may be inserted by the parton
shower. A systematic
inclusion of such events in the prescription would not require dramatic changes to the algorithm. Firstly, the definition of what constitutes a \HEJ state would need to be extended;
secondly, the appropriate tree-level matrix elements should be used when calculating the veto probability of trial emissions. 
Nevertheless, the impact upon the observables discussed in this
paper should be relatively modest; this was assessed by studying the relative size of the 
contributions of fixed-order non-FKL events in pure \HEJ.

As discussed in \cref{sec:ModifiedMergingAlgo}, a limitation to the method is that only
the hardest emission of the shower received subtractions in their associated splitting kernel.
This limitation could be addressed by re-inserting \HEJ emissions in those events that were modified by 
\pyt above the merging scale, at the appropriate evolution scales reconstructed during the parton shower history.
However, such a procedure has several ambiguities, such as where in the (modified) colour flow the emission should be
inserted, and precisely how the recoils should be performed. Preliminary studies indicate the effect of
reinserting \HEJ emissions has a small effect, even upon the most sensitive observables shown in \cref{fig:avejetspure}.
However, we postpone a systematic study of these effects to a future publication. 

Finally, also noted  in \cref{sec:ModifiedMergingAlgo}, a more advanced treatment for the weighting of colour flows in \HEJ events
that is informed by the parton shower may be necessary. The impact of this last effect is not obvious, and its resolution will require further study. 

In this paper we considered the effects of our merging algorithm in pure dijet analyses. 
Partially this was due to the availability of data; in addition it is worthwhile to first consider the effects of 
a new method in a simpler environment where there is no expectation of new physics.
Nevertheless it  is also important to apply this method to processes other than those which are purely QCD.
Since one of the primary motivations was to assess the impact of jet vetoes
that are relevant for Higgs plus dijets studies, this process is the next natural arena for study. 
We emphasise that this should not require any significant modifications to the method; 
the task is primarily an exercise in software development, rather than a theoretical challenge.

Finally, although we chose to implement this method for \pyt, in 
principle it should be possible to implement for other parton showers.  
It would be interesting to compare the effect of merging \HEJ with different choices of parton shower. 
It would also be informative to perform the jet analysis with a much harder
jet threshold, such that the sensitivity to the tunings of the
non-perturbative effects are reduced. This would result in a much cleaner
comparison of the perturbative merits.

% It is intriguing that \pyt performs as well or better than the predictions for \pyt + POWHEG provided for the original experimental studies.
% There are parameters in \pyt which are tuned to data that facilitate the modelling of non-perturbative effects; one would not expect this to have an impact on 
% observables which measure the number of hard emissions. 
% It would therefore be worthwhile to investigate what is the impact of tuning on the observables we have considered, and determine whether this is consistent with what is expected for soft QCD. 

Although we have been able to draw many positive conclusions by comparing with experimental data, the 
cuts that were chosen are not conducive to examining the effect of high energy logarithms. Both these points entail that it is difficult to discriminate 
between theoretical predictions that model different physics and should be expected to differ. 
We hope that as more data is collected, future analyses will examine a
similar set of observables but with more inclusive cuts, as discussed.

This work has reinforced the notion that the interplay between different types of logarithms is not necessarily straightforward, and that there are circumstances under which
the combination of two all-order summations is necessary. We hope that the merging algorithm we have developed may be used in future as a tool to inform 
analyses what selection of cuts and observables are sensitive to parton shower effects, high energy effects, or both. 

%%% Local Variables:
%%% mode: latex
%%% TeX-master: "main"
%%% End:
