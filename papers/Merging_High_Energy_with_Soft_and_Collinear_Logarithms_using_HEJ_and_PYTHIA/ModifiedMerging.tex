\def\sigmab{\ensuremath{\sigma_2^{\star}}}

\section{Modified Merging for \HEJ}
\label{sec:matching}

In merging matrix elements with parton showers there are two primary challenges encountered, which we recapitulate here so as to compare the corresponding 
challenges in merging \HEJ with a parton shower. The first challenge is to ensure there is no double-counting between the fixed order matrix elements and the parton shower.
In fixed order merging algorithms, this is achieved through the merging scale, which provides a clear partition of phase space. Above the merging scale, the multiplicity
of hard jets should not be increased by the parton shower, and the distribution of hard jets should be determined by the fixed order matrix elements.
Below the merging scale, soft and collinear radiation from the parton shower is added, smearing the energy of the original hard partons, but leaving the 
original jets' energies largely unchanged.

We want the merging of \HEJ and \pyt to obtain the logarithmic accuracy of
both. Therefore, the parton shower should not change the jet multiplicity relative to 
\HEJ in the MRK limit (namely, at large rapidities with no transverse momentum hierarchies).
The parton shower should however be able to add collinear radiation inside the jet cone. 
% 2) Talk about slicing versus subtraction
One could envisage using a phase space slicing mechanism such that
regions populated by \HEJ and parton shower are not allowed to
overlap.
% (For example, the jet cone radius could be used).
However, in combining \HEJ with a parton shower 
we are aiming to correctly model the amount of radiation (for example, the multiplicity of jets)
in regions of phase space sensitive to both high energy and soft-collinear logarithms, which is hard to achieve with a strict partition.
Instead we will allow both
formalisms to populate their respective overlapping phase spaces and define a subtraction
term for the splitting functions and corresponding no-emission probabilities in the shower. 
Double-counting is then avoided by reducing
the probability of producing a certain emission in the shower by the probability that
\HEJ had already performed that emission. 
% This procedure will indeed be employed;
% however in practice it will be more efficient to recluster \HEJ partons below a given cut-off scale,
% and allow the parton shower to populate below that scale freely. 
% This scale will play the role of the merging scale in our prescription, 
% however it should be noted that its inclusion is not absolutely necessary to the procedure. 

The second challenge in fixed order
  merging is to avoid double-counting between the inclusive event
  samples that are combined, which is resolved by making those event
  samples exclusive through reweighting with Sudakov factors.  The
  picture for merging \HEJ with a parton shower is slightly different,
  because a given $n$-parton event generated by \HEJ is already
  exclusive. (This is ensured by the inclusion of virtual corrections
  at the level of the matrix elements to all orders in
  \cref{eq:MHEJ}.)  It would therefore be inappropriate to na\"ively
  reweight events with the Sudakov factors in \cref{eq:noem} whose kernels correspond to
  the full Altarelli-Parisi splitting functions.
%
Instead we have devised a procedure where
  collinear emissions from the shower are added to states produced by
  \HEJ in a way such that the corresponding collinear Sudakov form factors only
  change the relative weight of different \HEJ multiplicities, and retain the
  inclusive cross section. It does so by inserting emissions also in
  early stages of the reconstructed parton shower history to avoid
  \textit{under}-counting of collinear emissions due to phase space
  limitations set by the full generated \HEJ state, which was the main
  drawback of the previous approach in \cite{Andersen:2011zd}.

In \cref{sec:ckkw-l-hej} we will outline the merging procedure without specifying the particulars of how the division of phase space
between \HEJ and parton shower is achieved. We simply assume that there exists a consistent way to classify a given emission as being 
belonging to the either the \HEJ or parton shower regimes. 
We use the jet cone radius as an example of a cut 
to highlight some features of our algorithm.
Interpreting this statement in the language of the parton shower
implies that it is possible to define both \HEJ and collinear (or subtracted) Sudakov factors. 
We will then develop these ideas, in particular the
definition of and procedure to calculate these subtracted Sudakov factors in \cref{sec:subtracted}.
Finally, the algorithm will be disclosed in full in \cref{sec:ModifiedMergingAlgo}.

% inspired by the CKKW-L method that provides a 
% mechanism by which collinear radiation may be inserted. We shall describe this new approach in 
% following section; the algorithm will be disclosed in full in \cref{sec:ModifiedMergingAlgo}.

% \todo[inline]{
% In the intro here, explain that the aim is that in the MRK limit, merging algorithm should reproduce HEJ (e.g. at large rapidities with no transverse momentum hierarchies). 
% In particular, Pythia should not increase the multiplicty in this region.
% In collinear limit, should reproduce Pythia (e.g. inside jet cone).
% Improvement with respect to data in observables than neither can describe.}

% \st{In designing a merging algorithm for consistently adding collinear splittings
% to partonic events produced by \HEJ, in some sense we may think of \HEJ } 
% \st{as a standard tree-level matrix element, however with a crucial difference: 
% contrary to the case for any fixed-order generator (such as \madgraph}
% \st{) a given $n$-parton event generated  by \HEJ }
% \st{is already exclusive. This is ensured by 
% the inclusion of virtual corrections at the level of the matrix elements to all orders in} \cref{eq:MHEJ}.
% \st{It would therefore be inappropriate to na\"ively reweight events with the Sudakov factors whose
% kernel corresponds to the full Altarelli-Parisi splitting functions.
% Nevertheless, as indicated earlier, it is still desired that the events be corrected 
% to take into account the probability that the parton shower might have produced collinear emissions an an early stage in the reconstructed parton shower
% history. This shall correspond to instead reweighting by \textit{subtracted} Sudakov factors which we shall define later in} \cref{sec:subtracted}.
% \st{Moreover, we design a treatment inspired by the CKKW-L method that provides a 
% mechanism by which collinear radiation may be inserted. We shall describe this new approach in 
% following section; the algorithm will be disclosed in full in} \cref{sec:ModifiedMergingAlgo}.


\subsection{CKKW-L and \HEJ}
\label{sec:ckkw-l-hej}

A prescription for dressing \HEJ events with collinear radiation may be 
obtained in a manner analogous to how MPI were added to samples of tree-level
events in CKKW-L. To understand how this works, we first note that the MPI algorithm
could have been reformulated such that one first does a normal reweighting
with the no-emission probabilities \textit{excluding MPI} in the trial
emissions, and then go through the reconstructed states
a second time making trial emissions \textit{using only MPI}. In this second
round, starting from $S_0$, as soon as an MPI trial emission from
$S_i$ with a scale $k_\perp > {k_\perp}_{i+1}$ is found one simply
replaces the original $S_n$ state with the $S_i$ state plus the
additional generated MPI emission. The shower subsequently evolves from 
the MPI emission scale $k_\perp$.

In the analogous procedure for \HEJ, since the states are already
exclusive we completely skip the first round of reweighting, and
proceed directly to the adding of collinear splittings and MPI.  As
before this is done by first constructing the parton shower history,
however the reconstructed states should only correspond to
configurations which \HEJ could have generated. We will define such
`\HEJ states' more precisely in the next section.  This is followed by the generation
of trial emissions from each reconstructed state \textit{which \HEJ
  could not have done}, namely the collinear emissions and MPI. 

If a trial emission from state $S_i$ is generated that has 
a scale $k_\perp > {k_\perp}_{i+1}$, the original
$S_n$ state is replaced by the reconstructed state $S_i$ with the
additional trial emission. If the original event is replaced, the
shower is allowed to evolve freely from the scale $k_\perp$.  In such
a prescription, the $N$ hardest emissions that are neither collinear
nor correspond to an MPI are generated by \HEJ; everything else is
generated by the shower.  We also skip the reweighting of \as\ in
\cref{eq:alpharew}, since as discussed in \cref{sec:hej} the scale
used in \HEJ has been chosen to be characteristic of the event
topology.  We will however still use $\as(k_\perp^2)$ in the addition
of collinear emissions from the shower.

% \todo[inline,color=purple]{The logarithmic accuracy of both \HEJ and
%   \pyt can be included in the event sample by following the procedure
%   described here: First generate events with weights of events
%   according to Eq.~(\ref{eq:resumdijetFKLmatched}). These are
%   exclusive samples to the accuracy of \HEJ. The accuracy of the
%   samples is upgraded to include that of \pyt by allowing \pyt to
%   shower in all of its phase space with the procedure described in
%   this section. Specifically, double counting is avoided in the second
%   covering of phase space by subtracting from the \pyt splitting
%   function the contribution to this from \HEJ, as described in the
%   following.}  \todo[inline,color=purple]{In the following paragraph
%   we should mention that this is how \HEJ would be interpreted within
%   the shower formalism, but that this would not generate the accuracy
%   of \HEJ (and that is why we do not just use \pyt to generate \HEJ)}
\def\kti#1{\ensuremath{{k^2_\perp}_{#1}}}
% 
To illustrate how the
  algorithm works quantitatively we will in the following assume that
  there is a clean separation in phase space between the \HEJ states
  and the region where we want to dress the jets with collinear
  emissions from \pyt. For instance, we could imagine a simple phase space
  cut, where \HEJ states are required to have a $\Delta R$ between any two partons
  larger than some value, and the \pyt splitting functions are set to
  zero if they result in such states.

We start by reformulating the $n$-parton state produced by \HEJ within the
shower-formalism, written as a basic two-parton inclusive cross section
multiplied by a series of `\HEJ splittings' with decreasing values of
the scale reconstructed by the merging algorithm.
%
% \todo[color=green]{Maybe we should here instead explain what happen with \HEJ states that cannot be forced into an $\kti{i}$-ordered sequence of \HEJ states.}
%
% \sout{(This is why only \HEJ states
% are allowed in the construction of the parton shower history)}.  
The fact that
the \HEJ states are exclusive means we can write the cross section for an
$n$-parton state produced by \HEJ (that is, prior to merging) as:
\begin{equation}
  \label{eq:HEJ-history}
  d\sigma_n^H=
  d\sigmab \left(\prod_{i=3}^nP_{i\!-\!1}^H(\kti{i})\Delta_{i\!-\!1}^H(\kti{i\!-\!1},\kti{i})
    d\kti{i}\Theta(\kti{i\!-\!1}-\kti{i})\right)\times\Delta_{n}^H(\kti{n},\kti{M}).
\end{equation}
Here $P^H_i(k_\perp^2)$ is the splitting function for
emitting a parton at the scale $k_\perp^2$ from the state $i$
\textit{according to \HEJ}, integrated over the energy fraction $z$;
$\Delta_i^H(\kti{i},k_\perp^2)$ is the probability that there were no
`\HEJ-like' emissions from the state $i$ between the scales $\kti{i}$
and $k_\perp^2$.  Finally, $d\sigmab$ is the inclusive differential
cross section for the initial two-parton state. 

The shower-merging will have to construct all shower-histories, which could
have produced a given $n$-parton state from \HEJ. This ends up being the
time-consuming step for the high-multiplicity states produced by \HEJ. These
states can be of much higher multiplicity than the current limit experienced
with fixed-order matchings, where the shower-histories are also
reconstructed. In order to reduce the complexity of the shower history
reconstruction, we trim the parton-content of the high-multiplicity
states from \HEJ before they are passed to the shower. This is done removing any parton with a
transverse momentum smaller than a scale $k_{\perp M}$ from the event record
(and reshuffling the remaining momenta to absorb the transverse momentum thus
removed).  The effect of
introducing the trimming though is that the event record contains no partons
with transverse momenta less than $k_{\perp M}$. After the trimming, this
phase space is therefore left completely for the shower to populate, and the
trimming scale is thus the final scale for the last Sudakov factor in
Eq.~(\ref{eq:HEJ-history}).

The inclusion of trimming can speed up the merging significantly;
however, it should be emphasised that \textit{formally} we should consider the 
limit where $k_{\perp M}\to 0$. Nevertheless, as the weight of the event is kept unchanged,
as long as $k_{\perp M}$ is smaller than the scale of the jet threshold, any observable
based on jet momenta is only weakly dependent on this trimming if at
all. We will later (in figure \ref{fig:avejetspure})
investigate directly the numerical impact of the transverse scale used in the
trimming of the event record in passing events from \HEJ to \pyt, which
indeed is found to be insignificant even on the observables which are very
sensitive to the jet multiplicities of the events.

Before continuing,
some comments may be needed to clarify \cref{eq:HEJ-history}:
\begin{itemize}\itemsep 0mm
\item At this stage we do not need to know anything about $P^H$. The
  fact that we have a sequence of emissions means that we can describe
  it as a product of splitting functions accompanied by corresponding
  no-emission probabilities which are of the form given in
  \cref{eq:hardest}, even if the states were not produced that way by
  \HEJ.
\item In rare cases it is not possible to reconstruct an ordered
  history of shower emission. Such cases are handled by joining two
  (or more) subsequent steps into one, as described
  in~\cite{Lonnblad:2011xx}. Such unordered paths are by definition
  far the parton shower resummation regions and does not affect the
  logarithmic accuracy of the procedure.
\item In other rare cases, it is not possible to find intermediate
  states corresponding to \HEJ states. Again we treat these by joining
  several steps into one, so that the trial emissions always come
  from \HEJ-like states.
\item The total inclusive dijet cross section is given by $\sigmab$ and
  is not the basic tree-level $2\to2$ cross section. This is because
  \HEJ includes non-unitary corrections beyond leading order, and
  these we would like to preserve.
\end{itemize}
We shall now consider the different possibilities which may arise in the merging procedure described above.
In the first case, the original state generated by \HEJ is not replaced by one
generated by the shower. This will occur only if at each reconstructed state in the history,
no trial (non-\HEJ-like) emission is generated above the scale ${k_\perp}_{i+1}$ of the next reconstructed state.
This corresponds to multiplying \cref{eq:HEJ-history} by the following product of no-emission factors:
\begin{equation}
\label{eq:collinear_sudakovs}
\Delta^C_n(\kti{n},\kti{M}) \cdot \prod_{i=3}^n \Delta^C_i(\kti{i\!-\!1},\kti{i}),
\end{equation}
where $\Delta^C_i$ is a suitably modified (collinear) Sudakov
  factor, for example, corresponding to the exponentiation of a \pyt splitting
  function with the $\Delta R$ cut assumed above. 
Defining
$\Delta^M_i=\Delta^H_i\Delta^C_i$, such events will contribute to the cross section according to:
\begin{equation}
  \label{eq:mergems}
  d\tilde{\sigma}_n=
  d\sigmab \left(\prod_{i=3}^nP_{i\!-\!1}^H(\kti{i})\Delta^M_i(\kti{i\!-\!1},\kti{i})
    d\kti{i}\Theta(\kti{i\!-\!1}-\kti{i})\right)\times\Delta^M_n(\kti{n},\kti{M}).
\end{equation}
Furthermore it is clear that we can freely dress these states with full \pyt
splittings below the merging scale.

If instead a trial emission is generated from a reconstructed state
$m\le n$ at the scale $\kti{C}>\kti{M}$ (and above the scale of the
next reconstructed state) the original $n$-parton state from \HEJ will
be replaced by the reconstructed $m$-parton state plus the accepted
trial emission. Calculating the contribution to the cross section from
such states requires summing and integrating over all possible
reconstructed \HEJ emissions below $\kti{C}$,
\begin{align}
  \label{eq:mergeall}
  d\tilde{\sigma}_{m/n}=&d\sigmab \left(\prod_{i=3}^mP^H_{i\!-\!1}(\kti{i})
    \Delta^M_{i\!-\!1}(\kti{i\!-\!1},\kti{i})d\kti{i}\Theta(\kti{i\!-\!1}-\kti{i})\right)
  \times\nonumber\\
  &\times P^C_m(\kti{C})\Delta^M(\kti{m},\kti{C})d\kti{C}\Theta(\kti{m}-\kti{C})\\
  &\times \sum_{j=m+1}^{n} \int_{\kti{M}}^{\kti{C}}\text{all \HEJ emissions}, \nonumber
\end{align}
where $P^C_m$ is the $\Delta R$-truncated \pyt splitting function. For
$n=m$ there is no integral and we just get the probability that there
are no extra emissions. For $n=m+1$, we get the probability that there
is exactly one extra emission, for $n=m+2$ exactly two, etc. The sum
of all these must necessarily sum up to unity, and the final result is
just the two first lines of \cref{eq:mergeall}.

In practice there is an upper cut, $N$, on the parton multiplicity in
\HEJ. Assuming that the corresponding cross section is inclusive over
the last emission, the last integral in the third line of
\cref{eq:mergeall}, becomes
\begin{equation}
  \label{eq:mergelast}
  \int_{\kti{M}}^{\kti{N\!-\!1}}d\kti{N}
  P^H_{N\!-\!1}(\kti{N})\Delta_{N\!-\!1}^H(\kti{N\!-\!1},\kti{N})=
  1-\Delta_{N\!-\!1}^H(\kti{N\!-\!1},\kti{M}),
\end{equation}
where we have used the property of no-emission probabilities that its
derivative is simply itself times the splitting function,
$\frac{d}{dk_\perp^2}\Delta_{i}(\kti{i},k_\perp^2)=P_{i}(k_\perp^2)\Delta_{i}(\kti{i},k_\perp^2)$. When adding this to
the $N-1$ contribution, this will explicitly cancel the last
no-emission factor there, and we can do the last integral in the $N-1$
contribution in the same way, and so on, until we cancel also the last
no-emission factor in the $m=n$ contribution.

Adding full \pyt shower splittings below \kti{C}, we can now write the
exclusive probability that we have exactly $n$ partons above the merging
scale as
\begin{eqnarray}
  \label{eq:mergingtotal}
  d\tilde{\sigma}_{n}&=&
  d\sigmab \left(\prod_{i=3}^nP_{i\!-\!1}^H(\kti{i})\Delta^M_i(\kti{i\!-\!1},\kti{i})
    d\kti{i}\Theta(\kti{i\!-\!1}-\kti{i})\right)\times\Delta^M_n(\kti{n},\kti{M})\nonumber\\
  &+&\sum_{m=3}^{n-1}d\sigmab 
  \left(\prod_{i=3}^mP^H_{i\!-\!1}(\kti{i})
    \Delta^M_{i\!-\!1}(\kti{i\!-\!1},\kti{i})d\kti{i}\Theta(\kti{i\!-\!1}-\kti{i})\right)
  \times\nonumber\\
  & &\qquad\times P^C_m(\kti{C})\Delta^M(\kti{m},\kti{C})d\kti{C}\Theta(\kti{m}-\kti{C})\\
  & &\qquad\times \left(\prod_{i=m+1}^nP^P_{i\!-\!1}(\kti{i})
    \Delta^P_{i\!-\!1}(\kti{i\!-\!1},\kti{i})d\kti{i}\Theta(\kti{i\!-\!1}-\kti{i})\right)
  \times\Delta^P_n(\kti{n},\kti{M}),\nonumber
\end{eqnarray}
where $P^P$ is now the full \pyt splitting function (possibly also
including MPI) and $\Delta^P$ the corresponding no-emission
probability. Comparing this with what \pyt would give on its own,
\begin{equation}
  \label{eq:justpythia}
  d\sigma_n^P=
  d\sigmab \left(\prod_{i=3}^nP_{i\!-\!1}^P(\kti{i})\Delta_{i\!-\!1}^P(\kti{i\!-\!1},\kti{i})
    d\kti{i}\Theta(\kti{i\!-\!1}-\kti{i})\right)\times\Delta_{n}^P(\kti{n},\kti{M}),
\end{equation}
we see that for $n$ partons above the merging scale the $m$ hardest
ones will always be produced by \HEJ, and if there are partons from
\pyt above the merging scale the hardest one of these will always be a
collinear splitting.
We also see that the procedure is unitary,
  in that the inclusive jet cross section is still given by $\sigmab$
  as calculated by \HEJ. All we have done is to add (unitary) parton
  shower emissions and, in some cases where these are harder than the
  \HEJ ones, reclustered the original \HEJ state into a lower
  multiplicity state, and then added the parton shower.

We note that the action of multiplying by \cref{eq:collinear_sudakovs}
was not present in the algorithm presented in \cite{Andersen:2011zd}
for matching \HEJ with \ariadne. That is to say, the probability that
the parton shower might have produced a collinear emission at an
earlier stage in the reconstructed history was not taken into
account. It was the lack of this step which allowed the inclusion of
soft gluons from \HEJ that interfered with the ordering of the parton
shower and prevented a proper parton shower evolution in the full
phase space. Furthermore these collinear emissions which according to
the parton shower should have occurred were not inserted; instead such
emissions could only be included \textit{below} the matching scale.
We emphasise that in this regard the approach we take here is
fundamentally different from how \HEJ was matched with \ariadne.



\subsection{The Subtracted Shower}
\label{sec:subtracted}

% In the procedure described in the previous section it was crucial that when trial emissions were generated between reconstructed \HEJ states
% this was 
% \todo[color=green,inline]{We start this section by explaining why we replace the phase
%   space slicing by the subtraction. Basically saying that where pythia
%   much larger than HEJ, we are either in the collinear resummation
%   region or in the large pt-hiearchy resummation region, and if HEJ is
%   much larger than pythia we are in the MRK resummation region. }
%
In the previous subsection it was assumed that we could make a simple
phase space cut between collinear splittings to be described by \pyt
and large angle splittings from \HEJ. However, the MRK-limit in \HEJ
does not take into account large logarithms that arise in case we have
large transverse momentum hierarchies between (possibly widely
separated) jets. Such logarithms are included in the parton shower,
and we would like to include them in our merging.

To accomplish this we go beyond a simple phase space cut and use
\textit{subtracted} splitting functions instead. The idea is that
where resummation is important the splitting functions are large, and
we could naively say that where the \pyt splitting function is less
than the \HEJ one we set it to zero, and vice versa. This would still
correspond to a simple phase space cut, albeit more complicated than
the $\Delta R$ cut assumed above. However, this would be fairly
wasteful as we would throw away many of the jets produced by
\HEJ. Instead we have introduced a procedure where the \HEJ splitting
function is subtracted from the \pyt one.

To do this it is now necessary to obtain an explicit definition of the
splitting functions and no emission probabilities for \HEJ.  Although
no such expressions appear explicitly within the \HEJ formalism, we
note that the Altarelli-Parisi splitting functions may be derived as
the soft and collinear limit of a ratio of matrix elements
\cite{Ellis:1991qj}:
\begin{equation}
  d k_\perp^2 dz  \int d\phi \frac{1}{16\pi^2}
  \frac{|\mathcal{M}^{n+1}|^2}{|\mathcal{M}^n|^2}\sim 
  \frac{d k_\perp^2}{k_\perp^2} dz \frac{ \alpha_s}{2\pi} P_{gg}(z)\;. \label{eq:ME2ratio}
\end{equation}
This is just the normal universal behaviour of
  matrix elements in the soft and collinear limit. The 
  Altarelli-Parisi splitting functions precisely capture the 
  soft and collinear singularities which must be exponentiated to
  calculate the leading DGLAP logarithms in the parton shower no emission probabilities.
  If instead we replace the full matrix elements by the \HEJ
  ones, this will no longer contain any collinear singularities, but only
  the soft singularities. Such a function is precisely what is needed
  to define a subtraction term for the parton shower. 
  Of course, we could take the MRK limit of this and retain
  the same logarithmic accuracy, but by using the full matrix elements we
  retain more of the \HEJ accuracy.

Therefore, as in the approach of \cite{Andersen:2011zd}, we define the \HEJ splitting function as a ratio of \HEJ matrix elements given by \cref{eq:MHEJ} corresponding to 
an event before and after the insertion of an emission as generated by the parton shower. Of course as noted in \cref{sec:hej}, 
these matrix elements are only valid for FKL configurations, but there is no restriction upon the kind of configuration which may be generated by the parton shower. 
We must therefore assert that the following criteria define a `HEJ state':
\begin{enumerate}\itemsep -0.5mm
 \item The most forwards outgoing parton should have the same flavour as the parton incoming along the positive z axis.\label{fklcondition1}
 \item The most backwards outgoing parton should have the same flavour as the parton incoming along the negative z axis. \label{fklcondition2}
 \item All other outgoing partons must be gluons. \label{fklcondition3}
 \item It must be possible to untangle the colour connections into two `ladders' of rapidity-ordered partons.\label{colcondition}
 \item The outgoing partons must cluster into at least two jets. \label{jetcondition1}
 \item Each extremal (most forwards or backwards) parton must be a member of the corresponding extremal jet.\label{jetcondition2}
 \item Each parton must have a transverse momentum above the merging scale ${k_\perp}_M$. \label{mergingscalecondition}
\end{enumerate}
Criteria \ref{fklcondition1}-\ref{fklcondition3} simply define an FKL
configuration; criterion \ref{colcondition} is required since a given
set of colour connections is chosen (as described in
\cref{sec:colo-conn-high}) and this has an impact upon which dipoles
arise in the \pyt parton shower; criteria
\ref{jetcondition1}-\ref{jetcondition2} are kinematic constraints on
\HEJ events.  Finally, although not strictly necessary for the purpose
of efficiency events generated by \HEJ containing soft emissions below
$\kti{M}$ are reclustered (in a manner that does not alter the
rapidities of the resulting jets) and this is therefore reflected by
the requirement given in condition \ref{mergingscalecondition}.  The
reclustering reduces the complexity of constructing all possible
shower histories for the state passed to \pyt. This reduces the CPU
time needed to obtained merged predictions, and explicit tests
indicate that the impact of the reclustering is unnoticeable on
observables based on hard jets, such as those studied in this paper.

Now, for any emission resulting in a configuration not corresponding to a \HEJ state we set
$P^H = 0$ (because there is nothing to subtract for non-FKL states),
and otherwise we can define:
\begin{equation}
P^H  = \frac{1}{2} \frac{1}{16 \pi^2} \frac{|\mathcal{M}^{n+1}_\mathrm{HEJ}|^2}{|\mathcal{M}^n_\mathrm{HEJ}|^2}\;.\label{eq:hej_split}
\end{equation}
The factor of $\frac{1}{2}$ accounts for the fact that the matrix elements are summed over all possible colour connections, but for each parton shower emission we wish to calculate the splitting function for
one of two possible choices (each of which contribute equally in the MRK limit). This expression however only accounts for time-like emissions. 
For a space-like branching $i\rightarrow jk$, where parton $j$ is evolved backwards to parton $i$ with a higher momentum fraction $x_i = (1/z) x_j$, we instead define:
\begin{equation}
\label{eq:hej_split_space}
P^H_\mathrm{spacelike} = \frac{1}{2}  \frac{1}{16 \pi^2}  \frac{|\mathcal{M}^{n+1}_\mathrm{HEJ}|^2}{|\mathcal{M}^n_\mathrm{HEJ}|^2} \, \frac{x_i f_i(x_i,\mu_F^2)}{x_j f_j(x_j,\mu_F^2)}\;.
\end{equation}
where the PDFs $f_{i,j}$ should be evaluated at an appropriately
chosen factorisation scale $\mu_F$.  Our effective Sudakov factor $\Delta^H$
from the previous section would then simply be the exponentiation of this
splitting kernel, however, we will not need to compute this explicitly in the numerical
implementation of the algorithm.
For completeness we shall also write down the \py
splitting functions, evaluated as a function of the evolution
variables $k^2_\perp$ and $z$:
\begin{equation}
 P^P(k^2_\perp,z) = \frac{\alpha_s}{(2\pi)^2} \frac{1}{k^2_\perp}P(z)\;,\label{eq:pythia_split}
\end{equation}
where $P(z)$ is the appropriate unregulated Altarelli-Parisi splitting
function. There is an additional factor of
$1/(2\pi)$ to average over azimuthal angle, since the matrix elements
in \cref{eq:hej_split} will be evaluated for a given choice of
azimuthal angle for the generated emission. In the case of a space-like
branching we modify \cref{eq:pythia_split} to include the ratio
of PDFs for the branching, as in \cref{eq:hej_split_space}. 


 To illustrate the differences between the
  \HEJ and \pyt splitting functions, in \cref{fig:splitPlots} we show
  their typical behaviour as a function of (\subref{fig:splitDR}) the angular distance between the
  emitted gluon and the nearest parton, $\Delta R$, and (\subref{fig:splitkt}) the transverse
  momentum of the emitted parton in the lab frame, $p_\perp$. What is
  shown is the average value of the splitting functions in the first
  emission in \HEJ-generated $qQ\rightarrow qQ$ events, excluding
  factors of $\alpha_S$ and ratios of PDFs.
  In \cref{fig:splitDR} we average over emissions with $p_\perp>10$\GeV. 
  The
  discontinuity in the \HEJ splitting function at the jet radius is an
  artefact of the regularisation procedure used for emissions inside the
  jet cone of extremal jets \cite{Andersen:2011hs}. 
  This is in any case the
  phase space region we want to populate with collinear shower
  splittings.

  \begin{figure}[t]
\centering
\begin{subfigure}{0.495\linewidth}
%     \includegraphics[width=\linewidth]{figures/SplitPlots_qQ/AveSplit_DR}
        \includegraphics[width=\linewidth]{AveSplit_DR}
    \caption{}
    \label{fig:splitDR}
  \end{subfigure}
  \begin{subfigure}{0.495\linewidth}
%     \includegraphics[width=\linewidth]{figures/SplitPlots_qQ/AveSplit_kt}
        \includegraphics[width=\linewidth]{AveSplit_kt}
    \caption{}
    \label{fig:splitkt}
  \end{subfigure}
  \caption{Plots comparing the average splitting function for \HEJ and \pyt for the first parton shower
  emission in \HEJ-generated $qQ\rightarrow qQ$ events.}
  \label{fig:splitPlots}
\end{figure}
  
For the $p_\perp$ plot we average of all
  emissions with $\Delta R>0.6$. We clearly see that for small
  $p_\perp$ the \pyt splitting function exceeds the \HEJ one (and
  also the analytic MRK-limit splitting shown for comparison). This
  is the region of large transverse momentum hierarchies, where the
  MRK approximation fails to properly resum the corresponding
  logarithms. Such large logarithms are present and are resummed by
  \pyt, and we would therefore like to add such splittings even if
  they are far away from the collinear region.

  From this we see that it makes sense to use a simple phase space cut
  based on the relative sizes of the splitting functions. 
  However, instead we can go one step further and in regions where 
  $P^P>P^H$ we subtract the \HEJ splitting function from the \pyt one.
  There we define the subtracted \pyt splitting function as

  \begin{equation}
\label{eq:subtracted_splitting_function}
P^S(k^2_\perp,z)= \max\left(P^P(k^2_\perp,z) -P^H(k^2_\perp,z), 0\right).
\end{equation}
%
% \todo[inline]{Clarify in following paragraph why PC $>$ 0. Mention that it is
%   necessary for this to be the case in order to use a probabilistic
%   (Sudakov) vetoing algorithm. Note that if pythia splitting function
%   is small we are happy to take the HEJ answer. Later on explain what
%   the implications in the context of the algorithm are on doing this,
%   and that it is sensible.}
%
Here the arguments of $P^H$ are intended to be schematic. This is
intended to denote that having generated an emission with
corresponding evolution variables $k^2_\perp$ and $z$, and having
inserted this into the event with an appropriate recoil strategy, the
matrix element containing $n+1$ partons should be evaluated with the
resulting set of $n+1$ final-state (recoiled) momenta. 
% It is clear
% that in the collinear regions we have $P^P>P^H$, and even away from
% these regions we can have enhancements of $P^P$ for the case
% where there is a large difference between the transverse momentum of
% the generated emission and the parton identified as the radiator.
% In regions having both a large transverse momentum hierarchy
% and large rapidity separations  we expect both DGLAP and BFKL
% resummation to be important, and the strategy is to simply to define
% their relative importance using the size of the respective splitting functions. In
% the pure FKL region, we expect $P^H>P^P$, and here we simply
% say that $P^C=0$.
With this notation we can now define the subtracted Sudakov factor:
\begin{equation}
 \label{eq:modsudakov_merging}
 \Delta^S(\kti{i},\kti{i+1})= \exp \left\{ - \int_{\kti{i+1}}^{\kti{i}} d k_\perp^2  \int dz \Theta(P^P-P^H) 
   \left[P^P(k^2_\perp,z) - P^H(k^2_\perp,z) \right] \right\}	.
\end{equation}
It should be clear that we then want $\Delta^M = \Delta^S \Delta^H$,
and in order that such a Sudakov factor might be employed during a
trial shower, it is sufficient to generate emissions according to the
full \pyt splitting function $P^P(k^2_\perp,z)$, but veto emissions
with probability
\begin{equation}
 \mathcal{P}_\mathrm{veto} = P^H(k^2_\perp,z) /P^P(k^2_\perp,z),
\end{equation}
in accordance with the Sudakov veto algorithm.

Armed with this we can go through the steps in \cref{sec:ckkw-l-hej}
again and arrive at exactly the same formulae except with $P^C$ and
$\Delta^C$ replaced by $P^S$ and $\Delta^S$. The net result is that in
phase space regions where $P^H>P^P$, where we believe \HEJ is doing a
good job, we never add any \pyt splittings, while in the complementary
region emissions are added in proportion to the subtracted splitting
function so that in total they will correspond to populating that
region only with \pyt splittings.



% We now briefly comment on the enforced positivity of
% \cref{eq:subtracted_splitting_function}.  Firstly, in order to
% generate the correct probabilty distribution for
% \cref{eq:modsudakov_merging} through the Sudakov veto algorithm (and
% where we generate according to $P^P(k^2_\perp,z)$), it is necessary
% that the we have $P^P>P^H$.  It is clear that in the collinear regions
% we have $P^P>P^H$, and such emissions will therefore be vetoed with a
% low probability.  In the pure FKL region, we expect $P^H>P^P$, so we
% simply say that $P^C=0$; emissions in this region will \textit{always}
% be vetoed.  Away from these regions we can have enhancements of $P^P$
% for the case where there is a large difference between the transverse
% momentum of the generated emission and the parton identified as the
% radiator.  In regions having both a large transverse momentum
% hierarchy and large rapidity separations we expect both DGLAP and BFKL
% resummation to be important, and the strategy is to simply to define
% their relative importance using the size of the respective splitting
% functions.

% To see how this will work in practice we show as illustration plots of the average splitting function $\langle P\rangle$
% for \HEJ and \pyt in \cref{fig:splitPlots}. This was evalutated for the first parton shower emission generated from a sample of $qQ\rightarrow qQ$ events
% (provided that this gives rise to a \HEJ state). Where the trial emission resulted in initial state recoil and ratios of PDFs were used, these factors were removed for a direct comparison of the splitting functions themselves. For the same reason, 
% due to potentially different scale choices for $\mu_R$ and conventions for the running of $\alpha_s$ in \HEJ and \pyt, these factors were also removed in the comparison. 

% In \cref{fig:splitDR} we show the average splitting function as a function of the angular distance between the emission and the nearest parton, $\Delta R$, for emissions having a transverse momentum greater
% than 10 GeV. The discontinuity in the average \HEJ splitting function at the jet radius is due to the fact that a different (regulated) matrix element is used for emissions which are inside the jet cone.
% Qualitatively we see that for (semi-hard) wide-angle emissions the \HEJ splitting function always exceeds that for \textsc{Pythia}, namely such emissions will always be vetoed in the parton shower.
% However, inside the jet cone, the splitting function for \pyt  greatly exceeds that for \HEJ, thus there is a very low probability that such emissions will be vetoed. For hard emissions, we can see that
% our subtraction method is equivalent to performing a phase space cut at the jet cone radius.

% \todo{choose a notation for transverse momentum that doesn't conflict with any existing definitions}
% In \cref{fig:splitkt} we show the average splitting function as a function of the (lab frame) transverse momentum $k_\perp$ of the trial emission, with the additional requirement
% that the angular separation of the emission from the nearest parton $\Delta R$ exceeds the jet radius ($R = 0.6$), namely it is a wide-angle emission.
% We also compare with the MRK limit of $C_A/(2\pi k_\perp^2)$.  We observe that all but the softest wide-angle emissions will be accounted for by \HEJ, and will be vetoed in the parton shower.
% However for soft emissions with $k_\perp \lesssim 10$ GeV we have $P^P>P^H$; therefore these emissions will not be vetoed, but will be treated with the appropriate subtracted
% splitting kernel.



\subsection{The Merging Algorithm}
\label{sec:ModifiedMergingAlgo}
For completeness we now explicitly disclose the full algorithm for merging \HEJ with \pyt as follows:
\begin{enumerate}
\item Generate samples of $n$-parton \HEJ states with $n \leq N$. Recluster any partons that have momenta above ${k_\perp}_{M}$ in such a way that the rapidities
of the resulting jets is unchanged. 

\item 
  For each $n$-parton state from \HEJ ($2<n\leq N$), reconstruct all possible \pyt
  shower histories where each clustering has the reconstructed scale $\kti{i}$, and set $\kti{n+1}$ $ = \kti{M}$. If $n=2$
  calculate the scale $\kti{2}$ and continue to step \ref{step:shower}, and otherwise continue as follows.
  \begin{enumerate}

  \item Throw away all histories that do not correspond to a sequence
    of \HEJ states.
  
  \item If there is at least one history that is correctly ordered in
    $\kti{i}$, throw away every other history.
  
  \item Give each history that is left a weight proportional
    to the \HEJ matrix element squared for the lowest multiplicity (\HEJ) state, times
    the product of \pyt splitting functions for the sequence of emissions 
    that gives the original $n$-parton state. Pick a history at random according to its relative weight.
    
  \item Starting from the most clustered state in the history, 
    make a trial emission from each intermediate state in the selected history
    starting from $\kti{i}$.

    \begin{enumerate}
     \item \label{step:trialemission} If the emission scale is below the reconstructed scale of the next state in the history, $\kti{i+1}$ ,
     continue to the next state in the history. If this is the original event we started with, 
     continue to step \ref{step:shower}.
     
     \item If the emission scale is above the reconstructed scale of the next state in the history,
     but has produced a \HEJ state, veto the emission with probability $P^H/P^P$.
     If the emission is vetoed, generate a new trial emission starting from the current
       emission scale, and return to \ref{step:trialemission}.
       If the emission is not vetoed replace the original event with this state 
       and continue to step \ref{step:shower}.
    \item If the emission scale is above the reconstructed scale and has not produced a \HEJ state, we
       substitute the original event with this state and continue to step \ref{step:shower}.
    \end{enumerate}  
  
  \end{enumerate}
  \item \label{step:shower}
  \begin{enumerate}
   \item If in the previous step we replaced the original event with one that could not have been produced by \HEJ,
    continue the shower from the emission scale of the new state without restriction.
   \item If this is the original event and we have $n < N$ start the shower from the reconstructed scale $\kti{n}$ and check the first emission. 
   If it gives a new  \HEJ state, discard the emission with probability $P^H/P^P$ and continue generating the first emission starting from 
   the scale $\kti{n}$.
   Once a first emission is accepted, the shower continues from the emission scale, radiating freely.
   \item If $n=N$, let the shower radiate freely from the scale $\kti{N}$.
  \end{enumerate}
  \item Once the parton shower has evolved below the cut-off scale, hadronise the event.
\end{enumerate}

This method represents one of the possibilities for merging \HEJ with a
parton shower. In particular, it retains the dijet cross section and
logarithmic accuracy of \HEJ: indeed, each event configuration and weight is
first generated by \HEJ, and all of phase space is thus covered. In the MRK
limits of similar transverse scales for all emissions, the Sudakov factors
introduced in equation~(\ref{eq:mergems}) all evaluate to unity, since
the scale used in the evolution of \pyt in the MRK limit tends to the
transverse scale of the lab frame. Since the MRK phase space is populated,
and the matrix elements are unchanged in this limit, the merging maintains the
logarithmic accuracy of \HEJ.

The logarithmic accuracy of \pyt is ensured since the full allowed phase
space in \pyt is covered, and the appropriate Sudakov factors between
emissions are applied in the shower evolution, with the possibility of
generating further emissions from the shower evolution. Such emissions are
then vetoed with a probability that said emissions were also generated from
\HEJ, such that double-counting is avoided.

For completeness we here mention three potential issues with the
algorithm. One limitation of the proposed method is that only the hardest
emissions (as ordered by \pyt) will be merged to \HEJ, which is not itself
ordered in hardness: it is possible for the parton shower to modify a state
classified as non-FKL (according to the momenta above the merging scale) to a
FKL state (accounted for in \HEJ) through an emission, and such emissions
will not have their splitting kernels subtracted. 
% \todo{Remove this
%   comment?}\todo[inline]{However, it is assumed that once an emission has
%   been performed in the shower domain, this will change the phase space
%   sufficiently that the \HEJ matrix elements will not be fully accurate.}
However, the non-FKL configurations account for a logarithmically suppressed
part of the cross section, quickly diminishing with increasing
rapidity \cite{Andersen:2017kfc}. Furthermore, future accounting for
next-to-leading logarithmic contributions in \HEJ will decrease further the
significance of the parton shower changing non-\HEJ to \HEJ states.


In addition, we reiterate that the method we are presenting is currently only applicable to FKL configurations. 
The impact of non-FKL corrections on the observables presented in this study is relatively small and within the indicated scale variations of the FKL results.
To include non-FKL configurations, it would be necessary to extend the definition of what constitutes a \HEJ state, and ensure that the appropriate tree-level matrix elements are used when calculating the veto probability for non-FKL states.
This is so that no problems arise from double-counting. Primarily such changes would affect what states may be included in the parton shower history, and which states may be inserted by the parton shower.

Finally we note that the factor one half in \cref{eq:hej_split} is based on
the fact that the colour flows which have a leading logarithmic
contribution will contribute equally to the colour-summed matrix element
squared in the MRK limit. This means that there will always be just two
possible colour flows for inserting a gluon in the exchange, and they will
have the same leading kinematic term in the MRK limit. While it is relevant
to take into account the different kinematic contribution from each possible
colour connection when matching full matrix elements to the parton shower,
the fact that the collinear divergences are absent from the formalism of \HEJ
means that the kinematic contributions from different colour connections
differ far less than in the full theory. While it would be possible to
account for the colour flow dependence in the contribution from the
sub-leading (and non-divergent) terms introduced in \HEJ compared to BFKL, we
choose in this study to assign an equal weight to the each of the possible
colour flows, just as will be the case in the MRK limit. This allows for the
simple attribution of $\frac 1 2 $ in \cref{eq:hej_split}.

% The parton
% shower might instead prefer one colour flow over the other possibility
% offered.

%  This turns out to be particularly so in the MRK limit. 
% \todo[color=green]{I added this to indicate that this is not a big
%   problem, but I'm not sure I'm being completely honest.}
% Even though in this limit normally we would have $P^H>P^P$  and therefore no
% \pyt splittings, it might be worth investigating a different
% weighting of the possible colour connections, inspired by the shower,
% or alternatively to further improve the behaviour of the shower
% splitting function itself in this limit, where the MRK-approximations
% to the colour flow are guaranteed to hold.

%%% Local Variables:
%%% mode: latex
%%% TeX-master: "main"
%%% End:
