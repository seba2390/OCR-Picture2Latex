\section{The High Energy Jets Monte Carlo}
\label{sec:hej}

\subsection{The High Energy Jets Formalism}
\label{sec:hejformalism}
The framework of \emph{High Energy Jets} (\HEJ)~\cite{Andersen:2009nu,Andersen:2009he,Andersen:2011hs} provides an
approximation to the perturbative hard scattering matrix
elements for jet production to any order in the strong coupling. The results are
exact in the limit of large invariant mass between all particles. The
formalism is inspired by the high energy factorisation of matrix elements (as
pioneered by
BFKL~\cite{Fadin:1975cb,Kuraev:1976ge,Kuraev:1977fs,Balitsky:1978ic}), and
obtains a power series in $\hat s$ for the square of the scattering matrix
elements. 
Within \HEJ, approximations are only applied to the matrix elements. This is different to the framework
of BFKL, where numerous kinematic approximations are applied in order to cast
the cross section in the form of a two-dimensional integral equation.
The highest power in $\hat s/p_t^2$ from the square of the matrix element gives the leading-logarithmic
contribution (in $\hat s/p_t^2$) to the cross section. Logarithmic
corrections additionally arise from virtual corrections. Recently it was shown that 
some next-to-leading contributions may be reached within \HEJ \cite{Andersen:2017kfc} 
by including so-called unordered emissions, which have the square of the matrix elements suppressed by one
power in $\hat s$ compared to the leading flavour-configuration for the same
rapidity-ordered momenta. However, in the present study we consider only the leading-logarithmic
contributions to the cross section, where only certain Fadin-Kuraev-Lipatov~\cite{Kuraev:1976ge} 
(FKL) partonic configurations contribute. 


We will now discuss in more detail the features of \HEJ that are
relevant to the construction of an algorithm for merging with a parton shower. The all-order
perturbative treatment of  $pp\to jj$ scattering in \HEJ starts with an
approximation to the tree-level amplitude for the scattering
process $f_1 f_2\to f_1 g\cdots gf_2$, where the final-state
particles are listed according to their ordering in rapidity, and $f_1, f_2$ can be
quarks, antiquarks or gluons. These are the FKL configurations that
give rise to the leading contribution to the inclusive $n$-jet cross section in the
\emph{Multi-Regge-Kinematic} (MRK) limit (see ref.\@ \cite{Andersen:2017kfc} for a
recent discussion of the power-suppression of other partonic contributions to
the same multi-jet process). The MRK limit can be specified as the limit of 
large rapidity separations between all particles, for fixed transverse momentum scales:
\begin{align}
 \label{eq:MRKlimt}
\forall i:&&  y_1 \gg \dots \gg y_{i-1}\gg y_i\gg \dots \gg y_n;&& {p_\perp}_i \sim p_\perp
\end{align}
It should be noted that the existence of large transverse momentum
hierarchies is not compatible with the MRK limit, which
will be of importance later.

The
$2\to n$ scattering amplitude is approximated at lowest order by the following
expression~\cite{Andersen:2011hs}:
\begin{align}
  \label{eq:multijetVs}
  \begin{split}
    \left|\overline{\mathcal{M}}^t_{f_1 f_2\to f_1g\ldots gf_2}\right|^2\ =\ &\frac 1 {4\
      (\Nc^2-1)}\ \left\|S_{f_1f_2\to f_1f_2}\right\|^2\\
    &\cdot\ \left(g^2\ K_{f_1}\ \frac 1 {t_1}\right) \cdot\ \left(g^2\ K_{f_2}\ \frac 1
      {t_{n-1}}\right)\\
    & \cdot \prod_{i=1}^{n-2} \left( \frac{-g^2 C_A}{t_it_{i+1}}\
      V^\mu(q_i,q_{i+1})V_\mu(q_i,q_{i+1}) \right),
  \end{split}
\end{align}
where $\left\|S_{f_1 f_2\to f_1 f_2}\right\|^2$ denotes the square of a pure
current-current scattering, $K_{f_1}, K_{f_2}$ are flavour-dependent
colour-factors (which can depend also on the momentum of the particles of
each flavour $f_1, f_2$, see ref.\@ \cite{Andersen:2011hs} for more
details); $q_i$ are the momenta of the colour-octets exchanged in the t-channel, and $t_i=q_i^2$.  The leading-logarithmic contribution to jet production beyond the
first two jets is given by
gluon emission from the underlying $2\to2$ process $f_1 f_2\to f_1 f_2$, and the effective vertex for gluon emissions takes the form \cite{Andersen:2009nu}:
\begin{align}
  \label{eq:GenEmissionV}
  \begin{split}
  V^\rho(q_i,q_{i+1})=&-(q_i+q_{i+1})^\rho \\
  &+ \frac{p_A^\rho}{2} \left( \frac{q_i^2}{p_{i+1}\cdot p_A} +
  \frac{p_{i+1}\cdot p_B}{p_A\cdot p_B} + \frac{p_{i+1}\cdot p_n}{p_A\cdot p_n}\right) +
p_A \leftrightarrow p_1 \\ 
  &- \frac{p_B^\rho}{2} \left( \frac{q_{i+1}^2}{p_{i+1} \cdot p_B} + \frac{p_{i+1}\cdot
      p_A}{p_B\cdot p_A} + \frac{p_{i+1}\cdot p_1}{p_B\cdot p_1} \right) - p_B
  \leftrightarrow p_n.
  \end{split}
\end{align}
This form of the effective vertex is fully gauge invariant; the Ward
Identity, $p_j\cdot V=0$ ($j=2,...,n-1$) can easily be checked, and is
valid for any values for the outgoing momenta $p_j$ (that is, not just in the MRK limit).

The virtual corrections to the amplitude for each multiplicity are
approximated in $D=4+2\varepsilon$ dimensions with the \emph{Lipatov
 ansatz} \cite{Balitsky:1978ic} for the $t$-channel gluon propagators (see
ref.\@ \cite{Andersen:2009nu} for more details). This is obtained by the simple
replacement
\begin{align}
  \label{eq:LipatovAnsatz} \frac 1 {t_i}\ \to\ \frac 1 {t_i}\ \exp\left[\hat
\alpha (q_i)(y_{i-1}-y_i)\right]
\end{align}
in \cref{eq:multijetVs}, where $y_i$ are the rapidities of the outgoing partons and
\begin{align} 
  \hat{\alpha}(q_i)&=-\gs^2\ \Ca\
  \frac{\Gamma(1-\varepsilon)}{(4\pi)^{2+\varepsilon}}\frac 2
  \varepsilon\left({\bf q}_i^2/\mu^2\right)^\varepsilon\label{eq:ahatdimreg},
\end{align} 
is the Regge trajectory which is regulated in $D=4+2\varepsilon$ dimensions, in which
$\mathbf{q}_i^2$ is the Euclidean square of the transverse components of
$q_i$. The cancellation of the poles in $\varepsilon$ between the real and
virtual corrections is organised using a subtraction term,
such that the regulated matrix elements used in the  all-order summation of \HEJ
are given by \cite{Andersen:2011hs}:
\begin{align}
  \label{eq:MHEJ}
  \begin{split}
    \overline{\left|\mathcal{M}_{\rm HEJ}^{\mathrm{reg}, f_1 f_2\to f_1 g
          \cdots g f_2}(\{ p_i\})\right|}^2 = \ &\frac 1 {4\
       (\Nc^2-1)}\ \left\|S_{f_1 f_2\to f_1 f_2}\right\|^2\\
     &\cdot\ \left(g^2\ K_{f_1}\ \frac 1 {t_1}\right) \cdot\ \left(g^2\ K_{f_2}\ \frac 1
       {t_{n-1}}\right)\\
     & \cdot \prod_{i=1}^{n-2} \left( {g^2 C_A}\
       \left(\frac {-1}{t_it_{i+1}} V^\mu(q_i,q_{i+1})V_\mu(q_i,q_{i+1}) \right. \right. \\
     &\left. \left. \phantom{\cdot \prod_{i=1}^{n-2} g^2 C_A \frac {-1}{t_it_{i+1}}}-\frac{4}{\mathbf{p}_i^2} \ \Theta\left(\mathbf{p}_i^2<\lambda^ 2\right)\right)\right)\\
     & \cdot \prod_{j=1}^{n-1} \exp\left[\omega^0(q_j,\lambda)(y_{j-1}-y_j)\right],
  \end{split}
\end{align}
where
\begin{equation}
\omega^0(q_j,\lambda)=\ -\frac{\alpha_s(\mu^2_R) \Ca}{\pi} \log\frac{{\bf q}_j^2}{\lambda^2}.
\end{equation}
and $\lambda$ is a regularisation parameter describing the extent of the
subtraction terms in the real emissions phase space.

Here $\alpha_s$ is evaluated using a renormalisation scale $\mu_R$, which
typically is chosen to reflect the momenta of the final-state
partons. Possible choices include half the scalar sum of transverse momenta
($H_T/2$) and the maximum jet transverse momentum (${p_T}_\mathrm{max}$).
Since the matrix elements have been regulated, this allows for a finite numerical approximation to the
all-order scattering amplitude to be constructed, and for this to be integrated
over all of phase space using a Monte Carlo approach (allowing for the application of arbitrary phase space cuts).
Just as for perturbative fixed-order calculations, the parton momenta in
Eq.~(\ref{eq:MHEJ}) are interpreted as arising from identifiable partons. An
NLO calculation of the production of dijets would deliver the \emph{exclusive} dijet
cross-section to order $\as^3$ and the \emph{inclusive} trijet cross-section
at the same order in $\alpha_s$. The perturbative result in Eq.~(\ref{eq:MHEJ}) contains real
and virtual corrections to any order, and the momenta and multiplicities
should all be considered exclusive (to the logarithmic accuracy of \HEJ).

Indeed, the all-order dijet cross section is  
simply calculated by explicitly summing the exclusive 
$n$-parton cross sections (calculated by numerically integrating the 
matrix elements squared from \cref{eq:MHEJ} over all of phase space)
over all numbers of gluon emissions from 
the initial scattering $f_1 f_2 \to f_1 f_2$.
In addition, matching to tree-level matrix
elements is performed by reweighting each exclusive $m$-jet event
with the factor:
\begin{align}
  w_{m-\mathrm{jet}}\equiv\frac{\overline{\left|\mathcal{M}^{f_1f_2\to f_1g\cdots
          gf_2}\left(\left\{p^\mathrm{new}_{\mathcal{J}_l}(\{p_i\})\right\}\right)\right|}^2}{\overline{\left|\mathcal{M}_{\mathrm{HEJ}}^{t,f_1f_2\to
          f_1g\cdots
          gf_2}\left(\left\{p^\mathrm{new}_{\mathcal{J}_l}(\{p_i\})\right\}\right)\right|}^2}.
\end{align}
This is just the ratio of the square of
the full tree-level matrix element (evaluated using \madgraph \cite{Alwall:2014hca})
to the approximation of this in \cref{eq:multijetVs}, both evaluated on
a set of shuffled momenta $p^\mathrm{new}_{\mathcal{J}_l}(\{p_i\})$ derived from the hard jets only. 
This procedure is summarised in the following formula:
\begin{align}
  \begin{split}
    \label{eq:resumdijetFKLmatched}
    \sigma_{2j}^\mathrm{resum, match}=&\sum_{f_1, f_2}\ \sum_{n=2}^\infty\
    \prod_{i=1}^n\left(\int_{p_{i\perp}=0}^{p_{i\perp}=\infty}
      \frac{\mathrm{d}^2\mathbf{p}_{i\perp}}{(2\pi)^3}\ 
      \int_{y_{i-1}}^{y_\mathrm{max}} \frac{\mathrm{d} y_i}{2}
    \right)\
    \frac{\overline{|\mathcal{M}_{\mathrm{HEJ}}^{\mathrm{reg},f_1 f_2\to f_1 g\cdots gf_2}(\{ p_i\})|}^2}{\hat s^2} \\
    &\times\ \sum_m \mathcal{O}_{mj}^e(\{p_i\})\ w_{m-\mathrm{jet}}\\
    &\times\ \ x_a f_{A,f_1}(x_a, Q^2_a)\ x_2 f_{B,f_2}(x_b, Q^2_b)\ (2\pi)^4\ \delta^2\!\!\left(\sum_{i=1}^n
      \mathbf{p}_{i\perp}\right )\ \mathcal{O}_{2j}(\{p_i\}),
  \end{split}
\end{align}
where $n$ is the partonic multiplicity of the final state, 
and the operator $\mathcal{O}^e_{mj}$ returns one if there are exactly $m$ jets, and zero otherwise.
We also define the inclusive dijet operator $\mathcal{O}_{2j} = \sum^\infty_{n=2} \mathcal{O}^e_{mj}$,
and require that the extremal partons from \HEJ are
members of the extremal jets, in order to ensure that the partonic
configuration matches the situation for which the \HEJ framework was
developed. 
The last line in \cref{eq:resumdijetFKLmatched} corresponds to the inclusion of
parton density functions (PDFs) and the momentum-conserving delta-functional. 
Finally we note that while the sum in the first line of \cref{eq:resumdijetFKLmatched}
is over all numbers of final-state partons, $2 \leq n < \infty$, in practice
the sum needs to include only a finite number of terms: for
finite rapidities and collider energies, the contribution beyond a certain
number of gluons is perturbatively suppressed. The upper bound
$N$ is chosen such that the results are insensitive to further
emissions. This check is performed by simply keeping track of the
contribution from each term in the series, and $N=22$ is sufficient for this study.
Nevertheless, for completeness this choice will enter into the merging algorithm described in \cref{sec:matching}.

The matching of \HEJ to tree-level accuracy is currently
performed up to four jets. The limit on the multiplicity is determined by the time taken
to evaluate the full expressions.  
In addition, the partonic configurations not conforming to the ordering described above
are included in \HEJ by simply adding the contributions order by order (again
using \madgraph \cite{Alwall:2014hca}), but no all-order summation is
performed for these \emph{non-FKL} configurations. 
In the current study, we
will focus on the FKL configurations, since 
this is where special attention is needed in order to avoid double-counting.
This is unlike the challenge addressed by typical fixed-order merging algorithms, 
because the description in \HEJ goes
beyond approximating leading-order matrix elements.
As previously discussed, application of the \emph{Lipatov ansatz} through \cref{eq:LipatovAnsatz} is used to 
sum to all-orders the leading-logarithmic virtual corrections to the $t$-channel poles. 
Although the approaches of \HEJ and \py are complementary and 
calculate different all-order contributions to the perturbative series, 
they cover overlapping regions of phase space, and the
combination of \HEJ and \py therefore requires a new merging algorithm.

% \todo[inline]{Talk about a naive slicing here, and then say that we go further and instead define a subtraction term. 
% This should have the property that in a pure shower region description should be mostly shower, and for HEJ regions mostly HEJ.}
% \todo[inline,color=purple]{Change terminology of 'subtraction term', to
%   distinguish from the subtraction term in Eq.~(\ref{eq:MHEJ}) to regularise
%   the \HEJ cross section.}
% \todo[color=purple]{Should we not move all discussion of the interplay
%   between \HEJ and\pyt to the section of
%   merging?}\todo[color=green]{Agreed.}\todo{Jeppe: Read through and remove}
% \textbf{By construction, the \HEJ framework provides a description directed
% particularly at hard, wide-angle QCD radiation. 
% However, although the subtraction term in \cref{eq:MHEJ} 
% suppresses the contribution from emissions
% with a transverse momentum less than $\lambda$,
% the transverse momenta of
% gluons emitted in-between the two extremal partons (in rapidity) can take on
% any value. Since the
% emissions of \HEJ populate all of phase space (in-between the rapidities of the
% scatterers with flavours $f_1$ and $f_2$), in principle we do not need to 
% define specific regions to populate with \HEJ and with the shower. 
% Instead we could allow both
% formalisms to populate their respective phase spaces and define a subtraction
% term for the Sudakov form factor of the shower. 
% Double-counting is thus avoided by reducing
% the probability of producing a certain emission in the shower by the probability that
% \HEJ had already performed the given emission. 
% This procedure will indeed be employed;
% however in practice it will be more efficient to recluster HEJ partons above a given cut-off scale,
% and allow the parton shower to populate below that scale freely. 
% This scale will play the role of the merging scale in our prescription, 
% however it should be noted that its inclusion is not absolutely necessary to the procedure. 
% }

% \textbf{
% The construction of an appropriate subtraction term also presents a challenge.
% Although \cref{eq:MHEJ} contains the exponential of the logarithm of a momentum,
% it is not directly related to a Sudakov factor of a normal parton shower ---
% firstly, the logarithm is not of the emitted momentum; secondly, the all-order
% summation is not formulated as a unitary evolution (i.e.~of constant total cross section). In
% order to perform merging with a parton shower, we have to \emph{define} the
% relation between the emissions of \HEJ and those of the shower. We will do this in 
% \cref{sec:subtracted}.
% }

We conclude this overview on \HEJ by reiterating that 
a parton shower framework such as \py\cite{Sjostrand:2014zea} is
necessary in order to evolve the partonic state of \HEJ to the state of
hadronisation, primarily by populating the partonic state with further soft
and collinear radiation. In order to obtain the logarithmic accuracy of the shower, it
should also populate (with the appropriate probability) any region between
disparate transverse scales,
which might be generated by \HEJ. Since the shower, as well as the subsequent string
hadronisation, relies on well-defined colour connections between
partons, we now briefly discuss the colour connections arising in \HEJ.

\subsection{The Colour Connections of High Energy Jets}
\label{sec:colo-conn-high}
The colour-ordered Parke-Taylor amplitudes~\cite{Parke:1986gb} for tree-level
$gg\to g\cdots g$ scattering allow for a very neat
analysis~\cite{DelDuca:1993pp,DelDuca:1995zy} of the dominant colour
configurations in the limit of widely-separated, hard gluons. The
conclusion, as presented in references \cite{DelDuca:1993pp,DelDuca:1995zy}, is
that the leading contribution in the MRK limit is provided by the colour configurations which can be untwisted
into two \textit{non-crossing} ladders that connect the rapidity-ordered
gluons. \Cref{fig:Colours} (left) contains an example of a configuration
contributing in the MRK limit, and one (right) which is suppressed. The
numbering of the final-state partons is assigned according to their rapidity; as drawn their vertical ordering also 
coincides with their ordering in rapidity.


\input ColourConnect
The colour connections in \cref{fig:Colours} (left) can be summarised as
$a134b2a$. It is possible to arrange the final-state partons such that no colour lines cross 
without modifying the vertical order of the final-state particles, namely by moving particle 2
to the left side of the same plot. Since the vertical ordering is unchanged,
the rungs of the resulting
un-crossed ladders are also ordered in rapidity.
Such manoeuvres are always possible when the order of the particles 
in the colour connection string between the two
initial-state gluons $a \dots b$ and $b \dots a$  
also reflects their order in rapidity, as in the case of $\{134\}$ and $\{2\}$
in $a134b2a$. 

The colour connections in \cref{fig:Colours} (right) can be
summarised as $a1324ba$; in this case the string $\{1324\}$ between
 $a$ and $b$ is \textit{not} ordered in rapidity. The only manoeuvre
 which will untangle the colour connections requires flipping the vertical arrangement of
particles $2$ and $3$ such that their vertical ordering is no longer equivalent to their ordering in rapidity. 
This configuration is therefore suppressed in the
MRK limit, because the two un-crossed ladders are not rapidity-ordered.

Furthermore, the study of references \cite{DelDuca:1993pp,DelDuca:1995zy} shows
that all the leading configurations each have the same limit in the
MRK limit, resulting in a colour
factor $\Ca$ for every final-state gluon. The limit agrees with that
predicted by the amplitudes of Fadin-Kuraev-Lipatov (FKL)~\cite{Kuraev:1976ge}.
When we pass an event from \HEJ to \py, we choose a
colour configuration at random from the set of colour connections which are
leading in the MRK limit, and pass the event using an interface conforming
to the \emph{Les Houches accord}~\cite{Boos:2001cv}. This method is identical
to that applied in ref.\@ \cite{Andersen:2011zd}.

% While the leading logarithmic accuracy of the cross sections is ensured by
% integrating over just the MRK phase space, all of phase space is explored by
% \HEJ, including that of emissions with vanishing momenta (see
% Eq.~(\ref{eq:resumdijetFKLmatched})). The generated state from \HEJ which
% should be parsed to \pyt can therefore contain many partons (technically we
% sum corrections with up to around 30 partons in the final state, where the
% upper limit is sufficiently large for the observables studied to be stable
% against variations). As part of the merging algorithm, and in line with the
% procedure of CKKW-L, \pyt will have to construct all possible shower
% histories leading to the specific state from \HEJ. The procedure for doing so
% is however too slow for the high multiplicities generated by \HEJ. Before
% parsing the event record to \pyt, we therefore remove all soft particles with
% a transverse momentum less than a \emph{merging scale} $k_{\perp M}$ from the
% \HEJ event record and redistribute the sum of their transverse momenta onto
% the remaining particles. The event weight etc.~is unchanged, and so is any
% observables dependent only on the momenta of hard jets (such as those studied
% in the current paper), as long as $k_{\perp M}$ is sufficiently small
% compared to the minimum jet transverse momentum. This clustering procedure is
% similar to that used for fixed-order matching in \HEJ and does not change the
% logarithms resummed by \HEJ.

%
% \todo[inline]{Comment here about MRK limit of colour connections - no
%   additional information on colour connection can be taken, and hence
%   later the factor of one half is the best we can do for now. If one
%   wants to weight colour connections differently, need to take
%   information from elsewhere, and it is not clear how to do this.}
% \todo[color=green,inline]{We should probably ask what \pyt has to say about it. In principle one could calculate the \pyt splitting function for the other colour flow and split the \HEJ ME ratio accordingly. Leif to say someithing in 4.3 instead.}

%%% Local Variables:
%%% mode: latex
%%% TeX-master: "main"
%%% End:
