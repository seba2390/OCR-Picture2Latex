\section{Introduction}
\label{sec:intro}
The analyses of collider data collected at both the Tevatron \cite{Abazov:2013gpa}
($\sqrt{s}=1.96$\TeV) and the LHC \cite{Aad:2011jz,Aad:2014pua,Aad:2014qxa,Chatrchyan:2012gwa,Aad:2015nda,Khachatryan:2016udy}
($\sqrt{s}=7,8$\TeV)
indicate that perturbative terms beyond fixed order are required for the
description of observables in processes involving at least two jets, in the
region of large partonic centre-of-mass energy $\sqrt{\hat s}$ compared to
the typical transverse momentum scale $p_\perp$,
$\sqrt{\hat s}/p_\perp>5$. This corresponds to the jets spanning
more than 3 units of rapidity. It is of course well-known
and indeed not surprising that the convergence of the perturbative series
requires input beyond fixed-order perturbation theory in certain regions of
phase space. The dominant and large corrections in this particular region of
phase space is the focus of one of the applications of the theory of
Balitsky-Fadin-Kuraev-Lipatov
(BFKL) \cite{Fadin:1975cb,Kuraev:1976ge,Kuraev:1977fs,Balitsky:1978ic}. The
study of such perturbative effects has received renewed interest, not only because the
increasing energy of colliders allows for detailed study of
observables in these regions, but also because some 
measurements specifically concentrate on experimental signatures with 
large rapidity spans \cite{Chatrchyan:2012pb,Chatrchyan:2012gwa,Chatrchyan:2013jya,Aad:2014pua,Aad:2014qxa,Aad:2015nda,Khachatryan:2016udy,Aaboud:2017fye}.

The BFKL formalism can provide a systematic description of large
perturbative corrections in two separate kinematic 
limits. Firstly, in the small-$x$ limit of $\hat{s}/s\ll 1$ (where $\sqrt{\hat{s}}$ is
the centre-of-mass energy for the partonic process and $\sqrt{s}$ is the
centre-of-mass energy of the collider), BFKL theory can be used to
describe the evolution of the PDFs in $x$. Secondly, in the limit $\hat s/|\hat t|\gg 1$
(where $|\hat t|^{1/2}$ is a typical jet transverse momentum scale and
$\hat s\le s$), BFKL theory captures the single-logarithmic
corrections in $\hat s/|\hat t| \sim e^{\Delta y}$ to the hard-scattering matrix element for
processes with a colour-octet exchange between two jets. These two applications
of BFKL are valid in two opposite kinematic regions. 
In the former case, a formalism to combine logarithms of BFKL and
Dokshitzer-Gribov-Lipatov-Altarelli-Parisi
(DGLAP) \cite{Dokshitzer:1977sg,Gribov:1972ri,Altarelli:1977zs} origin
was developed resulting in the Ciafaloni-Catani-Fiorani-Marchesini
(CCFM) equation \cite{Ciafaloni:1987ur,Catani:1989yc,Catani:1989sg,Marchesini:1994wr},
with an explicit partonic evolution implemented in
\textsc{Cascade} \cite{Jung:2000hk,Jung:2010si}. This has the non-perturbative stages of
the evolution handled by \pysix\cite{Sjostrand:2006za}.

The second \textit{high energy} limit of $\hat s/|\hat t|\gg 1$ is the focus of
Mueller-Navelet-style studies of QCD processes involving at
least two jets \cite{Mueller:1986ey}. Monte Carlo approaches to solving the BFKL equation were developed for the detailed study of
such processes \cite{Orr:1997im,Orr:1998ps}. A more accurate description of the scattering matrix
elements that still captures the BFKL logarithms was later obtained, which also 
included matching to fixed-order high multiplicity matrix
elements \cite{Andersen:2009nu,Andersen:2009he,Andersen:2011hs}. 
This approach
is implemented in the parton-level Monte Carlo of \HIGHEJ (\HEJ), 
which shall be described further in \cref{sec:hej}. 

An application of the \HEJ formalism that is of particular interest is the study
of the production of a Higgs boson in association with dijets that have a
large rapidity separation. 
The ability to model jets in the rapidity interval
permits an examination of the sensitivity of predictions
to the placement of vetoes upon additional radiation, 
which is particularly relevant to measurements
of the vector boson fusion (VBF) production channel \cite{DelDuca:2001fn,Klamke:2007cu,Andersen:2010zx}.
In order to fully understand the challenges in the theoretical modelling of QCD in the presence of vetoes,
and to expose deficiencies in different approaches, observables sensitive to additional radiation 
(such as gap fractions and average jet multiplicities) were measured by ATLAS in refs.~\cite{Aad:2011jz,Aad:2014pua}.
These analyses provided evidence that both high energy and DGLAP logarithms are necessary for 
an adequate description of data.

An algorithm to combine the high energy BFKL logarithms of \HEJ with the soft and collinear logarithms of DGLAP evolution 
was developed for the parton shower \ariadne \cite{Lonnblad:1992tz} in ref.~\cite{Andersen:2011zd}. 
The benefit of such a treatment is not only that the logarithmic accuracies of both descriptions are maintained (such
that emissions under both small and large invariant masses are described
correctly), but also that the partonic results of \HEJ are showered and
hadronised, thus obtaining a more realistic description of the various
stages of a hard scattering. This combined approach compared favourably to
data for several observables. 

Missing from this approach however were two important features that resulted in an inability to correctly describe jet profiles. 
Firstly, the method did not allow for the incorporation of the underlying event \cite{Sjostrand:2004ef}, which is required for a 
successful description of jet profiles. Furthermore, even when the effect of the underlying event was taken into account, 
there was a discrepancy in the profiles of high transverse momentum jets. It transpired that this could be 
understood in terms of certain soft gluons being produced in \HEJ that in \ariadne would only have been produced at a late stage in evolution. 
The presence of such soft emissions thereby inhibited the further evolution of the parton shower and 
such events would not contain the correct amount of collinear radiation.
Although the algorithm properly prevented the double-counting of such soft emissions, there was no mechanism in place
to account for the probability that the parton shower might preferentially have inserted collinear emissions at an earlier stage. 

In this paper we therefore present a new
method for combining the effects of soft and collinear logarithms with 
those of the all-order summation of \HEJ based on the advances made
in the merging of parton showers with fixed-order matrix elements. A crucial feature of our approach 
is that the exclusive $n$-parton events generated according to the \HEJ all-order matrix elements 
will be reweighted using properly subtracted collinear Sudakov factors, and moreover the parton shower will be able to 
insert collinear emissions where it is appropriate to do so. 
This has been implemented for the interleaved parton shower of \py \cite{Sjostrand:2014zea}, allowing for
the inclusion of multiple partonic interactions as well as the
subsequent hadronisation of the event.

% The benefit 
% \todo{Also mention how the lack of reweighting with collinear Sudakovs in this method interfered with the shower ordering, which was also visible as a discrepancy beyond the omission of MPI.}
% over the previous merging of \HEJ and \ariadne is that \py is a modern
% parton-shower with an inter-leaved description of the underlying event,
% components which were previously missing in the merged procedure. The
% underlying event is necessary for a successful description of jet profiles,
% i.e.~the energy-distribution within each jet-cone.

The outline of the paper is as follows. The all-order calculation of \HEJ is
described in \cref{sec:hej}. This is followed in \cref{sec:pythia} by a brief description of the
relevant parts of \py and the tree-level matrix element merging procedures from which our algorithm is inspired.
\Cref{sec:matching} will present the method for combining the calculations of \py with \HEJ.
In \cref{sec:results} we examine the performance our merged description, firstly by 
demonstrating the capacity of the new approach to describe jet profiles.
Secondly we compare to data for a set of observables that measure additional radiation in inclusive dijet events. 
We note that in this paper we restrict our focus to pure dijet studies, despite the relevance to Higgs phenomenology.
The reasons for this are two-fold. Firstly, the observables of interest have not yet been measured for Higgs plus dijet processes,
and secondly it is preferable to test newly developed tools in cleaner environments where there is no expectation of new physics.
Nevertheless the method we present should be easily applicable to other processes.
Finally we present the conclusions and outlook in \cref{sec:outlook}.

%%% Local Variables:
%%% mode: latex
%%% TeX-master: "main"
%%% End:
