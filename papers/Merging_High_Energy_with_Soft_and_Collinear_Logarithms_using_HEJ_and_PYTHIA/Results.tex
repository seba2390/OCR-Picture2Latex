\section{Results}
\label{sec:results}
\label{sec:descr-jet-struct}

\begin{figure}[t]
  \centering
%   \includegraphics[width=.75\linewidth]{figures/profile3D.pdf}
   \includegraphics[width=.75\linewidth]{profile3D.pdf}
  \caption{A Lego-plot of the momenta of partons arising from a single event
    from HEJ (blue) and average of the momenta arising from particles from
    that event showered 10,000 times with \py using the merging of
    \HEJpy. For this event configuration from \HEJ, which contains partons of
    similar transverse momenta, the effect of the showering is mostly to
    distribute physical particles around the partons of \HEJ.}
  \label{fig:lego}
\end{figure}

In this section, we will present the results of the formalism developed, and
contrast it with experimental data. We start however by qualitatively examining the
effect of the parton shower and hadronisation on a specific partonic event from \HEJ.
This is presented in \cref{fig:lego}, which shows a LEGO-plot of the average transverse momentum
deposit (greater than 100\MeV) in bins of $0.2\times 0.2$ units of rapidity and azimuthal angle. 
A single \HEJ event with 5 partons (and with fixed colour connections),
of which 4 are sufficiently hard to form individual jets of
$p_T>30$\GeV, is shown in blue. The average result of passing this event 10,000 times to \py
is shown in red. 

The effect of the shower on average is to spread out
the momentum of each \HEJ parton over an area with radius
$R\sim 1$ around that parton.
Indeed, for events similar to the chosen one, the effect of
the shower seems to be limited to filling the jet cones, and in \cref{sec:jetprofiles}
we study in more detail the accuracy with which the jet
cones are filled. In \cref{sec:impact-multi-jet} we will study
multi-jet observables, some of which probe large hierarchies in transverse momentum, and in such regions 
\py can additionally supplement the jets produced by \HEJ. 

We will mainly look at LHC analyses especially designed to probe
effects of high energy logarithms. It should be noted however, that
these have so far employed a relatively soft definition of
hadronic jets, typically requiring a transverse momentum of less than
40\GeV. This results in broad shower profiles, where the description
of the spill-over outside the jet cones is necessary for accurate
results. Furthermore, it was noted in ref.~\cite{Alioli:2012tp} that
these analyses often use cuts that also enhance soft and collinear
logarithms. In \cref{sec:impact-multi-jet} we therefore
propose to use a slightly harder threshold for jets to reduce the dependence
on shower and hadronisation effects, and crucially, investigate the full
rapidity range of the hard event rather than just the region in-between the
two hardest jets. This allows for a much
cleaner probe of the high energy logarithms.

% All the relevant LHC analyses concerned with probing the effects of
% BFKL logarithms that have been produced thus far employed a relatively
% soft definition of hadronic jets, typically requiring a transverse
% momentum of less than 40\GeV. This results in broad shower profiles,
% where the description of a spill-over outside the jet cones is
% necessary for accurate results. Furthermore, as discussed in
% ref. \cite{Alioli:2012tp}, the cuts and analyses are often influenced
% by regions of both soft, collinear and (in the best cases) \todo{This
%   sentence is not completely clear. Leif to rephrase the paragraph}
% high energy logarithms. A cleaner perturbative probe could be achieved
% by using a harder threshold for jets and a simple analysis, as
% discussed in \cref{sec:impact-multi-jet}.

We note that there exist many parameters in \py that 
control non-perturbative effects, and which are fixed by 
tuning to measurements of certain soft observables. We investigate an example of such an observable in the next section. 
As we will shortly see, the combination of \HEJ and \py obtains a very similar description
to \py alone. Therefore, in the results that follow 
we do not retune \py for use with \HEJ, 
even if this might further improve the agreement of
\HEJpy with data; instead we use the default Monash 2013 tune \cite{Skands:2014pea}
for both \py and \HEJpy.

\subsection{The Description of the Profile of Jets}
\label{sec:jetprofiles}

\begin{figure}[thp]
  \centering
%       \includegraphics[width=\linewidth]{figures/ATLAS_2011_S8924791/jet_profile_rho.pdf}
     \includegraphics[width=\linewidth]{jet_profile_rho.pdf}
  \caption{The data and predictions for the differential jet profile as
    defined in \cref{eq:rhoofr}. The parton shower of \py gives a very good
    description of data, which is inherited by \HEJpy.}
  \label{fig:jetprofile}
\end{figure}

The \emph{jet profiles} were measured at the LHC 
in early 7\TeV\ runs, for example by ATLAS in ref.\@ \cite{Aad:2011kq},  
accepting events with just one primary vertex (no pile-up) and at least one
jet with transverse momentum $p_\perp>30$\GeV\ and rapidity $|y|<2.8$. For such events, the differential
jet profile $\rho(r)$ as a function of the distance
$r=\sqrt{\Delta y^2+\Delta \phi^2}$ to the jet axis is defined as the average
fraction of the jet transverse momentum in an annulus between $r-\Delta r/2$
and $r+\Delta r/2$ around the jet axis in the $(y,\phi)$-plane. As such, 
\begin{equation}
  \label{eq:rhoofr}
  \rho(r)=\frac 1 {\Delta r} \frac 1
  {N_{\mathrm{jets}}}\sum_{\mathrm{jets}}\frac{p_\perp(r-\Delta r/2,r+\Delta r/2)}{p_\perp(0,R)},
\end{equation}
where $p_\perp(r_1,r_2)$ is the summed $p_\perp$ in the annulus of the two
circles of radii $r_1$ and $r_2$, and $N_{\mathrm{jets}}$ is the number of
jets. The measurement of ref.\@ \cite{Aad:2011kq} used $\Delta r=0.1$ and
the anti-$k_T$ jet algorithm \cite{Cacciari:2008gp} with $R=0.6$. This
analysis is implemented in recent versions of Rivet \cite{Buckley:2010ar},
which we use to analyse generated events (both here and in \cref{sec:impact-multi-jet}).

The \HEJ formalism captures the logarithms associated with wide-angle
emissions, but not those associated with the collinear emissions. \HEJ is thus
not expected to fill the jet cones with radiation, and it is expected that
the results of \HEJ\!+\py for the jet shapes is similar to those of pure
\py (since the merging procedure should produce results similar to \py in
regions where \HEJ does not radiate). In \cref{fig:jetprofile} we compare 
the predictions of \py and \HEJpy for $\rho(r)$ in slices of jet transverse momentum
to data \cite{Aad:2011kq}. While
\HEJ alone would primarily have filled just the first bin in each distribution,
\HEJpy gives the same very good description of the jet shapes
as the parton shower of \py. The merging procedure has therefore performed a
perfect job of populating the jet areas (through collinear emissions), which
are mostly empty in the pure partonic description of \HEJ --- and has of
course furthermore fully hadronised the partonic states.
The ability to describe this observable represents an improvement relative to
the matching of \HEJ + \ariadne.

\subsection{Impact on Multi-jet Observables}
\label{sec:impact-multi-jet}


In this section we will investigate the impact of the merging on
observables which depend only on the identified hard jets of the event. 
We shall make comparisons between pure \HEJ, \py and \HEJpy. Events in \HEJ 
(both with and without showering) were
generated with the PDF set CT14nlo \cite{Dulat:2015mca,Buckley:2014ana}, and 
the renormalisation and factorisation scales were taken to 
be the maximum jet transverse momentum $\mu_R=\mu_F = {p_T}_\mathrm{max}$.
In the case of pure \HEJ the scale uncertainties were estimated by varying 
$\mu_R$ and $\mu_F$ independently between twice and half the central scale choice,
and these uncertainties will be denoted as a band around the central predictions.
The vertical lines indicate the statistical uncertainty on the results. 
As before, \py predictions were generated using the Monash 2013 tune. The
expectation is that there should be little impact on the results of \HEJ in
phase space regions where the jets have similar transverse momenta but are widely
separated in rapidity. This is the region where \HEJ should already control
the dominant logarithms to all orders. The parton shower should therefore not
introduce sizeable corrections. On the other hand, as mentioned in \cref{sec:hejformalism}
regions with large disparate transverse scales are not encompassed by the kinematic assumptions
of the \HEJ formalism, and should therefore receive additional hard emissions from the parton
shower. 

\begin{figure}[t]
  \centering
     \includegraphics[width=\linewidth]{03_gapfrac_dy_12_q20.pdf}
%     \includegraphics[width=\linewidth]{figures/ATLAS_2011_S9126244/03_gapfrac_dy_12_q20.pdf}
  \caption{Plot showing a comparison between \HEJ, \py, \HEJpy  and ATLAS data \cite{Aad:2011jz} for
  the gap fraction as a function of the rapidity separation of the tagging jets, in slices of the average transverse momentum of the tagging dijets.
  The impact of the parton shower in \HEJpy is modest.}
  \label{fig:gapfrac}
\end{figure}

We will first consider two ATLAS analyses \cite{Aad:2011jz,Aad:2014pua}
that measure the amount of additional radiation in inclusive dijet events. 
Dijet systems are of course simple
at the Born level, characterised by two jets of equal transverse momenta that are
back-to-back in the azimuthal plane; however, this simple topology is in
general spoiled by radiative corrections. 
The analyses in question both require the existence of a dijet pair above 
some transverse momentum cut, defining the tagging jets; in what follows 
the tagging jets are identified as the two hardest (leading) jets in the event.
The number of jets in the rapidity interval between the tagging 
jets, each having a transverse momentum above a given \textit{veto scale} $Q_0$,
is then measured. 
This allows the definition of two observables: first, the \textit{gap fraction}, 
and secondly the average number of jets in the rapidity interval $\overline{N_\mathrm{jet}}$.
Events having no jets above the veto scale in the rapidity interval between the tagging jets
are classified as gap events. The gap fraction as defined by ATLAS \cite{Aad:2011jz,Aad:2014pua}
is then simply the ratio of the contribution to the cross section from these gap events to 
the inclusive dijet cross section.


% We will start by studying the effect on the so-called \emph{gap fraction} in dijet events as studied by ATLAS in ref.\@  \cite{Aad:2011jz}
% before studying the average number of jets within an event selection chosen
% by ATLAS in ref.\@ \cite{Aad:2014pua} and finally one promoted in
% ref.\@  \cite{Alioli:2012tp} to better disentangle the effects of the parton
% shower and that of high-energy logarithms.  ,
% and we will see that superficially similar definitions of observables and
% event samples can lead to very different results. This is discussed further
% in ref.\@  \cite{Alioli:2012tp}.

\begin{figure}[t]
  \centering
   \includegraphics[width=\linewidth]{05_gapfrac_q0_12.pdf}
%     \includegraphics[width=\linewidth]{figures/ATLAS_2011_S9126244/05_gapfrac_q0_12.pdf}
  \caption{Plot showing a comparison between \HEJ, \py, \HEJpy  and ATLAS data \cite{Aad:2011jz} for
  the gap fraction as a function of the veto scale $Q_0$, in slices of both the average transverse momentum and rapidity separation 
  of the tagging dijets.  For sufficiently large $Q_0\sim 50$\GeV\ HEJ alone
    achieving a good description; \py and \HEJpy are consistent, with a good description across all bins. 
%     For the largest rapidity spans and transverse momenta, there are hints
%     that \HEJpy provides and improvement relative to \py. 
    }
    \label{fig:gapfracb}
  \label{fig:gapfrac2}
\end{figure}

We start with the ATLAS analysis presented in ref.\@  \cite{Aad:2011jz}, 
in which jets were defined using the anti-$k_T$ jet
algorithm with $R=0.6$ and having rapidity $|y_j|<4.4$.
In \cref{fig:gapfrac} we show a plot of gap fraction as a function of the rapidity interval between the tagging jets
$|\Delta y|$,
where the veto scale was taken to be $Q_0 = 20$\GeV.
This is shown in bins of the average transverse momentum of the tagging dijets $\overline{p_T}$,
from  70\GeV\ -- 90\GeV\ to 240\GeV\ -- 270\GeV.
By construction,
the gap fraction will be 1 at $|\Delta y|=0$, since the phase space where a
third jet would be counted is vanishing (since only jets in-between the two
hardest jets are counted, and the rapidity difference between the two
hardest jets is zero). 
The variation between the predictions is small, and
discernible only for the bins with the largest $\overline{p_T}$.
Here there is a large hierarchy between the scale of the tagging jets
and the scale of any additional jets (which are characterised by the veto scale).
It is therefore not surprising that \HEJ predicts too few additional jets
in this region instead requiring the DGLAP resummation of 
the parton shower. Moreover the combination of \HEJpy results in a description 
that at least as good as, or better than, \py or \HEJ
individually. 


% ATLAS investigates in ref.\@  \cite{Aad:2011jz} the radiation from
% dijet systems, and we will here discuss the impact on results on the
% gap fraction and the average number of jets as defined in that paper. The ATLAS analysis in
% question starts from the system of the two hardest jets of the events with a
% transverse momentum above 20\GeV. The gap fraction is defined then as the
% fraction of events with no further jets above a scale $Q_0$ detected in the
% rapidity region in-between the two hardest jets. This is investigated in 7
% bins of the average transverse momentum of the two hardest jets, from
% 70\GeV\ -- 90\GeV\ to 240\GeV\ -- 270\GeV. The predictions obtained with pure \HEJ, \py
% and \HEJpy are compared to data in \cref{fig:gapfrac}. By construction,
% the gap fraction will be 1 at $\Delta y=0$, since the phase space where a
% third jet would be counted is vanishing (since only jets in-between the two
% hardest jets are counted, and the rapidity difference between the two
% hardest jets is zero). The variation between the predictions is small, and
% discernible only for the bin with the largest average $p_t$ of the two
% hardest jets. As the transverse momentum of the hardest jets is increased
% while $Q_0$ (the transverse scale at which additional jets are counted) is
% kept fixed at 20\GeV, \HEJ predicts too few additional jets. This is not
% surprising, since a successful prediction in regions with a large hierarchy
% between the hardest jets and a very low jet scale necessitates a shower
% resummation. Indeed, for each bin the description obtained by \HEJpy is as
% good or better than that of \py or \HEJ individually.




In \cref{fig:gapfrac2} the gap fraction is instead shown as a function of the 
veto scale $Q_0$, now binned in both $\overline{p_T}$ and $|\Delta y|$.
It is evident that even a modest increase in $Q_0$ to 50\GeV\ in \cref{fig:gapfrac}
would have brought the predictions
from pure \HEJ into perfect agreement with data across all regions in
$\overline{p_T}	$ and $\Delta y$. Furthermore, there are indications (e.g.~from the
region of
$210\GeV\le\overline{p_T}\le240\GeV, 2\le\Delta y\le3$) that
the high energy logarithms of \HEJ in \HEJpy improve the predictions of \py
alone. 



\begin{figure}[t]
  \centering
  \begin{subfigure}[t]{0.495\linewidth}
   \includegraphics[width=\linewidth]{04a_avenj_dy12.pdf}
%     \includegraphics[width=\linewidth]{figures/ATLAS_2014_I1307243/04a_avenj_dy12.pdf}
    \caption{}
    \label{fig:avgjets20GeVHardestDY}
  \end{subfigure}
  \begin{subfigure}[t]{0.495\linewidth}
%     \includegraphics[width=\linewidth]{figures/ATLAS_2014_I1307243/04b_avenj_pt.pdf}
\includegraphics[width=\linewidth]{04b_avenj_pt.pdf}
    \caption{}
    \label{fig:avgjets20GeVHardestPT}
  \end{subfigure}
  \caption{Plot showing a comparison between \HEJ, \py, \HEJpy  and ATLAS data \cite{Aad:2014pua} for
  the average number of jets in the rapidity interval between the two tagging jets, as a
    function of  (\subref{fig:avgjets20GeVHardestDY}) the rapidity interval
    between the two tagging jets, and (\subref{fig:avgjets20GeVHardestPT}) the
    average transverse momentum of the two tagging jets.}
    \label{fig:avgjets20GeVHardest}
\end{figure}

The average number of hard jets is a potentially better discriminant between
the predictions than the gap fraction, simply because the average number of
jets has a larger range of variation. We now consider this observable as measured
by ATLAS in ref.\@  \cite{Aad:2014pua}, where the hardest and second hardest jets 
(also defining the tagging jets)
were required to have transverse momenta above 60\GeV\ and 50\GeV\ respectively\footnote{
Asymmetric cuts are required in order for a meaningful
  comparison to NLO calculations, which suffers a spurious logarithmic
  dependence on the soft
  emissions \cite{Frixione:1997ks}.}.
Jets were again defined using the anti-$k_T$ algorithm with $R=0.6$.
In \cref{fig:avgjets20GeVHardestDY} the average number of jets in the interval between
the tagging jets is shown as a function of 
the rapidity interval between the tagging jets (with $Q_0 = 20$\GeV).
While the differences in the predictions are again small, we observe
that although the data from ATLAS lie within the scale uncertainty band for 
pure \HEJ, the central line for \HEJ nevertheless underestimates the 
number of additional jets. Meanwhile, the predictions of \HEJpy are better in line with data, and
are of a similar quality to that of \py. 

In \cref{fig:avgjets20GeVHardestPT} the average number of jets in the interval between
the tagging jets is instead shown as a function of the average transverse
momentum of the tagging jets (with $Q_0 = 30$\GeV), and where the dijets were
required to be separated by at least one unit of rapidity.
 As the average transverse momentum of the two
hardest jets increases to 1\TeV, the number of 30\GeV\ jets is unsurprisingly no
longer well-described without a shower resummation. Indeed, for increasing
$\overline{p_T}$, the predictions of pure \HEJ rises from 0.12 additional jets
to 0.3, whereas data rises from 0.15 to 0.5. Both \py and \HEJpy give a good
description of this distribution. For such large ratios of transverse scales,
the effect of the shower resummation is large, and therefore the results for 
\HEJpy are outside the scale uncertainty band for pure \HEJ.
% 
% The average number of hard jets is a potentially better discriminant between
% the predictions than the gap fraction, simply because the average number of
% jets has a larger range of variation. This was investigated further by ATLAS
% in ref.\@  \cite{Aad:2014pua} for various selection cuts and analyses, for which
% we will make predictions and comparisons. Two selection cuts were considered:
% A set of jets of transverse momenta above 60\GeV\ and 50\GeV\ were
% required\footnote{Asymmetric cuts are required in order for a meaningful
%   comparison to NLO calculations, which suffers a spurious logarithmic
%   dependence on the soft
%   emissions\cite{Frixione:1997ks}.}. \Cref{fig:avgjets20GeVHardestDY}
% investigates then average number of jets (anti-$k_T$, $R=0.6$) above 20\GeV\
% between the two hardest jets above 50\GeV\ (with at least one of these above
% 60\GeV), as a function of the rapidity span between the two hardest
% jets. While the difference in the predictions are again small, we observe
% that although the data from ATLAS is within the scale variation of the
% predictions from pure \HEJ, the number of jets above 20\GeV\ is
% underestimated. The predictions of \HEJpy are better in line with data, and
% of similar quality as that of \py. 
% %% 20GeV is too soft for clear discrimination
% 
% \Cref{fig:avgjets20GeVHardestPT} investigates the average number of jets
% above 30\GeV\ between the two hardest jets as a function of the average
% transverse momentum of these, and requiring a rapidity separation of at least
% 1 between the hardest jets. As the average transverse momentum of the two
% hardest jets increases to 1\TeV, the number of 30\GeV\ jets is unsurprisingly no
% longer well-described without a shower-resummation. Indeed, for increasing
% $\overline p_T$, the predictions of pure \HEJ rises from 0.12 additional jets
% to 0.3, whereas data rises from 0.15 to 0.5. Both \py and \HEJpy give a good
% description of this distribution. For such large ratios of transverse scales,
% the effect of the shower resummation is large, meaning that the results for
% this distributions of \HEJpy are outside the region of scale variations of of
% the results of pure \HEJ.



It should be apparent at this stage that in distributions that probe large differences in transverse momentum such as 
\cref{fig:avgjets20GeVHardestPT}  a parton shower is necessary for an accurate description, and therefore the
addition of \py to \HEJ gives rise to a notable improvement relative to \HEJ. 
Likewise, in distributions that probe large rapidity spans, one might have expected that \HEJ (and hence \HEJpy)
would provide a superior description relative to \py. Indeed, it is perhaps surprising that the predictions of \HEJ and \py are so
similar for the rapidity distributions studied so far. 
(In fact, we note that in some cases the description of \py is closer to data
than the predictions of \pyt+\texttt{POWHEG}
\cite{Nason:2004rx,Frixione:2007vw,Alioli:2010xa} which were used in the
original analyses \cite{Aad:2011jz,Aad:2014pua}. This could be an effect of
the later tunings of the non-perturbative parameters of \pyt, and reiterates
the possible benefits of performing similar analysis with much harder jet
scales, such that the sensitivity to the tunings of the MPI and
non-perturbative effects are reduced).
Firstly, the restrictive definition of the chosen observables prevents much variation in 
their values. Also, as discussed in ref.\@  \cite{Alioli:2012tp}, the softness of the
veto scale relative to the tagging dijets' transverse momentum results in 
event samples that are influenced by \textit{both} high energy and soft-collinear logarithms,
spoiling the applicability of the \HEJ formalism. 
%(i.e.~hierarchies in light-cone momenta)

\begin{figure}[t]
  \centering
  \begin{subfigure}[t]{0.495\linewidth}
  \centering
   \includegraphics[width=\linewidth]{avenj_dyfb.pdf}
%   \includegraphics[width=\linewidth]{figures/DIJETS_1202_1475/MSV/avenj_dyfb.pdf}
  \caption{}
  \label{fig:avejetspure_dyfb}
\end{subfigure}
  \begin{subfigure}[t]{0.495\linewidth}
    \centering
     \includegraphics[width=\linewidth]{avenj_HT.pdf}
%   \includegraphics[width=\linewidth]{figures/DIJETS_1202_1475/MSV/avenj_HT.pdf}
    \caption{}
    \label{fig:avejetspure_HT}
\end{subfigure}  
\caption{Plot showing a comparison between \HEJ, \py and \HEJpy
for the average number of jets as function of (\subref{fig:avejetspure_dyfb}) 
the rapidity interval between the most forward and backward jets, and (\subref{fig:avejetspure_HT})
the scalar sum of transverse momenta. The event selection and definition of observables was taken 
from ref.\@  \cite{Alioli:2012tp}, chosen to better disentangle effects originating from high energy or soft and collinear logarithms.}
  \label{fig:avejetspure}
\end{figure}

Simpler event samples were
suggested in ref.\@ \cite{Alioli:2012tp} to disentangle the two sources of
logarithmic corrections, together with more inclusive observables that better expose the
differences in the description of a fixed-order calculation, a parton shower
and \HEJ. The analysis considered inclusive dijet events, requiring at least one jet with 
transverse momentum above 45\GeV, and with all other jets required to have transverse momentum above 35\GeV. 
Jets are defined using the anti-$k_T$ algorithm with $R=0.5$ and with rapidities $|y_j|<4.7$.
Comparisons between \py, \HEJ, and \HEJpy were made for this analysis
and the results for the average total number of jets are shown in \cref{fig:avejetspure}.
We emphasise that the additional jets are no longer required to be in the rapidity interval
between the two hardest jets, and there is no longer a significant disparity between their transverse
momenta and that of the two hardest jets. This results in a greater number of jets 
passing the selection cuts, and consequently the potential for variation between 
different predictions is slightly higher.

Also shown in \cref{fig:avejetspure} as a shaded red band around the central
predictions for \HEJpy are variations of the merging scale ${k_\perp}_{M}$
(with a central scale of 15\GeV) between 7.5\GeV and 30\GeV. ${k_\perp}_{M}$
should be set to a value below the minimum jet transverse momentum used in
the analysis, which in this case is 35\GeV.  We see that even for these very
exclusive multiplicity-dependent observables, allowing the merging scale to
get very close to the analysis scale leads to only modest variations, and do
not exceed the size of the \HEJ renormalisation and factorisation scale
uncertainties.  As this plot is most sensitive to differences between \HEJ,
\py and \HEJpy, we expect the merging scale dependence in other plots to be
comparable or smaller than that observed here.

% \Cref{fig:avejetspure} compares the predictions for \py, \HEJ
% and \HEJpy for the number of jets above 35\GeV, in an event sample requiring
% just one jet above 45\GeV\ and one above 35\GeV. All jets above 35\GeV\ and
% with $|y_j|<4.7$ are considered, and $\Delta y_{fb}$ is the rapidity
% difference between the most forward and most backward of these hard
% jets. For every event, the jet count in the rapidity region between the most forward and
% backward hard jets is obviously larger than the jet-count in the
% rapidity region between 
% the two jets of largest transverse momenta. Therefore, the differences
% between the predictions is also slightly larger, when analysed versus the
% rapidity difference between the forward/backward jets, rather than between the two
% hardest jets.


In \cref{fig:avejetspure_dyfb} the average number of jets is shown as a function 
of the rapidity interval between the most forward and backward jets $\Delta y_{fb}$;
we expect this to be particularly sensitive to the logarithms in $\hat s/|\hat t| \sim e^{\Delta y}$
summed by \HEJ.
The predictions of \py are significantly lower than those of \HEJ and
\HEJpy, and moreover are outside the scale variation band for pure
\HEJ beyond $\Delta y_{fb}>4.5$. This implies that in this regime, the 
merging of \HEJ with \py increases the number of wide-angle jets relative to \py alone,
as we should expect.  Such differences should be even more pronounced
with a larger centre-of-mass energy than the choice of $\sqrt{s} =7$\TeV which was used for these
comparisons.


It is interesting to note that not only is the spread of predictions 
significantly larger in \cref{fig:avejetspure_dyfb} than in \cref{fig:avgjets20GeVHardestDY},
but also that the prediction of \HEJpy is
\emph{lower} than that of \HEJ alone.
There are several possible explanations for this. 
Firstly the addition of a parton shower extends the shower profile beyond the
jet radius, such that potentially fewer of the jets from the partonic calculation
pass the relevant criteria.
Secondly, at $\Delta y=10$ two partonic
jets of 45\GeV\ transverse momentum would take up all the energy available at
a 7\TeV\ collider, and all predictions for the average number of jets would
therefore have to return to 2 at this point. Since the parton shower uses some of
the available collider energy in (for example) the description of the underlying event,
the turnover of the average number of jets will have to happen
earlier than in the pure partonic prediction.
Alternatively it could be that too many non-FKL configurations of lower multiplicity are being inserted by
the merging algorithm, an issue that could be resolved by the extension of this method
to merge non-FKL events.
% Refinements of the method may indeed be necessary, although without a comparison to data however,
% it is difficult to determine what is the correct prescription. 
% We hope that an experimental study with similarly inclusive cuts will be conducted in future.

% We see in \cref{fig:avejetspure} (left) that the predictions for the
% average number of jets of \py are significantly lower than those of \HEJ and
% \HEJpy, and outside that of the scale variation on the predictions of pure
% \HEJ from $\Delta y_{fb}>4.5$. The differences should be even more pronounced
% with a larger centre-of-mass energy than the 7\TeV\ considered for these
% results. This observable and simpler selection cuts leads to a clearer
% separation of the effects of the logarithms included in the parton shower and
% those of the high-energy resummation of BFKL or \HEJ. It is interesting not
% only that the predictions from the high-energy logarithms included in \HEJ
% and the soft and collinear logarithms of \py are so diverse for the analysis
% of large rapidity spans in \cref{fig:avejetspure} (left) compared to that
% of \cref{fig:avgjets20GeVHardestDY}, but that the result of \HEJpy is
% \emph{lower} than that of \HEJ alone. There are several possible explanations
% for this. Firstly, since no hierarchies of transverse scales is introduced,
% compared to \cref{fig:avgjets20GeVHardestDY} the parton shower itself
% radiates fewer hard jets. Secondly, the addition of the parton shower
% introduces a mechanism to reduce the transverse momentum of the jets arising
% from \HEJ by extending the shower profile beyond the jet radius. The jet
% count is reduced, when the jet transverse momentum is pushed below the
% threshold set by the jet algorithm. And lastly, at $\Delta y=10$, two partonic
% jets of 45\GeV\ transverse momentum would take up all the energy available at
% a 7\TeV\ collider, and all predictions for the average number of jets would
% therefore have to return to 2 at this point. The parton shower uses some of
% the available collider energy in the description of the underlying event
% etc., and so the turn-over of the average number of jets will have to happen
% earlier than in the pure partonic prediction.

Finally, in \cref{fig:avejetspure_HT} the average number of jets is shown
as a function of the scalar sum of transverse momentum $H_T$; we expect this observable to 
be sensitive to the double logarithms in transverse momentum summed by the parton shower.
\py now adds further hard radiation to that of \HEJ, which is both as expected 
and is consistent with the previous results. 

The choice of more inclusive observables and simpler selection cuts leads to a clearer
separation of the effects of the logarithms included in the parton shower and
those of the all-order summation of high energy logarithms in BFKL or \HEJ.
A simple experimental investigation with a similar set of cuts and distributions 
would be extremely interesting in exposing
the shortcomings of either predictions, and the benefits of the combined
formalism presented in this paper. Such an experimental analysis would
further aid the development of predictions valid for the separation of the
VBF and GF contribution to Higgs-boson production in association with dijets.




% Contrary to the situation in the analysis of events versus the rapidity
% span, \py adds further hard radiation to that of \HEJ for
% the distribution in $H_T$. This is exactly as na\"ively expected, and a
% simple experimental investigation would be extremely interesting in exposing
% the shortcomings of either predictions, and the benefits of the combined
% formalism presented in this paper. Such an experimental analysis would
% further aid the development of predictions valid for the separation of the
% VBF and GF contribution to Higgs-boson production in association with dijets.



%%% Local Variables:
%%% mode: latex
%%% TeX-master: "main"
%%% End:
