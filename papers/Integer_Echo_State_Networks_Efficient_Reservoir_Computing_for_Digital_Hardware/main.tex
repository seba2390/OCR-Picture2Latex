% -*- TeX:SE -*-
%
% main.tex

 
  
%\documentclass[conference]{IEEEtran}
\documentclass[journal]{IEEEtran}
 
%\usepackage[latin1]{inputenc}
\usepackage[T1]{fontenc}
\usepackage{times}
%\usepackage{natbib} %IEEEtrans bibliography doesn't like natbib apparently
\usepackage{dcolumn}
\usepackage{endnotes}
\usepackage{graphics}
%\usepackage{subfigure}
%\usepackage{caption}
%\usepackage{subcaption}
\usepackage{times}
\usepackage{epsfig}
\usepackage{graphicx}
\usepackage{textcomp}
\usepackage{amssymb,amsmath}
\usepackage{balance}

\usepackage{listings}
\usepackage[]{algorithm2e}
\usepackage{flushend}
\usepackage{color,soul}
%\usepackage[hyphenbreaks]{breakurl}
%\usepackage[hyphens]{url}
%\usepackage{hyperref}
%\def\UrlBreaks{\do\/\do-}
%\usepackage{url}
%\usepackage[hyphens]{url}
\usepackage{hyperref}
%For making to do notes
\usepackage[dvipsnames]{xcolor}
\usepackage{todonotes}

\usepackage{siunitx}
\usepackage{pifont}% http://ctan.org/pkg/pifont
\newcommand{\cmark}{\ding{51}}%
\newcommand{\xmark}{\ding{55}}%

%For hl command
\soulregister\cite7
\soulregister\ref7
\soulregister\pageref7

%\usepackage[hyphenbreaks]{breakurl}
\def\UrlBreaks{\do\/\do-}

%\usepackage[colorlinks,urlcolor=blue]{hyperref}

%\usepackage[color=yellow,textsize=footnotesize]{todonotes}
%{\color{red} highlighted}


%\bibliographystyle{unsrt}
\bibliographystyle{IEEEtran} % IEEEtrans for @Patent
\urlstyle{sf}

\begin{document}

%\title{Integer Echo State Networks: Hyperdimensional Reservoir Computing}
\title{Integer Echo State Networks: Efficient Reservoir Computing for Digital Hardware}

% Approach for distributed representation and processing of a sensory data based on principles of hyper-dimensional computing


\author{Denis~Kleyko,
        E.~Paxon~Frady,
        Mansour~Kheffache,
        and~Evgeny~Osipov % <-this % stops a space
\thanks{%Manuscript received on May 1, 2019; 
This work  was  supported in part  by the Swedish Research Council (grant No. 2015-04677). 
The work of DK was supported by the European Union’s Horizon 2020 Research and Innovation Programme under the Marie Skłodowska-Curie Individual Fellowship Grant Agreement 839179 and in part by the DARPA’s VIP (Super-HD Project) and AIE (HyDDENN Project) programs.
%revised May 3, 2018; accepted May 3, 2018. Date of publication May 3, 2018; date of current version May 3, 2018.}
%\thanks{This work was supported by the Swedish Research Council (grant no. 2015-04677) and Systems on Nanoscale Information fabriCs (SONIC), one of the six SRC STARnet Centers, sponsored by MARCO and DARPA. DK also acknowledges Stiftelsen Seth M Kempes Stipendiefond for partially funding his research visit to UC Berkeley.
}
\thanks{\mbox{*}D. Kleyko is with the Redwood Center for Theoretical Neuroscience at the University of California, Berkeley, CA 94720, USA and also with Intelligent Systems Lab at Research Institutes of Sweden, 164 40 Kista, Sweden. \mbox{E-mail}: \mbox{denis.kleyko@ri.se}}% <-this % stops a space 
%\thanks{E. P. Frady is with Neuromorphic Computing Lab, Intel Labs and also with the Redwood Center for Theoretical Neuroscience at the University of California, Berkeley, CA 94720, USA. \mbox{E-mail}: \mbox{epaxon@berkeley.edu}
\thanks{E. P. Frady is with the Redwood Center for Theoretical Neuroscience at the University of California, Berkeley, CA 94720, USA. \mbox{E-mail}: \mbox{epaxon@berkeley.edu}
}% <-this % stops a space
\thanks{M. Kheffache is with Netlight Consulting AB, 111 53 Stockholm, Sweden. \mbox{E-mail}: \mbox{mansour.kheffache@netlight.com}}% <-this % stops a space 
\thanks{E. Osipov is with the Department of Computer  Science Electrical and Space Engineering, Lule\aa{} University of Technology, 971 87 Lule\aa{}, Sweden. \mbox{E-mail}: \mbox{evgeny.osipov@ltu.se} }% <-this % stops a space
 }% <-this % stops a space







\markboth{}%
{Kleyko \MakeLowercase{\textit{et al.}}: Integer Echo State Networks}


\maketitle

% I would change the first sentence (the rest is fine) to:
%
\begin{abstract}

%We propose an integer approximation of Echo State Networks (ESN) based on the
%mathematics of hyperdimensional computing.
%The reservoir of the proposed Integer Echo State Network (intESN) contains only \textit{n}-bits integers and replaces the recurrent matrix multiply %with an efficient cyclic shift operation. Such an architecture results in dramatic improvements in memory footprint and computational efficiency, with minimal performance loss.
%Our architecture naturally supports the usage of the trained reservoir in  symbolic processing tasks of analogy making and logical inference. 

We propose an approximation of Echo State Networks (ESN) that can be efficiently implemented on digital hardware based on the mathematics of hyperdimensional computing. The reservoir of the proposed integer Echo State Network (intESN) is a vector containing only n-bits integers (where n<8 is normally sufficient for a satisfactory performance). The recurrent matrix multiplication is replaced with an efficient cyclic shift operation. The proposed intESN approach is verified with typical tasks in reservoir computing: memorizing of a sequence of inputs; classifying time-series; learning dynamic processes. Such architecture results in dramatic improvements in memory footprint and computational efficiency, with minimal performance loss. 
%{\color{red}
The experiments on a field-programmable gate array confirm that the proposed intESN approach is much more energy efficient than the conventional ESN.
%}
\end{abstract}

\begin{IEEEkeywords}
reservoir computing, echo state networks, vector symbolic architectures, hyperdimensional computing, memory capacity, time-series classification, dynamic systems modelling
 \end{IEEEkeywords}


\section{Introduction}  \label{sec:introduction}

\newcommand\inexpIntro[3]{#1?(#2,#3).}
\newcommand\rinexpIntro[3]{*#1?(#2,#3).}
\newcommand\outexpIntro[3]{#1!(#2,#3).}
\newcommand\outatomIntro[3]{#1!(#2,#3)}

We propose a fully automated method for proving termination of \(\pi\)-calculus processes.
Although there have been a lot of studies on termination analysis for the \(\pi\)-calculus
and related calculi~\cite{Deng06IC,Demangeon07,SangiorgiTermination,KobayashiHybrid,Yoshida04IC,DBLP:journals/jlp/DemangeonHS10,Venet98SAS}, most of them have been rather theoretical,
and there have been surprisingly little efforts in developing  fully automated termination
verification methods and tools based on them. To our knowledge,
Kobayashi's \typical{}~\cite{TyPiCal,KobayashiHybrid} is the only exception that
can prove termination of \(\pi\)-calculus processes (extended with natural numbers)
fully automatically, but its termination analysis is quite limited (see Section~\ref{sec:relatedwork}).

Our method is based on a reduction to termination analysis for sequential programs:
we translate a \(\pi\)-calculus process \(P\) to a sequential program \(S_P\), so that
if \(S_P\) is terminating, so is \(P\). The reduction allows us to use
powerful, mature methods and tools
for termination analysis of sequential programs~\cite{heizmann2016ultimate,freqterm,DBLP:conf/lics/PodelskiR04,Kuwahara2014Termination,DBLP:journals/cacm/CookPR11}.

The idea of the translation is to convert a chain of communications on replicated input
channels to a chain of recursive function calls of the target sequential program.
Let us consider the following Fibonacci process:
\begin{align*}
    & \rinexpIntro{\fib}{n}{r}
        \ifexp{n<2}{ \soutatom{r}{1} \\ &\quad}
                   { \nuexp{s_1} \nuexp{s_2} (\outatomIntro{\fib}{n-1}{s_1} \PAR \outatomIntro{\fib}{n-2}{s_2} \PAR \sinexp{s_1}{x}\sinexp{s_2}{y}\soutatom{r}{x+y}) \\}
    & \PAR \outatomIntro{\fib}{m}{r}
\end{align*}
Here, the process
$\rinexpIntro{\fib}{n}{r} \ldots$ is a function server that computes the \(n\)-th Fibonacci number
in parallel and returns the result to \(r\),
and $\outatom{\fib}{m}{r}$ sends a request for computing the \(m\)-th Fibonacci number;
those who are not familiar with the syntax of the \(\pi\)-calculus may wish to consult
Section~\ref{sec:targetlanguage} first.
To prove that the process above is terminating for any integer \(m\),
it suffices to show that there is no infinite chain of communications on $\fib$:
\[
    \fib(m,r) \to \fib(m_1,r_1) \to \fib(m_2,r_2) \to \cdots.
\]
We convert the process above to the following program:\footnote{The actual translation
  given later is a little more complex.}
\begin{verbatim}
 let rec fib(n) = if n<2 then () else (fib(n-1) [] fib(n-2)) in
 fib(m)
\end{verbatim}
Here, \texttt{[]} represents the non-deterministic choice.
Note that, although the calculation of Fibonacci numbers is not preserved,
for each chain of communications on \texttt{fib}, there is a corresponding
sequence of recursive calls:
\[
\mathtt{fib}(m) \to \mathtt{fib}(m_1) \to \mathtt{fib}(m_2) \to \cdots.
\]
Thus, the termination of the sequential program above implies the termination of
the original process.
As shown in the example above, (i) each communication on a replicated input channel
is converted to a function call, (ii) each communication on a non-replicated input
channel is just removed (or, in the actual translation, replaced by a call of
a trivial function defined by \(f(\seq{x})=(\,)\)), and (iii) parallel composition
is replaced by a non-deterministic choice.
We formalize the translation outlined above and prove its correctness.

The basic translation sketched above sometimes loses too much information.
For example, consider the following process:
\begin{align*}
    & \rinexpIntro{\pre}{n}{r} \soutatom{r}{n-1} \\
    & \PAR \rinexpIntro{f}{n}{r} \ifexp{n<0}{ \soutatom{r}{1} }
                                       { \nuexp{s} (\outatomIntro{\pre}{n}{s} \PAR \sinexp{s}{x}\outatomIntro{f}{x}{r}) } \\
    & \PAR \outatomIntro{f}{m}{r}
\end{align*}
The translation sketched above would yield:
\begin{verbatim}
  let pred(n) = n-1 in
  let rec f(n) = if n<0 then () else (pred(n) [] f(*)) in
  f(m)
\end{verbatim}
Here, \texttt{*} represents a non-deterministic integer: since we have removed
the input $\sinatom{s}{x}$, we do not have information about the value of \( x \).
As a result, the sequential program above is non-terminating, although the original
process is terminating.
To remedy this problem, we also refine the basic translation above by using a refinement
type system for the \(\pi\)-calculus. Using the refinement type system,
we can infer that the value of \(x\) in the original process is less than \(n\),
so that we can refine the definition of \texttt{f} to:
\begin{verbatim}
 let rec f(n) = ... else (pred(n) [] let x=* in assume(x<n);f(x))
\end{verbatim}
The target program is now terminating, from which
we can deduce that the original process is also terminating.
We have implemented an automated tool based on the refined translation above.

The contributions of this paper are summarized as follows.
\begin{itemize}
\item The formalization of the basic translation from the \(\pi\)-calculus
  (extended with integers) to sequential programs, and a proof of its correctness.
\item The formalization of a refined translation based on a refinement type system.
\item An implementation of the refined translation, including automated refinement type
  inference based on CHC solving, and experiments to evaluate the effectiveness of
  our method.
\end{itemize}

The rest of this paper is structured as follows.
Section~\ref{sec:targetlanguage} introduces the source and target languages
of our translation.
Section~\ref{sec:approach} 
formalizes the basic translation, and proves its correctness.
Section~\ref{sec:refinement} refines the basic translation by using a refinement type system.
Section~\ref{sec:implementation} reports an implementation and experiments.
Section~\ref{sec:relatedwork} discusses related work,
and Section~\ref{sec:conclusion} concludes the paper.
 
\section{Related Work}
The loopholes we present in this paper are explored using packet injection, in which an attacker sends fake WiFi packets to devices in a secured WiFi network.
Packet injection has been used in the past to perform various types of attacks against WiFi networks
such as denial of service attacks for a particular client device or total disruption of the network~\cite{vanhoef2020protecting, dos, rogue-ap, deauth}. These attacks use different approaches such as beacon stuffing to send false information to WiFi devices~\cite{beacon-stuffing-1, beacon-stuffing-2}, or Traffic Indication Map (TIM) forgery to prevent clients from receiving data ~\cite{bellardo2003802, tim-forgery}. However, all of these attacks focus on spoofing 802.11 MAC-layer management frames to interrupt the normal operation of WiFi networks. 
To provide a countermeasure for some of these attacks, the 802.11w standard~\cite{ieee802.11w} 
introduces a protected management frame that prevents attackers from spoofing 802.11 management frames. 
Instead of spoofing 802.11 MAC frames, we exploit properties of the 802.11 physical layer to force a device to stay awake and respond when it should not. 
These loopholes open the door to multiple research avenues including new security and privacy threats. 


%\textcolor{blue}{Our recent preliminary work has shown that all WiFi devices respond with ACKs to packets received from outside of their network~\cite{abedi2020wifi}. However, this workshop paper does not show how to keep WiFi devices awake and avoid going to sleep mode. Moreover, it does not explore turning WiFi devices into sensitive motion sensors and monitoring people's breathing rates. In contrast,  \name\ shows the first attack which forces a WiFi device to be awake and pushes it to continuously transmit. We show how our technique enables an attacker to monitor the breathing rate of people by analyzing their WiFi signals.}

% The related work can be divided into three categories. The first category is WiFi sensing systems that use WiFi signals to infer some information, such as gesture detection, to enable a useful application for the user (the good). The second category is WiFi attacks which show how an attacker can interfere with the normal operation of WiFi networks (the bad). Finally, some recent studies show new privacy attacks against users by analyzing their WiFi signals (the ugly).


%Over the past decade, there has been a significant amount of research on WiFi sensing where WiFi signals are used to detect human activities~\cite{wifi-sensing-survey} to enable useful applications. These systems target different applications such as tracking and localization\cite{adib2013see, activity-recognition-1}, human detection \cite{gong2016adaptive, gong2015wifi}, gesture recognition \cite{abdelnasser2015wigest, gesture-recognition-1, gesture-recognition-2, gesture-recognition-3} and vital measurement such as respiration rate~\cite{abdelnasser2015ubibreathe, adib2015smart, breathing-rate-1, breathing-rate-2}. However, these systems target applications with social benefits and cannot be easily used by an attacker to create privacy and security threats. This is because either these techniques require cooperation from the target WiFi device or the attacker needs to be very close to the target to use these systems.


%WiFi Sensing is a technique that uses ambient WiFi signals to detect events or human activities~\cite{wifi-sensing-survey}. The motivation behind WiFi sensing is that we can obtain certain information without dedicated sensors. In particular, WiFi sensing techniques analyze changes in WiFi signals to infer different types of information. As mentioned earlier in Section~\ref{sec:csi}, CSI has been shown to be well suited for sensing techniques. For instance, there are applications that can identify the number of people in a closed room and their relative locations\cite{adib2013see, activity-recognition-1}. Applications that  develop wireless device-free human detection \cite{gong2016adaptive, gong2015wifi} are also implemented to determine the presence of human activities. Other than this, subtle movements like gesture recognition \cite{abdelnasser2015wigest, gesture-recognition-1, gesture-recognition-2, gesture-recognition-3} and vital measurement such as respiration rate~\cite{abdelnasser2015ubibreathe, adib2015smart, breathing-rate-1, breathing-rate-2}, can also be measured using wireless sensing. These applications are great tools that bring convenience to people's lives.However, for these techniques to work WiFi devices should cooperate to enable WiFi sensing. Therefore, an attacker cannot use these techniques to perform WiFi sensing since he/she has no access to the target building and the devices inside it.

% \subsection{The Bad}
% \label{sec:stealth}


%\emph{Beacon Injection:} 
%This attack is performed by forging 802.11 beacons and broadcasting them to all devices in a WiFi network~\cite{beacon-stuffing-1, beacon-stuffing-2}. The attacking device pretends to be the actual access point and injects false information in the forged beacons to enable a variety of attacks such as the ``evil twin access point'' attack~\cite{evil-twin}. 
%Since beacons are broadcasted to all devices, this will attack all the devices at the same time. The forged beacon frames can also be sent (i.e., unicast) to a particular device to attack individual devices rather than the entire network.

%\emph{TIM Forgery:}
%Traffic Indication Map (TIM), as mentioned in Section \ref{sec:beacon}, is used in 802.11 beacon frames. It contains information about whether sleeping devices have buffered packets at Access Point (AP) or not. It is suggested that an adversary can manipulate the Time Indication Map (TIM) inside beacons to change the behavior of WiFi devices~\cite{bellardo2003802, tim-forgery}. \name\ builds on these attacks. In particular, \name\ forges the TIM to make a device believe that it has buffered data to receive, so it cannot enter the sleep mode.
%For instance, the adversary can forge the TIM to make a device believe that it has buffered data to receive, so it cannot enter the sleep mode. In contrast, it can also prevent a device from receiving any data by telling it that the AP  has no buffered data for it. 


\textbf{WiFi sensing attack:} Over the past decade, there has been a significant amount of research on WiFi sensing where WiFi signals are used to detect human activities~\cite{iot-wifi-localization, rf-sensing, wifi-sensing-survey,adib2015smart, breathing-rate-1, breathing-rate-2,gesture-recognition-1, gesture-recognition-2, gesture-recognition-3,pu2013whole}. However, these systems target applications with social benefits and cannot be easily used by an attacker to create privacy and security threats. This is because either these techniques require cooperation from the target WiFi device or the attacker needs to be very close to the target to use these systems. A recent study shows that by capturing WiFi signals coming out of a private building, it is possible for an adversary to track user movements inside that building~\cite{zhu2018tu}. However, this attack has a bootstrapping stage which requires the attacker to walk around the target building for a long time to find the location of the WiFi devices. Furthermore, since this work relies on only the normal intermittent WiFi activities, it cannot capture continuous data such as breathing rate.  

\textbf{Battery draining attack:} 
Battery draining attacks date back to 1999 \cite{stajano1999resurrecting} and there have been many studies on such attacks and potential defense mechanisms since then~\cite{caviglione2012energy}.
%Battery draining attack was first introduced by Stajano, F. and Anderson, R. in 1999 \cite{stajano1999resurrecting}, where they prophesied that the battery of a mobile device can be exhausted by a malicious user even via legitimate usage. Extensive studies have been conducted to investigate the attack and defense manners. 
%One possible approach is by constantly launching canonical attacks and forcing defense systems to attempt to defeat them. Running defending software consumes CPU resources which leads to quick drainage of battery energy \cite{caviglione2012energy}. 
Battery discharge models and energy vulnerability due to operating systems have been investigated \cite{zhang2010accurate,jindal2013hypnos}. A more recent study plays multimedia files implicitly to increase power consumption during web browsing \cite{fiore2014multimedia, fiore2017exploiting}. In terms of defending, a monitoring agent that searches for abnormal current draw is discussed in \cite{buennemeyer2008mobile}. In contrast, our attack exploits the loopholes in the 802.11 physical layer protocol and the power-hungry WiFi transmission to quickly drain a target device's battery. We will discuss in Section~\ref{sec:cannot-be-fixed} that stopping our proposed attack is nearly impossible on today's WiFi devices.


This paper is an extension of our previous workshop publication ~\cite{polite-wifi}. The workshop paper shows preliminary results for our finding that WiFi devices respond with ACKs to packets received from outside of their network, and provides a brief discussion on potential privacy and security concerns of this behavior without studying them. We have also explored how the WiFi power saving mechanism can be exploited to keep a target device awake in a localization attack~\cite{wi-peep}. 
In this paper, we provide an in-depth study of these previously discovered loopholes. We also design and perform two privacy and security attacks, based on these loopholes. Finally, we implement these attacks on off-the-shelve WiFi devices and present detailed performance evaluations.


% To the best of our knowledge, \name is the first attack that forces WiFi devices to continuously transmit, enabling an attacker, who does not have access to the network, to estimate the breathing rate of a person from outside of the building using a low-cost WiFi module.

%\name\ combines techniques from WiFi sensing, attacks against WiFi networks to enable a new privacy attack to show for the first time that an attacker can estimate the respiration rate of a person from outside of the building.


%These attacks relies on the fact that wireless signals pass through walls; therefore, an attacker who is outside a building can receive signals coming from inside that building. The attacker can analyze the distortions in WiFi signals, caused by the body of a target person, to infer various types of information from a target building.

%To make the problem worse, it has been shown that WiFi devices send acknowledgments (ACK) when they receive fake packets coming from outside the their network~\cite{abedi2020wifi}. This behavior, called \emph{Polite WiFi}, enlightens the possibility of turning any WiFi device into a secret sensor.  This is because the CSI information can be extract from the ACKs send by a victim device to perform WiFi sensing and potentially obtain some sensitive information.
%\name\ combines techniques from WiFi sensing, attacks against WiFi networks to enable a new privacy attack to show for the first time that an attacker can estimate the respiration rate of a person from outside of the building.
%We also propose potential methodologies to protect against this attack.


% Another privacy attack which uses WiFi to extract the International Mobile Subscriber Identity (IMSI) from mobile phones \cite{o2017mobile}. The leakage of IMSI may results in potential tracking of the devices through link to the target user's other hardware addresses such as WiFi MAC addresses. As a result, the mobile device can be turned into a secret tracker which works for the attacker.


% some features of the techniques mentioned above like \emph{Polite WiFi} and TIM forgery, and it can detect the target people's breathing rate without their notice. The leak of breathing rate may seem to be harmless, but attacker can use this information to further infer people's presence in the house.





%\vspace*{-0.6cm}
\section{Integer Echo State Networks}
\label{sect:intesn}
%\vspace*{-0.3cm}
% 
% \begin{figure}[!ht]%[t!]
% \centering
% \includegraphics[width=0.7\columnwidth]{img/HD_ESN}
% \caption{Architecture of the Integer Echo State Network.}
% \label{fig:intesn}
% %\vspace*{-0.5cm}
% \end{figure}

% \begin{figure}[hbt]
% \minipage{0.49\textwidth}
%   \includegraphics[width=\linewidth]{img/HD_ESN}
%   \caption{Architecture of the Integer Echo State Network.}
% \label{fig:intesn}
% \endminipage\hfill
% \minipage{0.49\textwidth}%
%   \includegraphics[width=\linewidth]{img/Discretization}
%   \caption{Quantization and discretization of a continuous signal.}
% \label{fig:quantization}
% \endminipage
% \end{figure}


\begin{figure}[tb]%[!ht]%[t!]
\centering
\includegraphics[width=1.0\columnwidth]{img/HD_ESN_new}
\caption{Architecture of the proposed integer Echo State Network.}
\label{fig:intesn}
%\vspace*{-0.5cm}
\end{figure}

This section presents the main contribution of the article -- an architecture for integer Echo State Network. 
The architecture is illustrated in Fig.~\ref{fig:intesn}. The proposed intESN is structurally identical to the the conventional ESN (see Fig.~\ref{fig:esn}) with three layers of neurons: input ($\textbf{u}(n)$, $K$ neurons), output ($\textbf{y}(n)$, $L$ neurons), and reservoir ($\textbf{x}(n)$, $N$ neurons). It is important to note from the beginning that training the readout matrix $\textbf{W}^{\text{out}}$  for intESN is the same as for the conventional ESN (Section \ref{sect:training}). 

However, other components of intESN differs from the conventional ESN.  First,
activations of input and output layers are projected into the  reservoir in the
form of bipolar HD vectors \cite{MAP} of size $N$  (denoted as
$\textbf{u}^{\text{HD}}(n)$ and $\textbf{y}^{\text{HD}}(n)$). 
 % you said this in the previous section:
% Such projection is achieved by mapping activations into high-dimensional vectors.  According to the principles of HDC, each ``symbol''  is assigned with a random high-dimensional vector used as representation in HDC.  The correspondences are stored in the so-called item memory (Fig.~\ref{fig:intesn}),  which given the symbol issues the corresponding high-dimensional vector  (denoted as HD in the figure). 
For problems where input and output data are described by finite
alphabets and each symbol can be treated independently,  the mapping to $N$-dimensional space is achieved by simply assigning a random bipolar HD vector
to each symbol in the alphabet and storing them in the item memory \cite{Kanerva09, Kleyko2015}.  In the
case with continuous data (e.g., real numbers), we quantized the continuous values
into a finite alphabet.
%A symbol in such alphabet is a quantization level.  
The quantization scheme (denoted as $Q$) and the granularity of the
quantization are problem dependent. Additionally, when there is a need to
preserve similarity between quantization levels, distance preserving mapping
schemes are applied (see, e.g., \cite{Scalarencoding, Widdows15}), which can
preserve, for example, linear or nonlinear similarity between levels. An example
of a discretization and quantization of a continuous signal as well as its
HD vectors in the item memory is illustrated in Fig.~\ref{fig:intesn}. 
%\hl{
Continuous values can be also represented in HD vectors by varying their density. For a recent overview of several mapping approaches readers are referred to \cite{TNNLS18}. Also, an example of applying such mapping is presented in Section~\ref{sect:perf:analog}.
%}
%Thus after choosing the mapping scheme activations of neurons are projected into bipolar HD vector and this vector is added to reservoir.
Another feature of intESN is the way the recurrence in the reservoir is implemented. Rather than a matrix multiply, recurrence is implemented via the permutation of the reservoir vector. Note that permutation of a vector can be described in matrix form, which can play the role of $\textbf{W}$ in intESN. Note that the spectral radius of this matrix equals one.
However, an efficient implementation of permutation can be achieved for a special case -- cyclic shift (denoted as $\text{Sh}()$). 
%{\color{red}
It is important to note that we have shown in~\cite{Frady17} that the recurrent weight matrix $\textbf{W}$ creates key-value pairs of the input data.
Note that $\textbf{W}$ is chosen randomly and kept fixed, and this always leads to the same properties.
Moreover, there is no advantage of the fully connected random recurrent weight matrix over the simple cyclic shift operation for storing the input history.
Thus, the use of the cyclic shift in place of a random  recurrent weight matrix does not limit intESN's ability to produce linearly separable representations. 
%}
Fig.~\ref{fig:intesn} shows the recurrent connections of neurons in a reservoir with recurrence by cyclic shift of one position. In this case, vector-matrix multiplication $ \textbf{W} \textbf{x}(n) $ is equivalent to $ \text{Sh}(\textbf{x}(n),1)$.   




% In the ESN the reservoir matrix is a random matrix  which posses specific properties as described in the previous section.
%\hl{
Finally, to keep the integer values of neurons, intESN uses different nonlinear activation function for the reservoir -- clipping (\ref{eq:clipping}).
%}
Note that the simplest bundling operation is an elementwise addition. However, when using the elementwise
addition, the activity of a reservoir (i.e., a composite HD vector) is no longer bipolar. From the implementation
point of view, it is practical to keep the values of the elements of the HD vector in the limited range using  a threshold value (denoted as $\kappa$). % and make it a configurable parameter. The bounding operation is called {\it clipping}. The clipping is done as follows:
~
\begin{equation}
f_\kappa (x) = 
\begin{cases}
-\kappa & x \leq -\kappa \\
x & -\kappa < x < \kappa \\
\kappa & x \geq \kappa
\end{cases}
\label{eq:clipping}
\end{equation}
~
The clipping threshold $\kappa$ is regulating nonlinear behavior of the reservoir and limiting the range of activation values. Note that in intESN the reservoir is updated only with integer bipolar vectors, and after clipping the values of neurons are still integers in the range between $-\kappa$ and $\kappa$. Thus, each neuron can be represented using only $\log_2(2\kappa+1)$ bits of memory. For example, when $\kappa=7$, there are fifteen unique values of a neuron, which can be stored with just four bits. 
We have also shown recently that the usage of the clipping might be beneficial when implementing resource-efficient alternatives of Self-Organizing Maps~\cite{intSOM}.


Summarizing the aforementioned differences, the update of intESN is described as: 
~
\begin{equation}
\textbf{x}(n)= f_\kappa (\text{Sh}(\textbf{x}(n-1),1)+\textbf{u}^{\text{HD}}(n)+\textbf{y}^{\text{HD}}(n-1)).
\label{eq:intesnres}
 \end{equation}

















\section{Performance evaluation}
\label{sect:perf}

%\hl{
In this section, the proposed intESN approach is verified and compared to the conventional ESN and the ring-based ESN~\cite{MinESN} on a set of typical RC tasks.
In particular, three aspects are evaluated: short-term memory, classification of time-series, and modeling of dynamic processes.
Short-term memories are compared using the trajectory association task
\cite{PlateBook}, introduced in the area of holographic reduced representations
\cite{PlateTr}. Additionally, an approach for storing and decoding analog values using intESN is demonstrated on image patches. 
Classification of time-series is studied using the standard datasets from UCI and UCR.
Modeling of dynamic processes is tested on two typical cases. First, the task of learning a simple sinusoidal function is considered. Next, networks are trained to reproduce a complex dynamical system produced by a Mackey-Glass series.
Unless otherwise stated, ridge regression (the regularization coefficient is denoted as $\lambda$) with the  Moore-Penrose pseudo-inverse was used to learn the
readout matrix $\textbf{W}^{\text{out}}$. Values of input neurons $\textbf{u}(n)$ were not used for training the readout in any of the experiments below. 
%}
%Note about studies of capacity for both RC and HPC
%Moore-Penrose pseudo-inverse of matrix
















\subsection{Short-term memory}
\subsubsection{Sequence recall task}
\label{sect:perf:trajectory}






%While we leave formal analysis of the capacity for future work 

The sequence recall task includes  two stages: memorization and recall.
At the memorization stage,  a network continuously stores a sequence of tokens 
(e.g., letters, phonemes, etc).
The number of unique tokens is denoted as $D$ ($D=27$ in the experiments), and one token is presented as input each timestep.
At the recall stage, the network uses the content of its reservoir to
retrieve the token stored $d$ steps ago, where $d$ denotes delay. In the
experiments, the range of delay varied between 0 and 15.



\begin{figure}[tb]%[!ht]%[t!]
\centering
\includegraphics[width=1.0\linewidth]{img/short_memory_ort}
\caption{The accuracy of the  correct decoding of tokens for the conventional ESN, ring-based ESN, and integer ESN for
three different values of $N$.
}
\label{fig:memory}
%\vspace*{-0.5cm}
\end{figure}




For the conventional and ring-based ESNs, the  dictionary of tokens was represented by a one-hot encoding, i.e.
the number of input layer neurons was set to the  size of the dictionary
$K=D=27$. The same encoding scheme was adopted for the output layer, $L=27$.
%\hl{
The input vector  was projected to the reservoir by the projection matrix
$\textbf{W}^{in}$ where each entry was independently generated from the uniform distribution in the
range $[-1,1]$, the projection gain was set to $\beta=0.1$.  
The reservoir connection matrix $\textbf{W}$ for the conventional ESN  was first generated from the standard normal distribution and then orthogonalized. 
The reservoir connection matrix $\textbf{W}$ for the ring-based ESN  was generated as a permutation matrix. 
The feedback strength of both reservoir connection matrices was set to $\rho=0.94$. 
%}


For intESN, the item memory was populated with $D$ random high-dimensional
bipolar vectors.  The threshold for the clipping function was set to $\kappa=3$. The
output layer was the same as in ESN with $L=27$ and one-hot encoding of tokens.
%{\color{brown}
It is worth noting that $\rho$ and $\kappa$ were chosen in such a way that the accuracy curves would resemble each other as close as possible. 
The diligent readers are kindly referred to the Supplementary materials (Fig. S.1) where for the case $N=200$ the curves for the range of $\rho$ and $\kappa$ values are presented. 
%}



For each value of of the delay $d$ a readout matrix $\textbf{W}^{\text{out}}$  was trained, producing 16 matrices in total. 
The training sequence presented 2000 random tokens to the network, and only the last 1500 steps were used to compute the readout matrices. The regularization parameter for ridge regression was set to $\lambda=0$.
The training sequence of tokens delayed by the particular $d$ was used as the ground truth for the the activations of the output layer.  
During the operating phase,  both the inclusion of a new token into the reservoir and the recall of the delayed token from the reservoir 
were simultaneous.  Experiments were performed for three different sizes of the reservoir: $N=100$, $N=200$, and $N=300$.


\begin{figure}[tb]%[!ht]%[t!]
\centering
\includegraphics[width=1.0\linewidth]{img/short_memory_ort_lonf_ESN}
\caption{The accuracy of the  correct decoding of tokens for the conventional ESN, ring-based ESN, and integer ESN for
three different values of $N$. ``intESN-large'' refers to the fact that the number of neurons in intESN was equivalent to the memory footprint required by ESN for the stated number of neurons.  
}
\label{fig:memory:long}
%\vspace*{-0.5cm}
\end{figure}


The memory capacity of the network is characterized by the 
accuracy of the correct decoding of tokens for different values of the delay. 
%{\color{brown}
Fig.~\ref{fig:memory} depicts the accuracy for all networks conventional ESN (solid lines), ring-based ESN (dash-dotted line)
and intESN (dashed lines). The capacities of all the networks grow with the
increased number of neurons in the reservoir.  
Since the capacities of the conventional ESN and the ring-based ESN are almost identical, which is in line with~\cite{Strauss2012}, for the rest of this subsection we assume both of these networks when using the term ESN.  
%}


\begin{figure*}[tb]%[h]
\center{
\begin{minipage}[h]{0.8\linewidth}
\center{\includegraphics[width=1.0\linewidth]{img/intESN_images_or} }
\end{minipage}
\vfill
\begin{minipage}[h]{0.8\linewidth}
\center{\includegraphics[width=1.0\linewidth]{img/intESN_images_rec}}
\end{minipage}
\caption{
%\hl{
An example of image patches decoded from an intESN. Top row represents the original images stored in the reservoir.   
Other rows depict the patches reconstructed from intESN for different reservoir sizes and clipping thresholds.
%}
}
}
\label{fig:images}
\end{figure*}


%\hl{
The capacities of ESN and intESN  are comparable for small $d$, i.e., for the most recent tokens. 
For the increased delays the curves featured slightly different behaviors. 
With increase of the value of $d$ the performance of intESN started to decline faster compared to ESN.
%For all values of $N$ the accuracy of  intESN starts to deviate from 100\% earlier than that of  ESN. Also, intESN features slightly steeper decay than ESN. 
Eventually, all curves converge to the value of the random guess which equals $1/D$. 
%Moreover, it is possible to characterize the information capacity using a single number -- the amount of information which can be decoded from the reservoir. 
Moreover, the information capacity of a network is characterized by the amount of the decoded from the reservoir information.
This amount is determined using the amount of information per token ($\log_2D$), the probability of correctly decoding a token at each delay value, and the concept of mutual information. We calculated the amount of information for all networks in Fig.~\ref{fig:memory} in the considered delay range. For 100 neurons intESN preserved 19.3\% less information, for 200 and 300 neurons 21.7\% less.
%}




%\hl{
%On the other hand, intESN with $\kappa=3$ requires only 3-bit per neuron. It is assumed that one ESN neuron requires 32-bit then if the number of neurons in %intESN is increased ten times the reservoir memory footprints of two networks are going to be comparable. The results for this case are presented in Fig.~\ref{fig:memory:long} (the training sequence was prolonged to 9000 random tokens). In such setting intESN has clearly higher information capacity. In particular, %for ESN memory footprint with 100 neurons the decoded amount of information has increased 2.2 times while for 200 and 300 neurons it increased 1.6 and 1.3 times %respectively. 
%}











%\hl{
These results highlight a very important trade-off: the performance versus a complexity of implementation. While
the performance of intESN is somewhat poorer in this task, one has to bear in mind its memory footprint. With the
clipping threshold $\kappa=3$ only 3-bit are needed to represent the state of a neuron compared to 32-bit
per neuron (the size of type float) in ESN. 
%{\color{red}
In other words, intESN allowed lowering the memory footprint of the reservoir by an order of magnitude by sacrificing only a fraction of the performance with respect to the information capacity. 
Thus, we conjecture that some reduction in the performance for ten folds
memory footprint reduction is an acceptable price in applications on resource-constrained computing
devices.
%}
On the other hand, we can check the performance of the networks with equal memory
footprints. 
For this we increased the number of neurons in intESN so that the total memory consumed by
the reservoir with the same clipping threshold $\kappa=3$ would match that of the conventional or ring-based ESN. 
This network is denoted as ``intESN-large''.
%{\color{red}
Since $\kappa=3$ requires only 3-bit, in order to get the memory footprint corresponding to ESN, intESN could use more than ten times more neurons. 
Thus, the memory footprint of intESN with $1000$ neurons corresponds to ESN with $100$ neurons; while intESN with $2000$ and $3000$  neurons correspond to ESN with $200$ and $300$ neurons respectively.
%}
 The results for this case are presented in Fig.~\ref{fig:memory:long} (the training sequence was prolonged to
9000 random tokens). With such settings intESN-large has clearly higher information capacity. In
particular, for ESN memory footprint with 100 neurons the decoded amount of information has increased
2.2 times while for 200 and 300 neurons it increased 1.6 and 1.3 times respectively.
%}
%{\color{brown}
It is important to note, however, that while the memory consumed by the reservoir of intESN-large was comparable to the corresponding ESN, the readout matrix for intESN-large was larger and more computationally demanding than the ESN readout matrix since the size of a readout matrix is proportional to the number of neurons in the reservoir. 
%}





%ESN: 100 - 36.9;  200 - 58.6; 300 - 70.5; 
%intESN: 100 - 29.8;  200 - 45.9; 300 - 55.2; 
%Long training.
%ESN: 100 - 46.9;  200 - 74.9; 300 - 95.43; 
%intESN: 100 - 103.3;  200 - 117.4; 300 - 122.8; 

\subsubsection{Storage of analog values in intESN}
\label{sect:perf:analog}







%\hl{
This subsection presents the feasibility of storing analog values in intESN using image patches as a showcase. 
%{\color{red}
It is important to emphasize that this subsection  does not go into detailed comparisons with other methods as the main purpose here is 
the principal demonstration of the possibilities of storing continuous data in reservoirs consisting of integers in a limited range. 
In other words, with this showcase, we are aiming at demonstrating the feasibility of using integer approximation of neuron states in intESN to work with analog representations.
%}
A value of a pixel (in an RGB channel) can be treated as an analog value in the range between 0 and 1. 
For each pixel it is possible to generate a unique bipolar HD vector. 
%The pixel's value is encoded by multiplying all elements of the HD vector by that value. 
The typical approach to encode an analog value is to multiply all elements of the HD vector by that value.
The whole sequence is then represented using the bundling operation on all scaled HD vectors. The result of bundling can be used as an input to a reservoir. However, the resultant composite HD vector will not be in the integer range anymore. 
%This could be addressed in using scaling via sparsity. 
We address this problem by using sparsity.
Instead of scaling elements of an HD vector, we propose to randomly set the fraction of elements of the HD vector to zeros, i.e., the HD vector will become ternary. The proportion of zero elements is determined by the pixel's analog value. Pixels with values close to zero will have very sparse HD vectors while pixels with values close to one will have dense HD vectors, but all entries will always be
[-1, 0, or +1].
The result of bundling of such HD vectors (i.e., HD vector for an image) will still have integer values. Such representational scheme allows keeping integer values in the reservoir but it still can effectively store analog values.
%}




The examples of results are presented in Fig.~\ref{fig:images}. Top row depicts original images stored in the reservoir. The other rows depict images reconstructed from the reservoir. The following parameters of intESN were used (top to bottom): $N=64000$, $\kappa=11$; $N=32000$, $\kappa=8$;  $N=16000$, $\kappa=6$;  $N=8000$, $\kappa=4$. The values of $\kappa$ were optimized for a particular $N$. Columns correspond to the delay values (i.e., how many steps ago an image was stored in the reservoir) as in the previous experiment. 
As one would anticipate, the quality of the reconstructed images is improving for larger reservoir sizes. At the same time, the quality of the reconstructed images is deteriorating for larger delay values, i.e., the worst quality of the reconstructed image could be observed in the bottom right corner while the best reconstruction is located in the top left corner.
Nevertheless, the main observation for this experiment is that it is possible to project analog values into the reservoir with integer values using the mapping via varying sparsity and then retrieve the values from the reservoir.
Moreover, we have shown recently~\cite{intRVFL} that the mapping via varying sparsity could even be helpful when solving classification problems with a feed-forward variant of the ESN.

\begin{table}[tb]%[ht]
\renewcommand{\arraystretch}{1.3}
\caption{Details of datasets for time-series classification.\label{tab:datasets}
\vspace{-2mm}}
    % {\scriptsize
    \begin{center}
    \begin{tabular}{|c|c|c|c|c|}\hline
      	\multicolumn{5}{|c|}{\textbf{Univariate datasets from UCR}} \\ \hline\hline
        \textbf{Name} & \textit{\textbf{\#V}} & \textbf{Train} & \textbf{Test} & \textit{\textbf{\#C}} \\ \hline
        Swedish Leaf 	& 1	& 500 	& 625 	& 15 \\ \hline  
 	Distal Phalanx & 1	& 139 & 400 &	3 \\ \hline
	ECG 		& 1 	& 100 & 100 &	2 \\ \hline	
        Wafer  		& 1 	& 1000	 & 6164 &	2\\ \hline\hline       
        \multicolumn{5}{|c|}{\textbf{Multivariate datasets from UCI}} \\ \hline\hline
	 Character Trajectories & 3 		& 300 & 2558 & 20  \\\hline
	 Spoken Arabic Digit 	 & 13 	& 6600 & 2200 & 10  \\\hline
        Japanese Vowels 	& 12 	& 270 & 370 & 9 \\ \hline    	
    \end{tabular}
    \end{center}
%  }
%\vspace{-5mm}
\end{table}


\subsection{Classification of time-series}

%\hl{
In this section, ESN (conventional and ring-based) and intESN networks are compared in terms of classification accuracy obtained on standard time-series datasets. 
Following \cite{Bianchi2018} we used several (four) univariate datasets from UCR\footnote{UCR. Time Series Classification Archive [online], 2018. -- Available online: \url{https://www.cs.ucr.edu/\%7Eeamonn/time\_series\_data\_2018/}.}  and several (three) multivariate datasets from UCI\footnote{UCI. Machine Learning Repository [online], 2019. -- Available online: \url{http://archive.ics.uci.edu/ml/datasets.html}.}.
Details of datasets are presented in Table~\ref{tab:datasets}. 
For each dataset, the table includes the name, number of variables (\#\textit{V}), number of classes (\#\textit{C}), and the number of examples in training and testing datasets.  
%}






%\hl{
Configurations of the networks were kept fixed for all datasets. 
%{\color{brown}
In fact, the configuration of the conventional and ring-based ESNs were set in accordance to~\cite{Bianchi2018}: 
reservoir size was set to $N=800$, projection gain was set to $\beta=0.25$, the feedback strength was set to $\rho=0.99$. 
The regularization parameter for the ridge regression was set to $\lambda=1.0$.
The intESN was also trained with the same $\lambda$.
%}
The clipping threshold for the intESN  was set to $\kappa=7$. 
%{\color{red}
Also for intESN the quantized values of time-series were mapped to bipolar vectors using scatter codes \cite{TNNLS18, scatter}. 
The input signal \textbf{u}(n) was quantized as:
~
\begin{equation}
\textbf{u}(n)_q= \lfloor 200\textbf{u}(n) \rceil /200,
\label{eq:quan:timeser}
 \end{equation}
~
where $\lfloor * \rceil$ denotes rounding to the the closest integer. 
%}
Two sizes of intESN's reservoir were used. The first size corresponded to the size of the conventional and ring-based ESNs, i.e., $N=800$. The second size (``intESN-large'') corresponded to the same memory footprint\footnote{Except for the Japanese Vowels dataset where such reservoir size seemed to significantly overfit the training data. In that case, the number of neurons was increased twice.} required for ESN reservoir assuming that one ESN neuron requires 32-bit while one intESN neuron requires 4-bit (when $\kappa=7$). 
%{\color{red}
Thus, ``intESN-large'' had  $N=6400$ neurons. 
%}
%}




%\hl{
The output layers of networks were representing one-hot encodings of classes in a dataset, i.e., for the particular dataset $L=$\#\textit{C} of that dataset. 
The readout layers of all networks were trained using time-series from a training dataset in the so-called endpoints mode \cite{ComparisonReadOut} when only final temporal reservoir states for each time-series  are used for training a single readout matrix. 
%}


\begin{figure}[tb]%[!ht]%[t!]
\centering
\includegraphics[width=1.0\linewidth]{img/univariate}
\caption{The classification accuracy for univariate datasets from UCR. Bars depict mean values, lines depict standard deviations.  Bars denoted as ``ESN'' and ``intESN'' had the same number of neurons in their reservoirs while for ``intESN-large'' the number of neurons corresponded to ESN's memory footprint. 
}
\label{fig:univar}
\end{figure}




\begin{figure}[tb]%[!ht]%[t!]
\centering
\includegraphics[width=1.0\linewidth]{img/multivariate}
\caption{The classification accuracy for multivariate datasets from UCI. Bars depict mean values, lines depict standard deviations. Bars denoted as ``ESN'' and ``intESN'' had the same number of neurons in their reservoirs while for ``intESN-large'' the number of neurons corresponded to ESN's memory footprint. 
}
\label{fig:multivar}
\end{figure}




%\hl{
The experimental accuracies obtained from the networks for the considered datasets are presented in Fig.~\ref{fig:univar} and ~\ref{fig:multivar}. Fig.~\ref{fig:univar} presents the results for univariate datasets while Fig.~\ref{fig:multivar} presents the results for multivariate datasets. The figures depict mean and standard deviation values across ten independent random initializations of the networks.
%{\color{brown}
Similar to subsection~\ref{sect:perf:trajectory}, the accuracy of the conventional ESN and the ring-based ESN are almost identical, thus, for the rest of this subsection we assume both of these networks when using the term ESN.  
%}


%}

%\hl{
The obtained results strongly depend on the characteristics of the data. However, it was generally observed that intESN with the memory footprint equivalent to ESN demonstrated higher classification accuracy. 
On the other hand, the classification accuracy of intESN with the same number of neurons as in ESN was similar to ESN's performance for all considered datasets but two (``Swedish Leaf'' and ``Character Trajectories'') for which the accuracy degradation was sensible.
We, therefore, conjecture that in a general case, one cannot guarantee the same classification accuracy as for
ESN. The empirical evidence, however, shows that it is not infeasible. Since placing the reported results into
the general context of time-series classifications is outside the scope of this article we do not further
elaborate on fine-tuning of hyperparameters of intESN for the best classification performance.
%On the other hand, classification accuracy of intESN with the same number of neurons as in ESN was similar to ESN's performance for most of the datasets but it %was significantly lower for two datasets: ``Swedish Leaf'' and ``Character Trajectories''. Therefore, in a general case, one cannot guarantee the same accuracy %but the empirical evidence shows that it is not unlikely. 
%Finally, it should be noted that we did not aim at placing the reported results into the general context of time-series classifications. Instead, the goal was to compare the results of intESN and ESN.
%}
%{\color{brown}
However, the interested readers are kindly referred to the Supplementary materials (Fig. S.2 and S.3) where several different values of $N$, $\kappa$, and $\rho$ were examined for each dataset. 
%}







\subsection{Modeling of dynamic processes}
\subsubsection{Learning Sinusoidal Function}


\begin{figure}[tb]%[h]%[htb]
\centering
\includegraphics[width=1.0\linewidth]{img/Sinus.png}
\caption{Generation of a sinusoidal signal.}
\label{fig:Sinus}
\end{figure}

The task of learning a sinusoidal function \cite{ESN02} is
an example of a learning  simple dynamic system with the constant cyclic behavior. The 
ground truth signal was generated as follows:
~
\begin{equation}
y(n)=0.5\sin(n/4).
\label{eq:sin}
 \end{equation}
~
In this task, the input layer was not used, i.e. $K=0$ but the network
projected the activations  of the output layer back to the reservoir using $\textbf{W}^{\text{back}}$. The output
layer had only one neuron ($L=1$).  The reservoir size was fixed to $N=1000$ neurons.
The length of the training  sequence was 3000 (first 1000 steps were discarded
from the calculation). For ESN, the feedback strength for the reservoir connection matrix was set to $\rho=0.8$,
for both networks  $\lambda$ was set to 0.
A continuous value of the ground truth signal was fed-in to ESN during the training. 

For  intESN, in order to map the input signal to a bipolar vector the quantization was used. The signal was quantized as:
~
\begin{equation}
y(n)_q=\lfloor100y(n)\rceil/100.
\label{eq:quan}
 \end{equation}
~
The item memory for the projection of the output layer was populated with
bipolar vectors preserving linear  (in terms of dot product) similarity between
quantization levels \cite{Widdows15}.  The threshold for the clipping function
was set to $\kappa=3$.
 

\begin{figure*}[tb]%[htb]
\centering
\includegraphics[width=1.0\linewidth]{img/Mackey-Glass.png}
\caption{Prediction of the Mackey-Glass series.}
\label{fig:Mackey-Glass}
\end{figure*}


In the operating phase,  the network acted as the generator of the signal
feeding its previous prediction (at time $n-1$) back to the reservoir. 
Fig.~\ref{fig:Sinus} demonstrates  the behavior of intESN during the first 100
prediction steps.  The ground truth is depicted by dashed line while the
prediction of intESN  is  illustrated by the shaded area between 10\% and 90\%
percentiles (100 simulations were performed).  The figure does not show the
performance of the conventional ESN as it just followed the ground truth without
visible deviations. intESN clearly follows the values of the ground truth
but the deviation  from the ground truth is increasing with the number of
prediction steps.  It is unavoidable for the increasing prediction horizon but,
in this scenario,  it is additionally accelerated due to the presence of the
quantization error at each prediction step. 
%{\color{brown}
It is worth noting, however, that the quality of predictions for this task could be improved by increasing the value of $\kappa=3$, i.e., at the cost of extract memory allocated for each neuron. 
The diligent readers are kindly referred to the Supplementary materials (Fig. S.4) where several different values of $\kappa$ were examined. 
%}
The next subsection will clearly demonstrate effects caused by the quantization process. 
The error is accumulated because every time when feeding the prediction back to the reservoir of intESN it should be quantized in order to fetch a vector from the item memory. 




\subsubsection{Mackey-Glass series prediction}

A Mackey-Glass series is generated by the nonlinear time delay differential equation. 
It is commonly used to assess the predictive power of an RC approach.  
In this scenario, we followed  the preprocessing of data and the parameters of
ESN described in \cite{ESN04}.
The parameters of intESN  (including quantization scheme) were the same as in
the subsection above.  
%{\color{brown}
The interested readers are kindly referred to the Supplementary materials (Fig. S.5) where several different values of $N$ and $\kappa$ were examined. 
%}
The length of the training sequence was 3000 (first 1000
steps were discarded from the calculation).
Fig.~\ref{fig:Mackey-Glass} depicts  results for the first 300 prediction
steps. The results were calculated from 100 simulation runs.  The figure
includes four panels. Each panel depicts the  ground truth, the mean value of
predictions as well  as areas marking percentiles between 10\% and 90\%. The
lower right  corner corresponds to intESN while three other panels show
performance of ESN in three different cases related to the quantization of
the data.

In these scenarios ESN was trained to learn the model from the  quantized
data in order to see to which extent  it affects the network.
The upper left  corner corresponds to ESN without data quantization. In this
case, the predictions precisely follow the ground truth.  The upper right corner
corresponds to ESN trained on the quantized data but  with no quantization
during the operational phase. In such settings, the network  closely follows the
ground truth for the first 150 steps but then it often explodes.  The lower left
corner corresponds to ESN where the data was quantized during  both training
and prediction. In this scenario, the network was able produce to  produce good
prediction just for the first few dozens of steps and then entered the  chaotic
mode where even the mean value does not reflect the ground truth.  These cases
demonstrate how the quantization error could affect the predictions  especially
when it is added at each prediction step.
Note that intESN operated in  the same mode as the third ESN. Despite this
fact, its performance rather resembles that of  the second ESN where the speed
deviation of the ground truth is faster.  At the same time, the deviation of
intESN grows smoothly without a sudden explosion  in contrast to ESN.

%All examples demonstrate decent approximation of the performance achieved by the conventional ESN.




\input{tex/fpga}
% !TEX root = Guillon2017_arxiv.tex

%% Short Recap

Graph analysis of brain networks have been largely exploited in the study of AD with the aim to extract new predictive diagnostics of disease progression.
Typical approaches in functional neuroimaging, characterized by oscillatory dynamics, analyze brain networks separately at different frequencies thus neglecting the available multivariate spectral information.
Here, we adopted a method to formally take into account the topological information of multi-frequency connectomes obtained from source-reconstructed MEG signals in a group of AD and healthy subjects during EC resting states.

%% Multiplex Results

Main results showed that, while flattening networks of different frequency bands attenuates differences between AD and HC populations, keeping the multiplex nature of MEG connectomes allow to capture higher-order discriminant information.
AD subjects exhibited an aberrant multiplex brain network structure that significantly reduced the global propensity to facilitate information propagation across frequency bands as compared to HC subjects (\autoref{fig:participation}b, inset). This could be in part explained by the higher variability of the individual node degrees across bands (\autoref{fig:coefficient_of_variation}).

% NOTE: High MPCi does not necessarily mean high oi (autoref fig:mpca) but it also seems that having a high number of connections (high oi) and a low MPC is not possible in the case of the brain.

% NOTE: In general, a ROI with a high MPC but with low oi, will have an even higher MPC if its oi increase (for another subject for instance). In clear: for a given i (i.e. a given ROI), the corrcoeff between oi and MPCi is always positive.

% NOTE: I tested different thresholds and with the ImCoh to check if it was not because of the week noisy connections, but the distribution of MPCi values seems to be always the same. Even with an average degree of 1 meaning that the brain always tends to keep connections in multiples frequency bands in the same time. Could it be explained by the fact that coherence is influenced by harmonics?

Such loss of inter-frequency centrality was mostly localized in association areas as well as in the cingulate cortex (\autoref{fig:participation}b; \autoref{tab:local_participation}), which resulted the most important hub promoting interaction across bands in the HC group (\autoref{fig:mpc}a).
Because all these areas are typically affected by AD atrophy \citep{wenk_neuropathologic_2003} we hypothesize that the anatomical withering might have impacted the neural oscillatory mechanisms supporting large-scale brain functional integration. Notably, the significant alteration of the connectivity across bands observed in the cingulate cortex could be ascribed to typical M/EEG connectivity changes observed in AD, such as reduced $alpha$ coherence \citep{stam_magnetoencephalographic_2006,jeong_eeg_2004,dauwels_diagnosis_2010,wang_power_2015} (\autoref{fig:mpc}b).
We also found a significant decrease in the primary motor cortex (right precentral gyrus). While previous studies have identified this specific region as a connector hub in human brain networks \citep{tijms_alzheimers_2013}, its role in AD still needs to be clarified in terms of node centrality's changes with respect to healthy conditions.
%For these affected ROIs the decreased centrality was reflected by fewer interactions with higher sensory rhythms ($>20$ Hz) \citep{basar_review_2013} and more connections to lower attentional ones ($<13$ Hz) \citep{klimesch_EEG_1999} (\autoref{fig:participation}c).

% Single-Layer Results
While flattening network layers represents in general an oversimplification, analyzing single layers can still be a valid approach that is worth of investigation.
Because the $MPC$ is a pure multiplex quantity, we considered the conceptually akin version for single-layer networks, the standard participation coefficient $PC$, which evaluates the tendency of nodes to integrate information from different modules, rather than from different layers \citep{guimera_cartography_2005, battiston_structural_2014}.
AD patients exhibited lower inter-modular connectivity in the \textit{gamma} band with respect to HC subjects (\autoref{fig:participation}a; \autoref{tab:local_participation}) that was localized in association areas including frontal, temporal, and parietal cortices (\autoref{fig:participation}a; \autoref{tab:local_participation}).
%
Damages to these regions can lead to deficits in attention, recognition and planning \citep{purves_neuroscience_2001}. Our results support the hypothesis that AD could include a disconnection syndrome  \citep{pearson_anatomical_1985,arnold_topographical_1991,catani_rises_2005}.
Furthermore, they are in line with previous findings showing $PC$ decrements in AD, although those declines were more evident in lower frequency bands and therefore ascribed to possible long-range low-frequency connectivity alteration \citep{de_haan_disrupted_2012,tijms_alzheimers_2013}.

%% Conclusion
Put together, our findings indicated that AD alters the global brain network organization through connection disruption in several association regions, which play important roles in sensory processing by integrating information from other cortical regions through high-frequency channels \citep{miltner_coherence_1999-1,buschman_top-down_2007, siegel_neuronal_2008, gregoriou_high-frequency_2009, hipp_oscillatory_2011}.
%
Notably, we showed that the global loss of inter-modular interactions in the \textit{gamma} band is paralleled by a diffused decrease of inter-frequency centrality.
Future studies, involving recordings of limbic structures and/or stimulation-based techniques, should elucidate whether these two distinct reorganizational processes are truly independent or linked through possible cross-frequency mechanisms which are known to be essential for normal memory formation \citep{canolty_high_2006,axmacher_cross-frequency_2010, goutagny_alterations_2013}.


%% Classification Results

As a confirmation of the complementary information carried out by the multi-layer approach, we reported an increased classification accuracy when combining the local $PC$ and $MPC$ features.
The observed diagnostic power is in line with previous accuracy values obtained with standard graph theoretic approaches (around $80\%$) but exhibits slightly higher sensitivity ($>90\%$), which is often desired to avoid false negatives \citep{li_discriminant_2012, wang_disrupted_2013, wee_enriched_2011, wee_identification_2012, horwitz_functional_2011}.
Other approaches should determine if and to what extent the use of more sophisticated machine learning algorithms, or the inclusion of basic connectivity features \citep{hutchison_network-based_2011, shao_prediction_2012, zhou_hierarchical_2011} and different imaging modalities \citep{dai_discriminative_2012}, can lead to higher classification performance and better diagnosis \citep{tijms_alzheimers_2013}.

%% Correlation With MMSE

Previous works have documented relationships between brain network properties and neuropsychological measurements in AD, suggesting a potential impact for monitoring disease progression and for the development of new therapies
\citep{de_haan_functional_2009,lo_diffusion_2010,sanz-arigita_loss_2010-1,shu_disrupted_2012,stam_small-world_2007,wang_disrupted_2013}.
This is especially true for the standard $PC$ which has exhibited stronger correlations and larger between-group differences \citep{tijms_alzheimers_2013}.
In line with this prediction, we also reported significant correlations between the MMSE cognitive scores and the $PC$ values of the AD patients in the \textit{gamma} band (\autoref{fig:correlations}a).
%
An even stronger correlation was found, however, for the global $MPC$ values and the TR scores (\autoref{fig:correlations}b, \autoref{tab:local_correlation}).
Recent studies suggest that TR scores could be more specific for AD \citep{grober_free_2010, velayudhan_review_2014} as compared to MMSE scores which could be biased by differences in years of education, lack of sensitivity to progressive changes occurring with AD, as well as fail in detecting impairment caused by focal lesions \citep{tombaugh_mini-mental_1992}.
Locally, the regions whose $MPC$ correlated with TR were part of the default-mode network (DMN) (\autoref{tab:local_correlation}), which is heavily involved in memory formation and retrieval \citep{buckner_brains_2008,sperling_functional_2010}. According to recent hypothesis, these areas are directly affected by atrophy and metabolism disruption, as well as amyloid-$\beta$ deposition \citep{buckner_molecular_2005, greicius_default-mode_2004}.
Put together, our results suggest that AD symptoms related to episodic memory losses could be determined by the lower capacity of strategic DMN association areas to let information flow across different frequency channels.

\subsection*{Methodological considerations}

We estimated brain networks by means of spectral coherence, a connectivity measure widely used in the electrophysiological literature because of its simplicity and relatively intuitive interpretation \citep{srinivasan_eeg_2007}.
While this measure is known to suffer from possible volume conduction effects, recent evidence showed that source reconstruction techniques, like the one we adopted here, could at least mitigate this bias \citep{schoffelen_source_2009} and generate connectivity patterns consistent within and between subjects \citep{colclough_how_2016}.
In a separate analysis, we used the imaginary coherence as a candidate alternative to eliminate volume conduction effects \citep{nolte_identifying_2004}. We demonstrated that while no significant between-group differences could be obtained in terms of $MPC$ (data not shown here), the spatial distribution of the $MPC$ values was very similar to that observed in the brain networks obtained with the spectral coherence, especially for the internal regions along the longitudinal fissure (\autoref{fig:mpc_imcoh}).

Differently from other multiplex network quantities, such as those based on paths and walks \citep{boccaletti_structure_2014}, the $MPC$ has the advantage to not depend on the weights of the inter-layer links which, in general, are difficult to estimate or to assign from empirically obtained biological data. This is especially true in network neuroscience where, so far, the strength of the inter-layer connections is parametric and subject to arbitrariness \citep{de_domenico_mapping_2016} or estimated through measures of cross-frequency coupling \citep{brookes_multi-layer_2016-1} whose biological interpretation remains still to be completely elucidated \citep{jirsa_cross-frequency_2013}.

\vspace{-1em}
\section{Conclusions}
\vspace{-0.6em}
\label{SEC:CONC}
In this paper, we investigated the impact of workload dependent parameters on the failure ratio of the SSDs under power outage. To this end, we presented a fault injection and failure detection platform which injects the realistic power faults to the under test SSDs. During power failure, SSDs experience the exact voltage drop behavior that occurs during power failures in data centers. The results of our experiments reveal that the failure ratio in SSDs due to power outage is significantly affected by the parameters of the running workloads in the application layer. In addition, we show that failures in SSDs are not only due to volatile DRAM cache but also we observe similar failures in SSDs with disabled internal cache.





\bibliography{bica}

\end{document}
