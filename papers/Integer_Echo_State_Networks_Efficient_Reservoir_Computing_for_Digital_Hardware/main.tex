% -*- TeX:SE -*-
%
% main.tex

 
  
%\documentclass[conference]{IEEEtran}
\documentclass[journal]{IEEEtran}
 
%\usepackage[latin1]{inputenc}
\usepackage[T1]{fontenc}
\usepackage{times}
%\usepackage{natbib} %IEEEtrans bibliography doesn't like natbib apparently
\usepackage{dcolumn}
\usepackage{endnotes}
\usepackage{graphics}
%\usepackage{subfigure}
%\usepackage{caption}
%\usepackage{subcaption}
\usepackage{times}
\usepackage{epsfig}
\usepackage{graphicx}
\usepackage{textcomp}
\usepackage{amssymb,amsmath}
\usepackage{balance}

\usepackage{listings}
\usepackage[]{algorithm2e}
\usepackage{flushend}
\usepackage{color,soul}
%\usepackage[hyphenbreaks]{breakurl}
%\usepackage[hyphens]{url}
%\usepackage{hyperref}
%\def\UrlBreaks{\do\/\do-}
%\usepackage{url}
%\usepackage[hyphens]{url}
\usepackage{hyperref}
%For making to do notes
\usepackage[dvipsnames]{xcolor}
\usepackage{todonotes}

\usepackage{siunitx}
\usepackage{pifont}% http://ctan.org/pkg/pifont
\newcommand{\cmark}{\ding{51}}%
\newcommand{\xmark}{\ding{55}}%

%For hl command
\soulregister\cite7
\soulregister\ref7
\soulregister\pageref7

%\usepackage[hyphenbreaks]{breakurl}
\def\UrlBreaks{\do\/\do-}

%\usepackage[colorlinks,urlcolor=blue]{hyperref}

%\usepackage[color=yellow,textsize=footnotesize]{todonotes}
%{\color{red} highlighted}


%\bibliographystyle{unsrt}
\bibliographystyle{IEEEtran} % IEEEtrans for @Patent
\urlstyle{sf}

\begin{document}

%\title{Integer Echo State Networks: Hyperdimensional Reservoir Computing}
\title{Integer Echo State Networks: Efficient Reservoir Computing for Digital Hardware}

% Approach for distributed representation and processing of a sensory data based on principles of hyper-dimensional computing


\author{Denis~Kleyko,
        E.~Paxon~Frady,
        Mansour~Kheffache,
        and~Evgeny~Osipov % <-this % stops a space
\thanks{%Manuscript received on May 1, 2019; 
This work  was  supported in part  by the Swedish Research Council (grant No. 2015-04677). 
The work of DK was supported by the European Union’s Horizon 2020 Research and Innovation Programme under the Marie Skłodowska-Curie Individual Fellowship Grant Agreement 839179 and in part by the DARPA’s VIP (Super-HD Project) and AIE (HyDDENN Project) programs.
%revised May 3, 2018; accepted May 3, 2018. Date of publication May 3, 2018; date of current version May 3, 2018.}
%\thanks{This work was supported by the Swedish Research Council (grant no. 2015-04677) and Systems on Nanoscale Information fabriCs (SONIC), one of the six SRC STARnet Centers, sponsored by MARCO and DARPA. DK also acknowledges Stiftelsen Seth M Kempes Stipendiefond for partially funding his research visit to UC Berkeley.
}
\thanks{\mbox{*}D. Kleyko is with the Redwood Center for Theoretical Neuroscience at the University of California, Berkeley, CA 94720, USA and also with Intelligent Systems Lab at Research Institutes of Sweden, 164 40 Kista, Sweden. \mbox{E-mail}: \mbox{denis.kleyko@ri.se}}% <-this % stops a space 
%\thanks{E. P. Frady is with Neuromorphic Computing Lab, Intel Labs and also with the Redwood Center for Theoretical Neuroscience at the University of California, Berkeley, CA 94720, USA. \mbox{E-mail}: \mbox{epaxon@berkeley.edu}
\thanks{E. P. Frady is with the Redwood Center for Theoretical Neuroscience at the University of California, Berkeley, CA 94720, USA. \mbox{E-mail}: \mbox{epaxon@berkeley.edu}
}% <-this % stops a space
\thanks{M. Kheffache is with Netlight Consulting AB, 111 53 Stockholm, Sweden. \mbox{E-mail}: \mbox{mansour.kheffache@netlight.com}}% <-this % stops a space 
\thanks{E. Osipov is with the Department of Computer  Science Electrical and Space Engineering, Lule\aa{} University of Technology, 971 87 Lule\aa{}, Sweden. \mbox{E-mail}: \mbox{evgeny.osipov@ltu.se} }% <-this % stops a space
 }% <-this % stops a space







\markboth{}%
{Kleyko \MakeLowercase{\textit{et al.}}: Integer Echo State Networks}


\maketitle

% I would change the first sentence (the rest is fine) to:
%
\begin{abstract}

%We propose an integer approximation of Echo State Networks (ESN) based on the
%mathematics of hyperdimensional computing.
%The reservoir of the proposed Integer Echo State Network (intESN) contains only \textit{n}-bits integers and replaces the recurrent matrix multiply %with an efficient cyclic shift operation. Such an architecture results in dramatic improvements in memory footprint and computational efficiency, with minimal performance loss.
%Our architecture naturally supports the usage of the trained reservoir in  symbolic processing tasks of analogy making and logical inference. 

We propose an approximation of Echo State Networks (ESN) that can be efficiently implemented on digital hardware based on the mathematics of hyperdimensional computing. The reservoir of the proposed integer Echo State Network (intESN) is a vector containing only n-bits integers (where n<8 is normally sufficient for a satisfactory performance). The recurrent matrix multiplication is replaced with an efficient cyclic shift operation. The proposed intESN approach is verified with typical tasks in reservoir computing: memorizing of a sequence of inputs; classifying time-series; learning dynamic processes. Such architecture results in dramatic improvements in memory footprint and computational efficiency, with minimal performance loss. 
%{\color{red}
The experiments on a field-programmable gate array confirm that the proposed intESN approach is much more energy efficient than the conventional ESN.
%}
\end{abstract}

\begin{IEEEkeywords}
reservoir computing, echo state networks, vector symbolic architectures, hyperdimensional computing, memory capacity, time-series classification, dynamic systems modelling
 \end{IEEEkeywords}


% \leavevmode
% \\
% \\
% \\
% \\
% \\
\section{Introduction}
\label{introduction}

AutoML is the process by which machine learning models are built automatically for a new dataset. Given a dataset, AutoML systems perform a search over valid data transformations and learners, along with hyper-parameter optimization for each learner~\cite{VolcanoML}. Choosing the transformations and learners over which to search is our focus.
A significant number of systems mine from prior runs of pipelines over a set of datasets to choose transformers and learners that are effective with different types of datasets (e.g. \cite{NEURIPS2018_b59a51a3}, \cite{10.14778/3415478.3415542}, \cite{autosklearn}). Thus, they build a database by actually running different pipelines with a diverse set of datasets to estimate the accuracy of potential pipelines. Hence, they can be used to effectively reduce the search space. A new dataset, based on a set of features (meta-features) is then matched to this database to find the most plausible candidates for both learner selection and hyper-parameter tuning. This process of choosing starting points in the search space is called meta-learning for the cold start problem.  

Other meta-learning approaches include mining existing data science code and their associated datasets to learn from human expertise. The AL~\cite{al} system mined existing Kaggle notebooks using dynamic analysis, i.e., actually running the scripts, and showed that such a system has promise.  However, this meta-learning approach does not scale because it is onerous to execute a large number of pipeline scripts on datasets, preprocessing datasets is never trivial, and older scripts cease to run at all as software evolves. It is not surprising that AL therefore performed dynamic analysis on just nine datasets.

Our system, {\sysname}, provides a scalable meta-learning approach to leverage human expertise, using static analysis to mine pipelines from large repositories of scripts. Static analysis has the advantage of scaling to thousands or millions of scripts \cite{graph4code} easily, but lacks the performance data gathered by dynamic analysis. The {\sysname} meta-learning approach guides the learning process by a scalable dataset similarity search, based on dataset embeddings, to find the most similar datasets and the semantics of ML pipelines applied on them.  Many existing systems, such as Auto-Sklearn \cite{autosklearn} and AL \cite{al}, compute a set of meta-features for each dataset. We developed a deep neural network model to generate embeddings at the granularity of a dataset, e.g., a table or CSV file, to capture similarity at the level of an entire dataset rather than relying on a set of meta-features.
 
Because we use static analysis to capture the semantics of the meta-learning process, we have no mechanism to choose the \textbf{best} pipeline from many seen pipelines, unlike the dynamic execution case where one can rely on runtime to choose the best performing pipeline.  Observing that pipelines are basically workflow graphs, we use graph generator neural models to succinctly capture the statically-observed pipelines for a single dataset. In {\sysname}, we formulate learner selection as a graph generation problem to predict optimized pipelines based on pipelines seen in actual notebooks.

%. This formulation enables {\sysname} for effective pruning of the AutoML search space to predict optimized pipelines based on pipelines seen in actual notebooks.}
%We note that increasingly, state-of-the-art performance in AutoML systems is being generated by more complex pipelines such as Directed Acyclic Graphs (DAGs) \cite{piper} rather than the linear pipelines used in earlier systems.  
 
{\sysname} does learner and transformation selection, and hence is a component of an AutoML systems. To evaluate this component, we integrated it into two existing AutoML systems, FLAML \cite{flaml} and Auto-Sklearn \cite{autosklearn}.  
% We evaluate each system with and without {\sysname}.  
We chose FLAML because it does not yet have any meta-learning component for the cold start problem and instead allows user selection of learners and transformers. The authors of FLAML explicitly pointed to the fact that FLAML might benefit from a meta-learning component and pointed to it as a possibility for future work. For FLAML, if mining historical pipelines provides an advantage, we should improve its performance. We also picked Auto-Sklearn as it does have a learner selection component based on meta-features, as described earlier~\cite{autosklearn2}. For Auto-Sklearn, we should at least match performance if our static mining of pipelines can match their extensive database. For context, we also compared {\sysname} with the recent VolcanoML~\cite{VolcanoML}, which provides an efficient decomposition and execution strategy for the AutoML search space. In contrast, {\sysname} prunes the search space using our meta-learning model to perform hyperparameter optimization only for the most promising candidates. 

The contributions of this paper are the following:
\begin{itemize}
    \item Section ~\ref{sec:mining} defines a scalable meta-learning approach based on representation learning of mined ML pipeline semantics and datasets for over 100 datasets and ~11K Python scripts.  
    \newline
    \item Sections~\ref{sec:kgpipGen} formulates AutoML pipeline generation as a graph generation problem. {\sysname} predicts efficiently an optimized ML pipeline for an unseen dataset based on our meta-learning model.  To the best of our knowledge, {\sysname} is the first approach to formulate  AutoML pipeline generation in such a way.
    \newline
    \item Section~\ref{sec:eval} presents a comprehensive evaluation using a large collection of 121 datasets from major AutoML benchmarks and Kaggle. Our experimental results show that {\sysname} outperforms all existing AutoML systems and achieves state-of-the-art results on the majority of these datasets. {\sysname} significantly improves the performance of both FLAML and Auto-Sklearn in classification and regression tasks. We also outperformed AL in 75 out of 77 datasets and VolcanoML in 75  out of 121 datasets, including 44 datasets used only by VolcanoML~\cite{VolcanoML}.  On average, {\sysname} achieves scores that are statistically better than the means of all other systems. 
\end{itemize}


%This approach does not need to apply cleaning or transformation methods to handle different variances among datasets. Moreover, we do not need to deal with complex analysis, such as dynamic code analysis. Thus, our approach proved to be scalable, as discussed in Sections~\ref{sec:mining}. 

\section{Related Work}

  \begin{figure*}[t]
    \centering
    \includegraphics[width=\linewidth]{figures/pipeline}
    % \vspace{-12pt}
    \vspace{-20pt}
    \caption{The overview of our approach. Given the six RGB stream inputs surrounding the performer and objects, our approach generates high-quality human-object meshes and free-view rendering results. ``DR'' indicates differentiable rendering.
    }
    \vspace{-10pt}
    \label{fig:pipeline}
  \end{figure*}
    
  
\noindent{\textbf{Human Performance Capture.}}
Markerless human performance capture techniques have been widely investigated to achieve human free-viewpoint video or reconstruct the geometry. 
%
The high-end solutions~\cite{motion2fusion,TotalCapture,collet2015high,chen2019tightcap} adopt studio-setup with dense cameras to produce high-quality reconstruction and surface motion, but the synchronized and calibrated multi-camera systems are both difficult to deploy and expensive.
%
The recent low-end approaches~\cite{Xiang_2019_CVPR,LiveCap2019tog,chen2021sportscap, he2021challencap} enable light-weight performance capture under the single-view setup or even hand-held capture setup or drone-based capture setup~\cite{xu2017flycap}.
%
However, these methods require a naked human model or pre-scanned template. 
Volumetric fusion based methods~\cite{newcombe2015CVPR,DoubleFusion,BodyFusion,HybridFusion} enables free-form dynamic reconstruction. But they still suffer from careful and orchestrated motions, especially for a self-scanning process where the performer turns around carefully to obtain complete reconstruction. 
%
\cite{robustfusion} breaks self-scanning constraint by introducing implicit occupancy method.
%
 All these methods suffer from the limited mesh resolution leading to uncanny texturing output. Recent method~\cite{mustafa2020temporally} leverages unsupervised temporally coherent human reconstruction to generate free-viewpoint rendering. It is still hard for this method to get photo-realistic rendering results.
%
Comparably, our approach enables the high-fidelity capture of human-object interactions and eliminates the additional motion constraint under the sparse view RGB camera settings.


\noindent{\textbf{Neural Rendering.}}
The recent progress of differentiable neural rendering brings huge potential for 3D scene modeling and photo-realistic novel view synthesis. Researchers explore various data representations to pursue better performance and characteristics, such as point-clouds~\cite{Wu_2020_CVPR,aliev2019neural,suo2020neural3d}, voxels~\cite{lombardi2019neural}, texture meshes~\cite{thies2019deferred,liu2019neural} or implicit functions~\cite{park2019deepsdf,nerf,meng2021gnerf,chen2021mvsnerf,wang2021mirrornerf,luo2021convolutional}. 
%
However, these methods require inevitable pre-scene training to a new scene.
%
For neural modeling and rendering of dynamic scenes, NHR~\cite{Wu_2020_CVPR} embeds spatial features into sparse dynamic point-clouds, Neural Volumes~\cite{NeuralVolumes} transforms input images into a 3D volume representation by a VAE network.
% 
More recently, \cite{park2020deformable,pumarola2020d,li2020neural,xian2020space,tretschk2020non,peng2021neural,zhang2021editable} extend neural radiance field (NeRF)~\cite{nerf} into the dynamic setting. 
%
They learn a spatial mapping from the canonical scene to the current scene at each time step and regress the canonical radiance field. 
% 
However, for all the dynamic approaches above, dense spatial views or full temporal frames are required in training for high fidelity novel view rendering, leading to deployment difficulty and unacceptable training time overhead. Recent approaches~\cite{peng2021neural} and ~\cite{NeuralHumanFVV2021CVPR} adopt a sparse set of camera views to synthesize photo-realistic novel views of a performer. However, in the scenario of human-object interaction, these methods fail to generate both realistic performers and realistic objects.
Comparably, our approach explores the sparse capture setup and fast generates photo-realistic texture of challenging human-object interaction in novel views.

% \myparagraph{\textbf{Human-object capture}}
\noindent{\textbf{Human-object capture.}}
%
Early high-end work~\cite{collet2015high} captures both human and objects by reconstruction and rendering with dense cameras. 
%
Recently, several works explore the relation between human and scene to estimate 3D human pose and locate human position~\cite{hassan2019resolving,HPS,liu20204d}, naturally place human~\cite{PSI2019,PLACE:3DV:2020,hassan2021populating} or predict human motion~\cite{cao2020long}. 
%
Another related direction~\cite{GRAB:2020,hampali2021handsformer,liu2021semi} models the relationship between hand and objects for generation or capture.
%
PHOSA~\cite{2020phosa_Arrangements} runs human-object capture without any scene- or object-level 3D supervision using constraints to resolve ambiguity. 
However, they only recover the naked human bodies and produce a visually reasonable spatial arrangement.
%
A concurrent close work is RobustFusion(journal)~\cite{su2021robustfusion}. They capture human and objects by volumetric fusion respectively, and track object by Iterative Closest Point (ICP). 
However, their texturing quality is limited by mesh resolution and color representation, and the occluded region is ambiguous in 3D space.
%
Comparably, our approach enables photo-realistic novel view synthesis and accurate human object arrangement in 3D world space
under the human-object interaction for the light-weight sparse RGB settings.
%\vspace*{-0.6cm}
\section{Integer Echo State Networks}
\label{sect:intesn}
%\vspace*{-0.3cm}
% 
% \begin{figure}[!ht]%[t!]
% \centering
% \includegraphics[width=0.7\columnwidth]{img/HD_ESN}
% \caption{Architecture of the Integer Echo State Network.}
% \label{fig:intesn}
% %\vspace*{-0.5cm}
% \end{figure}

% \begin{figure}[hbt]
% \minipage{0.49\textwidth}
%   \includegraphics[width=\linewidth]{img/HD_ESN}
%   \caption{Architecture of the Integer Echo State Network.}
% \label{fig:intesn}
% \endminipage\hfill
% \minipage{0.49\textwidth}%
%   \includegraphics[width=\linewidth]{img/Discretization}
%   \caption{Quantization and discretization of a continuous signal.}
% \label{fig:quantization}
% \endminipage
% \end{figure}


\begin{figure}[tb]%[!ht]%[t!]
\centering
\includegraphics[width=1.0\columnwidth]{img/HD_ESN_new}
\caption{Architecture of the proposed integer Echo State Network.}
\label{fig:intesn}
%\vspace*{-0.5cm}
\end{figure}

This section presents the main contribution of the article -- an architecture for integer Echo State Network. 
The architecture is illustrated in Fig.~\ref{fig:intesn}. The proposed intESN is structurally identical to the the conventional ESN (see Fig.~\ref{fig:esn}) with three layers of neurons: input ($\textbf{u}(n)$, $K$ neurons), output ($\textbf{y}(n)$, $L$ neurons), and reservoir ($\textbf{x}(n)$, $N$ neurons). It is important to note from the beginning that training the readout matrix $\textbf{W}^{\text{out}}$  for intESN is the same as for the conventional ESN (Section \ref{sect:training}). 

However, other components of intESN differs from the conventional ESN.  First,
activations of input and output layers are projected into the  reservoir in the
form of bipolar HD vectors \cite{MAP} of size $N$  (denoted as
$\textbf{u}^{\text{HD}}(n)$ and $\textbf{y}^{\text{HD}}(n)$). 
 % you said this in the previous section:
% Such projection is achieved by mapping activations into high-dimensional vectors.  According to the principles of HDC, each ``symbol''  is assigned with a random high-dimensional vector used as representation in HDC.  The correspondences are stored in the so-called item memory (Fig.~\ref{fig:intesn}),  which given the symbol issues the corresponding high-dimensional vector  (denoted as HD in the figure). 
For problems where input and output data are described by finite
alphabets and each symbol can be treated independently,  the mapping to $N$-dimensional space is achieved by simply assigning a random bipolar HD vector
to each symbol in the alphabet and storing them in the item memory \cite{Kanerva09, Kleyko2015}.  In the
case with continuous data (e.g., real numbers), we quantized the continuous values
into a finite alphabet.
%A symbol in such alphabet is a quantization level.  
The quantization scheme (denoted as $Q$) and the granularity of the
quantization are problem dependent. Additionally, when there is a need to
preserve similarity between quantization levels, distance preserving mapping
schemes are applied (see, e.g., \cite{Scalarencoding, Widdows15}), which can
preserve, for example, linear or nonlinear similarity between levels. An example
of a discretization and quantization of a continuous signal as well as its
HD vectors in the item memory is illustrated in Fig.~\ref{fig:intesn}. 
%\hl{
Continuous values can be also represented in HD vectors by varying their density. For a recent overview of several mapping approaches readers are referred to \cite{TNNLS18}. Also, an example of applying such mapping is presented in Section~\ref{sect:perf:analog}.
%}
%Thus after choosing the mapping scheme activations of neurons are projected into bipolar HD vector and this vector is added to reservoir.
Another feature of intESN is the way the recurrence in the reservoir is implemented. Rather than a matrix multiply, recurrence is implemented via the permutation of the reservoir vector. Note that permutation of a vector can be described in matrix form, which can play the role of $\textbf{W}$ in intESN. Note that the spectral radius of this matrix equals one.
However, an efficient implementation of permutation can be achieved for a special case -- cyclic shift (denoted as $\text{Sh}()$). 
%{\color{red}
It is important to note that we have shown in~\cite{Frady17} that the recurrent weight matrix $\textbf{W}$ creates key-value pairs of the input data.
Note that $\textbf{W}$ is chosen randomly and kept fixed, and this always leads to the same properties.
Moreover, there is no advantage of the fully connected random recurrent weight matrix over the simple cyclic shift operation for storing the input history.
Thus, the use of the cyclic shift in place of a random  recurrent weight matrix does not limit intESN's ability to produce linearly separable representations. 
%}
Fig.~\ref{fig:intesn} shows the recurrent connections of neurons in a reservoir with recurrence by cyclic shift of one position. In this case, vector-matrix multiplication $ \textbf{W} \textbf{x}(n) $ is equivalent to $ \text{Sh}(\textbf{x}(n),1)$.   




% In the ESN the reservoir matrix is a random matrix  which posses specific properties as described in the previous section.
%\hl{
Finally, to keep the integer values of neurons, intESN uses different nonlinear activation function for the reservoir -- clipping (\ref{eq:clipping}).
%}
Note that the simplest bundling operation is an elementwise addition. However, when using the elementwise
addition, the activity of a reservoir (i.e., a composite HD vector) is no longer bipolar. From the implementation
point of view, it is practical to keep the values of the elements of the HD vector in the limited range using  a threshold value (denoted as $\kappa$). % and make it a configurable parameter. The bounding operation is called {\it clipping}. The clipping is done as follows:
~
\begin{equation}
f_\kappa (x) = 
\begin{cases}
-\kappa & x \leq -\kappa \\
x & -\kappa < x < \kappa \\
\kappa & x \geq \kappa
\end{cases}
\label{eq:clipping}
\end{equation}
~
The clipping threshold $\kappa$ is regulating nonlinear behavior of the reservoir and limiting the range of activation values. Note that in intESN the reservoir is updated only with integer bipolar vectors, and after clipping the values of neurons are still integers in the range between $-\kappa$ and $\kappa$. Thus, each neuron can be represented using only $\log_2(2\kappa+1)$ bits of memory. For example, when $\kappa=7$, there are fifteen unique values of a neuron, which can be stored with just four bits. 
We have also shown recently that the usage of the clipping might be beneficial when implementing resource-efficient alternatives of Self-Organizing Maps~\cite{intSOM}.


Summarizing the aforementioned differences, the update of intESN is described as: 
~
\begin{equation}
\textbf{x}(n)= f_\kappa (\text{Sh}(\textbf{x}(n-1),1)+\textbf{u}^{\text{HD}}(n)+\textbf{y}^{\text{HD}}(n-1)).
\label{eq:intesnres}
 \end{equation}

















\section{Performance evaluation}
\label{sect:perf}

%\hl{
In this section, the proposed intESN approach is verified and compared to the conventional ESN and the ring-based ESN~\cite{MinESN} on a set of typical RC tasks.
In particular, three aspects are evaluated: short-term memory, classification of time-series, and modeling of dynamic processes.
Short-term memories are compared using the trajectory association task
\cite{PlateBook}, introduced in the area of holographic reduced representations
\cite{PlateTr}. Additionally, an approach for storing and decoding analog values using intESN is demonstrated on image patches. 
Classification of time-series is studied using the standard datasets from UCI and UCR.
Modeling of dynamic processes is tested on two typical cases. First, the task of learning a simple sinusoidal function is considered. Next, networks are trained to reproduce a complex dynamical system produced by a Mackey-Glass series.
Unless otherwise stated, ridge regression (the regularization coefficient is denoted as $\lambda$) with the  Moore-Penrose pseudo-inverse was used to learn the
readout matrix $\textbf{W}^{\text{out}}$. Values of input neurons $\textbf{u}(n)$ were not used for training the readout in any of the experiments below. 
%}
%Note about studies of capacity for both RC and HPC
%Moore-Penrose pseudo-inverse of matrix
















\subsection{Short-term memory}
\subsubsection{Sequence recall task}
\label{sect:perf:trajectory}






%While we leave formal analysis of the capacity for future work 

The sequence recall task includes  two stages: memorization and recall.
At the memorization stage,  a network continuously stores a sequence of tokens 
(e.g., letters, phonemes, etc).
The number of unique tokens is denoted as $D$ ($D=27$ in the experiments), and one token is presented as input each timestep.
At the recall stage, the network uses the content of its reservoir to
retrieve the token stored $d$ steps ago, where $d$ denotes delay. In the
experiments, the range of delay varied between 0 and 15.



\begin{figure}[tb]%[!ht]%[t!]
\centering
\includegraphics[width=1.0\linewidth]{img/short_memory_ort}
\caption{The accuracy of the  correct decoding of tokens for the conventional ESN, ring-based ESN, and integer ESN for
three different values of $N$.
}
\label{fig:memory}
%\vspace*{-0.5cm}
\end{figure}




For the conventional and ring-based ESNs, the  dictionary of tokens was represented by a one-hot encoding, i.e.
the number of input layer neurons was set to the  size of the dictionary
$K=D=27$. The same encoding scheme was adopted for the output layer, $L=27$.
%\hl{
The input vector  was projected to the reservoir by the projection matrix
$\textbf{W}^{in}$ where each entry was independently generated from the uniform distribution in the
range $[-1,1]$, the projection gain was set to $\beta=0.1$.  
The reservoir connection matrix $\textbf{W}$ for the conventional ESN  was first generated from the standard normal distribution and then orthogonalized. 
The reservoir connection matrix $\textbf{W}$ for the ring-based ESN  was generated as a permutation matrix. 
The feedback strength of both reservoir connection matrices was set to $\rho=0.94$. 
%}


For intESN, the item memory was populated with $D$ random high-dimensional
bipolar vectors.  The threshold for the clipping function was set to $\kappa=3$. The
output layer was the same as in ESN with $L=27$ and one-hot encoding of tokens.
%{\color{brown}
It is worth noting that $\rho$ and $\kappa$ were chosen in such a way that the accuracy curves would resemble each other as close as possible. 
The diligent readers are kindly referred to the Supplementary materials (Fig. S.1) where for the case $N=200$ the curves for the range of $\rho$ and $\kappa$ values are presented. 
%}



For each value of of the delay $d$ a readout matrix $\textbf{W}^{\text{out}}$  was trained, producing 16 matrices in total. 
The training sequence presented 2000 random tokens to the network, and only the last 1500 steps were used to compute the readout matrices. The regularization parameter for ridge regression was set to $\lambda=0$.
The training sequence of tokens delayed by the particular $d$ was used as the ground truth for the the activations of the output layer.  
During the operating phase,  both the inclusion of a new token into the reservoir and the recall of the delayed token from the reservoir 
were simultaneous.  Experiments were performed for three different sizes of the reservoir: $N=100$, $N=200$, and $N=300$.


\begin{figure}[tb]%[!ht]%[t!]
\centering
\includegraphics[width=1.0\linewidth]{img/short_memory_ort_lonf_ESN}
\caption{The accuracy of the  correct decoding of tokens for the conventional ESN, ring-based ESN, and integer ESN for
three different values of $N$. ``intESN-large'' refers to the fact that the number of neurons in intESN was equivalent to the memory footprint required by ESN for the stated number of neurons.  
}
\label{fig:memory:long}
%\vspace*{-0.5cm}
\end{figure}


The memory capacity of the network is characterized by the 
accuracy of the correct decoding of tokens for different values of the delay. 
%{\color{brown}
Fig.~\ref{fig:memory} depicts the accuracy for all networks conventional ESN (solid lines), ring-based ESN (dash-dotted line)
and intESN (dashed lines). The capacities of all the networks grow with the
increased number of neurons in the reservoir.  
Since the capacities of the conventional ESN and the ring-based ESN are almost identical, which is in line with~\cite{Strauss2012}, for the rest of this subsection we assume both of these networks when using the term ESN.  
%}


\begin{figure*}[tb]%[h]
\center{
\begin{minipage}[h]{0.8\linewidth}
\center{\includegraphics[width=1.0\linewidth]{img/intESN_images_or} }
\end{minipage}
\vfill
\begin{minipage}[h]{0.8\linewidth}
\center{\includegraphics[width=1.0\linewidth]{img/intESN_images_rec}}
\end{minipage}
\caption{
%\hl{
An example of image patches decoded from an intESN. Top row represents the original images stored in the reservoir.   
Other rows depict the patches reconstructed from intESN for different reservoir sizes and clipping thresholds.
%}
}
}
\label{fig:images}
\end{figure*}


%\hl{
The capacities of ESN and intESN  are comparable for small $d$, i.e., for the most recent tokens. 
For the increased delays the curves featured slightly different behaviors. 
With increase of the value of $d$ the performance of intESN started to decline faster compared to ESN.
%For all values of $N$ the accuracy of  intESN starts to deviate from 100\% earlier than that of  ESN. Also, intESN features slightly steeper decay than ESN. 
Eventually, all curves converge to the value of the random guess which equals $1/D$. 
%Moreover, it is possible to characterize the information capacity using a single number -- the amount of information which can be decoded from the reservoir. 
Moreover, the information capacity of a network is characterized by the amount of the decoded from the reservoir information.
This amount is determined using the amount of information per token ($\log_2D$), the probability of correctly decoding a token at each delay value, and the concept of mutual information. We calculated the amount of information for all networks in Fig.~\ref{fig:memory} in the considered delay range. For 100 neurons intESN preserved 19.3\% less information, for 200 and 300 neurons 21.7\% less.
%}




%\hl{
%On the other hand, intESN with $\kappa=3$ requires only 3-bit per neuron. It is assumed that one ESN neuron requires 32-bit then if the number of neurons in %intESN is increased ten times the reservoir memory footprints of two networks are going to be comparable. The results for this case are presented in Fig.~\ref{fig:memory:long} (the training sequence was prolonged to 9000 random tokens). In such setting intESN has clearly higher information capacity. In particular, %for ESN memory footprint with 100 neurons the decoded amount of information has increased 2.2 times while for 200 and 300 neurons it increased 1.6 and 1.3 times %respectively. 
%}











%\hl{
These results highlight a very important trade-off: the performance versus a complexity of implementation. While
the performance of intESN is somewhat poorer in this task, one has to bear in mind its memory footprint. With the
clipping threshold $\kappa=3$ only 3-bit are needed to represent the state of a neuron compared to 32-bit
per neuron (the size of type float) in ESN. 
%{\color{red}
In other words, intESN allowed lowering the memory footprint of the reservoir by an order of magnitude by sacrificing only a fraction of the performance with respect to the information capacity. 
Thus, we conjecture that some reduction in the performance for ten folds
memory footprint reduction is an acceptable price in applications on resource-constrained computing
devices.
%}
On the other hand, we can check the performance of the networks with equal memory
footprints. 
For this we increased the number of neurons in intESN so that the total memory consumed by
the reservoir with the same clipping threshold $\kappa=3$ would match that of the conventional or ring-based ESN. 
This network is denoted as ``intESN-large''.
%{\color{red}
Since $\kappa=3$ requires only 3-bit, in order to get the memory footprint corresponding to ESN, intESN could use more than ten times more neurons. 
Thus, the memory footprint of intESN with $1000$ neurons corresponds to ESN with $100$ neurons; while intESN with $2000$ and $3000$  neurons correspond to ESN with $200$ and $300$ neurons respectively.
%}
 The results for this case are presented in Fig.~\ref{fig:memory:long} (the training sequence was prolonged to
9000 random tokens). With such settings intESN-large has clearly higher information capacity. In
particular, for ESN memory footprint with 100 neurons the decoded amount of information has increased
2.2 times while for 200 and 300 neurons it increased 1.6 and 1.3 times respectively.
%}
%{\color{brown}
It is important to note, however, that while the memory consumed by the reservoir of intESN-large was comparable to the corresponding ESN, the readout matrix for intESN-large was larger and more computationally demanding than the ESN readout matrix since the size of a readout matrix is proportional to the number of neurons in the reservoir. 
%}





%ESN: 100 - 36.9;  200 - 58.6; 300 - 70.5; 
%intESN: 100 - 29.8;  200 - 45.9; 300 - 55.2; 
%Long training.
%ESN: 100 - 46.9;  200 - 74.9; 300 - 95.43; 
%intESN: 100 - 103.3;  200 - 117.4; 300 - 122.8; 

\subsubsection{Storage of analog values in intESN}
\label{sect:perf:analog}







%\hl{
This subsection presents the feasibility of storing analog values in intESN using image patches as a showcase. 
%{\color{red}
It is important to emphasize that this subsection  does not go into detailed comparisons with other methods as the main purpose here is 
the principal demonstration of the possibilities of storing continuous data in reservoirs consisting of integers in a limited range. 
In other words, with this showcase, we are aiming at demonstrating the feasibility of using integer approximation of neuron states in intESN to work with analog representations.
%}
A value of a pixel (in an RGB channel) can be treated as an analog value in the range between 0 and 1. 
For each pixel it is possible to generate a unique bipolar HD vector. 
%The pixel's value is encoded by multiplying all elements of the HD vector by that value. 
The typical approach to encode an analog value is to multiply all elements of the HD vector by that value.
The whole sequence is then represented using the bundling operation on all scaled HD vectors. The result of bundling can be used as an input to a reservoir. However, the resultant composite HD vector will not be in the integer range anymore. 
%This could be addressed in using scaling via sparsity. 
We address this problem by using sparsity.
Instead of scaling elements of an HD vector, we propose to randomly set the fraction of elements of the HD vector to zeros, i.e., the HD vector will become ternary. The proportion of zero elements is determined by the pixel's analog value. Pixels with values close to zero will have very sparse HD vectors while pixels with values close to one will have dense HD vectors, but all entries will always be
[-1, 0, or +1].
The result of bundling of such HD vectors (i.e., HD vector for an image) will still have integer values. Such representational scheme allows keeping integer values in the reservoir but it still can effectively store analog values.
%}




The examples of results are presented in Fig.~\ref{fig:images}. Top row depicts original images stored in the reservoir. The other rows depict images reconstructed from the reservoir. The following parameters of intESN were used (top to bottom): $N=64000$, $\kappa=11$; $N=32000$, $\kappa=8$;  $N=16000$, $\kappa=6$;  $N=8000$, $\kappa=4$. The values of $\kappa$ were optimized for a particular $N$. Columns correspond to the delay values (i.e., how many steps ago an image was stored in the reservoir) as in the previous experiment. 
As one would anticipate, the quality of the reconstructed images is improving for larger reservoir sizes. At the same time, the quality of the reconstructed images is deteriorating for larger delay values, i.e., the worst quality of the reconstructed image could be observed in the bottom right corner while the best reconstruction is located in the top left corner.
Nevertheless, the main observation for this experiment is that it is possible to project analog values into the reservoir with integer values using the mapping via varying sparsity and then retrieve the values from the reservoir.
Moreover, we have shown recently~\cite{intRVFL} that the mapping via varying sparsity could even be helpful when solving classification problems with a feed-forward variant of the ESN.

\begin{table}[tb]%[ht]
\renewcommand{\arraystretch}{1.3}
\caption{Details of datasets for time-series classification.\label{tab:datasets}
\vspace{-2mm}}
    % {\scriptsize
    \begin{center}
    \begin{tabular}{|c|c|c|c|c|}\hline
      	\multicolumn{5}{|c|}{\textbf{Univariate datasets from UCR}} \\ \hline\hline
        \textbf{Name} & \textit{\textbf{\#V}} & \textbf{Train} & \textbf{Test} & \textit{\textbf{\#C}} \\ \hline
        Swedish Leaf 	& 1	& 500 	& 625 	& 15 \\ \hline  
 	Distal Phalanx & 1	& 139 & 400 &	3 \\ \hline
	ECG 		& 1 	& 100 & 100 &	2 \\ \hline	
        Wafer  		& 1 	& 1000	 & 6164 &	2\\ \hline\hline       
        \multicolumn{5}{|c|}{\textbf{Multivariate datasets from UCI}} \\ \hline\hline
	 Character Trajectories & 3 		& 300 & 2558 & 20  \\\hline
	 Spoken Arabic Digit 	 & 13 	& 6600 & 2200 & 10  \\\hline
        Japanese Vowels 	& 12 	& 270 & 370 & 9 \\ \hline    	
    \end{tabular}
    \end{center}
%  }
%\vspace{-5mm}
\end{table}


\subsection{Classification of time-series}

%\hl{
In this section, ESN (conventional and ring-based) and intESN networks are compared in terms of classification accuracy obtained on standard time-series datasets. 
Following \cite{Bianchi2018} we used several (four) univariate datasets from UCR\footnote{UCR. Time Series Classification Archive [online], 2018. -- Available online: \url{https://www.cs.ucr.edu/\%7Eeamonn/time\_series\_data\_2018/}.}  and several (three) multivariate datasets from UCI\footnote{UCI. Machine Learning Repository [online], 2019. -- Available online: \url{http://archive.ics.uci.edu/ml/datasets.html}.}.
Details of datasets are presented in Table~\ref{tab:datasets}. 
For each dataset, the table includes the name, number of variables (\#\textit{V}), number of classes (\#\textit{C}), and the number of examples in training and testing datasets.  
%}






%\hl{
Configurations of the networks were kept fixed for all datasets. 
%{\color{brown}
In fact, the configuration of the conventional and ring-based ESNs were set in accordance to~\cite{Bianchi2018}: 
reservoir size was set to $N=800$, projection gain was set to $\beta=0.25$, the feedback strength was set to $\rho=0.99$. 
The regularization parameter for the ridge regression was set to $\lambda=1.0$.
The intESN was also trained with the same $\lambda$.
%}
The clipping threshold for the intESN  was set to $\kappa=7$. 
%{\color{red}
Also for intESN the quantized values of time-series were mapped to bipolar vectors using scatter codes \cite{TNNLS18, scatter}. 
The input signal \textbf{u}(n) was quantized as:
~
\begin{equation}
\textbf{u}(n)_q= \lfloor 200\textbf{u}(n) \rceil /200,
\label{eq:quan:timeser}
 \end{equation}
~
where $\lfloor * \rceil$ denotes rounding to the the closest integer. 
%}
Two sizes of intESN's reservoir were used. The first size corresponded to the size of the conventional and ring-based ESNs, i.e., $N=800$. The second size (``intESN-large'') corresponded to the same memory footprint\footnote{Except for the Japanese Vowels dataset where such reservoir size seemed to significantly overfit the training data. In that case, the number of neurons was increased twice.} required for ESN reservoir assuming that one ESN neuron requires 32-bit while one intESN neuron requires 4-bit (when $\kappa=7$). 
%{\color{red}
Thus, ``intESN-large'' had  $N=6400$ neurons. 
%}
%}




%\hl{
The output layers of networks were representing one-hot encodings of classes in a dataset, i.e., for the particular dataset $L=$\#\textit{C} of that dataset. 
The readout layers of all networks were trained using time-series from a training dataset in the so-called endpoints mode \cite{ComparisonReadOut} when only final temporal reservoir states for each time-series  are used for training a single readout matrix. 
%}


\begin{figure}[tb]%[!ht]%[t!]
\centering
\includegraphics[width=1.0\linewidth]{img/univariate}
\caption{The classification accuracy for univariate datasets from UCR. Bars depict mean values, lines depict standard deviations.  Bars denoted as ``ESN'' and ``intESN'' had the same number of neurons in their reservoirs while for ``intESN-large'' the number of neurons corresponded to ESN's memory footprint. 
}
\label{fig:univar}
\end{figure}




\begin{figure}[tb]%[!ht]%[t!]
\centering
\includegraphics[width=1.0\linewidth]{img/multivariate}
\caption{The classification accuracy for multivariate datasets from UCI. Bars depict mean values, lines depict standard deviations. Bars denoted as ``ESN'' and ``intESN'' had the same number of neurons in their reservoirs while for ``intESN-large'' the number of neurons corresponded to ESN's memory footprint. 
}
\label{fig:multivar}
\end{figure}




%\hl{
The experimental accuracies obtained from the networks for the considered datasets are presented in Fig.~\ref{fig:univar} and ~\ref{fig:multivar}. Fig.~\ref{fig:univar} presents the results for univariate datasets while Fig.~\ref{fig:multivar} presents the results for multivariate datasets. The figures depict mean and standard deviation values across ten independent random initializations of the networks.
%{\color{brown}
Similar to subsection~\ref{sect:perf:trajectory}, the accuracy of the conventional ESN and the ring-based ESN are almost identical, thus, for the rest of this subsection we assume both of these networks when using the term ESN.  
%}


%}

%\hl{
The obtained results strongly depend on the characteristics of the data. However, it was generally observed that intESN with the memory footprint equivalent to ESN demonstrated higher classification accuracy. 
On the other hand, the classification accuracy of intESN with the same number of neurons as in ESN was similar to ESN's performance for all considered datasets but two (``Swedish Leaf'' and ``Character Trajectories'') for which the accuracy degradation was sensible.
We, therefore, conjecture that in a general case, one cannot guarantee the same classification accuracy as for
ESN. The empirical evidence, however, shows that it is not infeasible. Since placing the reported results into
the general context of time-series classifications is outside the scope of this article we do not further
elaborate on fine-tuning of hyperparameters of intESN for the best classification performance.
%On the other hand, classification accuracy of intESN with the same number of neurons as in ESN was similar to ESN's performance for most of the datasets but it %was significantly lower for two datasets: ``Swedish Leaf'' and ``Character Trajectories''. Therefore, in a general case, one cannot guarantee the same accuracy %but the empirical evidence shows that it is not unlikely. 
%Finally, it should be noted that we did not aim at placing the reported results into the general context of time-series classifications. Instead, the goal was to compare the results of intESN and ESN.
%}
%{\color{brown}
However, the interested readers are kindly referred to the Supplementary materials (Fig. S.2 and S.3) where several different values of $N$, $\kappa$, and $\rho$ were examined for each dataset. 
%}







\subsection{Modeling of dynamic processes}
\subsubsection{Learning Sinusoidal Function}


\begin{figure}[tb]%[h]%[htb]
\centering
\includegraphics[width=1.0\linewidth]{img/Sinus.png}
\caption{Generation of a sinusoidal signal.}
\label{fig:Sinus}
\end{figure}

The task of learning a sinusoidal function \cite{ESN02} is
an example of a learning  simple dynamic system with the constant cyclic behavior. The 
ground truth signal was generated as follows:
~
\begin{equation}
y(n)=0.5\sin(n/4).
\label{eq:sin}
 \end{equation}
~
In this task, the input layer was not used, i.e. $K=0$ but the network
projected the activations  of the output layer back to the reservoir using $\textbf{W}^{\text{back}}$. The output
layer had only one neuron ($L=1$).  The reservoir size was fixed to $N=1000$ neurons.
The length of the training  sequence was 3000 (first 1000 steps were discarded
from the calculation). For ESN, the feedback strength for the reservoir connection matrix was set to $\rho=0.8$,
for both networks  $\lambda$ was set to 0.
A continuous value of the ground truth signal was fed-in to ESN during the training. 

For  intESN, in order to map the input signal to a bipolar vector the quantization was used. The signal was quantized as:
~
\begin{equation}
y(n)_q=\lfloor100y(n)\rceil/100.
\label{eq:quan}
 \end{equation}
~
The item memory for the projection of the output layer was populated with
bipolar vectors preserving linear  (in terms of dot product) similarity between
quantization levels \cite{Widdows15}.  The threshold for the clipping function
was set to $\kappa=3$.
 

\begin{figure*}[tb]%[htb]
\centering
\includegraphics[width=1.0\linewidth]{img/Mackey-Glass.png}
\caption{Prediction of the Mackey-Glass series.}
\label{fig:Mackey-Glass}
\end{figure*}


In the operating phase,  the network acted as the generator of the signal
feeding its previous prediction (at time $n-1$) back to the reservoir. 
Fig.~\ref{fig:Sinus} demonstrates  the behavior of intESN during the first 100
prediction steps.  The ground truth is depicted by dashed line while the
prediction of intESN  is  illustrated by the shaded area between 10\% and 90\%
percentiles (100 simulations were performed).  The figure does not show the
performance of the conventional ESN as it just followed the ground truth without
visible deviations. intESN clearly follows the values of the ground truth
but the deviation  from the ground truth is increasing with the number of
prediction steps.  It is unavoidable for the increasing prediction horizon but,
in this scenario,  it is additionally accelerated due to the presence of the
quantization error at each prediction step. 
%{\color{brown}
It is worth noting, however, that the quality of predictions for this task could be improved by increasing the value of $\kappa=3$, i.e., at the cost of extract memory allocated for each neuron. 
The diligent readers are kindly referred to the Supplementary materials (Fig. S.4) where several different values of $\kappa$ were examined. 
%}
The next subsection will clearly demonstrate effects caused by the quantization process. 
The error is accumulated because every time when feeding the prediction back to the reservoir of intESN it should be quantized in order to fetch a vector from the item memory. 




\subsubsection{Mackey-Glass series prediction}

A Mackey-Glass series is generated by the nonlinear time delay differential equation. 
It is commonly used to assess the predictive power of an RC approach.  
In this scenario, we followed  the preprocessing of data and the parameters of
ESN described in \cite{ESN04}.
The parameters of intESN  (including quantization scheme) were the same as in
the subsection above.  
%{\color{brown}
The interested readers are kindly referred to the Supplementary materials (Fig. S.5) where several different values of $N$ and $\kappa$ were examined. 
%}
The length of the training sequence was 3000 (first 1000
steps were discarded from the calculation).
Fig.~\ref{fig:Mackey-Glass} depicts  results for the first 300 prediction
steps. The results were calculated from 100 simulation runs.  The figure
includes four panels. Each panel depicts the  ground truth, the mean value of
predictions as well  as areas marking percentiles between 10\% and 90\%. The
lower right  corner corresponds to intESN while three other panels show
performance of ESN in three different cases related to the quantization of
the data.

In these scenarios ESN was trained to learn the model from the  quantized
data in order to see to which extent  it affects the network.
The upper left  corner corresponds to ESN without data quantization. In this
case, the predictions precisely follow the ground truth.  The upper right corner
corresponds to ESN trained on the quantized data but  with no quantization
during the operational phase. In such settings, the network  closely follows the
ground truth for the first 150 steps but then it often explodes.  The lower left
corner corresponds to ESN where the data was quantized during  both training
and prediction. In this scenario, the network was able produce to  produce good
prediction just for the first few dozens of steps and then entered the  chaotic
mode where even the mean value does not reflect the ground truth.  These cases
demonstrate how the quantization error could affect the predictions  especially
when it is added at each prediction step.
Note that intESN operated in  the same mode as the third ESN. Despite this
fact, its performance rather resembles that of  the second ESN where the speed
deviation of the ground truth is faster.  At the same time, the deviation of
intESN grows smoothly without a sudden explosion  in contrast to ESN.

%All examples demonstrate decent approximation of the performance achieved by the conventional ESN.




\input{tex/fpga}
To put our results in a broader context, let us briefly discuss alternative methods for 
classical simulation of quantum circuits. Vector-based simulators~\cite{de2007massively,smelyanskiy2016qhipster,haner20170}
represent $n$-qubit quantum states by complex vectors of size $2^n$
stored in a classical memory.
The state vector is updated
upon application of each gate by performing sparse
matrix-vector multiplication. The memory footprint limits 
the method to small number of qubits. For example, 
H{\"a}ner and Steiger~\cite{haner20170}
reported a simulation of
quantum circuits with $n=45$ qubits and a few hundred gates 
using a  supercomputer with $0.5$ petabytes of memory.
In certain special cases 
the memory footprint can be reduced 
by recasting the simulation problem as a 
tensor network contraction~\cite{markov2008simulating,boixo2017simulation,aaronson2016complexity}.
Several tensor-based simulators have been developed~\cite{pednault2017breaking,li2018quantum,chen2018classical} 
for geometrically local  shallow quantum  circuits that include only nearest-neighbor
gates on a 2D grid of qubits~\cite{boixo2018characterizing}.
These methods enabled simulations of systems with more than $100$ qubits~\cite{chen2018classical}.
However, it is expected~\cite{alibaba} that for general (geometrically non-local) circuits 
of size $poly(n)$  the runtime of tensor-based simulators scales as $2^{n-o(n)}$.

In contrast, Clifford simulators described in the present paper are applicable to large-scale circuits
without any locality properties as long as the circuit is dominated by Clifford gates. 
This regime may be important for verification of first fault-tolerant quantum circuits
where  logical non-Clifford gates are expected to be scarce due to their high implementation
cost~\cite{fowler2013surface,jones2013low}.
Another advantage of Clifford simulators is their ability to sample the output
distribution of the circuit (as opposed to computing individual output amplitudes).
This is more close to what one would expect from the actual quantum computer. 
For example, a single run of the heuristic sum-over-Cliffords simulator
described in Section~\ref{heuristic} produces thousands of samples from the (approximate) output distribution. 
In contrast, a single run of a tensor-based simulator typically computes a single amplitude of the
output state.  Thus we believe that our techniques 
extend the reach of classical simulation algorithms complementing
the existing vector- or tensor-based simulators.

%PC:
A version of the sum-over-Cliffords simulator using the Metropolis sampling method is also publicly available
as part of \texttt{Qiskit-Aer}, the classical simulation framework of IBM's quantum
programming suite \texttt{Qiskit}~\cite{Qiskit}. This enables classical simulation and verification of quantum
circuits built in Qiskit on system sizes above $30$ qubits, which quickly become inaccessible with the
default vector-based method. This version also supports parallel processing over the stabilizer state decomposition,
which improves the performance of the Metropolis step.

%SBB:
Let us briefly comment on how simulators based on the stabilizer rank compare
with quasi-probability  methods~\cite{pashayan15,Delfosse15rebits,kocia2017discrete}.
The latter use a discrete Wigner function representation of quantum states
and Monte Carlo sampling
to approximate a given output probability of the target circuit with a small
additive error. Negativity of the Wigner function is an important parameter
that quantifies severity of the ``sign problem" associated with the Monte Carlo sampling.
The negativity also controls the runtime of quasi-probability methods. 
For example, the simulator proposed in~\cite{pashayan15}
has runtime $\epsilon^{-2} M^2$, where $M$ is the negativity and $\epsilon$
is the desired approximation error. In contrast to stabilizer rank simulators,
quasi-probability methods do not directly apply to stabilizer
operations on qubits since the latter are not known to have a non-negative Wigner function 
representation~\cite{Delfosse15rebits,karanjai2018contextuality}.
Furthermore, such methods are not well-suited for sampling the output
distribution since this task requires a small {\em multiplicative} error in 
approximating individual output probabilities. 

Our work leaves several  open questions. 
Since the efficiency of Clifford simulators hinges on the ability to find low-rank
stabilizer decompositions of multi-qubit magic states, 
improved techniques for finding such decompositions are of great interest. 
For example, consider a magic state $|\psi\ra=U|+\ra^{\otimes n}$, where 
$U$ is a diagonal circuit composed of $Z,CZ$, and $CCZ$ gates.
We anticipate that a low-rank exact stabilizer decomposition of $\psi$ can be 
found by computing the {\em transversal number}~\cite{alon1990transversal} of a suitable hypergraph describing
the placement of CCZ gates. Such low-rank decompositions may lead to more efficient
simulation algorithms for Clifford+CCZ circuits. We leave as an open question whether
the stabilizer extent $\xi(\psi)$ is multiplicative under tensor products for general states $\psi$.
Finally, it is of great interest to derive lower bounds on the stabilizer rank
of $n$-qubit magic states scaling exponentially with $n$. 






\section{Experimental Results And Discussion}
\label{sec:results}
The results presented in this section test the performance of the Autoencoder model. We evaluate our model using the performance metrics: accuracy, precision, recall, and F1 score, defined as follow: 

\vspace{-5mm}
\begin{align*}
    Accuracy &= \frac{TP+TN}{TP+TN+FP+FN}
\end{align*}
\vspace{-3mm}
\begin{align*}
    Precision &= \frac{TP}{TP+FP}
\end{align*}
\vspace{-3mm}
\begin{align*}
    Recall &= \frac{TP}{TP+FN}
\end{align*}
\vspace{-3mm}
\begin{align*}
    F1 ~Score &= 2 \times \frac{Precision \times Recall}{Precision + Recall}
\end{align*}

In our experiments, a \textit{positive} outcome means an abnormal activity was detected, whereas a negative outcome means a normal activity was detected.
True Positive (TP) refers to an abnormal activity that was correctly classified as abnormal. 
True Negative (TN) refers to a normal activity that was correctly classified as normal.
False Positive (FP) refers to a normal activity that was misclassified as abnormal.
and False Negative (FN) refers to an abnormal activity that was misclassified as normal.

The success of our model is based on measuring the reconstruction error that is produced by any given data point. Figure \ref{fig:recon} shows an example of reconstructed data overlaid the original data that was inserted into the model. %To be clear, the values shown in this graph are not measurements of the reconstruction loss that are shown in Figure \ref{fig:thresh}. This figure only shows the normalized temperature measurement from each data point, that is why they are not being represented in degrees.
In this figure, extremely severe dips in temperature denoted by the blue line (representing our original data) can be noticed. The data reconstructed by the model, represented by the red line, does not dip as much as the original data. This is because our model was not able to reconstruct these points accurately due to the fact that they are anomalies. The reconstruction loss (i.e. different between the original and the reconstructed data), where the model recognizes normal or abnormal behavior, is shown in Figure \ref{fig:thresh}. The figure shows a visualization of the mean-squared-error (MSE) generated by the model after it was given each data point within the test data set. The dotted red line denotes the threshold determined as mentioned in Section \ref{sec:ml-model}. Each data point's actual label is represented either by blue color to denote a normal behavior or red color to denote an anomaly and every data point that lies above the threshold was classified as anomalous. This figure illustrates our model's capability to detect the majority of anomalies by measuring the MSE produced by each data point.

Overall, as shown in Figure \ref{fig:aeresults}, our model was able to attain high performance with over $90\%$ in all metrics. The precision is lower than the recall metric which shows that the model produced slightly more false positives than false negatives. In a smart farming environment, a higher rate of false positives would not have a dramatic affect on the productivity of day to day operations and would ensure a higher number of anomalous situations are detected. A rather problematic situation would be if there were more false negatives than positives. A user would much prefer receiving an alert when nothing was wrong than not receiving an alert and enabling potential harm to occur to the crops and hardware. In the future, we hope to further decrease the number of false positives and negatives in order to fine-tune an overall more accurate model. This can be done by using more training samples.


% \begin{table}[!t]
%     \caption{Results}
%     \centering
%     \begin{tabular}{| c | c | c | c |}
%     \hline
%     Accuracy & Precision & Recall & F1\\ [0.5ex] % inserts table %heading
%     \hline
    
%     98.98\% & 90\% & 92.95\% & 91.45\% \\
    
%     \hline
%     \end{tabular}
%     \label{table:results}
% \end{table}

\begin{figure}[t!]
    \centering
    \includegraphics[width=8cm]{figures/aeresults-v2.png}
    \caption{Performance metrics for Autoencoder Model}
    \label{fig:aeresults}
\end{figure}


\section{Conclusion and Future Work}
\label{sec:conclusion}
Our approach has shown that smart farming anomaly detection can be done at an extremely accurate level by using an Autoencoder. Our approach would allow vast scalability by only requiring non-anomalous data for training. Greenhouses provide controlled environments that create consistent conditions for crops and data collection. Environments such as this are a perfect use case for our approach since the performance of an Autoencoder can drastically improve when provided with large amounts of non-anomalous data. Our approach shows that it may not be entirely necessary for machine learning professionals that are working on anomaly detection within smart farming to be highly concerned with developing models that are trained using labeled data that contains both normal and anomalous data. 

In the future, we will explore more anomaly detection models in order to optimize the system's performance. Once the best model has been selected, the architecture could be brought online to be used and tested with the added interactions of Internet connectivity. By bringing the system online we will have the ability to alert users of potential threats or anomalous behavior. These alerts could be coupled with actuators such as fertilization, watering, video monitoring, etc. The introduction of cameras can be ``used to calculate biomass development and fertilization status of crops" \cite{Walter6148}. They can also be used to allow the system-user to monitor their property from afar. We plan to introduce photo and video monitoring as one of our next steps to improve security and broaden our scope.

\section{Acknowledgements}
\label{sec:ack}
We thank TTU Shipley Farms for allowing to use greenhouse, and setup smart farm testbed. Dr. Brian Leckie and his group were instrumental in our system and early stages of data collection. We are thankful to Ms. Deepti Gupta to provide helpful guidance on dealing with time-series, correlated data and gave input on our model selection. This research is partially supported by the NSF Grant 2025682 at TTU.



\bibliography{bica}

\end{document}
