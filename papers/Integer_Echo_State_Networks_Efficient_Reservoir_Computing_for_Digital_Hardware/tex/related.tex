\section{Background and related work}
\label{sect:related}




There are many practical tasks that require the history of inputs to be solved.
In the area of artificial neural networks (ANN), such tasks require working memory. This could be implemented by recurrent connections between neurons of an RNN. 
%Such networks are called recurrent neural networks (RNNs). 
Training RNNs is much harder than that of feed-forward ANNs (FFNNs) due to vanishing gradient problem \cite{Bengio94}.

The challenge of training RNNs was addressed from two approaches. 
One approach eliminates the vanishing gradient problem through neurons with special memory gates \cite{LSTM97}. 
%First, the vanishing gradient problem can be eliminated through neurons with special memory gates, as it is done in Long Short-Term Memory \cite{LSTM97}.
Another approach is to reformulate the training process by learning only connections to
the last readout layer while keeping the other connections fixed.
This approach originally appeared in two similar architectures: Liquid State Machines
\cite{LSM02} and ESNs \cite{ESN03}, now referred to as reservoir computing\cite{RC09}.
%\hl{
%Despite, the simplicity of the readout layer training this process has its own nuances as the trained layer can be unstable. This problem was recently covered in \cite{Kudithipudi2018}. 
%}


It is interesting to note, that similar ideas were conceived in the area of
FFNNs, which can be seen as an RNN without memory, and are known under the name
of Extreme Learning Machines (ELMs) \cite{ELM06}.
ELMs are used to solve various machine learning problems including
classification, clustering, and regression \cite{ELM15}.

%\hl{
%The RC is a  powerful tool for modeling and predicting dynamic systems both living \cite{ESN11NIPS} and technical \cite{ESN04, RCnature11}  systems. 
%The RC is useful for  modeling and predicting dynamic systems. Another important application of the RC is a classification of time-series. 
Important applications of RC are the modeling and predicting of complex dynamic systems. 
%{\color{brown}
Generating and predicting chaotic systems was an important use-case from the beginning~\cite{Jaeger2001}, for example, ESNs were used for chaotic time-series from low-order aberrations caused by turbulence~\cite{Weddell2008}.
A thorough study on emulating chaotic systems was recently presented in~\cite{Antonik2017}.
%}
It was also shown that ESNs can be used for forecasting of Electroencephalography signals and for solving classification problems in the context of Brain-Computer Interfaces \cite{ESNEEG}.
There are different classification strategies and readout methods when performing classification of time-series with ESN. 
In \cite{ComparisonReadOut} three classification strategies and three readout methods were explored under the conditions that testing data is purposefully polluted with noise. 
Interestingly, different readout methods are preferable in different noise conditions. 
Recent work in \cite{Bianchi2018} also studied classification of multivariate time-series with ESN using standard benchmarking datasets. 
The work covered several advanced approaches, which extend the conventional ESN architecture, for generating a representation of a time-series in a reservoir. 
%}

Another recent research area is binary RC with cellular automata (CARC) which started as a interdisciplinary research within three areas: cellular automata, RC, and HDC. CARC was initially explored in \cite{Yilmaz15a}  for projecting binarized features into high-dimensional space.  Further in \cite{ISBI}, it was applied for modality classification of medical images. The usage of CARC for symbolic reasoning is explored in \cite{Yilmaz15b}. The memory characteristics of a reservoir formed by CARC are presented in  \cite{CAHD17}. Work \cite{RCELMCA17} proposed the usage of coupled cellular automata in CARC.    
%\hl{
Examples of recent RC developments also include advanced architectures such as Laplacian ESN~\cite{Han2018}, learning of reservoir's size and topology~\cite{Qiao2017}, new tools for investigating reservoir dynamics~\cite{BianchiTNNLS18} and determining its edge of criticality~\cite{Livi2018}.
%}


%{\color{brown}
The design of ESNs has been an important research area (see, e.g.,~\cite{Ozturk2007, Busing2010, Strauss2012}). 
One important aspect of the design is, of course, the choice of network's parameters for a given task.
Another important aspect considered in this study is the computational complexity. 
One of the ways of reducing computational costs would be to use quantized reservoir states. 
It was explored in Fractal Prediction Machines~\cite{Tino00} and Neural Prediction Machines~\cite{Tino04, Tino07} RC models. 
Another way of reducing computational costs involves modifications of network's connectivity structure. 
In this respect, the approach which is ideologically closest to our intESN, was presented in~\cite{MinESN}. 
The authors demonstrated that a simple cycle reservoir (referred to as the ring-based ESN) can be used to achieve a performance similar to the conventional ESN. 
Similar conclusions about the ring-based ESN were obtained in~\cite{Strauss2012} when studying different design strategies for reservoir connection matrices in four typical RC tasks. 
While the ring-based ESN explored reservoir update solution, which is similar to one of our optimizations, the technical side is very different from our approach as intESN strives at using only integers as neurons activation values.
%}





%{\color{brown}
While in this article the main focus is on reservoir states comprised of integers only, it is worth mentioning related works considering the general problem of reducing the computational complexity of the conventional ESN. 
For example, several optimizations were used in~\cite{ESNAnomaly09} in order to deploy an ESN on a resource-constrained device for anomaly detection.
These optimizations include sparse matrix algebra via compressed row storage for weights of connections between between the input layer neurons and the reservoir; single floating point precision; and an activation function, which resembles $\tanh()$ function but has lower complexity. 
Similarly, in works~\cite{Bacciu2013, Bacciu2014} ESNs were used in the context of user movements prediction on resource-constrained devices, therefore, the authors studied how the parameters of the network will affect computational costs and task performance.
In particular, the varied parameters were sparsity of reservoir connection matrix, number of bits per weight, number of neurons in reservoir.
%}



% Overview of the major theories
 
\subsection{Echo State Networks}
\label{sect:esn}
% 


% \begin{figure}[hbt]
% \minipage{0.49\textwidth}
%   \includegraphics[width=\linewidth]{img/ESN_new}
%   \caption{Architecture of the conventional Echo State Network.}
% \label{fig:esn}
% \endminipage\hfill
% \minipage{0.49\textwidth}%
%   \includegraphics[width=\linewidth]{img/HD_ESN_new}
%   \caption{Architecture of the Integer Echo State Network.}
% \label{fig:intesn}
% \endminipage
% \end{figure}


%\hl{
This subsection summarizes the functionality of the conventional ESN, it follows the description in \cite{ESNtut12} for a special case of leaky integration when $\alpha=1$\footnote{For the detailed tutorial on ESNs diligent readers are referred to \cite{ESNtut12}.}.
%}
Fig.~\ref{fig:esn} depicts the architectural design of the conventional ESN, which includes three layers of neurons. The input layer with $K$ neurons represents the current value of input signal denoted as $\textbf{u}(n)$. The output layer ($L$ neurons) produces the output of the network (denoted as $\textbf{y}(n)$) during the operating phase. The reservoir is the hidden layer of the network with $N$ neurons, with the state of the reservoir at time $n$ denoted as $\textbf{x}(n)$. 

In general, the connectivity of ESN is described by four matrices. $\textbf{W}^{\text{in}}$ describes connections between the input layer neurons and the reservoir, and $\textbf{W}^{\text{back}}$ does the same for the output layer. Both matrices project the current input and output to the reservoir.   
The memory in ESN is due to the recurrent connections between neurons in the reservoir, which are described in the reservoir matrix $\textbf{W}$.  
Finally, the matrix of readout connections $\textbf{W}^{\text{out}}$ transforms the current activity levels in the input layer and reservoir ($\textbf{u}(n)$ and $\textbf{x}(n)$, respectively) into the network's output $\textbf{y}(n)$.  
~
\begin{figure}[tb]%[!ht]%[t!]
\centering
\includegraphics[width=1.0\columnwidth]{img/ESN}
\caption{Architecture of the conventional Echo State Network.}
\label{fig:esn}
%\vspace*{-0.5cm}
\end{figure}
~
Note that three matrices ($\textbf{W}^{\text{in}}$, $\textbf{W}^{\text{back}}$, and $\textbf{W}$) are randomly generated at the network initialization and stay fixed during the network's lifetime. Thus, the training process is focused on learning the readout matrix $\textbf{W}^{\text{out}}$. There are no strict restrictions for the generation of projection matrices   
$\textbf{W}^{\text{in}}$ and $\textbf{W}^{\text{back}}$. They are usually randomly drawn from either normal or uniform distributions and scaled as shown below. The reservoir connection matrix, however, is restricted to posses the echo state property. This property is achieved when the  spectral radius of the matrix  $\textbf{W}$ is less or equal than one. 
%\hl{
For example, $\textbf{W}$ can be generated from a normal distribution and then normalized by its maximal eigenvalue.
Unless otherwise stated, in this article an orthogonal matrix was used as the reservoir connection matrix; such a matrix was formed by applying QR decomposition to a random matrix generated from the standard normal distribution. Also, $\textbf{W}$ can be scaled by a feedback strength parameter, see (\ref{eq:esnres}).
%}

The update of the network's reservoir at time $n$ is described by the following equation:
~
\begin{equation}
\textbf{x}(n)=\tanh(\rho\textbf{W}\textbf{x}(n-1)+\beta\textbf{W}^{\text{in}}\textbf{u}(n)+\beta\textbf{W}^{\text{back}}\textbf{y}(n-1)),
\label{eq:esnres}
 \end{equation}
%\hl{
where  $\beta$ and $\rho$ denote projection gain and the feedback strength, respectively. Note that it is assumed that the spectral radius of the reservoir connection matrix $\textbf{W}$ is one.
%}
Note also that at each time step neurons in the reservoir apply $\tanh()$ as the activation function.
The nonlinearity prevents the network from exploding by restricting the range of possible values from -1 to 1. The activity in the output layer is calculated as:  
~
\begin{equation}
\hat{\textbf{y}}(n)=g(\textbf{W}^{\text{out}}[\textbf{x}(n);\textbf{u}(n)]),
\label{eq:esny}
 \end{equation}
where the semicolon denotes concatenation of two vectors and $g()$ the activation function of the output neurons, for example, linear or  Winner-take-all.  
~
\subsubsection{Training process}
\label{sect:training}

This article only considers training with supervised-learning when the network is provided with the ground truth desired output at each update step. The reservoir states $\textbf{x}(n)$ are collected together with the ground truth $\textbf{y}(n)$ for each training step. The weights of the output layer connections are acquired by solving the regression problem which minimizes the mean square error between predictions (\ref{eq:esny}) and the ground truth.
%For example, $\textbf{W}^{\text{out}}$ can be calculated using the pseudo-inverse matrix operation (adopted in this article). 
While this article does not focus on the readout training task, it should be noted that there are many alternatives reported in the literature including the usage of regression with regularization, online update rules, etc. \cite{ESNtut12}.





\subsection{Fundamentals of hyperdimensional computing}
\label{sect:sparse}

In a localist representation, which is used in all modern digital
computers, a group of bits is needed in its entirety to interpret a representation.  
In HDC, all entities
(objects, phonemes, symbols, items) are represented by vectors of very high
dimensionality -- thousands of bits. The information is spread out in a {\it distributed representation}, which contrary to the
localist representations, any subset of the bits can be interpreted. 
Computing with distributed representations utilizes statistical
properties of vector spaces with very high dimensionality, which allow for approximate, noise-tolerant, highly parallel computations.
%The results are evaluated only in terms of a well-defined similarity metric.
{\it Item memory} (also referred to as {\it clean-up} memory) is needed to recover
composite representations assigned to complex concepts. There are several flavors
of HDC with distributed representations, differentiated by the
random distribution of vector elements, which can be real numbers \cite{PlateTr,
Gallant, MAP, Gallant2016}, complex  numbers \cite{PlateBook}, binary numbers
\cite{Kanerva09, Rachkovskij2001}, or bipolar \cite{Gallant, HD_ICRC16}. 

We rely on the mathematics of HDC with  
{\it bipolar distributed representations} to develop intESN.
%\subsection{Computing with dense bipolar distributed representations} 
Kanerva \cite{Kanerva09} proposed the use of distributed representations comprising $N=10, 000$
{\it binary} elements (referred to as HD vectors). The values of each element of an HD vector are independent
equally probable, hence they are also called dense distributed representations.
Similarity between two binary HD vectors is characterized by  Hamming distance,
which (for two vectors) measures the number of elements in which they differ. 
In very high dimensions  Hamming distances (normalized by the dimensionality $N$)
 between any arbitrary chosen HD vector and all other vectors in the
HD space are concentrated around 0.5. Interested readers are referred to
\cite{Kanerva09} and \cite{KanervaBook} for comprehensive
analysis of probabilistic properties of the high-dimensional
representational space.

The binary HD vectors can be equivalently mapped to
the case of bipolar representations, i.e., where each vector's element is
encoded as  ``-1'' or ``+1''. This definition is sometimes more convenient for
purely computational reasons. 
%Bipolar dense distributed representations ,however, possess a set of distinctive properties. 
The distance metric for the bipolar case is a dot product:
~
\begin{equation}
\text{dist} = \textbf{x}^\top \textbf{y}
\end{equation}
%\begin{equation}
  %   dist=|\textbf{x} \wedge \textbf{y}|_1.
 %\end{equation}
~
%\hl{
Basic symbols in HDC are referred to as  atomic HD vectors. They are generated randomly and independently,  and due to
high dimensionality will be nearly orthogonal with very high probability, i.e., similarity
(dot product) between such HD vectors is approximately 0.
An ordered sequence of symbols can be encoded into a composite HD vector
%, resembling a reservoir in RC, 
using the atomic HD vectors, the permutation (e.g., cyclic shift as a special case of permutation) and bundling operations.
This vector encodes the entire sequence history in the composite HD vector and resembles a neural reservoir. 
%}

%\hl{
Normally in HDC, the recovery of component atomic HD vectors from a composite HD vector is performed by finding the most similar vectors stored in the item memory\footnote{It is not common to do such decoding in RC. Normally, in the scope of RC a readout matrix is learned. In this article, we follow this standard RC approach to extracting information back from a reservoir.}. 
However, as more vectors are bundled together there is more interference noise and the likelihood of recovering the correct atomic HD vector declines. 

Our recent work \cite{Frady17} reveals the impact of interference noise, and shows that different flavors of HDC have universal memory capacity. Thus, the different flavors of HDC can be interchanged without affecting performance. From these insights, we are able to design much more efficient networks for reservoir computing for digital hardware. 
%}


