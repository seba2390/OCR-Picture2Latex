%\vspace*{-0.6cm}
\section{Digital hardware experiments}
\label{sect:fpga}
%\vspace*{-0.3cm}


%{\color{red}

In order to demonstrate the amenability of intESN for digital hardware, we used a field-programmable gate array (FPGA) and implemented three different architectures: software ESN as well as hardware accelerated ESN and intESN. 
All architectures were implemented on a ZedBoard FPGA\footnote{ZedBoard. Hardware User's Guide [online], 2014. -- Available online: \url{http://zedboard.org/sites/default/files/documentations/ZedBoard_HW_UG_v2_2.pdf}.}, which contains a dual core ARM Cortex A9 CPU interfacing with a programmable logic fabric. 
The Xilinx Vivado Design Suite\footnote{Xilinx. Vivado Design Suite User Guide [online], 2018. -- Available online: \url{https://www.xilinx.com/support/documentation/sw_manuals/xilinx2018_1/ug910-vivado-getting-started.pdf}.} and Vivado SDK\footnote{Xilinx. Generating Basic Software Platforms [online], 2018. -- Available online: \url{https://www.xilinx.com/support/documentation/sw_manuals/xilinx2018_3/ug1138-generating-basic-software-platforms.pdf}.} were used to design the hardware architectures and program the FPGA board, respectively.
The efficiency evaluation (e.g., energy consumption) of the architectures was based on the recall stage of the sequence recall task  as described in Section~\ref{sect:perf:trajectory}.




\subsection{Architectures design}
\label{sect:fpga:design}


The software ESN was implemented using only CPU on the FPGA board. 
For the hardware acceleration experiments, we programed the CPU to generate the inputs and feed them into a hardware architecture (either for ESN or intESN), and then retrieve the outputs once the computation is over.
The hardware ESN and intESN architectures were designed using Vivado High Level Synthesis (HLS) using the C programming language, and later on synthesized into hardware Intellectual Property components to be integrated in a larger hardware system comprised of: the Zynq Processing System, AXI Direct Memory Access, AXI Interconnect, ESN/intESN architecture, and various other peripherals used for clocking and resetting mechanisms. 
It is important to note that for resource limitations on the ZedBoard, no pipelining or hardware optimization directives have been used on the HLS designs in order to cope with the resource usage for growing reservoir sizes and remain within the board's capacity. However, we still provide a comparison of a speed-up expected from the pipelining for the two hardware architectures.


\subsection{Evaluation methodology}


For each architecture, the reservoir size was varied between $100$, $200$, and $300$ neurons. 
In the following, we report the number of clock cycles required to accomplish the sequence recall task, the power consumption, and the area utilization (i.e., resources) required for the hardware designs.
The number of clock cycles was measured using a hardware timer incorporated into the designs. The area utilization is reported from the hardware synthesis reports provided by Vivado. 
All three architectures use the same clock frequency of $100$ MHz.
The ZedBoard contains a $10$ m$\Omega$ shunt-resistor in series with the input supply of the whole board, which could be used to obtain the overall power consumption by measuring the voltage across it. 
However, at the desired scale of changes voltage fluctuation and measurement sensitivity made it almost impossible to properly perform a precise comparison.
Therefore, instead, we used the Xilinx Power Estimator (XPE) tool\footnote{Xilinx. Power Estimator User Guide [online], 2018. -- Available online: \url{https://www.xilinx.com/support/documentation/sw_manuals/xilinx2018_3/ug440-xilinx-power-estimator.pdf}.} provided by the Vivado Design Suite to estimate the power consumption of each architecture, which is a standard option \cite{FPGASTDP2019, HACNN2018, XPEconf}.



\subsection{Efficiency evaluation results}

\begin{table}[tb]%[ht]
\renewcommand{\arraystretch}{1.3}
\caption{Area utilization of the hardware architectures.
\label{tab:fpga:resources}
\vspace{-2mm}}
    % {\scriptsize
    \begin{center}
    \begin{tabular}{|c|c|c|c|c|}\hline
        $N$ & \textbf{LUT} & \textbf{FF} & \textbf{BRAM} & \textbf{DSP48} \\ \hline \hline 
      	\multicolumn{5}{|c|}{\textbf{ESN}} \\ \hline\hline
        $100$	& $8317$	& 	$8724$ &	$31.5$ &	$47$ \\ \hline  
 	$200$ & $8368 $	& 	$8770 $ &	$87.5$ &	$47$	 \\ \hline
	$300$ 		& $15950 $	& 	$8920 $ &	$135.5$ &	$47$\\ \hline \hline   	    
        \multicolumn{5}{|c|}{\textbf{intESN}} \\ \hline\hline
	 $100$ &	$4577 $	& 	$5488 $ &	$7.5 $ &	$5 $	  \\\hline
	 $200$ 	 & $4602 $	& 	$5492 $ &	$11.5 $ &	$6 $	  \\\hline
        $300$	& $4671 $	& 	$5497 $ &	$12.5 $ &	$6 $	 \\ \hline    	
    \end{tabular}
    \end{center}
%  }
%\vspace{-5mm}
\end{table}


Table~\ref{tab:fpga:resources} presents the area utilization of the hardware architectures for different sizes of reservoir.
It is clear that the resource utilization of the hardware ESN is always larger than that of hardware intESN. 
This is an empirical manifestation of the facts that intESN: a) requires a lower memory footprint ($\kappa=3$) and b) its machinery uses simpler operations, e.g., the clipping instead of $\tanh()$ and the cyclic shift instead of vector-matrix multiplication. 
It is important to note, that the drastic increase of LUT utilization when the reservoir size of ESN was set to $300$ is due to resource constraints on the FPGA board since the total number of usable BRAM units is 140.
Thus, the increased number of LUTs was used as an alternative way of increasing the memory capacity of the board.



\begin{table}[tb]%[ht]
\renewcommand{\arraystretch}{1.3}
\caption{Number of clock cycles (time) involved in the sequence recall task.
\label{tab:fpga:speed}
\vspace{-2mm}}
    % {\scriptsize
    \begin{center}
    \begin{tabular}{|c|c|c|c|}\hline
        $N$ & \textbf{software ESN} & \textbf{hardware ESN} & \textbf{hardware intESN}  \\ \hline 
        $100$	&  \num{3.0e8}  (3.04 \si{s}) 	& 	\num{5.8e8}  (5.77 \si{s})  &	 \num{1.4e8}  (1.42 \si{s})   \\ \hline  
 	$200$ & \num{1.0e9}  (9.99 \si{s}) 	& 	\num{1.9e9}  (18.74 \si{s}) &	\num{2.8e8}  (2.82 \si{s})  \\ \hline
	$300$ & \num{2.1e9}  (21.16 \si{s}) 	& 	\num{3.9e9}  (38.90 \si{s}) &\num{4.5e8}  (4.46 \si{s})  \\ \hline 	    

    \end{tabular}
    \end{center}
%  }
%\vspace{-5mm}
\end{table}

Table~\ref{tab:fpga:resources} presents the number of clock cycles (time in seconds) necessary to perform the operating phase of the sequence recall task for each architecture.
The number of clock cycles was measured with the help of the hardware timer.
The time was calculated using the known frequency of the clock ($100$ MHz)
As expected, for each architecture the operation time increases with the increased reservoir size. 
However, for any reservoir size, intESN is several times faster than both implementations of ESN (at least $2.1$ against the software ESN when $N=100$). The gain is increasing with the increased reservoir size so that when $N=300$ the hardware intESN is $8.7$ times faster compared to the hardware ESN. 

An interesting remark in Table~\ref{tab:fpga:speed} is that the hardware ESN seems slower than the software ESN. 
It is counterintuitive since the hardware architecture is expected to accelerate the computations. 
In the consider case, this fact is explained by the absence of pipelining (cf. Section~\ref{sect:fpga:design}) what prevents further hardware optimizations.
The pipelining would certainly increase the speed at the price of the increased area utilization, which would make the hardware designs being inconceivable on the target board.
In order to evaluate the effect of the pipelining on the speed of the hardware architectures, we have used the Vivado HLS synthesis report to estimate the number of clock cycles per iteration when $N=300$. 
The results are presented in Table~\ref{tab:fpga: pipelining}. 
It shows that the pipelining significantly decreases the number of clock cycles, which makes the hardware ESN much faster than the software one. 
At the same time, the speed-up obtained for the hardware intESN is still higher than that of the hardware ESN, which makes intESN even more efficient than ESN. 

  


\begin{figure}[tb]%[!ht]%[t!]
\centering
\includegraphics[width=1.0\linewidth]{img/power_efficiency}
\caption{
The overall energy consumption of all three architectures against the reservoir size. 
}
\label{fig:fpga:power}
\end{figure}





\begin{table}[tb]%[ht]
\renewcommand{\arraystretch}{1.3}
\caption{Speed-up comparison with the use of pipelining.
\label{tab:fpga: pipelining}
\vspace{-2mm}}
    % {\scriptsize
    \begin{center}
    \begin{tabular}{|c|c|c|c|c|}\hline
	 & \multicolumn{2}{c|}{\textbf{ESN}} & \multicolumn{2}{c|}{\textbf{intESN}} \\ \hline
        Pipelining & \cmark & \xmark & \cmark & \xmark\\ \hline       	 
       Clock cycles	& $1,297,072$	& 	$51,948$ &	$148,241$ &	$2,856$ \\ \hline  
	Speed-up & \multicolumn{2}{c|}{$24.97$} & \multicolumn{2}{c|}{$51.91$} \\ \hline
    \end{tabular}
    \end{center}
%  }
%\vspace{-5mm}
\end{table}



\begin{table}[tb]%[ht]
\renewcommand{\arraystretch}{1.3}
\caption{XPE overall power consumption (Watts)
\label{tab:fpga:power}
\vspace{-2mm}}
    % {\scriptsize
    \begin{center}
    \begin{tabular}{|c|c|c|c|}\hline
        $N$ & \textbf{software ESN} & \textbf{hardware ESN} & \textbf{hardware intESN}  \\ \hline 
        $100$	& $1.53$ 	& 	$1.73$  &	 $1.59$    \\ \hline  
 	$200$ & $1.53$ 	& 	$1.78$ &	$1.60$  \\ \hline
	$300$ & $1.53$ 	& 	$1.95$ & $1.60$ \\ \hline 	    

    \end{tabular}
    \end{center}
%  }
%\vspace{-5mm}
\end{table}


Table~\ref{tab:fpga:power} presents  the power consumption of each architecture. 
The first observation is that the consumption of the  software ESN remains constant for different $N$.
This is because the hardware itself remains fixed, and the software computations that are performed on the Cortex CPU are the same for each configuration. 
Moreover, the software ESN has the lowest power consumption because it is only comprised of the CPU and the hardware timer, whereas the other two include other hardware components in addition to those. 
However, for the hardware implementation of ESN one could see a noticeable increase for a larger reservoir sizes. There is also an increase for the hardware intESN but it is slower compared to the hardware ESN. 
Moreover, for the fixed reservoir size, intESN's power consumption remains lower than that of ESN.

Fig.~\ref{fig:fpga:power} depicts the overall energy consumption of each architecture for different sizes of reservoir.
The energy consumption was calculated as a product of the overall power consumption reported in Table~\ref{tab:fpga:power} and the operating time reported in Table~\ref{tab:fpga:speed}. 
Fig.~\ref{fig:fpga:power} clearly demonstrates that the energy efficiency of the hardware intESN is much better than that of both the hardware and software ESNs.
Scaling the reservoir size leads to an increase of the energy consumption of all architectures.
However, the slope of the curve for intESN is lower than for ESN's architectures. 
Therefore, the energy saved by the use of intESN drastically increases with increasing size of reservoir (e.g., for $N=300$ it needs $10.6$ times lower energy than for the hardware ESN), which is a strong argument in favor of intESN. 
%}


%\todo[inline, color=olive]{
%DK: Redraw Fig.~\ref{fig:fpga:power} to get better quality \\
%}
