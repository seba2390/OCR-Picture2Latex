%%%%%%%%%%%%%%%%%%%%%%%%%%%%%%%%%%%%
% A. Table and Graph Titles
%%%%%%%%%%%%%%%%%%%%%%%%%%%%%%%%%%%%
%%%%%%%%%%%%%%%%%%%%%%%%%%%%%%%%%%%%
% A1. Main Tables Titles 
%%%%%%%%%%%%%%%%%%%%%%%%%%%%%%%%%%%%
\newcommand{\TITLETABMAINONE}{Summary statistics.\label{tab:mainone}}
% \newcommand{\TITLETABMAINONELONG}{Internal table for reference longer version of table 1}
\newcommand{\TITLETABMAINONELONG}{\TITLETABMAINONE}
% inner-city four districts  
\newcommand{\TITLETABMAINTWO}{Distribution of cumulative ambient \PARPMTEN, composite index and extreme temperature exposures during the course of pregnancy.\label{tab:maintwo}}
\newcommand{\TITLETABMAINTHREE}{OLS regression analysis of birth weight with interactions of ambient \PARPMTEN and extreme temperature with maternal education.\label{tab:mainthree}}
\newcommand{\TITLETABMAINTHREEALLCITY}{OLS regression analysis of birth weight with interactions of ambient \PARPMTEN and extreme temperature with maternal education. Using all city districts pollution measurements.\label{tab:mainthreeallcity}}
\newcommand{\TITLETABMAINTHREEFULLTERM}{OLS regression analysis of birth weight with interactions of ambient \PARPMTEN and extreme temperature with maternal education. Using only full-term births.\label{tab:mainthreefullterm}}
\newcommand{\TITLETABMAINTHREEBIMARGIN}{Marginal-effects from logistic regression analysis of binary birth outcomes with interactions of ambient \PARPMTEN and extreme temperature with maternal education.\label{tab:mainthreebimargin}}
\newcommand{\TITLETABMAINTHREEBIODDS}{Odds-ratio from Logistic regression analysis of binary birth outcomes with interactions of ambient \PARPMTEN and extreme temperature with maternal education.\label{tab:mainthreebiodds}}
\newcommand{\TITLETABMAINFOUR}{Conditional quantile regression analysis of birth weight with ambient \PARPMTEN and extreme temperature and maternal education.\label{tab:mainfour}}
\newcommand{\TITLETABMAINFIVE}{Conditional quantile regression analysis of birth weight with interactions of ambient \PARPMTEN and extreme temperature with maternal education.\label{tab:mainfive}}
%%%%%%%%%%%%%%%%%%%%%%%%%%%%%%%%%%%%
% A2. Appendix Tables Titles 
%%%%%%%%%%%%%%%%%%%%%%%%%%%%%%%%%%%%
\newcommand{\TITLETABAPPONE}{OLS regression analysis of birth weight with interactions of ambient composite index and extreme temperature with maternal education.\label{tab:appone}}
\newcommand{\TITLETABAPPTWO}{Conditional quantile regression analysis of birth weight with ambient composite index and extreme temperature and maternal education.\label{tab:apptwo}}
\newcommand{\TITLETABAPPTHREE}{Conditional quantile regression analysis of birth weight with interactions of ambient composite index and extreme temperature with maternal education.\label{tab:appthree}}
%%%%%%%%%%%%%%%%%%%%%%%%%%%%%%%%%%%%
% A3. Graph Titles
%%%%%%%%%%%%%%%%%%%%%%%%%%%%%%%%%%%%
\newcommand{\TITLEFIGONE}{Daily mean ambient air pollution levels for three major pollutants, 2008 to 2011, Guangzhou, China.\label{fig:mainone}}
\newcommand{\TITLEFIGONETEMP}{
Daily mean temperature (Universal Thermal Climate Index), Guangzhou, China, 2008 to 2011.\label{fig:mainonetemp}
}
\newcommand{\TITLEFIGTWO}{Daily mean ambient \PARPMTEN exposure by mother's education.\label{fig:maintwo}}
\newcommand{\TITLEFIGTHREE}{Graphical illustration of mean and conditional quantile estimation coefficients from models of birth weight (grams) for  average daily mean ambient \PARPMTEN exposures ($\mu g/m^3$) (top row) and composite index (bottom row) for mothers with college education and mothers with high school or less education (HS).\label{fig:mainthree}}
\newcommand{\TITLEFIGFOUR}{Graphical illustration of predicted college premium in birth weight for different conditional quantiles and at different ambient \PARPMTEN exposure levels.\label{fig:mainfour}}

%%%%%%%%%%%%%%%%%%%%%%%%%%%%%%%%%%%%
% B. Table and Graph Notes
%%%%%%%%%%%%%%%%%%%%%%%%%%%%%%%%%%%%
%%%%%%%%%%%%%%%%%%%%%%%%%%%%%%%%%%%%
% B0. Common Notes Strings
%%%%%%%%%%%%%%%%%%%%%%%%%%%%%%%%%%%%
\newcommand{\EMPHNOTES}{\emph{Note:}\thinspace}
\newcommand{\CONTROLS}{\EMPHNOTES Regressions control for conception year and month interaction fixed effects, day of the week at birth, parity, and daily mean rainfall.}
\newcommand{\SIGESTIEACH}{$^{\dagger}$ $p<0.10$; $\sym{*}$ $p<0.05$; $\sym{**}$ $p<0.01$.}
\newcommand{\SIGESTIEACHQUANT}{\SIGESTIEACH Bootstrap standard errors are shown in parentheses.}
\newcommand{\SIGQUANTCOMP}{\\Given bootstrapped simultaneous conditional quantile estimates, superscripts a, b, c, and d indicate whether estimates across conditional quantiles are statistically different for the three ambient environment or education variables.
\\\textsuperscript{a} P10, P25, and P50 are significantly different at 0.05 sig. level.
\\\textsuperscript{b} P10 and P90 are significantly different at 0.05 sig. level.
\\\textsuperscript{c} P25 and P75 are significantly different at 0.05 sig. level.
\\\textsuperscript{d} P50, P75, and P90 are significantly different at 0.05 sig. level.}
%%%%%%%%%%%%%%%%%%%%%%%%%%%%%%%%%%%%
% B1. Main Tables Notes
%%%%%%%%%%%%%%%%%%%%%%%%%%%%%%%%%%%%
\newcommand{\TABNOTESMAINTABONE}{\small \textit{Note: } The analytic sample has 53,879 observations. The college subsample has 17,967 observations and the high school or lower subsample has 35,912 observations. Sources and computations for ambient environmental variables are discussed in the data section of the paper.\\
$^{\ddagger}$ P-value from testing whether the mean gap for each variable between college-educated and non-college-educated mothers is statistically different.}
\newcommand{\TABNOTESMAINTABTWO}{\small \textit{Note: }Figures reported in the table represent distributional statistics along the marginal distribution of ambient pollution and extreme temperature variables.}
\newcommand{\TABNOTESMAINTABTHREE}{\small\CONTROLS\\\SIGESTIEACH}
\newcommand{\TABNOTESMAINTABTHREEALLCITY}{\small\CONTROLS\\\SIGESTIEACH}
\newcommand{\TABNOTESMAINTABTHREEFULLTERM}{\small\CONTROLS\\\SIGESTIEACH}
\newcommand{\TABNOTESMAINTABTHREEBIMARGIN}{\small\CONTROLS\\\SIGESTIEACH}
\newcommand{\TABNOTESMAINTABTHREEBIODDS}{\small\CONTROLS\\\SIGESTIEACH}
\newcommand{\TABNOTESMAINTABFOUR}{\small\CONTROLS\SIGQUANTCOMP\\\SIGESTIEACHQUANT}
\newcommand{\TABNOTESMAINTABFIVE}{\small\CONTROLS\SIGQUANTCOMP\\\SIGESTIEACHQUANT}
%%%%%%%%%%%%%%%%%%%%%%%%%%%%%%%%%%%%
% B2. Appendix Tables Notes 
%%%%%%%%%%%%%%%%%%%%%%%%%%%%%%%%%%%%
\newcommand{\TABNOTESAPPTABONE}{\small\CONTROLS\\\SIGESTIEACH}
\newcommand{\TABNOTESAPPTABTWO}{\small\CONTROLS\SIGQUANTCOMP\\\SIGESTIEACHQUANT}
\newcommand{\TABNOTESAPPTABTHREE}{\small\CONTROLS\SIGQUANTCOMP\\\SIGESTIEACHQUANT}
%%%%%%%%%%%%%%%%%%%%%%%%%%%%%%%%%%%%
% B3. Graph Notes
%%%%%%%%%%%%%%%%%%%%%%%%%%%%%%%%%%%%
\newcommand{\FIGNOTESMAINFIGONE}{\small\EMPHNOTES This figure shows daily means for three monitored ambient air pollutants in Guangzhou between 2008 and 2011.}
\newcommand{\FIGNOTESMAINFIGONETEMP}{\small
% \EMPHNOTES This figure shows daily means of Universal Thermal Climate Index temperatures in Guangzhou between 2008 and 2011.
}
\newcommand{\FIGNOTESMAINFIGTWO}{\small\EMPHNOTES The left panel contains observed data, while the right panel depicts the conditional distribution of daily mean ambient \PARPMTEN exposure after controlling for conception year and month interaction fixed effects.}
\newcommand{\FIGNOTESMAINFIGTHREE}{\small\EMPHNOTES The x-axis of the sub-plots corresponds to results from different conditional quantile estimations. 
We interpret lower conditional quantiles as corresponding to higher levels of unobserved innate vulnerabilities for babies. 
The two subfigures in the top panel are based on estimates from Tables \ref{tab:mainthree} and \ref{tab:mainfive}.
The two subfigures in the bottom panel are based on estimates from Appendix Tables \ref{tab:appone} and \ref{tab:appthree}. 
The y-axis reports regression coefficients. The left and right panels depict the same information in two ways.
The solid line in the right panel shows the gap between the HS and college quantile coefficient lines in the corresponding left panel. The gaps depicted on the right provide a visualization of the maternal college education and pollution interaction coefficients, from Table \ref{tab:mainfive} in the top panel for \PARPMTEN or from Table \ref{tab:appthree} for the composite index.
In the top panel, coefficients indicate grams of birth weight change for each additional $\mu g/m^3$ of average daily mean ambient \PARPMTEN exposures. In the bottom panel, coefficients indicate grams of birth weight change for each additional index unit of the composite index.}
\newcommand{\FIGNOTESMAINFIGFOUR}{\small\EMPHNOTES The x-axis of the figure represents different ambient \PARPMTEN exposure levels during the course of pregnancy, which correspond to the range of ambient \PARPMTEN exposure levels reported in Table \ref{tab:maintwo}. The y-axis reports predicted college premium in birth weight--the predicted gap in birth weight between college-educated and non-college-educated mothers. Given estimates from Tables \ref{tab:mainthree} and \ref{tab:mainfive}, each line corresponds to predictions for children at differing conditional quantiles. The premium is greater at higher ambient exposure levels and for children with higher levels of unobserved innate vulnerabilities--those at lower conditional quantiles.}


%%%%%%%%%%%%%%%%%%%%%%%%%%%%%%%%%%%%
% E. Graphs 
%%%%%%%%%%%%%%%%%%%%%%%%%%%%%%%%%%%%

\pagebreak 
\begin{figure}[H]
	\centering
	\caption{\TITLEFIGONE}
	\includegraphics[width=1.0\textwidth, center]{"paper_final_fig/Graph_pollution_Guangzhou_inner".eps}
	\captionsetup{width=1.0\textwidth}\caption*{\FIGNOTESMAINFIGONE}
\end{figure}
\pagebreak 
\begin{figure}[H]
	\centering
	\caption{\TITLEFIGONETEMP}
	\includegraphics[width=0.75\textwidth, center]{"paper_final_fig/graph_UTCI_distribution_Guangzhou".eps}
	\captionsetup{width=0.75\textwidth}\caption*{\FIGNOTESMAINFIGONETEMP}
\end{figure}
\pagebreak 
\begin{figure}[H]
	\centering
	\caption{\TITLEFIGTWO}
	\includegraphics[width=1.0\textwidth, center]{"paper_final_fig/pm10_day_distribution_correction_inner".eps}
	\captionsetup{width=1.0\textwidth}\caption*{\FIGNOTESMAINFIGTWO}
\end{figure}
\pagebreak 
\begin{figure}[H]
	\centering
	\captionsetup{width=1.00\textwidth}\caption{\TITLEFIGTHREE}
	\includegraphics[width=1.00\textwidth, center]{"paper_final_fig/FIG3_PM10PC1_EXCEL2PDF".eps}
	\captionsetup{width=1.00\textwidth}\caption*{\FIGNOTESMAINFIGTHREE}
\end{figure}
\pagebreak 
\begin{figure}[H]
	\centering
	\caption{\TITLEFIGFOUR}
	\includegraphics[width=1.0\textwidth, center]{"paper_final_fig/FIG4_PM10PREDICT_EXCEL2PDF".eps}
	\captionsetup{width=1.0\textwidth}\caption*{\FIGNOTESMAINFIGFOUR}
\end{figure}
\clearpage

%%%%%%%%%%%%%%%%%%%%%%%%%%%%%%%%%%%%
% F. Tables
%%%%%%%%%%%%%%%%%%%%%%%%%%%%%%%%%%%%

\pagebreak 
\begin{table}[htbp]
\centering
\caption{\hspace*{0mm}\TITLETABMAINONELONG}
\begin{adjustbox}{max width=1.0\textwidth}
\begin{tabular}{m{7.75cm}*{7}{>{\centering\arraybackslash}m{1.5cm}}}
\toprule
& & & \multicolumn{5}{c}{Comparison by mother's education}\\
\cmidrule(l{5pt}r{5pt}){4-8}
& \multicolumn{2}{c}{All} & \multicolumn{2}{c}{<= High school} & \multicolumn{2}{c}{>=College} & \multicolumn{1}{c}{Gap}\\
\cmidrule(l{5pt}r{5pt}){2-3} \cmidrule(l{5pt}r{5pt}){4-5} \cmidrule(l{5pt}r{5pt}){6-7} \cmidrule(l{5pt}r{5pt}){8-8} 
 & mean & s.d. & mean & s.d. & mean & s.d. & p-value \\
\midrule
\addlinespace
\multicolumn{8}{l}{\hspace*{0mm}Child variables}\\
\addlinespace
\hspace*{6mm}Sex (male=1) & 0.53 & 0.50 & 0.54 & 0.50 & 0.53 & 0.50 & (0.01)\\
\addlinespace
\hspace*{6mm}Birth weight (grams) & 3182 & 473 & 3167 & 492 & 3213 & 429 & (0.00)\\
\addlinespace
\hspace*{6mm}Gestational age (days) & 273 & 11.51 & 273 & 12.31 & 273 & 9.69 & (0.00)\\
\addlinespace
\hspace*{6mm}Low birth weight    &  0.06 &        0.23&        0.07&        0.25&        0.04&        0.19&      (0.00)\\
\addlinespace
\hspace*{6mm}Preterm (\%, gestational age $<$ 37weeks) & 0.07 & 0.26 & 0.08 & 0.27 & 0.05 & 0.23 & (0.00)\\
\addlinespace
\hspace*{6mm}Small for Gestational Age   &        0.09&        0.29&        0.10&        0.30&        0.08&        0.27&      (0.00)\\
\addlinespace
\addlinespace
\multicolumn{8}{l}{\hspace*{0mm}Maternal variables}\\
\addlinespace
\hspace*{6mm}Mother's age (years) & 29.08 & 4.17 & 28.84 & 4.51 & 29.57 & 3.35 & (0.00)\\
\addlinespace
\hspace*{6mm}Mother's schooling attainment (years) & 13.13 & 2.19 & 11.75 & 0.90 & 15.89 & 1.15 & (0.00)\\
\addlinespace
\hspace*{6mm}Parity  & 1.25 & 0.51 & 1.34 & 0.57 & 1.06 & 0.26 & (0.00)\\
\addlinespace
\addlinespace
\multicolumn{8}{l}{\hspace*{0mm}Average of daily mean potential pollution exposures during pregnancy }\\
\addlinespace
\addlinespace
\multicolumn{8}{l}{\hspace*{6mm}\textit{All city districts measurements}}\\
\addlinespace
\hspace*{6mm}\PARPMTEN ($\mu g/m^3$) & 73.22 & 7.16 & 72.52 & 6.89 & 74.64 & 7.49 & (0.00)\\
\addlinespace
\hspace*{6mm}NO\textsubscript{2} ($\mu g/m^3$) & 41.41 & 6.13 & 40.61 & 6.00 & 43.01 & 6.07 & (0.00)\\
\addlinespace
\hspace*{6mm}SO\textsubscript{2} ($\mu g/m^3$) & 35.18 & 4.63 & 34.46 & 4.64 & 36.62 & 4.26 & (0.00)\\
\addlinespace
\hspace*{6mm}Composite index of \PARPMTEN, NO\textsubscript{2}, SO\textsubscript{2} & -0.05 & 1.56 & -0.26 & 1.53 & 0.38 & 1.53 & (0.00)\\
\addlinespace
\addlinespace
\multicolumn{8}{l}{\hspace*{6mm}\textit{Center-city districts measurements}}\\
\addlinespace
\hspace*{6mm}\PARPMTEN ($\mu g/m^3$) & 72.10 & 6.63 & 71.53 & 6.36 & 73.23 & 6.99 & (0.00)\\
\addlinespace
\hspace*{6mm}NO\textsubscript{2} ($\mu g/m^3$) & 48.60 & 6.94 & 47.73 & 6.79 & 50.32 & 6.90 & (0.00)\\
\addlinespace
\hspace*{6mm}SO\textsubscript{2} ($\mu g/m^3$) & 34.44 & 4.62 & 33.74 & 4.68 & 35.83 & 4.18 & (0.00)\\
\addlinespace
\hspace*{6mm}Composite index of \PARPMTEN, NO\textsubscript{2}, SO\textsubscript{2} & -0.02 & 1.48 & -0.22 & 1.45 & 0.37 & 1.46 & (0.00)\\
\addlinespace
\addlinespace
\multicolumn{8}{l}{\hspace*{0mm}Temperature and rainfall during pregancy}\\
\addlinespace
\hspace*{6mm}Average daily rainfall (mm) & 5.15 & 1.61 & 5.20 & 1.64 & 5.03 & 1.55 & (0.00)\\
\addlinespace
\hspace*{6mm}Average daily mean temperature (C$^\circ$) & 22.38 & 1.83 & 22.32 & 1.86 & 22.50 & 1.77 & (0.00)\\
\addlinespace
\addlinespace
\multicolumn{8}{l}{\hspace*{6mm}\textit{Percent of pregnancy days with potential exposure to extreme temperatures}}\\
\addlinespace
\hspace*{6mm}Extreme heat, above past 99\% & 1.73 & 1.61 & 1.60 & 1.54 & 1.98 & 1.71 & (0.00)\\
\addlinespace
\hspace*{6mm}Extreme cold, below past 1\% & 1.70 & 1.00 & 1.77 & 1.01 & 1.57 & 0.95 & (0.00)\\
\addlinespace
\hspace*{6mm}Extreme heat, above past 97.5\%  & 3.73 & 2.87 & 3.54 & 2.75 & 4.11 & 3.08 & (0.00)\\
\addlinespace
\hspace*{6mm}Extreme cold, below past 2.5\% & 4.40 & 2.47 & 4.53 & 2.52 & 4.13 & 2.33 & (0.00)\\
\addlinespace
\bottomrule
\addlinespace[0.5em]
\multicolumn{8}{p{1.35\textwidth}}{\parbox[t]{1.35\textwidth}{\TABNOTESMAINTABONE}}\\
\end{tabular}
\end{adjustbox}
\end{table}

\pagebreak 
\begin{table}[htbp]
\centering
\captionsetup{width=.90\textwidth}
\caption{\hspace*{0mm}\TITLETABMAINTWO}
\begin{adjustbox}{max width=0.9\textwidth}
\begin{tabular}{m{2.5cm}*{6}{>{\centering\arraybackslash}m{2cm}}}
\toprule
& & & \multicolumn{4}{c}{Varying cutoffs for extreme temperature exposures}\\
\cmidrule(l{5pt}r{5pt}){4-7}
& \multicolumn{2}{c}{Pollution measures} & \multicolumn{2}{c}{1 percent cutoff} & \multicolumn{2}{c}{2.5 percent cutoff}\\
\cmidrule(l{5pt}r{5pt}){2-3} \cmidrule(l{5pt}r{5pt}){4-5} \cmidrule(l{5pt}r{5pt}){6-7}
Statistics & \PARPMTEN & composite & heat & cold & heat & cold \\
       & {\footnotesize$\mu g/m^3$} & {\footnotesize index} & {\footnotesize percent days} & {\footnotesize percent days} & {\footnotesize percent days} & {\footnotesize percent days} \\
\midrule
\addlinespace
\multicolumn{7}{l}{\hspace*{0mm}Percentiles}\\
\addlinespace
\hspace*{6mm}P1 & 63.05  & -2.31 & 0.00 & 0.00 & 0.00 & 0.00\\
\addlinespace
\hspace*{6mm}P5 & 64.50  & -2.18 & 0.00 & 0.36 & 0.00 & 0.37\\
\addlinespace
\hspace*{6mm}P10 & 65.10 & -1.71 & 0.00 & 0.43 & 0.38 & 1.08\\
\addlinespace
\hspace*{6mm}P25 & 66.80 & -1.33 & 0.71 & 0.73 & 1.75 & 2.50\\
\addlinespace
\hspace*{6mm}P50 & 70.01 & -0.25 & 0.76 & 1.79 & 2.59 & 4.23\\
\addlinespace
\hspace*{6mm}P75 & 76.41 & 1.20  & 3.65 & 2.63 & 6.88 & 6.83\\
\addlinespace
\hspace*{6mm}P90 & 82.56 & 2.29  & 4.38 & 2.93 & 8.39 & 7.33\\
\addlinespace
\hspace*{6mm}P95 & 86.45 & 2.66  & 4.48 & 3.00 & 8.58 & 7.49\\
\addlinespace
\hspace*{6mm}P99 & 88.52 & 2.92  & 4.71 & 3.17 & 9.02 & 7.94\\
\addlinespace
\addlinespace
\multicolumn{7}{l}{\hspace*{0mm}Min and Max}\\
\addlinespace
\hspace*{6mm}Min & 51.86 & -2.42 & 0.00 & 0.00 & 0.00 & 0.00\\
\addlinespace
\hspace*{6mm}Max & 99.91 & 4.86 & 6.35 & 4.35 & 12.17 & 10.87\\
\addlinespace
\bottomrule
\addlinespace[0.5em]
\multicolumn{7}{p{1.03\textwidth}}{\parbox[t]{1.03\textwidth}{\TABNOTESMAINTABTWO}}\\
\end{tabular}
\end{adjustbox}
\end{table}

\pagebreak 
\begin{table}[htbp]
\centering
\captionsetup{width=0.9\textwidth}
\caption{\hspace*{0mm}\TITLETABMAINTHREE}
\begin{adjustbox}{max width=0.9\textwidth}
\begin{tabular}{m{5.7cm}*{4}{>{\centering\arraybackslash}m{2cm}}}
\toprule
& \multicolumn{4}{c}{Varying cutoffs of extreme temperature exposures}\\
\cmidrule(l{5pt}r{5pt}){2-5}
& \multicolumn{2}{c}{1 percent cutoff} & \multicolumn{2}{c}{2.5 percent cutoff}\\
\cmidrule(l{5pt}r{5pt}){2-3} \cmidrule(l{5pt}r{5pt}){4-5} 
Variable & (1) & (2) & (3) & (4) \\
\midrule
\addlinespace
\multicolumn{5}{l}{\hspace*{0mm}Environmental exposure variables}\\
\addlinespace
\hspace*{6mm}\PARPMTEN & -17.83\sym{**} & -18.58\sym{**} & -14.59\sym{**} & -15.41\sym{**}\\
\addlinespace
 & (2.27) & (2.29) & (2.31) & (2.33)\\
\addlinespace
\hspace*{6mm}Extreme heat & -22.38\sym{*} & -27.06\sym{**} & 0.09 & -2.33\\
\addlinespace
 & (9.41) & (9.57) & (5.11) & (5.21)\\
\addlinespace
\hspace*{6mm}Extreme cold & -30.84\sym{**} & -30.33\sym{**} & -24.68\sym{**} & -24.33\sym{**}\\
\addlinespace
 & (9.80) & (9.96) & (4.37) & (4.43)\\
\addlinespace
\addlinespace
\multicolumn{5}{l}{\hspace*{0mm}Education and environmental exposure interactions}\\
\addlinespace
\hspace*{6mm}College educated & 44.55\sym{**} & -115.70\sym{*} & 44.85\sym{**} & -133.00\sym{*}\\
\addlinespace
 & (4.43) & (53.17) & (4.43) & (55.52)\\
\addlinespace
\hspace*{6mm}College x \PARPMTEN &  & 1.95\sym{**} &  & 2.19\sym{**}\\
\addlinespace
 &  & (0.68) &  & (0.70)\\
\addlinespace
\hspace*{6mm}College x extreme heat &  & 11.82\sym{**} &  & 6.26\sym{**}\\
\addlinespace
 &  & (2.93) &  & (1.71)\\
\addlinespace
\hspace*{6mm}College x extreme cold &  & -2.15 &  & -1.35\\
\addlinespace
 &  & (4.68) &  & (1.87)\\
\addlinespace
\addlinespace
\multicolumn{5}{l}{\hspace*{0mm}Control variables}\\
\addlinespace
\hspace*{6mm}Male & 104.00\sym{**} & 104.00\sym{**} & 103.90\sym{**} & 103.90\sym{**}\\
\addlinespace
 & (3.93) & (3.93) & (3.93) & (3.93)\\
\addlinespace
\hspace*{6mm}Mother’s age & 55.41\sym{**} & 54.90\sym{**} & 55.33\sym{**} & 54.85\sym{**}\\
\addlinespace
 & (4.98) & (4.98) & (4.98) & (4.98)\\
\addlinespace
\hspace*{6mm}Mother’s age$^2$ & -0.90\sym{**} & -0.89\sym{**} & -0.89\sym{**} & -0.89\sym{**}\\
\addlinespace
 & (0.08) & (0.08) & (0.08) & (0.08)\\
\addlinespace
\hspace*{0mm}Intercept & 1,205.00\sym{**} & 1,276.00\sym{**} & 1,019.00\sym{**} & 1,094.00\sym{**}\\
\addlinespace
 & (219.90) & (221.50) & (221.40) & (223.30)\\
\addlinespace
\midrule
Observations & 53,879 & 53,879 & 53,879 & 53,879\\
R$^2$	& 0.069 & 0.069 & 0.069 & 0.070\\
\bottomrule
\addlinespace[0.5em]
\multicolumn{5}{p{0.95\textwidth}}{\parbox[t]{0.95\textwidth}{\TABNOTESMAINTABTHREE}}\\
\end{tabular}
\end{adjustbox}
\end{table}

\pagebreak 
\begin{table}[htbp]
\centering
\captionsetup{width=1.0\textwidth}
\caption{\hspace*{0mm}\TITLETABMAINFOUR}
\begin{adjustbox}{max width=1.0\textwidth}
\begin{tabular}{m{5.7cm}*{5}{>{\centering\arraybackslash}m{2cm}}}
\toprule
& \multicolumn{5}{c}{Estimates at conditional quantiles}\\
\cmidrule(l{5pt}r{5pt}){2-6} 
Variable & P10 & P25 & P50 & P75 & P90\\
\midrule
\addlinespace
\multicolumn{6}{l}{\hspace*{0mm}Environmental exposure variables}\\
\addlinespace
\hspace*{6mm}\PARPMTEN $^{a,b,c,d}$ & -38.90\sym{**} & -26.34\sym{**} & -22.16\sym{**} & -15.30\sym{**} & -17.50\sym{**}\\
\addlinespace
 & (3.49) & (2.44) & (2.09) & (2.38) & (2.83)\\
\addlinespace
\hspace*{6mm}Extreme heat & -17.88 & -13.15$^\dagger$ & -19.06\sym{*} & -8.10 & -7.43\\
\addlinespace
 & (13.26) & (7.84) & (8.90) & (11.00) & (12.17)\\
\addlinespace
\hspace*{6mm}Extreme cold $^{a,b}$ & -62.61\sym{**} & -29.79\sym{**} & -19.68\sym{*} & -23.89\sym{*} & -16.82\\
\addlinespace
 & (16.30) & (9.85) & (9.37) & (10.15) & (13.97)\\
\addlinespace
\addlinespace
\multicolumn{6}{l}{\hspace*{0mm}Education}\\
\addlinespace
\hspace*{6mm}College educated $^{a,b,c,d}$ & 64.59\sym{**} & 33.20\sym{**} & 21.79\sym{**} & 9.10 & 11.19\\
\addlinespace
 & (8.08) & (5.72) & (4.81) & (5.87) & (7.39)\\
\addlinespace
\addlinespace
\multicolumn{6}{l}{\hspace*{0mm}Control variables}\\
\addlinespace
\hspace*{6mm}Male & 92.62\sym{**} & 98.92\sym{**} & 111.94\sym{**} & 117.30\sym{**} & 132.60\sym{**}\\
\addlinespace
 & (7.67) & (5.07) & (4.42) & (4.80) & (6.75)\\
\addlinespace
\hspace*{6mm}Mother’s age & 64.30\sym{**} & 45.91\sym{**} & 46.47\sym{**} & 51.18\sym{**} & 46.37\sym{**}\\
\addlinespace
 & (10.01) & (5.79) & (5.63) & (5.53) & (8.81)\\
\addlinespace
\hspace*{6mm}Mother’s age$^2$ & -1.09\sym{**} & -0.73\sym{**} & -0.73\sym{**} & -0.78\sym{**} & -0.69\sym{**}\\
\addlinespace
 & (0.17) & (0.10) & (0.09) & (0.09) & (0.15)\\
\addlinespace
\hspace*{0mm}Intercept & 917.89\sym{*} & 1,168.25\sym{**} & 1,835.52\sym{**} & 1,897.93\sym{**} & 2,226.62\sym{**}\\
\addlinespace
 & (378.70) & (252.98) & (200.27) & (232.69) & (334.94)\\
\addlinespace
\midrule
Observations & 53,879 & 53,879 & 53,879 & 53,879 & 53,879\\
\bottomrule
\addlinespace[0.5em]
\multicolumn{6}{p{1.09\textwidth}}{\parbox[t]{1.09\textwidth}{\TABNOTESMAINTABFOUR}}\\
\end{tabular}
\end{adjustbox}
\end{table}

\pagebreak 
\begin{table}[htbp]
\centering
\captionsetup{width=1.0\textwidth}
\caption{\hspace*{0mm}\TITLETABMAINFIVE}
\begin{adjustbox}{max width=1.0\textwidth}
\begin{tabular}{m{5.7cm}*{5}{>{\centering\arraybackslash}m{2cm}}}
\toprule
& \multicolumn{5}{c}{Estimates at conditional quantiles}\\
\cmidrule(l{5pt}r{5pt}){2-6} 
Variable & P10 & P25 & P50 & P75 & P90\\
\midrule
\addlinespace
\multicolumn{6}{l}{\hspace*{0mm}Environmental exposure variables}\\
\addlinespace
\hspace*{6mm}\PARPMTEN $^{a,b,c,d}$ & -40.75\sym{**} & -28.15\sym{**} & -22.55\sym{**} & -16.04\sym{**} & -18.06\sym{**}\\
\addlinespace
 & (3.03) & (2.83) & (2.14) & (2.15) & (3.09)\\
\addlinespace
\hspace*{6mm}Extreme heat & -37.30\sym{**} & -23.23\sym{*} & -23.50\sym{**} & -12.04 & -13.12\\
\addlinespace
 & (13.73) & (10.64) & (8.23) & (10.92) & (13.12)\\
\addlinespace
\hspace*{6mm}Extreme cold $^{a,b}$ & -65.43\sym{**} & -28.98\sym{**} & -17.15\sym{*} & -22.28\sym{*} & -13.74\\
\addlinespace
 & (15.79) & (10.57) & (8.75) & (10.45) & (12.38)\\
\addlinespace
\addlinespace
\multicolumn{6}{l}{\hspace*{0mm}Education and environmental exposure interactions}\\
\addlinespace
\hspace*{6mm}College educated $^{a,b,c}$ & -396.40\sym{**} & -216.76\sym{**} & -83.69 & -95.51 & 2.20\\
\addlinespace
 & (99.62) & (59.02) & (58.77) & (60.23) & (87.96)\\
\addlinespace
\hspace*{6mm}College x \PARPMTEN $^{a,b}$ & 5.69\sym{**} & 3.11\sym{**} & 1.31$^\dagger$ & 1.39$^\dagger$ & -0.02\\
\addlinespace
 & (1.30) & (0.78) & (0.72) & (0.78) & (1.10)\\
\addlinespace
\hspace*{6mm}College x extreme heat $^{a,b}$ & 26.35\sym{**} & 11.83\sym{**} & 7.30\sym{*} & 5.24 & 6.19\\
\addlinespace
 & (5.42) & (3.91) & (3.27) & (3.50) & (4.75)\\
\addlinespace
\hspace*{6mm}College x extreme cold & 3.30 & 2.21 & -2.68 & -3.97 & -1.34\\
\addlinespace
 & (9.12) & (6.28) & (4.94) & (5.87) & (7.69)\\
\addlinespace
\addlinespace
\multicolumn{6}{l}{\hspace*{0mm}Control variables}\\
\addlinespace
\hspace*{6mm}Male & 90.14\sym{**} & 98.32\sym{**} & 112.02\sym{**} & 116.40\sym{**} & 132.55\sym{**}\\
\addlinespace
 & (8.38) & (5.15) & (4.42) & (5.15) & (6.12)\\
\addlinespace
\hspace*{6mm}Mother’s age & 62.49\sym{**} & 44.68\sym{**} & 46.26\sym{**} & 48.33\sym{**} & 45.30\sym{**}\\
\addlinespace
 & (10.02) & (6.04) & (5.07) & (5.70) & (8.91)\\
\addlinespace
\hspace*{6mm}Mother’s age$^2$ & -1.06\sym{**} & -0.71\sym{**} & -0.73\sym{**} & -0.74\sym{**} & -0.67\sym{**}\\
\addlinespace
 & (0.17) & (0.10) & (0.09) & (0.10) & (0.15)\\
\addlinespace
\hspace*{0mm}Intercept & 1,240.85\sym{**} & 1,415.85\sym{**} & 1,888.54\sym{**} & 2,032.87\sym{**} & 2,357.66\sym{**}\\
\addlinespace
 & (371.67) & (263.70) & (206.46) & (202.99) & (369.89)\\
\addlinespace
\midrule
Observations & 53,879 & 53,879 & 53,879 & 53,879 & 53,879\\
\bottomrule
\addlinespace[0.5em]
\multicolumn{6}{p{1.09\textwidth}}{\parbox[t]{1.09\textwidth}{\TABNOTESMAINTABFIVE}}\\
\end{tabular}
\end{adjustbox}
\end{table}

\pagebreak
\clearpage