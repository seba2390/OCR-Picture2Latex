%%%%%%%%%%%%%%%%%%%%%%%%%%%%%%%%%%%%%%%%
% 1. Define Keywords, JEL
%%%%%%%%%%%%%%%%%%%%%%%%%%%%%%%%%%%%%%%%
\newcommand{\PAPERKEYWORDS}{\textbf{Keywords}: air pollution, birth weight, maternal education, extreme temperatures, China}
\newcommand{\PAPERJEL}{\textbf{JEL}: I14, J1,Q53, Q54}

\newcommand{\PAPERKEYWORDSSEC}{
\section*{Keywords}
\PAPERKEYWORDS
}

%%%%%%%%%%%%%%%%%%%%%%%%%%%%%%%%%%%%%%%%
% 2. Define Title
%%%%%%%%%%%%%%%%%%%%%%%%%%%%%%%%%%%%%%%%
\newcommand{\PAPERTITLE}{Same environment, stratified impacts? Air pollution, extreme temperatures, and birth weight in south China}

%%%%%%%%%%%%%%%%%%%%%%%%%%%%%%%%%%%%%%%%
% 3. Define Authors
%%%%%%%%%%%%%%%%%%%%%%%%%%%%%%%%%%%%%%%%
\newcommand{\AUTHORZHAO}{Qingguo Zhao}
\newcommand{\AUTHORZHAOURL}{https://www.scopus.com/authid/detail.uri?authorId=55766446700}
\newcommand{\AUTHORZHAOINFO}{\href{\AUTHORZHAOURL}{\AUTHORZHAO}: 
Epidemiological Research Office of Key Laboratory of Male Reproduction and Genetics National Health and Family Planning Commission, Family Planning Research Institute of Guangdong Province, Guangzhou, Guangdong, China}

\newcommand{\AUTHORBEHRMAN}{Jere R. Behrman}
\newcommand{\AUTHORBEHRMANURL}{http://orcid.org/0000-0001-7835-6283}
\newcommand{\AUTHORBEHRMANINFO}{\href{\AUTHORBEHRMANURL}{\AUTHORBEHRMAN}: Departments of Economics and Sociology and Population Studies Center, University of Pennsylvania, 133 South 36th Street, Philadelphia, Pennsylvania, USA}

\newcommand{\AUTHORHANNUM}{Emily Hannum}
\newcommand{\AUTHORHANNUMURL}{https://orcid.org/0000-0003-2011-9984}
\newcommand{\AUTHORHANNUMINFO}{\href{\AUTHORHANNUMURL}{\AUTHORHANNUM} (corresponding, \href{mailto:hannumem@soc.upenn.edu}{hannumem@soc.upenn.edu}): Department of Sociology and Population Studies Center, University of Pennsylvania, 3718 Locust Walk, Philadelphia, Pennsylvania, USA}

\newcommand{\AUTHORLIU}{Xiaoying Liu}
\newcommand{\AUTHORLIUURL}{https://orcid.org/0000-0001-9897-1291}
\newcommand{\AUTHORLIUINFO}{\href{\AUTHORLIUURL}{\AUTHORLIU}: Population Studies Center, University of Pennsylvania, 3718 Locust Walk, Philadelphia, Pennsylvania, USA}

\newcommand{\AUTHORWANG}{Fan Wang}
\newcommand{\AUTHORWANGURL}{https://orcid.org/0000-0003-2640-5420}
\newcommand{\AUTHORWANGINFO}{\href{\AUTHORWANGURL}{\AUTHORWANG}: Department of Economics, University of Houston, 3623 Cullen Boulevard, Houston, Texas, USA}


\newcommand{\PAPERAUTHORS}{
\AUTHORLIU,\footnote{\AUTHORLIUINFO}
\\
\AUTHORBEHRMAN,\footnote{\AUTHORBEHRMANINFO}
\\
\AUTHORHANNUM,\footnote{\AUTHORHANNUMINFO}
\\
\AUTHORWANG,\footnote{\AUTHORWANGINFO}
\\
\AUTHORZHAO\thinspace\footnote{\AUTHORZHAOINFO}}

%%%%%%%%%%%%%%%%%%%%%%%%%%%%%%%%%%%%%%%%
% 4. Define Highlights
%%%%%%%%%%%%%%%%%%%%%%%%%%%%%%%%%%%%%%%%
\newcommand{\HIGHLIGHT}{
\section*{Highlights}
\begin{enumerate}
    \item We test maternal education as an effect modifier in associations between air pollution/extreme temperature and birth weight.
    \item We link birth records to environmental data from Guangzhou, China during a period of high and variable air pollution.
    \item Infants with unobserved vulnerabilities---at lower conditional quantiles of birth weight---face more risk from ambient exposures.
    \item The protection associated with college-educated mothers with respect to pollution and extreme heat is substantial.
    \item Protection is amplified under more extreme ambient conditions and for infants with greater unobserved innate vulnerability.
\end{enumerate}
}

%%%%%%%%%%%%%%%%%%%%%%%%%%%%%%%%%%%%%%%%
% 5. Define Thanks
%%%%%%%%%%%%%%%%%%%%%%%%%%%%%%%%%%%%%%%%
\newcommand{\ACKNOWLEDGMENTS}{This paper is part of the project "Prenatal Air Pollution Exposures and Early Childhood Outcomes," which is supported by a grant from the Penn China Research and Engagement Fund (PIs: Behrman and Hannum) and by National Science Foundation Grant 1756738 (PI: Hannum).  The authors also gratefully acknowledge support from the University of Houston Research Fund and, for coverage of Wang's time, Scholar Grant GS040-A-18 from the Chiang Ching-Kuo Foundation.  We thank Yu Xie and James Raymo for the opportunity to receive feedback at the Princeton Research in East Asian Demography and Inequality Seminar. We also thank Guy Grossman, Harsha Thirumurthy, Heather Schofield, and other participants in the Penn Development Research Initiative (PDRI) Seminar for comments. Finally, the authors appreciate the helpful feedback provided by anonymous reviewers.}

%%%%%%%%%%%%%%%%%%%%%%%%%%%%%%%%%%%%%%%%
% 6. Define Abstract
%%%%%%%%%%%%%%%%%%%%%%%%%%%%%%%%%%%%%%%%
\newcommand{\PAPERABSTRACT}{
This paper investigates whether associations between birth weight and prenatal ambient environmental conditions---pollution and extreme temperatures---differ by 1) maternal education; 2) children’s innate health; and 3) interactions between these two. We link birth records from Guangzhou, China, during a period of high pollution, to ambient air pollution (\PARPMTEN and a composite measure) and extreme temperature data. We first use mean regressions to test whether, overall, maternal education is an ``effect modifier'' in the relationships between ambient air pollution, extreme temperature, and birth weight. We then use conditional quantile regressions to  test for effect heterogeneity according to the unobserved innate vulnerability of babies after conditioning on other confounders. Results show that 1) the negative association between ambient exposures and birth weight is twice as large at lower conditional quantiles of birth weights as at the median; 2) the protection associated with college-educated mothers with respect to pollution and extreme heat is heterogeneous and potentially substantial: between 0.02 and 0.34 standard deviations of birth weights, depending on the conditional quantiles; 3) this protection is amplified under more extreme ambient conditions and for infants with greater unobserved innate vulnerabilities.\\
\PAPERJEL}

%%%%%%%%%%%%%%%%%%%%%%%%%%%%%%%%%%%%%%%%
% 6. Define citation or availability of latest draft
%%%%%%%%%%%%%%%%%%%%%%%%%%%%%%%%%%%%%%%%
\newcommand{\PAPERDOIURL}{https://doi.org/10.1016/j.ssresearch.2021.102691}
\newcommand{\PAPERINFO}{
This paper has been accepted for publication at Social Science Research: \url{\PAPERDOIURL}.
}


