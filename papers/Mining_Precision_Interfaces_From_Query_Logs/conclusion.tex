\section{Conclusion and Discussion}

This paper introduced the use of query logs as the API for interface generation, formalized the problem of extracting and generating interactive interfaces from these logs, and presented \sys as a language-agnostic solution.   To do so, we introduced a unified model over queries, query changes, interfaces, and interactions; and presented algorithms and optimizations to solve the interface generation problem for SQL and SPARQL. Visual interactive interfaces are increasingly relied upon in analysis, and we believe \sys presents an exciting research direction towards improving accessibility for the long-tail of analyses.   From another perspective, we view \sys as a compact {\it visual summary}  of a query log.

%general, semi-automatic approach towards interface generation for the long tail of analysis.   Interactive interfaces are a powerful abstraction to decouple technical expertise from analysis goals. \sys presents a generic, semi-automatic approach to identify such interfaces from previous analyses. We believe that this is an exciting research direction, as increasing the accessibility of data analysis is currently a major goal for the research community.

We are extending this work in several directions. First is to go beyond syntactic changes and incorporate  metadata, language semantics, database content and other information can lead to richer output interfaces. Second, we will investigate how grammar induction methods~\cite{berwick1987learning} can help us learn or recommend \lang statements directly. Finally, \sys{} currently assumes that \emph{all} queries in the log are relevant to the user's task and in the same language; a third direction is to support complex, multi-language logs.