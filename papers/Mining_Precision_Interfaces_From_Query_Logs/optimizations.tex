\section{Optimization}\label{s:opt}

A naive implementation of \sys first materializes \difftable by computing tree alignments between all pairs of ASTs in the query log, filtering \difftable using the \lang statements, transforming the results into an interaction graph, and performing the graph contraction procedure to derive the interfaces.  However, the cost of these steps, in particular tree alignment and graph contraction, can be considerable for even thousands of queries in the log. Fortunately, we may exploit three properties of the problem to reduce the number of pairwise comparisons: the transitivity of transformations,the existence of templates and the fact that all transformations may not be relevant.

\subsection{Transitive \lang Cliques}

\label{transcliques}
A \lang statement $s$ is transitive if matches between $(p_1, p_2)$ and $(p_2,p_3)$ implies a match between $(p_1,p_3)$:
{
  $$s(p_1, p_2) \ne \emptyset \wedge s(p_2, p_3) \ne \emptyset \rightarrow s(p_1, p_3) \ne \emptyset$$
}
If $s$ is transitive, then the set of programs $C$ that it matches forms a clique, and a new program $p$ need only compare with an arbitrary program $p_c\in C$ to check if matches with all members of the set.  Algorithm~\ref{a:clique} uses this observation to efficiently evaluate transitive \lang statements directly on the program log.
{\small
\begin{algorithm}
 \caption{Clique detection for transitive \lang statements.}
 \label{a:clique}
 \KwData{\lang statement: s, Programs: P}
 \KwResult{C}
 initialize C = $\{\}$\;
 \While{$p \in P$}{
  matched = false\;
  \While{$c \in C$}{
    \If{$\exists_{p_c \in C} s(p, p_c) \ne \emptyset$}{
     c.add(p)\;
     matched = true\;
     }
  }
  \If{$\neg$\textrm{matched}}
  {
    C.add($\{p\}$);
  }
 }
\end{algorithm}
}

We use a simple heuristic that identifies transitive statements in the \lang statements we have developed.  We check that the \texttt{WHERE} clause only contains transitive logical expressions (e.g., $=, \ne$).  We leave richer analysis techniques as future work.

A welcome side effect of this method is that it allows us to \emph{compress} the interaction graph. Indeed, this graph is often unpractical because it is very dense. Typically, it contains $\mathcal{O}(N^2)$ edges for $N$ queries, a consequence of the \lang's statements transitivity. Thanks to Algorithm~\ref{a:clique}, we can store it as a set of \emph{cliques} rather than a set of \emph{edges}, and lower the storage cost by an order of magnitude. We will show in Section~\ref{sec:atscale} that this strategy let us process large query logs in main memory.

% \ewu{HOW DO WE IDENTIFY THAT A PILANG STATEMENT, BASED ON ITS CURRENT SEMANTICS, IS TRANSITIVE?}
% \hc{We assume that all queries in the query log are distinct. The Where clause of PI-Lang statements can be seen as $\text{expr}_1 \wedge \text{expr}_2 \wedge \text{expr}_3 ...$. If every expr contains only set operations, and all operations are either == or != (e.g. a@old != a@new), then the PI-Lang statement is transitive.}
% \strike{We only have to look at the where clause of PI-Lang statements to determine if it is transitive. The where clause consists of arithmetic operations and set operations. The following example shows an arithmetic and a set operation respectively. Arithmetic operations are not transitive, while the transitivity of set operations can be determined using (will cite).  Actually describe how you determine if something is transitive or not.  Example above provides intuitiion but not really a procedure and not really any guarantees.}

\iffalse
\begin{verbatim}
|P@old - P@new| == 1;
W@old != W@new;
\end{verbatim}
\fi

\subsection{Program Templates}\label{s:templates}

We observe that query logs often contain cliques due to queries that have identical parse structures (i.e., {\it templates}) but different values in the literals.  For instance, queries emitted by varying a distance threshold or function parameter are identical everywhere except for a single value.  We term these cliques {\it templated cliques}.
To detect such cliques, our procedure is similar to finding template functions in Section~\ref{s:interface}: we replace the literals in the query ASTs (e.g., all node attributes are primitive values) with unnamed variables, hash the resulting templates and group by the hash values. Thus, we represent a group of similar ASTs by a template followed by a list of literals.  The rest of the system performs tree alignment and \lang evaluation over templated cliques rather than individual queries.  For each templated clique, we index the paths to each variable and probe each clique with the path expressions in each \lang statement.  A matching statement can use the index to quickly identify the ASTs that change and evaluate those; a statement that does not match can skip the clique altogether.

Since these operations do not rely on user inputs (e.g., \lang statements), we can perform them offline, during a preprocessing step. Furthermore, the output can be reused as the developer adds new \lang statements and tunes the interface generation parameters.


\subsection{Restricting Program Pairs}

Since the interfaces generated by \sys are derived from differences between pairs of programs, we can restrict the pairs to compare for performance and personalization purposes.   For instance, we might only compare program pairs in sequence (e.g., $p_i$ with $p_{i+1}$), or filter the program log table \texttt{progs} (Section~\ref{s:model-progs}) by the user, timestamp or other metadata.  Modeling the program log as a relation provides the system with considerable flexibility in choosing the program pairs.
