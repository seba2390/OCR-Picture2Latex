\documentclass[10pt]{article}
\usepackage{fullpage, amsmath, amssymb, amsfonts, amsthm, stmaryrd, url, hyperref, indentfirst, bm, bbm, extarrows, tikz-cd}
\usepackage[all]{xy}

\newtheorem{theorem}{Theorem}[section]
\newtheorem{lemma}[theorem]{Lemma}
\newtheorem{cor}[theorem]{Corollary}
\newtheorem{conj}[theorem]{Conjecture}
\newtheorem{prop}[theorem]{Proposition}
\theoremstyle{definition}
\newtheorem{defn}[theorem]{Definition}
\newtheorem{hypothesis}[theorem]{Hypothesis}
\newtheorem{example}[theorem]{Example}
\newtheorem{rmk}[theorem]{Remark}
\newtheorem{convention}[theorem]{Convention}
\newtheorem{notation}[theorem]{Notation}
\newtheorem{ques}[theorem]{Question}

\numberwithin{equation}{theorem}

\def\limind{\mathop{\oalign{{\rm lim}\cr
\hidewidth$\longrightarrow$\hidewidth\cr}}}
\def\limproj{\mathop{\oalign{{\rm lim}\cr
\hidewidth$\longleftarrow$\hidewidth\cr}}}

\newcommand{\AAA}{\mathbb{A}}
\newcommand{\CC}{\mathbb{C}}
\newcommand{\Cp}{\mathbb{C}_p}
\newcommand{\DD}{\mathbb{D}}
\newcommand{\dd}{\mathrm{d}}
\newcommand{\FF}{\mathbb{F}}
\newcommand{\Fp}{\mathbb{F}_p}
\newcommand{\Gp}{G_{\Qp}}
\newcommand{\OCp}{\mathfrak{o}}
\newcommand{\Qp}{\mathbb{Q}_p}
\newcommand{\PP}{\mathbb{P}}
\newcommand{\QQ}{\mathbb{Q}}
\newcommand{\RR}{\mathbb{R}}
\newcommand{\Zp}{\mathbb{Z}_p}
\newcommand{\ZZ}{\mathbb{Z}}
\newcommand{\bA}{\mathbf{A}}
\newcommand{\bB}{\mathbf{B}}
\newcommand{\bC}{\mathbf{C}}
\newcommand{\bE}{\mathbf{E}}
\newcommand{\be}{\mathbf{e}}
\newcommand{\bv}{\mathbf{v}}
\newcommand{\bw}{\mathbf{w}}
\newcommand{\bx}{\mathbf{x}}
\newcommand{\calB}{\mathcal{B}}
\newcommand{\calC}{\mathcal{C}}
\newcommand{\calD}{\mathcal{D}}
\newcommand{\calE}{\mathcal{E}}
\newcommand{\calF}{\mathcal{F}}
\newcommand{\calG}{\mathcal{G}}
\newcommand{\calH}{\mathcal{H}}
\newcommand{\calI}{\mathcal{I}}
\newcommand{\calJ}{\mathcal{J}}
\newcommand{\calL}{\mathcal{L}}
\newcommand{\calM}{\mathcal{M}}
\newcommand{\calO}{\mathcal{O}}
\newcommand{\calP}{\mathcal{P}}
\newcommand{\calR}{\mathcal{R}}
\newcommand{\calW}{\mathcal{W}}
\newcommand{\calY}{\mathcal{Y}}
\newcommand{\calZ}{\mathcal{Z}}
\newcommand{\gothb}{\mathfrak{b}}
\newcommand{\gothc}{\mathfrak{c}}
\newcommand{\gothD}{\mathfrak{D}}
\newcommand{\gothe}{\mathfrak{e}}
\newcommand{\gothl}{\mathfrak{l}}
\newcommand{\gothm}{\mathfrak{m}}
\newcommand{\gothM}{\mathfrak{M}}
\newcommand{\gothn}{\mathfrak{n}}
\newcommand{\gothN}{\mathfrak{N}}
\newcommand{\gothp}{\mathfrak{p}}
\newcommand{\gotho}{\mathfrak{o}}
\newcommand{\gothq}{\mathfrak{q}}
\newcommand{\gothS}{\mathfrak{S}}
\newcommand{\gothu}{\mathfrak{u}}
\newcommand{\gothU}{\mathfrak{U}}
\newcommand{\gothV}{\mathfrak{V}}
\newcommand{\dual}{\vee}

\newcommand{\QpLoc}{\mathbb{Q}_p\text{-}\mathbf{Loc}}
\newcommand{\QpdLoc}{\mathbb{Q}_{p^d}\text{-}\mathbf{Loc}}
\newcommand{\ZpLoc}{\mathbb{Z}_{p}\text{-}\mathbf{Loc}}
\newcommand{\ZpILoc}{\mathbb{Z}_{p}\text{-}\mathbf{ILoc}}
\newcommand{\ZpdLoc}{\mathbb{Z}_{p^d}\text{-}\mathbf{Loc}}
\newcommand{\ZpdILoc}{\mathbb{Z}_{p^d}\text{-}\mathbf{ILoc}}

\DeclareMathOperator{\ab}{ab}
\DeclareMathOperator{\alg}{alg}
\DeclareMathOperator{\AdBan}{\mathbf{AdBan}}
\DeclareMathOperator{\Alg}{\mathbf{Alg}}
\DeclareMathOperator{\alHom}{alHom}
\DeclareMathOperator{\Art}{Art}
\DeclareMathOperator{\Aut}{Aut}
\DeclareMathOperator{\bd}{bd}
\DeclareMathOperator{\bExt}{\mathbf{Ext}}
\DeclareMathOperator{\BT}{BT}
\DeclareMathOperator{\Coh}{\mathbf{Coh}}
\DeclareMathOperator{\coker}{coker}
\DeclareMathOperator{\cont}{cont}
\DeclareMathOperator{\D}{D}
\DeclareMathOperator{\der}{der}
\DeclareMathOperator{\diag}{diag}
\DeclareMathOperator{\End}{End}
\DeclareMathOperator{\et}{\acute{e}t}
\DeclareMathOperator{\Et}{\acute{E}t}
\DeclareMathOperator{\Ext}{Ext}
\DeclareMathOperator{\FEt}{\mathbf{F\acute{E}t}}
\DeclareMathOperator{\fet}{f\acute{e}t}
\DeclareMathOperator{\FFC}{FF}
\DeclareMathOperator{\Fil}{Fil}
\DeclareMathOperator{\Fitt}{Fitt}
\DeclareMathOperator{\Frac}{Frac}
\DeclareMathOperator{\frep}{frep}
\DeclareMathOperator{\Frob}{Frob}
\DeclareMathOperator{\Fss}{F-ss}
\DeclareMathOperator{\Gal}{Gal}
\DeclareMathOperator{\GL}{GL}
\DeclareMathOperator{\Gr}{Gr}
\DeclareMathOperator{\Hom}{Hom}
\DeclareMathOperator{\image}{image}
\DeclareMathOperator{\Id}{Id}
\DeclareMathOperator{\ind}{ind}
\DeclareMathOperator{\Ind}{Ind}
\DeclareMathOperator{\inte}{int}
\DeclareMathOperator{\Irr}{Irr}
\DeclareMathOperator{\Loc}{\mathbf{Loc}}
\DeclareMathOperator{\M}{M}
\DeclareMathOperator{\Map}{Map}
\DeclareMathOperator{\Maxspec}{Maxspec}
\DeclareMathOperator{\Mod}{Mod}
\DeclareMathOperator{\Mor}{Mor}
\DeclareMathOperator{\N}{N}
\DeclareMathOperator{\Norm}{Norm}
\DeclareMathOperator{\op}{op}
\DeclareMathOperator{\perf}{perf}
\DeclareMathOperator{\Pic}{Pic}
\DeclareMathOperator{\pr}{pr}
\DeclareMathOperator{\proet}{pro\acute{e}t}
\DeclareMathOperator{\profet}{prof\acute{e}t}
\DeclareMathOperator{\Proj}{Proj}
\DeclareMathOperator{\rank}{rank}
\DeclareMathOperator{\rec}{rec}
\DeclareMathOperator{\rel}{rel}
\DeclareMathOperator{\rig}{rig}
\DeclareMathOperator{\Real}{Re}
\DeclareMathOperator{\Rep}{Rep}
\DeclareMathOperator{\Res}{Res}
\DeclareMathOperator{\Set}{\mathbf{Set}}
\DeclareMathOperator{\SL}{SL}
\DeclareMathOperator{\Spa}{Spa}
\DeclareMathOperator{\Spec}{Spec}
\DeclareMathOperator{\spect}{sp}
\DeclareMathOperator{\Sprv}{Sprv}
\DeclareMathOperator{\Spv}{Spv}
\DeclareMathOperator{\Ss}{ss}
\DeclareMathOperator{\supp}{supp}
\DeclareMathOperator{\Tor}{Tor}
\DeclareMathOperator{\Tr}{Tr}
\DeclareMathOperator{\trdeg}{trdeg}
\DeclareMathOperator{\unr}{unr}
\DeclareMathOperator{\WD}{WD}
\DeclareMathOperator{\WDrep}{WDrep}

\title{An Ordinary Rank-Two Case of Local-Global Compatibility for Automorphic Representations of Arbitrary Weight Over CM Fields}
\author{Yuji Yang}
\date{}

\begin{document}
\maketitle
\begin{abstract}
We prove a rank-two potential automorphy theorem for mod $l$ representations satisfying an ordinary condition. Combined with an ordinary automorphy lifting theorem from \cite{ACC+18}, we prove a rank-two, $p\ne l$ case of local-global compatibility for regular algebraic cuspidal automorphic representations of arbitrary weight over CM fields that is $\iota$-ordinary for some $\iota:\overline{\QQ}_l\xrightarrow{\sim}\CC$.
\end{abstract}
\section{Introduction}

The goal of this work is to prove a new case of local-global compatibility for automorphic Galois representations, which we will introduce in a general sense below.

Let $F$ be a number field. Let $\pi=\otimes'_v\pi_v$ be an algebraic cuspidal automorphic representation of $\GL_n(\AAA_F)$, where $\pi_v$ is an irreducible representation of $\GL_n(F_v)$ for each place $v$ in $F$. Let $l$ be a rational prime and fix an isomorphism $\iota:\overline{\QQ}_l\xrightarrow{\sim}\CC$. For each $v$ the local Langlands correspondence gives a bijection
$$\rec_{F_v}:\Irr(\GL_n(F_v))\to\WDrep_n(W_{F_v}),$$
normalized as in \cite{HT01}, where $\Irr(\GL_n(F_v))$ is the set of isomorphism classes of irreducible smooth admissible representations of $\GL_n(F_v)$ over $\CC$, and $\WDrep_n(W_{F_v})$ is the set of isomorphism classes of $n$-dimensional Frobenius semisimple Weil--Deligne representations of $W_{F_v}$ over $\CC$. Recall that a Weil--Deligne representation is a pair $(r,N)$, where $r$ is a representation of the Weil group $W_{F_v}$ of $F_v$ and $N$ is a nilpotent endomorphism of the representation space that intertwines with $r$. We can apply the local Langlands correspondence map to a twist of $\pi_v$ to get $$\rec_{F_v}(\pi_v\otimes|\det|^{\frac{1-n}{2}}),$$ 
a Weil--Deligne representation of $W_{F_v}$ over $\CC$.

Conjecturally there exists a continuous semisimple Galois representation $r_\iota(\pi):G_F\to \GL_n(\overline{\QQ}_l)$ attached to $\pi$ and $\iota$. This restricts to $G_{F_v}$, and if $v\nmid l$, by Grothendieck's $l$-adic monodromy theorem the restriction determines a Weil--Deligne represenation $\WD(r_\iota(\pi)|_{G_{F_v}})$. We can take its Frobenius semisimplification (recall that the Frobenius semisimplification $(r,N)^{\Fss}=(r^{\Ss},N)$) and then base change to $\CC$ to get $$\WD(r_\iota(\pi)|_{G_{F_v}})^{\Fss}\otimes_\iota\CC,$$
another Weil--Deligne representation of $W_{F_v}$ over $\CC$.

The local-global compatibility conjecture says that these two objects are isomorphic:
\begin{conj}[Local-global compatibility]
For all finite place $v$ of $F$, we have
$$\WD(r_\iota(\pi)|_{G_{F_v}})^{\Fss}\otimes_\iota\CC\cong\rec_{F_v}(\pi_v\otimes|\det|^{\frac{1-n}{2}}).$$
\end{conj}

\begin{rmk}
We only defined $\WD$ when $v\nmid l$. We call it the $p\ne l$ case, assuming that $p$ is the residue characteristic of $F_v$. We can also define a Weil--Deligne representation in the $p=l$ case.
\end{rmk}

From now on we will focus on the case where $F$ is a CM field and $p\ne l$. Fix a rational prime $l$ and an isomorphism $\iota: \overline{\QQ}_l\xrightarrow{\sim}\CC$ as above. Let $\pi$ be a regular algebraic cuspidal automorphic representation of $\GL_n(\AAA_F)$. If $\pi$ is polarizable, the result is well-known \cite{Car12}. For non-polarizable $\pi$, we mention the following two remarkable results. In \cite{HLTT16}, Harris--Lan--Taylor--Thorne showed the existence of $r_\iota(\pi)$. Moreover, if the other prime $p$ satisfies some unramified conditions, then local-global compatibility holds at any finite place $v\mid p$ in $F$. In \cite{Var14}, Varma showed that for any finite place $v\nmid l$, local-global compatibility holds up to semisimplification (recall that the semisimplification $(r,N)^{\Ss}=(r^{\Ss},0)$). The question is then how to deal with the monodromy.

The work of Allen--Newton \cite[Theorem~4.1]{AN19} addresses the question when $\pi$ is of weight $0$ in the $\GL_2$ case. They prove that there is set of primes $l$ of Dircichlet density one such that local-global compatibility holds for all finite places $v\nmid l$ in $F$. Their proof uses the Fontaine--Laffaille automorphy lifting theorem \cite[Theorem~6.1.1]{ACC+18}, which requires $\pi$ to be of weight $0$. However, they mentioned that the weight $0$ condition might be removed if we use the ordinary automorphy lifting theorem \cite[Theorem~6.1.2]{ACC+18} instead, at the cost of imposing some ordinary assumptions on $\pi$ at $l$. We will investigate it in this paper. We have the following result.

\begin{theorem}
Let $F$ be a CM field and let $\pi$ be a regular algebraic cuspidal automorphic representation of $\GL_2(\AAA_F)$. Suppose that 
\begin{enumerate}
    \item $l\ge5$, lies under no prime at which $\pi$ is ramified,
    \item $\pi$ is $\iota$-ordinary for some $\iota:\overline{\QQ}_l\cong\CC$,
    \item $\overline{r}_{\iota}(\pi)$ is decomposed generic, $\overline{r}_{\iota}(\pi)(G_{F(\zeta_l)})$ is enormous, and there is a $\sigma\in G_F-G_{F(\zeta_l)}$ such that $\overline{r}_{\iota}(\pi)(\sigma)$ is a scalar.
\end{enumerate}
Then for any finite place $v\nmid l$ in $F$, we have
$$\WD(r_{\iota}(\pi))|_{G_{F_v}})^{\Fss}\otimes_\iota\CC\cong\rec_{F_v}(\pi_v\otimes|\det|^{-1/2}).$$
\end{theorem}

The idea of proof is similar to that of \cite[Theorem~4.1]{AN19}. By the main result of \cite{Var14}, it suffices to show that when $\pi_v$ is special (i.e. a twist of Steinberg), $r_{\iota}(\pi)|_{G_{F_v}}$ has nontrivial monodromy. Assume the monodromy is trivial. After a base change we may assume that $\pi_v$ is an unramified twist of Steinberg. We then want to find an automorphic representation $\pi_1$ that is unramified at $v$, such that $\overline{r}_{\iota}(\pi_1)\cong\overline{r}_{\iota}(\pi)$ after some restriction. This step is done by a potential automorphy theorem \cite[Theorem~3.9]{AN19}. Then the automorphy lifting theorem will give an automorphic representation $\Pi$ with $r_{\iota}(\Pi)\cong r_{\iota}(\pi)$ after some restriction, from which we can deduce a contradiction with the unramified local-global compatibility known from \cite{HLTT16,Var14}.

However, since we use the ordinary automorphy lifting theorem instead of the Fontaine--Laffaille one, there are some major differences. First, the ordinary automorphy lifting theorem does not specify the weights of the input automorphic representations $\pi$ and $\pi_1$, so we are able to handle regular algebraic cuspidal automorphic representations of arbitrary weight. Also, we need to prove an ordinary version of potential automorphy theorem (Theorem \ref{g3.9} proved below) to construct the automorphic representation $\pi_1$ which in some sense inherits ordinarity. More precisely, we impose an ordinary condition (Definition \ref{GoodOrdinary} below) on the Galois representation and we show that the automorphic representation we construct is $\iota$-ordinary. Finally, since we are not in the Fontaine--Laffaille case, we allow the prime $l$ to ramify in $F$.

\hspace*{\fill} \\
\textbf{Notations.} Let $F$ be a CM field. Let $l$ be an odd prime and let $v$ be a prime of $F$ over $l$. Let $k/\FF_l$ be a finite field. Let $\calO$ be a discrete valuation ring finite over $W(k)$ with residue field $k$. Let $\gothl$ be a uniformiser in $\calO$. Let $\epsilon_l$ be the $l$-adic cyclotomic character. We normalize our Hodge--Tate weights such that $\epsilon_l$ has all labelled Hodge--Tate weights equal to $-1$.

Let $M$ be a totally real field. We recall the definition of $M$-Hilbert--Blumenthal abelian varieties ($M$-HBAV in short). Let $S$ be a scheme and let $A$ be an abelian scheme over $S$ equipped with a given embedding of rings (real multiplication) $\iota:\calO_M\hookrightarrow\End(A/S)$. Let $\calM_A$ be the polarization module of $A$, i.e., the module of  $\calO_M$-linear symmetric homomorphisms $A\to A^{\vee}$, and let $\calM_A^+$ be the positive cone of polarizations. An $M$-HBAV is an abelian scheme with real multiplication $(A,\iota)$ such that the natural map $A\otimes_{\calO_M}\calM_A\to A^{\vee}$ is an isomorphism. Let $\gothc$ be a nonzero fractional ideal of $M$ and let $\gothc^+$ be the totally positive elements in $\gothc$. A $\gothc$-polarization of $A$ is an isomorphism $j:\gothc\xrightarrow{\sim}\calM_A$ of $\calO_M$-modules such that $j(\gothc^+)=\calM_A^+$.

Finally we recall the definition of a representation being ordinary of some weight. Let $\ZZ^n_+$ be the set of $n$-tuples with coordinates in decreasing order. Take a continuous representation $\rho:G_F\to\GL_n(\overline{\QQ}_l)$ and let $\lambda\in(\ZZ^n_+)^{\Hom(F,\overline{\QQ}_l)}$. We say that $\rho$ is ordinary of weight $\lambda$ if for each $v\mid l$ in $F$
$$\rho|_{G_{F_v}}\cong\left(\begin{array}{cccc}
    \psi_1 & * & * & * \\
    0 & \psi_2 & * & * \\
    \vdots & \ddots & \ddots & * \\
    0 & \cdots & 0 & \psi_n
\end{array}\right)$$
where for each $1\le i\le n$, $\psi_i:G_{F_v}\to \overline{\QQ}_l^\times$ is a continuous character such that $$\psi_i(\Art_{F_v}(\sigma))=\prod_{\tau:F_v\hookrightarrow\overline{\QQ}_l}\tau(\sigma)^{-(\lambda_{\tau,n-i+1}+i-1)}$$ for all $\sigma$ in some open subgroup of $\calO_{F_v}^\times$.

\section{An Important Lemma}
In this section we prove a lemma constructing certain $\gothl$-divisible groups that is important for our potential automorphy result (Theorem \ref{g3.9}). We first make the following definition to simplify our statement. 

\begin{defn}\label{GoodOrdinary}
Let $\overline{r}:G_{F_v}\to\GL_2(k)$ be a continuous representation. We say $\overline{r}$ is \textbf{good ordinary} if $$\overline{r}\cong\left(\begin{array}{cc}
    \overline{\psi}\overline{\epsilon}_l & \overline{c} \\
    0 & \overline{\psi}^{-1}
\end{array}\right)$$ for some unramified character $\overline{\psi}$ and extension class $\overline{c}$, such that either $\overline{\psi}^2\ne1$ or $\overline{\psi}^2=1$ and $\overline{c}$ is peu ramifi\'e.
\end{defn}

Now we state our important lemma, which can be viewed as an ordinary version of \cite[Lemma~3.8]{AN19}. Since we allow $l$ to ramify in $F$, we use Breuil--Kisin theory instead of Fontaine--Laffaille theory.

\begin{lemma}\label{g3.8}
Let $\overline{r}:G_{F_v}\to\GL_2(k)$ be a continuous representation such that:
\begin{itemize}
    \item $\det\overline{r}=\overline{\epsilon}_l$,
    \item there is a crystalline lift $r:G_{F_v}\to\GL_2(\calO)$ with labelled Hodge--Tate weights all equal to \{$-1,0$\}.
\end{itemize}
Let $V_{\overline{r}}$ be the underlying \'etale $k$-vector space scheme over $F_v$ of $\overline{r}$. Let $\gothl$ be a uniformiser in $\calO$. Then there exists a divisible $\calO$-module $\calG$ over $\calO_{F_v}$ of height $2[\calO:\ZZ_l]$, equipped with an $\calO$-linear symmetric isomorphism $\lambda:\calG\cong\calG^\vee\otimes\calD_{\calO/\ZZ_l}$, and an isomorphism $i:V_{\overline{r}}\cong\calG[\gothl]_{F_v}$ such that $i^\vee\circ\lambda[\gothl]_{F_v}\circ i$ is the isomorphism induced by the standard symplectic pairing on $V_{\overline{r}}$.

Moreover, if $\overline{r}$ is good ordinary, then $\calG$ is ordinary.
\end{lemma}

\begin{proof}

\textbf{Step 1: Reduction.} By assumption we have $\overline{r}:G_{F_v}\to\GL_2(k)$ such that  $\det\overline{r}=\overline{\epsilon}_l$, with a crystalline lift $r:G_{F_v}\to\GL_2(\calO)$ with all labelled Hodge--Tate weights in $\{-1,0\}$, so $\det r|_{I_{F_v}}=\epsilon_l$. Then there is an unramified character $\psi:G_{F_v}\to\calO^\times$ given by $\psi^2=(\det r)^{-1}\epsilon_l$, such that $r':=r\otimes\psi$ has $\det r'=\epsilon_l$ and it is another crystalline lift of $\overline{r}$ satisfying the assumptions. Thus we may assume without loss of generality that $\det r=\epsilon_l$.

\hspace*{\fill} \\
\textbf{Step 2: Construction.} The existence of the $l$-divisible group $\calG$ over $\calO_{F_v}$ follows from \cite[Corollary~6.2.3]{SW13}. For a constructive proof we use Breuil--Kisin theory and refer to \cite[Lemma~2.1.15 and Corollary~2.2.6]{Kis06}. Also see \cite[Lemma~11.2.10 and Theorem~12.3.2]{BC09}. From the construction \cite[Corollary~2.2.6]{Kis06} or \cite[Theorem~12.3.2]{BC09} we know that there is a natural isomorphism of lattices $r\cong T_l(\calG)$.

\hspace*{\fill} \\
\textbf{Step 3: Properties.} We finally check that $\calG$ meets all the conditions we demand. First of all, there is an obvious $\calO$-action on the representation space $\calO^2$ and by functoriality this gives the $\calO$-action on $\calG$; in other words, we get $$\calO\hookrightarrow\End\calG.$$ The height of $\calG$ is equal to the $\ZZ_l$-rank of $T_l(\calG)$, which is $2[\calO:\ZZ_l]$. 
Next, we compute that $$V_{\overline{r}}=V_{r\text{ mod }\gothl}\cong T_l(\calG)\text{ mod }\gothl=\calG[\gothl]_{F_v}(\overline{F}_v).$$
It remains to produce $\lambda$. Take $T=\calO^2$ with $G_{F_v}$-action given by $r$. We have the (Galois equivariant) standard symplectic pairing $\langle\cdot,\cdot\rangle:\calO^2\times\calO^2\to\calO(1)$, so $T\cong\Hom_\calO(T,\calO(1))$. The trace pairing from $\calO$ to $\ZZ_l$ gives $\Hom_\calO(T,\calO(1))\otimes\calD^{-1}_{\calO/\ZZ_l}\cong\Hom_{\ZZ_l}(T,\ZZ_l(1))$, and hence there is an isomorphism of Tate modules $$T_l(\calG)\cong T_l(\calG^\vee)\otimes\calD_{\calO/\ZZ_l}\cong T_l(\calG^\vee\otimes\calD_{\calO/\ZZ_l}).$$ 
Hence by the main result of \cite{Tat67} we get an isomorphism of $l$-divisible groups $\calG\cong\calG^\vee\otimes\calD_{\calO/\ZZ_l}$. The compatiblility with the standard symplectic pairing on $V_{\overline{r}}$ follows by construction.

Now we let $\overline{r}$ be good ordinary. We first point out that it is a stronger assumption: obviously $\det(\overline{r})=\overline{\epsilon}_l$, and \cite[Lemma~6.1.6]{BLGG12} implies that there is a potentially crystalline lift of $\overline{r}$ with labelled Hodge--Tate weights all equal to $\{-1,0\}$. Note that we normalize the representation differently, but the lemma still holds since the proof only cares about the quotient of the diagonal characters. Moreover, it follows from the proof that the lift could be made crystalline since we further assume that $\overline{\psi}$ is unramified. 

Therefore we get $\calG$ and it remains to show that it is ordinary. We know that $r\cong T_l(\calG)$ is ordinary, and by the matrix of $\overline{r}$ we know $r|_{I_{F_v}}$ has a subrepresentation $\epsilon_l\otimes_\ZZ\calO$. Note that $T_l(\mu_{l^\infty})\cong\epsilon_l$. By the main result of \cite{Tat67} we get an embedding $\mu_{l^\infty}\otimes_\ZZ\calO\hookrightarrow\calG_{\calO_{F_v}^\text{nr}}$.
\end{proof}

\section{Potential Automorphy}
We first state an ordinary version of \cite[Proposition~3.6]{AN19}. 

\begin{prop}\label{g3.6}
Let $k$ be an algebraically closed field of characteristic $l$. Let $M$ be a totally real field and let $\gothl$ be a prime in $M$ over $l$. Suppose that $(\calG,\lambda)$ is a divisible $\calO_{M,\gothl}$-module over $k$ of height $2[M_\gothl:\QQ_l]$ equipped with an $\calO_{M,\gothl}$-linear symmetric isomorphism $\lambda:\calG\cong\calG^\vee$.

Then there exists an $M$-HBAV $(A,\iota,j)$ over $k$ with $\calD_M^{-1}$-polarization and an isomorphism $i:A[\gothl^\infty]\cong\calG$ compatible with $\calO_{M,\gothl}$-actions on both sides such that $i^\vee\circ\lambda\circ i=j(1)$.

Moreover, if $\calG$ is ordinary, then $A$ is ordinary.
\end{prop}

\begin{proof}
Only the ordinary statement in the end is not done in \cite[Proposition~3.6]{AN19}, so we investigate it here. In the proof of \cite[Proposition~3.6]{AN19} they obtain a $\calD_M^{-1}$-polarized $M$-HBAV $A_0$ such that $A_0[\gothl^\infty]\cong\calG$, and in order to apply Dieudonn\'e theory to $A_0[l^\infty]$, they set $\calG_l=\calG\times\prod_{\gothl'\ne\gothl, \gothl'\mid l}A_0[\gothl'^\infty]$. Now since $\calG$ is ordinary, we have that $A_0[\gothl^\infty]$ is ordinary, and we demand $A_0[\gothl'^\infty]$ to be ordinary for all $\gothl'\mid l$, $\gothl'\ne\gothl$. This will yield an ordinary $M$-HBAV $A$ with desired properties.
\end{proof}

We then state our potential automorphy theorem, which is an ordinary version of \cite[Theorem~3.9]{AN19}.

\begin{theorem}\label{g3.9}
Suppose that $F$ is a CM field, $l$ is an odd prime and $k/\FF_l$ finite. Let $\calO$ be a discrete valuation ring finite over $W(k)$ with residue field $k$. Let $\overline{\rho}:G_F\to\GL_2(k)$ be a continuous absolutely irreducible representation such that 
\begin{itemize}
    \item $\det(\overline{\rho})=\overline{\epsilon}_l^{-1}$;
    \item for each $v\mid l$, $\overline{\rho}|_{G_{F_v}}$ admits a crystalline lift $\rho_v:G_{F_v}\to\GL_2(\calO)$ with all labelled Hodge–Tate weights equal to $\{0, 1\}$.
\end{itemize}

Suppose moreover that $F^\text{avoid}/F$ is a finite extension. Then we can find
\begin{itemize}
    \item a finite CM extension $F_1/F$ that is linearly disjoint from $F^\text{avoid}$ over $F$, such that if $v\mid l$ in $F$, then $v$ is unramified in $F_1$;
    \item a regular algebraic cuspidal automorphic representation $\pi$ for $GL_2(\AAA_{F_1})$ of weight $0$, unramified at places above $l$;
    \item an isomorphism $\iota:\overline{\QQ}_l\xrightarrow{\sim}\CC$
    such that (composing $\overline{\rho}$ with some embedding $k\hookrightarrow\overline{\FF}_l$)
    $$\overline{r}_\iota(\pi)\cong\overline{\rho}|_{G_{F_1}}.$$
\end{itemize}

Moreover, we can ensure the following:
\begin{enumerate}
    \item If $\overline{v}_0\nmid l$ is a finite place of $F^+$, then we can moreover find $F_1$ and $\pi$ as above with $\pi$ unramified above $\overline{v}_0$.
    \item If for each $v\mid l$, $\overline{\rho}^\vee|_{G_{F_v}}$ is good ordinary, then $\pi$ is furthermore $\iota$-ordinary.
\end{enumerate}

\end{theorem}

\begin{proof}
Choose a totally real field $M$ in which $l$ is unramified and a prime $\gothl$ of $M$ over $l$, such that $\calO_{M,\gothl}\cong\calO$ and the residue field $\calO_{M,\gothl}/\gothl\calO_{M,\gothl}=k_\gothl$ is isomorphic to $k$. Fix an isomorphism $k\cong k_\gothl$ and view $\overline{\rho}$ as a representation over $k_\gothl$.

Choose a non-CM elliptic curve $E/\QQ$ with good reduction at $l$ and the rational prime $q$ under $\overline{v}_0$. Choose a rational prime $p\ne l$ such that
\begin{itemize}
    \item $p>5$ splits completely in $FM$,
    \item $\SL_2(\FF_p)\subset\overline{r}_{E,p}(G_F)$, $E$ has good reduction at $p$ and $\overline{\rho}$ is unramified at places over $p$.
\end{itemize}
There are a positive density of primes satisfying the first condition by Chebotarev density theorem, and all but finitely many primes satisfy the second condition, so such $p$ exists. We fix a prime $\gothp$ of $M$ over $p$.

Let $V_{\overline{\rho}}^\vee$ be the $k_\gothl$-vector space scheme over $F$ underlying $\overline{\rho}^\vee$ and fix the standard symplectic pairing on it, which by assumption is preserved by $\overline{\rho}^\vee$ up to multiplier $\overline{\epsilon}_l$. Also we have the $k_\gothp$-vector space scheme $E[p]\cong(E\otimes_\ZZ\calO_M)[\gothp]$ over $F$, equipped with the Weil pairing.

Let $\calD^{-1}=\calD^{-1}_M$ be the inverse different of $M$. We consider the scheme $Y$ over $F$ classifying tuples $(A,j,\alpha_{\overline{\rho}},\alpha_E)$ where
\begin{itemize}
    \item $A$ is an $M$-HBAV with $\calD^{-1}$-polarization $j$,
    \item $\alpha_{\overline{\rho}}:A[\gothl]\to V_{\overline{\rho}}^\vee$ and $\alpha_E:A[\gothp]\to E[p]$ are isomorphisms of vector space schemes compatible with the fixed symplectic pairings on the right hand sides and with the pairings $A[\gothl]\times A[\gothl]\to\calO_M/\gothl(1)$ and $A[\gothp]\times A[\gothp]\to\calO_M/\gothp(1)$ on the left hand sides.
\end{itemize}
We know by \cite{Tay06} that $Y/F$ is a smooth and geometrically connected variety. Let $X=\Res_{F/F^+}Y$. Then $X/F^+$ is also smooth and geometrically connected.

Applying \cite[Theorem~7.2.4]{ACC+18} (Taylor's potential modularity of elliptic curves) with $\calL=\{l,p\}$ and $L_1^\text{avoid}$ the normal closure of $F^\text{avoid}\overline{F}^{\ker(\overline{\rho}\times\overline{r}_{E,p})}$ over $\QQ$, we get a finite Galois extension $L_2^\text{avoid}/\QQ$ linearly disjoint from $L_1^\text{avoid}$ over $\QQ$ and a finite totally real Galois extension $L^\text{suffices}/\QQ$ unramified above $p$ and $l$ and linearly disjoint from $L_1^\text{avoid}L_2^\text{avoid}$ over $\QQ$, such that for any finite totally real extension $L_2/L^\text{suffices}$ which is linearly disjoint from $L_2^\text{avoid}$ over $\QQ$, there is a regular algebraic cuspidal automorphic representation $\sigma$ of $\GL_2(\AAA_{L_2})$ of weight $0$ such that for any rational prime $p'$ and any isomorphism $\iota_{p'}:\overline{\QQ}_{p'}\cong\CC$, we have $r_{\iota_{p'}}(\sigma)\cong r_{E,p'}^\vee|_{G_{L_2}}$. 

We then apply a theorem of Moret-Bailly in the form stated in \cite[Theorem~3.1]{AN19} to $X$ with
\begin{itemize}
    \item $L=F^+$, $S_1=\{\overline{v}\mid lp\}$, $S_2=\{\overline{v}_0\}$, $L^\text{avoid}=L_1^\text{avoid}L_2^\text{avoid}L^\text{suffices}$,
    \item for $\overline{v}\mid lp$, $\Omega_{\overline{v}}\subset X((F^+_{\overline{v}})^\text{nr})=\prod_{v\mid\overline{v}}Y(F_v^\text{nr})$ is the subset given by abelian varieties $A$ with good reduction at $v$, and furthermore with ordinary reduction at $v$ if $\overline{\rho}^\vee|_{G_{F_v}}$ is good ordinary,
    \item $\Omega_{\overline{v}_0}\subset X(\overline{F^+_{\overline{v}_0}})=\prod_{v_0\mid\overline{v}_0}Y(\overline{F}_{v_0})$ is the subset given by abelian varieties $A$ with good reduction at $v_0$.
\end{itemize}

We need to check that all the assumptions of Moret--Bailly. Real places case is trivial since for each real place $\overline{v}$ of $F^+$, $v$ is the unique complex place of $F$ extending $\overline{v}$, and $X(F^+_{\overline{v}})=Y(F_v)=Y(\CC)$ is clearly non-empty.

For $v\mid p$ in $F$, the subset $\Omega_{\overline{v}}$ is open because having good reduction at $v$ is an open condition and it is Galois invariant because the Galois conjugate of an abelian variety with good reduction at $v$ still has good reduction at $v$, so we only need to show that the subset is nonempty. The two representations $\overline{\rho}$ and $\overline{r}_{E,l}$ over $k_\gothl$ which are unramified at $v$ can be trivialized by passing to some power. More precisely, we choose a positive integer $f$ such that $\overline{\rho}(\Frob_v)^{-f}$ and $\overline{r}_{E,l}(\Frob_v)^f$ are trivial. Now let $A$ be the base change of $E\otimes\calO_M$ to the unramified degree $f$ extension of $F_v$, and let $j$ be induced by the Weil pairing on $E$. By construction of $A$ we know that $j$ is a $\calD^{-1}$-polarization. We have isomorphisms over this extension of $F_v$, $\alpha_{\overline{\rho}}:A[\gothl]\to V_{\overline{\rho}}^\vee$ since both representations are trivial and $\alpha_E:A[\gothp]\to E[p]$ since $p$ splits completely in $M$. It is easy to check compatibility with pairings on both sides. A similar argument shows that $\Omega_{\overline{v}_0}$ is nonempty.

The only thing new here is to check that $\Omega_{\overline{v}}$ is nonempty when $v\mid l$. Applying Lemma \ref{g3.8} to $\overline{\rho}^\vee|_{G_{F_v}}$, we get a divisible $\calO$-module $\calG$ over $\calO_{F_v}$ with an isomorphism of $l$-divisible groups $\lambda:\calG\cong\calG^\vee$ ($\calD_{\calO/\ZZ_l}=\calD_{\calO_{M,\gothl}/\ZZ_l}$ is trivial since $l$ is unramified in $M$) and an isomorphism of finite flat group schemes $i:V_{\overline{\rho}}^\vee\cong\calG[\gothl]_{F_v}$ such that $\lambda$ induces the standard symplectic pairing on $V_{\overline{\rho}}^\vee$. Moreover, $\calG$ is ordinary if $\overline{\rho}^\vee|_{G_{F_v}}$ is good ordinary. Now consider an integral model $\calY/\calO_{F_v}$ of $Y_{F_v}$ classifying tuples $(A,j,\alpha_{\overline{\rho}},\alpha_E)$, where 
\begin{itemize}
    \item $A/S$ ($S$ an $\calO_{F_v}$-scheme)is an $M$-HBAV with $\calD^{-1}$-polarization $j$,
    \item $\alpha_{\overline{\rho}}:A[\gothl]\to\calG[\gothl]$ is an isomorphism of vector space schemes, compatible with the isomorphisms $A[\gothl]\cong A[\gothl]^\vee\otimes\calD_{\calO/\ZZ_l}$ induced by $j$ and $\calG[\gothl]\cong\calG[\gothl]^\vee$ induced by $\lambda$,
    \item $\alpha_E:A[\gothp]\to E[p]$ is an isomorphism of vector space schemes, compatible with the isomorphisms $A[\gothp]\cong A[\gothp]^\vee$ induced by $j$ ($\calD_{\calO_{M,\gothp}/\ZZ_p}$ is trivial since $p$ splits in $M$) and $E[p]\cong E[p]^\vee$ ($E$ has good reduction at $l$ so $E[p]$ extends to a vector space scheme over $\calO_{F_v}$ with the above canonical isomorphism).
\end{itemize}
We need to show that $\calY(\calO_{F_v}^\text{nr})$ is nonempty. By Greenberg's approximation theorem, it suffices to show that $\calY(\check{\calO})$, where $\check{\calO}$ is the $l$-adic completion of $\calO_{F_v}^\text{nr}$. Applying Proposition \ref{g3.6} to $\calG_{k_v}$, where $k_v=\calO_{F_v}^\text{nr}/v$ is an algebraically closed field, we get an $M$-HBAV $A_1/k_v$ with $\calD^{-1}$-polarization $j_1$ such that $A_1[\gothl^\infty]\cong\calG_{k_v}$ and $j_1(1)$ coincides with $\lambda_{k_v}:\calG_{k_v}\cong\calG_{k_v}^\vee$ under this isomorphism. Moreover, $A_1$ is ordinary if $\calG_{k_v}$ is ordinary. By Serre--Tate deformation theory, (since naturally we can define $\calG$ over $\check{\calO}$) we can lift $A_1$ to $\widetilde{A}_1$ over $\check{\calO}$ with $\calD^{-1}$-polarization $\widetilde{j}_1$ such that $\widetilde{A}_1[\gothl^\infty]\cong\calG_{\check{\calO}}$ and $\widetilde{j}_1(1)$ coincides with $\lambda_{\check{\calO}}:\calG_{\check{\calO}}\cong\calG_{\check{\calO}}^\vee$ under this isomorphism. By taking $\gothl$-torsion we get $\alpha_{\overline{\rho}}: \widetilde{A}_1[\gothl]\cong\calG_{\check{\calO}}[\gothl]$ and this isomorphism is compatible with the induced pairings on both sides. Finally we let $\alpha_E:\widetilde{A}_1[\gothp]\cong E[p]$ be the isomorphism between two trivial vector space schemes compatible with the pairings on both sides.

Now we checked the assumptions of Morel--Bailly, and hence obtain a finite Galois totally real extension $F_0^+/F^+$ and a point $(A,j,\alpha_{\overline{\rho}},\alpha_E)$ of $X(F_0^+)$ such that
\begin{itemize}
    \item $F_0^+/F^+$ is linearly disjoint from $L_1^\text{avoid}L_2^\text{avoid}L^\text{suffices}$ over $F^+$,
    \item $l$ and $p$ are unramified in $F_0^+$,
    \item $A$ has good reduction above $\overline{v}_0lp$,
    \item if for each $v\mid l$, $\overline{\rho}^\vee|_{G_{F_v}}$ is good ordinary, then $A$ has ordinary reduction at $v$.
\end{itemize}

Set $F_1=F_0^+L^\text{suffices}F$. Then $F_1/F$ is a CM extension unramified above $p$ and $v\mid l$. By \cite[Theorem~3.2]{AN19}, since $F_1^+/L^\text{suffices}$ is a finite totally real extension that is linearly disjoint from $L_2^\text{avoid}$ over $\QQ$, there exists a regular algebraic conjugate self-dual cuspidal automorphic representation $\sigma$ of $\GL_2(\AAA_{F_1})$ of weight 0 (initially of $\GL_2(\AAA_{F_1^+})$, and it extends by solvable base change), such that $\overline{r}_{\iota_p}(\sigma)\cong\overline{r}_{E,p}^\vee|_{G_{F_1}}$ for any $\iota_p:\overline{\QQ}_p\cong\CC$. Moreover $\sigma$ is unramified above $\overline{v}_0lp$ (since $E$ has good reduction above $\overline{v}_0lp$). Since $F_1$ is linearly disjoint from $L_1^\text{avoid}$ over $F$, $\SL_2(\FF_p)\subset\overline{r}_{E,p}(G_{F_1})$ (since $p$ is chosen to satisfy $\SL_2(\FF_p)\subset\overline{r}_{E,p}(G_F)$).

Now we fix a choice of $\iota$ and apply the automorphy lifting theorem \cite[Theorem~6.1.1]{ACC+18} to $\rho_0=r_{A,\gothp}^\vee$ (so $\overline{\rho}_0^\vee=A[\gothp]=E[p]$). Then there is a regular algebraic cuspidal automorphic representation $\pi$ of $\GL_2(\AAA_{F_1})$ of weight 0, unramified above $\overline{v}_0lp$, such that $r_{A,\gothp}^\vee\cong r_\iota(\pi)$. Since the representations $\{r_{A,\gothq}\}_\gothq$ ($\gothq$ ranges over places of $M$) coming from an abelian variety form a compatible system, if we fix $\iota_l:\overline{\QQ}_l\cong\CC$ we have $r_{A,\gothl}^\vee\cong r_{\iota_l}(\pi)$, and hence $\overline{r}_{A,\gothl}^\vee\cong\overline{r}_{\iota_l}(\pi)$. This finishes the proof since we have $\alpha_{\overline{\rho}}:A[\gothl]\cong V_{\overline{\rho}}^\vee$.

Finally we use the Hecke eigenvalue description to show that the automorphic representation $\pi$ is $\iota$-ordinary: Say $A$ is ordinary at some $v\mid l$, then the linear coefficient of the characteristic polynomial of $r_{A,\gothp}(\Frob_v)$ is a unit. Since $r_{A,\gothp}^\vee\cong r_{\iota}(\pi)$, the linear coefficient of the characteristic polynomial of $r_{\iota}(\pi)(\Frob_v)$ is a unit.
\end{proof}


\section{Local-Global Compatibility}
We first prove the following property of $\iota$-ordinarity.

\begin{theorem}\label{g2.8}
Let $F$ be a CM field and let $\pi$ be a regular algebraic cuspidal automorphic representation of $\GL_2(\AAA_F)$ of weight $\iota\lambda$, where $\iota:\overline{\QQ}_l\to\CC$ is a fixed isomorphism. Assume that 
\begin{itemize}
    \item $\overline{r}_\iota(\pi)$ is absolutely irreducible and decomposed generic,
    \item $\pi$ is $\iota$-ordinary.
\end{itemize}
Then $r_\iota(\pi)$ is ordinary of weight $\lambda$.
\end{theorem}

\begin{proof}
Fix $w\mid l$. Replacing $F$ with a finite solvable extension in which $l$ splits, we may assume that $F=F^+F_0$ with $F^+\ne\QQ$ totally real and $F_0$ an imaginary quadratic field in which $l$ splits. By \cite[Theorem~5.5.1]{ACC+18}, we know that there exists a Hecke algebra-valued lift $\rho_\gothm$ of $\overline{r}_\iota(\pi)$ such that
\begin{itemize}
    \item For any $g\in G_{F_w}$, the characteristic polynomial of $\rho_\gothm(g)$ equals $(X-\chi_{\lambda,w,1}(g))(X-\chi_{\lambda,w,2}(g))$,
    \item For any $g_1, g_2\in G_{F_w}$, $(\rho_\gothm(g_1)-\chi_{\lambda,w,1}(g_1))(\rho_\gothm(g_2)-\chi_{\lambda,w,2}(g_2))=0$,
\end{itemize}
where $\chi_{\lambda,w,i}$ is a Hecke algebra-valued character such that 
$$\chi_{\lambda,w,i}(\Art_{F_w}(\sigma))=\prod_{\tau:F_w\hookrightarrow\overline{\QQ}_l}\tau(\sigma)^{-(\lambda_{\tau,n-i+1}+i-1)}\langle\diag(1,\ldots,\sigma,\ldots,1)\rangle$$
($\sigma$ in the $i$-th place) for $\sigma\in\calO_{F_w}^\times$. Passing to the Hecke eigenvalue on $\pi$ the image of the diamond operator $\langle\diag(1,\ldots,\sigma,\ldots,1)\rangle$ is of finite order, so we can choose an open subgroup of $\calO_{F_w}^\times$ such that it is trivialized. Therefore the $\overline{\QQ}_l$-valued representation $r_\iota(\pi)$ corresponding to $\rho_{\gothm}$ satisfies the analogous properties with the characters $\chi_{\lambda,w,i}$ replaced by $\overline{\QQ}_l$-valued characters $\psi_{\lambda,w,i}$, such that
$$\psi_{\lambda,w,i}(\Art_{F_w}(\sigma))=\prod_{\tau:F_w\hookrightarrow\overline{\QQ}_l}\tau(\sigma)^{-(\lambda_{\tau,n-i+1}+i-1)}$$ for all $\sigma$ in some open subgroup of $\calO_{F_w}^\times$.
By Lemma \ref{BrauerNesbitt} below we know that $r_\iota(\pi)|_{G_{F_w}}$ is conjugate to an upper triangular matrix with (ordered) diagonal entries $\psi_{\lambda,w,1},\psi_{\lambda,w,2}$. This implies the ordinarity of $r_\iota(\pi)$.
\end{proof}

\begin{lemma}\label{BrauerNesbitt}
Let $G$ be a group and let $K$ be an algebraically closed field of characteristic 0. Fix two characters $\chi_1,\chi_2$. Suppose that $\rho:G\to\GL_2(K)$ is a representation satisfying
\begin{itemize}
    \item For any $g\in G$, the characteristic polynomial of $\rho(g)$ equals $(X-\chi_1(g))(X-\chi_2(g))$,
    \item For any $g_1, g_2\in G$, $(\rho(g_1)-\chi_1(g_1))(\rho(g_2)-\chi_2(g_2))=0$.
\end{itemize}
Then $\rho$ is conjugate to $\left(\begin{array}{cc}
    \chi_1 & \kappa \\
    0 & \chi_2
    \end{array}\right)$ for some $\kappa$.
\end{lemma}

\begin{proof}
The semisimplification of $\rho$ has the same characteristic polynomial as the semisimple representation $\chi_1\oplus\chi_2$. By the Brauer-Nesbitt theorem they are isomorphic, so $\rho$ is conjugate to an upper triangular matrix with (unordered) diagonal entries $\chi_1,\chi_2$. To show that they are put in the right order, suppose that $\rho$ is conjugate to $\left(\begin{array}{cc}
    \chi_2 & \kappa \\
    0 & \chi_1
    \end{array}\right)$.
If $\chi_1=\chi_2$ or $\rho$ is split then we are done. If $\chi_1\ne\chi_2$ and $\rho$ is nonsplit, then $H=\image(\rho)$ is nonabelian (otherwise under some basis every element of $H$ is diagonal). We can find $g_1,g_2\in G$ such that $(\rho(g_1)-\chi_1(g_1))(\rho(g_2)-\chi_2(g_2))$ is nonzero, which contradicts the second condition. Indeed, we choose $g_1$ such that $\chi_1(g_1)=\chi_2(g_1)=1$ and $\kappa(g_1)\ne0$, and choose $g_2$ such that $\chi_1(g_2)\ne\chi_2(g_2)$. Such $g_1$ exists since otherwise the intersection of $H$ and the unipotent subgroup $U$ of the upper triangular Borel $B$ is trivial and hence $H$ embeds into the quotient $B/U$, which is abelian. We can verify that the top right entry of the matrix $(\rho(g_1)-\chi_1(g_1))(\rho(g_2)-\chi_2(g_2))$ is nonzero.
\end{proof}

Now we are ready to prove our main result.

\begin{proof}[Proof of Theorem 1.3]
Fix a prime $p\ne l$ for which $\overline{r}_{\iota}(\pi)$ is decomposed generic. By the main result of \cite{Var14}, it suffices to show that if $v\nmid l$ is a finite place at which $\pi$ is special (i.e., $\pi_v$ is a twist of Steinberg), then $\WD(r_{\iota}(\pi)|_{G_{F_v}})$ has nontrivial monodromy. Let $N$ be the monodromy operator. To show that $N$ is nontrivial it suffices to do so after restricting to a finite extension. In particular, we may go to a solvable base change that is disjoint from $\overline{F}^{\ker(\overline{r}_{\iota}(\pi))}$ in which $p$ splits, and assume that 
\begin{itemize}
    \item $\pi_v$ is an unramified twist of Steinberg,
    \item $\overline{r}_{\iota}(\pi)$ is unramified at $v$ and $v^c$.
\end{itemize}

Now assume that $N=0$. By the main result of \cite{Var14} we have that $r_{\iota}(\pi)|_{G_{F_v}}\cong\chi\oplus\chi\epsilon_l$ for some unramified character $\chi:G_{F_v}\to\overline{\QQ}_l^\times$. We then apply Theorem \ref{g3.9} with $\overline{\rho}=\overline{r}_{\iota}(\pi)$ and $F^\text{avoid}$ equal to the Galois closure of $\overline{F}^{\ker(\overline{r}_{\iota}(\pi))}(\zeta_l)/\QQ$, to get a CM Galois extension $F_1/F$ linearly disjoint from $F^\text{avoid}$ over $F$, such that $\overline{r}_{\iota}(\pi)|_{G_{F_1}}\cong\overline{r}_{\iota}(\pi_1)$ for some weight 0 automorphic representation $\pi_1$ that is unramified above $v$ and $l$ and is $\iota$-ordinary. 

We wish to apply the automorphy lifting theorem \cite[Theorem~6.1.2]{ACC+18}. To do so, we check the following assumptions:
\begin{enumerate}
    \item $r_{\iota}(\pi)|_{G_{F_1}}$ is unramified almost everywhere. We know that $r_{\iota}(\pi)$ is unramified almost everywhere by the main result of \cite{HLTT16} and so is the restriction.
    \item For any $w_1\mid l$ in $F_1$, $r_{\iota}(\pi)|_{G_{F_{w_1}}}$ is ordinary (of some weight $\lambda$ if $\pi$ is of weight $\iota\lambda$). This is by Theorem \ref{g2.8}.
    \item $\overline{r}_{\iota}(\pi)|_{G_{F_1}}$ is absolutely irreducible (encoded in the definition of enormous image) and decomposed generic (by a similar argument in \cite[Theorem~4.1]{AN19}). The image of $\overline{r}_{\iota}(\pi)|_{G_{F_1(\zeta_l)}}$ is enormous (by our choice of $F^\text{avoid}$).
    \item There exists $\sigma\in G_{F_1}-G_{F_1(\zeta_l)}$ such that $r_{\iota}(\pi)(\sigma)$ is a scalar. This is still by our choice of $F^\text{avoid}$.
    \item There exists a regular algebraic cuspidal automorphic representation $\pi_1$ of $\GL_2(\AAA_{F_1})$ and an isomorphism $\iota:\overline{\QQ}_l\cong\CC$ such that $\pi_1$ is $\iota$-ordinary and $\overline{r}_{\iota}(\pi_1)\cong \overline{r}_{\iota}(\pi)|_{G_{F_1}}$. This is by Theorem \ref{g3.9}.
\end{enumerate}

The automorphy lifting theorem gives an $\iota$-ordinary cuspidal automorphic representation $\Pi$ of $\GL_2(\AAA_{F_1})$ such that $r_{\iota}(\Pi)\cong r_{\iota}(\pi)|_{G_{F_1}}$ and $\Pi_{v_1}$ is unramified at all $v_1\mid v$ in $F_1$. Then for any $v_1\mid v$ in $F_1$, $r_{\iota}(\Pi)|_{G_{F_{1,v_1}}}\cong \chi|_{G_{F_{1,v_1}}}\oplus\chi|_{G_{F_{1,v_1}}}\epsilon_l$, and $\Pi_{v_1}$ is an unramified principal series. By local-global compatibility at unramified places \cite{HLTT16} and \cite[Theorem~1]{Var14}, this contradicts the genericity of $\Pi$.
\end{proof}


\bibliographystyle{amsalpha}
\bibliography{bibtex}

\end{document}