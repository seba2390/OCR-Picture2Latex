\resizebox{!}{5cm}{
\tdplotsetmaincoords{61}{105}
\begin{tikzpicture}[tdplot_main_coords,scale=1.0]
%\begin{tiny}
%\begin{small}
\def\v{2.0};
\coordinate (C0) at (0,0,0);
\coordinate (C1) at (\v,0,0);
\coordinate (C2) at (0,\v,0);
\coordinate (C3) at (-\v,0,0);
\coordinate (C4) at (0,-\v,0);
\coordinate (C5) at (0,0,-\v);
\coordinate (C6) at (0,0,\v);

\coordinate (C7) at (\v,\v,0);
\coordinate (C8) at (-\v,\v,0);
\coordinate (C9) at (-\v,-\v,0);
\coordinate (C10) at (\v,-\v,0);
\coordinate (C11) at (\v,0,\v);
\coordinate (C12) at (0,\v,\v);
\coordinate (C13) at (-\v,0,\v);
\coordinate (C14) at (0,-\v,\v);
\coordinate (C15) at (\v,0,-\v);
\coordinate (C16) at (0,\v,-\v);
\coordinate (C17) at (-\v,0,-\v);
\coordinate (C18) at (0,-\v,-\v);

\coordinate (C19) at (\v,\v,\v);
\coordinate (C20) at (-\v,\v,\v);
\coordinate (C21) at (-\v,-\v,\v);
\coordinate (C22) at (\v,-\v,\v);
\coordinate (C23) at (\v,\v,-\v);
\coordinate (C24) at (-\v,\v,-\v);
\coordinate (C25) at (-\v,-\v,-\v);
\coordinate (C26) at (\v,-\v,-\v);
%


%
\foreach \p in {0,1,...,26} {
	%\shadedraw (C\p) node (D\p) [circle,shade,fill=gray!20,draw=black,opacity=0.7,minimum size={8pt}]{\p};
	\draw (C\p) node[inner sep=1pt] (D\p){\p};
};

\draw[dashed,opacity=0.5] (D19)--(D12);
\draw[dashed,opacity=0.5] (D12)--(D20);

\draw[dashed,opacity=0.5] (D20)--(D13);
\draw[dashed,opacity=0.5] (D13)--(D21);
\draw[dashed,opacity=0.5] (D21)--(D14);
\draw[dashed,opacity=0.5] (D14)--(D22);

\draw[dashed,opacity=0.5] (D22)--(D19);

\draw[dashed,opacity=0.5] (D23)--(D24);
\draw[dashed,opacity=0.5] (D24)--(D25);
\draw[dashed,opacity=0.5] (D25)--(D26);
\draw[dashed,opacity=0.5] (D26)--(D15);
\draw[dashed,opacity=0.5] (D15)--(D23);

%
\draw[dashed,opacity=0.5] (D7)--(D8);
\draw[dashed,opacity=0.5] (D8)--(D9);
\draw[dashed,opacity=0.5] (D9)--(D4);
\draw[dashed,opacity=0.5] (D4)--(D10);
\draw[dashed,opacity=0.5] (D10)--(D7);
%
\draw[dashed,opacity=0.5] (D12)--(D6);
\draw[dashed,opacity=0.5] (D6)--(D14);

\draw[dashed,opacity=0.5] (D14)--(D4);
\draw[dashed,opacity=0.5] (D4)--(D18);

\draw[dashed,opacity=0.5] (D18)--(D5);
\draw[dashed,opacity=0.5] (D5)--(D16);

\draw[dashed,opacity=0.5] (D16)--(D12);
%
\draw[dashed,opacity=0.5] (D19)--(D23);
\draw[dashed,opacity=0.5] (D20)--(D8);
\draw[dashed,opacity=0.5] (D8)--(D24);
\draw[dashed,opacity=0.5] (D21)--(D25);
\draw[dashed,opacity=0.5] (D22)--(D10);
\draw[dashed,opacity=0.5] (D10)--(D26);

%
\draw[dashed,opacity=0.5] (D11)--(D15);
\draw[dashed,opacity=0.5] (D13)--(D17);
\draw[dashed,opacity=0.5] (D11)--(D6);
\draw[dashed,opacity=0.5] (D6)--(D13);

\draw[dashed,opacity=0.5] (D15)--(D5);
\draw[dashed,opacity=0.5] (D5)--(D17);

\def\mycolora{red}
\def\mycolorb{blue}
\def\mycolorc{orange}
\def\arrowcolors{\mycolora,\mycolora,\mycolora,\mycolora,\mycolora,\mycolora,\mycolorb,\mycolorb,\mycolorb,\mycolorb,\mycolorb,\mycolorb,\mycolorb,\mycolorb,\mycolorc,\mycolorc,\mycolorc,\mycolorc,\mycolorc,\mycolorc,\mycolorc,\mycolorc,\mycolorc,\mycolorc,\mycolorc,\mycolorc}
%\foreach \p in {1,2,...,26} {
\foreach \c [count = \i] in \arrowcolors{
	pgfmatheval{\i+1}
	\draw[-latex,\c] (D0)--(D\pgfmathresult);
};
%
%\draw (D0) node{0};
%\end{tiny}
%\end{small}
\coordinate (T) at (0,0,-4);
\draw (T) node{D3Q27};
\end{tikzpicture}
}
