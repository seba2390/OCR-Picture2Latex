% !TEX root = 21ITW-JEDI.tex
% DO NOT REMOVE THE ABOVE COMMENT!
\section{Prerequisites}

We now introduce the system model and MAP-JED optimization problem from which we derive a computationally efficient soft-output JED algorithm in \fref{sec:JED}.

\subsection{System Model}\label{sec:system model}
 
We focus on the uplink of a MU-MIMO communication system in which  $U$ single-antenna UEs transmit pilots and data to a BS equipped with $B$ antennas.
%
We assume a block-fading scenario with a coherence time of $K=T+D$ time slots; $T$ time slots are reserved for pilots and $D$ time slots are used for payload data.
%
The transmitted data matrix $\bS=[\bS_T,\bS_D]$ contains the pilots $\bS_T\in\complexset^{U\times T}$  and the transmit symbols $\bS_D\in\setQ^{U\times D}$ of all UEs, where $\setQ$ is the constellation set  (e.g., QPSK). 
%
In what follows, we consider a frequency-flat channel\footnote{Frequency selective channels can be transformed into parallel frequency-flat subcarriers by means of orthogonal frequency-division multiplexing.} with the following input-output relation~\cite{gesbert2003theory}:
\begin{align}\label{eq:mimo model}
\bY = \bH\bS + \bN.
\end{align} 
Here, $\bY \in \complexset^{B\times K}$ is the receive matrix containing all received symbols at the $B$ BS antennas over the $K$ time slots, $\bH\in\complexset^{B\times U}$ is the (unknown) channel matrix, and $\bN\in\complexset^{B\times K}$ models thermal noise with i.i.d.\ circularly-symmetric complex Gaussian entries and variance $\No$ per complex dimension.

%
%
%
\subsection{MAP-JED Optimization Problem}
 
We start by formulating the MAP-JED problem.
%
Our goal is to \emph{jointly} estimate the channel matrix and recover the most likely transmit symbols which requires us to assume priors for the channel matrix and the transmit symbols. For simplicity, we assume i.i.d.\ circularly-symmetric Gaussian entries in $\bH$ with entry-wise variance $E_h$ and equally likely transmit symbols.
%
These assumptions result in the MAP-JED problem
%
\begin{align}\label{eq:joint ML with H prior}
	\big\{\widehat \bH,\widehat\bS_D\big\} 
	& =  \argmin_{\substack{\bH\in\complexset^{B\times U}\\\bS_D\in\setQ^{U \times D}} } \normfro{\bY - \bH\bS}^2 + \lambda\normfro{\bH}^2
\end{align}
with $\lambda=\No/E_h$.\footnote{In \fref{sec:hypernetwork}, we let the hyper-network determine an appropriate choice of the parameter $\lambda^{(t)}$ for every iteration $t$, which enables us to apply our JED algorithm to channel matrices that are not necessarily i.i.d.\ Gaussian.}
%
For simplicity of exposition, we will directly work with the transmit symbol matrix $\bS$ instead of the pilot and data matrices $\bS_T$ and $\bS_D$, respectively. 

Since \fref{eq:joint ML with H prior} can be written as two nested optimization problems in the variables $\bS$ and $\bH$, we can first determine the optimal channel estimate $\widehat{\bH}$ given the transmit symbol matrix~$\bS$ and then find the optimal transmit symbol matrix $\widehat{\bS}$.
%
Since the optimization problem is quadratic in $\bH$, the optimal channel estimate $\widehat{\bH}$ has the following closed-form solution:
\begin{align} \label{eq:optchannelest}
    \widehat\bH =\bY\bS^H\bM^{-1},
\end{align}
where we use the auxiliary matrix $\bM = \bS\bS^H + \lambda\bI_{U}$.
%
We can now substitute $\widehat\bH$ into the objective of \fref{eq:joint ML with H prior} and perform algebraic simplifications, which leads to an equivalent MAP-JED problem that only depends on the transmit symbol matrix: 
%
\begin{align}\label{eq:opt1}
	\widehat{\bS} = \argmax\limits_{\bS\in\setQ^{U\times K}}\, \Tr\!\left[\bY^H\bY\bS^H\bM^{-1}\bS\right]\!.
\end{align}
%
After solving the MAP-JED problem in \fref{eq:opt1}, one can determine the optimal channel estimate by plugging $\widehat\bS$ into~\fref{eq:optchannelest}. 
%
Note that for $\lambda = 0$, the MAP-JED problem in  \fref{eq:opt1} reduces to the well-known maximum likelihood JED problem in \cite[Eq.~6]{xu2008exact}.