% !TEX root = 21ITW-JEDI.tex
% DO NOT REMOVE THE ABOVE COMMENT!
\section{Simulation Results}

We now demonstrate the efficacy our soft-output JED algorithm and compare it to baseline algorithms. We first detail the system setup and then show simulation results. 

\subsection{System Setup}\label{sec:simulation setup}

We simulate a  MU-MIMO system as described in \fref{sec:system model} with $B=8$ BS antennas and $U=4$ single-antenna UEs transmitting QPSK symbols for $K=244$ time slots. The UEs transmit orthogonal pilots in  $\bS_T$ from a $4\times4$ Hadamard matrix.
%
The channel matrices are modelled as Rayleigh fading with i.i.d. complex standard Gaussian entries.
%
We consider per-UE coding with a  rate-1/2 low-density parity-check (LDPC) code as in IEEE 802.11n~\cite{wlan80211nPHY} with a block-length of $480$ bits; for LDPC decoding, we use a sum-product and layered decoding algorithm with $10$ iterations.
%
The hyper-network is trained using an NVIDIA GTX1080 with $1\,$M transmissions and batch size of $1\,$k.
%
We use Monte-Carlo simulations to extract the coded packet error rate (PER), uncoded bit error rate (BER), and BCE as in \fref{eq:loss function}.
%
We run $T_\text{max}=10$ iterations of our soft-output JED algorithm (called ``S-JED''). 



\subsection{Baseline Algorithms}
In order to evaluate the effectiveness of our S-JED algorithm, we simulate the SIMO lower bound, which cancels  MU interference with perfect CSI in a genie-aided fashion~\cite{zhang2006non}.
%
We also compare our algorithm to conventional methods that separate channel estimation from soft-output data detection. 
%
For such methods, we simulate a SIMO lower bound with estimated CSI (called ``SIMO (est. CSI)''), where we use a least-squares channel estimator to compute~$\widehat{\bH}^\text{LS}$. 
%
We also compare S-JED to the widely used soft-output linear minimum mean-square error (L-MMSE) equalizer \cite{seethalerEfficient2004,fateh2009vlsi} and the max-log optimal single-tree-search sphere decoder (STS-SD)~\cite{studer08a}.
%
\subsection{Simulation Results}
\begin{figure*}[tp]
\centering
\subfigure[]{\includegraphics[width=0.31\textwidth]{figs/JEDI4PSK_PER_vs_SNR.pdf}}
\hspace{0.2cm}
\subfigure[]{\includegraphics[width=0.31\textwidth]{figs/JEDI4PSK_BER_vs_SNR.pdf}}
\hspace{0.2cm}
\subfigure[]{\includegraphics[width=0.31\textwidth]{figs/JEDI4PSK_BCE_vs_SNR.pdf}}\\
\caption{Coded PER (a), uncoded BER (b), and BCE (c) performance for a $B=8$ BS antenna, $U=4$ UE MU-MIMO system with  transmitting QPSK for $K = 240$ time slots.
%
The proposed soft-output JED (S-JED) algorithm approaches the SIMO lower bound and outperforms the SIMO bound with estimated CSI as well as the max-log optimal soft-output STS-SD and the widely used L-MMSE equalizer which separate channel estimation from  data detection.}
\label{fig:8x4 numerical results}
\vspace{-0.2cm}
\end{figure*}

Figure~\ref{fig:8x4 numerical results} shows our simulation results. 
%
In \fref{fig:8x4 numerical results}(a), we see that S-JED approaches the SIMO lower bound by less than $3\,$dB at a coded PER of $0.1$\% and outperforms the SIMO lower bound that uses estimated CSI. 
%
S-JED significantly outperforms the max-log optimal soft-output STS-SD algorithm and the widely used L-MMSE equalizer, which both separate channel estimation from detection, by $2$\,dB and $4$\,dB, respectively.
%
In \fref{fig:8x4 numerical results}(b), we see that the uncoded BER results behave similarly. 
%
The results in \fref{fig:8x4 numerical results}(c) demonstrate that the BCE accurately characterizes the performance of all methods, as the order between algorithms is preserved with respect to coded PER and uncoded BER---this implies that the BCE loss in \fref{eq:loss function} is well suited to train soft-output data detectors. 