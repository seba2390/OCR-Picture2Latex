
\documentclass[12pt]{article}

\usepackage{multirow}
\usepackage{color}



\usepackage{comment}

%% Please use the following statements for
%% managing the text and math fonts for your papers:
\usepackage{times}
%\usepackage[cmbold]{mathtime}
\usepackage{bm}

\usepackage{amsmath}
\usepackage{amssymb}
\usepackage{amsfonts}

\usepackage{amssymb}
\usepackage{amsmath,amsthm, mathtools,commath}
\usepackage{graphics, color}

\usepackage{graphicx}
\usepackage{epsfig}
\usepackage{makeidx}
%\usepackage{hangul}
%\makeindex

\hbadness=10000 \tolerance=10000 \hyphenation{en-vi-ron-ment
in-ven-tory e-num-er-ate char-ac-ter-is-tic}



\usepackage[round]{natbib}
%\bibliographystyle{apalike2}
% \bibliographystyle{jmr}


\newcommand{\biblist}{\begin{list}{}
{\listparindent 0.0cm \leftmargin 0.50cm \itemindent -0.50 cm
\labelwidth 0 cm \labelsep 0.50 cm
\usecounter{list}}\clubpenalty4000\widowpenalty4000}
\newcommand{\ebiblist}{\end{list}}

\newcounter{list}


%\usepackage{setspace}
\usepackage{latexsym}
\usepackage{amsmath, amssymb, amsfonts, amsthm, bbm}
\usepackage{graphicx}
\usepackage{mathrsfs}

%\usepackage{hangpar}
\newcommand{\lbl}[1]{\label{#1}{\ensuremath{^{\fbox{\tiny\upshape#1}}}}}
% remove % from next line for final copy
\renewcommand{\lbl}[1]{\label{#1}}

\newtheorem{theorem}{Theorem}
\newtheorem{corollary}{Corollary}
\newtheorem{definition}{Definition}
\newtheorem{example}{Example}
\newtheorem{remark}{Remark}
\newtheorem{result}{Result}
\newtheorem{lemma}{Lemma}

\DeclareMathOperator*{\argmin}{arg\,min}
\newcommand{\ba}{{a}}
\newcommand{\bA}{\mathbf{A}}
\newcommand{\br}{\mathbf{r}}
\newcommand{\bx}{{x}}
\newcommand{\be}{{e}}
\newcommand{\bb}{{b}}
\newcommand{\bd}{{d}}
\newcommand{\by}{{y}}
\newcommand{\bu}{{u}}
\newcommand{\bv}{{v}}
\newcommand{\bt}{\bm{t}}
\newcommand{\bs}{\bm{s}}
\newcommand{\bz}{\bm{z}}
\newcommand{\bh}{\bm{h}}
\newcommand{\bn}{\bm{n}}
\newcommand{\bq}{\bm{q}}
\newcommand{\bmm}{\bm{m}}
\newcommand{\bD}{\mathbf{D}}
\newcommand{\bM}{\mathbf{M}}
\newcommand{\bN}{\mathbf{N}}
\newcommand{\bI}{\mathbf{I}}
\newcommand{\bG}{\mathbf{G}}
\newcommand{\bO}{\mathbf{O}}
\newcommand{\bE}{\mathbf{E}}
\newcommand{\bB}{\mathbf{B}}
\newcommand{\bC}{\mathbf{C}}
\newcommand{\bK}{\mathbf{K}}
\newcommand{\bX}{\mathbf{X}}
\newcommand{\bY}{\mathbf{Y}}
\newcommand{\bU}{\mathbf{U}}
\newcommand{\bV}{\mathbf{V}}
\newcommand{\bW}{\mathbf{W}}
\newcommand{\bZ}{\mathbf{Z}}
\newcommand{\bR}{\mathbf{R}}
\newcommand{\bP}{\mathbf{P}}
\newcommand{\bQ}{\mathbf{Q}}
\newcommand{\bT}{\mathbf{T}}
\newcommand{\bw}{\mathbf{w}}
\newcommand{\bS}{\mathbf{S}}
\newcommand{\bH}{\mathbf{H}}
\newcommand{\biI}{\mathcal{I}}
\newcommand{\biN}{\mathcal{N}}
\newcommand{\biT}{\mathcal{T}}
\newcommand{\bphi}{\boldsymbol{\phi}}
\newcommand{\bpi}{\boldsymbol{\pi}}
\newcommand{\bone}{\mathbf{1}}
\newcommand{\bbeta}{\boldsymbol{\beta}}
\newcommand{\bvar}{\boldsymbol{\varepsilon}}
\newcommand{\bdelta}{{\delta}}
\newcommand{\bepsilon}{{\epsilon}}
\newcommand{\btheta}{\boldsymbol{\theta}}
\newcommand{\btau}{\boldsymbol{\tau}}
\newcommand{\etab}{\boldsymbol{\eta}}
\newcommand{\bgamma}{\boldsymbol{\gamma}}
\newcommand{\blambda}{{\lambda}}
\newcommand{\bpsi}{\boldsymbol{\psi}}
\newcommand{\bmu}{\boldsymbol{\mu}}
\newcommand{\balpha}{\boldsymbol{\alpha}}
\newcommand{\bSigma}{\boldsymbol{\Sigma}}
\newcommand{\bGamma}{\boldsymbol{\Gamma}}
\newcommand{\bOmega}{{\Omega}}
\newcommand{\bDelta}{\boldsymbol{\Delta}}
\newcommand{\bPhi}{\boldsymbol{\Phi}}
\newcommand{\bPi}{\boldsymbol{\Pi}}
\newcommand{\bkappa}{\boldsymbol{\kappa}}
%\newcommand{\btheta}{\boldsymbol{\theta}}
\newcommand{\bzero}{\mathbf{0}}
\newcommand{\var}{\mathrm{var}}
\newcommand{\pr}{\mathrm{pr}}
\newcommand{\logit}{\mathrm{logit\,}}
\newcommand{\Var}{\mathrm{var}}
\newcommand{\Cov}{\mathrm{cov}}
\newcommand{\T}{\mathrm{T}}
\newcommand{\dps}{\displaystyle}
\def\E{E}
\newcommand{\xhi}{X_{hi}}
\newcommand{\yhi}{Y_{hi}}
\newcommand{\vhi}{v_{hi}}
\newcommand{\uhi}{u_{hi}}
\newcommand{\hrange}{{_{h=1}^{H}}}
\newcommand{\irange}{{_{i=1}^{m_h}}}
\newcommand{\widehatyhi}{\widehat{Y}_{hi}}
\newcommand{\hirange}{\hrange{_,}\irange}
\def\upstep#1{{^{(#1)}}}
\def\trans{^{\rm T}}
\usepackage{mathabx}
\def\wh{\widehat}
\def\wt{\widetilde}
\newcommand{\Norm}[1]{\left\Vert#1\right\Vert}
\newcommand{\Abs}[1]{\left\vert#1\right\vert}

\newcommand{\hf}[1]{\textcolor{blue}{\textbf{[Hengfang: #1]}}}



\newcommand{\jk}[1]{\textcolor{red}{\textbf{[JK: #1]}}}



\usepackage{natbib}

\providecommand{\keywords}[1]{\textbf{Key words:} #1}

\begin{document}


\baselineskip .3in


\title{Statistical Inference after Kernel Ridge Regression Imputation under item nonresponse}

\author{Hengfang Wang \and Jae Kwang Kim}

\date{} 

\maketitle

\begin{abstract}
Imputation is a popular technique for handling missing data. 
We consider a nonparametric approach to imputation using the kernel ridge regression technique and {propose consistent variance estimation}. The proposed variance estimator is based on a linearization approach which employs the entropy method to estimate the density ratio.
The $\sqrt{n}$-consistency of the imputation estimator is established when a Sobolev space is utilized in the kernel ridge regression imputation, which enables us to develop the proposed variance estimator.
Synthetic data experiments are presented to confirm our theory. 
\end{abstract}

\keywords{Reproducing kernel Hilbert space;  Missing data; Nonparametric method}



%\bibliographystyle{chicago}
%\bibliography{ref}

\newpage 

 
\section{Introduction}



Missing data is a universal problem in statistics.  Ignoring the cases with missing values  can lead to  misleading results \citep{kim2013statistical, little2019statistical}. To avoid the potential problem with missing data, imputation is commonly used.  
 After imputation, the imputed dataset can serve as a complete dataset that has no missing values, which in turn makes results from different analysis methods consistent. However, treating imputed data as if observed and applying the standard estimation procedure may result in misleading inference, leading to  underestimation of the variance  of  imputed point estimators.  As a result, how to make statistical {inferences} with  {imputed point estimators} is an important statistical problem. 
An overview of imputation method can be found in \citet{haziza2009imputation}. 

Multiple imputation,  proposed by \citet{rubin2004multiple}, addresses the uncertainty associated with imputation.  
However,  variance estimation using Rubin's formula  requires  certain conditions \citep{wang1998large,kim2006bias,yang2016note}, which do not necessarily hold in practice. An alternative method  is fractional imputation, originally proposed by \citet{kalton1984some}. The main idea of fractional imputation is to generate multiple imputed values and the corresponding fractional weights. 
%Hot deck imputation is a popular method of imputation  where the imputed values are taken  from  the observed values. In this vein, \citet{fay1996alternative,kim2004fractional,fuller2005hot,durrant2005imputation,durrant2006using} discussed fractional hot deck imputation.
In particular,
\citet{kim2011parametric} and \citet{kim2014fractional} employ fully parametric approach to handling nonresponse items with fractional imputation. However, such parametric fractional imputation relies heavily on the parametric model assumptions.  To mitigate the effects of parametric model assumption, empirical likelihood \citep{owen2001empirical,qin1994empirical} as a semiparametric approach was considered. In particular,  \citet{wang2009empirical} employed the kernel smoothing approach to do empirical likelihood inference with missing values. %\citet{chen2017semiparametric} extended \citet{muller2009estimating}'s work to develop fractional imputation with the first moment  assumption only.   
\citet{cheng1994nonparametric} utilized the kernel-based nonparametric regression approach to do the imputation and established the $\sqrt{n}$-consistency of the imputed estimator.

Kernel ridge regression \citep{friedman2001elements,shawe2004kernel} is a popular data-driven approach which can alleviate the effect of model assumption. {By using} a regularized \textit{M}-estimator {in reproducing} kernel Hilbert space (RKHS), kernel ridge regression can capture the model with {complex reproducing kernel Hilbert space} while a regularized term makes the original infinite dimensional estimation problem viable \citep{wahba1990spline}. \citet{geer2000empirical,mendelson2002geometric,zhang2005learning,koltchinskii2006local,steinwart2009optimal} studied the error bounds for the estimates of kernel ridge regression method. 
%Recently, \citet{zhang2013divide} employed truncation analysis to estimate the error bound in a distributed fashion. \citet{yang2017randomized} considered randomized sketches for KRR and studied projection dimension which can preserve minimax optimal approximations for KRR.

%\subsection{Our Work}

In this paper, we apply kernel ridge regression as a nonparametric imputation method and propose a consistent variance estimator for the corresponding imputation estimator under missing at random framework. Because the kernel ridge regression is a general tool for nonparametric regression with flexible assumptions, the proposed imputation method is practically useful. Variance estimation after the kernel ridge regression imputation is a challenging but important problem.   
To the best of our knowledge, this is the first paper which considers kernel ridge regression technique and discusses its variance estimation in the imputation framework.  Specifically, we first prove $\sqrt{n}$-consistency of the kernel ridge regression imputation estimator and obtain influence function for linearization. After that, we 
employ the maximum entropy method \citep{nguyen2010} for density ratio estimation to get a valid estimate of the inverse of the propensity scores. The consistency of our variance estimator can then  be established. 


The paper is organized as follows. In Section 2, the basic setup and  the proposed method are introduced. In Section 3, main theory is established. We also introduce a novel  nonparametric estimator of the propensity score function.  Results from two limited simulation studies are presented in Section 4. An illustration of the proposed method to a real data example is presented in Section 5. Some concluding remarks are made in Section 6. 


\section{Proposed Method}

Consider the problem of estimating $\theta=\mathbb{E}(Y)$ from an independent and identically distributed (IID) sample  $\{(\bx_i, y_i), i=1, \cdots, n\}$ of random {vector} $(X,Y)$. Instead of always {observing  $y_i$}, suppose that we observe $y_i$ only if $\delta_i=1$, where $\delta_i$ is the response indicator function of unit $i$ taking values on $\{0,1\}$.  The auxiliary variable $\bx_i$ are always observed. 
We assume that the response mechanism is missing at random (MAR)   in the sense of   \cite{rubin1976}. 
%$$ \delta \perp Y \mid \bx . $$

Under MAR, we can develop a nonparametric estimator $\wh{m} (\bx)$ of $m(\bx)=\mathbb{E}( Y \mid \bx)$ and construct the following imputation estimator: 
\begin{equation} 
\wh{\theta}_I = \frac{1}{n} \sum_{i=1}^n \left\{ \delta_i y_i + (1-\delta_i )\wh{m} (\bx_i) \right\}. 
\label{1} 
\end{equation} 
If $\wh{m} (\bx)$ is constructed by the kernel-based nonparametric regression method, we can express 
\begin{equation} 
 \wh{m} (\bx) = \frac{ \sum_{i=1}^n \delta_i K_h( \bx_i, \bx ) y_i}{ \sum_{i=1}^n \delta_i K_h( \bx_i, \bx )} 
\label{2}
\end{equation} 
where $K_h (\cdot)$ is the kernel function with bandwidth $h$. Under some suitable choice of the bandwidth $h$, \citet{cheng1994nonparametric} first established the $\sqrt{n}$-consistency of the imputation estimator (\ref{1}) with nonparametric  function in (\ref{2}).  However, the kernel-based regression imputation in (\ref{2}) is applicable only when the dimension of $x$ is small. 


%\textcolor{red}{Suppose we can express 
%$$ \hat{\theta}_I = \frac{1}{n} \sum_{i=1}^n \hat{m}(\bx_i) = \frac{1}{n} \sum_{i =1}^n\delta_i  \hat{g} (\bx_i) y_i , $$
%then 
% $\hat{g}(\bx)$ is a nonparametric estimator of $1/\pi(\bx)$. Thus, the linearization form of $\hat{\theta}_I$ is 
% $$ \tilde{\theta}_I = \frac{1}{n} \sum_{i=1}^n \{ \hat{m}_i+ \delta_i \hat{g}( \bx_i) (y_i - \hat{m}_i  ) \}$$
% where $\hat{m}_i = \hat{m} (\bx_i)$. The $\hat{g}( \bx)$ satisfies 
% $$ \sum_{i=1}^n \hat{m}_i = \sum_{i=1}^n\delta_i  \hat{g} (\bx_i) \hat{m}_i . $$
%}
%



In this paper, we extend the work of \citet{cheng1994nonparametric} by considering a more general type of the nonparametric imputation, called kernel ridge regression (KRR) imputation. The KKR technique can be understood using the reproducing kernel Hilbert space (RKHS) theory \citep{aronszajn1950theory} and can be described as 
\begin{equation} 
\wh{m} = \argmin_{m\in \mathcal{H}} \left[   \sum_{i=1}^{n} \delta_{i}\left\{ y_{i} - m(\bx_{i}) \right\}^{2} + \lambda \Norm{m}_{\mathcal{H}}^{2}   \right],
\label{3} 
\end{equation} 
where $ \norm{m}_{\mathcal{H}}^{2}  $ is the norm of $m$ in the Hilbert space $\mathcal{H}$. Here, the inner product $\langle \cdot, \cdot \rangle_{\mathcal{H}}$ is induced by {such} a kernel function, i.e., 
\begin{align}
\langle f, K(\cdot, \bx) \rangle_{\mathcal{H}} = f(\bx), \forall \bx \in \mathcal{X},  f\in \mathcal{H},
\end{align}
namely, the reproducing property of $\mathcal{H}$. Naturally, this  reproducing property implies the $\mathcal{H}$ norm of $f$: $\norm{f}_{\mathcal{H}} = \langle f, f  \rangle_{\mathcal{H}}^{1/2}$.

One 
canonical example of such a space is the Sobolev space. Specifically, assuming that  the domain of such functional space is $[0,1]$,
 the Sobolev space of order $l$ can be denoted as 
\begin{eqnarray}
	\mathcal{W}_{2}^{l} &=& \left\{ f:[0,1] \rightarrow \mathbb{R} |
	 f, f^{(1)}, \dots, f^{(l-1)} \mbox{ are absolute continuous and } f^{(l)} \in L^{2}[0,1]    \right\}. \notag 
\end{eqnarray}
One possible norm for this space can be
\begin{eqnarray}
	\norm{f}_{\mathcal{W}_{2}^{l}}^{2} = \sum_{q = 0}^{l-1}\left\{   \int_{0}^{1}f^{(q)}(t)dt      \right\}^{2} +
	 \int_{0}^{1}\left\{f^{(l)}(t) \right\}^{2}dt . \notag 
\end{eqnarray}
%Readers can refer to \cite{wahba1990spline} for a thorough treatment of the RKHS technique. 
In this section, we employ the Sobolev space of second order as the approximation space. 
For Sobolev space of order $\ell$, we have the kernel function
\begin{align}
K(x,y) = \sum_{q = 0}^{\ell-1}k_{q}(x)k_{q}(y) + k_{\ell}(x)k_{\ell}(y)
 + (-1)^{\ell}   k_{2\ell}(|x-y|),\notag 
\end{align}
where $k_{q}(x) = (q!)^{-1}B_{q}(x)$ and $B_{q}(\cdot)$ is the Bernoulli polynomial of order $q$.




By the representer theorem for RKHS \citep{wahba1990spline}, the estimate  in (\ref{3})  lies in the linear span of $\{K(\cdot, \bx_{i}), i = 1,\ldots, n\}$.
Specifically, we have 
\begin{align}\label{KRR}
\wh{m}(\cdot) = \sum_{i=1}^{n}\wh{\alpha}_{i,\lambda}K(\cdot, \bx_{i}),
\end{align}
where 
\begin{align}
\wh{\balpha}_{\lambda} = \left(\bDelta_{n}\bK +  \lambda\bI_{n}\right)^{-1}\bDelta_n \by,\notag
\end{align}
$\bDelta_{n} = \mbox{diag}(\delta_{1}, \ldots, \delta_{n})$, $\bK = (K(\bx_{i}, \bx_{j}))_{ij} $,  $\by = (y_{1},\ldots, y_{n})\trans$ and $\bI_{n}$ is the $n\times n$ identity matrix.

The tuning parameter $\lambda$ is selected via generalized cross-validation (GCV) in KRR, where the GCV criterion for $\lambda$ is
\begin{align}\label{GCV}
 \mbox{GCV}(\lambda) = \frac{n^{-1}   \Norm{ \left\{\bDelta_{n} - \bA(\lambda)\right\}\by }_{2}^{2}   }{n^{-1}  \mbox{Trace}(\bDelta_{n} - \bA(\lambda) )  },
\end{align}
and $\bA(\lambda) = \bDelta_{n}\bK  ( \bDelta_{n} \bK + \lambda \bI_{n}  )^{-1} \bDelta_{n}  $. The value of $\lambda$ minimizing the GCV is used for the selected tuning parameter. 





Using the KRR imputation in (\ref{3}), we aim to establish the following two goals:
\begin{enumerate}
	\item Find the sufficient conditions for the   $\sqrt{n}$-consistency of the imputation estimator $\wh{\theta}_I$ using (\ref{KRR}) and give a formal proof. 
	\item Find a linearization variance formula for the imputation estimator $\wh{\theta}_I$ using the KRR imputation. 
\end{enumerate}
The first part is formally presented in Theorem 1 in Section 3. For the second part, we employ the density ratio estimation method of \cite{nguyen2010} 
to get a consistent estimator of  $\omega (\bx) = \{\pi (\bx)\}^{-1}$ in the linearized version of $\hat{\theta}_I$. 
\begin{comment}
By Theorem 1, we use the following estimator to estimate the variance of $\wh{\theta}_I$ in (\ref{2}): 
\begin{align}\label{variance estimation}
\wh{\mbox{V}} (\wh{\theta}_{I}) = \frac{1}{n(n-1)}\sum_{i=1}^{n}\left( \hat{\eta}_i - \bar{\eta} \right)^2 
\end{align}
where 
$$  \hat{\eta}_i =  \wh{m}(\bx_{i}) + \delta_{i} \wh{\omega}_{i} \left\{ y_{i} - \wh{m}(\bx_{i})\right\},
$$
and $ \wh{\omega}_{i} $ is a consistent estimator of $\omega (\bx) = \{\pi (\bx)\}^{-1}$. 
\end{comment} 



\section{Main Theory}


Before we develop our main theory, we first introduce Mercer's theorem. 
\begin{lemma}[Mercer's theorem]\label{Mercer}
Given a continuous, symmetric, positive definite kernel function $K: \mathcal{X} \times \mathcal{X} \mapsto \mathbb{R}$. For $\bx, \bz \in \mathcal{X}$, under some regularity conditions, Mercer's theorem characterizes $K$ by the following expansion
\begin{align}
 K(\bx, \bz) = \sum_{j=1}^{\infty}\lambda_{j}\phi_{j}(\bx) \phi_{j}(\bz),\notag
\end{align}
where $\lambda_{1} \geq \lambda_{2} \geq \ldots \geq 0$ are a non-negative sequence of eigenvalues and $\{\phi_{j} \}_{j=1}^{\infty}$ is an orthonormal basis for $L^{2}(\mathbb{P})$.
\end{lemma}

To develop our theory, we  make the following  assumptions.
 \begin{description}
 
 \item {[A1]}
 \label{A1}
 	For some $k \geq 2$, there is a constant $\rho < \infty$ such that $E[ \phi_{j}(X)^{2k} ] \leq \rho^{2k}$ for all 
 	$j \in \mathbb{N}$, where $\{\phi_{j}\}_{j=1}^{\infty}$ are orthonormal basis by expansion from Mercer's theorem.

 \item {[A2]}
 \label{A2}
 	The function $m \in \mathcal{H}$, and for $\bx \in \mathcal{X}$, we have $E[\left\{ Y -  m(\bx)\right\}^{2} ] \leq \sigma^{2}$, {for some  $\sigma^{2} < \infty$.}

\item {[A3]}
\label{A3}
     The propensity score $\pi(\cdot)$ is uniformly bounded away from zero. In particular, there exists a positive constant $c > 0$ such that 
     	$\pi(\bx_{i}) \geq c$, for $i = 1, \ldots, n$.

\item {[A4]}
 \label{A4}
     The ratio $d/\ell < 2$ for $d$-dimensional Sobolev space of order $\ell$, where $d$ is the dimension of covariate $\bx$.
\end{description}


The first assumption is a technical assumption which controls the tail behavior of $\{\phi_{j}\}_{j=1}^{\infty}$. Assumption 2 indicates that the noises have bounded variance. Assumption 1 and Assumption 2 together aim to control the error bound of the kernel ridge regression estimate $\wh{m}$. {Furthermore}, Assumption 3 means that the support for the respondents should be the same as  the original sample support. Assumption 3 is a standard assumption for missing data analysis. Assumption 4 is a technical assumption for entropy analysis. Intuitively, when the dimension is large, the Sobolev space should be large enough to capture the true model. 


\begin{theorem}\label{main theorem}
Suppose Assumption $1 \sim 4$ hold for a Sobolev kernel of order $\ell$, $\lambda \asymp   n^{1-\ell}$, we have
\begin{align}\label{rate}
 \sqrt{n}(\wh{\theta}_{I} - \wt{\theta}_{I} ) = o_{p}(1), 
\end{align}
where 
\begin{align}\label{tilde_theta}
\wt{\theta}_{I} &= \frac{1}{n}\sum_{i=1}^{n} \left[ m(\bx_{i}) + \delta_{i} \frac{1}{\pi(\bx_{i})}  \left\{ y_{i} - m(\bx_{i})\right\} 
   \right] 
\end{align}
and 
\textcolor{blue}{}
$$\sqrt{n} \left( \tilde{\theta}_I - \theta \right) \stackrel{\mathcal{L}}{\longrightarrow}  N(0, \sigma^2 ) ,
$$
 with 
$$ \sigma^2 = V\{ E( Y \mid \bx) \} + E\{ V( Y \mid \bx)/\pi( \bx)   \} . $$ 

%Let $\omega_{i}^{\star} = \pi(\bx_{i})^{-1}$. In particular,  $\pmb{\omega}^{\star} = \{\omega_{i}^{\star}: \delta_{i} = 1\}$  can be estimated by\cite{wong2018kernel}.
\end{theorem}


Theorem \ref{main theorem} guarantees the asymptotic equivalence of $\wh{\theta}_{I}$ and $\wt{\theta}_{I}$ in \eqref{tilde_theta}. Specifically, the reference distribution is a combination of an outcome model and a propensity score model for sampling mechanism. The variance of $\wt{\theta}_I$ achieves the semiparametric lower bound of \citet{robins94}. 
%{Additionally, \eqref{tilde_theta} suggests a linearization form of variance estimation of $\wh{\theta}_{I}$. To estimate $\omega_{i}^{\star} = \pi(\bx_{i})^{-1}$, we can use employ the covariate balancing idea of \citet{wong2018kernel} to obtain $\hat{\pmb{\omega}}$ as provided in \eqref{WC_Opt}. }  
The proof of Theorem \ref{main theorem} is presented in the Appendix.
%\hf{Here, in some sense, I think we do not have a valid variance estimator for \citet{wong2018kernel}. In particular, they have uncertainty of $\wh{\omega}_{i}$. In our case, we only consider the variance estimator for $\wh{\theta}_{I}$, which only involves $\wh{m}$.}




The linearization formula in (\ref{tilde_theta}) can be used for  variance estimation. The idea is to estimate the influence function 
$\eta_i = m(\bx_{i}) + \delta_{i} \{ \pi(\bx_{i})\}^{-1}  \left\{ y_{i} - m(\bx_{i}) \right\} $ and apply the standard variance estimator using $\hat{\eta}_i$. To estimate $\eta_i$, we need an estimator of $\pi(x)$. We propose a version of KRR method to estimate $\omega(x) = \{ \pi(x) \}^{-1}$ directly.  
In order to estimate $\omega(x) = \{ \pi(x) \}^{-1}$, we wish to develop a {KRR} version of estimating $\omega (x)$. To do this, first define 
\begin{equation} 
 g( x) = \frac{ f(x \mid  \delta =0 ) }{ f( x \mid  \delta = 1 ) },
\label{dr2}
\end{equation} 
and, by Bayes theorem, we have  
$$ \omega(x)= \frac{1}{ \pi(x) }  = 1+ \frac{n_0}{n_1}  g(x).
$$
Thus, to estimate $\omega(x)$, we have only to estimate the density ration function $g(x)$ in (\ref{dr2}). 
Now, to estimate $g(x)$ nonparametrically, we use the idea of \cite{nguyen2010} for  the KRR approach to density ratio estimation. 

%in order to  estimate $g( x)$ and obtain a nonparametric estimator of $\omega(x)$. 

%\jk{Some details on the estimation of $g(x)$ should be placed here } 


To explain the KRR estimation of $g(x)$, note that $g(x)$ can be understood as the  maximizer of 
 \begin{align} 
     Q (g) = \int \log \left( g \right) f( \bx \mid \delta =0)  d \mu(\bx)   -   \int g (x) f( \bx \mid \delta = 1)   d \mu(x)    \label{qq} 
    \end{align} 
    with constraint 
    $$  \int g (x) f( \bx \mid \delta = 1)   d \mu(x) = 1 . 
    $$ 
    The sample version objective function is 
 \begin{equation} 
\hat{Q}( g) =  \frac{1}{n_0} \sum_{i=1}^n \mathbb{I} ( \delta_i=0) \log \{ g(\bx_i)  \} - \frac{1}{n_1} \sum_{i=1}^n \mathbb{I} ( \delta_i=1) g (\bx_i) 
\label{qq2} 
\end{equation}  
where $n_k = \sum_{i=1}^n \mathbb{I} (\delta_i = k ) $. The maximizer of $\hat{Q} (g)$ is an M-estimator of the density ratio function $g$. 



%Define 
%$h (\bx)= \log \{ g(\bx)\}$. The loss function  $L ( \cdot)$  derived from the optimization problem in (\ref{qq2}) can be written as 
%\begin{equation*} 
% L( \delta, h(\bx)  ) = \frac{1}{n_0}  \mathbb{I} (\delta=0) h(\bx) - \frac{1}{n_1} \mathbb{I} (\delta=1) \exp \{ h(\bx) \}. 
%\end{equation*} 



Further, define 
$h (\bx)= \log \{ g(\bx)\}$.
 The loss function 
 $L ( \cdot)$  derived from the optimization problem in (\ref{qq2}) can be written as 
\begin{equation*} 
 L( \delta, h(\bx)  ) = \frac{1}{n_1} \mathbb{I} (\delta=1) \exp \{ h(\bx) \} - \frac{1}{n_0}  \mathbb{I} (\delta=0) h(\bx).
\end{equation*} 
%such that  $$ \hat{Q} ( r_k) = \sum_{i=1}^N L( y_i, h_k (x_i) ) . $$



In our  problem, we wish to find $h$ that minimizes 
\begin{equation}
\sum_{i=1}^n L( \delta_i, \alpha_{0} + h(x_i) ) + \tau \left\| h \right\|_{\mathcal{H} }^{2} \label{18} 
\end{equation} 
over $\alpha_0 \in \mathbb{R}$ and $h \in \mathcal{H} $, where $L ( \cdot) $ is the loss function  derived from the optimization problem in (\ref{qq}) using maximum entropy. 
%That is, 



Hence, using the representer theorem again, the solution to (\ref{18}) can be obtained as 
\begin{equation} \label{entropy_method}
\min_{ \alpha \in \mathbb{R}^n  } \left\{ \sum_{i=1}^n L( \delta_i, \alpha_0 + \sum_{j=1}^n   \alpha_j K( x_i, x_j )) + \tau \balpha' \mathbf{K}  \balpha  \right\} 
\end{equation}
and $\alpha_0$ is a normalizing constant satisfying 
$$ n_1 = \sum_{i=1}^n  \mathbb{I} (\delta_i = 1) \exp \{ \alpha_0 + \sum_{j=1}^n  \hat{\alpha}_j K( x_i, x_j ) \} . $$
 Thus, we use 
\begin{equation}
 \hat{g} (x) = \exp \{ \hat{\alpha}_0 + \sum_{j=1}^n  \hat{\alpha}_j K( x, x_j)  \}
 \label{final} 
 \end{equation} 
 as a nonparametric approximation of the density ratio function $g( x)$. Also, 
 \begin{equation} 
 \hat{\omega} (x)= 1+ \frac{n_0}{n_1} \hat{g} (x)
 \label{final2}
 \end{equation} 
 is the nonparametric approximation of $\omega(x) = \{ \pi(x) \}^{-1}$. Note that $\tau$ is the tuning parameter that determines the model complexity of $g(x)$. The tuning parameter selection is discussed in Appendix B.  
 
 
 
 
Therefore, we can use 
\begin{equation} 
\wh{\mbox{V}}  = \frac{1}{n} \frac{1}{n-1} \sum_{i=1}^n \left( \hat{\eta}_i - \bar{\eta}_n \right)^2  
\label{varh}
\end{equation} 
as a variance estimator of $\hat{\theta}_I$, where 
\begin{equation} 
\hat{\eta}_i = \wh{m}(\bx_{i}) + \delta_{i} \wh{\omega}_{i} (x_i) \left\{ y_{i} - \wh{m}(\bx_{i})\right\}
\end{equation} 
and $\bar{\eta}_n = n^{-1} \sum_{i=1}^n \hat{\eta}_i$. 






\section{Simulation Study}

%\jk{We need to update our simulation result because we use a different function of $\hat{\omega}(x)$ for variance estimation. } 

%\hf{Simulation results updated.}
\subsection{Simulation study one} 

To evaluate the performance of the proposed imputation method and its variance estimator, we conduct two simulation studies. In the first simulation study, a continuous study variable is considered with three different data generating models.   In the three models, we keep the response rate around $70\%$ and $\mbox{Var}(Y) \approx 10$. Also, $\bx_{i} =  (x_{i1}, x_{i2}, x_{i3}, x_{i4})\trans$ are generated independently {element-wise} from the {uniform} distribution on the support $(1,3)$. In the first model (Model A), we use a linear regression model 
%The responses for linear case (Model A) are generated by 
\begin{align}
y_{i} =& 3 + 2.5x_{i1} + 2.75 x_{i2} +  2.5 x_{i3} + 2.25 x_{i4} + \sigma\epsilon_{i},\notag
\end{align}
to obtain $y_i$, 
where $\{\epsilon_{i}\}_{i=1}^{n}$ are generated from standard normal distribution and $\sigma = \sqrt{3}$. In the second model  (Model B), we use  \begin{align}
y_{i} =& 3 + (1/35)x_{i1}^{2}x_{i2}^{3}x_{i3} +  0.1x_{i4} + \sigma\epsilon_{i} \notag
\end{align}
to generate data with a {nonlinear} structure. The third model  (Model C) for generating the study variable is 
\begin{align}
y_{i} =& 3 + (1/180)x_{i1}^{2}x_{i2}^{3}x_{i3}x_{i4}^{2} + \sigma\epsilon_{i}.\notag
\end{align}


In addition to $\{(x_i\trans, y_i)\trans, i = 1, \ldots, n\}$, the response indicator variable $\delta$'s are independently generated from the {Bernoulli} distribution with probability $\logit(\bx_i' \beta + 2.5)$, where  ${\beta} = (-1, 0.5, -0.25, -0.1)\trans$ and $\mbox{logit}(p) = \log\{p / (1-p)\}$. We considered three sample sizes $n = 200$, $n = 500$ and $n = 1,000$ with 1,000 Monte Carlo replications. The reproducing kernel Hilbert space we employed is the second-order Sobolev space. 

We also compare three imputation methods: kernel ridge regression (KRR), B-spline, linear regression (Linear). We compute the Monte Carlo biases, variance, and the mean squared {errors} of the imputation estimators for each case. 
The corresponding results are presented in   Table \ref{MAR, linear, comparison}.  
%In Table \ref{MAR, linear, comparison}, the performance of the three imputation estimators are presented.   


\begin{table}[!ht]
\centering
\caption{Biases, Variances and Mean Squared Errors (MSEs) of three imputation estimators for continuous responses}\label{MAR, linear, comparison}
{
\begin{tabular}{cccccc}
  \hline
Model & Sample Size & Criteria & KRR & B-spline & Linear  \\ 
  \hline
 \multirow{9}{*}{A}& \multirow{3}{*}{$200$} & Bias & -0.0577 & 0.0027 & 0.0023 \\ 
 & & Var & 0.0724 & 0.0679 & 0.0682 \\ 
 & & MSE & 0.0757 & 0.0679 & 0.0682 \\ 
 \cline{3-6}
 &\multirow{3}{*}{$500$} & Bias & -0.0358 & 0.0038 & 0.0038 \\ 
 & & Var & 0.0275 & 0.0263 & 0.0263 \\ 
 & & MSE & 0.0288 & 0.0263 & 0.0263 \\
 \cline{3-6} 
 &\multirow{3}{*}{$1000$} & Bias & -0.0292 & 0.0002 & 0.0002 \\ 
 & & Var & 0.0132 & 0.0128 & 0.0129 \\ 
 & & MSE & 0.0141 & 0.0128 & 0.0129 \\
   \hline
\multirow{9}{*}{B} &  \multirow{3}{*}{$200$} & Bias & -0.0188 & 0.0493 & 0.0372 \\ 
 & & Var & 0.0644 & 0.0674 & 0.0666 \\ 
 & & MSE & 0.0648 & 0.0698 & 0.0680 \\ 
  \cline{3-6} 
 &\multirow{3}{*}{$500$} & Bias & -0.0136 & 0.0463 & 0.0356 \\ 
 & & Var & 0.0261 & 0.0275 & 0.0272 \\ 
 & & MSE & 0.0263 & 0.0296 & 0.0285 \\ 
  \cline{3-6} 
 &\multirow{3}{*}{$1000$} & Bias & -0.0122 & 0.0426 & 0.0313 \\ 
 & & Var & 0.0121 & 0.0129 & 0.0129 \\ 
 & & MSE & 0.0123 & 0.0147 & 0.0139 \\ 
\hline
  \multirow{9}{*}{C} &  \multirow{3}{*}{$200$}& Bias & -0.0223 & 0.0384 & 0.0283 \\ 
 & & Var & 0.0748 & 0.0811 & 0.0792 \\ 
 & & MSE & 0.0753 & 0.0825 & 0.0800 \\ 
 \cline{3-6}
 &\multirow{3}{*}{$500$} & Bias & -0.0141 & 0.0369 & 0.0287 \\ 
 & & Var & 0.0281 & 0.0307 & 0.0301 \\ 
 & & MSE & 0.0283 & 0.0320 & 0.0309 \\ 
 \cline{3-6}
 &\multirow{3}{*}{$1000$} & Bias & -0.0142 & 0.0310 & 0.0221 \\ 
 & & Var & 0.0124 & 0.0138 & 0.0136 \\ 
 & & MSE & 0.0126 & 0.0148 & 0.0141 \\ 
 \hline
\end{tabular}
}
\end{table}


The simulation results in  Table \ref{MAR, linear, comparison} shows that  the three methods show similar results under the linear model (Model A), but   kernel ridge regression imputation shows the best performance in terms of {the mean square errors} under the nonlinear models (Models B and C). Linear regression imputation still provides unbiased estimates, because the residual terms in the linear regression model are  approximately unbiased to zero. However, use of linear regression model for imputation leads to efficiency loss because it is not the best model. 



In addition, we have computed the proposed variance estimator under kernel ridge regression imputation. 
%The behavior of variance estimation for the imputation estimator is presented in Table \ref{MAR, linear, KRR}. 
In Table \ref{MAR, linear, KRR}, the relative biases of the proposed variance estimator  and the coverage rates of two interval estimators under $90\%$ and $95\%$ nominal coverage rates are presented. 
The relative bias of the variance estimator decreases as the sample size increases, which confirms  the validity of the proposed variance estimator. Furthermore, the interval estimators show good performances in terms of the coverage rates. 

%calculated by the variance estimation based on our method are well approximated.





\begin{table}[!ht]
\centering
\caption{Relative biases (R.B.)  of the proposed variance estimator, coverage rates (C.R.) of the $90\%$ and $95\%$ confidence intervals for imputed estimators under kernel ridge regression imputation for continuous responses}\label{MAR, linear, KRR}
\begin{tabular}{ccccc}
  \hline
\multirow{2}{*}{Model} & \multirow{2}{*}{Criteria}  & \multicolumn{3}{c}{Sample Size} \\
  \cline{3-5} 
    &  &   200 & 500 & 1000 \\ 
  \hline
 \multirow{3}{*}{A} & R.B. & -0.1050 & -0.0643 & -0.0315 \\ 
 &  C.R. (90\%) & 87.5\% & 89.6\% & 89.9\% \\ 
 &  C.R. (95\%)  & 94.0\% & 94.7\% & 94.9\% \\ 
 \hline
\multirow{3}{*}{B} & R.B. &-0.1016 & -0.1086 & -0.0276 \\ 
 & C.R. (90\%) & 87.6\% & 87.0\% & 89.2\% \\ 
 & C.R. (95\%)  & 92.6\% & 93.3\% & 94.8\% \\ 
   \hline
 \multirow{3}{*}{C} &  R.B. & -0.1934 & -0.1310 & -0.0054 \\
 &  C.R. (90\%)  & 85.0\% & 86.2\% & 90.4\% \\ 
 &  C.R. (95\%)  & 91.4\% & 93.4\% & 94.6\% \\ 
 \hline
\end{tabular}
\end{table}


\subsection{Simulation study two} 

The second simulation study is similar to the first simulation study except that the study variable $Y$ is binary. We use the same simulation setup for generating $\bx_i = (x_{1i}, x_{2i}, x_{3i}, x_{4i})$ and $\delta_i$ as the first simulation study. We consider three models for generating $Y$ 
\begin{equation}
y_{i} \sim \mbox{Bernoulli}(p_{i}), \label{model2} 
\end{equation}
where $p_{i}$ is chosen differently  for each model. 
For model D, we have
\begin{align}
\mbox{logit}(p_{i}) =  0.5 + (1/35)x_{i1}^{2}x_{i2}^{3}x_{i3} +  0.1x_{i4}.\notag
\end{align}
The responses for Model E are generated by (\ref{model2}) with 
\begin{align}
\mbox{logit}(p_{i}) =  0.5 + (1/180)x_{i1}^{2}x_{i2}^{3}x_{i3}x_{i4}^{2}.\notag
\end{align}
The responses for Model F are generated by (\ref{model2}) with 
\begin{align}
\mbox{logit}(p_{i}) =  0.5 + 0.15x_{i1}x_{i2}x_{i3}^{2} +  0.4x_{i2}x_{i3}.\notag
\end{align}

For each model, we consider three imputation estimators: kernel ridge regression (KRR), B-spline, linear regression (Linear). We compute the Monte Carlo biases, variance, and the mean squared errors of the imputation estimators for each case.  {The comparison of the simulation 
results for different estimators  are} presented in Table \ref{MAR, nonlinear, comparison, DEF}. In addition, the relative biases and the coverage rates of the interval estimators  are presented in Table \ref{MAR, binary, KRR}. The simulation results in Table \ref{MAR, binary, KRR} show that the relative biases of the  variance estimators are negligible  and the coverage rates of the interval estimators are close to the nominal levels. 





\begin{table}[!ht]
\centering
\caption{Biases, Variances and Mean Squared Errors (MSEs) of three imputation estimators for binary responses}\label{MAR, nonlinear, comparison, DEF}
{
\begin{tabular}{cccccc}
  \hline

Model & Sample Size & Criterion & KRR & B-spline & Linear  \\ 
  \hline
 \multirow{9}{*}{D} &  \multirow{3}{*}{$200$}& Bias & 0.00028 & 0.00007 & 0.00009 \\ 
 & & Var & 0.00199 & 0.00208 & 0.00206 \\ 
 & & MSE & 0.00199 & 0.00208 & 0.00206 \\ 
 \cline{3-6}
 &\multirow{3}{*}{$500$} & Bias & -0.00019 & -0.00014 & -0.00019 \\ 
 & & Var & 0.00080 & 0.00081 & 0.00081 \\ 
 & & MSE & 0.00080 & 0.00081 & 0.00081 \\ 
 \cline{3-6}
 &\multirow{3}{*}{$1000$} & Bias & -0.00006 & -0.00010 & -0.00010 \\ 
 & & Var & 0.00042 & 0.00042 & 0.00042 \\ 
  \hline
 \multirow{9}{*}{E} &  \multirow{3}{*}{$200$}& Bias & 0.00027 & -0.00001 & -0.00003 \\ 
 & & Var & 0.00195 & 0.00204 & 0.00202 \\ 
 & & MSE & 0.00195 & 0.00204 & 0.00202 \\ 
 \cline{3-6}
 &\multirow{3}{*}{$500$}  & Bias & -0.00039 & -0.00042 & -0.00044 \\ 
 & & Var & 0.00079 & 0.00080 & 0.00080 \\ 
 & & MSE & 0.00079 & 0.00080 & 0.00080 \\ 
  \cline{3-6}
 &\multirow{3}{*}{$1000$}  & Bias & -0.00005 & -0.00013 & -0.00010 \\ 
 & & Var & 0.00042 & 0.00043 & 0.00043 \\ 
 & & MSE & 0.00042 & 0.00043 & 0.00043 \\ 
   \hline
 \multirow{9}{*}{F} &  \multirow{3}{*}{$200$}& Bias & 0.00077 & 0.00102 & 0.00100 \\ 
 & & Var & 0.00199 & 0.00208 & 0.00206 \\ 
 & & MSE & 0.00199 & 0.00208 & 0.00206 \\ 
  \cline{3-6}
 &\multirow{3}{*}{$500$}  & Bias & -0.00002 & 0.00054 & 0.00047 \\ 
 & & Var & 0.00079 & 0.00080 & 0.00080 \\ 
 & & MSE & 0.00079 & 0.00080 & 0.00080 \\ 
  \cline{3-6}
 &\multirow{3}{*}{$1000$}  & Bias & 0.00007 & 0.00055 & 0.00060 \\ 
 & & Var & 0.00042 & 0.00043 & 0.00043 \\ 
 & & MSE & 0.00042 & 0.00043 & 0.00043 \\ 
\hline
\end{tabular}
}
\end{table}



\begin{table}[!ht]
\centering
\caption{
Relative biases (R.B.)  of the proposed variance estimator, coverage rates (C.R.) of the $90\%$ and $95\%$ confidence intervals for imputed estimators under kernel ridge regression imputation 
for binary responses}\label{MAR, binary, KRR}
\begin{tabular}{ccccc}
  \hline
\multirow{2}{*}{Model} & \multirow{2}{*}{Criteria}  & \multicolumn{3}{c}{Sample Size} \\
  \cline{3-5} 
    &  &   200 & 500 & 1000 \\ 
  \hline
 \multirow{3}{*}{D} & R.B. & -0.0061 & 0.0068 & -0.0392 \\ 
 & C.R. (90\%)  & 88.6\% & 90.2\% & 90.4\% \\ 
 &  C.R. (95\%)  & 94.6\% & 94.1\% & 94.3\% \\ 
 \hline
\multirow{3}{*}{E} & R.B.  & 0.0165 & 0.0222 & -0.0487 \\ 
 &  C.R. (90\%)  & 89.2\% & 89.9\% & 89.6\% \\ 
 & C.R. (95\%)   & 94.6\% & 94.7\% & 93.9\% \\  
  \hline
 \multirow{3}{*}{F} &   R.B. & -0.0062 & 0.0187 & -0.0437 \\ 
 & C.R. (90\%)   & 89.9\% & 89.7\% & 89.9\% \\ 
 & C.R. (95\%)   & 94.7\% & 94.8\% & 94.3\% \\ 
 \hline
\end{tabular}
\end{table}


%{We also compute the confidence intervals  using the asymptotic normality of the kernel ridge regression imputed estimator. The proposed variance estimator is used in computing the confidence intervals. Table 3 shows the coverage rates of the confidence intervals. The realized coverage probabilities are close to the nominal coverage probabilities, confirming the validity of the proposed interval estimator. }


\section{Application} 


\begin{comment}
We applied the KRR with kernels of second-order Sobolev space and Gaussian kernel to study the $\mbox{NO}_{2}$ measured in a city in Italy (\textcolor{red}{citation?}). Hourly weather conditions: temperature, absolute humidity, relative humidity are available for the whole year. Meanwhile, the averaged sensor response is subject to the missingness. We are interested in whether there are significant difference of $\mbox{NO}_{2}$ among seasons. We take March to May as spring, June to August as summer, September to November as fall, December to Februrary in the following year as winter. The corresponding missing rates for each season are presentaed in Table \ref{Missing_Table}. The estimates and 95\% confidence intervals for each season are presented in the Figure \ref{CI}. As a benchmark, the confidence interval computed from complete cases are also presented there. As we can see, the behaviors of KRR by two kernels are similar, except that KRR with Sobolev space seems have larger confidence interval. This is because second order Sobolev space is more complex than the RKHS induced by Gaussian kernel.  For the data application, we can see that 
the pollution in fall and winter are significantly more severe than those in spring and summer, where winter has highest $\mbox{NO}_{2}$ concentration. 


\begin{table}[!ht]
\centering
\caption{Missing Rates for Each Season}\label{Missing_Table}
\begin{tabular}{ccccc}
  \hline
Season & Spring & Summer & Fall & Winter \\ 
\hline
  Missing Rate & 20.36\% & 15.76\% & 31.82\% &  8.61\% \\ 
   \hline
\end{tabular}
\end{table}




\begin{figure}[!ht]
    \centering  \label{CI}
            \caption{Estimated seasonally  mean NO2 concentration in 2004  with 95\% confidence interval.}
    \label{fig:mesh1}
    \includegraphics[width=1\textwidth]{NO2_CI.png}
\end{figure}
\end{comment}


We applied the KRR with kernels of second-order Sobolev space and Gaussian kernel to study the $\mbox{PM}_{2.5}(\mu g/m^{3})$ concentration measured in Beijing, China \citep{liang2015assessing}. Hourly weather conditions: temperature, air pressure, cumulative wind speed, cumulative hours of snow and cumulative hours of rain are available from 2011 to 2015. Meanwhile, the averaged sensor response is subject to missingness. In December 2012, the missing rate of $\mbox{PM}_{2.5}$ is relatively high with missing rate $17.47\%$. We are interested in estimating the mean $\mbox{PM}_{2.5}$ in December with imputed KRR estimates. The point estimates and their 95\% confidence intervals are presented in the Table \ref{CI_Table}. The corresponding results are presented in the Figure \ref{CI_Fig}.  As a benchmark, the confidence interval computed from complete cases (Complete in Table \ref{CI_Table}) and confidence intervals for the imputed estimator under linear model (Linear) \citep{kim2009unified}  are also presented there. 

\begin{table}[!ht]
\centering
\caption{Imputed estimates (I.E.), standard error (S.E.) and $95\%$ confidence intervals (C.I.) for imputed mean $\mbox{PM}_{2.5}$ in December, 2012 under kernel ridge regression}\label{CI_Table}
\begin{tabular}{cccc}
  \hline
 Estimator & I.E. & S.E. & $95\%$ C.I. \\ 
  \hline
  Complete & 109.20 & 3.91 & (101.53, 116.87) \\ 
  Linear & 99.61 & 3.68 & (92.39, 106.83) \\ 
  Sobolev & 102.25 & 3.50 & (95.39, 109.12) \\ 
  Gaussian & 101.30 & 3.53 & (94.37, 108.22) \\ 
   \hline
\end{tabular}
\end{table}

\begin{figure}[!ht]
    \centering  \label{CI_Fig}
            \caption{Estimated mean $\mbox{PM}_{2.5}$ concentration in December 2012 with 95\% confidence interval.}
    \label{fig:mesh1}
    \includegraphics[width=1\textwidth]{PM_CI.png}
\end{figure}

As we can see, the performances of {KRR imputation estimators are similar} and created narrower $95\%$ confidence intervals. Furthermore,  the imputed $\mbox{PM}_{2.5}$ concentration during the missing period 
is relatively lower than  the fully observed weather conditions on average.  Therefore, if we only utilize the complete cases to estimate the  mean of $\mbox{PM}_{2.5}$, the severeness of air pollution would be over-estimated.





\section{Discussion}
 We consider kernel ridge regression  as a tool for nonparametric imputation and establish its asymptotic properties. In addition, we propose a linearized approach for variance estimation of the imputed estimator. For variance estimation, we also propose a novel approach of the maximum entropy method for  propensity score estimation.   The proposed Kernel ridge regression imputation can be used as a general tool for nonparametric imputation. By choosing different kernel functions, different  nonparametric imputation methods can be developed. The unified theory developed in this paper can cover various type of the kernel ridge regression imputation and enables us to make valid statistical inferences about the population means. 
 
 
% Numerical studies confirm our theoretical results.

There are several possible extensions of the research. First, the theory can be directly applicable to other nonparametric imputation methods, such as smoothing splines \citep{claeskens2009}. Second, instead of using ridge-type penalty term, one can also consider other penalty functions such as SCAD penalty \citep{FL2001} or adaptive Lasso \citep{zou2006}. Also, the maximum entropy method for propensity score estimation should be investigated more rigorously. Such extensions will be future research topics. 

\appendix

\section*{Appendix} 
\subsection*{A. 
Proof of Theorem 1 }

\renewcommand{\theequation}{A.\arabic{equation}}
\setcounter{equation}{0} 




Before we prove the main theorem, we first introduce the following lemma. 

\begin{lemma}[modified Lemma 7 in \citet{zhang2013divide}]\label{order}
Suppose Assumption [A1] and [A2] hold, for a random vector $\bz = \mathbb{E}(\bz) + \sigma \bvar$, let $\wt{\lambda} = \lambda/n$ we have
\begin{align}
 \bS_{\lambda}\bz = \mathbb{E}(\bz \mid \bx) + \mathcal{O}_{p}\left( \wt{\lambda} + \sqrt{\frac{ \gamma(\wt{\lambda})  }{n} } \right )\bone_{n},\notag
\end{align}
as long as $\mathbb{E}(\norm{z_{i}}_{\mathcal{H}})$ and $\sigma^{2}$ is bounded from above, for $i=1, \ldots, n$,  where $\bvar$ are noise vector with mean zero and bounded variance and 
 \begin{equation}\label{effective_dimension}
 \gamma(\wt{\lambda}) := \sum_{j=1}^{\infty} \frac{1}{1+\wt{\lambda}/\mu_{j}},\notag
 \end{equation}
is the effective dimension and $\{\mu_{j}\}_{j=1}^{\infty}$ are the eigenvalues of kernel $K$ used in $\hat{m}(\bx)$. 
%See Lemma \ref{Mercer} for the definition of the eigenvalues of kernel $K$
\end{lemma}
%\textcolor{red}{What is $\gamma(\wt{\lambda}) $? }


Now, to prove our main theorem, we write 
\begin{align}
\wh{\theta}_{I} &= \frac{1}{n} \sum_{i=1}^{n}\left\{ \delta_{i}y_{i} + (1-\delta_{i}) \wh{m}(\bx_{i}) \right\} \notag \\
&= \underbrace{\frac{1}{n}\sum_{i=1}^{n}m(\bx_{i})}_{:= R_{n}} + \underbrace{\frac{1}{n}\sum_{i=1}^{n}\delta_{i}\left\{   y_{i}  - m(\bx_{i}) \right\}}_{:= S_{n} }  + \underbrace{\frac{1}{n}\sum_{i=1}^{n} (1-\delta_{i})\left\{ \wh{m}(\bx_{i}) - m(\bx_{i})  \right\}}_{:=T_{n}}.
\end{align}
Therefore, as long as we show 
\begin{align}
T_{n} = \frac{1}{n} \sum_{i=1}^{n}\delta_{i}\left\{ \frac{1}{\pi(\bx_{i})} - 1  \right\}\left\{ y_{i} - m(\bx_{i})  \right\} + o_{p}(n^{-1/2}),\label{12} 
\end{align}
then the main theorem automatically holds.

To show (\ref{12}), recall that the KRR can be regarded as the following optimization problem
\begin{align}
 \wh{\balpha}_{\lambda} = \argmin_{\balpha \in \mathbb{R}^{n}} (\by - \bK\balpha)\trans \bDelta_{n} (\by - \bK \balpha) + \lambda \balpha\trans \bK \balpha.\notag 
\end{align}
Further, we have
\begin{align}
  \wh{\balpha}_{\lambda} = \left( \bDelta_{n} \bK + \lambda \bI_{n}  \right)^{-1} \bDelta_{n}\by,\notag
\end{align}
and 
\begin{align}
 \wh{\bmm} &= \bK \left(  \bDelta_{n}\bK + \lambda \bI_{n}    \right)^{-1} \bDelta_{n}\by \notag \\
 &= \bK  \left\{  \left(  \bDelta_{n} + \lambda \bK^{-1}    \right) \bK     \right\}^{-1} \bDelta_{n} \by \notag \\
 &=  \left(    \bDelta_{n} + \lambda \bK^{-1}   \right)^{-1}  \bDelta_{n}\by,\notag
\end{align}
where $\wh{\bmm} = (\wh{m}(\bx_{1}), \ldots, \wh{m}(\bx_{n}))\trans$. Let $\bS_{\lambda} =  (  \bI_{n} + \lambda \bK^{-1}   )^{-1}$, we have
\begin{align}
 \wh{\bmm} = \left( \bDelta_{n} + \lambda \bK^{-1}  \right)^{-1}\bDelta_{n}\by = \bC_{n}^{-1} \bd_{n},\notag
\end{align}
where
\begin{align}
\bC_{n} &= \bS_{\lambda}\left( \bDelta_{n} + \lambda\bK^{-1}   \right),\notag\\
\bd_{n} &= \bS_{\lambda}\bDelta_{n}\by\notag.
\end{align}

By Lemma \ref{order}, let $\wt{\lambda} = \lambda/n$, we obtain
\begin{align}
\bC_{n} &= \mathbb{E}(\bDelta_{n} \mid \bx) + \mathcal{O}_{p}\left( \wt{\lambda} + \sqrt{\frac{ \gamma(\wt{\lambda})  }{n} } \right )\bone_{n} \notag \\
&:= \bPi + \mathcal{O}_{p}\left( \wt{\lambda} + \sqrt{\frac{ \gamma(\wt{\lambda})  }{n} } \right )\bone_{n},\notag
\end{align}
where $\bPi = \mbox{diag}(\pi(\bx_{1}), \ldots, \pi(\bx_{n}))$ and 
 $\gamma(\wt{\lambda})$ is the effective dimension of kernel $K$. Similarly, we have 
\begin{align}
\bd_{n} &= \mathbb{E}(\bDelta_{n}\by \mid \bx) + \mathcal{O}_{p}\left( \wt{\lambda} + \sqrt{\frac{ \gamma(\wt{\lambda})  }{n} }\right)\bone_{n} \notag \\
&= \bPi\bmm + \mathcal{O}_{p}\left( \wt{\lambda} + \sqrt{\frac{ \gamma(\wt{\lambda})  }{n} } \right )\bone_{n}.\notag
\end{align}
Consequently, letting $a_n = \wt{\lambda} +  \sqrt{\gamma(\wt{\lambda})/n} $ and applying 
 Taylor expansion, we have
\begin{align}
 \wh{\bmm} &= \bmm + \bPi^{-1} \left(    \bd_{n} - \bC_{n}\bmm      \right) + o_{p}\left( a_n  \right )\bone_{n}\notag\\ 
 &= \bmm + \bPi^{-1} \left\{   \bS_{\lambda}\bDelta_{n}\by - \bS_{\lambda}\left( \bDelta_{n} + \lambda\bK^{-1}   \right)\bmm      \right\} \notag\\
 &\quad+ o_{p}\left( a_n  \right )\bone_{n}\notag \\
 &= \bmm + \bPi^{-1} \bS_{\lambda}\bDelta_{n}  \left(  \by - \bmm      \right) + \mathcal{O}_{p}\left(a_n  \right )\bone_{n},\notag
 \end{align}
where the last equality holds because 
\begin{align}
\bS_{\lambda} \lambda\bK^{-1}\bmm &= \bS_{\lambda}\left\{ \left( \bI_{n} + \lambda\bK^{-1} \right) - \bI_{n}  \right\}\bmm \notag \\
&= \bmm -  \bS_{\lambda} \bmm = \mathcal{O}_{p}\left( a_n  \right ). \notag 
\end{align}
%where the last equality is by Lemma \ref{order}.

Therefore, we have
 \begin{align}
T_{n} &= n^{-1} \bone\trans \left(\bI_{n} - \bDelta_{n}\right) (\wh{\bmm} - \bmm) \notag \\
&= {n}^{-1} \bone\trans \left(\bI_{n} - \bDelta_{n}\right) \bPi^{-1} \bS_{\lambda}\bDelta_{n}  \left(  \by - \bmm      \right) +  \mathcal{O}_{p}\left( a_n  \right)\notag \\
& = n^{-1} \bone\trans \left(\bI_{n} -  \bPi \right) \bPi^{-1} \bDelta_{n}  \left(  \by - \bmm      \right) +  \mathcal{O}_{p}\left( a_n  \right) \notag \\
 &= n^{-1} \bone\trans \left(\bPi^{-1} - \bI_{n}\right) \bDelta_{n}  \left(  \by - \bmm      \right) +  \mathcal{O}_{p}\left( a_n  \right)
\notag. 
\end{align}


By Corollary 5 in \citet{zhang2013divide}, for $\ell$-th order of Sobolev space, we have
\begin{align}
  \gamma(\wt{\lambda})  &= \sum_{j=1}^{\infty} \frac{1}{1 + j^{2\ell}\wt{\lambda} } \notag\\
  &\leq \wt{\lambda}^{-\frac{1}{2\ell}}  + \sum_{j >\wt{\lambda} ^{-\frac{1}{2\ell}}} \frac{1}{ 1 + j^{2\ell} \wt{\lambda}} \notag\\
  &\leq \wt{\lambda}^{-\frac{1}{2\ell}} + \wt{\lambda}^{-1} \int_{\wt{\lambda}^{-\frac{1}{2\ell}}}^{\infty}z dz \notag \\
  &= \wt{\lambda}^{-\frac{1}{2\ell}}  + \frac{1}{2\ell-1}\wt{\lambda} ^{-\frac{1}{2\ell}} \notag\\
  &= O\left(\wt{\lambda}^{-\frac{1}{2\ell}}\right).
\end{align}
Consequently, as long as $\wt{\lambda}^{-\frac{1}{2\ell}} / n = o(1)$ and $\wt{\lambda} = o(n^{-1/2})$, we have
\begin{align} 
T_{n} &=  \frac{1}{n}\bone\trans \left(\bPi^{-1} - \bI_{n}\right) \bDelta_{n}  \left(  \by - \bmm      \right) + o_{p}(n^{-1/2}).
\end{align}
One legitimate of such $\wt{\lambda}$ can be chosen as $n^{-\ell}$, i.e., $\lambda = \mathcal{O}(n^{1-\ell})$.

\subsection*{B. Computational Details }


As the objective function in \eqref{entropy_method} is convex \citep{nguyen2010}, we apply the limited-memory Broyden-Fletcher-Goldfarb-Shanno (L-BFGS) algorithm to solve 
the optimization problem with the following first order partial derivatives:
\begin{align}
  \frac{\partial U }{\partial \alpha_{0}} = & 
   \frac{1}{n_{1}}\sum_{i=1}^{n}\mathbb{I}(\delta_{i} = 0)\exp\left( \alpha_{0} + \sum_{j=1}^{n}\alpha_{j}K(x_{i}, x_{j})  \right) - 1,\notag\\
   \frac{\partial U }{\partial \alpha_{k}} = & \frac{1}{n_{1}}\sum_{i=1}^{n}\mathbb{I}(\delta_{i} = 0)k(x_{i}, x_{k})\exp\left( \alpha_{0} + \sum_{j=1}^{n}\alpha_{j}K(x_{i}, x_{j})  \right) 
   - \frac{1}{n_{0}}\sum_{i=1}^{n}K(x_{i}, x_{k}) \notag\\
   &+ 2  \tau\sum_{i=1}^{n}K(x_{i}, x_{k})\alpha_{i}, k = 1, \ldots, n. \notag
\end{align}



For tuning parameter selection $\tau$ in (\ref{18}),  we adopt a cross-validation (CV) strategy. In particular, we may firstly stratify the sample $S = \{1, \ldots, n\}$ into two strata 
	$S_{0} =\{i\in S: \delta_{i} = 0\}$  and $S_{1} =\{i\in S: \delta_{i} = 1\}$. 
Within each $S_{h}$, we make $K$ random partition $\mathcal{A}_{k}^{(h)}$ such that
\begin{align}
\begin{gathered}
     \bigcup_{k=1}^{K}\mathcal{A}_{k}^{(h)} = S_{h}, h = 0, 1\notag\\
     \mathcal{A}_{k_{1}}^{(h)} \bigcap \mathcal{A}_{k_{2}}^{(h)} =  \emptyset, k_{1} \neq k_{2}, k_{1}, k_{2} \in \{1, \ldots, K\}, \notag\\
     \Abs{\mathcal{A}_{1}}^{h} \approx  \Abs{\mathcal{A}_{2}}^{(h)} \approx \cdots \approx \Abs{\mathcal{A}_{K}}^{(h)},  h = 0, 1,\notag
\end{gathered}
\end{align}
where $|\cdot|$ is the cardinality of a specific set. For a fixed $\tau > 0$, the corresponding CV criterion is
\begin{align}\label{cv}
	\mbox{CV}(\tau) = \frac{1}{K}\sum_{k=1}^{K}\sum_{j\in \mathcal{A}_{k}} \tilde{L}(\delta_{j},  \hat{g}^{(-k)}(x_{j}, \tau) ),
\end{align}
where $\hat{g}^{(-k)}$ is the trained model with data with data points except for $\mathcal{A}_{k} = \mathcal{A}_{k}^{(0)} \cup  \mathcal{A}_{k}^{(1)}$. Regarding the loss function in \eqref{cv}, we can use
\begin{align}
	\tilde{L}(\delta; \hat{g}) = \mathbb{I}(\delta = 1, \hat{p}(x) < 0.5   ) + \mathbb{I}(\delta = 0, \hat{p}(x) > 0.5   ),\notag
\end{align}
where $ 
	\hat{p}(x) = n_{1} /\{  n_{1} + n_{0}\hat{g}(x)    \} $ 
as an estimator for $p(x) = Pr(\delta = 1 \mid x)$. As a result, we may select the tuning parameter $\tau$ which minimizes the CV criteria in \eqref{cv}.


%\IEEEraisesectionheading{\section{Introduction}}

\IEEEPARstart{V}{ision} system is studied in orthogonal disciplines spanning from neurophysiology and psychophysics to computer science all with uniform objective: understand the vision system and develop it into an integrated theory of vision. In general, vision or visual perception is the ability of information acquisition from environment, and it's interpretation. According to Gestalt theory, visual elements are perceived as patterns of wholes rather than the sum of constituent parts~\cite{koffka2013principles}. The Gestalt theory through \textit{emergence}, \textit{invariance}, \textit{multistability}, and \textit{reification} properties (aka Gestalt principles), describes how vision recognizes an object as a \textit{whole} from constituent parts. There is an increasing interested to model the cognitive aptitude of visual perception; however, the process is challenging. In the following, a challenge (as an example) per object and motion perception is discussed. 



\subsection{Why do things look as they do?}
In addition to Gestalt principles, an object is characterized with its spatial parameters and material properties. Despite of the novel approaches proposed for material recognition (e.g.,~\cite{sharan2013recognizing}), objects tend to get the attention. Leveraging on an object's spatial properties, material, illumination, and background; the mapping from real world 3D patterns (distal stimulus) to 2D patterns onto retina (proximal stimulus) is many-to-one non-uniquely-invertible mapping~\cite{dicarlo2007untangling,horn1986robot}. There have been novel biology-driven studies for constructing computational models to emulate anatomy and physiology of the brain for real world object recognition (e.g.,~\cite{lowe2004distinctive,serre2007robust,zhang2006svm}), and some studies lead to impressive accuracy. For instance, testing such computational models on gold standard controlled shape sets such as Caltech101 and Caltech256, some methods resulted $<$60\% true-positives~\cite{zhang2006svm,lazebnik2006beyond,mutch2006multiclass,wang2006using}. However, Pinto et al.~\cite{pinto2008real} raised a caution against the pervasiveness of such shape sets by highlighting the unsystematic variations in objects features such as spatial aspects, both between and within object categories. For instance, using a V1-like model (a neuroscientist's null model) with two categories of systematically variant objects, a rapid derogate of performance to 50\% (chance level) is observed~\cite{zhang2006svm}. This observation accentuates the challenges that the infinite number of 2D shapes casted on retina from 3D objects introduces to object recognition. 

Material recognition of an object requires in-depth features to be determined. A mineralogist may describe the luster (i.e., optical quality of the surface) with a vocabulary like greasy, pearly, vitreous, resinous or submetallic; he may describe rocks and minerals with their typical forms such as acicular, dendritic, porous, nodular, or oolitic. We perceive materials from early age even though many of us lack such a rich visual vocabulary as formalized as the mineralogists~\cite{adelson2001seeing}. However, methodizing material perception can be far from trivial. For instance, consider a chrome sphere with every pixel having a correspondence in the environment; hence, the material of the sphere is hidden and shall be inferred implicitly~\cite{shafer2000color,adelson2001seeing}. Therefore, considering object material, object recognition requires surface reflectance, various light sources, and observer's point-of-view to be taken into consideration.


\subsection{What went where?}
Motion is an important aspect in interpreting the interaction with subjects, making the visual perception of movement a critical cognitive ability that helps us with complex tasks such as discriminating moving objects from background, or depth perception by motion parallax. Cognitive susceptibility enables the inference of 2D/3D motion from a sequence of 2D shapes (e.g., movies~\cite{niyogi1994analyzing,little1998recognizing,hayfron2003automatic}), or from a single image frame (e.g., the pose of an athlete runner~\cite{wang2013learning,ramanan2006learning}). However, its challenging to model the susceptibility because of many-to-one relation between distal and proximal stimulus, which makes the local measurements of proximal stimulus inadequate to reason the proper global interpretation. One of the various challenges is called \textit{motion correspondence problem}~\cite{attneave1974apparent,ullman1979interpretation,ramachandran1986perception,dawson1991and}, which refers to recognition of any individual component of proximal stimulus in frame-1 and another component in frame-2 as constituting different glimpses of the same moving component. If one-to-one mapping is intended, $n!$ correspondence matches between $n$ components of two frames exist, which is increased to $2^n$  for one-to-any mappings. To address the challenge, Ullman~\cite{ullman1979interpretation} proposed a method based on nearest neighbor principle, and Dawson~\cite{dawson1991and} introduced an auto associative network model. Dawson's network model~\cite{dawson1991and} iteratively modifies the activation pattern of local measurements to achieve a stable global interpretation. In general, his model applies three constraints as it follows
\begin{inlinelist}
	\item \textit{nearest neighbor principle} (shorter motion correspondence matches are assigned lower costs)
	\item \textit{relative velocity principle} (differences between two motion correspondence matches)
	\item \textit{element integrity principle} (physical coherence of surfaces)
\end{inlinelist}.
According to experimental evaluations (e.g.,~\cite{ullman1979interpretation,ramachandran1986perception,cutting1982minimum}), these three constraints are the aspects of how human visual system solves the motion correspondence problem. Eom et al.~\cite{eom2012heuristic} tackled the motion correspondence problem by considering the relative velocity and the element integrity principles. They studied one-to-any mapping between elements of corresponding fuzzy clusters of two consecutive frames. They have obtained a ranked list of all possible mappings by performing a state-space search. 



\subsection{How a stimuli is recognized in the environment?}

Human subjects are often able to recognize a 3D object from its 2D projections in different orientations~\cite{bartoshuk1960mental}. A common hypothesis for this \textit{spatial ability} is that, an object is represented in memory in its canonical orientation, and a \textit{mental rotation} transformation is applied on the input image, and the transformed image is compared with the object in its canonical orientation~\cite{bartoshuk1960mental}. The time to determine whether two projections portray the same 3D object
\begin{inlinelist}
	\item increase linearly with respect to the angular disparity~\cite{bartoshuk1960mental,cooperau1973time,cooper1976demonstration}
	\item is independent from the complexity of the 3D object~\cite{cooper1973chronometric}
\end{inlinelist}.
Shepard and Metzler~\cite{shepard1971mental} interpreted this finding as it follows: \textit{human subjects mentally rotate one portray at a constant speed until it is aligned with the other portray.}



\subsection{State of the Art}

The linear mapping transformation determination between two objects is generalized as determining optimal linear transformation matrix for a set of observed vectors, which is first proposed by Grace Wahba in 1965~\cite{wahba1965least} as it follows. 
\textit{Given two sets of $n$ points $\{v_1, v_2, \dots v_n\}$, and $\{v_1^*, v_2^* \dots v_n^*\}$, where $n \geq 2$, find the rotation matrix $M$ (i.e., the orthogonal matrix with determinant +1) which brings the first set into the best least squares coincidence with the second. That is, find $M$ matrix which minimizes}
\begin{equation}
	\sum_{j=1}^{n} \vert v_j^* - Mv_j \vert^2
\end{equation}

Multiple solutions for the \textit{Wahba's problem} have been published, such as Paul Davenport's q-method. Some notable algorithms after Davenport's q-method were published; of that QUaternion ESTimator (QU\-EST)~\cite{shuster2012three}, Fast Optimal Attitude Matrix \-(FOAM)~\cite{markley1993attitude} and Slower Optimal Matrix Algorithm (SOMA)~\cite{markley1993attitude}, and singular value decomposition (SVD) based algorithms, such as Markley’s SVD-based method~\cite{markley1988attitude}. 

In statistical shape analysis, the linear mapping transformation determination challenge is studied as Procrustes problem. Procrustes analysis finds a transformation matrix that maps two input shapes closest possible on each other. Solutions for Procrustes problem are reviewed in~\cite{gower2004procrustes,viklands2006algorithms}. For orthogonal Procrustes problem, Wolfgang Kabsch proposed a SVD-based method~\cite{kabsch1976solution} by minimizing the root mean squared deviation of two input sets when the determinant of rotation matrix is $1$. In addition to Kabsch’s partial Procrustes superimposition (covers translation and rotation), other full Procrustes superimpositions (covers translation, uniform scaling, rotation/reflection) have been proposed~\cite{gower2004procrustes,viklands2006algorithms}. The determination of optimal linear mapping transformation matrix using different approaches of Procrustes analysis has wide range of applications, spanning from forging human hand mimics in anthropomorphic robotic hand~\cite{xu2012design}, to the assessment of two-dimensional perimeter spread models such as fire~\cite{duff2012procrustes}, and the analysis of MRI scans in brain morphology studies~\cite{martin2013correlation}.

\subsection{Our Contribution}

The present study methodizes the aforementioned mentioned cognitive susceptibilities into a cognitive-driven linear mapping transformation determination algorithm. The method leverages on mental rotation cognitive stages~\cite{johnson1990speed} which are defined as it follows
\begin{inlinelist}
	\item a mental image of the object is created
	\item object is mentally rotated until a comparison is made
	\item objects are assessed whether they are the same
	\item the decision is reported
\end{inlinelist}.
Accordingly, the proposed method creates hierarchical abstractions of shapes~\cite{greene2009briefest} with increasing level of details~\cite{konkle2010scene}. The abstractions are presented in a vector space. A graph of linear transformations is created by circular-shift permutations (i.e., rotation superimposition) of vectors. The graph is then hierarchically traversed for closest mapping linear transformation determination. 

Despite of numerous novel algorithms to calculate linear mapping transformation, such as those proposed for Procrustes analysis, the novelty of the presented method is being a cognitive-driven approach. This method augments promising discoveries on motion/object perception into a linear mapping transformation determination algorithm.


 
%\section{Method}\label{sec:method}
%

\subsection{Interaction-aware Human-Object Capture}\label{sec:human_capture}
Classical multi-view stereo reconstruction approaches \citep{Furukawa2013,Strecha2008,Newcombe2011,collet2015high} and recent neural rendering approaches \citep{Wu_2020_CVPR,NeuralVolumes,nerf} rely on multi-view dome based setup to achieve high-fidelity reconstruction and rendering results.
%
However, they suffer from both sparse-view inputs and occlusion of objects.
%
To this end, we propose a novel implicit human-object capture scheme to model the mutual influence between human and object from only sparse-view RGB inputs.

\noindent{\textbf{(a) Occlusion-aware Implicit Human Reconstruction.}}
For the human reconstruction, we perform a neural implicit geometry generation to jointly utilize both the pixel-aligned image features and global human motion priors with the aid of an occlusion-aware training data augmentation.
% 

Without dense RGB cameras and depth cameras, traditional multi-view stereo approaches \citep{collet2015high,motion2fusion} and depth-fusion approaches \citep{KinectFusion,UnstructureLan,robustfusion} can hardly reconstruct high-quality human meshes.
%
With implicit function approaches \citep{PIFU_2019ICCV,PIFuHD}, we can generate fine-detailed human meshes with sparse-view RGB inputs.
%
However, the occlusion from human-object interaction can still cause severe artifacts.
%
To end this, we thus utilize the pixel-aligned image features and global human motion priors.

% 
Specifically, we adopt the off-the-shelf instance segmentation approach \citep{Bolya_2019_ICCV} to obtain human and object masks, thus distinguishing the human and object separately from the sparse-view RGB input streams.
%
Meanwhile, we apply the parametric model estimation to provide human motion priors for our implicit human reconstruction.
%
We voxelized the mesh of this estimated human model to represent it with a volume field.

We give both the pixel-aligned image features and global human motion priors in volume representation to two different encoders of our implicit function, as shown in Fig. \ref{fig:pipeline} (a).
%
Different from \cite{2020phosa_Arrangements} with only a single RGB input, we use pixel-aligned image features from the multi-view inputs and concatenate them with our encoded voxel-aligned features.
%
We finally decode the pixel-aligned and voxel-aligned feature to occupancy values with a multilayer perceptron (MLP).

For each query 3D point $P$ on the volume grid, we follow PIFu \citep{saito2019pifu} to formulate the implicit function $f$ as:
\begin{align}
	f( \Phi(P),\Psi(P),Z(P)) & = \sigma : \sigma \in [0.0, 1.0],             \\
	\Phi(P)                  & = \frac{1}{n} \sum_{i}^{n}F_{I_{i}}(\pi_{i}(P)), \\
	\Psi(P)                  & =  G(F_{V},P),
\end{align}
where $p = \pi_{i}(P)$ denotes the projection of 3D point to camera view $i$, $F_{I_{i}}(x)= g(I_{i}(p))$ is the image feature at $p$.
%
$\Psi(P) = G(F_{V},P)$ denotes the voxel aligned features at $P$, $F_{V}$ is the voxel feature.
%
To better deal with occlusion, we introduce an occlusion-aware reconstruction loss to enhance the prediction at the occluded part of human.
% 
It is formulated as:
	\begin{align}
		 & \mathcal{L}_{\sigma} = \lambda_{occ}\sum_{t=1}^T \left\| \sigma_{occ}^{gt} - \sigma_{occ}^{pred} \right\|_2^2 + \lambda_{vis}\sum_{t=1}^T \left\| \sigma_{vis}^{gt} - \sigma_{vis}^{pred} \right\|_2^2.
	\end{align}

% 
Here, $\lambda_{occ}$ and $\lambda_{vis}$ represent the weight of occlusion points and visible points, respectively.
%
$\sigma_{occ}$ and $\sigma_{vis}$ are the training sampling points at the occlusion area and visible area.


\begin{figure}[t]
    \centering
    \includegraphics[width=\linewidth]{figures/data_augmentation}
    \vspace{-10pt}
    \caption{Illustration of our synthetic 3D data with both human and objects.}
    % \vspace{-1mm}
    \vspace{-15pt}
    \label{fig:DataAugmentation}
\end{figure}

\begin{figure*}[t]
	\centering
	\includegraphics[width=\linewidth]{figures/pipeline_net}
	\caption{Illustration of our layered human-object rendering approach, which not only includes a direction-aware neural texture blending scheme to encode the occlusion information explicitly but also adopts a spatial-temporal texture completion for the occluded regions based on the human motion priors.}
	\vspace{-10pt}
	\label{fig:pipeline_net}
\end{figure*}

For the detail of the parametric model estimation, we fit the parametric human model, SMPL \citep{SMPL2015}, to capture occluded human with the predicted 2D keypoints.
%
Specifically, we use Openpose \citep{Openpose} as our joint detector to estimate 2D human keypoints from sparse-view RGB inputs.
%
To estimate the pose/shape parameters of SMPL as our human prior for occluded human, we formulate the energy function $\boldsymbol{E}_{\mathrm{prior}}$ of this optimization as:
\begin{align} \label{eq:opt}
	\boldsymbol{E}_{\mathrm{prior}}(\boldsymbol{\theta}_t, \boldsymbol{\beta}) = \boldsymbol{E}_{\mathrm{2D}} + \lambda_{\mathrm{T}}\boldsymbol{E}_{\mathrm{T}}
\end{align}
% 
Here, $\boldsymbol{E}_{\mathrm{2D}}$ represents the re-projection constraint on 2D keypoints detected from sparse-view RGB inputs, while $\boldsymbol{E}_{\mathrm{T}}$ enforces the final pose and shape to be temporally smooth.
%
$\boldsymbol{\theta}_t$ is the pose parameters of frame $t$, while $\boldsymbol{\beta}$ is the shape parameters.
%
Note that this temporal smoothing enables globally consistent capture during the whole sequence, and benefits the parametric model estimation when some part of the body is gradually occluded.
%
We follow \cite{he2021challencap} to formulate the 2D term $\boldsymbol{E}_{\mathrm{2D}}$ and the temporal term $\boldsymbol{E}_{\mathrm{T}}$ under the sparse-view setting.
%

Moreover, we apply an occlusion-aware data augmentation to reduce the domain gap between our training set and the challenging human-object interaction testing set.
%
Specially, we randomly sample some objects from ShapeNet dataset~\cite{chang2015shapenet}.
%
We then randomly rotate and place them around human before training, as shown in Fig. \ref{fig:DataAugmentation}.
%
By simulating the occlusion of human-object interaction, our network is more robust to occluded human features.

With both the pixel-aligned image features and the statistical human motion priors under this occlusion-aware data augmentation training, our implicit function generates high-quality human meshes with only spare RGB inputs and occlusions from human-object interaction.

\noindent{\textbf{(b) Human-aware Object Tracking.}}
%
For the objects around the human, people recover them from depth maps~\cite{new2011kinect}, implicit fields~\cite{mescheder2019occupancy}, or semantic parts~\cite{chen2018autosweep}. We perform a template-based object alignment for the first frame and human-aware tracking to maintain temporal consistency and prevent the segmentation uncertainty caused by interaction. With the inspiration of PHOSA \cite{2020phosa_Arrangements}, we consider each object as a rigid body mesh.


To faithfully and robustly capture object in 3D space as time going, we introduce a human-aware tracking method.
%
Expressly, we assume objects are rigid bodies and transforming rigidly in the human-object interaction activities.
%
So the object mesh $O_{t}$ at frame $t$ can be represented as: $O_{t} = R_{t}O_{t-1}+T_{t}$.
% 
Based on the soft rasterization rendering~\cite{ravi2020accelerating}, the rotation $R_{t}$ and the translation $T_{t}$ can be naively optimized by comparing $\mathcal{L}_{2}$ norm between the rendered silhouette $S_{t}^{i}$ and object mask $\mathcal{M}o_{t}^{i}$.
%

Human is also an important cue to locate the object position.
%
From the 2D perspective, when objects are occluded by the human at a camera view, the  $\mathcal{L}_{2}$ loss between rendered silhouette and occluded mask will lead to the wrong object location due to the wrong guidelines at the occluded area.
%
So we remove the occluded area affected by human mask $\mathcal{M}h_{t}^{i}$ when computing the $\mathcal{L}_{2}$ loss.
%
From the 3D perspective, human can not interpenetrate an rigid object, so we also add an interpenetration loss $\mathcal{L}_{P}$~\cite{jiang2020mpshape} to regularize optimization. Our total object tracking loss is:
\begin{align}
	\mathcal{L}_{track} = \lambda_{1}\sum_{i=0}^{n}\| \mathcal{B}(\mathcal{M}h_{t}^{i}==0) \odot  S_{t}^{i} - \mathcal{M}o_{t}^{i}  \|  + \lambda_{2}\mathcal{L}_{P},
\end{align}
where n denotes view numbers, $\lambda_{1}$ denotes weight of silhouette loss, $\lambda_{2}$ denotes weight of interpenetration loss, $\mathcal{B}$ represents an binary operation, it returns 0 when the condition is true, else 1.
% 	}

Our implicit human-object capture utilizes both the pixel-aligned image features and global human motion priors with the aid of an occlusion-aware training data augmentation, and captures objects with the template-based alignment and the human-aware tracking to maintain temporal consistency and prevent the segmentation uncertainty caused by interaction. Thus, our approaches can generate high-quality human-object geometry with sparse inputs and occlusions.

\subsection{Layered Human-Object Rendering}\label{sec:rendering}
We introduce a neural human-object rendering pipeline to encode local fine-detailed human geometry and texture features from adjacent input views, so as to produce photo-realistic layered output in the target view, as illustrated in Fig. \ref{fig:pipeline_net}.

\begin{figure*}[t]
	\centering
	\includegraphics[width=\linewidth]{figures/gallery}
	\vspace{-20pt}
	\caption{The geometry and texture results of our proposed approach, which generates photo-realistic rendering results and high fidelity geometry on a various of sequences, such as rolling a box, playing with balls.}
    \vspace{-10pt}
	\label{fig:gallery}
\end{figure*}

\noindent{\textbf{(c) Direction-aware Neural Texture Blending.}} \label{sec:neuralBlending}
%
While traditional image-based rendering approaches always show the artifacts with the sparse-view texture blending, we follow \cite{NeuralHumanFVV2021CVPR} to propose a direction-aware neural texture blending approach to render photo-realistic human in the novel view.
% %
For a novel view image $I_{n}$, the linear combination of two source view $I_{1}$ and $I_{2}$ with blending weight map $W$ is formulate as:
\begin{align}
	I_{n} = W \cdot I_{1} + (1 - W) \cdot I_{2}.
\end{align}
However, in the sparse-view setting, the neural blending approach \citep{NeuralHumanFVV2021CVPR} can still generate unsmooth results. As the reason of these artifacts, the imbalance of angles between two source views with a novel view will lead to the imbalance wrapped image quality.
%

Different from \citet{NeuralHumanFVV2021CVPR}, we thus propose a direction-aware neural texture blending to eliminate such artifacts, as shown in Fig. \ref{fig:pipeline_net}.
%
The direction and angle between the two source views and target view will be an important cue for neural rendering quality, especially under occluded scenarios. 
%
Given novel view depth $D_{n}$ and source view depth $D_{1}$, $D_{2}$, we wrap them to the novel view $D_{1}^{n}$ and $D_{2}^{n}$, then compute the occlusion map $O_{i} = D_{n}- D_{i}^{n} (i=1,2)$.
%
Then, we unproject $D_{i}$ to point-clouds.
%
For each point $P$, we compute the cosine value between $\overrightarrow{c_{i}P}$ and $\overrightarrow{c_{n}P}$ to get angle map $A_{i}$, where $c_{i}$ denotes the optical center of source camera $i$, $c_{n}$ denotes the optical center of novel view camera.
%
Thus, we introduce a direction-aware blending network $\Theta_{DAN}$ to utilize global feature from image and local feature from human geometry to generate the blending weight map $W$, which can be formulated as:
\begin{align}
	W = \Theta_{DAN}(I_{1},O_{1},A_{1},I_{2},O_{2},A_{2}),
\end{align}
% 

\noindent{\textbf{(d) Spatial-temporal Texture Completion.}}
%
While human-object interaction activities consistently lead to occlusion, the missing texture on human, therefore, leads to severe artifacts for free-viewpoint rendering.
%
To end this, we propose a spatial-temporal texture completion method to generate a texture-completed proxy in the canonical human space.
%
We use the temporal and spatial information to complete the missing texture at view $i$ and time $t$ from different times and different views.

Specifically, we first use the non-rigid deformation to register an up-sampled SMPL model (41330 vertices) with the captured human meshes.
%	
Then, for each point on the proxy, we find the nearest visible points along with all views and all frames, then assign an interpolation color to this point.
% 
We thus generate a canonical human space with the fused texture.
%
For the occluded part of human in novel view, we render the texture-completed image and blend it with the neural rendering results in Sec. \ref{sec:neuralBlending} (c).

We utilize a layered human-object rendering strategy to render human-object together with the reconstruction and tracking of object.
%
For each frame, we render human with our novel neural texture blending while rendering objects through a classical graphics pipeline with color correction matrix (CCM).
%
To combine human and object rendering results, we utilize the depth buffer from the geometry of our human-object capture.
%

\noindent{\textbf{Training Strategy.}} To enable our sparse-view neural human performance rendering under human-object interaction, we need to train the direction-aware blending network $\Theta_{DAN}$ properly.
% 

We follow \citet{NeuralHumanFVV2021CVPR} to utilize 1457 scans from the Twindom dataset \cite{Twindom} to train our DAN $\Theta_{DAN}$ properly.
%
Differently, we randomly place the performers inside the virtual camera views and augment this dataset by randomly placing some objects from ShapeNet dataset~\cite{chang2015shapenet}.
%
By simulating the occlusion of human-object interaction, we make our network more robust to occluded human.
%
Our training dataset contains RGB images, depth maps and angle maps for all the views and models.

For the training of our direction-aware blending network $\Theta_{DAN}$, we set out to apply the following learning scheme to enable more robust blending weight learning.
%
The appearance loss function with the perceptual term ~\cite{Johnson2016Perceptual} is to make the blended texture as close as possible to the ground truth, which is formulated as:
\begin{align}
	\left.\mathcal{L}_{r g b}=\frac{1}{n} \sum_{j}^{n}
	\left(
	\left\|I_{r}^{j}-I_{g t}^{j}\right\|_{2}^{2}
% 	\right)
	+\left\|\varphi
	\left(
	I_{r}^{j}
	\right)
	-\varphi
	\left(
	I_{g t}^{j}
	\right)
	\right\|_{2}^{2}
	\right) \right.
\end{align}
where $I_{g t}$ is the ground truth RGB images; $\varphi(\cdot)$ denotes the output features of the third-layer of pre-trained VGG-19.

With the aid of such occlusion analysis, our texturing scheme maps the input adjacent images into a photo-realistic texture output of human-object activities in the target view through efficient blending weight learning, without requiring further per-scene training.

%\include{Theory} 
%\documentclass[a4paper,twoside]{article}

\usepackage{epsfig}
\usepackage{subfigure}
\usepackage{calc}
\usepackage{amssymb}
\usepackage{amstext}
\usepackage{amsmath}
\usepackage{amsthm}
\usepackage{comment}
\usepackage{bm}
\usepackage{multirow}
\usepackage{cite}
\usepackage{multicol}
\usepackage{pslatex}
\DeclareMathOperator*{\argmax}{argmax}
\usepackage{apalike}
\usepackage{SCITEPRESS}     % Please add other packages that you may need BEFORE the SCITEPRESS.sty package.

\subfigtopskip=0pt
\subfigcapskip=0pt
\subfigbottomskip=0pt

\begin{document}

\title{Efficient and Effective Single-Document Summarizations and A Word-Embedding Measurement of Quality}


\author{\authorname{Liqun Shao, Hao Zhang, Ming Jia and Jie Wang}
\affiliation{Department of Computer Science, University of Massachusetts, Lowell, MA, USA}
\email{\{Liqun\_Shao, Hao\_Zhang, Ming\_Jia\}@student.uml.edu, Jie\_Wang@uml.edu}
}

\keywords{Single-Document Summarizations, Keyword Ranking, Topic Clustering, Word Embedding, SoftPlus Function, Semantic Similarity, Summarization Evaluation, Realtime.}

\abstract{
Our task is to generate an effective summary for a given document with specific realtime requirements. %In our applications there is no labeled data available for training a model.
%We study how to extract efficiently a summary from a single document within a given length boundary to capture the main topics of the document and be human readable.
%Following the common approach to extract multiple key sentences as coherent as possible and cover the major topics of the original document
%as diverse as possible,
We use the softplus function %$\ln(1+e^x)$
to enhance keyword rankings to favor important sentences,
%that are more important.
based on which we present a number of summarization algorithms using various
keyword extraction and topic clustering methods. We show %called ERAKE and ERAKE-TT % using RAKE keyword extractions and topic clusterings
%two algorithms called RAKEN and RAKETT %using word-pair co-occurrences to identify keyword phrases and TextTiling to promote coverage of diverse topics,
that our algorithms meet the realtime requirements and yield the best ROUGE recall scores on DUC-02 over all previously-known
algorithms. % are better than the previously-known best algorithm.
%For instance, our algorithm ET3Rank generates the scores of (49.2, 25.6, 27.5) for (ROUGE-1, ROUGE-2, ROUGE-SU4),
%
%
%the (ROUGE-1, ROUGE-2, ROUGE-SU4) scores over DUC-02 of our algorithm ET3Rank are (49.2, 25.6, 27.5),
%while the previously-reported best results are (49.0, 24.7, 25.8).
%
%In particular, both RANEN and RAKETT
%have higher ROUGE scores over DUC-02 and meet the realtime requirement, with RAKETT slightly better on ROUGE scores
%and RAKEN faster in runtime.
%The ROUGE measure requires summaries written by human experts as benchmarks for comparisons by NLP algorithms.
%This presents an obstacle when dealing with a large number of documents of various topics and lengths, for it is infeasible to produce a sufficient number %of benchmarks.
%To overcome this barrier,
To evaluate the quality of summaries %against the original documents
without human-generated benchmarks, % of particular topics,
we define a measure called WESM based on word-embedding using %a word-embedding method to measure the quality of a summary. In particular, we
%to use
Word Mover's Distance.
% over paragraphs, denoted by WESM,
%to measure similarities between a summary and its original document.
%and
We show that
%the best and the second best algorithms remain the same under both average ROUGE and WESM over different datasets, and
the orderings of the ROUGE and WESM scores of our algorithms are highly comparable, %, with the average $L_1$-norm equal to 0.4. %Moreover, we show that
%the orderings over DUC-02 and NewsIR-16 are also comparable, with the average $L_1$-norm equal to 2.
%where
%over DUC-02 have a similar ordering of the WEP scores as that of the ROUGE scores. %We argue that WEP may serve as a viable alternative
%for evaluating the quality of a summary.
%We show that under WEP, these algorithms have almost the same orderings over the NewsIR-16 and DUC-02 datasets.
%that ofsimilar toand
%Moreover, the best algorithm  also ranks the highest under WEP over both NewsIR-16 and DUC-02.
suggesting that WESM may serve as a viable alternative for measuring the quality of a summary.
%over the NewsIR-16 dataset,
% is similar to the order of ROGUE rankings produced by these algorithms over the DUC 2012 dataset.
%in place of ROGUE when human-generated benchmarks are not available for comparisons.
%Our WMD method obtains competitive performance and our summarization system outperforms prior work on ROUGE.
}

\onecolumn \maketitle \normalsize \vfill

\section{\uppercase{Introduction}}
\label{sec:introduction}
\noindent Text summarization algorithms have been studied intensively and extensively.
%For practical purposes,
An effective summary must
be human-readable and %is considered effective if it
convey the central meanings of the original document within
a given length boundary. % and is human readable.
% over the last several decades.
%A number of unsupervised and supervised algorithms have been devised to achieve these goals.
%To capture the key points of a given document and achieve coherence,
The common approach of unsupervised summarization algorithms
extracts sentences based on importance rankings (e.g., see \cite{DUC02,Mihalcea04,Rose:10,lin:11,ParveenR015}),
where a keyword may also
be a phrase. %Sentence extraction in the original order of the document also produces reasonable coherence. % is a sequence of one or more words that provides a compact representation of the content in the document.
A sentence with a larger number of keywords of higher ranking scores is considered more important %a candidate
for extraction.
Supervised algorithms include
%Attempts were also made to use
CNN and RNN models % in recent years
for generating extractive and abstractive summaries (e.g., see \cite{RushCW15,NallapatiZSGX16,ChengL16a}). %List the 3 papers published in 2016
%These algorithms
%depend on large sets of labeled data to train models, and their ROUGE scores over DUC data are yet to catch up with unsupervised algorithms.
%Significant advancements were made in recent years in supervised and unsupervised summarization algorithms
%over a single document, multiple documents, and queries, including CNNs \cite{}
%
%An adequate summary for a given document should succinctly convey the central meanings for the document. Summarization systems generate a concise document as %output and take a long document as input. Several summarization variants
%

%In a project on
We were asked to construct a general-purpose text-automation tool to produce, among other things, an effective
summary for a given document with the following realtime requirements: Generate a summary instantly for a document of
up to 2,000 words, under 1 second for
a document of slightly over 5,000 words, and under 3 seconds for a very long document of around 10,000 words.
Moreover, we need to deal with documents of arbitrary topics without knowing what the topics are in advance.
%and we would like to measure the quality of a summary directly against the original document without
After investigating all existing summarization algorithms,
we conclude that unsupervised single-document summarization algorithms would be the best approach to meeting
our requirements. %We show that by modifying the common approach of sentence extractions, we can achieve both efficiency and effectiveness.
%learning is our only choice.

%Methods can be characterized as supervised and unsupervised, and summaries may be obtained over a single document, multiple documents, and based on queries.
%can be query based, multi document, and single document.
%For the above different summarization types, the basic requirements remain the same, which contain salient information and let readers not miss anything from the original document. Readers are not %interested in redundant information, so summaries should cover many diverse topics. Finally, summaries should represent main information, be concise and cover different topics.





\begin{comment}
Recent methods used sentence compression to convert a sentence into a shorter sentence or phrase, while trying to maintain syntactic correctness. For example, Turner and Charniak ~\shortcite{Turner:05} used a language model to trim sentences. Vandegehinste and
Pan ~\shortcite{Vandeghinste:04} used context-free grammar (CFG) trees to compress a sentence. However, CFG is ambiguous and constructing CFG trees' complexity is high. Dependency trees are widely developed because they offer better syntactic representations of sentences ~\cite{kahane:12}. To increase the expressive capacity of our model, we apply compression of the selected individual sentence from a syntactic aspect with the dependency tree compression model ~\cite{Shao:16}. This model defines the set of empirical rules to specify what can and cannot be trimmed. These rules were formed based on experiences from working with a large number of text documents. It showed that this model can be used to generate titles which have higher F1 scores than those generated by the previous methods.
\end{comment}

We use topic clusterings to obtain a good topic coverage in the summary when extracting key sentences.
%we may simply treat each paragraph as a separate topic. For a long document, however, a topic may spread out among several paragraphs and so a topic clustering method would be needed. %are often not %independent with each other.
%Based on
%We achieve this using topic clusterings to
In particular, we first determine which topic a sentence belongs to,
and then %by computing the probability.
%We then
extract key sentences to cover as many topics as possible within the given length boundary.
%In particular, we choose sentences of higher scores so that they belong to as many topics as possible %from each topics
%until the summary covers all the topics or reaches the size bound.

Human judgement is the best evaluation of the quality of a summarization algorithm. %For example, %either abstractive or extractive,
It is a standard practice to run an algorithm over DUC data and
compute the ROUGE recall scores with a set of DUC benchmarks,
%The ROUGE measure requires summaries
which are human-generated summaries for articles of a moderate size.
DUC-02 \cite{DUC02}, in particular, is a small set of benchmarks for single-document summarizations.
% as benchmarks for comparisons by NLP algorithms.
When dealing with a large number of documents of unknown topics and various sizes, human judgement may be
impractical, and so
we would like to have an alternative mechanism of measurement %suitable for evaluating the quality of a summary
without human involvement.
%this approach presents a barrier, for it infeasible to produce a sufficient number of benchmarks.
%To solve this issue
%we would like a new measure to compute how similar.
Ideally, this mechanism should preserve the same ordering as %able to distinguish better summaries and are comparable with
ROUGE over DUC data; namely,
if $S_1$ and $S_2$ are two summaries of the same DUC document %in DUC
produced by two algorithms,
and the ROUGE score of $S_1$ is higher than that of $S_2$, then it should also be the case under the new measure.
%The new measure is then referred to as ROUGE-compatible.
%To overcome the obstacle when no benchmarks are available to evaluate summaries for documents of particular topics,
%we define a measure based on a word-embedding method to measure the quality of a summary. In particular, we
%to use

%For this purpose, we devise WESM (Word-Embedding Similarity Measure)
%based on Word Mover's Distance (WMD) \cite{wmd}
%to measure word-embedding similarity of the summary and the original document.
Louis and Nenkova \cite{Louis:2009} devised an unsupervised method to
evaluate summarization without human models using
common similarity measures: Kullback-Leibler divergence, Jensen-Shannon divergence, and  cosine similarity.
%to measure similarities between a summary and its original document.
%
%The reason we choose WMD as a baseline similarity measurement is its capability to deal with abstractive summaries.
%
%that human-generated benchmarks
%are abstractive, and
These measures, %similarity measurements such as
%cosine similarity and
as well as the information-theoretic similarity measure \cite{Aslam03},
are meant to measure lexical similarities, which are unsuitable for measuring semantic similarities.

Word embeddings such as Word2Vec can
be used to fill this void %measure similarities from semantic aspects
and
we devise WESM (Word-Embedding Similarity Measure)
based on Word Mover's Distance (WMD) \cite{wmd}
to measure word-embedding similarity of the summary and the original document. WESM is meant to evaluate summaries for new datasets when no human-generated benchmarks are available. WESM has an advantage that it can measure the semantic similarity of documents. We show that WESM correlates well with ROUGE on DUC-02. Thus, WESM may be used as an alternative summarization evaluation method when benchmarks are unavailable.
%we choose WMD to measure semantic similarities. %as a summarization evaluation method.
%are not able to capture abstraction the similarity between an abstractive summary and the original document.
%This means that we would need to consider semantic similarity and context similarity, in addition to lexical similarity.
%WMD is the first successful attempt in this direction.

%that ofsimilar toand
%Moreover, RAKETT also ranks the highest under WEP over both DUC-02 and NewsIR-16.
%over the NewsIR-16 dataset,
% is similar to the order of ROGUE rankings produced by these algorithms over the DUC 2012 dataset.

%in place of ROGUE when human-generated benchmarks are not available for comparisons.
%Our WMD method obtains competitive performance and our summarization system outperforms prior work on ROUGE.

The major contributions of this paper are summarized below:

\vspace*{-3pt}
\begin{enumerate}
\item We present a number of summarization algorithms using topic clustering methods and enhanced keyword rankings by the softplus function,
%over keyword rankings by various
%keyword extraction algorithms and topic clustering methods,
and show that they meet the realtime requirements and outperform all the previously-known summarization algorithms under
the ROUGE measures over DUC-02.
%For instance, our algorithm ET3Rank generates the scores of (49.2, 25.6, 27.5) for (ROUGE-1, ROUGE-2, ROUGE-SU4),
%
%
%the (ROUGE-1, ROUGE-2, ROUGE-SU4) scores over DUC-02 of our algorithm ET3Rank are (49.2, 25.6, 27.5),
%while the previously-reported best results are (49.0, 24.7, 25.8) \cite{parveen2016}.

%over the DUC-02 dataset under ROUGE and they meet the realtime requirement.
%For instance, under the common measures of ROUGE-1, ROUGE-2, and ROUGE-SU4, our ET3Rank algorithm has scores of (49.2, 25.6, 27.5) over DUC-02, while
%the previously-known best scores are (48.1, 24.3, 24.2), generated by Tgraph \cite{ParveenR015}.
%ERAKE (Enhanced Rapid Automatic Keyword Extraction) and
%ERAKE-TT (ERAKE-TextTiling) %an algorithm using word-pair co-occurrences to identify keyword phrases and TextTiling to promote coverage of diverse topics,
%is the state of the art for single-document summarizations.  % superior to an extensive list of summarization algorithms.
%In particular, we show %evaluate the quality and runtime of an extensive list of summarization algorithms and show
%that both ERAKE and ERAKE-TT
%with ERAKE-TT better on ROUGE scores and ERAKE faster on runtime. For example,
%For a long document of 5,000 words, ET3Rank generates a summary of 30\% length in
%ess than 1 second on an average computer.

%combine the above approaches to generate a summary from two aspects, which are central meaning representation and topic diversity coverage with sentence %compression using the dependency tree compression model.
\vspace*{-3pt}
\item We propose a new mechanism WESM as
%
%show that these summarization algorithms
%over DUC-02 have a similar ordering of the WEP scores as that of the ROUGE scores.
%In particular, we show that under WEP, these algorithms have almost the same orderings over the DUC-02 and NewsIR-16 \cite{Signal1M2016} datasets.
%Thus, WEP may serve as
%we may use WEP as a viable
an alternative measurement of summary quality when human-generated benchmarks are unavailable.
%
%\item We de
%
%We test our unsupervised summary approach on datasets from NewsIR-16 ~\cite{Signal1M2016} and Document Understanding Evaluations 2002 ~\cite{DUC:02}. %Experimental results show that our approach outperforms the state-of-the-art results on DUC 2002 with ROUGE scores ~\cite{rouge}.
%\item We first use a novel word embedding method by Word Mover's Distance (WMD) ~\cite{wmd} to evaluate the quality of summaries without reference summaries. %Our experiments show that our approach performs comparable to ROUGE scores.
\end{enumerate}

The rest of the paper is organized as follows:
We survey in Section \ref{sect:work} unsupervised single-document summarization algorithms.
We present in Section \ref{sect:smethod} the details of our summarization algorithms and describe WESM in Section \ref{sect:emethod}. % the  automatic summarization evaluation methods.
We report the results of extensive experiments in Section \ref{sec:experiments} and conclude the paper in
%reports the experimental results of our approaches. Conclusions are given in
Section \ref{sect:conclusion}.

\section{\uppercase{Early Work}}
\label{sect:work}

\noindent Early work on single-topic summarizations can be described in the following three categories: keyword extractions, coverage and diversity optimizations,
and topic clusterings.

\subsection{Keyword extractions}

To identify keywords in a document over a corpus of documents, the measure of term-frequency-inverse-document-frequency (TF-IDF) \cite{salton87} is often used.
%  over
%which is %takes advantage of The preassumption for using TF-IDF is that
%a reasonably sized corpus of documents.
%However, TF-IDF has a shortcoming.
%Otherwise the TF-IDF value of a keyword may just be the term frequency of the keyword in the document.
%To identify keywords in a single document without using
When document corpora are unavailable,
the measure of word co-occurrences (WCO) can produce a %identify keywords with %\cite{Matsuo:03}. %which applies to a single document without a corpus.
comparable performance to TF-IDF over a large corpus of documents \cite{Matsuo:03}.
The methods of TextRank \cite{Mihalcea04} and RAKE (Rapid Automatic Keyword Extraction) \cite{Rose:10}
further refine the WCO method from different perspectives, %. We note that both TextRank and RAKE
which are also sufficiently fast to become candidates %for keyword extractions
for meeting the realtime requirements. %In particular,
%are unsupervised keyword extraction methods with weights.

TextRank %is a graph-based ranking model %for text processing and it can obtain competitive results with state-of-the-art systems developed in these areas.
%Graph-based ranking algorithms are a way of
%for deciding
computes the rank of a word in an undirected, weighted word-graph using a slightly modified PageRank algorithm \cite{Brin:98}.
To construct a word-graph for a given document, first remove stop words and represent each remaining word
as a node, then link two words if they both appear in a sliding window of a small size. Finally,
assign the number of co-occurrences of the endpoints of an edge as a weight to the edge.
%which are based on global information recursively drawn from the entire graph.
%This graph-based ranking model is that of ��voting�� or ��recommendation��. The higher the number of votes for a vertex, the higher the importance of the vertex.

RAKE %assigns a ranking to each word using word pair co-occurrences: It
first removes stop words using a stoplist, and then generates words (including phrases) using a set of word delimiters and %generates
%keyword phrases using
a set of phrase delimiters.
For each remaining word $w$, the degree of $w$ is the frequency of $w$ plus the number of co-occurrences of consecutive word pairs $ww'$ and $w''w$ in the document, where $w'$ and $w''$ are
remaining words.
The score of $w$ is the degree of $w$ divided by the frequency of $w$. We note that
the quality of RAKE also depends on a properly-chosen stoplist, which is language dependent.
%may be easier to determine for some languages and
%harder for other languages.
%The score of a keyword phrase is the summation of
%the scores of the words in the phrase.
%
%. RAKE is superior to WCO because it is simpler and achieves a higher precision rate. We use TextRank and RAKE as our keyword extraction methods for sentence ranking.
%RAKE is a linear-time algorithm, while the runtime of TextRank depends on
%the speed of convergence. Experimental results


\subsection{Coverage and diversity optimization}
%Our work is inspired by the concept of a class of submodular functions for document summarization ~\cite{lin:11}.
%Compared to all the other known extractive summarization algorithms known at the time it was published, Tgraph offered the best ROUGE scores on DUC-02.

 %, Lin and Bilmes \cite{lin:11}
The general framework of selecting sentences gives rise to optimization problems with objective functions
being monotone submodular \cite{lin:11} to promote coverage and diversity.
Among them is an objective function in the form of $L(S) + \lambda R(S)$ with
%
%that is the sum of
%two components.
%$S$ is
a summary $S$ and a coefficient $\lambda \geq 0$, where
$L(S)$ measures the coverage of the summary and $R(S)$ rewards diversity.
We use SubmodularF to denote the algorithm computing this objective function.
%and the algorithm computing it to be described later as SubmodularF.
SubmodularF uses TF-IDF values of words in sentences to compute the cosine similarity of two sentences.
% on the generalized TF-IDF values of the words contained in the respective sentence,
%where the TF value of a word is at the sentence level while the IDF value is still at the document level over a corpus.
%and the IDF value is for the entire corpus.
%The ranking of a sentence $s$ in a document
%is the similarity of $s$ and the document.
%$L(S)$ is measured by the total ranking of the sentences in $S$ with a parameterized upper bound different for each sentence.
%$R(S)$ is measured by first partitioning the document (often using its paragraphs as partitions)
%and then summing up, for each partition, the square root of the total average ranking for each sentence in the partition that is also in $S$.
While
it is NP-hard to maximize a submodular objective function subject to a summary length constraint, the submodularity allows
a greedy approximation with a proven approximation ratio of $1-1/\sqrt{e}$.
%
%Given a document $D$ in a corpus, let $S$ be a set of sentences extracted from
%$${\cal F}(S) = {\cal L}(S) + \lambda{\calR}(S),$$
%where ${\cal L}(S)$
%
%that
%This method
%encourages the summary to be representative of the corpus and positively rewards diversity. They used monotone nondecreasing and submodular functions which is %an efficient scalable greedy optimization scheme.
%However, they

SubmodularF needs labeled data to train the parameters in the objective function to achieve a better summary and it is intended to work on multiple-document summarizations.
While it is possible to work on a single document without a corpus, we note that the greedy algorithm has at least a quadratic-time complexity and it produces a summary with
low ROUGE scores over DUC-02 (see Section \ref{sec:other}),
and so it would not be a good candidate to meet our needs. This also applies to a generalized objective function
%
%
%its ROUGE scores over DUC-02 are much lower than those of RAKETT (see Section \ref{sec:experiments}).
%SubmodularF also incurs much longer time that does not meet the realtime requirement.
%
%The submodularity approach was generalized to include dispersion where the objective
%function
consisting of a submodular component and a non-submodular component \cite{DasguptaKR13}.
%While this generalization provides a slightly better ROUGE-1 score over DUC-04 than SubmodularF, it incurs
%a higher time complexity.

\subsection{Topic clusterings}

Two unsupervised approaches to topic clusterings for a given document have been investigated.
One is TextTiling \cite{hearst:97} and the other is LDA (Latent Dirichlet Allocation) \cite{blei:03}.
TextTiling %is a lighter-weight method,
%Another method we use for topic coverage is TextTiling ~\cite{hearst:97}.
%TextTiling is a technique for
%which
represents a topic as a set of consecutive paragraphs in the document.
It merges adjacent paragraphs that belong to the same topic. TextTiling identifies major topic-shifts based on patterns of lexical co-occurrences and distributions.
LDA computes for each word a distribution under a pre-determined number of topics. %is a heavy-weight method, which
%and represents a topic % would be better
%to identify the topics contained in the document,
%where each topic is represented
%as a set of keywords with larger probabilities. It needs to predetermine the number of topics contained in a document corpus.
LDA is a computation-heavy algorithm
that incurs a runtime too high to meet our realtime requirements.
TextTiling has a time complexity of almost linear, which meets the requirements of efficiency. % and runs much faster than LDA.
%Sentences from the same subtopic won't be selected unless all the subtopics have been covered to ensure the topic coverage.
%We note that LDA incurs a time complexity too high to meet the requirement of realtime summarizations.

\subsection{Other algorithms}
\label{sec:other}

Following the general framework of selecting sentences to meet the requirements of topic coverage and diversity,
a number of unsupervised single-document summarization algorithms have been devised.
The most notable is $CP_3$ \cite{parveen2016}, which produces the best ROUGE-1 (R-1), ROUGE-2 (R-2), and ROUGE-SU4 (R-SU4) scores on DUC-02
among all early algorithms,
including
%is an extractive single-document summarization algorithm devised recently.
%is the most recent algorithm
%, which
%offers the better ROUGE scores over DUC-02 compared to an extensive list of algorithms,
%including
Lead \cite{ParveenR015}, DUC-02 Best, TextRank, LREG \cite{ChengL16a},
Mead \cite{radev2004}, $ILP_{phrase}$ \cite{woodsend2010}, URANK \cite{wan2010}, UniformLink \cite{WanX10}, Egraph + Coherence \cite{Parveen015},
Tgraph + Coherence (Topical Coherence for Graph-based Extractive Summarization) \cite{ParveenR015},
NN-SE \cite{ChengL16a}, and SubmodularF.

$CP_3$ maximizes importance, non-redundancy, and pattern-based coherence of sentences to generate a coherent summary using ILP. 
It computes %considers
the ranks of selected sentences for the summary by the Hubs and Authorities algorithm (HITS) \cite{kleinberg1999}, %(will add a cite).
%Non-redundancy represents if the summary
and ensures that each selected sentence has unique information. % in every sentence.
It then uses mined patterns to extract sentences if the connectivity among nodes in the projection graph matches the connectivity among nodes in a coherence pattern. Because of space limitation, we omit the descriptions of the other algorithms.

Table \ref{CP3} shows the comparison results, where the results for SubmodularF
is obtained using
%To compare with SubmodularF,
%we use
the best parameters trained on DUC-03 \cite{lin:11}.
%%%table to be inserted here
%Our computation show that the ROUGE-1, ROUGE-2, and ROUGE-Su4 scores for SubmodularF are (39.6, 16.9, 17.8), which are
%much lower than those of $CP_3$.
%$CP_3$ \cite{parveen2016} is the most recent algorithm that has the best results reported over DUC-02 so far.
Thus, to demonstrate the effectiveness of our algorithms, we will compare our algorithms with only $CP_3$ %under these common ROUGE measures
over DUC-02.
\begin{table}[h]
\begin{center}
\caption{\label{CP3} ROUGE scores (\%) on DUC-02 data.}
\begin{tabular}{l|c|c|c}
\hline
\bf Methods & \bf R-1 & \bf R-2 & \bf R-SU4 \\
\hline
Lead & 45.9 & 18.0 & 20.1 \\
DUC 2002 Best & 48.0 & 22.8 & \\
TextRank & 47.0 & 19.5 & 21.7 \\
LREG & 43.8 & 20.7 & \\
Mead & 44.5 & 20.0 & 21.0 \\
$ILP_{phrase}$ & 45.4 & 21.3 & \\
URANK & 48.5 & 21.5 & \\
UniformLink & 47.1 & 20.1 &   \\
Egraph + Coh. & 48.5 & 23.0 & 25.3  \\
Tgraph + Coh. & 48.1 & 24.3 & 24.2 \\
NN-SE & 47.4 & 23.0 & \\
SubmodularF & 39.6 & 16.9 & 17.8 \\
$\bm{CP_3}$& \bf 49.0 & \bf 24.7 & \bf 25.8  \\
\hline
\end{tabular}

\end{center}
\end{table}

Solving ILP, however, is
time consuming even on documents of a moderate size, for ILP is NP-hard.
%
%to compute the optimization problem, and ILP is NP-hard.
Thus, $CP_3$ does not meet the requirements of time efficiency. We will need to investigate new methods.

%Important function
 %It uses coherence patterns with 3 nodes.
%
\begin{comment}
Tgraph is based on a weighted graphical representation of documents using LDA topic modeling. It %Tgraph + Coherence
uses ILP to optimize the objectives of sentence importance, coherence, and non-redundancy simultaneously. $CP_3$ optimizes Tgraph % Tgraph has considered simultaneously
using Mixed Integer Programming.
\end{comment}
%Because $CP_3$ offers %and Tgraph are the algorithms with
%the best ROUGE scores and the space is limited,
%
%mentioned above are omitted because of space limitation.
%We will be only
%and it suffices to compare our algorithms with only $CP_3$ under the common ROUGE measures over DUC-02 to demonstrate the effectiveness of our algorithms. %with $CP_3$. % and Tgraph.
%Tgraph, however, incurs much higher time complexity because of LDA (see Section \ref{sec:experiments}).
%, which does not need the realtime requirement.
%Moreover, we will show in Section \ref{sec:experiments} that its ROUGE scores on DUC-02 is not as good as RAKETT.
%Our method is superior %resulting in higher ROUGE scores and our summary evaluation methods leading to higher WMD scores on single documents without development sets.

%Another approach for extractive single-document summarization is called

%
%compared ROUGE scores with state-of-the-art results on DUC-02 data. Our experiments show that our method has higher ROUGE scores than Tgraph and other state-of-the-art methods such as TextRank,
%are other and so on using the same data. We also obtain comparable results from our summary evaluation methods by WMD.

\begin{comment}
ROUGE has been widely accepted for the evaluation of summaries. It includes measures to automatically determine the quality of a summary by comparing it to ideal summaries created by humans. These measures count the number of overlapping units between summary to be evaluated and the ideal summaries created by humans. However, the performance of the ROUGE method fully depends on the quality of reference summaries made by humans. Thus, it cannot ensure similarity between the original document and the summary to be evaluated. ROUGE does not take topic diversity coverage into account. WMD is a novel distance function between text documents. This work is based on word embeddings that learn semantically meaningful representations for words from local co-occurrences in setences. We used WMD to measure the similarity to the original document and take topic diversity coverage into consideration by comparing different paragraphs.
\end{comment}


\section{\uppercase{Our Methods}}

\label{sect:smethod}

\begin{comment}
\subsection{Sentence Ranking}
\label{ssect:sr}
The summary should contain only important sentences. We find all the keywords with positive numerical scores made from RAKE ~\cite{Rose:10} and TextRank  ~\cite{Mihalcea:04} and give a score to every sentence based the number of keywords it contains. Following Shao and Wang ~\shortcite{Shao:16}, we use their central sentence extraction algorithm for ranking sentences by importance.
Assume that a sentence contains $n$ keywords $w_1, \cdots, w_n$, and $w_i$ has a positive score $s_i$.

We use the power of 2 to amplify the differences of the rankings and the ranking score of the sentence is as follows:
\begin{equation} \label{rank1}
\mbox{Rank} = \sum_{i=1}^{n}2^{s_{i}}
\end{equation}

After we calculate the ranking scores of all the sentences, we can order the sentences by their scores. We try to pick as many sentences with high scores as possible. From high to low scoring sentences, we put each sentence into the summary based on different topic diversity coverage constraints in Section \ref{ssect:tdc}.


\subsection{Compression Constrains}
\label{ssect:cc}
\begin{figure}[h]
\includegraphics[width=3in]{fig1_1.png}
\includegraphics[width=3in]{fig1_2.png}
%\DeclareGraphicsExtensions.
\caption{The grammatical relations and trimmed dependency tree using DTCM for sentence ``Market concerns about the deficit has hit the greenback''}%: $\Lambda$ restricts the $\theta$ to the labels a document belongs to}
\label{fig:fig1}
\end{figure}

Our goal is to be able to compress sentences so we can pack more information into a summary. We use the Dependency tree compression model (DTCM) in Shao and Wang ~\shortcite{Shao:16} to do compressions. A dependency tree can provide relations between each word in the sentence. We use the Standord Dependency Parser (SDP) \footnote{\tt http://nlp.stanford.edu/software/\\ \-\hspace{.75cm} stanford-dependencies.shtml} to obtain grammatical relations between words in a sentence. This model follows a set of empirical rules to specify what can or cannot be trimmed. DTCM can delete all the unnecessary branches from sematic aspects. Figure \ref{fig:fig1}shows one example sentence ``Market concerns about the deficit has hit the greenback'' . The compressed sentence is ``Market concerns about deficit hit greenback''.
\end{comment}

%\subsection{Topic Diversity Coverage Constrains}
%\label{ssect:tdc}
\noindent We use TextRank and RAKE to obtain initial ranking scores of keywords, and
%In general, TextRank ranking scores are small while RAKE ranking scores are much larger.
use the softplus function \cite{softplus}
\begin{equation}\label{eq1}
sp(x) = \ln (1+e^x)
\end{equation}
to
enhance keyword rankings to favor sentences that are more important. % that are more important.
%enhance the rankings of sentences
%that are more important.

\subsection{Softplus ranking}

%as compared to the direct summation of the ranking scores of the keywords contained in the sentence.
Assume that after filtering, a sentence $s$ consists of $k$ keywords $w_1, \cdots, w_k$, and $w_i$ has a ranking score $r_i$
produced by TextRank or RAKE. 
Following Shao and Wang ~\cite{Shao:16}, we use their central sentence extraction algorithm for ranking sentences by importance as $\mbox{Rank}(s)$.
We can rank $s$ using one of the following two methods:
\begin{equation}\label{eq2}
{Rank}(s) = \sum_{i=1}^{k} r_i
\end{equation}
\begin{equation}\label{eq3}
{Rank}_{sp}(s) = \sum_{i=1}^{k} sp(r_i)
\end{equation}
\begin{comment}
\[
\mbox{Rank}(s) = \sum_{i=1}^{k} r_i;~~
\mbox{Rank}_{sp}(s) = \sum_{i=1}^{k} sp(r_i).
\]
\end{comment}
%A straightforward ranking of $s$ is to sum up the scores of the keywords contained in it; that is,
%$$\mbox{Rank}(s) = \sum_{i=1}^{k} r_i.$$
%We use the softplus function to enhance the ranking of $s$ as follows:
%$$\mbox{Rank}_{sp}(s) = \sum_{i=1}^{k} sp(r_i).$$
%We use DTRank (Direct TextRank) and DRAKE (Direct RAKE)
%to refer to this method of computing sentence ranking.
Let DTRank (Direct TextRank) and ETRank (Enhanced TextRank) denote
the methods of ranking sentences using, respectively,
$\mbox{Rank}(s)$ and $\mbox{Rank}_{sp}(s)$ over TextRank keyword rankings,
and
%
%and DRAKE (Direct RAKE) denote
%the method of ranking sentences using $\mbox{Rank}(s)$
%over TextRank and RAKE keyword rankings, respectively;
%and
DRAKE (Direct RAKE) and ERAKE (Enhanced RAKE)
%ETRank (Enhanced TextRank) and ERAKE (Enhanced RAKE)
to denote the methods of ranking sentences
using, respectively, $\mbox{Rank}(s)$ and $\mbox{Rank}_{sp}(s)$ over RAKE keyword rankings.

%To enhance the ranking of a sentence that is more important,
%We use the softplus function to enhance the ranking of $s$ as follows:
%$$\mbox{Rank}_{sp}(s) = \sum_{i=1}^{k} sp(r_i).$$
%
%
%If $r_i$ is produced by TextRank, and respectively by RAKE, then the ranking score of $s$ is denoted by
%Then the softplus ranking of $s$ is computed by % $\mbox{Rank}(s)$ and $\mbox{Rank}_R(s)$, which are computed by
%\begin{eqnarray*}
%$$\mbox{Rank}(s) = \sum_{i=1}^{k} \ln(1+e^{r_i}).$$
%$\mbox{Rank}_R(s) = \sum_{i=1}^{k} 2^{r_i}$.
%\end{eqnarray*}
%We refer to TextRank and RAKE using this measure of sentence ranking as ETRank (Enhanced TextRank) and ERAKE (Enhanced RAKE).
%We use ETRank (Enhanced TextRank) and ERAKE (Enhanced RAKE) to denote this method of computing sentence ranking based
%on TextRank and RAKE keyword rankings, respectively.

The softplus function is helpful because %enhance the ranking of a more important sentence,
%let $sp(x) = \ln (1+e^x)$. We note that
%we note that
when $x$ is a small positive number, $sp(x)$ increases the value of $x$ significantly (see Figure \ref{fig:softplus})
and when $x$ is large, $sp(x) \approx x$.
\begin{figure}[h]
\centering
\includegraphics[width=2.5in]{softplus1.png}
%\DeclareGraphicsExtensions.
\caption{Softplus function $\ln(1+e^x)$.} %: $\Lambda$ restricts the $\theta$ to the labels a document belongs to}
\label{fig:softplus}
\end{figure}
In particular,
given two sentences $s_1$ and $s_2$, suppose that $s_1$ has a few keywords with high rankings and the rest of the keywords with low rankings,
while $s_2$ has medium rankings for almost all the keywords. In this case, we would consider $s_1$ more important than $s_2$. However,
%because the keywords in $s_1$ gave low ranking scores,
we may end up with $\mbox{Rank}(s_1) < \mbox{Rank}(s_2)$.
To illustrate this using a numerical example, assume that $s_1$ and $s_2$ each consists of 5 keywords, with
original scores (sc) and softplus scores (sp) given in the following table \ref{example}.

\medskip

%The ranking of $s_1$ can be enhanced after using softplus. For example, the following two sentences are extracted from a news article using TextRank to rank keywords, with
%$s_1$ being more important than $s_2$ because $s_1$ specifies the name, strength, and direction of the hurricane, as well as the name of the place it is expected to hit.
%\begin{itemize}
%\item[$s_1$:] {\sf Hurricane Gilbert is heading toward Jamaica with 100 mph winds.}
%\item[$s_2$:] {\sf A hurricane warning has been issued for the island.}
%\end{itemize}
%The ranking score of each keyword is depicted in Table \ref{table:example}, from which we
%can see that $s_2$ is selected without using softplus. After using softplus, $s_1$ is selected as it should be.

\noindent
\begin{table}[h]
\begin{center}
\caption{\label{example} Numerical examples with given sc and sp scores.}
\begin{tabular}{l||c|c|c|c|c||c}
\hline
$s_1$ & $w_{11}$ & $w_{12}$ & $w_{13}$ & $w_{14}$ & $w_{15}$&   Rank \\
\hline
sc & 2.6 & 2.2 & 2.1 & 0.3 & 0.2 & 7.4\\
sp & 2.67 & 2.31 & 2.22 & 0.85& 0.80 & \bf 8.84\\
\hline
$s_2$ & $w_{21}$ & $w_{22}$ & $w_{23}$ & $w_{24}$ & $w_{25}$ & \\
\hline
sc & 1.6 & 1.5 & 1.5 & 1.5 & 1.4  & \bf 7.5  \\
sp & 1.78& 1.70& 1.70& 1.70& 1.62&  8.51 \\
\hline
\end{tabular}
\end{center}
\end{table}

\medskip
Sentence $s_1$ is more important than $s_2$
because it contains three keywords of much higher ranking scores than those of $s_2$.
However, $s_2$ will be selected without using softplus. After using softplus, $s_1$ is selected as it should be.

\begin{comment}
For example, Hurricane Gilbert is heading toward Jamaica with 100 mph winds.
����1��(0.7, 0.6, 0.01, 0.02, 0.03)
softplus:(1.10318604889, 1.03748795049, 0.698159680508, 0.703197179727, 0.708259676341)
softplusֵ�ĺͣ�4.25029053595
Ȩ�غͣ�1.36


A hurricane warning has been issued for the island.
����2��(0.2, 0.2, 0.3, 0.3, 0.4)
softplus:(0.798138869382, 0.798138869382, 0.854355244469, 0.854355244469, 0.9130152524)
softplusֵ�ĺͣ�4.2180034801
Ȩ�غͣ�1.4
\end{comment}

%\medskip
For a real-life example, consider the following two sentences from an article in DUC-02:
\vspace*{-1pt}
\begin{itemize}
\item[$s_1$:] {\small\sf Hurricane Gilbert swept toward Jamaica yesterday with 100-mile-an-hour winds, and officials issued warnings to residents on the southern coasts of the Dominican Republic, Haiti and Cuba.}
\vspace*{-1pt}\item[$s_2$:] {\small\sf Forecasters said the hurricane was gaining strength as it passed over the ocean and would dump heavy rain on the Dominican Republic and Haiti as it moved south of Hispaniola, the Caribbean island they share, and headed west.}
\end{itemize}
We consider $s_1$ more important as it
specifies the name, strength, and direction of the hurricane, the places affected, and the official warnings.
Using TextRank to compute
keyword scores, we have $\mbox{Rank}(s_1) = 1.538 < \mbox{Rank}(s_2) = 1.603$, which returns a less important sentence $s_2$. After computing
softplus,
we have $\mbox{Rank}_{sp}(s_1) = 8.430 > \mbox{Rank}_{sp}(s_2) = 7.773$; the more important sentence $s_1$ is selected.

Note that not any exponential function would do the trick. What we want is a function to return roughly the same value as the input when the input is large, and a significantly larger value than the input when the input is much less than 1. The softplus function meets this requirement.

\begin{comment}
(u'Forecasters said the hurricane was gaining strength as it passed over the ocean and would dump heavy rain on the Dominican Republic and Haiti as it moved south of Hispaniola, the Caribbean island they share, and headed west.', [1.6033724386659647, u'hurricane#caribbean island#south#republic#forecasters#haiti#west#headed#heavy rain#dominican#', '0.376361942128#0.222770119935#0.152622310017#0.144567553334#0.140516484532#0.140516484532#0.117560427986#0.117560427986#0.11182141385#0.0790752743651#'])

(u' Hurricane Gilbert swept toward Jamaica yesterday with 100-mile-an-hour winds, and officials issued warnings to residents on the southern coasts of the Dominican Republic, Haiti and Cuba.', [1.538394147633235, u'hurricane#south#coast#republic#jamaica yesterday#haiti#southern coasts#mile#winds#dominican#warnings#', '0.376361942128#0.152622310017#0.152622310017#0.144567553334#0.142542018933#0.140516484532#0.117560427986#0.0824985459553#0.0790752743651#0.0790752743651#0.0709520059994#'])
assume that sentence
$s$ consists of five keywords: $(w_1, w_2, w_3, w_4, w_5)$ whose corresponding ranking scores
are
%How we select sentences is critical. In general,
We would want to select sentences with
ranking scores as high as possible, while avoiding selections of multiple sentences from one topic
and no sentence at all from another topic.
\end{comment}


\subsection{Topic clustering schemes}

We consider four topic clustering schemes: TCS, TCP, TCTT, and TCLDA.
%The first one is called naive TCDS, denoted by
\begin{enumerate}
\item TCS %(TC based on sentences)
selects sentences without checking topics.
\item TCP %(TC based on paragraphs)
treats each paragraph as a separate topic.
%sentences based on their scores and also ensure the selected sentences distribute different paragraphs.
%For example, we first select a sentence with the highest ranking score (break ties arbitrarily) and place the paragraph number that contains this
%sentence into KTS. We then select the second highest score sentence. If the second sentence is not in the KTS, we put the second one into the summary and the %paragraph number into the KTS. Otherwise, we skip it to the next sentence. The following parts are the same as the basic procedure of TDCC.
\item TCTT %(TC based on TextTiling)
%The third TDCC is called TextTiling TDCC (t-TDCC). TextTiling ~\cite{hearst:97} is used to
partitions a document into a set of multi-paragraph %topical
segments using TextTiling.
%It may merge or divide paragraphs into several topics. In t-TDCC, we try not to pick sentences from the same topic. The procedure is similar to the basic TDCC.
\item TCLDA %(TC based on LDA)
%The last TDCC is called LDA TDCC (l-TDCC). LDA ~\cite{blei:03} is a model for topic modeling where topic probabilities are assigned words in documents.
computes a topic distribution for each word using LDA. %To use LDA we must first fix the number of topics for a document corpus.
%The probabilities can be sued to measure the semantic relatedness between words. The number of topics in a document should be set by us.
We set the number of topics from 5 to 8 depending on the length of the document.
Assume that a document contains $K$ topics ($5 \leq K \leq 8$) and the topic $j$ consists of $k_j$ words $w_{1j}, \cdots, w_{k_j,j}$, where $1 \leq j \leq K$
and
$w_{ij}$ has a probability $p_{ij} > 0$. For a document with $n$ sentences $s_1, \cdots, s_n$,
%if $s_z$ does not contain word $w_{ij}$, then the probability $p_{ij}$ in the topic $j$ is set to $1$, otherwise is $p_{ij}$.
we use the following maximization to determine which topic $t_z$ the sentence $s_z$ belongs to ($1 \leq t \leq K$):
\begin{equation} \label{eq4}
t_z = \argmax_{1 \leq j\leq k}\bigg(\prod_{i:w_{ij} \in s_z} {p_{ij}}\bigg)
\end{equation}
\end{enumerate}




\begin{comment}
Figure \ref{fig:fig3} shows the bipartite topical graph of TTCD, where $\bm{w}_j = (w_{1i}, \cdots, w_{k_j,j})$ is the vector words under topic $j$
($1 \leq j \leq m$), $s_z$ ($1 \leq z \leq n$) is a sentence in a document, $p_j$ ($1 \leq j \leq m$) is the product of possibilities in $wi$. $tz$ is the topic for the sentence $sz$ with the highest possibility. From equation \ref{topic}, we know that each sentence belongs to certain topic. Then we can use the basic TDCC to ensure the summary cover topics as many as possible.

 \begin{figure}[h]
\includegraphics[width=3in]{fig3.png}
%\DeclareGraphicsExtensions.
\caption{Bipartite topical graph of TTCD}%: $\Lambda$ restricts the $\theta$ to the labels a document belongs to}
\label{fig:fig3}
\end{figure}
\end{comment}

\subsection{Summarization algorithms}
\label{ssect:sa}

%The goal of topic diversity coverage constraints (TDCC) is to avoid sentences coming from the same topic, which means to make the summary cover as many topics %as possible.
The length of a summary may be specified by users,
either as a number of words or as a percentage of the number of characters of the original document.
By a ``30\% summary'' we mean that the number of characters of the summary does not exceed 30\% of that of the original document.

Let $L$ be the summary length (the total number of characters) specified by the user and $S$ a summary.
If $S$ consist of $m$ sentences
%Assume that a document contains $m$ topics and $n$ sentences
$s_1, \cdots, s_m$, and the number of characters of $s_i$ is $\ell_i$, then the following inequality must hold:
$\sum_{i=1}^m \ell_i \leq L.$
%The summary must satisfy additional length requirement:
%\begin{equation} \label{length}
%\mbox{Len(summary)} \geq \sum_{i=1}^{n}{l_{i}}
%\end{equation}

Depending on which sentence-ranking algorithm and which topic-clustering scheme to use, we have eight combinations
using ETRank and ERAKE, and eight combinations using DTRank and DRAKE, shown in Table \ref{algorithm_names}.
%
%ESTRank, EPTRank, ET3Rank, ELDATRank when using ETRank to compute sentence rankings;
%ESRAKE, EPRAKE, ET2RAKE, and ELDARAKE when using ERAKE to compute sentence rankings. %extract keywords.
For example, ET3Rank (Enhanced TextTiling TRank) means to use $\mbox{Rank}_{sp}(s)$ to rank sentences and
TextTiling to compute topic clusterings, and
%
%Likewise, using DTRank and DRAKE we have
%the following combinations: STRank, PTRank, T3Rank, LDATRank,
%SRAKE, PRAKE, T2RAKE, LDARAKE. For example,
T2RAKE (TextTiling RAKE)
means to use $\mbox{Rank}(s)$ rank sentences over RAKE keywords and TextTiling
to compute topic clusterings. %The description of all the algorithms are shown in Table \ref{algorithm_names}.

\begin{table}[h]
\begin{center}
\caption{\label{algorithm_names} Description of all the Algorithms with different sentence-ranking (S-R) and topic-clustering (T-C) schemes.}
\begin{tabular}{l|c|c}
\hline
\bf Methods & \bf S-R & \bf T-C \\
\hline
ESTRank & ETRank & TCS \\
EPTRank & ETRank & TCP \\
ET3Rank & ETRank & TCTT \\
ELDATRank & ETRank & TCLDA \\
\hline
ESRAKE & ERAKE & TCS \\
EPRAKE & ERAKE & TCP \\
ET2RAKE & ERAKE & TCTT \\
ELDARAKE & ERAKE & TCLDA \\
\hline
STRank & DTRank & TCS \\
PTRank & DTRank & TCP \\
T3Rank & DTRank & TCTT \\
LDATRank & DTRank & TCLDA \\
\hline
SRAKE & DRAKE & TCS \\
PRAKE & DRAKE & TCP \\
T2RAKE & DRAKE & TCTT \\
LDARAKE & DRAKE & TCLDA \\
\hline
\end{tabular}
\end{center}
\end{table}

All algorithms follow the following procedure for selecting sentences:

\vspace*{-2pt}
\begin{enumerate}
\item Preprocessing phase
\begin{enumerate}
\vspace*{-1pt}\item Identify keywords and compute the ranking of each keyword.
\vspace*{-1pt}\item Compute the ranking of each sentence.
\end{enumerate}
\vspace*{-2mm}
\item Sentence selection phase
\begin{enumerate}
\vspace*{-1pt}\item Sort the sentences in descending order of their ranking scores. % according to the underlying keyword extraction algorithm.

\vspace*{-1pt}\item Select sentences one at a time with a higher score to a lower score.
Check if the selected sentence $s$ belongs to the known-topic set (KTS) according to the underlying
topic clustering scheme, where KTS is a set of topics from sentences placed in the summary so far. If $s$ is in KTS, then discard it; otherwise, place $s$ into the summary and its topic into KTS.
\vspace*{-1pt}\item Continue this procedure until the summary reaches its length constraint.

\vspace*{-1pt}\item If the number of topics contained in the KTS is equal to the number of topics in the document,
% (assume a document contains $m$ topics), we
empty KTS and repeat the procedure from Step 1. %to empty and select the unpicked sentences from the start.
\end{enumerate}
\end{enumerate}

%Given a keyword extraction algorithm and a topic clustering scheme, it is straightforward to replace them in the algorithm. For example,
%ET3Rank use ETank to compute the ranking of each sentence and use TextTiling for topic clustering.

Figure \ref{fig:fig5} shows an example of 30\% summary generated by ET3Rank on an article in NewsIR-16.
\begin{figure*}[t]
\includegraphics[width=6in]{fig5.png}
%\DeclareGraphicsExtensions.
\caption{An example of 30\% summary of an article in NewsIR-16 by ET3Rank, where the
original document is on the left and the summary is on the right.}
\label{fig:fig5}
\end{figure*}


\begin{comment}
Figure \ref{fig:fig2} depicts this procedure.

\begin{figure}[h]
\includegraphics[width=3in]{fig2.png}
%\DeclareGraphicsExtensions.
\caption{The general procedure of extracting sentences}%: $\Lambda$ restricts the $\theta$ to the labels a document belongs to}
\label{fig:fig2}
\end{figure}

Our basic procedure of selecting sentences is as follows: (1) Sort the sentences in descending order of their ranking scores. (2) Select sentences one at a time with a higher score to a lower score. (3) Check if the selected sentence $s$ belongs to the known topic set (KTS), which is a set of topics from sentences placed in the summary so far. If $s$ is in KTS, then discard it; otherwise, place $s$ into the summary and its topic into KTS. (4) Continue this procedure until the summary reaches its length constraint. (5) If the number of topics contained in the KTS is equal to the number of topics in the document,
% (assume a document contains $m$ topics), we
empty KTS and repeat the procedure from Step 1. %to empty and select the unpicked sentences from the start.
Figure \ref{fig:fig2} depicts this procedure. % basic procedure of TDCC.


In this section, we introduce our four different summarization algorithms based on the above TDCC. We name summarization methods as n-SA, p-SA, t-SA and l-SA based on n-TDCC, p-TDCC, t-TDCC and l-TDCC. The only difference between these four algorithms is used different TDCC and other parts are the same. For utilizing TextRank or RAKE as the sentence ranking method, we can divide four summarization algorithms into eight, which are TextRankN, RAKEN, TextRankP, RAKEP, t-SA-T, RAKETT, TextRankLDA and RAKELDA. We describe our algorithm in general as follows:
\begin{itemize}
\item Get keywords set with weights from TextRank or RAKE.
\item Split the original text into sentences.
\item Find all the keywords in each sentence and use sentence ranking algorithm in Section \ref{ssect:sr} to get the sentence score.
\item Sort the sentences by their scores.
\item Iterate the sentences from high to low score and select the sentences based on TDCC in Section \ref{ssect:tdc} until it reaches the limitation of the summary.
\item Compress the selected sentences by algorithm in Section \ref{ssect:cc}.
\item Order the selected trimmed sentences by their order in the original text to generate the final summary.
\end{itemize}
\end{comment}
\section{\uppercase{A Word-Embedding Measurement of Quality}}
\label{sect:emethod}

%\subsection{Word2Vec Embedding}
%\label{ssec:Word2Vec}
\noindent Word2vec \cite{mikolov13,mikolov2013}
%introduced a novel word-embedding procedure Word2Vec. Their model
is an NN model that learns a vector representation for each word contained in a corpus of documents.
%In addition, they used the skip-gram model with neural network architecture.
The model consists of an input layer, a projection layer, and an output layer to predict nearby words in the context. In particular,
a sequence of $T$ words $w_1, \cdots, w_T$ are used to train a Word2Vec model for maximizing the probability of neighboring words:
\begin{equation} \label{eq5}
{\frac{1}{T}\sum_{t=1}^{T}{\sum_{j\in b(t)}{\log p(w_j|w_t)}}}
\end{equation}
where $b(t) = [t-c, t + c]$ is the set of center word $w_t$'s neighboring words, $c$ is the size of the training context, and $p(w_j|w_t)$ is defined by the softmax function.
%(See more details in ~\cite{mikolov2013}.)
Word2Vec can learn complex word relationships if it trains on a very large data set.
%A commonly cited example is that vec(king) + vec(man) - vec(woman) $\approx$ vec(queen).

\subsection{Word Mover's Distance}
\label{ssec:wmd}
Word Mover's Distance (WMD) \cite{wmd} uses Word2Vec as a word embedding representation method.
It measures the dissimilarity between two documents and calculates the minimum cumulative distance to ``travel'' from the embedded words of one document to the other. Although two documents may not share any words in common, WMD can still measure the semantical similarity by considering their word embeddings, while other bag-of-words or TF-IDF methods only measure the similarity by the appearance of words. A smaller value of WMD indicates that the two sentences are more similar.

\begin{comment}
 \begin{figure}[h]
\includegraphics[width=3in]{fig4.png}
%\DeclareGraphicsExtensions.
\caption{The WMD metric on two sentences S1, S2 compared with the query sentence Q \cite{wmd}}%: $\Lambda$ restricts the $\theta$ to the labels a document belongs to}
\label{fig:fig4}
\end{figure}
Figure \ref{fig:fig4}, extracted from \cite{wmd}, shows an example of the WMD metric on two sentences $S_1$, $S_2$ compared with the query sentence $Q$.
The arrows represent flow between two words, which are labeled with their distance costs.
For each sentence, stop words are removed. Comparing the main components one by one and adding their distance contributions together, we obtain the WMD of two sentences. The lower the value of WMD, the more similar two sentences are. Intuitively, traveling from $Illinois$ to $Chicago$ is much closer than is $Japan$ to $Chicago$. Also, Word2Vec embedding generates the vector vec ($Illinois$) closer vec ($Chicago$) than vec ($Japan$). Thus, sentence $S_1$ to query $Q$ is more similar than sentence $S_2$ to query $Q$.
\end{comment}

\subsection{A word-embedding similarity measure}

Based on WMD's ability of measuring the semantic similarity of documents, %we design innovative methods to evaluate the quality of the summary.
we propose a summarization evaluation measure WESM (Word-Embedding Similarity Measure). %One is based on the original document, denoted by WED and the other is based on paragraphs, denoted by WEP.
Given two documents $D_1$ and $D_2$, let $\mbox{WMD}(D_1,D_2)$ denote
the distance of $D_1$ and $D_2$.
Given a document $D$, assume that it consists of $\ell$ paragraphs $P_1, \cdots, P_\ell$.
Let $S$ be a summary of $D$.
We compare the word-embedding similarity of a summary $S$ with $D$ using WESM$(S,D)$ as follows:
\begin{equation} \label{eq6}
\mbox{WESM}(S,D) = \frac{1}{\ell}\sum_{i=1}^{\ell} \frac{1}{1 + \mbox{WMD}(S,P_i)}
\end{equation}
%$s$ is system summary and $t$ is original text. $wmd(s, t)$ is the WMD distance between the system summary and the original text, and the result we can obtain from Section \ref{ssec:wmd}.
The value of WESM$(S,D)$ %(Word-Embedding Document comparison) %we can learn that the range of WMD-o valu
is between 0 and 1. Under this measure, the higher the WESM$(S,D)$ value, the more similar $S$ is to $D$. %the original text.

%\paragraph{WEP similarities}

%WED takes semantic aspects into account.
 %We calculate the sum of WED scores with each paragraph of the original text compared with the system summary and then take the average of the sum.
%Assume that a document $D$ consists of $\ell$ paragraphs $P_1, \cdots, P_\ell$. %$t$ is the original text.
%Define WEP$(S,D)$ as follows:
% \begin{equation*} \label{wmd-p}
%\mbox{WEP}(S,D) = \frac{\sum_{i=1}^{\ell}\mbox{WED}(S,P_i)}{\ell}.
%
%{\frac{1}{1+\mbox{WMD}(S,P_{i})}}}{n}
%\end{equation*}
%From the above formula, we can conclude that the range of the value of WEP is from 0 to 1.
%The value of WEP (Word-Embedding Paragraph comparison) %we can learn that the range of WMD-o valu
%is between 0 and 1. The higher the WEP value, the more similar the summary is with the original text.
%We experimented on WMD-o and WMD-p with Word2Vec using different training sets and use them to evaluate eight summarization methods we mentioned in Section %\ref{ssect:sa} compared with ROUGE score %in Section \ref{sec:experiments}.
%Note that WEP measures %we also think of
%the influence of topic diversity to a summary.

\begin{comment}
\subsection{Absolute vs. relative measurements}
We note that different training sets may cause WMD to produce different ranges of values, and
so unless we fix a standard dataset to train Word2Vec,
we do not have a standard base for evaluating the absolute qualities of summaries using WESM. %by different training sets.
However, we can use it to evaluate relative qualities by
the score orderings.
%compare various summarizations over the same training set by looking at the ordering of their values.
%because the relative rank of summarization evaluated by WED does not change.

%On the other hand, we note that  trained on different datasets preserve similar orderings (see Sections 5.2 and 5.3), which provides
%a level of assurance of robustness.
%
% we may train WED on one dataset and use it to measure summaries produced by different algorithms on a different dataset;
%regardless what the training dataset is used, the ordering of WED scores remains relatively the same (see analysis in Section \ref{}).
\end{comment}

\section{\uppercase{Numerical Analysis}}
\label{sec:experiments}

%We describe the datasets in our experiments in the next section. %and report empirical results in this section.
%
%\subsection{Datasets}

\noindent We evaluate the qualities of summarizations using the DUC-02 dataset \cite{DUC02}
and the NewsIR-16 dataset \cite{Signal1M2016}.
DUC-02 consists of 60 reference sets, each of which consists of a number of documents, single-document summary benchmarks, and multi-document abstracts/extracts.
The common ROUGE recall measures of ROUGE-1, ROUGE-2, and ROUGE-SU4 are used to compare the quality of summarization algorithms over DUC data.
%By using DUC-02 data, we can evaluate our methods with other state-of-the-art summarization methods using ROUGE.
%DUC-02 contains single-document summary benchmarks.
NewsIR-16 consists of 1 million articles %that are mainly
from English news media sites and blogs.
%The average length of an article is 405 words.

We use various software packages to implement TextRank (with window size = 2) \cite{TRurl}, RAKE \cite{RAurl}, TexTiling \cite{TTurl}, LDA and Word2Vec \cite{gen}.
% from \url{https://github.com/summanlp/textrank},
%\url{https://github.com/aneesha/RAKE},
%url{https://pypi.python.org/pypi/nltk},
%and \url{https://pypi.python.org/pypi/gensim}.
%
%LDA��Word2Vec: �õ�gensim python������
%https://pypi.python.org/pypi/gensim
%TextTiling�õ���nltk������
%https://pypi.python.org/pypi/nltk
%
%We generate summaries by our summarization algorithms on NewsIR-16 dataset
%and evaluate
%evaluated by WED and WEP to find the summarization method with the best performance.

We use the existing Word2Vec model trained on English Wikipedia \cite{wiki},
%and the other on the GoogleNews-vectors model\footnote{\tt https://github.com/mmihaltz/Word2Vec-\\ \-\hspace{.75cm} GoogleNews-vectors} by genism.
%The English Wikipedia dataset
which consists of 3.75 million articles formatted in XML. The reason to choose this dataset is
for its large size and the diverse topics it covers.
%and the GoogleNews-vectors is the pre-trained Google News corpus that contains 3 million 300-dimension English word vectors.

%To use the WED and WEP measures, we need to train Word2Vec models or use pre-trained Word2Vec models.
% on different English training datasets.
%to present that
%We note that although the ranges of WMD values may differ on various datasets,
%we can still compare the quality of summaries
%for they present the
%
%. % and acquire comparable results with ROUGE.
%In other words, WED and WEP can evaluate the quality of summarization methods using the same dataset and different datasets can still obtain similar results.
%In particular, we trained our Word2Vec models over English Wikipedia\footnote{\tt https://dumps.wikimedia.org/enwiki/latest\\ \-\hspace{.75cm} /enwiki-latest-pages-articles.xml.bz2}
%and used GoogleNews-vectors model\footnote{\tt https://github.com/mmihaltz/Word2Vec-\\ \-\hspace{.75cm} GoogleNews-vectors} for Word2Vec by genism. English Wikipedia dataset contains 3 million %articles formated in XML. GoogleNews-vectors is the pre-trained Google News corpus (3 billion running words) word vector model, which is 3 million 300-dimension English word vectors.

\subsection{ROUGE evaluations over DUC-02}

%We compare the ROUGE-1, ROUGE-2, and ROUGE-SU4 recall scores for single-document summarizations produced by all algorithms over DUC-02.
As mentioned before, we use %Tgraph and
$CP_3$ to cover all previously known algorithms for the purpose of comparing qualities of summaries, as
%Tgraph produces the better results over these algorithms and
$CP_3$ produces the best results among them. %Since the $CP_3$ paper \cite{Parveen2016} does not cover SubmodularF,
%To compare with SubmodularF,
%we use the best parameters trained on DUC-03 \cite{lin:11}.
%Our computation show that the ROUGE-1, ROUGE-2, and ROUGE-Su4 scores for SubmodularF are (39.6, 16.9, 17.8), which are
%much lower than those of $CP_3$.

Among all the algorithms we devise, we only present those with at least one ROUGE recall score better than or equal to the corresponding score of $CP_3$, %Tgraph,
identified in bold %, where R-1, R-2, and R-SU4 stand for ROUGE-1,
%ROUGE-2, and ROUGE-SU4, respectively
(see Table \ref{duc}).
%The results are shown in %. In particular,
%we compare the ROUGE-1, ROUGE-2, and ROUGE-SU4 scores
Also shown in the table is the average of the three ROUGE scores (R-AVG). We can see that
ET3Rank is the winner, followed by T2RAKE; both are superior to $CP_3$.
%We can also see that all our algorithms
%in Table \ref{duc}
%perform better than Tgraph under ROUGE-1 and ROUGE-SU4.
Moreover, ET2RAKE offers the highest
ROUGE-1 score of 49.3.
%where the results for Lead, DUC-02 Best, TextRank, UniformLink, Egraph + Coh., and Tgraph +
%Coh. are copied from \cite{.}.
%
%with ROUGE to check against the state-of-the-art in Table \ref{duc}.
%\textit{Lead} selects the top five ranking sentences.
%DUC-02 Best is best result reported at DUC-02.
%We also compare with other popular summarization methods \textit{TextRank}, \textit{Submodular function},
%\textit{UniformLink (k=10)}, \textit{Egraph} and \textit{Tgraph (n=2000)}.
%We can see that RAKEN and RAKETT all outperform the rest of the algorithms on
%all of the ROUGE-1, ROUGE-2 and ROUGE-SU4 measures, and RAKETT is better than RAKEN on all of these measures. Moreover,
%RAKELAD also outperforms Tgraph on ROUGE-1 and ROUGE-SU4.
%
%perform better than the well known best systems on DUC-02. It shows that our sentence ranking system using RAKE can produce more informative summaries.
\begin{table}[h]
\begin{center}
\caption{\label{duc} ROUGE scores (\%) on DUC-02 data.}
\begin{tabular}{l|c|c|c||c}
\hline
\bf Methods & \bf R-1 & \bf R-2 & \bf R-SU4 & \bf R-AVG \\
\hline
$CP_3$& 49.0 & 24.7 & 25.8 & 33.17 \\
\hline
\bf ET3Rank & \bf 49.2 & \bf 25.6 & \bf 27.5 & \bf 34.10 \\
ESRAKE & \bf 49.0 & 23.6 & \bf 26.1 &  32.90 \\
%EPRAKE & \bf 48.9 & 22.8 & \bf 25.7  & \bf 32.46\\
ET2RAKE & \bf 49.3 & 21.4 & 24.5  & 31.73\\
%ELDARAKE & \bf 48.3 & 21.8 & \bf 24.5  & 31.53\\
PRAKE & \bf 49.0 & 24.5 & 25.3 & 32.93\\
\bf T2RAKE & \bf 49.1 & \bf 25.4 & \bf 25.8 & \bf 33.43 \\
%LDARAKE & \bf 48.3 & 22.5 & \bf 25.3 & 32.03\\
%SubmodularF & 39.6 & 16.9 & 17.8  \\
%Lead & 45.9 & 18.0 & 20.1  \\
%DUC-02 Best & 48.0 & 22.8 &   \\
%TextRank & 47.0 & 19.5 & 21.7  \\
%UniformLink & 47.1 & 20.1 &   \\
%Egraph + Coh. & 47.9 & 23.8 & 23.0  \\
\hline
%Tgraph + Coh. & 48.1 & 24.3 & 24.2 & 32.20 \\
%$CP_3$& 49.0 & 24.7 & 25.8 & 33.17 \\
%\hline
\end{tabular}
\end{center}
\end{table}

\subsection{WESM evaluations over DUC-02 and NewsIR-16}

%In this section, we generated summaries with 30\% and 50\% of the NewsIR-16 dataset using TextRankN, RAKEN, TextRankP, RAKEP, TextRankTT, RAKETT, TextRankLDA and RAKELDA and evaluated them by WMD-o %and WMD-p.
%We evaluate our algorithms listed
%in Table \ref{duc} using WESM over the Word2Vec model trained on English Wikipedia
%and GoogleNews.
%
Table \ref{results1} shows the evaluation results on DUC-02 and NewsIR-16 using WESM based on the Word2Vec model trained on English Wikipedia.
%and GoogleNews, respectively.
%Under each measure of WED and WEP,
The first number in the third row is the average score on all benchmark summaries in DUC-02.
For the rest of the rows, each number is the average
score of summaries produced by the corresponding algorithm for all documents in DUC-02 and NewsIR-16.
The size constraint of a summary on DUC-02 for each document is the same as that of the corresponding DUC-02 summary benchmark.

For NewsIR-16, we select at random 1,000 documents from NewsIR-16 and remove the title, references, and other unrelated content from each article.
%We then merge several articles at random to generate a new article of different sizes up to 10,000 words in an article.
Based on an observation that a 30\% summary allows for a good summary,
we compute 30\% summaries of these articles using each algorithm.
%and compute the average WED and WEP scores. The results are given in Table \ref{NewsIR-16}.
%
%for each algorithm, the
%average score of summaries for all documents in DUC-02 such that
%the size of the summary for each document is the same as that of the corresponding DUC-02 summary benchmark.
%The right column-hand is the average score

%WMD-o and WMD-p are trained on English Wikipedia.
%As shown in Table \ref{results}, RAKETT has the best WMD-o and WMD-p scores with 30\% and the best WMD-p score with 50\%. The summarization methods based on LDA which are TextRankLDA and RAKELDA %have the worst performance. The results are consistent with human judges. We can also conclude that WMD-p can better evaluate summaries than WMD-o, because RAKETT has a lower WMD-o score but a %higher WMD-p score. From WMD-p, we learn that summaries generated by RAKE based methods can produce better performance.

\begin{table}[h]
\begin{center}
\caption{\label{results1} Scores (\%) over DUC-02  and NewsIR-16 under WESM trained on English-Wikipedia.}
\begin{tabular}{l|c|c}
\hline
%\multirow{2}{*}{\bf Methods} &
%\bf Methods &
%\bf Measures & \multicolumn{2}{c}{\bf WESM EW}  & \multicolumn{2}{|c}{\bf WESM GN} \\
%\cline{2-5}
%Methods & WED & WEP & WED & WEP \\
%\hline
\bf Datasets & \bf DUC-02 & \bf NewsIR-16 \\ % & \bf D02 & \bf NIR \\
%{\bf Methods} & \bf WED & \bf WEP \\
\hline
%\bf Methods &
%\multicolumn{2}{c}{\bf WED}  & \multicolumn{2}{|c}{\bf WEP} \\
%\cline{2-3}
%\bf Methods &
Benchmarks & 3.021  & \\% &67.96 & \\
\hline









ET3Rank &\bf 3.382 &\bf 2.002 \\ % &\bf 69.45      &\bf 17.29 \\
ESRAKE      &3.175 & 1.956  \\ %&69.20      &17.15     \\
%EPRAKE      &3.154 &1.964   \\ %   &69.21     &17.04      \\
ET2RAKE     &3.148 &1.923   \\ %   &69.03      &17.01      \\
%ELDARAKE    &3.149 &1.960   \\ %   & 68.92  &17.02     \\
PRAKE       &3.150 &1.970   \\ %  &68.85      &17.21     \\
T2RAKE      &3.247 &1.990   \\ %  &69.17      &17.19    \\
%LDARAKE     &3.157 &1.914   \\ %   &68.45      &17.19    \\
\hline
\end{tabular}
\end{center}
\end{table}
It is expected that scores of our algorithms are better than the score for benchmarks under each measure, for the benchmarks often use different words not in the original documents, and hence would
have smaller similarities.

%We also note that WESM trained
%on
%To evaluate our algorithms over NewsIR-16 under WED and WEP, we select at random 1,000 documents from NewsIR-16 and remove the title, references, and other unrelated content from each article.
%We then merge several articles at random to generate a new article of different sizes up to 10,000 words in an article.
%Based on an observation that a 30\% summary allows for a good summary,
%we compute 30\% summaries of these articles using each algorithm
%and compute the average WED and WEP scores. The results are given in Table \ref{NewsIR-16}.

%summaries and average the scores
%
%The results shown in Tables \ref{results} and \ref{results1} are the average scores of the scores for each article's summary.

\begin{comment}
\begin{table}[h]
\begin{center}
\begin{tabular}{l|c|c|c|c}
\hline
\multirow{2}{*}{\bf Methods} &
%\bf Methods &
\multicolumn{2}{c}{\bf E-Wikipedia}  & \multicolumn{2}{|c}{\bf GoogleNews} \\
\cline{2-5}
%Methods & WED & WEP & WED & WEP \\
& \bf WED & \bf WEP & \bf WED & \bf WEP \\
\hline
ET3Rank &\bf 5.695  &\bf 2.002  &\bf 23.25  &\bf 11.32 \\
ESRAKE  &5.398      &1.956          &23.12  &11.18 \\
EPRAKE  &5.377      &1.964          &22.86  &11.21 \\
ET2RAKE &5.585      &1.922          &22.85  &11.17 \\
ELDARAKE &5.359     &1.960          &22.96  &11.08 \\
PRAKE   &5.377      &1.970          &23.14  &11.28 \\
T2RAKE &5.688       &1.990          &23.17  &11.20 \\
LDARAKE &5.554      &1.914          &23.18  &11.20 \\
\hline
\end{tabular}
\caption{\label{NewsIR-16} Scores (\%) over NewsIR-16 under WED and WEP trained on English Wikipedia and GoogleNews}
\end{center}
\end{table}
%Figure \ref{fig:fig5} shows a 30\% summary example of an article in the NewsIR-16 dataset by the RAKETT method. In Figure \ref{fig:fig5}, the left side is the content of the article and the right %side is the summary generated by the RAKETT method. The RAKETT method can merge sentences from the same subtopic, let the summary cover as many subtopics as possible and trim sentences from semantic %aspects.
\end{comment}

\subsection{Normalized $L_1$-norm}

We would like to determine if WESM is a viable measure. From our experiments, we know that the all-around best algorithm ET3Rank, the second best
algorithm T2RAKE, and ET2RAKE remain the same positions under R-AVG over DUC-02 and under WESM over both DUC-02 and NewsIR-16 (see Table \ref{ordering}),
ESRAKE and PRAKE remain the same positions under R-AVG over DUC-02 and under WESM over NewsIR-16, while ESRAKE and PRAKE only differ by one place under R-AVG and WESM over DUC-02.
\begin{table}[h]
\begin{center}
\caption{\label{ordering} Orderings of R-AVG scores over DUC-02 and
WESM scores over DUC-02 and NewsIR-16.}
\begin{tabular}{l|c|c|c}
\hline
\multirow{2}{*}{\bf Methods} &
%\bf Methods &
%\bf Measures &
\bf R-AVG  & \multicolumn{2}{|c}{\bf WESM} \\
\cline{2-4}
%Methods
& DUC-02 & DUC-02 & NewsIR-16 \\
\hline
%\bf Datasets & \bf DUC-02 & \bf NewsIR-16 \\ % & \bf D02 & \bf NIR \\
%{\bf Methods} & \bf WED & \bf WEP \\
%\hline
%\bf Methods &
%\multicolumn{2}{c}{\bf WED}  & \multicolumn{2}{|c}{\bf WEP} \\
%\cline{2-3}
%\bf Methods &
%Benchmarks & 4.22  & \\% &67.96 & \\
%\hline
\bf ET3Rank     &\bf 1  &\bf 1 &\bf 1  \\ % &\bf 69.45      &\bf 17.29 \\
ESRAKE          &4  &3 &4  \\ %&69.20      &17.15     \\
%\bf EPRAKE      &\bf 5  &\bf 5 &\bf 4  \\ %   &69.21     &17.04      \\
\bf ET2RAKE     &\bf 5  &\bf 5 &\bf 5  \\ %   &69.03      &17.01      \\
%ELDARAKE      &8  &7 &5   \\ %   & 68.92  &17.02     \\
PRAKE          &3  &4 &3  \\ %  &68.85      &17.21     \\
\bf T2RAKE      &\bf 2  &\bf 2 &\bf 2  \\ %  &69.17      &17.19    \\
%LDARAKE     &6  &4 &8  \\ %   &68.45      &17.19    \\
\hline
& ${\bm O}_1$ & ${\bm O}_2$ & ${\bm O}_3$ \\
\hline
\end{tabular}
\end{center}
\end{table}

Next, we compare the ordering of the R-AVG scores and the WESM scores over DUC-02. For this purpose, we use the normalized $L_1$-norm to compare the distance of two orderings. Let ${\bm X} = (x_1, x_2, \cdots, x_k)$ be a sequence of $k$ objects, where
each $x_i$ has two values $a_i$ and $b_i$ such that
$a_1, a_2, \ldots, a_k$ and $b_1,b_2,\ldots, b_k$ are, respectively, permutations of $1,2,\ldots,k$.
Let
\[
D_k= \sum_{i=1}^k |(k-i+1)-i|,
\]
which is the maximum distance two permutations can possibly have. Then
the normalized $L_1$-norm of ${\bm A} = (a_1, a_2, \cdots, b_k)$ and ${\bm B} = (b_1, b_2, \cdots, b_k)$ is defined by
$$||{\bm A}, {\bm B}||_1 = \frac{1}{D_k}\sum_{i=1}^k |a_i - b_i|.$$
Table \ref{ordering} shows the orderings of the R-AVG scores over DUC-02 and WESM scores over DUC-02 and NewsIR-16 (from Tables \ref{duc} and \ref{results1}).

It is straightforward to see that $D_5 = 12$,
$||{\bm O}_1, {\bm O}_2||_1 = ||{\bm O}_2, {\bm O}_3||_1 = 2/12 = 1/6$ and $||{\bf O}_1,{\bf O}_3||_1 = 0$.
%Note that
%the average $L_1$-norm on $(1,2,\cdots, 5)$ and the revers order is 2.4.
This indicates that WESM and ROUGE are highly comparable over DUC-02 and NewsIR-16,
and the orderings of WESM on different datasets, while with larger spread, are
still similar.

\begin{comment}
\subsection{WED and WEP evaluations on a different model}

We now apply a different model of Word2Vec trained on the GoogleNews dataset to evaluate the algorithms listed in Tabel \ref{duc}
over DUC-02 and NewsIR-16.
The results are shown in Table \ref{other}.

%Results on DUC-02 data are shown in Table \ref{other}. We used WMD-o and WMD-p trained by GoogleNews datasets to evaluate our summarization methods. o-s is the score of WMD-o comparing our system %summary with the DUC document. p-r is the score of WMD-p comparing the DUC reference summary with the DUC document. p-s is the score of WMD-p comparing our system summary with the DUC document. o-r %is the score of WMD-o comparing the DUC reference summary with the DUC document. From Table \ref{other}, although the absolute values of WMD-o and WMD-p are larger than the results in Table %\ref{results} because of using different training sets, we can still conclude the similar results with Table \ref{results} that RAKETT has the best performance and our methods are better than the %submodular function. The scores of WMD-o and WMD-p with our system summaries are higher than with DUC reference summaries.
\begin{table}[h]
\begin{center}
\begin{tabular}{l|c|c|c|c}
\hline
\multirow{2}{*}{\bf Methods} &
%\bf Methods &
\multicolumn{2}{c}{\bf WED}  & \multicolumn{2}{|c}{\bf WEP} \\
\cline{2-5}
%Methods & WED & WEP & WED & WEP \\
& 30\% & 50\% & 30\% & 50\% \\
\hline
\bf ET3Rank &82.32      &\bf 81.38&\bf 55.21&\bf 54.53 \\
ESRAKE      &83.38      &\bf 81.38&54.92    &\bf 54.53 \\
EPRAKE       &83.69     &\bf 81.38&55.03    &\bf 54.53  \\
ET2RAKE     &83.12      &\bf 81.38&54.94    &\bf 54.53  \\
ELDARAKE    &\bf 83.71  &\bf 81.38&55.11    &\bf 54.53 \\
PRAKE       &82.46      &\bf 81.38&54.81    &\bf 54.53 \\
T2RAKE      &82.10      &\bf 81.38&54.77    &\bf 54.53\\
LDARAKE     &82.02      &\bf 81.38&54.87    &\bf 54.53\\
\hline
\end{tabular}
\caption{\label{other1} Scores (\%) of GoogleNews-based WED and WEP on 30\% and 50\% summaries over DUC-02}
\end{center}
\end{table}

\begin{table}[h]
\begin{center}
\begin{tabular}{l|c|c|c|c}
\hline
\multirow{2}{*}{\bf Methods} &
%\bf Methods &
\multicolumn{2}{c}{\bf WED}  & \multicolumn{2}{|c}{\bf WEP} \\
\cline{2-5}
%Methods & WED & WEP & WED & WEP \\
& 30\% & 50\% & 30\% &50\% \\
\hline
\bf ET3Rank &\bf 23.35 &\bf 23.55   &\bf 11.36 &11.38 \\
ESRAKE      &23.22      &23.08      &11.31      &11.34 \\
EPRAKE      &23.06      &23.25      &11.21      &11.31  \\
ET2RAKE     &23.08      &23.38      &11.24      &  11.33\\
ELDARAKE    &23.25      &\bf 23.55  &11.26      &\bf 11.42  \\
PRAKE       &23.06      &23.25      &11.21      &11.31 \\
T2RAKE      &23.18      &23.18      &11.23      &11.34 \\
LDARAKE     &23.25      &23.28      &11.32      &11.33\\
\hline
\end{tabular}
\caption{\label{other} Scores (\%) of GoogleNews-based WED and WEP on 30\% and 50\% summaries over NewsIR-16}
\end{center}
\end{table}
\end{comment}

\subsection{Runtime analysis}

We carried out runtime analysis through experiments on a computer with a 3.5 GHz Intel Xeon CPU E5-1620 v3. %2.7 GHz dual-core Intel Core i5 CPU and 8 GB memory.
We used a Python implementation of our summarization algorithms. %methods on the NewsIR-16 dataset.
Since DUC-02 are short, all but LDA-based algorithms run in about the same time.
To obtain a finer distinction, we ran our experiments on NewsIR-16. Since the average size of NewsIR-16 articles is 405 words,
we selected at random a number of articles from NewsIR-16 and merged them to generate a new article.
For each size from around 500 to around 10,000 words, with increments of 500 words, we selected at random 100 articles and
computed the average runtime of different algorithms to produce 30\% summary (see Figure \ref{fig:runtime}).
We note that the time complexity of each of our algorithms incurs mainly in
the preprocessing phase; %that is, the runtime incurs in the sentence selection phase
%is minor; namely,
the size of summaries in the sentence selection phase only incur minor fluctuations of computation time, and
so it suffice to compare the runtime for producing 30\% summaries.

%and use summary rate with 30\%, 50\% and 70\%.
\begin{figure}[h]
\centering
\includegraphics[width=3.3in]{runtime-color.png}
\caption{Runtime analysis, where the unit on the x-axis is 100 words and the unit of the y-axis is seconds.} \label{fig:runtime}
\end{figure}
We can see from Figure \ref{fig:runtime} that ESRAKE and PRAKE incur about the same linear time and they are extremely fast.
Also, ET3RANK, ET2RAKE, and T2RAKE incur about the same time. While the time is higher because of the use of TextTiling and
is closed to being linear, it meets the realtime requirements. For example, for a document of up to 3,000 words, over
3,000 but less than 5,500 words, and 10,000 words, respectively,
the runtime of ET3Rank is under 0.5, 1, and 2.75 seconds.
%,
%
%for a document less than 5,500 words,
%the runtime is
%less than 1 second, % and for a very long document of 10,000 words, the runtime is
%and less than 2.75 seconds.

The runtime of SubmodularF is acceptable for documents of moderate sizes (not shown in the paper); but for a document of about 10,000 words, the runtime is close to 4 seconds.
LDA-based algorithms is much higher. For example, LDARAKE incurs about 16 seconds for a document of
about 2,000 words, about 41 seconds for a document of about 5,000 words, and about 79 seconds for a document of about 10,000 words.

%\begin{table}
%\begin{tabular}
%
%\end{tabular}
%\end{table}




\begin{comment}
\begin{table}[h]
\begin{center}
\begin{tabular}{l|c|c|c|c}
\hline
\bf Methods & \bf 10\% & \bf 30\% & \bf 50\% & \bf 70\% \\
\hline
\bf ET3Rank & &&& \\
ESRAKE & &&& \\
EPRAKE & &&&  \\
ET2RAKE & &&&  \\
ELDARAKE & &&&  \\
PRAKE & &&& \\
T2RAKE & &&& \\
LDARAKE & &&&\\
\hline
\end{tabular}
\caption{\label{runtime} Runtime (seconds) of different summary rate on the NewsIR-16 dataset.}
\end{center}
\end{table}

The Runtime of different summary rate on the NewsIR-16 dataset is given in Table \ref{runtime}. We can find that RAKE based summarization methods are much faster than TextRank based ones, which is bacause TextRank needs to calculate iterations until it converges while RAKE only calculates word co-occurrence. The runtime of generating summaries with different rate is similar and LDA based methods take more time.
\end{comment}

\section{\uppercase{Conclusions}}
\label{sect:conclusion}
\noindent We presented a number of unsupervised single-document summarization algorithms for generating effective summaries in realtime and
a new measure based on word-embedding similarities to evaluate the quality of a summary. We showed that ET3Rank is the best all-around algorithm. A web-based summarization tool using ET3Rank and T2RAKE will be made available to the public.

To further obtain better topic clusterings efficiently, we plan to extend TextTiling over non-consecutive paragraphs.
To obtain a better understanding of word-embedding similarity measures, we plan to compare WESM with human evaluation and
other unsupervised methods including those devised by Louis and Nenkova \cite{Louis:2009}.
We also plan to
explore new ways to measure summary qualities
without human-generated benchmarks.


\section*{\uppercase{Acknowledgements}}

\noindent
We thank the members of the Text Automation Lab at UMass Lowell for their support and fruitful discussions.



%the major topics of the original document as diverse as possible. We also

%We take sentence ranking and topic diversity coverage into account. We also present summarization evaluation methods based on Word2Vec without reference summaries. We experimented our system with several news datasets. Our summarization methods improve substantially over state-of-the-art systems on ROUGE while still maintaining good linguistic quality and the runtime is competitive. Our summarization evaluation methods can produce comparable results with ROUGE. We plan to explore sentence compression model with pronoun anaphora and trim sentences from phrases instead of words.

% \section*{\uppercase{Acknowledgements}}


\vfill
\bibliographystyle{apalike}
{\small
\bibliography{example}}


% \section*{\uppercase{Appendix}}

\vfill
\end{document}

 
%\section{Experiments}
\label{sec:experiments}
\subsection{Experimental Setup}
We evaluate the effectiveness of our approach on three different domain adaptation datasets: DomainNet~\cite{peng2019moment}, Office-Home~\cite{Venkateswara2017DeepHN} and Office31~\cite{Saenko2010AdaptingVC}. DomainNet ~\cite{peng2019moment} is a large-scale domain adaptation dataset with 345 classes across 6 domains. Following MME ~\cite{Saito2019SemiSupervisedDA}, we use a subset of the dataset containing 126 categories across four domains: Real(R), Clipart(C), Sketch(S), and Painting(P). The performance on DomainNet is evaluated using 7 different combinations out of possible 12 combinations. Office-Home~\cite{Venkateswara2017DeepHN} is another widely used domain adaptation benchmark dataset with 65 classes across four domains: Art(Ar), Product(Pr), Clipart(Cl), and Real (Rl). We perform experiments on all possible combinations of 4 domains. Office31~\cite{Saenko2010AdaptingVC} is a relatively smaller dataset containing just 31 categories of data across three domains- Amazon(A), Dslr(D), Webcam(W). Following prior work ~\cite{Saito2019SemiSupervisedDA, Kim2020AttractPA}, we evaluate our approach on two combinations for the office31 dataset. 

For the fair comparison, we use the data-splits (train, validation, and test splits) released by ~\cite{Saito2019SemiSupervisedDA} on Github \footnote{\url{https://github.com/VisionLearningGroup/SSDA_MME}}. We use the same settings for the benchmark datasets as in the prior work ~\cite{Saito2019SemiSupervisedDA, Kim2020AttractPA}, including the number of labeled samples in the target domain, which are consistent across all experiments.

\subsection{Implementation Details}
Similar to the previous works on SSDA ~\cite{Saito2019SemiSupervisedDA, Kim2020AttractPA, Li2020OnlineMF}, we use Resnet34 and Alexnet as the backbone networks in our paper. We only used VGG for Office31 due to its higher memory requirements. The feature generator model is initialized with ImageNet weights, and the classifier is randomly initialized and has the same architecture as in ~\cite{Saito2019SemiSupervisedDA, Kim2020AttractPA, Li2020OnlineMF}. All our experiments are performed using Pytorch ~\cite{Paszke2019PyTorchAI}.We use an identical set of hyperparameters ($\alpha= 4$, $\beta= 1$ ) across all our experiments other than minibatch size. All the hyperparameters values are decided using validation performance on Product to Art experiments on the Office-Home dataset. We have set $\tau=5$ in our experiments. Each minibatch of size $B$  contains an equal number of source and labeled target examples, while the number of unlabeled target samples is $\mu \times B$. We study the effect of $\mu$ in section \ref{ablation}. Resnet34 experiments are performed with minibatch size, $B= 32$ and Alexnet models are trained with $B= 24$. We use $\mu=4$ for all our experiments. We use SGD optimizer with a momentum of $0.9$ and an initial learning rate of $0.01$ with cosine learning rate decay for all our experiments. Weight decay is set to $0.0005$ for all our models. Other details of the experiments are included in the Appendix.
\subsection{Baselines}
We compare our CLDA framework with previous state-of-the-art SSDA approaches : \textbf{MME} ~\cite{Saito2019SemiSupervisedDA}, \textbf{APE} ~\cite{Kim2020AttractPA}, \textbf{BiAT} ~\cite{Jiang2020BidirectionalAT} , \textbf{UODA} ~\cite{Qin2020Contradictory}, ~\textbf{Meta-MME} ~\cite{Li2020OnlineMF} and ~\textbf{ENT} ~\cite{Grandvalet2004SemisupervisedLB} using the performance reported by these papers. papers. We also included the results from adversarial based baseline methods:
% DANN [12] ,ADR
% 212 [43] and CDANWe also include the results from UDA (Unsupervised Domain Adaptation) based baseline methods using the adversarial approach for the Semi-Supervised Domain Adaptation task: 
\textbf{DANN} ~\cite{Ganin2016DomainAdversarialTO}, \textbf{ADR} ~\cite{Saito2018AdversarialDR} and \textbf{CDAN} ~\cite{Long2018ConditionalAD} as reported in \cite{Saito2019SemiSupervisedDA}. We also provide the \textbf{S+T} results where the model is trained using all the labeled samples across domains.
\begin{table}[!t]
\caption{ \textbf{Performance Comparison in Office-Home.} Numbers show top-1 accuracy values for different domain adaptation scenarios under 3-shot setting using Alexnet and Resnet34 as backbone networks. We have highlighted the best method for each transfer task. CLDA surpasses all the baseline methods in most adaptation scenarios. Our Proposed framework achieves the best average performance among all compared methods.
}
\renewcommand{\arraystretch}{1.2}
\vspace{2mm}
\centering
\label{base_office_home_table}
\resizebox{\columnwidth}{!}{
\begin{tabular}{c|c|cccccccccccc|c}
\specialrule{.1em}{.05em}{.05em}
Net & Method & Rl$\rightarrow$Cl & Rl$\rightarrow$Pr & Rl$\rightarrow$Ar & Pr$\rightarrow$Rl & Pr$\rightarrow$Cl & Pr$\rightarrow$Ar & Ar$\rightarrow$Pl & Ar$\rightarrow$Cl & Ar$\rightarrow$Rl & Cl$\rightarrow$Rl & Cl$\rightarrow$Ar & Cl$\rightarrow$Pr & Mean \\
\hline
\multirow{8}{*}{Alexnet} & S+T & 44.6 & 66.7 & 47.7 & 57.8 & 44.4 & 36.1 & 57.6 & 38.8 & 57.0 & 54.3 & 37.5 & 57.9 & 50.0 \\
 & DANN & 47.2 & 66.7 & 46.6 & 58.1 & 44.4 & 36.1 & 57.2 & 39.8 & 56.6 & 54.3 & 38.6 & 57.9 & 50.3 \\
 & ADR  & 37.8 & 63.5 & 45.4 & 53.5 & 32.5 & 32.2 & 49.5 & 31.8 & 53.4 & 49.7 & 34.2 & 50.4 & 44.5 \\
& CDAN & 36.1 & 62.3 & 42.2 & 52.7 & 28.0 & 27.8 & 48.7 & 28.0 & 51.3 & 41.0 & 26.8 & 49.9 & 41.2 \\
 & ENT & 44.9 & 70.4 & 47.1 & 60.3 & 41.2 & 34.6 & 60.7 & 37.8 & 60.5 & 58.0 & 31.8 & 63.4 & 50.9 \\
 & MME & 51.2 & 73.0 & 50.3 & 61.6 & 47.2 & 40.7 & 63.9 & 43.8 & 61.4 & 59.9 & 44.7 & 64.7 & 55.2 \\
 & Meta-MME & 50.3 & - & - & - & 48.3 & 40.3 & - & 44.5 & - & - & 44.5 & - & - \\
 & BiAT & - & - & - & - & - & - & - & - & - & - & - & - & 56.4 \\
 & APE & \textbf{51.9} & \textbf{74.6} & 51.2 & 61.6 & 47.9 & 42.1 & 65.5 & 44.5 & 60.9 & 58.1 & 44.3 & 64.8 & 55.6 \\
 & \textbf{CLDA}(ours) & 51.5 & 74.1 & \textbf{54.3} & \textbf{67.0} & \textbf{47.9} & \textbf{47.0} & \textbf{65.8} & \textbf{47.4} & \textbf{66.6} & \textbf{64.1} & \textbf{46.8} & \textbf{67.5} & \textbf{58.3} \\
\hline
\hline
\multirow{7}{*}{Resnet34} & S+T & 55.7 & 80.8 & 67.8 & 73.1 & 53.8 & 63.5 & 73.1 & 54.0 & 74.2 & 68.3 & 57.6 & 72.3 & 66.2 \\
 & DANN & 57.3 & 75.5 & 65.2 & 69.2 & 51.8 & 56.6 & 68.3 & 54.7 & 73.8 & 67.1 & 55.1 & 67.5 & 63.5 \\
 & ENT & 62.6 & 85.7 & 70.2 & 79.9 & 60.5 & 63.9 & 79.5 & 61.3 & 79.1 & 76.4 & 64.7 & 79.1 & 71.9 \\
 & MME & 64.6 & 85.5 & 71.3 & 80.1 & 64.6 & 65.5 & 79.0 & 63.6 & 79.7 & 76.6 & 67.2 & 79.3 & 73.1 \\
 & Meta-MME & 65.2 & - & - & - & 64.5 & 66.7 & - & 63.3 & - & - & 67.5 & - & - \\
 & APE & \textbf{66.4} & 86.2 & 73.4 & 82.0 & \textbf{65.2} & 66.1 & 81.1 & \textbf{63.9} & 80.2 & 76.8 & 66.6 & 79.9 & 74.0 \\
 & \textbf{CLDA} (ours) & 66.0 & \textbf{87.6} & \textbf{76.7} & \textbf{82.2} & 63.9 & \textbf{72.4} & \textbf{81.4} & 63.4 & \textbf{81.3} & \textbf{80.3} & \textbf{70.5} & \textbf{80.9} & \textbf{75.5} \\
\specialrule{.1em}{.05em}{.05em}
\end{tabular}}

\vspace{2mm}
\end{table}
% %------------------------------------------------------------------------- 
% %-------------------------------------------------------------------------
\subsection{Results}
Table  ~\ref{base_office_home_table}- ~\ref{base_office_table} show  top-1 accuracies  and mean accuracies for different combination of domain adaptation scenarios for all three datasets in comparison with baseline SSDA methods.

\noindent\textbf{Office-Home.} Table ~\ref{base_office_home_table} contains the results of the Office-Home dataset for 3-shot setting with Alexnet and Resnet34 as backbone networks. Results for the $1$-shot adaptation scenarios are included in the Appendix ~\ref{office_home_1_shot}. 
Our method consistently performs better than the baseline approaches and achieves $58.3\%$  and $75.5\%$ mean accuracy with Alexnet and Resnet34, respectively. Our approach surpasses the state-of-the-art SSDA approaches in most of the adaptation tasks. In some domain adaptation cases, such as Cl $\rightarrow$ Rl, Rl $\rightarrow$ Ar and Pr $\rightarrow$ Ar, we exceeded APE by more than $3\%$.

\noindent\textbf{DomainNet}: Our CLDA approach surpasses the performance of existing SSDA baselines as shown in Table ~\ref{base_domainNet_table}. Using Alexnet backbone, our method improves over BiAT by $5.2\%$ and $4.9\%$ in 1-shot and 3-shot settings, respectively. We obtain similarly improved performance when we switch the neural backbone from Alexnet to Resnet34. With Resnet34 as the backbone, we gain $4.3\%$ and $3.6\%$ over APE in 1-shot and 3-shot settings, respectively. Similar to the Office-Home, our approach surpasses the well-known domain adaptation benchmarks methods in most domain adaptation tasks of the DomainNet dataset. Such consistent improved performance shows that our approach reduces both inter and intra domain discrepancy prevalent in SSDA. 

\noindent\textbf{Office31}: Similar to other datasets, our proposed method with Alexnet and VGG as neural backbone achieves the best performance in both domain adaption scenarios for office31 as shown in Table ~\ref{base_office_table}. Using Alexnet backbone, we beat the APE ~\cite{Kim2020AttractPA} by $3.2\%$ in 3-shot and BiAT by $7.3\%$ in 1-shot settings. We observe similar gains over all the baselines methods with VGG as the neural network backbone. This shows the efficacy of our proposed approach irrespective of the used backbone.







\begin{table}[!t]

\caption{ \textbf{Performance Comparison in DomainNet.} Numbers show Top-1 accuracy values for different domain adaptation scenarios under 1-shot and 3-shot settings using Alexnet and Resnet34 as backbone networks. CLDA achieves better performance than all the baseline methods in most of the domain adaptation tasks. We have highlighted the best approach for each domain adaptation task. Our Proposed framework achieves the best average performance among all compared methods.
}
\renewcommand{\arraystretch}{1.2}
\vspace{2mm}
\label{base_domainNet_table}
\begin{center}{
\resizebox{\columnwidth}{!}{
\begin{tabular}{c|c|cccccccccccccc|cc}
\specialrule{.1em}{.05em}{.05em}
\multirow{2}{*}{Net} & \multirow{2}{*}{Method} & \multicolumn{2}{c}{R$\rightarrow$C} & \multicolumn{2}{c}{R$\rightarrow$P} & \multicolumn{2}{c}{P$\rightarrow$C} & \multicolumn{2}{c}{C$\rightarrow$S} & \multicolumn{2}{c}{S$\rightarrow$P} & \multicolumn{2}{c}{R$\rightarrow$S} & \multicolumn{2}{c|}{P$\rightarrow$R} & \multicolumn{2}{c}{Mean} \\
 & & 1-shot & 3-shot & 1-shot & 3-shot & 1-shot & 3-shot & 1-shot & 3-shot & 1-shot & 3-shot & 1-shot & 3-shot & 1-shot & 3-shot & 1-shot & 3-shot \\ \hline
\multirow{8}{*}{Alexnet} & S+T & 43.3 & 47.1 & 42.4 & 45.0 & 40.1 & 44.9 & 33.6 & 36.4 & 35.7 & 38.4 & 29.1 & 33.3 & 55.8 & 58.7 & 40.0 & 43.4 \\
 & DANN & 43.3 & 46.1 & 41.6 & 43.8 & 39.1 & 41.0 & 35.9 & 36.5 & 36.9 & 38.9 & 32.5 & 33.4 & 53.5 & 57.3 & 40.4 & 42.4 \\
 & ADR      & 43.1 & 46.2 & 41.4 & 44.4 & 39.3 & 43.6 & 32.8 & 36.4 & 33.1 & 38.9 & 29.1 & 32.4 & 55.9 & 57.3 & 39.2 & 42.7 \\
 & CDAN     & 46.3 & 46.8 & 45.7 & 45.0 & 38.3 & 42.3 & 27.5 & 29.5 & 30.2 & 33.7 & 28.8 & 31.3 & 56.7 & 58.7 & 39.1 & 41.0 \\
 & ENT & 37.0 & 45.5 & 35.6 & 42.6 & 26.8 & 40.4 & 18.9 & 31.1 & 15.1 & 29.6 & 18.0 & 29.6 & 52.2 & 60.0 & 29.1 & 39.8 \\
 & MME & 48.9 & 55.6 & 48.0 & 49.0 & 46.7 & 51.7 & 36.3 & 39.4 & 39.4 & 43.0 & 33.3 & 37.9 & 56.8 & 60.7 & 44.2 & 48.2 \\
 & Meta-MME & - & 56.4 & - & 50.2 & & 51.9 & - & 39.6 & - & 43.7 & - & 38.7 & - & 60.7 & - & 48.8 \\
 & BiAT & 54.2 & 58.6 & 49.2 & 50.6 & 44.0 & 52.0 & 37.7 & 41.9 & 39.6 & 42.1 & 37.2 & 42.0 & 56.9 & 58.8 & 45.5 & 49.4 \\
 & APE & 47.7 & 54.6 & 49.0 & 50.5 & 46.9 & 52.1 & 38.5 & 42.6 & 38.5 & 42.2 & 33.8 & 38.7 & 57.5 & 61.4 & 44.6 & 48.9 \\
 & \textbf{CLDA} (ours) & \textbf{56.3} & \textbf{59.9} & \textbf{56.0} & \textbf{57.2} & \textbf{50.8} & \textbf{54.6} & \textbf{42.5} & \textbf{47.3} & \textbf{46.8} & \textbf{51.4} & \textbf{38.0} & \textbf{42.7} & \textbf{64.4} & \textbf{67.0} & \textbf{50.7} & \textbf{54.3} \\ 
\hline
\hline
\multirow{9}{*}{Resnet34} & S+T & 55.6 & 60.0 & 60.6 & 62.2 & 56.8 & 59.4 & 50.8 & 55.0 & 56.0 & 59.5 & 46.3 & 50.1 & 71.8 & 73.9 & 56.9 & 60.0 \\
 & DANN & 58.2 & 59.8 & 61.4 & 62.8 & 56.3 & 59.6 & 52.8 & 55.4 & 57.4 & 59.9 & 52.2 & 54.9 & 70.3 & 72.2 & 58.4 & 60.7 \\
  & ADR      & 57.1 & 60.7 & 61.3 & 61.9 & 57.0 & 60.7 & 51.0 & 54.4 & 56.0 & 59.9 & 49.0 & 51.1 & 72.0 & 74.2 & 57.6 & 60.4 \\
 & CDAN     & 65.0 & 69.0 & 64.9 & 67.3 & 63.7 & 68.4 & 53.1 & 57.8 & 63.4 & 65.3 & 54.5 & 59.0 & 73.2 & 78.5 & 62.5 & 66.5 \\
 & ENT & 65.2 & 71.0 & 65.9 & 69.2 & 65.4 & 71.1 & 54.6 & 60.0 & 59.7 & 62.1 & 52.1 & 61.1 & 75.0 & 78.6 & 62.6 & 67.6 \\
 & MME & 70.0 & 72.2 & 67.7 & 69.7 & 69.0 & 71.7 & 56.3 & 61.8 & 64.8 & 66.8 & 61.0 & 61.9 & 76.1 & 78.5 & 66.4 & 68.9 \\
 & UODA & 72.7 & 75.4 & 70.3 & 71.5 & 69.8 & 73.2 & 60.5 & 64.1 & 66.4 & 69.4 & 62.7 & 64.2 & 77.3 & 80.8 & 68.5 & 71.2 \\
 & Meta-MME & - & 73.5 & - & 70.3 & - & 72.8 & - & 62.8 & - & 68.0 & - & 63.8 & - & 79.2 & - & 70.1 \\
 & BiAT & 73.0 & 74.9 & 68.0 & 68.8 & 71.6 & 74.6 & 57.9 & 61.5 & 63.9 & 67.5 & 58.5 & 62.1 & 77.0 & 78.6 & 67.1 & 69.7 \\
 & APE & 70.4 & 76.6 & 70.8 & 72.1 & \textbf{72.9} & \textbf{76.7} & 56.7 & 63.1 & 64.5 & 66.1 & 63.0 & 67.8 & 76.6 & 79.4 & 67.6 & 71.7 \\
 & \textbf{CLDA} (ours) & \textbf{76.1} & \textbf{77.7} & \textbf{75.1} & \textbf{75.7} & 71.0 & 76.4 & \textbf{63.7} & \textbf{69.7} & \textbf{70.2} & \textbf{73.7} & \textbf{67.1} & \textbf{71.1} & \textbf{80.1} & \textbf{82.9} & \textbf{71.9} & \textbf{75.3} \\ 
 \specialrule{.1em}{.05em}{.05em}
 
\end{tabular}}}
\end{center}
\end{table}
%------------------------------------------------------------------------- 

% %-------------------------------------------------------------------------

\begin{table}[t]
\centering
\caption{ \textbf{Performance Comparison in Office31.} Numbers show Top-1 accuracy values for different domain adaptation scenarios under 1-shot and 3-shot settings using Alexnet and VGG as backbone networks. CLDA outperforms all the baseline approaches in both scenarios. We have highlighted the superior method on each domain adaptation task. Our Proposed framework achieves the best mean accuracy among all baseline methods.
}
\renewcommand{\arraystretch}{1.2}
\label{base_office_table}
\begin{center}{
\resizebox{\columnwidth}{!}{
\begin{tabular}{c|cc|cc|cc||cc|cc|cc}
\specialrule{.1em}{.05em}{.05em}
\multicolumn{7}{c}{Alexnet} & \multicolumn{6}{c}{VGG} \\
\hline
 \multirow{3}{*}{Method} & \multicolumn{2}{c}{W$\rightarrow$A} & \multicolumn{2}{c|}{D$\rightarrow$A} & \multicolumn{2}{c}{Mean} & \multicolumn{2}{c}{W$\rightarrow$A} & \multicolumn{2}{c|}{D$\rightarrow$A} & \multicolumn{2}{c}{Mean}\\
 & 1-shot & 3-shot & 1-shot & 3-shot & 1-shot & 3-shot & 1-shot & 3-shot & 1-shot & 3-shot & 1-shot & 3-shot \\ \hline
 S+T & 50.4 & 61.2 & 50.0 & 62.4 & 50.2 & 61.8 &169.2 &73.2 &68.2 &73.3 &68.7 &73.25 \\
 DANN & 57.0 & 64.4 & 54.5 & 65.2 & 55.8 & 64.8 &69.3 &75.4 &70.4 &74.6 &69.85 &75.0\\
 ADR & 50.2 & 61.2 & 50.9 & 61.4 & 50.6 & 61.3 &69.7 &73.3 &69.2 &74.1 &69.45 &73.7\\
 CDAN & 50.4 & 60.3 & 48.5 & 61.4 & 49.5 & 60.8 &65.9 &74.4 &64.4 &71.4 &65.15 &72.9\\
 ENT & 50.7 & 64.0 & 50.0 & 66.2 & 50.4 & 65.1 &69.1 &75.4 &72.1 &75.1 &70.6 &75.25\\
 MME & 57.2 & 67.3 & 55.8 & 67.8 & 56.5 & 67.6 &73.1 &76.3 &73.6 &\textbf{77.6} &73.35 &76.95\\
BiAT & 57.9 & 68.2 & 54.6 & 68.5 & 56.3 & 68.4 &- &- &- &- &- &- \\
 APE & - & 67.6 & - & 69.0 & - & 68.3 &- &- &- &- &- &-\\
 CLDA & \textbf{64.6} & \textbf{70.5} & \textbf{62.7} & \textbf{72.5} & \textbf{63.6} & \textbf{71.5} &\textbf{76.2} &\textbf{78.6} &\textbf{75.1} &76.7 &\textbf{75.6} &\textbf{77.6} \\
\specialrule{.1em}{.05em}{.05em}
\end{tabular}}}
% \vspace{-2.0mm}
\end{center}
\end{table}


\begin{table}[t]
\small
\begin{center}
\begin{tabular}{ccccc}
\shline
\multirow{2}{*}{{Method}} & {CIFAR-10}&{CIFAR-100} \\
& Acc $\uparrow$(Forget $\downarrow$) & Acc $\uparrow$(Forget $\downarrow$) \\ 
\midrule
baseline & 46.4\std{$\pm$1.2}(36.0\std{$\pm$}2.1) & 18.8\std{$\pm$0.8}(18.5\std{$\pm$}0.7) \\
w/o \methodname & 53.1\std{$\pm$1.4}(24.7\std{$\pm$2.0}) & 19.3\std{$\pm$0.7}(15.9\std{$\pm$0.9}) \\
w/o \dataaugname & 52.0\std{$\pm$1.5}(34.6\std{$\pm$2.4}) & 21.5\std{$\pm$0.5}(16.3\std{$\pm$0.8}) \\ 
\hline
w/o $\mathcal{L}^{\mathrm{new}}_{\mathrm{pro}}$ & 54.8\std{$\pm$1.2}(\textbf{22.1}\std{$\pm$3.0}) & 19.6\std{$\pm$0.8}(19.9\std{$\pm$0.7}) \\
w/o $\mathcal{L}^{\mathrm{seen}}_{\mathrm{pro}}$ & 55.7\std{$\pm$1.4}(25.5\std{$\pm$1.5}) & 20.1\std{$\pm$0.4}(16.2\std{$\pm$0.6}) \\ 
$\mathcal{L}^{\mathrm{seen}}_{\mathrm{pro}}$ w/o $\mathcal{C}^\mathrm{new}$ & 56.2\std{$\pm$1.2}(26.4\std{$\pm$2.3}) & 20.8\std{$\pm$0.6}(17.9\std{$\pm$0.7}) \\ 
\hline
{\frameworkName} (\textbf{ours}) & \textbf{57.8}\std{$\pm$1.1}(23.2\std{$\pm$1.3}) & \textbf{22.7}\std{$\pm$0.7}(\textbf{15.0}\std{$\pm$0.8}) \\ 
\shline 
\end{tabular}
\end{center}
\caption{Ablation studies on CIFAR-10 ($M=0.1k$) and CIFAR-100 ($M=0.5k$). 
``baseline'' means $\mathcal{L}_\mathrm{INS}+\mathcal{L}_\mathrm{CE}$.
``$\mathcal{L}^{\mathrm{seen}}_{\mathrm{pro}}$ w/o $\mathcal{C}^\mathrm{new}$'' means $\mathcal{L}^{\mathrm{seen}}_{\mathrm{pro}}$ do not consider new classes in current task.
}
\label{tab:ablation}
\end{table} 
%\section{Conclusion}
We have presented a neural performance rendering system to generate high-quality geometry and photo-realistic textures of human-object interaction activities in novel views using sparse RGB cameras only. 
%
Our layer-wise scene decoupling strategy enables explicit disentanglement of human and object for robust reconstruction and photo-realistic rendering under challenging occlusion caused by interactions. 
%
Specifically, the proposed implicit human-object capture scheme with occlusion-aware human implicit regression and human-aware object tracking enables consistent 4D human-object dynamic geometry reconstruction.
%
Additionally, our layer-wise human-object rendering scheme encodes the occlusion information and human motion priors to provide high-resolution and photo-realistic texture results of interaction activities in the novel views.
%
Extensive experimental results demonstrate the effectiveness of our approach for compelling performance capture and rendering in various challenging scenarios with human-object interactions under the sparse setting.
%
We believe that it is a critical step for dynamic reconstruction under human-object interactions and neural human performance analysis, with many potential applications in VR/AR, entertainment,  human behavior analysis and immersive telepresence.



 
%\newpage
\appendix
\section{Pricing equations}
\subsection{Credit default swap}
\label{CDS_pricing}
A credit default swap (CDS) is a contract designed to exchange credit risk of a Reference Name (RN) between a Protection Buyer (PB) and a Protection Seller (PS). PB makes periodic coupon payments to PS conditional on no default of RN, up to the nearest payment date, in the exchange for receiving from PS the loss given RN's default.

Consider a CDS contract written on the first bank (RN), denote its price $C_1(t, x)$.\footnote{For the CDS contracts written on the second bank, the similar expression could be provided by analogy.} We assume that the coupon is paid continuously and equals to $c$. Then, the value of a standard CDS contract can be given (\cite{BieleckiRutkowski}) by the solution of  (\ref{kolm_1})--(\ref{kolm_2})  with $\chi(t, x) = c$ and terminal condition
\begin{equation*}
	\psi(x) = 
	\begin{cases}
		1 - \min(R_1, \tilde{R}_1(1)), \quad (x_1, x_2) \in D_2, \\
		1 - \min(R_1, \tilde{R}_1(\omega_2)), \quad (x_1, x_2) \in D_{12}, \\		
	\end{cases}
\end{equation*}
where $\omega_2 = \omega_2(x)$ is defined in (\ref{term_cond}) and 
\begin{equation*}
	\tilde{R}_1(\omega_2) = \min \left[1, \frac{A_1(T) +  \omega_2 L_{2 1}(T)}{L_1(T) + \omega_2 L_{12}(T)}\right].
\end{equation*}
Thus, the pricing problem for CDS contract on the first bank is
\begin{equation}
\begin{aligned}
		& \frac{\partial}{\partial t} C_1(t, x) + \mathcal{L} C_1(t, x) = c, \\
		& C_1(t, 0, x_2) = 1 - R_1, \quad C_1(t, \infty, x_2) = -c(T-t), \\
		& C_1(t, x_1, 0) = \Xi(t, x_1) = 
		\begin{cases}
			c_{1,0}(t, x_1), & x_1 \ge \tilde{\mu}_1, \\
			1-R_1, & x_1 < \tilde{\mu}_i,
		\end{cases} \quad C_1(t, x_1, \infty) = c_{1,\infty}(t, x_1),\\
		& C_1(T, x) = \psi(x) = 
	\begin{cases}
		1 - \min(R_1, \tilde{R}_1(1)), \quad (x_1, x_2) \in D_2, \\
		1 - \min(R_1, \tilde{R}_1(\omega_2)), \quad (x_1, x_2) \in D_{12}, \\		
	\end{cases}
\end{aligned}
\end{equation}
where $c_{1,0}(t, x_1)$ is the solution of the following boundary value problem:
\begin{equation}
\begin{aligned}
		& \frac{\partial}{\partial t} c_{1, 0}(t, x_1) + \mathcal{L}_1 c_{1, 0}(t, x_1) = c, \\
		& c_{1, 0}(t, \tilde{\mu}_1^{<}) = 1 - R_1, \quad c_{1, 0}(t, \infty) = -c(T-t), \\
		& c_{1, 0}(T, x_1) = (1 - R_1) \mathbbm{1}_{\{\tilde{\mu}_1^{<} \le x_1 \le \tilde{\mu}_1^{=}\}}, 
\end{aligned}
\end{equation}
and $c_{1,\infty}(t, x_1)$ is the solution of the following boundary value problem
\begin{equation}
\begin{aligned}
		& \frac{\partial}{\partial t} c_{1, \infty}(t, x_1) + \mathcal{L}_1 c_{1, \infty}(t, x_1) = c, \\
		& c_{1, \infty}(t, 0) = 1 - R_1, \quad c_{1, \infty}(t, \infty) = -c(T-t), \\
		& c_{1, \infty}(T, x_1) = (1 - R_1) \mathbbm{1}_{\{x_1 \le \mu_1^{=}\}}.
\end{aligned}
\end{equation}

\subsection{First-to-default swap}
An FTD contract refers to a basket of reference names (RN). Similar to a regular CDS, the Protection Buyer (PB) pays a regular coupon payment $c$ to the Protection Seller (PS) up to the first default of any of the RN in the basket or maturity time $T$. In return, PS compensates PB the loss caused by the first default.

Consider the FTD contract referenced on $2$ banks, and denote its price $F(t, x)$. We assume that the coupon is paid continuously and equals to $c$. Then, the value of FTD contract can be given (\cite{LiptonItkin2015}) by the solution of  (\ref{kolm_1})--(\ref{kolm_2})  with $\chi(t, x) = c$ and terminal condition
\begin{equation*}
	\psi(x) = \beta_0  \mathbbm{1}_{\{x \in D_{12}\}} + \beta_1 \mathbbm{1}_{\{x \in D_{1}\}} + \beta_2 \mathbbm{1}_{\{x \in D_{2}\}},
\end{equation*}
where
\begin{equation*}
	\begin{aligned}
		\beta_0 = 1 - \min[\min(R_1, \tilde{R}_1(\omega_2), \min(R_2, \tilde{R}_2(\omega_1)], \\
		\beta_1 = 1 - \min(R_2, \tilde{R}_2(1)), \quad \beta_2 = 1 - \min(R_1, \tilde{R}_1(1)),
	\end{aligned}
\end{equation*}
and
\begin{equation*}
	\tilde{R}_1(\omega_2) = \min \left[1, \frac{A_1(T) +  \omega_2 L_{2 1}(T)}{L_1(T) + \omega_2 L_{12}(T)}\right], \quad \tilde{R}_2(\omega_1) = \min \left[1, \frac{A_2(T) +  \omega_1 L_{1 2}(T)}{L_2(T) + \omega_1 L_{21}(T)}\right].
\end{equation*}
with $\omega_1 = \omega_1(x)$ and $\omega_2 = \omega_2(x)$ defined in (\ref{term_cond}).

Thus, the pricing problem for a FTD contract is
\begin{equation}
\begin{aligned}
		& \frac{\partial}{\partial t} F(t, x) + \mathcal{L} F(t, x) = c, \\
		& F(t, x_1, 0) = 1 - R_2,  \quad F(t, 0, x_2) = 1 - R_1, \\
		& F(t, x_1, \infty) = f_{2,\infty}(t, x_1), \quad F(t, \infty, x_2) = f_{1,\infty}(t, x_2), \\
		& F(T, x) = \beta_0  \mathbbm{1}_{\{x \in D_{12}\}} + \beta_1 \mathbbm{1}_{\{x \in D_{1}\}} + \beta_2 \mathbbm{1}_{\{x \in D_{2}\}},
\end{aligned}
\end{equation}
where $f_{1,\infty}(t, x_1)$ and $f_{2,\infty}(t, x_2)$ are the solutions of the following boundary value problems
\begin{equation}
\begin{aligned}
		& \frac{\partial}{\partial t} f_{i, \infty}(t, x_i) + \mathcal{L}_i f_{i, \infty}(t, x_i) = c, \\
		& f_{i, \infty}(t, 0) = 1 - R_i, \quad f_{i, \infty}(t, \infty) = -c(T-t), \\
		& f_{1, \infty}(T, x_i) = (1 - R_i) \mathbbm{1}_{\{x_i \le \mu_i^{=}\}}.
\end{aligned}
\end{equation}

\subsection{Credit and Debt Value Adjustments for CDS}

Credit Value Adjustment and Debt Value Adjustment can be considered either unilateral or bilateral. For unilateral counterparty risk, we need to consider only two banks (RN, and PS for CVA and PB for DVA), and a two-dimensional problem can be formulated, while bilateral counterparty risk requires a three-dimensional problem, where Reference Name, Protection Buyer, and Protection Seller are all taken into account. We follow \cite{LiptonSav} for the pricing problem formulation but include jumps and mutual liabilities, which affects the boundary conditions.

\paragraph{Unilateral CVA and DVA}
The Credit Value Adjustment represents the additional price associated with the possibility of a counterparty's default. Then, CVA can be defined as
\begin{equation}
	V^{CVA} = (1- R_{PS}) \mathbb{E}[\mathbbm{1}_{\{\tau^{PS} < \min(T, \tau^{RN}) \}} (V_{\tau^{PS}}^{CDS})^{+} \, | \mathcal{F}_t],
\end{equation}
where $R_{PS}$ is the recovery rate of PS, $\tau^{PS}$ and $\tau^{RN}$ are the default times of PS and RN, and $V_t^{CDS}$ is the price of a CDS without counterparty credit risk.

We associate $x_1$ with the Protection Seller and $x_2$ with the Reference Name, then CVA can be given by the solution of  (\ref{kolm_1})--(\ref{kolm_2})  with $\chi(t, x) = 0$ and $\psi(x) = 0$. Thus,
\begin{equation}
\begin{aligned}
		& \frac{\partial}{\partial t} V^{CVA}+ \mathcal{L} V^{CVA} = 0, \\
		& V^{CVA}(t, 0, x_2) = (1 - R_{PS}) V^{CDS}(t, x_2)^{+}, \quad V^{CVA}(t, x_1, 0) = 0, \\
		& V^{CVA}(T, x_1, x_2) = 0.
\end{aligned}
\end{equation}

Similar, Debt Value Adjustment represents the additional price associated with the default and defined as
\begin{equation}
	V^{DVA} = (1- R_{PB}) \mathbb{E}[\mathbbm{1}_{\{\tau^{PB} < \min(T, \tau^{RN}) \}} (V_{\tau^{PB}}^{CDS})^{-} \, | \mathcal{F}_t],
\end{equation}
where $R_{PB}$ and $\tau^{PB}$ are the recovery rate and default time of the protection buyer.

Here, we associate $x_1$ with the Protection Buyer and $x_2$ with the Reference Name, then, similar to CVA,  DVA can be given by the solution of  (\ref{kolm_1})--(\ref{kolm_2}),
\begin{equation}
\begin{aligned}
		& \frac{\partial}{\partial t} V^{DVA}+ \mathcal{L} V^{DVA} = 0, \\
		& V^{DVA}(t, 0, x_2) = (1 - R_{PB}) V^{CDS}(t, x_2)^{-}, \quad V^{DVA}(t, x_1, 0) = 0, \\
		& V^{DVA}(T, x_1, x_2) = 0.
\end{aligned}
\end{equation}

\paragraph{Bilateral CVA and DVA}

When we defined unilateral CVA and DVA, we assumed that either protection  buyer, or protection seller are risk-free. Here we assume that they are both risky. Then, 
The Credit Value Adjustment represents the additional price associated with the possibility of counterparty's default and defined as
\begin{equation}
	V^{CVA} = (1 - R_{PS}) \mathbb{E}[\mathbbm{1}_{\{\tau^{PS} < \min(\tau^{PB}, \tau^{RN}, T)\}} (V^{CDS}_{\tau^{PS}})^{+} \, | \mathcal{F}_t],
\end{equation} 

Similar, for DVA
\begin{equation}
	V^{DVA} = (1 - R_{PB}) \mathbb{E}[\mathbbm{1}_{\{\tau^{PB} < \min(\tau^{PS}, \tau^{RN}, T)\}} (V^{CDS}_{\tau^{PB}})^{-} \, | \mathcal{F}_t],
\end{equation} 


We associate $x_1$ with protection seller, $x_2$ with protection buyer, and $x_3$ with reference name. Here, we have a three-dimensional process. Applying three-dimensional version of (\ref{kolm_1})--(\ref{kolm_2}) with $\psi(x) = 0, \chi(t, x) = 0$, we get
\begin{equation}
	\label{CVA_pde}
\begin{aligned}
		& \frac{\partial}{\partial t} V^{CVA} + \mathcal{L}_3 V^{CVA} = 0, \\
		& V^{CVA}(t, 0, x_2, x_3) = (1 - R_{PS}) V^{CDS}(t, x_3)^{+}, \\
		& V^{CVA}(t, x_1, 0, x_3 ) = 0, \quad V^{CVA}(t, x_1, x_2, 0)  = 0, \\
		& V^{CVA}(T, x_1, x_2, x_3) = 0,
\end{aligned}
\end{equation}
and
\begin{equation}
\label{DVA_pde}
\begin{aligned}
		& \frac{\partial}{\partial t} V^{DVA} + \mathcal{L}_3 V^{DVA} = 0, \\
		& V^{DVA}(t, 0, x_2, x_3) = (1 - R_{PB}) V^{CDS}(t, x_3)^{-}, \\
		& V^{DVA}(t, x_1, 0, x_3 ) = 0, \quad V^{DVA}(t, x_1, x_2, 0)  = 0, \\
		& V^{DVA}(T, x_1, x_2, x_3) = 0,
\end{aligned}
\end{equation}
where $\mathcal{L}_3 f$ is the three-dimensional infinitesimal generator.


 


\bibliographystyle{chicago}
\bibliography{ref}










\end{document} 