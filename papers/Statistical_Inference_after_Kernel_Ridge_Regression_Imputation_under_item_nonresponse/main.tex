
\documentclass[12pt]{article}

\usepackage{multirow}
\usepackage{color}



\usepackage{comment}

%% Please use the following statements for
%% managing the text and math fonts for your papers:
\usepackage{times}
%\usepackage[cmbold]{mathtime}
\usepackage{bm}

\usepackage{amsmath}
\usepackage{amssymb}
\usepackage{amsfonts}

\usepackage{amssymb}
\usepackage{amsmath,amsthm, mathtools,commath}
\usepackage{graphics, color}

\usepackage{graphicx}
\usepackage{epsfig}
\usepackage{makeidx}
%\usepackage{hangul}
%\makeindex

\hbadness=10000 \tolerance=10000 \hyphenation{en-vi-ron-ment
in-ven-tory e-num-er-ate char-ac-ter-is-tic}



\usepackage[round]{natbib}
%\bibliographystyle{apalike2}
% \bibliographystyle{jmr}


\newcommand{\biblist}{\begin{list}{}
{\listparindent 0.0cm \leftmargin 0.50cm \itemindent -0.50 cm
\labelwidth 0 cm \labelsep 0.50 cm
\usecounter{list}}\clubpenalty4000\widowpenalty4000}
\newcommand{\ebiblist}{\end{list}}

\newcounter{list}


%\usepackage{setspace}
\usepackage{latexsym}
\usepackage{amsmath, amssymb, amsfonts, amsthm, bbm}
\usepackage{graphicx}
\usepackage{mathrsfs}

%\usepackage{hangpar}
\newcommand{\lbl}[1]{\label{#1}{\ensuremath{^{\fbox{\tiny\upshape#1}}}}}
% remove % from next line for final copy
\renewcommand{\lbl}[1]{\label{#1}}

\newtheorem{theorem}{Theorem}
\newtheorem{corollary}{Corollary}
\newtheorem{definition}{Definition}
\newtheorem{example}{Example}
\newtheorem{remark}{Remark}
\newtheorem{result}{Result}
\newtheorem{lemma}{Lemma}

\DeclareMathOperator*{\argmin}{arg\,min}
\newcommand{\ba}{{a}}
\newcommand{\bA}{\mathbf{A}}
\newcommand{\br}{\mathbf{r}}
\newcommand{\bx}{{x}}
\newcommand{\be}{{e}}
\newcommand{\bb}{{b}}
\newcommand{\bd}{{d}}
\newcommand{\by}{{y}}
\newcommand{\bu}{{u}}
\newcommand{\bv}{{v}}
\newcommand{\bt}{\bm{t}}
\newcommand{\bs}{\bm{s}}
\newcommand{\bz}{\bm{z}}
\newcommand{\bh}{\bm{h}}
\newcommand{\bn}{\bm{n}}
\newcommand{\bq}{\bm{q}}
\newcommand{\bmm}{\bm{m}}
\newcommand{\bD}{\mathbf{D}}
\newcommand{\bM}{\mathbf{M}}
\newcommand{\bN}{\mathbf{N}}
\newcommand{\bI}{\mathbf{I}}
\newcommand{\bG}{\mathbf{G}}
\newcommand{\bO}{\mathbf{O}}
\newcommand{\bE}{\mathbf{E}}
\newcommand{\bB}{\mathbf{B}}
\newcommand{\bC}{\mathbf{C}}
\newcommand{\bK}{\mathbf{K}}
\newcommand{\bX}{\mathbf{X}}
\newcommand{\bY}{\mathbf{Y}}
\newcommand{\bU}{\mathbf{U}}
\newcommand{\bV}{\mathbf{V}}
\newcommand{\bW}{\mathbf{W}}
\newcommand{\bZ}{\mathbf{Z}}
\newcommand{\bR}{\mathbf{R}}
\newcommand{\bP}{\mathbf{P}}
\newcommand{\bQ}{\mathbf{Q}}
\newcommand{\bT}{\mathbf{T}}
\newcommand{\bw}{\mathbf{w}}
\newcommand{\bS}{\mathbf{S}}
\newcommand{\bH}{\mathbf{H}}
\newcommand{\biI}{\mathcal{I}}
\newcommand{\biN}{\mathcal{N}}
\newcommand{\biT}{\mathcal{T}}
\newcommand{\bphi}{\boldsymbol{\phi}}
\newcommand{\bpi}{\boldsymbol{\pi}}
\newcommand{\bone}{\mathbf{1}}
\newcommand{\bbeta}{\boldsymbol{\beta}}
\newcommand{\bvar}{\boldsymbol{\varepsilon}}
\newcommand{\bdelta}{{\delta}}
\newcommand{\bepsilon}{{\epsilon}}
\newcommand{\btheta}{\boldsymbol{\theta}}
\newcommand{\btau}{\boldsymbol{\tau}}
\newcommand{\etab}{\boldsymbol{\eta}}
\newcommand{\bgamma}{\boldsymbol{\gamma}}
\newcommand{\blambda}{{\lambda}}
\newcommand{\bpsi}{\boldsymbol{\psi}}
\newcommand{\bmu}{\boldsymbol{\mu}}
\newcommand{\balpha}{\boldsymbol{\alpha}}
\newcommand{\bSigma}{\boldsymbol{\Sigma}}
\newcommand{\bGamma}{\boldsymbol{\Gamma}}
\newcommand{\bOmega}{{\Omega}}
\newcommand{\bDelta}{\boldsymbol{\Delta}}
\newcommand{\bPhi}{\boldsymbol{\Phi}}
\newcommand{\bPi}{\boldsymbol{\Pi}}
\newcommand{\bkappa}{\boldsymbol{\kappa}}
%\newcommand{\btheta}{\boldsymbol{\theta}}
\newcommand{\bzero}{\mathbf{0}}
\newcommand{\var}{\mathrm{var}}
\newcommand{\pr}{\mathrm{pr}}
\newcommand{\logit}{\mathrm{logit\,}}
\newcommand{\Var}{\mathrm{var}}
\newcommand{\Cov}{\mathrm{cov}}
\newcommand{\T}{\mathrm{T}}
\newcommand{\dps}{\displaystyle}
\def\E{E}
\newcommand{\xhi}{X_{hi}}
\newcommand{\yhi}{Y_{hi}}
\newcommand{\vhi}{v_{hi}}
\newcommand{\uhi}{u_{hi}}
\newcommand{\hrange}{{_{h=1}^{H}}}
\newcommand{\irange}{{_{i=1}^{m_h}}}
\newcommand{\widehatyhi}{\widehat{Y}_{hi}}
\newcommand{\hirange}{\hrange{_,}\irange}
\def\upstep#1{{^{(#1)}}}
\def\trans{^{\rm T}}
\usepackage{mathabx}
\def\wh{\widehat}
\def\wt{\widetilde}
\newcommand{\Norm}[1]{\left\Vert#1\right\Vert}
\newcommand{\Abs}[1]{\left\vert#1\right\vert}

\newcommand{\hf}[1]{\textcolor{blue}{\textbf{[Hengfang: #1]}}}



\newcommand{\jk}[1]{\textcolor{red}{\textbf{[JK: #1]}}}



\usepackage{natbib}

\providecommand{\keywords}[1]{\textbf{Key words:} #1}

\begin{document}


\baselineskip .3in


\title{Statistical Inference after Kernel Ridge Regression Imputation under item nonresponse}

\author{Hengfang Wang \and Jae Kwang Kim}

\date{} 

\maketitle

\begin{abstract}
Imputation is a popular technique for handling missing data. 
We consider a nonparametric approach to imputation using the kernel ridge regression technique and {propose consistent variance estimation}. The proposed variance estimator is based on a linearization approach which employs the entropy method to estimate the density ratio.
The $\sqrt{n}$-consistency of the imputation estimator is established when a Sobolev space is utilized in the kernel ridge regression imputation, which enables us to develop the proposed variance estimator.
Synthetic data experiments are presented to confirm our theory. 
\end{abstract}

\keywords{Reproducing kernel Hilbert space;  Missing data; Nonparametric method}



%\bibliographystyle{chicago}
%\bibliography{ref}

\newpage 

 
\section{Introduction}



Missing data is a universal problem in statistics.  Ignoring the cases with missing values  can lead to  misleading results \citep{kim2013statistical, little2019statistical}. To avoid the potential problem with missing data, imputation is commonly used.  
 After imputation, the imputed dataset can serve as a complete dataset that has no missing values, which in turn makes results from different analysis methods consistent. However, treating imputed data as if observed and applying the standard estimation procedure may result in misleading inference, leading to  underestimation of the variance  of  imputed point estimators.  As a result, how to make statistical {inferences} with  {imputed point estimators} is an important statistical problem. 
An overview of imputation method can be found in \citet{haziza2009imputation}. 

Multiple imputation,  proposed by \citet{rubin2004multiple}, addresses the uncertainty associated with imputation.  
However,  variance estimation using Rubin's formula  requires  certain conditions \citep{wang1998large,kim2006bias,yang2016note}, which do not necessarily hold in practice. An alternative method  is fractional imputation, originally proposed by \citet{kalton1984some}. The main idea of fractional imputation is to generate multiple imputed values and the corresponding fractional weights. 
%Hot deck imputation is a popular method of imputation  where the imputed values are taken  from  the observed values. In this vein, \citet{fay1996alternative,kim2004fractional,fuller2005hot,durrant2005imputation,durrant2006using} discussed fractional hot deck imputation.
In particular,
\citet{kim2011parametric} and \citet{kim2014fractional} employ fully parametric approach to handling nonresponse items with fractional imputation. However, such parametric fractional imputation relies heavily on the parametric model assumptions.  To mitigate the effects of parametric model assumption, empirical likelihood \citep{owen2001empirical,qin1994empirical} as a semiparametric approach was considered. In particular,  \citet{wang2009empirical} employed the kernel smoothing approach to do empirical likelihood inference with missing values. %\citet{chen2017semiparametric} extended \citet{muller2009estimating}'s work to develop fractional imputation with the first moment  assumption only.   
\citet{cheng1994nonparametric} utilized the kernel-based nonparametric regression approach to do the imputation and established the $\sqrt{n}$-consistency of the imputed estimator.

Kernel ridge regression \citep{friedman2001elements,shawe2004kernel} is a popular data-driven approach which can alleviate the effect of model assumption. {By using} a regularized \textit{M}-estimator {in reproducing} kernel Hilbert space (RKHS), kernel ridge regression can capture the model with {complex reproducing kernel Hilbert space} while a regularized term makes the original infinite dimensional estimation problem viable \citep{wahba1990spline}. \citet{geer2000empirical,mendelson2002geometric,zhang2005learning,koltchinskii2006local,steinwart2009optimal} studied the error bounds for the estimates of kernel ridge regression method. 
%Recently, \citet{zhang2013divide} employed truncation analysis to estimate the error bound in a distributed fashion. \citet{yang2017randomized} considered randomized sketches for KRR and studied projection dimension which can preserve minimax optimal approximations for KRR.

%\subsection{Our Work}

In this paper, we apply kernel ridge regression as a nonparametric imputation method and propose a consistent variance estimator for the corresponding imputation estimator under missing at random framework. Because the kernel ridge regression is a general tool for nonparametric regression with flexible assumptions, the proposed imputation method is practically useful. Variance estimation after the kernel ridge regression imputation is a challenging but important problem.   
To the best of our knowledge, this is the first paper which considers kernel ridge regression technique and discusses its variance estimation in the imputation framework.  Specifically, we first prove $\sqrt{n}$-consistency of the kernel ridge regression imputation estimator and obtain influence function for linearization. After that, we 
employ the maximum entropy method \citep{nguyen2010} for density ratio estimation to get a valid estimate of the inverse of the propensity scores. The consistency of our variance estimator can then  be established. 


The paper is organized as follows. In Section 2, the basic setup and  the proposed method are introduced. In Section 3, main theory is established. We also introduce a novel  nonparametric estimator of the propensity score function.  Results from two limited simulation studies are presented in Section 4. An illustration of the proposed method to a real data example is presented in Section 5. Some concluding remarks are made in Section 6. 


\section{Proposed Method}

Consider the problem of estimating $\theta=\mathbb{E}(Y)$ from an independent and identically distributed (IID) sample  $\{(\bx_i, y_i), i=1, \cdots, n\}$ of random {vector} $(X,Y)$. Instead of always {observing  $y_i$}, suppose that we observe $y_i$ only if $\delta_i=1$, where $\delta_i$ is the response indicator function of unit $i$ taking values on $\{0,1\}$.  The auxiliary variable $\bx_i$ are always observed. 
We assume that the response mechanism is missing at random (MAR)   in the sense of   \cite{rubin1976}. 
%$$ \delta \perp Y \mid \bx . $$

Under MAR, we can develop a nonparametric estimator $\wh{m} (\bx)$ of $m(\bx)=\mathbb{E}( Y \mid \bx)$ and construct the following imputation estimator: 
\begin{equation} 
\wh{\theta}_I = \frac{1}{n} \sum_{i=1}^n \left\{ \delta_i y_i + (1-\delta_i )\wh{m} (\bx_i) \right\}. 
\label{1} 
\end{equation} 
If $\wh{m} (\bx)$ is constructed by the kernel-based nonparametric regression method, we can express 
\begin{equation} 
 \wh{m} (\bx) = \frac{ \sum_{i=1}^n \delta_i K_h( \bx_i, \bx ) y_i}{ \sum_{i=1}^n \delta_i K_h( \bx_i, \bx )} 
\label{2}
\end{equation} 
where $K_h (\cdot)$ is the kernel function with bandwidth $h$. Under some suitable choice of the bandwidth $h$, \citet{cheng1994nonparametric} first established the $\sqrt{n}$-consistency of the imputation estimator (\ref{1}) with nonparametric  function in (\ref{2}).  However, the kernel-based regression imputation in (\ref{2}) is applicable only when the dimension of $x$ is small. 


%\textcolor{red}{Suppose we can express 
%$$ \hat{\theta}_I = \frac{1}{n} \sum_{i=1}^n \hat{m}(\bx_i) = \frac{1}{n} \sum_{i =1}^n\delta_i  \hat{g} (\bx_i) y_i , $$
%then 
% $\hat{g}(\bx)$ is a nonparametric estimator of $1/\pi(\bx)$. Thus, the linearization form of $\hat{\theta}_I$ is 
% $$ \tilde{\theta}_I = \frac{1}{n} \sum_{i=1}^n \{ \hat{m}_i+ \delta_i \hat{g}( \bx_i) (y_i - \hat{m}_i  ) \}$$
% where $\hat{m}_i = \hat{m} (\bx_i)$. The $\hat{g}( \bx)$ satisfies 
% $$ \sum_{i=1}^n \hat{m}_i = \sum_{i=1}^n\delta_i  \hat{g} (\bx_i) \hat{m}_i . $$
%}
%



In this paper, we extend the work of \citet{cheng1994nonparametric} by considering a more general type of the nonparametric imputation, called kernel ridge regression (KRR) imputation. The KKR technique can be understood using the reproducing kernel Hilbert space (RKHS) theory \citep{aronszajn1950theory} and can be described as 
\begin{equation} 
\wh{m} = \argmin_{m\in \mathcal{H}} \left[   \sum_{i=1}^{n} \delta_{i}\left\{ y_{i} - m(\bx_{i}) \right\}^{2} + \lambda \Norm{m}_{\mathcal{H}}^{2}   \right],
\label{3} 
\end{equation} 
where $ \norm{m}_{\mathcal{H}}^{2}  $ is the norm of $m$ in the Hilbert space $\mathcal{H}$. Here, the inner product $\langle \cdot, \cdot \rangle_{\mathcal{H}}$ is induced by {such} a kernel function, i.e., 
\begin{align}
\langle f, K(\cdot, \bx) \rangle_{\mathcal{H}} = f(\bx), \forall \bx \in \mathcal{X},  f\in \mathcal{H},
\end{align}
namely, the reproducing property of $\mathcal{H}$. Naturally, this  reproducing property implies the $\mathcal{H}$ norm of $f$: $\norm{f}_{\mathcal{H}} = \langle f, f  \rangle_{\mathcal{H}}^{1/2}$.

One 
canonical example of such a space is the Sobolev space. Specifically, assuming that  the domain of such functional space is $[0,1]$,
 the Sobolev space of order $l$ can be denoted as 
\begin{eqnarray}
	\mathcal{W}_{2}^{l} &=& \left\{ f:[0,1] \rightarrow \mathbb{R} |
	 f, f^{(1)}, \dots, f^{(l-1)} \mbox{ are absolute continuous and } f^{(l)} \in L^{2}[0,1]    \right\}. \notag 
\end{eqnarray}
One possible norm for this space can be
\begin{eqnarray}
	\norm{f}_{\mathcal{W}_{2}^{l}}^{2} = \sum_{q = 0}^{l-1}\left\{   \int_{0}^{1}f^{(q)}(t)dt      \right\}^{2} +
	 \int_{0}^{1}\left\{f^{(l)}(t) \right\}^{2}dt . \notag 
\end{eqnarray}
%Readers can refer to \cite{wahba1990spline} for a thorough treatment of the RKHS technique. 
In this section, we employ the Sobolev space of second order as the approximation space. 
For Sobolev space of order $\ell$, we have the kernel function
\begin{align}
K(x,y) = \sum_{q = 0}^{\ell-1}k_{q}(x)k_{q}(y) + k_{\ell}(x)k_{\ell}(y)
 + (-1)^{\ell}   k_{2\ell}(|x-y|),\notag 
\end{align}
where $k_{q}(x) = (q!)^{-1}B_{q}(x)$ and $B_{q}(\cdot)$ is the Bernoulli polynomial of order $q$.




By the representer theorem for RKHS \citep{wahba1990spline}, the estimate  in (\ref{3})  lies in the linear span of $\{K(\cdot, \bx_{i}), i = 1,\ldots, n\}$.
Specifically, we have 
\begin{align}\label{KRR}
\wh{m}(\cdot) = \sum_{i=1}^{n}\wh{\alpha}_{i,\lambda}K(\cdot, \bx_{i}),
\end{align}
where 
\begin{align}
\wh{\balpha}_{\lambda} = \left(\bDelta_{n}\bK +  \lambda\bI_{n}\right)^{-1}\bDelta_n \by,\notag
\end{align}
$\bDelta_{n} = \mbox{diag}(\delta_{1}, \ldots, \delta_{n})$, $\bK = (K(\bx_{i}, \bx_{j}))_{ij} $,  $\by = (y_{1},\ldots, y_{n})\trans$ and $\bI_{n}$ is the $n\times n$ identity matrix.

The tuning parameter $\lambda$ is selected via generalized cross-validation (GCV) in KRR, where the GCV criterion for $\lambda$ is
\begin{align}\label{GCV}
 \mbox{GCV}(\lambda) = \frac{n^{-1}   \Norm{ \left\{\bDelta_{n} - \bA(\lambda)\right\}\by }_{2}^{2}   }{n^{-1}  \mbox{Trace}(\bDelta_{n} - \bA(\lambda) )  },
\end{align}
and $\bA(\lambda) = \bDelta_{n}\bK  ( \bDelta_{n} \bK + \lambda \bI_{n}  )^{-1} \bDelta_{n}  $. The value of $\lambda$ minimizing the GCV is used for the selected tuning parameter. 





Using the KRR imputation in (\ref{3}), we aim to establish the following two goals:
\begin{enumerate}
	\item Find the sufficient conditions for the   $\sqrt{n}$-consistency of the imputation estimator $\wh{\theta}_I$ using (\ref{KRR}) and give a formal proof. 
	\item Find a linearization variance formula for the imputation estimator $\wh{\theta}_I$ using the KRR imputation. 
\end{enumerate}
The first part is formally presented in Theorem 1 in Section 3. For the second part, we employ the density ratio estimation method of \cite{nguyen2010} 
to get a consistent estimator of  $\omega (\bx) = \{\pi (\bx)\}^{-1}$ in the linearized version of $\hat{\theta}_I$. 
\begin{comment}
By Theorem 1, we use the following estimator to estimate the variance of $\wh{\theta}_I$ in (\ref{2}): 
\begin{align}\label{variance estimation}
\wh{\mbox{V}} (\wh{\theta}_{I}) = \frac{1}{n(n-1)}\sum_{i=1}^{n}\left( \hat{\eta}_i - \bar{\eta} \right)^2 
\end{align}
where 
$$  \hat{\eta}_i =  \wh{m}(\bx_{i}) + \delta_{i} \wh{\omega}_{i} \left\{ y_{i} - \wh{m}(\bx_{i})\right\},
$$
and $ \wh{\omega}_{i} $ is a consistent estimator of $\omega (\bx) = \{\pi (\bx)\}^{-1}$. 
\end{comment} 



\section{Main Theory}


Before we develop our main theory, we first introduce Mercer's theorem. 
\begin{lemma}[Mercer's theorem]\label{Mercer}
Given a continuous, symmetric, positive definite kernel function $K: \mathcal{X} \times \mathcal{X} \mapsto \mathbb{R}$. For $\bx, \bz \in \mathcal{X}$, under some regularity conditions, Mercer's theorem characterizes $K$ by the following expansion
\begin{align}
 K(\bx, \bz) = \sum_{j=1}^{\infty}\lambda_{j}\phi_{j}(\bx) \phi_{j}(\bz),\notag
\end{align}
where $\lambda_{1} \geq \lambda_{2} \geq \ldots \geq 0$ are a non-negative sequence of eigenvalues and $\{\phi_{j} \}_{j=1}^{\infty}$ is an orthonormal basis for $L^{2}(\mathbb{P})$.
\end{lemma}

To develop our theory, we  make the following  assumptions.
 \begin{description}
 
 \item {[A1]}
 \label{A1}
 	For some $k \geq 2$, there is a constant $\rho < \infty$ such that $E[ \phi_{j}(X)^{2k} ] \leq \rho^{2k}$ for all 
 	$j \in \mathbb{N}$, where $\{\phi_{j}\}_{j=1}^{\infty}$ are orthonormal basis by expansion from Mercer's theorem.

 \item {[A2]}
 \label{A2}
 	The function $m \in \mathcal{H}$, and for $\bx \in \mathcal{X}$, we have $E[\left\{ Y -  m(\bx)\right\}^{2} ] \leq \sigma^{2}$, {for some  $\sigma^{2} < \infty$.}

\item {[A3]}
\label{A3}
     The propensity score $\pi(\cdot)$ is uniformly bounded away from zero. In particular, there exists a positive constant $c > 0$ such that 
     	$\pi(\bx_{i}) \geq c$, for $i = 1, \ldots, n$.

\item {[A4]}
 \label{A4}
     The ratio $d/\ell < 2$ for $d$-dimensional Sobolev space of order $\ell$, where $d$ is the dimension of covariate $\bx$.
\end{description}


The first assumption is a technical assumption which controls the tail behavior of $\{\phi_{j}\}_{j=1}^{\infty}$. Assumption 2 indicates that the noises have bounded variance. Assumption 1 and Assumption 2 together aim to control the error bound of the kernel ridge regression estimate $\wh{m}$. {Furthermore}, Assumption 3 means that the support for the respondents should be the same as  the original sample support. Assumption 3 is a standard assumption for missing data analysis. Assumption 4 is a technical assumption for entropy analysis. Intuitively, when the dimension is large, the Sobolev space should be large enough to capture the true model. 


\begin{theorem}\label{main theorem}
Suppose Assumption $1 \sim 4$ hold for a Sobolev kernel of order $\ell$, $\lambda \asymp   n^{1-\ell}$, we have
\begin{align}\label{rate}
 \sqrt{n}(\wh{\theta}_{I} - \wt{\theta}_{I} ) = o_{p}(1), 
\end{align}
where 
\begin{align}\label{tilde_theta}
\wt{\theta}_{I} &= \frac{1}{n}\sum_{i=1}^{n} \left[ m(\bx_{i}) + \delta_{i} \frac{1}{\pi(\bx_{i})}  \left\{ y_{i} - m(\bx_{i})\right\} 
   \right] 
\end{align}
and 
\textcolor{blue}{}
$$\sqrt{n} \left( \tilde{\theta}_I - \theta \right) \stackrel{\mathcal{L}}{\longrightarrow}  N(0, \sigma^2 ) ,
$$
 with 
$$ \sigma^2 = V\{ E( Y \mid \bx) \} + E\{ V( Y \mid \bx)/\pi( \bx)   \} . $$ 

%Let $\omega_{i}^{\star} = \pi(\bx_{i})^{-1}$. In particular,  $\pmb{\omega}^{\star} = \{\omega_{i}^{\star}: \delta_{i} = 1\}$  can be estimated by\cite{wong2018kernel}.
\end{theorem}


Theorem \ref{main theorem} guarantees the asymptotic equivalence of $\wh{\theta}_{I}$ and $\wt{\theta}_{I}$ in \eqref{tilde_theta}. Specifically, the reference distribution is a combination of an outcome model and a propensity score model for sampling mechanism. The variance of $\wt{\theta}_I$ achieves the semiparametric lower bound of \citet{robins94}. 
%{Additionally, \eqref{tilde_theta} suggests a linearization form of variance estimation of $\wh{\theta}_{I}$. To estimate $\omega_{i}^{\star} = \pi(\bx_{i})^{-1}$, we can use employ the covariate balancing idea of \citet{wong2018kernel} to obtain $\hat{\pmb{\omega}}$ as provided in \eqref{WC_Opt}. }  
The proof of Theorem \ref{main theorem} is presented in the Appendix.
%\hf{Here, in some sense, I think we do not have a valid variance estimator for \citet{wong2018kernel}. In particular, they have uncertainty of $\wh{\omega}_{i}$. In our case, we only consider the variance estimator for $\wh{\theta}_{I}$, which only involves $\wh{m}$.}




The linearization formula in (\ref{tilde_theta}) can be used for  variance estimation. The idea is to estimate the influence function 
$\eta_i = m(\bx_{i}) + \delta_{i} \{ \pi(\bx_{i})\}^{-1}  \left\{ y_{i} - m(\bx_{i}) \right\} $ and apply the standard variance estimator using $\hat{\eta}_i$. To estimate $\eta_i$, we need an estimator of $\pi(x)$. We propose a version of KRR method to estimate $\omega(x) = \{ \pi(x) \}^{-1}$ directly.  
In order to estimate $\omega(x) = \{ \pi(x) \}^{-1}$, we wish to develop a {KRR} version of estimating $\omega (x)$. To do this, first define 
\begin{equation} 
 g( x) = \frac{ f(x \mid  \delta =0 ) }{ f( x \mid  \delta = 1 ) },
\label{dr2}
\end{equation} 
and, by Bayes theorem, we have  
$$ \omega(x)= \frac{1}{ \pi(x) }  = 1+ \frac{n_0}{n_1}  g(x).
$$
Thus, to estimate $\omega(x)$, we have only to estimate the density ration function $g(x)$ in (\ref{dr2}). 
Now, to estimate $g(x)$ nonparametrically, we use the idea of \cite{nguyen2010} for  the KRR approach to density ratio estimation. 

%in order to  estimate $g( x)$ and obtain a nonparametric estimator of $\omega(x)$. 

%\jk{Some details on the estimation of $g(x)$ should be placed here } 


To explain the KRR estimation of $g(x)$, note that $g(x)$ can be understood as the  maximizer of 
 \begin{align} 
     Q (g) = \int \log \left( g \right) f( \bx \mid \delta =0)  d \mu(\bx)   -   \int g (x) f( \bx \mid \delta = 1)   d \mu(x)    \label{qq} 
    \end{align} 
    with constraint 
    $$  \int g (x) f( \bx \mid \delta = 1)   d \mu(x) = 1 . 
    $$ 
    The sample version objective function is 
 \begin{equation} 
\hat{Q}( g) =  \frac{1}{n_0} \sum_{i=1}^n \mathbb{I} ( \delta_i=0) \log \{ g(\bx_i)  \} - \frac{1}{n_1} \sum_{i=1}^n \mathbb{I} ( \delta_i=1) g (\bx_i) 
\label{qq2} 
\end{equation}  
where $n_k = \sum_{i=1}^n \mathbb{I} (\delta_i = k ) $. The maximizer of $\hat{Q} (g)$ is an M-estimator of the density ratio function $g$. 



%Define 
%$h (\bx)= \log \{ g(\bx)\}$. The loss function  $L ( \cdot)$  derived from the optimization problem in (\ref{qq2}) can be written as 
%\begin{equation*} 
% L( \delta, h(\bx)  ) = \frac{1}{n_0}  \mathbb{I} (\delta=0) h(\bx) - \frac{1}{n_1} \mathbb{I} (\delta=1) \exp \{ h(\bx) \}. 
%\end{equation*} 



Further, define 
$h (\bx)= \log \{ g(\bx)\}$.
 The loss function 
 $L ( \cdot)$  derived from the optimization problem in (\ref{qq2}) can be written as 
\begin{equation*} 
 L( \delta, h(\bx)  ) = \frac{1}{n_1} \mathbb{I} (\delta=1) \exp \{ h(\bx) \} - \frac{1}{n_0}  \mathbb{I} (\delta=0) h(\bx).
\end{equation*} 
%such that  $$ \hat{Q} ( r_k) = \sum_{i=1}^N L( y_i, h_k (x_i) ) . $$



In our  problem, we wish to find $h$ that minimizes 
\begin{equation}
\sum_{i=1}^n L( \delta_i, \alpha_{0} + h(x_i) ) + \tau \left\| h \right\|_{\mathcal{H} }^{2} \label{18} 
\end{equation} 
over $\alpha_0 \in \mathbb{R}$ and $h \in \mathcal{H} $, where $L ( \cdot) $ is the loss function  derived from the optimization problem in (\ref{qq}) using maximum entropy. 
%That is, 



Hence, using the representer theorem again, the solution to (\ref{18}) can be obtained as 
\begin{equation} \label{entropy_method}
\min_{ \alpha \in \mathbb{R}^n  } \left\{ \sum_{i=1}^n L( \delta_i, \alpha_0 + \sum_{j=1}^n   \alpha_j K( x_i, x_j )) + \tau \balpha' \mathbf{K}  \balpha  \right\} 
\end{equation}
and $\alpha_0$ is a normalizing constant satisfying 
$$ n_1 = \sum_{i=1}^n  \mathbb{I} (\delta_i = 1) \exp \{ \alpha_0 + \sum_{j=1}^n  \hat{\alpha}_j K( x_i, x_j ) \} . $$
 Thus, we use 
\begin{equation}
 \hat{g} (x) = \exp \{ \hat{\alpha}_0 + \sum_{j=1}^n  \hat{\alpha}_j K( x, x_j)  \}
 \label{final} 
 \end{equation} 
 as a nonparametric approximation of the density ratio function $g( x)$. Also, 
 \begin{equation} 
 \hat{\omega} (x)= 1+ \frac{n_0}{n_1} \hat{g} (x)
 \label{final2}
 \end{equation} 
 is the nonparametric approximation of $\omega(x) = \{ \pi(x) \}^{-1}$. Note that $\tau$ is the tuning parameter that determines the model complexity of $g(x)$. The tuning parameter selection is discussed in Appendix B.  
 
 
 
 
Therefore, we can use 
\begin{equation} 
\wh{\mbox{V}}  = \frac{1}{n} \frac{1}{n-1} \sum_{i=1}^n \left( \hat{\eta}_i - \bar{\eta}_n \right)^2  
\label{varh}
\end{equation} 
as a variance estimator of $\hat{\theta}_I$, where 
\begin{equation} 
\hat{\eta}_i = \wh{m}(\bx_{i}) + \delta_{i} \wh{\omega}_{i} (x_i) \left\{ y_{i} - \wh{m}(\bx_{i})\right\}
\end{equation} 
and $\bar{\eta}_n = n^{-1} \sum_{i=1}^n \hat{\eta}_i$. 






\section{Simulation Study}

%\jk{We need to update our simulation result because we use a different function of $\hat{\omega}(x)$ for variance estimation. } 

%\hf{Simulation results updated.}
\subsection{Simulation study one} 

To evaluate the performance of the proposed imputation method and its variance estimator, we conduct two simulation studies. In the first simulation study, a continuous study variable is considered with three different data generating models.   In the three models, we keep the response rate around $70\%$ and $\mbox{Var}(Y) \approx 10$. Also, $\bx_{i} =  (x_{i1}, x_{i2}, x_{i3}, x_{i4})\trans$ are generated independently {element-wise} from the {uniform} distribution on the support $(1,3)$. In the first model (Model A), we use a linear regression model 
%The responses for linear case (Model A) are generated by 
\begin{align}
y_{i} =& 3 + 2.5x_{i1} + 2.75 x_{i2} +  2.5 x_{i3} + 2.25 x_{i4} + \sigma\epsilon_{i},\notag
\end{align}
to obtain $y_i$, 
where $\{\epsilon_{i}\}_{i=1}^{n}$ are generated from standard normal distribution and $\sigma = \sqrt{3}$. In the second model  (Model B), we use  \begin{align}
y_{i} =& 3 + (1/35)x_{i1}^{2}x_{i2}^{3}x_{i3} +  0.1x_{i4} + \sigma\epsilon_{i} \notag
\end{align}
to generate data with a {nonlinear} structure. The third model  (Model C) for generating the study variable is 
\begin{align}
y_{i} =& 3 + (1/180)x_{i1}^{2}x_{i2}^{3}x_{i3}x_{i4}^{2} + \sigma\epsilon_{i}.\notag
\end{align}


In addition to $\{(x_i\trans, y_i)\trans, i = 1, \ldots, n\}$, the response indicator variable $\delta$'s are independently generated from the {Bernoulli} distribution with probability $\logit(\bx_i' \beta + 2.5)$, where  ${\beta} = (-1, 0.5, -0.25, -0.1)\trans$ and $\mbox{logit}(p) = \log\{p / (1-p)\}$. We considered three sample sizes $n = 200$, $n = 500$ and $n = 1,000$ with 1,000 Monte Carlo replications. The reproducing kernel Hilbert space we employed is the second-order Sobolev space. 

We also compare three imputation methods: kernel ridge regression (KRR), B-spline, linear regression (Linear). We compute the Monte Carlo biases, variance, and the mean squared {errors} of the imputation estimators for each case. 
The corresponding results are presented in   Table \ref{MAR, linear, comparison}.  
%In Table \ref{MAR, linear, comparison}, the performance of the three imputation estimators are presented.   


\begin{table}[!ht]
\centering
\caption{Biases, Variances and Mean Squared Errors (MSEs) of three imputation estimators for continuous responses}\label{MAR, linear, comparison}
{
\begin{tabular}{cccccc}
  \hline
Model & Sample Size & Criteria & KRR & B-spline & Linear  \\ 
  \hline
 \multirow{9}{*}{A}& \multirow{3}{*}{$200$} & Bias & -0.0577 & 0.0027 & 0.0023 \\ 
 & & Var & 0.0724 & 0.0679 & 0.0682 \\ 
 & & MSE & 0.0757 & 0.0679 & 0.0682 \\ 
 \cline{3-6}
 &\multirow{3}{*}{$500$} & Bias & -0.0358 & 0.0038 & 0.0038 \\ 
 & & Var & 0.0275 & 0.0263 & 0.0263 \\ 
 & & MSE & 0.0288 & 0.0263 & 0.0263 \\
 \cline{3-6} 
 &\multirow{3}{*}{$1000$} & Bias & -0.0292 & 0.0002 & 0.0002 \\ 
 & & Var & 0.0132 & 0.0128 & 0.0129 \\ 
 & & MSE & 0.0141 & 0.0128 & 0.0129 \\
   \hline
\multirow{9}{*}{B} &  \multirow{3}{*}{$200$} & Bias & -0.0188 & 0.0493 & 0.0372 \\ 
 & & Var & 0.0644 & 0.0674 & 0.0666 \\ 
 & & MSE & 0.0648 & 0.0698 & 0.0680 \\ 
  \cline{3-6} 
 &\multirow{3}{*}{$500$} & Bias & -0.0136 & 0.0463 & 0.0356 \\ 
 & & Var & 0.0261 & 0.0275 & 0.0272 \\ 
 & & MSE & 0.0263 & 0.0296 & 0.0285 \\ 
  \cline{3-6} 
 &\multirow{3}{*}{$1000$} & Bias & -0.0122 & 0.0426 & 0.0313 \\ 
 & & Var & 0.0121 & 0.0129 & 0.0129 \\ 
 & & MSE & 0.0123 & 0.0147 & 0.0139 \\ 
\hline
  \multirow{9}{*}{C} &  \multirow{3}{*}{$200$}& Bias & -0.0223 & 0.0384 & 0.0283 \\ 
 & & Var & 0.0748 & 0.0811 & 0.0792 \\ 
 & & MSE & 0.0753 & 0.0825 & 0.0800 \\ 
 \cline{3-6}
 &\multirow{3}{*}{$500$} & Bias & -0.0141 & 0.0369 & 0.0287 \\ 
 & & Var & 0.0281 & 0.0307 & 0.0301 \\ 
 & & MSE & 0.0283 & 0.0320 & 0.0309 \\ 
 \cline{3-6}
 &\multirow{3}{*}{$1000$} & Bias & -0.0142 & 0.0310 & 0.0221 \\ 
 & & Var & 0.0124 & 0.0138 & 0.0136 \\ 
 & & MSE & 0.0126 & 0.0148 & 0.0141 \\ 
 \hline
\end{tabular}
}
\end{table}


The simulation results in  Table \ref{MAR, linear, comparison} shows that  the three methods show similar results under the linear model (Model A), but   kernel ridge regression imputation shows the best performance in terms of {the mean square errors} under the nonlinear models (Models B and C). Linear regression imputation still provides unbiased estimates, because the residual terms in the linear regression model are  approximately unbiased to zero. However, use of linear regression model for imputation leads to efficiency loss because it is not the best model. 



In addition, we have computed the proposed variance estimator under kernel ridge regression imputation. 
%The behavior of variance estimation for the imputation estimator is presented in Table \ref{MAR, linear, KRR}. 
In Table \ref{MAR, linear, KRR}, the relative biases of the proposed variance estimator  and the coverage rates of two interval estimators under $90\%$ and $95\%$ nominal coverage rates are presented. 
The relative bias of the variance estimator decreases as the sample size increases, which confirms  the validity of the proposed variance estimator. Furthermore, the interval estimators show good performances in terms of the coverage rates. 

%calculated by the variance estimation based on our method are well approximated.





\begin{table}[!ht]
\centering
\caption{Relative biases (R.B.)  of the proposed variance estimator, coverage rates (C.R.) of the $90\%$ and $95\%$ confidence intervals for imputed estimators under kernel ridge regression imputation for continuous responses}\label{MAR, linear, KRR}
\begin{tabular}{ccccc}
  \hline
\multirow{2}{*}{Model} & \multirow{2}{*}{Criteria}  & \multicolumn{3}{c}{Sample Size} \\
  \cline{3-5} 
    &  &   200 & 500 & 1000 \\ 
  \hline
 \multirow{3}{*}{A} & R.B. & -0.1050 & -0.0643 & -0.0315 \\ 
 &  C.R. (90\%) & 87.5\% & 89.6\% & 89.9\% \\ 
 &  C.R. (95\%)  & 94.0\% & 94.7\% & 94.9\% \\ 
 \hline
\multirow{3}{*}{B} & R.B. &-0.1016 & -0.1086 & -0.0276 \\ 
 & C.R. (90\%) & 87.6\% & 87.0\% & 89.2\% \\ 
 & C.R. (95\%)  & 92.6\% & 93.3\% & 94.8\% \\ 
   \hline
 \multirow{3}{*}{C} &  R.B. & -0.1934 & -0.1310 & -0.0054 \\
 &  C.R. (90\%)  & 85.0\% & 86.2\% & 90.4\% \\ 
 &  C.R. (95\%)  & 91.4\% & 93.4\% & 94.6\% \\ 
 \hline
\end{tabular}
\end{table}


\subsection{Simulation study two} 

The second simulation study is similar to the first simulation study except that the study variable $Y$ is binary. We use the same simulation setup for generating $\bx_i = (x_{1i}, x_{2i}, x_{3i}, x_{4i})$ and $\delta_i$ as the first simulation study. We consider three models for generating $Y$ 
\begin{equation}
y_{i} \sim \mbox{Bernoulli}(p_{i}), \label{model2} 
\end{equation}
where $p_{i}$ is chosen differently  for each model. 
For model D, we have
\begin{align}
\mbox{logit}(p_{i}) =  0.5 + (1/35)x_{i1}^{2}x_{i2}^{3}x_{i3} +  0.1x_{i4}.\notag
\end{align}
The responses for Model E are generated by (\ref{model2}) with 
\begin{align}
\mbox{logit}(p_{i}) =  0.5 + (1/180)x_{i1}^{2}x_{i2}^{3}x_{i3}x_{i4}^{2}.\notag
\end{align}
The responses for Model F are generated by (\ref{model2}) with 
\begin{align}
\mbox{logit}(p_{i}) =  0.5 + 0.15x_{i1}x_{i2}x_{i3}^{2} +  0.4x_{i2}x_{i3}.\notag
\end{align}

For each model, we consider three imputation estimators: kernel ridge regression (KRR), B-spline, linear regression (Linear). We compute the Monte Carlo biases, variance, and the mean squared errors of the imputation estimators for each case.  {The comparison of the simulation 
results for different estimators  are} presented in Table \ref{MAR, nonlinear, comparison, DEF}. In addition, the relative biases and the coverage rates of the interval estimators  are presented in Table \ref{MAR, binary, KRR}. The simulation results in Table \ref{MAR, binary, KRR} show that the relative biases of the  variance estimators are negligible  and the coverage rates of the interval estimators are close to the nominal levels. 





\begin{table}[!ht]
\centering
\caption{Biases, Variances and Mean Squared Errors (MSEs) of three imputation estimators for binary responses}\label{MAR, nonlinear, comparison, DEF}
{
\begin{tabular}{cccccc}
  \hline

Model & Sample Size & Criterion & KRR & B-spline & Linear  \\ 
  \hline
 \multirow{9}{*}{D} &  \multirow{3}{*}{$200$}& Bias & 0.00028 & 0.00007 & 0.00009 \\ 
 & & Var & 0.00199 & 0.00208 & 0.00206 \\ 
 & & MSE & 0.00199 & 0.00208 & 0.00206 \\ 
 \cline{3-6}
 &\multirow{3}{*}{$500$} & Bias & -0.00019 & -0.00014 & -0.00019 \\ 
 & & Var & 0.00080 & 0.00081 & 0.00081 \\ 
 & & MSE & 0.00080 & 0.00081 & 0.00081 \\ 
 \cline{3-6}
 &\multirow{3}{*}{$1000$} & Bias & -0.00006 & -0.00010 & -0.00010 \\ 
 & & Var & 0.00042 & 0.00042 & 0.00042 \\ 
  \hline
 \multirow{9}{*}{E} &  \multirow{3}{*}{$200$}& Bias & 0.00027 & -0.00001 & -0.00003 \\ 
 & & Var & 0.00195 & 0.00204 & 0.00202 \\ 
 & & MSE & 0.00195 & 0.00204 & 0.00202 \\ 
 \cline{3-6}
 &\multirow{3}{*}{$500$}  & Bias & -0.00039 & -0.00042 & -0.00044 \\ 
 & & Var & 0.00079 & 0.00080 & 0.00080 \\ 
 & & MSE & 0.00079 & 0.00080 & 0.00080 \\ 
  \cline{3-6}
 &\multirow{3}{*}{$1000$}  & Bias & -0.00005 & -0.00013 & -0.00010 \\ 
 & & Var & 0.00042 & 0.00043 & 0.00043 \\ 
 & & MSE & 0.00042 & 0.00043 & 0.00043 \\ 
   \hline
 \multirow{9}{*}{F} &  \multirow{3}{*}{$200$}& Bias & 0.00077 & 0.00102 & 0.00100 \\ 
 & & Var & 0.00199 & 0.00208 & 0.00206 \\ 
 & & MSE & 0.00199 & 0.00208 & 0.00206 \\ 
  \cline{3-6}
 &\multirow{3}{*}{$500$}  & Bias & -0.00002 & 0.00054 & 0.00047 \\ 
 & & Var & 0.00079 & 0.00080 & 0.00080 \\ 
 & & MSE & 0.00079 & 0.00080 & 0.00080 \\ 
  \cline{3-6}
 &\multirow{3}{*}{$1000$}  & Bias & 0.00007 & 0.00055 & 0.00060 \\ 
 & & Var & 0.00042 & 0.00043 & 0.00043 \\ 
 & & MSE & 0.00042 & 0.00043 & 0.00043 \\ 
\hline
\end{tabular}
}
\end{table}



\begin{table}[!ht]
\centering
\caption{
Relative biases (R.B.)  of the proposed variance estimator, coverage rates (C.R.) of the $90\%$ and $95\%$ confidence intervals for imputed estimators under kernel ridge regression imputation 
for binary responses}\label{MAR, binary, KRR}
\begin{tabular}{ccccc}
  \hline
\multirow{2}{*}{Model} & \multirow{2}{*}{Criteria}  & \multicolumn{3}{c}{Sample Size} \\
  \cline{3-5} 
    &  &   200 & 500 & 1000 \\ 
  \hline
 \multirow{3}{*}{D} & R.B. & -0.0061 & 0.0068 & -0.0392 \\ 
 & C.R. (90\%)  & 88.6\% & 90.2\% & 90.4\% \\ 
 &  C.R. (95\%)  & 94.6\% & 94.1\% & 94.3\% \\ 
 \hline
\multirow{3}{*}{E} & R.B.  & 0.0165 & 0.0222 & -0.0487 \\ 
 &  C.R. (90\%)  & 89.2\% & 89.9\% & 89.6\% \\ 
 & C.R. (95\%)   & 94.6\% & 94.7\% & 93.9\% \\  
  \hline
 \multirow{3}{*}{F} &   R.B. & -0.0062 & 0.0187 & -0.0437 \\ 
 & C.R. (90\%)   & 89.9\% & 89.7\% & 89.9\% \\ 
 & C.R. (95\%)   & 94.7\% & 94.8\% & 94.3\% \\ 
 \hline
\end{tabular}
\end{table}


%{We also compute the confidence intervals  using the asymptotic normality of the kernel ridge regression imputed estimator. The proposed variance estimator is used in computing the confidence intervals. Table 3 shows the coverage rates of the confidence intervals. The realized coverage probabilities are close to the nominal coverage probabilities, confirming the validity of the proposed interval estimator. }


\section{Application} 


\begin{comment}
We applied the KRR with kernels of second-order Sobolev space and Gaussian kernel to study the $\mbox{NO}_{2}$ measured in a city in Italy (\textcolor{red}{citation?}). Hourly weather conditions: temperature, absolute humidity, relative humidity are available for the whole year. Meanwhile, the averaged sensor response is subject to the missingness. We are interested in whether there are significant difference of $\mbox{NO}_{2}$ among seasons. We take March to May as spring, June to August as summer, September to November as fall, December to Februrary in the following year as winter. The corresponding missing rates for each season are presentaed in Table \ref{Missing_Table}. The estimates and 95\% confidence intervals for each season are presented in the Figure \ref{CI}. As a benchmark, the confidence interval computed from complete cases are also presented there. As we can see, the behaviors of KRR by two kernels are similar, except that KRR with Sobolev space seems have larger confidence interval. This is because second order Sobolev space is more complex than the RKHS induced by Gaussian kernel.  For the data application, we can see that 
the pollution in fall and winter are significantly more severe than those in spring and summer, where winter has highest $\mbox{NO}_{2}$ concentration. 


\begin{table}[!ht]
\centering
\caption{Missing Rates for Each Season}\label{Missing_Table}
\begin{tabular}{ccccc}
  \hline
Season & Spring & Summer & Fall & Winter \\ 
\hline
  Missing Rate & 20.36\% & 15.76\% & 31.82\% &  8.61\% \\ 
   \hline
\end{tabular}
\end{table}




\begin{figure}[!ht]
    \centering  \label{CI}
            \caption{Estimated seasonally  mean NO2 concentration in 2004  with 95\% confidence interval.}
    \label{fig:mesh1}
    \includegraphics[width=1\textwidth]{NO2_CI.png}
\end{figure}
\end{comment}


We applied the KRR with kernels of second-order Sobolev space and Gaussian kernel to study the $\mbox{PM}_{2.5}(\mu g/m^{3})$ concentration measured in Beijing, China \citep{liang2015assessing}. Hourly weather conditions: temperature, air pressure, cumulative wind speed, cumulative hours of snow and cumulative hours of rain are available from 2011 to 2015. Meanwhile, the averaged sensor response is subject to missingness. In December 2012, the missing rate of $\mbox{PM}_{2.5}$ is relatively high with missing rate $17.47\%$. We are interested in estimating the mean $\mbox{PM}_{2.5}$ in December with imputed KRR estimates. The point estimates and their 95\% confidence intervals are presented in the Table \ref{CI_Table}. The corresponding results are presented in the Figure \ref{CI_Fig}.  As a benchmark, the confidence interval computed from complete cases (Complete in Table \ref{CI_Table}) and confidence intervals for the imputed estimator under linear model (Linear) \citep{kim2009unified}  are also presented there. 

\begin{table}[!ht]
\centering
\caption{Imputed estimates (I.E.), standard error (S.E.) and $95\%$ confidence intervals (C.I.) for imputed mean $\mbox{PM}_{2.5}$ in December, 2012 under kernel ridge regression}\label{CI_Table}
\begin{tabular}{cccc}
  \hline
 Estimator & I.E. & S.E. & $95\%$ C.I. \\ 
  \hline
  Complete & 109.20 & 3.91 & (101.53, 116.87) \\ 
  Linear & 99.61 & 3.68 & (92.39, 106.83) \\ 
  Sobolev & 102.25 & 3.50 & (95.39, 109.12) \\ 
  Gaussian & 101.30 & 3.53 & (94.37, 108.22) \\ 
   \hline
\end{tabular}
\end{table}

\begin{figure}[!ht]
    \centering  \label{CI_Fig}
            \caption{Estimated mean $\mbox{PM}_{2.5}$ concentration in December 2012 with 95\% confidence interval.}
    \label{fig:mesh1}
    \includegraphics[width=1\textwidth]{PM_CI.png}
\end{figure}

As we can see, the performances of {KRR imputation estimators are similar} and created narrower $95\%$ confidence intervals. Furthermore,  the imputed $\mbox{PM}_{2.5}$ concentration during the missing period 
is relatively lower than  the fully observed weather conditions on average.  Therefore, if we only utilize the complete cases to estimate the  mean of $\mbox{PM}_{2.5}$, the severeness of air pollution would be over-estimated.





\section{Discussion}
 We consider kernel ridge regression  as a tool for nonparametric imputation and establish its asymptotic properties. In addition, we propose a linearized approach for variance estimation of the imputed estimator. For variance estimation, we also propose a novel approach of the maximum entropy method for  propensity score estimation.   The proposed Kernel ridge regression imputation can be used as a general tool for nonparametric imputation. By choosing different kernel functions, different  nonparametric imputation methods can be developed. The unified theory developed in this paper can cover various type of the kernel ridge regression imputation and enables us to make valid statistical inferences about the population means. 
 
 
% Numerical studies confirm our theoretical results.

There are several possible extensions of the research. First, the theory can be directly applicable to other nonparametric imputation methods, such as smoothing splines \citep{claeskens2009}. Second, instead of using ridge-type penalty term, one can also consider other penalty functions such as SCAD penalty \citep{FL2001} or adaptive Lasso \citep{zou2006}. Also, the maximum entropy method for propensity score estimation should be investigated more rigorously. Such extensions will be future research topics. 

\appendix

\section*{Appendix} 
\subsection*{A. 
Proof of Theorem 1 }

\renewcommand{\theequation}{A.\arabic{equation}}
\setcounter{equation}{0} 




Before we prove the main theorem, we first introduce the following lemma. 

\begin{lemma}[modified Lemma 7 in \citet{zhang2013divide}]\label{order}
Suppose Assumption [A1] and [A2] hold, for a random vector $\bz = \mathbb{E}(\bz) + \sigma \bvar$, let $\wt{\lambda} = \lambda/n$ we have
\begin{align}
 \bS_{\lambda}\bz = \mathbb{E}(\bz \mid \bx) + \mathcal{O}_{p}\left( \wt{\lambda} + \sqrt{\frac{ \gamma(\wt{\lambda})  }{n} } \right )\bone_{n},\notag
\end{align}
as long as $\mathbb{E}(\norm{z_{i}}_{\mathcal{H}})$ and $\sigma^{2}$ is bounded from above, for $i=1, \ldots, n$,  where $\bvar$ are noise vector with mean zero and bounded variance and 
 \begin{equation}\label{effective_dimension}
 \gamma(\wt{\lambda}) := \sum_{j=1}^{\infty} \frac{1}{1+\wt{\lambda}/\mu_{j}},\notag
 \end{equation}
is the effective dimension and $\{\mu_{j}\}_{j=1}^{\infty}$ are the eigenvalues of kernel $K$ used in $\hat{m}(\bx)$. 
%See Lemma \ref{Mercer} for the definition of the eigenvalues of kernel $K$
\end{lemma}
%\textcolor{red}{What is $\gamma(\wt{\lambda}) $? }


Now, to prove our main theorem, we write 
\begin{align}
\wh{\theta}_{I} &= \frac{1}{n} \sum_{i=1}^{n}\left\{ \delta_{i}y_{i} + (1-\delta_{i}) \wh{m}(\bx_{i}) \right\} \notag \\
&= \underbrace{\frac{1}{n}\sum_{i=1}^{n}m(\bx_{i})}_{:= R_{n}} + \underbrace{\frac{1}{n}\sum_{i=1}^{n}\delta_{i}\left\{   y_{i}  - m(\bx_{i}) \right\}}_{:= S_{n} }  + \underbrace{\frac{1}{n}\sum_{i=1}^{n} (1-\delta_{i})\left\{ \wh{m}(\bx_{i}) - m(\bx_{i})  \right\}}_{:=T_{n}}.
\end{align}
Therefore, as long as we show 
\begin{align}
T_{n} = \frac{1}{n} \sum_{i=1}^{n}\delta_{i}\left\{ \frac{1}{\pi(\bx_{i})} - 1  \right\}\left\{ y_{i} - m(\bx_{i})  \right\} + o_{p}(n^{-1/2}),\label{12} 
\end{align}
then the main theorem automatically holds.

To show (\ref{12}), recall that the KRR can be regarded as the following optimization problem
\begin{align}
 \wh{\balpha}_{\lambda} = \argmin_{\balpha \in \mathbb{R}^{n}} (\by - \bK\balpha)\trans \bDelta_{n} (\by - \bK \balpha) + \lambda \balpha\trans \bK \balpha.\notag 
\end{align}
Further, we have
\begin{align}
  \wh{\balpha}_{\lambda} = \left( \bDelta_{n} \bK + \lambda \bI_{n}  \right)^{-1} \bDelta_{n}\by,\notag
\end{align}
and 
\begin{align}
 \wh{\bmm} &= \bK \left(  \bDelta_{n}\bK + \lambda \bI_{n}    \right)^{-1} \bDelta_{n}\by \notag \\
 &= \bK  \left\{  \left(  \bDelta_{n} + \lambda \bK^{-1}    \right) \bK     \right\}^{-1} \bDelta_{n} \by \notag \\
 &=  \left(    \bDelta_{n} + \lambda \bK^{-1}   \right)^{-1}  \bDelta_{n}\by,\notag
\end{align}
where $\wh{\bmm} = (\wh{m}(\bx_{1}), \ldots, \wh{m}(\bx_{n}))\trans$. Let $\bS_{\lambda} =  (  \bI_{n} + \lambda \bK^{-1}   )^{-1}$, we have
\begin{align}
 \wh{\bmm} = \left( \bDelta_{n} + \lambda \bK^{-1}  \right)^{-1}\bDelta_{n}\by = \bC_{n}^{-1} \bd_{n},\notag
\end{align}
where
\begin{align}
\bC_{n} &= \bS_{\lambda}\left( \bDelta_{n} + \lambda\bK^{-1}   \right),\notag\\
\bd_{n} &= \bS_{\lambda}\bDelta_{n}\by\notag.
\end{align}

By Lemma \ref{order}, let $\wt{\lambda} = \lambda/n$, we obtain
\begin{align}
\bC_{n} &= \mathbb{E}(\bDelta_{n} \mid \bx) + \mathcal{O}_{p}\left( \wt{\lambda} + \sqrt{\frac{ \gamma(\wt{\lambda})  }{n} } \right )\bone_{n} \notag \\
&:= \bPi + \mathcal{O}_{p}\left( \wt{\lambda} + \sqrt{\frac{ \gamma(\wt{\lambda})  }{n} } \right )\bone_{n},\notag
\end{align}
where $\bPi = \mbox{diag}(\pi(\bx_{1}), \ldots, \pi(\bx_{n}))$ and 
 $\gamma(\wt{\lambda})$ is the effective dimension of kernel $K$. Similarly, we have 
\begin{align}
\bd_{n} &= \mathbb{E}(\bDelta_{n}\by \mid \bx) + \mathcal{O}_{p}\left( \wt{\lambda} + \sqrt{\frac{ \gamma(\wt{\lambda})  }{n} }\right)\bone_{n} \notag \\
&= \bPi\bmm + \mathcal{O}_{p}\left( \wt{\lambda} + \sqrt{\frac{ \gamma(\wt{\lambda})  }{n} } \right )\bone_{n}.\notag
\end{align}
Consequently, letting $a_n = \wt{\lambda} +  \sqrt{\gamma(\wt{\lambda})/n} $ and applying 
 Taylor expansion, we have
\begin{align}
 \wh{\bmm} &= \bmm + \bPi^{-1} \left(    \bd_{n} - \bC_{n}\bmm      \right) + o_{p}\left( a_n  \right )\bone_{n}\notag\\ 
 &= \bmm + \bPi^{-1} \left\{   \bS_{\lambda}\bDelta_{n}\by - \bS_{\lambda}\left( \bDelta_{n} + \lambda\bK^{-1}   \right)\bmm      \right\} \notag\\
 &\quad+ o_{p}\left( a_n  \right )\bone_{n}\notag \\
 &= \bmm + \bPi^{-1} \bS_{\lambda}\bDelta_{n}  \left(  \by - \bmm      \right) + \mathcal{O}_{p}\left(a_n  \right )\bone_{n},\notag
 \end{align}
where the last equality holds because 
\begin{align}
\bS_{\lambda} \lambda\bK^{-1}\bmm &= \bS_{\lambda}\left\{ \left( \bI_{n} + \lambda\bK^{-1} \right) - \bI_{n}  \right\}\bmm \notag \\
&= \bmm -  \bS_{\lambda} \bmm = \mathcal{O}_{p}\left( a_n  \right ). \notag 
\end{align}
%where the last equality is by Lemma \ref{order}.

Therefore, we have
 \begin{align}
T_{n} &= n^{-1} \bone\trans \left(\bI_{n} - \bDelta_{n}\right) (\wh{\bmm} - \bmm) \notag \\
&= {n}^{-1} \bone\trans \left(\bI_{n} - \bDelta_{n}\right) \bPi^{-1} \bS_{\lambda}\bDelta_{n}  \left(  \by - \bmm      \right) +  \mathcal{O}_{p}\left( a_n  \right)\notag \\
& = n^{-1} \bone\trans \left(\bI_{n} -  \bPi \right) \bPi^{-1} \bDelta_{n}  \left(  \by - \bmm      \right) +  \mathcal{O}_{p}\left( a_n  \right) \notag \\
 &= n^{-1} \bone\trans \left(\bPi^{-1} - \bI_{n}\right) \bDelta_{n}  \left(  \by - \bmm      \right) +  \mathcal{O}_{p}\left( a_n  \right)
\notag. 
\end{align}


By Corollary 5 in \citet{zhang2013divide}, for $\ell$-th order of Sobolev space, we have
\begin{align}
  \gamma(\wt{\lambda})  &= \sum_{j=1}^{\infty} \frac{1}{1 + j^{2\ell}\wt{\lambda} } \notag\\
  &\leq \wt{\lambda}^{-\frac{1}{2\ell}}  + \sum_{j >\wt{\lambda} ^{-\frac{1}{2\ell}}} \frac{1}{ 1 + j^{2\ell} \wt{\lambda}} \notag\\
  &\leq \wt{\lambda}^{-\frac{1}{2\ell}} + \wt{\lambda}^{-1} \int_{\wt{\lambda}^{-\frac{1}{2\ell}}}^{\infty}z dz \notag \\
  &= \wt{\lambda}^{-\frac{1}{2\ell}}  + \frac{1}{2\ell-1}\wt{\lambda} ^{-\frac{1}{2\ell}} \notag\\
  &= O\left(\wt{\lambda}^{-\frac{1}{2\ell}}\right).
\end{align}
Consequently, as long as $\wt{\lambda}^{-\frac{1}{2\ell}} / n = o(1)$ and $\wt{\lambda} = o(n^{-1/2})$, we have
\begin{align} 
T_{n} &=  \frac{1}{n}\bone\trans \left(\bPi^{-1} - \bI_{n}\right) \bDelta_{n}  \left(  \by - \bmm      \right) + o_{p}(n^{-1/2}).
\end{align}
One legitimate of such $\wt{\lambda}$ can be chosen as $n^{-\ell}$, i.e., $\lambda = \mathcal{O}(n^{1-\ell})$.

\subsection*{B. Computational Details }


As the objective function in \eqref{entropy_method} is convex \citep{nguyen2010}, we apply the limited-memory Broyden-Fletcher-Goldfarb-Shanno (L-BFGS) algorithm to solve 
the optimization problem with the following first order partial derivatives:
\begin{align}
  \frac{\partial U }{\partial \alpha_{0}} = & 
   \frac{1}{n_{1}}\sum_{i=1}^{n}\mathbb{I}(\delta_{i} = 0)\exp\left( \alpha_{0} + \sum_{j=1}^{n}\alpha_{j}K(x_{i}, x_{j})  \right) - 1,\notag\\
   \frac{\partial U }{\partial \alpha_{k}} = & \frac{1}{n_{1}}\sum_{i=1}^{n}\mathbb{I}(\delta_{i} = 0)k(x_{i}, x_{k})\exp\left( \alpha_{0} + \sum_{j=1}^{n}\alpha_{j}K(x_{i}, x_{j})  \right) 
   - \frac{1}{n_{0}}\sum_{i=1}^{n}K(x_{i}, x_{k}) \notag\\
   &+ 2  \tau\sum_{i=1}^{n}K(x_{i}, x_{k})\alpha_{i}, k = 1, \ldots, n. \notag
\end{align}



For tuning parameter selection $\tau$ in (\ref{18}),  we adopt a cross-validation (CV) strategy. In particular, we may firstly stratify the sample $S = \{1, \ldots, n\}$ into two strata 
	$S_{0} =\{i\in S: \delta_{i} = 0\}$  and $S_{1} =\{i\in S: \delta_{i} = 1\}$. 
Within each $S_{h}$, we make $K$ random partition $\mathcal{A}_{k}^{(h)}$ such that
\begin{align}
\begin{gathered}
     \bigcup_{k=1}^{K}\mathcal{A}_{k}^{(h)} = S_{h}, h = 0, 1\notag\\
     \mathcal{A}_{k_{1}}^{(h)} \bigcap \mathcal{A}_{k_{2}}^{(h)} =  \emptyset, k_{1} \neq k_{2}, k_{1}, k_{2} \in \{1, \ldots, K\}, \notag\\
     \Abs{\mathcal{A}_{1}}^{h} \approx  \Abs{\mathcal{A}_{2}}^{(h)} \approx \cdots \approx \Abs{\mathcal{A}_{K}}^{(h)},  h = 0, 1,\notag
\end{gathered}
\end{align}
where $|\cdot|$ is the cardinality of a specific set. For a fixed $\tau > 0$, the corresponding CV criterion is
\begin{align}\label{cv}
	\mbox{CV}(\tau) = \frac{1}{K}\sum_{k=1}^{K}\sum_{j\in \mathcal{A}_{k}} \tilde{L}(\delta_{j},  \hat{g}^{(-k)}(x_{j}, \tau) ),
\end{align}
where $\hat{g}^{(-k)}$ is the trained model with data with data points except for $\mathcal{A}_{k} = \mathcal{A}_{k}^{(0)} \cup  \mathcal{A}_{k}^{(1)}$. Regarding the loss function in \eqref{cv}, we can use
\begin{align}
	\tilde{L}(\delta; \hat{g}) = \mathbb{I}(\delta = 1, \hat{p}(x) < 0.5   ) + \mathbb{I}(\delta = 0, \hat{p}(x) > 0.5   ),\notag
\end{align}
where $ 
	\hat{p}(x) = n_{1} /\{  n_{1} + n_{0}\hat{g}(x)    \} $ 
as an estimator for $p(x) = Pr(\delta = 1 \mid x)$. As a result, we may select the tuning parameter $\tau$ which minimizes the CV criteria in \eqref{cv}.


%\section{Introduction}
\label{sec:Introduction}


The goal in top-$\size$ recommendation is to recommend to each
consumer a small set of $\size$ items from a large collection of
items~\cite{cremonesi2010performance}.  For example, Netflix may want
to recommend $\size$ appealing movies to each consumer.  Collaborative
Filtering (CF)~\cite{herlocker2002empirical,lee2012comparative} is a
common top-$\size$ recommendation method.  CF infers user interests by
analyzing partially observed user-item interaction data, such as user
ratings on movies or historical purchase
logs~\cite{kanagal2012supercharging}. The main assumption in CF is that
users with similar interaction patterns have similar interests.


Standard CF methods for top-$\size$ recommendation focus on making  suggestions  that accurately reflect the user's preference history. However, as  observed in previous work,  CF recommendations are generally biased toward  popular items, leading to a rich get richer effect~\cite{vargas2014improving,steck2011item}.  The major reasons for this are \textit{popularity bias} and \textit{sparsity} of CF interaction data (detailed in Section~\ref{sec:related-work}). In a nutshell, to maintain  accuracy, recommendations are generated from the dense regions of the data,  where the popular items lie.  

However,  accurately suggesting popular items, may not be satisfactory for the consumers. For example, in Netflix, an accuracy-focused movie recommender may recommend ``Star Wars: The Force Awakens'' to users who have seen ``Star Wars: Rogue One''.  But, those users are probably already aware of ``The Force Awakens''. Considering additional factors, such as novelty of recommendations,  can lead to more effective suggestions~\cite{cremonesi2010performance,Castells2015,zhang2008avoiding,ziegler2005improving,zhang2012auralist}. 
%Second, accuracy-focused models typically achieve a   overall item-space coverage across their recommendations,  whereas high item-space coverage helps providers of the items increase revenue
%, users satisfaction since they are  likely already aware of or can find these items on their own.  

Focusing on popular items also adversely affects the satisfaction of  the providers of the items. This is because  accuracy-focused models typically achieve a  low overall item space coverage across their recommendations, whereas   high item space coverage helps providers of the items increase their revenue~\cite{vargas2014improving,Castells2015,adomavicius2011maximizing,anderson2006thelongtail, yin2012challenging,adomavicius2012improving}.
%accuracy-focused models typically achieve a

In contrast to the relatively small number of popular items, there are copious  {\it long-tail\/} items that have fewer observations (e.g., ratings) available. More precisely,  using the Pareto  principle (i.e.,~the $80/20$ rule),  long-tail items can be defined as items that generate the lower $20\%$ of observations~\cite{yin2012challenging}. Experimentally we found that these items correspond to almost $85\%$ of the items in several datasets (Sections~\ref{sec:Notation} and \ref{sec:Experiments}). %Table~\ref{tab:DatasetStatsticsSmall})


As previously shown, one way to improve the novelty of top-$\size$ sets is to recommend interesting long-tail items~\cite{cremonesi2010performance,ge2010beyond}.  The intuition  is that since they have fewer observations available,  they are more likely to be unseen~\cite{Kaminskas:2016:DSN:3028254.2926720}.  
 %For example, in online commerce,  newly added items are long-tail items that are yet to be discovered.  
Moreover, long-tail item promotion also results in higher overall coverage of the item space%, which increases profits for providers of the items
~\cite{vargas2014improving,Castells2015,zhang2008avoiding,zhang2012auralist,adomavicius2011maximizing,anderson2006thelongtail,yin2012challenging,jambor2010optimizing}. Because long-tail promotion reduces accuracy~\cite{steck2011item}, there are trade-offs to be explored.


%original submitted to ICDE
%This work studies three aspects of top-$\size$ recommendation: accuracy, novelty, and item-space coverage, and examines their trade-offs. In most previous work, predictions of a base recommendation system are re-ranked to handle their trade-offs~\cite{adomavicius2012improving,jambor2010optimizing,zhang2013personalize,wang2009portfolio}. Due to performance considerations, however, these techniques are not customized per user. For example,  parameters that balance the trade-off between novelty and accuracy are cross-validated at a global level.  This can be detrimental since users have varying preferences for  objectives such as long-tail novelty. We explore how to  automatically infer  user  preference for long-tail novelty, and how to leverage  it to correct  the popularity bias in standard recommender models. Our work does not rely on any additional contextual data, although such data, if available, can help promote newly-added long-tail items~\cite{agarwal2009regression,Saveski:2014:ICR:2645710.2645751}.

This work studies three aspects of top-$\size$ recommendation: accuracy, novelty, and item space coverage, and examines their trade-offs. In most previous work, predictions of a base recommendation algorithm are \textit{re-ranked} to handle these trade-offs~\cite{adomavicius2012improving,jambor2010optimizing,zhang2013personalize,wang2009portfolio}. The re-ranking models are computationally efficient but suffer from two drawbacks. First, due to performance considerations,  parameters that balance the trade-off between novelty and accuracy  are not customized per user. Instead they are cross-validated at a global level.  This can be detrimental since users have varying preferences for  objectives such as long-tail novelty. Second,  the re-ranking methods are often limited to a specific base recommender  that may be sensitive to dataset density. 
As a result, the datasets are pruned and the problem is studied in dense settings~\cite{adomavicius2012improving,ho2014likes}; but real world  scenarios are often sparse~\cite{kanagal2012supercharging,liu2017experimental}.   
% Because  dataset density can impact the performance of most base recommenders (like R-SVD), which in turn affects the performance of the re-ranking model, 

\iffalse
We address these limitations by directly inferring  user  preference for long-tail novelty  from interaction data.  This  allows us to customize the re-ranking  per user, and design a \textit{generic} framework, which resolves the second problem. In particular, since the long-tail novelty preferences are estimated independently of any base  recommender model, we can  plug-in an appropriate base recommender w.r.t. the dataset sparsity.% including ones that are more suitable for sparse settings.  

Modelling  user  preference for  long-tail novelty using only item popularity statistics, e.g., the average popularity of rated items as in~\cite{jugovac2017efficient}, disregards additional information like whether the user found the item interesting and the long-tail preferences of other users  of the items. \iffalse To incorporate them, we introduce the notion of  \emph{item long-tail importance}. Both  user long-tail preferences and item long-tail importance are dependent:  a user has high preference for discovering long-tail items if she is interested in important long-tail items, and an item that is associated with many of these kinds of users is likely to be more important.  We propose a joint optimization framework to directly learn,  from interaction data, both the users' long-tail preferences and the  items' long-tail importance. \fi
We propose an optimization approach that  incorporates  this information and  directly learns,  from interaction data, the users' long-tail novelty preferences.

Next, we use these learned preferences  to design a  top-$\size$ recommendation framework thats is generic, and provides customized balance between accuracy, novelty, and coverage. We refer to it as framework as GANC.  Using GANC, we design a novel algorithm, {\it Ordered Sampling-based Locally Greedy (OSLG)\/}, that relies on the learned long-tail novelty preferences  to scalably correct for popularity bias. Our work does not rely on any additional contextual data, although such data, if available, can help promote newly-added long-tail items~\cite{agarwal2009regression,Saveski:2014:ICR:2645710.2645751}. In summary:
\fi

We address the first limitation by directly inferring  user  preference for long-tail novelty  from interaction data.   Estimating these  preferences  using only item popularity statistics, e.g., the average popularity of rated items as in~\cite{jugovac2017efficient}, disregards additional information, like whether the user found the item interesting or the long-tail preferences of other users  of the items. We propose an approach that  incorporates  this information and  learns the users' long-tail novelty preferences from interaction data.

This approach allows us to customize the re-ranking  per user, and  design a \textit{generic} re-ranking framework, which resolves the second limitation of prior work. In particular, since the long-tail novelty preferences are estimated independently of any base recommender, we can  plug-in an appropriate one w.r.t. different factors, such as the dataset sparsity.

Our top-$\size$ recommendation framework, \textbf{GANC}, is \textbf{G}eneric, and provides customized balance between \textbf{A}ccuracy, \textbf{N}ovelty, and \textbf{C}overage. % Moreover, based on the learned long-tail novelty preferences, we also design a novel algorithm, {\it Ordered Sampling-based Locally Greedy (OSLG)\/}, that relies on the learned long-tail novelty preferences  to scalably correct for popularity bias. 
Our work does not rely on any additional contextual data, although such data, if available, can help promote newly-added long-tail items~\cite{agarwal2009regression,Saveski:2014:ICR:2645710.2645751}. In summary:

%Consider  the following toy example:
\vspace{-0.2cm}
\begin{table}[htb]
\centering
\scriptsize
%\small
\begin{tabular}{ccccccc} 
%\toprule
%&\multirow{2}{*}{}&\multicolumn{7}{c}{Ratings}\\
& & \cellcolor{blue!35}$w_1$ &\cellcolor{blue!18} $w_2$ & $\dots$ &\cellcolor{blue!8} $w_{89}$  &\cellcolor{blue!8} $w_{99}$   
\\
&   &$i_1$&$i_2$&$\dots$&$i_{89}$&$i_{90}$\\ 
\cmidrule(r){3-7} 	 
%\midrule
\cellcolor{red!35}$\theta_1$  &$u_1 $   &5 &   & $\dots$ &  &   \\
\cellcolor{red!28}$\theta_2$  &$u_2$     &5 &    & $\dots$ &  &  \\
 $\theta_3=?$  &$\bf u_3$  &5 &  &   $\dots$ &  &  \\
\cellcolor{red!10}$\theta_4$ & $u_4$  &  &5   & $\dots$ & &\\ 
\cellcolor{red!10}$\theta_5$ & $u_5$  &  & 5  & $\dots$ & &\\ 
$\theta_6=?$  & $\bf u_6$ & &5  &      $\dots$& &  \\ 
 & & $\hdots$  &$\hdots$   &$\hdots$   &$\hdots$   &$\hdots$  \\
%\midrule 
\cmidrule(r){3-7} 	 
\multicolumn{2}{c}{item pop.}  & 3  & 3  & $\dots$ &50&60\\  
%\bottomrule
%$ f_i$    &3  &3  &1  &3  &1  &2  \\  \hline
\end{tabular}
%#.
\caption{Simplified user-item interaction data. The user long-tail novelty preference ($\theta_u$), item long-tail importance weight ($w_i$) are highlighted. Darker colors indicate larger values. } \label{tab:example}
\end{table} 
\vspace{-0.2cm}
\begin{example}  
In Table~\ref{tab:example}, we are interested in estimating $\theta_3$ and $\theta_6$,  the long-tail preference of users $u_3$ and $u_6$ who have each rated a single movie. Additional ratings for other users  are not included here.  Considering only rating information, we observe $i_1$ and $i_2$ are  equally popular $|\mathcal{U}_{i_1}^{\trainset}| = |\mathcal{U}_{i_2}^{\trainset}|=3$, and $r_{31}=5$ and $r_{62}=5$. Using Eq.~\ref{eq:tfidf-risk}  we have $\theta_3 = \theta_6$. However, if we were given the long-tail preferences of the each item's user set, specifically that $u_1$ and $u_2$ have high long-tail preference (darker red), while $u_4$ and $u_5$ have lower long-tail preference (lighter red), we could conclude $i_1$ is a more important long-tail item compared to $i_2$ (indicated by a darker blue shade for $w_1$), and we expect  $\theta_3 \geq \theta_6$.

% On the other hand, if we knew that $u_4$ and $u_5$ have lower long-tail preference, we could conclude $i_2$ is a  less significant long-tail item. Therefore, However, if we  consider the long-tail preferences of other users, we may reason differently.    We need another variable $w_i$ which captures this information. 
%we would conclude that $u_3$ has higher long-tail preference compared to $u_6$, since the users $i_1$ is a more prominent long-tail item. 

% Relying only  on item popularity information, we would  conclude   $u_3$ and $u_6$ have equal long-tail preference, since $i_1$ and $i_2$ are  equally popular. However, considering  the second column,  long-tail preference of users,  long-tail importance for each item,  which captures the long-tail preference of its users. Since  that  both users of $i_1$ have high long-tail preference while  the users of $i_2$ have lower preference,  we may conclude $i_1$ is a more important long-tail item compared to $i_2$. Therefore, $u_3$'s long-tail preference should be at least as large as $u_6$'s preference. Specifically, consider two  items $i_1$ and $i_2$, with the following rating data: $i_1=\{u_1:5, u_2:5, u_3:5 \}$, $i_2=\{u_4:5, u_5:5, u_6:5\}$.  

%Table~\ref{tab:example} shows  simplified rating data. We want an estimate of the long-tail preference of $u_3$ and $u_6$, who have each  rated a single movie.  Relying only  on movie popularity information, we would  conclude   $u_3$ and $u_6$ have similar long-tail preference, since $m_1$ and $m_2$ are  equally popular. However, considering the long-tail preferences of other users of those movies, we may reason differently: since $u_1$ and $u_2$ have high long-tail preference, and $u_4$ and $u_5$ have low long-tail preference, $m_1$ is a more prominent long-tail item compared to $m_2$. Therefore, it is likely that $u_3$ has higher long-tail preference compared to $u_6$.considering the long-tail preferences of other users of those movies, we may reason differently.  For example, 
\label{ex:running}
\end{example}



%------------------------------

\iffalse
\begin{example}
Table~\ref{tab:example} shows rating data for a simplified system. %Note the user-item interaction matrix is sparse.
For this example, we define popular movies as those that have received  three or more ratings; $\{m_1, m_2, m_4\}$ are popular and  $\{m_3, m_5, m_6\}$ are niche movies. We observe $u_1$ and $u_3$  have rated relatively popular movies (risk-averse) while $u_2$ and $u_4$ have rated niche movies (risk-loving). 
\label{ex:running}
\end{example}

\begin{table}[htb]
\centering
\scriptsize
\begin{tabular}{ccccccc} 
\toprule
			&$m_1$ &$m_2$   &$m_3$    &$m_4$   &$m_5$ &$m_6$  \\ \hline 
$u_1 $ &5  &4  & - &-  &-  &-   \\
$u_2$  &-  &-  &-  &-  &5  &5   \\
$u_3$  &-  &4  &-  &5  &-  &-   \\
$u_4$  &-  &-  &3  &-  &-  &4   \\ 
$u_5$  &5  &-  &-  &3  &-  &-   \\ 
$u_6$  &4  &2  &-  &4  &-  &-   \\ 
\bottomrule
%$ f_i$    &3  &3  &1  &3  &1  &2  \\  \hline
\end{tabular}
\caption{User-Movie rating data} \label{tab:example}
\end{table}

It is essential to consider consumer characteristics in designing recommender systems so that they promote long-tail items to the right group of users and spread demand evenly between hit and niche items.  

\fi





%------------------------------
\iffalse
\begin{table}[htb]
\centering
\scriptsize
\begin{tabular}{ccccccc} 
\toprule
			&$m_1$ &$m_2$   &$m_3$    &$m_4$   &$m_5$ &$m_6$  \\ \hline 
$u_1 $ &\textbf{5}  & \textbf{4}  &\textcolor{gray}{ 1.2} &-  &-  &-   \\
$u_2$  &-  &-  &-  &-  & \textbf{5}  &\textbf{5}   \\
$u_3$  &-  &\textbf{4}  &-  &\textbf{5}  &-  &-   \\
$u_4$  &-  &-  &\textbf{3}  &-  &-  &\textbf{4}   \\ 
$u_5$  &\textbf{5}  &-  &-  &\textbf{3}  &-  &-   \\ 
$u_6$  &\textbf{4}  &\textbf{2}  &-  &\textbf{4}  &-  &-   \\ 
\bottomrule
%$ f_i$    &3  &3  &1  &3  &1  &2  \\  \hline
\end{tabular}
\caption{User-Movie rating data} \label{tab:example}
\end{table}
% $\mathcal{P}^1= \{ \mathcal{P}_1^1 \{i_1,i_2,i_3\}, \mathcal{P}_2^1:\{i_2,i_3,i_5\}  \}$
 %$\mathcal{P}^2= \{ \mathcal{P}_1^2: \{i_1,i_2,i_3\}, \mathcal{P}_2^2:\{i_2,i_5,i_6\}  \}$
 %$\mathcal{P}^3= \{ \mathcal{P}_1^3: \{i_7,i_8,i_9\}, \mathcal{P}_2^3:\{i_{10},i_{11},i_{12}\}  \}$
\begin{table}[htb]
\centering
\tiny
\begin{tabular}{ccc} 
\toprule
		&$u_1$&$u_2$  \\ \hline 
$\mathcal{P}^1 $ & $\{i_1,i_2,i_3\}$ & $\{i_2,i_3,i_5\} $ \\
$\mathcal{P}^2$ & $\{i_1,i_2,i_3\}$ & $\{i_2,i_5,i_6\} $ \\
$\mathcal{P}^3$ & $\{i_7,i_8,i_9\}$ & $\{i_{10},i_{11},i_{12} \}$ \\
\bottomrule
%$ f_i$    &3  &3  &1  &3  &1  &2  \\  \hline
\end{tabular}
\caption{Top-$\size$ allocations to users.} \label{tab:paretoExamples}
\end{table}
\fi


\iffalse
When considering long-tail items, it is important to consider consumers' willingness  to explore niche or unpopular items and their propensity towards similar items. In particular, they can be characterized by their  {\it risk degree\/} and {\it focusing degree\/}, respectively.  We compute these estimates  based on historical rating information. The following example further describes these notions in the context of movie rating data. 

\begin{example}  
Table~\ref{tab:example} shows rating data for a simplified system with $6$ users, $6$ movies, and $3$ genres. $m_i^{j}$ implies that movie $m_i$ belongs to genre $j$. Note the user-item interaction matrix is sparse. 
  For this setting, we define popular movies as those that have received  three or more ratings; $\{m_1, m_2, m_4\}$ are popular and  $\{m_3, m_5, m_6\}$ are niche movies. We now profile the users according to their risk and focusing degree. E.g., $u_1$ has rated relatively popular movies belonging to the same genre (risk-averse, high focusing degree); $u_2$ has rated niches movies in the same genre (risk-loving, high focusing degree); $u_3$ has rated popular movies in two different genres (risk-averse, low focusing degree), and $u_4$ has rated niches movies in two different genres (risk-loving, low focusing degree). 
\label{ex:running}
\end{example}
\begin{table}[htb]
\centering
\tiny
\begin{tabular}{ccccccc} 
\toprule
			&$m_1^{1}$ &$m_2^{1}$   &$m_3^{2}$    &$m_4^{3}$   &$m_5^{3}$ &$m_6^{3}$  \\ \hline 
$u_1 $ &5  &4  &-  &-  &-  &-   \\
$u_2$  &-  &-  &-  &-  &5  &5   \\
$u_3$  &-  &4  &-  &5  &-  &-   \\
$u_4$  &-  &-  &3  &-  &-  &4   \\ 
$u_5$  &5  &-  &-  &3  &-  &-   \\ 
$u_6$  &4  &2  &-  &4  &-  &-   \\ 
\bottomrule
%$ f_i$    &3  &3  &1  &3  &1  &2  \\  \hline
\end{tabular}
\caption{User-Movie rating data} \label{tab:example}
\end{table}
It is essential to consider these consumer characteristics in designing recommender systems so that they promote long-tail items to the right group of users and spread demand evenly between the hit and niche items.  
\fi
\iffalse
\begin{center}
\begin{figure*}[tp]
%\scalebox{0.5}{%
\resizebox{1\textwidth}{!}{%
%\small%\addtolength{\tabcolsep}{5pt}% below sums to 8
\begin{tabularx}{1.5\textwidth}{>{\hsize=2.5\hsize}X>{\hsize=2.5\hsize}X>{\hsize=0.5\hsize}X>{\hsize=0.5\hsize}X>{\hsize=0.5\hsize}X>{\hsize=0.5\hsize}X>{\hsize=0.5\hsize}X>{\hsize=0.5\hsize}X}
    \multirow{12}{*}{\includegraphics[scale=0.3]{codeForExample/popularity-movie.png}} & \multirow{12}{*}{\includegraphics[scale=0.3]{codeForExample/scatterplot.png}} & & & & & & \\
%   & &               &       &       &       &       &       \\
    & &\multicolumn{1}{l|}{}               &$m_1^{g1}$   	&$m_2^{g1}$    	&$m_3^{g2}$    &$m_4^{g2}$      &$m_5^{g3}$    \\ \cline{3-8}%\hline
    & &\multicolumn{1}{l|}{u1}          &5  &5  &-  &-   &-  \\
    & &\multicolumn{1}{l|}{u2}    		&-  &-  &4  &4  &5  \\
    & &\multicolumn{1}{l|}{u3}   			&1  &2  &1  &-  &-   \\
    & &\multicolumn{1}{l|}{u4}     		&1  &-  &-  &-  &-  \\
    & &               &       &       &       &       &       \\
    & &               &       &       &       &       &       \\
    & &               &       &       &       &       &       \\
    & &               &       &       &       &       &	\\
    \\
\end{tabularx}}
\caption{User-Movie interaction data a) Popularity-Movie histogram b)Movie genres/clusters c) User-Movie rating data} \label{fig:example}
\end{figure*}
\end{center}
\fi



%We propose a novel approach that allows us to  promote long-tail items in a targeted manner, thereby improving the novelty of top-$\size$ sets, the overall item-space coverage across recommendations, while maintaining reasonable levels of accuracy.

%Next, we integrate these learned preferences  in a generic  top-$\size$ recommendation framework to provide customized balance between accuracy and coverage.

%sequentially make recommendations, while adjusting its parameters with regard to the set of top-$\size$ recommendations made so far. However, since  sequential parameter updates  cause  scalability issues, we propose a sampling based algorithm. This variant of our framework, called {\it Ordered Sampling-based Locally Greedy (OSLG)\/},  allows us to  correct for the popularity bias in recommendations with regard to individual user long-tail preferences. 

%ICDE submission
%Our framework differs with  prior work in the following aspects:  unlike~\cite{adomavicius2011maximizing,adomavicius2012improving,zhang2013personalize,ho2014likes},  the long-tail preference personalization in our framework is learned rather than optimized using cross-validation or parameter tuning. In other words, our personalization method is independent of the underlying base  recommendation models.  Moreover, our framework is  generic. This enables us to  plug-in several base recommenders, and evaluate their  effectiveness without requiring  extensive tuning for the accuracy and coverage trade-off. 


%\vspace{-2.8pt}
\begin{itemize}

\item  We examine various measures for estimating user long-tail novelty preference in Section~\ref{sec:lt-pref} and formulate an optimization problem  to directly learn users' preferences for long-tail  items from interaction data in Section~\ref{sec:learning-lt-pref}. %In addition, we introduce several heuristics for measuring the user preference for less common items from historical rating data.% 

\item  We integrate the user preference estimates into GANC %, a generic re-ranking framework that provides customized balance between accuracy, novelty, and coverage 
(Section~\ref{sec:RiskbasedReranking}), and  introduce {\it Ordered Sampling-based Locally Greedy (OSLG)\/}, a scalable algorithm that relies  on user long-tail preferences to correct the popularity bias (Section~\ref{sec:optimizationAlgorithm}).
%We introduce OSLG, a scalable algorithm that relies  on user long-tail preferences to  maximize item space coverage \textcolor{red}{while maintaining acceptable levels of accuracy} (Section~\ref{sec:optimizationAlgorithm}).

\item   We conduct an extensive empirical study and evaluate performance from  accuracy, novelty, and coverage perspectives (Section~\ref{sec:Experiments}).  We use five  datasets with varying density and difficulty levels. %:  Netflix, MovieTweetings, and MovieLens (100K, 1M, 10M). 
  In contrast to most related work,  our evaluation considers realistic settings that include a large number of infrequent  items and users. %This enables us to study the impact of  data density on the performance trade-offs of several  state of the art top-$\size$ recommendation algorithms. %   %,  and use the all-items ranking protocol~\cite{steck2013evaluation,vargas2014improving}, where performance is measured using all items with train data. to evaluate the performance of several  state of the art top-$\size$ recommendation algorithms 
 
\item Our empirical results confirm that the performance of re-ranking models is impacted by the underlying   base recommender and the dataset density. Our generic approach enables us to easily incorporate a suitable base recommender to devise an effective solution for both dense and sparse settings. In dense settings, we use the same base recommender as existing re-ranking approaches, and we outperform them in accuracy and coverage metrics. For sparse settings, we plug-in a more suitable base recommender, and devise an effective solution that is competitive with existing top-$\size$ recommendation methods in accuracy and novelty. 

%Directly estimating the long-tail novelty preferences allows us to customize re-ranking per user, and  devise a generic framework.   
 
\end{itemize}

Section~\ref{sec:related-work} describes related work. Section~\ref{sec:conclusion} concludes.
 
%\begin{figure*}[!h]
 \centering
 \centerline{\includegraphics[width=17cm]{framework}}
 \caption{The proposed multi-source remote sensing image registration framework of MS-HLMO, including Harris feature point detection, Histogram of Local Main Orientation feature extraction, and multi-scale registration strategy.}
 \label{fig:framework}
\end{figure*}

The framework of the proposed MS-HLMO registration algorithm is shown in Fig.\ref{fig:framework}. The input multi-source image pair to be registered is preprocessed, which includes data normalization and basic denoising. Then, the preprocessed single-band images are used for feature points detection and feature extraction. Harris corner point detection, which contains detail treatments for multi-source images, is adopted to generate feature points between the image pair for matching. The key process of the proposed HLMO feature extraction is carried out in a multi-scale strategy, in which Gaussian pyramids are built to create a scale-space of the images. The HLMO feature descriptors of each Harris corner point are extracted on the PMOM of the images. The feature points between the image pair are then matched according to the descriptors, and Fast Sample Consensus (FSC) is carried out to remove the outliers. The matching results in the scale-space are combined through a multi-scale matching strategy. Finally, the spatial transformation between the original image pair is determined by the coordinate relationship between matched feature points according to a selected transformation model.



\subsection{Harris Feature Point Detection}
\label{ssec:subhead}
Harris corner \cite{harris1988combined} is one of the most stable feature points, which is slightly affected by intensity and scale difference and has high computational efficiency \cite{gao2021multi,2021Multi}. It has the advantage in multi-source remote sensing images with multi-modal properties and large data size. Here, the similar strategy \cite{2021Multi} is used for feature points detection. The Harris corner response of each pixel is calculated by:
\begin{equation}
cornerness = \frac{{{\rm{det}}(\textbf{\emph{M}})}}{{{\rm{tr}}(\textbf{\emph{M}})}}
\end{equation}
\begin{equation}
\textbf{\emph{M}} = \left[ {\begin{array}{*{20}{l}}
{\sum\limits_{{\bf{W}_\sigma}} {{{\bf{G}}_x}^2} }&{\sum\limits_{{\bf{W}_\sigma}} {{{\bf{G}}_x} {{\bf{G}}_y}}}\\
{\sum\limits_{{\bf{W}_\sigma}} {{{\bf{G}}_x} {{\bf{G}}_y}} }&{\sum\limits_{{\bf{W}_\sigma}} {{{\bf{G}}_y}^2}}
\end{array}} \right]
\end{equation}
where $\rm{det}(\textbf{\emph{M}})$ and $\rm{tr}(\textbf{\emph{M}})$ are the determinant and trace of $\textbf{\emph{M}}$, respectively, ${{\bf{G}}_x}$ and ${{\bf{G}}_y}$ are the image's gradient along $x$ and $y$ directions, respectively, and $\bf{W}_\sigma$ is a Gaussian window with variance $\sigma$. Pixels with strong response are considered to be feature points with distinct structure and stability between multi-source images.

An important issue in practical multi-source remote sensing image registration is that the data size and scale relations between images to be registered are diverse. For example, an image with high resolution covers a smaller spatial area. Many of the existing algorithms only deal with the ideal case that the image pair has the same scale and size. This paper focuses on solving several key problems at the same time, that is, two uncertain factors of image scale and size should be considered simultaneously. The proposed MS-HLMO adopts local non-maximum suppression (LNMS) to solve this problem. Since the size and scale difference of the image pairs are uncertain, in Harris corner detection, it is expected that the feature points in the image pair are distributed as uniformly as possible with the use of LNSM. Then, the ratio of the window size in LNMS is set depending on the ratio of the data size of the image pair:
\begin{equation}
ratio = \sqrt {\frac{{M \times N}}{{m \times {\rm{n}}}}}
\end{equation}
where $M,N$ and $m,n$ are the length and width of the two images, respectively.

\begin{figure}[!h]
    \centering
        \subfloat[]{\includegraphics[width=2.5in]{keypoints_1.pdf}
        \label{fig:harris:a}}
    \hfil
        \subfloat[]{\includegraphics[width=2.5in]{keypoints_2.pdf}
        \label{fig:harris:b}}
    \caption{Examples of feature points detection results with LNMS. (a) Detection results of the image pair with different scale. (b) Detection results of the image pair with different scale and size.}
    \label{fig:harris}
\end{figure}

Consequently, the distribution of feature points is consistent with the images' scale proportion when there are scale differences, as shown in Fig.\ref{fig:harris:a}. When there is a size difference, the feature points are far more uniformly distributed, and the repeatability is higher, as shown in Fig.\ref{fig:harris:b}.



\subsection{Histogram of Local Main Orientation}
\label{ssec:subhead}

%\subsubsection{Robust Feature of Gradient Orientation}
%
%Most feature-point-based registration algorithms use a local descriptor to extract the neighborhood information of the keypoints, and generate their feature vectors for similarity matching.  For example, SIFT, SURF, HOG and PIIFD use local gradient information for statistics. However, the performance of these algorithms is greatly reduced when processing multi-source , especially multi-sensor images. Through practice and analysis on , it is believed that multi-modal images will have unpredictable and serious differences in the magnitude of gradient, but the orientation of gradient is a more stable feature.
%
%\begin{figure}[h!]
% \begin{center}
%  \includegraphics[width=3.5in]{Robust_Gradient_Orientation.pdf}
%  \caption{Robust gradient orientation.}
%  \label{fig:res}
% \end{center}
%\end{figure}
%
%To intuitively show the characteristics of this orientation, we briefly summarize the common intensity distortion in multi-modal images, as shown in Fig.0. The four images can be taken as the edge of an object or the interface of two substances in the image. Assume that the center point is the detected feature point, and the image block represents the intensity information of the neighborhood of it. In Fig.0 (a), the intensity amplitude of the left part is lower than that of the right part. Obviously, the gradient orientation is horizontal to the right, which is 0°; Take Fig.0 (a) as the original or reference image, then Fig.0 (b), Fig.0 (c), and Fig.0 (d) is considered as intensity distortions ${{\cal F}_{{\rm{Ra}}}}( \bullet )$ of (a). In (b), the magnitude relationship of the two parts remains unchanged, but the difference between them changes. So the gradient amplitude changes, but the orientation remains 0°; In (c), due to large intensity distortion, the magnitude relationship changes. At this time, the gradient orientation is reversed compared with (a), but notice that it is still on the same line with the orientation in (a). If the gradient orientation is limited to [0°, 180°), the orientation in (c) is still 0°. (d) represents a general situation, which is considered as non-linear distortion of intensity, or some degradation of (a), such as down-sampling or blurring. At this time, the amplitude is hard to determine, but the orientation is still 0° to the right horizontally. A typical multi-source data is shown in Fig.0, which is an optical-infrared image pair. It basically contains the above intensity distortion problems, which is discussed in detail in the next subsection.
%
%In multi-source images, due to the differences in sensors, tempors, environments, etc., various intensity distortions may be caused, resulting in the multi-modal attributes of the image. The morphology of the detailed part of the image is basically in the four cases in Fig.0. In summary, the magnitude of the local gradient varies, but the orientation is basically stable, which defines the feature information that should be focused.


\subsubsection{Partial main orientation map}
Feature-point-based registration algorithms often use a local descriptor to extract the neighborhood information of the keypoints, and generate their feature vectors for similarity matching. For example, SIFT \cite{lowe2004distinctive}, HOG \cite{dalal2005histograms}, SURF \cite{bay2006surf} and PIIFD \cite{chen2010partial} employ gradient information as the basic feature. However, the performance of these algorithms is greatly degraded when processing multi-source, especially multi-sensor images. So, it is critical to extract invariant feature that is robust to ${{\cal F}_{\rm{Int}}}( \bullet )$, ${{\cal F}_{\rm{Rot}}}( \bullet )$, and ${{\cal F}_{\rm{Chg}}}( \bullet )$ for feature points description. A partial main orientation map (PMOM) is designed as the feature map in MS-HLMO, in which the Average Squared Gradient (ASG)\cite{kass1987analyzing} is adopted.

The ASG is a gradient weighting method. The elementary gradient of the image along $x$ and $y$ directions, i.e., ${\bf{G}}_x$ and ${\bf{G}}_y$ are calculated as
\begin{equation} \label{eqPMOM1}
\left[ \begin{array}{l}
{{\bf{G}}_x}(x,y)\\
{{\bf{G}}_y}(x,y)
\end{array} \right] = \left[ \begin{array}{l}
\frac{\partial }{{\partial x}}{\bf{I}}(x,y)\\
\frac{\partial }{{\partial y}}{\bf{I}}(x,y)
\end{array} \right]
\end{equation}
where ${\bf{I}}(x,y)$ represents the single-layer gray-scale image. The magnitude and orientation of its gradient, i.e., ${\bf{G}}_\rho$ and ${\bf{G}}_\varphi$ are
\begin{equation} \label{eqPMOM2}
\left[ \begin{array}{l}
{{\bf{G}}_\rho}\\
{{\bf{G}}_\varphi}
\end{array} \right] = \left[ \begin{array}{l}
\sqrt {{{\bf{G}}_x}^2 + {{\bf{G}}_y}^2} \\
\arctan \frac{{\bf{G}}_y}{{\bf{G}}_x}
\end{array} \right]\end
{equation}

In the ASG, a locally weighted squared gradient \cite{kass1987analyzing} along $x$ and $y$ directions, ${\bf{G}}_{{{\bf{W}}_\sigma},s,x}$ and ${\bf{G}}_{{\bf{W}_\sigma},s,y}$ are obtained as
\begin{equation} \label{eqPMOM3}
\left[ \begin{array}{l}
{{\bf{G}}_{{{\bf{W}}_\sigma},s,x}}\\
{{\bf{G}}_{{{\bf{W}}_\sigma},s,y}}
\end{array} \right] = \left[ \begin{array}{l}
\sum\limits_{{\bf{W}}_\sigma} {{{\bf{G}}_x}^2 - {{\bf{G}}_y}^2} \\
\sum\limits_{{\bf{W}}_\sigma} {2{{\bf{G}}_x}{{\bf{G}}_y}}
\end{array} \right]
\end{equation}
where ${\bf{W}}_\sigma$ is a Gaussian window with variance $\sigma$. Accordingly, the orientation of this gradient is
\begin{equation} \label{eqPMOM4}
{{\bf{G}}_{{{\bf{W}}_\sigma },s,\varphi }} = \angle ({{\bf{G}}_{{{\bf{W}}_\sigma },s,x}},{{\bf{G}}_{{{\bf{W}}_\sigma },s,y}})
\end{equation}
where $\angle (X,Y)$ is defined as
\begin{equation} \label{eqPMOM5}
\angle (X,Y) = \left\{ \begin{array}{l}
\arctan (\frac{Y}{X}), X \ge 0\\
\arctan (\frac{Y}{X}) + \pi , X < 0,Y \ge 0\\
\arctan (\frac{Y}{X}) - \pi , X < 0,Y < 0
\end{array} \right.
\end{equation}
making ${\bf{G}}_{{{\bf{W}}_\sigma },s,\varphi }$ within $(-\pi,\pi)$. According to \cite{kass1987analyzing}, this gradient is obtained by doubling the angle of the original gradient, so the orientation of the ASG is
\begin{equation} \label{eqPMOM6}
{{\bf{G}}_{{{\bf{W}}_\sigma},\varphi }} = \frac{1}{2}{{\bf{G}}_{{{\bf{W}}_\sigma},s,\varphi }}
\end{equation}

Compared with the classical gradient operator, the ASG orientation ${{\bf{G}}_{{{\bf{W}}_\sigma},\varphi}}\in(-\frac{\pi}{2},\frac{\pi}{2})$ reflects the weighted gradient orientation of a local region ${\bf{W}}_\sigma$, which is more robust and stable. In addition, the $x$ direction gradient is constant, and this characteristic meets the requirement that not affected by the reversal of gradient in intensity difference. Note that when $\sigma$ increases, the scale of ASG increases, which makes the local orientation more invariant to intensity difference and noise, but the uniqueness of local features decreases. From this, the following function is defined:
\begin{equation} \label{eqPMOM7}
{{\bf{G}}_{PMOM}} = \frac{1}{2}\angle (\sum\limits_\sigma  {\sum\limits_{{\bf{W}}_\sigma } {{{\bf{G}}_x}^2 - {{\bf{G}}_y}^2}} ,\sum\limits_\sigma  {\sum\limits_{{\bf{W}}_\sigma } {2{{\bf{G}}_x}{{\bf{G}}_y}} } )
\end{equation}
where a series of scale $\sigma$ are preset, the weighted responses ${\bf{G}}_{{{\bf{W}}_\sigma},s,x}$ and ${\bf{G}}_{{{\bf{W}}_\sigma},s,y}$ at each scale are added, and the ASG orientation is obtained. By filtering the image with Eq.(\ref{eqPMOM7}), the PMOM is obtained, where its value reflects the overall orientation of multiple scales in each partial area of the image.

\begin{figure}[h!]
 \begin{center}
  \includegraphics[width=3.5in]{Comparion_of_Feature_Maps.pdf}
  \caption{Comparison of feature maps of a selected scene, including a visible and an infrared image.}
  \label{fig:maps}
 \end{center}
\end{figure}

A visualized comparison of PMOM with other feature maps of typical multi-source data is shown in Fig.\ref{fig:maps}. The original data is a visible-infrared image pair, which contains obvious intensity difference. For comparison purposes, the images have been registered manually, basically eliminating the spatial differences of scale, rotation, and size. The magnitude and orientation of images' gradient are shown in Fig.\ref{fig:maps}, which are obtained using Eqs.(\ref{eqPMOM1})-(\ref{eqPMOM2}). These are the basic feature information in most algorithms \cite{lowe2004distinctive,dalal2005histograms,bay2006surf,chen2010partial,2021Multi}. It is observed that, due to the multi-modal properties of the original image pair, these two feature maps have large differences and instability, which is the main reason for the failure of most algorithms. The MIM \cite{li2019rift} of RIFT shown in Fig.\ref{fig:maps} also focuses on the local orientation of the image, where the maximum index is the main orientation among several ones. Compared with the directional gradient, Gabor transformation has a more stable response, which leads to RIFT robust to intensity difference. However, its value will also mutate due to local changes in images, and the rotation invariance is slightly poor. The image pair's PMOMs are shown at the bottom of Fig.\ref{fig:maps}. Compared with MIM, the proposed PMOM is not only more robust and stable between multi-modal images, but also continuous in value, which is conducive to achieving effective rotation invariance. In HLMO, PMOM is used as the unique feature information to extract local features of keypoints that are invariant to multi-modal properties.


\subsubsection{Descriptor extraction}

After determining the feature points and the specific feature for discrimination, the next step is to make use of the local feature information around each point and generate descriptors. Gradient Location and Orientation Histogram (GLOH) has shown excellent ability through experiments \cite{mikolajczyk2005performance}, and has been applied in multi-source remote sensing image registration \cite{dellinger2014sar,ma2016remote}. The original GLOH descriptor is a circular region divided by three circles, similar to that shown in Fig.\ref{fig:des180}, in which the two outer circular regions are divided into 4 parts, and the radius of the circular region divided are 5, 9, and 11. The partition size and the number are then improved \cite{mikolajczyk2005performance,dellinger2014sar,ma2016remote}. However, different parameters have various effects when treating multifarious types of images. In addition, if the number of outer ring regions is too small, the character of feature points is not significant, which makes it difficult to match accurately. If it is too large, the features are unstable, and the dimension of the descriptor is too high, which increases the burden of redundant calculation. To deal with this, a generalized GLOH-like (GGLOH) descriptor is proposed, and its structure of which is shown in Fig.\ref{fig:des:a}.

\begin{figure}[!h]
    \centering
        \subfloat[]{\includegraphics[width=2.2in]{Descriptor_Structure_a.png}%
        \label{fig:des:a}}
    \hfil
        \subfloat[]{\includegraphics[width=1.1in]{Descriptor_Structure_b.png}%
        \label{fig:des:b}}
    \caption{Descriptor structure of the proposed GGLOH. (a) Subregion partition of the local neighborhood of a feature point. (b) Angle quantification within each subregion.}
    \label{fig:des}
\end{figure}

Let ${{\rm{A}}^0}$ denote the central circular region, and ${\rm{A}}_j^i,(i=1,2, j=1,...,{N_{\rm{A}}})$ represent the sector subregion $j$ in the outer ring region $i$. Let ${N_{\rm{A}}}$ be the number of the subregions in each out ring region, which is even, ${\theta _{\rm{0}}}$ be the main orientation of the feature point, and $R_0$, $R_1$, $R_2$ be the radii of the central and outer regions, respectively. Note that the orientations of pixels’ gradient in each region are counted as feature information, therefore, fair use of information in each region is expected. The number of pixels in each region should be roughly the same, and the weight of the outer regions should not change due to the change of ${N_{\rm{A}}}$. So the area of each region should be the same, that is
\begin{equation} \label{eqggloh}
{N_{\rm{A}}} \cdot \pi {\rm{R}}_{\rm{0}}^2 = \pi ({\rm{R}}_{\rm{1}}^2 - {\rm{R}}_{\rm{0}}^2) = \pi ({\rm{R}}_{\rm{2}}^2 - {\rm{R}}_{\rm{1}}^2)
\end{equation}
which also fixes the relationship between $R_0$, $R_1$, $R_2$ and ${N_{\rm{A}}}$. When ${N_{\rm{A}}}$ is given different values, the stability and importance of each region’s feature remains the same. In HLMO, the GGLOH is used to extract local features on the PMOM, where the orientation values within $(-\frac{\pi}{2},\frac{\pi}{2})$ are uniformly quantified to ${N_{\rm{O}}}$ values, as shown in Fig.\ref{fig:des:b}, where ${\phi_k}(k=1,2,...,{N_{\rm{O}}})$ is the angle after quantization. A histogram with ${N_{\rm{O}}}$ bins is obtained in each region.

It is simple to achieve rotation invariance of HLMO. For each keypoint, the PMOM value at its location is the main orientation, that is, the reference orientation ${\theta _0}$ of the GGLOH. Then, all of the PMOM values within the local area of GGLOH also take ${\theta _0}$ as the reference (0°), that is, all angle values minus ${\theta _0}$, and those beyond $(-\frac{\pi}{2},\frac{\pi}{2})$ are flipped to their opposite angles.

\begin{figure}[h!]
 \begin{center}
  \includegraphics[width=3.3in]{Problem_of_180.pdf}
  \caption{The problem caused by the jump of main orientation near $-\frac{\pi}{2}$ or $\frac{\pi}{2}$.}
  \label{fig:des180}
 \end{center}
\end{figure}

Another key problem is that the rotation and nonlinear intensity difference may cause the jump of the main orientations of some feature points near $-\frac{\pi}{2}$ and $\frac{\pi}{2}$. An example is shown in Fig.\ref{fig:des180}. In PIIFD \cite{chen2010partial}, a similar problem has been discovered and improvement has been made for SIFT. Then a similar strategy is adopted to process GLOH-like descriptors,

\begin{equation}
{{\bf{D}}_1} = \left[ {\begin{array}{*{20}{c}}
{\begin{array}{*{20}{c}}
{{\bf{H}}_1^1}&{{\bf{H}}_2^1}& \cdots &{{\bf{H}}_{{{{N_{\rm{A}}}} \mathord{\left/
 {\vphantom {{{N_{\rm{A}}}} 2}} \right.
 \kern-\nulldelimiterspace} 2}}^1}
\end{array}}\\
{\begin{array}{*{20}{c}}
{{\bf{H}}_1^2}&{{\bf{H}}_2^2}& \cdots &{{\bf{H}}_{{{{N_{\rm{A}}}} \mathord{\left/
 {\vphantom {{{N_{\rm{A}}}} 2}} \right.
 \kern-\nulldelimiterspace} 2}}^2}
\end{array}}
\end{array}} \right]
\end{equation}

\begin{equation}
{{\bf{D}}_2} = \left[ {\begin{array}{*{20}{c}}
{\begin{array}{*{20}{c}}
{{\bf{H}}_{{{{N_{\rm{A}}}} \mathord{\left/
 {\vphantom {{{N_{\rm{A}}}} 2}} \right.
 \kern-\nulldelimiterspace} 2} + 1}^1}&{{\bf{H}}_{{{{N_{\rm{A}}}} \mathord{\left/
 {\vphantom {{{N_{\rm{A}}}} 2}} \right.
 \kern-\nulldelimiterspace} 2} + 2}^1}& \cdots &{{\bf{H}}_{{N_{\rm{A}}}}^1}
\end{array}}\\
{\begin{array}{*{20}{c}}
{{\bf{H}}_{{{{N_{\rm{A}}}} \mathord{\left/
 {\vphantom {{{N_{\rm{A}}}} 2}} \right.
 \kern-\nulldelimiterspace} 2} + 1}^2}&{{\bf{H}}_{{{{N_{\rm{A}}}} \mathord{\left/
 {\vphantom {{{N_{\rm{A}}}} 2}} \right.
 \kern-\nulldelimiterspace} 2} + 2}^2}& \cdots &{{\bf{H}}_{{N_{\rm{A}}}}^2}
\end{array}}
\end{array}} \right]
\end{equation}

\begin{equation} \label{eqdes}
{\bf{D}} = \left[ {\begin{array}{*{20}{c}}
{{{\bf{D}}_1} + {{\bf{D}}_2}}\\
{c\left| {{{\bf{D}}_1} - {{\bf{D}}_2}} \right|}
\end{array}} \right]
\end{equation}
where ${\bf{H}}_j^i$ is the histogram vector of gradient orientation of region ${\rm{A}}_j^i$. In this way, no matter whether the main orientation of feature points is reversed 180° or not, descriptor ${\bf{D}}$ is composed of the addition and subtraction of the upper and lower parts of GGLOH according to the main orientation axis, without changing the regions' order. Finally, a descriptor vector ${\bf{D}}_P$ is generated for the feature point $P$, whose dimension is $(2 \cdot {N_{\rm{A}}}+1) \cdot {N_{\rm{O}}}$.

%\subsection{Advanced Outlier Removal}
%\label{ssec:subhead}


\subsection{Multi-scale Registration Strategy}
\label{ssec:subhead}
Scale difference ${{\cal F}_{{\rm{Sc}}}}( \bullet )$ of multi-source images has a great influence on local features. Some algorithms have quantitative scale judging methods, such as SIFT \cite{lowe2004distinctive} and LHOPC. However, it is found that these methods are invalid in images with large modal differences. The reason is that when images do not belong to the same degradation model, it is not credible to judge the scale quantitatively through local image feature information. Multi-source images often have scale differences, and sometimes the scale proportion is unknown. In order to deal with this key problem and realize scale robustness, a multi-scale feature extraction and matching strategy is designed in MS-HLMO.
%由于传感器成像能力或成像条件的差异,图像显示出了尺度的差异。这种尺度差异往往体现在两个方面,一个是分辨率,一个是模糊程度。其中分辨率是指图像中一个像素对应实际空间的尺寸,另外一个是指

\begin{figure}[h!]
 \begin{center}
  \includegraphics[width=2.5in]{Pyramid.pdf}
  \caption{Structure of the Gaussian pyramid used in MS-HLMO.}
  \label{fig:pyramid}
 \end{center}
\end{figure}

Local information of feature points is extracted in the scale-space of the images. Based on the scale-space theory \cite{lindeberg1994scale}, the method of building image's Gaussian pyramids is adopted. The schematic diagram of establishing Gaussian pyramid of the image in the proposed algorithm is shown in Fig.\ref{fig:pyramid}. The original image is first sampled down step by step to obtain a series of images with different resolutions, that is, the first layer in each octave. Then in each octave, a series of Gaussian blurs are performed:
\begin{equation}\label{eqGauss1}
{\bf{L}} = {\bf{G}} * {\bf{I}}
\end{equation}
\begin{equation}\label{eqGauss2}
{\bf{G}} = \frac{1}{{\sqrt {2\pi {\sigma ^2}} }}{e^{\frac{{-({x^2} + {y^2})}}{{2{\sigma ^2}}}}}
\end{equation}
where ${\bf{I}}$ is the original image, ${\bf{G}}$ is a Gaussian kernel with a standard deviation of $\sigma$, and ${\bf{L}}$ is the Gaussian blur image.
%高斯尺度空间模拟出了图像尺度变化的效果,为获取图像多个尺度层面的特征提供了重要帮助。

\begin{algorithm}[htp]
\caption{\label{multi1} \footnotesize Proposed MS-HLMO Feature Extraction}
\begin{algorithmic}
\footnotesize
\STATE {\bf Input:} single-band image $\mathbf{I}$, feature point set $P_{\mathbf{I}}$, total number of octaves ${N_{\rm{GO}}}$ and layers ${N_{\rm{GL}}}$ in Gaussian pyramid, subregion and angle partition parameters ${N_{\rm{A}}}$, ${N_{\rm{O}}}$ in GGLOH , patch size $S$ of HLMO.
\STATE Through down-sampling and Eq.(\ref{eqGauss1})(\ref{eqGauss2}), the Gaussian pyramid ${\bf{G}}_{\bf{I}}(O,L)$ of image $\mathbf{I}$ is built with ${N_{\rm{GO}}}$ octaves and ${N_{\rm{GL}}}$ layers in each octave.
\STATE In each layer of ${\bf{G}}_{\bf{I}}(O,L)$:
\STATE \hspace*{0.1in}Calculate the PMOM of this layer to get ${\bf{F}}_{\bf{I}}(O,L)$ according to Eq.(\ref{eqPMOM1})(\ref{eqPMOM7})
\STATE \hspace*{0.1in}For each feature point $P$ in $P_{\bf{I}}$:
\STATE \hspace*{0.2in}Calculate the corresponding position
\STATE \hspace*{0.2in}Take the PMOM value at the position as the main orientation ${\theta_0}$
\STATE \hspace*{0.2in}Taking the main orientation as the reference direction ($0^{\circ}$), establish a GGLOH window with size (diameter) of S
\STATE \hspace*{0.2in}Statistics the PMOM value within each region of GGLOH to obtain the basic feature descriptor $D_{1}(P,O,L)$ and $D_{2}(P,O,L)$
\STATE \hspace*{0.2in}Obtain the descriptor $D(P,O,L)$ of $P$ with Eq.\ref{eqdes}.
\STATE {\bf Output:} feature descriptor set $D_{P_{\bf{I}}}(O,L)$
\end{algorithmic}
\end{algorithm}

After the scale-space of the images is established, for each Harris corner point, the HLMO descriptor is calculated by obtaining the local information at the corresponding location of each feature point in the scale-space. The proposed multi-scale HLMO feature extraction method is provided in Algorithm 1, where $O$ is the octave number in the Gaussian pyramid, and $L$ is the layer number. The algorithm outputs the feature point descriptor set $D_{P_{\bf{I}}}(O,L)$, which contains $(2 \cdot {N_{\rm{A}}}+1) \cdot {N_{\rm{O}}}$-dimensional vectors for each feature point at each scale.

The next step is to match the feature point sets of the image pair according to the descriptor sets. The process of the multi-scale feature matching is provided in Algorithm 2. In the process, each scale is matched in turn. Then the matching results are merged step by step while the outlier removal is carried out to realize the optimization of matching points. The final matching results are used to determine the spatial transformation between images. The most critical is to combine all the matching results of feature points and remove outliers, so as to maximize the correct matches of all scales. Fig.\ref{fig:pyramids} shows this process visually. Obviously, this is a general approach to handle all kinds of unknown scale differences. When the scale proportion of images is known or can be estimated, then the above process can be greatly simplified. In this case, the proposed multi-scale strategy still has the advantages of enhancing feature matching and maximizing the number of matching points.

\begin{algorithm}[htp]
\caption{\label{multi2} \footnotesize Proposed MS-HLMO Feature Matching}
\begin{algorithmic}
\footnotesize
\STATE {\bf Input:} feature point set of the image pair $P_{{\bf{I}}1}$, $P_{{\bf{I}}2}$, feature descriptor set of the image pair $D_{P_{{\bf{I}}1}}(O_{1},L_{1})$, $D_{P_{{\bf{I}}2}}(O_{2},L_{2})$
\STATE Take each layer of $D_{P_{{\bf{I}}1}}(O_{1},L_{1})$:
\STATE \hspace*{0.1in}Take each layer of $D_{P_{{\bf{I}}2}}(O_{2},L_{2})$:
\STATE \hspace*{0.2in}Match $P_{{\bf{I}}1}$ and $P_{{\bf{I}}2}$ using Euclidean distance of the descriptorss
\STATE \hspace*{0.2in}Remove outliers, producing the matching result of a single scale $M(O_{1},O_{2},L_{1},L_{2})$
\STATE The matching results of all layers in each octave of the scale-space are union and then optimized with outlier removal, producing the matching result $M_{L}(O_{1},O_{2})$
\STATE The matching results of all octaves in $M_{L}(O_{1},O_{2})$ are union and then optimized with outlier removal, producing the final matching result $M_{OL}(P_{{\bf{I}}1},P_{{\bf{I}}2})$
\STATE {\bf Output:} feature points matching set $M_{OL}(P_{{\bf{I}}1},P_{{\bf{I}}2})$
\end{algorithmic}
\end{algorithm}

\begin{figure}[h!]
 \begin{center}
  \includegraphics[width=3.5in]{MS_Matching.pdf}
  \caption{Multi-scale keypoints matching strategy in MS-HLMO.}
  \label{fig:pyramids}
 \end{center}
\end{figure}
%\include{Theory} 
%\documentclass[a4paper,twoside]{article}

\usepackage{epsfig}
\usepackage{subfigure}
\usepackage{calc}
\usepackage{amssymb}
\usepackage{amstext}
\usepackage{amsmath}
\usepackage{amsthm}
\usepackage{comment}
\usepackage{bm}
\usepackage{multirow}
\usepackage{cite}
\usepackage{multicol}
\usepackage{pslatex}
\DeclareMathOperator*{\argmax}{argmax}
\usepackage{apalike}
\usepackage{SCITEPRESS}     % Please add other packages that you may need BEFORE the SCITEPRESS.sty package.

\subfigtopskip=0pt
\subfigcapskip=0pt
\subfigbottomskip=0pt

\begin{document}

\title{Efficient and Effective Single-Document Summarizations and A Word-Embedding Measurement of Quality}


\author{\authorname{Liqun Shao, Hao Zhang, Ming Jia and Jie Wang}
\affiliation{Department of Computer Science, University of Massachusetts, Lowell, MA, USA}
\email{\{Liqun\_Shao, Hao\_Zhang, Ming\_Jia\}@student.uml.edu, Jie\_Wang@uml.edu}
}

\keywords{Single-Document Summarizations, Keyword Ranking, Topic Clustering, Word Embedding, SoftPlus Function, Semantic Similarity, Summarization Evaluation, Realtime.}

\abstract{
Our task is to generate an effective summary for a given document with specific realtime requirements. %In our applications there is no labeled data available for training a model.
%We study how to extract efficiently a summary from a single document within a given length boundary to capture the main topics of the document and be human readable.
%Following the common approach to extract multiple key sentences as coherent as possible and cover the major topics of the original document
%as diverse as possible,
We use the softplus function %$\ln(1+e^x)$
to enhance keyword rankings to favor important sentences,
%that are more important.
based on which we present a number of summarization algorithms using various
keyword extraction and topic clustering methods. We show %called ERAKE and ERAKE-TT % using RAKE keyword extractions and topic clusterings
%two algorithms called RAKEN and RAKETT %using word-pair co-occurrences to identify keyword phrases and TextTiling to promote coverage of diverse topics,
that our algorithms meet the realtime requirements and yield the best ROUGE recall scores on DUC-02 over all previously-known
algorithms. % are better than the previously-known best algorithm.
%For instance, our algorithm ET3Rank generates the scores of (49.2, 25.6, 27.5) for (ROUGE-1, ROUGE-2, ROUGE-SU4),
%
%
%the (ROUGE-1, ROUGE-2, ROUGE-SU4) scores over DUC-02 of our algorithm ET3Rank are (49.2, 25.6, 27.5),
%while the previously-reported best results are (49.0, 24.7, 25.8).
%
%In particular, both RANEN and RAKETT
%have higher ROUGE scores over DUC-02 and meet the realtime requirement, with RAKETT slightly better on ROUGE scores
%and RAKEN faster in runtime.
%The ROUGE measure requires summaries written by human experts as benchmarks for comparisons by NLP algorithms.
%This presents an obstacle when dealing with a large number of documents of various topics and lengths, for it is infeasible to produce a sufficient number %of benchmarks.
%To overcome this barrier,
To evaluate the quality of summaries %against the original documents
without human-generated benchmarks, % of particular topics,
we define a measure called WESM based on word-embedding using %a word-embedding method to measure the quality of a summary. In particular, we
%to use
Word Mover's Distance.
% over paragraphs, denoted by WESM,
%to measure similarities between a summary and its original document.
%and
We show that
%the best and the second best algorithms remain the same under both average ROUGE and WESM over different datasets, and
the orderings of the ROUGE and WESM scores of our algorithms are highly comparable, %, with the average $L_1$-norm equal to 0.4. %Moreover, we show that
%the orderings over DUC-02 and NewsIR-16 are also comparable, with the average $L_1$-norm equal to 2.
%where
%over DUC-02 have a similar ordering of the WEP scores as that of the ROUGE scores. %We argue that WEP may serve as a viable alternative
%for evaluating the quality of a summary.
%We show that under WEP, these algorithms have almost the same orderings over the NewsIR-16 and DUC-02 datasets.
%that ofsimilar toand
%Moreover, the best algorithm  also ranks the highest under WEP over both NewsIR-16 and DUC-02.
suggesting that WESM may serve as a viable alternative for measuring the quality of a summary.
%over the NewsIR-16 dataset,
% is similar to the order of ROGUE rankings produced by these algorithms over the DUC 2012 dataset.
%in place of ROGUE when human-generated benchmarks are not available for comparisons.
%Our WMD method obtains competitive performance and our summarization system outperforms prior work on ROUGE.
}

\onecolumn \maketitle \normalsize \vfill

\section{\uppercase{Introduction}}
\label{sec:introduction}
\noindent Text summarization algorithms have been studied intensively and extensively.
%For practical purposes,
An effective summary must
be human-readable and %is considered effective if it
convey the central meanings of the original document within
a given length boundary. % and is human readable.
% over the last several decades.
%A number of unsupervised and supervised algorithms have been devised to achieve these goals.
%To capture the key points of a given document and achieve coherence,
The common approach of unsupervised summarization algorithms
extracts sentences based on importance rankings (e.g., see \cite{DUC02,Mihalcea04,Rose:10,lin:11,ParveenR015}),
where a keyword may also
be a phrase. %Sentence extraction in the original order of the document also produces reasonable coherence. % is a sequence of one or more words that provides a compact representation of the content in the document.
A sentence with a larger number of keywords of higher ranking scores is considered more important %a candidate
for extraction.
Supervised algorithms include
%Attempts were also made to use
CNN and RNN models % in recent years
for generating extractive and abstractive summaries (e.g., see \cite{RushCW15,NallapatiZSGX16,ChengL16a}). %List the 3 papers published in 2016
%These algorithms
%depend on large sets of labeled data to train models, and their ROUGE scores over DUC data are yet to catch up with unsupervised algorithms.
%Significant advancements were made in recent years in supervised and unsupervised summarization algorithms
%over a single document, multiple documents, and queries, including CNNs \cite{}
%
%An adequate summary for a given document should succinctly convey the central meanings for the document. Summarization systems generate a concise document as %output and take a long document as input. Several summarization variants
%

%In a project on
We were asked to construct a general-purpose text-automation tool to produce, among other things, an effective
summary for a given document with the following realtime requirements: Generate a summary instantly for a document of
up to 2,000 words, under 1 second for
a document of slightly over 5,000 words, and under 3 seconds for a very long document of around 10,000 words.
Moreover, we need to deal with documents of arbitrary topics without knowing what the topics are in advance.
%and we would like to measure the quality of a summary directly against the original document without
After investigating all existing summarization algorithms,
we conclude that unsupervised single-document summarization algorithms would be the best approach to meeting
our requirements. %We show that by modifying the common approach of sentence extractions, we can achieve both efficiency and effectiveness.
%learning is our only choice.

%Methods can be characterized as supervised and unsupervised, and summaries may be obtained over a single document, multiple documents, and based on queries.
%can be query based, multi document, and single document.
%For the above different summarization types, the basic requirements remain the same, which contain salient information and let readers not miss anything from the original document. Readers are not %interested in redundant information, so summaries should cover many diverse topics. Finally, summaries should represent main information, be concise and cover different topics.





\begin{comment}
Recent methods used sentence compression to convert a sentence into a shorter sentence or phrase, while trying to maintain syntactic correctness. For example, Turner and Charniak ~\shortcite{Turner:05} used a language model to trim sentences. Vandegehinste and
Pan ~\shortcite{Vandeghinste:04} used context-free grammar (CFG) trees to compress a sentence. However, CFG is ambiguous and constructing CFG trees' complexity is high. Dependency trees are widely developed because they offer better syntactic representations of sentences ~\cite{kahane:12}. To increase the expressive capacity of our model, we apply compression of the selected individual sentence from a syntactic aspect with the dependency tree compression model ~\cite{Shao:16}. This model defines the set of empirical rules to specify what can and cannot be trimmed. These rules were formed based on experiences from working with a large number of text documents. It showed that this model can be used to generate titles which have higher F1 scores than those generated by the previous methods.
\end{comment}

We use topic clusterings to obtain a good topic coverage in the summary when extracting key sentences.
%we may simply treat each paragraph as a separate topic. For a long document, however, a topic may spread out among several paragraphs and so a topic clustering method would be needed. %are often not %independent with each other.
%Based on
%We achieve this using topic clusterings to
In particular, we first determine which topic a sentence belongs to,
and then %by computing the probability.
%We then
extract key sentences to cover as many topics as possible within the given length boundary.
%In particular, we choose sentences of higher scores so that they belong to as many topics as possible %from each topics
%until the summary covers all the topics or reaches the size bound.

Human judgement is the best evaluation of the quality of a summarization algorithm. %For example, %either abstractive or extractive,
It is a standard practice to run an algorithm over DUC data and
compute the ROUGE recall scores with a set of DUC benchmarks,
%The ROUGE measure requires summaries
which are human-generated summaries for articles of a moderate size.
DUC-02 \cite{DUC02}, in particular, is a small set of benchmarks for single-document summarizations.
% as benchmarks for comparisons by NLP algorithms.
When dealing with a large number of documents of unknown topics and various sizes, human judgement may be
impractical, and so
we would like to have an alternative mechanism of measurement %suitable for evaluating the quality of a summary
without human involvement.
%this approach presents a barrier, for it infeasible to produce a sufficient number of benchmarks.
%To solve this issue
%we would like a new measure to compute how similar.
Ideally, this mechanism should preserve the same ordering as %able to distinguish better summaries and are comparable with
ROUGE over DUC data; namely,
if $S_1$ and $S_2$ are two summaries of the same DUC document %in DUC
produced by two algorithms,
and the ROUGE score of $S_1$ is higher than that of $S_2$, then it should also be the case under the new measure.
%The new measure is then referred to as ROUGE-compatible.
%To overcome the obstacle when no benchmarks are available to evaluate summaries for documents of particular topics,
%we define a measure based on a word-embedding method to measure the quality of a summary. In particular, we
%to use

%For this purpose, we devise WESM (Word-Embedding Similarity Measure)
%based on Word Mover's Distance (WMD) \cite{wmd}
%to measure word-embedding similarity of the summary and the original document.
Louis and Nenkova \cite{Louis:2009} devised an unsupervised method to
evaluate summarization without human models using
common similarity measures: Kullback-Leibler divergence, Jensen-Shannon divergence, and  cosine similarity.
%to measure similarities between a summary and its original document.
%
%The reason we choose WMD as a baseline similarity measurement is its capability to deal with abstractive summaries.
%
%that human-generated benchmarks
%are abstractive, and
These measures, %similarity measurements such as
%cosine similarity and
as well as the information-theoretic similarity measure \cite{Aslam03},
are meant to measure lexical similarities, which are unsuitable for measuring semantic similarities.

Word embeddings such as Word2Vec can
be used to fill this void %measure similarities from semantic aspects
and
we devise WESM (Word-Embedding Similarity Measure)
based on Word Mover's Distance (WMD) \cite{wmd}
to measure word-embedding similarity of the summary and the original document. WESM is meant to evaluate summaries for new datasets when no human-generated benchmarks are available. WESM has an advantage that it can measure the semantic similarity of documents. We show that WESM correlates well with ROUGE on DUC-02. Thus, WESM may be used as an alternative summarization evaluation method when benchmarks are unavailable.
%we choose WMD to measure semantic similarities. %as a summarization evaluation method.
%are not able to capture abstraction the similarity between an abstractive summary and the original document.
%This means that we would need to consider semantic similarity and context similarity, in addition to lexical similarity.
%WMD is the first successful attempt in this direction.

%that ofsimilar toand
%Moreover, RAKETT also ranks the highest under WEP over both DUC-02 and NewsIR-16.
%over the NewsIR-16 dataset,
% is similar to the order of ROGUE rankings produced by these algorithms over the DUC 2012 dataset.

%in place of ROGUE when human-generated benchmarks are not available for comparisons.
%Our WMD method obtains competitive performance and our summarization system outperforms prior work on ROUGE.

The major contributions of this paper are summarized below:

\vspace*{-3pt}
\begin{enumerate}
\item We present a number of summarization algorithms using topic clustering methods and enhanced keyword rankings by the softplus function,
%over keyword rankings by various
%keyword extraction algorithms and topic clustering methods,
and show that they meet the realtime requirements and outperform all the previously-known summarization algorithms under
the ROUGE measures over DUC-02.
%For instance, our algorithm ET3Rank generates the scores of (49.2, 25.6, 27.5) for (ROUGE-1, ROUGE-2, ROUGE-SU4),
%
%
%the (ROUGE-1, ROUGE-2, ROUGE-SU4) scores over DUC-02 of our algorithm ET3Rank are (49.2, 25.6, 27.5),
%while the previously-reported best results are (49.0, 24.7, 25.8) \cite{parveen2016}.

%over the DUC-02 dataset under ROUGE and they meet the realtime requirement.
%For instance, under the common measures of ROUGE-1, ROUGE-2, and ROUGE-SU4, our ET3Rank algorithm has scores of (49.2, 25.6, 27.5) over DUC-02, while
%the previously-known best scores are (48.1, 24.3, 24.2), generated by Tgraph \cite{ParveenR015}.
%ERAKE (Enhanced Rapid Automatic Keyword Extraction) and
%ERAKE-TT (ERAKE-TextTiling) %an algorithm using word-pair co-occurrences to identify keyword phrases and TextTiling to promote coverage of diverse topics,
%is the state of the art for single-document summarizations.  % superior to an extensive list of summarization algorithms.
%In particular, we show %evaluate the quality and runtime of an extensive list of summarization algorithms and show
%that both ERAKE and ERAKE-TT
%with ERAKE-TT better on ROUGE scores and ERAKE faster on runtime. For example,
%For a long document of 5,000 words, ET3Rank generates a summary of 30\% length in
%ess than 1 second on an average computer.

%combine the above approaches to generate a summary from two aspects, which are central meaning representation and topic diversity coverage with sentence %compression using the dependency tree compression model.
\vspace*{-3pt}
\item We propose a new mechanism WESM as
%
%show that these summarization algorithms
%over DUC-02 have a similar ordering of the WEP scores as that of the ROUGE scores.
%In particular, we show that under WEP, these algorithms have almost the same orderings over the DUC-02 and NewsIR-16 \cite{Signal1M2016} datasets.
%Thus, WEP may serve as
%we may use WEP as a viable
an alternative measurement of summary quality when human-generated benchmarks are unavailable.
%
%\item We de
%
%We test our unsupervised summary approach on datasets from NewsIR-16 ~\cite{Signal1M2016} and Document Understanding Evaluations 2002 ~\cite{DUC:02}. %Experimental results show that our approach outperforms the state-of-the-art results on DUC 2002 with ROUGE scores ~\cite{rouge}.
%\item We first use a novel word embedding method by Word Mover's Distance (WMD) ~\cite{wmd} to evaluate the quality of summaries without reference summaries. %Our experiments show that our approach performs comparable to ROUGE scores.
\end{enumerate}

The rest of the paper is organized as follows:
We survey in Section \ref{sect:work} unsupervised single-document summarization algorithms.
We present in Section \ref{sect:smethod} the details of our summarization algorithms and describe WESM in Section \ref{sect:emethod}. % the  automatic summarization evaluation methods.
We report the results of extensive experiments in Section \ref{sec:experiments} and conclude the paper in
%reports the experimental results of our approaches. Conclusions are given in
Section \ref{sect:conclusion}.

\section{\uppercase{Early Work}}
\label{sect:work}

\noindent Early work on single-topic summarizations can be described in the following three categories: keyword extractions, coverage and diversity optimizations,
and topic clusterings.

\subsection{Keyword extractions}

To identify keywords in a document over a corpus of documents, the measure of term-frequency-inverse-document-frequency (TF-IDF) \cite{salton87} is often used.
%  over
%which is %takes advantage of The preassumption for using TF-IDF is that
%a reasonably sized corpus of documents.
%However, TF-IDF has a shortcoming.
%Otherwise the TF-IDF value of a keyword may just be the term frequency of the keyword in the document.
%To identify keywords in a single document without using
When document corpora are unavailable,
the measure of word co-occurrences (WCO) can produce a %identify keywords with %\cite{Matsuo:03}. %which applies to a single document without a corpus.
comparable performance to TF-IDF over a large corpus of documents \cite{Matsuo:03}.
The methods of TextRank \cite{Mihalcea04} and RAKE (Rapid Automatic Keyword Extraction) \cite{Rose:10}
further refine the WCO method from different perspectives, %. We note that both TextRank and RAKE
which are also sufficiently fast to become candidates %for keyword extractions
for meeting the realtime requirements. %In particular,
%are unsupervised keyword extraction methods with weights.

TextRank %is a graph-based ranking model %for text processing and it can obtain competitive results with state-of-the-art systems developed in these areas.
%Graph-based ranking algorithms are a way of
%for deciding
computes the rank of a word in an undirected, weighted word-graph using a slightly modified PageRank algorithm \cite{Brin:98}.
To construct a word-graph for a given document, first remove stop words and represent each remaining word
as a node, then link two words if they both appear in a sliding window of a small size. Finally,
assign the number of co-occurrences of the endpoints of an edge as a weight to the edge.
%which are based on global information recursively drawn from the entire graph.
%This graph-based ranking model is that of ��voting�� or ��recommendation��. The higher the number of votes for a vertex, the higher the importance of the vertex.

RAKE %assigns a ranking to each word using word pair co-occurrences: It
first removes stop words using a stoplist, and then generates words (including phrases) using a set of word delimiters and %generates
%keyword phrases using
a set of phrase delimiters.
For each remaining word $w$, the degree of $w$ is the frequency of $w$ plus the number of co-occurrences of consecutive word pairs $ww'$ and $w''w$ in the document, where $w'$ and $w''$ are
remaining words.
The score of $w$ is the degree of $w$ divided by the frequency of $w$. We note that
the quality of RAKE also depends on a properly-chosen stoplist, which is language dependent.
%may be easier to determine for some languages and
%harder for other languages.
%The score of a keyword phrase is the summation of
%the scores of the words in the phrase.
%
%. RAKE is superior to WCO because it is simpler and achieves a higher precision rate. We use TextRank and RAKE as our keyword extraction methods for sentence ranking.
%RAKE is a linear-time algorithm, while the runtime of TextRank depends on
%the speed of convergence. Experimental results


\subsection{Coverage and diversity optimization}
%Our work is inspired by the concept of a class of submodular functions for document summarization ~\cite{lin:11}.
%Compared to all the other known extractive summarization algorithms known at the time it was published, Tgraph offered the best ROUGE scores on DUC-02.

 %, Lin and Bilmes \cite{lin:11}
The general framework of selecting sentences gives rise to optimization problems with objective functions
being monotone submodular \cite{lin:11} to promote coverage and diversity.
Among them is an objective function in the form of $L(S) + \lambda R(S)$ with
%
%that is the sum of
%two components.
%$S$ is
a summary $S$ and a coefficient $\lambda \geq 0$, where
$L(S)$ measures the coverage of the summary and $R(S)$ rewards diversity.
We use SubmodularF to denote the algorithm computing this objective function.
%and the algorithm computing it to be described later as SubmodularF.
SubmodularF uses TF-IDF values of words in sentences to compute the cosine similarity of two sentences.
% on the generalized TF-IDF values of the words contained in the respective sentence,
%where the TF value of a word is at the sentence level while the IDF value is still at the document level over a corpus.
%and the IDF value is for the entire corpus.
%The ranking of a sentence $s$ in a document
%is the similarity of $s$ and the document.
%$L(S)$ is measured by the total ranking of the sentences in $S$ with a parameterized upper bound different for each sentence.
%$R(S)$ is measured by first partitioning the document (often using its paragraphs as partitions)
%and then summing up, for each partition, the square root of the total average ranking for each sentence in the partition that is also in $S$.
While
it is NP-hard to maximize a submodular objective function subject to a summary length constraint, the submodularity allows
a greedy approximation with a proven approximation ratio of $1-1/\sqrt{e}$.
%
%Given a document $D$ in a corpus, let $S$ be a set of sentences extracted from
%$${\cal F}(S) = {\cal L}(S) + \lambda{\calR}(S),$$
%where ${\cal L}(S)$
%
%that
%This method
%encourages the summary to be representative of the corpus and positively rewards diversity. They used monotone nondecreasing and submodular functions which is %an efficient scalable greedy optimization scheme.
%However, they

SubmodularF needs labeled data to train the parameters in the objective function to achieve a better summary and it is intended to work on multiple-document summarizations.
While it is possible to work on a single document without a corpus, we note that the greedy algorithm has at least a quadratic-time complexity and it produces a summary with
low ROUGE scores over DUC-02 (see Section \ref{sec:other}),
and so it would not be a good candidate to meet our needs. This also applies to a generalized objective function
%
%
%its ROUGE scores over DUC-02 are much lower than those of RAKETT (see Section \ref{sec:experiments}).
%SubmodularF also incurs much longer time that does not meet the realtime requirement.
%
%The submodularity approach was generalized to include dispersion where the objective
%function
consisting of a submodular component and a non-submodular component \cite{DasguptaKR13}.
%While this generalization provides a slightly better ROUGE-1 score over DUC-04 than SubmodularF, it incurs
%a higher time complexity.

\subsection{Topic clusterings}

Two unsupervised approaches to topic clusterings for a given document have been investigated.
One is TextTiling \cite{hearst:97} and the other is LDA (Latent Dirichlet Allocation) \cite{blei:03}.
TextTiling %is a lighter-weight method,
%Another method we use for topic coverage is TextTiling ~\cite{hearst:97}.
%TextTiling is a technique for
%which
represents a topic as a set of consecutive paragraphs in the document.
It merges adjacent paragraphs that belong to the same topic. TextTiling identifies major topic-shifts based on patterns of lexical co-occurrences and distributions.
LDA computes for each word a distribution under a pre-determined number of topics. %is a heavy-weight method, which
%and represents a topic % would be better
%to identify the topics contained in the document,
%where each topic is represented
%as a set of keywords with larger probabilities. It needs to predetermine the number of topics contained in a document corpus.
LDA is a computation-heavy algorithm
that incurs a runtime too high to meet our realtime requirements.
TextTiling has a time complexity of almost linear, which meets the requirements of efficiency. % and runs much faster than LDA.
%Sentences from the same subtopic won't be selected unless all the subtopics have been covered to ensure the topic coverage.
%We note that LDA incurs a time complexity too high to meet the requirement of realtime summarizations.

\subsection{Other algorithms}
\label{sec:other}

Following the general framework of selecting sentences to meet the requirements of topic coverage and diversity,
a number of unsupervised single-document summarization algorithms have been devised.
The most notable is $CP_3$ \cite{parveen2016}, which produces the best ROUGE-1 (R-1), ROUGE-2 (R-2), and ROUGE-SU4 (R-SU4) scores on DUC-02
among all early algorithms,
including
%is an extractive single-document summarization algorithm devised recently.
%is the most recent algorithm
%, which
%offers the better ROUGE scores over DUC-02 compared to an extensive list of algorithms,
%including
Lead \cite{ParveenR015}, DUC-02 Best, TextRank, LREG \cite{ChengL16a},
Mead \cite{radev2004}, $ILP_{phrase}$ \cite{woodsend2010}, URANK \cite{wan2010}, UniformLink \cite{WanX10}, Egraph + Coherence \cite{Parveen015},
Tgraph + Coherence (Topical Coherence for Graph-based Extractive Summarization) \cite{ParveenR015},
NN-SE \cite{ChengL16a}, and SubmodularF.

$CP_3$ maximizes importance, non-redundancy, and pattern-based coherence of sentences to generate a coherent summary using ILP. 
It computes %considers
the ranks of selected sentences for the summary by the Hubs and Authorities algorithm (HITS) \cite{kleinberg1999}, %(will add a cite).
%Non-redundancy represents if the summary
and ensures that each selected sentence has unique information. % in every sentence.
It then uses mined patterns to extract sentences if the connectivity among nodes in the projection graph matches the connectivity among nodes in a coherence pattern. Because of space limitation, we omit the descriptions of the other algorithms.

Table \ref{CP3} shows the comparison results, where the results for SubmodularF
is obtained using
%To compare with SubmodularF,
%we use
the best parameters trained on DUC-03 \cite{lin:11}.
%%%table to be inserted here
%Our computation show that the ROUGE-1, ROUGE-2, and ROUGE-Su4 scores for SubmodularF are (39.6, 16.9, 17.8), which are
%much lower than those of $CP_3$.
%$CP_3$ \cite{parveen2016} is the most recent algorithm that has the best results reported over DUC-02 so far.
Thus, to demonstrate the effectiveness of our algorithms, we will compare our algorithms with only $CP_3$ %under these common ROUGE measures
over DUC-02.
\begin{table}[h]
\begin{center}
\caption{\label{CP3} ROUGE scores (\%) on DUC-02 data.}
\begin{tabular}{l|c|c|c}
\hline
\bf Methods & \bf R-1 & \bf R-2 & \bf R-SU4 \\
\hline
Lead & 45.9 & 18.0 & 20.1 \\
DUC 2002 Best & 48.0 & 22.8 & \\
TextRank & 47.0 & 19.5 & 21.7 \\
LREG & 43.8 & 20.7 & \\
Mead & 44.5 & 20.0 & 21.0 \\
$ILP_{phrase}$ & 45.4 & 21.3 & \\
URANK & 48.5 & 21.5 & \\
UniformLink & 47.1 & 20.1 &   \\
Egraph + Coh. & 48.5 & 23.0 & 25.3  \\
Tgraph + Coh. & 48.1 & 24.3 & 24.2 \\
NN-SE & 47.4 & 23.0 & \\
SubmodularF & 39.6 & 16.9 & 17.8 \\
$\bm{CP_3}$& \bf 49.0 & \bf 24.7 & \bf 25.8  \\
\hline
\end{tabular}

\end{center}
\end{table}

Solving ILP, however, is
time consuming even on documents of a moderate size, for ILP is NP-hard.
%
%to compute the optimization problem, and ILP is NP-hard.
Thus, $CP_3$ does not meet the requirements of time efficiency. We will need to investigate new methods.

%Important function
 %It uses coherence patterns with 3 nodes.
%
\begin{comment}
Tgraph is based on a weighted graphical representation of documents using LDA topic modeling. It %Tgraph + Coherence
uses ILP to optimize the objectives of sentence importance, coherence, and non-redundancy simultaneously. $CP_3$ optimizes Tgraph % Tgraph has considered simultaneously
using Mixed Integer Programming.
\end{comment}
%Because $CP_3$ offers %and Tgraph are the algorithms with
%the best ROUGE scores and the space is limited,
%
%mentioned above are omitted because of space limitation.
%We will be only
%and it suffices to compare our algorithms with only $CP_3$ under the common ROUGE measures over DUC-02 to demonstrate the effectiveness of our algorithms. %with $CP_3$. % and Tgraph.
%Tgraph, however, incurs much higher time complexity because of LDA (see Section \ref{sec:experiments}).
%, which does not need the realtime requirement.
%Moreover, we will show in Section \ref{sec:experiments} that its ROUGE scores on DUC-02 is not as good as RAKETT.
%Our method is superior %resulting in higher ROUGE scores and our summary evaluation methods leading to higher WMD scores on single documents without development sets.

%Another approach for extractive single-document summarization is called

%
%compared ROUGE scores with state-of-the-art results on DUC-02 data. Our experiments show that our method has higher ROUGE scores than Tgraph and other state-of-the-art methods such as TextRank,
%are other and so on using the same data. We also obtain comparable results from our summary evaluation methods by WMD.

\begin{comment}
ROUGE has been widely accepted for the evaluation of summaries. It includes measures to automatically determine the quality of a summary by comparing it to ideal summaries created by humans. These measures count the number of overlapping units between summary to be evaluated and the ideal summaries created by humans. However, the performance of the ROUGE method fully depends on the quality of reference summaries made by humans. Thus, it cannot ensure similarity between the original document and the summary to be evaluated. ROUGE does not take topic diversity coverage into account. WMD is a novel distance function between text documents. This work is based on word embeddings that learn semantically meaningful representations for words from local co-occurrences in setences. We used WMD to measure the similarity to the original document and take topic diversity coverage into consideration by comparing different paragraphs.
\end{comment}


\section{\uppercase{Our Methods}}

\label{sect:smethod}

\begin{comment}
\subsection{Sentence Ranking}
\label{ssect:sr}
The summary should contain only important sentences. We find all the keywords with positive numerical scores made from RAKE ~\cite{Rose:10} and TextRank  ~\cite{Mihalcea:04} and give a score to every sentence based the number of keywords it contains. Following Shao and Wang ~\shortcite{Shao:16}, we use their central sentence extraction algorithm for ranking sentences by importance.
Assume that a sentence contains $n$ keywords $w_1, \cdots, w_n$, and $w_i$ has a positive score $s_i$.

We use the power of 2 to amplify the differences of the rankings and the ranking score of the sentence is as follows:
\begin{equation} \label{rank1}
\mbox{Rank} = \sum_{i=1}^{n}2^{s_{i}}
\end{equation}

After we calculate the ranking scores of all the sentences, we can order the sentences by their scores. We try to pick as many sentences with high scores as possible. From high to low scoring sentences, we put each sentence into the summary based on different topic diversity coverage constraints in Section \ref{ssect:tdc}.


\subsection{Compression Constrains}
\label{ssect:cc}
\begin{figure}[h]
\includegraphics[width=3in]{fig1_1.png}
\includegraphics[width=3in]{fig1_2.png}
%\DeclareGraphicsExtensions.
\caption{The grammatical relations and trimmed dependency tree using DTCM for sentence ``Market concerns about the deficit has hit the greenback''}%: $\Lambda$ restricts the $\theta$ to the labels a document belongs to}
\label{fig:fig1}
\end{figure}

Our goal is to be able to compress sentences so we can pack more information into a summary. We use the Dependency tree compression model (DTCM) in Shao and Wang ~\shortcite{Shao:16} to do compressions. A dependency tree can provide relations between each word in the sentence. We use the Standord Dependency Parser (SDP) \footnote{\tt http://nlp.stanford.edu/software/\\ \-\hspace{.75cm} stanford-dependencies.shtml} to obtain grammatical relations between words in a sentence. This model follows a set of empirical rules to specify what can or cannot be trimmed. DTCM can delete all the unnecessary branches from sematic aspects. Figure \ref{fig:fig1}shows one example sentence ``Market concerns about the deficit has hit the greenback'' . The compressed sentence is ``Market concerns about deficit hit greenback''.
\end{comment}

%\subsection{Topic Diversity Coverage Constrains}
%\label{ssect:tdc}
\noindent We use TextRank and RAKE to obtain initial ranking scores of keywords, and
%In general, TextRank ranking scores are small while RAKE ranking scores are much larger.
use the softplus function \cite{softplus}
\begin{equation}\label{eq1}
sp(x) = \ln (1+e^x)
\end{equation}
to
enhance keyword rankings to favor sentences that are more important. % that are more important.
%enhance the rankings of sentences
%that are more important.

\subsection{Softplus ranking}

%as compared to the direct summation of the ranking scores of the keywords contained in the sentence.
Assume that after filtering, a sentence $s$ consists of $k$ keywords $w_1, \cdots, w_k$, and $w_i$ has a ranking score $r_i$
produced by TextRank or RAKE. 
Following Shao and Wang ~\cite{Shao:16}, we use their central sentence extraction algorithm for ranking sentences by importance as $\mbox{Rank}(s)$.
We can rank $s$ using one of the following two methods:
\begin{equation}\label{eq2}
{Rank}(s) = \sum_{i=1}^{k} r_i
\end{equation}
\begin{equation}\label{eq3}
{Rank}_{sp}(s) = \sum_{i=1}^{k} sp(r_i)
\end{equation}
\begin{comment}
\[
\mbox{Rank}(s) = \sum_{i=1}^{k} r_i;~~
\mbox{Rank}_{sp}(s) = \sum_{i=1}^{k} sp(r_i).
\]
\end{comment}
%A straightforward ranking of $s$ is to sum up the scores of the keywords contained in it; that is,
%$$\mbox{Rank}(s) = \sum_{i=1}^{k} r_i.$$
%We use the softplus function to enhance the ranking of $s$ as follows:
%$$\mbox{Rank}_{sp}(s) = \sum_{i=1}^{k} sp(r_i).$$
%We use DTRank (Direct TextRank) and DRAKE (Direct RAKE)
%to refer to this method of computing sentence ranking.
Let DTRank (Direct TextRank) and ETRank (Enhanced TextRank) denote
the methods of ranking sentences using, respectively,
$\mbox{Rank}(s)$ and $\mbox{Rank}_{sp}(s)$ over TextRank keyword rankings,
and
%
%and DRAKE (Direct RAKE) denote
%the method of ranking sentences using $\mbox{Rank}(s)$
%over TextRank and RAKE keyword rankings, respectively;
%and
DRAKE (Direct RAKE) and ERAKE (Enhanced RAKE)
%ETRank (Enhanced TextRank) and ERAKE (Enhanced RAKE)
to denote the methods of ranking sentences
using, respectively, $\mbox{Rank}(s)$ and $\mbox{Rank}_{sp}(s)$ over RAKE keyword rankings.

%To enhance the ranking of a sentence that is more important,
%We use the softplus function to enhance the ranking of $s$ as follows:
%$$\mbox{Rank}_{sp}(s) = \sum_{i=1}^{k} sp(r_i).$$
%
%
%If $r_i$ is produced by TextRank, and respectively by RAKE, then the ranking score of $s$ is denoted by
%Then the softplus ranking of $s$ is computed by % $\mbox{Rank}(s)$ and $\mbox{Rank}_R(s)$, which are computed by
%\begin{eqnarray*}
%$$\mbox{Rank}(s) = \sum_{i=1}^{k} \ln(1+e^{r_i}).$$
%$\mbox{Rank}_R(s) = \sum_{i=1}^{k} 2^{r_i}$.
%\end{eqnarray*}
%We refer to TextRank and RAKE using this measure of sentence ranking as ETRank (Enhanced TextRank) and ERAKE (Enhanced RAKE).
%We use ETRank (Enhanced TextRank) and ERAKE (Enhanced RAKE) to denote this method of computing sentence ranking based
%on TextRank and RAKE keyword rankings, respectively.

The softplus function is helpful because %enhance the ranking of a more important sentence,
%let $sp(x) = \ln (1+e^x)$. We note that
%we note that
when $x$ is a small positive number, $sp(x)$ increases the value of $x$ significantly (see Figure \ref{fig:softplus})
and when $x$ is large, $sp(x) \approx x$.
\begin{figure}[h]
\centering
\includegraphics[width=2.5in]{softplus1.png}
%\DeclareGraphicsExtensions.
\caption{Softplus function $\ln(1+e^x)$.} %: $\Lambda$ restricts the $\theta$ to the labels a document belongs to}
\label{fig:softplus}
\end{figure}
In particular,
given two sentences $s_1$ and $s_2$, suppose that $s_1$ has a few keywords with high rankings and the rest of the keywords with low rankings,
while $s_2$ has medium rankings for almost all the keywords. In this case, we would consider $s_1$ more important than $s_2$. However,
%because the keywords in $s_1$ gave low ranking scores,
we may end up with $\mbox{Rank}(s_1) < \mbox{Rank}(s_2)$.
To illustrate this using a numerical example, assume that $s_1$ and $s_2$ each consists of 5 keywords, with
original scores (sc) and softplus scores (sp) given in the following table \ref{example}.

\medskip

%The ranking of $s_1$ can be enhanced after using softplus. For example, the following two sentences are extracted from a news article using TextRank to rank keywords, with
%$s_1$ being more important than $s_2$ because $s_1$ specifies the name, strength, and direction of the hurricane, as well as the name of the place it is expected to hit.
%\begin{itemize}
%\item[$s_1$:] {\sf Hurricane Gilbert is heading toward Jamaica with 100 mph winds.}
%\item[$s_2$:] {\sf A hurricane warning has been issued for the island.}
%\end{itemize}
%The ranking score of each keyword is depicted in Table \ref{table:example}, from which we
%can see that $s_2$ is selected without using softplus. After using softplus, $s_1$ is selected as it should be.

\noindent
\begin{table}[h]
\begin{center}
\caption{\label{example} Numerical examples with given sc and sp scores.}
\begin{tabular}{l||c|c|c|c|c||c}
\hline
$s_1$ & $w_{11}$ & $w_{12}$ & $w_{13}$ & $w_{14}$ & $w_{15}$&   Rank \\
\hline
sc & 2.6 & 2.2 & 2.1 & 0.3 & 0.2 & 7.4\\
sp & 2.67 & 2.31 & 2.22 & 0.85& 0.80 & \bf 8.84\\
\hline
$s_2$ & $w_{21}$ & $w_{22}$ & $w_{23}$ & $w_{24}$ & $w_{25}$ & \\
\hline
sc & 1.6 & 1.5 & 1.5 & 1.5 & 1.4  & \bf 7.5  \\
sp & 1.78& 1.70& 1.70& 1.70& 1.62&  8.51 \\
\hline
\end{tabular}
\end{center}
\end{table}

\medskip
Sentence $s_1$ is more important than $s_2$
because it contains three keywords of much higher ranking scores than those of $s_2$.
However, $s_2$ will be selected without using softplus. After using softplus, $s_1$ is selected as it should be.

\begin{comment}
For example, Hurricane Gilbert is heading toward Jamaica with 100 mph winds.
����1��(0.7, 0.6, 0.01, 0.02, 0.03)
softplus:(1.10318604889, 1.03748795049, 0.698159680508, 0.703197179727, 0.708259676341)
softplusֵ�ĺͣ�4.25029053595
Ȩ�غͣ�1.36


A hurricane warning has been issued for the island.
����2��(0.2, 0.2, 0.3, 0.3, 0.4)
softplus:(0.798138869382, 0.798138869382, 0.854355244469, 0.854355244469, 0.9130152524)
softplusֵ�ĺͣ�4.2180034801
Ȩ�غͣ�1.4
\end{comment}

%\medskip
For a real-life example, consider the following two sentences from an article in DUC-02:
\vspace*{-1pt}
\begin{itemize}
\item[$s_1$:] {\small\sf Hurricane Gilbert swept toward Jamaica yesterday with 100-mile-an-hour winds, and officials issued warnings to residents on the southern coasts of the Dominican Republic, Haiti and Cuba.}
\vspace*{-1pt}\item[$s_2$:] {\small\sf Forecasters said the hurricane was gaining strength as it passed over the ocean and would dump heavy rain on the Dominican Republic and Haiti as it moved south of Hispaniola, the Caribbean island they share, and headed west.}
\end{itemize}
We consider $s_1$ more important as it
specifies the name, strength, and direction of the hurricane, the places affected, and the official warnings.
Using TextRank to compute
keyword scores, we have $\mbox{Rank}(s_1) = 1.538 < \mbox{Rank}(s_2) = 1.603$, which returns a less important sentence $s_2$. After computing
softplus,
we have $\mbox{Rank}_{sp}(s_1) = 8.430 > \mbox{Rank}_{sp}(s_2) = 7.773$; the more important sentence $s_1$ is selected.

Note that not any exponential function would do the trick. What we want is a function to return roughly the same value as the input when the input is large, and a significantly larger value than the input when the input is much less than 1. The softplus function meets this requirement.

\begin{comment}
(u'Forecasters said the hurricane was gaining strength as it passed over the ocean and would dump heavy rain on the Dominican Republic and Haiti as it moved south of Hispaniola, the Caribbean island they share, and headed west.', [1.6033724386659647, u'hurricane#caribbean island#south#republic#forecasters#haiti#west#headed#heavy rain#dominican#', '0.376361942128#0.222770119935#0.152622310017#0.144567553334#0.140516484532#0.140516484532#0.117560427986#0.117560427986#0.11182141385#0.0790752743651#'])

(u' Hurricane Gilbert swept toward Jamaica yesterday with 100-mile-an-hour winds, and officials issued warnings to residents on the southern coasts of the Dominican Republic, Haiti and Cuba.', [1.538394147633235, u'hurricane#south#coast#republic#jamaica yesterday#haiti#southern coasts#mile#winds#dominican#warnings#', '0.376361942128#0.152622310017#0.152622310017#0.144567553334#0.142542018933#0.140516484532#0.117560427986#0.0824985459553#0.0790752743651#0.0790752743651#0.0709520059994#'])
assume that sentence
$s$ consists of five keywords: $(w_1, w_2, w_3, w_4, w_5)$ whose corresponding ranking scores
are
%How we select sentences is critical. In general,
We would want to select sentences with
ranking scores as high as possible, while avoiding selections of multiple sentences from one topic
and no sentence at all from another topic.
\end{comment}


\subsection{Topic clustering schemes}

We consider four topic clustering schemes: TCS, TCP, TCTT, and TCLDA.
%The first one is called naive TCDS, denoted by
\begin{enumerate}
\item TCS %(TC based on sentences)
selects sentences without checking topics.
\item TCP %(TC based on paragraphs)
treats each paragraph as a separate topic.
%sentences based on their scores and also ensure the selected sentences distribute different paragraphs.
%For example, we first select a sentence with the highest ranking score (break ties arbitrarily) and place the paragraph number that contains this
%sentence into KTS. We then select the second highest score sentence. If the second sentence is not in the KTS, we put the second one into the summary and the %paragraph number into the KTS. Otherwise, we skip it to the next sentence. The following parts are the same as the basic procedure of TDCC.
\item TCTT %(TC based on TextTiling)
%The third TDCC is called TextTiling TDCC (t-TDCC). TextTiling ~\cite{hearst:97} is used to
partitions a document into a set of multi-paragraph %topical
segments using TextTiling.
%It may merge or divide paragraphs into several topics. In t-TDCC, we try not to pick sentences from the same topic. The procedure is similar to the basic TDCC.
\item TCLDA %(TC based on LDA)
%The last TDCC is called LDA TDCC (l-TDCC). LDA ~\cite{blei:03} is a model for topic modeling where topic probabilities are assigned words in documents.
computes a topic distribution for each word using LDA. %To use LDA we must first fix the number of topics for a document corpus.
%The probabilities can be sued to measure the semantic relatedness between words. The number of topics in a document should be set by us.
We set the number of topics from 5 to 8 depending on the length of the document.
Assume that a document contains $K$ topics ($5 \leq K \leq 8$) and the topic $j$ consists of $k_j$ words $w_{1j}, \cdots, w_{k_j,j}$, where $1 \leq j \leq K$
and
$w_{ij}$ has a probability $p_{ij} > 0$. For a document with $n$ sentences $s_1, \cdots, s_n$,
%if $s_z$ does not contain word $w_{ij}$, then the probability $p_{ij}$ in the topic $j$ is set to $1$, otherwise is $p_{ij}$.
we use the following maximization to determine which topic $t_z$ the sentence $s_z$ belongs to ($1 \leq t \leq K$):
\begin{equation} \label{eq4}
t_z = \argmax_{1 \leq j\leq k}\bigg(\prod_{i:w_{ij} \in s_z} {p_{ij}}\bigg)
\end{equation}
\end{enumerate}




\begin{comment}
Figure \ref{fig:fig3} shows the bipartite topical graph of TTCD, where $\bm{w}_j = (w_{1i}, \cdots, w_{k_j,j})$ is the vector words under topic $j$
($1 \leq j \leq m$), $s_z$ ($1 \leq z \leq n$) is a sentence in a document, $p_j$ ($1 \leq j \leq m$) is the product of possibilities in $wi$. $tz$ is the topic for the sentence $sz$ with the highest possibility. From equation \ref{topic}, we know that each sentence belongs to certain topic. Then we can use the basic TDCC to ensure the summary cover topics as many as possible.

 \begin{figure}[h]
\includegraphics[width=3in]{fig3.png}
%\DeclareGraphicsExtensions.
\caption{Bipartite topical graph of TTCD}%: $\Lambda$ restricts the $\theta$ to the labels a document belongs to}
\label{fig:fig3}
\end{figure}
\end{comment}

\subsection{Summarization algorithms}
\label{ssect:sa}

%The goal of topic diversity coverage constraints (TDCC) is to avoid sentences coming from the same topic, which means to make the summary cover as many topics %as possible.
The length of a summary may be specified by users,
either as a number of words or as a percentage of the number of characters of the original document.
By a ``30\% summary'' we mean that the number of characters of the summary does not exceed 30\% of that of the original document.

Let $L$ be the summary length (the total number of characters) specified by the user and $S$ a summary.
If $S$ consist of $m$ sentences
%Assume that a document contains $m$ topics and $n$ sentences
$s_1, \cdots, s_m$, and the number of characters of $s_i$ is $\ell_i$, then the following inequality must hold:
$\sum_{i=1}^m \ell_i \leq L.$
%The summary must satisfy additional length requirement:
%\begin{equation} \label{length}
%\mbox{Len(summary)} \geq \sum_{i=1}^{n}{l_{i}}
%\end{equation}

Depending on which sentence-ranking algorithm and which topic-clustering scheme to use, we have eight combinations
using ETRank and ERAKE, and eight combinations using DTRank and DRAKE, shown in Table \ref{algorithm_names}.
%
%ESTRank, EPTRank, ET3Rank, ELDATRank when using ETRank to compute sentence rankings;
%ESRAKE, EPRAKE, ET2RAKE, and ELDARAKE when using ERAKE to compute sentence rankings. %extract keywords.
For example, ET3Rank (Enhanced TextTiling TRank) means to use $\mbox{Rank}_{sp}(s)$ to rank sentences and
TextTiling to compute topic clusterings, and
%
%Likewise, using DTRank and DRAKE we have
%the following combinations: STRank, PTRank, T3Rank, LDATRank,
%SRAKE, PRAKE, T2RAKE, LDARAKE. For example,
T2RAKE (TextTiling RAKE)
means to use $\mbox{Rank}(s)$ rank sentences over RAKE keywords and TextTiling
to compute topic clusterings. %The description of all the algorithms are shown in Table \ref{algorithm_names}.

\begin{table}[h]
\begin{center}
\caption{\label{algorithm_names} Description of all the Algorithms with different sentence-ranking (S-R) and topic-clustering (T-C) schemes.}
\begin{tabular}{l|c|c}
\hline
\bf Methods & \bf S-R & \bf T-C \\
\hline
ESTRank & ETRank & TCS \\
EPTRank & ETRank & TCP \\
ET3Rank & ETRank & TCTT \\
ELDATRank & ETRank & TCLDA \\
\hline
ESRAKE & ERAKE & TCS \\
EPRAKE & ERAKE & TCP \\
ET2RAKE & ERAKE & TCTT \\
ELDARAKE & ERAKE & TCLDA \\
\hline
STRank & DTRank & TCS \\
PTRank & DTRank & TCP \\
T3Rank & DTRank & TCTT \\
LDATRank & DTRank & TCLDA \\
\hline
SRAKE & DRAKE & TCS \\
PRAKE & DRAKE & TCP \\
T2RAKE & DRAKE & TCTT \\
LDARAKE & DRAKE & TCLDA \\
\hline
\end{tabular}
\end{center}
\end{table}

All algorithms follow the following procedure for selecting sentences:

\vspace*{-2pt}
\begin{enumerate}
\item Preprocessing phase
\begin{enumerate}
\vspace*{-1pt}\item Identify keywords and compute the ranking of each keyword.
\vspace*{-1pt}\item Compute the ranking of each sentence.
\end{enumerate}
\vspace*{-2mm}
\item Sentence selection phase
\begin{enumerate}
\vspace*{-1pt}\item Sort the sentences in descending order of their ranking scores. % according to the underlying keyword extraction algorithm.

\vspace*{-1pt}\item Select sentences one at a time with a higher score to a lower score.
Check if the selected sentence $s$ belongs to the known-topic set (KTS) according to the underlying
topic clustering scheme, where KTS is a set of topics from sentences placed in the summary so far. If $s$ is in KTS, then discard it; otherwise, place $s$ into the summary and its topic into KTS.
\vspace*{-1pt}\item Continue this procedure until the summary reaches its length constraint.

\vspace*{-1pt}\item If the number of topics contained in the KTS is equal to the number of topics in the document,
% (assume a document contains $m$ topics), we
empty KTS and repeat the procedure from Step 1. %to empty and select the unpicked sentences from the start.
\end{enumerate}
\end{enumerate}

%Given a keyword extraction algorithm and a topic clustering scheme, it is straightforward to replace them in the algorithm. For example,
%ET3Rank use ETank to compute the ranking of each sentence and use TextTiling for topic clustering.

Figure \ref{fig:fig5} shows an example of 30\% summary generated by ET3Rank on an article in NewsIR-16.
\begin{figure*}[t]
\includegraphics[width=6in]{fig5.png}
%\DeclareGraphicsExtensions.
\caption{An example of 30\% summary of an article in NewsIR-16 by ET3Rank, where the
original document is on the left and the summary is on the right.}
\label{fig:fig5}
\end{figure*}


\begin{comment}
Figure \ref{fig:fig2} depicts this procedure.

\begin{figure}[h]
\includegraphics[width=3in]{fig2.png}
%\DeclareGraphicsExtensions.
\caption{The general procedure of extracting sentences}%: $\Lambda$ restricts the $\theta$ to the labels a document belongs to}
\label{fig:fig2}
\end{figure}

Our basic procedure of selecting sentences is as follows: (1) Sort the sentences in descending order of their ranking scores. (2) Select sentences one at a time with a higher score to a lower score. (3) Check if the selected sentence $s$ belongs to the known topic set (KTS), which is a set of topics from sentences placed in the summary so far. If $s$ is in KTS, then discard it; otherwise, place $s$ into the summary and its topic into KTS. (4) Continue this procedure until the summary reaches its length constraint. (5) If the number of topics contained in the KTS is equal to the number of topics in the document,
% (assume a document contains $m$ topics), we
empty KTS and repeat the procedure from Step 1. %to empty and select the unpicked sentences from the start.
Figure \ref{fig:fig2} depicts this procedure. % basic procedure of TDCC.


In this section, we introduce our four different summarization algorithms based on the above TDCC. We name summarization methods as n-SA, p-SA, t-SA and l-SA based on n-TDCC, p-TDCC, t-TDCC and l-TDCC. The only difference between these four algorithms is used different TDCC and other parts are the same. For utilizing TextRank or RAKE as the sentence ranking method, we can divide four summarization algorithms into eight, which are TextRankN, RAKEN, TextRankP, RAKEP, t-SA-T, RAKETT, TextRankLDA and RAKELDA. We describe our algorithm in general as follows:
\begin{itemize}
\item Get keywords set with weights from TextRank or RAKE.
\item Split the original text into sentences.
\item Find all the keywords in each sentence and use sentence ranking algorithm in Section \ref{ssect:sr} to get the sentence score.
\item Sort the sentences by their scores.
\item Iterate the sentences from high to low score and select the sentences based on TDCC in Section \ref{ssect:tdc} until it reaches the limitation of the summary.
\item Compress the selected sentences by algorithm in Section \ref{ssect:cc}.
\item Order the selected trimmed sentences by their order in the original text to generate the final summary.
\end{itemize}
\end{comment}
\section{\uppercase{A Word-Embedding Measurement of Quality}}
\label{sect:emethod}

%\subsection{Word2Vec Embedding}
%\label{ssec:Word2Vec}
\noindent Word2vec \cite{mikolov13,mikolov2013}
%introduced a novel word-embedding procedure Word2Vec. Their model
is an NN model that learns a vector representation for each word contained in a corpus of documents.
%In addition, they used the skip-gram model with neural network architecture.
The model consists of an input layer, a projection layer, and an output layer to predict nearby words in the context. In particular,
a sequence of $T$ words $w_1, \cdots, w_T$ are used to train a Word2Vec model for maximizing the probability of neighboring words:
\begin{equation} \label{eq5}
{\frac{1}{T}\sum_{t=1}^{T}{\sum_{j\in b(t)}{\log p(w_j|w_t)}}}
\end{equation}
where $b(t) = [t-c, t + c]$ is the set of center word $w_t$'s neighboring words, $c$ is the size of the training context, and $p(w_j|w_t)$ is defined by the softmax function.
%(See more details in ~\cite{mikolov2013}.)
Word2Vec can learn complex word relationships if it trains on a very large data set.
%A commonly cited example is that vec(king) + vec(man) - vec(woman) $\approx$ vec(queen).

\subsection{Word Mover's Distance}
\label{ssec:wmd}
Word Mover's Distance (WMD) \cite{wmd} uses Word2Vec as a word embedding representation method.
It measures the dissimilarity between two documents and calculates the minimum cumulative distance to ``travel'' from the embedded words of one document to the other. Although two documents may not share any words in common, WMD can still measure the semantical similarity by considering their word embeddings, while other bag-of-words or TF-IDF methods only measure the similarity by the appearance of words. A smaller value of WMD indicates that the two sentences are more similar.

\begin{comment}
 \begin{figure}[h]
\includegraphics[width=3in]{fig4.png}
%\DeclareGraphicsExtensions.
\caption{The WMD metric on two sentences S1, S2 compared with the query sentence Q \cite{wmd}}%: $\Lambda$ restricts the $\theta$ to the labels a document belongs to}
\label{fig:fig4}
\end{figure}
Figure \ref{fig:fig4}, extracted from \cite{wmd}, shows an example of the WMD metric on two sentences $S_1$, $S_2$ compared with the query sentence $Q$.
The arrows represent flow between two words, which are labeled with their distance costs.
For each sentence, stop words are removed. Comparing the main components one by one and adding their distance contributions together, we obtain the WMD of two sentences. The lower the value of WMD, the more similar two sentences are. Intuitively, traveling from $Illinois$ to $Chicago$ is much closer than is $Japan$ to $Chicago$. Also, Word2Vec embedding generates the vector vec ($Illinois$) closer vec ($Chicago$) than vec ($Japan$). Thus, sentence $S_1$ to query $Q$ is more similar than sentence $S_2$ to query $Q$.
\end{comment}

\subsection{A word-embedding similarity measure}

Based on WMD's ability of measuring the semantic similarity of documents, %we design innovative methods to evaluate the quality of the summary.
we propose a summarization evaluation measure WESM (Word-Embedding Similarity Measure). %One is based on the original document, denoted by WED and the other is based on paragraphs, denoted by WEP.
Given two documents $D_1$ and $D_2$, let $\mbox{WMD}(D_1,D_2)$ denote
the distance of $D_1$ and $D_2$.
Given a document $D$, assume that it consists of $\ell$ paragraphs $P_1, \cdots, P_\ell$.
Let $S$ be a summary of $D$.
We compare the word-embedding similarity of a summary $S$ with $D$ using WESM$(S,D)$ as follows:
\begin{equation} \label{eq6}
\mbox{WESM}(S,D) = \frac{1}{\ell}\sum_{i=1}^{\ell} \frac{1}{1 + \mbox{WMD}(S,P_i)}
\end{equation}
%$s$ is system summary and $t$ is original text. $wmd(s, t)$ is the WMD distance between the system summary and the original text, and the result we can obtain from Section \ref{ssec:wmd}.
The value of WESM$(S,D)$ %(Word-Embedding Document comparison) %we can learn that the range of WMD-o valu
is between 0 and 1. Under this measure, the higher the WESM$(S,D)$ value, the more similar $S$ is to $D$. %the original text.

%\paragraph{WEP similarities}

%WED takes semantic aspects into account.
 %We calculate the sum of WED scores with each paragraph of the original text compared with the system summary and then take the average of the sum.
%Assume that a document $D$ consists of $\ell$ paragraphs $P_1, \cdots, P_\ell$. %$t$ is the original text.
%Define WEP$(S,D)$ as follows:
% \begin{equation*} \label{wmd-p}
%\mbox{WEP}(S,D) = \frac{\sum_{i=1}^{\ell}\mbox{WED}(S,P_i)}{\ell}.
%
%{\frac{1}{1+\mbox{WMD}(S,P_{i})}}}{n}
%\end{equation*}
%From the above formula, we can conclude that the range of the value of WEP is from 0 to 1.
%The value of WEP (Word-Embedding Paragraph comparison) %we can learn that the range of WMD-o valu
%is between 0 and 1. The higher the WEP value, the more similar the summary is with the original text.
%We experimented on WMD-o and WMD-p with Word2Vec using different training sets and use them to evaluate eight summarization methods we mentioned in Section %\ref{ssect:sa} compared with ROUGE score %in Section \ref{sec:experiments}.
%Note that WEP measures %we also think of
%the influence of topic diversity to a summary.

\begin{comment}
\subsection{Absolute vs. relative measurements}
We note that different training sets may cause WMD to produce different ranges of values, and
so unless we fix a standard dataset to train Word2Vec,
we do not have a standard base for evaluating the absolute qualities of summaries using WESM. %by different training sets.
However, we can use it to evaluate relative qualities by
the score orderings.
%compare various summarizations over the same training set by looking at the ordering of their values.
%because the relative rank of summarization evaluated by WED does not change.

%On the other hand, we note that  trained on different datasets preserve similar orderings (see Sections 5.2 and 5.3), which provides
%a level of assurance of robustness.
%
% we may train WED on one dataset and use it to measure summaries produced by different algorithms on a different dataset;
%regardless what the training dataset is used, the ordering of WED scores remains relatively the same (see analysis in Section \ref{}).
\end{comment}

\section{\uppercase{Numerical Analysis}}
\label{sec:experiments}

%We describe the datasets in our experiments in the next section. %and report empirical results in this section.
%
%\subsection{Datasets}

\noindent We evaluate the qualities of summarizations using the DUC-02 dataset \cite{DUC02}
and the NewsIR-16 dataset \cite{Signal1M2016}.
DUC-02 consists of 60 reference sets, each of which consists of a number of documents, single-document summary benchmarks, and multi-document abstracts/extracts.
The common ROUGE recall measures of ROUGE-1, ROUGE-2, and ROUGE-SU4 are used to compare the quality of summarization algorithms over DUC data.
%By using DUC-02 data, we can evaluate our methods with other state-of-the-art summarization methods using ROUGE.
%DUC-02 contains single-document summary benchmarks.
NewsIR-16 consists of 1 million articles %that are mainly
from English news media sites and blogs.
%The average length of an article is 405 words.

We use various software packages to implement TextRank (with window size = 2) \cite{TRurl}, RAKE \cite{RAurl}, TexTiling \cite{TTurl}, LDA and Word2Vec \cite{gen}.
% from \url{https://github.com/summanlp/textrank},
%\url{https://github.com/aneesha/RAKE},
%url{https://pypi.python.org/pypi/nltk},
%and \url{https://pypi.python.org/pypi/gensim}.
%
%LDA��Word2Vec: �õ�gensim python������
%https://pypi.python.org/pypi/gensim
%TextTiling�õ���nltk������
%https://pypi.python.org/pypi/nltk
%
%We generate summaries by our summarization algorithms on NewsIR-16 dataset
%and evaluate
%evaluated by WED and WEP to find the summarization method with the best performance.

We use the existing Word2Vec model trained on English Wikipedia \cite{wiki},
%and the other on the GoogleNews-vectors model\footnote{\tt https://github.com/mmihaltz/Word2Vec-\\ \-\hspace{.75cm} GoogleNews-vectors} by genism.
%The English Wikipedia dataset
which consists of 3.75 million articles formatted in XML. The reason to choose this dataset is
for its large size and the diverse topics it covers.
%and the GoogleNews-vectors is the pre-trained Google News corpus that contains 3 million 300-dimension English word vectors.

%To use the WED and WEP measures, we need to train Word2Vec models or use pre-trained Word2Vec models.
% on different English training datasets.
%to present that
%We note that although the ranges of WMD values may differ on various datasets,
%we can still compare the quality of summaries
%for they present the
%
%. % and acquire comparable results with ROUGE.
%In other words, WED and WEP can evaluate the quality of summarization methods using the same dataset and different datasets can still obtain similar results.
%In particular, we trained our Word2Vec models over English Wikipedia\footnote{\tt https://dumps.wikimedia.org/enwiki/latest\\ \-\hspace{.75cm} /enwiki-latest-pages-articles.xml.bz2}
%and used GoogleNews-vectors model\footnote{\tt https://github.com/mmihaltz/Word2Vec-\\ \-\hspace{.75cm} GoogleNews-vectors} for Word2Vec by genism. English Wikipedia dataset contains 3 million %articles formated in XML. GoogleNews-vectors is the pre-trained Google News corpus (3 billion running words) word vector model, which is 3 million 300-dimension English word vectors.

\subsection{ROUGE evaluations over DUC-02}

%We compare the ROUGE-1, ROUGE-2, and ROUGE-SU4 recall scores for single-document summarizations produced by all algorithms over DUC-02.
As mentioned before, we use %Tgraph and
$CP_3$ to cover all previously known algorithms for the purpose of comparing qualities of summaries, as
%Tgraph produces the better results over these algorithms and
$CP_3$ produces the best results among them. %Since the $CP_3$ paper \cite{Parveen2016} does not cover SubmodularF,
%To compare with SubmodularF,
%we use the best parameters trained on DUC-03 \cite{lin:11}.
%Our computation show that the ROUGE-1, ROUGE-2, and ROUGE-Su4 scores for SubmodularF are (39.6, 16.9, 17.8), which are
%much lower than those of $CP_3$.

Among all the algorithms we devise, we only present those with at least one ROUGE recall score better than or equal to the corresponding score of $CP_3$, %Tgraph,
identified in bold %, where R-1, R-2, and R-SU4 stand for ROUGE-1,
%ROUGE-2, and ROUGE-SU4, respectively
(see Table \ref{duc}).
%The results are shown in %. In particular,
%we compare the ROUGE-1, ROUGE-2, and ROUGE-SU4 scores
Also shown in the table is the average of the three ROUGE scores (R-AVG). We can see that
ET3Rank is the winner, followed by T2RAKE; both are superior to $CP_3$.
%We can also see that all our algorithms
%in Table \ref{duc}
%perform better than Tgraph under ROUGE-1 and ROUGE-SU4.
Moreover, ET2RAKE offers the highest
ROUGE-1 score of 49.3.
%where the results for Lead, DUC-02 Best, TextRank, UniformLink, Egraph + Coh., and Tgraph +
%Coh. are copied from \cite{.}.
%
%with ROUGE to check against the state-of-the-art in Table \ref{duc}.
%\textit{Lead} selects the top five ranking sentences.
%DUC-02 Best is best result reported at DUC-02.
%We also compare with other popular summarization methods \textit{TextRank}, \textit{Submodular function},
%\textit{UniformLink (k=10)}, \textit{Egraph} and \textit{Tgraph (n=2000)}.
%We can see that RAKEN and RAKETT all outperform the rest of the algorithms on
%all of the ROUGE-1, ROUGE-2 and ROUGE-SU4 measures, and RAKETT is better than RAKEN on all of these measures. Moreover,
%RAKELAD also outperforms Tgraph on ROUGE-1 and ROUGE-SU4.
%
%perform better than the well known best systems on DUC-02. It shows that our sentence ranking system using RAKE can produce more informative summaries.
\begin{table}[h]
\begin{center}
\caption{\label{duc} ROUGE scores (\%) on DUC-02 data.}
\begin{tabular}{l|c|c|c||c}
\hline
\bf Methods & \bf R-1 & \bf R-2 & \bf R-SU4 & \bf R-AVG \\
\hline
$CP_3$& 49.0 & 24.7 & 25.8 & 33.17 \\
\hline
\bf ET3Rank & \bf 49.2 & \bf 25.6 & \bf 27.5 & \bf 34.10 \\
ESRAKE & \bf 49.0 & 23.6 & \bf 26.1 &  32.90 \\
%EPRAKE & \bf 48.9 & 22.8 & \bf 25.7  & \bf 32.46\\
ET2RAKE & \bf 49.3 & 21.4 & 24.5  & 31.73\\
%ELDARAKE & \bf 48.3 & 21.8 & \bf 24.5  & 31.53\\
PRAKE & \bf 49.0 & 24.5 & 25.3 & 32.93\\
\bf T2RAKE & \bf 49.1 & \bf 25.4 & \bf 25.8 & \bf 33.43 \\
%LDARAKE & \bf 48.3 & 22.5 & \bf 25.3 & 32.03\\
%SubmodularF & 39.6 & 16.9 & 17.8  \\
%Lead & 45.9 & 18.0 & 20.1  \\
%DUC-02 Best & 48.0 & 22.8 &   \\
%TextRank & 47.0 & 19.5 & 21.7  \\
%UniformLink & 47.1 & 20.1 &   \\
%Egraph + Coh. & 47.9 & 23.8 & 23.0  \\
\hline
%Tgraph + Coh. & 48.1 & 24.3 & 24.2 & 32.20 \\
%$CP_3$& 49.0 & 24.7 & 25.8 & 33.17 \\
%\hline
\end{tabular}
\end{center}
\end{table}

\subsection{WESM evaluations over DUC-02 and NewsIR-16}

%In this section, we generated summaries with 30\% and 50\% of the NewsIR-16 dataset using TextRankN, RAKEN, TextRankP, RAKEP, TextRankTT, RAKETT, TextRankLDA and RAKELDA and evaluated them by WMD-o %and WMD-p.
%We evaluate our algorithms listed
%in Table \ref{duc} using WESM over the Word2Vec model trained on English Wikipedia
%and GoogleNews.
%
Table \ref{results1} shows the evaluation results on DUC-02 and NewsIR-16 using WESM based on the Word2Vec model trained on English Wikipedia.
%and GoogleNews, respectively.
%Under each measure of WED and WEP,
The first number in the third row is the average score on all benchmark summaries in DUC-02.
For the rest of the rows, each number is the average
score of summaries produced by the corresponding algorithm for all documents in DUC-02 and NewsIR-16.
The size constraint of a summary on DUC-02 for each document is the same as that of the corresponding DUC-02 summary benchmark.

For NewsIR-16, we select at random 1,000 documents from NewsIR-16 and remove the title, references, and other unrelated content from each article.
%We then merge several articles at random to generate a new article of different sizes up to 10,000 words in an article.
Based on an observation that a 30\% summary allows for a good summary,
we compute 30\% summaries of these articles using each algorithm.
%and compute the average WED and WEP scores. The results are given in Table \ref{NewsIR-16}.
%
%for each algorithm, the
%average score of summaries for all documents in DUC-02 such that
%the size of the summary for each document is the same as that of the corresponding DUC-02 summary benchmark.
%The right column-hand is the average score

%WMD-o and WMD-p are trained on English Wikipedia.
%As shown in Table \ref{results}, RAKETT has the best WMD-o and WMD-p scores with 30\% and the best WMD-p score with 50\%. The summarization methods based on LDA which are TextRankLDA and RAKELDA %have the worst performance. The results are consistent with human judges. We can also conclude that WMD-p can better evaluate summaries than WMD-o, because RAKETT has a lower WMD-o score but a %higher WMD-p score. From WMD-p, we learn that summaries generated by RAKE based methods can produce better performance.

\begin{table}[h]
\begin{center}
\caption{\label{results1} Scores (\%) over DUC-02  and NewsIR-16 under WESM trained on English-Wikipedia.}
\begin{tabular}{l|c|c}
\hline
%\multirow{2}{*}{\bf Methods} &
%\bf Methods &
%\bf Measures & \multicolumn{2}{c}{\bf WESM EW}  & \multicolumn{2}{|c}{\bf WESM GN} \\
%\cline{2-5}
%Methods & WED & WEP & WED & WEP \\
%\hline
\bf Datasets & \bf DUC-02 & \bf NewsIR-16 \\ % & \bf D02 & \bf NIR \\
%{\bf Methods} & \bf WED & \bf WEP \\
\hline
%\bf Methods &
%\multicolumn{2}{c}{\bf WED}  & \multicolumn{2}{|c}{\bf WEP} \\
%\cline{2-3}
%\bf Methods &
Benchmarks & 3.021  & \\% &67.96 & \\
\hline









ET3Rank &\bf 3.382 &\bf 2.002 \\ % &\bf 69.45      &\bf 17.29 \\
ESRAKE      &3.175 & 1.956  \\ %&69.20      &17.15     \\
%EPRAKE      &3.154 &1.964   \\ %   &69.21     &17.04      \\
ET2RAKE     &3.148 &1.923   \\ %   &69.03      &17.01      \\
%ELDARAKE    &3.149 &1.960   \\ %   & 68.92  &17.02     \\
PRAKE       &3.150 &1.970   \\ %  &68.85      &17.21     \\
T2RAKE      &3.247 &1.990   \\ %  &69.17      &17.19    \\
%LDARAKE     &3.157 &1.914   \\ %   &68.45      &17.19    \\
\hline
\end{tabular}
\end{center}
\end{table}
It is expected that scores of our algorithms are better than the score for benchmarks under each measure, for the benchmarks often use different words not in the original documents, and hence would
have smaller similarities.

%We also note that WESM trained
%on
%To evaluate our algorithms over NewsIR-16 under WED and WEP, we select at random 1,000 documents from NewsIR-16 and remove the title, references, and other unrelated content from each article.
%We then merge several articles at random to generate a new article of different sizes up to 10,000 words in an article.
%Based on an observation that a 30\% summary allows for a good summary,
%we compute 30\% summaries of these articles using each algorithm
%and compute the average WED and WEP scores. The results are given in Table \ref{NewsIR-16}.

%summaries and average the scores
%
%The results shown in Tables \ref{results} and \ref{results1} are the average scores of the scores for each article's summary.

\begin{comment}
\begin{table}[h]
\begin{center}
\begin{tabular}{l|c|c|c|c}
\hline
\multirow{2}{*}{\bf Methods} &
%\bf Methods &
\multicolumn{2}{c}{\bf E-Wikipedia}  & \multicolumn{2}{|c}{\bf GoogleNews} \\
\cline{2-5}
%Methods & WED & WEP & WED & WEP \\
& \bf WED & \bf WEP & \bf WED & \bf WEP \\
\hline
ET3Rank &\bf 5.695  &\bf 2.002  &\bf 23.25  &\bf 11.32 \\
ESRAKE  &5.398      &1.956          &23.12  &11.18 \\
EPRAKE  &5.377      &1.964          &22.86  &11.21 \\
ET2RAKE &5.585      &1.922          &22.85  &11.17 \\
ELDARAKE &5.359     &1.960          &22.96  &11.08 \\
PRAKE   &5.377      &1.970          &23.14  &11.28 \\
T2RAKE &5.688       &1.990          &23.17  &11.20 \\
LDARAKE &5.554      &1.914          &23.18  &11.20 \\
\hline
\end{tabular}
\caption{\label{NewsIR-16} Scores (\%) over NewsIR-16 under WED and WEP trained on English Wikipedia and GoogleNews}
\end{center}
\end{table}
%Figure \ref{fig:fig5} shows a 30\% summary example of an article in the NewsIR-16 dataset by the RAKETT method. In Figure \ref{fig:fig5}, the left side is the content of the article and the right %side is the summary generated by the RAKETT method. The RAKETT method can merge sentences from the same subtopic, let the summary cover as many subtopics as possible and trim sentences from semantic %aspects.
\end{comment}

\subsection{Normalized $L_1$-norm}

We would like to determine if WESM is a viable measure. From our experiments, we know that the all-around best algorithm ET3Rank, the second best
algorithm T2RAKE, and ET2RAKE remain the same positions under R-AVG over DUC-02 and under WESM over both DUC-02 and NewsIR-16 (see Table \ref{ordering}),
ESRAKE and PRAKE remain the same positions under R-AVG over DUC-02 and under WESM over NewsIR-16, while ESRAKE and PRAKE only differ by one place under R-AVG and WESM over DUC-02.
\begin{table}[h]
\begin{center}
\caption{\label{ordering} Orderings of R-AVG scores over DUC-02 and
WESM scores over DUC-02 and NewsIR-16.}
\begin{tabular}{l|c|c|c}
\hline
\multirow{2}{*}{\bf Methods} &
%\bf Methods &
%\bf Measures &
\bf R-AVG  & \multicolumn{2}{|c}{\bf WESM} \\
\cline{2-4}
%Methods
& DUC-02 & DUC-02 & NewsIR-16 \\
\hline
%\bf Datasets & \bf DUC-02 & \bf NewsIR-16 \\ % & \bf D02 & \bf NIR \\
%{\bf Methods} & \bf WED & \bf WEP \\
%\hline
%\bf Methods &
%\multicolumn{2}{c}{\bf WED}  & \multicolumn{2}{|c}{\bf WEP} \\
%\cline{2-3}
%\bf Methods &
%Benchmarks & 4.22  & \\% &67.96 & \\
%\hline
\bf ET3Rank     &\bf 1  &\bf 1 &\bf 1  \\ % &\bf 69.45      &\bf 17.29 \\
ESRAKE          &4  &3 &4  \\ %&69.20      &17.15     \\
%\bf EPRAKE      &\bf 5  &\bf 5 &\bf 4  \\ %   &69.21     &17.04      \\
\bf ET2RAKE     &\bf 5  &\bf 5 &\bf 5  \\ %   &69.03      &17.01      \\
%ELDARAKE      &8  &7 &5   \\ %   & 68.92  &17.02     \\
PRAKE          &3  &4 &3  \\ %  &68.85      &17.21     \\
\bf T2RAKE      &\bf 2  &\bf 2 &\bf 2  \\ %  &69.17      &17.19    \\
%LDARAKE     &6  &4 &8  \\ %   &68.45      &17.19    \\
\hline
& ${\bm O}_1$ & ${\bm O}_2$ & ${\bm O}_3$ \\
\hline
\end{tabular}
\end{center}
\end{table}

Next, we compare the ordering of the R-AVG scores and the WESM scores over DUC-02. For this purpose, we use the normalized $L_1$-norm to compare the distance of two orderings. Let ${\bm X} = (x_1, x_2, \cdots, x_k)$ be a sequence of $k$ objects, where
each $x_i$ has two values $a_i$ and $b_i$ such that
$a_1, a_2, \ldots, a_k$ and $b_1,b_2,\ldots, b_k$ are, respectively, permutations of $1,2,\ldots,k$.
Let
\[
D_k= \sum_{i=1}^k |(k-i+1)-i|,
\]
which is the maximum distance two permutations can possibly have. Then
the normalized $L_1$-norm of ${\bm A} = (a_1, a_2, \cdots, b_k)$ and ${\bm B} = (b_1, b_2, \cdots, b_k)$ is defined by
$$||{\bm A}, {\bm B}||_1 = \frac{1}{D_k}\sum_{i=1}^k |a_i - b_i|.$$
Table \ref{ordering} shows the orderings of the R-AVG scores over DUC-02 and WESM scores over DUC-02 and NewsIR-16 (from Tables \ref{duc} and \ref{results1}).

It is straightforward to see that $D_5 = 12$,
$||{\bm O}_1, {\bm O}_2||_1 = ||{\bm O}_2, {\bm O}_3||_1 = 2/12 = 1/6$ and $||{\bf O}_1,{\bf O}_3||_1 = 0$.
%Note that
%the average $L_1$-norm on $(1,2,\cdots, 5)$ and the revers order is 2.4.
This indicates that WESM and ROUGE are highly comparable over DUC-02 and NewsIR-16,
and the orderings of WESM on different datasets, while with larger spread, are
still similar.

\begin{comment}
\subsection{WED and WEP evaluations on a different model}

We now apply a different model of Word2Vec trained on the GoogleNews dataset to evaluate the algorithms listed in Tabel \ref{duc}
over DUC-02 and NewsIR-16.
The results are shown in Table \ref{other}.

%Results on DUC-02 data are shown in Table \ref{other}. We used WMD-o and WMD-p trained by GoogleNews datasets to evaluate our summarization methods. o-s is the score of WMD-o comparing our system %summary with the DUC document. p-r is the score of WMD-p comparing the DUC reference summary with the DUC document. p-s is the score of WMD-p comparing our system summary with the DUC document. o-r %is the score of WMD-o comparing the DUC reference summary with the DUC document. From Table \ref{other}, although the absolute values of WMD-o and WMD-p are larger than the results in Table %\ref{results} because of using different training sets, we can still conclude the similar results with Table \ref{results} that RAKETT has the best performance and our methods are better than the %submodular function. The scores of WMD-o and WMD-p with our system summaries are higher than with DUC reference summaries.
\begin{table}[h]
\begin{center}
\begin{tabular}{l|c|c|c|c}
\hline
\multirow{2}{*}{\bf Methods} &
%\bf Methods &
\multicolumn{2}{c}{\bf WED}  & \multicolumn{2}{|c}{\bf WEP} \\
\cline{2-5}
%Methods & WED & WEP & WED & WEP \\
& 30\% & 50\% & 30\% & 50\% \\
\hline
\bf ET3Rank &82.32      &\bf 81.38&\bf 55.21&\bf 54.53 \\
ESRAKE      &83.38      &\bf 81.38&54.92    &\bf 54.53 \\
EPRAKE       &83.69     &\bf 81.38&55.03    &\bf 54.53  \\
ET2RAKE     &83.12      &\bf 81.38&54.94    &\bf 54.53  \\
ELDARAKE    &\bf 83.71  &\bf 81.38&55.11    &\bf 54.53 \\
PRAKE       &82.46      &\bf 81.38&54.81    &\bf 54.53 \\
T2RAKE      &82.10      &\bf 81.38&54.77    &\bf 54.53\\
LDARAKE     &82.02      &\bf 81.38&54.87    &\bf 54.53\\
\hline
\end{tabular}
\caption{\label{other1} Scores (\%) of GoogleNews-based WED and WEP on 30\% and 50\% summaries over DUC-02}
\end{center}
\end{table}

\begin{table}[h]
\begin{center}
\begin{tabular}{l|c|c|c|c}
\hline
\multirow{2}{*}{\bf Methods} &
%\bf Methods &
\multicolumn{2}{c}{\bf WED}  & \multicolumn{2}{|c}{\bf WEP} \\
\cline{2-5}
%Methods & WED & WEP & WED & WEP \\
& 30\% & 50\% & 30\% &50\% \\
\hline
\bf ET3Rank &\bf 23.35 &\bf 23.55   &\bf 11.36 &11.38 \\
ESRAKE      &23.22      &23.08      &11.31      &11.34 \\
EPRAKE      &23.06      &23.25      &11.21      &11.31  \\
ET2RAKE     &23.08      &23.38      &11.24      &  11.33\\
ELDARAKE    &23.25      &\bf 23.55  &11.26      &\bf 11.42  \\
PRAKE       &23.06      &23.25      &11.21      &11.31 \\
T2RAKE      &23.18      &23.18      &11.23      &11.34 \\
LDARAKE     &23.25      &23.28      &11.32      &11.33\\
\hline
\end{tabular}
\caption{\label{other} Scores (\%) of GoogleNews-based WED and WEP on 30\% and 50\% summaries over NewsIR-16}
\end{center}
\end{table}
\end{comment}

\subsection{Runtime analysis}

We carried out runtime analysis through experiments on a computer with a 3.5 GHz Intel Xeon CPU E5-1620 v3. %2.7 GHz dual-core Intel Core i5 CPU and 8 GB memory.
We used a Python implementation of our summarization algorithms. %methods on the NewsIR-16 dataset.
Since DUC-02 are short, all but LDA-based algorithms run in about the same time.
To obtain a finer distinction, we ran our experiments on NewsIR-16. Since the average size of NewsIR-16 articles is 405 words,
we selected at random a number of articles from NewsIR-16 and merged them to generate a new article.
For each size from around 500 to around 10,000 words, with increments of 500 words, we selected at random 100 articles and
computed the average runtime of different algorithms to produce 30\% summary (see Figure \ref{fig:runtime}).
We note that the time complexity of each of our algorithms incurs mainly in
the preprocessing phase; %that is, the runtime incurs in the sentence selection phase
%is minor; namely,
the size of summaries in the sentence selection phase only incur minor fluctuations of computation time, and
so it suffice to compare the runtime for producing 30\% summaries.

%and use summary rate with 30\%, 50\% and 70\%.
\begin{figure}[h]
\centering
\includegraphics[width=3.3in]{runtime-color.png}
\caption{Runtime analysis, where the unit on the x-axis is 100 words and the unit of the y-axis is seconds.} \label{fig:runtime}
\end{figure}
We can see from Figure \ref{fig:runtime} that ESRAKE and PRAKE incur about the same linear time and they are extremely fast.
Also, ET3RANK, ET2RAKE, and T2RAKE incur about the same time. While the time is higher because of the use of TextTiling and
is closed to being linear, it meets the realtime requirements. For example, for a document of up to 3,000 words, over
3,000 but less than 5,500 words, and 10,000 words, respectively,
the runtime of ET3Rank is under 0.5, 1, and 2.75 seconds.
%,
%
%for a document less than 5,500 words,
%the runtime is
%less than 1 second, % and for a very long document of 10,000 words, the runtime is
%and less than 2.75 seconds.

The runtime of SubmodularF is acceptable for documents of moderate sizes (not shown in the paper); but for a document of about 10,000 words, the runtime is close to 4 seconds.
LDA-based algorithms is much higher. For example, LDARAKE incurs about 16 seconds for a document of
about 2,000 words, about 41 seconds for a document of about 5,000 words, and about 79 seconds for a document of about 10,000 words.

%\begin{table}
%\begin{tabular}
%
%\end{tabular}
%\end{table}




\begin{comment}
\begin{table}[h]
\begin{center}
\begin{tabular}{l|c|c|c|c}
\hline
\bf Methods & \bf 10\% & \bf 30\% & \bf 50\% & \bf 70\% \\
\hline
\bf ET3Rank & &&& \\
ESRAKE & &&& \\
EPRAKE & &&&  \\
ET2RAKE & &&&  \\
ELDARAKE & &&&  \\
PRAKE & &&& \\
T2RAKE & &&& \\
LDARAKE & &&&\\
\hline
\end{tabular}
\caption{\label{runtime} Runtime (seconds) of different summary rate on the NewsIR-16 dataset.}
\end{center}
\end{table}

The Runtime of different summary rate on the NewsIR-16 dataset is given in Table \ref{runtime}. We can find that RAKE based summarization methods are much faster than TextRank based ones, which is bacause TextRank needs to calculate iterations until it converges while RAKE only calculates word co-occurrence. The runtime of generating summaries with different rate is similar and LDA based methods take more time.
\end{comment}

\section{\uppercase{Conclusions}}
\label{sect:conclusion}
\noindent We presented a number of unsupervised single-document summarization algorithms for generating effective summaries in realtime and
a new measure based on word-embedding similarities to evaluate the quality of a summary. We showed that ET3Rank is the best all-around algorithm. A web-based summarization tool using ET3Rank and T2RAKE will be made available to the public.

To further obtain better topic clusterings efficiently, we plan to extend TextTiling over non-consecutive paragraphs.
To obtain a better understanding of word-embedding similarity measures, we plan to compare WESM with human evaluation and
other unsupervised methods including those devised by Louis and Nenkova \cite{Louis:2009}.
We also plan to
explore new ways to measure summary qualities
without human-generated benchmarks.


\section*{\uppercase{Acknowledgements}}

\noindent
We thank the members of the Text Automation Lab at UMass Lowell for their support and fruitful discussions.



%the major topics of the original document as diverse as possible. We also

%We take sentence ranking and topic diversity coverage into account. We also present summarization evaluation methods based on Word2Vec without reference summaries. We experimented our system with several news datasets. Our summarization methods improve substantially over state-of-the-art systems on ROUGE while still maintaining good linguistic quality and the runtime is competitive. Our summarization evaluation methods can produce comparable results with ROUGE. We plan to explore sentence compression model with pronoun anaphora and trim sentences from phrases instead of words.

% \section*{\uppercase{Acknowledgements}}


\vfill
\bibliographystyle{apalike}
{\small
\bibliography{example}}


% \section*{\uppercase{Appendix}}

\vfill
\end{document}

 
%

\subsection{Experimental Settings}
We evaluated the proposed models on reduced SUN RGB-D and Places365 dataset. To be noticed, aim to investigate the generalizability of our model, we evaluate our model pretrained on Places365-7 dataset on the SUN RGB-D dataset without retraining it. We'll introduce the implementation details and training procedure and different experiment settings.

\subsubsection{Implementation Details}
For the PlacesCNN model, ResNet18 or ResNet50 architecture is adopted in our experiment for ablation study. The optimizer used is the Stochastic Gradient Descent (SGD) with an initial learning rate of 0.01, the momentum of 0.9, and the weight decay of 0.0001. We decrease the learning rate 10 times every 10 epoch, and every time when updating the learning rate, we reload the parameters which have the best accuracy before this timestamp. The total number of epoch during training is 40. To be noticed, we use the training sets of Places365-7 and Places365-14 dataset for learning the BORM statistically, while testing on the SUN RGB-D dataset, where the BORM is the same as the Places365-7 dataset. 


\subsubsection{Dataset Settings}

\begin{table}[]
\centering
	\caption{Dataset split setting, where the number of training set and number of test set are listed below.}
	\label{tab:dataset_split}
	\begin{tabular}{l|l|l}
		\hline
		Dataset      & Training & Test  \\ \hline
		Places365-7  &  35000   & 701     \\ 
		Places365-14 &  75000   & 1500    \\ 
		SUN RGB-D    &  35000(from Places365-7)        & 2077    \\ \hline
	\end{tabular}
\end{table}
\ \\
\textbf{Places365 Dataset:}
In this paper, we use the reduced Places365 \cite{zhou2017places}  dataset to test our methods, since it is the most largest and challenging scene classification dataset yet, and it contains broad categories in the indoor environment. In the experiment, we only consider the indoor scene recognition. There are two different settings on the reduced Places365 dataset. The one is Places365 with 7 classes includes Corridor, Dinning Room, Kitchen, Living Room, Bedroom, Office, and Bathroom, denoted as Places365-7.  The test set setting is the same as the official dataset and described in \cite{pal2019deduce}.
In addition, we use the reduced Places365 with 14 indoor scenes in Home environment includes Wet bar, Home theater, Balcony, Closet, Kitchen, Bedroom, Playroom, Laundromat, Bathroom, Living Room, Home office, Dining room, Staircase, and Garage denoted as Places365-14. The dataset splitting follows the same setting as described in \cite{chen2019scene}. The dataset splition can be seen from Table. \ref{tab:dataset_split}.

\textbf{SUN RGB-D Dataset:}
SUN RGB-D dataset \cite{song2015sun} is a challenging dataset for scene understanding that contains not only RGB images but also depth information of each image. It contains 3784 images collected by Kinect V2 and 1159 collected by Intel RealSense. Moreover, it incorporates 1449 images from the NYUDepth V2 \cite{silberman2012indoor}, and 554 images from the Berkeley B3DO Dataset \cite{janoch2013category}, both captured by Kinect V1. Finally, it takes 3389 manually selected distinguished frames without significant motion blur from the SUN3D videos \cite{xiao2013sun3d} captured by Asus Xtion.

In our experiment, we mainly consider the indoor environment understanding. Therefore, we use the reduced SUN RGB-D dataset includes Office, Kitchen, Bedroom, Corridor, Bathroom, Living room, and Dining room, where the test set split is the same as the official dataset. There are 3741 RGB images in total for testing. And we test our model pretrained on the Place365-7 on SUN RGB-D dataset without retraining.



\subsection{Experimental Results}
\subsubsection{Effect of Object Knowledge}
As illustrated in Fig. \ref{fig:om_ablation}, a group of ablation studies have been conducted for evaluating the effect of object knowledge to indoor scene recognition on the Places365-14, Places365-7, and SUN RGB-D dataset, respectively. The x-axis represent the number of object information IOM have about the indoor scene and is added by 20 from 90 to 150 sequentially selected from the vector. Plus, IOM-80 is the baseline accuracy. Obviously, as the number of object information increases, the much better scene recognition accuracy is achieved, e.g., on the Places365-14 dataset, the IOM-150 reaches 74.1\% accuracy, which is \textbf{10.0\%} higher than IOM-80. This improvement shows the number of object information is proportional to scene recognition accuracy. Moreover, the comparison experiments between the IOM and OM on three datasets, shows an average of \textbf{20.5\%} improvements can be achieved. After analyzing the object categories of OM pretrained on the MS COCO, we observed there are only half of the object categories are related to indoor scenes. In contrast, the other half is related to outdoor scenes. Therefore, the relevance of object categories of object model with the scenes is essential for scene recognition, e.g., the information of elephant and giraffe in OM will not be valuable for indoor scene recognition. Similarly, the bus and train are not beneficial to indoor scene recognition.

\subsubsection{{Analysis of BORM}}

\begin{figure}[tbp]
	\centering
	\includegraphics[width=0.4\textwidth]{Fig//om_abl.jpg}
	\caption{Ablation study of improved object model (IOM) with different number of object knowledge ranging from 80 to 150 categories as shown in horizontal axis. The vertical axis shows the accuracy on percentage.}
	\label{fig:om_ablation}
\end{figure}

%\begin{table*}[]
%	\centering
%	\scriptsize
%	\caption{Ablation study of accuracy on percentage of improved object model (IOM) with different number of object knowledge ranging from 80 to 150 categories.}
%	\label{tab:om_ablation}
%	\begin{tabular}{l|c|c|cccccccc}
%		\hline
%		\textbf{} & {$\Phi_{obj}$ \cite{pal2019deduce}} & OM-80 & {IOM-80} & {IOM-90} & {IOM-100} & {IOM-110} & {IOM-120} & {IOM-130} & {IOM-140} & {IOM-150} \\ \hline 
%		{Places365-14} & - & 47.0 & 64.1 & 64.4 & 66.1 & 71.5 & 71.9 & 72.5 & 73.7 & \textbf{74.1} \\ 
%		{Places365-7} & 62.6 & 73.0 & 80.9 & 81.6 & 81.6 & 81.9 & 81.7 & 82.0 & 82.1 & \textbf{82.6} \\ 
%		{SUN RGBD} & 53.6 & 59.2 & 65.9 & 66.8 & 66.7 & 67.0 & 66.8 & 67.3 & 67.8 & \textbf{68.1} \\ \hline
%	\end{tabular}
%\end{table*}


%\begin{table*}[htbp]
%\centering
%\scriptsize
%\caption{Scene recognition accuracy in percentage on the reduced Places365-7 dataset}
%\label{tab:place365_7}
%\begin{tabular}{cccccccccc}
%\hline
%            & ResNet18 & ResNet50 & $\Phi_{obj}$ \cite{pal2019deduce} & OM  & IOM  & BORM  & CBORM \\ \hline
%Bathroom    & 87 & 94 & 65 & 71  & 87      &  88     &  95   \\
%Bedroom     & 82 & 83 & 74 & 84  & 92      &  92     &  81   \\
%Corridor    & 96 & 93 & 90 & 92  & 91      &  89     &  95   \\
%Dining room & 81 & 71 & 94 & 74  & 85      &  80     &  93   \\
%Kitchen     & 83 & 84 & 62 & 65  & 73      &  83     &  94   \\
%Living room & 55 & 66 & 25 & 66  & 73      &  72     &  92   \\
%Office      & 79 & 88 & 29 & 58  & 76      &  78     &  81   \\ \hline
%Avg         & 80.4 & 82.7 & 62.6 & 72.9   &  82.4   & \textbf{83.1}  & \textbf{90.1}    \\ \hline
%\end{tabular}
%\end{table*}



We conduct experiments for BORM and IOM in the reduced Places365-7 dataset, and results are displayed in Table \ref{tab:place365_7_sota}. The experiment results show the BORM and IOM model has an advantage over the OM and yields an average accuracy of 83.1\% and 82.4\%, surpassing the OM model about \textbf{20\%} accuracy. The experiment result proves that with more object knowledge about the surrounding environment, the greater scene recognition accuracy can be reached.  Then, we test the model pretrained on the Places365-7 dataset on the SUN RGB-D test set, and similar conclusion can be drawn. Moreover, the BORM outperforms the IOM with 0.7\% and 1.1\% accuracy on the Places365-7 and SUN RGB-D dataset, respectively, which validates the knowledge of the co-occurrences between object pairs and their probabilistic relation forms an important indoor scene representation.


Similarly, in the Table \ref{tab:place365_14_sota}, we have conducted experiments on the reduced Places365-14 dataset, and experiment results show the IOM and BORM tremendously improves the performance over OM with \textbf{27\%} accuracy. The results suggest the effectiveness of BORM and IOM over the OM especially when the number of scene classes of dataset is large.



\subsubsection{Performance Comparison}


\begin{table}[htbp]
	\centering
	\scriptsize
	\begin{minipage}[t]{.44\textwidth}
		\centering
		\caption{Comparison with the state-of-the-art methods on the reduced Places365-7 Dataset and SUN dataset of scene recognition accuracy}
		\label{tab:place365_7_sota}
		\begin{tabular}{c|ccc}
			\hline
			Method     & Config   & Acc(Places365-7) & Acc(SUN) \\ \hline
			\multirow{2}{*}{PlacesCNN \cite{zhou2017places}}     & ResNet18 & 80.4  & 63.3  \\
			& ResNet50 & 82.7 & 67.2   \\ \hline
			\multirow{3}{*}{Deduce \cite{pal2019deduce}}     &  $\Phi_{obj}$ (OM)      & 62.6 & 53.6      \\
			& $\Phi_{scene}$    & 87.3 & 66.8    \\
			& $\Phi_{comb.}$ & 88.1 & 70.1    \\ \hline
			%\multirow{1}{*} {Baseline}
			%& OM   & 72.9 & 59.2   \\   \hline 
			\multirow{3}{*}{Ours} 	
			& IOM  & 82.4 & 68.1   \\
			& BORM & 83.1 & 69.2   \\ \cdashline{2-4}
%			& CIOM & 90.1 & 71.8   \\ 
			& CBORM& \textbf{90.1} & \textbf{72.1\%}   \\	\hline
		\end{tabular}
	\end{minipage}

	\hspace{1cm}

	\begin{minipage}[t]{.44\textwidth}
		\centering
		\caption{Comparison with the state-of-the-art methods on the reduced Places365-14 Dataset of scene recognition accuracy, the * indicates the re-implement of the method. }
		\label{tab:place365_14_sota}
		\begin{tabular}{c|cc}
			\hline
			Method                    & Config           & Acc \\ \hline
			\multirow{2}{*}{PlacesCNN \cite{zhou2017places}}     & ResNet18 & 76.0  \\
			& ResNet50 & 80.0   \\ \hline
			\multirow{1}{*}{Word2Vec \cite{chen2019scene}} 
			%& ResNet50         & 83.5    \\
			& ResNet50+Word2Vec         & 83.7    \\ \hline
%			\multirow{1}{*}{Deduce \cite{pal2019deduce}}     & Obj(re-implement)      & 47.0     \\ \hline
            \multirow{1}{*}{{*}Deduce \cite{pal2019deduce}} 
            
            & $\Phi_{obj}$ (OM) 		   & 47.0	 \\ \hline
			\multirow{3}{*}{Ours}  
			& IOM          & 74.1    \\
			& BORM         & 74.9    \\ \cdashline{2-3}
%			& CIOM & 85.5    		\\
			& CBORM & \textbf{85.8}				\\	\hline
		\end{tabular}
	\end{minipage}
\end{table}



As shown in Table \ref{tab:place365_7_sota}, we conduct the ablation study of using only the BORM model and the CBORM model. We found the CBORM yield an average accuracy of \textbf{90.1\%}, which greatly outperforms the ResNet18 and ResNet50 of PlacesCNN baselines about \textbf{10\%} and \textbf{8\%} respectively. Meanwhile, CBORM outperforms the BORM with 7\% accuracy and 3\% on the Places365-7 and SUN RGB-D dataset, respectively.

%Also, it surpass the Scene (ResNet18 model that pretrained on Places365 and finetuned on the Places365-7 dataset). 

In comparison with the state of the art, Table \ref{tab:place365_7_sota} shows that the CBORM improves the scene recognition by \textbf{2.0\%} in the reduce Places365-7 dataset.  Also, as shown in Table \ref{tab:place365_14_sota}, the CBORM improve the recognition accuracy by \textbf{2.1\%} in the reduced Places365-14 dataset. Both results show the combined model achieves comparable results to some recent approaches that use the word-embedding method to extract the semantic meaning of the environment, or use the combination of scene and object representations for better scene understanding. Moreover, Table \ref{tab:place365_7_sota} shows the performance of our method over the method in \cite{pal2019deduce} with \textbf{2\%}, showing the excellent generalization ability of CBORM over other methods on the Reduced SUN RGB-D dataset.

These results demonstrate that CBORM is successful in recognizing the scene images with a competitive accuracy. This improved effectiveness of CBORM over the state-of-the-art justifies our reasonable assumption that relation of object pairs is an essential complementary information for indoor scene recognition.



 
%\section{Conclusion}
\label{sec:conclusion}
This paper presents a generic top-$\size$ recommendation framework for  trading-off accuracy, novelty, and coverage. To achieve this, we profile the users according to their preference for long-tail novelty. We examine various measures, and formulate an optimization problem to learn these user preferences from interaction data.  We then integrate the user preference estimates in our generic framework, GANC.  Extensive experiments on several datasets confirm that there are trade-offs between accuracy, coverage, and novelty. Almost all re-ranking models increase coverage and novelty at the cost of accuracy. However, existing re-ranking models typically rely on rating prediction models, and are hence more effective in dense settings. Using a generic approach, we can easily incorporate a suitable base accuracy recommender to devise an effective solution for both sparse and dense settings.  %Our results  also indicate there is no single method that outperforms other methods in all metrics. However our techniques obtain a significant improvement in coverage, while  . 
Although we integrated the  long-tail novelty preference estimates into a re-ranking framework, their use-case is not limited to these frameworks. In  the future, we intend to explore the temporal and topical dynamics of long-tail novelty preference, particularly in settings where contextual information is  available.  
%We achieve these objectives without using any additional contextual information.


\iffalse
While we focused on promoting long-tail items across users, we did not consider diversity of individual top-$\size$ recommendations, a factor that has been shown to positively affect consumer satisfaction. This is one direction for future work. Moreover, the sequential setting  in our work, creates a dependency between different batches, where,  the items recommended to a batch of users, depends on those recommended to previous batches. This dependency is created through the parameter $\mathbf{f}$, that is updated every time a top-$\size$ set  is allocated to a batch of users. A future direction for our work is to estimate a distribution over $\mathbf{f}$ that allows us to independently solve the problem for each user, leading to improvements across all performance metrics, including recommendation time. 

We design algorithms that take advantage of the structure in the value functions to obtain both efficient and scalable solutions. 
We design an algorithm that takes advantage of the structure in the value functions to obtain both efficient and scalable solutions. 

\textcolor{red}{Our  sequential  algorithms can be applied for batch recommendation contexts,~e.g., personalized email marketing, where based on prior interaction data between users and items,  a new round of recommendations must be sent to all users in the system.  However, the independent coverage algorithms lift the sequential setting restrictions and allow it be applied for re-ranking the output of base recommender in any setting. }A future direction for our work is to incorporate explicit diversity metrics in the framework. 
\fi


%We have a presented a submodular maximization framework to systematically trade-off relevance and diversity in recommendations to individual users and coverage across the item-space. This ensures both consumer and producer satisfaction. We model users according to their risk and focusing degrees and promote long-tail items to the right group of consumers. Consequently, we obtain a significant improvement in coverage while maintaining reasonable levels of user satisfaction. Furthermore, our methods are able to achieve a more balanced distribution across the set of recommended items. In the future, we plan to investigate the effect of using alternative base recommender systems. 

%Future Work
%However most of these methods assume that the ratings are missing at random (MAR). Since our method of generating recommendations is based on the completed matrix, assuming MAR might introduce additional bias, we will use methods which assume that the ratings at missing not at random (MNAR),explored in~\cite{steck2010training, icml2014c2_hernandez-lobatob14}. 	 
%Long Tail %Recently, authors in~\cite{cremonesi2010performance} conducted extensive experiments to evaluate the performances of various matrix factorization-based algorithms and neighborhood models on the task of recommending long tail items. Their experimental results show that long tail recommendation leads to a decrease in accuracy for all algorithms. They also showed that for this task, SVD outperforms other state-of-the-art algorithms. 
 
%\newpage
\section{Dataset Visualizations}
\label{sec:app_dataset_visuals}

%%%%%%
%%
%%
\subsection{Examples of each view class}
\newcommand{\BC}{0.33}
\setlength{\tabcolsep}{0.1cm}
\begin{figure}[!h]
\begin{tabular}{c c c c}
    PLAX  & PSAX & OTHER 
    \\
    \includegraphics[width=\BC\textwidth]{figures/small_appendix/Appendix_PLAX1.jpg}
    &
    \includegraphics[width=\BC\textwidth]{figures/small_appendix/Appendix_PSAX1.jpg}
    &
    \includegraphics[width=\BC\textwidth]{figures/small_appendix/Appendix_Other1.jpg}
    &
   
    \\
    
    \includegraphics[width=\BC\textwidth]{figures/small_appendix/Appendix_PLAX2.jpg}
    &
    \includegraphics[width=\BC\textwidth]{figures/small_appendix/Appendix_PSAX2.jpg}
    &
    \includegraphics[width=\BC\textwidth]{figures/small_appendix/Appendix_Other2.jpg}
    &
   
     \\
     
     \includegraphics[width=\BC\textwidth]{figures/small_appendix/Appendix_PLAX3.jpg}
    &
    \includegraphics[width=\BC\textwidth]{figures/small_appendix/Appendix_PSAX3.jpg}
    &
    \includegraphics[width=\BC\textwidth]{figures/small_appendix/Appendix_Other3.jpg}
    &
   
     \\
     
     \includegraphics[width=\BC\textwidth]{figures/small_appendix/Appendix_PLAX4.jpg}
    &
    \includegraphics[width=\BC\textwidth]{figures/small_appendix/Appendix_PSAX4.jpg}
    &
    \includegraphics[width=\BC\textwidth]{figures/small_appendix/Appendix_Other4.jpg}
    &
   
    \end{tabular}	
    \caption{Examples of images for each possible view label in our dataset. \emph{From left to right:} Four examples of peristernal long axis (PLAX) view, four examples of peristernal short axis (PSAX) view, and four examples of other kinds of view in our ``Other'' class. }
    \label{fig:VIEW_SAMPLES_APPENDIX}
\end{figure}

%%%%%%
%%
%%
\newpage
\subsection{Examples of each view for a Severe AS patient}
\newcommand{\BA}{0.33}
\setlength{\tabcolsep}{0.1cm}
\begin{figure}[!h]
\begin{tabular}{c c c c}
    PLAX  & PSAX & OTHER 
    \\
    \includegraphics[width=\BA\textwidth]{figures/small_appendix/SevereAS_11112007_PLAX1.jpg}
    &
    \includegraphics[width=\BA\textwidth]{figures/small_appendix/SevereAS_11112007_PSAX1.jpg}
    &
    \includegraphics[width=\BA\textwidth]{figures/small_appendix/SevereAS_11112007_Other1.jpg}
    &
    
    \\
    
    \includegraphics[width=\BA\textwidth]{figures/small_appendix/SevereAS_11112007_PLAX2.jpg}
    &
    \includegraphics[width=\BA\textwidth]{figures/small_appendix/SevereAS_11112007_PSAX2.jpg}
    &
    \includegraphics[width=\BA\textwidth]{figures/small_appendix/SevereAS_11112007_Other2.jpg}
    &
   
     \\
     
     \includegraphics[width=\BA\textwidth]{figures/small_appendix/SevereAS_11112007_PLAX3.jpg}
    &
    \includegraphics[width=\BA\textwidth]{figures/small_appendix/SevereAS_11112007_PSAX3.jpg}
    &
    \includegraphics[width=\BA\textwidth]{figures/small_appendix/SevereAS_11112007_Other3.jpg}
    &
  
    \end{tabular}	
    \caption{Examples of images from a patient with Severe AS in our dataset. \emph{From left to right:} Three examples of parasternal long axis (PLAX) view, three examples of parasternal short axis (PSAX) view, and three examples of other kinds of view in our ``Other'' class. }
    \label{fig:PatientSevereAS}
\end{figure}


%%%%%%
%%
%%
\newpage
\subsection{Examples of each view for a No AS patient}
\newcommand{\BB}{0.33}
\setlength{\tabcolsep}{0.1cm}
\begin{figure}[!h]
\begin{tabular}{c c c c}
    PLAX  & PSAX & OTHER 
    \\
    \includegraphics[width=\BB\textwidth]{figures/small_appendix/NoAS_1996889_PLAX1.jpg}
    &
    \includegraphics[width=\BB\textwidth]{figures/small_appendix/NoAS_1996889_PSAX1.jpg}
    &
    \includegraphics[width=\BB\textwidth]{figures/small_appendix/NoAS_1996889_Other1.jpg}
    &
    
    \\
    
    \includegraphics[width=\BB\textwidth]{figures/small_appendix/NoAS_1996889_PLAX2.jpg}
    &
    \includegraphics[width=\BB\textwidth]{figures/small_appendix/NoAS_1996889_PSAX2.jpg}
    &
    \includegraphics[width=\BB\textwidth]{figures/small_appendix/NoAS_1996889_Other2.jpg}
    &
   
     \\
     
     \includegraphics[width=\BB\textwidth]{figures/small_appendix/NoAS_1996889_PLAX3.jpg}
    &
    \includegraphics[width=\BB\textwidth]{figures/small_appendix/NoAS_1996889_PSAX3.jpg}
    &
    \includegraphics[width=\BB\textwidth]{figures/small_appendix/NoAS_1996889_Other3.jpg}
    &
  
    \end{tabular}	
    \caption{Examples of images from a patient with No AS in our dataset. \emph{From left to right:} Three examples of parasternal long axis (PLAX) view, three examples of parasternal short axis (PSAX) view, and three examples of other kinds of view in our ``Other'' class. }
    \label{fig:PatientNoAS}
\end{figure}



\newpage 
\section{Further Results}

\subsection{Assessment of ensembling}

Table~\ref{tab:best_single_checkpoint_VS_ensemble_FS_echo260} compares using a single checkpoint (one point estimate of neural network weight vector $\theta$) to using an ensemble of parameters aggregated from the last 25 checkpoints (one per epoch).

\begin{table}[!h]
    \centering
    \begin{tabular}{c|cccc|c}
    \textit{Diagnosis classification} & Split 1  & Split 2 & Split 3 & Split 4 & Average\\
    \hline
    Best single checkpoint  & 61.81 & 59.79 & 56.05 & 64.21 & 60.46\\
    Ensemble  & 62.95 & 61.03 & 56.58 & 63.84 & \textbf{61.13}
	\\ \hline
    \textit{View classification}  &   &  &  &  & 
    \\ \hline
    Best single checkpoint  & 93.03 & 93.24 & 92.39 & 93.79 & 93.11\\
    Ensemble  & 92.37 & 93.24 & 93.72 & 93.87 & \textbf{93.30}\\
    \end{tabular}
    \caption{Comparing best single checkpoint performance with ensemble performance on \textbf{Full-size \datasetName-156-52}}
    \label{tab:best_single_checkpoint_VS_ensemble_FS_echo260}
\end{table}


%%%%%%
%%
%%
\subsection{Patient-level diagnosis performance on bonus heldout set}

Table~\ref{tab:diagnosis classification patient unlabeled_heldout_174} examines the performance of the best labeled-set-only methods and MixMatch methods on the 174 patient studies that have diagnosis but no view labels.
 While the images used here were originally included in the unlabeled training set (which was used to train SSL methods like MixMatch), the diagnosis labels were not provided at all during training time. 
 We thus still believe this is an authentic test of generalization given the scarcity of labeled data available for our task.
 Of course, additional independent evaluation (especially from another institution) is needed.

\begin{table}[!h]
    \centering
    \begin{tabular}{l l l|rrrr|c}
    Pretrain & Method & Voting
    & Split 1  & Split 2 & Split 3 & Split 4 & average\\
    \hline
    & Basic WRN & Simple average & 76.73 & 75.25 & 76.87 & 81.88 & 77.68\\
    & Basic WRN & View-prioritized & 73.63 & 83.21 & 79.70 & 80.08 & 79.18\\
    %SSL & FS & MixMatch & Priority view + confidence & 94.58 & 84.17 & 77.50 & 92.5 & 87.19\\
    \hline
    & MixMatch & Simple average & 85.32 & 76.29 & 74.14 & 79.95 & 78.93\\
    view & MixMatch & Simple average & 83.36 & 77.96 & 75.61 & 81.37 & 79.58\\
    & MixMatch & View-prioritized & 83.27 & 83.76 & 82.34 & 82.83 & \textbf{83.05}\\
    view & MixMatch & View-prioritized & 82.53 & 86.15 & 79.62 & 83.27 & 82.89\\
    %view & MixMatch & LR with view-priority & 80.42 & 84.24 & 76.58 & 80.67 & 80.48\\
    %(MixMatch transfered) + MysteryMethod & NA & NA & NA\\ 
    \end{tabular}
    \caption{Patient-level AS Severity Diagnosis Classification on the \textbf{bonus heldout set} of 174 patients for whom we have diagnosis labels only (no view labels). We show balanced accuracy on models trained on each of the four folds on four \textbf{full-size \datasetName-156-52} dataset.
    }%endcaption
    \label{tab:diagnosis classification patient unlabeled_heldout_174}
\end{table}


%%%%%%
%%
%%
\subsection{Assessment of MixMatch hyperparameter sensitivity}

In Table~\ref{tab:MixMatch hyperparameters ablation study}, we consider four possible strategies for setting the hyperparameters of MixMatch, varying two  key settings for the weight on unlabeled loss $\lambda$. First, we vary whether the final value of $\lambda$ is set to its \emph{best} value among a grid of candidates (based on validation set performance), or \emph{fixed} to a constant.
Second, we vary whether $\lambda$ remains fixed over iterations throughout a training run, or is updated over iterations on a linear ramp schedule from 0 to its final target value. 

From this comparison, we see we consistent gains across splits (average gain across splits of over 1.6\% balanced accuracy) for using a delayed ramp up schedule with target value selected via grid search.

\begin{table}[!h]
    \centering
    \begin{tabular}{l l| rrrr | r}
    Final $\lambda$ value & $\lambda$ update schedule & Split 1  & Split 2 & Split 3 & Split 4 & Average\\
    \hline
    best on val & Delayed ramp-up  & 65.57 & 62.69 & 60.87 & 66.29 & 63.86\\
    best on val & Immediate ramp-up & 65.07 & 61.87 & 60.82 & 65.37 & 63.28\\
    best on val & Constant  & 65.03 & 61.52 & 58.87 & 65.22 & 62.66\\
    100 (fixed) & Constant & 63.94 & 61.79 & 58.87 & 64.35 & 62.24\\
    \end{tabular}
    \caption{Ablation study of different settings of the unlabeled loss weight $\lambda$ for MixMatch. AS severity diagnosis classification for individual images on the \textbf{full-size \datasetName-156-52} dataset. showing balanced accuracy averaged over the test sets from multiple folds (each fold’s test set contains all images from 52 patients). }%endcaption
    \label{tab:MixMatch hyperparameters ablation study}
\end{table}



%%%%%%
%%
%%
\subsection{Assessment of alternative view prioritization strategy using thresholding}


An anonymous reviewer suggested an alternative strategy for prioritizing images of relevant view.
The alternative strategy works as follows: for each image, we compute the predicted probability that the image is a ``relevant view'' (either PLAX and PSAX) by summing the probabilities of each view type.
However, instead of using this raw probability as a weight (as our chosen method does), we use a \emph{cutoff threshold} and simply average the diagnosis predictions of images whose relevant view probability is above the cutoff.
For each patient, we use the majority vote prediction of the diagnosis from the images of relevant views.
The value of the cutoff threshold is selected using the validation set to maximize balanced accuracy.

Table~\ref{tab:Suggested_Aggregation_Ablation} shows the performance of this strategy (``threshold-then-average'') on the full-size dataset.
Using this alternative prioritization strategy together with our suggested methodology for patient-level diagnosis (using MixMatch, pretraining on view), we find the average test set balanced accuracy is around 85.8\%, while the weighted average strategy in the main paper achieves over 90\% balanced accuracy. We take this as reasonably decisive evidence that a weighted average (rather than a simple cutoff) should be preferred.

\begin{table}[!h]
    \centering
    \begin{tabular}{l l l|rrrr|c}
    Pretrain & Method & Aggregation across images
    & Split 1  & Split 2 & Split 3 & Split 4 & average\\
    \hline
    & Basic WRN & Threshold-then-Average & 85.42 & 86.25 & 79.17 & 92.50 & 85.84 \\
    %SSL & FS & MixMatch & Priority view + confidence & 94.58 & 84.17 & 77.50 & 92.5 & 87.19\\
    & MixMatch & Threshold-then-Average & 83.33 & 84.17 & 77.50 & 94.58 & 84.90 \\
    view & MixMatch & Threshold-then-Averagen & 86.67 & 80.00 & 82.50 & 94.17 & 85.84\\
    %view & MixMatch & LR with view-priority & 87.08 & 82.08 & 85.00 & 88.75 & 85.73\\
    %(MixMatch transfered) + MysteryMethod & NA & NA & NA\\ 
    \end{tabular}
    \caption{Alternative view-prioritizing strategy for patient-level AS severity diagnosis classification on the \textbf{full-size \datasetName-156-52} dataset, showing balanced accuracy on the test set across multiple folds (each fold’s test set contains 52 patients).}
    %endcaption
    \label{tab:Suggested_Aggregation_Ablation}
\end{table}



%%%%%%
%%
%%
\subsection{ROC Curve of patient-level diagnosis: no AS vs. mild/moderate/severe AS}

Fig.~\ref{fig: No AS vs Some AS} shows receiver operating curves for several methods for the task of distinguishing no AS vs Some AS (which aggregates both the mild/moderate and severe levels in the 3-level diagnosis task of the main paper).

\begin{figure}[!h]
\begin{tabular}{c c}
	\includegraphics[width=0.43\textwidth]{figures/fold0_multitask_PatientLevel_NoVSSome_NormalizedPriorityStrategyClassProbabilityScore.pdf}
	&
    \includegraphics[width=0.43\textwidth]{figures/fold1_multitask_PatientLevel_NoVSSome_NormalizedPriorityStrategyClassProbabilityScore.pdf}
	\\
	(a) Split 1 & (b) Split 2
	\\
	\includegraphics[width=0.43\textwidth]{figures/fold2_multitask_PatientLevel_NoVSSome_NormalizedPriorityStrategyClassProbabilityScore.pdf}
	&
    \includegraphics[width=0.43\textwidth]{figures/fold3_multitask_PatientLevel_NoVSSome_NormalizedPriorityStrategyClassProbabilityScore.pdf}
	\\
	(c) Split 3 & (d) Split 4
\end{tabular}
    
\caption{ROC curves for binary diagnosis task (no AS vs ``mild/moderate/severe AS'') on \textbf{full-size \datasetName-156-52}.
    }%endcaption
    \label{fig: No AS vs Some AS}
\end{figure}

\section{Methodological Details}

\subsection{Image processing details}
\label{sec:removing_doppler}

\paragraph{Removing doppler images.}
In the raw data of all imagery available for an echocardiogram study, 
we obtained TIFF files that represent both cineloops and Doppler images.

We verified in our labeled set that all Doppler images have one of the following landscape aspect ratio $(831, 323)$, $(901, 384)$, $(901, 390)$, $(704, 305)$, $(831, 421)$, $(901, 469)$ or $(563, 294)$. Only the Dopplers have these aspect ratios. We thus filtered out Doppler completely via these aspect ratios. 

\paragraph{Downsizing}
The original images are provided as high-resolution TIFF format images (hundreds of pixels per side) of varying aspect ratios. Generally, we can expect that both view and diagnosis classifiers would perform better given higher-resolution input (and holding other factors the same). The main trade-off of processing higher-resolution images is increased runtime and memory requirements. In our preliminary experiments, we compared downsizing all images to a standard square aspect ratio at 3 possible sizes: 32x32, 64x64 and 128x128. We found that 64x64 achieves a good balance between model performance and computation cost. 
A prior study by \citet{madaniDeepEchocardiographyDataefficient2018} provides a more extensive study of optimal resolution size. The interested reader can refer to their work for more details. 


\subsection{Architecture Settings and Hyperparameters}
\label{sec:arch_and_hyperparameters}

\paragraph{Weighted cross-entropy for labeled loss}
To counteract the effect of class imbalance in the dataset, we use weighted cross-entropy for the labeled loss. For an input image $x$ whose true label $y$ indicates it belongs to class $c$, the weighted cross-entropy assumes the following form:
\begin{align}
\mathcal{L}^L(\theta, x) = - w_{c} \log \hat{p}_{c}(\theta, x),
\end{align}
where $\hat{p}_{c}$ is the predicted probability of class $c$. The weight $w_{c}$ is calculated using the training set statistics as follow:
\begin{align}
w_{c} = \frac{\prod_{k\neq c}{N_{k}}}{\sum_{j}\prod_{k \neq j}{N_{k}}}
\end{align}
where $N_{k}$ is the number of images of class $k$ in the training set.

\paragraph{Common architecture.}
Following~\citet{oliverRealisticEvaluationDeep2018}, for all considered methods, we use the \emph{same} backbone neural network architecture: a wide residual network~\citep{zagoruykoWideResidualNetworks2017} with 28 layers (WRN-28), which has total of 5,931,683 parameters.
This same network architecture is used in the original MixMatch evaluation~\citep{berthelotMixmatchHolisticApproach2019} with promising results.

\paragraph{Common training protocol.}
All SSL methods we consider follow the loss minimization framework with two primary losses (one for ``labeled'' data and one for ``unlabeled'' data) in Eq.~\eqref{eq:standard-SSL-loss-template}.
We allow every method to train for 32 epochs (where each epoch processes $2^{16}$ images, as in \citet{berthelotMixmatchHolisticApproach2019}).
Our preliminary experiments suggest that after 30 epochs all methods effectively converge in terms of validation balanced accuracy. 

\paragraph{Common regularization.}
For all methods, we expect performance will be vulnerable to overfitting, so we impose an L2-norm penalty on the weights $\theta$, also known as weight decay. Each method selects its preferred value of this penalty strength hyperparameter. We searched values in [0.0002, 0.002, 0.02].

\paragraph{Common optimization.}
We use ADAM \citep{kingma2014adam} to optimize each model.
Each method selects the value of the step size (learning rate) as a hyperparameter. We experimented with 0.002 and 0.0007
%HZ: 'performance being sensitive to learning rate' is very reasonable. But we don't have an ablation to back it. 
%We find performance is sensitive to the step size (learning rate) hyperparameter, so we perform a grid search and select the value that maximizes balanced accuracy on the validation set.

\paragraph{Hyperparameters for Pseudo-Label.}
Beyond the usual hyperparameters for our loss-minimization SSL framework, another important hyperparameter for pseudo-label is the threshold $\tau$. We find that performance is not very sensitive to the chosen $\tau$ value as long as it is within a certain range. We set $\tau$ to 0.95, as done in past literature that evaluates Pseudo-Label as an SSL method ~\citep{oliverRealisticEvaluationDeep2018,berthelotMixmatchHolisticApproach2019, berthelotRemixmatchSemisupervisedLearning2019, sohnFixmatchSimplifyingSemisupervised2020}.


\paragraph{Hyperparameters for VAT.}
Beyond the usual hyperparameters for our SSL framework, for VAT we need to select a value for $\epsilon$.
In \citet{miyatoVirtualAdversarialTraining2019}, the authors claimed that they can achieve superior performance by tuning only $\epsilon$ and fixing $\lambda$ to 1. In our experiment, we used the default $\lambda$ as in \cite{berthelotMixmatchHolisticApproach2019} and searched the value of $\epsilon$ in [2, 6, 18], together with learning rate and weight decay. We select the best hyperparameters using validation set performance. 


\paragraph{Hyperparameters for MixMatch.}
Beyond the usual hyperparameters for our SSL framework, the key hyperparameters for MixMatch include the number of augmentations $K$, the temperature $T>0$ used for sharpening, interpolation hyperparameter $\alpha$ and unlabeled loss coefficient $\lambda$. We set $K=2$, $T=0.5$, and $\alpha=0.75$ as done in \citet{berthelotMixmatchHolisticApproach2019}, and search for $\lambda$ in the range [10, 30, 75, 100, 130] using validation set. 

\paragraph{Hyperparameters for Multitask training.}
We searched $\gamma$, the hyperparameter that control the strength of the auxilliary view loss in Eq.~\eqref{eq:multitask}, in the range [10, 3, 1, 0.3, 0.1]. The best $\alpha$ is selected together with other hyperparameters on validation set. 
 


\bibliographystyle{chicago}
\bibliography{ref}










\end{document} 