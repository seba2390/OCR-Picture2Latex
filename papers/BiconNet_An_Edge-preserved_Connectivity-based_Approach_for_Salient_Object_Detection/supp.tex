% This version of CVPR template is provided by Ming-Ming Cheng.
% Please leave an issue if you found a bug:
% https://github.com/MCG-NKU/CVPR_Template.

\documentclass[review]{cvpr}
%\documentclass[final]{cvpr}
\usepackage{stfloats}
\usepackage{times}
\usepackage{epsfig}
\usepackage{graphicx}
\usepackage{amsmath}
\usepackage{amssymb}
\newcommand*{\Scale}[2][4]{\scalebox{#1}{$#2$}}%
% Include other packages here, before hyperref.

% If you comment hyperref and then uncomment it, you should delete
% egpaper.aux before re-running latex.  (Or just hit 'q' on the first latexhttps://www.overleaf.com/project/5fa74b342fa1906cff9bc289
% run, let it finish, and you should be clear).
\usepackage[pagebackref=true,breaklinks=true,colorlinks,bookmarks=false]{hyperref}


\def\cvprPaperID{4306} % *** Enter the CVPR Paper ID here
\def\confYear{CVPR 2021}
%\setcounter{page}{4321} % For final version only


\begin{document}


%%%%%%%%% TITLE
\title{BiconNet: An Edge-preserved Connectivity-based Approach for Salient Object Detection --- Supplementary Material}

\author{First Author\\
Institution1\\
Institution1 address\\
{\tt\small firstauthor@i1.org}
% For a paper whose authors are all at the same institution,
% omit the following lines up until the closing ``}''.
% Additional authors and addresses can be added with ``\and'',
% just like the second author.
% To save space, use either the email address or home page, not both
\and
Second Author\\
Institution2\\
First line of institution2 address\\
{\tt\small secondauthor@i2.org}
}

\maketitle


\

%%%%%%%%% BODY TEXT
This supplementary material provides complementary information about Bilateral Connectivity Network (BiconNet). 

\section{Analysis of the Model Size}
As illustrated in the paper, the numbers of parameters of proposed BiconNet are very closed to those of baselines since we only changed the last fully connection layers. This conclusion can be also viewed in Tab. \ref{size}. This means we can improve the baselines without generating too much extra computational costs.

\section{Experiment Setting Details}
As we illustrated in the paper, we adopted the same data pre-processing tricks and network settings as described the baselines' paper \cite{gcpa,poolnet,MINet,egnet,cpd,ITSD} for every baseline experiment. For the corresponding BiconNet version of the baseline, we mainly changed the starting learning rates of that of the baselines. The details of the parameter settings of all our experiments are shown in Tab. \ref{settings}. Note that we used the official codes for the baselines and strictly followed the instruction on their websites. All experiments were trained from scratch.
\

\begin{table}
\begin{center}
\setlength{\abovecaptionskip}{0.cm}
\caption{The model sizes of baselines and their corresponding BiconNet version.}
\label{size}
\includegraphics[width=0.98\linewidth]{{latex/size.png}}
\end{center}
\vspace{-9pt}
\end{table}
\section{The Effect of $L_{bimap}$}
In this section, we will illustrate the effect of $L_{bimap}$ in the training. As mentioned in the paper, $L_{bimap}$ computes the multi-channel binary cross entropy loss based on the Bicon map $\widetilde C$. Different from Conn map $C$ which is the direct output of the network, Bicon map $\widetilde C$ is generated by multiplying the two unidirectional probability through bilateral voting. Therefore, the elements in $\widetilde C$ are exponentially correlated with those in $C$. For the hard salient pixel \cite{f3net}, such as an edge pixel, since it is likely to be predicted as a low probability value in $C$, it will generate an even smaller value in $\widetilde C$. This means the $L_{bimap}$ will generate much larger losses at these pixels as shown in Fig. \ref{bimap}. As a result, $L_{bimap}$ can encourage the network to pay more attention to the structural information and help the network to learn to handle the complex scenes.

\begin{table*}[b]
\begin{center}
\caption{The detailed network settings for BiconNets with different backbones.}
\label{settings}
\includegraphics[width=0.98\linewidth]{{latex/settings.png}}
\end{center}
\end{table*}


\begin{figure*}[b]
\begin{center}
\setlength{\abovecaptionskip}{0.cm}
\includegraphics[width=0.95\linewidth]{{latex/bimap_l.png}}
\caption{The visualization of $L_{bimap}$ and $L_{connmap}$. We can see that different from $L_{connmap}$ which puts global attention on the image, $L_{bimap}$ focuses more on the hard salient pixels, such as the edges of objects or the salient regions where the intensities have sharp changes, and generates much larger loss at these regions.}
\label{bimap}
\end{center}
\vspace{-9pt}
\end{figure*}

{\small
\bibliographystyle{ieee_fullname}
\bibliography{Bicon_library}
}


\end{document}
