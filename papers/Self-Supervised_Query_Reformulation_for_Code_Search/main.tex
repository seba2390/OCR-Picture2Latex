\documentclass[sigconf,screen]{acmart}
%\documentclass[sigconf,review,anonymous]{acmart}

%%
%% \BibTeX command to typeset BibTeX logo in the docs
\AtBeginDocument{%
  \providecommand\BibTeX{{%
    \normalfont B\kern-0.5em{\scshape i\kern-0.25em b}\kern-0.8em\TeX}}}
    

\setcopyright{acmcopyright}
\copyrightyear{2023}
\acmYear{2023}
\acmDOI{XXXXXXX.XXXXXXX}

\acmConference[ESEC/FSE 2023]{The 31st ACM Joint European Software Engineering Conference and Symposium on the Foundations of Software Engineering}{11 - 17 November, 2023}{San Francisco, USA}

\acmPrice{15.00}
\acmISBN{978-1-4503-XXXX-X/18/06}


% The preceding line is only needed to identify funding in the first footnote. If that is unneeded, please comment it out.
% \usepackage{cite}
\usepackage{amsmath,amsfonts}
\usepackage{algorithmic}
\usepackage{graphicx}
\usepackage{textcomp}
\usepackage{ulem}
\usepackage{xcolor}
\usepackage{color, colortbl}
\usepackage{balance}
\usepackage{url}
% \def\BibTeX{{\rm B\kern-.05em{\sc i\kern-.025em b}\kern-.08em
%     T\kern-.1667em\lower.7ex\hbox{E}\kern-.125emX}}
    
\usepackage{xspace}

\newcommand{\etal}{\hbox{et al.}\xspace}
\newcommand{\eg}{\hbox{e.g.}\xspace}
\newcommand{\ie}{\hbox{i.e.}\xspace}
\newcommand{\wrt}{\hbox{w.r.t.}\xspace}
\newcommand{\etc}{\hbox{etc.}\xspace}
\newcommand{\vs}{\hbox{vs.}\xspace}

\newcommand{\ourmethod}{\textit{SSQR}\xspace}  

\newcommand{\new}[1]{\textcolor{red}{[mao: #1]}}
\newcommand{\shen}[1]{\textcolor{red}{[shen: #1]}}
\newcommand{\gu}[1]{\textcolor{blue}{[gu: #1]}}
\newcommand{\mao}[1]{\textcolor{red}{[mao: #1]}}
\newcommand{\wan}[1]{\textcolor{cyan!70!blue}{[Wan: #1]}}

% \usepackage{algorithm} 
\usepackage{mathrsfs}
% \floatname{Algorithm}{Procedure}
\renewcommand{\algorithmicrequire}{\textbf{Input:}}
\renewcommand{\algorithmicensure}{\textbf{Output:}}
\newcommand{\codecomment}[1]{\hfill$\triangleright$ {#1}}
% Please add the following required packages to your document preamble:
\usepackage{multirow}
% for split clines in table
\usepackage{booktabs}
%for demonstrate case study code translations
\usepackage{listings}
% for figure
\usepackage{graphicx}
%% subfigure
\usepackage{subcaption}
\usepackage{cleveref}
% for line space
\usepackage{setspace}
% for table
\usepackage{threeparttable}
% for algorithm
\usepackage[ruled]{algorithm2e}
\normalem    

\begin{document}

\title{Self-Supervised Query Reformulation for Code Search}

\settopmatter{authorsperrow=4}
\author{Yuetian Mao}
\authornote{Both authors contributed equally to this research.}
\affiliation{%
  \institution{Shanghai Jiao Tong University}
  \city{Shanghai}
  \country{China}
}
\email{mytkeroro@sjtu.edu.cn}

\author{Chengcheng Wan}
\authornotemark[1]
%\authornote{Chengcheng Wan is the co-corresponding author.}
\affiliation{%
  \institution{East China Normal University}
  \city{Shanghai}
  \country{China}
}
\email{wancc1995@gmail.com}

\author{Yuze Jiang}
\affiliation{%
  \institution{Shanghai Jiao Tong University}
  \city{Shanghai}
  \country{China}
}
\email{jyz-1201@sjtu.edu.cn}

\author{Xiaodong Gu}
\authornote{Xiaodong Gu is the corresponding author.}
\affiliation{%
  \institution{Shanghai Jiao Tong University}
  \city{Shanghai}
  \country{China}
}
\email{xiaodong.gu@sjtu.edu.cn}

%%
%%
%% Sometimes the addresses are too long to fit on the page.  In this
%% case uncomment the lines below and fill them accodingly.
%%
%% \authorsaddresses{Corresponding author: Ben Trovato,
%% \href{mailto:trovato@corporation.com}{trovato@corporation.com};
%% Institute for Clarity in Documentation, P.O. Box 1212, Dublin,
%% Ohio, USA, 43017-6221}
%%
%%
%% Keywords. The author(s) should pick words that accurately describe
%% the work being presented. Separate the keywords with commas.

\begin{abstract}
Automatic query reformulation is a widely utilized technology for enriching user requirements and enhancing the outcomes of code search. It can be conceptualized as a machine translation task, wherein the objective is to rephrase a given query into a more comprehensive alternative. While showing promising results, training such a model typically requires a large parallel corpus of query pairs (\ie, the original query and a reformulated query) that are confidential and unpublished by online code search engines. This restricts its practicality in software development processes. In this paper, we propose \ourmethod, a self-supervised query reformulation method that does not rely on any parallel query corpus. Inspired by pre-trained models, \ourmethod treats query reformulation as a masked language modeling task conducted on an extensive unannotated corpus of queries. \ourmethod extends T5 (a sequence-to-sequence model based on Transformer) with a new pre-training objective named \textit{corrupted query completion} (CQC), which randomly masks words within a complete query and trains T5 to predict the masked content. Subsequently, for a given query to be reformulated, \ourmethod identifies potential locations for expansion and leverages the pre-trained T5 model to generate appropriate content to fill these gaps. The selection of expansions is then based on the information gain associated with each candidate. Evaluation results demonstrate that \textit{our method} outperforms unsupervised baselines significantly and achieves competitive performance compared to supervised methods.
\end{abstract}

% \begin{CCSXML}
% <ccs2012>
%  <concept>
%   <concept_id>10010520.10010553.10010562</concept_id>
%   <concept_desc>Computer systems organization~Embedded systems</concept_desc>
%   <concept_significance>500</concept_significance>
%  </concept>
%  <concept>
%   <concept_id>10010520.10010575.10010755</concept_id>
%   <concept_desc>Computer systems organization~Redundancy</concept_desc>
%   <concept_significance>300</concept_significance>
%  </concept>
%  <concept>
%   <concept_id>10010520.10010553.10010554</concept_id>
%   <concept_desc>Computer systems organization~Robotics</concept_desc>
%   <concept_significance>100</concept_significance>
%  </concept>
%  <concept>
%   <concept_id>10003033.10003083.10003095</concept_id>
%   <concept_desc>Networks~Network reliability</concept_desc>
%   <concept_significance>100</concept_significance>
%  </concept>
% </ccs2012>
% \end{CCSXML}

\begin{CCSXML}
<ccs2012>
<concept>
<concept_id>10002951.10003317.10003325.10003330</concept_id>
<concept_desc>Information systems~Query reformulation</concept_desc>
<concept_significance>500</concept_significance>
</concept>
</ccs2012>
\end{CCSXML}

% \ccsdesc[500]{Computer systems organization~Embedded systems}
% \ccsdesc[300]{Computer systems organization~Redundancy}
% \ccsdesc{Computer systems organization~Robotics}
% \ccsdesc[100]{Networks~Network reliability}

\ccsdesc[500]{Information systems~Query reformulation}

\keywords{Query Reformulation, Code Search, Self-supervised Learning}

\maketitle

\section{Introduction}

% 突出有监督方法现在的局限性,强调本文motivation
Searching through a vast repository of source code has been an indispensable activity for developers throughout the software development process~\cite{XiaBLKHX17}. The objective of code search is to retrieve and reuse code snippets from existing projects that align with a developer's intent expressed as a natural language query \cite{YanYCSJ20}.
%The results of code search depend largely on the quality of the query.
However, it has been observed that developers often struggle to articulate their information needs optimally when submitting queries~\cite{KoMCA06, eberhart2022generating}. 
This difficulty may arise from factors such as inconsistent terminology used in the query or a limited understanding of the specific domain in which information is sought. 
Developers may constantly reformulate their queries until the queries reflect their real query intention and retrieve the most relevant code snippets. 
Studies~\cite{sequer} have shown that in Stack Overflow, approximately 24.62\% of queries on Stack Overflow have undergone reformulation. Moreover, developers, on average, reformulate their queries 1.46 times before selecting a particular result to view.
%The query reformulation process is tedious for developers, especially for novices.

One common solution to this problem is automatic query reformulation, namely, rephrasing a given query into a more comprehensive alternative~\cite{HaiducBMOLM13, liu2022formulate}. 
A natural first way to accomplish this objective is to replace words in a query with synonyms based on external knowledge such as WordNet and thesauri \cite{sesimilarwords, SWordNet, lusearch, li2022cooperative}. However, this methodology restricts the expansion to the word level. Besides, gathering and maintaining domain knowledge is usually costly. The knowledge base might always lag behind the fast-growing code corpora. 
There have been other attempts that consider pseudo-relevance feedback, i.e., emerging keywords in the initial search results \cite{nlp2api, pseudo, huang2017query, zhu2022lol}. They search for an initial set of results using the original query, select new keywords from the top \textit{k} results using TF-IDF weighting, and finally expand the original query with the emerging keywords. 
%For example, Huang \etal~\cite{huang2017query} identify the possible intentions of the user from the initial search results and revise the user’s query using the intention words. 
Nevertheless, despite expanding queries at a word level, this approach also has a risk of expanding queries with noisy words. Hence, the expanded query can be semantically irrelevant to the original one.

In recent years, driven by the prevalence of deep learning, researchers seek the idea of casting query reformulation as a machine translation task: the original query is taken as input to a neural sequence-to-sequence model and is translated into a more comprehensive alternative~\cite{sequer}. 
Despite showing substantial gains, such models require to be trained on a large-scale parallel corpus of query pairs (\ie, the original query and a reformulated query). Unfortunately, acquiring large query pairs is infeasible given that real-world search engines (\eg, Google and Stack Overflow) do not publicly release the evolution of queries. For example, the state-of-the-art method SEQUER \cite{sequer} relies on a confidential parallel dataset that cannot (likely to be impossible) be accessed by external researchers. Replicating the performance of SEQUER becomes challenging or even impossible for those who lack access to such privileged datasets. This lack of replicability hampers the wider adoption and evaluation of the method by the research community.

%Our Work
In this paper, we present \ourmethod, a self-supervised query reformulation method that achieves competitive performance to the state-of-the-art supervised approaches, while not relying on the availability of parallel query data for supervision. Inspired by the pre-trained models, \ourmethod automatically acquires the supervision of query expansion through self-supervised training on a large-scale corpus of code comments.  
Specifically, we design a new pre-training objective called \textit{corrupted query completion} (CQC) to simulate the query expansion process. CQC masks keywords in long, comprehensive queries and asks the model to predict the missing contents. In such a way, the trained model is encouraged to expand incomplete queries with keywords.
\ourmethod leverages T5~\cite{t5}, the state-of-the-art language model for code. The methodology of \ourmethod involves a two-step process. Firstly, T5 is pre-trained using the CQC objective on a vast unannotated corpus of queries. This pre-training phase aims to equip T5 with the ability to predict masked content within queries.
When presented with a query to be reformulated, \ourmethod enumerates potential positions within the query that can be expanded. It then utilizes the pre-trained T5 model to generate appropriate content to fill these identified positions. Subsequently, \ourmethod employs an information gain criterion to select the expansion positions that contribute the most valuable information to the original query, resulting in the reformulated query.

We evaluate \ourmethod on two search engines through both automatic and human evaluations, and compare with state-of-the-art approaches, including SEQUER~\cite{sequer}, NLP2API~\cite{nlp2api}, LuSearch~\cite{lusearch}, and GooglePS~\cite{GooglePS}. Experimental results show that \ourmethod improves the MRR score by over 50\% compared with the unsupervised baselines and gains competitive performance over the fully-supervised approach. 
Human evaluation reveals that our approach can generate more natural and informative queries, with improvements of 19.31\% and 26.35\% to the original queries, respectively. %Experiments on the code question answering task demonstrate the good generality of our approach to other code tasks. 

%Contribution
Our contributions are summarized as follows:
\begin{itemize}
    \item To the best of our knowledge, \ourmethod is the first self-supervised query reformulation approach, which does not rely on a parallel corpus of reformulations.
    \item We propose a novel information gain criterion to select the pertinent expansion positions that contribute the most valuable information to the original query.
    \item We perform automatic and human evaluations on the proposed method. % on the widely used CodeSearchNet benchmark. 
    Quantitative and qualitative results show significant improvements over the state-of-the-art approaches.     %significantly outperforms the baseline approaches.
\end{itemize}



\section{Background}
%This section presents the technical background that motivates this work.
\subsection{Code Search}
Code search is a technology to retrieve and reuse code from pre-existing projects~\cite{deepcs,ChaiZSG22,YanYCSJ20}. 
Similar to general-purpose search engines, developers often encounter challenges when attempting to implement specific tasks. In such scenarios, they can leverage a code search engine by submitting a natural language query. The search engine then traverses an extensive repository of code snippets collected from various projects, identifying code that is semantically relevant to the given query. Code search can be broadly classified into two categories: search within the context of a specific project or as an open search across multiple projects. The search results may include individual code snippets, functions, or entire projects. 


\subsection{Query Reformulation}
\begin{figure}
 \centering
  \includegraphics[width=0.8\linewidth, trim=30 10 30 10 clip]{figures/GooglePS.pdf}
  \caption{An example of Google query reformulation.}
  \label{fig:QueryReformulationExample}
%\vspace{-3mm}
\end{figure}

\begin{figure*}[ht]
 \centering
  \includegraphics[width=0.8\linewidth, trim=0 10 0 5, clip]{figures/framework.pdf}
  \caption{An illustration of the main pipeline}
  \label{fig:framework}
\end{figure*}
  %\wan{1. If possible, please make font not too smaller than the paper body. 2. At the first glance, ``T5'' looks like unrelated to the black box. A possible fix is to put T5 on one side of the box and then draw the lines as $\swarrow \searrow$. If you want to keep the current line style, you may also increase the size of black box to make it close to T5 (only if it is the core component of \ourmethod).  3. ``code Search'' should have same font as ``Span Generation'', unless you want to emphasis it.}

 \begin{figure}[h]
 \centering
  \includegraphics[width=0.85\linewidth, trim=10 10 10 40 clip]{figures/T5.pdf}
  \caption{Illustration of T5}
  \label{fig:T5Structure}
\end{figure}

Query reformulation provides an effective way to enhance the performance of search engines~\cite{zhu2022lol,li2022cooperative,chen2021towards}. 
The quality of queries is often a bottleneck of search experience in web search~\cite{chen2021towards}. This is because the initial query entered by the user is often short, generic, and ambiguous. Therefore, the search results could hardly meet the specific intents of the user. This requires the user to revise his query through multiple rounds. Query reformulation is a technology that reformulates user's queries into more concrete and comprehensive alternatives~\cite{huang2009analyzing}.
Figure~\ref{fig:QueryReformulationExample} shows an example of query reformulation in Google search engine. When a user enters the query ``convert string'' in the search box, there may exist multiple possible intents, such as ``convert something to a string'' or ``convert a string to something''. Additionally, the specific programming language for implementing the conversion function is not specified. In such cases, conventional search engines like Google face challenges in accurately determining the user's true intent. 

To address this issue, search engines often employ tools like the Google Prediction Service (GooglePS). GooglePS automatically suggests multiple reformulations of the original query. These reformulations provide alternative options that the user can consider to refine their search. By presenting a range of reformulations, users can narrow down their search target by selecting the most relevant reformulation that aligns with their intended query. This process helps users in finding more precise and tailored search results.


Query reformulation broadly encompasses various techniques, including query expansion, reduction, and replacement~\cite{jansen2009patterns}. While query expansion involves augmenting the original query with additional information, such as synonyms and related entities, to enhance its content, query reduction focuses on eliminating ambiguous or inaccurate expressions. Query replacement, on the other hand, involves substituting incorrect or uncommon keywords in the original query with more commonly used and precise terms. Among these types, query expansion constitutes the predominant approach, accounting for approximately 80\% of real-world search scenarios~\cite{sadowski2015developers}.




\subsection{Self-Supervised Learning and Pre-trained Models}

Supervised learning is a class of machine learning methods that train algorithms to classify data or predict outcomes by leveraging labeled datasets.
It is known to be expensive in manual labeling, and the bottleneck of data annotation further causes generalization errors~\cite{jakubovitz2019generalization}, spurious correlations~\cite{kronmal1993spurious}, and adversarial attacks~\cite{madry2017towards}. 
Self-supervised learning alleviates these limitations by automatically mining supervision signals from large-scale unsupervised data using auxiliary tasks \cite{Self-supervised-Learning}. This enables a neural network model to learn rich representations without the need for manual labeling \cite{Yang0S22}. For example, the cloze test masks words in an input sentence and asks the model to predict the original words. In this way, the model can learn the semantic representations of sentences from large unlabeled text corpora. 
%There are three main techniques for creating self-supervision: generative, contrastive, and generative-contrastive~\cite{ericsson2022self}. The generative methods train an encoder to encode input x into an explicit vector z and a decoder to reconstruct x from z (\eg, the cloze test, graph generation). Contrastive methods train an encoder to encode input x into an explicit vector z to measure similarity (\eg, mutual information maximization, instance discrimination). Generative-contrastive methods train a decoder to generate fake samples and a discriminator to distinguish them from real samples (\eg, GAN~\cite{GAN})%

%PLM
Pre-trained language models (PLMs) such as BERT~\cite{BERT}, GPT~\cite{GPT}, and T5~\cite{t5} are the most typical self-supervised learning technology. A PLM aims to learn language's generic representations on a large unlabeled corpus and then transfer them to specific tasks through fine-tuning on labeled task-specific datasets. This requires the model to create self-supervised learning objectives from the unlabeled corpora. 
% Take the T5 (Text-to-Text Transfer Transformer)~\cite{t5} as an example. %T5 aims to learn representations of texts on unlabeled corpora.  
% As illustrated in Figure~\ref{fig:T5Structure}, T5 employs the Transformer~\cite{vaswani2017attention} architecture where an encoder takes a text as input and outputs the encoded vector. 
Take the Text-to-Text Transfer Transformer (T5)~\cite{t5} in Figure~\ref{fig:T5Structure} as an example. T5 employs the Transformer~\cite{vaswani2017attention} architecture where an encoder accepts a text as input and outputs the encoded vector. 
A decoder generates the target sequence based on the encodings. To efficiently learn the text representations, T5 designs three self-supervised pre-training tasks, namely, masked span prediction, masked language modeling, and corrupted sentence reconstruction.
By pre-training on large-scale text corpora, T5 achieves state-of-the-art performance in a variety of NLP tasks, such as sentence acceptability judgment~\cite{zomer2021beyond}, sentiment analysis~\cite{pipalia2020comparative}, paraphrasing similarity calculation~\cite{nighojkar2021improving}, and question answering~\cite{jiang2021can}.
%\vspace{2mm}

%\section{A Motivating Example}
%\gu{The reviewers (both 1 and 2) suggest providing a usage scenario of SSQR}
%\wan{How about using ``A motivated example'' as section heading? I also feel placing this section after background would be a bit better}



\section{Method}

The primary focus of this paper is on query expansion, the most typical (accounting for 80\%) technique for query reformulation. Query expansion aims to insert key phrases into a query thereby making it more specific and comprehensive. %This amounts to solving a \textit{pinpoint-then-expand} problem: for a given query, we find a position where information is likely to be missing and generate a span of keywords to fill it into. 
Essentially, query expansion addresses a \textit{pinpoint-then-expand} problem, wherein the goal is to identify potential information gaps within a given query and generate a set of keywords to fill those gaps. 

Inspired by the masked language modeling (MLM) task introduced by pre-trained models like BERT~\cite{BERT}, our proposed method adopts a self-supervised idea. Specifically, we mask keywords within complete code search queries and train a model to accurately predict and recover the masked information. This allows the model to learn the underlying patterns and relationships within the queries, enabling it to generate meaningful expansions for query reformulation.

\begin{figure*}[t]
 \centering
  \includegraphics[width=1.0\linewidth]{figures/Example.pdf}
  \caption{A working example of each expansion step}
  \label{fig:example}
\end{figure*}

\subsection{Overview}
%\vspace{1mm}
Figure~\ref{fig:framework} shows the main framework as well as the usage scenario of our method. The pipeline involves two main phases: an offline pre-training phase and an online expansion phase. 
During the pre-training phase, \ourmethod continually pre-trains a PLM named T5~\cite{t5} with a newly designed \textit{corrupted query completion} task on an unlabelled corpus of long queries (\S \ref{ss:approach:pretrain}). This enables the T5 to learn how to expand incomplete queries into longer ones. 
%The usage stage involves three steps, namely, expanding candidate spans, selecting expansion positions, and code search. 
During the runtime of \ourmethod, when a user presents a query for code search, \ourmethod employs a two-step process for query expansion. Firstly, it enumerates candidate positions within the query that can be expanded and utilizes the pre-trained T5 model to generate content that fills these positions (as discussed in \S \ref{ss:approach:step2}). 
Following the expansion step, \ourmethod proceeds to select the position that offer the highest information gain after the expansion (introduced in \S \ref{ss:approach:selection}). This selection process ensures that the most valuable and informative expansions are chosen, thereby enhancing the reformulated query in terms of its relevance and comprehensiveness.

Finally, once the query has been expanded, users conduct code search by selecting the most relevant reformulation that aligns with their intended query. Our approach specifically focuses on the function-level code search scenario, which involves the retrieval of relevant functions from a vast collection of code snippets spanning multiple projects.



The following sections elaborate on each step of our approach respectively. 

\subsection{Pre-training T5 with Corrupted Query Completion}
\label{ss:approach:pretrain}



We start by pre-training a PLM which can predict the missing span in a query. 
We take the state-of-the-art T5~\cite{t5} as the backbone model since it has a sequence-to-sequence architecture and is more compatible with generative tasks. Besides, T5 is specialized in predicting masked spans (\ie, a number of words). 

To enable T5 to learn how to express a query more comprehensively, we design a new pre-training objective called \textit{corrupted query completion} (CQC) using a large-scale corpus of unlabelled queries. 
Similar to the MLM objective, CQC randomly masks a span of words in the query and asks the model to predict the masked span. 
More specifically, given an original query $q=\left(w_{1}, \ldots,  w_{n}\right)$ that consists of a sequence of $n$ words, \ourmethod masks out a span of 15\%$\times$$n$ consecutive words from a randomly selected position $i$, namely, $s_{i: j}$=$\left(w_{i}, \ldots, w_{j}\right)$, and replaces it with a \texttt{[MASK]} token. 
Then, the corrupted query is taken as input to T5 which predicts the words in the masked span. 
We use the teacher-forcing strategy for pre-training. When predicting a word in the corrupted span, the context visible to the model consists of two parts: 1) the uncorrupted words in the original query, denoted as $q_{\backslash s_{i: j}}=\left(w_{1}, \ldots, w_{i-1}, w_{j+1}, \cdots, w_{n}\right)$; and 2) the ground truth words appeared before the current predicting position $w_{t}$, denoted as $w_{i: t-1}$. We pre-train the model using the cross-entropy loss, namely, minimizing
\begin{equation}
\mathcal{L}_{\mathrm{cqc}}=-\sum_{t=i}^{j} \log p\left(w_{t} \mid q_{\backslash s_{i: j}}, w_{i: t-1}\right).
\end{equation}
%T5's pre-training tags (Targets) pass $<$X$>$, $<$Y$>$, $<$Z$>$ and other special characters with sequence information to align with special identifiers in the input for the requirement of the Text-to-Text framework adopted by T5.
Figure~\ref{fig:example}(a) shows an example of the CQC task. For a query ``how to reverse an array in Java'' taken from the training corpus, the algorithm corrupts the query by replacing the modifier ``in Java'' with \texttt{[MASK]}. The corrupted query is taken as input to T5 which predicts the original masked tokens ``in Java''.


% %\vspace{1mm}
% \subsection{Predicting Expansion Positions} 
% %\vspace{1mm}
% \begin{algorithm}[t]
%     \caption{Predicting Expansion Positions}
%     \label{algorithm:PredictingMaskPosition}
%     \LinesNumbered
%     \KwIn {$q$: the input query written in natural language\;
%     T5: the pre-trained T5 model\;
%     $k$: the number of candidate positions.}
%     \KwOut {$\mathbf{Q}$: a set of candidate masked queries.}
%     $w_1,...,w_n$ = tokenize($q$)\;
%     $\mathbf{Q}$ = \{\}; // initialize the query set\\
%     \For{i = 1 to n}{
%         $\tilde{q}$ = $w_1,...,w_{i}$, \texttt{[MASK]}, $w_{i+1},...,w_n$;
%         ~~// insert a \texttt{[MASK]} token to the $i$-th position in $q$\\
%         $v_{1},...,v_{m}$ = T5($\tilde{q}$);~~// predict the content for the \texttt{[MASK]} token;\\
%         $\epsilon$ = - $\frac{1}{m}\sum_{j=1}^m p(v_j)\mathrm{log}p(v_j)$;~~// calculate the entropy for the prediction. \\
%         $\mathbf{Q} = \mathbf{Q}\cup\langle\tilde{q},\epsilon\rangle $\;
%     }  
%    % $Sq$ = list of generating confidence score for the candidate masked queries\;
%   %  \For{i = 1 to len($Cq$)}{
%  %        Input the $i$th masked query in $Cq$ into T5 model $m$ and calculate the generating entropy $score$ for the [MASK] token based on the context in the query\;
%    %     Append $score$ to $Sq$\;
%   %  }
%   %  Sort $Sq$ in descending order\;
%     Sort $\mathbf{Q}$ based on entropy in an ascending order\;
%     $\mathbf{Q}$= top$(\mathbf{Q},k)$;~~// select the top-$k$ queries.\\
%     \Return $\mathbf{Q}$
% \end{algorithm}



\subsection{Expanding Candidate Spans} 
\label{ss:approach:step2}
The pre-trained T5 model is then leveraged to expand queries. 
We consider a query that needs to be expanded as an incomplete query where a span of words is missing at a position (denoted as a masked token), we want the model to generate a sequence of words to fill in the span. This is exactly the problem of \emph{masked span prediction} as T5 aims to solve. Therefore, we leverage the pre-trained T5 to expand the incomplete queries. 

However, a query with $n$ words have $n$+$1$ positions for expansion. Therefore, we design a \textit{best-first} strategy: we enumerate all the $n$+$1$ positions as the masked spans, perform the CQC task, and select the top-\textit{k} positions that have the most information gain of predictions.
Specially, given a original query $q$ =$\{w_1,w_2,...,w_n\}$, \ourmethod enumerates the $n$+$1$ positions between words. For each position, it inserts a \texttt{[MASK]} token. This results in $n$+$1$ candidate masked queries. 
Each incomplete query $\tilde{q}$=[$w_1,\ldots,$\verb|[MASK]|$,\ldots,w_n$] is taken as input to the pre-trained T5.
The decoder of T5 generates a span of words $s$=[$v_1,\ldots,v_m$] for the \verb|[MASK]| token by sampling words according to the predicted probabilities. 
Finally, \ourmethod replaces the \verb|[MASK]| token with the generated span $s$, yielding the reformulated query.

Figure~\ref{fig:example}(b) shows an example. Given the first two masked queries, the T5 model generates ``in Java'' and ``integer'' for the masked tokens, respectively.
The former refers to the language used to implement the function, while the latter refers to the data type of the target data structure. The reformulated queries supplement the original queries with additional information, revealing user potential intents from different aspects. %, resulting in more accurate search results.


\subsection{Selecting Expansion Positions} 
\label{ss:approach:selection}

%\vspace{1mm}
\begin{algorithm}[t]
    \caption{Span Expansion and Selection}
    \label{algorithm:PredictingMaskPosition}
    \LinesNumbered
    \KwIn {$q$: the input query written in natural language\;
    ~~~~~~~T5: the pre-trained T5 model\;
    ~~~~~~~~\,$k$: the number of candidate positions\;
    ~~~~~~~$m$: the maximum target length.
    }
    \KwOut {$\mathbf{Q}$: a set of candidate masked queries.}
    $w_1,...,w_n$ = tokenize($q$)\;
    $\mathbf{Q}$ = \{\}; \codecomment{initialize the query set.} \\
    \For{i = 1 to n}{
        $\tilde{q}$ = $w_1,...,w_{i}$, \texttt{[MASK]}, $w_{i+1},...,w_n$;
        ~~\codecomment{insert a}\\\hfill \texttt{[MASK]} token to the $i$-th position in $q$.\\
        $v_{1},...,v_{m}$ = T5($\tilde{q}$);~~ \codecomment{predict the content for the}\\\hfill \texttt{[MASK]} token.\\
        $IG$ = $\frac{1}{m}\sum_{j=1}^m p(v_j)\mathrm{log}p(v_j)$;~~\codecomment{calculate the}\\\hfill information gain for the expansion. \\
        $\mathbf{Q} = \mathbf{Q}\cup\langle\tilde{q},IG\rangle $\;
    }  
    Sort $\mathbf{Q}$ based on entropy in an descending order\;
    $\mathbf{Q}$= top$(\mathbf{Q},k)$;~~\codecomment{ select the top-$k$ queries.}\\
    \Return $\mathbf{Q}$
\end{algorithm}

A query with $n$ words has $n$+$1$ candidate positions for expansion, but not all of them are necessary for expansion. Hence, we must determine which positions are the most proper to be expanded. 
\ourmethod selects the top-\textit{k} candidate queries that have the most missing information in the masked span. 
The resulting expanded queries are more likely to gain information after span filling. 

%\begin{itemize}
%    \item [1)] \textit{Enumerating expansion positions}:
%    The algorithm enumerates expansion positions. For each position, it inserts a \verb|[MASK]| token and produces a candidate masked query (Line 4).
%    \item [2)] \textit{Calculating the generating confidences}:
%    The pre-trained T5 model calculates the entropy for the predicted span for each candidate (Line 5-7).
%    \item [3)] \textit{Selecting the top-K candidates}:
%    The algorithm chooses the top-K candidates with the highest generating confidence, \ie, the lowest entropy (Line 8-10).
%\end{itemize}

The key issue here is how to measure the information gain after filling each span. In our approach, we define \emph{information gain} for a span expansion as the negative entropy over the predicted probability distribution of the generated words~\cite{jiang2015relative, luukka2011feature}. In information theory, entropy~\cite{renyi1961measures} characterizes the uncertainty of an event in a system. Suppose the probability that an event will happen follows a distribution of ($p_1,\ldots,p_n$), the entropy of the event can be computed as $-\sum_{i=1}^np_i\mathrm{log}p_i$. The lower the entropy, the more certain that the event can happen. That means the event brings more information to humans.
This can be analogized to the span prediction problem: when the probability distribution of the generated words over the vocabulary is uniform, the entropy (uncertainty) becomes high because every word is likely to be generated. By contrast, smaller entropy means that there is a greatly different likelihood of generating each word and thus the certainty of the generation is high. The lower the entropy, the higher the certainty that this span contains the word, and the more information that the expansion brings to the query.
If the span contains multiple words, we can measure the information gain of the span prediction using their average negative entropy.  

For each candidate query $\tilde{q}$, we predict a span $s$=$[v_1,...,v_m]$ using the pretrained T5 model: 
\begin{equation}
p(v_i)=T5(\tilde{q}, v_{<i}), i=1,\ldots,L
\end{equation}
where each $v_i$ denotes a sub-token in the predicted span.
Each prediction $p(v_i)$ follows a probability distribution of $p_1,\ldots,p_{|V|}$ over the vocabulary of the entire set of queries in the training corpus, indicating the likelihood of each token in the vocabulary appearing in the span. 

Our next step is to compute the information gain of each expansion using negative entropy: For each sub-token $v_i$, the information gain can be calculated by 
\begin{equation}
IG(v_i) = -H(v_i)=\sum_{v=1}^{|V|}p(v_i=v)\mathrm{log} p(v_i=v). 
\end{equation}
The higher the IG, the more certainty that the prediction is. For a predicted span $s$ = [$v_1,\ldots,v_m$] with $m$ sub-tokens, we compute the average IG of all its tokens, namely, 
\begin{equation}
IG(s) = \frac{1}{m}\sum_{i=1}^LIG(v_i). 
\end{equation}

Finally, we select the top $K$ expansions with the highest information gain and then replace the \texttt{[MASK]} token with the predicted span. The top-$k$ expansions are provided to users for choosing the most relevant one that aligns with their intention. 
The specific details of the method are summarized in Algorithm~\ref{algorithm:PredictingMaskPosition}.

Figure~\ref{fig:example}(c) shows an example query expansion. For a given query ``convert string to list'' to be expanded, \ourmethod firstly enumerates five expansion positions of the original query, inserting a \texttt{[MASK]} token into each one. Next, the pre-trained T5 model takes these candidate queries as input and calculates the information gain from the prediction for these candidate queries. Finally, \ourmethod recommends the top-2 masked queries (here \textit{k}=2) with the highest information gain (\ie, minimum entropy values) for users to choose. 


\section{Experimental Setup}

\subsection{Research Questions}
We evaluate the performance of \ourmethod in query reformulation through both automatic and human studies. We further explore the impact of different configurations on performance. %Finally, we investigate the effectiveness of \ourmethod in other code intelligence tasks such as code question answering.  
In specific, we address the following research questions:

\begin{itemize}
    \item \textbf{RQ1: How effective is \ourmethod in query reformulation for code search?}
    
    We apply query reformulation to Lucene and CodeBERT based code search engines and compare the search accuracy before and after query reformulation by various approaches.
    
    \item \textbf{RQ2: Whether the queries reformulated by \ourmethod are more contentful and easy to understand?}
    
    In addition to the automatic evaluation of code search performance, we also want to assess the intrinsic quality of the reformulated queries. To this end, we perform a human study to assess whether the reformulated queries contain more information than the original ones and meanwhile conform to human reading habits. 
    
    \item \textbf{RQ3: How do different configurations impact the performance of \ourmethod?}

    To obtain a better insight into \ourmethod, we investigate the performance of \ourmethod under different configurations. We firstly investigate the effect of different positioning strategies, i.e., what is the best criteria to select the expansion position in the original query. We are also interested in the number of expansions for each query. 

    %\item \textbf{RQ4: How effective is \ourmethod for code question answering?}
    
    %Besides code search, there are other tasks (\eg, code question answering) that also get benefits from query reformulation. We wonder whether our approach can have the same effect on other tasks. Hence, we apply \ourmethod to code question answering and evaluate the accuracy improvement after query reformulation.   
\end{itemize}


\subsection{Datasets}
We pre-train, fine-tune, and test all models using two large code search corpora: CODEnn~\cite{deepcs} and CodeXGLUE~\cite{CodeXGlue}. They are non-overlapping and thus alleviate the duplicated code issue between pre-training and downstream tasks~\cite{Allamanis19}.

\smallskip\textbf{Dataset for Pre-training}. We pre-train T5 using code comments from the large-scale CODEnn dataset. CODEnn has been specifically processed for code search. %We download it from its Github repository \cite{DeepCodeSearch}\wan{Do we really need to mention this? It seems too detailed} and extract the ``description" attribute for each instance. According to the original paper, it was generated by parsing AST from code methods and extracting the code comments from the AST. The original data was compressed in a numerical format, so we also decode it into texts using the provided vocabulary. 
Compared to CodeSearchNet~\cite{CodeSearchNet}, this dataset has a much larger volume (i.e., more than 10 million queries). We take the first 1 million for pre-training.

\smallskip\textbf{Dataset for Code Search}. We use the code search dataset of CodeXGLUE which provides queries and the corresponding code segments from multiple projects. %which is initially provided in the CodeSearchNet benchmark\wan{If we say ``initial'', then the reader would expect description of ``later''. How about we simply remove this clause?}. %{It contains both queries and the corresponding ground-truth code segments, while other baselines just provide tool demos or executable programs without training datasets.} 
In this dataset, each record has five attributes, including the code segment (in Python), repository URL, code tokens, doc string (i.e., NL description of the function), and the index of the code segment. 
We split the original dataset into training and test sets, with 251,820 and 19,210 samples respectively.
The training set is used to fine-tune the CodeBERT search engine, while the test set is used as the search pool from which a search engine retrieves code. All queries in the test phase are tests on the same pool.
%\reviewer{what’s the size of the corpus per query? Is it the same corpus for all the queries (in both the training and testing)? Do the queries come from different projects?}
The statistics of our datasets are summarized in Table~\ref{tab:dataset}.

\begin{table}[!t]
\caption{\rmfamily Statistics of Datasets}
\rmfamily
\centering
\label{tab:dataset}
\begin{threeparttable}
\begin{tabular}{lcc}
\toprule
\textbf{Stage} & \textbf{Dataset} & \textbf{\# of Samples}   \\ \bottomrule
Pre-training                 & CODEnn~\cite{deepcs} & 1,000,000  \\
Fine-tuning                 & CodeXGLUE~\cite{CodeXGlue} &  251,820 \\
Search    &   CodeXGLUE~\cite{CodeXGlue} & 19,210  \\ \bottomrule
\end{tabular}
\end{threeparttable}
\vspace{-3mm}
\end{table}


\subsection{Implementation Details}
\textbf{Implementation of the pre-trained model}. Our model is implemented based on \verb|T5-base| from the open-source collection of HuggingFace~\cite{t5base}. We use the default tokenizer and input-output lengths.
Since HuggingFace does not provide an official PyTorch script for T5 pre-training, we implement the pre-training script based on the PyTorch Lightening framework~\cite{PyTorch-Lightning}. We initialize T5 with the default checkpoint provided by Huggingface and continually pre-train it with our proposed CQC task. The pre-training takes 3 epochs with a learning rate of 1e-3. The batch size is set to 32 in all experiments. We set $m$ and $k$ in Algorithm 1 to 10 and 3 respectively.
Since T5 is non-deterministic, \ourmethod can generate different queries for an input query at each run. To guarantee the same output at each run, we fix the random seeds to 101 and reload the same \emph{state-dict} of T5.

\smallskip\textbf{Implementation of the search engines}. We experiment under two search engines based on CodeBERT~\cite{feng2020codebert} and Lucene~\cite{Lucene}. 

1) \textit{A CodeBERT-based search engine}: As our approach is built on pre-trained models, we first verify the effectiveness on a pre-training based search engine. Specifically, we test our approach on the default search engine by CodeBERT. We reuse the implementation of the code search (\ie, Text-Code) task in CodeXGLUE~\cite{CodeXGlue}. Then, we fine-tune the CodeBERT-base checkpoint on a training set from CodeXGLUE for 2 epochs with a constant learning rate of 5e-5. 

2) \textit{The Lucene search engine}: Besides the pre-training based search engine, we also test our approach on a classic search engine named Lucene~\cite{Lucene}. Lucene is a keyword-based search library that is widely adapted to a variety of code search engines and platforms~\cite{solr, indextank, gormley2015elasticsearch}. We implement the Lucene search engine based on the Lucene core in Java. We extract the code segments from the test dataset from CodeXGLUE, parse them by Lucene's StandardAnalyzer, and build their indexes. 

We train all models on a Linux server with Ubuntu 18.04.1 and a GPU of Nvidia GeForce RTX 2080 Ti.


\subsection{Baselines}
We compare our method with the state-of-the-art query reformulation approaches, including a supervised method called SEQUER, and unsupervised methods such as NLP2API, LuSearch, and GooglePS. 
\begin{itemize}
\item[1)] \textbf{Supervised}~\cite{sequer}: a supervised learning approach for query reformulation named SEQUER. SEQUER leverages Transformer~\cite{vaswani2017attention} to learn the sequence-to-sequence mapping between the original and the reformulated queries. The method relies on a confidential parallel dataset of query evolution logs provided by Stack Overflow. The dataset contains internal HTTP requests processed by Stack Overflow’s web servers within one year. %Each request involves the user information, the event type (\eg, searching, post visiting, and question list browsing), and the event content.
\item[2)] \textbf{NLP2API}~\cite{nlp2api}: a feedback based approach that expands query with recommended APIs.
NLP2API automatically identifies relevant API classes collected from Stack Overflow using keywords in the initial search results and then expands the query with these API classes.
\item[3)] \textbf{LuSearch}~\cite{lusearch}: a knowledge-based approach that expands a query with synonyms in WordNet~\cite{WordNet}. The reformulated queries by LuSearch are based on Lucene's structural syntax, which are too long and contain too many Lucene-specific keywords. 
Due to the constraint on the input length of T5 model (\ie, 512 tokens), we keep the synonyms and remove keywords about attribute names in Lucene such as ``methbody'' and ``methname''.
\item[4)] \textbf{GooglePS}~\cite{GooglePS}: the Google query prediction service that gives real-time suggestions on query reformulation. We directly enter test queries into the Google search box and manually collect the reformulated queries in RQ2 and the case study. We do not compare our method with GooglePS in RQ1 because its search API is unavailable to us for processing a large number of queries. Besides, our baseline model has demonstrated a great improvement over it in terms of MRR~\cite{sequer}.

\end{itemize}


\subsection{Evaluation Metrics}
The ultimate goal of query reformulation is to enhance search accuracy by using the reformulated queries. In our experiments, we first evaluate the search accuracy measured by the widely used mean reciprocal rank (MRR). MRR is defined as the average of the reciprocal ranks (\ie, the multiplicative inverse of the target post’s rank) of the search results for all the queries, namely,
\begin{equation}
MRR=\frac{1}{Q} \sum_{i=1}^{|Q|} \frac{1}{rank_{i}}
\end{equation}
where $Q$ refers to a set of queries and  $rank_{i}$ stands for the position of the first relevant document for the $i$-th query. A higher MRR indicates better search performance. 

Besides the indirect criteria in search performance, query reformulation also aims to help users write more precise and high-quality queries. Therefore, we further define two metrics to measure the intrinsic quality of the reformulated queries:
\begin{itemize}
\item \textit{Informativeness} measures how much information a query contains that contributes to code search. We use this metric to evaluate how much information gain the reformulation brings to the original query.
\item \textit{Naturalness} measures how well a query is grammatically correct and follows human reading habits. By using this metric, we want the reformulation to be semantically coherent with the original query.
\end{itemize}

Both metrics range from 1 to 5. Higher scores indicate better performance.

\section{Results}

\subsection{RQ1: Performance on Code Search}

As the ultimate goal of query reformulation, we first evaluate whether the reformulated queries by \ourmethod lead to better code search performance. 
We experiment under both search engines and compare the improvement of MRR scores before and after query reformulation by various methods. For each query, we calculate its similarity to code instances in the test set. The top 100 instances with the highest similarity are selected as the search results. Each query has one ground-truth code instance in the test set. We calculate the MRR scores by comparing the results and the ground-truth code.
Then, for each method, we select the first three reformulations and report the highest MRR score among them. 
Since the purpose of query reformulation is to hit the potential search intent of the user. We believe that results with the maximum MRR in the top-$k$ reformulations are the most likely to satisfy this goal and are therefore considered meaningful.

\subsection{Performance estimation}
\label{sec:performance}

In this section, we describe the first phase of Elo-MMR. For notational convenience, we assume all probability expressions to be conditioned on the \textbf{prior context} $P_{i,< t}$, and omit the subscript $t$.

Our prior belief on each player's skill $S_i$ implies a prior distribution on $P_i$. Let's denote its probability density function (pdf) by
\looseness=-1
\begin{equation}
\label{eq:perf-prior} 
f_i(p) := \Pr(P_i = p) = \int \pi_i(s) \Pr(P_i = p \mid S_i=s) \,\mathrm{d}s,
\end{equation}
where $\pi_i(s)$ was defined in \Cref{eq:pi-s}. Let
\[F_i(p) := \Pr(P_i\le p) = \int_{-\infty}^p f_i(x) \,\dx,\]
be the corresponding cumulative distribution function (cdf). For the purpose of analysis, we'll also define the following ``loss'', ``draw'', and ``victory'' functions:
\begin{align*}
l_i(p) &:= \ddp\ln(1-F_i(p)) = \frac{-f_i(p)}{1 - F_i(p)},
\\d_i(p) &:= \ddp\ln f_i(p) = \frac{f'_i(p)}{f_i(p)},
\\v_i(p) &:= \ddp\ln F_i(p) = \frac{f_i(p)}{F_i(p)}.
\end{align*}

Evidently, $l_i(p) < 0 < v_i(p)$. Now we define what it means for the deviation $P_i - S_i$ to be log-concave.
\begin{definition}
\label{def:log-concave}
An absolutely continuous random variable on a convex domain is \textbf{log-concave} if its probability density function $f$ is positive on its domain and satisfies
\[f(\theta x + (1-\theta) y) > f(x)^\theta f(y)^{1-\theta},\;\forall\theta\in(0,1),x\neq y.\]
\end{definition}

We note that log-concave distributions appear widely, and include the Gaussian and logistic distributions used in Glicko, TrueSkill, and many others. We'll see inductively that our prior $\pi_i$ is log-concave at every round. Since log-concave densities are closed under convolution~\cite{concave}, the independent sum $P_i=S_i+(P_i-S_i)$ is also log-concave. The following lemma (proved in the appendix) makes log-concavity very convenient:
\begin{lemma}
\label{lem:decrease}
If $f_i$ is continuously differentiable and log-concave, then the functions $l_i,d_i,v_i$ are continuous, strictly decreasing, and
\[l_i(p) < d_i(p) < v_i(p) \text{ for all }p.\]
\end{lemma}

For the remainder of this section, we fix the analysis with respect to some player $i$. As argued in \Cref{sec:bayes_model}, $P_i$ concentrates very narrowly in the posterior. Hence, we can estimate $P_i$ by its MAP, choosing $p$ so as to maximize:
\[\Pr(P_i=p\mid E^L_i,E^W_i) \propto f_i(p) \Pr(E^L_i,E^W_i\mid P_i=p).\]

Define $j\succ i$, $j\prec i$, $j\sim i$ as shorthand for $j\in E^L_i$, $j\in E^W_i$, $j\in \mathcal P\setminus (E^L_i\cup E^W_i)$ (that is, $P_j>P_i$, $P_j<P_i$, $P_j=P_i$), respectively. The following theorem yields our MAP estimate:
\begin{theorem}
\label{thm:uniq-max}
Suppose that for all $j$, $f_j$ is continuously differentiable and log-concave. Then the unique maximizer of $\Pr(P_i=p\mid E^L_i,E^W_i)$ is given by the unique zero of
\[Q_i(p) := \sum_{j \succ i} l_j(p) + \sum_{j \sim i} d_j(p) + \sum_{j \prec i} v_j(p).\]
\end{theorem}
The proof is relegated to the appendix. Intuitively, we're saying that the performance is the balance point between appropriately weighted wins, draws, and losses. Let's look at two specializations of our general model, to serve as running examples in this paper.

\paragraph{Gaussian performance model}
If both $S_j$ and $P_j-S_j$ are assumed to be Gaussian with known means and variances, then their independent sum $P_j$ will also be a known Gaussian. It is analytic and log-concave, so \Cref{thm:uniq-max} applies.

We substitute the well-known Gaussian pdf and cdf for $f_j$ and $F_j$, respectively. A simple binary search, or faster numerical techniques such as the Illinois algorithm or Newton's method, can be employed to solve for the maximizing $p$.

\begin{figure}
    \centering
    \includegraphics[width=1.05\columnwidth]{l2-lr-plot.eps}
    \caption{$L_2$ versus $L_R$ for typical values (left). Gaussian versus logistic probability density functions (right).}
    \label{fig:l2-lr-plot}
\end{figure}

\paragraph{Logistic performance model}
Now we assume the performance deviation $P_j-S_j$ has a logistic distribution with mean 0 and variance $\beta^2$. In general, the rating system administrator is free to set $\beta$ differently for each contest. Since shorter contests tend to be more variable, one reasonable choice might be to make $1/\beta^2$ proportional to the contest duration.

Given the mean and variance of the skill prior, the independent sum $P_j = S_j + (P_j-S_j)$ would have the same mean, and a variance that's increased by $\beta^2$. Unfortunately, we'll see that the logistic performance model implies a form of skill prior from which it's tough to extract a mean and variance. Even if we could, the sum does not yield a simple distribution.

For experienced players, we expect $S_j$ to contribute much less variance than $P_j-S_j$; thus, in our heuristic approximation, we take $P_j$ to have the same form of distribution as the latter. That is, we take $P_j$ to be logistic, centered at the prior rating $\mu^\pi_j = \argmax \pi_j$, with variance $\delta_j^2 = \sigma_j^2 + \beta^2$, where $\sigma_j$ will be given by \Cref{eq:variance}. This distribution is analytic and log-concave, so the same methods based on \Cref{thm:uniq-max} apply. 
Define the scale parameter $\bar\delta_j := \frac{\sqrt{3}}{\pi} \delta_j$. A logistic distribution with variance $\delta_j^2$ has cdf and pdf:
\begin{align*}
F_j(x) &= \frac { 1 } { 1 + e^{-(x-\mu^\pi_j)/\bar\delta_j} }
= \frac 12 \left(1 + \tanh\frac{x-\mu^\pi_j}{2\bar\delta_j} \right),
\\f_j(x) &= \frac { e^{(x-\mu^\pi_j)/\bar\delta_j} } { \bar\delta_j\left( 1 + e^{(x-\mu^\pi_j)/\bar\delta_j} \right)^2}
= \frac { 1 } { 4\bar\delta_j} \sech^2\frac{x-\mu^\pi_j}{2\bar\delta_j}.
\end{align*}

The logistic distribution satisfies two very convenient relations:
\begin{align*}
F'_j(x) = f_j(x) &= F_j(x) (1 - F_j(x)) / \bar\delta_j,
\\f'_j(x) &= f_j(x) (1 - 2F_j(x)) / \bar\delta_j,
\end{align*}
from which it follows that
\[d_j(p)
= \frac{1 - 2F_j(p)}{\bar\delta}
= \frac{-F_j(p)}{\bar\delta} + \frac{1 - F_j(p)}{\bar\delta}
= l_j(p) + v_j(p).\]

In other words, a tie counts as the sum of a win and a loss. This can be compared to the approach (used in Elo, Glicko, TopCoder, and Codeforces) of treating each tie as half a win plus half a loss.\footnote{Elo-MMR, too, can be modified to split ties into half win plus half loss. It's easy to check that \Cref{lem:decrease} still holds if $d_j(p)$ is replaced by
$w_l l_j(p) + w_v v_j(p)$
for some $w_l,w_v\in [0,1]$ with $|w_l-w_v|<1$.
In particular, we can set $w_l=w_v=0.5$. The results in \Cref{sec:properties} won't be altered by this change.}

Finally, putting everything together:
\[Q_i(p) = \sum_{j \succeq i} l_j(p) + \sum_{j \preceq i} v_j(p)
= \sum_{j \succeq i} \frac{-F_j(p)}{\bar\delta_j} + \sum_{j \preceq i} \frac{1 - F_j(p)}{\bar\delta_j}.\]
Our estimate for $P_i$ is the zero of this expression. The terms on the right correspond to probabilities of winning and losing against each player $j$, weighted by $1/\bar\delta_j$. Accordingly, we can interpret $\sum_{j\in \cP} (1-F_j(p))/\bar\delta_j$ as a weighted expected rank of a player whose performance is $p$. Similar to the performance computations in Codeforces and TopCoder, $P_i$ can thus be viewed as the performance level at which one's expected rank would equal $i$'s actual rank.


The experimental results are presented in Table \ref{tab:performance}. \ourmethod enhances the MRR by 9.90\% and 12.23\% on the two search engines, respectively. %As can be seen, \ourmethod outperforms the two unsupervised baselines significantly on both search engines. 
Compared to the two unsupervised baselines, LuSearch and NLP2API, it brings a giant leap of over 50\% in search accuracy. 
More surprisingly, \ourmethod achieves competitive results to the supervised counterpart though it is not given with any annotations. 
This indicates that our self-supervised approach can assist developers to write high-quality queries, which ultimately leads to better code search results. 

We notice that the performance of \ourmethod is slightly worse than the supervised counterpart with the Lucene search engine. This is probably because the supervised approach applies fixed expansion patterns to queries and therefore tends to expand queries with common, fixed keywords. These keywords can be easily hit by search engines based on keyword matching (e.g., Lucene). On the contrary, \ourmethod does not use fixed expansion patterns and thus has more various keywords. Lucene is not able to perform keyword matching on them. Instead, CodeBERT, which models the semantic relationships between keywords, can understand queries expanded by \ourmethod.


Another interesting point is that LuSearch and NLP2API do not contribute to the CodeBERT-based search engine. This is probably because both approaches append words to the tail of the original query, hence perturbing the semantics of the original query when we use deep learning based search engines such as CodeBERT.
%CodeBERT performs code search by calculating the semantic similarity between queries and code segments. 

%Under the Lucene search engine, \ourmethod wins the second place, slightly behind SEQUER. 
%Compared to NLP2API and LuSearch, \ourmethod and SEQUER gain the greater improvements of 12.23\% and 16.91\%. It indicates that these DL-based methods perform well on both the grammatical coherence and the rationality of information supplementation, and can be effectively adapted to different search engines.
%NLP2API achieves a 2.87\% boost in terms of MRR, while LuSearch still slightly degrades the search performance. %The reason could be that we implement a simplified version of LuSearch which only  extracts the top k synonyms of the query keywords from the code segment comment corpus and ignores other information. 

%\textit{Answer to RQ1}: \ourmethod improves the MRR score by over 50\% compared with the state-of-the-art knowledge-based approaches and obtains the competitive performance over fully supervised models. 

\subsection{RQ2: Qualitative Evaluation}

% Besides the automatic evaluation, we also perform a human study to evaluate the intrinsic quality of the reformulated queries. We recruited volunteers from our department through email invitations. Six candidates applied for human evaluation.
% They were postgraduates from different labs to ours. We accepted four of them who are in the area of software engineering and natural language processing. All participants have programming experience for over four years and have experience in code search.
To evaluate the intrinsic quality of the reformulated queries, we perform a human study with programmers. Four participants from author's institution, but different labs, are recruited through invitations. All participants are postgraduates in the area of software engineering or natural language processing, having over-four-year programming experience. %\wan{I shrinked the description. How about now?}
We took the first 100 queries from the test set in RQ1, 
%\gu{reviewer: why were the first 100 queries selected? What’s the quality of these queries? Ideally, the queries are of different qualities in terms of content/readability and different effectiveness levels for code search.}
and reformulated them using various methods, including SEQUER, LuSearch, NLP2API, \ourmethod, and GooglePS. We assigned 100 search tasks to human annotators using these 100 queries and present the reformulated queries by various approaches. The annotators were asked to search code using Google and provide their ratings (on a scale of 1 to 5) towards the reformulation in terms of informativeness and naturalness, without knowing the source of the reformulation tool. 

% Table~\ref{tab:HumanEvaluation} summarizes the quality ratings by annotators. We observe that both \ourmethod and SEQUER significantly improve the original query in terms of naturalness and informativeness. This indicates that they are capable of reformulating queries into clearer and more informative alternatives. We notice that LuSearch and NLP2API have a noticeable degradation of naturalness though the informativeness increases a little bit. This is because the two methods directly append relevant APIs or synonyms to the tail of the original query, which is likely to break the coherence of the query. GooglePS reformulates a query through two strategies: if the query is a common prefix of other queries, it will append words to the tail. Otherwise, it suggests common queries that contain keywords in the original query. GooglePS achieves a slight increase in both naturalness and informativeness. Comparatively, the queries reformulated by \ourmethod and SEQUER are more human-like.
Table~\ref{tab:HumanEvaluation} summarizes the quality ratings by annotators. Overall, \ourmethod achieves the most improvement in terms of naturalness (19\%) and informativeness (26\%), showing that it reformulated queries are more human-like. Comparatively, GooglePS and SEQUER have much less improvement. LuSearch and NLP2API even decrease the naturalness and informativeness, as they directly append relevant APIs or synonyms to the tail of the original query, and thus break the coherence of the query.%\wan{Shrinked.}

\begin{table}[!t]
\caption{\rmfamily Human Evaluation Result}%\wan{We use SEQUER in text, but supervised in the table. It seems a bit confusing}}
\rmfamily
\centering
\scalebox{1.0}{
\label{tab:HumanEvaluation}
\begin{threeparttable}
%\setlength\tabcolsep{10pt}
\begin{tabular}{lll}
\toprule
\textbf{Approach} & \textbf{Naturalness} & \textbf{Informativeness}   \\ \bottomrule
No Reformulation                 & 3.21 & 3.15 \\
Supervised~\cite{sequer}                 & 3.63 (+13.08\%) 
& 3.44 (+9.21\%) \\
\hline
LuSearch~\cite{lusearch}                & 2.63 (-18.07\%) & 3.17 (+0.63\%) \\
NLP2API~\cite{nlp2api}                & 2.80 (-12.77\%) & 3.50 (+11.11\%) \\
GooglePS~\cite{GooglePS}     &   3.27 (+1.87\%) &  3.33 (+5.71\%) \\
\ourmethod (ours)                & \textbf{3.83} (+\textbf{19.31\%}) & \textbf{3.98} (+\textbf{26.35\%}) \\ \bottomrule
\end{tabular}
\end{threeparttable}
}
\vspace{-3mm}
\end{table}

In particular, compared with the strong baseline SEQUER, \ourmethod obtains a greater improvement of 17.14\% in terms of informativeness, while outperforming slightly in terms of naturalness.
The main reason could be that SEQUER applies three reformulation patterns, \ie, deleting unimportant words, rewriting typos, and adding keywords, where only the last pattern increases the informativeness of the query.
Besides, SEQUER often adds keywords in a monotonous pattern, such as appending ``in Java'' at the tail of the queries; meanwhile, our method can generate diverse spans at the proper positions of the original queries. 
Consequently, the reformulated queries by our approach are more informative. 


%\textit{Answer to RQ2}: \ourmethod gains 26.35\% and 19.31\% improvements in terms of informativeness and naturalness, respectively, which are 17.14\% and 6.23\% greater than the strong baseline SEQUER.


\subsection{RQ3: Performance under Different Configs}

In this experiment, we evaluate the performance of \ourmethod in code search under different configurations with the CodeBERT search engine. We vary the positioning strategy and the number of candidate positions in order to search for the optimal configuration. 

\smallskip \textbf{Positioning Strategies}. 
% As a main step in our approach, selecting the expansion positions can be critical to the performance. Besides the entropy-based criterion, we also tried two other strategies. We compare three strategies for the selection of expansion positions:
Selecting expansion positions is critical to the performance. We compare three strategies, including the entropy-based criterion:
\begin{itemize}
\item RAND randomly selects \textit{k} positions in the original query for expansion.
\item PROB selects the top-\textit{k} positions that have  the maximum probability while predicting their missing content.
\item ENTR selects the top-\textit{k} positions that have the  minimum entropy while predicting their missing content.
\end{itemize}

The results are shown in Table \ref{tab:ablation}. The PROB and ENTR strategies bring a large improvement (around 10\%) to the code search performance. This indicates that both criteria correctly quantify the missing information at various positions. Between these two strategies, ENTR performs slightly better than PROB, probably because ENTR considers the entire distribution of the prediction while PROB just considers the maximum one. 
As expected, the RAND strategy causes a degradation of 6.68\% in code search performance because it selects expansion positions without any guidance, which results in incorrect or redundant expansions.

\begin{table}[t]
\vspace{-10pt}
\caption{Ablation results, evaluated using F1-score (\%). '\textit{w/o Sep}' excludes the separator from the backbone architecture; '\textit{w/o Decomp}' replaces \(L_{\text{dec}}\) with \(L_{\text{rec}}\); '\textit{w/o Augment}' omits pretraining on a synthetic dataset; and '\textit{Iterative}' involves iterative training between the decomposition and anomaly detection tasks.}\label{tab:ablation}
\centering
\resizebox{0.9\linewidth}{!}{
\renewcommand{\multirowsetup}{\centering}
% \setlength{\tabcolsep}{3pt}
\begin{tabular}{c|ccccc}
\toprule
Ablation & UCR & SMD & SWaT & PSM & WADI \\
\midrule
TADNet              & 98.74          & \textbf{93.35} & \textbf{90.21} & \textbf{98.66} & 88.15          \\
\midrule
\textit{w/o Sep}    & 32.68          & 66.24          & 76.89          & 83.28          & 47.66          \\
% r/ DPRNN w/ Conv  & 96.32          & 91.58          & 89.22          & 97.16          & 86.23          \\
\textit{w/o Decomp} & 48.69          & 84.12          & 88.41          & 95.57          & 65.72          \\
\textit{w/o Augment}& 40.12          & 74.17          & 83.26          & 98.01          & 62.15          \\
\textit{Iterative}  & \textbf{99.12} & 92.14          & 86.55          & 96.58          & \textbf{92.06} \\
\bottomrule
\end{tabular}
}
\vspace{-10pt}
\end{table}


\begin{table}[!t]
\caption{\rmfamily Performance of \ourmethod under Different Candidate Position Numbers}
\rmfamily
\centering
\scalebox{1.0}{
\label{tab:ablation2}
\begin{threeparttable}
\setlength\tabcolsep{15pt}
\begin{tabular}{ccc}
\toprule
\textbf{\# Positions} & \textbf{MRR} & \textbf{Improvement}   \\ \bottomrule
1                 & 0.1726 & \textminus14.60\% \\
2                 & 0.2074 & +~\,2.62\% \\
3                 & \textbf{0.2221} & +~\,\textbf{9.90}\% \\ \bottomrule
\end{tabular}
\end{threeparttable}
}
\end{table}

\smallskip \textbf{Number of Candidate Positions}. We also investigate how many expansions lead to the best performance. We vary the number of candidate positions from 1 to 3 and verify their effects on performance.
%The reason we do not use more number of candidate positions is that the minimum query length in the dataset is 3, which means that at most 4 candidate positions can be inserted. When the number of candidate positions is greater than or equal to 4, there may not be enough empty positions to insert \verb|[MASK]| tokens or \verb|[MASK]| tokens are inserted in all available positions, in this case, the positioning strategies don't work.
Table \ref{tab:ablation2} shows the results. We observe that increasing the number of candidate positions has a positive effect on performance. 
The best performance is achieved when 3 candidate positions are expanded. Meanwhile, only one candidate position can have a negative effect on query reformulation. 
The reason can be that our method reformulates the original query with a variety of query intents. 
A larger number of candidate positions can hit more user intents and hence leads to better search accuracy. 
%And it is similar to common search engines provide users with multiple possible search reformulation results to correspond to different user potential intent.

%\textit{Answer to RQ3}: The effectiveness of our method is affected by the strategy for expansion positioning and the number of candidate positions. Selecting expansion positions according to the prediction entropy can provide the best expansion; three candidate positions are sufficient to expand most queries.

%\subsection{RQ4: Performance on Code Question Answering}
%\begin{figure}[!t]
% \centering
%  \includegraphics[width=\linewidth]{figures/Code Question-Answering Example.pdf}
%  \caption{An example of code question answering.}
%  \label{fig:codeQA}
%%  \vspace{-3mm}
%\end{figure}

%\begin{table*}[!htbp]
\caption{\rmfamily Effectiveness of \ourmethod in Code Question Answering}
\rmfamily
\centering
\large
\scalebox{0.8}{
\begin{tabular}{cllllll}
\toprule
\multicolumn{1}{l}{\textbf{Language}} & \textbf{Approach} & \textbf{BLUE}    & \textbf{ROUGE-L} & \textbf{Precision} & \bf Recall  & \textbf{F1}               \\ \hline
\multirow{2}{*}{\textbf{Java}}          & \bf CodeQA~\cite{CodeQA}  & 33.08            & 28.91            & 36.63            & 28.3             & 29.91            \\
                                        & \bf CodeQA+\ourmethod    & 36.77 (+11.15\%) & 33.03 (+14.25\%) & 44.06 (+20.28\%) & 31.86 (+12.58\%) & 34.32 (+14.74\%) \\ \hline
\multirow{2}{*}{\textbf{Python}}        & \bf CodeQA~\cite{CodeQA}  & 35.16            & 31.06            & 39.12            & 30.59            & 32.34            \\
                                        & \bf CodeQA+\ourmethod   & 38.83 (+10.44\%) & 34.86 (+12.23\%) & 46.06 (+17.74\%) & 33.89 (+10.79\%) & 36.42 (+12.62\%) \\ \bottomrule
\end{tabular}
\label{tab:codeqa}
}
\end{table*}
%To verify the generality of \ourmethod, we further apply our method to the code question answering task. Code question answering can be considered as a machine reading comprehension task: given a code snippet and a question that is related to it, a machine learning model is trained to generate a textual answer to the question. We apply \ourmethod to the original questions and input the reformulated questions to the QA model. Then we assess the improvement of accuracy after question reformulation.
%Figure~\ref{fig:codeQA} shows an example of question answering for the function \texttt{hashCode}.
%We evaluate \ourmethod on the CodeQA~\cite{CodeQA} benchmark, which contains code segments written in both Java and Python. The statistics of the benchmark can be found in the original paper \cite{CodeQA}. Following the CodeQA paper, we implement the code QA engine~\cite{CodeQASystem} based on CodeBERT.  
%We measure the performance using five popular metrics, namely, BLUE-4~\cite{papineni2002bleu}, ROUGE-L~\cite{lin2004rouge}, Precision, Recall, and F1. %BLUE, ROUGE\_L is widely used evaluation metrics for generative tasks. Precision, Recall and F1 evaluate the ratio of words in which the generated answers overlap with the ground truth answers.

%The results are presented in Table~\ref{tab:codeqa}. As can be seen, by query reformulation with \ourmethod, the performance of code QA is improved by 15\% on average in terms of all metrics. For example, \ourmethod increases the precision by 20.28\% and 17.74\% on the Java and Python datasets, respectively.
%This indicates that our method can be well generalized to the code QA task. 
%We notice that results on the Java dataset gain more improvement than that on the Python dataset, probably because of the different scales (259k vs. 88k) of Java and Python datasets used for the self-supervised training.

%\textit{Answer to RQ4}: 
%Our reformulation method is effective not only for natural language understanding tasks such as code search but also for code comprehension tasks such as code question answering. 

\subsection{Qualitative Analysis}
\section{Case Study and Evaluation}
\label{casestudy}
\begin{table*}[tbp]
  \centering
  \begin{tabular}{r|ccc}
    version & \texttt{join} & \texttt{vjoin} & \texttt{udot}, \texttt{sortVector}, \texttt{roundVector}\\ \hline
    $<$ 0.15     & available  & undefined & undefined\\
    $\geq$ 0.16  & deleted & available & available\\
  \end{tabular}
  \caption{Availability of functions in hmatrix before and after tha update.}
  \label{table:join}
\end{table*}
\subsection{Case Study}
We demonstrate that \mylang{} programming achieves the two benefits of programming with versions. 
The case study simulated the incompatibility of hmatrix,\footnote{\url{https://github.com/haskell-numerics/hmatrix/blob/master/packages/base/CHANGELOG}} a popular Haskell library for numeric linear algebra and matrix computations, in the VL module \mn{Matrix}.
This simulation involved updating the applications \mn{Main} depending on \mn{Matrix} to reflect incompatible changes.

Table \ref{table:join} shows the changes introduced in version 0.16 of hmatrix. Before version 0.15, hmatrix provided a \texttt{join} function for concatenating multiple vectors.
The update from version 0.15 to 0.16 replaced \texttt{join} with \texttt{vjoin}. Moreover, several new functions were introduced.
We implement two versions of \mn{Matrix} to simulate backward incompatible changes in \mylang{}.
Also, due to the absence of user-defined types in \mylang{}, we represent \texttt{Vector a} and \texttt{Matrix a} as \texttt{List Int} and \texttt{List (List Int)} respectively, using \mn{List}, a partial port of \texttt{Data.List} from the Haskell standard library.

\begin{figure}[t]

\begin{minipage}{.5\textwidth}
% \begin{lstlisting}[style=haskell]
\begin{minted}{haskell}
module Main where
import Matrix
import List
main = let
  vec = [2, 1]
  sorted = sortVector vec
  m22 = join -- [[1,2],[2,1]]
          (singleton sorted)
          (singleton vec)
  in determinant m22
-- error: version inconsistent
\end{minted}
% \end{lstlisting}
\end{minipage}
\begin{minipage}{.5\textwidth}
\begin{minted}{haskell}
module Main where
import Matrix
import List
main = let
  vec = [2, 1]
  sorted = @\vlkey{unversion}@
             (sortVector vec)
  m22 = join -- [[1,2],[2,1]]
          (singleton sorted)
          (singleton vec)
  in determinant m22 -- ->* -3
\end{minted}
\end{minipage}
\caption{Snippets of \texttt{Main} before (left) and after (right) rewriting.}
\label{fig6-5}

\end{figure}
We implement \mn{Main} working with two conflicting versions of \mn{Matrix}. The left side of Figure \ref{fig6-5} shows a snippet of \mn{Main} in the process of updating \mn{Matrix} from version 0.15.0 to 0.16.0. \fn{main} uses functions from both versions of \mn{Matrix} together: \fn{join} and \fn{sortVector} are available only in version 0.15.0 and 0.16.0 respectively, hence \mn{Main} has conflicting dependencies on both versions of \mn{Matrix}. Therefore, it will be impossible to successfully build this program in existing languages unless the developer gives up using either \fn{join} or \fn{sortVector}.

\begin{itemize}
\item \textbf{Detecting Inconsistent Version}:
\mylang{} can accept \mn{Main} in two stages. First, the compiler flags a version inconsistency error.
It is unclear which \mn{Matrix} version the \fn{main} function depends on as \fn{join} requires version 0.15.0 while \fn{sortVector} requires version 0.16.0.
The error prevents using such incompatible version combinations, which are not allowed in a single expression.

\item \textbf{Simultaneous Use of Multiple Versions}:
In this case, using \fn{join} and \fn{sortVector} simultaneously is acceptable, as their return values are vectors and matrices. Therefore, we apply \texttt{\vlkey{unversion} t} for $t$ to collaborate with other versions.
The right side of Figure \ref{fig6-5} shows a rewritten snippet of \mn{Main}, where \texttt{sortVector vec} is replaced by \texttt{\vlkey{unversion} (sortVector vec)}. Assuming we avoid using programs that depend on a specific version elsewhere in the program, we can successfully compile and execute \fn{main}.
\end{itemize}

\subsection{Scalability of Constraint Resolution}
\begin{figure}[tbp]
    \centering
    \includegraphics[height=6.5cm]{figs/ret.png}
    \caption{Constraint resolution time for the duplicated \mn{List} by \texttt{\#mod} $\times$ \texttt{\#ver}.}
    \label{fig:consres}
\end{figure}

We conducted experiments on the constraint resolution time of the \mylang{} compiler. In the experiment, we duplicated a \mylang{} module, renaming it to \texttt{\#mod} like \mn{List\_i}, and imported each module sequentially. Every module had the same number of versions, denoted as \texttt{\#ver}. Each module version was implemented identically to \mn{List}, with top-level symbols distinguished by the module name, such as \fn{concat\_List\_i}. The experiments were performed ten times on a Ryzen 9 7950X running Ubuntu 22.04, with \texttt{\#mod} and \texttt{\#ver} ranging from 1 to 5.

Figure \ref{fig:consres} shows the average constraint resolution time. 
The data suggests that the resolution time increases polynomially (at least square) for both \texttt{\#mod} and \texttt{\#ver}.
Several issues in the current implementation contribute to this inefficiency:
First, we employ sbv as a z3 interface, generating numerous redundant variables in the SMT-Lib2 script.
For instance, in a code comprising 2600 LOC (with $\texttt{\#mod} =5$ and $\texttt{\#ver} =5$), the \mylang{} compiler produces 6090 version resource variables and the sbv library creates SMT-Lib2 scripts with approximately 210,000 intermediate symbolic variables.
Second, z3 solves versions for all AST nodes, whereas the compiler's main focus should be on external variables and the subterms of \texttt{\vlkey{unversion}}.
Third, the current \mylang{} nests the constraint network, combined with $\lor$, \texttt{\#mod} times at each bundling. This approach results in an overly complex constraint network for standard programs.
Hence, to accelerate constraint solving, we can develop a more efficient constraint compiler for SMT-Lib2 scripts, implement preprocess to reduce constraints, and employ a greedy constraint resolution for each module.






% \section{Limitations of the Current \mylang{}}
% % This section discusses the limitations of the current VL language and possible solutions.

% \subsection{Lack of Support for Structural Incompatibility}
% One of the apparent problems with the current VL system is that it does not support \emph{type incompatibilities}, a key element of structural incompatibilities. We will first analyze the types of incompatibilities and then discuss ways to extend the current VL system.

% % \paragraph{Types of Incompatibilities}
% % \label{sec:typesofincompatibility}
% % Incompatibilities between old and new versions of a package caused by updates can be broadly classified into two categories \emph{structural incompatibilities} and \emph{behavioral incompatibilities}.

% % \paragraph{Structural Incompatibilities}
% % \begin{table*}[tbp]
  \centering
  \begin{tabular}{r|ccc}
    version & \texttt{join} & \texttt{vjoin} & \texttt{udot}, \texttt{sortVector}, \texttt{roundVector}\\ \hline
    $<$ 0.15     & available  & undefined & undefined\\
    $\geq$ 0.16  & deleted & available & available\\
  \end{tabular}
  \caption{Availability of functions in hmatrix before and after tha update.}
  \label{table:join}
\end{table*}
% % A structural incompatibility occurs when multiple versions of a package provide different set of definitions including function names and data structures.
% % Structural incompatibilities are caused by adding and removing definitions, internal changes to data structures, and renaming.
% % Table \ref{table6-1} shows an example of structural incompatibility in GIMP Drawing Kit (GDK).
% % GDK is a C library for creating graphical user interfaces and is used by many projects, including GNOME.

% % If the deprecated functions are not available, version 3.22 is structurally incompatible with version 3.20 because the former lacks \mscreen{} that is available in the latter.
% % GDK versions before 3.22 provide \mscreen{} that tells the number of connected physical monitors.
% % However, versions 3.22 later provide the same functionality function \mdisplay{} and deprecate \mscreen{}.
% % When we upgrade GDK to version 3.22 and build software that uses this function without modifying anything, the build system will give us an undefined reference error.
% % With a static type check, the programmer will be informed of the incompatibility problem as a compilation error.

% % \paragraph{Behavioral Incompatibilities}
% % \input{./figs/table6-2.tex}
% % A behavioral incompatibility is a situation where multiple versions of a package provide the same definition but differ in their behavior.
% % Code changes may also cause behavioral incompatibilities that include additions, removals, and changes in side effects, even if there is no change in name or type.
% % Table \ref{table6-2} shows an example of behavioral incompatibility in the Android Platform API (henceforth Android API).
% % The Android API is the standard library written mostly in Java, and its version synchronizes with Android OS.

% % Before version 19\footnote{The Android API uses \textit{levels} instead of versions as identifiers for API revisions, but we will call them versions for consistency.},
% % the Android API provided the \mset{} method in the \texttt{AlarmManager} class that schedules an alarm at a specified time.
% % However, since version 19, the Android API has changed its behavior for power management.
% % Despite having the same name and type definitions, \mset{} no longer guarantees accurate alarm delivery.
% % For developers who require accurate delivery, the method \msetExact{} is provided instead.

% % \paragraph{Extending \mylang{} to Support Structural Incompatibility}
% The current VL language system forces terms of different versions to have the same type, both on the theoretical (typing rules in \corelang{}) and implementation (bundling in \vlmini{}) aspects.
% In \corelang{}, definitions of the same type can be combined as a versioned record (even if the programmer has given them different names), while terms with different types cannot be in a versioned record. Also, the VL language system will stop compilation if it finds a definition with the same name but a different type in more than one version of the same module.

% However, more feature is needed to deal with broader incompatibilities. Raemaekers et al. conducted a comprehensive analysis of the seven-year history of library releases in Maven Central. They found that about one-third of all releases introduced at least one structural incompatibility change. The top three causes of structural incompatibilities were class, method, or field deletions, and the remaining seven were type changes.~\cite{RAEMAEKERS2017140}
% It seems an important step to extend the language system to support a wider variety of type incompatibilities and to help programmers improve dependencies.

% The current \corelang{} design is motivated by the basic design that "the type of a versioned record is similar to the type \vertype{r}{A}, a type with a resource in coeffect calculli." In the current \corelang{}, the type of versioned record $\{\overline{l=t}\,|\,l_k\}$ is $\vertype{r}{A} (r = \{\overline{l}\})$, and no difference exists between a type of versioned records and promotions of a term of type $A$. This design has the advantage that versioned records and promotions could be treated in a unified manner, making it easier to formalize dynamic and static semantics.

% One useful idea to address this problem is to decouple version inference from the type inference of coeffect calculus and implement a type system that guarantees version consistency on top of the polymorphic record calculus.~\cite{10.1145/218570.218572} The idea stems from the fact that the type $\vertype{\{l_1,l_2\}}{A}$ is structurally similar to the variant type $\langle l_1 : A,\, l_2 : A\rangle$ of $\Lambda^{\forall,\bullet}$. It is no longer required with variants that types be the same, so terms with different types can be stored as a single value, such as $\langle l_1 = true,\, l_2 : 100\rangle : \langle l_1 : Bool,\, l_2 : Int\rangle$. Although the current version inference is uniformly defined with type inference, we believe it is possible to separate its algorithm and implement it in another calculus because the type and version inference in the type system of \vlmini{} is orthogonal to each other. In the current VL system, constraints generated from type inference and constraints generated from version inference are completely independent, and all constraints passed to z3 are version constraints.



% \subsection{Inadequate Version Polymorphism}
% As we attempt to scale VL programming to a realistically sized development, incomplete version polymorphism via duplication described in section \ref{sec:adhocversionpolymorphism} becomes an obstacle. The following examples are VL programs that depend on modules \texttt{A} and \texttt{B} in Figure \ref{fig:smallexample}. Both use functions \texttt{g} and \texttt{h} provided by module \texttt{B} and the variable \texttt{a} provided by module \texttt{A}.

% \input{figs/fig6-8.tex}
% The first problem is the difference between the treatment of local and external variables. The two programs in Figure \ref{fig6-8} illustrate this problem.  The only difference between the two programs is that the program on the left is written to apply functions without local variables, whereas the program on the right binds \texttt{a} to \texttt{a'}. However, the left one succeeds, while the right fails in version inference.

% The reason for this problem is the type inference system assigns the only resource variable to the local variable \texttt{a'}. The applications \texttt{g a'} and \texttt{h a'} generate constraints that require \texttt{a'} to depend on versions 1 and 2 of module \texttt{B}, respectively, but there is no version label that satisfies both. All external variables are given unique names by duplication, but local variables are not. Therefore, the type inference results differ in the two programs in Figure \ref{fig6-8}.

% \input{figs/fig6-9.tex}
% The second problem is that there is only one version on which each version of the top-level symbol can depend. The programs in Figure \ref{fig6-9} illustrate this problem.

% The top program requires \texttt{a} of \texttt{A} versions 2.0.0 and 1.0.0 as arguments of \texttt{g} and \texttt{h}, respectively, whereas the bottom program requires \texttt{A} version 2.0.0 for both arguments. The result of type inference is that the top program has a label that satisfies this requirement, while the bottom program does not.

% The cause of this problem is that the inference system produces a variable dependency on one of the versions of the original top-level symbol. The current VL type inference creates a variable dependency on either version of the source when creating a resource variable with the same constraints as the source of the duplication. In this example, the two copies of \texttt{a}, \texttt{a\_0} (for \texttt{g (version {A=2.0.0} of a))} and \texttt{a\_1} (for \texttt{h (version {A=2.0.0} of a))}, are expected to select either version of \texttt{a}. Furthermore, the generated constraints constrain the selected version of \texttt{a}. In line 7, \texttt{g} requires \texttt{a\_0} to have a dependency on version 1.0.0 of \texttt{B}, and version {A=2.0.0} of \texttt{a\_0} requires that \texttt{a\_0} is equal to the label selected for version 2.0.0 of a, resulting in version 2.0.0 of \texttt{a}. Similarly, line 8 generates a constraint that requires that the label for version 2.0.0 of \texttt{a} must contain version 1.0.0 of \texttt{B}, so no label satisfies both simultaneously.

% It is necessary to introduce full-resource polymorphism in the core calculus instead of duplication to solve this irrational problem,.
% The idea is to store external variables and constraints that behave in a version polymorphic manner in a top-level definition environment and instantiate them with a new resource variable for each symbol occurrence. This kind of resource polymorphism is similar to that already implemented in the Gr language~\cite{Orchard:2019:Granule}. However, unlike Gr, \vlmini{} provides a type inference algorithm that collects constraints on a per-module basis, so we need the well-defined form of the principal type.
% This extension is future work.

To further understand the capability of \ourmethod, we qualitatively examine the reformulation samples by various methods. Four examples are provided in Table~\ref{tab:casestudy}.
Example 1 compares the reformulation for the query ``The total CPU load for the Synology DSM'' by various methods. The original query aims to find the code that monitors the CPU load of a DSM. The reformulated query by \ourmethod is more precise to the real scenario since CPU load and memory usage are often important indicators that need to be monitored simultaneously. In contrast, SEQUER only removes the ``total'' at the beginning of the query during reformulation. LuSearch appends the query with the synonyms of the keyword ``total'' such as ``sum'' and ``aggreg''. Meanwhile, NLP2API appends the query with APIs that are relevant to CPU and operating system, which are more useful compared to those of SEQUER and LuSearch. GooglePS appends the word ``7'' after ``DSM'' to indicate the version of DSM, which helps to narrow the range of possible solutions.

In Example 2, the original query ``Fetch the events'' is incomplete and ambiguous because the user does not specify what events to fetch and where to fetch them from. The reformulated query by \ourmethod is more informative than that by SEQUER: \ourmethod specifies the source of events, i.e., from the server, which makes the query more concrete and understandable; meanwhile, SEQUER only restricts the programming language of the target code, without alleviating the ambiguity of the original query. LuSearch expands the synonyms of ``Fetch'' such as ``get'' and ``convei'' to the tail of the query. NLP2API adds APIs relevant to events to the original query. But these synonyms and APIs have limited effect on improving search accuracy. GooglePS specifies the requirement of the query to be a service by adding ``Service worker'' at the beginning of the original query. But such a specification has a limited effect on narrowing the search space.

Example 3 shows the results for the query ``Get method that raises MissingSetting if the value was unset.'' The reformulated query by \ourmethod recognizes that MissingSetting is an exception and prepends it with the exception keyword. This facilitates the search engine to find code with similar functionality. In contrast, SEQUER just specifies the programming language of the target code. Compared to \ourmethod and SEQUER, LuSearch and NLP2API only append irrelevant APIs and synonyms to the original query. Hence, the semantics of the query are broken. GooglePS fails to reformulate such a long query. Instead, it returns a search query from other users that contains the keyword ``unset''. The returned results by GooglePS discard much information from the original query, making  deviate from the user intent.

%\gu{Reviewer 1: it would be interesting and insightful to show what kinds of sequences of words are added to expand the query that leads to good search results, while others lead to worse performance. More qualitative analysis is required.}


%In some cases, however, \ourmethod might produce worse reformulations. 
Finally, the last example shows a worse case.
Although \ourmethod achieves the new state-of-the-art, it might occasionally produce error reformulations.
\ourmethod prepends a modifier ``a list of'' in front of the word ``values'', which conflicts with ``dictionary'' in the given query and thus hampers the code search performance. This is probably because the word ``values'' occurs frequently in the training corpus and often refers to elements in arrays and lists. Therefore, \ourmethod tends to expand it with modifiers such as ``all the'' and ``a list of''. Comparably, SEQUER does nothing to the original query. LuSearch concerns ``Load'' as the keyword and expands it. NLP2API adds APIs relevant to the key and value of the dictionary data structure, which results in better search performance. GooglePS cannot handle such a long query and just gives an irrelevant reformulation.

These examples demonstrate the superiority of \ourmethod in query reformulation for code search, affirming the strong ability of both position prediction and span generation.
In future work, we will conduct empirical research on the error types, and improve our model for the challenging reformulations.



\section{Discussion}
\subsection{Strength of \ourmethod over fully supervised approaches?}
%Unsupervised methods such as the LuSearch and NLP2API expand queries in the word level. Moreover, their expansion is limited to co-occurred or similar words. They simply add the expanded words at the end of the original query. Such approaches can improve the search performance for keyword-based search engines such as Lucene. Neural code search engine, on the other hand, consider more on the semantics and naturalness of search queries. the broken naturalness of the query increases the difficulty of the neural network model to understand the search query, thus reducing the accuracy of code search. 
%Compared to the unsupervised methods, \ourmethod leverages the self-supervised learning method to learn higher-level linguistic knowledge such as grammatical structure or sentence semantics from a large code search corpus. When \ourmethod reformulate a query, it can evaluate the rationality of words that should be added to the original query in more dimensions so as to improve the reformulation effect. In addition, \ourmethod preserves and even improves the naturalness of the original query through reformulation, and is therefore applicable to different search engines

One debatable question is what are the benefits of \ourmethod since it does not beat the SOTA fully-supervised approach in terms of the code search metrics. 

%We argue that lacking of supervised datasets is one of the biggest problems in the field of code query reformulation. 
%Earlier works such as LuSearch and NLP2API try to address this problem by augmenting queries with external knowledge such as WordNet and co-occurring APIs. However, knowledge-based approaches cannot identify ambiguous words, \ie, words that have different meanings in different contexts~\cite{eddington2015meaning}; Furthermore, manually constructing a knowledge base is costly and tend to be outdated, resulting in weak generalization to rare reformulation patterns in unsupervised data.

Fully-supervised methods such as SEQUER achieve the state-of-the-art performance by sequence-to-sequence learning on a parallel query set. However, acquiring such parallel queries is infeasible since the query evolution log by search engines such as Google and Stack Overflow is not publicly available. Besides, the sequence-to-sequence approach tends to learn generic reformulation patterns, \eg, specifying the programming language or deleting a few irrelevant words. 


%Compared with existing methods, \ourmethod provides a third alternative way of query reformulation. \ourmethod is a data-driven approach based on self-supervised learning, which enables it to automatically learn knowledge and reformulation patterns from large-scale unsupervised data in a simple and uniform way. 
Compared to SEQUER, \ourmethod does not rely on the supervision of parallel queries, instead, it is trained on a nonparallel dataset (queries only) that does not need to collect the ground-truth reformulations. This significantly scales up the size of training data, and therefore allows the model to learn diverse reformulation patterns from a large number of code search queries. 
\ourmethod provides an alternative feasible and cheap way of achieving the same performance. %The position selection strategy that effectively increases the reformulation patterns by selecting the positions most likely to miss information. 

\subsection{Limitations and Threats}
We have identified the following limitations and threats to our method:


%\smallskip\textit{Human evaluation.}
%In our experiments, the naturalness and informativeness of returned results are manually graded and could suffer from subjectivity bias. To mitigate this threat, four developers with more than three years of development experience are selected to perform this task independently, while their results are discussed and confirmed by all of the authors. In the future, we will further mitigate this threat by inviting more developers for the grading.



\smallskip\emph{Patterns of query reformulation.} In this work, we mainly explore query expansion, the most typical class of query reformulation. While query expansion is only designed to supplement queries with more information, redundant or misspelled words in the query can also hamper the code search performance, which cannot be handled by our method. Thus, in future work, we will extend our approach to support more reformulating patterns, including query simplification and modification. For example, in addition to only inserting a \verb|[MASK]| token in the CQC task, we can also replace the original words with a \verb|[MASK]| token or simply delete a token and ask the pre-trained model to predict the deletion position. A classification model can also be employed to decide whether to add, delete or modify keywords in the original query.


\smallskip\emph{Code comments as queries.} As obtaining real code queries from search websites is difficult, we use code comments from code search datasets to approximate code queries in building and evaluating our model. Although code comments are widely used for training machine learning models on NL-PL matching~\cite{deepcs,feng2020codebert,wang2021codet5}, they may not represent the performance of queries in real-world code search engines. 


%\smallskip\textit{The selection of the pre-trained model.}
%Our method is built upon T5 for the reason that T5 is one of the most popular PLMs for generation tasks and its pre-trained objective (\ie, corrupted span replacement) has a similar form to the downstream span generation task. However, there have been many other PLMs such as GPT~\cite{} and BERT~{}. Experiments on these PLMs may have different results. We leave these experiments with different neural network models for future work. \gu{this might not be a threat since our research subject is not PLMs} 

%这个在文中好像并没的提及,且不是Threats to Validity,而是本方法的不足,故建议在此不提,可放在future work中,一句话就可以。
%\textit{Number of MASK tokens.} When inserting [MASK] tokens for the original queries to indicate the location of missing information, \ourmethod fixes the number of inserted [MASK] token to 1. Thus, \ourmethod may not be able to fill in all missing information for queries that have multiple information incompleteness, while \ourmethod may also add redundant information to queries that have already been well-structured. In future, we will support dynamically inserting a varying number of [MASK] tokens for each query.


\section{Related Work}
\subsection{Query Reformulation for Code Search}
Query reformulation for code search has gained much attention in recent years~\cite{sesimilarwords, co-occurrence, SWordNet, nlp2api, pseudo, lusearch}. There are approximately three categories of technologies, namely, knowledge-based, feedback-based, and deep learning based approaches.

The knowledge-based approaches aim to expand or revise the initial query based on external knowledge such as WordNet~\cite{WordNet} and thesauri. 
For example, Howard \etal~\cite{sesimilarwords} reformulated queries using semantically similar words mined from method signatures and corresponding comments in the source code. Satter and Sakib~\cite{co-occurrence} proposed to expand queries with co-occurring words in past queries mined from code search logs. Yang and Tan~\cite{SWordNet} constructed a software-specific thesaurus named SWordNet by mining code- comment mappings. They expanded queries with similar words in the thesaurus. Lu \etal~\cite{lusearch} proposed LuSearch which extends queries with synonyms generated from WordNet.

Unlike knowledge-based approaches, feedback-based approaches identify the possible intentions of the user from the initial search results and use them to update the original query. 
For example, Rahman and Roy~\cite{nlp2api} proposed to search Stack Overflow posts using pseudo-relevance feedback. Their approach identifies important API classes from code snippets in the posts using TF-IDF, and then uses the top-ranked API classes to expand the original queries. Hill \etal~\cite{pseudo} presented a novel approach to extract natural language phrases from source code identifiers and hierarchically classify phrases and search results, which helps developers quickly identify relevant program elements for investigation or identify alternative words for query reformulation.

Recently, deep learning has advanced query reformulation significantly~\cite{sequer, seq2seqOfBugReport}. Researchers regard query reformulation as a machine translation task and employ neural sequence-to-sequence models. 
For example, Cao \etal~\cite{sequer} trained a sequence-to-sequence model with an attention mechanism on a parallel corpus of original and reformulated queries. The trained model can be used to reformulate a new query from Stack Overflow. %Lawrie and Binkley~\cite{seq2seqOfBugReport} trained a sequence-to-sequence model to automatically reformulate the bug report into a summary in order to support the bug localization task.

While deep learning based approaches show more promising results than previous approaches, they rely on the availability of large, high-quality query pairs. For example, Cao \etal's work requires the availability of query pairs within the same session in the search logs of Stack Overflow. But such logs are confidential and unavailable to researchers. This restricts their practicality in real-world code search.

Unlike these works, \ourmethod is a data-driven approach based on self-supervised learning. \ourmethod expands queries by pre-training a Transformer model with corrupt query completion on large unlabeled data. Results demonstrate that \ourmethod achieves competitive results to that of fully-supervised models without requiring data labeling.


\subsection{Code Intelligence with Pre-trained Language Models}

In recent years, there is an emerging trend in applying pre-trained language models to code intelligence \cite{feng2020codebert,wang2021codet5,niu2022spt,hadi2022effectiveness}. 
For example, Feng \etal\cite{feng2020codebert} pre-trained the CodeBERT model based on the Transformer architecture using programming and natural languages. %Their model extends the pre-training task of BERT by a \textit{replaced token detection} (RTD) pre-training objective. 
CodeBERT can learn the generic representations of both natural and programming languages that can broadly support NL-PL comprehension tasks (\eg, code defect detection, and natural language code search) and generation tasks (\eg, code comment generation, and code translation). Wang \etal~\cite{wang2021codet5} proposed CodeT5, which extends the T5 with an identifier-aware pre-training task. Unlike encoder-only CodeBERT, CodeT5 is built upon a Transformer encoder-decoder model. It achieves state-of-the-art performance on both code comprehension and generation tasks in all directions, including PL-NL, NL-PL, and PL-PL. %Niu \etal~\cite{niu2022spt} present SPT-Code, a sequence-to-sequence pre-trained model for source code. SPT-Code enables both the encoder and the decoder of Transformer to be jointly pre-trained. Each data instance for SPT-Code is composed of three types of information derived from a method, namely the code sequence, its AST, and the associated natural language description. SPT-Code achieves competitive performance on five code-related downstream tasks. Hadi \etal~\cite{hadi2022effectiveness} propose two novel tokenization techniques to help improve the performance of the pre-trained Transformer based models in the API sequence generation task.

To the best of our knowledge, \ourmethod is the first attempt to apply PLM in query reformulation, which aims to leverage the knowledge learned by PLM to expand queries.


\section{Conclusion}
In this paper, we propose \ourmethod, a novel self-supervised approach for query reformulation. 
\ourmethod formulates query expansion as a masked query completion task and pre-trains T5 to learn general knowledge from large unlabeled query corpora. For a search query, \ourmethod guides T5 through enumerating multiple positions for expansion and selecting positions that have the best information gain for expansion.
We perform both automatic and human evaluations to verify the effectiveness of \ourmethod. %on the CodeSearchNet benchmark
The results show that \ourmethod generates useful and natural-sounding reformulated queries, outperforming baselines by a remarkable margin.
In the future, we will explore other reformulation patterns such as query simplification and modification besides query expansion. 
%We also plan to develop a query reformulation plugin for code search engines and then investigate our approach in real usage scenarios.
%We also plan to design new pre-training objectives to learn more code search related knowledge from unsupervised data. Furthermore, it is valuable to develop a query reformulation plugin for code search engines and then investigate our approach in real usage scenarios.
We also plan to compare the performance of our approach with large language models such as GPT-4.

\section*{Data Availability}
Our source code and experimental data are publicly available at \href{https://github.com/RedSmallPanda/SSQR}{https://github.com/RedSmallPanda/SSQR}.

\section*{Acknowledgments}
This research is supported by National Natural Science Foundation of China (Grant No. 62232003, 62102244, 62032004) and CCF-Tencent Open Research Fund (RAGR20220129). 

\balance
\bibliographystyle{ACM-Reference-Format}
\bibliography{ref}

\end{document}
