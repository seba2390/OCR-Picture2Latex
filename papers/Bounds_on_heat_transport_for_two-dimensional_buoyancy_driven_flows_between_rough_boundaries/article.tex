\documentclass{article}
\usepackage{authblk}
\usepackage{bookmark}
\usepackage[margin = 4cm]{geometry} % default margin for article 1.875in
\usepackage{amsthm, amssymb, latexsym, graphics}
\usepackage{amsmath,amsfonts}
\usepackage{color}
\usepackage{hyperref}
%\usepackage{showkeys}
\usepackage{todonotes}
%\usepackage{pdftex}


%%% Fabian's packages %%%
\usepackage{caption}    % the caption of the subdomains figure was not centered
%\usepackage{mathtools}  % for mathrlap in flalign like at \eqref{energy-bound-inter}
\usepackage[style = alphabetic, doi = false, url = false]{biblatex}
\addbibresource{bibliography.bib}



\newtheorem{theorem}{Theorem}[section]
\newtheorem{lemma}[theorem]{Lemma}
\newtheorem{definition}{Definition}[section]
\newtheorem{proposition}[theorem]{Proposition}
\newtheorem{corollary}[theorem]{Corollary}
\theoremstyle{definition}
\newtheorem{assumption}{Assumption}[section]
\theoremstyle{definition}
\newtheorem{example}{Example}[section]
\newtheorem{remark}{Remark}[section]


\DeclareMathOperator*{\essinf}{ess\,inf}
\DeclareMathOperator*{\esssup}{ess\,sup}

%%%%%%%%Camilla`s command%%%%%%%%%

\newcommand\N{{\mathbb N}}
\newcommand\R{{\mathbb R}}
\newcommand\T{{\mathbb T}}
\newcommand\Z{{\mathbb Z}}

\newcommand{\HH}{\mathcal{H}}
\newcommand{\DD}{\mathcal{D}}
\newcommand{\Range}{\mathcal{R}}
\newcommand{\Real}{{\rm{Re}}}
\newcommand{\Hnu}{H_{\nu}}
\newcommand{\tf}{\tilde{f}}
\newcommand{\tg}{\tilde{g}}
\newcommand{\gplus}{\gamma^+}
\newcommand{\gminus}{\gamma^-}
\newcommand{\Pra}{\rm{Pr}}
\newcommand{\Ra}{{\rm{Ra}}}
\newcommand{\Nu}{{\rm{Nu}}}
\newcommand{\la}{\langle}
\newcommand{\ra}{\rangle}

\providecommand{\keywords}[1]
{
  \noindent\small	
  \textbf{Keywords:} #1
}

\title{Bounds on heat transport for two-dimensional buoyancy driven flows between rough boundaries}


\author[1]{Fabian Bleitner}
\author[2]{Camilla Nobili}

\affil[1]{\small{Department of Mathematics, University of Hamburg, Germany}}
\affil[2]{\small{Department of Mathematics, University of Surrey, United Kingdom}}



%\affil[1]{\footnotesize{Department of Mathematics, University of Hamburg, Germany, \textit{fabian.bleitner@uni-hamburg.de}}}
%\affil[2]{\footnotesize{Department of Mathematics, University of Surrey, United Kingdom, \textit{c.nobili@surrey.ac.uk}}}


%\affil[1]{\small{Department of Mathematics, University of Hamburg, Bundesstraße 55, 20146 Hamburg, Germany}}
%\affil[2]{\small{Department of Mathematics, University of Surrey, Guildford, Surrey, GU2 7XH, United Kingdom}}

\date{}

\begin{document}

\maketitle

\begin{abstract}
We consider the two-dimensional Rayleigh-B\'enard convection in a layer of fluid between rough Navier-slip boundaries.
The top and bottom boundaries are described by the same height function $h$.
We prove rigorous upper bounds on the Nusselt number which capture the dependence on the curvature of the boundary $\kappa$ and the (non-constant) friction coefficient $\alpha$ explicitly.
For $h\in W^{2,\infty}$ and $\kappa$ satisfying a smallness condition with respect to $\alpha$, we find 
\begin{equation*}
 \Nu\lesssim \Ra^{\frac 12}+\|\kappa\|_{\infty}\,,
\end{equation*}
which agrees with the predicted Spiegel-Kraichnan scaling when $\kappa=0$.
This bound is obtained via local regularity estimates in a small strip at the boundary.
When $h\in W^{3,\infty}$ and the functions $\kappa$ and $\alpha$ are sufficiently small in $L^{\infty}$, we prove upper bounds using the background field method, which interpolate between $\Ra^{\frac 12}$ and $\Ra^{\frac{5}{12}}$ with non-trivial dependence on $\alpha$ and $\kappa$.
These bounds agree with the result in \cite{drivasNguyenNobiliBoundsOnHeatFluxForRayleighBenardConvectionBetweenNavierSlipFixedTemperatureBoundaries} for flat boundaries and constant friction coefficient.
Furthermore, in the regime $\Pr\geq \Ra^{\frac 57}$, we improve the $\Ra^{\frac 12}$-upper bound, showing
$$\Nu\lesssim_{\alpha,\kappa}\Ra^{\frac 37}\,,$$
where $\lesssim_{\alpha,\kappa}$ hides an additional dependency of the implicit constant on $\alpha$ and $\kappa$. 
\end{abstract}

\keywords{Rayleigh-B\'enard convection, Navier-slip boundary conditions, rough boundaries, scaling laws, upper bounds}


%\tableofcontents





\section{Introduction}
In this paper we deal with the Rayleigh-B\'enard convection problem, modelled by the Boussinesq system for the velocity field $u=(u_1,u_2)\in \R^2$, the scalar temperature field $T$ and the pressure $p$
\begin{align}
    \frac{1}{\Pra}(u_t + u\cdot \nabla u ) -\Delta u +\nabla p &= \Ra T e_2\tag{NS}\label{navierStokes}\\
    \nabla \cdot u &= 0\tag{DF}\label{divergenceFreeIncompressibilityCondition}\\
    T_t +u\cdot \nabla T - \Delta T &= 0\tag{AD}\label{heatEquation}
\end{align}
set in the regular and bounded domain 
\begin{align*}
    \Omega=\{(y_1,y_2) \in \mathbb{R}^2 \ | \ 0 < y_1 < \Gamma, h(y_1) < y_2 < 1+h(y_1)\}
\end{align*}
where $h$ describes the height of the bottom boundary and $e_2=(0,1)$. The nondimensional numbers $\Ra$ and $\Pr$ are the Rayleigh and Prandtl number respectively. 
We assume periodicity of all variables in the $e_1=(1,0)$-direction and impose
 \begin{equation}
    \label{BC-T}
    \begin{aligned}
        T &= 1  & &\textnormal{ on }\gamma^- \\
        T &= 0  & &\textnormal{ on }\gamma^+
    \end{aligned}
\end{equation}
where $\gamma^-=\lbrace (y_1,y_2)\in \mathbb{R}^2\ \vert \ 0<y_1<\Gamma, y_2=h(y_1)\rbrace$ and $\gamma^+=\lbrace (y_1,y_2)\in \mathbb{R}^2\ \vert \ 0<y_1<\Gamma, y_2=1+h(y_1)\rbrace$ are the bottom and top boundary respectively.
Our goal is to derive physically meaningful upper bounds for the Nusselt number, a quantity that measures the average heat transport in the vertical direction.
In the last thirty years, the problem of deriving scaling laws for the Nusselt number in turbulent convection between two horizontal plates has been highly investigated both in experiments and numerical studies (cf.\cite{AGL09} and references therein). 
Our analysis wants to capture the role of geometry and boundary conditions in the scaling laws for the Nusselt number. 

Intentionally, we did not yet specify the boundary conditions for $u$ at $\gamma^+$ and $\gamma^-$. Before doing that, we recall some important results under \textit{no-slip} and \textit{free-slip} boundary conditions, which are the most studied for this problem.
The no-slip boundary conditions in the case of flat boundaries (i.e. $h=0$) read
\begin{align*}
    u=0\quad \mbox{ at }\quad y_2=\{0,1\}\,.
\end{align*}
In this setting, Doering and Constantin in 1996 \cite{DC96} rigorously proved the upper bound $\Nu\lesssim \Ra^{\frac 12}$ in three-dimensions. From this seminal result, many works followed aiming at optimizing the $\Ra^{\frac 12}$-upper bound (see \cite{N21}, \cite{FAW21} and references therein). In the infinite-Prandtl number setting, a series of works \cite{CD99}, \cite{DOR06}, \cite{OS11} established the bound $\Nu\lesssim \Ra^{\frac 13}$ up to a logarithmic correction. A lower bound within the method proposed in \cite{CD99} was proved in \cite{NO17}. In the finite Prandtl number case, Choffrut, Otto and the second author of this paper in \cite{choffrutNobiliOttoUpperBoundsOnNusseltNumberAtFinitePrandtlNumber} improved the perturbative result of Wang \cite{WA08_boundOn} showing $\Nu\lesssim (\Ra\ln\Ra)^{\frac 13}$ for $\Pra\gtrsim (\Ra\ln\Ra)^{\frac 13}$ and the crossover to the bound $\Nu\lesssim (\Pr^{-1}\Ra\ln\Ra)^{\frac 12}$ for $\Pra\lesssim(\Ra\ln\Ra)^{\frac 13}$. The methods proposed by Doering and Constantin in the nineties to produce scaling laws for the Nusselt number in boundary-driven convection continues to be fruitfully employed in other (related) problems \cite{tobasco2022},\cite{DC01},\cite{FNW20}, \cite{AFW21} (see also \cite{Goluskin2016} and references therein). Recently, remarkable results have been produced concerning the saturation of the upper bounds \cite{DT19},\cite{kumar2022}.
%
The question around the "ultimate scaling law" in the Rayleigh-B\'enard convection problem was recently object of animated discussions \cite{doering2019} and it remains unclear whether at large Rayleigh number the scaling $\Nu\sim \Ra^{\frac 12}$ will prevail over the scaling $\Nu\sim \Ra^{\frac 13}$ or whether another scaling arises. 

The no-slip boundary conditions are the most used in theoretical and numerical studies, but whether or not they represent the most "realistic" conditions, is subject to debate (see \cite{N21} and references therein). In 2011 Doering and Whitehead \cite{whiteheadDoeringUltimateState} remarkably proved $\Nu\lesssim \Ra^{\frac{5}{12}}$ in the two-dimensional Rayleigh-B\'enard convection problem between flat horizontal boundaries, with free-slip boundary conditions, i.e.
$$u_2=0 \quad \mbox{ and } \quad\partial_{2}u_1=0 \quad \mbox{ at }\quad y_2=\{0,1\}\,. $$ 
This result rules out the $\Nu\sim \Ra^{\frac 12}$ scaling in this setting. But the question about the optimality of this upper bound remains along with the question whether the scaling $\Nu\sim \Ra^{\frac{5}{12}}$ carries a physical meaning. Motivated by this result, in \cite{drivasNguyenNobiliBoundsOnHeatFluxForRayleighBenardConvectionBetweenNavierSlipFixedTemperatureBoundaries} Drivas, Nguyen and the second author of this paper considered the two-dimensional Rayleigh-B\'enard convection problem with the following conditions on the horizontal plates:
\begin{equation}
    \label{Navier-slip-straight}
    u_2=0 \quad \mbox{ and } \quad \partial_{2}u_1=\frac{1}{L_s}u_1 \quad\mbox{ at }\quad y_2=\{0,1\}\,,
\end{equation}
%\qquad \mbox{ on } \Gamma=[0,L]\times\{0,1\}\,,
where $L_s$ is the (constant) "slip-length". Under these assumptions the authors proved the \textit{interpolation bound}
\begin{align*}
    \Nu\lesssim \Ra^{\frac{5}{12}}+\frac{1}{L_s^2}\Ra^{\frac 12}\qquad \mbox{ in the regime } \qquad\Pr\gtrsim \frac{1}{L_s^2} \Ra^{\frac{3}{4}}\,.  
\end{align*}

Relatively few theoretical works have addressed the problem of the Nusselt number scaling in the case of \textit{rough boundaries} in Rayleigh-B\'enard convection: In \cite{SW11} Shishkina and Wagner developed an analytical model to estimate the Nusselt number deviations caused by the wall roughness and in \cite{WS15} the same authors performed direct numerical simulations.
%The role of boundary conditions and of the geometry of the boundaries in the scaling laws for the Nusselt number is not yet clearly understood and only recently this problem was object of few studies.
In \cite{goluskinDoeringBoundsForConvectionBetweenRoughBoundaries} Goluskin and Doering considered the three-dimensional Rayleigh-B\'enard problem between rough boundaries. In particular, under the assumption that the profile $h$ (which may differ at the top and bottom boundaries) is a continuous and piecewise differentiable function of the horizontal coordinates and has squared-integrable gradients, the authors showed $\Nu\lesssim C(\|\nabla h\|_2^2)\Ra^{\frac 12}$.
Here we want to cite also \cite{K16} and \cite{K18}, where Kerswell derives an upper bound for the energy dissipation rate per unit mass for a pressure-driven flow in a rough channel.

The boundary conditions in \eqref{Navier-slip-straight} are the simplified version of the original \textit{Navier-slip} boundary conditions \cite{Navier1823}
\begin{equation}
    \label{BC-u}
    \begin{aligned}
        \tau \cdot \mathbb{D}u\cdot n +\alpha u_\tau &= 0 & &\textnormal{ on }\gamma^+\cup \gamma^-\\
        u\cdot n &= 0 & &\textnormal{ on }\gamma^+\cup \gamma^-, 
    \end{aligned}
\end{equation}
Here we denoted with $n$ and $\tau$ the outward unit normal and tangential vector to the boundary respectively, with $u_\tau = u\cdot \tau$ the tangential velocity and with $\mathbb{D}u_{ij} = \frac{1}{2}(\partial_i u_j +\partial_j u_i)$ the strain tensor. The space-dependent function $\alpha\geq 0$ is the friction coefficient and measures the tendency of the fluid to slip on the boundary. It may vary along $\gamma^-$ and $\gamma^+$. The problem is illustrated in Figure \ref{fig:overview}. We notice that the Navier-slip boundary conditions have been studied in a variety of problems in fluid mechanics. They have been considered in problems related to mixing \cite{huWu} as well as rotating systems \cite{dalibardVaret}.
In \cite{amroucheEscobedoGhosh} and \cite{acevdeoAmroucheConcaGhosh} the authors studied the Stokes operator with Navier-slip boundary conditions and the convergence to no-slip boundary conditions as $\alpha\to\infty$.
 
In this paper we want to generalize the result in \cite{drivasNguyenNobiliBoundsOnHeatFluxForRayleighBenardConvectionBetweenNavierSlipFixedTemperatureBoundaries}, considering the original \textit{Navier-slip} boundary conditions \eqref{BC-u}, hence allowing a non-constant (space dependent) friction coefficient. Furthermore, we allow a certain degree of \textit{roughness} at the upper and lower plates. 


%\smallskip
%To our knowledge, this is the first work that explores the role of Navier-slip boundary conditions in the scaling-laws for the Nusselt number.
%Furthermore, we allow a certain degree of \textit{roughness} at the upper and lower plates.
Our first, more general result is the following
\begin{theorem}
\label{Lemma-Ra-One-Half-Bound}
Let $u$ and $T$ solve system \eqref{navierStokes}-\eqref{divergenceFreeIncompressibilityCondition}-\eqref{heatEquation}, with boundary conditions \eqref{BC-T} and \eqref{BC-u}. Assume $h \in W^{2,\infty}[0,\Gamma]$, $\alpha>0$
%and that the conditions of Lemma \ref{lemma-H1-bounded-by-grad-and-bdry-terms} are satisfied. 
and $\kappa$ satisfies
\begin{align}
    \label{ec}
    |\kappa| \leq 2\alpha + \frac{1}{4\sqrt{1+(h')^2}} \min\left\lbrace 1, \sqrt{\alpha} \right\rbrace
\end{align}
pointwise almost everywhere on $\gamma^-\cup\gamma^+$. 
Then there exists a constant $C>0$ depending only on $\|h'\|_{\infty}$ and $|\Omega|$ such that
\begin{align}\label{ub-R12}
    \Nu \leq C \Ra^\frac{1}{2}+C\|\kappa\|_\infty
\end{align}
for all $\Ra\geq 1$.
\end{theorem}
This upper bound catches the classical Kraichnan-Spiegel %\cite{kraichnan}\cite{spiegel}
scaling $\Ra^{\frac 12}$ and is uniform in the Prandtl number. The novelty of this (expected) result is to show explicitly the role played by the functions $\alpha$ and $\kappa$ in the bound. We notice that the bound holds under assumption \eqref{ec} which relates the magnitude of $|\kappa|$ and $\alpha$. Since $\kappa$ changes sign, this assumption is crucial to obtain the energy decay estimate (see Lemma \ref{lemmaEnergyDecay}). As we can see from \eqref{ub-R12}, the bound deteriorates as roughness (quantified by $\|\kappa\|_\infty$) increases.

Under more restrictive regularity assumptions on the height-function, we can prove the following
\begin{theorem}
\label{main-theorem}
Let $u$ and $T$ solve the system \eqref{navierStokes}-\eqref{divergenceFreeIncompressibilityCondition}-\eqref{heatEquation}, with boundary conditions \eqref{BC-T} and \eqref{BC-u}.
Let $h\in W^{3,\infty}[0,\Gamma]$, $\alpha\in W^{1,\infty}(\gamma^-\cup\gamma^+)$ and $u_0\in W^{1,r}(\Omega)$ for some $r>2$. Set $\underline{\alpha}:=\min_{\gamma^-\cup\gamma^+}\alpha>0$. Then there exists a constant $0<\bar C<1$ such that for all $\alpha$ and $\kappa$ with
\begin{align}
    \label{theorem-condition-alpha+kappa-small}
    \|\alpha+\kappa\|_\infty \leq \bar C
\end{align}
the following bounds on the Nusselt number hold:
\begin{enumerate}
    \item\label{main-theorem-case-interpolation-kappa-leq-alpha}
        If $|\kappa|\leq \alpha$ on $\gamma^-\cup\gamma^+$, $\Pra\geq \underline{\alpha}^{-\frac{3}{2}}\Ra^\frac{3}{4}$ and $\Ra^{-\frac{1}{2}}\leq \underline{\alpha}$ then %Nusselt number is bounded by
        \begin{align}
            \label{main-theorem-bound-interpolation-kappa-leq-alpha}
            \Nu \leq C_\frac{1}{2} \|\alpha+\kappa\|_{W^{1,\infty}}^2 \Ra^\frac{1}{2} + C_\frac{5}{12}\Ra^\frac{5}{12}.
        \end{align}
    \item\label{main-theorem-case-interpolation-more-general-kappa}
        If $|\kappa|\leq 2\alpha + \frac{1}{4\sqrt{1+(h')^2}}\sqrt{\alpha}$ on $\gamma^-\cup\gamma^+$, $\Pra\geq \underline{\alpha}^{-\frac{3}{2}}\Ra^\frac{3}{4}$ and $\Ra^{-1}\leq \underline{\alpha}$ then %Nusselt number is bounded by
        \begin{align}
            \label{main-theorem-bound-interpolation-more-general-kappa}
            \Nu \leq C_\frac{1}{2} \sqrt{\underline{\alpha}}\|\alpha+\kappa\|_{W^{1,\infty}}^2 \Ra^\frac{1}{2} + C_\frac{5}{12} \underline{\alpha}^{-\frac{1}{12}} \Ra^\frac{5}{12}.
        \end{align}
    \item\label{main-theorem-case-3over7bound}
        If $|\kappa|\leq 2\alpha + \frac{1}{4\sqrt{1+(h')^2}}\sqrt{\alpha}$ on $\gamma^-\cup\gamma^+$ and $\Pra\geq \Ra^\frac{5}{7}$ then %the Nusselt number is bounded by
        \begin{align}
            \label{main-theorem-bound-3over7bound}
            \Nu \leq C_\frac{3}{7} \Ra^\frac{3}{7}.
        \end{align}
\end{enumerate}
The constants are given by 
\begin{align*}
    C_\frac{1}{2} &= C(1+\|u_0\|_{W^{1,r}}^2)^{-1}
    \\
    C_\frac{5}{12} &= C \left(\|u_0\|_{W^{1,r}}+\|\dot\alpha\|_\infty+\|\dot\kappa\|_\infty+1\right)^\frac{1}{3}
    \\
    C_\frac{3}{7} &= C \left(\|\alpha+\kappa\|_{W^{1,\infty}}^2 + \underline{\alpha}^{-\frac{1}{2}} + \underline{\alpha}^{-\frac{1}{6}}(\|u_0\|_{W^{1,r}}+\|\dot\alpha\|_\infty+\|\dot\kappa\|_\infty+1)^\frac{1}{3}\right),
\end{align*}
where $\dot \alpha$ and $\dot \kappa$ are the derivatives of $\alpha$ and $\kappa$ along the boundary and $C>0$ denotes a constant depending only on the size of the domain $|\Omega|$, $\|h'\|_{\infty}$ and $r$.
\end{theorem}

First of all, we notice that our results in Theorem \ref{Lemma-Ra-One-Half-Bound} and Theorem \ref{main-theorem} also cover the case of flat-boundaries and generic friction coefficient $\alpha$. In fact, all results (i.e. \eqref{ub-R12}, \eqref{main-theorem-bound-interpolation-kappa-leq-alpha}, \eqref{main-theorem-bound-interpolation-more-general-kappa} and \eqref{main-theorem-bound-3over7bound}) hold setting $\kappa=0$.

We observe that Theorem \ref{main-theorem} only holds under the smallness assumption \eqref{theorem-condition-alpha+kappa-small}, while there is no such restriction in Theorem \ref{Lemma-Ra-One-Half-Bound}. If $\kappa$ and $\alpha$ are big, and related through \eqref{ec}, then the bound \eqref{ub-R12} holds. Additionally, note that Theorem \ref{main-theorem} requires the rather restrictive assumption $h\in W^{3,\infty}$ to ensure $\kappa\in W^{1,\infty}$.

The interpolation bound \eqref{main-theorem-bound-interpolation-kappa-leq-alpha} coincides with the result in \cite{drivasNguyenNobiliBoundsOnHeatFluxForRayleighBenardConvectionBetweenNavierSlipFixedTemperatureBoundaries}, when $\kappa=0$ and $\alpha=\frac{1}{L_s}$. The main advantage of the interpolation results \eqref{main-theorem-bound-interpolation-kappa-leq-alpha} and \eqref{main-theorem-bound-interpolation-more-general-kappa} is in the regime of small $\|\alpha\|_{W^{1,\infty}}$ and $\|\kappa\|_{W^{1,\infty}}$. In fact, notice that if $\kappa\sim \Ra^{-\nu}$ and $\alpha\sim \Ra^{-\mu}$, then \eqref{main-theorem-bound-interpolation-kappa-leq-alpha} translates into 
$$\Nu\lesssim \begin{cases} \Ra^{-2\mu+\frac 12}& \mbox{ if } \mu<\frac{1}{24}\\\Ra^{\frac{5}{12}}& \mbox{ if } \frac{1}{24}\leq\mu<\frac{1}{2}\end{cases}$$
and $\nu\geq \mu$. This means that the best bound is achieved in the case $\frac{1}{24}\leq\mu<\frac{1}{2}$ and $\Pr\gtrsim \Ra^{\frac 32\mu+\frac 34}$. Notice that, while \eqref{main-theorem-bound-interpolation-kappa-leq-alpha} holds only under the assumption $|\kappa|\leq \alpha$, the upper bound \eqref{main-theorem-bound-interpolation-more-general-kappa} holds under the weaker assumption $|\kappa|\leq 2\alpha + \frac{1}{4\sqrt{1+(h')^2}}\sqrt{\alpha}$ on $\gamma^-\cup\gamma^+$, allowing $|\kappa|>\alpha$ especially when $\alpha$ and $|\kappa|$ are very small.
We observe that in the regime of small $\alpha$ and $\kappa$, say $\|\alpha\|_{W^{1,\infty}},\|\kappa\|_{W^{1,\infty}}\leq 1$, the constant $C_\frac{5}{12}$ becomes independent of $\alpha$ and $\kappa$.
Finally we remark that if the condition $\Ra^{-1}\leq \underline{\alpha}$ for \eqref{main-theorem-bound-interpolation-more-general-kappa} is violated, then \eqref{ub-R12} yields a stricter bound.

Interestingly \eqref{main-theorem-bound-3over7bound} improves the upper bound \eqref{ub-R12} in the case of small $\|\alpha+\kappa\|_\infty$ and big $\Pra$, i.e. $\Pra\geq \Ra^\frac{5}{7}$. While the interpolation bound \eqref{main-theorem-bound-interpolation-kappa-leq-alpha} yields a better result if $\alpha\sim \Ra^{-\mu}$, \eqref{main-theorem-bound-3over7bound} provides the sharpest bound if $\alpha$ and $\kappa$ are independent of $\Ra$. Notice that, differently from the constant prefactor in \eqref{ub-R12}, the constant $C_{\frac 37}$ also depends on $\alpha$ and $\kappa$. 
%\todo{I didn't include the precise $\mu=\frac{1}{35}$}% Notice that in that case, depending on whether $\alpha\leq \Ra^{-\frac{1}{35}}$, the upper bound in \eqref{main-theorem-bound-interpolation-kappa-leq-alpha} is smaller than the bound in \eqref{main-theorem-bound-3over7bound}.



%Overview of the results, where $\alpha \sim \Ra^{-\mu}$. The bound $\Nu\lesssim \Ra^\frac{1}{2}$ holds for all $\Pra>0$ and $\alpha$ and $\kappa$ fulfilling \eqref{energy_condition_explicite_kappa}. The other bounds only hold in case of big Prandtl numbers and $\|\alpha+\kappa\|_{W^{1,\infty}}\leq \bar C<1$. In case $\alpha$ and $\kappa$ do not scale with regards to the Rayleigh number the best bound is given by $\Nu\lesssim \Ra^\frac{3}{7}$. However if $\mu>\frac{1}{28}$ the interpolation estimates give the best bound, which in case of $|\kappa|\leq |\alpha|$ persists for big $\mu$, while in the case of \eqref{energy_condition_explicite_kappa} the best result given $\Nu \lesssim \Ra^\frac{3}{7}$. Also notice that the condition on the Prandtl number is slightly weaker in the case of the $\frac{3}{7}$ bound.



%%%%%%%%%%%%%%%%%%%%%%%%%%%%%%%%%%%%%%%%%%%%%%%%%%%%%%%%
%\begin{figure}
%    \includegraphics[width=\textwidth]{pictures/scaling.pdf}
%    \caption{Overview of the results, where $\alpha \sim \Ra^{-\mu}$. The bound $\Nu\lesssim \Ra^\frac{1}{2}$ holds for all $\Pra>0$ and $\alpha$ and $\kappa$ fulfilling \eqref{energy_condition_explicite_kappa}. The other bounds only hold in case of big Prandtl numbers and $\|\alpha+\kappa\|_{W^{1,\infty}}\leq \bar C<1$. In case $\alpha$ and $\kappa$ do not scale with regards to the Rayleigh number the best bound is given by $\Nu\lesssim \Ra^\frac{3}{7}$. However if $\mu>\frac{1}{28}$ the interpolation estimates give the best bound, which in case of $|\kappa|\leq |\alpha|$ persists for big $\mu$, while in the case of \eqref{energy_condition_explicite_kappa} the best result given $\Nu \lesssim \Ra^\frac{3}{7}$. Also notice that the condition on the Prandtl number is slightly weaker in the case of the $\frac{3}{7}$ bound.}
%    \label{fig:scaling}
%\end{figure}
%%%%%%%%%%%%%%%%%%%%%%%%%%%%%%%%%%%%%%%%%%%%%%%%%%%%%%%%%%%%


\section{Preliminaries}
We start with introducing some notation and facts we will use in the whole paper. 
We will often use that the tangential vector to the boundary $\tau$ can be written as $\tau = n^\perp=(-n_2,n_1)$, where $n$ is outward unit normal and $u_\tau = u\cdot \tau$. 
 The curvature of the boundaries $\kappa$ is given by $\frac{d}{d\lambda} \tau = \kappa n$, where $\lambda$ is the parameterization of the boundaries by arc length in $\tau$-direction. The problem is illustrated in Figure \ref{fig:overview}.

\begin{figure}
\centering
   % \includegraphics[width=\textwidth]{pictures/overview.pdf}
   \includegraphics[scale=0.75]{pictures/overview_h.pdf}
    \caption{Illustration of the Rayleigh-B\'enard convection problem with Navier-slip boundary conditions and rough boundaries considered in this paper: \eqref{navierStokes}, \eqref{divergenceFreeIncompressibilityCondition}, \eqref{heatEquation}, \eqref{BC-T},\eqref{BC-u}.}
    \label{fig:overview}
\end{figure}


We will use the following notation to denote space-time averages
\begin{align*}
    \langle f \rangle &= \limsup_{t\to\infty}\frac{1}{t}\int_0^t \frac{1}{|\Omega|}\int_\Omega f \ dy\ dt,\\
    \langle f \rangle_{\gamma^-} &= \limsup_{t\to\infty}\frac{1}{t}\int_0^t \frac{1}{|\Omega|}\int_{\gamma^-} f \ dS\ dt,\\
    \langle f \rangle_{\gamma^-\cup\gamma^+} &= \limsup_{t\to\infty}\frac{1}{t}\int_0^t \frac{1}{|\Omega|}\int_{\gamma^-\cup\gamma^+} f \ dS\ dt.
\end{align*}
Moreover, on the set 
\begin{equation}\label{strip-av}
    \gamma(x_2)=\left\lbrace(y_1,y_2)\ \middle | \ 0<y_1<\Gamma, y_2=h(y_1)+x_2 \right\rbrace,
\end{equation}
where $0\leq x_2\leq 1$, illustrated in Figure \ref{fig:subdomains}, we define the average
\begin{align*}
    \langle f \rangle_{\gamma(x_2)} &=\limsup_{t\to\infty}\frac{1}{t}\int_0^t \frac{1}{|\Omega|}\int_{\gamma(x_2)} f\ dS\,dt.
\end{align*}

%As the temperature field $T$ satisfies the advection-diffusion equation it is bounded by the maximum principle. 
In what follows we will consider $0\leq T_0\leq 1$, implying
\begin{align}
    \label{maximum-principle}
    \|T\|_\infty \leq 1
\end{align}
for all times $t>0$, by the maximum principle for the temperature.

\begin{figure}%[h]
    \includegraphics[width=\textwidth]{pictures/subdomains.pdf}
    \caption{Illustrations of $\gamma(x_2)$, $\tilde\Omega(x_2)$ and $\Omega^\star(z)$}
    \label{fig:subdomains}
\end{figure}

%Next we state a definition for the Nusselt number.
Next we define the Nusselt number.

\begin{definition}
    The Nusselt number is defined as
    \begin{equation}\label{nusselt-def}
        \Nu=\la|\nabla T|^2\ra.
    \end{equation}
\end{definition}

It also admits other equivalent representations, as stated in the following.

\begin{proposition}
The Nusselt number satisfies
\begin{align}
    %\label{Nusselt-nusselt-representation}
    %\Nu :={}& \langle |\nabla T|^2\rangle\notag\\
    \Nu={}&\langle n \cdot \nabla T \rangle_{\gamma^-}\label{nusselt-bot}\\ 
    ={}& \langle  (uT - \nabla T)\cdot n_+ \rangle_{\gamma(x_2)} \label{nusselt-strip}\\
    %={}& \langle  n_+ \cdot (u - \nabla)T \rangle\\
    %={}&\langle(uT - \nabla T)\cdot  n_+  \rangle
    %\label{nusselt-3}\\
    \geq{}& \frac{1}{1+\max h -\min h}\langle (u_2-\partial_2)T \rangle\,.\label{nusselt-ineq}
\end{align}
where $n_+$ is the normal at the curve $\gamma(x_2)$ pointing in the same direction as $n$ on $\gamma^+$, illustrated in Figure \ref{fig:subdomains}.
\end{proposition}
%\begin{remark} Note that in the case of flat boundaries the identities simplify to
%\begin{align*}
%    \Nu :={}& \langle |\nabla T|^2\rangle
%    ={} -\langle \partial_2 T \rangle_{\lbrace y_2 =0\rbrace}
%    ={} \langle (u_2 - \partial_2) T \rangle_{\lbrace y_2 = x_2\rbrace }
%    ={} \langle (u_2-\partial_2)T \rangle\,.
%\end{align*}
%\end{remark}
\begin{proof}
Argument for \eqref{nusselt-bot}: Testing the temperature equation with $T$, integrating by parts and using the the divergence-free condition and the boundary conditions for $T$, we obtain
\begin{equation*}
    \frac{1}{2}\frac{d}{dt}\|T\|_2^ 2 = -\|\nabla T\|_2^2 + \int_{\gamma^-} n \cdot \nabla T.
\end{equation*}
Taking the long-time averages and using \eqref{maximum-principle}, the maximum principle for the temperature, $T$ is universally bounded in time and we get
\begin{equation*}
    0=\lim_{t\to\infty} \frac{1}{2t} \frac{1}{|\Omega|} \left(\|T\|_2^2-\|T_0\|_2^2\right) 
    =-\langle |\nabla T|^2\rangle +\langle n\cdot \nabla T\rangle_{\gamma^-}.
\end{equation*}

Argument for \eqref{nusselt-strip}: For $0\leq x_2\leq 1$, we define the sets 
\begin{equation}
    \label{definition_OmegaTilde}
            \tilde\Omega(x_2)=\left\lbrace (y_1,y_2)\in \mathbb{R}^2\ \middle \vert\ 0<y_1<\Gamma,\ h(y_1)<y_2<h(y_1)+x_2 \right\rbrace,
\end{equation}
illustrated in Figure \ref{fig:subdomains}, and integrate the equation for $T$, obtaining
 \begin{align*}
            \int_{\tilde\Omega(x_2)} T_t &= \int_{\tilde\Omega(x_2)} \nabla \cdot(\nabla T -uT) 
            =\int_{\partial\tilde\Omega(x_2)} n\cdot (\nabla T - u T)\\
            &=\int_{\gamma^-} n_-\cdot (\nabla T - u T)+\int_{\gamma(x_2)} n_+\cdot (\nabla T - u T)\\
            &=\int_{\gamma^-} n_-\cdot \nabla T +\int_{\gamma(x_2)} n_+\cdot (\nabla T - u T)
            %\int_{\gamma^-} n_-\cdot \nabla T+\int_{\gamma(x_2)} n_+\cdot (\nabla T - u T)=
            %\int_{\gamma^-} n_-\cdot \nabla T-\int_{\gamma(x_2)} n_-\cdot (\nabla T - u T),
\end{align*}
where we used the incompressibility condition and that $u\cdot n=0$ at $\gminus$.
Taking the long-time average we get
\begin{equation*}
    \langle n\cdot \nabla T\rangle_{\gamma^-}=\la (uT-\nabla T)\cdot n_+\ra_{\gamma(x_2)}.    
\end{equation*}
In particular, from this identity we infer that the Nusselt number is independent of $x_2$.


%Argument for \eqref{nusselt-3}: This identity follows simply by integrating \eqref{nusselt-strip} in $x_2$ between $0$ and $1$.
%\todo[inline]{This needs to be removed if not used}

Argument for \eqref{nusselt-ineq}: Let us define the set $\Omega^\star(z)=\{y_2\leq z\}\cap \Omega$, as shown in Figure \ref{fig:subdomains}, and write
\begin{align*}
    \int_{\partial\Omega^{\star}(z)}n\cdot (\nabla T-uT)&=\int_{\gminus\cap\{y_2\leq z\}}n_-\cdot (\nabla T-uT)+\int_{\gplus\cap\{y_2\leq z\}}n_+\cdot (\nabla T-uT)
    \\&\qquad+\int_{\Omega\cap\{y_2=z\}}n_+\cdot (\nabla T-uT).
\end{align*}

\begin{figure}
    \begin{center}
        \includegraphics[width=0.5\textwidth]{pictures/subdomainCasesTop.pdf}    
    \end{center}
    \caption{Illustration of $\Omega^\star(z)$}
    \label{fig:nusselt-inequality}
\end{figure}
Integrate the previous equation in $z$ between $\min h$ and $1+\max h$ and write
\begin{equation*}
    \int_{\min h}^{1+\max h}\int_{\partial\Omega^{\star}(z)}n\cdot (\nabla T-uT)\, dS\, dz=A+B+C,
\end{equation*}
where 
\begin{align*}
    A&=\int_{\min h}^{1+\max h}\int_{\gminus\cap\{y_2\leq z\}}n_-\cdot (\nabla T-uT)\, dS\, dz\\
    B&=\int_{\min h}^{1+\max h}\int_{\gplus\cap\{y_2\leq z\}}n_+\cdot (\nabla T-uT)\, dS\, dz\\
    C&=\int_{\min h}^{1+\max h}\int_{\Omega\cap\{y_2=z\}}n_+\cdot (\nabla T-uT)\, dS\, dz.
\end{align*}
We analyze the three terms separately: 

Term C: We notice that this term can be simply rewritten as
\begin{align*}
    C=\int_{\min h}^{1+\max h}\int_{\Omega\cap\{y_2=z\}} (\partial_2 T-u_2T)\, dS\, dz=\int_{\Omega} (\partial_2T-u_2T).
\end{align*}

Term B: This integral has a sign, in fact
\begin{align*}
    B=\int_{\min h}^{1+\max h}\int_{\gplus\cap\{y_2\leq z\}}n_+\cdot \nabla T\, dS\, dz\leq 0,  
\end{align*}
since, at $\gamma^+$ we have $n_+\cdot \nabla T\leq 0$.

Term A: We decompose the integral in A further:
\begin{align*}
    A=\left(\int_{\min h}^{\max h}+\int_{\max h}^{1+\max h}\right)\int_{\gminus\cap\{y_2\leq z\}}n_-\cdot (\nabla T-uT)\, dS\, dz=:A_1+A_2.
\end{align*}
Notice that
\begin{align*}
    A_2&=\int_{\max h}^{1+\max h}\int_{\gminus}n_-\cdot (\nabla T-uT)\, dS\, dz=\int_{0}^{1} dz\int_{\gminus}n_-\cdot (\nabla T-uT)\, dS 
    \\
    &=\int_{\gminus}n_-\cdot \nabla T.
\end{align*}
The term $A_1$ instead will be estimated as follows
\begin{align*}
    \int_{\min h}^{\max h}&\int_{\gminus\cap\{y_2\leq z\}}n_-\cdot (\nabla T-uT)\, dS\, dz
    \\
    &=\int_{\min h}^{\max h}\int_{\gminus\cap\{y_2\leq z\}}n_-\cdot \nabla T\, dS\, dz
    %\\
    \leq\int_{\min h}^{\max h}\int_{\gminus}n_-\cdot \nabla T\, dS\, dz
    %\\
    %&=\int_{\min \gminus}^{\max \gminus}\,dz\int_{\gminus}n_-\cdot \nabla T\, dS
    \\
    &= (\max h - \min h)\int_{\gminus}n_-\cdot \nabla T,
\end{align*}
where in the first inequality we used that $n_-\cdot \nabla T\geq 0$ at $\gminus$.

Putting all together, we obtain
\begin{align*}
    \int_{\min h}^{1+\max h}&\int_{\partial\Omega^{\star}}n\cdot (\nabla T-uT)\, dS\, dz
    \\
    &\leq (1+\max h - \min h)\int_{\gminus}n_-\cdot \nabla T+\int_{\Omega} (\partial_2T-u_2T).
\end{align*}
Taking the long-time average and observing that
\begin{align*}
    \limsup_{t\rightarrow \infty}\frac{1}{t}\int_0^t &\int_{\min h}^{1+\max h}\int_{\partial\Omega^{\star}(z)}n\cdot (\nabla T-uT)\ dS\ dz\ dt
    \\
    &=\limsup_{t\rightarrow \infty}\frac{1}{t}\int_0^t \int_{\min h}^{1+\max h}\int_{\Omega^{\star}(z)} (\Delta T-u\cdot \nabla T)\ dy\ dz\ dt
    \\
    %&\qquad=\limsup_{t\rightarrow \infty}\frac{1}{t}\int_0^t \int_{\min \gminus}^{\max \gplus}\int_{\Omega^{\star}(z)} T_t \ dy\ dz\ dt \\
    &=\limsup_{t\rightarrow \infty}\frac{1}{t}\int_0^t\frac{d}{dt} \int_{\min h}^{1+\max h}\int_{\Omega^{\star}(z)} T \ dy\ dz\ dt %\\
    %&\qquad=\limsup_{t\rightarrow \infty}\frac{1}{t}\int_{\min \gminus}^{\max \gplus}\int_{\Omega^{\star}(z)} (T(t)-T(0)) \ dy\ dz\ dt
    =0\,
\end{align*}
by the maximum principle for $T$, we have 
\begin{align*}
    0\leq (1+\max h-\min h)\la n_-\cdot \nabla T\, \ra_{\gminus}+\la\partial_2T-u_2T\ra
\end{align*}
implying
\begin{align*}
     \la u_2T-\partial_2T\ra\leq(1+\max h-\min h)\la n_-\cdot \nabla T\ra_{\gminus}.
\end{align*}
\end{proof}
%\begin{remark}
%Notice that identity \eqref{nusselt-strip} does not hold if $\gminus$ and $\gplus$ are chosen to be different.
%\todo[inline]{That is wrong. Actually this holds for every "nice" curve where the endpoints at $0$ and $\Gamma$ match by changing the domain so that the upper boundary is that curve. \eqref{nusselt-ineq} would need to be changed if they wouldn't match but I think it would still be possible with a little bit changed prefactor as we throw away the upper boundary part.}
%\todo[inline]{If you meant \eqref{nusselt-ineq}, even then something similar (with probably changed constant) should hold as the term coming from the top boundary gets just thrown away.}
%\end{remark}










\section{A-priori bounds}\label{Section-A-Priori-Bounds}
In this section we collect a-priori bounds on the energy and enstrophy of the solution $u$ and derive pressure estimates that will be used to prove the main result in the Section \ref{section-four}. 

\subsection{A-priori estimate for the velocity}
%
\begin{proposition}[Energy Balance]
Strong solutions of \eqref{navierStokes}, \eqref{divergenceFreeIncompressibilityCondition}, \eqref{BC-u} satisfy
\begin{align}
    \label{energy-balance}
    \frac{1}{2 \Pra} \frac{d}{dt}\|u\|_2^2 + \|\nabla u\|_2^2 + \int_{\gamma^+\cup \gamma^-} (2\alpha+\kappa) u_\tau^2= \Ra\int_{\Omega} T u_2.
\end{align}
\end{proposition}

\begin{proof}
The balance follows by testing the Navier Stokes equations with $u$, integrating by parts and observing that
\begin{align*}
    \int_\Omega u \cdot \nabla p
    %&= -\int \nabla \cdot u p + \int_{\partial\Omega} p u\cdot n 
    =0
\end{align*}
and
\begin{align*}
    \int_\Omega u \cdot (u\cdot \nabla ) u 
    %&= \int u_i u_j \partial_j u_i 
    %= -\int \partial_j u_i u_j u_i-\int u_i \partial_j u_j u_i +\int_{\partial \Omega} u_i u_j u_i n_j 
    %=-\int \partial_j u_i u_j u_i =
    =-\int_\Omega u\cdot (u\cdot\nabla)u
\end{align*}
by the incompressibility and boundary conditions, and
\begin{align*}
    \int_\Omega u\cdot \Delta u
    &=-\|\nabla u\|_2^2 +\int_{\partial \Omega} n_i u_j\partial_i u_j
    \\
    &= -\|\nabla u\|_2^2 +2 \int_{\gamma^+\cup \gamma^-} u\cdot(\mathbb{D}u\ n) -\int_{\gamma^+\cup \gamma^-} n_iu_j \partial_j u_i\\
    &= -\|\nabla u\|_2^2 - 2\int_{\gamma^+\cup \gamma^-} \alpha u_{\tau}^2 -\int_{\gamma^+\cup \gamma^-} n\cdot (u\cdot \nabla) u
    \\
    &= -\|\nabla u\|_2^2 - \int_{\gamma^+\cup \gamma^-} (2\alpha+\kappa) u_\tau^2\,.
\end{align*}
In this identity we used the algebraic identity

\begin{equation*}
    n_iu_j\partial_iu_j=2u\cdot(\mathbb{D}u \,n)-n_iu_j\partial_ju_i,    
\end{equation*}
the Navier slip boundary conditions to deduce
\begin{equation*}
    u\cdot(\mathbb{D}u \,n)+\alpha u_{\tau}^2=0  
\end{equation*}
and the equality
\begin{equation}
    \label{id-kappaUtau2-1}
    n\cdot(u\cdot \nabla)u=\kappa u_{\tau}^2\,,
\end{equation}
proved in \eqref{appendix-proof-id-kappaUtau2} in the Appendix.

\end{proof}


Under a smallness assumption on $\kappa$, the second and third term on the left-hand side of \eqref{energy-balance} are positive and bounded from below by the $H^1$-norm of $u$ even though $(2\alpha+\kappa)$ might be negative on some parts of the boundary. This will be essential in what follows, especially in order to prove the energy decay and the main theorem.

\begin{lemma}
\label{lemma-H1-bounded-by-grad-and-bdry-terms}
Assume $\alpha \geq \underline{\alpha}$ almost everywhere on $\gamma^-\cup\gamma^+$ for some constant $\underline{\alpha}>0$ and $\kappa$ satisfies
\begin{align}
    \label{energy_condition_explicite_kappa}
    |\kappa| \leq 2\alpha + \frac{1}{4\sqrt{1+(h')^2}} \min\left\lbrace 1, \sqrt{\alpha} \right\rbrace
\end{align}
almost everywhere on $\gamma^-\cup\gamma^+$. Then
\begin{align}
    \label{H1-bounded-by-grad-and-bdry-terms}
    \frac{3}{4}\|\nabla u\|_2^2 + \int_{\gamma^-\cup\gamma^+} (2\alpha+\kappa) u_\tau^2 \geq \frac{1}{4}\min\lbrace 1, \underline{\alpha}\rbrace \|u\|_{H^1}^2.
\end{align}
\end{lemma}

\begin{proof}
In this proof we use the notation 
\begin{equation}
    \label{defKappaAlphaUpm}
    \begin{aligned}
        \kappa_-&=\kappa(y_1,h(y_1)), & \alpha_-&=\alpha(y_1,h(y_1)), & u_-&=u_\tau(y_1,h(y_1)),
        \\
        \kappa_+&=\kappa(y_1,1+h(y_1)), & \alpha_+&=\alpha(y_1,1+h(y_1)), & u_+&=u_\tau(y_1,1+h(y_1)),
    \end{aligned}
\end{equation}
which is the evaluation of the functions on the bottom or top boundary. Notice that, because of the symmetry of the domain, $\kappa_-=-\kappa_+$.


The idea of the proof is that if $\kappa_-$ is negative for some $y_1$, then $\kappa_+$ is positive and we can compensate by the fundamental theorem of calculus.

By the fundamental theorem of calculus, Young's and Hölder's inequality
\begin{equation}
    \label{fundamental-theorem-calc-estimate-for-u_minus}
    \begin{aligned}
        |u(y_1,y_2)|^2&\leq \left(u_- +\int_{h(y_1)}^{y_2} \partial_2 u(y_1,z) \ dz\right)^2
        \\
        &\leq (1+\epsilon) u_-^2 + (1+\epsilon^{-1}) \left(\int_{h(y_1)}^{y_2} \partial_2 u(y_1,z) \ dz\right)^2
        \\
        &\leq (1+\epsilon) u_-^2 + (1+\epsilon^{-1}) (y_2-h(y_1)) \|\partial_2 u \|_{L^2(\gamma^-,\gamma^+)}^2
    \end{aligned}
\end{equation}
for all $\epsilon>0$, where $\|\partial_2 u\|_{L^2(\gamma^-,\gamma^+)}^2=\int_{h(y_1)}^{1+h(y_1)}|\partial_2u(y_1,y_2)|^2 dy_2$ and analogously
\begin{align}
    \label{fundamental-theorem-calc-estimate-for-u_plus}
    |u(y_1,y_2)|^2
    &\leq (1+\epsilon) u_+^2 + (1+\epsilon^{-1}) (1+h(y_1)-y_2) \|\partial_2 u \|_{L^2(\gamma^-,\gamma^+)}^2.
\end{align}
Integrating \eqref{fundamental-theorem-calc-estimate-for-u_minus} and \eqref{fundamental-theorem-calc-estimate-for-u_plus} in $y_2$ one gets
\begin{equation}
    \label{u-estimated-by-bdry-and-grad}
    \begin{aligned}
        \|u\|_{L^2(\gamma^-,\gamma^+)}^2 &\leq (1+\epsilon) \min \lbrace u_-^2,u_+^2 \rbrace +\frac{1+\epsilon^{-1}}{2} \|\partial_2 u\|_{L^2(\gamma^-,\gamma^+)}^2
        \\
        &\leq (1+\epsilon) \max\lbrace \alpha_-^{-1}, \alpha_+^{-1}, 1 \rbrace \Big( \min\lbrace \alpha_-, \alpha_+\rbrace \sqrt{1+(h')^2}\min\lbrace u_-^2,u_+^2 \rbrace 
        \\
        &\qquad\qquad\qquad  + (2\epsilon)^{-1} \|\partial_2 u\|_{L^2(\gamma^-,\gamma^+)}^2\Big),
    \end{aligned}
\end{equation}
where in the last inequality we have smuggled in the factor $\sqrt{1+(h')^2}>1$. Next we claim that
\begin{equation}
    \label{energy-decay-kappa-removed-estimate}
    \begin{aligned}
        \min\lbrace\alpha_-,\alpha_+\rbrace \sqrt{1+(h')^2}u_{i}^2
        %\\
        &
        \leq \frac{5}{16}\|\partial_2 u\|_{L^2(\gamma^-,\gamma^+)}^2 + (2\alpha_-+\kappa_-)\sqrt{1+(h')^2}u_-^2
        \\
        &\qquad+ (2\alpha_++\kappa_+)\sqrt{1+(h')^2}u_+^2
    \end{aligned}
\end{equation}
holds for either $i=+$ or $i=-$. Using \eqref{energy-decay-kappa-removed-estimate}, \eqref{u-estimated-by-bdry-and-grad} turns into
\begin{align*}
    \|u\|_{L^2(\gamma^-,\gamma^+)}^2
    &\leq (1+\epsilon) \max\lbrace \alpha_-^{-1}, \alpha_+^{-1}, 1 \rbrace \bigg[(2\alpha_-+\kappa_-)\sqrt{1+(h')^2}u_-^2 
    \\
    &\qquad +(2\alpha_++\kappa_+)\sqrt{1+(h')^2}u_+^2 +\left(\frac{5}{16}+(2\epsilon)^{-1}\right) \|\partial_2 u\|_{L^2(\gamma^-,\gamma^+)}^2\bigg].
\end{align*}
Integrating in $y_1$ and choosing $\epsilon=3$ yields
\begin{align*}
    \|u\|_2^2 &\leq 4 \max\lbrace \underline{\alpha}^{-1}, 1 \rbrace \left[ \int_{\gamma^-\cup\gamma^+} (2\alpha+\kappa) u_\tau^2 + \frac{1}{2}\|\partial_2 u\|_2^2\right]
\end{align*}
which implies the following bound for the $H^1$-norm
\begin{align*}
    \|u\|_{H^1}^2 &\leq 4 \max\lbrace \underline{\alpha}^{-1}, 1 \rbrace \left[ \int_{\gamma^-\cup\gamma^+} (2\alpha+\kappa) u_\tau^2 + \frac{3}{4} \|\nabla u\|_2^2\right].
\end{align*}

It is only left to show that \eqref{energy-decay-kappa-removed-estimate} holds. In order to prove the claim we distinguish between two cases.
\begin{itemize}
    \item
    If $|\kappa| \leq 2\alpha$, then both $2\alpha_\pm+\kappa_\pm>0$ and as $\kappa_-=-\kappa_+$ either $\kappa_-$ or $\kappa_+$ is non-negative. Assume first $\kappa_-\geq 0$. Then by these observations
    \begin{align*}
        %\phantom{}% intending because of the itemize
        \min\lbrace&\alpha_-,\alpha_+\rbrace\sqrt{1+(h')^2}u_{i}^2 
        \\&\leq 2\alpha_- \sqrt{1+(h')^2}u_{-}^2 
        \\
        &\leq (2\alpha_- + \kappa_-) \sqrt{1+(h')^2}u_{-}^2 + (2\alpha_+ + \kappa_+) \sqrt{1+(h')^2}u_{+}^2
    \end{align*}
    The case $\kappa_+>0$ follows similar with $u_+^2$ instead of $u_-^2$.
    \item
    Now assume that $\kappa_+<0$ and $|\kappa_+|>2\alpha_+$. The case $\kappa_-<0$ and $|\kappa_-|>2\alpha_-$ follows similarly by exchanging $+$ and $-$. Using \eqref{fundamental-theorem-calc-estimate-for-u_minus} with $y_2 = 1 + h(y_1)$, respectively \eqref{fundamental-theorem-calc-estimate-for-u_plus} with $y_2 = h(y_1)$, and observing that $\kappa_-=-\kappa_+>2\alpha_+>0$ it holds
    \begin{align*}
        %\phantom{}% intending because of the itemize
        - (2\alpha_-+\kappa_-)&\sqrt{1+(h')^2}u_-^2 -(2\alpha_++\kappa_+)\sqrt{1+(h')^2}u_+^2
        \\
        &= - (2\alpha_-+\kappa_-)\sqrt{1+(h')^2}u_-^2 + (\kappa_--2\alpha_+)\sqrt{1+(h')^2}u_+^2
        \\
        &\leq (\kappa_--2\alpha_+)(1+\epsilon^{-1})\sqrt{1+(h')^2}\|\partial_2 u\|_{L^2(\gamma^-,\gamma^+)}^2
        \\
        &\qquad - \left[2\alpha_-+2\alpha_+-\epsilon(\kappa_--2\alpha_+)\right]\sqrt{1+(h')^2}u_-^2.
    \end{align*}
    In order for the squared bracket to be positive we choose $\epsilon = \frac{\alpha_-+\alpha_+}{\kappa_--2\alpha_+}$ to get
    \begin{align*}
        %\phantom{}% intending because of the itemize
        - (2\alpha_-+\kappa_-)&\sqrt{1+(h')^2}u_-^2 -(2\alpha_++\kappa_+)\sqrt{1+(h')^2}u_+^2
        \\
        &\leq \left(\kappa_--2\alpha_+ + \frac{(\kappa_- - 2 \alpha_+)^2}{\alpha_- + \alpha_+}\right)\sqrt{1+(h')^2}\|\partial_2 u\|_{L^2(\gamma^-,\gamma^+)}^2
        \\
        &\qquad - \left[\alpha_-+\alpha_+\right]\sqrt{1+(h')^2}u_-^2.
    \end{align*}
    Then, as the smallness condition \eqref{energy_condition_explicite_kappa} implies
    \begin{align*}
        \kappa_- &= |\kappa_+| \leq 2\alpha_+ + \frac{1}{4\sqrt{1+(h')^2}} \min\left\lbrace 1, \sqrt{\alpha_+} \right\rbrace
        \\
        &\leq 2\alpha_+ + \frac{1}{4} \min\left\lbrace \frac{1}{\sqrt{1+(h')^2}}, \frac{\sqrt{\alpha_++\alpha_-}}{(1+(h')^2)^\frac{1}{4}} \right\rbrace,
    \end{align*}
    we get \vspace{-5pt}    \begin{align*}
        %\phantom{}% intending because of the itemize
        - (2\alpha_-+\kappa_-)&\sqrt{1+(h')^2}u_-^2 -(2\alpha_++\kappa_+)\sqrt{1+(h')^2}u_+^2
        \\
        &\leq  \frac{5}{16}\|\partial_2 u\|_{L^2(\gamma^-,\gamma^+)}^2 - \left[\alpha_-+\alpha_+\right]\sqrt{1+(h')^2}u_-^2,
    \end{align*}
    proving the claim.
\end{itemize}
\vspace{-5pt}
\end{proof}

\begin{remark}
Note that this Lemma can be improved: In fact for every $\kappa$ and $\alpha$ with $\kappa_\mp < \sqrt{\frac{2\alpha_\mp+2\alpha_\pm}{\sqrt{1+(h')^2}}+(2\alpha_\pm)^2}$,
where we use the notation \eqref{defKappaAlphaUpm}, it holds $\|\nabla u\|_2^2+\int (2\alpha+\kappa)u_\tau^2 > 0$. This implies energy decay in \eqref{energy-balance}.
\newline
Nevertheless, in order to simplify the estimates and improve readability we will work with assumption \eqref{energy_condition_explicite_kappa} instead. This choice will have no effects in terms of optimality of the bounds for the Nusselt number. 
%The constant on the right-hand side could also be improved by removing the prefactor in front of $\|\nabla u\|_2^2$, but this factor of $\frac{3}{4}$ will be exploited in Theorem \ref{main-theorem}.
\end{remark}


We will use the energy balance and \eqref{H1-bounded-by-grad-and-bdry-terms} to prove the following decay estimate for the energy of $u$:
\begin{lemma}[Energy Decay]
\label{lemmaEnergyDecay}
Let the assumptions of Lemma \ref{lemma-H1-bounded-by-grad-and-bdry-terms} be satisfied. Then the energy of $u$ is bounded by
\begin{align}
    \label{Energy-decay}
    \|u\|_2^2 \leq e^{-\frac{1}{4}\min\lbrace 1, \underline{\alpha}\rbrace \Pra\phantom{.}  t}\|u_0\|_2^2 + 256 \max\left\lbrace 1, \underline{\alpha}^{-2}\right\rbrace|\Omega|\Ra^2.
\end{align}
\end{lemma}

%\begin{remark}
%If the condition \eqref{energy_condition_explicite_kappa} is changed to
%\begin{align*}
%    \kappa_\pm < \sqrt{\frac{2\alpha_\pm+2\alpha_\mp}{\sqrt{1+(\gamma')^2}}+(2\alpha_\mp)^2}
%\end{align*}
%one still gets a uniform energy bound
%\begin{align*}
%    \|u\|_2^2 \leq e^{-C\Pra\ t} \|u_0\|_2^2 + C \Ra^2,
%\end{align*}
%where $C>0$ depends on $\alpha$, $\kappa$ and $|\Omega|$. This condition has the advantage that for portions of the boundary with big curvature $\kappa$ it suffices that only one of the slip coefficients either top or bottom in that portion suffices to get the energy bound. \\
%In what follows we will consider problems with small curvature $|\kappa|\sim |\alpha|\ll 1$ for which \eqref{energy_condition_explicite_kappa} is fulfilled anyways.
%\todo[inline]{Actually this condition isn't much better than the one where we have the explicit constant. Also I am not happy with how this is formulated}
%\end{remark}

\begin{proof}
By the energy balance \eqref{energy-balance}, Young's inequality
%\begin{align*}
%    \frac{1}{2 \Pra} \frac{d}{dt}\|u\|_2^2 
%    &= -\|\nabla u\|_2^2 - \int_{\gamma^+\cup \gamma^-} (2\alpha+\kappa) u_\tau^2 + \Ra\int_\Omega T u_2
%    \\
%    &\leq -\|\nabla u\|_2^2 - \int_{\gamma^+\cup \gamma^-} (2\alpha+\kappa) u_\tau^2 + \epsilon \|u_2\|_2^2 + \frac{4}{\epsilon}\Ra^2 \|T\|_2^2 
%\end{align*}
%for all $\epsilon>0$. As $\|T\|_\infty \leq 1$ by the 
 and the maximum principle \eqref{maximum-principle}, we obtain
\begin{align}
    \label{energy-estimate-derivative}
    \frac{1}{2 \Pra} \frac{d}{dt}\|u\|_2^2 
    &\leq -\|\nabla u\|_2^2 - \int_{\gamma^+\cup \gamma^-} (2\alpha+\kappa) u_\tau^2 + \epsilon \|u_2\|_2^2 + \frac{4}{\epsilon}|\Omega|\Ra^2\,.
\end{align}
Plugging \eqref{H1-bounded-by-grad-and-bdry-terms} into \eqref{energy-estimate-derivative}, we find
\begin{align*}
    \frac{1}{2 \Pra} \frac{d}{dt}\|u\|_2^2 
    %&\leq -\|\nabla u\|_2^2 - \int_{\gamma^+\cup \gamma^-} (2\alpha+\kappa) u_\tau^2 + \epsilon \|u_2\|_2^2 + \frac{4}{\epsilon}|\Omega|\Ra^2
    %\\
    &\leq -\left(\frac{1}{4}\min\lbrace 1, \underline{\alpha}\rbrace -\epsilon\right) \|u\|_2^2 + \frac{4}{\epsilon}|\Omega|\Ra^2\,,
\end{align*}
and choosing $\epsilon=\frac{1}{8}\min\lbrace 1, \underline{\alpha}\rbrace$ yields
\begin{align*}
    \frac{d}{dt}\|u\|_2^2 
    &\leq -\frac{1}{4}\min\lbrace 1, \underline{\alpha}\rbrace \Pra  \|u\|_2^2 + 64 \Pra \max \lbrace 1, \underline{\alpha}^{-1}\rbrace |\Omega|\Ra^2\,.
\end{align*}
Applying Gr\"onwall's inequality we obtain \eqref{Energy-decay}.
%\begin{align*}
%    \|u\|_2^2 \leq e^{-\frac{1}{4}\min\lbrace 1, \underline{\alpha}\rbrace \Pra\phantom{.}  t}\|u_0\|_2^2 + 256 \max\left\lbrace 1, \underline{\alpha}^{-2}\right\rbrace|\Omega|\Ra^2.
%\end{align*}
\end{proof}
Taking the long-time average of the energy balance \eqref{energy-balance}, using the fact that
    $$\limsup_{t\to\infty}\frac{1}{t}\int_0^t \frac{d}{dt}\|u\|_2^2 = \limsup_{t\to\infty}\frac{1}{t}\left(\|u\|_2^2-\|u_0\|_2^2\right)=0$$
thanks to the uniform bound \eqref{Energy-decay} one gets
\begin{align*}
    \langle |\nabla u|^2\rangle + \langle (2\alpha+\kappa) u_\tau^2\rangle_{\gamma^-\cup\gamma^+}= \Ra \langle T u_2 \rangle
\end{align*}
and observing that, by \eqref{nusselt-ineq}, 
\begin{equation*}
     \Ra\langle T u_2\rangle
    =  \Ra(\langle u_2T-\partial_2 T\rangle -1)
    \leq \Ra\left((1+\max h - \min h) \Nu-1\right),
\end{equation*}
%where in the inequality we have used \eqref{nusselt-ineq}
we deduce the following
\begin{corollary}
\begin{align}
    \label{average-energy-balance}
    \langle|\nabla u|^2\rangle + \langle (2\alpha+\kappa) u_\tau^2\rangle_{\gamma^+\cup \gamma^-}\leq \Ra\left((1+\max h - \min h) \Nu-1\right).
\end{align}
\end{corollary}




\subsection{A-priori estimate for the vorticity}
We now introduce the two-dimensional vorticity $\omega=\nabla^{\perp}\cdot u$, where $\nabla^\perp = (-\partial_2,\partial_1)$. It is easy to see that $\omega$ satisfies the equation
\begin{equation}
    \label{vorticity-equation}
    \begin{aligned}
        \frac{1}{\Pra}\left(\omega_t +(u\cdot\nabla)\omega\right)-\Delta \omega &=\Ra \partial_1 T & \textnormal{ in }&\Omega,\\
        \omega &= -2(\alpha+\kappa)u_\tau & \textnormal{ on }&\gamma^\pm.
    \end{aligned}
\end{equation}
Notice that the boundary term is deduced from the following computation 
\begin{align*}
    \omega&=\omega (\tau\cdot\tau)= \omega (-\tau_1 n_2 +\tau_2 n_1) = \tau_i n_j (-\partial_i u_j+\partial_j u_i)\\
    &=2 \tau \cdot \mathbb{D}u \cdot n- 2n\cdot(\tau\cdot \nabla) u = -2(\alpha+\kappa)u_\tau ,
\end{align*}
where 
%in the first identity
we used that $\tau=(-n_2,n_1)$, the boundary conditions \eqref{BC-u} and the identity 
\begin{equation}
    \label{id-kappa}
    \kappa u_{\tau}=n\cdot (\tau\cdot \nabla )u,
\end{equation}
proved in \eqref{appendix-proof-id-kappa} in the Appendix.


\begin{proposition}\label{vorticity-balance}
The following vorticity balance holds
%\begin{flalign*}
%    \frac{1}{2\Pra}\frac{d}{dt}\|\omega\|^2 &+ \mathrlap{\frac{1}{\Pra}\frac{d}{dt}\int_{\gamma^-,\gamma^+} (\alpha+\kappa)u_\tau^2 +\|\nabla \omega\|^2} && \\
%    &= \mathrlap{-2 \int_{\gamma^-\cup\gamma^+} (\alpha+\kappa) u \cdot \nabla p +\Ra\int_{\Omega} \omega \partial_1 T} && \\
%    & &&&{}- \frac{2}{3\Pra}\int_{\gamma^-\cup\gamma^+} (\alpha+\kappa) u\cdot (u\cdot\nabla) u 
%    -2 \Ra \int_{\gamma^-\cup\gamma^+} (\alpha+\kappa)u_\tau T n_1
%\end{flalign*}
\begin{align*}
    \frac{1}{2\Pra}\frac{d}{dt}&\|\omega\|^2 +\frac{1}{\Pra}\frac{d}{dt}\int_{\gamma^-,\gamma^+} (\alpha+\kappa)u_\tau^2 +\|\nabla \omega\|_2^2 - \Ra\int_{\Omega} \omega \partial_1 T 
    \\
    &=
    -2 \int_{\gamma^-\cup\gamma^+} (\alpha+\kappa) u \cdot \nabla p
    - \frac{2}{3\Pra}\int_{\gamma^-\cup\gamma^+} (\alpha+\kappa) u\cdot (u\cdot\nabla) u 
    \\
    &\qquad -2 \Ra \int_{\gamma^-} (\alpha+\kappa)u_\tau n_1.
\end{align*}
\end{proposition}

\begin{proof}
Testing the vorticity equation with $\omega$ yields
\begin{equation}
    \label{vort-id}
    \frac{1}{2\Pra}\frac{d}{dt}\|\omega\|_2^2 = -\frac{1}{\Pra}\int_{\Omega} \omega (u\cdot \nabla)\omega + \int_{\Omega} \omega\Delta\omega +\Ra\int_{\Omega} \omega \partial_1 T
\end{equation}
Using the incompressibility condition, it is easy to see that the first term on the right-hand side vanishes since $u\cdot n=0$ at $\partial\Omega$.
In order to analyze the second term, we first notice that 
\begin{equation}
    \label{vort-balance-bdry-id-a}
    n\cdot\nabla\omega=\tau \cdot \Delta u=\frac{1}{\Pra}\tau \cdot u_t + \frac{1}{\Pra}\tau \cdot (u\cdot \nabla)u +\tau\cdot \nabla p -\Ra T n_1\,,
\end{equation}
where we used incompressibility in the first identity. Then, using the boundary conditions for the vorticity and temperature and \eqref{vort-balance-bdry-id-a}, we have 
\begin{align*}
    \int_{\Omega}\omega\Delta\omega 
    %&=& -\|\nabla \omega\|^2 +\int_{\partial\Omega} \omega n\cdot\nabla \omega\\
    &= -\|\nabla \omega\|_2^2 +\int_{\gamma^-\cup\gamma^+} \omega n\cdot\nabla \omega\\
    &= -\|\nabla \omega\|_2^2 -2 \int_{\gamma^-\cup\gamma^+} (\alpha+\kappa)u_\tau n\cdot\nabla \omega\\
    &=-\|\nabla \omega\|_2^2 -\frac{2}{\Pra}\int_{\gamma^-\cup\gamma^+} (\alpha+\kappa)u_\tau \tau \cdot u_t - \frac{2}{\Pra}\int_{\gamma^-\cup\gamma^+} (\alpha+\kappa)u_\tau \tau \cdot (u\cdot \nabla)u \\
    &\qquad -2 \int_{\gamma^-\cup\gamma^+} (\alpha+\kappa)u_\tau \tau \cdot \nabla p + 2 \Ra \int_{\gamma^-\cup\gamma^+} (\alpha+\kappa)u_\tau T n_1\\
    &= -\|\nabla \omega\|_2^2 -\frac{1}{\Pra}\frac{d}{dt}\int_{\gamma^-\cup\gamma^+} (\alpha+\kappa)u_\tau^2 - \frac{2}{\Pra}\int_{\gamma^-\cup\gamma^+} (\alpha+\kappa) u \cdot (u\cdot \nabla)u \\
    &\qquad -2 \int_{\gamma^-\cup\gamma^+} (\alpha+\kappa) u \cdot \nabla p + 2 \Ra \int_{\gamma^-} (\alpha+\kappa)u_\tau  n_1\,.
    %&=-\|\nabla \omega\|^2 -\frac{1}{\Pra}\frac{d}{dt}\int_{\gamma^-,\gamma^+} (\alpha+\kappa)u_\tau^2 - \frac{2}{3\Pra}\int_{\gamma^-,\gamma^+} (\alpha+\kappa) \frac{d}{d\lambda}u_\tau^3 \\
    %&\qquad -2 \int_{\gamma^-,\gamma^+} (\alpha+\kappa) u \cdot \nabla p - 2 \Ra \int_{\gamma^-,\gamma^+} (\alpha+\kappa)u_\tau T n_1.
\end{align*}
Plugging it into \eqref{vort-id} yields
\begin{align*}
    \frac{1}{2\Pra}\frac{d}{dt}\|\omega\|_2^2
    %&= -\frac{1}{\Pra}\int_{\Omega} \omega (u\cdot \nabla)\omega + \int_{\Omega} \omega\Delta\omega +\Ra\int_{\Omega} \omega \partial_1 T
    %\\
    &=\int_{\Omega} \omega\Delta\omega +\Ra\int_{\Omega} \omega \partial_1 T
    \\
    &=-\|\nabla \omega\|_2^2 -\frac{1}{\Pra}\frac{d}{dt}\int_{\gamma^-\cup\gamma^+} (\alpha+\kappa)u_\tau^2 - \frac{2}{\Pra}\int_{\gamma^-\cup\gamma^+} (\alpha+\kappa) u\cdot (u\cdot\nabla) u 
    \\
    &\qquad -2 \int_{\gamma^-\cup\gamma^+} (\alpha+\kappa) u \cdot \nabla p -2 \Ra\int_{\gamma^-} (\alpha+\kappa)u_\tau n_1 +\Ra\int_{\Omega} \omega \partial_1 T\,.
\end{align*}
Let us observe that the term $\int(\alpha+\kappa) u\cdot (u\cdot\nabla) u$ is, in general, non-zero as the parameters $\alpha$ and $\kappa$ depend on the space variables.
\end{proof}

%In order to interchange between bounds on the vorticity and the gradients of the velocity we need the following Lemma. 
We now want to relate the $L^2$-norm of the vorticity with the $L^2$-norm of the enstrophy.


\begin{lemma}
\label{lemma_u_bounded_by_omega}
Let $2\leq q \leq p$. If $h\in W^{3,\infty}$ and $\omega\in W^{1,p}$
\begin{equation*}
    \begin{aligned}
        \|\nabla u\|_2^2 &= \|\omega\|_2^2 + \int_{\gamma^-\cup\gamma^+} \kappa u_\tau^2
        \\
        \| u \|_{W^{1,p}} &\leq C \left(\|\omega\|_p + \left(1+\|\kappa\|_\infty^{1+\frac{2}{q}-\frac{2}{p}}\right)\|u\|_q\right)
        \\
        \| u \|_{W^{2,p}} &\leq C \left(\|\nabla \omega\|_p + (1+\|\kappa\|_\infty)\| \omega\|_p +\left(1+\|\kappa\|_\infty^2 + \|\dot\kappa\|_\infty\right)\|u\|_p\right)
    \end{aligned}
\end{equation*}
holds, where the constant $C$ only depends on $p$, $|\Omega|$ and $\|h'\|_{\infty}$.
\end{lemma}


\begin{remark}
Note that in the case of flat boundaries the estimates simplify to
\begin{align*}
    \|\nabla u\|_2^2 = \|\omega\|_2^2, \qquad \|\nabla u\|_{W^{m,p}} \leq C \|\omega\|_{W^{m,p}}
\end{align*}
as proven in Lemma 7 in \cite{drivasNguyenNobiliBoundsOnHeatFluxForRayleighBenardConvectionBetweenNavierSlipFixedTemperatureBoundaries}.
\end{remark}

\begin{proof}
\leavevmode
\begin{itemize}
    \item
    %By the incompressibility condition of $u$
    %which, together with 
    Integrating by parts twice, we find
    \begin{equation}
        \label{appendix-gradu-equals-omega+boundary-1}
        \begin{aligned}
            %\phantom{}% intending because of the itemize
            \|\nabla u\|_2^2 &= \int_{\gamma^-\cup\gamma^+} u \cdot (n \cdot \nabla) u - \int u \cdot \Delta u 
            \\
            %&= \int_{\gamma^-\cup\gamma^+} u \cdot (n \cdot \nabla) u - \int u \cdot \nabla^\perp \omega
            %\\
            &= \int_{\gamma^-\cup\gamma^+} u \cdot (n \cdot \nabla) u + \int u^\perp \cdot \nabla \omega 
            \\
            %&=  \int_{\gamma^-\cup\gamma^+} u \cdot (n \cdot \nabla) u + \int_{\gamma^-\cup\gamma^+} n \cdot u^\perp \omega - \int \nabla \cdot u^\perp \omega
            %\\
            &= \int_{\gamma^-\cup\gamma^+} u \cdot (n \cdot \nabla) u - \int_{\gamma^-\cup\gamma^+} u_\tau \omega + \|\omega\|_2^2,
        \end{aligned}
    \end{equation}
    where we used the identity
     \begin{align*}
        %\phantom{}% intending because of the itemize
        \nabla^\perp \omega = \begin{pmatrix}\partial_2^2 u_1 -\partial_1\partial_2 u_2\\-\partial_1\partial_2 u_1 +\partial_1^2 u_2\end{pmatrix} = \Delta u,
    \end{align*}
    due to incompressibility.
    Next notice that $\tau_i\tau_j + n_in_j = \delta_{ij}$. Therefore the second boundary term of the right-hand side of \eqref{appendix-gradu-equals-omega+boundary-1} can be rewritten as
    \begin{equation}
        \label{appendix-gradu-equals-omega+boundary-2}
        \begin{aligned}
            %\phantom{}% intending because of the itemize
            -\int_{\gamma^-\cup\gamma^+} u_\tau \omega &= -\int_{\gamma^-\cup\gamma^+} u_\tau \tau \cdot (\tau\cdot \nabla^\perp) u - \int_{\gamma^-\cup\gamma^+} u_\tau n \cdot (n\cdot \nabla^\perp) u
            \\
            %&= -\int_{\gamma^-\cup\gamma^+} u \cdot (\tau\cdot \nabla^\perp) u - \int_{\gamma^-\cup\gamma^+} u_\tau n \cdot (n\cdot \nabla^\perp) u
            %\\
            %&= \int_{\gamma^-\cup\gamma^+} u \cdot (\tau^\perp\cdot \nabla) u + \int_{\gamma^-\cup\gamma^+} u_\tau n \cdot (n^\perp\cdot \nabla) u
            %\\
            &= -\int_{\gamma^-\cup\gamma^+} u \cdot (n\cdot \nabla) u + \int_{\gamma^-\cup\gamma^+} n \cdot (u\cdot \nabla) u,
        \end{aligned}
    \end{equation}
    where in the last identity we used that $\tau\cdot \nabla^\perp = -\tau^\perp \cdot \nabla=n \cdot\nabla$ and $u_\tau n\cdot\nabla^\perp= -u_\tau\cdot n^\perp \cdot\nabla = -u_\tau \tau  \cdot\nabla=-u\cdot\nabla$.
    The first term on the right-hand side of \eqref{appendix-gradu-equals-omega+boundary-1} cancels with the first term on the right-hand side of \eqref{appendix-gradu-equals-omega+boundary-2}, implying
    \begin{align*}
        %\phantom{}% intending because of the itemize
        \|\nabla u\|_2^2 = \int_{\gamma^-\cup\gamma^+} n\cdot (u\cdot \nabla) u + \|\omega\|_2^2
    \end{align*}
    Finally using 
    \begin{align}
        \label{id-kappaUtau2-2}
        %\phantom{}% intending because of the itemize
        n\cdot (u\cdot \nabla) u = \kappa u_\tau^2
    \end{align}
    on $\gamma^-\cup\gamma^+$, which is proven in \eqref{appendix-proof-id-kappaUtau2} in the Appendix, yields the claim.
    \item 
    Let $\phi$ be the stream function of $u$, i.e. $\nabla^\perp\phi=u$, then
    \begin{align*}
        %\phantom{}% intending because of the itemize
        \Delta \phi &= \omega\\
        \phi\vert_{\gamma^\pm} &= \phi_\pm
    \end{align*}
    with constants $\phi_+$ and $\phi_-$ and without loss of generality set $\phi_-=0$. We can calculate $\phi_+$ by
    \begin{align*}
        %\phantom{}% intending because of the itemize
        \phi_+ &= \frac{1}{|\Omega|} \int_0^\Gamma \phi_+ dy_1 = \frac{1}{|\Omega|} \int_0^\Gamma \left[\phi_- + \int_{h(y_1)}^{1+h(y_1)} \partial_2 \phi\ dy_2\right] dy_1 = -\frac{1}{|\Omega|}\int_{\Omega}u_1 \ dy.
    \end{align*}
    Therefore $\bar \phi = \phi + (y_2-h(y_1))\frac{1}{|\Omega|}\int_{\Omega} u_1\ dy$ solves
    \begin{equation*}
        \begin{aligned}
            %\phantom{}% intending because of the itemize
            \Delta \bar\phi &= \omega - h''\frac{1}{|\Omega|}\int_{\Omega} u_1 \ dy
            \\
            \bar\phi\vert_{\gamma^\pm} &= 0.
        \end{aligned}
    \end{equation*}
    In order to flatten the boundary we introduce the change of variables
    \begin{align*}
        %\phantom{}% intending because of the itemize
        x=\Phi(y) = \begin{pmatrix}y_1\\y_2 - h(y_1)\end{pmatrix}, \qquad y=\Psi(x) = \begin{pmatrix}x_1\\x_1 + h(x_1)\end{pmatrix}.
    \end{align*}
    Here and in the rest of the paper $C>0$ denotes a constant that possibly depends on $\|h'\|_{\infty}$, the size of the domain $|\Omega|$ and the Sobolev exponent and may change from line to line. Note that
    \begin{align*}
        \|h'\|_\infty \leq C,\qquad \|h''\|_\infty \leq C\|\kappa\|_\infty,\qquad \|h'''\|_\infty\leq C (\|\dot\kappa\|_\infty+\|\kappa\|_\infty),
    \end{align*}
    and for $\xi(x)=\chi(\Psi(x))=\chi(y)$ one has
    \begin{equation}
        \label{changeOfVariablesDerivative}
        \begin{aligned}
            \|\nabla \xi\|_p&\leq C\|\nabla\chi\|_p
            \\
            \|\nabla^2\xi\|_p&\leq C(\|\nabla^2\chi\|_p+\|\kappa\|_\infty \|\nabla \chi\|_p)
            \\
            \|\nabla^3\xi\|_p &\leq C\left(\|\nabla^3\chi\|_p+\|\kappa\|_\infty\|\nabla^2\chi\|_p+(\|\dot\kappa\|_\infty+\|\kappa\|_\infty)\|\nabla \chi\|_p\right)
        \end{aligned}
    \end{equation}
    and analogous for the transformation in the other direction.
    Then
    \begin{equation}
        \label{elliptic-regularity-pde-tilde-phi}
        \begin{aligned}
            \tilde L\tilde \phi&=\tilde f & \textnormal{ in }&[0,\Gamma]\times [0,1]
            \\
            \tilde \phi &= 0  & \textnormal{ on }&[0,\Gamma]\times \lbrace x_2=0\rbrace \cup \lbrace x_2=1\rbrace,
        \end{aligned}
    \end{equation}
    where $\tilde L \tilde \phi=\sum_{i,j} \partial_{x_i}(\tilde a_{i,j}\partial_{x_j}\tilde \phi(x))$ with $\tilde a_{1,1}=1$, $\tilde a_{1,2}=\tilde a_{2,1}=-h'$ and $\tilde a_{2,2}=1+(h')^2$, $\tilde \phi(x)=\bar \phi(\Psi(x))$ and $\tilde f=\tilde\omega-h''\frac{1}{|\Omega|}\int_{\Omega} u_1 dy$ with $\tilde \omega (x) = \omega(\Psi(x))$. As this operator is elliptic we get
    \begin{align}
        \label{appendix-straightened-elliptic-reg-1}
        \|\tilde \phi\|_{W^{2,p}}\leq C \|\tilde f\|_p
    \end{align}
    for some constant $C>0$ depending only on $p$, $|\Omega|$ and $\|h'\|_{\infty}$. Using Hölder's inequality and the estimates for the change of variables \eqref{changeOfVariablesDerivative}, \eqref{appendix-straightened-elliptic-reg-1} becomes
    \begin{align}
        \label{elliptic-regularity-phi-W2p}
        \|\tilde\phi\|_{W^{2,p}} \leq C\|\tilde f\|_p \leq C \|\omega\|_p + C \|\kappa\|_\infty \|u\|_p.
    \end{align}
    Going back to the definition of $\bar \phi$, we find that \eqref{changeOfVariablesDerivative}, Hölder's inequality and \eqref{elliptic-regularity-phi-W2p} yield
    \begin{equation}
        \label{ellptic-regularity-gradu-by-omega}
        \begin{aligned}
            \|\nabla u\|_p &= \|\nabla^2\phi\|_p \leq \|\nabla^2\bar \phi\|_p + C\|h''\|_\infty \|u_1\|_1
            \\
            &
            \leq C \left(\|\nabla^2 \tilde \phi\|_p + \|\kappa\|_\infty \|\nabla \tilde\phi\|_p + \|h''\|_\infty \|u_1\|_1\right)
            \\
            &
            \leq C \left(\|\omega\|_p+ \|\kappa\|_\infty \|u\|_p\right),
        \end{aligned}
    \end{equation}
    implying
    \begin{align}
        \label{elliptic-regularity-u-in-Wonep-bound}
        \|u\|_{W^{1,p}} &\leq C \left(\|\omega\|_p+(1+ \|\kappa\|_\infty) \|u\|_p\right).
    \end{align}
    In order to estimate the $L^p$-norm of $u$ by the $L^q$-norm use interpolation and Young's inequality to get
    \begin{align*}
        \|u\|_p \leq C \|\nabla u\|_p^\theta \|u\|_q^{1-\theta} + C \|u\|_q \leq \epsilon \theta C \|\nabla u\|_p + C \left( (1-\theta)\epsilon^{-\frac{\theta}{1-\theta}}+1\right)\|u\|_q 
    \end{align*}
    for $\frac{1}{p}=\theta \left(\frac{1}{p}-\frac{1}{2}\right)+\frac{1-\theta}{q}$ and all $\epsilon>0$. Then choosing $\epsilon^{-1}=(1+\|\kappa\|_\infty)$ and plugging in $\theta = \frac{2(q-p)}{2(q-p)-pq}$
    \begin{align*}
        (1+\|\kappa\|_\infty) \|u\|_p &\leq C \|\nabla u\|_p + C\left(1+\|\kappa\|_\infty^{\frac{1}{1-\theta}}\right)\|u\|_q 
        \\
        &\leq C \|\nabla u\|_p + C\left(1+\|\kappa\|_\infty^{1+\frac{2}{q}-\frac{2}{p}}\right)\|u\|_q
    \end{align*}
    proving the claim.

    \item In order to prove the $W^{2,p}$ bound notice that by \eqref{elliptic-regularity-pde-tilde-phi} $\hat \phi = \partial_{x_1}\tilde \phi$ solves
    \begin{equation*}
        \begin{aligned}
            \tilde L\hat\phi&=\hat f & \textnormal{ in }&[0,\Gamma]\times [0,1]
            \\
            \hat\phi &= 0  & \textnormal{ on }& \lbrace x_2=0\rbrace \cup \lbrace x_2=1\rbrace,
        \end{aligned}
    \end{equation*}
    with $\hat f = \partial_{x_1}\tilde\omega + h'''\frac{1}{|\Omega|}\int_{\Omega} u_1 dy + \partial_{x_1}(h''\partial_{x_2}\tilde \phi)+\partial_{x_2}(h''\partial_{x_1}\tilde \phi) - 2h'h'' \partial_{x_2}^2 \tilde \phi \in L^p$. Again using elliptic regularity and Hölder's inequality we find
    \begin{equation}
        \label{elliptic-regularity-ddx1W2}
        \begin{aligned}
            \|\partial_{x_1}\tilde \phi\|_{W^{2,p}} &= \|\hat\phi\|_{W^{2,p}}\leq C \|\hat f\|_p
            \\
            &\leq C\left(\|\nabla \tilde\omega\|_p + \|\dot\kappa\|_\infty (\|u\|_p + \|\nabla \tilde\phi\|_p) + \|\kappa\|_\infty \|\nabla^2\tilde\phi\|_p\right).
        \end{aligned}
    \end{equation}
    In order to estimate the missing term $\partial_{x_2}^3\tilde \phi$ notice as $h'$ is independent of $x_2$
    \begin{equation}
        \label{elliptic-regularity-dx2hoch3}
        \begin{aligned}
            \partial_{x_2}^3\tilde \phi &= \frac{1}{1+(h')^2}\partial_{x_2} \left(\partial_{x_2}((1+(h')^2)\partial_{x_2}\tilde \phi\right) 
            \\
            &= \frac{1}{1+(h')^2}\partial_{x_2} \left(\tilde L\tilde\phi-\partial_{x_1}^2\tilde \phi +\partial_{x_1}(h'\partial_{x_2}\tilde \phi ) + \partial_{x_2} (h'\partial_{x_1}\tilde \phi)\right)
            \\
            &= \frac{1}{1+(h')^2}\partial_{x_2} \left(\tilde f - \partial_{x_1}^2 \tilde \phi + \partial_{x_1} (h'\partial_{x_2}\tilde \phi)+\partial_{x_2}(h'\partial_{x_1}\tilde \phi)\right).
        \end{aligned}
    \end{equation}
    Taking the norm in \eqref{elliptic-regularity-dx2hoch3} we find
    \begin{equation}
        \label{appendix-stream-function-ddx2hoch3}
        \begin{aligned}
            \|\partial_{x_2}^3\tilde \phi\|_{p}&\leq C \left( \|\nabla \tilde f\|_p + \|\kappa\|_\infty \|\partial_{x_2}^2\tilde\phi\|_p + \|\partial_{x_1}\tilde \phi\|_{W^{2,p}} \right).
        \end{aligned}
    \end{equation}
    Combining \eqref{elliptic-regularity-ddx1W2} and \eqref{appendix-stream-function-ddx2hoch3}
    \begin{align*}
        \|\tilde\phi\|_{W^{3,p}}\leq C \left(\|\nabla \tilde f\|_p + \|\kappa\|_\infty \|\nabla^2\tilde\phi\|_p +\|\nabla \tilde\omega\|_p + \|\dot\kappa\|_\infty (\|u\|_p + \|\nabla \tilde\phi\|_p)  \right).
    \end{align*}
    Using Hölder's inequality we find
    \begin{align*}
        \|\nabla \tilde f\|_p \leq C(\|\nabla \tilde \omega\|_p + \|\dot\kappa\|_\infty \|u\|_p),
    \end{align*}
    which together with \eqref{elliptic-regularity-phi-W2p} yields
    \begin{equation}
        \label{elliptic-regularity-phi-W3p}
        \begin{aligned}
            \|\tilde\phi\|_{W^{3,p}}&\leq C \left(\|\nabla \tilde\omega\|_p + \|\kappa\|_\infty \|\omega\|_p+ \|\kappa\|_\infty^2 \|u\|_p + \|\dot\kappa\|_\infty (\|u\|_p + \|\nabla \tilde\phi\|_p) \right)
            \\
            &\leq  C \left(\|\nabla \omega\|_p + \|\kappa\|_\infty \|\omega\|_p+ (\|\kappa\|_\infty^2 + \|\dot\kappa\|_\infty) \|u\|_p \right),
        \end{aligned}
    \end{equation}
    where in the last inequality we used
    \begin{align}
        \label{elliptic-regularity-gradphi}
        \|\nabla\tilde\omega\|_p\leq C \|\nabla\omega\|_p, \qquad \|\nabla\tilde\phi\|_p\leq C\|\nabla\bar\phi\|_p \leq C (\|\nabla\phi\|_p+\|u_1\|_1)\leq C\|u\|_p.
    \end{align}
    By the definitions and the change of variables estimate \eqref{changeOfVariablesDerivative} and Hölder's inequality one gets
    \begin{align*}
        \|\nabla^2 u\|_p &= \|\nabla^3 \phi\|_p\leq \|\nabla^3\bar \phi\|_p + h'''\frac{1}{|\Omega|}\int |u_1| \\
        &\leq C\left(\|\nabla^3 \tilde\phi\|_p + \|\kappa\|_\infty \|\nabla^2 \tilde\phi\|_p + (\|\dot\kappa\|_\infty+\|\kappa\|_\infty)\|\nabla\tilde\phi\|_p + \|\kappa\|_\infty\|u\|_p \right)
        \\
        &\leq C\left(\|\nabla \omega\|_p +\|\kappa\|_\infty\|\omega\|_{p} + (\|\kappa\|_\infty+\|\kappa\|_\infty^2+\|\dot\kappa\|_\infty) \|u\|_p\right),
    \end{align*}
    where in the last inequality we used \eqref{elliptic-regularity-phi-W3p}, \eqref{elliptic-regularity-phi-W2p} and \eqref{elliptic-regularity-gradphi}.
    Finally using the $W^{1,r}$-bound for $u$, \eqref{elliptic-regularity-u-in-Wonep-bound}, and Young's inequality yields the claim.
\end{itemize}
\end{proof}

The next result concerns a crucial $L_t^{\infty}L^p_x-$bound for the vorticity.
\begin{lemma}\label{LemmaVorticityBound}
Let $p\in (2,\infty)$ and assume that the conditions of Lemma \ref{lemma-H1-bounded-by-grad-and-bdry-terms} are satisfied. Then there exists a constant $C$ depending only on $p$, $|\Omega|$ and $\|h'\|_{\infty}$ such that
\begin{align*}
    \|\omega\|_{p} &\leq  C\left[ \|\omega_0\|_p  +\left(1+\|\alpha+\kappa\|_\infty^{\frac{2(p-1)}{p-2}}\right)\|u_0\|_2+ C_{\alpha,\kappa}\Ra\right],
\end{align*}
where $C_{\alpha,\kappa}= \left(1+\|\alpha+\kappa\|_\infty^{\frac{2(p-1)}{p-2}}\right)\max\left\lbrace 1, \underline{\alpha}^{-1}\right\rbrace$.
\end{lemma}

\begin{proof}
Fix an arbitrary time $\bar t>0$ and decompose the solution $\omega$ to \eqref{vorticity-equation} as
$$\omega=\bar\omega_\pm +\tilde \omega_\pm $$
where $\tilde \omega_\pm$ solves
\begin{equation*}
    \begin{aligned}
        \frac{1}{\Pra} (\partial_t \tilde\omega_\pm + u\cdot \nabla \tilde \omega_\pm)-\Delta \tilde \omega_\pm &= \Ra \partial_1 T  &\textnormal{ in } &\Omega\\
        \tilde \omega_\pm &= \pm \Lambda &\textnormal{ on } &\gamma^+\cup\gamma^-\\
        \tilde \omega_{\pm,0} &= \pm |\omega_0| &\textnormal{ in } &\Omega\,,
    \end{aligned}
\end{equation*}
with $\Lambda= 2 \|(\alpha+\kappa)u_\tau\|_{L^\infty([0,\bar t]\times\lbrace\gamma^+\cup\gamma^-\rbrace)}$ and the difference $\bar\omega_\pm = \omega -\tilde \omega_\pm$ solves
\begin{equation*}
    \begin{aligned}
        \frac{1}{\Pra} (\partial_t \bar\omega_\pm + u\cdot \nabla \bar \omega_\pm)-\Delta \bar \omega_\pm &= 0  &\textnormal{ in } &\Omega\\
        \bar \omega_\pm &= -2(\alpha+\kappa) u_\tau \mp \Lambda  &\textnormal{ on } &\gamma^+\cup\gamma^-\\
        \bar \omega_{\pm,0} &= \omega_0 \mp |\omega_0| &\textnormal{ in } &\Omega\,.
    \end{aligned}
\end{equation*}
Since the boundary and the initial values have a sign, i.e. $\bar{\omega}_+\leq 0$ on $\gplus\cup\gminus$, $\bar{\omega}_{+,0}\leq 0$ in $\Omega$ and $\bar{\omega}_-\geq 0$ on $\gplus\cup\gminus$, $\bar{\omega}_{-,0}\geq 0$ in $\Omega$, then, by the maximum principle, $\omega -\tilde \omega_+ =\bar \omega_+ \leq 0 $ and $0\leq \bar \omega_- = \omega - \tilde \omega_-$ yielding $\tilde \omega_- \leq \omega \leq \tilde \omega_+$.
In particular
\begin{align}
    |\omega| \leq \max \lbrace |\tilde\omega_-|,|\tilde \omega_+| \rbrace\,.
\end{align}
Hence, it remains to find upper bounds for $|\tilde\omega_-|$ and $|\tilde \omega_+|$. By symmetry, it suffices to show an upper bound for $\tilde \omega_+$. We divide the proof in three steps:

\medskip
\textbf{Step 1}: Omitting the indices, we define
\begin{align*}
    \hat \omega = \tilde \omega - \Lambda,
\end{align*}
then $\hat{\omega}$ satisfies
\begin{align*}
    \frac{1}{\Pra} (\partial_t \hat\omega + u\cdot \nabla \hat \omega)-\Delta \hat \omega &= \Ra \partial_1 T  &\textnormal{ in } &\Omega\\
    \hat \omega &= \Lambda-\Lambda = 0 &\textnormal{ on } &\gamma^+\cup\gamma^-\\
    \hat \omega_{0} &= |\omega_0|-\Lambda &\textnormal{ in } &\Omega\,.
\end{align*}
Testing the equation with $\hat{\omega}|\hat{\omega}|^{p-2}$ we obtain
\begin{align*}
    \frac{1}{p\Pra}\frac{d}{dt} \|\hat\omega\|_p^p 
    %&= -\frac{1}{\Pra}\int |\hat \omega|^{p-2} \hat \omega u \cdot\nabla \hat \omega + \int |\hat \omega|^{p-2} \hat \omega \Delta \hat \omega - \Ra \int |\hat\omega|^{p-2}\hat \omega \partial_1 T\\
    &= -(p-1)\int_\Omega |\nabla \hat\omega|^2 |\hat\omega|^{p-2} -\Ra \int_\Omega T  \partial_1(|\hat{\omega}|^{p-2}\hat\omega).
\end{align*}
Using $\|T\|_{\infty}=1$, Young's and Hölder's inequality, we estimate the second term of the right-hand side as
\begin{align*}
    \left|\Ra\int_\Omega T  \partial_1(|\hat{\omega}|^{p-2}\hat\omega)\right|
    \leq \frac{p-1}{2}\left(\Ra^2 |\Omega|^{\frac{2}{p}} \|\hat\omega\|_p^{p-2} + \int_\Omega |\nabla\hat\omega|^2 |\hat\omega|^{p-2}\right).  
\end{align*}
Then 
\begin{align*}
    \frac{1}{p\Pra}\frac{d}{dt} \|\hat\omega\|_p^p 
    &\leq \frac{p-1}{2}\left( \Ra^2 |\Omega|^\frac{2}{p} \|\hat\omega\|_p^{p-2} -\int_\Omega |\nabla \hat\omega|^2 |\hat\omega|^{p-2} \right)
    \\
    &=\frac{p-1}{2}\left( \Ra^2 |\Omega|^\frac{2}{p} \|\hat\omega\|_p^{p-2} -\frac{4}{p^2}\big\|\nabla |\hat\omega|^\frac{p}{2}\big\|_2^2 \right).
\end{align*}
By the Poincar\'e estimate applied to the second term of the right-hand side (remember that $\hat{\omega}$ vanishes at the boundary by definition), we obtain
  \begin{align*}
    \frac{1}{p\Pra}\frac{d}{dt} \|\hat\omega\|_p^p \leq \frac{p-1}{2} \Ra^2 |\Omega|^\frac{2}{p} \|\hat\omega\|_p^{p-2} - 2\frac{p-1}{p^2C_p^2} \|\hat\omega\|_p^p\,,
\end{align*}
where $C_p$ denotes the Poincar\'e constant.
Dividing through by $\|\hat{\omega}\|_p^{p-2}$ we obtain the inequality
\begin{align*}
    \frac{d}{dt}\|\hat\omega\|_p^2 \leq \frac{p-1}{p} \Pra \Ra^2 |\Omega|^\frac{2}{p} - 4 \Pra \frac{p-1}{p^3 C_p^2} \|\hat\omega\|_p^2.
\end{align*}
By the Gr\"onwall inequality
\begin{align*}
    \|\hat\omega\|_p^2 
    &\leq e^{- 4 \Pra \frac{p-1}{p^3 C_p^2} t} \|\hat\omega_0\|_p^2 +\frac{1}{4} C_p^2 p^2 \Ra^2 |\Omega|^\frac{2}{p}
    %\\
    %&
    \leq e^{- 4 \Pra \frac{p-1}{p^3 C_p^2} t}  \|\omega_0-\Lambda\|_p^2 +\frac{1}{4} C_p^2 p^2 \Ra^2 |\Omega|^\frac{2}{p}
    \\
    &
    \leq e^{- 4 \Pra \frac{p-1}{p^3 C_p^2} t}  (\|\omega_0\|_p^2+\Lambda^2|\Omega|^{\frac 2p})+\frac{1}{4} C_p^2 p^2 \Ra^2 |\Omega|^\frac{2}{p}
    %\\
    %&
    \leq C^2(\|\omega_0\|_p^2+\Lambda^2+ \Ra^2).
    %\\&\leq e^{-c(p,\Omega) t} \|\omega_0-\Lambda\|_p^2 + C_1^2(p,\Omega) \Ra^2
\end{align*}


\medskip
\textbf{Step 2:}
We now turn to the estimate for $\Lambda$. We have 
\begin{equation*}
    \Lambda = 2\|(\alpha+\kappa)u_\tau\|_{L^\infty(\gamma^-\cup\gamma^+)}
    \leq 2 \|\alpha+\kappa\|_\infty\|u\|_{L^\infty(\Omega\times[0,\bar t])},
\end{equation*}
which can be bounded using interpolation and Young's inequality by
%\textcolor{red}{For $p=2$ it does not work as the only interpolation that is possible is the $\|\nabla u\|_2^\theta\|u\|_\infty^{1-\theta}$ one and there one can only compensate either $\|\omega\|_2$ or $\|u\|_\infty$ and not both at the same time.}
\begin{align*}
    2&\|\alpha+\kappa\|_\infty\|u\|_{L^\infty(\Omega\times[0,\bar t])} 
    \\
    &\qquad\leq C\|\alpha+\kappa\|_\infty\|\nabla u\|_{L^\infty_t (L^p_x)}^\theta \|u\|_{L^\infty_t (L^2_x)}^{1-\theta}+C\|\alpha+\kappa\|_{\infty}\|u\|_{L^\infty_t (L^2_x)}
    \\
    &\qquad\leq \epsilon C \theta \|\nabla u\|_{L^\infty_t (L^p_x)}+ C\left[1+(1-\theta)\epsilon^{\frac{p}{2-p}} \|\alpha+\kappa\|_{\infty}^{\frac{p}{p-2}}  \right] \|\alpha+\kappa\|_{\infty}\|u\|_{L^\infty_t (L^2_x)}
\end{align*}
for $p>2$ and arbitrary $\epsilon>0$, where $\theta= \frac{p}{2(p-1)}$ and $L_t^\infty(L_x^p)=L^\infty([0,\bar t];L^p(\Omega))$. According to Lemma \ref{lemma_u_bounded_by_omega}
\begin{align*}
    \| u \|_{W^{1,p}} &\leq C \left(\|\omega\|_p + \left(1+\|\kappa\|_\infty^{2-\frac{2}{p}}\right)\|u\|_2\right)
\end{align*}
resulting in
\begin{align*}
    \Lambda &\leq \epsilon C \|\omega\|_{L_t^\infty(L_x^p)} + C\left[\epsilon\left(1+\|\kappa\|_\infty^{2-\frac{2}{p}}\right)+\|\alpha+\kappa\|_{\infty}+\epsilon^{\frac{p}{2-p}} \|\alpha+\kappa\|_{\infty}^{\frac{2(p-1)}{p-2}}  \right] \|u\|_{L^\infty_t (L^2_x)}.
\end{align*}



\medskip
\textbf{Step 3:}
Recalling that $|\omega|\leq \max \lbrace |\tilde \omega_-|,|\tilde \omega_+|\rbrace$ and $\hat\omega_+ =\tilde \omega_+ - \Lambda$ and using the results of Step 1 and Step 2 one gets
\begin{align*}
        \|\omega\|_{L_t^\infty (L_x^p)} &\leq\|\tilde \omega\|_{L_t^\infty(L_x^p)} =\|\hat \omega+\Lambda\|_{L_t^\infty(L_x^p)}
        \leq  \|\hat\omega\|_{L_t^\infty(L_x^p)} + |\Omega|^\frac{1}{p} \Lambda
        \leq C(  \|\omega_0\|_p + \Ra +\Lambda)
        %\\
        %&\leq  C\left(\|\omega_0\|_p + \Ra + 2\|\alpha+\kappa\|_{L^\infty(\gamma^-\cup\gamma^+)} \|u_{\tau}\|_{L^\infty([0,T]\times\lbrace\gamma^- \cup \gamma^+\rbrace)}\right)
        \\
        &\leq \epsilon C \|\omega\|_{L_t^\infty ( L_x^p)} + C\|\omega_0\|_p + C \Ra 
        \\
        &\qquad+ C\left[\epsilon\left(1+\|\kappa\|_\infty^{2-\frac{2}{p}}\right)+\|\alpha+\kappa\|_\infty+\epsilon^{\frac{p}{2-p}} \|\alpha+\kappa\|_\infty^{\frac{2(p-1)}{p-2}}\right] \|u\|_{L^\infty_t (L^2_x)}
\end{align*}
    
Choosing $\epsilon$ small we can compensate the vorticity term on the right-hand side. By symmetry of the two boundaries $\max_{\gamma^-\cup\gamma^+} \kappa=-\min_{\gamma^-\cup\gamma^+} \kappa$, which, as $\alpha>0$, implies $\|\kappa\|_\infty\leq \|\alpha+\kappa\|_\infty$. Combining these observations we find
\begin{align*}
    \|\omega\|_{L_t^\infty(L_x^p)} \leq C \|\omega_0\|_p + C\left(1+\|\alpha+\kappa\|_\infty^{\frac{2(p-1)}{p-2}}\right)\|u\|_{L_t^\infty(L_x^2)} + C\Ra.
\end{align*}
Finally Lemma \ref{lemmaEnergyDecay} yields
\begin{align*}
    \|\omega\|_{L_t^\infty(L_x^p)} 
    %&\leq C \|\omega_0\|_p + C\left(1+\|\alpha+\kappa\|_{L^\infty(\gamma^-\cup\gamma^+)}^{\frac{2(p-1)}{p-2}}\right)\left(\|u_0\|_{2}+ C_{\alpha,\kappa}\Ra\right) + C\Ra
    %\\
    &\leq C \left[ \|\omega_0\|_p + \left(1+\|\alpha+\kappa\|_\infty^{\frac{2(p-1)}{p-2}}\right)\|u_0\|_{2}+ C_{\alpha,\kappa}\Ra\right],
\end{align*}
where $C_{\alpha,\kappa}=  \left(1+\|\alpha+\kappa\|_\infty^{\frac{2(p-1)}{p-2}}\right)\max\left\lbrace 1, \underline{\alpha}^{-1}\right\rbrace $ and $C$ only depends on $|\Omega|$, $\|h'\|_{\infty}$ and $p$. As the constants are independent of $\bar t$ this bound holds universally in time.

%\textcolor{orange}{
%Actually no. The constant depends indirectly also $\kappa$. Namely we use Gagliardo Nirenberg, Poincare and also elliptic regularity for \eqref{gradu-in-Lp} which all (I think) for general domains only rely on the Stein Extension property constant. For our case this depends on the Lipschitz constant of the boundary, i.e. $\max \gamma'$, see for example Stein Singular Integrals and Differentiability Properties of Functions, 1970, Chapter VI, §3, Theorem 5c (p. 181). And also on $|\Omega|$. So if we should specify then we could either put an upper bound for $\gamma'$ and then all the lipschitz constants for different domains can be bounded by that lipschitz constant or explicitly carry through the lipschitz constant which might be a bit tricky. With regards to the calulations i have to check.
%}
%$$ \|\omega\|_{L_t^\infty(L_x^p)}
 %   &\leq&  C(\|\omega_0\|_p + \Ra)\\
 %   &+& C\theta\left(\frac{2C}{\varepsilon^{1-\theta}}\right)^{\frac{1}{\theta}}\|\alpha+\kappa\|_{L^{\infty}(\gamma^-\cup\gamma^+)}^{\frac{1}{\theta}}(\|u_0\|_{2}+C_D\Ra),,$$
 
\end{proof}

As $\Omega$ is bounded Hölder inequality and Lemma \ref{LemmaVorticityBound} yield that if $\omega_0 \in L^p$ for any $p<\infty$ then $\|\omega\|_p$ is universally bounded in time. By trace Theorem and Lemma \ref{lemma_u_bounded_by_omega}
\begin{align*}
    \|u\|_{L^2(\gamma^-\cup\gamma^+)}\leq \|u\|_{H^1} \leq C\left(\|\omega\|_2+(1+\|\kappa\|_\infty)\|u\|_2\right).
\end{align*}
Using Lemma \ref{lemmaEnergyDecay} this is also universally bounded in time, therefore taking the long time average of the vorticity balance, see Lemma \ref{vorticity-balance}, we get the following.

\begin{corollary}
Assume that the conditions of Lemma \ref{LemmaVorticityBound} are satisfied and $\omega_0\in L^p$ for some $p>2$, then
\begin{equation}
    \label{average-enstrophy-balance}
    \begin{aligned}
        0 &= \langle |\nabla \omega|^2\rangle - 2 \langle (\alpha+\kappa) u\cdot \nabla p \rangle_{\gamma^-\cup\gamma^+} - \Ra \langle \omega\partial_1 T\rangle 
        \\
        &\qquad + \frac{2}{3\Pra} \langle (\alpha+\kappa) u\cdot (u\cdot\nabla) u\rangle_{\gamma^-\cup\gamma^+} + 2\Ra \langle (\alpha+\kappa) u_\tau n_1\rangle_{\gamma^-}.
    \end{aligned}
\end{equation}
\end{corollary}




\subsection{A-priori estimate for the pressure}
\leavevmode

The pressure satisfies
\begin{equation}
    \label{pressure-eq}
    \begin{aligned}
        \Delta p &= -\frac{1}{\Pra}(\nabla u)^T \colon \nabla u +\Ra \partial_2 T   & \textnormal{ in }&\Omega\\
        n\cdot \nabla p &= -\frac{1}{\Pra}\kappa u^2 +2 \tau\cdot\nabla\left((\alpha+\kappa)u_\tau\right)  & \textnormal{ on }&\gamma^+\\
        n\cdot \nabla p &= -\frac{1}{\Pra}\kappa u^2 +2 \tau\cdot\nabla\left((\alpha+\kappa)u_\tau \right) +n_2\Ra  & \textnormal{ on }&\gamma^-
    \end{aligned}
\end{equation}

The equation in the bulk is easy to obtain by applying the divergence to the Navier-Stokes equations, using incompressibility and writing compactly $\nabla\cdot((u\cdot\nabla)u)=(\nabla u)^T:\nabla u$.
In order to track the pressure at the boundary we look at Navier-Stokes equations at $\gminus\cup\gplus$ 
\begin{align*}
    \left(\frac{1}{\Pra}\left( u_t + (u\cdot \nabla)u \right) - \Delta u +  \nabla p - \Ra Te_2\right)\cdot n =0
\end{align*}
where $n$ is the normal at the boundary.
It is clear that
\begin{align*}
    n\cdot u_t &= \frac{d}{dt}(n\cdot u) = 0
   % n\cdot (u\cdot \nabla ) u \overset{\eqref{shortBoundaryIdentities}}&{=} \kappa u_\tau^2
\end{align*}
and
\begin{align*}
    n\cdot \Delta u= n \cdot\nabla^\perp \omega =  -2n\cdot \nabla^\perp \left((\alpha+\kappa)u_\tau\right)
    = 2 \tau \cdot \nabla \left((\alpha+\kappa)u_\tau\right),
    %= 2\frac{d}{d\lambda}\left((\alpha+\kappa)u_\tau \right),
\end{align*}
using the boundary condition for the vorticity in \eqref{vorticity-equation}.
Thanks to \eqref{appendix-proof-id-kappaUtau2} in the Appendix we also have 
\begin{align}
    \label{id-kappaUtau2-3}
    n\cdot (u\cdot \nabla ) u =\kappa u_\tau^2.
\end{align}
Hence
\begin{align*}
    n\cdot \nabla p = -\frac{\kappa}{\Pra} u_\tau^2 +2\tau \cdot \nabla ((\alpha+\kappa)u_\tau) + n_2T\Ra
\end{align*}
at the boundary. Now, it is only left to observe that $T=0$ at $\gplus$ and $T=1$ at $\gminus$.

\begin{proposition}
\label{proposition-pressure-bound}
For any $r\in (2,\infty)$ there exists a constant $C$ depending on $|\Omega|$, $r$ and $\|h'\|_{\infty}$ such that
\begin{align*}
    \| p\|_{H^1}
    \leq C\left[ \Ra\|T\|_2 + \|\alpha+\kappa\|_\infty\|u\|_{H^2}+\left(\frac{1+\|\kappa\|_\infty}{\Pra}\|u\|_{W^{1,r}}+\|\dot\alpha+\dot\kappa\|_\infty \right)\|u\|_{H^1}\right].
\end{align*}

\end{proposition}

\begin{proof}
On one hand, integrating by parts and using the boundary conditions (for $u$ and $T$), we have 
\begin{align*}
    \int_{\Omega} p\Delta p 
    &= - \| \nabla p\|_2^2 - \frac{1}{\Pra} \int_{\gamma^-\cup\gamma^+} p\kappa u_\tau^2 +2\int_{\gamma^-\cup\gamma^+}p \tau\cdot \nabla\left((\alpha+\kappa)u_\tau\right)+\Ra\int_{\gamma^-} p n_2
\end{align*}
On the other hand using the equation satisfied by the pressure \eqref{pressure-eq}
\begin{align*}
    \int_{\Omega} p\Delta p 
     &= -\frac{1}{\Pra}\int_{\Omega} p(\nabla u)^T \colon \nabla u + \Ra \int_{\Omega} p\partial_2 T \\
     %&=& -\frac{1}{\Pra}\int_{\Omega} p(\nabla u)^T \colon \nabla u + \Ra \int_{\Omega} \partial_2 (pT)- \Ra \int_{\Omega} T\partial_2 p\\
     &= -\frac{1}{\Pra}\int_{\Omega} p(\nabla u)^T \colon \nabla u  + \Ra \int_{\gminus\cup\gplus} pT n_2- \Ra \int_{\Omega} T\partial_2 p\\
     &= -\frac{1}{\Pra}\int_{\Omega} p(\nabla u)^T \colon \nabla u + \Ra\int_{\gminus} pn_2- \Ra \int_{\Omega} T\partial_2 p,
\end{align*}
where we used the boundary conditions for $T$ in the last identity.
Combining these estimates one gets
\begin{align*}
     \| \nabla p\|^2 
     &= - \frac{1}{\Pra} \int_{\gamma^-\cup\gamma^+} p\kappa u_\tau^2 +2\int_{\gamma^-\cup\gamma^+}p \tau\cdot \nabla\left((\alpha+\kappa)u_\tau\right) +\frac{1}{\Pra}\int_{\Omega} p(\nabla u)^T \colon \nabla u
     \\&\qquad + \Ra \int_{\Omega} T\partial_2 p.
\end{align*}
We estimate the right-hand side:
By H\"older inequality with $\frac{1}{r}+\frac{1}{q}+\frac{1}{2}=1$ and Sobolev embedding\footnote{ Since $\Omega$ is bounded, for any $1\leq \mu<\infty$ choose $s>\max\lbrace 2,\mu \rbrace$ and $q=\frac{s}{\mu}>1$, such that $\frac{ns}{n+s}=\frac{2s}{2+s}\leq \frac{2s}{s}=2$. By H\"older and Sobolev inequality we obtain
\begin{align*}
    \|f\|_\mu \leq \|1\|_{\frac{\mu q}{q-1}}\|f\|_{\mu q} = |\Omega|^{\frac{q-1}{\mu q}}\|f\|_s =|\Omega|^{\frac{\mu s-1}{s}}\|f\|_s \leq C \|f\|_{W^{1,\frac{ns}{n+s}}} \leq C \|f\|_{H^1}.
\end{align*}
}
\begin{align}
    \label{pressure-3-term-estimate}
    \|p(\nabla u)^T\colon \nabla u\|_1 &\leq \|p\|_q \|\nabla u\|_2 \|\nabla u\|_r 
    \leq C \|p\|_{H^1} \|\nabla u\|_2 \|\nabla u\|_r,
\end{align}
where $C$ depends on $|\Omega|$, $r$ and $\|h'\|_{\infty}$. For the temperature term we apply H\"older's inequality use that $\|T\|_\infty\leq 1$ by the maximum principle \eqref{maximum-principle}
\begin{align*}
    \|T\partial_2 p\|_1\leq \|T\|_2 \|p\|_{H^1}\leq |\Omega|^\frac{1}{2} \|p\|_{H^1}\,,
\end{align*}
and for the second term we compute
\begin{align*}
    &\int_{\gamma^-\cup\gamma^+} \left|p \tau\cdot \nabla ((\alpha+\kappa)u_{\tau})\right|
    \\
    &\quad\leq \|\alpha+\kappa\|_\infty\|p\|_{L^2(\gamma^-\cup\gamma^+)} \|\nabla u\|_{L^2(\gamma^-\cup\gamma^+)} + \|\dot\alpha+\dot\kappa\|_\infty \|p\|_{L^2(\gamma^-\cup\gamma^+)}\|u\|_{L^2(\gamma^-\cup\gamma^+)}
    \\
    &\quad\leq C \|\alpha+\kappa\|_\infty \|p\|_{H^1} \|u\|_{H^2} + C \|\dot\alpha+\dot\kappa\|_\infty \|p\|_{H^1} \|u\|_{H^1},
\end{align*}
where in the last inequality we use the trace estimate.
Finally, we estimate the first term: Similar to \eqref{pressure-3-term-estimate} for $\frac{1}{r}+\frac{1}{q}+\frac{1}{2}=1$ and every $r>2$
\begin{align*}
    \left|\int_{\gamma^-\cup \gamma^+} \kappa p u_\tau^2\right| &\leq \|\kappa\|_\infty \|pu_\tau^2\|_{L^1(\gamma^-\cup\gamma^+)}\\
    &\leq C\|\kappa\|_\infty  \|pu^2\|_{W^{1,1}} \\
    %&= C\|\kappa\|_\infty \left(\|pu^2\|_1 + \|\nabla(pu^2_\tau)\|_1 \right)\\
    %&= C\|\kappa\|_\infty \left( \|pu^2\|_1 + \|u^2 \nabla p\|_1 + 2\| p u \nabla u \|_1\right)\\ 
    &\leq C\|\kappa\|_\infty \left( \|p\|_q \|u\|_2\|u\|_r + \|\nabla p\|_2 \|u\|_r \|u\|_q + \| p \|_q \|u\|_r \|\nabla u\|_2\right)\\
    &\leq C\|\kappa\|_\infty \left( \|p\|_{H^1} \|u\|_{H^1}\| u\|_{W^{1,r}} + \|\nabla p\|_2 \|u\|_r \|u\|_{H^1} + \| p \|_{H^1} \|u\|_r \|\nabla u\|_2\right)\\
    &\leq C \|\kappa\|_\infty \|p\|_{H^1} \|u\|_{H^1}\| u\|_{W^{1,r}}\,,
\end{align*}
where $C$ depends on $|\Omega|$, $r$ and $\|h'\|_{\infty}$.

\smallskip 

Combining the estimates we find
\begin{align*}
     \| \nabla p\|_2^2 
     %&= - \frac{1}{\Pra} \int_{\gamma^-\cup\gamma^+} p\kappa u_\tau^2 +2\int_{\gamma^-\cup\gamma^+}p \frac{d}{d\lambda}\left( (\alpha+\kappa)u_\tau\right) +\frac{1}{\Pra}\int_{\Omega} p(\nabla u)^T \colon \nabla u + \Ra \int_{\Omega} T\partial_2 p
     &\leq C \|p\|_{H^1}\left[ \Ra + \|\alpha+\kappa\|_\infty\|u\|_{H^2}+\left(\frac{1+\|\kappa\|_\infty}{\Pra}\|u\|_{W^{1,r}}+\|\dot\alpha+\dot\kappa\|_\infty \right)\|u\|_{H^1}\right].
\end{align*}

Using that the pressure $p$ is only defined up to a constant so we choose $p$ to have zero mean such that Poincaré yields $\|p\|_q\leq C \|\nabla p\|_q$ which implies $\|p\|_{H^1}^2 = \|\nabla p\|_2^2+\|p\|_2^2 \leq (1+C^2) \|\nabla p\|_2^2$. Then 
\begin{align*}
    \| p\|_{H^1}^2
    &\leq C \| p\|_{H^1}\left[ \Ra + \|\alpha+\kappa\|_\infty\|u\|_{H^2}+\left(\frac{1+\|\kappa\|_\infty}{\Pra}\|u\|_{W^{1,r}}+\|\dot\alpha+\dot\kappa\|_\infty \right)\|u\|_{H^1}\right]\,.
\end{align*}
Finally dividing by $\| p\|_{H^1}$ we conclude that there exists a constant $C>0$ depending on $\Omega$ and $r$ such that
\begin{eqnarray*}
    \| p\|_{H^1}
    \leq C\left[ \Ra + \|\alpha+\kappa\|_\infty\|u\|_{H^2}+\left(\frac{1+\|\kappa\|_\infty}{\Pra}\|u\|_{W^{1,r}}+\|\dot\alpha+\dot\kappa\|_\infty \right)\|u\|_{H^1}\right]\,.
\end{eqnarray*}
for any $r>2$.
\end{proof}








\section{Upper bounds on the Nusselt number}\label{section-four}
Combining the a-priori estimates derived in the previous section we are now able to prove the $\Ra^\frac{1}{2}$ bound, that was first derived for the flat, no slip case in 3 dimensions by Doering and Constantin \cite{DC96}.
\subsection{Proof of Theorem \ref{Lemma-Ra-One-Half-Bound}}
%\begin{lemma}
%\label{Lemma-Ra-One-Half-Bound}
%Assume $h \in W^{2,\infty}$ and that the conditions of Lemma \ref{lemma-H1-bounded-by-grad-and-bdry-terms} are satisfied. Then there exists a constant $C>0$ depending only on the Lipschitz constant of the boundary and $|\Omega|$ such that
%\begin{align*}
%    \Nu \leq C \Ra^\frac{1}{2}+\|\kappa\|_\infty
%\end{align*}
%for all $\Ra\geq 1$.
%\end{lemma}



%\begin{proof}
Let $\Omega_\delta$ be given by $$\Omega_\delta = \lbrace (y_1,y_2) \ \vert \ 0\leq y_1\leq \Gamma, 1+h(y_1)-\delta\leq y_2\leq 1+h(y_1)\rbrace$$ as illustrated in Figure \ref{fig:omega_delta}.
\begin{figure}
    \begin{center}
        \includegraphics[width=0.5\textwidth]{pictures/omega_delta.pdf}
    \end{center}
    \caption{Illustration of $\Omega_\delta$.}
    \label{fig:omega_delta}
\end{figure}
%Then according to \eqref{nusselt-strip} the Nusselt number is given by
%\begin{align*}
%    \Nu &= \langle  (uT - \nabla T)\cdot n_+ \rangle_{\gamma(x_2)} \qquad \textnormal{ for any }0\leq x_2 \leq 1.
%\end{align*}
Taking the average in $z\in (1-\delta,1)$ in the representation of the Nusselt number \eqref{nusselt-strip} we find
\begin{equation}
    \label{halfBound-nusselt-on-Omega_delta}
    \begin{aligned}
        \Nu &= \limsup_{T\to\infty} \frac{1}{T}\int_0^T \frac{1}{\delta} \frac{1}{|\Omega|} \int_{1-\delta}^{1}\int_{\gamma(z)} n_+\cdot (u-\nabla)T \ dS\ dz\ dt 
        \\
        &= \limsup_{T\to\infty} \frac{1}{T}\int_0^T \frac{1}{\delta} \frac{1}{|\Omega|} \int_{\Omega_\delta} n_+\cdot u T \ dy\ dt - \limsup_{T\to\infty} \frac{1}{T}\int_0^T \frac{1}{\delta} \frac{1}{|\Omega|} \int_{\Omega_\delta} n_+\cdot \nabla T \ dy\ dt
    \end{aligned}
\end{equation}
In order to estimate the first term on the right-hand side notice that by the fundamental theorem of calculus for $(y_1,y_2)\in \Omega_\delta$
\begin{equation}
    \label{halfBound-velocity-term}
    \begin{aligned}
        |n_+\cdot u| (y_1,y_2)&= \left|n_+\cdot u\vert_{\gamma^+}+\int_{1+h(y_1)}^{y_2} \partial_2 (n_+\cdot u)\ dz\right| \leq  \int_{1+h(y_1)-\delta}^{1+h(y_1)} |\partial_2 u|\ dz 
        \\
        &\leq \delta^\frac{1}{2}\|\nabla u\|_{L^2(\gamma^-,\gamma^+)},
    \end{aligned}
\end{equation}
where $\| \nabla u\|_{L^2(\gamma^-,\gamma^+)}=\|\nabla u(y_1,\cdot)\|_{L^2(h(y_1),1+h(y_1))}$ and we used the non-penetration boundary condition for $u$ and that $n_+$ is constant in $y_2$-direction in the first inequality and Hölder's inequality in the second estimate. Analogously for the temperature and $(y_1,y_2)\in \Omega_\delta$ it holds
\begin{align}
    \label{halfBound-temperature-term}
    |T|(y_1,y_2) \leq \delta^\frac{1}{2} \| \nabla T\|_{L^2(\gamma^-,\gamma^+)}
\end{align}
as $T=0$ on $\gamma^+$. In order to estimate the second integral in \eqref{halfBound-nusselt-on-Omega_delta} partial integration and the boundary condition $T=0$ on $\gamma^+$ yields
\begin{align}
    \label{halfBound-nusselt-second-integral-estimate-1}
    \left|\int_{\Omega_\delta} n_+\cdot \nabla T \ dy \right| \leq \int_{\gamma^+} |T| \ dS+ \int_{\gamma(1-\delta)} |n_+\cdot n_- T| \ dS + \int_{\Omega_\delta} |T\nabla \cdot n_+ | \ dy.
\end{align}
By the maximum principle \eqref{maximum-principle} the temperature is bounded by $\|T\|_\infty \leq 1$, so the first two terms on the right-hand side of \eqref{halfBound-nusselt-second-integral-estimate-1} are bounded by a constant depending on $\|h'\|_{\infty}$ and $|\Omega|$. In order to estimate the last term notice that
\begin{align*}
    n_+=\frac{1}{\sqrt{1+(h')^2}}\begin{pmatrix}-h'\\1\end{pmatrix}\qquad \textnormal{ and }\qquad |\kappa| = \frac{|h''|}{(1+(h')^2)^\frac{3}{2}},
\end{align*}
where $h'=\partial_1 h(y_1)$ and $h''= \partial_1^2h(y_1)$ as derived in \eqref{appendix-normal-vector-representation} and \eqref{appendix-kappa-representation} in the Appendix. Therefore $|\nabla\cdot n_+ |=|\kappa|$, which implies
\begin{align*}
    \int_{\Omega_\delta} |T\nabla \cdot n_+ | \ dy \leq \|\kappa\|{_\infty}|\Omega_\delta|
\end{align*}
for the last term in \eqref{halfBound-nusselt-second-integral-estimate-1}.
% actually with kappa as on the top boundary $\nabla\cdot n_+ = -\kappa$
Combining these observations
\begin{align}
    \label{halfBound-nusselt-second-integral-estimate-2}
    \left|\int_{\Omega_\delta} n_+\cdot \nabla T \ dy \right| \leq C (1+\delta \|\kappa\|_\infty).
\end{align}
Plugging \eqref{halfBound-velocity-term}, \eqref{halfBound-temperature-term} and \eqref{halfBound-nusselt-second-integral-estimate-2} into \eqref{halfBound-nusselt-on-Omega_delta} and using Hölder inequality, there exists a constant depending on $\|h'\|_{\infty}$ and $|\Omega|$ such that
\begin{align*}
    \Nu &\leq
    \limsup_{T\to\infty} \frac{1}{T}\int_0^T \frac{1}{|\Omega|} \int_{\Omega_\delta} \|\nabla u\|_{L^2(\gamma^-,\gamma^+)}\|\nabla T\|_{L^2(\gamma^-,\gamma^+)} \ dy\ dt + C\left(\frac{1}{\delta}+\|\kappa\|_\infty\right)
    \\
    &\leq C \left(\delta \langle |\nabla u|^2 \rangle^\frac{1}{2} \langle |\nabla T|^2 \rangle^\frac{1}{2}+\frac{1}{\delta}+\|\kappa\|_\infty\right).
\end{align*}
By \eqref{nusselt-def} and \eqref{average-energy-balance} we can substitute both gradients and get
\begin{align*}
    \Nu &\leq C \left(\delta\Ra^\frac{1}{2}\left((1+\max h-\min h) \Nu-1\right)^\frac{1}{2} \Nu^\frac{1}{2}+\frac{1}{\delta}+\|\kappa\|_\infty\right) 
    \\
    &\leq  C\left(\delta\Ra^\frac{1}{2} \Nu+\frac{1}{\delta}+\|\kappa\|_\infty\right).
\end{align*}
Balancing the terms by choosing $\delta=\Nu^{-\frac{1}{2}}\Ra^{-\frac{1}{4}}$ we get
%\begin{align*}
%    \Nu \leq  C\left(\Ra^\frac{1}{4}\Nu^\frac{1}{2}+\|\kappa\|_\infty\right) \leq \frac{1}{2}\Nu + C\left(\Ra^\frac{1}{2}+\|\kappa\|_\infty\right),
%\end{align*}
%where we used Young's inequality in the last estimate, which implies
\begin{align*}
    \Nu \leq  C\left(\Ra^\frac{1}{2}+\|\kappa\|_\infty\right)
\end{align*}
for $\Ra \geq 1$.
%\end{proof}


\subsection{Introduction of the background field method}
\leavevmode


In order to improve the bound of Theorem \ref{Lemma-Ra-One-Half-Bound} we follow the "background field" strategy used \cite{drivasNguyenNobiliBoundsOnHeatFluxForRayleighBenardConvectionBetweenNavierSlipFixedTemperatureBoundaries}, which is based on \cite{whiteheadDoeringUltimateState}. This approach consists of specifying a stationary background field for the temperature and show its "marginal stability" as we will explain in what follows. This will be achieved by applying the a-priori bounds derived in Section \ref{Section-A-Priori-Bounds}.
%applying the a-priori estimates derived in Section \ref{Section-A-Priori-Bounds} and solving a variational problem in order to get a bound on the Nusselt number.

To this end we define the background profile for the temperature by
\begin{align}
    \label{def-eta}
    \eta(y_1,y_2)=1-\frac{1}{2\delta}
    \begin{cases}
        \begin{aligned}
            &2\delta+y_2-(1+h(y_1)) & &\textnormal{for}& 1+h(y_1)-\delta&\leq y_2 \leq 1+h(y_1)\\
            &\delta & &\textnormal{for}& h(y_1)+\delta&< y_2 < 1+h(y_1)-\delta\\
            &y_2-h(y_1) & &\textnormal{for}& h(y_1)&\leq y_2 \leq h(y_1)+\delta
        \end{aligned}
    \end{cases}
\end{align}
for $\delta>0$ and the difference $\theta$ by
\begin{align}
    \label{def-theta}
    \theta = T-\eta.
\end{align}
This profile is illustrated in Figure \ref{fig:background_profile}.

\begin{figure}
    \begin{center}
        \includegraphics[width=0.5\textwidth]{pictures/background_profile.pdf}
    \end{center}
    \caption{Illustrations of the background profile $\eta$.}
    \label{fig:background_profile}
\end{figure}

Note that $\eta$ fulfills the boundary conditions of $T$, so $\theta$ vanishes on $\gamma^\pm$. Also since $n_-$ can be expressed by
\begin{align*}
    n_- = \frac{1}{\sqrt{1+(h')^2}}\begin{pmatrix}h'\\ -1\end{pmatrix}
\end{align*}
as derived in \eqref{appendix-normal-vector-representation}, its gradient is given by
\begin{equation}
    \label{nabla-eta-identity}
    \nabla \eta = 
    \begin{cases}
        \begin{aligned}
            &0 &&\textnormal{for}& h(y_1)+\delta < y_2 < 1+h(y_1)-\delta \\%[5pt]
            &\frac{1}{2\delta} \sqrt{1+(h')^2}n_- &&\textnormal{else}
        \end{aligned}
    \end{cases}
\end{equation}
for almost every $y\in \Omega$. 
%In order to express the Nusselt number in terms of $\eta$ we need the following proposition.
Inserting this decomposition in the definition of the Nusselt number, we have

\begin{proposition}
Let $\eta$ and $\theta$ be defined by \eqref{def-eta} and \eqref{def-theta}. Then
\begin{align}
    \label{nusselt-eta-theta-representation}
    \Nu = \langle |\nabla \eta|^2\rangle -\langle |\nabla\theta|^2\rangle - 2\langle \theta u\cdot \nabla \eta\rangle
\end{align}
\end{proposition}

The proof of this identity is essentially the same as the one for Proposition 7 in \cite{drivasNguyenNobiliBoundsOnHeatFluxForRayleighBenardConvectionBetweenNavierSlipFixedTemperatureBoundaries} and we report it here just for convenience of the reader.
\begin{proof}
Plugging the definitions of $\eta$ and $\theta$ into \eqref{heatEquation} we have
\begin{align*}
    \theta_t + u\cdot\nabla \eta +u\cdot \nabla \theta -\Delta \eta -\Delta \theta = 0,
\end{align*}
and, integrating this equation against $\theta$ we find
\begin{align}
    \label{theta-energy-identity}
    0=\frac{1}{2}\frac{d}{dt}\|\theta\|_2^2 + \int_\Omega \theta u\cdot \nabla \eta + \int_\Omega \theta u\cdot \nabla \theta -\int_\Omega \theta\Delta \eta-\int_\Omega \theta\Delta\theta.
\end{align}
%We study each term individually. 
The third term on the right-hand side of \eqref{theta-energy-identity} vanishes since $u\cdot n=0$ and $u$ is divergence-free.
%as
%\begin{align*}
%    \int_\Omega \theta u\cdot \nabla \theta = \frac{1}{2}\int_\Omega u\cdot\nabla \theta^2 = -\frac{1}{2}\int_\Omega \theta^2\nabla\cdot u + \frac{1}{2} \int_{\gamma^-\cup\gamma^+} \theta^2 u\cdot n = 0,
%\end{align*}
%where in the last equality we used the divergence free and boundary condition for $u$. 
For the fourth term on the right-hand side of \eqref{theta-energy-identity} we get
\begin{align*}
    -\int_\Omega \theta \Delta \eta  &= - \int_{\gamma^-\cup\gamma^+} \theta n\cdot \nabla \eta  + \int_\Omega \nabla\theta \cdot\nabla \eta  = \int_\Omega \nabla\theta \cdot\nabla \eta ,
\end{align*}
where in the last equality we used that $\theta$ vanishes on the boundary by definition. Similarly
\begin{align*}
    -\int_\Omega \theta \Delta \theta = \|\nabla\theta\|_2^2.
\end{align*}
Therefore taking the long time average of \eqref{theta-energy-identity} and using that $\theta$ is universally bounded in time as both $T$ and $\eta$ fulfill $0\leq T,\eta\leq 1$ we find
\begin{align*}
    \langle \theta u\cdot\nabla \eta \rangle + \langle \nabla\theta \cdot \nabla \eta \rangle + \langle |\nabla\theta|^2\rangle = 0.
\end{align*}
Using this identity in the definition of the Nusselt number \eqref{nusselt-def} we find
\begin{align*}
    \Nu = \langle|\nabla T|^2 \rangle = \langle |\nabla \eta|^2\rangle + 2\langle \nabla \theta \cdot\nabla\eta \rangle + \langle |\nabla \theta|^2\rangle = \langle |\nabla \eta|^2\rangle - \langle |\nabla \theta|^2\rangle -2\langle \theta u \cdot\nabla\eta \rangle
\end{align*}
as a representation of $\Nu$.\footnote{The argument can be rigorously justified via mollification of $\eta$.}
\end{proof}
%We add and subtract the balancing term $M\Ra^2$ to \eqref{nusselt-eta-theta-representation}, and
%define the quadratic form
%\begin{equation}\label{qf1}  
%\mathcal{Q}[u,\theta,\eta]=M\Ra^2+\langle|\nabla %\theta|^2\rangle+2\langle\theta u\cdot \nabla \eta\rangle\,,
%\end{equation}
%where $M>0$ will be selected at the end.
%
Next we define
\begin{equation*}
    \label{def-a}
    \begin{aligned}
        \mathbf{a} &:= \langle |\nabla \omega|^2\rangle -2 \langle (\alpha+\kappa)u\cdot\nabla p\rangle_{\gamma^-\cup\gamma^+} - \Ra \langle \omega\partial_1 T\rangle 
        \\
        &\qquad + \frac{2}{3\Pra} \langle (\alpha+\kappa) u\cdot (u\cdot\nabla) u\rangle_{\gamma^-\cup\gamma^+} + 2\Ra \langle (\alpha+\kappa) u_\tau n_1\rangle_{\gamma^-} =0,
    \end{aligned}
\end{equation*}
where the last identity is due to \eqref{average-enstrophy-balance} and 
\begin{align}
    \label{def-b}
    \mathbf{b} &:= \langle|\nabla u|^2\rangle + \langle (2\alpha+\kappa) u_\tau^2\rangle_{\gamma^-\cup \gamma^+} - \Ra\left((1+\max h - \min h) \Nu-1\right)\,.% \leq 0,
\end{align}
%where the inequality is due to \eqref{average-energy-balance} and
Using \eqref{nusselt-eta-theta-representation} we can rewrite the Nusselt number as
\begin{align}
    \label{nu-identity-including-Q}
    (1-b(1+&\max h -\min h))\Nu + b 
    %&= \langle |\nabla\eta|^2\rangle - \langle |\nabla\theta|^2\rangle -2\langle\theta u\cdot\nabla \eta\rangle - b((1+\max\gamma^--\min\gamma^-)\Nu-1)
    %\\
    = M\Ra^2+2\langle |\nabla\eta|^2\rangle-\mathcal{Q}[u,\theta,\eta]
\end{align}
where the quadratic form $\mathcal{Q}$ is defined as
\begin{multline}\label{quadratic-form}
    \mathcal{Q}[u,\theta,\eta]:=M \Ra^2 + \langle |\nabla \eta|^2\rangle + \langle |\nabla \theta|^2\rangle + 2\langle \theta u\cdot\nabla \eta\rangle 
    \\
    \qquad + \frac{b}{\Ra}\langle |\nabla u|^2\rangle + \frac{b}{\Ra}\langle (2\alpha+\kappa)u_\tau^2\rangle_{\gamma^-\cup\gamma^+} - \frac{b}{\Ra}\mathbf{b} + a \mathbf{a}\,.
\end{multline}
In this new representation $a>0$ and $0\leq b<(1+\max h - \min h)^{-1}$ and notice that the balancing term $M\Ra^2$, with $M>0$, was introduced. The choice of the parameters $a,b,M$ will follow from an optimization procedure at the end.

%Then by \eqref{nusselt-eta-theta-representation} and \eqref{def-b}
%and using $\mathbf{a}=0$ and the definition of $\mathcal{Q}$ yields
%\begin{align}
 %   \label{nu-identity-including-Q}
 %   (1-b(1+\max\gamma^--\min\gamma^-))\Nu + b = M \Ra^2 + 2\langle |\nabla \eta|^2\rangle - \mathcal{Q}[u,\theta,\eta].
%\end{align}

We now want to prove that for a suitable choice of $\delta$ the form $\mathcal{Q}$ is non negative. In order to do so we need the following Lemma.
\begin{lemma}
\label{lemma-estimate-theta-u-gradeta}
One has
\begin{align*}
    2|\langle\theta u\cdot \nabla \eta\rangle| &\leq  \delta^6 C (a\epsilon)^{-1}\langle |\partial_2 u|^2\rangle + a\epsilon \langle |\partial_2^2 u|^2\rangle +\frac{1}{2}\langle |\partial_2 \theta|^2\rangle
\end{align*}
for any $\epsilon>0$.
\end{lemma}



\begin{proof}
By \eqref{nabla-eta-identity}
\begin{equation}
    \label{theta-u-grad-eta-estimate-1}
    \begin{aligned}
        2\int_\Omega \theta u\cdot\nabla \eta &= \frac{1}{\delta} \int_0^\Gamma \int_{h(y_1)}^{h(y_1)+\delta} \sqrt{1+(h')^2} \theta u\cdot n_-\ dy_2\ dy_1
        \\
        &\qquad + \frac{1}{\delta} \int_0^\Gamma\int_{1+h(y_1)-\delta}^{1+h(y_1)} \sqrt{1+(h')^2} \theta u\cdot n_-\ dy_2 \ dy_1
    \end{aligned}
\end{equation}
We focus on the first term on the right-hand side. The second one can be treated similarly. By the fundamental theorem of calculus and Hölder's inequality
\begin{equation}
    \label{theta-u-grad-eta-estimate-u}
    \begin{aligned}
        |u\cdot n_-|(y_1,y_2)&=\left|(u\cdot n_-)(y_1,h(y_1))+ \int_{h(y_1)}^{y_2} \partial_2(u\cdot n_-)(y_1,z) \ dz\right| 
        \\
        &\leq \delta \|\partial_2(u\cdot n_-)\|_{L^\infty(\gamma^-,\gamma^+)}
    \end{aligned}
\end{equation}
for $h(y_1)\leq y_2\leq h(y_1)+\delta$, where in the last inequality we used the boundary condition for $u$. Similarly
\begin{align}
    \label{theta-u-grad-eta-estimate-theta}
    |\theta(y_1,y_2)|\leq \delta^\frac{1}{2}\|\partial_2 \theta\|_ {L^2(\gamma^-,\gamma^+)}
\end{align}
as $\theta$ vanishes on the boundary. In order to estimate $\partial_2 (n_-\cdot u)$ notice that by partial integration and the boundary condition for $u$
\begin{align*}
    \int_{h(y_1)}^{1+h(y_1)} \partial_2 (n_- \cdot u) = n_2 \ n_-\cdot u\vert_{\gamma^-} - n_2 \ n_-\cdot u\vert_{\gamma^+} = 0.
\end{align*}
Therefore for every $y_1$ there exists $h(y_1)\leq \bar y_2\leq 1+h(y_1)$ such that $\partial_2(u\cdot n_-)(y_1,\bar y_2)=0$. Applying the fundamental theorem of calculus again we find
\begin{equation}
    \label{theta-u-grad-eta-estimate-d22u}
    \begin{aligned}
        (\partial_2 (u\cdot n_-))^2(y_1,y_2)&= (\partial_2 (u\cdot n_-))^2 (y_1,\bar y_2) + \int_{\bar y_2}^{y_2} \partial_2 \left((\partial_2(u\cdot n_-))^2\right)(y_1,z)\ dz
        \\
        &\leq 2 \|\partial_2 u\|_{L^2(\gamma^-,\gamma^+)}\|\partial_2^2 u\|_{L^2(\gamma^-,\gamma^+)},
    \end{aligned}
\end{equation}
where in the last inequality we used Hölder's inequality, that $n_-$ is constant in $e_2$ direction and $|n_-|=1$. Combing \eqref{theta-u-grad-eta-estimate-1} with \eqref{theta-u-grad-eta-estimate-u}, \eqref{theta-u-grad-eta-estimate-theta} and \eqref{theta-u-grad-eta-estimate-d22u} and using Young's inequality twice yields
\begin{align*}
    \bigg|2\int_\Omega &\theta u\cdot\nabla \eta\ dy\bigg|
    \\
    %&\leq \frac{2}{\delta} \delta \int_0^\Gamma \|\partial_2(u\cdot n_-)\|_{L^\infty(\gamma^-,\gamma^+)}\sqrt{1+(\gamma')^2}\delta^\frac{1}{2}\|\partial_2\theta\|_{L^2(\gamma^-,\gamma^+)} \int_0^\delta dz \ dy_1
    %\\
    &\leq (2 \delta)^\frac{3}{2}\int_0^\Gamma \|\partial_2 u\|_{L^2(\gamma^-,\gamma^+)}^\frac{1}{2}\|\partial_2^2 u\|_{L^2(\gamma^-,\gamma^+)}^\frac{1}{2} \|\partial_2\theta\|_{L^2(\gamma^-,\gamma^+)}\sqrt{1+(h')^2}\ dy_1
    \\
    %&= 2 \delta^\frac{3}{2}\int_0^\Gamma \left(2\|\partial_2 u\|_{L^2(\gamma^-,\gamma^+)}\|\partial_2^2  u\|_{L^2(\gamma^-,\gamma^+)}\right)^\frac{1}{2} \|\partial_2\theta\|_{L^2(\gamma^-,\gamma^+)}\sqrt{1+(\gamma')^2}\ dy_1
    %\\
    %&\leq 2 \delta^\frac{3}{2}\int_0^\Gamma \left(\nu\|\partial_2 u\|_{L^2(\gamma^-,\gamma^+)}^2 +\nu^{-1}\|\partial_2^2 u\|_{L^2(\gamma^-,\gamma^+)}^2\right)^\frac{1}{2} \|\partial_2\theta\|_{L^2(\gamma^-,\gamma^+)}\sqrt{1+(\gamma')^2}\ dy_1
    %\\
    %&\leq  \delta^\frac{3}{2}\int_0^\Gamma \left(\mu\nu\|\partial_2 u\|_{L^2(\gamma^-,\gamma^+)}^2+\mu\nu^{-1}\|\partial_2^2 u\|_{L^2(\gamma^-,\gamma^+)}^2 + \mu^{-1} \|\partial_2\theta\|_{L^2(\gamma^-,\gamma^+)}^2\right)\sqrt{1+(\gamma')^2}\ dy_1
    %\\
    &\leq  C\delta^\frac{3}{2} \int_0^\Gamma\mu\nu\|\partial_2 u\|_{L^2(\gamma^-,\gamma^+)}^2+\mu\nu^{-1}\|\partial_2^2 u\|_{L^2(\gamma^-,\gamma^+)}^2 + \mu^{-1} \|\partial_2\theta\|_{L^2(\gamma^-,\gamma^+)}^2\ dy_1
\end{align*}
for some $\mu,\nu>0$ that will be determined later and $C=\left\|\sqrt{1+(h')^2}\right\|_{\infty}$.

Taking the long time average
\begin{align*}
    2|\langle\theta u\cdot \nabla \eta\rangle| \leq C\delta^\frac{3}{2} \left(\mu\nu \langle |\partial_2 u|^2\rangle + \mu \nu^{-1} \langle |\partial_2^2 u|^2\rangle +\mu^{-1}\langle |\partial_2 \theta|^2\rangle \right)
\end{align*}
and setting $\mu = 2\delta^\frac{3}{2}C$ and $\nu=2\delta^3 C^2 (a\epsilon)^{-1}$ yields the result.
\end{proof}

With all these preparations at hand we are able to prove the main result in the next subsection.
\subsection{Proof of Theorem \ref{main-theorem}}
%\begin{theorem}
%\label{main-theorem}
%Let $h\in W^{3,\infty}[0,\Gamma]$, $\alpha\in W^{1,\infty}(\gamma^-\cup\gamma^+)$, $\alpha>0$ on $\gamma^-\cup\gamma^+$ and $u_0\in W^{1,r}(\Omega)$ for some $r>2$. There exists a constant $0<\bar C<1$ such that for all $\alpha$ and $\kappa$ with
%\begin{align}
%    \label{theorem-condition-alpha+kappa-small}
%    \|\alpha+\kappa\|_\infty \leq \bar C
%\end{align}
%the following bounds on the Nusselt number hold.
%\begin{enumerate}
%    \item\label{main-theorem-bound-interpolation-kappa-leq-alpha}
%        If $|\kappa|\leq \alpha$ on $\gamma^-\cup\gamma^+$, $\Pra\geq \underline{\alpha}^{-\frac{3}{2}}\Ra^\frac{3}{4}$ and $\Ra^{-\frac{1}{2}}\leq \underline{\alpha}$ then %Nusselt number is bounded by
%        \begin{align*}
%            \Nu \leq C_\frac{1}{2} \|\alpha+\kappa\|_{W^{1,\infty}}^2 \Ra^\frac{1}{2} + C_\frac{5}{12}\Ra^\frac{5}{12}.
%        \end{align*}
%    \item\label{main-theorem-bound-interpolation-more-general-kappa}
%        If $|\kappa|\leq 2\alpha + \frac{1}{4\sqrt{1+(h')^2}}\sqrt{\alpha}$ on $\gamma^-\cup\gamma^+$, $\Pra\geq \underline{\alpha}^{-\frac{3}{2}}\Ra^\frac{3}{4}$ and $\Ra^{-1}\leq \underline{\alpha}$ then %Nusselt number is bounded by
%        \begin{align*}
%            \Nu \leq C_\frac{1}{2} \sqrt{\underline{\alpha}}\|\alpha+\kappa\|_{W^{1,\infty}}^2 \Ra^\frac{1}{2} + C_\frac{5}{12} \underline{\alpha}^{-\frac{1}{12}} \Ra^\frac{5}{12}.
%        \end{align*}
%    \item\label{main-theorem-bound-3over7bound}
%        If $|\kappa|\leq 2\alpha + \frac{1}{4\sqrt{1+(h')^2}}\sqrt{\alpha}$ on $\gamma^-\cup\gamma^+$ and $\Pra\geq \Ra^\frac{5}{7}$ then %the Nusselt number is bounded by
%        \begin{align*}
%            \Nu \leq C_\frac{3}{7} \Ra^\frac{3}{7}.
%        \end{align*}
%\end{enumerate}
%The constants are given by 
%\begin{align*}
%    C_\frac{1}{2} &= C(1+\|u_0\|_{W^{1,r}}^2)^{-1}
%    \\
%    C_\frac{5}{12} &= C \left(\|u_0\|_{W^{1,r}}+\|\dot\alpha\|_\infty+\|\dot\kappa\|_\infty\right)^\frac{1}{3}
%    \\
%    C_\frac{3}{7} &= C \|\alpha+\kappa\|_{W^{1,\infty}}^2 + C \left(1 + \underline{\alpha}^{-\frac{1}{6}}\left(1+\|u_0\|_{W^{1,r}}+\|\dot\alpha\|_\infty+\|\dot\kappa\|_\infty\right)\right)^\frac{1}{3},
%\end{align*}
%where $C>0$ denotes a constant depending only on the size of the domain $|\Omega|$, the Lipschitz constant of the boundary and $r$.
%\end{theorem}

%\begin{proof}
In the following we will extensively use
\begin{align}
    \label{proof-main-theorem-essinf-alpha-and-kappa-leq-1}
    \underline{\alpha}\leq 1, \qquad \|\kappa\|_\infty\leq 1.
\end{align}
The first inequality is justified as $\essinf \alpha \leq \essinf_{\kappa>0} (\alpha+\kappa) \leq \|\alpha+\kappa\|_\infty \leq 1$ by assumption \eqref{theorem-condition-alpha+kappa-small} and the second one as $\kappa(y_1,h(y_1))=-\kappa(y_1,1+h(y_1))$ and $\alpha>0$ almost everywhere one has $-\essinf_{\kappa <0}\kappa=\esssup_{\kappa>0}\kappa\leq \esssup_{\kappa>0} \alpha+\kappa\leq \|\alpha+\kappa\|_\infty\leq 1$ by assumption \eqref{theorem-condition-alpha+kappa-small}.


We will show that $\mathcal{Q}$ is non-negative for some appropriate choice of $\delta$. Then \eqref{nu-identity-including-Q} will yield the bound.

As $\mathbf{b}\leq 0$ by \eqref{average-energy-balance} plugging in the definition of $\mathbf{a}$, i.e. \eqref{def-a}, yields
\begin{equation}
    \label{Q-estimation-1}
    \begin{aligned}
        \mathcal{Q}[u,\theta,\eta]&=M \Ra^2 + \langle |\nabla\eta|^2\rangle + \langle |\nabla \theta|^2\rangle + 2\langle \theta u\cdot\nabla \eta\rangle + \frac{b}{\Ra}\langle |\nabla u|^2\rangle + \frac{b}{\Ra}\langle (2\alpha+\kappa)u_\tau^2\rangle_{\gamma^-\cup\gamma^+}
        \\
        &\qquad - \frac{b}{\Ra}\mathbf{b} + a \mathbf{a}
        \\
        &\geq M \Ra^2 + \langle |\nabla\eta|^2\rangle + \langle |\nabla \theta|^2\rangle + 2\langle \theta u\cdot\nabla \eta\rangle + \frac{b}{\Ra}\langle |\nabla u|^2\rangle + \frac{b}{\Ra}\langle (2\alpha+\kappa)u_\tau^2\rangle_{\gamma^-\cup\gamma^+} 
        \\
        &\qquad + a\langle |\nabla \omega|^2\rangle - 2 a \langle (\alpha+\kappa)u\cdot\nabla p\rangle_{\gamma^-\cup\gamma^+} - a \Ra \langle \omega\partial_1 T\rangle 
        \\
        &\qquad + \frac{2}{3\Pra} a \langle (\alpha+\kappa) u\cdot (u\cdot\nabla) u\rangle_{\gamma^-\cup\gamma^+} + 2 a \Ra \langle (\alpha+\kappa) u_\tau n_1\rangle_{\gamma^-}.
    \end{aligned}
\end{equation}
Next we estimate some of the terms individually. 
\begin{itemize}
    \item
    For the eighth term on the right-hand side of \eqref{Q-estimation-1} we can shift the derivative onto $u$ and $\alpha+\kappa$ as the boundary is periodic and get
    \begin{align*}
        -\int_{\gamma^-\cup\gamma^+} (\alpha+\kappa)u\cdot \nabla p \ dS &= \langle p \tau \cdot \nabla ((\alpha+\kappa) u_\tau)\rangle_{\gamma^-\cup\gamma^+} 
        \\
        &= \langle (\alpha+\kappa) p \tau \cdot \nabla u_\tau\rangle_{\gamma^-\cup\gamma^+} + \langle p (\dot\alpha+\dot\kappa) u_\tau\rangle_{\gamma^-\cup\gamma^+} 
    \end{align*}
    Using Hölder's inequality and Trace Theorem one gets
    \begin{align*}
        \left|\int_{\gamma^-\cup\gamma^+} (\alpha+\kappa)u\cdot \nabla p \ dS\right| \leq C \left( \|\alpha+\kappa\|_\infty \|u\|_{H^2} + \|\dot\alpha+\dot\kappa\|_\infty\|u\|_{H^1} \right)\|p\|_{H^1}
    \end{align*}
    where $\dot \alpha$ and $\dot \kappa$ denotes the derivative of $\alpha$ and $\kappa$ along the boundary. The pressure bound derived in Proposition \ref{proposition-pressure-bound} and Young's inequality imply
    \begin{equation}
        \label{Q-estimation-up}
        \begin{aligned}
            &2\left|\int_{\gamma^-\cup\gamma^+} (\alpha+\kappa)u\cdot \nabla p  \ dS\right|
            \\
            &\qquad \leq C \left( \|\alpha+\kappa\|_\infty \|u\|_{H^2} + \|\dot\alpha+\dot\kappa\|_\infty\|u\|_{H^1} \right)
            \\
            &\qquad\qquad\cdot\left[Ra\|T\|_2 + \|\alpha+\kappa\|_\infty\|u\|_{H^2}+\left(\frac{1+\|\kappa\|_\infty}{\Pra}\|u\|_{W^{1,r}}+\|\dot\alpha+\dot\kappa\|_\infty \right)\|u\|_{H^1}\right]
            \\
            &\qquad \leq \left(\epsilon+C\|\alpha+\kappa\|_\infty^2\right) \|u\|_{H^2}^2 + C_\epsilon \|\alpha+\kappa\|_{W^{1,\infty}}^2 \Ra^2  
            \\
            &\qquad\qquad + C\left( \left(\frac{1}{\Pra}\|u\|_{W^{1,r}}\right)^2+\|\alpha+\kappa\|_{W^{1,\infty}}^2+1 \right)\|u\|_{H^1}^2
        \end{aligned}
    \end{equation}
    for all $\epsilon>0$, where $C_\epsilon>0$ depends on $|\Omega|$, $r$, $\|h'\|_{\infty}$ and $\epsilon$ and in the last inequality we used that $\|\kappa\|\leq 1$.
    \item
    For the ninth term on the right-hand side of \eqref{Q-estimation-1} Hölder's and Young's inequality yield
    \begin{align*}
        |a\Ra \langle \omega\partial_1 T\rangle| &= |a\Ra \langle \omega\partial_1 (\eta+\theta)\rangle| \leq \frac{1}{2} \langle|\nabla \eta|^2\rangle + \frac{1}{2} \langle|\nabla \theta|^2\rangle + a^2 \Ra^2 \langle |\omega|^2\rangle.
    \end{align*}
    \item
    In order to estimate the tenth term on the right-hand side of \eqref{Q-estimation-1} we first use Hölder's inequality and Trace Theorem to get
    \begin{align}
        \label{Q-estimation-uuu-1}
        \frac{1}{\Pra}&\left|\int_{\gamma^-\cup\gamma^+} (\alpha+\kappa) u\cdot(u\cdot \nabla)u \ dS\right| 
        \leq C\frac{\|\alpha+\kappa\|_\infty}{\Pra} \left\|u^2 |\nabla u|\right\|_{W^{1,1}} .
    \end{align}
    Again Hölder's inequality with $\frac{1}{r}+\frac{1}{p}+\frac{1}{2}=1$ and Sobolev Theorem as in the proof of Proposition \ref{proposition-pressure-bound} imply
    \begin{equation}
        \label{Q-estimation-uuu-2}
        \begin{aligned}
            \left\|u^2 |\nabla u|\right\|_{W^{1,1}} &\leq C\left(\|u\|_q\|u\|_r \|\nabla u\|_2 + \|u\|_q\|\nabla u\|_r \|\nabla u\|_2+\|u\|_q\|u\|_r \|\nabla^2 u\|_2 \right) 
            \\
            &\leq C \left(\|u\|_{W^{1,r}}\|u\|_{H^1}+\| u\|_{H^2}\|u\|_{W^{1,r}}\right)\|u\|_{H^1}
        \end{aligned}
    \end{equation}
    for all $r>2$. Combining \eqref{Q-estimation-uuu-1} and \eqref{Q-estimation-uuu-2} and using Young's inequality and the assumption $\|\alpha+\kappa\|_\infty\leq 1$ yields
    \begin{equation}
        \label{Q-estimation-uuu-3}
        \begin{aligned}
            \frac{2}{3\Pra}\bigg|\int_{\gamma^-\cup\gamma^+} (\alpha+\kappa) &u\cdot(u\cdot \nabla)u \ dS\bigg| 
            \\
            &\leq C\frac{1}{\Pra} \left(\|u\|_{W^{1,r}}\|u\|_{H^1}+\| u\|_{H^2}\|u\|_{W^{1,r}}\right)\|u\|_{H^1}
            \\ 
            &\leq \epsilon \|u\|_{H^2}^2 + C\left( C_\epsilon\left(\frac{1}{\Pra} \|u\|_{W^{1,r}}\right)^2+1\right) \|u\|_{H^1}^2
        \end{aligned}
    \end{equation}
    for all $\epsilon>0$ where $C_\epsilon>0$ depends on $|\Omega|$, $r$, $\|h'\|_{\infty}$ and $\epsilon$.
    \item
    In order to estimate the eleventh term on the right-hand side of \eqref{Q-estimation-1} notice that by Trace Theorem and Young's inequality
    \begin{align}
        \label{Q-estimation-un}
        2\Ra \left|\int_{\gamma^-\cup\gamma^+} (\alpha+\kappa) u_\tau n_1 \ dS\right| &\leq C \Ra \|\alpha+\kappa\|_\infty \|u\|_{H^1} \leq C \|\alpha+\kappa\|_\infty^2\Ra^2 +  \|u\|_{H^1}^2.
    \end{align}
\end{itemize}
In order to apply these estimates we first notice that by Lemma \ref{lemma_u_bounded_by_omega}
\begin{align*}
    \|u\|_{W^{1,r}}\leq C \left(\|\omega\|_r + (1+\|\kappa\|_\infty)^{2-\frac{2}{r}}\|u\|_2\right).
\end{align*}
The $L^p$-norm of the vorticity and energy are bounded by Lemma \ref{LemmaVorticityBound} and Lemma \ref{lemmaEnergyDecay} respectively, implying
\begin{align*}
    \|u\|_{W^{1,r}}\leq C \left(\|u_0\|_{W^{1,r}} +\underline{\alpha}^{-1}\Ra\right),
\end{align*}
where we exploited \eqref{proof-main-theorem-essinf-alpha-and-kappa-leq-1}.
Using this bound for the $W^{1,r}$ norm of $u$, the prefactors in the individual estimates are independent of time. Then taking the long time average of \eqref{Q-estimation-up}, \eqref{Q-estimation-uuu-3} and \eqref{Q-estimation-un} and plugging the bounds into \eqref{Q-estimation-1} yields
\begin{align*}
        \mathcal{Q}[u,\theta,\eta]
        %&\geq M \Ra^2 + \langle |\nabla\eta|^2\rangle + \langle |\nabla \theta|^2\rangle + 2\langle \theta u\cdot\nabla \eta\rangle + \frac{b}{\Ra}\langle |\nabla u|^2\rangle + \frac{b}{\Ra}\langle (2\alpha+\kappa)u_\tau^2\rangle_{\gamma^-\cup\gamma^+} 
        %\\
        %&\qquad + a\langle |\nabla \omega|^2\rangle - 2 a \langle (\alpha+\kappa)u\cdot\nabla p\rangle_{\gamma^-\cup\gamma^+} - a \Ra \langle \omega\partial_1 T\rangle 
        %\\
        %&\qquad + \frac{2}{3\Pra} a \langle (\alpha+\kappa) u\cdot (u\cdot\nabla) u\rangle_{\gamma^-\cup\gamma^+} + 2 a \Ra \langle (\alpha+\kappa) u_\tau n_1\rangle_{\gamma^-}
        %\\
        &\geq M \Ra^2 + \frac{1}{2} \langle |\nabla\eta|^2\rangle + \frac{1}{2} \langle |\nabla \theta|^2\rangle + 2\langle \theta u\cdot\nabla \eta\rangle + \frac{b}{\Ra}\langle |\nabla u|^2\rangle + \frac{b}{\Ra}\langle (2\alpha+\kappa)u_\tau^2\rangle_{\gamma^-\cup\gamma^+} 
        \\
        &\qquad + a\langle |\nabla \omega|^2\rangle - a\left[2\epsilon+C\|\alpha+\kappa\|_\infty^2 \right]\left\langle \|u\|_{H^2}^2\right\rangle - C_\epsilon a \|\alpha+\kappa\|_{W^{1,\infty}}^2 \Ra^2
        \\
        &\qquad - aC\left[ C_\epsilon\left(\frac{1}{\Pra}\left(\|u_0\|_{W^{1,r}} + \underline{\alpha}^{-1}\Ra\right)\right)^2+\|\alpha+\kappa\|_{W^{1,\infty}}^2+1 \right]\left\langle\|u\|_{H^1}^2\right\rangle 
        \\
        &\qquad - a^2 \Ra^2 \langle |\omega|^2\rangle.
\end{align*}
Choosing $M= C_\epsilon a\|\alpha+\kappa\|_{W^{1,\infty}}^2$
\begin{equation}
    \label{Q-estimation-2}
    \begin{aligned}
        \mathcal{Q}[u,\theta,\eta]
        &\geq \frac{1}{2} \langle |\nabla \theta|^2\rangle + 2\langle \theta u\cdot\nabla \eta\rangle + \frac{b}{\Ra}\langle |\nabla u|^2\rangle + \frac{b}{\Ra}\langle (2\alpha+\kappa)u_\tau^2\rangle_{\gamma^-\cup\gamma^+} 
        \\
        &\qquad + a\langle |\nabla \omega|^2\rangle - a\left[2\epsilon+C\|\alpha+\kappa\|_\infty^2 \right]\left\langle \|u\|_{H^2}^2\right\rangle
        \\
        &\qquad - aC\left[ C_\epsilon\left(\frac{1}{\Pra}\left(\|u_0\|_{W^{1,r}} + \underline{\alpha}^{-1}\Ra\right)\right)^2+\|\alpha+\kappa\|_{W^{1,\infty}}^2+1 \right]\left\langle\|u\|_{H^1}^2\right\rangle 
        \\
        &\qquad - a^2 \Ra^2 \langle |\omega|^2\rangle.
    \end{aligned}
\end{equation}
%where $C_\epsilon$ additionally depends on $\epsilon$.
In order to estimate the second term on the right-hand side of \eqref{Q-estimation-2} use Lemma \eqref{lemma-estimate-theta-u-gradeta} to get
\begin{align*}
        \mathcal{Q}[u,\theta,\eta]
        &\geq \frac{b}{\Ra}\langle |\nabla u|^2\rangle + \frac{b}{\Ra}\langle (2\alpha+\kappa)u_\tau^2\rangle_{\gamma^-\cup\gamma^+} 
        \\
        &\qquad + a\langle |\nabla \omega|^2\rangle - a\left[3\epsilon+C\|\alpha+\kappa\|_\infty^2 \right]\left\langle \|u\|_{H^2}^2\right\rangle
        \\
        &\qquad - aC\left[ C_\epsilon\left(\frac{1}{\Pra}\left(\|u_0\|_{W^{1,r}} + \underline{\alpha}^{-1}\Ra\right)\right)^2+\|\alpha+\kappa\|_{W^{1,\infty}}^2+1 \right]\left\langle\|u\|_{H^1}^2\right\rangle 
        \\
        &\qquad - a^2 \Ra^2 \langle |\omega|^2\rangle- C_{\epsilon}\delta^6 a^{-1}\langle |\nabla u|^2\rangle.
\end{align*}
Next according to Lemma \eqref{lemma-H1-bounded-by-grad-and-bdry-terms} the first two terms on the right-hand side can be estimated by the $H^1$ norm, i.e.
\begin{align}
    \label{main-theorem-lemma-H1-bounded-by-grad-bdry-terms-explicite-used-formula}
    \frac{3b}{8\Ra}\langle |\nabla u|^2\rangle + \frac{b}{2\Ra} \langle (2\alpha+\kappa) u_\tau^2 \rangle_{\gamma^-\cup\gamma^+} \geq \frac{\underline{\alpha}b}{8\Ra}   \langle \|u\|_{H^1}^2 \rangle.
\end{align}
Then
\begin{align*}
        \mathcal{Q}[u,\theta,\eta]
        &\geq a\langle |\nabla \omega|^2\rangle - a\left[3\epsilon+C\|\alpha+\kappa\|_\infty^2 \right]\left\langle \|u\|_{H^2}^2\right\rangle
        \\
        &\qquad + \left[\frac{\underline{\alpha} b}{8\Ra} - a C_\epsilon\left(\left(\frac{1}{\Pra}\left(\|u_0\|_{W^{1,r}} +\underline{\alpha}^{-1}\Ra\right)\right)^2+\|\alpha+\kappa\|_{W^{1,\infty}}^2+1\right) \right]\left\langle\|u\|_{H^1}^2\right\rangle
        \\
        &\qquad - a^2 \Ra^2 \langle |\omega|^2\rangle + \left(\frac{5b}{8\Ra}-C_\epsilon\delta^6 a^{-1}\right) \langle |\nabla u|^2\rangle + \frac{b}{2\Ra}\langle (2\alpha+\kappa)u_\tau^2\rangle_{\gamma^-\cup\gamma^+}
\end{align*}
and by Lemma \ref{lemma_u_bounded_by_omega} and the smallness conditions \eqref{proof-main-theorem-essinf-alpha-and-kappa-leq-1}
\begin{align*}
    \langle \|u\|_{H^2}^2 \rangle \leq C \langle |\nabla \omega|^2\rangle + C (1+\|\dot\kappa\|_\infty) \langle \|u\|_{H^1}^2 \rangle.
\end{align*}
For $\mathcal{Q}$ one gets
\begin{align*}
        \mathcal{Q}[u,\theta,\eta]&\geq  a\left[1-3\epsilon C_1-C_2\|\alpha+\kappa\|_\infty^2 \right]\langle |\nabla \omega|^2\rangle 
        \\
        &\qquad + \left[\frac{\underline{\alpha} b}{8\Ra} - a 3\epsilon - a C_\epsilon\left(\left(\frac{1}{\Pra}\left(\|u_0\|_{W^{1,r}} +\underline{\alpha}^{-1}\Ra\right)\right)^2+\|\dot\alpha\|_\infty^2+\|\dot\kappa\|_\infty^2+1\right) \right]\left\langle\|u\|_{H^1}^2\right\rangle
        \\
        &\qquad - a^2 \Ra^2 \langle |\omega|^2\rangle + \left(\frac{5b}{8\Ra}-C\delta^6 a^{-1}\right) \langle |\nabla u|^2\rangle + \frac{b}{2\Ra}\langle (2\alpha+\kappa)u_\tau^2\rangle_{\gamma^-\cup\gamma^+},
\end{align*}
where we used Young's inequality and that $\|\alpha\|_\infty \leq 2$ as $\|\kappa\|_\infty \leq 1$ by \eqref{proof-main-theorem-essinf-alpha-and-kappa-leq-1} and $\|\alpha+\kappa\|_\infty \leq 1$. Setting $\epsilon = \frac{1}{6 C_1}$ and using the smallness assumption $\|\alpha+\kappa\|_\infty^2\leq \bar C= \frac{1}{2C_2}$ the first bracket is positive and we are left with
\begin{equation}
    \label{Q-estimate-before-a0}
    \begin{aligned}
        \mathcal{Q}[u,\theta,\eta] &\geq \left[\frac{\underline{\alpha} b}{8\Ra} - a C\left(\left(\frac{1}{\Pra}\left(\|u_0\|_{W^{1,r}} +\underline{\alpha}^{-1}\Ra\right)\right)^2+\|\dot\alpha\|_\infty^2+\|\dot\kappa\|_\infty^2+1\right) \right]\left\langle\|u\|_{H^1}^2\right\rangle
        \\
        &\qquad - a^2 \Ra^2 \langle |\omega|^2\rangle + \left(\frac{5b}{8\Ra}-C\delta^6 a^{-1}\right) \langle |\nabla u|^2\rangle + \frac{b}{2\Ra}\langle (2\alpha+\kappa)u_\tau^2\rangle_{\gamma^-\cup\gamma^+}.        
    \end{aligned}
\end{equation}
Next we have to differentiate between the two conditions on $\kappa$.
\begin{itemize}
    \item Case $|\kappa|\leq \alpha$
    %\item \emph{$|\kappa|\leq \alpha$ for the \ref{main-theorem-case-interpolation-kappa-leq-alpha}. case in Theorem \ref{main-theorem} with $\Ra^{-\frac{1}{2}}<\underline{\alpha}$ and $\Pra\geq \frac{1}{\underline{\alpha}^\frac{3}{2}}\Ra^\frac{3}{4}$}
    \vspace{10pt}\\
        In order to estimate the vorticity term notice that by Lemma \ref{lemma_u_bounded_by_omega} and the condition $|\kappa|\leq \alpha$ one has
        \begin{align*}
            \|\omega\|_2^2 &\leq \|\nabla u\|_2^2 + \int_{\gamma^-\cup\gamma^+} |\kappa| u_\tau^2 \leq \|\nabla u\|_2^2 + \int_{\gamma^-\cup\gamma^+} \alpha u_\tau^2 
            \\
            &\leq \|\nabla u\|_2^2 + \int_{\gamma^-\cup\gamma^+} (2\alpha+\kappa) u_\tau^2
        \end{align*}
        and taking the long time average \eqref{Q-estimate-before-a0} turns into
        \begin{align*}
            \mathcal{Q}[u,\theta,\eta] &\geq \left[\frac{\underline{\alpha} b}{8\Ra} - a C\left(\left(\frac{1}{\Pra}\left(\|u_0\|_{W^{1,r}} +\underline{\alpha}^{-1}\Ra\right)\right)^2+\|\dot\alpha\|_\infty^2+\|\dot\kappa\|_\infty^2+1\right) \right]\left\langle\|u\|_{H^1}^2\right\rangle
            \\
            &\qquad + \left[\frac{b}{2\Ra}- a^2\Ra^2\right]  \langle |\omega|^2\rangle + \left(\frac{b}{8\Ra}-C\delta^6 a^{-1}\right) \langle |\nabla u|^2\rangle.
        \end{align*}
        From the second squared bracket on the right-hand side it becomes clear that $a$ has to decay at least as fast as $\Ra^{-\frac{3}{2}}$ for $\mathcal{Q}$ to be non negative. Setting $a=a_0 \Ra^{-\frac{3}{2}}$
        \begin{align*}
            \mathcal{Q}[u,\theta,\eta] &\geq \left[\frac{\underline{\alpha} b}{8\Ra} - \frac{a_0 C}{\Ra^{\frac{3}{2}}} \left(\left(\frac{1}{\Pra}\left(\|u_0\|_{W^{1,r}} +\underline{\alpha}^{-1}\Ra\right)\right)^2+\|\dot\alpha\|_\infty^2+\|\dot\kappa\|_\infty^2+1\right) \right]\left\langle\|u\|_{H^1}^2\right\rangle
            \\
            &\qquad + \frac{1}{2\Ra}\left[b- 2a_0^2\right]  \langle |\omega|^2\rangle + \left(\frac{b}{8\Ra}-C\delta^6 a_0^{-1}\Ra^\frac{3}{2}\right) \langle |\nabla u|^2\rangle.
        \end{align*}
        The assumption $\Pra\geq \frac{1}{\underline{\alpha}^\frac{3}{2}}\Ra^\frac{3}{4}$ implies
        %\begin{align*}
        %    \mathcal{Q}[u,\theta,\eta] &\geq \frac{1}{\Ra} \left[\frac{\underline{\alpha} b}{8} - \frac{a_0 C}{\Ra^{\frac{1}{2}}} \left(\left(\underline{\alpha}^\frac{3}{2}\Ra^{-\frac{3}{4}}\left(\|u_0\|_{W^{1,r}} +\underline{\alpha}^{-1}\Ra\right)\right)^2+\|\dot\alpha\|_\infty^2+\|\dot\kappa\|_\infty^2+1\right) \right]\left\langle\|u\|_{H^1}^2\right\rangle
        %    \\
        %    &\qquad + \frac{1}{2\Ra}\left[b- 2a_0^2\right]  \langle |\omega|^2\rangle + \left(\frac{b}{8\Ra}-C\delta^6 a_0^{-1}\Ra^\frac{3}{2}\right) \langle |\nabla u|^2\rangle.
        %\end{align*}
        \begin{align*}
            \mathcal{Q}[u,\theta,\eta] &\geq \frac{\underline{\alpha}}{\Ra} \left[\frac{b}{8} - a_0 C \left(\|u_0\|_{W^{1,r}}^2+1+\underline{\alpha}^{-1}\Ra^{-\frac{1}{2}}\left(\|\dot\alpha\|_\infty^2+\|\dot\kappa\|_\infty^2+1\right)\right) \right]\left\langle\|u\|_{H^1}^2\right\rangle
            \\
            &\qquad + \frac{1}{2\Ra}\left[b- 2a_0^2\right]  \langle |\omega|^2\rangle + \left(\frac{b}{8\Ra}-C\delta^6 a_0^{-1}\Ra^\frac{3}{2}\right) \langle |\nabla u|^2\rangle
        \end{align*}
        and since $\Ra^{-\frac{1}{2}}<\underline{\alpha}$
        \begin{align*}
            \mathcal{Q}[u,\theta,\eta] &\geq \frac{\underline{\alpha}}{\Ra} \left[\frac{b}{8} - a_0 C \left(\|u_0\|_{W^{1,r}}^2+\|\dot\alpha\|_\infty^2+\|\dot\kappa\|_\infty^2+1\right) \right]\left\langle\|u\|_{H^1}^2\right\rangle
            \\
            &\qquad + \frac{1}{2\Ra}\left[b- 2a_0^2\right]  \langle |\omega|^2\rangle + \left(\frac{b}{8\Ra}-C\delta^6 a_0^{-1}\Ra^\frac{3}{2}\right) \langle |\nabla u|^2\rangle,
        \end{align*}
        where without loss of generality $C\geq 1$. In order for the two squared brackets to be non-negative we choose
        \begin{align*}
            a_0 = \frac{b}{8C\left(\|u_0\|_{W^{1,r}}^2+\|\dot\alpha\|_\infty^2+\|\dot\kappa\|_\infty^2+1\right)}
        \end{align*}
        and get
        \begin{align*}
            \mathcal{Q}[u,\theta,\eta]
            &\geq \left(\frac{b}{8\Ra} - C\delta^6 a_0^{-1}\Ra^\frac{3}{2}\right)\langle |\nabla u|^2\rangle.
        \end{align*}
        Letting $\delta$ solve $\frac{b}{8\Ra}= C \delta^6 a_0^{-1}\Ra^\frac{3}{2}$, i.e.
        \begin{align*}
            \delta = \left(\frac{a_0 b}{8C}\right)^\frac{1}{6}\Ra^{-\frac{5}{12}}
        \end{align*}
        $\mathcal{Q}$ is non-negative. Now we can come the estimating the Nusselt number. By \eqref{nu-identity-including-Q}
        \begin{align*}
            (1-b(1+\max h -\min h))\Nu + b \leq M \Ra^2 + 2\langle |\nabla \eta|^2\rangle.
        \end{align*}
        The gradient can be estimated by \eqref{nabla-eta-identity}, which yields
        \begin{align*}
            \langle |\nabla \eta|^2\rangle \leq C \delta^{-1}
        \end{align*}
        and plugging in $\delta$ and $M=C a\|\alpha+\kappa\|_{W^{1,\infty}}^2$ and choosing $b = \frac{1}{2(1+\max h - \min h)}$ we find
        \begin{align*}
            \Nu \leq C_\frac{1}{2} \|\alpha+\kappa\|_{W^{1,\infty}}^2 \Ra^\frac{1}{2} + C_\frac{5}{12} \Ra^\frac{5}{12}
        \end{align*}
        with $C_\frac{1}{2} = C(1+\|u_0\|_{W^{1,r}}^2)^{-1}$ and $C_\frac{5}{12} = C \left(\|u_0\|_{W^{1,r}}+\|\dot\alpha\|_\infty+\|\dot\kappa\|_\infty+1\right)^\frac{1}{3}$.
    \item Case $|\kappa|\leq 2\alpha + \frac{1}{4\sqrt{1+(h')^2}}\sqrt{\alpha}$\vspace{10pt}\\
        Using Lemma \ref{lemma_u_bounded_by_omega}, Trace Theorem and $\|\kappa\|_\infty \leq 1$ we can bound the vorticity term by
        \begin{align*}
            \|\omega\|_2^2 \leq C \|u\|_{H^1}^2
        \end{align*}
        and taking the long time average \eqref{Q-estimate-before-a0} turns into
        \begin{align*}
            \mathcal{Q}[u,\theta,\eta] &\geq \left[\frac{\underline{\alpha} b}{8\Ra} - a C\left(\left(\frac{1}{\Pra}\left(\|u_0\|_{W^{1,r}} +\underline{\alpha}^{-1}\Ra\right)\right)^2+\|\dot\alpha\|_\infty^2+\|\dot\kappa\|_\infty^2+1\right) \right]\left\langle\|u\|_{H^1}^2\right\rangle
            \\
            &\qquad - C a^2 \Ra^2 \left\langle\|u\|_{H^1}^2\right\rangle + \left(\frac{5b}{8\Ra}-C\delta^6 a^{-1}\right) \langle |\nabla u|^2\rangle + \frac{b}{2\Ra}\langle (2\alpha+\kappa)u_\tau^2\rangle_{\gamma^-\cup\gamma^+}.        
        \end{align*}
        Again applying \eqref{main-theorem-lemma-H1-bounded-by-grad-bdry-terms-explicite-used-formula} we find
        \begin{equation}
            \label{proof-main-theorem-before-new-second-cases}
            \begin{aligned}
                \mathcal{Q}[u,\theta,\eta] &\geq \left[\frac{\underline{\alpha} b}{8\Ra} - a C\left(\left(\frac{1}{\Pra}\left(\|u_0\|_{W^{1,r}} +\underline{\alpha}^{-1}\Ra\right)\right)^2+\|\dot\alpha\|_\infty^2+\|\dot\kappa\|_\infty^2+1\right) \right]\left\langle\|u\|_{H^1}^2\right\rangle
                \\
                &\qquad + \left[\frac{\underline{\alpha}b}{8\Ra}  - C a^2 \Ra^2\right] \left\langle\|u\|_{H^1}^2\right\rangle + \left(\frac{b}{4\Ra}-C\delta^6 a^{-1}\right) \langle |\nabla u|^2\rangle,
            \end{aligned}
        \end{equation}
        which because of the second squared bracket on the right-hand side imposes the condition on $a$ to decay at least as fast as $\Ra^{-\frac{3}{2}}$. We differentiate between two choices of $a$.
        \begin{itemize}
            \item[\small{$\blacktriangleright$}] 
                Setting $a=a_0\Ra^{-\frac{3}{2}}$ in \eqref{proof-main-theorem-before-new-second-cases} and estimating similar to before, using the assumptions $\Ra^{-1}\leq \underline{\alpha}$ and $\Pra\geq \frac{1}{\underline{\alpha}^\frac{3}{2}}\Ra^\frac{3}{4}$, we find
                %\begin{align*}
                %    \mathcal{Q}[u,\theta,\eta]
                %    &\geq \frac{\underline{\alpha}}{\Ra} \left[\frac{b}{8} - a_0 C\left(\|u_0\|_{W^{1,r}}^2+1 + \underline{\alpha}^{-\frac{1}{2}} \left(\|\dot\alpha\|_\infty^2+\|\dot\kappa\|_\infty^2+1\right)\right)\right]\left\langle\|u\|_{H^1}^2\right\rangle
                %    \\
                %    &\qquad - a_0^2 \Ra^{-1} \langle |\omega|^2\rangle + \left(\frac{5b}{8\Ra} - C\delta^6 a_0^{-1}\Ra^\frac{3}{2}\right)\langle |\nabla u|^2\rangle
                %    \\
                %    &\qquad + \frac{b}{2\Ra}\langle (2\alpha+\kappa)u_\tau^2\rangle_{\gamma^-\cup\gamma^+}.
                %\end{align*} 
                \begin{align*}
                    \mathcal{Q}[u,\theta,\eta]
                    &\geq \frac{\underline{\alpha}}{\Ra} \left[\frac{b}{8} - a_0 C\left(\|u_0\|_{W^{1,r}}^2+\underline{\alpha}^{-\frac{1}{2}}\left(\|\dot\alpha\|_\infty^2+\|\dot\kappa\|_\infty^2+1\right)\right) \right]\left\langle\|u\|_{H^1}^2\right\rangle
                    \\
                    &\qquad  +\frac{1}{\Ra}\left(\frac{\underline{\alpha}b}{8} - C a_0^2\right) \langle \|u\|_{H^1}^2\rangle + \left(\frac{b}{4\Ra} - C\delta^6 a_0^{-1}\Ra^\frac{3}{2}\right)\langle |\nabla u|^2\rangle.
                \end{align*}
                Choosing
                \begin{equation*}
                    \begin{gathered}
                        a_0=\frac{\underline{\alpha}^\frac{1}{2} b}{8} \frac{1}{C\left(\|u_0\|_{W^{1,r}}^2+\|\dot\alpha\|_\infty^2+\|\dot\kappa\|_\infty^2+1\right)},\qquad
                        \delta = \left(\frac{a_0 b}{4C}\right)^\frac{1}{6}\Ra^{-\frac{5}{12}},
                        \\
                        b = \frac{1}{2(1+\max h - \min h)}
                    \end{gathered}
                \end{equation*}
                $\mathcal{Q}$ is non-negative and hence
                \begin{align*}
                    \Nu &\leq  C a_0 \|\alpha+\kappa\|_{W^{1,\infty}}^2 \Ra^\frac{1}{2} + C \delta^{-1}
                    \\
                    &\leq C_\frac{1}{2} \underline{\alpha}^\frac{1}{2}\|\alpha+\kappa\|_{W^{1,\infty}}^2 \Ra^\frac{1}{2} + C_\frac{5}{12} \underline{\alpha}^{-\frac{1}{12}} \Ra^\frac{5}{12}
                \end{align*}
                with $C_\frac{1}{2} = C(1+\|u_0\|_{W^{1,r}}^2)^{-1}$ and $C_\frac{5}{12} = C \left(\|u_0\|_{W^{1,r}}+\|\dot\alpha\|_\infty+\|\dot\kappa\|_\infty+1\right)^\frac{1}{3}$.
            \item[\small{$\blacktriangleright$}]
                %Notice that by choice of $a_0$ and therefore $a$ in the previous case we are able to decrease the prefactor of $\Ra^\frac{1}{2}$ which came with the cost of increasing the prefactor of $\Ra^\frac{5}{12}$. We now want to exploit this by optimizing with regards to the Rayleigh number.
                Setting $a=a_0\Ra^{-\frac{11}{7}}$ in \eqref{proof-main-theorem-before-new-second-cases} and estimating similar to before, using $\Pra\geq \Ra^\frac{5}{7}$, we find
                \begin{align*}
                    \mathcal{Q}[u,\theta,\eta]
                    &\geq \frac{1}{\Ra} \left[\frac{\underline{\alpha} b}{8} - a_0 C \left( \Ra^{-\frac{9}{7}} \|u_0\|_{W^{1,r}}^2 +  \underline{\alpha}^{-2} + \Ra^{-\frac{4}{7}} \left(\|\dot\alpha\|_\infty^2 + \|\dot\kappa\|_\infty^2 + 1\right)\right)\right]\left\langle\|u\|_{H^1}^2\right\rangle
                    \\
                    &\qquad + \frac{1}{\Ra}\left(\frac{\underline{\alpha}b}{8} - C a_0^2 \Ra^{-\frac{1}{7}}\right) \langle \|u\|_{H^1}^2\rangle + \left(\frac{b}{4\Ra} - C\delta^6 a_0^{-1}\Ra^\frac{11}{7}\right)\langle |\nabla u|^2\rangle,
                \end{align*}
                which after choosing
                \begin{equation*}
                    \begin{gathered}
                    a_0 = \frac{\underline{\alpha}b}{8}\frac{1}{C(\|u_0\|_{W^{1,r}}^2+\underline{\alpha}^{-2}+ \|\dot\alpha\|_\infty^2 + \|\dot\kappa\|_\infty^2 + 1)},\qquad \delta = \left(\frac{a_0 b}{4C}\right)^\frac{1}{6} \Ra^{-\frac{3}{7}},
                    \\
                    b = \frac{1}{2(1+\max h - \min h)}
                    \end{gathered}
                \end{equation*}
                is non-negative, implying
                \begin{align*}
                    \Nu \leq C a \|\alpha+\kappa\|_{W^{1,\infty}}^2 \Ra^2 + C \delta^{-1}\leq  C_\frac{3}{7} \Ra^\frac{3}{7},
                \end{align*}
                where $C_\frac{3}{7}= C \left(\|\alpha+\kappa\|_{W^{1,\infty}}^2 + \underline{\alpha}^{-\frac{1}{2}} + \underline{\alpha}^{-\frac{1}{6}}(\|u_0\|_{W^{1,r}}+\|\dot\alpha\|_\infty+\|\dot\kappa\|_\infty+1)^\frac{1}{3}\right)$.
        \end{itemize}
\end{itemize}







%\end{proof}






%\begin{table}[h]
%    \centering
%    \textcolor{olive}{
%    \scalebox{0.85}{
%    \begin{tabular}{l|c|c|c|c|c|c|c|}
%         case & \multicolumn{3}{|c|}{$|\kappa|\leq \alpha$} & \multicolumn{2}{|c|}{\eqref{energy_condition_explicite_kappa}} & \eqref{energy_condition_explicite_kappa}\\
%         regime & \multicolumn{3}{|c|}{$\mu\leq \frac{1}{2}$} & \multicolumn{2}{|c|}{$\mu\leq 1$} & all \\
%         condition & $\frac{1}{2}\geq \mu\geq \frac{1}{24}$ & $\mu \leq \frac{1}{24}$ & & $\mu\leq \nu$ & $\nu = \frac{\mu}{2}\leq \frac{1}{2}$  & none \\
%         comment & best & general & \eqref{energy_condition_explicite_kappa} $\mu<\nu$ reached  & better bound  & biggest kappa & independent of $\alpha$, $\kappa$   \\
%         \hline
%         scaling & $\Ra^{\frac{5}{12}}$ & $\Ra^{\frac{1}{2}-2\mu}$ & $\Ra^\frac{21}{50}$ & $\Ra^{\frac{1}{2} - 2\mu}+ \Ra^{\frac{5+\mu}{12}}$ &  $\Ra^{\frac{1}{2}-\mu}+\Ra^{\frac{5+\mu}{12}}$ & $\Ra^\frac{3}{7}$ \\
%         \hline
%         optimal & $\Ra^{\frac{5}{12}}$ & $\Ra^{\frac{5}{12}}$ & $\Ra^\frac{21}{50}$ & $\Ra^{\frac{21}{50}}$ & $\Ra^{\frac{11}{26}}$ & $\Ra^\frac{3}{7}$ \\
%         approx & $\Ra^{0.4167}$ & $\Ra^{0.4167}$ & $\Ra^{0.42}$ &$\Ra^{0.42}$ & $\Ra^{0.4231}$ & $\Ra^{0.4286}$\\
%         for &  & $\mu=\frac{1}{24}$ & $\mu=\frac{1}{26}$ & $\mu=\frac{1}{25}$ & $\mu = \frac{1}{13}$ &  \\
%         \hline
%    \end{tabular}}
%    }
%    \caption{scaling overview}
%\end{table}





%\newpage

\section{Notation}
\begin{itemize}
    \item
    Perpendicular direction:
    \begin{align*}
        a^\perp = \begin{pmatrix}-a_2\\a_1\end{pmatrix}
    \end{align*}
    \item
    Vorticity:
    \begin{align*}
        \omega = \nabla^\perp \cdot u.
    \end{align*}
    \item
    $u_\tau = \tau\cdot u$ is the scalar velocity along the boundary.
    \\
    \item
    Tensor product:
    \begin{align*}
        A \ \colon B = a_{ij}b_{ij}
    \end{align*}
    \item
    If not explicitly stated differently $C$ will denote a positive constant, which might depend on the size of the domain $|\Omega|=\Gamma$, $\|h'\|_{\infty}$ and potentially on the exponent of the Sobolev norm.
    \\
    \item
    $n_+$ is the normal vector pointing upwards.
    \\
    \item
    $n_-$ is the normal vector pointing downwards.
    \\
    \item
    $n$ is the general normal vector with direction pointing outwards its domain.
    \\
    \item
    $\tau=n^\perp$ is the tangential vector oriented in the direction of $n^\perp$. 
    \\
    \item
    $\lambda$ is the parameterization of the boundary by arc-length in the direction of $\tau$.
    \\
    \item
    Variable in the curved domain: $y=\begin{pmatrix}y_1,y_2\end{pmatrix}\in \Omega$
    \\
    \item
    Variable in the straightened domain: $x=\begin{pmatrix}x_1,x_2\end{pmatrix}\in [0,\Gamma]\times[0,1]$
    \\
    \item
    $S$ denotes the integration variable over one-dimensional curves.
    \\
    \item
    If the Lebesgue and Sobolev norms are taken over the whole domain of definition of the function we abbreviate like the following
    \begin{align*}
        \|\alpha+\kappa\|_\infty &= \|\alpha+\kappa\|_{L^\infty(\gamma^-\cup\gamma^+)}, & \|u\|_p &= \|u\|_{L^p(\Omega)}
    \end{align*}
    \item
    For single integrals over the whole domain of integration we skip the integration variable for easier readability, i.e.
    \begin{align*}
        \int_{\gamma^-} \kappa u_\tau^2 &= \int_{\gamma^-} \kappa u_\tau^2 \ dS, & \int_\Omega \omega^2 = \int_{\Omega} \omega^2 \ dy
    \end{align*}
    \item
    $L^p(\gamma^-,\gamma^+)$, depending on $y_1$, denotes the $L^p$-norm along a vertical line defined by
    \begin{align*}
        \|u\|_{L^p(\gamma^-,\gamma^+)}^p=\int_{h(y_1)}^{1+h(y_1)} |u(y_1,y_2)|^p\ dy_2.
    \end{align*}
\end{itemize}



\section{Appendix}
\begin{itemize}
    \item The curvature on the boundary.\\
        As the bottom boundary can be parameterized by $(y_1,h(y_1))$ the tangential is parallel to $(1,h')$. Taking into consideration the symmetry of the domain, the outward pointing convention for $n$ and the definition of $\tau$, normalizing yields for the normal and tangent vectors
        \begin{align}
            \label{appendix-normal-vector-representation}
            n_\pm &= \pm \frac{1}{\sqrt{1+(h')^2}}\begin{pmatrix}-h'\\1\end{pmatrix},& \tau_\pm &= \mp \frac{1}{\sqrt{1+(h')^2}}\begin{pmatrix}1\\h'\end{pmatrix}.
        \end{align}
        In order to calculate the curvature we find by explicitly calculating
        \begin{align}
            \label{appendix-derivative-of-n}
            \frac{d}{dy_1} \tau_\pm = -\frac{h''}{1+(h')^2} n_\pm
        \end{align}
        and as the arc length parameterization in direction of $\tau_\pm$ is given by
        \begin{align*}
            \lambda(y_1) &= \int_{0}^{y_1} \sqrt{1+(h'(s))^2} \ ds \textnormal{ on }\gamma^-,
            & \lambda(y_1) &= \int_{y_1}^{\Gamma} \sqrt{1+(h'(s))^2} \ ds \textnormal{ on }\gamma^+
        \end{align*}
        one gets
        \begin{align*}
            \frac{d}{d y_1}\lambda(y_1) = \mp \sqrt{1+(h')^2} \textnormal{ on }\gamma^\pm
        \end{align*}
        and using \eqref{appendix-derivative-of-n}
        \begin{align*}
            \frac{d}{d\lambda} \tau_\pm = \mp \frac{1}{\sqrt{1+(h')^2}} \frac{d}{dy_1} \tau_\pm = \pm \frac{h''}{(1+(h')^2)^\frac{3}{2}} n_\pm,
        \end{align*}
        implying
        \begin{align}
            \label{appendix-kappa-representation}
            \kappa = \pm \frac{h''}{(1+(h')^2)^\frac{3}{2}} \textnormal{ on } \gamma^\pm.
        \end{align}
    \item Argument for \eqref{id-kappa}
        \begin{align}
            \label{appendix-proof-id-kappa}
            n\cdot(\tau\cdot \nabla) u
            &= n\cdot (\tau\cdot \nabla) (u_\tau \tau)
            = n\cdot \frac{d}{d\lambda} (u_\tau \tau)
            = u_\tau \kappa\ n\cdot n + n\cdot \tau \frac{d}{d\lambda}u_\tau =\kappa u_\tau.
        \end{align}
    \item Argument for \eqref{id-kappaUtau2-1}, \eqref{id-kappaUtau2-2} and \eqref{id-kappaUtau2-3}\\
    Using \eqref{appendix-proof-id-kappa}
        \begin{align}
            \label{appendix-proof-id-kappaUtau2}
            n\cdot(u\cdot \nabla) u
            &=u_{\tau} n\cdot (\tau\cdot \nabla) u = \kappa u_\tau^2.
        \end{align}
\end{itemize}


\subsection*{Acknowledgements}
The authors thank Steffen Pottel for useful feedback and suggestions regarding the manuscript.
FB acknowledges the support by the Deutsche Forschungsgemeinschaft (DFG) within the Research Training Group GRK 2583 "Modeling, Simulation and Optimization of Fluid Dynamic Applications”. CN was partially supported by DFG-TRR181 and GRK-2583. 
\printbibliography


\end{document}