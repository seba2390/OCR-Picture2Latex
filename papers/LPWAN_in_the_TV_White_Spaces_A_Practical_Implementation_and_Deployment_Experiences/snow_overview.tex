\section{Background and System Model}\label{sec:model}
\subsection{An Overview of SNOW}\label{sec:snow_overview}
\begin{figure*}[!htbp]
    \centering
      \subfigure[Network Architecture\label{fig:network}]{
    \includegraphics[width=0.4\textwidth]{figs/snow-arch.eps}
      }\hfill
      \subfigure[Dual-radio BS and subcarriers\label{fig:dualradio}]{
        \includegraphics[width=.4\textwidth]{figs/dual-radio.eps}
      }%\vspace{-0.05in}
    \caption{The SNOW architecture~\cite{snow2}.}
    \label{fig:arch}
\end{figure*}
% \begin{figure}[!htpb]
% \centering
% \includegraphics[width=0.5\textwidth]{figs/dual-radio.eps}
% \caption{Dual-radio BS and subcarriers~\cite{snow_ton}.}
% \label{fig:dualradio}
% \end{figure}
In this section, we provide a brief overview of SNOW. Its complete design and description is available in~\cite{mydissertation}. SNOW is a highly scalable LPWAN technology operating in the TV white spaces. It supports asynchronous, reliable, bi-directional, and concurrent communication between a BS and numerous nodes. Due to its long-range, SNOW forms a star topology allowing the BS and the nodes to communicate directly, as shown in Figure~\ref{fig:network}. The BS is powerful, Internet-connected, and line-powered while the nodes are power-constrained and do not have access to the Internet. To determine white space availability in a particular area, the BS queries a cloud-hosted geo-location database via the Internet. A node depends on the BS to learn its white space availability. In SNOW, all the complexities are offloaded to the BS to make the node design simple. 
Each node is equipped with a single half-duplex radio. To support simultaneous uplink and downlink communications, the BS uses a dual-radio architecture for reception (Rx) and transmission (Tx), as shown in Figure~\ref{fig:dualradio}.

%\subsubsection{Physical Layer}
The SNOW PHY uses a distributed implementation of OFDM called {\em D-OFDM}. D-OFDM enables the BS to receive concurrent transmissions from {\em asynchronous} nodes using a single-antenna radio (Rx-radio). Also, using a single-antenna radio (Tx-Radio), the BS can transmit different data to different nodes concurrently~\cite{snow, snow2, snow_ton, snow3, isnow_ton, isnow_p2p}. Note that the SNOW PHY is different from MIMO radio design adopted in other wireless domains such as LTE, WiMAX, and 802.11n~\cite{snow2} as the latter use multiple antennas to enable multiple transmissions and receptions.
The BS operates on a wideband channel split into orthogonal narrowband channels/subcarriers (Figure~\ref{fig:dualradio}). Each node is assigned a single subcarrier. 
For encoding and decoding, the BS runs inverse fast Fourier transform (IFFT) and global fast Fourier transform (G-FFT) over the entire wideband channel, respectively.
When the number of nodes is no greater than the number of subcarriers, every node is assigned a unique subcarrier. Otherwise, a subcarrier is shared by more than one node. 
%Narrower bands offer more extended communication range and consume less energy making it practical for IoT applications.
SNOW supports ASK and BPSK modulation techniques, supporting different bitrates. 
%Additionally, SNOW is capable of exploiting fragmented white space spectrum.


%\subsubsection{Medium Access Control Layer}
%The nodes in SNOW use a lightweight carrier sense multiple access/collision avoidance (CSMA/CA) medium access control (MAC) protocol similar to TinyOS~\cite{tinyos}. Additionally, the nodes can autonomously transmit, remain in receive mode, or sleep. A node runs clear channel assessment (CCA) before transmitting. If its subcarrier is occupied, the node makes a random back-off in a fixed congestion back-off window. After this back-off expires, the node transmits immediately if its subcarrier is free. This operation is repeated until the node makes this transmission and gets the acknowledgment (ACK).

%The nodes in SNOW use a lightweight CSMA/CA (carrier sense multiple access with collision avoidance)-based  media access control (MAC) protocol similar to TinyOS~\cite{tinyos}. Additionally, the nodes can autonomously transmit, remain in receive mode, or sleep. A node runs clear channel assessment (CCA) before transmitting. If its subcarrier is occupied, the node makes a random back-off in a fixed congestion back-off window. After this back-off expires, the node transmits immediately if its subcarrier is free and repeats this operation until it sends the packet and gets an acknowledgment (ACK).

The nodes in SNOW use a lightweight CSMA/CA (carrier sense multiple access with collision avoidance)-based media access control (MAC) protocol similar to TinyOS~\cite{tinyos}. The nodes can autonomously transmit, remain in receive mode, or sleep. Specifically, when a node has data to send, it wakes up by turning its radio on. Then it performs a random back-off in a fixed initial back-off window. When the back-off timer expires, it runs CCA (Clear Channel Assessment). If the subcarrier is clear, it transmits the data. If the subcarrier is occupied, then the node makes a random back-off in a fixed congestion back-off window. After this back-off expires, if the subcarrier is clean the node transmits immediately. This process is repeated until it makes the transmission and gets an acknowledgment (ACK).


\subsection{An Overview of TI CC13x0 LaunchPads}
Texas Instruments introduced TI CC1310 and TI CC1350 LaunchPads as a part of the SimpleLink microcontroller (MCU) platform to support ultra-low-power and long-range communication having a Cortex-M3 processor with 8KB SRAM and up to 128KB of in-system programmable storage~\cite{cc1350, snow_cots}.
With a small form-factor (length: 8cm, width: 4cm), both CC1310 and CC1350 are designed to operate in the lower frequency bands (287--351MHz, 359--527MHz, and 718--1054MHz) including the TV band. As an added feature, CC1350 can also operate in the 2.4GHz band while using Bluetooth low energy radio.
The packet structure of the CC13x0 devices includes a \code{preamble}, followed by \code{sync word}, \code{payload length}, \code{payload}, and \code{CRC}, chronologically. They support different data modulation techniques including Frequency Shift Keying (FSK), Gaussian FSK (GFSK), On-Off Keying (OOK), and a proprietary long-range modulation. They are capable of using a Tx/Rx bandwidth that ranges between 39 and 3767kHz. 
Additionally, with a supply voltage in the range of 1.8 to 3.8 volts, their Rx and Tx current consumption is 5.4mA and 13.4mA at +10dBm, respectively, offering ultra-low-power communication. 
The greatest advantage is that  
they have a programmable and reconfigurable physical layer, offering flexibility and feasibility for customized protocol implementation.











%In 2.2 provide a brief introduction of TI CC1310 (physical description, band, energy, and advantages) so that reader know both SNOW and 1310 before the technical section starts. 

% The physical (PHY) layer and the media access control (MAC) layer of SNOW is designed in~\cite{snow, snow2, snow_ton}. SNOW is an LPWAN platform to operate over TV white spaces. A SNOW node has a single half-duplex narrowband radio. Due to the long transmission (Tx) range, the nodes are directly connected to the BS and vice versa (Figure~\ref{fig:network}). SNOW thus forms a star topology.    
% The BS determines white spaces in the area by accessing a cloud-hosted database  through the Internet. 
% The nodes  are power constrained and not directly connected to the Internet.  They do not do spectrum sensing or cloud access.
% The BS uses a wide channel split into orthogonal subcarriers.   As shown in Figure~\ref{fig:dualradio}, the BS uses two radios, both operating on the same spectrum --  one for only transmission (called {\bf\slshape Tx radio}),  and the other for only reception (called {\bf\slshape Rx radio}). Such a dual-radio of the BS allows concurrent bi-directional communication in SNOW.

% The PHY layer of SNOW uses a {\bf D}istributed implementation of {\bf OFDM} (Orthogonal Frequency Division Multiplexing) for multi-user access, called {\bf D-OFDM}~\cite{snow, snow2}. In SNOW, the BS's wide white space spectrum is split into narrowband orthogonal subcarriers which carry parallel data streams to/from the distributed nodes from/to the BS as D-OFDM. A subcarrier bandwidth can be chosen as low as 100kHz, 200kHz, 400kHz depending on the packet and expected bit rate.  Narrower bands have lower bit rate but longer range, and consume less power~\cite{channelwidth}. Thus, SNOW adopted D-OFDM by assigning the orthogonal subcarriers to different nodes. Each node transmits and receives on the assigned subcarrier. Each subcarrier can be modulated using Amplitude Shift Keying (ASK) or Binary Phase Shift Keying (BPSK). 
% D-OFDM enables multiple receptions using a single antenna and also enables different data transmissions to different nodes using a single antenna. SNOW is a fully asynchronous network, requiring no time synchronization among the nodes for D-OFDM. The SNOW BS can  also exploit fragmented spectrum. If the BS spectrum is split into  $n$  subcarriers, then it can receive from $n$ nodes simultaneously. Similarly, it can transmit $n$ different data at a time. 

% \subsection{SNOW MAC Layer}

% This BS spectrum  is split into  $n$ overlapping orthogonal subcarriers -- $f_1, f_2, \cdots, f_n$ -- each of equal width. Each node is assigned one subcarrier. When the number of nodes is no greater than the number of subcarriers, every node is assigned a unique subcarrier.  Otherwise, a subcarrier is shared by more than one node. 
% The nodes that share the same subcarrier will contend for and access it using a CSMA/CA (Carrier Sense Multiple Access with Collision Avoidance) policy.

% The subcarrier allocation is done by the BS.
% The nodes in SNOW use a lightweight CSMA/CA protocol for transmission that uses a static interval for random back-off like the one used in TinyOS~\cite{tinyos} . 
% Specifically, when a node has data to send, it wakes up by turning its radio on. Then it performs a random back-off in a fixed {\slshape initial back-off window}.  When the back-off timer expires, it runs CCA (Clear Channel Assessment) and if the subcarrier is clear, it transmits the data. If the subcarrier is occupied, then the node makes a random back-off in a fixed {\slshape congestion back-off window}. After this back-off expires, if the subcarrier is clean the node transmits immediately. This process is repeated until it makes the transmission. The node then can go to sleep again.

% The nodes can autonomously transmit, remain in receive (Rx) mode, or sleep. 
% Since D-OFDM allows handling asynchronous Tx and 
% Rx, the link layer can send acknowledgment (ACK) for any transmission in either direction.   As shown in Figure~\ref{fig:dualradio}, both radios of the BS use the same spectrum and subcarriers - the subcarriers in the Rx radio are for receiving while those in the Tx radio are for transmitting. Since each node (non BS) has just a single half-duplex radio, it can be either receiving or transmitting, but not doing both at the same time.  
% Both experiments and large-scale simulations show high efficiency  of SNOW in latency and energy with a linear increase in throughput with the number of nodes, demonstrating its superiority  over existing LPWAN designs~\cite{snow, snow2}.







