\section{Evaluation}\label{sec:eval}
In this section, we provide an extensive evaluation (both uplink and downlink performances) of our CC13x0-based SNOW implementation, considering both stationary and mobile CC13x0 nodes.

\subsection{Setup}\label{sec:expsetup}
Figure~\ref{fig:testbed} shows our deployment in the city of Detroit, Michigan. Specifically, we deploy 22 CC1310 devices and 3 CC1350 devices (25 CC13x0 devices in total).
Due to our limited resources (e.g., batteries/power supply options outdoors), we create five (denoted by black pins) different clusters from 25 nodes at distances 200, 400, 600, 800, and 1000m from the BS (denoted by orange pin as well as a label). In each cluster, we randomly place 5 CC1310 devices or a mixture of CC1310 and CC1350 devices that are connected to a laptop via a USB hub so that they can operate without interruptions.
In our deployment, one group of nodes may be hidden from another group of nodes, thus creating hidden terminal scenarios. For example, the nodes at 1.05km and 800m (see Figure~\ref{fig:testbed}) may be hidden from each other since they are placed in the opposite directions of the BS.
We use a white space channel with frequency band 500--506MHz and split into 29 (numbered 1--29) overlapping (50\%) orthogonal subcarriers, each 400kHz wide. The USRP-based SNOW also uses a similar subcarrier bandwidth~\cite{snow_ton}.
\begin{figure}[!htb]
\centering
\includegraphics[width=0.35\textwidth]{figs/testbed.eps}
\caption{SNOW deployment in Detroit, Michigan.}
\label{fig:testbed}
\end{figure}
We use the 28th subcarrier as the join subcarrier and the 26th subcarrier as the downlink subcarrier. We do not use the 29th and the 27th subcarriers so that the join subcarrier may remain ICI-free.
The remaining 25 subcarriers are assigned to different nodes.
%A CC1310 device can operate in 470--527MHz of TV white space band. 
%talk about device configurations: packet structure, modulation, tx power, bandwidths,

We use the packet structure of the CC13x0 devices (\code{preamble}: (32 bits), \code{sync word}: (32 bits), \code{paylod length}:, \code{payload}: variable length, and \code{CRC} (16 bits)). Our default payload length is 30 bytes and contains random data. Our default bandwidth at the CC13x0 nodes is 39kHz. We use OOK modulation supported by the CC13x0 devices. Unlike the USRP-based SNOW, we do not use any spreading factor. Since the subcarrier bandwidths at the BS and the CC13x0 nodes are 400kHz and 39kHz, respectively, the oversampling at each subcarrier in the BS compensates for the spreading factor. Our default transmission power at the BS and the nodes is 15dBm. However, a CC13x0 device may choose to operate with any transmission power between 0 and 15dBm, as needed by our ATPC model. The receive sensitivity at the BS is set to -114dBm, as per the white space regulations~\cite{whitespaceSurvey}. Unless stated otherwise, these are the default parameter settings.

\begin{figure*}[!htbp]
    \centering
      \subfigure[Packet reception rate vs. distances\label{fig:prr_dist}]{
    \includegraphics[width=0.35\textwidth]{figs/eval/network/prr-dist.eps}
      }\hfill
      \subfigure[Packet reception rate in uplink\label{fig:prr_txs}]{
        \includegraphics[width=.35\textwidth]{figs/eval/network/prr-txs.eps}
      }\hfill
      \subfigure[Packet reception rate in downlink\label{fig:down_prr}]{
        \includegraphics[width=0.35\textwidth]{figs/eval/network/downprr.eps}
      }
    \caption{Reliability in long distance communication.}
    \label{fig:prr}
\end{figure*}
\subsection{Reliability over Long Distance}\label{sec:prr_dist}
\subsubsection{Achievable Distance}
We first test the achievable communication range of CC13x0-based SNOW. We take one CC13x0 device and transmit to the BS from different distances between 200 and 1000m. We keep our antenna height at 3 meters above the ground for both the BS and node. At each distance, each CC13x0 transmits 1000 packets with a random interval between 0 and 500ms. The transmission power is set to 15dBm. To show comparison, we repeat the same experiments without compensating for CSI and CFO as well. Additionally, we test the achievable distance between two LoRa SX1276 devices (bandwidth: 125kHz, spreading factor: 12, coding rate: 4/5) with the above settings. We choose a bandwidth of 125kHz in LoRa since it is the closest value compared to the bandwidth in CC13x0 (39kHz).


Figure~\ref{fig:prr_dist} shows that the {\em packet reception rate (PRR)} at the SNOW BS when packets are sent with and without compensating for CSI and CFO, comparing with LoRa. PRR is defined as the ratio of the number of successfully decoded packets to the number of total packets sent.
As shown in this figure, with CSI and CFO compensation, the BS achieves 95\% of PRR from a distance of 1km . 
Without CSI and CFO compensation, the PRR at the BS is as low as 30\% from 1km distance. This figure also shows that a LoRa SX1276 device can deliver packets to another over 1km with a PRR of 96\%, which is similar to the CC13x0-based SNOW node (CSI and CFO compensated). The results thus demonstrate that SNOW on the new platform is highly competitive against LoRa, an LPWAN leader that operates in the ISM band. Additionally,
we find that beyond approximately 1km, PRR stars decreasing in our implementation. Our best guess is that if we can place the BS or the node at a higher altitude (FCC allows up to 30 meters), we may achieve high reliability over much longer communication range.

\subsubsection{Uplink Reliability}
To show the uplink reliability under concurrent transmissions from different numbers of nodes (CFO and CSI compensated), we transmit from 1, 5, 10, 15, 20, and 25 nodes to the BS, respectively. In this experiment, all the nodes are distributed within 200 and 1000m of the BS, as shown in Figure~\ref{fig:testbed}. Each node uses different subcarrier bandwidths between 39 and 392kHz. For each bandwidth starting from 39kHz, a node sends consecutive 1000 packets. Between each bandwidth, a node sleeps for 500ms. Thus, the BS knows the change in the bandwidth. 
Note that in practical deployment scenarios, a node can let know the BS of its bandwidth during the joining process.
In this experiment, we show the performance of a node for different bandwidths. Figure~\ref{fig:prr_txs} shows that we can achieve up to 99\% PRR when 25 nodes transmit concurrently using 39kHz, and up to 98.1\% PRR using 392kHz. Additionally, the number of concurrent transmissions does not affect the trend in PRR for any given bandwidth.
Thus, ensuring high uplink reliability of our CC13x0-based implementation over long distances.

\subsubsection{Downlink Reliability}
In the downlink, we test the reliability by sending 100 consecutive 30-byte (payload length) packets to each of the 25 nodes that are distributed within 200 and 1000m of the BS. We repeat the same experiment 50 times with an interval between 0 and 500ms. In this experiment, we compensate for both CSI and CFO.
Figure~\ref{fig:down_prr} shows our downlink reliability at different distances observed by different nodes. For better representation, we cluster the nodes that are located approximately at the same distance and plot the PRR against distance.
As shown in this figure, the PRR in the downlink is as high as 99\% for 75\% of the nodes that are approximately 200m away from the BS. Also, 75\% of the nodes that are approximately 1km away from the BS achieve a PRR of 96\%. Thus, this experiment confirms high downlink reliability of our CC13x0-based implementation over long distances.


\subsection{Performance in Uplink Communication}\label{sec:tputuplink}
 
In this section, we evaluate the uplink network performance in terms of throughput (achievable bitrate), end-to-end delay (time between the transmission and ACK reception of a packet), and energy consumption. We allow from 1, 5, 10, 15, 20, and 25 nodes to transmit concurrently to the BS, respectively. We distribute the nodes between 200 and 1000m in our testbed. Each node transmits 1000 30-byte (payload length) packets with a random packet interval between 0 and 100ms. 
Such packet interval confirms that the node's transmissions are indeed asynchronous to the BS.
Each node uses a bandwidth of 39kHz. 
We evaluate the uplink network performance for three different cases: (1) nodes or/and BS {\bf compensate} for CSI, CFO, and ATPC; (2) nodes or/and BS {\bf compensate only} for CSI and CFO, but not ATPC; (3) nodes or/and BS {\bf do not compensate} for CSI, CFO, and ATPC. Note that ATPC applies to the nodes only, and hence we use "or/and" in the above three cases.
For each case, we run the experiments as long as at least 90\% of the packets are delivered to the BS. A node may thus try several retransmissions.
\begin{figure*}[!htbp]
    \centering
      \subfigure[Throughput\label{fig:tput}]{
    \includegraphics[width=0.35\textwidth]{figs/eval/network/throughput.eps}
      }\hfill
      \subfigure[End-to-end delay\label{fig:delay}]{
        \includegraphics[width=.35\textwidth]{figs/eval/network/delay.eps}
      }\hfill 
      \subfigure[Energy consumption\label{fig:energy}]{
        \includegraphics[width=.35\textwidth]{figs/eval/network/energybit.eps}
      }
    \caption{Network performance in uplink under varying number of nodes.}
    \label{fig:performance}
 \end{figure*}

\subsubsection{Throughput}
Figure~\ref{fig:tput} shows that the BS achieves 279kbps of throughput when 25 nodes transmit concurrently (case 1), yielding approximately 11.16kbps per node. Additionally, the overall throughput at the BS increases linearly with an increase in the number of nodes transmitting concurrently. When only CSI and CFO are compensated for, the overall throughput at the BS also increases with an increase in the number of concurrent transmissions, however, it depends on the nodes' distribution (physical) across the network. If there is no near-far power problem, the overall throughput may be the same as observed in case 1. With no compensation, the throughput per node is approximately 5kbps, thus more than 2x lesser than case 1. Note that a CC13x0 device can generate a baseband signal with a symbol rate of 11.2kbaud (OOK modulated). Thus, using a node bandwidth of 39kHz or 392kHz will not affect the per node throughput. However, a lower node bandwidth gives higher PRR (Section~\ref{sec:prr_dist}) due to longer symbol duration, combating the ICI to some extent.
Additionally, if we use any other COTS device that can generate a higher symbol rate for OOK at higher node bandwidth, the per node throughput may also increase with an increase in the node bandwidth. Similarly, adopting a quadrature amplitude modulation (QAM) or frequency modulation (FM) (both yet to be explored in SNOW) at the nodes may increase the throughput at the SNOW BS. In the future, we shall adopt QAM-based or FM-based modulations in SNOW.
%Typical LoRa deployments use a bandwidth of 125kHz, a spreading factor of 7, and a coding rate of 4/5 that provides a theoretical PHY transmission rate of 5092bps~\cite{sx1301}. Thus, with CC1310 devices deployed as SNOW nodes, we can achieve a throughput $11200/5092 \approx 2.2x$ higher than the typical LoRa deployments. On the other hand, SigFox can achieve a total uplink throughput of 100bps~\cite{sigfox}, which is $11200/100 = 112x$ lower than our uplink throughput. 
Overall, our CC13x0-based SNOW implementation shows high potential for practical deployments of low-rate IoT applications~\cite{whitespaceSurvey, ismail2018low}.

\subsubsection{End-to-end Delay}
Figure~\ref{fig:delay} shows the average end-to-end delay per packet at the nodes. When CSI, CFO, and ATPC are compensated for, the average end-to-end delay per packet in the network is 53ms with 25 concurrent transmissions. Also, for case 1, the average end-to-end delay per packet almost remains constant for any number of concurrent transmissions. For case 2, where only CSI and CFO are compensated for, the average end-to-end delay per packet increases a little bit with an increase in the number of concurrent transmissions. With no compensation, the average end-to-end delay per packet increases almost linearly with an increase in the number of concurrent transmissions. The reason is that a node retransmits several packets several times.  
%A typical LoRa deployment with a payload size of 30-byte takes approximately 90ms of air-time (without ACK)~\cite{lorawan}, which is $90/55 = 1.63x$ higher than CC13x0-based SNOW. 
Overall, our CC13x0-based SNOW implementation shows great promise for low-latency Industry 4.0 applications and their deployments~\cite{modekurthy2018utilization}.




\subsubsection{Energy Consumption}
Figure~\ref{fig:energy} shows the average energy consumption per bit at the nodes. We use the CC13x0 energy profile to calculate the energy consumption during Tx, Rx, and idle time~\cite{cc1350}.
For case 1, where the CSI, CFO, and ATPC are compensated for, the average energy consumption per bit in the network is approximately 11.6$\mu$J with 25 concurrent transmissions. Also, the average energy consumption per bit almost remains constant for any number of concurrent transmissions. For case 2, where only CSI and CFO are compensated for, the average energy consumption per bit increases to 16$\mu$J for 25 concurrent transmissions. Also, when nothing is compensated for, the average energy consumption per packet increases almost linearly with an increase in the number of concurrent transmissions. The reason is that a node retransmits several packets several times. Overall, small energy consumption in case 1 confirms that our CC13x0-based SNOW implementation may host long-lasting deployments.
%for the IoT applications.


%Our observation here is that with the increase in the number of parallel transmissions per packet energy consumption at the nodes increases in LoRaWAN, while it stays almost same in SNOW. Due to our limited number of LoRaWAN devices, in this paper, we are not able to show at what level of parallelism SNOW outperforms LoRaWAN. However, we have performed large scale simulations (with 2000 nodes) in our earlier work in~\cite{snow_ton}, showing superiority of SNOW over LoRaWAN in terms of energy consumption.

\subsection{Performance in Downlink Communication}
\begin{figure}[!htb]
\centering
\includegraphics[width=0.35\textwidth]{figs/eval/network/downtput.eps}
\caption{Average throughput per node in downlink.}
\label{fig:down_tput}
\end{figure}
In this section, we evaluate the downlink network performance in terms of throughput. The BS sends 1000 consecutive 30-byte (payload length) packets to each of the 25 nodes. Also, the BS and the nodes compensate for both CSI and CFO. In the downlink, the BS uses a Tx bandwidth of 39kHz. We repeat the above experiment without compensating for CSI and CFO as well. Figure~\ref{fig:down_tput} shows the average throughput per node at different distances. For better representation, we cluster the nodes that are located approximately at the same distance and plot average throughput against the distance. As shown in this figure, a node that is approximately 200m away from the BS can achieve an average downlink throughput of 4.75kpbs, while both the BS and the node compensate for CSI and CFO. The average throughput remains almost the same as those observed at other distances, up to 1km as well. In contrast, the average throughput drops sharply with an increase in the distance when CSI and CFO are not compensated for. Note that a CC13x0 device can successfully receive an OOK-modulated signal with 4.8kbaud symbol rate and 39kHz bandwidth~\cite{snow_cots}.
%Compared to SigFox that has a downlink throughput of 600bps~\cite{sigfox}, CC13x0-based SNOW may achieve $4800/600 = 8x$ higher throughput, 
Overall, our CC13x0-based SNOW implementation holds the potentials for low-rate IoT applications.


\subsection{Performance under Mobility}\label{sec:mobility_per}
In this section, we evaluate the network performance under CC13x0 node's mobility in terms of throughput, energy consumption, and end-to-end delay. 
We allow all 25 nodes to transmit concurrently to the BS. However, due to our limited resources, we enable mobility in only one node that is approximately 600m far from the BS and calculate its performance.
All nodes except the mobile node continuously transmit to the BS 30-byte (payload size) packets with a random interval between 0 and 50ms, using their assigned subcarriers, each 39kHz wide. Our CFO estimation technique (Section~\ref{sec:cfo}) allows a mobile node to travel in any direction but at a uniform speed during a packet transmission.
We vary the speed of the mobile node for different packets to 5mph, 10mph, and 20mph in any arbitrary direction within our network range. At each speed, we change the payload size of the mobile node between 10 and 120bytes. For each payload size, the mobile node transmits to the BS 1000 packets with a random interval between 0 and 50ms. We run experiments with the above settings for two cases: (1) the mobile node or/and the BS {\bf compensate} for CSI, CFO, and ATPC; (2) the mobile node or/and the BS {\bf do not compensate} for CSI, CFO, and ATPC.
\begin{figure}[t]
    \centering
      \subfigure[Throughput \label{fig:m_tput}]{
    \includegraphics[width=0.35\textwidth]{figs/eval/mobility/throughput.eps} 
      }\hfill%\vspace{-0.13in}
      \subfigure[Energy consumption \label{fig:menergy}]{
        \includegraphics[width=.35\textwidth]{figs/eval/mobility/energybit.eps}
      }
    \caption{Throughput and energy consumption under mobility.}
    \label{fig:mobility}
\end{figure}

\subsubsection{Throughput}
Figure~\ref{fig:m_tput} shows the throughput at the BS (of the mobile node) for different speeds and payload sizes. As this figure suggests, the throughput decreases slightly from 11.18kbps to 10.3kbps at 5mph, 10.35kbps at 10mph, and 10.3kbps at 20mph for an increase in the payload size between 10 and 120bytes, as CSI, CFO, and ATPC are compensated for. When the mobile node or/and the BS do not compensate for CSI, CFO, and ATPC, the throughput decreases sharply with an increase in speed and packet size. For example, at 20mph, the throughput drops to approximately 0 for payload size of 60bytes. In general, the packet size is susceptible to node's mobility. In fact, if CSI and CFO are not compensated for, the effects of unknown channel conditions and frequency offset ripple through a longer packet and increase the BER. Thus, our SNOW implementation is resilient and robust under the mobility of the nodes.

\subsubsection{Energy Consumption}
%Figure~\ref{fig:menergy} shows that the average energy consumption per packet increases slightly higher than linear with an increase in the payload size, when CSI, CFO, or/and ATPC are compensated for. For example, at 5mph, it takes on average 1.78mJoule, 2.85mJoule, 4.5mJoule, 8.2mJoule, and 10.2mJoule to transmit a payload of size 10, 30, 60, 90 and 120bytes, respectively. Also, the average energy consumption per packet increases with an increase in the speed. As shown in this figure, the average energy consumption per packet is approximately 1.78mJoule at 5mph and 2.14mJoule at 20mph, for a payload of size 10bytes. Our best guess is that at higher speeds the mobile node retransmits several packets multiple times due to ACK loss, high BER at BS, or/and ATPC. Overall, Figure~\ref{fig:menergy} confirms that our CC13x0-based SNOW implementation is energy efficient under the mobility of the nodes.

Figure~\ref{fig:menergy} shows an interesting behavior of the energy consumption per bit of the CC13x0 devices. At each speed (CSI, CFO, or/and ATPC compensated), the average energy consumption per bit decreases when we increase the payload size from 10 to 60bytes. But, when we increase the payload size beyond 60bytes (e.g., 90 and 120bytes), the energy consumption per bit starts to increase. For example, the average energy consumption per bit (at 5mph) for payload sizes 10, 30, 60, 90, and 120bytes are 22.2, 11.8, 9.3, 11.2, and 12.1$\mu$J, respectively. At speeds 10 and 20mph, the trends in energy consumption per bit are also similar to the above trend.
To the best of our knowledge, CC13x0 devices show such behavior due to their arbitrary end-to-end delays of packets with different payload lengths (to be discussed in Section~\ref{sec:e2edspeed}). Overall,  a payload of length 60bytes may be preferable (in terms of energy per bit) in the CC13x0-based SNOW implementation.

%the average energy consumption per bit increases slightly higher than linear with an increase in the payload size, when CSI, CFO, or/and ATPC are compensated for. For example, at 5mph, it takes on average 1.78mJoule, 2.85mJoule, 4.5mJoule, 8.2mJoule, and 10.2mJoule to transmit a payload of size 10, 30, 60, 90 and 120bytes, respectively. Also, the average energy consumption per packet increases with an increase in the speed. As shown in this figure, the average energy consumption per packet is approximately 1.78mJoule at 5mph and 2.14mJoule at 20mph, for a payload of size 10bytes. Our best guess is that at higher speeds the mobile node retransmits several packets multiple times due to ACK loss, high BER at BS, or/and ATPC. Overall, Figure~\ref{fig:menergy} confirms that our CC13x0-based SNOW implementation is energy efficient under the mobility of the nodes.


\subsubsection{End-to-end Delay}\label{sec:e2edspeed}
Figure~\ref{fig:mdelay} shows that the average per-packet end-to-end delay at the mobile node increases with an increase in speed and payload size. For example, at 5mph, the average per-packet end-to-end delays with payloads of sizes 10, 30, 60, 90, and 120bytes are 35, 56, 88, 160, 200ms, respectively; at 20mph, the average end-to-end delays are 42, 65, 93, 170, 220ms, respectively. Note that these delays in terms of payload lengths are in fact arbitrary. For example, the ratio between the delays of 30-byte and 10-byte payloads is not 30/10=3x, but 56/35=1.6x at 5mph and 65/42$\approx$1.5x at 20mph speed at the node.

%at 10mph, the average end-to-end delays are 37, 60, 90, 162, 210ms, respectively;
Figure~\ref{fig:m_cdfpayload} shows the cumulative distribution function (CDF) of the end-to-end delay at a constant speed of 5mph with varying payload sizes. This figure shows that 60\% of the 10-byte (payload length) packets observe an end-to-end delay more than 35ms, 65\% of the 30-byte (payload length) packets observe an end-to-end delay more than 55ms, 50\% of the 60-byte (payload length) packets observe an end-to-end delay more than 90ms, 98\% of the 90-byte (payload length) packets observe an end-to-end delay more than 150ms, and 95\% of the 120-byte (payload length) packets observe an end-to-end delay more than 195ms. 
Furthermore, Figure~\ref{fig:m_cdfmph} shows the CDF of end-to-end delays for a fixed payload length of 30bytes at varying speed. As this figure shows, 98\% of the packets at 5mph observe an end-to-end delay up to 55ms, 99.99\% of the packets at 10mph observe an end-to-end delay up to 60ms, and 98\% of the packets at 20mph observe an end-to-end delay up to 65ms. Overall, Figure~\ref{fig:m_eted} confirms that our CC13x0-based SNOW implementation may provide very low latency under the mobility of the nodes.
\begin{figure*}[t]
    \centering
      \subfigure[End-to-end delay\label{fig:mdelay}]{
        \includegraphics[width=.35\textwidth]{figs/eval/mobility/delay.eps}
      }\hfill %\vspace{-0.15in}
      \subfigure[CDF of end-to-end delay\label{fig:m_cdfpayload}]{
        \includegraphics[width=0.35\textwidth]{figs/eval/mobility/cdf-delay.eps}
      }\hfill
      \subfigure[CDF of end-to-end delay\label{fig:m_cdfmph}]{
        \includegraphics[width=.35\textwidth]{figs/eval/mobility/cdf-mph.eps}
      }
    \caption{End-to-end delay analysis of mobile node under varying payload length.}
    \label{fig:m_eted}
\end{figure*}
\begin{figure}[!htb]
\centering
\includegraphics[width=0.35\textwidth]{figs/eval/ws/interference.eps}
\caption{Performance under interference.}
\label{fig:ws_interf}
\end{figure}
\subsection{Performance in the Presence of Interference}
In this section, we evaluate the performance of our implementation in the presence of interference. Due to the COVID-19 restrictions, we limit this experiment in an indoor area of (20$x$30)m$^2$.
We place a USRP B200 device within 10m of the BS to act as an interferer. Within 200ms intervals, the interferer randomly operates near the 25 subcarriers used by the CC13x0 devices and transmits random 40-byte payloads. The CC13x0 devices send packets to the BS concurrently and incessantly 
from distances between 20 and 30m. 
We let the interferer overlap 20\%, 40\%, 60\%, 80\%, and 100\% with the legitimate subcarriers. For each magnitude of overlaps, we run this experiment for 2 minutes. In Figure~\ref{fig:ws_interf}, we show the distribution of PRR at the BS (CSI, CFO, and ATPC are compensated) for 30 runs of the above experiment.
As shown in this figure, even with 100\% overlap, the PRR at the BS can be as high as 84\%. For 60\% overlap, the PRR at the BS is as high as 92\%. Overall, as we decrease the percentage of overlaps, the PRR increases at the BS. This experiments thus confirms that the impact of external interference is less severe or negligible when the interferer's spectrum partially overlaps with the legitimate subcarriers in our CC13x0-based SNOW implementation.

%To evaluate the performance of our implementation in the presence of interference, we place a USRP B200 device within 10m of the BS to act as an interferer. Within 200ms intervals, the interferer randomly operates near the 25 subcarriers used by the CC13x0 devices and transmits random 40-byte payloads. The CC13x0 devices send packets to the BS concurrently and incessantly from distances between 20 and 30m. Due to the COVID-19 restrictions, we limit this experiment in an indoor area of 600m$^2$. We let the interferer overlap 20\%, 40\%, 60\%, 80\%, and 100\% with the legitimate subcarriers. For each magnitude of overlaps, we run this experiment for 2 minutes. In Figure~\ref{fig:ws_interf}, we show the distribution of PRR at the BS (CSI, CFO, and ATPC are compensated) for 30 runs of the above experiment. As shown in this figure, even with 100\% overlap, the PRR at the BS can be as high as 84\%. For 60\% overlap, the PRR at the BS is as high as 92\%. Overall, as we decrease the percentage of overlaps, the PRR increases at the BS. This experiments thus confirms that the impact of external interference is less severe or negligible when the interferer's spectrum partially overlaps with the legitimate subcarriers in our CC13x0-based SNOW implementation.


\subsection{Performance Comparison with LoRaWAN}

In this section, we experimentally compare the performance of our CC13x0-based SNOW implementation with a LoRaWAN network. We have 8 Dragino LoRa/GPS-Hat Sx1276 transceivers that can transmit or receive on a single channel. We create a LoRaWAN gateway capable of receiving on 3 channels simultaneously using 3 of our LoRa-Hats, while the remaining 5 devices act as LoRaWAN nodes. For a fair comparison, we allow 5 SNOW nodes (3 CC1350 devices and 2 CC1310 devices) to transmit to the SNOW BS, allowing only 3 subcarriers for data Rx/Tx. Similarly, in LoRaWAN, 5 nodes transmit on three 500kHz channels using a spreading factor of 7 and a coding rate of $\frac{4}{5}$.
In SNOW, the nodes use a subcarrier bandwidth of 392kHz with no bit spreading factor. While choosing 500kHz or 392kHz has no differentiable impact in our CC13x0-based SNOW implementation (as discussed in Section~\ref{sec:tputuplink}), we choose the latter due to the configurable Tx bandwidth limitation of the devices. The LoRaWAN gateway uses 3 adjacent 500kHz channels in the 915MHz frequency band (in the US), while the SNOW BS, in this setup, uses 3 adjacent overlapping subcarriers, numbered 10, 11, and 12 (please refer to Section~\ref{sec:expsetup} for subcarrier allocation).

The above configuration for LoRaWAN will result in its best possible throughput and energy-efficiency~\cite{snow_ton}. Additionally,  choosing LoRaWAN's largest spreading factor (e.g., 12) and 125kHz channel bandwidth for better reliability will make the comparison unfair with SNOW (e.g., SNOW uses 392kHz bandwidth and no spreading factor).
Each node (for both LoRaWAN and SNOW) transmits 1000 thirty-byte (payload size) packets from a distance of approximately 1km to the gateway/BS with a random inter-packet interval between 500 and 1000ms and a Tx power of 15dBm. Each node randomly hops to a different channel/subcarrier after sending 200 packets. In LoRaWAN, the nodes use the pure ALOHA MAC protocol, and thus operating as the Class-A LoRaWAN nodes~\cite{ismail2018low}. In SNOW, the nodes use the lightweight CSMA/CA MAC protocol (as discussed in Section~\ref{sec:snow_overview}).
In the following, we compare LoRaWAN and SNOW in terms of reliability, throughput, and energy consumption with the above settings. Note that achievable distance comparison between SNOW and LoRaWAN has already been presented in Section~\ref{sec:prr_dist}.

\begin{figure*}[t]
    \centering
      \subfigure[Packet reception rate at the gateway/BS\label{fig:lora_prr}]{
    \includegraphics[width=0.35\textwidth]{figs/lora/loraprr.eps}
      }\hfill
      \subfigure[Throughput at the gateway/BS\label{fig:lora_tput}]{
        \includegraphics[width=.35\textwidth]{figs/lora/loratput.eps}
      }
      \hfill 
      \subfigure[Energy consumption at the nodes\label{fig:lora_en}]{
        \includegraphics[width=.35\textwidth]{figs/lora/loraenbits.eps}
      }
    \caption{Uplink performance comparison between SNOW and LoRaWAN.}
    \label{fig:lora-comp}
 \end{figure*}
\subsubsection{Reliability Comparison with Parallel Tx/Rx}
Figure~\ref{fig:lora_prr} shows the PRR at the gateway/BS for LoRaWAN and CC13x0-based SNOW implementation under varying number of nodes that transmit concurrently. As shown in this figure, when only one node transmits, the PRR is approximately 95\% in both LoRaWAN and SNOW. Also, the PRR of LoRaWAN decreases with the increase in the number of parallel transmissions. For SNOW, it remains almost similar with the increase in the number of parallel transmissions. For example, when 5 nodes transmit in parallel, LoRaWAN achieves a PRR of 59\%, compared to 87\% in SNOW. Such performance degradation in LoRaWAN happens as it uses an ALOHA-based MAC protocol without any collision avoidance. The PRR of LoRaWAN may increase if we increase the inter-packet interval and will remain the same for SNOW even if we decrease the inter-packet interval. 

\subsubsection{Throughput Comparison}
Figure~\ref{fig:lora_tput} shows the overall throughput (kbps) comparison at the gateway/BS between LoRaWAN and SNOW. As shown in this figure, the throughput at the LoRaWAN gateway is approximately 20.8kbps, compared to 10.64kbps at the SNOW BS when only one node transmits. However, the throughput at the SNOW BS surpasses that at the LoRaWAN gateway when 2 or more nodes transmit concurrently. As shown in Figure~\ref{fig:lora_tput}, the throughput at the SNOW BS is $\frac{48.12}{12.9} \approx 3.7$x higher compared to LoRaWAN when 5 nodes transmit concurrently. The throughput in LoRaWAN decreases as we increase the number of LoRaWAN nodes because of the following reason. The LoRaWAN nodes adopt the ALOHA-based MAC protocol. Thus, a LoRaWAN node does not check if a channel is free before transmitting a packet, which results in a collision with another ongoing packet transmission (if any) from a different LoRaWAN node in the same channel. Consequently, both packets are lost as well as not considered in the throughput calculation.
Packet collisions might happen in our setup since 5 LoRaWAN nodes share 3 channels. Note that our setup is realistic, which emulates the scenario of having hundreds of LoRaWAN nodes under a 64-channel (maximum possible) LoRaWAN gateway. For 100 LoRaWAN nodes, the channel to node ratio is $\frac{64}{100} \approx \frac{3}{5}$, which is also the same in our setup. The SNOW nodes, on the other hand, avoid packet collisions on the same channel by adopting the CSMA/CA MAC protocol.
Compared to LoRaWAN, our CC13x0-based SNOW implementation thus observes better throughput.

\subsubsection{Energy Consumption Comparison}
Figure~\ref{fig:lora_en} shows the per packet energy consumption
%(in mJ based on the correctly received packets) 
at the nodes of LoRaWAN and our CC13x0-based SNOW implementation. As the figure shows, when 5 nodes transmit in parallel, a LoRaWAN node spends 3.7$\mu$J/bit, compared to 10.6$\mu$J/bit in SNOW. Here, the per-bit energy consumption in SNOW is slightly higher than that in LoRaWAN. However, this figure shows that the per-bit energy consumption increases in LoRaWAN and remains almost steady in SNOW when the number of concurrent nodes increases. SNOW is designed to enable a large number of concurrent transmissions to the BS and such a tendency in energy consumption shows its energy efficiency under that scenario. On the other hand, the number of retransmissions to deliver a packet increases with the increase in the number of nodes in LoRaWAN, thereby increasing the per packet energy consumption. Due to a limited number of devices, we are unable to demonstrate this in real experiment. Note that the LoRaWAN configuration used in this experiment (e.g., channel bandwidth of 500kHz, spreading factor of 7, coding rate of $\frac{4}{5}$) is the most energy-efficient (for a given Tx power) for an individual LoRaWAN node. 
As reported in~\cite{xu2019measurement}, increasing the spreading factor increases energy consumption at the LoRaWAN nodes. Similarly, decreasing the channel bandwidth also increases energy consumption at the LoRaWAN nodes~\cite{xu2019measurement}.


\subsection{Discussion on the Performance of CC1310 and CC1350}
As we experiment with both CC1310 and CC1350 devices as the SNOW nodes (Sections~\ref{sec:prr_dist}--\ref{sec:mobility_per}), we see no noticeable performance difference at the SNOW BS in terms of reliability and throughput, compared to our previous experiments with the CC1310 devices only in~\cite{snow_cots}. Similarly, there is no noticeable performance difference in terms of end-to-end latency and energy consumption between the two platforms. 
The reason is that our SNOW implementation is minimally invasive to the IoT devices having almost similar PHY layer properties. We envision that any IoT device with a programmable PHY, capable of operating in the white spaces, and capable of OOK/BPSK modulation may work as a SNOW node by addressing the same set of challenges as described through Sections~\ref{sec:csi}--\ref{sec:near-far}. In this paper, we practically demonstrate this by implementing SNOW on CC1310 and CC1350 devices and showing that their performance is similar.
























