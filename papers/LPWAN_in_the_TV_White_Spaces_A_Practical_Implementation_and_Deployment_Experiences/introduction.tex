%\vspace{-0.1in}
\section{Introduction}\label{sec:intro}

%-Introduce LPWANs
Low-Power Wide-Area Network (LPWAN) is an emerging communication technology that supports long-range, low-power, and low-cost connectivity to numerous devices. It is regarded as a key technology to drive the Internet-of-Things (IoT). Due to their escalating demand, recently multiple LPWAN technologies have been developed that operate in the licensed/cellular (NB-IoT, LTE-M, 5G, etc.) or unlicensed/non-cellular (SNOW, LoRa, SigFox, etc.) bands~\cite{ismail2018low}. Most of the non-cellular technologies operate in the sub-1GHz ISM band except SNOW (Sensor Network over White Spaces) and WEIGHTLESS-W that operate in the TV white spaces~\cite{ismail2018low, whitespaceSurvey}.
%In this paper, we implement the first LPWAN over white spaces designed for practical deployment.
%-then motivate the usage of white spaces for LPWAN and introduce the evolution of SNOW\\

\emph{White spaces} are the allocated but locally unused TV spectrum (54-698MHz in the US) that can be used by unlicensed devices as the secondary users.
Compared to the crowded ISM band, white spaces offer less crowded and much wider spectrum in both urban and rural areas, boasting an abundance in rural and suburbs~\cite{snow2}. Due to their low frequency, white spaces have excellent propagation and obstacle penetration characteristics enabling long-range communication. Thus, they hold the potentials for LPWAN to support various IoT applications. 
To our knowledge, WEIGHTLESS-W (which, to the best of our knowledge, has been decommissioned~\cite{ismail2018low}) and SNOW~\cite{snow_ton} are the only two efforts to exploit the TV white spaces for LPWAN. 
Initially introduced in~\cite{snow}, SNOW is a highly scalable LPWAN technology offering reliable, bi-directional, concurrent, and asynchronous communication between a base station (BS) and numerous nodes.


%- Limitation of current SNOW in terms of practical deployment (i.e., it is implemented on USRP) + the importance of practical deployment (i.e. the need for cheap, small form-factor, and deployable SNOW devices) which should be in terms of cost and deployment ease.\\
% Despite such promise, SNOW has received very less focus from the research community due to its limited availability regarding practical deployment. Existing SNOW implementation uses the Universal Software Radio Peripheral (USRP) platform for both the BS and the nodes. USRP is a hardware platform developed for software-defined radio applications~\cite{usrp}. Using the USRP platform as the SNOW node limits the practical deployment of SNOW in real-world applications due to several factors including its high cost and large form-factor. For example, a USRP B200 device with a half-duplex radio costs approximately US \$750, as of today. As such, it inherently becomes costly to deploy a large-scale SNOW network. Today, IoT applications including smart city (i.e., waste management, smart lighting, smart grid), transportation and logistics (i.e., connected vehicles), agricultural and smart farming (i.e., Microsoft FarmBeats), process management (i.e.,oil field monitoring) ,  and healthcare require collection of information from thousands of IoT nodes.


Despite its promise as a LPWAN technology, SNOW has not yet received sufficient attention from the research community due to its limited availability for practical deployment.
The SNOW implementation in~\cite{snow_ton}, which is also available as open-source~\cite{snow_bs}, uses Universal Software Radio Peripheral (USRP) devices as LPWAN nodes, hindering the applicability of this technology in practical and large-scale deployment.
USRP is a hardware platform developed for software-defined radio applications~\cite{usrp}. Using them as the SNOW node limits the practical deployment of SNOW in real-world applications due to several factors including its high cost and large form-factor. Currently, a USRP B200 device with a half-duplex radio costs $\approx$ \$750 USD. As such, it inherently becomes costly to deploy a large-scale SNOW network. 
Today, IoT applications including smart city (e.g., waste management and smart grid), transportation and logistics (e.g., connected vehicles), agricultural and smart farming (i.e., Microsoft FarmBeats), process management (e.g.,oil field monitoring), and healthcare require collection of information from thousands of IoT nodes~\cite{ismail2018low}.

In this paper, we address the above practical limitations of the existing SNOW technology by implementing it on low-cost and low form-factor commercial off-the-shelf (COTS) devices.
Through this implementation, we demonstrate that any COTS device that has a programmable physical layer (PHY), operates in the white spaces, and supports amplitude-shift-keying (ASK) or binary phase-shift-keying (BPSK) may work as a SNOW node in practical deployments. Along with our original TI CC1310-based SNOW implementation in~\cite{snow_cots}, we thus implement SNOW on TI CC1350 that has a programmable PHY, costs approximately \$30 USD (retail price) and is 10x smaller than a USRP B200 device, thereby making SNOW adoptable in practical IoT applications. Additionally, we emphasize the fact that this implementation maybe is adaptable in the diverse range of COTS IoT devices with minimal efforts (e.g., only needs to deal with new development software), which makes it highly portable and practical choice for SNOW.





%Through this implementation, we demonstrate that any COTS device that have a programmable physical layer (PHY), operate in the white spaces, and support amplitude-shift-keying (ASK) or binary phase-shift-keying (BPSK) may work as a SNOW node in practical deployments. Along with our original TI CC1310-based SNOW implementation in~\cite{snow_cots}, we thus implement SNOW on TI CC1350 that has a programmable PHY, costs approximately \$30 USD (retail price) and is 10x smaller than a USRP B200 device, thereby making SNOW adoptable in practical IoT applications.



%with heterogeneous devices.




%Specifically, thanks to its programmable PHY, we use the widely available and low-power TI CC13x0 (CC1310~\cite{cc1310} or CC1350~\cite{cc1350} LaunchPad) IoT device which costs approximately \$30 USD (retail price) and is 10x smaller than a USRP B200 device (including antenna), thereby making SNOW adoptable in practical IoT applications.
%\revise{Note that, commercially available LoRa or SigFox devices cannot be used as an alternative to the CC1310 device since they are tied to their proprietary protocols and are not allowed for any modification. Additionally, to adapt SNOW on LoRa or SigFix devices, it will require completely new PHY and media access control (MAC) protocol.}



%- Our key contribution. Say sth like “ (1) We make a bold effort and implement on cheap, small form-factor, and deployable SNOW devices. Specifically, we take TI CC1310 devices that has programmable PHY”… say some great points and challenges in implementation….. Also say this can be implemented in any similar COTs devices that has programmable PHY (name some). (2) We have explored several issues like CSI, CFO, near-far…. and implemented to enable reliable communication on these COTs devices….. (3) We run experiments with the new implementation…. Saye sth about what our results say…. (for example peerformance similar to USRP… or close/better than LoRa….)\\

%The SNOW technology has never been implemented on IoT devices before. 
The existing USRP-based SNOW implementation does not face the following practical challenges due to the expensive and powerful hardware design of USRP (as reflected by evaluation in~\cite{snow_ton, snow, snow2}), which the implementation on CC13x0 (CC1310 or CC1350) has to address. 
{\bf First}, due to its orthogonal frequency division multiplexing (OFDM)-based design, the SNOW BS transmitter is subject to high peak-to-average power ratio (PAPR). Thus, the overall reliability at the CC13x0 device during downlink communication degrades severely. 
{\bf Second}, due to the asymmetric bandwidth requirements of the SNOW BS and the nodes, channel state information (CSI) estimation between the BS and a CC13x0 device plays a vital role in both uplink and downlink communications. Without CSI estimation, the overall reliability and the communication range decreases. 
{\bf Third}, Carrier frequency offset (CFO) needs to be handled robustly as its effects are much more pronounced in the low-cost CC13x0 devices, leading to severe inter-carrier-interference (ICI). ICI decreases the overall bitrate in both uplink and downlink communications. 
While these challenges are quite common in the wireless domain, due to the novel design of SNOW, the existing solutions/approaches may not be adopted in SNOW.
Along with addressing these challenges, through this new implementation, we also make SNOW resilient to the classic near-far power problem. Due to the near-far power problem, where a far node's transmission gets buried under a near node's transmission radiation, the reliability in the uplink communication may be degraded. We thus address the above challenge as well.
%, topsep=0pt [leftmargin=*]
Specifically, we make the following key contributions.
\begin{itemize}
	\item We implement SNOW for practical deployment on CC13x0s to work as SNOW nodes. Compared to the current USRP-based SNOW implementation, the cost and the form-factor of a single CC13x0-based SNOW node are decreased approximately 25x and 10x, respectively.
	\item In our implementation, we address several practical challenges including the PAPR problem, CSI and CFO estimation, and near-far power problem. Specifically,
	%We address PAPR by slightly modifying the downlink strategy of SNOW which offers a trade-off between reliability and latency. 
	we propose a data-aided CSI estimation technique that allows a CC13x0 device to communicate directly with the SNOW BS from a distance of approximately 1km. Additionally, we propose a pilot-based CFO estimation technique that takes into account the device mobility and increases reliability in both uplink and downlink communications. Finally, we address the near-far power problem in SNOW through an adaptive transmission power control (ATPC) protocol that improves the reliability in the uplink communications.
	\item We experiment with the CC13x0-based SNOW implementation through deployment in the city of Detroit, Michigan. Our results demonstrate that we achieve an uplink throughput of 11.2kbps per node.
	%, which is 2.2x higher compared to LoRa theoretical uplink throughput in typical setup (channel bandwidth: 125kHz, spreading factor: 7, and coding rate: 4/5)~\cite{lorawan}. 
	Additionally, our overall uplink throughput increases \emph{linearly} with the increase in the number of SNOW nodes. In the downlink, we achieve a throughput of 4.8kbps per node.
	Compared to a typical LoRa deployment (channel bandwidth: 500kHz, spreading factor: 7, and coding rate: 4/5), our uplink throughput is approximately 3.7x higher when 5 nodes transmit to a gateway that can receive concurrent packets using 3 channels.
	%Compared to SigFox, our uplink and downlink throughputs are 112x and 8x higher, respectively.
	%Also, the overall uplink throughput increases \emph{linearly} with the increase in the number of SNOW nodes. Our uplink 
	%Compared to typical LoRa deployment (channel bandwidth: 125kHz, spreading factor: 7, and coding rate: 4/5~\cite{lorawan}), our uplink throughput is approximately 2.2x higher. Also, 
\end{itemize}

%lora uplink downlink: 5468 bps  for 125kHz SF = 7 and CR = 4/5
%SigFox UL: 100bps, DL: 600bps


%This work will make it possible to adopt SNOW for real IoT applications. We will make the implementation available for both research and development purposes, which will not be limited to the CC1310 device only. Thus, our experience in this paper will help others to deploy SNOW network with commercially available low-cost, low-power off-the-shelf devices that can operate on TV white spaces and support ASK or/and BPSK modulation with minimal configuration changes.


In the rest of the paper, Section~\ref{sec:model} provides an overview of SNOW and TI CC13x0. Section~\ref{sec:implementation} presents our SNOW implementation. Section~\ref{sec:near-far} describes the near-far power problem in SNOW. Sections~\ref{sec:deploy} and~\ref{sec:eval}  analyze the deployment cost and performance of our CC13x0-based SNOW, respectively. Section~\ref{sec:related} overviews related work. Finally, Section~\ref{sec:conclusion} concludes our paper.



















