\section{Network Architecture and Deployment Cost}\label{sec:deploy}
\begin{figure}[!htb]
\centering
\includegraphics[width=0.5\textwidth]{figs/deployment.eps}
\caption{The SNOW architecture for practical deployment (The PC may be replaced by a Raspberry Pi device. The two USRP B200 devices can be replaced by a USRP B2100 device that has two half-duplex radios.)}
\label{fig:deployment}
\end{figure}
In this section, we discuss the practical applicability of our implementation. Figure~\ref{fig:deployment} shows our network view. The SNOW BS is a PC that connects two USRP B200 devices (Tx-Radio and Rx-Radio). The BS is also connected to the Internet. In the BS, a USRP B210 device may be used which has two half-duplex radios. Also, a Raspberry Pi device may be used instead of the PC. All the CC13x0 nodes are battery-powered and directly connected to the BS. 
\begin{figure}[!htb]
\centering
\includegraphics[width=0.35\textwidth]{figs/cost/costs.eps}
\caption{Practical deployment cost with numerous nodes.}
\label{fig:cost}
\end{figure}

We now analyze the deployment cost of our CC13x0-based SNOW implementation and compare with the original USRP-based SNOW implementation in~\cite{snow_ton}. 
%and LoRa. We are unable to quantify the cost of a SigFox deployment due to the unspecified (not publicly available) charges of SigFox network subscription, hence do not compare with SigFox. 
Figure~\ref{fig:cost} shows the total deployment cost of our CC13x0-based SNOW implementation for different numbers of nodes between 1000 and 10,000. A CC1310 or CC1350 device costs approximately \$30 USD (retail price). The price for the BS is approximately \$1600 USD (two USRP B200 devices \$750 USD each, and two antennas \$50 USD each). In this comparison, the cost of the PC is not considered since it is common for both implementations. For SNOW implementation in~\cite{snow_ton}, a node is a  USRP B200 device that has an antenna and runs on a Raspberry Pi. A Raspberry Pi device costs approximately \$35 USD.

To provide an insight into the deployment cost of a LoRaWAN network, we consider the Dragino LoRa/GPS-Hat (e.g., SX1276 chip) that runs on Raspberry Pi and costs approximately \$32 USD (retail price)~\cite{dragino}. We choose this LoRaWAN node since it has almost identical computational and RF capabilities, compared to the TI CC13x0 devices (e.g., both have the Cortex-M series MCU, similar energy profiles, same set of sensors, and software support). In addition, we consider a LoRaWAN gateway that costs approximately \$299 USD and can receive packets on multiple channels simultaneously~\cite{loracost}. We rule out cheaper LoRaWAN devices (costs $\approx$\$10 USD) from the calculation since they do not have a similar profile as CC13x0 and do not provide software support.
%A LoRa gateway and an end-device (node) cost approximately \$299 and \$97 USD, respectively~\cite{loracost}. 

As shown in Figure~\ref{fig:cost}, to deploy an LPWAN with 1000 nodes, the CC13x0-based SNOW implementation may cost approximately \$31.6K USD, compared to \$836.6K USD for the USRP-based SNOW implementation proposed in~\cite{snow_ton}, and \$67.3K USD for the Dragino LoRa-Hat-based LoRaWAN. 
For a deployment of 10,000 nodes, the costs are \$301.6K, \$8.3M, and \$670.3K USD
for CC13x0-based SNOW implementation, USRP-based SNOW implementation, and Dragino LoRa-Hat-based LoRaWAN, 
respectively. As shown in Figure~\ref{fig:cost}, the cost of each LPWAN increases linearly with the increase in the number of nodes. However, the cost of our CC13x0-based SNOW implementation in unnoticeable.
Our new implementation of SNOW on the CC13x0 devices thus becomes highly scalable in terms of cost, making SNOW deployable for practical applications.

