%%
%% This is file `sample-acmsmall.tex',
%% generated with the docstrip utility.
%%
%% The original source files were:
%%
%% samples.dtx  (with options: `acmsmall')
%% 
%% IMPORTANT NOTICE:
%% 
%% For the copyright see the source file.
%% 
%% Any modified versions of this file must be renamed
%% with new filenames distinct from sample-acmsmall.tex.
%% 
%% For distribution of the original source see the terms
%% for copying and modification in the file samples.dtx.
%% 
%% This generated file may be distributed as long as the
%% original source files, as listed above, are part of the
%% same distribution. (The sources need not necessarily be
%% in the same archive or directory.)
%%
%% The first command in your LaTeX source must be the \documentclass command.
\documentclass[acmsmall]{acmart}

%%
%% \BibTeX command to typeset BibTeX logo in the docs
\AtBeginDocument{%
  \providecommand\BibTeX{{%
    \normalfont B\kern-0.5em{\scshape i\kern-0.25em b}\kern-0.8em\TeX}}}

%% Rights management information.  This information is sent to you
%% when you complete the rights form.  These commands have SAMPLE
%% values in them; it is your responsibility as an author to replace
%% the commands and values with those provided to you when you

\usepackage{graphicx, url}
%\usepackage[noadjust]{cite}
\usepackage{balance}
\usepackage{subfigure,captcont,wrapfig, setspace}
\usepackage{comment, ragged2e}
\usepackage{subfigure, color, colortbl}
\usepackage{amsmath}
\usepackage[linesnumbered, ruled, vlined]{algorithm2e}


%% complete the rights form.
\setcopyright{acmcopyright}
\copyrightyear{2020}
\acmYear{2020}
\acmDOI{0000001.0000001}


%%
%% These commands are for a JOURNAL article.
\acmJournal{TECS}
\acmVolume{0}
\acmNumber{0}
\acmArticle{0}
\acmMonth{8}

\definecolor{blue}{rgb}{0.0,0.0,1}
\newcommand*{\revise}{\textcolor{blue}}

\definecolor{red}{rgb}{1,0.0,0.0}
\newcommand*{\edit}{\textcolor{red}}
\newcommand{\code}[1]{\texttt{#1}}

%%
%% Submission ID.
%% Use this when submitting an article to a sponsored event. You'll
%% receive a unique submission ID from the organizers
%% of the event, and this ID should be used as the parameter to this command.
%%\acmSubmissionID{123-A56-BU3}

%%
%% The majority of ACM publications use numbered citations and
%% references.  The command \citestyle{authoryear} switches to the
%% "author year" style.
%%
%% If you are preparing content for an event
%% sponsored by ACM SIGGRAPH, you must use the "author year" style of
%% citations and references.
%% Uncommenting
%% the next command will enable that style.
%%\citestyle{acmauthoryear}

%%
%% end of the preamble, start of the body of the document source.
\begin{document}

%%
%% The "title" command has an optional parameter,
%% allowing the author to define a "short title" to be used in page headers.
\title{LPWAN in the TV White Spaces: A Practical Implementation and Deployment Experiences}

%%
%% The "author" command and its associated commands are used to define
%% the authors and their affiliations.
%% Of note is the shared affiliation of the first two authors, and the
%% "authornote" and "authornotemark" commands
%% used to denote shared contribution to the research.
\author{Mahbubur Rahman}
\orcid{0000-0003-2353-6687}
\affiliation{%
  \institution{City University of New York}
  \streetaddress{Queens College Science Building, 65-30 Kissena Boulevard}
  \city{Flushing}
  \state{NY}
  \postcode{11367}
  \country{USA}}
\email{mdmahbubur.rahman@qc.cuny.edu}

\author{Dali Ismail}
\affiliation{%
 \institution{Wayne State University}
  \streetaddress{5057 Woodward Avenue, Suite 3010}
  \city{Detroit}
  \state{MI}
  \postcode{48202}
  \country{USA}}
\email{dali.ismail@wayne.edu}

\author{Venkata P. Modekurthy}
\affiliation{%
 \institution{Wayne State University}
  \streetaddress{5057 Woodward Avenue, Suite 3010}
  \city{Detroit}
  \state{MI}
  \postcode{48202}
  \country{USA}}
\email{modekurthy@wayne.edu}

\author{Abusayeed Saifullah}
\affiliation{%
 \institution{Wayne State University}
  \streetaddress{5057 Woodward Avenue, Suite 3010}
  \city{Detroit}
  \state{MI}
  \postcode{48202}
  \country{USA}}
\email{saifullah@wayne.edu}

%%
%% By default, the full list of authors will be used in the page
%% headers. Often, this list is too long, and will overlap
%% other information printed in the page headers. This command allows
%% the author to define a more concise list
%% of authors' names for this purpose.
\renewcommand{\shortauthors}{Trovato and Tobin, et al.}

%%
%% The abstract is a short summary of the work to be presented in the
%% article.
\begin{abstract}
Low-Power Wide-Area Network (LPWAN) is an enabling Internet-of-Things (IoT) technology that supports long-range, low-power, and low-cost connectivity to numerous devices. To avoid the crowd in the limited ISM band (where most LPWANs operate) and cost of licensed band, the recently proposed SNOW (Sensor Network over White Spaces) is a promising LPWAN platform that operates over the TV white spaces. As it is a very recent technology and is still in its infancy, the current SNOW implementation uses the USRP devices as LPWAN nodes, which has high costs ($\approx$ \$750 USD per device) and large form-factors, hindering its applicability in practical deployment. In this paper, we implement SNOW using low-cost, low form-factor, low-power, and widely available commercial off-the-shelf (COTS) devices to enable its practical and large-scale deployment. Our choice of the COTS device (TI CC13x0: CC1310 or CC1350) consequently brings down the cost and form-factor of a SNOW node by 25x and 10x, respectively. Such implementation of SNOW on the CC13x0 devices, however, faces a number of challenges to enable link reliability and communication range. Our implementation addresses these challenges by handling peak-to-average power ratio problem, channel state information estimation, carrier frequency offset estimation, and near-far power problem. Our deployment in the city of Detroit, Michigan demonstrates that CC13x0-based SNOW can achieve uplink and downlink throughputs of 11.2kbps and 4.8kbps per node, respectively, over a distance of 1km. Also, the overall throughput in the uplink increases linearly with the increase in the number of SNOW nodes.
\end{abstract}

%%
%% The code below is generated by the tool at http://dl.acm.org/ccs.cfm.
%% Please copy and paste the code instead of the example below.
%%
\begin{CCSXML}
<ccs2012>
<concept>
<concept_id>10003033.10003034</concept_id>
<concept_desc>Networks~Network architectures</concept_desc>
<concept_significance>500</concept_significance>
</concept>
<concept>
<concept_id>10003033.10003079</concept_id>
<concept_desc>Networks~Network performance evaluation</concept_desc>
<concept_significance>500</concept_significance>
</concept>
<concept>
<concept_id>10003033.10003083.10003097</concept_id>
<concept_desc>Networks~Network mobility</concept_desc>
<concept_significance>300</concept_significance>
</concept>
<concept>
<concept_id>10003033.10003039.10003040</concept_id>
<concept_desc>Networks~Network protocol design</concept_desc>
<concept_significance>100</concept_significance>
</concept>
<concept>
<concept_id>10010520.10010553.10003238</concept_id>
<concept_desc>Computer systems organization~Sensor networks</concept_desc>
<concept_significance>300</concept_significance>
</concept>
</ccs2012>
\end{CCSXML}

\ccsdesc[500]{Networks~Network architectures}
\ccsdesc[500]{Networks~Network performance evaluation}
\ccsdesc[300]{Networks~Network mobility}
\ccsdesc[100]{Networks~Network protocol design}
\ccsdesc[300]{Computer systems organization~Sensor networks}


\keywords{LPWAN, SNOW, White spaces, OFDM}

%%
%% This command processes the author and affiliation and title
%% information and builds the first part of the formatted document.
\maketitle
\renewcommand{\shortauthors}{Rahman, M. et al.}

\section{Introduction}  \label{sec:introduction}

\newcommand\inexpIntro[3]{#1?(#2,#3).}
\newcommand\rinexpIntro[3]{*#1?(#2,#3).}
\newcommand\outexpIntro[3]{#1!(#2,#3).}
\newcommand\outatomIntro[3]{#1!(#2,#3)}

We propose a fully automated method for proving termination of \(\pi\)-calculus processes.
Although there have been a lot of studies on termination analysis for the \(\pi\)-calculus
and related calculi~\cite{Deng06IC,Demangeon07,SangiorgiTermination,KobayashiHybrid,Yoshida04IC,DBLP:journals/jlp/DemangeonHS10,Venet98SAS}, most of them have been rather theoretical,
and there have been surprisingly little efforts in developing  fully automated termination
verification methods and tools based on them. To our knowledge,
Kobayashi's \typical{}~\cite{TyPiCal,KobayashiHybrid} is the only exception that
can prove termination of \(\pi\)-calculus processes (extended with natural numbers)
fully automatically, but its termination analysis is quite limited (see Section~\ref{sec:relatedwork}).

Our method is based on a reduction to termination analysis for sequential programs:
we translate a \(\pi\)-calculus process \(P\) to a sequential program \(S_P\), so that
if \(S_P\) is terminating, so is \(P\). The reduction allows us to use
powerful, mature methods and tools
for termination analysis of sequential programs~\cite{heizmann2016ultimate,freqterm,DBLP:conf/lics/PodelskiR04,Kuwahara2014Termination,DBLP:journals/cacm/CookPR11}.

The idea of the translation is to convert a chain of communications on replicated input
channels to a chain of recursive function calls of the target sequential program.
Let us consider the following Fibonacci process:
\begin{align*}
    & \rinexpIntro{\fib}{n}{r}
        \ifexp{n<2}{ \soutatom{r}{1} \\ &\quad}
                   { \nuexp{s_1} \nuexp{s_2} (\outatomIntro{\fib}{n-1}{s_1} \PAR \outatomIntro{\fib}{n-2}{s_2} \PAR \sinexp{s_1}{x}\sinexp{s_2}{y}\soutatom{r}{x+y}) \\}
    & \PAR \outatomIntro{\fib}{m}{r}
\end{align*}
Here, the process
$\rinexpIntro{\fib}{n}{r} \ldots$ is a function server that computes the \(n\)-th Fibonacci number
in parallel and returns the result to \(r\),
and $\outatom{\fib}{m}{r}$ sends a request for computing the \(m\)-th Fibonacci number;
those who are not familiar with the syntax of the \(\pi\)-calculus may wish to consult
Section~\ref{sec:targetlanguage} first.
To prove that the process above is terminating for any integer \(m\),
it suffices to show that there is no infinite chain of communications on $\fib$:
\[
    \fib(m,r) \to \fib(m_1,r_1) \to \fib(m_2,r_2) \to \cdots.
\]
We convert the process above to the following program:\footnote{The actual translation
  given later is a little more complex.}
\begin{verbatim}
 let rec fib(n) = if n<2 then () else (fib(n-1) [] fib(n-2)) in
 fib(m)
\end{verbatim}
Here, \texttt{[]} represents the non-deterministic choice.
Note that, although the calculation of Fibonacci numbers is not preserved,
for each chain of communications on \texttt{fib}, there is a corresponding
sequence of recursive calls:
\[
\mathtt{fib}(m) \to \mathtt{fib}(m_1) \to \mathtt{fib}(m_2) \to \cdots.
\]
Thus, the termination of the sequential program above implies the termination of
the original process.
As shown in the example above, (i) each communication on a replicated input channel
is converted to a function call, (ii) each communication on a non-replicated input
channel is just removed (or, in the actual translation, replaced by a call of
a trivial function defined by \(f(\seq{x})=(\,)\)), and (iii) parallel composition
is replaced by a non-deterministic choice.
We formalize the translation outlined above and prove its correctness.

The basic translation sketched above sometimes loses too much information.
For example, consider the following process:
\begin{align*}
    & \rinexpIntro{\pre}{n}{r} \soutatom{r}{n-1} \\
    & \PAR \rinexpIntro{f}{n}{r} \ifexp{n<0}{ \soutatom{r}{1} }
                                       { \nuexp{s} (\outatomIntro{\pre}{n}{s} \PAR \sinexp{s}{x}\outatomIntro{f}{x}{r}) } \\
    & \PAR \outatomIntro{f}{m}{r}
\end{align*}
The translation sketched above would yield:
\begin{verbatim}
  let pred(n) = n-1 in
  let rec f(n) = if n<0 then () else (pred(n) [] f(*)) in
  f(m)
\end{verbatim}
Here, \texttt{*} represents a non-deterministic integer: since we have removed
the input $\sinatom{s}{x}$, we do not have information about the value of \( x \).
As a result, the sequential program above is non-terminating, although the original
process is terminating.
To remedy this problem, we also refine the basic translation above by using a refinement
type system for the \(\pi\)-calculus. Using the refinement type system,
we can infer that the value of \(x\) in the original process is less than \(n\),
so that we can refine the definition of \texttt{f} to:
\begin{verbatim}
 let rec f(n) = ... else (pred(n) [] let x=* in assume(x<n);f(x))
\end{verbatim}
The target program is now terminating, from which
we can deduce that the original process is also terminating.
We have implemented an automated tool based on the refined translation above.

The contributions of this paper are summarized as follows.
\begin{itemize}
\item The formalization of the basic translation from the \(\pi\)-calculus
  (extended with integers) to sequential programs, and a proof of its correctness.
\item The formalization of a refined translation based on a refinement type system.
\item An implementation of the refined translation, including automated refinement type
  inference based on CHC solving, and experiments to evaluate the effectiveness of
  our method.
\end{itemize}

The rest of this paper is structured as follows.
Section~\ref{sec:targetlanguage} introduces the source and target languages
of our translation.
Section~\ref{sec:approach} 
formalizes the basic translation, and proves its correctness.
Section~\ref{sec:refinement} refines the basic translation by using a refinement type system.
Section~\ref{sec:implementation} reports an implementation and experiments.
Section~\ref{sec:relatedwork} discusses related work,
and Section~\ref{sec:conclusion} concludes the paper.

\section{Background and System Model}\label{sec:model}
\subsection{An Overview of SNOW}\label{sec:snow_overview}
\begin{figure*}[!htbp]
    \centering
      \subfigure[Network Architecture\label{fig:network}]{
    \includegraphics[width=0.4\textwidth]{figs/snow-arch.eps}
      }\hfill
      \subfigure[Dual-radio BS and subcarriers\label{fig:dualradio}]{
        \includegraphics[width=.4\textwidth]{figs/dual-radio.eps}
      }%\vspace{-0.05in}
    \caption{The SNOW architecture~\cite{snow2}.}
    \label{fig:arch}
\end{figure*}
% \begin{figure}[!htpb]
% \centering
% \includegraphics[width=0.5\textwidth]{figs/dual-radio.eps}
% \caption{Dual-radio BS and subcarriers~\cite{snow_ton}.}
% \label{fig:dualradio}
% \end{figure}
In this section, we provide a brief overview of SNOW. Its complete design and description is available in~\cite{mydissertation}. SNOW is a highly scalable LPWAN technology operating in the TV white spaces. It supports asynchronous, reliable, bi-directional, and concurrent communication between a BS and numerous nodes. Due to its long-range, SNOW forms a star topology allowing the BS and the nodes to communicate directly, as shown in Figure~\ref{fig:network}. The BS is powerful, Internet-connected, and line-powered while the nodes are power-constrained and do not have access to the Internet. To determine white space availability in a particular area, the BS queries a cloud-hosted geo-location database via the Internet. A node depends on the BS to learn its white space availability. In SNOW, all the complexities are offloaded to the BS to make the node design simple. 
Each node is equipped with a single half-duplex radio. To support simultaneous uplink and downlink communications, the BS uses a dual-radio architecture for reception (Rx) and transmission (Tx), as shown in Figure~\ref{fig:dualradio}.

%\subsubsection{Physical Layer}
The SNOW PHY uses a distributed implementation of OFDM called {\em D-OFDM}. D-OFDM enables the BS to receive concurrent transmissions from {\em asynchronous} nodes using a single-antenna radio (Rx-radio). Also, using a single-antenna radio (Tx-Radio), the BS can transmit different data to different nodes concurrently~\cite{snow, snow2, snow_ton, snow3, isnow_ton, isnow_p2p}. Note that the SNOW PHY is different from MIMO radio design adopted in other wireless domains such as LTE, WiMAX, and 802.11n~\cite{snow2} as the latter use multiple antennas to enable multiple transmissions and receptions.
The BS operates on a wideband channel split into orthogonal narrowband channels/subcarriers (Figure~\ref{fig:dualradio}). Each node is assigned a single subcarrier. 
For encoding and decoding, the BS runs inverse fast Fourier transform (IFFT) and global fast Fourier transform (G-FFT) over the entire wideband channel, respectively.
When the number of nodes is no greater than the number of subcarriers, every node is assigned a unique subcarrier. Otherwise, a subcarrier is shared by more than one node. 
%Narrower bands offer more extended communication range and consume less energy making it practical for IoT applications.
SNOW supports ASK and BPSK modulation techniques, supporting different bitrates. 
%Additionally, SNOW is capable of exploiting fragmented white space spectrum.


%\subsubsection{Medium Access Control Layer}
%The nodes in SNOW use a lightweight carrier sense multiple access/collision avoidance (CSMA/CA) medium access control (MAC) protocol similar to TinyOS~\cite{tinyos}. Additionally, the nodes can autonomously transmit, remain in receive mode, or sleep. A node runs clear channel assessment (CCA) before transmitting. If its subcarrier is occupied, the node makes a random back-off in a fixed congestion back-off window. After this back-off expires, the node transmits immediately if its subcarrier is free. This operation is repeated until the node makes this transmission and gets the acknowledgment (ACK).

%The nodes in SNOW use a lightweight CSMA/CA (carrier sense multiple access with collision avoidance)-based  media access control (MAC) protocol similar to TinyOS~\cite{tinyos}. Additionally, the nodes can autonomously transmit, remain in receive mode, or sleep. A node runs clear channel assessment (CCA) before transmitting. If its subcarrier is occupied, the node makes a random back-off in a fixed congestion back-off window. After this back-off expires, the node transmits immediately if its subcarrier is free and repeats this operation until it sends the packet and gets an acknowledgment (ACK).

The nodes in SNOW use a lightweight CSMA/CA (carrier sense multiple access with collision avoidance)-based media access control (MAC) protocol similar to TinyOS~\cite{tinyos}. The nodes can autonomously transmit, remain in receive mode, or sleep. Specifically, when a node has data to send, it wakes up by turning its radio on. Then it performs a random back-off in a fixed initial back-off window. When the back-off timer expires, it runs CCA (Clear Channel Assessment). If the subcarrier is clear, it transmits the data. If the subcarrier is occupied, then the node makes a random back-off in a fixed congestion back-off window. After this back-off expires, if the subcarrier is clean the node transmits immediately. This process is repeated until it makes the transmission and gets an acknowledgment (ACK).


\subsection{An Overview of TI CC13x0 LaunchPads}
Texas Instruments introduced TI CC1310 and TI CC1350 LaunchPads as a part of the SimpleLink microcontroller (MCU) platform to support ultra-low-power and long-range communication having a Cortex-M3 processor with 8KB SRAM and up to 128KB of in-system programmable storage~\cite{cc1350, snow_cots}.
With a small form-factor (length: 8cm, width: 4cm), both CC1310 and CC1350 are designed to operate in the lower frequency bands (287--351MHz, 359--527MHz, and 718--1054MHz) including the TV band. As an added feature, CC1350 can also operate in the 2.4GHz band while using Bluetooth low energy radio.
The packet structure of the CC13x0 devices includes a \code{preamble}, followed by \code{sync word}, \code{payload length}, \code{payload}, and \code{CRC}, chronologically. They support different data modulation techniques including Frequency Shift Keying (FSK), Gaussian FSK (GFSK), On-Off Keying (OOK), and a proprietary long-range modulation. They are capable of using a Tx/Rx bandwidth that ranges between 39 and 3767kHz. 
Additionally, with a supply voltage in the range of 1.8 to 3.8 volts, their Rx and Tx current consumption is 5.4mA and 13.4mA at +10dBm, respectively, offering ultra-low-power communication. 
The greatest advantage is that  
they have a programmable and reconfigurable physical layer, offering flexibility and feasibility for customized protocol implementation.











%In 2.2 provide a brief introduction of TI CC1310 (physical description, band, energy, and advantages) so that reader know both SNOW and 1310 before the technical section starts. 

% The physical (PHY) layer and the media access control (MAC) layer of SNOW is designed in~\cite{snow, snow2, snow_ton}. SNOW is an LPWAN platform to operate over TV white spaces. A SNOW node has a single half-duplex narrowband radio. Due to the long transmission (Tx) range, the nodes are directly connected to the BS and vice versa (Figure~\ref{fig:network}). SNOW thus forms a star topology.    
% The BS determines white spaces in the area by accessing a cloud-hosted database  through the Internet. 
% The nodes  are power constrained and not directly connected to the Internet.  They do not do spectrum sensing or cloud access.
% The BS uses a wide channel split into orthogonal subcarriers.   As shown in Figure~\ref{fig:dualradio}, the BS uses two radios, both operating on the same spectrum --  one for only transmission (called {\bf\slshape Tx radio}),  and the other for only reception (called {\bf\slshape Rx radio}). Such a dual-radio of the BS allows concurrent bi-directional communication in SNOW.

% The PHY layer of SNOW uses a {\bf D}istributed implementation of {\bf OFDM} (Orthogonal Frequency Division Multiplexing) for multi-user access, called {\bf D-OFDM}~\cite{snow, snow2}. In SNOW, the BS's wide white space spectrum is split into narrowband orthogonal subcarriers which carry parallel data streams to/from the distributed nodes from/to the BS as D-OFDM. A subcarrier bandwidth can be chosen as low as 100kHz, 200kHz, 400kHz depending on the packet and expected bit rate.  Narrower bands have lower bit rate but longer range, and consume less power~\cite{channelwidth}. Thus, SNOW adopted D-OFDM by assigning the orthogonal subcarriers to different nodes. Each node transmits and receives on the assigned subcarrier. Each subcarrier can be modulated using Amplitude Shift Keying (ASK) or Binary Phase Shift Keying (BPSK). 
% D-OFDM enables multiple receptions using a single antenna and also enables different data transmissions to different nodes using a single antenna. SNOW is a fully asynchronous network, requiring no time synchronization among the nodes for D-OFDM. The SNOW BS can  also exploit fragmented spectrum. If the BS spectrum is split into  $n$  subcarriers, then it can receive from $n$ nodes simultaneously. Similarly, it can transmit $n$ different data at a time. 

% \subsection{SNOW MAC Layer}

% This BS spectrum  is split into  $n$ overlapping orthogonal subcarriers -- $f_1, f_2, \cdots, f_n$ -- each of equal width. Each node is assigned one subcarrier. When the number of nodes is no greater than the number of subcarriers, every node is assigned a unique subcarrier.  Otherwise, a subcarrier is shared by more than one node. 
% The nodes that share the same subcarrier will contend for and access it using a CSMA/CA (Carrier Sense Multiple Access with Collision Avoidance) policy.

% The subcarrier allocation is done by the BS.
% The nodes in SNOW use a lightweight CSMA/CA protocol for transmission that uses a static interval for random back-off like the one used in TinyOS~\cite{tinyos} . 
% Specifically, when a node has data to send, it wakes up by turning its radio on. Then it performs a random back-off in a fixed {\slshape initial back-off window}.  When the back-off timer expires, it runs CCA (Clear Channel Assessment) and if the subcarrier is clear, it transmits the data. If the subcarrier is occupied, then the node makes a random back-off in a fixed {\slshape congestion back-off window}. After this back-off expires, if the subcarrier is clean the node transmits immediately. This process is repeated until it makes the transmission. The node then can go to sleep again.

% The nodes can autonomously transmit, remain in receive (Rx) mode, or sleep. 
% Since D-OFDM allows handling asynchronous Tx and 
% Rx, the link layer can send acknowledgment (ACK) for any transmission in either direction.   As shown in Figure~\ref{fig:dualradio}, both radios of the BS use the same spectrum and subcarriers - the subcarriers in the Rx radio are for receiving while those in the Tx radio are for transmitting. Since each node (non BS) has just a single half-duplex radio, it can be either receiving or transmitting, but not doing both at the same time.  
% Both experiments and large-scale simulations show high efficiency  of SNOW in latency and energy with a linear increase in throughput with the number of nodes, demonstrating its superiority  over existing LPWAN designs~\cite{snow, snow2}.








\section{SNOW Implementation on TI CC13x0}\label{sec:implementation}
\begin{figure}[!htb]
\centering
\includegraphics[width=0.49\textwidth]{figs/devices-new.eps}
\caption{Devices used in our SNOW implementation. A node is a CC1310 or CC1350 device. The BS has two USRP B200s, each having its own antenna. An antenna is approximately 2x bigger than a B200.}
\label{fig:devices}
\end{figure}
%talk about what devices are used in BS and as nodes
The original SNOW implementation in~\cite{snow_ton} uses the USRP hardware platform for both the BS and the nodes. In our implementation, we use the CC13x0 devices as SNOW nodes and USRP in the BS (Figure~\ref{fig:devices}).
%A USRP B200 device with a half-duplex radio costs approximately \$750 USD, as of today. As such, it becomes costly to deploy the SNOW network and examine its scalability. In contrast, in this work, we realize the functionality of a SNOW node in TI CC1310 LaunchPad~\cite{cc1310} that costs approximately \$30 USD, thus much cheaper and widely available to the research community to develop and deploy SNOW networks. 
For BS implementation, we adopt the open-source code provided in~\cite{snow_bs}. The BS uses two half-duplex USRP devices (Rx-Radio and Tx-Radio), each having its own antenna. Also. the BS is implemented on the GNURadio software platform that gives a high magnitude of freedom to perform baseband signal processing~\cite{gnuradio}.
In the following, we explore a number of implementation considerations and feasibility for a CC13x0 device to work as a SNOW node in practical deployments. 
First, we show how to configure a CC13x0 device to make it work as a SNOW node. We then address the practical challenges (e.g., PAPR problem, CSI estimation, and CFO estimation) associated with our CC13x0-based SNOW implementation.

\subsection{Configuring TI CC13x0}
%talk about how to configure nodes and BS
We configure the subcarrier center frequency, bandwidth, modulation, and the Tx power by setting appropriate values to the CC13x0 command inputs \code{centerFreq, rxBw, modulation}, and \code{txPower}, respectively, using {\em Code Composer Studio} (CCS) provided by Texas Instruments~\cite{snow_cots}. A graphical user interface alternative to CCS is {\em SmartRF Studio}. The MAC protocol of SNOW in CC13x0 is implemented on top of the example CSMA/CA project that comes with CCS. Note that the functionalities of a SNOW node are very simple and may be incorporated easily in the IoT devices that have both storage and computational limitations like the CC13x0 devices.

\subsection{Peak-to-Average Power Ratio Observation}\label{sec:papr}
By transmitting on a large number of subcarriers simultaneously (in the downlink), the BS suffers from a traditional OFDM problem called {\em peak-to-average power ratio (PAPR)}. PAPR of an OFDM signal is defined as the ratio of the maximum instantaneous power to its average power.
In the SNOW downlink communications (i.e., BS to nodes), after the IFFT is performed by the BS, the composite signal can be represented as
$\nonumber x(t) = \frac{1}{\sqrt{N}}\sum_{k=0}^{N-1}X_k~e^{j2 \pi f_k t},~~0 \le t \le NT.$
% \begin{equation}
% \nonumber x(t) = \frac{1}{\sqrt{N}}\sum_{k=0}^{N-1}X_k~e^{j2 \pi f_k t},~~0 \le t \le NT
% \end{equation} 
Here, $X_k$ is the modulated data symbol for node $k = \{0, 1, \cdots, N-1\}$ on subcarrier center frequency $f_k = k\Delta f$, where $\Delta f = \frac{1}{NT}$ and $T$ is the symbol period. Therefore, the PAPR may be calculated as%~\cite{jiang2008overview}
\begin{equation}
\nonumber \text{PAPR}[x(t)] = 10\log_{10}\Bigg( \frac{\max\limits_{0~ \le ~t~ \le~ NT} [|x(t)|^2 ]}{P_{\text{avg}}}\Bigg)~~dB.
\end{equation}
Here, the average power $P_{\text{avg}} = E [|x(t)|^2]$.
A node's signal detection on its subcarrier is very sensitive to the nonlinear signal processing components used in the BS, i.e., the digital-to-analog converter (DAC) and high power amplifier (HPA), which may severely impair the bit error rate (BER) in the nodes due to the induced spectral regrowth. If the HPA does not operate in the linear region with a large power back-off due to high PAPR, the out-of-band power will exceed the specified limit and introduce severe ICI~\cite{jiang2008overview}. Moreover, the in-band distortion (constellation tilting and scattering) due to high PAPR may cause severe performance degradation~\cite{kamali2012understanding}. It has been shown that the PAPR reduction results in significant power saving at the transmitters~\cite{baxley2004power}.
\begin{figure}[!htb]
\centering
\includegraphics[width=0.35\textwidth]{figs/papr/papr.eps}
\caption{PAPR distribution of D-OFDM signal in Tx-Radio.}
\label{fig:papr}
\end{figure}


As shown in Figure~\ref{fig:papr}, the PAPR in the SNOW downlink communications (for N = 64) follows the Gaussian distribution. Thus, the peak signal occurs quite rarely and the transmitted D-OFDM signal will cause the HPA to operate in the nonlinear region, resulting in a very inefficient amplification. To illustrate the power efficiency of the HPA for N = 64, let us assume the probability of the clipped D-OFDM frames is less than 0.01\%. We thus need to apply an input back-off (IBO)~\cite{baxley2004power} equivalent to the PAPR at a probability of $10^{-4}$. Here, PAPR $\approx$ 14dB or 25.12. Thus, the efficiency ($\eta = 0.5/\text{PAPR}$) of the HPA~\cite{jiang2008overview} is $\eta = 0.5/25.12 \approx 1.99\%$. Such low efficiency at the HPA motivates us to explore the high PAPR in SNOW for practical deployments.
%show some results explaining how PARP affects the BS-Node communication
Several uplink PAPR reduction techniques for single-user OFDM systems have been proposed (see survey~\cite{jiang2008overview}). However, the characteristics of the downlink PAPR in SNOW, where different data are concurrently transmitted to different nodes, are entirely different from the PAPR observed in a single-user OFDM system. To adopt an uplink PAPR reduction technique used in the single-user OFDM systems for the downlink PAPR reduction in SNOW, each node has to process the entire data frame transmitted by the BS and then demodulate its own data. However, a SNOW node has less computational power and does not apply FFT to decode its data~\cite{snow_ton}, or any other node's data. Thus, the existing PAPR reduction techniques will not work in our implementation.

%To this extent, we address the PAPR problem in SNOW by allocating a special subcarrier  called {\em downlink subcarrier} for the downlink communications.
We propose to handle the PAPR problem in SNOW by using only one subcarrier (called {\em downlink subcarrier}) for downlink communication. All the nodes use this subcarrier to receive from the BS. Namely, the Tx-Radio transmits only on one subcarrier that is not used by any node for uplink communication.
The BS may send any broadcast message, ACK, or data to the nodes using that downlink subcarrier. A node has to switch to the downlink subcarrier to listen to any broadcast message, ACK, or data.
The BS may reserve multiple subcarriers  as {\slshape backup subcarriers} for downlink communication. 
If the currently used downlink subcarrier becomes overly noisy or unreliable, it can be replaced by a backup subcarrier.
Note that the dual-radio in the BS allows it to receive concurrent packets from a set of nodes (uplink) and transmit broadcast/ACK/data packets to another set of nodes (downlink), simultaneously. 
The BS can acknowledge several nodes using a single transmission by using a bit-vector of size equals to the number of subcarriers.
If the BS receives a packet from a node operating on subcarrier $i$, it will set the $i$-th bit in the bit-vector. Upon receiving the bit-vector, that node may get an ACK by looking at the $i$-th bit of the vector. Because of the bit-vector, the downlink ACKs also scale up like the uplink traffic. In the case of different packets for different nodes, the volume of downlink traffic (compared to the uplink traffic) is also practical since the IoT applications may not require high volume downlink traffic~\cite{whitespaceSurvey}.

%A node retransmits the packet if that packet is not acknowledged in the first ACK received by that node. 
%In the following, we describe our technique below to handle a {\bf rare} case in practical SNOW deployments, and hence may be kept optional in implementation.

%\revise{Let nodes $A$ and $B$ share subcarrier $i$. The BS may receive a packet from node $B$ while preparing the ACK for node $A$'s packet. If both packets are decoded correctly, the BS acknowledges them by setting the $i$-th bit of the vector. However, if only one packet is decoded correctly, the BS resets the $i$-th bit of the vector. Thus, none of the packets are acknowledged. To compensate for this, the BS (Tx-Radio) switches to subcarrier $i$ and sends separate ACKs for nodes $A$ and $B$. On the other hand, if a node finds that its packet is not acknowledged in the downlink subcarrier, it listens to its subcarrier for a short fixed window before attempting a retransmission. A node knows about that fixed window when it joins the network. Note that {\em a very few} nodes (sharing the same subcarrier) may be involved in this scenario since the ACK generation time at the BS is very small. Other ways of addressing this issue may include the use of \emph{hash functions}, which we do not consider due to the scalability issues in hash-related collisions.}

When a node $u$ transmits to the BS, if another node $v$ sharing the same subcarrier wants to transmit, $v$ senses the channel as busy and refrain from transmitting. When the BS transmits ACK to $u$ on the downlink subcarrier using the Tx-Radio, node $v$ may also transmit to the BS. Since the Tx-Radio at that time is making a downlink transmission, it may not send the ACK upon $v$'s transmission immediately. However, the Tx-Radio can send $v$'s ACK immediately after completing its current downlink transmission. Thus, $v$ may need to wait for ACK for a little longer than the time needed to send a downlink transmission from the BS. A node may go to sleep mode or its next state right after receiving an ACK. However, if a node that has transmitted but not yet received ACK, should wait for a little longer (e.g., up to one or two downlink transmission time). Note that a very few nodes (sharing the same subcarrier) may be involved in this scenario since the ACK generation time at the BS is very small. For the same reason, the waiting time for ACK will also not be very long (e.g., up to one or two downlink transmission time). Note that this scenario is quite rare and most of the times the nodes will receive ACK immediately upon transmission.


%When a subcarrier (say, $i$) is shared by multiple nodes, the BS may receive a packet (say, from node A) before transmitting the ACK for another packet (say, from node B). In this case, both nodes A and B may be acknowledged by setting the $i$-th bit of the vector. However, if the packet from node A (or, B) is valid and the packet from node B (or, A) is invalid, the BS will reset the $i$-th bit of the vector and transmit the ACK. Thus, none of the packets are acknowledged even if one of them is valid. To compensate for that, the BS (Tx-Radio) will switch to node A's (or, B's) subcarrier and transmit an ACK packet. Thus, in our implementation, if a node finds that its packet is not acknowledged in the first valid ACK it received, before retransmission it listens to its subcarrier for a fixed amount of time. Each node may know this fixed time when it joins the network. Typically, if a subcarrier is shared by $G$ nodes, the fixed amount of time (worst case) may be set to $GD_p$ (ignoring the frequency switching time in the Tx-Radio), where $D_p$ is the time to transmit one packet. Other ways of addressing such issue may include the use of \emph{hash functions}. However, we do not explore that in our implementation for scalability issue due to hash collision.

%In the case where each subcarrier is assigned to only a node, the size of the bit-vector may be set to the total number of subcarriers. Thus, if the BS receives a packet from a node operating on subcarrier $i$, it will set the $i$-th bit in the bit-vector. Upon receiving the bit-vector in ACK subcarrier, that node can check the $i$-th bit of the vector. However, in practical deployments with thousands of nodes, a subcarrier may be shared by multiple nodes, making the creation of the bit-vector non-trivial. We cannot also use any hashing technique because of the hash collisions and scalability issues. 

\begin{figure*}[!htbp] 
    \centering
      \subfigure[RSSI under varying distance\label{fig:csi_rssi}]{
    \includegraphics[width=0.35\textwidth]{figs/csi/rssi.eps}
      }\hfill
      \subfigure[Path Loss under varying distance\label{fig:csi_pathloss}]{
        \includegraphics[width=.35\textwidth]{figs/csi/pathloss.eps}
      }\hfill
      \subfigure[BER under varying distance\label{fig:csi_ber}]{
        \includegraphics[width=.35\textwidth]{figs/csi/ber.eps}
      }
    \caption{RSSI, path loss, and BER at the SNOW BS for a TI CC1310 node.}
    \label{fig:csi}
 \end{figure*}
\subsection{Channel State Information Estimation}\label{sec:csi}

Multi-user OFDM communication requires channel estimation and tracking to ensure high data rate at the BS. One way to avoid channel estimation is to use the \emph{differential phase-shift keying (DPSK)} modulation. DPSK, however, results in a lower bitrate at the BS due to a 3dB loss in the signal-to-noise ratio (SNR)~\cite{van1995channel}. Additionally, the current SNOW design does not support DPSK modulation. SNR at the BS for each node is different in SNOW. Also, SNR of each node is affected differently due to channel conditions, deteriorating the overall bitrate in the uplinks. Thus, it requires handling of the channel estimation in SNOW.

Figure~\ref{fig:csi} shows the experimentally found received signal strength indicator (RSSI), path loss, and BER at the SNOW BS for a CC1310 device that transmits successive 1000 30-byte (payload) packets from 200 to 1000m distances, respectively, with a Tx power of 15dBm, subcarrier center frequency at 500MHz, and a bandwidth of 98kHz. Figure~\ref{fig:csi_rssi} indicates that the RSSI decreases rapidly with the increase in distance. Also, the path loss in Figure~\ref{fig:csi_pathloss} shows that it is significantly higher than the theoretical free space loss~\cite{rappaport1996wireless}. We also compare with the Okumura-Hata~\cite{rappaport1996wireless} loss to check if it fits the model, however, it does not. Finally, Figure~\ref{fig:csi_ber} confirms that the BER goes above $10^{-3}$ (which is not acceptable~\cite{rnr}) beyond 400m due to the unknown channel conditions. Figure~\ref{fig:csi_ber} also shows that the BER worsens for an increase in the subcarrier bandwidth. Thus, to make our implementation more resilient, we need to incorporate the CSI estimation in SNOW.

We calculate the CSI for each SNOW node independently on its subcarrier. We consider a slow flat-fading model~\cite{tse2005fundamentals}, where the channel conditions vary slowly with respect to a single node to BS packet duration. Note that joint-CSI estimation~\cite{jiang2007iterative, ribeiro2008uplink} in SNOW is not our design goal since it would require SNOW nodes to be strongly time-synchronized.  
Similar to IEEE 802.16e, we run CSI estimation independently for each node because of their different fading and noise characteristics. In the following, we explain the CSI estimation technique for one node on its subcarrier for each packet. The BS uses the same technique to estimate CSI for all other nodes. 

For a node, in a narrowband flat-fading subcarrier, the system is modeled as $y = Hx + w$,
% \begin{equation}
% \nonumber y = Hx + w,
% \end{equation}
where $y$, $x$, and $w$ are the receive vector, transmit vector, and noise vector, respectively. $H$ is the channel matrix. 
We model the noise as additive white Gaussian noise, i.e., a circular symmetric complex normal ($CN$) with $w \sim CN(0, W)$, where the mean is zero and noise covariance matrix $W$ is known.
%Noise is modeled as circular symmetric complex normal ($CN$) with $w \sim CN(0, W)$, where the mean is zero and noise covariance matrix $W$ is known, thus an additive white Gaussian noise. 
As the subcarrier conditions vary, we estimate the CSI on a short-term basis based on popular approach called {\em training sequence}. We use the known preamble transmitted at the beginning of each packet. $H$ is estimated using the combined knowledge of the received and the transmitted preambles. To make the estimation robust, we divide the preamble into $n$ equal parts (preamble sequence). E.g., n = 4, which is similar to the estimation in IEEE 802.11.

Let the preamble sequence be $(p_1, p_2, \cdots, p_n)$, where vector $p_i$ is transmitted as $y_i = Hp_i + w_i$.
% \begin{equation}
% \nonumber y_i = Hp_i + w_i.
% \end{equation}
Combining the received preamble sequences, we get $Y = [y_1, \cdots, y_n] = HP + W$, where 
% \begin{equation}
% \nonumber Y = [y_1, \cdots, y_n] = HP + W.
% \end{equation}
$P = [p_1, \cdots, p_n]$ and $W = [w_1, \cdots, w_n]$. With combined knowledge of $Y$ and $P$, channel matrix $H$ is estimated. Similar to the CSI estimation in the uplink communications by the BS, each node also estimates the CSI during its downlink communications. Note that the computational complexity of CSI estimation at the nodes is lightweight since each SNOW packet has a 32-bit preamble~\cite{snow_ton}, divided into four equal parts. A node thus processes a vector of only 8 bits at a time.
%during CSI estimation.




\subsection{Carrier Frequency Offset Estimation} \label{sec:cfo}

Multi-user OFDM systems are very sensitive to the CFO between the transmitters and the receiver. CFO causes the OFDM systems to lose orthogonality between subcarriers, which results in severe ICI. 
A transmitter and a receiver observe CFO due to (i) the mismatch in their local oscillator frequency as a result of hardware imperfections; (ii) the relative motion that causes a Doppler shift. 
%CFO originates in a transmitter and a receiver due to their (i) local oscillator's frequency mismatch as a result of hardware imperfections; (ii) relative motion that causes a Doppler shift. 
ICI degrades the SNR between an OFDM transmitter and a receiver, which results in significant BER. Thus, we investigate the needs for CFO estimation in our implementation.
\begin{figure}[!htb]
\centering
\includegraphics[width=0.35\textwidth]{figs/cfo/ber.eps}
\caption{BER at different $E_b/N_0$.}
\label{fig:cfo}
\end{figure}
The loss in SNR due to the CFO between the SNOW BS and a node can be estimated as 
$SNR_{loss} = 1 + \frac{1}{3}(\pi \delta f T)^2\frac{E_s}{N_0}$~\cite{nee2000ofdm}, where
% \begin{equation} \scriptsize
%  \nonumber SNR_{loss} = 1 + \frac{1}{3}(\pi \delta f T)^2\frac{E_s}{N_0} 
% \end{equation}
$\delta f$ is the frequency offset, $T$ is the symbol duration, $E_s$ is the average received subcarrier energy, and $N_0/2$ is the two-sided spectral density of the noise power.

To observe the effects of CFO, we choose two neighboring orthogonal subcarriers in the BS and send concurrent packets from two nodes at 200m distance. Each node sends successive 1000 30-byte packets. We repeat this experiment varying the transmission powers at the nodes to generate signals with different $E_b/N_0$, where $E_b$ is the average energy per bit in the received signals. 
Figure~\ref{fig:cfo} shows the BER at the BS while receiving packets from these two nodes. This figure shows that BER is nearly $10^{-3}$ even for very high $E_b/N_0$ ($\approx 40$dB), which is also very high compared to the theoretical BER~\cite{choi2000carrier}. Thus, CFO is heavily pronounced in SNOW.
The distributed and asynchronous nature of SNOW does not allow CFO estimation similar to the traditional multi-user OFDM systems.
While the USRP-based SNOW implementation provides a trivial and {\em coarse} CFO estimation, it is not robust and does not account for the mobility of the nodes~\cite{snow_ton}.
We propose a pilot-based robust CFO estimation technique, combining both coarse and finer estimations, which accounts for the mobility of the nodes as well. We use training symbols for CFO estimation in an ICI free environment for each node independently, while it joins the network by communicating with the BS using a non-overlapping {\em join subcarrier}.


We explain the CFO estimation technique between a node and the BS (uplink) on a join subcarrier $f$ based on time-domain samples. Note that the BS keeps running the G-FFT on the entire BS spectrum. We thus extract the corresponding time-domain samples of the join subcarrier by applying IFFT during a node join. The join subcarrier does not overlap with other subcarriers; hence it is ICI-free. If $f_{\text{node}}$ and $f_{\text{BS}}$ are the frequencies at a node and the BS, respectively, then their frequency offset $\delta f = f_{\text{node}}-  f_{\text{BS}}$.
For transmitted signal $x(t)$ from a node, the received signal  $y(t)$ at the BS that experiences a CFO of $\delta f$ is given by 
$y(t)  = x(t) e^{j2\pi \delta f t}$.
Similar to IEEE 802.11a, we estimate $\delta f$ based on short and long preamble approach. Note that the USRP-based implementation has considered only one preamble to estimate CFO.
In our implementation, the BS first divides a $n$-bit preamble from a node into short and long preambles of lengths $n/4$ and $3n/4$, respectively. Thus for a 32-bit preamble (typically used in SNOW), the lengths of the short and long preambles are  8 and 24, respectively. 
The short preamble and the long preamble are used for coarse and finer CFO estimation, respectively. 
Considering $\delta t_s$ as the short preamble duration and $\delta f_s$ as the coarse CFO estimation, we have
$y(t-\delta t_s)  = x(t) e^{j2\pi \delta f_s (t-\delta t_s)}.$

Since $y(t)$ and $y(t-\delta t_s)$ are known at the BS, we have
\begin{align*}
y(t-\delta t_s) y^*(t)  & = x(t) e^{j2\pi \delta f_s (t-\delta t_s)}       x^*(t) e^{-j2\pi  \delta f_s t}
                           = |x(t)|^2  e^{j 2\pi  \delta f_s -\delta t_s }.
\end{align*}
Taking angle of both sides gives us as follows.
% $$\sphericalangle  y(t-\delta t_s) y^*(t)   =  \sphericalangle     |x(t)|^2  e^{j 2\pi  \delta f_s -\delta t_s }  =      - 2\pi  \delta f_s \delta t_s$$
\begin{align*}
\sphericalangle  y(t-\delta t_s) y^*(t)   &=  \sphericalangle     |x(t)|^2  e^{j 2\pi  \delta f_s -\delta t_s } =      - 2\pi  \delta f_s \delta t_s
                                          %&=      - 2\pi  \delta f_s \delta t_s
\end{align*}
By rearranging the above equation, we get
$$\delta f_s   =  - \frac{\sphericalangle  y(t-\delta t_s) y^*(t) }{2\pi\delta t_s}.$$

Now that we have the coarse CFO $\delta f_s$, we correct each time domain sample (say, $P$) received in the long preamble as $ P_a = P_a e^{-ja \delta f_s}$, where $a = \{1, 2, \cdots, A\}$ and $A$ is the number of time-domain samples in the long preamble. Taking into account the corrected samples of the long preamble and considering $\delta t_l$ as the long preamble duration, we estimate the finer CFO as follows. 
\begin{equation} 
\delta f  =  - \frac{\sphericalangle  y(t-\delta t_l) y^*(t) }{2\pi\delta t_l} \label{eqn:finer_cfo}
\end{equation}
To this extent, considering the join subcarrier $f$, the {\slshape ppm (parts per million)} on the BS's crystal is given by $ \text{ppm}_\text{BS} = 10^6  \big(\frac{\delta f}{f}\big) $. Thus, the BS calculates $ \delta f_i$ on subcarrier $f_i$ (assigned for node $i$) as 
$\delta f_i =  \frac{(f_i * \text{ppm}_\text{BS})}{10^6}.$ The CFO between the Tx-Radio and the Rx-radio can be estimated using a basic SISO CFO estimation technique~\cite{yao2005blind}. Thus, BS also knows the CFO for the downlink.


We now explain the CFO estimation to compensate for the Doppler shift. Note that if the signal bandwidth is sufficiently narrow at a given carrier frequency and mobile velocity, the Doppler shift can be approximated as a common shift across the entire signal bandwidth~\cite{talbot2007mobility}. Thus, the Doppler shift in the join subcarrier for a node also represents the Doppler shift at its assigned subcarrier, and hence the estimated CFO in Equation (\ref{eqn:finer_cfo}) is not affected due to the Doppler Shift.
For simplicity, we consider that a node's velocity is constant and the change in Doppler shift is negligible during a single packet transmission in SNOW.
Considering $\delta f_d$ as the CFO due to the Doppler shift, $v$ as the velocity of the node, and $\theta$ as the angle of the arrived signal at the BS from the node, we have $f_d = f_i\big(\frac{v}{c}\big)\cos(\theta)$~\cite{talbot2007mobility}, where
% \begin{equation}
% 	\nonumber \delta f_d = f_i\big(\frac{v}{c}\big)\cos(\theta).
% \end{equation}
$f_i$ is the subcarrier center frequency and $c$ is the speed of light. The node itself may consider its motion as circular and approximate $\theta = \frac{\delta s}{r}$, where $\delta s$ is the amount of anticipated change in position during a packet transmission and $r$ is the {\em line-of-sight} distance between the node and BS. Thus, CFO compensation due to the Doppler shift is done at the nodes during uplink communications. In the downlink communications, the BS Tx-Radio can also compensate for the node's mobility as the node can report its Doppler shift to the BS during the uplink communications.

In summary, as the nodes asynchronously transmit, estimating joint-CFO of the subcarriers at the BS is very difficult. We thus use a simple feedback approach for proactive CFO correction in the uplink communications. Specifically, 
$\delta f_i$  estimated at the BS for subcarrier $f_i$ is given to the node (during joining process at subcarrier $f_i$).
The node may then adjust its transmitted signal based on $\delta f_i$ and $\delta f_d$, calculated as $(\delta f_i + \delta f_d)$, which will align its signal so that the BS does not need to compensate for CFO in the uplink communications. Such feedback-based proactive compensation scheme was studied before for multi-user OFDM and is also used in global system for mobile communication (GSM)~\cite{van1999time}.

\section{Handling the Near-Far Power Problem} \label{sec:near-far}
\begin{figure}[!htb]
\centering
\includegraphics[width=0.5\textwidth]{figs/near-far-flat.eps}
\caption{An illustration of the near-far power problem. B is farther from the BS than A and both transmit concurrently using the same Tx power.}
\label{fig:near-far}
\end{figure}
Wireless communication is susceptible to the near-far power problem, especially in CDMA (Code Division Multiple Access)~\cite{muqattash2003cdma}. Multi-user D-OFDM system in SNOW may also suffer from this problem. Figure~\ref{fig:near-far} illustrates the near-far power problem in SNOW. Suppose, nodes A and B are operating on two adjacent subcarriers. Node A is closer to the BS compared to node B. When both nodes A and B transmit concurrently to the BS, the received frequency domain signals from node A and B may look as shown on the right of Figure~\ref{fig:near-far}. Here, transmission from node B is severely interfered by the strong radiations of node A's transmission. As such, node B's signal may be buried under node A's signal making it difficult for the BS to decode the packet from node B. 
A typical SNOW deployment may have such scenarios if the nodes operating on adjacent subcarriers use the same transmission power and transmit concurrently at the BS from different distances. 
\begin{figure}[t]
    \centering 
      \subfigure[Avg. PDR at different Tx powers\label{fig:nf_pdr}]{
    \includegraphics[width=0.35\textwidth]{figs/nearfar/pdr.eps}
      }\hfill
      \subfigure[Avg. PDR at different Tx powers and time\label{fig:nf_time}]{
        \includegraphics[width=.35\textwidth]{figs/nearfar/pdr-time.eps}
      }
    \caption{Packet delivery ratio at different Tx powers}
    \label{fig:nf-effects}
 \end{figure}


To observe the near-far power problem in SNOW, we run experiments by choosing 3 different adjacent subcarriers, where the middle subcarrier observes the near-far power problem introduced by both subcarriers on its left and right. We place two CC1310 nodes within 20m of the BS that use the left and the right subcarrier, respectively. We use another CC1310 node that uses the middle subcarrier and is placed at different distances between 200 and 1000m from the BS. Nodes that are within 20m of the BS transmit packets continuously with a transmission power of 0dBm. At each distance, for each transmission power between 8 and 15dBm, the node that uses the middle subcarrier sends 100 rounds of 1000 consecutive packets (sends one packet then waits for the ACK and then sends another packet, and so on) to the BS and with a random interval of 0-500ms. For each transmission power level, at each distance, that node calculates its average {\em packet delivery ratio (PDR)}. PDR is defined as the ratio of the number of successfully acknowledged packets to the number of total packets sent.
We repeat the same experiments for 7 days at 9 AM, 2 PM, and 6 PM.

Figure~\ref{fig:nf_pdr} shows that the average PDR increases at each distance with the increase in the transmission power. Figure~\ref{fig:nf_time} depicts the result for 7-day experiments (only at a distance of 200m) and shows that the average PDR changes at different time of the day. Overall, Figure~\ref{fig:nf_pdr} and~\ref{fig:nf_time} confirms that the average PDR increases with the increase in the transmission power. To ensure the energy-efficiency at the nodes, i.e., to find a  transmission power  that suffices to eliminate the effects of near-far power problem, we propose an adaptive transmission power control for the SNOW design, as described below.

% To demonstrate the effects of near-far problem, we run experiments in SNOW by placing 5 nodes at different distances ranging between 200-1000m. The subcarriers assigned to the nodes are chosen in a way such that they observe the near-far problem as shown in Figure~\ref{fig:nf-effects}.  
% At each distance, a node transmits 1000 packets using a fixed transmission power (concurrently with other nodes at other distances using the same transmission power) and calculate its packet delivery ratio (PDR). A packet is correctly delivered if the node receives an ACK for that packet. At each location we vary the transmission power between 0-15dBm and repeat the same strategy. Figure~\ref{fig:nf_rssi} shows that the average RSSI at the BS increases with the increase in the transmission power, as expected. However, Figure~\ref{fig:nf_pdr} shows that the PDR at nodes at different distances vary unexpectedly. For example, node at distance 400m observes very low PDR due to the node at 200m, and so on. Thus, the near-far problem needs to be addressed in SNOW. To this extent, we propose an adaptive transmission power control in SNOW.


\subsection{Adaptive Transmission Power Control}\label{sec:atpc}
Our design objective for the adaptive Tx power control is to correlate the subcarrier-level Tx power and link quality (i.e., PDR) between each node and the BS. We thus formulate a predictive model to provide each node with a proper Tx power to make a successful transmission to the BS using its assigned subcarrier. Note that our work differs from the work in~\cite{lin2016atpc} in fundamental concepts of the network design and architecture. In~\cite{lin2016atpc}, the authors have considered a multi-hop wireless sensor network based on IEEE 802.15.4 with no concurrency between a set of transmitters and a receiver. Additionally, our model is much more simpler since we deal with single hop communications. As such, the overheads (i.e., energy consumption and latency at each node) associated with our model are fundamentally lesser than that in~\cite{lin2016atpc}, or the other techniques developed for multi-hop wireless networks~\cite{son2006experimental, li2005cone}. In the following, we describe our model.


Whenever a node is assigned a new subcarrier or observes a lower PDR, e.g., PDR below quality of service (QoS) requirements due to mobility, it runs a lightweight predictive model to determine the convenient Tx power to make successful transmissions to the BS.
Our predictive model uses an approximation function to estimate the PDR distribution at different Tx power levels. Over time, that function is modified to adapt to the node's changes. The function is built from the sample pairs of the Tx power levels and PDRs between the node and the BS via a curve-fitting approach. A node collects these samples by sending groups of packets to the BS at different Tx power levels. A node may not be assigned new subcarriers or may not observe lower PDR due to mobility (as per our CSI and CFO estimations) frequently. Thus, the overhead (e.g., energy consumption) for collecting these samples may be negligible compared to the overall network lifetime (which is several years).

Specifically, our predictive model uses two vectors: $TP$ and $L$, where $TP = \{ tp_1, tp_2, \cdots, tp_m \}$ contains $m$ different Tx power levels that the node uses to send $m$ groups of packets to the BS and $L = \{ l_1, l_2, \cdots, l_m \}$ contains the corresponding PDR values at different Tx power levels. Thus, $l_i$ represents the PDR value at Tx power level $tp_i$. We use the following linear function to correlate between Tx power and PDR.
\begin{equation}
	l(tp_i) = a~.~tp_i + b \label{eqn:linear_model}
\end{equation}
To lessen the computational overhead in the node, we adopt the {\em least square approximation} technique to determine the unknown coefficients $a$ and $b$ in Equation (\ref{eqn:linear_model}). Thus, we find the minimum of the function $S(a, b)$, where $\nonumber S(a, b) = \sum |l_i - l(tp_i)|^2.$
% \begin{equation}
% \nonumber	S(a, b) = \sum |l_i - l(tp_i)|^2.
% \end{equation}
The minimum of $S(a, b)$ is determined by taking the partial derivatives of $S(a, b)$ with respect to $a$ and $b$, respectively, and setting them to zero. Thus, $ \frac{\partial S}{\partial a} = 0$ and $\frac{\partial S}{\partial b} = 0$ give us
\begin{align}
	\nonumber a~\sum (tp_i)^2 + b~\sum tp_i &= \sum l_i.tp_i \text{ and} \\ 
  \nonumber a~\sum tp_i + b~m &= \sum l_i.
\end{align}
Simplifying the above two equations, we find the estimated values of $a$ and $b$ as follows.
\begin{equation}\nonumber
\begin{split}
	\begin{bmatrix}
		\hat{a}\\
        \hat{b}
	\end{bmatrix}
    = \frac{1}{m \sum (tp_i)^2 - (\sum tp_i)^2} \times \\
    \begin{bmatrix}
    	m \sum l_i.tp_i - \sum l_i \sum tp_i\\
    	\sum l_i \sum (tp_i)^2 - \sum l_i.tp_i \sum tp_i
    \end{bmatrix}
\end{split}
\end{equation}
Using the estimated values of $a$ and $b$, the node can calculate the appropriate Tx power as follows.
\begin{equation}\label{eqn:estimated}
tp = \big[\frac{PDR_{\text{threshold}} - \hat{b}}{\hat{a}}\big] \in TP
\end{equation}
Here, $PDR_{\text{threshold}}$ is the threshold set empirically or according to QoS requirements, and $[.]$ denotes the function that rounds the value to the nearest integer in the vector $TP$.

Now that the initial model has been established in Equation (\ref{eqn:estimated}), this needs to be updated continuously with the node's changes over time. In Equation (\ref{eqn:linear_model}), both $a$ and $b$ are functions of time that allow the node to use the latest samples to adjust the curve-fitting model dynamically. 
It is empirically found that (Figure~\ref{fig:nf_pdr}) the slope of the curve does not change much over time; hence $a$ is assumed time-invariant in the predictive model. On the other hand, the value of $b$ changes drastically over time (Figure~\ref{fig:nf_time}). Thus, Equation (\ref{eqn:linear_model}) is rewritten as follows that characterizes the actual relationship between Tx power and PDR.
\begin{equation}
	\nonumber l(tp(t)) = a.tp(t) + b(t)
\end{equation}
Now, $b(t)$ is determined by the latest Tx power and PDR pairs using the following feedback-based control equation~\cite{lin2016atpc}.
\begin{align}
	\nonumber \Delta \hat{b}(t) &= \hat{b}(t) - \hat{b}(t+1) \\
    			\nonumber	  &= \frac{\sum^K_{k=1} [PDR_{\text{threshold}} - l_k(t - 1)]}{K} \\ 
                      &= PDR_{\text{threshold}} - l(t-1) \label{eqn:control}
\end{align}
Here, $l(t-1)$ is the average value of $K$ readings denoted as 
\begin{equation}
	\nonumber l(t-1) = \frac{\sum^K_{k=1} l_k(t - 1)}{K}.
\end{equation}
Here, $l_k(t-1)$, for $k = \{1, 2, \cdots, K\}$, is one reading of PDR during the time period $t-1$ and $K$ is the number of feedback responses at time period $t-1$. Now, the error in Equation (\ref{eqn:control}) is deducted from the previous estimation; hence the new estimation of $b(t)$ can be written as: $\hat{b}(t) = \hat{b}(t-1) - \Delta \hat{b}(t)$.
Given the newly estimated $\hat{b}(t)$, the node now can set the Tx power at time $t$ as
\begin{equation}
	\nonumber tp(t) = \big[\frac{PDR_{\text{threshold}} - \hat{b}(t)}{\hat{a}}\big].
\end{equation}














\section{Applications and Deployment}
Document image classification can find many enterprise level applications in insurance companies to reduce manual review and stream line claims processes. For example, casualty injuries claims usually have multiple correspondences with hospitals and clinics, each involving documents to be processed by the insurance company. Personal automobile claims processing can use automatic document classification of letters of guarantee, purchase receipts, and others in order to auto-approve a certain percentage of auto claim reimbursements. Each year, millions of claim related document images sent to insurance companies belonging to hundreds of categorizes related to financial institutions, medical providers, or legal organizations. A scalable document classifier can assign the correct categories as meta data to a document, which can be used to route to appropriate downstream tasks. Given the scale document image classification can be deployed in an enterprise, 
% an effective machine learning approach, 
an efficient framework for model training and iterations, together with an economical model hosting solutions has clear cost advantages.  
To support all those applications and achieve potential scalability , we are in the progress of deploying this trained model as an API under Container as a Service (CaaS) on AWS using Fargate managed Elastic Container Service (ECS). The efficient model size facilitates faster deployment and ease of maintenance. A standard ECS task with v4CPUs and 30G RAM can be used and the weights of the trained model can be conveniently packaged in a dockerized image and deployed at scale. 
% \textcolor{red}{but even large models can still be dockerized}









% 


%!TEX root = main.tex
\section{Evaluation}
\label{sec:eval}

In this section, we evaluate the performance of our unsupervised Ordered Word Mover's Distance metric and supervised Multi-scale Sentence Matching model with factorized sentences as input. We apply our algorithms to semantic textual similarity estimation tasks and sentence pair paraphrase identification tasks, based on four datasets: STSbenchmark, SICK, MSRP and MSRvid. 

\subsection{Experimental Setup}
\label{subsec:setup}


\begin{table}[tb]
  \caption{Description of evaluation datasets.}
  \label{tab:datasets}
  \begin{tabular}{lllll}
    \toprule
    Dataset & Task & Train & Dev & Test\\
    \midrule
    STSbenchmark & Similarity scoring & $5748$ & $1500$ & $1378$ \\
    SICK & Similarity scoring & $4500$ & $500$ & $4927$ \\
    MSRP & Paraphrase identification & $4076$ & - & $1725$ \\
    MSRvid & Similarity scoring & $750$ & - & $750$ \\
    \bottomrule
  \end{tabular}
  \vspace{-2mm}
\end{table}

We will start with a brief description for each dataset:
\begin{itemize}
\item \textbf{STSbenchmark}\cite{cer2017semeval}: it is a dataset for semantic textual similarity (STS) estimation. The task is to assign a similarity score to each sentence pair on a scale of 0.0 to 5.0, with 5.0 being the most similar.

\item \textbf{SICK}\cite{marelli2014sick}: it is another STS dataset from the SemEval 2014 task 1. It has the same scoring mechanism as STSbenchmark, where 0.0 represents the least amount of relatedness and 5.0 represents the most.

\item \textbf{MSRvid}: the Microsoft Research Video Description Corpus contains 1500 sentences that are concise summaries on the content of a short video. Each pair of sentences is also assigned a semantic similarity score between 0.0 and 5.0. 

\item \textbf{MSRP}\cite{quirk2004monolingual}: the Microsoft Research Paraphrase Corpus is a set of 5800 sentence pairs collected from news articles on the Internet. Each sentence pair is labeled 0 or 1, with 1 indicating that the two sentences are paraphrases of each other.
\end{itemize}

Table \ref{tab:datasets} shows a detailed breakdown of the datasets used in evaluation.
For STSbenchmark dataset we use the provided train/dev/test split.
The SICK dataset does not provide development set out of the box, so we extracted 500 instances from the training set as the development set.
For MSRP and MSRvid, since their sizes are relatively small to begin with, we did not create any development set for them.

One metric we used to evaluate the performance of our proposed models on the task of semantic textual similarity estimation is the Pearson Correlation coefficient, commonly denoted by $r$. Pearson Correlation is defined as:
\begin{equation}
\label{eq:pearson}
 r = cov(X,Y) /( \sigma_X \sigma_Y),
\end{equation}
where $cov(X,Y)$ is the co-variance between distributions X and Y, and $\sigma_X$, $\sigma_Y$ are the standard deviations of X and Y.
The Pearson Correlation coefficient can be thought as a measure of how well two distributions fit on a straight line. Its value has range [-1, 1], where a value of 1 indicates that data points from two distribution lie on the same line with a positive slope.
% Due to this unique property, we believe the Pearson Correlation coefficient is a strong indicator of the performance of our metric. 

Another metric we utilized is the Spearman's Rank Correlation coefficient. Commonly denoted by $r_s$, the Spearman's Rank Correlation coefficient shares a similar mathematical expression with the Pearson Correlation coefficient, but it is applied to ranked variables.
Formally it is defined as \cite{wiki:spearman}:
\begin{equation}
\label{eq:spearman}
 \rho = cov(rg_X, rg_Y) / (\sigma_{rg_X} \sigma_{rg_Y}),
\end{equation}
where $rg_X$, $rg_Y$ denotes the ranked variables derived from $X$ and $Y$. $cov(rg_X,rg_Y)$, $\sigma_{rg_X}$, $\sigma_{rg_Y}$ corresponds to the co-variance and standard deviations of the rank variables. The term ranked simply means that each instance in X is ranked higher or lower against every other instances in X and the same for Y. We then compare the rank values of X and Y with \ref{eq:spearman}. Like the Pearson Correlation coefficient, the Spearman's Rank Correlation coefficient has an output range of [-1, 1], and it measures the monotonic relationship between X and Y. A Spearman's Rank Correlation value of 1 implies that as X increases, Y is guaranteed to increase as well.
The Spearman's Rank Correlation is also less sensitive to noise created by outliers compared to the Pearson Correlation.

For the task of paraphrase identification, the classification accuracy of label $1$ and the F1 score are used as metrics. 

In the supervised learning portion, we conduct the experiments on the aforementioned four datasets. We use training sets to train the models, development set to tune the hyper-parameters and each test set is only used once in the final evaluation. For datasets without any development set, we will use cross-validation in the training process to prevent overfitting, that is, use $10\%$ of the training data for validation and the rest is used in training. For each model, we carry out training for 10 epochs. We then choose the model with the best validation performance to be evaluated on the test set.  


\subsection{Unsupervised Matching with OWMD}
\label{subsec:eval-owmd}

To evaluate the effectiveness of our Ordered Word Mover's Distance metric, we first take an unsupervised approach towards the similarity estimation task on the STSbenchmark, SICK and MSRvid datasets. Using the distance metrics listed in Table \ref{tab:compare-pearson} and \ref{tab:compare-spearman}, we first computed the distance between two sentences, then calculated the Pearson Correlation coefficients and the Spearman's Rank Correlation coefficients between all pair's distances and their labeled scores. We did not use the MSRP dataset since it is a binary classification problem.


In our proposed Ordered Word Mover's Distance metric, distance between two sentences is calculated using the order preserving Word Mover's Distance algorithm. For all three datasets, we performed hyper-parameter tuning using the training set and calculated the Pearson Correlation coefficients on the test and development set. We found that for the STSbenchmark dataset, setting $\lambda_1=10$, $\lambda_2=0.03$ produces the most optimal result. For the SICK dataset, a combination of $\lambda_1=3.5$, $\lambda_2=0.015$ works best. And for the MSRvid dataset, the highest Pearson Correlation is attained when $\lambda_1=0.01$, $\lambda_2=0.02$.
We maintain a max iteration of 20 since in our experiments we found that it is sufficient for the correlation result to converge.
During hyper-parameter tuning we discovered that using the Euclidean metric along with $\sigma=10$ produces better results, so all OWMD results summarized in Table \ref{tab:compare-pearson} and \ref{tab:compare-spearman} are acquired under these parameter settings. Finally, it is worth mentioning that our OWMD metric calculates the distances using factorized versions of sentences, while all other metrics use the original sentences. Sentence factorization is a necessary preprocessing step for the OWMD metric.


We compared the performance of Ordered Word Mover's Distance metric with the following methods:

\begin{itemize}
\item \textbf{Bag-of-Words (BoW)}: in the Bag-of-Words metric, distance between two sentences is computed as the cosine similarity between the word counts of the sentences.

\item \textbf{LexVec}~\cite{salle2016enhancing}: calculate the cosine similarity between the  averaged 300-dimensional LexVec word embedding of the two sentences. 

\item \textbf{GloVe}~\cite{pennington2014glove}: calculate the cosine similarity between the averaged 300-dimensional GloVe 6B word embedding of the two sentences. 

\item \textbf{Fastext}~\cite{joulin2016bag}: calculate the cosine similarity between the averaged 300-dimensional Fastext word embedding of the two sentences. 

\item \textbf{Word2vec}~\cite{mikolov2013efficient}: calculate the cosine similarity between the averaged 300-dimensional Word2vec word embedding of the two sentences.

\item \textbf{Word Mover's Distance (WMD)}~\cite{kusner2015word}: estimating the semantic distance between two sentences by WMD introduced in Sec.~\ref{sec:owmd}.
\end{itemize} 


\begin{table}[tb]
  \caption{Pearson Correlation results on different distance metrics.}
  \label{tab:compare-pearson}
  \begin{tabular}{c|cc|cc|c}
    \toprule
    \multirow{2}{*}{Algorithm} & \multicolumn{2}{c}{STSbenchmark} & \multicolumn{2}{c}{SICK} & MSRvid\\ 
     & Test & Dev & Test & Dev & Test\\
    \midrule
    BoW & $0.5705$ & $0.6561$ & $0.6114$ & $0.6087$ & $0.5044$ \\
    LexVec & $0.5759$ & $0.6852$ & $0.6948$ & $\mathbf{0.6811}$ & $0.7318$\\
    GloVe & $0.4064$ & $0.5207$ & $0.6297$ & $0.5892$  & $0.5481$ \\
    Fastext & $0.5079$ & $0.6247$ & $0.6517$ & $0.6421$  & $0.5517$  \\
    Word2vec & $0.5550$ & $0.6911$ & $\mathbf{0.7021}$ & $0.6730$  & $0.7209$  \\
    WMD & $0.4241$ & $0.5679$ & $0.5962$ & $0.5953$  & $0.3430$  \\
    OWMD & $\mathbf{0.6144}$ & $\mathbf{0.7240}$ & $0.6797$ & $0.6772$  & $\mathbf{0.7519}$  \\
    \bottomrule
  \end{tabular}
  \vspace{-4mm}
\end{table}

\begin{table}[tb]
  \caption{Spearman's Rank Correlation results on different distance metrics.}
  \label{tab:compare-spearman}
  \begin{tabular}{c|cc|cc|c}
    \toprule
    \multirow{2}{*}{Algorithm} & \multicolumn{2}{c}{STSbenchmark} & \multicolumn{2}{c}{SICK} & MSRvid\\ 
     & Test & Dev & Test & Dev & Test\\
    \midrule
    BoW & $0.5592$ & $0.6572$ & $0.5727$ & $0.5894$ & $0.5233$ \\
    LexVec & $0.5472$ & $0.7032$ & $0.5872$ & $0.5879$ & $0.7311$\\
    GloVe & $0.4268$ & $0.5862$ & $0.5505$ & $0.5490$  & $0.5828$ \\
    Fastext & $0.4874$ & $0.6424$ & $0.5739$ & $0.5941$  & $0.5634$  \\
    Word2vec & $0.5184$ & $0.7021$ & $0.6082$ & $0.6056$  & $0.7175$  \\
    WMD & $0.4270$ & $0.5781$ & $0.5488$ & $0.5612$  & $0.3699$  \\
    OWMD & $\mathbf{0.5855}$ & $\mathbf{0.7253}$ & $\mathbf{0.6133}$ & $\mathbf{0.6188}$  & $\mathbf{0.7543}$  \\
    \bottomrule
  \end{tabular}
  \vspace{-2mm}
\end{table}


Table \ref{tab:compare-pearson} and Table \ref{tab:compare-spearman} compare the performance of different metrics in terms of the Pearson Correlation coefficients and the Spearman's Rank Correlation coefficients.
We can see that the result of our OWMD metric achieves the best performance on all the datasets in terms of the Spearman's Rank Correlation coefficients.
It also produced the best Pearson Correlation results on the STSbenchmark and the MSRvid dataset, while the performance on SICK datasets are close to the best.
This can be attributed to the two characteristics of OWMD. First, the input sentence is re-organized into a predicate-argument structure using the sentence factorization tree. Therefore, corresponding semantic units in the two sentences will be aligned roughly in order. Second, our OWMD metric takes word positions into consideration and penalizes disordered matches. Therefore, it will produce less mismatches compared with the WMD metric.

% On the SICK dataset, although the result of our metric falls slightly behind Word2vec, LexVec on the test set and Word2vec on the development set, we still believe that it is a superior metric because it produced competitive results across multiple datasets. 

% Table \ref{tab:compare-spearman} presents the Spearman's Rank Correlation coefficients acquired with the same distance metrics. We can observe that our OWMD metric achieves the highest correlation scores on all three datasets. Which proves once again that OWMD is a better distance metric for the task of semantic similarity detection.

\subsection{Supervised Multi-scale Semantic Matching}
\label{subsec:eval-multilayer}

\begin{table*}[tb]
  \caption{A comparison among different supervised learning models in terms of accuracy, F1 score, Pearson's $r$ and Spearman's $\rho$ on various test sets.}
  \label{tab:sts}
  \begin{tabular}{c|cc|cc|cc|cc}
    \toprule
    \multirow{2}{*}{Model} & \multicolumn{2}{c}{MSRP} & \multicolumn{2}{c}{SICK} & \multicolumn{2}{c}{MSRvid} & \multicolumn{2}{c}{STSbenchmark}\\ 
     & Acc.(\%) & F1(\%) & $r$ & $\rho$ & $r$ & $\rho$ & $r$ & $\rho$ \\
    \midrule
    MaLSTM & $66.95$ & $73.95$ & $0.7824$ & $0.71843$ & $0.7325$ & $0.7193$ & $0.5739$ & $0.5558$\\
    Multi-scale MaLSTM & $\mathbf{74.09}$ & $\mathbf{82.18}$ & $\mathbf{0.8168}$ & $\mathbf{0.74226}$ & $\mathbf{0.8236}$ & $\mathbf{0.8188}$ & $\mathbf{0.6839}$ & $\mathbf{0.6575}$\\
    \midrule
    HCTI & $73.80$ & $80.85$ & $0.8408$ & $0.7698$ & $\mathbf{0.8848}$ & $\mathbf{0.8763}$  & $\mathbf{0.7697}$ & $\mathbf{0.7549}$ \\
    Multi-scale HCTI & $\mathbf{74.03}$ & $\mathbf{81.76}$ & $\mathbf{0.8437}$ & $\mathbf{0.7729}$ & $0.8763$ & $0.8686$  & $0.7269$ & $0.7033$  \\
    \bottomrule
  \end{tabular}
  \vspace{-2mm}
\end{table*}

The use of sentence factorization can improve both existing unsupervised metrics and existing supervised models. 
% We extend the normal Siamese model to Fig. \ref{fig:network} to take advantage of different level of information in the factorized sentence. 
To evaluate how the performance of existing Siamese neural networks can be improved by our sentence factorization technique and the multi-scale Siamese architecture, we implemented two types of Siamese sentence matching models, HCTI \cite{mueller2016siamese} and MaLSTM \cite{shao2017hcti}. HCTI is a Convolutional Neural Network (CNN) based Siamese model, which achieves the best Pearson Correlation coefficient on STSbenchmark dataset in SemEval2017 competition (compared with all the other neural network models). MaLSTM is a Siamese adaptation of the Long Short-Term Memory (LSTM) network for learning sentence similarity. As the source code of HCTI is not released in public, we implemented it according to \cite{shao2017hcti} by Keras \cite{chollet2015keras}. With the same parameter settings listed in paper \cite{shao2017hcti} and tried our best to optimize the model, we got a Pearson correlation of 0.7697 (0.7833 in paper \cite{shao2017hcti}) in STSbencmark test dataset.

We extended HCTI and MaLSTM to our proposed Siamese architecture in Fig. \ref{fig:network}, namely the Multi-scale MaLSTM and the Multi-scale HCTI. To evaluate the performance of our models, the experiment is conducted on two tasks: 1) semantic textual similarity estimation based on the STSbenchmark, MSRvid, and SICK2014 datasets; 2) paraphrase identification based on the MSRP dataset.

Table \ref{tab:sts} shows the results of HCTI, MaLSTM and our multi-scale models on different datasets. Compared with the original models, our models with multi-scale semantic units of the input sentences as network inputs significantly improved the performance on most datasets. 
Furthermore, the improvements on different tasks and datasets also proved the general applicability of our proposed architecture.

Compared with MaLSTM, our multi-scaled Siamese models with factorized sentences as input perform much better on each dataset. For MSRvid and STSbenmark dataset, both Pearson's $r$ and Spearman's $\rho$ increase about $10\%$ with Multi-scale MaLSTM. Moreover, the Multi-scale MaLSTM achieves the highest accuracy and F1 score on the MSRP dataset compared with other models listed in Table \ref{tab:sts}.

There are two reasons why our Multi-scale MaLSTM significantly outperforms MaLSTM model. First, for an input sentence pair, 
we explicitly model their semantic units with the factorization algorithm.
%we explicitly model the different scales of semantics of them with the semantic units produced by our sentence factorization algorithm. 
Second, our multi-scaled network architecture is 
specifically designed
%specially adapted to 
for multi-scaled sentences representations. Therefore, it is able to explicitly match a pair of sentences at different granularities.

We also report the results of HCTI and Multi-scale HCTI in Table \ref{tab:sts}. For the paraphrase identification task, our model shows better accuracy and F1 score on MSRP dataset. For the semantic textual similarity estimation task, the performance varies across datasets. On the SICK dataset, the performance of Multi-scale HCTI is close to HCTI with slightly better Pearson' $r$ and Spearman's $\rho$. However, the Multi-scale HCTI is not able to outperform HCTI on MSRvid and STSbenchmark. HCTI is still the best neural network model on the STSbenchmark dataset, and the MSRvid dataset is a subset of STSbenchmark.
Although HCTI has strong performance on these two datasets, it performs worse than our model on other datasets.
% Overall, the experimental results demonstrated the superior applicability and generalizability of our proposed models.
Overall, the experimental results demonstrated the general applicability of our proposed model architecture, which performs well on various semantic matching tasks.

% \begin{table}[tb]
%   \caption{Results of Accuracy and F1 score on MSRP test dataset.}
%   \label{tab:MSRP result}
%   \begin{tabular}{lllll}
%     \toprule
%     Model & Acc.(\%) & F1(\%)  \\
%     \midrule
%     MaLSTM & $66.95$ & $73.95$ \\
%     Factorized MaLSTM & $\mathbf{74.09}$ & $\mathbf{82.18}$ \\
%     HCTI & $73.80$ & $80.85$ \\
%     Factorized HCTI & $\mathbf{74.03}$ & $\mathbf{81.76}$ \\
%     \bottomrule
%   \end{tabular}
%   \vspace{0mm}
% \end{table}


% \begin{table}[tb]
%   \caption{Results of Pearson's $r$ and Spearman's $\rho$ on SICK test dataset.}
%   \label{tab:SICK result}
%   \begin{tabular}{lllll}
%     \toprule
%     Model & r & $\rho$ \\
%     \midrule
%     MaLSTM & $0.7824$ & $0.71843$ \\
%     Factorized MaLSTM & $\mathbf{0.8168}$ & $\mathbf{0.74226}$ \\
%     HCTI & $0.8408$ & $\mathbf{0.7698}$ \\
%     Factorized HCTI & $\mathbf{0.8429}$ & $0.7676$ \\
%     \bottomrule
%   \end{tabular}
%   \vspace{0mm}
% \end{table}


% \begin{table}[tb]
%   \caption{Results of Pearson's $r$ and Spearman's $\rho$ on MSRvid test dataset.}
%   \label{tab:MSRvid result}
%   \begin{tabular}{lll}
%     \toprule
%     Model & r & $\rho$  \\
%     \midrule
%     MaLSTM & $0.7325$ & $0.7193$ \\
%     Factorized MaLSTM & $\mathbf{0.8236}$ & $\mathbf{0.8188}$ \\
%     HCTI & $\mathbf{0.8848}$ & $\mathbf{0.8763}$ \\
%     Factorized HCTI & $0.8763$ & $0.8686$ \\
%     \bottomrule
%   \end{tabular}
%   \vspace{0mm}
% \end{table}



% \begin{table}[tb]
%   \caption{Results of Pearson's $r$ and Spearman's $\rho$ on STSbenchmark test dataset.}
%   \label{tab:STSbenchmark result}
%   \begin{tabular}{lllll}
%     \toprule
%     Model & r & $\rho$ \\
%     \midrule
%     MaLSTM & $0.5739$ & $0.5558$ \\
%     Factorized MaLSTM & $\mathbf{0.6839}$ & $\mathbf{0.6575}$ \\
%     HCTI & $\mathbf{0.7697}$ & $\mathbf{0.7549}$ \\
%     Factorized HCTI & $0.7269$ & $0.7033$ \\
%     \bottomrule
%   \end{tabular}
%   \vspace{0mm}
% \end{table}




\textbf{Related work}:
% Object detection related datasets/algo in non-medical domain
% Locally labeled CXR dataset
A few CXR datasets have localized abnormality annotations \cite{shih2019augmenting,filice2020crowdsourcing,jaeger2014two} that are curated manually. These are high quality gold standard ground truth datasets but tend to be smaller in scale (< 30,000 images) and have a narrow coverage, with typically only 1-2 labels. In addition, since most labeling efforts only have abnormality semantics attached, no direct relationships with the affected anatomical locations are available. 

%MEHDI: repeated concepts from above. I am removing the following: 

%The lack of anatomic semantics in the annotation is a limitation for complex multi-modal clinical reasoning work, e.g., differential diagnosis, since clinicians often integrate information along anatomical lines, and for downstream report generation tasks, which often requires describing not only the abnormality but also correctly communicate the location of the abnormalities (and medical devices) to the receiving clinicians. 

Two recent CXR datasets have labels for anatomies described in the reports. In \cite{datta2020dataset}, a small manually annotated dataset (2000 reports) included 10 abnormalities that are individually associated with 29 unique spatial locations (anatomies) at the report level. Another CXR dataset has automatically extracted abnormality and anatomy labels as disconnected concepts that are only correlated at the study level from  160,000 reports using a supervised NLP algorithm \cite{bustos2020padchest}. This was trained on a smaller set of manually annotated data. Neither datasets contain localized annotations for the associated CXR images, nor any comparison relation annotations between sequential exams, both of which are available in the Chest ImaGenome dataset. In Table \ref{tab:related}, we present a comparison of our Chest ImagGenome dataset with other datasets available in the literature.

% Table -- Kashyap

% MEdical imaging datasets to go here: Discussed that we will only focus on cxr datasets that are available for this paper. 
% \caption{\color{red} Kashyap, feel free to continue with the table. We should remove the questionmarks and add a line for our dataset (since all others are not graph). For longer text, using abbreviations and explaining them in the caption often works better. If fill in the values is not possible, it is better to remove the table altogether.}


\begin{table}[t!]
\caption{Summary of existing chest X-ray datasets}
\resizebox{\textwidth}{!}{%
\begin{tabular}{@{}lllllllll@{}}
\toprule
\textbf{Dataset} & \textbf{Annotation Level} & \textbf{Annotation Method} & \textbf{Num Labels} & \textbf{Anatomy Labeled} & \textbf{Graph} & \textbf{Dataset Size} & \textbf{Temporal Labels} & \textbf{Reports} \\ \midrule
SIIM-ACR Pneumothorax Segmentation \cite{filice2020crowdsourcing} & Segmentation & Manual + augmented & 1 & No & No & 12,047 & No & No \\
RSNA Pneumonia Detection Challenge   \cite{shih2019augmenting} & Bounding Boxes & Manual & 1 & No & No & 30,000 & No & No \\
Indiana University Chest X-ray collection \cite{demner2016preparing} & Global & Automated & 10 & No & No & 3,813 & No & Yes \\
NIH CXR dataset \cite{wang2017chestx} & Global & Automated & 14 & No & No & 112,120 & No & No \\
PLCO \cite{team2000prostate} & Global & Automated & 24 & Yes & No & 236,000 & Yes & No \\
Stanford CheXpert \cite{irvin2019chexpert} & Global & Automated & 14 & No & No & 224,316 & No & No \\
MIMIC-CXR \cite{johnson2019mimic} & Global & Automated & 14 & No & No & 377,110 & No & Yes \\
Dutta \cite{datta2020dataset} & Global & Manual & 10 & Yes & Yes & 2,000 & No & Yes \\
PadChest \cite{bustos2020padchest} & Global & Manual + automated & 297 & Yes & No & 160,868 & No & Yes \\
Montgomery County Chest X-ray   \cite{jaeger2014two} & Segmentation & Manual & 1 & Yes & No & 138 & No & No \\
Shenzen Hospital Chest X-ray   \cite{jaeger2014two} & Segmentation & Manual & 1 & Yes & No & 662 & No & No \\  \hline \hline
\textbf{Chest ImaGenome} & Bounding Boxes & Automated & 131 & Yes & Yes & 242,072 & Yes & Yes \\
\bottomrule
\end{tabular}%
}
\label{tab:related}
\vspace{-0.4cm}
\end{table}
% removed (Derived from MIMIC-CXR \cite{johnson2019mimic}) % makes table really small


\section{Conclusions}\label{sec:conclusion}
The recently proposed LPWAN technology -- SNOW -- has the potential to enable connectivity to numerous IoT devices over long distances. However, the high cost and the large form-factor of the USRP-based SNOW nodes hinder its practical deployments. In this paper, we have implemented SNOW for practical deployments using the CC13x0 devices as SNOW nodes. Our CC13x0-based SNOW implementation decreases the cost and the form-factor of a single SNOW node by 25x and 10x, respectively. 
We have also addressed several practical deployment challenges that include PAPR reduction, CSI estimation, CFO estimation, and near-far power problem. 
We have deployed our CC13x0-based SNOW in the city of Detroit, Michigan and achieved per node uplink and downlink throughputs of 11.2kbps and 4.8kbps, respectively, over a distance of 1km. 
%Additionally, our overall uplink throughput at the BS have increased linearly with the increase in the number of nodes. 
Our experiments also show that SNOW can achieve throughput several times higher than LoRaWAN under typical settings.
Finally, our extensive experiments have demonstrated the CC13x0-based SNOW as a feasible LPWAN technology that can be deployed practically at low-cost and in large-scale for future IoT applications.
%%
%% The acknowledgments section is defined using the "acks" environment
%% (and NOT an unnumbered section). This ensures the proper
%% identification of the section in the article metadata, and the
%% consistent spelling of the heading.
\begin{acks}
This work was supported by NSF through grants CAREER-1846126 and CNS-2006467.
\end{acks}

%%
%% The next two lines define the bibliography style to be used, and
%% the bibliography file.
\bibliographystyle{ACM-Reference-Format}
\bibliography{whitespacebib}


\end{document}
\endinput
%%
%% End of file `sample-acmsmall.tex'.
