
\section{Related Work}\label{sec:related}
Recently, a number of LPWAN technologies have been developed that operate in the licensed (e.g., LTE Cat M1, NB-IoT, EC-GSM-IoT, and 5G) or unlicensed (e.g., LoRa, SigFox, RPMA (INGENU), IQRF, Telensa,  DASH7, WEIGHTLESS-N, WEIGHTLESS-P, IEEE 802.11ah, IEEE 802.15.4k, and IEEE 802.15.4g) spectrum~\cite{ismail2018low, whitespaceSurvey, saxena2016achievable, chen2017narrowband, gozalvez2016new, akpakwu2017survey, kouvelas2020p}.
%\revise{Please see two surveys in~\cite{ismail2018low, whitespaceSurvey} for detailed descriptions of these technologies.}
Operating in the licensed band is costly due to high service fee and costly infrastructure. 
On the contrary, most non-cellular LPWANs including LoRa and SigFox operate in the sub-1GHz ISM band, for example, between 902 and 928MHz in the continents of North America and South America. While the ISM band is unlicensed, it is heavily crowded due to the proliferation of LPWANs as well as other wireless technologies in this band. 
To avoid the high cost of licensed band and the crowd of the ISM band, SNOW was designed to exploit the TV white spaces. As reported in~\cite{whitespaceSurvey, ismail2018low, snow2, ws_sigcomm09},
white spaces are widely available in both urban and rural areas, less crowded (compared to the ISM band), and offer a wider spectrum (compared to the available frequency bands for LPWANs in the ISM band) that a SNOW BS can utilize.

The existing work on white space focused on exploiting the white spaces for broadband access~\cite{whitespaceSurvey, zhang2015design, hasan2014gsm, kumar, harrison2015whitespace, ws_sigcomm09, WATCH, videostreaming, linkasymmetry, vehiclebased, ws_mobicom13, ws_nsdi10} and spectrum determination through spectrum sensing~\cite{saeed2017local, ws_dyspan08_kim, ws_mobicom08, ws_dyspan11, FIWEX} and/or geo-location database approach~\cite{database1, database2, database3, vehiclebased, hysim}. 
Alongside, various standards bodies (e.g., IEEE 802.11af, IEEE 802.15.4m, IEEE 802.19.1, IEEE 802.22, IEEE 1900.4a, IEEE 1900.7, and ECMA-392) and industry leaders (e.g., Microsoft and Google) have also targeted the white spaces for unlicensed personal or commercial use~\cite{kocks2012spectrum, MSRAfrica, ismail2018low, whitespaceSurvey}. In contrast, SNOW, as a LPWAN, exploits white spaces for highly scalable and wide-area sensor network applications.
With the rapid growth of IoT,  LPWANs will suffer from crowded spectrum due to long range. It is hence critical to exploit white spaces for IoT. 
Our paper focuses on implementing SNOW using the cheap and widely available COTS devices for practical and scalable deployment.




%\revise{On the contrary, most non-cellular LPWANs, except SNOW and WEIGHTLESS-W, operate in the 433MHz or sub-1GHz ISM band. For example, LoRa operates only between unlicensed 433.05--434.79MHz, 863--870MHz, or 902--928MHz frequencies, as per the adopting country's regulations. SigFox is available only in four unlicensed frequency bands: 868--878.6MHz, 902.1375--904.6625MHz, 922.3-923.5MHz, and 920.1375-922.6625MHz, around the world. While RPMA is known to operate in the unlicensed 2.4GHz band, no LPWAN has been reported to operate in the unlicensed 5GHz band or mmWave. While the ISM band is unlicensed, it is getting heavily crowded due to the proliferation of LPWANs (as discussed above) as well as other wireless technologies in this band. To avoid the high cost of licensed band and the crowd of the ISM band, SNOW was designed to exploit the TV white spaces. White spaces are widely available in both urban and rural areas, less crowded (compared to the ISM bands), and offer a wider spectrum (compared to the available widths for LPWANs in the ISM bands) that a SNOW BS can utilize.}


%Recently, a number of LPWAN technologies have been developed that operate in the licensed (LTE Cat M1, NB-IoT, EC-GSM-IoT, 5G) or unlicensed spectrum (LoRa, SigFox, RPMA (INGENU), IQRF , Telensa,  DASH7, WEIGHTLESS-N, WEIGHTLESS-P, IEEE 802.11ah, IEEE 802.15.4k, IEEE 802.15.4g)~\cite{ismail2018low, whitespaceSurvey}. Operating in the licensed band is costly due to high service fee and costly infrastructure. On the contrary, most non-cellular LPWANs, except SNOW and WEIGHTLESS-W, operate in the ISM band. While the ISM band is unlicensed, it is getting heavily crowded due to the proliferation of LPWANs as well as other wireless technologies in this band. To avoid the high cost of licensed band and the crowd of the ISM band, SNOW was designed to exploit the widely available, less crowded, and wide spectrum of the TV white spaces. The existing work on white space focused on exploiting the white spaces for broadband access~\cite{zhang2015design, ws_sigcomm09, WATCH, ws_mobicom13}, spectrum determination through spectrum sensing~\cite{ ws_dyspan08_kim, ws_mobicom08, ws_dyspan11}, or/and geo-location approach~\cite{database1, database2, vehiclebased, hysim}. We provided a comprehensive review of the existing white space technologies in~\cite{whitespaceSurvey}. Alongside, various standards bodies (IEEE 802.11af, IEEE 802.15.4m, IEEE 802.19.1, IEEE 802.22, IEEE 1900.4a, IEEE 1900.7, ECMA-392) and industry leaders (Microsoft, Google) have also targeted the white spaces for unlicensed personal or commercial use~\cite{ismail2018low, whitespaceSurvey}. In contrast, SNOW exploits white spaces for highly scalable LPWAN. To the best of our knowledge, our work is the first to implement LPWAN over white spaces using COTS devices for real-world deployments. With the rapid growth of IoT,  LPWANs will suffer from crowded spectrum due to long range. It is hence critical to exploit white spaces for IoT. Our paper focuses on implementing SNOW using the cheap and widely available COTS devices for practical and scalable deployment.



%Recently, a number of LPWAN technologies have been developed that operate in the licensed (LTE Cat M1~\cite{cat, saxena2016achievable}, NB-IoT~\cite{nbiot, chen2017narrowband}, EC-GSM-IoT~\cite{ecgsmiot, gozalvez2016new}, 5G~\cite{3gpp, akpakwu2017survey}) or unlicensed (LoRa~\cite{lorawan, kouvelas2020p}, SigFox~\cite{sigfox}, RPMA (INGENU)~\cite{rpma}, IQRF \cite{iqrf}, Telensa~\cite{telensa},  DASH7~\cite{dash7}, WEIGHTLESS-N~\cite{weightless}, WEIGHTLESS-P~\cite{weightless-p}, IEEE 802.11ah~\cite{IEE80211_ah}, IEEE 802.15.4k~\cite{IEE802154_k}, IEEE 802.15.4g~\cite{IEE802154_g}) spectrum. Operating in the licensed band is costly due to high service fee and costly infrastructure. On the contrary, most non-cellular LPWANs, except SNOW and WEIGHTLESS-W, operate in the ISM band. While the ISM band is unlicensed, it is getting heavily crowded due to the proliferation of LPWANs as well as other wireless technologies in this band. To avoid the high cost of licensed band and the crowd of the ISM band, SNOW was designed to exploit the widely available, less crowded, and wide spectrum of the TV white spaces. 

%The existing work on white space focused on exploiting the white spaces for broadband access~\cite{whitespaceSurvey, zhang2015design, hasan2014gsm, kumar, harrison2015whitespace, ws_sigcomm09, WATCH, videostreaming, linkasymmetry, vehiclebased, ws_mobicom13, ws_nsdi10} and spectrum determination through spectrum sensing~\cite{saeed2017local, ws_dyspan08_kim, ws_mobicom08, ws_dyspan11, FIWEX} or/and geo-location approach~\cite{dbreq, database1, database2, database3, vehiclebased, hysim}. Alongside, various standards bodies (IEEE 802.11af~\cite{IEE802_af}, IEEE 802.15.4m~\cite{ieee154}, IEEE 802.19.1~\cite{IEE802_19}, IEEE 802.22~\cite{IEEE802_22}, IEEE 1900.4a~\cite{IEE19004_a}, IEEE 1900.7~\cite{IEE19007}, ECMA-392~\cite{ECMA392, kocks2012spectrum}) and industry leaders (Microsoft~\cite{4Africa, MSRAfrica}, Google~\cite{GoogleAfrica}) have also targeted the white spaces for unlicensed personal or commercial use. In contrast, SNOW exploits white spaces for highly scalable LPWAN. With the rapid growth of IoT,  LPWANs will suffer from crowded spectrum due to long range. It is hence critical to exploit white spaces for IoT. Our paper focuses on implementing SNOW using the cheap and widely available COTS devices for practical and scalable deployment.