\section{Cognitive Biases: Background and Experimental Setup}\label{sec:cog_biases}

% \gabis{This section feels very long and overspecified. Can't we do with like a paragraph on each bias + references? Why does the reader needs to know this much detail on each bias? Will we use these details later? I'd only mention stuff we actually need later}
% \itay{This section was expended for two reasons - 1. to avoid describing the data generation in another section (as everyone asked in the restructure). 2. Nir felt there was no info about these biases and we should strongly establish the biases understanding as a stand-alone phenomenon by the reader before describing their testing in LMs.}
% \gabis{So maybe at tell the reader why we're elaborating so much and how this will be useful later. Readers in NLP can be sometimes impatient in reading background.}


% Our work sets out to investigate the extent to which different language models display certain
% cognitive-like
% % \gabis{again, we should be consistent, is it ``cognitive'' or ``cognitive-like''?} \itay{I hope this was addressed by a change in the coining of the term. cognitive is for humans, cognitive-like is for LMs}\gabis{did we make this distinction clear somewhere?} \itay{yes, it's marked in quotes and referred, hopefully in a clear manner.}
% biases, i.e., similar to those observed in human behavior.
% We begin with a definition of cognitive biases,
% why we chose the biases we investigate,
% and proceed to delve into a detailed description of each bias.

%\paragraph{Cognitive Bias Definition.}
% A cognitive bias refers to the observation of systematic patterns of deviation from rational or logical thinking,
% % Cognitive biases are behavioral patterns that systematically deviate from rational or logical thinking
% % in a particular manner.
% leading individuals
% %\nir{i don't think it's precise to say that biases `lead' to something - it's more of an observation about how we make choices} \itay{it's not the biases that lead, but the systematic pattern. changed the phrasing to make it more clear.}
% to make judgments or decisions 
% \nir{we should be careful with where and how we use decisions vs. judgment. there is a technical definition for each (see comments below)} \itay{ok, I'm not sure what it means here - is it not right to use here both 'judgment' and 'decisions'? if not, what is right?}
% that are influenced more by subjective factors and less by objective evidence. %\nir{all decisions are inherently subjective - even rational}
% Cognitive-like biases can be measured by a shift in preferences or choices triggered by contextual or alternative changes, which, from a rational or logical perspective, should not have a significant impact.
% \nir{Suggested revision:
% start by saying that rational choice reflects fixed preferences; then say that biases aim to explain behavior that is inconsistent with preferences,
% or that is affected by supposedly irrelevant factors, such as context or framing}

Rational choice theory depicts humans as making choices in a manner that maximizes value
on the basis of fixed preferences.
% Rational choices reflect people's fixed preferences that guide decision-making and reasoning processes.
A large body of literature is devoted to describing how actual human behavior deviates from this ideal.
Cognitive biases aim to explain regular inconsistencies in choice behavior
by revealing our susceptibility to `supposedly irrelevant' factors,
such as the context of the decision task, or its framing.
Cognitive biases are therefore defined and measured by how judgments and decisions
deviate from the rational or logical ideal in response to contextual changes.
% , which, from a rational or logical perspective, should not have a significant impact.

%\paragraph{Why We Chose These Biases.}
 Our research targets three biases
that are both prevalent and well-established.
The first two are the \emph{decoy effect} and the \emph{certainty effect} --- decision-making biases that relate to special cases of the more general \emph{prospect theory} \cite{kahneman1979prospect},
with each capturing one of its distinct aspects: the perception of value, and the perception of uncertainty.
%\nir{if you want, you can say that together both biases consider that two basic elements of prospect theory - how people perceive (or act on the perception of) (i) value, and (ii) probabilities} \itay{I prefer to keep it concise here.}
The third bias is the \emph{belief bias} --  a logical fallacy in judgment, 
which was previously observed in a closed-off pretrained model.%\nir{what do you mean by `solely' here?}\itay{delete `solely', didn't add meaning.}
%\nir{i didn't understand this last statement} \itay{rephrased it.}

In this section, we provide for each bias some general background and a description of its classic experimental setup, which we later build on.

%Additional information on the definition of cognitive bias and why we chose these biases can be found in Appendix \ref{appendix:cog_bias_extra_info}. %\gabis{I think that ``why we chose these biases'' should appear here, seems like an important design choice which should be conveyed in the main text. Maybe work with Nir on a sentence or two motivating our choice?}

%\nir{add transition sentence (``we now describe each bias...'')?}
%We now proceed to delve into the detailed description of each bias.

% \nir{perhaps write some more on cognitive biases in general, and ours in particular? give some history, context, describe their scope and prevalence, etc}
% \itay{I suspect it would be too much for the average NLP reader. We can refer to a survey maybe?}

\subsection{Decoy Effect} \label{subsec:def_decoy}
% \nir{i suggest the following structure for the `background' pars (for all three biases): start with setup, and what rational choice predicts (`when given a set of alternatives to choose from, a rational agent would choose her most preferred alternative, i.e., the one having the highest intrinsic value'); then say how general deviations from this look like (`but in reality, human choices are often affected by the set of alternatives presented to them'); then drill in to the particular effect; (`for example, in choosing between A and B, the existence of a clearly inferior alternative C can affect the choice of A vs. B'); then describe the experimental technique to elicit this sort of behavior (`to study this, experiments use a `decoy' option...');
% then say what the bias is (`when affected by the existence of a dummy option, choice behavior is said to exhibit the `decoy bias' effect').}


\paragraph{Background.} 
When choosing from a set of alternatives, a rational agent chooses the item having the highest intrinsic value.
% Consider a setting in which an agent must choose one item from a set of alternatives.
% individuals often assess multiple options based on various choice criteria.
% Whereas rational agents make choices on the basis of intrinsic item value,
Human choices, however, are often affected by context, and in particular,
by the set of available alternatives.
For example, a decision maker who chooses $A$ from the set $\{A, B\}$
may decide to choose $B$ from the set $\{A,B,C\}$ -- behavior which cannot be consistent with any underlying preference ordering \citep{mcfadden1974conditional}.\footnote{A rational agent would necessarily choose either $A$ or $C$.}
The extreme case in which $C$ is clearly inferior to both $A$ and $B$,
has been coined as the \emph{decoy effect}, to portray $C$ as `decoy' item
whose only role is to shift the choice from $A$ and $B$.
The decoy effect can be traced back to its first empirical observations in \citet{huber1982adding},
whose experimental setup we adopt and extend, and describe next.

% in choosing between two options A and B, an existence of a clearly inferior alternative option C that can affect the choice of A or B is considered a deviation from rational choice.

% To study this, experiments compare the choices done by people when presented with only two base options or with an additional 'decoy' option.
% When the presentation of a 'decoy' causes a choice shift between the two base options when this shift is regarded as the decoy bias.

% A decoy option strategically serves as a reference point to another option.
%\nir{this is an hypothesis about the mechanism; i think we should stick to saying what phenomena was empirically observed, not how or why we think it happens. if not, should say that prospect theory explains this by asserting that choices are made on the basis of perceived values, which are relative rather than absolute, and that depend on some reference point}

%By intentionally appearing inferior, the decoy option influences the perception of the other option as dominant and preferable, regardless of its intrinsic value.
%Therefore, the addition of a decoy option is an extreme case of people exhibiting context dependence choice.


%\gabis{Can we again make use of Table 1 and point to it to explain how the concepts we discuss here and in the following biases manifest in those examples?}
%\itay{added examples}
% \nir{can say that decoy is a special (even extreme) case of people exhibiting `context dependence' or `choice-set dependence' in choice}

\paragraph{Experimental Setup.}
% \nir{in my mind, these sub-sections should drill in on the experimental details - but at this point the reader should already have a broad idea of the bias and how its treated experimentally} \itay{I'm not sure what is the suggestion here - should the bias be explained better beforehand or the sub-section should drill into more detail? or both? is this assuming the other changes you suggested will be done correctly?}
To study the decoy effect, \citet{huber1982adding} proposed to measure 
how the choice between two items changes when a third \emph{asymmetrically dominated} item---the decoy--- is added to the choice set.
% The decoy effect materializes when humans are presented with two options, and will tend to have a specific change in preference when also presented with a third decoy option that is asymmetrically dominated \cite{Huber1981AddingAD}.
Items in the experiments are described by their attributes (e.g., quality and price).
In the control condition, subjects are asked to choose one item out of two comparable alternatives;
in the treatment condition, an additional \textit{Decoy Option} is added to the choice set.
The decoy's attributes are set so that it is asymmetrically dominated
(i.e., is worse in all dimensions) by one of the original items, referred to as the \textit{Target Option}, but not by the other item, referred to as the \textit{Competitor Option}.
% an asymmetrically dominated item (\textit{Decoy Option}) 
% when it has a lower rating in all dimensions compared to one option (\textit{Target Option}); but, in comparison to the other option, it has a lower rating in some dimensions and higher in others (\textit{Competitor Option}).
Table \ref{table:examples_biases} (first row) provides a concrete example:
Brand 1 and Brand 2 are comparable, whereas Brand 3 (the decoy) is inferior to Brand 2 (target), but not to Brand 1 (competitor).
%\nir{i think it's confusing that you point to an example, which is essentially how an experiment is designed, before you talk about the experimental design}
% The treatment sample has the same options Brand 1 (\textit{Competitor Option}) and Brand 2 (\textit{Target Option}) as the control sample but was added a third Brand 3 (\textit{Decoy Option}).
% Brand 3 is directly inferior compared to Brand 2.

Choice behavior is said to exhibit the `decoy effect' if 
% The decoy bias describes the phenomenon where people
subjects tend to choose Brand 1 in the control condition,
but prefer Brand 2 in the treatment condition.
By design, this means that choices are affected by a supposedly irrelevant factor---the availability of an alternative that in itself will never be chosen,
suggesting that choices are biased.
% This shift of choice happens despite these two options being the exact same options in both samples.

Note that for any fixed pair of target and competitor items,
there is a range of possible values for price and quality which the decoy can take.
\citet{huber1982adding} studies four sub-types of decoys, with attributes relative
to the target option being:
% Given pre-chosen Target and competitor options with two-dimensional quality ratings, the definition for a decoy option gives us a specific range of possible values that a third option within these values range will be considered a decoy.
% Decoys divide into four sub-types, according to their changed values relative to the target option. The sub-types are 
higher price bur same quality (\textit{R});
extremely higher price but same quality (\textit{R*});
higher price and worse quality (\textit{RF});
and same price but worse quality (\textit{F}).
We compare the effects of each sub-type on models and human choices.
%\nir{do we use this also? if so, state this. if not - consider dropping this last paragraph}
% \nir{was this in the original Huber paper? or is this our addition?} \itay{Yes, this was in the original Huber paper.}


\subsection{Certainty Effect} \label{subsec:def_certainty}
% \nir{see prev. comment on the layout of pars: rational choice (here, maximizing expected value); general deviation (make suboptimal decisions due to distorted perception of probabilities, especially small and large); specific setting (extreme case - p=1); etc}

\paragraph{Background.} Most decision settings involve some degree of uncertainty.
% In most situations, choices involve uncertain outcomes that include obtainable values in certain probabilities.
Given a set of alternatives describing possible outcomes and their probability,
utility theory \citep{friedman1948utility}
determines that rational agents will choose the option with the highest expected value.
% According to classic theory \cite{friedman1948utility}, a rational individual  will usually prefer options with a higher expected value, taking into account the values and probability of each outcome. 
Human choice, however, tends to deviate from this ideal,
especially when given probabilities are either very small or very large.
%
% However, a deviation from rational thinking would be a sub-optimal choice in terms of expected value.
% This deviation can happen when one option is presented as \textit{certain}, possibly due to over-weighting the probability of the certain choice.
%
% A certain option leads to a preference for options with lower expected value but greater certainty.
% \nir{we can't really say that preferences shift, or that something `lead' to something... again this is trying to explain the underlying mechanism, but we should stick to descriptive statements}
% tend to display a preference for options with guaranteed or certain outcomes, even when the expected value of an alternative option is objectively higher.
The \emph{certainty effect} describes people's tendency to prefer outcomes that occur with certainty to alternatives that yield higher expected value, but contain a risk.
This effect was initially explored in the seminal work of \citet{kahneman1979prospect},
whose experimental setup we describe next.\footnote{We explore an additional adjacent effect from the same work and provide details in Appendix \ref{appendix:sec:not_probable}.}

% underweight outcomes
% that are merely probable in comparison with outcomes that are obtained with certainty.

% The following experimental setup was laid out by \citet{kahneman1979prospect} which described the certainty effect for the first time.

\paragraph{Experimental Setup.}
In \citet{kahneman1979prospect}, human subjects were asked to choose between
two `lotteries', each describing a simple distribution over potential monetary rewards
(e.g., 80\% to win \$100 and 20\% to win nothing).
In the control condition, subjects were given two lotteries $A,B$ each having some degree of risk;
in the treatment condition, alternative $B$, having lower expected value,
was modified to provide its original expected value but with a probability one
(i.e., the same expected value but at no risk).

Table \ref{table:examples_biases} (second row) presents an example.
% These experiments are illustrated using the samples in the second row of Table \ref{table:examples_biases}.
% In this example, both the treatment and control samples present two options with varying probabilities and prize gaps.
In both conditions, Option A prize remains the same and has a higher expected reward than Option B,
whose certainty varies across conditions.
% The bias represents a tendency to choose Option A with higher expected utility in the control sample, yet favoring the certain Option B in the treatment sample despite its lower expected utility.
As in the example,
finding in \citet{kahneman1979prospect} (and many followup works)
revealed while control subjects tend to choose rationally,
treatment subjects display a strong preference towards the certain alternative despite its lower expected reward.


% The experiments conducted by  \citet{kahneman1979prospect}, involved presenting human participants with two options containing potential prize money along with corresponding winning probabilities.
% In one option, the expected utility was higher than the second option, whereas the second option had a 100\% winning probability, indicating certainty.

% The findings revealed that while individuals usually prefer options with higher expected utility, in these experiments they exhibit a stronger preference for the certain option, even when the option with higher expected utility is available.

% Table \ref{table:examples_biases} (second row) presents an example.
% % These experiments are illustrated using the samples in the second row of Table \ref{table:examples_biases}.
% % In this example, both the treatment and control samples present two options with varying probabilities and prize gaps.
% In both conditions, Option A remains the same, and has higher expected reward than Option B,
% whose certainty varies across conditions.
% The bias represents a tendency to choose Option A with higher expected utility in the control sample, yet favoring the certain Option B in the treatment sample despite its lower expected utility.

% \nir{next par: consider shortening considerably and turning into footnote}
% In the same work \citet{kahneman1979prospect} show another adjacent effect where people underweight probabilities of outcomes that are possible but not probable (e.g. probabilities lower than 0.03), but overweight probabilities when they are merely probable.
% We report this adjacent effect results in Appendix \ref{appendix:sec:not_probable} as it shows a similar trend to the certainty effect.
% We aggregate these effects together, as the source of bias is the same and the results trend are similar.
%That is,  individuals prefer options with a higher expected value, taking into account the probability of each outcome.
%However, this preference may shift when one option is presented as \textit{certain}, leading to a preference for options with lower expected value but greater certainty.

\subsection{Belief Bias} \label{subsec:def_belief}
\paragraph{Background.}
%\nir{here, need to say what part of this is `judgment'}
% Syllogisms are a class of reasoning problems characterized by a straightforward argument structure, involving two true statements that logically necessitate a third statement \cite{smith2000aristotle}. 
% These multi-step reasoning tasks provide an essential framework for investigating deductive reasoning abilities and the cognitive processes involved in drawing valid conclusions.
Syllogisms are a class of reasoning problems involving two true statements and a third conclusion statement, that is either logically deductible from the true statements, or is not \cite{sep-aristotle-logic}. 
To make a rational judgment of the conclusion,
it is both necessary and sufficient to apply logical reasoning
to the true statements---and to them alone. %regardless of other world knowledge. 
\emph{Belief bias} occurs when a person's evaluation of the validity of the conclusion is affected also by her personal knowledge, beliefs, or values,
which can sometimes result in false reasoning.
% rather than actual evidence presented by the two true statements.
This bias was empirically demonstrated by \citet{evans1983conflict},
whose results suggest that human judgment can be affected by the
`believability' of the conclusions,
i.e., that subjects' perception of logical validity depends on the degree
to which the conclusion statement is believable (or not).


\paragraph{Experimental Setup.}
In \citet{evans1983conflict}, human subjects were given sets of two premises and a conclusion, and asked whether the conclusion logically followed from the premises \cite{evans1983conflict}.
Half of the conclusions were phrased to be \textit{believable}---aligned with general world knowledge (e.g., ``cigarettes are addictive''), and the other half was constructed to be \textit{non-believable} (``cigarettes are non-addictive'').

Table \ref{table:examples_biases} (third row) shows an example.
Both treatment and control tasks include two premises and an invalid conclusion;
while the control includes fictitious objects,
the treatment includes real-world objects---which in this case are believable, and entail an erroneous answer (`valid').
% , , leading to more relatable conclusions.
% The bias reflects a propensity to invalidate the control sample's conclusion while validating the treatment's conclusion.
The results of \citet{evans1983conflict} show that subjects were more likely to consider believable conclusions as valid and unbelievable conclusions as invalid,
suggesting the presence of belief bias in their judgments.


% In recent work, \citet{dasgupta2022language} showed that Chinchilla  \citep{hoffmann2022training} exhibits belief-bias like behavior.
% % Our work recreates their results on this bias and expands them to other models.
% Our work investigates more models from different models' families, focusing on the effect of IT and RLHF on the extent of the biases.\nir{is this the place to state this? (didn't we already say something more general before?}
