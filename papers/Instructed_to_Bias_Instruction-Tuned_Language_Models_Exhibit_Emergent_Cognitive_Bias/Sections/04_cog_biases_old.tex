\section{Cognitive Biases}\label{sec:cog_biases}
We propose to investigate the extent to which different language models display cognitive-like biases similar to those observed in human behavior.

A cognitive bias refers to systematic patterns of deviation from rational or logical thinking, leading individuals to make judgments or decisions that are influenced by subjective factors, preconceptions, or inherent cognitive limitations rather than objective evidence or rationality.
Cognitive-like biases can be measured by a shift in preferences or choices triggered by contextual or alternative changes, which, from a rational or logical perspective, should not have a significant impact.

% Given the diverse range of cognitive biases, we have chosen to focus on three well-established biases: the \textit{Decoy Effect}, \textit{Certainty Effect}, and \textit{Belief Bias}.

Our investigation centered on the decoy effect, certainty effect, and belief bias, which are distinctive biases associated with decision-making and judgment.
The decoy and certainty effects specifically address various facets of prospect theory and possess notable characteristics that enhance their distinctiveness.
The belief bias represents a judgment-related bias and has previously been observed in a different pretrained model, making it an interesting additional bias to explore.

% \nir{perhaps write some more on cognitive biases in general, and ours in particular? give some history, context, describe their scope and prevalence, etc}
% \itay{I suspect it would be too much for the average NLP reader. We can refer to a survey maybe?}

\subsection{Decoy Effect}
When making comparative judgments of value, individuals often assess multiple options based on various choice criteria.
A decoy option strategically serves as a reference point to another option by intentionally appearing inferior, thereby influencing the perception of the other option as dominant and preferable.

The decoy effect materializes when humans are presented with two options, and will tend to have a specific change in preference when also presented with a third decoy option that is asymmetrically dominated \cite{Huber1981AddingAD}.
An option is asymmetrically dominated (\textit{Decoy Option}) when it has a lower rating in all dimensions compared to one option (\textit{Target Option}); but, in comparison to the other option, it has a lower rating in some dimensions and higher in others (\textit{Competitor Option}).

Given pre-chosen Target and competitor options with two-dimensional quality ratings, the definition for a decoy option gives us a specific range of possible values that a third option within these values range will be considered a decoy.

\subsection{Certainty Effect}
In situations where choices involve uncertain outcomes,  Individuals generally prefer options with a higher expected value, taking into account the probability of each outcome. 
However, this preference may shift when one option is presented as \textit{certain}, leading to a preference for options with lower expected value but greater certainty.
% tend to display a preference for options with guaranteed or certain outcomes, even when the expected value of an alternative option is objectively higher.
The certainty effect describes the phenomenon where people underweight outcomes that are merely probable in comparison with outcomes that are obtained with \textit{certainty} as shown by \cite{Kahneman1979ProspectTA}.

In the same work \citet{Kahneman1979ProspectTA} show another adjacent effect where people underweight probabilities of outcomes that are possible but not probable (e.g. probabilities lower than 0.03), but overweight probabilities when they are merely probable.
We aggregate these effects together, as the source of bias is the same and results trend are similar.
%That is,  individuals prefer options with a higher expected value, taking into account the probability of each outcome.
%However, this preference may shift when one option is presented as \textit{certain}, leading to a preference for options with lower expected value but greater certainty.

\subsection{Belief Bias}
Syllogisms are a class of reasoning problems characterized by a straightforward argument structure, involving two true statements that logically necessitate a third statement \cite{smith2000aristotle}. 
These multi-step reasoning tasks provide an essential framework for investigating deductive reasoning abilities and the cognitive processes involved in drawing valid conclusions.

Belief bias occurs when a person evaluates the strength of an argument based on their beliefs about the world and personal values, rather than actual evidence presented with the two true statements in the argument.

This bias was shown in a series of experiments where human subjects were given sets of two premises and a conclusion and asked whatever the conclusion followed directly from the premises \cite{evans1983conflict}.
Half of the conclusions were created so they would be \textit{believable} - aligned with general world knowledge (e.g., "cigarettes are addictive"), and the other half was constructed to be \textit{Unbelievable} ("cigarettes are non-addictive").

The results showed that humans were more likely to consider believable conclusions as valid and unbelievable conclusions as invalid, highlighting the presence of belief bias.

In recent work, \citet{dasgupta2022language} showed that the Belief bias exists in Chinchilla  \citep{hoffmann2022training}.
Our work recreates their results on this bias and expands them to other models.


