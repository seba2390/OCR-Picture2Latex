%!TEX root = main.tex
\section{Patched Estimation and\\ Spatial Spline Regression}
\label{sec:SSR}
In this section, we first explore some tentative solutions, and then point out their insufficiency and limitations in handling our constrained spatial sparse recovery problem.

\textbf{Patched Piece-wise Constant Estimation}.
In practice, we only know the locations of all $\mathbf p_{B_i}$'s and their corresponding aggregated volumes $z_i$'s. What is not known is the fine-grained distribution of each volume $z_i$ across $\Omega_{B_i}$, the subregion covers the observed point  $B_i$.

As a first heuristic, we can assume that the density is  distributed uniformly within each $\Omega_{B_i}$ and estimate $f(\mathbf p_j)$ as the volume $z_i$ divided by its area:
\begin{equation}\label{eq:patched}
	\bar f(\mathbf p_j) = \frac{z_i}{|\Omega_{B_i}|}, \text{ for each }\mathbf p_j \in \Omega_{B_i},
\end{equation}
where $|\Omega_{B_i}|$ is the area of $\Omega_{B_i}$.
%That is, first, we compute the average density for each base station by dividing corresponding volume by the area. 
%Further we assume that the density for each base station is constant on the area covered by this base station. 
Hence we obtain patched piece-wise constant estimation. In this paper, we may use {\it patch} to refer to $\Omega_{B_i}$, the subregion covered $B_i$. 

However, the patched estimation oversimplifies the solution, since the obtained estimates $\bar f(\mathbf p_j)$ are far from being smooth and in fact may have discontinuous jumps on the borders of patches, which will be illustrated in our evaluation. %\red{(Unless all average densities are the same)}
In practice, however, the piece-wise uniformity assumption fails: $f(\mathbf p_j)$ is not constant within a certain patch $\Omega_{B_i}$. In fact, $f(\mathbf p_j)$ should slowly change across neighboring points, as the underlying geographical and demographical characteristics also change smoothly across  regions.

\textbf{Spatial Spline Regression}.
The observations above naturally lead to the idea of using spatial smoothing techniques to smoothen the patched estimation $\bar f(\mathbf p_j)$ to remove the discontinuities and jumps. In the following, we briefly describe a recently proposed powerful smoothing technique called Spatial Spline Regression (SSR) \cite{Sanga13}. After demonstrating its usage for our particular problem, we point out the major limitations for the spatial sparse recovery problem.


Given a set of $l$ spatial data points in $\Omega$, which contains the following information: \emph{1)} the values of these $l$ points: $\{h_j\}_{j=1}^l$, \emph{2)} their positions $\{\mathbf p_j\}_{j=1}^l$, and \emph{3)} their attribute vectors $\{\mathbf w_j\}_{j=1}^l$, SSR fits a smooth spatial field $f$ by minimizing the following penalized sum of square errors \cite{Sanga13}, \cite{Ramsay02}, i.e.,
\begin{equation}\label{eq:minSSR}
	\underset{\boldsymbol{\beta}, f}{\text{minimize}}\sum_{j=1}^{l}\big(h_j- {\mathbf w}_j^{\mathsf T} \boldsymbol{\beta} - f(\mathbf p_j) \big)^2 + \lambda\int_\Omega(\nabla^2 f)^2\ud\mathbf p,
\end{equation}
where $f$ is assumed to be twice-differentiable over $\Omega$, and
$\nabla^2 f= \frac{\partial^2 f}{\partial x^2}+\frac{\partial^2 f}{\partial y^2}$ denotes the \emph{Laplacian} of $f$ to smoothen out the roughness of the spatial field $f$. The tuning parameter $\lambda$ is used to trade the smoothness of $f$ off for a better approximation to data value $h_j$.


% and can be selected using some data-driven or {\it ad hoc} methods.


% $\{ \mathbf p_j, h_j\}$, where $\mathbf p_j\in\Omega$ is the position and $h_j$ is the corresponding value, the SSR assumes that $h_j$ is modeled by

%\begin{equation}\label{eq:spatial_model}
%	h_j = f(\mathbf p_j) + {\mathbf w}_j^{\mathsf T}\boldsymbol{\beta} +\epsilon_j,
%\end{equation}
%where $f$ is a spatial field that give the underlying smooth surface of  $f(\mathbf p_j)$, and $\mathbf w_j$ represents the attributes at position $p_j$. Furthermore, $f$ is assumed to be twice-differentiable over $\Omega$. 
%The smoothness of $f$ enables a key assumption that many spatial data studies hinge upon, that is, the points close to each other are likely to have similar spatial field values.

%Both the spatial field $f$ and the attribute coefficients $\boldsymbol{\beta}$ can be learned based on $l$ training data points in $\Omega$, which contain the following information: \emph{1)} the values of these $l$ points: $\{h_j\}_{j=1}^l$, \emph{2)} their positions $\{\mathbf p_j\}_{j=1}^l$, and \emph{3)} their attribute vectors $\{\mathbf w_j}_{j=1}^l$.

%Once $f$ and $\boldsymbol{\beta}$ are learned from the training data, the value $z_j$ can be estimated by plugging its attribute information ${\mathbf w}_j$ and position $\mathbf p_j$ into \eqref{eq:spatial_model}.
%The model \eqref{eq:spatial_model} can be trained using either kernel-based methods \cite{Clapp04, Chopra07, Caplin08} or finite element analysis \cite{Sanga13, wood2003thin, Ramsay02}.

% However, the challenge to solving problem \eqref{eq:minSSR} is that it involves searching for a functional $f$ over a possibly non-convex domain $\Omega$ that may have strong concavities, complicated boundaries, and even interior holes.
% Although kernel-based methods \cite{Chopra07} are also a commonly used smoothing technique, their major drawback is that by using uniformly damping weights in distance-based kernels, they tend to link data points across unrelated or weakly related subregions in an irregularly shaped non-convex domain.

We now briefly describe how spatial spline regression \cite{Sanga13} can solve problem \eqref{eq:minSSR} via \emph{finite element analysis} for any irregularly shaped domain $\Omega$. 
% For details, interested readers are referred to \cite{Sanga13}.
In SSR, the domain $\Omega$ is divided into small disjoint triangles, which can be done for example by the means of Delaunay triangulation \cite{hje06}. Then a polynomial function is defined on each of these triangles, such that the summation of these polynomial functions defined on different pieces closely approximates the desired spatial field $f$. It is shown in \cite{Sanga13} that the best approximation is achieved by simply solving a set of linear equations (see \cite{Sanga13} for more details).

%Specifically, let $\zeta_1,\ldots,\zeta_K$ denote the vertices of all the small triangles, which are called control points and can be adaptively selected by available data points. Define a piecewise linear or quadratic basis function $\psi_k(x,y)$ called {\it Lagrangian finite element} with $(x,y)\in\Omega$, associated with each control point $\zeta_k$ such that $\psi_k$ evaluates to $1$ at $\zeta_k$ and is equal to $0$ at all other control points. Therefore, according to the {\it Lagrangian property of the basis} we can approximate $f(x,y)$ for any $(x,y)\in\Omega$ only using the values of $f$ on the $K$ control points, i.e., $\mathbf f_K := (f(\zeta_1),\ldots,f(\zeta_K))^{\mathsf T}$. That is, if we let $\psi(x,y) :=(\psi_1(x,y),\ldots,\psi_K(x,y))^{\mathsf T}$ denote the $K$ predefined basis functions, each corresponding to a control point, then we have
%\begin{equation}\label{eq:triApprox}
%	f(x,y) = \mbox{$\sum_{k=1}^{K}$} f(\zeta_k)\psi_k(x,y) = \mathbf f_K^{\mathsf T}\mathbf\psi(x,y),
%\end{equation}  
%Since $\psi_1(x,y),\ldots,\psi_K(x,y)$ are predefined and known \emph{a priori}, the variational estimation of $f$ in problem \eqref{eq:minSSR} boils down to the estimation of only $K$ scalar values, i.e., $\mathbf f_K = (f(\zeta_1),\ldots,f(\zeta_K))^{\mathsf T}$.

%In fact, it is shown in \cite{Sanga13} that with the piece-wise approximation given by \eqref{eq:triApprox}, solving \eqref{eq:minSSR} is simply solving a set of linear equations for $\hat f(\zeta_1),\ldots,\hat f(\zeta_K)$. 
%The estimator $\hat f(x,y)$ for $f$ can then be derived from \eqref{eq:triApprox} as
%\[
%		\hat f(x,y) = \hat{\mathbf f}_K^{\mathsf T}\mathbf\psi(x,y),
%\]

%It is worth noting that commodity triangulation software for finite element analysis is readily available in many free and commercial finite element packages. For example, Delaunay triangulations of a set of data location points (e.g., \cite{hje06}) $V$ are such that no point in $V$ is inside the circumcircle of any triangle; they maximize the minimum angle of all the triangle angles, avoiding stretched triangles. 

Now we can see that if $l=n$ and we plug $h_j = \bar f(\mathbf p_j)$, $j=1,\ldots,n$ into problem \eqref{eq:minSSR}, we will get a new density surface $\hat f$ as a solution to the SSR problem \eqref{eq:minSSR} that is a smoothened approximation of the patched estimates $\bar f(\mathbf p_j)$.

However, SSR given by \eqref{eq:minSSR}
% has a major drawback in our particular case, that is, it 
can not accommodate any constraints, and especially, does not enforce the aggregated volume constraint \eqref{eq:BSvolumes}, or equivalently, the constraint $\mathbf z = \mathbf A\mathbf f$ in  \eqref{eq:prob0}. Therefore, if we smoothen the patched estimates $\bar f(\mathbf p_j)$ out to get a smooth surface estimate $\hat f$, there is no guarantee that the estimated densities in each patch $\Omega_{B_i}$ will sum up to the observed volume $z_i$ on the point $B_i$. Violating this constraint would likely cause large density estimation errors.