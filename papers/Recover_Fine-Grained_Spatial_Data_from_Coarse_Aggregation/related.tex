%!TEX root = main.tex
\section{Related Work}
\label{sec:related}






% % surface 1
% \begin{figure}[!ht]
%         %\hspace{7mm}
%         \includegraphics[width=3.5in]{figures/3d_200BS_Nov_trueval.png}
%         %\vspace{-5mm}
%         \caption{Real cell phone activity distribution.}
%         \label{fig:3Dtrueval.png}
%         \vspace{1.5mm}
% \end{figure}

% % surface 2
% \begin{figure}[!htb]
%         %\hspace{7mm}
%         \includegraphics[width=3.5in]{figures/3d_200BS_Nov_baseline1.png}
%         %\vspace{-5mm}
%         \caption{Estimated cell phone activity distribution by Patched Estimation.}
%         \label{fig:3Dbaseline1.png}
%         \vspace{1.5mm}
% \end{figure}

% % surface 3
% \begin{figure}[!htb]
%         %\hspace{7mm}
%         \includegraphics[width=3.5in]{figures/3d_200BS_Nov_SsrAdmm.png}
%         %\vspace{-5mm}
%         \caption{Estimated cell phone activity distribution by Constrained Spatial Smoothing + Features.}
%         \label{fig:3DSsrAdmm.png}
%         \vspace{1.5mm}
% \end{figure}





The Telecom Italia Big Data Challenge dataset is a multi-source dataset that contains a variety of informations, including aggregation of telecommunication activities, news, social networks, weather, and electricity data from the city of Milan. With the important information about human activities contained in the dataset, especially the cellphone activity records, researchers utilized the data to study different problems, such as modeling human mobility patterns~\cite{gonzalez2008understanding}, population density estimation~\cite{douglass2015high}, models the spread of diseases~\cite{blondel2015survey}, modeling city ecology~\cite{bcici_mobihoc15}, etc. However, few research work has been done to estimate the spatial distribution of cellphone activity itself, despite the great value of this problem.

There are various tasks where the key problem is estimating a spatial field over a region based on observations of sampled points, such as house price estimation and population density estimation. \cite{chopra2007discovering} models the underlying surface of land desirability using kernel-based interpolation. However, it is hard to choose the form of kernel functions and tune a large number of hyper-parameters. 
Spatial Spline Regression technique is applied to the problem of population density estimation in~\cite{Sanga13}. However, in our problem, we only get the accumulated activity density in base stations, rather than real densities in each base station location. Besides, BS locations distribution is highly sparse in our case. 



% \red{Spatial smoothing:} Although model \eqref{eq:spatial_model} can be trained by a range of kernel-based methods \cite{Clapp04, Chopra07, Caplin08}, the common drawback of these approaches is that by using uniformly damping weights in distance-based kernels, they tend to link weakly related data points across areas in a non-convex domain. Spatial spline regression \cite{Sanga13} on the other hand uses finite-element analysis approach to jointly solve for $f$ and $\boldsymbol{\beta}$ from model \eqref{eq:spatial_model} over any irregularly shaped domain $\Omega$.

The fine-grained data for the distribution of the volume of calls and SMS is not usually available. A common type of data is the data collected by cell phone base stations.
% Sometimes cell phone providers interpolate the data collected by the base stations as is discussed in \cite{manfredini2014toward}.
Some researchers interpolate the data to obtain fine grained distributions as in \cite{ratti2006mobile}. However in \cite{ratti2006mobile} authors do not evaluate the performance of the interpolated distribution. To the best of our knowledge 
there is no extensive work done in trying to obtain optimal reconstructions of fine grained cell phone data distribution. 
We are the first to apply latest spatial functional analysis techniques to cellphone activity distribution modeling, assuming the activity densities consists of a regression part based on social or demographical statistic features and a spatial field that captures the underlying smoothness property of cellphone activities. 
%In particular, we leverage the idea of spatial spline regression to handle any irregularly shaped geographic regions. We have developed a novel Constrained Spatial Smoothing approach and corresponding training algorithm to recover spatial distribution of cellphone activities from highly sparse observations.