\documentclass[10pt, conference, letterpaper]{IEEEtran}

\usepackage{multicol}
\usepackage{multirow}
\usepackage{graphicx}
%% Patch to make old LaTeX understand dates in yyyy-mm-dd format
\makeatletter
\@ifundefined{@parse@version@dash}{%
	\def\@parse@version#1{\@parse@version@0#1}
	\def\@parse@version@#1/#2/#3#4#5\@nil{%
		\@parse@version@dash#1-#2-#3#4\@nil}
	\def\@parse@version@dash#1-#2-#3#4#5\@nil{%
		\if\relax#2\relax\else#1\fi#2#3#4 }
}{}
\makeatother

\documentclass[aps,pre,twocolumn,groupedaddress,longbibliography]{revtex4-2}
\usepackage{amsfonts,amssymb,amsmath} %%%add
\usepackage{color} %%%add
\usepackage{graphicx} %%%add
\usepackage{epstopdf,mathrsfs} %%%add
\usepackage[colorlinks,linktocpage,linkcolor=blue]{hyperref}

\newcommand{\red}[1]{\textcolor{red}{#1}}
\newcommand{\blue}[1]{\textcolor{blue}{#1}}
\newcommand{\cyan}[1]{\textcolor{cyan}{#1}}
\newcommand{\yellow}[1]{\textcolor{yellow}{#1}}

%\usepackage{xcolor}
%\pagecolor[rgb]{0.9, 0.99, 0.9}


\begin{document}

\title{Random diffusivity processes in an external force field}
\author{Xudong Wang$^1$}
%\email{xdwang14@njust.edu.cn}
\author{Yao Chen$^2$}
\email{ychen@njau.edu.cn}
\affiliation{$^1$School of Mathematics and Statistics, Nanjing University of Science and Technology, Nanjing, 210094, P.R. China \\
$^2$College of Sciences, Nanjing Agricultural University, Nanjing, 210094, P.R. China}


\begin{abstract}
Brownian yet non-Gaussian processes have recently been observed in numerous biological systems and the corresponding theories have been built based on random diffusivity models. Considering the particularity of random diffusivity, this paper studies the effect of an external force acting on two kinds of random diffusivity models whose difference is embodied in whether the fluctuation-dissipation theorem is valid. Based on the two random diffusivity models, we derive the Fokker-Planck equations with an arbitrary external force, and analyse various observables in the case with a constant force, including the Einstein relation, the moments, the kurtosis, and the asymptotic behaviors of the probability density function of particle's displacement at different time scales.
Both the theoretical results and numerical simulations of these observables show significant difference between the two kinds of random diffusivity models, which implies the important role of the fluctuation-dissipation theorem in random diffusivity systems.

\end{abstract}


\maketitle

\section{Introduction}\label{Sec1}
It is ubiquitous to find that particles diffuse under some kind of external force fields in the natural world. Under the effect of external forces, the motion of particles shows many kinds of anomalous diffusion phenomena in complex systems \cite{BouchaudGeorges:1990,MetzlerKlafter:2000,MagdziarzWeronKlafter:2008,EuleFriedrich:2009,CairoliBaule:2015,FedotovKorabel:2015}.
Particularly, the particles might undergo a biased random walk with a nonzero mean of displacement. The corresponding ensemble-averaged mean-squared displacement (MSD) is defined as
\begin{equation}\label{1}
\langle \Delta x^2(t)\rangle=\langle[x(t)-\langle x(t)\rangle]^2\rangle \,{\propto}\,t^\beta  \quad  (\beta\neq1),
\end{equation}
where normal Brownian motion belongs to $\beta=1$, and anomalous diffusion is characterized by the nonlinear evolution in time with $\beta\neq1$.


In addition to the normal diffusion of Brownian motion, the probability density function (PDF) of its displacement is Gaussian-shaped \cite{VanKampen:1992,CoffeyKalmykovWaldron:2004}
\begin{equation}\label{Gaussian}
  G(x,t|D)=\frac{1}{\sqrt{4\pi Dt}}\exp\left(-\frac{x^2}{4Dt}\right)
\end{equation}
for a given diffusivity $D$. In contrast to the Gaussian-shaped PDF, a new class of normal diffusion process has recently been observed with a non-Gaussian PDF, which is thus named as Brownian yet non-Gaussian process. This phenomenon has been found in a large range of complex systems, including polystyrene beads diffusing on the surface of lipid tubes \cite{WangAnthonyBaeGranick:2009} or in networks \cite{WangAnthonyBaeGranick:2009,ToyotaHeadSchmidtMizuno:2011,SilvaStuhrmannBetzKoenderink:2014}, as well as the diffusion of tracer molecules on polymer thin films \cite{Bhattacharya-etal:2013} and in simulations of two-dimensional discs \cite{KimKimSung:2013}. Instead of the Gaussian shape, the PDF of the Brownian yet non-Gaussian process is characterized by exponential distribution
\begin{equation}\label{Exponential}
  p(x,t)= \frac{1}{2\sqrt{D_0t}}\exp\left(-\frac{|x|}{\sqrt{D_0t}}\right)
\end{equation}
with $D_0$ being the effective diffusivity.

The interesting phenomenon of the non-Gaussian feature can be interpreted by the superstatistical approach of assuming the diffusivity $D$ in Eq. \eqref{Gaussian} being a random variable \cite{Beck:2001,BeckCohen:2003,Beck:2006}.
More precisely, each particle undergoes a normal Brownian motion with its own diffusivity which does not change considerably in a short time. The diffusivity $D$ of each particle obeys the exponential distribution
$\pi(D)=\exp(-D/D_0)/D_0$, and the randomness of diffusivity results from a spatially inhomogeneous environment. Averaging the Gaussian distribution in Eq. \eqref{Gaussian} over the diffusivity with the exponential distribution $\pi(D)$ yields \cite{WangKuoBaeGranick:2012,HapcaCrawfordYoung:2009}
\begin{equation}\label{PDF-x}
\begin{split}
    p(x,t)&=\int_0^\infty \pi(D)G(x,t|D)dD  \\
    &=\frac{1}{\sqrt{4D_0t}}\exp\left(-\frac{|x|}{\sqrt{D_0t}}\right).
\end{split}
\end{equation}
Besides the superstatistical approach, the exponential tail is found to be universal for short-time dynamics of the continuous-time random walk by using large deviation theory  \cite{BarkaiBurov:2020,WangBarkaiBurov:2020}.

Furthermore, the phenomenon observed in experiments also shows that the PDF undergoes a crossover from exponential distribution to Gaussian distribution \cite{WangAnthonyBaeGranick:2009,WangKuoBaeGranick:2012}. This crossover cannot reappear in the approach of the superstatistical dynamics. To interpret the phenomenon of such a crossover in the PDF of the Brownian yet non-Gaussian process, Chubynsky and Slater proposed a diffusing diffusivity model, in which the diffusion coefficient of the tracer particle evolves in time like the coordinate of a Brownian particle in a gravitational field \cite{ChubynskySlater:2014}. Chechkin {\it et al.} established a minimal model under the framework of Langevin equation with the diffusivity being the square of an Ornstein-Uhlenbeck process \cite{ChechkinSenoMetzlerSokolov:2017}. Due to the widespread applications of random diffusivity when describing the particle's motion in complex environments, the researches on systems with random parameters have been extended to many physical models, including underdamped Langevin equation \cite{SlezakMetzlerMagdziarz:2018,Vitali.etal:2018,ChenWang:2021}, generalized grey Brownian motion \cite{SposiniChechkinSenoPagniniMetzler:2018} and fractional Brownian motion \cite{JainSebastian:2018,MackalaMagdziarz:2019,Wang-etal:2020,Wang-etal:2020-2}, together with some discussions on ergodic property of random diffusivity systems \cite{CherstvyMetzler:2016,WangChen:2021,WangChen:2022}.

Our aim here is to consider the effect of an external force field on the Brownian yet non-Gaussian processes. Since it is convenient to describe a motion under an external force or an environment with fluctuation in a Langevin equation, we will investigate the effect of a force on the
minimal Langevin model with diffusing diffusivity proposed in Ref. \cite{ChechkinSenoMetzlerSokolov:2017}, where a Brownian particle with a random diffusivity $D(t)$ is described by
\begin{equation}\label{model0}
  \frac{d}{d t}x(t)=\sqrt{2D(t)}\xi(t).
\end{equation}
Here, $\xi(t)$ is the Gaussian white noise with mean zero and correlation function $\langle\xi(t_1)\xi(t_2)\rangle=\delta(t_1-t_2)$, and $D(t)$ is the square of an Ornstein-Uhlenbeck process to guarantee its positivity and randomness.

When considering the response of such a random diffusivity model to an external disturbance or the internal fluctuation of the system, we need to pay attention to whether the fluctuation-dissipation theorem (FDT) is valid or not in this system. The FDT plays a fundamental role in the statistical mechanics of nonequilibrium states and of irreversible processes \cite{Kubo:1966,MarconiPuglisiRondoniVulpiani:2008}. For this reason, two kinds of random diffusivity models, one satisfies FDT and one not, are considered, and their difference is also a main concerned object in this paper.

In addition to the FDT, Brownian motion also has a good property about Einstein relation which connects the fluctuation of an ensemble of particles with their mobility under a constant force $F$ by an equality \cite{Kubo:1966,MetzlerKlafter:2000}
\begin{equation}\label{ER}
  \langle x_F(t)\rangle=\frac{\langle x_0^2(t)\rangle}{2k_B\mathcal{T}}F.
\end{equation}
Here, $k_B$ is the Boltzmann constant, $\mathcal{T}$ is the absolute temperature of a heat bath, $x_F(t)$ and $x_0(t)$ denote the particle positions with and without the constant force $F$, respectively. Furthermore, the Einstein relation has been found to be valid for both normal and anomalous processes close to equilibrium in the limit $F\rightarrow0$, which can be derived from linear response theory \cite{BarkaiFleurov:1998,BenichouOshanin:2002,ShemerBarkai:2009,FroembergBarkai:2013-3}.
It will be interesting to find whether the Einstein relation holds or not in random diffusivity models.

In this paper, taking the two kinds of random diffusivity models satisfying the FDT or not as the main object, we first derive the Fokker-Planck equation of the PDF of particle's displacement for the two models under an arbitrary external force $F(x)$, and then make some specific analyses on the two models under a constant force $F$. The concerned observables mainly include the Einstein relation, the moments, the kurtosis of PDF, and the asymptotic behaviors of PDF.

The structure of this paper is as follows. In Sec. \ref{Sec2}, the two kinds of random diffusivity models are introduced. For arbitrary external force, the Fokker-Planck equations corresponding to the two models are derived in Sec. \ref{Sec3}. The detailed discussions on the observables for two models under a constant force are given in Secs. \ref{Sec4} and \ref{Sec5}, respectively. In Sec. \ref{Sec6}, we present the simulation results to verify the theoretical analyses on the observables for the case with constant force, and make a detailed comparison between the two models. Some discussions and summaries are provided in Sec. \ref{Sec5}. For convenience, we put some mathematical details in Appendix.




\section{Two random diffusivity models}\label{Sec2}
Since the FDT plays an important role on the diffusion behavior of a Langevin system, the difference between the two models concerned here is embodied in whether the FDT is valid or not.
Based on the random diffusivity model in Eq. \eqref{model0} characterizing the motion of a free particle, two kinds of models under an external force $F(x)$ can be written as
\begin{equation}\label{model-FDT}
  \frac{d}{d t}x(t)=\sqrt{2k_B\mathcal{T}D(t)}\xi(t)+D(t)F(x),
\end{equation}
and
\begin{equation}\label{model-NFDT}
  \frac{d}{d t}x(t)=\sqrt{2k_B\mathcal{T}D(t)}\xi(t)+F(x),
\end{equation}
respectively. The FDT is satisfied in Eq. \eqref{model-FDT}, which can be verified by dividing $D(t)$ on both sides, i.e.,
\begin{equation}\label{FDT-Eq}
  \frac{1}{D(t)}\frac{d}{d t}x(t)=\sqrt{\frac{2k_B\mathcal{T}}{D(t)}}\xi(t)+F(x).
\end{equation}
It can be seen that the dissipation memory kernel and correlation function of noise satisfy the relation \cite{Kubo:1966,KuboTodaHashitsume:1985,Zwanzig:2001,WangChenDeng:2019}
\begin{equation}\label{FDT-Relation}
  2k_B\mathcal{T}K(t_1-t_2)=\langle R(t_1)R(t_2)\rangle,
\end{equation}
where $K(t_1-t_2)=\delta(t_1-t_2)/D(t)$ is the dissipation memory kernel and $R(t)=\sqrt{\frac{2k_B\mathcal{T}}{D(t)}}\xi(t)$ is the internal noise in Eq. \eqref{FDT-Eq}. The FDT describes the phenomenon that the friction force and the random driving force come from the same origin and thus are closely related through Eq. \eqref{FDT-Relation}. For the Langevin system with a diffusing diffusivity $D(t)$ describing a spatially inhomogeneous environment, the FDT is still valid for each realization of $D(t)$.

Generalizing the idea in Refs. \cite{ChubynskySlater:2014,ChechkinSenoMetzlerSokolov:2017}, we use a generic overdampered Langevin equation to describe the diffusing diffusivity $D(t)$, i.e.,
\begin{equation}\label{model-DD}
\begin{split}
D(t)&=y^2(t),\\
\frac{d}{d t}y(t)&=f(y,t)+g(y,t) \eta(t),
\end{split}
\end{equation}
where the first equation is to guarantee the non-negativity of diffusivity $D(t)$, the second equation gives the evolution of auxiliary variable $y(t)$ with arbitrary functions $f(y,t)$ and $g(y,t)$ representing the external force and multiplicative noise on process $y(t)$. In addition, the noise $\eta(t)$ is also a Gaussion white noise with correlation function $\langle \eta(t_1)\eta(t_2)\rangle=\delta(t_1-t_2)$, similar to $\xi(t)$ but independent of $\xi(t)$. A special case that $f(y,t)=-y$ and $g(y,t)\equiv1$ yields the Ornstein-Uhlenbeck process $y(t)$ discussed in Ref. \cite{ChechkinSenoMetzlerSokolov:2017}.
Here, the arbitrary functions $f(y,t)$ and $g(y,t)$ in an overdamped Langevin equation result in a large range of diffusion processes beyond the Ornstein-Uhlenbeck process, including those reaching a steady state or not at long time limit, which is determined by the competitive roles between $f(y,t)$ and $g(y,t)$ \cite{WangDengChen:2019}. Many theoretical foundations have been established in the discussions on the ergodic properties and Feynman-Kac equations of the general overdamped Langevin equation \cite{WangDengChen:2019,WangChenDeng:2018,CairoliBaule:2017}.



\section{Fokker-Planck equations}\label{Sec3}
The Fokker-Planck equation governs the PDF $p(x,t)$ of finding the particle at position $x$ at time $t$, which describes the particle's stochastic motion in a macroscopic way. Compared with the Fokker-Planck equations containing integer derivatives for Brownian motion with or without an external force, those contain the fractional derivatives for many kinds of anomalous diffusion processes \cite{MetzlerKlafter:00,FriedrichJenkoBauleEule:2006,TurgemanCarmiBarkai:2009,KosztolowiczDutkiewicz:2021}. The Fokker-Planck equation for the random diffusivity model in Eq. \eqref{model0} have been derived in Ref. \cite{ChechkinSenoMetzlerSokolov:2017}.
Here we extend the model to the one containing an arbitrary external force and derive the corresponding Fokker-Planck equation.
Since the Langevin system includes three variables (the concerned process $x(t)$, diffusing diffusivity $D(t)$ and the auxiliary variable $y(t)$), and $D(t)$ depends on $y(t)$ explicitly as $D(t)=y^2(t)$, the bivariate PDF $p(x,y,t)$ is the underlying variable in the Fokker-Planck equation.
For convenience, we take $k_B\mathcal{T}=1$ in Eqs. \eqref{model-FDT} and \eqref{model-NFDT}, and take a space-dependent force $F(x)$. It should be noted that the results in this section are also valid for the case with time-dependent external force $F(x,t)$. The corresponding derivations can be obtained directly by replacing $F(x(s))$ with $F(x(s),s)$ in Eq. \eqref{split} and replacing $F(x(t))$ with $F(x(t),t)$ in Eq. \eqref{incre-x-y}.


Let us drive the Fokker-Planck equation corresponding to Eq. \eqref{model-FDT} firstly. Due to the FDT, the subordination method proposed in Ref. \cite{ChechkinSenoMetzlerSokolov:2017} for free particles can be applied here, i.e., rewriting the concerned process $x(t)$ as a compound process $x(t):=x(s(t))$ and splitting Eq. \eqref{model-FDT} into a Langevin system in subordinated form
\begin{equation}\label{split}
\begin{split}
\frac{d}{d s}x(s)&=\sqrt{2}\xi(s)+F(x(s)),\\
\frac{d}{d t}s(t)&=D(t),
\end{split}
\end{equation}
with the proof of the equivalence between them presented in Appendix \ref{App1}. The subordination method has been commonly used in Langevin system to describe subdiffusion \cite{Fogedby:1994,MetzlerKlafter:2000-2} or superdiffusion \cite{FriedrichJenkoBauleEule:2006,EuleZaburdaevFriedrichGeisel:2012,WangChenDeng:2019}.


The PDF $G(x,s)$ of process $x(s)$ in the first equation of Eq. \eqref{split} satisfies the classical Fokker-Planck equation \cite{Risken:1989,CoffeyKalmykovWaldron:2004}
\begin{equation}\label{Gxs}
\frac{\partial}{\partial s}G(x,s)=\left(-\frac{\partial}{\partial x}F(x)+\frac{\partial^2}{\partial x^2}\right)G(x,s).
\end{equation}
Combining the latter equation in Eq. \eqref{split}, we find $s(t)=\int_0^ty^2(t')dt'$. Therefore, $s(t)$ can be regarded as a functional of process $y(t)$, and the joint PDF $Q(s,y,t)$ satisfies the Feynman-Kac equation \cite{WangChenDeng:2018,CairoliBaule:2017,TurgemanCarmiBarkai:2009}
\begin{equation}\label{Qst}
\begin{split}
\frac{\partial}{\partial t}Q(s,y,t)&=\left(-\frac{\partial}{\partial y}f(y,t)+\frac{1}{2}\frac{\partial^2}{\partial y^2}g^2(y,t)\right)Q(s,y,t)\\
&~~~~-y^2\frac{\partial}{\partial s}Q(s,y,t).
\end{split}
\end{equation}
Since the two equations in Eq. \eqref{split} evolve independently, it holds that
\begin{equation}
\begin{split}
p(x,y,t)=\int_0^\infty G(x,s)Q(s,y,t)ds.
\end{split}
\end{equation}
Then combining the equations satisfied by $G(x,t)$ and $Q(s,y,t)$ in Eqs. \eqref{Gxs} and \eqref{Qst}, we obtain
\begin{equation}\label{FKE1}
\begin{split}
\frac{\partial}{\partial t}p(x,y,t)=&\int_0^\infty G(x,s)\frac{\partial}{\partial t}Q(s,y,t)ds\\
=&\left(-\frac{\partial}{\partial y}f(y,t)+\frac{1}{2}\frac{\partial^2}{\partial y^2}g^2(y,t)\right)p(x,y,t)\\
&-y^2\int_0^\infty G(x,s)\frac{\partial}{\partial s}Q(s,y,t) ds\\
=&\left(-\frac{\partial}{\partial y}f(y,t)+\frac{1}{2}\frac{\partial^2}{\partial y^2}g^2(y,t)\right)p(x,y,t)\\
&+y^2\left(-\frac{\partial}{\partial x}F(x)+\frac{\partial^2}{\partial x^2}\right)p(x,y,t),
\end{split}
\end{equation}
where the integration by parts has been used in the last equality and the corresponding boundary terms vanish.

For another model violating the FDT in Eq. \eqref{model-NFDT}, it cannot be split into two independent equations as Eq. \eqref{split}, and the subordination method is not applicable for this case. Instead, we adopt a universal Fourier transform method, which has been successfully used in deriving Fokker-Planck equation and Feynman-Kac equation \cite{DenisovHorsthemkeHanggi:2009,WangChenDeng:2018}.
Since the bivariate PDF $p(x,y,t)$ can be written as
$p(x,y,t)=\langle \delta(x-x(t))\delta(y-y(t))\rangle$, its Fourier transform ($x\rightarrow k_1, y\rightarrow k_2$) is
\begin{equation}
\begin{split}
  \tilde{p}(k_1,k_2,t)=& \int_{-\infty}^\infty\int_{-\infty}^\infty e^{-ik_1x-ik_2y}p(x,y,t)dxdy  \\
  =& \langle e^{-ik_1x(t)}e^{-ik_2y(t)}\rangle.
\end{split}
\end{equation}
The key point of this method is to derive the increment of $\tilde{p}(k_1,k_2,t)$ of order $\mathcal{O}(\tau)$ within a time interval $[t,t+\tau]$ when $\tau\rightarrow0$. Based on Eq. \eqref{model-NFDT} and the second equation of Eq. \eqref{model-DD}, we get the increments of $x(t)$ and $y(t)$ by omitting the higher order terms:
\begin{equation}\label{incre-x-y}
\begin{split}
x(t+\tau)-x(t)&\simeq\sqrt{2D(t)}\delta B_1(t)+F(x(t))\tau, \\[3pt]
y(t+\tau)-y(t)&\simeq f(y(t),t)\tau+g(y(t),t)\delta B_2(t),
\end{split}
\end{equation}
where $\delta B_i(t)=B_i(t+\tau)-B_i(t)$ is the increment of Brownian motion, $B_1(t)$ and $B_2(t)$ are independent from each other. By use of Eq. \eqref{incre-x-y}, the increment of $\tilde{p}(k_1,k_2,t)$ as $\delta \tilde{p}(k_1,k_2,t):=\tilde{p}(k_1,k_2,t+\tau)-\tilde{p}(k_1,k_2,t)$ can be evaluated as \\
\begin{widetext}
\begin{equation}\label{increment}
\begin{split}
\delta \tilde{p}(k_1,k_2,t)&=\langle e^{-ik_1x(t+\tau)}e^{-ik_2y(t+\tau)}\rangle-\langle e^{-ik_1x(t)}e^{-ik_2y(t)}\rangle\\
&\simeq \langle e^{-ik_1x(t)}e^{-ik_2y(t)}(e^{-ik_1(\sqrt{2D(t)}\delta B_1(t)+F(x(t))\tau)}e^{-ik_2(f(y(t),t)\tau+g(y(t),t)\delta B_2(t))}-1)\rangle\\
&\simeq \langle e^{-ik_1x(t)}e^{-ik_2y(t)}(-k_1^2D(t)\tau-ik_1F(x(t))\tau-ik_2f(y(t),t)\tau-\frac{1}{2}k_2^2g^2(y(t),t)\tau)\rangle,
\end{split}
\end{equation}
where we perform the ensemble average on $\delta B_1(t)$ and $\delta B_2(t)$ in the last line. More precisely, Eq. \eqref{incre-x-y} implies that both $x(t)$, $y(t)$ and $D(t)$ only depend on the increments $B_i$ of Brownian motion before time $t$, and thus they are independent from the increment $\delta B_i(t)$. We deal with the last two exponential functions in the second line by Taylor's series and only retain the terms of order $\mathcal{O}(\tau)$ as the last line shows.
Then dividing Eq. \eqref{increment} by $\tau$ on both sides, and taking the limit $\tau\rightarrow 0$, one arrives at
\begin{equation}
\begin{split}
\frac{\partial}{\partial t}\tilde{p}(k_1,k_2,t)=&-k^2_1\langle D(t)e^{-ik_1x(t)}e^{-ik_2y(t)}\rangle-ik_1\langle F(x(t))e^{-ik_1x(t)}e^{-ik_2y(t)}\rangle\\
&-ik_2\langle f(y(t),t)e^{-ik_1x(t)}e^{-ik_2y(t)}\rangle-\frac{1}{2}k^2_2\langle g^2(y(t),t)e^{-ik_1x(t)}e^{-ik_2y(t)}\rangle.
\end{split}
\end{equation}
\end{widetext}
Using the relation $D(t)=y^2(t)$ on the first term on the right-hand side, and performing inverse Fourier transform, we obtain the Fokker-Planck equation for the bivariate PDF $p(x,y,t)$ as
\begin{equation}\label{FKE2}
\begin{split}
\frac{\partial}{\partial t}p(x,y,t)&=\left(-\frac{\partial}{\partial x}F(x)+y^2\frac{\partial^2}{\partial x^2}\right)p(x,y,t)\\
&~~~+\left(-\frac{\partial}{\partial y}f(y,t)+\frac{1}{2}\frac{\partial^2}{\partial y^2}g^2(y,t)\right)p(x,y,t).
\end{split}
\end{equation}

Comparing the Fokker-Planck equations \eqref{FKE1} and \eqref{FKE2} for two different models in Eqs. \eqref{model-FDT} and \eqref{model-NFDT}, respectively, we find the main difference is embodied at the term containing external force $F(x)$. The former is $y^2F(x)$, i.e., $D(t)F(x)$ due to $D(t)=y^2(t)$ while the latter is $F(x)$. This difference is consistent to the discrepancy between the original models, i.e., $D(t)F(x)$ versus $F(x)$ in Eqs. \eqref{model-FDT} and \eqref{model-NFDT}. Actually, the Fokker-Planck equations \eqref{FKE1} can also be derived with the method of Fourier transform as Eq. \eqref{FKE2} by replacing $F(x)$ with $D(t)F(x)$ in the procedure.

Although the procedure of deriving the two Fokker-Planck equations looks a little complicated, the final form of Fokker-Planck equations can be understood in a simple way.
With a given $D(t)$, the corresponding Fokker-Planck equations governing the PDF $p(x,t)$ of displacement are
\begin{equation}\label{FK1}
\begin{split}
\frac{\partial}{\partial t}p(x,t)=D(t)\left[-\frac{\partial}{\partial x}F(x)+\frac{\partial^2}{\partial x^2}\right]p(x,t),
\end{split}
\end{equation}
and
\begin{equation}\label{FK2}
\begin{split}
\frac{\partial}{\partial t}p(x,t)=\left[-\frac{\partial}{\partial x}F(x)+D(t)\frac{\partial^2}{\partial x^2}\right]p(x,t)
\end{split}
\end{equation}
for Eqs. \eqref{model-FDT} and \eqref{model-NFDT}, respectively.
Then taking Eq. \eqref{FKE2} as an example, the terms on right-hand side can be divided into two parts. The first two terms are the ones in Fokker-Planck equation \eqref{FK2} by replacing $D(t)$ with $y^2$, while the last two terms come from the Fokker-Planck equation governing the PDF $p(y,t)$.
Albeit $D(t)$ is a diffusion process here, when we derive the Fokker-Planck equation governing the bivariate PDF $p(x,y,t)$, the role of $D(t)$ at the Fokker-Planck equation acts similarly to a deterministic function.


\section{Constant force field in Eq. (\ref{model-NFDT})}\label{Sec4}
For a comparison with the force-free case of Brownian yet non-Gaussian diffusion in Ref. \cite{ChechkinSenoMetzlerSokolov:2017}, we take $y(t)$ to be the Ornstein-Uhlenbeck process in the following discussions.
Let us first focus on the case that a constant force $F$ acts on the model \eqref{model-NFDT} where the FDT is broken. In this case, the Langevin system is written as
\begin{equation}\label{model_not_cons}
\begin{split}
\frac{d}{d t}x(t)&=\sqrt{2D(t)}\xi(t)+F,\\
D(t)&=y^2(t),\\
\frac{d}{d t}y(t)&=-y(t)+\eta(t).
\end{split}
\end{equation}
Based on the first equation, the process $x(t)$ can be written as
\begin{equation}\label{x-xf}
  x(t)=x_0(t)+Ft,
\end{equation}
where $x_0(t)$ denotes the trajectory of a free particle satisfying $dx_0(t)/dt=\sqrt{2D(t)}\xi(t)$ \cite{ChechkinSenoMetzlerSokolov:2017}. By the relation in Eq. \eqref{x-xf}, one has
\begin{equation}\label{relation}
\langle\Delta x^n(t)\rangle:=\langle (x(t)-\langle x(t)\rangle)^n\rangle=\langle x^n_0(t)\rangle,
\end{equation}
where $\langle x(t)\rangle=Ft$, and $x_0(t)$ is unbiased due to the symmetry of $\xi(t)$. Therefore, the ensemble-averaged MSD is $\langle \Delta x^2(t)\rangle=\langle x^2_0(t)\rangle\simeq t$.
The constant force here does not change the diffusion behavior and behaves as a decoupled force, which implies that model \eqref{model_not_cons} is Galilei invariant \cite{MetzlerKlafter:2000,CairoliKlagesBaule:2018,ChenWangDeng:2019-2}.
In addition, the drift $Ft$ dominates the diffusion process, and it holds that
\begin{equation}\label{moments1}
\langle x^n(t)\rangle\simeq F^nt^n.
\end{equation}
The relation between the first moment for the case with a constant force and the second moment of a free particle is
\begin{equation}
  \langle x(t)\rangle \simeq F\langle x^2_0(t)\rangle,
\end{equation}
which does not satisfy the Einstein relation in Eq. \eqref{ER}. This also relates to the violation of the FDT in Eq. \eqref{model-NFDT}.

Based on the moments, we can calculate the kurtosis to evaluate the deviation of the shape of a PDF from Gaussian distribution. The kurtosis of a one-dimensional Gaussian process is equal to $3$. Now for a biased process, the kurtosis is defined as
\begin{equation}\label{kurtosis}
\begin{split}
K=\frac{\langle \Delta x^4(t)\rangle}{\langle \Delta x^2(t)\rangle^2}.
\end{split}
\end{equation}
By use of Eq. \eqref{relation}, the kurtosis of the random diffusivity process under a constant force is
\begin{equation}\label{K1}
\begin{split}
K=\frac{\langle x^4_0(t)\rangle}{\langle x^2_0(t)\rangle^2}\simeq \left\{
\begin{array}{ll}
  9, &~ t\rightarrow 0, \\[5pt]
  3,  & ~t\rightarrow\infty,
\end{array}\right.
\end{split}
\end{equation}
consistent to the force-free case in Ref. \cite{ChechkinSenoMetzlerSokolov:2017}, where the PDF exhibits a crossover from exponential distribution to Gaussian distribution.

To be more delicate than the kurtosis, the explicit expression of the PDF $p(x,t)$ can be obtained through a translation of the PDF $p_0(x,t)$ of free particles to the positive direction with magnitude $Ft$, i.e.,
\begin{equation}\label{p1jianjin}
\begin{split}
p(x,t)&=p_0(x-Ft,t) \\[3pt]
&\simeq \left\{
\begin{array}{ll}
  \frac{1}{\pi t^{1/2}}K_0\left(\frac{x-Ft}{t^{1/2}}\right), &~ t\rightarrow 0,  \\[9pt]
  \frac{1}{(2\pi t)^{1/2}}\exp\left(-\frac{(x-Ft)^2}{2t}\right),  & ~t\rightarrow\infty,
\end{array}\right.
\end{split}
\end{equation}
where the expression of $p_0(x,t)$ is explicitly given in Eqs. (63) and (79) of Ref. \cite{ChechkinSenoMetzlerSokolov:2017} and $K_0$ is the Bessel function \cite{GradshteynRyzhikGeraniumsTseytlin:1980}.
In the short time limit, considering the asymptotics $K_0(z)\simeq \sqrt{\frac{\pi}{2z}}e^{-z}$ for $z\rightarrow\infty$, we have
\begin{equation}\label{PDF-short1}
\begin{split}
p(x,t)\simeq \frac{1}{\sqrt{2\pi|x-Ft|t^{1/2}}}\exp\left(-\frac{|x-Ft|}{t^{1/2}}\right),
\end{split}
\end{equation}
being an exponential distribution centered at $Ft$.

On the other hand, the short time asymptotics can be obtained from a superstatistical approach. For the time shorter than the diffusivity correlation time of the Ornstein-Uhlenbeck process, the diffusivity does not change considerably, and thus the initial condition in equilibrium of the Ornstein-Uhlenbeck process describes an ensemble of particles which diffuse with their own diffusion coefficient, resulting in a superstatistical result \cite{ChechkinSenoMetzlerSokolov:2017}.
In detail, the PDF $p_s(x,t)$ in superstatistical sense is given as the weighted average of a single Gaussian distribution $G(x,t|D)$ over the stationary distribution $p_D(D)$ of diffusivity $D$. The stationary distribution $p_D(D)$ can be obtained through the stationary distribution $f_{\textrm{st}}(y)=e^{-y^2}/\sqrt{\pi}$ of Ornstein-Uhlenbeck process in Eq. \eqref{model_not_cons}, i.e., \cite{ChechkinSenoMetzlerSokolov:2017}
\begin{equation}
\begin{split}
    p_D(D)=\int_{-\infty}^\infty f_{\textrm{st}}(y)\delta(D-y^2)dy
    =\frac{1}{\sqrt{\pi D}}e^{-D}.
\end{split}
\end{equation}
Then, it holds that
\begin{equation}
\begin{split}
p_s(x,t)&=\int_0^\infty p_D(D)G(x,t|D)dD  \\
&=\int_0^\infty \frac{1}{\sqrt{\pi D}}e^{-D} \cdot\frac{1}{\sqrt{4\pi Dt}}e^{-\frac{(x-Ft)^2}{4Dt}}dD \\
&=\frac{1}{\pi t^{1/2}}K_0\left(\frac{x-Ft}{t^{1/2}}\right),
\end{split}
\end{equation}
which is consistent to the short time asymptotics in Eq. \eqref{p1jianjin}.


\section{Constant force field in Eq. (\ref{model-FDT})}\label{Sec5}

The case that the constant force affects the diffusing diffusivity model \eqref{model-FDT} satisfying the FDT is
\begin{equation}\label{model_cons}
\begin{split}
\frac{d}{d t}x(t)&=\sqrt{2D(t)}\xi(t)+D(t)F,\\
D(t)&=y^2(t),\\
\frac{d}{d t}y(t)&=-y(t)+\eta(t).
\end{split}
\end{equation}
Similar to the way of deriving Fokker-Planck equation in Eq. \eqref{split}, it also brings convenience to rewrite the first equation of Eq. \eqref{model_cons} into a Langevin equation in the subordinated form, i.e.,
\begin{equation}\label{model_cons2}
\begin{split}
\frac{d}{d s}x(s)&=\sqrt{2}\xi(s)+F,\\
\frac{d }{d t}s(t)&=D(t),
\end{split}
\end{equation}
where the displacement is denoted as a compound process $x(t):=x(s(t))$.
Due to the independence between the two equations in Eq. \eqref{model_cons2}, it holds that
\begin{equation}\label{split-relation2}
\begin{split}
p(x,t)=\int_0^\infty G(x,s)O(s,t)ds,
\end{split}
\end{equation}
where $G(x,s)$ is the PDF of finding a Brownian particle under a constant force at position $x$ at time $s$, and $O(s,t)$ is the PDF of finding process $s(t)$ taking the value $s$ at time $t$. Therefore, $G(x,s)$ is a Gaussian distribution centered at $Fs$, i.e., $G(x,s)=\frac{1}{\sqrt{4\pi s}}e^{-\frac{(x-Fs)^2}{4s}}$ and $\tilde{G}(k,s)=e^{-ikFs-s k^2}$ in Fourier space ($x\rightarrow k$). Then we perform Fourier transform on Eq. \eqref{split-relation2} and obtain
\begin{equation}\label{pkt2}
\begin{split}
\tilde{p}(k,t)&=\int_0^\infty \tilde{G}(k,s)O(s,t)ds\\
&=\int_0^\infty e^{-(ikF+k^2)s} O(s,t)ds\\
&=\hat{O}(ikF+k^2,t),
\end{split}
\end{equation}
where $\hat{O}(ikF+k^2,t)$ denotes the Laplace transform $(s\rightarrow ikF+k^2)$ of the PDF $O(s,t)$. By use of the known result on the Laplace transform of $O(s,t)$ for the integrated square of the Ornstein-Uhlenbeck process \cite{Dankel:1991,ChechkinSenoMetzlerSokolov:2017}, we have
\begin{equation}\label{pkt-exact}
\begin{split}
    \tilde{p}(k,t)&=\exp\left(\frac{t}{2}\right)\left/\left[\frac{1}{2}\left(\sqrt{1+2\tilde{k}}+\frac{1}{\sqrt{1+2\tilde{k}}}\right)\right.\right.\\
&~~~\left.\times\textrm{sinh}\left(t\sqrt{1+2\tilde{k}}\right)
+\textrm{cosh}\left(t\sqrt{1+2\tilde{k}}\right)\right]^{\frac{1}{2}},
\end{split}
\end{equation}
where $\tilde{k}=ikF+k^2$. In order to satisfy the condition of Eq. \eqref{pkt-exact} proposed in Ref. \cite{Dankel:1991}, we assume that the initial position $y_0$ in Eqs. \eqref{model_not_cons} and \eqref{model_cons} obeys the equilibrium distribution of the Ornstein-Uhlenbeck process $y(t)$, i.e., a Gaussian distribution with mean zero and variance $1/2$:
\begin{equation}\label{EquilibriumDistribution}
  p_{\textrm{eq}}(y_0)=\frac{1}{\sqrt{\pi}}\exp(-y_0^2).
\end{equation}
This equilibrium distribution is also employed throughout all the simulations in Sec. \ref{Sec6}.
The expression of $\tilde{p}(k,t)$ in Eq. \eqref{pkt-exact} is exact for any time $t$, based on which we can evaluate the asymptotic moments and PDFs in $x$ space for short and long times.

For the moments, performing the Taylor expansion of exponential function in Eq. \eqref{pkt2} yields
\begin{equation}
\begin{split}
\tilde{p}(k,t)
&=\int_0^\infty e^{-\tilde{k}s} O(s,t)ds\\
&=1-\tilde{k}\langle s(t)\rangle+\frac{\tilde{k}^2}{2}\langle s^2(t)\rangle +\cdots.
\end{split}
\end{equation}
Then we use the formula $\langle x^n(t)\rangle=i^n\left.\frac{\partial^n}{\partial k^n}p(k,t)\right|_{k=0}$ and obtain the first four moments
\begin{equation}\label{Moments}
\begin{split}
\langle x(t)\rangle&=F\langle s(t)\rangle,\\
\langle x^2(t)\rangle&=2\langle s(t)\rangle+F^2\langle s^2(t)\rangle,\\
\langle x^3(t)\rangle&=6F\langle s^2(t)\rangle+F^3\langle s^3(t)\rangle,\\
\langle x^4(t)\rangle&=12\langle s^2(t)\rangle+12F^2\langle s^3(t)\rangle+F^4\langle s^4(t)\rangle.\\
\end{split}
\end{equation}
To obtain both short time and long time asymptotics, we need the accurate expressions of $\langle s^n(t)\rangle$, which are presented in Appendix \ref{App2}.
We find that for long times,
\begin{equation}\label{moments2}
  \langle x^n(t)\rangle\simeq \frac{F^n}{2^n}t^n.
\end{equation}
The relation between the first moment for the case with a constant force and the second moment for the force-free case is
\begin{equation}\label{ER-S}
  \langle x(t)\rangle \simeq \frac{F}{2}\langle x^2_0(t)\rangle,
\end{equation}
which satisfies the Einstein relation in Eq. \eqref{ER}.

Based on Eq. \eqref{Moments} and the accurate expression of $\langle s^n(t)\rangle$ in Appendix \ref{App2}, the MSD is equal to
\begin{equation}\label{M2}
\begin{split}
\langle\Delta x^2(t)\rangle&=\left(\frac{F^2}{2}+1\right)t+\frac{F^2}{4}(e^{-2t}-1)\\
&\simeq \left\{
\begin{array}{ll}
  t, &~t\rightarrow 0,  \\[5pt]
  \left(\frac{F^2}{2}+1\right)t,  & ~t\rightarrow \infty.
\end{array}\right.
\end{split}
\end{equation}
When $F=0$, it recovers to the constantly normal diffusion $\langle x^2(t)\rangle=t$.
Under the influence of a constant force, the particles still exhibit normal diffusion, but the effective diffusion coefficient increases from $1$ to $F^2/2+1$ as time goes.
Similar to the MSD in Eq. \eqref{M2}, the asymptotic expressions of fourth moment can be obtained from Eqs. \eqref{Moments} and Appendix \ref{App2}:
\begin{equation}\label{M4}
\begin{split}
\langle \Delta x^4(t)\rangle\simeq \left\{
\begin{array}{ll}
  9t^2, &~t\rightarrow 0,  \\[5pt]
  3\left(\frac{F^2}{2}+1\right)^2t^2,  & ~t\rightarrow \infty.
\end{array}\right.
\end{split}
\end{equation}
The constant force enhances the diffusion slightly since it only increases the diffusion coefficient without changing the diffusion behavior at long time limit.

Here we also evaluate the kurtosis to predict the shape of the PDF $p(x,t)$ for the case satisfying FDT. Considering the definition of kurtosis in Eq. \eqref{kurtosis}, and combining the moments in Eqs. \eqref{M2} and \eqref{M4}, we find
\begin{equation}\label{K2}
\begin{split}
K\simeq \left\{
\begin{array}{ll}
  9, &~ t\rightarrow 0, \\[5pt]
  3,  & ~t\rightarrow\infty.
\end{array}\right.
\end{split}
\end{equation}
Surprisingly, this result is consistent to the force-free case and the result in Eq. \eqref{K1}, which implies a possible crossover of PDF from exponential distribution to Gaussian distribution as the force-free case.

For the asymptotic expression of PDF $p(x,t)$, taking $t\rightarrow 0$ in Eq. \eqref{pkt-exact} yields
\begin{equation}\label{pkt-ST}
\begin{split}
\tilde{p}(k,t)\simeq t^{-\frac{1}{2}}\left(ikF+k^2+\frac{1}{t}\right)^{-\frac{1}{2}}.
\end{split}
\end{equation}
The normalization of the asymptotic PDF can be verified by $\tilde{p}(k=0,t)=1$. The inverse Fourier transform of $\tilde{p}(k,t)$ cannot be obtained easily. Since $t\rightarrow 0$, whenever $k\rightarrow0$ or $k\rightarrow\infty$, the imaginary part in the brackets of Eq. \eqref{pkt-ST} is much smaller than the real part, i.e., $kF\ll k^2+1/t$. Therefore, the constant force $F$ here only makes a slight biase on the original PDF. The expression of the biased PDF will be explicitly given through a superstatistical approach in the following. The asymptotic behavior at short time limit should be consistent to the corresponding superstatistical result.

In superstatistical approach, the effective PDF $p_s(x,t)$ is given as the weighted average of the conditional Gaussian distribution over the stationary distribution $p_D(D)$, i.e.,
\begin{equation}\label{short}
\begin{split}
p_s(x,t)&=\int_0^\infty p_D(D)G(x,t|D)dD  \\
&=\frac{1}{\sqrt{4\pi^2t}}e^{\frac{Fx}{2}}\int_0^\infty \frac{1}{D}e^{-D\left(1+\frac{F^2}{4}t\right)}e^{-\frac{x^2}{4Dt}}dD\\
&=\frac{1}{\pi\sqrt{t}}e^{\frac{Fx}{2}}K_0\left(\frac{\sqrt{4+F^2t}x}{2\sqrt{t}}\right),
\end{split}
\end{equation}
where  $G(x,t|D)=\frac{1}{\sqrt{4\pi Dt}}e^{-\frac{(x-FDt)^2}{4Dt}}$ has been used.
Then using the asymptotic behavior $K_0(z)\simeq \sqrt{\frac{\pi}{2z}}e^{-z}$ as $z\rightarrow\infty$, we arrive at
\begin{equation}\label{psxt}
\begin{split}
p_s(x,t)&\simeq \frac{1}{\sqrt{2\pi|x|t^{1/2}}}\frac{1}{\sqrt{(1+F^2t/4)^{1/2}}}  \\
&~~~~\times
\exp\left(\frac{Fx}{2}-\sqrt{1+F^2t/4}\frac{|x|}{t^{1/2}}\right).
\end{split}
\end{equation}
Corresponding to the short time aymptotics in Eq. \eqref{pkt-ST}, we take $t\ll 4/F^2$ in Eq. \eqref{psxt}, and obtain
\begin{equation}\label{psxt-ST}
\begin{split}
p_s(x,t)\simeq p_0(x,t) \exp\left(\frac{Fx}{2}\right),
\end{split}
\end{equation}
where
\begin{equation}\label{p0xt}
  p_0(x,t)=\frac{1}{\sqrt{2\pi|x|t^{1/2}}}\exp\left(-\frac{|x|}{t^{1/2}}\right)
\end{equation}
is the PDF of free particles in the superstatistical case. It can be seen that the constant force only adds a time-independent correction $e^{Fx/2}$ to the PDF of free particles at short time limit. Compared with the exponential part in $p_0(x,t)$, the exponential correction $e^{Fx/2}$ is negligible for short time since the exponential coefficient satisfies $F/2\ll 1/t^{1/2}$. This result is consistent to the previous kurtosis $K\simeq9$ in Eq. \eqref{K2} at short time limit and the analyses following Eq. \eqref{pkt-ST}.

On the other hand, the long time asymptotics $t\gg 4/F^2$ of $p_s(x,t)$ is
\begin{equation}\label{psxt-LT}
\begin{split}
p_s(x,t)\simeq p_0(x,t)C_F(x,t),
\end{split}
\end{equation}
where
\begin{equation}
  \begin{split}
    C_F(x,t)=\left\{
    \begin{array}{ll}
    \frac{1}{\sqrt{Ft^{1/2}/2}}\exp\left(\frac{x}{2t^{1/2}}\right),  &  x>0,  \\[5pt]
    \frac{1}{\sqrt{Ft^{1/2}/2}}\exp\left(Fx\right),  & x<0.
    \end{array}\right.
  \end{split}
\end{equation}
The constant force makes the PDF biased to the positive direction, i.e., decaying more slowly for $x>0$ but faster for $x<0$. Furthermore, the change in PDF at $x<0$ is more obvious than that at $x>0$. For long time limit, the exponential coefficient $F$ in $C_F(x,t)$ is much larger than $t^{-1/2}$ in $p_0(x,t)$, i.e., $F\gg 1/t^{1/2}$. So the dominating term of decaying when $x<0$ is $e^{Fx}$.


In contrast to the superstatistical results above, the real long time asymptotics of the Langevin system in Eq. \eqref{model_cons} can be found by taking $t\rightarrow \infty$ in Eq. \eqref{pkt-exact}. The asymptotic result is
\begin{equation}
\begin{split}
\tilde{p}(k,t)\simeq \frac{\sqrt{2}\exp\left(\frac{t}{2}(1-\sqrt{1+2\tilde{k}})\right)}
{\left[\frac{1}{2}\left(\sqrt{1+2\tilde{k}}+\frac{1}{\sqrt{1+2\tilde{k}}}\right)+1\right]^{1/2}}.
\end{split}
\end{equation}
Then we consider the large-$x$ behavior by taking $k\rightarrow 0$, and obtain
\begin{equation}
\begin{split}
\tilde{p}(k,t)\simeq \exp\left(-\frac{iFt}{2}k -\frac{(2+F^2)t}{4}k^2\right).
\end{split}
\end{equation}
With the inverse Fourier transform, the Gaussian distribution with mean $Ft/2$ and variance $(F^2/2+1)t$ is obtained:
\begin{equation}\label{long}
\begin{split}
p(x,t)\simeq \frac{1}{\sqrt{2\pi\left(\frac{F^2}{2}+1\right)t}}
\exp\left(-\frac{\left(x-\frac{F}{2}t\right)^2}{2\left(\frac{F^2}{2}+1\right)t}\right).
\end{split}
\end{equation}
This Gaussian shape is also consistent to the previous kurtosis $K\simeq3$ in Eq. \eqref{K2} at long time limit.

\section{Simulations}\label{Sec6}
In all our simulations, the initial position $y_0$ of the Langevin systems in Eqs. \eqref{model_not_cons} and \eqref{model_cons} is taken from the equilibrium distribution $N(0,1/\sqrt{2})$ in Eq. \eqref{EquilibriumDistribution}, and the two models in Eqs. \eqref{model_not_cons} and \eqref{model_cons} are recorded briefly as ``Model I'' and ``Model II'', respectively.
For a clear comparison between the two models, we put the simulation results of the same observable in one figure, with their moments in Fig. \ref{fig1}, kurtosis in Fig. \ref{fig2}, short-time PDFs in Fig. \ref{fig3}, and long-time PDFs in Fig. \ref{fig4}.

\begin{figure}
  \centering
  % Requires \usepackage{graphicx}
  \includegraphics[scale=0.5]{fig1}\\
  \caption{Moments $\langle x^n(t)\rangle$ with $n=1,2,3,4$. Model I (in red) and Model II (in blue) represent the Langevin systems in Eqs. \eqref{model_not_cons} and \eqref{model_cons}, respectively.
  %From the bottom to the top, each two lines represent the moment order $n$ from $n=1$ to $n=4$.
  The circle and star markers denote the simulation results, while the solid and dashed lines denote the theoretical results in Eqs. \eqref{moments1} and \eqref{moments2}, respectively. Based on Eqs. \eqref{moments1} and \eqref{moments2}, the two lines with the same $n$ are parallel for two models, i.e., $\langle x^n(t)\rangle_I=2^n\langle x^n(t)\rangle_{I\!I}$. Correspondingly, each two lines (or markers) from the bottom to the top represent the first, second, third, and fourth moments, respectively. Parameters: $T=10^3$, $F=2$, and $10^3$ samples are used for ensemble average.
}\label{fig1}
\end{figure}

\begin{figure}
  \centering
  % Requires \usepackage{graphicx}
  \includegraphics[scale=0.5]{fig2}\\
  \caption{Kurtosis (defined in Eq. \eqref{kurtosis}) in Model I (in red) and Model II (in blue) which represent the Langevin systems in Eqs. \eqref{model_not_cons} and \eqref{model_cons}, respectively. For the two models, the circle and star markers denote the simulation results, while the solid and dashed lines denote the theoretical results in Eqs. \eqref{K1-exact} and \eqref{K2-exact}, respectively. The kurtosis in Eqs. \eqref{K1} and \eqref{K2} both have the same asymptotics as the force-free case.
In contrast to the monotone decreasing behavior of the kurtosis line of Model I, that of Model II has a maximum value around $t=0.5$. Parameters: $T=10^2$, $F=2$, and $10^6$ samples are used for ensemble average.
}\label{fig2}
\end{figure}

In Fig. \ref{fig1}, we simulate the first four moments $\langle x^n(t)\rangle$ of two models, which agree with the theoretical results very well. According to the theoretical results in Eqs. \eqref{moments1} and \eqref{moments2}, we find the moments of two models only differ by a constant multiplier, i.e.,
\begin{equation}
  \langle x^n(t)\rangle_I=2^n\langle x^n(t)\rangle_{I\!I}.
\end{equation}
As a result, the solid and dashed lines (or circle and star markers) in Fig. \ref{fig1} are parallel for the same $n$.



In Fig. \ref{fig2}, we simulate the kurtosis for two models. They have the same asymptotic results in Eqs. \eqref{K1} and \eqref{K2} with a crossover from $K=9$ at the beginning to $K=3$ at the infinity. In addition to the asymptotic results, the exact expressions of kurtosis can be obtained by use of the definition in Eq. \eqref{kurtosis} and the first four moments $\langle x^n(t)\rangle$ in Eqs. \eqref{relation} and \eqref{Moments}. For convenience, the exact expressions are presented in Appendix \ref{App3}, where the latter (Eq. \eqref{K2-exact}) recovers the former (Eq. \eqref{K1-exact}) when $F=0$. The kurtosis of Model I is the same as the force-free case \cite{ChechkinSenoMetzlerSokolov:2017} due to its Galilei invariant property.
In contrast to the monotone decreasing kurtosis from $9$ to $3$ in Model I, the kurtosis of Model II has a maximum value around $t=0.5$, which means that for short time, the PDF of Model II undergoes a significant deviation from the Gaussian distribution. The reason can be found from the asymptotic PDF at short time limit in Eq. \eqref{psxt-ST}. The additional term $e^{Fx/2}$ brings a biase to the original exponential distribution $p_0(x,t)$ in Eq. \eqref{p0xt}. At long time limit, the PDF converges to the Gaussian distribution in Eq. \eqref{long}, corresponding to the monotone decreasing kurtosis after $t=0.5$ in Model II.



The asymptotic PDFs of two models for short time are presented in Fig. \ref{fig3}. The corresponding theoretical results are given in Eqs. \eqref{PDF-short1} and \eqref{psxt-ST}, respectively. For Model I, the PDF is exactly a translation to the positive direction with the magnitude $x=Ft$ of the original PDF $p_0(x,t)$ for force-free case. In contrast to Model I, the PDF of Model II is asymmetric due to the term $e^{Fx/2}$ in Eq. \eqref{psxt-ST}. It can be found that the lines in a semi-log graph (Fig. \ref{fig3}) are not exactly straight. The slight deviation from straight lines comes from the power-law correction term $|x|^{-1/2}$ in $p_0(x,t)$ in Eq. \eqref{p0xt}.

\begin{figure}
  \centering
  % Requires \usepackage{graphicx}
  \includegraphics[scale=0.5]{fig3}\\
  \caption{Short-time PDFs in Model I (in red) and Model II (in blue) which represent the Langevin systems in Eqs. \eqref{model_not_cons} and \eqref{model_cons}, respectively. For the two models, the circle and star markers denote the simulation results, while the solid and dashed lines denote the theoretical results in Eqs. \eqref{PDF-short1} and \eqref{psxt-ST}, respectively. The PDF of Model I is a symmetric exponential distribution with the center at $x=Ft$, while the PDF of Model II is an asymmetric skewed exponential distribution.
Parameters: $T=0.1$, $F=1$, and $10^7$ samples are used for ensemble average.}\label{fig3}
\end{figure}


The asymptotic PDFs of two models for long time are presented in Fig. \ref{fig4}. The corresponding theoretical results are given in Eqs. \eqref{p1jianjin} and \eqref{long}, respectively. Corresponding to the behavior of the kurtosis tending to $3$ in Fig. \ref{fig2}, the PDFs for two models both converge to the Gaussian distribution at long time limit. As the shape of PDFs in Fig. \ref{fig4} shows, the PDF of Model I has the mean $Ft$ and the variance $t$, while the one of Model II has a smaller mean $Ft/2$ but a larger variance $(F^2/2+1)t$. This feature comes from the fact that the constant force $F$ is multiplied by a stochastic process $D(t)$ which enhances the fluctuation, and that the mean of $D(t)$ at steady state is $1/2$ which weakens the effective drift by half.


\begin{figure}
  \centering
  % Requires \usepackage{graphicx}
  \includegraphics[scale=0.5]{fig4}\\
  \caption{Long-time PDFs in Model I (in red) and Model II (in blue) which represent the Langevin systems in Eqs. \eqref{model_not_cons} and \eqref{model_cons}, respectively. For the two models, the circle and star markers denote the simulation results, while the solid and dashed lines denote the theoretical results in Eqs. \eqref{p1jianjin} and \eqref{long}, respectively. Both the PDFs of two models are Gaussian shapes. The PDF of Model I has the mean $Ft$ and the variance $t$, while the one of Model II has a smaller mean $Ft/2$ but a larger variance $(F^2/2+1)t$.
Parameters: $T=20$, $F=1$, and $10^7$ samples are used for ensemble average.}\label{fig4}
\end{figure}



\section{Conclusion}\label{Sec7}
%This paper mainly focus on the effect of external force acting on the random diffusivity models.
Much attention has been taken to the scenarios of how external force (or constant force) influences a dynamic system with a power-law distributed waiting time \cite{BarkaiFleurov:1998,MetzlerKlafter:2000,FroembergBarkai:2013-3,ChenWangDeng:2019-2,ChenWangDeng:2019-3}.
This paper extends this issue to the random diffusivity model with a diffusing diffusivity $D(t)$, and explores how the diffusing diffusivity $D(t)$ acts in a system under an external force.
Considering the importance of the FDT in the statistical mechanics of nonequilibrium dynamics, we build two kinds of random diffusivity models with an external force based on whether the FDT satisfies or not.

The main studies on the two models can be divided into two parts: one derives the Fokker-Planck equation of random diffusivity models with arbitrary external force, and another one investigates in detail some common quantities by taking a specific constant force.
In the first part, the Fokker-Planck equations for the bivariate PDF $p(x,y,t)$ of
two random diffusivity models under an arbitrary external force field are derived in Eqs. \eqref{FKE1} and \eqref{FKE2}. Corresponding to the fact that the only difference between the original Langevin equations \eqref{model-FDT} and \eqref{model-NFDT} is $F(x)$ versus $D(t)F(x)$, the difference between the Fokker-Planck equations is only embodied at the external force term, $F(x)$ versus $y^2F(x)$. Although $D(t)$ is a diffusion process, the role of $D(t)$ at the expression of Fokker-Planck equations is similar to a deterministic function.
The structure of the derived Fokker-Planck equations has striking character. Due to the independence between the evolution of concerned process $x(t)$ and auxiliary process $y(t)$, the right-hand side of Fokker-Planck equations \eqref{FKE1} and \eqref{FKE2} can be divided into two parts, being the terms in the corresponding Fokker-Planck equation governing the PDF $p(x,t)$ and $p(y,t)$, respectively.

In the second part, we investigate the case with constant force field and the diffusivity $D(t)$ being
the square of Ornstein-Uhlenbeck process by studying the moments, Einstein relation, the kurtosis and the asymptotic behaviors of the PDF in detail.
For random diffusivity model in Eq. \eqref{model_not_cons} with the FDT broken, we establish the relation between the concerned process $x(t)$ under the effect of a constant force and the displacement $x_0(t)$ of a free particle by $x(t)=x_0(t) +Ft$. Thus we find this model is Galilei invariant, similar to the discussed anomalous processes \cite{MetzlerKlafter:2000,CairoliKlagesBaule:2018,ChenWangDeng:2019-2}. The diffusion behavior is not changed by the constant force. The mean value is $Ft$ and the Einstein relation is not valid in this model. Compared with the PDF of force-free case, the PDF is translated to the positive direction with a biase $Ft$, with the kurtosis and the asymptotic behaviors of PDF unchanged.

For the random diffusivity model in Eq. \eqref{model_cons} satisfying the FDT, the results are quite different from the force-free case. The theoretical derivations are based on the technique of splitting the first equation of Eq. \eqref{model_cons} into a Langevin equation in subordinated form. We find the mean value of displacement is $\langle x(t)\rangle=Ft/2$ in this case, satisfying the Einstein relation Eq. \eqref{ER-S}. Although the kurtosis has the same asymptotic behavior at $t\rightarrow0$ and $t\rightarrow\infty$, it is not monotone any more. It increases for short time and reaches the maximum around $t=0.5$ as Fig. \ref{fig2} shows.
For long time, the PDF surprisingly converges to a Gaussian distribution as the force-free case, while the PDF in short-time limit is biased due to a correction $e^{Fx/2}$ compared with the force-free case.

Many significant differences between the two models imply that the FDT also plays an important role in random diffusivity systems. Through detailed analyses on the kurtosis and the shape of PDF, the model satisfying the FDT shows many interesting dynamic behaviors due to the existence of random diffusivity $D(t)$. These results will bring benefits to the discussions on how anomalous diffusion particles response to the external force in more random diffusivity systems.



\section*{Acknowledgments}
This work was supported by the National Natural Science
Foundation of China under Grant No. 12105145, the Natural Science Foundation of Jiangsu Province under Grant No. BK20210325.

\appendix

\section{Equivalence between Eqs. \eqref{model-FDT} and \eqref{split}}\label{App1}
The main idea of proving the equivalence is to combine the two equations in Eq. \eqref{split} and to transform them into Eq. \eqref{model-FDT}. Noting that the diffusing diffusivity $D(t)$ is independent of the noise $\xi$, $D(t)$ can be regarded as a deterministic function and the ensemble average only acts on $\xi$ in the following.
Integrating the first equation in Eq. \eqref{split} yields
\begin{equation}\label{A1}
  x(s)=\sqrt{2}\int_0^s\xi(s')ds'+\int_0^sF(x(s'))ds',
\end{equation}
where we have assumed the initial condition $x(0)=0$.
Since the concerned process $x(t)$ has been written as a compound process $x(t):=x(s(t))$, $x(t)$ can be obtained by replacing $s$ with $s(t)$ in Eq. \eqref{A1}, i.e.,
\begin{equation}\label{A2}
  x(t)=\sqrt{2}\int_0^{s(t)}\xi(s')ds'+\int_0^{s(t)}F(x(s'))ds'.
\end{equation}
By using the second equation of Eq. \eqref{split} and performing the derivative over time $t$ on both sides of Eq. \eqref{A2}, one arrives at
\begin{equation}\label{A3}
  \frac{d}{dt}x(t)=\sqrt{2}D(t)\xi(s(t))+D(t)F(x(t)).
\end{equation}
Now the only difference between Eqs. \eqref{A3} and \eqref{model-FDT} is the first term on the right-hand side. It is sufficient to prove that they share the same correlation function since $\xi$ is white Gaussian noise. A formula about $\delta$-function
\begin{equation}
  \delta(h(x))=\sum_{i}\frac{\delta(x-x_i)}{|h'(x_i)|}
\end{equation}
will be used, where $x_i$ is the $i$th simple root of $h(x)=0$. Utilizing this formula and a truth that $s(t)$ is monotone increasing, we have
\begin{equation}
\begin{split}
    \langle\xi(s(t_1))\xi(s(t_2))\rangle
&=\delta(s(t_1)-s(t_2))  \\
&=\frac{1}{D(t_1)}\delta(t_1-t_2).
\end{split}
\end{equation}
Therefore, it can be found that both the correlation functions of the first term in Eqs. \eqref{A3} and \eqref{model-FDT} are
\begin{equation}
  2D(t_1)\delta(t_1-t_2).
\end{equation}




\section{Moments of process $s(t)$}\label{App2}
The moments of process $s(t)$ in Eq. \eqref{model_cons2} can be obtained from its PDF in Laplace space by use of the formula
\begin{equation}
  \langle s^n(t)\rangle=(-1)^n\left.\frac{\partial^n}{\partial \lambda^n}\hat{O}(\lambda,t)\right|_{\lambda=0},
\end{equation}
where $\hat{O}(\lambda,t)$ is the Laplace transform of $O(s,t)$, and \cite{Dankel:1991,ChechkinSenoMetzlerSokolov:2017}
\begin{equation}
\begin{split}
    \hat{O}(\lambda,t)&=\exp\left(\frac{t}{2}\right)\left/\left[\frac{1}{2}\left(\sqrt{1+2\lambda}+\frac{1}{\sqrt{1+2\lambda}}\right)\right.\right.\\
&~~~\left.\times\textrm{sinh}\left(t\sqrt{1+2\lambda}\right)
+\textrm{cosh}\left(t\sqrt{1+2\lambda}\right)\right]^{\frac{1}{2}}.
\end{split}
\end{equation}
\vspace{5mm}
With some tedious calculations, it holds that
%\begin{equation}
%\begin{split}
%\langle s(t)\rangle&=\frac{t}{2},\\[4pt]
%\langle s^2(t)\rangle&=\frac{1}{4}(e^{-2t}-1+2t+t^2),\\[4pt]
%\langle s^3(t)\rangle&=\frac{1}{8}\Big(3(4+5t)e^{-2t}-12+9t+6t^2+t^3\Big),\\[4pt]
%\langle s^4(t)\rangle&=\frac{1}{16}\Big(6(27+50t+25t^2)e^{-2t}+9e^{-4t} \\[2pt]
%   &~~~~-171+60t+54t^2+12t^3+t^4\Big).
%\end{split}
%\end{equation}
\begin{equation}
\langle s(t)\rangle=\frac{t}{2},
\end{equation}
\begin{equation*}
\langle s^2(t)\rangle=\frac{1}{4}(e^{-2t}-1+2t+t^2),
\end{equation*}
\begin{equation*}
\langle s^3(t)\rangle=\frac{1}{8}\Big(3(4+5t)e^{-2t}-12+9t+6t^2+t^3\Big),
\end{equation*}
\begin{equation*}
\begin{split}
\langle s^4(t)\rangle&=\frac{1}{16}\Big(6(27+50t+25t^2)e^{-2t}+9e^{-4t} \\[2pt]
   &~~~~-171+60t+54t^2+12t^3+t^4\Big).
\end{split}
\end{equation*}

\section{Exact kurtosis}\label{App3}
The exact theoretical expressions of kurtosis for two models in Eqs. \eqref{model_not_cons} and \eqref{model_cons} are
\begin{equation}\label{K1-exact}
  K=\frac{3}{t^2}(-1+e^{-2t}+2t+t^2)
\end{equation}
for Model I, and
\begin{widetext}
\begin{equation}\label{K2-exact}
  \begin{split}
  K&=\left\{-3-18F^2-\frac{171}{16}F^4+\left(6+\frac{33}{2}F^2+\frac{27}{4}F^4\right)t
    +\left(3+3F^2+\frac{3}{4}F^4\right)t^2 \right.\\
&~~~~+ \left. \left[3+\left(18+\frac{39}{2}t\right)F^2
    +\left(\frac{81}{8}+\frac{63}{4}t+6t^2\right)F^4\right]e^{-2t}+\frac{9}{16}F^4e^{-4t}  \right\}
\left/ \left[\left(\frac{F^2}{2}+1\right)t+\frac{F^2}{4}(e^{-2t}-1)\right]^2                        \right.
\end{split}
\end{equation}
for Model II.
\end{widetext}


\section*{References}
\bibliography{ReferenceCW}

\end{document}




% *** CITATION PACKAGES ***
\usepackage{cite}
\usepackage{url}

% *** MATH PACKAGES ***
\usepackage{amssymb}
\usepackage{amsthm, amsmath}
\usepackage{amsbsy}
\usepackage{color}

% *** SUBFIGURE PACKAGES ***
\usepackage{subfigure}

% *** PDF, URL AND HYPERLINK PACKAGES ***
\usepackage{cite,url}

% *** ALGORITHM ***
% \usepackage{algorithm}
% \usepackage{algpseudocode}
% \makeatletter
% \def\BState{\State\hskip-\ALG@thistlm}
% \makeatother

%%%%%% Added by Bang begin
\usepackage{dsfont}  % for indicator function
\usepackage{algorithmic}
\usepackage[ruled]{algorithm2e}
\SetKwInput{kwInitStep}{Initialization Step}
\SetKwInput{kwGibbsStep}{Gibbs Sampling Step}
\SetKwInput{kwTrainStep}{Training}
\SetKwInput{KwPredictStep}{Prediction}
%%%%%% Added by Bang end


\def\squareforqed{\IEEEQED}%\hbox{\rlap{$\sqcap$}$\sqcup$}}
\def\qed{\ifmmode\squareforqed\else{\unskip\nobreak\hfil
\penalty50\hskip1em\null\nobreak\hfil\squareforqed
\parfillskip=0pt\finalhyphendemerits=0\endgraf}\fi}

\newtheorem{theorem}{Theorem}
\newtheorem{lemma}[theorem]{Lemma}
\newtheorem{corollary}[theorem]{Corollary}
\newtheorem{claim}[theorem]{Claim}
\newtheorem{proposition}[theorem]{Proposition}
\newtheorem{predicate}[theorem]{Predicate}
\newtheorem{observation}[theorem]{Observation}
\newtheorem{openProb}{Open Problem}
\newtheorem{definition}{Definition}

\newcommand{\e}{\mathrm{e}}
\newcommand{\ud}{\mathrm{d}}
\newcommand{\E}{\boldsymbol{\mathrm{E}}}
\newcommand{\WN}{\mbox{WN}}
\newcommand{\var}{\boldsymbol{\mathbf{Var}}}
\newcommand{\cov}{\boldsymbol{\mathbf{Cov}}}
\DeclareMathOperator*{\argmax}{arg\,max}
\newcommand{\red}[1]{\textcolor{red}{#1}}
\newcommand{\blue}[1]{\textcolor{blue}{#1}}


\begin{document}

\title{Recover Fine-Grained Spatial Data \\from Coarse Aggregation}

% author names and affiliations
% use a multiple column layout for up to three different
% affiliations
\author{\IEEEauthorblockN{Bang Liu}\thanks{ This work was partially supported by the NSERC-CRD and NSERC-RGPIN grants.}
\IEEEauthorblockA{
University of Alberta\\
bang3@ualberta.ca}
\and
\IEEEauthorblockN{Borislav Mavrin}
\IEEEauthorblockA{
University of Alberta\\
mavrin@ualberta.ca}
\and
\IEEEauthorblockN{Linglong Kong}
\IEEEauthorblockA{
University of Alberta\\
lkong@ualberta.ca}
\and
\IEEEauthorblockN{Di Niu}
\IEEEauthorblockA{
University of Alberta\\
dniu@ualberta.ca}}



\maketitle
\thispagestyle{empty}
\pagestyle{empty}

\begin{abstract}
	%With the almost universal adoption of mobile devices in modern society, enormous amount of behavioral traces data are generated. One such kind of data is the cell phone activities records and its distribution. 
	%We study a challenging problem of reconstructing fine-grained spatial densities from coarse-grained measurements, i.e., the aggregate observations recorded for each subregion in a spatial field of interest. 
	In this paper, we study a new type of \emph{spatial sparse recovery} problem, that is to infer the fine-grained spatial distribution of certain density data in a region only based on the aggregate observations recorded for each of its subregions.
	One typical example of this spatial sparse recovery problem is to infer spatial distribution of cellphone activities based on aggregate mobile traffic volumes observed at sparsely scattered base stations.  
	We propose a novel \emph{Constrained Spatial Smoothing} (CSS) approach, which exploits the local continuity that exists in many types of spatial data to perform sparse recovery via finite-element methods, while enforcing the aggregated observation constraints through an innovative use of the ADMM algorithm. We also improve the approach to further utilize additional geographical attributes.
	%Understanding the spatial distribution of cell phone activities is not only of great value for telecommunication companies to conduct resource planning, but also provides insights into the study of human activity and the underlying urban ecology. However, in reality, due to technical overhead and privacy concerns related to mobile phone tracking, cell phone activity data with a fine-grained geographical distribution are not always available.
	Extensive evaluations based on a large dataset of phone call records and a demographical dataset from the city of Milan show that our approach significantly outperforms various state-of-the-art approaches, including Spatial Spline Regression (SSR).
%	Though a bunch of research works have been done based on cell phone activities, such kind of data is actually highly rare due to both technical overhead and privacy issues. Instead of having access to the distribution of cell phone activities on a whole region, in reality, usually only the aggregated volumes of cell phone activities associated with base stations that sparsely distributed within a region are observable. Therefore, how to infer the spatial distribution of cell phone activities from sparse aggregated observations is a very important problem.
%	In this paper, we study the problem of inferring cell phone activity distributions from aggregated sparse observations. We formulate the problem as sparse recovery and propose a new Constrained Spatial Smoothing approach. Our approach exploits the smoothness property of cell phone activities distribution to perform sparse spatial recovery. Besides, it is also able to incorporate external information as a regression add-on to further enhance recovery performance. Furthermore, we propose an Alternating Direction Method of Multipliers algorithm to decouple the problem and learn model parameters effectively.

\end{abstract}


% !TEX root = ../arxiv.tex

Unsupervised domain adaptation (UDA) is a variant of semi-supervised learning \cite{blum1998combining}, where the available unlabelled data comes from a different distribution than the annotated dataset \cite{Ben-DavidBCP06}.
A case in point is to exploit synthetic data, where annotation is more accessible compared to the costly labelling of real-world images \cite{RichterVRK16,RosSMVL16}.
Along with some success in addressing UDA for semantic segmentation \cite{TsaiHSS0C18,VuJBCP19,0001S20,ZouYKW18}, the developed methods are growing increasingly sophisticated and often combine style transfer networks, adversarial training or network ensembles \cite{KimB20a,LiYV19,TsaiSSC19,Yang_2020_ECCV}.
This increase in model complexity impedes reproducibility, potentially slowing further progress.

In this work, we propose a UDA framework reaching state-of-the-art segmentation accuracy (measured by the Intersection-over-Union, IoU) without incurring substantial training efforts.
Toward this goal, we adopt a simple semi-supervised approach, \emph{self-training} \cite{ChenWB11,lee2013pseudo,ZouYKW18}, used in recent works only in conjunction with adversarial training or network ensembles \cite{ChoiKK19,KimB20a,Mei_2020_ECCV,Wang_2020_ECCV,0001S20,Zheng_2020_IJCV,ZhengY20}.
By contrast, we use self-training \emph{standalone}.
Compared to previous self-training methods \cite{ChenLCCCZAS20,Li_2020_ECCV,subhani2020learning,ZouYKW18,ZouYLKW19}, our approach also sidesteps the inconvenience of multiple training rounds, as they often require expert intervention between consecutive rounds.
We train our model using co-evolving pseudo labels end-to-end without such need.

\begin{figure}[t]%
    \centering
    \def\svgwidth{\linewidth}
    \input{figures/preview/bars.pdf_tex}
    \caption{\textbf{Results preview.} Unlike much recent work that combines multiple training paradigms, such as adversarial training and style transfer, our approach retains the modest single-round training complexity of self-training, yet improves the state of the art for adapting semantic segmentation by a significant margin.}
    \label{fig:preview}
\end{figure}

Our method leverages the ubiquitous \emph{data augmentation} techniques from fully supervised learning \cite{deeplabv3plus2018,ZhaoSQWJ17}: photometric jitter, flipping and multi-scale cropping.
We enforce \emph{consistency} of the semantic maps produced by the model across these image perturbations.
The following assumption formalises the key premise:

\myparagraph{Assumption 1.}
Let $f: \mathcal{I} \rightarrow \mathcal{M}$ represent a pixelwise mapping from images $\mathcal{I}$ to semantic output $\mathcal{M}$.
Denote $\rho_{\bm{\epsilon}}: \mathcal{I} \rightarrow \mathcal{I}$ a photometric image transform and, similarly, $\tau_{\bm{\epsilon}'}: \mathcal{I} \rightarrow \mathcal{I}$ a spatial similarity transformation, where $\bm{\epsilon},\bm{\epsilon}'\sim p(\cdot)$ are control variables following some pre-defined density (\eg, $p \equiv \mathcal{N}(0, 1)$).
Then, for any image $I \in \mathcal{I}$, $f$ is \emph{invariant} under $\rho_{\bm{\epsilon}}$ and \emph{equivariant} under $\tau_{\bm{\epsilon}'}$, \ie~$f(\rho_{\bm{\epsilon}}(I)) = f(I)$ and $f(\tau_{\bm{\epsilon}'}(I)) = \tau_{\bm{\epsilon}'}(f(I))$.

\smallskip
\noindent Next, we introduce a training framework using a \emph{momentum network} -- a slowly advancing copy of the original model.
The momentum network provides stable, yet recent targets for model updates, as opposed to the fixed supervision in model distillation \cite{Chen0G18,Zheng_2020_IJCV,ZhengY20}.
We also re-visit the problem of long-tail recognition in the context of generating pseudo labels for self-supervision.
In particular, we maintain an \emph{exponentially moving class prior} used to discount the confidence thresholds for those classes with few samples and increase their relative contribution to the training loss.
Our framework is simple to train, adds moderate computational overhead compared to a fully supervised setup, yet sets a new state of the art on established benchmarks (\cf \cref{fig:preview}).

\section{Preliminaries}
\label{sec:def}
\subsection{Problem Definition}
    Let $G=(V,E)$ denote a graph, where $V$ is a set of nodes and $E \subseteq V \times V $ is a set of edges. Assume each node $v_i$ is associated with a $k-$dimensional real-valued feature vector $\mathbf{w}_i \in \mathbb{R}^k$ and a label $y_i \in \{0,...,M-1\}$.
    If the label $y_i$ of node $v_i$ is unknown, we say node $v_i$ is an unlabeled node. We denote the set of labeled nodes as $V^L$ and the set of unlabeled nodes as $V^U=V\backslash V^L$. Usually, we have $|V^L| \ll |V^U|$. We also call the graph $G$ as \textit{partially labeled graph}~\cite{Tang:11PKDD}.
    Given this, we can formally define the semi-supervised learning problem on graph.

%\vspace{0.06in}
\begin{definition}\textbf{Semi-supervised Learning on Graph.}
	Given a partially labeled graph $G=(V^L\cup V^U, E)$, the objective here is to learn a function $f$ using features $\mathbf{w}$ associated with each node and the graphical structure, in order to predict the labels of unlabeled nodes in the graph.
\end{definition}
%\vspace{0.06in}
 
 Please note that in semi-supervised learning, training and prediction are usually performed simultaneously. In this case, the learning considers both labeled nodes and unlabeled nodes, as well as the structure of the whole graph.
    In this paper, we mainly consider transductive learning setting, though the proposed model can be also applied to other machine learning settings. 
    Moreover, we only consider undirected graphs, but the extension to directed graphs is straightforward.

\subsection{Generative Adversarial Nets (GANs)}

GAN~\cite{goodfellow2014generative} is a new framework for estimating generative models via an adversarial process, in which a generative model $G$ is trained to best fit the original training data and a discriminative model $D$ is trained to distinguish real samples from samples generated by model $G$.
The process can be formalized as a min-max game between $G$ and $D$, with the following loss (value) function:

\beq{\label{gan_aim}
\min\limits_G\max\limits_D V(G,D) = \mathbb{E}_{\mathbf{x}\sim p_{d}(\mathbf{x})} \log D(\mathbf{x}) 
+ \mathbb{E}_{\mathbf{z}\sim p_z(\mathbf{z})} \log [1-D(G(\mathbf{z}))]
}

\noindent where $p_{d}$ is the data distribution from the training data, $p_z(\mathbf{z})$ is a prior on input noise variables.





%!TEX root = main.tex
\section{Patched Estimation and\\ Spatial Spline Regression}
\label{sec:SSR}
In this section, we first explore some tentative solutions, and then point out their insufficiency and limitations in handling our constrained spatial sparse recovery problem.

\textbf{Patched Piece-wise Constant Estimation}.
In practice, we only know the locations of all $\mathbf p_{B_i}$'s and their corresponding aggregated volumes $z_i$'s. What is not known is the fine-grained distribution of each volume $z_i$ across $\Omega_{B_i}$, the subregion covers the observed point  $B_i$.

As a first heuristic, we can assume that the density is  distributed uniformly within each $\Omega_{B_i}$ and estimate $f(\mathbf p_j)$ as the volume $z_i$ divided by its area:
\begin{equation}\label{eq:patched}
	\bar f(\mathbf p_j) = \frac{z_i}{|\Omega_{B_i}|}, \text{ for each }\mathbf p_j \in \Omega_{B_i},
\end{equation}
where $|\Omega_{B_i}|$ is the area of $\Omega_{B_i}$.
%That is, first, we compute the average density for each base station by dividing corresponding volume by the area. 
%Further we assume that the density for each base station is constant on the area covered by this base station. 
Hence we obtain patched piece-wise constant estimation. In this paper, we may use {\it patch} to refer to $\Omega_{B_i}$, the subregion covered $B_i$. 

However, the patched estimation oversimplifies the solution, since the obtained estimates $\bar f(\mathbf p_j)$ are far from being smooth and in fact may have discontinuous jumps on the borders of patches, which will be illustrated in our evaluation. %\red{(Unless all average densities are the same)}
In practice, however, the piece-wise uniformity assumption fails: $f(\mathbf p_j)$ is not constant within a certain patch $\Omega_{B_i}$. In fact, $f(\mathbf p_j)$ should slowly change across neighboring points, as the underlying geographical and demographical characteristics also change smoothly across  regions.

\textbf{Spatial Spline Regression}.
The observations above naturally lead to the idea of using spatial smoothing techniques to smoothen the patched estimation $\bar f(\mathbf p_j)$ to remove the discontinuities and jumps. In the following, we briefly describe a recently proposed powerful smoothing technique called Spatial Spline Regression (SSR) \cite{Sanga13}. After demonstrating its usage for our particular problem, we point out the major limitations for the spatial sparse recovery problem.


Given a set of $l$ spatial data points in $\Omega$, which contains the following information: \emph{1)} the values of these $l$ points: $\{h_j\}_{j=1}^l$, \emph{2)} their positions $\{\mathbf p_j\}_{j=1}^l$, and \emph{3)} their attribute vectors $\{\mathbf w_j\}_{j=1}^l$, SSR fits a smooth spatial field $f$ by minimizing the following penalized sum of square errors \cite{Sanga13}, \cite{Ramsay02}, i.e.,
\begin{equation}\label{eq:minSSR}
	\underset{\boldsymbol{\beta}, f}{\text{minimize}}\sum_{j=1}^{l}\big(h_j- {\mathbf w}_j^{\mathsf T} \boldsymbol{\beta} - f(\mathbf p_j) \big)^2 + \lambda\int_\Omega(\nabla^2 f)^2\ud\mathbf p,
\end{equation}
where $f$ is assumed to be twice-differentiable over $\Omega$, and
$\nabla^2 f= \frac{\partial^2 f}{\partial x^2}+\frac{\partial^2 f}{\partial y^2}$ denotes the \emph{Laplacian} of $f$ to smoothen out the roughness of the spatial field $f$. The tuning parameter $\lambda$ is used to trade the smoothness of $f$ off for a better approximation to data value $h_j$.


% and can be selected using some data-driven or {\it ad hoc} methods.


% $\{ \mathbf p_j, h_j\}$, where $\mathbf p_j\in\Omega$ is the position and $h_j$ is the corresponding value, the SSR assumes that $h_j$ is modeled by

%\begin{equation}\label{eq:spatial_model}
%	h_j = f(\mathbf p_j) + {\mathbf w}_j^{\mathsf T}\boldsymbol{\beta} +\epsilon_j,
%\end{equation}
%where $f$ is a spatial field that give the underlying smooth surface of  $f(\mathbf p_j)$, and $\mathbf w_j$ represents the attributes at position $p_j$. Furthermore, $f$ is assumed to be twice-differentiable over $\Omega$. 
%The smoothness of $f$ enables a key assumption that many spatial data studies hinge upon, that is, the points close to each other are likely to have similar spatial field values.

%Both the spatial field $f$ and the attribute coefficients $\boldsymbol{\beta}$ can be learned based on $l$ training data points in $\Omega$, which contain the following information: \emph{1)} the values of these $l$ points: $\{h_j\}_{j=1}^l$, \emph{2)} their positions $\{\mathbf p_j\}_{j=1}^l$, and \emph{3)} their attribute vectors $\{\mathbf w_j}_{j=1}^l$.

%Once $f$ and $\boldsymbol{\beta}$ are learned from the training data, the value $z_j$ can be estimated by plugging its attribute information ${\mathbf w}_j$ and position $\mathbf p_j$ into \eqref{eq:spatial_model}.
%The model \eqref{eq:spatial_model} can be trained using either kernel-based methods \cite{Clapp04, Chopra07, Caplin08} or finite element analysis \cite{Sanga13, wood2003thin, Ramsay02}.

% However, the challenge to solving problem \eqref{eq:minSSR} is that it involves searching for a functional $f$ over a possibly non-convex domain $\Omega$ that may have strong concavities, complicated boundaries, and even interior holes.
% Although kernel-based methods \cite{Chopra07} are also a commonly used smoothing technique, their major drawback is that by using uniformly damping weights in distance-based kernels, they tend to link data points across unrelated or weakly related subregions in an irregularly shaped non-convex domain.

We now briefly describe how spatial spline regression \cite{Sanga13} can solve problem \eqref{eq:minSSR} via \emph{finite element analysis} for any irregularly shaped domain $\Omega$. 
% For details, interested readers are referred to \cite{Sanga13}.
In SSR, the domain $\Omega$ is divided into small disjoint triangles, which can be done for example by the means of Delaunay triangulation \cite{hje06}. Then a polynomial function is defined on each of these triangles, such that the summation of these polynomial functions defined on different pieces closely approximates the desired spatial field $f$. It is shown in \cite{Sanga13} that the best approximation is achieved by simply solving a set of linear equations (see \cite{Sanga13} for more details).

%Specifically, let $\zeta_1,\ldots,\zeta_K$ denote the vertices of all the small triangles, which are called control points and can be adaptively selected by available data points. Define a piecewise linear or quadratic basis function $\psi_k(x,y)$ called {\it Lagrangian finite element} with $(x,y)\in\Omega$, associated with each control point $\zeta_k$ such that $\psi_k$ evaluates to $1$ at $\zeta_k$ and is equal to $0$ at all other control points. Therefore, according to the {\it Lagrangian property of the basis} we can approximate $f(x,y)$ for any $(x,y)\in\Omega$ only using the values of $f$ on the $K$ control points, i.e., $\mathbf f_K := (f(\zeta_1),\ldots,f(\zeta_K))^{\mathsf T}$. That is, if we let $\psi(x,y) :=(\psi_1(x,y),\ldots,\psi_K(x,y))^{\mathsf T}$ denote the $K$ predefined basis functions, each corresponding to a control point, then we have
%\begin{equation}\label{eq:triApprox}
%	f(x,y) = \mbox{$\sum_{k=1}^{K}$} f(\zeta_k)\psi_k(x,y) = \mathbf f_K^{\mathsf T}\mathbf\psi(x,y),
%\end{equation}  
%Since $\psi_1(x,y),\ldots,\psi_K(x,y)$ are predefined and known \emph{a priori}, the variational estimation of $f$ in problem \eqref{eq:minSSR} boils down to the estimation of only $K$ scalar values, i.e., $\mathbf f_K = (f(\zeta_1),\ldots,f(\zeta_K))^{\mathsf T}$.

%In fact, it is shown in \cite{Sanga13} that with the piece-wise approximation given by \eqref{eq:triApprox}, solving \eqref{eq:minSSR} is simply solving a set of linear equations for $\hat f(\zeta_1),\ldots,\hat f(\zeta_K)$. 
%The estimator $\hat f(x,y)$ for $f$ can then be derived from \eqref{eq:triApprox} as
%\[
%		\hat f(x,y) = \hat{\mathbf f}_K^{\mathsf T}\mathbf\psi(x,y),
%\]

%It is worth noting that commodity triangulation software for finite element analysis is readily available in many free and commercial finite element packages. For example, Delaunay triangulations of a set of data location points (e.g., \cite{hje06}) $V$ are such that no point in $V$ is inside the circumcircle of any triangle; they maximize the minimum angle of all the triangle angles, avoiding stretched triangles. 

Now we can see that if $l=n$ and we plug $h_j = \bar f(\mathbf p_j)$, $j=1,\ldots,n$ into problem \eqref{eq:minSSR}, we will get a new density surface $\hat f$ as a solution to the SSR problem \eqref{eq:minSSR} that is a smoothened approximation of the patched estimates $\bar f(\mathbf p_j)$.

However, SSR given by \eqref{eq:minSSR}
% has a major drawback in our particular case, that is, it 
can not accommodate any constraints, and especially, does not enforce the aggregated volume constraint \eqref{eq:BSvolumes}, or equivalently, the constraint $\mathbf z = \mathbf A\mathbf f$ in  \eqref{eq:prob0}. Therefore, if we smoothen the patched estimates $\bar f(\mathbf p_j)$ out to get a smooth surface estimate $\hat f$, there is no guarantee that the estimated densities in each patch $\Omega_{B_i}$ will sum up to the observed volume $z_i$ on the point $B_i$. Violating this constraint would likely cause large density estimation errors.
Online convex optimization with memory has emerged as an important and challenging area with a wide array of applications, see \citep{lin2012online,anava2015online,chen2018smoothed,goel2019beyond,agarwal2019online,bubeck2019competitively} and the references therein.  Many results in this area have focused on the case of online optimization with switching costs (movement costs), a form of one-step memory, e.g., \citep{chen2018smoothed,goel2019beyond,bubeck2019competitively}, though some papers have focused on more general forms of memory, e.g., \citep{anava2015online,agarwal2019online}. In this paper we, for the first time, study the impact of feedback delay and nonlinear switching cost in online optimization with switching costs. 

An instance consists of a convex action set $\mathcal{K}\subset\mathbb{R}^d$, an initial point $y_0\in\mathcal{K}$, a sequence of non-negative convex cost functions $f_1,\cdots,f_T:\mathbb{R}^d\to\mathbb{R}_{\ge0}$, and a switching cost $c:\mathbb{R}^{d\times(p+1)}\to\mathbb{R}_{\ge0}$. To incorporate feedback delay, we consider a situation where the online learner only knows the geometry of the hitting cost function at each round, i.e., $f_t$, but that the minimizer of $f_t$ is revealed only after a delay of $k$ steps, i.e., at time $t+k$.  This captures practical scenarios where the form of the loss function or tracking function is known by the online learner, but the target moves over time and measurement lag means that the position of the target is not known until some time after an action must be taken. 
To incorporate nonlinear (and potentially nonconvex) switching costs, we consider the addition of a known nonlinear function $\delta$ from $\mathbb{R}^{d\times p}$ to $\mathbb{R}^d$ to the structured memory model introduced previously.  Specifically, we have
\begin{align}
c(y_{t:t-p}) = \frac{1}{2}\|y_t-\delta(y_{t-1:t-p})\|^2,    \label{e.newswitching}
\end{align}
where we use $y_{i:j}$ to denote either $\{y_i, y_{i+1}, \cdots, y_j\}$ if $i\leq j$, or  $\{y_i, y_{i-1}, \cdots, y_j\}$ if $i > j$ throughout the paper. Additionally, we use $\|\cdot\|$ to denote the 2-norm of a vector or the spectral norm of a matrix.

In summary, we consider an online agent that interacts with the environment as follows:
% \begin{inparaenum}[(i)] 
\begin{enumerate}%[leftmargin=*]
    \item The adversary reveals a function $h_t$, which is the geometry of the $t^\mathrm{th}$ hitting cost, and a point $v_{t-k}$, which is the minimizer of the $(t-k)^\mathrm{th}$ hitting cost. Assume that $h_t$ is $m$-strongly convex and $l$-strongly smooth, and that $\arg\min_y h_t(y)=0$.
    \item The online learner picks $y_t$ as its decision point at time step $t$ after observing $h_t,$  $v_{t-k}$.
    \item The adversary picks the minimizer of the hitting cost at time step $t$: $v_t$. 
    \item The learner pays hitting cost $f_t(y_t)=h_t(y_t-v_t)$ and switching cost $c(y_{t:t-p})$ of the form \eqref{e.newswitching}.
\end{enumerate}

The goal of the online learner is to minimize the total cost incurred over $T$ time steps, $cost(ALG)=\sum_{t=1}^Tf_t(y_t)+c(y_{t:t-p})$, with the goal of (nearly) matching the performance of the offline optimal algorithm with the optimal cost $cost(OPT)$. The performance metric used to evaluate an algorithm is typically the \textit{competitive ratio} because the goal is to learn in an environment that is changing dynamically and is potentially adversarial. Formally, the competitive ratio (CR) of the online algorithm is defined as the worst-case ratio between the total cost incurred by the online learner and the offline optimal cost: $CR(ALG)=\sup_{f_{1:T}}\frac{cost(ALG)}{cost(OPT)}$.

It is important to emphasize that the online learner decides $y_t$ based on the knowledge of the previous decisions $y_1\cdots y_{t-1}$, the geometry of cost functions $h_1\cdots h_t$, and the delayed feedback on the minimizer $v_1\cdots v_{t-k}$. Thus, the learner has perfect knowledge of cost functions $f_1\cdots f_{t-k}$, but incomplete knowledge of $f_{t-k+1}\cdots f_t$ (recall that $f_t(y)=h_t(y-v_t)$).

Both feedback delay and nonlinear switching cost add considerable difficulty for the online learner compared to versions of online optimization studied previously. Delay hides crucial information from the online learner and so makes adaptation to changes in the environment more challenging. As the learner makes decisions it is unaware of the true cost it is experiencing, and thus it is difficult to track the optimal solution. This is magnified by the fact that nonlinear switching costs increase the dependency of the variables on each other. It further stresses the influence of the delay, because an inaccurate estimation on the unknown data, potentially magnifying the mistakes of the learner. 

The impact of feedback delay has been studied previously in online learning settings without switching costs, with a focus on regret, e.g., \citep{joulani2013online,shamir2017online}.  However, in settings with switching costs the impact of delay is magnified since delay may lead to not only more hitting cost in individual rounds, but significantly larger switching costs since the arrival of delayed information may trigger a very large chance in action.  To the best of our knowledge, we give the first competitive ratio for delayed feedback in online optimization with switching costs. 

We illustrate a concrete example application of our setting in the following.

\begin{example}[Drone tracking problem]
\label{example:drone} \emph{
Consider a drone with vertical speed $y_t\in\mathbb{R}$. The goal of the drone is to track a sequence of desired speeds $y^d_1,\cdots,y^d_T$ with the following tracking cost:}
\begin{equation}
    \sum_{t=1}^T \frac{1}{2}(y_t-y^d_t)^2 + \frac{1}{2}(y_t-y_{t-1}+g(y_{t-1}))^2,
\end{equation}
\emph{where $g(y_{t-1})$ accounts for the gravity and the aerodynamic drag. One example is $g(y)=C_1+C_2\cdot|y|\cdot y$ where $C_1,C_2>0$ are two constants~\cite{shi2019neural}. Note that the desired speed $y_t^d$ is typically sent from a remote computer/server. Due to the communication delay, at time step $t$ the drone only knows $y_1^d,\cdots,y_{t-k}^d$.}

\emph{This example is beyond the scope of existing results in online optimization, e.g.,~\cite{shi2020online,goel2019beyond,goel2019online}, because of (i) the $k$-step delay in the hitting cost $\frac{1}{2}(y_t-y_t^d)$ and (ii) the nonlinearity in the switching cost $\frac{1}{2}(y_t-y_{t-1}+g(y_{t-1}))^2$ with respective to $y_{t-1}$. However, in this paper, because we directly incorporate the effect of delay and nonlinearity in the algorithm design, our algorithms immediately provide constant-competitive policies for this setting.}
\end{example}


\subsection{Related Work}
This paper contributes to the growing literature on online convex optimization with memory.  
Initial results in this area focused on developing constant-competitive algorithms for the special case of 1-step memory, a.k.a., the Smoothed Online Convex Optimization (SOCO) problem, e.g., \citep{chen2018smoothed,goel2019beyond}. In that setting, \citep{chen2018smoothed} was the first to develop a constant, dimension-free competitive algorithm for high-dimensional problems.  The proposed algorithm, Online Balanced Descent (OBD), achieves a competitive ratio of $3+O(1/\beta)$ when cost functions are $\beta$-locally polyhedral.  This result was improved by \citep{goel2019beyond}, which proposed two new algorithms, Greedy OBD and Regularized OBD (ROBD), that both achieve $1+O(m^{-1/2})$ competitive ratios for $m$-strongly convex cost functions.  Recently, \citep{shi2020online} gave the first competitive analysis that holds beyond one step of memory.  It holds for a form of structured memory where the switching cost is linear:
$
    c(y_{t:t-p})=\frac{1}{2}\|y_t-\sum_{i=1}^pC_iy_{t-i}\|^2,
$
with known $C_i\in\mathbb{R}^{d\times d}$, $i=1,\cdots,p$. If the memory length $p = 1$ and $C_1$ is an identity matrix, this is equivalent to SOCO. In this setting, \citep{shi2020online} shows that ROBD has a competitive ratio of 
\begin{align}
    \frac{1}{2}\left( 1 + \frac{\alpha^2 - 1}{m} + \sqrt{\Big( 1 + \frac{\alpha^2 - 1}{m}\Big)^2 + \frac{4}{m}} \right),
\end{align}
when hitting costs are $m$-strongly convex and $\alpha=\sum_{i=1}^p\|C_i\|$. 


Prior to this paper, competitive algorithms for online optimization have nearly always assumed that the online learner acts \emph{after} observing the cost function in the current round, i.e., have zero delay.  The only exception is \citep{shi2020online}, which considered the case where the learner must act before observing the cost function, i.e., a one-step delay.  Even that small addition of delay requires a significant modification to the algorithm (from ROBD to Optimistic ROBD) and analysis compared to previous work. 

As the above highlights, there is no previous work that addresses either the setting of nonlinear switching costs nor the setting of multi-step delay. However, the prior work highlights that ROBD is a promising algorithmic framework and our work in this paper extends the ROBD framework in order to address the challenges of delay and non-linear switching costs. Given its importance to our work, we describe the workings of ROBD in detail in Algorithm~\ref{robd}. 

\begin{algorithm}[t!]
  \caption{ROBD \citep{goel2019beyond}}
  \label{robd}
\begin{algorithmic}[1]
  \STATE {\bfseries Parameter:} $\lambda_1\ge0,\lambda_2\ge0$
  \FOR{$t=1$ {\bfseries to} $T$}
  \STATE {\bfseries Input:} Hitting cost function $f_t$, previous decision points $y_{t-p:t-1}$
  \STATE $v_t\leftarrow\arg\min_yf_t(y)$
  \STATE $y_t\leftarrow\arg\min_yf_t(y)+\lambda_1c(y,y_{t-1:t-p})+\frac{\lambda_2}{2}\|y-v_t\|^2_2$
  \STATE {\bfseries Output:} $y_t$
  \ENDFOR
   
\end{algorithmic}
\end{algorithm}

Another line of literature that this paper contributes to is the growing understanding of the connection between online optimization and adaptive control. The reduction from adaptive control to online optimization with memory was first studied in \citep{agarwal2019online} to obtain a sublinear static regret guarantee against the best linear state-feedback controller, where the approach is to consider a disturbance-action policy class with some fixed horizon.  Many follow-up works adopt similar reduction techniques \citep{agarwal2019logarithmic, brukhim2020online, gradu2020adaptive}. A different reduction approach using control canonical form is proposed by \citep{li2019online} and further exploited by \citep{shi2020online}. Our work falls into this category.  The most general results so far focus on Input-Disturbed Squared Regulators, which can be reduced to online convex optimization with structured memory (without delay or nonlinear switching costs).  As we show in \Cref{Control}, the addition of delay and nonlinear switching costs leads to a significant extension of the generality of control models that can be reduced to online optimization. 
%!TEX root = main.tex
\section{Performance Evaluation}
\label{sec:simu}

% compare CDFs
\begin{figure*}[!htb]
    \centering
            \subfigure[November, $n_{\text{BS}} = 200$]{
                \includegraphics[width=1.7in]{figures/CDF_200_Nov}
                    \label{fig:CDF200Nov}
            }
            \hspace{-3mm}
            \subfigure[December, $n_{\text{BS}} = 200$]{
                \includegraphics[width=1.7in]{figures/CDF_200_Dec}
                    \label{fig:CDF200Dec}
            }
            \hspace{-3mm}
            \subfigure[November, $n_{\text{BS}} = 100$]{
                \includegraphics[width=1.7in]{figures/CDF_100_Nov}
                    \label{fig:CDF100Nov}
            }
            \hspace{-3mm}
            \subfigure[December, $n_{\text{BS}} = 100$]{
                \includegraphics[width=1.7in]{figures/CDF_100_Dec}
                    \label{fig:CDF100Dec}
            }
            \vspace{-3mm}
    \caption{The comparison of the CDFs of relative errors given by different estimation methods when $n_{\text{BS}} = 200$ and $n_{\text{BS}} = 100$ for stress-testing. The legends follow the same order as the curves at relative error $= 0.5$.}
    \label{fig:compareCDF100}
\vspace{-3mm}
\end{figure*}

% % % compare bar plot. 4 in one line.
% % \begin{figure*}[t]
%     \centering
%             \subfigure[November, $n_{\text{BS}} = 200$]{
%                 \includegraphics[width=3.2in]{figures/barplot_200_Nov}
%                     \label{fig:barplot200Nov}
%             }
%             \hspace{-3mm}
%             \subfigure[December, $n_{\text{BS}} = 200$]{
%                 \includegraphics[width=3.2in]{figures/barplot_200_Dec}
%                     \label{fig:barplot200Dec}
%             }
%             \vspace{0mm}
%             \subfigure[November, $n_{\text{BS}} = 100$]{
%                 \includegraphics[width=3.2in]{figures/barplot_100_Nov}
%                     \label{fig:barplot100Nov}
%             }
%             \hspace{-3mm}
%             \subfigure[December, $n_{\text{BS}} = 100$]{
%                 \includegraphics[width=3.2in]{figures/barplot_100_Dec}
%                     \label{fig:barplot100Dec}
%             }
%             \vspace{0mm}
%     \caption{Comparison of the estimation's Mean Relative Error of different methods when $n_{\text{BS}} = 200$ or $n_{\text{BS}} = 100$ for stress-testing. In each figure, the bars from left to right stands for Patched Estimation, Patched Estimation + SSR 1, Patched Estimation + SSR 2, Constrained Spatial Smoothing, and Constrained Spatial Smoothing + Features respectively.}
%     \label{fig:compareMRE}
% \vspace{0mm}
% \end{figure*}

% compare bar plot. 4 in one line.
\begin{figure*}[t]
    \centering
            \subfigure[November, $n_{\text{BS}} = 200$]{
                \includegraphics[width=1.7in]{figures/barplot_200_Nov}
                    \label{fig:barplot200Nov}
            }
            \hspace{-3mm}
            \subfigure[December, $n_{\text{BS}} = 200$]{
                \includegraphics[width=1.7in]{figures/barplot_200_Dec}
                    \label{fig:barplot200Dec}
            }
            \hspace{-3mm}
            \subfigure[November, $n_{\text{BS}} = 100$]{
                \includegraphics[width=1.7in]{figures/barplot_100_Nov}
                    \label{fig:barplot100Nov}
            }
            \hspace{-3mm}
            \subfigure[December, $n_{\text{BS}} = 100$]{
                \includegraphics[width=1.7in]{figures/barplot_100_Dec}
                    \label{fig:barplot100Dec}
            }
            \vspace{-3mm}
    \caption{The comparison of the Mean Relative Error of different estimation methods when $n_{\text{BS}} = 200$ or $n_{\text{BS}} = 100$ for stress-testing. In each figure, the bars from left to right represent Patched Estimation, Patched Estimation + SSR 1, Patched Estimation + SSR 2, Constrained Spatial Smoothing, and Constrained Spatial Smoothing + Features, respectively.}
    \label{fig:compareMRE}
\vspace{-6mm}
\end{figure*}


% % compare bar plots
% \begin{figure*}[t]
%     \centering
%             \subfigure[November]{
%                 \includegraphics[width=3.2in]{figures/barplot_200_Nov}
%                     \label{fig:barplot200Nov}
%             }
%             \hspace{-3mm}
%             \subfigure[December]{
%                 \includegraphics[width=3.2in]{figures/barplot_200_Dec}
%                     \label{fig:barplot200Dec}
%             }
%             \vspace{-3mm}
%     \caption{Comparison of the estimation's Mean Relative Error of different methods when $n_{\text{BS}} = 200$.}
%     \label{fig:compareMRE200}
% \vspace{1mm}
% %\end{figure*}

% %\begin{figure*}[t]
%     \centering
%             \subfigure[November]{
%                 \includegraphics[width=3.2in]{figures/barplot_100_Nov}
%                     \label{fig:barplot100Nov}
%             }
%             \hspace{-3mm}
%             \subfigure[December]{
%                 \includegraphics[width=3.2in]{figures/barplot_100_Dec}
%                     \label{fig:barplot100Dec}
%             }
%             \vspace{-3mm}
%     \caption{Comparison of the estimation's Mean Relative Error of different methods when $n_{\text{BS}} = 100$ for stress-testing.}
%     \label{fig:compareMRE100}
% \vspace{-5mm}
% \end{figure*}




% In this section, we perform an extensive case study of the approach we described above in order to demonstrate its applicability. 
The model in \eqref{eq:add-auxiliary} is not attached to any particular empirical problem and does not contain many implicit assumptions, it is general. However, in order to measure its performance we evaluate the model using real-world cell phone data.
% We picked the cell phone data as an example of how the model can solve empirical problem and compare the model's performance to other approaches.

% \subsection{Dataset Description}
% \label{sec:activity-recovery}

% The model in \eqref{eq:add-auxiliary} is not attached to any particular empirical problem and does not contain many implicit assumptions, it is general. However, in order to measure its performance we evaluate the model using real-world data. Due to generality of the proposed learning algorithm the range of possible data sets is potentially big. For our empirical case study we chose cell phone data, where there exists a problem of recovering a spatial field from coarse aggregations observed at sparse cell phone towers. We do not overestimate the problem, but rather see this particular data set suitable for extensive case study.
%To give a more intuitive idea about our problem, here we %introduce the datasets we utilized, and describe how we %process the data to study the problem of inferring cell %phone activities spatial distribution.

The Milan Call Description Records (CDR) dataset
% is a part of the Telecom Italia Big Data Challenge dataset provided by Telecom Italia Mobile.
% It
contains the telecommunications activity records from November 1$st$, 2013 to December 31$th$, 2013 in the city of Milan~\cite{bigdatachallenge}. In the Milan CDR dataset, the city of Milan is divided into a $100\times 100$ square grid. Each square is size of about 235m $\times$ 235m. Each activity record consists of the following entries: square ID, time-stamp of 10-minute time slot, incoming SMS activity, outgoing SMS activity, incoming call activity and outgoing call activity. The values of the four types of activities are normalized to the same scale.



% % grid
% \begin{figure}[t]
%         %\hspace{7mm}
%         \includegraphics[width=3.3in]{figures/grid}
%         %\vspace{-5mm}
%         \caption{The map shows the metropolitan area of Milan, Italy, and the area covered by the 2726 grid squares.}
%         \label{fig:grid}
%         \vspace{-3mm}
% \end{figure}


Another dataset we utilized is the Milan geographical attribute dataset available from the Municipality of Milan's Open Data website \cite{barlacchi2015multi}. This dataset consists of features of central 2726 squares among the whole $10,000$ squares. The features of each square include: population, green area percentage, number of sport centers, number of universities, number of businesses, and number of bus stops.
% Fig.~\ref{fig:grid} shows the map of Milan and the area covered by these grid squares. The 2726 squares covers the central part of the Milan city and contains the majority of telecommunication activities in the dataset.
% We refer to~\cite{bcici_mobihoc15} for more detailed description about this dataset.
In our empirical study, we focus on these squares to compare the performance of different algorithms.

% % value distribution
% \begin{figure}[t]
%                         \centering
%                         \subfigure[November]{
%                 \includegraphics[width=1.5in]{figures/heatmap_Nov_call_sms}
%                                 \label{fig:heatmapNov}
%                         }
%                 \hspace{-4mm}
%                         \subfigure[December]{
%                 \includegraphics[width=1.5in]{figures/heatmap_Dec_call_sms}
%                                 \label{fig:heatmapDec}
%                         }
%                         \vspace{-1mm}
%                 \caption{The heat map of call + sms activities during November and December.}
%                 \label{fig:NovDecHeatmap}
% \vspace{-5mm}
% \end{figure}



The general \textbf{\textit{problem of recovering a spatial field from coarse aggregations observed at sparse points in
the field}} in this particular case study is reformulated into \textbf{\textit{the problem of recovering the distribution of cell phone activities over the whole 2726 square regions given that only aggregated activity observations in base stations are known}}. To study this problem, we need to further process the Milan CDR dataset.

\textit{First}, we sum up the four types of activities during November and December respectively to come up with the activity volume of each square during the two months. These two datasets are served as the ground-truth datasets of Milan cell phone activity distributions.
% Fig.~\ref{fig:heatmapNov} and Fig.~\ref{fig:heatmapDec} show the heat maps of activity volumes in each square during November and December. 
\textit{Second}, after we aggregated the two months' activities for each square, we need to set the locations of base stations (BSs). According to \cite{ratti2006mobile}, there are roughly $200$ base stations in Milan. However, the exact locations are not available. Thus, we assume the $n_{\text{BS}}$ ($n_{\text{BS}} = 200$ or $100$ for stress-test) BSs are randomly distributed according to the probability distribution
$
\Pr (\text{Set square $i$ as BS}) = f(\mathbf p_i) / \sum_{j=1}^{N}f(\mathbf p_j),
$
where $f(\mathbf p_i)$ is the cell phone activity volume in square $i$, $i=\{1, \ldots, N\}$, $N=2726$ is the number of squares we are focusing on. 
% Notice that when we have 200 base station's aggregated observations, they only cover $7.34 \%$ of the whole 2726 squares region. This is extremely sparse and makes our problem highly challenging.
% In addition, we also assume $n_{\text{BS}} = 100$ and choose $100$ squares as BSs according to the same probability distribution to stress-test our algorithm's capability under even sparser observations.
\textit{Third}, after the base station locations are sampled, the activity of each square will be assigned to its closest base station. If multiple base stations are equidistant from the square, then the activity of this square will be evenly distributed among these base stations. We then assume we only know the aggregated activities in base station squares, which is usually the true case in reality.
Fig.~\ref{fig:100BS} and Fig.~\ref{fig:100BSincharge} show the base station distributions and the region charged by each base station for $n_{\text{BS}} = 100$ respectively.
% To save space, we don't present the figure for 200 base stations. 

% % sample BS location
% \begin{figure}[t]
%                         \centering
%                         \subfigure[$n_{\text{BS}} = 200$]{
%                 \includegraphics[width=1.5in]{figures/heatmap_200BS}
%                                 \label{fig:200BS}
%                         }
%                 \hspace{-4mm}
%                         \subfigure[$n_{\text{BS}} = 100$]{
%                 \includegraphics[width=1.5in]{figures/heatmap_100BS}
%                                 \label{fig:100BS}
%                         }
%                         \vspace{-1mm}
%                 \caption{The sampled base station distributions for $n_{\text{BS}} = 200$ and $n_{\text{BS}} = 100$.}
%                 \label{fig:BSLocations}
% \vspace{-5mm}
% \end{figure}







% \subsection{Experimental Setup}
% \subsubsection{\bf Algorithms Evaluated}
We test our proposed approach and compare it with 3 baseline methods.
% In particular, we evaluate and compare the following models using the aggregated November and December datasets, with number of base stations $n_{\text{BS}} = 200$ or $n_{\text{BS}} = 100$ for stress testing.
\begin{itemize}
\item \textbf{Patched Estimation (PE)}:
% estimate the cell phone activity distribution by patched piece-wise constant estimation, that is, 
assume cell phone activity density is distributed uniformly within each sub-region $\Omega_{B_i}$
% , i.e., the area covered by base station $B_i$,
and estimate each square's activity volume by \eqref{eq:patched}.
\item \textbf{Patched Estimation + SSR 1}: first estimate \textit{only base station} activity volumes by \eqref{eq:patched}. Use these sparse points to fit a smooth surface by running Spatial Spline Regression to obtain the estimated cell phone activity in all squares. 
\item \textbf{Patched Estimation + SSR 2}: first estimate the activity volumes of \textit{all squares} by Patched Estimation. Then use all these points to fit a smooth surface by running Spatial Spline Regression to obtain the final estimated cell phone activity in all squares.
\item \textbf{Constrained Spatial Smoothing (CSS)}: first get the initial estimation of the activity volumes of all squares by Patched Estimation, then run Constrained Spatial Smoothing algorithm to get the final activity volumes estimation of all squares.
\item \textbf{Constrained Spatial Smoothing + Features}: in this case, we incorporate the geographical features into the Constrained Spatial Smoothing algorithm.
\end{itemize}

We set the penalty parameter $\lambda = 1$ when $n_{\text{BS}} = 200$ and $\lambda = 10$ when $n_{\text{BS}} = 100$, for all methods that utilize SSR. The geographical features of Milan are only incorporated in the last algorithm described above.
% Besides, for the implementation of Spatial Spline Regression, we use the \emph{fdaPDE} R Package~\cite{fdaPDE}.
We evaluate the performance by the Mean Relative Error (MRE) of the produced activity estimates for the true activity values. 
% The relative error of an estimation $\hat{f}(\mathbf{p}_j)$ compared to the true value $f(\mathbf{p}_j)$ is defined as $|\hat{f}(\mathbf{p}_j) - f(\mathbf{p}_j)| / f(\mathbf{p}_j)$.

\subsection{Performance Evaluation}
\subsubsection{\bf Comparison of Different Algorithms}


We show the cumulative distribution function (CDF) of Relative Errors given by each approach in Fig.~\ref{fig:compareCDF100}. In addition, we compare the estimation's Mean Relative Errors of different approaches in Fig.~\ref{fig:compareMRE}. It is quite clear that our proposed algorithms outperform other three baseline approaches significantly in all the cases ($n_{\text{BS}} = 200$ and $n_{\text{BS}}=100$, data aggregated in November and in December). 

By comparing Patched Estimation + SSR 1 with Patched Estimation approach, we can see that using spatial smoothing based on only base station squares' observations leads to worse performance than patched estimation. This can be explained by the smoothing property of SSR and the way we set the values of base station squares. As we described, we set the activity values of base stations by averaging the total activity amount of each base station on all the squares it covers. Thus, given the activity $\frac{z_i}{|\Omega_{B_i}|}$ ($|\Omega_{B_i}|$ denotes the number of squares within region $\Omega_{B_i}$) of a base station $B_i$, the true activities of itself and its surrounding squares within region $B_i$ are distributed with a mean of $\frac{z_i}{|\Omega_{B_i}|}$. Given two base stations $B_1$ and $B_2$ that are close to each other, with aggregated activities of $z_1$ and $z_2$ respectively, the Spatial Smoothing approach will fit a smooth surface between the two base stations. Suppose $z_1 > z_2$, in this case, in overall the activities of $B_1$'s neighbour squares will be under estimated, and that of $B_2$ will be over estimated. Therefore, Patched Estimation + SSR 1's performance is not as good as Patched Estimation.

By comparing Patched Estimation + SSR 2 with Patched Estimation and Patched Estimation + SSR 1, we can observe that applying spatial smoothing on the results of patched estimation improves the performance. This proves the rationality and effectiveness of introducing smoothness into the estimated cell phone activity distribution surface.

Our proposed approaches achieves much better performance compared with the three baseline methods. By using Constrained Spatial Smoothing instead of applying Spatial Spline Regression directly, we are able to fit a smooth activity distribution while forcing it to match the observations of base station squares (the aggregated activity volumes) at the same time. By comparing Constrained Spatial Smoothing that incorporates additional features of each square with the version without features, we can see that the performance is further improved. The reason is that the heterogeneity of different locations will influence the telecommunication activity distribution, therefore making the distribution not everywhere smooth. Incorporating additional features into our model can help to explain the residuals between estimated smooth distribution and the true activity distribution, therefore further increases estimation accuracy.
% By comparing Fig.~\ref{fig:compareCDF200} and Fig.~\ref{fig:compareCDF100}, we also can see that incorporating additional features into Constrained Spatial Smoothing becomes more important when the base stations are more sparse.

The performance of different methods on December dataset is worse than on November dataset. The reason is that, there are multiple holidays during December, therefore the cell phone activities will be much more irregular than usual. 

% % 3d plots
% \begin{figure*}[t]
%                         \centering
%                         \subfigure[Real distribution]{
%                 \includegraphics[width=2.3in]{figures/3d_200BS_Nov_trueval.png}
%                                 \label{fig:3Dtrueval.png}
%                         }
%                 \hspace{-4mm}
%                         \subfigure[Estimation of Patched Estimation]{
%                 \includegraphics[width=2.3in]{figures/3d_200BS_Nov_baseline1.png}
%                                 \label{fig:3Dbaseline1.png}
%                         }
%                         \hspace{-4mm}
%                         \subfigure[Estimation of Constrained Spatial Smoothing + Features]{
%                 \includegraphics[width=2.3in]{figures/3d_200BS_Nov_SsrAdmm.png}
%                                 \label{fig:3DSsrAdmm.png}
%                         }
%                         \vspace{0mm}
%                 \caption{The true telecommunication activity distribution on November and the fitted surface of Patched Estimation and our method SSR + ADMM when $n_{\text{BS}} = 200$.}
%                 \label{fig:compareSurface}
% \vspace{-5mm}
% \end{figure*}








% % map and heatmap
% \begin{figure*}[t]
%                         \centering
%                         \subfigure[Map of Milan]{
%                 \includegraphics[width=1.95in]{figures/grid}
%                                 \label{fig:grid}
%                         }
%                 \hspace{-4mm}
%                         \subfigure[Activity Distribution in November]{
%                 \includegraphics[width=2.5in]{figures/heatmap_Nov_call_sms}
%                                 \label{fig:heatmapNov}
%                         }
%                         \hspace{-4mm}
%                         \subfigure[Activity Distribution in December]{
%                 \includegraphics[width=2.5in]{figures/heatmap_Dec_call_sms}
%                                 \label{fig:heatmapDec}
%                         }
%                         \vspace{0mm}
%                 \caption{ (a) shows the metropolitan area of Milan, Italy, and the area covered by the 2726 grid squares. (b) and (c) show the heat map of cell phone activities (Call + SMS) during November and December.}
%                 \label{fig:compareSurface}
% \vspace{1mm}
% \end{figure*}

% sample BS location
\begin{figure}[t]
                        \centering
                %         \subfigure[$n_{\text{BS}} = 200$]{
                % \includegraphics[width=2.2in]{figures/heatmap_200BS}
                %                 \label{fig:200BS}
                %         }
                % \hspace{-0mm}
                        \subfigure[Distribution of BSs]{
                \includegraphics[width=1.3in]{figures/heatmap_100BS}
                                \label{fig:100BS}
                        }
                \hspace{-0mm}
                        \subfigure[Areas covered by each BS]{
                \includegraphics[width=1.3in]{figures/incharge_100BS}
                                \label{fig:100BSincharge}
                        }
                        \vspace{-3mm}
                \caption{(a) The geographical distribution of sampled base stations for $n_{\text{BS}} = 100$. (b) The areas that the individual base stations are responsible for, when $n_{\text{BS}} = 100$.}
                \label{fig:BSLocations}
\vspace{-6mm}
\end{figure}












% Fig.~\ref{fig:3Dtrueval.png}, Fig.~\ref{fig:3Dbaseline1.png} and Fig.~\ref{fig:3DSsrAdmm.png} show the distribution surfaces of true cell phone activity volumes, estimated volumes by Patched Estimation, and estimated volumes by Constrained Spatial Smoothing with features when $n_{\text{BS}} = 200$ using the November dataset. We can see that the Patched Estimation approach fits a stepped surface, while our approach gives a much smoother surface.





% \subsubsection{\bf Impact of Smooth Penalty Parameter $\lambda$}

% % influence of lambda
% \begin{figure}[t]
%         %\hspace{7mm}
%         \includegraphics[width=3.4in]{figures/lambda}
%         %\vspace{-5mm}
%         \caption{Influence of $\lambda$ to estimation's Mean Relative Error when $n_{\text{BS}} = 200$ and $n_{\text{BS}}=100$ for stress-testing. The figure is based on the November dataset. Result on the December dataset is similar.}
%         \label{fig:lambda}
%         \vspace{1.5mm}
% \end{figure}

% Fig.~\ref{fig:lambda} shows how the the estimation's Mean Relative Error varies when $\lambda$ increases from $10^{-4}$ to $10^3$. We make two interesting observations. First, $\lambda$ around $1 \sim 10$ usually gives the best performance. Too big or too small $\lambda$ will decrease the estimation accuracy. This is reasonable, as when $\lambda$ is too small, we put little emphasis on the smoothness of estimated surface, thus the performance will suffer. If $\lambda$ is too big, it enforces a smooth surface, which also doesn't match the reality. 
% Second, when we have less base stations, $\lambda$ that gives the best performance will increase (from 1 to 10). Besides, we can see that the performance of the model with $\lambda$ between $1 \sim 100$ does not significantly change when $n_{\text{BS}} = 100$. That indicates the following: when the base station distribution is more sparse, the estimation performance is less sensitive to $\lambda$ when it is around the best value ($1$ for $n_{\text{BS}} = 200$ and $10$ for $n_{\text{BS}} = 100$).  
















\section{Related Work}\label{sec:related}
 
The authors in \cite{humphreys2007noncontact} showed that it is possible to extract the PPG signal from the video using a complementary metal-oxide semiconductor camera by illuminating a region of tissue using through external light-emitting diodes at dual-wavelength (760nm and 880nm).  Further, the authors of  \cite{verkruysse2008remote} demonstrated that the PPG signal can be estimated by just using ambient light as a source of illumination along with a simple digital camera.  Further in \cite{poh2011advancements}, the PPG waveform was estimated from the videos recorded using a low-cost webcam. The red, green, and blue channels of the images were decomposed into independent sources using independent component analysis. One of the independent sources was selected to estimate PPG and further calculate HR, and HRV. All these works showed the possibility of extracting PPG signals from the videos and proved the similarity of this signal with the one obtained using a contact device. Further, the authors in \cite{10.1109/CVPR.2013.440} showed that heart rate can be extracted from features from the head as well by capturing the subtle head movements that happen due to blood flow.

%
The authors of \cite{kumar2015distanceppg} proposed a methodology that overcomes a challenge in extracting PPG for people with darker skin tones. The challenge due to slight movement and low lighting conditions during recording a video was also addressed. They implemented the method where PPG signal is extracted from different regions of the face and signal from each region is combined using their weighted average making weights different for different people depending on their skin color. 
%

There are other attempts where authors of \cite{6523142,6909939, 7410772, 7412627} have introduced different methodologies to make algorithms for estimating pulse rate robust to illumination variation and motion of the subjects. The paper \cite{6523142} introduces a chrominance-based method to reduce the effect of motion in estimating pulse rate. The authors of \cite{6909939} used a technique in which face tracking and normalized least square adaptive filtering is used to counter the effects of variations due to illumination and subject movement. 
The paper \cite{7410772} resolves the issue of subject movement by choosing the rectangular ROI's on the face relative to the facial landmarks and facial landmarks are tracked in the video using pose-free facial landmark fitting tracker discussed in \cite{yu2016face} followed by the removal of noise due to illumination to extract noise-free PPG signal for estimating pulse rate. 

Recently, the use of machine learning in the prediction of health parameters have gained attention. The paper \cite{osman2015supervised} used a supervised learning methodology to predict the pulse rate from the videos taken from any off-the-shelf camera. Their model showed the possibility of using machine learning methods to estimate the pulse rate. However, our method outperforms their results when the root mean squared error of the predicted pulse rate is compared. The authors in \cite{hsu2017deep} proposed a deep learning methodology to predict the pulse rate from the facial videos. The researchers trained a convolutional neural network (CNN) on the images generated using Short-Time Fourier Transform (STFT) applied on the R, G, \& B channels from the facial region of interests.
The authors of \cite{osman2015supervised, hsu2017deep} only predicted pulse rate, and we extended our work in predicting variance in the pulse rate measurements as well.

All the related work discussed above utilizes filtering and digital signal processing to extract PPG signals from the video which is further used to estimate the PR and PRV.  %
The method proposed in \cite{kumar2015distanceppg} is person dependent since the weights will be different for people with different skin tone. In contrast, we propose a deep learning model to predict the PR which is independent of the person who is being trained. Thus, the model would work even if there is no prior training model built for that individual and hence, making our model robust. 

%
\section{Conclusion}
\label{sec:conclude}

We propose an algorithm named \ebdjoin+ for computing edit similarity join, one of the most important operations in database systems.  Different from all previous approaches, we first embed the input strings from the edit space to the Hamming space, and then try to perform a filtering (for reducing candidate pairs) in the Hamming space where efficient tools like locality sensitive hashing are available.  Our experiments have shown that \ebdjoin+ significantly outperforms, at a very small cost of accuracy, all existing algorithms on long strings and large thresholds.
%, which are critical to applications such as the analysis of genome sequences in bioinformatics. 


\bibliographystyle{IEEEtran}
\bibliography{main}

%\chapter{Supplementary Material}
\label{appendix}

In this appendix, we present supplementary material for the techniques and
experiments presented in the main text.

\section{Baseline Results and Analysis for Informed Sampler}
\label{appendix:chap3}

Here, we give an in-depth
performance analysis of the various samplers and the effect of their
hyperparameters. We choose hyperparameters with the lowest PSRF value
after $10k$ iterations, for each sampler individually. If the
differences between PSRF are not significantly different among
multiple values, we choose the one that has the highest acceptance
rate.

\subsection{Experiment: Estimating Camera Extrinsics}
\label{appendix:chap3:room}

\subsubsection{Parameter Selection}
\paragraph{Metropolis Hastings (\MH)}

Figure~\ref{fig:exp1_MH} shows the median acceptance rates and PSRF
values corresponding to various proposal standard deviations of plain
\MH~sampling. Mixing gets better and the acceptance rate gets worse as
the standard deviation increases. The value $0.3$ is selected standard
deviation for this sampler.

\paragraph{Metropolis Hastings Within Gibbs (\MHWG)}

As mentioned in Section~\ref{sec:room}, the \MHWG~sampler with one-dimensional
updates did not converge for any value of proposal standard deviation.
This problem has high correlation of the camera parameters and is of
multi-modal nature, which this sampler has problems with.

\paragraph{Parallel Tempering (\PT)}

For \PT~sampling, we took the best performing \MH~sampler and used
different temperature chains to improve the mixing of the
sampler. Figure~\ref{fig:exp1_PT} shows the results corresponding to
different combination of temperature levels. The sampler with
temperature levels of $[1,3,27]$ performed best in terms of both
mixing and acceptance rate.

\paragraph{Effect of Mixture Coefficient in Informed Sampling (\MIXLMH)}

Figure~\ref{fig:exp1_alpha} shows the effect of mixture
coefficient ($\alpha$) on the informed sampling
\MIXLMH. Since there is no significant different in PSRF values for
$0 \le \alpha \le 0.7$, we chose $0.7$ due to its high acceptance
rate.


% \end{multicols}

\begin{figure}[h]
\centering
  \subfigure[MH]{%
    \includegraphics[width=.48\textwidth]{figures/supplementary/camPose_MH.pdf} \label{fig:exp1_MH}
  }
  \subfigure[PT]{%
    \includegraphics[width=.48\textwidth]{figures/supplementary/camPose_PT.pdf} \label{fig:exp1_PT}
  }
\\
  \subfigure[INF-MH]{%
    \includegraphics[width=.48\textwidth]{figures/supplementary/camPose_alpha.pdf} \label{fig:exp1_alpha}
  }
  \mycaption{Results of the `Estimating Camera Extrinsics' experiment}{PRSFs and Acceptance rates corresponding to (a) various standard deviations of \MH, (b) various temperature level combinations of \PT sampling and (c) various mixture coefficients of \MIXLMH sampling.}
\end{figure}



\begin{figure}[!t]
\centering
  \subfigure[\MH]{%
    \includegraphics[width=.48\textwidth]{figures/supplementary/occlusionExp_MH.pdf} \label{fig:exp2_MH}
  }
  \subfigure[\BMHWG]{%
    \includegraphics[width=.48\textwidth]{figures/supplementary/occlusionExp_BMHWG.pdf} \label{fig:exp2_BMHWG}
  }
\\
  \subfigure[\MHWG]{%
    \includegraphics[width=.48\textwidth]{figures/supplementary/occlusionExp_MHWG.pdf} \label{fig:exp2_MHWG}
  }
  \subfigure[\PT]{%
    \includegraphics[width=.48\textwidth]{figures/supplementary/occlusionExp_PT.pdf} \label{fig:exp2_PT}
  }
\\
  \subfigure[\INFBMHWG]{%
    \includegraphics[width=.5\textwidth]{figures/supplementary/occlusionExp_alpha.pdf} \label{fig:exp2_alpha}
  }
  \mycaption{Results of the `Occluding Tiles' experiment}{PRSF and
    Acceptance rates corresponding to various standard deviations of
    (a) \MH, (b) \BMHWG, (c) \MHWG, (d) various temperature level
    combinations of \PT~sampling and; (e) various mixture coefficients
    of our informed \INFBMHWG sampling.}
\end{figure}

%\onecolumn\newpage\twocolumn
\subsection{Experiment: Occluding Tiles}
\label{appendix:chap3:tiles}

\subsubsection{Parameter Selection}

\paragraph{Metropolis Hastings (\MH)}

Figure~\ref{fig:exp2_MH} shows the results of
\MH~sampling. Results show the poor convergence for all proposal
standard deviations and rapid decrease of AR with increasing standard
deviation. This is due to the high-dimensional nature of
the problem. We selected a standard deviation of $1.1$.

\paragraph{Blocked Metropolis Hastings Within Gibbs (\BMHWG)}

The results of \BMHWG are shown in Figure~\ref{fig:exp2_BMHWG}. In
this sampler we update only one block of tile variables (of dimension
four) in each sampling step. Results show much better performance
compared to plain \MH. The optimal proposal standard deviation for
this sampler is $0.7$.

\paragraph{Metropolis Hastings Within Gibbs (\MHWG)}

Figure~\ref{fig:exp2_MHWG} shows the result of \MHWG sampling. This
sampler is better than \BMHWG and converges much more quickly. Here
a standard deviation of $0.9$ is found to be best.

\paragraph{Parallel Tempering (\PT)}

Figure~\ref{fig:exp2_PT} shows the results of \PT sampling with various
temperature combinations. Results show no improvement in AR from plain
\MH sampling and again $[1,3,27]$ temperature levels are found to be optimal.

\paragraph{Effect of Mixture Coefficient in Informed Sampling (\INFBMHWG)}

Figure~\ref{fig:exp2_alpha} shows the effect of mixture
coefficient ($\alpha$) on the blocked informed sampling
\INFBMHWG. Since there is no significant different in PSRF values for
$0 \le \alpha \le 0.8$, we chose $0.8$ due to its high acceptance
rate.



\subsection{Experiment: Estimating Body Shape}
\label{appendix:chap3:body}

\subsubsection{Parameter Selection}
\paragraph{Metropolis Hastings (\MH)}

Figure~\ref{fig:exp3_MH} shows the result of \MH~sampling with various
proposal standard deviations. The value of $0.1$ is found to be
best.

\paragraph{Metropolis Hastings Within Gibbs (\MHWG)}

For \MHWG sampling we select $0.3$ proposal standard
deviation. Results are shown in Fig.~\ref{fig:exp3_MHWG}.


\paragraph{Parallel Tempering (\PT)}

As before, results in Fig.~\ref{fig:exp3_PT}, the temperature levels
were selected to be $[1,3,27]$ due its slightly higher AR.

\paragraph{Effect of Mixture Coefficient in Informed Sampling (\MIXLMH)}

Figure~\ref{fig:exp3_alpha} shows the effect of $\alpha$ on PSRF and
AR. Since there is no significant differences in PSRF values for $0 \le
\alpha \le 0.8$, we choose $0.8$.


\begin{figure}[t]
\centering
  \subfigure[\MH]{%
    \includegraphics[width=.48\textwidth]{figures/supplementary/bodyShape_MH.pdf} \label{fig:exp3_MH}
  }
  \subfigure[\MHWG]{%
    \includegraphics[width=.48\textwidth]{figures/supplementary/bodyShape_MHWG.pdf} \label{fig:exp3_MHWG}
  }
\\
  \subfigure[\PT]{%
    \includegraphics[width=.48\textwidth]{figures/supplementary/bodyShape_PT.pdf} \label{fig:exp3_PT}
  }
  \subfigure[\MIXLMH]{%
    \includegraphics[width=.48\textwidth]{figures/supplementary/bodyShape_alpha.pdf} \label{fig:exp3_alpha}
  }
\\
  \mycaption{Results of the `Body Shape Estimation' experiment}{PRSFs and
    Acceptance rates corresponding to various standard deviations of
    (a) \MH, (b) \MHWG; (c) various temperature level combinations
    of \PT sampling and; (d) various mixture coefficients of the
    informed \MIXLMH sampling.}
\end{figure}


\subsection{Results Overview}
Figure~\ref{fig:exp_summary} shows the summary results of the all the three
experimental studies related to informed sampler.
\begin{figure*}[h!]
\centering
  \subfigure[Results for: Estimating Camera Extrinsics]{%
    \includegraphics[width=0.9\textwidth]{figures/supplementary/camPose_ALL.pdf} \label{fig:exp1_all}
  }
  \subfigure[Results for: Occluding Tiles]{%
    \includegraphics[width=0.9\textwidth]{figures/supplementary/occlusionExp_ALL.pdf} \label{fig:exp2_all}
  }
  \subfigure[Results for: Estimating Body Shape]{%
    \includegraphics[width=0.9\textwidth]{figures/supplementary/bodyShape_ALL.pdf} \label{fig:exp3_all}
  }
  \label{fig:exp_summary}
  \mycaption{Summary of the statistics for the three experiments}{Shown are
    for several baseline methods and the informed samplers the
    acceptance rates (left), PSRFs (middle), and RMSE values
    (right). All results are median results over multiple test
    examples.}
\end{figure*}

\subsection{Additional Qualitative Results}

\subsubsection{Occluding Tiles}
In Figure~\ref{fig:exp2_visual_more} more qualitative results of the
occluding tiles experiment are shown. The informed sampling approach
(\INFBMHWG) is better than the best baseline (\MHWG). This still is a
very challenging problem since the parameters for occluded tiles are
flat over a large region. Some of the posterior variance of the
occluded tiles is already captured by the informed sampler.

\begin{figure*}[h!]
\begin{center}
\centerline{\includegraphics[width=0.95\textwidth]{figures/supplementary/occlusionExp_Visual.pdf}}
\mycaption{Additional qualitative results of the occluding tiles experiment}
  {From left to right: (a)
  Given image, (b) Ground truth tiles, (c) OpenCV heuristic and most probable estimates
  from 5000 samples obtained by (d) MHWG sampler (best baseline) and
  (e) our INF-BMHWG sampler. (f) Posterior expectation of the tiles
  boundaries obtained by INF-BMHWG sampling (First 2000 samples are
  discarded as burn-in).}
\label{fig:exp2_visual_more}
\end{center}
\end{figure*}

\subsubsection{Body Shape}
Figure~\ref{fig:exp3_bodyMeshes} shows some more results of 3D mesh
reconstruction using posterior samples obtained by our informed
sampling \MIXLMH.

\begin{figure*}[t]
\begin{center}
\centerline{\includegraphics[width=0.75\textwidth]{figures/supplementary/bodyMeshResults.pdf}}
\mycaption{Qualitative results for the body shape experiment}
  {Shown is the 3D mesh reconstruction results with first 1000 samples obtained
  using the \MIXLMH informed sampling method. (blue indicates small
  values and red indicates high values)}
\label{fig:exp3_bodyMeshes}
\end{center}
\end{figure*}

\clearpage



\section{Additional Results on the Face Problem with CMP}

Figure~\ref{fig:shading-qualitative-multiple-subjects-supp} shows inference results for reflectance maps, normal maps and lights for randomly chosen test images, and Fig.~\ref{fig:shading-qualitative-same-subject-supp} shows reflectance estimation results on multiple images of the same subject produced under different illumination conditions. CMP is able to produce estimates that are closer to the groundtruth across different subjects and illumination conditions.

\begin{figure*}[h]
  \begin{center}
  \centerline{\includegraphics[width=1.0\columnwidth]{figures/face_cmp_visual_results_supp.pdf}}
  \vspace{-1.2cm}
  \end{center}
	\mycaption{A visual comparison of inference results}{(a)~Observed images. (b)~Inferred reflectance maps. \textit{GT} is the photometric stereo groundtruth, \textit{BU} is the Biswas \etal (2009) reflectance estimate and \textit{Forest} is the consensus prediction. (c)~The variance of the inferred reflectance estimate produced by \MTD (normalized across rows).(d)~Visualization of inferred light directions. (e)~Inferred normal maps.}
	\label{fig:shading-qualitative-multiple-subjects-supp}
\end{figure*}


\begin{figure*}[h]
	\centering
	\setlength\fboxsep{0.2mm}
	\setlength\fboxrule{0pt}
	\begin{tikzpicture}

		\matrix at (0, 0) [matrix of nodes, nodes={anchor=east}, column sep=-0.05cm, row sep=-0.2cm]
		{
			\fbox{\includegraphics[width=1cm]{figures/sample_3_4_X.png}} &
			\fbox{\includegraphics[width=1cm]{figures/sample_3_4_GT.png}} &
			\fbox{\includegraphics[width=1cm]{figures/sample_3_4_BISWAS.png}}  &
			\fbox{\includegraphics[width=1cm]{figures/sample_3_4_VMP.png}}  &
			\fbox{\includegraphics[width=1cm]{figures/sample_3_4_FOREST.png}}  &
			\fbox{\includegraphics[width=1cm]{figures/sample_3_4_CMP.png}}  &
			\fbox{\includegraphics[width=1cm]{figures/sample_3_4_CMPVAR.png}}
			 \\

			\fbox{\includegraphics[width=1cm]{figures/sample_3_5_X.png}} &
			\fbox{\includegraphics[width=1cm]{figures/sample_3_5_GT.png}} &
			\fbox{\includegraphics[width=1cm]{figures/sample_3_5_BISWAS.png}}  &
			\fbox{\includegraphics[width=1cm]{figures/sample_3_5_VMP.png}}  &
			\fbox{\includegraphics[width=1cm]{figures/sample_3_5_FOREST.png}}  &
			\fbox{\includegraphics[width=1cm]{figures/sample_3_5_CMP.png}}  &
			\fbox{\includegraphics[width=1cm]{figures/sample_3_5_CMPVAR.png}}
			 \\

			\fbox{\includegraphics[width=1cm]{figures/sample_3_6_X.png}} &
			\fbox{\includegraphics[width=1cm]{figures/sample_3_6_GT.png}} &
			\fbox{\includegraphics[width=1cm]{figures/sample_3_6_BISWAS.png}}  &
			\fbox{\includegraphics[width=1cm]{figures/sample_3_6_VMP.png}}  &
			\fbox{\includegraphics[width=1cm]{figures/sample_3_6_FOREST.png}}  &
			\fbox{\includegraphics[width=1cm]{figures/sample_3_6_CMP.png}}  &
			\fbox{\includegraphics[width=1cm]{figures/sample_3_6_CMPVAR.png}}
			 \\
	     };

       \node at (-3.85, -2.0) {\small Observed};
       \node at (-2.55, -2.0) {\small `GT'};
       \node at (-1.27, -2.0) {\small BU};
       \node at (0.0, -2.0) {\small MP};
       \node at (1.27, -2.0) {\small Forest};
       \node at (2.55, -2.0) {\small \textbf{CMP}};
       \node at (3.85, -2.0) {\small Variance};

	\end{tikzpicture}
	\mycaption{Robustness to varying illumination}{Reflectance estimation on a subject images with varying illumination. Left to right: observed image, photometric stereo estimate (GT)
  which is used as a proxy for groundtruth, bottom-up estimate of \cite{Biswas2009}, VMP result, consensus forest estimate, CMP mean, and CMP variance.}
	\label{fig:shading-qualitative-same-subject-supp}
\end{figure*}

\clearpage

\section{Additional Material for Learning Sparse High Dimensional Filters}
\label{sec:appendix-bnn}

This part of supplementary material contains a more detailed overview of the permutohedral
lattice convolution in Section~\ref{sec:permconv}, more experiments in
Section~\ref{sec:addexps} and additional results with protocols for
the experiments presented in Chapter~\ref{chap:bnn} in Section~\ref{sec:addresults}.

\vspace{-0.2cm}
\subsection{General Permutohedral Convolutions}
\label{sec:permconv}

A core technical contribution of this work is the generalization of the Gaussian permutohedral lattice
convolution proposed in~\cite{adams2010fast} to the full non-separable case with the
ability to perform back-propagation. Although, conceptually, there are minor
differences between Gaussian and general parameterized filters, there are non-trivial practical
differences in terms of the algorithmic implementation. The Gauss filters belong to
the separable class and can thus be decomposed into multiple
sequential one dimensional convolutions. We are interested in the general filter
convolutions, which can not be decomposed. Thus, performing a general permutohedral
convolution at a lattice point requires the computation of the inner product with the
neighboring elements in all the directions in the high-dimensional space.

Here, we give more details of the implementation differences of separable
and non-separable filters. In the following, we will explain the scalar case first.
Recall, that the forward pass of general permutohedral convolution
involves 3 steps: \textit{splatting}, \textit{convolving} and \textit{slicing}.
We follow the same splatting and slicing strategies as in~\cite{adams2010fast}
since these operations do not depend on the filter kernel. The main difference
between our work and the existing implementation of~\cite{adams2010fast} is
the way that the convolution operation is executed. This proceeds by constructing
a \emph{blur neighbor} matrix $K$ that stores for every lattice point all
values of the lattice neighbors that are needed to compute the filter output.

\begin{figure}[t!]
  \centering
    \includegraphics[width=0.6\columnwidth]{figures/supplementary/lattice_construction}
  \mycaption{Illustration of 1D permutohedral lattice construction}
  {A $4\times 4$ $(x,y)$ grid lattice is projected onto the plane defined by the normal
  vector $(1,1)^{\top}$. This grid has $s+1=4$ and $d=2$ $(s+1)^{d}=4^2=16$ elements.
  In the projection, all points of the same color are projected onto the same points in the plane.
  The number of elements of the projected lattice is $t=(s+1)^d-s^d=4^2-3^2=7$, that is
  the $(4\times 4)$ grid minus the size of lattice that is $1$ smaller at each size, in this
  case a $(3\times 3)$ lattice (the upper right $(3\times 3)$ elements).
  }
\label{fig:latticeconstruction}
\end{figure}

The blur neighbor matrix is constructed by traversing through all the populated
lattice points and their neighboring elements.
% For efficiency, we do this matrix construction recursively with shared computations
% since $n^{th}$ neighbourhood elements are $1^{st}$ neighborhood elements of $n-1^{th}$ neighbourhood elements. \pg{do not understand}
This is done recursively to share computations. For any lattice point, the neighbors that are
$n$ hops away are the direct neighbors of the points that are $n-1$ hops away.
The size of a $d$ dimensional spatial filter with width $s+1$ is $(s+1)^{d}$ (\eg, a
$3\times 3$ filter, $s=2$ in $d=2$ has $3^2=9$ elements) and this size grows
exponentially in the number of dimensions $d$. The permutohedral lattice is constructed by
projecting a regular grid onto the plane spanned by the $d$ dimensional normal vector ${(1,\ldots,1)}^{\top}$. See
Fig.~\ref{fig:latticeconstruction} for an illustration of the 1D lattice construction.
Many corners of a grid filter are projected onto the same point, in total $t = {(s+1)}^{d} -
s^{d}$ elements remain in the permutohedral filter with $s$ neighborhood in $d-1$ dimensions.
If the lattice has $m$ populated elements, the
matrix $K$ has size $t\times m$. Note that, since the input signal is typically
sparse, only a few lattice corners are being populated in the \textit{slicing} step.
We use a hash-table to keep track of these points and traverse only through
the populated lattice points for this neighborhood matrix construction.

Once the blur neighbor matrix $K$ is constructed, we can perform the convolution
by the matrix vector multiplication
\begin{equation}
\ell' = BK,
\label{eq:conv}
\end{equation}
where $B$ is the $1 \times t$ filter kernel (whose values we will learn) and $\ell'\in\mathbb{R}^{1\times m}$
is the result of the filtering at the $m$ lattice points. In practice, we found that the
matrix $K$ is sometimes too large to fit into GPU memory and we divided the matrix $K$
into smaller pieces to compute Eq.~\ref{eq:conv} sequentially.

In the general multi-dimensional case, the signal $\ell$ is of $c$ dimensions. Then
the kernel $B$ is of size $c \times t$ and $K$ stores the $c$ dimensional vectors
accordingly. When the input and output points are different, we slice only the
input points and splat only at the output points.


\subsection{Additional Experiments}
\label{sec:addexps}
In this section, we discuss more use-cases for the learned bilateral filters, one
use-case of BNNs and two single filter applications for image and 3D mesh denoising.

\subsubsection{Recognition of subsampled MNIST}\label{sec:app_mnist}

One of the strengths of the proposed filter convolution is that it does not
require the input to lie on a regular grid. The only requirement is to define a distance
between features of the input signal.
We highlight this feature with the following experiment using the
classical MNIST ten class classification problem~\cite{lecun1998mnist}. We sample a
sparse set of $N$ points $(x,y)\in [0,1]\times [0,1]$
uniformly at random in the input image, use their interpolated values
as signal and the \emph{continuous} $(x,y)$ positions as features. This mimics
sub-sampling of a high-dimensional signal. To compare against a spatial convolution,
we interpolate the sparse set of values at the grid positions.

We take a reference implementation of LeNet~\cite{lecun1998gradient} that
is part of the Caffe project~\cite{jia2014caffe} and compare it
against the same architecture but replacing the first convolutional
layer with a bilateral convolution layer (BCL). The filter size
and numbers are adjusted to get a comparable number of parameters
($5\times 5$ for LeNet, $2$-neighborhood for BCL).

The results are shown in Table~\ref{tab:all-results}. We see that training
on the original MNIST data (column Original, LeNet vs. BNN) leads to a slight
decrease in performance of the BNN (99.03\%) compared to LeNet
(99.19\%). The BNN can be trained and evaluated on sparse
signals, and we resample the image as described above for $N=$ 100\%, 60\% and
20\% of the total number of pixels. The methods are also evaluated
on test images that are subsampled in the same way. Note that we can
train and test with different subsampling rates. We introduce an additional
bilinear interpolation layer for the LeNet architecture to train on the same
data. In essence, both models perform a spatial interpolation and thus we
expect them to yield a similar classification accuracy. Once the data is of
higher dimensions, the permutohedral convolution will be faster due to hashing
the sparse input points, as well as less memory demanding in comparison to
naive application of a spatial convolution with interpolated values.

\begin{table}[t]
  \begin{center}
    \footnotesize
    \centering
    \begin{tabular}[t]{lllll}
      \toprule
              &     & \multicolumn{3}{c}{Test Subsampling} \\
       Method  & Original & 100\% & 60\% & 20\%\\
      \midrule
       LeNet &  \textbf{0.9919} & 0.9660 & 0.9348 & \textbf{0.6434} \\
       BNN &  0.9903 & \textbf{0.9844} & \textbf{0.9534} & 0.5767 \\
      \hline
       LeNet 100\% & 0.9856 & 0.9809 & 0.9678 & \textbf{0.7386} \\
       BNN 100\% & \textbf{0.9900} & \textbf{0.9863} & \textbf{0.9699} & 0.6910 \\
      \hline
       LeNet 60\% & 0.9848 & 0.9821 & 0.9740 & 0.8151 \\
       BNN 60\% & \textbf{0.9885} & \textbf{0.9864} & \textbf{0.9771} & \textbf{0.8214}\\
      \hline
       LeNet 20\% & \textbf{0.9763} & \textbf{0.9754} & 0.9695 & 0.8928 \\
       BNN 20\% & 0.9728 & 0.9735 & \textbf{0.9701} & \textbf{0.9042}\\
      \bottomrule
    \end{tabular}
  \end{center}
\vspace{-.2cm}
\caption{Classification accuracy on MNIST. We compare the
    LeNet~\cite{lecun1998gradient} implementation that is part of
    Caffe~\cite{jia2014caffe} to the network with the first layer
    replaced by a bilateral convolution layer (BCL). Both are trained
    on the original image resolution (first two rows). Three more BNN
    and CNN models are trained with randomly subsampled images (100\%,
    60\% and 20\% of the pixels). An additional bilinear interpolation
    layer samples the input signal on a spatial grid for the CNN model.
  }
  \label{tab:all-results}
\vspace{-.5cm}
\end{table}

\subsubsection{Image Denoising}

The main application that inspired the development of the bilateral
filtering operation is image denoising~\cite{aurich1995non}, there
using a single Gaussian kernel. Our development allows to learn this
kernel function from data and we explore how to improve using a \emph{single}
but more general bilateral filter.

We use the Berkeley segmentation dataset
(BSDS500)~\cite{arbelaezi2011bsds500} as a test bed. The color
images in the dataset are converted to gray-scale,
and corrupted with Gaussian noise with a standard deviation of
$\frac {25} {255}$.

We compare the performance of four different filter models on a
denoising task.
The first baseline model (`Spatial' in Table \ref{tab:denoising}, $25$
weights) uses a single spatial filter with a kernel size of
$5$ and predicts the scalar gray-scale value at the center pixel. The next model
(`Gauss Bilateral') applies a bilateral \emph{Gaussian}
filter to the noisy input, using position and intensity features $\f=(x,y,v)^\top$.
The third setup (`Learned Bilateral', $65$ weights)
takes a Gauss kernel as initialization and
fits all filter weights on the train set to minimize the
mean squared error with respect to the clean images.
We run a combination
of spatial and permutohedral convolutions on spatial and bilateral
features (`Spatial + Bilateral (Learned)') to check for a complementary
performance of the two convolutions.

\label{sec:exp:denoising}
\begin{table}[!h]
\begin{center}
  \footnotesize
  \begin{tabular}[t]{lr}
    \toprule
    Method & PSNR \\
    \midrule
    Noisy Input & $20.17$ \\
    Spatial & $26.27$ \\
    Gauss Bilateral & $26.51$ \\
    Learned Bilateral & $26.58$ \\
    Spatial + Bilateral (Learned) & \textbf{$26.65$} \\
    \bottomrule
  \end{tabular}
\end{center}
\vspace{-0.5em}
\caption{PSNR results of a denoising task using the BSDS500
  dataset~\cite{arbelaezi2011bsds500}}
\vspace{-0.5em}
\label{tab:denoising}
\end{table}
\vspace{-0.2em}

The PSNR scores evaluated on full images of the test set are
shown in Table \ref{tab:denoising}. We find that an untrained bilateral
filter already performs better than a trained spatial convolution
($26.27$ to $26.51$). A learned convolution further improve the
performance slightly. We chose this simple one-kernel setup to
validate an advantage of the generalized bilateral filter. A competitive
denoising system would employ RGB color information and also
needs to be properly adjusted in network size. Multi-layer perceptrons
have obtained state-of-the-art denoising results~\cite{burger12cvpr}
and the permutohedral lattice layer can readily be used in such an
architecture, which is intended future work.

\subsection{Additional results}
\label{sec:addresults}

This section contains more qualitative results for the experiments presented in Chapter~\ref{chap:bnn}.

\begin{figure*}[th!]
  \centering
    \includegraphics[width=\columnwidth,trim={5cm 2.5cm 5cm 4.5cm},clip]{figures/supplementary/lattice_viz.pdf}
    \vspace{-0.7cm}
  \mycaption{Visualization of the Permutohedral Lattice}
  {Sample lattice visualizations for different feature spaces. All pixels falling in the same simplex cell are shown with
  the same color. $(x,y)$ features correspond to image pixel positions, and $(r,g,b) \in [0,255]$ correspond
  to the red, green and blue color values.}
\label{fig:latticeviz}
\end{figure*}

\subsubsection{Lattice Visualization}

Figure~\ref{fig:latticeviz} shows sample lattice visualizations for different feature spaces.

\newcolumntype{L}[1]{>{\raggedright\let\newline\\\arraybackslash\hspace{0pt}}b{#1}}
\newcolumntype{C}[1]{>{\centering\let\newline\\\arraybackslash\hspace{0pt}}b{#1}}
\newcolumntype{R}[1]{>{\raggedleft\let\newline\\\arraybackslash\hspace{0pt}}b{#1}}

\subsubsection{Color Upsampling}\label{sec:color_upsampling}
\label{sec:col_upsample_extra}

Some images of the upsampling for the Pascal
VOC12 dataset are shown in Fig.~\ref{fig:Colour_upsample_visuals}. It is
especially the low level image details that are better preserved with
a learned bilateral filter compared to the Gaussian case.

\begin{figure*}[t!]
  \centering
    \subfigure{%
   \raisebox{2.0em}{
    \includegraphics[width=.06\columnwidth]{figures/supplementary/2007_004969.jpg}
   }
  }
  \subfigure{%
    \includegraphics[width=.17\columnwidth]{figures/supplementary/2007_004969_gray.pdf}
  }
  \subfigure{%
    \includegraphics[width=.17\columnwidth]{figures/supplementary/2007_004969_gt.pdf}
  }
  \subfigure{%
    \includegraphics[width=.17\columnwidth]{figures/supplementary/2007_004969_bicubic.pdf}
  }
  \subfigure{%
    \includegraphics[width=.17\columnwidth]{figures/supplementary/2007_004969_gauss.pdf}
  }
  \subfigure{%
    \includegraphics[width=.17\columnwidth]{figures/supplementary/2007_004969_learnt.pdf}
  }\\
    \subfigure{%
   \raisebox{2.0em}{
    \includegraphics[width=.06\columnwidth]{figures/supplementary/2007_003106.jpg}
   }
  }
  \subfigure{%
    \includegraphics[width=.17\columnwidth]{figures/supplementary/2007_003106_gray.pdf}
  }
  \subfigure{%
    \includegraphics[width=.17\columnwidth]{figures/supplementary/2007_003106_gt.pdf}
  }
  \subfigure{%
    \includegraphics[width=.17\columnwidth]{figures/supplementary/2007_003106_bicubic.pdf}
  }
  \subfigure{%
    \includegraphics[width=.17\columnwidth]{figures/supplementary/2007_003106_gauss.pdf}
  }
  \subfigure{%
    \includegraphics[width=.17\columnwidth]{figures/supplementary/2007_003106_learnt.pdf}
  }\\
  \setcounter{subfigure}{0}
  \small{
  \subfigure[Inp.]{%
  \raisebox{2.0em}{
    \includegraphics[width=.06\columnwidth]{figures/supplementary/2007_006837.jpg}
   }
  }
  \subfigure[Guidance]{%
    \includegraphics[width=.17\columnwidth]{figures/supplementary/2007_006837_gray.pdf}
  }
   \subfigure[GT]{%
    \includegraphics[width=.17\columnwidth]{figures/supplementary/2007_006837_gt.pdf}
  }
  \subfigure[Bicubic]{%
    \includegraphics[width=.17\columnwidth]{figures/supplementary/2007_006837_bicubic.pdf}
  }
  \subfigure[Gauss-BF]{%
    \includegraphics[width=.17\columnwidth]{figures/supplementary/2007_006837_gauss.pdf}
  }
  \subfigure[Learned-BF]{%
    \includegraphics[width=.17\columnwidth]{figures/supplementary/2007_006837_learnt.pdf}
  }
  }
  \vspace{-0.5cm}
  \mycaption{Color Upsampling}{Color $8\times$ upsampling results
  using different methods, from left to right, (a)~Low-resolution input color image (Inp.),
  (b)~Gray scale guidance image, (c)~Ground-truth color image; Upsampled color images with
  (d)~Bicubic interpolation, (e) Gauss bilateral upsampling and, (f)~Learned bilateral
  updampgling (best viewed on screen).}

\label{fig:Colour_upsample_visuals}
\end{figure*}

\subsubsection{Depth Upsampling}
\label{sec:depth_upsample_extra}

Figure~\ref{fig:depth_upsample_visuals} presents some more qualitative results comparing bicubic interpolation, Gauss
bilateral and learned bilateral upsampling on NYU depth dataset image~\cite{silberman2012indoor}.

\subsubsection{Character Recognition}\label{sec:app_character}

 Figure~\ref{fig:nnrecognition} shows the schematic of different layers
 of the network architecture for LeNet-7~\cite{lecun1998mnist}
 and DeepCNet(5, 50)~\cite{ciresan2012multi,graham2014spatially}. For the BNN variants, the first layer filters are replaced
 with learned bilateral filters and are learned end-to-end.

\subsubsection{Semantic Segmentation}\label{sec:app_semantic_segmentation}
\label{sec:semantic_bnn_extra}

Some more visual results for semantic segmentation are shown in Figure~\ref{fig:semantic_visuals}.
These include the underlying DeepLab CNN\cite{chen2014semantic} result (DeepLab),
the 2 step mean-field result with Gaussian edge potentials (+2stepMF-GaussCRF)
and also corresponding results with learned edge potentials (+2stepMF-LearnedCRF).
In general, we observe that mean-field in learned CRF leads to slightly dilated
classification regions in comparison to using Gaussian CRF thereby filling-in the
false negative pixels and also correcting some mis-classified regions.

\begin{figure*}[t!]
  \centering
    \subfigure{%
   \raisebox{2.0em}{
    \includegraphics[width=.06\columnwidth]{figures/supplementary/2bicubic}
   }
  }
  \subfigure{%
    \includegraphics[width=.17\columnwidth]{figures/supplementary/2given_image}
  }
  \subfigure{%
    \includegraphics[width=.17\columnwidth]{figures/supplementary/2ground_truth}
  }
  \subfigure{%
    \includegraphics[width=.17\columnwidth]{figures/supplementary/2bicubic}
  }
  \subfigure{%
    \includegraphics[width=.17\columnwidth]{figures/supplementary/2gauss}
  }
  \subfigure{%
    \includegraphics[width=.17\columnwidth]{figures/supplementary/2learnt}
  }\\
    \subfigure{%
   \raisebox{2.0em}{
    \includegraphics[width=.06\columnwidth]{figures/supplementary/32bicubic}
   }
  }
  \subfigure{%
    \includegraphics[width=.17\columnwidth]{figures/supplementary/32given_image}
  }
  \subfigure{%
    \includegraphics[width=.17\columnwidth]{figures/supplementary/32ground_truth}
  }
  \subfigure{%
    \includegraphics[width=.17\columnwidth]{figures/supplementary/32bicubic}
  }
  \subfigure{%
    \includegraphics[width=.17\columnwidth]{figures/supplementary/32gauss}
  }
  \subfigure{%
    \includegraphics[width=.17\columnwidth]{figures/supplementary/32learnt}
  }\\
  \setcounter{subfigure}{0}
  \small{
  \subfigure[Inp.]{%
  \raisebox{2.0em}{
    \includegraphics[width=.06\columnwidth]{figures/supplementary/41bicubic}
   }
  }
  \subfigure[Guidance]{%
    \includegraphics[width=.17\columnwidth]{figures/supplementary/41given_image}
  }
   \subfigure[GT]{%
    \includegraphics[width=.17\columnwidth]{figures/supplementary/41ground_truth}
  }
  \subfigure[Bicubic]{%
    \includegraphics[width=.17\columnwidth]{figures/supplementary/41bicubic}
  }
  \subfigure[Gauss-BF]{%
    \includegraphics[width=.17\columnwidth]{figures/supplementary/41gauss}
  }
  \subfigure[Learned-BF]{%
    \includegraphics[width=.17\columnwidth]{figures/supplementary/41learnt}
  }
  }
  \mycaption{Depth Upsampling}{Depth $8\times$ upsampling results
  using different upsampling strategies, from left to right,
  (a)~Low-resolution input depth image (Inp.),
  (b)~High-resolution guidance image, (c)~Ground-truth depth; Upsampled depth images with
  (d)~Bicubic interpolation, (e) Gauss bilateral upsampling and, (f)~Learned bilateral
  updampgling (best viewed on screen).}

\label{fig:depth_upsample_visuals}
\end{figure*}

\subsubsection{Material Segmentation}\label{sec:app_material_segmentation}
\label{sec:material_bnn_extra}

In Fig.~\ref{fig:material_visuals-app2}, we present visual results comparing 2 step
mean-field inference with Gaussian and learned pairwise CRF potentials. In
general, we observe that the pixels belonging to dominant classes in the
training data are being more accurately classified with learned CRF. This leads to
a significant improvements in overall pixel accuracy. This also results
in a slight decrease of the accuracy from less frequent class pixels thereby
slightly reducing the average class accuracy with learning. We attribute this
to the type of annotation that is available for this dataset, which is not
for the entire image but for some segments in the image. We have very few
images of the infrequent classes to combat this behaviour during training.

\subsubsection{Experiment Protocols}
\label{sec:protocols}

Table~\ref{tbl:parameters} shows experiment protocols of different experiments.

 \begin{figure*}[t!]
  \centering
  \subfigure[LeNet-7]{
    \includegraphics[width=0.7\columnwidth]{figures/supplementary/lenet_cnn_network}
    }\\
    \subfigure[DeepCNet]{
    \includegraphics[width=\columnwidth]{figures/supplementary/deepcnet_cnn_network}
    }
  \mycaption{CNNs for Character Recognition}
  {Schematic of (top) LeNet-7~\cite{lecun1998mnist} and (bottom) DeepCNet(5,50)~\cite{ciresan2012multi,graham2014spatially} architectures used in Assamese
  character recognition experiments.}
\label{fig:nnrecognition}
\end{figure*}

\definecolor{voc_1}{RGB}{0, 0, 0}
\definecolor{voc_2}{RGB}{128, 0, 0}
\definecolor{voc_3}{RGB}{0, 128, 0}
\definecolor{voc_4}{RGB}{128, 128, 0}
\definecolor{voc_5}{RGB}{0, 0, 128}
\definecolor{voc_6}{RGB}{128, 0, 128}
\definecolor{voc_7}{RGB}{0, 128, 128}
\definecolor{voc_8}{RGB}{128, 128, 128}
\definecolor{voc_9}{RGB}{64, 0, 0}
\definecolor{voc_10}{RGB}{192, 0, 0}
\definecolor{voc_11}{RGB}{64, 128, 0}
\definecolor{voc_12}{RGB}{192, 128, 0}
\definecolor{voc_13}{RGB}{64, 0, 128}
\definecolor{voc_14}{RGB}{192, 0, 128}
\definecolor{voc_15}{RGB}{64, 128, 128}
\definecolor{voc_16}{RGB}{192, 128, 128}
\definecolor{voc_17}{RGB}{0, 64, 0}
\definecolor{voc_18}{RGB}{128, 64, 0}
\definecolor{voc_19}{RGB}{0, 192, 0}
\definecolor{voc_20}{RGB}{128, 192, 0}
\definecolor{voc_21}{RGB}{0, 64, 128}
\definecolor{voc_22}{RGB}{128, 64, 128}

\begin{figure*}[t]
  \centering
  \small{
  \fcolorbox{white}{voc_1}{\rule{0pt}{6pt}\rule{6pt}{0pt}} Background~~
  \fcolorbox{white}{voc_2}{\rule{0pt}{6pt}\rule{6pt}{0pt}} Aeroplane~~
  \fcolorbox{white}{voc_3}{\rule{0pt}{6pt}\rule{6pt}{0pt}} Bicycle~~
  \fcolorbox{white}{voc_4}{\rule{0pt}{6pt}\rule{6pt}{0pt}} Bird~~
  \fcolorbox{white}{voc_5}{\rule{0pt}{6pt}\rule{6pt}{0pt}} Boat~~
  \fcolorbox{white}{voc_6}{\rule{0pt}{6pt}\rule{6pt}{0pt}} Bottle~~
  \fcolorbox{white}{voc_7}{\rule{0pt}{6pt}\rule{6pt}{0pt}} Bus~~
  \fcolorbox{white}{voc_8}{\rule{0pt}{6pt}\rule{6pt}{0pt}} Car~~ \\
  \fcolorbox{white}{voc_9}{\rule{0pt}{6pt}\rule{6pt}{0pt}} Cat~~
  \fcolorbox{white}{voc_10}{\rule{0pt}{6pt}\rule{6pt}{0pt}} Chair~~
  \fcolorbox{white}{voc_11}{\rule{0pt}{6pt}\rule{6pt}{0pt}} Cow~~
  \fcolorbox{white}{voc_12}{\rule{0pt}{6pt}\rule{6pt}{0pt}} Dining Table~~
  \fcolorbox{white}{voc_13}{\rule{0pt}{6pt}\rule{6pt}{0pt}} Dog~~
  \fcolorbox{white}{voc_14}{\rule{0pt}{6pt}\rule{6pt}{0pt}} Horse~~
  \fcolorbox{white}{voc_15}{\rule{0pt}{6pt}\rule{6pt}{0pt}} Motorbike~~
  \fcolorbox{white}{voc_16}{\rule{0pt}{6pt}\rule{6pt}{0pt}} Person~~ \\
  \fcolorbox{white}{voc_17}{\rule{0pt}{6pt}\rule{6pt}{0pt}} Potted Plant~~
  \fcolorbox{white}{voc_18}{\rule{0pt}{6pt}\rule{6pt}{0pt}} Sheep~~
  \fcolorbox{white}{voc_19}{\rule{0pt}{6pt}\rule{6pt}{0pt}} Sofa~~
  \fcolorbox{white}{voc_20}{\rule{0pt}{6pt}\rule{6pt}{0pt}} Train~~
  \fcolorbox{white}{voc_21}{\rule{0pt}{6pt}\rule{6pt}{0pt}} TV monitor~~ \\
  }
  \subfigure{%
    \includegraphics[width=.18\columnwidth]{figures/supplementary/2007_001423_given.jpg}
  }
  \subfigure{%
    \includegraphics[width=.18\columnwidth]{figures/supplementary/2007_001423_gt.png}
  }
  \subfigure{%
    \includegraphics[width=.18\columnwidth]{figures/supplementary/2007_001423_cnn.png}
  }
  \subfigure{%
    \includegraphics[width=.18\columnwidth]{figures/supplementary/2007_001423_gauss.png}
  }
  \subfigure{%
    \includegraphics[width=.18\columnwidth]{figures/supplementary/2007_001423_learnt.png}
  }\\
  \subfigure{%
    \includegraphics[width=.18\columnwidth]{figures/supplementary/2007_001430_given.jpg}
  }
  \subfigure{%
    \includegraphics[width=.18\columnwidth]{figures/supplementary/2007_001430_gt.png}
  }
  \subfigure{%
    \includegraphics[width=.18\columnwidth]{figures/supplementary/2007_001430_cnn.png}
  }
  \subfigure{%
    \includegraphics[width=.18\columnwidth]{figures/supplementary/2007_001430_gauss.png}
  }
  \subfigure{%
    \includegraphics[width=.18\columnwidth]{figures/supplementary/2007_001430_learnt.png}
  }\\
    \subfigure{%
    \includegraphics[width=.18\columnwidth]{figures/supplementary/2007_007996_given.jpg}
  }
  \subfigure{%
    \includegraphics[width=.18\columnwidth]{figures/supplementary/2007_007996_gt.png}
  }
  \subfigure{%
    \includegraphics[width=.18\columnwidth]{figures/supplementary/2007_007996_cnn.png}
  }
  \subfigure{%
    \includegraphics[width=.18\columnwidth]{figures/supplementary/2007_007996_gauss.png}
  }
  \subfigure{%
    \includegraphics[width=.18\columnwidth]{figures/supplementary/2007_007996_learnt.png}
  }\\
   \subfigure{%
    \includegraphics[width=.18\columnwidth]{figures/supplementary/2010_002682_given.jpg}
  }
  \subfigure{%
    \includegraphics[width=.18\columnwidth]{figures/supplementary/2010_002682_gt.png}
  }
  \subfigure{%
    \includegraphics[width=.18\columnwidth]{figures/supplementary/2010_002682_cnn.png}
  }
  \subfigure{%
    \includegraphics[width=.18\columnwidth]{figures/supplementary/2010_002682_gauss.png}
  }
  \subfigure{%
    \includegraphics[width=.18\columnwidth]{figures/supplementary/2010_002682_learnt.png}
  }\\
     \subfigure{%
    \includegraphics[width=.18\columnwidth]{figures/supplementary/2010_004789_given.jpg}
  }
  \subfigure{%
    \includegraphics[width=.18\columnwidth]{figures/supplementary/2010_004789_gt.png}
  }
  \subfigure{%
    \includegraphics[width=.18\columnwidth]{figures/supplementary/2010_004789_cnn.png}
  }
  \subfigure{%
    \includegraphics[width=.18\columnwidth]{figures/supplementary/2010_004789_gauss.png}
  }
  \subfigure{%
    \includegraphics[width=.18\columnwidth]{figures/supplementary/2010_004789_learnt.png}
  }\\
       \subfigure{%
    \includegraphics[width=.18\columnwidth]{figures/supplementary/2007_001311_given.jpg}
  }
  \subfigure{%
    \includegraphics[width=.18\columnwidth]{figures/supplementary/2007_001311_gt.png}
  }
  \subfigure{%
    \includegraphics[width=.18\columnwidth]{figures/supplementary/2007_001311_cnn.png}
  }
  \subfigure{%
    \includegraphics[width=.18\columnwidth]{figures/supplementary/2007_001311_gauss.png}
  }
  \subfigure{%
    \includegraphics[width=.18\columnwidth]{figures/supplementary/2007_001311_learnt.png}
  }\\
  \setcounter{subfigure}{0}
  \subfigure[Input]{%
    \includegraphics[width=.18\columnwidth]{figures/supplementary/2010_003531_given.jpg}
  }
  \subfigure[Ground Truth]{%
    \includegraphics[width=.18\columnwidth]{figures/supplementary/2010_003531_gt.png}
  }
  \subfigure[DeepLab]{%
    \includegraphics[width=.18\columnwidth]{figures/supplementary/2010_003531_cnn.png}
  }
  \subfigure[+GaussCRF]{%
    \includegraphics[width=.18\columnwidth]{figures/supplementary/2010_003531_gauss.png}
  }
  \subfigure[+LearnedCRF]{%
    \includegraphics[width=.18\columnwidth]{figures/supplementary/2010_003531_learnt.png}
  }
  \vspace{-0.3cm}
  \mycaption{Semantic Segmentation}{Example results of semantic segmentation.
  (c)~depicts the unary results before application of MF, (d)~after two steps of MF with Gaussian edge CRF potentials, (e)~after
  two steps of MF with learned edge CRF potentials.}
    \label{fig:semantic_visuals}
\end{figure*}


\definecolor{minc_1}{HTML}{771111}
\definecolor{minc_2}{HTML}{CAC690}
\definecolor{minc_3}{HTML}{EEEEEE}
\definecolor{minc_4}{HTML}{7C8FA6}
\definecolor{minc_5}{HTML}{597D31}
\definecolor{minc_6}{HTML}{104410}
\definecolor{minc_7}{HTML}{BB819C}
\definecolor{minc_8}{HTML}{D0CE48}
\definecolor{minc_9}{HTML}{622745}
\definecolor{minc_10}{HTML}{666666}
\definecolor{minc_11}{HTML}{D54A31}
\definecolor{minc_12}{HTML}{101044}
\definecolor{minc_13}{HTML}{444126}
\definecolor{minc_14}{HTML}{75D646}
\definecolor{minc_15}{HTML}{DD4348}
\definecolor{minc_16}{HTML}{5C8577}
\definecolor{minc_17}{HTML}{C78472}
\definecolor{minc_18}{HTML}{75D6D0}
\definecolor{minc_19}{HTML}{5B4586}
\definecolor{minc_20}{HTML}{C04393}
\definecolor{minc_21}{HTML}{D69948}
\definecolor{minc_22}{HTML}{7370D8}
\definecolor{minc_23}{HTML}{7A3622}
\definecolor{minc_24}{HTML}{000000}

\begin{figure*}[t]
  \centering
  \small{
  \fcolorbox{white}{minc_1}{\rule{0pt}{6pt}\rule{6pt}{0pt}} Brick~~
  \fcolorbox{white}{minc_2}{\rule{0pt}{6pt}\rule{6pt}{0pt}} Carpet~~
  \fcolorbox{white}{minc_3}{\rule{0pt}{6pt}\rule{6pt}{0pt}} Ceramic~~
  \fcolorbox{white}{minc_4}{\rule{0pt}{6pt}\rule{6pt}{0pt}} Fabric~~
  \fcolorbox{white}{minc_5}{\rule{0pt}{6pt}\rule{6pt}{0pt}} Foliage~~
  \fcolorbox{white}{minc_6}{\rule{0pt}{6pt}\rule{6pt}{0pt}} Food~~
  \fcolorbox{white}{minc_7}{\rule{0pt}{6pt}\rule{6pt}{0pt}} Glass~~
  \fcolorbox{white}{minc_8}{\rule{0pt}{6pt}\rule{6pt}{0pt}} Hair~~ \\
  \fcolorbox{white}{minc_9}{\rule{0pt}{6pt}\rule{6pt}{0pt}} Leather~~
  \fcolorbox{white}{minc_10}{\rule{0pt}{6pt}\rule{6pt}{0pt}} Metal~~
  \fcolorbox{white}{minc_11}{\rule{0pt}{6pt}\rule{6pt}{0pt}} Mirror~~
  \fcolorbox{white}{minc_12}{\rule{0pt}{6pt}\rule{6pt}{0pt}} Other~~
  \fcolorbox{white}{minc_13}{\rule{0pt}{6pt}\rule{6pt}{0pt}} Painted~~
  \fcolorbox{white}{minc_14}{\rule{0pt}{6pt}\rule{6pt}{0pt}} Paper~~
  \fcolorbox{white}{minc_15}{\rule{0pt}{6pt}\rule{6pt}{0pt}} Plastic~~\\
  \fcolorbox{white}{minc_16}{\rule{0pt}{6pt}\rule{6pt}{0pt}} Polished Stone~~
  \fcolorbox{white}{minc_17}{\rule{0pt}{6pt}\rule{6pt}{0pt}} Skin~~
  \fcolorbox{white}{minc_18}{\rule{0pt}{6pt}\rule{6pt}{0pt}} Sky~~
  \fcolorbox{white}{minc_19}{\rule{0pt}{6pt}\rule{6pt}{0pt}} Stone~~
  \fcolorbox{white}{minc_20}{\rule{0pt}{6pt}\rule{6pt}{0pt}} Tile~~
  \fcolorbox{white}{minc_21}{\rule{0pt}{6pt}\rule{6pt}{0pt}} Wallpaper~~
  \fcolorbox{white}{minc_22}{\rule{0pt}{6pt}\rule{6pt}{0pt}} Water~~
  \fcolorbox{white}{minc_23}{\rule{0pt}{6pt}\rule{6pt}{0pt}} Wood~~ \\
  }
  \subfigure{%
    \includegraphics[width=.18\columnwidth]{figures/supplementary/000010868_given.jpg}
  }
  \subfigure{%
    \includegraphics[width=.18\columnwidth]{figures/supplementary/000010868_gt.png}
  }
  \subfigure{%
    \includegraphics[width=.18\columnwidth]{figures/supplementary/000010868_cnn.png}
  }
  \subfigure{%
    \includegraphics[width=.18\columnwidth]{figures/supplementary/000010868_gauss.png}
  }
  \subfigure{%
    \includegraphics[width=.18\columnwidth]{figures/supplementary/000010868_learnt.png}
  }\\[-2ex]
  \subfigure{%
    \includegraphics[width=.18\columnwidth]{figures/supplementary/000006011_given.jpg}
  }
  \subfigure{%
    \includegraphics[width=.18\columnwidth]{figures/supplementary/000006011_gt.png}
  }
  \subfigure{%
    \includegraphics[width=.18\columnwidth]{figures/supplementary/000006011_cnn.png}
  }
  \subfigure{%
    \includegraphics[width=.18\columnwidth]{figures/supplementary/000006011_gauss.png}
  }
  \subfigure{%
    \includegraphics[width=.18\columnwidth]{figures/supplementary/000006011_learnt.png}
  }\\[-2ex]
    \subfigure{%
    \includegraphics[width=.18\columnwidth]{figures/supplementary/000008553_given.jpg}
  }
  \subfigure{%
    \includegraphics[width=.18\columnwidth]{figures/supplementary/000008553_gt.png}
  }
  \subfigure{%
    \includegraphics[width=.18\columnwidth]{figures/supplementary/000008553_cnn.png}
  }
  \subfigure{%
    \includegraphics[width=.18\columnwidth]{figures/supplementary/000008553_gauss.png}
  }
  \subfigure{%
    \includegraphics[width=.18\columnwidth]{figures/supplementary/000008553_learnt.png}
  }\\[-2ex]
   \subfigure{%
    \includegraphics[width=.18\columnwidth]{figures/supplementary/000009188_given.jpg}
  }
  \subfigure{%
    \includegraphics[width=.18\columnwidth]{figures/supplementary/000009188_gt.png}
  }
  \subfigure{%
    \includegraphics[width=.18\columnwidth]{figures/supplementary/000009188_cnn.png}
  }
  \subfigure{%
    \includegraphics[width=.18\columnwidth]{figures/supplementary/000009188_gauss.png}
  }
  \subfigure{%
    \includegraphics[width=.18\columnwidth]{figures/supplementary/000009188_learnt.png}
  }\\[-2ex]
  \setcounter{subfigure}{0}
  \subfigure[Input]{%
    \includegraphics[width=.18\columnwidth]{figures/supplementary/000023570_given.jpg}
  }
  \subfigure[Ground Truth]{%
    \includegraphics[width=.18\columnwidth]{figures/supplementary/000023570_gt.png}
  }
  \subfigure[DeepLab]{%
    \includegraphics[width=.18\columnwidth]{figures/supplementary/000023570_cnn.png}
  }
  \subfigure[+GaussCRF]{%
    \includegraphics[width=.18\columnwidth]{figures/supplementary/000023570_gauss.png}
  }
  \subfigure[+LearnedCRF]{%
    \includegraphics[width=.18\columnwidth]{figures/supplementary/000023570_learnt.png}
  }
  \mycaption{Material Segmentation}{Example results of material segmentation.
  (c)~depicts the unary results before application of MF, (d)~after two steps of MF with Gaussian edge CRF potentials, (e)~after two steps of MF with learned edge CRF potentials.}
    \label{fig:material_visuals-app2}
\end{figure*}


\begin{table*}[h]
\tiny
  \centering
    \begin{tabular}{L{2.3cm} L{2.25cm} C{1.5cm} C{0.7cm} C{0.6cm} C{0.7cm} C{0.7cm} C{0.7cm} C{1.6cm} C{0.6cm} C{0.6cm} C{0.6cm}}
      \toprule
& & & & & \multicolumn{3}{c}{\textbf{Data Statistics}} & \multicolumn{4}{c}{\textbf{Training Protocol}} \\

\textbf{Experiment} & \textbf{Feature Types} & \textbf{Feature Scales} & \textbf{Filter Size} & \textbf{Filter Nbr.} & \textbf{Train}  & \textbf{Val.} & \textbf{Test} & \textbf{Loss Type} & \textbf{LR} & \textbf{Batch} & \textbf{Epochs} \\
      \midrule
      \multicolumn{2}{c}{\textbf{Single Bilateral Filter Applications}} & & & & & & & & & \\
      \textbf{2$\times$ Color Upsampling} & Position$_{1}$, Intensity (3D) & 0.13, 0.17 & 65 & 2 & 10581 & 1449 & 1456 & MSE & 1e-06 & 200 & 94.5\\
      \textbf{4$\times$ Color Upsampling} & Position$_{1}$, Intensity (3D) & 0.06, 0.17 & 65 & 2 & 10581 & 1449 & 1456 & MSE & 1e-06 & 200 & 94.5\\
      \textbf{8$\times$ Color Upsampling} & Position$_{1}$, Intensity (3D) & 0.03, 0.17 & 65 & 2 & 10581 & 1449 & 1456 & MSE & 1e-06 & 200 & 94.5\\
      \textbf{16$\times$ Color Upsampling} & Position$_{1}$, Intensity (3D) & 0.02, 0.17 & 65 & 2 & 10581 & 1449 & 1456 & MSE & 1e-06 & 200 & 94.5\\
      \textbf{Depth Upsampling} & Position$_{1}$, Color (5D) & 0.05, 0.02 & 665 & 2 & 795 & 100 & 654 & MSE & 1e-07 & 50 & 251.6\\
      \textbf{Mesh Denoising} & Isomap (4D) & 46.00 & 63 & 2 & 1000 & 200 & 500 & MSE & 100 & 10 & 100.0 \\
      \midrule
      \multicolumn{2}{c}{\textbf{DenseCRF Applications}} & & & & & & & & &\\
      \multicolumn{2}{l}{\textbf{Semantic Segmentation}} & & & & & & & & &\\
      \textbf{- 1step MF} & Position$_{1}$, Color (5D); Position$_{1}$ (2D) & 0.01, 0.34; 0.34  & 665; 19  & 2; 2 & 10581 & 1449 & 1456 & Logistic & 0.1 & 5 & 1.4 \\
      \textbf{- 2step MF} & Position$_{1}$, Color (5D); Position$_{1}$ (2D) & 0.01, 0.34; 0.34 & 665; 19 & 2; 2 & 10581 & 1449 & 1456 & Logistic & 0.1 & 5 & 1.4 \\
      \textbf{- \textit{loose} 2step MF} & Position$_{1}$, Color (5D); Position$_{1}$ (2D) & 0.01, 0.34; 0.34 & 665; 19 & 2; 2 &10581 & 1449 & 1456 & Logistic & 0.1 & 5 & +1.9  \\ \\
      \multicolumn{2}{l}{\textbf{Material Segmentation}} & & & & & & & & &\\
      \textbf{- 1step MF} & Position$_{2}$, Lab-Color (5D) & 5.00, 0.05, 0.30  & 665 & 2 & 928 & 150 & 1798 & Weighted Logistic & 1e-04 & 24 & 2.6 \\
      \textbf{- 2step MF} & Position$_{2}$, Lab-Color (5D) & 5.00, 0.05, 0.30 & 665 & 2 & 928 & 150 & 1798 & Weighted Logistic & 1e-04 & 12 & +0.7 \\
      \textbf{- \textit{loose} 2step MF} & Position$_{2}$, Lab-Color (5D) & 5.00, 0.05, 0.30 & 665 & 2 & 928 & 150 & 1798 & Weighted Logistic & 1e-04 & 12 & +0.2\\
      \midrule
      \multicolumn{2}{c}{\textbf{Neural Network Applications}} & & & & & & & & &\\
      \textbf{Tiles: CNN-9$\times$9} & - & - & 81 & 4 & 10000 & 1000 & 1000 & Logistic & 0.01 & 100 & 500.0 \\
      \textbf{Tiles: CNN-13$\times$13} & - & - & 169 & 6 & 10000 & 1000 & 1000 & Logistic & 0.01 & 100 & 500.0 \\
      \textbf{Tiles: CNN-17$\times$17} & - & - & 289 & 8 & 10000 & 1000 & 1000 & Logistic & 0.01 & 100 & 500.0 \\
      \textbf{Tiles: CNN-21$\times$21} & - & - & 441 & 10 & 10000 & 1000 & 1000 & Logistic & 0.01 & 100 & 500.0 \\
      \textbf{Tiles: BNN} & Position$_{1}$, Color (5D) & 0.05, 0.04 & 63 & 1 & 10000 & 1000 & 1000 & Logistic & 0.01 & 100 & 30.0 \\
      \textbf{LeNet} & - & - & 25 & 2 & 5490 & 1098 & 1647 & Logistic & 0.1 & 100 & 182.2 \\
      \textbf{Crop-LeNet} & - & - & 25 & 2 & 5490 & 1098 & 1647 & Logistic & 0.1 & 100 & 182.2 \\
      \textbf{BNN-LeNet} & Position$_{2}$ (2D) & 20.00 & 7 & 1 & 5490 & 1098 & 1647 & Logistic & 0.1 & 100 & 182.2 \\
      \textbf{DeepCNet} & - & - & 9 & 1 & 5490 & 1098 & 1647 & Logistic & 0.1 & 100 & 182.2 \\
      \textbf{Crop-DeepCNet} & - & - & 9 & 1 & 5490 & 1098 & 1647 & Logistic & 0.1 & 100 & 182.2 \\
      \textbf{BNN-DeepCNet} & Position$_{2}$ (2D) & 40.00  & 7 & 1 & 5490 & 1098 & 1647 & Logistic & 0.1 & 100 & 182.2 \\
      \bottomrule
      \\
    \end{tabular}
    \mycaption{Experiment Protocols} {Experiment protocols for the different experiments presented in this work. \textbf{Feature Types}:
    Feature spaces used for the bilateral convolutions. Position$_1$ corresponds to un-normalized pixel positions whereas Position$_2$ corresponds
    to pixel positions normalized to $[0,1]$ with respect to the given image. \textbf{Feature Scales}: Cross-validated scales for the features used.
     \textbf{Filter Size}: Number of elements in the filter that is being learned. \textbf{Filter Nbr.}: Half-width of the filter. \textbf{Train},
     \textbf{Val.} and \textbf{Test} corresponds to the number of train, validation and test images used in the experiment. \textbf{Loss Type}: Type
     of loss used for back-propagation. ``MSE'' corresponds to Euclidean mean squared error loss and ``Logistic'' corresponds to multinomial logistic
     loss. ``Weighted Logistic'' is the class-weighted multinomial logistic loss. We weighted the loss with inverse class probability for material
     segmentation task due to the small availability of training data with class imbalance. \textbf{LR}: Fixed learning rate used in stochastic gradient
     descent. \textbf{Batch}: Number of images used in one parameter update step. \textbf{Epochs}: Number of training epochs. In all the experiments,
     we used fixed momentum of 0.9 and weight decay of 0.0005 for stochastic gradient descent. ```Color Upsampling'' experiments in this Table corresponds
     to those performed on Pascal VOC12 dataset images. For all experiments using Pascal VOC12 images, we use extended
     training segmentation dataset available from~\cite{hariharan2011moredata}, and used standard validation and test splits
     from the main dataset~\cite{voc2012segmentation}.}
  \label{tbl:parameters}
\end{table*}

\clearpage

\section{Parameters and Additional Results for Video Propagation Networks}

In this Section, we present experiment protocols and additional qualitative results for experiments
on video object segmentation, semantic video segmentation and video color
propagation. Table~\ref{tbl:parameters_supp} shows the feature scales and other parameters used in different experiments.
Figures~\ref{fig:video_seg_pos_supp} show some qualitative results on video object segmentation
with some failure cases in Fig.~\ref{fig:video_seg_neg_supp}.
Figure~\ref{fig:semantic_visuals_supp} shows some qualitative results on semantic video segmentation and
Fig.~\ref{fig:color_visuals_supp} shows results on video color propagation.

\newcolumntype{L}[1]{>{\raggedright\let\newline\\\arraybackslash\hspace{0pt}}b{#1}}
\newcolumntype{C}[1]{>{\centering\let\newline\\\arraybackslash\hspace{0pt}}b{#1}}
\newcolumntype{R}[1]{>{\raggedleft\let\newline\\\arraybackslash\hspace{0pt}}b{#1}}

\begin{table*}[h]
\tiny
  \centering
    \begin{tabular}{L{3.0cm} L{2.4cm} L{2.8cm} L{2.8cm} C{0.5cm} C{1.0cm} L{1.2cm}}
      \toprule
\textbf{Experiment} & \textbf{Feature Type} & \textbf{Feature Scale-1, $\Lambda_a$} & \textbf{Feature Scale-2, $\Lambda_b$} & \textbf{$\alpha$} & \textbf{Input Frames} & \textbf{Loss Type} \\
      \midrule
      \textbf{Video Object Segmentation} & ($x,y,Y,Cb,Cr,t$) & (0.02,0.02,0.07,0.4,0.4,0.01) & (0.03,0.03,0.09,0.5,0.5,0.2) & 0.5 & 9 & Logistic\\
      \midrule
      \textbf{Semantic Video Segmentation} & & & & & \\
      \textbf{with CNN1~\cite{yu2015multi}-NoFlow} & ($x,y,R,G,B,t$) & (0.08,0.08,0.2,0.2,0.2,0.04) & (0.11,0.11,0.2,0.2,0.2,0.04) & 0.5 & 3 & Logistic \\
      \textbf{with CNN1~\cite{yu2015multi}-Flow} & ($x+u_x,y+u_y,R,G,B,t$) & (0.11,0.11,0.14,0.14,0.14,0.03) & (0.08,0.08,0.12,0.12,0.12,0.01) & 0.65 & 3 & Logistic\\
      \textbf{with CNN2~\cite{richter2016playing}-Flow} & ($x+u_x,y+u_y,R,G,B,t$) & (0.08,0.08,0.2,0.2,0.2,0.04) & (0.09,0.09,0.25,0.25,0.25,0.03) & 0.5 & 4 & Logistic\\
      \midrule
      \textbf{Video Color Propagation} & ($x,y,I,t$)  & (0.04,0.04,0.2,0.04) & No second kernel & 1 & 4 & MSE\\
      \bottomrule
      \\
    \end{tabular}
    \mycaption{Experiment Protocols} {Experiment protocols for the different experiments presented in this work. \textbf{Feature Types}:
    Feature spaces used for the bilateral convolutions, with position ($x,y$) and color
    ($R,G,B$ or $Y,Cb,Cr$) features $\in [0,255]$. $u_x$, $u_y$ denotes optical flow with respect
    to the present frame and $I$ denotes grayscale intensity.
    \textbf{Feature Scales ($\Lambda_a, \Lambda_b$)}: Cross-validated scales for the features used.
    \textbf{$\alpha$}: Exponential time decay for the input frames.
    \textbf{Input Frames}: Number of input frames for VPN.
    \textbf{Loss Type}: Type
     of loss used for back-propagation. ``MSE'' corresponds to Euclidean mean squared error loss and ``Logistic'' corresponds to multinomial logistic loss.}
  \label{tbl:parameters_supp}
\end{table*}

% \begin{figure}[th!]
% \begin{center}
%   \centerline{\includegraphics[width=\textwidth]{figures/video_seg_visuals_supp_small.pdf}}
%     \mycaption{Video Object Segmentation}
%     {Shown are the different frames in example videos with the corresponding
%     ground truth (GT) masks, predictions from BVS~\cite{marki2016bilateral},
%     OFL~\cite{tsaivideo}, VPN (VPN-Stage2) and VPN-DLab (VPN-DeepLab) models.}
%     \label{fig:video_seg_small_supp}
% \end{center}
% \vspace{-1.0cm}
% \end{figure}

\begin{figure}[th!]
\begin{center}
  \centerline{\includegraphics[width=0.7\textwidth]{figures/video_seg_visuals_supp_positive.pdf}}
    \mycaption{Video Object Segmentation}
    {Shown are the different frames in example videos with the corresponding
    ground truth (GT) masks, predictions from BVS~\cite{marki2016bilateral},
    OFL~\cite{tsaivideo}, VPN (VPN-Stage2) and VPN-DLab (VPN-DeepLab) models.}
    \label{fig:video_seg_pos_supp}
\end{center}
\vspace{-1.0cm}
\end{figure}

\begin{figure}[th!]
\begin{center}
  \centerline{\includegraphics[width=0.7\textwidth]{figures/video_seg_visuals_supp_negative.pdf}}
    \mycaption{Failure Cases for Video Object Segmentation}
    {Shown are the different frames in example videos with the corresponding
    ground truth (GT) masks, predictions from BVS~\cite{marki2016bilateral},
    OFL~\cite{tsaivideo}, VPN (VPN-Stage2) and VPN-DLab (VPN-DeepLab) models.}
    \label{fig:video_seg_neg_supp}
\end{center}
\vspace{-1.0cm}
\end{figure}

\begin{figure}[th!]
\begin{center}
  \centerline{\includegraphics[width=0.9\textwidth]{figures/supp_semantic_visual.pdf}}
    \mycaption{Semantic Video Segmentation}
    {Input video frames and the corresponding ground truth (GT)
    segmentation together with the predictions of CNN~\cite{yu2015multi} and with
    VPN-Flow.}
    \label{fig:semantic_visuals_supp}
\end{center}
\vspace{-0.7cm}
\end{figure}

\begin{figure}[th!]
\begin{center}
  \centerline{\includegraphics[width=\textwidth]{figures/colorization_visuals_supp.pdf}}
  \mycaption{Video Color Propagation}
  {Input grayscale video frames and corresponding ground-truth (GT) color images
  together with color predictions of Levin et al.~\cite{levin2004colorization} and VPN-Stage1 models.}
  \label{fig:color_visuals_supp}
\end{center}
\vspace{-0.7cm}
\end{figure}

\clearpage

\section{Additional Material for Bilateral Inception Networks}
\label{sec:binception-app}

In this section of the Appendix, we first discuss the use of approximate bilateral
filtering in BI modules (Sec.~\ref{sec:lattice}).
Later, we present some qualitative results using different models for the approach presented in
Chapter~\ref{chap:binception} (Sec.~\ref{sec:qualitative-app}).

\subsection{Approximate Bilateral Filtering}
\label{sec:lattice}

The bilateral inception module presented in Chapter~\ref{chap:binception} computes a matrix-vector
product between a Gaussian filter $K$ and a vector of activations $\bz_c$.
Bilateral filtering is an important operation and many algorithmic techniques have been
proposed to speed-up this operation~\cite{paris2006fast,adams2010fast,gastal2011domain}.
In the main paper we opted to implement what can be considered the
brute-force variant of explicitly constructing $K$ and then using BLAS to compute the
matrix-vector product. This resulted in a few millisecond operation.
The explicit way to compute is possible due to the
reduction to super-pixels, e.g., it would not work for DenseCRF variants
that operate on the full image resolution.

Here, we present experiments where we use the fast approximate bilateral filtering
algorithm of~\cite{adams2010fast}, which is also used in Chapter~\ref{chap:bnn}
for learning sparse high dimensional filters. This
choice allows for larger dimensions of matrix-vector multiplication. The reason for choosing
the explicit multiplication in Chapter~\ref{chap:binception} was that it was computationally faster.
For the small sizes of the involved matrices and vectors, the explicit computation is sufficient and we had no
GPU implementation of an approximate technique that matched this runtime. Also it
is conceptually easier and the gradient to the feature transformations ($\Lambda \mathbf{f}$) is
obtained using standard matrix calculus.

\subsubsection{Experiments}

We modified the existing segmentation architectures analogous to those in Chapter~\ref{chap:binception}.
The main difference is that, here, the inception modules use the lattice
approximation~\cite{adams2010fast} to compute the bilateral filtering.
Using the lattice approximation did not allow us to back-propagate through feature transformations ($\Lambda$)
and thus we used hand-specified feature scales as will be explained later.
Specifically, we take CNN architectures from the works
of~\cite{chen2014semantic,zheng2015conditional,bell2015minc} and insert the BI modules between
the spatial FC layers.
We use superpixels from~\cite{DollarICCV13edges}
for all the experiments with the lattice approximation. Experiments are
performed using Caffe neural network framework~\cite{jia2014caffe}.

\begin{table}
  \small
  \centering
  \begin{tabular}{p{5.5cm}>{\raggedright\arraybackslash}p{1.4cm}>{\centering\arraybackslash}p{2.2cm}}
    \toprule
		\textbf{Model} & \emph{IoU} & \emph{Runtime}(ms) \\
    \midrule

    %%%%%%%%%%%% Scores computed by us)%%%%%%%%%%%%
		\deeplablargefov & 68.9 & 145ms\\
    \midrule
    \bi{7}{2}-\bi{8}{10}& \textbf{73.8} & +600 \\
    \midrule
    \deeplablargefovcrf~\cite{chen2014semantic} & 72.7 & +830\\
    \deeplabmsclargefovcrf~\cite{chen2014semantic} & \textbf{73.6} & +880\\
    DeepLab-EdgeNet~\cite{chen2015semantic} & 71.7 & +30\\
    DeepLab-EdgeNet-CRF~\cite{chen2015semantic} & \textbf{73.6} & +860\\
  \bottomrule \\
  \end{tabular}
  \mycaption{Semantic Segmentation using the DeepLab model}
  {IoU scores on the Pascal VOC12 segmentation test dataset
  with different models and our modified inception model.
  Also shown are the corresponding runtimes in milliseconds. Runtimes
  also include superpixel computations (300 ms with Dollar superpixels~\cite{DollarICCV13edges})}
  \label{tab:largefovresults}
\end{table}

\paragraph{Semantic Segmentation}
The experiments in this section use the Pascal VOC12 segmentation dataset~\cite{voc2012segmentation} with 21 object classes and the images have a maximum resolution of 0.25 megapixels.
For all experiments on VOC12, we train using the extended training set of
10581 images collected by~\cite{hariharan2011moredata}.
We modified the \deeplab~network architecture of~\cite{chen2014semantic} and
the CRFasRNN architecture from~\cite{zheng2015conditional} which uses a CNN with
deconvolution layers followed by DenseCRF trained end-to-end.

\paragraph{DeepLab Model}\label{sec:deeplabmodel}
We experimented with the \bi{7}{2}-\bi{8}{10} inception model.
Results using the~\deeplab~model are summarized in Tab.~\ref{tab:largefovresults}.
Although we get similar improvements with inception modules as with the
explicit kernel computation, using lattice approximation is slower.

\begin{table}
  \small
  \centering
  \begin{tabular}{p{6.4cm}>{\raggedright\arraybackslash}p{1.8cm}>{\raggedright\arraybackslash}p{1.8cm}}
    \toprule
    \textbf{Model} & \emph{IoU (Val)} & \emph{IoU (Test)}\\
    \midrule
    %%%%%%%%%%%% Scores computed by us)%%%%%%%%%%%%
    CNN &  67.5 & - \\
    \deconv (CNN+Deconvolutions) & 69.8 & 72.0 \\
    \midrule
    \bi{3}{6}-\bi{4}{6}-\bi{7}{2}-\bi{8}{6}& 71.9 & - \\
    \bi{3}{6}-\bi{4}{6}-\bi{7}{2}-\bi{8}{6}-\gi{6}& 73.6 &  \href{http://host.robots.ox.ac.uk:8080/anonymous/VOTV5E.html}{\textbf{75.2}}\\
    \midrule
    \deconvcrf (CRF-RNN)~\cite{zheng2015conditional} & 73.0 & 74.7\\
    Context-CRF-RNN~\cite{yu2015multi} & ~~ - ~ & \textbf{75.3} \\
    \bottomrule \\
  \end{tabular}
  \mycaption{Semantic Segmentation using the CRFasRNN model}{IoU score corresponding to different models
  on Pascal VOC12 reduced validation / test segmentation dataset. The reduced validation set consists of 346 images
  as used in~\cite{zheng2015conditional} where we adapted the model from.}
  \label{tab:deconvresults-app}
\end{table}

\paragraph{CRFasRNN Model}\label{sec:deepinception}
We add BI modules after score-pool3, score-pool4, \fc{7} and \fc{8} $1\times1$ convolution layers
resulting in the \bi{3}{6}-\bi{4}{6}-\bi{7}{2}-\bi{8}{6}
model and also experimented with another variant where $BI_8$ is followed by another inception
module, G$(6)$, with 6 Gaussian kernels.
Note that here also we discarded both deconvolution and DenseCRF parts of the original model~\cite{zheng2015conditional}
and inserted the BI modules in the base CNN and found similar improvements compared to the inception modules with explicit
kernel computaion. See Tab.~\ref{tab:deconvresults-app} for results on the CRFasRNN model.

\paragraph{Material Segmentation}
Table~\ref{tab:mincresults-app} shows the results on the MINC dataset~\cite{bell2015minc}
obtained by modifying the AlexNet architecture with our inception modules. We observe
similar improvements as with explicit kernel construction.
For this model, we do not provide any learned setup due to very limited segment training
data. The weights to combine outputs in the bilateral inception layer are
found by validation on the validation set.

\begin{table}[t]
  \small
  \centering
  \begin{tabular}{p{3.5cm}>{\centering\arraybackslash}p{4.0cm}}
    \toprule
    \textbf{Model} & Class / Total accuracy\\
    \midrule

    %%%%%%%%%%%% Scores computed by us)%%%%%%%%%%%%
    AlexNet CNN & 55.3 / 58.9 \\
    \midrule
    \bi{7}{2}-\bi{8}{6}& 68.5 / 71.8 \\
    \bi{7}{2}-\bi{8}{6}-G$(6)$& 67.6 / 73.1 \\
    \midrule
    AlexNet-CRF & 65.5 / 71.0 \\
    \bottomrule \\
  \end{tabular}
  \mycaption{Material Segmentation using AlexNet}{Pixel accuracy of different models on
  the MINC material segmentation test dataset~\cite{bell2015minc}.}
  \label{tab:mincresults-app}
\end{table}

\paragraph{Scales of Bilateral Inception Modules}
\label{sec:scales}

Unlike the explicit kernel technique presented in the main text (Chapter~\ref{chap:binception}),
we didn't back-propagate through feature transformation ($\Lambda$)
using the approximate bilateral filter technique.
So, the feature scales are hand-specified and validated, which are as follows.
The optimal scale values for the \bi{7}{2}-\bi{8}{2} model are found by validation for the best performance which are
$\sigma_{xy}$ = (0.1, 0.1) for the spatial (XY) kernel and $\sigma_{rgbxy}$ = (0.1, 0.1, 0.1, 0.01, 0.01) for color and position (RGBXY)  kernel.
Next, as more kernels are added to \bi{8}{2}, we set scales to be $\alpha$*($\sigma_{xy}$, $\sigma_{rgbxy}$).
The value of $\alpha$ is chosen as  1, 0.5, 0.1, 0.05, 0.1, at uniform interval, for the \bi{8}{10} bilateral inception module.


\subsection{Qualitative Results}
\label{sec:qualitative-app}

In this section, we present more qualitative results obtained using the BI module with explicit
kernel computation technique presented in Chapter~\ref{chap:binception}. Results on the Pascal VOC12
dataset~\cite{voc2012segmentation} using the DeepLab-LargeFOV model are shown in Fig.~\ref{fig:semantic_visuals-app},
followed by the results on MINC dataset~\cite{bell2015minc}
in Fig.~\ref{fig:material_visuals-app} and on
Cityscapes dataset~\cite{Cordts2015Cvprw} in Fig.~\ref{fig:street_visuals-app}.


\definecolor{voc_1}{RGB}{0, 0, 0}
\definecolor{voc_2}{RGB}{128, 0, 0}
\definecolor{voc_3}{RGB}{0, 128, 0}
\definecolor{voc_4}{RGB}{128, 128, 0}
\definecolor{voc_5}{RGB}{0, 0, 128}
\definecolor{voc_6}{RGB}{128, 0, 128}
\definecolor{voc_7}{RGB}{0, 128, 128}
\definecolor{voc_8}{RGB}{128, 128, 128}
\definecolor{voc_9}{RGB}{64, 0, 0}
\definecolor{voc_10}{RGB}{192, 0, 0}
\definecolor{voc_11}{RGB}{64, 128, 0}
\definecolor{voc_12}{RGB}{192, 128, 0}
\definecolor{voc_13}{RGB}{64, 0, 128}
\definecolor{voc_14}{RGB}{192, 0, 128}
\definecolor{voc_15}{RGB}{64, 128, 128}
\definecolor{voc_16}{RGB}{192, 128, 128}
\definecolor{voc_17}{RGB}{0, 64, 0}
\definecolor{voc_18}{RGB}{128, 64, 0}
\definecolor{voc_19}{RGB}{0, 192, 0}
\definecolor{voc_20}{RGB}{128, 192, 0}
\definecolor{voc_21}{RGB}{0, 64, 128}
\definecolor{voc_22}{RGB}{128, 64, 128}

\begin{figure*}[!ht]
  \small
  \centering
  \fcolorbox{white}{voc_1}{\rule{0pt}{4pt}\rule{4pt}{0pt}} Background~~
  \fcolorbox{white}{voc_2}{\rule{0pt}{4pt}\rule{4pt}{0pt}} Aeroplane~~
  \fcolorbox{white}{voc_3}{\rule{0pt}{4pt}\rule{4pt}{0pt}} Bicycle~~
  \fcolorbox{white}{voc_4}{\rule{0pt}{4pt}\rule{4pt}{0pt}} Bird~~
  \fcolorbox{white}{voc_5}{\rule{0pt}{4pt}\rule{4pt}{0pt}} Boat~~
  \fcolorbox{white}{voc_6}{\rule{0pt}{4pt}\rule{4pt}{0pt}} Bottle~~
  \fcolorbox{white}{voc_7}{\rule{0pt}{4pt}\rule{4pt}{0pt}} Bus~~
  \fcolorbox{white}{voc_8}{\rule{0pt}{4pt}\rule{4pt}{0pt}} Car~~\\
  \fcolorbox{white}{voc_9}{\rule{0pt}{4pt}\rule{4pt}{0pt}} Cat~~
  \fcolorbox{white}{voc_10}{\rule{0pt}{4pt}\rule{4pt}{0pt}} Chair~~
  \fcolorbox{white}{voc_11}{\rule{0pt}{4pt}\rule{4pt}{0pt}} Cow~~
  \fcolorbox{white}{voc_12}{\rule{0pt}{4pt}\rule{4pt}{0pt}} Dining Table~~
  \fcolorbox{white}{voc_13}{\rule{0pt}{4pt}\rule{4pt}{0pt}} Dog~~
  \fcolorbox{white}{voc_14}{\rule{0pt}{4pt}\rule{4pt}{0pt}} Horse~~
  \fcolorbox{white}{voc_15}{\rule{0pt}{4pt}\rule{4pt}{0pt}} Motorbike~~
  \fcolorbox{white}{voc_16}{\rule{0pt}{4pt}\rule{4pt}{0pt}} Person~~\\
  \fcolorbox{white}{voc_17}{\rule{0pt}{4pt}\rule{4pt}{0pt}} Potted Plant~~
  \fcolorbox{white}{voc_18}{\rule{0pt}{4pt}\rule{4pt}{0pt}} Sheep~~
  \fcolorbox{white}{voc_19}{\rule{0pt}{4pt}\rule{4pt}{0pt}} Sofa~~
  \fcolorbox{white}{voc_20}{\rule{0pt}{4pt}\rule{4pt}{0pt}} Train~~
  \fcolorbox{white}{voc_21}{\rule{0pt}{4pt}\rule{4pt}{0pt}} TV monitor~~\\


  \subfigure{%
    \includegraphics[width=.15\columnwidth]{figures/supplementary/2008_001308_given.png}
  }
  \subfigure{%
    \includegraphics[width=.15\columnwidth]{figures/supplementary/2008_001308_sp.png}
  }
  \subfigure{%
    \includegraphics[width=.15\columnwidth]{figures/supplementary/2008_001308_gt.png}
  }
  \subfigure{%
    \includegraphics[width=.15\columnwidth]{figures/supplementary/2008_001308_cnn.png}
  }
  \subfigure{%
    \includegraphics[width=.15\columnwidth]{figures/supplementary/2008_001308_crf.png}
  }
  \subfigure{%
    \includegraphics[width=.15\columnwidth]{figures/supplementary/2008_001308_ours.png}
  }\\[-2ex]


  \subfigure{%
    \includegraphics[width=.15\columnwidth]{figures/supplementary/2008_001821_given.png}
  }
  \subfigure{%
    \includegraphics[width=.15\columnwidth]{figures/supplementary/2008_001821_sp.png}
  }
  \subfigure{%
    \includegraphics[width=.15\columnwidth]{figures/supplementary/2008_001821_gt.png}
  }
  \subfigure{%
    \includegraphics[width=.15\columnwidth]{figures/supplementary/2008_001821_cnn.png}
  }
  \subfigure{%
    \includegraphics[width=.15\columnwidth]{figures/supplementary/2008_001821_crf.png}
  }
  \subfigure{%
    \includegraphics[width=.15\columnwidth]{figures/supplementary/2008_001821_ours.png}
  }\\[-2ex]



  \subfigure{%
    \includegraphics[width=.15\columnwidth]{figures/supplementary/2008_004612_given.png}
  }
  \subfigure{%
    \includegraphics[width=.15\columnwidth]{figures/supplementary/2008_004612_sp.png}
  }
  \subfigure{%
    \includegraphics[width=.15\columnwidth]{figures/supplementary/2008_004612_gt.png}
  }
  \subfigure{%
    \includegraphics[width=.15\columnwidth]{figures/supplementary/2008_004612_cnn.png}
  }
  \subfigure{%
    \includegraphics[width=.15\columnwidth]{figures/supplementary/2008_004612_crf.png}
  }
  \subfigure{%
    \includegraphics[width=.15\columnwidth]{figures/supplementary/2008_004612_ours.png}
  }\\[-2ex]


  \subfigure{%
    \includegraphics[width=.15\columnwidth]{figures/supplementary/2009_001008_given.png}
  }
  \subfigure{%
    \includegraphics[width=.15\columnwidth]{figures/supplementary/2009_001008_sp.png}
  }
  \subfigure{%
    \includegraphics[width=.15\columnwidth]{figures/supplementary/2009_001008_gt.png}
  }
  \subfigure{%
    \includegraphics[width=.15\columnwidth]{figures/supplementary/2009_001008_cnn.png}
  }
  \subfigure{%
    \includegraphics[width=.15\columnwidth]{figures/supplementary/2009_001008_crf.png}
  }
  \subfigure{%
    \includegraphics[width=.15\columnwidth]{figures/supplementary/2009_001008_ours.png}
  }\\[-2ex]




  \subfigure{%
    \includegraphics[width=.15\columnwidth]{figures/supplementary/2009_004497_given.png}
  }
  \subfigure{%
    \includegraphics[width=.15\columnwidth]{figures/supplementary/2009_004497_sp.png}
  }
  \subfigure{%
    \includegraphics[width=.15\columnwidth]{figures/supplementary/2009_004497_gt.png}
  }
  \subfigure{%
    \includegraphics[width=.15\columnwidth]{figures/supplementary/2009_004497_cnn.png}
  }
  \subfigure{%
    \includegraphics[width=.15\columnwidth]{figures/supplementary/2009_004497_crf.png}
  }
  \subfigure{%
    \includegraphics[width=.15\columnwidth]{figures/supplementary/2009_004497_ours.png}
  }\\[-2ex]



  \setcounter{subfigure}{0}
  \subfigure[\scriptsize Input]{%
    \includegraphics[width=.15\columnwidth]{figures/supplementary/2010_001327_given.png}
  }
  \subfigure[\scriptsize Superpixels]{%
    \includegraphics[width=.15\columnwidth]{figures/supplementary/2010_001327_sp.png}
  }
  \subfigure[\scriptsize GT]{%
    \includegraphics[width=.15\columnwidth]{figures/supplementary/2010_001327_gt.png}
  }
  \subfigure[\scriptsize Deeplab]{%
    \includegraphics[width=.15\columnwidth]{figures/supplementary/2010_001327_cnn.png}
  }
  \subfigure[\scriptsize +DenseCRF]{%
    \includegraphics[width=.15\columnwidth]{figures/supplementary/2010_001327_crf.png}
  }
  \subfigure[\scriptsize Using BI]{%
    \includegraphics[width=.15\columnwidth]{figures/supplementary/2010_001327_ours.png}
  }
  \mycaption{Semantic Segmentation}{Example results of semantic segmentation
  on the Pascal VOC12 dataset.
  (d)~depicts the DeepLab CNN result, (e)~CNN + 10 steps of mean-field inference,
  (f~result obtained with bilateral inception (BI) modules (\bi{6}{2}+\bi{7}{6}) between \fc~layers.}
  \label{fig:semantic_visuals-app}
\end{figure*}


\definecolor{minc_1}{HTML}{771111}
\definecolor{minc_2}{HTML}{CAC690}
\definecolor{minc_3}{HTML}{EEEEEE}
\definecolor{minc_4}{HTML}{7C8FA6}
\definecolor{minc_5}{HTML}{597D31}
\definecolor{minc_6}{HTML}{104410}
\definecolor{minc_7}{HTML}{BB819C}
\definecolor{minc_8}{HTML}{D0CE48}
\definecolor{minc_9}{HTML}{622745}
\definecolor{minc_10}{HTML}{666666}
\definecolor{minc_11}{HTML}{D54A31}
\definecolor{minc_12}{HTML}{101044}
\definecolor{minc_13}{HTML}{444126}
\definecolor{minc_14}{HTML}{75D646}
\definecolor{minc_15}{HTML}{DD4348}
\definecolor{minc_16}{HTML}{5C8577}
\definecolor{minc_17}{HTML}{C78472}
\definecolor{minc_18}{HTML}{75D6D0}
\definecolor{minc_19}{HTML}{5B4586}
\definecolor{minc_20}{HTML}{C04393}
\definecolor{minc_21}{HTML}{D69948}
\definecolor{minc_22}{HTML}{7370D8}
\definecolor{minc_23}{HTML}{7A3622}
\definecolor{minc_24}{HTML}{000000}

\begin{figure*}[!ht]
  \small % scriptsize
  \centering
  \fcolorbox{white}{minc_1}{\rule{0pt}{4pt}\rule{4pt}{0pt}} Brick~~
  \fcolorbox{white}{minc_2}{\rule{0pt}{4pt}\rule{4pt}{0pt}} Carpet~~
  \fcolorbox{white}{minc_3}{\rule{0pt}{4pt}\rule{4pt}{0pt}} Ceramic~~
  \fcolorbox{white}{minc_4}{\rule{0pt}{4pt}\rule{4pt}{0pt}} Fabric~~
  \fcolorbox{white}{minc_5}{\rule{0pt}{4pt}\rule{4pt}{0pt}} Foliage~~
  \fcolorbox{white}{minc_6}{\rule{0pt}{4pt}\rule{4pt}{0pt}} Food~~
  \fcolorbox{white}{minc_7}{\rule{0pt}{4pt}\rule{4pt}{0pt}} Glass~~
  \fcolorbox{white}{minc_8}{\rule{0pt}{4pt}\rule{4pt}{0pt}} Hair~~\\
  \fcolorbox{white}{minc_9}{\rule{0pt}{4pt}\rule{4pt}{0pt}} Leather~~
  \fcolorbox{white}{minc_10}{\rule{0pt}{4pt}\rule{4pt}{0pt}} Metal~~
  \fcolorbox{white}{minc_11}{\rule{0pt}{4pt}\rule{4pt}{0pt}} Mirror~~
  \fcolorbox{white}{minc_12}{\rule{0pt}{4pt}\rule{4pt}{0pt}} Other~~
  \fcolorbox{white}{minc_13}{\rule{0pt}{4pt}\rule{4pt}{0pt}} Painted~~
  \fcolorbox{white}{minc_14}{\rule{0pt}{4pt}\rule{4pt}{0pt}} Paper~~
  \fcolorbox{white}{minc_15}{\rule{0pt}{4pt}\rule{4pt}{0pt}} Plastic~~\\
  \fcolorbox{white}{minc_16}{\rule{0pt}{4pt}\rule{4pt}{0pt}} Polished Stone~~
  \fcolorbox{white}{minc_17}{\rule{0pt}{4pt}\rule{4pt}{0pt}} Skin~~
  \fcolorbox{white}{minc_18}{\rule{0pt}{4pt}\rule{4pt}{0pt}} Sky~~
  \fcolorbox{white}{minc_19}{\rule{0pt}{4pt}\rule{4pt}{0pt}} Stone~~
  \fcolorbox{white}{minc_20}{\rule{0pt}{4pt}\rule{4pt}{0pt}} Tile~~
  \fcolorbox{white}{minc_21}{\rule{0pt}{4pt}\rule{4pt}{0pt}} Wallpaper~~
  \fcolorbox{white}{minc_22}{\rule{0pt}{4pt}\rule{4pt}{0pt}} Water~~
  \fcolorbox{white}{minc_23}{\rule{0pt}{4pt}\rule{4pt}{0pt}} Wood~~\\
  \subfigure{%
    \includegraphics[width=.15\columnwidth]{figures/supplementary/000008468_given.png}
  }
  \subfigure{%
    \includegraphics[width=.15\columnwidth]{figures/supplementary/000008468_sp.png}
  }
  \subfigure{%
    \includegraphics[width=.15\columnwidth]{figures/supplementary/000008468_gt.png}
  }
  \subfigure{%
    \includegraphics[width=.15\columnwidth]{figures/supplementary/000008468_cnn.png}
  }
  \subfigure{%
    \includegraphics[width=.15\columnwidth]{figures/supplementary/000008468_crf.png}
  }
  \subfigure{%
    \includegraphics[width=.15\columnwidth]{figures/supplementary/000008468_ours.png}
  }\\[-2ex]

  \subfigure{%
    \includegraphics[width=.15\columnwidth]{figures/supplementary/000009053_given.png}
  }
  \subfigure{%
    \includegraphics[width=.15\columnwidth]{figures/supplementary/000009053_sp.png}
  }
  \subfigure{%
    \includegraphics[width=.15\columnwidth]{figures/supplementary/000009053_gt.png}
  }
  \subfigure{%
    \includegraphics[width=.15\columnwidth]{figures/supplementary/000009053_cnn.png}
  }
  \subfigure{%
    \includegraphics[width=.15\columnwidth]{figures/supplementary/000009053_crf.png}
  }
  \subfigure{%
    \includegraphics[width=.15\columnwidth]{figures/supplementary/000009053_ours.png}
  }\\[-2ex]




  \subfigure{%
    \includegraphics[width=.15\columnwidth]{figures/supplementary/000014977_given.png}
  }
  \subfigure{%
    \includegraphics[width=.15\columnwidth]{figures/supplementary/000014977_sp.png}
  }
  \subfigure{%
    \includegraphics[width=.15\columnwidth]{figures/supplementary/000014977_gt.png}
  }
  \subfigure{%
    \includegraphics[width=.15\columnwidth]{figures/supplementary/000014977_cnn.png}
  }
  \subfigure{%
    \includegraphics[width=.15\columnwidth]{figures/supplementary/000014977_crf.png}
  }
  \subfigure{%
    \includegraphics[width=.15\columnwidth]{figures/supplementary/000014977_ours.png}
  }\\[-2ex]


  \subfigure{%
    \includegraphics[width=.15\columnwidth]{figures/supplementary/000022922_given.png}
  }
  \subfigure{%
    \includegraphics[width=.15\columnwidth]{figures/supplementary/000022922_sp.png}
  }
  \subfigure{%
    \includegraphics[width=.15\columnwidth]{figures/supplementary/000022922_gt.png}
  }
  \subfigure{%
    \includegraphics[width=.15\columnwidth]{figures/supplementary/000022922_cnn.png}
  }
  \subfigure{%
    \includegraphics[width=.15\columnwidth]{figures/supplementary/000022922_crf.png}
  }
  \subfigure{%
    \includegraphics[width=.15\columnwidth]{figures/supplementary/000022922_ours.png}
  }\\[-2ex]


  \subfigure{%
    \includegraphics[width=.15\columnwidth]{figures/supplementary/000025711_given.png}
  }
  \subfigure{%
    \includegraphics[width=.15\columnwidth]{figures/supplementary/000025711_sp.png}
  }
  \subfigure{%
    \includegraphics[width=.15\columnwidth]{figures/supplementary/000025711_gt.png}
  }
  \subfigure{%
    \includegraphics[width=.15\columnwidth]{figures/supplementary/000025711_cnn.png}
  }
  \subfigure{%
    \includegraphics[width=.15\columnwidth]{figures/supplementary/000025711_crf.png}
  }
  \subfigure{%
    \includegraphics[width=.15\columnwidth]{figures/supplementary/000025711_ours.png}
  }\\[-2ex]


  \subfigure{%
    \includegraphics[width=.15\columnwidth]{figures/supplementary/000034473_given.png}
  }
  \subfigure{%
    \includegraphics[width=.15\columnwidth]{figures/supplementary/000034473_sp.png}
  }
  \subfigure{%
    \includegraphics[width=.15\columnwidth]{figures/supplementary/000034473_gt.png}
  }
  \subfigure{%
    \includegraphics[width=.15\columnwidth]{figures/supplementary/000034473_cnn.png}
  }
  \subfigure{%
    \includegraphics[width=.15\columnwidth]{figures/supplementary/000034473_crf.png}
  }
  \subfigure{%
    \includegraphics[width=.15\columnwidth]{figures/supplementary/000034473_ours.png}
  }\\[-2ex]


  \subfigure{%
    \includegraphics[width=.15\columnwidth]{figures/supplementary/000035463_given.png}
  }
  \subfigure{%
    \includegraphics[width=.15\columnwidth]{figures/supplementary/000035463_sp.png}
  }
  \subfigure{%
    \includegraphics[width=.15\columnwidth]{figures/supplementary/000035463_gt.png}
  }
  \subfigure{%
    \includegraphics[width=.15\columnwidth]{figures/supplementary/000035463_cnn.png}
  }
  \subfigure{%
    \includegraphics[width=.15\columnwidth]{figures/supplementary/000035463_crf.png}
  }
  \subfigure{%
    \includegraphics[width=.15\columnwidth]{figures/supplementary/000035463_ours.png}
  }\\[-2ex]


  \setcounter{subfigure}{0}
  \subfigure[\scriptsize Input]{%
    \includegraphics[width=.15\columnwidth]{figures/supplementary/000035993_given.png}
  }
  \subfigure[\scriptsize Superpixels]{%
    \includegraphics[width=.15\columnwidth]{figures/supplementary/000035993_sp.png}
  }
  \subfigure[\scriptsize GT]{%
    \includegraphics[width=.15\columnwidth]{figures/supplementary/000035993_gt.png}
  }
  \subfigure[\scriptsize AlexNet]{%
    \includegraphics[width=.15\columnwidth]{figures/supplementary/000035993_cnn.png}
  }
  \subfigure[\scriptsize +DenseCRF]{%
    \includegraphics[width=.15\columnwidth]{figures/supplementary/000035993_crf.png}
  }
  \subfigure[\scriptsize Using BI]{%
    \includegraphics[width=.15\columnwidth]{figures/supplementary/000035993_ours.png}
  }
  \mycaption{Material Segmentation}{Example results of material segmentation.
  (d)~depicts the AlexNet CNN result, (e)~CNN + 10 steps of mean-field inference,
  (f)~result obtained with bilateral inception (BI) modules (\bi{7}{2}+\bi{8}{6}) between
  \fc~layers.}
\label{fig:material_visuals-app}
\end{figure*}


\definecolor{city_1}{RGB}{128, 64, 128}
\definecolor{city_2}{RGB}{244, 35, 232}
\definecolor{city_3}{RGB}{70, 70, 70}
\definecolor{city_4}{RGB}{102, 102, 156}
\definecolor{city_5}{RGB}{190, 153, 153}
\definecolor{city_6}{RGB}{153, 153, 153}
\definecolor{city_7}{RGB}{250, 170, 30}
\definecolor{city_8}{RGB}{220, 220, 0}
\definecolor{city_9}{RGB}{107, 142, 35}
\definecolor{city_10}{RGB}{152, 251, 152}
\definecolor{city_11}{RGB}{70, 130, 180}
\definecolor{city_12}{RGB}{220, 20, 60}
\definecolor{city_13}{RGB}{255, 0, 0}
\definecolor{city_14}{RGB}{0, 0, 142}
\definecolor{city_15}{RGB}{0, 0, 70}
\definecolor{city_16}{RGB}{0, 60, 100}
\definecolor{city_17}{RGB}{0, 80, 100}
\definecolor{city_18}{RGB}{0, 0, 230}
\definecolor{city_19}{RGB}{119, 11, 32}
\begin{figure*}[!ht]
  \small % scriptsize
  \centering


  \subfigure{%
    \includegraphics[width=.18\columnwidth]{figures/supplementary/frankfurt00000_016005_given.png}
  }
  \subfigure{%
    \includegraphics[width=.18\columnwidth]{figures/supplementary/frankfurt00000_016005_sp.png}
  }
  \subfigure{%
    \includegraphics[width=.18\columnwidth]{figures/supplementary/frankfurt00000_016005_gt.png}
  }
  \subfigure{%
    \includegraphics[width=.18\columnwidth]{figures/supplementary/frankfurt00000_016005_cnn.png}
  }
  \subfigure{%
    \includegraphics[width=.18\columnwidth]{figures/supplementary/frankfurt00000_016005_ours.png}
  }\\[-2ex]

  \subfigure{%
    \includegraphics[width=.18\columnwidth]{figures/supplementary/frankfurt00000_004617_given.png}
  }
  \subfigure{%
    \includegraphics[width=.18\columnwidth]{figures/supplementary/frankfurt00000_004617_sp.png}
  }
  \subfigure{%
    \includegraphics[width=.18\columnwidth]{figures/supplementary/frankfurt00000_004617_gt.png}
  }
  \subfigure{%
    \includegraphics[width=.18\columnwidth]{figures/supplementary/frankfurt00000_004617_cnn.png}
  }
  \subfigure{%
    \includegraphics[width=.18\columnwidth]{figures/supplementary/frankfurt00000_004617_ours.png}
  }\\[-2ex]

  \subfigure{%
    \includegraphics[width=.18\columnwidth]{figures/supplementary/frankfurt00000_020880_given.png}
  }
  \subfigure{%
    \includegraphics[width=.18\columnwidth]{figures/supplementary/frankfurt00000_020880_sp.png}
  }
  \subfigure{%
    \includegraphics[width=.18\columnwidth]{figures/supplementary/frankfurt00000_020880_gt.png}
  }
  \subfigure{%
    \includegraphics[width=.18\columnwidth]{figures/supplementary/frankfurt00000_020880_cnn.png}
  }
  \subfigure{%
    \includegraphics[width=.18\columnwidth]{figures/supplementary/frankfurt00000_020880_ours.png}
  }\\[-2ex]



  \subfigure{%
    \includegraphics[width=.18\columnwidth]{figures/supplementary/frankfurt00001_007285_given.png}
  }
  \subfigure{%
    \includegraphics[width=.18\columnwidth]{figures/supplementary/frankfurt00001_007285_sp.png}
  }
  \subfigure{%
    \includegraphics[width=.18\columnwidth]{figures/supplementary/frankfurt00001_007285_gt.png}
  }
  \subfigure{%
    \includegraphics[width=.18\columnwidth]{figures/supplementary/frankfurt00001_007285_cnn.png}
  }
  \subfigure{%
    \includegraphics[width=.18\columnwidth]{figures/supplementary/frankfurt00001_007285_ours.png}
  }\\[-2ex]


  \subfigure{%
    \includegraphics[width=.18\columnwidth]{figures/supplementary/frankfurt00001_059789_given.png}
  }
  \subfigure{%
    \includegraphics[width=.18\columnwidth]{figures/supplementary/frankfurt00001_059789_sp.png}
  }
  \subfigure{%
    \includegraphics[width=.18\columnwidth]{figures/supplementary/frankfurt00001_059789_gt.png}
  }
  \subfigure{%
    \includegraphics[width=.18\columnwidth]{figures/supplementary/frankfurt00001_059789_cnn.png}
  }
  \subfigure{%
    \includegraphics[width=.18\columnwidth]{figures/supplementary/frankfurt00001_059789_ours.png}
  }\\[-2ex]


  \subfigure{%
    \includegraphics[width=.18\columnwidth]{figures/supplementary/frankfurt00001_068208_given.png}
  }
  \subfigure{%
    \includegraphics[width=.18\columnwidth]{figures/supplementary/frankfurt00001_068208_sp.png}
  }
  \subfigure{%
    \includegraphics[width=.18\columnwidth]{figures/supplementary/frankfurt00001_068208_gt.png}
  }
  \subfigure{%
    \includegraphics[width=.18\columnwidth]{figures/supplementary/frankfurt00001_068208_cnn.png}
  }
  \subfigure{%
    \includegraphics[width=.18\columnwidth]{figures/supplementary/frankfurt00001_068208_ours.png}
  }\\[-2ex]

  \subfigure{%
    \includegraphics[width=.18\columnwidth]{figures/supplementary/frankfurt00001_082466_given.png}
  }
  \subfigure{%
    \includegraphics[width=.18\columnwidth]{figures/supplementary/frankfurt00001_082466_sp.png}
  }
  \subfigure{%
    \includegraphics[width=.18\columnwidth]{figures/supplementary/frankfurt00001_082466_gt.png}
  }
  \subfigure{%
    \includegraphics[width=.18\columnwidth]{figures/supplementary/frankfurt00001_082466_cnn.png}
  }
  \subfigure{%
    \includegraphics[width=.18\columnwidth]{figures/supplementary/frankfurt00001_082466_ours.png}
  }\\[-2ex]

  \subfigure{%
    \includegraphics[width=.18\columnwidth]{figures/supplementary/lindau00033_000019_given.png}
  }
  \subfigure{%
    \includegraphics[width=.18\columnwidth]{figures/supplementary/lindau00033_000019_sp.png}
  }
  \subfigure{%
    \includegraphics[width=.18\columnwidth]{figures/supplementary/lindau00033_000019_gt.png}
  }
  \subfigure{%
    \includegraphics[width=.18\columnwidth]{figures/supplementary/lindau00033_000019_cnn.png}
  }
  \subfigure{%
    \includegraphics[width=.18\columnwidth]{figures/supplementary/lindau00033_000019_ours.png}
  }\\[-2ex]

  \subfigure{%
    \includegraphics[width=.18\columnwidth]{figures/supplementary/lindau00052_000019_given.png}
  }
  \subfigure{%
    \includegraphics[width=.18\columnwidth]{figures/supplementary/lindau00052_000019_sp.png}
  }
  \subfigure{%
    \includegraphics[width=.18\columnwidth]{figures/supplementary/lindau00052_000019_gt.png}
  }
  \subfigure{%
    \includegraphics[width=.18\columnwidth]{figures/supplementary/lindau00052_000019_cnn.png}
  }
  \subfigure{%
    \includegraphics[width=.18\columnwidth]{figures/supplementary/lindau00052_000019_ours.png}
  }\\[-2ex]




  \subfigure{%
    \includegraphics[width=.18\columnwidth]{figures/supplementary/lindau00027_000019_given.png}
  }
  \subfigure{%
    \includegraphics[width=.18\columnwidth]{figures/supplementary/lindau00027_000019_sp.png}
  }
  \subfigure{%
    \includegraphics[width=.18\columnwidth]{figures/supplementary/lindau00027_000019_gt.png}
  }
  \subfigure{%
    \includegraphics[width=.18\columnwidth]{figures/supplementary/lindau00027_000019_cnn.png}
  }
  \subfigure{%
    \includegraphics[width=.18\columnwidth]{figures/supplementary/lindau00027_000019_ours.png}
  }\\[-2ex]



  \setcounter{subfigure}{0}
  \subfigure[\scriptsize Input]{%
    \includegraphics[width=.18\columnwidth]{figures/supplementary/lindau00029_000019_given.png}
  }
  \subfigure[\scriptsize Superpixels]{%
    \includegraphics[width=.18\columnwidth]{figures/supplementary/lindau00029_000019_sp.png}
  }
  \subfigure[\scriptsize GT]{%
    \includegraphics[width=.18\columnwidth]{figures/supplementary/lindau00029_000019_gt.png}
  }
  \subfigure[\scriptsize Deeplab]{%
    \includegraphics[width=.18\columnwidth]{figures/supplementary/lindau00029_000019_cnn.png}
  }
  \subfigure[\scriptsize Using BI]{%
    \includegraphics[width=.18\columnwidth]{figures/supplementary/lindau00029_000019_ours.png}
  }%\\[-2ex]

  \mycaption{Street Scene Segmentation}{Example results of street scene segmentation.
  (d)~depicts the DeepLab results, (e)~result obtained by adding bilateral inception (BI) modules (\bi{6}{2}+\bi{7}{6}) between \fc~layers.}
\label{fig:street_visuals-app}
\end{figure*}


\end{document}
% \grid
% \grid
% \grid
% \grid
% \grid
% \grid
% \grid
% \grid
