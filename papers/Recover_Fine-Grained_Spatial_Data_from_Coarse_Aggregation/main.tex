\documentclass[10pt, conference, letterpaper]{IEEEtran}

\usepackage{multicol}
\usepackage{multirow}
\usepackage{graphicx}
%% Patch to make old LaTeX understand dates in yyyy-mm-dd format
\makeatletter
\@ifundefined{@parse@version@dash}{%
	\def\@parse@version#1{\@parse@version@0#1}
	\def\@parse@version@#1/#2/#3#4#5\@nil{%
		\@parse@version@dash#1-#2-#3#4\@nil}
	\def\@parse@version@dash#1-#2-#3#4#5\@nil{%
		\if\relax#2\relax\else#1\fi#2#3#4 }
}{}
\makeatother

\documentclass[aps,pre,twocolumn,groupedaddress,longbibliography]{revtex4-2}
\usepackage{amsfonts,amssymb,amsmath} %%%add
\usepackage{color} %%%add
\usepackage{graphicx} %%%add
\usepackage{epstopdf,mathrsfs} %%%add
\usepackage[colorlinks,linktocpage,linkcolor=blue]{hyperref}

\newcommand{\red}[1]{\textcolor{red}{#1}}
\newcommand{\blue}[1]{\textcolor{blue}{#1}}
\newcommand{\cyan}[1]{\textcolor{cyan}{#1}}
\newcommand{\yellow}[1]{\textcolor{yellow}{#1}}

%\usepackage{xcolor}
%\pagecolor[rgb]{0.9, 0.99, 0.9}


\begin{document}

\title{Random diffusivity processes in an external force field}
\author{Xudong Wang$^1$}
%\email{xdwang14@njust.edu.cn}
\author{Yao Chen$^2$}
\email{ychen@njau.edu.cn}
\affiliation{$^1$School of Mathematics and Statistics, Nanjing University of Science and Technology, Nanjing, 210094, P.R. China \\
$^2$College of Sciences, Nanjing Agricultural University, Nanjing, 210094, P.R. China}


\begin{abstract}
Brownian yet non-Gaussian processes have recently been observed in numerous biological systems and the corresponding theories have been built based on random diffusivity models. Considering the particularity of random diffusivity, this paper studies the effect of an external force acting on two kinds of random diffusivity models whose difference is embodied in whether the fluctuation-dissipation theorem is valid. Based on the two random diffusivity models, we derive the Fokker-Planck equations with an arbitrary external force, and analyse various observables in the case with a constant force, including the Einstein relation, the moments, the kurtosis, and the asymptotic behaviors of the probability density function of particle's displacement at different time scales.
Both the theoretical results and numerical simulations of these observables show significant difference between the two kinds of random diffusivity models, which implies the important role of the fluctuation-dissipation theorem in random diffusivity systems.

\end{abstract}


\maketitle

\section{Introduction}\label{Sec1}
It is ubiquitous to find that particles diffuse under some kind of external force fields in the natural world. Under the effect of external forces, the motion of particles shows many kinds of anomalous diffusion phenomena in complex systems \cite{BouchaudGeorges:1990,MetzlerKlafter:2000,MagdziarzWeronKlafter:2008,EuleFriedrich:2009,CairoliBaule:2015,FedotovKorabel:2015}.
Particularly, the particles might undergo a biased random walk with a nonzero mean of displacement. The corresponding ensemble-averaged mean-squared displacement (MSD) is defined as
\begin{equation}\label{1}
\langle \Delta x^2(t)\rangle=\langle[x(t)-\langle x(t)\rangle]^2\rangle \,{\propto}\,t^\beta  \quad  (\beta\neq1),
\end{equation}
where normal Brownian motion belongs to $\beta=1$, and anomalous diffusion is characterized by the nonlinear evolution in time with $\beta\neq1$.


In addition to the normal diffusion of Brownian motion, the probability density function (PDF) of its displacement is Gaussian-shaped \cite{VanKampen:1992,CoffeyKalmykovWaldron:2004}
\begin{equation}\label{Gaussian}
  G(x,t|D)=\frac{1}{\sqrt{4\pi Dt}}\exp\left(-\frac{x^2}{4Dt}\right)
\end{equation}
for a given diffusivity $D$. In contrast to the Gaussian-shaped PDF, a new class of normal diffusion process has recently been observed with a non-Gaussian PDF, which is thus named as Brownian yet non-Gaussian process. This phenomenon has been found in a large range of complex systems, including polystyrene beads diffusing on the surface of lipid tubes \cite{WangAnthonyBaeGranick:2009} or in networks \cite{WangAnthonyBaeGranick:2009,ToyotaHeadSchmidtMizuno:2011,SilvaStuhrmannBetzKoenderink:2014}, as well as the diffusion of tracer molecules on polymer thin films \cite{Bhattacharya-etal:2013} and in simulations of two-dimensional discs \cite{KimKimSung:2013}. Instead of the Gaussian shape, the PDF of the Brownian yet non-Gaussian process is characterized by exponential distribution
\begin{equation}\label{Exponential}
  p(x,t)= \frac{1}{2\sqrt{D_0t}}\exp\left(-\frac{|x|}{\sqrt{D_0t}}\right)
\end{equation}
with $D_0$ being the effective diffusivity.

The interesting phenomenon of the non-Gaussian feature can be interpreted by the superstatistical approach of assuming the diffusivity $D$ in Eq. \eqref{Gaussian} being a random variable \cite{Beck:2001,BeckCohen:2003,Beck:2006}.
More precisely, each particle undergoes a normal Brownian motion with its own diffusivity which does not change considerably in a short time. The diffusivity $D$ of each particle obeys the exponential distribution
$\pi(D)=\exp(-D/D_0)/D_0$, and the randomness of diffusivity results from a spatially inhomogeneous environment. Averaging the Gaussian distribution in Eq. \eqref{Gaussian} over the diffusivity with the exponential distribution $\pi(D)$ yields \cite{WangKuoBaeGranick:2012,HapcaCrawfordYoung:2009}
\begin{equation}\label{PDF-x}
\begin{split}
    p(x,t)&=\int_0^\infty \pi(D)G(x,t|D)dD  \\
    &=\frac{1}{\sqrt{4D_0t}}\exp\left(-\frac{|x|}{\sqrt{D_0t}}\right).
\end{split}
\end{equation}
Besides the superstatistical approach, the exponential tail is found to be universal for short-time dynamics of the continuous-time random walk by using large deviation theory  \cite{BarkaiBurov:2020,WangBarkaiBurov:2020}.

Furthermore, the phenomenon observed in experiments also shows that the PDF undergoes a crossover from exponential distribution to Gaussian distribution \cite{WangAnthonyBaeGranick:2009,WangKuoBaeGranick:2012}. This crossover cannot reappear in the approach of the superstatistical dynamics. To interpret the phenomenon of such a crossover in the PDF of the Brownian yet non-Gaussian process, Chubynsky and Slater proposed a diffusing diffusivity model, in which the diffusion coefficient of the tracer particle evolves in time like the coordinate of a Brownian particle in a gravitational field \cite{ChubynskySlater:2014}. Chechkin {\it et al.} established a minimal model under the framework of Langevin equation with the diffusivity being the square of an Ornstein-Uhlenbeck process \cite{ChechkinSenoMetzlerSokolov:2017}. Due to the widespread applications of random diffusivity when describing the particle's motion in complex environments, the researches on systems with random parameters have been extended to many physical models, including underdamped Langevin equation \cite{SlezakMetzlerMagdziarz:2018,Vitali.etal:2018,ChenWang:2021}, generalized grey Brownian motion \cite{SposiniChechkinSenoPagniniMetzler:2018} and fractional Brownian motion \cite{JainSebastian:2018,MackalaMagdziarz:2019,Wang-etal:2020,Wang-etal:2020-2}, together with some discussions on ergodic property of random diffusivity systems \cite{CherstvyMetzler:2016,WangChen:2021,WangChen:2022}.

Our aim here is to consider the effect of an external force field on the Brownian yet non-Gaussian processes. Since it is convenient to describe a motion under an external force or an environment with fluctuation in a Langevin equation, we will investigate the effect of a force on the
minimal Langevin model with diffusing diffusivity proposed in Ref. \cite{ChechkinSenoMetzlerSokolov:2017}, where a Brownian particle with a random diffusivity $D(t)$ is described by
\begin{equation}\label{model0}
  \frac{d}{d t}x(t)=\sqrt{2D(t)}\xi(t).
\end{equation}
Here, $\xi(t)$ is the Gaussian white noise with mean zero and correlation function $\langle\xi(t_1)\xi(t_2)\rangle=\delta(t_1-t_2)$, and $D(t)$ is the square of an Ornstein-Uhlenbeck process to guarantee its positivity and randomness.

When considering the response of such a random diffusivity model to an external disturbance or the internal fluctuation of the system, we need to pay attention to whether the fluctuation-dissipation theorem (FDT) is valid or not in this system. The FDT plays a fundamental role in the statistical mechanics of nonequilibrium states and of irreversible processes \cite{Kubo:1966,MarconiPuglisiRondoniVulpiani:2008}. For this reason, two kinds of random diffusivity models, one satisfies FDT and one not, are considered, and their difference is also a main concerned object in this paper.

In addition to the FDT, Brownian motion also has a good property about Einstein relation which connects the fluctuation of an ensemble of particles with their mobility under a constant force $F$ by an equality \cite{Kubo:1966,MetzlerKlafter:2000}
\begin{equation}\label{ER}
  \langle x_F(t)\rangle=\frac{\langle x_0^2(t)\rangle}{2k_B\mathcal{T}}F.
\end{equation}
Here, $k_B$ is the Boltzmann constant, $\mathcal{T}$ is the absolute temperature of a heat bath, $x_F(t)$ and $x_0(t)$ denote the particle positions with and without the constant force $F$, respectively. Furthermore, the Einstein relation has been found to be valid for both normal and anomalous processes close to equilibrium in the limit $F\rightarrow0$, which can be derived from linear response theory \cite{BarkaiFleurov:1998,BenichouOshanin:2002,ShemerBarkai:2009,FroembergBarkai:2013-3}.
It will be interesting to find whether the Einstein relation holds or not in random diffusivity models.

In this paper, taking the two kinds of random diffusivity models satisfying the FDT or not as the main object, we first derive the Fokker-Planck equation of the PDF of particle's displacement for the two models under an arbitrary external force $F(x)$, and then make some specific analyses on the two models under a constant force $F$. The concerned observables mainly include the Einstein relation, the moments, the kurtosis of PDF, and the asymptotic behaviors of PDF.

The structure of this paper is as follows. In Sec. \ref{Sec2}, the two kinds of random diffusivity models are introduced. For arbitrary external force, the Fokker-Planck equations corresponding to the two models are derived in Sec. \ref{Sec3}. The detailed discussions on the observables for two models under a constant force are given in Secs. \ref{Sec4} and \ref{Sec5}, respectively. In Sec. \ref{Sec6}, we present the simulation results to verify the theoretical analyses on the observables for the case with constant force, and make a detailed comparison between the two models. Some discussions and summaries are provided in Sec. \ref{Sec5}. For convenience, we put some mathematical details in Appendix.




\section{Two random diffusivity models}\label{Sec2}
Since the FDT plays an important role on the diffusion behavior of a Langevin system, the difference between the two models concerned here is embodied in whether the FDT is valid or not.
Based on the random diffusivity model in Eq. \eqref{model0} characterizing the motion of a free particle, two kinds of models under an external force $F(x)$ can be written as
\begin{equation}\label{model-FDT}
  \frac{d}{d t}x(t)=\sqrt{2k_B\mathcal{T}D(t)}\xi(t)+D(t)F(x),
\end{equation}
and
\begin{equation}\label{model-NFDT}
  \frac{d}{d t}x(t)=\sqrt{2k_B\mathcal{T}D(t)}\xi(t)+F(x),
\end{equation}
respectively. The FDT is satisfied in Eq. \eqref{model-FDT}, which can be verified by dividing $D(t)$ on both sides, i.e.,
\begin{equation}\label{FDT-Eq}
  \frac{1}{D(t)}\frac{d}{d t}x(t)=\sqrt{\frac{2k_B\mathcal{T}}{D(t)}}\xi(t)+F(x).
\end{equation}
It can be seen that the dissipation memory kernel and correlation function of noise satisfy the relation \cite{Kubo:1966,KuboTodaHashitsume:1985,Zwanzig:2001,WangChenDeng:2019}
\begin{equation}\label{FDT-Relation}
  2k_B\mathcal{T}K(t_1-t_2)=\langle R(t_1)R(t_2)\rangle,
\end{equation}
where $K(t_1-t_2)=\delta(t_1-t_2)/D(t)$ is the dissipation memory kernel and $R(t)=\sqrt{\frac{2k_B\mathcal{T}}{D(t)}}\xi(t)$ is the internal noise in Eq. \eqref{FDT-Eq}. The FDT describes the phenomenon that the friction force and the random driving force come from the same origin and thus are closely related through Eq. \eqref{FDT-Relation}. For the Langevin system with a diffusing diffusivity $D(t)$ describing a spatially inhomogeneous environment, the FDT is still valid for each realization of $D(t)$.

Generalizing the idea in Refs. \cite{ChubynskySlater:2014,ChechkinSenoMetzlerSokolov:2017}, we use a generic overdampered Langevin equation to describe the diffusing diffusivity $D(t)$, i.e.,
\begin{equation}\label{model-DD}
\begin{split}
D(t)&=y^2(t),\\
\frac{d}{d t}y(t)&=f(y,t)+g(y,t) \eta(t),
\end{split}
\end{equation}
where the first equation is to guarantee the non-negativity of diffusivity $D(t)$, the second equation gives the evolution of auxiliary variable $y(t)$ with arbitrary functions $f(y,t)$ and $g(y,t)$ representing the external force and multiplicative noise on process $y(t)$. In addition, the noise $\eta(t)$ is also a Gaussion white noise with correlation function $\langle \eta(t_1)\eta(t_2)\rangle=\delta(t_1-t_2)$, similar to $\xi(t)$ but independent of $\xi(t)$. A special case that $f(y,t)=-y$ and $g(y,t)\equiv1$ yields the Ornstein-Uhlenbeck process $y(t)$ discussed in Ref. \cite{ChechkinSenoMetzlerSokolov:2017}.
Here, the arbitrary functions $f(y,t)$ and $g(y,t)$ in an overdamped Langevin equation result in a large range of diffusion processes beyond the Ornstein-Uhlenbeck process, including those reaching a steady state or not at long time limit, which is determined by the competitive roles between $f(y,t)$ and $g(y,t)$ \cite{WangDengChen:2019}. Many theoretical foundations have been established in the discussions on the ergodic properties and Feynman-Kac equations of the general overdamped Langevin equation \cite{WangDengChen:2019,WangChenDeng:2018,CairoliBaule:2017}.



\section{Fokker-Planck equations}\label{Sec3}
The Fokker-Planck equation governs the PDF $p(x,t)$ of finding the particle at position $x$ at time $t$, which describes the particle's stochastic motion in a macroscopic way. Compared with the Fokker-Planck equations containing integer derivatives for Brownian motion with or without an external force, those contain the fractional derivatives for many kinds of anomalous diffusion processes \cite{MetzlerKlafter:00,FriedrichJenkoBauleEule:2006,TurgemanCarmiBarkai:2009,KosztolowiczDutkiewicz:2021}. The Fokker-Planck equation for the random diffusivity model in Eq. \eqref{model0} have been derived in Ref. \cite{ChechkinSenoMetzlerSokolov:2017}.
Here we extend the model to the one containing an arbitrary external force and derive the corresponding Fokker-Planck equation.
Since the Langevin system includes three variables (the concerned process $x(t)$, diffusing diffusivity $D(t)$ and the auxiliary variable $y(t)$), and $D(t)$ depends on $y(t)$ explicitly as $D(t)=y^2(t)$, the bivariate PDF $p(x,y,t)$ is the underlying variable in the Fokker-Planck equation.
For convenience, we take $k_B\mathcal{T}=1$ in Eqs. \eqref{model-FDT} and \eqref{model-NFDT}, and take a space-dependent force $F(x)$. It should be noted that the results in this section are also valid for the case with time-dependent external force $F(x,t)$. The corresponding derivations can be obtained directly by replacing $F(x(s))$ with $F(x(s),s)$ in Eq. \eqref{split} and replacing $F(x(t))$ with $F(x(t),t)$ in Eq. \eqref{incre-x-y}.


Let us drive the Fokker-Planck equation corresponding to Eq. \eqref{model-FDT} firstly. Due to the FDT, the subordination method proposed in Ref. \cite{ChechkinSenoMetzlerSokolov:2017} for free particles can be applied here, i.e., rewriting the concerned process $x(t)$ as a compound process $x(t):=x(s(t))$ and splitting Eq. \eqref{model-FDT} into a Langevin system in subordinated form
\begin{equation}\label{split}
\begin{split}
\frac{d}{d s}x(s)&=\sqrt{2}\xi(s)+F(x(s)),\\
\frac{d}{d t}s(t)&=D(t),
\end{split}
\end{equation}
with the proof of the equivalence between them presented in Appendix \ref{App1}. The subordination method has been commonly used in Langevin system to describe subdiffusion \cite{Fogedby:1994,MetzlerKlafter:2000-2} or superdiffusion \cite{FriedrichJenkoBauleEule:2006,EuleZaburdaevFriedrichGeisel:2012,WangChenDeng:2019}.


The PDF $G(x,s)$ of process $x(s)$ in the first equation of Eq. \eqref{split} satisfies the classical Fokker-Planck equation \cite{Risken:1989,CoffeyKalmykovWaldron:2004}
\begin{equation}\label{Gxs}
\frac{\partial}{\partial s}G(x,s)=\left(-\frac{\partial}{\partial x}F(x)+\frac{\partial^2}{\partial x^2}\right)G(x,s).
\end{equation}
Combining the latter equation in Eq. \eqref{split}, we find $s(t)=\int_0^ty^2(t')dt'$. Therefore, $s(t)$ can be regarded as a functional of process $y(t)$, and the joint PDF $Q(s,y,t)$ satisfies the Feynman-Kac equation \cite{WangChenDeng:2018,CairoliBaule:2017,TurgemanCarmiBarkai:2009}
\begin{equation}\label{Qst}
\begin{split}
\frac{\partial}{\partial t}Q(s,y,t)&=\left(-\frac{\partial}{\partial y}f(y,t)+\frac{1}{2}\frac{\partial^2}{\partial y^2}g^2(y,t)\right)Q(s,y,t)\\
&~~~~-y^2\frac{\partial}{\partial s}Q(s,y,t).
\end{split}
\end{equation}
Since the two equations in Eq. \eqref{split} evolve independently, it holds that
\begin{equation}
\begin{split}
p(x,y,t)=\int_0^\infty G(x,s)Q(s,y,t)ds.
\end{split}
\end{equation}
Then combining the equations satisfied by $G(x,t)$ and $Q(s,y,t)$ in Eqs. \eqref{Gxs} and \eqref{Qst}, we obtain
\begin{equation}\label{FKE1}
\begin{split}
\frac{\partial}{\partial t}p(x,y,t)=&\int_0^\infty G(x,s)\frac{\partial}{\partial t}Q(s,y,t)ds\\
=&\left(-\frac{\partial}{\partial y}f(y,t)+\frac{1}{2}\frac{\partial^2}{\partial y^2}g^2(y,t)\right)p(x,y,t)\\
&-y^2\int_0^\infty G(x,s)\frac{\partial}{\partial s}Q(s,y,t) ds\\
=&\left(-\frac{\partial}{\partial y}f(y,t)+\frac{1}{2}\frac{\partial^2}{\partial y^2}g^2(y,t)\right)p(x,y,t)\\
&+y^2\left(-\frac{\partial}{\partial x}F(x)+\frac{\partial^2}{\partial x^2}\right)p(x,y,t),
\end{split}
\end{equation}
where the integration by parts has been used in the last equality and the corresponding boundary terms vanish.

For another model violating the FDT in Eq. \eqref{model-NFDT}, it cannot be split into two independent equations as Eq. \eqref{split}, and the subordination method is not applicable for this case. Instead, we adopt a universal Fourier transform method, which has been successfully used in deriving Fokker-Planck equation and Feynman-Kac equation \cite{DenisovHorsthemkeHanggi:2009,WangChenDeng:2018}.
Since the bivariate PDF $p(x,y,t)$ can be written as
$p(x,y,t)=\langle \delta(x-x(t))\delta(y-y(t))\rangle$, its Fourier transform ($x\rightarrow k_1, y\rightarrow k_2$) is
\begin{equation}
\begin{split}
  \tilde{p}(k_1,k_2,t)=& \int_{-\infty}^\infty\int_{-\infty}^\infty e^{-ik_1x-ik_2y}p(x,y,t)dxdy  \\
  =& \langle e^{-ik_1x(t)}e^{-ik_2y(t)}\rangle.
\end{split}
\end{equation}
The key point of this method is to derive the increment of $\tilde{p}(k_1,k_2,t)$ of order $\mathcal{O}(\tau)$ within a time interval $[t,t+\tau]$ when $\tau\rightarrow0$. Based on Eq. \eqref{model-NFDT} and the second equation of Eq. \eqref{model-DD}, we get the increments of $x(t)$ and $y(t)$ by omitting the higher order terms:
\begin{equation}\label{incre-x-y}
\begin{split}
x(t+\tau)-x(t)&\simeq\sqrt{2D(t)}\delta B_1(t)+F(x(t))\tau, \\[3pt]
y(t+\tau)-y(t)&\simeq f(y(t),t)\tau+g(y(t),t)\delta B_2(t),
\end{split}
\end{equation}
where $\delta B_i(t)=B_i(t+\tau)-B_i(t)$ is the increment of Brownian motion, $B_1(t)$ and $B_2(t)$ are independent from each other. By use of Eq. \eqref{incre-x-y}, the increment of $\tilde{p}(k_1,k_2,t)$ as $\delta \tilde{p}(k_1,k_2,t):=\tilde{p}(k_1,k_2,t+\tau)-\tilde{p}(k_1,k_2,t)$ can be evaluated as \\
\begin{widetext}
\begin{equation}\label{increment}
\begin{split}
\delta \tilde{p}(k_1,k_2,t)&=\langle e^{-ik_1x(t+\tau)}e^{-ik_2y(t+\tau)}\rangle-\langle e^{-ik_1x(t)}e^{-ik_2y(t)}\rangle\\
&\simeq \langle e^{-ik_1x(t)}e^{-ik_2y(t)}(e^{-ik_1(\sqrt{2D(t)}\delta B_1(t)+F(x(t))\tau)}e^{-ik_2(f(y(t),t)\tau+g(y(t),t)\delta B_2(t))}-1)\rangle\\
&\simeq \langle e^{-ik_1x(t)}e^{-ik_2y(t)}(-k_1^2D(t)\tau-ik_1F(x(t))\tau-ik_2f(y(t),t)\tau-\frac{1}{2}k_2^2g^2(y(t),t)\tau)\rangle,
\end{split}
\end{equation}
where we perform the ensemble average on $\delta B_1(t)$ and $\delta B_2(t)$ in the last line. More precisely, Eq. \eqref{incre-x-y} implies that both $x(t)$, $y(t)$ and $D(t)$ only depend on the increments $B_i$ of Brownian motion before time $t$, and thus they are independent from the increment $\delta B_i(t)$. We deal with the last two exponential functions in the second line by Taylor's series and only retain the terms of order $\mathcal{O}(\tau)$ as the last line shows.
Then dividing Eq. \eqref{increment} by $\tau$ on both sides, and taking the limit $\tau\rightarrow 0$, one arrives at
\begin{equation}
\begin{split}
\frac{\partial}{\partial t}\tilde{p}(k_1,k_2,t)=&-k^2_1\langle D(t)e^{-ik_1x(t)}e^{-ik_2y(t)}\rangle-ik_1\langle F(x(t))e^{-ik_1x(t)}e^{-ik_2y(t)}\rangle\\
&-ik_2\langle f(y(t),t)e^{-ik_1x(t)}e^{-ik_2y(t)}\rangle-\frac{1}{2}k^2_2\langle g^2(y(t),t)e^{-ik_1x(t)}e^{-ik_2y(t)}\rangle.
\end{split}
\end{equation}
\end{widetext}
Using the relation $D(t)=y^2(t)$ on the first term on the right-hand side, and performing inverse Fourier transform, we obtain the Fokker-Planck equation for the bivariate PDF $p(x,y,t)$ as
\begin{equation}\label{FKE2}
\begin{split}
\frac{\partial}{\partial t}p(x,y,t)&=\left(-\frac{\partial}{\partial x}F(x)+y^2\frac{\partial^2}{\partial x^2}\right)p(x,y,t)\\
&~~~+\left(-\frac{\partial}{\partial y}f(y,t)+\frac{1}{2}\frac{\partial^2}{\partial y^2}g^2(y,t)\right)p(x,y,t).
\end{split}
\end{equation}

Comparing the Fokker-Planck equations \eqref{FKE1} and \eqref{FKE2} for two different models in Eqs. \eqref{model-FDT} and \eqref{model-NFDT}, respectively, we find the main difference is embodied at the term containing external force $F(x)$. The former is $y^2F(x)$, i.e., $D(t)F(x)$ due to $D(t)=y^2(t)$ while the latter is $F(x)$. This difference is consistent to the discrepancy between the original models, i.e., $D(t)F(x)$ versus $F(x)$ in Eqs. \eqref{model-FDT} and \eqref{model-NFDT}. Actually, the Fokker-Planck equations \eqref{FKE1} can also be derived with the method of Fourier transform as Eq. \eqref{FKE2} by replacing $F(x)$ with $D(t)F(x)$ in the procedure.

Although the procedure of deriving the two Fokker-Planck equations looks a little complicated, the final form of Fokker-Planck equations can be understood in a simple way.
With a given $D(t)$, the corresponding Fokker-Planck equations governing the PDF $p(x,t)$ of displacement are
\begin{equation}\label{FK1}
\begin{split}
\frac{\partial}{\partial t}p(x,t)=D(t)\left[-\frac{\partial}{\partial x}F(x)+\frac{\partial^2}{\partial x^2}\right]p(x,t),
\end{split}
\end{equation}
and
\begin{equation}\label{FK2}
\begin{split}
\frac{\partial}{\partial t}p(x,t)=\left[-\frac{\partial}{\partial x}F(x)+D(t)\frac{\partial^2}{\partial x^2}\right]p(x,t)
\end{split}
\end{equation}
for Eqs. \eqref{model-FDT} and \eqref{model-NFDT}, respectively.
Then taking Eq. \eqref{FKE2} as an example, the terms on right-hand side can be divided into two parts. The first two terms are the ones in Fokker-Planck equation \eqref{FK2} by replacing $D(t)$ with $y^2$, while the last two terms come from the Fokker-Planck equation governing the PDF $p(y,t)$.
Albeit $D(t)$ is a diffusion process here, when we derive the Fokker-Planck equation governing the bivariate PDF $p(x,y,t)$, the role of $D(t)$ at the Fokker-Planck equation acts similarly to a deterministic function.


\section{Constant force field in Eq. (\ref{model-NFDT})}\label{Sec4}
For a comparison with the force-free case of Brownian yet non-Gaussian diffusion in Ref. \cite{ChechkinSenoMetzlerSokolov:2017}, we take $y(t)$ to be the Ornstein-Uhlenbeck process in the following discussions.
Let us first focus on the case that a constant force $F$ acts on the model \eqref{model-NFDT} where the FDT is broken. In this case, the Langevin system is written as
\begin{equation}\label{model_not_cons}
\begin{split}
\frac{d}{d t}x(t)&=\sqrt{2D(t)}\xi(t)+F,\\
D(t)&=y^2(t),\\
\frac{d}{d t}y(t)&=-y(t)+\eta(t).
\end{split}
\end{equation}
Based on the first equation, the process $x(t)$ can be written as
\begin{equation}\label{x-xf}
  x(t)=x_0(t)+Ft,
\end{equation}
where $x_0(t)$ denotes the trajectory of a free particle satisfying $dx_0(t)/dt=\sqrt{2D(t)}\xi(t)$ \cite{ChechkinSenoMetzlerSokolov:2017}. By the relation in Eq. \eqref{x-xf}, one has
\begin{equation}\label{relation}
\langle\Delta x^n(t)\rangle:=\langle (x(t)-\langle x(t)\rangle)^n\rangle=\langle x^n_0(t)\rangle,
\end{equation}
where $\langle x(t)\rangle=Ft$, and $x_0(t)$ is unbiased due to the symmetry of $\xi(t)$. Therefore, the ensemble-averaged MSD is $\langle \Delta x^2(t)\rangle=\langle x^2_0(t)\rangle\simeq t$.
The constant force here does not change the diffusion behavior and behaves as a decoupled force, which implies that model \eqref{model_not_cons} is Galilei invariant \cite{MetzlerKlafter:2000,CairoliKlagesBaule:2018,ChenWangDeng:2019-2}.
In addition, the drift $Ft$ dominates the diffusion process, and it holds that
\begin{equation}\label{moments1}
\langle x^n(t)\rangle\simeq F^nt^n.
\end{equation}
The relation between the first moment for the case with a constant force and the second moment of a free particle is
\begin{equation}
  \langle x(t)\rangle \simeq F\langle x^2_0(t)\rangle,
\end{equation}
which does not satisfy the Einstein relation in Eq. \eqref{ER}. This also relates to the violation of the FDT in Eq. \eqref{model-NFDT}.

Based on the moments, we can calculate the kurtosis to evaluate the deviation of the shape of a PDF from Gaussian distribution. The kurtosis of a one-dimensional Gaussian process is equal to $3$. Now for a biased process, the kurtosis is defined as
\begin{equation}\label{kurtosis}
\begin{split}
K=\frac{\langle \Delta x^4(t)\rangle}{\langle \Delta x^2(t)\rangle^2}.
\end{split}
\end{equation}
By use of Eq. \eqref{relation}, the kurtosis of the random diffusivity process under a constant force is
\begin{equation}\label{K1}
\begin{split}
K=\frac{\langle x^4_0(t)\rangle}{\langle x^2_0(t)\rangle^2}\simeq \left\{
\begin{array}{ll}
  9, &~ t\rightarrow 0, \\[5pt]
  3,  & ~t\rightarrow\infty,
\end{array}\right.
\end{split}
\end{equation}
consistent to the force-free case in Ref. \cite{ChechkinSenoMetzlerSokolov:2017}, where the PDF exhibits a crossover from exponential distribution to Gaussian distribution.

To be more delicate than the kurtosis, the explicit expression of the PDF $p(x,t)$ can be obtained through a translation of the PDF $p_0(x,t)$ of free particles to the positive direction with magnitude $Ft$, i.e.,
\begin{equation}\label{p1jianjin}
\begin{split}
p(x,t)&=p_0(x-Ft,t) \\[3pt]
&\simeq \left\{
\begin{array}{ll}
  \frac{1}{\pi t^{1/2}}K_0\left(\frac{x-Ft}{t^{1/2}}\right), &~ t\rightarrow 0,  \\[9pt]
  \frac{1}{(2\pi t)^{1/2}}\exp\left(-\frac{(x-Ft)^2}{2t}\right),  & ~t\rightarrow\infty,
\end{array}\right.
\end{split}
\end{equation}
where the expression of $p_0(x,t)$ is explicitly given in Eqs. (63) and (79) of Ref. \cite{ChechkinSenoMetzlerSokolov:2017} and $K_0$ is the Bessel function \cite{GradshteynRyzhikGeraniumsTseytlin:1980}.
In the short time limit, considering the asymptotics $K_0(z)\simeq \sqrt{\frac{\pi}{2z}}e^{-z}$ for $z\rightarrow\infty$, we have
\begin{equation}\label{PDF-short1}
\begin{split}
p(x,t)\simeq \frac{1}{\sqrt{2\pi|x-Ft|t^{1/2}}}\exp\left(-\frac{|x-Ft|}{t^{1/2}}\right),
\end{split}
\end{equation}
being an exponential distribution centered at $Ft$.

On the other hand, the short time asymptotics can be obtained from a superstatistical approach. For the time shorter than the diffusivity correlation time of the Ornstein-Uhlenbeck process, the diffusivity does not change considerably, and thus the initial condition in equilibrium of the Ornstein-Uhlenbeck process describes an ensemble of particles which diffuse with their own diffusion coefficient, resulting in a superstatistical result \cite{ChechkinSenoMetzlerSokolov:2017}.
In detail, the PDF $p_s(x,t)$ in superstatistical sense is given as the weighted average of a single Gaussian distribution $G(x,t|D)$ over the stationary distribution $p_D(D)$ of diffusivity $D$. The stationary distribution $p_D(D)$ can be obtained through the stationary distribution $f_{\textrm{st}}(y)=e^{-y^2}/\sqrt{\pi}$ of Ornstein-Uhlenbeck process in Eq. \eqref{model_not_cons}, i.e., \cite{ChechkinSenoMetzlerSokolov:2017}
\begin{equation}
\begin{split}
    p_D(D)=\int_{-\infty}^\infty f_{\textrm{st}}(y)\delta(D-y^2)dy
    =\frac{1}{\sqrt{\pi D}}e^{-D}.
\end{split}
\end{equation}
Then, it holds that
\begin{equation}
\begin{split}
p_s(x,t)&=\int_0^\infty p_D(D)G(x,t|D)dD  \\
&=\int_0^\infty \frac{1}{\sqrt{\pi D}}e^{-D} \cdot\frac{1}{\sqrt{4\pi Dt}}e^{-\frac{(x-Ft)^2}{4Dt}}dD \\
&=\frac{1}{\pi t^{1/2}}K_0\left(\frac{x-Ft}{t^{1/2}}\right),
\end{split}
\end{equation}
which is consistent to the short time asymptotics in Eq. \eqref{p1jianjin}.


\section{Constant force field in Eq. (\ref{model-FDT})}\label{Sec5}

The case that the constant force affects the diffusing diffusivity model \eqref{model-FDT} satisfying the FDT is
\begin{equation}\label{model_cons}
\begin{split}
\frac{d}{d t}x(t)&=\sqrt{2D(t)}\xi(t)+D(t)F,\\
D(t)&=y^2(t),\\
\frac{d}{d t}y(t)&=-y(t)+\eta(t).
\end{split}
\end{equation}
Similar to the way of deriving Fokker-Planck equation in Eq. \eqref{split}, it also brings convenience to rewrite the first equation of Eq. \eqref{model_cons} into a Langevin equation in the subordinated form, i.e.,
\begin{equation}\label{model_cons2}
\begin{split}
\frac{d}{d s}x(s)&=\sqrt{2}\xi(s)+F,\\
\frac{d }{d t}s(t)&=D(t),
\end{split}
\end{equation}
where the displacement is denoted as a compound process $x(t):=x(s(t))$.
Due to the independence between the two equations in Eq. \eqref{model_cons2}, it holds that
\begin{equation}\label{split-relation2}
\begin{split}
p(x,t)=\int_0^\infty G(x,s)O(s,t)ds,
\end{split}
\end{equation}
where $G(x,s)$ is the PDF of finding a Brownian particle under a constant force at position $x$ at time $s$, and $O(s,t)$ is the PDF of finding process $s(t)$ taking the value $s$ at time $t$. Therefore, $G(x,s)$ is a Gaussian distribution centered at $Fs$, i.e., $G(x,s)=\frac{1}{\sqrt{4\pi s}}e^{-\frac{(x-Fs)^2}{4s}}$ and $\tilde{G}(k,s)=e^{-ikFs-s k^2}$ in Fourier space ($x\rightarrow k$). Then we perform Fourier transform on Eq. \eqref{split-relation2} and obtain
\begin{equation}\label{pkt2}
\begin{split}
\tilde{p}(k,t)&=\int_0^\infty \tilde{G}(k,s)O(s,t)ds\\
&=\int_0^\infty e^{-(ikF+k^2)s} O(s,t)ds\\
&=\hat{O}(ikF+k^2,t),
\end{split}
\end{equation}
where $\hat{O}(ikF+k^2,t)$ denotes the Laplace transform $(s\rightarrow ikF+k^2)$ of the PDF $O(s,t)$. By use of the known result on the Laplace transform of $O(s,t)$ for the integrated square of the Ornstein-Uhlenbeck process \cite{Dankel:1991,ChechkinSenoMetzlerSokolov:2017}, we have
\begin{equation}\label{pkt-exact}
\begin{split}
    \tilde{p}(k,t)&=\exp\left(\frac{t}{2}\right)\left/\left[\frac{1}{2}\left(\sqrt{1+2\tilde{k}}+\frac{1}{\sqrt{1+2\tilde{k}}}\right)\right.\right.\\
&~~~\left.\times\textrm{sinh}\left(t\sqrt{1+2\tilde{k}}\right)
+\textrm{cosh}\left(t\sqrt{1+2\tilde{k}}\right)\right]^{\frac{1}{2}},
\end{split}
\end{equation}
where $\tilde{k}=ikF+k^2$. In order to satisfy the condition of Eq. \eqref{pkt-exact} proposed in Ref. \cite{Dankel:1991}, we assume that the initial position $y_0$ in Eqs. \eqref{model_not_cons} and \eqref{model_cons} obeys the equilibrium distribution of the Ornstein-Uhlenbeck process $y(t)$, i.e., a Gaussian distribution with mean zero and variance $1/2$:
\begin{equation}\label{EquilibriumDistribution}
  p_{\textrm{eq}}(y_0)=\frac{1}{\sqrt{\pi}}\exp(-y_0^2).
\end{equation}
This equilibrium distribution is also employed throughout all the simulations in Sec. \ref{Sec6}.
The expression of $\tilde{p}(k,t)$ in Eq. \eqref{pkt-exact} is exact for any time $t$, based on which we can evaluate the asymptotic moments and PDFs in $x$ space for short and long times.

For the moments, performing the Taylor expansion of exponential function in Eq. \eqref{pkt2} yields
\begin{equation}
\begin{split}
\tilde{p}(k,t)
&=\int_0^\infty e^{-\tilde{k}s} O(s,t)ds\\
&=1-\tilde{k}\langle s(t)\rangle+\frac{\tilde{k}^2}{2}\langle s^2(t)\rangle +\cdots.
\end{split}
\end{equation}
Then we use the formula $\langle x^n(t)\rangle=i^n\left.\frac{\partial^n}{\partial k^n}p(k,t)\right|_{k=0}$ and obtain the first four moments
\begin{equation}\label{Moments}
\begin{split}
\langle x(t)\rangle&=F\langle s(t)\rangle,\\
\langle x^2(t)\rangle&=2\langle s(t)\rangle+F^2\langle s^2(t)\rangle,\\
\langle x^3(t)\rangle&=6F\langle s^2(t)\rangle+F^3\langle s^3(t)\rangle,\\
\langle x^4(t)\rangle&=12\langle s^2(t)\rangle+12F^2\langle s^3(t)\rangle+F^4\langle s^4(t)\rangle.\\
\end{split}
\end{equation}
To obtain both short time and long time asymptotics, we need the accurate expressions of $\langle s^n(t)\rangle$, which are presented in Appendix \ref{App2}.
We find that for long times,
\begin{equation}\label{moments2}
  \langle x^n(t)\rangle\simeq \frac{F^n}{2^n}t^n.
\end{equation}
The relation between the first moment for the case with a constant force and the second moment for the force-free case is
\begin{equation}\label{ER-S}
  \langle x(t)\rangle \simeq \frac{F}{2}\langle x^2_0(t)\rangle,
\end{equation}
which satisfies the Einstein relation in Eq. \eqref{ER}.

Based on Eq. \eqref{Moments} and the accurate expression of $\langle s^n(t)\rangle$ in Appendix \ref{App2}, the MSD is equal to
\begin{equation}\label{M2}
\begin{split}
\langle\Delta x^2(t)\rangle&=\left(\frac{F^2}{2}+1\right)t+\frac{F^2}{4}(e^{-2t}-1)\\
&\simeq \left\{
\begin{array}{ll}
  t, &~t\rightarrow 0,  \\[5pt]
  \left(\frac{F^2}{2}+1\right)t,  & ~t\rightarrow \infty.
\end{array}\right.
\end{split}
\end{equation}
When $F=0$, it recovers to the constantly normal diffusion $\langle x^2(t)\rangle=t$.
Under the influence of a constant force, the particles still exhibit normal diffusion, but the effective diffusion coefficient increases from $1$ to $F^2/2+1$ as time goes.
Similar to the MSD in Eq. \eqref{M2}, the asymptotic expressions of fourth moment can be obtained from Eqs. \eqref{Moments} and Appendix \ref{App2}:
\begin{equation}\label{M4}
\begin{split}
\langle \Delta x^4(t)\rangle\simeq \left\{
\begin{array}{ll}
  9t^2, &~t\rightarrow 0,  \\[5pt]
  3\left(\frac{F^2}{2}+1\right)^2t^2,  & ~t\rightarrow \infty.
\end{array}\right.
\end{split}
\end{equation}
The constant force enhances the diffusion slightly since it only increases the diffusion coefficient without changing the diffusion behavior at long time limit.

Here we also evaluate the kurtosis to predict the shape of the PDF $p(x,t)$ for the case satisfying FDT. Considering the definition of kurtosis in Eq. \eqref{kurtosis}, and combining the moments in Eqs. \eqref{M2} and \eqref{M4}, we find
\begin{equation}\label{K2}
\begin{split}
K\simeq \left\{
\begin{array}{ll}
  9, &~ t\rightarrow 0, \\[5pt]
  3,  & ~t\rightarrow\infty.
\end{array}\right.
\end{split}
\end{equation}
Surprisingly, this result is consistent to the force-free case and the result in Eq. \eqref{K1}, which implies a possible crossover of PDF from exponential distribution to Gaussian distribution as the force-free case.

For the asymptotic expression of PDF $p(x,t)$, taking $t\rightarrow 0$ in Eq. \eqref{pkt-exact} yields
\begin{equation}\label{pkt-ST}
\begin{split}
\tilde{p}(k,t)\simeq t^{-\frac{1}{2}}\left(ikF+k^2+\frac{1}{t}\right)^{-\frac{1}{2}}.
\end{split}
\end{equation}
The normalization of the asymptotic PDF can be verified by $\tilde{p}(k=0,t)=1$. The inverse Fourier transform of $\tilde{p}(k,t)$ cannot be obtained easily. Since $t\rightarrow 0$, whenever $k\rightarrow0$ or $k\rightarrow\infty$, the imaginary part in the brackets of Eq. \eqref{pkt-ST} is much smaller than the real part, i.e., $kF\ll k^2+1/t$. Therefore, the constant force $F$ here only makes a slight biase on the original PDF. The expression of the biased PDF will be explicitly given through a superstatistical approach in the following. The asymptotic behavior at short time limit should be consistent to the corresponding superstatistical result.

In superstatistical approach, the effective PDF $p_s(x,t)$ is given as the weighted average of the conditional Gaussian distribution over the stationary distribution $p_D(D)$, i.e.,
\begin{equation}\label{short}
\begin{split}
p_s(x,t)&=\int_0^\infty p_D(D)G(x,t|D)dD  \\
&=\frac{1}{\sqrt{4\pi^2t}}e^{\frac{Fx}{2}}\int_0^\infty \frac{1}{D}e^{-D\left(1+\frac{F^2}{4}t\right)}e^{-\frac{x^2}{4Dt}}dD\\
&=\frac{1}{\pi\sqrt{t}}e^{\frac{Fx}{2}}K_0\left(\frac{\sqrt{4+F^2t}x}{2\sqrt{t}}\right),
\end{split}
\end{equation}
where  $G(x,t|D)=\frac{1}{\sqrt{4\pi Dt}}e^{-\frac{(x-FDt)^2}{4Dt}}$ has been used.
Then using the asymptotic behavior $K_0(z)\simeq \sqrt{\frac{\pi}{2z}}e^{-z}$ as $z\rightarrow\infty$, we arrive at
\begin{equation}\label{psxt}
\begin{split}
p_s(x,t)&\simeq \frac{1}{\sqrt{2\pi|x|t^{1/2}}}\frac{1}{\sqrt{(1+F^2t/4)^{1/2}}}  \\
&~~~~\times
\exp\left(\frac{Fx}{2}-\sqrt{1+F^2t/4}\frac{|x|}{t^{1/2}}\right).
\end{split}
\end{equation}
Corresponding to the short time aymptotics in Eq. \eqref{pkt-ST}, we take $t\ll 4/F^2$ in Eq. \eqref{psxt}, and obtain
\begin{equation}\label{psxt-ST}
\begin{split}
p_s(x,t)\simeq p_0(x,t) \exp\left(\frac{Fx}{2}\right),
\end{split}
\end{equation}
where
\begin{equation}\label{p0xt}
  p_0(x,t)=\frac{1}{\sqrt{2\pi|x|t^{1/2}}}\exp\left(-\frac{|x|}{t^{1/2}}\right)
\end{equation}
is the PDF of free particles in the superstatistical case. It can be seen that the constant force only adds a time-independent correction $e^{Fx/2}$ to the PDF of free particles at short time limit. Compared with the exponential part in $p_0(x,t)$, the exponential correction $e^{Fx/2}$ is negligible for short time since the exponential coefficient satisfies $F/2\ll 1/t^{1/2}$. This result is consistent to the previous kurtosis $K\simeq9$ in Eq. \eqref{K2} at short time limit and the analyses following Eq. \eqref{pkt-ST}.

On the other hand, the long time asymptotics $t\gg 4/F^2$ of $p_s(x,t)$ is
\begin{equation}\label{psxt-LT}
\begin{split}
p_s(x,t)\simeq p_0(x,t)C_F(x,t),
\end{split}
\end{equation}
where
\begin{equation}
  \begin{split}
    C_F(x,t)=\left\{
    \begin{array}{ll}
    \frac{1}{\sqrt{Ft^{1/2}/2}}\exp\left(\frac{x}{2t^{1/2}}\right),  &  x>0,  \\[5pt]
    \frac{1}{\sqrt{Ft^{1/2}/2}}\exp\left(Fx\right),  & x<0.
    \end{array}\right.
  \end{split}
\end{equation}
The constant force makes the PDF biased to the positive direction, i.e., decaying more slowly for $x>0$ but faster for $x<0$. Furthermore, the change in PDF at $x<0$ is more obvious than that at $x>0$. For long time limit, the exponential coefficient $F$ in $C_F(x,t)$ is much larger than $t^{-1/2}$ in $p_0(x,t)$, i.e., $F\gg 1/t^{1/2}$. So the dominating term of decaying when $x<0$ is $e^{Fx}$.


In contrast to the superstatistical results above, the real long time asymptotics of the Langevin system in Eq. \eqref{model_cons} can be found by taking $t\rightarrow \infty$ in Eq. \eqref{pkt-exact}. The asymptotic result is
\begin{equation}
\begin{split}
\tilde{p}(k,t)\simeq \frac{\sqrt{2}\exp\left(\frac{t}{2}(1-\sqrt{1+2\tilde{k}})\right)}
{\left[\frac{1}{2}\left(\sqrt{1+2\tilde{k}}+\frac{1}{\sqrt{1+2\tilde{k}}}\right)+1\right]^{1/2}}.
\end{split}
\end{equation}
Then we consider the large-$x$ behavior by taking $k\rightarrow 0$, and obtain
\begin{equation}
\begin{split}
\tilde{p}(k,t)\simeq \exp\left(-\frac{iFt}{2}k -\frac{(2+F^2)t}{4}k^2\right).
\end{split}
\end{equation}
With the inverse Fourier transform, the Gaussian distribution with mean $Ft/2$ and variance $(F^2/2+1)t$ is obtained:
\begin{equation}\label{long}
\begin{split}
p(x,t)\simeq \frac{1}{\sqrt{2\pi\left(\frac{F^2}{2}+1\right)t}}
\exp\left(-\frac{\left(x-\frac{F}{2}t\right)^2}{2\left(\frac{F^2}{2}+1\right)t}\right).
\end{split}
\end{equation}
This Gaussian shape is also consistent to the previous kurtosis $K\simeq3$ in Eq. \eqref{K2} at long time limit.

\section{Simulations}\label{Sec6}
In all our simulations, the initial position $y_0$ of the Langevin systems in Eqs. \eqref{model_not_cons} and \eqref{model_cons} is taken from the equilibrium distribution $N(0,1/\sqrt{2})$ in Eq. \eqref{EquilibriumDistribution}, and the two models in Eqs. \eqref{model_not_cons} and \eqref{model_cons} are recorded briefly as ``Model I'' and ``Model II'', respectively.
For a clear comparison between the two models, we put the simulation results of the same observable in one figure, with their moments in Fig. \ref{fig1}, kurtosis in Fig. \ref{fig2}, short-time PDFs in Fig. \ref{fig3}, and long-time PDFs in Fig. \ref{fig4}.

\begin{figure}
  \centering
  % Requires \usepackage{graphicx}
  \includegraphics[scale=0.5]{fig1}\\
  \caption{Moments $\langle x^n(t)\rangle$ with $n=1,2,3,4$. Model I (in red) and Model II (in blue) represent the Langevin systems in Eqs. \eqref{model_not_cons} and \eqref{model_cons}, respectively.
  %From the bottom to the top, each two lines represent the moment order $n$ from $n=1$ to $n=4$.
  The circle and star markers denote the simulation results, while the solid and dashed lines denote the theoretical results in Eqs. \eqref{moments1} and \eqref{moments2}, respectively. Based on Eqs. \eqref{moments1} and \eqref{moments2}, the two lines with the same $n$ are parallel for two models, i.e., $\langle x^n(t)\rangle_I=2^n\langle x^n(t)\rangle_{I\!I}$. Correspondingly, each two lines (or markers) from the bottom to the top represent the first, second, third, and fourth moments, respectively. Parameters: $T=10^3$, $F=2$, and $10^3$ samples are used for ensemble average.
}\label{fig1}
\end{figure}

\begin{figure}
  \centering
  % Requires \usepackage{graphicx}
  \includegraphics[scale=0.5]{fig2}\\
  \caption{Kurtosis (defined in Eq. \eqref{kurtosis}) in Model I (in red) and Model II (in blue) which represent the Langevin systems in Eqs. \eqref{model_not_cons} and \eqref{model_cons}, respectively. For the two models, the circle and star markers denote the simulation results, while the solid and dashed lines denote the theoretical results in Eqs. \eqref{K1-exact} and \eqref{K2-exact}, respectively. The kurtosis in Eqs. \eqref{K1} and \eqref{K2} both have the same asymptotics as the force-free case.
In contrast to the monotone decreasing behavior of the kurtosis line of Model I, that of Model II has a maximum value around $t=0.5$. Parameters: $T=10^2$, $F=2$, and $10^6$ samples are used for ensemble average.
}\label{fig2}
\end{figure}

In Fig. \ref{fig1}, we simulate the first four moments $\langle x^n(t)\rangle$ of two models, which agree with the theoretical results very well. According to the theoretical results in Eqs. \eqref{moments1} and \eqref{moments2}, we find the moments of two models only differ by a constant multiplier, i.e.,
\begin{equation}
  \langle x^n(t)\rangle_I=2^n\langle x^n(t)\rangle_{I\!I}.
\end{equation}
As a result, the solid and dashed lines (or circle and star markers) in Fig. \ref{fig1} are parallel for the same $n$.



In Fig. \ref{fig2}, we simulate the kurtosis for two models. They have the same asymptotic results in Eqs. \eqref{K1} and \eqref{K2} with a crossover from $K=9$ at the beginning to $K=3$ at the infinity. In addition to the asymptotic results, the exact expressions of kurtosis can be obtained by use of the definition in Eq. \eqref{kurtosis} and the first four moments $\langle x^n(t)\rangle$ in Eqs. \eqref{relation} and \eqref{Moments}. For convenience, the exact expressions are presented in Appendix \ref{App3}, where the latter (Eq. \eqref{K2-exact}) recovers the former (Eq. \eqref{K1-exact}) when $F=0$. The kurtosis of Model I is the same as the force-free case \cite{ChechkinSenoMetzlerSokolov:2017} due to its Galilei invariant property.
In contrast to the monotone decreasing kurtosis from $9$ to $3$ in Model I, the kurtosis of Model II has a maximum value around $t=0.5$, which means that for short time, the PDF of Model II undergoes a significant deviation from the Gaussian distribution. The reason can be found from the asymptotic PDF at short time limit in Eq. \eqref{psxt-ST}. The additional term $e^{Fx/2}$ brings a biase to the original exponential distribution $p_0(x,t)$ in Eq. \eqref{p0xt}. At long time limit, the PDF converges to the Gaussian distribution in Eq. \eqref{long}, corresponding to the monotone decreasing kurtosis after $t=0.5$ in Model II.



The asymptotic PDFs of two models for short time are presented in Fig. \ref{fig3}. The corresponding theoretical results are given in Eqs. \eqref{PDF-short1} and \eqref{psxt-ST}, respectively. For Model I, the PDF is exactly a translation to the positive direction with the magnitude $x=Ft$ of the original PDF $p_0(x,t)$ for force-free case. In contrast to Model I, the PDF of Model II is asymmetric due to the term $e^{Fx/2}$ in Eq. \eqref{psxt-ST}. It can be found that the lines in a semi-log graph (Fig. \ref{fig3}) are not exactly straight. The slight deviation from straight lines comes from the power-law correction term $|x|^{-1/2}$ in $p_0(x,t)$ in Eq. \eqref{p0xt}.

\begin{figure}
  \centering
  % Requires \usepackage{graphicx}
  \includegraphics[scale=0.5]{fig3}\\
  \caption{Short-time PDFs in Model I (in red) and Model II (in blue) which represent the Langevin systems in Eqs. \eqref{model_not_cons} and \eqref{model_cons}, respectively. For the two models, the circle and star markers denote the simulation results, while the solid and dashed lines denote the theoretical results in Eqs. \eqref{PDF-short1} and \eqref{psxt-ST}, respectively. The PDF of Model I is a symmetric exponential distribution with the center at $x=Ft$, while the PDF of Model II is an asymmetric skewed exponential distribution.
Parameters: $T=0.1$, $F=1$, and $10^7$ samples are used for ensemble average.}\label{fig3}
\end{figure}


The asymptotic PDFs of two models for long time are presented in Fig. \ref{fig4}. The corresponding theoretical results are given in Eqs. \eqref{p1jianjin} and \eqref{long}, respectively. Corresponding to the behavior of the kurtosis tending to $3$ in Fig. \ref{fig2}, the PDFs for two models both converge to the Gaussian distribution at long time limit. As the shape of PDFs in Fig. \ref{fig4} shows, the PDF of Model I has the mean $Ft$ and the variance $t$, while the one of Model II has a smaller mean $Ft/2$ but a larger variance $(F^2/2+1)t$. This feature comes from the fact that the constant force $F$ is multiplied by a stochastic process $D(t)$ which enhances the fluctuation, and that the mean of $D(t)$ at steady state is $1/2$ which weakens the effective drift by half.


\begin{figure}
  \centering
  % Requires \usepackage{graphicx}
  \includegraphics[scale=0.5]{fig4}\\
  \caption{Long-time PDFs in Model I (in red) and Model II (in blue) which represent the Langevin systems in Eqs. \eqref{model_not_cons} and \eqref{model_cons}, respectively. For the two models, the circle and star markers denote the simulation results, while the solid and dashed lines denote the theoretical results in Eqs. \eqref{p1jianjin} and \eqref{long}, respectively. Both the PDFs of two models are Gaussian shapes. The PDF of Model I has the mean $Ft$ and the variance $t$, while the one of Model II has a smaller mean $Ft/2$ but a larger variance $(F^2/2+1)t$.
Parameters: $T=20$, $F=1$, and $10^7$ samples are used for ensemble average.}\label{fig4}
\end{figure}



\section{Conclusion}\label{Sec7}
%This paper mainly focus on the effect of external force acting on the random diffusivity models.
Much attention has been taken to the scenarios of how external force (or constant force) influences a dynamic system with a power-law distributed waiting time \cite{BarkaiFleurov:1998,MetzlerKlafter:2000,FroembergBarkai:2013-3,ChenWangDeng:2019-2,ChenWangDeng:2019-3}.
This paper extends this issue to the random diffusivity model with a diffusing diffusivity $D(t)$, and explores how the diffusing diffusivity $D(t)$ acts in a system under an external force.
Considering the importance of the FDT in the statistical mechanics of nonequilibrium dynamics, we build two kinds of random diffusivity models with an external force based on whether the FDT satisfies or not.

The main studies on the two models can be divided into two parts: one derives the Fokker-Planck equation of random diffusivity models with arbitrary external force, and another one investigates in detail some common quantities by taking a specific constant force.
In the first part, the Fokker-Planck equations for the bivariate PDF $p(x,y,t)$ of
two random diffusivity models under an arbitrary external force field are derived in Eqs. \eqref{FKE1} and \eqref{FKE2}. Corresponding to the fact that the only difference between the original Langevin equations \eqref{model-FDT} and \eqref{model-NFDT} is $F(x)$ versus $D(t)F(x)$, the difference between the Fokker-Planck equations is only embodied at the external force term, $F(x)$ versus $y^2F(x)$. Although $D(t)$ is a diffusion process, the role of $D(t)$ at the expression of Fokker-Planck equations is similar to a deterministic function.
The structure of the derived Fokker-Planck equations has striking character. Due to the independence between the evolution of concerned process $x(t)$ and auxiliary process $y(t)$, the right-hand side of Fokker-Planck equations \eqref{FKE1} and \eqref{FKE2} can be divided into two parts, being the terms in the corresponding Fokker-Planck equation governing the PDF $p(x,t)$ and $p(y,t)$, respectively.

In the second part, we investigate the case with constant force field and the diffusivity $D(t)$ being
the square of Ornstein-Uhlenbeck process by studying the moments, Einstein relation, the kurtosis and the asymptotic behaviors of the PDF in detail.
For random diffusivity model in Eq. \eqref{model_not_cons} with the FDT broken, we establish the relation between the concerned process $x(t)$ under the effect of a constant force and the displacement $x_0(t)$ of a free particle by $x(t)=x_0(t) +Ft$. Thus we find this model is Galilei invariant, similar to the discussed anomalous processes \cite{MetzlerKlafter:2000,CairoliKlagesBaule:2018,ChenWangDeng:2019-2}. The diffusion behavior is not changed by the constant force. The mean value is $Ft$ and the Einstein relation is not valid in this model. Compared with the PDF of force-free case, the PDF is translated to the positive direction with a biase $Ft$, with the kurtosis and the asymptotic behaviors of PDF unchanged.

For the random diffusivity model in Eq. \eqref{model_cons} satisfying the FDT, the results are quite different from the force-free case. The theoretical derivations are based on the technique of splitting the first equation of Eq. \eqref{model_cons} into a Langevin equation in subordinated form. We find the mean value of displacement is $\langle x(t)\rangle=Ft/2$ in this case, satisfying the Einstein relation Eq. \eqref{ER-S}. Although the kurtosis has the same asymptotic behavior at $t\rightarrow0$ and $t\rightarrow\infty$, it is not monotone any more. It increases for short time and reaches the maximum around $t=0.5$ as Fig. \ref{fig2} shows.
For long time, the PDF surprisingly converges to a Gaussian distribution as the force-free case, while the PDF in short-time limit is biased due to a correction $e^{Fx/2}$ compared with the force-free case.

Many significant differences between the two models imply that the FDT also plays an important role in random diffusivity systems. Through detailed analyses on the kurtosis and the shape of PDF, the model satisfying the FDT shows many interesting dynamic behaviors due to the existence of random diffusivity $D(t)$. These results will bring benefits to the discussions on how anomalous diffusion particles response to the external force in more random diffusivity systems.



\section*{Acknowledgments}
This work was supported by the National Natural Science
Foundation of China under Grant No. 12105145, the Natural Science Foundation of Jiangsu Province under Grant No. BK20210325.

\appendix

\section{Equivalence between Eqs. \eqref{model-FDT} and \eqref{split}}\label{App1}
The main idea of proving the equivalence is to combine the two equations in Eq. \eqref{split} and to transform them into Eq. \eqref{model-FDT}. Noting that the diffusing diffusivity $D(t)$ is independent of the noise $\xi$, $D(t)$ can be regarded as a deterministic function and the ensemble average only acts on $\xi$ in the following.
Integrating the first equation in Eq. \eqref{split} yields
\begin{equation}\label{A1}
  x(s)=\sqrt{2}\int_0^s\xi(s')ds'+\int_0^sF(x(s'))ds',
\end{equation}
where we have assumed the initial condition $x(0)=0$.
Since the concerned process $x(t)$ has been written as a compound process $x(t):=x(s(t))$, $x(t)$ can be obtained by replacing $s$ with $s(t)$ in Eq. \eqref{A1}, i.e.,
\begin{equation}\label{A2}
  x(t)=\sqrt{2}\int_0^{s(t)}\xi(s')ds'+\int_0^{s(t)}F(x(s'))ds'.
\end{equation}
By using the second equation of Eq. \eqref{split} and performing the derivative over time $t$ on both sides of Eq. \eqref{A2}, one arrives at
\begin{equation}\label{A3}
  \frac{d}{dt}x(t)=\sqrt{2}D(t)\xi(s(t))+D(t)F(x(t)).
\end{equation}
Now the only difference between Eqs. \eqref{A3} and \eqref{model-FDT} is the first term on the right-hand side. It is sufficient to prove that they share the same correlation function since $\xi$ is white Gaussian noise. A formula about $\delta$-function
\begin{equation}
  \delta(h(x))=\sum_{i}\frac{\delta(x-x_i)}{|h'(x_i)|}
\end{equation}
will be used, where $x_i$ is the $i$th simple root of $h(x)=0$. Utilizing this formula and a truth that $s(t)$ is monotone increasing, we have
\begin{equation}
\begin{split}
    \langle\xi(s(t_1))\xi(s(t_2))\rangle
&=\delta(s(t_1)-s(t_2))  \\
&=\frac{1}{D(t_1)}\delta(t_1-t_2).
\end{split}
\end{equation}
Therefore, it can be found that both the correlation functions of the first term in Eqs. \eqref{A3} and \eqref{model-FDT} are
\begin{equation}
  2D(t_1)\delta(t_1-t_2).
\end{equation}




\section{Moments of process $s(t)$}\label{App2}
The moments of process $s(t)$ in Eq. \eqref{model_cons2} can be obtained from its PDF in Laplace space by use of the formula
\begin{equation}
  \langle s^n(t)\rangle=(-1)^n\left.\frac{\partial^n}{\partial \lambda^n}\hat{O}(\lambda,t)\right|_{\lambda=0},
\end{equation}
where $\hat{O}(\lambda,t)$ is the Laplace transform of $O(s,t)$, and \cite{Dankel:1991,ChechkinSenoMetzlerSokolov:2017}
\begin{equation}
\begin{split}
    \hat{O}(\lambda,t)&=\exp\left(\frac{t}{2}\right)\left/\left[\frac{1}{2}\left(\sqrt{1+2\lambda}+\frac{1}{\sqrt{1+2\lambda}}\right)\right.\right.\\
&~~~\left.\times\textrm{sinh}\left(t\sqrt{1+2\lambda}\right)
+\textrm{cosh}\left(t\sqrt{1+2\lambda}\right)\right]^{\frac{1}{2}}.
\end{split}
\end{equation}
\vspace{5mm}
With some tedious calculations, it holds that
%\begin{equation}
%\begin{split}
%\langle s(t)\rangle&=\frac{t}{2},\\[4pt]
%\langle s^2(t)\rangle&=\frac{1}{4}(e^{-2t}-1+2t+t^2),\\[4pt]
%\langle s^3(t)\rangle&=\frac{1}{8}\Big(3(4+5t)e^{-2t}-12+9t+6t^2+t^3\Big),\\[4pt]
%\langle s^4(t)\rangle&=\frac{1}{16}\Big(6(27+50t+25t^2)e^{-2t}+9e^{-4t} \\[2pt]
%   &~~~~-171+60t+54t^2+12t^3+t^4\Big).
%\end{split}
%\end{equation}
\begin{equation}
\langle s(t)\rangle=\frac{t}{2},
\end{equation}
\begin{equation*}
\langle s^2(t)\rangle=\frac{1}{4}(e^{-2t}-1+2t+t^2),
\end{equation*}
\begin{equation*}
\langle s^3(t)\rangle=\frac{1}{8}\Big(3(4+5t)e^{-2t}-12+9t+6t^2+t^3\Big),
\end{equation*}
\begin{equation*}
\begin{split}
\langle s^4(t)\rangle&=\frac{1}{16}\Big(6(27+50t+25t^2)e^{-2t}+9e^{-4t} \\[2pt]
   &~~~~-171+60t+54t^2+12t^3+t^4\Big).
\end{split}
\end{equation*}

\section{Exact kurtosis}\label{App3}
The exact theoretical expressions of kurtosis for two models in Eqs. \eqref{model_not_cons} and \eqref{model_cons} are
\begin{equation}\label{K1-exact}
  K=\frac{3}{t^2}(-1+e^{-2t}+2t+t^2)
\end{equation}
for Model I, and
\begin{widetext}
\begin{equation}\label{K2-exact}
  \begin{split}
  K&=\left\{-3-18F^2-\frac{171}{16}F^4+\left(6+\frac{33}{2}F^2+\frac{27}{4}F^4\right)t
    +\left(3+3F^2+\frac{3}{4}F^4\right)t^2 \right.\\
&~~~~+ \left. \left[3+\left(18+\frac{39}{2}t\right)F^2
    +\left(\frac{81}{8}+\frac{63}{4}t+6t^2\right)F^4\right]e^{-2t}+\frac{9}{16}F^4e^{-4t}  \right\}
\left/ \left[\left(\frac{F^2}{2}+1\right)t+\frac{F^2}{4}(e^{-2t}-1)\right]^2                        \right.
\end{split}
\end{equation}
for Model II.
\end{widetext}


\section*{References}
\bibliography{ReferenceCW}

\end{document}




% *** CITATION PACKAGES ***
\usepackage{cite}
\usepackage{url}

% *** MATH PACKAGES ***
\usepackage{amssymb}
\usepackage{amsthm, amsmath}
\usepackage{amsbsy}
\usepackage{color}

% *** SUBFIGURE PACKAGES ***
\usepackage{subfigure}

% *** PDF, URL AND HYPERLINK PACKAGES ***
\usepackage{cite,url}

% *** ALGORITHM ***
% \usepackage{algorithm}
% \usepackage{algpseudocode}
% \makeatletter
% \def\BState{\State\hskip-\ALG@thistlm}
% \makeatother

%%%%%% Added by Bang begin
\usepackage{dsfont}  % for indicator function
\usepackage{algorithmic}
\usepackage[ruled]{algorithm2e}
\SetKwInput{kwInitStep}{Initialization Step}
\SetKwInput{kwGibbsStep}{Gibbs Sampling Step}
\SetKwInput{kwTrainStep}{Training}
\SetKwInput{KwPredictStep}{Prediction}
%%%%%% Added by Bang end


\def\squareforqed{\IEEEQED}%\hbox{\rlap{$\sqcap$}$\sqcup$}}
\def\qed{\ifmmode\squareforqed\else{\unskip\nobreak\hfil
\penalty50\hskip1em\null\nobreak\hfil\squareforqed
\parfillskip=0pt\finalhyphendemerits=0\endgraf}\fi}

\newtheorem{theorem}{Theorem}
\newtheorem{lemma}[theorem]{Lemma}
\newtheorem{corollary}[theorem]{Corollary}
\newtheorem{claim}[theorem]{Claim}
\newtheorem{proposition}[theorem]{Proposition}
\newtheorem{predicate}[theorem]{Predicate}
\newtheorem{observation}[theorem]{Observation}
\newtheorem{openProb}{Open Problem}
\newtheorem{definition}{Definition}

\newcommand{\e}{\mathrm{e}}
\newcommand{\ud}{\mathrm{d}}
\newcommand{\E}{\boldsymbol{\mathrm{E}}}
\newcommand{\WN}{\mbox{WN}}
\newcommand{\var}{\boldsymbol{\mathbf{Var}}}
\newcommand{\cov}{\boldsymbol{\mathbf{Cov}}}
\DeclareMathOperator*{\argmax}{arg\,max}
\newcommand{\red}[1]{\textcolor{red}{#1}}
\newcommand{\blue}[1]{\textcolor{blue}{#1}}


\begin{document}

\title{Recover Fine-Grained Spatial Data \\from Coarse Aggregation}

% author names and affiliations
% use a multiple column layout for up to three different
% affiliations
\author{\IEEEauthorblockN{Bang Liu}\thanks{ This work was partially supported by the NSERC-CRD and NSERC-RGPIN grants.}
\IEEEauthorblockA{
University of Alberta\\
bang3@ualberta.ca}
\and
\IEEEauthorblockN{Borislav Mavrin}
\IEEEauthorblockA{
University of Alberta\\
mavrin@ualberta.ca}
\and
\IEEEauthorblockN{Linglong Kong}
\IEEEauthorblockA{
University of Alberta\\
lkong@ualberta.ca}
\and
\IEEEauthorblockN{Di Niu}
\IEEEauthorblockA{
University of Alberta\\
dniu@ualberta.ca}}



\maketitle
\thispagestyle{empty}
\pagestyle{empty}

\begin{abstract}
	%With the almost universal adoption of mobile devices in modern society, enormous amount of behavioral traces data are generated. One such kind of data is the cell phone activities records and its distribution. 
	%We study a challenging problem of reconstructing fine-grained spatial densities from coarse-grained measurements, i.e., the aggregate observations recorded for each subregion in a spatial field of interest. 
	In this paper, we study a new type of \emph{spatial sparse recovery} problem, that is to infer the fine-grained spatial distribution of certain density data in a region only based on the aggregate observations recorded for each of its subregions.
	One typical example of this spatial sparse recovery problem is to infer spatial distribution of cellphone activities based on aggregate mobile traffic volumes observed at sparsely scattered base stations.  
	We propose a novel \emph{Constrained Spatial Smoothing} (CSS) approach, which exploits the local continuity that exists in many types of spatial data to perform sparse recovery via finite-element methods, while enforcing the aggregated observation constraints through an innovative use of the ADMM algorithm. We also improve the approach to further utilize additional geographical attributes.
	%Understanding the spatial distribution of cell phone activities is not only of great value for telecommunication companies to conduct resource planning, but also provides insights into the study of human activity and the underlying urban ecology. However, in reality, due to technical overhead and privacy concerns related to mobile phone tracking, cell phone activity data with a fine-grained geographical distribution are not always available.
	Extensive evaluations based on a large dataset of phone call records and a demographical dataset from the city of Milan show that our approach significantly outperforms various state-of-the-art approaches, including Spatial Spline Regression (SSR).
%	Though a bunch of research works have been done based on cell phone activities, such kind of data is actually highly rare due to both technical overhead and privacy issues. Instead of having access to the distribution of cell phone activities on a whole region, in reality, usually only the aggregated volumes of cell phone activities associated with base stations that sparsely distributed within a region are observable. Therefore, how to infer the spatial distribution of cell phone activities from sparse aggregated observations is a very important problem.
%	In this paper, we study the problem of inferring cell phone activity distributions from aggregated sparse observations. We formulate the problem as sparse recovery and propose a new Constrained Spatial Smoothing approach. Our approach exploits the smoothness property of cell phone activities distribution to perform sparse spatial recovery. Besides, it is also able to incorporate external information as a regression add-on to further enhance recovery performance. Furthermore, we propose an Alternating Direction Method of Multipliers algorithm to decouple the problem and learn model parameters effectively.

\end{abstract}


Reinforcement learning has achieved great success in areas such as Game-playing \citep{silver2018general,vinyals2019grandmaster}, robotics \cite{kober2013reinforcement}, large language models \citep{ouyang2022training}, etc.
However, due to safety concerns or physical limitations, in some real-world reinforcement learning problems, we must consider additional constraints that may influence the optimal policy and the learning process \citep{garcia2015comprehensive}.
% For example, a robotic arm must not take actions that may cause harm to itself or the environments.
A standard framework to handle such cases is the constrained Markov Decision Process (CMDP) \citep{altman1999constrained}.
Within the CMDP framework, the agent has to maximize
the expected cumulative reward while
obeying a finite number of constraints, which are usually in the form of expected cumulative cost criteria.

However, we are sometimes concerned with the problem with a continuum of constraints.
For example,
the constraints we meet might be time-evolving or subject to uncertain parameters, which
cannot be formulated as an ordinary CMDP
(see Examples \ref{Example_Time_Evolving} and  \ref{Example_Uncertain}).
In this paper we would study a generalized CMDP  
to address the above problem.  Because the constraints are not only infinite-number but also lie
in a continuous set,
the generalization is not trivial. Fortunately, we find that we can borrow the idea behind semi-infinite programming (SIP) \citep{remez1934determination, hettich1993semi} to deal with the semi-infinite constraints.
Accordingly, we propose \emph{semi-infinitely constrained Markov decision processes} (SICMDPs)
as a novel complement to the ordinary CMDP framework.
%More specifically,  an SICMDP model %, we consider 
%contains a continuum of constraints whereas an ordinary CMDP contains a finite number of constraints. 

%This generalization is natural but not trivial. However, we can brows the idea  
%The idea is quite natural and can be backtracked
%to the practice of extending linear programming to linear semi-infinite programming (LSIP) %\cite{remez1934determination, GobernaLSIO1998}.
%In addition, 
%As a complementary approach to the ordinary CMDP framework, 
%SICMDP can be used to model these problems  which cannot be described by a finite number of constraints
%that are not covered by .
%For example,
%the restrictions we consider can be time-evolving or subject to uncertain parameters
%, thus
%cannot be described by a finite number of constraints but a continuum of constraints 
%(see Examples \ref{Example_Time_Evolving} and  \ref{Example_Uncertain}).

We also present two reinforcement learning algorithms to solve SICMDPs called SI-CRL and SI-CPO, respectively.
SI-CRL is a model-based reinforcement learning algorithm designed for tabular cases, and SI-CPO is a policy optimization algorithm for non-tabular cases.
% and analyze its performance both theoretically and empirically.
The main challenge is that we need to deal with a continuum of constraints, thus reinforcement learning algorithms for ordinary CMDPs do not work anymore.
In SI-CRL, we tackle this difficulty by first transforming the reinforcement learning problem to an equivalent LSIP problem, which can then be solved using methods in the LSIP literature like the dual exchange methods \citep{Hu1990,reemtsen1998numerical}.
In SI-CPO, we resort to the idea of cooperative stochastic approximation developed in \cite{lan2020algorithms, wei2020comirror}.
As far as we know, we are the first to introduce tools from semi-infinitely programming (SIP) into the reinforcement learning community for solving constrained reinforcement learning problems.

% To the best of our knowledge, we are the first to apply tools from semi-infinitely programming (SIP) to solve reinforcement learning problems.
Furthermore, we give theoretical analysis for both SI-CRL and SI-CPO.
We decompose the error of SI-CRL into two parts: the statistical error from approximating the true SICMDP with an offline dataset and the optimization error due to the fact that the solution of the LSIP problem obtained by the dual exchange method is inexact.
On the optimization side, we show that the iteration complexity of SI-CRL is $O\left(\left\{\mathrm{diam}(Y)L\sqrt{|\gS|^2|\gA|m}/\left[(1-\gamma)\epsilon\right]\right\}^m\right)$.
On the statistical side, we show that the sample complexity of SI-CRL is $\widetilde O\left(\frac{|S|^2|A|^2}{\epsilon^2(1-\gamma)^3}\right)$ if the offline dataset is generated by a generative model, and $\widetilde O\left(\frac{|S||A|}{\nu_{\min} \epsilon^2(1-\gamma)^3}\right)$ if the dataset is generated by a probability measure $\nu$ as considered in \cite{chen2019information}.
Here $\widetilde O$ means that all logarithm terms are discarded.
For SI-CPO, things become a little more complicated because other than the statistical error and the optimization error, we also need to consider the function approximation error, which comes from imperfect policy parametrizations.
It is shown if the function approximation error can be controlled to $O(\epsilon)$ order, the iteration complexity of SI-CPO is $\widetilde{O}\left(\frac{1}{\epsilon^2(1-\gamma)^6}\right)$ and the sample complexity of SI-CPO is $\widetilde{O}(\frac{1}{\epsilon^4(1-\gamma)^{10}})$.
Here our iteration complexity bound is equivalent to a typical $\widetilde O(1/\sqrt{T})$ global convergence rate.

We perform a set of numerical experiments to illustrate the SICMDP model and validate our proposed algorithms.
Specifically, we examine two numerical examples, namely the discharge of sewage and ship route planning.
Through the discharge of sewage example, we show the advantage of the SICMDP framework over the CMDP baseline obtained by naive discretization in modeling realistic sequential decision-making problems.
Moreover, we demonstrate the effectiveness of the SI-CRL and SI-CPO algorithms in such tabular environments. 
In the ship route planning example, we illustrate the benefits of the SICMDP framework and the ability of the SI-CPO algorithm to address complex continuous control tasks involving continuous state spaces with modern deep reinforcement learning techniques.

% In summary, our contributions are listed as follows.
% First, we present the SICMDP model, which can be viewed as a generalization of the ordinary CMDP model.
% Second, we propose an algorithm to perform reinforcement learning for SICMDPs, which is called SI-CRL, and we believe that we are the first to apply tools from SIP
% to solve reinforcement learning problems.
% Third, we give a theoretical analysis of SI-CRL and identify both its sample complexity and iteration complexity.
% In addition, we perform numerical experiments to illustrate the SICMDP model and validate the SI-CRL algorithm.
% \{This paragraph can be removed!!! \}





\section{General Framework}
\label{sec:framework}
Our goal in this work is to demonstrate the utility of natural language descriptions in assisting policy transfer across domains. In this section, we first describe our environment setup and the general framework of our approach. The details of our model and algorithm follow in Section~\ref{sec:model}.

\subsection{Environment Setup} 
We model a single environment as a Markov Decision Process (MDP),  $E = \langle S, A, T, R, O, Z \rangle$. Here, $S$ is the state space, and $A$ is the set of actions available to the agent. In this work, we consider every state $s \in S$ to be a 2-dimensional grid of size $m \times n$, with each cell containing an entity symbol $o \in O$.\footnote{In our experiments, we relax this assumption to allow for multiple entities per cell, but for ease of description, we shall assume a single entity per cell. The assumption of 2-D worlds can also be easily relaxed to generalize our model to other situations.} $T$ is the transition distribution over all possible next states $s'$ conditioned on the agent choosing action $a$ in state $s$. $R$ determines the reward provided to the agent at each time step. The agent does not have access to the true $T$ and $R$ of the environment. Each domain also has a goal state $s_g \in S$ which determines when an episode terminates. Finally, $Z$ is the complete set of text descriptions provided to the agent for this particular environment. 

\subsection{Reinforcement Learning (RL)}
The goal of an autonomous agent is to maximize cumulative reward obtained from the environment. A traditional way to achieve this is by learning an action value function $Q(s,a)$ through reinforcement. The \emph{Q-function} predicts the expected future reward for choosing action~$a$ in state~$s$. A straightforward policy then is to simply choose the action that maximizes the $Q$-value in the current state: 

\begin{dmath*}
\pi(s) = \argmax_a Q(s,a)
\end{dmath*}

If we also make use of the descriptions, we have a text-conditioned policy: 
\begin{dmath}
\pi(s, Z) = \argmax_a Q(s, a, Z)
\end{dmath} 

A successful control policy for an environment will contain both knowledge of  the environment dynamics and the capability to identify goal states. While the latter is task-specific, the former characteristic is more useful for learning a general policy that transfers to different domains. Based on this hypothesis, we employ a model-aware RL approach that can learn the dynamics of the world while estimating the optimal $Q$. Specifically, we make use of \emph{Value Iteration (VI)}~\cite{sutton1998introduction}, an algorithm based on dynamic programming. The update equations for value iteration in our setup are:
\begin{align}
Q^{(n+1)}(s, a, Z) &= \sum_{s' \in S} T(s' | s, a, Z) [ R(s', Z) + \gamma V^{(n)}(s', Z) ]  \nonumber \\
V^{(n+1)}(s, Z) &= \max_a Q^{(n+1)}(s,a, Z) 
\label{eq:vi}
\end{align}
where $\gamma$ is a discount factor and $n$ is the iteration number. The updates require an estimate of $T$ and $R$, which the agent must obtain through exploration of the environment.

% Note that this assumes the agent has knowledge of the true $T$ and $R$ in order to estimate $Q$ and $V$. Since our setup does not provide this information, the agent has to estimate the transition and reward functions from its interactions with the world.

\subsection{Text Descriptions}
Estimating the dynamics of the environment from interactive experience can require a significant number of samples. Our main hypothesis is that if an agent can derive information about the dynamics from text descriptions, it can determine $T$ and $R$ faster and more accurately. 
% Hence, we work with text that provides such relevant particulars of the domain.

For instance, consider the sentence \emph{``Red bat that moves horizontally, left to right''}. This talks about the movement of a third-party entity (\emph{bat}), independent of the agent's goal. Provided the agent can learn to interpret this sentence, it can then infer the direction of movement of a different entity (e.g. \emph{``A tan car moving slowly to the left''}) in a different domain. Further, this inference is useful even if the agent has a completely different goal. On the other hand, instruction-like text such as \emph{``Move towards the wooden door''} is highly context-specific and only relevant to domains that have the mentioned goal.

With this in mind, we provide the agent with text descriptions that collectively portray characteristics of the world. These descriptions are crowdsourced by asking humans to view gameplay videos and describe entities.  A single description talks about one particular entity in the world. The text contains (partial) information about the entity's movement and interaction with the player avatar. Each description is also aligned to its corresponding entity in the environment and not all entities may have a description.
% We make sure that a simple mapping cannot be found between entities in different domains using just their names in text.\todo{clarify this}
Figure~\ref{fig:descriptions} provides some samples; more details on data collection and statistics are in Section~\ref{sec:experiments}. 

\begin{figure}
  \begin{annotationbox}
%     \centering
    \small
      \begin{itemize}
        \item Scorpion2: \emph{Red scorpion that moves up and down} 
        \item Alien3: \emph{This character slowly moves from right to left while having the ability to shoot upwards}
        \item Sword1: \emph{This item is picked up and used by the player for attacking enemies}
      \end{itemize}
%     }
  \end{annotationbox}
  \caption{Example text descriptions of entities in different environments, collected using Amazon Mechanical Turk. Turkers were shown videos of gameplay in the different environments and asked to describe each entity's behavior or role. Note that these sentences are not instructive, since they provide no direct information on how to act in the environment.}
  \label{fig:descriptions}
\end{figure}

\subsection{Transfer for RL}
In order to test our grounding hypothesis, we consider learning across multiple environments. Specifically, an agent can learn to ground language semantics in an environment $E_1$ and then we can test its understanding capability by placing it in a new unseen domain, $E_2$. The agent can obtain unlimited experience in $E_1$, and after convergence of its policy, it is allowed to interact with and learn a policy for $E_2$. We do not provide the agent with any explicit mapping between different entities or goals across domains, either directly or through the text. For instance, even though the boulders in \emph{Boulderchase} are impassable objects just like the walls in \emph{Bomberman}~\ref{fig:example}, the agent does not have access to a mapping between these entities. In this setup, the agent's goal is to re-utilize information obtained through its interactions in $E_1$ to learn more efficiently in $E_2$.


% \paragraph{Environment Setup} 
% We model a single environment as a Markov Decision Process (MDP), represented by $E = \langle S, A, T, R, O, Z \rangle$. Here, $S$ is the state space, and $A$ is the set of actions available to the agent. In this work, we consider every state $s \in S$ to be a 2-dimensional grid of size $m \times n$, with each cell containing an entity symbol $o \in O$.\footnote{In our experiments, we relax this assumption to allow for multiple entities per cell, but for ease of description, we shall assume a single entity. The assumption of 2-D worlds can also be easily relaxed to generalize our model to other situations.} $T$ is the transition distribution over all possible next states $s'$ conditioned on the agent choosing action $a$ in state $s$. $R$ determines the reward provided to the agent at each time step. The agent does not have access to the true $T$ and $R$ of the environment. Each domain also has a goal state $s_g \in S$ which determines when an episode terminates. Finally, $Z$ is the complete set of text descriptions provided to the agent for this particular environment. 

% \paragraph{Reinforcement learning (RL)}
% The goal of an autonomous agent is to maximize cumulative reward obtained from the environment. A traditional way to achieve this is by learning an action value function $Q(s,a)$ through reinforcement. The \emph{Q-function} predicts the expected future reward for choosing action~$a$ in state~$s$. A straightforward policy then is to simply choose the action that maximizes the $Q$-value in the current state: $\pi(s) = \argmax_a Q(s,a)$. If we also make use of the descriptions, we have a text-conditioned policy: $\pi(s, Z) = \argmax_a Q(s, a, Z)$. 

% A successful control policy for an environment will contain both knowledge of  the environment dynamics and the capability to identify goal states. While the latter is task-specific, the former characteristic is more useful for learning a general policy that transfers to different domains. Based on this hypothesis, we employ a model-aware RL approach that can learn the dynamics of the world while estimating the optimal $Q$. Specifically, we make use of \emph{Value Iteration (VI)}~\cite{sutton1998introduction}, an algorithm based on dynamic programming. The update equations are as follows:
% \begin{align}
% Q^{(n+1)}&(s, a, Z) = R(s, a, Z)  \nonumber \\ 
% &+ \gamma \sum_{s' \in S} T(s' | s, a, Z) V^{(n)}(s', Z)  \nonumber \\
% V^{(n+1)}&(s, Z) = \max_a Q^{(n+1)}(s,a, Z) 
% \label{eq:vi}
% \end{align}
% where $\gamma$ is a discount factor and $n$ is the iteration number. The updates require an estimate of $T$ and $R$, which the agent must obtain through exploration of the environment.

% % Note that this assumes the agent has knowledge of the true $T$ and $R$ in order to estimate $Q$ and $V$. Since our setup does not provide this information, the agent has to estimate the transition and reward functions from its interactions with the world.

% \paragraph{Text descriptions}
% Estimating the dynamics of the environment from interactive experience can require a significant number of samples. Our main hypothesis is that if an agent can derive information about the dynamics from text descriptions, it can determine $T$ and $R$ faster and more accurately. 
% % Hence, we work with text that provides such relevant particulars of the domain.

% For instance, consider the sentence \emph{``Red bat that moves horizontally, left to right.''}. This talks about the movement of a third-party entity ('bat'), independent of the agent's goal. Provided the agent can learn to interpret this sentence, it can then infer the direction of movement of a different entity (e.g. \emph{``A tan car moving slowly to the left''} in a different domain. Further, this inference is useful even if the agent has a completely different goal. On the other hand, instruction-like text, such as \emph{``Move towards the wooden door''}, is highly context-specific, only relevant to domains that have the mentioned goal.

% With this in mind, we provide the agent with text descriptions that collectively portray characteristics of the world. A single description talks about one particular entity in the world. The text contains (partial) information about the entity's movement and interaction with the player avatar. Each description is also aligned to its corresponding entity in the environment. 
% % We make sure that a simple mapping cannot be found between entities in different domains using just their names in text.\todo{clarify this}
% Figure~\ref{fig:descriptions} provides some samples; details on data collection and statistics are in Section~\ref{sec:experiments}.

% \begin{figure}
%   \begin{annotationbox}
% %     \centering
%     \small
%       \begin{itemize}[leftmargin=0.45cm]
%         \item Scorpion2: \emph{Red scorpion that moves up and down} 
%         \item Alien3: \emph{This character slowly moves from right to left while having the ability to shoot upwards}
%         \item Sword1: \emph{This item is picked up and used by the player for attacking enemies}
%       \end{itemize}
% %     }
%   \end{annotationbox}
%   \caption{Some example text descriptions of entities in different environments.}
%   \label{fig:descriptions}
% \end{figure}

% % \begin{table}[h]
% % \centering
% % \resizebox{\linewidth}{!}{%
% % \begin{tabular}{  l  } \toprule
% % \textit{Red scorpion that moves up and down} \\
% % \textit{This character slowly moves left to right} \\ 
% % \textit{and has the ability to shoot to the left which can kill the player} \\
% % \textit{this item is picked up and used by the player for attacking enemies}
% % \textit{Ghost1 moves horizontally and is an enemy} \\
% % \textit{Alien3 is an enemy bomber shooting upwards} \\
% % \bottomrule
% % \end{tabular}
% % }
% % \caption{Some example text descriptions of various entities for a game environment.}
% % \label{table:descriptions}
% % \end{table}

% \paragraph{Transfer for RL}
% A natural scenario to test our grounding hypothesis is to consider learning across multiple environments. The agent can learn to ground language semantics in an environment $E_1$ and then we can test its understanding capability by placing it in a new unseen domain, $E_2$. The agent is allowed unlimited experience in $E_1$, and after convergence of its policy, it is then allowed to interact with and learn a policy for $E_2$. We do not provide the agent with any mapping between entities or goals across domains, either directly or through the text. The agent's goal is to re-utilize information obtained in $E_1$ to learn more efficiently in $E_2$.

% % For example, a `bat' in $E_1$ is not given the same symbol as a `bat' in $E_2$.\footnote{Also note that the behavior of a bat in the two environments can be substantially different.} The aim of transfer is to re-utilize information obtained in $E_1$ to learn efficiently in $E_2$.


% % \paragraph{Environment}
% % In our setup, an environment consists of a state space $\mathcal{S}$ and a set of text descriptions $\mathcal{Z} = \{z_i\}$. The state is a $m \times n$ grid world containing entities drawn from a set $\mathcal{O}$ (with unique IDs). Given an input state $s \in \mathcal{S}$, the agent can take a discrete action $a \in \mathcal{A}$, and observe a new state $s'$ of the environment, which changes according to a transition distribution $\mathcal{T}(s' | s,a)$. The environment also provides the agent with a reward $\mathcal{R}(s,a)$ at every time step. Note that the agent does not have access to the true $\mathcal{T}$ and $\mathcal{R}$ of its environment. 

% % Each description $z_i$ is a sentence that provides information about one particular entity type such as its movements or interactions with other entities. We assume access to the mapping between each $z_i$ and its corresponding object $o_i$.

% % \paragraph{Reinforcement Learning (RL)}
% % In the RL framework, the goal of an autonomous agent is to perform actions that maximize the cumulative reward it obtains from the environment. This is done by learning an action value function $Q(s,a)$, which predicts the expected future reward of choosing action~$a$ in state~$s$. Using this, a straightforward policy is to simply choose the action that maximizes the $Q$-value in the current state: $\pi(s) = \argmax_a Q(s,a)$.
% % % \todo{define Q and policy}

% % \paragraph{Transfer setup}
% % We are given a source environment $e_u$ and a target environment $e_v$. In addition, we have access to corresponding sets of text descriptions $\mathcal{Z}_u$ and $\mathcal{Z}_v$, respectively. We first estimate parameters $\Theta$ of a policy $\pi_u(s, \mathcal{Z}_u)$ through several interactions with $e_u$. The policy is optimized to obtain maximum possible reward on the source environment. Now, using $\pi_u$, our goal is to learn an optimal policy for the target environment $e_v$ in as few interactions as possible, by transferring knowledge obtained in $e_u$.


% % % Formally, let us consider an environment $e \in \mathcal{E}$, consisting of entities $\mathcal{O}^e = \{o^e_i\}$ and correspondingly aligned text descriptions $\mathcal{Z}^e = \{z^e_i\}$. Our goal is to learn a mapping from these descriptions to the control dynamics, while simultaneously learning an optimal policy. In this work, we consider two-dimensional state spaces, but the main facets of our model can be extended to other scenarios.
%!TEX root = main.tex
\section{Patched Estimation and\\ Spatial Spline Regression}
\label{sec:SSR}
In this section, we first explore some tentative solutions, and then point out their insufficiency and limitations in handling our constrained spatial sparse recovery problem.

\textbf{Patched Piece-wise Constant Estimation}.
In practice, we only know the locations of all $\mathbf p_{B_i}$'s and their corresponding aggregated volumes $z_i$'s. What is not known is the fine-grained distribution of each volume $z_i$ across $\Omega_{B_i}$, the subregion covers the observed point  $B_i$.

As a first heuristic, we can assume that the density is  distributed uniformly within each $\Omega_{B_i}$ and estimate $f(\mathbf p_j)$ as the volume $z_i$ divided by its area:
\begin{equation}\label{eq:patched}
	\bar f(\mathbf p_j) = \frac{z_i}{|\Omega_{B_i}|}, \text{ for each }\mathbf p_j \in \Omega_{B_i},
\end{equation}
where $|\Omega_{B_i}|$ is the area of $\Omega_{B_i}$.
%That is, first, we compute the average density for each base station by dividing corresponding volume by the area. 
%Further we assume that the density for each base station is constant on the area covered by this base station. 
Hence we obtain patched piece-wise constant estimation. In this paper, we may use {\it patch} to refer to $\Omega_{B_i}$, the subregion covered $B_i$. 

However, the patched estimation oversimplifies the solution, since the obtained estimates $\bar f(\mathbf p_j)$ are far from being smooth and in fact may have discontinuous jumps on the borders of patches, which will be illustrated in our evaluation. %\red{(Unless all average densities are the same)}
In practice, however, the piece-wise uniformity assumption fails: $f(\mathbf p_j)$ is not constant within a certain patch $\Omega_{B_i}$. In fact, $f(\mathbf p_j)$ should slowly change across neighboring points, as the underlying geographical and demographical characteristics also change smoothly across  regions.

\textbf{Spatial Spline Regression}.
The observations above naturally lead to the idea of using spatial smoothing techniques to smoothen the patched estimation $\bar f(\mathbf p_j)$ to remove the discontinuities and jumps. In the following, we briefly describe a recently proposed powerful smoothing technique called Spatial Spline Regression (SSR) \cite{Sanga13}. After demonstrating its usage for our particular problem, we point out the major limitations for the spatial sparse recovery problem.


Given a set of $l$ spatial data points in $\Omega$, which contains the following information: \emph{1)} the values of these $l$ points: $\{h_j\}_{j=1}^l$, \emph{2)} their positions $\{\mathbf p_j\}_{j=1}^l$, and \emph{3)} their attribute vectors $\{\mathbf w_j\}_{j=1}^l$, SSR fits a smooth spatial field $f$ by minimizing the following penalized sum of square errors \cite{Sanga13}, \cite{Ramsay02}, i.e.,
\begin{equation}\label{eq:minSSR}
	\underset{\boldsymbol{\beta}, f}{\text{minimize}}\sum_{j=1}^{l}\big(h_j- {\mathbf w}_j^{\mathsf T} \boldsymbol{\beta} - f(\mathbf p_j) \big)^2 + \lambda\int_\Omega(\nabla^2 f)^2\ud\mathbf p,
\end{equation}
where $f$ is assumed to be twice-differentiable over $\Omega$, and
$\nabla^2 f= \frac{\partial^2 f}{\partial x^2}+\frac{\partial^2 f}{\partial y^2}$ denotes the \emph{Laplacian} of $f$ to smoothen out the roughness of the spatial field $f$. The tuning parameter $\lambda$ is used to trade the smoothness of $f$ off for a better approximation to data value $h_j$.


% and can be selected using some data-driven or {\it ad hoc} methods.


% $\{ \mathbf p_j, h_j\}$, where $\mathbf p_j\in\Omega$ is the position and $h_j$ is the corresponding value, the SSR assumes that $h_j$ is modeled by

%\begin{equation}\label{eq:spatial_model}
%	h_j = f(\mathbf p_j) + {\mathbf w}_j^{\mathsf T}\boldsymbol{\beta} +\epsilon_j,
%\end{equation}
%where $f$ is a spatial field that give the underlying smooth surface of  $f(\mathbf p_j)$, and $\mathbf w_j$ represents the attributes at position $p_j$. Furthermore, $f$ is assumed to be twice-differentiable over $\Omega$. 
%The smoothness of $f$ enables a key assumption that many spatial data studies hinge upon, that is, the points close to each other are likely to have similar spatial field values.

%Both the spatial field $f$ and the attribute coefficients $\boldsymbol{\beta}$ can be learned based on $l$ training data points in $\Omega$, which contain the following information: \emph{1)} the values of these $l$ points: $\{h_j\}_{j=1}^l$, \emph{2)} their positions $\{\mathbf p_j\}_{j=1}^l$, and \emph{3)} their attribute vectors $\{\mathbf w_j}_{j=1}^l$.

%Once $f$ and $\boldsymbol{\beta}$ are learned from the training data, the value $z_j$ can be estimated by plugging its attribute information ${\mathbf w}_j$ and position $\mathbf p_j$ into \eqref{eq:spatial_model}.
%The model \eqref{eq:spatial_model} can be trained using either kernel-based methods \cite{Clapp04, Chopra07, Caplin08} or finite element analysis \cite{Sanga13, wood2003thin, Ramsay02}.

% However, the challenge to solving problem \eqref{eq:minSSR} is that it involves searching for a functional $f$ over a possibly non-convex domain $\Omega$ that may have strong concavities, complicated boundaries, and even interior holes.
% Although kernel-based methods \cite{Chopra07} are also a commonly used smoothing technique, their major drawback is that by using uniformly damping weights in distance-based kernels, they tend to link data points across unrelated or weakly related subregions in an irregularly shaped non-convex domain.

We now briefly describe how spatial spline regression \cite{Sanga13} can solve problem \eqref{eq:minSSR} via \emph{finite element analysis} for any irregularly shaped domain $\Omega$. 
% For details, interested readers are referred to \cite{Sanga13}.
In SSR, the domain $\Omega$ is divided into small disjoint triangles, which can be done for example by the means of Delaunay triangulation \cite{hje06}. Then a polynomial function is defined on each of these triangles, such that the summation of these polynomial functions defined on different pieces closely approximates the desired spatial field $f$. It is shown in \cite{Sanga13} that the best approximation is achieved by simply solving a set of linear equations (see \cite{Sanga13} for more details).

%Specifically, let $\zeta_1,\ldots,\zeta_K$ denote the vertices of all the small triangles, which are called control points and can be adaptively selected by available data points. Define a piecewise linear or quadratic basis function $\psi_k(x,y)$ called {\it Lagrangian finite element} with $(x,y)\in\Omega$, associated with each control point $\zeta_k$ such that $\psi_k$ evaluates to $1$ at $\zeta_k$ and is equal to $0$ at all other control points. Therefore, according to the {\it Lagrangian property of the basis} we can approximate $f(x,y)$ for any $(x,y)\in\Omega$ only using the values of $f$ on the $K$ control points, i.e., $\mathbf f_K := (f(\zeta_1),\ldots,f(\zeta_K))^{\mathsf T}$. That is, if we let $\psi(x,y) :=(\psi_1(x,y),\ldots,\psi_K(x,y))^{\mathsf T}$ denote the $K$ predefined basis functions, each corresponding to a control point, then we have
%\begin{equation}\label{eq:triApprox}
%	f(x,y) = \mbox{$\sum_{k=1}^{K}$} f(\zeta_k)\psi_k(x,y) = \mathbf f_K^{\mathsf T}\mathbf\psi(x,y),
%\end{equation}  
%Since $\psi_1(x,y),\ldots,\psi_K(x,y)$ are predefined and known \emph{a priori}, the variational estimation of $f$ in problem \eqref{eq:minSSR} boils down to the estimation of only $K$ scalar values, i.e., $\mathbf f_K = (f(\zeta_1),\ldots,f(\zeta_K))^{\mathsf T}$.

%In fact, it is shown in \cite{Sanga13} that with the piece-wise approximation given by \eqref{eq:triApprox}, solving \eqref{eq:minSSR} is simply solving a set of linear equations for $\hat f(\zeta_1),\ldots,\hat f(\zeta_K)$. 
%The estimator $\hat f(x,y)$ for $f$ can then be derived from \eqref{eq:triApprox} as
%\[
%		\hat f(x,y) = \hat{\mathbf f}_K^{\mathsf T}\mathbf\psi(x,y),
%\]

%It is worth noting that commodity triangulation software for finite element analysis is readily available in many free and commercial finite element packages. For example, Delaunay triangulations of a set of data location points (e.g., \cite{hje06}) $V$ are such that no point in $V$ is inside the circumcircle of any triangle; they maximize the minimum angle of all the triangle angles, avoiding stretched triangles. 

Now we can see that if $l=n$ and we plug $h_j = \bar f(\mathbf p_j)$, $j=1,\ldots,n$ into problem \eqref{eq:minSSR}, we will get a new density surface $\hat f$ as a solution to the SSR problem \eqref{eq:minSSR} that is a smoothened approximation of the patched estimates $\bar f(\mathbf p_j)$.

However, SSR given by \eqref{eq:minSSR}
% has a major drawback in our particular case, that is, it 
can not accommodate any constraints, and especially, does not enforce the aggregated volume constraint \eqref{eq:BSvolumes}, or equivalently, the constraint $\mathbf z = \mathbf A\mathbf f$ in  \eqref{eq:prob0}. Therefore, if we smoothen the patched estimates $\bar f(\mathbf p_j)$ out to get a smooth surface estimate $\hat f$, there is no guarantee that the estimated densities in each patch $\Omega_{B_i}$ will sum up to the observed volume $z_i$ on the point $B_i$. Violating this constraint would likely cause large density estimation errors.
\section{The \MakeLowercase{i}W\MakeLowercase{inr}NFL model}
\label{sec:model}

In this section we are going to present the data we used to develop our in-game probability model as well as the design details of {\method}. 

{\bf Data: }In order to perform our analysis we utilize a dataset collected from NFL's Game Center for all the regular season games between the seasons 2009 and 2016. 
We access the data using the Python {\tt nflgame} API \cite{nflgame}. 
The dataset includes detailed play-by-play information for every game that took place during these seasons. 
This information is used to obtain the state of the game that will drive the design of {\method}. 
In total, we collected information for 2,048 regular season games and a total of 338,294 snaps/plays. 

{\bf Model: }
{\method} is based on a logistic regression model that calculates the probability of the home team winning given the current status of the game as: 

\begin{equation}
\Pr(H=1| \mathbf{x})= \frac{\exp(\mathbf{\weight}^T\cdot\mathbf{x})}{1+\exp(\mathbf{\weight}^T\cdot\mathbf{x})}
\label{eq:reg}
\end{equation}
where $H$ is the dependent random variable of our model representing whether the home team wins or not, $\mathbf{x}$ is the vector with the independent variables, while the coefficient vector $\mathbf{\weight}$ includes the weights for each independent variable and is estimated using the corresponding data.  
For a game of infinite duration a linear model could be a very good approximation.  
However, the boundary effects from the finite duration of a game create several non-linearities \cite{winston2012mathletics}.  
For this reason, we enhance our model - using the same set of features - with a Support Vector Machine classifier with radial kernel for the last three minutes of regulation.  
In order to obtain a probability output from the SVM classifier, we further use Platt's scaling \cite{platt1999probabilistic}: 

\begin{equation}
\Pr(H=1| \mathbf{x})= \frac{1}{1+\exp{(Af(x)+B)}}
\label{eq:platt}
\end{equation}
where $f(x)$ is the uncalibrated value produced by the SVM classifier: 

\begin{equation}
f(x) = \sum_{i} (\alpha_i y_i k(\mathbf{x}_i\cdot\mathbf{x}))+ b
\label{eq:svm}
\end{equation}
where $k(\mathbf{x},\mathbf{x}')$ is the kernel used for the SVM.   
Figure \ref{fig:iwinrNFL} depicts the simple flow chart of {\method}. 


\begin{figure}[t]
\begin{center}
\includegraphics[scale=0.35]{plots/iwinrNFL.pdf}%\vspacecap
 \caption{{\method} includes a linear and a non-linear component.}
 \label{fig:iwinrNFL}
\end{center}
\end{figure}

In order to describe the status of the game we use the following variables:

\begin{enumerate}
\item {\bf Ball Possession Team:} This binary feature captures whether the home or the visiting team has the ball possession
\item {\bf Score Differential:} This feature captures the current score differential (home - visiting)
\item {\bf Timeouts Remaining:} This feature is represented by two independent variables - one for the home and one for the away team - and they capture the number of timeouts remaining for each of the teams
%\item {\bf Quarter:} This feature captures the current quarter of the game
%\item {\bf Time Remaining:} This feature captures the time (in seconds) remaining for the current quarter to end
\item {\bf Time Elapsed: } This feature captures the time elapsed since the beginning of the game
\item {\bf Down:} This feature represents the down of the team in possession
\item {\bf Field Position:} This feature captures the distance covered by the team in possession from their own yard line
\item {\bf Yards-to-go:} This variables represents the number of yards needed for a first down
\item {\bf Ball Possession Time: } This variable captures the time that the offensive unit of the home team is on the field 
\item {\bf Ranking Differential: } This variable represents the difference of the win percentage for the two team (home - visiting)
\end{enumerate}

The last independent variable is representative of the power ranking difference between the two teams. 
Most of the existing models that include such a variable are using the Vegas line spread for each game.  
We choose not to do so for the following reason.  
The objective of the Vegas line is not to predict game outcomes but rather distribute money across the different bets.  
Exactly because of this objective the line is changing during the week before the game.  
While this line can change due to new information for the competing teams (e.g., injury updates), the line is mainly changing when a particular team has accumulated the majority of the bets. 
In this case it will also be hard to choose which line to use (e.g., the opening, the closing or some average of them).  
Therefore, we choose to use the win percentage differential of the two teams as an indicator of their strength (even though this has its own issues given the uneven schedule in NFL).  
However, note that if one would like to use the point spread as a variable this can be easily incorporated in the model. 
Table \ref{tab:iwinrnfl} presents the coefficients of the logistic regression model of {\method} with standardized independent variables for better comparisons. 


\begin{table}[ht]
\begin{center}
\def\sym#1{\ifmmode^{#1}\else\(^{#1}\)\fi}
\begin{tabular}{l*{1}{c}}
\toprule
                    &\multicolumn{1}{c}{(1)}\\
                    &\multicolumn{1}{c}{Winner}\\
\midrule
Possession Team (H)         &      0.41\sym{***}\\
                    &     (49.19)         \\
\addlinespace
Score Differential           &      3.59\sym{***}\\
                    &    (247.34)         \\
\addlinespace
Home Timeouts           &     0.12\sym{***}\\
                    &      (8.74)         \\
\addlinespace
Away Timeouts           &     -0.11\sym{***}\\
                    &    (-12.47)         \\
\addlinespace
Ball Possession Time  &     -0.05.\\
                    &    (-1.66)         \\
\addlinespace
Time Lapsed       &   -0.05.\\
                    &      (-1.66)         \\
\addlinespace
Down                &   -0.01         \\
                    &      (0.04)         \\
\addlinespace
Field Position            &   0.02\sym{**} \\
                    &      (2.71)         \\
\addlinespace
Yards-to-go                &  -0.01         \\
                    &      (0.23)         \\
\addlinespace
Rating differential         &       0.75\sym{***}\\
                    &     (80.47)         \\
\addlinespace
Intercept            &       0.57\sym{*}\\
                    &    (2.09)         \\
\midrule
Observations        &      338,294         \\
\bottomrule
\multicolumn{2}{l}{\footnotesize \textit{t} statistics in parentheses}\\
\multicolumn{2}{l}{\footnotesize \sym{$_.$} \(p<0.1\), \sym{*} \(p<0.05\), \sym{**} \(p<0.01\), \sym{***} \(p<0.001\)}\\
\end{tabular}
\end{center}
\caption{Standardized logisitic regression coefficients for {\method}.}
\label{tab:iwinrnfl}
\end{table}


As we can see, as one might have expected the current scoring differential exhibits the strongest correlation with the in-game win probability.  
The only factors that do not appear to be statistically significant predictors of the dependent variable are the down and the yards-to-go. 
Even though the corresponding coefficients are negative as one might have expected (e.g., being at an earlier down gives you more chances to advance the ball), they are not significant in estimating the win probability. 
On the contrary, all else being equal timeouts appear to be quiet important since they can help a team stop the clock, while teams with better win percentage appear to have an advantage as well, since this can be a sign of a better team. 
In the following section we provide a detailed evaluation of {\method}.
%!TEX root = main.tex
\section{Performance Evaluation}
\label{sec:simu}

% compare CDFs
\begin{figure*}[!htb]
    \centering
            \subfigure[November, $n_{\text{BS}} = 200$]{
                \includegraphics[width=1.7in]{figures/CDF_200_Nov}
                    \label{fig:CDF200Nov}
            }
            \hspace{-3mm}
            \subfigure[December, $n_{\text{BS}} = 200$]{
                \includegraphics[width=1.7in]{figures/CDF_200_Dec}
                    \label{fig:CDF200Dec}
            }
            \hspace{-3mm}
            \subfigure[November, $n_{\text{BS}} = 100$]{
                \includegraphics[width=1.7in]{figures/CDF_100_Nov}
                    \label{fig:CDF100Nov}
            }
            \hspace{-3mm}
            \subfigure[December, $n_{\text{BS}} = 100$]{
                \includegraphics[width=1.7in]{figures/CDF_100_Dec}
                    \label{fig:CDF100Dec}
            }
            \vspace{-3mm}
    \caption{The comparison of the CDFs of relative errors given by different estimation methods when $n_{\text{BS}} = 200$ and $n_{\text{BS}} = 100$ for stress-testing. The legends follow the same order as the curves at relative error $= 0.5$.}
    \label{fig:compareCDF100}
\vspace{-3mm}
\end{figure*}

% % % compare bar plot. 4 in one line.
% % \begin{figure*}[t]
%     \centering
%             \subfigure[November, $n_{\text{BS}} = 200$]{
%                 \includegraphics[width=3.2in]{figures/barplot_200_Nov}
%                     \label{fig:barplot200Nov}
%             }
%             \hspace{-3mm}
%             \subfigure[December, $n_{\text{BS}} = 200$]{
%                 \includegraphics[width=3.2in]{figures/barplot_200_Dec}
%                     \label{fig:barplot200Dec}
%             }
%             \vspace{0mm}
%             \subfigure[November, $n_{\text{BS}} = 100$]{
%                 \includegraphics[width=3.2in]{figures/barplot_100_Nov}
%                     \label{fig:barplot100Nov}
%             }
%             \hspace{-3mm}
%             \subfigure[December, $n_{\text{BS}} = 100$]{
%                 \includegraphics[width=3.2in]{figures/barplot_100_Dec}
%                     \label{fig:barplot100Dec}
%             }
%             \vspace{0mm}
%     \caption{Comparison of the estimation's Mean Relative Error of different methods when $n_{\text{BS}} = 200$ or $n_{\text{BS}} = 100$ for stress-testing. In each figure, the bars from left to right stands for Patched Estimation, Patched Estimation + SSR 1, Patched Estimation + SSR 2, Constrained Spatial Smoothing, and Constrained Spatial Smoothing + Features respectively.}
%     \label{fig:compareMRE}
% \vspace{0mm}
% \end{figure*}

% compare bar plot. 4 in one line.
\begin{figure*}[t]
    \centering
            \subfigure[November, $n_{\text{BS}} = 200$]{
                \includegraphics[width=1.7in]{figures/barplot_200_Nov}
                    \label{fig:barplot200Nov}
            }
            \hspace{-3mm}
            \subfigure[December, $n_{\text{BS}} = 200$]{
                \includegraphics[width=1.7in]{figures/barplot_200_Dec}
                    \label{fig:barplot200Dec}
            }
            \hspace{-3mm}
            \subfigure[November, $n_{\text{BS}} = 100$]{
                \includegraphics[width=1.7in]{figures/barplot_100_Nov}
                    \label{fig:barplot100Nov}
            }
            \hspace{-3mm}
            \subfigure[December, $n_{\text{BS}} = 100$]{
                \includegraphics[width=1.7in]{figures/barplot_100_Dec}
                    \label{fig:barplot100Dec}
            }
            \vspace{-3mm}
    \caption{The comparison of the Mean Relative Error of different estimation methods when $n_{\text{BS}} = 200$ or $n_{\text{BS}} = 100$ for stress-testing. In each figure, the bars from left to right represent Patched Estimation, Patched Estimation + SSR 1, Patched Estimation + SSR 2, Constrained Spatial Smoothing, and Constrained Spatial Smoothing + Features, respectively.}
    \label{fig:compareMRE}
\vspace{-6mm}
\end{figure*}


% % compare bar plots
% \begin{figure*}[t]
%     \centering
%             \subfigure[November]{
%                 \includegraphics[width=3.2in]{figures/barplot_200_Nov}
%                     \label{fig:barplot200Nov}
%             }
%             \hspace{-3mm}
%             \subfigure[December]{
%                 \includegraphics[width=3.2in]{figures/barplot_200_Dec}
%                     \label{fig:barplot200Dec}
%             }
%             \vspace{-3mm}
%     \caption{Comparison of the estimation's Mean Relative Error of different methods when $n_{\text{BS}} = 200$.}
%     \label{fig:compareMRE200}
% \vspace{1mm}
% %\end{figure*}

% %\begin{figure*}[t]
%     \centering
%             \subfigure[November]{
%                 \includegraphics[width=3.2in]{figures/barplot_100_Nov}
%                     \label{fig:barplot100Nov}
%             }
%             \hspace{-3mm}
%             \subfigure[December]{
%                 \includegraphics[width=3.2in]{figures/barplot_100_Dec}
%                     \label{fig:barplot100Dec}
%             }
%             \vspace{-3mm}
%     \caption{Comparison of the estimation's Mean Relative Error of different methods when $n_{\text{BS}} = 100$ for stress-testing.}
%     \label{fig:compareMRE100}
% \vspace{-5mm}
% \end{figure*}




% In this section, we perform an extensive case study of the approach we described above in order to demonstrate its applicability. 
The model in \eqref{eq:add-auxiliary} is not attached to any particular empirical problem and does not contain many implicit assumptions, it is general. However, in order to measure its performance we evaluate the model using real-world cell phone data.
% We picked the cell phone data as an example of how the model can solve empirical problem and compare the model's performance to other approaches.

% \subsection{Dataset Description}
% \label{sec:activity-recovery}

% The model in \eqref{eq:add-auxiliary} is not attached to any particular empirical problem and does not contain many implicit assumptions, it is general. However, in order to measure its performance we evaluate the model using real-world data. Due to generality of the proposed learning algorithm the range of possible data sets is potentially big. For our empirical case study we chose cell phone data, where there exists a problem of recovering a spatial field from coarse aggregations observed at sparse cell phone towers. We do not overestimate the problem, but rather see this particular data set suitable for extensive case study.
%To give a more intuitive idea about our problem, here we %introduce the datasets we utilized, and describe how we %process the data to study the problem of inferring cell %phone activities spatial distribution.

The Milan Call Description Records (CDR) dataset
% is a part of the Telecom Italia Big Data Challenge dataset provided by Telecom Italia Mobile.
% It
contains the telecommunications activity records from November 1$st$, 2013 to December 31$th$, 2013 in the city of Milan~\cite{bigdatachallenge}. In the Milan CDR dataset, the city of Milan is divided into a $100\times 100$ square grid. Each square is size of about 235m $\times$ 235m. Each activity record consists of the following entries: square ID, time-stamp of 10-minute time slot, incoming SMS activity, outgoing SMS activity, incoming call activity and outgoing call activity. The values of the four types of activities are normalized to the same scale.



% % grid
% \begin{figure}[t]
%         %\hspace{7mm}
%         \includegraphics[width=3.3in]{figures/grid}
%         %\vspace{-5mm}
%         \caption{The map shows the metropolitan area of Milan, Italy, and the area covered by the 2726 grid squares.}
%         \label{fig:grid}
%         \vspace{-3mm}
% \end{figure}


Another dataset we utilized is the Milan geographical attribute dataset available from the Municipality of Milan's Open Data website \cite{barlacchi2015multi}. This dataset consists of features of central 2726 squares among the whole $10,000$ squares. The features of each square include: population, green area percentage, number of sport centers, number of universities, number of businesses, and number of bus stops.
% Fig.~\ref{fig:grid} shows the map of Milan and the area covered by these grid squares. The 2726 squares covers the central part of the Milan city and contains the majority of telecommunication activities in the dataset.
% We refer to~\cite{bcici_mobihoc15} for more detailed description about this dataset.
In our empirical study, we focus on these squares to compare the performance of different algorithms.

% % value distribution
% \begin{figure}[t]
%                         \centering
%                         \subfigure[November]{
%                 \includegraphics[width=1.5in]{figures/heatmap_Nov_call_sms}
%                                 \label{fig:heatmapNov}
%                         }
%                 \hspace{-4mm}
%                         \subfigure[December]{
%                 \includegraphics[width=1.5in]{figures/heatmap_Dec_call_sms}
%                                 \label{fig:heatmapDec}
%                         }
%                         \vspace{-1mm}
%                 \caption{The heat map of call + sms activities during November and December.}
%                 \label{fig:NovDecHeatmap}
% \vspace{-5mm}
% \end{figure}



The general \textbf{\textit{problem of recovering a spatial field from coarse aggregations observed at sparse points in
the field}} in this particular case study is reformulated into \textbf{\textit{the problem of recovering the distribution of cell phone activities over the whole 2726 square regions given that only aggregated activity observations in base stations are known}}. To study this problem, we need to further process the Milan CDR dataset.

\textit{First}, we sum up the four types of activities during November and December respectively to come up with the activity volume of each square during the two months. These two datasets are served as the ground-truth datasets of Milan cell phone activity distributions.
% Fig.~\ref{fig:heatmapNov} and Fig.~\ref{fig:heatmapDec} show the heat maps of activity volumes in each square during November and December. 
\textit{Second}, after we aggregated the two months' activities for each square, we need to set the locations of base stations (BSs). According to \cite{ratti2006mobile}, there are roughly $200$ base stations in Milan. However, the exact locations are not available. Thus, we assume the $n_{\text{BS}}$ ($n_{\text{BS}} = 200$ or $100$ for stress-test) BSs are randomly distributed according to the probability distribution
$
\Pr (\text{Set square $i$ as BS}) = f(\mathbf p_i) / \sum_{j=1}^{N}f(\mathbf p_j),
$
where $f(\mathbf p_i)$ is the cell phone activity volume in square $i$, $i=\{1, \ldots, N\}$, $N=2726$ is the number of squares we are focusing on. 
% Notice that when we have 200 base station's aggregated observations, they only cover $7.34 \%$ of the whole 2726 squares region. This is extremely sparse and makes our problem highly challenging.
% In addition, we also assume $n_{\text{BS}} = 100$ and choose $100$ squares as BSs according to the same probability distribution to stress-test our algorithm's capability under even sparser observations.
\textit{Third}, after the base station locations are sampled, the activity of each square will be assigned to its closest base station. If multiple base stations are equidistant from the square, then the activity of this square will be evenly distributed among these base stations. We then assume we only know the aggregated activities in base station squares, which is usually the true case in reality.
Fig.~\ref{fig:100BS} and Fig.~\ref{fig:100BSincharge} show the base station distributions and the region charged by each base station for $n_{\text{BS}} = 100$ respectively.
% To save space, we don't present the figure for 200 base stations. 

% % sample BS location
% \begin{figure}[t]
%                         \centering
%                         \subfigure[$n_{\text{BS}} = 200$]{
%                 \includegraphics[width=1.5in]{figures/heatmap_200BS}
%                                 \label{fig:200BS}
%                         }
%                 \hspace{-4mm}
%                         \subfigure[$n_{\text{BS}} = 100$]{
%                 \includegraphics[width=1.5in]{figures/heatmap_100BS}
%                                 \label{fig:100BS}
%                         }
%                         \vspace{-1mm}
%                 \caption{The sampled base station distributions for $n_{\text{BS}} = 200$ and $n_{\text{BS}} = 100$.}
%                 \label{fig:BSLocations}
% \vspace{-5mm}
% \end{figure}







% \subsection{Experimental Setup}
% \subsubsection{\bf Algorithms Evaluated}
We test our proposed approach and compare it with 3 baseline methods.
% In particular, we evaluate and compare the following models using the aggregated November and December datasets, with number of base stations $n_{\text{BS}} = 200$ or $n_{\text{BS}} = 100$ for stress testing.
\begin{itemize}
\item \textbf{Patched Estimation (PE)}:
% estimate the cell phone activity distribution by patched piece-wise constant estimation, that is, 
assume cell phone activity density is distributed uniformly within each sub-region $\Omega_{B_i}$
% , i.e., the area covered by base station $B_i$,
and estimate each square's activity volume by \eqref{eq:patched}.
\item \textbf{Patched Estimation + SSR 1}: first estimate \textit{only base station} activity volumes by \eqref{eq:patched}. Use these sparse points to fit a smooth surface by running Spatial Spline Regression to obtain the estimated cell phone activity in all squares. 
\item \textbf{Patched Estimation + SSR 2}: first estimate the activity volumes of \textit{all squares} by Patched Estimation. Then use all these points to fit a smooth surface by running Spatial Spline Regression to obtain the final estimated cell phone activity in all squares.
\item \textbf{Constrained Spatial Smoothing (CSS)}: first get the initial estimation of the activity volumes of all squares by Patched Estimation, then run Constrained Spatial Smoothing algorithm to get the final activity volumes estimation of all squares.
\item \textbf{Constrained Spatial Smoothing + Features}: in this case, we incorporate the geographical features into the Constrained Spatial Smoothing algorithm.
\end{itemize}

We set the penalty parameter $\lambda = 1$ when $n_{\text{BS}} = 200$ and $\lambda = 10$ when $n_{\text{BS}} = 100$, for all methods that utilize SSR. The geographical features of Milan are only incorporated in the last algorithm described above.
% Besides, for the implementation of Spatial Spline Regression, we use the \emph{fdaPDE} R Package~\cite{fdaPDE}.
We evaluate the performance by the Mean Relative Error (MRE) of the produced activity estimates for the true activity values. 
% The relative error of an estimation $\hat{f}(\mathbf{p}_j)$ compared to the true value $f(\mathbf{p}_j)$ is defined as $|\hat{f}(\mathbf{p}_j) - f(\mathbf{p}_j)| / f(\mathbf{p}_j)$.

\subsection{Performance Evaluation}
\subsubsection{\bf Comparison of Different Algorithms}


We show the cumulative distribution function (CDF) of Relative Errors given by each approach in Fig.~\ref{fig:compareCDF100}. In addition, we compare the estimation's Mean Relative Errors of different approaches in Fig.~\ref{fig:compareMRE}. It is quite clear that our proposed algorithms outperform other three baseline approaches significantly in all the cases ($n_{\text{BS}} = 200$ and $n_{\text{BS}}=100$, data aggregated in November and in December). 

By comparing Patched Estimation + SSR 1 with Patched Estimation approach, we can see that using spatial smoothing based on only base station squares' observations leads to worse performance than patched estimation. This can be explained by the smoothing property of SSR and the way we set the values of base station squares. As we described, we set the activity values of base stations by averaging the total activity amount of each base station on all the squares it covers. Thus, given the activity $\frac{z_i}{|\Omega_{B_i}|}$ ($|\Omega_{B_i}|$ denotes the number of squares within region $\Omega_{B_i}$) of a base station $B_i$, the true activities of itself and its surrounding squares within region $B_i$ are distributed with a mean of $\frac{z_i}{|\Omega_{B_i}|}$. Given two base stations $B_1$ and $B_2$ that are close to each other, with aggregated activities of $z_1$ and $z_2$ respectively, the Spatial Smoothing approach will fit a smooth surface between the two base stations. Suppose $z_1 > z_2$, in this case, in overall the activities of $B_1$'s neighbour squares will be under estimated, and that of $B_2$ will be over estimated. Therefore, Patched Estimation + SSR 1's performance is not as good as Patched Estimation.

By comparing Patched Estimation + SSR 2 with Patched Estimation and Patched Estimation + SSR 1, we can observe that applying spatial smoothing on the results of patched estimation improves the performance. This proves the rationality and effectiveness of introducing smoothness into the estimated cell phone activity distribution surface.

Our proposed approaches achieves much better performance compared with the three baseline methods. By using Constrained Spatial Smoothing instead of applying Spatial Spline Regression directly, we are able to fit a smooth activity distribution while forcing it to match the observations of base station squares (the aggregated activity volumes) at the same time. By comparing Constrained Spatial Smoothing that incorporates additional features of each square with the version without features, we can see that the performance is further improved. The reason is that the heterogeneity of different locations will influence the telecommunication activity distribution, therefore making the distribution not everywhere smooth. Incorporating additional features into our model can help to explain the residuals between estimated smooth distribution and the true activity distribution, therefore further increases estimation accuracy.
% By comparing Fig.~\ref{fig:compareCDF200} and Fig.~\ref{fig:compareCDF100}, we also can see that incorporating additional features into Constrained Spatial Smoothing becomes more important when the base stations are more sparse.

The performance of different methods on December dataset is worse than on November dataset. The reason is that, there are multiple holidays during December, therefore the cell phone activities will be much more irregular than usual. 

% % 3d plots
% \begin{figure*}[t]
%                         \centering
%                         \subfigure[Real distribution]{
%                 \includegraphics[width=2.3in]{figures/3d_200BS_Nov_trueval.png}
%                                 \label{fig:3Dtrueval.png}
%                         }
%                 \hspace{-4mm}
%                         \subfigure[Estimation of Patched Estimation]{
%                 \includegraphics[width=2.3in]{figures/3d_200BS_Nov_baseline1.png}
%                                 \label{fig:3Dbaseline1.png}
%                         }
%                         \hspace{-4mm}
%                         \subfigure[Estimation of Constrained Spatial Smoothing + Features]{
%                 \includegraphics[width=2.3in]{figures/3d_200BS_Nov_SsrAdmm.png}
%                                 \label{fig:3DSsrAdmm.png}
%                         }
%                         \vspace{0mm}
%                 \caption{The true telecommunication activity distribution on November and the fitted surface of Patched Estimation and our method SSR + ADMM when $n_{\text{BS}} = 200$.}
%                 \label{fig:compareSurface}
% \vspace{-5mm}
% \end{figure*}








% % map and heatmap
% \begin{figure*}[t]
%                         \centering
%                         \subfigure[Map of Milan]{
%                 \includegraphics[width=1.95in]{figures/grid}
%                                 \label{fig:grid}
%                         }
%                 \hspace{-4mm}
%                         \subfigure[Activity Distribution in November]{
%                 \includegraphics[width=2.5in]{figures/heatmap_Nov_call_sms}
%                                 \label{fig:heatmapNov}
%                         }
%                         \hspace{-4mm}
%                         \subfigure[Activity Distribution in December]{
%                 \includegraphics[width=2.5in]{figures/heatmap_Dec_call_sms}
%                                 \label{fig:heatmapDec}
%                         }
%                         \vspace{0mm}
%                 \caption{ (a) shows the metropolitan area of Milan, Italy, and the area covered by the 2726 grid squares. (b) and (c) show the heat map of cell phone activities (Call + SMS) during November and December.}
%                 \label{fig:compareSurface}
% \vspace{1mm}
% \end{figure*}

% sample BS location
\begin{figure}[t]
                        \centering
                %         \subfigure[$n_{\text{BS}} = 200$]{
                % \includegraphics[width=2.2in]{figures/heatmap_200BS}
                %                 \label{fig:200BS}
                %         }
                % \hspace{-0mm}
                        \subfigure[Distribution of BSs]{
                \includegraphics[width=1.3in]{figures/heatmap_100BS}
                                \label{fig:100BS}
                        }
                \hspace{-0mm}
                        \subfigure[Areas covered by each BS]{
                \includegraphics[width=1.3in]{figures/incharge_100BS}
                                \label{fig:100BSincharge}
                        }
                        \vspace{-3mm}
                \caption{(a) The geographical distribution of sampled base stations for $n_{\text{BS}} = 100$. (b) The areas that the individual base stations are responsible for, when $n_{\text{BS}} = 100$.}
                \label{fig:BSLocations}
\vspace{-6mm}
\end{figure}












% Fig.~\ref{fig:3Dtrueval.png}, Fig.~\ref{fig:3Dbaseline1.png} and Fig.~\ref{fig:3DSsrAdmm.png} show the distribution surfaces of true cell phone activity volumes, estimated volumes by Patched Estimation, and estimated volumes by Constrained Spatial Smoothing with features when $n_{\text{BS}} = 200$ using the November dataset. We can see that the Patched Estimation approach fits a stepped surface, while our approach gives a much smoother surface.





% \subsubsection{\bf Impact of Smooth Penalty Parameter $\lambda$}

% % influence of lambda
% \begin{figure}[t]
%         %\hspace{7mm}
%         \includegraphics[width=3.4in]{figures/lambda}
%         %\vspace{-5mm}
%         \caption{Influence of $\lambda$ to estimation's Mean Relative Error when $n_{\text{BS}} = 200$ and $n_{\text{BS}}=100$ for stress-testing. The figure is based on the November dataset. Result on the December dataset is similar.}
%         \label{fig:lambda}
%         \vspace{1.5mm}
% \end{figure}

% Fig.~\ref{fig:lambda} shows how the the estimation's Mean Relative Error varies when $\lambda$ increases from $10^{-4}$ to $10^3$. We make two interesting observations. First, $\lambda$ around $1 \sim 10$ usually gives the best performance. Too big or too small $\lambda$ will decrease the estimation accuracy. This is reasonable, as when $\lambda$ is too small, we put little emphasis on the smoothness of estimated surface, thus the performance will suffer. If $\lambda$ is too big, it enforces a smooth surface, which also doesn't match the reality. 
% Second, when we have less base stations, $\lambda$ that gives the best performance will increase (from 1 to 10). Besides, we can see that the performance of the model with $\lambda$ between $1 \sim 100$ does not significantly change when $n_{\text{BS}} = 100$. That indicates the following: when the base station distribution is more sparse, the estimation performance is less sensitive to $\lambda$ when it is around the best value ($1$ for $n_{\text{BS}} = 200$ and $10$ for $n_{\text{BS}} = 100$).  
















\textbf{Related work}:
% Object detection related datasets/algo in non-medical domain
% Locally labeled CXR dataset
A few CXR datasets have localized abnormality annotations \cite{shih2019augmenting,filice2020crowdsourcing,jaeger2014two} that are curated manually. These are high quality gold standard ground truth datasets but tend to be smaller in scale (< 30,000 images) and have a narrow coverage, with typically only 1-2 labels. In addition, since most labeling efforts only have abnormality semantics attached, no direct relationships with the affected anatomical locations are available. 

%MEHDI: repeated concepts from above. I am removing the following: 

%The lack of anatomic semantics in the annotation is a limitation for complex multi-modal clinical reasoning work, e.g., differential diagnosis, since clinicians often integrate information along anatomical lines, and for downstream report generation tasks, which often requires describing not only the abnormality but also correctly communicate the location of the abnormalities (and medical devices) to the receiving clinicians. 

Two recent CXR datasets have labels for anatomies described in the reports. In \cite{datta2020dataset}, a small manually annotated dataset (2000 reports) included 10 abnormalities that are individually associated with 29 unique spatial locations (anatomies) at the report level. Another CXR dataset has automatically extracted abnormality and anatomy labels as disconnected concepts that are only correlated at the study level from  160,000 reports using a supervised NLP algorithm \cite{bustos2020padchest}. This was trained on a smaller set of manually annotated data. Neither datasets contain localized annotations for the associated CXR images, nor any comparison relation annotations between sequential exams, both of which are available in the Chest ImaGenome dataset. In Table \ref{tab:related}, we present a comparison of our Chest ImagGenome dataset with other datasets available in the literature.

% Table -- Kashyap

% MEdical imaging datasets to go here: Discussed that we will only focus on cxr datasets that are available for this paper. 
% \caption{\color{red} Kashyap, feel free to continue with the table. We should remove the questionmarks and add a line for our dataset (since all others are not graph). For longer text, using abbreviations and explaining them in the caption often works better. If fill in the values is not possible, it is better to remove the table altogether.}


\begin{table}[t!]
\caption{Summary of existing chest X-ray datasets}
\resizebox{\textwidth}{!}{%
\begin{tabular}{@{}lllllllll@{}}
\toprule
\textbf{Dataset} & \textbf{Annotation Level} & \textbf{Annotation Method} & \textbf{Num Labels} & \textbf{Anatomy Labeled} & \textbf{Graph} & \textbf{Dataset Size} & \textbf{Temporal Labels} & \textbf{Reports} \\ \midrule
SIIM-ACR Pneumothorax Segmentation \cite{filice2020crowdsourcing} & Segmentation & Manual + augmented & 1 & No & No & 12,047 & No & No \\
RSNA Pneumonia Detection Challenge   \cite{shih2019augmenting} & Bounding Boxes & Manual & 1 & No & No & 30,000 & No & No \\
Indiana University Chest X-ray collection \cite{demner2016preparing} & Global & Automated & 10 & No & No & 3,813 & No & Yes \\
NIH CXR dataset \cite{wang2017chestx} & Global & Automated & 14 & No & No & 112,120 & No & No \\
PLCO \cite{team2000prostate} & Global & Automated & 24 & Yes & No & 236,000 & Yes & No \\
Stanford CheXpert \cite{irvin2019chexpert} & Global & Automated & 14 & No & No & 224,316 & No & No \\
MIMIC-CXR \cite{johnson2019mimic} & Global & Automated & 14 & No & No & 377,110 & No & Yes \\
Dutta \cite{datta2020dataset} & Global & Manual & 10 & Yes & Yes & 2,000 & No & Yes \\
PadChest \cite{bustos2020padchest} & Global & Manual + automated & 297 & Yes & No & 160,868 & No & Yes \\
Montgomery County Chest X-ray   \cite{jaeger2014two} & Segmentation & Manual & 1 & Yes & No & 138 & No & No \\
Shenzen Hospital Chest X-ray   \cite{jaeger2014two} & Segmentation & Manual & 1 & Yes & No & 662 & No & No \\  \hline \hline
\textbf{Chest ImaGenome} & Bounding Boxes & Automated & 131 & Yes & Yes & 242,072 & Yes & Yes \\
\bottomrule
\end{tabular}%
}
\label{tab:related}
\vspace{-0.4cm}
\end{table}
% removed (Derived from MIMIC-CXR \cite{johnson2019mimic}) % makes table really small

%!TEX root = main.tex
\section{Concluding Remarks}
\label{sec:conclude}

In this paper, we study the problem of inferring the fine-grained spatial distribution of certain density data in a region based on the aggregate observations recorded for each of its subregions.
% , which is extremely challenging and seldom visited before, and analyze the challenges of it.
We propose the Constrained Spatial Smoothing (CSS) approach that exploits both the intrinsic smooth property of underlying factors and the additional features from external social or domestic statics. We further propose a training algorithm which combines the Spatial Spline Regression (SSR) technique and ADMM technique to learn our model parameters efficiently. To evaluate our algorithm and compare it with various other approaches, we run extensive evaluation based on the Milan Call Detail Records dataset provided by Telecom Italia Mobile. The simulation results on the dataset show that our algorithm significantly outperforms other baseline approaches by a great percentage. 

% Although we use the data on cell phone activities to illustrate our methodology, our algorithm is not limited to solving the problem of inferring the distribution of cell phone activities, but also applicable to a variety of problems where estimating an implicit or explicit smooth surface is required, such as %population density estimation, land desirability estimation, human activity pattern modeling and so on.
% inferring the spatial distribution of population densities based on the aggregate population observed at sparsely scattered polling stations, reconstructing a fine-grained geographical distribution of users for an Internet media provider or retailer only from aggregated user counts observed at certain datacenters or points of presence (PoPs), and so on. 




\bibliographystyle{IEEEtran}
\bibliography{main}

%\onecolumn


% \tableofcontents{}

% \newpage

\section*{Supplementary Material}
\addcontentsline{toc}{section}{Supplementary Material}


Throughout this discussion, 
we will make frequently use 
of the following standard results
concerning the exponential concentration 
of random variables:

\begin{lemma}[Hoeffding's inequality for independent RVs~\citep{hoeffding1994probability}] Let $Z_1, Z_2, \ldots, Z_n$ be independent bounded random variables with $Z_i \in [a,b]$ for all $i$, then 
    \begin{align*}
        \prob\left( \frac{1}{n} \sum_{i=1}^n (Z_i - \Expo{Z_i}) \ge t \right) \le \exp{\left( -\frac{2nt^2}{(b-a)^2} \right) }
    \end{align*} 
    and 
    \begin{align*}
        \prob\left( \frac{1}{n} \sum_{i=1}^n (Z_i - \Expo{Z_i}) \le -t \right) \le \exp{\left( -\frac{2nt^2}{(b-a)^2} \right) }
    \end{align*} 
    for all $t \ge 0$. 
\end{lemma}

\begin{lemma}[Hoeffding's inequality for sampling with replacement~\citep{hoeffding1994probability}] \label{lem:hoeffding_sampling} Let $\calZ = (Z_1, Z_2, \ldots, Z_N)$ be a finite population of $N$ points with $Z_i \in [a.b]$ for all $i$. Let $X_1, X_2, \ldots X_n$ be a random sample drawn without replacement from $\calZ$. Then for all $t \ge 0$, we have 
    \begin{align*}
        \prob\left( \frac{1}{n} \sum_{i=1}^n (X_i - \mu ) \ge t \right) \le \exp{\left( -\frac{2nt^2}{(b-a)^2} \right) }
    \end{align*} 
    and 
    \begin{align*}
        \prob\left( \frac{1}{n} \sum_{i=1}^n (X_i - \mu ) \le -t \right) \le \exp{\left( -\frac{2nt^2}{(b-a)^2} \right) } \,,
    \end{align*} 
    where $\mu = \frac{1}{N} \sum_{i=1}^{N} Z_i$. 
\end{lemma}

We now discuss one condition that generalizes the exponential concentration to dependent random variables.
\begin{condition}[Bounded difference inequality] \label{cond:BDC} Let $\calZ$ be some set and $\phi: \calZ^n \to \Real$. We say that $\phi$ satisfies the bounded difference assumption if 
there exists $c_1, c_2, \ldots c_n \ge 0$ s.t. for all $i$, we have 
\begin{align*}
    \sup_{Z_1,Z_2, \ldots,Z_n, Z_i^\prime \in \calZ^{n+1} } \abs{\phi (Z_1, \ldots, Z_i, \ldots, Z_n ) - \phi (Z_1, \ldots, Z_i^\prime, \ldots, Z_n ) } \le c_i \,.
\end{align*} 
\end{condition}

\begin{lemma}[McDiarmid’s inequality~\citep{mcdiarmid1989}] \label{lem:McDiarmid} Let $Z_1, Z_2, \ldots, Z_n$ be independent random variables on set $\calZ$ and $\phi : \calZ^n \to \Real$ satisfy bounded difference inequality (\codref{cond:BDC}). Then for all $t>0$, we have 
    \begin{align*}
        \prob\left( \phi(Z_1, Z_2, \ldots, Z_n) - \Expo{\phi(Z_1, Z_2, \ldots, Z_n)} \ge t \right) \le \exp{\left( -\frac{2t^2}{\sum_{i=1}^n c_i^2} \right) } 
    \end{align*} 
    and 
    \begin{align*}
        \prob\left( \phi(Z_1, Z_2, \ldots, Z_n) - \Expo{\phi(Z_1, Z_2, \ldots, Z_n)} \le -t \right) \le \exp{\left( -\frac{2t^2}{\sum_{i=1}^n c_i^2} \right) } \,.
    \end{align*} 
\end{lemma}


\section{Proofs from \secref{sec:ERM_training}}\label{app:proof_erm}

\textbf{Additional notation {} {}} Let $m_1$ be the number of mislabeled points ($\wt S_M$) and $m_2$ be the number of correctly labeled points ($\wt S_C$). Note $m_1 + m_2 = m$. 


\subsection{Proof of \thmref{thm:error_ERM}}


\begin{proof}[Proof of \lemref{lem:fit_mislabeled}] 
    The main idea of our proof is to regard 
    the clean portion of the data 
    ($S \cup \wt S_C$) as fixed.   
    Then, there exists an (unknown) classifier $f^*$ 
    that minimizes the expected risk
    calculated on the (fixed) clean data
    and (random draws of) the mislabeled data $\wt S_M$. 
    % 
    % 
    Formally, 
    \begin{align}
    f^* \defeq \argmin_{f \in \calF} \error_{\widecheck {\calD}} (f) \,, \label{eq:modified_ERM}
    \end{align}
    where $$\widecheck \calD = \frac{n}{m+n} \calS + \frac{m_2}{m+n} \wt \calS_C  + \frac{m_1}{m+n}\calDm \,.$$ 
    Note here that $\widecheck \calD$ is a combination 
    of the \emph{empirical distribution} 
    over correctly labeled data $S \cup \wt S_C$
    and the (population) distribution 
    over mislabeled data $\calDm$.
    Recall that 
    \begin{align}
    \wh f \defeq \argmin_{f \in \calF} \error_{\calS \cup \wt S} (f) \,. \label{eq:orig_ERM}
    \end{align}
    % 
    % 
    Since, $\widehat f$ minimizes 0-1 error 
    on $S \cup \wt S$, using ERM optimality on \eqref{eq:orig_ERM},  
    we have 
    \begin{align}
        \error_{\calS \cup \wt \calS}(\widehat f) \le \error_{
            \calS \cup \wt \calS}(f^*) \,.    \label{eq:step1}
    \end{align}
    Moreover, since $f^*$ is independent of $\wt S_M$, using Hoeffding's bound,
    % \footnote{For a fully rigorous argument,
    % refer to the complete proof in App.~\ref{app:proof_erm}.} 
    we have with probability at least $1-\delta$ that
    \begin{align}
      \error_{\wt \calS_M}(f^*) \le \error_{ \calDm}(f^*) +  \sqrt{\frac{\log(1/\delta)}{2 m_1}} \,. \label{eq:step2} 
    \end{align}
    %$ 
    %for some constant $c_1\le 1/2$. 
    Finally, since $f^*$ is the optimal classifier on $\widecheck \calD$, 
    we have 
    \begin{align}
        \error_{\widecheck \calD}(f^*) \le \error_{\widecheck \calD}(\widehat f) \,. \label{eq:step3}
    \end{align}
    Now to relate \eqref{eq:step1} and \eqref{eq:step3}, we multiply \eqref{eq:step2} by $\frac{m_1}{m+n}$ and add $\frac{n}{m+n} \error_{\calS} (f)  + \frac{m_2}{m+n} \error_{\wt \calS_C} (f)$ both the sides. Hence, 
    we can rewrite \eqref{eq:step2} as follows: 
    \begin{align}
        \error_{\calS \cup \wt\calS}(f^*) \le \error_{ \widecheck \calD}(f^*) +  \frac{m_1}{m+n}\sqrt{\frac{\log(1/\delta)}{2 m_1}} \,. \label{eq:step4} 
    \end{align}
    Now we combine equations \eqref{eq:step1}, \eqref{eq:step4}, and \eqref{eq:step3}, to get 
    \begin{align}
        \error_{\calS \cup \wt \calS}(\wh f) \le \error_{\widecheck \calD}(\wh f) +  \frac{m_1}{m+n}\sqrt{\frac{\log(1/\delta)}{2 m_1}} \,, 
    \end{align}
    which implies 
    \begin{align}
        \error_{ \wt \calS_M}(\wh f) \le \error_{\calDm}(\wh f) + \sqrt{\frac{\log(1/\delta)}{2 m_1}} \,. \label{eq:lemma1_final}
    \end{align}
    Since $\wt S$ is obtained by randomly labeling an unlabeled dataset, we assume $2m_1 \approx m$ \footnote{Formally, with probability at least $1-\delta$, we have  $(m - 2m_1)\le \sqrt{m\log(1/\delta)/2}$.}. Moreover, using $\error_{\calDm} = 1 - \error_{\calD}$ we obtain the desired result.   
    % Combining the above steps and using the fact 
    % that $\error_\calD = 1- \error_{\calDm} $, 
    % we obtain the desired result.
\end{proof}

\begin{proof}[Proof of \lemref{lem:mislabeled_error}]
    Recall $\error_{\wt S} (f) = \frac{m_1}{m} \error_{\wt S_M}(f) + \frac{m_2}{m} \error_{\wt S_C}(f)$. Hence, we have 
    \begin{align}
        2\error_{\wt S}(f) - \error_{\wt S_M}(f) - \error_{\wt S_C}(f) &= \left(\frac{2m_1}{m} \error_{\wt S_M}(f) - \error_{\wt S_M}(f)\right) + \left(\frac{2m_2}{m} \error_{\wt S_C}(f) - \error_{\wt S_C}(f)\right) \\ &= \left(\frac{2m_1}{m} - 1\right) \error_{\wt S_M}(f) + \left(\frac{2m_2}{m} - 1 \right)\error_{\wt S_C} (f) \,.
    \end{align} 
    Since the dataset is labeled uniformly at random, with probability at least $1-\delta$, we have  $\left(\frac{2m_1}{m} - 1\right) \le \sqrt{\frac{\log(1/\delta)}{2m}}$. Similarly, we have with probability at least $1-\delta$, $\left(\frac{2m_2}{m} - 1\right) \le \sqrt{\frac{\log(1/\delta)}{2m}}$. Using union bound, with probability at least $1-\delta$, we have
    % \begin{align}
    %     2\error_{\wt S} - \error_{\wt S_M}(f) - \error_{\wt S_C}(f) \le \sqrt{\frac{\log(2/\delta)}{2m}} \left(\error_{\wt S_M}(f) + \error_{\wt S_C}(f) \right) \le 2\sqrt{\frac{\log(2/\delta)}{2m}} \,. \label{eq:lemma2_final}
    % \end{align}
    \begin{align}
        2\error_{\wt S} - \error_{\wt S_M}(f) - \error_{\wt S_C}(f) \le \sqrt{\frac{\log(2/\delta)}{2m}} \left(\error_{\wt S_M}(f) + \error_{\wt S_C}(f) \right) \,. \label{eq:lemma2_prefinal}
    \end{align}
    With re-arranging $\error_{\wt S_M}(f) + \error_{\wt S_C}(f)$ and using the inequality $ 1- a\le \frac{1}{1+a} $, we have  
    \begin{align}
        2\error_{\wt S} - \error_{\wt S_M}(f) - \error_{\wt S_C}(f) \le 2\error_{\wt \calS} \sqrt{\frac{\log(2/\delta)}{2m}}  \,. \label{eq:lemma2_final}
    \end{align}

    % We obtain the desired result by using 
\end{proof}

\begin{proof}[Proof of \lemref{lem:clear_error}]
% Recall 0-1 error on each point  $(x,y) \in S \cup \wt S$ is given by $\I{ f(x)\ne y}$.
In the set of correctly labeled points $S \cup \wt S_C$, we have $S$ as a random subset of $S \cup \wt S_C$. Hence, using Hoeffding's inequality for sampling without replacement (\lemref{lem:hoeffding_sampling}), we have with probability at least $1-\delta$
\begin{align}
    \error_{\wt \calS_C} (\wh f)- \error_{\calS \cup \wt \calS_C}( \wh f) \le  \sqrt{\frac{\log(1/\delta)}{2m_2}} \,.
\end{align}
Re-writing $\error_{\calS \cup \wt \calS_C}( \wh f)$ as $\frac{m_2}{m_2 + n} \error_{\wt \calS_C }(\wh f) + \frac{n}{m_2 + n} \error_{\calS }(\wh f)$, we have with probability at least $1-\delta$
\begin{align}
   \left(\frac{n}{n+m_2}\right) \left(\error_{\wt \calS_C} (\wh f)- \error_{\calS}( \wh f) \right) \le  \sqrt{\frac{\log(1/\delta)}{2m_2}} \,.
\end{align}
As before, assuming $2m_2 \approx m$, we have with probability at least $1-\delta$ 
\begin{align}
    \error_{\wt \calS_C} (\wh f)- \error_{\calS}( \wh f) \le \left(1+\frac{m_2}{n}\right)  \sqrt{\frac{\log(1/\delta)}{m}} \le \left(1 + \frac{m}{2n}\right) \sqrt{\frac{\log(1/\delta)}{m}} \,. \label{eq:lemma3_final}
\end{align} 
\end{proof}

\begin{proof}[Proof of \thmref{thm:error_ERM}] 
    Having established these core intermediate results, we can now combine above three lemmas to prove the main result. 
    In particular, we bound the population error on clean data ($\error_\calD(\wh f)$) as follows:  
    \begin{enumerate}[(i)]
        \item First, use \eqref{eq:lemma1_final}, to obtain an upper bound on the population error on clean data, i.e., with probability at least $1-\delta/4$, we have
        \begin{align}
            \error_{ \calD} (\wh f) \le 1 - \error_{ \wt \calS_M}(\wh f) + \sqrt{\frac{\log(4/\delta)}{m}} \,. 
        \end{align}
        \item  Second, use \eqref{eq:lemma2_final}, to relate the error on the mislabeled fraction with error on clean portion of randomly labeled data and error on whole randomly labeled dataset, i.e., with probability at least $1-\delta/2$, we have 
        \begin{align}
            - \error_{\wt S_M}(f) \le \error_{\wt S_C}(f) - 2\error_{\wt S}  + 2\error_{\wt S} \sqrt{\frac{\log(4/\delta)}{2m}}  \,. 
        \end{align} 
        \item Finally, use \eqref{eq:lemma3_final} to relate the error on the clean portion of randomly labeled data and error on clean training data, i.e., with probability $1-\delta/4$, we have 
        \begin{align}
            \error_{\wt \calS_C} (\wh f)\le - \error_{\calS}( \wh f) + \left(1 + \frac{m}{2n} \right) \sqrt{\frac{\log(4/\delta)}{m}} \,. 
        \end{align} 
    \end{enumerate}

    Using union bound on the above three steps, we have with probability at least $1-\delta$: 
    \begin{align}
        \error_\calD (\wh f) \le \error_{\calS}(\wh f)   + 1 - 2\error_{\wt \calS}(\wh f)   + \left(\sqrt{2} \error_{\wt S} + 2 + \frac{m}{2n}\right)  \sqrt{\frac{\log(4/\delta)}{m}} \,.
    \end{align}
    % Note that $(1/\sqrt{2} + 2.5)$ is a loose constant. In experiments, we use the ratio $\frac{m}{n}$
    %  the exact error $\error_{\wt \calS}(\wh f)$ 
    % to evaluate R.H.S.    
\end{proof}

\subsection{Proof of \propref{prop:rademacher}}

\begin{proof}[Proof of \propref{prop:rademacher}]
    For a classifier $ f: \calX \to \{-1, 1\}$, we have $1 - 2\,\indict{ f(x) \ne y} = y \cdot f(x)$. Hence, by definition of $\error$, we have 
    \begin{align}
        1 -2\error_{\wt \calS}(f) = \frac{1}{m}\sum_{i=1}^m y_i \cdot f(x_i) \le \sup_{f \in \calF} \, \frac{1}{m} \sum_{i=1}^m y_i \cdot f(x_i)  \,. \label{eq:error_rademacher}
    \end{align}
    Note that for fixed inputs $(x_1, x_2, \ldots, x_m)$ in $\wt S$, $(y_1, y_2, \ldots y_m)$ are random labels. Define $\phi_1 (y_1, y_2, \ldots, y_m) \defeq \sup_{f \in \calF} \, \frac{1}{m} \sum_{i=1}^m y_i \cdot f(x_i)$. We have the following bounded difference condition on $\phi_1$. For all i, 
    \begin{align}
        \sup_{y_1, \ldots y_m, y_i^\prime \in \{-1, 1\}^{m+1} } \abs{ \phi_1 (y_1,\ldots, y_i, \ldots, y_m) - \phi_1 (y_1,\ldots, y_i^\prime, \ldots, y_m)  } \le 1/m \,. \label{cond1_rademacher}
    \end{align} 
    
    Similarly, we define $\phi_2 (x_1, x_2, \ldots, x_m) \defeq \Expt{ y_i \sim_U \{-1, 1\}  }{ \sup_{f \in \calF} \, \frac{1}{m}  \sum_{i=1}^m y_i \cdot f(x_i)}$. We have the following bounded difference condition on $\phi_2$. 
    For all i,
    \begin{align}
        \sup_{x_1, \ldots x_m, x_i^\prime \in \calX^{m+1} } \abs{ \phi_2 (x_1,\ldots, x_i, \ldots, x_m) - \phi_1 (x_1,\ldots, x_i^\prime, \ldots, x_m)  } \le 1/m \,. \label{cond2_rademacher}
    \end{align}
    Using McDiarmid’s inequality (\lemref{lem:McDiarmid}) twice 
    with Condition \eqref{cond1_rademacher} and \eqref{cond2_rademacher}, 
    with probability at least $1-\delta$, we have
    \begin{align}
        \sup_{f \in \calF} \, \frac{1}{m} \sum_{i=1}^m y_i \cdot f(x_i)  - \Expt{x,y}{\sup_{f \in \calF} \, \frac{1}{m} \sum_{i=1}^m y_i \cdot f(x_i) } \le \sqrt{\frac{2\log(2/\delta)}{m}} \,. \label{eq:final_rademacher}
    \end{align} 
    Combining \eqref{eq:error_rademacher} and \eqref{eq:final_rademacher}, we obtain the desired result. 
\end{proof}


\subsection{Proof of \thmref{thm:error_regularized_ERM}}

Proof of \thmref{thm:error_regularized_ERM} follows similar to the proof of \thmref{thm:error_ERM}. Note that the same results in \lemref{lem:fit_mislabeled}, \lemref{lem:mislabeled_error}, and \lemref{lem:clear_error} hold in the regularized ERM case. However, the arguments in the proof of \lemref{lem:fit_mislabeled} change slightly. Hence, we state the lemma for regularized ERM and prove it here for completeness. 

\begin{lemma} \label{lem:lemma1_reg}
    Assume the same setup as \thmref{thm:error_regularized_ERM}. 
    Then for any $\delta >0$, with probability at least  $1-\delta$ 
    over the random draws of mislabeled data $\wt S_M$, we have 
    \begin{align}
        \error_\calD(\widehat f)  \le 1 -\error_{\wt \calS_M}(\widehat f) + \sqrt{\frac{\log(1/\delta)}{m}}\,. 
    \end{align} 
\end{lemma}
\begin{proof}
    The main idea of the proof remains the same, i.e. regard 
    the clean portion of the data 
    ($S \cup \wt S_C$) as fixed.   
    Then, there exists a classifier $f^*$ 
    that is optimal over draws 
    of the mislabeled data $\wt S_M$. 

    
    Formally, 
    \begin{align}
    f^* \defeq \argmin_{f \in \calF} \error_{\widecheck {\calD}} (f)  + \lambda R(f) \,, \label{eq:modified_ERM_reg}
    \end{align}
    where $$\widecheck \calD = \frac{n}{m+n} \calS + \frac{m_1}{m+n} \wt \calS_C  + \frac{m_2}{m+n}\calDm \,.$$ That is, $\widecheck \calD$ a combination of 
    the \emph{empirical distribution} 
    over correctly labeled data $S \cup \wt S_C$
    % in $S\cup \wt S$ 
    and the (population) distribution 
    over mislabeled data $\calDm$.
    Recall that 
    \begin{align}
    \wh f \defeq \argmin_{f \in \calF} \error_{\calS \cup \wt S} (f) + \lambda R(f) \,. \label{eq:orig_ERM_reg}
    \end{align}
    % 
    % 
    Since, $\widehat f$ minimizes 0-1 error 
    on $S \cup \wt S$, using ERM optimality on \eqref{eq:orig_ERM},  
    we have 
    \begin{align}
        \error_{\calS \cup \wt \calS}(\widehat f) + \lambda R(\wh f) \le \error_{
            \calS \cup \wt \calS}(f^*) + \lambda R(f^*) \,.    \label{eq:step1_reg}
    \end{align}
    Moreover, since $f^*$ is independent of $\wt S_M$, using Hoeffding's bound,
    % \footnote{For a fully rigorous argument,
    % refer to the complete proof in App.~\ref{app:proof_erm}.} 
    we have with probability at least $1-\delta$ that
    \begin{align}
      \error_{\wt \calS_M}(f^*) \le \error_{ \calDm}(f^*) +  \sqrt{\frac{\log(1/\delta)}{2 m_1}} \,. \label{eq:step2_reg} 
    \end{align}
    %$ 
    %for some constant $c_1\le 1/2$. 
    Finally, since $f^*$ is the optimal classifier on $\widecheck \calD$, 
    we have 
    \begin{align}
        \error_{\widecheck \calD}(f^*) + \lambda R(f^*) \le \error_{\widecheck \calD}(\widehat f) + \lambda R(\wh f) \,. \label{eq:step3_reg}
    \end{align}
     Now to relate \eqref{eq:step1_reg} and \eqref{eq:step3_reg}, we can re-write the \eqref{eq:step2_reg} as follows: 
    \begin{align}
        \error_{\calS \cup \wt\calS}(f^*) \le \error_{ \widecheck \calD}(f^*) +  \frac{m_1}{m+n}\sqrt{\frac{\log(1/\delta)}{2 m_1}} \,. \label{eq:step4_reg} 
    \end{align}
    After adding $\lambda R(f^*)$ on both sides in \eqref{eq:step4_reg}, we combine equations \eqref{eq:step1_reg}, \eqref{eq:step4_reg}, and \eqref{eq:step3_reg}, to get 
    \begin{align}
        \error_{\calS \cup \wt \calS}(\wh f) \le \error_{\widecheck \calD}(\wh f) +  \frac{m_1}{m+n}\sqrt{\frac{\log(1/\delta)}{2 m_1}} \,, 
    \end{align}
    which implies 
    \begin{align}
        \error_{ \wt \calS_M}(\wh f) \le \error_{\calDm}(\wh f) + \sqrt{\frac{\log(1/\delta)}{2 m_1}} \,. \label{eq:lemma_reg_final}
    \end{align}
    Similar as before, since $\wt S$ is obtained by randomly labeling an unlabeled dataset, we assume 
    $2m_1 \approx m$. Moreover, using $\error_{\calDm} = 1 - \error_{\calD}$ we obtain the desired result. 
\end{proof}
% \begin{proof}[Proof of ]
    
% \end{proof}

\subsection{Proof of \thmref{thm:multiclass_ERM}}

To prove our results in the multiclass case,
we first state and prove lemmas
parallel to those
% We first state and prove lemmas 
% parallel 
% to the three lemmas 
used in the proof of balanced binary case. 
We then combine these results 
% in the three lemmas 
to obtain the result in \thmref{thm:multiclass_ERM}. 

Before stating the result, 
we define mislabeled distribution $\calDm$ for any $\calD$.
While $\calDm$ and $\calD$ share 
the same marginal distribution over inputs $\calX$,
the conditional distribution over labels $y$ 
given an input $x\sim \calD_\calX$ is changed as follows:
For any $x$, the Probability Mass Function (PMF) over $y$ is defined as:  
$p_{\calDm} (\cdot \vert x) \defeq \frac{1 - p_{\calD}(\cdot \vert x)}{k - 1}$, where $ p_{\calD}(\cdot \vert x)$ is the PMF over $y$ for the distribution $\calD$. 

\begin{lemma} \label{lem:fit_mislabeled_multi}
    Assume the same setup as \thmref{thm:multiclass_ERM}. 
    Then for any $\delta >0$, with probability at least  $1-\delta$ 
    over the random draws of mislabeled data $\wt S_M$, we have 
    \begin{align}
        \error_\calD(\widehat f)  \le (k-1)\left(1 -\error_{\wt \calS_M}(\widehat f)\right) + (k-1)\sqrt{\frac{\log(1/\delta)}{m}}\,. \label{eq:lemma1_multi}
    \end{align}   
\end{lemma} 

\begin{proof}
   
    The main idea of the proof remains the same.
    We begin by regarding the clean portion of the data 
    ($S \cup \wt S_C$) as fixed. 
    Then, there exists a classifier $f^*$ 
    that is optimal over draws 
    of the mislabeled data $\wt S_M$. 
    
    However, in the multiclass case,
    we cannot as easily relate the population error on mislabeled data 
    to the population accuracy on clean data.   
    While for binary classification, 
    % we could upper bound $\error_{\wt \calS_M}$ 
    % with $1-\error_\calD$ 
    we could lower bound the population accuracy $1-\error_\calD$
    with the empirical error on mislabeled data $\error_{\wt \calS_M}$ 
    (in the proof of \lemref{lem:fit_mislabeled}), 
    for multiclass classification, 
    error on the mislabeled data 
    and accuracy on the clean data 
    in the population 
    are not so directly related.  
    To establish \eqref{eq:lemma1_multi},
    we break the error on the 
    (unknown) mislabeled data 
    into two parts: one term corresponds 
    to predicting the true label on mislabeled data, 
    and the other corresponds to predicting 
    neither the true label 
    nor the assigned (mis-)label.  
    Finally, we relate these errors to their
    population counterparts to establish \eqref{eq:lemma1_multi}. 
    
    Formally, 
    \begin{align}
    f^* \defeq \argmin_{f \in \calF} \error_{\widecheck {\calD}} (f)  + \lambda R(f) \,, \label{eq:modified_ERM_reg2}
    \end{align}
    where $$\widecheck \calD = \frac{n}{m+n} \calS + \frac{m_1}{m+n} \wt \calS_C  + \frac{m_2}{m+n}\calDm \,.$$ 
    That is, $\widecheck \calD$ is a combination 
    of the \emph{empirical distribution} 
    over correctly labeled data $S \cup \wt S_C$
    % in $S\cup \wt S$ 
    and the (population) distribution 
    over mislabeled data $\calDm$.
    Recall that 
    \begin{align}
    \wh f \defeq \argmin_{f \in \calF} \error_{\calS \cup \wt S} (f) + \lambda R(f) \,. \label{eq:orig_ERM_reg2}
    \end{align}
    % 
    % 
    Following the exact steps from the proof of \lemref{lem:lemma1_reg}, 
    with probability at least $1-\delta$, we have  
    \begin{align}
        \error_{ \wt \calS_M}(\wh f) \le \error_{\calDm}(\wh f) + \sqrt{\frac{\log(1/\delta)}{2 m_1}} \,. \label{eq:lemma1_final_multi_prev}
    \end{align}
    Similar to before, since $\wt S$ is obtained 
    by randomly labeling an unlabeled dataset, 
    we assume 
    $\frac{k}{k-1} m_1 \approx m$. 
    
    Now we will relate $\error_{\calDm} (\wh f)$ with $\error_{\calD}(\wh f)$. 
    Let $y^T$ denote the (unknown) true label 
    for a mislabeled point $(x, y)$ 
    (i.e., label before replacing it with a mislabel). 
    \begin{align*}    
         \Expt{(x, y) \in \sim \calDm}{\indict{ \wh f(x) \ne y }}  &= \underbrace{\Expt{(x, y) \in \sim \calDm}{\indict{ \wh f(x) \ne y \land \wh f(x) \ne y^T}}}_{\RN{1}} \\ &\qquad \qquad + \underbrace{\Expt{(x, y) \in \sim \calDm}{\indict{ \wh f(x) \ne y \land \wh f(x) = y^T}}}_{\RN{2}} \,. \numberthis \label{eq:excess_term}
    \end{align*}
    Clearly, term 2 is one minus the accuracy 
    on the clean unseen data, i.e.,
    \begin{align}
        \RN{2} = 1 - \Expt{{x,y} \sim \calD}{ \indict{ \wh f(x) \ne y}} = 1- \error_{\calD}(\wh f) \,. \label{eq:term1}    
    \end{align}
    Next, we relate term 1 with the error on the unseen clean data. 
    We show that term 1 is equal to the error on the unseen clean data 
    scaled by $\frac{k-2}{k-1}$,
    where $k$ is the number of labels.
    Using the definition of mislabeled distribution $\calDm$,  
    we have 
    \begin{align}
        \RN{1} = \frac{1}{k-1} \left( \Expt{(x, y) \in \sim \calD}{ \sum_{i \in \calY \land i\ne y}  \indict{ \wh f(x) \ne i \land \wh f(x) \ne y}} \right) = \frac{k-2}{k-1} \error_{\calD}(\wh f) \,.\label{eq:term2}
    \end{align}    

    Combining the result in \eqref{eq:term1}, \eqref{eq:term2} and \eqref{eq:excess_term}, we have 
    \begin{align}
        \error_{\calDm}(\wh f) = 1- \frac{1}{k-1} \error_{\calD}(\wh f) \,.\label{eq:combine_terms}
    \end{align}
    Finally, combining the result in \eqref{eq:combine_terms} 
    with equation \eqref{eq:lemma1_final_multi_prev}, 
    we have with probability $1-\delta$, 
    \begin{align}
      \error_{\calD}(\wh f) \le  (k-1) \left( 1- \error_{ \wt \calS_M}(\wh f) \right)  + (k-1) \sqrt{\frac{k \log(1/\delta)}{ 2(k-1)m}} \,. \label{eq:lemma1_final_multi}
    \end{align}
\end{proof}

\begin{lemma} \label{lem:mislabeled_error_multi}
    Assume the same setup as \thmref{thm:multiclass_ERM}. 
    Then for any $\delta >0$, 
    with probability at least $1-\delta$ 
    over the random draws of $\wt S$, we have  
    % \begin{align}
        $$\abs{k\error_{\wt \calS}(\widehat f) - \error_{\wt \calS_C}(\widehat f) -  (k-1)\error_{\wt \calS_M}(\widehat f) } \le  2k\sqrt{\frac{\log(4/\delta)}{2m}}\,. $$ % \label{eq:lemma2}
    % \end{align}   
    %  for some constant $c_3 \le 1.0\,$.
\end{lemma} 


\begin{proof}
    Recall $\error_{\wt S} (f) = \frac{m_1}{m} \error_{\wt S_M}(f) + \frac{m_2}{m} \error_{\wt S_C}(f)$. Hence, we have 
    \begin{align*}
        k\error_{\wt S}(f) - (k-1)\error_{\wt S_M}(f) - \error_{\wt S_C}(f) &= (k-1)\left(\frac{k m_1}{(k-1) m} \error_{\wt S_M}(f) - \error_{\wt S_M}(f)\right) \\ & \qquad \qquad + \left(\frac{km_2}{m} \error_{\wt S_C}(f) - \error_{\wt S_C}(f)\right) \\ &= k \left[ \left(\frac{m_1}{m} - \frac{k-1}{k}\right) \error_{\wt S_M}(f) + \left(\frac{m_2}{m} - \frac{1}{k} \right) \error_{\wt S_C} (f) \right] \,.
    \end{align*} 
    Since the dataset is randomly labeled, 
    we have with probability at least $1-\delta$, 
    $\left(\frac{m_1}{m} - \frac{k-1}{k}\right) \le \sqrt{\frac{\log(1/\delta)}{2m}}$. 
    Similarly, we have with probability at least $1-\delta$, 
    $\left(\frac{m_2}{m} - \frac{1}{k}\right) \le \sqrt{\frac{\log(1/\delta)}{2m}}$. 
    Using union bound, we have with probability at least $1-\delta$
    % \begin{align}
    %     2\error_{\wt S} - \error_{\wt S_M}(f) - \error_{\wt S_C}(f) \le \sqrt{\frac{\log(2/\delta)}{2m}} \left(\error_{\wt S_M}(f) + \error_{\wt S_C}(f) \right) \le 2\sqrt{\frac{\log(2/\delta)}{2m}} \,. \label{eq:lemma2_final}
    % \end{align}
    \begin{align}
        k\error_{\wt S}(f) - (k-1)\error_{\wt S_M}(f) - \error_{\wt S_C}(f)  \le k \sqrt{\frac{\log(2/\delta)}{2m}} \left(\error_{\wt S_M}(f) + \error_{\wt S_C}(f) \right) \,. \label{eq:lemma2_final_multi}
    \end{align}

    % We obtain the desired result by using 
\end{proof}

\begin{lemma} \label{lem:clear_error_multi}
    Assume the same setup as \thmref{thm:multiclass_ERM}. 
    Then for any $\delta >0$, with probability at least $1-\delta$ 
    over the random draws of $\wt S_C$ and $S$, we have 
    % \begin{align}
        $$\abs{\error_{\wt \calS_C}(\widehat f) - \error_{\calS}(\widehat f) } \le 1.5 \sqrt{\frac{k\log(2/\delta)}{2m}}\,.$$ %\label{eq:lemma3}
    % \end{align}   
    % for some constant $c_2 \le 1.2\,$.
\end{lemma} 
\begin{proof}
    % Recall 0-1 error on each point  $(x,y) \in S \cup \wt S$ is given by $\I{ f(x)\ne y}$.
    In the set of correctly labeled points $S \cup \wt S_C$,
    we have $S$ as a random subset of $S \cup \wt S_C$. 
    Hence, using Hoeffding's inequality 
    for sampling without replacement 
    (\lemref{lem:hoeffding_sampling}), 
    we have with probability at least $1-\delta$
    \begin{align}
        \error_{\wt \calS_c} (\wh f)- \error_{\calS \cup \wt \calS_C}( \wh f) \le  \sqrt{\frac{\log(1/\delta)}{2m_2}} \,.
    \end{align}
    Re-writing $\error_{\calS \cup \wt \calS_C}( \wh f)$ 
    as $\frac{m_2}{m_2 + n} \error_{\wt \calS_C }(\wh f) + \frac{n}{m_2 + n} \error_{\calS }(\wh f)$, 
    we have with probability at least $1-\delta$
    \begin{align}
       \left(\frac{n}{n+m_2}\right) \left(\error_{\wt \calS_c} (\wh f)- \error_{\calS}( \wh f) \right) \le  \sqrt{\frac{\log(1/\delta)}{2m_2}} \,.
    \end{align}
    As before, assuming $km_2 \approx m$, 
    we have with probability at least $1-\delta$ 
    \begin{align}
        \error_{\wt \calS_c} (\wh f)- \error_{\calS}( \wh f) \le \left(1+\frac{m_2}{n}\right)  \sqrt{\frac{k\log(1/\delta)}{2m}} \le \left( 1 + \frac{1}{k}\right) \sqrt{\frac{k\log(1/\delta)}{2m}} \,. \label{eq:lemma3_final_multi}
    \end{align} 
\end{proof}

\begin{proof}[Proof of \thmref{thm:multiclass_ERM}] 
    Having established these core intermediate results, 
    we can now combine above three lemmas. 
    In particular, we bound the population error 
    on clean data ($\error_\calD(\wh f)$) as follows:  
    \begin{enumerate}[(i)]
        \item First, use \eqref{eq:lemma1_final_multi}, 
        to obtain an upper bound on the population error on clean data, 
        i.e., with probability at least $1-\delta/4$, we have
        \begin{align}
            \error_{ \calD} (\wh f) \le (k-1)\left(1 - \error_{ \wt \calS_M}(\wh f) \right) + (k-1) \sqrt{\frac{k\log(4/\delta)}{2(k-1)m}} \,. 
        \end{align}
        \item  Second, use \eqref{eq:lemma2_final_multi}
        to relate the error on the mislabeled fraction 
        with error on clean portion of randomly labeled data 
        and error on whole randomly labeled dataset, 
        i.e., with probability at least $1-\delta/2$, we have 
        \begin{align}
            - (k-1)\error_{\wt S_M}(f) \le \error_{\wt S_C}(f) - k\error_{\wt S}  + k\sqrt{\frac{\log(4/\delta)}{2m}}  \,. 
        \end{align} 
        \item Finally, use \eqref{eq:lemma3_final_multi} 
        to relate the error on the clean portion of randomly labeled data 
        and error on clean training data, 
        i.e., with probability $1-\delta/4$, we have 
        \begin{align}
            \error_{\wt \calS_C} (\wh f)\le - \error_{\calS}( \wh f) + \left(1 + \frac{m}{kn} \right) \sqrt{\frac{k\log(4/\delta)}{2m}} \,. 
        \end{align} 
    \end{enumerate}

    Using union bound on the above three steps, 
    we have with probability at least $1-\delta$: 
    \begin{align}
        \error_\calD (\wh f) \le \error_{\calS}(\wh f) + (k-1) - k\error_{\wt \calS}(\wh f)   + (\sqrt{k(k-1)} + k + \sqrt{k} + \frac{m}{n\sqrt{k}})  \sqrt{\frac{\log(4/\delta)}{2m}} \,.\label{eq:multiclass_ERM_final}
    \end{align}
    Simplifying the term in RHS of \eqref{eq:multiclass_ERM_final}, 
    we get the desired result. 
    % Note that since $\frac{m}{n\sqrt{k}}$ 
    % is much smaller than the sum of the other terms
    % the other terms in summation, 
    % we ignore $\frac{m}{n\sqrt{k}}$  
    % Z: ??? --- great
    % that 
    % them
    in the final bound. 
    % we ignore that in the final bound. 
    % Note that $(1/\sqrt{2} + 2.5)$ is a loose constant. In experiments, we use the ratio $\frac{m}{n}$
    %  the exact error $\error_{\wt \calS}(\wh f)$ 
    % to evaluate R.H.S.    
\end{proof}

\newpage
\section{Proofs from \secref{sec:linear_models}}\label{app:proof_gd}
We suppose that the parameters of the linear function 
are obtained via gradient descent on 
the following $L_2$ regularized problem: 
\begin{align}
    % n in denominator is avoided deliberately
    \calL_S(w; \lambda) \defeq \sum_{i=1}^n{(w^Tx_i - y_i)^2} + \lambda \norm{w}{2}^2 \,, \label{eq:l2_MSE_app}   
\end{align}
where $\lambda\ge0$ is a regularization parameter. 
We assume access to a clean dataset 
$S = \{(x_i, y_i)\}_{i=1}^n \sim \calD^n$ 
and randomly labeled dataset 
$\wt S = \{(x_i, y_i)\}_{i=n+1}^{n+m} \sim \wt \calD^m$. 
Let $\bX = [x_1, x_2, \cdots, x_{m+n}]$ 
and $\by = [y_1, y_2, \cdots, y_{m+n}]$. 
Fix a positive learning rate $\eta$ such that 
$\eta \le 1/\left(\norm{\bX^T\bX}{\text{op}} + \lambda^2\right)$ 
and an initialization $w_0 = 0$. 
% \todos{Assumption made for simplicty}. 
Consider the following gradient descent iterates 
to minimize objective \eqref{eq:l2_MSE_app} on $S \cup \wt S$:
\begin{align}
w_t = w_{t-1} - \eta \grad_w \calL_{S \cup \wt S} (w_{t-1}; \lambda) \quad \forall t=1,2,\ldots \label{eq:GD_iterates_app}
\end{align} 
Then we have $\{ w_t\}$ converge to the limiting solution 
$\wh w = \left( \bX^T\bX+\lambda \boldsymbol{I}\right)^{-1}\bX^T\by$. Define $\widehat f (x) \defeq f(x ; \wh w) $.  

% \subsection{\textcolor{red}{Errata}}

% We wish to correct the following error in the body:
% \codref{cond:error_stability} is not enough 
% to guarantee the result in \thmref{thm:linear}. 
% We now present a slightly stronger condition 
% called \emph{hypothesis stability} 
% under which we obtain a result 
% similar to \thmref{thm:linear}. 

% This error doesn't change the main arguments of the proof,
% where we show that the empirical train error 
% is less than or equal to the leave-one-out error.
% We need a stronger condition to relate leave-one-out error 
% with the population error of the original classifier. 
% Specifically, while \codref{cond:error_stability} 
% relates the average population error of leave-one-out classifiers 
% with the population error of the original classifier, 
% we need the new condition to show the concentration 
% of the empirical leave-one-out error 
% and average population error of leave-one-out classifiers. 
% main takeaway 

% Note that the new condition, 
% while being stronger than the previous one, 
% still doesn't imply generalization \citep{bousquet2002stability,elisseeff2003leave,abou2019exponential}. 
% Overall, the main results in \secref{sec:ERM_training} 
% and takeaways of the paper remain unaffected by the error.  

% We now present the new condition 
% and a corrected statement of \thmref{thm:linear}. 
% Recall, for a given training set $S \sim \calD^n $, 
% we use $S_{(i)}$ to denote the training set $S$ 
% with the $i^{\text{th}}$ point removed.

% \begin{condition}[Hypothesis Stability] 
%     \label{cond:hypothesis_stability}
%     We have $\beta$ hypothesis stability 
%     if our training algorithm $\calA$ satisfies the following: 
%     \begin{align*}
%     % ${\sum_{i=1}^n \frac{\error_{\calD}( f(\calA, S_{(i)}))}{n} - \error_\calD(f(\calA, S))} \le \beta\,$.
%     \forall i \in \{1,2,\ldots, n\}, \quad  \Expt{\calS, (x,y) \in \calD}{ \abs{\error\left( f(x) ,y  \right) - \error\left( f_{(i)}(x), y \right) }} \le \frac{\beta}{n} \,,
%     \end{align*}
%     where $f_{(i)} \defeq f(\calA, S_{(i)})$ and $ f \defeq f(\calA, S)$.
% \end{condition}

% \begin{theorem}[Correct statement of \thmref{thm:linear}] \label{thm:new_linear}
%     Assume that this gradient descent algorithm satisfies \codref{cond:hypothesis_stability}
%     with $\beta=\calO(1)$.  
%     Then for any $\delta >0$, with probability at least $1-\delta$ 
%     over the random draws of datasets $\wt S$ and $S$, we have:
%     \begin{align}
%         \error_\calD(\widehat f) \le \error_\calS(\widehat f) + 1 - 2 \error_{\wt\calS}(\widehat f) + \left(\frac{1}{\sqrt{2}} + 1.5 \right) \sqrt{\frac{\log(4/\delta)}{m}} + \sqrt{\frac{4}{\delta}\left(\frac{1}{m} +\frac{3\beta}{m+n} \right)}  \,. \label{eq:gd_error}
%     \end{align} 
%     % for some constant $c\le 3.2$.
% \end{theorem}

\subsection{Proof of \thmref{thm:linear}}
We use a standard result from linear algebra, 
namely the Shermann-Morrison formula 
\citep{sherman1950adjustment} for matrix inversion:  

\begin{lemma}[\citet{sherman1950adjustment}] \label{lem:sherman}
    Suppose $\bA \in \Real^{n \times n}$ 
    is an invertible square matrix 
    and $u,v \in \Real^n$ are column vectors. 
    Then $\bA + uv^T$ is invertible iff $1 + v^T \bA u \ne 0$ 
    and in particular
    \begin{align}
        (\bA + u v^T)^{-1} = \bA^{-1}  - \frac{\bA^{-1} uv^T \bA^{-1} }{ 1 + v^T \bA^{-1} u} \,.
    \end{align}   
\end{lemma}
\newcommand\byy[1]{\by_{\left(#1\right)}}
\newcommand\bXX[1]{\bX_{\left(#1\right)}}
\newcommand\ff[1]{\wh f_{\left(#1\right)}}

For a given training set $S \cup \wt S_C$, 
define leave-one-out error 
on mislabeled points in the training data 
as $$\error_{\text{LOO}(\wt S_M) } = \frac{\sum_{(x_i, y_i) \in \wt S_M} \error( f_{(i)}( x_i), y_i)}{ \abs{\wt S_M }} \,, $$
where $f_{(i)} \defeq f(\calA, (S \cup \wt S)_{(i)})$. 
To relate empirical leave-one-out error and population error 
with hypothesis stability condition, 
we use the following lemma:   

\begin{lemma}[\citet{bousquet2002stability}] \label{lem:stability_error}
    For the leave-one-out error, we have
    \begin{align}
        \Expo{ \left( \error_{\calDm}(\wh f) -\error_{\text{LOO}(\wt S_M) } \right)^2 } \le \frac{1}{2m_1}+  \frac{3\beta}{n + m}\,.
    \end{align}   
    % where $ f \defeq f(\calA, S \cup \wt S) $.
\end{lemma}

Proof of the above lemma is similar 
to the proof of Lemma 9 in \citet{bousquet2002stability} 
and can be found in \appref{app:proof_lem_error}. 
% 
% Before presenting the result, we introduce some notation. 
Before presenting the proof of \thmref{thm:linear}, 
we introduce some more notation. 
Let $\bX_{(i)}$ denote the matrix of covariates 
with the $i^{\text{th}}$ point removed. 
Similarly, let $\by_{(i)}$ be the array of responses 
with the $i^{\text{th}}$ point removed. 
Define the corresponding regularized GD solution 
as $\wh w_{(i)} = \left( \bXX{i}^T\bXX{i}+\lambda \boldsymbol{I}\right)^{-1}\bXX{i}^T\byy{i}$. 
Define $\ff{i}(x) \defeq f(x ; \wh w_{(i)}) $.

\begin{proof}[Proof of \thmref{thm:linear}]
    Because squared loss minimization does not imply 0-1 error minimization, 
    we cannot use arguments from \lemref{lem:fit_mislabeled}. 
    This is the main technical difficulty. 
    To compare the 0-1 error at a train point with an unseen point, 
    we use the closed-form expression for $\widehat{w}$ 
    and Shermann-Morrison formula 
    to upper bound training error 
    with leave-one-out cross validation error. 
    
    The proof is divided into three parts: 
    In part one, we show that 0-1 error 
    on mislabeled points in the training set 
    is lower than the error obtained 
    by leave-one-out error at those points. 
    In part two, we relate this leave-one-out error 
    with the population error on mislabeled distribution
    using \codref{cond:hypothesis_stability}.
    While the empirical leave-one-out error is an unbiased estimator 
    of the average population error of leave-one-out classifiers, 
    we need hypothesis stability 
    to control the variance 
    of empirical leave-one-out error. 
    Finally, in part three, we show 
    that the error on the mislabeled training points 
    can be estimated with just the randomly labeled 
    and clean training data (as in proof of \thmref{thm:error_ERM}).  

    \textbf{Part 1 {} {}} First we relate training error with leave-one-out error.        
    For any training point $(x_i, y_i)$ in $\wt S \cup S$, we have 
    \begin{align}
        \error(\wh f(x_i), y_i ) &= \indict{ y_i \cdot x_i^T \wh w < 0 } = \indict{ y_i \cdot x_i^T \left( \bX^T\bX+\lambda \boldsymbol{I}\right)^{-1}\bX^T\by < 0 } \\
        &= \indict{ y_i \cdot x_i^T \underbrace{\left( \bXX{i}^T\bXX{i} + x_i ^T x_i +\lambda \boldsymbol{I}\right)^{-1}}_{\RN{1}} (\bXX{i}^T\byy{i} + y_i \cdot x_i) < 0 } \,.
    \end{align}
    Letting $\bA = \left(\bXX{i}^T\bXX{i} +\lambda \boldsymbol{I}\right)$ 
    and using \lemref{lem:sherman} on term 1, we have 
    \begin{align}
        \error(\wh f(x_i), y_i ) &= \indict{ y_i \cdot x_i^T \left[\bA^{-1} -  \frac{\bA^{-1} x_i x_i^T \bA^{-1}}{ 1 + x_i ^T \bA^{-1} x_i } \right] (\bXX{i}^T\byy{i} + y_i \cdot x_i) < 0 } \\
        &= \indict{ y_i \cdot\left[ \frac{ x_i^T \bA^{-1} ( 1 + x_i ^T \bA^{-1} x_i ) -  x_i^T \bA^{-1} x_i x_i^T \bA^{-1}}{ 1 + x_i ^T \bA ^{-1}x_i } \right] (\bXX{i}^T\byy{i} + y_i \cdot x_i) < 0 } \\
        &= \indict{ y_i \cdot\left[ \frac{ x_i^T \bA^{-1}}{ 1 + x_i ^T \bA ^{-1}x_i } \right] (\bXX{i}^T\byy{i} + y_i \cdot x_i) < 0 } \,.
    \end{align}

    Since $1 + x_i^T \bA^{-1} x_i > 0$, we have 
    \begin{align}
        \error(\wh f(x_i), y_i ) &= \indict{ y_i \cdot x_i^T \bA^{-1} (\bXX{i}^T\byy{i} + y_i \cdot x_i) < 0 } \\
        &= \indict{ x_i^T \bA^{-1} x_i +  y_i \cdot x_i^T \bA^{-1} (\bXX{i}^T\byy{i}) < 0 } \\
        &\le \indict{ y_i \cdot x_i^T \bA^{-1} (\bXX{i}^T\byy{i}) < 0 } = \error(\ff{i}(x_i), y_i ) \,.\label{eq:LOO_error}
    \end{align}

    Using \eqref{eq:LOO_error}, we have 
    \begin{align}
        \error_{\wt \calS_M } (\wh f) \le \error_{\text{LOO} (\wt S_M)} \defeq \frac{\sum_{(x_i, y_i) \in \wt S_M} \error(\ff{i}(x_i), y_i ) }{\abs{\wt \calS_M}}\label{eq:LOO_error_final} \,.
    \end{align}
    \textbf{Part 2 {}{}} We now relate RHS in \eqref{eq:LOO_error_final} 
    with the population error on mislabeled distribution. 
    To do this, we leverage \codref{cond:hypothesis_stability} 
    and \lemref{lem:stability_error}. 
    In particular, we have 

    \begin{align}
        \Expt{\calS \cup \wt \calS_M }{ \left(\error_{\calDm}(\wh f) - \error_{\text{LOO} (\wt S_M)}\right)^2 } \le \frac{1}{2m_1} + \frac{3\beta}{m+n} \,.
    \end{align}

    Using Chebyshev's inequality, with probability at least $1-\delta$, we have 
    \begin{align}
        \error_{\text{LOO} (\wt S_M)} \le  \error_{\calDm}(\wh f)   + \sqrt{\frac{1}{\delta}\left(\frac{1}{2m_1} +\frac{3\beta}{m+n} \right)} \,. \label{eq:final_mislabeled_linear}
    \end{align}
    

    \textbf{Part 3 {}{}} Combining \eqref{eq:final_mislabeled_linear} and \eqref{eq:LOO_error_final}, we have 

    \begin{align}
        \error_{\wt \calS_M } (\wh f) \le \error_{\calDm}(\wh f)   + \sqrt{\frac{1}{\delta}\left(\frac{1}{2m_1} +\frac{3\beta}{m+n} \right)} \,. \label{eq:linear_parallel_lem1}
    \end{align}

    Compare \eqref{eq:linear_parallel_lem1} with \eqref{eq:lemma1_final} 
    in the proof of \lemref{lem:fit_mislabeled}. 
    We obtain a similar relationship 
    between $\error_{\wt \calS_M }$ and $\error_{\calDm}$ 
    but with a polynomial concentration 
    instead of exponential concentration. 
    In addition, since we just use concentration arguments 
    to relate mislabeled error to the errors
    on the clean and unlabeled portions 
    of the randomly labeled data, 
    we can directly use the results 
    in \lemref{lem:mislabeled_error} and \lemref{lem:clear_error}. 
    Therefore, combining results in \lemref{lem:mislabeled_error}, \lemref{lem:clear_error}, and \eqref{eq:linear_parallel_lem1} with union bound, 
    we have with probability at least $1-\delta$
    \begin{align}
        \error_\calD(\widehat f) \le \error_\calS(\widehat f) + 1 - 2 \error_{\wt\calS}(\widehat f) + \left(\sqrt{2}\error_{\wt\calS}(\widehat f) + 1 + \frac{m}{2n} \right) \sqrt{\frac{\log(4/\delta)}{m}} + \sqrt{\frac{4}{\delta}\left(\frac{1}{m} +\frac{3\beta}{m+n} \right)}  \,.
    \end{align}
    

       
\end{proof}

\subsection{Extension to multiclass classification} \label{app:multiclass_linear}
For multiclass problems with squared loss minimization, as standard practice, we consider one-hot encoding for the underlying label, i.e., a class label $c \in [k]$ is treated as $(0, \cdot, 0,1,0, \cdot, 0) \in \Real^k$ (with $c$-th coordinate being 1).  As before, we suppose that the parameters of the linear function 
are obtained via gradient descent on the following $L_2$ regularized problem: 
\begin{align}
    % n in denominator is avoided deliberately
    \calL_S(w; \lambda) \defeq \sum_{i=1}^n\norm{w^Tx_i - y_i}{2}^2 + \lambda \sum_{j=1}^k \norm{w_j}{2}^2 \,, \label{eq:l2_multiclass_MSE_app}   
\end{align}
where $\lambda\ge0$ is a regularization parameter. 
We assume access to a clean dataset 
$S = \{(x_i, y_i)\}_{i=1}^n \sim \calD^n$ 
and randomly labeled dataset 
$\wt S = \{(x_i, y_i)\}_{i=n+1}^{n+m} \sim \wt \calD^m$. 
Let $\bX = [x_1, x_2, \cdots, x_{m+n}]$ 
and $\by = [e_{y_1}, e_{y_2}, \cdots, e_{y_{m+n}}]$. 
Fix a positive learning rate $\eta$ such that 
$\eta \le 1/\left(\norm{\bX^T\bX}{\text{op}} + \lambda^2\right)$ 
and an initialization $w_0 = 0$. 
% \todos{Assumption made for simplicty}. 
Consider the following gradient descent iterates 
to minimize objective \eqref{eq:l2_MSE_app} on $S \cup \wt S$:
\begin{align}
{w_j}^t = {w_j}^{t-1} - \eta \grad_{w_j} \calL_{S \cup \wt S} (w^{t-1}; \lambda) \quad \forall t=1,2,\ldots \text{ and } j=1,2,\ldots,k  \,. \label{eq:GD_multi_iterates_app}
\end{align} 
Then we have $\{ {w_j}^t\}$ for all $j =1,2,\cdots, k$ converge to the limiting solution 
$\wh w_j = \left( \bX^T\bX+\lambda \boldsymbol{I}\right)^{-1}\bX^T\by_j$. Define $\widehat f (x) \defeq f(x ; \wh w) $.  

\begin{theorem}\label{thm:multi_linear}
    Assume that this gradient descent algorithm satisfies \codref{cond:hypothesis_stability}
    with $\beta=\calO(1)$.  
    Then for a multiclass classification problem wth $k$ classes, for any $\delta >0$, with probability at least $1-\delta$, we have:
    \begin{align*}
        \error_\calD(\widehat f) \le \error_\calS(\widehat f) &+ (k-1)\left(1 - \frac{k}{k-1} \error_{\wt\calS}(\widehat f) \right) \\ &+ \left(k + \sqrt{k} + \frac{m}{n\sqrt{k}} \right) \sqrt{\frac{\log(4/\delta)}{2m}} + \sqrt{k(k-1)} \sqrt{\frac{4}{\delta}\left(\frac{1}{m} +\frac{3\beta}{m+n} \right)}  \,. \numberthis \label{eq:gd_multi_error}
    \end{align*} 
    % for some constant $c\le 3.2$.
\end{theorem}
\begin{proof}
    The proof of this theorem is divided into two parts. In the first part, we relate the error on the mislabeled samples with the population error on the mislabeled data. Similar to the proof of \thmref{thm:linear}, we use Shermann-Morrison formula to upper bound training error with leave-one-out error on each $\wh w^j$. Second part of the proof follows entirely from the proof of \thmref{thm:multiclass_ERM}. In essence, the first part derives an equivalent of \eqref{eq:lemma1_final_multi_prev} for GD training with squared loss and then the second part follows from the proof  of \thmref{thm:multiclass_ERM}. 
    
    \textbf{Part-1:} Consider a training point $(x_i,y_i)$ in $\wt S \cup S $. For simplicity, we use $c_i$ to denote the class of $i$-th point and use $y_i$ as the corresponding one-hot embedding. Recall error in multiclass point is given by $\error(\wh f(x_i), y_i ) = \indict{ c_i \not \in \argmax x_i^T \wh w }$. Thus, there exists a $j \ne c_i \in [k]$, such that we have
     \begin{align}
        \error(\wh f(x_i), y_i ) &= \indict{ c_i \not \in \argmax x_i^T \wh w } = \indict{ x_i^T \wh w_{c_i} < x_i^T \wh w_{j}  } \\ &= \indict{ x_i^T \left( \bX^T\bX+\lambda \boldsymbol{I}\right)^{-1}\bX^T\by_{c_i} < x_i^T \left( \bX^T\bX+\lambda \boldsymbol{I}\right)^{-1}\bX^T\by_{j} } \\
        &= \indict{ x_i^T \underbrace{\left( \bXX{i}^T\bXX{i} + x_i ^T x_i +\lambda \boldsymbol{I}\right)^{-1}}_{\RN{1}} \left(\bXX{i}^T{\by_{c_i}}_{(i)} + x_i - \bXX{i}^T{\by_{j}}_{(i)}\right) < 0 } \,.
    \end{align}
    Letting $\bA = \left(\bXX{i}^T\bXX{i} +\lambda \boldsymbol{I}\right)$ 
    and using \lemref{lem:sherman} on term 1, we have 
    \begin{align}
        \error(\wh f(x_i), y_i ) &= \indict{ x_i^T \left[\bA^{-1} -  \frac{\bA^{-1} x_i x_i^T \bA^{-1}}{ 1 + x_i ^T \bA^{-1} x_i } \right]  \left(\bXX{i}^T{\by_{c_i}}_{(i)} + x_i - \bXX{i}^T{\by_{j}}_{(i)}\right) < 0 } \\
        &= \indict{ \left[ \frac{ x_i^T \bA^{-1} ( 1 + x_i ^T \bA^{-1} x_i ) -  x_i^T \bA^{-1} x_i x_i^T \bA^{-1}}{ 1 + x_i ^T \bA ^{-1}x_i } \right]  \left(\bXX{i}^T{\by_{c_i}}_{(i)} + x_i - \bXX{i}^T{\by_{j}}_{(i)}\right) < 0 } \\
        &= \indict{ \left[ \frac{ x_i^T \bA^{-1}}{ 1 + x_i ^T \bA ^{-1}x_i } \right]  \left(\bXX{i}^T{\by_{c_i}}_{(i)} + x_i - \bXX{i}^T{\by_{j}}_{(i)}\right) < 0} \,.
    \end{align}
    Since $1 + x_i^T \bA^{-1} x_i > 0$, we have 
    \begin{align}
        \error(\wh f(x_i), y_i ) &= \indict{ x_i^T \bA^{-1}  \left(\bXX{i}^T{\by_{c_i}}_{(i)} + x_i - \bXX{i}^T{\by_{j}}_{(i)}\right) < 0 } \\
        &= \indict{ x_i^T \bA^{-1} x_i +  x_i^T \bA^{-1}  \bXX{i}^T{\by_{c_i}}_{(i)}  - x_i^T\bA^{-1}  \bXX{i}^T{\by_{j}}_{(i)} < 0 } \\
        &\le \indict{  x_i^T \bA^{-1}  \bXX{i}^T{\by_{c_i}}_{(i)}  - x_i^T\bA^{-1}  \bXX{i}^T{\by_{j}}_{(i)} < 0  } = \error(\ff{i}(x_i), y_i ) \,.\label{eq:LOO_error_multi}
    \end{align}
    Using \eqref{eq:LOO_error_multi}, we have 
    \begin{align}
        \error_{\wt \calS_M } (\wh f) \le \error_{\text{LOO} (\wt S_M)} \defeq \frac{\sum_{(x_i, y_i) \in \wt S_M} \error(\ff{i}(x_i), y_i ) }{\abs{\wt \calS_M}}\label{eq:LOO_error_multi_final} \,.
    \end{align}
    
    We now relate RHS in \eqref{eq:LOO_error_final} 
    with the population error on mislabeled distribution. 
    Similar as before, to do this, we leverage \codref{cond:hypothesis_stability} 
    and \lemref{lem:stability_error}. Using  \eqref{eq:final_mislabeled_linear} and \eqref{eq:LOO_error_multi_final}, we have 
    \begin{align}
        \error_{\wt \calS_M } (\wh f) \le \error_{\calDm}(\wh f)   + \sqrt{\frac{1}{\delta}\left(\frac{1}{2m_1} +\frac{3\beta}{m+n} \right)} \,. \label{eq:linear_multi_parallel_lem1}
    \end{align}
    
    We have now derived a parallel to \eqref{eq:lemma1_final_multi_prev}. Using the same arguments in the proof of \lemref{lem:fit_mislabeled_multi}, we have 
    \begin{align}
      \error_{\calD}(\wh f) \le  (k-1) \left( 1- \error_{ \wt \calS_M}(\wh f) \right)  + (k-1)\sqrt{\frac{k}{\delta(k-1)}\left(\frac{1}{2m_1} +\frac{3\beta}{m+n} \right)}  \,. \label{eq:lemma1_linear_final_multi}
    \end{align}
    
    \textbf{Part-2:} We now combine the results in \lemref{lem:mislabeled_error_multi} and \lemref{lem:clear_error_multi} to obtain the final inequality in terms of quantities that can be computed from just the randomly labeled and clean data. Similar to the binary case, we obtained a polynomial concentration instead of exponential concentration. Combining \eqref{eq:lemma1_linear_final_multi} with \lemref{lem:mislabeled_error_multi} and \lemref{lem:clear_error_multi}, we have with probability at least $1-\delta$
    \begin{align*}
        \error_\calD(\widehat f) \le \error_\calS(\widehat f) &+ (k-1)\left(1 - \frac{k}{k-1} \error_{\wt\calS}(\widehat f) \right) \\ &+ \left(k + \sqrt{k} + \frac{m}{n\sqrt{k}} \right) \sqrt{\frac{\log(4/\delta)}{2m}} + \sqrt{k(k-1)} \sqrt{\frac{4}{\delta}\left(\frac{1}{m} +\frac{3\beta}{m+n} \right)}  \,. \numberthis \label{eq:gd_multi_error_proof}
    \end{align*} 
\end{proof}

\subsection{Discussion on \codref{cond:hypothesis_stability}} \label{app:discuss_cond1}
The quantity in LHS of \codref{cond:hypothesis_stability} 
measures how much the function learned by the algorithm 
(in terms of error on unseen point) will change 
when one point in the training set is removed. 
% Discussion on exponential concentration and stronger condition. 
% Notice that hypothesis stability implies error stability, i.e., \codref{cond:error_stability} \citep{bousquet2002stability}.  
% In summary, while error stability allowed us 
% to relate the average population error 
% of the leave-one-out classifiers 
% with the population error of the original classifier, 
We need hypothesis stability condition 
to control the variance of the empirical leave-one-out error to show concentration of average leave-one-error with the population error. 

Additionally, we note that while the dominating term in the RHS of \thmref{thm:linear} matches with the dominating term in ERM bound in \thmref{thm:error_ERM}, there is a polynomial concentration term 
(dependence on $1/\delta$ instead of $\log(\sqrt{1/\delta})$) 
in \thmref{thm:linear}. 
Since with hypothesis stability, 
we just bound the variance, 
the polynomial concentration is due 
to the use of Chebyshev's inequality 
instead of an exponential tail inequality
(as in \lemref{lem:fit_mislabeled}).
Recent works have highlighted that 
a slightly stronger condition than hypothesis stability 
can be used to obtain an exponential concentration 
for leave-one-out error \citep{abou2019exponential},
but we leave this for future work for now. 
% We leave 
% However, the constants 

% we also want to highlight  

\subsection{Formal statement and proof of \propref{prop:early_stop}} \label{app:formal_early_stop}

Before formally presenting the result, 
we will introduce some notation.  
By $\calL_{S}(w)$, we denote 
the objective in \eqref{eq:l2_MSE_app} with $\lambda=0$. 
Assume Singular Value Decomposition (SVD) of $\bX$
as $\sqrt{n} \bU \bS^{1/2} \bV^T$. 
Hence $\bX^T \bX = \bV \bS \bV^T$.
Consider the GD iterates defined in \eqref{eq:GD_iterates_app}. 
% 
We now derive closed form expression 
for the $t^\text{th}$ iterate of gradient descent:  
% 
\begin{align}
    w_t = w_{t-1} + \eta \cdot \bX^T (\by - \bX w_{t-1}) = (\bI - \eta \bV \bS \bV^T )w_{k-1} + \eta \bX^T \by \,.
\end{align}
Rotating by $\bV^T$, we get 
\begin{align}
    \wt w_t = (\bI - \eta\bS )\wt w_{k-1} + \eta \wt \by \label{eq:GD_recur},
\end{align}
where $\wt w_t = \bV^T w_t $ and $\wt \by = \bV^T \bX^T \by$. 
Assuming the initial point $w_0 = 0$ 
and applying the recursion in \eqref{eq:GD_recur}, we get
\begin{align}
    \wt w_t = \bS ^{-1} ( \bI - (\bI - \eta \bS)^k ) \wt \by \,, 
\end{align} 
Projecting solution back to the original space, we have 
\begin{align}
     w_t = \bV \bS ^{-1} ( \bI - (\bI - \eta \bS)^k ) \bV^T \bX^T \by \,. 
\end{align} 
% We will work with this GD solution at any iterate $t$ in the next proposition. 
Define $f_t(x) \defeq f(x;w_t)$ 
as the solution at the $t^{\text{th}}$ iterate. 
Let $\wt w_{\lambda} = \argmin_{w} \calL_\calS (w;\lambda) = (\bX^T \bX + \lambda \bI)^{-1} \bX^T \by = \bV (\bS + \lambda \bI )^{-1} \bV^T \bX^T \by $. 
% ) \,,$ for all $t=1,2,\ldots\,.$ 
and define $\wt f_\lambda(x) \defeq f(x;\wt w_\lambda)$ as the regularized solution. 
Assume $\kappa$ be the condition number 
of the population covariance matrix 
and let $s_\text{min}$ be the minimum positive 
singular value of the empirical covariance matrix. 
Our proof idea is inspired from recent work 
on relating gradient flow solution 
and regularized solution 
for regression problems \citep{ali2018continuous}. 
We will use the following lemma in the proof: 
\begin{lemma} \label{lem:ineq_soln}
    For all $x \in [0,1]$ and for all $ k \in \mathbb{N}$, 
    we have (a) $ \frac{kx}{1+kx} \le 1- (1-x)^k$ 
    and (b) $ 1- (1-x)^k \le 2 \cdot \frac{kx}{kx+1} $.
    %  where $g(c)$ is a constant dependent on $c$. For $c = 1$, $g(c) = 2.0$.   
\end{lemma}
\begin{proof}
    % [Proof of \lemref{lem:ineq_soln}]
    % Part (a) is easy. 
    Using $ (1-x)^k \le \frac{1}{1+kx}$, we have part (a). 
    For part (b), we numerically maximize 
    $\frac{ (1+kx ) (1 - (1-x)^k) }{kx}$ 
    for all $k\ge 1$ and for all $x \in [0, 1]$.  
\end{proof}

% 
% Next, 

\begin{prop}[Formal statement of \propref{prop:early_stop}] \label{prop:formal_early_stop}
Let $\lambda = \frac{1}{t\eta}$. 
For a training point $x$, we have 
\begin{align*}
    \Expt{x \sim \calS}{(f_t(x) - \wt f_\lambda(x))^2} &\le c(t,\eta) \cdot \Expt{x \sim \calS}{f_t(x)^2} \,, %\label{eq:early_stop}
\end{align*}
where $c(t, \eta) \defeq \min( 0.25, \frac{1}{s_\text{min}^2 t^2 \eta^2})$. 
Similarly for a test point, we have 
\begin{align*}
    \Expt{x \sim \calD_\calX}{(f_t(x) - \wt f_\lambda(x))^2} &\le \kappa \cdot c(t,\eta) \cdot \Expt{x \sim \calD_\calX}{f_t(x)^2} \,. %\label{eq:early_stop}
\end{align*}
\end{prop} 

\begin{proof}
    %%%%%%%%%%%%% 
    We want to analyze the expected squared difference output 
    of regularized linear regression 
    with regularization constant $\lambda = \frac{1}{\eta t}$ 
    and the gradient descent solution at the $t^\text{th}$ iterate. 
    We separately expand the algebraic expression 
    for squared difference at a training point and a test point. 
    % We start by considering the difference  
    Then the main step is to show that 
    $\left[ \bS ^{-1} ( \bI - (\bI - \eta \bS)^k )  - (\bS + \lambda \bI )^{-1}\right] \preceq c(\eta, t) \cdot \bS ^{-1} ( \bI - (\bI - \eta \bS)^k ) $.

    %%%%%%%%%%%%%
    
   \textbf{Part 1 {} {}} 
    First, we will analyze the squared difference 
    of the output at a training point 
    (for simplicity, we refer to $S \cup \wt S$ as $S$), i.e., 
    \begin{align}
        \Expt{ x \sim \calS }{\left(f_t(x) - \wt f_\lambda (x)\right)^2} &= \norm{\bX w_t - \bX \wt w_\lambda}{2}^2\\ &=   \norm{\bX \bV \bS ^{-1} ( \bI - (\bI - \eta \bS)^t ) \bV^T \bX^T \by - \bX \bV (\bS + \lambda \bI )^{-1} \bV^T \bX^T \by }{2}^2 \\
        &= \norm{\bX \bV \left(\bS ^{-1} ( \bI - (\bI - \eta \bS)^t ) - (\bS + \lambda \bI )^{-1} \right) \bV^T \bX^T \by  }{2} \\
        &=  \by^T \bV \bX \left( \underbrace{\bS ^{-1} ( \bI - (\bI - \eta \bS)^t ) - (\bS + \lambda \bI )^{-1}}_{\RN{1}} \right)^2 \bS \bV^T \bX^T \by \label{eq:train_GD_rel} \,.
        %  (\bX \bV \bS ^{-1} ( \bI - (\bI - \eta \bS)^k ) \bV^T \bX^T \by)^T \bX \bV \bS ^{-1} ( \bI - (\bI - \eta \bS)^k ) \bV^T \bX^T \by
    \end{align}
    We now separately consider term 1. 
    Substituting $\lambda = \frac{1}{t \eta}$, 
    we get
    \begin{align}
        \bS ^{-1} ( \bI - (\bI - \eta \bS)^t ) - (\bS + \lambda \bI )^{-1} &= \bS^{-1} \left( ( \bI - (\bI - \eta \bS)^t ) - (\bI + \bS^{-1} \lambda )^{-1}\right) \\
        &= \underbrace{\bS^{-1} \left( ( \bI - (\bI - \eta \bS)^t ) - (\bI + ( \bS t \eta)^{-1}  )^{-1}\right)}_{\bA} \,.
    \end{align}

    We now separately bound the diagonal entries in matrix $\bA$. 
    With $s_i$, we denote $i^{\text{th}}$ diagonal entry of $\bS$.
    Note that since $ \eta\le 1/\norm{S}{\text{op}}$, 
    for all $i$, $\eta s_i  \le 1$.  
    Consider $i^{\text{th}}$ diagonal term (which is non-zero) 
    of the diagonal matrix $\bA$, we have 
    \begin{align}
        \bA_{ii} = \frac{1}{s_i} \left(  1 - (1 - s_i \eta)^t - \frac{t \eta s_i}{1 + t \eta s_i } \right) &=  \frac{1 - (1 - s_i \eta)^t}{s_i} \left( \underbrace{ 1 - \frac{t \eta s_i}{(1 + t \eta s_i)(1 - (1 - s_i \eta)^t)}}_{\RN{2}} \right) \\ 
         &\le \frac{1}{2}\left[ \frac{1 - (1 - s_i \eta)^t}{ s_i} \right] \tag*{(Using \lemref{lem:ineq_soln} (b))} \,.
    \end{align} 
    Additionally, we can also show the following upper bound on term 2: 
    \begin{align}
         1 - \frac{t \eta s_i}{(1 + t \eta s_i)(1 - (1 - s_i \eta)^t)} &= \frac{(1 + t \eta s_i)(1 - (1 - s_i \eta)^t) - t \eta s_i }{(1 + t \eta s_i)(1 - (1 - s_i \eta)^t)} \\
         & \le  \frac{ 1 -  (1 - s_i \eta)^t - t \eta s_i (1 - s_i \eta)^t}{(1 + t \eta s_i)(1 - (1 - s_i \eta)^t)} \\
         & \le \frac{1}{t\eta s_i} \,. \tag{Using \lemref{lem:ineq_soln} (a)}
        %  &\le \frac{1}{2}\left[ \frac{1 - (1 - s_i \eta)^t}{ s_i} \right] \tag*{(Using \lemref{lem:ineq_soln})} \,.
    \end{align} 

    Combining both the upper bounds 
    on each diagonal entry $\bA_{ii}$, we have 
    \begin{align}
    \bA \preceq c_1(\eta, t) \cdot \bS^{-1} ( \bI - (\bI - \eta \bS)^t ) \,, \label{eq:upperbound_diagonal}
    \end{align}
    where $c_1(\eta, t ) = \min(0.5, \frac{1}{t s_i \eta })$. Plugging this into \eqref{eq:train_GD_rel}, we have 
    \begin{align}
        \Expt{ x \sim \calS }{\left(f_t(x) - \wt f_\lambda (x)\right)^2} &\le c(\eta, t) \cdot \by^T \bV \bX  \left( \bS^{-1} ( \bI - (\bI - \eta \bS)^t ) \right)^2 \bS \bV^T \bX^T \by \\
        &=   c(\eta, t) \cdot \by^T \bV \bX  \left( \bS^{-1} ( \bI - (\bI - \eta \bS)^t ) \right) \bS \left( \bS^{-1} ( \bI - (\bI - \eta \bS)^t ) \right) \bV^T \bX^T \by \\
        & =  c(\eta, t) \cdot \norm{\bX w_t}{2}^2 \\
        &= c(\eta, t) \cdot  \Expt{ x \sim \calS }{\left(f_t(x) \right)^2} \,,
    \end{align}
    where $c(\eta, t ) = \min(0.25, \frac{1}{t^2 s^2_i \eta^2 })$.

    \textbf{Part 2 {} {}} With $\bSigma$, 
    we denote the underlying true covariance matrix. 
    We now consider the squared difference of output at an unseen point: 
    \begin{align}
        \Expt{ x \sim \calD_{\calX} }{\left(f_t(x) - \wt f_\lambda (x)\right)^2} &= \Expt{x \sim \calD_{\calX}}{\norm{x^T w_t - x^T \wt w_\lambda}{2}} \\
        &=   \norm{x^T \bV \bS ^{-1} ( \bI - (\bI - \eta \bS)^t ) \bV^T \bX^T \by - x^T \bV (\bS + \lambda \bI )^{-1} \bV^T \bX^T \by }{2} \\
        &= \norm{x^T \bV \left(\bS ^{-1} ( \bI - (\bI - \eta \bS)^t ) - (\bS + \lambda \bI )^{-1} \right) \bV^T \bX^T \by  }{2} \\
        &= \by^T \bV \bX \left( \bS ^{-1} ( \bI - (\bI - \eta \bS)^t ) - (\bS + \lambda \bI )^{-1} \right) \bV^T \bSigma \bV \\ &\qquad \qquad \qquad \qquad \qquad \left( (\bI - (\bI - \eta \bS)^t ) - (\bS + \lambda \bI )^{-1} \right) \bV^T \bX^T \by \\
        &\le \sigma_{\text{max}} \cdot \by^T \bV \bX \left( \underbrace{\bS ^{-1} ( \bI - (\bI - \eta \bS)^t ) - (\bS + \lambda \bI )^{-1}}_{\RN{1}} \right)^2 \bV^T \bX^T \by \,, \label{eq:test_GD_rel}
        %  (\bX \bV \bS ^{-1} ( \bI - (\bI - \eta \bS)^k ) \bV^T \bX^T \by)^T \bX \bV \bS ^{-1} ( \bI - (\bI - \eta \bS)^k ) \bV^T \bX^T \by
    \end{align}
    where $\sigma_{\text{max}}$ is the maximum eigenvalue 
    of the underlying covariance matrix $\bSigma$. 
    Using the upper bound on term 1 in \eqref{eq:upperbound_diagonal}, 
    we have 
    \begin{align}
        \Expt{ x \sim \calD_{\calX} }{\left(f_t(x) - \wt f_\lambda (x)\right)^2} &\le \sigma_{\text{max}} \cdot c(\eta, t) \cdot \by^T \bV \bX  \left( \bS^{-1} ( \bI - (\bI - \eta \bS)^t ) \right)^2 \bV^T \bX^T \by \\
        &=   \kappa \cdot c(\eta, t) \cdot \sigma_{\text{min}}\cdot \norm{\bV \left( \bS^{-1} ( \bI - (\bI - \eta \bS)^t ) \right) \bV^T \bX^T \by}{2}^2 \\
        &\le \kappa \cdot c(\eta, t) \cdot \left[ \bV \left( \bS^{-1} ( \bI - (\bI - \eta \bS)^t ) \right) \bV^T \bX^T \right]^T \bSigma \\
        &\qquad \qquad \qquad \qquad \qquad \left[ \bV \left( \bS^{-1} ( \bI - (\bI - \eta \bS)^t ) \right) \bV^T \bX^T \right] \by \\
        & = \kappa \cdot c(\eta, t) \cdot \Expt{x \sim \calD_{\calX}}{\norm{x^T w_t}{2}} \,.
    \end{align}
% 
% 
    % Since $ \eta\le 1/\norm{S}{\text{op}}$, invoking \lemref{lem:ineq_soln} to upper bound term 1 with
\end{proof}

\subsection{Extension to deep learning} \label{appsubsec:ext_DL}
Under \asmpref{appsubsec:justifying_assumption1}, we present the formal result parallel to \thmref{thm:multiclass_ERM}. 
\begin{theorem} \label{thm:multiclass_ERM_algoA}
    Consider a multiclass classification problem 
    with $k$ classes. Under \asmpref{asmp:deep_models}, 
    for any $\delta >0$, with probability at least $1-\delta$,
    we have
    \vspace{-10pt}
    \begin{align*}
        \error_\calD(\widehat f)  \le \error_\calS(\widehat f) + (k-1) \left(1 - \tfrac{k}{k-1} \error_{\wt\calS}(\widehat f)\right) + c\sqrt{\frac{\log(\frac{4}{\delta})}{2m}} \,,\numberthis \label{eq:multiclass_ERM_deep}
    % \vspace{-20pt}
    \end{align*}
    for some constant $c \le ((c+1) k+\sqrt{k} + \frac{m}{n\sqrt{k}})$.
\end{theorem}

The proof follows exactly as in step (i) to (iii) in \thmref{thm:multiclass_ERM}.  

\subsection{Justifying~\asmpref{asmp:deep_models}} \label{appsubsec:justifying_assumption1}

Motivated by the analysis on linear models, we now discuss alternate (and weaker) conditions that imply \asmpref{asmp:deep_models}. 
We need hypothesis stability (\codref{cond:hypothesis_stability}) and the following assumption relating training error and leave-one-error: 

\begin{assumption} \label{asmp:loo_error}
Let $\wh f$ be a model obtained by training with algorithm $\calA$ on a mixture of clean $S$ and randomly labeled data $\wt S$. Then we assume we have 
\begin{align*}
    \error_{\wt \calS_M} (\wh f) \le  \error_{\text{LOO} (\wt S_M)} \,, 
\end{align*}
for all $(x_i, y_i) \in  \wt S_M$ where $\wh f_{(i)} \defeq f(\calA, S \cup {{}\wt S_M}_{(i)})$ and  $\error_{\text{LOO} (\wt S_M)} \defeq  \frac{\sum_{(x_i, y_i) \in \wt S_M} \error(\ff{i}(x_i), y_i ) }{\abs{\wt \calS_M}}$.  
\end{assumption}

% we assume this to extend our result (parallel to \thmref{thm:multi_linear}) for deep models. 
Intuitively, this assumption states that the error on a (mislabeled) datum $(x,y)$ included in the training set is less than the error on that datum $(x,y)$ obtained by a model trained on the training set $S - \{(x,y)\}$. We proved this for linear models trained with GD in the proof of \thmref{thm:multi_linear}. 
% 
\codref{cond:hypothesis_stability} with $\beta = \calO(1)$ and \asmpref{asmp:loo_error} together with \lemref{lem:stability_error} implies \asmpref{asmp:deep_models} with a polynomial residual term (instead of logarithmic in $1/\delta$): 
\begin{align}
     \error_{\calS_M} (\wh f) \le  \error_{\calDm}(\wh f)   + \sqrt{\frac{1}{\delta}\left(\frac{1}{m} +\frac{3\beta}{m+n} \right)} \,.
\end{align}
% Note that this  

\newpage 
\section{Additional experiments and details}\label{app:exp}
\newcommand\tab[1][1cm]{\hspace*{#1}}

\subsection{Datasets} \label{sec:app_dataset}

\textbf{Toy Dataset {} {}} Assume fixed constants $\mu$ and $\sigma$. For a given label $y$, we simulate features $x$ in our toy classification setup as follows: 
\begin{align*}
    x \defeq \texttt{concat} \left[ x_1, x_2\right] \quad \text{where} \quad  x_1 \sim  \calN( y \cdot \mu, \sigma^2 I_{d \times d}) \ \  \text{and} \ \  x_1 \sim  \calN( 0, \sigma^2 I_{d \times d}) \,.
\end{align*}  
% where $y$ is the true label and $x$ is the corresponding feature vector. 
In experiements throughout the paper, we fix dimention $d=100$, $\mu = 1.0 $, and $\sigma = \sqrt{d}$. Intuitively, $x_1$ carries the information about the underlying label and $x_2$ is additional noise independent of the underlying label. 

\textbf{CV datasets {} {}} We use MNIST~\citep{lecun1998mnist} and CIFAR10~\cite{krizhevsky2009learning}. 
% For binary tasks, 
We produce a binary variant from the multiclass classification problem by mapping classes $\{0,1,2,3,4\}$ to label $1$ and $\{ 5,6,7,8,9\}$ to label $-1$. For CIFAR dataset, we also use the standard data augementation of random crop and horizontal flip. PyTorch code is as follows: 

\texttt{(transforms.RandomCrop(32, padding=4),\\
\tab transforms.RandomHorizontalFlip())}

\textbf{NLP dataset {} {}} We use IMDb Sentiment analysis~\citep{maas2011learning} corpus.  

\subsection{Architecture Details} 

All experiments were run on NVIDIA GeForce RTX 2080 Ti GPUs. We used PyTorch~\citep{NEURIPS2019a9015} and Keras with Tensorflow~\citep{abadi2016tensorflow} backend for experiments. 
% , ELMo embeddings~\citep{Peters:2018}, and Hugging Face Transformers~\citep{wolf-etal-2020-transformers}. 

\textbf{Linear model {} {}} For the toy dataset, we simulate a linear model with scalar output and the same number of parameters as the number of dimensions.   

\textbf{Wide nets {} {}} To simulate the NTK regime, we experiment with $2-$layered wide nets. The PyTorch code for 2-layer wide MLP is as follows: 


\texttt{ nn.Sequential( \\
\tab     nn.Flatten(),\\
\tab    nn.Linear(input\_dims, 200000, bias=True),\\
\tab    nn.ReLU(),\\
\tab    nn.Linear(200000, 1, bias=True)\\
\tab     )}


We experiment both (i) with the second layer fixed at random initialization; (ii)  and updating both layers' weights.     

\textbf{Deep nets for CV tasks {} {}} We consider a 4-layered MLP. The PyTorch code for 4-layer MLP is as follows: 

\texttt{ nn.Sequential(nn.Flatten(), \\
\tab        nn.Linear(input\_dim, 5000, bias=True),\\
\tab        nn.ReLU(),\\
\tab        nn.Linear(5000, 5000, bias=True),\\
\tab        nn.ReLU(),\\
\tab        nn.Linear(5000, 5000, bias=True),\\
\tab        nn.ReLU(),\\
% \tab        nn.Linear(5000, 5000, bias=True),\\
% \tab        nn.ReLU(),\\
\tab        nn.Linear(1024, num\_label, bias=True)\\
\tab        )}

For MNIST, we use $1000$ nodes instead of $5000$ nodes in the hidden layer. 
% 
We also experiment with convolutional nets. In particular, we use ResNet18 \citep{he2016deep}. Implementation adapted from:  \url{https://github.com/kuangliu/pytorch-cifar.git}. 

\textbf{Deep nets for NLP {} {}} We use a simple LSTM model with embeddings intialized with ELMo embeddings~\citep{Peters:2018}. Code adapted from: \url{https://github.com/kamujun/elmo_experiments/blob/master/elmo_experiment/notebooks/elmo_text_classification_on_imdb.ipynb} 

We also evaluate our bounds with a BERT model. In particular, we fine-tune an off-the-shelf uncased BERT model~\citep{devlin2018bert}. Code adapted from Hugging Face Transformers~\citep{wolf-etal-2020-transformers}: \url{https://huggingface.co/transformers/v3.1.0/custom_datasets.html}. 


\subsection{Additonal experiments}

\textbf{Results with SGD on underparameterized linear models {} {}} 

\begin{figure*}[h]
    \centering 
    % \vspace{-15pt}
    % \includegraphics[width=0.9\linewidth]{example-image-a}
    \includegraphics[width=0.3\linewidth]{figures/lowdim-Gaussian-SGD.pdf}
    % \includegraphics[width=0.9\linewidth]{figures/{CIFAR10_rn=0.1_lr=0.2_wd=0.005}.png}
    \vspace{-5pt}
    \caption{ 
    % Predicted lower bound 
    % on different
    We plot the accuracy and corresponding bound 
    (RHS in \eqref{eq:erm}) at $\delta = 0.1$
    for toy binary classification task. 
    Results aggregated over $3$ seeds. 
    % i.e., $1-\error$ where $\error$ is the term in the RHS of \eqref{eq:erm}
    Accuracy vs fraction of unlabeled data (w.r.t clean data) 
    in the toy setup with a linear model trained with SGD. Results parallel to \figref{fig:error_binary}(a) with SGD.  }
    \label{fig:error_binary_linear}
    \vspace{-5pt}
\end{figure*}

\textbf{Results with wide nets on binary MNIST {} {}}

\begin{figure*}[h]
    \centering 
    % \vspace{-15pt}
    % \includegraphics[width=0.9\linewidth]{example-image-a}
    \subfigure[GD with MSE loss]{\includegraphics[width=0.3\linewidth]{figures/MNIST-GD_MSE.pdf}} \hfil
    \subfigure[SGD with CE loss]{\includegraphics[width=0.3\linewidth]{figures/MNIST-SGD_CE.pdf}}
    \subfigure[SGD with MSE loss]{\includegraphics[width=0.3\linewidth]{figures/MNIST-SGD_MSE-first-layer.pdf}}
    % \includegraphics[width=0.9\linewidth]{figures/{CIFAR10_rn=0.1_lr=0.2_wd=0.005}.png}
    \vspace{-5pt}
    \caption{ 
    % Predicted lower bound 
    % on different
    We plot the accuracy and corresponding bound 
    (RHS in \eqref{eq:erm}) at $\delta = 0.1$ 
    for binary MNIST classification. 
    Results aggregated over $3$ seeds. 
    % i.e., $1-\error$ where $\error$ is the term in the RHS of \eqref{eq:erm}
    Accuracy vs fraction of unlabeled data 
    for a 2-layer wide network on binary MNIST with both the layers training in (a,b) and only first layer training in (c). 
    Results parallel to \figref{fig:error_binary}(b) .  }
    \label{fig:error_binary_MNIST}
    \vspace{-5pt}
\end{figure*}

% \begin{figure*}[h]
%     \centering 
%     % \vspace{-15pt}
%     % \includegraphics[width=0.9\linewidth]{example-image-a}
%     \subfigure[GD with MSE loss]{\includegraphics[width=0.3\linewidth]{figures/MNIST.pdf}} \hfil
    
%     \subfigure[SGD with CE loss]{\includegraphics[width=0.3\linewidth]{figures/MNIST.pdf}}
%     % \includegraphics[width=0.9\linewidth]{figures/{CIFAR10_rn=0.1_lr=0.2_wd=0.005}.png}
%     \vspace{-5pt}
%     \caption{ 
%     % Predicted lower bound 
%     % on different
%     We plot the accuracy and corresponding bound 
%     (RHS in \eqref{eq:erm}) at $\delta = 0.1$
%     for binary MNIST classification. 
%     Results aggregated over $3$ seeds. 
%     % i.e., $1-\error$ where $\error$ is the term in the RHS of \eqref{eq:erm}
%     Accuracy vs fraction of unlabeled data 
%     for a 2-layer wide network on binary MNIST with just the first layer training. 
%     Results parallel to \figref{fig:error_binary}(b) with only the first layer training.  }
%     \label{fig:error_binary_MNIST}
%     \vspace{-5pt}
% \end{figure*}

\textbf{Results on CIFAR 10 and MNIST {} {}} 
% 
We plot epoch wise error curve for results in \tabref{table:multiclass}(\figref{fig:error_epoch_CIFAR10} and \figref{fig:error_epoch_MNIST}). We observe the same trend as in \figref{fig:error_CIFAR10}. Additionally, we plot an \emph{oracle bound} obtained by tracking the error on mislabeled data which nevertheless were predicted as true label. To obtain an exact emprical value of the oracle bound, we need underlying true labels for the randomly labeled data. 
% Note that our bound in \thmref{thm:multiclass_ERM}, lower bounds the accuracy as predicted by the oracle bound. 
While with just access to extra unlabeled data we cannot calculate oracle bound, we note that the oracle bound is very tight and never violated in practice underscoring an importamt aspect of generalization in multiclass problems. This highlight that even a stronger conjecture may hold in multiclass classification, i.e., error on mislabeled data (where nevertheless true label was predicted) lower bounds the population error on the distribution of mislabeled data and hence, the error on (a specific) mislabeled portion predicts the population accuracy on clean data. 
% 
On the other hand, the dominating term of in \thmref{thm:multiclass_ERM} is loose when compared with the oracle bound. The main reason, we believe is the pessimistic upper bound in \eqref{eq:lemma1_final_multi_prev} in the proof of \lemref{lem:fit_mislabeled_multi}. We leave an investigation on this gap for future. 
% of fit 

% However, oracle bound highlights two . One,  



\begin{figure}[h]
    \centering 
    % \vspace{-15pt}
    % \includegraphics[width=0.9\linewidth]{example-image-a}
    \subfigure[MLP]{\includegraphics[width=0.3\linewidth]{figures/CIFAR10-FNN.pdf}} \hfil
    \subfigure[ResNet]{\includegraphics[width=0.3\linewidth]{figures/CIFAR10-Resnet.pdf}}
    % \includegraphics[width=0.9\linewidth]{figures/{CIFAR10_rn=0.1_lr=0.2_wd=0.005}.png}
    % \vspace{-10pt}
    \caption{ Per epoch curves for CIFAR10 corresponding results in \tabref{table:multiclass}. As before, we just plot the dominating term in the RHS of \eqref{eq:multiclass_ERM} as predicted bound. Additionally, we also plot the predicted lower bound by the error on mislabeled data which nevertheless were predicted as true label. We refer to this as ``Oracle bound''. See text for more details. 
    % 
    % except for the stopping point. 
    % The bound predicted by RATT (RHS in \eqref{eq:multiclass_ERM}) is vacuous. 
    }\label{fig:error_epoch_CIFAR10}
    % \vspace{-15pt}
\end{figure}


\begin{figure}[h]
    \centering 
    % \vspace{-15pt}
    % \includegraphics[width=0.9\linewidth]{example-image-a}
    \subfigure[MLP]{\includegraphics[width=0.3\linewidth]{figures/MNIST-FNN.pdf}} \hfil
    \subfigure[ResNet]{\includegraphics[width=0.3\linewidth]{figures/MNIST-Resnet.pdf}}
    % \includegraphics[width=0.9\linewidth]{figures/{CIFAR10_rn=0.1_lr=0.2_wd=0.005}.png}
    % \vspace{-10pt}
    \caption{ Per epoch curves for MNIST corresponding results in \tabref{table:multiclass}. As before, we just plot the dominating term in the RHS of \eqref{eq:multiclass_ERM} as predicted bound. Additionally, we also plot the predicted lower bound by the error on mislabeled data which nevertheless were predicted as true label. We refer to this as ``Oracle bound''. See text for more details. 
    % 
    % except for the stopping point. 
    % The bound predicted by RATT (RHS in \eqref{eq:multiclass_ERM}) is vacuous. 
    }\label{fig:error_epoch_MNIST}
    % \vspace{-15pt}
\end{figure}

\textbf{Results on CIFAR 100 {} {}} 
% 
On CIFAR100, our bound in \eqref{eq:multiclass_ERM} yields vacous bounds. However, the oracle bound as explained above yields tight guarantees in the initial phase of the learning (i.e., when learning rate is less than $0.1$) (\figref{fig:error_CIFAR100}).  

\begin{figure}[h]
    \centering 
    % \vspace{-15pt}
    % \includegraphics[width=0.9\linewidth]{example-image-a}
    \includegraphics[width=0.3\linewidth]{figures/CIFAR100-Resnet.pdf}
    % \includegraphics[width=0.9\linewidth]{figures/{CIFAR10_rn=0.1_lr=0.2_wd=0.005}.png}
    % \vspace{-10pt}
    \caption{ Predicted lower bound by the error on mislabeled data which nevertheless were predicted as true label with ResNet18 on CIFAR100. We refer to this as ``Oracle bound''. See text for more details. 
    % 
    % except for the stopping point. 
    The bound predicted by RATT (RHS in \eqref{eq:multiclass_ERM}) is vacuous. 
    }\label{fig:error_CIFAR100}
    % \vspace{-15pt}
\end{figure}


% \paragraph{Experiments on CIFAR100} 


% \subsection{Model Selection using RATT}


\subsection{Hyperparameter Details}


\textbf{\figref{fig:error_CIFAR10} {} {}} We use clean training dataset of size $40,000$. We fix the amount of unlabeled data at $20\%$ of the clean size, i.e. we include additional $8,000$ points with randomly assigned labels. We use test set of $10,000$ points. For both MLP and ResNet, we use SGD with an initial learning rate of $0.1$ and momentum $0.9$. We fix the weight decay parameter at $5\times 10^{-4}$. After $100$ epochs, we decay the learning rate to $0.01$. We use SGD batch size of $100$. 

\textbf{\figref{fig:error_binary} (a) {} {}} We obtain a toy dataset according to the process described in \secref{sec:app_dataset}. We fix $d=100$ and create a dataset of $50,000$ points with balanced classes. Moreover, we sample additional covariates with the same procedure to create randomly labeled dataset. For both SGD and GD training, we use a fixed learning rate $0.1$.    

\textbf{\figref{fig:error_binary} (b) {} {}} Similar to binary CIFAR, we use clean training dataset of size $40,000$ and fix the amount of unlabeled data at $20\%$ of the clean dataset size. To train wide nets, we use a fixed learning of $0.001$ with GD and SGD. We decide the weight decay parameter and the early stopping point that maximizes our generalization bound (i.e. without peeking at unseen data ).  We use SGD batch size of $100$. 

\textbf{\figref{fig:error_binary} (c) {} {}} With IMDb dataset, we use a clean dataset of size $20,000$ and as before, fix the amount of unlabeled data at $20\%$ of the clean data. To train ELMo model, we use Adam optimizer with a fixed learning rate $0.01$ and weight decay $10^{-6}$ to minimize cross entropy loss. We train with batch size $32$ for 3 epochs. To fine-tune BERT model, we use Adam optimizer with learning rate $5\times 10^{-5}$ to minimize cross entropy loss. We train with a batch size of $16$ for 1 epoch.    

\textbf{\tabref{table:multiclass} {} {}} For multiclass datasets, we train both MLP and ResNet with the same hyperparameters as described before. We sample a clean training dataset of size $40,000$ and fix the amount of unlabeled data at $20\%$ of the clean size. We use SGD with an initial learning rate of $0.1$ and momentum $0.9$. We fix the weight decay parameter at $5\times 10^{-4}$. After $30$ epochs for ResNet and after $50$ epochs for MLP, we decay the learning rate to $0.01$.  We use SGD with batch size $100$. 
For \figref{fig:error_CIFAR100}, we use the same hyperparameters as 
CIFAR10 training, except we now decay learning rate after $100$ epochs. 


In all experiments, to identify the best possible accuracy on just the clean data, we use the exact same set of hyperparamters except the stopping point. We choose a stopping point that maximizes test performance. 

\subsection{Summary of experiments }

\begin{center}
    \begin{table}[H] 
        \centering
        \begin{tabular}{|c|c|c|c|} 
        \hline
        Classification type & Model category & Model & Dataset  \\ [0.5ex] 
        \hline
        \hline
        \multirow{10}{*}{Binary} & Low dimensional & Linear model & Toy Gaussain dataset  \\
                        \cline{2-4}
                         & Overparameterized 
                        %  & Linear model & Toy Gaussain dataset \\
                        %  \cline{3-4}
                        %  & & 2-layer wide net& Toy Gaussain dataset \\
                        %  \cline{3-4}
                         & \multirow{2}{*}{2-layer wide net} & \multirow{2}{*}{Binary MNIST} \\
                         & linear nets & &  
                         \\
                         \cline{2-4}                 
                         & \multirow{6}{*}{Deep nets} & \multirow{2}{*}{MLP} & Binary MNIST \\
                         \cline{4-4}
                         & &  & Binary CIFAR \\
                         \cline{3-4}
                         &  & \multirow{2}{*}{ResNet} & Binary MNIST \\
                         \cline{4-4}
                         & &  & Binary CIFAR \\
                         \cline{3-4}
                         &  & ELMo-LSTM model & IMDb Sentiment Analysis \\
                         \cline{3-4}
                         & & BERT pre-trained model & IMDb Sentiment Analysis \\
        \hline
        \multirow{5}{*}{Multiclass} & \multirow{5}{*}{Deep nets} & \multirow{2}{*}{MLP} & MNIST \\
                        \cline{4-4} 
                        & & & CIFAR10 \\                   
                        \cline{3-4}
                         &   & \multirow{3}{*}{ResNet} & MNIST \\
                         \cline{4-4}
                         &   & & CIFAR10 \\
                         \cline{4-4}
                         &   & & CIFAR100 \\
        \hline
        \end{tabular}
        % \caption{Summary of experiments performed} \label{table:experiments}
    \end{table}    
    % \footnotetext[6]{We use both MSE loss and cross-entropy loss.}
    % \footnotetext[6]{We try 2 variants: one with a fixed first layer and the other with both layers trainable.}
\end{center}

\newpage
\section{Proof of \lemref{lem:stability_error}} \label{app:proof_lem_error}

\begin{proof}[Proof of \lemref{lem:stability_error}]
    Recall, we have a training set $S \cup \wt S_C$. We defined leave-one-out error on mislabeled points as $$\error_{\text{LOO}(\wt S_M) } = \frac{\sum_{(x_i, y_i) \in \wt S_M} \error( f_{(i)}( x_i), y_i)}{ \abs{\wt S_M }} \,, $$
    where $f_{(i)} \defeq f(\calA, (S \cup \wt S)_{(i)})$. Define $S^\prime \defeq S \cup \wt S$. Assume $(x,y)$ and $(x^\prime,y^\prime)$ as i.i.d. samples from ${\calDm}$. 
    Using Lemma 25 in \citet{bousquet2002stability}, we have
    \begin{align*}
        \Expo{ \left( \error_{\calDm}(\wh f) -\error_{\text{LOO}(\wt S_M) } \right)^2 } \le & \Expt{ S^\prime, (x,y), (x^\prime,y^\prime) }{ \error(\wh f(x), y ) \error(\wh f(x^\prime), y^\prime )} - 2 \Expt{ S^\prime, (x,y) }{ \error(\wh f(x), y ) \error(f_{(i)}(x_i), y_i )} \\
        & + \frac{m_1-1}{m_1}\Expt{ S^\prime }{  \error(f_{(i)}(x_i), y_i )  \error(f_{(j)}(x_j), y_j )} + \frac{1}{m_1} \Expt{ S^\prime }{  \error(f_{(i)}(x_i), y_i ) } \,. \numberthis \label{eq:main_reln}
    \end{align*}
    We can rewrite the equation above as : 
    \begin{align*}
        \Expo{ \left( \error_{\calDm}(\wh f) -\error_{\text{LOO}(\wt S_M) } \right)^2 } \le &  \, \underbrace{\Expt{ S^\prime, (x,y), (x^\prime,y^\prime) }{ \error(\wh f(x), y ) \error(\wh f(x^\prime), y^\prime ) - \error(\wh f(x), y ) \error(f_{(i)}(x_i), y_i )}}_{\RN{1}} \\
        & + \underbrace{\Expt{ S^\prime }{  \error(f_{(i)}(x_i), y_i )  \error(f_{(j)}(x_j), y_j ) -  \error(\wh f(x), y ) \error(f_{(i)}(x_i), y_i )}}_{\RN{2}} \\ &+ \underbrace{\frac{1}{m_1} \Expt{ S^\prime }{  \error(f_{(i)}(x_i), y_i ) - \error(f_{(i)}(x_i), y_i )  \error(f_{(j)}(x_j), y_j ) }}_{\RN{3}} \,. \numberthis \label{eq:main_reln2}
    \end{align*}
    
    We will now bound term $\RN{3}$.  Using Cauchy-Schwarz's inequality, we have
    
    \begin{align}
        \Expt{ S^\prime }{  \error(f_{(i)}(x_i), y_i ) - \error(f_{(i)}(x_i), y_i )  \error(f_{(j)}(x_j), y_j ) }^2 &\le  \Expt{ S^\prime }{  \error(f_{(i)}(x_i), y_i ) }^2 \Expt{S^\prime}{1 -   \error(f_{(j)}(x_j), y_j ) }^2 \\
        &\le \frac{1}{4} \,.\label{eq:term1_lem12}
    \end{align}
    
    Note that since $(x_i,y_i)$, $(x_j ,y_j )$, $(x,y)$, and $(x^\prime, y^\prime)$ are all from same distribution $\calDm$, we directly incorporate the bounds on term $\RN{1}$ and $\RN{2}$ from the proof of Lemma 9 in \citet{bousquet2002stability}. Combining that with \eqref{eq:term1_lem12} and our definition of hypothesis stability in \codref{cond:hypothesis_stability}, we have the required claim. 
    
    
    % We now re-write term $\RN{1}$ as
    % \begin{align*}
    %         &\Expt{S^\prime, (x,y), (x^\prime,y^\prime) }{ \error(\wh f(x), y ) \error(\wh f(x^\prime), y^\prime ) - \error(\wh f(x), y ) \error(f_{(i)}(x_i), y_i )} \\ & \qquad = \Expt{ S^\prime, (x,y), (x^\prime,y^\prime) }{ \error(\wh f(x), y ) \error(\wh f  (x^\prime), y^\prime ) - \error(\wh f ^\prime(x), y ) \error(f_{(i)}(x^\prime), y^\prime )} \tag{Exchanging $(x_i, y_i)$ with $(x^\prime, y^\prime)$ in the second term} \\
    %         & \qquad = \Expt{ S^\prime, (x,y), (x^\prime,y^\prime) }{  \left(\error(\wh f(x), y )-  \error(f_{(i)}(x), y ) \right) \error(\wh f  (x^\prime), y^\prime )  } \\
    %         & \qquad  + \Expt{ S^\prime, (x,y), (x^\prime,y^\prime) }{  \left(\error(f_{(i)}(x), y ) -\error(\wh f ^\prime(x), y ) \right) \error(\wh f  (x^\prime), y^\prime )}  \\
    %         & \qquad +\Expt{ S^\prime, (x,y), (x^\prime,y^\prime) }{  \left( \error(\wh f  (x^\prime), y^\prime ) -  \error(f_{(i)}(x^\prime), y^\prime ) \right) \error(\wh f ^\prime(x), y ) }  \,, \numberthis \label{eq:term1_final}
    % \end{align*}
    % where $\wh f^\prime$ is the classifier obtained by training on $ S^\prime_{(i)} \cup \{ (x^\prime, y^\prime) \} $. Similarly we can re-write term $\RN{2}$ as 
    % \begin{align*}
    %     & \Expt{ S^\prime }{  \error(f_{(i)}(x_i), y_i )  \error(f_{(j)}(x_j), y_j ) -  \error(\wh f(x), y ) \error(f_{(i)}(x_i), y_i )} \\
    %     &\quad  = \Expt{ S^\prime, (x,y), (x^\prime,y^\prime)}{  \error(f^{\prime\prime}_{(i)}(x), y )  \error(f_{(j)}^{\prime}(x^\prime), y^\prime ) -  \error(\wh f(x), y ) \error(f_{(i)}(x_i), y_i )} \tag{Exchanging $(x_i, y_i)$ with $(x, y)$ and $(x_j, y_j)$ with $(x^\prime, y^\prime)$ in the first term}\\
    %     &\quad = \Expt{ S^\prime, (x,y), (x^\prime,y^\prime)}{  \error(f^{\prime\prime}_{(j)}(x), y )  \error(f_{(i)}^{\prime}(x^\prime), y^\prime ) -  \error(\wh f^\prime (x), y ) \error(f^\prime_{(j)}(x^\prime), y^\prime )} \tag{Exchanging $(x_i, y_i)$ and $(x_j, y_j)$ and then replacing $(x_j, y_j)$ with $(x^\prime, y^\prime)$ in the second term} \\
    %     & \quad = \Expt{ S^\prime, (x,y), (x^\prime,y^\prime) }{  \left( \error(f_{(i)}^{\prime}(x^\prime), y^\prime )   -  \error(\wh f^{\prime\prime}  (x^\prime), y^\prime ) \right)  \error(f^{\prime\prime}_{(j)}(x), y )   } \\
    %     & \quad  + \Expt{ S^\prime, (x,y), (x^\prime,y^\prime) }{  \left( \error(f^{\prime\prime}_{(j)}(x), y )  -\error(\wh f ^\prime(x), y ) \right) \error(\wh f^{\prime\prime}  (x^\prime), y^\prime )  }  \\
    %     & \quad+ \Expt{ S^\prime, (x,y), (x^\prime,y^\prime) }{  \left( \error(\wh f^{\prime\prime}  (x^\prime), y^\prime )  -  \error(f^\prime_{(j)}(x^\prime), y^\prime ) \right)  \error(\wh f^\prime (x), y ) }   \\
    %     & \quad = \Expt{ S^\prime, (x,y), (x^\prime,y^\prime) }{  \left( \error(f_{(i)}^{\prime}(x^\prime), y^\prime )   -  \error(\wh f (x^\prime), y^\prime ) \right)  \error(f_{(i)}(x_j), y_j )   } \\
    %     & \quad  + \Expt{ S^\prime, (x,y), (x^\prime,y^\prime) }{  \left( \error(f^{\prime\prime}_{(j)}(x), y )  -\error(\wh f (x), y ) \right) \error(\wh f^{\prime\prime}  (x_j), y_j )  }  \\
    %     & \quad+ \Expt{ S^\prime, (x,y), (x^\prime,y^\prime) }{  \left( \error(\wh f^{\prime\prime}  (x^\prime), y^\prime )  -  \error(f^\prime_{(j)}(x^\prime), y^\prime ) \right)  \error(\wh f^\prime (x^\prime), y^\prime ) }  \,, \numberthis \label{eq:term2_final}
    % \end{align*}
    % where $f^{\prime\prime}_{(j)}$ is trained on $S^\prime_{(j,i)} \cup {(x,y)}$, $f^{\prime}_{(i)}$ is trained on $S^\prime_{(j,i)} \cup {(x^\prime,y^\prime)}$, and $\wh f^{\prime\prime} $ is trained on $S^\prime_{(j)} \cup {(x,y)}$. Note in the last line we replaced $(x,y)$ by $(x_j, y_j)$ in the first term, replaced $(x^\prime,y^\prime)$ by $(x_j, y_j)$ in the second term and exchanged $(x_i,y_i)$ with $(x_j,y_j)$ and also $(x,y)$ and $(x^\prime, y^\prime)$
    
    
\end{proof}


% 
% 16th Century Version Control 
% 

% \onecolumn

% \section*{Supplementary Material}
% We will be using the following standard results
% on exponential concentration of random variables 
% all throughout the discussion:

% \begin{lemma}[Hoeffding's inequality for independent RVs~\citep{hoeffding1994probability}] Let $Z_1, Z_2, \ldots, Z_n$ be independent bounded random variables with $Z_i \in [a,b]$ for all $i$, then 
%     \begin{align*}
%         \prob\left( \frac{1}{n} \sum_{i=1}^n (Z_i - \Expo{Z_i}) \ge t \right) \le \exp{\left( -\frac{2nt^2}{(b-a)^2} \right) }
%     \end{align*} 
%     and 
%     \begin{align*}
%         \prob\left( \frac{1}{n} \sum_{i=1}^n (Z_i - \Expo{Z_i}) \le -t \right) \le \exp{\left( -\frac{2nt^2}{(b-a)^2} \right) }
%     \end{align*} 
%     for all $t \ge 0$. 
% \end{lemma}

% \begin{lemma}[Hoeffding's inequality for sampling with replacement~\citep{hoeffding1994probability}] \label{lem:hoeffding_sampling} Let $\calZ = (Z_1, Z_2, \ldots, Z_N)$ be a finite population of $N$ points with $Z_i \in [a.b]$ for all $i$. Let $X_1, X_2, \ldots X_n$ be a random sample drawn without replacement from $\calZ$. Then for all $t \ge 0$, we have 
%     \begin{align*}
%         \prob\left( \frac{1}{n} \sum_{i=1}^n (X_i - \mu ) \ge t \right) \le \exp{\left( -\frac{2nt^2}{(b-a)^2} \right) }
%     \end{align*} 
%     and 
%     \begin{align*}
%         \prob\left( \frac{1}{n} \sum_{i=1}^n (X_i - \mu ) \le -t \right) \le \exp{\left( -\frac{2nt^2}{(b-a)^2} \right) } \,,
%     \end{align*} 
%     where $\mu = \frac{1}{N} \sum_{i=1}^{N} Z_i$. 
% \end{lemma}

% We now discuss one condition that generalizes the exponential concentration to dependent random variables.
% \begin{condition}[Bounded difference inequality] \label{cond:BDC} Let $\calZ$ be some set and $\phi: \calZ^n \to \Real$. We say that $\phi$ satisfies the bounded difference assumption if 
% there exists $c_1, c_2, \ldots c_n \ge 0$ s.t. for all $i$, we have 
% \begin{align*}
%     \sup_{Z_1,Z_2, \ldots,Z_n, Z_i^\prime in \calZ^{n+1} } \abs{\phi (Z_1, \ldots, Z_i, \ldots, Z_n ) - \phi (Z_1, \ldots, Z_i^\prime, \ldots, Z_n ) } \le c_i \,.
% \end{align*} 
% \end{condition}

% \begin{lemma}[McDiarmid’s inequality~\citep{mcdiarmid1989}] \label{lem:McDiarmid} Let $Z_1, Z_2, \ldots, Z_n$ be independent random variables on set $\calZ$ and $\phi : \calZ^n \to \Real$ satisfy bounded difference assumption (\codref{cond:BDC}). Then for all $t>0$, we have 
%     \begin{align*}
%         \prob\left( \phi(Z_1, Z_2, \ldots, Z_n) - \Expo{\phi(Z_1, Z_2, \ldots, Z_n)} \ge t \right) \le \exp{\left( -\frac{2t^2}{\sum_{i=1}^n c_i^2} \right) } 
%     \end{align*} 
%     and 
%     \begin{align*}
%         \prob\left( \phi(Z_1, Z_2, \ldots, Z_n) - \Expo{\phi(Z_1, Z_2, \ldots, Z_n)} \le -t \right) \le \exp{\left( -\frac{2t^2}{\sum_{i=1}^n c_i^2} \right) } \,
%     \end{align*} 
% \end{lemma}


% \section{Proofs from \secref{sec:ERM_training}}\label{app:proof_erm}

% \textbf{Additional notation {} {}} Let $m_1$ be the number of mislabeled points ($\wt S_M$) and $m_2$ be the number of correctly labeled points ($\wt S_C$). Note $m_1 + m_2 = m$. 


% \subsection{Proof of \thmref{thm:error_ERM}}


% \begin{proof}[Proof of \lemref{lem:fit_mislabeled}] 
%     The main idea of our proof is to regard 
%     the clean portion of the data 
%     ($S \cup \wt S_C$) as fixed.   
%     Then, there exists a classifier $f^*$ 
%     that is optimal over draws 
%     of the mislabeled data $\wt S_M$. 
% % 
%     % 
%     Formally, 
%     \begin{align}
%     f^* \defeq \argmin_{f \in \calF} \error_{\widecheck {\calD}} (f) \,, \label{eq:modified_ERM}
%     \end{align}
%     where $$\widecheck \calD = \frac{n}{m+n} \calS + \frac{m_1}{m+n} \wt \calS_C  + \frac{m_2}{m+n}\calDm \,.$$ That is, $\widecheck \calD$ a combination of 
%     the \emph{empirical distribution} 
%     over correctly labeled data $S \cup \wt S_C$
%     % in $S\cup \wt S$ 
%     and the (population) distribution 
%     over mislabeled data $\calDm$.
%     Recall that 
%     \begin{align}
%     \wh f \defeq \argmin_{f \in \calF} \error_{\calS \cup \wt S} (f) \,. \label{eq:orig_ERM}
%     \end{align}
%     % 
%     % 
%     Since, $\widehat f$ minimizes 0-1 error 
%     on $S \cup \wt S$, using ERM optimality on \eqref{eq:orig_ERM},  
%     we have 
%     \begin{align}
%         \error_{\calS \cup \wt \calS}(\widehat f) \le \error_{
%             \calS \cup \wt \calS}(f^*) \,.    \label{eq:step1}
%     \end{align}
%     Moreover, since $f^*$ is independent of $\wt S_M$, using Hoeffding's bound,
%     % \footnote{For a fully rigorous argument,
%     % refer to the complete proof in App.~\ref{app:proof_erm}.} 
%     we have with probability at least $1-\delta$ that
%     \begin{align}
%       \error_{\wt \calS_M}(f^*) \le \error_{ \calDm}(f^*) +  \sqrt{\frac{\log(1/\delta)}{2 m_1}} \,. \label{eq:step2} 
%     \end{align}
%     %$ 
%     %for some constant $c_1\le 1/2$. 
%     Finally, since $f^*$ is the optimal classifier on $\widecheck \calD$, 
%     we have 
%     \begin{align}
%         \error_{\widecheck \calD}(f^*) \le \error_{\widecheck \calD}(\widehat f) \label{eq:step3}
%     \end{align}
%      Now to relate \eqref{eq:step1} and \eqref{eq:step3}, we can re-write the \eqref{eq:step2} as follows: 
%     \begin{align}
%         \error_{\calS \cup \wt\calS}(f^*) \le \error_{ \widecheck \calD}(f^*) +  \frac{m_1}{m+n}\sqrt{\frac{\log(1/\delta)}{2 m_1}} \,. \label{eq:step4} 
%     \end{align}
%     Now we combine equations \eqref{eq:step1}, \eqref{eq:step4}, and \eqref{eq:step3}, to get 
%     \begin{align}
%         \error_{\calS \cup \wt \calS}(\wh f) \le \error_{\widecheck \calD}(\wh f) +  \frac{m_1}{m+n}\sqrt{\frac{\log(1/\delta)}{2 m_1}} \,, 
%     \end{align}
%     which implies 
%     \begin{align}
%         \error_{ \wt \calS_M}(\wh f) \le \error_{\calDm}(\wh f) + \sqrt{\frac{\log(1/\delta)}{2 m_1}} \,. \label{eq:lemma1_final}
%     \end{align}
%     Since $\wt S$ is obtained by randomly labeling an unlabeled dataset, we assume $2m_1 \approx m$ \footnote{Formally, with probability at least $1-\delta$, we have  $(m - 2m_1)\le \sqrt{m\log(1/\delta)/2}$ }. Moreover, using $\error_{\calDm} = 1 - \error_{\calD}$ we obtain the desired result.   
%     % Combining the above steps and using the fact 
%     % that $\error_\calD = 1- \error_{\calDm} $, 
%     % we obtain the desired result.
% \end{proof}

% \begin{proof}[Proof of \lemref{lem:mislabeled_error}]
%     Recall $\error_{\wt S} (f) = \frac{m_1}{m} \error_{\wt S_M}(f) + \frac{m_2}{m} \error_{\wt S_C}(f)$. Hence, we have 
%     \begin{align}
%         2\error_{\wt S}(f) - \error_{\wt S_M}(f) - \error_{\wt S_C}(f) &= \left(\frac{2m_1}{m} \error_{\wt S_M}(f) - \error_{\wt S_M}(f)\right) + \left(\frac{2m_2}{m} \error_{\wt S_C}(f) - \error_{\wt S_C}(f)\right) \\ &= \left(\frac{2m_1}{m} - 1\right) \error_{\wt S_M}(f) + \left(\frac{2m_2}{m} - 1 \right)\error_{\wt S_C} (f) \,.
%     \end{align} 
%     Since the dataset is randomly labeled, with probability at least $1-\delta$, we have  $\left(\frac{2m_1}{m} - 1\right) \le \sqrt{\frac{\log(1/\delta)}{2m}}$. Similarly, we have with probability at least $1-\delta$, $\left(\frac{2m_2}{m} - 1\right) \le \sqrt{\frac{\log(1/\delta)}{2m}}$. Using union bound, we have with probability at least $1-\delta$
%     % \begin{align}
%     %     2\error_{\wt S} - \error_{\wt S_M}(f) - \error_{\wt S_C}(f) \le \sqrt{\frac{\log(2/\delta)}{2m}} \left(\error_{\wt S_M}(f) + \error_{\wt S_C}(f) \right) \le 2\sqrt{\frac{\log(2/\delta)}{2m}} \,. \label{eq:lemma2_final}
%     % \end{align}
%     \begin{align}
%         2\error_{\wt S} - \error_{\wt S_M}(f) - \error_{\wt S_C}(f) \le \sqrt{\frac{\log(2/\delta)}{2m}} \left(\error_{\wt S_M}(f) + \error_{\wt S_C}(f) \right) \,. \label{eq:lemma2_prefinal}
%     \end{align}
%     With re-arranging $\error_{\wt S_M}(f) + \error_{\wt S_C}(f)$ and using the inequality $ 1- a\le \frac{1}{1+a} $, we have  
%     \begin{align}
%         2\error_{\wt S} - \error_{\wt S_M}(f) - \error_{\wt S_C}(f) \le 2\error_{\wt \calS} \sqrt{\frac{\log(2/\delta)}{2m}}  \,. \label{eq:lemma2_final}
%     \end{align}

%     % We obtain the desired result by using 
% \end{proof}

% \begin{proof}[Proof of \lemref{lem:clear_error}]
% % Recall 0-1 error on each point  $(x,y) \in S \cup \wt S$ is given by $\I{ f(x)\ne y}$.
% In the set of correctly labeled points $S \cup \wt S_C$, we have $S$ as a random subset of $S \cup \wt S_C$. Hence, using Hoeffding's inequality for sampling without replacement (\lemref{lem:hoeffding_sampling}), we have with probability at least $1-\delta$
% \begin{align}
%     \error_{\wt \calS_c} (\wh f)- \error_{\calS \cup \wt \calS_C}( \wh f) \le  \sqrt{\frac{\log(1/\delta)}{2m_2}} \,.
% \end{align}
% Re-writing $\error_{\calS \cup \wt \calS_C}( \wh f)$ as $\frac{m_2}{m_2 + n} \error_{\wt \calS_C }(\wh f) + \frac{n}{m_2 + n} \error_{\calS }(\wh f)$, we have with probability at least $1-\delta$
% \begin{align}
%   \left(\frac{n}{n+m_2}\right) \left(\error_{\wt \calS_c} (\wh f)- \error_{\calS}( \wh f) \right) \le  \sqrt{\frac{\log(1/\delta)}{2m_2}} \,.
% \end{align}
% As before, assuming $2m_2 \approx m$, we have with probability at least $1-\delta$ 
% \begin{align}
%     \error_{\wt \calS_c} (\wh f)- \error_{\calS}( \wh f) \le \left(1+\frac{m_2}{n}\right)  \sqrt{\frac{\log(1/\delta)}{m}} \le 1.5 \sqrt{\frac{\log(1/\delta)}{m}} \,. \label{eq:lemma3_final}
% \end{align} 
% \end{proof}

% \begin{proof}[Proof of \thmref{thm:error_ERM}] 
%     Having established these core intermediate results, we can now combine above three lemmas to prove the main result. 
%     In particular, we bound the population error on clean data ($\error_\calD(\wh f)$) as follows:  
%     \begin{enumerate}[(i)]
%         \item First, use \eqref{eq:lemma1_final}, to obtain an upper bound on the population error on clean data, i.e., with probability at least $1-\delta/4$, we have
%         \begin{align}
%             \error_{ \calD} (\wh f) \le 1 - \error_{ \wt \calS_M}(\wh f) + \sqrt{\frac{\log(4/\delta)}{m}} \,. 
%         \end{align}
%         \item  Second, use \eqref{eq:lemma2_final}, to relate the error on the mislabeled fraction with error on clean portion of randomly labeled data and error on whole randomly labeled dataset, i.e., with probability at least $1-\delta/2$, we have 
%         \begin{align}
%             - \error_{\wt S_M}(f) \le \error_{\wt S_C}(f) - 2\error_{\wt S}  + \sqrt{\frac{\log(4/\delta)}{2m}}  \,. 
%         \end{align} 
%         \item Finally, use \eqref{eq:lemma3_final} to relate the error on the clean portion of randomly labeled data and error on clean training data, i.e., with probability $1-\delta/4$, we have 
%         \begin{align}
%             \error_{\wt \calS_C} (\wh f)\le - \error_{\calS}( \wh f) + \left(1 + \frac{m}{2n} \right) \sqrt{\frac{\log(4/\delta)}{m}} \,. 
%         \end{align} 
%     \end{enumerate}

%     Using union bound on the above three steps, we have with probability at least $1-\delta$: 
%     \begin{align}
%         \error_\calD (\wh f) \le \error_{\calS}(\wh f)   + 1 - 2\error_{\wt \calS}(\wh f)   + (1/\sqrt{2} + 2.5)  \sqrt{\frac{\log(4/\delta)}{m}} \,.
%     \end{align}
%     Note that $(1/\sqrt{2} + 2.5)$ is a loose constant. In experiments, we use the ratio $\frac{m}{n}$
%     %  the exact error $\error_{\wt \calS}(\wh f)$ 
%     to evaluate R.H.S.    
% \end{proof}

% \subsection{Proof of \propref{prop:rademacher}}

% \begin{proof}[Proof of \propref{prop:rademacher}]
%     For a classifier $ f: \calX \to \{-1, 1\}$, we have $1 - 2\,\indict{ f(x) \ne y} = y \cdot f(x)$. Hence, by definition of $\error$, we have 
%     \begin{align}
%         1 -2\error_{\wt \calS}(f) = \frac{1}{m}\sum_{i=1}^m y_i \cdot f(x_i) \le \sup_{f \in \calF} \, \frac{1}{m} \sum_{i=1}^m y_i \cdot f(x_i)  \,. \label{eq:error_rademacher}
%     \end{align}
%     Note that for fixed inputs $(x_1, x_2, \ldots, x_m)$ in $\wt S$, $(y_1, y_2, \ldots y_m)$ are random labels. Define $\phi_1 (y_1, y_2, \ldots, y_m) \defeq \sup_{f \in \calF} \, \frac{1}{m} \sum_{i=1}^m y_i \cdot f(x_i)$. We have the following bounded difference condition on $\phi_1$. For all i, 
%     \begin{align}
%         \sup_{y_1, \ldots y_m, y_i^\prime \in \{-1, 1\}^{m+1} } \abs{ \phi_1 (y_1,\ldots, y_i, \ldots, y_m) - \phi_1 (y_1,\ldots, y_i^\prime, \ldots, y_m)  } \le 1/m \,. \label{cond1_rademacher}
%     \end{align} 
    
%     Similarly define $\phi_2 (x_1, x_2, \ldots, x_m) \defeq \Expt{ y_i \sim_U \{-1, 1\}  }{ \sup_{f \in \calF} \, \frac{1}{m}  \sum_{i=1}^m y_i \cdot f(x_i)}$. We have the following bounded difference condition on $\phi_2$. For all i,
%     \begin{align}
%         \sup_{x_1, \ldots x_m, x_i^\prime \in \calX^{m+1} } \abs{ \phi_2 (x_1,\ldots, x_i, \ldots, x_m) - \phi_1 (x_1,\ldots, x_i^\prime, \ldots, x_m)  } \le 1/m \,. \label{cond2_rademacher}
%     \end{align}
%     Using McDiarmid’s inequality (\lemref{lem:McDiarmid}) twice with Condition \eqref{cond1_rademacher} and \eqref{cond2_rademacher}, with probability at least $1-\delta$, we have
%     \begin{align}
%         \sup_{f \in \calF} \, \frac{1}{m} \sum_{i=1}^m y_i \cdot f(x_i)  - \Expt{x,y}{\sup_{f \in \calF} \, \frac{1}{m} \sum_{i=1}^m y_i \cdot f(x_i) } \le \sqrt{\frac{2\log(2/\delta)}{m}} \label{eq:final_rademacher}
%     \end{align} 
%     Combining \eqref{eq:error_rademacher} and \eqref{eq:final_rademacher}, we obtain the desired result. 
% \end{proof}


% \subsection{Proof of \thmref{thm:error_regularized_ERM}}

% Proof of \thmref{thm:error_regularized_ERM} follows similar to the proof of \thmref{thm:error_ERM}. Note that the same results in \lemref{lem:fit_mislabeled}, \lemref{lem:mislabeled_error}, and \lemref{lem:clear_error} hold in the regularized ERM case. However, the arguments in the proof of \lemref{lem:fit_mislabeled} changes slightly. Hence, we state and prove a lemma parallel to \lemref{lem:fit_mislabeled} for completeness. 

% \begin{lemma} \label{lem:lemma1_reg}
%     Assume the same setup as \thmref{thm:error_regularized_ERM}. 
%     Then for any $\delta >0$, with probability at least  $1-\delta$ 
%     over the random draws of mislabeled data $\wt S_M$, we have 
%     \begin{align}
%         \error_\calD(\widehat f)  \le 1 -\error_{\wt \calS_M}(\widehat f) + \sqrt{\frac{\log(1/\delta)}{m}}\,. 
%     \end{align} 
% \end{lemma}
% \begin{proof}
%     The main idea of the proof remains the same, i.e. regard 
%     the clean portion of the data 
%     ($S \cup \wt S_C$) as fixed.   
%     Then, there exists a classifier $f^*$ 
%     that is optimal over draws 
%     of the mislabeled data $\wt S_M$. 

    
%     Formally, 
%     \begin{align}
%     f^* \defeq \argmin_{f \in \calF} \error_{\widecheck {\calD}} (f)  + \lambda R(f) \,, \label{eq:modified_ERM_reg}
%     \end{align}
%     where $$\widecheck \calD = \frac{n}{m+n} \calS + \frac{m_1}{m+n} \wt \calS_C  + \frac{m_2}{m+n}\calDm \,.$$ That is, $\widecheck \calD$ a combination of 
%     the \emph{empirical distribution} 
%     over correctly labeled data $S \cup \wt S_C$
%     % in $S\cup \wt S$ 
%     and the (population) distribution 
%     over mislabeled data $\calDm$.
%     Recall that 
%     \begin{align}
%     \wh f \defeq \argmin_{f \in \calF} \error_{\calS \cup \wt S} (f) + \lambda R(f) \,. \label{eq:orig_ERM_reg}
%     \end{align}
%     % 
%     % 
%     Since, $\widehat f$ minimizes 0-1 error 
%     on $S \cup \wt S$, using ERM optimality on \eqref{eq:orig_ERM},  
%     we have 
%     \begin{align}
%         \error_{\calS \cup \wt \calS}(\widehat f) + \lambda R(\wh f) \le \error_{
%             \calS \cup \wt \calS}(f^*) + \lambda R(f^*) \,.    \label{eq:step1_reg}
%     \end{align}
%     Moreover, since $f^*$ is independent of $\wt S_M$, using Hoeffding's bound,
%     % \footnote{For a fully rigorous argument,
%     % refer to the complete proof in App.~\ref{app:proof_erm}.} 
%     we have with probability at least $1-\delta$ that
%     \begin{align}
%       \error_{\wt \calS_M}(f^*) \le \error_{ \calDm}(f^*) +  \sqrt{\frac{\log(1/\delta)}{2 m_1}} \,. \label{eq:step2_reg} 
%     \end{align}
%     %$ 
%     %for some constant $c_1\le 1/2$. 
%     Finally, since $f^*$ is the optimal classifier on $\widecheck \calD$, 
%     we have 
%     \begin{align}
%         \error_{\widecheck \calD}(f^*) + \lambda R(f^*) \le \error_{\widecheck \calD}(\widehat f) + \lambda R(\wh f) \label{eq:step3_reg}
%     \end{align}
%      Now to relate \eqref{eq:step1_reg} and \eqref{eq:step3_reg}, we can re-write the \eqref{eq:step2_reg} as follows: 
%     \begin{align}
%         \error_{\calS \cup \wt\calS}(f^*) \le \error_{ \widecheck \calD}(f^*) +  \frac{m_1}{m+n}\sqrt{\frac{\log(1/\delta)}{2 m_1}} \,. \label{eq:step4_reg} 
%     \end{align}
%     After adding $\lambda R(f^*)$ on both sides in \eqref{eq:step4_reg}, we combine equations \eqref{eq:step1_reg}, \eqref{eq:step4_reg}, and \eqref{eq:step3_reg}, to get 
%     \begin{align}
%         \error_{\calS \cup \wt \calS}(\wh f) \le \error_{\widecheck \calD}(\wh f) +  \frac{m_1}{m+n}\sqrt{\frac{\log(1/\delta)}{2 m_1}} \,, 
%     \end{align}
%     which implies 
%     \begin{align}
%         \error_{ \wt \calS_M}(\wh f) \le \error_{\calDm}(\wh f) + \sqrt{\frac{\log(1/\delta)}{2 m_1}} \,. \label{eq:lemma_reg_final}
%     \end{align}
%     Similar as before, since $\wt S$ is obtained by randomly labeling an unlabeled dataset, we assume 
%     $2m_1 \approx m$. Moreover, using $\error_{\calDm} = 1 - \error_{\calD}$ we obtain the desired result. 
% \end{proof}
% % \begin{proof}[Proof of ]
    
% % \end{proof}

% \subsection{Proof of \thmref{thm:multiclass_ERM}}

% We first state and prove lemmas parallel to three lemmas used in the proof of balanced binary case. Then we combine the results in the three lemmas to obtain the result in \thmref{thm:multiclass_ERM}. 

% Before stating the result, we define mislabeled distribution $\calDm$ for any $\calD$. While $\calDm$ and $\calD$ share 
% the same marginal distribution over $\calX$, 
% the distribution over labels $y$ 
% given an input $x\sim \calD_\calX$ is changed.
% In particular, for any $x$, the pdf over $y$ is changed to:  
% $p_{\calDm} (\cdot \vert x) \defeq \frac{1 - p_{\calD}(\cdot \vert x)}{k - 1}$.

% \begin{lemma} \label{lem:fit_mislabeled_multi}
%     Assume the same setup as \thmref{thm:multiclass_ERM}. 
%     Then for any $\delta >0$, with probability at least  $1-\delta$ 
%     over the random draws of mislabeled data $\wt S_M$, we have 
%     \begin{align}
%         \error_\calD(\widehat f)  \le (k-1)\left(1 -\error_{\wt \calS_M}(\widehat f)\right) + (k-1)\sqrt{\frac{\log(1/\delta)}{m}}\,. \label{eq:lemma1_multi}
%     \end{align}   
% \end{lemma} 

% \begin{proof}
%     The main idea of the proof remains the same, i.e. regard 
%     the clean portion of the data 
%     ($S \cup \wt S_C$) as fixed. 
%     Then, there exists a classifier $f^*$ 
%     that is optimal over draws 
%     of the mislabeled data $\wt S_M$. 
    
%     However, we need to be careful while relating population error on mislabeled data with population accuracy on clean data.   
%     While for binary classification,  we could upper bound $\error_{\wt \calS_M}$ 
%     with $1-\error_\calD$  (in the proof of \lemref{lem:fit_mislabeled}), 
%     for multiclass classification, 
%     error on the mislabeled data 
%     and accuracy on the clean data 
%     in the population 
%     are not so directly related.  
%     To establish \eqref{eq:lemma1_multi},
%     we break the error on the 
%     (unknown) mislabeled data 
%     into two parts: one term corresponds 
%     to predicting the true label on mislabeled data, 
%     and the other corresponds to predicting 
%     neither the true label 
%     nor the assigned (mis-)label.  
%     Finally, we relate these errors to their
%     population counterparts to establish \eqref{eq:lemma1_multi}. 
    
%     Formally, 
%     \begin{align}
%     f^* \defeq \argmin_{f \in \calF} \error_{\widecheck {\calD}} (f)  + \lambda R(f) \,, \label{eq:modified_ERM_reg2}
%     \end{align}
%     where $$\widecheck \calD = \frac{n}{m+n} \calS + \frac{m_1}{m+n} \wt \calS_C  + \frac{m_2}{m+n}\calDm \,.$$ That is, $\widecheck \calD$ a combination of 
%     the \emph{empirical distribution} 
%     over correctly labeled data $S \cup \wt S_C$
%     % in $S\cup \wt S$ 
%     and the (population) distribution 
%     over mislabeled data $\calDm$.
%     Recall that 
%     \begin{align}
%     \wh f \defeq \argmin_{f \in \calF} \error_{\calS \cup \wt S} (f) + \lambda R(f) \,. \label{eq:orig_ERM_reg2}
%     \end{align}
%     % 
%     % 
%     Following the exact steps from the proof of \lemref{lem:lemma1_reg}, with probability at least $1-\delta$, we have  
%     \begin{align}
%         \error_{ \wt \calS_M}(\wh f) \le \error_{\calDm}(\wh f) + \sqrt{\frac{\log(1/\delta)}{2 m_1}} \,. \label{eq:lemma1_final_multi_prev}
%     \end{align}
%     Similar to before, since $\wt S$ is obtained by randomly labeling an unlabeled dataset, we assume 
%     $\frac{k}{k-1} m_1 \approx m$. 
    
%     Now we will relate $\error_\calDm (\wh f)$ with $\error_{\calD}(\wh f)$. Let $y^T$ denote the (unknown) true label for a mislabeled point $(x, y)$ (i.e., label before replacing it with a mislabel). 
%     \begin{align}    
%          \Expt{(x, y) \in \sim \calDm}{\indict{ \wh f(x) \ne y }}  &= \underbrace{\Expt{(x, y) \in \sim \calDm}{\indict{ \wh f(x) \ne y \land \wh f(x) \ne y^T}}}_{\RN{1}} + \underbrace{\Expt{(x, y) \in \sim \calDm}{\indict{ \wh f(x) \ne y \land \wh f(x) = y^T}}}_{\RN{2}} \,. \label{eq:excess_term}
%     \end{align}
%     Clearly, term 2 is one minus the accuracy on the clean unseen data, i.e. 
%     \begin{align}
%         \RN{2} = 1 - \Expt{{x,y} \sim \calD}{ \indict{ \wh f(x) \ne y}} = 1- \error_{\calD}(\wh f) \,. \label{eq:term1}    
%     \end{align}
%     Next, we  relate term 1 with the error on the unseen clean data. We show that term 1 is equal to the error on the unseen clean data scaled by $\frac{k-2}{k-1}$ where $k$ is the number of labels. Using the definition of mislabeled distribution $\calDm$,  we have 
%     \begin{align}
%         \RN{1} = \frac{1}{k-1} \left( \Expt{(x, y) \in \sim \calD}{ \sum_{i \in \calY \land i\ne y}  \indict{ \wh f(x) \ne i \land \wh f(x) \ne y}} \right) = \frac{k-2}{k-1} \error_{\calD}(\wh f) \,.\label{eq:term2}
%     \end{align}    

%     Combining the result in \eqref{eq:term1}, \eqref{eq:term2} and \eqref{eq:excess_term}, we have 
%     \begin{align}
%         \error_{\calDm}(\wh f) = 1- \frac{1}{k-1} \error_{\calD}(\wh f) \,.\label{eq:combine_terms}
%     \end{align}
%     Finally, combining the result in \eqref{eq:combine_terms} with equation \eqref{eq:lemma1_final_multi_prev}, we have with probability $1-\delta$, 
%     \begin{align}
%       \error_{\calD}(\wh f) \le  (k-1) \left( 1- \error_{ \wt \calS_M}(\wh f) \right)  + (k-1) \sqrt{\frac{k \log(1/\delta)}{ 2(k-1)m}} \,. \label{eq:lemma1_final_multi}
%     \end{align}
% \end{proof}

% \begin{lemma} \label{lem:mislabeled_error_multi}
%     Assume the same setup as \thmref{thm:multiclass_ERM}.  Then for any $\delta >0$, with probability at least $1-\delta$ over the random draws of $\wt S$, we have  
%     % \begin{align}
%         $$\abs{k\error_{\wt \calS}(\widehat f) - \error_{\wt \calS_C}(\widehat f) -  (k-1)\error_{\wt \calS_M}(\widehat f) } \le  2k\sqrt{\frac{\log(4/\delta)}{2m}}\,. $$ % \label{eq:lemma2}
%     % \end{align}   
%     %  for some constant $c_3 \le 1.0\,$.
% \end{lemma} 


% \begin{proof}
%     Recall $\error_{\wt S} (f) = \frac{m_1}{m} \error_{\wt S_M}(f) + \frac{m_2}{m} \error_{\wt S_C}(f)$. Hence, we have 
%     \begin{align}
%         k\error_{\wt S}(f) - (k-1)\error_{\wt S_M}(f) - \error_{\wt S_C}(f) &= (k-1)\left(\frac{k m_1}{(k-1) m} \error_{\wt S_M}(f) - \error_{\wt S_M}(f)\right) + \left(\frac{km_2}{m} \error_{\wt S_C}(f) - \error_{\wt S_C}(f)\right) \\ &= k \left[ \left(\frac{m_1}{m} - \frac{k-1}{k}\right) \error_{\wt S_M}(f) + \left(\frac{m_2}{m} - \frac{1}{k} \right) \error_{\wt S_C} (f) \right] \,.
%     \end{align} 
%     Since the dataset is randomly labeled, we have with probability at least $1-\delta$, $\left(\frac{m_1}{m} - \frac{k-1}{k}\right) \le \sqrt{\frac{\log(1/\delta)}{2m}}$. Similarly, we have with probability at least $1-\delta$, $\left(\frac{m_2}{m} - \frac{1}{k}\right) \le \sqrt{\frac{\log(1/\delta)}{2m}}$. Using union bound, we have with probability at least $1-\delta$
%     % \begin{align}
%     %     2\error_{\wt S} - \error_{\wt S_M}(f) - \error_{\wt S_C}(f) \le \sqrt{\frac{\log(2/\delta)}{2m}} \left(\error_{\wt S_M}(f) + \error_{\wt S_C}(f) \right) \le 2\sqrt{\frac{\log(2/\delta)}{2m}} \,. \label{eq:lemma2_final}
%     % \end{align}
%     \begin{align}
%         k\error_{\wt S}(f) - (k-1)\error_{\wt S_M}(f) - \error_{\wt S_C}(f)  \le k \sqrt{\frac{\log(2/\delta)}{2m}} \left(\error_{\wt S_M}(f) + \error_{\wt S_C}(f) \right) \,. \label{eq:lemma2_final_multi}
%     \end{align}

%     % We obtain the desired result by using 
% \end{proof}

% \begin{lemma} \label{lem:clear_error_multi}
%     Assume the same setup as \thmref{thm:multiclass_ERM}. 
%     Then for any $\delta >0$, with probability at least $1-\delta$ 
%     over the random draws of $\wt S_C$ and $S$, we have 
%     % \begin{align}
%         $$\abs{\error_{\wt \calS_C}(\widehat f) - \error_{\calS}(\widehat f) } \le 1.5 \sqrt{\frac{k\log(2/\delta)}{2m}}\,.$$ %\label{eq:lemma3}
%     % \end{align}   
%     % for some constant $c_2 \le 1.2\,$.
% \end{lemma} 
% \begin{proof}
%     % Recall 0-1 error on each point  $(x,y) \in S \cup \wt S$ is given by $\I{ f(x)\ne y}$.
%     In the set of correctly labeled points $S \cup \wt S_C$, we have $S$ as a random subset of $S \cup \wt S_C$. Hence, using Hoeffding's inequality for sampling without replacement (\lemref{lem:hoeffding_sampling}), we have with probability at least $1-\delta$
%     \begin{align}
%         \error_{\wt \calS_c} (\wh f)- \error_{\calS \cup \wt \calS_C}( \wh f) \le  \sqrt{\frac{\log(1/\delta)}{2m_2}} \,.
%     \end{align}
%     Re-writing $\error_{\calS \cup \wt \calS_C}( \wh f)$ as $\frac{m_2}{m_2 + n} \error_{\wt \calS_C }(\wh f) + \frac{n}{m_2 + n} \error_{\calS }(\wh f)$, we have with probability at least $1-\delta$
%     \begin{align}
%       \left(\frac{n}{n+m_2}\right) \left(\error_{\wt \calS_c} (\wh f)- \error_{\calS}( \wh f) \right) \le  \sqrt{\frac{\log(1/\delta)}{2m_2}} \,.
%     \end{align}
%     As before, assuming $km_2 \approx m$, we have with probability at least $1-\delta$ 
%     \begin{align}
%         \error_{\wt \calS_c} (\wh f)- \error_{\calS}( \wh f) \le \left(1+\frac{m_2}{n}\right)  \sqrt{\frac{k\log(1/\delta)}{2m}} \le \left( 1 + \frac{1}{k}\right) \sqrt{\frac{k\log(1/\delta)}{2m}} \,. \label{eq:lemma3_final_multi}
%     \end{align} 
% \end{proof}

% \begin{proof}[Proof of \thmref{thm:multiclass_ERM}] 
%     Having established these core intermediate results, we can now combine above three lemmas. 
%     In particular, we bound the population error on clean data ($\error_\calD(\wh f)$) as follows:  
%     \begin{enumerate}[(i)]
%         \item First, use \eqref{eq:lemma1_final_multi}, to obtain an upper bound on the population error on clean data, i.e., with probability at least $1-\delta/4$, we have
%         \begin{align}
%             \error_{ \calD} (\wh f) \le (k-1)\left(1 - \error_{ \wt \calS_M}(\wh f) \right) + (k-1) \sqrt{\frac{k\log(4/\delta)}{2(k-1)m}} \,. 
%         \end{align}
%         \item  Second, use \eqref{eq:lemma2_final_multi}, to relate the error on the mislabeled fraction with error on clean portion of randomly labeled data and error on whole randomly labeled dataset, i.e., with probability at least $1-\delta/2$, we have 
%         \begin{align}
%             - (k-1)\error_{\wt S_M}(f) \le \error_{\wt S_C}(f) - k\error_{\wt S}  + k\sqrt{\frac{\log(4/\delta)}{2m}}  \,. 
%         \end{align} 
%         \item Finally, use \eqref{eq:lemma3_final_multi} to relate the error on the clean portion of randomly labeled data and error on clean training data, i.e., with probability $1-\delta/4$, we have 
%         \begin{align}
%             \error_{\wt \calS_C} (\wh f)\le - \error_{\calS}( \wh f) + \left(1 + \frac{m}{kn} \right) \sqrt{\frac{k\log(4/\delta)}{2m}} \,. 
%         \end{align} 
%     \end{enumerate}

%     Using union bound on the above three steps, we have with probability at least $1-\delta$: 
%     \begin{align}
%         \error_\calD (\wh f) \le \error_{\calS}(\wh f) + (k-1) - k\error_{\wt \calS}(\wh f)   + (\sqrt{k(k-1)} + k + \sqrt{k} + \frac{m}{n\sqrt{k}})  \sqrt{\frac{\log(4/\delta)}{2m}} \,.
%     \end{align}
%     % Note that $\frac{m}{n\sqrt{k}}$ is much smaller than the other terms in addition. Hence, we ignore this in the final bound. 
%     % Note that $(1/\sqrt{2} + 2.5)$ is a loose constant. In experiments, we use the ratio $\frac{m}{n}$
%     %  the exact error $\error_{\wt \calS}(\wh f)$ 
%     % to evaluate R.H.S.    
% \end{proof}

% \newpage
% \section{Proofs from \secref{sec:linear_models}}\label{app:proof_gd}

% We suppose that the parameters of the linear function 
% are obtained via gradient descent on 
% the following $L_2$ regularized problem: 
% \begin{align}
%     % n in denominator is avoided deliberately
%     \calL_S(w; \lambda) \defeq \sum_{i=1}^n{(w^Tx_i - y_i)^2} + \lambda \norm{w}{2}^2 \,, \label{eq:l2_MSE_app}   
% \end{align}
% where $\lambda\ge0$ is a regularization parameter. 
% We assume access to a clean dataset 
% $S = \{(x_i, y_i)\}_{i=1}^n \sim \calD^n$ 
% and randomly labeled dataset 
% $\wt S = \{(x_i, y_i)\}_{i=n+1}^{n+m} \sim \wt \calD^m$. 
% Let $\bX = [x_1, x_2, \cdots, x_{m+n}]$ 
% and $\by = [y_1, y_2, \cdots, y_{m+n}]$. 
% Fix a positive learning rate $\eta$ such that 
% $\eta \le 1/\left(\norm{\bX^T\bX}{\text{op}} + \lambda^2\right)$ 
% and an initialization $w_0 = 0$. 
% % \todos{Assumption made for simplicty}. 
% Consider the following gradient descent iterates 
% to minimize objective \eqref{eq:l2_MSE_app} on $S \cup \wt S$:
% \begin{align}
% w_t = w_{t-1} - \eta \grad_w \calL_{S \cup \wt S} (w_{t-1}; \lambda) \quad \forall t=1,2,\ldots \label{eq:GD_iterates_app}
% \end{align} 
% Then we have $\{ w_t\}$ converge to the limiting solution 
% $\wh w = \left( \bX^T\bX+\lambda \boldsymbol{I}\right)^{-1}\bX^T\by$. Define $\widehat f (x) \defeq f(x ; \wh w) $.  

% \subsection{\textcolor{red}{Errata}}

% We wish to correct the following error in the body: \codref{cond:error_stability} is not enough to guarantee the result in \thmref{thm:linear}. We now present a slightly stronger condition called \emph{hypothesis stability} under which we obtain a result similar to \thmref{thm:linear}. 

% This error doesn't change the main arguments of the proof where we show that the empirical train error is less than or equal to the leave-one-out error. We need a stronger condition to relate leave-one-out error with the population error of the original classifier. Specifically, while \codref{cond:error_stability} relates the average population error of leave-one-out classifiers with the population error of the original classifier, we need the new condition to show the concentration of the empirical leave-one-out error and  average population error of leave-one-out classifiers. 
% % main takeaway 

% Note that the new condition, while being stronger than the previous one, still doesn't imply generalization~\cite{bousquet2002stability,elisseeff2003leave,abou2019exponential}. Overall, the main results in \secref{sec:ERM_training} and takeaways of the paper remain unaffected by the error.  

% We now present the new condition and a corrected statement of \thmref{thm:linear}. Recall, for a given training set $S \sim \calD^n $, 
% we use $S_{(i)}$ to denote the training set $S$ 
% with the $i^{\text{th}}$ point removed.

% \begin{condition}[Hypothesis Stability] 
%     \label{cond:hypothesis_stability}
%     We have $\beta$ hypothesis stability 
%     if our training algorithm $\calA$ satisfies the following: 
%     \begin{align*}
%     % ${\sum_{i=1}^n \frac{\error_{\calD}( f(\calA, S_{(i)}))}{n} - \error_\calD(f(\calA, S))} \le \beta\,$.
%     \forall i \in \{1,2,\ldots, n\}, \quad  \Expt{\calS, (x,y) \in \calD}{ \abs{\error\left( f(x) ,y  \right) - \error\left( f_{(i)}(x), y \right) }} \le \frac{\beta}{n} \,,
%     \end{align*}
%     where $f_{(i)} \defeq f(\calA, S_{(i)})$ and $ f \defeq f(\calA, S)$.
% \end{condition}

% \begin{theorem}[Correct statement of \thmref{thm:linear}] \label{thm:new_linear}
%     Assume that this gradient descent algorithm satisfies \codref{cond:hypothesis_stability}
%     with $\beta=\calO(1)$.  
%     Then for any $\delta >0$, with probability at least $1-\delta$ 
%     over the random draws of datasets $\wt S$ and $S$, we have:
%     \begin{align}
%         \error_\calD(\widehat f) \le \error_\calS(\widehat f) + 1 - 2 \error_{\wt\calS}(\widehat f) + \left(\frac{1}{\sqrt{2}} + 1.5 \right) \sqrt{\frac{\log(4/\delta)}{m}} + \sqrt{\frac{4}{\delta}\left(\frac{1}{m} +\frac{3\beta}{m+n} \right)}  \,. \label{eq:gd_error}
%     \end{align} 
%     % for some constant $c\le 3.2$.
% \end{theorem}

% \subsection{Proof of \thmref{thm:new_linear}}
% We use a standard result from linear algebra, namely Shermann-Morrison formula~\citep{sherman1950adjustment} for matrix inversion:  

% \begin{lemma}[\citet{sherman1950adjustment}] \label{lem:sherman}
%     Suppose $\bA \in \Real^{n \times n}$ is an invertible square matrix and $u,v \in \Real^n$ are column vectors. Then $\bA + uv^T$ is invertible iff $1 + v^T \bA u \ne 0$ and in particular
%     \begin{align}
%         (\bA + u v^T)^{-1} = \bA^{-1}  - \frac{\bA^{-1} uv^T \bA^{-1} }{ 1 + v^T \bA^{-1} u} \,.
%     \end{align}   
% \end{lemma}
% \newcommand\byy[1]{\by_{\left(#1\right)}}
% \newcommand\bXX[1]{\bX_{\left(#1\right)}}
% \newcommand\ff[1]{\wh f_{\left(#1\right)}}

% For a given training set $S \cup \wt S_C$, define leave-one-out error on mislabeled points in the training data as $$\error_{\text{LOO}(\wt S_M) } = \frac{\sum_{(x_i, y_i) \in \wt S_M} \error( f_{(i)}( x_i), y_i)}{ \abs{\wt S_M }} \,, $$
% where $f_{(i)} \defeq f(\calA, (S \cup \wt S)_{(i)})$. To relate empirical leave-one-out error and population error with hypothesis stability condition, we use the following lemma:   

% \begin{lemma}[\citet{bousquet2002stability}] \label{lem:stability_error}
%     For the leave-one-out error, we have
%     \begin{align}
%         \Expo{ \left( \error_{\calDm}(\wh f) -\error_{\text{LOO}(\wt S_M) } \right)^2 } \le \frac{1}{2m_1}+  \frac{3\beta}{n + m}\,.
%     \end{align}   
%     % where $ f \defeq f(\calA, S \cup \wt S) $.
% \end{lemma}

% Proof of the above lemma is similar to the proof of  Lemma 9 in \citet{bousquet2002stability} and can be found in \appref{app:proof_lem_error}. 
% % 
% % Before presenting the result, we introduce some notation. 
% Before presenting the proof of \thmref{thm:new_linear}, we introduce some more notation. Let $\bX_{(i)}$ denote the matrix of covariates with $i^{\text{th}}$ point removed. Similarly let $\by_{(i)}$ be the array of responses with $i^{\text{th}}$ point removed. Define the corresponding regularized GD solution as $\wh w_{(i)} = \left( \bXX{i}^T\bXX{i}+\lambda \boldsymbol{I}\right)^{-1}\bXX{i}^T\byy{i}$. Define $\ff{i}(x) \defeq f(x ; \wh w_{(i)}) $.

% \begin{proof}[Proof of \thmref{thm:new_linear}]
%     Because squared loss minimization does not imply 0-1 error minimization, we cannot use arguments from \lemref{lem:fit_mislabeled}. This is the main technical difficulty. To compare the 0-1 error at a train point with an unseen point, 
%     we use the closed-form expression for $\widehat{w}$ and Shermann-Morrison formula to upper bound training error with leave-one-out cross validation error. 
    
%     The proof is divided into three parts: In part one, we show that 0-1 error on mislabeled points in the training set is lower than the error obtained by leave-one-out error at those points. In part two, we relate this leave-one-out error with the population error on mislabeled distribution using \codref{cond:hypothesis_stability}. While the empirical leave-one-out error is unbiased estimator of the average population error of leave-one-out classifiers, we need hypothesis stability to control the variance of empirical leave-one-out error. Finally in part three, we show that the error on the mislabeled training points can be estimated with just the randomly labeled and  clean training data (as in proof of \thmref{thm:error_ERM}).  

%     \textbf{Part 1 {} {}} First we relate training error with leave-one-out error.        
%     For any 
%     training point $(x_i, y_i)$ in $\wt S \cup S$, we have 
%     \begin{align}
%         \error(\wh f(x_i), y_i ) &= \indict{ y_i \cdot x_i^T \wh w < 0 } = \indict{ y_i \cdot x_i^T \left( \bX^T\bX+\lambda \boldsymbol{I}\right)^{-1}\bX^T\by < 0 } \\
%         &= \indict{ y_i \cdot x_i^T \underbrace{\left( \bXX{i}^T\bXX{i} + x_i ^T x_i +\lambda \boldsymbol{I}\right)^{-1}}_{\RN{1}} (\bXX{i}^T\byy{i} + y \cdot x_i) < 0 }
%     \end{align}
%     Letting $\bA = \left(\bXX{i}^T\bXX{i} +\lambda \boldsymbol{I}\right)$ and using \lemref{lem:sherman} on term 1, we have 
%     \begin{align}
%         \error(\wh f(x_i), y_i ) &= \indict{ y_i \cdot x_i^T \left[\bA^{-1} -  \frac{\bA^{-1} x_i x_i^T \bA^{-1}}{ 1 + x_i ^T \bA^{-1} x_i } \right] (\bXX{i}^T\byy{i} + y \cdot x_i) < 0 } \\
%         &= \indict{ y_i \cdot\left[ \frac{ x_i^T \bA^{-1} ( 1 + x_i ^T \bA^{-1} x_i ) -  x_i^T \bA^{-1} x_i x_i^T \bA^{-1}}{ 1 + x_i ^T \bA ^{-1}x_i } \right] (\bXX{i}^T\byy{i} + y \cdot x_i) < 0 } \\
%         &= \indict{ y_i \cdot\left[ \frac{ x_i^T \bA^{-1}}{ 1 + x_i ^T \bA ^{-1}x_i } \right] (\bXX{i}^T\byy{i} + y \cdot x_i) < 0 } \,.
%     \end{align}

%     Since $1 + x_i^T \bA^{-1} x_i > 0$, we have 
%     \begin{align}
%         \error(\wh f(x_i), y_i ) &= \indict{ y_i \cdot x_i^T \bA^{-1} (\bXX{i}^T\byy{i} + y \cdot x_i) < 0 } \\
%         &= \indict{ x_i^T \bA^{-1} x_i +  y_i \cdot x_i^T \bA^{-1} (\bXX{i}^T\byy{i}) < 0 } \\
%         &\le \indict{ y_i \cdot x_i^T \bA^{-1} (\bXX{i}^T\byy{i}) < 0 } = \error(\ff{i}(x_i), y_i ) \,.\label{eq:LOO_error}
%     \end{align}

%     Using \eqref{eq:LOO_error}, we have 
%     \begin{align}
%         \error_{\wt \calS_M } (\wh f) \le \error_{\text{LOO} (S_M)} \defeq \frac{\sum_{(x_i, y_i) \in \wt S_M} \error(\ff{i}(x_i), y_i ) }{\abs{\wt \calS_M}}\label{eq:LOO_error_final}
%     \end{align}
%     \textbf{Part 2 {}{}} We now relate RHS in \eqref{eq:LOO_error_final} with the population error on mislabeled distribution. To do this, we leverage \codref{cond:hypothesis_stability} and \lemref{lem:stability_error}. In particular, we have 

%     \begin{align}
%         \Expt{\calS \cup \wt \calS_M }{ \left(\error_{\calDm}(\wh f) - \error_{\text{LOO} (S_M)}\right)^2 } \le \frac{1}{2m_1} + \frac{3\beta}{m+n} \,.
%     \end{align}

%     Using Chebyshev's inequality, with probability at least $1-\delta$, we have 
%     \begin{align}
%         \error_{\text{LOO} (S_M)} \le  \error_{\calDm}(\wh f)   + \sqrt{\frac{1}{\delta}\left(\frac{1}{2m_1} +\frac{3\beta}{m+n} \right)} \,. \label{eq:final_mislabeled_linear}
%     \end{align}
    

%     \textbf{Part 3 {}{}} Combining \eqref{eq:final_mislabeled_linear} and \eqref{eq:LOO_error_final}, we have 

%     \begin{align}
%         \error_{\wt \calS_M } (\wh f) \le \error_{\calDm}(\wh f)   + \sqrt{\frac{1}{\delta}\left(\frac{1}{2m_1} +\frac{3\beta}{m+n} \right)} \,. \label{eq:linear_parallel_lem1}
%     \end{align}

%     Compare \eqref{eq:linear_parallel_lem1}, with \eqref{eq:lemma1_final} in the proof of \lemref{lem:fit_mislabeled}. We obtain a similar relationship between $\error_{\wt \calS_M }$ and $\error_{\calDm}$ but with a polynomial concentration instead of exponential concentration. 
%     In addition, since we just use concentration arguments to relate mislabeled error with the error on clean portion and unlabeled portion, we can directly use the results in \lemref{lem:mislabeled_error} and \lemref{lem:clear_error}. Therefore, combining results in \lemref{lem:mislabeled_error}, \lemref{lem:clear_error}, and \eqref{eq:linear_parallel_lem1} with union bound, we have with probability at least $1-\delta$

%     \begin{align}
%         \error_\calD(\widehat f) \le \error_\calS(\widehat f) + 1 - 2 \error_{\wt\calS}(\widehat f) + \left(\frac{1}{\sqrt{2}} + 1.5 \right) \sqrt{\frac{\log(4/\delta)}{m}} + \sqrt{\frac{4}{\delta}\left(\frac{1}{m} +\frac{3\beta}{m+n} \right)}  \,.
%     \end{align}
    

       
% \end{proof}

% \subsection{Discussion on \codref{cond:hypothesis_stability}}

% The quantity in LHS of \codref{cond:hypothesis_stability} measures how much the function learned by the algorithm (in terms of error on unseen point) will change when one point in the training set is removed. 
% % Discussion on exponential concentration and stronger condition. 
% Notice that hypothesis stability implies error stability, i.e., \codref{cond:error_stability} ~\cite{bousquet2002stability}.  In summary, while error stability allowed us to relate the average population error of the leave-one-out classifiers with the population error of the original classifier, we need hypothesis stability condition to control the variance of the empirical leave-one-out error. 

% Additionally, we note that while the dominating term in the RHS of \thmref{thm:new_linear} matches with the dominating term in ERM bound in \thmref{thm:error_ERM}, there is a polynomial concentration term (dependence on $1/\delta$ instead of $\log(\sqrt{1/\delta})$) in  \thmref{thm:new_linear}. 
% Since with hypothesis stability, we just bound the variance,  the polynomial concentration is due to the use of Chebyshev's inequality instead of an exponential tail inequality (as in \lemref{lem:fit_mislabeled}).
% Recent works have highlighted that slightly stronger condition than hypothesis stability can be used to obtained an exponential concentration for leave-one-out error~\citep{abou2019exponential}, but we leave this for future work for now. 
% % We leave 
% % However, the constants 

% % we also want to highlight  

% \subsection{Formal statement and proof of  of \propref{prop:early_stop}}

% Before formally presenting the result, we will introduce some notation.  By $\calL_{S}(w)$, we denote 
% the objective in \eqref{eq:l2_MSE_app} with $\lambda=0$. 
% Assume Singular Value Decomposition (SVD) of $\bX$  as $\sqrt{n} \bU \bS^{1/2} \bV^T$. Hence $\bX^T \bX = \bV \bS \bV^T$.
% Consider the GD iterates defined in \eqref{eq:GD_iterates_app}. 
% % 
% We now derive closed form expression for the $t^\text{th}$ iterate of gradient descent:  
% % 
% \begin{align}
%     w_t = w_{t-1} + \eta \cdot \bX^T (\by - \bX w_{t-1}) = (\bI - \eta \bV \bS \bV^T )w_{k-1} + \eta \bX^T \by \,.
% \end{align}
% Rotating by $\bV^T$, we get 
% \begin{align}
%     \wt w_t = (\bI - \eta\bS )\wt w_{k-1} + \eta \wt \by \,, \label{eq:GD_recur}
% \end{align}
% where $\wt w_t = \bV^T w_t $ and $\wt \by = \bV^T \bX^T \by$. Assuming the initial point $w_0 = 0$ and applying the recursion in \eqref{eq:GD_recur}, we get
% \begin{align}
%     \wt w_t = \bS ^{-1} ( \bI - (\bI - \eta \bS)^k ) \wt \by \,, 
% \end{align} 
% Projecting solution back to the original space, we have 
% \begin{align}
%      w_t = \bV \bS ^{-1} ( \bI - (\bI - \eta \bS)^k ) \bV^T \bX^T \by \,, 
% \end{align} 
% % We will work with this GD solution at any iterate $t$ in the next proposition. 
% Define $f_t(x) \defeq f(x;w_t)$ as the solution at the $t^{\text{th}}$ iterate. 
% Let $\wt w_{\lambda} = \argmin_{w} \calL_\calS (w;\lambda) = (\bX^T \bX + \lambda \bI)^{-1} \bX^T \by = \bV (\bS + \lambda \bI )^{-1} \bV^T \bX^T \by $. 
% % ) \,,$ for all $t=1,2,\ldots\,.$ 
% and define $\wt f_\lambda(x) \defeq f(x;\wt w_\lambda)$ as the regularized solution. 
% Assume $\kappa$ be the condition number of the population covariance matrix 
% and 
% let $s_\text{min}$ be the minimum positive singular value of the empirical covariance matrix. Our proof idea is inspired from recent work on relating gradient flow solution and regularized solution for regression problems \citep{ali2018continuous}. We will use the following lemma in the proof: 
% \begin{lemma} \label{lem:ineq_soln}
%     For all $x \in [0,1]$ and for all $ k \in \mathbb{N}$, we have (a) $ \frac{kx}{1+kx} \le 1- (1-x)^k$ and (b) $ 1- (1-x)^k \le 2 \cdot \frac{kx}{kx+1} $.
%     %  where $g(c)$ is a constant dependent on $c$. For $c = 1$, $g(c) = 2.0$.   
% \end{lemma}
% \begin{proof}
%     % [Proof of \lemref{lem:ineq_soln}]
%     % Part (a) is easy. 
%     Using $ (1-x)^k \le \frac{1}{1+kx}$, we have part (a). For part (b), we numerically maximize $\frac{ (1+kx ) (1 - (1-x)^k) }{kx}$ for all $k\ge 1$ and for all $x \in [0, 1]$.  
% \end{proof}

% % 
% % Next, 

% \begin{prop}[Formal statement of \propref{prop:early_stop}] \label{prop:formal_early_stop}
% Let $\lambda = \frac{1}{t\eta}$. For a training point $x$, we have 
% \begin{align*}
%     \Expt{x \sim \calS}{(f_t(x) - \wt f_\lambda(x))^2} &\le c(t,\eta) \cdot \Expt{x \sim \calS}{f_t(x)^2} \,, %\label{eq:early_stop}
% \end{align*}
% where $c(t, \eta) \defeq \min( 0.25, \frac{1}{s_\text{min}^2 t^2 \eta^2})$. Similarly for a test point, we have 
% \begin{align*}
%     \Expt{x \sim \calD_\calX}{(f_t(x) - \wt f_\lambda(x))^2} &\le \kappa \cdot c(t,\eta) \cdot \Expt{x \sim \calD_\calX}{f_t(x)^2} \,. %\label{eq:early_stop}
% \end{align*}
% \end{prop} 

% \begin{proof}
%     %%%%%%%%%%%%% 
%     We want to analyze the expected squared difference output of regularized linear regression with regularization constant $\lambda = \frac{1}{\eta t}$ and gradient descent solution at $t^\text{th}$ iterate. We separately expand the algebraic expression for squared difference at a training point and a test point. 
%     % We start by considering the difference  
%     Then the main step is to show that  $\left[ \bS ^{-1} ( \bI - (\bI - \eta \bS)^k )  - (\bS + \lambda \bI )^{-1}\right] \preceq c(\eta, t) \cdot \bS ^{-1} ( \bI - (\bI - \eta \bS)^k ) $.

%     %%%%%%%%%%%%%
    
%   \textbf{Part 1 {} {}} 
%     First, we will analyze the squared difference of output at a training point (for simplicity, we refer to $S \cup \wt S$ as $S$), i.e. 
%     \begin{align}
%         \Expt{ x \sim \calS }{\left(f_t(x) - \wt f_\lambda (x)\right)^2} &= \norm{\bX w_t - \bX \wt w_\lambda}{2}^2 =   \norm{\bX \bV \bS ^{-1} ( \bI - (\bI - \eta \bS)^t ) \bV^T \bX^T \by - \bX \bV (\bS + \lambda \bI )^{-1} \bV^T \bX^T \by }{2}^2 \\
%         &= \norm{\bX \bV \left(\bS ^{-1} ( \bI - (\bI - \eta \bS)^t ) - (\bS + \lambda \bI )^{-1} \right) \bV^T \bX^T \by  }{2} \\
%         &=  \by^T \bV \bX \left( \underbrace{\bS ^{-1} ( \bI - (\bI - \eta \bS)^t ) - (\bS + \lambda \bI )^{-1}}_{\RN{1}} \right)^2 \bS \bV^T \bX^T \by \label{eq:train_GD_rel}
%         %  (\bX \bV \bS ^{-1} ( \bI - (\bI - \eta \bS)^k ) \bV^T \bX^T \by)^T \bX \bV \bS ^{-1} ( \bI - (\bI - \eta \bS)^k ) \bV^T \bX^T \by
%     \end{align}
%     We now separately consider term 1. Substituting $\lambda = \frac{1}{t \eta}$, we get
%     \begin{align}
%         \bS ^{-1} ( \bI - (\bI - \eta \bS)^t ) - (\bS + \lambda \bI )^{-1} &= \bS^{-1} \left( ( \bI - (\bI - \eta \bS)^t ) - (\bI + \bS^{-1} \lambda )^{-1}\right) \\
%         &= \underbrace{\bS^{-1} \left( ( \bI - (\bI - \eta \bS)^t ) - (\bI + ( \bS t \eta)^{-1}  )^{-1}\right)}_{\bA}
%     \end{align}

%     We now separately bound the diagonal entries in matrix $\bA$. 
%     With $s_i$, we denote $i^{\text{th}}$ diagonal entry of $\bS$. Note that since $ \eta\le 1/\norm{S}{\text{op}}$, for all $i$, $\eta s_i  \le 1$.  Consider $i^{\text{th}}$ diagonal term (which is non-zero) of the diagonal matrix $\bA$, we have 
%     \begin{align}
%         \bA_{ii} = \frac{1}{s_i} \left(  1 - (1 - s_i \eta)^t - \frac{t \eta s_i}{1 + t \eta s_i } \right) &=  \frac{1 - (1 - s_i \eta)^t}{s_i} \left( \underbrace{ 1 - \frac{t \eta s_i}{(1 + t \eta s_i)(1 - (1 - s_i \eta)^t)}}_{\RN{2}} \right) \\ 
%          &\le \frac{1}{2}\left[ \frac{1 - (1 - s_i \eta)^t}{ s_i} \right] \tag*{(Using \lemref{lem:ineq_soln} (b))} \,.
%     \end{align} 
%     Additionally, we can also show the following upper bound on term 2: 
%     \begin{align}
%          1 - \frac{t \eta s_i}{(1 + t \eta s_i)(1 - (1 - s_i \eta)^t)} &= \frac{(1 + t \eta s_i)(1 - (1 - s_i \eta)^t) - t \eta s_i }{(1 + t \eta s_i)(1 - (1 - s_i \eta)^t)} \\
%          & \le  \frac{ 1 -  (1 - s_i \eta)^t - t \eta s_i (1 - s_i \eta)^t}{(1 + t \eta s_i)(1 - (1 - s_i \eta)^t)} \\
%          & \le \frac{1}{t\eta s_i} \,. \tag{Using \lemref{lem:ineq_soln} (a)}
%         %  &\le \frac{1}{2}\left[ \frac{1 - (1 - s_i \eta)^t}{ s_i} \right] \tag*{(Using \lemref{lem:ineq_soln})} \,.
%     \end{align} 

%     Combining both the upper bounds on each diagonal entry $\bA_{ii}$, we have 
%     \begin{align}
%     \bA \preceq c_1(\eta, t) \cdot \bS^{-1} ( \bI - (\bI - \eta \bS)^t ) \,, \label{eq:upperbound_diagonal}
%     \end{align}
%     where $c_1(\eta, t ) = \min(0.5, \frac{1}{t s_i \eta })$. Plugging this into \eqref{eq:train_GD_rel}, we have 
%     \begin{align}
%         \Expt{ x \sim \calS }{\left(f_t(x) - \wt f_\lambda (x)\right)^2} &\le c(\eta, t) \cdot \by^T \bV \bX  \left( \bS^{-1} ( \bI - (\bI - \eta \bS)^t ) \right)^2 \bS \bV^T \bX^T \by \\
%         &=   c(\eta, t) \cdot \by^T \bV \bX  \left( \bS^{-1} ( \bI - (\bI - \eta \bS)^t ) \right) \bS \left( \bS^{-1} ( \bI - (\bI - \eta \bS)^t ) \right) \bV^T \bX^T \by \\
%         & =  c(\eta, t) \cdot \norm{\bX w_t}{2}^2 \\
%         &= c(\eta, t) \cdot  \Expt{ x \sim \calS }{\left(f_t(x) \right)^2} \,,
%     \end{align}
%     where $c(\eta, t ) = \min(0.25, \frac{1}{t^2 s^2_i \eta^2 })$.

%     \textbf{Part 2 {} {}} With $\bSigma$, we denote the underlying true covariance matrix. We now consider the squared difference of output at an unseen point: 
%     \begin{align}
%         \Expt{ x \sim \calD_{\calX} }{\left(f_t(x) - \wt f_\lambda (x)\right)^2} &= \Expt{x \sim \calD_{\calX}}{\norm{x^T w_t - x^T \wt w_\lambda}{2}} \\
%         &=   \norm{x^T \bV \bS ^{-1} ( \bI - (\bI - \eta \bS)^t ) \bV^T \bX^T \by - x^T \bV (\bS + \lambda \bI )^{-1} \bV^T \bX^T \by }{2} \\
%         &= \norm{x^T \bV \left(\bS ^{-1} ( \bI - (\bI - \eta \bS)^t ) - (\bS + \lambda \bI )^{-1} \right) \bV^T \bX^T \by  }{2} \\
%         &= \by^T \bV \bX \left( \bS ^{-1} ( \bI - (\bI - \eta \bS)^t ) - (\bS + \lambda \bI )^{-1} \right) \bV^T \bSigma \bV \\ &\qquad \qquad \qquad \qquad \qquad \left( (\bI - (\bI - \eta \bS)^t ) - (\bS + \lambda \bI )^{-1} \right) \bV^T \bX^T \by \\
%         &\le \sigma_{\text{max}} \cdot \by^T \bV \bX \left( \underbrace{\bS ^{-1} ( \bI - (\bI - \eta \bS)^t ) - (\bS + \lambda \bI )^{-1}}_{\RN{1}} \right)^2 \bV^T \bX^T \by \,, \label{eq:test_GD_rel}
%         %  (\bX \bV \bS ^{-1} ( \bI - (\bI - \eta \bS)^k ) \bV^T \bX^T \by)^T \bX \bV \bS ^{-1} ( \bI - (\bI - \eta \bS)^k ) \bV^T \bX^T \by
%     \end{align}
%     where $\sigma_{\text{max}}$ is the maximum eigenvalue of the underlying covariance matrix $\bSigma$. Using the upper bound on term 1 in \eqref{eq:upperbound_diagonal}, we have 
%     \begin{align}
%         \Expt{ x \sim \calD_{\calX} }{\left(f_t(x) - \wt f_\lambda (x)\right)^2} &\le \sigma_{\text{max}} \cdot c(\eta, t) \cdot \by^T \bV \bX  \left( \bS^{-1} ( \bI - (\bI - \eta \bS)^t ) \right)^2 \bV^T \bX^T \by \\
%         &=   \kappa \cdot c(\eta, t) \cdot \sigma_{\text{min}}\cdot \norm{\bV \left( \bS^{-1} ( \bI - (\bI - \eta \bS)^t ) \right) \bV^T \bX^T \by}{2}^2 \\
%         &\le \kappa \cdot c(\eta, t) \cdot \left[ \bV \left( \bS^{-1} ( \bI - (\bI - \eta \bS)^t ) \right) \bV^T \bX^T \right]^T \bSigma \\
%         &\qquad \qquad \qquad \qquad \qquad \left[ \bV \left( \bS^{-1} ( \bI - (\bI - \eta \bS)^t ) \right) \bV^T \bX^T \right] \by \\
%         & = \kappa \cdot c(\eta, t) \cdot \Expt{x \sim \calD_{\calX}}{\norm{x^T w_t}{2}} \,.
%     \end{align}
% % 
% % 
%     % Since $ \eta\le 1/\norm{S}{\text{op}}$, invoking \lemref{lem:ineq_soln} to upper bound term 1 with
% \end{proof}


% \newpage
% \section{Additional experiments and details}\label{app:exp}
% \newcommand\tab[1][1cm]{\hspace*{#1}}

% \subsection{Datasets} \label{sec:app_dataset}

% \textbf{Toy Dataset {} {}} Assume fixed constants $\mu$ and $\sigma$. For a given label $y$, we simulate features $x$ in our toy classification setup as follows: 
% \begin{align*}
%     x \defeq \texttt{concat} \left[ x_1, x_2\right] \quad \text{where} \quad  x_1 \sim  \calN( y \cdot \mu, \sigma^2 I_{d \times d}) \ \  \text{and} \ \  x_1 \sim  \calN( 0, \sigma^2 I_{d \times d}) \,.
% \end{align*}  
% % where $y$ is the true label and $x$ is the corresponding feature vector. 
% In experiements throughout the paper, we fix dimention $d=100$, $\mu = 1.0 $, and $\sigma = \sqrt{d}$. Intuitively, $x_1$ carries the information about the underlying label and $x_2$ is additional noise independent of the underlying label. 

% \textbf{CV datasets {} {}} We use MNIST~\citep{lecun1998mnist} and CIFAR10~\cite{krizhevsky2009learning}. 
% % For binary tasks, 
% We produce a binary variant from the multiclass classification problem by mapping classes $\{0,1,2,3,4\}$ to label $1$ and $\{ 5,6,7,8,9\}$ to label $-1$. For CIFAR dataset, we also use the standard data augementation of random crop and horizontal flip. PyTorch code is as follows: 

% \texttt{(transforms.RandomCrop(32, padding=4),\\
% \tab transforms.RandomHorizontalFlip())}

% \textbf{NLP dataset {} {}} We use IMDb Sentiment analysis~\citep{maas2011learning} corpus.  

% \subsection{Architecture Details} 

% All experiments were run on NVIDIA GeForce RTX 2080 Ti GPUs. We used PyTorch~\citep{NEURIPS2019a9015} and Keras with Tensorflow~\citep{abadi2016tensorflow} backend for experiments. 
% % , ELMo embeddings~\citep{Peters:2018}, and Hugging Face Transformers~\citep{wolf-etal-2020-transformers}. 

% \textbf{Linear model {} {}} For the toy dataset, we simulate a linear model with scalar output and the same number of parameters as the number of dimensions.   

% \textbf{Wide nets {} {}} To simulate the NTK regime, we experiment with $2-$layered wide nets. The PyTorch code for 2-layer wide MLP is as follows: 


% \texttt{ nn.Sequential( \\
% \tab     nn.Flatten(),\\
% \tab    nn.Linear(input\_dims, 200000, bias=True),\\
% \tab    nn.ReLU(),\\
% \tab    nn.Linear(200000, 1, bias=True)\\
% \tab     )}


% We experiment both (i) with the first layer fixed at random initialization; (ii)  and updating both layers' weights.     

% \textbf{Deep nets for CV tasks {} {}} We consider a 4-layered MLP. The PyTorch code for 4-layer MLP is as follows: 

% \texttt{ nn.Sequential(nn.Flatten(), \\
% \tab        nn.Linear(input\_dim, 5000, bias=True),\\
% \tab        nn.ReLU(),\\
% \tab        nn.Linear(5000, 5000, bias=True),\\
% \tab        nn.ReLU(),\\
% \tab        nn.Linear(5000, 5000, bias=True),\\
% \tab        nn.ReLU(),\\
% % \tab        nn.Linear(5000, 5000, bias=True),\\
% % \tab        nn.ReLU(),\\
% \tab        nn.Linear(1024, num\_label, bias=True)\\
% \tab        )}

% For MNIST, we use $1000$ nodes instead of $5000$ nodes in the hidden layer. 
% % 
% We also experiment with convolutional nets. In particular, we use ResNet18 \citep{he2016deep}. Implementation adapted from:  \url{https://github.com/kuangliu/pytorch-cifar.git}. 

% \textbf{Deep nets for NLP {} {}} We use a simple LSTM model with embeddings intialized with ELMo embeddings~\citep{Peters:2018}. Code adapted from: \url{https://github.com/kamujun/elmo_experiments/blob/master/elmo_experiment/notebooks/elmo_text_classification_on_imdb.ipynb} 

% We also evaluate our bounds with a BERT model. In particular, we fine-tune an off-the-shelf uncased BERT model~\citep{devlin2018bert}. Code adapted from Hugging Face Transformers~\citep{wolf-etal-2020-transformers}: \url{https://huggingface.co/transformers/v3.1.0/custom_datasets.html}. 


% \subsection{Additonal experiments}

% 1. SGD with linear models on cross entropy and MSE loss. 

% 2. CE loss and SGD. GD with MSE loss 

% 3. Binary MNIST with MLP. multiclass MNIST  

% \textbf{Results on CIFAR 10 {} {}} 
% % 
% We plot epoch wise error curve for results in \tabref{table:multiclass}. We observe the same trend as in \figref{fig:error_CIFAR10}. Additionally, we plot an \emph{oracle bound} obtained by tracking the error on mislabeled data which nevertheless were predicted as true label. To obtain an exact emprical value of the oracle bound, we need underlying true labels for the randomly labeled data. 
% % Note that our bound in \thmref{thm:multiclass_ERM}, lower bounds the accuracy as predicted by the oracle bound. 
% While with just access to extra unlabeled data we cannot calculate oracle bound, we note that the oracle bound is very tight and never violated in practice underscoring an importamt aspect of generalization in multiclass problems. This highlight that even a stronger conjecture may hold in multiclass classification, i.e., error on mislabeled data (where nevertheless true label was predicted) lower bounds the population error on the distribution of mislabeled data and hence, the error on (a specific) mislabeled portion predicts the population accuracy on clean data. 
% % 
% On the other hand, the dominating term of in \thmref{thm:multiclass_ERM} is loose when compared with the oracle bound. The main reason, we believe is the pessimistic upper bound in \eqref{eq:lemma1_final_multi_prev} in the proof of \lemref{lem:fit_mislabeled_multi}. We leave an investigation on this gap for future. 
% % of fit 

% % However, oracle bound highlights two . One,  



% \begin{figure}[h]
%     \centering 
%     % \vspace{-15pt}
%     % \includegraphics[width=0.9\linewidth]{example-image-a}
%     \includegraphics[width=0.4\linewidth]{figures/CIFAR10-FNN.pdf} \hfil
%     \includegraphics[width=0.4\linewidth]{figures/CIFAR10-Resnet.pdf}
%     % \includegraphics[width=0.9\linewidth]{figures/{CIFAR10_rn=0.1_lr=0.2_wd=0.005}.png}
%     % \vspace{-10pt}
%     \caption{ Per epoch curves for CIFAR10 corresponding results in \tabref{table:multiclass}. As before, we just plot the dominating term in the RHS of \eqref{eq:multiclass_ERM} as predicted bound. Additionally, we also plot the predicted lower bound by the error on mislabeled data which nevertheless were predicted as true label. We refer to this as ``Oracle bound''. See text for more details. 
%     % 
%     % except for the stopping point. 
%     % The bound predicted by RATT (RHS in \eqref{eq:multiclass_ERM}) is vacuous. 
%     }\label{fig:error_epoch_CIFAR10}
%     % \vspace{-15pt}
% \end{figure}


% \textbf{Results on CIFAR 100 {} {}} 
% % 
% On CIFAR100, our bound in \eqref{eq:multiclass_ERM} yields vacous bounds. However, the oracle bound as explained above yields tight guarantees in the initial phase of the learning (i.e., when learning rate is less than $0.1$). 

% \begin{figure}[h]
%     \centering 
%     % \vspace{-15pt}
%     % \includegraphics[width=0.9\linewidth]{example-image-a}
%     \includegraphics[width=0.4\linewidth]{figures/CIFAR100-Resnet.pdf}
%     % \includegraphics[width=0.9\linewidth]{figures/{CIFAR10_rn=0.1_lr=0.2_wd=0.005}.png}
%     % \vspace{-10pt}
%     \caption{ Predicted lower bound by the error on mislabeled data which nevertheless were predicted as true label with ResNet18 on CIFAR100. We refer to this as ``Oracle bound''. See text for more details. 
%     % 
%     % except for the stopping point. 
%     The bound predicted by RATT (RHS in \eqref{eq:multiclass_ERM}) is vacuous. 
%     }\label{fig:error_CIFAR100}
%     % \vspace{-15pt}
% \end{figure}


% % \paragraph{Experiments on CIFAR100} 



% \subsection{Hyperparameter Details}


% \textbf{\figref{fig:error_CIFAR10} {} {}} We use clean training dataset of size $40,000$. We fix the amount of unlabeled data at $20\%$ of the clean size, i.e. we include additional $8,000$ points with randomly assigned labels. We use test set of $10,000$ points. For both MLP and ResNet, we use SGD with an initial learning rate of $0.1$ and momentum $0.9$. We fix the weight decay parameter at $5\times 10^{-4}$. After $100$ epochs, we decay the learning rate to $0.01$. We use SGD batch size of $100$. 

% \textbf{\figref{fig:error_binary} (a) {} {}} We obtain a toy dataset according to the process described in \secref{sec:app_dataset}. We fix $d=100$ and create a dataset of $50,000$ points with balanced classes. Moreover, we sample additional covariates with the same procedure to create randomly labeled dataset. For both SGD and GD training, we use a fixed learning rate $0.1$.    

% \textbf{\figref{fig:error_binary} (b) {} {}} Similar to binary CIFAR, we use clean training dataset of size $40,000$ and fix the amount of unlabeled data at $20\%$ of the clean dataset size. To train wide nets, we use a fixed learning of $0.001$ with GD and SGD. We decide the weight decay parameter and the early stopping point that maximizes our generalization bound (i.e. without peeking at unseen data ).  We use SGD batch size of $100$. 

% \textbf{\figref{fig:error_binary} (c) {} {}} With IMDb dataset, we use a clean dataset of size $20,000$ and as before, fix the amount of unlabeled data at $20\%$ of the clean data. To train ELMo model, we use Adam optimizer with a fixed learning rate $0.01$ and weight decay $10^{-6}$ to minimize cross entropy loss. We train with batch size $32$ for 3 epochs. To fine-tune BERT model, we use Adam optimizer with learning rate $5\times 10^{-5}$ to minimize cross entropy loss. We train with a batch size of $16$ for 1 epoch.    

% \textbf{\tabref{table:multiclass} {} {}} For multiclass datasets, we train both MLP and ResNet with the same hyperparameters as described before. We sample a clean training dataset of size $40,000$ and fix the amount of unlabeled data at $20\%$ of the clean size. We use SGD with an initial learning rate of $0.1$ and momentum $0.9$. We fix the weight decay parameter at $5\times 10^{-4}$. After $30$ epochs for ResNet and after $50$ epochs for MLP, we decay the learning rate to $0.01$.  We use SGD with batch size $100$. 
% For \figref{fig:error_CIFAR100}, we use the same hyperparameters as 
% CIFAR10 training, except we now decay learning rate after $100$ epochs. 


% In all experiments, to identify the best possible accuracy on just the clean data, we use the exact same set of hyperparamters except the stopping point. We choose a stopping point that maximizes test performance. 

% \subsection{Summary of experiments }

% \begin{center}
%     \begin{table}[H] 
%         \centering
%         \begin{tabular}{|c|c|c|c|} 
%         \hline
%         Classification type & Model category & Model & Dataset  \\ [0.5ex] 
%         \hline
%         \hline
%         \multirow{9}{*}{Binary} & Low dimensional & Linear model & Toy Gaussain dataset  \\
%                         \cline{2-4}
%                          & \multirow{1}{*}{Overparameterized linear nets} 
%                         %  & Linear model & Toy Gaussain dataset \\
%                         %  \cline{3-4}
%                         %  & & 2-layer wide net& Toy Gaussain dataset \\
%                         %  \cline{3-4}
%                          & 2-layer wide net & Binary MNIST \\
%                          \cline{2-4}                 
%                          & \multirow{6}{*}{Deep nets} & \multirow{2}{*}{MLP} & Binary MNIST \\
%                          \cline{4-4}
%                          & &  & Binary CIFAR \\
%                          \cline{3-4}
%                          &  & \multirow{2}{*}{ResNet} & Binary MNIST \\
%                          \cline{4-4}
%                          & &  & Binary CIFAR \\
%                          \cline{3-4}
%                          &  & ELMo-LSTM model & IMDb Sentiment Analysis \\
%                          \cline{3-4}
%                          & & BERT pre-trained model & IMDb Sentiment Analysis \\
%         \hline
%         \multirow{5}{*}{Multiclass} & \multirow{5}{*}{Deep nets} & \multirow{2}{*}{MLP} & MNIST \\
%                         \cline{4-4} 
%                         & & & CIFAR10 \\                   
%                         \cline{3-4}
%                          &   & \multirow{3}{*}{ResNet} & MNIST \\
%                          \cline{4-4}
%                          &   & & CIFAR10 \\
%                          \cline{4-4}
%                          &   & & CIFAR100 \\
%         \hline
%         \end{tabular}
%         % \caption{Summary of experiments performed} \label{table:experiments}
%     \end{table}    
%     % \footnotetext[6]{We use both MSE loss and cross-entropy loss.}
%     % \footnotetext[6]{We try 2 variants: one with a fixed first layer and the other with both layers trainable.}
% \end{center}

% \newpage
% \section{Proof of \lemref{lem:stability_error}} \label{app:proof_lem_error}

% \begin{proof}[Proof of \lemref{lem:stability_error}]
%     Recall, we have a training set $S \cup \wt S_C$. We defined leave-one-out error on mislabeled points as $$\error_{\text{LOO}(\wt S_M) } = \frac{\sum_{(x_i, y_i) \in \wt S_M} \error( f_{(i)}( x_i), y_i)}{ \abs{\wt S_M }} \,, $$
%     where $f_{(i)} \defeq f(\calA, (S \cup \wt S)_{(i)})$. Define $S^\prime \defeq S \cup \wt S$. Assume $(x,y)$ and $(x^\prime,y^\prime)$ as i.i.d. samples from ${\calDm}$. 
%     Using Lemma 25 in \citet{bousquet2002stability}, we have
%     \begin{align*}
%         \Expo{ \left( \error_{\calDm}(\wh f) -\error_{\text{LOO}(\wt S_M) } \right)^2 } \le & \Expt{ S^\prime, (x,y), (x^\prime,y^\prime) }{ \error(\wh f(x), y ) \error(\wh f(x^\prime), y^\prime )} - 2 \Expt{ S^\prime, (x,y) }{ \error(\wh f(x), y ) \error(f_{(i)}(x_i), y_i )} \\
%         & + \frac{m_1-1}{m_1}\Expt{ S^\prime }{  \error(f_{(i)}(x_i), y_i )  \error(f_{(j)}(x_j), y_j )} + \frac{1}{m_1} \Expt{ S^\prime }{  \error(f_{(i)}(x_i), y_i ) } \,. \numberthis \label{eq:main_reln}
%     \end{align*}
%     We can rewrite the equation above as : 
%     \begin{align*}
%         \Expo{ \left( \error_{\calDm}(\wh f) -\error_{\text{LOO}(\wt S_M) } \right)^2 } \le &  \, \underbrace{\Expt{ S^\prime, (x,y), (x^\prime,y^\prime) }{ \error(\wh f(x), y ) \error(\wh f(x^\prime), y^\prime ) - \error(\wh f(x), y ) \error(f_{(i)}(x_i), y_i )}}_{\RN{1}} \\
%         & + \underbrace{\Expt{ S^\prime }{  \error(f_{(i)}(x_i), y_i )  \error(f_{(j)}(x_j), y_j ) -  \error(\wh f(x), y ) \error(f_{(i)}(x_i), y_i )}}_{\RN{2}} \\ &+ \underbrace{\frac{1}{m_1} \Expt{ S^\prime }{  \error(f_{(i)}(x_i), y_i ) - \error(f_{(i)}(x_i), y_i )  \error(f_{(j)}(x_j), y_j ) }}_{\RN{3}} \,. \numberthis \label{eq:main_reln2}
%     \end{align*}
    
%     We will now bound term $\RN{3}$.  Using Schwarz's inequality, we have
    
%     \begin{align}
%         \Expt{ S^\prime }{  \error(f_{(i)}(x_i), y_i ) - \error(f_{(i)}(x_i), y_i )  \error(f_{(j)}(x_j), y_j ) }^2 &\le  \Expt{ S^\prime }{  \error(f_{(i)}(x_i), y_i ) }^2 \Expt{S^\prime}{1 -   \error(f_{(j)}(x_j), y_j ) }^2 \\
%         &\le \frac{1}{4} \label{eq:term1_lem12}
%     \end{align}
    
%     Note that since $(x_i,y_i)$, $(x_j ,y_j )$, $(x,y)$, and $(x^\prime, y^\prime)$ are all from same distribution $\calDm$, we directly incorporate the bounds on term $\RN{1}$ and $\RN{2}$ from proof of Lemma 9 in \citet{bousquet2002stability}. Combining that with \eqref{eq:term1_lem12} and our definition of hypothesis stability in \codref{cond:hypothesis_stability}, we have the required claim. 
    
    
%     % We now re-write term $\RN{1}$ as
%     % \begin{align*}
%     %         &\Expt{S^\prime, (x,y), (x^\prime,y^\prime) }{ \error(\wh f(x), y ) \error(\wh f(x^\prime), y^\prime ) - \error(\wh f(x), y ) \error(f_{(i)}(x_i), y_i )} \\ & \qquad = \Expt{ S^\prime, (x,y), (x^\prime,y^\prime) }{ \error(\wh f(x), y ) \error(\wh f  (x^\prime), y^\prime ) - \error(\wh f ^\prime(x), y ) \error(f_{(i)}(x^\prime), y^\prime )} \tag{Exchanging $(x_i, y_i)$ with $(x^\prime, y^\prime)$ in the second term} \\
%     %         & \qquad = \Expt{ S^\prime, (x,y), (x^\prime,y^\prime) }{  \left(\error(\wh f(x), y )-  \error(f_{(i)}(x), y ) \right) \error(\wh f  (x^\prime), y^\prime )  } \\
%     %         & \qquad  + \Expt{ S^\prime, (x,y), (x^\prime,y^\prime) }{  \left(\error(f_{(i)}(x), y ) -\error(\wh f ^\prime(x), y ) \right) \error(\wh f  (x^\prime), y^\prime )}  \\
%     %         & \qquad +\Expt{ S^\prime, (x,y), (x^\prime,y^\prime) }{  \left( \error(\wh f  (x^\prime), y^\prime ) -  \error(f_{(i)}(x^\prime), y^\prime ) \right) \error(\wh f ^\prime(x), y ) }  \,, \numberthis \label{eq:term1_final}
%     % \end{align*}
%     % where $\wh f^\prime$ is the classifier obtained by training on $ S^\prime_{(i)} \cup \{ (x^\prime, y^\prime) \} $. Similarly we can re-write term $\RN{2}$ as 
%     % \begin{align*}
%     %     & \Expt{ S^\prime }{  \error(f_{(i)}(x_i), y_i )  \error(f_{(j)}(x_j), y_j ) -  \error(\wh f(x), y ) \error(f_{(i)}(x_i), y_i )} \\
%     %     &\quad  = \Expt{ S^\prime, (x,y), (x^\prime,y^\prime)}{  \error(f^{\prime\prime}_{(i)}(x), y )  \error(f_{(j)}^{\prime}(x^\prime), y^\prime ) -  \error(\wh f(x), y ) \error(f_{(i)}(x_i), y_i )} \tag{Exchanging $(x_i, y_i)$ with $(x, y)$ and $(x_j, y_j)$ with $(x^\prime, y^\prime)$ in the first term}\\
%     %     &\quad = \Expt{ S^\prime, (x,y), (x^\prime,y^\prime)}{  \error(f^{\prime\prime}_{(j)}(x), y )  \error(f_{(i)}^{\prime}(x^\prime), y^\prime ) -  \error(\wh f^\prime (x), y ) \error(f^\prime_{(j)}(x^\prime), y^\prime )} \tag{Exchanging $(x_i, y_i)$ and $(x_j, y_j)$ and then replacing $(x_j, y_j)$ with $(x^\prime, y^\prime)$ in the second term} \\
%     %     & \quad = \Expt{ S^\prime, (x,y), (x^\prime,y^\prime) }{  \left( \error(f_{(i)}^{\prime}(x^\prime), y^\prime )   -  \error(\wh f^{\prime\prime}  (x^\prime), y^\prime ) \right)  \error(f^{\prime\prime}_{(j)}(x), y )   } \\
%     %     & \quad  + \Expt{ S^\prime, (x,y), (x^\prime,y^\prime) }{  \left( \error(f^{\prime\prime}_{(j)}(x), y )  -\error(\wh f ^\prime(x), y ) \right) \error(\wh f^{\prime\prime}  (x^\prime), y^\prime )  }  \\
%     %     & \quad+ \Expt{ S^\prime, (x,y), (x^\prime,y^\prime) }{  \left( \error(\wh f^{\prime\prime}  (x^\prime), y^\prime )  -  \error(f^\prime_{(j)}(x^\prime), y^\prime ) \right)  \error(\wh f^\prime (x), y ) }   \\
%     %     & \quad = \Expt{ S^\prime, (x,y), (x^\prime,y^\prime) }{  \left( \error(f_{(i)}^{\prime}(x^\prime), y^\prime )   -  \error(\wh f (x^\prime), y^\prime ) \right)  \error(f_{(i)}(x_j), y_j )   } \\
%     %     & \quad  + \Expt{ S^\prime, (x,y), (x^\prime,y^\prime) }{  \left( \error(f^{\prime\prime}_{(j)}(x), y )  -\error(\wh f (x), y ) \right) \error(\wh f^{\prime\prime}  (x_j), y_j )  }  \\
%     %     & \quad+ \Expt{ S^\prime, (x,y), (x^\prime,y^\prime) }{  \left( \error(\wh f^{\prime\prime}  (x^\prime), y^\prime )  -  \error(f^\prime_{(j)}(x^\prime), y^\prime ) \right)  \error(\wh f^\prime (x^\prime), y^\prime ) }  \,, \numberthis \label{eq:term2_final}
%     % \end{align*}
%     % where $f^{\prime\prime}_{(j)}$ is trained on $S^\prime_{(j,i)} \cup {(x,y)}$, $f^{\prime}_{(i)}$ is trained on $S^\prime_{(j,i)} \cup {(x^\prime,y^\prime)}$, and $\wh f^{\prime\prime} $ is trained on $S^\prime_{(j)} \cup {(x,y)}$. Note in the last line we replaced $(x,y)$ by $(x_j, y_j)$ in the first term, replaced $(x^\prime,y^\prime)$ by $(x_j, y_j)$ in the second term and exchanged $(x_i,y_i)$ with $(x_j,y_j)$ and also $(x,y)$ and $(x^\prime, y^\prime)$
    
    
% \end{proof}

\end{document}
% \grid
% \grid
% \grid
% \grid
% \grid
% \grid
% \grid
% \grid
