%!TEX root = main.tex
\section{Introduction}
\label{sec:intro}

In this paper, we consider a new type of \emph{spatial sparse recovery} problem, that is to infer the fine-grained spatial distribution of certain density data in a region based on the aggregate observations recorded for each of its subregions. Such a spatial sparse recovery problem is of particular interests to many big data applications, where acquiring fine-grained spatial data involves either additional technical overhead or privacy issues. For example, a telecommunication service provider may only be able to monitor the aggregate mobile traffic on each cell tower (base station), but want to infer the fine-grained spatial distribution of cell phone activities for capacity planning, device installation, urban ecology~\cite{bcici_mobihoc15}, and the study of population density estimation~\cite{douglass2015high}. Another example is to infer the spatial distribution of population densities (e.g., voter population for a certain party) based on the aggregate population observed at sparsely scattered polling stations. A further example is for an Internet media provider or retailer, e.g., Google, Amazon, etc., to reconstruct a fine-grained geographical distribution of its users only from aggregated user counts observed at certain datacenters or points of presence (PoPs). The ability of spatial recovery from coarse aggregation will be critical in all these cases, as tracking the position of each individual may incur either technical overhead or privacy concerns.

However, the above-mentioned problem is very challenging. % and has seldom been studied. 
A straightforward solution is to have a patched piece-wise constant estimation by assuming the density is uniformly distributed within each subregion in which there is an aggregate observation. However, this approach gives a non-smooth piece-wise constant spatial field, which offers little value in terms of discovering hot spots. 
Moreover, the densities estimated in this way will jumps between neighboring subregions, ignoring the local continuity of the spatial densities across subregions. Such local continuity, however, is present in many spatial data, e.g., cell phone/Internet activities, which are highly dependent on underlying factors such as urban geography, area functionality, and population density, all of which are usually smoothly changing. 
Alternatively, we may use spatial smoothing techniques \cite{Sanga13} developed in statistics in the past decade to smoothen the patched estimation. 
However, nearly all existing spatial smoothing techniques \cite{Wood08, Sanga13, guillas2010bivariate} aim to reconstruct a spatial field of densities based on \emph{sampled} observations, e.g., recovering a spatial field of temperatures according to temperature readings at a few sample points, instead of based on coarse-grained \emph{aggregate observations}. As a result, applying existing spatial smoothing techniques to our new problem may violate the necessary constraint that the estimated spatial data in each subregion must sum up to its corresponding aggregate observation in the first place, leading to systematic errors.
%However, the estimates derived this way may violate the necessary constraint that the estimated activity densities in a subregion covered by a base station should sum up to the volume observed on that base station in the first place.

In this paper, we propose a new approach called \emph{Constrained Spatial Smoothing} (CSS) to overcome the difficulties mentioned above. Specifically, we aim to reconstruct a spatial field $f$ of densities over a region $\Omega$ by penalizing the ``roughness'' of $f$ (especially across subregions), subject to the constraint that the aggregation of discretized values of $f$ in each patched subregion $\Omega_{B_i}$ equals to the aggregate value $z_i$ observed in $\Omega_{B_i}$. Our new approach is different from traditional spatial smoothing due to the presence of the additional constraint.
We propose an Alternating Direction Method of Multipliers (ADMM) \cite{boyd2011distributed} algorithm to decouple the problem into the alternated minimizations of two subproblems: a quadratic program (QP) and a spatial smoothing subproblem, where we use the QP to iteratively enforce the aggregation constraints, while solving the spatial smoothing subproblem with a recently proposed finite element technique called Spatial Spline Regression (SSR) \cite{Sanga13}. 
Moreover, our proposed algorithm not only leverages the local continuity to perform spatial sparse recovery, but is also able to take into account additional external information %such as the number of schools, number of bus stops, population, etc.,
in the underlying geographical region 
%as a regression add-on component 
to further enhance recovery performance.
%Last but not least, our CSSE algorithm is not limited to handling the problem of inferring cell phone activities distribution, but also can be applied to a variety of sparse recovery problem where intrinsic smoothness exists.


% Another important contribution of the paper is that we conducted extensive evaluation of the proposed algorithms in a case study of reconstructing mobile phone activity distributions in Milan, Italy from base station observations.
% Our evaluation is based on the Milan Call Detail Records (CDR) dataset, provided by Telecom Italia Mobile 
% ~\cite{bigdatachallenge}, which contains the phone call and SMS activity records of two months in each grid square of 235m $\times$ 235m in the city of Milan, Italy. 
% We consider a region that consists of 2726 grid squares in an irregularly bounded region in the city of Milan. To stress-test the algorithm performance, we assume we only know the aggregate phone activities observed on 100 or 200 base stations and aim to recover the entire spatial field of phone activities.  
%Note that since Milan has about 200 base stations, such a fine granularity of data means that the dataset was collected using means beyond merely observing base station volumes (with legal permissions). The availability of this dataset makes it feasible to effectively evaluate our newly proposed spatial sparse recovery procedure, which has not been possible in the past. 
% We also use another geographical attribute dataset available from the Municipality of Milan's Open Data website \cite{barlacchi2015multi} as the additional external attribute data to improve performance.


%First, in many real-world scenarios, the observation data points are very sparse, while what needs to be reconstructed is the entire spatial field. For example, in the cell phone activity problem,  base stations are normally 400-800 meters apart even in dense urban areas and are commonly spaced 2–3 km apart in suburban areas.


We conducted extensive evaluation of the proposed algorithm in a case study of reconstructing mobile phone activity distributions of Milan, Italy from aggregated observations on base stations.
Results suggest that our algorithm achieves significant improvement, as compared to various other methods including the state-of-the-art Spatial Spline Regression (SSR) \cite{Sanga13} approach. 
Our algorithm can recover the fine-grained spatial distribution of cell phone activities in Milan only from observations on $200$ base stations, with a mean absolute percentage error of $0.309$, representing a $26.3\%$ improvement from SSR.

%while the MRAE of the base line method we mentioned above is $\red{0.000}$.

% The remainder of this paper is organized as follows.
% In Sec.~\ref{sec:formulation}, we formulate the problem of spatial field reconstruction from coarse aggregate observations.
% In Sec.~\ref{sec:SSR}, we describe existing solutions, including a state-of-the-art Spatial Spline Regression (SSR) technique for spatial smoothing. 
% In Sec.~\ref{sec:model}, we propose our Constrained Spatial Smoothing method which respects both the local continuity in the spatial field and the aggregation constraints at the same time. In Sec.~\ref{sec:simu}, we conduct extensive evaluation in comparison with various other methods through a solid and extensive case study of cell phone activity density estimation in the city of Milan. We discuss related literature in Sec.~\ref{sec:related} and conclude the paper in Sec.~\ref{sec:conclude}.







 %In fact, since a powered mobile phone always communicates wirelessly with the closest base station, knowledge of the location of the base stations together with the number of cell phones associated with them only provides a very coarse-grained spatial distribution of cell phone activities, from which a fine-grained distribution needs to be recovered. 

%Second, existing spatial smoothing techniques \cite{wood2003thin, Wood08, Ramsay02, Sanga13, guillas2010bivariate} all aim to reconstruct a spatial field of densities based on \emph{sampled} observations, e.g., recovering a spatial field of temperatures according to temperature readings at a few points, instead of coarse-grained aggregate observations. Therefore, applying existing spatial smoothing techniques to our new problem cannot ensure that inferred spatial data in each subregion sums up to its corresponding aggregate observation, leading to systematic errors.



%while we are hoping to reveal the spatial pattern of cell phone activity density to find ``hot spots'' that the activity densities are high to help with tasks such as determining the locations of new BSs or cell towers, 


%Third, besides the spatial smoothness property, the method also fails to utilize external information of different regions, such as the number of schools, number of bus stops, population, etc., which can be helpful for activity density estimation.

%due to a lack of publicly available dataset about fine-grained cell phone call activity distributions in the past, it has also been hard to evaluate the performance of such spatial sparse recovery techniques.

%Mobile devices and smart phones are almost universally adopted in modern society, generating an enormous amount of behavioral traces that can be used to provide useful insights into the studies of human activity and urban planning~\cite{barlacchi2015multi}. Among such diverse behavioral traces generated from mobile devices, one kind of information that researchers have long been interested in is the spatial distribution of cell phone activities including call descriptions and SMS activities. On one hand, understanding the distribution of cell phone activities can be helpful for service providers to install new base stations (BS) or small cell towers in the city to improve coverage and avoid hot spots. On the other hand,  the estimated cell phone activities can also be leveraged in a variety of emerging big data challenges, including but not limited to population density estimation~\cite{douglass2015high,deville2014dynamic}, human mobility prediction \cite{cho2011friendship}, city structure modeling~\cite{louail2014mobile}, urban ecology pattern studies~\cite{bcici_mobihoc15} and so on. 

%Despite the great value contained in such kind of data, in reality, the collection of fine-grained spatial distribution of cell phone activities at a large scale is always hampered by both technical overhead and complicated privacy issues. First, although mobile phone tracking and positioning can be done via a number of mature technologies \cite{wiki:mobiletracking}, including GPS, via multilateration of radio signals, or via signal strengths to nearby antennas, etc., telecommunication service providers may have no business incentives to install the additional technical components required for positioning.  Moreover, due to delicate privacy and human right issues, in most countries, it is either illegal or not ethical for service providers to track a user's position (except in special cases such as emergencies~\cite{wiki:mobiletracking}).  Therefore, in reality, what a service provider can readily collect without sparking any privacy concerns is the volume of cell phone activities associated with each base station.



%only have access to the aggregated activity density in BSs, i.e., the total activity density of the region that the BS is in charge of. Therefore, we can only use the BS activities to infer the entire region's activity distribution. However, the extreme sparsity of BS activity observations pose great challenge on estimating the whole region activity distribution, therefore the problem of cell phone activity estimation is seldom visited before and few effective approach has been proposed to our best knowledge.

%In this paper, we conduct an in-depth analysis on the problem of inferring spatial distribution of cell phone activities from sparse aggregated observations. More specifically, our problem is that, given aggregated cell phone activities in BSs that are sparsely distributed in a city, recovering the spatial distribution of cell phone activities over the whole region that the BSs are in charge of.



