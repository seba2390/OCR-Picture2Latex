%!TEX root = main.tex
\section{Problem Formulation}
\label{sec:formulation}

% influence of lambda
% \begin{figure}[t]
%         %\hspace{7mm}
%         \includegraphics[width=3.5in]{figures/BS}
%         %\vspace{-5mm}
%         \caption{Cell phone activities will be aggregated into the most nearby base station.}
%         \label{fig:grid}
%         \vspace{0mm}
% \end{figure}


%\red{The following can be changed to a parameter table.}
In this section, we formally introduce the problem of recovering a spatial field from coarse aggregations observed at sparse points in the field.
%\label{sec:problem-formulation}
% Our problem can be formulated as a new type of sparse recovery problems. 
% To ease the presentation, we may use cell phone activity recovery as an example. 

Suppose the entire region of interest is modelled by an irregularly bounded domain $\Omega \subset\mathbb R^2$ that excludes the uninhabited areas such as rivers, ocean coasts, hills and so on.
Let $f(\mathbf p)$ be a real-valued function or a spatial field that models certain densities (e.g., cell phone activities) over different geographical positions $\mathbf p = (x,y)\in\Omega$.
Suppose there is a set of $m$ observation points (e.g., base stations) $B = \{B_1,\ldots, B_m\}$ sparsely distributed in $\Omega$,
where each observation point $B_i$ has a position $\mathbf p_{B_i}$ and has observed an \emph{aggregated volume} $z_i$ in the subregion $\Omega_{B_i}$ it is in charge of. 

For example, in the case of cell phone activity recovery, a mobile phone user will always connect to the closest base station (cell tower). Therefore, we have 
$
	z_i = \int_{\Omega_{B_i}}f(\mathbf p)\ud \mathbf p,
$
where the subregion $\Omega_{B_i}$ that $B_i$ represnts is given by 
\[
	\Omega_{B_i} = \{\mathbf p\in \Omega: \|\mathbf p-\mathbf p_{B_i}\|< \|\mathbf p-\mathbf p_{B_{i'}}\|, \forall B_{i'}\in B, \ i'\ne i\}.
\]
Our goal is to recover the entire spatial field $f$ of cell phone activity densities in the domain $\Omega$, based on the aggregated activities $z_1,\ldots, z_m$ observed on $m$ sparsely distributed base stations. In this case, we may call the aggregate observations $z_1,\ldots, z_m$ \emph{base station volumes}. However, such a problem is almost computationally infeasible as the continuous nature of $\Omega_{B_i}$ can hardly be handled by a personal computer.

%In reality, we only need to recover $f$ to a certain granularity required by the operator, e.g., 235m$\times$235m squares in the dataset provided by Telecom Italia Mobile.
%To make the computation feasible, 
To fix notations, suppose $\Omega$ is discretized into $n$ small grid squares $\mathbf p_1,\ldots, \mathbf p_n$, where $\mathbf p_j = (x_j,y_j)\in \Omega$, $j=1,\ldots,n$ also represents the position of square $j$'s center in $\Omega$. Without loss of generality, assume each grid square has an area of $\Delta=1$. And we have $m \ll n$, i.e, the number of aggregate observations (e.g., base station volumes) is way smaller than the number of squares to be recovered.

With the discretization of the domain, the observed volume on base station $B_i$ is given by
\begin{equation}\label{eq:BSvolumes}
	z_i = \sum_{\mathbf p_j \in \Omega_{B_i}}f(\mathbf p_j)\cdot \Delta,\quad i = 1,\ldots,m,
\end{equation}
where the subregion that $B_i$ represents is given by
\begin{equation}\label{eq:omegaBi}
	\Omega_{B_i} = \{\mathbf p_j: 1\le j\le n, \|\mathbf p_j-\mathbf p_{B_i}\|< \|\mathbf p_j-\mathbf p_{B_{i'}}\|, \ \forall i'\ne i\}.
\end{equation}
Therefore, our objective is to recover the unknown spatial field $f$, and especially the activity densities 
\[\mathbf f := (f(\mathbf p_1),\ldots, f(\mathbf p_n))^{\sf T}\] in all $n$ grid squares if the desired granularity is on a per-square level, based on the aggregated observations $z_i$ in \eqref{eq:BSvolumes}.

Note that the above problem description is not only applicable to cell phone activity density recovery, but also applies to a wide range of applications.
% , e.g., inferring a fine-grained geographical user distribution for a certain app or website based on aggregated user counts collected at sparsely distributed Presence of Points (PoPs) or datacenters, and recovering the voter distribution for a certain party based on aggregate statistics at polling stations.
The nonessential difference is that each application has its own way to define each subregion $\Omega_{B_i}$, from which volume $z_i$ is aggregated.


\subsection{Constrained Spatial Smoothing Problem}

Let $\mathbf z = (z_1,\ldots,z_m)^{\sf T}$.
Since all $z_i$ are known and $\Omega_{B_i}$ can be predetermined, e.g., from \eqref{eq:omegaBi} for the cell phone activity density recovery problem, reconstructing $\mathbf f$ from \eqref{eq:BSvolumes} is essentially solving a linear equation
$
	\mathbf z = \mathbf A \mathbf f,
$
where the elements of matrix $\mathbf A\in \mathbb R^{m\times n}$ are given by 
$
A_{ij} = \left\{
	\begin{array}{ll}
		1 & \quad\text{if $\mathbf p_j \in \Omega_{B_i}$,}\\
		0 & \quad\text{otherwise.}\\
	\end{array}
\right.
$
To recover $f(\mathbf p_1),\ldots, f(\mathbf p_n)$ from $z_1,\ldots, z_m$ is apparently a sparse recovery problem, since $m\ll n$. Such a task seems infeasible, since the linear system of equations \eqref{eq:BSvolumes} is an underdetermined system which has an infinite many solutions. 

However, we may further utilize some spatial property of $f$ to make the sparse recovery problem feasible. That is, we can leverage the fact that spatial data often exhibit local correlation or local continuity within $\Omega$. For example, the cell phone activity density at a certain place critically depends on the underlying overall human population and activity at that point, e.g., downtown is more crowded than suburb residential areas, and business areas such as office buildings feature different cell phone activity patterns than leisure areas such as night clubs and restaurants. And the underlying spatial distributions of human activity density and area functionality are often slowly changing over the domain $\Omega$. 

Therefore, taking into account the local spatial continuity and non-negative property of $f$, we formulate our constrained spatial sparse recovery problem as
\begin{equation}
\label{eq:prob0}
\begin{split}
	\underset{f}{\mbox{minimize}}\quad 		& \int_\Omega(\nabla^2 f)^2\ud\mathbf p \\
	\mbox{subject to}\quad & \mathbf z = \mathbf A \mathbf f,\\
						&\mathbf f\ge 0,
\end{split}
\end{equation}
where $\nabla^2 f= \frac{\partial^2 f}{\partial x^2}+\frac{\partial^2 f}{\partial y^2}$ denotes the Laplacian of the functional $f$  to penalize the roughness of the spatial field $f$ and encourage local similarity.
% It is worth noting that once $f$ is reconstructed, we have not only recovered the densities $\mathbf f$ at the square centers $\mathbf p_1,\ldots,\mathbf p_n$, but can also recover the density $f(\mathbf p)$ of any point $\mathbf p\in \Omega$, e.g., between the centers of two neighboring grid squares.
% , although such a fine-grained recovery may not be needed in every application.

Furthermore, we can also use additional external social or demographical features at each place to enhance the recovery capability.
% In the case of cell phone activity density recovery, cell phone activities are often correlated with the underlying population density and social functionalities (e.g., the percentage of green area, the number of schools, the number of businesses/restaurants, the number of sport facilities, and the number of bus stops, etc.) of the considered regions.
Specifically, for each grid square $j$, let $\mathbf w_j = (w_{j1},\ldots,w_{jq})^\mathsf{T}$ be a vector of $q$ external attribute values associated with square $j$. With the additional input of external attributes, we assume that the spatial density data to be recovered in each square $j$ is given by 
\begin{equation}\label{eq:f_lr}
	f(\mathbf p_j) = f'(\mathbf p_j) + \mathbf w_j^{\sf T}\boldsymbol{\beta},
\end{equation}
where $f'(\mathbf p)$ is an underlying spatial field functional that preserves local spatial continuity, while $\mathbf w_j^{\sf T}\boldsymbol{\beta}$ is a linear regression part based on the attributes of square $\mathbf p_j$ that allows position-specific variation or jumps.

In the presence of attributes, the constrained spatial sparse recovery problem can be formulated as
\begin{equation}
\label{eq:prob1}
\begin{split}
	\underset{f', \boldsymbol{\beta}}{\mbox{minimize}}\quad 		& \int_\Omega(\nabla^2 f')^2\ud\mathbf p \\
	\mbox{subject to}
				  \quad & f(\mathbf p_j) = f'(\mathbf p_j) + \mathbf w_j^{\sf T}\boldsymbol{\beta},\quad j = 1,\ldots,n,\\
						& \mathbf z = \mathbf A \mathbf f,\\
						&\mathbf f\ge 0.
\end{split}
\end{equation}
Once the spatial field $f'$ and $\boldsymbol{\beta}$ are found, we can recover $f(\mathbf p_j)$ for all the squares using \eqref{eq:f_lr}. For example, with external attributes, the cell phone activity $f(\mathbf p_j)$ at a certain point $\mathbf p_j$ is modeled as the summation of a spatial field functional $f'(\mathbf p_j)$ and the linear regression from the attributes that permit jumps if neighboring subregions have distinct attributes and functionalities.












%Denote $n$ the number of all squares to be estimated, $n_{BS}$ the number of sampled base stations, $m$ the number of different features available, and $\mathbf{w_i}$ is the $m$-dimensional feature vector of square $i$.

%Suppose we are aiming to estimate the communication activities of $n$ squares $\mathbf{z} := (z_1, z_2, ..., z_n)$ from $n_{BS}$ base stations' aggregated activities $\mathbf{b} := (b_1, b_2, ..., b_{n_{BS}})$. For each base station $i$, the aggregated activity $b_i$ is the sum of its neighbor squares' activities. A square is the neighbor of a base station, if the base station is closest to this square in terms of euclidean distance. Thus, the relation between $\mathbf{b}$ and $\mathbf{z}$ can be denoted by $\mathbf{b} = \mathbf{Az}$, where $\mathbf{A}$ is a matrix of $n_{BS} \times n$. If square $i$ is a neighbor of base station $j$, then $a_{i,j} = 1$, otherwise $a_{i,j} = 0$.

%We assume
%\begin{equation}\label{eq:spatial_model}
%	z_i = {\mathbf w}_i^{\mathsf T}\boldsymbol{\beta} + f(\mathbf p_i) + \epsilon_i,
%\end{equation}


