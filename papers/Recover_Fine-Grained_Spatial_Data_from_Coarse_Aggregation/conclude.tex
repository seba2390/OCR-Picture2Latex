%!TEX root = main.tex
\section{Concluding Remarks}
\label{sec:conclude}

In this paper, we study the problem of inferring the fine-grained spatial distribution of certain density data in a region based on the aggregate observations recorded for each of its subregions.
% , which is extremely challenging and seldom visited before, and analyze the challenges of it.
We propose the Constrained Spatial Smoothing (CSS) approach that exploits both the intrinsic smooth property of underlying factors and the additional features from external social or domestic statics. We further propose a training algorithm which combines the Spatial Spline Regression (SSR) technique and ADMM technique to learn our model parameters efficiently. To evaluate our algorithm and compare it with various other approaches, we run extensive evaluation based on the Milan Call Detail Records dataset provided by Telecom Italia Mobile. The simulation results on the dataset show that our algorithm significantly outperforms other baseline approaches by a great percentage. 

% Although we use the data on cell phone activities to illustrate our methodology, our algorithm is not limited to solving the problem of inferring the distribution of cell phone activities, but also applicable to a variety of problems where estimating an implicit or explicit smooth surface is required, such as %population density estimation, land desirability estimation, human activity pattern modeling and so on.
% inferring the spatial distribution of population densities based on the aggregate population observed at sparsely scattered polling stations, reconstructing a fine-grained geographical distribution of users for an Internet media provider or retailer only from aggregated user counts observed at certain datacenters or points of presence (PoPs), and so on. 

