\section{Related Work}\label{sec:related}

% A lot of work has been done previously in machine reading and reinforcement learning for natural language processing (NLP).

The past few years have seen a large body of work on information extraction (IE), particularly in the biomedical domain. 
This work is too vast to be comprehensively discussed here. We refer the interested reader to the BioNLP community~\cite[inter alia]{nedellec2013overview,kim2012genia,kim2009overview} for a starting point. 
However, most of this work focuses on {\em how} to read, not on {\em what} to read given a goal. To our knowledge, we are the first to focus on the latter task.

% Reinforcement learning (RL) provides a framework for solving the ``credit assignment problem'', where a sequence of decisions needs to be made, but the signal indicating whether the decisions were good may not be received until after a number of decisions have been made.
% Several NLP applications can be described as sequential decision problems and in these cases RL has been used to successfully achieve state of the art performance.
Reinforcement learning has been used to achieve state of the art performance in several natural language processing (NLP) and information retrieval (IR) tasks.
For example, RL has been used to guide IR and filter irrelevant web content~\cite{seo2000reinforcement, zhang2001personalized}.
More recently, RL has been combined with deep learning with great success, e.g., for improving coreference resolution~\cite{clark2016deep}. 
Finally, RL has been used to improve the efficiency of IE by learning how to incrementally reconcile new information and help choose what to look for next \cite{narasimhan2016improving}, a task close to ours.
This serves as an inspiration for the work we present here, but with a critical difference: \citet{narasimhan2016improving} focus on slot filling using a pre-existing template. This makes both the information integration and stopping criteria well-defined. On the other hand, in our focused reading domain, we do not know ahead of time which new pieces of information are necessarily relevant and must be taken in context.