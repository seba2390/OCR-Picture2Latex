%\title{emnlp 2017 instructions}
% File emnlp2017.tex
%

\documentclass[11pt,letterpaper]{article}
\usepackage{emnlp2017}
\usepackage{times}
\usepackage{latexsym}
\usepackage{todonotes}
\usepackage{algorithm}
\usepackage{algpseudocode}

% Uncomment this line for the final submission:
\emnlpfinalcopy

%  Enter the EMNLP Paper ID here:
\def\emnlppaperid{***}

% To expand the titlebox for more authors, uncomment
% below and set accordingly.
% \addtolength\titlebox{.5in}    

\newcommand\BibTeX{B{\sc ib}\TeX}

\usepackage{enumitem}  % control line separation space in lists
\usepackage{url}
\urldef\pubmedquery\url{https://www.ncbi.nlm.nih.gov/pubmed?term=(%222016%22%5BDate%20-%20Publication%5D%20%3A%20%222016%22%5BDate%20-%20Publication%5D)}


\title{Learning what to read: Focused machine reading}  % Focused Reading for Influence Paths

% Author information can be set in various styles:
% For several authors from the same institution:
% \author{Author 1 \and ... \and Author n \\
%         Address line \\ ... \\ Address line}
% if the names do not fit well on one line use
%         Author 1 \\ {\bf Author 2} \\ ... \\ {\bf Author n} \\
% For authors from different institutions:
% \author{Author 1 \\ Address line \\  ... \\ Address line
%         \And  ... \And
%         Author n \\ Address line \\ ... \\ Address line}
% To start a seperate ``row'' of authors use \AND, as in
% \author{Author 1 \\ Address line \\  ... \\ Address line
%         \AND
%         Author 2 \\ Address line \\ ... \\ Address line \And
%         Author 3 \\ Address line \\ ... \\ Address line}
% If the title and author information does not fit in the area allocated,
% place \setlength\titlebox{<new height>} right after
% at the top, where <new height> can be something larger than 2.25in
%\author{Enrique Noriega-Atala \\ The University of Arizona \\ {\tt enoriega@email.arizona.edu} \And Marco A. Valenzuela-Escarcega \\ The University of Arizona \\ {\tt marcov@email.arizona.edu} \AND  Clayton T. Morrison \\ {\tt claytonm@email.arizona.edu} \And  Mihai Surdeanu\\ {\tt surdeanu@email.arizona.edu} }

\author{Enrique Noriega-Atala \\ {\bf Marco A. Valenzuela-Esc\'{a}rcega}  \And Clayton T. Morrison \\  {\bf Mihai Surdeanu} \AND {\normalfont University of Arizona}\\ Tucson, Arizona, USA \\ \\ {\tt \{enoriega,marcov,claytonm,msurdeanu\}@email.arizona.edu} }

   
\date{}

\begin{document}

\maketitle

\begin{abstract}
Recent efforts in bioinformatics have achieved tremendous progress in the machine reading of biomedical literature, and the assembly of the extracted biochemical interactions into large-scale models such as protein signaling pathways. 
However, batch machine reading of literature at today's scale (PubMed alone indexes over 1 million papers per year) is unfeasible due to both cost and processing overhead. In this work,
we introduce a focused reading approach to guide the machine reading of biomedical literature towards {\em what} literature should be read to answer a biomedical query as efficiently as possible. We introduce a family of algorithms for focused reading, including an intuitive, strong baseline, and a second approach which uses a reinforcement learning (RL) framework that learns when to explore (widen the search) or exploit (narrow it). 
We demonstrate that the RL approach is capable of answering more queries than the  baseline, while being more efficient, i.e., reading fewer documents.


%Machine reading of large document corpora is expensive.
%{\em Focused reading} incrementally and efficiently searches for information in a document corpus.
%We present a first family of algorithms for focused reading.  
%We develop a reinforcement learning framework for learning better focused reading policies.
%We demonstrate that the reinforcement learning approach finds more information than a strong baseline, while reading fewer documents.
% \todo{Must be much punchier.}
\end{abstract}

% \leavevmode
% \\
% \\
% \\
% \\
% \\
\section{Introduction}
\label{introduction}

AutoML is the process by which machine learning models are built automatically for a new dataset. Given a dataset, AutoML systems perform a search over valid data transformations and learners, along with hyper-parameter optimization for each learner~\cite{VolcanoML}. Choosing the transformations and learners over which to search is our focus.
A significant number of systems mine from prior runs of pipelines over a set of datasets to choose transformers and learners that are effective with different types of datasets (e.g. \cite{NEURIPS2018_b59a51a3}, \cite{10.14778/3415478.3415542}, \cite{autosklearn}). Thus, they build a database by actually running different pipelines with a diverse set of datasets to estimate the accuracy of potential pipelines. Hence, they can be used to effectively reduce the search space. A new dataset, based on a set of features (meta-features) is then matched to this database to find the most plausible candidates for both learner selection and hyper-parameter tuning. This process of choosing starting points in the search space is called meta-learning for the cold start problem.  

Other meta-learning approaches include mining existing data science code and their associated datasets to learn from human expertise. The AL~\cite{al} system mined existing Kaggle notebooks using dynamic analysis, i.e., actually running the scripts, and showed that such a system has promise.  However, this meta-learning approach does not scale because it is onerous to execute a large number of pipeline scripts on datasets, preprocessing datasets is never trivial, and older scripts cease to run at all as software evolves. It is not surprising that AL therefore performed dynamic analysis on just nine datasets.

Our system, {\sysname}, provides a scalable meta-learning approach to leverage human expertise, using static analysis to mine pipelines from large repositories of scripts. Static analysis has the advantage of scaling to thousands or millions of scripts \cite{graph4code} easily, but lacks the performance data gathered by dynamic analysis. The {\sysname} meta-learning approach guides the learning process by a scalable dataset similarity search, based on dataset embeddings, to find the most similar datasets and the semantics of ML pipelines applied on them.  Many existing systems, such as Auto-Sklearn \cite{autosklearn} and AL \cite{al}, compute a set of meta-features for each dataset. We developed a deep neural network model to generate embeddings at the granularity of a dataset, e.g., a table or CSV file, to capture similarity at the level of an entire dataset rather than relying on a set of meta-features.
 
Because we use static analysis to capture the semantics of the meta-learning process, we have no mechanism to choose the \textbf{best} pipeline from many seen pipelines, unlike the dynamic execution case where one can rely on runtime to choose the best performing pipeline.  Observing that pipelines are basically workflow graphs, we use graph generator neural models to succinctly capture the statically-observed pipelines for a single dataset. In {\sysname}, we formulate learner selection as a graph generation problem to predict optimized pipelines based on pipelines seen in actual notebooks.

%. This formulation enables {\sysname} for effective pruning of the AutoML search space to predict optimized pipelines based on pipelines seen in actual notebooks.}
%We note that increasingly, state-of-the-art performance in AutoML systems is being generated by more complex pipelines such as Directed Acyclic Graphs (DAGs) \cite{piper} rather than the linear pipelines used in earlier systems.  
 
{\sysname} does learner and transformation selection, and hence is a component of an AutoML systems. To evaluate this component, we integrated it into two existing AutoML systems, FLAML \cite{flaml} and Auto-Sklearn \cite{autosklearn}.  
% We evaluate each system with and without {\sysname}.  
We chose FLAML because it does not yet have any meta-learning component for the cold start problem and instead allows user selection of learners and transformers. The authors of FLAML explicitly pointed to the fact that FLAML might benefit from a meta-learning component and pointed to it as a possibility for future work. For FLAML, if mining historical pipelines provides an advantage, we should improve its performance. We also picked Auto-Sklearn as it does have a learner selection component based on meta-features, as described earlier~\cite{autosklearn2}. For Auto-Sklearn, we should at least match performance if our static mining of pipelines can match their extensive database. For context, we also compared {\sysname} with the recent VolcanoML~\cite{VolcanoML}, which provides an efficient decomposition and execution strategy for the AutoML search space. In contrast, {\sysname} prunes the search space using our meta-learning model to perform hyperparameter optimization only for the most promising candidates. 

The contributions of this paper are the following:
\begin{itemize}
    \item Section ~\ref{sec:mining} defines a scalable meta-learning approach based on representation learning of mined ML pipeline semantics and datasets for over 100 datasets and ~11K Python scripts.  
    \newline
    \item Sections~\ref{sec:kgpipGen} formulates AutoML pipeline generation as a graph generation problem. {\sysname} predicts efficiently an optimized ML pipeline for an unseen dataset based on our meta-learning model.  To the best of our knowledge, {\sysname} is the first approach to formulate  AutoML pipeline generation in such a way.
    \newline
    \item Section~\ref{sec:eval} presents a comprehensive evaluation using a large collection of 121 datasets from major AutoML benchmarks and Kaggle. Our experimental results show that {\sysname} outperforms all existing AutoML systems and achieves state-of-the-art results on the majority of these datasets. {\sysname} significantly improves the performance of both FLAML and Auto-Sklearn in classification and regression tasks. We also outperformed AL in 75 out of 77 datasets and VolcanoML in 75  out of 121 datasets, including 44 datasets used only by VolcanoML~\cite{VolcanoML}.  On average, {\sysname} achieves scores that are statistically better than the means of all other systems. 
\end{itemize}


%This approach does not need to apply cleaning or transformation methods to handle different variances among datasets. Moreover, we do not need to deal with complex analysis, such as dynamic code analysis. Thus, our approach proved to be scalable, as discussed in Sections~\ref{sec:mining}.
\section{Related Work}\label{sec:related}
 
The authors in \cite{humphreys2007noncontact} showed that it is possible to extract the PPG signal from the video using a complementary metal-oxide semiconductor camera by illuminating a region of tissue using through external light-emitting diodes at dual-wavelength (760nm and 880nm).  Further, the authors of  \cite{verkruysse2008remote} demonstrated that the PPG signal can be estimated by just using ambient light as a source of illumination along with a simple digital camera.  Further in \cite{poh2011advancements}, the PPG waveform was estimated from the videos recorded using a low-cost webcam. The red, green, and blue channels of the images were decomposed into independent sources using independent component analysis. One of the independent sources was selected to estimate PPG and further calculate HR, and HRV. All these works showed the possibility of extracting PPG signals from the videos and proved the similarity of this signal with the one obtained using a contact device. Further, the authors in \cite{10.1109/CVPR.2013.440} showed that heart rate can be extracted from features from the head as well by capturing the subtle head movements that happen due to blood flow.

%
The authors of \cite{kumar2015distanceppg} proposed a methodology that overcomes a challenge in extracting PPG for people with darker skin tones. The challenge due to slight movement and low lighting conditions during recording a video was also addressed. They implemented the method where PPG signal is extracted from different regions of the face and signal from each region is combined using their weighted average making weights different for different people depending on their skin color. 
%

There are other attempts where authors of \cite{6523142,6909939, 7410772, 7412627} have introduced different methodologies to make algorithms for estimating pulse rate robust to illumination variation and motion of the subjects. The paper \cite{6523142} introduces a chrominance-based method to reduce the effect of motion in estimating pulse rate. The authors of \cite{6909939} used a technique in which face tracking and normalized least square adaptive filtering is used to counter the effects of variations due to illumination and subject movement. 
The paper \cite{7410772} resolves the issue of subject movement by choosing the rectangular ROI's on the face relative to the facial landmarks and facial landmarks are tracked in the video using pose-free facial landmark fitting tracker discussed in \cite{yu2016face} followed by the removal of noise due to illumination to extract noise-free PPG signal for estimating pulse rate. 

Recently, the use of machine learning in the prediction of health parameters have gained attention. The paper \cite{osman2015supervised} used a supervised learning methodology to predict the pulse rate from the videos taken from any off-the-shelf camera. Their model showed the possibility of using machine learning methods to estimate the pulse rate. However, our method outperforms their results when the root mean squared error of the predicted pulse rate is compared. The authors in \cite{hsu2017deep} proposed a deep learning methodology to predict the pulse rate from the facial videos. The researchers trained a convolutional neural network (CNN) on the images generated using Short-Time Fourier Transform (STFT) applied on the R, G, \& B channels from the facial region of interests.
The authors of \cite{osman2015supervised, hsu2017deep} only predicted pulse rate, and we extended our work in predicting variance in the pulse rate measurements as well.

All the related work discussed above utilizes filtering and digital signal processing to extract PPG signals from the video which is further used to estimate the PR and PRV.  %
The method proposed in \cite{kumar2015distanceppg} is person dependent since the weights will be different for people with different skin tone. In contrast, we propose a deep learning model to predict the PR which is independent of the person who is being trained. Thus, the model would work even if there is no prior training model built for that individual and hence, making our model robust. 

%
\section{Focused Reading}\label{sec:focusedreading}

Here we consider focused reading for the biomedical domain, and we focus on binary promotion/inhibition interactions between biochemical entities. 
%in the domain of biological interactions among entities and processes within a cell.  
%We assume a simplified model of interactions that can be represented as a directed graph.  
In this setting, the machine reading (or IE) component constructs a directed graph, where vertices represent {\em participants} in an interaction (e.g., protein, gene, or a biological process), and edges represent directed activation interactions. 
% Interactions relate a set of one or more participants ({\em controllers}) to a single outcome ({\em controlled}) participant in the interaction.  
Edge labels indicate whether the controller entity has a {\em positive} (promoting) or {\em negative} (inhibitory) influence on the controlled participant. 
Figure~\ref{fig:fragment} shows an example edge in this graph.

% This representation allows us to encode statements such as {\em mTOR triggers cellular apoptosis} as a graph, as shown in Figure~\ref{fig:fragment}.

\begin{figure}[t]
\begin{center}
  \includegraphics[width=.4\textwidth]{figs/fragment.pdf}
  \caption{Example of a graph edge encoding the relation extracted from the text: \emph{mTOR triggers cellular apoptosis}.}\label{fig:fragment}
  \vspace{-3mm}
  \end{center}
\end{figure}

We use REACH\footnote{\url{https://github.com/clulab/reach}}, an open source IE system~\cite{Valenzuela:15}, to extract interactions from unstructured biomedical text and construct the graph above. %  REACH output maps directly to our simplified graph model representation.
We couple this IE system with a Lucene\footnote{\url{https://lucene.apache.org}} index of biomedical publications to retrieve papers based on queries about participant mentions in the text (as discussed below).

Importantly, we essentially use IE as a black box (thus, our method could potentially work with any IE system), and focus on strategies that guide {\em what} the IE system reads for a complex information need.
In particular, we consider the common scenario where a biologist (or other model-building process) queries the literature on:
\begin{quote}
 How does one participant (source) affect another (destination), where the connection is typically indirect?
 \end{quote}
 This type of queries is common in biology, where such direct/indirect interactions are observed in experiments, but the explanation of why these dependencies exist is unclear.
 
Algorithm~\ref{alg:focusedreading} outlines the general focused reading algorithm for this task.  
% \todo{describe the use case we address here: Focused reading searches for a directed path of interactions from a source to a dest.}.
In the algorithm, $S, D, A$, and $B$ represent individual participants, where $S$ and $D$ are the {\bf s}ource and {\bf d}estination entities in the initial user query. $G$ is the interaction graph that is iteratively constructed during the focused reading procedure, with $V$ being the set of vertices (biochemical entities), and $E$ the set of edges (promotion/inhibition interactions). 
$\mathit{\Sigma}$ is the strategy that chooses which two entities/vertices to be used in the next information retrieval iteration.
$Q$ is a Lucene query automatically constructed in each iteration to retrieve new papers to read.

%\begin{algorithm}
%\caption{Focused Reading}
%\begin{algorithmic}[1]
%{\small
%\Procedure{FocusedReading}{$S,D$}%\Comment{$S$ and $D$ are the src and dst}
%   \State {\scriptsize $G \gets \{\{S, D\}, \emptyset\}$} \label{alg:fr:graph}%\Comment{$G$ holds the search graph}
%   \Repeat \label{alg:fr:start}
%		\State {\scriptsize $(A,B) \gets $\Call{ChooseEndPoints}{$G$}} \label{alg:chendpoints}
%   		\State {\scriptsize $Q \gets$ \Call{ChooseQuery}{$A,B,G$}} \label{alg:chquery}
%   		\State {\scriptsize $(V, E) \gets$\Call{Lucene+Reach}{$Q$}} \label{alg:ie}%\Comment{Query and IE}
%   		\State {\scriptsize \Call{Expand}{$V, E, G$}} \label{alg:reconcile}%\Comment{Add the new elements to the model}
%   \Until{{\scriptsize \Call{IsConnected}{$S,D$}} OR {\scriptsize \Call{StopConditionMet}{$G$}}}\label{alg:fr:stop}
%\EndProcedure
%}
%\end{algorithmic}
%\label{alg:focusedreading}
%\end{algorithm}

\begin{algorithm}
\caption{Focused reading framework}
\begin{algorithmic}[1]
{\small
\Procedure{FocusedReading}{$S,D$}%\Comment{$S$ and $D$ are the src and dst}
   \State {\scriptsize $G \gets \{\{S, D\}, \emptyset\}$} \label{alg:fr:graph}%\Comment{$G$ holds the search graph}
   \Repeat \label{alg:fr:start}
   		\State {\scriptsize $\mathit{\Sigma} \gets $ \Call{EndpointStrategy}{$G$}} \label{alg:endpointsstrategy}
		\State {\scriptsize $(A,B) \gets $ \Call{ChooseEndPoints}{$\mathit{\Sigma},G$}} \label{alg:chendpoints}
   		\State {\scriptsize $Q \gets $ \Call{ChooseQuery}{$A,B,G$}} \label{alg:chquery}
   		\State {\scriptsize $(V, E) \gets$ \Call{Lucene+Reach}{$Q$}} \label{alg:ie}%\Comment{Query and IE}
   		\State {\scriptsize \Call{Expand}{$V, E, G$}} \label{alg:reconcile}%\Comment{Add the new elements to the model}
   \Until{{\scriptsize \Call{IsConnected}{$S,D$}} OR {\scriptsize \Call{StopConditionMet}{$G$}}}\label{alg:fr:stop}
\EndProcedure
}
\end{algorithmic}
\label{alg:focusedreading}
\end{algorithm}


The algorithm initializes the search graph as containing the two unconnected participants as vertices: $\{S, D\}$ (line~\ref{alg:fr:graph}).
%
The algorithm then enters into its central loop (lines \ref{alg:fr:start} through \ref{alg:fr:stop}). The loop terminates when one or more directed paths connecting $S$ to $D$ are found, or when a stopping condition is met: either $G$ has not changed since the previous run through the loop, or after exceeding some number of iterations through the loop (in this work, ten).

At each pass through the loop the algorithm grows the search graph as follows:
\setlist{nolistsep}
%\begin{enumerate}[noitemsep]
%  \item Given the current graph, choose the next participants to attempt to link: $(A,B)$ (line~\ref{alg:chendpoints}).
%  \item Choose a query strategy: {\em exploration} or {\em exploitation} (line~\ref{alg:chquery}).  
%  In general, exploration widens the search space by adding many more nodes to the graph, whereas exploitation narrows the search by focusing on a specific region of the graph.
%  % ms: I did not understand the sentence below, so I rewrote it above.
%  %Exploration makes queries likely to find interactions directly linking $A$ to $B$, while exploitation makes queries more likely to directly link $A$ to $B$.
%%\todo{Explain what explore and exploit mean!}
%  \item Run the Lucene query to retrieve papers and process the papers using the IE system.  The result of this call is a set of interactions, similar to that in Figure~\ref{fig:fragment} (line~\ref{alg:ie}).
%  \item Add the new interaction participants (vertices $V$) and directed influences (edges $E$) to the search graph (line~\ref{alg:reconcile}).
%\end{enumerate}

\begin{enumerate}%[noitemsep]
\item The graph $G$ is initialized with two nodes, the source and destination in the user's information need, and no edges (because we have not read any papers yet).
  \item Given the current graph, choose a strategy, $\mathit{\Sigma}$, for selecting which entities to query next: {\em exploration} or {\em exploitation} (line~\ref{alg:endpointsstrategy}). In general, exploration aims to widen the search space by adding many more nodes to the graph, whereas exploitation aims to narrow the search by focusing on entities in a specific region of the graph.
  \item Using strategy $\mathit{\Sigma}$, choose the next entities to attempt to link: $(A,B)$ (line~\ref{alg:chendpoints}). 
  \item Choose a query, $Q$: again, using {\em exploration} or {\em exploitation}, following the same intuition as with the entity choice strategy (line~\ref{alg:chquery}).
Here exploration queries retrieve a wider range of documents, while exploitation queries are more restrictive.
  
  % ms: I did not understand the sentence below, so I rewrote it above.
  %Exploration makes queries likely to find interactions directly linking $A$ to $B$, while exploitation makes queries more likely to directly link $A$ to $B$.
%\todo{Explain what explore and exploit mean!}
  \item Run the Lucene query to retrieve papers and process the papers using the IE system.  The result of this call is a set of interactions, similar to that in Figure~\ref{fig:fragment} (line~\ref{alg:ie}).
  \item Add the new interaction participant entities (vertices $V$) and directed influences (edges $E$) to the search graph (line~\ref{alg:reconcile}).
    \item If the source and destination entities are connected in $G$, stop: the user's information need has been addressed. Otherwise, continue from step 2.
\end{enumerate}

The central loop performs a bidirectional search in which each iteration expands the search horizon outward from $S$ and $D$.  
% \todo{Add that intuitive figure here to illustrate how the search horizon is grown?}
Algorithm~\ref{alg:focusedreading} represents a family of possible focused reading algorithms, differentiated by how each of the functions in the main loop are implemented.  In this work, {\sc IsConnected} stops after a single path is found, but a variant could consider finding multiple paths, paths of some length, or incorporate other criteria about the properties of the path.  We next consider particular choices for the inner loop functions.
% describe and evaluate
% In the first pass through the loop, $S$ and $D$ are the only vertices in the graph, so the algorithm queries and retrieves papers that mention $S$ and $D$, the IE system reads the papers, and all new interactions are added to the graph.  In the second pass through the loop, {\sc ChooseEndPoints} can now select participants from vertices around $S$ and $D$, and {\sc ChooseQuery} retrieves additional documents that add more participants, growing the search frontier.  As search continues, directed paths outward from $S$ and paths inward toward $D$ are extended until eventually they meet (thus answering the query on how $S$ and $D$ are connected), or the algorithm reaches another terminating condition.

%Figure~\ref{fig:search-graph} illustrates how the graph $G$ evolves during the search.  In this example, $S$ is the protein Pi3k and $D$ is the protein KTF.  In the first pass through the loop, $S$ and $D$ are the only vertices in the graph, so the algorithm queries for and retrieves papers that mention $S$ and $D$, REACH reads the papers, and all new interactions are added to the graph -- in this case, these include all vertices and edges within the inner dashed circles around Pi3k and KTF.  In the second pass through the loop, {\sc ChooseEndPoints} now selects participants from vertices in the subgraphs around Pi3k and KTF -- in this example, we assume one is chosen from each subgraph.  Again, these are used to seed the retrieval documents and extracted interactions are added to the graph; these new additions are represented in second layer of vertices and edges between the dashed and solid lines.  In this way, the central loop performs a kind of bidirectional search in which each iteration expands the search horizon outward from $S$ and $D$. \todo{This paragraph can be simplified. Not sure the figure is needed?}
%
%\begin{figure}[htb]
%\begin{center}
%  \includegraphics[width=.4\textwidth]{figs/search-graph}
%  \caption{Evolution of graph during search}\label{fig:search-graph}
%  \end{center}
%\end{figure}



% Algorithm~\ref{alg:focusedreading} represents a family of possible focused reading algorithms, differentiated by how each of the functions in the main loop are implemented.  For example, in this work, {\sc IsConnected} stops after a single path is found, but it could implemented to consider finding multiple paths, paths of some length, or incorporate other criteria about the properties of the path.  {\sc ChooseEndPoints} is currently implemented to prefer selecting participants that are more highly connected in $G$, as a proxy for their apparent importance in the graph (similar to the intuition behind PageRank~\cite{page1999pagerank}).  {\sc ChooseQuery} is responsible for constructing the Lucene query for retrieving documents involving the current target participants; we describe this further, below.

% It is worth mentioning that so far, the outline of the algorithm has been left  abstract on purpose because the different implementation of function calls in it can customize its behavior. We can override the \texttt{IsConnected} function to stop as soon as a path is found or until a path of length $n$ is found; override \texttt{ChooseEndPoints} to randomize the chosen endpoints of to chose them according to a random criteria, i.e. PageRank \cite{page1999pagerank}; \texttt{ChooseQuery} let's us define a policy to decide whether to query for exploration, exploitation or a composition of both.

% This flexibility is what allows to search for a data driven policy that aims to minimize the number of costly operations (information retrieval and information extraction) while looking for the explanations. 

% In the rest of the paper we will show that a deterministic implementation of focused reading can find explanations otherwise intractable and that it is possible to use reinforcement learning to learn a policy that improves over the deterministic implementation.
\section{Experimental Evaluation}
\label{sec:experiment}
To demonstrate the viability of our modeling methodology, we show experimentally how through the deliberate combination and configuration of parallel FREEs, full control over 2DOF spacial forces can be achieved by using only the minimum combination of three FREEs.
To this end, we carefully chose the fiber angle $\Gamma$ of each of these actuators to achieve a well-balanced force zonotope (Fig.~\ref{fig:rigDiagram}).
We combined a contracting and counterclockwise twisting FREE with a fiber angle of $\Gamma = 48^\circ$, a contracting and clockwise twisting FREE with $\Gamma = -48^\circ$, and an extending FREE with $\Gamma = -85^\circ$.
All three FREEs were designed with a nominal radius of $R$ = \unit[5]{mm} and a length of $L$ = \unit[100]{mm}.
%
\begin{figure}
    \centering
    \includegraphics[width=0.75\linewidth]{figures/rigDiagram_wlabels10.pdf}
    \caption{In the experimental evaluation, we employed a parallel combination of three FREEs (top) to yield forces along and moments about the $z$-axis of an end effector.
    The FREEs were carefully selected to yield a well-balanced force zonotope (bottom) to gain full control authority over $F^{\hat{z}_e}$ and $M^{\hat{z}_e}$.
    To this end, we used one extending FREE, and two contracting FREEs which generate antagonistic moments about the end effector $z$-axis.}
    \label{fig:rigDiagram}
\end{figure}


\subsection{Experimental Setup}
To measure the forces generated by this actuator combination under a varying state $\vec{x}$ and pressure input $\vec{p}$, we developed a custom built test platform (Fig.~\ref{fig:rig}). 
%
\begin{figure}
    \centering
    \includegraphics[width=0.9\linewidth]{figures/photos/rig_labeled.pdf}
    \caption{\revcomment{1.3}{This experimental platform is used to generate a targeted displacement (extension and twist) of the end effector and to measure the forces and torques created by a parallel combination of three FREEs. A linear actuator and servomotor impose an extension and a twist, respectively, while the net force and moment generated by the FREEs is measured with a force load cell and moment load cell mounted in series.}}
    \label{fig:rig}
\end{figure}
%
In the test platform, a linear actuator (ServoCity HDA 6-50) and a rotational servomotor (Hitec HS-645mg) were used to impose a 2-dimensional displacement on the end effector. 
A force load cell (LoadStar  RAS1-25lb) and a moment load cell (LoadStar RST1-6Nm) measured the end-effector forces $F^{\hat{z_e}}$ and moments $M^{\hat{z_e}}$, respectively.
During the experiments, the pressures inside the FREEs were varied using pneumatic pressure regulators (Enfield TR-010-g10-s). 

The FREE attachment points (measured from the end effector origin) were measured to be:
\begin{align}
    \vec{d}_1 &= \bmx 0.013 & 0 & 0 \emx^T  \text{m}\\
    \vec{d}_2 &= \bmx -0.006 & 0.011 & 0 \emx^T  \text{m}\\
    \vec{d}_3 &= \bmx -0.006 & -0.011 & 0 \emx^T \text{m}
%    \vec{d}_i &= \bmx 0 & 0 & 0 \emx^T , && \text{for } i = 1,2,3
\end{align}
All three FREEs were oriented parallel to the end effector $z$-axis:
\begin{align}
    \hat{a}_i &= \bmx 0 & 0 & 1 \emx^T, \hspace{20pt} \text{for } i = 1,2,3
\end{align}
Based on this geometry, the transformation matrices $\bar{\mathcal{D}}_i$ were given by:
\begin{align}
    \bar{\mathcal{D}}_1 &= \bmx 0 & 0 & 1 & 0 & -0.013 & 0 \\ 0 & 0 & 0 & 0 & 0 & 1 \emx^T  \\
    \bar{\mathcal{D}}_2 &= \bmx 0 & 0 & 1 & 0.011 & 0.006 & 0 \\ 0 & 0 & 0 & 0 & 0 & 1 \emx^T  \\
    \bar{\mathcal{D}}_3 &= \bmx 0 & 0 & 1 & -0.011 & 0.006 & 0 \\ 0 & 0 & 0 & 0 & 0 & 1 \emx^T 
%    \bar{\mathcal{D}}_i &= \bmx 0 & 0 & 1 & 0 & 0 & 0 \\ 0 & 0 & 0 & 0 & 0 & 1 \emx^T , && \text{for } i = 1,2,3
\end{align}
These matrices were used in equation \eqref{eq:zeta} to yield the state-dependent fluid Jacobian $\bar{J}_x$ and to compute the resulting force zontopes.
%while using measured values of $\vec{\zeta}^{\,\text{meas}} (\vec{q}, \vec{P})$ and $\vec{\zeta}^{\,\text{meas}} (\vec{q}, 0)$ in \eqref{eq:fiberIso} yields the empirical measurements of the active force.



\subsection{Isolating the Active Force}
To compare our model force predictions (which focus only on the active forces induced by the fibers)
to those measured empirically on a physical system, we had to remove the elastic force components attributed to the elastomer. 
Under the assumption that the elastomer force is merely a function of the displacement $\vec{x}$ and independent of pressure $\vec{p}$ \cite{bruder2017model}, this force component can be approximated by the measured force at a pressure of $\vec{p}=0$. 
That is: 
\begin{align}
    \vec{f}_{\text{elast}} (\vec{x}) = \vec{f}_{\text{\,meas}} (\vec{x}, 0)
\end{align}
With this, the active generalized forces were measured indirectly by subtracting off the force generated at zero pressure:
\begin{align}
    \vec{f} (\vec{x}, \vec{p})  &= \vec{f}_{\text{meas}} (\vec{x}, \vec{p}) - \vec{f}_{\text{meas}} (\vec{x}, 0)     \label{eq:fiberIso}
\end{align}


%To validate our parallel force model, we compare its force predictions, $\vec{\zeta}_{\text{pred}}$, to those measured empirically on a physical system, $\vec{\zeta}_\text{meas}$. 
%From \eqref{eq:Z} and \eqref{eq:zeta}, the force at the end effector is given by:
%\begin{align}
%    \vec{\zeta}(\vec{q}, \vec{P}) &= \sum_{i=1}^n \bar{\mathcal{D}}_i \left( {\bar{J}_V}_i^T(\vec{q_i}) P_i + \vec{Z}_i^{\text{elast}} (\vec{q_i}) \right) \\
%    &= \underbrace{\sum_{i=1}^n \bar{\mathcal{D}}_i {\bar{J}_V}_i^T(\vec{q_i}) P_i}_{\vec{\zeta}^{\,\text{fiber}} (\vec{q}, \vec{P})} + \underbrace{\sum_{i=1}^n \bar{\mathcal{D}}_i \vec{Z}_i^{\text{elast}} (\vec{q_i})}_{\vec{\zeta}^{\text{elast}} (\vec{q})}   \label{eq:zetaSplit}
%     &= \vec{\zeta}^{\,\text{fiber}} (\vec{q}, \vec{P}) + \vec{\zeta}^{\text{elast}} (\vec{q})
%\end{align}
%\Dan{These will need to reflect changes made to previous section.}
%The model presented in this paper does not specify the elastomer forces, $\vec{\zeta}^{\text{elast}}$, therefore we only validate its predictions %of the fiber forces, $\vec{\zeta}^{\,\text{fiber}}$. 
%We isolate the fiber forces by noting that $\vec{\zeta}^{\text{elast}} (\vec{q}) = \vec{\zeta}(\vec{q}, 0)$ and rearranging \eqref{eq:zetaSplit}
%\begin{align}
%    \vec{\zeta}^{\,\text{fiber}} (\vec{q}, \vec{P})  &= \vec{\zeta}(\vec{q}, \vec{P}) - \vec{\zeta}(\vec{q}, 0)     \label{eq:fiberIso}
%%    \vec{\zeta}^{\,\text{fiber}}_{\text{emp}} (\vec{q}, \vec{P})  &= \vec{\zeta}_{\text{emp}}(\vec{q}, \vec{P}) - %\vec{\zeta}_{\text{emp}}(\vec{q}, 0)
%\end{align}
%Thus we measure the fiber forces indirectly by subtracting off the forces generated at zero pressure.  


\subsection{Experimental Protocol}
The force and moment generated by the parallel combination of FREEs about the end effector $z$-axis  was measured in four different geometric configurations: neutral, extended, twisted, and simultaneously extended and twisted (see Table \ref{table:RMSE} for the exact deformation amounts). 
At each of these configurations, the forces were measured at all pressure combinations in the set
\begin{align}
    \mathcal{P} &= \left\{ \bmx \alpha_1 & \alpha_2 & \alpha_3 \emx^T p^{\text{max}} \, : \, \alpha_i = \left\{ 0, \frac{1}{4}, \frac{1}{2}, \frac{3}{4}, 1 \right\} \right\}
\end{align}
with $p^{\text{max}}$ = \unit[103.4]{kPa}. 
\revcomment{3.2}{The experiment was performed twice using two different sets of FREEs to observe how fabrication variability might affect performance. The results from Trial 1 are displayed in Fig.~\ref{fig:results} and the error for both trials is given in Table \ref{table:RMSE}.}



\subsection{Results}

\begin{figure*}[ht]
\centering

\def\picScale{0.08}    % define variable for scaling all pictures evenly
\def\plotScale{0.2}    % define variable for scaling all plots evenly
\def\colWidth{0.22\linewidth}

\begin{tikzpicture} %[every node/.style={draw=black}]
% \draw[help lines] (0,0) grid (4,2);
\matrix [row sep=0cm, column sep=0cm, style={align=center}] (my matrix) at (0,0) %(2,1)
{
& \node (q1) {(a) $\Delta l = 0, \Delta \phi = 0$}; & \node (q2) {(b) $\Delta l = 5\text{mm}, \Delta \phi = 0$}; & \node (q3) {(c) $\Delta l = 0, \Delta \phi = 20^\circ$}; & \node (q4) {(d) $\Delta l = 5\text{mm}, \Delta \phi = 20^\circ$};

\\

&
\node[style={anchor=center}] {\includegraphics[width=\colWidth]{figures/photos/s0w0pic_colored.pdf}}; %\fill[blue] (0,0) circle (2pt);
&
\node[style={anchor=center}] {\includegraphics[width=\colWidth]{figures/photos/s5w0pic_colored.pdf}}; %\fill[blue] (0,0) circle (2pt);
&
\node[style={anchor=center}] {\includegraphics[width=\colWidth]{figures/photos/s0w20pic_colored.pdf}}; %\fill[blue] (0,0) circle (2pt);
&
\node[style={anchor=center}] {\includegraphics[width=\colWidth]{figures/photos/s5w20pic_colored.pdf}}; %\fill[blue] (0,0) circle (2pt);

\\

\node[rotate=90] (ylabel) {Moment, $M^{\hat{z}_e}$ (N-m)};
&
\node[style={anchor=center}] {\includegraphics[width=\colWidth]{figures/plots3/s0w0.pdf}}; %\fill[blue] (0,0) circle (2pt);
&
\node[style={anchor=center}] {\includegraphics[width=\colWidth]{figures/plots3/s5w0.pdf}}; %\fill[blue] (0,0) circle (2pt);
&
\node[style={anchor=center}] {\includegraphics[width=\colWidth]{figures/plots3/s0w20.pdf}}; %\fill[blue] (0,0) circle (2pt);
&
\node[style={anchor=center}] {\includegraphics[width=\colWidth]{figures/plots3/s5w20.pdf}}; %\fill[blue] (0,0) circle (2pt);

\\

& \node (xlabel1) {Force, $F^{\hat{z}_e}$ (N)}; & \node (xlabel2) {Force, $F^{\hat{z}_e}$ (N)}; & \node (xlabel3) {Force, $F^{\hat{z}_e}$ (N)}; & \node (xlabel4) {Force, $F^{\hat{z}_e}$ (N)};

\\
};
\end{tikzpicture}

\caption{For four different deformed configurations (top row), we compare the predicted and the measured forces for the parallel combination of three FREEs (bottom row). 
\revcomment{2.6}{Data points and predictions corresponding to the same input pressures are connected by a thin line, and the convex hull of the measured data points is outlined in black.}
The Trial 1 data is overlaid on top of the theoretical force zonotopes (grey areas) for each of the four configurations.
Identical colors indicate correspondence between a FREE and its resulting force/torque direction.}
\label{fig:results}
\end{figure*}






% & \node (a) {(a)}; & \node (b) {(b)}; & \node (c) {(c)}; & \node (d) {(d)};


For comparison, the measured forces are superimposed over the force zonotope generated by our model in Fig.~\ref{fig:results}a-~\ref{fig:results}d.
To quantify the accuracy of the model, we defined the error at the $j^{th}$ evaluation point as the difference between the modeled and measured forces
\begin{align}
%    \vec{e}_j &= \left( {\vec{\zeta}_{\,\text{mod}}} - {\vec{\zeta}_{\,\text{emp}}} \right)_j
%    e_j &= \left( F/M_{\,\text{mod}} - F/M_{\,\text{emp}} \right)_j
    e^F_j &= \left( F^{\hat{z}_e}_{\text{pred}, j} - F^{\hat{z}_e}_{\text{meas}, j} \right) \\
    e^M_j &= \left( M^{\hat{z}_e}_{\text{pred}, j} - M^{\hat{z}_e}_{\text{meas}, j} \right)
\end{align}
and evaluated the error across all $N = 125$ trials of a given end effector configuration.
% using the following metrics:
% \begin{align}
%     \text{RMSE} &= \sqrt{ \frac{\sum_{j=1}^{N} e_j^2}{N} } \\
%     \text{Max Error} &= \max \{ \left| e_j \right| : j = 1, ... , N \}
% \end{align}
As shown in Table \ref{table:RMSE}, the root-mean-square error (RMSE) is less than \unit[1.5]{N} (\unit[${8 \times 10^{-3}}$]{Nm}), and the maximum error is less than \unit[3]{N}  (\unit[${19 \times 10^{-3}}$]{Nm}) across all trials and configurations.

\begin{table}[H]
\centering
\caption{Root-mean-square error and maximum error}
\begin{tabular}{| c | c || c | c | c | c|}
    \hline
     & \rule{0pt}{2ex} \textbf{Disp.} & \multicolumn{2}{c |}{\textbf{RMSE}} & \multicolumn{2}{c |}{\textbf{Max Error}} \\ 
     \cline{2-6}
     & \rule{0pt}{2ex} (mm, $^\circ$) & F (N) & M (Nm) & F (N) & M (Nm) \\
     \hline
     \multirow{4}{*}{\rotatebox[origin=c]{90}{\textbf{Trial 1}}}
     & 0, 0 & 1.13 & $3.8 \times 10^{-3}$ & 2.96 & $7.8 \times 10^{-3}$ \\
     & 5, 0 & 0.74 & $3.2 \times 10^{-3}$ & 2.31 & $7.4 \times 10^{-3}$ \\
     & 0, 20 & 1.47 & $6.3 \times 10^{-3}$ & 2.52 & $15.6 \times 10^{-3}$\\
     & 5, 20 & 1.18 & $4.6 \times 10^{-3}$ & 2.85 & $12.4 \times 10^{-3}$ \\  
     \hline
     \multirow{4}{*}{\rotatebox[origin=c]{90}{\textbf{Trial 2}}}
     & 0, 0 & 0.93 & $6.0 \times 10^{-3}$ & 1.90 & $13.3 \times 10^{-3}$ \\
     & 5, 0 & 1.00 & $7.7 \times 10^{-3}$ & 2.97 & $19.0 \times 10^{-3}$ \\
     & 0, 20 & 0.77 & $6.9 \times 10^{-3}$ & 2.89 & $15.7 \times 10^{-3}$\\
     & 5, 20 & 0.95 & $5.3 \times 10^{-3}$ & 2.22 & $13.3 \times 10^{-3}$ \\  
     \hline
\end{tabular}
\label{table:RMSE}
\end{table}

\begin{figure}
    \centering
    \includegraphics[width=\linewidth]{figures/photos/buckling.pdf}
    \caption{At high fluid pressure the FREE with fiber angle of $-85^\circ$ started to buckle.  This effect was less pronounced when the system was extended along the $z$-axis.}
    \label{fig:buckling}
\end{figure}

%Experimental precision was limited by unmodeled material defects in the FREEs, as well as sensor inaccuracy. While the commercial force and moment sensors used have a quoted accuracy of 0.02\% for the force sensor and 0.2\% for the moment sensor (LoadStar Sensors, 2015), a drifting of up to 0.5 N away from zero was noticed on the force sensor during testing.

It should be noted, that throughout the experiments, the FREE with a fiber angle of $-85^\circ$ exhibited noticeable buckling behavior at pressures above $\approx$ \unit[50]{kPa} (Fig.~\ref{fig:buckling}). 
This behavior was more pronounced during testing in the non-extended configurations (Fig.~\ref{fig:results}a~and~\ref{fig:results}c). 
The buckling might explain the noticeable leftward offset of many of the points in Fig.~\ref{fig:results}a and Fig.~\ref{fig:results}c, since it is reasonable to assume that buckling reduces the efficacy of of the FREE to exert force in the direction normal to the force sensor. 

\begin{figure}
    \centering
    \includegraphics[width=\linewidth]{figures/zntp_vs_x4.pdf}
    \caption{A visualization of how the \emph{force zonotope} of the parallel combination of three FREEs (see Fig.~\ref{fig:rig}) changes as a function of the end effector state $x$. One can observe that the change in the zonotope ultimately limits the work-space of such a system.  In particular the zonotope will collapse for compressions of more than \unit[-10]{mm}.  For \revcomment{2.5}{scale and comparison, the convex hulls of the measured points from Fig.~\ref{fig:results}} are superimposed over their corresponding zonotope at the configurations that were evaluated experimentally.}
    % \marginnote{\#2.5}
    \label{fig:zntp_vs_x}
\end{figure}

\section{Case Studies}
\label{sec:case_studies}
In this section, we present a case study of Facebook posts from an Australian public page.
The page shifts between early 2020 (\emph{2019-2020 Australian bushfire season}) and late 2020 (\emph{COVID-19 crises}) from being a moderate-right group for discussion around climate change to a far-right extremist group for conspiracy theories.


\begin{figure*}[!tbp]
	\begin{subfigure}{0.21\textwidth}
		\includegraphics[width=\textwidth]{images/facebook1.png}
		\caption{}
		\label{subfig:first-posting}
		\includegraphics[width=0.9\textwidth]{images/facebook3.jpg}
		\caption{}
		\label{subfig:comment-post-1}
	\end{subfigure}
    \begin{subfigure}{0.28\textwidth}
		\includegraphics[width=\textwidth]{images/facebook2.jpg}
		\caption{}
		\label{subfig:second-posting}
	\end{subfigure}
    \begin{subfigure}{0.23\textwidth}
		\includegraphics[width=\textwidth]{images/facebook4.jpg}
		\caption{}
		\label{subfig:comment-post-2a}
	\end{subfigure}
    \begin{subfigure}{0.23\textwidth}
		\includegraphics[width=\textwidth]{images/facebook5.jpg}
		\caption{}
		\label{subfig:comment-post-2b}
	\end{subfigure}
	\caption{
		Examples of postings and comment threads from a public Facebook page from two periods of time early 2020 (a) and late 2020 (b)-(e), which show a shift from climate change debates to extremist and far-right messaging.
	}
	\label{fig:facebook}
\end{figure*}

We focus on a sample of 2 postings and commenting threads from one Australian Facebook page we classified as ``far-right'' based on the content on the page. 
We have anonymized the users in \Cref{fig:facebook} to avoid re-identification.
The first posting and comment thread (see \Cref{subfig:first-posting}) was collected on Jan 10, 2020, and responded to the Australian bushfire crisis that began in late 2019 and was still ongoing in January 2020. It contains an ambivalent text-based provocation that references disputes in the community regarding the validity of climate change and climate science. 

The second posting and comment thread (see \Cref{subfig:second-posting}) was collected from the same page in September 2020, months after the bushfire crisis had abated.
At that time, a new crisis was energizing and connecting the far-right groups in our dataset --- i.e., the COVID-19 pandemic and the government interventions to curb the spread of the virus. 
The post is different in style compared to the first.
It is image-based instead of text-based and highly emotive, with a photo collage bringing together images of prison inmates with iron masks on their faces (top row) juxtaposed to people wearing face masks during COVID-19 (bottom row). 
The image references the public health orders issued during Melbourne's second lockdown and suggests that being ordered to wear masks is an infringement of citizen rights and freedoms, similar to dehumanizing restraints used on prisoners.

To analyze reactions to the posts, two researchers used a deductive analytical approach to separately code and to analyze the commenting threads --- see \Cref{subfig:comment-post-1} for comments of the first posting, and \Cref{subfig:comment-post-2a,subfig:comment-post-2b} for comments on the second posting. 
Conversations were also inductively coded for emerging themes. 
During the analysis, we observed qualitative differences in the types of content users posted, interactions between commenters, tone and language of debate, linked media shared in the commenting thread, and the opinions expressed.
The rest of this section further details these differences.
To ensure this was not a random occurrence, we tested the exemplar threads against field notes collected on the group during the entire study.
We also used Facebook's search function within pages to find a sample of posts from the same period and which dealt with similar topics. 
After this analysis, we can confidently say that key changes occurred in the group between the bushfire crisis and COVID-19, that we detail next.

\subsubsection*{Exemplar 1 --- climate change skepticism.}
To explore this transformation in more depth, we analyzed comments scraped on the first posting --- \cref{sub@subfig:comment-post-1} shows a small sample of these comments.
The language used was similar to comments that we observed on numerous far-right nationalist pages at the time of the bushfires.
These comments are usually text-based, employing emojis to denote emotions, and sometimes being mocking or provocative in tone. 
Noteworthy for this commenting thread is the 50/50 split in the number of members posting in favor of action on climate change (on one side) and those who posted anti-Greens and anti-climate change science posts and memes (on the other side).
The two sides aligned strongly with political partisanship --- either with Liberal/National coalition (climate change deniers) or Labor/Green (climate change believers) parties. 
This is rather unusual for pages classified as far-right. 

We observed trolling practices between the climate change deniers and believers, which often descend into \emph{flame wars} --- i.e., online ``firefights that take place between disembodied combatants on electronic bulletin boards''~\citep{bukatman1994flame}.
The result is a boosted engagement on the post but also the frustration and confusion of community members and lurkers who came to the discussions to become informed or debate rationally on key differences between the two positions.
They often even become targeted, victimized, and baited by trolls on both sides of the partisan divide. 
The opinions expressed by deniers in commenting sections range from skepticism regarding climate change science to plain denial.
Deniers also regard a range of targets as embroiled in a climate change conspiracy to deceive the public, such as The Greens and their environmental policy, in some cases the government, the United Nations, and climate change celebrities like David Attenborough and Greta Thunberg. 
These figures are blamed for either exaggerating risks of climate change or creating a climate change hoax to increase the influence of the UN on domestic governments or to increase domestic governments' social control over citizens. 

Both coders noted that flame wars between these opposing personas contained very few links to external media. 
Where links were added, they often seemed disconnected from the rest of the conversation and were from users whose profiles suggested they believed in more radical conspiracy theories.
One such example is ``geo-engineering'' (see \cref{sub@subfig:comment-post-1}).
Its adherents believe that solar geo-engineering programs designed to combat climate change are secretly used by a global elite to depopulate the world through sterilization or to control and weaponize the weather.

Nonetheless, apart from the random comments that hijack the thread, redirecting users to external ``alternative'' news sites and Twitter, and the trolls who seem to delight in victimizing unsuspecting victims, the discussion was pretty healthy.
There are many questions, rational inquiries, and debates between users of different political persuasion and views on climate change.
This, however, changes in the span of only a couple of months.

\subsubsection*{Exemplar 2 --- posting and commenting thread.}
We observe a shift in the comment section of the post collected during the second wave of the COVID pandemic (\Cref{sub@subfig:second-posting}) --- which coincided with government laws mandating the public to wear masks and stay at home in Victoria, Australia.
There emerges much more extreme far-right content that converges with anti-vaccination opinions and content.
We also note a much higher prevalence of conspiracy theories often implicating racialized targets.
This is exemplified in the comments on the second post (\Cref{sub@subfig:comment-post-2a,sub@subfig:comment-post-2b}) where Islamophobia and antisemitism are confidently asserted alongside anti-mask rhetoric.
These comments consider face masks similar to the religious head coverings worn by some Muslim women, which users describe as ``oppressive'' and ``silencing''. 
In this way, anti-maskers cast women as a distinct, sympathetic marginalized demographic.
However, this is enacted alongside the racialization and demonization of Islam as an oppressive religion. 

Given the extreme racialization of anti-mask rhetoric, some commenters contest these positions, arguing that the page is becoming less an anti-Scott Morrison page (Australia's Prime Minister at the time) and changing into a page that harbors ``far-right dickheads''.
This questioning is actively challenged by far-right commenters and conspiracy theorists on the page, who regarded pro-mask users and the Scott Morrison government as ``puppets'' being manipulated by higher forces (see \Cref{sub@subfig:comment-post-2b}). 

This indicates a significant change on the page's membership towards the extreme-right, who employs more extreme forms of racialized imagery, with more extreme opinion being shared.
Conspiracy theorists become more active and vocal, and they consistently challenge the opinions of both center conservative and left-leaning users. 
This is evident in the final two comments in \Cref{subfig:comment-post-2b}, which reflect QAnon style conspiracy theories and language.
Public health orders to wear masks are being connected to a conspiracy that all of these decisions are directed by a secret network of global Jewish elites, who manipulate the pandemic to increase their power and control. 
This rhetoric intersects with the contemporary ``QAnon'' conspiracy theory, which evolved from the ``Pizzagate'' conspiracy theory.
They also heavily draw on well-established antisemitic blood libel conspiracy theories, which foster beliefs that a powerful global elite is controlling the decisions of organizations such as WHO and are responsible for the vaccine rollout and public health orders related to the pandemic.
The QAnon conspiracy is also influenced by Bill Gates' Microchips conspiracy theory, i.e., the theory that the WHO and the Bill Gates Foundation global vaccine programs are used to inject tracking microchips into people.

These conspiracy theories have, since COVID-19, connected formerly separate communities and discourses, uniting existing anti-vaxxer communities, older demographics who are mistrustful of technology, far-right communities suspicious of global and national left-wing agendas, communities protesting against 5G mobile networks (for fear that they will brainwash, control, or harm people), as well as generating its own followers out of those anxious during the 2020 onset of the COVID-19 pandemic.
We detect and describe some of these opinion dynamics in the next section.

\mySection{Related Works and Discussion}{}
\label{chap3:sec:discussion}

In this section we briefly discuss the similarities and differences of the model presented in this chapter, comparing it with some related work presented earlier (Chapter \ref{chap1:artifact-centric-bpm}). We will mention a few related studies and discuss directly; a more formal comparative study using qualitative and quantitative metrics should be the subject of future work.

Hull et al. \citeyearpar{hull2009facilitating} provide an interoperation framework in which, data are hosted on central infrastructures named \textit{artifact-centric hubs}. As in the work presented in this chapter, they propose mechanisms (including user views) for controlling access to these data. Compared to choreography-like approach as the one presented in this chapter, their settings has the advantage of providing a conceptual rendezvous point to exchange status information. The same purpose can be replicated in this chapter's approach by introducing a new type of agent called "\textit{monitor}", which will serve as a rendezvous point; the behaviour of the agents will therefore have to be slightly adapted to take into account the monitor and to preserve as much as possible the autonomy of agents.

Lohmann and Wolf \citeyearpar{lohmann2010artifact} abandon the concept of having a single artifact hub \cite{hull2009facilitating} and they introduce the idea of having several agents which operate on artifacts. Some of those artifacts are mobile; thus, the authors provide a systematic approach for modelling artifact location and its impact on the accessibility of actions using a Petri net. Even though we also manipulate mobile artifacts, we do not model artifact location; rather, our agents are equipped with capabilities that allow them to manipulate the artifacts appropriately (taking into account their location). Moreover, our approach considers that artifacts can not be remotely accessed, this increases the autonomy of agents.

The process design approach presented in this chapter, has some conceptual similarities with the concept of \textit{proclets} proposed by Wil M. P. van der Aalst et al. \citeyearpar{van2001proclets, van2009workflow}: they both split the process when designing it. In the model presented in this chapter, the process is split into execution scenarios and its specification consists in the diagramming of each of them. Proclets \cite{van2001proclets, van2009workflow} uses the concept of \textit{proclet-class} to model different levels of granularity and cardinality of processes. Additionally, proclets act like agents and are autonomous enough to decide how to interact with each other.

The model presented in this chapter uses an attributed grammar as its mathematical foundation. This is also the case of the AWGAG model by Badouel et al. \citeyearpar{badouel14, badouel2015active}. However, their model puts stress on modelling process data and users as first class citizens and it is designed for Adaptive Case Management.

To summarise, the proposed approach in this chapter allows the modelling and decentralized execution of administrative processes using autonomous agents. In it, process management is very simply done in two steps. The designer only needs to focus on modelling the artifacts in the form of task trees and the rest is easily deduced. Moreover, we propose a simple but powerful mechanism for securing data based on the notion of accreditation; this mechanism is perfectly composed with that of artifacts. The main strengths of our model are therefore : 
\begin{itemize}
	\item The simplicity of its syntax (process specification language), which moreover (well helped by the accreditation model), is suitable for administrative processes;
	\item The simplicity of its execution model; the latter is very close to the blockchain's execution model \cite{hull2017blockchain, mendling2018blockchains}. On condition of a formal study, the latter could possess the same qualities (fault tolerance, distributivity, security, peer autonomy, etc.) that emanate from the blockchain;
	\item Its formal character, which makes it verifiable using appropriate mathematical tools;
	\item The conformity of its execution model with the agent paradigm and service technology.
\end{itemize}
In view of all these benefits, we can say that the objectives set for this thesis have indeed been achieved. However, the proposed model is perfectible. For example, it can be modified to permit agents to respond incrementally to incoming requests as soon as any prefix of the extension of a bud is produced. This makes it possible to avoid the situation observed on figure \ref{chap3:fig:execution-figure-4} where the associated editor is informed of the evolution of the subtree resulting from $C$ only when this one is closed. All the criticisms we can make of the proposed model in particular, and of this thesis in general, have been introduced in the general conclusion (page \pageref{chap5:general-conclusion}) of this manuscript.





\section*{Acknowledgments}

This work was partially funded by the DARPA Big Mechanism
program under ARO contract W911NF-14-1-0395.

Dr. Mihai Surdeanu discloses a financial interest in Lum.ai. This interest has been disclosed to the University of Arizona Institutional Review Committee and is being managed in accordance with its conflict of interest policies.


\bibliography{emnlp2017}
\bibliographystyle{emnlp_natbib}

\end{document}
