%\title{emnlp 2017 instructions}
% File emnlp2017.tex
%

\documentclass[11pt,letterpaper]{article}
\usepackage{emnlp2017}
\usepackage{times}
\usepackage{latexsym}
\usepackage{todonotes}
\usepackage{algorithm}
\usepackage{algpseudocode}

% Uncomment this line for the final submission:
\emnlpfinalcopy

%  Enter the EMNLP Paper ID here:
\def\emnlppaperid{***}

% To expand the titlebox for more authors, uncomment
% below and set accordingly.
% \addtolength\titlebox{.5in}    

\newcommand\BibTeX{B{\sc ib}\TeX}

\usepackage{enumitem}  % control line separation space in lists
\usepackage{url}
\urldef\pubmedquery\url{https://www.ncbi.nlm.nih.gov/pubmed?term=(%222016%22%5BDate%20-%20Publication%5D%20%3A%20%222016%22%5BDate%20-%20Publication%5D)}


\title{Learning what to read: Focused machine reading}  % Focused Reading for Influence Paths

% Author information can be set in various styles:
% For several authors from the same institution:
% \author{Author 1 \and ... \and Author n \\
%         Address line \\ ... \\ Address line}
% if the names do not fit well on one line use
%         Author 1 \\ {\bf Author 2} \\ ... \\ {\bf Author n} \\
% For authors from different institutions:
% \author{Author 1 \\ Address line \\  ... \\ Address line
%         \And  ... \And
%         Author n \\ Address line \\ ... \\ Address line}
% To start a seperate ``row'' of authors use \AND, as in
% \author{Author 1 \\ Address line \\  ... \\ Address line
%         \AND
%         Author 2 \\ Address line \\ ... \\ Address line \And
%         Author 3 \\ Address line \\ ... \\ Address line}
% If the title and author information does not fit in the area allocated,
% place \setlength\titlebox{<new height>} right after
% at the top, where <new height> can be something larger than 2.25in
%\author{Enrique Noriega-Atala \\ The University of Arizona \\ {\tt enoriega@email.arizona.edu} \And Marco A. Valenzuela-Escarcega \\ The University of Arizona \\ {\tt marcov@email.arizona.edu} \AND  Clayton T. Morrison \\ {\tt claytonm@email.arizona.edu} \And  Mihai Surdeanu\\ {\tt surdeanu@email.arizona.edu} }

\author{Enrique Noriega-Atala \\ {\bf Marco A. Valenzuela-Esc\'{a}rcega}  \And Clayton T. Morrison \\  {\bf Mihai Surdeanu} \AND {\normalfont University of Arizona}\\ Tucson, Arizona, USA \\ \\ {\tt \{enoriega,marcov,claytonm,msurdeanu\}@email.arizona.edu} }

   
\date{}

\begin{document}

\maketitle

\begin{abstract}
Recent efforts in bioinformatics have achieved tremendous progress in the machine reading of biomedical literature, and the assembly of the extracted biochemical interactions into large-scale models such as protein signaling pathways. 
However, batch machine reading of literature at today's scale (PubMed alone indexes over 1 million papers per year) is unfeasible due to both cost and processing overhead. In this work,
we introduce a focused reading approach to guide the machine reading of biomedical literature towards {\em what} literature should be read to answer a biomedical query as efficiently as possible. We introduce a family of algorithms for focused reading, including an intuitive, strong baseline, and a second approach which uses a reinforcement learning (RL) framework that learns when to explore (widen the search) or exploit (narrow it). 
We demonstrate that the RL approach is capable of answering more queries than the  baseline, while being more efficient, i.e., reading fewer documents.


%Machine reading of large document corpora is expensive.
%{\em Focused reading} incrementally and efficiently searches for information in a document corpus.
%We present a first family of algorithms for focused reading.  
%We develop a reinforcement learning framework for learning better focused reading policies.
%We demonstrate that the reinforcement learning approach finds more information than a strong baseline, while reading fewer documents.
% \todo{Must be much punchier.}
\end{abstract}

\section{Introduction}  \label{sec:introduction}

\newcommand\inexpIntro[3]{#1?(#2,#3).}
\newcommand\rinexpIntro[3]{*#1?(#2,#3).}
\newcommand\outexpIntro[3]{#1!(#2,#3).}
\newcommand\outatomIntro[3]{#1!(#2,#3)}

We propose a fully automated method for proving termination of \(\pi\)-calculus processes.
Although there have been a lot of studies on termination analysis for the \(\pi\)-calculus
and related calculi~\cite{Deng06IC,Demangeon07,SangiorgiTermination,KobayashiHybrid,Yoshida04IC,DBLP:journals/jlp/DemangeonHS10,Venet98SAS}, most of them have been rather theoretical,
and there have been surprisingly little efforts in developing  fully automated termination
verification methods and tools based on them. To our knowledge,
Kobayashi's \typical{}~\cite{TyPiCal,KobayashiHybrid} is the only exception that
can prove termination of \(\pi\)-calculus processes (extended with natural numbers)
fully automatically, but its termination analysis is quite limited (see Section~\ref{sec:relatedwork}).

Our method is based on a reduction to termination analysis for sequential programs:
we translate a \(\pi\)-calculus process \(P\) to a sequential program \(S_P\), so that
if \(S_P\) is terminating, so is \(P\). The reduction allows us to use
powerful, mature methods and tools
for termination analysis of sequential programs~\cite{heizmann2016ultimate,freqterm,DBLP:conf/lics/PodelskiR04,Kuwahara2014Termination,DBLP:journals/cacm/CookPR11}.

The idea of the translation is to convert a chain of communications on replicated input
channels to a chain of recursive function calls of the target sequential program.
Let us consider the following Fibonacci process:
\begin{align*}
    & \rinexpIntro{\fib}{n}{r}
        \ifexp{n<2}{ \soutatom{r}{1} \\ &\quad}
                   { \nuexp{s_1} \nuexp{s_2} (\outatomIntro{\fib}{n-1}{s_1} \PAR \outatomIntro{\fib}{n-2}{s_2} \PAR \sinexp{s_1}{x}\sinexp{s_2}{y}\soutatom{r}{x+y}) \\}
    & \PAR \outatomIntro{\fib}{m}{r}
\end{align*}
Here, the process
$\rinexpIntro{\fib}{n}{r} \ldots$ is a function server that computes the \(n\)-th Fibonacci number
in parallel and returns the result to \(r\),
and $\outatom{\fib}{m}{r}$ sends a request for computing the \(m\)-th Fibonacci number;
those who are not familiar with the syntax of the \(\pi\)-calculus may wish to consult
Section~\ref{sec:targetlanguage} first.
To prove that the process above is terminating for any integer \(m\),
it suffices to show that there is no infinite chain of communications on $\fib$:
\[
    \fib(m,r) \to \fib(m_1,r_1) \to \fib(m_2,r_2) \to \cdots.
\]
We convert the process above to the following program:\footnote{The actual translation
  given later is a little more complex.}
\begin{verbatim}
 let rec fib(n) = if n<2 then () else (fib(n-1) [] fib(n-2)) in
 fib(m)
\end{verbatim}
Here, \texttt{[]} represents the non-deterministic choice.
Note that, although the calculation of Fibonacci numbers is not preserved,
for each chain of communications on \texttt{fib}, there is a corresponding
sequence of recursive calls:
\[
\mathtt{fib}(m) \to \mathtt{fib}(m_1) \to \mathtt{fib}(m_2) \to \cdots.
\]
Thus, the termination of the sequential program above implies the termination of
the original process.
As shown in the example above, (i) each communication on a replicated input channel
is converted to a function call, (ii) each communication on a non-replicated input
channel is just removed (or, in the actual translation, replaced by a call of
a trivial function defined by \(f(\seq{x})=(\,)\)), and (iii) parallel composition
is replaced by a non-deterministic choice.
We formalize the translation outlined above and prove its correctness.

The basic translation sketched above sometimes loses too much information.
For example, consider the following process:
\begin{align*}
    & \rinexpIntro{\pre}{n}{r} \soutatom{r}{n-1} \\
    & \PAR \rinexpIntro{f}{n}{r} \ifexp{n<0}{ \soutatom{r}{1} }
                                       { \nuexp{s} (\outatomIntro{\pre}{n}{s} \PAR \sinexp{s}{x}\outatomIntro{f}{x}{r}) } \\
    & \PAR \outatomIntro{f}{m}{r}
\end{align*}
The translation sketched above would yield:
\begin{verbatim}
  let pred(n) = n-1 in
  let rec f(n) = if n<0 then () else (pred(n) [] f(*)) in
  f(m)
\end{verbatim}
Here, \texttt{*} represents a non-deterministic integer: since we have removed
the input $\sinatom{s}{x}$, we do not have information about the value of \( x \).
As a result, the sequential program above is non-terminating, although the original
process is terminating.
To remedy this problem, we also refine the basic translation above by using a refinement
type system for the \(\pi\)-calculus. Using the refinement type system,
we can infer that the value of \(x\) in the original process is less than \(n\),
so that we can refine the definition of \texttt{f} to:
\begin{verbatim}
 let rec f(n) = ... else (pred(n) [] let x=* in assume(x<n);f(x))
\end{verbatim}
The target program is now terminating, from which
we can deduce that the original process is also terminating.
We have implemented an automated tool based on the refined translation above.

The contributions of this paper are summarized as follows.
\begin{itemize}
\item The formalization of the basic translation from the \(\pi\)-calculus
  (extended with integers) to sequential programs, and a proof of its correctness.
\item The formalization of a refined translation based on a refinement type system.
\item An implementation of the refined translation, including automated refinement type
  inference based on CHC solving, and experiments to evaluate the effectiveness of
  our method.
\end{itemize}

The rest of this paper is structured as follows.
Section~\ref{sec:targetlanguage} introduces the source and target languages
of our translation.
Section~\ref{sec:approach} 
formalizes the basic translation, and proves its correctness.
Section~\ref{sec:refinement} refines the basic translation by using a refinement type system.
Section~\ref{sec:implementation} reports an implementation and experiments.
Section~\ref{sec:relatedwork} discusses related work,
and Section~\ref{sec:conclusion} concludes the paper.

\textbf{Related work}:
% Object detection related datasets/algo in non-medical domain
% Locally labeled CXR dataset
A few CXR datasets have localized abnormality annotations \cite{shih2019augmenting,filice2020crowdsourcing,jaeger2014two} that are curated manually. These are high quality gold standard ground truth datasets but tend to be smaller in scale (< 30,000 images) and have a narrow coverage, with typically only 1-2 labels. In addition, since most labeling efforts only have abnormality semantics attached, no direct relationships with the affected anatomical locations are available. 

%MEHDI: repeated concepts from above. I am removing the following: 

%The lack of anatomic semantics in the annotation is a limitation for complex multi-modal clinical reasoning work, e.g., differential diagnosis, since clinicians often integrate information along anatomical lines, and for downstream report generation tasks, which often requires describing not only the abnormality but also correctly communicate the location of the abnormalities (and medical devices) to the receiving clinicians. 

Two recent CXR datasets have labels for anatomies described in the reports. In \cite{datta2020dataset}, a small manually annotated dataset (2000 reports) included 10 abnormalities that are individually associated with 29 unique spatial locations (anatomies) at the report level. Another CXR dataset has automatically extracted abnormality and anatomy labels as disconnected concepts that are only correlated at the study level from  160,000 reports using a supervised NLP algorithm \cite{bustos2020padchest}. This was trained on a smaller set of manually annotated data. Neither datasets contain localized annotations for the associated CXR images, nor any comparison relation annotations between sequential exams, both of which are available in the Chest ImaGenome dataset. In Table \ref{tab:related}, we present a comparison of our Chest ImagGenome dataset with other datasets available in the literature.

% Table -- Kashyap

% MEdical imaging datasets to go here: Discussed that we will only focus on cxr datasets that are available for this paper. 
% \caption{\color{red} Kashyap, feel free to continue with the table. We should remove the questionmarks and add a line for our dataset (since all others are not graph). For longer text, using abbreviations and explaining them in the caption often works better. If fill in the values is not possible, it is better to remove the table altogether.}


\begin{table}[t!]
\caption{Summary of existing chest X-ray datasets}
\resizebox{\textwidth}{!}{%
\begin{tabular}{@{}lllllllll@{}}
\toprule
\textbf{Dataset} & \textbf{Annotation Level} & \textbf{Annotation Method} & \textbf{Num Labels} & \textbf{Anatomy Labeled} & \textbf{Graph} & \textbf{Dataset Size} & \textbf{Temporal Labels} & \textbf{Reports} \\ \midrule
SIIM-ACR Pneumothorax Segmentation \cite{filice2020crowdsourcing} & Segmentation & Manual + augmented & 1 & No & No & 12,047 & No & No \\
RSNA Pneumonia Detection Challenge   \cite{shih2019augmenting} & Bounding Boxes & Manual & 1 & No & No & 30,000 & No & No \\
Indiana University Chest X-ray collection \cite{demner2016preparing} & Global & Automated & 10 & No & No & 3,813 & No & Yes \\
NIH CXR dataset \cite{wang2017chestx} & Global & Automated & 14 & No & No & 112,120 & No & No \\
PLCO \cite{team2000prostate} & Global & Automated & 24 & Yes & No & 236,000 & Yes & No \\
Stanford CheXpert \cite{irvin2019chexpert} & Global & Automated & 14 & No & No & 224,316 & No & No \\
MIMIC-CXR \cite{johnson2019mimic} & Global & Automated & 14 & No & No & 377,110 & No & Yes \\
Dutta \cite{datta2020dataset} & Global & Manual & 10 & Yes & Yes & 2,000 & No & Yes \\
PadChest \cite{bustos2020padchest} & Global & Manual + automated & 297 & Yes & No & 160,868 & No & Yes \\
Montgomery County Chest X-ray   \cite{jaeger2014two} & Segmentation & Manual & 1 & Yes & No & 138 & No & No \\
Shenzen Hospital Chest X-ray   \cite{jaeger2014two} & Segmentation & Manual & 1 & Yes & No & 662 & No & No \\  \hline \hline
\textbf{Chest ImaGenome} & Bounding Boxes & Automated & 131 & Yes & Yes & 242,072 & Yes & Yes \\
\bottomrule
\end{tabular}%
}
\label{tab:related}
\vspace{-0.4cm}
\end{table}
% removed (Derived from MIMIC-CXR \cite{johnson2019mimic}) % makes table really small

\section{Focused Reading}\label{sec:focusedreading}

Here we consider focused reading for the biomedical domain, and we focus on binary promotion/inhibition interactions between biochemical entities. 
%in the domain of biological interactions among entities and processes within a cell.  
%We assume a simplified model of interactions that can be represented as a directed graph.  
In this setting, the machine reading (or IE) component constructs a directed graph, where vertices represent {\em participants} in an interaction (e.g., protein, gene, or a biological process), and edges represent directed activation interactions. 
% Interactions relate a set of one or more participants ({\em controllers}) to a single outcome ({\em controlled}) participant in the interaction.  
Edge labels indicate whether the controller entity has a {\em positive} (promoting) or {\em negative} (inhibitory) influence on the controlled participant. 
Figure~\ref{fig:fragment} shows an example edge in this graph.

% This representation allows us to encode statements such as {\em mTOR triggers cellular apoptosis} as a graph, as shown in Figure~\ref{fig:fragment}.

\begin{figure}[t]
\begin{center}
  \includegraphics[width=.4\textwidth]{figs/fragment.pdf}
  \caption{Example of a graph edge encoding the relation extracted from the text: \emph{mTOR triggers cellular apoptosis}.}\label{fig:fragment}
  \vspace{-3mm}
  \end{center}
\end{figure}

We use REACH\footnote{\url{https://github.com/clulab/reach}}, an open source IE system~\cite{Valenzuela:15}, to extract interactions from unstructured biomedical text and construct the graph above. %  REACH output maps directly to our simplified graph model representation.
We couple this IE system with a Lucene\footnote{\url{https://lucene.apache.org}} index of biomedical publications to retrieve papers based on queries about participant mentions in the text (as discussed below).

Importantly, we essentially use IE as a black box (thus, our method could potentially work with any IE system), and focus on strategies that guide {\em what} the IE system reads for a complex information need.
In particular, we consider the common scenario where a biologist (or other model-building process) queries the literature on:
\begin{quote}
 How does one participant (source) affect another (destination), where the connection is typically indirect?
 \end{quote}
 This type of queries is common in biology, where such direct/indirect interactions are observed in experiments, but the explanation of why these dependencies exist is unclear.
 
Algorithm~\ref{alg:focusedreading} outlines the general focused reading algorithm for this task.  
% \todo{describe the use case we address here: Focused reading searches for a directed path of interactions from a source to a dest.}.
In the algorithm, $S, D, A$, and $B$ represent individual participants, where $S$ and $D$ are the {\bf s}ource and {\bf d}estination entities in the initial user query. $G$ is the interaction graph that is iteratively constructed during the focused reading procedure, with $V$ being the set of vertices (biochemical entities), and $E$ the set of edges (promotion/inhibition interactions). 
$\mathit{\Sigma}$ is the strategy that chooses which two entities/vertices to be used in the next information retrieval iteration.
$Q$ is a Lucene query automatically constructed in each iteration to retrieve new papers to read.

%\begin{algorithm}
%\caption{Focused Reading}
%\begin{algorithmic}[1]
%{\small
%\Procedure{FocusedReading}{$S,D$}%\Comment{$S$ and $D$ are the src and dst}
%   \State {\scriptsize $G \gets \{\{S, D\}, \emptyset\}$} \label{alg:fr:graph}%\Comment{$G$ holds the search graph}
%   \Repeat \label{alg:fr:start}
%		\State {\scriptsize $(A,B) \gets $\Call{ChooseEndPoints}{$G$}} \label{alg:chendpoints}
%   		\State {\scriptsize $Q \gets$ \Call{ChooseQuery}{$A,B,G$}} \label{alg:chquery}
%   		\State {\scriptsize $(V, E) \gets$\Call{Lucene+Reach}{$Q$}} \label{alg:ie}%\Comment{Query and IE}
%   		\State {\scriptsize \Call{Expand}{$V, E, G$}} \label{alg:reconcile}%\Comment{Add the new elements to the model}
%   \Until{{\scriptsize \Call{IsConnected}{$S,D$}} OR {\scriptsize \Call{StopConditionMet}{$G$}}}\label{alg:fr:stop}
%\EndProcedure
%}
%\end{algorithmic}
%\label{alg:focusedreading}
%\end{algorithm}

\begin{algorithm}
\caption{Focused reading framework}
\begin{algorithmic}[1]
{\small
\Procedure{FocusedReading}{$S,D$}%\Comment{$S$ and $D$ are the src and dst}
   \State {\scriptsize $G \gets \{\{S, D\}, \emptyset\}$} \label{alg:fr:graph}%\Comment{$G$ holds the search graph}
   \Repeat \label{alg:fr:start}
   		\State {\scriptsize $\mathit{\Sigma} \gets $ \Call{EndpointStrategy}{$G$}} \label{alg:endpointsstrategy}
		\State {\scriptsize $(A,B) \gets $ \Call{ChooseEndPoints}{$\mathit{\Sigma},G$}} \label{alg:chendpoints}
   		\State {\scriptsize $Q \gets $ \Call{ChooseQuery}{$A,B,G$}} \label{alg:chquery}
   		\State {\scriptsize $(V, E) \gets$ \Call{Lucene+Reach}{$Q$}} \label{alg:ie}%\Comment{Query and IE}
   		\State {\scriptsize \Call{Expand}{$V, E, G$}} \label{alg:reconcile}%\Comment{Add the new elements to the model}
   \Until{{\scriptsize \Call{IsConnected}{$S,D$}} OR {\scriptsize \Call{StopConditionMet}{$G$}}}\label{alg:fr:stop}
\EndProcedure
}
\end{algorithmic}
\label{alg:focusedreading}
\end{algorithm}


The algorithm initializes the search graph as containing the two unconnected participants as vertices: $\{S, D\}$ (line~\ref{alg:fr:graph}).
%
The algorithm then enters into its central loop (lines \ref{alg:fr:start} through \ref{alg:fr:stop}). The loop terminates when one or more directed paths connecting $S$ to $D$ are found, or when a stopping condition is met: either $G$ has not changed since the previous run through the loop, or after exceeding some number of iterations through the loop (in this work, ten).

At each pass through the loop the algorithm grows the search graph as follows:
\setlist{nolistsep}
%\begin{enumerate}[noitemsep]
%  \item Given the current graph, choose the next participants to attempt to link: $(A,B)$ (line~\ref{alg:chendpoints}).
%  \item Choose a query strategy: {\em exploration} or {\em exploitation} (line~\ref{alg:chquery}).  
%  In general, exploration widens the search space by adding many more nodes to the graph, whereas exploitation narrows the search by focusing on a specific region of the graph.
%  % ms: I did not understand the sentence below, so I rewrote it above.
%  %Exploration makes queries likely to find interactions directly linking $A$ to $B$, while exploitation makes queries more likely to directly link $A$ to $B$.
%%\todo{Explain what explore and exploit mean!}
%  \item Run the Lucene query to retrieve papers and process the papers using the IE system.  The result of this call is a set of interactions, similar to that in Figure~\ref{fig:fragment} (line~\ref{alg:ie}).
%  \item Add the new interaction participants (vertices $V$) and directed influences (edges $E$) to the search graph (line~\ref{alg:reconcile}).
%\end{enumerate}

\begin{enumerate}%[noitemsep]
\item The graph $G$ is initialized with two nodes, the source and destination in the user's information need, and no edges (because we have not read any papers yet).
  \item Given the current graph, choose a strategy, $\mathit{\Sigma}$, for selecting which entities to query next: {\em exploration} or {\em exploitation} (line~\ref{alg:endpointsstrategy}). In general, exploration aims to widen the search space by adding many more nodes to the graph, whereas exploitation aims to narrow the search by focusing on entities in a specific region of the graph.
  \item Using strategy $\mathit{\Sigma}$, choose the next entities to attempt to link: $(A,B)$ (line~\ref{alg:chendpoints}). 
  \item Choose a query, $Q$: again, using {\em exploration} or {\em exploitation}, following the same intuition as with the entity choice strategy (line~\ref{alg:chquery}).
Here exploration queries retrieve a wider range of documents, while exploitation queries are more restrictive.
  
  % ms: I did not understand the sentence below, so I rewrote it above.
  %Exploration makes queries likely to find interactions directly linking $A$ to $B$, while exploitation makes queries more likely to directly link $A$ to $B$.
%\todo{Explain what explore and exploit mean!}
  \item Run the Lucene query to retrieve papers and process the papers using the IE system.  The result of this call is a set of interactions, similar to that in Figure~\ref{fig:fragment} (line~\ref{alg:ie}).
  \item Add the new interaction participant entities (vertices $V$) and directed influences (edges $E$) to the search graph (line~\ref{alg:reconcile}).
    \item If the source and destination entities are connected in $G$, stop: the user's information need has been addressed. Otherwise, continue from step 2.
\end{enumerate}

The central loop performs a bidirectional search in which each iteration expands the search horizon outward from $S$ and $D$.  
% \todo{Add that intuitive figure here to illustrate how the search horizon is grown?}
Algorithm~\ref{alg:focusedreading} represents a family of possible focused reading algorithms, differentiated by how each of the functions in the main loop are implemented.  In this work, {\sc IsConnected} stops after a single path is found, but a variant could consider finding multiple paths, paths of some length, or incorporate other criteria about the properties of the path.  We next consider particular choices for the inner loop functions.
% describe and evaluate
% In the first pass through the loop, $S$ and $D$ are the only vertices in the graph, so the algorithm queries and retrieves papers that mention $S$ and $D$, the IE system reads the papers, and all new interactions are added to the graph.  In the second pass through the loop, {\sc ChooseEndPoints} can now select participants from vertices around $S$ and $D$, and {\sc ChooseQuery} retrieves additional documents that add more participants, growing the search frontier.  As search continues, directed paths outward from $S$ and paths inward toward $D$ are extended until eventually they meet (thus answering the query on how $S$ and $D$ are connected), or the algorithm reaches another terminating condition.

%Figure~\ref{fig:search-graph} illustrates how the graph $G$ evolves during the search.  In this example, $S$ is the protein Pi3k and $D$ is the protein KTF.  In the first pass through the loop, $S$ and $D$ are the only vertices in the graph, so the algorithm queries for and retrieves papers that mention $S$ and $D$, REACH reads the papers, and all new interactions are added to the graph -- in this case, these include all vertices and edges within the inner dashed circles around Pi3k and KTF.  In the second pass through the loop, {\sc ChooseEndPoints} now selects participants from vertices in the subgraphs around Pi3k and KTF -- in this example, we assume one is chosen from each subgraph.  Again, these are used to seed the retrieval documents and extracted interactions are added to the graph; these new additions are represented in second layer of vertices and edges between the dashed and solid lines.  In this way, the central loop performs a kind of bidirectional search in which each iteration expands the search horizon outward from $S$ and $D$. \todo{This paragraph can be simplified. Not sure the figure is needed?}
%
%\begin{figure}[htb]
%\begin{center}
%  \includegraphics[width=.4\textwidth]{figs/search-graph}
%  \caption{Evolution of graph during search}\label{fig:search-graph}
%  \end{center}
%\end{figure}



% Algorithm~\ref{alg:focusedreading} represents a family of possible focused reading algorithms, differentiated by how each of the functions in the main loop are implemented.  For example, in this work, {\sc IsConnected} stops after a single path is found, but it could implemented to consider finding multiple paths, paths of some length, or incorporate other criteria about the properties of the path.  {\sc ChooseEndPoints} is currently implemented to prefer selecting participants that are more highly connected in $G$, as a proxy for their apparent importance in the graph (similar to the intuition behind PageRank~\cite{page1999pagerank}).  {\sc ChooseQuery} is responsible for constructing the Lucene query for retrieving documents involving the current target participants; we describe this further, below.

% It is worth mentioning that so far, the outline of the algorithm has been left  abstract on purpose because the different implementation of function calls in it can customize its behavior. We can override the \texttt{IsConnected} function to stop as soon as a path is found or until a path of length $n$ is found; override \texttt{ChooseEndPoints} to randomize the chosen endpoints of to chose them according to a random criteria, i.e. PageRank \cite{page1999pagerank}; \texttt{ChooseQuery} let's us define a policy to decide whether to query for exploration, exploitation or a composition of both.

% This flexibility is what allows to search for a data driven policy that aims to minimize the number of costly operations (information retrieval and information extraction) while looking for the explanations. 

% In the rest of the paper we will show that a deterministic implementation of focused reading can find explanations otherwise intractable and that it is possible to use reinforcement learning to learn a policy that improves over the deterministic implementation.
\newcommand{\twomoons}{{\tt Twomoons}}
\newcommand{\gauss}{{\tt Gauss}}
\newcommand{\sculpture}{{\tt Sculpture}}
\newcommand{\baseline}{{\tt Baseline}}
\newcommand{\MM}{{\tt MsgPassing}}
\newcommand{\blackboard}{{\tt Blackboard}}
\newcommand{\ncut}{\text{ncut}}
\newcommand{\chensays}[2][]{\textcolor{blue} {\textsc{Jiecao #1:} \emph{#2}}}

\section{Experiments}
In this section we present experimental results for  graph clustering in the message passing and blackboard models. We will compare the following three algorithms. (1) \baseline: each site sends all the data to the coordinator directly; (2) \MM: our algorithm in the message passing model (Section~\ref{sec:gcmessage}); (3) 
\blackboard: our algorithm in  the blackboard model (Section~\ref{sec:bb}).


%Since both of our algorithms are crucially based on the use of spectral scarification, our main focus in the experiments is to investigate to what extend the quality of the spectral clustering algorithms will be affected by using spectral sparsification, the saving of communication costs by using spectral sparsificaion, ...
%
%
%The goal of this experiment is not to demonstrate the effectiveness of the spectral clustering algorithm. We mainly want to investigate the following, 
%\begin{itemize}
%\item to what extend the quality of clustered results will be affected by using spectral sparsification.
%\item saving of communication costs by using spectral sparsifier.
%\item the affect of constants in algorithms of the message passing/blackboard model.
%\end{itemize}
%
%
%\subsection{The Setup}
%\paragraph{Reference Algorithms}
%We compare different algorithms in our experiment.

%Note that we can also run \MM~ in the blackboard model.

Besides giving the visualized results of these algorithms on various datasets, we also measure the qualities of the results via the {\em normalized cut}, defined as 
\[
\ncut(A_1, \ldots, A_{k}) = \frac{1}{2}\sum_{i\in[k]}\frac{w(A_i, V\backslash A_i)}{\vol(A_i)},
\]
 which is a standard objective function to be minimized for spectral clustering algorithms. 
%We will compare the communication costs of these algorithms in different settings.

%We also compare the total communication costs of different algorithms/models. As the unit does not matter in our case, we normalize all communication costs by the cost of \baseline.  Whenever possible, we will visualize the clustered results.

We implemented the algorithms using multiple languages, including Matlab, Python and C++. Our experiments were conducted on an IBM NeXtScale nx360 M4 server, which is equipped with 2 Intel Xeon E5-2652 v2 8-core processors, 32GB RAM and 250GB local storage.


\subsection{Datasets.}
We test the algorithms in the following real and synthetic datasets, which is visualized in \figref{visualization}.


\begin{figure}[h]
     \centering
     \subfigure[\twomoons]{\includegraphics[width=0.23\textwidth]{twomoons-14000-original.png}\label{fig:twomoons}}
     ~~
     \subfigure[\gauss]{\includegraphics[width=0.23\textwidth]{gauss-10000-original.png}\label{fig:gauss}}
     ~~
     \subfigure[\sculpture]{\includegraphics[width=0.13\textwidth,height=0.16\textwidth]{sculpture-11680-original.jpg}\label{fig:sculpture}}
     \caption{Visualization of the datasets for our experiments.}
     \label{fig:visualization}
\end{figure}



\vspace{-1mm}
\begin{itemize}
\item \twomoons : this dataset contains $n=14,000$ coordinates in $\mathbb{R}^2$. We consider each point to be a vertex. For any two vertices $u, v$, we add an edge with weight $w(u,v) = \exp\{-\|u-v\|_2^2/\sigma^2\}$ with $\sigma = 0.1$ when one vertex is among the $7000$-nearest points of the other.  This construction results in a graph with about $110,000,000$ edges.

\item  \gauss : this dataset contains $n = 10,000$ points in $\mathbb{R}^2$. There are $4$ clusters in this dataset, each generated using a Gaussian distribution. We construct a complete graph as the similarity graph.  For any two vertices $u, v$, we define the weight $w(u,v) = \exp\{-\|u-v\|_2^2/\sigma^2\}$ with $\sigma = 1$. The resulting graph has about $100,000,000$ edges.

\item \sculpture : a photo of \textit{The Greek Slave}~\footnote{Available in e.g., \url{http://artgallery.yale.edu/collections/objects/14794}}. We use an $80\times 150$ version of this photo where each pixel is viewed as a vertex. To construct a similarity graph, we map each pixel to a point in $\mathbb{R}^5$, i.e., $(x, y, r, g, b)$, where the latter three coordinates are the RGB values. For any two vertices $u, v$, we  put an edge between $u, v$ with weight $w(u,v) = \exp\{-\|u-v\|_2^2/\sigma^2\}$ with $\sigma = 0.5$ if one of $u, v$ is among the $5000$-nearest points of the other. This results in a graph with about $70,000,000$ edges.
\end{itemize}
\vspace{-1mm}
In the distributed model edges are randomly partitioned across $s$ sites. 

%\vspace{-1.5mm}



\subsection{Results on clustering quality}
%{\em Quality.} \
\begin{figure*}[ht]
     \centering
     \subfigure[\baseline]{\includegraphics[width=0.2\textwidth]{twomoons-14000-original-clustered.png}\label{fig:twomoons-clustered-original}}
     \subfigure[\MM]{\includegraphics[width=0.2\textwidth]{twomoons-14000-sparsify-clustered-15.png}\label{fig:twomoons-clustered-sparsify}}
     \subfigure[\blackboard]{\includegraphics[width=0.2\textwidth]{twomoons-14000-chain-clustered.png}\label{fig:twomoons-clustered-chain}}
     \caption*{\twomoons, $k = 2$;}

\subfigure[\baseline]{\includegraphics[width=0.2\textwidth]{gauss-10000-original-clustered.png}\label{fig:gauss-clustered-original}}
     \subfigure[\MM]{\includegraphics[width=0.2\textwidth]{gauss-10000-sparsify-clustered-15.png}\label{fig:gauss-clustered-sparsify}}
     \subfigure[\blackboard]{\includegraphics[width=0.2\textwidth]{gauss-10000-chain-clustered.png}\label{fig:gauss-clustered-chain}}
     \caption*{\gauss, $k = 4$}


     \subfigure[\baseline]{\includegraphics[width=0.2\textwidth,height=0.2\textwidth]{sculpture-11680-original-clustered.png}\label{fig:sculpture-clustered-original}}  
     \subfigure[\MM]{\includegraphics[width=0.2\textwidth,height=0.2\textwidth]{sculpture-11680-sparsify-clustered-15.png}\label{fig:sculpture-clustered-sparsify}}
     \subfigure[\blackboard]{\includegraphics[width=0.2\textwidth,height=0.2\textwidth]{sculpture-11680-chain-clustered.png}\label{fig:sculpture-clustered-chain}}
     \caption*{\sculpture, $k = 3$. }


     
     \caption{Visualization of the results on \twomoons, \gauss\ and \sculpture. In the message passing model each site samples $5 n$ edges; in the blackboard model all sites jointly sample $10n$ edges (in \twomoons~ and \gauss) or $20n$ edges (in \sculpture) and the chain has length $18$. $s = 15$.}
     \label{fig:quality-1}
\end{figure*}

We visualize the clustered results for 
the \twomoons, \gauss\ and \sculpture\ in Figure~\ref{fig:quality-1}.
% and visualize the clustered results for \gauss\ and \sculpture in Figure~\ref{fig:quality-2}.
It can be seen that \baseline, \MM\ and \blackboard\ give results of very similar qualities.  For simplicity, here we only present the visualization for $s=15$. Similar results were observed when we varied the values of $s$.  
%\he{To Qin: Do you plan to have two titles (Results \& Quality)?}


% \begin{figure*}[h]
%      \centering
% \subfigure[\baseline]{\includegraphics[width=0.3\textwidth]{gauss-10000-original-clustered.png}\label{fig:gauss-clustered-original}}
%      \subfigure[\MM]{\includegraphics[width=0.3\textwidth]{gauss-10000-sparsify-clustered-15.png}\label{fig:gauss-clustered-sparsify}}
%      \subfigure[\blackboard]{\includegraphics[width=0.3\textwidth]{gauss-10000-chain-clustered.png}\label{fig:gauss-clustered-chain}}
%      \caption*{\gauss, $k = 4$}


%      \subfigure[\baseline]{\includegraphics[width=0.2\textwidth]{sculpture-11680-original-clustered.png}\label{fig:sculpture-clustered-original}}  
%      \subfigure[\MM]{\includegraphics[width=0.2\textwidth]{sculpture-11680-sparsify-clustered-15.png}\label{fig:sculpture-clustered-sparsify}}
%      \subfigure[\blackboard]{\includegraphics[width=0.2\textwidth]{sculpture-11680-chain-clustered.png}\label{fig:sculpture-clustered-chain}}
%      \caption*{\sculpture, $k = 3$. }

%      \caption{Visualization of results on \gauss\ and \sculpture; in the message passing model each site samples $5 n$ edges; in the blackboard model all sites jointly sample $10n$ (in \gauss) or $20n$ (in \sculpture) edges and the chain has length $18$.}
%      \label{fig:quality-2}
% \end{figure*}


We also compare the normalized cut (ncut) values of the clustering results of different algorithms.  The results are presented in Figure \ref{fig:quality}. In all datasets, the ncut values of different algorithms are very close. The ncut value of \MM\ slightly decreases when we increase the value of $s$, while the ncut value of \blackboard\ is independent of $s$.
%We comment that in general, it is difficult to compare \MM\ and \blackboard\ directly because they are affected by different parameters.


\begin{figure*}[!ht]
  \centering
  \subfigure[\twomoons]{\includegraphics[width=0.33\textwidth]{twomoons-14000-ncut.png}\label{fig:twomoons-quality}}\hspace*{-1.1em}
  \subfigure[\gauss]{\includegraphics[width=0.31\textwidth]{gauss-10000-ncut.png}\label{fig:gauss-quality}}\hspace*{-1.1em}
  \subfigure[\sculpture]{\includegraphics[width=0.31\textwidth]{sculpture-11680-ncut.png}\label{fig:sculpture-quality}}\hspace*{-1.1em}
  \subfigure{\includegraphics[width=0.14\textwidth]{legend.png}}
     \caption{Comparisons on normalized cuts. In the message passing model, each site samples $5n$ edges; in each round of the algorithm in the blackboard model, all sites jointly sample $10n$ edges (in \twomoons~and \gauss) or $20n$ edges (in \sculpture) edges and the chain has length $18$.}
     \label{fig:quality}
\end{figure*}

%\textcolor{red}{To Jiecao: Can you put the color lines indicating baseline, message passing, and blackboard within one row in Pic 2? Withthis we can save some space.}

%\vspace{-1.5mm}

\subsection{Results on communication costs} 
\begin{figure*}[!ht]
     \centering
     \subfigure[\twomoons]{\includegraphics[width=0.3\textwidth]{twomoons-14000-communication.png}\label{fig:twomoons-communication}}
     \subfigure[\gauss]{\includegraphics[width=0.3\textwidth]{gauss-10000-communication.png}\label{fig:gauss-communication}}
     \subfigure[\sculpture]{\includegraphics[width=0.3\textwidth]{sculpture-11680-communication.png}\label{fig:sculpture-communication}}


     \subfigure[\twomoons]{\includegraphics[width=0.32\textwidth]{twomoons-14000-communication-2.png}\label{fig:twomoons-communication-2}}
     \subfigure[\gauss]{\includegraphics[width=0.32\textwidth]{gauss-10000-communication-2.png}\label{fig:gauss-communication-2}}
     \subfigure[\sculpture]{\includegraphics[width=0.32\textwidth]{sculpture-11680-communication-2.png}\label{fig:sculpture-communication-2}}
     \caption{Comparisons on communication costs. In the message passing model, each site samples $5n$ edges; in each round of the algorithm in the blackboard model, all sites jointly sample $10n$ (in \twomoons~and \gauss) or $20n$ (in \sculpture) edges and the chain has length $18$. }
     \label{fig:communication}
\end{figure*}

We compare the communication costs of different algorithms in Figure \ref{fig:communication}. We observe that while achieving similar clustering qualities as \baseline, both \MM\ and \blackboard\ are significantly more communication-efficient (by one or two orders of magnitudes in our experiments). We also notice that the value of $s$ does not affect the communication cost of \blackboard, while the communication cost of \MM\ grows almost linearly with $s$; when $s$ is large, \MM\ uses significantly more communication than \blackboard. These confirm our theory.  %In Figure~\ref{fig:mm-const} and Figure~\ref{fig:blackboard-const}   in Appendix~\ref{sec:parameters} we present how the performance of \MM\ and \blackboard\ are affected by their parameters.

%
%
%\vspace{-1.5mm}
%\paragraph{Summary.}  From our experimental results we conclude that \MM\ and \blackboard\ achieve similar clustering quality as the native algorithm \baseline, while significantly reduce the communication cost.  When the number of sites is large, \blackboard\ is more communication efficient than \MM, as predicted by our theory.



\subsection{Parameters in \MM\ and \blackboard}
\label{sec:parameters}

Figure \ref{fig:mm-const} shows in \MM how the value of ncut is affected by the number of sites and the number of edges sampled in each site. 
Here, each site samples $cn$ edges. 
When $c=3$ and $s=1$, the ncut value diverges in all datasets. This is because with such a small $c$, the algorithm does not generate a valid sparsifier. In general, increasing $c$ or $s$ will slightly decrease the ncut value. But once they are above some thresholds, the ncut values of \MM\ and \baseline\ become very close.

Figure \ref{fig:blackboard-const} shows in \blackboard  how the ncut value is affected by the number of iterations and the number of edges sampled. When the number of iterations is set to be $5$, ncut values diverge in all datasets. This is because we cannot expect to generate a valid sparsifier by using such few iterations. It can be seen from \ref{fig:bb-gauss-constant} that for a fixed $c$, performing more iterations will help to reduce ncut values. From the same figure, one can also conclude that for fixed iterations, increasing $c$ also helps to reduce the ncut values.



\begin{figure*}[h!t]
     \centering
     \subfigure[\twomoons]{\includegraphics[width=0.3\textwidth]{twomoons-c.png}\label{fig:mm-twomoons-constant}}
     \subfigure[\gauss~dataset]{\includegraphics[width=0.3\textwidth]{gauss-c.png}\label{fig:mm-gauss-constant}}
     \subfigure[\sculpture]{\includegraphics[width=0.3\textwidth]{sculpture-c.png}\label{fig:mm-sculpture-constant}}
     \caption{The pictures above show the $\ncut$ values with respect to the values of $c$ and $s$ for the \MM\ algorithm. Here  
 each site samples $c n$ edges.}
     \label{fig:mm-const}
\end{figure*}


\begin{figure*}[h!t]
     \centering
     \subfigure[\twomoons]{\includegraphics[width=0.3\textwidth]{twomoons-iter.png}\label{fig:bb-twomoons-constant}}
     \subfigure[\gauss]{\includegraphics[width=0.3\textwidth]{gauss-iter.png}\label{fig:bb-gauss-constant}}
     \subfigure[\sculpture]{\includegraphics[width=0.3\textwidth]{sculpture-iter.png}\label{fig:bb-sculpture-constant}}
     \caption{The pictures above show how the $\ncut$ values are affected by the number of iterations and the value of $c$ for the \blackboard\ algorithm. Here 
all sites jointly sample $c n$ edges. }
     \label{fig:blackboard-const}
\end{figure*}






% \vspace{-0.20in}
%\subsection{Analysis:}


\textbf{Use of Multiple Projection Heads:} The use of different projection heads for each view on OpenImages classification gives us a boost of $1.1$ mAP on Obj-Obj+Dilate crop. Pre-training on COCO and finetuning on VOC dataset for object-detection task gives a boost of $0.4$ mAP. Hence using multiple projection heads results in a consistent improvement. 

\textbf{Varying Dilation Parameter:} Table 3 (appendix) shows the effect of varying the dilation parameter. A sweet spot exists at a moderate dilation value of $\delta=0.1$ for COCO object detection. 

% \textbf{Computational Cost:} BING adds negligible time to the pre-training. Generating object proposals takes ~29 mins for the full OHMS dataset (one-time cost) and ~16 mins for COCO. Instead of pre-generating, adding the BING operator to the data loader pipeline has a trivial overhead (+$0.1\%$). %As an example, the wall-clock time taken for 1 epoch of training is 1'46'' for the Dense-CL baseline and 1'45'' for our method.
% \textbf{}



%Between two views, we measure the number of common pixels; and then measure the fraction of these common pixels that overlap with a ground truth bounding box (object). We find that this fraction for COCO is $99\%$ for object-scene crops and $92.1\%$ for the scene-scene crop. In the case of OpenImages-Subset, the numbers are, respectively, $99.1\%$ and $87.3\%$. This is another way of seeing that OpenImages-Subset can benefit more from object-scene crops, borne out by the numbers in Tables \ref{tab:ssl_comparison_classification} and \ref{tab:coco_detection}. 


% \as{Shlok: could you please make this description a little better and clear?}
% We find the overlapping pixels between two crops ($C_{int} = C_1 \cap C_2$). Next we calculate intersection of $C_{int}$  with the most overlapping ground truth object ($O$) and calculate the score $\frac{C_{int} \cap O}{C_{int}}$ for each image and average it. 
% To do this, we calculate the \% intersection of the most overlapping ground truth object with the inter
% Next we try to find the probability of an actual ground truth object co-occuring in between two crops. We find  object-overlap between both Scene-Object crops and Scene-Scene crops. To do this we firstly calculate the overlapping region between two crops. Overlapping region is the area of overlap between two crops before the resize operation. Then for all the ground truth objects present in the original image  we find the object with maximum overlap in the overlapping region. Intuitively for a object to have high overlap, the object should be present in both the crops. 

% Similarly instead of taking an crop with maximum overlap we calculate average of all the crops that are present in the image. We find this average probability to be 65.12 \% for Object-Scene crop and 73.47 \% for Scene-Scene crop. 
% This is consistent with the findings of the InfoMin \cite{tian2020contrastive} that there is a tradeoff between how much information views can share.  

% Similarly in the case of OpenImages we can see from Fig \ref{fig:radius_openimages} that as we increase the radius of the object-object crops the performance firstly increases and then decreases, suggesting there is a sweet point on mutual information on OpenImages dataset as well.
% \\

% \textbf{Performance on 5 classes per image images?}









In this paper, 2D and 3D CNN models were used to generate pelvic sCTs from T1-weighted MR images. Our sCT generation methods were fully automated, requiring no deformable registration or manual segmentation of bone tissues. As shown in Figure~\ref{fig3}, the 2D and 3D CNN models generated high quality sCTs. MAE curves shown in Figure~\ref{fig4} indicated that both models could precisely estimate soft-tissue HU values but had difficulty in reproducing air and high-density bone tissues. 

The MAEs within the body contour across all patients were 40.5 $\pm$ 5.4 HU and 37.6 $\pm$ 5.1 HU for the 2D and 3D models, respectively. The time required for generating a pelvic sCT using our CNN models was about 5.5 s. Our MAE results are comparable to previous studies. Kim $et \ al.$\cite{RN41} presented a voxel-based weighted summation method that produced an MAE of 74.3 $\pm$ 3.9 HU. However, manual contouring of bone tissues required for this method can be tedious and time-consuming. An MAE of 40.5 $\pm$ 8.2 HU was achieved by Dowling $et \ al.$\cite{RN11} using an average MRI-CT atlas from 38 patients. Andreasen $et \ al.$\cite{RN42} reported an MAE of 54 $\pm$ 8 HU using an atlas-based method with pattern recognition, and its prediction time was about 20.8 min. Another random forest model proposed by Andreasen $et \ al.$\cite{RN43} generated sCTs with an MAE of 58 $pm$ 9 HU. A hybrid method suggested by Siversson $et \ al.$ \cite{RN45} obtained an MAE of 36.5 $\pm$ 4.1 HU when ignoring errors introduced by gas cavities. This hybrid method was implemented in the cloud-based commercial software MriPlanner (Spectronic Medical AB, Helsingborg, Sweden), which required 50 to 80 min to generate a sCT.\cite{RN45} The patch-based 3D context-aware generative adversarial network presented by Nie $et \ al.$\cite{RN26} achieved an MAE of 39.0 $\pm$ 4.6 HU. 

Our CNN models reproduced low-density bone as shown in Figure ~\ref{fig4}. The bone-region DSCs were 0.81 $\pm$ 0.04 and 0.82 $\pm$ 0.04 from the 2D and 3D models, respectively. These results are comparable to reported DSC results of 0.79 $\pm$ 0.12\cite{RN10} and 0.91$\pm$0.03{\cite{RN11}}, where the authors compared bone contours manually drawn on the sCT and CT.

It was feasible to train the proposed 3D model with 16 image volumes from scratch. Results of the Wilcoxon signed-rank tests shown in Table~\ref{tab1} demonstrated a statistically significant improvement in overall MAE, bone DSC, and bone precision of the 3D model compared to the 2D model. However, as shown in Figure~\ref{fig4}, the 2D model seemed to perform better in estimating the high-density bone HU values. It should be noted that smaller overall MAEs do not guarantee improved sCT dose calculation and patient positioning performance. While the models performed well, we will continue to acquire more patient data to potentially improve model accuracy and further test model differences.

As this was a retrospective study, the MR image voxel sizes were not matched, resulting in different voxel intensities between images. This may have affected the sCT generation accuracy although we applied intensity normalization. A potential study could examine how voxel size variations affects sCT estimation. 

The proposed 3D model can be implemented on a 12 GB GPU to process volumetric images with dimensions of 256 $\times$ 256 $\times$ 30. More GPU memory would be required to process higher resolution 3D images. Considering the limited access to multi-GPU systems, a 3D architecture with fewer convolutional layers could be considered to deal with higher resolutions. However, the performance could be affected by the reduced parameters and smaller receptive fields of the less complex model. Another approach would be to extract 30-slice sub-volumes from CT and MR images for training the 3D model. The sCT could then be generated by averaging 30-slice sCT sub-volumes produced by the model. 

A number of techniques could be investigated for improving model performance.  Nie $et \ al.$\cite{RN26} showed that introducing an additional adversarial discriminator improved overall sCT quality. The same approach could be adapted in our proposed 2D and 3D CNN models.  Non-rigid deformation\cite{RN44} could also be applied to both CT and MR images in the process of the on-the-fly data augmentation to produce more training pairs. Multiple MR images acquired with different sequences could be fed into models to provide more information for distinguishing different tissues. Multi-GPU systems with more memory would enable the exploration of larger batch sizes for training CNN models, which could reduce variances in gradient estimation and accelerate the training. 



\section*{Acknowledgments}

This work was partially funded by the DARPA Big Mechanism
program under ARO contract W911NF-14-1-0395.

Dr. Mihai Surdeanu discloses a financial interest in Lum.ai. This interest has been disclosed to the University of Arizona Institutional Review Committee and is being managed in accordance with its conflict of interest policies.


\bibliography{emnlp2017}
\bibliographystyle{emnlp_natbib}

\end{document}
