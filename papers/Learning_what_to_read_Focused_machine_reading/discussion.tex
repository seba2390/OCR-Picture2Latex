\section{Discussion and future work}

We introduced a framework for the focused reading of biomedical literature, which is necessary to handle the data overload that plagues even machine reading approaches.
  We have presented a generic focused reading algorithm, an intuitive, strong baseline algorithm that instantiates it,  and formulated an RL approach that learns how to efficiently query the paper repository that feeds the machine reading component. % an improved policy for {\sc ChooseQuery}.  
 % The baseline dramatically reduces the number of papers that may be read while still recovering 68\% of the paths, while the RL Query Policy is even more efficient (i.e., reading fewer papers) while recovering more paths.
% reduces the number of papers further while recovering 72\% of the paths.
We showed that the RL-based focused reading is more efficient than the baseline (i.e., it reads 24\% fewer papers), while answering 7\% more queries.

There are many exciting directions in which to take this work.  First, more of the focused reading algorithm can be subject to RL, with the {\sc ChooseEndPoints} policy being the clear next candidate.  
Second, we can expand focused reading to efficiently search for multiple paths between $S$ and $D$.  
Finally, we will incorporate additional biological constraints (e.g., focus on pathways that exist in specific species) into the search itself. 
% ; biologists are interested in paths that occur in particular biological contexts (e.g., in specific tissues), and paths that help explain specific kinds of influence.  
%Finally, we must incorporate additional constraints into the search itself; biologists are interested in paths that occur in particular biological contexts (e.g., healthy versus cancer pancreatic cells), and are also often not just interested in any path, but in paths that help explain specific kinds of influence.  
% These constraints should be incorporated into the ranking of candidate search strategies.



%We have introduced a a framework to do focused reading to find influence paths that explain how biochemical reactions a cellular processes affect each other according to the scientific literature made available by the research community. We do this while avoid doing exhaustive machine reading against very large datasets by learning a policy that steers us towards selecting documents which are likely informative for our purposed and attempting to ignore irrelevant information at each step of the process.
%
%While we have shown that this is possible, we treated this problem very abstractly and ignored certain constraints that could lead to a more useful solution for researchers.
%
%Improvements to this work can be explored in two different lines: Improving the modeling of focused reading and incorporating semantics into the decision process.
%
%On the modeling side, we are interested in improving the query and endpoints choosing strategies by exploring more sophisticated and finer grained algorithms. Instead of having two coarse grained strategies, we could have a family of parameterized strategies, for example, by selecting the amount of papers to retrieve depending on the state. This would create a larger, finer grained action space which may yield improvements in the amount of resources consumed by machine reading. Better endpoint choosing strategies could allow us to take advantage of them  with the policy. We found a result using a very simple reinforcement learning algorithm because we considered very simple and restricted state and action spaces. Refining the modeling of the problem will require more modern RL algorithms to deal with the increase in complexity. More sophisticated and modern reinforcement learning algorithms could helps deal with larger action and space states.
%
%Another useful improvement would be to tune the stop conditions to let focused reading retrieve more than one explanation and rank them according to a useful criteria.

%On the semantics side we are interested in incorporating \emph{contextual information}: Build the search graph in such a way that the individual interactions in an explanation path are consistent within their biological container context, for example, if we are concerned about an explanation of the influence between an protein and cellular apoptosis in the human pancreas, we would like to avoid getting an explanation path where the supporting evidence came from a sentence that states a phenomena happening on mice brain, but consistent with the chain of influence. We would also like to verify whether an explanation is feasible according to a \todo{give more details about this}{\emph{biological model}}. This is equivalent to asking a scientist whether the explanation yield by focused reading makes sense biologically.