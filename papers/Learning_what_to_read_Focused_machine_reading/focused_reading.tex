\section{Focused Reading}\label{sec:focusedreading}

Here we consider focused reading for the biomedical domain, and we focus on binary promotion/inhibition interactions between biochemical entities. 
%in the domain of biological interactions among entities and processes within a cell.  
%We assume a simplified model of interactions that can be represented as a directed graph.  
In this setting, the machine reading (or IE) component constructs a directed graph, where vertices represent {\em participants} in an interaction (e.g., protein, gene, or a biological process), and edges represent directed activation interactions. 
% Interactions relate a set of one or more participants ({\em controllers}) to a single outcome ({\em controlled}) participant in the interaction.  
Edge labels indicate whether the controller entity has a {\em positive} (promoting) or {\em negative} (inhibitory) influence on the controlled participant. 
Figure~\ref{fig:fragment} shows an example edge in this graph.

% This representation allows us to encode statements such as {\em mTOR triggers cellular apoptosis} as a graph, as shown in Figure~\ref{fig:fragment}.

\begin{figure}[t]
\begin{center}
  \includegraphics[width=.4\textwidth]{figs/fragment.pdf}
  \caption{Example of a graph edge encoding the relation extracted from the text: \emph{mTOR triggers cellular apoptosis}.}\label{fig:fragment}
  \vspace{-3mm}
  \end{center}
\end{figure}

We use REACH\footnote{\url{https://github.com/clulab/reach}}, an open source IE system~\cite{Valenzuela:15}, to extract interactions from unstructured biomedical text and construct the graph above. %  REACH output maps directly to our simplified graph model representation.
We couple this IE system with a Lucene\footnote{\url{https://lucene.apache.org}} index of biomedical publications to retrieve papers based on queries about participant mentions in the text (as discussed below).

Importantly, we essentially use IE as a black box (thus, our method could potentially work with any IE system), and focus on strategies that guide {\em what} the IE system reads for a complex information need.
In particular, we consider the common scenario where a biologist (or other model-building process) queries the literature on:
\begin{quote}
 How does one participant (source) affect another (destination), where the connection is typically indirect?
 \end{quote}
 This type of queries is common in biology, where such direct/indirect interactions are observed in experiments, but the explanation of why these dependencies exist is unclear.
 
Algorithm~\ref{alg:focusedreading} outlines the general focused reading algorithm for this task.  
% \todo{describe the use case we address here: Focused reading searches for a directed path of interactions from a source to a dest.}.
In the algorithm, $S, D, A$, and $B$ represent individual participants, where $S$ and $D$ are the {\bf s}ource and {\bf d}estination entities in the initial user query. $G$ is the interaction graph that is iteratively constructed during the focused reading procedure, with $V$ being the set of vertices (biochemical entities), and $E$ the set of edges (promotion/inhibition interactions). 
$\mathit{\Sigma}$ is the strategy that chooses which two entities/vertices to be used in the next information retrieval iteration.
$Q$ is a Lucene query automatically constructed in each iteration to retrieve new papers to read.

%\begin{algorithm}
%\caption{Focused Reading}
%\begin{algorithmic}[1]
%{\small
%\Procedure{FocusedReading}{$S,D$}%\Comment{$S$ and $D$ are the src and dst}
%   \State {\scriptsize $G \gets \{\{S, D\}, \emptyset\}$} \label{alg:fr:graph}%\Comment{$G$ holds the search graph}
%   \Repeat \label{alg:fr:start}
%		\State {\scriptsize $(A,B) \gets $\Call{ChooseEndPoints}{$G$}} \label{alg:chendpoints}
%   		\State {\scriptsize $Q \gets$ \Call{ChooseQuery}{$A,B,G$}} \label{alg:chquery}
%   		\State {\scriptsize $(V, E) \gets$\Call{Lucene+Reach}{$Q$}} \label{alg:ie}%\Comment{Query and IE}
%   		\State {\scriptsize \Call{Expand}{$V, E, G$}} \label{alg:reconcile}%\Comment{Add the new elements to the model}
%   \Until{{\scriptsize \Call{IsConnected}{$S,D$}} OR {\scriptsize \Call{StopConditionMet}{$G$}}}\label{alg:fr:stop}
%\EndProcedure
%}
%\end{algorithmic}
%\label{alg:focusedreading}
%\end{algorithm}

\begin{algorithm}
\caption{Focused reading framework}
\begin{algorithmic}[1]
{\small
\Procedure{FocusedReading}{$S,D$}%\Comment{$S$ and $D$ are the src and dst}
   \State {\scriptsize $G \gets \{\{S, D\}, \emptyset\}$} \label{alg:fr:graph}%\Comment{$G$ holds the search graph}
   \Repeat \label{alg:fr:start}
   		\State {\scriptsize $\mathit{\Sigma} \gets $ \Call{EndpointStrategy}{$G$}} \label{alg:endpointsstrategy}
		\State {\scriptsize $(A,B) \gets $ \Call{ChooseEndPoints}{$\mathit{\Sigma},G$}} \label{alg:chendpoints}
   		\State {\scriptsize $Q \gets $ \Call{ChooseQuery}{$A,B,G$}} \label{alg:chquery}
   		\State {\scriptsize $(V, E) \gets$ \Call{Lucene+Reach}{$Q$}} \label{alg:ie}%\Comment{Query and IE}
   		\State {\scriptsize \Call{Expand}{$V, E, G$}} \label{alg:reconcile}%\Comment{Add the new elements to the model}
   \Until{{\scriptsize \Call{IsConnected}{$S,D$}} OR {\scriptsize \Call{StopConditionMet}{$G$}}}\label{alg:fr:stop}
\EndProcedure
}
\end{algorithmic}
\label{alg:focusedreading}
\end{algorithm}


The algorithm initializes the search graph as containing the two unconnected participants as vertices: $\{S, D\}$ (line~\ref{alg:fr:graph}).
%
The algorithm then enters into its central loop (lines \ref{alg:fr:start} through \ref{alg:fr:stop}). The loop terminates when one or more directed paths connecting $S$ to $D$ are found, or when a stopping condition is met: either $G$ has not changed since the previous run through the loop, or after exceeding some number of iterations through the loop (in this work, ten).

At each pass through the loop the algorithm grows the search graph as follows:
\setlist{nolistsep}
%\begin{enumerate}[noitemsep]
%  \item Given the current graph, choose the next participants to attempt to link: $(A,B)$ (line~\ref{alg:chendpoints}).
%  \item Choose a query strategy: {\em exploration} or {\em exploitation} (line~\ref{alg:chquery}).  
%  In general, exploration widens the search space by adding many more nodes to the graph, whereas exploitation narrows the search by focusing on a specific region of the graph.
%  % ms: I did not understand the sentence below, so I rewrote it above.
%  %Exploration makes queries likely to find interactions directly linking $A$ to $B$, while exploitation makes queries more likely to directly link $A$ to $B$.
%%\todo{Explain what explore and exploit mean!}
%  \item Run the Lucene query to retrieve papers and process the papers using the IE system.  The result of this call is a set of interactions, similar to that in Figure~\ref{fig:fragment} (line~\ref{alg:ie}).
%  \item Add the new interaction participants (vertices $V$) and directed influences (edges $E$) to the search graph (line~\ref{alg:reconcile}).
%\end{enumerate}

\begin{enumerate}%[noitemsep]
\item The graph $G$ is initialized with two nodes, the source and destination in the user's information need, and no edges (because we have not read any papers yet).
  \item Given the current graph, choose a strategy, $\mathit{\Sigma}$, for selecting which entities to query next: {\em exploration} or {\em exploitation} (line~\ref{alg:endpointsstrategy}). In general, exploration aims to widen the search space by adding many more nodes to the graph, whereas exploitation aims to narrow the search by focusing on entities in a specific region of the graph.
  \item Using strategy $\mathit{\Sigma}$, choose the next entities to attempt to link: $(A,B)$ (line~\ref{alg:chendpoints}). 
  \item Choose a query, $Q$: again, using {\em exploration} or {\em exploitation}, following the same intuition as with the entity choice strategy (line~\ref{alg:chquery}).
Here exploration queries retrieve a wider range of documents, while exploitation queries are more restrictive.
  
  % ms: I did not understand the sentence below, so I rewrote it above.
  %Exploration makes queries likely to find interactions directly linking $A$ to $B$, while exploitation makes queries more likely to directly link $A$ to $B$.
%\todo{Explain what explore and exploit mean!}
  \item Run the Lucene query to retrieve papers and process the papers using the IE system.  The result of this call is a set of interactions, similar to that in Figure~\ref{fig:fragment} (line~\ref{alg:ie}).
  \item Add the new interaction participant entities (vertices $V$) and directed influences (edges $E$) to the search graph (line~\ref{alg:reconcile}).
    \item If the source and destination entities are connected in $G$, stop: the user's information need has been addressed. Otherwise, continue from step 2.
\end{enumerate}

The central loop performs a bidirectional search in which each iteration expands the search horizon outward from $S$ and $D$.  
% \todo{Add that intuitive figure here to illustrate how the search horizon is grown?}
Algorithm~\ref{alg:focusedreading} represents a family of possible focused reading algorithms, differentiated by how each of the functions in the main loop are implemented.  In this work, {\sc IsConnected} stops after a single path is found, but a variant could consider finding multiple paths, paths of some length, or incorporate other criteria about the properties of the path.  We next consider particular choices for the inner loop functions.
% describe and evaluate
% In the first pass through the loop, $S$ and $D$ are the only vertices in the graph, so the algorithm queries and retrieves papers that mention $S$ and $D$, the IE system reads the papers, and all new interactions are added to the graph.  In the second pass through the loop, {\sc ChooseEndPoints} can now select participants from vertices around $S$ and $D$, and {\sc ChooseQuery} retrieves additional documents that add more participants, growing the search frontier.  As search continues, directed paths outward from $S$ and paths inward toward $D$ are extended until eventually they meet (thus answering the query on how $S$ and $D$ are connected), or the algorithm reaches another terminating condition.

%Figure~\ref{fig:search-graph} illustrates how the graph $G$ evolves during the search.  In this example, $S$ is the protein Pi3k and $D$ is the protein KTF.  In the first pass through the loop, $S$ and $D$ are the only vertices in the graph, so the algorithm queries for and retrieves papers that mention $S$ and $D$, REACH reads the papers, and all new interactions are added to the graph -- in this case, these include all vertices and edges within the inner dashed circles around Pi3k and KTF.  In the second pass through the loop, {\sc ChooseEndPoints} now selects participants from vertices in the subgraphs around Pi3k and KTF -- in this example, we assume one is chosen from each subgraph.  Again, these are used to seed the retrieval documents and extracted interactions are added to the graph; these new additions are represented in second layer of vertices and edges between the dashed and solid lines.  In this way, the central loop performs a kind of bidirectional search in which each iteration expands the search horizon outward from $S$ and $D$. \todo{This paragraph can be simplified. Not sure the figure is needed?}
%
%\begin{figure}[htb]
%\begin{center}
%  \includegraphics[width=.4\textwidth]{figs/search-graph}
%  \caption{Evolution of graph during search}\label{fig:search-graph}
%  \end{center}
%\end{figure}



% Algorithm~\ref{alg:focusedreading} represents a family of possible focused reading algorithms, differentiated by how each of the functions in the main loop are implemented.  For example, in this work, {\sc IsConnected} stops after a single path is found, but it could implemented to consider finding multiple paths, paths of some length, or incorporate other criteria about the properties of the path.  {\sc ChooseEndPoints} is currently implemented to prefer selecting participants that are more highly connected in $G$, as a proxy for their apparent importance in the graph (similar to the intuition behind PageRank~\cite{page1999pagerank}).  {\sc ChooseQuery} is responsible for constructing the Lucene query for retrieving documents involving the current target participants; we describe this further, below.

% It is worth mentioning that so far, the outline of the algorithm has been left  abstract on purpose because the different implementation of function calls in it can customize its behavior. We can override the \texttt{IsConnected} function to stop as soon as a path is found or until a path of length $n$ is found; override \texttt{ChooseEndPoints} to randomize the chosen endpoints of to chose them according to a random criteria, i.e. PageRank \cite{page1999pagerank}; \texttt{ChooseQuery} let's us define a policy to decide whether to query for exploration, exploitation or a composition of both.

% This flexibility is what allows to search for a data driven policy that aims to minimize the number of costly operations (information retrieval and information extraction) while looking for the explanations. 

% In the rest of the paper we will show that a deterministic implementation of focused reading can find explanations otherwise intractable and that it is possible to use reinforcement learning to learn a policy that improves over the deterministic implementation.