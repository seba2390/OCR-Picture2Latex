

We have provided compelling evidence that the runtime of \Alfa{} follows a long-tailed or lognormal distribution.
According to~\cite{SMP11RestartStrategies}, the usefulness of restarts is a necessary, however not 
a sufficient, condition to obtain super-linear speedups by 
parallelization. Since we have shown that the necessary 
condition is (presumably) satisfied, this immediately raises the 
question of whether super-linear speedups are obtained by parallelizing 
\Alfa{}-type algorithms.



We additionally want to pose the question whether some of the Conjectures~\ref{conj:strong} or~\ref{conj:weak} can be theoretically proven.
A first line of attack would be to analyze the special case of an %
solver like \SRWA{} whose runtime was already theoretically analyzed.


The technique of analyzing the runtime distribution of \Alfa{} could be further developed to help better understand the behavior of CDCL solvers.
These kind of solvers heavily employ the technique of adding new clauses and deleting some clauses.
This can be thought of as solving a new logically equivalent formula of the base instance.

Preliminary results on the solvers excluded for heuristic reasons seem to suggest that the \Alfa{}-method forces the runtime of the base solver to exhibit a multimodal behavior. Thus, the lognormal distribution is not a good fit in this case. However, an initial visual inspection of the data indicates an even heavier tail.

