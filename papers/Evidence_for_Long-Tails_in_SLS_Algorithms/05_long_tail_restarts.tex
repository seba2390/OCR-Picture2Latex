If the Strong Conjecture holds, \ie if the runtimes are lognormally distributed, then restarts are useful~\cite{Lorenz18RuntimeDistributions}.
This section extends this result and mathematically proves that restarts are useful even if only the Weak Conjecture holds.
This will be achieved by showing that restarts are useful for long-tailed distributions.


A condition 
for the usefulness of restarts, as defined in %
Definition~\ref{def:RestartsUseful}, was 
proven
in~\cite{Lorenz18RuntimeDistributions}. %
We will show the result using this theorem that is restated below.

\begin{theorem}[\cite{Lorenz18RuntimeDistributions}]
	\label{theo:sufficient}
	Let $X$ be a positive, real-valued random variable having quantile function~$Q$, then restarts are useful if and only if there is a quantile $p \in (0,1)$ such that
	\begin{equation*}
		R(p,X) :=
		(1-p)\cdot \frac{Q(p)}{\expectation{X}}+\frac{\integralLow[u]{0}{p}{Q(u)}}{\expectation{X}}<p.
	\end{equation*}
\end{theorem}


Even if the quantile function and the expected value are unknown, 
$R(p,X)$
can be characterized for large values of~$p$.

\begin{lemma}
	\label{lem:exp_infinite_restart}
	Consider a positive, real-valued random variable~$X$ with pdf~$f$ and quantile function~$Q$ such that~\mbox{$\expectation{X}<\infty$}.
	Also, assume that the limit $\lim_{t\rightarrow\infty} t^2 \cdot f(t)$ exists.
	Then,
	\begin{align*}
		\lim\limits_{p\rightarrow 1} R(p,X)
		= \lim\limits_{p\rightarrow 1} \, \left((1-p)\cdot \frac{Q(p)}{\expectation{X}}+\frac{\integralLow[u]{0}{p}{Q(u)}}{\expectation{X}}\right)
		= 1.
	\end{align*}
\end{lemma}

\begin{proof}
	In the following, let~$F$ and~$f$ be the cdf and pdf of~$X$, respectively.
	We start by specifying the derivative of~$Q$ with respect to~$p$ as a preliminary consideration.
	From $F = Q^{-1}$ and the application of the inverse function theorem~\cite{rudin1964principles} it follows:
	\begin{align}
	\label{eq:quantil_deriv}
		Q'(p) \coloneqq \frac{\D}{\D p} Q(p)=\frac{1}{f\bigl(Q(p)\bigr)}. 
	\end{align}
	
	As the first step in our proof, we consider the limiting value of the second summand of~$R(p,X)$.
	This value can be determined by integration by substitution with~\mbox{$x=Q(u)$} followed by applying the change of variable method with~\mbox{$p=F(t)$}:
	\begin{align*}
		\lim\limits_{p\rightarrow 1}\frac{\integralLow[u]{0}{p}{Q(u)}}{\expectation{X}}
		=\lim\limits_{p\rightarrow 1}\frac{\integralLow{0}{Q(p)}{x\cdot f(x)}}{\expectation{X}}
		=\lim\limits_{t\rightarrow \infty} \frac{\integralLow{0}{t}{x\cdot f(x)}}{\expectation{X}} =1.
	\end{align*}
	The last equality 
	holds %
	because the 
	numerator
	matches the definition of the expected value.
	
	Next, we examine the limit
	of 
	$(1-p)Q(p)/\expectation{X}$.
	Since 
	$\lim_{p \to 1} (1-p) = 0$,
	the limit of \mbox{$(1-p)\cdot Q(p)/\expectation{X}$} needs to be examined 
	more closely.
	For this purpose, L'Hospital's rule is applied twice as well as the change of variable method with \mbox{$p=F(t)$} is used in the following:
	\begin{align*}
		\lim\limits_{p\rightarrow 1} (1-p)\cdot Q(p)
		= \lim\limits_{p\rightarrow 1} Q(p)^2\cdot f\bigl(Q(p)\bigr)
		= \lim\limits_{t\rightarrow \infty} t^2\cdot f(t).
	\end{align*}
	It is well-known that if \mbox{$\liminf_{t\rightarrow \infty} t^2 \cdot f(t) >0$} were to hold, then the expected value $\expectation{X}$ would be infinite (this statement is, for example, implicitly given in~\cite{foss2011introduction}).
	This would contradict the premise of the lemma; therefore, \mbox{$\liminf_{t\rightarrow \infty} t^2 \cdot f(t) = 0$}.
	Moreover, since, by assumption, $\lim_{t\rightarrow\infty} t^2 \cdot f(t)$ exists, we may conclude that 
	\[
	\lim\limits_{t\rightarrow\infty} t^2 \cdot f(t) = \limsup\limits_{t\rightarrow\infty} t^2 \cdot f(t) = \liminf\limits_{t\rightarrow\infty} t^2 \cdot f(t) = 0. \qedhere
	\]
\end{proof}

A frequently used tool for the description of distributions is the hazard rate function.

\begin{definition}[\cite{rausand2020system}]
	\label{def:hazard_rate}
	Let $X$ be a positive, real-valued random variable having cdf~$F$ and pdf~$f$.
	The \introduceterm{hazard rate function} $r \colon \Rpos \to \Rpos$ of $X$ is given by %
		\begin{align*}
			r(t) := \frac{f(t)}{1-F(t)}.
		\end{align*}
\end{definition}

In particular, there is an interesting relationship between the long-tail property and the hazard rate function's behavior.





\begin{restatelemmaorig}[\cite{nair2020fundamentals}]
	\label{lem:restateorig}
	\label{lem:appendixlemma}
	Let $X$ be a positive, real-valued random variable with hazard rate function~$r$ such that the limit \mbox{$\lim\limits_{t\rightarrow\infty}r(t)$} exists.
	Then $X$ is long-tailed if and only if \mbox{$\LIM{t} r(t)=0$}.
\end{restatelemmaorig}



\begin{proof}
	The appendix contains a proof of this lemma since the manuscript~\cite{nair2020fundamentals} was still unpublished at the time of writing.
\end{proof}


With the help of these preliminary considerations, we are now ready to show that restarts are useful for long-tailed distributions.

\begin{theorem}
	\label{theo:long_tail_restarts}
	Consider a positive, long-tailed random variable~$X$ with continuous pdf~$f$ and hazard rate function~$r$. 
	Also assume that either \mbox{$\expectation{X}=\infty$} holds or the limits \mbox{$\lim_{t\rightarrow\infty}r(t)$} and \mbox{$\lim_{t\rightarrow\infty} t^2 \cdot f(t)$} both exist.
	In both cases, restarts are useful for $X$.
\end{theorem}

\begin{proof}
	Let $F$ be the cdf and $Q$ the quantile function of $X$.
	We begin with the case 
	$\expectation{X} = \infty$.
	According to \refTheo{theo:sufficient}, restarts are useful if and only if
		\[
		(1-p)\cdot \frac{Q(p)}{\expectation{X}} + \frac{1}{\expectation{X}} \cdot \integralLow[u]{0}{p}{Q(u)} < p
		\]
	for some $p \in (0,1)$. However, if the expected value~$\expectation{X}$ is infinite, then the left side of this inequality is zero and the inequality is obviously satisfied. Hence, the statement follows.
	
	Secondly, we assume that $\expectation{X} < \infty$ and that
	both
	\mbox{$\Lim{t}r(t)$} and \mbox{$\Lim{t} t^2 \cdot f(t)$} exist. \refEq{eq:quantil_deriv} can now be used to calculate the following derivative:
	\begin{align*}
		\frac{\D}{\D p} \Big( R(p,X) - p \Big)
		= \frac{\D}{\D p} \left((1-p)\cdot \frac{Q(p)}{\expectation{X}}+\frac{\integralLow[u]{0}{p}{Q(u)}}{\expectation{X}}-p\right)
		= \frac{1-p}{\expectation{X}\cdot f\bigl(Q(p)\bigr)}-1.
	\end{align*}
	Consider the limit of this expression for $p\rightarrow 1$. Once again, the change of variable method is applied with $p=F(t)$, resulting in:
	\begin{align*}
		\lim\limits_{p\rightarrow 1} \frac{1-p}{\expectation{X}\cdot f\bigl(Q(p)\bigr)}-1 
		= \lim\limits_{t\rightarrow \infty} \frac{1-F(t)}{\expectation{X}\cdot f(t)}-1 
		= \lim\limits_{t\rightarrow \infty} \frac{1}{\expectation{X}\cdot r(t)}-1.
	\end{align*}
	By assumption, $X$ has a long-tail distribution and the limit of \mbox{$\Lim{t}r(t)$} exists. For this reason, \mbox{$\lim_{t\rightarrow \infty}r(t)=0$} follows as a result of \refLem{lem:appendixlemma}. Furthermore, since $\expectation{X} < \infty$ holds, we may conclude that  	
	\begin{align}
		\label{eq:long_tail_exp_infinity}
		\lim\limits_{p\rightarrow 1} \frac{1-p}{\expectation{X}\cdot f\bigl(Q(p)\bigr)}-1 
		= \lim\limits_{t\rightarrow \infty} \frac{1}{\expectation{X}\cdot r(t)}-1 = \infty.
	\end{align}
	The condition from \refTheo{theo:sufficient} can be rephrased in such a way that restarts are useful if and only if 
	$R(p,X) - p < 0$.
	According to \refLem{lem:exp_infinite_restart}, the left-hand side of this inequality approaches $0$ for $p\rightarrow 1$.
	However, as has been shown in \refEq{eq:long_tail_exp_infinity}, the derivative of 
	$R(p,X)-p$
	approaches infinity for $p\rightarrow 1$. These two observations imply that there is a $p\in (0,1)$ satisfying 
	$R(p,X) - p < 0$.
	Consequently, restarts are useful for~$X$.
\end{proof}

It should be noted that the conditions of this theorem are not restrictive since all naturally occurring long-tail distributions satisfy these conditions (see also~\cite{nair2020fundamentals}).


\begin{conjecture}[Corollary of the Weak Conjecture]
	Restarts are useful for \Alfa{} with $\textsf{SLS} \in \set{\texttt{SRWA}, \texttt{probSAT}, \texttt{YalSAT}}$.
\end{conjecture}

If \refCon{conj:weak}  is true, then this statement follows immediately by \refTheo{theo:long_tail_restarts}.


