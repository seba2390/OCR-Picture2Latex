

This section proves the connection between long-tailed distributed random variables and their hazard rate functions of \refLem{lem:restateorig}, restated next. This restatement is presented in a stronger form 
by providing
an additional equivalent statement.


\begin{restatelemma}	
	Let $X$ be a positive, real-valued random variable with hazard rate function~$r$ such that the limit \mbox{$\lim_{t\rightarrow\infty}r(t)$} exists.
	Then, the following three statements are equivalent:
	\begin{enumerate}
		\item $X$ is long-tailed. \label{proof:long_tail_1}
		\item $\LIM{x} \integral[t]{x}{x+y}{r(t)} = 0$, \quantorsep $\forall y > 0$.
		\label{proof:long_tail_2}
		\item $\LIM{t} r(t)=0$.
		\label{proof:long_tail_3}
	\end{enumerate}
\end{restatelemma}

\begin{proof}[Proof of \refLem{lem:restateorig}]
	The proof is taken from~\cite{nair2020fundamentals}. A well-known property of the hazard rate function is:
	\begin{align}
		\prob{X > x} = \exp{\left(-\integralLow[t]{0}{x}{r(t)}\right)}.\label{eq:hazard_rate_connection_survival}
	\end{align}
	In the following, let $y \in \Rpos$ be any positive real number. First, we show the equivalence of the first and second statement.
	In line with the definition of long-tail distributions (see \refDef{def:long_tail}), we examine the quotient~$\frac{\prob{X > x+y}}{\prob{X>x}}$ for this purpose and apply the characterization from \refEq{eq:hazard_rate_connection_survival}:
	\begin{align*}
		\frac{\prob{X > x+y}}
		{\prob{X > x}} =
		\frac{\exp{\Big(-\integralLow[t]{0}{x+y}{r(t)}\Big)}}
		{\exp{\Big(-\integralLow[t]{0}{x}{r(t)}\Big)}} =
		\exp{\left(-\integral[t]{x}{x+y}{r(t)}\right)}.
	\end{align*}
	As one can readily infer from this equation, \mbox{$\Lim{x} \frac{\prob{X > x+y}}{\prob{X>x}}=1$} is true if and only if \mbox{$\Lim{x} \integralLow[t]{x}{x+y}{r(t)} = 0$} is true. Thus, the equivalence of the first and second statements has been demonstrated.
	
	In the next step, we show that the second statement implies the third statement, using logical contraposition to prove this. We, therefore, assume that \mbox{$\liminf\limits_{t\rightarrow \infty}r(t) > 0$} holds, or in other words, the following holds:
	\begin{align*}
		\exists C>0 \quantorsep \exists x_0 > 0 \quantorsep \forall x > x_0: \quantorsep
		r(x) \geq C.
	\end{align*}
	By exploiting this property, we are able to estimate the integral \mbox{$\integralLow[t]{x}{x+y}{r(t)}$}:
	\begin{align*}
		\forall x > x_0:\quantorsep
		\integral[t]{x}{x+y}{r(t)} \geq \integral[t]{x}{x+y}{C} = y\cdot C > 0.
	\end{align*}
	Thus it is shown that if the third statement does not hold, the second statement does not hold either.
	
	In the last step, we shall demonstrate that the third statement implies the second statement. Since $r(t)\geq 0$ holds for all $t\in \Rpos$, one may express \mbox{$\Lim{t} r(t)=0$} as follows:
	\begin{align*}
		\forall C>0 \quantorsep \exists x_0 > 0 \quantorsep \forall x > x_0:\quantorsep
		r(x) < C.
	\end{align*}
	Once again, we may exploit this property to estimate the integral~\mbox{$\integralLow[t]{x}{x+y}{r(t)}$}:
	\begin{align*}
		\forall x > x_0:\quantorsep
		\integral[t]{x}{x+y}{r(t)} < \integral[t]{x}{x+y}{C} = y\cdot C.
	\end{align*}
	Since this estimate holds for all $C>0$, this immediately yields \mbox{$\LIM{x} \integral[t]{x}{x+y}{r(t)} = 0$}.
\end{proof}