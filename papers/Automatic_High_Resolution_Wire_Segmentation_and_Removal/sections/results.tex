%auto-ignore
\section{Experiments} \label{sec:results}

\subsection{Implementation Details}
\paragraph{Wire Segmentation Network.}
We experiment with ResNet-50~\cite{resnet} and MixTransformer-B2~\cite{segformer} as our shared feature extractor. We expand the input RGB channel to six channels by concatenating the conditional probability map, min- and max-filtered luminance channels. For the min and max filtering, we use a fixed 6x6 kernel. We use separate decoders for the coarse and fine modules, denoted as $D_C$ and $D_F$ respectively.

%\textcolor{purple}{
We use the MLP decoder proposed in~\cite{segformer} for the MixTransformer segmentation model, and the ASPP decoder in~\cite{deeplabv3p} for our ResNet-50 segmentation model. In both the segmentation and inpainting modules, we take the per-pixel average of the predicted probability when merging overlapping patches. To crop $P_\mathrm{con}$ from $P_\mathrm{glo}$, we upsample the predicted $P_\mathrm{glo}$ to the original resolution, then crop the predicted regions according to the sliding window position.
%}

%\textcolor{purple}{
To train the segmentation module, we downsample the image $I_\mathrm{glo}$ to $p\times p$ to obtain $I_\mathrm{glo}^\mathrm{ds}$. From $I_\mathrm{glo}$, we randomly crop one $p\times p$ patch $I_\mathrm{loc}$ that contains at least 1\% wire pixels. This gives a pair of $I_\mathrm{glo}^\mathrm{ds}$ and $I_\mathrm{loc}$ to compute the losses. During inference, $I_\mathrm{glo}^\mathrm{ds}$ is obtained in the same way as training, while multiple $I_\mathrm{loc}$ are obtained via a sliding window sampled only when the proportion of wire pixels is above $\alpha$. All feature extractors are pretrained on ImageNet.
%, while the decoders are trained from scratch.
%}

We train our model on 5000 training images. The model is trained for 80k iterations with a batch size of 4. We set patch size $p = 512$ during training. 
%Global images are downsampled to 512$\times$512. We feed the global images to the coarse module and obtain a two channel class logit map. A single 1$\times$1 convolution is used to transform this logit map into a single channel map, which is then concatenated with the local RGB image and binary location mask. The five-channel input is finally fed into the fine module to obtain the local logit map.
For all ResNet models, we use SGD with a learning rate of 0.01, a momentum of 0.9 and weight decay of 0.0005. For MixTransformer models, we use AdamW~\cite{adamw} with a learning rate of 0.0002 and weight decay of 0.0001. Our training follows the ``poly" learning rate schedule with a power of 0.9. During inference, we set both the global image size and local patch size $p$ to 1024. Unless otherwise specified, we set the percentage for local refinement to $1\%$ ($\alpha=0.01$).

%In this section, we provide a set of quantitative results and comparisons between our two-stage model and other methods. We then provide ablation studies, qualitative visualizations and discuss several failure cases. For additional results and details, please refer to our supplementary materials.

%\todo{illurstrate the sampling strategy in training and inference strategy here?}
\vspace{-4mm}
\paragraph{Wire Inpainting Network.}
We adopt LaMa \cite{suvorov2022resolution} for wire inpainting by finetuning on an augmented wire dataset. 
To prepare the wire training set, we randomly crop ten $680\times680$ patches from the non-wire regions of each image in our training partition. In total, we have 50K more training images in addition to 
%\textcolor{purple}{
the 8M
%} 
Places2 \cite{zhou2017places} dataset, and increase its sampling rate %\textcolor{purple}{
by $10\times$
%} 
to balance the dataset. We also use all the ground truth segmentation maps in our training set to sample wire-like masks. During training, we start from Big-LaMa weights, and train the model on $512\times 512$ patches. We also prepare a synthetic wire inpainting quality evaluation dataset, containing 1000 images at $512\times 512$ with synthetic wire masks.
%containing 1000 images of $512 \times 512$ and synthetic wire masks on that. 
While running inference on full-resolution images, we apply a tile-based approach, by fixing the window size at $512\times 512$ with an $32$-pixel overlap. 

\subsection{Wire Segmentation Evaluation}
\paragraph{Quantitative Evaluation}

%\subsection{Globally Conditioned Inference}

We compare with several widely-used object semantic segmentation and high-resolution semantic segmentation models. Specifically, we train DeepLabv3+~\cite{deeplabv3p} with ResNet-50~\cite{resnet} backbone under two settings: global and local. In the global setting, the original images are resized to 1024$\times$1024. In the local setting, we randomly crop 1024$\times$1024 patches from the original images. We train our models on 4 Nvidia V100 GPUs and test them on a single V100 GPU. For high-resolution semantic segmentation models, we compare with CascadePSP~\cite{cascadepsp}, MagNet~\cite{magnet} and ISDNet~\cite{isdnet}. We describe the training details of these works in the supplement.

We present the results of in Table~\ref{table:results} tested on \benchmark. We report wire IoU, F1-score, precision and recall for quantitative evaluation. We also report wire IoUs for images at three scales, small (0 -- 3000$\times$3000), medium (3000$\times$3000 -- 6000$\times$6000) and large (6000$\times$6000+), which are useful for analyzing model characteristics. Finally, we report average, minimum and maximum inference times on \benchmark.

\begin{table}[t!]
\centering
\caption{Voice conversion \& F0 manipulation results. MOS results are reported with 95\% confidence interval. VDE, and FFE are reported for F0 manipulation while PER, WER, EER, and MOS are reported for voice conversion. Notice, for VDE, and FFE higher is the better since F0 was flattened.}
\label{tab:conv}

\resizebox{1\columnwidth}{!}{
\begin{tabular}{c@{~} | c@{~} | c@{~}c@{~} | c@{~} | c@{~} ||  c@{~}c@{~} }
\toprule
\multirow{2}{*}{Dataset} & \multirow{2}{*}{Method} & \multicolumn{4}{c||}{Voice Conversion} & \multicolumn{2}{c}{F0 Manipulation} \\
\cmidrule{3-8}
& & PER~$\downarrow$ & WER~$\downarrow$ & EER~$\downarrow$ & MOS~$\uparrow$ & VDE~$\uparrow$ & FFE~$\uparrow$ \\
\midrule
VCTK & GT  & 17.16 & 4.32 & 3.25 & 4.11$\pm$0.29 & -- & -- \\
\midrule 
\multirow{3}{*}{LJ}
% & ASR-TTS   & 50.74  & --     & 66.08 & 32.96 & 1.46 \\
& CPC       & 22.22 	& 16.11 		& 0.46 		& 3.57$\pm$0.15 		& \bf 46.68 & \bf 48.71\\
& HuBERT    & \bf 19.09 & \bf 12.23 & \bf 0.31  & \bf 3.71$\pm$0.24 & 39.20 		& 48.42\\
& VQ-VAE    & 40.88 	& 36.96 		& 9.65 		& 2.90$\pm$0.17 		& 10.54 	& 12.08 \\
\midrule 
\multirow{3}{*}{VCTK} 
% & ASR-TTS   & 68.88  & --    & 41.77 & 13.55 & 6.48 \\
& CPC       &  23.58 		& 15.98 		& \bf 4.83  &  3.42 $\pm$ 0.24 		& \bf 25.29 & \bf 26.97 \\
& HuBERT    &  \bf 20.85 	& \bf 12.72 & 6.01  		& \bf  3.58 $\pm$ 0.28 	& 23.46 	& 26.67 \\
& VQ-VAE    & 36.88  		& 29.44 		& 11.56 		& 3.08 $\pm$ 0.34 		& 7.03  	& 7.80  \\
\bottomrule
\end{tabular}}
\vspace{-0.4cm}
\end{table}

\vspace{-0.1cm}
\section{Results}
\vspace{-0.1cm}
Our results cover
% We report results for 
three different settings: (i) speech reconstruction experiments; (ii) speaker conversion and F0 manipulation; (iii) bitrate analysis with subjective tests for speech codec evaluation. We employ two datasets: LJ~\cite{ljspeech17} single speaker dataset and VCTK~\cite{vctk} multi-speaker dataset. All datasets were resampled to a 16kHz sample rate.

% \paragraph*{Implementation Details.}
% \smallskip
\noindent{\bf Implementation Details\quad} 
\label{sec:impl}
We follow the same setup as in~\cite{lakhotia2021generative}. For CPC, we used the model from~\cite{Riviere2020}, which was trained on a ``clean'' 6k hour sub-sample of the LibriLight dataset~\cite{Kahn2020,Riviere2020}. We extract a downsampled representation from an intermediate layer with a 256-dimensional embedding and a hop size of 160 audio samples. For HuBERT we used a \textsc{Base} 12 transformer-layer model trained for two iterations~\cite{hsu2020hubert} on 960 hours of LibriSpeech corpus~\cite{Panayotov2015}. 
% This model encodes every 320 raw audio samples into a 768-dimensional vector. 
This model downsamples the raw audio $\times320$ into a sequence of 768-dimensional vectors. Similarly to~\cite{lakhotia2021generative}, activations were extracted from the sixth layer.

%CPC: We use a dictionary of 100 units, leading to a bitrate of 700bps.
%HuBERT: A dictionary of 100 units is used, leading to a bitrate of 350bps. 
%VQVE: The VQ-VAE discrete code operates at a bitrate of 800bps.
% For both CPC and HuBERT, the k-means algorithm is applied to convert continuous frames to discrete codes, using the LibriSpeech clean-100h~\cite{Panayotov2015} dataset. 
For CPC and HuBERT, the k-means algorithm is trained on LibriSpeech clean-100h~\cite{Panayotov2015} dataset to convert continuous frames to discrete codes. We quantize both learned representations with $K=100$ centroids. Leading to a bitrate of 700bps for CPC and 350bps for HuBERT.

% VQ-VAE
Similarly to CPC models, we trained the VQ-VAE content encoder model on the ``clean'' 6K hours subset from the LibriLight dataset. We use an encoder operating on the raw signal to extract discrete units, similar to~\cite{jukebox}. In addition, ``random restarts'' were performed when the mean usage of a codebook vector fell below a predetermined threshold. Finally, we used HiFiGAN (architecture and objective) as the decoder instead of a simple convolutional decoder, as it improved the overall audio quality. This model encodes the raw audio into a sequence of discrete tokens from 256 possible tokens~\cite{garbacea2019low} with a hop size of 160 raw audio samples. The VQ-VAE discrete code operates at a bitrate of 800bps. We additionally experimented with 100 discrete units for VQ-VAE, however results were the best for 256. This finding is consistent with~\cite{garbacea2019low}.

% verification model
The speaker verification network uses the architecture proposed in~\cite{heigold2016end}. It was trained on the VoxCeleb2~\cite{voxceleb2} dataset, achieving a 7.4\% Equal Error Rate (EER) for speaker verification on the test split of the VoxCeleb1~\cite{Nagrani17} dataset.

% pitch
Only a single F0 representation is considered across all evaluated models, trained on the VCTK dataset.
% The F0 is extracted from the raw audio using YAAPT~\cite{yaapt} algorithm, using a window size of 20ms and a 5ms hop. 
The F0 is extracted from the raw audio using a window size of 20ms and a 5ms hop. 
As a result, the F0 sequence is sampled at 200Hz. 
% We apply the quantization described at Sec.~\ref{sec:method}, using a pitch codebook of $K'=20$ tokens and an encoder that downsamples the pitch by $\times16$. 
The quantization described at Sec.~\ref{sec:method}, is applied using an F0 codebook of $K'=20$ tokens and an encoder that downsamples the signal by $\times16$. Hence, the discrete F0 representation is sampled at 12.5Hz, leading to a bitrate of 65bps. The final bitrate of the evaluated codecs is the sum of the pitch code bitrate with the content code bitrate.

% \paragraph*{Evaluation Metrics}
% \smallskip
\noindent{\bf Evaluation Metrics\quad} 
We consider both subjective and objective evaluation metrics. For subjective tests, we report the Mean Opinion Scores (MOS). In which human evaluators rate the naturalness of audio samples on a scale of 1--5. Each experiment, included 50 randomly selected samples rated by 30 raters. For objective evaluation, we consider: (i) Equal Error Rate~(EER) as an automatic speaker verification metric obtained using a pre-trained speaker verification network. We report EER between test utterances and enrolled speakers; (ii) Voicing Decision Error (VDE)~\cite{nakatani2008method}, which measures the portion of frames with voicing decision error; (iii) F0 Frame Error (FFE)~\cite{chu2009reducing}, measures the percentage of frames that contain a deviation of more than 20\% in pitch value or have a voicing decision error; (iv) Word Error Rate (WER) and Phoneme Error Rate (PER), proxy metrics to the intelligibility of the generated audio. We used a pre-trained ASR network~\cite{baevski2020wav2vec} on both reconstructed and converted samples to calculate both metrics. %To generate target phonemes, the g2p-en~\cite{g2pE2019} Grapheme2Phoneme module was used.

% \vspace{-0.1cm}
% \smallskip
\noindent{\bf Reconstruction \& Conversion}
% \vspace{-0.1cm}
We start by reporting the reconstruction performance. Results are summarized in Table~\ref{tab:recon}. When considering the intelligibility of the reconstructed signal HuBERT reaches the lowest PER and WER scores across all models, where both CPC and HuBERT are superior to VQ-VAE. However, when considering F0 reconstruction VQ-VAE outperforms both HuBERT and CPC by a significant margin. This results are somewhat intuitive, bearing in mind VQ-VAE objective is to fully reconstruct the input signal. In terms of subjective evaluation, all models reach similar MOS scores, with one exception of CPC on LJ. 

%Notice, since the same F0 units are used for each method, this result implies the VQ-VAE units contain some information about the F0 of the signal, enabling better reconstruction. Regarding speaker information, the CPC gets the lowest EER. 

To better evaluate the disentanglement properties of each method with respect to speaker identity and F0, we conducted an additional set of experiments aiming at speaker conversion and F0 manipulation. For voice conversion, we converted each test utterance into five random target speakers. Next, we employed a speaker verification network, which extracts \emph{d-vector} representation to evaluate speaker-converted utterances' similarity to real speaker utterances (low error-rate indicates good conversion), providing measurement to the speaker identity's disentanglement from the evaluated coding method. The error-rate is reported between converted test utterances and enrolled speakers. For the LJ speech single speaker dataset, we converted samples from the VCTK dataset to the single speaker and enrolled all VCTK speakers together with the single speaker. Results are summarized in Table~\ref{tab:conv} (left). Unlike resynthesis results, on voice conversion CPC and HuBERT outperform VQ-VAE on both LJ and VCTK datasets, indicating VQ-VAE contains more information about the speaker in the encoded units, hence producing more artifacts. Notice, this also affects WER, PER, and the overall subjective quality (MOS). 

Next, to evaluate the presence of F0 in the discrete units, we flattened the F0 units before synthesizing the signal and calculated VDE and FFE with respect to the original F0 values. F0 flattening was done by setting the speakers' mean F0 value across all voiced frames. In this experiment, we expected units that contain F0 information to be better at F0 reconstruction over disentangled units. Results are summarized in Table~\ref{tab:conv} (right). Notice VQ-VAE can still reconstruct the F0 almost at the same level as when using the original F0 as conditioning (5.2 vs 7.03, and 5.59 vs 7.8), in contrast to CPC and HuBERT.

\begin{figure}[t!]
\centering
\includegraphics[width=0.65\columnwidth, trim={50 20 70 20}]{figures/codec_2.pdf}
% \caption{MUSHRA subjective listening test results as a function of bitrate per second for various methods. Purple dots denote the baseline methods, and green dots the proposed SSL based method.} 
\caption{MUSHRA subjective quality results as a function of bitrate per second. Purple dots denote the baseline methods, and green dots the proposed SSL based method.} 
\label{fig:codec}
\vspace{-0.5cm}
\end{figure}

% \vspace{-0.1cm}
% \smallskip
\noindent{\bf Speech Codec}
Our final experiment evaluates the obtained speech units as a low bitrate speech codec. 
% Therefore, we evaluate how the performance varies as a function of the number of discrete units. Changing the number of units is equivalent to varying the bitrate of the encoded signal. 
We use a subjective MUSHRA-type listening test~\cite{series2014method} to measure the perceived quality of the proposed speech codec with regard to its bitrate constraints. In MUSHRA evaluations, listeners are presented with a labeled uncompressed signal for reference, a set of test samples to rate, a copy of the uncompressed reference, and a low-quality anchor. Listeners are asked to rate each test utterance and the copy of the uncompressed reference with respect to the labeled reference in a scale of 1-100.

The experiment is performed on the VCTK dataset~\cite{vctk}. For evaluation, we used 20 utterances from 5 speakers. The set of speakers in the test data is disjoint with those in the training data. For this experiment, HuBERT models with 50, 100, and 200 units were trained as described in Sec.~\ref{sec:impl}. For comparison, we included other speech codecs in our evaluation: Opus~\cite{valin2012definition} wideband at 9 kbps VBR, Codec2~\cite{rowe2011codec} at 2.4 kbps and LPCNet~\cite{valin2019real} operating at 1.6 kbps. The LPCNet model was trained from scratch on the VCTK dataset following the experimental setup in~\cite{valin2019real}. The VQ-VAE model employs the HiFiGAN decoder trained on the LibriLight dataset to match the amount of data reported in~\cite{garbacea2019low}. We compressed the anchor sample with Speex~\cite{valin2016speex} at 4 kbps as a low anchor. Fig.~\ref{fig:codec} depicts the results. HuBERT with 50 units reaches the best MUSHRA score while its bitrate is only 365bps, which is significantly lower than the baseline methods.

As shown in Table~\ref{table:results}, while the global model runs fast, it has lower wire IoUs. In contrast, the local model produces high-quality predictions, but requires a very long inference time.
Meanwhile, although CascadePSP is a class-agnostic refinement model designed for high-resolution segmentation refinement, it primarily targets common objects and does not generalize to wires. 
% \textcolor{red}{\sout{We thus retrain CascadePSP on our dataset but find their data perturbation does not realistic model coarse wire segmentation, thus cannot effectively conduct refinement.}}
For MagNet, its refinement module only takes in probability maps without image features, thus failing to refine when the input prediction is inaccurate. 
% As a result, the refinement module cannot accurately produce high-resolution wire predictions.
Among these works, ISDNet is relatively effective and efficient at wire prediction. 
% In fact, their inference time is on par with the global network even at high image resolution, while maintaining relatively high wire IoU. 
However, their shallow network design trades off capacity for efficiency, limiting the performance of wire segmentation that is thin and sparse. 
% \textcolor{red}{\sout{For a fair comparison, we tried to replace the shallow branch in ISDNet with a MixTransformer backbone but failed due to GPU memory limitation. We thus only replace their deep network module, which only yields minor improvement, as shown in Table~\ref{table:results}.}}

% \textcolor{purple}{
Compared to the methods above, our model achieves the best trade-off between accuracy and memory consumption. By leveraging the fact that wires are sparse and thin, our pipeline captures both global and local features more efficiently, thus saving a lot of computation while maintaining high segmentation quality.
% }
% There are two main reasons for this poor performance. 
% First, the global model performs inference on downsampled images, which leads to degraded image quality around thin wire-like objects and causes imprecise or disjointed predictions. Second, since prediction maps are upsampled to the original resolution, the final segmentation map may contain artifacts such as aliasing and over-predicted regions.
% \mt{
% First, since whole-image models predict segmentation maps on downsampled images, predictions on extremely small wires may be incomplete or of low quality. Second, whole-image segmentation maps are obtained by bilinearly upsampling the model output which leads to loose predictions.
% }
% These issues are rare in common object semantic segmentation since their maximum downsampling rate is usually no greater than 3$\times$, while the downsampling rate in our task can reach 10$\times$ (e.g. downsampling 10k$\times$10k images to 1024$\times$1024.). As a result, common object semantic segmentation methods fail to maintain their performances in our task.

% This file was created by matplotlib2tikz v0.6.3.
\begin{tikzpicture}

\definecolor{color1}{RGB}{000,125,122} % MPG green
\definecolor{color0}{HTML}{FF9933} %EI orange

\begin{axis}[
xmin=-100.1212309, xmax=2151.4841009,
ymin=2.53430998098285e-09, ymax=4.2233316248538,
ymode=log,
width=\figwidth,
height=\figheight,
tick align=outside,
xlabel={CPU time [$\mathrm{s}$]},
ylabel={relative residual},
x grid style={lightgray!92.026143790849673!black},
y grid style={lightgray!92.026143790849673!black},
extra y ticks=1e-8,
extra y tick style={grid=major, grid style={dashed,black}},
legend style={draw=white!80.0!black},
legend entries={{CG},{def-CG(8,12)}},
legend cell align={left},
mystyle
]
\addplot [semithick, color0,]
table {%
2.338201 1
4.44607500000001 0.803355123226035
6.645453 0.189723157739223
8.847655 0.0509960204551763
10.740944 0.0389494404201885
12.681271 0.0179898688439675
14.469114 0.0157165818767063
16.244612 0.0109148932485257
18.092409 0.00744639345220638
19.873594 0.00565091648891621
21.661486 0.0045672164315901
23.440174 0.00655416754453231
25.218376 0.00441835200120656
27.00421 0.00286988796812528
28.777418 0.00229073118286192
30.554253 0.00184696207454626
32.335071 0.00170948531149521
34.115036 0.00135532406667619
35.917596 0.0011314129954786
37.717639 0.000995099708333701
39.509563 0.00124467117208623
41.308675 0.00127919261125498
43.100188 0.000745617290911589
44.877516 0.000645207200387889
46.65213 0.00125335739014907
48.495569 0.000566671088344948
50.35954 0.000465779260059283
52.157285 0.000472832773089773
53.98484 0.000706055999276449
56.200514 0.00039557438505856
58.268514 0.00218584404908133
60.050247 0.000329904898579162
61.831192 0.000302504584667049
63.608298 0.000282349949251237
65.390862 0.00027375733823991
67.162513 0.000345627901558582
68.938952 0.000234773985971648
70.717576 0.000218933309531953
72.495541 0.000264446005703375
74.268853 0.00102511903397826
76.052934 0.000177183810119042
77.905004 0.000164585352865341
79.68121 0.000156758732823845
81.463985 0.000271408914110484
83.244103 0.000137873326241013
85.020203 0.000209610075810377
86.794582 0.000131164238986208
88.571451 0.000120571009634111
90.35171 0.000472924440607135
92.134302 0.000104111867724803
93.91936 0.000215831811934717
95.689264 0.000108344843979051
97.467556 0.000145406337107936
99.249585 9.83319255693291e-05
101.029253 7.9487119502651e-05
102.804547 6.9054667817296e-05
104.582844 9.89496263218335e-05
106.362049 8.76192149021089e-05
108.198216 6.96416350425278e-05
109.979402 0.000115284771851654
111.758071 8.63971853518058e-05
113.538694 9.66753862630729e-05
115.325372 7.83450597254574e-05
117.11079 5.31825919489748e-05
118.889627 6.20404156580138e-05
120.667321 5.49629834279613e-05
122.442378 0.00030823276302295
124.257998 6.91611168839292e-05
126.088834 4.64409689899364e-05
127.964068 4.1347240542324e-05
129.926714 5.84786482145998e-05
131.717248 3.6371742320493e-05
133.514839 7.81016437735521e-05
135.295211 3.19694953292453e-05
137.10018 4.8206838178297e-05
139.311255 0.000131282086413705
141.456877 2.95912943760967e-05
143.520072 2.62136224738801e-05
145.565082 4.96911563905546e-05
147.893901 4.47914934213392e-05
150.103769 4.24372369989435e-05
152.038478 2.89649614903263e-05
154.135428 2.06097244640404e-05
155.917417 2.04425661632688e-05
157.697338 7.33055032296042e-05
159.47799 5.57994015255639e-05
161.270785 1.89621655930058e-05
163.050111 2.32149615651989e-05
164.832774 1.917168942288e-05
166.639812 2.38945127824719e-05
168.57736 1.85833834033099e-05
170.782275 1.47446788164736e-05
172.70998 3.07085402215687e-05
174.677387 1.83196245982353e-05
176.860804 6.27528159346378e-05
178.868486 1.60151633576816e-05
180.994855 2.76801388864942e-05
183.112804 1.20206223901352e-05
185.320823 3.31946439301624e-05
187.547378 1.10926071890035e-05
189.775237 9.4849723170377e-06
191.980904 8.66702062223968e-06
194.22611 1.33812332422693e-05
196.410421 4.20534792627039e-05
198.380116 9.87248514735162e-06
200.404731 1.07072202690904e-05
202.424446 7.04485740535497e-06
204.383753 1.81801055858743e-05
206.178944 1.2318902778476e-05
208.009403 6.83609948340216e-06
209.887263 9.7815196887844e-06
212.098816 1.26733265782489e-05
214.3046 1.74476735041725e-05
216.339067 6.27326363064533e-06
218.475676 1.08830719167175e-05
220.561385 6.01750198590015e-06
222.741816 1.94412969696978e-05
224.83513 4.58107886856281e-06
226.64807 7.66490811369606e-06
228.488537 4.91966509463524e-06
230.270216 5.65438808258883e-06
232.227768 1.11606189759778e-05
234.325583 3.9513132800811e-06
236.453178 7.98742878527026e-06
238.653277 4.60243296866998e-06
240.857861 5.30994424478289e-06
242.966102 4.2631298318275e-06
244.999716 4.17060567204133e-06
247.040269 4.9521921937178e-06
249.075768 2.81434243472579e-06
251.138078 2.53381900440129e-05
253.353627 3.62877818960533e-06
255.519492 5.79065437344612e-06
257.673829 3.81501327973428e-06
259.498022 4.41125423694519e-06
261.281506 2.98596763496664e-06
263.231347 2.88701774135704e-06
265.30782 6.40870524569588e-06
267.258152 2.3020460420503e-06
269.184366 2.9617821781976e-06
271.585356 1.12170725426271e-05
273.824399 1.92744750815247e-06
276.09324 1.85123981537782e-06
278.388852 3.1652393314562e-06
280.514078 2.34480432564453e-06
282.609874 1.50771212984554e-06
284.606495 1.46728969469757e-06
286.459528 2.5387782910075e-06
288.331303 9.20990293955467e-06
290.361776 1.69090060045012e-06
292.472552 3.28713504324724e-06
294.519925 1.54867660786071e-06
296.552238 2.20460788169525e-06
298.59819 1.28858013741564e-06
300.758958 2.28949956912794e-06
302.977901 1.26440924020322e-06
305.047106 1.53250668648788e-06
307.248305 2.02469489887452e-06
309.277452 5.82783252743368e-06
311.283774 2.30716484539411e-06
313.498701 8.95382050802481e-07
315.445712 1.19027406565537e-06
317.580155 9.50379943558741e-07
319.555588 1.09011919340949e-06
321.419928 6.9483841758977e-07
323.192191 1.16614183972475e-06
324.971577 7.90381684433361e-07
327.041162 3.08178994890944e-06
329.060966 1.03760383795321e-06
330.877475 1.5216072227517e-06
332.664154 1.00821898282087e-06
334.540456 5.68559994423168e-07
336.558855 5.05917187612717e-07
338.481377 5.96003687564043e-07
340.664056 5.86768563194272e-07
342.655193 7.07241889247326e-07
344.515758 2.54940696385307e-06
346.717585 1.16871035189459e-06
348.858481 7.96416102535291e-07
350.798598 4.86781827240055e-07
352.815058 7.32731090048667e-07
354.859185 7.86865457870226e-07
356.910023 3.84759482407189e-07
358.967741 7.09138343674972e-07
360.976073 3.19060013913001e-07
363.195237 6.22596622166789e-07
365.405211 4.87610919073146e-07
367.348338 5.99377354796012e-07
369.322147 4.07436862774794e-07
371.497061 3.4125699307762e-07
373.543511 3.72597211615982e-07
375.441781 3.53086391548906e-07
377.464704 2.71439627694361e-07
379.472276 3.00772598839259e-07
381.659171 6.60802967590278e-07
383.675025 2.22023788220869e-07
385.810649 2.60646008558975e-07
388.00976 2.8467347221045e-07
390.000912 2.06328828309457e-07
391.889738 4.04562759325244e-07
393.691625 1.72173943351701e-07
395.773745 2.9980832344414e-07
397.983738 2.08984014081915e-07
400.075862 4.81362687324553e-07
401.968513 3.19293565350821e-07
403.818335 2.61291157832008e-07
405.91258 2.91008075524331e-07
407.882786 1.26840547569393e-07
409.939741 1.57541858549776e-07
411.772569 2.13413363664401e-07
413.686072 1.30821150905034e-07
415.705307 1.4440711653756e-07
417.70636 2.71571428450793e-07
419.835866 1.7083290185777e-07
421.769995 1.69734922414137e-07
423.824447 1.1361833431033e-07
425.704792 1.17728606387229e-07
427.490536 1.33116961703577e-07
429.417078 1.05139352450153e-07
431.325926 1.39259352406075e-07
433.464613 1.54548931788732e-07
435.51068 1.46477016087427e-07
437.398136 2.09946784222416e-07
439.317214 1.505516081264e-07
441.25917 8.29329684320074e-08
443.23825 9.47698040005921e-08
445.345181 9.34863948626903e-08
447.219825 6.59789709325083e-08
449.021049 1.74877392556874e-07
450.809176 7.35633748656762e-08
452.590264 7.06333602257365e-08
454.370597 4.1141633194515e-07
456.151862 7.38636058797e-08
458.170466 5.538996909896e-08
460.388864 6.04532009544436e-08
462.336406 8.66246006460668e-08
464.304225 4.11137870206209e-08
466.087653 6.12876010497824e-08
467.892891 6.44424248653451e-08
469.710288 8.93149242336997e-08
471.506856 9.8085964971925e-08
473.686097 5.12127589946733e-08
475.851004 9.35977469492779e-08
478.066035 7.49654333598934e-08
480.270371 5.23990680640004e-08
482.234778 5.17801527848953e-08
484.164215 4.25592987973186e-08
486.123476 5.10230952115914e-08
488.253673 3.02851599610119e-08
490.238791 1.97618347654298e-07
492.06488 2.5607939622142e-08
493.843298 5.08623643285511e-08
495.622514 3.22037227176915e-08
497.412571 5.51916478545168e-08
499.244174 3.01661628436876e-08
501.047335 3.0893641001417e-08
502.824304 2.96390250672025e-08
504.603488 3.52550575355969e-08
506.386801 3.75916213525004e-08
508.173135 2.66444763533281e-08
509.951188 1.63352975273386e-08
511.737682 2.35445950198274e-08
513.515971 4.20728014346737e-08
515.291317 2.67069250777757e-08
517.068324 1.45463061165461e-08
518.851024 1.26473636318996e-08
520.627858 4.71535035106676e-08
522.403976 2.40417356677701e-08
524.202576 1.86162202249697e-08
525.994692 2.24707199894313e-08
527.789523 1.25236620686235e-08
529.750549 1.61462618135815e-08
531.684237 2.01524212212717e-08
533.46938 1.1020618434418e-08
535.243347 3.79774087073663e-08
537.020355 1.71280102719761e-08
538.796092 3.22786474090885e-08
540.578704 1.07235377431837e-08
542.363263 1.94078266109902e-08
544.142314 1.27330531704612e-08
545.924869 8.46042523209604e-09
};
\addplot [semithick, color1]
table {%
2.22446599999967 1
4.19218799999999 0.803355123226035
5.98680299999978 0.189723157739223
7.76439500000015 0.0509960204551763
9.54846399999951 0.0389494404201885
11.3233769999997 0.0179898688439675
13.0934209999996 0.0157165818767063
14.8648519999997 0.0109148932485257
16.7672469999998 0.00744639345220638
18.9040519999999 0.00565091648891621
20.7229090000001 0.0045672164315901
22.5221819999997 0.00655416754453231
24.2965999999997 0.00441835200120656
26.0706829999999 0.00286988796812528
27.8376129999997 0.00229073118286192
29.6076379999995 0.00184696207454626
31.3784839999998 0.00170948531149521
33.1586429999998 0.00135532406667619
34.931979 0.0011314129954786
36.7055209999999 0.000995099708333701
38.4824440000002 0.00124467117208623
40.3009350000002 0.00127919261125498
42.4953749999995 0.000745617290911589
44.7041339999996 0.000645207200387889
46.7680979999996 0.00125335739014907
48.8930559999999 0.000566671088344948
51.0705399999997 0.000465779260059283
53.0762919999997 0.000472832773089773
55.2792819999995 0.000706055999276449
57.20543 0.00039557438505856
59.2867630000001 0.00218584404908133
61.4789639999999 0.000329904898579162
63.5860720000001 0.000302504584667049
65.3556900000003 0.000282349949251237
67.1415850000003 0.00027375733823991
68.928664 0.000345627901558582
70.7004710000001 0.000234773985971648
72.4768880000001 0.000218933309531953
74.2511139999997 0.000264446005703375
76.0227669999995 0.00102511903397826
77.7931280000003 0.000177183810119042
79.5560679999999 0.000164585352865341
81.3223429999998 0.000156758732823845
83.095577 0.000271408914110484
84.861868 0.000137873326241013
86.6282590000001 0.000209610075810377
88.5654709999999 0.000131164238986208
90.4364690000002 0.000120571009634111
92.2144859999999 0.000472924440607135
93.9810669999997 0.000104111867724803
95.7490749999997 0.000215831811934717
97.5169310000001 0.000108344843979051
99.2862029999997 0.000145406337107936
101.071564 9.83319255693291e-05
102.842027 7.9487119502651e-05
104.612332 6.9054667817296e-05
106.384138 9.89496263218335e-05
108.153292 8.76192149021089e-05
109.924455 6.96416350425278e-05
111.694002 0.000115284771851654
113.476625 8.63971853518058e-05
115.250521 9.66753862630729e-05
117.018706 7.83450597254574e-05
118.788136 5.31825919489748e-05
120.560193 6.20404156580138e-05
122.336273 5.49629834279613e-05
124.111134 0.00030823276302295
125.880635 6.91611168839292e-05
127.653796 4.64409689899364e-05
129.440205 4.1347240542324e-05
131.214586999999 5.84786482145998e-05
132.986638 3.6371742320493e-05
135.136946 7.81016437735521e-05
137.163728 3.19694953292453e-05
139.142878 4.8206838178297e-05
141.336893 0.000131282086413705
143.541996 2.95912943760967e-05
145.735251 2.62136224738801e-05
147.542289 4.96911563905546e-05
149.310673 4.47914934213392e-05
151.083446 4.24372369989435e-05
152.860701 2.89649614903263e-05
154.62869 2.06097244640404e-05
156.403672 2.04425661632688e-05
158.166777 7.33055032296042e-05
159.940034 5.57994015255639e-05
161.712177 1.89621655930058e-05
163.492911 2.32149615651989e-05
165.269185 1.917168942288e-05
167.042007 2.38945127824719e-05
168.810693 1.85833834033099e-05
170.577724 1.47446788164736e-05
172.345684 3.07085402215687e-05
174.119295 1.83196245982353e-05
175.894581 6.27528159346378e-05
177.666609 1.60151633576816e-05
179.439276 2.76801388864942e-05
181.208162 1.20206223901352e-05
182.989824 3.31946439301624e-05
184.761413 1.10926071890035e-05
186.547602 9.4849723170377e-06
188.333492 8.66702062223968e-06
190.113738 1.33812332422693e-05
191.900723 4.20534792627039e-05
193.688798 9.87248514735162e-06
195.474046 1.07072202690904e-05
197.282048999999 7.04485740535497e-06
199.065946 1.81801055858743e-05
200.878821 1.2318902778476e-05
202.658414 6.83609948340216e-06
204.439488 9.7815196887844e-06
206.221798 1.26733265782489e-05
208.000516999999 1.74476735041725e-05
209.776125 6.27326363064533e-06
211.550275 1.08830719167175e-05
213.318036 6.01750198590015e-06
215.097348 1.94412969696978e-05
216.871671 4.58107886856281e-06
218.645511 7.66490811369606e-06
220.415583 4.91966509463524e-06
222.187818 5.65438808258883e-06
223.968736 1.11606189759778e-05
225.740724 3.9513132800811e-06
227.512031 7.98742878527026e-06
229.294886 4.60243296866998e-06
231.064859999999 5.30994424478289e-06
232.843573 4.2631298318275e-06
234.611613999999 4.17060567204133e-06
236.37876 4.9521921937178e-06
238.153952 2.81434243472579e-06
239.936802 2.53381900440129e-05
241.710253 3.62877818960533e-06
243.496681 5.79065437344612e-06
245.27204 3.81501327973428e-06
247.041607 4.41125423694519e-06
248.811899 2.98596763496664e-06
250.586556 2.88701774135704e-06
252.358173 6.40870524569588e-06
254.133049 2.3020460420503e-06
255.907339 2.9617821781976e-06
257.679241 1.12170725426271e-05
259.449399 1.92744750815247e-06
261.218518 1.85123981537782e-06
262.98727 3.1652393314562e-06
264.758334 2.34480432564453e-06
266.526025 1.50771212984554e-06
268.292112 1.46728969469757e-06
270.056889 2.5387782910075e-06
271.820917 9.20990293955467e-06
273.591663 1.69090060045012e-06
275.380991 3.28713504324724e-06
277.148391 1.54867660786071e-06
278.918891 2.20460788169525e-06
280.688185 1.28858013741564e-06
282.452136 2.28949956912794e-06
284.220461 1.26440924020322e-06
285.992494 1.53250668648788e-06
287.759636 2.02469489887452e-06
289.532626 5.82783252743368e-06
291.299367 2.30716484539411e-06
293.163358 8.95382050802481e-07
294.933479 1.19027406565537e-06
296.699176 9.50379943558741e-07
298.474821 1.09011919340949e-06
300.243727 6.9483841758977e-07
302.013184 1.16614183972475e-06
303.78463 7.90381684433361e-07
305.54939 3.08178994890944e-06
307.319703 1.03760383795321e-06
309.080318 1.5216072227517e-06
310.84569 1.00821898282087e-06
312.615683 5.68559994423168e-07
314.38778 5.05917187612717e-07
316.160316 5.96003687564043e-07
317.932616 5.86768563194272e-07
319.705391 7.07241889247326e-07
321.477002 2.54940696385307e-06
323.249755 1.16871035189459e-06
325.022163 7.96416102535291e-07
326.822678 4.86781827240055e-07
328.845966 7.32731090048667e-07
330.864743 7.86865457870226e-07
332.903497 3.84759482407189e-07
334.955528 7.09138343674972e-07
337.161867 3.19060013913001e-07
339.215036 6.22596622166789e-07
341.152724 4.87610919073146e-07
343.005508 5.99377354796012e-07
345.100635 4.07436862774794e-07
347.013422 3.4125699307762e-07
348.781882 3.72597211615982e-07
350.562654 3.53086391548906e-07
352.368244 2.71439627694361e-07
354.146013 3.00772598839259e-07
355.915945 6.60802967590278e-07
357.684158 2.22023788220869e-07
359.447429 2.60646008558975e-07
361.216865 2.8467347221045e-07
363.019247 2.06328828309457e-07
364.848731 4.04562759325244e-07
366.621995 1.72173943351701e-07
368.391514 2.9980832344414e-07
370.161582 2.08984014081915e-07
371.932469 4.81362687324553e-07
373.711446 3.19293565350821e-07
375.480089 2.61291157832008e-07
377.249441 2.91008075524331e-07
379.015735 1.26840547569393e-07
380.789417 1.57541858549776e-07
382.561863 2.13413363664401e-07
384.333979 1.30821150905034e-07
386.10125 1.4440711653756e-07
387.940037 2.71571428450793e-07
390.010831 1.7083290185777e-07
391.79038 1.69734922414137e-07
393.559244999999 1.1361833431033e-07
395.32804 1.17728606387229e-07
397.099475 1.33116961703577e-07
398.877590999999 1.05139352450153e-07
400.65009 1.39259352406075e-07
402.422142 1.54548931788732e-07
404.19011 1.46477016087427e-07
405.957033 2.09946784222416e-07
407.720349 1.505516081264e-07
409.492905999999 8.29329684320074e-08
411.267612 9.47698040005921e-08
413.043241 9.34863948626903e-08
414.814588 6.59789709325083e-08
416.593885 1.74877392556874e-07
418.366715 7.35633748656762e-08
420.13807 7.06333602257365e-08
421.907922 4.1141633194515e-07
423.680072 7.38636058797e-08
425.443371 5.538996909896e-08
427.215354 6.04532009544436e-08
428.989978 8.66246006460668e-08
431.121993 4.11137870206209e-08
433.074396 6.12876010497824e-08
435.046712 6.44424248653451e-08
436.896206 8.93149242336997e-08
438.671736 9.8085964971925e-08
440.437752 5.12127589946733e-08
442.210488 9.35977469492779e-08
443.989649 7.49654333598934e-08
445.762587 5.23990680640004e-08
447.584348 5.17801527848953e-08
449.604417 4.25592987973186e-08
451.375092 5.10230952115914e-08
453.140997 3.02851599610119e-08
454.915563 1.97618347654298e-07
456.683235 2.5607939622142e-08
458.449288 5.08623643285511e-08
460.222291999999 3.22037227176915e-08
461.989305999999 5.51916478545168e-08
463.757513 3.01661628436876e-08
465.524748 3.0893641001417e-08
467.294398 2.96390250672025e-08
469.064138 3.52550575355969e-08
470.838889 3.75916213525004e-08
472.616002 2.66444763533281e-08
474.390451 1.63352975273386e-08
476.162384 2.35445950198274e-08
477.929805 4.20728014346737e-08
479.699537 2.67069250777757e-08
481.470825 1.45463061165461e-08
483.24925 1.26473636318996e-08
485.02701 4.71535035106676e-08
486.796453 2.40417356677701e-08
488.565196 1.86162202249697e-08
490.335886 2.24707199894313e-08
492.109134 1.25236620686235e-08
493.881522 1.61462618135815e-08
495.653009 2.01524212212717e-08
497.422271 1.1020618434418e-08
499.190762 3.79774087073663e-08
500.965683 1.71280102719761e-08
502.73518 3.22786474090885e-08
504.513565 1.07235377431837e-08
506.28157 1.94078266109902e-08
508.055238 1.27330531704612e-08
509.823391 8.46042523209604e-09
};
\addplot [semithick, color0]
table {%
548.47659 1
550.604753 0.804575147171834
552.563441 0.194574803468408
554.659528 0.0524702710386406
556.697261 0.0403274484736291
558.892463 0.0187561729091768
561.091696 0.0165370912766679
563.164666 0.0115256957522894
565.15932 0.00795507095065917
567.169867 0.00603031544117425
569.29098 0.00481715637993339
571.327516 0.00394061134811415
573.454215 0.014360747699105
575.576079 0.00306217856628012
577.746553 0.0024477155993977
579.924173 0.00194251536143408
582.109603 0.00180798618746539
584.158484 0.00141692253262101
586.344373 0.00118349559957808
588.527284 0.00104840707038964
590.710789 0.000955152306238132
592.885008 0.00208689287483618
595.053558 0.000825049650553405
597.265646 0.0006589620211942
599.308357 0.00130359203122416
601.082953 0.000595655378265468
603.247191 0.000474008293798442
605.250841 0.000480906427342067
607.261268 0.00079653823535531
609.430221 0.000382003569928178
611.587421 0.000329864833929373
613.659875 0.00176894964924077
615.67224 0.000299190516312295
617.795844 0.000267501004799817
619.806948 0.000418143275401713
621.850822 0.000237857500753838
624.061447 0.000211955878682798
625.93069 0.000198898988204841
627.890499 0.000242371597449379
629.930084 0.000162484581424941
631.867678 0.000726734086528887
634.042072 0.000144360744603273
636.221873 0.000124849982940771
638.4118 0.000219541672526653
640.567673 0.000137929860874211
642.497476 0.000184011099998317
644.71775 0.000104456445270027
646.911135 9.42339492780038e-05
649.094938 8.82555076188387e-05
651.317188 0.000332543818754805
653.505771 0.000175431225171645
655.705655 7.94487008276935e-05
657.879182 7.97250665375181e-05
660.054973 9.58632008036806e-05
662.274372 5.74586868542394e-05
664.333927 4.86931230699351e-05
666.545133 4.60342963871433e-05
668.581046 8.24021236889478e-05
670.431855 9.02404516530702e-05
672.198337 5.29976876838894e-05
674.122718 5.54407600850154e-05
676.308048 4.16616369077177e-05
678.176432 7.52268505124214e-05
680.021383 3.58205820419052e-05
681.979756 3.11910350573727e-05
684.026912 3.92315623508584e-05
686.035073 2.71039335298286e-05
688.042125 4.03358175663988e-05
690.053189 0.000122917082338616
692.065288 2.6448684122239e-05
694.067912 3.00782917692962e-05
696.078445 2.32100036450274e-05
698.07786 2.85900385818874e-05
700.081267 1.63625098611934e-05
702.096093 2.14825775764505e-05
704.183685 1.61089455600537e-05
706.361447 1.38496335938258e-05
708.535115 5.6249888493482e-05
710.585747 2.45333425648609e-05
712.337665 1.14114814467705e-05
714.131219 1.91310404911606e-05
716.219171 1.97121785311141e-05
718.264634 1.08959638592439e-05
720.179871 1.03169836644582e-05
722.101241 1.56883638775297e-05
724.080067 8.08185173814642e-06
725.889302 1.05643320110876e-05
727.635559 2.00047109248023e-05
729.374709 8.23406831499031e-06
731.209218 1.02046215485314e-05
733.3525 7.03527235466151e-06
735.349107 5.60683242777821e-06
737.350814 1.34005130877811e-05
739.35364 4.80888563319628e-06
741.356235 7.01408688683218e-06
743.358756 5.44625053754784e-06
745.395997 1.23915630042102e-05
747.402372 6.4769049303526e-06
749.391192 3.60953730554169e-06
751.395154 9.47559185518589e-06
753.520366 3.31689516089873e-06
755.51977 2.82217423055585e-06
757.512973 4.1417020471358e-06
759.507524 3.14917285931108e-06
761.324609 3.05318699727901e-06
763.265182 2.01324648249847e-05
765.288105 2.151782729349e-06
767.299212 5.21320469533925e-06
769.357418 2.5806239523675e-06
771.380413 2.15754963687444e-06
773.377641 2.97621882910623e-06
775.212268 1.77755155551252e-06
777.038487 3.11894827027881e-06
778.775846 2.95501259972015e-06
780.523314 1.54884044494216e-06
782.261258 3.37620993152913e-06
784.012175 1.35514797889014e-06
785.775116 1.58509547404461e-06
787.587409 2.08854768613547e-06
789.535996 1.16499874944654e-06
791.720608 1.39916825150902e-06
793.918407 8.26365012415274e-07
796.11489 1.02097000461651e-06
798.140151 1.51525768317309e-06
800.275473 1.9659208662347e-06
802.292403 1.3861786556825e-06
804.296321 8.38692448202519e-07
806.342942 8.8949215171609e-07
808.296346 9.81951455512862e-07
810.37127 5.77030515331467e-07
812.404324 5.28491049242911e-07
814.247347 1.26110528024692e-06
816.321663 7.55525911643681e-07
818.144741 2.18447625977866e-06
819.891641 9.80138225609487e-07
821.710382 4.49492299999525e-07
823.905139 4.77287994622972e-07
826.07849 5.49762903514828e-07
828.251675 8.46292150464813e-07
830.446565 4.99040717902428e-07
832.623409 3.38197255623425e-07
834.814965 2.60618927489777e-07
836.996945 1.83432171859272e-06
839.174914 6.35886136086282e-07
841.345636 2.54129859594308e-07
843.528212 2.97014176321847e-07
845.712828 2.50440978936578e-07
847.891232 2.90400547931784e-07
850.066809 2.45358484997045e-07
852.250802 2.18022511074055e-07
854.440633 1.80045624553343e-07
856.621618 5.34009250040473e-07
858.77905 2.825629076779e-07
860.858703 3.67858098175075e-07
862.938723 2.29997691876114e-07
864.937349 1.24443476672167e-07
866.900707 4.12849268964643e-07
868.949616 1.72633806500882e-07
870.882403 1.21941397474908e-07
872.781461 2.12532327967004e-07
874.836176 1.29704462042672e-07
876.899542 2.12578762708253e-07
878.932452 1.54720503626113e-07
880.932119 1.50181684008079e-07
882.977788 1.12577556770473e-07
884.792403 9.04763356497316e-08
886.549591 7.65609381332944e-08
888.590863 1.02923977888375e-07
890.779244 8.41964226950324e-08
892.778498 5.41145975360906e-08
894.95869 5.54194375664303e-07
897.130813 5.30467905822276e-08
899.308092 8.21704845192905e-08
901.488313 7.84203278452345e-08
903.656449 6.43895891719471e-08
905.835825 5.22976235310876e-08
907.831899 7.74909720254807e-08
909.995946 3.88364292108061e-08
912.167161 5.27207474930154e-08
914.237783 4.37758133172815e-08
916.408537 6.23803676764154e-08
918.581 4.13944400024831e-08
920.414365 6.66071523362415e-08
922.160339 2.26168806919645e-08
923.908492 4.07823313946301e-08
925.655222 2.98526544570711e-08
927.411947 2.5637902643913e-08
929.1654 2.19290658593133e-08
930.905676 3.1925208186662e-08
932.648685 3.88627167423959e-08
934.584322 2.7149101341402e-08
936.329491 1.49908134793956e-08
938.068394 4.2107343284912e-08
939.805639 2.83900849040112e-08
941.553117 1.63216863919426e-08
943.315289 2.08105443213209e-08
945.055333 2.40842599056665e-08
946.796017 1.55609848413076e-08
948.536992 2.61816966050055e-08
950.278747 1.11829488289751e-08
952.019609 9.17970665249584e-09
};
\addplot [semithick, color1]
table {%
513.593691 1
520.429815 0.0897552067633381
522.361958 0.0241981817251772
524.376379 0.0140607115939351
526.326719 0.00602391122871624
528.199284 0.00408331427437668
529.984365 0.000808146812399748
531.778952 0.00155537747561955
533.562237000001 0.00127766321052858
535.703714000001 0.00071706472859905
537.893067 0.000696868707444039
539.690655 0.000854313528022512
541.47251 0.000543446783628864
543.259685 0.00047469131055112
545.041482 0.000469185696986753
546.816849 0.000378272672897542
548.59543 0.000420219355936952
550.380287 0.000308726644628059
552.285490000001 0.000281369058136524
554.488402 0.000261792957284976
556.360107 0.000227597072448966
558.133589 0.000199679755822227
559.916025 0.000198074733048661
561.724019 0.000155846072429997
563.53959 0.000145689959146884
565.341556 0.000141820296638258
567.13665 0.000122331211446387
569.086765 0.000115931264263827
571.120195 0.000102471437404643
572.902513 0.000118107572153779
574.889148000001 0.000103013276388464
576.883522000001 8.25140638173735e-05
578.933867 7.98367549348567e-05
581.223928 7.8296655715013e-05
583.355695000001 5.94798291262974e-05
585.393037 5.62678607314492e-05
587.407095 7.83912860344326e-05
589.604643 5.22249198803824e-05
591.56895 4.72995040938741e-05
593.683376 5.50386125643815e-05
595.954442000001 4.33693623754731e-05
598.150437 3.90597992496879e-05
600.112269000001 4.31168002199793e-05
601.893185 3.59395668742404e-05
603.668344 3.0593568008885e-05
605.447768 2.51722790449838e-05
607.228842 2.29344280939236e-05
609.00146 2.15044623623177e-05
610.781466 1.80100639273911e-05
612.561891 2.68935393340049e-05
614.346506 1.86206589665474e-05
616.129067 2.89566768755171e-05
617.909697000001 1.57228544860959e-05
619.709626 1.95600711052586e-05
621.490999 1.38702850663627e-05
623.269019 1.22392287571959e-05
625.043507 1.11907818363793e-05
626.822979 9.82061222498215e-06
628.596352 9.10229177514851e-06
630.371559000001 1.03777704124574e-05
632.148293000001 9.49194524278866e-06
633.932875 1.20248894899714e-05
635.708065 7.4277346701762e-06
637.487052 7.12088822973319e-06
639.267972 7.89190413789684e-06
641.048939 5.27414886426533e-06
642.831609 4.84791943807432e-06
644.612071 7.56903042899967e-06
646.388816 4.18236168870855e-06
648.16738 4.98849413070217e-06
649.948852 4.44395294841323e-06
651.724134000001 3.98192252880781e-06
653.51786 4.79470834798883e-06
655.309768000001 4.4846828174337e-06
657.082095 3.17189709787398e-06
658.857757 2.40120195830553e-06
660.653381000001 2.59609605026357e-06
662.432535 2.72248316304636e-06
664.306505 2.06232523436312e-06
666.084568000001 1.59103005980662e-06
667.859941 2.39021719224409e-06
669.636842 1.47240642569083e-06
671.41656 1.71639819407172e-06
673.201571 1.7055619774947e-06
674.978758000001 1.87006245930523e-06
676.754892 1.30289939557844e-06
678.577437 1.62403253886185e-06
680.374558 9.80402248552936e-07
682.156570000001 8.6523124730344e-07
683.935492 1.20855767548888e-06
685.710471 8.66719947685454e-07
687.516516 8.94372216832799e-07
689.312601 8.1732192126493e-07
691.115017000001 9.13640837367273e-07
692.917226 8.99495447912707e-07
695.023646 5.45223059923352e-07
697.036388 4.2815130376724e-07
698.857779 8.58640054267441e-07
700.642387 4.49427404009971e-07
702.430156 4.79750309884542e-07
704.209135000001 3.61600588056346e-07
705.998667 4.39717121578615e-07
707.779609000001 4.10208723874874e-07
709.558689 2.61074594441271e-07
711.344714 2.66800289876789e-07
713.121893 2.6670250682126e-07
714.905499 2.6736831379393e-07
716.683140000001 2.53623677126009e-07
718.464393000001 2.20509852960166e-07
720.242486000001 2.54016287100818e-07
722.020778000001 2.09284054464626e-07
723.795683 1.70250522984431e-07
725.573845 1.81707234575347e-07
727.356604 1.40241425886295e-07
729.174331000001 2.49650235202672e-07
730.955013 1.04975469491967e-07
732.739076 9.91257054173385e-08
734.580177 1.41893010533713e-07
736.592365 1.09369448336331e-07
738.62446 1.06521310610131e-07
740.989736 1.08027941365595e-07
743.180803000001 6.92103857854144e-08
745.339928 5.72674845590428e-08
747.378331 6.79709603550055e-08
749.669954 6.05855440771429e-08
751.704162 5.24939141833735e-08
753.839640000001 4.72284134367944e-08
755.939106 4.6772427232461e-08
757.999406 4.95331791908153e-08
759.938337 4.94771659940246e-08
762.066789 4.26021426236251e-08
764.107793000001 3.68297433523107e-08
766.130635 5.04028096371161e-08
768.161835 3.07372696040459e-08
770.223183000001 3.85694056198679e-08
772.360838 2.39313202871365e-08
774.573022 3.38912251494762e-08
776.647865 3.04436501485057e-08
778.578961000001 2.05422662705129e-08
780.3569 1.47871925503288e-08
782.135026 2.01171191249903e-08
783.90993 1.37990722253255e-08
785.688375 1.91533956650386e-08
787.465502 1.14939754282137e-08
789.243138 1.32231279634868e-08
791.019009 9.41393594701483e-09
};
\addplot [semithick, color0]
table {%
954.252842 1
956.056406 0.807403182734015
957.830998 0.209157114227959
959.562094 0.0574498774819136
961.293164 0.0446522480661267
963.024031 0.021515678867764
964.758472 0.01898376625362
966.484904 0.0135876740716136
968.21915 0.0097633141335379
969.950451 0.00719162288226795
971.678773 0.0056447829620256
973.409405 0.00546213918872563
975.13899 0.00782191004677242
976.868217 0.00354062408711678
978.595327 0.00293077112260834
980.322316 0.00226161916542685
982.07403 0.00211147008791219
983.808548 0.00158923643794072
985.539013 0.00132321276618765
987.268546 0.00122176486723707
988.999697 0.00126752461784391
990.733182 0.00227100326614295
992.463512 0.000865550973060411
994.194713 0.000757840274601954
995.923735 0.00132290242930215
997.66378 0.000650559728886852
999.396855 0.000497834137512665
1001.126026 0.000484482082311342
1002.854396 0.000397790354591021
1004.583735 0.00074561922245561
1006.318012 0.00168431595146445
1008.046016 0.000324518751709368
1009.774064 0.000279859328742144
1011.499864 0.000238440666883214
1013.232084 0.000220165377066704
1014.96103 0.000338852078456284
1016.691578 0.000177489429138718
1018.421969 0.000142270254458325
1020.153875 0.000159501309910695
1021.885226 0.000212061945448712
1023.614805 0.000209648088037511
1025.347773 0.000111691730066137
1027.079123 9.16810210602272e-05
1028.809422 0.000180035915238713
1030.53943 8.83870313585687e-05
1032.269105 0.000130595844544588
1034.000632 7.02311933404634e-05
1035.733258 5.68297645914652e-05
1037.459981 0.000258934230200693
1039.187105 5.04355111432053e-05
1040.916571 4.73762192540598e-05
1042.651316 8.16791906138276e-05
1044.378748 4.05947264633152e-05
1046.105285 5.24617180539135e-05
1047.83513 3.24212464087842e-05
1049.565634 2.4368516382776e-05
1051.29804 3.50133555204801e-05
1053.026507 2.9822027252813e-05
1054.759629 2.07094735804864e-05
1056.490415 3.7457645772989e-05
1058.218881 1.91396065178415e-05
1059.946415 2.13446692831919e-05
1061.674131 2.75260994599881e-05
1063.403061 1.49341099551232e-05
1065.133763 1.0860205766136e-05
1066.862059 1.46389166817763e-05
1068.593969 1.78671602748245e-05
1070.322901 1.11031852053794e-05
1072.052026 9.23627940321391e-06
1073.778427 1.15838426987398e-05
1075.504351 1.16625303453547e-05
1077.231937 7.09566927830926e-06
1078.965969 5.49690866714071e-06
1080.698491 8.41254450610259e-06
1082.425353 1.15476033958252e-05
1084.15628 2.10957994179589e-05
1085.887571 4.65859258687873e-06
1087.613989 4.53046830906909e-06
1089.342405 5.00417946939788e-06
1091.069814 3.70827709781267e-06
1092.800131 4.18350189483589e-06
1094.529111 4.8532798718905e-06
1096.257504 2.91052216683018e-06
1097.984043 2.11123759840904e-06
1099.714875 1.69706825613142e-05
1101.442393 1.93635846353211e-06
1103.170292 2.49724846501887e-06
1104.899348 2.1482938601225e-06
1106.628361 2.07555137796272e-06
1108.357662 1.33382577306514e-06
1110.085068 1.11569496150927e-06
1111.815588 1.78186305929105e-06
1113.547964 1.02950308906753e-06
1115.284992 2.06075102679317e-06
1117.012553 1.14267862893736e-06
1118.739771 9.38344168164203e-07
1120.469751 1.37073445321779e-06
1122.197745 6.25581512992715e-07
1123.927227 9.48062299177348e-07
1125.655068 6.90850374150217e-07
1127.382716 9.43664068923229e-07
1129.113978 4.03508236254046e-07
1130.843514 3.75035282333256e-06
1132.574914 3.39710293232837e-07
1134.300243 5.04279806626755e-07
1136.029686 4.11847057317572e-07
1137.759154 3.99350062824171e-07
1139.488454 2.19606602154228e-07
1141.220981 5.1033114827104e-07
1142.954758 2.37306173851025e-07
1144.688954 3.76463096543722e-07
1146.417568 9.57076822669299e-07
1148.144531 2.69570308643675e-07
1149.874049 1.65084644131239e-07
1151.603592 2.52828564942088e-07
1153.334169 1.18417407378208e-07
1155.063932 1.89798195459004e-07
1156.795551 1.82703903127406e-07
1158.52298 1.19046156343376e-07
1160.253131 2.68669391511545e-07
1161.995767 1.57975850567369e-07
1163.72656 1.45138287498865e-07
1165.452266 9.28719796512181e-08
1167.181696 7.58191210310504e-08
1168.910628 9.79943289968511e-08
1170.640676 5.74798952982427e-08
1172.372418 7.00413077416874e-08
1174.102732 4.70681097360707e-08
1175.829201 7.32093012470285e-08
1177.557825 3.83147668879135e-07
1179.28566 3.72633515029962e-08
1181.01439 5.65965518493804e-08
1182.745321 3.70864168336453e-08
1184.475895 3.33515292710637e-08
1186.209012 4.27205437214068e-08
1187.938526 4.71793225355922e-08
1189.666472 3.856842018558e-08
1191.394174 2.7975251130289e-08
1193.124153 4.96557065601286e-08
1194.853257 2.22926349586089e-08
1196.579864 2.18027178469088e-08
1198.308944 4.37559402946719e-08
1200.041889 1.47637472246613e-08
1201.770579 2.74915914415221e-08
1203.500046 1.62098218694057e-08
1205.23034 1.16568639792656e-08
1206.959924 8.89342010875944e-09
};
\addplot [semithick, color1]
table {%
793.557571 1
800.13767 0.0911427645143914
802.196634 0.0265460284643973
804.242931 0.0173918933167727
806.134463 0.0106697408679668
808.020554 0.00718434276370028
810.218269 0.0059451486867994
812.429697 0.00433426407875372
814.44103 0.00320165343098197
816.376916 0.00236278380739917
818.278823 0.00220587518437407
820.081995 0.00168009747355593
822.187107 0.00129557758420921
824.335999 0.00110850750341525
826.392442 0.000899516553039836
828.426403 0.000798780552315318
830.479868 0.000665987531215606
832.524814 0.000532838217230583
834.562139 0.0005156920881273
836.41557 0.0004500637145942
838.333796 0.000371819768007053
840.388018 0.000339162631367577
842.42963 0.000284859629643476
844.369789 0.000248735298034049
846.586415 0.000236389684194195
848.680458 0.000196114453155886
850.867666 0.000184813127254151
852.899749 0.000163807909102225
854.863096 0.000142435892639105
856.825664 0.000122934130068368
858.943024 0.000116381051480572
860.996721 0.000124709158517965
863.100602 0.000113119160576514
865.116846 0.000102659673570746
867.336542 7.52584160591281e-05
869.249137 7.02924360700946e-05
871.289531 9.75968738366879e-05
873.166319 6.49425179471096e-05
874.941634 5.05939243124861e-05
876.715872 4.30297196663709e-05
878.489589 3.87138524133327e-05
880.261527 3.85377869538616e-05
882.034813 4.7410047074645e-05
883.809265 2.75073774483697e-05
885.579378 2.24484045605451e-05
887.349907 2.10531286132097e-05
889.125424 2.93697223491933e-05
891.050474 1.7362107023584e-05
893.190901 1.48300289899869e-05
894.980768 1.44804775138738e-05
896.770073 1.63985687657753e-05
898.560165 1.05779312175704e-05
900.351593 8.69882987843914e-06
902.150887 8.24829580366975e-06
903.944684 8.20662625990197e-06
905.71562 8.12357072589293e-06
907.489364 7.38463308911791e-06
909.267188 6.6437176821422e-06
911.06608 4.81033387294915e-06
912.836138 4.04421314046696e-06
914.603568 3.38553932464208e-06
916.384413 5.60370681452225e-06
918.159257 3.03374256979834e-06
919.928939 3.63335992710881e-06
921.698976 2.37671192775622e-06
923.470367 2.32211118284288e-06
925.241168 2.54800342613899e-06
927.012569 1.55688579704828e-06
928.791271 1.24851917305657e-06
930.579604 1.089411429342e-06
932.351616 1.19780745028929e-06
934.125523 1.16125314997807e-06
935.893273 7.52912778914789e-07
937.664416 6.68472894720509e-07
939.433895 8.36895725441414e-07
941.235124 4.93597812638444e-07
943.040994 7.58270738226459e-07
944.828032 4.82776327203579e-07
946.600427 5.51510175313411e-07
948.372467 3.65682351411149e-07
950.144463 5.77271729765553e-07
951.973781 2.71195426786616e-07
953.752473 2.5768860423996e-07
955.52223 3.01334364369782e-07
957.306155 1.75072512846909e-07
959.083904 2.19792227628896e-07
960.855595 1.69753392356333e-07
962.631365 2.27710265932426e-07
964.411263 1.4584269290771e-07
966.185419 1.39758028486441e-07
967.957381 1.10000419769023e-07
969.727136 1.02996819657457e-07
971.497093 7.4177566513742e-08
973.267975 9.11045871190199e-08
975.041695 5.98608589855025e-08
976.817889 8.31116232384855e-08
978.583932 4.82981438922929e-08
980.358288 3.9313119457933e-08
982.132489 3.06257332725343e-08
983.900567 5.15425351329012e-08
985.678454 3.15086626360504e-08
987.449922 3.10649939652032e-08
989.230628 2.06028889207575e-08
991.001615 3.13353552281446e-08
992.7737 1.89835659628716e-08
994.549308 1.75218332149065e-08
996.31959 1.63770858570623e-08
998.095802 1.68780845709502e-08
999.882839 1.15569242916659e-08
1001.658698 1.01080054544374e-08
1003.431066 1.13059852481353e-08
1005.204014 9.63252484795002e-09
};
\addplot [semithick, color0]
table {%
1208.805235 1
1210.599267 0.818272315902229
1212.364867 0.242259321513288
1214.096429 0.0710913023207334
1215.826827 0.0556398448146228
1217.560121 0.0293573872881461
1219.291423 0.0244467802092831
1221.030234 0.018876000113289
1222.76633 0.0146852573463301
1224.498175 0.0100442269234059
1226.232733 0.00769706451104584
1227.965849 0.0147300271265303
1229.697161 0.0066261787876248
1231.432284 0.00462561180337677
1233.166055 0.00371654280960053
1234.896576 0.0031330192027257
1236.62781 0.00263286899937473
1238.35978 0.00196412558348577
1240.091343 0.00159251555946919
1241.822085 0.00136424434615748
1243.552951 0.00231762050031698
1245.284398 0.00122468103688415
1247.016679 0.000947809494064252
1248.749816 0.000706626621682135
1250.488116 0.000790674521935505
1252.232636 0.000949542829447962
1253.966364 0.000556709642374778
1255.697341 0.000484319921080416
1257.429234 0.000377723181005276
1259.161354 0.000623244987678761
1260.891935 0.0010554229636783
1262.62218 0.000272202217708809
1264.355646 0.000234619553298099
1266.087816 0.000194464164953167
1267.818116 0.000159516955075347
1269.549193 0.000261338034655031
1271.279625 0.000132677495893211
1273.010685 9.28776262420741e-05
1274.742025 8.72960374667698e-05
1276.473806 0.000355980076217417
1278.207786 0.000110314641985049
1279.945361 6.70357473157585e-05
1281.677766 5.26030935087663e-05
1283.409459 4.6612590571314e-05
1285.138275 6.41647155277276e-05
1286.868134 6.41451134254005e-05
1288.602128 3.66401395363573e-05
1290.335256 2.82704255270533e-05
1292.069597 0.000171431785847554
1293.80237 2.38691572997502e-05
1295.536019 1.7702520771327e-05
1297.265549 3.12936155691308e-05
1298.996015 1.90744630601135e-05
1300.727345 1.27879934823477e-05
1302.460451 9.24395305244678e-06
1304.193401 8.47735722519072e-06
1305.925163 2.39824869043666e-05
1307.657572 7.73857605881764e-06
1309.388381 8.93455016341e-06
1311.12859 5.81913773060934e-06
1312.860257 1.09251957080432e-05
1314.590103 4.43513888357525e-06
1316.324022 5.93268483333126e-06
1318.05986 3.48712072513073e-06
1319.793793 6.53278212473133e-06
1321.525455 2.81284629139959e-06
1323.256953 4.57188042380182e-06
1324.990108 2.86303222681479e-06
1326.723934 2.24290652882191e-06
1328.457352 2.52445594546981e-06
1330.191019 1.78064188233118e-06
1331.922605 1.47538838909502e-06
1333.653328 1.4612330842354e-06
1335.385559 1.18829122215443e-06
1337.117359 2.3301508608057e-06
1338.851614 2.66631313960355e-06
1340.589064 1.37337200020953e-06
1342.319896 6.82829221450436e-07
1344.051321 7.08653441443475e-07
1345.780725 8.16341906620547e-07
1347.51382 6.18384548198649e-07
1349.244433 4.87808863166941e-07
1350.980022 3.27400830837085e-07
1352.712949 1.40613222722748e-06
1354.446524 1.04775348040622e-06
1356.177946 2.74444745464533e-07
1357.913605 2.40215681854498e-07
1359.648842 2.80679847491535e-07
1361.380996 3.9119459747821e-07
1363.113911 1.6877731031517e-07
1364.847487 1.30363645981031e-07
1366.577923 1.06284502519644e-07
1368.31047 2.93965754844562e-07
1370.042615 1.29074461612528e-07
1371.779649 7.77591938405139e-08
1373.511741 2.13230220749015e-07
1375.242996 5.63950486028745e-08
1376.974264 6.69144576779616e-08
1378.706843 6.2682215350718e-08
1380.438608 5.26029015069904e-08
1382.166257 7.73653963252643e-08
1383.896085 4.1348679137388e-08
1385.624843 7.34040638254236e-08
1387.353472 5.45694037316012e-08
1389.085417 2.75133692473097e-08
1390.813716 2.21402557574557e-08
1392.54167 2.38147117772554e-08
1394.269562 1.64391008919924e-08
1396.000311 2.91792117845172e-08
1397.732901 6.40116666148779e-08
1399.460709 1.19096080549985e-08
1401.198659 2.7067697549069e-08
1402.928654 1.00314415199265e-08
1404.658869 1.33535134615926e-08
1406.389994 6.75843686782672e-09
};
\addplot [semithick, color1]
table {%
1008.026384 1
1014.739535 0.10156732384356
1016.591494 0.034333629422772
1018.685083 0.021487107744974
1020.554408 0.0138854234494908
1022.352164 0.00955073799095624
1024.124393 0.00777911114272731
1025.908281 0.00605273401915171
1027.689475 0.00488180646261698
1029.489517 0.00356105219105137
1031.265839 0.0029941091248344
1033.048677 0.00245762483640191
1034.820926 0.00191275880848553
1036.967902 0.00157445887118763
1038.918645 0.00127088380669054
1040.698597 0.00122616446681266
1042.47317 0.000919890690026271
1044.548743 0.000760802437338683
1046.646332 0.000727005094669489
1048.68873 0.000667927679653431
1050.726804 0.000519185496661831
1052.718366 0.000458152528791348
1054.547967 0.000410201370307888
1056.325502 0.000333359942648819
1058.099874 0.000314954159889192
1059.884369 0.00022964971893214
1061.7221 0.000220898959037529
1063.512981 0.000186867176685361
1065.532969 0.000151058030942693
1067.31272 0.000125294983620029
1069.085252 0.000117519028426575
1070.860361 9.30152481163594e-05
1072.651263 0.000118722038892285
1074.522409 7.8521292930063e-05
1076.620049 5.37699796259381e-05
1078.512259 4.55365510355356e-05
1080.694707 3.80995269133743e-05
1082.637067 4.65841952017412e-05
1084.416867 3.23320843077165e-05
1086.190113 2.53200748840635e-05
1087.966385 2.0877762682147e-05
1089.760755 2.14041361689364e-05
1091.566886 1.99249040396012e-05
1093.338929 1.20244797711863e-05
1095.112637 1.07392430590736e-05
1096.912029 1.32194145497855e-05
1099.115307 8.30165360466446e-06
1101.000035 6.44623980666414e-06
1102.941137 5.51847291357396e-06
1105.161721 4.10323806522608e-06
1107.385973 4.3511679118104e-06
1109.338271 3.96337805581642e-06
1111.151733 2.48620031448565e-06
1112.978024 2.2368982930796e-06
1115.174664 2.68876788633142e-06
1117.270347 2.1576943229498e-06
1119.34991 2.12493141145042e-06
1121.33849 1.23668312345997e-06
1123.323516 1.09024037994297e-06
1125.175029 8.14937324216612e-07
1127.002628 6.39699828103158e-07
1128.92552 6.07664316514598e-07
1130.844026 6.45717500065571e-07
1132.617209 4.83414718006493e-07
1134.392666 4.64397473398051e-07
1136.189561 3.2387585852231e-07
1137.989604 3.59119990097014e-07
1139.770151 2.37948425856868e-07
1141.543852 1.72862562392539e-07
1143.491325 1.43998487909675e-07
1145.365677 1.26411093038564e-07
1147.179914 1.62786090936517e-07
1148.992596 8.02907907287171e-08
1151.186248 6.30940909105587e-08
1153.158036 7.01825829657808e-08
1155.028882 5.47233223150631e-08
1157.000604 6.53818319404438e-08
1158.782543 3.89237122835261e-08
1160.770756 5.74004655882073e-08
1162.599959 3.09746006127651e-08
1164.397871 2.8447839205526e-08
1166.194125 1.74408932904179e-08
1167.987887 1.88443915062669e-08
1170.070453 1.43130021924602e-08
1172.229325 1.04925205360079e-08
1174.233667 1.17765795673953e-08
1176.221598 7.80492179169907e-09
};
\addplot [semithick, color0]
table {%
1408.471484 1
1410.525314 0.882677761201002
1412.76457 0.309670955099524
1414.940532 0.10576572804898
1417.12426 0.0841112047270652
1419.316068 0.0502922147108168
1421.381363 0.0403269882840885
1423.120328 0.0315104234797059
1424.85213 0.027613822219445
1426.586204 0.0171940460380396
1428.320403 0.0125649054617688
1430.053485 0.0342342385468219
1431.787783 0.0107315233431743
1433.528906 0.00761682477566453
1435.260607 0.00566665432208058
1436.992684 0.00489004668308005
1438.733467 0.00352785854055746
1440.498089 0.00282697958975048
1442.254582 0.00235946084946117
1444.381321 0.00182507379066131
1446.28954 0.00208056048933845
1448.462228 0.00211087048621502
1450.62845 0.00124616187543843
1452.800526 0.000836458243197229
1454.969365 0.000724884148310258
1457.14051 0.000694596454165217
1459.155862 0.000615514181703636
1461.326489 0.000395194927668283
1463.196686 0.0004208377585385
1464.929456 0.000395598198124697
1466.663921 0.000918697799807139
1468.398112 0.000278425181621581
1470.132473 0.000209302300691924
1471.86441 0.000162037682858397
1473.596435 0.000172972758771172
1475.341813 0.000172795670286419
1477.074182 0.000106841712176908
1478.806106 7.17351870660708e-05
1480.540289 6.996355438425e-05
1482.273229 0.000169351233257836
1484.007309 9.26325676648504e-05
1485.740006 4.12270510874369e-05
1487.472585 3.57924453080998e-05
1489.203017 2.7372889238935e-05
1490.933779 2.12147511537555e-05
1492.667396 1.87129615904653e-05
1494.402713 2.49156838541279e-05
1496.569641 1.26337037957049e-05
1498.739859 6.9573822985203e-05
1500.909401 1.26408242352784e-05
1503.081451 1.17676306081066e-05
1505.258774 7.76051210680125e-06
1507.418268 7.03555586866665e-06
1509.592748 7.05708037656616e-06
1511.34872 4.01992387792646e-06
1513.08271 3.25292429229784e-06
1514.816428 2.89266193791756e-06
1516.549863 7.84981550740253e-06
1518.280908 4.47804441175053e-06
1520.458121 1.97169134247244e-06
1522.617644 2.07539211294119e-06
1524.792819 1.50822500545013e-06
1526.955486 1.045288234975e-06
1529.124467 1.43537780006783e-06
1531.285463 8.02656255925569e-07
1533.459457 1.14550723615524e-06
1535.638873 1.07220168230004e-06
1537.522257 8.18439245667697e-07
1539.254492 1.2441629989285e-06
1540.986792 6.8774926389615e-07
1542.720383 4.19806217553207e-07
1544.454752 3.36708429312322e-07
1546.187276 3.7560073000031e-07
1547.9211 2.2196092391392e-07
1549.654423 4.34596119915634e-07
1551.387388 1.71274263888174e-07
1553.125294 1.13821555298461e-06
1554.85788 1.555080215204e-07
1556.590172 1.39607463487244e-07
1558.320689 8.99884722858898e-08
1560.05309 6.19499067838696e-08
1561.78439 9.31021016056566e-08
1563.518012 6.01216944570034e-08
1565.250678 6.53646779220535e-08
1566.982449 3.69775889570879e-08
1568.714828 9.62830557047104e-08
1570.446952 5.18306051100901e-08
1572.178525 2.90717947577393e-08
1573.910763 4.54180198390428e-08
1575.642456 3.15081233426165e-08
1577.373373 1.83680779528913e-08
1579.10786 1.25065458903252e-08
1580.840944 1.26269733196054e-08
1582.57256 1.10461286522775e-08
1584.312023 2.85007190283326e-08
1586.044742 1.75190956363716e-08
1587.77907 8.70914992295231e-09
};
\addplot [semithick, color1]
table {%
1178.841828 1
1185.119341 0.13520718401907
1186.937365 0.0543183634497806
1188.909694 0.0329497808419938
1190.961682 0.0234263668176792
1193.099187 0.0169864028726687
1195.156529 0.0139925486597589
1197.189612 0.0108878574031763
1199.255085 0.00835649750440487
1201.28406 0.00599055969610091
1203.178066 0.00516434095737581
1204.953956 0.00398818462354353
1206.735219 0.00334485994762383
1208.513017 0.00268269421467095
1210.296356 0.002213496586031
1212.068388 0.00217778682619692
1213.837311 0.00147603175232371
1215.610468 0.00140803123645132
1217.382912 0.00111901873216995
1219.154319 0.00100821084068454
1220.928202 0.000789506322812134
1222.696997 0.000628416467606209
1224.48652 0.000496562012986069
1226.253883 0.000416613206417103
1228.025905 0.000326130903870491
1229.974396 0.000240681733847384
1231.797897 0.00019939158291641
1233.568639 0.000153429129976108
1235.339356 0.000125459126038543
1237.137466 0.000107667269272456
1238.910248 8.08197106722744e-05
1240.685926 6.05792561752281e-05
1242.457218 4.62158857679012e-05
1244.227439 5.00116746900274e-05
1246.002863 4.08836028273995e-05
1247.772727 2.86250313377046e-05
1249.543954 2.29849390035916e-05
1251.322226 2.38844045026314e-05
1253.094297 1.61717841866269e-05
1254.864975 1.28511216688041e-05
1256.638189 9.2082271165195e-06
1258.413594 7.46076994452548e-06
1260.188039 8.48595947701931e-06
1261.960844 5.3869116265434e-06
1263.735948 4.25051634464303e-06
1265.523114 3.16872208231232e-06
1267.334596 2.56724956173524e-06
1269.123327 2.92011637242875e-06
1270.92699 1.85091885574955e-06
1272.776528 1.34180340620055e-06
1274.559252 1.04021183408033e-06
1276.50411 8.88284719387982e-07
1278.590479 9.00360980531013e-07
1280.806595 5.72383741803389e-07
1283.016483 4.33376277687863e-07
1284.988005 3.65196509836731e-07
1287.054042 3.49028192103689e-07
1288.941353 4.06189863793683e-07
1290.720339 2.16513536739336e-07
1292.520548 1.62231314001954e-07
1294.294527 1.34268313240392e-07
1296.069142 1.28666785902168e-07
1297.849235 8.43227511086363e-08
1299.625488 6.65527680240578e-08
1301.400341 7.61233551018288e-08
1303.174367 6.61199555504632e-08
1304.950902 4.29516124953038e-08
1306.726496 2.72205754419616e-08
1308.710397 2.05953997160241e-08
1310.926988 1.63769786953586e-08
1313.132204 1.55773465574139e-08
1315.35098 1.19390572108686e-08
1317.588887 9.864500766699e-09
};
\addplot [semithick, color0]
table {%
1590.072993 1
1591.996014 1.25892340553418
1593.743862 0.442958859796317
1595.473873 0.185694102027004
1597.210363 0.151451458936597
1598.945029 0.110107164027802
1600.678151 0.091285960116365
1602.410667 0.0668819947984154
1604.145524 0.0628093350737482
1605.878837 0.0382698521275427
1607.614893 0.0284050184408119
1609.349519 0.033009727833984
1611.082547 0.0296224079852868
1612.816953 0.0153204175148354
1614.551214 0.0117143666948396
1616.287004 0.00918396629108533
1618.043492 0.00728446409947007
1619.780719 0.00461891472592516
1621.527364 0.00359158743000327
1623.262503 0.00274889589344499
1624.997354 0.00234963525990905
1626.731281 0.00498093613847313
1628.464912 0.00168576115129762
1630.199232 0.00123569166507798
1631.93316 0.0010187562771903
1633.682345 0.000807146533896264
1635.414763 0.000961809132624422
1637.148322 0.000717089619685102
1638.883645 0.00054770179952566
1640.619241 0.000612130759018172
1642.354858 0.000682541181725462
1644.090482 0.000471364143750875
1645.824617 0.000228299866220261
1647.5583 0.000187122410793583
1649.291422 0.000170318550035927
1651.028993 0.000232088725388518
1652.764687 0.000115406604247722
1654.499762 8.18384851495256e-05
1656.235887 6.84440300056229e-05
1657.970962 9.30080873146337e-05
1659.703651 0.000186988534823705
1661.43703 4.29866688706606e-05
1663.172153 3.0907492083339e-05
1664.9097 2.58953729816432e-05
1666.6441 1.8045045124242e-05
1668.377166 1.38973259073298e-05
1670.116167 1.51074310372519e-05
1671.850153 1.36125838215815e-05
1673.586075 1.30768739075335e-05
1675.321337 2.06918168410443e-05
1677.055453 8.38913499861498e-06
1678.786231 1.28357867859607e-05
1680.522656 4.86294924314107e-06
1682.257798 4.08087258479645e-06
1683.99204 2.96812707794602e-06
1685.724492 2.70870310039537e-06
1687.457157 2.34157594631056e-06
1689.191914 1.60298452760894e-06
1690.925322 8.62164357617996e-06
1692.656514 2.69109454005255e-06
1694.3901 1.10412322931446e-06
1696.12125 8.01960121460129e-07
1697.851742 6.31862840211085e-07
1699.58141 7.31915168134056e-07
1701.316647 3.9440803740715e-07
1703.050034 3.6731612072615e-07
1704.78348 4.94461537583423e-07
1706.518148 9.78266498181808e-07
1708.252351 2.33359068066861e-07
1709.98883 4.94624990325069e-07
1711.72497 1.59385161322244e-07
1713.459568 2.6601747708775e-07
1715.192126 1.87988661888046e-07
1716.924748 1.12245659277174e-07
1718.657787 7.98322771954845e-08
1720.391543 9.81216548199836e-08
1722.125686 4.0279891026414e-07
1723.85745 5.51054433793084e-08
1725.590938 3.74640692593947e-08
1727.322162 3.75950902710955e-08
1729.053481 2.94659355287259e-08
1730.787947 1.80174587536791e-08
1732.522982 3.76465844157703e-08
1734.256435 1.76188211405712e-08
1735.99344 1.87362006990659e-08
1737.752661 1.67487427214135e-08
1739.505631 1.70493400741162e-08
1741.252861 7.23071475363706e-09
};
\addplot [semithick, color1]
table {%
1320.172017 1
1326.824743 0.248820955880855
1328.898003 0.110938529457055
1330.981807 0.0680584079472129
1332.964065 0.0530319842512949
1335.035651 0.0432812590168041
1337.024247 0.0290319309620613
1339.100885 0.0215095709670067
1341.321199 0.0165463544529106
1343.264962 0.0116356076744143
1345.137517 0.0103846620281675
1346.908663 0.00733379388851304
1348.689059 0.0061812640986991
1350.461473 0.00503817009499729
1352.23187 0.00447537959462417
1354.000812 0.00403918739648224
1355.777635 0.00261750702591131
1357.549266 0.00244847962430531
1359.319457 0.001807193859749
1361.086574 0.00156688199664357
1362.851893 0.00117020423617186
1364.626302 0.00080464978310612
1366.397237 0.000588750070398256
1368.167652 0.000529442505587108
1369.960278 0.000390064757151127
1371.729834 0.000278719310274855
1373.494386 0.000221103058918435
1375.268717 0.000167335997845522
1377.042393 0.000128817456067335
1378.817544 9.85794604413643e-05
1380.584571 7.32331435181233e-05
1382.35231 5.86309366721891e-05
1384.130212 6.27102454367108e-05
1385.899304 4.92368810137248e-05
1387.671045 3.03859760972182e-05
1389.440414 2.55207676238092e-05
1391.208498 1.66665476925435e-05
1392.977207 1.35957861821071e-05
1394.744511 9.36607641539168e-06
1396.508564 1.21060531480456e-05
1398.278595 6.98556633115141e-06
1400.052354 4.91233816940033e-06
1401.820311 3.88509637596899e-06
1403.588055 2.90601404228149e-06
1405.357952 3.32252067601299e-06
1407.123394 1.97786965152602e-06
1408.891356 1.38612881252431e-06
1410.680053 1.59783277047001e-06
1412.454561 9.83046042466924e-07
1414.218987 6.59162320027348e-07
1415.991977 6.22255018953725e-07
1417.760024 5.70648123453716e-07
1419.529344 3.72820586568715e-07
1421.290176 2.58925618781072e-07
1423.062691 1.77268371316047e-07
1424.837578 1.99827673818756e-07
1426.608988 1.53273920435383e-07
1428.377263 1.23145007188581e-07
1430.159132 9.50118086640836e-08
1431.958279 6.35138877843641e-08
1433.728565 6.97286083710095e-08
1435.499225 3.94692042075118e-08
1437.265621 3.08024622860708e-08
1439.038878 3.07869168197978e-08
1440.808669 1.76577180060294e-08
1442.574609 1.19313316220521e-08
1444.348112 8.19325548405378e-09
};
\addplot [semithick, color0]
table {%
1743.081169 1
1744.837177 1.6087281600166
1746.569698 0.612555880206803
1748.30407 0.299152102915402
1750.035878 0.22578444367571
1751.772563 0.20018263128823
1753.504789 0.197681977754026
1755.234285 0.142765506974438
1756.965293 0.124288691845791
1758.697984 0.0792975588056127
1760.429985 0.0569366242232655
1762.161353 0.0403555901346313
1763.893457 0.169941208598842
1765.624901 0.0277019738404317
1767.356214 0.0207037950721268
1769.089007 0.0153863239446574
1770.827023 0.01232196007754
1772.57467 0.00759756077679174
1774.304391 0.00605331258245564
1776.036406 0.00473727048218437
1777.771687 0.0036466881102338
1779.503999 0.0144757330779211
1781.237009 0.00271976864732875
1782.970098 0.00201117440806909
1784.701284 0.0014362629351664
1786.439858 0.00117220300212632
1788.173764 0.00118742340991963
1789.905262 0.00112767108830797
1791.636974 0.000843674848908517
1793.370425 0.000753957886684349
1795.105767 0.00333452281850322
1796.838609 0.000424798653575849
1798.569465 0.000383475486763922
1800.303753 0.000245431838664922
1802.038162 0.000205782629008761
1803.772779 0.000293485858723227
1805.505167 0.000154889851826305
1807.237562 0.00011113088340707
1808.972213 0.000130098810389568
1810.704808 0.000134255897430585
1812.439238 0.000114491063082427
1814.170661 5.92295296251871e-05
1815.89989 3.69088063694307e-05
1817.633735 3.38979856246879e-05
1819.364751 2.11799976931397e-05
1821.095563 1.60173824157012e-05
1822.827076 2.54866034409014e-05
1824.558366 1.77367131290851e-05
1826.290908 1.41656668462055e-05
1828.020872 2.07643033991534e-05
1829.752398 8.69657910757747e-06
1831.488751 1.62855602744456e-05
1833.22151 5.24287492846242e-06
1834.951268 3.99701014946331e-06
1836.680444 3.3162210231057e-06
1838.413004 4.58477845992848e-06
1840.143487 4.06028282602606e-06
1841.874651 2.22043435992531e-06
1843.605665 1.51957930207912e-06
1845.336458 1.39316571883172e-06
1847.069223 1.55621092681295e-06
1848.800348 1.04847372456203e-06
1850.531879 8.63736940517262e-07
1852.263831 5.03621528054854e-07
1853.993922 3.7412409988474e-06
1855.724394 3.35328587008411e-07
1857.456996 6.02706331113166e-07
1859.188915 2.58143501846866e-07
1860.931455 2.25158311744436e-07
1862.66886 2.17672344134753e-07
1864.40331 2.89419910988686e-07
1866.137952 1.27161749539068e-07
1867.869257 1.76259821333223e-07
1869.598891 3.07405572612669e-07
1871.329954 9.04060318348974e-08
1873.089656 8.40568981406434e-08
1874.820289 6.53056491989719e-08
1876.552431 3.56952539713205e-08
1878.2839 2.75971298851568e-08
1880.014758 2.02004631003805e-08
1881.744379 1.56038499490164e-08
1883.47592 6.70587097891658e-08
1885.207601 1.82194004176855e-08
1886.941591 1.04182971392006e-08
1888.673493 1.25039199229798e-08
1890.406906 1.23938650793195e-08
1892.143682 6.65322566974713e-09
};
\addplot [semithick, color1]
table {%
1446.742298 1
1453.942018 0.462463990035682
1456.024538 0.220454760080254
1458.156198 0.143243260718361
1460.102175 0.11374811977842
1462.305647 0.0885366158767486
1464.209153 0.0576879343029282
1466.182507 0.0386524078028073
1468.159069 0.0290530859483476
1470.380217 0.0207234434443234
1472.29642 0.017687825704345
1474.072631 0.0125626346138141
1475.843736 0.0107730687095972
1477.622151 0.00860285057191416
1479.408135 0.00686192462637046
1481.187231 0.00500734795190893
1482.96076 0.00372117130603528
1484.747816 0.00293474986743842
1486.515929 0.00288232682526603
1488.290221 0.00179098897016448
1490.064122 0.00146874469030106
1491.836347 0.00132455250346479
1493.606052 0.000859943607649922
1495.382367 0.000664281431034654
1497.151724 0.000544940222915885
1498.93018 0.000351859080644449
1500.7151 0.000295217641772103
1502.487155 0.000218394908454212
1504.262271 0.000150733893476565
1506.036649 0.000113443121893792
1507.815119 8.77812300387742e-05
1509.587957 8.0817571941136e-05
1511.358618 6.84928731478125e-05
1513.127968 4.60042641426994e-05
1514.903094 3.40307701289556e-05
1516.671278 2.2977819028737e-05
1518.45845 1.69775235891245e-05
1520.336821 1.32872536731276e-05
1522.53478 8.7447279774397e-06
1524.597956 6.8673291979135e-06
1526.568557 6.90311790188025e-06
1528.651328 4.69654276875943e-06
1530.859438 3.25579616969358e-06
1532.928032 2.31678357237145e-06
1535.021159 2.50225635255009e-06
1536.822597 1.66410519234052e-06
1538.599561 1.17045100333878e-06
1540.37169 8.49354302047422e-07
1542.149035 6.12857934367282e-07
1543.923435 4.68834861809905e-07
1545.697113 4.67865506071294e-07
1547.472964 3.17112820986606e-07
1549.24114 2.28135198557699e-07
1551.017161 2.72115425801557e-07
1552.791613 1.44295595848111e-07
1554.559443 1.5720567023961e-07
1556.32982 8.95871393903555e-08
1558.098082 6.83756160352466e-08
1559.870402 7.75648895535855e-08
1561.688677 3.8153441630022e-08
1563.45981 3.46663322001941e-08
1565.226822 2.98051227293421e-08
1566.99629 1.67068694587224e-08
1568.765712 1.73865701083947e-08
1570.537633 1.08279317373893e-08
1572.307504 7.69155557707964e-09
};
\addplot [semithick, color0]
table {%
1893.95521 1
1895.744726 0.978578487844548
1897.483758 0.452630563012444
1899.212295 0.274506903191449
1900.940334 0.230636837425874
1902.665726 0.205634860470716
1904.393607 0.194905875311664
1906.125145 0.125777342856689
1907.852947 0.110853460026038
1909.580229 0.0751821745110982
1911.309202 0.0523902427312033
1913.039182 0.0534836584554684
1914.767786 0.0468410307120514
1916.492335 0.0250704821030212
1918.218525 0.0179723169546249
1919.94537 0.0126781550712591
1921.687047 0.0103906532419307
1923.424296 0.00647661407922633
1925.602025 0.00515565497167974
1927.777058 0.00419943050401319
1929.937437 0.00528616058830989
1932.10021 0.00356924231719373
1934.269856 0.0021050195767773
1936.436515 0.00168785945796043
1938.611329 0.00117694991622965
1940.79023 0.00114137418621747
1942.963462 0.0012595414942975
1945.133707 0.000701007980341736
1947.306059 0.000510373561962083
1949.470402 0.000908121008188929
1951.645411 0.00178312939009155
1953.815741 0.000332659735507866
1955.985272 0.000242016394424124
1958.158576 0.000199956950225066
1960.330906 0.000169630048693226
1962.506606 0.0002105572371367
1964.68185 0.000122838759098158
1966.605568 7.79197040421013e-05
1968.342036 8.79692660001444e-05
1970.074556 0.000132502315850497
1971.805332 7.25339475016501e-05
1973.535714 5.43068058497639e-05
1975.266802 3.14615270801098e-05
1976.996362 2.70563742576924e-05
1978.727238 1.66898627284937e-05
1980.456596 1.64722788963308e-05
1982.189199 1.64316168148198e-05
1984.362486 8.50499946109101e-06
1986.527689 2.57065762104438e-05
1988.701118 8.72205463337251e-06
1990.873575 8.16807893516338e-06
1993.045228 8.4258445422712e-06
1995.214866 4.29879752529914e-06
1997.375534 2.93090947467014e-06
1999.351499 2.34502604490608e-06
2001.515745 1.80445393824927e-06
2003.697579 1.99123899973758e-06
2005.876048 2.22919136150953e-06
2008.047288 1.28194182769298e-06
2010.211511 7.80871393310539e-07
2012.31591 1.30118071933961e-06
2014.468429 7.84109815477666e-07
2016.619627 5.55260509393333e-07
2018.776062 3.55224195546022e-07
2020.707765 2.8117081372453e-07
2022.552615 3.99012994904481e-07
2024.686814 3.85357106086966e-07
2026.565159 2.23205152649638e-07
2028.296411 3.30299610360842e-07
2030.025155 1.22597770804339e-07
2031.750427 9.99988249960541e-08
2033.482027 1.04661614490551e-07
2035.211237 1.42071353790736e-07
2036.973279 6.11181144876138e-08
2038.701056 4.17217084338014e-08
2040.427175 4.52974888977384e-08
2042.156508 3.53898800085139e-08
2043.885484 3.82022082772067e-08
2045.679575 1.9922256886248e-08
2047.408205 1.39429825929878e-08
2049.138404 9.38682032132173e-09
};
\addplot [semithick, color1]
table {%
1575.095645 1
1581.582654 0.415138940640922
1583.599673 0.201786927513696
1585.697131 0.153465543069231
1587.799058 0.138416196785426
1589.937933 0.0948650869994917
1592.057908 0.0588001068933884
1593.921823 0.0409193790337813
1595.712404 0.0270938656936288
1597.489363 0.0184627650968213
1599.268831 0.0141737862276452
1601.044916 0.00970778588435321
1602.822931 0.00804460483706191
1604.601172 0.00654858093199894
1606.381415 0.00535640605312761
1608.179395 0.00389457406828932
1610.120965 0.00305031768163603
1612.303671 0.00272901347237773
1614.510266 0.00192725340983415
1616.492969 0.00133868271419171
1618.542896 0.00122108529963149
1620.579514 0.000883020505791603
1622.4595 0.000624586217139113
1624.240743 0.000481627268158425
1626.019575 0.000357234466519351
1627.799685 0.000256912520271691
1629.578433 0.000204857796809637
1631.368002 0.000138094002682542
1633.146728 0.000100140907137282
1634.9298 7.58271608846032e-05
1636.711288 5.99594594544956e-05
1638.488559 4.18097269430846e-05
1640.267339 5.41397883831579e-05
1642.042139 3.04663280621217e-05
1643.823679 2.13390731639165e-05
1645.602281 1.39255969181318e-05
1647.380922 9.92869882951228e-06
1649.160344 7.23231268488847e-06
1650.943968 5.18832537139365e-06
1652.776797 5.91038234848873e-06
1654.618954 3.92958359441134e-06
1656.417104 2.5189604902865e-06
1658.200082 1.70733758399763e-06
1659.98373 1.31287972177135e-06
1662.112682 1.1205546024035e-06
1664.24488 1.038567921471e-06
1666.05977 6.56269946016138e-07
1668.216229 4.55284074944354e-07
1670.42154 3.03739644204172e-07
1672.436426 2.23766353002694e-07
1674.337045 2.32691913907532e-07
1676.128125 1.58045659834813e-07
1677.908011 1.03401052527595e-07
1679.684536 1.08555934737461e-07
1681.464106 6.72858130184609e-08
1683.242564 8.41847317733126e-08
1685.028091 4.19631870484859e-08
1686.99174 4.32045163999037e-08
1689.06768 3.03694836805597e-08
1691.295143 2.01252993604734e-08
1693.321207 1.77806790591312e-08
1695.308185 1.48806038779074e-08
1697.134838 8.8254013463881e-09
};
%\addplot [semithick, black, dashed]
%table {%
%0 1e-08
%1697.134838 1e-08
%};
\end{axis}

\end{tikzpicture}

% Meanwhile, local models yield higher IoUs, but have significantly longer inference times. This is because in order to preserve resolutions of very large images, many iterations of sliding-window have to be performed (e.g. 25 iterations are required for an image with size 5120$\times$5120 and a window size of 1024$\times$1024), which leads to significantly slower inference speeds.

% In contrast, our two-stage model achieves better results than sliding-window models while only requiring less than half the inference time. The coarse module in our two-stage model determines potential wire regions for refinement, and skips the refinement step on regions with no wires. This saves inference time and increases the overall inference speed. In regions where there are wires, the fine module leverages information from the coarse module to more accurately predict a tight segmentation mask. These two factors together yield an effective and efficient model for wire semantic segmentation on high-resolution images.
\vspace{-4.5mm}
% \subsubsection{Comparing with SOTA}
\paragraph{Qualitative Evaluation}
%\todo{section subject to change}\\
We provide visual comparisons of segmentation models in Figure~\ref{fig:visual}. We show the ``local'' DeepLabv3+ model as it consistently outperforms its ``global'' variant given that ``local'' predicts wire masks in a sliding-window manner at the original image resolution. As a trade-off, without global context, the model suffers from over-prediction. CascadePSP is designed to refine common object masks given a coarse input mask, thus fails to produce satisfactory results when the input is inaccurate or incomplete. Similarly, the refinement module of MagNet does not handle inaccurate wire predictions. ISDNet performs the best among related methods, but the quality is still unsatisfactory as it uses a lightweight model with limited capacity. Compared to all these methods, our model captures both global context and local details, thus producing more accurate mask predictions.
\vspace{-4mm}
\paragraph{Ablation Studies}

% \subsubsection{Effectiveness of global logit map}

% We show that using the global logit map as input to the fine module conveys more contextual information effectively than using the local logit map. For comparison, we train a separate two-stage model, where we only crop and resize the logit map at the location of the local image patch as input to the fine module.

% %auto-ignore
\begin{table}[h!]
\centering
\resizebox{0.98\linewidth}{!}{
    \renewcommand{\arraystretch}{1.1}
    \begin{tabular}{p{0.27\linewidth}|c|ccc}
    \hline
    Model & IoU (\%) & F1 (\%) & Precision (\%) & Recall (\%)\\ \hline\hline
    Local logit only & 71.3 & 83.2 & 84.5 & \textbf{81.9} \\ \hline
    Global logit + binary location map (ours) & \textbf{71.9} & \textbf{83.6} & \textbf{86.2} & 81.3\\ \hline
    
\end{tabular}
}
\vspace{-1mm}
\caption{Comparison between using only local logit map and global logit map as fine module input. By including the logit map of the entire image, our model avoids over-prediction. }
\label{table:logit}
\vspace{-2mm}
\end{table}

% As shown in Table~\ref{table:logit}, using only the local logit map yields inferior performances. Specifically, we find that in situations where local refinement is taken place, the global logit map provides sufficient information for the fine module to identify confusing non-wire objects, thus avoiding over-prediction. Figure~\ref{fig:overpredict} demonstrates that our fine module with global logit map input successfully avoids a pattern on the building that strongly resembles a wire. 

% %auto-ignore
\begin{figure}[h!]
    \centering
    \begin{tabular}{@{}c@{\hspace{1mm}}c@{\hspace{1mm}}c@{}}
    \captionsetup{type=figure}
    \includegraphics[width=0.9\linewidth]{figures/overpredict/02709_global.jpg} \\
    (a) Fine module output with global logit map \\
    and binary location map. \\
    \includegraphics[width=0.9\linewidth]{figures/overpredict/02709_local.jpg} \\
    (b) Fine module output with only local logit map. \\
    \end{tabular}
    \captionof{figure}{(Image to be updated)}
    \label{fig:overpredict}
\end{figure}

In Table~\ref{table:component_ablation}, we report wire IoUs after removing each component in our model, including MinMax, MaxPool, and Coarse condition concatenation. We find that all components play a significant role for accurate wire prediction, particularly in large images. Both MinMax and MaxPool are effective in encouraging prediction, which is shown by the drop in recall without either component, also shown in Figure~\ref{fig:visual}. Coarse condition, as described in Section~\ref{sec:segmentation}, is crucial in providing global context to the local network, without which the wire IoU drops significantly.

Table~\ref{table:thresholds} shows the wire IoUs and inference speed of our two-stage model as $\alpha$ changes. We observe a consistent decrease in performance as $\alpha$ increases. On the other hand, setting $\alpha$ to 0.01 barely decreases IoU, while significantly boosting inference speed, which means the coarse network is effectively activated at wire regions.
% We believe that in addition to providing a condition for refinement, the coarse module also acts as a suppressor to eliminate false positives, it does so by skipping potential false positives that would otherwise be mis-classified as wires. As a result, setting the optimal $\alpha$ allows the network to predict accurate segmentation masks, avoid over-prediction, and maintain a high inference speed all at the same time.

% Note that when $\alpha=0.0\%$ (refine on all windows), our model still outperforms a single sliding-window model (70.3\% vs. 69.0\%). This means that the coarse module indeed provides useful information to the fine module via the logit map, which further justifies our two-stage design.

% \subsubsection{Effectiveness of model components}

% We find that the global context provides vital information to the local branch during inference to avoid false positives. When the global branch softmax is fed into the local branch together with the local image, the network suppresses predictions at regions that look like wires, such as pavement cracks. We show these in Figure X. Quantitatively, the wire IoU drops significantly without conditioning on the global branch.

% \subsubsection{Effectiveness of Maxpool resize}

% \subsubsection{Effectiveness of MinMax input}

% Here we show the useful information from the minmax image. Figure X compares a model trained with/without minmax. As can be seen, for extremely bright/dark wires, minmax is able to emphasize this feature and ensure prediction in those areas where they are not easily seen after downsampling.



\subsection{Wire Inpainting Evaluation}
\vspace{-1mm}
We evaluate our wire inpainting model using the synthetic dataset. Results are shown in Table~\ref{exp:wire_inp}. Our model structure is highly related to LaMa~\cite{suvorov2022resolution}. The difference is the training data and the proposed color adjustment module to address color inconsistency. We also compare our methods with PatchMatch \cite{barnes2009patchmatch} based on patch synthesis, DeepFillv2~\cite{yu2019free} based on Contextual Attention, CMGAN~\cite{zheng2022cm} and FcF~\cite{jain2022keys} based on StyleGAN2~\cite{karras2020analyzing} and LDM~\cite{rombach2022high} based on Diffusion. Inference speed is measured on a single A100-80G GPU. Visual results on synthetic and real images are shown in Figure \ref{fig:wire_inp}. PatchMatch, as a traditional patch synthesis method, produces consistent color and texture that leads to high PSNR. However, it performs severely worse on complicated structural completion. StyleGAN-based CMGAN and FcF are both too heavy for wires that are thin and sparse. Besides, diffusion-based models like LDM tends to generate arbitrary objects and patterns. DeepFill and the official Big-LaMa both have severe color inconsistency issue, especially in the sky region. Our model has a balanced quality and efficiency, and performs well on structural completion and color consistency. 
Note that we use a tile-based method at inference time.
% with a window size of $512 \times 512$ and an overlap of $32$.
The reason the tile-based strategy can be employed is due to the wire characteristics: sparse, thin and lengthy. More high-resolution inpainting results are in the supplementary materials.

% \begin{figure}[h!]
% \centering
% \captionsetup{type=figure}
% \includegraphics[width=1.\linewidth]{figures/inpainting_result.pdf}
% \vspace{-6mm}
% \captionof{figure}{\textbf{Inpainting Comparison}. Our model performs well on complicated structure completion and color consistency, especially on building facades and sky regions containing plain and uniform color. 
% \vspace{-3mm}
% }
% \label{fig:wire_inp}
% \end{figure}


% \begin{table}[t]\setlength{\tabcolsep}{5pt}
% \setlength{\abovecaptionskip}{8pt}
% \centering
% \footnotesize
% % \scriptsize
% % \tiny

% %\vspace{-2ex}
% % \resizebox{\columnwidth}{!}{
% \begin{tabular}{r|c c c|c}
% \hline
% Model &PSNR$\uparrow$&LPIPS$\downarrow$&FID$\downarrow$ &Speed (s/img)\\ \hline
% %Photoshop\\ %should be easy to run batch testing
% PatchMatch \cite{barnes2009patchmatch}&50.29 &0.0294 & 5.0403 & -\\
% DeepFillv2 \cite{yu2019free} &47.01 &0.0374&8.0086 &0.009\\
% CMGAN \cite{zheng2022cm} &50.07 &0.0255 &3.8286 &0.141\\
% FcF \cite{jain2022keys}&48.82&0.0322&4.7848&0.048\\
% LDM \cite{rombach2022high} & 45.96 & 0.0401& 10.1687 & 4.280\\
% Big-LaMa \cite{suvorov2022resolution} & 49.63 & 0.0267& 4.1245 &0.034\\
% Ours (LaMa-Wire) & 50.06 & 0.0259 & 3.6950 &0.034\\
% \hline
% \end{tabular}
% \caption{Quantitative results of inpainting on our synthetic wire inpainting evaluation dataset (1000 images). Our model achieves the highest perceptual quality in terms of FID, and has a balanced speed and quality.}
% % }
% \label{exp:wire_inp}
% % \vspace{-4mm}
% \end{table}