%auto-ignore

\section{Discussion}

% \subsection{Wire completion}

% In some cases, a part of a wire goes in front of a textured object and becomes difficult to recognize. When this happens, a light post-processing step can be performed to connect two wire segments and form a complete wire. We achieve this by fitting a parabola on each predicted wire segment. Pixels along the fitted curve that are originally predicted as background with a wire confidence of above 0.4 are reclassified as wire. We show in Figure~\ref{fig:completion} how this simple post-processing method can help picking up any missed wires due to blending with backgrounds.

\begin{comment}
\subsection{Wire removal}
%\input{figure_tex/removal}

Figure~\ref{fig:pixel6} (B) show two examples of automatic wire removal using wire segmentation masks from our proposed model. We use the ``Content-aware Fill''~\cite{barnes2009patchmatch} feature in Photoshop to remove wires. The full process of wire segmentation and removal is fully automatic without the need of any user interaction.
\end{comment}

% \subsection{Generalizing to large images}

% %auto-ignore
\begin{figure*}[t!]
\centering
\includegraphics[width=1.0\textwidth]{figures/panorama.pdf}
\vspace{-7mm}
\caption{\textbf{Segmentation result for a panoramic image.} Our two-stage model leverages the sparsity of wires in natural images, and efficiently generalizes to ultra-high resolution images such as this panorama of $10$K by $2$K resolution. Note that our method produces high quality wire segmentation that covers wires that are almost invisible.}
\vspace{-4mm}
\label{fig:panorama}
\end{figure*}

% Given the fact that wires are sparse in natural images, we show that our two-stage model generalizes to ultra-high resolution images such as panoramas without significant computation overhead. A $10$K by $2$K panorama with its segmentation result is shown in Figure~\ref{fig:panorama}. The full inference takes $6.7$ seconds and produces high-quality wire segmentation.
\vspace{-1.5mm}
\subsection{Comparison with Google Pixel 6}
\vspace{-1.5mm}
% %auto-ignore
\begin{figure}[h!]
    \setlength{\abovecaptionskip}{1mm}
    \centering
    \captionsetup{type=figure}
    \includegraphics[width=0.43\textwidth]{figures/wire-pixel6.pdf}
    \vspace{-1mm}
    \captionof{figure}{\textbf{Comparison with Pixel 6}. Our model can pick up hardly visible wires that even in complicated backgrounds}
    \label{fig:pixel6}
\end{figure}
Recently, Google Pixel 6~\cite{Pixel6as49} announced the ``Magic Eraser'' photo feature that automatically detects and removes distractors. Note that this is a product feature and is not specifically designed for wires, and thus is hardly comparable with our method. We compare against this feature by uploading the images to Google Photos and applying ``Magic Eraser'' without manual intervention. We find that ``Magic Eraser'' performs well on wires with clear background, but it suffers from thin wires that are hardly visible and wires with complicated background. We show two examples in the supplementary material.

\subsection{Failure cases}
\vspace*{-2.5mm}
% %auto-ignore
\begin{figure}[h!]
\centering
\captionsetup{type=figure}
\includegraphics[width=1.\linewidth]{figures/failure.pdf}
\vspace{-6mm}
\captionof{figure}{\textbf{Failure cases}. Our model can fail for cases where wires are heavily blended with the background (upper row), or on thick wires with complex patterns that are rarely seen.
\vspace{-5mm}
}
\label{fig:failure}
\end{figure}
%\cezhang{todo: fill in caption for figure and revise this paragraph}
%Figure~\ref{fig:failure} shows examples where our model does not perform well. In the upper row example, the diagonal and vertical wires are ambiguously blended with the background texture, which make the wires hardly distinguishable. In the lower row case, thicker wires with complex structures rarely appear in the dataset, causing predictions of only part of the wire.

While our proposed wire segmentation model produces high-quality masks in most situations, there are still some challenging cases that our model cannot resolve. In particular, wires that are heavily blended in with surrounding structures/background, or wires under extreme lighting conditions are challenging to segment accurately. We show several examples in the supplementary material.

%auto-ignore

\begin{table}[t!]
\centering
\resizebox{0.49\textwidth}{!}{
    \renewcommand{\arraystretch}{1}
    \addtolength{\tabcolsep}{-2pt}
    \begin{tabular}{r|c|ccc|ccc}
    \hline
    Model & \begin{tabular}[x]{@{}c@{}}Wire\\IoU\end{tabular} & F1 & Precision & Recall & \begin{tabular}[x]{@{}c@{}}IoU\\(Small)\end{tabular} & \begin{tabular}[x]{@{}c@{}}IoU\\(Medium)\end{tabular} & \begin{tabular}[x]{@{}c@{}}IoU\\(Large)\end{tabular} \\\hline\hline
    %& \begin{tabular}[x]{@{}c@{}}Speed\\(s/img)\end{tabular}  & \begin{tabular}[x]{@{}c@{}}Memory\\(GB)\end{tabular}\\ \hline\hline 
    Ours & 60.83 & 75.65 & 83.62 & 69.06 & 63.52 & 59.83 & 62.93 \\ \hline
    -- MinMax & 60.01 & 75.01 & 84.87 & 67.2 & 63.67 & 58.99 & 61.97 \\
    -- MaxPool & 59.86 & 74.89 & 85.25 & 66.78 & 61.45 & 59.40 & 60.76 \\
    -- Coarse & 56.92 & 72.55 & 82.91 & 64.49 & 62.83 & 57.42 & 54.47 \\ \hline
\end{tabular}
\addtolength{\tabcolsep}{2pt}
}
\vspace{-1mm}
\caption{Ablation study of our model components.
% \yq{This table can be made one-column to save space.}\cezhang{+1, we're running out of space.}
}
\label{table:component_ablation}
\vspace{-2mm}
\end{table}
%auto-ignore
\begin{table}[t!]
\centering
\resizebox{0.9\columnwidth}{!}{
    \renewcommand{\arraystretch}{1.1}
    \begin{tabular}{r|c|ccc|cc}
    \hline
    $\alpha$& \begin{tabular}[x]{@{}c@{}}Wire\\IoU\end{tabular} & F1 & Precision & Recall & \begin{tabular}[x]{@{}c@{}}Avg. Time\\(s/img)\end{tabular} &
    Speed up\\ \hline
    0.0 & 60.97 & 75.75 & 82.63 & 69.93 & 1.91 & 1$\times$ \\
    0.01 & 60.83 & 75.65 & 83.62 & 69.06 & 0.82 & 2.3$\times$ \\
    0.02 & 60.35 & 75.27 & 83.97 & 68.20 & 0.75 & 2.5$\times$ \\
    0.05 & 55.17 & 71.11 & 84.84 & 61.20 & 0.58 & 3.3$\times$ \\
    0.1 & 42.44 & 59.59 & 86.06 & 45.57 & 0.4 & 4.8$\times$ \\
    \hline
\end{tabular}
}
\vspace{-1mm}
\caption{Ablation on the threshold for refinement. At $\alpha=0.0$, all windows are passed to the fine module.}
\label{table:thresholds}
%\vspace{-2mm}
\end{table}
%auto-ignore

\begin{table}[t]\setlength{\tabcolsep}{5pt}
\setlength{\abovecaptionskip}{8pt}
\centering
\footnotesize
% \scriptsize
% \tiny

%\vspace{-2ex}
% \resizebox{\columnwidth}{!}{
\begin{tabular}{r|c c c|c}
\hline
Model &PSNR$\uparrow$&LPIPS$\downarrow$&FID$\downarrow$ &Speed (s/img)\\ \hline
%Photoshop\\ %should be easy to run batch testing
PatchMatch \cite{barnes2009patchmatch}&50.29 &0.0294 & 5.0403 & -\\
DeepFillv2 \cite{yu2019free} &47.01 &0.0374&8.0086 &0.009\\
CMGAN \cite{zheng2022cm} &50.07 &0.0255 &3.8286 &0.141\\
FcF \cite{jain2022keys}&48.82&0.0322&4.7848&0.048\\
LDM \cite{rombach2022high} & 45.96 & 0.0401& 10.1687 & 4.280\\
Big-LaMa \cite{suvorov2022resolution} & 49.63 & 0.0267& 4.1245 &0.034\\
Ours (LaMa-Wire) & 50.06 & 0.0259 & 3.6950 &0.034\\
\hline
\end{tabular}
\vspace{-1mm}
\caption{Quantitative results of inpainting on our synthetic wire inpainting evaluation dataset (1000 images). Our model achieves the highest perceptual quality in terms of FID, and has a balanced speed and quality.}
% }
\label{exp:wire_inp}
% \vspace{-4mm}
\end{table}
%auto-ignore

\begin{figure}[h!]
\centering
\captionsetup{type=figure}
\includegraphics[width=1.\linewidth]{figures/inpainting_result.pdf}
\vspace{-6mm}
\captionof{figure}{\textbf{Inpainting Comparison}. Our model performs well on complicated structure completion and color consistency, especially on building facades and sky regions containing plain and uniform color. 
\vspace{-3mm}
}
\label{fig:wire_inp}
\end{figure}

% \subsection{Potential negative impacts}

% While our work is primarily used in image retouching applications, improper use of our wire segmentation model and content removal may spread misinformation for malicious intents. Content authentication could be used to mark the integrity of the shared content. Caution must be taken on the potential negative social impacts that come with the application of this work.
% \vspace{-2mm}