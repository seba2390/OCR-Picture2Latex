%auto-ignore

\section{Dataset Collection and \benchmark} \label{sec:dataset}
% We present 
%We will release the test set (407 images?) to provide a standardized platform for benchmarking. \yq{Should we mention here we only release the testing dataset, or just do not mention now whether we want to release the training one.} 
%In this section, we describe the unique properties and provide statistics of WireSeg-HR, a high-resolution photographic image benchmark dataset for wire semantic segmentation.

\subsection{Image Source and Annotations}

% We collected high-resolution images with wires and wire-like objects from multiple sources. 20\% of the images are taken with DSLRs and smartphone cameras, while the others are sourced from the Internet (e.g. Flickr). As a result, the images have a variety of resolutions and have gone through different image processing pipelines. We target images with scenes that include city, street, rural area and landscape. We restricted the number of images in which the wire is the major object. We manually filter images so that wires in these images appear differently, for example, in front of buildings, partially occluded by vegetation or be hardly visible in the sky. This leads to a rich set of wire appearances. The final dataset after manual selection contains 3201 images.
Our definition of wires include electrical wires/cables, power lines, supporting/connecting wires, and any wire-like object that resemble a wire structure.
We collect high-resolution images with wires from two sources: 80\% of the images are from photo sharing platforms (Flickr, Pixabay, etc.), and 20\% of the images are captured with different cameras (DSLRs and smartphones) in multiple countries on our own. For the internet images, we collect 400K candidate images by keyword-searching. Then, we remove duplicates and images where wires are the only subjects. We then curate the final 6K images that cover sufficient scene diversity like city, street, rural area and landscape.

%\subsection{Wire annotation}

%auto-ignore
\begin{figure}[t!]
    \centering
    \captionsetup{type=figure}
    \includegraphics[width=\linewidth]{figures/annotation.pdf}\\
    \vspace{-1mm}
    \captionof{figure}{\textbf{Wire Annotation Example.} An example wire annotation in our dataset. Our annotation (B) is accurate in different wire thicknesses (\textcolor{red}{red}), variations in wire shapes (\textcolor{orange}{orange}) and accurate wire occlusions (\textcolor{yellow}{yellow}).}
    % \textcolor{red}{This image is unchanged from last year, do we want to say this is one of the collected images rather than saying it's in the benchmark dataset?}
\vspace{-4mm}
    \label{fig:annotation}
\end{figure}
%auto-ignore
\begin{figure*}[hbt!]
\centering
\includegraphics[width=1.0\textwidth]{figures/wire-pipeline.png}
\vspace{-2mm}
\caption{\textbf{Our wire removal system}. A system overview of our wire segmentation and removal for high resolution images. Input is concatenated with min- and max-filtered luminance channels. The downsampled input is fed into the coarse module to obtain the global probability. In the local stage, original-resolution patches are concatenated with the global probability map to obtain the local logit map. After a segmentation mask is predicted, we adopt LaMa architecture and use a tile-based approach to achieve wire removal. See Section~\ref{sec:segmentation},~\ref{sec:inpainting} for details.}
\vspace{-1mm}
\label{fig:pipeline}
\end{figure*}
Our wire annotation process contains two rounds. In the first round, annotators draw detailed masks over wires at full-resolution. The annotated masks enclose the main wire body and the boundary, oftentimes including a gradient falloff due to aliasing or defocus.
The boundary region annotation is crucial so as to avoid residual artifacts during wire removal.
In the second round, quality assurance is carried out to re-annotate unsatisfactory annotations. We show an example of our high-quality wire annotations in Figure~\ref{fig:annotation}. 
% As shown, our dataset contains highly detailed masks for wires of various shapes and appearances.

\subsection{Dataset Statistics}
%\yq{Do we want to release the entire dataset. It's better to have a table comparing the statistics of different exisiting datasets and show the special cases we covered but others do not. }\cezhang{agreed, we should also mention and justify that this is the first wire dataset, unlike previous datasets such as the one for line detection or for satellite images, or any other available ones.}

In Table~\ref{table:stats}, we list the statistics of our dataset and compare them with existing wire-like datasets. Our dataset is the first wire dataset that contains high-resolution photographic images. The dataset is randomly split into 5000 training, 500 validation, and 500 testing images. We release 420 copyright-free test images with annotations.
%We will release the 500 test images (the \benchmark) with annotations upon paper acceptance.

%We will release \todo{how many?} test set as the first wire segmentation benchmark dataset.

%auto-ignore
\begin{table}[h!]
\centering
\resizebox{\linewidth}{!}{
    \renewcommand{\arraystretch}{1.1}
    \begin{tabular}{r|cccc}
    \hline
    
    Dataset & \begin{tabular}[x]{@{}c@{}}\# Wire\\Images\end{tabular}  & \begin{tabular}[x]{@{}c@{}}Min.\\Image Size\end{tabular}  &\begin{tabular}[x]{@{}c@{}}Max.\\Image Size\end{tabular}  &\begin{tabular}[x]{@{}c@{}}Median\\Image Size\end{tabular} \\ \hline
    Powerline~\cite{powerlinedataset} & 2000 & 128$\times$128 & 128$\times$128 & 128$\times$128\\
    PLDU~\cite{pldu} & 573 & 540$\times$360 & 540$\times$360 & 540$\times$360 \\
    PLDM~\cite{pldu} & 287 & 540$\times$360 & 540$\times$360 & 540$\times$360 \\
    TTPLA~\cite{ttpla} & 1100 & 3840$\times$2160 & 3840$\times$2160 & 3840$\times$2160\\ \hline
    \textbf{Ours} & 6000 & 360$\times$240 & 15904$\times$10608 & 5040$\times$3360 \\ \hline

\end{tabular}
}
\vspace{-2mm}
\caption{Statistics of our wire dataset compared to others.}%Image and annotation statistics of our test set.\yq{You can make it as a 4x4 table, with each row: item, min, max, avg; and each column: item name, image size, wire thickness, and percentage. (mt: done)}}
\vspace{-5mm}
\label{table:stats}
\end{table}

