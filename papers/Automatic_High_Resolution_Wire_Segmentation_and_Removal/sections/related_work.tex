%auto-ignore
\section{Related Work} \label{sec:related}

\section{Motivations for Empirical Study}
\label{sec:motivations}
The key question that we try to answer is when and why we should use standard
iteration space tiling over cache oblivious tiling.  The two approaches
perform similar partitioning of the iteration space, but the schedules given
to the partitions are different.  Theoretically, cache oblivious code seems to
have advantages over iteration space tiling.  However, many factors complicate
the actual performance, which made our initial experiments difficult to
interpret.  In this section, we describe the obstacles between the theory and
practice we have identified.

We use Single-Level Tiling (SLT) for iteration space tiling, and Cache
Oblivious Tiling (COT) for cache oblivious techniques in this
paper, which are further described in Section~\ref{sec:background}.

\paragraph{Recursion Overhead} This is a well-known overhead of
COT~\cite{yotov2007experimental}.  The recursion introduces overheads, such as
function call overhead, and increased register pressure.  Furthemore, the
functions force inter-procedural analysis/optimization, known to be more
difficult for compilers well.  Thus, the leaf tiles must be ``sufficiently
large'' to avoid excessive overhead due to the recursion.

 \paragraph{Recursive Split Constraints the Tile Sizes} In typical cache
 oblivious algorithms, the problem is recursively split into halves in each
 dimension. This is in fact a rather coarse-grained exploration of the
 hierarchical partitioning of the iteration space. For instance, if the
 current problem size is $B^3$, then the next sub-problem would be
 $(\frac{B}{2})^3$.  If the best problem size for utilizing a level of cache
 is $(B-x)^3$ where $x\ll \frac{B}{2}$ then the subproblems due to
 divide-and-conquer will not match the best.  This is another factor that
 necessitates fine tuning of leaf tile sizes even for COT, since the utilization
 rate of L1 cache has strong impact on performance.  

%\paragraph{COT Leads to Imbalanced Tiles} Current COT tools recursively split
%the problem into halves in each dimension.  If the original bounds are not
%powers of two, every power-of-two leaf will be paired with a non-power-of-two
%leaf.  Since leaf tile sizes are often carefully tuned, thismeans that half
%the leaves will be suboptimal.  Our code generator incorporates a simple
%optimization that ensures that such suboptimal leaf nodes only occur at the
%boundaries of the iteration space.

\paragraph{COT has more Conflict Misses} The divide-and-conquer execution
order may negatively affect cache interference, especially with high
dimensional data.  This happens when the memory is allocated such that the
accesses are contiguous along some direction in the iteration space (typically
along innermost canonical axis).  With lexicographic order of execution, this
contiguity is largely preserved in the tiled execution.  However,
divide-and-conquer executes neighboring tiles in all dimensions, and many of
those tiles access some distant location in memory.  In contrast to accessing
contiguous regions of memory, accessing various segments of the memory
increases the chances of conflicts.

\paragraph{Hardware Prefetching}  Modern architectures are equipped with
hardware prefetchers that can bring data to the L1 cache. When
having sufficient locality at L2 or LLC makes the program compute-bound, then
the latency to L2/LLC can be hidden by the prefetcher. For such programs, it is
unnecessary to tile for the fastest cache, and larger tiles targeting slower
caches improve performance by maximizing prefetcher
effectiveness~\cite{mehta2016turbotiling}. When the primary objective is speed,
the leaf tiles for COT should also be large, which negates the benefit of
divide-and-conquer, as the leafs are already targeting slower caches.
Prefetching have little impact on parallel executions, since prefetching is
bandwidth limited. When multiple cores try to prefetch at the same time,
the bandwidth limit is quickly reached, and the latency hiding effect is
lost. Furthermore, smaller tile sizes are better for parallel execution for
load balancing  reasons.


These factors limit the effectiveness of COT in various ways and are also
closely tied to the characteristics of the computation. Our empirical study
illustrate the impact of these factors on polyhedral computations.

% Local Variables: ***
% TeX-master: "TACO2017.tex" ***
% fill-column: 78 ***
% End: ***


% \textcolor{blue}{1. semantic segemtnation,however, as we will discuss, directly using these methods will nto help.  , 3. high resolution: xxx, - directly using them is not working 3. wire segmentnion.}

%We introduce related works in the following way. We first describe several widely-used semantic segmentation methods, followed by methods that tackle more specific challenges such as high-resolution segmentation and wire segmentation. We describe key principles of these works, where we can draw inspirations and improve upon.

\paragraph{Semantic segmentation}
Semantic segmentation has been actively researched over the past decade. For example, the DeepLab series~\cite{deeplab, deeplabv3, deeplabv3p} has been one of the most widely used set of semantic segmentation methods. They leverage dilated convolutions to capture long-range pixel correlations. Similarly, CCNet~\cite{ccnet} attend to non-local regions via a two-step axial attention mechanism. PSPNet~\cite{pspnet} use multi-scale pooling to extract high-resolution features.

Recently, the self-attention mechanism~\cite{attention} has gained increasing popularity. 
% Originally applied in Natural Language Processing tasks, its vision-based variants~\cite{vit, swin} have shown to be superior to traditional convolution-based methods. 
Transformer-based models for semantic segmentation~\cite{dpt, setr, swin, segformer, hassani2022neighborhood, hassani2022dilated, jain2021semask, jain2022oneformer} significantly outperform convolution-based networks since the attention modules benefit from their global receptive fields~\cite{segformer}, which let the models attend to objects that span across larger portions of the feature map.

While these above methods work well in common object semantic segmentation, when applied to our task of wire segmentation in high-resolution images, they either drop significantly in segmentation quality or require long inference times. We show in Section~\ref{sec:results} that directly applying these methods to our task yields undesirable results.

\vspace{-5mm}
\paragraph{High-resolution image segmentation}
Segmentation in high-resolution images involves additional design considerations. It is computationally infeasible to perform inference on the full-resolution image with a deep network. As a result, to maximally preserve image details within the available computation resources, many methods employ a global-local inference pipeline. For instance, GLNet~\cite{glnet} simultaneously predict a coarse segmentation map on the downsampled image and a fine segmentation map on local patches at the original resolution, then fuse them to produce the final prediction. 
%Their model shares features produced by both the global and local branch, thus achieving feature fusion.
MagNet~\cite{magnet} is a recent method that proposes to iteratively predict and refine coarse segmentation maps at multiple scales using a single feature extractor and multiple lightweight refinement modules. CascadePSP~\cite{cascadepsp} train a standalone class-agnostic model to refine predictions at a higher resolution from a pretrained segmentation model. ISDNet~\cite{isdnet} propose to use an extremely lightweight subnetwork to take in the entire full-resolution image. However, the subnetwork is limited in capacity and thus segmentation quality. We share the same idea with these past works on using a coarse-to-fine approach for wire segmentation, but modify the architecture and data processing to tailor to wires.

% Most of these methods share a similar coarse-to-fine approach, where instead of performing a single inference on high-resolution images, they divide them into a coarse-to-fine approach. We repurpose important components of these methods into a two-stage model design optimized for wire segmentation in high-resolution images.
%While our model pipeline is similar to the above methods, there are a few fundamental differences that makes our pipeline more effective and efficient in our task. First, GLNet essentially uses two separate networks for their global and local branches. The combined network does not share weights, and requires a three-stage training scheme. Meanwhile, CascadePSP only does segmentation refinement, which means they require an entire separate network to produce the initial prediction. MagNet uses an iterative refinement method at multiple scales, where the inference time scales quadratically with number of refinement scales. They also use a much smaller module for refinement, which limits quality of refinement. In contrast, our proposed network shares the feature extractor and can be trained end-to-end. Our model is also capable of predicting accurate wire masks on very high-resolution images with only two stages and without requiring any separately trained model for initial prediction.
\vspace{-5mm}
\paragraph{Wire/Curve segmentation}
While few works tackle wire segmentation in high-resolution images, there are prior works that handle similar objects. For example, Transmission Line Detection (TLD) is an actively researched area in aerial imagery for drone applications. Convolutional neural networks are used~\cite{ttpla, pldu, cable_inst, lsnet} to segment overhanging power cables in outdoor scenes. However, wire patterns in TLD datasets are relatively consistent in appearance and shape -- evenly spaced and only spanning locally. In contrast, we handle more generic wires seen in regular photographic contents, where the wire appearance has much higher variety. 
%We will discuss in Section~\ref{sec:results} that the models from these works do not generalize to wires in our task.

Some other topics are loosely related to our task. Lane detection~\cite{lanedet,swiftlane,structurelane} aims to segment lanes for autonomous driving applications. These methods benefit from simple line parameterization (e.g., as two end-points), and strong positional priors. In contrast, as shown in Figure~\ref{fig:motivation}, wires vary drastically in shapes and sizes in our task, thus making them difficult to parameterize.

\vspace{-5mm}
\paragraph{High-Resolution Image Inpainting}
Image inpainting has been well-explored using patch synthesis-based methods \cite{barnes2009patchmatch, wexler2007space, darabi2012image, kaspar2015self} or deep neural networks \cite{contextencoder, globallocal, partialconv, contextual, yu2019free, xu2022image}. Zhao \textit{et al.} leveraged the powerful image sysnthesis ability of StyleGAN2 \cite{karras2020analyzing} and proposed CoModGAN \cite{comodgan} to push the image generation quality to a newer level, and was followed by \cite{zheng2022cm, jain2022keys}. Most of these deep models cannot be applied to inpainting tasks at high-resolution images. The latest diffusion-based inpainting model like DALLE-2 \cite{dalle}, LDM \cite{rombach2022high}, and StableDiffusion etc. also suffer from long inference time and low output resolution. ProFill \cite{zeng2020high} was first proposed to address high resolution inpainting via a guided super resolution module. HiFill \cite{hifill} utilized a contextual residual aggregation module and the resolution can be up to 8K. LaMa \cite{suvorov2022resolution} applied the fourier convoluational residual blocks to make the propagation of image structures well. LaMa was trained on only $256 \times 256$ images, but can be used for images up to 2K with high quality. Recently, Zhang \textit{et al.} \cite{supercaf} proposed to use guided PatchMatch for any-resolution inpainting and extended the deep inpainting results from LaMa to modern camera resolution. The textures are better reused, while the structure and line completion at high-resolution can still be challenging. In this paper, we aim at removing wires from high resolution photos. The problem can become easier if we run inpainting in a local manner since wires are usually thin and long. Therefore, we propose to revisit LaMa for wire removal, and run the inference in a tile-based fashion. 