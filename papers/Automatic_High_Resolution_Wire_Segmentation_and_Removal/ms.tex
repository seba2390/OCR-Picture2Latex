% CVPR 2023 Paper Template
% based on the CVPR template provided by Ming-Ming Cheng (https://github.com/MCG-NKU/CVPR_Template)
% modified and extended by Stefan Roth (stefan.roth@NOSPAMtu-darmstadt.de)

\documentclass[10pt,twocolumn,letterpaper]{article}

%%%%%%%%% PAPER TYPE  - PLEASE UPDATE FOR FINAL VERSION
% \usepackage[review]{cvpr}      % To produce the REVIEW version
\usepackage{cvpr}              % To produce the CAMERA-READY version
%\usepackage[pagenumbers]{cvpr} % To force page numbers, e.g. for an arXiv version

\usepackage[accsupp]{axessibility}  % Improves PDF readability for those with disabilities.

% Include other packages here, before hyperref.
\usepackage[normalem]{ulem}
\usepackage{graphicx}
\usepackage{amsmath}
\usepackage{amssymb}
\usepackage{booktabs}
\usepackage{xcolor}
\usepackage{comment}
\usepackage{enumitem}
\newcommand{\todo}[1]{{\color{red}#1}}

\newcommand{\benchmark}{WireSegHR}

% It is strongly recommended to use hyperref, especially for the review version.
% hyperref with option pagebackref eases the reviewers' job.
% Please disable hyperref *only* if you encounter grave issues, e.g. with the
% file validation for the camera-ready version.
%
% If you comment hyperref and then uncomment it, you should delete
% ReviewTempalte.aux before re-running LaTeX.
% (Or just hit 'q' on the first LaTeX run, let it finish, and you
%  should be clear).
\usepackage[pagebackref,breaklinks,colorlinks]{hyperref}


% Support for easy cross-referencing
\usepackage[capitalize]{cleveref}
\crefname{section}{Sec.}{Secs.}
\Crefname{section}{Section}{Sections}
\Crefname{table}{Table}{Tables}
\crefname{table}{Tab.}{Tabs.}


%%%%%%%%% PAPER ID  - PLEASE UPDATE
\def\cvprPaperID{699} % *** Enter the CVPR Paper ID here
\def\confName{CVPR}
\def\confYear{2023}

\begin{document}

%%%%%%%%% TITLE - PLEASE UPDATE
% \title{Automatic Wire Segmentation for Inpainting}
% \title{Segmentation to the Extreme: \\A Large-Scale Wire Segmentation Dataset and a Pilot Study}
%\title{Segmentation to the Extreme: \\A Large-Scale High-Resolution Wire Segmentation Dataset and a Pilot Study}
%\title{WINE: WIre NEver Appeared in Your Photos}
\title{Automatic High Resolution Wire Segmentation and Removal}

\author{Mang Tik Chiu$^{1,2}$, Xuaner Zhang$^2$, Zijun Wei$^{2}$, Yuqian Zhou$^2$, Eli Shechtman$^2$, \\Connelly Barnes$^{2}$, Zhe Lin$^2$, Florian Kainz$^2$, Sohrab Amirghodsi$^2$, Humphrey Shi$^{1,3}$ 
\\
{\small $^1$UIUC, $^2$Adobe, $^3$University of Oregon}\\
{\small \textbf{\url{https://github.com/adobe-research/auto-wire-removal}}}}

\twocolumn[{%
\renewcommand\twocolumn[1][]{#1}%
\maketitle
\begin{center}
    \centering
    \captionsetup{type=figure}
    \includegraphics[width=\textwidth]{figures/teaser2.pdf}
    \captionof{figure}{We present an automatic high-resolution wire segmentation and removal pipeline. Each triad shows the high-resolution input image, our automatic wire segmentation result masked in red, and our full-resolution wire removal result. The visual quality of these photographs is greatly improved with our fully-automated wire clean-up system.}
    \label{fig:demo}
\end{center}%
}]

% \vspace{3mm}
%[INTERNAL]: \href{https://session-0f3f6c51-53bd-446a-ad30-ea59ed8b6445.devbox.training.adobesensei.io/tensorboard/result.html}{Visualizations} of current models.

%%%%%%%%% ABSTRACT
% \begin{figure}[t!]
  \centering
  \includegraphics[width=\linewidth]{figs/teaser/teaser_fig.pdf}
  \caption{Given only one facial image, \ours \ can efficiently generate countless identity-preserved and  text-coherent images in different context without any test-time optimization.}
  % \vspace{0.4cm}
  \label{fig:teaser}
\end{figure}


\begin{abstract}
\label{sec:abstract}

%% 1. what is the problem 
Scientific applications that run on leadership computing facilities often face the challenge 
of being unable to fit leading science cases onto accelerator devices due to memory constraints 
(memory-bound applications).
%
% 2. what is your solution 
In this work, the authors studied one such US Department of Energy mission-critical condensed matter 
physics application, Dynamical Cluster Approximation (DCA++), and this paper discusses how device memory-bound challenges were successfully reduced  by proposing an effective 
``all-to-all'' communication method---a ring communication algorithm. 
%
This implementation takes advantage of acceleration on GPUs and remote direct memory access (RDMA) for fast data exchange between GPUs. 
%
\\Additionally, the ring algorithm was optimized with sub-ring communicators
and multi-threaded support to further reduce communication overhead and 
expose more concurrency, respectively.
%
% 3. What's the cherry-picked evaluation result you want to mention
The computation and communication were also analyzed 
by using the Autonomic Performance Environment for Exascale 
(APEX) profiling tool,  and this paper further discusses the 
performance trade-off for the ring algorithm implementation. 
%
The memory analysis on the ring algorithm shows that the allocation size for the authors' most 
memory-intensive data structure per GPU is now reduced to $1/p$ of the original size, where $p$ is the number of GPUs in the ring communicator.
%
The communication analysis suggests that 
the distributed Quantum Monte Carlo execution time grows linearly as sub-ring size increases, and the cost of messages passing through the network interface connector could be a limiting factor.


%
% \todoRed{Ronnie: Next sentence needs rewrite, too much information about Green's function that no one knows in the abstract; recommend generalizing.} \emph {However, DCA++ is currently facing memory-bound challenge as 
% a larger device array $G_t$ is limited by device memory size, where
% $G_t$ is a two-particle Green's function that allows condensed matter
% scientists to explore larger and more complex (higher fidelity)
% physics cases.}

\end{abstract}

\keywords{DCA++, Quantum Monte Carlo, GPU Remote Direct Memory Access, memory-bound issue, exascale machines}


%%%%%%%%% BODY TEXT
\section{Introduction}  \label{sec:introduction}

\newcommand\inexpIntro[3]{#1?(#2,#3).}
\newcommand\rinexpIntro[3]{*#1?(#2,#3).}
\newcommand\outexpIntro[3]{#1!(#2,#3).}
\newcommand\outatomIntro[3]{#1!(#2,#3)}

We propose a fully automated method for proving termination of \(\pi\)-calculus processes.
Although there have been a lot of studies on termination analysis for the \(\pi\)-calculus
and related calculi~\cite{Deng06IC,Demangeon07,SangiorgiTermination,KobayashiHybrid,Yoshida04IC,DBLP:journals/jlp/DemangeonHS10,Venet98SAS}, most of them have been rather theoretical,
and there have been surprisingly little efforts in developing  fully automated termination
verification methods and tools based on them. To our knowledge,
Kobayashi's \typical{}~\cite{TyPiCal,KobayashiHybrid} is the only exception that
can prove termination of \(\pi\)-calculus processes (extended with natural numbers)
fully automatically, but its termination analysis is quite limited (see Section~\ref{sec:relatedwork}).

Our method is based on a reduction to termination analysis for sequential programs:
we translate a \(\pi\)-calculus process \(P\) to a sequential program \(S_P\), so that
if \(S_P\) is terminating, so is \(P\). The reduction allows us to use
powerful, mature methods and tools
for termination analysis of sequential programs~\cite{heizmann2016ultimate,freqterm,DBLP:conf/lics/PodelskiR04,Kuwahara2014Termination,DBLP:journals/cacm/CookPR11}.

The idea of the translation is to convert a chain of communications on replicated input
channels to a chain of recursive function calls of the target sequential program.
Let us consider the following Fibonacci process:
\begin{align*}
    & \rinexpIntro{\fib}{n}{r}
        \ifexp{n<2}{ \soutatom{r}{1} \\ &\quad}
                   { \nuexp{s_1} \nuexp{s_2} (\outatomIntro{\fib}{n-1}{s_1} \PAR \outatomIntro{\fib}{n-2}{s_2} \PAR \sinexp{s_1}{x}\sinexp{s_2}{y}\soutatom{r}{x+y}) \\}
    & \PAR \outatomIntro{\fib}{m}{r}
\end{align*}
Here, the process
$\rinexpIntro{\fib}{n}{r} \ldots$ is a function server that computes the \(n\)-th Fibonacci number
in parallel and returns the result to \(r\),
and $\outatom{\fib}{m}{r}$ sends a request for computing the \(m\)-th Fibonacci number;
those who are not familiar with the syntax of the \(\pi\)-calculus may wish to consult
Section~\ref{sec:targetlanguage} first.
To prove that the process above is terminating for any integer \(m\),
it suffices to show that there is no infinite chain of communications on $\fib$:
\[
    \fib(m,r) \to \fib(m_1,r_1) \to \fib(m_2,r_2) \to \cdots.
\]
We convert the process above to the following program:\footnote{The actual translation
  given later is a little more complex.}
\begin{verbatim}
 let rec fib(n) = if n<2 then () else (fib(n-1) [] fib(n-2)) in
 fib(m)
\end{verbatim}
Here, \texttt{[]} represents the non-deterministic choice.
Note that, although the calculation of Fibonacci numbers is not preserved,
for each chain of communications on \texttt{fib}, there is a corresponding
sequence of recursive calls:
\[
\mathtt{fib}(m) \to \mathtt{fib}(m_1) \to \mathtt{fib}(m_2) \to \cdots.
\]
Thus, the termination of the sequential program above implies the termination of
the original process.
As shown in the example above, (i) each communication on a replicated input channel
is converted to a function call, (ii) each communication on a non-replicated input
channel is just removed (or, in the actual translation, replaced by a call of
a trivial function defined by \(f(\seq{x})=(\,)\)), and (iii) parallel composition
is replaced by a non-deterministic choice.
We formalize the translation outlined above and prove its correctness.

The basic translation sketched above sometimes loses too much information.
For example, consider the following process:
\begin{align*}
    & \rinexpIntro{\pre}{n}{r} \soutatom{r}{n-1} \\
    & \PAR \rinexpIntro{f}{n}{r} \ifexp{n<0}{ \soutatom{r}{1} }
                                       { \nuexp{s} (\outatomIntro{\pre}{n}{s} \PAR \sinexp{s}{x}\outatomIntro{f}{x}{r}) } \\
    & \PAR \outatomIntro{f}{m}{r}
\end{align*}
The translation sketched above would yield:
\begin{verbatim}
  let pred(n) = n-1 in
  let rec f(n) = if n<0 then () else (pred(n) [] f(*)) in
  f(m)
\end{verbatim}
Here, \texttt{*} represents a non-deterministic integer: since we have removed
the input $\sinatom{s}{x}$, we do not have information about the value of \( x \).
As a result, the sequential program above is non-terminating, although the original
process is terminating.
To remedy this problem, we also refine the basic translation above by using a refinement
type system for the \(\pi\)-calculus. Using the refinement type system,
we can infer that the value of \(x\) in the original process is less than \(n\),
so that we can refine the definition of \texttt{f} to:
\begin{verbatim}
 let rec f(n) = ... else (pred(n) [] let x=* in assume(x<n);f(x))
\end{verbatim}
The target program is now terminating, from which
we can deduce that the original process is also terminating.
We have implemented an automated tool based on the refined translation above.

The contributions of this paper are summarized as follows.
\begin{itemize}
\item The formalization of the basic translation from the \(\pi\)-calculus
  (extended with integers) to sequential programs, and a proof of its correctness.
\item The formalization of a refined translation based on a refinement type system.
\item An implementation of the refined translation, including automated refinement type
  inference based on CHC solving, and experiments to evaluate the effectiveness of
  our method.
\end{itemize}

The rest of this paper is structured as follows.
Section~\ref{sec:targetlanguage} introduces the source and target languages
of our translation.
Section~\ref{sec:approach} 
formalizes the basic translation, and proves its correctness.
Section~\ref{sec:refinement} refines the basic translation by using a refinement type system.
Section~\ref{sec:implementation} reports an implementation and experiments.
Section~\ref{sec:relatedwork} discusses related work,
and Section~\ref{sec:conclusion} concludes the paper.

The industry standard for pose edition is to create rigs, a collection of pieces of software designed to manipulate a character's skeleton. The rig describes the skeleton's bones, how they relate to each other, are constrained in their possible motion and are deformed. These rules are loosely specified and creating a good rig requires a detailed understanding of physics and anatomy, as well as technical and artistic skills. Rigging is thus a time consuming task even for experienced animators, and even more so in large scale productions which often require a different in-depth rig for each character in the cast.
Previous work has helped alleviate this difficulty by providing efficient tools to speed up/and or ease the rigging process, relying on inverse kinematics or data-driven methods.
\subsection{Character pose design}
\subsubsection{Inverse Kinematics (IK)}
IK solvers are a family of methods commonly used in robotics, engineering and computer graphics, in which the parameterization of a kinematic chain is determined from the position of its end effector.
They are a staple tool in pose design software, ensuring the respect of elementary constraints during pose edition. Their de-facto role is to guarantee the length of the limbs, and in some cases to enforce the orientation angle range of a joint.
Many IK solutions have been studied over the years \cite{aristidou_inverse_2018}; usually revolving around approximated linearizations or heuristics. 

Numerical methods require a set of iterations to achieve a satisfactory solution formulated by a cost function to be minimized.
IK solutions can generally be divided into three sub-categories: Jacobian \cite{Siciliano_Handbook_Robot_2007}, Newtonians \cite{cohen_ik_1996} and Heuristics. Most software implement heuristic methods such as Cyclic Coordinate Descent (CCD) \cite{wang_ccd_1991} or 
Forward-Backward Reaching IK (FABRIK) \cite{aristidou_fabrik:_2011} due to their simplicity and extensibility. 

The main drawback of 
these solvers is that they manipulate kinematic chains without taking into account many morphological aspects that make a pose more or less plausible. They offer a first level of help to users but are not sufficient to guarantee a realistic pose. Many joints constraints are dependent on each other and require subjective, human-made approximations.

\subsubsection{Data-driven pose edition}
Data-driven methods offer promising opportunities to solve these approximations. Using real-life data can help in modelling the complex inter-dependencies of skeletons and providing users with smarter edition tools.
While it is still an early field of research, some solutions have been studied. Wu \etal \cite{wu_posing_2009} propose a method for natural character posing from a large motion database. It employs adaptive KD-clustering to select a representative frame from a database and sparse approximations to accelerate training and posing. 
Huang \etal in \cite{Huang_IK_MGDM_2017} present a method based on the formulation of multi-variate Gaussian distribution models (MGDMs), which learn the joint constraints of a kinematic skeleton from motion capture data. 

Some work has also been dedicated to finding new editing interfaces. \modify{}{Instead of the usual setup manipulating joints directly, Guay \etal \cite{guay_line_2013} articulate a framework based on the conceptual "line of action" which describes the overall pose dynamics. They provide a mathematical definition of the line of action, and a interface in which the software modifies the pose to follow a user-provided line. In the same line of though} Garcia \etal \cite{garcia_sketching_2019} propose \modify{a method transforming doodle of trajectories (position and orientation over time) }{a virtual reality-based interface where the user's hands motion (position and orientation over time) are transformed} into sequences of actions and then into detailed character animations using a dataset of parametrized motion clips automatically fitted to the trajectory. 

% ==> DL et Latent Space. 
\subsection{Neural modelling of human motion}
Neural networks have received a great amount of attention over the last decade and shown impressive result in modelling complex data. Human motion has not been spared and deep learning methods have proven their capability of generating realistic motion in a number of difficult cases. 

The literature in neural-based animation include example in user-controlled character navigation \cite{Holden2017} and interactions with the environment \cite{starke_neural_2019}. 
Holden \etal \cite{Holden2020} also show that neural networks can be used to replace parts of existing data-driven methods, improving their scalability potential.
More recently, some work has also focused on improving smaller parts of the animation pipeline rather than replacing it completely. Berson et al. \cite{berson_intuitive_2020} leverage neural networks to provide an interactive system to edit facial animation. 

% Wrap up
Data-driven IK and pose editing can relieve animators from time-consuming, back-and-forth pose adjustments by applying constraints extracted from real-world data. Recently, neural-network-based approaches have demonstrated their ability to model the intricacies of human motion while scaling to large amount of data and retaining a fast inference time. In this paper we seek to take advantage of these properties to create an efficient posing tool, intuitively usable even by a inexperienced user.
\section{The Semantic Urban Mesh Dataset}\label{sec:framework}
\subsection{Dataset Specification}

We have used Helsinki's 3D texture meshes as input and annotated them as a benchmark dataset of semantic urban meshes. 
The Helsinki's raw dataset covers about 12 $ km^2 $, and it was generated in 2017 from oblique aerial images that have about a 7.5 $cm$  ground sampling distance (GSD) using an off-the-shelf commercial software namely ContextCapture~\citep{contextcap}.
The source images have three colour channels (i.e., red, green, and blue) and are collected from an airplane with five cameras that have $80\%$ length coverage and $60\%$ side coverage.
To recover the 3D water bodies that do not fulfil the Lambertian hypothesis, 2D vector maps and ortho-photos are used when performing the surface reconstruction.
Furthermore, processing like aerial triangulation, dense image matching, and mesh surface reconstruction were all performed with ContextCapture.
It should be noticed that the entire region of Helsinki is split into tiles, and each of them covers about 250 $ m^2 $~\citep{kalasatamaReport}.
As shown in Figure \ref{fig:overview},  we have selected the central region of Helsinki as the study area, which includes 64 tiles and covers about 4 $km^2$ map area (8 $km^2$ surface area) in total.   

\subsection{Object Classes}
We define the semantic categories for urban meshes by the most common objects in the urban environment with unambiguous geometry and texture appearance.
Moreover, each triangle face is assigned to a label of one of the six semantic classes. 
Ambiguous regions (which account for about 2.6\% of the total mesh surface area), such as shadowed regions or distorted surfaces, are labelled as unclassified (see Figure \ref{fig:ambigious}).
The object classes we consider in the benchmark dataset are: 
\begin{itemize}
	\item \textbf{terrain}: roads, bridges, grass fields, and impervious surfaces;
	\item \textbf{building}: houses,high-rises, monuments, and security booths;
	\item \textbf{high vegetation}: trees, shrubs, and bushes;
	\item \textbf{water}: rivers, sea, and pools;
	\item \textbf{vehicle}: cars, buses, and lorries;  
	\item \textbf{boat}: boats, ships, freighters, and sailboats;
	\item \textbf{unclassified}: incomplete objects like buses and trains, distorted surfaces like tables, tents and facades, construction sites, underground walls.
\end{itemize}

\begin{figure}[!tb]
	\includegraphics[height=0.48\textwidth]{figures/overview_grids/yaxis.png}
	\begin{subfigure}[t]{0.48\textwidth}
		\includegraphics[width=\linewidth]{figures/overview_grids/texture_global_birdsview00.png}
		\includegraphics[width=\linewidth]{figures/overview_grids/xaxis.png}
		\label{fig:textop}
	\end{subfigure}
	\hspace*{\fill}
	\begin{subfigure}[t]{0.48\textwidth}		
		\includegraphics[width=\linewidth]{figures/overview_grids/semantic_global_birdsview00.png}
		\vspace*{-0.78cm}
		\begin{center}
		\includegraphics[width=0.8\linewidth]{figures/semantic_results/semantic_legend2.png}
		\end{center}
		\label{fig:semtop}
	\end{subfigure}
	\vspace*{-0.7cm}
	\caption{Overview of the semantic urban mesh benchmark.
	Left: the texture meshes covering about 4 $km^2$ map area. Right: the ground truth meshes.
	More views of the same scene (with different visualization styles) are shown in Figures \ref{fig:texside} and \ref{fig:semside}.}
	\label{fig:overview}
\end{figure}

\begin{figure}[!tb]
	\centering
	\begin{subfigure}[t]{0.48\textwidth}
		\includegraphics[width=\linewidth]{figures/ambigious/shadow_tex_zoom.png}
		\caption{}
	\end{subfigure}
	\hspace*{\fill}
	\begin{subfigure}[t]{0.48\textwidth}
		\includegraphics[width=\linewidth]{figures/ambigious/shadow_fc_zoom.png}
		\caption{}
	\end{subfigure}
	\begin{subfigure}[t]{0.48\textwidth}
		\includegraphics[width=\linewidth]{figures/ambigious/distort_tex_zoom.png}
		\caption{}
	\end{subfigure}
	\hspace*{\fill}
	\begin{subfigure}[t]{0.48\textwidth}
		\includegraphics[width=\linewidth]{figures/ambigious/distort_fc_zoom.png}
		\caption{}
	\end{subfigure}
	\caption{Ambiguous regions are labelled as unclassified (in black). 
		(a) Shadow region with texture.
		(b) Shadow region with semantic colour.
		(c) Distorted region with texture.
		(d) Distorted region with semantic colour.} 
	\label{fig:ambigious}
\end{figure}


\subsection{Semi-automatic Mesh Annotation}  \label{sec:mesh_annota}
Rather than manually labelling each triangle face of the raw meshes, we design a semi-automatic mesh labelling framework to accelerate the labelling process. Figure~\ref{fig:pipeline} shows the overall pipeline of our labelling workflow.

Given the fact that urban environments consist of a large number of planar regions in the data, we opt to label the data at the segment level instead of individual triangle faces. 
Specifically, we over-segment the input meshes into a set of planar segments. 
These segments can enrich local contextual information for feature extraction and serve as the basic annotation unit to improve annotation efficiency.

\begin{figure}[!tb]
	\centering
	\includegraphics[width=\textwidth]{figures/pipeline/pipeline_L1.png}
	\caption{The pipeline of the labelling workflow.}
	\label{fig:pipeline}
\end{figure}

Instead of randomly choosing a mesh tile as input for annotation and refinement, which is insufficient for manual annotation progress, we favour picking a mesh tile that is more difficult to classify.
Similar to active learning, we first compute the feature diversity (see Equation \ref{eq:fea_div}) to optimally select a mesh tile containing a variety of classes and objects at different scales and complexity.
The feature diversity $F_{m}$ of tile $m$ is computed as
\begin{equation}\label{eq:fea_div}
	F_{m}=\frac{\sum_{i=1}^{N_{f}}\left ( f_i - \bar{f} \right )^{2}}{N_{f}}
\end{equation}
where $f_i$ represents each handcrafted feature which describe in Section \ref{sec:initial_seg}, and $\bar{f}$ is mean value of a $N_{f}$ dimensional feature vector.
To acquire the first ground truth data, we manually annotate the mesh (with segments) that is selected with the highest feature diversity.
Then, we add the first labelled mesh into the training dataset for the supervised classification.
Specifically, we use the segment-based features as input for the classifier, and the output is a pre-labelled mesh dataset.
Next, we use the mesh annotation tool to manually refine the pre-labelled mesh according to the feature diversity.
Finally, the new refined mesh will be added to the training dataset to improve the automatic classification accuracy incrementally.


\subsubsection{Initial Segmentation}\label{sec:initial_seg}

To avoid redundant computations of numerous triangles, we first apply mesh over-segmentation (i.e., linear least-squares fitting of planes) based on region growing on the input data to group triangle faces into homogeneous regions~\citep{lafarge2012creating}.
Such grouped regions are beneficial for computing local contextual features.
We then extract both geometric and radiometric features from those mesh segments as follows: 
\begin{itemize}
	\item[$\bullet$] \textit{Eigen-based features} are computed from the covariance matrix of the triangle vertices with respect to the average centre within each segment, which is beneficial for identifying urban objects with various surface distributions.
	The linearity $= (\lambda_{1} - \lambda_{2}) / \lambda_{1}$, sphericity $= \lambda_{3}/ \lambda_{1}$ and change of curvature $= \lambda_{3} / (\lambda_{1} + \lambda_{2} + \lambda_{3})$ are computed based on the three eigenvalues $\lambda_{1} \geq \lambda_{2} \geq \lambda_{3}\geq 0$.
	The local eigenvectors $\mathbf{n}_{i} $ and the unit normal vector $\mathbf{n}_{z} $ along Z-axis are used to compute the verticality $=1-\left | \mathbf{n}_{i}\cdot \mathbf{n}_{z} \right | $~\citep{hackel2016fast}.
	Note that many eigen-based features have been studied in literature~\citep{hackel2016fast,west2004context,weinmann2013feature}, and some of them were designed for and tested on LiDAR point clouds. 
	\textcolor{ao}{
	These eigen-based features are mostly computed per point based on its spherical neighbourhood, which often contains noise and does not form a surface. 
	Our chosen eigen-based features are defined on a segment representing the surface of a mesh, and thus they can capture non-local geometric properties of an object.
	}
	Additionally, in this work, we have tested all eigen-based features from the literature~\citep{hackel2016fast}, and we only present the ones that are effective for texture meshes.
	\item[$\bullet$] \textit{Elevation} is divided into absolute elevation $z_{a}$, relative elevation $z_{r}$ and multiscale elevations $z_{m}$.
	Where $z_{a}$ is the average elevation of the segment;
	the relative elevation is computed as $z_{r} = z_{a}-z_{r_{min}}$;
	the multiscale elevation~\citep{Verdie2015,Rouhani2017} $z_{m} = \sqrt{\frac{z_{a} - z_{min}}{z_{max} - z_{min}}}$.
	And $z_{r_{min}}$ denotes the lowest elevation of the local largest ground segment computed within a cylindrical neighbourhood with 30 meters radius around the segment centre.
	$z_{min}$ and $z_{max}$ represent the local minimum and maximum elevation values of a cylindrical neighbourhood within the scale of 10 meters, 20 meters, and 40 meters.
	Such large cylindrical neighbourhoods allow to find the local ground considering the resilience to hilly environments, \textcolor{ao}{and the square root ensures that small relative height values (i.e., values smaller than 1 $ m $) get a larger elevation attribute to enlarge elevation differences between small objects and the local ground (e.g., cars against the ground, boats against the water surfaces).}
	More importantly, due to the influence of terrain fluctuations and various scales of urban objects, the elevation of these three categories can complement each other.
	\item[$\bullet$] \textit{Segment area} is computed as $area(S_k) = \sum_{i = 1}^{N} area(f_i) $, where $f_i$ denotes a triangle of the segment $S_k$, and $N$ denotes the total number of triangles in $S_k$.
	\item[$\bullet$] \textit{Triangle density} is defined as $density(S_k) = \frac{N}{area(S_k)} $,  which reveals the object complexity, especially for adaptive urban meshes.
	\item[$\bullet$] \textit{Interior radius of 3D medial axis transform (InMAT)}~\citep{ma20123d,peters2016robust} of a segment $S_k$ is formulated as $r_k = \frac{\sum_{i=1}^{M} r_i}{M}$, where $M$ denotes the total number of triangle vertices of $S_k$, and $r_i$ denotes the interior radius of the shrinking ball that touches the vertex $v_i$ within the segment $S_k$. 
	It is designed to distinguish objects with different scales. 
	\item[$\bullet$] \textit{HSV colour-based features} are derived from the RGB channel of the entire texture map.
	We use the HSV colour space since it can better differentiate different objects than RGB.
	We compute the average colour, the variance of the colour distribution of all pixels within each segment, and we further discretize it into a histogram that consists of 15 bins of the hue channel, five bins of the saturation channel, and five bins of the value channel.
	\item[$\bullet$] \textit{Greenness} $a_{g}$ is used to classify objects that are similar to green vegetation.
	Specifically, it is computed according to the averaged RGB colour of each segment via $a_{g}=G-0.39\cdot R-0.61\cdot B$~\citep{mckinnon2017comparing}. 
\end{itemize}
	All the above features are concatenated into a 44-dimensional feature vector used by our random forest (RF) classifier in the initial segmentation. 

\subsubsection{Annotation Tool for Refinement}

Because of the under-segmentation errors and the imperfect results of the semantic mesh segmentation process, we design a mesh annotation tool (see Figure \ref{fig:annotator}) to manually correct the labelling errors.
Our mesh annotation tool is developed based on the labelling tool of CGAL~\citep{cgal:eb-20b}.

\begin{figure}[!tb]
	\centering
	\includegraphics[width=\textwidth]{figures/annotator/annotator.png}
	\caption{The interface of our annotation tool for 3D texture meshes. }
	\label{fig:annotator}
\end{figure}

As shown in Table \ref{tab:annotation_operation}, it consists of three operation categories: view, selection, and annotation.
The	view operations provide essential functions for the user to manipulate the scene camera, such as translate, rotate, zoom, or set the new pivot for the scene.
In addition, to use textures as a reference for labelling, we map texture and face colour with a certain degree of transparency, and we visualize the segment border to differentiate each segment. 

\begin{table}[!tb]
	\centering
	\noindent\adjustbox{max width=0.8\textwidth}
	{
		\begin{threeparttable}
			\centering
			\begin{tabular}{ccc}
				\toprule
				Categories & Operations & Objects \\
				\midrule
				\multirow{4}[2]{*}{View} & Translate & Camera \\
				& Rotate & Camera \\
				& Zoom in / out & Camera \\
				& Set pivot & Camera \\
				\midrule
				\multirow{6}[2]{*}{Selection} & Multi-selection / Lasso & Triangles / Segments \\
				& Expand / Reduce & Triangles / Segments \\
				& Semantic selection & Segments \\
				& Split region & Segments \\
				& Planar region extraction & Triangles \\
				& Split mesh & Triangles \\
				\midrule
				\multirow{3}[2]{*}{Annotation} & Probability slider & Segments \\
				& Segment area slider & Segments \\
				& Progress bar & Triangles \\
				& Switch semantic view & Triangles \\ 
				& Labelling & Triangles / Segments \\
				\bottomrule
			\end{tabular}%
		\end{threeparttable}
	}
	\caption{Basic operations in our annotation tool.} 
	\label{tab:annotation_operation}%
\end{table}%


The	selection operations allow the user to select or deselect either triangle faces (see Figure \ref{fig:tri_sel}) or segments (see Figure \ref{fig:seg_sel}) freely via a brush or a lasso.
Specifically, the face selection operation is used to fix the under-segmentation errors and generate new segments, and the segment selection operation is to fix incorrect segment labels.

\begin{figure}[!tb]
	\centering
	\begin{subfigure}[t]{0.32\textwidth}
		\includegraphics[width=\linewidth]{figures/pipeline/tri_select_a.png}
		\caption{}
	\end{subfigure}
	\hspace*{\fill}
	\begin{subfigure}[t]{0.32\textwidth}
		\includegraphics[width=\linewidth]{figures/pipeline/tri_select_b.png}
		\caption{}
	\end{subfigure}
	\hspace*{\fill}
	\begin{subfigure}[t]{0.32\textwidth}
		\includegraphics[width=\linewidth]{figures/pipeline/tri_select_c.png}
		\caption{}
	\end{subfigure}
	\caption{An example of labelling by selecting triangles using the lasso tool (blue edges: segment boundaries). 
		(a) Before selection.
		(b) Lasso selection result (in red).
		(c) The correct label has been assigned to the selected region. 
		In this example, the label of the selected region has been changed from `ground' to `vehicle'.
	} 
	\label{fig:tri_sel}
\end{figure}


\begin{figure}[!tb]
	\centering
	\begin{subfigure}[t]{0.32\textwidth}
		\includegraphics[width=\linewidth]{figures/pipeline/seg_select_a.png}
		\caption{}
	\end{subfigure}
	\hspace*{\fill}
	\begin{subfigure}[t]{0.32\textwidth}
		\includegraphics[width=\linewidth]{figures/pipeline/seg_select_b.png}
		\caption{}
	\end{subfigure}
	\hspace*{\fill}
	\begin{subfigure}[t]{0.32\textwidth}
		\includegraphics[width=\linewidth]{figures/pipeline/seg_select_c.png}
		\caption{}
	\end{subfigure}
	\caption{An example of segment labelling. 
		(a) Part of a wall of the building was previously labelled as `high vegetation' (in green).
		(b) Segment selection result (in red).
		(c) The label of the selected segment has been corrected with the new label `building'.
	}
	\label{fig:seg_sel}
\end{figure}

We also allow the user to edit the selection of each individual segment with splitting functions (see Figure \ref{fig:pnp_func}) and automatic extraction of the most planar region (see Figure \ref{fig:seg_func}). 
As for splitting, we first detect the potential planar and non-planar segments marked by user strokes, and then the non-planar one is split according to the vertex-to-plane distance.
It allows generating candidate non-planar regions (with respect to the detected planar segment) for the user to edit, and
it is useful to split a segment that covers large non-planar regions or contains more than one dominant planar area.
To extract the most planar region, we apply the region growing algorithm~\citep{lafarge2012creating} within the selected segment to automatically generate the candidate triangle faces with user-defined thresholds (i.e., the maximum distance to the plane, the maximum accepted angle, and the minimum region size).
Such an operation allows the user to filter out some small bumpy regions of the selected segment.

\begin{figure}[!tb]
	\centering
	\begin{subfigure}[t]{0.48\textwidth}
		\includegraphics[width=\linewidth]{figures/annotator/pnp_pipeline1.png}
		\caption{}
	\end{subfigure}
	\hspace*{\fill}
	\begin{subfigure}[t]{0.48\textwidth}
		\includegraphics[width=\linewidth]{figures/annotator/pnp_pipeline2.png}
		\caption{}
	\end{subfigure}
	\caption{An example splitting planar and non-planar regions. 
		(a) The user draws a stroke (in red) across the border of the non-planar segment and the planar segment. 
		(b) The detected non-planar segment has been split into two parts (i.e., a non-planar region shown in red and a planar segment shown in green).
	} 
	\label{fig:pnp_func}
\end{figure}

\begin{figure}[!tb]
	\centering
	\begin{subfigure}[t]{0.48\textwidth}
		\includegraphics[width=\linewidth]{figures/annotator/planar_split_pipeline1.png}
		\caption{}
	\end{subfigure}
	\hspace*{\fill}
	\begin{subfigure}[t]{0.48\textwidth}
		\includegraphics[width=\linewidth]{figures/annotator/planar_split_pipeline3.png}
		\caption{}
	\end{subfigure}
	\caption{Editing an individual segment. 
		(a) A segment is selected (highlighted in green) for splitting. 
		(b) Automatic extraction of the most planar region (shown in red) within the selected segment according to user-defined thresholds.} 
	\label{fig:seg_func}
\end{figure}

Besides, probability and area-based sliders and a progress bar are provided in the annotation panel to improve annotation efficiency and experience, respectively. 
Specifically, the probability slider is introduced for the user to visually inspect the segments that are most likely misclassified.
Moreover, the user can further use it to inspect a specific class by switching the view to highlight a specific semantic class.
The segment area slider is used to identify isolated tiny segments, which commonly appear as errors.
The progress bar is used to indicate the estimated labelling progress during the annotation.
After performing the selection, the user can easily assign the corresponding label to the selected area.

%auto-ignore

% \section{Method} \label{sec:method}

% In this section, we describe our two-stage model architecture. Figure~\ref{fig:pipeline} shows the overall model structure. We will first discuss motivations of our two-stage design, then describe our model architecture and training details.

%In this section, we first discuss motivations of our two-stage model architecture, then describe our model architecture and training details. Figure~\ref{fig:pipeline} shows the overall model structure. 

\section{High-Resolution Wire Segmentation}
\label{sec:segmentation}
Wires appear visually different from common objects -- being thin, long, sparse and oftentimes partially occluded. 
We find the following two design choices crucial to building an effective wire segmentation system: 1) having a two stage framework so that coarse prediction from global context guides precise segmentation from local patches and 2) maximally preserving and augmenting image features and annotations of wires throughout the pipeline.  
% We thus propose a two-stage coarse-to-fine pipeline, in which we inject feature augmentation and architectural choices to maximally preserve the wire features throughout.

\subsection{The Two-stage Coarse to Fine Model}
%\yqe{
Figure \ref{fig:pipeline} shows the two-stage segmentation pipeline. It consists of a coarse and a fine module, which share an encoder $E$ and have their own decoder $D_C$ and $D_F$. Intuitively, the coarse module aims to capture the global contextual information from the entire image and highlight the image regions possibly containing wires. Conditioned on the predictions from the coarse module, the fine module achieves high-resolution wire segmentation by only looking at local patches likely containing wires.

Given a high-resolution image $I_\textrm{glo}$, we first bilinearly downsample it to $I_\textrm{glo}^{ds}$ with a fixed size $p\times p$ and feed it into the coarse module. The module predicts the global probability map $P_\textrm{glo} = \textrm{SoftMax}(D_C(E(I_\textrm{glo}^{ds})))$ containing the activation of the wire regions.% with rich global contextual information.

For each patch $I_\textrm{loc}$ of size $p \times p$ cropped from the full-resolution image $I_\textrm{glo}$, and the corresponding conditional probability map $P_\textrm{con}$ cropped from $P_\textrm{glo}$, we predict the local probability $P_\textrm{loc} = \textrm{SoftMax}(D_F(E(I_\textrm{loc}, P_\textrm{con})))$. 
%$M$ is a binary location mask indicating the patch region of $I_{loc}$ in $I_{glo}$ with 1, and elsewhere 0.
Note that $E$ is shared between the coarse and the fine module, thus it should take inputs with the same number of channels. Therefore, for the coarse module, we concatenate an additional zero channel with the input image to make the channel number consistent.
% We formulate our problem as a semantic segmentation task, so we can regard it as a pixel-wise classification problem. Each image pixel is labeled as 0 or 1 indicating background or wire pixel.

% We apply Cross Entropy (CE) loss to both the global $P_\textrm{glo} = \textrm{SoftMax}(Z_\textrm{glo})$ and local probability map $P_\textrm{loc} = \textrm{SoftMax}(Z_\textrm{loc})$, comparing with their ground truth annotations $G_\textrm{glo}$ and $G_\textrm{loc}$.

We apply Cross Entropy (CE) loss to both the global $P_\textrm{glo}$ and local probability map $P_\textrm{loc}$, comparing with their ground truth annotations $G_\textrm{glo}$ and $G_\textrm{loc}$.

\vspace{-2mm}
\begin{equation}
\begin{aligned}
%    \mathcal{L}_{glo} &= -\sum_{c\in C}\log\ Softmax(Z_{glo})_c \\
    %\mathcal{L}_{loc} &= -\sum_{c\in C}\log\ Softmax(Z_{loc})_c \\
    \mathcal{L}_\textrm{glo} &= CE (P_\textrm{glo}, G_\textrm{glo}) \\
    \mathcal{L}_\textrm{loc} &= CE (P_\textrm{loc}, G_\textrm{loc}) \\
\end{aligned}
\end{equation}
%Both $Z_{glo}$ and $Z_{loc}$ are trained by computing Cross Entropy losses of the logit maps after softmax against their respective ground truth wire annotations. 
The final loss $\mathcal{L}$ is the sum of the two:

\vspace{-3mm}
\begin{equation}
\begin{aligned}
    \mathcal{L} &= \mathcal{L}_ \textrm{glo} + \lambda \mathcal{L}_ \textrm{loc},
\end{aligned}
\end{equation}

%Given a high-resolution image $I$, the coarse module of our two-stage model aims to recognize semantics of the entire image and pick up wire regions at a coarse level. Looking at the entire image allows the coarse module to capture more contextual information. To fit the large images into a GPU for training and inference, images are first bilinearly downsampled, then fed into the coarse module to predict a global logit map $Z_{glo}$. We use $Z_{glo}$ as a conditional input to guide the fine module in the next stage.

%The fine module is conditioned on the output from the coarse module. This module takes in a local image patch $I_{loc}$ of the whole image at full resolution, the global logit map $Z_{glo}$ from the coarse module, and a binary location mask $M$ that sets the patch relative to the full image to $1$ and other regions to $0$. The fine module then predicts the local wire logit map $Z_{loc}$. Empirically, we find that concatenating the entire global logit map rather than cropping the logit map at the location of the image patch yields slightly improved results.

%The designs of the coarse and fine modules are conceptually the same as those in GLNet~\cite{glnet} and MagNet~\cite{magnet}, where a global network is trained on entire downsampled images and a local network is trained on higher-resolution image patches conditioned on some global features. However, unlike GLNet, where intermediate features are shared between the global and local branch bidirectionally, we opt for a simpler late fusion by concatenating the logit map directly to the fine module. We also only use two stages instead of up to four stage as done in MagNet, since a single fine module is already sufficient at refining annotations that are only several pixels thick, and additional stages can drastically increase inference time.

where we 
%empirically 
set $\lambda = 1$ for training. 
% During training, 
% we apply data augmentation including random scaling, horizontal flipping and photometric distortion to the full-resolution image to obtain the input $I_{glo}$. After that, 
% we randomly crop one local patch $I_{loc}$ with patch size $p=512$ from the augmented image. 
Similar to Focal loss~\cite{focal} and Online Hard Example Mining~\cite{ohem}, we balance the wire and background samples in the training set by selecting patches that contain at least 1\% of wire pixels. %We find that this simple constrained cropping approach yields better performances than other well known balancing methods, including Focal loss~\cite{focal} and Online Hard Example Mining~\cite{ohem}.

%The global image $I_{glo}$ is generated by simply downsampling this augmented image. We then generate the local image patch $I_{loc}$ by randomly cropping a 512$\times$512 window from the augmented image that contains at least 1\% wire pixels. This helps balance between wire and background pixels. We find that this simple constrained cropping approach yields better performances than other well known balancing methods, including Focal loss~\cite{focal} and Online Hard Example Mining~\cite{ohem}.

%We share the same feature extractor network between both the coarse and fine module, but train separate feature decoders. 
%To account for the additional inputs to the fine module, we expand the input channels of the feature extractor from 3 to 5. The last two channels are set to 0 when passing an image through the coarse module, and set to the global logit map $Z_{glo}$ and the binary location map $M$ for the fine module. 
To perform inference, we first feed the downsampled image to the coarse module, which is the same as training. 
% We then compute the global wire segmentation map by taking the $\mathrm{argmax}$ over the two classes for each pixel. 
Local inference is done by running a sliding window over the entire image, where the patch is sampled only when there is at least some percentage of wire pixels (determined by~$\alpha$). This brings two advantages: First, we save computation time in regions where there are no wires. Second, the local fine module can leverage the information from the global branch for better inference quality.

% The design of our two-stage pipeline is conceptually similar to GLNet~\cite{glnet} and MagNet~\cite{magnet}, where a global network is trained on downsampled images and a local network is trained on image patches at the original resolution. However, unlike GLNet whose intermediate features are shared between the global and local branch bidirectionally, we opt for a simpler late fusion by concatenating the probability map directly to the fine module. We only use two stages instead of four stages in MagNet. Our pipeline demonstrates its efficiency and simplicity while coping with wire-like objects which are only several pixels thick, and avoids high computational demands from some previous works.

\subsection{Wire Feature Preservation}
\label{sec:wire_feature_preserve}
As wires are thin and sparse, applying downsampling to the input images may make the wire features vanish entirely. To mitigate this challenge, we propose a simple feature augmentation technique by taking the min and max pixel luminance values of the input image over a local window. Either the local min or the max value makes the wire pixels more visually apparent. In practice, we concatenate the min- and max-filtered luminance channels to the RGB image and condition map, resulting in 6 total channels as input. We name this component MinMax.
%to form a 5-channel input.
% Unlike common object segmentation, with prominent textures and colors, wires, especially outdoor cables, tend to either be shadowed or reflective under sunlight. As a result, it is common to see either very bright or very dark wires in images. An intuitive approach is to find the min and max of an image before downsampling, and preserve this feature as input to the model.
% \todo{show an example patch of luminance min max filter:}

%\subsection{Motivation}
% We design a two-stage coarse-to-fine pipeline for wire-like object segmentation. This model structure 
% we divide our model into two major components -- a coarse module and a fine module. Our two-stage model design 
% is motivated by two observations. First, wires in our dataset are extremely long and thin, many spanning over several thousand pixels but only occupy several pixels across. Limited by the memory size of modern GPUs, we cannot simply pass the entire image at full-resolution to a model for inference. As a result, two separate modules are required for high-resolution inference, where the coarse module captures the entire wire and its surrounding contextual information at a lower resolution, and the fine module captures detailed textures of the wire at the original resolution.

% Second, we observe that wire pixels are sparse, where a typical high-resolution image contains a small percentages of wire pixels. This means that we can also use the coarse module to guide the fine module on what regions to capture the details. The fine module can save computation by only predicting segmentation masks where there are wires.

% \subsection{Global Condition}
% Many high-resolution segmentation methods share the idea of using global-local refinement module. But each method differs in details that are tailored to their targeted applications. 
% MagNet~\cite{magnet} trains a single model by randomly sampling global/local patches, and then trains a refinement module that takes as input the two predictions. This neglects the causal relationship between global and local features. The refinement module does not take into account any image feature. The recently proposed ISDNet~\cite{isdnet} use a shallow network and takes the entire image as input. This is proven effective in some high-resolution segmentation datasets such as DeepGlobe~\cite{deepglobe}, where resolution is fixed and manageable (5k$\times$5k). But the model is unable to scale to larger images. The shallow model, while being efficient, is limited in model capacity, especially with sparse labels.
% the fact that they take the entire image as input strictly limits their model to use lightweight networks as the shallow branch (i.e. STDC~\cite{stdc} with output stride=8), which has limited performance in cases where the target label is also small. 
% In fact, we were unable to train ISDNet with a stronger shallow network to produce high-resolution outputs.
% ~\cite{learning_downsample} attempts to learn a downsampling network, which we tried but found was detrimental for our thin and sparse wire masks. 
% As a result, 

Besides feature augmentations, we also adapt the architecture to maximally preserve the sparse wire annotations.
% Different from existing methods, we find it crucial to keep the image at a close-to-original resolution to maximally preserve the image features of wires. In addition, we find it similarly important to preserve the sparse wire annotations. 
% We use the global branch as a conditioning network for training both coarse and high resolution segmentation tasks rather than a simple coarse branch that is only learned for the coarse resolution. 
We propose to use ``overprediction" and achieve this by using max-pool downsampling on the coarse labels during training, which preserves activation throughout the coarse branch. We name this component MaxPool. We provide ablation studies for these components in Section~\ref{sec:results}.

\vspace{-5mm}
\section{High-Resolution Wire Inpainting}
\label{sec:inpainting}
Given a full-resolution wire segmentation mask estimated by our wire segmentation model, we propose an inpainting pipeline to remove and fill in the wire regions. Our approach addresses two major challenges in wire inpainting. First, recent state-of-the-art deep inpainting methods do not handle arbitrary resolution images, which is critical for high-resolution wire removal. Second, deep inpainting methods often suffer from color inconsistency when the background has uniform (or slowly varying) colors. This issue is particularly significant for wires, as they are often in front of uniform backgrounds, such as the sky or building facades. The commonly used reconstruction loss, such as L1, is not sensitive to color inconsistency, which further exacerbates this issue.

We thus revisit the efficient deep inpainting method LaMa \cite{suvorov2022resolution}. Compared with other inpainting models, LaMa has two major advantages. First, it contains the Fourier convolutional layers which enables an efficient and high-quality structural completion. This helps complete building facades and other man-made structures with fewer artifacts. Second, its high inference efficiency makes a tile-based inference approach possible for high resolution images. 

% To tailor LaMa for wire removal, we fix the input size of the model to $512\times512$ and train the model on an augmented dataset by including synthetic wire masks and cropped patches from full-resolution images. 
To address color inconsistency, we propose a novel ``onion-peel" color adjustment module. Specifically, we compute the mean of the RGB channels within the onion-peel regions $M_o = D(M, d) - M$ of the wire mask $M$, where $D$ is the binary dilation operator, and $d$ is the kernel size. The color difference for each channel $c \in {R, G, B}$ becomes $\textrm{Bias}_c = \mathbb{E}[M_o (x_c - y_c)]$, where $x$ is the input image, and $y$ is the output from the inpainting network. The final output of the inpainting model is: $\hat{y_c} = y_c + \textrm{Bias}_{c}$. Loss functions are then applied to $\hat{y_c}$ to achieve color consistency while compositing the final result $y_{out} = (1 - M) \odot x + M \odot \hat{y}$. 

% While running inference on full-resolution images, we apply a tile-based approach, by fixing the window size at $512\times 512$ with an $32$-pixel overlap. 
% This makes the model consistent in training and testing settings, and gives good textural and structural details for local regions.
\begin{table}[t!]
\centering
\caption{Voice conversion \& F0 manipulation results. MOS results are reported with 95\% confidence interval. VDE, and FFE are reported for F0 manipulation while PER, WER, EER, and MOS are reported for voice conversion. Notice, for VDE, and FFE higher is the better since F0 was flattened.}
\label{tab:conv}

\resizebox{1\columnwidth}{!}{
\begin{tabular}{c@{~} | c@{~} | c@{~}c@{~} | c@{~} | c@{~} ||  c@{~}c@{~} }
\toprule
\multirow{2}{*}{Dataset} & \multirow{2}{*}{Method} & \multicolumn{4}{c||}{Voice Conversion} & \multicolumn{2}{c}{F0 Manipulation} \\
\cmidrule{3-8}
& & PER~$\downarrow$ & WER~$\downarrow$ & EER~$\downarrow$ & MOS~$\uparrow$ & VDE~$\uparrow$ & FFE~$\uparrow$ \\
\midrule
VCTK & GT  & 17.16 & 4.32 & 3.25 & 4.11$\pm$0.29 & -- & -- \\
\midrule 
\multirow{3}{*}{LJ}
% & ASR-TTS   & 50.74  & --     & 66.08 & 32.96 & 1.46 \\
& CPC       & 22.22 	& 16.11 		& 0.46 		& 3.57$\pm$0.15 		& \bf 46.68 & \bf 48.71\\
& HuBERT    & \bf 19.09 & \bf 12.23 & \bf 0.31  & \bf 3.71$\pm$0.24 & 39.20 		& 48.42\\
& VQ-VAE    & 40.88 	& 36.96 		& 9.65 		& 2.90$\pm$0.17 		& 10.54 	& 12.08 \\
\midrule 
\multirow{3}{*}{VCTK} 
% & ASR-TTS   & 68.88  & --    & 41.77 & 13.55 & 6.48 \\
& CPC       &  23.58 		& 15.98 		& \bf 4.83  &  3.42 $\pm$ 0.24 		& \bf 25.29 & \bf 26.97 \\
& HuBERT    &  \bf 20.85 	& \bf 12.72 & 6.01  		& \bf  3.58 $\pm$ 0.28 	& 23.46 	& 26.67 \\
& VQ-VAE    & 36.88  		& 29.44 		& 11.56 		& 3.08 $\pm$ 0.34 		& 7.03  	& 7.80  \\
\bottomrule
\end{tabular}}
\vspace{-0.4cm}
\end{table}

\vspace{-0.1cm}
\section{Results}
\vspace{-0.1cm}
Our results cover
% We report results for 
three different settings: (i) speech reconstruction experiments; (ii) speaker conversion and F0 manipulation; (iii) bitrate analysis with subjective tests for speech codec evaluation. We employ two datasets: LJ~\cite{ljspeech17} single speaker dataset and VCTK~\cite{vctk} multi-speaker dataset. All datasets were resampled to a 16kHz sample rate.

% \paragraph*{Implementation Details.}
% \smallskip
\noindent{\bf Implementation Details\quad} 
\label{sec:impl}
We follow the same setup as in~\cite{lakhotia2021generative}. For CPC, we used the model from~\cite{Riviere2020}, which was trained on a ``clean'' 6k hour sub-sample of the LibriLight dataset~\cite{Kahn2020,Riviere2020}. We extract a downsampled representation from an intermediate layer with a 256-dimensional embedding and a hop size of 160 audio samples. For HuBERT we used a \textsc{Base} 12 transformer-layer model trained for two iterations~\cite{hsu2020hubert} on 960 hours of LibriSpeech corpus~\cite{Panayotov2015}. 
% This model encodes every 320 raw audio samples into a 768-dimensional vector. 
This model downsamples the raw audio $\times320$ into a sequence of 768-dimensional vectors. Similarly to~\cite{lakhotia2021generative}, activations were extracted from the sixth layer.

%CPC: We use a dictionary of 100 units, leading to a bitrate of 700bps.
%HuBERT: A dictionary of 100 units is used, leading to a bitrate of 350bps. 
%VQVE: The VQ-VAE discrete code operates at a bitrate of 800bps.
% For both CPC and HuBERT, the k-means algorithm is applied to convert continuous frames to discrete codes, using the LibriSpeech clean-100h~\cite{Panayotov2015} dataset. 
For CPC and HuBERT, the k-means algorithm is trained on LibriSpeech clean-100h~\cite{Panayotov2015} dataset to convert continuous frames to discrete codes. We quantize both learned representations with $K=100$ centroids. Leading to a bitrate of 700bps for CPC and 350bps for HuBERT.

% VQ-VAE
Similarly to CPC models, we trained the VQ-VAE content encoder model on the ``clean'' 6K hours subset from the LibriLight dataset. We use an encoder operating on the raw signal to extract discrete units, similar to~\cite{jukebox}. In addition, ``random restarts'' were performed when the mean usage of a codebook vector fell below a predetermined threshold. Finally, we used HiFiGAN (architecture and objective) as the decoder instead of a simple convolutional decoder, as it improved the overall audio quality. This model encodes the raw audio into a sequence of discrete tokens from 256 possible tokens~\cite{garbacea2019low} with a hop size of 160 raw audio samples. The VQ-VAE discrete code operates at a bitrate of 800bps. We additionally experimented with 100 discrete units for VQ-VAE, however results were the best for 256. This finding is consistent with~\cite{garbacea2019low}.

% verification model
The speaker verification network uses the architecture proposed in~\cite{heigold2016end}. It was trained on the VoxCeleb2~\cite{voxceleb2} dataset, achieving a 7.4\% Equal Error Rate (EER) for speaker verification on the test split of the VoxCeleb1~\cite{Nagrani17} dataset.

% pitch
Only a single F0 representation is considered across all evaluated models, trained on the VCTK dataset.
% The F0 is extracted from the raw audio using YAAPT~\cite{yaapt} algorithm, using a window size of 20ms and a 5ms hop. 
The F0 is extracted from the raw audio using a window size of 20ms and a 5ms hop. 
As a result, the F0 sequence is sampled at 200Hz. 
% We apply the quantization described at Sec.~\ref{sec:method}, using a pitch codebook of $K'=20$ tokens and an encoder that downsamples the pitch by $\times16$. 
The quantization described at Sec.~\ref{sec:method}, is applied using an F0 codebook of $K'=20$ tokens and an encoder that downsamples the signal by $\times16$. Hence, the discrete F0 representation is sampled at 12.5Hz, leading to a bitrate of 65bps. The final bitrate of the evaluated codecs is the sum of the pitch code bitrate with the content code bitrate.

% \paragraph*{Evaluation Metrics}
% \smallskip
\noindent{\bf Evaluation Metrics\quad} 
We consider both subjective and objective evaluation metrics. For subjective tests, we report the Mean Opinion Scores (MOS). In which human evaluators rate the naturalness of audio samples on a scale of 1--5. Each experiment, included 50 randomly selected samples rated by 30 raters. For objective evaluation, we consider: (i) Equal Error Rate~(EER) as an automatic speaker verification metric obtained using a pre-trained speaker verification network. We report EER between test utterances and enrolled speakers; (ii) Voicing Decision Error (VDE)~\cite{nakatani2008method}, which measures the portion of frames with voicing decision error; (iii) F0 Frame Error (FFE)~\cite{chu2009reducing}, measures the percentage of frames that contain a deviation of more than 20\% in pitch value or have a voicing decision error; (iv) Word Error Rate (WER) and Phoneme Error Rate (PER), proxy metrics to the intelligibility of the generated audio. We used a pre-trained ASR network~\cite{baevski2020wav2vec} on both reconstructed and converted samples to calculate both metrics. %To generate target phonemes, the g2p-en~\cite{g2pE2019} Grapheme2Phoneme module was used.

% \vspace{-0.1cm}
% \smallskip
\noindent{\bf Reconstruction \& Conversion}
% \vspace{-0.1cm}
We start by reporting the reconstruction performance. Results are summarized in Table~\ref{tab:recon}. When considering the intelligibility of the reconstructed signal HuBERT reaches the lowest PER and WER scores across all models, where both CPC and HuBERT are superior to VQ-VAE. However, when considering F0 reconstruction VQ-VAE outperforms both HuBERT and CPC by a significant margin. This results are somewhat intuitive, bearing in mind VQ-VAE objective is to fully reconstruct the input signal. In terms of subjective evaluation, all models reach similar MOS scores, with one exception of CPC on LJ. 

%Notice, since the same F0 units are used for each method, this result implies the VQ-VAE units contain some information about the F0 of the signal, enabling better reconstruction. Regarding speaker information, the CPC gets the lowest EER. 

To better evaluate the disentanglement properties of each method with respect to speaker identity and F0, we conducted an additional set of experiments aiming at speaker conversion and F0 manipulation. For voice conversion, we converted each test utterance into five random target speakers. Next, we employed a speaker verification network, which extracts \emph{d-vector} representation to evaluate speaker-converted utterances' similarity to real speaker utterances (low error-rate indicates good conversion), providing measurement to the speaker identity's disentanglement from the evaluated coding method. The error-rate is reported between converted test utterances and enrolled speakers. For the LJ speech single speaker dataset, we converted samples from the VCTK dataset to the single speaker and enrolled all VCTK speakers together with the single speaker. Results are summarized in Table~\ref{tab:conv} (left). Unlike resynthesis results, on voice conversion CPC and HuBERT outperform VQ-VAE on both LJ and VCTK datasets, indicating VQ-VAE contains more information about the speaker in the encoded units, hence producing more artifacts. Notice, this also affects WER, PER, and the overall subjective quality (MOS). 

Next, to evaluate the presence of F0 in the discrete units, we flattened the F0 units before synthesizing the signal and calculated VDE and FFE with respect to the original F0 values. F0 flattening was done by setting the speakers' mean F0 value across all voiced frames. In this experiment, we expected units that contain F0 information to be better at F0 reconstruction over disentangled units. Results are summarized in Table~\ref{tab:conv} (right). Notice VQ-VAE can still reconstruct the F0 almost at the same level as when using the original F0 as conditioning (5.2 vs 7.03, and 5.59 vs 7.8), in contrast to CPC and HuBERT.

\begin{figure}[t!]
\centering
\includegraphics[width=0.65\columnwidth, trim={50 20 70 20}]{figures/codec_2.pdf}
% \caption{MUSHRA subjective listening test results as a function of bitrate per second for various methods. Purple dots denote the baseline methods, and green dots the proposed SSL based method.} 
\caption{MUSHRA subjective quality results as a function of bitrate per second. Purple dots denote the baseline methods, and green dots the proposed SSL based method.} 
\label{fig:codec}
\vspace{-0.5cm}
\end{figure}

% \vspace{-0.1cm}
% \smallskip
\noindent{\bf Speech Codec}
Our final experiment evaluates the obtained speech units as a low bitrate speech codec. 
% Therefore, we evaluate how the performance varies as a function of the number of discrete units. Changing the number of units is equivalent to varying the bitrate of the encoded signal. 
We use a subjective MUSHRA-type listening test~\cite{series2014method} to measure the perceived quality of the proposed speech codec with regard to its bitrate constraints. In MUSHRA evaluations, listeners are presented with a labeled uncompressed signal for reference, a set of test samples to rate, a copy of the uncompressed reference, and a low-quality anchor. Listeners are asked to rate each test utterance and the copy of the uncompressed reference with respect to the labeled reference in a scale of 1-100.

The experiment is performed on the VCTK dataset~\cite{vctk}. For evaluation, we used 20 utterances from 5 speakers. The set of speakers in the test data is disjoint with those in the training data. For this experiment, HuBERT models with 50, 100, and 200 units were trained as described in Sec.~\ref{sec:impl}. For comparison, we included other speech codecs in our evaluation: Opus~\cite{valin2012definition} wideband at 9 kbps VBR, Codec2~\cite{rowe2011codec} at 2.4 kbps and LPCNet~\cite{valin2019real} operating at 1.6 kbps. The LPCNet model was trained from scratch on the VCTK dataset following the experimental setup in~\cite{valin2019real}. The VQ-VAE model employs the HiFiGAN decoder trained on the LibriLight dataset to match the amount of data reported in~\cite{garbacea2019low}. We compressed the anchor sample with Speex~\cite{valin2016speex} at 4 kbps as a low anchor. Fig.~\ref{fig:codec} depicts the results. HuBERT with 50 units reaches the best MUSHRA score while its bitrate is only 365bps, which is significantly lower than the baseline methods.
In this paper, 2D and 3D CNN models were used to generate pelvic sCTs from T1-weighted MR images. Our sCT generation methods were fully automated, requiring no deformable registration or manual segmentation of bone tissues. As shown in Figure~\ref{fig3}, the 2D and 3D CNN models generated high quality sCTs. MAE curves shown in Figure~\ref{fig4} indicated that both models could precisely estimate soft-tissue HU values but had difficulty in reproducing air and high-density bone tissues. 

The MAEs within the body contour across all patients were 40.5 $\pm$ 5.4 HU and 37.6 $\pm$ 5.1 HU for the 2D and 3D models, respectively. The time required for generating a pelvic sCT using our CNN models was about 5.5 s. Our MAE results are comparable to previous studies. Kim $et \ al.$\cite{RN41} presented a voxel-based weighted summation method that produced an MAE of 74.3 $\pm$ 3.9 HU. However, manual contouring of bone tissues required for this method can be tedious and time-consuming. An MAE of 40.5 $\pm$ 8.2 HU was achieved by Dowling $et \ al.$\cite{RN11} using an average MRI-CT atlas from 38 patients. Andreasen $et \ al.$\cite{RN42} reported an MAE of 54 $\pm$ 8 HU using an atlas-based method with pattern recognition, and its prediction time was about 20.8 min. Another random forest model proposed by Andreasen $et \ al.$\cite{RN43} generated sCTs with an MAE of 58 $pm$ 9 HU. A hybrid method suggested by Siversson $et \ al.$ \cite{RN45} obtained an MAE of 36.5 $\pm$ 4.1 HU when ignoring errors introduced by gas cavities. This hybrid method was implemented in the cloud-based commercial software MriPlanner (Spectronic Medical AB, Helsingborg, Sweden), which required 50 to 80 min to generate a sCT.\cite{RN45} The patch-based 3D context-aware generative adversarial network presented by Nie $et \ al.$\cite{RN26} achieved an MAE of 39.0 $\pm$ 4.6 HU. 

Our CNN models reproduced low-density bone as shown in Figure ~\ref{fig4}. The bone-region DSCs were 0.81 $\pm$ 0.04 and 0.82 $\pm$ 0.04 from the 2D and 3D models, respectively. These results are comparable to reported DSC results of 0.79 $\pm$ 0.12\cite{RN10} and 0.91$\pm$0.03{\cite{RN11}}, where the authors compared bone contours manually drawn on the sCT and CT.

It was feasible to train the proposed 3D model with 16 image volumes from scratch. Results of the Wilcoxon signed-rank tests shown in Table~\ref{tab1} demonstrated a statistically significant improvement in overall MAE, bone DSC, and bone precision of the 3D model compared to the 2D model. However, as shown in Figure~\ref{fig4}, the 2D model seemed to perform better in estimating the high-density bone HU values. It should be noted that smaller overall MAEs do not guarantee improved sCT dose calculation and patient positioning performance. While the models performed well, we will continue to acquire more patient data to potentially improve model accuracy and further test model differences.

As this was a retrospective study, the MR image voxel sizes were not matched, resulting in different voxel intensities between images. This may have affected the sCT generation accuracy although we applied intensity normalization. A potential study could examine how voxel size variations affects sCT estimation. 

The proposed 3D model can be implemented on a 12 GB GPU to process volumetric images with dimensions of 256 $\times$ 256 $\times$ 30. More GPU memory would be required to process higher resolution 3D images. Considering the limited access to multi-GPU systems, a 3D architecture with fewer convolutional layers could be considered to deal with higher resolutions. However, the performance could be affected by the reduced parameters and smaller receptive fields of the less complex model. Another approach would be to extract 30-slice sub-volumes from CT and MR images for training the 3D model. The sCT could then be generated by averaging 30-slice sCT sub-volumes produced by the model. 

A number of techniques could be investigated for improving model performance.  Nie $et \ al.$\cite{RN26} showed that introducing an additional adversarial discriminator improved overall sCT quality. The same approach could be adapted in our proposed 2D and 3D CNN models.  Non-rigid deformation\cite{RN44} could also be applied to both CT and MR images in the process of the on-the-fly data augmentation to produce more training pairs. Multiple MR images acquired with different sequences could be fed into models to provide more information for distinguishing different tissues. Multi-GPU systems with more memory would enable the exploration of larger batch sizes for training CNN models, which could reduce variances in gradient estimation and accelerate the training. 



\begin{comment}
\begin{figure}
\includegraphics[width=\linewidth]{figs/beyond_tss_lesion.pdf}
\caption[]{End-to-End runtime lesion study of the entire MNIST dataset and the FMA featurized music dataset. Each of DROP's contributions provides a runtime improvement.}
\label{fig:beyond_lesion}
\end{figure}
\end{comment}



\section{Conclusion}
\label{sec:conclusion}

Advanced data analytics techniques must scale to rising data volumes. 
DR techniques offer a powerful toolkit when processing these datasets, with PCA frequently outperforming popular techniques in exchange for high computational cost. 
In response, we propose DROP, a new dimensionality reduction optimizer. 
DROP combines progressive sampling, progress estimation, and online aggregation to identify high quality low dimensional bases via PCA without processing the entire dataset by balancing the runtime of downstream tasks and achieved dimensionality. 
Thus, DROP provides a first step in bridging the gap between quality and efficiency in end-to-end DR for downstream \red{analytics}. 

%We revisit canonical operators for time series dimensionality reduction and the measurement study of~\cite{keogh-study}, and show that PCA is more effective than popular alternatives in the data mining literature often by a margin of over $2\times$ on average on gold-standard time series benchmark data sets with respect to output data dimension. More surprisingly, we empirically demonstrate that a small number of samples are sufficient to accurately characterize directions of maximum variance and obtain a high-quality low-dimensional transformation.




%%%%%%%%% REFERENCES
{\small
\bibliographystyle{ieee_fullname}
\bibliography{egbib}
}

\end{document}
