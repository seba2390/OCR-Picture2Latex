% CVPR 2023 Paper Template
% based on the CVPR template provided by Ming-Ming Cheng (https://github.com/MCG-NKU/CVPR_Template)
% modified and extended by Stefan Roth (stefan.roth@NOSPAMtu-darmstadt.de)

\documentclass[10pt,twocolumn,letterpaper]{article}

%%%%%%%%% PAPER TYPE  - PLEASE UPDATE FOR FINAL VERSION
% \usepackage[review]{cvpr}      % To produce the REVIEW version
\usepackage{cvpr}              % To produce the CAMERA-READY version
%\usepackage[pagenumbers]{cvpr} % To force page numbers, e.g. for an arXiv version

\usepackage[accsupp]{axessibility}  % Improves PDF readability for those with disabilities.

% Include other packages here, before hyperref.
\usepackage[normalem]{ulem}
\usepackage{graphicx}
\usepackage{amsmath}
\usepackage{amssymb}
\usepackage{booktabs}
\usepackage{xcolor}
\usepackage{comment}
\usepackage{enumitem}
\newcommand{\todo}[1]{{\color{red}#1}}

\newcommand{\benchmark}{WireSegHR}

% It is strongly recommended to use hyperref, especially for the review version.
% hyperref with option pagebackref eases the reviewers' job.
% Please disable hyperref *only* if you encounter grave issues, e.g. with the
% file validation for the camera-ready version.
%
% If you comment hyperref and then uncomment it, you should delete
% ReviewTempalte.aux before re-running LaTeX.
% (Or just hit 'q' on the first LaTeX run, let it finish, and you
%  should be clear).
\usepackage[pagebackref,breaklinks,colorlinks]{hyperref}


% Support for easy cross-referencing
\usepackage[capitalize]{cleveref}
\crefname{section}{Sec.}{Secs.}
\Crefname{section}{Section}{Sections}
\Crefname{table}{Table}{Tables}
\crefname{table}{Tab.}{Tabs.}


%%%%%%%%% PAPER ID  - PLEASE UPDATE
\def\cvprPaperID{699} % *** Enter the CVPR Paper ID here
\def\confName{CVPR}
\def\confYear{2023}

\begin{document}

%%%%%%%%% TITLE - PLEASE UPDATE
% \title{Automatic Wire Segmentation for Inpainting}
% \title{Segmentation to the Extreme: \\A Large-Scale Wire Segmentation Dataset and a Pilot Study}
%\title{Segmentation to the Extreme: \\A Large-Scale High-Resolution Wire Segmentation Dataset and a Pilot Study}
%\title{WINE: WIre NEver Appeared in Your Photos}
\title{Automatic High Resolution Wire Segmentation and Removal}

\author{Mang Tik Chiu$^{1,2}$, Xuaner Zhang$^2$, Zijun Wei$^{2}$, Yuqian Zhou$^2$, Eli Shechtman$^2$, \\Connelly Barnes$^{2}$, Zhe Lin$^2$, Florian Kainz$^2$, Sohrab Amirghodsi$^2$, Humphrey Shi$^{1,3}$ 
\\
{\small $^1$UIUC, $^2$Adobe, $^3$University of Oregon}\\
{\small \textbf{\url{https://github.com/adobe-research/auto-wire-removal}}}}

\twocolumn[{%
\renewcommand\twocolumn[1][]{#1}%
\maketitle
\begin{center}
    \centering
    \captionsetup{type=figure}
    \includegraphics[width=\textwidth]{figures/teaser2.pdf}
    \captionof{figure}{We present an automatic high-resolution wire segmentation and removal pipeline. Each triad shows the high-resolution input image, our automatic wire segmentation result masked in red, and our full-resolution wire removal result. The visual quality of these photographs is greatly improved with our fully-automated wire clean-up system.}
    \label{fig:demo}
\end{center}%
}]

% \vspace{3mm}
%[INTERNAL]: \href{https://session-0f3f6c51-53bd-446a-ad30-ea59ed8b6445.devbox.training.adobesensei.io/tensorboard/result.html}{Visualizations} of current models.

%%%%%%%%% ABSTRACT
% %auto-ignore
\twocolumn[{%
\renewcommand\twocolumn[1][]{#1}%
\begin{center}
\maketitle
    \centering
    \begin{tabular}{@{}c@{\hspace{1mm}}c@{\hspace{1mm}}c@{}}
    \captionsetup{type=figure}
    \includegraphics[width=0.5\linewidth]{figures/teaser/Picture2.jpg} &
    \includegraphics[width=0.5\linewidth]{figures/teaser/Picture4.jpg} \\
    Input Image & Our Wire Segmentation Result \\
    % \includegraphics[width=0.33\linewidth]{figures/teaser/1_caf.png} \\
    \end{tabular}
    \captionof{figure}{\textbf{Our wire-like object segmentation result}. The input image (left) is a 12 Megapixel image taken by a smartphone. Our wire-like object segmentation shows high quality result (right) despite the complexity and variation of wires in the input image. (All figures in this paper are best viewed in full resolution on a big screen.)}
\end{center}%
}]
  In this paper, we explore the connection between secret key agreement and secure omniscience within the setting of the multiterminal source model with a wiretapper who has side information. While the secret key agreement problem considers the generation of a maximum-rate secret key through public discussion, the secure omniscience problem is concerned with communication protocols for omniscience that minimize the rate of information leakage to the wiretapper. The starting point of our work is a lower bound on the minimum leakage rate for omniscience, $\rl$, in terms of the wiretap secret key capacity, $\wskc$. Our interest is in identifying broad classes of sources for which this lower bound is met with equality, in which case we say that there is a duality between secure omniscience and secret key agreement. We show that this duality holds in the case of certain finite linear source (FLS) models, such as two-terminal FLS models and pairwise independent network models on trees with a linear wiretapper. Duality also holds for any FLS model in which $\wskc$ is achieved by a perfect linear secret key agreement scheme. We conjecture that the duality in fact holds unconditionally for any FLS model. On the negative side, we give an example of a (non-FLS) source model for which duality does not hold if we limit ourselves to communication-for-omniscience protocols with at most two (interactive) communications.  We also address the secure function computation problem and explore the connection between the minimum leakage rate for computing a function and the wiretap secret key capacity.
  
%   Finally, we demonstrate the usefulness of our lower bound on $\rl$ by using it to derive equivalent conditions for the positivity of $\wskc$ in the multiterminal model. This extends a recent result of Gohari, G\"{u}nl\"{u} and Kramer (2020) obtained for the two-user setting.
  
   
%   In this paper, we study the problem of secret key generation through an omniscience achieving communication that minimizes the 
%   leakage rate to a wiretapper who has side information in the setting of multiterminal source model.  We explore this problem by deriving a lower bound on the wiretap secret key capacity $\wskc$ in terms of the minimum leakage rate for omniscience, $\rl$. 
%   %The former quantity is defined to be the maximum secret key rate achievable, and the latter one is defined as the minimum possible leakage rate about the source through an omniscience scheme to a wiretapper. 
%   The main focus of our work is the characterization of the sources for which the lower bound holds with equality \textemdash it is referred to as a duality between secure omniscience and wiretap secret key agreement. For general source models, we show that duality need not hold if we limit to the communication protocols with at most two (interactive) communications. In the case when there is no restriction on the number of communications, whether the duality holds or not is still unknown. However, we resolve this question affirmatively for two-user finite linear sources (FLS) and pairwise independent networks (PIN) defined on trees, a subclass of FLS. Moreover, for these sources, we give a single-letter expression for $\wskc$. Furthermore, in the direction of proving the conjecture that duality holds for all FLS, we show that if $\wskc$ is achieved by a \emph{perfect} secret key agreement scheme for FLS then the duality must hold. All these results mount up the evidence in favor of the conjecture on FLS. Moreover, we demonstrate the usefulness of our lower bound on $\wskc$ in terms of $\rl$ by deriving some equivalent conditions on the positivity of secret key capacity for multiterminal source model. Our result indeed extends the work of Gohari, G\"{u}nl\"{u} and Kramer in two-user case.

%%%%%%%%% BODY TEXT
% \leavevmode
% \\
% \\
% \\
% \\
% \\
\section{Introduction}
\label{introduction}

AutoML is the process by which machine learning models are built automatically for a new dataset. Given a dataset, AutoML systems perform a search over valid data transformations and learners, along with hyper-parameter optimization for each learner~\cite{VolcanoML}. Choosing the transformations and learners over which to search is our focus.
A significant number of systems mine from prior runs of pipelines over a set of datasets to choose transformers and learners that are effective with different types of datasets (e.g. \cite{NEURIPS2018_b59a51a3}, \cite{10.14778/3415478.3415542}, \cite{autosklearn}). Thus, they build a database by actually running different pipelines with a diverse set of datasets to estimate the accuracy of potential pipelines. Hence, they can be used to effectively reduce the search space. A new dataset, based on a set of features (meta-features) is then matched to this database to find the most plausible candidates for both learner selection and hyper-parameter tuning. This process of choosing starting points in the search space is called meta-learning for the cold start problem.  

Other meta-learning approaches include mining existing data science code and their associated datasets to learn from human expertise. The AL~\cite{al} system mined existing Kaggle notebooks using dynamic analysis, i.e., actually running the scripts, and showed that such a system has promise.  However, this meta-learning approach does not scale because it is onerous to execute a large number of pipeline scripts on datasets, preprocessing datasets is never trivial, and older scripts cease to run at all as software evolves. It is not surprising that AL therefore performed dynamic analysis on just nine datasets.

Our system, {\sysname}, provides a scalable meta-learning approach to leverage human expertise, using static analysis to mine pipelines from large repositories of scripts. Static analysis has the advantage of scaling to thousands or millions of scripts \cite{graph4code} easily, but lacks the performance data gathered by dynamic analysis. The {\sysname} meta-learning approach guides the learning process by a scalable dataset similarity search, based on dataset embeddings, to find the most similar datasets and the semantics of ML pipelines applied on them.  Many existing systems, such as Auto-Sklearn \cite{autosklearn} and AL \cite{al}, compute a set of meta-features for each dataset. We developed a deep neural network model to generate embeddings at the granularity of a dataset, e.g., a table or CSV file, to capture similarity at the level of an entire dataset rather than relying on a set of meta-features.
 
Because we use static analysis to capture the semantics of the meta-learning process, we have no mechanism to choose the \textbf{best} pipeline from many seen pipelines, unlike the dynamic execution case where one can rely on runtime to choose the best performing pipeline.  Observing that pipelines are basically workflow graphs, we use graph generator neural models to succinctly capture the statically-observed pipelines for a single dataset. In {\sysname}, we formulate learner selection as a graph generation problem to predict optimized pipelines based on pipelines seen in actual notebooks.

%. This formulation enables {\sysname} for effective pruning of the AutoML search space to predict optimized pipelines based on pipelines seen in actual notebooks.}
%We note that increasingly, state-of-the-art performance in AutoML systems is being generated by more complex pipelines such as Directed Acyclic Graphs (DAGs) \cite{piper} rather than the linear pipelines used in earlier systems.  
 
{\sysname} does learner and transformation selection, and hence is a component of an AutoML systems. To evaluate this component, we integrated it into two existing AutoML systems, FLAML \cite{flaml} and Auto-Sklearn \cite{autosklearn}.  
% We evaluate each system with and without {\sysname}.  
We chose FLAML because it does not yet have any meta-learning component for the cold start problem and instead allows user selection of learners and transformers. The authors of FLAML explicitly pointed to the fact that FLAML might benefit from a meta-learning component and pointed to it as a possibility for future work. For FLAML, if mining historical pipelines provides an advantage, we should improve its performance. We also picked Auto-Sklearn as it does have a learner selection component based on meta-features, as described earlier~\cite{autosklearn2}. For Auto-Sklearn, we should at least match performance if our static mining of pipelines can match their extensive database. For context, we also compared {\sysname} with the recent VolcanoML~\cite{VolcanoML}, which provides an efficient decomposition and execution strategy for the AutoML search space. In contrast, {\sysname} prunes the search space using our meta-learning model to perform hyperparameter optimization only for the most promising candidates. 

The contributions of this paper are the following:
\begin{itemize}
    \item Section ~\ref{sec:mining} defines a scalable meta-learning approach based on representation learning of mined ML pipeline semantics and datasets for over 100 datasets and ~11K Python scripts.  
    \newline
    \item Sections~\ref{sec:kgpipGen} formulates AutoML pipeline generation as a graph generation problem. {\sysname} predicts efficiently an optimized ML pipeline for an unseen dataset based on our meta-learning model.  To the best of our knowledge, {\sysname} is the first approach to formulate  AutoML pipeline generation in such a way.
    \newline
    \item Section~\ref{sec:eval} presents a comprehensive evaluation using a large collection of 121 datasets from major AutoML benchmarks and Kaggle. Our experimental results show that {\sysname} outperforms all existing AutoML systems and achieves state-of-the-art results on the majority of these datasets. {\sysname} significantly improves the performance of both FLAML and Auto-Sklearn in classification and regression tasks. We also outperformed AL in 75 out of 77 datasets and VolcanoML in 75  out of 121 datasets, including 44 datasets used only by VolcanoML~\cite{VolcanoML}.  On average, {\sysname} achieves scores that are statistically better than the means of all other systems. 
\end{itemize}


%This approach does not need to apply cleaning or transformation methods to handle different variances among datasets. Moreover, we do not need to deal with complex analysis, such as dynamic code analysis. Thus, our approach proved to be scalable, as discussed in Sections~\ref{sec:mining}.
\section{Related Work}
%\mz{We lack a comparison to this paper: https://arxiv.org/abs/2305.14877}
%\anirudh{refine to be more on-topic?}
\iffalse
\paragraph{In-Context Learning} As language models have scaled, the ability to learn in-context, without any weight updates, has emerged. \cite{brown2020language}. While other families of large language models have emerged, in-context learning remains ubiquitous \cite{llama, bloom, gptneo, opt}. Although such as HELM \cite{helm} have arisen for systematic evaluation of \emph{models}, there is no systematic framework to our knowledge for evaluating \emph{prompting methods}, and validating prompt engineering heuristics. The test-suite we propose will ensure that progress in the field of prompt-engineering is structured and objectively evaluated. 

\paragraph{Prompt Engineering Methods} Researchers are interested in the automatic design of high performing instructions for downstream tasks. Some focus on simple heuristics, such as selecting instructions that have the lowest perplexity \cite{lowperplexityprompts}. Other methods try to use large language models to induce an instruction when provided with a few input-output pairs \cite{ape}. Researchers have also used RL objectives to create discrete token sequences that can serve as instructions \cite{rlprompt}. Since the datasets and models used in these works have very little intersection, it is impossible to compare these methods objectively and glean insights. In our work, we evaluate these three methods on a diverse set of tasks and models, and analyze their relative performance. Additionally, we recognize that there are many other interesting angles of prompting that are not covered by instruction engineering \cite{weichain, react, selfconsistency}, but we leave these to future work.

\paragraph{Analysis of Prompting Methods} While most prompt engineering methods focus on accuracy, there are many other interesting dimensions of performance as well. For instance, researchers have found that for most tasks, the selection of demonstrations plays a large role in few-shot accuracy \cite{whatmakesgoodicexamples, selectionmachinetranslation, knnprompting}. Additionally, many researchers have found that even permuting the ordering of a fixed set of demonstrations has a significant effect on downstream accuracy \cite{fantasticallyorderedprompts}. Prompts that are sensitive to the permutation of demonstrations have been shown to also have lower accuracies \cite{relationsensitivityaccuracy}. Especially in low-resource domains, which includes the large public usage of in-context learning, these large swings in accuracy make prompting less dependable. In our test-suite we include sensitivity metrics that go beyond accuracy and allow us to find methods that are not only performant but reliable.

\paragraph{Existing Benchmarks} We recognize that other holistic in-context learning benchmarks exist. BigBench is a large benchmark of 204 tasks that are beyond the capabilities of current LLMs. BigBench seeks to evaluate the few-shot abilities of state of the art large language models, focusing on performance metrics such as accuracy \cite{bigbench}. Similarly, HELM is another benchmark for language model in-context learning ability. Rather than only focusing on performance, HELM branches out and considers many other metrics such as robustness and bias \cite{helm}. Both BigBench and HELM focus on ranking different language model, while fix a generic instruction and prompt format. We instead choose to evaluate instruction induction / selection methods over a fixed set of models. We are the first ever evaluation script that compares different prompt-engineering methods head to head. 
\fi

\paragraph{In-Context Learning and Existing Benchmarks} As language models have scaled, in-context learning has emerged as a popular paradigm and remains ubiquitous among several autoregressive LLM families \cite{brown2020language, llama, bloom, gptneo, opt}. Benchmarks like BigBench \cite{bigbench} and HELM \cite{helm} have been created for the holistic evaluation of these models. BigBench focuses on few-shot abilities of state-of-the-art large language models, while HELM extends to consider metrics like robustness and bias. However, these benchmarks focus on evaluating and ranking \emph{language models}, and do not address the systematic evaluation of \emph{prompting methods}. Although contemporary work by \citet{yang2023improving} also aims to perform a similar systematic analysis of prompting methods, they focus on simple probability-based prompt selection while we evaluate a broader range of methods including trivial instruction baselines, curated manually selected instructions, and sophisticated automated instruction selection.

\paragraph{Automated Prompt Engineering Methods} There has been interest in performing automated prompt-engineering for target downstream tasks within ICL. This has led to the exploration of various prompting methods, ranging from simple heuristics such as selecting instructions with the lowest perplexity \cite{lowperplexityprompts}, inducing instructions from large language models using a few annotated input-output pairs \cite{ape}, to utilizing RL objectives to create discrete token sequences as prompts \cite{rlprompt}. However, these works restrict their evaluation to small sets of models and tasks with little intersection, hindering their objective comparison. %\mz{For paragraphs that only have one work in the last line, try to shorten the paragraph to squeeze in context.}

\paragraph{Understanding in-context learning} There has been much recent work attempting to understand the mechanisms that drive in-context learning. Studies have found that the selection of demonstrations included in prompts significantly impacts few-shot accuracy across most tasks \cite{whatmakesgoodicexamples, selectionmachinetranslation, knnprompting}. Works like \cite{fantasticallyorderedprompts} also show that altering the ordering of a fixed set of demonstrations can affect downstream accuracy. Prompts sensitive to demonstration permutation often exhibit lower accuracies \cite{relationsensitivityaccuracy}, making them less reliable, particularly in low-resource domains.

Our work aims to bridge these gaps by systematically evaluating the efficacy of popular instruction selection approaches over a diverse set of tasks and models, facilitating objective comparison. We evaluate these methods not only on accuracy metrics, but also on sensitivity metrics to glean additional insights. We recognize that other facets of prompting not covered by instruction engineering exist \cite{weichain, react, selfconsistency}, and defer these explorations to future work. 
\section{Dataset}
\label{sec:dataset}
%\sarah{add statistics about distribution of merge patterns}
%\alexey{I added some numbers in the section 4 (around line 270). Detailed numbers are in Appendix. We can move it up here if needed...}
%To create a dataset for self-supervised pretraining, we clone all non-fork repositories with more than 20 stars in GitHub that have C, C++, C\#, Python, Java, JavaScript, TypeScript, PHP, Go, and Ruby as their top language. The resulting dataset comprises over 64 million source code files. 
%\chris{why do we list languages here that we don't ever evaluate on?  A reviewer will find this confusing and ask about it.  We found that language specific models work better than multi-lingual models, right?}

The finetuning dataset is mined from over 100,000 open source software repositories in multiple programming languages with merge conflicts. It contains commits from git histories with exactly two parents, which resulted in a merge conflict.  We replay \texttt{git merge} on the two parents to see if it generates any conflicts. Otherwise, we ignore the merge from our dataset. We use the approach introduced by~\citet{Dinella2021} to extract resolution regions---however, we do not restrict ourselves to conflicts with less than 30 lines only.  Lastly, we extract token-level conflicts and conflict resolution classification labels (introduced in Section \ref{formulation}) from line-level conflicts and resolutions. Tab.~\ref{tab:fintuning_dataset} provides a summary of the finetuning dataset.

\begin{table}[htb]
\centering
\caption{Number of merge conflicts in the dataset.}
\begin{tabular}{llllllllllll} \toprule
\textbf{Programming language} & \textbf{Development set}  & \textbf{Test set} \\ \midrule
C\# & 27874 & 6969 \\ 
JavaScript & 66573 & 16644\\ 
TypeScript & 22422 & 5606\\ 
Java & 103065 & 25767 \\ 
\bottomrule
\end{tabular}
\label{tab:fintuning_dataset}
\end{table}
The finetuning dataset is split into development and test sets in the proportion 80/20 at random at the file-level. The development set is further split into training and validation sets in 80/20 proportion at the merge conflict level.    

%auto-ignore

% \section{Method} \label{sec:method}

% In this section, we describe our two-stage model architecture. Figure~\ref{fig:pipeline} shows the overall model structure. We will first discuss motivations of our two-stage design, then describe our model architecture and training details.

%In this section, we first discuss motivations of our two-stage model architecture, then describe our model architecture and training details. Figure~\ref{fig:pipeline} shows the overall model structure. 

\section{High-Resolution Wire Segmentation}
\label{sec:segmentation}
Wires appear visually different from common objects -- being thin, long, sparse and oftentimes partially occluded. 
We find the following two design choices crucial to building an effective wire segmentation system: 1) having a two stage framework so that coarse prediction from global context guides precise segmentation from local patches and 2) maximally preserving and augmenting image features and annotations of wires throughout the pipeline.  
% We thus propose a two-stage coarse-to-fine pipeline, in which we inject feature augmentation and architectural choices to maximally preserve the wire features throughout.

\subsection{The Two-stage Coarse to Fine Model}
%\yqe{
Figure \ref{fig:pipeline} shows the two-stage segmentation pipeline. It consists of a coarse and a fine module, which share an encoder $E$ and have their own decoder $D_C$ and $D_F$. Intuitively, the coarse module aims to capture the global contextual information from the entire image and highlight the image regions possibly containing wires. Conditioned on the predictions from the coarse module, the fine module achieves high-resolution wire segmentation by only looking at local patches likely containing wires.

Given a high-resolution image $I_\textrm{glo}$, we first bilinearly downsample it to $I_\textrm{glo}^{ds}$ with a fixed size $p\times p$ and feed it into the coarse module. The module predicts the global probability map $P_\textrm{glo} = \textrm{SoftMax}(D_C(E(I_\textrm{glo}^{ds})))$ containing the activation of the wire regions.% with rich global contextual information.

For each patch $I_\textrm{loc}$ of size $p \times p$ cropped from the full-resolution image $I_\textrm{glo}$, and the corresponding conditional probability map $P_\textrm{con}$ cropped from $P_\textrm{glo}$, we predict the local probability $P_\textrm{loc} = \textrm{SoftMax}(D_F(E(I_\textrm{loc}, P_\textrm{con})))$. 
%$M$ is a binary location mask indicating the patch region of $I_{loc}$ in $I_{glo}$ with 1, and elsewhere 0.
Note that $E$ is shared between the coarse and the fine module, thus it should take inputs with the same number of channels. Therefore, for the coarse module, we concatenate an additional zero channel with the input image to make the channel number consistent.
% We formulate our problem as a semantic segmentation task, so we can regard it as a pixel-wise classification problem. Each image pixel is labeled as 0 or 1 indicating background or wire pixel.

% We apply Cross Entropy (CE) loss to both the global $P_\textrm{glo} = \textrm{SoftMax}(Z_\textrm{glo})$ and local probability map $P_\textrm{loc} = \textrm{SoftMax}(Z_\textrm{loc})$, comparing with their ground truth annotations $G_\textrm{glo}$ and $G_\textrm{loc}$.

We apply Cross Entropy (CE) loss to both the global $P_\textrm{glo}$ and local probability map $P_\textrm{loc}$, comparing with their ground truth annotations $G_\textrm{glo}$ and $G_\textrm{loc}$.

\vspace{-2mm}
\begin{equation}
\begin{aligned}
%    \mathcal{L}_{glo} &= -\sum_{c\in C}\log\ Softmax(Z_{glo})_c \\
    %\mathcal{L}_{loc} &= -\sum_{c\in C}\log\ Softmax(Z_{loc})_c \\
    \mathcal{L}_\textrm{glo} &= CE (P_\textrm{glo}, G_\textrm{glo}) \\
    \mathcal{L}_\textrm{loc} &= CE (P_\textrm{loc}, G_\textrm{loc}) \\
\end{aligned}
\end{equation}
%Both $Z_{glo}$ and $Z_{loc}$ are trained by computing Cross Entropy losses of the logit maps after softmax against their respective ground truth wire annotations. 
The final loss $\mathcal{L}$ is the sum of the two:

\vspace{-3mm}
\begin{equation}
\begin{aligned}
    \mathcal{L} &= \mathcal{L}_ \textrm{glo} + \lambda \mathcal{L}_ \textrm{loc},
\end{aligned}
\end{equation}

%Given a high-resolution image $I$, the coarse module of our two-stage model aims to recognize semantics of the entire image and pick up wire regions at a coarse level. Looking at the entire image allows the coarse module to capture more contextual information. To fit the large images into a GPU for training and inference, images are first bilinearly downsampled, then fed into the coarse module to predict a global logit map $Z_{glo}$. We use $Z_{glo}$ as a conditional input to guide the fine module in the next stage.

%The fine module is conditioned on the output from the coarse module. This module takes in a local image patch $I_{loc}$ of the whole image at full resolution, the global logit map $Z_{glo}$ from the coarse module, and a binary location mask $M$ that sets the patch relative to the full image to $1$ and other regions to $0$. The fine module then predicts the local wire logit map $Z_{loc}$. Empirically, we find that concatenating the entire global logit map rather than cropping the logit map at the location of the image patch yields slightly improved results.

%The designs of the coarse and fine modules are conceptually the same as those in GLNet~\cite{glnet} and MagNet~\cite{magnet}, where a global network is trained on entire downsampled images and a local network is trained on higher-resolution image patches conditioned on some global features. However, unlike GLNet, where intermediate features are shared between the global and local branch bidirectionally, we opt for a simpler late fusion by concatenating the logit map directly to the fine module. We also only use two stages instead of up to four stage as done in MagNet, since a single fine module is already sufficient at refining annotations that are only several pixels thick, and additional stages can drastically increase inference time.

where we 
%empirically 
set $\lambda = 1$ for training. 
% During training, 
% we apply data augmentation including random scaling, horizontal flipping and photometric distortion to the full-resolution image to obtain the input $I_{glo}$. After that, 
% we randomly crop one local patch $I_{loc}$ with patch size $p=512$ from the augmented image. 
Similar to Focal loss~\cite{focal} and Online Hard Example Mining~\cite{ohem}, we balance the wire and background samples in the training set by selecting patches that contain at least 1\% of wire pixels. %We find that this simple constrained cropping approach yields better performances than other well known balancing methods, including Focal loss~\cite{focal} and Online Hard Example Mining~\cite{ohem}.

%The global image $I_{glo}$ is generated by simply downsampling this augmented image. We then generate the local image patch $I_{loc}$ by randomly cropping a 512$\times$512 window from the augmented image that contains at least 1\% wire pixels. This helps balance between wire and background pixels. We find that this simple constrained cropping approach yields better performances than other well known balancing methods, including Focal loss~\cite{focal} and Online Hard Example Mining~\cite{ohem}.

%We share the same feature extractor network between both the coarse and fine module, but train separate feature decoders. 
%To account for the additional inputs to the fine module, we expand the input channels of the feature extractor from 3 to 5. The last two channels are set to 0 when passing an image through the coarse module, and set to the global logit map $Z_{glo}$ and the binary location map $M$ for the fine module. 
To perform inference, we first feed the downsampled image to the coarse module, which is the same as training. 
% We then compute the global wire segmentation map by taking the $\mathrm{argmax}$ over the two classes for each pixel. 
Local inference is done by running a sliding window over the entire image, where the patch is sampled only when there is at least some percentage of wire pixels (determined by~$\alpha$). This brings two advantages: First, we save computation time in regions where there are no wires. Second, the local fine module can leverage the information from the global branch for better inference quality.

% The design of our two-stage pipeline is conceptually similar to GLNet~\cite{glnet} and MagNet~\cite{magnet}, where a global network is trained on downsampled images and a local network is trained on image patches at the original resolution. However, unlike GLNet whose intermediate features are shared between the global and local branch bidirectionally, we opt for a simpler late fusion by concatenating the probability map directly to the fine module. We only use two stages instead of four stages in MagNet. Our pipeline demonstrates its efficiency and simplicity while coping with wire-like objects which are only several pixels thick, and avoids high computational demands from some previous works.

\subsection{Wire Feature Preservation}
\label{sec:wire_feature_preserve}
As wires are thin and sparse, applying downsampling to the input images may make the wire features vanish entirely. To mitigate this challenge, we propose a simple feature augmentation technique by taking the min and max pixel luminance values of the input image over a local window. Either the local min or the max value makes the wire pixels more visually apparent. In practice, we concatenate the min- and max-filtered luminance channels to the RGB image and condition map, resulting in 6 total channels as input. We name this component MinMax.
%to form a 5-channel input.
% Unlike common object segmentation, with prominent textures and colors, wires, especially outdoor cables, tend to either be shadowed or reflective under sunlight. As a result, it is common to see either very bright or very dark wires in images. An intuitive approach is to find the min and max of an image before downsampling, and preserve this feature as input to the model.
% \todo{show an example patch of luminance min max filter:}

%\subsection{Motivation}
% We design a two-stage coarse-to-fine pipeline for wire-like object segmentation. This model structure 
% we divide our model into two major components -- a coarse module and a fine module. Our two-stage model design 
% is motivated by two observations. First, wires in our dataset are extremely long and thin, many spanning over several thousand pixels but only occupy several pixels across. Limited by the memory size of modern GPUs, we cannot simply pass the entire image at full-resolution to a model for inference. As a result, two separate modules are required for high-resolution inference, where the coarse module captures the entire wire and its surrounding contextual information at a lower resolution, and the fine module captures detailed textures of the wire at the original resolution.

% Second, we observe that wire pixels are sparse, where a typical high-resolution image contains a small percentages of wire pixels. This means that we can also use the coarse module to guide the fine module on what regions to capture the details. The fine module can save computation by only predicting segmentation masks where there are wires.

% \subsection{Global Condition}
% Many high-resolution segmentation methods share the idea of using global-local refinement module. But each method differs in details that are tailored to their targeted applications. 
% MagNet~\cite{magnet} trains a single model by randomly sampling global/local patches, and then trains a refinement module that takes as input the two predictions. This neglects the causal relationship between global and local features. The refinement module does not take into account any image feature. The recently proposed ISDNet~\cite{isdnet} use a shallow network and takes the entire image as input. This is proven effective in some high-resolution segmentation datasets such as DeepGlobe~\cite{deepglobe}, where resolution is fixed and manageable (5k$\times$5k). But the model is unable to scale to larger images. The shallow model, while being efficient, is limited in model capacity, especially with sparse labels.
% the fact that they take the entire image as input strictly limits their model to use lightweight networks as the shallow branch (i.e. STDC~\cite{stdc} with output stride=8), which has limited performance in cases where the target label is also small. 
% In fact, we were unable to train ISDNet with a stronger shallow network to produce high-resolution outputs.
% ~\cite{learning_downsample} attempts to learn a downsampling network, which we tried but found was detrimental for our thin and sparse wire masks. 
% As a result, 

Besides feature augmentations, we also adapt the architecture to maximally preserve the sparse wire annotations.
% Different from existing methods, we find it crucial to keep the image at a close-to-original resolution to maximally preserve the image features of wires. In addition, we find it similarly important to preserve the sparse wire annotations. 
% We use the global branch as a conditioning network for training both coarse and high resolution segmentation tasks rather than a simple coarse branch that is only learned for the coarse resolution. 
We propose to use ``overprediction" and achieve this by using max-pool downsampling on the coarse labels during training, which preserves activation throughout the coarse branch. We name this component MaxPool. We provide ablation studies for these components in Section~\ref{sec:results}.

\vspace{-5mm}
\section{High-Resolution Wire Inpainting}
\label{sec:inpainting}
Given a full-resolution wire segmentation mask estimated by our wire segmentation model, we propose an inpainting pipeline to remove and fill in the wire regions. Our approach addresses two major challenges in wire inpainting. First, recent state-of-the-art deep inpainting methods do not handle arbitrary resolution images, which is critical for high-resolution wire removal. Second, deep inpainting methods often suffer from color inconsistency when the background has uniform (or slowly varying) colors. This issue is particularly significant for wires, as they are often in front of uniform backgrounds, such as the sky or building facades. The commonly used reconstruction loss, such as L1, is not sensitive to color inconsistency, which further exacerbates this issue.

We thus revisit the efficient deep inpainting method LaMa \cite{suvorov2022resolution}. Compared with other inpainting models, LaMa has two major advantages. First, it contains the Fourier convolutional layers which enables an efficient and high-quality structural completion. This helps complete building facades and other man-made structures with fewer artifacts. Second, its high inference efficiency makes a tile-based inference approach possible for high resolution images. 

% To tailor LaMa for wire removal, we fix the input size of the model to $512\times512$ and train the model on an augmented dataset by including synthetic wire masks and cropped patches from full-resolution images. 
To address color inconsistency, we propose a novel ``onion-peel" color adjustment module. Specifically, we compute the mean of the RGB channels within the onion-peel regions $M_o = D(M, d) - M$ of the wire mask $M$, where $D$ is the binary dilation operator, and $d$ is the kernel size. The color difference for each channel $c \in {R, G, B}$ becomes $\textrm{Bias}_c = \mathbb{E}[M_o (x_c - y_c)]$, where $x$ is the input image, and $y$ is the output from the inpainting network. The final output of the inpainting model is: $\hat{y_c} = y_c + \textrm{Bias}_{c}$. Loss functions are then applied to $\hat{y_c}$ to achieve color consistency while compositing the final result $y_{out} = (1 - M) \odot x + M \odot \hat{y}$. 

% While running inference on full-resolution images, we apply a tile-based approach, by fixing the window size at $512\times 512$ with an $32$-pixel overlap. 
% This makes the model consistent in training and testing settings, and gives good textural and structural details for local regions.
%!TEX ROOT = ../../centralized_vs_distributed.tex

\section{{\titlecap{the centralized-distributed trade-off}}}\label{sec:numerical-results}

\revision{In the previous sections we formulated the optimal control problem for a given controller architecture
(\ie the number of links) parametrized by $ n $
and showed how to compute minimum-variance objective function and the corresponding constraints.
In this section, we present our main result:
%\red{for a ring topology with multiple options for the parameter $ n $},
we solve the optimal control problem for each $ n $ and compare the best achievable closed-loop performance with different control architectures.\footnote{
\revision{Recall that small (large) values of $ n $ mean sparse (dense) architectures.}}
For delays that increase linearly with $n$,
\ie $ f(n) \propto n $, 
we demonstrate that distributed controllers with} {few communication links outperform controllers with larger number of communication links.}

\textcolor{subsectioncolor}{Figure~\ref{fig:cont-time-single-int-opt-var}} shows the steady-state variances
obtained with single-integrator dynamics~\eqref{eq:cont-time-single-int-variance-minimization}
%where we compare the standard multi-parameter design 
%with a simplified version \tcb{that utilizes spatially-constant feedback gains
and the quadratic approximation~\eqref{eq:quadratic-approximation} for \revision{ring topology}
with $ N = 50 $ nodes. % and $ n\in\{1,\dots,10\} $.
%with $ N = 50 $, $ f(n) = n $ and $ \tau_{\textit{min}} = 0.1 $.
%\autoref{fig:cont-time-single-int-err} shows the relative error, defined as
%\begin{equation}\label{eq:relative-error}
%	e \doteq \dfrac{\optvarx-\optvar}{\optvar}
%\end{equation}
%where $ \optvar $ and $ \optvarx $ denote the the optimal and sub-optimal scalar variances, respectively.
%The performance gap is small
%and becomes negligible for large $ n $.
{The best performance is achieved for a sparse architecture with  $ n = 2 $ 
in which each agent communicates with the two closest pairs of neighboring nodes. 
This should be compared and contrasted to nearest-neighbor and all-to-all 
communication topologies which induce higher closed-loop variances. 
Thus, 
the advantage of introducing additional communication links diminishes 
beyond}
{a certain threshold because of communication delays.}

%For a linear increase in the delay,
\textcolor{subsectioncolor}{Figure~\ref{fig:cont-time-double-int-opt-var}} shows that the use of approximation~\eqref{eq:cont-time-double-int-min-var-simplified} with $ \tilde{\gvel}^* = 70 $
identifies nearest-neighbor information exchange as the {near-optimal} architecture for a double-integrator model
with ring topology. 
This can be explained by noting that the variance of the process noise $ n(t) $
in the reduced model~\eqref{eq:x-dynamics-1st-order-approximation}
is proportional to $ \nicefrac{1}{\gvel} $ and thereby to $ \taun $,
according to~\eqref{eq:substitutions-4-normalization},
making the variance scale with the delay.

%\mjmargin{i feel that we need to comment about different results that we obtained for CT and DT double-intergrator dynamics (monotonic deterioration of performance for the former and oscillations for the latter)}
\revision{\textcolor{subsectioncolor}{Figures~\ref{fig:disc-time-single-int-opt-var}--\ref{fig:disc-time-double-int-opt-var}}
show the results obtained by solving the optimal control problem for discrete-time dynamics.
%which exhibit similar trade-offs.
The oscillations about the minimum in~\autoref{fig:disc-time-double-int-opt-var}
are compatible with the investigated \tradeoff~\eqref{eq:trade-off}:
in general, 
the sum of two monotone functions does not have a unique local minimum.
Details about discrete-time systems are deferred to~\autoref{sec:disc-time}.
Interestingly,
double integrators with continuous- (\autoref{fig:cont-time-double-int-opt-var}) ad discrete-time (\autoref{fig:disc-time-double-int-opt-var}) dynamics
exhibits very different trade-off curves,
whereby performance monotonically deteriorates for the former and oscillates for the latter.
While a clear interpretation is difficult because there is no explicit expression of the variance as a function of $ n $,
one possible explanation might be the first-order approximation used to compute gains in the continuous-time case.
%which reinforce our thesis exposed in~\autoref{sec:contribution}.

%\begin{figure}
%	\centering
%	\includegraphics[width=.6\linewidth]{cont-time-double-int-opt-var-n}
%	\caption{Steady-state scalar variance for continuous-time double integrators with $ \taun = 0.1n $.
%		Here, the \tradeoff is optimized by nearest-neighbor interaction.
%	}
%	\label{fig:cont-time-double-int-opt-var-lin}
%\end{figure}
}

\begin{figure}
	\centering
	\begin{minipage}[l]{.5\linewidth}
		\centering
		\includegraphics[width=\linewidth]{random-graph}
	\end{minipage}%
	\begin{minipage}[r]{.5\linewidth}
		\centering
		\includegraphics[width=\linewidth]{disc-time-single-int-random-graph-opt-var}
	\end{minipage}
	\caption{Network topology and its optimal {closed-loop} variance.}
	\label{fig:general-graph}
\end{figure}

Finally,
\autoref{fig:general-graph} shows the optimization results for a random graph topology with discrete-time single integrator agents. % with a linear increase in the delay, $ \taun = n $.
Here, $ n $ denotes the number of communication hops in the ``original" network, shown in~\autoref{fig:general-graph}:
as $ n $ increases, each agent can first communicate with its nearest neighbors,
then with its neighbors' neighbors, and so on. For a control architecture that utilizes different feedback gains for each communication link
	(\ie we only require $ K = K^\top $) we demonstrate that, in this case, two communication hops provide optimal closed-loop performance. % of the system.}

Additional computational experiments performed with different rates $ f(\cdot) $ show that the optimal number of links increases for slower rates: 
for example, 
the optimal number of links is larger for $ f(n) = \sqrt{n} $ than for $ f(n) = n $. 
\revision{These results are not reported because of space limitations.}
\mySection{Related Works and Discussion}{}
\label{chap3:sec:discussion}

In this section we briefly discuss the similarities and differences of the model presented in this chapter, comparing it with some related work presented earlier (Chapter \ref{chap1:artifact-centric-bpm}). We will mention a few related studies and discuss directly; a more formal comparative study using qualitative and quantitative metrics should be the subject of future work.

Hull et al. \citeyearpar{hull2009facilitating} provide an interoperation framework in which, data are hosted on central infrastructures named \textit{artifact-centric hubs}. As in the work presented in this chapter, they propose mechanisms (including user views) for controlling access to these data. Compared to choreography-like approach as the one presented in this chapter, their settings has the advantage of providing a conceptual rendezvous point to exchange status information. The same purpose can be replicated in this chapter's approach by introducing a new type of agent called "\textit{monitor}", which will serve as a rendezvous point; the behaviour of the agents will therefore have to be slightly adapted to take into account the monitor and to preserve as much as possible the autonomy of agents.

Lohmann and Wolf \citeyearpar{lohmann2010artifact} abandon the concept of having a single artifact hub \cite{hull2009facilitating} and they introduce the idea of having several agents which operate on artifacts. Some of those artifacts are mobile; thus, the authors provide a systematic approach for modelling artifact location and its impact on the accessibility of actions using a Petri net. Even though we also manipulate mobile artifacts, we do not model artifact location; rather, our agents are equipped with capabilities that allow them to manipulate the artifacts appropriately (taking into account their location). Moreover, our approach considers that artifacts can not be remotely accessed, this increases the autonomy of agents.

The process design approach presented in this chapter, has some conceptual similarities with the concept of \textit{proclets} proposed by Wil M. P. van der Aalst et al. \citeyearpar{van2001proclets, van2009workflow}: they both split the process when designing it. In the model presented in this chapter, the process is split into execution scenarios and its specification consists in the diagramming of each of them. Proclets \cite{van2001proclets, van2009workflow} uses the concept of \textit{proclet-class} to model different levels of granularity and cardinality of processes. Additionally, proclets act like agents and are autonomous enough to decide how to interact with each other.

The model presented in this chapter uses an attributed grammar as its mathematical foundation. This is also the case of the AWGAG model by Badouel et al. \citeyearpar{badouel14, badouel2015active}. However, their model puts stress on modelling process data and users as first class citizens and it is designed for Adaptive Case Management.

To summarise, the proposed approach in this chapter allows the modelling and decentralized execution of administrative processes using autonomous agents. In it, process management is very simply done in two steps. The designer only needs to focus on modelling the artifacts in the form of task trees and the rest is easily deduced. Moreover, we propose a simple but powerful mechanism for securing data based on the notion of accreditation; this mechanism is perfectly composed with that of artifacts. The main strengths of our model are therefore : 
\begin{itemize}
	\item The simplicity of its syntax (process specification language), which moreover (well helped by the accreditation model), is suitable for administrative processes;
	\item The simplicity of its execution model; the latter is very close to the blockchain's execution model \cite{hull2017blockchain, mendling2018blockchains}. On condition of a formal study, the latter could possess the same qualities (fault tolerance, distributivity, security, peer autonomy, etc.) that emanate from the blockchain;
	\item Its formal character, which makes it verifiable using appropriate mathematical tools;
	\item The conformity of its execution model with the agent paradigm and service technology.
\end{itemize}
In view of all these benefits, we can say that the objectives set for this thesis have indeed been achieved. However, the proposed model is perfectible. For example, it can be modified to permit agents to respond incrementally to incoming requests as soon as any prefix of the extension of a bud is produced. This makes it possible to avoid the situation observed on figure \ref{chap3:fig:execution-figure-4} where the associated editor is informed of the evolution of the subtree resulting from $C$ only when this one is closed. All the criticisms we can make of the proposed model in particular, and of this thesis in general, have been introduced in the general conclusion (page \pageref{chap5:general-conclusion}) of this manuscript.




% \vspace{-0.5em}
\section{Conclusion}
% \vspace{-0.5em}
Recent advances in multimodal single-cell technology have enabled the simultaneous profiling of the transcriptome alongside other cellular modalities, leading to an increase in the availability of multimodal single-cell data. In this paper, we present \method{}, a multimodal transformer model for single-cell surface protein abundance from gene expression measurements. We combined the data with prior biological interaction knowledge from the STRING database into a richly connected heterogeneous graph and leveraged the transformer architectures to learn an accurate mapping between gene expression and surface protein abundance. Remarkably, \method{} achieves superior and more stable performance than other baselines on both 2021 and 2022 NeurIPS single-cell datasets.

\noindent\textbf{Future Work.}
% Our work is an extension of the model we implemented in the NeurIPS 2022 competition. 
Our framework of multimodal transformers with the cross-modality heterogeneous graph goes far beyond the specific downstream task of modality prediction, and there are lots of potentials to be further explored. Our graph contains three types of nodes. While the cell embeddings are used for predictions, the remaining protein embeddings and gene embeddings may be further interpreted for other tasks. The similarities between proteins may show data-specific protein-protein relationships, while the attention matrix of the gene transformer may help to identify marker genes of each cell type. Additionally, we may achieve gene interaction prediction using the attention mechanism.
% under adequate regulations. 
% We expect \method{} to be capable of much more than just modality prediction. Note that currently, we fuse information from different transformers with message-passing GNNs. 
To extend more on transformers, a potential next step is implementing cross-attention cross-modalities. Ideally, all three types of nodes, namely genes, proteins, and cells, would be jointly modeled using a large transformer that includes specific regulations for each modality. 

% insight of protein and gene embedding (diff task)

% all in one transformer

% \noindent\textbf{Limitations and future work}
% Despite the noticeable performance improvement by utilizing transformers with the cross-modality heterogeneous graph, there are still bottlenecks in the current settings. To begin with, we noticed that the performance variations of all methods are consistently higher in the ``CITE'' dataset compared to the ``GEX2ADT'' dataset. We hypothesized that the increased variability in ``CITE'' was due to both less number of training samples (43k vs. 66k cells) and a significantly more number of testing samples used (28k vs. 1k cells). One straightforward solution to alleviate the high variation issue is to include more training samples, which is not always possible given the training data availability. Nevertheless, publicly available single-cell datasets have been accumulated over the past decades and are still being collected on an ever-increasing scale. Taking advantage of these large-scale atlases is the key to a more stable and well-performing model, as some of the intra-cell variations could be common across different datasets. For example, reference-based methods are commonly used to identify the cell identity of a single cell, or cell-type compositions of a mixture of cells. (other examples for pretrained, e.g., scbert)


%\noindent\textbf{Future work.}
% Our work is an extension of the model we implemented in the NeurIPS 2022 competition. Now our framework of multimodal transformers with the cross-modality heterogeneous graph goes far beyond the specific downstream task of modality prediction, and there are lots of potentials to be further explored. Our graph contains three types of nodes. while the cell embeddings are used for predictions, the remaining protein embeddings and gene embeddings may be further interpreted for other tasks. The similarities between proteins may show data-specific protein-protein relationships, while the attention matrix of the gene transformer may help to identify marker genes of each cell type. Additionally, we may achieve gene interaction prediction using the attention mechanism under adequate regulations. We expect \method{} to be capable of much more than just modality prediction. Note that currently, we fuse information from different transformers with message-passing GNNs. To extend more on transformers, a potential next step is implementing cross-attention cross-modalities. Ideally, all three types of nodes, namely genes, proteins, and cells, would be jointly modeled using a large transformer that includes specific regulations for each modality. The self-attention within each modality would reconstruct the prior interaction network, while the cross-attention between modalities would be supervised by the data observations. Then, The attention matrix will provide insights into all the internal interactions and cross-relationships. With the linearized transformer, this idea would be both practical and versatile.

% \begin{acks}
% This research is supported by the National Science Foundation (NSF) and Johnson \& Johnson.
% \end{acks}

%%%%%%%%% REFERENCES
{\small
\bibliographystyle{ieee_fullname}
\bibliography{egbib}
}

\end{document}
