%auto-ignore
\begin{table}[h!]
\centering
\resizebox{\linewidth}{!}{
    \renewcommand{\arraystretch}{1.1}
    \begin{tabular}{r|cccc}
    \hline
    
    Dataset & \begin{tabular}[x]{@{}c@{}}\# Wire\\Images\end{tabular}  & \begin{tabular}[x]{@{}c@{}}Min.\\Image Size\end{tabular}  &\begin{tabular}[x]{@{}c@{}}Max.\\Image Size\end{tabular}  &\begin{tabular}[x]{@{}c@{}}Median\\Image Size\end{tabular} \\ \hline
    Powerline~\cite{powerlinedataset} & 2000 & 128$\times$128 & 128$\times$128 & 128$\times$128\\
    PLDU~\cite{pldu} & 573 & 540$\times$360 & 540$\times$360 & 540$\times$360 \\
    PLDM~\cite{pldu} & 287 & 540$\times$360 & 540$\times$360 & 540$\times$360 \\
    TTPLA~\cite{ttpla} & 1100 & 3840$\times$2160 & 3840$\times$2160 & 3840$\times$2160\\ \hline
    \textbf{Ours} & 6000 & 360$\times$240 & 15904$\times$10608 & 5040$\times$3360 \\ \hline

\end{tabular}
}
\vspace{-2mm}
\caption{Statistics of our wire dataset compared to others.}%Image and annotation statistics of our test set.\yq{You can make it as a 4x4 table, with each row: item, min, max, avg; and each column: item name, image size, wire thickness, and percentage. (mt: done)}}
\vspace{-5mm}
\label{table:stats}
\end{table}