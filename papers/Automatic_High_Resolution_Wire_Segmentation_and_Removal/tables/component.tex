%auto-ignore

\begin{table}[t!]
\centering
\resizebox{0.49\textwidth}{!}{
    \renewcommand{\arraystretch}{1}
    \addtolength{\tabcolsep}{-2pt}
    \begin{tabular}{r|c|ccc|ccc}
    \hline
    Model & \begin{tabular}[x]{@{}c@{}}Wire\\IoU\end{tabular} & F1 & Precision & Recall & \begin{tabular}[x]{@{}c@{}}IoU\\(Small)\end{tabular} & \begin{tabular}[x]{@{}c@{}}IoU\\(Medium)\end{tabular} & \begin{tabular}[x]{@{}c@{}}IoU\\(Large)\end{tabular} \\\hline\hline
    %& \begin{tabular}[x]{@{}c@{}}Speed\\(s/img)\end{tabular}  & \begin{tabular}[x]{@{}c@{}}Memory\\(GB)\end{tabular}\\ \hline\hline 
    Ours & 60.83 & 75.65 & 83.62 & 69.06 & 63.52 & 59.83 & 62.93 \\ \hline
    -- MinMax & 60.01 & 75.01 & 84.87 & 67.2 & 63.67 & 58.99 & 61.97 \\
    -- MaxPool & 59.86 & 74.89 & 85.25 & 66.78 & 61.45 & 59.40 & 60.76 \\
    -- Coarse & 56.92 & 72.55 & 82.91 & 64.49 & 62.83 & 57.42 & 54.47 \\ \hline
\end{tabular}
\addtolength{\tabcolsep}{2pt}
}
\vspace{-1mm}
\caption{Ablation study of our model components.
% \yq{This table can be made one-column to save space.}\cezhang{+1, we're running out of space.}
}
\label{table:component_ablation}
\vspace{-2mm}
\end{table}