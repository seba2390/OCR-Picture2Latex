\subsubsection{TDSS Motivation} 

The variable sky is the focus of many recent and upcoming large-scale
photometric surveys. For example, the SDSS Supernova program included
100 epochs of $ugriz$ imaging on a $2.5^\circ$ wide region on the
Celestial Equator in the SGC (Stripe 82; \citealt{sesar07a}). Recently
concluded and ongoing surveys include Pan-STARRS1
(PS1; \citealt{kaiser10a}), the Catalina Real-Team Transient Survey
(CRTS; \citealt{drake09a}), and the Palomar Transient Factory
(PTF; \citealt{law09a}), to be followed by the Zwicky Transient
Factory (ZTF; \citealt{bellm14a, smith14a}). In the 2020s, the Large
Synoptic Survey Telescope (LSST; \citealt{lsst09a}) will provide an
unprecedented number of transients and variable stars and quasars. The
study of variable sources will improve our understanding of
fundamental processes regarding the evolution of astrophysical
objects.  Accreting supermassive black holes, manifesting themselves
as active galactic nuclei, quasars, and blazars, often vary by tens of
percent or more in the optical on month- to year-long time
scales. Stellar variability reveals magnetic activity on stellar
surfaces, interactions between members of binaries, and pulsations.

To physically characterize the variable objects in these surveys, a
number of targeted programs have conducted spectroscopy on selected
variable types such as quasars, RR Lyrae stars, subdwarfs, white
dwarfs, and binaries (e.g., \citealt{geier11a, palanquedelabrouille11a,
rebassamansergas11a, badenes13a, drake13a}). The aim of TDSS
is to conduct a large-scale, statistically complete survey of all
variable types, without an imposed bias to either color or specific
light-curve character. This survey provides critical information
necessary to map photometric variability properties onto physical
classifications for currently ongoing projects, and future endeavors
such as LSST.

TDSS is creating a sample of single-epoch spectroscopy of 200,000
variable sources selected from PS1 over the 7,500 deg$^2$ of eBOSS;
about 140,000 of these are selected already for eBOSS or have had
spectra in SDSS-I/II/III. For a subset of selected objects ($\sim
10,000$) TDSS is conducting few-epoch spectroscopy (two to three
visits over the duration of SDSS-IV) to use spectroscopic variability
to characterize the objects.

\subsubsection{TDSS Target Selection} 
\label{sec:tdss:targeting}

\citet{morganson15a} describes the target selection for TDSS single-epoch
spectroscopy, and \citet{ruan16a} and describes early spectroscopic
results.  In brief, $griz$ imaging is used to select targets from SDSS
DR9 and PS1. SDSS data were taken between 1998 and 2009, with
typically only one epoch per observation. The PS1 3$\pi$ survey
acquired 10--15 epochs of imaging between 2010 and 2013.  TDSS uses
the SDSS-PS1 comparison as a measure of long-term variability, and the
variation among PS1 epochs as a measure of short-term
variability. Adopting the Stripe 82 database as a
testbed, \citet{morganson15a} developed an estimator $E$ related to
the probability of a specific source being variable based on the
short- and long-term variability, and the apparent magnitude. This
estimate is applied to a set of isolated point sources with $17<i<22$
and defined a threshold $E$ above which to select objects as likely
variables. Across most of the sky (80\%) TDSS randomly selects 10 targets
per deg$^2$ that pass this threshold and are not already eBOSS quasar
targets. In the remaining sky (20\%) there are fewer than 10 unique
targets that pass the threshold, and TDSS selects some targets at lower
$E$.

About 10\% of the fibers devoted to TDSS are dedicated to repeat
spectroscopy of previously known objects already having at least one
extant SDSS spectrum in the archive, and which are anticipated to
reveal astrophysically interesting spectral variablity with an
additional epoch or two of further spectroscopy. This few-epoch
spectroscopy was initially conducted in eight planned programs. The
subjects of these programs are: radial velocities of dwarf carbon
stars; M-dwarf/white dwarf binaries; active ultracool dwarfs; highly
variable ($>0.2$ mag) stars; broad absorption line quasars
(\citealt{grier16a}); Balmer-line variability in bright quasars
(\citealt{runnoe16a}); double-peaked broad emission-line quasars; and
Mg II velocity variability in quasars.

TDSS data is processed through the same pipeline that processes eBOSS
observations. Figure \ref{fig:eboss-spectra} displays an example
spectrum from the first year of TDSS: a variable broad absorption line
quasar selected for few-epoch spectroscopy.
