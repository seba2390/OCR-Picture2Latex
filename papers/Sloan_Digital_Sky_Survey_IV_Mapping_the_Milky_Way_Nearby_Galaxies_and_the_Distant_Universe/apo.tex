The 2.5~m Sloan Foundation Telescope at APO is a modified
two-corrector Ritchey--Chr{\'e}tien design, with a Gascoigne
astigmatic corrector, and a highly aspheric corrector designed for
spectroscopy near the focal plane.  It has a 3$^\circ$ diameter usable
field of view, and a focal ratio of $f/5$.  Commissioned during the
late 1990s, it has been acquiring survey data for the past 19 years.
It performed photometric imaging through 2009; for this purpose, there
was an alternative corrector near the focal plane designed for imaging
mode. It has performed multi-object fiber-fed spectroscopy through the
present, and is devoted to this task exclusive in SDSS-IV. The on-axis
focal plane scale is nominally 217.736 mm deg$^{-1}$.

The telescope system is maintained and operated throughout the year by
engineering and administrative staff plus a team of nine full-time
observers and two to three plate-pluggers.  On each night of
observing, two observers are on duty. The field change operation
involves the manipulation of the cartridges, which weigh 100--130
kilograms, on the telescope pier near the telescope, in dark, often
cold, and occasionally icy conditions. The presence of two observers
on site is necessary to ensure instrument and personnel safety. The
use of dedicated, full time employees as observers is necessary for
maintaining safe working conditions and contributes to the high
reliability of the system and the homogeneity of the resulting data
set.

We conduct multiplexed spectroscopic observations on the Sloan
Foundation Telescope in the following manner. Each day, the plugging
technicians prepare a set of cartridges with aluminum plates plugged
with optical fibers. Each plate corresponds to a specific field on the
sky to be observed at a specific hour angle. When the cartridge is
engaged on the telescope, the plate is bent to conform to the
telescope focal plane in the optical. Depending on the cartridge
configuration, the optical fibers feed either the BOSS optical
spectrographs (\citealt{smee13a}), the APOGEE spectrograph
(\citealt{wilson12a}), or both. The cartridges are initially staged in
a bay near the telescope and allowed to equilibrate with the outside
air temperature. During the night, the observers can swap the
cartridges efficiently so that a number of fields can be observed
throughout the night. \citet{dawson13a} provide a detailed description
of this procedure.  The APO observers submit observing reports each
morning and track time lost due to weather and technical problems on a
monthly basis. Technical issues have led to $<1\%$ time loss overall
over the past few years.

The system has seventeen cartridges used for spectroscopy. Eight have
1000 fibers that emanate at two slit heads (500 fibers each). The slit
heads directly interface with the two pairs of BOSS optical
spectrographs. Each pair consists of a red spectrograph and a blue
spectrograph that together cover the optical regime from 356 nm to
1040 nm, with $R\sim 1500$--$2500$. The fibers have 120 $\mu$m active
cores, which subtend 2'' on the sky.

The other nine cartridges contain 300 short fibers that are grouped in
sets of 30 into harnesses and terminate in US Conec MTP fiber
connectors.  The 10 fiber connectors are in turn grouped into a
precision gang connector that connects to a set of long ($\sim$40 m)
fibers extending from the telescope into the APOGEE instrument room
and terminating on the APOGEE spectrograph slit head. The APOGEE
instrument has a wavelength coverage of 1.5--1.7 $\mu$m, with $R\sim$
22,500. As in the case of BOSS, the fibers have 120 $\mu$m active
cores.  Through most of SDSS-III, there were eight APOGEE-2
cartridges; in early 2014, one BOSS cartridge was converted to an
APOGEE cartridge.

New in SDSS-IV, six of the nine APOGEE cartridges have an additional
short fiber system for MaNGA that interfaces with the BOSS
spectrographs (\citealt{drory15a}). The MaNGA fiber system consists of
17 IFUs and 12 mini-IFUs, plus 92 sky fibers, for 1423 fibers in
total. These fibers are spaced more densely on the spectrograph slit
head, which leads to a greater degree of blending between the spectra;
this blending is more tolerable in MaNGA than in BOSS because
neighboring MaNGA spectra on the spectrograph are also neighboring on
the sky, which reduces the dynamic range in flux between neighboring
spectra. As is the case for the APOGEE and BOSS systems, all of these
fibers have \mbox{120 $\mu$m} active cores; however, the cladding and
buffer on the fibers were reduced to increase the filling factor of
the IFU.  The resulting spectra have nearly the same properties of
those taken with the BOSS spectrographs. These six cartridges are
capable of simultaneous APOGEE and MaNGA observations. The first MaNGA
cartridge was commissioned in 2014 March, and the final one became
operational in 2015 January. Section \ref{sec:manga} describes the
system and its use in more detail.

In addition to the science fibers, each cartridge contains a set of 16
coherent fiber bundles that are plugged into holes centered on bright
stars and are routed to a guide camera that functions at visible
wavelengths ($\sim 5500$ \AA). The operations software uses the guide
camera feedback to control telescope position, rotator position, and
focal plane scale. During APOGEE observations, the guiding software
accounts for the chromatic differential refraction between visible
wavelengths and APOGEE wavelengths in order to best align the APOGEE
fibers with the images in the focal plane at 1.66 $\mu$m.

A special purpose fiber connection exists between the NMSU 1~m
Telescope and the APOGEE spectrograph. Seven fibers are deployed in
the NMSU 1~m focal plane in a fixed pattern; one fiber is used for
a science target and the remainder for sky measurements. This mode can
be activated when the APOGEE spectrograph is not being used by the
Sloan Foundation Telescope.

A database ({\tt apodb}) at APO tracks the status and location of all
plates and cartridges. An automatic scheduling program ({\tt
autoscheduler}) determines which plates should be plugged or observed
at any given time. The pluggers and observers use a web application
({\tt Petunia}) to interface with the database and view autoscheduler
output.  Occasionally, human intervention and re-prioritization of the
automatic schedule is required; this action is performed by {\tt
Petunia}.  The observers use a graphical user interface ({\tt STUI})
to send commands to and receive feedback from the operations software
controlling the telescope and instruments.

In SDSS-IV, APOGEE-2N, MaNGA, and eBOSS share the APO observing time
from 2014 July 1 to 2020 June 30. The observatory functions all year
except for the summer shutdown period, a roughly six-week hiatus for
engineering and maintenance in July and August, during the season with
the worst weather for observing.  Major engineering work is scheduled
for this period. The baseline plan for observations allocates the
bright time to APOGEE-2 and splits the dark time between eBOSS and
MaNGA; the exact allocations are adjusted to best achieve the overall
science goals depending on progress during the survey. We describe
here the baseline plan at the start of the survey. The overall number
of hours available in the survey is 18,826 (excluding engineering
nights, typically taken at full moon). This number (and those below)
assumes uneventful recommissioning of the telescope after each summer
shutdown.

APOGEE-2 uses the 8,424 of those hours that are deemed bright time,
because the APOGEE-2 observations are of sources typically much
brighter than the sky background. We define bright time as when the
moon is illuminated more than 35\% and is above the horizon. For
APOGEE-2, science observations occur between 8$^\circ$ twilight in the
``summer'' (roughly between the vernal and autumnal equinoxes) and
between 12$^\circ$ twilight in the ``winter,'' to avoid overworking
the observers. In the ``summer'' period, APOGEE-2 also utilizes dark
time in the morning twilight between 15$^\circ$ and 8$^\circ$, which
eBOSS and MaNGA cannot use.

eBOSS and MaNGA use the remaining hours, when the moon is below the
horizon or illuminated at less than 35\%.  For these dark time
programs, science observations occur between 15$^\circ$
twilights. Although eBOSS and MaNGA split the effective observing time
in SDSS-IV, in practice, the implementation is complicated by
observational limitations. MaNGA requires the bulk of its time to be
spent when the NGC is observable. Because MaNGA target selection is
based on the Legacy spectroscopic survey, it has available 7,500
deg$^2$ of targeting in the NGC but only 500 deg$^2$ in three isolated
stripes in the SGC \citep{abazajian09a}. Providing sufficient
targeting, and assuring that three-dimensional environmental
information is available for each target, requires MaNGA to be
NGC-focused and eBOSS to be SGC-focused. In addition, the SGC is more
difficult to observe because of Galactic dust foregrounds. Therefore,
in accounting for the time balance between eBOSS and MaNGA, 1.4 hr
of SGC dark time is effectively equivalent to 1.0 hr of NGC dark
time. As a result, eBOSS is assigned 5,497 hr and MaNGA 4,904
hr.

The inital time allocation for the three surveys as a function of
Local Sidereal Time (LST) is shown in Table~\ref{tab:timeallocation}.

\begin{table}[htp]
\centering
\caption{
\label{tab:timeallocation}
Initial allocations for SDSS-IV APO programs.}
\begin{tabular}{l c c c}
\hline\hline
LST (Hours) & \multicolumn{3}{c}{Time Allocated (Hours)}\\
  & APOGEE-2 & MaNGA & eBOSS \\ \hline
0--1 & 322.9 & 22.4 & 423.0 \cr
1--2 & 350.0 & 55.1 & 434.0 \cr
2--3 & 372.1 & 99.5 & 409.4 \cr
3--4 & 377.9 & 168.7 & 337.1 \cr
4--5 & 375.8 & 215.6 & 290.5 \cr
5--6 & 377.3 & 225.1 & 279.4 \cr
6--7 & 373.7 & 239.5 & 261.4 \cr
7--8 & 373.2 & 283.8 & 219.3 \cr
8--9 & 379.0 & 296.3 & 202.5 \cr
9--10 & 377.2 & 287.5 & 210.4 \cr
10--11 & 385.2 & 284.0 & 206.5 \cr
11--12 & 384.1 & 308.1 & 177.8 \cr
12--13 & 388.8 & 291.5 & 185.6 \cr
13--14 & 388.1 & 273.3 & 193.2 \cr
14--15 & 390.7 & 317.9 & 135.1 \cr
15--16 & 380.7 & 351.3 & 72.9 \cr
16--17 & 339.6 & 284.6 & 67.0 \cr
17--18 & 316.3 & 230.2 & 49.8 \cr
18--19 & 316.8 & 212.5 & 65.8 \cr
19--20 & 294.0 & 171.4 & 134.6 \cr
20--21 & 288.6 & 132.3 & 194.7 \cr
21--22 & 285.2 & 96.0 & 250.7 \cr
22--23 & 288.7 & 40.4 & 319.0 \cr
23--24 & 298.0 & 16.8 & 377.6 \cr
\hline
\end{tabular}
\end{table}
