\subsubsection{SPIDERS Motivation}

Within the main eBOSS program of quasars and LRGs, an average of 50
fibers per plate are allocated to sources associated with X-ray
emission, primarily AGNs and cluster galaxies. The goal of these
observations are twofold: first, to obtain a statistically complete
sample of X-ray emitting accreting black holes to better understand
quasar evolution and physics; second, to obtain redshifts and velocity
dispersions for a large sample of X-ray clusters.  The samples are
defined using the ROSAT All-Sky Survey
(RASS; \citealt{voges99a,boller16a}), the XMM Slew Survey
(XMMSL; \citealt{warwick12a}), and the upcoming eROSITA instrument
(\citealt{merloni12a}). In total, 22,000 spectra of X-ray emitting
AGNs
will be acquired, about 25\% of which will be targets in common with
the eBOSS cosmological program, and redshifts of about 58,000 galaxies
in 5,000 galaxy clusters.

SPIDERS uses this X-ray census of AGNs to better understand the
relationships among the growth of galaxies, the growth of their
central black holes, and the growth of their dark matter halos;
Section \ref{sec:quasars} describes these goals in more detail. The
SPIDERS cluster sample better establishes cluster scaling relations
and their evolution, and to use them to constrain cosmological
parameters through the evolution of the cluster mass function
(\citealt{allen11a, weinberg13a}).  For all of these science goals,
the existing statistically complete X-ray selected samples are too
small; they consist primarily of the sample of RASS sources observed
in SDSS-I and -II (\citealt{anderson03a}) and of much narrower
field of view and deeper observations in, for example, COSMOS
(\citealt{cappelluti09a, civano16a}), AEGIS (\citealt{laird09a,
nandra15a}), CDFS (\citealt{luo08a, xue11a}), and XBo{\"o}tes
(\citealt{kenter05a, murray05a}). Systematic, moderate resolution
spectroscopic follow-up of large area X-ray surveys, which sample
massive galaxy clusters and the bright end of the AGN luminosity
function, are currently lacking, and can yield important insights into
demographics, evolution, and physical characteristics of galaxies in
the densest large-scale structure environments, and of AGNs, including
the obscured populations.

\subsubsection{SPIDERS Target Selection}

eROSITA's planned launch is in early 2018 and data will become
available in Fall 2018. The satellite will observe the whole sky every
six months, and over four years will produce a series of eight
successively deeper eROSITA All Sky X-ray Survey catalogs (eRASS:1
through eRASS:8). Given this timeline, the targeting strategy for
SPIDERS is divided into several tiers depending on the available data
at the time of observation.
\begin{itemize}
\item {\it Tier 0}: Prior to the availability of eRASS data, SPIDERS 
targets RASS and XMMSL targets. 
\item {\it Tier 1}: SPIDERS will begin targeting eROSITA data with 
eRASS:1, which will be a factor of four to five times deeper than RASS (for
point sources). eRASS:1 data is planned to be available in Fall 2018
and SDSS-IV observations can begin in early 2019.
\item {\it Tier 2}: eRASS:3 is planned to be available mid-2019, and 
SPIDERS will target it beginning late 2019.
\end{itemize}

SDSS-IV does not observe eRASS sources over the entire sky. The survey
only has access to sources in the half of the sky defined in
Galactic coordinates ($180^\circ<l<360^\circ$). This hemisphere is
accessible to the eROSITA-DE consortium, with which SDSS-IV has a data
sharing agreement. Under current plans, the other half of the sky
is accessible only to the Russian eROSITA consortium.

For Tier 0 point sources, RASS identifies on average 3 deg$^{-2}$, of
which about 0.8 deg$^{-2}$ are not previously observed
spectroscopically and not too bright to observe within an eBOSS
exposure (which means, typically, $r>17$).  The uncertainty in the
coordinates of each point source is about 20$''$--30$''$, making the
identification of optical counterparts challenging. The match to the
optical counterpart is performed in two steps: (1) the {\it WISE}
counterparts are found using a Bayesian method based on that
of \citet{budavari09a}, taking into account priors in color-magnitude
space; (2) counterparts in the SDSS DR9 imaging data are determined
with a simple positional match to the {\it WISE} coordinates.  XMMSL covers
about 50\% of the eBOSS area and provides an additional 0.2 deg$^{-2}$
new point sources on average. The selection of the RASS and XMMSL
point sources is limited at $r=22$ (Galactic extinction
corrected). Details of the targeting scheme for Tier 0 AGN will be
described in \citet{dwelly17a}.

For Tier 0 extended sources, the Constrain Dark Energy with X-ray
Clusters (CODEX) team has identified photon overdensities in RASS that
correspond to galaxy clusters (\citealt{finoguenov12a}). These
clusters, plus Planck-detected clusters, have been matched to likely
cluster members using SDSS DR9 imaging, specifically using the
red-sequence Matched-filter Probabilistic Percolation method
(redMaPPer; \citealt{rykoff14a}). There are about 5,000 such clusters
within the eBOSS footprint. In addition, $\sim 300$ clusters are
identified serendipitously by XMM and also matched to DR9
(XCLASS; \citealt{clerc12a, sadibekova14a}). SPIDERS targets cluster
galaxies down to $i_{\rm fiber}=21$ (Galactic extinction
corrected). From these cluster samples, there is a target density of
up to 20 deg$^{-2}$ on average; because these targets are concentrated
in dense clusters and are subject to fiber collisions, only 7--8
deg$^{-2}$ are assigned fibers. When including previous SDSS legacy
spectroscopic observations, SPIDERS reaches a median of approximately
10 galaxies per cluster with spectroscopic redshifts. Details of the
clusters targeting algorithms and of the analysis steps are presented
in Clerc et al.~(2016).

For Tiers 1 and 2 point sources (AGN), eRASS:1 and eRASS:3 will be
matched to SDSS DR9 imaging. We will target AGNs with $17<r<22$. In the
eROSITA-DE sky area, this procedure will yield about 4,000 targets in
eRASS:1 and 7,000 in eRASS:3 that are not already targeted by
eBOSS. Including both eBOSS and SPIDERS, there will be $\sim 15,000$
eROSITA-detected AGNs with optical spectra from SDSS-IV.
%Note: I am assuming here two seasons of NGC eBOSS observations (year
%5 and 6), at about 500 deg^2 of useful area coverage per season

For Tiers 1 and 2 extended sources (clusters), member galaxies will be
identified using the same methods as for CODEX and XCLASS, but the
improved spatial resolution and depth of eRASS relative to RASS will
allow the targeting of intrinsically less massive and/or more distant
clusters. The number of galaxies assigned fibers per cluster range
from 1 to 10 depending on distance and cluster richness. Based on
estimated cluster counts in eROSITA simulations, SPIDERS expects
target densities of 7 deg$^{-2}$ in eRASS:1 and 10 deg$^{-2}$ in
eRASS:3.

SPIDERS data are processed through the same pipeline that
processes eBOSS data. Figure \ref{fig:eboss-spectra} shows an example
spectrum from the first year of SPIDERS: an AGN selected as an X-ray
emitter in RASS.
