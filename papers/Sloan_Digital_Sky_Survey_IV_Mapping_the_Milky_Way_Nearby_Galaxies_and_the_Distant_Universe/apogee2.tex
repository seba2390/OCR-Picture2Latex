\subsection{APOGEE-2 Motivation} 

APOGEE-2 is conducting high-resolution, high signal-to-noise ratio
spectroscopy in the near infrared for a large sample of Milky Way
stars.  A key challenge in astrophysics is the characterization of the
archeological record, chemical evolution, dynamics, and flows of mass
and energy within galaxies. The Milky Way provides a unique
opportunity to examine these processes in detail, star-by-star. Large
spectroscopic samples are critical for mapping the Galaxy's numerous
spatial, chemical, and kinematic Galactic sub-populations.

APOGEE-2 is creating a Galactic archeology sample designed to
understand the history of all components of the Milky Way, including
the dust-obscured ones (Fig.~\ref{fig:apogee2:overview}), and to
better understand the stellar astrophysics necessary to uncover that
history. APOGEE-2 is accomplishing this goal by continuing the overall
strategy of APOGEE-1 (\citealt{zasowski13a, majewski15a}), increasing
to 400,000 the number of stars sampled, and expanding to cover the
inner Galaxy from the Southern Hemisphere. The primary sample is a set
of red giant branch stars that trace Galactic structure and
evolution. Several smaller sets of targets explore more specific
aspects of Galactic and stellar astrophysics.  These spectra yield
precise radial velocities, stellar parameters, and abundances of at
least 15 elements. The Sloan Foundation Telescope at APO and the du
Pont Telescope at LCO are mapping both hemispheres of the Milky Way.

APOGEE-2 is distinguished from all other Galactic archeology
experiments planned or in progress by its combination of high spectral
resolution, near infrared wavelength coverage, high signal-to-noise
ratio, homogeneity, dual-hemisphere capability, and large statistical
sample.  It improves upon other Milky Way spectroscopic surveys that
lack the combined high resolution and $S/N$ needed by current
methodology for the determination of accurate stellar parameters and
chemical abundances (RAVE, \citealt{steinmetz06a, kordopatis13a};
BRAVA, \citealt{howard08a}; SEGUE-1 and SEGUE-2, \citealt{yanny09a};
ARGOS, \citealt{freeman13a}; and LAMOST, \citealt{cui12a,
  zhao12a}). APOGEE-2 complements existing or future wide-angle,
high-resolution stellar spectroscopic surveys or instruments that are
single-hemisphere and are optical, experiencing heavy dust extinction
at low Galactic latitudes and in the inner Galaxy (GALAH,
\citealt{zucker12a, desilva15a}; {\it Gaia}-ESO,
\citealt{gilmore12a}; WEAVE, \citealt{dalton14a}; 4MOST,
\citealt{dejong14a}). MOONS (\citealt{cirasuolo14a}) is the closest
analog and is complementary in ambition; it is a near-infrared
instrument under construction for the Very Large Telescope in the
Southern Hemisphere, with a larger number of fibers (1024) and
telescope aperture size (8.2~m), but twenty times smaller
field of view (500 arcmin$^2$).

Like other high resolution surveys and instruments, APOGEE-2
complements the optical {\it Gaia} satellite measurements of parallax,
proper motion, and spectroscopy of a much larger number of stars
(\citealt{prusti16a}). APOGEE-2 will benefit from the accurate
measurements of distance and proper motion from {\it Gaia} for its
stars. Our understanding of the Galactic chemical and dynamical
structure will be strengthened using the APOGEE-2 information
available for these stars: more precise radial velocities, more
precise stellar atmospheric parameters, and more precise abundances
for a larger set of elements.

\subsection{APOGEE-2 Science}

The combined APOGEE-1 and APOGEE-2 data sets yield multi-element
chemical abundances and kinematic information for stars from the inner
bulge out to the more distant halo in all longitudinal directions and
include both Galactic satellites and star clusters.  To effectively
exploit these data, APOGEE-2 is collecting additional observations on
fundamental aspects of stellar physics necessary to promote the
overall understanding of the formation of the Galaxy.

Near-infrared spectra are excellent for studies of stars in the
Galactic disk and bulge. The bulk of these regions suffer high
extinction from foreground dust in the visible, with regions in the
Galactic plane frequently yielding $A_{V} > 10$ \citep{nidever12a}.
With $A_{H}/A_{V} \sim 0.16$, NIR observations can peer through the
dust far more efficiently than optical data. The \Hband\ is rich in
stellar atomic (e.g., Fe, Ti, Si, Mg, and Ca) and molecular (e.g., CO, OH,
CN) absorption lines that can be used to determine stellar properties
and elemental abundances \citep{meszaros13a, holtzman15a,
  shetrone15a}. In particular, lines in the \Hband\ are sensitive to
the most common metals in the universe, C, N, and O, which are
difficult to measure in the optical.  The luminous red giant branch
(RGB) population dominates useful source catalogs like 2MASS, and
selecting targets by \Hband\ flux and red $J-K_s$ color
yields a population relatively unbiased in age and metallicity.

As shown in APOGEE-1 \citep{holtzman15a} typical APOGEE-2 spectra
enable measurements of at least 15 separate chemical abundances with
0.1 dex precision and high precision radial velocities (better than
100~m~s$^{-1}$).  The final spectra are the result of coadding several
observations spaced up to a month or more apart; these time series
data can identify radial velocity variables and detect interesting
binaries and substellar companions.

APOGEE-2's magnitude and color selection criteria result in a main
survey sample dominated by distant red giant, subgiant, and red clump
stars, but with some contribution from nearby late-type dwarf stars.
Through the inclusion of supplementary science programs, the final
APOGEE-2 program also includes observations of RR Lyrae stars,
high-mass and early main-sequence objects, as well as
pre-main-sequence stars. Combined, these programs will address a
number of topics in Galactic and stellar astrophysics.

\begin{itemize}
\item Mapping of the thick and thin disk at all Galactic
  longitudes, including the inner disk regions, and at the full range
  of Galactic radii, with substantial samples at least 6 kpc from the
  sun and with a significant subsample having reliably determined
  ages. These maps expand upon APOGEE-1 results
  (\citealt{anders14a, nidever14a, hayden15a}), and further test
  scenarios of inside-out growth, radial migration, and the origin of
  the $\alpha$-enriched population (\citealt{chiappini15a, martig15a,
    bovy16c}).

\item 
  Accurate stellar ages and masses from the combination of APOGEE data
  with asteroseismology (e.g., \citealt{epstein14a, chiappini15a,
    martig15a}), establishing critical benchmarks in the analysis of
  Galactic chemistry and dynamics in numerous directions sampled by
  {\it Kepler} and its subsequent K2 mission.

\item Dynamics of the disk and the Galactic rotation curve, including
  non-axisymmetric influences of the bar and spiral arms (e.g.,
  \citealt{bovy12a, bovy15a}).

\item Three-dimensional mapping of the Galactic bulge and bar, measuring
  dynamics of the bar, bulge, and nuclear disk (\citealt{nidever12a,
    schonrich15a, ness16a}) and their chemistry
  (\citealt{garciaperez13a, ness15a}).  Southern Hemisphere operations
  as well as the inclusion of standard candles such as red clump and
  RR Lyrae stars will make this mapping more complete and precise than
  APOGEE-1.

\item Chemistry and dynamics in the inner and outer halo across all
  Galactic longitudes, including a large area of the NGC, and sampling
  known halo substructure and stars reaching to at least 25 kpc.

\item Stellar populations, chemistry and dynamics of nascent star
  clusters, open clusters, globular clusters at various evolutionary
  stages, dwarf spheroidals, the Magellanic Clouds, and other
  important components of the Milky Way system (e.g.,
  \citealt{frinchaboy13a, majewski13a, meszaros13a, cottaar14a,
    cottaar15a, foster15a, garciahernandez15a, meszaros15a, bovy16b}).

\item Exoplanet host observations in {\it Kepler} fields to
  characterize host versus non-host properties and assess false
  positive rates (\citealt{fleming15a}).

\item 
  Detection of stellar companions of stellar, brown dwarf, and
  planetary mass across the Galaxy (e.g., \citealt{troup16a}).

\item Mapping the interstellar medium using Diffuse Interstellar Bands
  (\citealt{zasowski15b, zasowski15a}), or dust reddening effects
  (\citealt{schultheis14a}).

\end{itemize}

APOGEE-2 is also pursuing ancillary science programs with a small
fraction of the available fibers to utilize more targeted and
exploratory uses of the APOGEE instruments.

\subsection{APOGEE-2 Hardware}

APOGEE-2 utilizes one existing spectrograph at APO
(\citealt{eisenstein11a, wilson12a, majewski15a}) and a second
instrument at LCO. Each spectrograph is fed with 300 fibers with 120
$\mu$m cores; both yield nearly complete spectral coverage between
1.51 $\mu$m $< \lambda < $ 1.70 $\mu$m, high spectral resolution
($R\sim$ 22,500, as measured for the first spectrograph) and high S/N
($> 100 $ pixel$^{-1}$) for most targets \citep{majewski15a}. The
APOGEE spectrographs each utilize a large mosaic volume-phase
holographic (VPH) grating.  At APO, the first spectrograph's VPH
grating consisted of three aligned panels on the same substrate. The
spectrograph cameras consist of four monocrystalline silicon lenses
and two fused silica lenses. The spectra are dispersed onto three
Teledyne H2RG array detectors with 18 $\mu$m pixels, sampling three
adjacent spectral ranges; all elements of each array are sampled
``up-the-ramp'' at 10.7 second intervals within each exposure.  This
procedure yields an effective detector read-noise of $\sim$10 e$^-$
per pixel.  The geometric demagnification of the camera and collimator
optics delivers slightly over 2 pixels sampling of the fiber diameter
in the spatial dimension, but the spectra are slightly undersampled in
the blue part of the spectrum. To fully sample the spectra, the three
detectors are dithered by a half pixel in the spectral dimension
between exposures, which therefore are routinely taken in pairs. The
measured throughput of the APOGEE-1 instrument is $20\pm2$\%
\citep{majewski15a}.

At APO, the spectrograph is fed by long fibers extending from the
Sloan Foundation Telescope and the NMSU 1-meter Telescope, as
described in Section \ref{sec:apo}.  The NMSU 1-meter Telescope is
used to observe bright stars, such as previously well-characterized
spectral standards and HIPPARCOS targets (\citealt{feuillet16a}), when
the spectrograph is not otherwise in use with the Sloan Telescope.

The APOGEE-South spectrograph at LCO is a near-clone of the APOGEE
spectrograph with some slight differences. First, the mosaic VPH
grating uses two panels instead of three, a simplification with
negligible impact on the net instrument throughput. Nevertheless, the
pair of panel exposures were not perfectly aligned; therefore, an
optical wedge is added to compensate for this misalignment to optimize
spectral resolution.  Second, the spectrograph optical bench is
mounted within the instrument cryostat with greater consideration of
seismic events, given its location in Chile.  Other more minor
modifications in the optical bench and cryostat configuration have
been adopted as well.

We anticipate that the data from the second spectrograph will be, in
most respects, quite similar to those from the original.  The fibers
will typically have lower sky backgrounds because they subtend a
smaller angular size. In addition, the du Pont optical correctors have
less loss in the \Hband, which is $\sim 40\%$ on the Sloan Foundation
Telescope.

The APOGEE-South spectrograph was installed at the du Pont Telescope
in 2017 February and survey operations are planned to start soon
thereafter.

\begin{figure}[t!]
\centering
\includegraphics[width=0.49\textwidth,
  angle=0]{apogee2-footprint.pdf}
\caption{ \label{fig:apogee2:overview} APOGEE-1 and planned APOGEE-2
  spectroscopic footprint in equatorial coordinates, centered at
  $\alpha_{\rm J2000}=270^\circ$, with East to the left.  Black shows
  APOGEE-1 data, orange indicates APOGEE-2N, and red is APOGEE-2S.
  Blue shows projected MaNGA coverage for which APOGEE-2 can
  potentially have observations of stars (see also Figure
  \ref{fig:manga-footprint}). Because of logistical constraints and
  potential changes in the MaNGA plans, the final coverage of the halo
  may differ somewhat from this figure.}
\end{figure}

\subsection{APOGEE-2 Targeting and Observing Strategy}
\label{sec:apogee2:targeting}

APOGEE-2 continues much of the observational strategy for APOGEE-1
(\citealt{zasowski13a}).  Its standard targeting uses the 2MASS
survey, selecting stars based on dereddened $J-K_s$.  Additional
information from the Optical Gravitational Lensing Experiment (e.g.,
OGLE-III and OGLE-IV; \citealt{udalski08a, udalski15a}), Vista
Variables in the Via Lactea (VVV: \citealt{minniti10a, saito12a,
  hempel14a}), and the VVV Extended ESO Public Survey (VVVX) surveys
are incorporated for certain subsamples. Dereddened magnitude limits
range from $H=12.2$ to $13.8$ mag (depending on cohort, as explained
below) for the bright-time observations, and are $H=11.5$ during
co-observing with MaNGA.

To estimate extinction in the disk and bulge, APOGEE-2 supplements
2MASS imaging with the Spitzer-IRAC Galactic Legacy Infrared Mid-Plane
Survey Extraordinaire and extensions (GLIMPSE; \citealt{benjamin03a,
  churchwell09a}).  Where GLIMPSE data are not available, APOGEE-2
uses data from the all-sky {\it WISE}
mission (\citealt{wright10a}).  The reddening estimates employ the
Rayleigh-Jeans Color Excess method (\citealt{zasowski09a,
  majewski11a}).

To efficiently separate dwarfs and giants in the stellar halo,
APOGEE-2 obtained Washington $M$ and $T_2$ and DDO~51 stellar
photometry using the Array Camera on the 1.3~m telescope of the
U.S. Naval Observatory in Flagstaff, with additional data anticipated
for the Magellanic Cloud targeting in the Southern survey
component. In the ($M-T_2$) versus ($M-{\rm DDO~51}$) color plane, dwarfs
and giants lie in distinct locations, which allows relatively clean
separation of these stellar classes (\citealt{geisler84b, munoz05a,
  zasowski13a}).

To collect sufficient signals on fainter stars while still acquiring
data on large numbers of brighter stars, APOGEE-1 and APOGEE-2 employ
a system of ``cohorts,'' groups of stars observed together for the
same length of time.  The 3-visit cohorts correspond to the brighter
magnitude limits ($H=12.2$) and the longer cohorts correspond to
deeper magnitude limits (down to $H=13.8$). Each 3$^\circ$ diameter
field on the sky is observed with one or more plate designs, each of
which consists of a combination of cohorts.  Stars are predominantly
divided into cohorts according to brightness, and observed
(``visited'') long enough to obtain the required S/N goals: typically
S/N $\sim$ 100 per half-resolution element for the core programs
sampling Milky Way giant stars; S/N $\sim$ 70 for some exceptional
target classes such as luminous stars in Local Group dSph and the
Magellanic clouds; and S/N $\sim$ 10 for RR Lyrae in the bulge. For
example, in a 12-visit field, ``short'' cohort stars are observed on 3
visits, ``medium'' cohort stars are observed on 6 visits, and ``long''
cohort stars are observed on all 12 visits.  \citet{zasowski13a}
provide additional examples. Each visit corresponds to 67 minutes of
exposure time in nominal conditions (see
\S\ref{sec:apogee2:observations} for further visit details), with
fields visited anywhere from 3 to 24 times.

Visits per field have cadences between 3 and 25 days. This strategy is
adopted to yield detections of spectroscopic variability, most
commonly velocity shifts due to binary companions with a typical
radial velocity precision of $\sim$100--200 m s$^{-1}$.  For stars
observed more than the nominal three visits, it is possible to detect
brown dwarf and planet mass companions (\citealt{fleming15a,
  troup16a}).

The APOGEE-2 observations are divided into northern and southern
components, and each of these are sub-divided into different target
classes identifying different Galactic regions or special target
classes. The sky coverage is summarized in Figure
\ref{fig:apogee2:overview}. The target categories summarized in
Table~\ref{tab:apogeetargeting}, providing the number of plates,
visits, and stars observed in each class from respective hemispheres
(N or S). All targeted stars will have observations yielding radial
velocities and stellar atmospheric parameters, but, depending on the
target faintness (e.g., giants in the Magellanic clouds) or type
(e.g., RR Lyrae), abundance information may only be partial or
unavailable, as noted.

\begin{figure}[t!]
\centering
\includegraphics[width=0.49\textwidth, angle=0]{apogee-limits.pdf}
\caption{ \label{fig:apogee2-galaxy} Map of APOGEE-1 and APOGEE-2
  distance limits at $b=0^\circ$ within the Galactic plane, compared
  to other Galactic plane spectroscopic surveys. These limits assume
  observations of stars at the tip of the red giant branch (for
  solar metallicity and 2 Gyr of age) using isochrones from
  \citet{bressan12a}. To calculate the distance limit, we use the dust
  extinction prescription of \citet{bovy16a} and limits of $H=12.2$
  for APOGEE-2, $V=14$ for GALAH, and $V=19$ for {\it Gaia}-ESO (their
  faintest limit across all fields). Longer cohorts in APOGEE-2 
  extend correspondingly further.}
\end{figure}

APOGEE-2N continues observations of red giant branch (RGB) and red
clump (RC) stars in the inner and outer Galactic disk, and of the
stellar halo in the NGC and in the SGC. The distance limits for this
sample in the Galactic plane ($b=0^\circ$) are shown in Figure
\ref{fig:apogee2-galaxy}. Some halo fields specifically target areas
with known tidal streams; these samples are anticipated to total
$\sim$58,000 stars.  Additional Galactic evolution programs target
dwarf spheroidals, as well as open and globular clusters. Because it
shares cartridges with MaNGA, APOGEE-2N is co-observing with MaNGA
during dark time. Due to the MaNGA observing strategy, these exposures
are typically three hours of integration.  However, MaNGA's dithers
mean a lower overall throughput (see Section
\ref{sec:apogee2:observations}) and therefore the magnitude limit in
these fields is $H=11.5$.  We anticipate an additional 120,000 stars,
primarily selected as red giants, in ``halo'' (i.e., high latitude)
fields. These locations are displayed in blue in Figure
\ref{fig:apogee2:overview}. These co-observed stars represent a
substantial increase in numbers of halo stars over what was possible
in APOGEE-1.

APOGEE-2 expands an ancillary APOGEE-1 program in the {\it Kepler}
satellite Cygnus field into a main survey objective including the
fields observed with the K2 mission.  Two main goals focus on
asteroseismology and gyrochronology targets and observations relating to
Kepler exoplanets. The APOKASC collaboration combines the resources of
APOGEE and the {\it Kepler} Asteroseismology Science Consortium (KASC)
to determine precise age and mass constraints on stars of a range of
stellar types (\citealt{pinsonneault14a}).  The {\it Kepler} Object of
Interest (KOI) program provides multi-epoch observations on five of
the modules in the original {\it Kepler} Cygnus field, targeting KOIs
to characterize planet-host versus control star properties as well as
to improve our understanding of the frequency of false positives
within the KOI sample.  In addition to these observations of the
primary Kepler field, APOGEE-2N is conducting a campaign of {\it
  Kepler} K2 fields, using the combined space
asteroseismology/gyrochronology plus APOGEE spectral data to determine
high-quality ages for stars in a wide range of Galactic directions.

The samples listed in Table~\ref{tab:apogeetargeting} complete
APOGEE-2's homogeneous sampling of all Galactic regions with the RGB
and RC survey. We are also targeting fainter stars from the upper RGB of
the LMC, SMC, and several dSphs, and probing the chemistry of open and
globular clusters. A new program observes RR Lyrae stars in the
bulge from OGLE-IV and VVV to measure the detailed structure and
kinematics of the ancient bulge.

Both the northern and the southern components also contain ancillary
program targets with a diverse range of science goals. These programs
include using low extinction windows to examine the far disk at
distances of over 15 kpc in the plane, measuring Cepheid metallicities
across the disk, characterizing young moving groups, determining the
detailed and precision abundance trends in clusters, and studying
massive AGB stars. APOGEE is also conducting an extensive
cross-calibration program between APOGEE, SEGUE, GALAH, and {\it
  Gaia}-ESO, and between the APOGEE and APOGEE-South spectrographs.

\begin{table}[htp]
\centering
\caption{
\label{tab:apogeetargeting}
APOGEE-2 Targeting Description}
\begin{tabular}{l c r r r l}
\hline\hline
%
Target      & N or S & $N_{\rm plate}$ & $N_{\rm visit}$ & $N_{\rm
  star}$ & Abundances \\ \hline
%
Clusters    & N &  31 &   63 &  2340 & complete \\
            & S &  63 &  158 &  8715 & complete \\
\multicolumn{6}{c}{} \\ [-0.1in]
Bulge       & N &   1 &   18 &   230 & complete \\
            & S & 213 &  321 & 38310 & complete \\
\multicolumn{6}{c}{} \\ [-0.1in]
Inner Disk  & N & 116 &  348 & 20010 & complete \\
Outer Disk  & N &  93 &  279 & 21390 & complete \\
Disk        & S & 179 &  537 & 30470 & complete \\
dSph        & N &  12 &   72 &   780 & partial \\
            & S &  12 &   72 &   780 & partial \\
\multicolumn{6}{c}{} \\ [-0.1in]
Halo-NGC    & N &  84 &  504 &  5460 & complete \\
            & S &   4 &   48 &   480 & complete \\
\multicolumn{6}{c}{} \\ [-0.1in]
Halo-SGC    & N &  28 &   87 &  6670 & complete \\
            & S &  24 &   72 &  5520 & complete \\
\multicolumn{6}{c}{} \\ [-0.1in]
Streams-NGC & N &  48 &  288 &  3840 & partial \\
Streams-SGC & N &   9 &   39 &  1410 & partial \\
            & S &   2 &   12 &   345 & partial \\
\multicolumn{6}{c}{} \\ [-0.1in]
APOKASC     & N &  56 &   56 & 12880 & complete \\
KOI         & N &   5 &   90 &  1150 & complete \\
Halo Co-obs & N & 600 &  600 &120000 & complete \\
LMC         & S &  51 &  153 &  4930 & partial \\
SMC         & S &  24 &   78 &  1920 & partial \\
SGR         & S &   4 &   30 &  1405 & complete \\
RRLyrae     & S &  31 &   31 &  4000 & ---\\
\multicolumn{6}{c}{} \\ [-0.1in]
TOTALS      & N &1084 & 2444 &196160 & \\
            & S & 607 & 1512 & 96875 & \\ 
\hline
\end{tabular}
\end{table}

\clearpage
\subsection{APOGEE-2 Observations}
\label{sec:apogee2:observations}

APOGEE-2N utilizes the bright time at APO. Details of the division of
observations across the SDSS-IV surveys at APO are given in Section
\ref{sec:apo}. APOGEE-2S primarily utilizes the bright time at
LCO, and conducts observations 75 nights each year. Section
\ref{sec:lco} describes the operational model; otherwise, APOGEE-2S
largely employs the same observing strategies as APOGEE-2N. 

Each APOGEE-2N fiber is encased in a metal ferrule whose tip is
relatively narrow at 2.154 mm and is inserted fully into the plate
hole, but whose base is around 3.722 in mm diameter and sits flat on
the back of the plate.  A buffer of 0.3 mm around each ferrule is
maintained to prevent plugging difficulty. Given the plate scale on
the Sloan Foundation Telescope, on the same plate no two APOGEE-2N
fibers can be separate by less than 72$''$ on the sky. As described in
Section \ref{sec:plates}, the APOGEE-2N holes are counterbored so that
the fiber tips lie on the \Hband\ focal plane.

Each APOGEE-2S fiber has a larger 3.25 mm tip and a 4.76 mm base. No
buffer is used around each ferrule. Given the plate scale of the du
Pont Telescope, on the same plate no two APOGEE-2S fibers can be
separated by less than 52$''$ on the sky. Because at LCO the plate is
curved to match the \Hband\ focal plane, there is no counterboring of
the APOGEE-2S plates.

Each plate is designed for a specific hour angle of observation.  The
observability window is designed such that no image falls more than
0.3$''$ from the fiber center during guiding.  These limits on the
LST of observation are slightly larger than for eBOSS because APOGEE-2
operates in the near-infrared where the refraction effects are
smaller. In addition, for APOGEE-2N, we add 30 minutes on either side
to ease scheduling constraints.

%With observations taken in both hemispheres, APOGEE-2 can greatly
%reduce the number of high airmass observations to special situations
%only, such as cross-calibration, K2 field observations, and bulge
%target testing. In the past, such high airmass observations
%(particularly of the bulge as observed from APO) reduced the useful
%FOV because of differential refraction.

An APOGEE-2 visit typically consists of eight 500~s exposures taken
in two ABBA sequences (a total of 66.7 minutes), where A and B are two
detector dither positions in the spectral dimension described above to
ensure critical sampling. Each exposure consists of 47 non-destructive
detector reads spaced every 10.7~s.  Each visit requires 20 minutes
overhead in cartridge changes, calibrations, and field acquisition.
Whereas in APOGEE-1 and the beginning of APOGEE-2, we had a fixed
number of exposures per visit, starting in 2016 we have adapted the
number of exposures based on the accumulated signal-to-noise ratio
relative to the requirement, as eBOSS and MaNGA do. This change allows
more efficient use of resources; initial estimates from the first few
months indicate that the net increase in the survey completion rate is
significant (roughly $15\%$).

During MaNGA time, APOGEE fibers are placed on APOGEE-2 targets.  The
MaNGA observations are dithered on the sky and their schedule
constrains the APOGEE exposures to have 10\% shorter exposure times
than the standard APOGEE exposures. Both of these effects lead to a
net throughput reduction per exposure of almost a factor of two; a
reduction of about 40\% due to the offset under typical seeing, and
about 10\% more due to the shorter exposure times. In some cases, the
MaNGA-led observing yields more than the standard number of APOGEE
exposures per field, but this is generally insufficient to compensate
for the reduced throughput per exposure. As a result, the faint limit
for targets on the MaNGA-led co-observing plates is chosen to be
$\sim$0.7 mag brighter than it is for standard APOGEE plates ($H<11.5$
instead of $H<12.2$), so that the standard APOGEE signal-to-noise
ratio requirement is met for targets in the MaNGA fields.

\begin{figure*}[t!]
\centering
~\\
~\\
\includegraphics[width=0.99\textwidth, angle=0]{apogee2-spectra.pdf}
~\\
\includegraphics[width=0.99\textwidth,
  angle=0]{apogee2-abundances.pdf}
\caption{ \label{fig:apogee2-data} Top panel: Several subregions of
  the full APOGEE spectra for seven stars of a range of metallicities,
  as labeled on the right (plotted using the software described in
  \citealt{bovy16b}). The black lines are the data; the red lines are
  the best-fit {\tt ASPCAP} model; the areas where the data are
  missing are masked due to sky contamination or other issues. Both
  data and model have been normalized to the pseudo-continuum
  $f_c(\lambda)$ (\citealt{holtzman15a}). Clean, strong lines
  identified by \citet{smith13a} are labeled.  Bottom panels:
  Elemental abundances relative to Fe for several of the species whose
  lines exist in the top panel, as a function of [Fe/H], for the
  APOGEE DR13 sample of 164,562 stars. APOGEE-2 can examine the major
  patterns as a function of Galactic location (e.g.,
  \citealt{nidever14a}, \citealt{hayden15a}).  \\ ~\\ ~\\ }
\end{figure*}

\subsection{APOGEE-2 Data}

The APOGEE-2 spectroscopic data consist of $R\sim 22,500$ spectra in
the $H$ band (1.51 $\mu$m $< \lambda < $ 1.70 $\mu$m), at high
signal-to-noise ratio ($> 100 $ pixel$^{-1}$) for most targets
\citep{majewski15a}. From these data, we determine radial velocities,
stellar parameters, and abundances. \citet{garciaperez16a},
\citet{holtzman15a}, and \citet{nidever15a} describe the APOGEE data
processing pipelines. The fundamentals remain unchanged for APOGEE-2,
and are summarized below.

The APOGEE Quicklook pipeline ({\tt apogeeql}) analyzes the
observations during each exposure to estimate the signal-to-noise
ratio and make decisions about continuing to subsequent exposures. The
observers use these data but they are not used for scientific
analysis.

Each morning, the APOGEE Reduction Pipeline ({\tt APRED}) produces
spectra for each new visit for the observed plates, extracting
individual spectra (\citealt{horne86a}).  Multiple exposures taken on
the same night are combined into ``visit'' spectra. In most cases,
multiple visits are made to each star, sometimes with the same plate
and sometimes with multiple plates. APOGEE-2 measures radial velocities from
each visit spectrum, aligns the spectra in their rest frame, and
creates a combined spectrum. 

The APOGEE Stellar Parameters and Chemical Abundance Pipeline ({\tt
  ASPCAP}) analyzes the combined spectrum.  This pipeline divides each
spectrum by a pseudo-continuum, and then performs two analyses.
First, {\tt ASPCAP} determines the key stellar parameters influencing
the spectrum --- effective temperature ($T_{\rm eff}$), surface
gravity ($\log g$), overall scaled-solar metal abundance [M/H],
$\alpha$-element abundance [$\alpha$/M], carbon abundance [C/M], and
nitrogen abundance [N/M] --- via optimization against a set of large,
multidimensional libraries of synthetic spectra (\citealt{zamora15a}).
{\tt ASPCAP} uses the FERRE\footnote{\tt
  http://github.com/callendeprieto/ferre} code to minimize $\chi^2$
differences between the pseudo-continuum-normalized spectrum and
synthesized stellar spectra interpolated from a precomputed grid
(\citealt{allendeprieto06a}). The synthetic spectra used in ASPCAP are
computed using the model atmospheres described by \citet{meszaros12a}
based on the ATLAS9\footnote{\tt
  http://www.iac.es/proyecto/ATLAS-APOGEE/} (\citealt{kurucz79a}) or
MARCS\footnote{\tt http://marcs.astro.uu.se} (\citealt{gustafsson08a})
model atmospheres. These models consider variations in carbon and the
$\alpha$ elements of $\pm 1$ dex from the solar abundance ratios. In
DR13 and DR14, the radiative transfer calculations are performed with
the code Turbospectrum (\citealt{alvarez98a, plez12a}). This code
differs from the code ASS$\epsilon$T (\citealt{koesterke09a}) used in
DR12, and includes an upgrade of the $H$-band atomic and molecular
line lists presented by \citet{shetrone15a}. In the fitting, we
usually tie the micro-turbulence ($v_{\rm micro}$) to the surface
gravity.  In the models, oxygen abundance is taken to scale with
$\alpha$.

Second, {\tt ASPCAP} performs a detailed chemical abundance
determination, conducting a series of one-dimensional parameter
searches for a set of 15 elements (C, N, O, Na, Mg, Al, Si, S, K, Ca,
Ti, V, Mn, Fe, and Ni).  For each element, a set of weighted regions
of the pseudo-continuum-normalized spectrum is compared to the models
(\citealt{garciaperez16a}).  The same underlying stellar parameter
grid is used for these searches as for the stellar parameter
determination. In each case $T_{\rm eff}$, $\log g$, and $v_{\rm
  micro}$ are fixed; only one metallicity parameter is varied. For C
and N, the [C/M] and [N/M] dimensions are varied, respectively; for O,
Mg, Si, S, Ca, and Ti, the [$\alpha$/M] dimension is varied; for Na,
Al, K, V, Mn, Fe, and Ni, the [M/H] dimension is varied. The spectroscopic
windows defined by \citet{garciaperez16a} are designed such that the
procedure in each case is sensitive primarily to the variation in the
desired element; the precise windows have changed since
DR12. Additional elemental abundances can be estimated from the
spectra and {\tt ASPCAP} is being developed over time to incorporate
these.

The {\tt ASPCAP} pipeline abundances are calibrated in several ways to
minimize systematic errors both internally and with respect to other
abundance scales. An internal temperature-dependent calibration of the
raw abundances returned by {\tt ASPCAP} is derived using the
assumption that abundances within open clusters and first-generation
stars in globular clusters (apart from C and N in giants) are
homogeneous (\citealt{desilva06a, desilva07a}). Some elements show
temperature-dependent abundance trends that are removed by this
calibration.  To improve the external accuracy, APOGEE-2 applies an
external correction that sets the median abundances of solar
metallicity stars ($-0.1<$[M/H]$<0.1$) near the solar circle to have
solar abundance ratios; this differs from DR12, where no external
correction was applied to quantities other than [M/H].  After this
calibration, most abundances have a typical precision near 0.05 dex,
though uncertainties for some elements with just a few weak lines
can be considerably larger; in detail, the precision is a function of
effective temperature, metallicity, and signal-to-noise.

The top panel of Figure \ref{fig:apogee2-data} displays several
spectra of varying metallicities from APOGEE-2 along with the best-fit
{\tt ASPCAP} model. The bottom panel presents the distribution of
several abundance ratios within the sample.

The first SDSS-IV data release (DR13; 2016 July) contains a
rereduction of APOGEE-1 data through the latest version of the
pipeline. In DR14 (summer 2017) the first two years of APOGEE-2 data
will be released.
