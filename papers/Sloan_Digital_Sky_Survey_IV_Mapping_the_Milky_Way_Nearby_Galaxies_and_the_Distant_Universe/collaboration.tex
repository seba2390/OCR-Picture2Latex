The Science Collaboration is led by the Project Spokesperson, who is
elected for a three-year term by the collaboration. A Collaboration
Council consisting of representatives from the participating
institutions advises the Spokesperson. The Spokesperson and the
Collaboration Council developed the Publication Policy for SDSS-IV.

Following previous SDSS collaborations, the Publication Policy's
guiding principle is that all participants can pursue any project so
long as they notify the entire collaboration of their plans and update
the collaboration as projects progress.  Groups pursuing similar
science projects are encouraged to collaborate, but they are not
required to do so. There is no binding internal refereeing process.
Instead, draft publications using non-public data must be posted to
the whole collaboration for a review period of at least three weeks
prior to submission to any journal or online archive.  Participants
outside of the core analysis team may request co-authorship on a paper
if they played a significant role in producing the data or analysis
tools that enabled it. Scientists who have contributed at least one
year of effort to SDSS-IV infrastructure development or operations can
request ``Architect'' status, which entitles them to request
co-authorship on any science publications for those surveys to which
they contributed.  All SDSS-IV authorship requests are expected to
comply with the professional guidelines of the American Physical
Society.

Each of the SDSS-IV programs has Science Working Groups to coordinate
and promote scientific collaboration within the team.  These working
groups overlap and interact with the SDSS-IV project personnel but are
more focused on science analysis.  The working groups communicate and
collaborate through archived e-mail lists, wiki pages, regular
teleconferences, and in-person meetings.  Importantly, the science
activities of these working groups are not funded by the SDSS-IV
project.

The policies of SDSS-IV allow limited proprietary data rights to
astronomers outside the collaboration under specific conditions that
fall into two categories. First, when an SDSS-IV member leaves for a
non-SDSS institution. The member can ask the Collaboration Council and
Management Committee for ``Continuing External Collaborator'' status
to complete a defined scientific investigation that had been
substantially started before the change in institutions. Second, if
crucial skills to complete science of interest to SDSS-IV members are
not available within the SDSS-IV collaboration, due to either
personnel or time constraints, SDSS-IV members can ask the
collaboration, with approval from the Collaboration Council and
Management Committee, for ``External Collaborator'' status for
non-SDSS members to work on specific aspects of declared projects. The
collaboration evaluates whether the contributions of the non-members
are unique and necessary to produce cutting-edge science from the SDSS
collaboration for a limited number of papers.

The projects, publications, and other activities are tracked in a
central database as part of the SDSS-IV data system. Collaboration
members use a web application to interact with this internal database.
This system lends clarity to the status of approvals and decisions
with regard to internal collaboration activities.
