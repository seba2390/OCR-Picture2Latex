The governance and management structure of SDSS-IV continues the
highly successful structure developed over its previous phases.
SDSS-IV is ultimately overseen by the Astrophysical Research
Consortium (ARC) and its Board of Governors.  The ARC Board has
established a set of SDSS-IV Principles of
Operations\footnote{See \link{http://www.sdss.org/collaboration/}.}
which provides the governance and management structure of the project.

Institutions join the collaboration via contributions, both technical
and financial, committed to through Memoranda of Understanding
(MOUs). Scientists at these institutions have data rights to all of
the SDSS-IV surveys. ``Full membership'' yields data rights for all
employees at an institution. ``Associate membership,'' which requires
a smaller contribution, yields data rights for a limited number of
scientists. Technical contributions must directly address items in the
survey budget.

The ARC Board has established an Advisory Council (AC) that oversees
the Director and the project. The AC consists of representatives from
the member institutions. It approves each new MOU and has authority
over significant changes in policy, changes in the project scope, and
fundraising activities. 

Figure \ref{fig:orgchart} shows the high-level organizational chart.
The management structure is designed to unify decision-making and
establish clear lines of authority for the allocation of resources by
the Central Project Office.

\begin{figure*}[t!]
\centering
\includegraphics[width=0.98\textwidth, angle=0]{sdss4-orgchart.pdf}
\caption{ High-level organizational chart for SDSS-IV, as of 
2017 February. Positions have rotated somewhat during the project and will
continue to do so.}
\label{fig:orgchart}
\end{figure*}

The Central Project Office contains the Director, the Project
Scientist, the Program Manager, and the Project Spokesperson. The
Director makes spending, budget, and fundraising decisions, and
resolves decision-making conflicts.  The Project Scientist's role is
to ensure the scientific quality and integrity of the project, through
reviews of the scientific plans and products.  The Program Manager is
the full-time manager of the project, tracking the schedule and
project personnel issues. The co-Chairs of Education and Public
Engagement and the FAST Science Liaison are part of the Central
Project.

The Project Spokesperson is the leader of the Science Collaboration
and represents SDSS-IV to the scientific community. The Science
Collaboration is described more fully in the next subsection.

The leadership teams of each core program in SDSS-IV (APOGEE-2, eBOSS,
MaNGA) have a common structure. Each program has a Principal
Investigator (PI), a Survey Scientist, and an Instrument Scientist.
The PI is responsible for leading each survey, both scientifically and
in terms of its management.  The Survey Scientist is responsible for
the proper execution of the survey. The particular focus of the Survey
Scientist differs from survey to survey, and ranges from overall
scientific strategy to pipeline development.  The Instrument Scientist
is responsible for the development and maintenance of the
instrument. For eBOSS, the instrument is stable and not under
development; in this case the Instrument Scientist takes on many of
the operational tasks. For MaNGA and APOGEE-2, which have major
hardware upgrades and development, the instrument scientists are much
more focused on that development.  For the same reason, MaNGA and
APOGEE-2 have Project Managers to lead the hardware construction.
SPIDERS and TDSS each have PIs but not the other leadership positions.

Several positions exist to support common goals and coordination.  The
Data Management team leads the data management and distribution.  A
Survey Coordinator plans and monitors the survey observational
strategy. The co-Chairs of Education and Public Engagement lead a
committee coordinating the development of educational materials and
public engagement activities.

At APO, the Sloan Telescope Lead Scientist manages the infrastructure
development and maintenance of the telescope and the APO Operations
Manager manages the day-to-day operations, including site
maintenance. At LCO, the LCO Project Manager leads the hardware
development and the LCO Operations Manager manages the day-to-day
survey operations. The LCO site maintenance and telescope maintenance
is handled by the Observatories of the Carnegie Institution for
Science.

Logistical responsibility for handling scientific, technical, and data
release papers rests with the Scientific Publications Coordinator
(SPC), Technical Publications Coordinator (TPC), and Scientific
Spokesperson, respectively. Publications Coordinators ensure
that publications follow standard survey publication processes, and they
maintain a common electronic web-based archive of all scientific,
technical, and data release publications of the SDSS-IV, accessible to
collaboration members.  The TPC coordinates the publication of technical
papers, ensuring that the technical documentation of the project is
disseminated efficiently and promptly.  The SPC is responsible for
tracking SDSS-IV scientific papers through the publication policy
process and assuring that all SDSS-IV papers (scientific, technical,
and data release) reference the appropriate technical papers. The
Scientific Spokesperson has overall responsibility for the
Publications Archive, and coordinates the publication of the data
release papers.

The individuals filling these roles and the teams they lead are
geographically distributed at over twenty institutions. Each team
communicates through email lists, weekly phone meetings, and periodic
in-person meetings. A Management Committee consisting of individuals
in the positions listed here meets weekly to monitor the project
progress.

