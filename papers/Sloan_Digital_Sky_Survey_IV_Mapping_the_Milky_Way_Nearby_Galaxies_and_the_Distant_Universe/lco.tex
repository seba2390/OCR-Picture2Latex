The 2.5~m Ir\'{e}n\'{e}e du Pont telescope is a modified
Ritchey--Chr\'{e}tien optical design held in an equatorial fork mount.
With a Gascoigne corrector lens, it has a 2.1 degree diameter usable
field of view \citep{bowen73a} with a focal ratio of $f/7.5$. The
on-axis focal plane scale is nominally 329.310 mm~deg$^{-1}$. The du
Pont telescope design informed a number of features of the Sloan
Foundation telescope at APO \citep{gunn06a}.

Completed in 1977, the du Pont telescope pioneered early wide-field
fiber spectroscopy. \cite{shectman93a} describes the fiber system used
for the Las Campanas Redshift Survey \citep[LCRS,][]{shectman96a} that
formed a basis for the design of the SDSS observing systems. Since the
completion of the LCRS, the du Pont telescope has not been used for
wide-field spectroscopy.  SDSS-IV is creating the infrastructure to
return to this mode of operation with improved efficiency. The primary
system upgrades include an expanded range of motion for the corrector
lens (to optimize wide-field image quality in the \Hband), improved
servo-control of the instrument rotator, and re-design of the
secondary mirror mounting structure for increased stiffness and
enhanced collimation and focus control.  In addition, implementation
of a new flat-field system is planned to optimize observing
efficiency. The telescope drives, control electronics, and control
software have also been recently modernized.

The SDSS-IV project is designing, fabricating and installing an
optical fiber cartridge and plugging system for LCO that is similar to
that at APO. We use five interchangeable cartridges with 300 short
fibers that can be re-plugged throughout each night, with a plan to
support observations of up to ten plates per night. The short fibers
in each cartridge are precisely connected through a fiber link
(the ``telescope link'') to a set of long fibers that transmit light
to the spectrograph on the ground floor of the telescope building. The
fibers run along a long metal boom attached to the wall of the dome,
and which can rotate to lie along the wall to keep the fibers safe
during observations and to provide safe storage.

Each cartridge includes a plug plate mechanically bent to conform to
the telescope's focal surface, which at \mbox{1.6 $\mu$m} has a radius
of curvature of 8800 mm. The focal plane position parallel to the
optical axis varies around \mbox{6 mm} between the center and edge of
the field (\citealt{shectman93a}), compared to around 2 mm for the
Sloan Foundation Telescope in the optical.  To achieve this large
flexure, the outer part of the plate is held at a fixed angle with a
bending ring (as done at APO). The plate profile is verified and the
profile measurements are stored in the SDSS-IV LCO database ({\tt
lcodb}).

Figure \ref{fig:lco-layout} shows the configuration during
observations, in particular, the fiber run. The bottom of the du Pont
Telescope and the primary mirror are shown as the yellow box and the
inset gray annulus, respectively. The secondary focal plane is located
approximately 8 feet above the dome floor when the telescope points to
zenith. During APOGEE-2S operations, a focal plane scaling mechanism
is attached at the secondary focus. Cartridges must latch to this
scaling mechanism in order to be observed. As shown, the fibers exit
the cartridge, run along a boom to the dome wall, and travel down a
level to the instrument room.

The scaling mechanism allows real-time changes in plate position along
the optical axis. With corresponding movement of the telescope's
secondary mirror, this can be used to alter the focal plane scale to
compensate for changes introduced by differential refraction, thermal
expansion and contraction of the plate, and stellar aberration. The
scaling mechanism is controlled by the SDSS operations software as
part of the overall guiding system.

In order to implement efficient cartridge changes on the scaling
mechanism, we have constructed a stable three-rope hoist system, which
lifts the cartridges into place in the focal plane. The five
cartridges themselves are stored on custom-built dollies so they can
be maneuvered about the observing floor and plugging room.  Cartridges
are plugged in a room next to the dome, then placed in the dome to
equilibrate with the dome temperature. When a cartridge is ready to be
observed, it is rolled to the hoist, attached to the three ropes, and
lifted to the focal plane. Electrical cabinets attached to the scaling
ring house the motion control electronics, while a second electrical
cabinet at the end of the fiber boom contains an LCD touch screen
(VMI), allowing the user to control the system.  The VMI communicates
with the scaling ring electronics through a Bluetooth connection.  A set
of interlocks prevent the cartridge from being lifted in an unsafe
state (e.g., not fully attached to the hoists) or from being left
unsecured to the scaling mechanism.

\begin{figure}[!t]
\centering
\includegraphics[width=0.49\textwidth,
  angle=0]{lco-layout-clipped.png}
\caption{ \label{fig:lco-layout} Model of the du Pont Telescope
configuration during APOGEE-2S observations. The yellow transparent
box indicates the bottom of the telescope, with the gray annulus
indicating the location of the primary mirror. The scaling ring
mechanism with a cartridge attached is just below the primary. A
telescope fiber link connects the cartridge to a patch panel at the end
of the boom. The instrument fibers travel down a movable boom to the
wall of the dome, and are directed to the instrument room in the level
below the telescope dome. The room on the dome level on the right side
of the diagram is used for plate plugging and mapping during the
night.}
\end{figure}

The focal plane and its distortions are estimated using Zemax and an
adjusted version of the specifications from \citet{bowen73a}. From the
analysis of test images the ``best'' focal distance is 254 mm below
the rotator (993 mm from the secondary). We have directly measured the
on-axis scale and distortions at 229 mm and 279 mm below the
rotator by observing star fields using a camera positioned at various
radii in the focal plane. We have found that the specifications
in \citet{bowen73a} do not reproduce these scales well. Their Table 1
entry of the telescope focal length does not include the contribution
of the corrector.  We use a Zemax model based on the surface
specifications in their Table 2, including the corrector, with the
curvature of the primary and secondary adjusted to be consistent with
our observed scales. The resulting nominal scale and distortion is
modeled with a quintic function $s = s_0 \theta + s_3\theta^3 +
s_5\theta^5$. Our best current estimates yield, in the $H$-band, $s_0
= 329.342$ mm deg$^{-1}$, $s_3 = 2.109$ mm deg$^{-3}$, and $s_5 =
0.033$ mm deg$^{-5}$, and at the guider camera wavelength of 7600 \AA,
$s_0 = 329.297$ mm deg$^{-1}$, $s_3 = 2.168$ mm deg$^{-3}$, and $s_5 =
0.021$ mm deg$^{-5}$.  These estimates may be further refined in the
course of commissioning the system.

The cartridges contain guide systems similar to those used on the
telescope at APO. Because the system is being solely designed for use
with APOGEE-2S, we have designed a camera with effective wavelength
around 7600 \AA, which should increase its ability to use guide stars
in the more reddened part of the Milky Way.  The camera is an Andor
iKon-M 394 with a 1024 $\times$ 1024 pixel CCD, with 13 $\mu$m
pixels. This configuration is similar to that currently used at APO
(\citealt{smee13a}).  The effective wavelength is defined by an
Astrodon Photometrics Gen 2 Sloan $i$ filter. The filter is mounted in
the parallel beam between the two Nikon $f/1.4$ 35~mm lenses that
comprise the transfer optics from the output fiber block to the CCD.
The guide fibers and transfer optics preserve the telescope focal
plane scale.  Each 13 $\mu$m guider pixel subtends 0.142'' on the
sky.  The camera is operated binned $2\times2$ for guiding; thus each binned
pixel subtends 0.284'' on the sky.

The plug plates for APOGEE-2S are nearly identical to those used at
APO.  On the du Pont telescope, we use a 1.9$^\circ$ diameter field of
view, which is similar in physical size to the Sloan Foundation
Telescope. As at APO, the fibers have 120 $\mu$m diameter cores to
preserve the instrumental resolution of the spectrograph.  The fiber
core size corresponds to 1.3'' on the sky. The smaller angular size at
LCO relative to APO is appropriate for the better median seeing at LCO
($\sim$0.7'' FWHM in the
\Hband). Relative to APO, this configuration does place stricter
constraints on telescope pointing and focus (despite the slower beam of
the du Pont).

The fibers feed the APOGEE-South spectrograph, a near-clone of
the APOGEE spectrograph at APO. Changes in the new spectrograph are
described in more detail in Section \ref{sec:apogee2}.

APOGEE-2S uses approximately the equivalent of 75 nights per year on
the du Pont telescope starting in 2017 and continuing through 2020
June. In addition, up to 25 nights per year are available to guest
observers through Carnegie Observatories and the Chilean Time
Allocation Committee. All observations are conducted in $\sim$10 night
observing runs throughout the year. The southern \mbox{APOGEE-2}
program has led to a developing partnership between SDSS-IV and
astronomers at seven Chilean universities that have joined the SDSS-IV
project in a collaboration on the design, construction, engineering,
and execution of the survey. This Chilean Participation Group is an
unprecedentedly broad collaboration among Chilean universities in
astronomy and dovetails with the interest of the Chilean government in
developing astronomical engineering as a national strategy in
technology transfer and development of science.
