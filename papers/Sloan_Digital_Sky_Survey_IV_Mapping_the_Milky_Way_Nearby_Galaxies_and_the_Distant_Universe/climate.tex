
The past success of the SDSS collaboration has hinged on tapping into
a diverse talent base. We have worked and continue to work within
SDSS-IV on this issue. Other collaborations may find the SDSS-IV
experience described here informative as they configure their policies
or face similar situations.

The SDSS-IV organization does not directly hire any of the staff, so
all recruitment of staff paid on contracts to institutions from ARC
also must go through each institution's human resources
process. Similarly, in cases of personnel issues, each institution has
its own policies on workplace environment.  The interleaving of
SDSS-IV processes with institutional policies represents an
interesting complication to international, multi-institutional
organizations such as the SDSS.

As discussed in \citet{lundgren15a}, SDSS-IV identified early a
disparity in the gender balance of its leadership structure. In order
to identify the causes of, monitor, and address this issue, we created
a Committee on the Participation of Women in the SDSS (CPWS). The CPWS
initiated regular demographic surveys of the SDSS in order to monitor
the make up of the collaboration and the project over time.  The CPWS
also compiled information on how the project leadership recruitment
proceeded. Near the beginning of SDSS-IV, and in previous phases of
the project, the recruitment for survey positions such as those in
Figure \ref{fig:orgchart} or others such as working group chairs, was
conducted informally and in a relatively federated manner across the
project.

In 2013, SDSS-IV began to implement an early recommendation of the
CPWS to formalize the recruitment process. SDSS-IV policy is that open
project leadership roles are defined and necessary qualifications
discussed prior to searching for candidates.  Roles now usually are
defined with fixed duration to allow rotation and to mitigate the
level of commitment required.  We publicly advertise for candidates
within the collaboration.  Once candidates are identified, the slate
of candidates is reviewed by the Central Project; at this point, if
there is a paucity of female candidates, the reasons for this are
explored and an attempt is made to redress the issue by encouraging
qualified female candidates to apply. The process is tracked by the
Central Project, which needs to approve all appointments.
\citet{lundgren15a} represents an initial attempt to
assess the effectiveness of this process in increasing participation
of women in the survey leadership; the results are as yet unclear for
SDSS-IV.

In the same year, SDSS-IV formed a Committee on the Participation of
Minorities in SDSS (CPMS) to address the underrepresentation of
minorities in the survey.  While the goal of the CPWS was to ensure
gender balance in SDSS leadership, the CPMS was faced with the more
fundamental goal of recruiting and retaining underrepresented minority
talent in the collaboration at all. CPWS identified a lack of
resources, training, and contact with the SDSS collaboration that is a
barrier to full participation of minorities in the survey. In
response, SDSS-IV implemented two immediate and strategic programs to
have the most meaningful impact: the Faculty And Student Team (FAST)
program deliberately focuses on building serious, long-term research
relationships between faculty/student teams and SDSS partners; the
distributed SDSS REU program targets talented minority students at the
undergraduate level, and can be used as a recruitment tool into
graduate school in astronomy.

The FAST program has been independently funded by the Sloan Foundation
for an initial three-year period. It actively recruits and trains
underrepresented minority (URM) talent to participate in SDSS
science. To qualify for FAST, at least one team member is expected to
be a URM and/or to have a track record serving URM scholars.  FAST
scholar teams are matched with established SDSS partners to work on a
research project of mutual interest and receive specialized training,
mentoring, and financial support in order to introduce teams to SDSS
science and to cement their participation within the
collaboration. FAST team faculty become full members of the SDSS
collaboration, with all data rights, access to centralized computing,
and ability to lead projects that this implies. We selected our first
FAST cohort of three teams in 2015 and recruited five FAST teams in 2016.
The distributed SDSS REU program has also been funded by the Sloan
Foundation for one pilot summer in 2016, with six students at four
institutions.

With regard to the climate of the SDSS-IV collaboration, the global
nature of the survey poses unique challenges in developing an
effective and positive work environment.  Project personnel and
science collaboration are distributed at dozens of institutions, in a
number of countries. Opportunities for in-person interaction are
often limited, with most communication happening through email and
phone conversations. There is no central institution recruiting the
leadership and personnel; in addition, a number of project personnel
work on a voluntary basis or for ``in-kind'' credit for their
technical work. Recognizing the potential issues that could arise in
this environment, we requested that an advisory committee from the
American Physical Society conduct a site visit at the 2014
collaboration meeting. There were numerous comments and suggestions
from the visiting committee. In 2015, CPWS crafted these suggestions
into a set of specific recommendations for the project to prioritize
in order to maintain and improve the quality of the climate in the
collaboration.

The CPWS and CPMS have now been combined into a single Committee on
Inclusion in the SDSS (COINS) with the mandate of both original
committees.

In order to address specific issues that may arise within the
collaboration or other problems, the ARC Board has appointed two
Ombudspersons for SDSS-IV that can be consulted to mediate problems
within the collaboration. The position of Ombudsperson is particularly
designed for cases where handling the matter through formal project
channels would lead to a conflict of interest or cases where anonymity
is desired. In addition, SDSS-IV is in the process of developing a
formal Code of Conduct.

