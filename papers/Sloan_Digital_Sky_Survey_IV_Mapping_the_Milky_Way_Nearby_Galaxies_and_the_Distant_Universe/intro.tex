The Sloan Digital Sky Survey (SDSS; \citealt{york00a}) started
observations in 1998 and has completed three different phases. The
data collected includes optical imaging of most of the northern high
Galactic latitude sky as well as optical and near-infrared
spectroscopy of over 3.5 million stars, galaxies, and quasars. These
observations all used the 2.5~m Sloan Foundation Telescope at
Apache Point Observatory (APO;
\citealt{gunn06a}). This paper describes SDSS-IV, the fourth phase,
and how it builds upon and extends both the infrastructure and
scientific legacy of the previous generations of surveys.

\subsection{The SDSS-I through SDSS-III legacy}

Between 2000 April and 2005 June, as described by \citet{york00a},
SDSS-I began the SDSS Legacy Survey, imaging the sky in five
bandpasses \citep[$u$, $g$, $r$, $i$ and $z$;][]{fukugita96a} using
the SDSS imaging camera (\citealt{gunn98a}). As part of the Legacy
Survey, SDSS-I also observed spectra, mostly of galaxies and
quasars,\footnote{To refer to objects thought to have actively
accreting supermassive black holes, we use the terms ``quasar'' or
``Active Galactic Nuclei (AGN),'' sometimes interchangeably,
throughout this paper.}  using a pair of dual-channel optical fiber
spectrographs fed by 640 fibers with 3$''$ diameters
(\citealt{smee13a}). The galaxies were divided into two samples, a
flux-limited Main Sample with a median redshift of $z\sim 0.1$
(\citealt{strauss02a}) and a color-selected sample of Luminous Red
Galaxies which extended to $z\sim 0.5$ (\citealt{eisenstein01a}). The
quasar sample included both ultraviolet excess quasars out to $z\sim
2$ and a set of high-redshift quasars with redshifts beyond $z=5$
(\citealt{richards02a}).

Between 2005 July and 2008 June, SDSS-II completed the Legacy Survey
with 1.3 million spectra over 8000 deg$^2$; the area covered was a
large contiguous region in the Northern Galactic Cap (NGC) and three
long, thin stripes in the Southern Galactic Cap (SGC). SDSS-II also
executed two new programs: The Sloan Extension for Galactic
Understanding and Exploration 1 (\mbox{SEGUE-1;} \citealt{yanny09a}) obtained
around 3,000 deg$^2$ of new imaging over a larger range of Galactic
latitudes and spectra of 240,000 unique stars over a range of spectral
types to investigate Milky Way structure. The Sloan Digital Sky Survey
II Supernova Survey (\citealt{frieman08b, sako14a}) cataloged over
10,000 transient and variable sources, including 1,400 SN Type Ia,
over a 200 deg$^2$ region on the equatorial stripe in the SGC,
referred to as Stripe 82. These two surveys primarily utilized the
dark time.

Between 2008 July and 2014 June, SDSS-III conducted four surveys
(\citealt{eisenstein11a}). Stellar spectroscopy continued with
SEGUE-2, which obtained 130,000 more stars during the first year of
SDSS-III (\citealt{aihara11a}). SDSS-III continued the imaging
campaign, adding 2,350 deg$^2$ of unique area and creating a
contiguous footprint in the Southern Galactic Cap; at the end of 2009
the imaging camera was retired. In Summer 2009, for the Baryon
Oscillation Spectroscopic Survey (BOSS; \citealt{dawson13a}), SDSS-III
upgraded the optical spectrographs to cover a larger optical range and
accommodate 1000 fibers (\citealt{smee13a}). By the end of SDSS-III,
BOSS spectroscopically surveyed 10,338 deg$^{2}$, gathering 1.2
million galaxy spectra to extend the original luminous red galaxy
sample from SDSS-I and SDSS-II to $z\sim 0.7$ and to increase its
sampling density at lower redshifts. It simultaneously used the
Ly$\alpha$ forest in 140,000 spectra drawn from a sample of 180,000
observed quasars to map the fluctuations in neutral hydrogen at
redshifts $2.1<z<3.5$. Both SEGUE-2 and BOSS were conducted using the
dark time.

SDSS-III also employed the Sloan Foundation Telescope in bright
time. From Fall 2008 through 2012 July, the Multi-Object APO Radial
Velocity Exoplanet Large-area Survey (MARVELS; \citealt{ge09a})
observed 5,500 bright stars ($7.6<V<12$) with a 60-fiber
interferometric spectrograph to measure high precision radial
velocities, searching for extra-solar planets and brown
dwarfs. Starting in 2011 May through 2014 June, the APO Galactic
Evolution Experiment 1 (APOGEE-1; \citealt{majewski15a}) observed
140,000 stars with a 300-fiber, $R\sim$ 22,500, \Hband\ spectrograph.

Because the weather efficiency of BOSS exceeded expectations, it
finished its primary observations early, and during its last few
months SDSS-III conducted several special programs in dark time
(\citealt{alam15b}). The Sloan Extended QUasar, ELG and LRG Survey
(SEQUELS) observed 300 deg$^2$ using the BOSS spectrograph to obtain a
dense set of quasars, emission line galaxies (ELGs), and luminous red
galaxies (LRGs), which was used to test target selection for
SDSS-IV. The SDSS Reverberation Mapping program
(SDSS-RM; \citealt{shen15a}) observed a single field containing 849
quasars over more than 30 epochs in order to monitor quasar
variability. During dark time when the inner galaxy was visible (local
sidereal times 15--20 hr) the bulk of the time was allocated to the
APOGEE-1 program.

Data from these surveys have been publicly released.  The SDSS-I and
SDSS-II Legacy, Supernova, and SEGUE-I survey data were released in a
set of data releases beginning in 2001 and culminating in 2008 October
with Data Release 7 (DR7; \citealt{abazajian09a}). The complete
SDSS-III data set was released in 2015 January in DR12
(\citealt{alam15b}).

\subsection{SDSS-IV}

SDSS-IV has new goals that build upon the scientific results of
previous SDSS surveys in the areas of Galactic archeology, galaxy
evolution, and cosmology. In so doing, SDSS-IV observations enable the
detailed astrophysical study of stars and stellar systems, the
interstellar and intergalactic medium, and supermassive black holes;
some of the emerging science themes are described below. The primary
goals of SDSS-IV are achieved in the following three core programs,
two of which required new infrastructure.

\begin{itemize}
\item 
{\it APO Galactic Evolution Experiment 2} (APOGEE-2;
Section \ref{sec:apogee2}) aims to improve our understanding of the
history of the Milky Way and of stellar astrophysics. It expands the
APOGEE-1 probe of the Milky Way history through mapping the chemical
and dynamical patterns of the Galaxy's stars via high resolution,
near-infrared spectroscopy.  The second-generation program has
northern and southern components, APOGEE-2N and APOGEE-2S,
respectively.  APOGEE-2N continues at APO, with primary use of the
bright time. APOGEE-2S utilizes new infrastructure and a new
spectrograph now installed at the 2.5 m du Pont Telescope at Las
Campanas Observatory (LCO).  The pair of spectrographs at APO and LCO
together target a total sample of around 400,000 stars.  APOGEE-2's
near-infrared observations yield access to key regions of the Galaxy
unobservable by virtually all other existing surveys of the Milky Way,
which are predominantly conducted at optical wavelengths.

\item 
{\it Mapping Nearby Galaxies at APO} (MaNGA; \citealt{bundy15a};
Section \ref{sec:manga}) aims to better understand the evolutionary
histories of galaxies and what regulates their star formation.  It
provides a comprehensive census of the internal structure of nearby
galaxies (median redshift $z\sim 0.03$), rendered via integral field
spectroscopy (IFS) --- a new observing mode for SDSS. This census
includes the spatial distribution of both gas and stars, enabling
assessments of the dynamics, stellar populations, and chemical
abundance patterns within galaxies as a function of environment. Using
half of the dark time at APO, MaNGA relies on novel fiber bundle
technology to observe 17 galaxies simultaneously by feeding the fiber
output of independent integral field units into the optical BOSS
spectrographs.  MaNGA plans to observe 10,000 nearby galaxies spanning
all environments and the stellar mass range $10^9$--$10^{11}$
$M_\odot$. The MaNGA observations cover 3500 \AA\ to 1 $\mu$m with
about 65 km s$^{-1}$ velocity resolution and 1--2 kpc spatial
resolution.

\item
{\it extended Baryon Oscillation Spectroscopic Survey}
(eBOSS; \citealt{dawson16a}; Section \ref{sec:ets}) aims to better
understand dark matter, dark energy, the properties of neutrinos, and
inflation. It pushes large-scale structure measurements into a new
redshift regime ($0.6<z<2.2$). Using single-fiber spectroscopy, it
targets galaxies in the range $0.6<z<1.1$ and quasars at redshifts
$z>0.9$. These samples allow an investigation of the expansion of the
universe using the Baryon Acoustic Oscillation (BAO) and the growth of
structure using large-scale redshift space distortions. The
large-scale structure measurements also constrain the mass of the
neutrino and primordial non-Gaussianity. Using half of the dark time
at APO, eBOSS is to observe $\sim$ 250,000 new LRGs ($0.6<z<1.0$) and
$\sim$ 450,000 new quasars ($0.9<z<3.5$) over 7,500 deg$^{2}$. Using
300 plates to cover a portion of this footprint, it also aims to
obtain spectra of $\sim$195,000 new ELGs ($0.7<z<1.1$).

\end{itemize}

There are two major subprograms executed concurrently with eBOSS, also
described in Section \ref{sec:ets}:

\begin{itemize}
\item 
{\it SPectroscopic IDentification of ERosita Sources} (SPIDERS) 
investigates the nature of $X$-ray emitting sources, including active
galactic nuclei and galaxy clusters. It uses
$\sim$5\% of the eBOSS fibers on sources related to X-ray
emission. Most of its targets are X-ray emitting active galactic
nuclei, and a portion are galaxies associated with X-ray
clusters. Initially, SPIDERS targets X-ray sources detected mainly in
the ROSAT All Sky Survey (RASS; \citealt{voges99a}), which has
recently been reprocessed (\citealt{boller16a}). In late 2018, SPIDERS
plans to begin targeting sources from the eROSITA instrument on board
the Spectrum Roentgen Gamma satellite (\citealt{predehl10a,
merloni12a}). Together with eBOSS, SPIDERS targets a sample of 80,000
X-ray identified sources ($\sim$ 57,000 X-ray cluster galaxies and
22,000 AGNs, of which around 5,000 are already included in
eBOSS targeting).

\item 
{\it Time Domain Spectroscopic Survey} (TDSS; \citealt{morganson15a})
investigates the physical nature of time-variable sources through
spectroscopy.  It also uses $\sim$5\% of the eBOSS fibers, primarily
on sources detected to be variable in Pan-STARRS1 data
(PS1; \citealt{kaiser10a}), or between SDSS and PS1 imaging. The
targets identified in PS1 are a mix of quasars (about 60\%) and
stellar variables (about 40\%). A majority of the quasars are already
targeted by eBOSS. TDSS aims to produce a spectroscopic
characterization of a statistically complete selection of
$\sim$200,000 variables on the sky down to $i=21$. TDSS targets a
total of around 80,000 objects not otherwise included by eBOSS
targeting.

\end{itemize}

In executing these programs, we exploit several efficiencies allowed
by the SDSS observing facilities. First, there is substantial common
infrastructure and technology invested in the plate and cartridge
hardware at APO and in the associated software.  Second, the SDSS-IV
survey teams closely coordinate the observing schedule on long and
short time scales to maximize efficiency. Finally, MaNGA and APOGEE-2
are able to co-observe, which allows APOGEE-2 to observe a large
number of halo stars during dark time and for MaNGA to create a unique
optical stellar library in bright time.

In addition to these overlaps in infrastructure, there exist
substantial scientific synergies between the SDSS-IV programs. These
connections allow the surveys to explore a number of critical aspects
of baryon processing into and out of gravitational potentials from
scales of stars to galaxy clusters. We remark on two emerging themes
that we expect to grow over the course of the survey. First, the
science goals of APOGEE-2 and MaNGA are closely aligned in the context
of understanding galaxy formation and evolution. APOGEE-2 treats the
Milky Way as a detailed laboratory for asking questions about galaxy
evolution similar to those MaNGA asks using a set of more distant
galaxies observed in less detail. These vantage points are highly
complementary because APOGEE-2 has access to chemo-dynamical structure
on a star-by-star basis, while MaNGA samples all viewing angles for
both gas and stars over a wide range of galaxy masses and
environments. These disparate perspectives facilitate understanding
the kind of galaxy we live in, and by extension, the detailed
processes occurring in other galaxies.

Second, the eBOSS, TDSS, and SPIDERS programs create an
unprecedentedly large and complete sample of quasars, essentially
complete down to Seyfert luminosities out to nearly $z\sim 2$ (further
discussion of quasar science is in Section \ref{sec:quasars}).  This
sample serves as a critically important tool for understanding the
evolution and decline in accretion rates of supermassive black holes,
and in turn how active galactic nuclei impact the hosts in which they
reside.

This paper describes the facilities that make these programs possible
as well as the scientific goals, observational strategy, and
management of the project and its associated collaboration. We pay
particular attention to the new hardware developments of the program,
which are primarily related to APOGEE-2S and MaNGA. More detail on all
programs, and in particular how each survey's design addresses its
high level requirements, is or will be available in existing and
upcoming technical papers (\citealt{bundy15a, morganson15a, clerc16a,
dawson16a, dwelly17a}, and APOGEE-2 and TDSS papers in preparation).

Section 
\ref{sec:facilities} provides an overview of the APO and LCO facilities.
Section \ref{sec:imaging} describes the imaging data utilized in
SDSS-IV, which includes significant reanalysis of SDSS and {\it
Wide-field Infrared Survey Explorer (WISE)} images.
Sections \ref{sec:apogee2} through \ref{sec:ets} present the survey
programs.  Section \ref{sec:data} describes the data management and
distribution plan for the project. Section \ref{sec:epo} provides a
summary of the education and public engagement strategies employed by
the project. Section \ref{sec:management} describes the project
management and organization of the science collaboration, including
the activities associated with fostering and maintaining a healthy
climate within SDSS-IV. Section \ref{sec:summary} provides a brief
summary.
