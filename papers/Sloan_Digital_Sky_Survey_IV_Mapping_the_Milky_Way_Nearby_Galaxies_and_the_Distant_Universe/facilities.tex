The primary departure in SDSS-IV from previous survey generations is
the expansion of our observing facilities to include telescopes in
both hemispheres. In contrast to the requirements for extragalactic
surveys on scales where the universe is isotropic, such as MaNGA and
eBOSS, this expansion is essential for the study of the Milky Way in
APOGEE-2.  In particular, the south affords much more efficient access
to the Galactic bulge and the inner disk, even for near-infrared
surveys that can operate at high airmass; full mapping of the Milky
Way, including the disk and bulge where APOGEE's near-infrared view has
the greatest advantage, requires all-sky coverage.

Since its inception, SDSS has used the 2.5~m Sloan Foundation
Telescope at the Apache Point Observatory (APO), located in the
Sacramento Mountains of south-central New Mexico. Since the advent of
APOGEE-1 in SDSS-III, the NMSU 1~m Telescope
(\citealt{holtzman10a}) at APO has also been used with the APOGEE
spectrograph. SDSS-IV adds the 2.5~m du Pont Telescope
(\citealt{bowen73a}) located at the Las Campanas Observatory (LCO) in
the Andean foothills of Chile.  On the 2.5~m Sloan Foundation
Telescope, we continue to operate the BOSS spectrographs for the eBOSS
and MaNGA programs during dark time, and the APOGEE spectrograph
during bright time.  For the 2.5~m du Pont Telescope, a second,
nearly identical APOGEE spectrograph was constructed for the southern
component of the APOGEE-2 survey.
