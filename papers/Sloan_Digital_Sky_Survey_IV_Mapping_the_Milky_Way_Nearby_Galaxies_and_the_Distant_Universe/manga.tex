\subsection{MaNGA Motivation} 

MaNGA is gathering two-dimensional optical spectroscopic maps
(integral field spectroscopy) over a broad wavelength range for a
sample of 10,000 nearby galaxies. In contrast, the original SDSS
Legacy survey of the nearby galaxy population, and all similar efforts
of similar scope to it, obtained single fiber spectroscopy. Single
fiber spectroscopy constrains the ionized gas content, stellar
populations, and kinematics of each galaxy, but only averaged over one
specific (typically central) region. These surveys revealed in broad
terms how the properties of galaxies, including their stellar mass,
photometric structure, dynamics, and environment, relate to their
star-formation activity and its bimodal distribution.  However, to
fully understand how galaxy growth proceeds, how star-formation ends,
and how the assembly process shapes the final observed galaxy
properties, detailed mapping of gas and stellar structure across the
entire volume of each galaxy is required. MaNGA's integral field
spectroscopic data allows study and characterization of the spatial
distribution of stars and gas as well as of the detailed dynamical
structure, including rotation, non-circular motions, and spatial maps
of higher moments of the velocity distribution function.

MaNGA is the latest and most comprehensive of a series of integral
field spectroscopic galaxy surveys of ever-increasing size. The
Spectrographic Areal Unit for Research on Optical Nebulae (SAURON;
\citealt{dezeeuw02a}), DiskMass (\citealt{bershady10a}), ATLAS$^{\rm
  3D}$ (\citealt{cappellari11a}), and the Calar Alto Legacy Integral
Field Area Survey (CALIFA; \citealt{sanchez11a}) have created a total
sample of around $1000$ well-resolved galaxies. The Sydney-AAO
Multi-object Integral field spectrograph (SAMI; \citealt{croom12a})
survey is now operating at the Anglo-Australian Observatory and plans
to observe 3400 galaxies.

MaNGA's distinguishing characteristics in this context are as
follows. First, it is the largest planned survey. Relative to CALIFA
and ATLAS$^{\rm 3D}$, the larger sample sizes of both MaNGA and SAMI
are made possible through multiplexing; by having multiple,
independently positionable IFUs across the telescope field of view,
both surveys are able to observe more than one galaxy at once, and
hence dramatically increase survey speed.  A consequence of requiring
all targets to be contained within the telescope field of view is that
both MaNGA and SAMI target more distant objects than SAURON or CALIFA,
and achieve lower physical resolution. Second, MaNGA uses the BOSS
spectrograph, which has broader wavelength coverage than SAMI, CALIFA,
or previous surveys. MaNGA is the only large integral field survey
with spectroscopic coverage out to 1 $\mu$m to allow coverage of the
calcium triplet and iron hydride features informative of stellar
populations, and [S III] emission lines from ionized gas. Third, MaNGA
covers the radial scale of galaxies in a uniform manner regardless of
mass or other characteristics; one-third of MaNGA galaxies have
coverage to at least 2.5$R_e$ and two-thirds have coverage to at least
1.5$R_e$ ($R_e$ is equivalent to the half-light radius for any profile
shape).  Finally, MaNGA has statistically well-defined selection
criteria across galaxy mass, color, environment, and redshift.

\subsection{MaNGA Science}

The primary science goal of MaNGA is to investigate the evolution of
galaxy growth. It is designed to supply critical information for
addressing four questions. (1) How are galaxy disks growing at the
present day and what is the source of the gas supplying this growth?
(2) What are the relative contributions of stellar accretion, major
mergers, and secular evolution processes to the present-day growth of
galactic bulges and ellipticals? (3) How is the shutdown of star
formation regulated by internal processes within galaxies and
externally driven processes that may depend on environment? (4) How is
mass and angular momentum distributed among different components and
how has their assembly affected the components through time?

MaNGA's resolved spectroscopy provides critical observations to
address these questions. The stellar continuum of the galaxies reveals
the star-formation history and stellar chemistry (e.g.,
\citealt{thomas03a}). Nebular emission characterizes active galactic
nuclei, star formation, and other processes
(e.g. \citealt{osterbrock06a}).  When star formation dominates the
emission, line fluxes and flux ratios indicate the rate of star
formation and the metallicity of the ionized gas around the stars
(e.g. \citealt{tremonti04a}). Both nebular emission and stellar light
provide key dynamical information related to the mass and mass profile
of the galaxies (e.g. \citealt{cappellari08a, li16a}).

The MaNGA hardware and survey are designed with the aim to constrain
the distribution of physical properties of galaxies by gathering a
sample large enough to probe the natural variation of these properties
in the three dimensions of environment, mass, and galaxy
star-formation rate. The sample size (10,000 galaxies) is justified by
the desire to resolve the variation of galaxy properties in six bins
in each of these three dimensions with about 50 galaxies in each
bin. This number of galaxies per bin is sufficient such that
differences between bins can be determined accurately.

The major areas of study for MaNGA follow from and map into the four
science questions above.
\begin{itemize}
\item Growth of galaxy disks, through the determination of
  star-formation rate surface densities and gas metallicity gradients.
\item Quenching of star formation, through star-formation rates and
  star-formation history gradients.
\item Assembly of bulges and spheroids, through star-formation
  histories and metallicity and abundance gradients.
\item The distribution and transfer of angular momentum in the stellar
  and gas components.
\item Weighing galaxy subcomponents, using the dynamically determined
  masses (from both gas and star kinematics) and the stellar masses.
\end{itemize}

The MaNGA exposure times are designed to achieve sufficient
signal-to-noise ratio spectra to address these questions.  The driving
requirements on exposure time are the precision requirements at
1.5$R_e$ on star-formation rates (0.2 dex per spatial resolution
element), stellar population ages, metallicities, and
$\alpha$-abundances (0.12 dex when averaged over an annular ring), and
dynamical mass determinations (10\%). When these goals are achieved,
other precision requirements on ionized gas and stellar population
properties necessary to study the above questions are typically
satisfied. For the majority of galaxies in the MaNGA sample, these
requirements are met by achieving the signal-to-noise ratio criteria
described below (Section \ref{sec:manga:observations}).

\subsection{MaNGA Hardware}
\label{sec:manga:hardware}

\citet{drory15a} describe the MaNGA fiber bundle technology in
detail. This technology allows precise hex-packed bundles of optical
fibers to be fed to the BOSS spectrograph. As described in Section
\ref{sec:apo}, for each of six cartridges there are 17 fiber bundles,
12 7-fiber minibundles used for standard stars, and 92 single fibers
for sky. The 17 large bundles are normally used to target galaxies and
have a range of sizes tuned to the MaNGA target galaxy distribution;
there are 2 19-fiber bundles, 4 37-fiber bundles, 4 61-fiber bundles, 2
91-fiber bundles, and 5 127-fiber bundles. Each fiber has a 120 $\mu$m
active core (2$''$ on the sky); in addition, there are 6 $\mu$m of
cladding and 9 $\mu$m of buffer, for a total diameter of 150 $\mu$m,
which defines the hexagonal spacing. When deployed, the fiber system
has high throughput (97\% $\pm$ 0.5\% in lab throughput tests).
Each fiber has a focal ratio degradation that is small and is
equivalent to the BOSS single fiber system. The overall throughput is
improved slightly relative to BOSS through the use of antireflective
coatings.

Each fiber bundle has associated sky fibers. Minibundles have
a single sky fiber, 19-fiber and 37-fiber bundles have two, 61-fiber
bundles have four, 91-fiber bundles have six, and 127-fiber bundles
have eight.  These sky fibers are constrained physically to be placed
in holes within 14$'$ of their associated IFU. This
configuration leads to sky fibers always being available close to the
science fibers both on the focal plane and on the BOSS slit head (see
\citealt{law16a}).

\begin{figure}[t!]
\centering
\includegraphics[width=0.49\textwidth,
  angle=0]{manga-footprint.pdf}
\caption{ \label{fig:manga-footprint} Planned MaNGA spectroscopic
  footprint in equatorial coordinates, centered at $\alpha_{\rm
    J2000}=270^\circ$, with East to the left.  Black shows the
  available MaNGA tiles; orange indicates example coverage for a
  simulated SDSS-IV MaNGA survey.}
\end{figure}

\subsection{MaNGA Target Selection}

Wake et al.~(submitted) describe the galaxy targeting
strategy. The primary goals are to obtain a statistically
representative sample of 10,000 galaxies with uniform spatial
coverage, an approximately flat distribution in $\log M_\ast$, and the
maximum spatial resolution and signal-to-noise ratio with these
constraints. To ensure that the sample definition is simple and fully
reproducible, selection functions are defined in redshift, rest-frame
$r$-band absolute magnitude, rest-frame $g-r$ color, and (for the
color-enhanced sample) rest-frame NUV$-i$ color only.

MaNGA selects galaxies from the NASA-Sloan Atlas (NSA;
\citealt{blanton11a}), which is based on the Main Galaxy Sample of
\citet{strauss02a} but includes a number of nearby galaxies without
SDSS spectroscopy and incorporates better photometric analysis than
the standard SDSS pipeline.  The version of NSA used ({\tt v1\_0\_1})
is limited to galaxies with $z<0.15$.  For selection and targeting
purposes, $R_e$ is defined in the MaNGA survey as the major-axis
elliptical Petrosian radius in the $r$ band.  Galaxies are matched to
IFUs of different size based on this $R_e$ value and the effective
size of the IFU.

MaNGA target selection is limited to the redshift range
$0.01<z<0.15$. We seek an approximately flat stellar mass
distribution, and to cover most galaxies out to a roughly uniform
radius in terms of $R_e$. Achieving these goals requires targeting
more luminous, and consequently intrinsically larger, galaxies at
larger redshifts.  MaNGA defines three major samples across the
footprint of the Main Sample of galaxies from the SDSS-II Legacy
Survey; about one-third of this full sample is targeted for
observation. The observed sample is to include the following.
\begin{itemize}
\item 5000 Primary galaxies: selected in a narrow band of rest-frame
  $i$-band luminosity and redshift such that 80\% have coverage out to
  $1.5 R_e$.
\item 1700 Color-enhanced galaxies: selected according to $i$-band
  luminosity and redshift as for Primary, but with a well-defined
  upweighting as a function of NUV$-i$ color to better sample the
  rarer colors. The Primary and the Color-enhanced sample together
  are referred to as the Primary+ sample.
\item 3300 Secondary galaxies: selected in a band of rest-frame
  $i$-band luminosity and redshift, somewhat higher redshift relative
  to Primary, such that 80\% have coverage out to $2.5R_e$.
\end{itemize}
The Primary sample has a median redshift of $\langle
z\rangle\sim 0.03$, whereas the Secondary sample is at a larger median
redshift $\langle z\rangle\sim 0.05$. 

These targets are defined over most of the 7800 deg$^{2}$ area of the
SDSS Main Galaxy Sample, which is a large contiguous region in the NGC
and three 2.5$^\circ$ stripes in the SGC. Since the density of MaNGA
target galaxies varies substantially over the sky, Wake et
al.~(submitted) have designed the potential field locations to adjust
to cover the dense regions more densely, using a version of the
algorithm described by \citet{blanton03a}.  Figure
\ref{fig:manga-footprint} shows these potential locations as black
circles (each 1.5$^\circ$ in radius). As in eBOSS, each pointing is
referred to as a tile, which is typically associated with a single
physical plate.  MaNGA will be able to observe about one-third of the
available tiles during its six years of operations. Figure
\ref{fig:manga-footprint} shows a simulated projection of this
coverage (depending on weather patterns).

For each plate, minibundles are associated with standard stars, which
are F stars selected similarly to those in eBOSS and are used for
spectrophotometric calibration (\citealt{yan16a}). The sky fibers
associated with each bundle are assigned to locations that are empty
in SDSS imaging.

In addition, MaNGA is targeting a set of ancillary targets observed in
fields for which the above samples do not use all the bundles. These
ancillary samples are described in the data release papers
(e.g. for DR13 in \citealt{albareti16a}).

\begin{figure*}[t!]
\centering
\includegraphics[width=0.98\textwidth,
  angle=0]{manga-data.pdf}
\caption{Top left: image of a MaNGA target (UGC 02705) from SDSS, with
  MaNGA 127-fiber bundle footprint overlaid (37$''$ $\times$ 37$''$).
  Top right: maps of derived quantities from the DAP pipeline: stellar
  velocity dispersion $\sigma_\ast$, stellar mean velocity $v_\ast$,
  the stellar population age indicator $D_n$4000, the metallicity
  indicator $\langle$Fe$\rangle = 0.5($Fe5270$+$Fe5335$)$, the [OIII]
  $\lambda$5007 flux in $10^{-17}$ erg cm$^{-2}$ s$^{-1}$, and the
  H$\alpha$ flux in the same units.  Bottom: sum of MaNGA spectra in
  elliptical annuli of increasing radii.
\label{fig:manga-data}}
\end{figure*}

\subsection{MaNGA Observations}
\label{sec:manga:observations}

MaNGA utilizes approximately 50\% of the dark time at APO. Details of
the division of observations across the SDSS-IV surveys are given in
Section \ref{sec:apo}.

Each MaNGA fiber bundle is encased in a small metal ferrule 20 mm in
length, which protects the bundle and contains a pin for keeping the
ferrule in constant alignment on the plate. The resulting ferrule is 7
mm in diameter, larger than that for individual eBOSS or APOGEE-2
fibers. This constraint prevents two fiber bundles on the same plate
from being closer than about 116$''$.

The fiber bundles do not optimally sample the typical atmospheric and
telescope point spread function. To provide better sampling, each
plate is observed in a set of three successive 15 minute exposures
offset from each other by 1.44$''$ in a triangular pattern on the sky
(\citealt{law15a}). Typically, these dithered exposures are all taken 
in succession to make sure a full set exists for each plate and night.

Each plate is designed for a specific hour angle of observation and is
observable over a certain visibility window, as described in
\citet{law15a}. The window is defined according to how quickly the
position of the IFU shifts in sky coordinates due to differential
refraction across the field (accounting for the telescope's guiding
adjustments). The condition is that the maximum shift at any
wavelength for an IFU at any location on the plate over an hour
duration is 0.5$''$ or less.  If a dither set is begun within any part
of the observing window, all subsequent dithers must be taken at
similar hour angles in order to be combined, such that they are all
within an hour of each other (the data do not have to be taken on the
same night).

MaNGA requires a signal-to-noise ratio of 5 \AA$^{-1}$ fiber$^{-1}$ in
the $r$-band continuum at a Galactic extinction corrected $r$-band
surface brightness of 23 mag arcsec$^{-2}$ \citep[AB
  magnitude;][]{oke83a}. This goal is achieved by setting a threshold
for determining whether the plate is complete as follows for the blue
and red BOSS spectrograph data. We do so using the $(S/N)^2$ per
spectroscopic pixel summed across exposures. A plate is deemed
complete when this $(S/N)^2$ exceeds a threshold at a fiducial $g_{\rm
  fiber2}$ and $i_{\rm fiber2}$ (these are magnitudes from SDSS DR13
imaging \citep{ahn12a} within a 2$''$ diameter aperture convolved with
2$''$ FWHM seeing).  For Galactic extinction corrected $g_{\rm
  fiber2}=22$, the threshold is $(S/N)^2>20$ in the blue spectrograph.
For Galactic extinction corrected $i_{\rm fiber2}=21$, the threshold
is $(S/N)^2>36$ in the red spectrograph.  Typically three sets of
dithers (nine total exposures) are required for completion; in regions
of greater Galactic extinction more than three sets are required.
Usually, only two sets can be taken in succession while still
satisfying the hour-angle criteria described above. Observations of
the same plate are therefore typically split across nights.

For some sets, if the observing conditions are changing rapidly, some
dithers are good quality but others are not. The good-quality dithers
in this situation are considered ``orphan'' exposures since they
cannot be easily combined with exposures in other sets. These good
exposures are processed but are not included in the reconstructed data
cubes because they would lead to non-uniform images. Major changes in
the reduction procedure might allow a more efficient use of these
otherwise good-quality observations. Doing so is not in the pipeline
development plans; nevertheless, the fully calibrated row-stacked
spectra are available for such analysis.

In the mean, each plate requires 3.3 sets of 3 exposures, or about 2.5
hr of open shutter time. Each set requires 20 minutes overhead in
cartridge changes, calibrations, and field acquisition. The orphaned
exposures produce an additional 10\% loss in efficiency.

In addition to the galaxy survey, MaNGA uses their IFUs for the
development of a new optical stellar library (the MaNGA Stellar
Library, or MaSTAR). Because MaNGA IFUs share cartridges with APOGEE
fibers, during APOGEE-2N time the MaNGA IFUs are placed on MaSTAR
targets. These observations are not dithered. The MaSTAR library
provides several advantages over existing libraries. Totaling around
6,000 stars, MaSTAR is several times larger than previous efforts,
including those few that span a comparable spectral range, e.g.,
STELIB (\citealt{leborgne03a}) or INDO-US (\citealt{valdes04a}).  Its
target selection utilizes stellar parameter estimates from
\mbox{APOGEE-1} (\citealt{garciaperez16a}), SEGUE
(\citealt{allendeprieto08a}), and LAMOST (\citealt{lee15b}) to better
cover underrepresented ranges of parameter space of effective
temperature, surface gravity, metallicity, and abundance.  While the
Milky Way imposes certain practical limits, say, on the available
dynamic range in age and abundance, there are known significant gaps
in parameter coverage, e.g., at low temperatures for both dwarfs and
giants, and at low metallicity, that MaSTAR is be able to fill.  While
SEGUE (\citealt{yanny09a}) sampled a large number of stars over a
range of spectral types and surface gravities, their goal of broadly
studying the kinematics and stellar populations of our Galaxy did not
lead to an adequate sampling of some of these regions of parameter
space where stars in the Milky Way are rare in the magnitude ranges
probed. MaSTAR is the first stellar library of significant size with
wavelength coverage from 3600 \AA\ to beyond 1 $\mu$m. Finally, for
the purposes of stellar population synthesis of MaNGA galaxies, using
an empirical library with the same instrument minimizes systematics in
resolution mismatch and offers significant improvements and
consistency in spectrophotometry.

\subsection{MaNGA Data}
\label{sec:manga:data}

MaNGA spectroscopic data consists of $R\sim 2,000$ spectra in the
optical (approximately 3600 \AA $< \lambda <$ 10,350 \AA), at
signal-to-noise ratios of at least 5 per pixel, spatially resolved
across galaxies at $\sim 2.5''$ resolution FWHM, from which we create
maps of velocities, velocity dispersion, line emission, and stellar
population indicators. MaNGA data are processed using a pipeline
derived from and similar to that used for eBOSS, and utilizing similar
infrastructure.

MaNGA data are processed through a quicklook pipeline (Daughter Of
Spectro; DOS) during each observation to estimate the signal-to-noise
ratio in real time and make decisions about continuing to subsequent
exposures. Quality assurance plots are studied each day to identify
unexpected failures of the observing system or pipelines.

A Data Reduction Pipeline (DRP; \citealt{law16a}) reduces the single
fibers in each exposure into individual spectra using optimal
extraction. This pipeline is similar to and shares a code base
with the pipeline that processes BOSS spectrograph data
(\citealt{bolton12a}).  There is a subtle difference in the sky
estimation. As in BOSS and eBOSS, all fibers are used to define the
model sky spectrum; however, this model spectrum can be scaled in the DRP to
match the local sky background near each IFU.  A second and more
fundamental difference is the spectrophotometric calibration
procedure.  An important factor in the single fiber eBOSS
spectrophotometric calibration is the wavelength-dependent loss due to
atmospheric differential refraction (ADR; for a detailed discussion,
see \citealt{margala16a}).  However, for MaNGA, this effect is better
interpreted as a variation with wavelength of the effective location
of the fiber center on the sky; i.e., the blue light samples a slightly
different part of the galaxy than the red light. Loosely speaking,
light is no longer ``lost'' from a given fiber due to ADR, but instead
shifted toward a neighboring fiber.  Thus, the spectrophotometric
correction should not include ADR losses. As \citet{yan16a} describe,
the correction is performed using standard stars observed through 7-fiber
minibundles, which allow for the geometric effects to be disentangled
from the effective throughput of the system. 
The DRP produces a set of wavelength and flux calibrated ``row
stacked spectra'' for each exposure.

In the second stage of processing, the DRP associates each fiber in a
given exposure with its effective on-sky location using the
as-measured fiber bundle metrology in combination with the known
dither offsets and a model for the ADR and guider corrections.  This
astrometry is further refined on a per-exposure basis by comparing the
fiber fluxes to reference broadband imaging in order to correct small
rotations and/or offsets in the fiber bundle location from the
intended position.  The DRP then uses a flux-conserving variation of
Shepard's method \citep{sanchez12} to interpolate the row-stacked spectra onto a
three-dimensional data cube with regularly spaced dimensions, one in
wavelength and two Cartesian spatial dimensions.  Details on the DRP
can be found in \citet{law16a}.

Based on the row-stacked spectra and data cubes, a Data Analysis
Pipeline (DAP) calculates maps of derived quantities such as Lick
indices \citep[e.g.,][]{worthey94}, emission-line fluxes, and
kinematic quantities such as gas velocity, stellar velocity, and
stellar velocity dispersion. The list of calculated quantities remains
under development.  Future plans for DAP include deriving high-level
quantities such as stellar mass and abundance maps, metallicity maps,
and kinematic models.

Figure \ref{fig:manga-data} shows some typical MaNGA data for
UGC~02705, for which observations through a 127-fiber bundle finished
on 2014 October 26.

The first SDSS-IV data release (DR13; 2016 July) contains MaNGA
results data taken through 2015 July.  In DR14, the MaNGA data through
2016 May will be released.

