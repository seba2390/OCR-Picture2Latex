\subsubsection{eBOSS Motivation}
\label{sec:eboss:motivation}

eBOSS is conducting cosmological measurements of dark matter, dark
energy, and the gravitational growth of structure.  Current data from
other large-scale structure measurements, Supernovae Type Ia, and the
cosmic microwave background are consistent with a spatially flat cold
dark matter model and a cosmological constant
($\Lambda$CDM; \citealt{weinberg13a, aubourg15a}). The cosmological
constant or some other mechanism is required due to the observed
late-time acceleration in the cosmic
expansion \citep[e.g.,][]{riess98, perlmutter99a}.

The cosmological constant can be generated through a nonzero, but very
small, vacuum energy density; however, the particle physics mechanism
to generate this level of vacuum energy is unknown. The acceleration
could also be caused by some more general fluid with negative
pressure, referred to typically as ``dark energy;'' the equation of
state of this fluid is constrained to be fairly similar to that of the
vacuum energy. Alternatively, the acceleration may be caused due to
modifications of general relativity that affect gravity at large
scales (e.g. \citealt{randall99a, dvali00a, sahni03a, sotiriou10a,
battye12a}). Many of these explanations of the acceleration are
theoretically plausible, and the challenge is to observationally bound
the possibilities. One critical constraint arises from precisely
measuring the rate of expansion and gravitational growth of structure
throughout all cosmic epochs.

eBOSS is creating the largest volume map of the universe usable for
large-scale structure to date. This data set will allow exploration of
dark energy and other phenomena in epochs where no precision
cosmological measurements currently exist, pursuing four key goals:
BAO measurements of the Hubble parameter and distance as a function of
redshift, redshift space distortion measurements of the gravitational
growth of structure, constraints on and possible detection of the
neutrino mass sum, and constraints on inflation through measurements
of non-Gaussianity.

Among currently operating experiments, only the Hobby-Eberly Telescope
Dark Energy Experiment (HETDEX; \citealt{hill08a}) and the Dark Energy
Survey (DES; \citealt{abbott16a}) will measure the universe's
expansion history at comparable precision and accuracy. HETDEX is a
wide-field integral field spectrograph survey that will map \lya\
emitting objects at $z\sim 2$--$3$. DES is an imaging survey that will
measure BAO as a function of redshift using angular clustering and
photometric redshifts.  Future spectroscopic experiments are planned
that will exceed the precision in measuring expansion of any current
program. These experiments include DESI (\citealt{levi13a}) and the
Prime Focus Spectrograph at {\it Subaru} (PFS; \citealt{takada14a}). eBOSS's
large-scale structure results precede the beginning of either of these
experiments and is poised to deliver the first accurate measurements
of expansion in the redshift range $1<z<2$.

\subsubsection{eBOSS Science}
\label{sec:eboss:science}

The primary cosmological constraints from eBOSS are BAO measurements
of the angular diameter distance $D_A(z)$ relative to that of the CMB,
and the Hubble parameter $H(z)$ as a function of
redshift. \citet{weinberg13a} includes a recent review of this
technique. The LRG, ELG, and low-redshift quasar samples are used as
tracers to measure BAO in large-scale structure; the high-redshift
quasar sample is used for \lya\ forest measurements of BAO in the
neutral gas clustering. These measurements in real and redshift space
yield constraints on the Hubble parameter $H(z)$ and the angular
diameter distance $D_A(z)$, which can be combined into a constraint on
a combined distance $R(z)$. Full details on the definition of these
quantities, and projections regarding the precision on BAO from eBOSS
can be found in \citet{dawson16a} and \citet{zhao16a}.
Table \ref{table:eboss_samples} summarizes the expected precision from
the LRG, ELG, quasar, and \lya\ samples.  In terms of the Dark Energy
Task Force (DETF) Figure of Merit (FoM; \citealt{albrecht06a}), the
eBOSS sample improves the FoM over the existing constraints to date by
a factor of three. These projections assume only measurements of the
BAO feature itself. Addition of the broadband power spectrum, redshift
space distortions, and geometric distortions is expected to produce a
further increase in the FoM \citep{mcdonald09a}, though with greater
theoretical systematics.

\begin{table}[t!]
\caption{
\label{table:eboss_samples} Cosmological precision in eBOSS}
\begin{tabular}{lccccc}
\hline\hline
Target Class & $z$ & $\sigma_H/H$ & $\sigma_{D_A}/D_A$
& $\sigma_R/R$ & $\sigma_{f\sigma_8}/f\sigma_8$\tablenotemark{a}\\
\hline
LRG\tablenotemark{b} & 0.71 & 0.025 & 0.016 & 0.010 & 0.025 \\
ELG\tablenotemark{c} & 0.86 & 0.050 & 0.035 & 0.022 & 0.034 \\
Quasar & 1.37 & 0.033 & 0.025 & 0.016 & 0.028 \\
Ly-$\alpha$ & 2.54 & 0.014 & 0.017 & --- & --- \\
\hline
\end{tabular}
\tablecomments{
Results derived from \citet{zhao16a}. 
}
\tablenotetext{a}{$f\sigma_8$ forecasts use assumptions similar to the
model-independent constraints cited in
Section \ref{sec:eboss:science}, holding other cosmological parameters
fixed.}
\tablenotetext{b}{Includes LRGs observed in SDSS-III within the overlapping
redshift range.}
\tablenotetext{c}{Numbers correspond to the ``high density'' ELG sample
in \citet{zhao16a}, which is close to the current
plan. }
\end{table}

Redshift space surveys, as opposed to imaging surveys, yield a unique
additional constraint on cosmology; since galaxy motions reflect the
gravitational growth of structure, measuring the anisotropic
distortion they produce in clustering yields constraints on cosmological
parameters and general relativity (GR) (\citealt{weinberg13a}). In the
context of cosmic acceleration, clustering measurements can
distinguish between models for acceleration that rely on dark energy
and those that require modified gravity (\citealt{huterer15a}).  This
measurement yields $f\sigma_8$, where $f$ measures the growth rate and
$\sigma_8$ measures the amplitude of matter fluctuations. Currently
the most robust constraints on $f\sigma_8$ are from BOSS, with
large-scale model-independent constraints of $\sim 6$\% (9\% when
marginalizing over other parameters;
\citealt{beutler14a, samushia14a, alam15a}) and model-dependent
constraints on smaller scales of 2.5\% (\citealt{reid14a}). These
critical tests distinguishing dark energy and modified gravity models
are possible only with a spectroscopic redshift program such as eBOSS.

The fundamental properties of neutrinos are imprinted in the
distribution of galaxies. eBOSS's large volume permits tight new
constraints on, and perhaps finally allows for a measure of, the
neutrino mass.  Flavor oscillation measurements place lower limits on
the neutrino masses of $0.05$--$0.10$~eV depending on the model
(\citealt{fogli12a}).  Cosmological observations place upper limits on
the sum of neutrino flavor masses, due to the suppression of power by the
neutrino component in fluctuations at scales smaller than 100 Mpc. The
best existing cosmological constraint is that $\sum m_\nu < 0.23$ eV
(95\% confidence, when assuming zero curvature; \citealt{planckXVI14}),
from CMB measurements and BAO.  Adding eBOSS constraints from the LRG,
ELG, and $z<2.2$ quasars improves this limit to $\sum m_{\nu}
<0.108$ eV, close to the minimum allowed neutrino mass in conventional
particle physics theories.  eBOSS clustering data therefore have a
significant chance of measuring the neutrino mass sum, which would be
a major breakthrough in fundamental physics.

eBOSS pioneers tests of cosmic inflation through the measurement of
very-large-scale fluctuations.  Departures from the standard
inflationary scenario commonly yield small deviations from Gaussian
fluctuations, quantifiable by \fnl ($=0$ for Gaussian). A natural form
of non-Gaussianity (the ``local'' form; \citealt{wands10a}) can be
tested using two-point statistics at $>200$ Mpc
(\citealt{dalal08a}). eBOSS yields the only constraints ($\sigma_{\rm
fnl}=12$) comparable in precision to (but completely independent of)
current Planck limits (local
$\fnl=2.5\pm5.7$; \citealt{ade16a}). Furthermore, galaxy bispectrum
measurements have the potential to improve eBOSS constraints
dramatically.  Future improvements will likely be best achieved with
redshift surveys such as eBOSS.

eBOSS yields the largest existing statistical sample available for a
broad array of other science topics.
\begin{itemize}
\item Galaxy formation and evolution through interpretation of the
small-scale correlation functions (\citealt{zheng07a, leauthaud12a,
guo13a}).
\item Evolution of the most luminous galaxies out to \mbox{$z\sim 1$}
(e.g., \citealt{maraston13a, bundy15b, monterodorta16a}).
\item Nature of the circumgalactic medium through statistical absorption
studies (\citealt{steidel10a, zhu14a, zhu15a}).
\item Calibration of photometric redshifts through cross-correlation;
eBOSS provides this calibration for DES and validates this method
for use in future surveys such as LSST (\citealt{newman15a}).
\item Nature of the intergalactic medium in the range
$2<z<3.5$, using the damped Lyman$alpha$ systems, Lyman limit systems
, and the Lyman-$\alpha$ and Ly$\beta$ forests and their
cross-correlations with other tracers of structure.
(e.g. \citealt{becker13a, pieri14a, lee15a}). These techniques can
reveal signatures of He II reionization, the clustering of ionizing
sources, and can potentially detect Ly$\alpha$ emission.
\end{itemize}

We will discuss the quasar science in more detail in
Section \ref{sec:quasars}.

\subsubsection{eBOSS Targeting Strategy}
\label{sec:eboss:targeting}

\citet{dawson16a} presents an overview of the eBOSS targeting strategy, 
which aims primarily at surveying a large volume of the universe. The
eBOSS footprint covers 7500 deg$^{2}$, with approximately 4500
deg$^{2}$ in the North Galactic Cap (NGC) and 3000 deg$^{2}$ in the
South Galactic Cap (SGC). Luminous red galaxies (LRGs) and quasars
are targeted over the full eBOSS footprint.  An emission-line
galaxy (ELG) sample is targeted over 1000--1500 square degrees
starting in Fall 2016.  A 466 deg$^2$ pilot program was conducted in
SDSS-III and early SDSS-IV, designated the Sloan Extended Quasar, ELG,
and LRG Survey (SEQUELS; \citealt{dawson16a, alam15b}). SEQUELS tested
these target selection techniques. Figure \ref{fig:eboss_footprint}
shows the the currently planned eBOSS footprint, and
Table \ref{table:eboss_samples} summarizes the planned eBOSS samples
and the resulting cosmological constraints.

\begin{figure}[t!]
\centering
\includegraphics[width=0.49\textwidth,
  angle=0]{eboss_footprint.pdf}
\caption{ \label{fig:eboss_footprint}
Planned eBOSS spectroscopic footprint in equatorial coordinates,
centered at $\alpha_{\rm J2000} = 270^\circ$, with East to the left.
Grey areas are the BOSS spectroscopic footprint, and for eBOSS red
represents the planned LRG and quasar sample footprint, and blue shows
the planned ELG footprint.}
\end{figure}

The targeting strategy is driven by a desire to fill the existing gap
in cosmological large-scale structure measurements between $z\sim 0.6$
and $z\sim 2.5$, which is the transition from cosmic deceleration to
acceleration. With existing facilities, this range cannot be covered
over wide fields using a single tracer. Thus, we adopt a multi-tracer
strategy: extend the BOSS LRG sample to $z\sim 0.8$, introduce an
emission-line galaxy sample that can be selected and successfully
observed to $z\sim 1.1$, conduct a dense survey of quasars to $z\sim
2.2$, and enhance the BOSS quasar sample at $z>2.2$.
 
The full quasar sample is designed to cover $0.9<z<3.5$. The quasars
at redshifts $z<2.2$ are utilized as tracers of large-scale structure
themselves.  The quasars at $z>2.1$ are utilized as backlights for
\lya\ absorption, which measures the density of neutral gas
along the line of sight at those redshifts.  The core quasar target
selection is described by \citet{myers15a}, utilizing a redshift-binned
version of the Extreme Deconvolution (XD) algorithm applied to quasars
(XDQSOz; \citet{bovy11a, bovy12b}).  In the SDSS-IV case, we apply
XD on the SDSS photometry and its associated uncertainties to select
quasars, and then consult {\it WISE} photometry to veto sources likely to be
stars.  We do not observe quasars at $z<2.1$ that were
spectroscopically classified in prior SDSS surveys (which have a
density $\sim 13$ deg$^{-2}$), but these are included in
clustering analyses. eBOSS re-observes the fainter quasars at $z>2.1$ to
improve the signal-to-noise ratio in the \lya\ forest by a factor of
1.4.

The LRG sample is designed to cover $0.6<z<1.0$, with a median $z\sim
0.71$. eBOSS achieves this selection using a combination of SDSS $r$,
$i$, and $z$ photometry and {\it WISE} 3.4 $\mu$m photometry, as
described by
\citet{prakash15a}. The sample is limited at $z<19.95$ (using
Galactic extinction corrected SDSS model magnitudes).

The ELG sample is designed to cover $0.7<z<1.1$, with a median $z\sim
0.86$ \citep{comparat16a, jouvel16a}. The selection uses the deep $g$,
$r$, and $z$ band imaging from the Dark Energy Camera
(DECam; \citealt{flaugher12a}). The imaging is primarily drawn from a
combination of DES imaging and of the DECam Legacy Survey
(DECaLS\footnote{{\tt http://legacysurvey.org}}), a wide footprint
extragalactic imaging survey being conducted in preparation for DESI.
The ELG targets are observed at a high density ($>180$ deg$^{-2}$)
over 1000--1500 deg$^{2}$ split about equally between the SGC and
NGC. Because of the available imaging depth, the target density in the
SGC is high ($\sim 240$ deg$^{-2}$) and the efficiency of selecting
ELGs in the desired redshift range is around 80\%, whereas the density
($\sim 190$ deg$^{-2}$) and efficiency ($75\%$) are lower in the
NGC. In both regions, the median redshift is similar. These targets are
observed on separate plates from the LRG and quasar cosmological
surveys. These plates do not contain SPIDERS targets, but, as
described in Section \ref{sec:tdss}, they do include Repeat Quasar
Spectroscopy targets. ELG observations began in Fall 2016. A future
paper will describe the exact selection function, its redshift
distribution, as well as systematic weights to be applied for
large-scale structure analysis.

The eBOSS team also considered the use of other imaging data sets.  In
SEQUELS, \citet{comparat15a} drew ELG targets from the South Galactic
Cap U-band Sky Survey (SCUSS; \citealt{zou15a}) and SDSS. In the last
round of tests before the ELG program was
finalized, \citet{comparat16a} and \citet{raichoor16a} combined {\it WISE}
(\citealt{wright10a}), SCUSS, and SDSS to select ELG targets. The
final selection functions are nearly as efficient as the DECaLS
targeting but yielded a lower effective redshift.

For the LRG, ELG, and quasar clustering samples, eBOSS aims to create
uniform target selection with a maximum absolute variation (peak to
peak) of 15\% in the expected target number density. The expected
target number density is defined with respect to its estimated
dependence on imaging survey sensitivity, calibration errors, stellar
density, and Galactic extinction (\citealt{myers15a, prakash15a,
dawson16a}).

The targets are assigned to plates using a descendant of the tiling
algorithm adopted in the Legacy and BOSS surveys
(\citealt{blanton03a}). The eBOSS pointings are designed to cover
large contiguous areas in the NGC and SGC. Each pointing is referred
to as a tile, which typically (but not always) is associated with a
single physical plate.  Of the 1000 available fibers, 80 are assigned
to estimate the sky and 20 are assigned to bright $F$ stars used as
standard sources.  The TDSS and SPIDERS programs are included in the
tiling assignments and observed on the same plates as the eBOSS
targets.

eBOSS adopted a tiered-priority system for assigning survey targets to
plates, which leads to an efficient assignment of fibers and a
satisfactory level of completeness. All non-LRG targets receive
maximal priority and the tiling solution must achieve 100\% tiling
completeness for a set of all non-LRG targets that do not collide with
each other (a ``decollided'' set; see
\citealt{blanton03a}). For LRGs, eBOSS does not require full
decollided completeness. Rather, the density of LRG targets
intentionally oversubscribes the remaining fiber budget.  The average
density of LRGs assigned to fibers spectra is about 50 deg$^{-2}$.  In
areas of lower density in non-LRG targets, the LRGs can be observed up
to a density of about 60 deg$^{-2}$.  In areas of higher density in
non-LRG targets, the LRGs can be incomplete; however, eBOSS does require
that the total completeness of the decollided LRG targets be greater
than 95\%.  This layered tiling scheme allows 8\% more area to be
covered than otherwise would, at the cost of the variable completeness
of LRGs.

In the first round of fiber assignments --- the non-LRG targets --- 
eBOSS specifies the priority for fiber assignments when fiber collisions
occur.  Because the quasar targets have significantly higher density
than TDSS and SPIDERS targets, quasar-TDSS/SPIDERS collisions are
fractionally more common for TDSS/SPIDERS target classes.  Collisions
are resolved in the following order (highest to lowest priority):
SPIDERS, TDSS, reobservation of known quasars, clustering quasars, and
variability-selected quasars.  Quasars found in the FIRST survey
\citep{becker95a} and white dwarf stars that can be used as possible
calibration standards are given the lowest priorities for resolving
fiber collisions.

\citet{dawson16a} summarizes the overall expected numbers of
spectra. Nominal weather performance provides completion of $\sim$1800
plates, which would yield 1.62 million object spectra including about
180,000 unique TDSS and SPIDERS targets. Table
\ref{table:eboss_tdss_spiders} lists the numbers of confirmed
quasars at $z<2.1$, new and repeated BOSS quasars at $z>2.2$,
confirmed LRGs, and confirmed ELGs, assuming our estimated
efficiencies and redshift success rates.  The spectra that are
contaminants to the eBOSS cosmological sample are primarily blue stars
for quasar targeting and M stars for LRG targeting.

\begin{figure}[t!]
\centering
\includegraphics[width=0.49\textwidth,
  angle=0]{eboss_pie.png}
\caption{ \label{fig:eboss_pie} Slice along right ascension through
the eBOSS redshift sample, 5$^\circ$ wide in declination and centered
at $\delta = +22^\circ$. Black points indicate previously known
redshifts from SDSS-I through SDSS-III. Cyan points show eBOSS quasars
and red points represent eBOSS LRGs, each category selected as
described in Section \ref{sec:eboss:targeting}.  }
\end{figure}

\subsubsection{eBOSS Observations}
\label{sec:eboss:observations}

eBOSS utilizes approximately 50\% of the dark time at APO. Details of the
division of observations across the SDSS-IV surveys are given in
Section \ref{sec:apo}.

Each BOSS fiber is encased in a metal ferrule whose tip is relatively
narrow (2.154~mm) and is inserted fully into the plate hole, but whose
base is 3.722~mm in diameter and sits flat on the back of the plate.
Two fibers on the same plate therefore cannot be placed more closely
than 62$''$ from each other on the sky.  Thus, except where two tiles
overlap, only one of such a pair can be observed; these fiber
collisions affect both the small- and large-scale clustering signal
from the sample and must be accounted for in the analysis
(e.g. \citealt{guo12a}).

Each plate is designed for a specific hour angle of observation.  The
observability window is designed such that no image falls more than
0.3$''$ from the fiber center during guiding. This restriction limits
the range of LSTs in which a plate is observable.
% These limits are propagated to the SDSS-IV observing software and
% used by the scheduling software and observers in planning each
% night's observing.

eBOSS is designed for LRGs, ELGs, and quasars with $z<1.5$ to have a
redshift accuracy $<300$\,km s$^{-1}$ (root mean squared) at all
redshifts. Larger redshift errors have the potential to damp the BAO
feature in the radial direction, thus diluting the precision
achievable on $H(z)$. We require catastrophic errors (defined as
redshift errors exceeding $1000$\,km s$^{-1}$ that are not flagged) to
be $<1\%$. At higher redshifts, we aim for quasars to have a redshift
measurement accuracy $< 300 + 400(z - 1.5)$\,km s$^{-1}$. The increase
at higher redshift reflects the expected rising difficulty of accurate
redshift measurement. A small number of repeat spectra are obtained
where fibers are available, which allow an estimate of the
uncertainties in the redshifts.

To achieve these goals, eBOSS observations are designed to obtain
median $i$-band (S/N)$^2>22$ per pixel at a fiducial target magnitude
$i_{\rm fiber2}=21$ and median $g$-band (S/N)$^2>10$ per pixel at a
fiducial target magnitude $g_{\rm fiber2}=22$.  The dispersion of the
BOSS spectrographs delivers roughly 1~\AA\ per pixel.  Plates are
exposed until they satisfy this signal-to-noise ratio requirement.
First year data indicate that plates require 4.7 15-minute exposures
to exceed these requirements; during the first year, we slightly
exceeded the requirements and averaged 5.3 exposures per plate. The
mean overhead per completed plate is around 22 minutes (this time
averages over cases where a plate was observed on multiple nights).
These thresholds are designed to satisfy the above requirements on
redshift accuracy.  The observing depths are also established to
achieve a reliable classification of all targets, whereby catastrophic
errors are required to occur at a rate of less than 1\% for all target
classes.

%In past SDSS large-scale structure surveys, a set of non-colliding
%targets has been defined, the decollided targets
%(\citealt{blanton03a}). These decollided targets have typically been
%%observed at virtually 100\% completeness; this completeness comes at a
%cost, in that not all the fibers are used for the primary target
%set. The quasar and ELG samples in eBOSS are targeted in this
%fashion. However, the LRGs are not; they are targeted using all of
%the spare fibers after the quasar samples have been targeted. This
%procedure leads to the use of all of the available fibers, but a
%lower total LRG completeness. In addition, the completeness is
%variable and correlates with the background quasar density; this
%variable completeness needs to be accounted for in the analysis.

\subsubsection{eBOSS Data}
\label{sec:eboss:data}

eBOSS spectroscopic data consists of single-fiber $R\sim 2,000$
spectra in the optical (approximately 3600 \AA $< \lambda <$
10,350 \AA), at signal-to-noise ratios of $\sim$ 2--4 per pixel for
most targets, from which we determine redshifts and
classifications. The eBOSS pipeline is a slightly modified version of
the BOSS pipeline described by \citet{bolton12a}.
Figure \ref{fig:eboss-spectra} displays six example spectra from the
first year of eBOSS, processed through a preliminary version of the
eBOSS pipeline.

eBOSS data are processed through a quicklook pipeline (Son Of Spectro,
SOS) during each observation to estimate the signal-to-noise ratio in
real time and inform decisions about continuing to subsequent
exposures.  Quality assurance plots are examined each day to identify
unexpected failures of the observing system or pipelines.

Each morning following a night of eBOSS observations the data are
processed by the pipeline and made available for the collaboration.
The pipeline extracts the individual spectra using optimal extraction
(\citealt{horne86a}), and builds a spatially dependent model of the
sky spectrum from the 80 sky fibers and subtracts that model from each
object fiber. It determines the spectrophotometric calibration, which
includes the telluric line correction, using a set of 20 calibrator
standard stars observed on each plate, selected to have colors similar
to F stars and in the magnitude range $16<r_{\rm fiber2}<18$.
Redshifts are determined using a set of templates, with separate sets
for stars, galaxies, and quasars. For stars, the templates consist of
individual archetypes; for galaxies and quasars, the templates consist
of Principal Component Analysis (PCA) basis sets that are linearly
combined to fit the data at each potential redshift. The best redshift
and classification (star, galaxy, or quasar) is determined based on
the $\chi^2$ differences between the models and the data. For
galaxies, the pipeline also fits the velocity dispersion of the
galaxy, by comparing the spectra with linear combinations of a set of
high-resolution stellar templates. The pipeline conducts emission-line
flux and equivalent width measurements as well for a number of major
emission lines.

The pipeline undergoes continuous improvement as problems are
identified and repaired. Future versions will benefit from ongoing
efforts to improve sky subtraction and spectrophotometric
calibration. A new procedure and set of templates for fitting
redshifts is being developed to handle better the lower
signal-to-noise ratio of the fainter eBOSS targets. Specifically,
quasars and galaxies will use a large number of fixed archetypes
rather than a PCA basis set (\citealt{hutchinson16a}).

\begin{figure*}[t!]
\centering
\includegraphics[width=0.99\textwidth,
  angle=0]{eboss-spectra.pdf}
\caption{ \label{fig:eboss-spectra}
Six representative eBOSS spectra, showing an emission line galaxy, a
luminous red galaxy, a quasar from the core ``cosmological'' sample, a
quasar selected at $z>2.2$ for Ly-$\alpha$ forest studies, an X-ray
emitting quasar selected by SPIDERS, and a TDSS-selected variable
broad absorption line quasar (listed left-to-right, and
top-to-bottom). The locations of emission lines are labeled in blue,
and for the luminous red galaxy, those of absorption features are
labeled in red.}
\end{figure*}

The eBOSS pipeline has been applied to all SDSS-III BOSS data as well,
which were taken with the same instrument. We do not have plans to
reanalyze the previous SDSS-I and SDSS-II data from the SDSS
spectrographs.

The first SDSS-IV data release (DR13; 2016 July) contains a
rereduction of BOSS data through the latest version of the pipeline
and includes plates from SDSS-IV completing the SEQUELS sample. In
DR14, the first two years of eBOSS data will be released.

The quasar science team within eBOSS plans to continue to maintain the
SDSS quasar catalog, the latest version of which is DR12Q
(\citealt{paris14a}). This catalog includes visually vetted redshifts
and classifications and has greater reliability than the standard
pipeline results. In DR12Q, all quasar spectra were inspected visually
by at least two people. However, in eBOSS a greater amount of
automatic vetting reduces the number of quasars that need to be
inspected visually.

