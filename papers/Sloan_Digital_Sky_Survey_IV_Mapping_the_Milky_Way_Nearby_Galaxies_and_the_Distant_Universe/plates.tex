The plates used at APO and LCO are produced for SDSS-IV using the same
systems used in previous SDSS programs.  The plates themselves are 3.2
mm thick aluminum plates, 80 cm in diameter, with a 65.2 cm diameter
region in which holes can be drilled to place fibers. Each fiber or
IFU is housed in a metal ferrule whose tip ranges in size from 2.154
mm to 3.25 mm in diameter. The larger diameter ferrules are employed
in the MaNGA and LCO systems; all others use a 2.154 mm diameter (see
Sections \ref{sec:apogee2:observations}, \ref{sec:manga:observations},
and \ref{sec:eboss:observations} for details). The ferrules have a
larger base that rests on the back side of the plate to keep the fiber
tip position fixed in focus.

Each survey plans potential observations several months in advance and
determines the sky coordinates and optimal Local Sidereal Times (LSTs)
for a set of plates. Based on the target selection results, the
potential targets in each field are assigned fibers. The fiber
placements have some physical constraints, most significantly with
regard to the minimum separation of fibers. Other constraints on the
fiber assignment based on target type and brightness can be applied.
These constraints are described below for APOGEE-2, MaNGA, and eBOSS.

Given a desired observation at a given celestial location and LST, the
target coordinates are translated into observed altitude and azimuth
given atmospheric refraction and the observatory location.  These
coordinates are translated into the physical focal plane location of
each target image, based on telescope scale and distortions. Finally,
the focal plane location is translated into a drilling location taking
into account the relative bending of the plate and the thermal
expansion of the plate due to the difference between the drill shop
temperature and the estimated observing temperature.

A large format vertical milling machine (a Dah Lih MCV-2100) at the
University of Washington drills each plate
(\citealt{siegmund98a}). During drilling, the APO plates are bent on a
mandrel such that the fiber angle will be aligned with the chief ray
at that position on the focal plane. The LCO plates are fixed to a
flat fixture, since, for the du Pont Telescope, the chief ray is
normal to the focal plane.

When observed at APO, the plates are bent to match the focal plane
curvature at around 5400 \AA. The \Hband\ focal plane has a slightly
smaller radius of curvature. In order for APOGEE fibers to remain near
the \Hband\ focus in the outer parts of the plate, a shallow
``counterbore'' is drilled on the back side of the plate, so that when
the base of the ferrule rests inside this counterbore, the fiber tip
extends beyond the plate surface slightly in order to reach the
\Hband\ focal plane.  When observed at LCO, the plates are bent to
match the focal plane in the \Hband, so no counterboring is
necessary.

At both observatories, the bending is achieved using a center post
with a 4.87 mm radius. We insert a further 1.1 mm buffer between the
post and the outer diameter of any ferrule, restricting the placement
of targets very near the centers of plates.

A Coordinate Measuring Machine measures a subset of holes on each
plate for quality assurance purposes. The typical errors measured in
hole position are 10 $\mu$m. This error has increased somewhat over
time from 7 $\mu$m since the system was first installed in
1996. However, this contribution to the total fiber position error is
subdominant.  As plugged, the median fiber position offset is 13
$\mu$m; 90\% of fibers do better than 22 $\mu$m.  The most important
error contribution arises from the slight ``clearance'' tilts induced
when each fiber is plugged, because the holes are by necessity
slightly larger than the ferrules.
