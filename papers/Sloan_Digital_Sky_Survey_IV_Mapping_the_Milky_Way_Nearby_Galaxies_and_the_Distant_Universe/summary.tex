We have described SDSS-IV, which began operations in 2014 July, with
plans to continue until mid-2020. The collaboration has over 1,000
participating astronomers from over 50 institutions worldwide. Three
major programs (APOGEE-2, MaNGA, and eBOSS) and two subprograms (TDSS
and SPIDERS) will address a number of key scientific topics using
dual-hemisphere wide-field spectroscopic facilities. The major
elements of this science program are as follows.
\begin{itemize}
\item Milky Way formation history and evolution, using chemical and
  dynamical mapping of all of its stellar components with APOGEE-2.
\item Stellar astrophysics, using APOGEE-2 infrared spectra alone and
  in combination with asteroseismology, using TDSS's optical
  observations of variable stars, and using MaNGA's bright-time
  optical stellar library.
\item Formation history and evolution of the diverse array of galaxy
  types, using chemical and dynamical mapping of stars and gas with
  MaNGA integral field spectroscopy, using the distant galaxy
  populations in the eBOSS LRG and ELG programs, and the cluster
  galaxies in SPIDERS.
\item Quasar properties and evolution using the massive sample of
  quasars in eBOSS, reaching nearly down to Seyfert galaxy
  luminosities out to $z\sim 2$, complemented with quasars selected
  via variability (TDSS) and X-ray emission (SPIDERS).
\item The most powerful cosmological constraints to date from
  large-scale structure, precisely investigating the Hubble diagram
  and the growth of structure in the redshift range $1<z<2$ for the
  first time, using the largest volume cosmological large-scale
  structure survey to date from eBOSS.
\end{itemize}

The science program is coupled to a robust education and public
engagement program. All of the raw and reduced data will be released
on a well-defined schedule using innovative public interfaces.
