\begin{figure*}[t!]
\centering
\includegraphics[width=0.9\textwidth,
  angle=0]{nbar-qso.pdf}
\caption{ \label{fig:nbar-qso} Distribution of quasars in redshift and
rest-frame $i$-band absolute magnitude. Top panel: contours
show the density of Legacy and BOSS quasars in this plane from SDSS-I
through SDSS-III.  The grayscale represents the density of eBOSS,
TDSS, and SPIDERS quasars from SDSS-IV from the first year results. In
the range $1<z<2$ the SDSS-IV quasars probe much lower luminosities
than previous SDSS samples. The gray horizontal line corresponds to
$M_\ast$ for galaxies (\citealt{blanton04b}); the SDSS-IV quasars out
to $z\sim 2$ approach the faintness of Seyfert galaxies in optical
luminosity. Bottom panel: each histogram shows the density of
quasars as a function of redshift. The gray histogram is for Legacy
and BOSS quasars from SDSS-I through SDSS-III. The blue histogram
shows the estimated density of eBOSS quasars from the first year
results. In the range $1<z<2$ the eBOSS sample represents an increase
in density by factors of 5--10.}
\end{figure*}

eBOSS, TDSS, and SPIDERS together select more than half a million
quasar targets.  This enormous quasar catalog (tripling the world's
number of quasar spectra) includes objects targeted by optical and
mid-IR ({\it WISE}) colors, variability (TDSS), radio (FIRST), and
X-ray emission (SPIDERS).  Combined with previous SDSS and BOSS
observations, the catalog spans a factor of more than $\sim1000$ in
accretion luminosity from $z=0$ to $z=5$.  Whereas previous surveys
have sampled different quasar luminosity classes at different
redshifts, the SDSS-IV sample enables an understanding of individual
classes of quasars across epochs and better trace the full history of
active BH growth since $z\approx3$.  Figure \ref{fig:nbar-qso} shows
the increased density of quasars in SDSS-IV relative to previous SDSS
surveys, as well as its extension to fainter luminosities in the range
$1<z<2$.

The best measurements of the Type I quasar luminosity function at
$z<2$ from optical survey data come from 10,000 quasars compiled by
the 2dF-SDSS LRG and QSO (2SLAQ) survey \citep{croom09a}; using deeper
data, previous SDSS programs have extended to higher redshifts but
have not probed these lower redshifts as densely
(\citealt{palanquedelabrouille13a}). This survey targeted quasars to a
similar depth as eBOSS (though the eBOSS limit of $r<22$ reaches
many more quasars than the 2SLAQ limit of $g<21.85$), but over an area
$\sim40$ times smaller.  The statistical power provided by the large
--- and highly complete --- eBOSS sample provides a powerful new
probe of the evolution of the faint-end slope of the luminosity
function over the interval from $z=1$ to $z=2$, strongly constraining
feedback models for black hole growth \citep[e.g.,][]{hopkins06b}.

Combining measurements of the faint end of the luminosity function
with precision probes of quasar clustering constrains models for
quasar lifetimes, the typical halos hosting quasars, the co-evolution
of quasars and spheroidal galaxies, and the evolution in black hole
mass of active quasars (using virial mass estimators). Within the
redshift range $1<z<2$, the mass of black holes powering quasars is
expected to decrease with increasing redshift by an order of
magnitude, perhaps symptomatic of the characteristic fueling mechanism
shifting from major mergers to secular processes
(\citealt{hopkinshernquist06}). This prediction can be robustly tested
with eBOSS's measurements of the luminosity dependence of quasar
clustering. Finally, cross-correlation analyses of eBOSS galaxies and
quasars at redshifts where samples overlap provides unique insight
into the connection between quasars and galaxies (both quenched and
star-forming).

Selecting quasars using several different techniques within eBOSS,
TDSS, and SPIDERS allows SDSS-IV to account for the selection biases
that affect any individual quasar selection technique.  For example,
the SDSS-iV data enables the comparison of high-redshift quasars
with lower luminosity, X-ray selected AGNs at low redshift that may
represent their descendants.  A further advantage provided by SDSS-IV
is the ability to tie together the faint quasar population at optical
(eBOSS) and X-ray (SPIDERS) wavelengths within the same
survey. Reaching the optically fainter quasar population provides
access to a much larger number of significantly reddened quasars,
yielding a more complete census of narrow-line and reddened broad-line
AGNs.

Large quasar samples are useful not only for demographic studies, but
also for yielding rare phenomena. Repeat spectroscopy of known quasars
through TDSS captures changes in the absorption profiles of clouds
along the line of sight to quasar nuclear
regions \citep[e.g.,][]{ak13a}, rare state changes when the nuclear
emission effectively vanishes \citep[so-called ``changing-look''
quasars;][]{lamassa15a,runnoe16a}, and a variety of other
time-dependent phenomena traced by multi-epoch quasar
spectroscopy. The unprecedented density of quasar targeting
within SDSS-IV, particularly when considering that most known quasars
will not be re-targeted and thus can have nearby objects targeted
within the fiber collision radius, probes the environments of
quasars through small-scale clustering with far greater numbers and
more uniformity than achieved even by dedicated surveys of quasar
pairs \citep[e.g.,][]{hennawi06a}.  Combining small-scale quasar pairs
with the large-scale clustering sample from eBOSS constrains halo
occupation models of quasars over a wide range of both luminosity and
spatial scales and permit detailed examination of the relationship
between quasar triggering and environment.

There are three quasar programs that SDSS-IV is executing to
enhance quasar science: a complete sample of AGNs on Stripe 82, a
continuation of the SDSS-RM program, and a program for repeat quasar
spectroscopy.

First, ``Stripe82X'' provides a focused effort to build
a complete sample of AGNs with SDSS-IV spectroscopy, with a set of six
spectroscopic plates dedicated to AGN targets.  The plates span a
footprint of $\sim35~{\rm deg}^2$ within the SDSS Stripe 82 region,
bounding the area defined by the Stripe 82 X-ray survey of
\citet{lamassa2016} between $\alpha_{\rm J2000}=14^\circ$ and 
$\alpha_{\rm J2000}=28^\circ$. X-ray sources drawn
from \citet{lamassa2016} with optical counterparts having
$r<22.5$ provide the primary target class for the Stripe82X survey,
totaling nearly 900 objects. The remaining fibers on each plate are
primarily assigned to {\it WISE}-selected AGN \citep[using the R75 color
criteria of][]{assef13a} and variability-selected quasars
\citep{peters15a, palanquedelabrouille16a}. A small number of high-redshift 
quasar candidates and repeat observations of ``changing look" and
related quasar candidates using TDSS selection criteria are also
included. The tiling includes roughly 5,000 AGN targets. The primary
goals of the Stripe82X program are: (1) to better characterize AGN
bolometric corrections by combining the spectroscopy with the
extensive multiwavelength photometry available on Stripe 82; (2) to
explore and compare the diverse classes of AGN selected by different
wavelength regimes; and (3) to construct a bolometric AGN luminosity
function from a highly complete, faint AGN sample.

Second, during dark time, SDSS-IV is continuing the SDSS-RM program
(\citealt{shen15a}) initiated during the last observing semester of
SDSS-III in 2014 (\citealt{alam15b}). SDSS-RM monitors a sample of 849
quasars within a single 7 deg$^2$ field with BOSS spectroscopy and
accompanying photometry to measure quasar broad-line time lags with
the reverberation mapping technique (e.g., \citealt{blandford82a,
peterson93a}).  In eBOSS, the SDSS-RM spectroscopy has a cadence of 2
epochs (similar depth to eBOSS) per month (12 epochs/year) since 2015,
and provides an extended temporal baseline to detect broad-line lags
on multi-year timescales in high-redshift quasars when combined with
earlier SDSS-RM data.

Third, a Repeat Quasar Spectroscopy (RQS) program emphasizing known
quasars is being observed in the eBOSS ELG region discussed in
Section \ref{sec:eboss:targeting}, supplementing the TDSS few-epoch
spectroscopy. In this $\sim10^3$~deg$^2$ region, TDSS is also
obtaining a new epoch of spectroscopy for previously-known SDSS
quasars. In this region, we include quasars with $17<i<21$ (also
including morphologically extended AGNs) from the DR7 or DR12 quasar
catalogs, or SDSS-IV objects with spectro-pipeline class ``QSO'' that
have been vetted as quasars/AGNs by our own visual inspection of the
spectra. As part of the ELG plates, TDSS observes a total of
$\sim10^4$ known quasars/AGNs for an additional epoch of spectroscopy,
including the bulk of all known SDSS quasars in this region to
$i<19.1$, as well as filling additional available fibers for RQS with
either: known SDSS quasars extending to $i<20.5$ already having more
than one extant epoch of on-hand spectroscopy; and/or additional of
the most highly variable known SDSS quasars in the ELG region, as
determined from a reduced chi-squared measure of their photometric
variability in SDSS and PS1 imaging. Details of RQS target selection
will be reported in a future publication (MacLeod et al. 2017, in
preparation).

SDSS-IV maintains the tradition established by the previous
incarnations of the survey to publicly release quasar catalogs
\citep[e.g.,][]{schneider10a,paris14a} associated with each release of
new spectroscopic data.  In SDSS-III, starting from the output of the
the SDSS pipeline \citep{bolton12a}, the spectrum of each quasar
target was visually inspected to confirm both its identification and
redshift.  This procedure ensured the high purity of the catalog
content and contributed to improvements in the SDSS pipeline.  The
quasar target density of SDSS-IV is approximately three times larger
than in SDSS-III. This increase combined with the amount of time
required to perform a systematic visual inspection of all quasar
targets forces us to adapt our strategy to construct quasar catalogs.
Hence, we developed a semi-automated scheme: starting from the output
of the SDSS pipeline, we identify spectra for which the identification
and/or redshift produced by the automated pipeline are
questionable. The spectra of these objects ($\sim$7\% of the targets)
are then visually inspected. This automated strategy was tested
against a fully visually inspected sample drawn from the SDSS-IV pilot
survey performed at the end of SDSS-III and its design delivers a
quasar catalog with a purity larger than 99\% and a loss of less than
1\% of actual quasars \citep[see ][ for more details]{dawson16a}.

The content of the SDSS-IV quasar catalog is similar to the previous
ones. Multiwavelength information is provided when available along
with spectroscopic properties such as emission-line fitting, presence
of broad absorption lines and improved redshift estimates.  At the
conclusion of SDSS-IV, the photometric and spectroscopic properties of
about a million quasars will be released.
