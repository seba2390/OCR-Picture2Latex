For the purposes of the SDSS-IV survey targeting, we have undertaken
the reanalysis of a variety of existing imaging data sets.  We will refer
to these data sets in subsequent sections describing the survey
programs.

We have applied a photometric recalibration to the SDSS imaging data
set. Using the PS1 photometric calibrations
of \citet{schlafly12a}, \citet{finkbeiner16a} have rederived the $g$,
$r$, $i$, and $z$ band zero points and the flat fields in all five SDSS
bands (including $u$).  The residual systematics are reduced to 0.9,
0.7, 0.7, and 0.8\% in the $griz$ bands, respectively; several
uncertain calibrations of specific imaging scans are also now much
better constrained. The resulting recalibrated images and imaging
catalogs are the basis for the eBOSS and MaNGA targeting.  They are
now included as the default imaging data set in SDSS-IV public data
releases, starting in DR13. 

All the targeting based on SDSS imaging in SDSS-IV uses the DR9
astrometric calibration \citep{pier03a,ahn12a} for both targets and
for guide stars. The SDSS-III BOSS survey used the previous DR8
astrometric calibration, which has known systematic errors. Because
the systematic errors were fairly coherent over the SDSS
field-of-view, the fiber flux losses due to these errors were
relatively minor.

For the purposes of the MaNGA target selection, we are using the
NASA-Sloan Atlas (NSA; \citealt{blanton11a}), a reanalysis of the SDSS
photometric data using sky subtraction and deblending better tuned for
large galaxies. Relative to the originally distributed version of that
catalog, we have used the new calibrations mentioned above, increased
the redshift range to $z=0.15$, and have added an elliptical aperture
Petrosian measurement of flux, which MaNGA targeting is based upon.

For the purposes of eBOSS target selection, \citet{lang16a} reanalyzed
data from {\it WISE} (\citealt{wright10a}). Using positions and galaxy
profile measurements from SDSS photometry as input structural models,
they constrained {\it WISE} band fluxes using the {\it WISE}
imaging. These results agree with the standard {\it WISE} photometry
to within 0.03 mag for high signal-to-noise ratio, isolated point
sources in {\it WISE}.  However, the new reductions also provide flux
measurements for low signal-to-noise ratio ($<5\sigma$) objects
detected in the SDSS but not in {\it WISE} (over 200 million
objects). Despite the fact that the objects are undetected, their flux
measurements are nevertheless informative to target selection, in
particular, for distinguishing stars from quasars.  These results have
been used for eBOSS targeting and have been released in DR13.

Several additional imaging analyses have been performed for targeting
SDSS-IV data; these extra sources of imaging will not necessarily be
incorporated into the SDSS public data releases, although some of them
have been released separately. 
\begin{itemize}
\item Variability analysis of Palomar Transient Factory
(PTF; \citealt{law09a}) catalogs to detect quasars 
(\citealt{palanquedelabrouille16a};
Section \ref{sec:eboss:targeting}). 
\item Selection of variable sources from PS1 (\citealt{morganson15a};
Section \ref{sec:tdss:targeting}). 
\item Intermediate-band imaging in Washington $M$, $T_2$ and 
DDO~51 filters for APOGEE-2 (\citealt{majewski00a,zasowski13a};
Section \ref{sec:apogee2:targeting}).
\item Selection of emission-line galaxies from the Dark Energy Camera
Legacy Survey (DECaLS), a $g$, $r$ and $z$ band photometric survey
being performed in preparation for the Dark Energy Spectroscopic
Instrument (DESI; \citealt{levi13a}) project.
\end{itemize}

For the purposes of eBOSS and MaNGA targeting, we correct magnitudes
for Galactic extinction using the \citet{schlegel98a} models of dust
absorption.  Galactic extinction coefficients have been updated as
recommended by \citet{schlafly11a}.  The extinction coefficients
$R_u$, $R_g$, $R_r$, $R_i$, and $R_z$ are changed from the values used
in BOSS (5.155, 3.793, 2.751, 2.086, and 1.479) to updated values
(4.239, 3.303, 2.285, 1.698, and 1.263).  We set $R_{W1}=0.184$ for
the {\it WISE} 3.4 $\mu$m band and $R_{W2}=0.113$ for the 4.6 $\mu$m
band \citep{fitzpatrick99a}.
