\documentclass[a4paper,11pt]{article}

\usepackage[english]{babel} 	%langue
\usepackage[utf8]{inputenc}      	%encodage des caractères 
\usepackage[T1]{fontenc}
\usepackage{aas_macros}
\usepackage{graphicx}
\usepackage{caption}
\usepackage{natbib}
\usepackage[margin=0.3in]{geometry}
\usepackage{amsmath}  % Advanced maths commands
%\usepackage{amssymb} % Extra maths symbols
\usepackage{hyperref}
\usepackage{pdflscape}
\usepackage{wasysym}
\usepackage{siunitx}
\usepackage{mathtools}
\usepackage{color}
\usepackage{gensymb}
\captionsetup[figure]{labelfont={bf},name={Figure},labelsep=period}
\usepackage[font=small]{caption}


\makeatletter
\def\@endpart{\vskip50pt}
\makeatother

\addtolength{\topmargin}{0.4in}
\addtolength{\textheight}{-0.4in}

\begin{document}
  

\onecolumn
  
  \begin{center}
  \Large \textbf{SUPPLEMENTARY APPENDICES}\\[0.5cm]
  \end{center}


\setcounter{topnumber}{4}% Just for this example
\setcounter{totalnumber}{4}% Just for this example

\newcommand{\fignumprefix}{}
\renewcommand{\thefigure}{\fignumprefix\arabic{figure}}
\renewcommand{\theHfigure}{figure.\thefigure}
\newcommand{\setfignumprefix}[1]{%
	\renewcommand{\fignumprefix}{#1}% Update figure counter prefix
	\setcounter{figure}{0}% Reset figure counter
}

\textbf{APPENDIX B: ORBITAL EVOLUTION}\\ \label{appendix:orb_evolution_all}

\textbf{B1  Semi-major axis and eccentricity of TC candidates}\\

\setfignumprefix{B}
\begin{figure*}[!ht]
  \centering
  \includegraphics[width=0.47\textwidth]{ae_1.png}
  \includegraphics[width=0.47\textwidth]{ae_2.png}\\
  \includegraphics[width=0.47\textwidth]{ae_3.png}
  \includegraphics[width=0.47\textwidth]{ae_4.png}\\
  \includegraphics[width=0.47\textwidth]{ae_5.png}
  \includegraphics[width=0.47\textwidth]{ae_6.png}
  \caption{Nominal eccentricity and semi-major axis of the 51 NEAs integrated and comet 2P/Encke over $\pm$20 000 years. The orbital elements of the bodies every year are linked with a solid line, while their value every 100 years is represented by the filled circles along the line.}
  \label{fig:ae_allNEAs}
\end{figure*}

\textbf{B2  Precession cycle of 2004 TG10, 2005 TF50, 2005 UR and 2015 TX24}\\

\begin{figure*}[!ht]
  \centering
  2004 TG10 \hspace{7cm} 2005 TF50\\
  \includegraphics[width=.49\textwidth]{2004TG10_angles_zoom.png}
  \includegraphics[width=.49\textwidth]{2005TF50_angles_zoom.png}\\[0.2cm]
  2005 UR \hspace{7cm} 2015 TX24\\
  \includegraphics[width=.49\textwidth]{2005UR_angles_zoom.png}
  \includegraphics[width=.49\textwidth]{2015TX24_angles_zoom.png}\\
  \caption{Inclination, perihelion argument and longitude of the ascending node of NEAs 2004 TG10, 2005 TF50, 2005 UR and 2015 TX24. The asteroids were integrated from the nominal solutions provided in Table A1 over a period of 30 000 years.}
  \label{fig:All_angles}
\end{figure*}


\newpage
\textbf{APPENDIX C: \text{7:2} MEAN MOTION RESONANCE}\\ \label{appendix:resonant_asteroids}

Mean motion resonances (MMRs) with Jupiter are known to influence the orbital evolution of the objects associated with the Taurid Complex. In particular, the 7:2 MMR is responsible for the formation of a resonant branch of meteoroids, that increases the level of activity of the Taurid meteor shower when approaching Earth. The resonant branch may also include large NEAs, of order a few hundred meters to 1 km in size, that are protected from dynamical diffusion by the resonance over long periods of time. 

Since the 7:2 MMR may play an important role on the evolution of the TC members investigated, we determined the proportion of objects in our sample that were likely trapped in the resonance during several millennia. To that end, for each body integrated, we computed the value of the resonant argument $\sigma$ as described in \cite{Asher1991}:  

\begin{equation}
\sigma =  7\lambda_J - 2\lambda - 5\varpi
\end{equation}

where $\lambda_J$ is the mean longitude of Jupiter, and $\lambda$ and $\varpi$ the mean longitude and  longitude of perihelion of the body integrated. If the body evolves inside the resonance, the resonant angle $\sigma$ librates around 0, with an amplitude between 0 and 180$\degree$. If the amplitude of the oscillations is low, the object is strongly trapped into the resonance. The top panel of figure \ref{fig:example_resonant} illustrates the case of a body not evolving into the 7:2 MMR (i.e., comet 2P/Encke), while the bottom panel represents the evolution of an object strongly trapped in the resonance (namely 2005 UY6). 

For each body in our sample, we analyzed the evolution of the resonant argument $\sigma$ for every clone created from the body's nominal orbit. We then determined the percentage of clones trapped into the 7:2 MMR over time. Figure \ref{fig:percent_resonant} presents the percentage of resonant clones of each body integrated over the past and future 20 000 years. For clarity reasons, the objects with the highest percentage of resonant clones are represented in color, while objects with less than 20\% of the clones evolving in the 7:2 MMR most of the time are drawn in grey. 


\setfignumprefix{C}
\begin{figure}[!ht]
	\centering
	\includegraphics[trim={0cm 0.0cm 0.0cm 0.0cm}, clip, width=\textwidth]{Example_resonant2.png}
	\caption{7:2 resonant arguments for comet 2P/Encke and asteroid 2005 UY6 based on integrations of the nominal orbits provided in Table A1. 2005 UY6 has been trapped into the 7:2 MMR with Jupiter over the past millennia, while 2P/Encke evolves outside the resonance. }
	\label{fig:example_resonant}
\end{figure}

From this figure, we see that NEAs 2005 UY6, 2015 TX24, 2005 TF50, 2019 UN12, 2003 UL3 and 2005 UR have at least 50\% of their clones trapped into the resonance during several thousand of years. In particular, most of the clones of 2005 UY6, 2015 TX24, 2005 TF50 and 2019 UN12 stay in the resonance during the whole integration period. In a lower extent, about 20\% of the clones of 2019 AN12, 2019 RV3 and 2020 JV remained in the resonance during the integration period. The recently observed NEA 2019 BJ1 clearly evolves in the 7:2 MMR at the current epoch, but its number of resonant clones decreases significantly after 5000 years of integration. 

In figure \ref{fig:percent_fireballs}, we present the results of the same analysis applied to the fireballs investigated in Section 5. As expected, all the fireballs except the comparator Southern Taurid EN241015\_185031 evolved in the 7:2 MMR at the epoch of observation in 2015. During the backward integration, the percentage of resonant clones decreases regularly for most of the fireballs, reaching a minimum of about 30\% resonant clones around 10 000 BCE. We therefore conclude that all the fireballs remained inside or close to the 7:2 MMR during the integration period, confirming their membership in the Taurid Swarm Complex. 

\begin{figure}[!ht]
	\centering
	\includegraphics[width=.49\textwidth]{percent_resonant_past.png}
	\includegraphics[width=.49\textwidth]{percent_resonant_future.png}\\[0.2cm]
	\includegraphics[width=.55\textwidth]{caption.png}
	\caption{Percentage of clones, for each body, evolving into the 7:2 MMR with Jupiter. Objects with at least 20\% of clones belonging to the resonance in the past are shown in color, while the other ones are drawn in grey.}
	\label{fig:percent_resonant}
\end{figure}

\begin{figure}[!ht]
	\centering
	\includegraphics[width=.5\textwidth]{percent_resonant_fireballs.png}\\[0.2cm]
	\includegraphics[width=.5\textwidth]{caption_fireballs.png}
	\caption{Percentage of clones, for each fireball, evolving into the 7:2 MMR with Jupiter.}
	\label{fig:percent_fireballs}
\end{figure}



\onecolumn
\textbf{APPENDIX D: SENSITIVITY ANALYSIS} \label{appendix:sensitivity} \\

\textbf{D1  Different MOID thresholds}\\

\setfignumprefix{D}
\begin{figure*}[!ht]
  \centering
  \includegraphics[width=.49\textwidth]{CA_2004TG10_2P0_MOIDs_500ms.png}
  \includegraphics[width=.49\textwidth]{CA_2005TF50_2P0_MOIDs_500ms.png}\\
  \includegraphics[width=.49\textwidth]{CA_2005UR_2P0_MOIDs_500ms.png}
  \includegraphics[width=.49\textwidth]{CA_2015TX24_2P0_MOIDs_500ms.png}
  \caption{Percentage of clones of NEAs 2004 TG10, 2005 TF50, 2005 UR and 2015 TX24 approaching at least one clone of comet 2P/Encke with a relative velocity below 500 m/s and different MOID thresholds (0.05, 0.01, 0.005, 0.001, 0.0005, 0.0001, 0.00005 and 0.00001 AU). }
  \label{fig:different_MOIDs}
\end{figure*}

\newpage
\textbf{D2  Different relative velocity thresholds}\\

\begin{figure*}[!ht]
  \centering
  \includegraphics[width=.49\textwidth]{CA_2004TG10_2P0_relVs_05.png}
  \includegraphics[width=.49\textwidth]{CA_2005TF50_2P0_relVs_05.png}\\
  \includegraphics[width=.49\textwidth]{CA_2005UR_2P0_relVs_05.png}
  \includegraphics[width=.49\textwidth]{CA_2015TX24_2P0_relVs_05.png}
  \caption{Percentage of clones of NEAs 2004 TG10, 2005 TF50, 2005 UR and 2015 TX24 approaching at least one clone of comet 2P/Encke with a MOID below 0.05 AU and different relative velocities limits.}
  \label{fig:different_relVs}
\end{figure*}

\newpage
\textbf{D3  Sensitivity to 2015 TX24's initial orbit}\\

\begin{figure*}[!ht]
  \centering
  \includegraphics[width=.8\textwidth]{2015TX24_variations.png}
  \caption{Percentage of clones of NEA 2015 TX24 (with slightly modified initial orbital elements as shown in the subplot legends) approaching at least one clone of comet 2P/Encke with a MOID below 0.05 AU.}
  \label{fig:2015TX24_variations}
\end{figure*}

\newpage
\textbf{APPENDIX E: COMPLEMENTARY RAPPROCHEMENTS ANALYSIS}\\ \label{appendix:additional_rapprochements} 

\textbf{E1  G5: Extension to $\pm$ 100 000 years} \label{sec:100Kyr}\\

\setfignumprefix{E}
\begin{figure*}[!ht]
  \centering
  \includegraphics[width=.9\textwidth]{CA_2P_100Kyr.png}\\
  \includegraphics[width=.9\textwidth]{CA_2004TG10_100Kyr.png}\\
  \includegraphics[width=.9\textwidth]{CA_other3_100Kyr.png}\\
  \includegraphics[width=.9\textwidth]{CA_2004TG10_2P_100Kyr.png}\\
  \caption{Percentage of clones of body \#1 (in blue) and body \#2 (in black) approaching at least a clone of the other object with a MOID smaller than 0.05 AU and a relative velocity below 500 m/s. }
  \label{fig:Main5_100Kyr}
\end{figure*}

\newpage
\textbf{E2  NEA 2005 UY6}\\

\begin{figure*}[!ht]
  \centering
  \includegraphics[width=\textwidth]{CA_2005UY6_500ms.png}\\
  \caption{Percentage of clones of body \#1 (in blue) and body \#2 (in black) approaching at least a clone of the other object with a MOID smaller than 0.05 AU and a relative velocity below 500 m/s. }
  \label{fig:2005UY6}
\end{figure*}

\textbf{E3  Fireballs' close approaches with 2P/Encke, $V_r$ threshold of 500 m/s}\\

\begin{figure*}[!ht]
  \centering
  \includegraphics[width=\textwidth]{Ev1_05AU_500ms.png}
  \caption{Percentage of clones of the fireballs (in blue) and comet 2P/Encke (in black) approaching at least a clone of the other object with a MOID smaller than 0.05 AU and a relative velocity below 500 m/s. The grey shaded areas encompasses the period 2500 BCE to 5500 BCE, and the two magenta areas the years 3000 BCE to 3500 BCE and 4400 BCE to 4900 BCE respectively.}
  \label{fig:fireballs_with_peak_500ms}
\end{figure*}

\begin{figure*}[!ht]
  \centering
  \includegraphics[width=\textwidth]{Ev2_05AU_500ms.png}
  \caption{Percentage of clones of the fireballs (in blue) and comet 2P/Encke (in black) approaching at least a clone of the other object with a MOID smaller than 0.05 AU and a relative velocity below 500 m/s. The grey shaded areas encompasses the period 2500 BCE to 5500 BCE, and the two magenta areas the years 3000 BCE to 3500 BCE and 4400 BCE to 4900 BCE respectively.}
  \label{fig:fireballs_without_peak_500ms}
\end{figure*}

\clearpage
\newpage
\textbf{E4  Fireballs' close approaches with 2004 TG10}\\

\begin{figure*}[!ht]
  \centering
  \includegraphics[width=\textwidth]{Ev1_05AU_500ms_2004TG10.png}
  \caption{Percentage of clones of the fireballs (in blue) and 2004 TG10 (in black) approaching at least a clone of the other object with a MOID smaller than 0.05 AU and a relative velocity below 500 m/s. The grey shaded areas encompasses the period 2500 BCE to 5500 BCE, and the two magenta areas the years 3000 BCE to 3500 BCE and 4400 BCE to 4900 BCE respectively.}
  \label{fig:fireballs1_2004TG10}
\end{figure*}

\begin{figure*}[!ht]
  \centering
  \includegraphics[width=\textwidth]{Ev2_05AU_500ms_2004TG10.png}
  \caption{Percentage of clones of the fireballs (in blue) and 2004 TG10 (in black) approaching at least a clone of the other object with a MOID smaller than 0.05 AU and a relative velocity below 500 m/s. The grey shaded areas encompasses the period 2500 BCE to 5500 BCE, and the two magenta areas the years 3000 BCE to 3500 BCE and 4400 BCE to 4900 BCE respectively.}
  \label{fig:fireballs2_2004TG10}
\end{figure*}

\clearpage
\textbf{APPENDIX F: CLONES ORBITAL ELEMENTS AROUND 3200 BCE}\\ \label{appendix:clones_selected} 

\textbf{F1  G5}\\


\setfignumprefix{F}
\begin{figure*}[!ht]
  \centering
  2004 TG10 \hspace{7cm} 2005 TF50\\ 
  \includegraphics[width=.49\textwidth]{2004TG10_w2P_clones.png}
  \includegraphics[width=.49\textwidth]{2005TF50_w2P_clones.png}\\
  2005 UR \hspace{7cm} 2015 TX24\\
  \includegraphics[width=.49\textwidth]{2005UR_w2P_clones.png}
  \includegraphics[width=.49\textwidth]{2015TX24_w2P_clones.png}\\
  \caption{Semi-major axis and eccentricity of the clones created from 2004 TG10, 2005 TF50, 2005 UR or 2015TX24's nominal solution and integrated until 10 000 BCE. The plot presents a compilation of the clones' a and e between 3000 BCE and 3500 BCE (in blue). The clones belonging to the 7:2 MMR are presented in yellow, while those approaching at least one clone of comet 2P/Encke with a low MOID and relative velocity are circled in red. \label{fig:clones_selected}}
\end{figure*}

\newpage
\textbf{F2  Fireballs}\\

\begin{figure*}[!ht]
  \centering
  \includegraphics[width=0.32\textwidth]{011115_174410_w2P_clones.png}
  \includegraphics[width=0.32\textwidth]{021115_235259_w2P_clones.png}
  \includegraphics[width=0.32\textwidth]{041115_021452_w2P_clones.png}\\
  \includegraphics[width=0.32\textwidth]{051115_213433_w2P_clones.png}
  \includegraphics[width=0.32\textwidth]{051115_220108_w2P_clones.png}
  \includegraphics[width=0.32\textwidth]{051115_234939_w2P_clones.png}\\
  \includegraphics[width=0.32\textwidth]{061115_040629_w2P_clones.png}
  \includegraphics[width=0.32\textwidth]{061115_164758_w2P_clones.png}
  \includegraphics[width=0.32\textwidth]{081115_181258_w2P_clones.png}
  \caption{Semi-major axis and eccentricity of the clones created for the fireballs of Figure 12 and integrated until 10 000 BCE. The plot presents a compilation of the clones' a and e between 3000 BCE and 3500 BCE (in blue). The clones belonging to the 7:2 MMR are presented in yellow, while those approaching at least one clone of comet 2P/Encke with a low MOID and relative velocity are circled in red.}
  \label{fig:clones_selected_fireballs1}
\end{figure*}


\begin{figure*}[!ht]
  \centering
  \includegraphics[width=0.32\textwidth]{011115_013625_w2P_clones.png}
  \includegraphics[width=0.32\textwidth]{021115_022525_w2P_clones.png}
  \includegraphics[width=0.32\textwidth]{171115_020907_w2P_clones.png}\\
  \includegraphics[width=0.32\textwidth]{251015_022301_w2P_clones.png}
  \includegraphics[width=0.32\textwidth]{311015_202117_w2P_clones.png}
  \includegraphics[width=0.32\textwidth]{311015_211904_w2P_clones.png}
  \includegraphics[width=0.32\textwidth]{241015_185031_w2P_clones.png}
  \caption{Semi-major axis and eccentricity of the clones created for the fireballs of Figure 13 and integrated until 10 000 BCE. The plot presents a compilation of the clones' a and e between 3000 BCE and 3500 BCE (in blue). The clones belonging to the 7:2 MMR are presented in yellow, while those approaching at least one clone of comet 2P/Encke with a low MOID and relative velocity are circled in red.}
  \label{fig:clones_selected_fireballs2}
\end{figure*}

\clearpage
\newpage
\textbf{APPENDIX G: SIMULATED METEOROIDS EJECTION FROM THE G5 BODIES}\\ \label{appendix:ejection} 

\setfignumprefix{G}
\begin{figure*}[!ht]
  \centering
  \includegraphics[width=.32\textwidth]{2P_ejection.png}
  \includegraphics[width=.32\textwidth]{2004TG10_ejection.png}
  \includegraphics[width=.32\textwidth]{2005TF50_ejection.png}\\
  \large 2P/Encke \hspace{3.5cm} 2004 TG10 \hspace{3cm} 2005 TF50\\[0.2cm]
  \includegraphics[width=.32\textwidth]{2005UR_ejection.png}
  \includegraphics[width=.32\textwidth]{2015TX24_ejection.png}\\
  \large 2005 UR \hspace{4cm} 2015 TX24 \hspace{2cm}\\
  \caption{Distribution of nodal locations for model meteoroids crossing the ecliptic plane in 2015 based on the three ejection scenarios for the G5 bodies' orbits circa 3200 BCE (see Section 6 for details). The nodes are color coded by the meteoroid ejection velocity appropriate to the possible three scenarios discussed in the text, namely with values of 2 m/s (in red), 1 to 125 m/s (in orange) and 1 km/s (in grey). The time periods where the annual Taurid meteor showers (NTA, STA, BTA and ZPE) are active are indicated by the colored arcs of circle close to the Earth's orbit (in blue). The solar longitude range where the Taurid resonant Swarm was observed in 2015 by \protect\cite{Spurny2017} is illustrated by the black circle arc.}
  \label{fig:Ejections_3200BCE}
\end{figure*}

\begin{figure*}[!ht]
  \centering
  \includegraphics[width=.95\textwidth]{radiants_3200BC.png}
  \caption{Right ascension and declination of the meteoroids' geocentric radiant as observed and simulated in 2015. The observed radiants of the 2015 fireballs reported by \protect\cite{Spurny2017} are represented by black symbols. The simulated radiants in 2015 corresponding to ejection from 2004 TG10, 2005 TF50, 2005 UR, 2015 TX24 and 2P around 3200 BCE are represented in orange, purple, green, blue and red respectively. Meteoroids ejected with low velocities (corresponding to scenario 2) are represented by circles, while ejecta with velocities of 1 km/s are represented by crosses. Only meteoroids that cross the ecliptic plane at a maximum distance of 0.05 AU from Earth's orbit were selected for radiant computation. See Section 6 for details.}
  \label{fig:radiants}
\end{figure*}

\clearpage
\newpage
\textbf{APPENDIX H: FUTURE POSSIBLE ASSOCIATIONS} \label{sec:future}\\


\setfignumprefix{H}
\begin{figure}[!ht]
  \centering
  \includegraphics[width=\textwidth]{Map_future.png}
  \caption{Possible rapprochement between TC candidates over the period 2000 CE to 23 000 CE. Abscissas and ordinates indicate the bodies integrated. Circles at position (body \#1, body \#2) indicate the percentage of clones of body \#1 approaching at least one clone of body \#2 with a MOID smaller than 0.05 AU and a relative velocity below 1 km/s. The size of the circle is proportional to the maximum percentage of clones retained for body \#1 (in abscissa) compared to body \#2 (in ordinate) over the integration period. The color of the circle indicates the epoch when this highest percentage of clones retained is attained.}
  \label{fig:map_future}
\end{figure}

\newpage
\bibliographystyle{mnras}
\bibliography{references2} % if your bibtex file is called example.bib


\end{document}

