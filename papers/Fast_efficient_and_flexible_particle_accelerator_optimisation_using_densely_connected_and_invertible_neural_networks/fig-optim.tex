\begin{figure}[ht!]
    \centering
    \begin{tikzpicture}[scale=0.6]
        %%%%%%%%%%%%%%%%%%%%%%
        % OPAL optimisation.
        %%%%%%%%%%%%%%%%%%%%%%
        \node[inner sep=0pt] (russell) at (-2.4, 10) {\Large\bf a};
        
        \def\arrowlength{1};
        \def\blocksep{2};
        \def\layersep{5}
        
        \begin{scope}[font=\small]
            % Initialisation: Selection.
            \node[very thick, align=center, draw, fill=gray!20] (initselect) at (1, 2) {Initialise. \\ $i = 0$};
            
            % Initialisation: Random.
            \node[minimum width=1cm, minimum height=1cm, draw, anchor=north, above=2 of initselect] (rect) {}; 
            \foreach \one/\two in {0.1cm/0.9cm, 0.35cm/0.45cm, 0.85cm/0.95cm, 0.8cm/0.5cm, 0.1cm/0.15cm, 0.4cm/0.7cm, 0.615cm/0.135cm} {
                \fill let \p1=(rect.south) in (1.5*\one + \x1 - 0.5cm, \y1 + 1.5*\two) circle (2pt);
            }
            \node[above=0.2 of rect.north] (init) {Random initialisation};
            \node[align=center, anchor=north, below=0.2 of rect.south] (initresult) {$x_{0,j} \sim \mathrm{Unif}$};
            
            \begin{scope}[on background layer]
                \node[very thick, fit={(rect) (init) (initresult)}, draw, fill=gray!20] (initbox) {};
            \end{scope}
            
            % Surrogate evaluation.
            \node[right=2*\arrowlength of initselect.east, anchor=west, yshift=0.8cm, align=center] (opaltext) {\opal\ calculates\\ beam properties};
            \node[right=2*\arrowlength of initselect.east, anchor=west, yshift=-1cm] (opalformula) {$y_{i,j} = f \left( x_{i,j} \right)$};
            
            %\node[draw, minimum width=3.3cm, minimum height=2.9cm, anchor=west, right=2*\arrowlength of initselect.east, yshift=0.35cm] (opalbox) {};
            \begin{scope}[on background layer]
                \node [very thick, fit={(opaltext) (opalformula)}, draw, fill=gray!20] (opalbox) {};
            \end{scope}
            
            %\node[fit={(opaltext) (opalformula)}, draw, yshift=-1.6cm] (opalbox) {};
            
            % Condition.
            \node[very thick, align=center, anchor=west, draw=outputcolor, shape=diamond, fill=outputcolor!20] (cond) at ($ (opalbox.east) + (2*\arrowlength, 0) $) {$i \leq n_g$?};
            
            % GA part.
            \node (ga) at ($ (cond.north) + (0, 5*\arrowlength) $) {GA advances population};
            \node [align=left, below=0.5 of ga] (gatext) {$\{ x_{i+1, j} \} = \mathrm{NSGA2}(\{ x_{i,j} \}, \{ y_{i,j} \})$};
            
            \begin{scope}[on background layer]
                \node[very thick, fit={(ga) (gatext)}, draw=outputcolor, fill=outputcolor!20] (gabox) {};
            \end{scope}
            
            % Pareto front.
            \node[very thick, align=left, anchor=west, draw=outputcolor, fill=outputcolor!20] (pareto) at ($ (cond.east) + (\arrowlength, 0) $) {Select Pareto front};
            
            % Result.
            \node[anchor=west, very thick, draw=outputcolor, fill=outputcolor!20] (result) at ($ (pareto.east) + (\arrowlength, 0) $) {$\{x\}^*$};
            
            % Arrows.
            \draw[myarrow, dashed] (initbox.south) -- (initselect.north);
            \draw[myarrow] (initselect.east) -- (opalbox.west);
            \draw[myarrow] (opalbox.east) -- (cond.west);
            \draw[myarrow]  (cond.north) -- (gabox.south) node[pos=0.35, anchor=west, xshift=2mm] {\textcolor{outputcolor}{yes}};
            \draw[myarrow] (gabox.west) -| (opalbox.north) node[pos=0.2, anchor=south east] {\textcolor{outputcolor}{$\mathbf{i \leftarrow i+1}$}};
            \draw[myarrow] (cond.east) -- (pareto.west) node[pos=0.3, anchor=south, yshift=2mm] {\textcolor{outputcolor}{no}} node[pos=1] (thirdarrow) {};
            \draw[myarrow] (pareto.east) -- (result.west) node (fourtharrow) {};
        \end{scope}
        
        %%%%%%%%%%%%%%%%%%%%%%%%%%%%%%%%%
        % Surrogate optimisation.
        %%%%%%%%%%%%%%%%%%%%%%%%%%%%%%%%%
        \node[inner sep=0pt] (russell) at (-2.4, -0.8) {\Large\bf b};
        
        \begin{scope}[font=\small, yshift=-11cm]
            \def\layersep{6.4};
            % Initialisation: Selection.
            \node[very thick, align=center, draw=hiddencolor, fill=hiddencolor!20] (initselect) at (1, 2) {Initialise. \\ $i = 0$};
            
            % Initialisation: Random.
            \node[minimum width=1cm, minimum height=1cm, draw, anchor=north, above=2 of initselect] (rect) {}; 
            \foreach \one/\two in {0.1cm/0.9cm, 0.35cm/0.45cm, 0.85cm/0.95cm, 0.8cm/0.5cm, 0.1cm/0.15cm, 0.4cm/0.7cm, 0.615cm/0.135cm} {
                \fill let \p1=(rect.south) in (1.5*\one + \x1 - 0.5cm, \y1 + 1.5*\two) circle (2pt);
            }
            \node[above=0.2 of rect.north] (init) {Random initialisation};
            \node[align=center, anchor=north, below=0.2 of rect.south] (initresult) {$x_{0,j} \sim \mathrm{Unif}$};
            
            \begin{scope}[on background layer]
                \node[very thick, fit={(rect) (init) (initresult)}, draw, fill=gray!20] (initbox) {};
            \end{scope}
            
            % Initialisation: Biased.
            \node[minimum width=1cm, minimum height=1cm, draw, anchor=north, above=-4.35 of initselect] (rect2) {}; 
            \foreach \one/\two in {0.2cm/1.6cm, 0.65cm/1.1cm, 0.4cm/1.6cm, 0.3cm/1cm, 0.45cm/1.3cm, 0.2cm/1.3cm, 0.6cm/1.47cm} {
                \fill let \p1=(rect2.south) in (\one, \y1 + \two) circle (2pt);
            }
            \node[align=center, above=0.2 of rect2.north] (init2) {Initialisation using\\\invertiblemodel};
            \node[align=center, anchor=north, below=0.2 of rect2.south] (initresult2) {$x_{0,j} = \hat{f}^{-1}(y_t)$};
            
            \begin{scope}[on background layer]
                \node[very thick, fit={(rect2) (init2) (initresult2)}, draw=hiddencolor, fill=hiddencolor!20] (initbox2) {};
            \end{scope}
            
            % Surrogate evaluation.
            \node[right=2*\arrowlength of initselect.east, anchor=west, yshift=1.6cm, align=center] (opaltext) {\forwardmodel \\ predicts beam};
            \node[below=2.5cm of opaltext] (opalformula) {$y_{i,j} = \tilde{f} \left( x_{i,j} \right)$};
            
            \begin{scope}[on background layer]
                \node[very thick, draw=hiddencolor, fill=hiddencolor!20, minimum width=3cm, minimum height=4.4cm, anchor=west, right=2*\arrowlength of initselect.east] (opalbox) {};
            \end{scope}
            
            % Draw picture of a neural network.
            \begin{scope}[xshift=6.25cm, yshift=0.1cm, scale=0.16, y=3cm]
                % Input layer.
                \begin{scope}[color=black]
                \node[draw, circle] (input1) at (0, 2) {};
                \node[draw, circle] (input2) at (0, 4) {};
                \node[draw, circle] (input3) at (0, 6) {};
                \end{scope}
                
                % Hidden layers.
                \begin{scope}[black]
                    \node[draw, circle] (hidden11) at (1*\layersep, 1) {};
                    \node[draw, circle] (hidden12) at (1*\layersep, 3) {};
                    \node[draw, circle] (hidden13) at (1*\layersep, 5) {};
                    \node[draw, circle] (hidden14) at (1*\layersep, 7) {};
                    
                    \node[draw, circle] (hidden21) at (2*\layersep, 1) {};
                    \node[draw, circle] (hidden22) at (2*\layersep, 3) {};
                    \node[draw, circle] (hidden23) at (2*\layersep, 5) {};
                    \node[draw, circle] (hidden24) at (2*\layersep, 7) {};
                    
                    \node[draw, circle] (hidden31) at (3*\layersep, 1) {};
                    \node[draw, circle] (hidden32) at (3*\layersep, 3) {};
                    \node[draw, circle] (hidden33) at (3*\layersep, 5) {};
                    \node[draw, circle] (hidden34) at (3*\layersep, 7) {};
                \end{scope}
                    
                % Output layer.
                \begin{scope}[color=black]
                    \node[draw, circle] (out1) at (4*\layersep, 0) {};
                    \node[draw, circle] (out2) at (4*\layersep, 2) {};
                    \node[draw, circle] (out3) at (4*\layersep, 4) {};
                    \node[draw, circle] (out4) at (4*\layersep, 6) {};
                    \node[draw, circle] (out5) at (4*\layersep, 8) {};
                \end{scope}
            
                % Connect the layers.
                \foreach \x in {(input1), (input2), (input3)} {
                    \foreach \y in {(hidden11), (hidden12), (hidden13), (hidden14)} {
                        \draw \x -- \y;
                    }
                }
                
                \foreach \x in {(hidden11), (hidden12), (hidden13), (hidden14)} {
                    \foreach \y in {(hidden21), (hidden22), (hidden23), (hidden24)} {
                        \draw \x -- \y;
                    }
                }
                
                \foreach \x in {(hidden21), (hidden22), (hidden23), (hidden24)} {
                    \foreach \y in {(hidden31), (hidden32), (hidden33), (hidden34)} {
                        \draw \x -- \y;
                    }
                }
                
                \foreach \x in {(hidden31), (hidden32), (hidden33), (hidden34)} {
                    \foreach \y in {(out1), (out2), (out3), (out4), (out5)} {
                        \draw \x -- \y;
                    }
                }
            \end{scope}
            
            %\node[fit={(opaltext) (opalformula)}, draw, yshift=-1.6cm] (opalbox) {};
            
            % Condition.
            \node[very thick, align=center, anchor=west, draw=outputcolor, fill=outputcolor!20, shape=diamond] (cond) at ($ (opalbox.east) + (2*\arrowlength, 0) $) {$i \leq n_g$?};
            
            % GA part.
            \node (ga) at ($ (cond.north) + (0, 5*\arrowlength) $) {GA advances population};
            \node [align=left, below=0.5 of ga] (gatext) {$\{ x_{i+1, j} \} = \mathrm{NSGA2}(\{ x_{i,j} \}, \{ y_{i,j} \})$};
            
            \begin{scope}[on background layer]
                \node[very thick, fit={(ga) (gatext)}, draw=outputcolor, fill=outputcolor!20] (gabox) {};
            \end{scope}
            
            % Pareto front.
            \node[very thick, align=left, anchor=west, draw=outputcolor, fill=outputcolor!20] (pareto) at ($ (cond.east) + (\arrowlength, 0) $) {Select Pareto front};
            
            % Result.
            \node[anchor=west, very thick, draw=outputcolor, fill=outputcolor!20] (result) at ($ (pareto.east) + (\arrowlength, 0) $) {$\{x\}^*$};
            
            % Arrows.
            \draw[myarrow, dashed] (initbox.south) -- (initselect.north);
            \draw[myarrow, dashed] (initbox2.north) -- (initselect.south);
            \draw[myarrow] (initselect.east) -- (opalbox.west);
            \draw[myarrow] (opalbox.east) -- (cond.west);
            \draw[myarrow]  (cond.north) -- (gabox.south) node[pos=0.35, anchor=west, xshift=2mm] {\textcolor{outputcolor}{yes}};
            \draw[myarrow] (gabox.west) -| (opalbox.north) node[pos=0.2, anchor=south east] {\textcolor{outputcolor}{$\mathbf{i \leftarrow i+1}$}};
            \draw[myarrow] (cond.east) -- (pareto.west) node[pos=0.3, anchor=south, yshift=2mm] {\textcolor{outputcolor}{no}} node[pos=1] (thirdarrow) {};
            \draw[myarrow] (pareto.east) -- (result.west) node (fourtharrow) {};
            
            % Explanation of variables.
            \node[anchor=north west, align=left] at ($ (opalbox.south east) + (0, -1cm) $) {$x_{i, j}$: $j$-th \dvar\ configuration in generation $i$\\$y_{i, j}$: $j$-th quantity of interest configuration in generation $i$};
        \end{scope}
        
        \node[inner sep=0pt] at (-2.4, -18) {\Large\bf c};
        \node[inner sep=0pt] at (9, -18) {\Large\bf d};
        
        \begin{scope}[yshift=-24.5cm]
            %%%%%%%%%%%%%%%%%%%%
            % Forward model
            %%%%%%%%%%%%%%%%%%%%
        
            \begin{scope}[scale=0.5, xshift=-2cm, font=\footnotesize]
                \def\layersep{3}
                
                \begin{scope}[ultra thick]
                    % Input layer.
                    \begin{scope}[color=inputcolor]
                    \node[draw, circle] (input1) at (0, 2) {};
                    \node[draw, circle] (input2) at (0, 4) {};
                    \node[draw, circle] (input3) at (0, 6) {};
                    \end{scope}
                    
                    % Hidden layers.
                    \begin{scope}[hiddencolor]
                        \node[draw, circle] (hidden11) at (1.5*\layersep, 1) {};
                        \node[draw, circle] (hidden12) at (1.5*\layersep, 3) {};
                        \node[draw, circle] (hidden13) at (1.5*\layersep, 5) {};
                        \node[draw, circle] (hidden14) at (1.5*\layersep, 7) {};
                        
                        \node[draw, circle] (hidden21) at (2.5*\layersep, 1) {};
                        \node[draw, circle] (hidden22) at (2.5*\layersep, 3) {};
                        \node[draw, circle] (hidden23) at (2.5*\layersep, 5) {};
                        \node[draw, circle] (hidden24) at (2.5*\layersep, 7) {};
                        
                        \node[draw, circle] (hidden31) at (3.5*\layersep, 1) {};
                        \node[draw, circle] (hidden32) at (3.5*\layersep, 3) {};
                        \node[draw, circle] (hidden33) at (3.5*\layersep, 5) {};
                        \node[draw, circle] (hidden34) at (3.5*\layersep, 7) {};
                    \end{scope}
                        
                    % Output layer.
                    \begin{scope}[color=outputcolor]
                        \node[draw, circle] (out1) at (5*\layersep, 0) {};
                        \node[draw, circle] (out2) at (5*\layersep, 2) {};
                        \node[draw, circle] (out3) at (5*\layersep, 4) {};
                        \node[draw, circle] (out4) at (5*\layersep, 6) {};
                        \node[draw, circle] (out5) at (5*\layersep, 8) {};
                    \end{scope}
                \end{scope}
                
                % Connect the layers.
                \foreach \x in {(input1), (input2), (input3)} {
                    \foreach \y in {(hidden11), (hidden12), (hidden13), (hidden14)} {
                        \draw \x -- \y;
                    }
                }
                
                \foreach \x in {(hidden11), (hidden12), (hidden13), (hidden14)} {
                    \foreach \y in {(hidden21), (hidden22), (hidden23), (hidden24)} {
                        \draw \x -- \y;
                    }
                }
                
                \foreach \x in {(hidden21), (hidden22), (hidden23), (hidden24)} {
                    \foreach \y in {(hidden31), (hidden32), (hidden33), (hidden34)} {
                        \draw \x -- \y;
                    }
                }
                
                \begin{scope}[even odd rule]
                    %\draw (3.9 * \layersep, 0.2) -- ++(0.65 * \layersep, 0) -- ++(0, 7.5) -- ++(-0.75 * \layersep, 0) -- cycle ++(2.8 * \layersep, 14) -- ++(1.7 * \layersep, 0) -- ++(0, 10) -- ++(-1.7 * \layersep, 0) -- ++(0, -10);
                    \clip (3.9 * \layersep, 0.2) -- ++(0.65 * \layersep, 0) -- ++(0, 7.5) -- ++(-0.75 * \layersep, 0) -- cycle ++(2.8 * \layersep, 14) -- ++(1.7 * \layersep, 0) -- ++(0, 10) -- ++(-1.7 * \layersep, 0) -- ++(0, -10);
                    \foreach \x in {(hidden31), (hidden32), (hidden33), (hidden34)} {
                        \foreach \y in {(out1), (out2), (out3), (out4), (out5)} {
                            \draw \x -- \y;
                        }
                    }
                \end{scope}
                
                
                % Annotate the layers.
                \filldraw[color=inputcolor] (-1.6, -1.5) rectangle +(3.3, 0.25) node[below, midway, black, yshift=-0.3cm] {\textcolor{inputcolor}{Input Layer}};
                \filldraw[color=hiddencolor] (\layersep, -1.5) rectangle +(3*\layersep, 0.25) node[below, midway, black, yshift=-0.3cm] {\textcolor{hiddencolor}{Hidden Layers}};
                \filldraw[color=outputcolor] (5*\layersep - 2, -1.5) rectangle +(4.0, 0.25) node[below, midway, black, yshift=-0.3cm] {\textcolor{outputcolor}{Output Layer}};
                
                % Draw variables.
                \draw[decorate, decoration={brace, amplitude=0.2cm, raise=0.3cm}, color=inputcolor, very thick] (input1.south) -- (input3.north) node[midway, xshift=-0.8cm, align=center, font=\linespread{0.8}\selectfont, color=inputcolor] {$\mathbf{x}$ \\ $s$};
                
                \draw[decorate, decoration={brace, amplitude=0.2cm, raise=0.3cm, mirror}, color=outputcolor, very thick] (out1.south) -- (out5.north) node[midway, xshift=0.8cm, align=center, color=outputcolor] {$\mathbf{y}$};
                
                % Annotate depth and width.
                \draw[color=hiddencolor, |-|, very thick] (1.5*\layersep, 8.8) -- +(2*\layersep, 0) node[midway, yshift=0.3cm] {\textbf{Depth}};
                \draw[color=hiddencolor, |-|, very thick] ([xshift=1.5cm]hidden31.south) -- ([xshift=1.5cm]hidden34.north) node[rotate=90, midway, yshift=-0.3cm] {\textbf{Width}};
                
                % Direction arrow.
                \draw [-{Triangle[width=4mm,angle'=45]}, line width=1mm, color=gray] (0, 12) -- (5*\layersep, 12);
            \end{scope}
            
            %%%%%%%%%%%%%%%%%%%%%
            % Invertible Model
            %%%%%%%%%%%%%%%%%%%%%
            
            \begin{scope}[scale=0.5, y=2cm, xshift=22cm, font=\footnotesize]
                \def\blockonepos{3}
                \def\blockwidth{5}
                \def\permwidth{2.5}
                \def\permpos{\blockonepos + \blockwidth + 3}
                \def\blocktwopos{\permpos + \permwidth + 3}
                \def\outpos{\blocktwopos + \blockwidth + 3}
                
                % Define the nodes, draw input & output layers.
                \begin{scope}[yshift=-1cm]
                    \foreach \i in {1, ..., 4} {
                        \node[draw, circle, ultra thick, color=inputcolor] (input\i) at (0, \i) {};
                        \node (block1in\i) at (\blockonepos, \i) {};
                        \node (block1out\i) at (\blockonepos + \blockwidth, \i) {};
                        \node (permin\i) at (\permpos, \i) {};
                        \node (permout\i) at (\permpos + \permwidth, \i) {};
                        \node (block2in\i) at (\blocktwopos, \i) {};
                        \node (block2out\i) at (\blocktwopos + \blockwidth, \i) {};
                        \node[draw, circle, ultra thick, color=outputcolor] (output\i) at (\outpos, \i) {};
                        
                        \draw[xshift=0, yshift=0] (input\i.east) -- (block1in\i.center);
                        \draw (block1out\i.center) -- (permin\i.center);
                        \draw (permout\i.center) -- (block2in\i.center);
                        \draw (block2out\i.center) -- (output\i.west);
                    }
                \end{scope}
                
                \begin{scope}[yshift=-1cm]
                    % First & second block.
                    \foreach \pos in {\blockonepos, \blocktwopos} {
                        \filldraw[ultra thick, draw=hiddencolor, fill=hiddencolor!20] (\pos, 0) rectangle ++(\blockwidth, 5);
                        \node[align=center, font=\footnotesize\linespread{0.8}\selectfont] at (\pos + 0.5 * \blockwidth, 0.5 * 5) {Affine \\ Coupling \\ Block};
                    }
                    % Permutation layer.
                    \draw[ultra thick] (\permpos, 0) rectangle ++(\permwidth, 5);
                \end{scope}
                
                % Draw permutations.
                \draw (permin1.center) -- (permout3.center);
                \draw (permin2.center) -- (permout1.center);
                \draw (permin3.center) -- (permout4.center);
                \draw (permin4.center) -- (permout2.center);
                
                % Annotate the layers.
                \node at (-1, -1.65) {\textcolor{inputcolor}{Settings}};
                \node at (\blockonepos + 0.5 * \blockwidth, -1.57) {\textcolor{hiddencolor}{Block 1}};
                \node[align=center] at (\permpos + 0.5 * \permwidth, -1.65) {Permutation\\ Layer};
                \node at (\blocktwopos + 0.5 * \blockwidth, -1.57) {\textcolor{hiddencolor}{Block 2}};
                \node at (\outpos + 1, -1.57) {\textcolor{outputcolor}{Labels}};
                
                % Annotate input/output layers.
                \draw[decorate, decoration={brace, amplitude=0.2cm, raise=0.3cm}, color=inputcolor, very thick] (input1.south) -- (input4.north) node[midway, xshift=-0.85cm, align=left, font=\linespread{0.8}\selectfont, color=inputcolor] {$\mathbf{x}$ \\ $s$ \\ $\mathbf{x_\mathrm{pad}}$};
                \draw[decorate, decoration={brace, amplitude=0.2cm, raise=0.3cm, mirror}, color=outputcolor, very thick] (output1.south) -- (output4.north) node[midway, xshift=1cm, align=left, font=\linespread{0.8}\selectfont, color=outputcolor] {$\mathbf{y}$ \\ $s$ \\ $\mathbf{z}$ \\ $\mathbf{y_\mathrm{pad}}$};
                
                % Direction arrow.
                \draw [{Triangle[width=4mm,angle'=45]}-{Triangle[width=4mm,angle'=45]}, line width=1mm, color=gray] (0, 6) -- (\outpos, 6);
            \end{scope}
        \end{scope}
    \end{tikzpicture}
    
    \caption{Optimisations with \opal\ (\textbf{a}) and the \forwardmodel\ (\textbf{b}). The gray parts correspond to current practice, the blue parts belong to the \ga\ (GA) NSGA-II, and the orange parts are our novel contributions. The GA generates $n_g$ generations.  Schematics of the \forwardmodel\ (\textbf{c}) and \invertiblemodel\ (\textbf{d}) architectures.
    %The depth and width of the hidden layers, and the number and architecture of the hidden coupling blocks is tuned to the modelled particle accelerator.
    The grey arrows denote the prediction directions.}
    \label{fig:optimisations}
\end{figure}