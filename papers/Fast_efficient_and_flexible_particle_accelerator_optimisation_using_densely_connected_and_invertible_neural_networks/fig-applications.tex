\begin{figure}
\resizebox{\textwidth}{!}{
\begin{tikzpicture}[scale=3.5]
%
% IsoDAR
%
\begin{scope}[xscale=1,yscale=1,xshift=0,yshift=5] % or whatever you want
\draw[thick,rotate=90] plot file {newData.dat} ; 

\fill[cyan, opacity=0.15] (0, 0) -- ({1.5*cos(110)}, {1.5*sin(110)}) arc (110:70:1.5);
\fill[cyan, opacity=0.15] (0, 0) -- ({1.5*cos(20)}, {1.5*sin(20)}) arc (20:-20:1.5);
\fill[cyan, opacity=0.15] (0, 0) -- ({1.5*cos(-110)}, {1.5*sin(-70)}) arc (-110:-70:1.5);
\fill[cyan, opacity=0.15] (0, 0) -- ({-1.5*cos(20)}, {-1.5*sin(-20)}) arc (160:200:1.5);
\fill[orange, opacity=0.15] (0, 0) -- ({1.3*cos(60)}, {1.3*sin(60)}) arc (60:30:1.3);
\fill[orange, opacity=0.15] (0, 0) -- ({1.3*cos(-30)}, {1.3*sin(-30)}) arc (-30:-60:1.3);
\fill[orange, opacity=0.15] (0, 0) -- ({1.3*cos(210)}, {1.3*sin(210)}) arc (210:240:1.3);
\fill[orange, opacity=0.15] (0, 0) -- ({1.3*cos(150)}, {1.3*sin(150)}) arc (150:120:1.3);

% Yoke.
\filldraw[even odd rule,inner color=red,outer color=white] 
(0,0) circle (1.5)
(0,0) circle (1.7);
\node at (0, 1.6) {\large Yoke};

% Annotations.
\node [orange] (rf) at (-1.2, 0.8) {\large RF Cavity};
%\draw [->, orange, thick] (rf) -- (-0.8, 0.8);

\node [cyan] (hill) at (-1.65, 0) {\large Sector Magnet};
%\draw [->, cyan, thick] (hill) -- (-1.3, 0.3);

% Dots
\fill[OrangeRed] (0.08, -0.07) circle (0.03);
\fill[outputcolor] (0.7, -0.9) circle (0.03);

% QOIs
\node[align=left, anchor=west] (output) at (1.6, 1) {\Large \textbf{Quantities of interest}\\
\textcolor{outputcolor}{$\mathbf{E, \Delta E}$}\\ 
\textcolor{outputcolor}{$\mathbf{\sigma_{x, y, z}}$}\\ 
\textcolor{outputcolor}{$\mathbf{\epsilon_{x, y, z}}$}\\ 
\textcolor{outputcolor}{$\mathbf{h_{x, y, z}}$}\\
\textcolor{outputcolor}{$\mathbf{N_l}$}};

\draw [decorate,decoration={brace,amplitude=5pt},xshift=-4pt,outputcolor,thick]
(1.72, 0.62) -- (1.72,1.225) node [OrangeRed,midway] {};

\draw [->, >=stealth, outputcolor, thick] (1.5, 0.92) -- (0.7, -0.9); % (0.88, 0.71);

% DVARs
\node[align=left, below=9.4cm of output.west, anchor=west] 
{\Large \textbf{Design variables}\\ 
\textcolor{OrangeRed}{$\mathbf{p_{r0}}$}\\ 
\textcolor{OrangeRed}{$\mathbf{r_0}$}\\
\textcolor{OrangeRed}{$\mathbf{\sigma_{x, y, z}}$}\\
\textcolor{OrangeRed}{$\mathbf{\phi_{rf}}$}};

\draw [decorate,decoration={brace,amplitude=5pt},xshift=-4pt,OrangeRed,thick]
(1.72,-2) -- (1.72,-1.55) node [OrangeRed,midway] {};

\draw [->, >=stealth, OrangeRed, thick] (1.5, -1.77) -- (0.08, -0.07); %(0.44, 0.32);

% Add the crosses.
\foreach \position in {(0, -0.25),(-0.08, -0.3), (-0.3, -0.03), (-0.27, 0.07), (-0.18, 0.08), (0.3, 0.29), (0.4, 3.1), (0.11, 0.22), (0.13, 0.02), (0.18, -0.02), (0.3, 0.02)}
{
	\draw \position node[cross=2pt, very thick, green] {};
}
\end{scope}


\begin{scope}[xscale=0.3,yscale=1,xshift=300,yshift=5] % or whatever you want
% Draw the gun.
\node[inner sep=0pt, xshift=1.7cm] (gaussians) {\includegraphics[width=2cm]{img/2D_gaussians_no_background.png}};
\draw[very thick, latex-latex] (0.9cm, 0.275cm) -- +(0.7cm, 0cm) node[color=matplotlibblue,anchor=south, pos=0.5]{$\mathbf{\lambda}$};
\draw[very thick, latex-latex] (0.6cm, 0.22cm) -- +(0cm, -0.44cm) node[color=matplotlibblue,anchor=east, pos=0.5] {SIGXY};
%\node[single arrow, draw, single arrow tip angle=120, single arrow head extend=0.2cm, minimum size=1cm] (arrow) at (0.7cm, 0.5cm) {};
%\node[above=0.15cm of arrow.north, align=center] {Bunched $e^-$ beam\\of charge $C$};

% \newcommand{\solenoidbox}{
%     \begin{scope}[thick, draw=matplotlibblue, fill=matplotlibblue!20]
%         \filldraw (0, 0) rectangle +(1cm, \solenoidheight);
%     \end{scope}
% }
% %https://www.overleaf.com/project/5ef0952f08397d0001104358
% % Argument: x-position
% \newcommand{\solenoid}[1]{
%     \begin{scope}[xshift=#1]
%         \begin{scope}[yshift=\solenoidshift]
%             \solenoidbox
%         \end{scope}
%         \begin{scope}[yshift=-\solenoidshift - \solenoidheight]
%             \solenoidbox
%         \end{scope}
%     \end{scope}
% }
\filldraw[draw=matplotlibblue, fill=matplotlibblue!20] (2cm, \solenoidshift) rectangle +(0.5cm, 0.5*\solenoidheight);
\filldraw[draw=matplotlibblue, fill=matplotlibblue!20] (2cm, -\solenoidshift - 0.5*\solenoidheight) rectangle +(0.5cm, 0.5*\solenoidheight);

\filldraw[draw=matplotlibblue, fill=matplotlibblue!20] (2cm, \solenoidshift + 0.6*\solenoidheight) rectangle +(0.5cm, 0.5*\solenoidheight);
\filldraw[draw=matplotlibblue, fill=matplotlibblue!20] (2cm, -\solenoidshift - 0.5*\solenoidheight - 0.6*\solenoidheight) rectangle +(0.5cm, 0.5*\solenoidheight);
\node[color=matplotlibblue, xshift=-0.3cm] at (1.8cm, -0.27cm) {Q};
\node[color=matplotlibblue, xshift=-0.3cm] at (1.8cm, -0.67cm) {IM};
\node[color=matplotlibblue, xshift=-0.3cm] at (1.8cm, -0.47cm) {IBF};
\node[color=matplotlibblue, xshift=-0.3cm] at (1.8cm, -0.87cm) {$\phi$};
% \node[shape=isosceles triangle, draw, inner sep=0pt,
%   text width=\gunwidth, minimum size=\gunwidth, xshift=0.17cm, very thick] (gun) at (0, 0) {};
\draw[line width=1mm] (0.5cm, -1cm) -- + (2cm, 0cm) node[anchor=north, pos=0.5] {Gun};

%\draw [-{Triangle[width=4mm,angle'=45]}, line width=1mm] (0.5cm, -1.4cm) -- +(11.6cm, 0cm) node[anchor=north, pos=0.5] {Bunched $e^-$ beam of charge $C$};

% Draw linacs.
\foreach \x\l in {\Lone/{Cavity\\1}, \Ltwo/{Cavity\\2}, \Lthree/{Cavity\\3}, \Lfive/{Cavity\\4}} {
    \cavity{\x}{0cm}{\l}
}

% Draw solenoids.
\foreach \x\l in {\LSone/$\mathrm{ILS}_1$, \LStwo/$\mathrm{ILS}_2$, \LSthree/$\mathrm{ILS}_3$} {
    \solenoid{\x}
    \node[yshift=0cm, xshift=2\lslength-0.2cm, align=center, color=matplotlibblue] at (\x + \lslength/2, -0.87cm) {\l};
}

% Draw 0m marker.
\draw[dotted] (2.8cm, -0.8cm) -- +(0, 1.6cm) node[yshift=0.25cm] {0m};

% Draw 10m marker.
%\draw[dotted] (10.9cm, -0.8cm) -- +(0, 1.6cm) node[yshift=0.25cm] {10m};

% Draw the end of the simulated machine.
\draw[dotted] (12cm, -0.8cm) -- +(0, 1.6cm) node[yshift=0.25cm] {26m};

\begin{scope}[very thick]
    \draw (\gunwidth + 2.5cm, 0) -- (\Lone, 0);
    \draw ( {\Lone + \cavitylength - 0.045cm}, 0) -- (\Ltwo, 0);
    \draw (\Ltwo + \cavitylength - 0.045cm, 0) -- (\Lthree, 0);
    \draw (\Lthree + \cavitylength - 0.045cm, 0) -- (\Lfive, 0);
    \draw (\Lfive + \cavitylength - 0.045cm, 0) -- (11.1cm, 0) edge[dotted] (11.8cm, 0);
    \draw (11.8cm, 0) -- (12cm, 0);
\end{scope}
\draw[line width=1mm] (11.9cm, -1cm) -- + (0.1cm, 0cm) node[anchor=north, pos=0.5] {Experiment};

\end{scope}

% place drawings 

\node[inner sep=0pt] (russell) at (0,3)
    {\includegraphics[width=12cm, height=7.8cm]{fig1_isodar_at_kamland}};

\node[inner sep=0pt] (russell) at (5,3)
    {\includegraphics[width=12cm, height=7.8cm]{awa-pic-1}};


\node[inner sep=0pt] (russell) at (-1.8,4) {\Large\bf a};
\node[inner sep=0pt] (russell) at (-1.8,1.2) {\Large\bf b};
\node[inner sep=0pt] (russell) at (3.2,4) {\Large\bf c};
\node[inner sep=0pt] (russell) at (3.2,1.2) {\Large\bf d};
\end{tikzpicture}
}
\caption{Overview of the use-cases. \textbf{a} Artistic representation of the \isodar\ experiment at KamLAND (Kamioka Observatory, Japan), from left to right the cyclotron, target and detector are depicted. \textbf{b} A schematic of the \isodar\ cyclotron with relevant parameter. \textbf{c} The AWA facility at Argonne National Laboratory (US), pictured from the gun downstream. \textbf{d} Schematic of the relevant parts of the AWA machine.}
\label{fig:overview}
\end{figure}