\pdfoutput=1

\documentclass[cernpreprint,texlive=2016,txfonts,UKenglish]{latex/atlasdoc}

\usepackage[biblatex=true]{latex/atlaspackage}
\usepackage{latex/atlasphysics}

\newcommand{\AtlasCoordFootnote}{
ATLAS uses a right-handed coordinate system with its origin at the
nominal interaction point (IP) in the centre of the detector and the
$z$-axis along the beam pipe. The $x$-axis points from the IP to the
centre of the LHC ring, and the $y$-axis points upwards. Cylindrical
coordinates $(r,\phi)$ are used in the transverse plane, $\phi$ being
the azimuthal angle around the $z$-axis. The pseudorapidity is defined
in terms of the polar angle $\theta$ as $\eta=-\ln\tan(\theta/2)$. The
angular distance is measured in units of 
$\Delta R\equiv\sqrt{(\Delta\eta)^{2}+(\Delta\phi)^{2}}$. The rapidity
is  defined as $y=0.5\ln[(E+p_z)/(E-p_z)]$, where $E$ is the energy
and $p_z$ is the $z$-component of the momentum, and transverse energy
is defined as $E_{\mathrm T}=E\sin\theta$.}

\graphicspath{{logos/}}

\usepackage{latex/atlasbiblatex}

\usepackage{epsfig}

\addbibresource{photonjet2015_paper.bib}

\def\ptjetl{p_{\mathrm{T}}^{\text{jet-lead}}}
\def\etg{E_{\mathrm{T}}^{\gamma}}
\def\rapjet{y^{\mathrm{jet}}}
\def\etag{\eta^{\gamma}}
\def\mgj{m^{\gamma-{\mathrm{jet}}}}
\def\ctgj{\cos\theta^*}
\def\dr{\Delta R^{\gamma-{\mathrm{jet}}}}
\def\deltaphigj{\Delta\phi^{\gamma-{\mathrm{jet}}}}
\def\pt{p_{\mathrm{T}}}
\def\pb1{pb$^{-1}$}
\def\fb1{fb$^{-1}$}
\def\qq{q\bar q}
\def\dgen{\frac{{\mathrm{d}}\sigma}{{\mathrm{d}}O}}
\def\dgenn{{\mathrm{d}}\sigma/{\mathrm{d}}O}
\def\dsetg{{\mathrm{d}}\sigma/{\mathrm{d}}\etg}
\def\dspt{{\mathrm{d}}\sigma/{\mathrm{d}}\ptjetl}
\def\dsphi{{\mathrm{d}}\sigma/{\mathrm{d}}\deltaphigj}
\def\dsmgj{{\mathrm{d}}\sigma/{\mathrm{d}}\mgj}
\def\dsctgj{{\mathrm{d}}\sigma/{\mathrm{d}}|\ctgj|}
\def\rbg{R^{\mathrm{bg}}}
\def\ppgj{pp\rightarrow\gamma + {\mathrm{jet}} + X}
\def\colab#1{#1 Collaboration}
\def\mz{m_Z}
\def\as{\alpha_{\mathrm{s}}}
\def\asz{\alpha_{\mathrm{s}}(\mz)}
\def\muR{\mu_{\mathrm{R}}}
\def\muF{\mu_{\mathrm{F}}}
\def\muf{\mu_{\mathrm{f}}}
\def\etal{et al.}
\def\etiso{E_{\mathrm{T}}^{\mathrm{iso}}}
\def\etisoc{E_{\mathrm{T,cut}}^{\mathrm{iso}}}
\def\etisocut{$4.2\cdot 10^{-3}\cdot \etg +4.8\ {\mathrm{\GeV}}$}
\def\etisocutp{$4.2\cdot 10^{-3}\cdot \etg +10\ {\mathrm{\GeV}}$}
\def\ptjet{p_{\mathrm{T}}^{\text{jet}}}
\def\rapjetl{y^{\text{jet-lead}}}
\def\sher{{\textsc{Sherpa}}}
\def\pyt{{\textsc{Pythia}}}
\def\jetp{{\textsc{Jetphox}}}
\mathchardef\mhyphen="2D

\AtlasTitle{Measurement of the cross section for isolated-photon plus
jet production in {\boldmath $pp$} collisions at {\boldmath $\sqrt
s=13$}~\TeV\ using the ATLAS detector}

\author{The ATLAS Collaboration}

\AtlasJournalRef{Phys. Lett. B 780 (2018) 578}
\AtlasDOI{DOI:10.1016/j.physletb.2018.03.035}
\PreprintIdNumber{CERN-EP-2017-265} 

\AtlasAbstract{
The dynamics of isolated-photon production in association with a jet
in proton--proton collisions at a centre-of-mass energy of $13$~\TeV\
are studied with the ATLAS detector at the LHC using a dataset with an
integrated luminosity of $3.2$~fb$^{-1}$. Photons are required to have
transverse energies above $125$~\GeV. Jets are identified using the
anti-$k_t$ algorithm with radius parameter $R=0.4$ and required to
have transverse momenta above $100$~\GeV.  Measurements of
isolated-photon plus jet cross sections are presented as functions of
the leading-photon transverse energy, the leading-jet transverse
momentum, the azimuthal angular separation between the photon and the
jet, the photon--jet invariant mass and the scattering angle in the
photon--jet centre-of-mass system. Tree-level plus parton-shower
predictions from {\textsc{Sherpa}} and {\textsc{Pythia}} as well as
next-to-leading-order QCD predictions from {\textsc{Jetphox}} and
{\textsc{Sherpa}} are compared to the measurements.
}


\begin{document}

\maketitle

\section{Introduction}
\label{intro}

The production of prompt photons in association with at least one jet
in proton--proton ($pp$) collisions provides a testing ground for
perturbative QCD (pQCD). In $pp$ collisions, all photons that are not
secondaries from hadron decays are considered to be ``prompt''. The
measurements of angular correlations between the photon and the jet
can be used to probe the dynamics of the hard-scattering process. The
dominant source in $pp$ collisions at the LHC is the $qg\rightarrow
q\gamma$ process. These measurements are also useful for tuning Monte
Carlo (MC) models and testing $t$-channel quark
exchange~\cite{np:b875:483,np:b918:257}. Furthermore, precise
measurements of these processes validate the generators used for
background studies in searches for physics beyond the Standard Model
which involve photons, such as the search for new phenomena in final
states with a photon and a  jet~\cite{jhep:1603:041,1709.10440}.

The production of $\ppgj$ events proceeds via two processes: direct,
in which the photon originates from the hard process, and
fragmentation, in which the photon arises from the fragmentation of a
coloured high transverse momentum\footnote{\AtlasCoordFootnote}
($\pt$) parton~\cite{pr:d76:034003,pr:d79:114024}. The direct and
fragmentation contributions are only well defined at leading order
(LO) in QCD; at higher orders this distinction is no longer
possible. These two processes exhibit distinct behaviours in the
observables considered here. Precise measurements test the interplay
of direct and fragmentation processes.

Measurements of prompt-photon production in a final state with
accompanying hadrons necessitate an isolation requirement on the
photon to avoid the large contribution from neutral-hadron decays into
photons. The production of isolated photons in association with jets
in $pp$ collisions at $\sqrt s=7$ and $8$~\TeV\ was studied by the
ATLAS~\cite{pr:d85:092014,np:b875:483,np:b918:257} and
CMS~\cite{pr:d88:112009,jhep:1406:009,jhep:1510:128}
Collaborations. The increase in the centre-of-mass energy of $pp$
collisions at the LHC to $13$~\TeV\ allows the exploration of the
dynamics of photon+jet production in a new regime with the goal of
testing the pQCD predictions at higher energy transfers than achieved
before. It is also possible to investigate whether the data in the new
energy regime are well described by the predictions of parton-shower
event generators, such as \sher~\cite{jhep:0902:007} and
\pyt~\cite{cpc:178:852}.

The dynamics of the underlying processes in $2\rightarrow 2$ hard
collinear scattering can be investigated using the variable
$\theta^*$, where $\ctgj\equiv\tanh(\Delta y/2)$ and $\Delta y$ is the
difference between the rapidities of the two final-state
particles. The variable $\theta^*$ coincides with the scattering polar
angle in the centre-of-mass frame for collinear scattering of massless
particles, and its distribution is sensitive to the spin of the
exchanged particle. For processes dominated by $t$-channel gluon
exchange, such as dijet production in $pp$ collisions, the
differential cross section behaves as $(1-|\ctgj|)^{-2}$ when
$|\ctgj|\rightarrow 1$. In contrast, processes dominated by
$t$-channel quark exchange are expected to exhibit a
$(1-|\ctgj|)^{-1}$ behaviour when $|\ctgj|\rightarrow 1$. This
fundamental prediction of QCD can be tested in photon plus jet
production in $pp$ collisions. The direct-photon contribution,
dominated by $t$-channel quark exchange, is expected to exhibit a
$(1-|\ctgj|)^{-1}$ dependence when $|\ctgj|\rightarrow 1$, whereas
that of fragmentation processes is predicted to be the same as in
dijet production, namely $(1-|\ctgj|)^{-2}$. For both processes, there
are also $s$-channel contributions which are, however, non-singular
when $|\ctgj|\rightarrow 1$. As a result, a measurement of the cross
section for prompt-photon plus jet production as a function of
$|\ctgj|$ provides a handle on the relative contributions of the
direct and fragmentation components as well as the possibility of
testing the dominance of $t$-channel quark exchange.

The results presented here include a study of the kinematics of the
photon plus one-jet system via measurements of the cross sections as
functions of the leading-photon transverse energy ($\etg$) and the
leading-jet transverse momentum ($\ptjetl$). The dynamics of the
photon plus one-jet system are studied by measuring the azimuthal
angular separation between the leading photon and the leading jet
($\deltaphigj$), the invariant mass of the leading photon and the
leading jet ($\mgj$) and $\ctgj$. The distribution in $\mgj$ is
predicted to be monotonically decreasing in QCD due to the absence of
resonances that decay into a photon and a jet. The analysis is
performed using $3.2$~\fb1\ of $\sqrt s=13$~\TeV\ $pp$ collision data
recorded by ATLAS. Next-to-leading-order (NLO) QCD predictions from
\jetp~\cite{jhep:0205:028,pr:d73:094007} and \sher\ as well as the
tree-level predictions of \pyt\ and \sher\ are compared to the
measurements.

\section{ATLAS detector}
\label{detector}
The ATLAS detector~\cite{PERF-2007-01} is a multi-purpose detector
with a forward-backward symmetric cylindrical geometry. It consists of
an inner tracking detector surrounded by a thin superconducting
solenoid, electromagnetic and hadronic calorimeters, and a muon
spectrometer incorporating three large superconducting toroid
magnets. The inner-detector system is immersed in a $2$~T axial
magnetic field and provides charged-particle tracking in the range
$|\eta|<2.5$. The high-granularity silicon pixel detector is closest
to the interaction region and provides four measurements per track;
the innermost layer, known as the insertable
B-layer~\cite{ATLAS-TDR-19}, provides high-resolution hits at small
radius to improve the tracking performance. The pixel detector is
followed by the silicon microstrip tracker, which typically provides
four three-dimensional space point measurements per track. These
silicon detectors are complemented by the transition radiation
tracker, which enables radially extended track reconstruction up to
$|\eta|=2.0$. The calorimeter system covers the range
$|\eta|<4.9$. Within the region $|\eta|<3.2$, electromagnetic (EM)
calorimetry is provided by barrel and endcap high-granularity
lead/liquid-argon (LAr) EM calorimeters, with an additional thin LAr
presampler covering $|\eta|<1.8$ to correct for energy loss in
material upstream of the calorimeters; for $|\eta|<2.5$ the EM
calorimeter is divided into three layers in depth. Hadronic
calorimetry is provided by a steel/scintillator-tile calorimeter,
segmented into three barrel structures within $|\eta|<1.7$, and two
copper/LAr hadronic endcap calorimeters, which cover the region
$1.5<|\eta|<3.2$. The solid-angle coverage is completed out to
$|\eta|=4.9$ with forward copper/LAr and tungsten/LAr calorimeter
modules, which are optimised for EM and hadronic measurements,
respectively. Events are selected using a first-level trigger
implemented in custom electronics, which reduces the maximum event
rate of 40~MHz to a design value of 100~kHz using a subset of detector
information. Software algorithms with access to the full detector
information are then used in the high-level trigger to yield a
recorded event rate of about 1~kHz~\cite{epj:c77:317}.

\section{Data sample and Monte Carlo simulations}
\label{datmc}
The data used in this analysis were collected with the ATLAS detector
during the $pp$ collision running period of 2015, when the LHC
operated with a bunch spacing of $25$~ns and at a centre-of-mass
energy of $13$~\TeV. Only events taken during stable beam conditions
and satisfying detector and data-quality requirements, which include
the calorimeters and inner tracking detectors being in nominal
operation, are considered. The average number of $pp$ interactions per
bunch crossing in the dataset is $13$.  The total integrated
luminosity of the collected sample is {\mbox{$3.16 \pm 0.07$~\fb1}}.
The uncertainty in the integrated luminosity is $2.1\%$ and is
derived, following a methodology similar to that detailed in
Ref.~\cite{epj:c76:653}, from a calibration of the luminosity scale
using $x$--$y$ beam-separation scans performed in August 2015.

Samples of MC events were generated to study the characteristics of
signal events. The MC programs \pyt~8.186 and \sher~2.1.1 were used to
generate the simulated events. In both generators, the partonic
processes were simulated using tree-level matrix elements, with the
inclusion of initial- and final-state parton showers. Fragmentation
into hadrons was performed using the Lund string
model~\cite{prep:97:31} in the case of  \pyt, and in \sher\ by a
modified version of the cluster model~\cite{epj:c36:381}. The LO
NNPDF2.3~\cite{np:b867:244} parton distribution functions (PDF) set
was  used for \pyt\ while the NLO CT10~\cite{pr:d82:074024} PDF set
was used for \sher\ to parameterise the proton structure. Both samples
include a simulation of the underlying event (UE). The event-generator
parameters were set according to the ATLAS  2014 tune series (A14
tune) for \pyt~\cite{ATL-PHYS-PUB-2014-021} and to the tune developed
in conjunction with the NLO CT10 PDF set for \sher. The \pyt\
simulation of the signal includes LO photon-plus-jet events from both
direct processes (the hard subprocesses  $qg\rightarrow q\gamma$ and
$\qq\rightarrow g\gamma$, called the ``hard'' component) and photon
bremsstrahlung in LO QCD dijet events (called the ``bremsstrahlung''
component). The bremsstrahlung component is modelled by final-state
QED radiation arising from calculations of all $2\rightarrow 2$ QCD
processes. In the particle-level phase space of the presented
measurements the fraction of the bremsstrahlung component decreases
from $35\%$ at $\etg=125$~\GeV\ to $15\%$ at $\etg=1$~\TeV. The \sher\
samples were generated with LO matrix elements for photon-plus-jet
final states with up to three additional partons ($2\rightarrow n$
processes with $n$ from $2$ to $5$); the matrix elements were merged
with the \sher\ parton shower using the ME+PS@LO
prescription~\cite{jhep:0905:053}. The bremsstrahlung component is
accounted for in \sher\ through the matrix elements of $2\rightarrow
n$ processes with $n\geq 3$. In the generation of the \sher\ samples,
a requirement on the photon isolation at the matrix-element level was
imposed using the criterion defined in Ref.~\cite{pl:b429:369}. This
criterion, commonly called Frixione's criterion, requires the total
transverse energy inside a cone of size ${\cal V}$ around the
generated final-state photon, excluding the photon itself, to be below
a certain threshold,
$E_{\mathrm{T}}^{\mathrm{max}}({\cal V})=\epsilon\etg ((1-\cos{{\cal V}})/(1-\cos{{\cal R}}))^{n}$,
for all ${\cal V} < {\cal R}$. The parameters for the threshold were
chosen to be ${\cal R}=0.3$, $n=2$ and $\epsilon=0.025$. The \sher\
predictions from this computation are referred to as LO \sher. 

All the samples of generated events were passed through the
{\textsc{Geant4}}-based~\cite{nim:a506:250} ATLAS detector and trigger
full simulation programs~\cite{epj:c70:823}. They are reconstructed
and analysed with the same program chain as the data. Pile-up from
additional $pp$ collisions  in the same and neighbouring bunch
crossings was simulated by overlaying each MC event with a variable
number of simulated inelastic $pp$ collisions generated using
\pyt~8.153 with the ATLAS set of tuned parameters for minimum bias
events (A2 tune)~\cite{ATL-PHYS-PUB-2012-003}. The MC events are
weighted to reproduce the distribution of the average number of
interactions per bunch crossing ($\langle\mu\rangle$) observed in the
data, referred to as ``pile-up reweighting''. In this procedure, the
$\langle\mu\rangle$ value from the data is divided by a factor of
$1.16\pm 0.07$, a rescaling which makes the number  of reconstructed
primary vertices agree better between data and simulation and
reproduces the visible cross section of inelastic $pp$ collisions as
measured in the data~\cite{nc:2:463}.

\section{Event selection}
\label{evesel}
Events were recorded using a single-photon trigger, with a transverse
energy threshold of $120$~\GeV\ and ``loose'' identification
requirements based on the shower shapes in the second layer of the EM
calorimeter as well as on  the energy leaking into the hadronic
calorimeter from the EM calorimeter~\cite{epj:c77:317}. Events are
required to have a reconstructed primary vertex. If multiple primary
vertices are reconstructed, the one with the highest sum of the
$\pt^2$ of the associated tracks is selected as the primary vertex.

\nopagebreak
Photon candidates are reconstructed from clusters of energy deposited
in the EM calorimeter and  classified~\cite{epj:c76:666} as
unconverted photons (candidates without a matching track or matching
reconstructed conversion vertex in the inner detector) or converted
photons (candidates with a matching reconstructed conversion vertex or
a matching track consistent with originating from a photon
conversion). The measurement of the photon energy is based on the
energy collected in calorimeter cells in an area of size
$\Delta\eta\times\Delta\phi=0.075\times 0.175$ in the barrel and
$0.125\times 0.125$ in the endcaps. A dedicated energy
calibration~\cite{epj:c74:3071} is then applied to the candidates to
account for upstream energy loss and both lateral and longitudinal
leakage. The photon identification is based primarily on shower shapes
in the calorimeter~\cite{epj:c76:666}. An initial selection is derived
using the information from the hadronic calorimeter and the lateral
shower shape in the second layer of the EM calorimeter, where most of
the photon energy is contained. The final tight selection applies
stringent criteria~\cite{epj:c76:666} to the same variables used in
the initial selection, separately for converted and unconverted photon
candidates. It also places requirements on the shower shape in the
finely segmented first calorimeter layer to ensure the compatibility
of the measured shower profile with that originating from a single
photon impacting the calorimeter. When applying the photon
identification criteria to simulated events, corrections are made for
small differences in the average values of the shower-shape variables
between data and simulation. Events with at least one photon candidate
with calibrated $\etg>125$~\GeV, where the trigger is maximally
efficient, and $|\etag|<2.37$ are selected. Candidates in the region
$1.37<|\etag|<1.56$, which includes the transition region between the
barrel and endcap calorimeters, are not considered. The photon
candidate is required to be isolated based on the amount of transverse
energy inside a cone of size $\Delta R=0.4$ in the $\eta$--$\phi$
plane around the photon candidate, excluding an area of size
$\Delta\eta \times \Delta\phi = 0.125 \times 0.175$ centred on the
photon. The isolation transverse energy is computed from topological
clusters of calorimeter cells~\cite{epj:c77:490} and is denoted by
$\etiso$. Topological clusters are built from neighbouring calorimeter
cells containing energy significantly above a noise threshold that is
estimated from measurements of calorimeter electronic noise and
simulated pile-up noise. The measured value of $\etiso$ is corrected
for the expected leakage of the photon energy into the isolation cone
as well as for the estimated contributions from the UE and
pile-up~\cite{pr:d83:052005,pl:b706:150}. The corrections for pile-up
and the UE are computed simultaneously on an event-by-event basis
using the jet-area method~\cite{jhep:0804:005,jhep:1004:065} as
follows: the $k_\perp$ jet algorithm~\cite{pr:d48:3160,np:b406:187}
with jet radius $R=0.5$ is used to reconstruct all jets, taking
topological clusters of calorimeter cells as input; no explicit
transverse momentum threshold is applied. The ambient transverse
energy density for the event ($\rho$), from pile-up and the UE, is
computed using the median of the distribution of the ratio of the
jet's transverse energy to its area. Finally, $\rho$ is multiplied by
the area of the isolation cone to compute the correction to
$\etiso$. The combined correction is typically $2$~\GeV\ and depends
weakly on $\etg$. In addition, for simulated events, data-driven
corrections to $\etiso$ are applied such that the peak position in the
$\etiso$ distribution coincides in data and simulation. After all
these corrections, $\etiso$ is required to be less than
$\etisoc\equiv$~\etisocut~\cite{jhep:1608:005}. The isolation
requirement significantly reduces the main background, which consists
of multi-jet events where one jet typically contains a $\pi^0$ or
$\eta$ meson that carries most of the jet energy and is misidentified
as a photon because it decays into an almost collinear photon pair. A
small fraction of the events contain more than one photon candidate
satisfying the selection criteria. In such events, the highest-$\etg$
(leading) photon is considered for further study.

Jets are reconstructed using the anti-$k_t$
algorithm~\cite{jhep:0804:063,epj:c72:1896} with a radius parameter
$R=0.4$, using topological clusters as input. The calorimeter cell
energies are measured at the EM scale, corresponding to the energy
deposited by electromagnetically interacting particles. The jet
four-momenta are computed from the sum of the jet-constituent
four-momenta, treating each as a four-vector with zero mass. The jets
are then further calibrated using the method described in
Ref.~\cite{pr:d96:072002} and these jets are referred to as
detector-level jets. The four-momentum of each jet is recalculated to
point to the selected primary vertex of the event rather than the
centre of the detector. The contribution from the UE and pile-up is
then subtracted on a jet-by-jet basis using the jet-area method. A
jet-energy calibration is derived from MC simulations as a correction
relating the calorimeter response to the true jet energy. To determine
these corrections, the jet reconstruction procedure applied to the
topological clusters is also applied to the generated stable
particles, which are defined as those with a decay length of
$c\tau>10$~mm, excluding muons and neutrinos; these jets are referred
to as particle-level jets. In addition, sequential jet corrections,
derived from MC simulated events and using global properties of the
jet such as tracking information, calorimeter energy deposits and muon
spectrometer information, are
applied~\cite{ATLAS-CONF-2015-002}. Finally, the detector-level jets
are further calibrated with additional correction factors derived in
situ from a combination of $\gamma+{\mathrm{jet}}$, $Z+{\mathrm{jet}}$
and multi-jet $p_{\mathrm{T}}$ balance
methods~\cite{epj:c75:17,ATLAS-CONF-2015-037}.\footnote{The effect of
  the correlation between the events used in the in situ $\gamma+$jet
  analysis and the events selected here has a negligible effect on the
  experimental uncertainties associated to the measurements.}

Jets reconstructed from calorimeter signals not originating from a
$pp$ collision are rejected by applying jet-quality
criteria~\cite{epj:c75:17,epj:c73:2304}. These criteria suppress
spurious jets from electronic noise in the calorimeter, cosmic rays
and beam-related backgrounds. Remaining jets are required to have
calibrated transverse momenta greater than $60$~\GeV\ and rapidity
$|\rapjet|<2.37$. Jets overlapping with the candidate photon are not
considered if the jet axis lies within a cone of size $\Delta R=0.8$
around the photon candidate; this requirement prevents any overlap
between the photon isolation cone ($\Delta R=0.4$) and the jet cone
($\Delta R=0.4$). Finally, the event is retained if the jet with
highest transverse energy (leading jet) has $\ptjet>100$~\GeV.

The total number of data events selected by using the requirements
discussed above is $895\ 726$.  This sample of events is used to
measure the cross section as a function of $\etg$, $\ptjetl$ and
$\deltaphigj$. For the measurements of the cross sections as functions
of $\mgj$ and $|\ctgj|$, the additional constraints
$|\etag+\rapjetl|<2.37$, $|\ctgj|<0.83$ and $\mgj>450$~\GeV\ are
imposed to remove the bias~\cite{np:b875:483,np:b918:257} due to the
rapidity and transverse-momentum requirements on the photon and the
jet;\footnote{The first two constraints avoid the bias induced by
  requirements on $\etag$ and $\rapjetl$, yielding slices of $\ctgj$
  with the same length along the $\etag+\rapjetl$ axis. The third
  constraint avoids the bias due to the $\etg>125$~\GeV\ requirement.}
the number of events selected in the data after these additional
requirements is $137\ 738$.

\section{Background estimation and signal extraction}
\label{bacsub}
After the requirements on photon identification and isolation are
applied to the sample of events, there is still a residual background
contribution, which arises primarily from jets identified as photons
in multi-jet events. This background contribution is estimated and
subtracted bin-by-bin using a data-driven technique which makes use of
the same two-dimensional sideband method employed in a previous
analysis~\cite{pl:b770:473}. In this approach, the photon is
classified as:
\begin{itemize}
\item ``Isolated'', if $\etiso<\etisoc$.
\item ``Non-isolated'', if $\etiso>\etisoc +2\ {\mathrm{\GeV}}$ and
  $\etiso< 50$~\GeV. The non-isolated region is separated by $2$~\GeV\
  from the isolated region to reduce the signal contamination. The
  upper bound is applied to avoid highly non-isolated
  photons.\footnote{In this way, the determination of the signal yield
    does not depend on the description by the MC generators of the
    distribution of $\etiso$ for prompt photons with high values of
    $\etiso$.} 
\item ``Tight'', if it satisfies the tight photon identification criteria.
\item ``Non-tight'', if it fails at least one of four tight
  requirements on the shower-shape variables computed from the energy
  deposits in the first layer of the EM calorimeter, but satisfies the
  tight requirement on the total lateral shower width in the first
  layer and all the other tight identification criteria in other
  layers~\cite{epj:c76:666}.
\end{itemize}

The distributions in $\etiso$ for tight and non-tight photon
candidates with $|\eta^{\gamma}|<0.6$ in the data are shown separately
in Figure~\ref{fig0} for two regions in $\etg$. The MC simulation of
the prompt-photon signal using \pyt\ is also shown. A fit of the sum
of the distributions of the \pyt\ signal photons and the non-tight
photon candidates to the distribution of the tight photon candidates
is also included. A clear signal of prompt photons centred at $\etiso$
about zero is observed.

\begin{figure}[h]
\setlength{\unitlength}{1.0cm}
\begin{picture} (18.0,6.0)
\put (0.0,-0.4){\includegraphics[width=6.8cm,height=6.8cm]{figures/fig_01a.pdf}}
\put (8.0,-0.4){\includegraphics[width=6.8cm,height=6.8cm]{figures/fig_01b.pdf}}
\end{picture}
\caption
{
  $\etiso$ distribution for tight (black dots) and non-tight (dashed
  histogram, normalised according to the fit, see text) photon
  candidates in data with $|\etag|<0.6$ in different $\etg$
  regions. The MC simulation of the signal using \pyt\ is also shown
  (dotted histogram). The solid histogram is the sum of the
  contributions of the MC simulation of the signal using \pyt\  and
  that of the non-tight photon candidates normalised according to the
  fit.
}
\label{fig0}
\end{figure}

For the estimation of the background contamination in the signal
region a two-dimensional plane is formed by $\etiso$ and a binary
variable (``tight'' vs. ``non-tight'' photon candidate) since these
two variables are expected to be largely uncorrelated for background
events. In this plane, four regions are defined: the ``signal'' region
($A$), containing tight and isolated photon candidates; the
``non-isolated'' background control region ($B$), containing tight and
non-isolated photon candidates; the ``non-tight'' background control
region ($C$), containing isolated and non-tight photon candidates; the
background control region containing non-isolated and non-tight photon
candidates ($D$).

The signal yield $N_A^{\mathrm{sig}}$ in region $A$ is estimated by
using the relation
\begin{equation}
  N_A^{\mathrm{sig}}=N_A-R^{\mathrm{bg}}\cdot(N_B-f_B N_A^{\mathrm{sig}})\cdot
  \frac{(N_C-f_C N_A^{\mathrm{sig}})}{(N_D-f_D N_A^{\mathrm{sig}})},
  \label{eqone}
  \end{equation}
where $N_K$, with $K=A,B,C,D$, is the number of events in region $K$
and $R^{\mathrm{bg}}=N_A^{\mathrm{bg}}\cdot N_D^{\mathrm{bg}}/(N_B^{\mathrm{bg}}\cdot N_C^{\mathrm{bg}})$
is the so-called background correlation and is taken as
$R^{\mathrm{bg}}=1$ for the nominal results; $N_K^{\mathrm{bg}}$ with
$K=A,B,C,D$ is the unknown number of background events in each
region. Equation~(\ref{eqone}) takes into account the expected number
of signal events in each of the three background control regions
($N_K^{\mathrm{sig}}$) via the signal leakage fractions,
$f_K=N_K^{\mathrm{sig}}/N_A^{\mathrm{sig}}$ with $K=B,C,D$, which are
estimated using the MC simulations of the signal.

The only assumption underlying Eq.~(\ref{eqone}) is that the isolation
and identification variables are uncorrelated for background events,
thus $R^{\mathrm{bg}}=1$. This assumption is verified in data
typically within $\pm 10\%$ in validation regions,\footnote{The
  validation regions are defined within the control regions $B$ and
  $D$ by splitting each of them into two subregions.} which are
dominated by background. Deviations of $R^{\mathrm{bg}}$ from unity in
the validation regions, after accounting for signal leakage using
either the \pyt\ or LO \sher\ simulations, are propagated  through
Eq.~(\ref{eqone}) and taken as systematic uncertainties. The signal
purity, defined as $N_A^{\mathrm{sig}}/N_A$, is above $90\%$ in all
bins of the measured distributions. The use of \pyt\ or LO \sher\ to
extract the signal leakage fractions lead to similar signal purities;
the difference in the signal purity is taken as a systematic
uncertainty.

The background from electrons misidentified as photons, mainly
produced in Drell--Yan $Z^{(*)}/\gamma^*\rightarrow e^+e^-$ and
$W^{(*)}\rightarrow e\nu$ processes, is also studied. Such
misidentified electrons are largely suppressed by the photon
selection. This background is estimated using MC samples of fully
simulated events and found to be negligible in the phase-space region
of the analysis presented here. These processes were simulated using
the \sher~2.2.1 generator. Matrix-elements were calculated for up to
two additional partons at NLO and up to four partons at
LO~\cite{jhep:0202:044,prl:108:111601}.

\section{Fiducial phase space and unfolding}

\subsection{Fiducial phase space}
\label{fiducial}
The cross sections are unfolded to a phase-space region close to the
applied event selection. The fiducial phase-space region is defined at
the particle level. A summary of the requirements at particle level
that define the fiducial phase-space region of the measurements is
given in Table~\ref{table1}. The cross sections as functions of
$|\ctgj|$ and $\mgj$ are measured in a fiducial phase-space region
with additional requirements, as detailed in the last row of
Table~\ref{table1}. The particle-level isolation requirement on the
photon is built by summing the transverse energy of all stable
particles (see Section~\ref{evesel}), except for muons and neutrinos,
in a cone of size $\Delta R=0.4$ around the photon direction after the
contribution from the UE is subtracted. The same underlying-event
subtraction procedure used at the reconstruction level is applied at
the particle level. The particle-level requirement on $\etiso$ is
optimised to best match the acceptance at reconstruction level using
the \pyt\ and LO \sher\ MC samples by comparing the calorimeter
isolation transverse energy with the particle-level isolation
transverse energy on an event-by-event basis. The particle-level
requirement on $\etiso$ thus optimised is
$\etiso({\mathrm{particle}})<$~\etisocutp; the same requirement is
obtained whether \pyt\ or LO \sher\ is used. Particle-level jets are
reconstructed using the anti-$k_t$ jet algorithm with radius parameter
$R=0.4$ and are built from stable particles, excluding muons and
neutrinos. At particle level, the particles associated with the
overlaid $pp$ collisions (pile-up) are not considered. 

\vspace{-0.3cm}
\begin{table}[h]
\caption{
  Summary of the requirements at particle level that define the
  fiducial phase-space region of the measurements.
}
\setlength{\unitlength}{1.0cm}
\begin{picture} (18.0,4.0)
\put (0.0,1.5){\centerline{\scalebox{1.0}{
\begin{tabular}{|l|} \hline
  {\textbf{Requirements on photons}} \\ 
  $\etg>125$~\GeV, $|\etag|<2.37$ (excluding $1.37<|\etag|<1.56$) \\
  $\etiso<$~\etisocutp \\ \hline
  {\textbf{Requirements on jets}} \\ 
  anti-$k_t$ algorithm with $R=0.4$  \\
  the leading jet within $|\rapjet|<2.37$ and $\dr>0.8$ is selected\\
  $\ptjetl>100$~\GeV \\ \hline
  {\textbf{UE subtraction using $k_\perp$ algorithm with $R=0.5$ (cf. Section~\ref{evesel})}} \\ \hline 
  {\textbf{Additional requirements for $\dsmgj$ and $\dsctgj$}} \\ 
  $|\etag+\rapjetl|<2.37$, $|\ctgj|<0.83$ and $\mgj>450$~\GeV\  \\ \hline
\end{tabular}
}}}
\end{picture}
\label{table1}
\end{table}

\subsection{Unfolding}
\label{cor}
The distributions of the background-subtracted signal yield as
functions of $\etg$, $\ptjetl$, $\deltaphigj$, $\mgj$ and $|\ctgj|$
are used to measure the corresponding differential cross sections for
isolated-photon plus jet production. The distributions are unfolded to
the particle level using MC samples of events via a bin-by-bin
technique which corrects for resolution effects and the efficiency of
the photon and jet reconstruction through the formula

\begin{equation}
  \dgen(i)=\frac{N_A^{\mathrm{sig}}(i)\ C^{\mathrm{MC}}(i)}{{\cal L}\ \Delta O(i)}, \nonumber
  \label{eqthree}
\end{equation}
where $\dgenn(i)$ is the cross section as a function of observable $O$
in bin $i$, $N_A^{\mathrm{sig}}(i)$ is the number of
background-subtracted data events in bin $i$, $C^{\mathrm{MC}}(i)$ is
the correction factor in bin $i$, ${\cal L}$ is the integrated
luminosity and $\Delta O(i)$ is the width of bin $i$. The correction
factors are computed from the MC samples as
$C^{\mathrm{MC}}(i)=N^{\mathrm{MC}}_{\mathrm{part}}(i)/N^{\mathrm{MC}}_{\mathrm{reco}}(i)$,
where $N^{\mathrm{MC}}_{\mathrm{part}}(i)$ is the number of events
that satisfy the kinematic constraints of the phase-space region at
the particle level, and $N^{\mathrm{MC}}_{\mathrm{reco}}(i)$ is the
number of events which meet all the selection criteria at the
reconstruction level.

The distributions of the signal yields as functions of $\etg$,
$\ptjetl$, $\deltaphigj$, $\mgj$ and $|\ctgj|$ in data after
background subtraction are well described by the LO \sher\ MC
simulations, but some differences are observed when compared to the
\pyt\ MC simulations, in particular in the tail of the $\ptjetl$
distribution. A better description of the data distributions as
functions of $\etg$, $\ptjetl$, $\deltaphigj$, $\mgj$ and $|\ctgj|$ by
\pyt\ is obtained by increasing/decreasing the relative contribution
from direct processes with respect to bremsstrahlung
processes~\cite{np:b875:483,np:b918:257}; the resulting \pyt\
simulations are referred to as \pyt-adjusted simulations.

The unfolded cross sections are measured using the signal leakage
fractions from \pyt-adjusted simulations since these include an
unbiased sample of non-isolated photons,\footnote{In the LO \sher\
  samples, the application of the Frixione's criterion to the photon
  isolation at matrix-element level prevents the radiated photon from
  being close to a parton. In the \pyt\ samples, the bremsstrahlung
  component is simulated with a parton shower approach and, as a
  result, the radiated photon can be close to a parton.} and the
correction factors, $C^{\mathrm{MC}}$, from LO \sher\ since these
simulations give somewhat better agreement with the data distributions
as functions of $\etg$, $\ptjetl$, $\deltaphigj$, $\mgj$ and
$|\ctgj|$. The correction factors vary between $1.08$ and $1.21$
depending on the observable. The results of the bin-by-bin unfolding
procedure are checked with an iterative Bayesian unfolding
method~\cite{nim:a362:487} based on LO \sher\ simulations, giving
consistent results.

\section{Uncertainties in the cross-section measurements}
\label{syst}
{\textbf{Photon energy scale and resolution.}}
A detailed assessment of the uncertainties in the photon energy scale
and resolution is made using the method reported in
Ref.~\cite{pl:b770:473}. The photon energy scale uncertainties come
mostly from calibration studies using 8~\TeV\
data~\cite{epj:c74:3071}, with additional systematic uncertainties to
take into account the differences between the 2012 and 2015
configurations. The uncertainties are split into independent
components to account for correlations of the uncertainties between
different bins of the measured cross sections. The individual sources
of uncertainty are varied  by $\pm 1\sigma$ in the MC simulations and
propagated through the analysis separately to maintain the full
information about the correlations of the uncertainties between
different bins of the measured cross sections. The impact of the
photon energy resolution uncertainty is much smaller than that of the
photon energy scale uncertainty. The resulting uncertainty in the
measured cross sections is obtained by adding in quadrature all the
individual components and increases from $1\%$ at $\etg=125$~\GeV\ to
$4.5\%$ at $\etg\sim 1.5$~\TeV.

{\textbf{Jet energy scale and resolution.}}
A detailed assessment of the uncertainties in the jet energy scale and
resolution is made using the method reported in
Ref.~\cite{pr:d96:072002}. The individual sources of
uncertainty~\cite{pr:d96:072002} are varied  by $\pm 1\sigma$ in the MC
simulations and propagated through the analysis separately, to
maintain the full information about the correlations of the
uncertainties between different bins of the measured cross
sections. The resulting uncertainty in the measured cross sections is
obtained by adding in quadrature all the individual components and
increases from $1.9\%$ at $\ptjetl=100$~\GeV\ to $7.5\%$ at
$\ptjetl\sim 1$~\TeV.

{\textbf{Parton-shower and hadronisation model dependence.}}
The effects due to the parton-shower and hadronisation models on the
signal purity and detector-to-particle-level correction factors are
studied separately. The effects on the signal purity are estimated as
the differences observed between the nominal results and those
obtained using either the (non-adjusted) \pyt\ or LO \sher\ MC samples
for the determination of the signal leakage fractions. The difference
between the nominal results and those obtained using the \pyt-adjusted
MC samples for the determination of the unfolding correction factors
is taken as a systematic uncertainty. The resulting uncertainties in
the measured cross sections are typically smaller than $2\%$.

{\textbf{Photon identification efficiency.}}
The uncertainty in the photon identification efficiency is estimated
from the effect of differences between shower-shape variable
distributions in data and simulation. From the studies presented in
Refs.~\cite{epj:c76:666,ATL-PHYS-PUB-2016-014}, this procedure is
found to provide a conservative estimate of the uncertainties. The
resulting uncertainty in the measured cross sections is in the range
$1$--$2\%$. The effects on the measured cross sections due to the
uncertainty in the photon reconstruction efficiency, which are
evaluated by repeating the full analysis using a different detector
simulation with increased material in front of the calorimeter, are
found to be negligible.

{\textbf{Photon isolation modelling.}}
The differences between the nominal results and those obtained without
applying the data-driven corrections to $\etiso$ in simulated events
are taken as systematic uncertainties in the measurements due to the
modelling of $\etiso$ in the MC simulation. The resulting uncertainty
in the measured cross sections is less than $1.1\%$.

{\textbf{Definition of the background control regions.}}
The estimation of the background contamination in the signal region is
affected by the choice of background control regions. The control
regions $B$ and $D$ are defined by the lower and upper limits on
$\etiso$ and the choice of inverted photon identification variables
used in the selection of non-tight photons. To study the dependence on
the specific choices, these definitions are varied over a wide
range. The lower limit on $\etiso$ in regions $B$ and $D$ is varied by
$\pm 1$~\GeV, which is larger than any difference between data and
simulations and still provides a sufficient sample to perform the
data-driven subtraction. The upper limit on $\etiso$ in regions $B$
and $D$ is removed. The resulting uncertainty in the measured cross
sections is negligible. Likewise, the choice of inverted photon
identification variables is varied. The analysis is repeated using
different sets of variables: tighter (looser) identification criteria
are defined by applying tight requirements to an extended (restricted)
set of shower-shape variables in the first calorimeter
layer~\cite{epj:c76:666,ATL-PHYS-PUB-2016-014}. The resulting
uncertainty in the measured cross sections is smaller than $1.3\%$.

{\textbf{Photon identification and isolation correlation in the
    background.}}
The photon isolation and identification variables used to define the
plane in the two-dimensional sideband method to subtract the
background are assumed to be independent for background events
($\rbg=1$ in Eq.~(\ref{eqone})). Any correlation between these
variables affects the estimation of the purity of the signal sample
and leads to systematic uncertainties in the background-subtraction
procedure. A range in $\rbg$ is set to cover the deviations from unity
measured in the validation regions after subtracting the signal
leakage with either \pyt-adjusted or LO \sher\ MC samples. The
resulting uncertainty in all measured cross sections is less than
$2\%$.

{\textbf{Pile-up.}}
The uncertainty is estimated by changing the nominal rescaling factor
of $1.16$ to $1.09$ or $1.23$ and re-evaluating the reweighting
factors. The resulting uncertainty in the measured cross sections is
typically less than $0.5\%$.

{\textbf{Unfolding procedure.}}
The uncertainty is estimated by comparing the nominal results with
those obtained by unfolding with LO \sher\ MC samples reweigthed to
match the data distributions. The resulting uncertainty in the
measured cross sections is negligible.

The total systematic uncertainty is computed by adding in quadrature
the uncertainties from the sources listed above, the statistical
uncertainty of the MC samples, the uncertainty in the trigger
efficiency ($1\%$) and the uncertainty in the integrated luminosity,
which is fully correlated between all bins of all the measured cross
sections. There are large correlations in the systematic uncertainties
across bins of one observable, particularly in the uncertainties due
to the photon and jet energy scales, which are dominant. The total
systematic uncertainty, excluding that in the luminosity, is less than
$5\%$ for $\etg$, $4\%$ for $\deltaphigj$, $6\%$ for $\mgj$ and $4\%$
for $|\ctgj|$ and increases from $4\%$ at $\ptjetl=100$~\GeV\ to
$10\%$ at $\ptjetl\sim 1.5$~\TeV. Figure~\ref{fig1} shows the total
systematic uncertainty  for each measured cross section, excluding
that in the luminosity; the dominant components are shown separately
in this Figure. The systematic uncertainty dominates the total
experimental uncertainty for $\etg\lesssim 700$~\GeV\ and
$\mgj\lesssim 1.5$~\TeV, whereas for higher $\etg$ and $\mgj$ values,
the statistical uncertainty of the data limits the precision of the
measurements. For $\ptjetl$, $\deltaphigj$ and $|\ctgj|$, the
systematic uncertainty dominates in the whole measured range.

\section{Theoretical predictions}
\label{nlo}
The NLO pQCD predictions presented in this Letter are computed using
two programs, namely \jetp~1.3.1\_2 and \sher~2.2.2. The \jetp\
program includes a full NLO pQCD calculation of both the direct and
fragmentation contributions to the cross section for the $\ppgj$
process. The number of massless quark flavours is set to five. The
renormalisation scale $\muR$, factorisation scale $\muF$ and
\begin{figure}[h]
\setlength{\unitlength}{1.0cm}
\begin{picture} (18.0,20.0)
\put (0.0,14.0){\includegraphics[width=7cm,height=7cm]{figures/fig_02a.pdf}}
\put (8.0,14.0){\includegraphics[width=7cm,height=7cm]{figures/fig_02b.pdf}}
\put (0.0,7.0){\includegraphics[width=7cm,height=7cm]{figures/fig_02c.pdf}}
\put (0.0,0.0){\includegraphics[width=7cm,height=7cm]{figures/fig_02d.pdf}}
\put (8.0,0.0){\includegraphics[width=7cm,height=7cm]{figures/fig_02e.pdf}}
\end{picture}
\caption
{
  Total relative systematic uncertainty (solid lines), excluding that
  in the luminosity measurement, as a function of  $\etg$, $\ptjetl$,
  $\deltaphigj$, $\mgj$ and $|\ctgj|$. The three dominant
  contributions are also included separately: the jet energy scale
  (dashed lines), the photon energy scale (dotted lines) and the
  photon identification (dot-dashed lines). The shaded band displays
  the relative statistical uncertainty; for the last point in $\etg\
  (\mgj)$ the relative statistical uncertainty is $32\%$ ($30\%$).
}
\label{fig1}
\end{figure}
\FloatBarrier
fragmentation scale $\muf$ are chosen to be
$\muR=\muF=\muf=\etg$~\cite{pr:d73:094007}. The calculations are
performed using the MMHT2014~\cite{epj:c75:204} PDF set and the BFG
set II of parton-to-photon fragmentation functions~\cite{epj:c2:529},
both at NLO. The strong coupling constant is set to $\asz=0.120$. The
calculations are performed using a parton-level  isolation criterion
which requires the total transverse energy from the partons inside a
cone of size $\Delta R=0.4$ around the photon direction to be below
\etisocutp. The NLO pQCD predictions from \jetp\ are at the parton
level while the measurements are at the particle level. Thus, there
can be differences between the two levels concerning the photon
isolation as well as the photon and jet four-momenta. Since the data
are corrected for pile-up and UE effects and the distributions are
unfolded to a phase-space definition in which the requirement on
$\etiso$ at particle level is applied after subtraction of the UE, it
is expected that parton-to-hadron corrections to the NLO pQCD
predictions are small. Correction factors to the \jetp\ predictions
are estimated by computing the ratio of the particle-level cross
section for a \pyt\ sample with UE effects to the parton-level cross
section without UE effects. These factors are close to unity within
$\pm 5\%$ for the observables studied, except for $\ptjetl\gtrsim
600$~\GeV; in this region, which is dominated by the bremsstrahlung
component, the factors can differ by up to $30\%$ from unity since
hadronisation of a nearby parton can significantly change the
particle-level isolation compared to the parton-level isolation.

The \sher~2.2.2 program consistently combines parton-level
calculations of $\gamma+ {\mathrm{(1,\ 2)-jet}}$ events at NLO and
$\gamma+{\mathrm{(3,\ 4)-jet}}$ events at
LO~\cite{jhep:0202:044,prl:108:111601} supplemented with a parton
shower~\cite{jhep:0803:038} while avoiding double-counting
effects~\cite{jhep:1304:027}. A requirement on the photon isolation at
the matrix-element level is imposed using Frixione's criterion with
${\cal R}=0.1$, $n=2$ and $\epsilon=0.1$. Dynamic factorisation and
renormalisation scales are adopted as well as a dynamical merging
scale with $\bar Q_{\mathrm{cut}}=20$~\GeV~\cite{jp:g44:044007}. The
strong coupling constant is set to $\asz=0.118$. Fragmentation into
hadrons and simulation of the UE are performed using the same models
as for the LO \sher\ samples. The next-to-next-to-leading-order (NNLO)
NNPDF3.0 PDF set~\cite{jhep:1504:040} is used in conjunction with the
corresponding \sher\ tuning. All the NLO \sher\  predictions are based
on the particle-level observables from this computation after applying
the requirements listed in Table~\ref{table1}.

\FloatBarrier
\subsection{Uncertainties in the predictions}
The uncertainty in the NLO pQCD predictions from \jetp\ due to terms
beyond NLO is estimated by repeating the calculations using values of
$\muR$, $\muF$ and $\muf$ scaled by the factors $0.5$ and $2$. The
three scales  are either varied simultaneously, individually or by
fixing one and varying the other two. In all cases, the condition
$0.5\leq\mu_A/\mu_B\leq 2$ is imposed, where $A,B={\mathrm{R,\ F,\
    f}}$. The final uncertainty is taken as the largest deviation from
the nominal value among the 14 possible variations. In the case of the
NLO \sher\ prediction, which does not include the fragmentation
contribution, $\muR$ and $\muF$ are varied as above and the largest
deviation from the nominal value among the 6 possible variations is
taken as the uncertainty.

The uncertainty in the NLO pQCD predictions from \jetp\ due to the
choice of proton PDFs is estimated by repeating the calculations using
the $50$ sets from the MMHT2014 error analysis~\cite{epj:c75:204} and
applying the Hessian method~\cite{pr:d65:014013} for evaluation of the
PDF uncertainties. In the case of NLO \sher, it is estimated using 100
replicas from the NNPDF3.0 analysis~\cite{jhep:1504:040}.

The uncertainty in the NLO pQCD predictions from \jetp\ (NLO \sher)
due to the uncertainty in $\as$ is estimated by repeating the
calculations using two additional sets of proton PDFs from the
MMHT2014 (NNPDF3.0) analysis, for which different values of $\as$ at
$\mz$ were assumed in the fits, namely $0.118$ ($0.117$) and $0.122$
($0.119$); in this way, the correlation between $\as$ and the PDFs is
preserved.

The uncertainty in the parton-to-hadron correction is estimated by
comparing the values obtained using different tunes of \pyt: the ATLAS
set of tuned parameters for the underlying event (tune
AU2)~\cite{ATL-PHYS-PUB-2012-003} with the CTEQ6L1 PDF
set~\cite{jhep:0207:012}, the A14 tune with the LO NNPDF2.3 PDF set as
well as the tunes in which the parameter settings of the latter
related to the modelling of the UE are
varied~\cite{ATL-PHYS-PUB-2014-021}. Larger differences are obtained
from the comparison of the two central tunes than from the variations
around the A14 tune. The nominal correction is taken as the average of
the corrections using the two central tunes, while the uncertainty is
estimated as half of the difference between the two central tunes.

The dominant theoretical uncertainty is that arising from the terms
beyond NLO and, in the case of \jetp\ (NLO \sher), is $\approx 10\%$
($15$--$25\%$) for $\etg$, $\mgj$ and $|\ctgj|$ and increases from
$5\%$ ($15\%$) at $\ptjetl=130$~\GeV\ to $30\%$ ($30\%$) for
$\ptjetl=1.5$~\TeV. In the case of the NLO \sher\ prediction for
$\dsphi$, the uncertainty increases from $10\%$ at
$\deltaphigj\sim\pi$ to $40\%$ at $\deltaphigj\sim\pi/2$. The
uncertainty in the predictions of \jetp\ (NLO \sher) arising from that
in the PDFs is $\lesssim 2\%$ ($3\%$) for all observables. The
uncertainty arising from the value of $\asz$ is below $2\%$ ($5\%$).
The uncertainty in the parton-to-hadron correction is in the range
$1$--$3\%$ except for $\ptjetl\gtrsim 600$~\GeV, where it increases to
$20\%$ at $\ptjetl = 1.5$~\TeV; this uncertainty is included in the
\jetp\ predictions, but not in the case of  NLO \sher\ since it is a
particle-level Monte Carlo generator.\footnote{An uncertainty related
  to the modelling of the hadronisation process should also be
  assigned to the NLO \sher\ predictions, but no tune other than the
  default one is available. It is expected that the uncertainty should
  be of similar size as that evaluated using \pyt.} The total
theoretical uncertainty is obtained by adding in quadrature the
individual uncertainties listed above and, in the case of \jetp\ (NLO
\sher), is $10$--$15\%$ ($15$--$25\%$) except for $\ptjetl$, where it
is in the range $10$--$40\%$ ($15$--$30\%$); in the case of the NLO
\sher\ prediction for $\dsphi$, the total uncertainty is
$10$--$40\%$.

\section{Results}
\label{res}
The measurements presented here apply to isolated prompt photons with
$\etiso<$~\etisocutp\ at particle level and jets of hadrons. The
measured fiducial cross section for isolated-photon plus one-jet
production in the phase-space region given in Table~\ref{table1} is 
$\sigma_{\mathrm{meas}} = 300 \pm 10\ ({\mathrm{exp.}})\ \pm 6\ ({\mathrm{lumi.}})\ {\mathrm{pb}},$
where ``exp.'' denotes the sum in quadrature of the statistical and
systematic uncertainties excluding that due to the luminosity and
``lumi.'' denotes the uncertainty due to that in the integrated
luminosity. The fiducial cross sections predicted by NLO QCD \jetp\
(multi-leg NLO QCD plus parton-shower \sher) using the MMHT2014
(NNPDF3.0) PDF set are

\vspace{0.25cm}
\centerline{$\sigma_{\mathrm{\jetp}}=291_{-21}^{+25}\ ({\mathrm{scale}})\  _{-3}^{+2}\ ({\mathrm{PDF}})\ _{-5}^{+4} \ ({\mathrm{\alpha_{\mathrm{s}}}})\ \pm 6 \ ({\mathrm{non\mhyphen perturb.}})\ {\mathrm {pb}}$}

and

\centerline{$\sigma_{\mathrm{NLO \sher}}=319_{-45}^{+54}\ ({\mathrm{scale}})\ \pm 3\ ({\mathrm{PDF}})\ _{-11}^{+10} \ ({\mathrm{\alpha_{\mathrm{s}}}})\ {\mathrm {pb}},$}
which are consistent with the measurement within the theoretical uncertainties.

Figure~\ref{fig2} shows the isolated-photon plus jet cross sections as
functions of $\etg$, $\ptjetl$, $\deltaphigj$, $\mgj$ and
$|\ctgj|$. The measured $\dsetg$ decreases by almost six orders of
magnitude over the $\etg$ range studied. Values of $\etg$ up to
$1.5$~\TeV\ are measured. The measured $\dspt$ decreases by more than
four orders of magnitude from $\ptjetl = 100$~\GeV\ up to the highest
measured value, $\ptjetl\approx 1.5$~\TeV. The measurement of $\dsphi$
is restricted to $\deltaphigj>\pi/2$ to avoid the phase-space region
dominated by photon production in association with a multi-jet
system. The measured $\dsphi$ increases as
\begin{figure}[h]
\setlength{\unitlength}{1.0cm}
\begin{picture} (18.0,20.0)
\put (0.0,13.0){\includegraphics[width=8cm,height=8cm]{figures/fig_03a.pdf}}
\put (8.0,13.0){\includegraphics[width=8cm,height=8cm]{figures/fig_03b.pdf}}
\put (0.0,6.0){\includegraphics[width=8cm,height=8cm]{figures/fig_03c.pdf}}
\put (0.0,-1.0){\includegraphics[width=8cm,height=8cm]{figures/fig_03d.pdf}}
\put (8.0,-1.0){\includegraphics[width=8cm,height=8cm]{figures/fig_03e.pdf}}
\end{picture}
\caption
{
  Measured cross sections for isolated-photon plus jet production
  (dots) as functions of $\etg$, $\ptjetl$, $\deltaphigj$, $\mgj$ and
  $|\ctgj|$; the observables are constructed using the leading photon
  and the leading jet.  For comparison, the tree-level plus
  parton-shower predictions from LO \sher\  (solid lines)  and \pyt\
  (dashed lines) normalised to the integrated measured cross sections
  (using the factors indicated in parentheses) are also shown. The
  theoretical uncertainties associated with the tree-level predictions
  are not included. The bottom part of each figure shows the ratios of
  the MC predictions to the measured cross section. The inner (outer)
  error bars represent the statistical uncertainties (the statistical
  and systematic uncertainties added in quadrature). For most of the
  points, the inner error bars are smaller than the marker size and,
  thus, not visible.
}
\label{fig2}
\end{figure}
\FloatBarrier
$\deltaphigj$ increases. The measured $\dsmgj$ decreases by more than
four orders of magnitude up to the highest measured value,
$\mgj=3.25$~\TeV. The measured $\dsctgj$ increases as $|\ctgj|$
increases.

The tree-level predictions of the \pyt\ and LO \sher\ MC models are
compared to the measurements in Figure~\ref{fig2}. These predictions
are normalised to the measured integrated fiducial cross section. The
difference in normalisation between data and \pyt\ (LO \sher) is $\sim
+10\%\ (+40\%)$ and attributed to the fact that these generators are
based on tree-level matrix elements, which are affected by a large
normalisation uncertainty due to missing higher-order terms; for this
reason, the theoretical uncertainties are not included in
Figure~\ref{fig2}. Both predictions give an adequate description of
the shape of the measured $\dsetg$, although \pyt\ is slightly better
than LO \sher\ for $\etg\lesssim 600$~\GeV. For $\dspt$, the
prediction from LO \sher\ gives an adequate description of the data in
the whole measured range, whereas that from \pyt\ overestimates the
data for $\ptjetl\gtrsim 200$~\GeV; the overestimation is attributed
to a large contribution from photon bremsstrahlung predicted by the
tune used in \pyt\ (see Section~\ref{datmc}). The prediction from LO
\sher\ gives a good description of the measured $\dsphi$, whereas
\pyt\ underestimates the data for
$3\pi/5<\deltaphigj<4\pi/5$~rad; this is expected from the inclusion
of additional partons in the matrix elements in \sher\ as compared to
\pyt, for which additional partons must necessarily come from the
parton shower. Both predictions give a good description of the data
for $\mgj<1.25$~\TeV\ and for all of the measured $|\ctgj|$ range.

To illustrate the sensitivity to $t$-channel quark or gluon exchange,
the predicted cross-sections $\dsctgj$ from \jetp\ for LO direct and
fragmentation processes are compared to the measurement in
Figure~\ref{fig5}. Even though the two components are no longer
distinguishable at NLO, the LO calculations are useful in illustrating
the basic differences in the dynamics of the two processes. The
contribution from fragmentation, dominated by gluon exchange, shows a
steeper increase as $|\ctgj| \rightarrow 1$ than that from direct
processes, dominated by quark exchange. The shape of the measured
cross-section $\dsctgj$ is closer to that of the direct processes than
that of fragmentation. This is consistent with the dominance of
processes in which the exchanged particle is a quark.

\begin{figure}[h]
\setlength{\unitlength}{1.0cm}
\begin{picture} (18.0,5.5)
\put (0.0,-2.2){\centerline{\includegraphics[width=8cm,height=8cm]{figures/fig_04.pdf}}}
\end{picture}
\vspace{-1.0cm}
\caption
{
  Measured cross section for isolated-photon plus jet production
  (dots) as a function of $|\ctgj|$; the observable is constructed
  using the leading photon and the leading jet. For comparison, the LO
  QCD predictions from \jetp, normalised to the integrated measured
  cross section by the factors shown in parentheses, of direct (dashed
  lines) and fragmentation (dotted lines) processes are shown
  separately. The error bars are smaller than the marker size and,
  thus, not visible.
}
\label{fig5}
\end{figure}

The predictions of the fixed-order NLO QCD calculations of \jetp\
based on the MMHT2014 proton PDF set and corrected for hadronisation
and UE effects as explained in Section~\ref{nlo} are compared to the

\begin{figure}[p]
\setlength{\unitlength}{1.0cm}
\begin{picture} (18.0,20.0)
\put (0.0,13.0){\includegraphics[width=8cm,height=8cm]{figures/fig_05a.pdf}}
\put (8.0,13.0){\includegraphics[width=8cm,height=8cm]{figures/fig_05b.pdf}}
\put (0.0,6.1){\includegraphics[width=8cm,height=8cm]{figures/fig_05c.pdf}}
\put (0.0,-0.8){\includegraphics[width=8cm,height=8cm]{figures/fig_05d.pdf}}
\put (8.0,-0.8){\includegraphics[width=8cm,height=8cm]{figures/fig_05e.pdf}}
\end{picture}
\vspace{-0.7cm}
\caption
{
  Measured cross sections for isolated-photon plus jet production
  (dots) as functions of $\etg$, $\ptjetl$,  $\deltaphigj$,  $\mgj$
  and $|\ctgj|$; the observables are constructed using the leading
  photon and the leading jet.  For comparison, the multi-leg NLO QCD
  plus parton shower predictions from NLO \sher\ (dashed lines) and
  the NLO QCD predictions from \jetp\ corrected for hadronisation and
  UE effects (solid lines) are also shown. The bottom part of each
  figure shows the ratios of the predictions to the measured cross
  section. The inner (outer) error bars represent the statistical
  uncertainties (the statistical and systematic uncertainties added in
  quadrature) and the bands display the theoretical uncertainty. For
  most of the points, the inner error bars are smaller than the marker
  size and, thus, not visible.
}
\label{fig4}
\end{figure}
\FloatBarrier
measurements\footnote{As shown in Ref.~\cite{np:b875:483}, the NLO QCD
  predictions of \jetp\ cannot describe $\dsphi$ due to the limited
  number of final-state partons.} in Figure~\ref{fig4}. The
predictions of the multi-leg NLO QCD plus parton-shower
calculations of \sher\ based on the NNPDF3.0 PDF set are also compared to the
measurements in Figure~\ref{fig4}. Both types of predictions describe
the data within the experimental and theoretical uncertainties. For
the cross section as a function of $\deltaphigj$, the only
well-founded prediction is that of NLO \sher, which is able to
reproduce the data down to $\deltaphigj=\pi/2$ due to the inclusion of
the matrix elements for $2\rightarrow n$ processes with $n=4$ and
$5$. For most of the points, the theoretical uncertainties are larger
than those of experimental origin. Predictions for \jetp\ (\sher\ NLO)
are also obtained with other PDF sets,  namely NLO
CT14~\cite{pr:d93:033006} and NLO NNPDF3.0 (CT14 and MMHT2014),  and
differ by less than $5\%$ with respect to those using MMHT2014
(NNPDF3.0). Thus, the description of the data achieved by the
predictions does not depend significantly on the specific PDF set
used. It is concluded that the NLO pQCD predictions provide an
adequate description of the measurements within the uncertainties.

\section{Summary}
\label{conc}
Measurements of the cross section for the production of an isolated
photon in association with jets in $pp$ collisions at $\sqrt
s=13$~\TeV, $\ppgj$, are presented. These measurements are based on an
integrated luminosity of $3.2$~\fb1\ of ATLAS data recorded at the
LHC. The photon is required to have $\etg>125$~\GeV\ and
$|\etag|<2.37$, excluding the region $1.37<|\etag|<1.56$. The jets are
reconstructed using the anti-$k_t$ algorithm with radius parameter
$R=0.4$. The cross sections are measured as functions of $\etg$,
$\ptjetl$ and $\deltaphigj$ with $\ptjetl>100$~\GeV; the measurements
extend up to values of $1.5$~\TeV\ in $\etg$ and $\ptjetl$. The
dependence on $\mgj$ and $|\ctgj|$ is measured for $\mgj>450$~\GeV.

The predictions of the tree-level plus parton-shower MC models by
\pyt\ and LO \sher\ give a satisfactory description of the shape of
the data distributions, except for $\ptjetl$ in the case of \pyt. The
fixed-order NLO QCD calculations of \jetp, corrected for hadronisation
and UE effects, and the multi-leg NLO QCD plus parton-shower
calculations of \sher\ describe the measured cross sections within the
experimental and theoretical uncertainties. The comparison of
predictions based on different parameterisations of the proton PDFs
shows that the description of the data achieved does not depend
significantly on the specific PDF set used. The only well-founded
prediction for $\dsphi$ is that of NLO \sher, which is able to
reproduce the data down to $\deltaphigj=\pi/2$ due to the inclusion of
the matrix elements for $2\rightarrow n$ processes with $n=4$ and
$5$. The measured dependence on $|\ctgj|$ is consistent with the
dominance of processes in which a quark is exchanged. All these
studies provide tests of the pQCD description of the dynamics of
isolated-photon plus jet production in $pp$ collisions at $\sqrt
s=13$~\TeV. The experimental uncertainties are, in general, much
smaller than the uncertainties in the predictions and, thus,
calculations with higher precision will allow stringent tests of the
theory.

\section*{Acknowledgements}

\section*{Acknowledgements}
This document is an updated version of HPCA 2022 and 2023, which, in
turn, has been derived from two previous conferences, in particular
HPCA 2021 and MICRO 2021, which, in turn, are derived from past MICRO,
HPCA, ISCA, and ASPLOS conferences.



\printbibliography

\newpage
% ATLAS Collaboration author list
% Data extracted on 17-Jul-2017 for paper reference HIGG-2016-14
%\documentclass[11pt]{article}
%\usepackage{a4wide}\begin{document}
\begin{flushleft}
{\Large The ATLAS Collaboration}

\bigskip

M.~Aaboud$^\textrm{\scriptsize 137d}$,
G.~Aad$^\textrm{\scriptsize 88}$,
B.~Abbott$^\textrm{\scriptsize 115}$,
O.~Abdinov$^\textrm{\scriptsize 12}$$^{,*}$,
B.~Abeloos$^\textrm{\scriptsize 119}$,
S.H.~Abidi$^\textrm{\scriptsize 161}$,
O.S.~AbouZeid$^\textrm{\scriptsize 139}$,
N.L.~Abraham$^\textrm{\scriptsize 151}$,
H.~Abramowicz$^\textrm{\scriptsize 155}$,
H.~Abreu$^\textrm{\scriptsize 154}$,
R.~Abreu$^\textrm{\scriptsize 118}$,
Y.~Abulaiti$^\textrm{\scriptsize 148a,148b}$,
B.S.~Acharya$^\textrm{\scriptsize 167a,167b}$$^{,a}$,
S.~Adachi$^\textrm{\scriptsize 157}$,
L.~Adamczyk$^\textrm{\scriptsize 41a}$,
J.~Adelman$^\textrm{\scriptsize 110}$,
M.~Adersberger$^\textrm{\scriptsize 102}$,
T.~Adye$^\textrm{\scriptsize 133}$,
A.A.~Affolder$^\textrm{\scriptsize 139}$,
T.~Agatonovic-Jovin$^\textrm{\scriptsize 14}$,
C.~Agheorghiesei$^\textrm{\scriptsize 28c}$,
J.A.~Aguilar-Saavedra$^\textrm{\scriptsize 128a,128f}$,
S.P.~Ahlen$^\textrm{\scriptsize 24}$,
F.~Ahmadov$^\textrm{\scriptsize 68}$$^{,b}$,
G.~Aielli$^\textrm{\scriptsize 135a,135b}$,
S.~Akatsuka$^\textrm{\scriptsize 71}$,
H.~Akerstedt$^\textrm{\scriptsize 148a,148b}$,
T.P.A.~{\AA}kesson$^\textrm{\scriptsize 84}$,
E.~Akilli$^\textrm{\scriptsize 52}$,
A.V.~Akimov$^\textrm{\scriptsize 98}$,
G.L.~Alberghi$^\textrm{\scriptsize 22a,22b}$,
J.~Albert$^\textrm{\scriptsize 172}$,
P.~Albicocco$^\textrm{\scriptsize 50}$,
M.J.~Alconada~Verzini$^\textrm{\scriptsize 74}$,
S.C.~Alderweireldt$^\textrm{\scriptsize 108}$,
M.~Aleksa$^\textrm{\scriptsize 32}$,
I.N.~Aleksandrov$^\textrm{\scriptsize 68}$,
C.~Alexa$^\textrm{\scriptsize 28b}$,
G.~Alexander$^\textrm{\scriptsize 155}$,
T.~Alexopoulos$^\textrm{\scriptsize 10}$,
M.~Alhroob$^\textrm{\scriptsize 115}$,
B.~Ali$^\textrm{\scriptsize 130}$,
M.~Aliev$^\textrm{\scriptsize 76a,76b}$,
G.~Alimonti$^\textrm{\scriptsize 94a}$,
J.~Alison$^\textrm{\scriptsize 33}$,
S.P.~Alkire$^\textrm{\scriptsize 38}$,
B.M.M.~Allbrooke$^\textrm{\scriptsize 151}$,
B.W.~Allen$^\textrm{\scriptsize 118}$,
P.P.~Allport$^\textrm{\scriptsize 19}$,
A.~Aloisio$^\textrm{\scriptsize 106a,106b}$,
A.~Alonso$^\textrm{\scriptsize 39}$,
F.~Alonso$^\textrm{\scriptsize 74}$,
C.~Alpigiani$^\textrm{\scriptsize 140}$,
A.A.~Alshehri$^\textrm{\scriptsize 56}$,
M.I.~Alstaty$^\textrm{\scriptsize 88}$,
B.~Alvarez~Gonzalez$^\textrm{\scriptsize 32}$,
D.~\'{A}lvarez~Piqueras$^\textrm{\scriptsize 170}$,
M.G.~Alviggi$^\textrm{\scriptsize 106a,106b}$,
B.T.~Amadio$^\textrm{\scriptsize 16}$,
Y.~Amaral~Coutinho$^\textrm{\scriptsize 26a}$,
C.~Amelung$^\textrm{\scriptsize 25}$,
D.~Amidei$^\textrm{\scriptsize 92}$,
S.P.~Amor~Dos~Santos$^\textrm{\scriptsize 128a,128c}$,
S.~Amoroso$^\textrm{\scriptsize 32}$,
G.~Amundsen$^\textrm{\scriptsize 25}$,
C.~Anastopoulos$^\textrm{\scriptsize 141}$,
L.S.~Ancu$^\textrm{\scriptsize 52}$,
N.~Andari$^\textrm{\scriptsize 19}$,
T.~Andeen$^\textrm{\scriptsize 11}$,
C.F.~Anders$^\textrm{\scriptsize 60b}$,
J.K.~Anders$^\textrm{\scriptsize 77}$,
K.J.~Anderson$^\textrm{\scriptsize 33}$,
A.~Andreazza$^\textrm{\scriptsize 94a,94b}$,
V.~Andrei$^\textrm{\scriptsize 60a}$,
S.~Angelidakis$^\textrm{\scriptsize 9}$,
I.~Angelozzi$^\textrm{\scriptsize 109}$,
A.~Angerami$^\textrm{\scriptsize 38}$,
A.V.~Anisenkov$^\textrm{\scriptsize 111}$$^{,c}$,
N.~Anjos$^\textrm{\scriptsize 13}$,
A.~Annovi$^\textrm{\scriptsize 126a,126b}$,
C.~Antel$^\textrm{\scriptsize 60a}$,
M.~Antonelli$^\textrm{\scriptsize 50}$,
A.~Antonov$^\textrm{\scriptsize 100}$$^{,*}$,
D.J.~Antrim$^\textrm{\scriptsize 166}$,
F.~Anulli$^\textrm{\scriptsize 134a}$,
M.~Aoki$^\textrm{\scriptsize 69}$,
L.~Aperio~Bella$^\textrm{\scriptsize 32}$,
G.~Arabidze$^\textrm{\scriptsize 93}$,
Y.~Arai$^\textrm{\scriptsize 69}$,
J.P.~Araque$^\textrm{\scriptsize 128a}$,
V.~Araujo~Ferraz$^\textrm{\scriptsize 26a}$,
A.T.H.~Arce$^\textrm{\scriptsize 48}$,
R.E.~Ardell$^\textrm{\scriptsize 80}$,
F.A.~Arduh$^\textrm{\scriptsize 74}$,
J-F.~Arguin$^\textrm{\scriptsize 97}$,
S.~Argyropoulos$^\textrm{\scriptsize 66}$,
M.~Arik$^\textrm{\scriptsize 20a}$,
A.J.~Armbruster$^\textrm{\scriptsize 32}$,
L.J.~Armitage$^\textrm{\scriptsize 79}$,
O.~Arnaez$^\textrm{\scriptsize 161}$,
H.~Arnold$^\textrm{\scriptsize 51}$,
M.~Arratia$^\textrm{\scriptsize 30}$,
O.~Arslan$^\textrm{\scriptsize 23}$,
A.~Artamonov$^\textrm{\scriptsize 99}$,
G.~Artoni$^\textrm{\scriptsize 122}$,
S.~Artz$^\textrm{\scriptsize 86}$,
S.~Asai$^\textrm{\scriptsize 157}$,
N.~Asbah$^\textrm{\scriptsize 45}$,
A.~Ashkenazi$^\textrm{\scriptsize 155}$,
L.~Asquith$^\textrm{\scriptsize 151}$,
K.~Assamagan$^\textrm{\scriptsize 27}$,
R.~Astalos$^\textrm{\scriptsize 146a}$,
M.~Atkinson$^\textrm{\scriptsize 169}$,
N.B.~Atlay$^\textrm{\scriptsize 143}$,
K.~Augsten$^\textrm{\scriptsize 130}$,
G.~Avolio$^\textrm{\scriptsize 32}$,
B.~Axen$^\textrm{\scriptsize 16}$,
M.K.~Ayoub$^\textrm{\scriptsize 119}$,
G.~Azuelos$^\textrm{\scriptsize 97}$$^{,d}$,
A.E.~Baas$^\textrm{\scriptsize 60a}$,
M.J.~Baca$^\textrm{\scriptsize 19}$,
H.~Bachacou$^\textrm{\scriptsize 138}$,
K.~Bachas$^\textrm{\scriptsize 76a,76b}$,
M.~Backes$^\textrm{\scriptsize 122}$,
M.~Backhaus$^\textrm{\scriptsize 32}$,
P.~Bagnaia$^\textrm{\scriptsize 134a,134b}$,
M.~Bahmani$^\textrm{\scriptsize 42}$,
H.~Bahrasemani$^\textrm{\scriptsize 144}$,
J.T.~Baines$^\textrm{\scriptsize 133}$,
M.~Bajic$^\textrm{\scriptsize 39}$,
O.K.~Baker$^\textrm{\scriptsize 179}$,
E.M.~Baldin$^\textrm{\scriptsize 111}$$^{,c}$,
P.~Balek$^\textrm{\scriptsize 175}$,
F.~Balli$^\textrm{\scriptsize 138}$,
W.K.~Balunas$^\textrm{\scriptsize 124}$,
E.~Banas$^\textrm{\scriptsize 42}$,
A.~Bandyopadhyay$^\textrm{\scriptsize 23}$,
Sw.~Banerjee$^\textrm{\scriptsize 176}$$^{,e}$,
A.A.E.~Bannoura$^\textrm{\scriptsize 178}$,
L.~Barak$^\textrm{\scriptsize 32}$,
E.L.~Barberio$^\textrm{\scriptsize 91}$,
D.~Barberis$^\textrm{\scriptsize 53a,53b}$,
M.~Barbero$^\textrm{\scriptsize 88}$,
T.~Barillari$^\textrm{\scriptsize 103}$,
M-S~Barisits$^\textrm{\scriptsize 32}$,
J.T.~Barkeloo$^\textrm{\scriptsize 118}$,
T.~Barklow$^\textrm{\scriptsize 145}$,
N.~Barlow$^\textrm{\scriptsize 30}$,
S.L.~Barnes$^\textrm{\scriptsize 36c}$,
B.M.~Barnett$^\textrm{\scriptsize 133}$,
R.M.~Barnett$^\textrm{\scriptsize 16}$,
Z.~Barnovska-Blenessy$^\textrm{\scriptsize 36a}$,
A.~Baroncelli$^\textrm{\scriptsize 136a}$,
G.~Barone$^\textrm{\scriptsize 25}$,
A.J.~Barr$^\textrm{\scriptsize 122}$,
L.~Barranco~Navarro$^\textrm{\scriptsize 170}$,
F.~Barreiro$^\textrm{\scriptsize 85}$,
J.~Barreiro~Guimar\~{a}es~da~Costa$^\textrm{\scriptsize 35a}$,
R.~Bartoldus$^\textrm{\scriptsize 145}$,
A.E.~Barton$^\textrm{\scriptsize 75}$,
P.~Bartos$^\textrm{\scriptsize 146a}$,
A.~Basalaev$^\textrm{\scriptsize 125}$,
A.~Bassalat$^\textrm{\scriptsize 119}$$^{,f}$,
R.L.~Bates$^\textrm{\scriptsize 56}$,
S.J.~Batista$^\textrm{\scriptsize 161}$,
J.R.~Batley$^\textrm{\scriptsize 30}$,
M.~Battaglia$^\textrm{\scriptsize 139}$,
M.~Bauce$^\textrm{\scriptsize 134a,134b}$,
F.~Bauer$^\textrm{\scriptsize 138}$,
H.S.~Bawa$^\textrm{\scriptsize 145}$$^{,g}$,
J.B.~Beacham$^\textrm{\scriptsize 113}$,
M.D.~Beattie$^\textrm{\scriptsize 75}$,
T.~Beau$^\textrm{\scriptsize 83}$,
P.H.~Beauchemin$^\textrm{\scriptsize 165}$,
P.~Bechtle$^\textrm{\scriptsize 23}$,
H.P.~Beck$^\textrm{\scriptsize 18}$$^{,h}$,
H.C.~Beck$^\textrm{\scriptsize 57}$,
K.~Becker$^\textrm{\scriptsize 122}$,
M.~Becker$^\textrm{\scriptsize 86}$,
M.~Beckingham$^\textrm{\scriptsize 173}$,
C.~Becot$^\textrm{\scriptsize 112}$,
A.J.~Beddall$^\textrm{\scriptsize 20e}$,
A.~Beddall$^\textrm{\scriptsize 20b}$,
V.A.~Bednyakov$^\textrm{\scriptsize 68}$,
M.~Bedognetti$^\textrm{\scriptsize 109}$,
C.P.~Bee$^\textrm{\scriptsize 150}$,
T.A.~Beermann$^\textrm{\scriptsize 32}$,
M.~Begalli$^\textrm{\scriptsize 26a}$,
M.~Begel$^\textrm{\scriptsize 27}$,
J.K.~Behr$^\textrm{\scriptsize 45}$,
A.S.~Bell$^\textrm{\scriptsize 81}$,
G.~Bella$^\textrm{\scriptsize 155}$,
L.~Bellagamba$^\textrm{\scriptsize 22a}$,
A.~Bellerive$^\textrm{\scriptsize 31}$,
M.~Bellomo$^\textrm{\scriptsize 154}$,
K.~Belotskiy$^\textrm{\scriptsize 100}$,
O.~Beltramello$^\textrm{\scriptsize 32}$,
N.L.~Belyaev$^\textrm{\scriptsize 100}$,
O.~Benary$^\textrm{\scriptsize 155}$$^{,*}$,
D.~Benchekroun$^\textrm{\scriptsize 137a}$,
M.~Bender$^\textrm{\scriptsize 102}$,
K.~Bendtz$^\textrm{\scriptsize 148a,148b}$,
N.~Benekos$^\textrm{\scriptsize 10}$,
Y.~Benhammou$^\textrm{\scriptsize 155}$,
E.~Benhar~Noccioli$^\textrm{\scriptsize 179}$,
J.~Benitez$^\textrm{\scriptsize 66}$,
D.P.~Benjamin$^\textrm{\scriptsize 48}$,
M.~Benoit$^\textrm{\scriptsize 52}$,
J.R.~Bensinger$^\textrm{\scriptsize 25}$,
S.~Bentvelsen$^\textrm{\scriptsize 109}$,
L.~Beresford$^\textrm{\scriptsize 122}$,
M.~Beretta$^\textrm{\scriptsize 50}$,
D.~Berge$^\textrm{\scriptsize 109}$,
E.~Bergeaas~Kuutmann$^\textrm{\scriptsize 168}$,
N.~Berger$^\textrm{\scriptsize 5}$,
J.~Beringer$^\textrm{\scriptsize 16}$,
S.~Berlendis$^\textrm{\scriptsize 58}$,
N.R.~Bernard$^\textrm{\scriptsize 89}$,
G.~Bernardi$^\textrm{\scriptsize 83}$,
C.~Bernius$^\textrm{\scriptsize 145}$,
F.U.~Bernlochner$^\textrm{\scriptsize 23}$,
T.~Berry$^\textrm{\scriptsize 80}$,
P.~Berta$^\textrm{\scriptsize 86}$,
C.~Bertella$^\textrm{\scriptsize 35a}$,
G.~Bertoli$^\textrm{\scriptsize 148a,148b}$,
F.~Bertolucci$^\textrm{\scriptsize 126a,126b}$,
I.A.~Bertram$^\textrm{\scriptsize 75}$,
C.~Bertsche$^\textrm{\scriptsize 45}$,
D.~Bertsche$^\textrm{\scriptsize 115}$,
G.J.~Besjes$^\textrm{\scriptsize 39}$,
O.~Bessidskaia~Bylund$^\textrm{\scriptsize 148a,148b}$,
M.~Bessner$^\textrm{\scriptsize 45}$,
N.~Besson$^\textrm{\scriptsize 138}$,
C.~Betancourt$^\textrm{\scriptsize 51}$,
A.~Bethani$^\textrm{\scriptsize 87}$,
S.~Bethke$^\textrm{\scriptsize 103}$,
A.J.~Bevan$^\textrm{\scriptsize 79}$,
J.~Beyer$^\textrm{\scriptsize 103}$,
R.M.~Bianchi$^\textrm{\scriptsize 127}$,
O.~Biebel$^\textrm{\scriptsize 102}$,
D.~Biedermann$^\textrm{\scriptsize 17}$,
R.~Bielski$^\textrm{\scriptsize 87}$,
K.~Bierwagen$^\textrm{\scriptsize 86}$,
N.V.~Biesuz$^\textrm{\scriptsize 126a,126b}$,
M.~Biglietti$^\textrm{\scriptsize 136a}$,
T.R.V.~Billoud$^\textrm{\scriptsize 97}$,
H.~Bilokon$^\textrm{\scriptsize 50}$,
M.~Bindi$^\textrm{\scriptsize 57}$,
A.~Bingul$^\textrm{\scriptsize 20b}$,
C.~Bini$^\textrm{\scriptsize 134a,134b}$,
S.~Biondi$^\textrm{\scriptsize 22a,22b}$,
T.~Bisanz$^\textrm{\scriptsize 57}$,
C.~Bittrich$^\textrm{\scriptsize 47}$,
D.M.~Bjergaard$^\textrm{\scriptsize 48}$,
C.W.~Black$^\textrm{\scriptsize 152}$,
J.E.~Black$^\textrm{\scriptsize 145}$,
K.M.~Black$^\textrm{\scriptsize 24}$,
R.E.~Blair$^\textrm{\scriptsize 6}$,
T.~Blazek$^\textrm{\scriptsize 146a}$,
I.~Bloch$^\textrm{\scriptsize 45}$,
C.~Blocker$^\textrm{\scriptsize 25}$,
A.~Blue$^\textrm{\scriptsize 56}$,
W.~Blum$^\textrm{\scriptsize 86}$$^{,*}$,
U.~Blumenschein$^\textrm{\scriptsize 79}$,
S.~Blunier$^\textrm{\scriptsize 34a}$,
G.J.~Bobbink$^\textrm{\scriptsize 109}$,
V.S.~Bobrovnikov$^\textrm{\scriptsize 111}$$^{,c}$,
S.S.~Bocchetta$^\textrm{\scriptsize 84}$,
A.~Bocci$^\textrm{\scriptsize 48}$,
C.~Bock$^\textrm{\scriptsize 102}$,
M.~Boehler$^\textrm{\scriptsize 51}$,
D.~Boerner$^\textrm{\scriptsize 178}$,
D.~Bogavac$^\textrm{\scriptsize 102}$,
A.G.~Bogdanchikov$^\textrm{\scriptsize 111}$,
C.~Bohm$^\textrm{\scriptsize 148a}$,
V.~Boisvert$^\textrm{\scriptsize 80}$,
P.~Bokan$^\textrm{\scriptsize 168}$$^{,i}$,
T.~Bold$^\textrm{\scriptsize 41a}$,
A.S.~Boldyrev$^\textrm{\scriptsize 101}$,
A.E.~Bolz$^\textrm{\scriptsize 60b}$,
M.~Bomben$^\textrm{\scriptsize 83}$,
M.~Bona$^\textrm{\scriptsize 79}$,
M.~Boonekamp$^\textrm{\scriptsize 138}$,
A.~Borisov$^\textrm{\scriptsize 132}$,
G.~Borissov$^\textrm{\scriptsize 75}$,
J.~Bortfeldt$^\textrm{\scriptsize 32}$,
D.~Bortoletto$^\textrm{\scriptsize 122}$,
V.~Bortolotto$^\textrm{\scriptsize 62a}$,
D.~Boscherini$^\textrm{\scriptsize 22a}$,
M.~Bosman$^\textrm{\scriptsize 13}$,
J.D.~Bossio~Sola$^\textrm{\scriptsize 29}$,
J.~Boudreau$^\textrm{\scriptsize 127}$,
J.~Bouffard$^\textrm{\scriptsize 2}$,
E.V.~Bouhova-Thacker$^\textrm{\scriptsize 75}$,
D.~Boumediene$^\textrm{\scriptsize 37}$,
C.~Bourdarios$^\textrm{\scriptsize 119}$,
S.K.~Boutle$^\textrm{\scriptsize 56}$,
A.~Boveia$^\textrm{\scriptsize 113}$,
J.~Boyd$^\textrm{\scriptsize 32}$,
I.R.~Boyko$^\textrm{\scriptsize 68}$,
J.~Bracinik$^\textrm{\scriptsize 19}$,
A.~Brandt$^\textrm{\scriptsize 8}$,
G.~Brandt$^\textrm{\scriptsize 57}$,
O.~Brandt$^\textrm{\scriptsize 60a}$,
U.~Bratzler$^\textrm{\scriptsize 158}$,
B.~Brau$^\textrm{\scriptsize 89}$,
J.E.~Brau$^\textrm{\scriptsize 118}$,
W.D.~Breaden~Madden$^\textrm{\scriptsize 56}$,
K.~Brendlinger$^\textrm{\scriptsize 45}$,
A.J.~Brennan$^\textrm{\scriptsize 91}$,
L.~Brenner$^\textrm{\scriptsize 109}$,
R.~Brenner$^\textrm{\scriptsize 168}$,
S.~Bressler$^\textrm{\scriptsize 175}$,
D.L.~Briglin$^\textrm{\scriptsize 19}$,
T.M.~Bristow$^\textrm{\scriptsize 49}$,
D.~Britton$^\textrm{\scriptsize 56}$,
D.~Britzger$^\textrm{\scriptsize 45}$,
F.M.~Brochu$^\textrm{\scriptsize 30}$,
I.~Brock$^\textrm{\scriptsize 23}$,
R.~Brock$^\textrm{\scriptsize 93}$,
G.~Brooijmans$^\textrm{\scriptsize 38}$,
T.~Brooks$^\textrm{\scriptsize 80}$,
W.K.~Brooks$^\textrm{\scriptsize 34b}$,
J.~Brosamer$^\textrm{\scriptsize 16}$,
E.~Brost$^\textrm{\scriptsize 110}$,
J.H~Broughton$^\textrm{\scriptsize 19}$,
P.A.~Bruckman~de~Renstrom$^\textrm{\scriptsize 42}$,
D.~Bruncko$^\textrm{\scriptsize 146b}$,
A.~Bruni$^\textrm{\scriptsize 22a}$,
G.~Bruni$^\textrm{\scriptsize 22a}$,
L.S.~Bruni$^\textrm{\scriptsize 109}$,
BH~Brunt$^\textrm{\scriptsize 30}$,
M.~Bruschi$^\textrm{\scriptsize 22a}$,
N.~Bruscino$^\textrm{\scriptsize 23}$,
P.~Bryant$^\textrm{\scriptsize 33}$,
L.~Bryngemark$^\textrm{\scriptsize 45}$,
T.~Buanes$^\textrm{\scriptsize 15}$,
Q.~Buat$^\textrm{\scriptsize 144}$,
P.~Buchholz$^\textrm{\scriptsize 143}$,
A.G.~Buckley$^\textrm{\scriptsize 56}$,
I.A.~Budagov$^\textrm{\scriptsize 68}$,
F.~Buehrer$^\textrm{\scriptsize 51}$,
M.K.~Bugge$^\textrm{\scriptsize 121}$,
O.~Bulekov$^\textrm{\scriptsize 100}$,
D.~Bullock$^\textrm{\scriptsize 8}$,
T.J.~Burch$^\textrm{\scriptsize 110}$,
S.~Burdin$^\textrm{\scriptsize 77}$,
C.D.~Burgard$^\textrm{\scriptsize 51}$,
A.M.~Burger$^\textrm{\scriptsize 5}$,
B.~Burghgrave$^\textrm{\scriptsize 110}$,
K.~Burka$^\textrm{\scriptsize 42}$,
S.~Burke$^\textrm{\scriptsize 133}$,
I.~Burmeister$^\textrm{\scriptsize 46}$,
J.T.P.~Burr$^\textrm{\scriptsize 122}$,
E.~Busato$^\textrm{\scriptsize 37}$,
D.~B\"uscher$^\textrm{\scriptsize 51}$,
V.~B\"uscher$^\textrm{\scriptsize 86}$,
P.~Bussey$^\textrm{\scriptsize 56}$,
J.M.~Butler$^\textrm{\scriptsize 24}$,
C.M.~Buttar$^\textrm{\scriptsize 56}$,
J.M.~Butterworth$^\textrm{\scriptsize 81}$,
P.~Butti$^\textrm{\scriptsize 32}$,
W.~Buttinger$^\textrm{\scriptsize 27}$,
A.~Buzatu$^\textrm{\scriptsize 35c}$,
A.R.~Buzykaev$^\textrm{\scriptsize 111}$$^{,c}$,
S.~Cabrera~Urb\'an$^\textrm{\scriptsize 170}$,
D.~Caforio$^\textrm{\scriptsize 130}$,
V.M.~Cairo$^\textrm{\scriptsize 40a,40b}$,
O.~Cakir$^\textrm{\scriptsize 4a}$,
N.~Calace$^\textrm{\scriptsize 52}$,
P.~Calafiura$^\textrm{\scriptsize 16}$,
A.~Calandri$^\textrm{\scriptsize 88}$,
G.~Calderini$^\textrm{\scriptsize 83}$,
P.~Calfayan$^\textrm{\scriptsize 64}$,
G.~Callea$^\textrm{\scriptsize 40a,40b}$,
L.P.~Caloba$^\textrm{\scriptsize 26a}$,
S.~Calvente~Lopez$^\textrm{\scriptsize 85}$,
D.~Calvet$^\textrm{\scriptsize 37}$,
S.~Calvet$^\textrm{\scriptsize 37}$,
T.P.~Calvet$^\textrm{\scriptsize 88}$,
R.~Camacho~Toro$^\textrm{\scriptsize 33}$,
S.~Camarda$^\textrm{\scriptsize 32}$,
P.~Camarri$^\textrm{\scriptsize 135a,135b}$,
D.~Cameron$^\textrm{\scriptsize 121}$,
R.~Caminal~Armadans$^\textrm{\scriptsize 169}$,
C.~Camincher$^\textrm{\scriptsize 58}$,
S.~Campana$^\textrm{\scriptsize 32}$,
M.~Campanelli$^\textrm{\scriptsize 81}$,
A.~Camplani$^\textrm{\scriptsize 94a,94b}$,
A.~Campoverde$^\textrm{\scriptsize 143}$,
V.~Canale$^\textrm{\scriptsize 106a,106b}$,
M.~Cano~Bret$^\textrm{\scriptsize 36c}$,
J.~Cantero$^\textrm{\scriptsize 116}$,
T.~Cao$^\textrm{\scriptsize 155}$,
M.D.M.~Capeans~Garrido$^\textrm{\scriptsize 32}$,
I.~Caprini$^\textrm{\scriptsize 28b}$,
M.~Caprini$^\textrm{\scriptsize 28b}$,
M.~Capua$^\textrm{\scriptsize 40a,40b}$,
R.M.~Carbone$^\textrm{\scriptsize 38}$,
R.~Cardarelli$^\textrm{\scriptsize 135a}$,
F.~Cardillo$^\textrm{\scriptsize 51}$,
I.~Carli$^\textrm{\scriptsize 131}$,
T.~Carli$^\textrm{\scriptsize 32}$,
G.~Carlino$^\textrm{\scriptsize 106a}$,
B.T.~Carlson$^\textrm{\scriptsize 127}$,
L.~Carminati$^\textrm{\scriptsize 94a,94b}$,
R.M.D.~Carney$^\textrm{\scriptsize 148a,148b}$,
S.~Caron$^\textrm{\scriptsize 108}$,
E.~Carquin$^\textrm{\scriptsize 34b}$,
S.~Carr\'a$^\textrm{\scriptsize 94a,94b}$,
G.D.~Carrillo-Montoya$^\textrm{\scriptsize 32}$,
D.~Casadei$^\textrm{\scriptsize 19}$,
M.P.~Casado$^\textrm{\scriptsize 13}$$^{,j}$,
M.~Casolino$^\textrm{\scriptsize 13}$,
D.W.~Casper$^\textrm{\scriptsize 166}$,
R.~Castelijn$^\textrm{\scriptsize 109}$,
V.~Castillo~Gimenez$^\textrm{\scriptsize 170}$,
N.F.~Castro$^\textrm{\scriptsize 128a}$$^{,k}$,
A.~Catinaccio$^\textrm{\scriptsize 32}$,
J.R.~Catmore$^\textrm{\scriptsize 121}$,
A.~Cattai$^\textrm{\scriptsize 32}$,
J.~Caudron$^\textrm{\scriptsize 23}$,
V.~Cavaliere$^\textrm{\scriptsize 169}$,
E.~Cavallaro$^\textrm{\scriptsize 13}$,
D.~Cavalli$^\textrm{\scriptsize 94a}$,
M.~Cavalli-Sforza$^\textrm{\scriptsize 13}$,
V.~Cavasinni$^\textrm{\scriptsize 126a,126b}$,
E.~Celebi$^\textrm{\scriptsize 20d}$,
F.~Ceradini$^\textrm{\scriptsize 136a,136b}$,
L.~Cerda~Alberich$^\textrm{\scriptsize 170}$,
A.S.~Cerqueira$^\textrm{\scriptsize 26b}$,
A.~Cerri$^\textrm{\scriptsize 151}$,
L.~Cerrito$^\textrm{\scriptsize 135a,135b}$,
F.~Cerutti$^\textrm{\scriptsize 16}$,
A.~Cervelli$^\textrm{\scriptsize 18}$,
S.A.~Cetin$^\textrm{\scriptsize 20d}$,
A.~Chafaq$^\textrm{\scriptsize 137a}$,
D.~Chakraborty$^\textrm{\scriptsize 110}$,
S.K.~Chan$^\textrm{\scriptsize 59}$,
W.S.~Chan$^\textrm{\scriptsize 109}$,
Y.L.~Chan$^\textrm{\scriptsize 62a}$,
P.~Chang$^\textrm{\scriptsize 169}$,
J.D.~Chapman$^\textrm{\scriptsize 30}$,
D.G.~Charlton$^\textrm{\scriptsize 19}$,
C.C.~Chau$^\textrm{\scriptsize 31}$,
C.A.~Chavez~Barajas$^\textrm{\scriptsize 151}$,
S.~Che$^\textrm{\scriptsize 113}$,
S.~Cheatham$^\textrm{\scriptsize 167a,167c}$,
A.~Chegwidden$^\textrm{\scriptsize 93}$,
S.~Chekanov$^\textrm{\scriptsize 6}$,
S.V.~Chekulaev$^\textrm{\scriptsize 163a}$,
G.A.~Chelkov$^\textrm{\scriptsize 68}$$^{,l}$,
M.A.~Chelstowska$^\textrm{\scriptsize 32}$,
C.~Chen$^\textrm{\scriptsize 67}$,
H.~Chen$^\textrm{\scriptsize 27}$,
J.~Chen$^\textrm{\scriptsize 36a}$,
S.~Chen$^\textrm{\scriptsize 35b}$,
S.~Chen$^\textrm{\scriptsize 157}$,
X.~Chen$^\textrm{\scriptsize 35c}$$^{,m}$,
Y.~Chen$^\textrm{\scriptsize 70}$,
H.C.~Cheng$^\textrm{\scriptsize 92}$,
H.J.~Cheng$^\textrm{\scriptsize 35a,35d}$,
A.~Cheplakov$^\textrm{\scriptsize 68}$,
E.~Cheremushkina$^\textrm{\scriptsize 132}$,
R.~Cherkaoui~El~Moursli$^\textrm{\scriptsize 137e}$,
E.~Cheu$^\textrm{\scriptsize 7}$,
K.~Cheung$^\textrm{\scriptsize 63}$,
L.~Chevalier$^\textrm{\scriptsize 138}$,
V.~Chiarella$^\textrm{\scriptsize 50}$,
G.~Chiarelli$^\textrm{\scriptsize 126a,126b}$,
G.~Chiodini$^\textrm{\scriptsize 76a}$,
A.S.~Chisholm$^\textrm{\scriptsize 32}$,
A.~Chitan$^\textrm{\scriptsize 28b}$,
Y.H.~Chiu$^\textrm{\scriptsize 172}$,
M.V.~Chizhov$^\textrm{\scriptsize 68}$,
K.~Choi$^\textrm{\scriptsize 64}$,
A.R.~Chomont$^\textrm{\scriptsize 37}$,
S.~Chouridou$^\textrm{\scriptsize 156}$,
Y.S.~Chow$^\textrm{\scriptsize 62a}$,
V.~Christodoulou$^\textrm{\scriptsize 81}$,
M.C.~Chu$^\textrm{\scriptsize 62a}$,
J.~Chudoba$^\textrm{\scriptsize 129}$,
A.J.~Chuinard$^\textrm{\scriptsize 90}$,
J.J.~Chwastowski$^\textrm{\scriptsize 42}$,
L.~Chytka$^\textrm{\scriptsize 117}$,
A.K.~Ciftci$^\textrm{\scriptsize 4a}$,
D.~Cinca$^\textrm{\scriptsize 46}$,
V.~Cindro$^\textrm{\scriptsize 78}$,
I.A.~Cioara$^\textrm{\scriptsize 23}$,
C.~Ciocca$^\textrm{\scriptsize 22a,22b}$,
A.~Ciocio$^\textrm{\scriptsize 16}$,
F.~Cirotto$^\textrm{\scriptsize 106a,106b}$,
Z.H.~Citron$^\textrm{\scriptsize 175}$,
M.~Citterio$^\textrm{\scriptsize 94a}$,
M.~Ciubancan$^\textrm{\scriptsize 28b}$,
A.~Clark$^\textrm{\scriptsize 52}$,
B.L.~Clark$^\textrm{\scriptsize 59}$,
M.R.~Clark$^\textrm{\scriptsize 38}$,
P.J.~Clark$^\textrm{\scriptsize 49}$,
R.N.~Clarke$^\textrm{\scriptsize 16}$,
C.~Clement$^\textrm{\scriptsize 148a,148b}$,
Y.~Coadou$^\textrm{\scriptsize 88}$,
M.~Cobal$^\textrm{\scriptsize 167a,167c}$,
A.~Coccaro$^\textrm{\scriptsize 52}$,
J.~Cochran$^\textrm{\scriptsize 67}$,
L.~Colasurdo$^\textrm{\scriptsize 108}$,
B.~Cole$^\textrm{\scriptsize 38}$,
A.P.~Colijn$^\textrm{\scriptsize 109}$,
J.~Collot$^\textrm{\scriptsize 58}$,
T.~Colombo$^\textrm{\scriptsize 166}$,
P.~Conde~Mui\~no$^\textrm{\scriptsize 128a,128b}$,
E.~Coniavitis$^\textrm{\scriptsize 51}$,
S.H.~Connell$^\textrm{\scriptsize 147b}$,
I.A.~Connelly$^\textrm{\scriptsize 87}$,
S.~Constantinescu$^\textrm{\scriptsize 28b}$,
G.~Conti$^\textrm{\scriptsize 32}$,
F.~Conventi$^\textrm{\scriptsize 106a}$$^{,n}$,
M.~Cooke$^\textrm{\scriptsize 16}$,
A.M.~Cooper-Sarkar$^\textrm{\scriptsize 122}$,
F.~Cormier$^\textrm{\scriptsize 171}$,
K.J.R.~Cormier$^\textrm{\scriptsize 161}$,
M.~Corradi$^\textrm{\scriptsize 134a,134b}$,
F.~Corriveau$^\textrm{\scriptsize 90}$$^{,o}$,
A.~Cortes-Gonzalez$^\textrm{\scriptsize 32}$,
G.~Cortiana$^\textrm{\scriptsize 103}$,
G.~Costa$^\textrm{\scriptsize 94a}$,
M.J.~Costa$^\textrm{\scriptsize 170}$,
D.~Costanzo$^\textrm{\scriptsize 141}$,
G.~Cottin$^\textrm{\scriptsize 30}$,
G.~Cowan$^\textrm{\scriptsize 80}$,
B.E.~Cox$^\textrm{\scriptsize 87}$,
K.~Cranmer$^\textrm{\scriptsize 112}$,
S.J.~Crawley$^\textrm{\scriptsize 56}$,
R.A.~Creager$^\textrm{\scriptsize 124}$,
G.~Cree$^\textrm{\scriptsize 31}$,
S.~Cr\'ep\'e-Renaudin$^\textrm{\scriptsize 58}$,
F.~Crescioli$^\textrm{\scriptsize 83}$,
W.A.~Cribbs$^\textrm{\scriptsize 148a,148b}$,
M.~Cristinziani$^\textrm{\scriptsize 23}$,
V.~Croft$^\textrm{\scriptsize 108}$,
G.~Crosetti$^\textrm{\scriptsize 40a,40b}$,
A.~Cueto$^\textrm{\scriptsize 85}$,
T.~Cuhadar~Donszelmann$^\textrm{\scriptsize 141}$,
A.R.~Cukierman$^\textrm{\scriptsize 145}$,
J.~Cummings$^\textrm{\scriptsize 179}$,
M.~Curatolo$^\textrm{\scriptsize 50}$,
J.~C\'uth$^\textrm{\scriptsize 86}$,
S.~Czekierda$^\textrm{\scriptsize 42}$,
P.~Czodrowski$^\textrm{\scriptsize 32}$,
G.~D'amen$^\textrm{\scriptsize 22a,22b}$,
S.~D'Auria$^\textrm{\scriptsize 56}$,
L.~D'eramo$^\textrm{\scriptsize 83}$,
M.~D'Onofrio$^\textrm{\scriptsize 77}$,
M.J.~Da~Cunha~Sargedas~De~Sousa$^\textrm{\scriptsize 128a,128b}$,
C.~Da~Via$^\textrm{\scriptsize 87}$,
W.~Dabrowski$^\textrm{\scriptsize 41a}$,
T.~Dado$^\textrm{\scriptsize 146a}$,
T.~Dai$^\textrm{\scriptsize 92}$,
O.~Dale$^\textrm{\scriptsize 15}$,
F.~Dallaire$^\textrm{\scriptsize 97}$,
C.~Dallapiccola$^\textrm{\scriptsize 89}$,
M.~Dam$^\textrm{\scriptsize 39}$,
J.R.~Dandoy$^\textrm{\scriptsize 124}$,
M.F.~Daneri$^\textrm{\scriptsize 29}$,
N.P.~Dang$^\textrm{\scriptsize 176}$,
A.C.~Daniells$^\textrm{\scriptsize 19}$,
N.S.~Dann$^\textrm{\scriptsize 87}$,
M.~Danninger$^\textrm{\scriptsize 171}$,
M.~Dano~Hoffmann$^\textrm{\scriptsize 138}$,
V.~Dao$^\textrm{\scriptsize 150}$,
G.~Darbo$^\textrm{\scriptsize 53a}$,
S.~Darmora$^\textrm{\scriptsize 8}$,
J.~Dassoulas$^\textrm{\scriptsize 3}$,
A.~Dattagupta$^\textrm{\scriptsize 118}$,
T.~Daubney$^\textrm{\scriptsize 45}$,
W.~Davey$^\textrm{\scriptsize 23}$,
C.~David$^\textrm{\scriptsize 45}$,
T.~Davidek$^\textrm{\scriptsize 131}$,
D.R.~Davis$^\textrm{\scriptsize 48}$,
P.~Davison$^\textrm{\scriptsize 81}$,
E.~Dawe$^\textrm{\scriptsize 91}$,
I.~Dawson$^\textrm{\scriptsize 141}$,
K.~De$^\textrm{\scriptsize 8}$,
R.~de~Asmundis$^\textrm{\scriptsize 106a}$,
A.~De~Benedetti$^\textrm{\scriptsize 115}$,
S.~De~Castro$^\textrm{\scriptsize 22a,22b}$,
S.~De~Cecco$^\textrm{\scriptsize 83}$,
N.~De~Groot$^\textrm{\scriptsize 108}$,
P.~de~Jong$^\textrm{\scriptsize 109}$,
H.~De~la~Torre$^\textrm{\scriptsize 93}$,
F.~De~Lorenzi$^\textrm{\scriptsize 67}$,
A.~De~Maria$^\textrm{\scriptsize 57}$,
D.~De~Pedis$^\textrm{\scriptsize 134a}$,
A.~De~Salvo$^\textrm{\scriptsize 134a}$,
U.~De~Sanctis$^\textrm{\scriptsize 135a,135b}$,
A.~De~Santo$^\textrm{\scriptsize 151}$,
K.~De~Vasconcelos~Corga$^\textrm{\scriptsize 88}$,
J.B.~De~Vivie~De~Regie$^\textrm{\scriptsize 119}$,
W.J.~Dearnaley$^\textrm{\scriptsize 75}$,
R.~Debbe$^\textrm{\scriptsize 27}$,
C.~Debenedetti$^\textrm{\scriptsize 139}$,
D.V.~Dedovich$^\textrm{\scriptsize 68}$,
N.~Dehghanian$^\textrm{\scriptsize 3}$,
I.~Deigaard$^\textrm{\scriptsize 109}$,
M.~Del~Gaudio$^\textrm{\scriptsize 40a,40b}$,
J.~Del~Peso$^\textrm{\scriptsize 85}$,
D.~Delgove$^\textrm{\scriptsize 119}$,
F.~Deliot$^\textrm{\scriptsize 138}$,
C.M.~Delitzsch$^\textrm{\scriptsize 7}$,
A.~Dell'Acqua$^\textrm{\scriptsize 32}$,
L.~Dell'Asta$^\textrm{\scriptsize 24}$,
M.~Dell'Orso$^\textrm{\scriptsize 126a,126b}$,
M.~Della~Pietra$^\textrm{\scriptsize 106a,106b}$,
D.~della~Volpe$^\textrm{\scriptsize 52}$,
M.~Delmastro$^\textrm{\scriptsize 5}$,
C.~Delporte$^\textrm{\scriptsize 119}$,
P.A.~Delsart$^\textrm{\scriptsize 58}$,
D.A.~DeMarco$^\textrm{\scriptsize 161}$,
S.~Demers$^\textrm{\scriptsize 179}$,
M.~Demichev$^\textrm{\scriptsize 68}$,
A.~Demilly$^\textrm{\scriptsize 83}$,
S.P.~Denisov$^\textrm{\scriptsize 132}$,
D.~Denysiuk$^\textrm{\scriptsize 138}$,
D.~Derendarz$^\textrm{\scriptsize 42}$,
J.E.~Derkaoui$^\textrm{\scriptsize 137d}$,
F.~Derue$^\textrm{\scriptsize 83}$,
P.~Dervan$^\textrm{\scriptsize 77}$,
K.~Desch$^\textrm{\scriptsize 23}$,
C.~Deterre$^\textrm{\scriptsize 45}$,
K.~Dette$^\textrm{\scriptsize 46}$,
M.R.~Devesa$^\textrm{\scriptsize 29}$,
P.O.~Deviveiros$^\textrm{\scriptsize 32}$,
A.~Dewhurst$^\textrm{\scriptsize 133}$,
S.~Dhaliwal$^\textrm{\scriptsize 25}$,
F.A.~Di~Bello$^\textrm{\scriptsize 52}$,
A.~Di~Ciaccio$^\textrm{\scriptsize 135a,135b}$,
L.~Di~Ciaccio$^\textrm{\scriptsize 5}$,
W.K.~Di~Clemente$^\textrm{\scriptsize 124}$,
C.~Di~Donato$^\textrm{\scriptsize 106a,106b}$,
A.~Di~Girolamo$^\textrm{\scriptsize 32}$,
B.~Di~Girolamo$^\textrm{\scriptsize 32}$,
B.~Di~Micco$^\textrm{\scriptsize 136a,136b}$,
R.~Di~Nardo$^\textrm{\scriptsize 32}$,
K.F.~Di~Petrillo$^\textrm{\scriptsize 59}$,
A.~Di~Simone$^\textrm{\scriptsize 51}$,
R.~Di~Sipio$^\textrm{\scriptsize 161}$,
D.~Di~Valentino$^\textrm{\scriptsize 31}$,
C.~Diaconu$^\textrm{\scriptsize 88}$,
M.~Diamond$^\textrm{\scriptsize 161}$,
F.A.~Dias$^\textrm{\scriptsize 39}$,
M.A.~Diaz$^\textrm{\scriptsize 34a}$,
E.B.~Diehl$^\textrm{\scriptsize 92}$,
J.~Dietrich$^\textrm{\scriptsize 17}$,
S.~D\'iez~Cornell$^\textrm{\scriptsize 45}$,
A.~Dimitrievska$^\textrm{\scriptsize 14}$,
J.~Dingfelder$^\textrm{\scriptsize 23}$,
P.~Dita$^\textrm{\scriptsize 28b}$,
S.~Dita$^\textrm{\scriptsize 28b}$,
F.~Dittus$^\textrm{\scriptsize 32}$,
F.~Djama$^\textrm{\scriptsize 88}$,
T.~Djobava$^\textrm{\scriptsize 54b}$,
J.I.~Djuvsland$^\textrm{\scriptsize 60a}$,
M.A.B.~do~Vale$^\textrm{\scriptsize 26c}$,
D.~Dobos$^\textrm{\scriptsize 32}$,
M.~Dobre$^\textrm{\scriptsize 28b}$,
C.~Doglioni$^\textrm{\scriptsize 84}$,
J.~Dolejsi$^\textrm{\scriptsize 131}$,
Z.~Dolezal$^\textrm{\scriptsize 131}$,
M.~Donadelli$^\textrm{\scriptsize 26d}$,
S.~Donati$^\textrm{\scriptsize 126a,126b}$,
P.~Dondero$^\textrm{\scriptsize 123a,123b}$,
J.~Donini$^\textrm{\scriptsize 37}$,
J.~Dopke$^\textrm{\scriptsize 133}$,
A.~Doria$^\textrm{\scriptsize 106a}$,
M.T.~Dova$^\textrm{\scriptsize 74}$,
A.T.~Doyle$^\textrm{\scriptsize 56}$,
E.~Drechsler$^\textrm{\scriptsize 57}$,
M.~Dris$^\textrm{\scriptsize 10}$,
Y.~Du$^\textrm{\scriptsize 36b}$,
J.~Duarte-Campderros$^\textrm{\scriptsize 155}$,
A.~Dubreuil$^\textrm{\scriptsize 52}$,
E.~Duchovni$^\textrm{\scriptsize 175}$,
G.~Duckeck$^\textrm{\scriptsize 102}$,
A.~Ducourthial$^\textrm{\scriptsize 83}$,
O.A.~Ducu$^\textrm{\scriptsize 97}$$^{,p}$,
D.~Duda$^\textrm{\scriptsize 109}$,
A.~Dudarev$^\textrm{\scriptsize 32}$,
A.Chr.~Dudder$^\textrm{\scriptsize 86}$,
E.M.~Duffield$^\textrm{\scriptsize 16}$,
L.~Duflot$^\textrm{\scriptsize 119}$,
M.~D\"uhrssen$^\textrm{\scriptsize 32}$,
M.~Dumancic$^\textrm{\scriptsize 175}$,
A.E.~Dumitriu$^\textrm{\scriptsize 28b}$,
A.K.~Duncan$^\textrm{\scriptsize 56}$,
M.~Dunford$^\textrm{\scriptsize 60a}$,
H.~Duran~Yildiz$^\textrm{\scriptsize 4a}$,
M.~D\"uren$^\textrm{\scriptsize 55}$,
A.~Durglishvili$^\textrm{\scriptsize 54b}$,
D.~Duschinger$^\textrm{\scriptsize 47}$,
B.~Dutta$^\textrm{\scriptsize 45}$,
D.~Duvnjak$^\textrm{\scriptsize 1}$,
M.~Dyndal$^\textrm{\scriptsize 45}$,
B.S.~Dziedzic$^\textrm{\scriptsize 42}$,
C.~Eckardt$^\textrm{\scriptsize 45}$,
K.M.~Ecker$^\textrm{\scriptsize 103}$,
R.C.~Edgar$^\textrm{\scriptsize 92}$,
T.~Eifert$^\textrm{\scriptsize 32}$,
G.~Eigen$^\textrm{\scriptsize 15}$,
K.~Einsweiler$^\textrm{\scriptsize 16}$,
T.~Ekelof$^\textrm{\scriptsize 168}$,
M.~El~Kacimi$^\textrm{\scriptsize 137c}$,
R.~El~Kosseifi$^\textrm{\scriptsize 88}$,
V.~Ellajosyula$^\textrm{\scriptsize 88}$,
M.~Ellert$^\textrm{\scriptsize 168}$,
S.~Elles$^\textrm{\scriptsize 5}$,
F.~Ellinghaus$^\textrm{\scriptsize 178}$,
A.A.~Elliot$^\textrm{\scriptsize 172}$,
N.~Ellis$^\textrm{\scriptsize 32}$,
J.~Elmsheuser$^\textrm{\scriptsize 27}$,
M.~Elsing$^\textrm{\scriptsize 32}$,
D.~Emeliyanov$^\textrm{\scriptsize 133}$,
Y.~Enari$^\textrm{\scriptsize 157}$,
O.C.~Endner$^\textrm{\scriptsize 86}$,
J.S.~Ennis$^\textrm{\scriptsize 173}$,
J.~Erdmann$^\textrm{\scriptsize 46}$,
A.~Ereditato$^\textrm{\scriptsize 18}$,
M.~Ernst$^\textrm{\scriptsize 27}$,
S.~Errede$^\textrm{\scriptsize 169}$,
M.~Escalier$^\textrm{\scriptsize 119}$,
C.~Escobar$^\textrm{\scriptsize 170}$,
B.~Esposito$^\textrm{\scriptsize 50}$,
O.~Estrada~Pastor$^\textrm{\scriptsize 170}$,
A.I.~Etienvre$^\textrm{\scriptsize 138}$,
E.~Etzion$^\textrm{\scriptsize 155}$,
H.~Evans$^\textrm{\scriptsize 64}$,
A.~Ezhilov$^\textrm{\scriptsize 125}$,
M.~Ezzi$^\textrm{\scriptsize 137e}$,
F.~Fabbri$^\textrm{\scriptsize 22a,22b}$,
L.~Fabbri$^\textrm{\scriptsize 22a,22b}$,
V.~Fabiani$^\textrm{\scriptsize 108}$,
G.~Facini$^\textrm{\scriptsize 81}$,
R.M.~Fakhrutdinov$^\textrm{\scriptsize 132}$,
S.~Falciano$^\textrm{\scriptsize 134a}$,
R.J.~Falla$^\textrm{\scriptsize 81}$,
J.~Faltova$^\textrm{\scriptsize 32}$,
Y.~Fang$^\textrm{\scriptsize 35a}$,
M.~Fanti$^\textrm{\scriptsize 94a,94b}$,
A.~Farbin$^\textrm{\scriptsize 8}$,
A.~Farilla$^\textrm{\scriptsize 136a}$,
C.~Farina$^\textrm{\scriptsize 127}$,
E.M.~Farina$^\textrm{\scriptsize 123a,123b}$,
T.~Farooque$^\textrm{\scriptsize 93}$,
S.~Farrell$^\textrm{\scriptsize 16}$,
S.M.~Farrington$^\textrm{\scriptsize 173}$,
P.~Farthouat$^\textrm{\scriptsize 32}$,
F.~Fassi$^\textrm{\scriptsize 137e}$,
P.~Fassnacht$^\textrm{\scriptsize 32}$,
D.~Fassouliotis$^\textrm{\scriptsize 9}$,
M.~Faucci~Giannelli$^\textrm{\scriptsize 49}$,
A.~Favareto$^\textrm{\scriptsize 53a,53b}$,
W.J.~Fawcett$^\textrm{\scriptsize 122}$,
L.~Fayard$^\textrm{\scriptsize 119}$,
O.L.~Fedin$^\textrm{\scriptsize 125}$$^{,q}$,
W.~Fedorko$^\textrm{\scriptsize 171}$,
S.~Feigl$^\textrm{\scriptsize 121}$,
L.~Feligioni$^\textrm{\scriptsize 88}$,
C.~Feng$^\textrm{\scriptsize 36b}$,
E.J.~Feng$^\textrm{\scriptsize 32}$,
H.~Feng$^\textrm{\scriptsize 92}$,
M.J.~Fenton$^\textrm{\scriptsize 56}$,
A.B.~Fenyuk$^\textrm{\scriptsize 132}$,
L.~Feremenga$^\textrm{\scriptsize 8}$,
P.~Fernandez~Martinez$^\textrm{\scriptsize 170}$,
S.~Fernandez~Perez$^\textrm{\scriptsize 13}$,
J.~Ferrando$^\textrm{\scriptsize 45}$,
A.~Ferrari$^\textrm{\scriptsize 168}$,
P.~Ferrari$^\textrm{\scriptsize 109}$,
R.~Ferrari$^\textrm{\scriptsize 123a}$,
D.E.~Ferreira~de~Lima$^\textrm{\scriptsize 60b}$,
A.~Ferrer$^\textrm{\scriptsize 170}$,
D.~Ferrere$^\textrm{\scriptsize 52}$,
C.~Ferretti$^\textrm{\scriptsize 92}$,
F.~Fiedler$^\textrm{\scriptsize 86}$,
A.~Filip\v{c}i\v{c}$^\textrm{\scriptsize 78}$,
M.~Filipuzzi$^\textrm{\scriptsize 45}$,
F.~Filthaut$^\textrm{\scriptsize 108}$,
M.~Fincke-Keeler$^\textrm{\scriptsize 172}$,
K.D.~Finelli$^\textrm{\scriptsize 152}$,
M.C.N.~Fiolhais$^\textrm{\scriptsize 128a,128c}$$^{,r}$,
L.~Fiorini$^\textrm{\scriptsize 170}$,
A.~Fischer$^\textrm{\scriptsize 2}$,
C.~Fischer$^\textrm{\scriptsize 13}$,
J.~Fischer$^\textrm{\scriptsize 178}$,
W.C.~Fisher$^\textrm{\scriptsize 93}$,
N.~Flaschel$^\textrm{\scriptsize 45}$,
I.~Fleck$^\textrm{\scriptsize 143}$,
P.~Fleischmann$^\textrm{\scriptsize 92}$,
R.R.M.~Fletcher$^\textrm{\scriptsize 124}$,
T.~Flick$^\textrm{\scriptsize 178}$,
B.M.~Flierl$^\textrm{\scriptsize 102}$,
L.R.~Flores~Castillo$^\textrm{\scriptsize 62a}$,
M.J.~Flowerdew$^\textrm{\scriptsize 103}$,
G.T.~Forcolin$^\textrm{\scriptsize 87}$,
A.~Formica$^\textrm{\scriptsize 138}$,
F.A.~F\"orster$^\textrm{\scriptsize 13}$,
A.~Forti$^\textrm{\scriptsize 87}$,
A.G.~Foster$^\textrm{\scriptsize 19}$,
D.~Fournier$^\textrm{\scriptsize 119}$,
H.~Fox$^\textrm{\scriptsize 75}$,
S.~Fracchia$^\textrm{\scriptsize 141}$,
P.~Francavilla$^\textrm{\scriptsize 83}$,
M.~Franchini$^\textrm{\scriptsize 22a,22b}$,
S.~Franchino$^\textrm{\scriptsize 60a}$,
D.~Francis$^\textrm{\scriptsize 32}$,
L.~Franconi$^\textrm{\scriptsize 121}$,
M.~Franklin$^\textrm{\scriptsize 59}$,
M.~Frate$^\textrm{\scriptsize 166}$,
M.~Fraternali$^\textrm{\scriptsize 123a,123b}$,
D.~Freeborn$^\textrm{\scriptsize 81}$,
S.M.~Fressard-Batraneanu$^\textrm{\scriptsize 32}$,
B.~Freund$^\textrm{\scriptsize 97}$,
D.~Froidevaux$^\textrm{\scriptsize 32}$,
J.A.~Frost$^\textrm{\scriptsize 122}$,
C.~Fukunaga$^\textrm{\scriptsize 158}$,
T.~Fusayasu$^\textrm{\scriptsize 104}$,
J.~Fuster$^\textrm{\scriptsize 170}$,
C.~Gabaldon$^\textrm{\scriptsize 58}$,
O.~Gabizon$^\textrm{\scriptsize 154}$,
A.~Gabrielli$^\textrm{\scriptsize 22a,22b}$,
A.~Gabrielli$^\textrm{\scriptsize 16}$,
G.P.~Gach$^\textrm{\scriptsize 41a}$,
S.~Gadatsch$^\textrm{\scriptsize 32}$,
S.~Gadomski$^\textrm{\scriptsize 80}$,
G.~Gagliardi$^\textrm{\scriptsize 53a,53b}$,
L.G.~Gagnon$^\textrm{\scriptsize 97}$,
C.~Galea$^\textrm{\scriptsize 108}$,
B.~Galhardo$^\textrm{\scriptsize 128a,128c}$,
E.J.~Gallas$^\textrm{\scriptsize 122}$,
B.J.~Gallop$^\textrm{\scriptsize 133}$,
P.~Gallus$^\textrm{\scriptsize 130}$,
G.~Galster$^\textrm{\scriptsize 39}$,
K.K.~Gan$^\textrm{\scriptsize 113}$,
S.~Ganguly$^\textrm{\scriptsize 37}$,
Y.~Gao$^\textrm{\scriptsize 77}$,
Y.S.~Gao$^\textrm{\scriptsize 145}$$^{,g}$,
F.M.~Garay~Walls$^\textrm{\scriptsize 49}$,
C.~Garc\'ia$^\textrm{\scriptsize 170}$,
J.E.~Garc\'ia~Navarro$^\textrm{\scriptsize 170}$,
J.A.~Garc\'ia~Pascual$^\textrm{\scriptsize 35a}$,
M.~Garcia-Sciveres$^\textrm{\scriptsize 16}$,
R.W.~Gardner$^\textrm{\scriptsize 33}$,
N.~Garelli$^\textrm{\scriptsize 145}$,
V.~Garonne$^\textrm{\scriptsize 121}$,
A.~Gascon~Bravo$^\textrm{\scriptsize 45}$,
K.~Gasnikova$^\textrm{\scriptsize 45}$,
C.~Gatti$^\textrm{\scriptsize 50}$,
A.~Gaudiello$^\textrm{\scriptsize 53a,53b}$,
G.~Gaudio$^\textrm{\scriptsize 123a}$,
I.L.~Gavrilenko$^\textrm{\scriptsize 98}$,
C.~Gay$^\textrm{\scriptsize 171}$,
G.~Gaycken$^\textrm{\scriptsize 23}$,
E.N.~Gazis$^\textrm{\scriptsize 10}$,
C.N.P.~Gee$^\textrm{\scriptsize 133}$,
J.~Geisen$^\textrm{\scriptsize 57}$,
M.~Geisen$^\textrm{\scriptsize 86}$,
M.P.~Geisler$^\textrm{\scriptsize 60a}$,
K.~Gellerstedt$^\textrm{\scriptsize 148a,148b}$,
C.~Gemme$^\textrm{\scriptsize 53a}$,
M.H.~Genest$^\textrm{\scriptsize 58}$,
C.~Geng$^\textrm{\scriptsize 92}$,
S.~Gentile$^\textrm{\scriptsize 134a,134b}$,
C.~Gentsos$^\textrm{\scriptsize 156}$,
S.~George$^\textrm{\scriptsize 80}$,
D.~Gerbaudo$^\textrm{\scriptsize 13}$,
A.~Gershon$^\textrm{\scriptsize 155}$,
G.~Ge\ss{}ner$^\textrm{\scriptsize 46}$,
S.~Ghasemi$^\textrm{\scriptsize 143}$,
M.~Ghneimat$^\textrm{\scriptsize 23}$,
B.~Giacobbe$^\textrm{\scriptsize 22a}$,
S.~Giagu$^\textrm{\scriptsize 134a,134b}$,
N.~Giangiacomi$^\textrm{\scriptsize 22a,22b}$,
P.~Giannetti$^\textrm{\scriptsize 126a,126b}$,
S.M.~Gibson$^\textrm{\scriptsize 80}$,
M.~Gignac$^\textrm{\scriptsize 171}$,
M.~Gilchriese$^\textrm{\scriptsize 16}$,
D.~Gillberg$^\textrm{\scriptsize 31}$,
G.~Gilles$^\textrm{\scriptsize 178}$,
D.M.~Gingrich$^\textrm{\scriptsize 3}$$^{,d}$,
N.~Giokaris$^\textrm{\scriptsize 9}$$^{,*}$,
M.P.~Giordani$^\textrm{\scriptsize 167a,167c}$,
F.M.~Giorgi$^\textrm{\scriptsize 22a}$,
P.F.~Giraud$^\textrm{\scriptsize 138}$,
P.~Giromini$^\textrm{\scriptsize 59}$,
G.~Giugliarelli$^\textrm{\scriptsize 167a,167c}$,
D.~Giugni$^\textrm{\scriptsize 94a}$,
F.~Giuli$^\textrm{\scriptsize 122}$,
C.~Giuliani$^\textrm{\scriptsize 103}$,
M.~Giulini$^\textrm{\scriptsize 60b}$,
B.K.~Gjelsten$^\textrm{\scriptsize 121}$,
S.~Gkaitatzis$^\textrm{\scriptsize 156}$,
I.~Gkialas$^\textrm{\scriptsize 9}$$^{,s}$,
E.L.~Gkougkousis$^\textrm{\scriptsize 139}$,
P.~Gkountoumis$^\textrm{\scriptsize 10}$,
L.K.~Gladilin$^\textrm{\scriptsize 101}$,
C.~Glasman$^\textrm{\scriptsize 85}$,
J.~Glatzer$^\textrm{\scriptsize 13}$,
P.C.F.~Glaysher$^\textrm{\scriptsize 45}$,
A.~Glazov$^\textrm{\scriptsize 45}$,
M.~Goblirsch-Kolb$^\textrm{\scriptsize 25}$,
J.~Godlewski$^\textrm{\scriptsize 42}$,
S.~Goldfarb$^\textrm{\scriptsize 91}$,
T.~Golling$^\textrm{\scriptsize 52}$,
D.~Golubkov$^\textrm{\scriptsize 132}$,
A.~Gomes$^\textrm{\scriptsize 128a,128b,128d}$,
R.~Gon\c{c}alo$^\textrm{\scriptsize 128a}$,
R.~Goncalves~Gama$^\textrm{\scriptsize 26a}$,
J.~Goncalves~Pinto~Firmino~Da~Costa$^\textrm{\scriptsize 138}$,
G.~Gonella$^\textrm{\scriptsize 51}$,
L.~Gonella$^\textrm{\scriptsize 19}$,
A.~Gongadze$^\textrm{\scriptsize 68}$,
S.~Gonz\'alez~de~la~Hoz$^\textrm{\scriptsize 170}$,
S.~Gonzalez-Sevilla$^\textrm{\scriptsize 52}$,
L.~Goossens$^\textrm{\scriptsize 32}$,
P.A.~Gorbounov$^\textrm{\scriptsize 99}$,
H.A.~Gordon$^\textrm{\scriptsize 27}$,
I.~Gorelov$^\textrm{\scriptsize 107}$,
B.~Gorini$^\textrm{\scriptsize 32}$,
E.~Gorini$^\textrm{\scriptsize 76a,76b}$,
A.~Gori\v{s}ek$^\textrm{\scriptsize 78}$,
A.T.~Goshaw$^\textrm{\scriptsize 48}$,
C.~G\"ossling$^\textrm{\scriptsize 46}$,
M.I.~Gostkin$^\textrm{\scriptsize 68}$,
C.A.~Gottardo$^\textrm{\scriptsize 23}$,
C.R.~Goudet$^\textrm{\scriptsize 119}$,
D.~Goujdami$^\textrm{\scriptsize 137c}$,
A.G.~Goussiou$^\textrm{\scriptsize 140}$,
N.~Govender$^\textrm{\scriptsize 147b}$$^{,t}$,
E.~Gozani$^\textrm{\scriptsize 154}$,
L.~Graber$^\textrm{\scriptsize 57}$,
I.~Grabowska-Bold$^\textrm{\scriptsize 41a}$,
P.O.J.~Gradin$^\textrm{\scriptsize 168}$,
J.~Gramling$^\textrm{\scriptsize 166}$,
E.~Gramstad$^\textrm{\scriptsize 121}$,
S.~Grancagnolo$^\textrm{\scriptsize 17}$,
V.~Gratchev$^\textrm{\scriptsize 125}$,
P.M.~Gravila$^\textrm{\scriptsize 28f}$,
C.~Gray$^\textrm{\scriptsize 56}$,
H.M.~Gray$^\textrm{\scriptsize 16}$,
Z.D.~Greenwood$^\textrm{\scriptsize 82}$$^{,u}$,
C.~Grefe$^\textrm{\scriptsize 23}$,
K.~Gregersen$^\textrm{\scriptsize 81}$,
I.M.~Gregor$^\textrm{\scriptsize 45}$,
P.~Grenier$^\textrm{\scriptsize 145}$,
K.~Grevtsov$^\textrm{\scriptsize 5}$,
J.~Griffiths$^\textrm{\scriptsize 8}$,
A.A.~Grillo$^\textrm{\scriptsize 139}$,
K.~Grimm$^\textrm{\scriptsize 75}$,
S.~Grinstein$^\textrm{\scriptsize 13}$$^{,v}$,
Ph.~Gris$^\textrm{\scriptsize 37}$,
J.-F.~Grivaz$^\textrm{\scriptsize 119}$,
S.~Groh$^\textrm{\scriptsize 86}$,
E.~Gross$^\textrm{\scriptsize 175}$,
J.~Grosse-Knetter$^\textrm{\scriptsize 57}$,
G.C.~Grossi$^\textrm{\scriptsize 82}$,
Z.J.~Grout$^\textrm{\scriptsize 81}$,
A.~Grummer$^\textrm{\scriptsize 107}$,
L.~Guan$^\textrm{\scriptsize 92}$,
W.~Guan$^\textrm{\scriptsize 176}$,
J.~Guenther$^\textrm{\scriptsize 65}$,
F.~Guescini$^\textrm{\scriptsize 163a}$,
D.~Guest$^\textrm{\scriptsize 166}$,
O.~Gueta$^\textrm{\scriptsize 155}$,
B.~Gui$^\textrm{\scriptsize 113}$,
E.~Guido$^\textrm{\scriptsize 53a,53b}$,
T.~Guillemin$^\textrm{\scriptsize 5}$,
S.~Guindon$^\textrm{\scriptsize 2}$,
U.~Gul$^\textrm{\scriptsize 56}$,
C.~Gumpert$^\textrm{\scriptsize 32}$,
J.~Guo$^\textrm{\scriptsize 36c}$,
W.~Guo$^\textrm{\scriptsize 92}$,
Y.~Guo$^\textrm{\scriptsize 36a}$$^{,w}$,
R.~Gupta$^\textrm{\scriptsize 43}$,
S.~Gupta$^\textrm{\scriptsize 122}$,
G.~Gustavino$^\textrm{\scriptsize 115}$,
P.~Gutierrez$^\textrm{\scriptsize 115}$,
N.G.~Gutierrez~Ortiz$^\textrm{\scriptsize 81}$,
C.~Gutschow$^\textrm{\scriptsize 81}$,
C.~Guyot$^\textrm{\scriptsize 138}$,
M.P.~Guzik$^\textrm{\scriptsize 41a}$,
C.~Gwenlan$^\textrm{\scriptsize 122}$,
C.B.~Gwilliam$^\textrm{\scriptsize 77}$,
A.~Haas$^\textrm{\scriptsize 112}$,
C.~Haber$^\textrm{\scriptsize 16}$,
H.K.~Hadavand$^\textrm{\scriptsize 8}$,
N.~Haddad$^\textrm{\scriptsize 137e}$,
A.~Hadef$^\textrm{\scriptsize 88}$,
S.~Hageb\"ock$^\textrm{\scriptsize 23}$,
M.~Hagihara$^\textrm{\scriptsize 164}$,
H.~Hakobyan$^\textrm{\scriptsize 180}$$^{,*}$,
M.~Haleem$^\textrm{\scriptsize 45}$,
J.~Haley$^\textrm{\scriptsize 116}$,
G.~Halladjian$^\textrm{\scriptsize 93}$,
G.D.~Hallewell$^\textrm{\scriptsize 88}$,
K.~Hamacher$^\textrm{\scriptsize 178}$,
P.~Hamal$^\textrm{\scriptsize 117}$,
K.~Hamano$^\textrm{\scriptsize 172}$,
A.~Hamilton$^\textrm{\scriptsize 147a}$,
G.N.~Hamity$^\textrm{\scriptsize 141}$,
P.G.~Hamnett$^\textrm{\scriptsize 45}$,
L.~Han$^\textrm{\scriptsize 36a}$,
S.~Han$^\textrm{\scriptsize 35a,35d}$,
K.~Hanagaki$^\textrm{\scriptsize 69}$$^{,x}$,
K.~Hanawa$^\textrm{\scriptsize 157}$,
M.~Hance$^\textrm{\scriptsize 139}$,
B.~Haney$^\textrm{\scriptsize 124}$,
P.~Hanke$^\textrm{\scriptsize 60a}$,
J.B.~Hansen$^\textrm{\scriptsize 39}$,
J.D.~Hansen$^\textrm{\scriptsize 39}$,
M.C.~Hansen$^\textrm{\scriptsize 23}$,
P.H.~Hansen$^\textrm{\scriptsize 39}$,
K.~Hara$^\textrm{\scriptsize 164}$,
A.S.~Hard$^\textrm{\scriptsize 176}$,
T.~Harenberg$^\textrm{\scriptsize 178}$,
F.~Hariri$^\textrm{\scriptsize 119}$,
S.~Harkusha$^\textrm{\scriptsize 95}$,
R.D.~Harrington$^\textrm{\scriptsize 49}$,
P.F.~Harrison$^\textrm{\scriptsize 173}$,
N.M.~Hartmann$^\textrm{\scriptsize 102}$,
M.~Hasegawa$^\textrm{\scriptsize 70}$,
Y.~Hasegawa$^\textrm{\scriptsize 142}$,
A.~Hasib$^\textrm{\scriptsize 49}$,
S.~Hassani$^\textrm{\scriptsize 138}$,
S.~Haug$^\textrm{\scriptsize 18}$,
R.~Hauser$^\textrm{\scriptsize 93}$,
L.~Hauswald$^\textrm{\scriptsize 47}$,
L.B.~Havener$^\textrm{\scriptsize 38}$,
M.~Havranek$^\textrm{\scriptsize 130}$,
C.M.~Hawkes$^\textrm{\scriptsize 19}$,
R.J.~Hawkings$^\textrm{\scriptsize 32}$,
D.~Hayakawa$^\textrm{\scriptsize 159}$,
D.~Hayden$^\textrm{\scriptsize 93}$,
C.P.~Hays$^\textrm{\scriptsize 122}$,
J.M.~Hays$^\textrm{\scriptsize 79}$,
H.S.~Hayward$^\textrm{\scriptsize 77}$,
S.J.~Haywood$^\textrm{\scriptsize 133}$,
S.J.~Head$^\textrm{\scriptsize 19}$,
T.~Heck$^\textrm{\scriptsize 86}$,
V.~Hedberg$^\textrm{\scriptsize 84}$,
L.~Heelan$^\textrm{\scriptsize 8}$,
S.~Heer$^\textrm{\scriptsize 23}$,
K.K.~Heidegger$^\textrm{\scriptsize 51}$,
S.~Heim$^\textrm{\scriptsize 45}$,
T.~Heim$^\textrm{\scriptsize 16}$,
B.~Heinemann$^\textrm{\scriptsize 45}$$^{,y}$,
J.J.~Heinrich$^\textrm{\scriptsize 102}$,
L.~Heinrich$^\textrm{\scriptsize 112}$,
C.~Heinz$^\textrm{\scriptsize 55}$,
J.~Hejbal$^\textrm{\scriptsize 129}$,
L.~Helary$^\textrm{\scriptsize 32}$,
A.~Held$^\textrm{\scriptsize 171}$,
S.~Hellman$^\textrm{\scriptsize 148a,148b}$,
C.~Helsens$^\textrm{\scriptsize 32}$,
R.C.W.~Henderson$^\textrm{\scriptsize 75}$,
Y.~Heng$^\textrm{\scriptsize 176}$,
S.~Henkelmann$^\textrm{\scriptsize 171}$,
A.M.~Henriques~Correia$^\textrm{\scriptsize 32}$,
S.~Henrot-Versille$^\textrm{\scriptsize 119}$,
G.H.~Herbert$^\textrm{\scriptsize 17}$,
H.~Herde$^\textrm{\scriptsize 25}$,
V.~Herget$^\textrm{\scriptsize 177}$,
Y.~Hern\'andez~Jim\'enez$^\textrm{\scriptsize 147c}$,
H.~Herr$^\textrm{\scriptsize 86}$,
G.~Herten$^\textrm{\scriptsize 51}$,
R.~Hertenberger$^\textrm{\scriptsize 102}$,
L.~Hervas$^\textrm{\scriptsize 32}$,
T.C.~Herwig$^\textrm{\scriptsize 124}$,
G.G.~Hesketh$^\textrm{\scriptsize 81}$,
N.P.~Hessey$^\textrm{\scriptsize 163a}$,
J.W.~Hetherly$^\textrm{\scriptsize 43}$,
S.~Higashino$^\textrm{\scriptsize 69}$,
E.~Hig\'on-Rodriguez$^\textrm{\scriptsize 170}$,
K.~Hildebrand$^\textrm{\scriptsize 33}$,
E.~Hill$^\textrm{\scriptsize 172}$,
J.C.~Hill$^\textrm{\scriptsize 30}$,
K.H.~Hiller$^\textrm{\scriptsize 45}$,
S.J.~Hillier$^\textrm{\scriptsize 19}$,
M.~Hils$^\textrm{\scriptsize 47}$,
I.~Hinchliffe$^\textrm{\scriptsize 16}$,
M.~Hirose$^\textrm{\scriptsize 51}$,
D.~Hirschbuehl$^\textrm{\scriptsize 178}$,
B.~Hiti$^\textrm{\scriptsize 78}$,
O.~Hladik$^\textrm{\scriptsize 129}$,
X.~Hoad$^\textrm{\scriptsize 49}$,
J.~Hobbs$^\textrm{\scriptsize 150}$,
N.~Hod$^\textrm{\scriptsize 163a}$,
M.C.~Hodgkinson$^\textrm{\scriptsize 141}$,
P.~Hodgson$^\textrm{\scriptsize 141}$,
A.~Hoecker$^\textrm{\scriptsize 32}$,
M.R.~Hoeferkamp$^\textrm{\scriptsize 107}$,
F.~Hoenig$^\textrm{\scriptsize 102}$,
D.~Hohn$^\textrm{\scriptsize 23}$,
T.R.~Holmes$^\textrm{\scriptsize 33}$,
M.~Homann$^\textrm{\scriptsize 46}$,
S.~Honda$^\textrm{\scriptsize 164}$,
T.~Honda$^\textrm{\scriptsize 69}$,
T.M.~Hong$^\textrm{\scriptsize 127}$,
B.H.~Hooberman$^\textrm{\scriptsize 169}$,
W.H.~Hopkins$^\textrm{\scriptsize 118}$,
Y.~Horii$^\textrm{\scriptsize 105}$,
A.J.~Horton$^\textrm{\scriptsize 144}$,
J-Y.~Hostachy$^\textrm{\scriptsize 58}$,
S.~Hou$^\textrm{\scriptsize 153}$,
A.~Hoummada$^\textrm{\scriptsize 137a}$,
J.~Howarth$^\textrm{\scriptsize 87}$,
J.~Hoya$^\textrm{\scriptsize 74}$,
M.~Hrabovsky$^\textrm{\scriptsize 117}$,
J.~Hrdinka$^\textrm{\scriptsize 32}$,
I.~Hristova$^\textrm{\scriptsize 17}$,
J.~Hrivnac$^\textrm{\scriptsize 119}$,
T.~Hryn'ova$^\textrm{\scriptsize 5}$,
A.~Hrynevich$^\textrm{\scriptsize 96}$,
P.J.~Hsu$^\textrm{\scriptsize 63}$,
S.-C.~Hsu$^\textrm{\scriptsize 140}$,
Q.~Hu$^\textrm{\scriptsize 36a}$,
S.~Hu$^\textrm{\scriptsize 36c}$,
Y.~Huang$^\textrm{\scriptsize 35a}$,
Z.~Hubacek$^\textrm{\scriptsize 130}$,
F.~Hubaut$^\textrm{\scriptsize 88}$,
F.~Huegging$^\textrm{\scriptsize 23}$,
T.B.~Huffman$^\textrm{\scriptsize 122}$,
E.W.~Hughes$^\textrm{\scriptsize 38}$,
G.~Hughes$^\textrm{\scriptsize 75}$,
M.~Huhtinen$^\textrm{\scriptsize 32}$,
P.~Huo$^\textrm{\scriptsize 150}$,
N.~Huseynov$^\textrm{\scriptsize 68}$$^{,b}$,
J.~Huston$^\textrm{\scriptsize 93}$,
J.~Huth$^\textrm{\scriptsize 59}$,
G.~Iacobucci$^\textrm{\scriptsize 52}$,
G.~Iakovidis$^\textrm{\scriptsize 27}$,
I.~Ibragimov$^\textrm{\scriptsize 143}$,
L.~Iconomidou-Fayard$^\textrm{\scriptsize 119}$,
Z.~Idrissi$^\textrm{\scriptsize 137e}$,
P.~Iengo$^\textrm{\scriptsize 32}$,
O.~Igonkina$^\textrm{\scriptsize 109}$$^{,z}$,
T.~Iizawa$^\textrm{\scriptsize 174}$,
Y.~Ikegami$^\textrm{\scriptsize 69}$,
M.~Ikeno$^\textrm{\scriptsize 69}$,
Y.~Ilchenko$^\textrm{\scriptsize 11}$$^{,aa}$,
D.~Iliadis$^\textrm{\scriptsize 156}$,
N.~Ilic$^\textrm{\scriptsize 145}$,
G.~Introzzi$^\textrm{\scriptsize 123a,123b}$,
P.~Ioannou$^\textrm{\scriptsize 9}$$^{,*}$,
M.~Iodice$^\textrm{\scriptsize 136a}$,
K.~Iordanidou$^\textrm{\scriptsize 38}$,
V.~Ippolito$^\textrm{\scriptsize 59}$,
M.F.~Isacson$^\textrm{\scriptsize 168}$,
N.~Ishijima$^\textrm{\scriptsize 120}$,
M.~Ishino$^\textrm{\scriptsize 157}$,
M.~Ishitsuka$^\textrm{\scriptsize 159}$,
C.~Issever$^\textrm{\scriptsize 122}$,
S.~Istin$^\textrm{\scriptsize 20a}$,
F.~Ito$^\textrm{\scriptsize 164}$,
J.M.~Iturbe~Ponce$^\textrm{\scriptsize 62a}$,
R.~Iuppa$^\textrm{\scriptsize 162a,162b}$,
H.~Iwasaki$^\textrm{\scriptsize 69}$,
J.M.~Izen$^\textrm{\scriptsize 44}$,
V.~Izzo$^\textrm{\scriptsize 106a}$,
S.~Jabbar$^\textrm{\scriptsize 3}$,
P.~Jackson$^\textrm{\scriptsize 1}$,
R.M.~Jacobs$^\textrm{\scriptsize 23}$,
V.~Jain$^\textrm{\scriptsize 2}$,
K.B.~Jakobi$^\textrm{\scriptsize 86}$,
K.~Jakobs$^\textrm{\scriptsize 51}$,
S.~Jakobsen$^\textrm{\scriptsize 65}$,
T.~Jakoubek$^\textrm{\scriptsize 129}$,
D.O.~Jamin$^\textrm{\scriptsize 116}$,
D.K.~Jana$^\textrm{\scriptsize 82}$,
R.~Jansky$^\textrm{\scriptsize 52}$,
J.~Janssen$^\textrm{\scriptsize 23}$,
M.~Janus$^\textrm{\scriptsize 57}$,
P.A.~Janus$^\textrm{\scriptsize 41a}$,
G.~Jarlskog$^\textrm{\scriptsize 84}$,
N.~Javadov$^\textrm{\scriptsize 68}$$^{,b}$,
T.~Jav\r{u}rek$^\textrm{\scriptsize 51}$,
M.~Javurkova$^\textrm{\scriptsize 51}$,
F.~Jeanneau$^\textrm{\scriptsize 138}$,
L.~Jeanty$^\textrm{\scriptsize 16}$,
J.~Jejelava$^\textrm{\scriptsize 54a}$$^{,ab}$,
A.~Jelinskas$^\textrm{\scriptsize 173}$,
P.~Jenni$^\textrm{\scriptsize 51}$$^{,ac}$,
C.~Jeske$^\textrm{\scriptsize 173}$,
S.~J\'ez\'equel$^\textrm{\scriptsize 5}$,
H.~Ji$^\textrm{\scriptsize 176}$,
J.~Jia$^\textrm{\scriptsize 150}$,
H.~Jiang$^\textrm{\scriptsize 67}$,
Y.~Jiang$^\textrm{\scriptsize 36a}$,
Z.~Jiang$^\textrm{\scriptsize 145}$,
S.~Jiggins$^\textrm{\scriptsize 81}$,
J.~Jimenez~Pena$^\textrm{\scriptsize 170}$,
S.~Jin$^\textrm{\scriptsize 35a}$,
A.~Jinaru$^\textrm{\scriptsize 28b}$,
O.~Jinnouchi$^\textrm{\scriptsize 159}$,
H.~Jivan$^\textrm{\scriptsize 147c}$,
P.~Johansson$^\textrm{\scriptsize 141}$,
K.A.~Johns$^\textrm{\scriptsize 7}$,
C.A.~Johnson$^\textrm{\scriptsize 64}$,
W.J.~Johnson$^\textrm{\scriptsize 140}$,
K.~Jon-And$^\textrm{\scriptsize 148a,148b}$,
R.W.L.~Jones$^\textrm{\scriptsize 75}$,
S.D.~Jones$^\textrm{\scriptsize 151}$,
S.~Jones$^\textrm{\scriptsize 7}$,
T.J.~Jones$^\textrm{\scriptsize 77}$,
J.~Jongmanns$^\textrm{\scriptsize 60a}$,
P.M.~Jorge$^\textrm{\scriptsize 128a,128b}$,
J.~Jovicevic$^\textrm{\scriptsize 163a}$,
X.~Ju$^\textrm{\scriptsize 176}$,
A.~Juste~Rozas$^\textrm{\scriptsize 13}$$^{,v}$,
M.K.~K\"{o}hler$^\textrm{\scriptsize 175}$,
A.~Kaczmarska$^\textrm{\scriptsize 42}$,
M.~Kado$^\textrm{\scriptsize 119}$,
H.~Kagan$^\textrm{\scriptsize 113}$,
M.~Kagan$^\textrm{\scriptsize 145}$,
S.J.~Kahn$^\textrm{\scriptsize 88}$,
T.~Kaji$^\textrm{\scriptsize 174}$,
E.~Kajomovitz$^\textrm{\scriptsize 48}$,
C.W.~Kalderon$^\textrm{\scriptsize 84}$,
A.~Kaluza$^\textrm{\scriptsize 86}$,
S.~Kama$^\textrm{\scriptsize 43}$,
A.~Kamenshchikov$^\textrm{\scriptsize 132}$,
N.~Kanaya$^\textrm{\scriptsize 157}$,
L.~Kanjir$^\textrm{\scriptsize 78}$,
V.A.~Kantserov$^\textrm{\scriptsize 100}$,
J.~Kanzaki$^\textrm{\scriptsize 69}$,
B.~Kaplan$^\textrm{\scriptsize 112}$,
L.S.~Kaplan$^\textrm{\scriptsize 176}$,
D.~Kar$^\textrm{\scriptsize 147c}$,
K.~Karakostas$^\textrm{\scriptsize 10}$,
N.~Karastathis$^\textrm{\scriptsize 10}$,
M.J.~Kareem$^\textrm{\scriptsize 57}$,
E.~Karentzos$^\textrm{\scriptsize 10}$,
S.N.~Karpov$^\textrm{\scriptsize 68}$,
Z.M.~Karpova$^\textrm{\scriptsize 68}$,
K.~Karthik$^\textrm{\scriptsize 112}$,
V.~Kartvelishvili$^\textrm{\scriptsize 75}$,
A.N.~Karyukhin$^\textrm{\scriptsize 132}$,
K.~Kasahara$^\textrm{\scriptsize 164}$,
L.~Kashif$^\textrm{\scriptsize 176}$,
R.D.~Kass$^\textrm{\scriptsize 113}$,
A.~Kastanas$^\textrm{\scriptsize 149}$,
Y.~Kataoka$^\textrm{\scriptsize 157}$,
C.~Kato$^\textrm{\scriptsize 157}$,
A.~Katre$^\textrm{\scriptsize 52}$,
J.~Katzy$^\textrm{\scriptsize 45}$,
K.~Kawade$^\textrm{\scriptsize 70}$,
K.~Kawagoe$^\textrm{\scriptsize 73}$,
T.~Kawamoto$^\textrm{\scriptsize 157}$,
G.~Kawamura$^\textrm{\scriptsize 57}$,
E.F.~Kay$^\textrm{\scriptsize 77}$,
V.F.~Kazanin$^\textrm{\scriptsize 111}$$^{,c}$,
R.~Keeler$^\textrm{\scriptsize 172}$,
R.~Kehoe$^\textrm{\scriptsize 43}$,
J.S.~Keller$^\textrm{\scriptsize 31}$,
E.~Kellermann$^\textrm{\scriptsize 84}$,
J.J.~Kempster$^\textrm{\scriptsize 80}$,
J~Kendrick$^\textrm{\scriptsize 19}$,
H.~Keoshkerian$^\textrm{\scriptsize 161}$,
O.~Kepka$^\textrm{\scriptsize 129}$,
B.P.~Ker\v{s}evan$^\textrm{\scriptsize 78}$,
S.~Kersten$^\textrm{\scriptsize 178}$,
R.A.~Keyes$^\textrm{\scriptsize 90}$,
M.~Khader$^\textrm{\scriptsize 169}$,
F.~Khalil-zada$^\textrm{\scriptsize 12}$,
A.~Khanov$^\textrm{\scriptsize 116}$,
A.G.~Kharlamov$^\textrm{\scriptsize 111}$$^{,c}$,
T.~Kharlamova$^\textrm{\scriptsize 111}$$^{,c}$,
A.~Khodinov$^\textrm{\scriptsize 160}$,
T.J.~Khoo$^\textrm{\scriptsize 52}$,
V.~Khovanskiy$^\textrm{\scriptsize 99}$$^{,*}$,
E.~Khramov$^\textrm{\scriptsize 68}$,
J.~Khubua$^\textrm{\scriptsize 54b}$$^{,ad}$,
S.~Kido$^\textrm{\scriptsize 70}$,
C.R.~Kilby$^\textrm{\scriptsize 80}$,
H.Y.~Kim$^\textrm{\scriptsize 8}$,
S.H.~Kim$^\textrm{\scriptsize 164}$,
Y.K.~Kim$^\textrm{\scriptsize 33}$,
N.~Kimura$^\textrm{\scriptsize 156}$,
O.M.~Kind$^\textrm{\scriptsize 17}$,
B.T.~King$^\textrm{\scriptsize 77}$,
D.~Kirchmeier$^\textrm{\scriptsize 47}$,
J.~Kirk$^\textrm{\scriptsize 133}$,
A.E.~Kiryunin$^\textrm{\scriptsize 103}$,
T.~Kishimoto$^\textrm{\scriptsize 157}$,
D.~Kisielewska$^\textrm{\scriptsize 41a}$,
V.~Kitali$^\textrm{\scriptsize 45}$,
K.~Kiuchi$^\textrm{\scriptsize 164}$,
O.~Kivernyk$^\textrm{\scriptsize 5}$,
E.~Kladiva$^\textrm{\scriptsize 146b}$,
T.~Klapdor-Kleingrothaus$^\textrm{\scriptsize 51}$,
M.H.~Klein$^\textrm{\scriptsize 92}$,
M.~Klein$^\textrm{\scriptsize 77}$,
U.~Klein$^\textrm{\scriptsize 77}$,
K.~Kleinknecht$^\textrm{\scriptsize 86}$,
P.~Klimek$^\textrm{\scriptsize 110}$,
A.~Klimentov$^\textrm{\scriptsize 27}$,
R.~Klingenberg$^\textrm{\scriptsize 46}$,
T.~Klingl$^\textrm{\scriptsize 23}$,
T.~Klioutchnikova$^\textrm{\scriptsize 32}$,
E.-E.~Kluge$^\textrm{\scriptsize 60a}$,
P.~Kluit$^\textrm{\scriptsize 109}$,
S.~Kluth$^\textrm{\scriptsize 103}$,
E.~Kneringer$^\textrm{\scriptsize 65}$,
E.B.F.G.~Knoops$^\textrm{\scriptsize 88}$,
A.~Knue$^\textrm{\scriptsize 103}$,
A.~Kobayashi$^\textrm{\scriptsize 157}$,
D.~Kobayashi$^\textrm{\scriptsize 159}$,
T.~Kobayashi$^\textrm{\scriptsize 157}$,
M.~Kobel$^\textrm{\scriptsize 47}$,
M.~Kocian$^\textrm{\scriptsize 145}$,
P.~Kodys$^\textrm{\scriptsize 131}$,
T.~Koffas$^\textrm{\scriptsize 31}$,
E.~Koffeman$^\textrm{\scriptsize 109}$,
N.M.~K\"ohler$^\textrm{\scriptsize 103}$,
T.~Koi$^\textrm{\scriptsize 145}$,
M.~Kolb$^\textrm{\scriptsize 60b}$,
I.~Koletsou$^\textrm{\scriptsize 5}$,
A.A.~Komar$^\textrm{\scriptsize 98}$$^{,*}$,
Y.~Komori$^\textrm{\scriptsize 157}$,
T.~Kondo$^\textrm{\scriptsize 69}$,
N.~Kondrashova$^\textrm{\scriptsize 36c}$,
K.~K\"oneke$^\textrm{\scriptsize 51}$,
A.C.~K\"onig$^\textrm{\scriptsize 108}$,
T.~Kono$^\textrm{\scriptsize 69}$$^{,ae}$,
R.~Konoplich$^\textrm{\scriptsize 112}$$^{,af}$,
N.~Konstantinidis$^\textrm{\scriptsize 81}$,
R.~Kopeliansky$^\textrm{\scriptsize 64}$,
S.~Koperny$^\textrm{\scriptsize 41a}$,
A.K.~Kopp$^\textrm{\scriptsize 51}$,
K.~Korcyl$^\textrm{\scriptsize 42}$,
K.~Kordas$^\textrm{\scriptsize 156}$,
A.~Korn$^\textrm{\scriptsize 81}$,
A.A.~Korol$^\textrm{\scriptsize 111}$$^{,c}$,
I.~Korolkov$^\textrm{\scriptsize 13}$,
E.V.~Korolkova$^\textrm{\scriptsize 141}$,
O.~Kortner$^\textrm{\scriptsize 103}$,
S.~Kortner$^\textrm{\scriptsize 103}$,
T.~Kosek$^\textrm{\scriptsize 131}$,
V.V.~Kostyukhin$^\textrm{\scriptsize 23}$,
A.~Kotwal$^\textrm{\scriptsize 48}$,
A.~Koulouris$^\textrm{\scriptsize 10}$,
A.~Kourkoumeli-Charalampidi$^\textrm{\scriptsize 123a,123b}$,
C.~Kourkoumelis$^\textrm{\scriptsize 9}$,
E.~Kourlitis$^\textrm{\scriptsize 141}$,
V.~Kouskoura$^\textrm{\scriptsize 27}$,
A.B.~Kowalewska$^\textrm{\scriptsize 42}$,
R.~Kowalewski$^\textrm{\scriptsize 172}$,
T.Z.~Kowalski$^\textrm{\scriptsize 41a}$,
C.~Kozakai$^\textrm{\scriptsize 157}$,
W.~Kozanecki$^\textrm{\scriptsize 138}$,
A.S.~Kozhin$^\textrm{\scriptsize 132}$,
V.A.~Kramarenko$^\textrm{\scriptsize 101}$,
G.~Kramberger$^\textrm{\scriptsize 78}$,
D.~Krasnopevtsev$^\textrm{\scriptsize 100}$,
M.W.~Krasny$^\textrm{\scriptsize 83}$,
A.~Krasznahorkay$^\textrm{\scriptsize 32}$,
D.~Krauss$^\textrm{\scriptsize 103}$,
J.A.~Kremer$^\textrm{\scriptsize 41a}$,
J.~Kretzschmar$^\textrm{\scriptsize 77}$,
K.~Kreutzfeldt$^\textrm{\scriptsize 55}$,
P.~Krieger$^\textrm{\scriptsize 161}$,
K.~Krizka$^\textrm{\scriptsize 33}$,
K.~Kroeninger$^\textrm{\scriptsize 46}$,
H.~Kroha$^\textrm{\scriptsize 103}$,
J.~Kroll$^\textrm{\scriptsize 129}$,
J.~Kroll$^\textrm{\scriptsize 124}$,
J.~Kroseberg$^\textrm{\scriptsize 23}$,
J.~Krstic$^\textrm{\scriptsize 14}$,
U.~Kruchonak$^\textrm{\scriptsize 68}$,
H.~Kr\"uger$^\textrm{\scriptsize 23}$,
N.~Krumnack$^\textrm{\scriptsize 67}$,
M.C.~Kruse$^\textrm{\scriptsize 48}$,
T.~Kubota$^\textrm{\scriptsize 91}$,
H.~Kucuk$^\textrm{\scriptsize 81}$,
S.~Kuday$^\textrm{\scriptsize 4b}$,
J.T.~Kuechler$^\textrm{\scriptsize 178}$,
S.~Kuehn$^\textrm{\scriptsize 32}$,
A.~Kugel$^\textrm{\scriptsize 60a}$,
F.~Kuger$^\textrm{\scriptsize 177}$,
T.~Kuhl$^\textrm{\scriptsize 45}$,
V.~Kukhtin$^\textrm{\scriptsize 68}$,
R.~Kukla$^\textrm{\scriptsize 88}$,
Y.~Kulchitsky$^\textrm{\scriptsize 95}$,
S.~Kuleshov$^\textrm{\scriptsize 34b}$,
Y.P.~Kulinich$^\textrm{\scriptsize 169}$,
M.~Kuna$^\textrm{\scriptsize 134a,134b}$,
T.~Kunigo$^\textrm{\scriptsize 71}$,
A.~Kupco$^\textrm{\scriptsize 129}$,
T.~Kupfer$^\textrm{\scriptsize 46}$,
O.~Kuprash$^\textrm{\scriptsize 155}$,
H.~Kurashige$^\textrm{\scriptsize 70}$,
L.L.~Kurchaninov$^\textrm{\scriptsize 163a}$,
Y.A.~Kurochkin$^\textrm{\scriptsize 95}$,
M.G.~Kurth$^\textrm{\scriptsize 35a,35d}$,
V.~Kus$^\textrm{\scriptsize 129}$,
E.S.~Kuwertz$^\textrm{\scriptsize 172}$,
M.~Kuze$^\textrm{\scriptsize 159}$,
J.~Kvita$^\textrm{\scriptsize 117}$,
T.~Kwan$^\textrm{\scriptsize 172}$,
D.~Kyriazopoulos$^\textrm{\scriptsize 141}$,
A.~La~Rosa$^\textrm{\scriptsize 103}$,
J.L.~La~Rosa~Navarro$^\textrm{\scriptsize 26d}$,
L.~La~Rotonda$^\textrm{\scriptsize 40a,40b}$,
F.~La~Ruffa$^\textrm{\scriptsize 40a,40b}$,
C.~Lacasta$^\textrm{\scriptsize 170}$,
F.~Lacava$^\textrm{\scriptsize 134a,134b}$,
J.~Lacey$^\textrm{\scriptsize 45}$,
D.P.J.~Lack$^\textrm{\scriptsize 87}$,
H.~Lacker$^\textrm{\scriptsize 17}$,
D.~Lacour$^\textrm{\scriptsize 83}$,
E.~Ladygin$^\textrm{\scriptsize 68}$,
R.~Lafaye$^\textrm{\scriptsize 5}$,
B.~Laforge$^\textrm{\scriptsize 83}$,
T.~Lagouri$^\textrm{\scriptsize 179}$,
S.~Lai$^\textrm{\scriptsize 57}$,
S.~Lammers$^\textrm{\scriptsize 64}$,
W.~Lampl$^\textrm{\scriptsize 7}$,
E.~Lan\c{c}on$^\textrm{\scriptsize 27}$,
U.~Landgraf$^\textrm{\scriptsize 51}$,
M.P.J.~Landon$^\textrm{\scriptsize 79}$,
M.C.~Lanfermann$^\textrm{\scriptsize 52}$,
V.S.~Lang$^\textrm{\scriptsize 60a}$,
J.C.~Lange$^\textrm{\scriptsize 13}$,
R.J.~Langenberg$^\textrm{\scriptsize 32}$,
A.J.~Lankford$^\textrm{\scriptsize 166}$,
F.~Lanni$^\textrm{\scriptsize 27}$,
K.~Lantzsch$^\textrm{\scriptsize 23}$,
A.~Lanza$^\textrm{\scriptsize 123a}$,
A.~Lapertosa$^\textrm{\scriptsize 53a,53b}$,
S.~Laplace$^\textrm{\scriptsize 83}$,
J.F.~Laporte$^\textrm{\scriptsize 138}$,
T.~Lari$^\textrm{\scriptsize 94a}$,
F.~Lasagni~Manghi$^\textrm{\scriptsize 22a,22b}$,
M.~Lassnig$^\textrm{\scriptsize 32}$,
P.~Laurelli$^\textrm{\scriptsize 50}$,
W.~Lavrijsen$^\textrm{\scriptsize 16}$,
A.T.~Law$^\textrm{\scriptsize 139}$,
P.~Laycock$^\textrm{\scriptsize 77}$,
T.~Lazovich$^\textrm{\scriptsize 59}$,
M.~Lazzaroni$^\textrm{\scriptsize 94a,94b}$,
B.~Le$^\textrm{\scriptsize 91}$,
O.~Le~Dortz$^\textrm{\scriptsize 83}$,
E.~Le~Guirriec$^\textrm{\scriptsize 88}$,
E.P.~Le~Quilleuc$^\textrm{\scriptsize 138}$,
M.~LeBlanc$^\textrm{\scriptsize 172}$,
T.~LeCompte$^\textrm{\scriptsize 6}$,
F.~Ledroit-Guillon$^\textrm{\scriptsize 58}$,
C.A.~Lee$^\textrm{\scriptsize 27}$,
G.R.~Lee$^\textrm{\scriptsize 133}$$^{,ag}$,
S.C.~Lee$^\textrm{\scriptsize 153}$,
L.~Lee$^\textrm{\scriptsize 59}$,
B.~Lefebvre$^\textrm{\scriptsize 90}$,
G.~Lefebvre$^\textrm{\scriptsize 83}$,
M.~Lefebvre$^\textrm{\scriptsize 172}$,
F.~Legger$^\textrm{\scriptsize 102}$,
C.~Leggett$^\textrm{\scriptsize 16}$,
G.~Lehmann~Miotto$^\textrm{\scriptsize 32}$,
X.~Lei$^\textrm{\scriptsize 7}$,
W.A.~Leight$^\textrm{\scriptsize 45}$,
M.A.L.~Leite$^\textrm{\scriptsize 26d}$,
R.~Leitner$^\textrm{\scriptsize 131}$,
D.~Lellouch$^\textrm{\scriptsize 175}$,
B.~Lemmer$^\textrm{\scriptsize 57}$,
K.J.C.~Leney$^\textrm{\scriptsize 81}$,
T.~Lenz$^\textrm{\scriptsize 23}$,
B.~Lenzi$^\textrm{\scriptsize 32}$,
R.~Leone$^\textrm{\scriptsize 7}$,
S.~Leone$^\textrm{\scriptsize 126a,126b}$,
C.~Leonidopoulos$^\textrm{\scriptsize 49}$,
G.~Lerner$^\textrm{\scriptsize 151}$,
C.~Leroy$^\textrm{\scriptsize 97}$,
A.A.J.~Lesage$^\textrm{\scriptsize 138}$,
C.G.~Lester$^\textrm{\scriptsize 30}$,
M.~Levchenko$^\textrm{\scriptsize 125}$,
J.~Lev\^eque$^\textrm{\scriptsize 5}$,
D.~Levin$^\textrm{\scriptsize 92}$,
L.J.~Levinson$^\textrm{\scriptsize 175}$,
M.~Levy$^\textrm{\scriptsize 19}$,
D.~Lewis$^\textrm{\scriptsize 79}$,
B.~Li$^\textrm{\scriptsize 36a}$$^{,w}$,
Changqiao~Li$^\textrm{\scriptsize 36a}$,
H.~Li$^\textrm{\scriptsize 150}$,
L.~Li$^\textrm{\scriptsize 36c}$,
Q.~Li$^\textrm{\scriptsize 35a,35d}$,
Q.~Li$^\textrm{\scriptsize 36a}$,
S.~Li$^\textrm{\scriptsize 48}$,
X.~Li$^\textrm{\scriptsize 36c}$,
Y.~Li$^\textrm{\scriptsize 143}$,
Z.~Liang$^\textrm{\scriptsize 35a}$,
B.~Liberti$^\textrm{\scriptsize 135a}$,
A.~Liblong$^\textrm{\scriptsize 161}$,
K.~Lie$^\textrm{\scriptsize 62c}$,
J.~Liebal$^\textrm{\scriptsize 23}$,
W.~Liebig$^\textrm{\scriptsize 15}$,
A.~Limosani$^\textrm{\scriptsize 152}$,
S.C.~Lin$^\textrm{\scriptsize 182}$,
T.H.~Lin$^\textrm{\scriptsize 86}$,
R.A.~Linck$^\textrm{\scriptsize 64}$,
B.E.~Lindquist$^\textrm{\scriptsize 150}$,
A.E.~Lionti$^\textrm{\scriptsize 52}$,
E.~Lipeles$^\textrm{\scriptsize 124}$,
A.~Lipniacka$^\textrm{\scriptsize 15}$,
M.~Lisovyi$^\textrm{\scriptsize 60b}$,
T.M.~Liss$^\textrm{\scriptsize 169}$$^{,ah}$,
A.~Lister$^\textrm{\scriptsize 171}$,
A.M.~Litke$^\textrm{\scriptsize 139}$,
B.~Liu$^\textrm{\scriptsize 153}$$^{,ai}$,
H.~Liu$^\textrm{\scriptsize 92}$,
H.~Liu$^\textrm{\scriptsize 27}$,
J.K.K.~Liu$^\textrm{\scriptsize 122}$,
J.~Liu$^\textrm{\scriptsize 36b}$,
J.B.~Liu$^\textrm{\scriptsize 36a}$,
K.~Liu$^\textrm{\scriptsize 88}$,
L.~Liu$^\textrm{\scriptsize 169}$,
M.~Liu$^\textrm{\scriptsize 36a}$,
Y.L.~Liu$^\textrm{\scriptsize 36a}$,
Y.~Liu$^\textrm{\scriptsize 36a}$,
M.~Livan$^\textrm{\scriptsize 123a,123b}$,
A.~Lleres$^\textrm{\scriptsize 58}$,
J.~Llorente~Merino$^\textrm{\scriptsize 35a}$,
S.L.~Lloyd$^\textrm{\scriptsize 79}$,
C.Y.~Lo$^\textrm{\scriptsize 62b}$,
F.~Lo~Sterzo$^\textrm{\scriptsize 153}$,
E.M.~Lobodzinska$^\textrm{\scriptsize 45}$,
P.~Loch$^\textrm{\scriptsize 7}$,
F.K.~Loebinger$^\textrm{\scriptsize 87}$,
A.~Loesle$^\textrm{\scriptsize 51}$,
K.M.~Loew$^\textrm{\scriptsize 25}$,
A.~Loginov$^\textrm{\scriptsize 179}$$^{,*}$,
T.~Lohse$^\textrm{\scriptsize 17}$,
K.~Lohwasser$^\textrm{\scriptsize 141}$,
M.~Lokajicek$^\textrm{\scriptsize 129}$,
B.A.~Long$^\textrm{\scriptsize 24}$,
J.D.~Long$^\textrm{\scriptsize 169}$,
R.E.~Long$^\textrm{\scriptsize 75}$,
L.~Longo$^\textrm{\scriptsize 76a,76b}$,
K.A.~Looper$^\textrm{\scriptsize 113}$,
J.A.~Lopez$^\textrm{\scriptsize 34b}$,
D.~Lopez~Mateos$^\textrm{\scriptsize 59}$,
I.~Lopez~Paz$^\textrm{\scriptsize 13}$,
A.~Lopez~Solis$^\textrm{\scriptsize 83}$,
J.~Lorenz$^\textrm{\scriptsize 102}$,
N.~Lorenzo~Martinez$^\textrm{\scriptsize 5}$,
M.~Losada$^\textrm{\scriptsize 21}$,
P.J.~L{\"o}sel$^\textrm{\scriptsize 102}$,
X.~Lou$^\textrm{\scriptsize 35a}$,
A.~Lounis$^\textrm{\scriptsize 119}$,
J.~Love$^\textrm{\scriptsize 6}$,
P.A.~Love$^\textrm{\scriptsize 75}$,
H.~Lu$^\textrm{\scriptsize 62a}$,
N.~Lu$^\textrm{\scriptsize 92}$,
Y.J.~Lu$^\textrm{\scriptsize 63}$,
H.J.~Lubatti$^\textrm{\scriptsize 140}$,
C.~Luci$^\textrm{\scriptsize 134a,134b}$,
A.~Lucotte$^\textrm{\scriptsize 58}$,
C.~Luedtke$^\textrm{\scriptsize 51}$,
F.~Luehring$^\textrm{\scriptsize 64}$,
W.~Lukas$^\textrm{\scriptsize 65}$,
L.~Luminari$^\textrm{\scriptsize 134a}$,
O.~Lundberg$^\textrm{\scriptsize 148a,148b}$,
B.~Lund-Jensen$^\textrm{\scriptsize 149}$,
M.S.~Lutz$^\textrm{\scriptsize 89}$,
P.M.~Luzi$^\textrm{\scriptsize 83}$,
D.~Lynn$^\textrm{\scriptsize 27}$,
R.~Lysak$^\textrm{\scriptsize 129}$,
E.~Lytken$^\textrm{\scriptsize 84}$,
F.~Lyu$^\textrm{\scriptsize 35a}$,
V.~Lyubushkin$^\textrm{\scriptsize 68}$,
H.~Ma$^\textrm{\scriptsize 27}$,
L.L.~Ma$^\textrm{\scriptsize 36b}$,
Y.~Ma$^\textrm{\scriptsize 36b}$,
G.~Maccarrone$^\textrm{\scriptsize 50}$,
A.~Macchiolo$^\textrm{\scriptsize 103}$,
C.M.~Macdonald$^\textrm{\scriptsize 141}$,
B.~Ma\v{c}ek$^\textrm{\scriptsize 78}$,
J.~Machado~Miguens$^\textrm{\scriptsize 124,128b}$,
D.~Madaffari$^\textrm{\scriptsize 170}$,
R.~Madar$^\textrm{\scriptsize 37}$,
W.F.~Mader$^\textrm{\scriptsize 47}$,
A.~Madsen$^\textrm{\scriptsize 45}$,
J.~Maeda$^\textrm{\scriptsize 70}$,
S.~Maeland$^\textrm{\scriptsize 15}$,
T.~Maeno$^\textrm{\scriptsize 27}$,
A.S.~Maevskiy$^\textrm{\scriptsize 101}$,
V.~Magerl$^\textrm{\scriptsize 51}$,
J.~Mahlstedt$^\textrm{\scriptsize 109}$,
C.~Maiani$^\textrm{\scriptsize 119}$,
C.~Maidantchik$^\textrm{\scriptsize 26a}$,
A.A.~Maier$^\textrm{\scriptsize 103}$,
T.~Maier$^\textrm{\scriptsize 102}$,
A.~Maio$^\textrm{\scriptsize 128a,128b,128d}$,
O.~Majersky$^\textrm{\scriptsize 146a}$,
S.~Majewski$^\textrm{\scriptsize 118}$,
Y.~Makida$^\textrm{\scriptsize 69}$,
N.~Makovec$^\textrm{\scriptsize 119}$,
B.~Malaescu$^\textrm{\scriptsize 83}$,
Pa.~Malecki$^\textrm{\scriptsize 42}$,
V.P.~Maleev$^\textrm{\scriptsize 125}$,
F.~Malek$^\textrm{\scriptsize 58}$,
U.~Mallik$^\textrm{\scriptsize 66}$,
D.~Malon$^\textrm{\scriptsize 6}$,
C.~Malone$^\textrm{\scriptsize 30}$,
S.~Maltezos$^\textrm{\scriptsize 10}$,
S.~Malyukov$^\textrm{\scriptsize 32}$,
J.~Mamuzic$^\textrm{\scriptsize 170}$,
G.~Mancini$^\textrm{\scriptsize 50}$,
I.~Mandi\'{c}$^\textrm{\scriptsize 78}$,
J.~Maneira$^\textrm{\scriptsize 128a,128b}$,
L.~Manhaes~de~Andrade~Filho$^\textrm{\scriptsize 26b}$,
J.~Manjarres~Ramos$^\textrm{\scriptsize 47}$,
K.H.~Mankinen$^\textrm{\scriptsize 84}$,
A.~Mann$^\textrm{\scriptsize 102}$,
A.~Manousos$^\textrm{\scriptsize 32}$,
B.~Mansoulie$^\textrm{\scriptsize 138}$,
J.D.~Mansour$^\textrm{\scriptsize 35a}$,
R.~Mantifel$^\textrm{\scriptsize 90}$,
M.~Mantoani$^\textrm{\scriptsize 57}$,
S.~Manzoni$^\textrm{\scriptsize 94a,94b}$,
L.~Mapelli$^\textrm{\scriptsize 32}$,
G.~Marceca$^\textrm{\scriptsize 29}$,
L.~March$^\textrm{\scriptsize 52}$,
L.~Marchese$^\textrm{\scriptsize 122}$,
G.~Marchiori$^\textrm{\scriptsize 83}$,
M.~Marcisovsky$^\textrm{\scriptsize 129}$,
M.~Marjanovic$^\textrm{\scriptsize 37}$,
D.E.~Marley$^\textrm{\scriptsize 92}$,
F.~Marroquim$^\textrm{\scriptsize 26a}$,
S.P.~Marsden$^\textrm{\scriptsize 87}$,
Z.~Marshall$^\textrm{\scriptsize 16}$,
M.U.F~Martensson$^\textrm{\scriptsize 168}$,
S.~Marti-Garcia$^\textrm{\scriptsize 170}$,
C.B.~Martin$^\textrm{\scriptsize 113}$,
T.A.~Martin$^\textrm{\scriptsize 173}$,
V.J.~Martin$^\textrm{\scriptsize 49}$,
B.~Martin~dit~Latour$^\textrm{\scriptsize 15}$,
M.~Martinez$^\textrm{\scriptsize 13}$$^{,v}$,
V.I.~Martinez~Outschoorn$^\textrm{\scriptsize 169}$,
S.~Martin-Haugh$^\textrm{\scriptsize 133}$,
V.S.~Martoiu$^\textrm{\scriptsize 28b}$,
A.C.~Martyniuk$^\textrm{\scriptsize 81}$,
A.~Marzin$^\textrm{\scriptsize 32}$,
L.~Masetti$^\textrm{\scriptsize 86}$,
T.~Mashimo$^\textrm{\scriptsize 157}$,
R.~Mashinistov$^\textrm{\scriptsize 98}$,
J.~Masik$^\textrm{\scriptsize 87}$,
A.L.~Maslennikov$^\textrm{\scriptsize 111}$$^{,c}$,
L.~Massa$^\textrm{\scriptsize 135a,135b}$,
P.~Mastrandrea$^\textrm{\scriptsize 5}$,
A.~Mastroberardino$^\textrm{\scriptsize 40a,40b}$,
T.~Masubuchi$^\textrm{\scriptsize 157}$,
P.~M\"attig$^\textrm{\scriptsize 178}$,
J.~Maurer$^\textrm{\scriptsize 28b}$,
S.J.~Maxfield$^\textrm{\scriptsize 77}$,
D.A.~Maximov$^\textrm{\scriptsize 111}$$^{,c}$,
R.~Mazini$^\textrm{\scriptsize 153}$,
I.~Maznas$^\textrm{\scriptsize 156}$,
S.M.~Mazza$^\textrm{\scriptsize 94a,94b}$,
N.C.~Mc~Fadden$^\textrm{\scriptsize 107}$,
G.~Mc~Goldrick$^\textrm{\scriptsize 161}$,
S.P.~Mc~Kee$^\textrm{\scriptsize 92}$,
A.~McCarn$^\textrm{\scriptsize 92}$,
R.L.~McCarthy$^\textrm{\scriptsize 150}$,
T.G.~McCarthy$^\textrm{\scriptsize 103}$,
L.I.~McClymont$^\textrm{\scriptsize 81}$,
E.F.~McDonald$^\textrm{\scriptsize 91}$,
J.A.~Mcfayden$^\textrm{\scriptsize 81}$,
G.~Mchedlidze$^\textrm{\scriptsize 57}$,
S.J.~McMahon$^\textrm{\scriptsize 133}$,
P.C.~McNamara$^\textrm{\scriptsize 91}$,
R.A.~McPherson$^\textrm{\scriptsize 172}$$^{,o}$,
S.~Meehan$^\textrm{\scriptsize 140}$,
T.J.~Megy$^\textrm{\scriptsize 51}$,
S.~Mehlhase$^\textrm{\scriptsize 102}$,
A.~Mehta$^\textrm{\scriptsize 77}$,
T.~Meideck$^\textrm{\scriptsize 58}$,
K.~Meier$^\textrm{\scriptsize 60a}$,
B.~Meirose$^\textrm{\scriptsize 44}$,
D.~Melini$^\textrm{\scriptsize 170}$$^{,aj}$,
B.R.~Mellado~Garcia$^\textrm{\scriptsize 147c}$,
J.D.~Mellenthin$^\textrm{\scriptsize 57}$,
M.~Melo$^\textrm{\scriptsize 146a}$,
F.~Meloni$^\textrm{\scriptsize 18}$,
A.~Melzer$^\textrm{\scriptsize 23}$,
S.B.~Menary$^\textrm{\scriptsize 87}$,
L.~Meng$^\textrm{\scriptsize 77}$,
X.T.~Meng$^\textrm{\scriptsize 92}$,
A.~Mengarelli$^\textrm{\scriptsize 22a,22b}$,
S.~Menke$^\textrm{\scriptsize 103}$,
E.~Meoni$^\textrm{\scriptsize 40a,40b}$,
S.~Mergelmeyer$^\textrm{\scriptsize 17}$,
P.~Mermod$^\textrm{\scriptsize 52}$,
L.~Merola$^\textrm{\scriptsize 106a,106b}$,
C.~Meroni$^\textrm{\scriptsize 94a}$,
F.S.~Merritt$^\textrm{\scriptsize 33}$,
A.~Messina$^\textrm{\scriptsize 134a,134b}$,
J.~Metcalfe$^\textrm{\scriptsize 6}$,
A.S.~Mete$^\textrm{\scriptsize 166}$,
C.~Meyer$^\textrm{\scriptsize 124}$,
J-P.~Meyer$^\textrm{\scriptsize 138}$,
J.~Meyer$^\textrm{\scriptsize 109}$,
H.~Meyer~Zu~Theenhausen$^\textrm{\scriptsize 60a}$,
F.~Miano$^\textrm{\scriptsize 151}$,
R.P.~Middleton$^\textrm{\scriptsize 133}$,
S.~Miglioranzi$^\textrm{\scriptsize 53a,53b}$,
L.~Mijovi\'{c}$^\textrm{\scriptsize 49}$,
G.~Mikenberg$^\textrm{\scriptsize 175}$,
M.~Mikestikova$^\textrm{\scriptsize 129}$,
M.~Miku\v{z}$^\textrm{\scriptsize 78}$,
M.~Milesi$^\textrm{\scriptsize 91}$,
A.~Milic$^\textrm{\scriptsize 161}$,
D.W.~Miller$^\textrm{\scriptsize 33}$,
C.~Mills$^\textrm{\scriptsize 49}$,
A.~Milov$^\textrm{\scriptsize 175}$,
D.A.~Milstead$^\textrm{\scriptsize 148a,148b}$,
A.A.~Minaenko$^\textrm{\scriptsize 132}$,
Y.~Minami$^\textrm{\scriptsize 157}$,
I.A.~Minashvili$^\textrm{\scriptsize 54b}$,
A.I.~Mincer$^\textrm{\scriptsize 112}$,
B.~Mindur$^\textrm{\scriptsize 41a}$,
M.~Mineev$^\textrm{\scriptsize 68}$,
Y.~Minegishi$^\textrm{\scriptsize 157}$,
Y.~Ming$^\textrm{\scriptsize 176}$,
L.M.~Mir$^\textrm{\scriptsize 13}$,
K.P.~Mistry$^\textrm{\scriptsize 124}$,
T.~Mitani$^\textrm{\scriptsize 174}$,
J.~Mitrevski$^\textrm{\scriptsize 102}$,
V.A.~Mitsou$^\textrm{\scriptsize 170}$,
A.~Miucci$^\textrm{\scriptsize 18}$,
P.S.~Miyagawa$^\textrm{\scriptsize 141}$,
A.~Mizukami$^\textrm{\scriptsize 69}$,
J.U.~Mj\"ornmark$^\textrm{\scriptsize 84}$,
T.~Mkrtchyan$^\textrm{\scriptsize 180}$,
M.~Mlynarikova$^\textrm{\scriptsize 131}$,
T.~Moa$^\textrm{\scriptsize 148a,148b}$,
K.~Mochizuki$^\textrm{\scriptsize 97}$,
P.~Mogg$^\textrm{\scriptsize 51}$,
S.~Mohapatra$^\textrm{\scriptsize 38}$,
S.~Molander$^\textrm{\scriptsize 148a,148b}$,
R.~Moles-Valls$^\textrm{\scriptsize 23}$,
R.~Monden$^\textrm{\scriptsize 71}$,
M.C.~Mondragon$^\textrm{\scriptsize 93}$,
K.~M\"onig$^\textrm{\scriptsize 45}$,
J.~Monk$^\textrm{\scriptsize 39}$,
E.~Monnier$^\textrm{\scriptsize 88}$,
A.~Montalbano$^\textrm{\scriptsize 150}$,
J.~Montejo~Berlingen$^\textrm{\scriptsize 32}$,
F.~Monticelli$^\textrm{\scriptsize 74}$,
S.~Monzani$^\textrm{\scriptsize 94a,94b}$,
R.W.~Moore$^\textrm{\scriptsize 3}$,
N.~Morange$^\textrm{\scriptsize 119}$,
D.~Moreno$^\textrm{\scriptsize 21}$,
M.~Moreno~Ll\'acer$^\textrm{\scriptsize 32}$,
P.~Morettini$^\textrm{\scriptsize 53a}$,
S.~Morgenstern$^\textrm{\scriptsize 32}$,
D.~Mori$^\textrm{\scriptsize 144}$,
T.~Mori$^\textrm{\scriptsize 157}$,
M.~Morii$^\textrm{\scriptsize 59}$,
M.~Morinaga$^\textrm{\scriptsize 157}$,
V.~Morisbak$^\textrm{\scriptsize 121}$,
A.K.~Morley$^\textrm{\scriptsize 32}$,
G.~Mornacchi$^\textrm{\scriptsize 32}$,
J.D.~Morris$^\textrm{\scriptsize 79}$,
L.~Morvaj$^\textrm{\scriptsize 150}$,
P.~Moschovakos$^\textrm{\scriptsize 10}$,
M.~Mosidze$^\textrm{\scriptsize 54b}$,
H.J.~Moss$^\textrm{\scriptsize 141}$,
J.~Moss$^\textrm{\scriptsize 145}$$^{,ak}$,
K.~Motohashi$^\textrm{\scriptsize 159}$,
R.~Mount$^\textrm{\scriptsize 145}$,
E.~Mountricha$^\textrm{\scriptsize 27}$,
E.J.W.~Moyse$^\textrm{\scriptsize 89}$,
S.~Muanza$^\textrm{\scriptsize 88}$,
F.~Mueller$^\textrm{\scriptsize 103}$,
J.~Mueller$^\textrm{\scriptsize 127}$,
R.S.P.~Mueller$^\textrm{\scriptsize 102}$,
D.~Muenstermann$^\textrm{\scriptsize 75}$,
P.~Mullen$^\textrm{\scriptsize 56}$,
G.A.~Mullier$^\textrm{\scriptsize 18}$,
F.J.~Munoz~Sanchez$^\textrm{\scriptsize 87}$,
W.J.~Murray$^\textrm{\scriptsize 173,133}$,
H.~Musheghyan$^\textrm{\scriptsize 32}$,
M.~Mu\v{s}kinja$^\textrm{\scriptsize 78}$,
A.G.~Myagkov$^\textrm{\scriptsize 132}$$^{,al}$,
M.~Myska$^\textrm{\scriptsize 130}$,
B.P.~Nachman$^\textrm{\scriptsize 16}$,
O.~Nackenhorst$^\textrm{\scriptsize 52}$,
K.~Nagai$^\textrm{\scriptsize 122}$,
R.~Nagai$^\textrm{\scriptsize 69}$$^{,ae}$,
K.~Nagano$^\textrm{\scriptsize 69}$,
Y.~Nagasaka$^\textrm{\scriptsize 61}$,
K.~Nagata$^\textrm{\scriptsize 164}$,
M.~Nagel$^\textrm{\scriptsize 51}$,
E.~Nagy$^\textrm{\scriptsize 88}$,
A.M.~Nairz$^\textrm{\scriptsize 32}$,
Y.~Nakahama$^\textrm{\scriptsize 105}$,
K.~Nakamura$^\textrm{\scriptsize 69}$,
T.~Nakamura$^\textrm{\scriptsize 157}$,
I.~Nakano$^\textrm{\scriptsize 114}$,
R.F.~Naranjo~Garcia$^\textrm{\scriptsize 45}$,
R.~Narayan$^\textrm{\scriptsize 11}$,
D.I.~Narrias~Villar$^\textrm{\scriptsize 60a}$,
I.~Naryshkin$^\textrm{\scriptsize 125}$,
T.~Naumann$^\textrm{\scriptsize 45}$,
G.~Navarro$^\textrm{\scriptsize 21}$,
R.~Nayyar$^\textrm{\scriptsize 7}$,
H.A.~Neal$^\textrm{\scriptsize 92}$,
P.Yu.~Nechaeva$^\textrm{\scriptsize 98}$,
T.J.~Neep$^\textrm{\scriptsize 138}$,
A.~Negri$^\textrm{\scriptsize 123a,123b}$,
M.~Negrini$^\textrm{\scriptsize 22a}$,
S.~Nektarijevic$^\textrm{\scriptsize 108}$,
C.~Nellist$^\textrm{\scriptsize 119}$,
A.~Nelson$^\textrm{\scriptsize 166}$,
M.E.~Nelson$^\textrm{\scriptsize 122}$,
S.~Nemecek$^\textrm{\scriptsize 129}$,
P.~Nemethy$^\textrm{\scriptsize 112}$,
M.~Nessi$^\textrm{\scriptsize 32}$$^{,am}$,
M.S.~Neubauer$^\textrm{\scriptsize 169}$,
M.~Neumann$^\textrm{\scriptsize 178}$,
P.R.~Newman$^\textrm{\scriptsize 19}$,
T.Y.~Ng$^\textrm{\scriptsize 62c}$,
T.~Nguyen~Manh$^\textrm{\scriptsize 97}$,
R.B.~Nickerson$^\textrm{\scriptsize 122}$,
R.~Nicolaidou$^\textrm{\scriptsize 138}$,
J.~Nielsen$^\textrm{\scriptsize 139}$,
V.~Nikolaenko$^\textrm{\scriptsize 132}$$^{,al}$,
I.~Nikolic-Audit$^\textrm{\scriptsize 83}$,
K.~Nikolopoulos$^\textrm{\scriptsize 19}$,
J.K.~Nilsen$^\textrm{\scriptsize 121}$,
P.~Nilsson$^\textrm{\scriptsize 27}$,
Y.~Ninomiya$^\textrm{\scriptsize 157}$,
A.~Nisati$^\textrm{\scriptsize 134a}$,
N.~Nishu$^\textrm{\scriptsize 35c}$,
R.~Nisius$^\textrm{\scriptsize 103}$,
I.~Nitsche$^\textrm{\scriptsize 46}$,
T.~Nitta$^\textrm{\scriptsize 174}$,
T.~Nobe$^\textrm{\scriptsize 157}$,
Y.~Noguchi$^\textrm{\scriptsize 71}$,
M.~Nomachi$^\textrm{\scriptsize 120}$,
I.~Nomidis$^\textrm{\scriptsize 31}$,
M.A.~Nomura$^\textrm{\scriptsize 27}$,
T.~Nooney$^\textrm{\scriptsize 79}$,
M.~Nordberg$^\textrm{\scriptsize 32}$,
N.~Norjoharuddeen$^\textrm{\scriptsize 122}$,
O.~Novgorodova$^\textrm{\scriptsize 47}$,
M.~Nozaki$^\textrm{\scriptsize 69}$,
L.~Nozka$^\textrm{\scriptsize 117}$,
K.~Ntekas$^\textrm{\scriptsize 166}$,
E.~Nurse$^\textrm{\scriptsize 81}$,
F.~Nuti$^\textrm{\scriptsize 91}$,
K.~O'connor$^\textrm{\scriptsize 25}$,
D.C.~O'Neil$^\textrm{\scriptsize 144}$,
A.A.~O'Rourke$^\textrm{\scriptsize 45}$,
V.~O'Shea$^\textrm{\scriptsize 56}$,
F.G.~Oakham$^\textrm{\scriptsize 31}$$^{,d}$,
H.~Oberlack$^\textrm{\scriptsize 103}$,
T.~Obermann$^\textrm{\scriptsize 23}$,
J.~Ocariz$^\textrm{\scriptsize 83}$,
A.~Ochi$^\textrm{\scriptsize 70}$,
I.~Ochoa$^\textrm{\scriptsize 38}$,
J.P.~Ochoa-Ricoux$^\textrm{\scriptsize 34a}$,
S.~Oda$^\textrm{\scriptsize 73}$,
S.~Odaka$^\textrm{\scriptsize 69}$,
A.~Oh$^\textrm{\scriptsize 87}$,
S.H.~Oh$^\textrm{\scriptsize 48}$,
C.C.~Ohm$^\textrm{\scriptsize 16}$,
H.~Ohman$^\textrm{\scriptsize 168}$,
H.~Oide$^\textrm{\scriptsize 53a,53b}$,
H.~Okawa$^\textrm{\scriptsize 164}$,
Y.~Okumura$^\textrm{\scriptsize 157}$,
T.~Okuyama$^\textrm{\scriptsize 69}$,
A.~Olariu$^\textrm{\scriptsize 28b}$,
L.F.~Oleiro~Seabra$^\textrm{\scriptsize 128a}$,
S.A.~Olivares~Pino$^\textrm{\scriptsize 34a}$,
D.~Oliveira~Damazio$^\textrm{\scriptsize 27}$,
A.~Olszewski$^\textrm{\scriptsize 42}$,
J.~Olszowska$^\textrm{\scriptsize 42}$,
A.~Onofre$^\textrm{\scriptsize 128a,128e}$,
K.~Onogi$^\textrm{\scriptsize 105}$,
P.U.E.~Onyisi$^\textrm{\scriptsize 11}$$^{,aa}$,
H.~Oppen$^\textrm{\scriptsize 121}$,
M.J.~Oreglia$^\textrm{\scriptsize 33}$,
Y.~Oren$^\textrm{\scriptsize 155}$,
D.~Orestano$^\textrm{\scriptsize 136a,136b}$,
N.~Orlando$^\textrm{\scriptsize 62b}$,
R.S.~Orr$^\textrm{\scriptsize 161}$,
B.~Osculati$^\textrm{\scriptsize 53a,53b}$$^{,*}$,
R.~Ospanov$^\textrm{\scriptsize 36a}$,
G.~Otero~y~Garzon$^\textrm{\scriptsize 29}$,
H.~Otono$^\textrm{\scriptsize 73}$,
M.~Ouchrif$^\textrm{\scriptsize 137d}$,
F.~Ould-Saada$^\textrm{\scriptsize 121}$,
A.~Ouraou$^\textrm{\scriptsize 138}$,
K.P.~Oussoren$^\textrm{\scriptsize 109}$,
Q.~Ouyang$^\textrm{\scriptsize 35a}$,
M.~Owen$^\textrm{\scriptsize 56}$,
R.E.~Owen$^\textrm{\scriptsize 19}$,
V.E.~Ozcan$^\textrm{\scriptsize 20a}$,
N.~Ozturk$^\textrm{\scriptsize 8}$,
K.~Pachal$^\textrm{\scriptsize 144}$,
A.~Pacheco~Pages$^\textrm{\scriptsize 13}$,
L.~Pacheco~Rodriguez$^\textrm{\scriptsize 138}$,
C.~Padilla~Aranda$^\textrm{\scriptsize 13}$,
S.~Pagan~Griso$^\textrm{\scriptsize 16}$,
M.~Paganini$^\textrm{\scriptsize 179}$,
F.~Paige$^\textrm{\scriptsize 27}$,
G.~Palacino$^\textrm{\scriptsize 64}$,
S.~Palazzo$^\textrm{\scriptsize 40a,40b}$,
S.~Palestini$^\textrm{\scriptsize 32}$,
M.~Palka$^\textrm{\scriptsize 41b}$,
D.~Pallin$^\textrm{\scriptsize 37}$,
E.St.~Panagiotopoulou$^\textrm{\scriptsize 10}$,
I.~Panagoulias$^\textrm{\scriptsize 10}$,
C.E.~Pandini$^\textrm{\scriptsize 126a,126b}$,
J.G.~Panduro~Vazquez$^\textrm{\scriptsize 80}$,
P.~Pani$^\textrm{\scriptsize 32}$,
S.~Panitkin$^\textrm{\scriptsize 27}$,
D.~Pantea$^\textrm{\scriptsize 28b}$,
L.~Paolozzi$^\textrm{\scriptsize 52}$,
Th.D.~Papadopoulou$^\textrm{\scriptsize 10}$,
K.~Papageorgiou$^\textrm{\scriptsize 9}$$^{,s}$,
A.~Paramonov$^\textrm{\scriptsize 6}$,
D.~Paredes~Hernandez$^\textrm{\scriptsize 179}$,
A.J.~Parker$^\textrm{\scriptsize 75}$,
M.A.~Parker$^\textrm{\scriptsize 30}$,
K.A.~Parker$^\textrm{\scriptsize 45}$,
F.~Parodi$^\textrm{\scriptsize 53a,53b}$,
J.A.~Parsons$^\textrm{\scriptsize 38}$,
U.~Parzefall$^\textrm{\scriptsize 51}$,
V.R.~Pascuzzi$^\textrm{\scriptsize 161}$,
J.M.~Pasner$^\textrm{\scriptsize 139}$,
E.~Pasqualucci$^\textrm{\scriptsize 134a}$,
S.~Passaggio$^\textrm{\scriptsize 53a}$,
Fr.~Pastore$^\textrm{\scriptsize 80}$,
S.~Pataraia$^\textrm{\scriptsize 86}$,
J.R.~Pater$^\textrm{\scriptsize 87}$,
T.~Pauly$^\textrm{\scriptsize 32}$,
B.~Pearson$^\textrm{\scriptsize 103}$,
S.~Pedraza~Lopez$^\textrm{\scriptsize 170}$,
R.~Pedro$^\textrm{\scriptsize 128a,128b}$,
S.V.~Peleganchuk$^\textrm{\scriptsize 111}$$^{,c}$,
O.~Penc$^\textrm{\scriptsize 129}$,
C.~Peng$^\textrm{\scriptsize 35a,35d}$,
H.~Peng$^\textrm{\scriptsize 36a}$,
J.~Penwell$^\textrm{\scriptsize 64}$,
B.S.~Peralva$^\textrm{\scriptsize 26b}$,
M.M.~Perego$^\textrm{\scriptsize 138}$,
D.V.~Perepelitsa$^\textrm{\scriptsize 27}$,
F.~Peri$^\textrm{\scriptsize 17}$,
L.~Perini$^\textrm{\scriptsize 94a,94b}$,
H.~Pernegger$^\textrm{\scriptsize 32}$,
S.~Perrella$^\textrm{\scriptsize 106a,106b}$,
R.~Peschke$^\textrm{\scriptsize 45}$,
V.D.~Peshekhonov$^\textrm{\scriptsize 68}$$^{,*}$,
K.~Peters$^\textrm{\scriptsize 45}$,
R.F.Y.~Peters$^\textrm{\scriptsize 87}$,
B.A.~Petersen$^\textrm{\scriptsize 32}$,
T.C.~Petersen$^\textrm{\scriptsize 39}$,
E.~Petit$^\textrm{\scriptsize 58}$,
A.~Petridis$^\textrm{\scriptsize 1}$,
C.~Petridou$^\textrm{\scriptsize 156}$,
P.~Petroff$^\textrm{\scriptsize 119}$,
E.~Petrolo$^\textrm{\scriptsize 134a}$,
M.~Petrov$^\textrm{\scriptsize 122}$,
F.~Petrucci$^\textrm{\scriptsize 136a,136b}$,
N.E.~Pettersson$^\textrm{\scriptsize 89}$,
A.~Peyaud$^\textrm{\scriptsize 138}$,
R.~Pezoa$^\textrm{\scriptsize 34b}$,
F.H.~Phillips$^\textrm{\scriptsize 93}$,
P.W.~Phillips$^\textrm{\scriptsize 133}$,
G.~Piacquadio$^\textrm{\scriptsize 150}$,
E.~Pianori$^\textrm{\scriptsize 173}$,
A.~Picazio$^\textrm{\scriptsize 89}$,
E.~Piccaro$^\textrm{\scriptsize 79}$,
M.A.~Pickering$^\textrm{\scriptsize 122}$,
R.~Piegaia$^\textrm{\scriptsize 29}$,
J.E.~Pilcher$^\textrm{\scriptsize 33}$,
A.D.~Pilkington$^\textrm{\scriptsize 87}$,
A.W.J.~Pin$^\textrm{\scriptsize 87}$,
M.~Pinamonti$^\textrm{\scriptsize 135a,135b}$,
J.L.~Pinfold$^\textrm{\scriptsize 3}$,
H.~Pirumov$^\textrm{\scriptsize 45}$,
M.~Pitt$^\textrm{\scriptsize 175}$,
L.~Plazak$^\textrm{\scriptsize 146a}$,
M.-A.~Pleier$^\textrm{\scriptsize 27}$,
V.~Pleskot$^\textrm{\scriptsize 86}$,
E.~Plotnikova$^\textrm{\scriptsize 68}$,
D.~Pluth$^\textrm{\scriptsize 67}$,
P.~Podberezko$^\textrm{\scriptsize 111}$,
R.~Poettgen$^\textrm{\scriptsize 84}$,
R.~Poggi$^\textrm{\scriptsize 123a,123b}$,
L.~Poggioli$^\textrm{\scriptsize 119}$,
I.~Pogrebnyak$^\textrm{\scriptsize 93}$,
D.~Pohl$^\textrm{\scriptsize 23}$,
G.~Polesello$^\textrm{\scriptsize 123a}$,
A.~Poley$^\textrm{\scriptsize 45}$,
A.~Policicchio$^\textrm{\scriptsize 40a,40b}$,
R.~Polifka$^\textrm{\scriptsize 32}$,
A.~Polini$^\textrm{\scriptsize 22a}$,
C.S.~Pollard$^\textrm{\scriptsize 56}$,
V.~Polychronakos$^\textrm{\scriptsize 27}$,
K.~Pomm\`es$^\textrm{\scriptsize 32}$,
D.~Ponomarenko$^\textrm{\scriptsize 100}$,
L.~Pontecorvo$^\textrm{\scriptsize 134a}$,
G.A.~Popeneciu$^\textrm{\scriptsize 28d}$,
S.~Pospisil$^\textrm{\scriptsize 130}$,
K.~Potamianos$^\textrm{\scriptsize 16}$,
I.N.~Potrap$^\textrm{\scriptsize 68}$,
C.J.~Potter$^\textrm{\scriptsize 30}$,
T.~Poulsen$^\textrm{\scriptsize 84}$,
J.~Poveda$^\textrm{\scriptsize 32}$,
M.E.~Pozo~Astigarraga$^\textrm{\scriptsize 32}$,
P.~Pralavorio$^\textrm{\scriptsize 88}$,
A.~Pranko$^\textrm{\scriptsize 16}$,
S.~Prell$^\textrm{\scriptsize 67}$,
D.~Price$^\textrm{\scriptsize 87}$,
M.~Primavera$^\textrm{\scriptsize 76a}$,
S.~Prince$^\textrm{\scriptsize 90}$,
N.~Proklova$^\textrm{\scriptsize 100}$,
K.~Prokofiev$^\textrm{\scriptsize 62c}$,
F.~Prokoshin$^\textrm{\scriptsize 34b}$,
S.~Protopopescu$^\textrm{\scriptsize 27}$,
J.~Proudfoot$^\textrm{\scriptsize 6}$,
M.~Przybycien$^\textrm{\scriptsize 41a}$,
A.~Puri$^\textrm{\scriptsize 169}$,
P.~Puzo$^\textrm{\scriptsize 119}$,
J.~Qian$^\textrm{\scriptsize 92}$,
G.~Qin$^\textrm{\scriptsize 56}$,
Y.~Qin$^\textrm{\scriptsize 87}$,
A.~Quadt$^\textrm{\scriptsize 57}$,
M.~Queitsch-Maitland$^\textrm{\scriptsize 45}$,
D.~Quilty$^\textrm{\scriptsize 56}$,
S.~Raddum$^\textrm{\scriptsize 121}$,
V.~Radeka$^\textrm{\scriptsize 27}$,
V.~Radescu$^\textrm{\scriptsize 122}$,
S.K.~Radhakrishnan$^\textrm{\scriptsize 150}$,
P.~Radloff$^\textrm{\scriptsize 118}$,
P.~Rados$^\textrm{\scriptsize 91}$,
F.~Ragusa$^\textrm{\scriptsize 94a,94b}$,
G.~Rahal$^\textrm{\scriptsize 181}$,
J.A.~Raine$^\textrm{\scriptsize 87}$,
S.~Rajagopalan$^\textrm{\scriptsize 27}$,
C.~Rangel-Smith$^\textrm{\scriptsize 168}$,
T.~Rashid$^\textrm{\scriptsize 119}$,
S.~Raspopov$^\textrm{\scriptsize 5}$,
M.G.~Ratti$^\textrm{\scriptsize 94a,94b}$,
D.M.~Rauch$^\textrm{\scriptsize 45}$,
F.~Rauscher$^\textrm{\scriptsize 102}$,
S.~Rave$^\textrm{\scriptsize 86}$,
I.~Ravinovich$^\textrm{\scriptsize 175}$,
J.H.~Rawling$^\textrm{\scriptsize 87}$,
M.~Raymond$^\textrm{\scriptsize 32}$,
A.L.~Read$^\textrm{\scriptsize 121}$,
N.P.~Readioff$^\textrm{\scriptsize 58}$,
M.~Reale$^\textrm{\scriptsize 76a,76b}$,
D.M.~Rebuzzi$^\textrm{\scriptsize 123a,123b}$,
A.~Redelbach$^\textrm{\scriptsize 177}$,
G.~Redlinger$^\textrm{\scriptsize 27}$,
R.~Reece$^\textrm{\scriptsize 139}$,
R.G.~Reed$^\textrm{\scriptsize 147c}$,
K.~Reeves$^\textrm{\scriptsize 44}$,
L.~Rehnisch$^\textrm{\scriptsize 17}$,
J.~Reichert$^\textrm{\scriptsize 124}$,
A.~Reiss$^\textrm{\scriptsize 86}$,
C.~Rembser$^\textrm{\scriptsize 32}$,
H.~Ren$^\textrm{\scriptsize 35a,35d}$,
M.~Rescigno$^\textrm{\scriptsize 134a}$,
S.~Resconi$^\textrm{\scriptsize 94a}$,
E.D.~Resseguie$^\textrm{\scriptsize 124}$,
S.~Rettie$^\textrm{\scriptsize 171}$,
E.~Reynolds$^\textrm{\scriptsize 19}$,
O.L.~Rezanova$^\textrm{\scriptsize 111}$$^{,c}$,
P.~Reznicek$^\textrm{\scriptsize 131}$,
R.~Rezvani$^\textrm{\scriptsize 97}$,
R.~Richter$^\textrm{\scriptsize 103}$,
S.~Richter$^\textrm{\scriptsize 81}$,
E.~Richter-Was$^\textrm{\scriptsize 41b}$,
O.~Ricken$^\textrm{\scriptsize 23}$,
M.~Ridel$^\textrm{\scriptsize 83}$,
P.~Rieck$^\textrm{\scriptsize 103}$,
C.J.~Riegel$^\textrm{\scriptsize 178}$,
J.~Rieger$^\textrm{\scriptsize 57}$,
O.~Rifki$^\textrm{\scriptsize 115}$,
M.~Rijssenbeek$^\textrm{\scriptsize 150}$,
A.~Rimoldi$^\textrm{\scriptsize 123a,123b}$,
M.~Rimoldi$^\textrm{\scriptsize 18}$,
L.~Rinaldi$^\textrm{\scriptsize 22a}$,
G.~Ripellino$^\textrm{\scriptsize 149}$,
B.~Risti\'{c}$^\textrm{\scriptsize 32}$,
E.~Ritsch$^\textrm{\scriptsize 32}$,
I.~Riu$^\textrm{\scriptsize 13}$,
F.~Rizatdinova$^\textrm{\scriptsize 116}$,
E.~Rizvi$^\textrm{\scriptsize 79}$,
C.~Rizzi$^\textrm{\scriptsize 13}$,
R.T.~Roberts$^\textrm{\scriptsize 87}$,
S.H.~Robertson$^\textrm{\scriptsize 90}$$^{,o}$,
A.~Robichaud-Veronneau$^\textrm{\scriptsize 90}$,
D.~Robinson$^\textrm{\scriptsize 30}$,
J.E.M.~Robinson$^\textrm{\scriptsize 45}$,
A.~Robson$^\textrm{\scriptsize 56}$,
E.~Rocco$^\textrm{\scriptsize 86}$,
C.~Roda$^\textrm{\scriptsize 126a,126b}$,
Y.~Rodina$^\textrm{\scriptsize 88}$$^{,an}$,
S.~Rodriguez~Bosca$^\textrm{\scriptsize 170}$,
A.~Rodriguez~Perez$^\textrm{\scriptsize 13}$,
D.~Rodriguez~Rodriguez$^\textrm{\scriptsize 170}$,
S.~Roe$^\textrm{\scriptsize 32}$,
C.S.~Rogan$^\textrm{\scriptsize 59}$,
O.~R{\o}hne$^\textrm{\scriptsize 121}$,
J.~Roloff$^\textrm{\scriptsize 59}$,
A.~Romaniouk$^\textrm{\scriptsize 100}$,
M.~Romano$^\textrm{\scriptsize 22a,22b}$,
S.M.~Romano~Saez$^\textrm{\scriptsize 37}$,
E.~Romero~Adam$^\textrm{\scriptsize 170}$,
N.~Rompotis$^\textrm{\scriptsize 77}$,
M.~Ronzani$^\textrm{\scriptsize 51}$,
L.~Roos$^\textrm{\scriptsize 83}$,
S.~Rosati$^\textrm{\scriptsize 134a}$,
K.~Rosbach$^\textrm{\scriptsize 51}$,
P.~Rose$^\textrm{\scriptsize 139}$,
N.-A.~Rosien$^\textrm{\scriptsize 57}$,
E.~Rossi$^\textrm{\scriptsize 106a,106b}$,
L.P.~Rossi$^\textrm{\scriptsize 53a}$,
J.H.N.~Rosten$^\textrm{\scriptsize 30}$,
R.~Rosten$^\textrm{\scriptsize 140}$,
M.~Rotaru$^\textrm{\scriptsize 28b}$,
J.~Rothberg$^\textrm{\scriptsize 140}$,
D.~Rousseau$^\textrm{\scriptsize 119}$,
A.~Rozanov$^\textrm{\scriptsize 88}$,
Y.~Rozen$^\textrm{\scriptsize 154}$,
X.~Ruan$^\textrm{\scriptsize 147c}$,
F.~Rubbo$^\textrm{\scriptsize 145}$,
F.~R\"uhr$^\textrm{\scriptsize 51}$,
A.~Ruiz-Martinez$^\textrm{\scriptsize 31}$,
Z.~Rurikova$^\textrm{\scriptsize 51}$,
N.A.~Rusakovich$^\textrm{\scriptsize 68}$,
H.L.~Russell$^\textrm{\scriptsize 90}$,
J.P.~Rutherfoord$^\textrm{\scriptsize 7}$,
N.~Ruthmann$^\textrm{\scriptsize 32}$,
Y.F.~Ryabov$^\textrm{\scriptsize 125}$,
M.~Rybar$^\textrm{\scriptsize 169}$,
G.~Rybkin$^\textrm{\scriptsize 119}$,
S.~Ryu$^\textrm{\scriptsize 6}$,
A.~Ryzhov$^\textrm{\scriptsize 132}$,
G.F.~Rzehorz$^\textrm{\scriptsize 57}$,
A.F.~Saavedra$^\textrm{\scriptsize 152}$,
G.~Sabato$^\textrm{\scriptsize 109}$,
S.~Sacerdoti$^\textrm{\scriptsize 29}$,
H.F-W.~Sadrozinski$^\textrm{\scriptsize 139}$,
R.~Sadykov$^\textrm{\scriptsize 68}$,
F.~Safai~Tehrani$^\textrm{\scriptsize 134a}$,
P.~Saha$^\textrm{\scriptsize 110}$,
M.~Sahinsoy$^\textrm{\scriptsize 60a}$,
M.~Saimpert$^\textrm{\scriptsize 45}$,
M.~Saito$^\textrm{\scriptsize 157}$,
T.~Saito$^\textrm{\scriptsize 157}$,
H.~Sakamoto$^\textrm{\scriptsize 157}$,
Y.~Sakurai$^\textrm{\scriptsize 174}$,
G.~Salamanna$^\textrm{\scriptsize 136a,136b}$,
J.E.~Salazar~Loyola$^\textrm{\scriptsize 34b}$,
D.~Salek$^\textrm{\scriptsize 109}$,
P.H.~Sales~De~Bruin$^\textrm{\scriptsize 168}$,
D.~Salihagic$^\textrm{\scriptsize 103}$,
A.~Salnikov$^\textrm{\scriptsize 145}$,
J.~Salt$^\textrm{\scriptsize 170}$,
D.~Salvatore$^\textrm{\scriptsize 40a,40b}$,
F.~Salvatore$^\textrm{\scriptsize 151}$,
A.~Salvucci$^\textrm{\scriptsize 62a,62b,62c}$,
A.~Salzburger$^\textrm{\scriptsize 32}$,
D.~Sammel$^\textrm{\scriptsize 51}$,
D.~Sampsonidis$^\textrm{\scriptsize 156}$,
D.~Sampsonidou$^\textrm{\scriptsize 156}$,
J.~S\'anchez$^\textrm{\scriptsize 170}$,
V.~Sanchez~Martinez$^\textrm{\scriptsize 170}$,
A.~Sanchez~Pineda$^\textrm{\scriptsize 167a,167c}$,
H.~Sandaker$^\textrm{\scriptsize 121}$,
R.L.~Sandbach$^\textrm{\scriptsize 79}$,
C.O.~Sander$^\textrm{\scriptsize 45}$,
M.~Sandhoff$^\textrm{\scriptsize 178}$,
C.~Sandoval$^\textrm{\scriptsize 21}$,
D.P.C.~Sankey$^\textrm{\scriptsize 133}$,
M.~Sannino$^\textrm{\scriptsize 53a,53b}$,
Y.~Sano$^\textrm{\scriptsize 105}$,
A.~Sansoni$^\textrm{\scriptsize 50}$,
C.~Santoni$^\textrm{\scriptsize 37}$,
H.~Santos$^\textrm{\scriptsize 128a}$,
I.~Santoyo~Castillo$^\textrm{\scriptsize 151}$,
A.~Sapronov$^\textrm{\scriptsize 68}$,
J.G.~Saraiva$^\textrm{\scriptsize 128a,128d}$,
B.~Sarrazin$^\textrm{\scriptsize 23}$,
O.~Sasaki$^\textrm{\scriptsize 69}$,
K.~Sato$^\textrm{\scriptsize 164}$,
E.~Sauvan$^\textrm{\scriptsize 5}$,
G.~Savage$^\textrm{\scriptsize 80}$,
P.~Savard$^\textrm{\scriptsize 161}$$^{,d}$,
N.~Savic$^\textrm{\scriptsize 103}$,
C.~Sawyer$^\textrm{\scriptsize 133}$,
L.~Sawyer$^\textrm{\scriptsize 82}$$^{,u}$,
J.~Saxon$^\textrm{\scriptsize 33}$,
C.~Sbarra$^\textrm{\scriptsize 22a}$,
A.~Sbrizzi$^\textrm{\scriptsize 22a,22b}$,
T.~Scanlon$^\textrm{\scriptsize 81}$,
D.A.~Scannicchio$^\textrm{\scriptsize 166}$,
M.~Scarcella$^\textrm{\scriptsize 152}$,
J.~Schaarschmidt$^\textrm{\scriptsize 140}$,
P.~Schacht$^\textrm{\scriptsize 103}$,
B.M.~Schachtner$^\textrm{\scriptsize 102}$,
D.~Schaefer$^\textrm{\scriptsize 32}$,
L.~Schaefer$^\textrm{\scriptsize 124}$,
R.~Schaefer$^\textrm{\scriptsize 45}$,
J.~Schaeffer$^\textrm{\scriptsize 86}$,
S.~Schaepe$^\textrm{\scriptsize 23}$,
S.~Schaetzel$^\textrm{\scriptsize 60b}$,
U.~Sch\"afer$^\textrm{\scriptsize 86}$,
A.C.~Schaffer$^\textrm{\scriptsize 119}$,
D.~Schaile$^\textrm{\scriptsize 102}$,
R.D.~Schamberger$^\textrm{\scriptsize 150}$,
V.A.~Schegelsky$^\textrm{\scriptsize 125}$,
D.~Scheirich$^\textrm{\scriptsize 131}$,
M.~Schernau$^\textrm{\scriptsize 166}$,
C.~Schiavi$^\textrm{\scriptsize 53a,53b}$,
S.~Schier$^\textrm{\scriptsize 139}$,
L.K.~Schildgen$^\textrm{\scriptsize 23}$,
C.~Schillo$^\textrm{\scriptsize 51}$,
M.~Schioppa$^\textrm{\scriptsize 40a,40b}$,
S.~Schlenker$^\textrm{\scriptsize 32}$,
K.R.~Schmidt-Sommerfeld$^\textrm{\scriptsize 103}$,
K.~Schmieden$^\textrm{\scriptsize 32}$,
C.~Schmitt$^\textrm{\scriptsize 86}$,
S.~Schmitt$^\textrm{\scriptsize 45}$,
S.~Schmitz$^\textrm{\scriptsize 86}$,
U.~Schnoor$^\textrm{\scriptsize 51}$,
L.~Schoeffel$^\textrm{\scriptsize 138}$,
A.~Schoening$^\textrm{\scriptsize 60b}$,
B.D.~Schoenrock$^\textrm{\scriptsize 93}$,
E.~Schopf$^\textrm{\scriptsize 23}$,
M.~Schott$^\textrm{\scriptsize 86}$,
J.F.P.~Schouwenberg$^\textrm{\scriptsize 108}$,
J.~Schovancova$^\textrm{\scriptsize 32}$,
S.~Schramm$^\textrm{\scriptsize 52}$,
N.~Schuh$^\textrm{\scriptsize 86}$,
A.~Schulte$^\textrm{\scriptsize 86}$,
M.J.~Schultens$^\textrm{\scriptsize 23}$,
H.-C.~Schultz-Coulon$^\textrm{\scriptsize 60a}$,
H.~Schulz$^\textrm{\scriptsize 17}$,
M.~Schumacher$^\textrm{\scriptsize 51}$,
B.A.~Schumm$^\textrm{\scriptsize 139}$,
Ph.~Schune$^\textrm{\scriptsize 138}$,
A.~Schwartzman$^\textrm{\scriptsize 145}$,
T.A.~Schwarz$^\textrm{\scriptsize 92}$,
H.~Schweiger$^\textrm{\scriptsize 87}$,
Ph.~Schwemling$^\textrm{\scriptsize 138}$,
R.~Schwienhorst$^\textrm{\scriptsize 93}$,
J.~Schwindling$^\textrm{\scriptsize 138}$,
A.~Sciandra$^\textrm{\scriptsize 23}$,
G.~Sciolla$^\textrm{\scriptsize 25}$,
M.~Scornajenghi$^\textrm{\scriptsize 40a,40b}$,
F.~Scuri$^\textrm{\scriptsize 126a,126b}$,
F.~Scutti$^\textrm{\scriptsize 91}$,
J.~Searcy$^\textrm{\scriptsize 92}$,
P.~Seema$^\textrm{\scriptsize 23}$,
S.C.~Seidel$^\textrm{\scriptsize 107}$,
A.~Seiden$^\textrm{\scriptsize 139}$,
J.M.~Seixas$^\textrm{\scriptsize 26a}$,
G.~Sekhniaidze$^\textrm{\scriptsize 106a}$,
K.~Sekhon$^\textrm{\scriptsize 92}$,
S.J.~Sekula$^\textrm{\scriptsize 43}$,
N.~Semprini-Cesari$^\textrm{\scriptsize 22a,22b}$,
S.~Senkin$^\textrm{\scriptsize 37}$,
C.~Serfon$^\textrm{\scriptsize 121}$,
L.~Serin$^\textrm{\scriptsize 119}$,
L.~Serkin$^\textrm{\scriptsize 167a,167b}$,
M.~Sessa$^\textrm{\scriptsize 136a,136b}$,
R.~Seuster$^\textrm{\scriptsize 172}$,
H.~Severini$^\textrm{\scriptsize 115}$,
T.~Sfiligoj$^\textrm{\scriptsize 78}$,
F.~Sforza$^\textrm{\scriptsize 32}$,
A.~Sfyrla$^\textrm{\scriptsize 52}$,
E.~Shabalina$^\textrm{\scriptsize 57}$,
N.W.~Shaikh$^\textrm{\scriptsize 148a,148b}$,
L.Y.~Shan$^\textrm{\scriptsize 35a}$,
R.~Shang$^\textrm{\scriptsize 169}$,
J.T.~Shank$^\textrm{\scriptsize 24}$,
M.~Shapiro$^\textrm{\scriptsize 16}$,
P.B.~Shatalov$^\textrm{\scriptsize 99}$,
K.~Shaw$^\textrm{\scriptsize 167a,167b}$,
S.M.~Shaw$^\textrm{\scriptsize 87}$,
A.~Shcherbakova$^\textrm{\scriptsize 148a,148b}$,
C.Y.~Shehu$^\textrm{\scriptsize 151}$,
Y.~Shen$^\textrm{\scriptsize 115}$,
N.~Sherafati$^\textrm{\scriptsize 31}$,
P.~Sherwood$^\textrm{\scriptsize 81}$,
L.~Shi$^\textrm{\scriptsize 153}$$^{,ao}$,
S.~Shimizu$^\textrm{\scriptsize 70}$,
C.O.~Shimmin$^\textrm{\scriptsize 179}$,
M.~Shimojima$^\textrm{\scriptsize 104}$,
I.P.J.~Shipsey$^\textrm{\scriptsize 122}$,
S.~Shirabe$^\textrm{\scriptsize 73}$,
M.~Shiyakova$^\textrm{\scriptsize 68}$$^{,ap}$,
J.~Shlomi$^\textrm{\scriptsize 175}$,
A.~Shmeleva$^\textrm{\scriptsize 98}$,
D.~Shoaleh~Saadi$^\textrm{\scriptsize 97}$,
M.J.~Shochet$^\textrm{\scriptsize 33}$,
S.~Shojaii$^\textrm{\scriptsize 94a}$,
D.R.~Shope$^\textrm{\scriptsize 115}$,
S.~Shrestha$^\textrm{\scriptsize 113}$,
E.~Shulga$^\textrm{\scriptsize 100}$,
M.A.~Shupe$^\textrm{\scriptsize 7}$,
P.~Sicho$^\textrm{\scriptsize 129}$,
A.M.~Sickles$^\textrm{\scriptsize 169}$,
P.E.~Sidebo$^\textrm{\scriptsize 149}$,
E.~Sideras~Haddad$^\textrm{\scriptsize 147c}$,
O.~Sidiropoulou$^\textrm{\scriptsize 177}$,
A.~Sidoti$^\textrm{\scriptsize 22a,22b}$,
F.~Siegert$^\textrm{\scriptsize 47}$,
Dj.~Sijacki$^\textrm{\scriptsize 14}$,
J.~Silva$^\textrm{\scriptsize 128a,128d}$,
S.B.~Silverstein$^\textrm{\scriptsize 148a}$,
V.~Simak$^\textrm{\scriptsize 130}$,
L.~Simic$^\textrm{\scriptsize 14}$,
S.~Simion$^\textrm{\scriptsize 119}$,
E.~Simioni$^\textrm{\scriptsize 86}$,
B.~Simmons$^\textrm{\scriptsize 81}$,
M.~Simon$^\textrm{\scriptsize 86}$,
P.~Sinervo$^\textrm{\scriptsize 161}$,
N.B.~Sinev$^\textrm{\scriptsize 118}$,
M.~Sioli$^\textrm{\scriptsize 22a,22b}$,
G.~Siragusa$^\textrm{\scriptsize 177}$,
I.~Siral$^\textrm{\scriptsize 92}$,
S.Yu.~Sivoklokov$^\textrm{\scriptsize 101}$,
J.~Sj\"{o}lin$^\textrm{\scriptsize 148a,148b}$,
M.B.~Skinner$^\textrm{\scriptsize 75}$,
P.~Skubic$^\textrm{\scriptsize 115}$,
M.~Slater$^\textrm{\scriptsize 19}$,
T.~Slavicek$^\textrm{\scriptsize 130}$,
M.~Slawinska$^\textrm{\scriptsize 42}$,
K.~Sliwa$^\textrm{\scriptsize 165}$,
R.~Slovak$^\textrm{\scriptsize 131}$,
V.~Smakhtin$^\textrm{\scriptsize 175}$,
B.H.~Smart$^\textrm{\scriptsize 5}$,
J.~Smiesko$^\textrm{\scriptsize 146a}$,
N.~Smirnov$^\textrm{\scriptsize 100}$,
S.Yu.~Smirnov$^\textrm{\scriptsize 100}$,
Y.~Smirnov$^\textrm{\scriptsize 100}$,
L.N.~Smirnova$^\textrm{\scriptsize 101}$$^{,aq}$,
O.~Smirnova$^\textrm{\scriptsize 84}$,
J.W.~Smith$^\textrm{\scriptsize 57}$,
M.N.K.~Smith$^\textrm{\scriptsize 38}$,
R.W.~Smith$^\textrm{\scriptsize 38}$,
M.~Smizanska$^\textrm{\scriptsize 75}$,
K.~Smolek$^\textrm{\scriptsize 130}$,
A.A.~Snesarev$^\textrm{\scriptsize 98}$,
I.M.~Snyder$^\textrm{\scriptsize 118}$,
S.~Snyder$^\textrm{\scriptsize 27}$,
R.~Sobie$^\textrm{\scriptsize 172}$$^{,o}$,
F.~Socher$^\textrm{\scriptsize 47}$,
A.~Soffer$^\textrm{\scriptsize 155}$,
A.~S{\o}gaard$^\textrm{\scriptsize 49}$,
D.A.~Soh$^\textrm{\scriptsize 153}$,
G.~Sokhrannyi$^\textrm{\scriptsize 78}$,
C.A.~Solans~Sanchez$^\textrm{\scriptsize 32}$,
M.~Solar$^\textrm{\scriptsize 130}$,
E.Yu.~Soldatov$^\textrm{\scriptsize 100}$,
U.~Soldevila$^\textrm{\scriptsize 170}$,
A.A.~Solodkov$^\textrm{\scriptsize 132}$,
A.~Soloshenko$^\textrm{\scriptsize 68}$,
O.V.~Solovyanov$^\textrm{\scriptsize 132}$,
V.~Solovyev$^\textrm{\scriptsize 125}$,
P.~Sommer$^\textrm{\scriptsize 51}$,
H.~Son$^\textrm{\scriptsize 165}$,
A.~Sopczak$^\textrm{\scriptsize 130}$,
D.~Sosa$^\textrm{\scriptsize 60b}$,
C.L.~Sotiropoulou$^\textrm{\scriptsize 126a,126b}$,
R.~Soualah$^\textrm{\scriptsize 167a,167c}$,
A.M.~Soukharev$^\textrm{\scriptsize 111}$$^{,c}$,
D.~South$^\textrm{\scriptsize 45}$,
B.C.~Sowden$^\textrm{\scriptsize 80}$,
S.~Spagnolo$^\textrm{\scriptsize 76a,76b}$,
M.~Spalla$^\textrm{\scriptsize 126a,126b}$,
M.~Spangenberg$^\textrm{\scriptsize 173}$,
F.~Span\`o$^\textrm{\scriptsize 80}$,
D.~Sperlich$^\textrm{\scriptsize 17}$,
F.~Spettel$^\textrm{\scriptsize 103}$,
T.M.~Spieker$^\textrm{\scriptsize 60a}$,
R.~Spighi$^\textrm{\scriptsize 22a}$,
G.~Spigo$^\textrm{\scriptsize 32}$,
L.A.~Spiller$^\textrm{\scriptsize 91}$,
M.~Spousta$^\textrm{\scriptsize 131}$,
R.D.~St.~Denis$^\textrm{\scriptsize 56}$$^{,*}$,
A.~Stabile$^\textrm{\scriptsize 94a}$,
R.~Stamen$^\textrm{\scriptsize 60a}$,
S.~Stamm$^\textrm{\scriptsize 17}$,
E.~Stanecka$^\textrm{\scriptsize 42}$,
R.W.~Stanek$^\textrm{\scriptsize 6}$,
C.~Stanescu$^\textrm{\scriptsize 136a}$,
M.M.~Stanitzki$^\textrm{\scriptsize 45}$,
B.S.~Stapf$^\textrm{\scriptsize 109}$,
S.~Stapnes$^\textrm{\scriptsize 121}$,
E.A.~Starchenko$^\textrm{\scriptsize 132}$,
G.H.~Stark$^\textrm{\scriptsize 33}$,
J.~Stark$^\textrm{\scriptsize 58}$,
S.H~Stark$^\textrm{\scriptsize 39}$,
P.~Staroba$^\textrm{\scriptsize 129}$,
P.~Starovoitov$^\textrm{\scriptsize 60a}$,
S.~St\"arz$^\textrm{\scriptsize 32}$,
R.~Staszewski$^\textrm{\scriptsize 42}$,
P.~Steinberg$^\textrm{\scriptsize 27}$,
B.~Stelzer$^\textrm{\scriptsize 144}$,
H.J.~Stelzer$^\textrm{\scriptsize 32}$,
O.~Stelzer-Chilton$^\textrm{\scriptsize 163a}$,
H.~Stenzel$^\textrm{\scriptsize 55}$,
G.A.~Stewart$^\textrm{\scriptsize 56}$,
M.C.~Stockton$^\textrm{\scriptsize 118}$,
M.~Stoebe$^\textrm{\scriptsize 90}$,
G.~Stoicea$^\textrm{\scriptsize 28b}$,
P.~Stolte$^\textrm{\scriptsize 57}$,
S.~Stonjek$^\textrm{\scriptsize 103}$,
A.R.~Stradling$^\textrm{\scriptsize 8}$,
A.~Straessner$^\textrm{\scriptsize 47}$,
M.E.~Stramaglia$^\textrm{\scriptsize 18}$,
J.~Strandberg$^\textrm{\scriptsize 149}$,
S.~Strandberg$^\textrm{\scriptsize 148a,148b}$,
M.~Strauss$^\textrm{\scriptsize 115}$,
P.~Strizenec$^\textrm{\scriptsize 146b}$,
R.~Str\"ohmer$^\textrm{\scriptsize 177}$,
D.M.~Strom$^\textrm{\scriptsize 118}$,
R.~Stroynowski$^\textrm{\scriptsize 43}$,
A.~Strubig$^\textrm{\scriptsize 49}$,
S.A.~Stucci$^\textrm{\scriptsize 27}$,
B.~Stugu$^\textrm{\scriptsize 15}$,
N.A.~Styles$^\textrm{\scriptsize 45}$,
D.~Su$^\textrm{\scriptsize 145}$,
J.~Su$^\textrm{\scriptsize 127}$,
S.~Suchek$^\textrm{\scriptsize 60a}$,
Y.~Sugaya$^\textrm{\scriptsize 120}$,
M.~Suk$^\textrm{\scriptsize 130}$,
V.V.~Sulin$^\textrm{\scriptsize 98}$,
DMS~Sultan$^\textrm{\scriptsize 162a,162b}$,
S.~Sultansoy$^\textrm{\scriptsize 4c}$,
T.~Sumida$^\textrm{\scriptsize 71}$,
S.~Sun$^\textrm{\scriptsize 59}$,
X.~Sun$^\textrm{\scriptsize 3}$,
K.~Suruliz$^\textrm{\scriptsize 151}$,
C.J.E.~Suster$^\textrm{\scriptsize 152}$,
M.R.~Sutton$^\textrm{\scriptsize 151}$,
S.~Suzuki$^\textrm{\scriptsize 69}$,
M.~Svatos$^\textrm{\scriptsize 129}$,
M.~Swiatlowski$^\textrm{\scriptsize 33}$,
S.P.~Swift$^\textrm{\scriptsize 2}$,
I.~Sykora$^\textrm{\scriptsize 146a}$,
T.~Sykora$^\textrm{\scriptsize 131}$,
D.~Ta$^\textrm{\scriptsize 51}$,
K.~Tackmann$^\textrm{\scriptsize 45}$,
J.~Taenzer$^\textrm{\scriptsize 155}$,
A.~Taffard$^\textrm{\scriptsize 166}$,
R.~Tafirout$^\textrm{\scriptsize 163a}$,
E.~Tahirovic$^\textrm{\scriptsize 79}$,
N.~Taiblum$^\textrm{\scriptsize 155}$,
H.~Takai$^\textrm{\scriptsize 27}$,
R.~Takashima$^\textrm{\scriptsize 72}$,
E.H.~Takasugi$^\textrm{\scriptsize 103}$,
T.~Takeshita$^\textrm{\scriptsize 142}$,
Y.~Takubo$^\textrm{\scriptsize 69}$,
M.~Talby$^\textrm{\scriptsize 88}$,
A.A.~Talyshev$^\textrm{\scriptsize 111}$$^{,c}$,
J.~Tanaka$^\textrm{\scriptsize 157}$,
M.~Tanaka$^\textrm{\scriptsize 159}$,
R.~Tanaka$^\textrm{\scriptsize 119}$,
S.~Tanaka$^\textrm{\scriptsize 69}$,
R.~Tanioka$^\textrm{\scriptsize 70}$,
B.B.~Tannenwald$^\textrm{\scriptsize 113}$,
S.~Tapia~Araya$^\textrm{\scriptsize 34b}$,
S.~Tapprogge$^\textrm{\scriptsize 86}$,
S.~Tarem$^\textrm{\scriptsize 154}$,
G.F.~Tartarelli$^\textrm{\scriptsize 94a}$,
P.~Tas$^\textrm{\scriptsize 131}$,
M.~Tasevsky$^\textrm{\scriptsize 129}$,
T.~Tashiro$^\textrm{\scriptsize 71}$,
E.~Tassi$^\textrm{\scriptsize 40a,40b}$,
A.~Tavares~Delgado$^\textrm{\scriptsize 128a,128b}$,
Y.~Tayalati$^\textrm{\scriptsize 137e}$,
A.C.~Taylor$^\textrm{\scriptsize 107}$,
G.N.~Taylor$^\textrm{\scriptsize 91}$,
P.T.E.~Taylor$^\textrm{\scriptsize 91}$,
W.~Taylor$^\textrm{\scriptsize 163b}$,
P.~Teixeira-Dias$^\textrm{\scriptsize 80}$,
D.~Temple$^\textrm{\scriptsize 144}$,
H.~Ten~Kate$^\textrm{\scriptsize 32}$,
P.K.~Teng$^\textrm{\scriptsize 153}$,
J.J.~Teoh$^\textrm{\scriptsize 120}$,
F.~Tepel$^\textrm{\scriptsize 178}$,
S.~Terada$^\textrm{\scriptsize 69}$,
K.~Terashi$^\textrm{\scriptsize 157}$,
J.~Terron$^\textrm{\scriptsize 85}$,
S.~Terzo$^\textrm{\scriptsize 13}$,
M.~Testa$^\textrm{\scriptsize 50}$,
R.J.~Teuscher$^\textrm{\scriptsize 161}$$^{,o}$,
T.~Theveneaux-Pelzer$^\textrm{\scriptsize 88}$,
F.~Thiele$^\textrm{\scriptsize 39}$,
J.P.~Thomas$^\textrm{\scriptsize 19}$,
J.~Thomas-Wilsker$^\textrm{\scriptsize 80}$,
P.D.~Thompson$^\textrm{\scriptsize 19}$,
A.S.~Thompson$^\textrm{\scriptsize 56}$,
L.A.~Thomsen$^\textrm{\scriptsize 179}$,
E.~Thomson$^\textrm{\scriptsize 124}$,
M.J.~Tibbetts$^\textrm{\scriptsize 16}$,
R.E.~Ticse~Torres$^\textrm{\scriptsize 88}$,
V.O.~Tikhomirov$^\textrm{\scriptsize 98}$$^{,ar}$,
Yu.A.~Tikhonov$^\textrm{\scriptsize 111}$$^{,c}$,
S.~Timoshenko$^\textrm{\scriptsize 100}$,
P.~Tipton$^\textrm{\scriptsize 179}$,
S.~Tisserant$^\textrm{\scriptsize 88}$,
K.~Todome$^\textrm{\scriptsize 159}$,
S.~Todorova-Nova$^\textrm{\scriptsize 5}$,
S.~Todt$^\textrm{\scriptsize 47}$,
J.~Tojo$^\textrm{\scriptsize 73}$,
S.~Tok\'ar$^\textrm{\scriptsize 146a}$,
K.~Tokushuku$^\textrm{\scriptsize 69}$,
E.~Tolley$^\textrm{\scriptsize 113}$,
L.~Tomlinson$^\textrm{\scriptsize 87}$,
M.~Tomoto$^\textrm{\scriptsize 105}$,
L.~Tompkins$^\textrm{\scriptsize 145}$$^{,as}$,
K.~Toms$^\textrm{\scriptsize 107}$,
B.~Tong$^\textrm{\scriptsize 59}$,
P.~Tornambe$^\textrm{\scriptsize 51}$,
E.~Torrence$^\textrm{\scriptsize 118}$,
H.~Torres$^\textrm{\scriptsize 144}$,
E.~Torr\'o~Pastor$^\textrm{\scriptsize 140}$,
J.~Toth$^\textrm{\scriptsize 88}$$^{,at}$,
F.~Touchard$^\textrm{\scriptsize 88}$,
D.R.~Tovey$^\textrm{\scriptsize 141}$,
C.J.~Treado$^\textrm{\scriptsize 112}$,
T.~Trefzger$^\textrm{\scriptsize 177}$,
F.~Tresoldi$^\textrm{\scriptsize 151}$,
A.~Tricoli$^\textrm{\scriptsize 27}$,
I.M.~Trigger$^\textrm{\scriptsize 163a}$,
S.~Trincaz-Duvoid$^\textrm{\scriptsize 83}$,
M.F.~Tripiana$^\textrm{\scriptsize 13}$,
W.~Trischuk$^\textrm{\scriptsize 161}$,
B.~Trocm\'e$^\textrm{\scriptsize 58}$,
A.~Trofymov$^\textrm{\scriptsize 45}$,
C.~Troncon$^\textrm{\scriptsize 94a}$,
M.~Trottier-McDonald$^\textrm{\scriptsize 16}$,
M.~Trovatelli$^\textrm{\scriptsize 172}$,
L.~Truong$^\textrm{\scriptsize 147b}$,
M.~Trzebinski$^\textrm{\scriptsize 42}$,
A.~Trzupek$^\textrm{\scriptsize 42}$,
K.W.~Tsang$^\textrm{\scriptsize 62a}$,
J.C-L.~Tseng$^\textrm{\scriptsize 122}$,
P.V.~Tsiareshka$^\textrm{\scriptsize 95}$,
G.~Tsipolitis$^\textrm{\scriptsize 10}$,
N.~Tsirintanis$^\textrm{\scriptsize 9}$,
S.~Tsiskaridze$^\textrm{\scriptsize 13}$,
V.~Tsiskaridze$^\textrm{\scriptsize 51}$,
E.G.~Tskhadadze$^\textrm{\scriptsize 54a}$,
K.M.~Tsui$^\textrm{\scriptsize 62a}$,
I.I.~Tsukerman$^\textrm{\scriptsize 99}$,
V.~Tsulaia$^\textrm{\scriptsize 16}$,
S.~Tsuno$^\textrm{\scriptsize 69}$,
D.~Tsybychev$^\textrm{\scriptsize 150}$,
Y.~Tu$^\textrm{\scriptsize 62b}$,
A.~Tudorache$^\textrm{\scriptsize 28b}$,
V.~Tudorache$^\textrm{\scriptsize 28b}$,
T.T.~Tulbure$^\textrm{\scriptsize 28a}$,
A.N.~Tuna$^\textrm{\scriptsize 59}$,
S.A.~Tupputi$^\textrm{\scriptsize 22a,22b}$,
S.~Turchikhin$^\textrm{\scriptsize 68}$,
D.~Turgeman$^\textrm{\scriptsize 175}$,
I.~Turk~Cakir$^\textrm{\scriptsize 4b}$$^{,au}$,
R.~Turra$^\textrm{\scriptsize 94a}$,
P.M.~Tuts$^\textrm{\scriptsize 38}$,
G.~Ucchielli$^\textrm{\scriptsize 22a,22b}$,
I.~Ueda$^\textrm{\scriptsize 69}$,
M.~Ughetto$^\textrm{\scriptsize 148a,148b}$,
F.~Ukegawa$^\textrm{\scriptsize 164}$,
G.~Unal$^\textrm{\scriptsize 32}$,
A.~Undrus$^\textrm{\scriptsize 27}$,
G.~Unel$^\textrm{\scriptsize 166}$,
F.C.~Ungaro$^\textrm{\scriptsize 91}$,
Y.~Unno$^\textrm{\scriptsize 69}$,
C.~Unverdorben$^\textrm{\scriptsize 102}$,
J.~Urban$^\textrm{\scriptsize 146b}$,
P.~Urquijo$^\textrm{\scriptsize 91}$,
P.~Urrejola$^\textrm{\scriptsize 86}$,
G.~Usai$^\textrm{\scriptsize 8}$,
J.~Usui$^\textrm{\scriptsize 69}$,
L.~Vacavant$^\textrm{\scriptsize 88}$,
V.~Vacek$^\textrm{\scriptsize 130}$,
B.~Vachon$^\textrm{\scriptsize 90}$,
K.O.H.~Vadla$^\textrm{\scriptsize 121}$,
A.~Vaidya$^\textrm{\scriptsize 81}$,
C.~Valderanis$^\textrm{\scriptsize 102}$,
E.~Valdes~Santurio$^\textrm{\scriptsize 148a,148b}$,
M.~Valente$^\textrm{\scriptsize 52}$,
S.~Valentinetti$^\textrm{\scriptsize 22a,22b}$,
A.~Valero$^\textrm{\scriptsize 170}$,
L.~Val\'ery$^\textrm{\scriptsize 13}$,
S.~Valkar$^\textrm{\scriptsize 131}$,
A.~Vallier$^\textrm{\scriptsize 5}$,
J.A.~Valls~Ferrer$^\textrm{\scriptsize 170}$,
W.~Van~Den~Wollenberg$^\textrm{\scriptsize 109}$,
H.~van~der~Graaf$^\textrm{\scriptsize 109}$,
P.~van~Gemmeren$^\textrm{\scriptsize 6}$,
J.~Van~Nieuwkoop$^\textrm{\scriptsize 144}$,
I.~van~Vulpen$^\textrm{\scriptsize 109}$,
M.C.~van~Woerden$^\textrm{\scriptsize 109}$,
M.~Vanadia$^\textrm{\scriptsize 135a,135b}$,
W.~Vandelli$^\textrm{\scriptsize 32}$,
A.~Vaniachine$^\textrm{\scriptsize 160}$,
P.~Vankov$^\textrm{\scriptsize 109}$,
G.~Vardanyan$^\textrm{\scriptsize 180}$,
R.~Vari$^\textrm{\scriptsize 134a}$,
E.W.~Varnes$^\textrm{\scriptsize 7}$,
C.~Varni$^\textrm{\scriptsize 53a,53b}$,
T.~Varol$^\textrm{\scriptsize 43}$,
D.~Varouchas$^\textrm{\scriptsize 119}$,
A.~Vartapetian$^\textrm{\scriptsize 8}$,
K.E.~Varvell$^\textrm{\scriptsize 152}$,
J.G.~Vasquez$^\textrm{\scriptsize 179}$,
G.A.~Vasquez$^\textrm{\scriptsize 34b}$,
F.~Vazeille$^\textrm{\scriptsize 37}$,
T.~Vazquez~Schroeder$^\textrm{\scriptsize 90}$,
J.~Veatch$^\textrm{\scriptsize 57}$,
V.~Veeraraghavan$^\textrm{\scriptsize 7}$,
L.M.~Veloce$^\textrm{\scriptsize 161}$,
F.~Veloso$^\textrm{\scriptsize 128a,128c}$,
S.~Veneziano$^\textrm{\scriptsize 134a}$,
A.~Ventura$^\textrm{\scriptsize 76a,76b}$,
M.~Venturi$^\textrm{\scriptsize 172}$,
N.~Venturi$^\textrm{\scriptsize 32}$,
A.~Venturini$^\textrm{\scriptsize 25}$,
V.~Vercesi$^\textrm{\scriptsize 123a}$,
M.~Verducci$^\textrm{\scriptsize 136a,136b}$,
W.~Verkerke$^\textrm{\scriptsize 109}$,
A.T.~Vermeulen$^\textrm{\scriptsize 109}$,
J.C.~Vermeulen$^\textrm{\scriptsize 109}$,
M.C.~Vetterli$^\textrm{\scriptsize 144}$$^{,d}$,
N.~Viaux~Maira$^\textrm{\scriptsize 34b}$,
O.~Viazlo$^\textrm{\scriptsize 84}$,
I.~Vichou$^\textrm{\scriptsize 169}$$^{,*}$,
T.~Vickey$^\textrm{\scriptsize 141}$,
O.E.~Vickey~Boeriu$^\textrm{\scriptsize 141}$,
G.H.A.~Viehhauser$^\textrm{\scriptsize 122}$,
S.~Viel$^\textrm{\scriptsize 16}$,
L.~Vigani$^\textrm{\scriptsize 122}$,
M.~Villa$^\textrm{\scriptsize 22a,22b}$,
M.~Villaplana~Perez$^\textrm{\scriptsize 94a,94b}$,
E.~Vilucchi$^\textrm{\scriptsize 50}$,
M.G.~Vincter$^\textrm{\scriptsize 31}$,
V.B.~Vinogradov$^\textrm{\scriptsize 68}$,
A.~Vishwakarma$^\textrm{\scriptsize 45}$,
C.~Vittori$^\textrm{\scriptsize 22a,22b}$,
I.~Vivarelli$^\textrm{\scriptsize 151}$,
S.~Vlachos$^\textrm{\scriptsize 10}$,
M.~Vogel$^\textrm{\scriptsize 178}$,
P.~Vokac$^\textrm{\scriptsize 130}$,
G.~Volpi$^\textrm{\scriptsize 126a,126b}$,
H.~von~der~Schmitt$^\textrm{\scriptsize 103}$,
E.~von~Toerne$^\textrm{\scriptsize 23}$,
V.~Vorobel$^\textrm{\scriptsize 131}$,
K.~Vorobev$^\textrm{\scriptsize 100}$,
M.~Vos$^\textrm{\scriptsize 170}$,
R.~Voss$^\textrm{\scriptsize 32}$,
J.H.~Vossebeld$^\textrm{\scriptsize 77}$,
N.~Vranjes$^\textrm{\scriptsize 14}$,
M.~Vranjes~Milosavljevic$^\textrm{\scriptsize 14}$,
V.~Vrba$^\textrm{\scriptsize 130}$,
M.~Vreeswijk$^\textrm{\scriptsize 109}$,
R.~Vuillermet$^\textrm{\scriptsize 32}$,
I.~Vukotic$^\textrm{\scriptsize 33}$,
P.~Wagner$^\textrm{\scriptsize 23}$,
W.~Wagner$^\textrm{\scriptsize 178}$,
J.~Wagner-Kuhr$^\textrm{\scriptsize 102}$,
H.~Wahlberg$^\textrm{\scriptsize 74}$,
S.~Wahrmund$^\textrm{\scriptsize 47}$,
J.~Wakabayashi$^\textrm{\scriptsize 105}$,
J.~Walder$^\textrm{\scriptsize 75}$,
R.~Walker$^\textrm{\scriptsize 102}$,
W.~Walkowiak$^\textrm{\scriptsize 143}$,
V.~Wallangen$^\textrm{\scriptsize 148a,148b}$,
C.~Wang$^\textrm{\scriptsize 35b}$,
C.~Wang$^\textrm{\scriptsize 36b}$$^{,av}$,
F.~Wang$^\textrm{\scriptsize 176}$,
H.~Wang$^\textrm{\scriptsize 16}$,
H.~Wang$^\textrm{\scriptsize 3}$,
J.~Wang$^\textrm{\scriptsize 45}$,
J.~Wang$^\textrm{\scriptsize 152}$,
Q.~Wang$^\textrm{\scriptsize 115}$,
R.~Wang$^\textrm{\scriptsize 6}$,
S.M.~Wang$^\textrm{\scriptsize 153}$,
T.~Wang$^\textrm{\scriptsize 38}$,
W.~Wang$^\textrm{\scriptsize 153}$$^{,aw}$,
W.~Wang$^\textrm{\scriptsize 36a}$$^{,ax}$,
Z.~Wang$^\textrm{\scriptsize 36c}$,
C.~Wanotayaroj$^\textrm{\scriptsize 118}$,
A.~Warburton$^\textrm{\scriptsize 90}$,
C.P.~Ward$^\textrm{\scriptsize 30}$,
D.R.~Wardrope$^\textrm{\scriptsize 81}$,
A.~Washbrook$^\textrm{\scriptsize 49}$,
P.M.~Watkins$^\textrm{\scriptsize 19}$,
A.T.~Watson$^\textrm{\scriptsize 19}$,
M.F.~Watson$^\textrm{\scriptsize 19}$,
G.~Watts$^\textrm{\scriptsize 140}$,
S.~Watts$^\textrm{\scriptsize 87}$,
B.M.~Waugh$^\textrm{\scriptsize 81}$,
A.F.~Webb$^\textrm{\scriptsize 11}$,
S.~Webb$^\textrm{\scriptsize 86}$,
M.S.~Weber$^\textrm{\scriptsize 18}$,
S.W.~Weber$^\textrm{\scriptsize 177}$,
S.A.~Weber$^\textrm{\scriptsize 31}$,
J.S.~Webster$^\textrm{\scriptsize 6}$,
A.R.~Weidberg$^\textrm{\scriptsize 122}$,
B.~Weinert$^\textrm{\scriptsize 64}$,
J.~Weingarten$^\textrm{\scriptsize 57}$,
M.~Weirich$^\textrm{\scriptsize 86}$,
C.~Weiser$^\textrm{\scriptsize 51}$,
H.~Weits$^\textrm{\scriptsize 109}$,
P.S.~Wells$^\textrm{\scriptsize 32}$,
T.~Wenaus$^\textrm{\scriptsize 27}$,
T.~Wengler$^\textrm{\scriptsize 32}$,
S.~Wenig$^\textrm{\scriptsize 32}$,
N.~Wermes$^\textrm{\scriptsize 23}$,
M.D.~Werner$^\textrm{\scriptsize 67}$,
P.~Werner$^\textrm{\scriptsize 32}$,
M.~Wessels$^\textrm{\scriptsize 60a}$,
T.D.~Weston$^\textrm{\scriptsize 18}$,
K.~Whalen$^\textrm{\scriptsize 118}$,
N.L.~Whallon$^\textrm{\scriptsize 140}$,
A.M.~Wharton$^\textrm{\scriptsize 75}$,
A.S.~White$^\textrm{\scriptsize 92}$,
A.~White$^\textrm{\scriptsize 8}$,
M.J.~White$^\textrm{\scriptsize 1}$,
R.~White$^\textrm{\scriptsize 34b}$,
D.~Whiteson$^\textrm{\scriptsize 166}$,
B.W.~Whitmore$^\textrm{\scriptsize 75}$,
F.J.~Wickens$^\textrm{\scriptsize 133}$,
W.~Wiedenmann$^\textrm{\scriptsize 176}$,
M.~Wielers$^\textrm{\scriptsize 133}$,
C.~Wiglesworth$^\textrm{\scriptsize 39}$,
L.A.M.~Wiik-Fuchs$^\textrm{\scriptsize 51}$,
A.~Wildauer$^\textrm{\scriptsize 103}$,
F.~Wilk$^\textrm{\scriptsize 87}$,
H.G.~Wilkens$^\textrm{\scriptsize 32}$,
H.H.~Williams$^\textrm{\scriptsize 124}$,
S.~Williams$^\textrm{\scriptsize 109}$,
C.~Willis$^\textrm{\scriptsize 93}$,
S.~Willocq$^\textrm{\scriptsize 89}$,
J.A.~Wilson$^\textrm{\scriptsize 19}$,
I.~Wingerter-Seez$^\textrm{\scriptsize 5}$,
E.~Winkels$^\textrm{\scriptsize 151}$,
F.~Winklmeier$^\textrm{\scriptsize 118}$,
O.J.~Winston$^\textrm{\scriptsize 151}$,
B.T.~Winter$^\textrm{\scriptsize 23}$,
M.~Wittgen$^\textrm{\scriptsize 145}$,
M.~Wobisch$^\textrm{\scriptsize 82}$$^{,u}$,
T.M.H.~Wolf$^\textrm{\scriptsize 109}$,
R.~Wolff$^\textrm{\scriptsize 88}$,
M.W.~Wolter$^\textrm{\scriptsize 42}$,
H.~Wolters$^\textrm{\scriptsize 128a,128c}$,
V.W.S.~Wong$^\textrm{\scriptsize 171}$,
S.D.~Worm$^\textrm{\scriptsize 19}$,
B.K.~Wosiek$^\textrm{\scriptsize 42}$,
J.~Wotschack$^\textrm{\scriptsize 32}$,
K.W.~Wozniak$^\textrm{\scriptsize 42}$,
M.~Wu$^\textrm{\scriptsize 33}$,
S.L.~Wu$^\textrm{\scriptsize 176}$,
X.~Wu$^\textrm{\scriptsize 52}$,
Y.~Wu$^\textrm{\scriptsize 92}$,
T.R.~Wyatt$^\textrm{\scriptsize 87}$,
B.M.~Wynne$^\textrm{\scriptsize 49}$,
S.~Xella$^\textrm{\scriptsize 39}$,
Z.~Xi$^\textrm{\scriptsize 92}$,
L.~Xia$^\textrm{\scriptsize 35c}$,
D.~Xu$^\textrm{\scriptsize 35a}$,
L.~Xu$^\textrm{\scriptsize 27}$,
T.~Xu$^\textrm{\scriptsize 138}$,
B.~Yabsley$^\textrm{\scriptsize 152}$,
S.~Yacoob$^\textrm{\scriptsize 147a}$,
D.~Yamaguchi$^\textrm{\scriptsize 159}$,
Y.~Yamaguchi$^\textrm{\scriptsize 120}$,
A.~Yamamoto$^\textrm{\scriptsize 69}$,
S.~Yamamoto$^\textrm{\scriptsize 157}$,
T.~Yamanaka$^\textrm{\scriptsize 157}$,
M.~Yamatani$^\textrm{\scriptsize 157}$,
K.~Yamauchi$^\textrm{\scriptsize 105}$,
Y.~Yamazaki$^\textrm{\scriptsize 70}$,
Z.~Yan$^\textrm{\scriptsize 24}$,
H.~Yang$^\textrm{\scriptsize 36c}$,
H.~Yang$^\textrm{\scriptsize 16}$,
Y.~Yang$^\textrm{\scriptsize 153}$,
Z.~Yang$^\textrm{\scriptsize 15}$,
W-M.~Yao$^\textrm{\scriptsize 16}$,
Y.C.~Yap$^\textrm{\scriptsize 83}$,
Y.~Yasu$^\textrm{\scriptsize 69}$,
E.~Yatsenko$^\textrm{\scriptsize 5}$,
K.H.~Yau~Wong$^\textrm{\scriptsize 23}$,
J.~Ye$^\textrm{\scriptsize 43}$,
S.~Ye$^\textrm{\scriptsize 27}$,
I.~Yeletskikh$^\textrm{\scriptsize 68}$,
E.~Yigitbasi$^\textrm{\scriptsize 24}$,
E.~Yildirim$^\textrm{\scriptsize 86}$,
K.~Yorita$^\textrm{\scriptsize 174}$,
K.~Yoshihara$^\textrm{\scriptsize 124}$,
C.~Young$^\textrm{\scriptsize 145}$,
C.J.S.~Young$^\textrm{\scriptsize 32}$,
J.~Yu$^\textrm{\scriptsize 8}$,
J.~Yu$^\textrm{\scriptsize 67}$,
S.P.Y.~Yuen$^\textrm{\scriptsize 23}$,
I.~Yusuff$^\textrm{\scriptsize 30}$$^{,ay}$,
B.~Zabinski$^\textrm{\scriptsize 42}$,
G.~Zacharis$^\textrm{\scriptsize 10}$,
R.~Zaidan$^\textrm{\scriptsize 13}$,
A.M.~Zaitsev$^\textrm{\scriptsize 132}$$^{,al}$,
N.~Zakharchuk$^\textrm{\scriptsize 45}$,
J.~Zalieckas$^\textrm{\scriptsize 15}$,
A.~Zaman$^\textrm{\scriptsize 150}$,
S.~Zambito$^\textrm{\scriptsize 59}$,
D.~Zanzi$^\textrm{\scriptsize 91}$,
C.~Zeitnitz$^\textrm{\scriptsize 178}$,
G.~Zemaityte$^\textrm{\scriptsize 122}$,
A.~Zemla$^\textrm{\scriptsize 41a}$,
J.C.~Zeng$^\textrm{\scriptsize 169}$,
Q.~Zeng$^\textrm{\scriptsize 145}$,
O.~Zenin$^\textrm{\scriptsize 132}$,
T.~\v{Z}eni\v{s}$^\textrm{\scriptsize 146a}$,
D.~Zerwas$^\textrm{\scriptsize 119}$,
D.~Zhang$^\textrm{\scriptsize 92}$,
F.~Zhang$^\textrm{\scriptsize 176}$,
G.~Zhang$^\textrm{\scriptsize 36a}$$^{,ax}$,
H.~Zhang$^\textrm{\scriptsize 35b}$,
J.~Zhang$^\textrm{\scriptsize 6}$,
L.~Zhang$^\textrm{\scriptsize 51}$,
L.~Zhang$^\textrm{\scriptsize 36a}$,
M.~Zhang$^\textrm{\scriptsize 169}$,
P.~Zhang$^\textrm{\scriptsize 35b}$,
R.~Zhang$^\textrm{\scriptsize 23}$,
R.~Zhang$^\textrm{\scriptsize 36a}$$^{,av}$,
X.~Zhang$^\textrm{\scriptsize 36b}$,
Y.~Zhang$^\textrm{\scriptsize 35a,35d}$,
Z.~Zhang$^\textrm{\scriptsize 119}$,
X.~Zhao$^\textrm{\scriptsize 43}$,
Y.~Zhao$^\textrm{\scriptsize 36b}$$^{,az}$,
Z.~Zhao$^\textrm{\scriptsize 36a}$,
A.~Zhemchugov$^\textrm{\scriptsize 68}$,
B.~Zhou$^\textrm{\scriptsize 92}$,
C.~Zhou$^\textrm{\scriptsize 176}$,
L.~Zhou$^\textrm{\scriptsize 43}$,
M.~Zhou$^\textrm{\scriptsize 35a,35d}$,
M.~Zhou$^\textrm{\scriptsize 150}$,
N.~Zhou$^\textrm{\scriptsize 35c}$,
C.G.~Zhu$^\textrm{\scriptsize 36b}$,
H.~Zhu$^\textrm{\scriptsize 35a}$,
J.~Zhu$^\textrm{\scriptsize 92}$,
Y.~Zhu$^\textrm{\scriptsize 36a}$,
X.~Zhuang$^\textrm{\scriptsize 35a}$,
K.~Zhukov$^\textrm{\scriptsize 98}$,
A.~Zibell$^\textrm{\scriptsize 177}$,
D.~Zieminska$^\textrm{\scriptsize 64}$,
N.I.~Zimine$^\textrm{\scriptsize 68}$,
C.~Zimmermann$^\textrm{\scriptsize 86}$,
S.~Zimmermann$^\textrm{\scriptsize 51}$,
Z.~Zinonos$^\textrm{\scriptsize 103}$,
M.~Zinser$^\textrm{\scriptsize 86}$,
M.~Ziolkowski$^\textrm{\scriptsize 143}$,
L.~\v{Z}ivkovi\'{c}$^\textrm{\scriptsize 14}$,
G.~Zobernig$^\textrm{\scriptsize 176}$,
A.~Zoccoli$^\textrm{\scriptsize 22a,22b}$,
R.~Zou$^\textrm{\scriptsize 33}$,
M.~zur~Nedden$^\textrm{\scriptsize 17}$,
L.~Zwalinski$^\textrm{\scriptsize 32}$.
\bigskip
\\
$^{1}$ Department of Physics, University of Adelaide, Adelaide, Australia\\
$^{2}$ Physics Department, SUNY Albany, Albany NY, United States of America\\
$^{3}$ Department of Physics, University of Alberta, Edmonton AB, Canada\\
$^{4}$ $^{(a)}$ Department of Physics, Ankara University, Ankara; $^{(b)}$ Istanbul Aydin University, Istanbul; $^{(c)}$ Division of Physics, TOBB University of Economics and Technology, Ankara, Turkey\\
$^{5}$ LAPP, CNRS/IN2P3 and Universit{\'e} Savoie Mont Blanc, Annecy-le-Vieux, France\\
$^{6}$ High Energy Physics Division, Argonne National Laboratory, Argonne IL, United States of America\\
$^{7}$ Department of Physics, University of Arizona, Tucson AZ, United States of America\\
$^{8}$ Department of Physics, The University of Texas at Arlington, Arlington TX, United States of America\\
$^{9}$ Physics Department, National and Kapodistrian University of Athens, Athens, Greece\\
$^{10}$ Physics Department, National Technical University of Athens, Zografou, Greece\\
$^{11}$ Department of Physics, The University of Texas at Austin, Austin TX, United States of America\\
$^{12}$ Institute of Physics, Azerbaijan Academy of Sciences, Baku, Azerbaijan\\
$^{13}$ Institut de F{\'\i}sica d'Altes Energies (IFAE), The Barcelona Institute of Science and Technology, Barcelona, Spain\\
$^{14}$ Institute of Physics, University of Belgrade, Belgrade, Serbia\\
$^{15}$ Department for Physics and Technology, University of Bergen, Bergen, Norway\\
$^{16}$ Physics Division, Lawrence Berkeley National Laboratory and University of California, Berkeley CA, United States of America\\
$^{17}$ Department of Physics, Humboldt University, Berlin, Germany\\
$^{18}$ Albert Einstein Center for Fundamental Physics and Laboratory for High Energy Physics, University of Bern, Bern, Switzerland\\
$^{19}$ School of Physics and Astronomy, University of Birmingham, Birmingham, United Kingdom\\
$^{20}$ $^{(a)}$ Department of Physics, Bogazici University, Istanbul; $^{(b)}$ Department of Physics Engineering, Gaziantep University, Gaziantep; $^{(d)}$ Istanbul Bilgi University, Faculty of Engineering and Natural Sciences, Istanbul; $^{(e)}$ Bahcesehir University, Faculty of Engineering and Natural Sciences, Istanbul, Turkey\\
$^{21}$ Centro de Investigaciones, Universidad Antonio Narino, Bogota, Colombia\\
$^{22}$ $^{(a)}$ INFN Sezione di Bologna; $^{(b)}$ Dipartimento di Fisica e Astronomia, Universit{\`a} di Bologna, Bologna, Italy\\
$^{23}$ Physikalisches Institut, University of Bonn, Bonn, Germany\\
$^{24}$ Department of Physics, Boston University, Boston MA, United States of America\\
$^{25}$ Department of Physics, Brandeis University, Waltham MA, United States of America\\
$^{26}$ $^{(a)}$ Universidade Federal do Rio De Janeiro COPPE/EE/IF, Rio de Janeiro; $^{(b)}$ Electrical Circuits Department, Federal University of Juiz de Fora (UFJF), Juiz de Fora; $^{(c)}$ Federal University of Sao Joao del Rei (UFSJ), Sao Joao del Rei; $^{(d)}$ Instituto de Fisica, Universidade de Sao Paulo, Sao Paulo, Brazil\\
$^{27}$ Physics Department, Brookhaven National Laboratory, Upton NY, United States of America\\
$^{28}$ $^{(a)}$ Transilvania University of Brasov, Brasov; $^{(b)}$ Horia Hulubei National Institute of Physics and Nuclear Engineering, Bucharest; $^{(c)}$ Department of Physics, Alexandru Ioan Cuza University of Iasi, Iasi; $^{(d)}$ National Institute for Research and Development of Isotopic and Molecular Technologies, Physics Department, Cluj Napoca; $^{(e)}$ University Politehnica Bucharest, Bucharest; $^{(f)}$ West University in Timisoara, Timisoara, Romania\\
$^{29}$ Departamento de F{\'\i}sica, Universidad de Buenos Aires, Buenos Aires, Argentina\\
$^{30}$ Cavendish Laboratory, University of Cambridge, Cambridge, United Kingdom\\
$^{31}$ Department of Physics, Carleton University, Ottawa ON, Canada\\
$^{32}$ CERN, Geneva, Switzerland\\
$^{33}$ Enrico Fermi Institute, University of Chicago, Chicago IL, United States of America\\
$^{34}$ $^{(a)}$ Departamento de F{\'\i}sica, Pontificia Universidad Cat{\'o}lica de Chile, Santiago; $^{(b)}$ Departamento de F{\'\i}sica, Universidad T{\'e}cnica Federico Santa Mar{\'\i}a, Valpara{\'\i}so, Chile\\
$^{35}$ $^{(a)}$ Institute of High Energy Physics, Chinese Academy of Sciences, Beijing; $^{(b)}$ Department of Physics, Nanjing University, Jiangsu; $^{(c)}$ Physics Department, Tsinghua University, Beijing 100084; $^{(d)}$ University of Chinese Academy of Science (UCAS), Beijing, China\\
$^{36}$ $^{(a)}$ Department of Modern Physics and State Key Laboratory of Particle Detection and Electronics, University of Science and Technology of China, Anhui; $^{(b)}$ School of Physics, Shandong University, Shandong; $^{(c)}$ Department of Physics and Astronomy, Key Laboratory for Particle Physics, Astrophysics and Cosmology, Ministry of Education; Shanghai Key Laboratory for Particle Physics and Cosmology, Shanghai Jiao Tong University, Shanghai(also at PKU-CHEP), China\\
$^{37}$ Universit{\'e} Clermont Auvergne, CNRS/IN2P3, LPC, Clermont-Ferrand, France\\
$^{38}$ Nevis Laboratory, Columbia University, Irvington NY, United States of America\\
$^{39}$ Niels Bohr Institute, University of Copenhagen, Kobenhavn, Denmark\\
$^{40}$ $^{(a)}$ INFN Gruppo Collegato di Cosenza, Laboratori Nazionali di Frascati; $^{(b)}$ Dipartimento di Fisica, Universit{\`a} della Calabria, Rende, Italy\\
$^{41}$ $^{(a)}$ AGH University of Science and Technology, Faculty of Physics and Applied Computer Science, Krakow; $^{(b)}$ Marian Smoluchowski Institute of Physics, Jagiellonian University, Krakow, Poland\\
$^{42}$ Institute of Nuclear Physics Polish Academy of Sciences, Krakow, Poland\\
$^{43}$ Physics Department, Southern Methodist University, Dallas TX, United States of America\\
$^{44}$ Physics Department, University of Texas at Dallas, Richardson TX, United States of America\\
$^{45}$ DESY, Hamburg and Zeuthen, Germany\\
$^{46}$ Lehrstuhl f{\"u}r Experimentelle Physik IV, Technische Universit{\"a}t Dortmund, Dortmund, Germany\\
$^{47}$ Institut f{\"u}r Kern-{~}und Teilchenphysik, Technische Universit{\"a}t Dresden, Dresden, Germany\\
$^{48}$ Department of Physics, Duke University, Durham NC, United States of America\\
$^{49}$ SUPA - School of Physics and Astronomy, University of Edinburgh, Edinburgh, United Kingdom\\
$^{50}$ INFN e Laboratori Nazionali di Frascati, Frascati, Italy\\
$^{51}$ Fakult{\"a}t f{\"u}r Mathematik und Physik, Albert-Ludwigs-Universit{\"a}t, Freiburg, Germany\\
$^{52}$ Departement  de Physique Nucleaire et Corpusculaire, Universit{\'e} de Gen{\`e}ve, Geneva, Switzerland\\
$^{53}$ $^{(a)}$ INFN Sezione di Genova; $^{(b)}$ Dipartimento di Fisica, Universit{\`a} di Genova, Genova, Italy\\
$^{54}$ $^{(a)}$ E. Andronikashvili Institute of Physics, Iv. Javakhishvili Tbilisi State University, Tbilisi; $^{(b)}$ High Energy Physics Institute, Tbilisi State University, Tbilisi, Georgia\\
$^{55}$ II Physikalisches Institut, Justus-Liebig-Universit{\"a}t Giessen, Giessen, Germany\\
$^{56}$ SUPA - School of Physics and Astronomy, University of Glasgow, Glasgow, United Kingdom\\
$^{57}$ II Physikalisches Institut, Georg-August-Universit{\"a}t, G{\"o}ttingen, Germany\\
$^{58}$ Laboratoire de Physique Subatomique et de Cosmologie, Universit{\'e} Grenoble-Alpes, CNRS/IN2P3, Grenoble, France\\
$^{59}$ Laboratory for Particle Physics and Cosmology, Harvard University, Cambridge MA, United States of America\\
$^{60}$ $^{(a)}$ Kirchhoff-Institut f{\"u}r Physik, Ruprecht-Karls-Universit{\"a}t Heidelberg, Heidelberg; $^{(b)}$ Physikalisches Institut, Ruprecht-Karls-Universit{\"a}t Heidelberg, Heidelberg, Germany\\
$^{61}$ Faculty of Applied Information Science, Hiroshima Institute of Technology, Hiroshima, Japan\\
$^{62}$ $^{(a)}$ Department of Physics, The Chinese University of Hong Kong, Shatin, N.T., Hong Kong; $^{(b)}$ Department of Physics, The University of Hong Kong, Hong Kong; $^{(c)}$ Department of Physics and Institute for Advanced Study, The Hong Kong University of Science and Technology, Clear Water Bay, Kowloon, Hong Kong, China\\
$^{63}$ Department of Physics, National Tsing Hua University, Taiwan, Taiwan\\
$^{64}$ Department of Physics, Indiana University, Bloomington IN, United States of America\\
$^{65}$ Institut f{\"u}r Astro-{~}und Teilchenphysik, Leopold-Franzens-Universit{\"a}t, Innsbruck, Austria\\
$^{66}$ University of Iowa, Iowa City IA, United States of America\\
$^{67}$ Department of Physics and Astronomy, Iowa State University, Ames IA, United States of America\\
$^{68}$ Joint Institute for Nuclear Research, JINR Dubna, Dubna, Russia\\
$^{69}$ KEK, High Energy Accelerator Research Organization, Tsukuba, Japan\\
$^{70}$ Graduate School of Science, Kobe University, Kobe, Japan\\
$^{71}$ Faculty of Science, Kyoto University, Kyoto, Japan\\
$^{72}$ Kyoto University of Education, Kyoto, Japan\\
$^{73}$ Research Center for Advanced Particle Physics and Department of Physics, Kyushu University, Fukuoka, Japan\\
$^{74}$ Instituto de F{\'\i}sica La Plata, Universidad Nacional de La Plata and CONICET, La Plata, Argentina\\
$^{75}$ Physics Department, Lancaster University, Lancaster, United Kingdom\\
$^{76}$ $^{(a)}$ INFN Sezione di Lecce; $^{(b)}$ Dipartimento di Matematica e Fisica, Universit{\`a} del Salento, Lecce, Italy\\
$^{77}$ Oliver Lodge Laboratory, University of Liverpool, Liverpool, United Kingdom\\
$^{78}$ Department of Experimental Particle Physics, Jo{\v{z}}ef Stefan Institute and Department of Physics, University of Ljubljana, Ljubljana, Slovenia\\
$^{79}$ School of Physics and Astronomy, Queen Mary University of London, London, United Kingdom\\
$^{80}$ Department of Physics, Royal Holloway University of London, Surrey, United Kingdom\\
$^{81}$ Department of Physics and Astronomy, University College London, London, United Kingdom\\
$^{82}$ Louisiana Tech University, Ruston LA, United States of America\\
$^{83}$ Laboratoire de Physique Nucl{\'e}aire et de Hautes Energies, UPMC and Universit{\'e} Paris-Diderot and CNRS/IN2P3, Paris, France\\
$^{84}$ Fysiska institutionen, Lunds universitet, Lund, Sweden\\
$^{85}$ Departamento de Fisica Teorica C-15, Universidad Autonoma de Madrid, Madrid, Spain\\
$^{86}$ Institut f{\"u}r Physik, Universit{\"a}t Mainz, Mainz, Germany\\
$^{87}$ School of Physics and Astronomy, University of Manchester, Manchester, United Kingdom\\
$^{88}$ CPPM, Aix-Marseille Universit{\'e} and CNRS/IN2P3, Marseille, France\\
$^{89}$ Department of Physics, University of Massachusetts, Amherst MA, United States of America\\
$^{90}$ Department of Physics, McGill University, Montreal QC, Canada\\
$^{91}$ School of Physics, University of Melbourne, Victoria, Australia\\
$^{92}$ Department of Physics, The University of Michigan, Ann Arbor MI, United States of America\\
$^{93}$ Department of Physics and Astronomy, Michigan State University, East Lansing MI, United States of America\\
$^{94}$ $^{(a)}$ INFN Sezione di Milano; $^{(b)}$ Dipartimento di Fisica, Universit{\`a} di Milano, Milano, Italy\\
$^{95}$ B.I. Stepanov Institute of Physics, National Academy of Sciences of Belarus, Minsk, Republic of Belarus\\
$^{96}$ Research Institute for Nuclear Problems of Byelorussian State University, Minsk, Republic of Belarus\\
$^{97}$ Group of Particle Physics, University of Montreal, Montreal QC, Canada\\
$^{98}$ P.N. Lebedev Physical Institute of the Russian Academy of Sciences, Moscow, Russia\\
$^{99}$ Institute for Theoretical and Experimental Physics (ITEP), Moscow, Russia\\
$^{100}$ National Research Nuclear University MEPhI, Moscow, Russia\\
$^{101}$ D.V. Skobeltsyn Institute of Nuclear Physics, M.V. Lomonosov Moscow State University, Moscow, Russia\\
$^{102}$ Fakult{\"a}t f{\"u}r Physik, Ludwig-Maximilians-Universit{\"a}t M{\"u}nchen, M{\"u}nchen, Germany\\
$^{103}$ Max-Planck-Institut f{\"u}r Physik (Werner-Heisenberg-Institut), M{\"u}nchen, Germany\\
$^{104}$ Nagasaki Institute of Applied Science, Nagasaki, Japan\\
$^{105}$ Graduate School of Science and Kobayashi-Maskawa Institute, Nagoya University, Nagoya, Japan\\
$^{106}$ $^{(a)}$ INFN Sezione di Napoli; $^{(b)}$ Dipartimento di Fisica, Universit{\`a} di Napoli, Napoli, Italy\\
$^{107}$ Department of Physics and Astronomy, University of New Mexico, Albuquerque NM, United States of America\\
$^{108}$ Institute for Mathematics, Astrophysics and Particle Physics, Radboud University Nijmegen/Nikhef, Nijmegen, Netherlands\\
$^{109}$ Nikhef National Institute for Subatomic Physics and University of Amsterdam, Amsterdam, Netherlands\\
$^{110}$ Department of Physics, Northern Illinois University, DeKalb IL, United States of America\\
$^{111}$ Budker Institute of Nuclear Physics, SB RAS, Novosibirsk, Russia\\
$^{112}$ Department of Physics, New York University, New York NY, United States of America\\
$^{113}$ Ohio State University, Columbus OH, United States of America\\
$^{114}$ Faculty of Science, Okayama University, Okayama, Japan\\
$^{115}$ Homer L. Dodge Department of Physics and Astronomy, University of Oklahoma, Norman OK, United States of America\\
$^{116}$ Department of Physics, Oklahoma State University, Stillwater OK, United States of America\\
$^{117}$ Palack{\'y} University, RCPTM, Olomouc, Czech Republic\\
$^{118}$ Center for High Energy Physics, University of Oregon, Eugene OR, United States of America\\
$^{119}$ LAL, Univ. Paris-Sud, CNRS/IN2P3, Universit{\'e} Paris-Saclay, Orsay, France\\
$^{120}$ Graduate School of Science, Osaka University, Osaka, Japan\\
$^{121}$ Department of Physics, University of Oslo, Oslo, Norway\\
$^{122}$ Department of Physics, Oxford University, Oxford, United Kingdom\\
$^{123}$ $^{(a)}$ INFN Sezione di Pavia; $^{(b)}$ Dipartimento di Fisica, Universit{\`a} di Pavia, Pavia, Italy\\
$^{124}$ Department of Physics, University of Pennsylvania, Philadelphia PA, United States of America\\
$^{125}$ National Research Centre "Kurchatov Institute" B.P.Konstantinov Petersburg Nuclear Physics Institute, St. Petersburg, Russia\\
$^{126}$ $^{(a)}$ INFN Sezione di Pisa; $^{(b)}$ Dipartimento di Fisica E. Fermi, Universit{\`a} di Pisa, Pisa, Italy\\
$^{127}$ Department of Physics and Astronomy, University of Pittsburgh, Pittsburgh PA, United States of America\\
$^{128}$ $^{(a)}$ Laborat{\'o}rio de Instrumenta{\c{c}}{\~a}o e F{\'\i}sica Experimental de Part{\'\i}culas - LIP, Lisboa; $^{(b)}$ Faculdade de Ci{\^e}ncias, Universidade de Lisboa, Lisboa; $^{(c)}$ Department of Physics, University of Coimbra, Coimbra; $^{(d)}$ Centro de F{\'\i}sica Nuclear da Universidade de Lisboa, Lisboa; $^{(e)}$ Departamento de Fisica, Universidade do Minho, Braga; $^{(f)}$ Departamento de Fisica Teorica y del Cosmos, Universidad de Granada, Granada; $^{(g)}$ Dep Fisica and CEFITEC of Faculdade de Ciencias e Tecnologia, Universidade Nova de Lisboa, Caparica, Portugal\\
$^{129}$ Institute of Physics, Academy of Sciences of the Czech Republic, Praha, Czech Republic\\
$^{130}$ Czech Technical University in Prague, Praha, Czech Republic\\
$^{131}$ Charles University, Faculty of Mathematics and Physics, Prague, Czech Republic\\
$^{132}$ State Research Center Institute for High Energy Physics (Protvino), NRC KI, Russia\\
$^{133}$ Particle Physics Department, Rutherford Appleton Laboratory, Didcot, United Kingdom\\
$^{134}$ $^{(a)}$ INFN Sezione di Roma; $^{(b)}$ Dipartimento di Fisica, Sapienza Universit{\`a} di Roma, Roma, Italy\\
$^{135}$ $^{(a)}$ INFN Sezione di Roma Tor Vergata; $^{(b)}$ Dipartimento di Fisica, Universit{\`a} di Roma Tor Vergata, Roma, Italy\\
$^{136}$ $^{(a)}$ INFN Sezione di Roma Tre; $^{(b)}$ Dipartimento di Matematica e Fisica, Universit{\`a} Roma Tre, Roma, Italy\\
$^{137}$ $^{(a)}$ Facult{\'e} des Sciences Ain Chock, R{\'e}seau Universitaire de Physique des Hautes Energies - Universit{\'e} Hassan II, Casablanca; $^{(b)}$ Centre National de l'Energie des Sciences Techniques Nucleaires, Rabat; $^{(c)}$ Facult{\'e} des Sciences Semlalia, Universit{\'e} Cadi Ayyad, LPHEA-Marrakech; $^{(d)}$ Facult{\'e} des Sciences, Universit{\'e} Mohamed Premier and LPTPM, Oujda; $^{(e)}$ Facult{\'e} des sciences, Universit{\'e} Mohammed V, Rabat, Morocco\\
$^{138}$ DSM/IRFU (Institut de Recherches sur les Lois Fondamentales de l'Univers), CEA Saclay (Commissariat {\`a} l'Energie Atomique et aux Energies Alternatives), Gif-sur-Yvette, France\\
$^{139}$ Santa Cruz Institute for Particle Physics, University of California Santa Cruz, Santa Cruz CA, United States of America\\
$^{140}$ Department of Physics, University of Washington, Seattle WA, United States of America\\
$^{141}$ Department of Physics and Astronomy, University of Sheffield, Sheffield, United Kingdom\\
$^{142}$ Department of Physics, Shinshu University, Nagano, Japan\\
$^{143}$ Department Physik, Universit{\"a}t Siegen, Siegen, Germany\\
$^{144}$ Department of Physics, Simon Fraser University, Burnaby BC, Canada\\
$^{145}$ SLAC National Accelerator Laboratory, Stanford CA, United States of America\\
$^{146}$ $^{(a)}$ Faculty of Mathematics, Physics {\&} Informatics, Comenius University, Bratislava; $^{(b)}$ Department of Subnuclear Physics, Institute of Experimental Physics of the Slovak Academy of Sciences, Kosice, Slovak Republic\\
$^{147}$ $^{(a)}$ Department of Physics, University of Cape Town, Cape Town; $^{(b)}$ Department of Physics, University of Johannesburg, Johannesburg; $^{(c)}$ School of Physics, University of the Witwatersrand, Johannesburg, South Africa\\
$^{148}$ $^{(a)}$ Department of Physics, Stockholm University; $^{(b)}$ The Oskar Klein Centre, Stockholm, Sweden\\
$^{149}$ Physics Department, Royal Institute of Technology, Stockholm, Sweden\\
$^{150}$ Departments of Physics {\&} Astronomy and Chemistry, Stony Brook University, Stony Brook NY, United States of America\\
$^{151}$ Department of Physics and Astronomy, University of Sussex, Brighton, United Kingdom\\
$^{152}$ School of Physics, University of Sydney, Sydney, Australia\\
$^{153}$ Institute of Physics, Academia Sinica, Taipei, Taiwan\\
$^{154}$ Department of Physics, Technion: Israel Institute of Technology, Haifa, Israel\\
$^{155}$ Raymond and Beverly Sackler School of Physics and Astronomy, Tel Aviv University, Tel Aviv, Israel\\
$^{156}$ Department of Physics, Aristotle University of Thessaloniki, Thessaloniki, Greece\\
$^{157}$ International Center for Elementary Particle Physics and Department of Physics, The University of Tokyo, Tokyo, Japan\\
$^{158}$ Graduate School of Science and Technology, Tokyo Metropolitan University, Tokyo, Japan\\
$^{159}$ Department of Physics, Tokyo Institute of Technology, Tokyo, Japan\\
$^{160}$ Tomsk State University, Tomsk, Russia\\
$^{161}$ Department of Physics, University of Toronto, Toronto ON, Canada\\
$^{162}$ $^{(a)}$ INFN-TIFPA; $^{(b)}$ University of Trento, Trento, Italy\\
$^{163}$ $^{(a)}$ TRIUMF, Vancouver BC; $^{(b)}$ Department of Physics and Astronomy, York University, Toronto ON, Canada\\
$^{164}$ Faculty of Pure and Applied Sciences, and Center for Integrated Research in Fundamental Science and Engineering, University of Tsukuba, Tsukuba, Japan\\
$^{165}$ Department of Physics and Astronomy, Tufts University, Medford MA, United States of America\\
$^{166}$ Department of Physics and Astronomy, University of California Irvine, Irvine CA, United States of America\\
$^{167}$ $^{(a)}$ INFN Gruppo Collegato di Udine, Sezione di Trieste, Udine; $^{(b)}$ ICTP, Trieste; $^{(c)}$ Dipartimento di Chimica, Fisica e Ambiente, Universit{\`a} di Udine, Udine, Italy\\
$^{168}$ Department of Physics and Astronomy, University of Uppsala, Uppsala, Sweden\\
$^{169}$ Department of Physics, University of Illinois, Urbana IL, United States of America\\
$^{170}$ Instituto de Fisica Corpuscular (IFIC), Centro Mixto Universidad de Valencia - CSIC, Spain\\
$^{171}$ Department of Physics, University of British Columbia, Vancouver BC, Canada\\
$^{172}$ Department of Physics and Astronomy, University of Victoria, Victoria BC, Canada\\
$^{173}$ Department of Physics, University of Warwick, Coventry, United Kingdom\\
$^{174}$ Waseda University, Tokyo, Japan\\
$^{175}$ Department of Particle Physics, The Weizmann Institute of Science, Rehovot, Israel\\
$^{176}$ Department of Physics, University of Wisconsin, Madison WI, United States of America\\
$^{177}$ Fakult{\"a}t f{\"u}r Physik und Astronomie, Julius-Maximilians-Universit{\"a}t, W{\"u}rzburg, Germany\\
$^{178}$ Fakult{\"a}t f{\"u}r Mathematik und Naturwissenschaften, Fachgruppe Physik, Bergische Universit{\"a}t Wuppertal, Wuppertal, Germany\\
$^{179}$ Department of Physics, Yale University, New Haven CT, United States of America\\
$^{180}$ Yerevan Physics Institute, Yerevan, Armenia\\
$^{181}$ Centre de Calcul de l'Institut National de Physique Nucl{\'e}aire et de Physique des Particules (IN2P3), Villeurbanne, France\\
$^{182}$ Academia Sinica Grid Computing, Institute of Physics, Academia Sinica, Taipei, Taiwan\\
$^{a}$ Also at Department of Physics, King's College London, London, United Kingdom\\
$^{b}$ Also at Institute of Physics, Azerbaijan Academy of Sciences, Baku, Azerbaijan\\
$^{c}$ Also at Novosibirsk State University, Novosibirsk, Russia\\
$^{d}$ Also at TRIUMF, Vancouver BC, Canada\\
$^{e}$ Also at Department of Physics {\&} Astronomy, University of Louisville, Louisville, KY, United States of America\\
$^{f}$ Also at Physics Department, An-Najah National University, Nablus, Palestine\\
$^{g}$ Also at Department of Physics, California State University, Fresno CA, United States of America\\
$^{h}$ Also at Department of Physics, University of Fribourg, Fribourg, Switzerland\\
$^{i}$ Also at II Physikalisches Institut, Georg-August-Universit{\"a}t, G{\"o}ttingen, Germany\\
$^{j}$ Also at Departament de Fisica de la Universitat Autonoma de Barcelona, Barcelona, Spain\\
$^{k}$ Also at Departamento de Fisica e Astronomia, Faculdade de Ciencias, Universidade do Porto, Portugal\\
$^{l}$ Also at Tomsk State University, Tomsk, and Moscow Institute of Physics and Technology State University, Dolgoprudny, Russia\\
$^{m}$ Also at The Collaborative Innovation Center of Quantum Matter (CICQM), Beijing, China\\
$^{n}$ Also at Universita di Napoli Parthenope, Napoli, Italy\\
$^{o}$ Also at Institute of Particle Physics (IPP), Canada\\
$^{p}$ Also at Horia Hulubei National Institute of Physics and Nuclear Engineering, Bucharest, Romania\\
$^{q}$ Also at Department of Physics, St. Petersburg State Polytechnical University, St. Petersburg, Russia\\
$^{r}$ Also at Borough of Manhattan Community College, City University of New York, New York City, United States of America\\
$^{s}$ Also at Department of Financial and Management Engineering, University of the Aegean, Chios, Greece\\
$^{t}$ Also at Centre for High Performance Computing, CSIR Campus, Rosebank, Cape Town, South Africa\\
$^{u}$ Also at Louisiana Tech University, Ruston LA, United States of America\\
$^{v}$ Also at Institucio Catalana de Recerca i Estudis Avancats, ICREA, Barcelona, Spain\\
$^{w}$ Also at Department of Physics, The University of Michigan, Ann Arbor MI, United States of America\\
$^{x}$ Also at Graduate School of Science, Osaka University, Osaka, Japan\\
$^{y}$ Also at Fakult{\"a}t f{\"u}r Mathematik und Physik, Albert-Ludwigs-Universit{\"a}t, Freiburg, Germany\\
$^{z}$ Also at Institute for Mathematics, Astrophysics and Particle Physics, Radboud University Nijmegen/Nikhef, Nijmegen, Netherlands\\
$^{aa}$ Also at Department of Physics, The University of Texas at Austin, Austin TX, United States of America\\
$^{ab}$ Also at Institute of Theoretical Physics, Ilia State University, Tbilisi, Georgia\\
$^{ac}$ Also at CERN, Geneva, Switzerland\\
$^{ad}$ Also at Georgian Technical University (GTU),Tbilisi, Georgia\\
$^{ae}$ Also at Ochadai Academic Production, Ochanomizu University, Tokyo, Japan\\
$^{af}$ Also at Manhattan College, New York NY, United States of America\\
$^{ag}$ Also at Departamento de F{\'\i}sica, Pontificia Universidad Cat{\'o}lica de Chile, Santiago, Chile\\
$^{ah}$ Also at The City College of New York, New York NY, United States of America\\
$^{ai}$ Also at School of Physics, Shandong University, Shandong, China\\
$^{aj}$ Also at Departamento de Fisica Teorica y del Cosmos, Universidad de Granada, Granada, Portugal\\
$^{ak}$ Also at Department of Physics, California State University, Sacramento CA, United States of America\\
$^{al}$ Also at Moscow Institute of Physics and Technology State University, Dolgoprudny, Russia\\
$^{am}$ Also at Departement  de Physique Nucleaire et Corpusculaire, Universit{\'e} de Gen{\`e}ve, Geneva, Switzerland\\
$^{an}$ Also at Institut de F{\'\i}sica d'Altes Energies (IFAE), The Barcelona Institute of Science and Technology, Barcelona, Spain\\
$^{ao}$ Also at School of Physics, Sun Yat-sen University, Guangzhou, China\\
$^{ap}$ Also at Institute for Nuclear Research and Nuclear Energy (INRNE) of the Bulgarian Academy of Sciences, Sofia, Bulgaria\\
$^{aq}$ Also at Faculty of Physics, M.V.Lomonosov Moscow State University, Moscow, Russia\\
$^{ar}$ Also at National Research Nuclear University MEPhI, Moscow, Russia\\
$^{as}$ Also at Department of Physics, Stanford University, Stanford CA, United States of America\\
$^{at}$ Also at Institute for Particle and Nuclear Physics, Wigner Research Centre for Physics, Budapest, Hungary\\
$^{au}$ Also at Giresun University, Faculty of Engineering, Turkey\\
$^{av}$ Also at CPPM, Aix-Marseille Universit{\'e} and CNRS/IN2P3, Marseille, France\\
$^{aw}$ Also at Department of Physics, Nanjing University, Jiangsu, China\\
$^{ax}$ Also at Institute of Physics, Academia Sinica, Taipei, Taiwan\\
$^{ay}$ Also at University of Malaya, Department of Physics, Kuala Lumpur, Malaysia\\
$^{az}$ Also at LAL, Univ. Paris-Sud, CNRS/IN2P3, Universit{\'e} Paris-Saclay, Orsay, France\\
$^{*}$ Deceased
\end{flushleft}

%\end{document}
% Created with xml2latex.py


\end{document}
