% TO remove blinding search for BLINDING everywhere


% Generated by GrindEQ Word-to-LaTeX 2012 
% LaTeX/AMS-LaTeX

%BLINDING
\documentclass[12pt]{article}
%\documentclass{article}
%%% remove comment delimiter ('%') and specify encoding parameter if
%%% required,
%%% see TeX documentation for additional info (cp1252-Western,cp1251-Cyrillic)
%\usepackage[cp1252]{inputenc}

\usepackage{xr}
\externaldocument{paper}

%%% remove comment delimiter ('%') and select language if required
%\usepackage[english,spanish]{babel}
\usepackage{graphicx,subfigure,amsthm,amsmath,latexsym,amssymb,asa,enumitem}
\usepackage{float,epsfig,multirow,rotating,times,verbatim,wrapfig}%,bold-extra}
%\usepackage{parskip}
\usepackage[margin=0.9in]{geometry} % to get margins of 0.9 on all
                                % four sides
\usepackage{mdwlist} %http://www.terminally-incoherent.com/blog/2007/09/19/latex-squeezing-the-vertical-white-space/
\usepackage[affil-it]{authblk} 
\RequirePackage{setspace}
\RequirePackage{caption}
%\captionsetup[table]{font={stretch=1.4}}     %% change 1.2 as you like
%\captionsetup[figure]{font={stretch=1.4}}    %% change 1.2 as you like
%\captionsetup{font={stretch=1.2}} 
%BLINDING
\captionsetup{font={it,stretch=1.2}}
%\captionsetup{font={it,stretch=1}}
% https://tex.stackexchange.com/questions/153334/change-the-baselinestretch-line-spacing-only-for-figure-captions/170533#170533
\usepackage[small,compact]{titlesec}
\titlespacing{\section}{0pt}{*1}{*1}
\titlespacing{\subsection}{0pt}{*1}{*1}
\titlespacing{\subsubsection}{0pt}{*1}{*1}
\titlespacing{\paragraph}{0pt}{*1}{*1}
\titlespacing{\subparagraph}{0pt}{*1}{*1}

\setlist{nolistsep}
%\renewcommand\Authfont{\fontsize{12}{10.4}\selectfont}
%\renewcommand\Affilfont{\fontsize{11}{10.8}\selectfont}

\newenvironment{packed_enum}{
\begin{enumerate}[leftmargin=1.2em]
  \setlength{\itemsep}{1pt}
  \setlength{\parskip}{0pt}
  \setlength{\parsep}{0pt}
}{\end{enumerate}}

\usepackage[T1]{fontenc}
\usepackage{lmodern}
%\usepackage[nolists]{endfloat}
%\usepackage{euler}
%\setmainfont[Numbers=Lining,BoldFont={TeX Gyre Pagella Bold}]{EB Garamond}
%\usepackage{times}
%\usepackage[full]{textcomp}
%\usepackage{garamondx}
%\usepackage{kantlipsum}
%\usepackage{ebgaramond}
\usepackage[T1]{fontenc}
%\usepackage{gentium} %https://tex.stackexchange.com/questions/168780/how-to-use-boldface-from-another-package-while-using-ebgaramond-font
%\usepackage{times}
\usepackage{tgtermes} 
\RequirePackage{natbib}
\setlength{\bibsep}{1pt}
%\setlength{\oddsidemargin}{-0.1in}
%\setlength{\evensidemargin}{-0.1in}
%\setlength{\textwidth}{6.6in}
%\setlength{\topmargin}{-0.6in}
%\setlength{\textheight}{9.3in}
%\evensidemargin \oddsidemargin

%%% remove comment delimiter ('%') and select graphics package
%%% for DVI output:
%\usepackage[dvips]{graphicx}
%%% or for PDF output:
%\usepackage[pdftex]{graphicx}
%%% or for old LaTeX compilers:
%\usepackage[dvips]{graphics}
\newtheorem{theorem}{Theorem}[section]
\newtheorem{lemma}[theorem]{Lemma}
\newtheorem{corollary}[theorem]{Corollary}
\newtheorem{result}[theorem]{Result}
\newcommand{\abs}[1]{|#1|}
\newcommand{\norm}[1]{\Vert#1\Vert}
\newcommand{\R}{I\!\!R}
\newcommand{\bell}{\boldsymbol{\ell}}
\newcommand{\bd}{\boldsymbol{d}}
\newcommand{\blambda}{\boldsymbol{\lambda}}
\newcommand{\balpha}{\boldsymbol{\alpha}}
\newcommand{\btheta}{\boldsymbol{\theta}}
\newcommand{\bnabla}{\boldsymbol{\nabla}}
\newcommand{\bLambda}{\boldsymbol{\Lambda}}
\newcommand{\bDelta}{\boldsymbol{\Delta}}
\newcommand{\bkappa}{\boldsymbol{\kappa}}
\newcommand{\bbeta}{\boldsymbol{\beta}}
\newcommand{\bdelta}{\boldsymbol{\delta}}
\newcommand{\bphi}{\boldsymbol{\phi}}
\newcommand{\bPhi}{\boldsymbol{\Phi}}
\newcommand{\bPi}{\boldsymbol{\Pi}}
\newcommand{\bChii}{\boldsymbol{\Chi}}
\newcommand{\bXi}{\boldsymbol{\Xi}}
\newcommand{\bxi}{\boldsymbol{\xi}}
\newcommand{\bGamma}{\boldsymbol{\Gamma}}
\newcommand{\bgamma}{\boldsymbol{\gamma}}
\newcommand{\bSigma}{\boldsymbol{\Sigma}}
\newcommand{\bPsi}{\boldsymbol{\Psi}}
\newcommand{\bpsi}{\boldsymbol{\psi}}
\newcommand{\bepsilon}{\boldsymbol{\epsilon}}
\newcommand{\bUpsilon}{\boldsymbol{\Upsilon}}
\newcommand{\bmu}{\boldsymbol{\mu}}
\newcommand{\bzeta}{\boldsymbol{\zeta}}
\newcommand{\bfeta}{\boldsymbol{\eta}}
\newcommand{\bvarrho}{\boldsymbol{\varrho}}
\newcommand{\bvarphi}{\boldsymbol{\varphi}}
\newcommand{\bTheta}{\boldsymbol{\Theta}}
\newcommand{\bvartheta}{\boldsymbol{\vartheta}}
\newcommand{\btau}{\boldsymbol{\tau}}
\newcommand{\bOmega}{\boldsymbol{\Omega}}
\newcommand{\bomega}{\boldsymbol{\omega}}
\newcommand{\prm}{^{\prime}}
\newcommand{\bB}{\boldsymbol{B}}
\newcommand{\bC}{\boldsymbol{C}}
\newcommand{\bc}{\boldsymbol{c}}
\newcommand{\bD}{\boldsymbol{D}}
\newcommand{\bff}{\boldsymbol{f}}
\newcommand{\bV}{\boldsymbol{V}}
\newcommand{\bG}{\boldsymbol{G}}
\newcommand{\bH}{\boldsymbol{H}}
\newcommand{\bmm}{\boldsymbol{m}}
\newcommand{\bM}{\boldsymbol{M}}
\newcommand{\bi}{\boldsymbol{i}}
\newcommand{\bI}{\boldsymbol{I}}
\newcommand{\bJ}{\boldsymbol{J}}
\newcommand{\bK}{\boldsymbol{K}}
\newcommand{\bk}{\boldsymbol{k}}
\newcommand{\bA}{\boldsymbol{A}}
\newcommand{\bS}{\boldsymbol{S}}
\newcommand{\bt}{\boldsymbol{t}}
\newcommand{\bT}{\boldsymbol{T}}
\newcommand{\bE}{\boldsymbol{E}}
\newcommand{\be}{\boldsymbol{e}}
\newcommand{\bp}{\boldsymbol{p}}
\newcommand{\bP}{\boldsymbol{P}}
\newcommand{\bq}{\boldsymbol{q}}
\newcommand{\bQ}{\boldsymbol{Q}}
\newcommand{\bR}{\boldsymbol{R}}
\newcommand{\bW}{\boldsymbol{W}}
\newcommand{\bu}{\boldsymbol{u}}
\newcommand{\bv}{\boldsymbol{v}}
\newcommand{\bs}{\boldsymbol{s}}
\newcommand{\bU}{\boldsymbol{U}}
\newcommand{\bx}{\boldsymbol{x}}
\newcommand{\bX}{\boldsymbol{X}}
\newcommand{\by}{\boldsymbol{y}}
\newcommand{\bY}{\boldsymbol{Y}}
\newcommand{\bZ}{\boldsymbol{Z}}
\newcommand{\bz}{\boldsymbol{z}}
\newcommand{\blda}{\boldsymbol{a}}
\newcommand{\bOO}{\boldsymbol{O}}
\newcommand{\bw}{\boldsymbol{w}}
\newcommand{\bee}{\boldsymbol{e}}
\newcommand{\bzero}{\boldsymbol{0}}
\newcommand{\lsb}{\bigl\{}
\newcommand{\rsb}{\bigr\}}
\newcommand{\bj}{{\bf{J}}}
\newcommand{\bone}{\boldsymbol{1}}
\newcommand{\bmV}{\boldsymbol{\mathcal V}}
\newcommand{\bmI}{\boldsymbol{\mathcal I}}
\newcommand{\mM}{\mathcal M}
\newcommand{\mD}{\mathcal D}
\newcommand{\mR}{\mathcal R}
\newcommand{\mF}{\mathcal F}
\newcommand{\mK}{\mathcal K}
\newcommand{\mI}{\mathcal I}
\newcommand{\mE}{\mathcal E}
\newcommand{\mV}{\mathcal V}
\newcommand{\E}{\mbox{I\!E}}
\newcommand{\one}{$\phantom{1}$}
\newcommand{\oo}{$\phantom{00}$}
\newcommand{\ooo}{$\phantom{000}$}
\newcommand{\oc}{$\phantom{0,}$}
\newcommand{\ooc}{$\phantom{00,}$}
\newcommand{\oooc}{$\phantom{000,}$}
\newcommand{\Ncand}{N_{\rm cand}}
\newcommand{\tover}{t_{\rm over}}
\newcommand{\tstd}{t_{\rm std,1}}
\newcommand{\ttwostage}{t_{\rm 2S,1}}
\newcommand{\psp}{P(S')}
\newcommand{\pep}{P(E')}
\newcommand{\pacc}{P_{\rm acc}}
\newcommand{\prob}{\mbox{I}\!\mbox{Pr}}
\DeclareMathOperator*{\argmin}{argmin}
\DeclareMathOperator*{\argmax}{argmax}
\hyphenation{Lan-ge-vin Lan-ge-vins}
\newcommand{\apprasym}{
 \mathrel{\ooalign{$\sim$\cr\kern+1.25pt\large $\colon$}}}

\newcommand{\floor}[1]{{\lfloor{#1}\rfloor}}

%BLINDING
%\newcommand{\blinded}{1} 
\newcommand{\blinded}{0} 


\begin{document}
%\DeclareFontShape{OT1}{EBGaramond-OsF}{bx}{n}{ <-> ssub * cmr/bx/n }{}


\if0\blinded
{
\markboth{Ye et al{\it et al.}}{Automated Flaw Detection in NDE
  Images}
}
\fi
\if1\blinded
{
\markboth{}{Supplement to Automated Flaw Detection in NDE Images}
}
\fi

%\pagestyle{empty}
\title{\vspace{-0.8in} Supplement to ``A Statistical Framework for
  Improved Automatic Flaw Detection in Nondestructive Evaluation Images''} 
\if0\blinded
{
\author{
  \vspace{-0.1in}
  {\bf Ye Tian}\\
  \vspace{-0.2in}
  Department of Statistics and Statistical Laboratory\\
    Iowa State University\\
    Ames, IA 50011\\
    (tianye1984@gmail.com)\\
\and
    \vspace{-0.2in}
    {\bf Ranjan Maitra}\\
    \vspace{-0.2in}
    Department of Statistics and Statistical Laboratory\\
    Iowa State University\\
    Ames, IA 50011\\
    (maitra@iastate.edu)
    \and
    \vspace{-0.2in}
           {\bf William Q. Meeker}\\
           \vspace{-0.2in}
           Department of Statistics and the Center for Nondestructive
           Evaluation\\
           Iowa State University\\
           Ames, IA 50011\\
           (wqmeeker@iastate.edu)
           \and
           \vspace{-0.2in}
                  {\bf Stephen D. Holland}\\
                  \vspace{-0.2in}
                  Department of Aerospace Engineering and Center for Nondestructive Evaluation\\
                  Iowa State University\\
                  Ames, IA 50011\\
                  (sdh4@iastate.edu)
}
\date{\vspace{-0.5in}}
}
\fi
\if1\blinded
    {
      \date{\vspace{-1.2in}}
      \renewcommand{\baselinestretch}{1.4}\normalsize
    }
\fi
\maketitle
%BLINDING
\renewcommand{\baselinestretch}{1.4}\normalsize

\renewcommand\thefigure{S-\arabic{figure}}
\renewcommand\thetable{S-\arabic{table}}
\renewcommand\thesection{S-\arabic{section}}
\renewcommand\thesubsection{S-\arabic{section}.\arabic{subsection}}
\renewcommand\theequation{S-\arabic{equation}}
\section{Notations used in Supplement}
In this supplement, references to sections, figures and equations in
the main paper are referred to using the same identifiers as in the
main paper. References to sections, figures and equations in the
supplement use the suffix ``S-''. 
\section{Methodology -- Supplement}
\subsection{Illustrative Examples showing the Effect of $\lambda$}
\label{S-lambda}
\begin{figure}[!h]
\vspace{-0in}
\begin{center}
\mbox{
\subfigure[Strong signal,
  $\lambda=2$]{\includegraphics[width=0.33\textwidth]{plots/f5_strong_lambda2}}
\subfigure[Strong signal,
  $\lambda=100$]{\includegraphics[width=0.33\textwidth]{plots/f5_strong_lambda100}}
\subfigure[Strong signal,
  $\lambda=200$]{\includegraphics[width=0.33\textwidth]{plots/f5_strong_lambda200}}
}
\mbox{
\subfigure[Weak signal,
  $\lambda=2$]{\includegraphics[width=0.33\textwidth]{plots/f5_weak_lambda2}}
\subfigure[Weak signal,
  $\lambda=100$]{\includegraphics[width=0.33\textwidth]{plots/f5_weak_lambda100}}
\subfigure[Weak signal,
  $\lambda=200$]{\includegraphics[width=0.33\textwidth]{plots/f5_weak_lambda200}}}
\mbox{
\subfigure[Noise,
  $\lambda=2$]{\includegraphics[width=0.33\textwidth]{plots/f5_noise_lambda2}}
\subfigure[Noise,
  $\lambda=100$]{\includegraphics[width=0.33\textwidth]{plots/f5_noise_lambda100}}
\subfigure[Noise,
  $\lambda=200$]{\includegraphics[width=0.33\textwidth]{plots/f5_noise_lambda200}}}
\end{center}
\vspace{-0.2in}
\caption{The effect of $\lambda$ on optimal ellipses drawn on the 
  illustrative images of Figures~\ref{fig1} and~\ref{fig2} having
  strong (top row, a--c), weak (middle row, d--f) and no true (bottom row,
  g--i) signal.} 
\label{lambdaeffect}
\vspace{-0.1in}
%\end{wrapfigure}
\end{figure}
The objective function~\eqref{voleqn} in the main paper depends on
$\lambda$ so we now illustrate the effect of $\lambda$ with a bid to
make recommendations for its selection. 
Figure~\ref{lambdaeffect}  displays  the results -- for three different
values of $\lambda$ ($\lambda = 2$, first column; $\lambda = 100$,
second column; $\lambda = 200$, third column) -- of drawing the optimal
inner ellipse and the 
corresponding outer ellipse (using Step~\ref{step3} of our algorithm)
after optimizing $\mV(a,b,\theta)$ for the three illustrative cases of 
Section~\ref{vbexample} after processing with a matched filter,
resulting in images as in Figure~\ref{fig2}.  For presentation
clarity, the images in the top row with the strong true signal is drawn
using a different scale than the other two sets of images. 
The figures in the first column all have larger ellipses than their 
corresponding counterparts in the next two columns which have ellipses
in decreasing order of size. Thus, a larger regularization parameter results  
in smaller ellipses, because of the heavier penalty put on noise
pixels (which have negative $C_\sigma(\tau(u,v))$ values). The
differences in the sizes of the inner ellipses, however, decrease 
for larger values of $\lambda$ (last two columns). 

\subsection{Diagnostic Checks for Model Assumptions in
  Vibrothermography Specimens}
\label{supp.diag}
%\begin{rapfigure}{r}{0.5\textwidth}
We report results on some checks to evaluate the assumption of normality
\begin{figure}[h]
\begin{center}
\mbox{
 \subfigure[]{\includegraphics*[width=0.5\textwidth]{plots/f10new_a}}
  \subfigure[]{\includegraphics*[width=0.5\textwidth]{plots/f10new_b}}}
\end{center}
\vspace{-0.2in}
\caption{Quantile plots for evaluating the assumption of normality in the NIM
  model for the (a) flawed  and (b) flawless specimens.}
\label{diags}
\vspace{-0.1in}
%\end{wrapfigure}
\end{figure}
in the NIM model for the vibrothermography specimen images. Figure~\ref{diags} provides quantile-quantile plots
for the residuals obtained upon fitting the NIM to  flawed and
flawless specimens. (Note that since there is no signal
in a flawless specimen, the residuals are essentially the same as
$D_{\mbox{noise}}$.) These plots indicate reasonably good agreement  
with the normal distribution. A formal test for normality using the
\citet{shapiroandwilk65} approach yielded $p$-values of 0.109  and
0.07 for flawed and flawless specimens, providing support for the
normality assumption. (We recall that in most real applications, the
Shapiro-Wilk test  will tend to find departures from a normal
distribution with moderately large to large sample sizes.) We conclude
by noting that we are not making inferences on the tails of the
distribution and so the procedure is expected to be robust to moderate
departures from the  normality assumption in the NIM model.

\section{Performance Evaluations -- Supplement}
\subsection{Derivation of the NIM Model}
\label{supp.deriv}
 We have, from the NIM model in Section~\ref{NIM}
 \begin{equation*}
  D_{\mbox{obs}} =
  \max(D_{\mbox{signal}},D_{\mbox{noise}})
  \end{equation*}
  where
\begin{equation*}
  D_{\mbox{signal}} =
  \beta_0+\beta_1 \log_{10}(\mbox{(flaw size)}) + \epsilon_s,
\end{equation*}
with $\epsilon_s\sim N(0,\sigma_s^2)$ distributed independently of
$D_{\mbox{noise}} \sim N(0,\sigma_N^2)$. Then the POD is given by,
\begin{equation*}
\begin{split}
  \mbox{POD(flaw)} & = \Pr(D_{\mbox{obs}} > 0)\\
  & = \Pr(\max(D_{\mbox{signal}},D_{\mbox{noise}}) > 0) \\
  & = 1-\Pr(\max(D_{\mbox{signal}},D_{\mbox{noise}}) \leq 0)\\
  & =  1-{\rm \Phi }\left[ 
-\frac{\beta _{0} +\beta _{1} \log_{10} ({\rm flaw})}{\sigma _{S} 
} \right]\Phi \left(-\frac{\mu _{N} }{\sigma _{N} } \right). \\
\end{split}
\end{equation*}
 A similar argument holds for the POD(flaw) of the ``PeakAmp''
 method, which is derived in Equation (3) of \cite{liandmeeker09}. To
 elucidate, since the 
 peak intensity $\check Z$ of the raw unprocessed hottest 
 frame image  (in $\log_{10}$) is used to characterize the image, we
 have that a flaw is detected if $\log_{10}\check Z_{\mbox{obs}}$ is
 greater than some threshold given by $\log Z_{\mbox{th}}$. Then, from
 the NIM, we have $\log_{10}\check Z_{\mbox{obs}} = 
 \max(Y_{\mbox{signal}},Y_{\mbox{noise}})$ so that 
  \begin{equation*}
\begin{split}
  \mbox{POD(flaw)} & = \Pr[Z_{\mbox{obs}} > Z_{\rm th}]\\
  & = \Pr[\log_{10}(Z_{\mbox{obs}}) > \log_{10}(Z_{\rm th})]\\
  & = \Pr[\max(Y_{\mbox{signal}},T_{\mbox{noise}}) > \log_{10}(Z_{\rm th})] \\
  & = 1-\Pr[\max(Y_{\mbox{signal}},Y_{\mbox{noise}} ) \leq \log_{10}(Z_{\rm th})]\\
  & =  1-{\rm \Phi }\left[ 
\frac{\log_{10}Z_{\rm th} - \gamma _{0} - \gamma _{1} \log_{10} ({\rm flaw})}{\kappa _{S} } \right]\Phi \left(\frac{\log_{10}Z_{\rm th} -\nu _{N} }{\kappa _{N} } \right). \\
\end{split}
\end{equation*}
where 
$(\gamma_0,\gamma_1,\kappa_S,\nu_N,\kappa_N)$ are defined as in the
last paragraph of~Section~\ref{performance}.
\renewcommand{\baselinestretch}{1}\normalsize
\begingroup
\def\bibfont{\small}
\bibliographystyle{asa}
%\clearpage  %if it should start on a new page
\addcontentsline{toc}{section}{References}
\bibliography{refs}
\endgroup

\end{document}

