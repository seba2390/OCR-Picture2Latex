%% LyX 2.3.3 created this file.  For more info, see http://www.lyx.org/.
%% Do not edit unless you really know what you are doing.
% \documentclass[english]{article}
\documentclass[3p,review]{elsarticle}
%\usepackage[T1]{fontenc}
%\usepackage[latin9]{inputenc}
%\usepackage[a4paper]{geometry}
%\geometry{verbose,tmargin=2cm,bmargin=2cm,lmargin=2cm,rmargin=2cm}
%\setlength{\parindent}{0bp}
%\usepackage{babel}
%\usepackage{textcomp}
%\usepackage[unicode=true]{hyperref}
\usepackage{amsmath,amssymb}
\usepackage{verbatimbox}
\usepackage{graphicx}
\graphicspath{{./PicsFinal/}}
\usepackage{listings}
\usepackage[colorlinks=true,linkcolor=blue,citecolor=blue,urlcolor=blue]{hyperref}
%\usepackage{biblatex}


%\usepackage[margin=1in]{geometry}

\usepackage{siunitx}
%\num[group-separator={,}]{1234567890}
\usepackage{enumitem}
%\usepackage[digitsep={,},group-minimum-digits={3}]{siunitx}

\newcounter{hintcounter}

\newcommand{\hint}[1]{

  \begin{center}
    \emph{``Hint \thehintcounter:  {#1}''}
  \end{center}
  \stepcounter{hintcounter}
}

\newcommand{\mybox}[1]{
  \begin{center}
   \begin{tabular}{|p{0.9\textwidth}|}
    \hline
    {#1}\\
    \hline
 \end{tabular}
\end{center}
}
\newcommand{\N}{\mathcal{N}}
\newcommand{\T}{\mathcal{T}}
\newcommand{\D}{\mathcal{D}}
\newcommand{\Cl}{\mathcal{C}}
\newcommand{\I}{\mathrm{i}}
\newcommand{\Es}{\mathcal{S}}
\newcommand{\sphere}{{\mathbb{S}^2}}

\newcommand{\bx}{{\bf x}}
\newcommand{\by}{{\bf y}}
\newcommand{\bn}{{\bf n}}
\newcommand{\bz}{{\bf z}}
\newcommand{\bs}{{\bf s}}
\newcommand{\bd}{{\bf d}}
\newcommand{\bT}{{\bf T}}
\newcommand{\bN}{{\bf N}}
\newcommand{\bS}{{\bf S}}
\newcommand{\bD}{{\bf D}}
\newcommand{\de}{\mathrm{d}}

\newcommand{\meshtohrtf}{Mesh2HRTF{}}
\newcommand{\numcalc}{NumCalc{}}

 
%\makeatletter
%%%%%%%%%%%%%%%%%%%%%%%%%%%%%% Textclass specific LaTeX commands.
%\newcommand{\lyxaddress}[1]{
%	\par {\raggedright #1
%	\vspace{1.4em}
%	\noindent\par}
%}

%\@ifundefined{date}{}{\date{}}
%\makeatother

\begin{document}
\begin{frontmatter}{}
\title{NumCalc: An open source BEM code for solving acoustic scattering problems}
\author[ARI]{W. Kreuzer\corref{CORRAUTH}}
\ead{wolfgang.kreuzer@oeaw.ac.at}
\author[ARI]{K. Pollack}
\ead{katharina.pollack@oeaw.ac.at}
\author[TUB]{F. Brinkmann}
\ead{fabian.brinkmann@tu-berlin.de}
\author[ARI]{P. Majdak}
\ead{piotr.majdak@oeaw.ac.at}

\affiliation[ARI]{
  organization = {Austrian Academy of Sciences, Acoustics Research Institute},
  addressline = {Wohllebengasse 12-14},
  postcode = {1040},
  city = {Vienna},
  country = {Austria}
}

% \address[addressARI]{Austrian Academy of Sciences, Acoustics Research Institute, Wohllebengasse 12-14, 1040 Vienna}
\affiliation[TUB]{
  organization = {Audio Communication Group, Technical University of Berlin},
  addressline = {Straße des 17. Juni 135},
  postcode = {10623},
  city = {Berlin},
  country = {Germany}
}
%\address[addressTUB]{TU Berlin, }

\cortext[CORRAUTH]{Corresponding author.}



\begin{abstract}
The calculation of the acoustic field in or around objects is an important task in acoustic engineering. To numerically solve this task, the boundary element method (BEM) is a commonly used method especially for infinite domains. The open source tool \meshtohrtf{} and its BEM core \numcalc{} provide users with a collection of free software for acoustic simulations without the need of having an in-depth knowledge into numerical methods. However, we feel that users should have a basic understanding with respect to the methods behind the software they are using. We are convinced that this basic understanding helps in avoiding common mistakes and also helps to understand the requirements to use the software. To provide this background is the first motivation for this paper. A second motivation for this paper is to demonstrate the accuracy of \numcalc{} when solving benchmark problems. Thus, users can get an idea about the accuracy they can expect when using \numcalc{} as well as  the memory and CPU requirements of \numcalc{}. A third motivation for this paper is to give users  detailed information about some parts of the actual implementation that are usually not mentioned in literature, e.g., the specific version of the fast multipole method and its clustering process or how to use frequency-dependent admittance boundary conditions.
\end{abstract}

\begin{keyword}
  BEM \sep Software \sep Fast Multipole Method
\end{keyword}
%
\journal{Applied Acoustics}
\end{frontmatter}{}
\section{Introduction}\label{Sec:Introduction}
The boundary element method (BEM, ~\cite{chen2008,SauSch10}) has a long tradition for numerically solving the Helmholtz equation in 3D that models acoustic wave propagation in the frequency domain. Compared to other methods such as the finite element method~\cite{MarNol08} (FEM), the BEM has the advantage that only the surface of the scattering object needs to be considered and that for external problems the decay of the acoustic wave towards infinity is already included in the formulation. This renders the BEM an attractive tool for calculating sound-wave propagation. However, the BEM has the drawback that %integrals over a singular function (Green's function) need to be calculated, and that
the generated linear system of equations has a densely populated system matrix. To tackle this problem,  the combination of BEM and fast methods for matrix-vector multiplications such as the fast multipole method (FMM,~\cite{Coifmanetal93,Rokhlin90,Rokhlin93}) or $\mathcal{H}$-matrices~\cite{Hackbusch15} were introduced. Eventually, BEM-based acoustic modeling became an increasingly popular tool for engineers, as illustrated by various software packages such as COMSOL~\cite{comsol}, FastBEMAcoustics~\cite{fastbem}, AcouSTO~\cite{Acousto}, %FreeFEM, \cite{hecht2012new}
BEMpp~\cite{Betcke2021}, or OpenBEM~\cite{Henrquez2010OpenBEMA}. 

\numcalc{}~\cite{Kreuzeretal22a} is an open-source program written in C$++$ for solving the Helmholtz equation in 3D. It is based on collocation with constant elements in combination with an implementation of the FMM to speed up matrix-vector multiplications. \numcalc{} is distributed in the framework of 
\meshtohrtf{}~\cite{Fabianspaper,mesh2hrtf,Ziegelwangeretal15}, which is aimed at the numerical calculation and post processing of head-related transfer functions (HRTFs,~\cite{moller_head-related_1995}). HRTFs describe the direction-dependent filtering of a sound source due to the listener's body especially by the head and ears and are usually acoustically measured~\cite{majdak2007,pollack2022modern}.

The basic idea for \meshtohrtf{} is to provide an open-source code that is easy-to-use for researchers without the need for an extensive background in mathematics or physics, specifically targeting users not familiar with BEM. To this end, \meshtohrtf{} contains an add-on for the open-source program Blender \cite{blendermanual} and automatically generates the input for \numcalc. The accompanying project websites \cite{mesh2hrtf,mesh2hrtfwiki} provide tutorials for creating a valid description of the geometry of the scattering object and its materials. Despite the encapsulation of the BEM calculations from the typical \meshtohrtf{} user, some rules are required for correct calculations and their compliance cannot be ensured by the \meshtohrtf{} interface. Hence, the main goal of the article is to explain such rules, provide insights to what can happen if they are violated, and how to avoid common mistakes. To this end, throughout the article, we point out the most important rules as hints:
\hint{Look out for these lines.} 

Although part of the \meshtohrtf{}-project, \numcalc{} is \emph{not restricted} to calculations of HRTFs only. The input to \numcalc{} is a general definition of surface meshes, based on triangular or plane quadrilateral elements, arbitrary admittance boundary conditions can be assigned to each element, and external sound sources can be modelled as plane waves or as point sources (see \ref{Sec:Mesh} and \ref{App:Input}). \numcalc{} can be used to solve general wave scattering problems in 3D using the BEM.

This article describes the general features of \numcalc{} in detail for the first time in one manuscript. It provides all necessary information for users to responsibly use \numcalc{} for their problem at hand, including some benchmark tests to measure its accuracy and needs with respect to memory consumption. Besides  specific aspects of the implementation of the BEM, such as the quasi-singular quadrature or the possibility to define frequency-dependent boundary conditions, we for the first time give a detailed description on the specific flavour and implementation of the FMM. This includes a description, how the clusters are generated and how one can derive all matrices necessary for the fast multipole method. Compared to other standard implementations of the FMM the root level already consists of multitple clusters and the local element-to-cluster expansion matrices on \emph{all} levels of a multilevel FMM are stored explicitly, which reduces computing time. As the root level already consists of many clusters, the truncation length of the multipole expansion and in turn the memory necessary to use NumCalc can be kept relatively low. Finally, we provide numerical experiments that  illustrate the accuracy of \numcalc{}, show how the mesh is clustered in the default setting and discusses the memory required for the FMM algorithm based on the clustering. %In this manuscript we put the focus solely on \numcalc{}, that can be used for all sorts of acoustic BEM problems in 3D.


This article is structured as follows: In Sec.~\ref{Sec:BriefBEM} we describe the general aspects of the BEM and the way they are implemented in \numcalc. In Sec.~\ref{Sec:Implementation}, we provide specific details of the implementation of the quadrature and clustering in \numcalc{}. In Sec.~\ref{Sec:Critical}, we focus on the critical components of the implementation and their effect on accuracy and computational effort. Sec.~\ref{Sec:Benchmark} presents three benchmark problems solved with \numcalc{} to demonstrate the achievable accuracy and the need of computing power. Finally, we summarize and discuss the results in Sec.~\ref{Sec:Discussion}. To keep the paper self contained, \ref{Sec:Mesh} provides the definition of a mesh to describe the geometry of the scatterer, \ref{App:Input} provides examples for \numcalc{} input files including how to define frequency-dependent admittance boundary conditions, \ref{App:Commandline} gives a description of possible command line parameters and \ref{App:Output} lists the different output files generated by \numcalc.
%In Section \ref{ we that illustrate the accuracy that can be expected with the code and that points to some drawbacks of \numcalc{} in terms of memory consumption.


%The description of \numcalc{} will contain details of the specific implementation of the fast multipole (FMM) in \numcalc{} that for mid size problems, up to 100 000 elements, speeds up calculations but that on the other hand, has a higher memory consumption, which can become a problem for very large meshes. %Especially, at conferences, one is sometimes faced with questions like \emph{``how do you do the hypersingular integration?''} or \emph{``how do you do the clustering?''}.

%This paper is an extension of a conference paper \cite{Kreuzeretal22a}, its focus is set on \numcalc{} and the implementation of the boundary element method.  Users interested in the practical pipeline from mesh generation with Blender to the final HRTFs are referred to \cite{Fabianspaper}.

 %Using a benchmark example of a sphere, we also illustrate the effect of the regularity of the BEM elements and the dependence of the calculation results on the number of elements. 
%  Additionally, users can but don't need to 
%  \begin{itemize}
%  \item Why? The bald and not excessively bald and not excessively smart hamster obeyed a terrified and not excessively terrified hamster.
%  \item Simple to use vs tweakable
%  \item Accuracy about 0.1 dB is okay for most applications
%  \item not always the best method chosen for implementation, but the one that is fairly easy and still produces good results
%  \item Nevertheless, describe some of the mechanisms behind \meshtohrtf{} so
%    \begin{itemize}
%    \item users get a general idea about background
%    \item users get an idea on how to modify the code or use the input parameters
%    \end{itemize}
%  \end{itemize}
%\mybox{List of possible error sources}

%\section{An overview of \numcalc{}}\label{Sec:NumCalc}



\section{BEM and its Implementation in \numcalc}\label{Sec:BriefBEM}
% A major part of \meshtohrtf{} is the calculation of the acoustic field on and around the scatterer using the boundary element method, which is executed by \numcalc. To this end, the Helmholtz equation, that models harmonic wave propagation in the frequency domain, needs to be discretized and solved. \numcalc{} is based on collocation with constant elements, i.e. the sound field is assumed to be constant on each  element of the mesh. This approach is one of the simplest ways to generate a linear system of equations in connection with BEM, and we do not claim, that it is the best way for an implementation in terms of accuracy (see also Sec.~\ref{Sec:Benchmark}). But because of its simplicity, collocation is very popular in the engineering community. \meshtohrtf{} was originally aimed at applications in acoustics, where a relative error in the range of a few percent is acceptable, thus the known slow convergence (see also Section \ref{Sec:Benchmark},nach vorne ziehen?) of the collocation method with respect to the number of elements used is not a pressing issue here.

In a nutshell, \numcalc{} implements the direct BEM using collocation with constant elements in combination with the Burton-Miller method. Per default, an iterative solver is used to solve the system of equations, and the matrix-vector multiplications are sped up using the FMM. To keep the paper self-contained, we start with a brief introduction to collocation BEM, explaining the requirements for the meshes used for the BEM. %The subsections are aimed at users not familiar with BEM by listing the main components of the BEM and by pointing out the influence of these components on the accuracy and the stability of the calculated solution.

Figure~\ref{Fig:Schemedrawing} shows a typical situation of an (exterior) acoustics wave scattering problem. An incoming point source outside the scatterer $\Omega$ is positioned at a point $\bx_{\mathrm{inc}}$. The acoustic wave produced by this source is scattered and reflected at the surface $\Gamma$ of the scatterer. With the BEM, the acoustic field on the surface $\Gamma$ at nodes $\bx_\Gamma$ as well as on exterior nodes $\bx_{\text{o}}$ outside $\Gamma$ is determined. The main goal of \numcalc{} is to derive the acoustic pressure $p$ and the particle velocity $v$ at both the points on the surface $\Gamma$ of the scatterer $\Omega$ and the evaluation points $\bx_{\text{o}}$ in the free-field $\mathbb{R}^3\setminus \Gamma$. To this end, \numcalc{} solves the Helmholtz equation describing the free-field sound propagation in the frequency domain:

\begin{equation}\label{Equ:Helmholtz}
\nabla^2 p(\bx,\omega) + k^2 p(\bx,\omega) = p_0(\bx,\bx_{\text{inc}},\omega),\quad \bx \in \mathbb{R}^3\setminus \Gamma,
\end{equation}
with $p$ being the sound pressure, $k = \frac{\omega}{c} = \frac{2\pi f}{c}$ the wavenumber for a given frequency $f$ and speed of sound $c$, and $p_0$ being an external sound source located at $\bx_{\text{inc}}$. 

To model the acoustic properties of the surface $\Gamma$, a boundary condition (bc) is defined for each point on the surface $\Gamma$. Thus, either the sound pressure $p$ (Dirichlet bc), the particle velocity $v$ (Neumann bc) or a combination of both (Robin bc) are known:
\begin{align*}
  p(\bx) &= f(\bx)\qquad  \bx \in \Gamma_D,\\
  v(\bx) & = g(\bx)\qquad \bx \in \Gamma_N,\\
  \alpha p(\bx) + \beta v(\bx) &= h(\bx)\qquad \bx \in \Gamma_R,
\end{align*}
for given functions $f,g,$ $h$, and constants $\alpha,\beta$. % and $\Gamma = \Gamma_D \cup \Gamma_N \cup \Gamma_R$. 

%In Fig.\,\ref{Fig:Schemedrawing} a typical situation of an (exterior) acoustics wave scattering problem is given. An incoming point source outside the scatterer $\Omega$ is positioned at a point $\bx_{\mathrm{inc}}$. The acoustic wave produced by this source is scattered and reflected at the surface $\Gamma$ of the scatterer. With the BEM the acoustic field on the surface $\Gamma$ at nodes $\bx_\Gamma$ as well as on nodes $\bx_o$ outside $\Gamma$ is determined. 
%
\begin{figure}
  \begin{center}
  \includegraphics[width=0.3\textwidth]{Schematicdrawing}
  \caption{Scheme of an exterior problem with scatterer $\Omega$ with boundary $\Gamma = \Gamma_D \cup \Gamma_N \cup \Gamma_R$, a point-source at $\bx_{\text{inc}}$, interior point $\bx_i$, exterior point $\bx_o$ and a node on the boundary $\bx_{\Gamma}$.}\label{Fig:Schemedrawing}
  \end{center}
\end{figure}
%
%

\subsection{The boundary integral equation (BIE)}
The acoustic field at a point $\bx$ can be represented by the sound pressure $p(\bx)$ and the particle velocity $v(\bx)$ at this point, which in the frequency domain can both be derived from the velocity potential $\phi(\bx)$ using
\begin{equation}\label{Equ:utopandv}
  p(\bx) = -\I \omega \rho \phi(\bx),\quad v(\bx) = \frac{\partial \phi}{\partial \bn}\left(\bx\right) = \nabla \phi(\bx) \cdot \bn(\bx),
\end{equation}
where $\omega = 2\pi f$ is the angular frequency, and $\rho$ denotes the density of the medium. If $\bx \in \Gamma$ the particle velocity is defined using the normal vector $\bn$ to the surface pointing to the outside. If the point $\bx$ lies in the free field it can be any arbitrary vector with $||\bn|| = 1$.

In order to solve the Helmholtz equation, the partial differential equation needs to be transformed into an integral equation. In general, there are two possibilities for this transformation, direct and indirect BEM (cf.~\cite{SauSch10}). In \numcalc{}, the direct formulation based on Green's representation formula is used:
\begin{equation}\label{Equ:BIE_I}
  \lambda(\bx) \phi(\bx) - \tau \int\limits_\Gamma H(\bx,\by) \phi(\by) \de\by + \tau \int\limits_\Gamma G(\bx,\by) v(\by) \de\by = \phi_{\text{inc}}(\bx),
\end{equation}
%\mybox{K will mit lambda(x) anfangen}
%where $u(\bx)$ is the velocity potential at a point $\bx$ in space. The sound pressure $p(\bx)$, the particle velocity $v(\bx)$, and the velocity potential $u(\bx)$  are related bys
%
where
%\mybox{Muessen wir hier erwaehnen, dass der druck die ableitung nach der zeit ist? Yes, weil im Gegensatz zur Schallschnelle v der Druck auch keine gerichtete Groesse ist.}
$G(\bx,\by) = \frac{e^{\I k ||\bx - \by||}}{4\pi ||\bx - \by||}$ is the Green's function for the free-field Helmholtz equation, $H(\bx,\by) = \frac{\partial G}{\partial \bn_y}(\bx,\by) = \nabla G(\bx,\by) \cdot \bn(\by) = G(\bx,\by)\left(\I k - \frac{1}{r}\right) \frac{\bx - \by}{r}\cdot \bn(\by)$, $r = ||\bx - \by||$, and $\phi_{\text{inc}}$ is the velocity potential caused by an external sound source, e.g., a point source positioned at $\bx_{\mathrm{inc}}$. The parameter $\tau$ indicates the problem setting. If $\tau = -1$, the acoustic field \emph{inside} the object is of interest, in this case, we speak of an ``interior'' problem. If $\tau = 1$ the acoustic field outside the object is of interest, in that case, we have an ``exterior'' problem.

The integrands in Eq.~(\ref{Equ:BIE_I}) depend on the derivative with respect to the normal vector $\bn_y := {\bf n}(\by)$. Normal vectors pointing in the different directions cause a sign change, essentially mixing up the roles of interior and exterior problems. As a consequence, Eq.~(\ref{Equ:BIE_I}) may produce incorrect results. To avoid this, it is essential that $\bn_y$ points away from the scatterer, however, \numcalc{} does not check this. This check is left to the user: \hint{Check if your normal vectors point away from the scatterer.}

Note, that if the distance $r = ||\bx - \by||$ between two points $\bx$ and $\by$ becomes small, $G$ and $H$ behave like $\frac{1}{r}$. Thus, special methods are needed to numerically calculate the integrals over $\Gamma$. We describe these methods in Sec.~\ref{Sec:Quadrature}. 

Note also, that Eq.~(\ref{Equ:BIE_I}) cannot be solved uniquely at resonance frequencies of the inner problem~\cite{BurMil71,Schenck68}. For exterior problems, this can be handled using either CHIEF points~\cite{Schenck68} or the Burton-Miller method~\cite{BurMil71}. \numcalc{} implements the Burton-Miller method and the implementation is described in Sec.~\ref{Sec:Burton-Miller}.

Finally, note that $\lambda(\bx)$ depends on the position of the point $\bx$ (see also Fig.~\ref{Fig:Schemedrawing}). If $\bx $ lies in the interior of the object $\Omega$, $\lambda(\bx) = 0$. If $\bx$ lies outside the object, $\lambda(\bx) = 1$. If $\bx$ lies on a smooth part of the surface $\lambda(\bx) = \frac12$. %This has implications when deriving a linear system of equation based on Eq.~(\ref{Equ:BIE_I}), for which there are three options: collocation, Galerkin, and Nystr\"om, cf.~\cite{Caninoetal98,SauSch10}. \numcalc{} uses the collocation method, which implementation is described in Sec.~\ref{Sec:Collocation}.


%In the literature, several methods to ensure uniqueness of the solution can be found~\cite{Schenck68,BurMil71,BraWer65}.
% There are quadrature methods that can deal with such singularities, see Section \ref{Sec:Quadrature}. However, the right hand side (including $\phi_{\text{inc}}(\bx)$) also depends on the Green's function, thus if a sound source is placed too close to the surface, these singularities can cause numerical problems. 
%\subsection{Deriving a linear system of equations}
% To simplify the derivation and the notation we just concentrate on the transformation of Eq.~(\ref{Equ:BIE_I}) into a linear system, the transformation of the the integrals with $H'$ and $E$follows the same steps:

\subsection{Discretization of the Geometry and Shapefunctions}
To transform the BIE into a linear system of equations, the unknown solution of Eq.~(\ref{Equ:BIE_I}), i.e., $\phi(\bx)$ or $v(\bx)$ on the surface (depending on the boundary condition at \bx), needs to be approximated by simple ansatz functions (= shape functions). To this end, the geometry of the scatterer's surface is approximated by a mesh consisting of $N$ elements $\Gamma_j, j = 1,\dots,N$. In \numcalc{} these elements can be triangular or \emph{plane} quadrilateral, and on each of them, the BIE solution is assumed to be constant:
$$
\phi(\bx) = \left\{
  \begin{array}{cc}
    \phi_j & \bx \in \Gamma_j,\\
    0 & \text{otherwise}.
  \end{array}
\right.
$$
The number of elements $N$ is a compromise between calculation time and numerical accuracy. %For examples of a mesh, we refer to Fig.~\ref{Fig:Spheremesh} in Section \ref{Sec:Benchmark}.
The numerical accuracy is actually frequency dependent because the unknown solution with its oscillating parts needs to be represented accurately by the constant ansatz functions. As a rule of thumb, the mesh should contain six to eight elements per wavelength~\cite{marburgsix2002}. \numcalc{} does a check on that and displays a warning if the average size of the elements is too low, however, it will still do the calculations. This is because, in certain cases, acceptable results can be achieved, when ``irrelevant'' parts of the mesh have a coarser discretization~\cite{Ziegelwangeretal16,Palm2021a}. The decision on the number of elements is within the responsibility of the user: 
%
\hint{Check if your mesh contains 6-to-8 elements per wavelength in the relevant geometry regions.}
%
Also, it is known, that for some problems, and that especially close to sharp edges and corners, a finer discretization than 6-to-8 elements per wavelength might be necessary~\cite{Kreuzer22}, thus one should be aware, that the 6-to-8 elements per wavelength rule is just a rule of thumb.
 
The elements of the mesh are also used to calculated the integrals over the surface $\Gamma$. In \numcalc{}, integrals over $\Gamma$ are replaced by sums of integrals over each element: 
\begin{equation}\label{Equ:IntI}
\int\limits_{\Gamma} H(\bx,\by)\phi(\by) \de\by \approx  \sum_{j=1}^N \int_{\Gamma_j} H(\bx,\by) \phi(\by) \de\by \approx \sum_{j=1}^N \int\limits_{\Gamma_j} H(\bx,\by) \de\by \phi_j.
\end{equation}
As $\Gamma_j$ represents a simple geometric element, the integrals over each element can be calculated relatively easily. By using the shape functions in combination with the boundary conditions, the unknown solution of Eq.~(\ref{Equ:BIE_I}) can be reduced to $N$ unknowns (either $\phi_j$ or $v_j$, depending on the given boundary condition).

%  \subsubsection*{Step 2: The shape functions = Approximation of the unknown $\phi(\by)$}
%  The unknown velocity potential $\phi(\by)$ or its derivative $v(\by)$ is replaced by a linear combination of simple ansatz functions (= shape functions), that are, in general, defined locally on the elements of the mesh, thus
%  $$
%  \phi(\bx) \approx \sum_{i=1}^I N_i(\bx) \phi_i.
%  $$
%  With this approximation the integral in Eq.~(\ref{Equ:IntI}) becomes
%  $$
%  \sum_{j=1}^N \int_{\Gamma_j} H(\bx,\by) \phi(\by) \de\by \approx \sum_{i=1}^I \sum_{j=1}^N \int_{\Gamma_j} H(\bx,\by) N_i(\by) \phi_i \de\by.
%  $$
%  Dependending on accuracy requirements these shape functions need to be chosen accordingly (cf.~ \cite{SauSch10}).
%  % \subsubsection*{Step 3: The collocations nodes = Evaluation of Eq.~(\ref{Equ:BIE_I})}

\subsection{Collocation}\label{Sec:Collocation}
In order to derive a linear system of equation, \numcalc{} uses the collocation method, i.e., after the unknown solution is represented using the shape functions, Eq.~(\ref{Equ:BIE_I}) is evaluated at $N$ collocation nodes $\bx_i$ for $i = 1,\dots,N$. For constant elements these nodes are, in general, the midpoints of each element $\Gamma_i$ defined as the mean over the coordinates of the element's vertices. Note, that because the collocation nodes are located inside the smooth plane element, $\lambda(\bx_i) = \frac12$, for $i = 1,\dots, N$, and Eq.~(\ref{Equ:BIE_I}) becomes
\begin{equation}\label{Equ:BIESystem}
\frac{\phi_i}{2} - \tau \sum_{j=1}^N \left(\int_{\Gamma_j} H(\bx_i,\by)\de\by\right) \phi_j
+ \tau \sum_{j=1}^N \left(\int_{\Gamma_j} G(\bx_i,\by) \de\by \right)v_j = \phi_{\mathrm{inc}} (\bx_i),
\end{equation}
where $\phi_i = \phi(\bx_i)$ and $v_i = v(\bx_i),\; i = 1,\dots, N,$ are the velocity potential and the particle velocity at the midpoint $\bx_i$ of each element, respectively. By defining the vectors
$$
{\pmb \phi} := (\phi_1,\dots,\phi_N)^\top,\quad {\bf v} := ({v_1,\dots,v_N})^\top,\quad \pmb{\phi}_{\mathrm{inc}}=(\phi_{\mathrm{inc}}(\bx_1),\dots,\phi_{\mathrm{inc}}(\bx_N))^\top
$$
and the matrices
\begin{equation}\label{Equ:Matrices_I}
{\bf H}_{ij} = \int\limits_{\Gamma_j} H(\bx_i,\by) \de\by, \quad {\bf G}_{ij} = \int\limits_{\Gamma_j}G(\bx_i,\by)\de\by
\end{equation}
we obtain the discretized BIE in matrix-vector form:
%
\begin{equation}\label{Equ:LinSystem}
\frac{\pmb\phi}{2} - \tau {\bf H\pmb\phi} + \tau {\bf G v} = {\pmb\phi}_{\mathrm{inc}},
\end{equation}
which is a basis for further consideration to solve the Helmholtz equation numerically.
%
\subsection{Incoming Sound Sources}\label{Sec:Incoming}
Incoming sound sources $\phi_{\mathrm{inc}}$ are the sources not positioned on the surface of the scatterer but located in the free field. \numcalc{} implements two types of incoming sound sources: Plane waves and point sources. A plane wave is defined by its strength $S_0$ and its direction $\bd$:
$$
\phi_{\mathrm{inc}}(\bx) = S_0 e^{\I k \bx \cdot \bd}, 
$$
and a point source is defined by its strength $S_0$ and its position $\bx^*$:
\begin{equation}\label{Equ:PointSrc}
\phi_{\mathrm{inc}}(\bx) = S_0 \frac{e^{\I k ||\bx - \bx^*||}}{4\pi||\bx - \bx^*||},
\end{equation}
with $k$ being the wavenumber. 
%
From the definition of the point source it becomes clear, that if the source position $\bx^*$ is very close to the surface, the denominator in Eq.~(\ref{Equ:PointSrc}) will become zero, causing numerical problems. %Thus, \hint{Do not put external sound sources on the surface of the scatterer.}
% But also, having an external source close to the scatterer may cause numerical problems in the representation of  $\phi_{\mathrm{inc}}(\bx)$.
%
Thus, \hint{Have at least the distance of one average edge length between an external sound source and the surface.}
%
If sound sources very close to the surface or even on the surface are needed, a velocity boundary condition different from zero at the elements close to the source may be a better choice.

\subsection{The Burton-Miller Method}\label{Sec:Burton-Miller}
\numcalc{} implements the Burton-Miller approach to ensure a stable and unique solution of Eq.~(\ref{Equ:LinSystem}) for exterior problems at all frequencies. The final system of equations is derived by combining Eq.~(\ref{Equ:BIE_I}) and its derivative with respect to the normal vector at $\bx$
\begin{equation}\label{Equ:dBIE}
\lambda(\bx) v(\bx) - \tau \int_{\Gamma} E(\bx,\by) \phi(\by) \de\by + \tau \int_{\Gamma} H'(\bx,\by) v(\by) \de\by) = v_{\mathrm{inc}}(\bx), 
\end{equation}
where $E(\bx,\by) = \frac{\partial^2 G(\bx,\by)}{\partial \bn_x \partial \bn_y}$ and $H'(\bx, \by) = \frac{\partial G(\bx,\by)}{\partial \bn_x}$. The matrix-vector representation of Eq.~(\ref{Equ:dBIE}) is then derived in a similar manner as that for Eq.~(\ref{Equ:LinSystem}). 

The final system of equations to be solved reads then as
\begin{equation}\label{Equ:SystemI}
  \frac{\left( \pmb\phi - \gamma \bf v\right)}{2} - \tau  \left( {\bf H - \gamma E}\right)\pmb{\phi} + \tau \left( {\bf G - \gamma H'}\right){\bf v} = \pmb{\phi}_{\text{inc}} - \gamma {\bf v}_{\text{inc}},
\end{equation}
where 
\begin{equation}\label{Equ:Theintegrals}
  {\bf E}_{ij} =   \int\limits_{\Gamma_j}E(\bx_i,\by) \de\by,\; {\bf H}'_{ij} = \int\limits_{\Gamma_j}H'(\bx_i,\by) \de\by,\; {\bf v}_{\mathrm{inc}} = (v_{\mathrm{inc}}(\bx_1),\dots, v_{\mathrm{inc}}(\bx_N))^\top,
\end{equation}
%
and $\gamma := \frac{i}{k}$ is the coupling factor introduced by the Burton-Miller method \cite{BurMil71,Marburg14}. 

% The system equation Eq.~(\ref{Equ:SystemI}) is the one to be solved by \numcalc{}.
Because for each individual element, a boundary condition (pressure, velocity, or admittance) is given, either $\phi_i$, $v_i$, or a relation between them is known. Thus, the final system of equations Eq.~(\ref{Equ:SystemI}) has $N$ unknowns and there are as many collocation nodes as unknowns, meaning that it can be solved by regular methods, e.g., either a direct solver~\cite{Andersonetal99} or an iterative solvers~\cite{Saad03}. For most problems, the matrices in that system are, in general, densely populated and the number of unknowns can be large, thus, iterative solvers are the most common option. However, for large $N$, the cost of the matrix-vector multiplications needed by an iterative solver becomes too high. For those systems, the BEM can be coupled with methods for fast matrix-vector multiplications, e.g. $\mathcal{H}$-matrices~\cite{Hackbusch15} or the in \numcalc{} implemented FMM~\cite{Coifmanetal93}. %Further, numerical calculations of the integrals over each element $\Gamma_j$ require an appropriate quadrature method. The following section focuses on the details of solving the system equation Eq.~(\ref{Equ:SystemI}) in \numcalc{}.


\section{Specific Details on Deriving and Solving the System Equation}\label{Sec:Implementation}
\subsection{Quadrature}\label{Sec:Quadrature}
In order to solve the BIE, the integrals in ${\bf G, H, H'}$, and ${\bf E}$ defined in Eqs.~(\ref{Equ:Matrices_I}) and (\ref{Equ:Theintegrals}) need to be calculated numerically. If the collocation node $\bx_i$ lies in the element for which the integral needs to be calculated, the integrand becomes singular because of the singularities of the Green's function and its derivatives. In this case, these integrals have to be calculated using special methods for the singular and the hypersingular integrals. For integrals over elements that do not contain the collocation node, we distinguish between elements that are close to the collocation node and elements far away which are solved by the quasi-singular quadrature and the regular quadrature, respectively. 

\subsubsection{Singular Quadrature}\label{Sec:SingQuad}
For $i=j$, %the integrands in Eq.~(\ref{Equ:Matrices_I}) become
%$$
%{\bf G}_{ii} = \int_{\Gamma_i} G(\bx_i,\by) \de\by, \quad {\bf H}_{ii} = \int_{\Gamma_i} %H(\bx_i,\by) \de\by,
%$$
%weakly singular, because the integral is over the element that also contains the collocation node, and thus,
the integrands in Eq.~(\ref{Equ:Matrices_I}) behave %, the diagonals of the matrices ${\bf G}$ and ${\bf H}$  contain an integral with a  weakly singular integrands that 
like $\frac{1}{r}$ for $r:=||\bx_i - \by|| \rightarrow 0$. In this case, \numcalc{} regularizes the integrals by using a quadrature method based on~\cite{Duffy82}. If the element $\Gamma_i$ is triangular, it is subdivided into six triangles, where the collocation node $\bx_i$, at which the singularity occurs, is a vertex of each of the sub-triangles. For a quadrilateral element, eight triangular sub-elements are created. Similar to~\cite{Duffy82}, the integrals over each of these sub-triangles are transformed into integrals over the unit square which removes the singularity. After that transformation, the integral is computed using a Gauss quadrature with $4\times 4$ quadrature nodes. %To illustrate this principle, we look at the integral of the function $\frac{1}{r}$ over the unit triangle with nodes $(0,0), (0,1)$ and $(1,1)$ which has a weak singularity at $(0,0)$. With the substitution used in Eq.~(\ref{Equ:Duffy}), the integral over the weak singularity can be transformed into a regular integral:
%\begin{align}
%\nonumber \int_{0}^{1} \int_{0}^{x} \frac{1}{\sqrt{x^2 + y^2}}\de y dx &=
%\left|
%  \begin{array}{ll}
%    x = u, & dx = du\\
%    y = wu, & dy = udw
%  \end{array}
%\right| = \\
%\label{Equ:Duffy} &=\int_0^1 \int_0^1 \frac{u}{\sqrt{u^2 + u^2w^2}}dwdu =
%\int_0^1 \int_0^1 \frac{1}{\sqrt{1 + w^2}}dwdu.
%\end{align}


For the singular quadrature with kernel ${\bf H}(\bx,\by)$, we can use the fact that, if $\bx$ and $\by$ are in the same planar element, $(\bx - \by)$ is orthogonal to $\bn_y$. This yields $H(\bx,\by) = \nabla_y G(\bx,\by)\cdot \bn_y = \frac{d G}{dr}\frac{\partial r}{\partial \by} \cdot \bn_y = -\frac{d G}{dr} \frac{\bx - \by}{r} \cdot \bn_y = 0$. The same argument can be used for the contributions with respect to ${\bf H}'$.
%\mybox{Routinennamen einfÃŒgen?}
%\mybox{Das ist die theorie hinter $H$ obwohl in NumCalc numerisch gerechnet wird, sollen wir $H$ ueberhaupt erwÀhnen? KP: ich find schon!}

\subsubsection{Hypersingular Quadrature}
To calculate the integrals involving the hypersingular kernel $E(\bx,\by)$, the integral involving the hypersingular kernel is split into integrals over the edges of the element and the element itself~\cite{Krishnasamyetal90}:
\begin{equation}
{\bf E}_{ii} = \int\limits_{\Gamma_i}E(\bx_i,\by)\de\by= \sum_{\ell,m,n = 1}^3 \bn_\ell \epsilon_{\ell m n} \oint\limits_{\partial {\Gamma_i}}\frac{\partial G}{\partial y_m}(\bx_i,\by) dy_n + k^2\int\limits_{\Gamma_i}G(\bx_i,\by) \de\by, %= \frac{\partial G}{\partial n_y} \otimes 
\end{equation}
where $\epsilon_{\ell mn}$ is the Levi-Civita-symbol
$$
\epsilon_{\ell mn} = \left\{
  \begin{array}{rl}
    1 & \text{ if } (\ell,m,n) \in \{ (1,2,3), (2,3,1),(3,1,2) \},\\
    -1 & \text{ if } (\ell,m,n) \in \{ (3,2,1), (1,3,2), (2,1,3) \},\\
    0 & \text{ if } \ell=m, \ell=n, m=n,
  \end{array}
\right.
$$
and $\by = (y_1,y_2,y_3)$ is a point on the element $\Gamma_i$, $\bn_\ell$ is the $\ell$-th component of the normal vector $\bn$, and $\partial {\Gamma_i}$ denotes the edges of the element.
%
For the integral over the edges $\partial \Gamma_i$, each edge is subdivided into 4 equally sized parts, and on each part, a Gaussian quadrature with 3 nodes is used. The integral over the element $\Gamma_i$ is calculated using the singular quadrature described in Sec.~\ref{Sec:SingQuad}.

\subsubsection{Quasi-singular Quadrature}
\numcalc{} calculates the quasi-singular integrals in two steps: First, if the relative distance $\tilde{r} = \frac{||\bx_i - \bx_j||}{\sqrt{A_j}}$ of the midpoint $\bx_j$ of $\Gamma_j$ and the collocation node $\bx_i$ is smaller than 1.3, the element is subdivided into four subelements, with $A_j$ being the area of the subelement $\Gamma_j$. This procedure is repeated for all subelements until the condition is fulfilled. As result, the distance between the collocation node and the subelement will be larger than the edge length of that subelements. This procedure is visualized in Fig.~\ref{Fig:Subdivide}. Note, that this way, the quadrature grid gets finer close to the collocation node.  The factor 1.3 was heuristically chosen as a good compromise between accuracy and efficiency. Once the subelements are constructed, the integrals over them can be solved using the regular quadrature.
%
\begin{figure}[!h]
  \begin{center}
  \includegraphics[width=0.6\textwidth]{Subdivisionbw}
  \caption{Example of two levels of element subdivisions in the quasi-singular quadrature and a quadrilateral element. The dashed (red) line shows the distance between the two nodes, which is too small compared to the average edge length, thus the element needs to be divided. The dashed-dotted (green) line shows the distance which is sufficiently large such that no further division is necessary.}\label{Fig:Subdivide}
  \end{center}
\end{figure}

Note that the distance between the element midpoints is just an approximation for the actually required smallest distance between the collocation node and \emph{all} the points of an element. \numcalc{} uses the distance between the element midpoints because calculations can be done efficiently and it works well for regular elements. Thus,
%
\hint{Keep the elements as regular as possible.}
%
If an element is not regular, e.g., stretched in one direction, the approach of iterative subdivisions may fail. 
%
\subsubsection{Regular Quadrature}
%
If the element and the collocation node are sufficiently far apart from each other, \numcalc{} determines the order of the Gauss quadrature over the element by using an a-priori error estimator similar to \cite{LacWat76}. The number of quadrature nodes is determined by the smallest $m$ for which all of the 3 a-priori error estimates 
  \begin{align*}
  \varepsilon_G &:= 32\left(\frac{1}{2\tilde{r}}\right)^{2m+1},\\
  \varepsilon_H &:= 64(2m+1)\left(\frac{1}{2\tilde{r}}\right)^{2m+2},\\
  \varepsilon_E &:= 128(m+1)(2m+1)\left(\frac{1}{2\tilde{r}}\right)^{2m+3}
  \end{align*}
are smaller than the default tolerance of $10^{-3}$. If $m$ becomes larger than 6, $m$ is set to 6, which provides a compromise between accuracy and computational effort. Because of the splitting of elements into subelements done in the quasi-singular case, $m = 6$ is a feasible upper limit even in the regular quadrature.

For quadrilateral elements, the quadrature nodes are given by the cross product of $m \times m$ Gauss nodes on the interval $[-1,1]$,~\cite{AbrSte64}. For triangular elements, Gauss quadrature formulas for triangles based on $m$ are used ~\cite{Cowper73, Bathe02}.

\subsection{Linear System of Equations Solvers}
% Once the matrix describing the system equation is set up, the number of unknowns is reduced to the number of elements $N$ by inserting the boundary conditions.
\numcalc{} offers to choose between a direct solver (either from LAPACK~\cite{Andersonetal99}, if \numcalc{} is compiled with LAPACK-support, or a slower direct LU-decomposition) or an iterative conjugate gradient squared (CGS) solver~\cite{Saad03}. The iterative solver is preferred, because, the number of elements $N$ is generally large for a feasible computation with a direct solver. When using the iterative solver, \numcalc{} offers to precondition the system matrix by applying either row-scaling or an incomplete LU-decomposition~\cite[Chapter 5]{Meister99}. However, even with preconditioning, if the elements or the geometry of the scatterer are irregular, the CGS solver may have robustness problems, yielding in convergence issues. Thus, when using the iterative solver, 
%
\hint{Check if the iterative solver has converged.}
%


%The error tolerances for Gauss order may arguable seem rather large, however, in practice, they provide a good balance between efficiency and needed accuracy for most problems at hand, and  

%\mybox{Mention the order breakdown because of the gauss order restriction for the nearly singular integrals? KP: yes!}

%\mybox{Irgendwo den CGS solver noch erwÀhnen, wenn es fancy klingt, kann ich auch den direkten solver in mesh2hrtf auf LAPACK trimmen . KP: Laback?}

\subsection{The Fast Multipole Method}\label{Sec:FMM_I}
One drawback of the BEM is that the system matrix is densely populated, which makes numerical computations with even moderately sized meshes prohibitively expensive in terms of memory consumption and computation time. This dense structure is caused by the Green's function $\frac{e^{\I k ||\bx -\by||}}{4\pi ||\bx - \by||}$, because the norm $||\bx - \by||$ introduces a nonlinear coupling between every element $\Gamma_j$  and every collocation node $\bx_i$.

The FMM is based on a man-in-the-middle principle where the Green's function $G(\bx,\by)$ is approximated by a product of three functions
\begin{equation}
G(\bx,\by) \approx G_1(\bx - \bz_1 ) G_2( \bz_1 - \bz_2) G_3( \bz_2 - \by).
\end{equation}
This splitting has the advantage, that %$G_1(\bx)$ and $G_2(\by)$ could be calculated independently of each other, thus, computation time as well memory consumption can be reduced. 
the integral over the BEM elements $\Gamma_j$ can be calculated independently of the collocation node $\bx_i$, and thus can be reused multiple times. On the downside, this splitting is only numerically stable if $\bx$ and $\by$ are sufficiently apart from each other (for more details we refer to Sec.~\ref{Sec:Expansion}).

To derive this splitting, the elements of the mesh are grouped into different clusters $\Cl_i$. A cluster is a collection of elements that are within a certain distance of each other and for each cluster $\Cl_i$ its midpoint $\bz_i$ is defined as the average over the coordinates of all vertices of the elements in the cluster. For two clusters $\Cl_i$ and $\Cl_j$ with $\bz_i$ and $\bz_j$ that are sufficiently ``far'' apart, the element-to-element interactions of all the elements between both clusters are reduced to the local interactions between each elements of a cluster with its midpoint (functions $G_1$ and $G_3$), and the interactions between the two clusters (function $G_2$), respectively (cf. Fig.~\ref{Fig:FMMScheme}). Note that local-to-cluster and the cluster-to-cluster expansions can be calculated independently from each other.

%$$
%\int_{\Gamma_j} G(\bx,\by) d\by \approx \Es(\bx,\bz_1) \D(\bz_1,\bz_2)\int_{\Gamma_j} \T(\by,\bz_2) d\by,
%$$
\begin{figure}[!h]
  \begin{center}
    \includegraphics[width=0.4\textwidth]{FMMScheme}
  \end{center}
  \caption{Scheme of a FMM interaction between two clusters. The interactions (gray lines) between elements (black dots) located in two distant clusters (shaded areas) are reduced to fewer within and between cluster interactions  (black lines).}\label{Fig:FMMScheme}
\end{figure}


\subsubsection{The expansion of the Green's function}\label{Sec:Expansion}
The expansion of the Green's function is based on its representation as a series~\cite{Rahola96}:
%  \begin{equation}\label{Equ:Multipole}
%    \frac{e^{\I k ||\br_0 + \br||}}{4\pi ||\br_0 + \br||} =
%    \frac{\I k}{4\pi} \int\limits_\sphere e^{\i k \bs \cdot \br} \sum\limits{n_0}^\infty \I^n (2n + 1) h_n^{(1)}(k ||\br_0||) P_n(\bs \cdot \frac{\br_0}{||\br_0}d\bs,
%  \end{equation}
%where $\br_0 = \bz_1 - \bz_2$ and $\br = \bx - \bz_1 + z_2 - \by$.
\begin{align}
  \nonumber G(\bx,\by) &= \frac{e^{\I k ||\bx - \by||}}{4\pi ||\bx - \by||} = \\
  \nonumber &= \frac{\I k}{4\pi}\int\limits_\sphere e^{\I k (\bx - \bz_1 + \bz_2 - \by)\cdot \bs} \sum_{n = 0}^\infty \I^n (2n + 1) h_n(k||\bz_1 - \bz_2||) P_n \left(k \frac{(\bz_1 - \bz_2)\cdot \bs}{||\bz_1 - \bz_2||}\right) \de\bs\\
  \label{Equ:FMMBase}&\approx \frac{\I k}{4\pi} \int\limits_\sphere  \Es(\bx,\bz_1,\bs) \D_L(\bz_1 - \bz_2,\bs) \T(\by,\bz_2,\bs) \de\bs,
\end{align}
where
\begin{align*}
  \Es(\bx,\bz_1,\bs) &= e^{\I k(\bx - \bz_1)\cdot \bs},\quad \T(\by,\bz_2,\bs) = e^{\I k (\bz_2 - \by)\cdot \bs},\\
  \D_L(\bz_1,\bz_2,\bs) &= \sum_{n = 0}^L \I^n (2n + 1) h_n(k||\bz_1 - \bz_2||) P_n \left(k \frac{(\bz_1 - \bz_2)\cdot \bs}{||\bz_1 - \bz_2||}\right),
\end{align*}
and $h_n$ denotes the spherical Hankel function of order $n$, $P_n$ is given by the Legendre polynomial of order $n$, and $\sphere$ denotes the unit sphere in 3D. %and $\bs$ is a node on the unit-sphere $\sphere$.

The integral over $\sphere$ needs to be calculated numerically. To this end, the unit sphere $\sphere$ is discretized using $2L^2$ quadrature nodes $\bs_j, j = 1,\dots,2L^2$, where $L$ is the length of the multipole expansion in Eq.~(\ref{Equ:FMMBase}). The elevation angle $\theta \in [0,\pi]$ is discretized using $L$ Gaussian quadrature nodes, for the azimuth angle $\phi\in [0,2\pi]$, $2L$ equidistant quadrature nodes are used~\cite{Coifmanetal93}. 
%\item The spherical Hankel function becomes singular for small arguments, and for higher order $n$ this problem becomes more prominent. This has an effect on the truncation of the infinite sum in Eq.~\ref{Equ:FMMBase} and can also act as a motivation why the system is split in near and far-field. If $||\bz_1 - \bz_2||$ becomes to small, the Hankel function becomes instable.

%In the expansion, the truncation $L$ of the infinite sum in Eq.~(\ref{Equ:FMMBase}) plays an important role. %and will be discussed in the next subsection.

%\subsubsection{Truncation of the FMM}\label{Sec:Truncation}
It is not trivial to find an adequate truncation parameter $L$ for the sum in Eq.~(\ref{Equ:FMMBase}) and a lot of approaches have been proposed~\cite{Coifmanetal93,CecDar13}. On the one hand, $L$ needs be high enough to let the sum in Eq.~(\ref{Equ:FMMBase}) converge. On the other hand, a large $L$ implies Hankel functions with high order, which  become numerically unstable for small arguments. Thus, the FMM is not recommended if two clusters are close to each other.
%mybox{Do we need the previous sentence? This is one of the reasons that in order for the multipole expansion to work properly the distance between their midpoints must be bigger then the sum of their radii. KP: yes!} 
In~\cite{Coifmanetal93}, a semi-heuristic formula for finding the optimal $L$ is given by
$$
L = k d_{max} + 5 \ln (k d_{max} + \pi)
$$
for single precision and 

$$
L = k d_{max} + 10 \ln (k d_{max} + \pi)
$$
for double precision, where $d_{\max}$ is maximum distance $||(\bz_i - \by) + (\bx - \bz_j)||$ over all clusters. As $L$ depends also on the wave number $k$, the order of the FMM is also frequency dependent. \numcalc{} uses a similar bound:
\begin{equation}\label{Equ:FMMTruncation}
L = \max(8,2 r_{\max} k + 1.8 \log_{10}(2 r_{\max} k +  \pi)),
\end{equation}
where $r_{\max}$ is the radius\footnote{The radius of a cluster is the biggest distance between the vertices in the cluster and the cluster midpoint.} of the largest cluster. %, i.e. 
Our numerical experiments have shown that a lower bound for $L$ can enhance the stability and accuracy (especially at lower frequencies) and that a factor of 1.8 in \label{Equ:Truncation} is a good compromise between stability, accuracy, and efficiency. In Sec.~\ref{Sec:Benchmark} we present an example where a bigger factor was used to increase the accuracy of the BEM solution in a special case. %Note, however, that a high $L$ may have negative effect on the stability of the solution.



%\subsubsection{Clustering}
%
In Eq.~(\ref{Equ:FMMBase}), the functions $\Es$ and $\T$ represent the local element-to-cluster expansion inside each cluster, whereas the function $\D_L$ represents the cluster-to-cluster interaction. One key element in $\D_L$ are the spherical Hankel functions $h_n$ of order $n = 0,\dots, L$, that become singular at 0, $\lim_{x\rightarrow 0} h_n(x) = O\left(\frac{1}{x^{n+1}}\right)$~\cite{AbrSte64}. Thus, the FMM expansion is only stable for relatively large arguments $k||\bz_1 - \bz_2||$, which implies that the FMM can only be applied for cluster-pairs that are sufficiently apart from each other. To this end, following~\cite{CecDar13}, \numcalc{} analyzes the relation between the distance $||\bz_i - \bz_j||$ of the cluster midpoints to their radii $r_i$ and $r_j$, which are given by the maximum distance between the cluster midpoint and the vertices of the cluster elements. If
\begin{equation}\label{Equ:Farfieldcond}
||\bz_i - \bz_j|| > \frac{2}{\sqrt{3}} || r_i + r_j ||,
\end{equation}
the two clusters are defined to be in each others farfield and the FMM expansion is applied. %Otherwise the two clusters are in each others nearfield. %This criterion is chosen based on \cite{CecDar13}.% where it was stated that for that value the FMM expansion converges absolutely and uniformly.%, in former releases of $\numcalc$ a slightly smaller value $\frac{\sqrt{5}}{2}$ was used, which lead to stability problems for very few selected cases.
%\mybox{ErwÀhnen, siehe beispiel unten}
%
%\mybox{Wir haben hier einen kleinen unterschied zu darve, der behauptet, dass ich eigentlich $2/sqrt 3$ brauche, damit das die summa absolut und gleichmaessig konvergiert, erwaehnen? KP: Zumindest kommentieren, ja!}
%

If two clusters do not fulfil Eq.~(\ref{Equ:Farfieldcond}), the clusters and their elements are defined to be in each others nearfield. The interaction between such cluster pairs will be calculated using the conventional BEM approach with the non-seperable $G(\bx,\by)$ leading to the sparse nearfield matrix $\bN$. If two clusters are found to be in each others farfield, the Green's function $G(\bx,\by)$ will be approximated by Eq.~(\ref{Equ:FMMBase}). %When using the FMM,  the computational effort of a matrix-vector multiplication can be reduced from $O(N^2)$ to $O(N\log_2N)$ depending on the flavour and implementation of the FMM, \cite{Coifmanetal93,Bettessetal04,OhnChe05,GumDur08}. \numcalc{} offers two versions of the fast multipole method: the single-level (SLFMM) and the multi-level (MLFMM) fast multipole method.

Note, that based on Eq.~(\ref{Equ:FMMBase}), the expansions for the derivatives of $G$ are easy, because $\bx$ and $\by$ only occur as arguments of  exponential functions, for which the calculation of the derivative is trivial.
% \subsubsection{The matrices ${\bf T,D}$ and ${\bf S}$}
% \subsubsection{Discretization of  $\Es,\D_L$, and $\T$}

%
%
\subsubsection{Cluster generation}\label{Sec:SLFMM}
To actually generate the clusters in \numcalc{}, a bounding box is put around the mesh. This box is then subdivided into (approximately) equally sized sub-boxes for which the average edge length can be provided by the user. Note, that the sub-boxes have the same edge length in all three dimension, thus, the number of sub-boxes may vary along the different coordinate axes.

If the edge length is not specified, the default edge length will be $|e_\text{b}| = \left(\sqrt{N}A_0\right)^{1/2}$, where $N$ is the number of elements and $A_0$ is the average area of all elements. The number of subdivisions is estimated by rounding the quotient of the edge length of the original box and the target edge length. Thus, the actual edge lengths of the sub-boxes can slightly deviate from the target length $|e_\text{b}|$. If the midpoint of an element is within a sub-box, the element is assigned to the box, and all elements inside the $i$-th sub-box form the cluster element $\mathcal{C}_i$. The choice of initial edge length can be motivated by the following idea: First, it is assumed that all elements have roughly the same size. Secondly, the assumption of having about $n_0 = \sqrt{N}$ clusters with about $N_0 = \sqrt{N}$ elements in each cluster provides a nice balance between the number of the clusters and the number of elements in each cluster. If we assume that the surface of the scatterer is locally smooth, the elements in each cluster cover approximately a surface area of $A_\Cl = N_0 A_0$. As each cluster is fully contained in one bounding box, a good estimate for the edge length of such a box is $\sqrt{A_\Cl} = \sqrt{N_0 A_0} = \sqrt[4]{N} \sqrt{A_0}$. Note, that the assumption of having elements of roughly the same size, has an impact on the efficiency of the method, but also influences the stability of the FMM expansion. If the size of the elements inside a cluster varies to much, the local element-to-cluster expansions may run into numerical troubles, thus:
\hint{The elements should have approximately the same size, at least locally.}
%
Sub-boxes that contain no elements are discarded. %In the remaining boxes, the cluster midpoint $\bz$ is defined as the average coordinate over all vertices of the sub-box elements and the cluster radius $r$ is defined as the maximum distance between the cluster midpoint and the vertices of the cluster elements.
Clusters with a small radius (in comparison with the average cluster radius) are merged with the nearest cluster with large size.  

\begin{figure}[!h]
  \begin{center}
    \includegraphics[width=0.3\textwidth]{ClustvsBox}
    \caption{Example of a cluster (blue) and the bounding (sub)box (grey) around it.}\label{Fig:ClustervsBox}
  \end{center}
\end{figure}

Note, that there is a logical difference between the sub-box and the cluster. A cluster is just a collection of elements, which form a 2D manifold, whereas the bounding box is a full 3D object (see also Fig.~\ref{Fig:ClustervsBox}). %, which is just a collection of the elements.
As a consequence, the radius of the cluster is usually \emph{not} the same as half the edge length of the sub-box. Also, the cluster midpoint is usually different from the midpoint of the sub-box. 
%
\subsubsection{FMM System of Equations}
Combining the FMM expansion Eq.~(\ref{Equ:FMMBase}) with the discretized BIE  Eq.~(\ref{Equ:SystemI}), the system of equations for the FMM has the form 
\begin{equation}\label{Equ:SystemSLFMM}
\bN + {\bf S D} {\bf T},
\end{equation}
in which $\bN$ contains the contributions for cluster-pairs in the near-field where the FMM cannot be applied, and ${\bf S, D}$ and ${\bf T}$ contain the contributions of far-field cluster pairs, where the FMM expansion will be applied. %The latter are the discretized versions of $\Es$, $\D_L$, and $\T$, respectively, representing the local interactions inside each cluster, cluster-to-cluster interactions, and the elements-to-cluster-midpoint interactions, respectively.
Because of the clustering, the matrices ${\bf S, D}$ and ${\bf T}$ have a block-structure, see Fig.~\ref{Fig:Blockstructure} for a schematic  representation of block structure of ${\bf SDT}$.

To better illustrate the sparse structure of these matrices, we look at the entries of the different matrices with respect to the integral parts involving the Green's function $G(\bx,\by$), the matrix parts for the integrals with respect to $H, H',$ and $E$ can be derived in a similar way.

The entries of the near-field matrix $\bN$ are calculated for each collocation node $\bx_i$ using the quadrature techniques for standard BEM (see Sec.~\ref{Sec:Quadrature}), resulting in
  $$
  \bN_{ij} = \left\{
    \begin{array}{cc}
      \int\limits_{\Gamma_j} G(\bx_i,\by)\de\by & \Gamma_j \in \N_i,\\
      0 & \text{otherwise},
    \end{array}
  \right. 
  $$
where $\Gamma_j$ is in the nearfield $\N_i$ of the cluster containing the collocation node, and $i,j = 1,\dots,N$. %, and the integral kernel $K(\bx_i,\by)$ is given by a combination of the Green's functions with its derivatives (see Eq.~(\ref{Equ:Theintegrals})).
    % (see Eq.~(\ref{SomeEquation})).
%  As $\Gamma_i \in \N(i)$ these steps includes the calculation of the singular and nearly singular integrals, see also Sec.~\ref{Sec:Quadrature}.
%\item 

The matrix ${\bf D}$ represents the cluster-to-cluster interaction between Clusters $\Cl_m$ and $\Cl_n$. It contains $2L^2$ blocks of size $N_C \times N_C$ defined by
  $$
  (\bD_\nu)_{mn} := \left\{
    \begin{array}{cc}
      \D_L(\bz_m,\bz_n,\bs_\nu) &\text{ if } (C_m,\Cl_n) \text{ is a far-field cluster pair},\\
      0 &\text{ otherwise},
    \end{array}
    \right. 
  $$
where $N_C$ is the number of clusters, $m,n = 1,\dots N_C$, and $\nu = 1,\dots,2L^2$, with $2L^2$ being the number of quadrature nodes on the unit sphere. Thus, the total size of ${\bf D}$ is $2 N_C L^2 \times 2N_CL^2$. 

The matrix $\bT$ describes the local element-to-cluster expansion from each element to cluster's midpoint and is represented as a ${2L^2}\times 1$ dimensional block matrix
  $$
  \bT = (\bT_1,\cdots,\bT_{2L^2})^\top,
  $$
  where each single block has size $N_C \times N$ and is defined by
  $$
  (\bT_\nu)_{mj} = \left\{
    \begin{array}{cc}
      \int\limits_{\Gamma_j}e^{\I (\bz_m - \by) \cdot \bs_\nu}\de\by & \Gamma_j\in \Cl_m, \\
      0 & \text{otherwise}.
    \end{array}
    \right. 
  $$
%
%  For each element $\Gamma_i$ and each quadrature node $\bs_j$ for the integral over the unit sphere, the local expansion matrix between element and midpoint $\bz_m$ of the cluster $\Cl_m$ containing $\Gamma_i$ is defined as
%  $$
%  \bT_{ij}(m) = \int\limits_{\Gamma_i} e^{\I (\bz_m - \by)\cdot \bs_j} d\by.
%  $$
%  The matrices $\bT(m)$ for the different clusters are collected using a sparse matrix format, where the entries are addressed based on the cluster number and the number of the quadrature node on the sphere. Note that this matrix parts are defined for each level, thus \emph{no} interpolation of entries between different levels is currently necessary. As already mentioned, this means on the one hand a slightly bigger memory consumption, but on the other hand, faster computations.%The matrices $\bT$ at root level can also be used when calculating the acoustic field at the evaluation nodes.
%
%    $\D_L(\bz_m,\bz_n,\bs_i)$ for each cluster midpoint  $\bz_m$, each cluster midpoint $\bz_n$ not in the near field of $\Cl_m$, and each quadrature node on the sphere $\bs_j$:
%    $$
%    \bD_L(z_i)_{mn} = \sum_{j=0}^L \I^j (2j + 1) h_j(k||\bz_m - \bz_n||)P_n\left(
%      k\frac{(\bz_m - \bz_n)\cdot \bs_i}{||\bz_m - \bz_n||}
%      \right)
%    $$
The local distribution matrix $\bS$ of size $1\times 2L^2$ is defined by 
  $$
  \bS = (\bS_1,\dots,\bS_{2L^2}),
  $$
where each $N\times N_C$ block has the form
  $$
  (\bS_\nu)_{in} = \left\{
    \begin{array}{cc}
      w_\nu e^{\I(\bx_i - \bz_n)\cdot \bs_\nu} & \bx_i \in \Cl_n,\\
      0 & \text{otherwise},
    \end{array}
  \right.
  $$
where $w_\nu$ is the $\nu$-th weight used by the quadrature method for the integral over the unit sphere.
\begin{figure}
  $$
  \left(
\begin{array}{ccc}
  \fbox{\parbox[][100pt][c]{20pt}{%
  \phantom{ai}$\bS_1$
  }
  } &
  \cdots &
  \fbox{\parbox[][100pt][c]{20pt}{%
  \phantom{i}$\bS_{2L^2}$
  }
  }
\end{array}
\right)
\left(
  \begin{array}{ccc}
    \fbox{\parbox[][20pt][c]{20pt}{%
    \phantom{a}$\bD_1$
    }}
    & 0 & 0\\
    0 & \ddots & 0 \\
    0 & 0 &
            \fbox{\parbox[][20pt][c]{20pt}{%
            $\bD_{2L^2}$
            }}
  \end{array}
\right)
\left(
  \begin{array}{c}
    \fbox{\parbox[][20pt][c]{100pt}{%
    \centering
    $\bT_1$
    }
    }
    \\[13pt]
    \vdots\\[13pt]
    \fbox{\parbox[][20pt][c]{100pt}{%
    \centering
    $\bT_{2L^2}$
    }
    }
  \end{array}
\right)
$$
\caption{Block structure of the FMM matrices in the SLFMM. Each matrix block $\bS_i$ has size $N\times N_C$, where $N$ is the number of elements and $N_C$ is the number of clusters. Each block $\bD_i$ has size $N_C\times N_C$ and each block $\bT_i$ has size $N_C \times N$.}\label{Fig:Blockstructure}
\end{figure}
%
%The combination ${\bf S\cdot D\cdot T}$ can be seen as doing first the local expansion for each cluster followed by passing the local information to other clusters using ${\bf D}$. Finally, the integral over $\sphere$ is calculated and the information on the cluster is distributed to all its elements.
%
\subsubsection{Multi level FMM (MLFMM)}
For higher efficiency, the FMM also can be used in a multilevel form, in which a cluster tree is generated, i.e., the mesh is subdivided not only once into clusters in the first level (root level), but clusters are iteratively subdivided in the next level, until the clusters at the finest level (leaf level) have about 25 elements. As the FMM is based on multiple levels, we speak of the multilevel fast multipole method (MLFMM) compared to the single level fast multipole method (SLFMM) described in Sec.~\ref{Sec:SLFMM}.

In contrast to other implementations of the MLFMM where the cluster on the root level contains all elements of the mesh, in \numcalc{}, the root of the cluster tree is based on a clustering similar to that done in the SLFMM (see Sec.~\ref{Sec:SLFMM}).  In \numcalc{}, users can either define an initial bounding box edge length $|e_{\text{r}}|$ or use the default target edge length chosen such that each cluster contains approximately $N_0 = 0.9 \sqrt{N}$ elements. As in the SLFMM, it is assumed that the elements have similar sizes with an area of $A_0$ and an average edge length of $|e_0|$.
%
\begin{figure}[!h]
  \begin{center}
    \includegraphics[width=0.25\textwidth]{FMMMesh01a.pdf}
    \hspace{10pt}
    \includegraphics[width=0.25\textwidth]{FMMMesh1.pdf}
    \hspace{10pt}
   % \includegraphics[width=0.2\textwidth]{FMMMesh2.pdf}
    %\hspace{10pt}
    \includegraphics[width=0.25\textwidth]{FMMMesh3.pdf}
    \caption{Example of the MLFMM clustering: Mesh (dark/blue line) and the bounding boxes for the first two levels. At each level, empty (white) boxes are discarded, small (dark grey/red) boxes are merged into the neighbouring large (light grey/green) boxes, which are used for the clustering. }\label{Fig:Clustering}
  \end{center}
\end{figure}

The clusters at higher levels are constructed in two steps: First, new bounding boxes are created for each cluster based on the coordinates of element vertices in the respective cluster. Then, sub-boxes with target edge length $\frac{|e_{\text{r}}|}{2^\ell}$ are created similar to the root level, where $\ell$ is the number of the level. The elements within these sub-boxes then form the clusters in the new level of the MLFMM. The clusters contained in the sub-boxes in the new level are called children, the original cluster element is called parent cluster. Fig.~\ref{Fig:Clustering} shows an example of creating the bounding boxes and clusters for the root and next level. %While at each level, the boxes and the clusters are created in a similar way as in the SLFMM, for the next level of the MLFMM, new bounding boxes are placed around the clusters, and these bounding boxes are then subdivided into eight sub-boxes.


Currently, \numcalc{} does \emph{not} build the cluster tree adaptively, but the number of levels
$$
\ell_{\max} = \max\left(1, \text{round}\left[\log_2\left(\frac{|e_{\text{r}}|}{|e_{\text{l}}|}\right)\right]\right) = \max\left(\text{round}\left[\frac12 \log_2\left(\frac{\sqrt{N}}{22.5}\right)\right]\right),
$$ 
is determined beforehand, where $|e_{\text{r}}|$ and $|e_{\text{l}}|$ are the target bounding box lengths on the root and leaf level, respectively. %and thus the number of subdivisions is fixed beforehand. %This is based on two assumptions that each leaf clusters should contain about $N_\ell = 25$ elements.
Because in the default setting, each cluster on the root level contains $N_0 = 0.9 \sqrt{N}$ elements, the edge length of each bounding box can be estimated by $|e_{\text{r}}| = \sqrt{\frac{N}{N_0}} \sqrt{A}$. On the leaf level, each bounding box contains about 25 elements, thus it has an edge length of $|e_{\text{l}}| = 5 \sqrt{A}$. In that setting, $\ell_{\max}$ defines how often $|e_{\text{r}}|$ needs to be split to obtain $|e_{\text{l}}|$.  \numcalc{} produces cluster trees which are well-balanced and where all leaf clusters are on the same level. Note, however, that the bounding boxes are constructed to have about the same edge length in all directions, thus, if the mesh has large aspect ratios, the number of clusters in the respective dimension will differ.

%
% Similar to the SLFMM case, the clusters on each level are constructed based on each sub-box. Again, if a cluster is too small in terms of cluster radius it is merged with the nearest cluster that is big enough.

In practice, elements are rarely distributed regularly inside the bounding boxes. Thus our clustering approach may lead to leaf clusters with very few elements on the one hand and clusters with more then 25 elements on the other hand. Our numerical experiments have shown that having already about $\sqrt{N}$ clusters in the root level improves the stability and efficiency of the FMM, because, for example, the expansion length used by the FMM  (see Sec.~\ref{Sec:Expansion}) is kept relatively low.%This is partly because of the specific implementation of the MLFMM in \numcalc{}.


In order to solve the MLFMM case, nearfield and farfield cluster-pairs are defined on each level using the same rule as for the SLFMM. However, in contrast to the SLFMM, the FMM is \emph{not} applied for all far-field cluster pairs on each level, but only to a specific sub-set, the so called interaction list. For each cluster $\Cl_j$, the interaction list $I(\Cl_j)$ is defined as the elements in the child clusters of the near field clusters of the parent of $\Cl_j$.

Calculations for the MLFMM start at the leaf level $\ell_{\max}$. For each  cluster $\Cl_i^{\ell_{\max}}$ on this level, the near field matrix ${\bf N}$ is calculated for all near field cluster pairs as in the SLFMM case (for example the blue box and the two red boxes in Fig.~\ref{Fig:MLFMM}). The fast multipole expansion, however, is only applied between each cluster $\Cl^{\ell_{\max}}_i$ and the clusters in the respective interaction list $I(\Cl_i^{\ell_{\max}})$ (green boxes in Fig.~\ref{Fig:MLFMM} for the selected blue cluster box). All other cluster-pairings (white boxes) on this level are neglected.

For those pairs, the local element-to-cluster expansions at level $\ell_{\max}$ are transformed into local element-to-cluster expansions with respect to the parent cluster in level $\ell_{\max}\!-\!1$ (upward pass, blue box to parent blue box in Fig.~\ref{Fig:MLFMM}). This transformation is necessary because the parent cluster has a different cluster midpoint and maybe a different multipole expansion length as the child. At this level no near-field components need to be calculated, because these have already been ``covered'' by the calculations at the level of the children. For each pairing of cluster $\Cl^{\ell_{\max}-1}_i$ and clusters in its interaction list, the FMM is performed. % Thus, on this level the FMM needs to be done, which is again restricted only to pairings of clusters $\Cl_i$ and clusters in their interaction list. 
For all other cluster pairs the local expansions are again passed on to the parent, and the procedure is repeated at the level above.



%The system is built by starting at the leaf level. For each cluster $\Cl^{\ell_{\max}}_i$ on this level, the interaction between nearfield clusters is calculated using the conventional BEM leading to the nearfield matrix $\bN$. For clusters in the interaction list, the FMM-expansion Eq.~(\ref{Equ:FMMBase}) is used, thus, the entries of the  local element-to-cluster expansion matrices $\bT(\ell_{\max})$,  cluster- to- cluster-interaction matrices $\bD({\ell_{\max}})$, and cluster-to-element matrices $\bS({\ell_{\max}})$ are determined. For all other cluster pairs (white boxes) on this level, nothing is done yet.

 \begin{figure}[!h]
  \begin{center}
%    \includegraphics[width=0.8\textwidth]{MLFMM8}
    \includegraphics[width=0.8\textwidth]{mlfmmbw.png}
  \end{center}
  \caption{Scheme of the MLFMM. For a single cluster on the highest level $\ell_{\max}$ (= leaf level, shown in the bottom row by the dark blue box), the red boxes denote its nearfield clusters and the light green boxes denote the clusters stored in the interaction list, for which the FMM is applied. The interaction with the other clusters within this level (denoted by the white boxes) are considered in the parent level $\ell_{\max}\!-\!1 $. This procedure is repeated for all levels of the MLFMM. Arrows depict interactions between clusters and messages passing to the parent, dotted arrows depict a parent-child relation.}\label{Fig:MLFMM}
\end{figure} 

In summary, at each level $\ell$, the local element-to-cluster matrices $\bT(\ell)$, the local cluster-to-element matrices $\bS(\ell$), and the cluster-to-cluster interaction matrices $\bD(\ell)$ will be calculated, and the final system for the MLFMM is given by
%For these elements, the local expansion between element and cluster midpoint is transformed to a local expansion between element and the midpoint of the parent cluster (upward pass between cluster and its parent, blue boxes in Fig.~\ref{Fig:MLFMM}), resulting in new matrices $\bT(\ell)$ that compared to most other inplementations of the FMM are calculated and stored directly in \numcalc{}. Between clusters in the interaction list of the parent clusters we again define the cluster-to-cluster interaction matrices $\bD(\ell)$. On each level the matrices $\bS(\ell)$ pass down the information from parent to child cluster and handle the interactions between cluster midpoint and elements on the leaf level including the quadrature over the unit sphere. For clusters neither in the nearfield nor in the interaction list, the cluster information is again passed to the parent at level $\ell - 1$ and the procedure is repeated.  In the end the system will be given by the sum of several sparse matrices
$$
N(\ell_{\max}) + \sum_{\ell = 1}^{\ell_{\max}} \bS(\ell) \bD(\ell) \bT(\ell).
$$


While the MLFMM is recommended for most of the problems, \numcalc{} implements a few ``safeguards'' focused on an efficient calculation and stable results:
\begin{itemize}[leftmargin=*]
\item If the user selects the conventional BEM (i.e., no FMM) but the number of elements $N$ is larger than $20000$, then \numcalc{} will still use the FMM, namely, the SLFMM.
\item If the user selects the conventional BEM or the SLFMM, and the number of elements $N$ is larger than $50000$, then \numcalc{} will automatically switch to the MLFMM.
\item If the user selects the FMM, but the wavelength $\lambda$ is $80$ times larger (or more) than the maximum distance $r_{\max}$ between two nodes of the mesh, then \numcalc{} will sill use the conventional BEM (i.e., \emph{without} the FMM). This may especially the case at low frequencies and prevents the too small truncation of the multipole method (see Section~\ref{Sec:Expansion}).
\item If the user selects the MLFMM, but $\lambda$ is larger than $r_{\max}$ then \numcalc{} automatically switches back to the SLFMM.
\end{itemize}


One specific detail in \numcalc{} is the fact, that %as the cluster midpoints and the expansion lengths will, in general, vary from level to level,
\numcalc{} determines the matrices $\bT(\ell), \bD(\ell)$ and $\bS(\ell)$  beforehand for \emph{all} levels and explicitly keeps those sparse matrices in memory.
This is in contrast to many other implementations of the FMM, that only calculate the expansions on the leaf level, and then pass the information between levels by interpolation and filtering algorithms (cf.~ \cite{AmiPro03}).
The approach in \numcalc{} has the advantage of a faster calculation because no interpolation and filtering routines are necessary for a slightly higher memory consumption at least for mid-size problems.

%
%
Nevertheless, one drawback of storing $\bT(\ell)$ and $\bS(\ell)$ on each level $\ell \le \ell_{\max}$ is that these matrices may become large close to the root level, which can cause problems with memory consumption. The radii of the clusters are larger at smaller levels (close to the root of the cluster tree), %grows for smaller levels $\ell$,
and thus the truncation parameter $L$  in the multipole expansion (see Eq.~(\ref{Equ:FMMTruncation})) and, in turn, the number of necessary quadrature nodes for the integral over the unit sphere increases when getting closer to the root. This implies that the number of blocks in $\bT(\ell)$ and $\bS(\ell)$ also increases towards the root, which means higher memory consumption. %We will see in the next section that the number of terms in the expansion is related to the largest radius of the clusters.
However, with the default of about $\sqrt{N}$ clusters at root level, large clusters can be usually avoided and the memory consumption can be kept at a feasible level.
% matrices on \emph{all} levels of the MLFMM are stored explicitly.

%To illustrate the memory consumption, we provide a rough estimate on the non-zero entries of the different matrices. In Sec.~\ref{Sec:NonZero}, these values are compared to actual computations with a benchmark problem of the scattered field of a sound-hard sphere. %Additionally, since \numcalc{} version 1.0, there is the option, to just calculate the number of non-zeros at each frequency, without having to do the actual BEM calculations. Based on this number, a lower bound for the memory requirements can be given.
\subsubsection{Memory estimation SLFMM}\label{Sec:SLFMMEstim}
% The default approach of clustering in the single level is to get  about $\sqrt{N}$ clusters with $N_0 = \sqrt{N}$ elements per clusters, where $N$ is the number of boundary elements.
The number of non-zero entries of the nearfield matrix $\bN$ in Eq.~(\ref{Equ:SystemSLFMM}) is given by the product of the number of interactions between the elements of two clusters, the number of possible nearfield clusters for a cluster, and the number of all possible clusters. As every cluster has about $\sqrt{N}$ elements, %, where $N$ is the number of all elements,
a single nearfield cluster-to-cluster interaction adds  $N$ non-zero entries to the nearfield matrix. In the SLFMM the number of clusters is about $\sqrt{N}$, in the worst case, each cluster has 27 clusters in its nearfield, because a bounding box shares at least a face, an edge, or a vertex with 27 boxes (including itself). Therefore, in the worst case, the nearfield matrix $\bN$ has $27 N^{3/2}$ non-zero entries. However, one has to distinguish between the 3D bounding box and the cluster itself (see also Fig.~\ref{Fig:ClustervsBox}). In practice, one can assume that each cluster has on average about 9 to 10 nearfield clusters. for most calculations, it is fair to assume that the matrix has about $9 N^{3/2}$ to $10 N^{3/2}$ non-zero entries. This assumption works well for small clusters where the surface of the scatterer inside the cluster has only little curvature. For more complex geometries and larger clusters, more non-zero entries are necessary.

% The number of nearfield clusters can be motivated as follows: If the nearfield of a bounding box is defined as the number of all boxes that share at least one common vertex with the box, then every bounding box has 27 boxes in its nearfield, including the box itself. Thus, theoretically, there are up to 27 clusters in the nearfield of each cluster.
 %For most clusters, the elements will form an almost plane 2D manifold, thus, the cluster diameter, i.e. the biggest distance between two points of the cluster, may also be different from the edge length of the bounding box.
%
%Thus,  %because locally the clusters lie more or less in a plane. Two clusters are neighbours if they are in each others near-field.
The cluster in Fig.~\ref{Fig:ClustervsBox}, for example, would probably have 9 clusters in its nearfield, one for each edge and each vertex of the cluster patch plus the patch itself. However, especially at lower cluster levels with bigger cluster radii, the assumption of low curvature is not always correct. In this case, the number of nearfield cluster can become larger than 9 or 10.


The local element-to-cluster expansion matrices $\bT$ and $\bS$ both have exactly $2L^2 N$ non-zero entries, where $2L^2$ is the number of quadrature points on the sphere that depends on the truncation parameter $L$. As each element lies in exactly one cluster, there is only one local interaction between element and cluster midpoint, thus there are $N$ local element-to-cluster expansions. Therefore, the number of $2L^2N$ non-zero entries. %As already mentioned, the number of quadrature nodes on the sphere depends on the truncation $L$ of the multipole expansion. The sphere is parameterized using spherical coordinates, in the azimuth-direction  $2L$ quadrature nodes, in the elevation  $L$ quadrature nodes are used.

For the cluster-to-cluster interactions (matrix entries of $\bD$), it is assumed that each cluster has about 9 clusters in the nearfield. The number of non-zero entries of the midpoint-to-midpoint matrix is a product of  number of clusters, number of far-field clusters, and the number of quadrature nodes on the unit sphere. On average $\sqrt N(\sqrt N - 9) 2L^2$ non-zero entries can be expected in total. The accuracy of this estimation depends on the uniformness  of the discretization and the smoothness of the scattering object.
%In principle, the midpoint-to-midpoint interactions are symmetric, however, because the implementation on how the matrix components are calculated and addressed, the symmetry is not used in the current version of \numcalc{}.
%\mybox{ich wÌrde vorschlagen, wir erwÀhnen den Punkt bevor ein reviewer es tut. Weil im Abschnitt mit dem Benchmark sind die Daten so gegeben. Eine alternative wÀre die symmetry zu implementieren, aber dann heisst es wieder, ich arbeite zu viel am code}
%
%
%
\subsubsection{Memory Estimation MLFMM}\label{Sec:MLFMMEstim}
% Compared to other implementations of the MLFMM,  the default approach in \numcalc{} aims at a fixed number of clusters on the root level that is set to approximately $N_0 = 0.9\sqrt{N}$. %It is also possible for the user to provide an initial edge length of the bounding box for the root level.
We assume that the scatterer has a regular geometry without many notches and crests, thus each cluster can be assumed to have about 9 to 10 nearfield clusters, and although a parent box can be divided into 8 child-boxes, the same 2D manifold argument as above is used in the estimation that each parent has about 4 to 5 children.

In contrast to the SLFMM, the non-zero entries of the nearfield matrix $\bN$  now depend on the number of clusters on the highest (leaf-)level and the (average) number of elements in the leaf clusters. %In the literature concerning the FMM (cf. \cite{Coifmanetal93}), it is suggested that each cluster on the leaf level should contain about 25 elements.
The number of non-zero entries on the leaf level can be estimated by the product of the number of clusters at the leaf level, the number of element-to-element interactions between two clusters, and the number of nearfield clusters. Therefore, the nearfield matrix $\bN$ has about  $\frac{N}{25}\cdot 25^2 \cdot 9 = 225 N$ entries.%, where again $N$ is the number of all BEM elements.

The local element-to-cluster expansion matrices $\bT(\ell)$ and $\bS(\ell)$ depend on $L_\ell$, the number of terms used in the multipole expansion. On a given level, $\bT(\ell)$ and $\bS(\ell)$ have exactly $2 L^2_\ell N$ non-zero entries each.

On the root level, the number of non-zero entries of the midpoint-to-midpoint interaction matrix can be derived similar to the SLFMM. For higher levels, the number non-zero entries of $\bD(\ell)$ is given by the product of the number of clusters at the level, the number of quadrature nodes on the sphere, and the number of clusters in the interaction list of each cluster. Under the  assumption that each cluster has about 4 children and about 10 neighbouring clusters (without the cluster itself), the number of non-zero entries for $\bD(\ell))  \approx 4^{\ell-1} 0.9 \sqrt{N} \cdot 2 L^2_\ell \cdot 40,\ell > 1$. In Section \ref{Sec:NonZero} we will compare these estimates with the actual number of non-zeroes in the different multipole matrices for a simple example, which will show, that the estimates can only be rough estimates, because the assumptions about the number of clusters on each level is rather vague.%It is reasonable to assume that each parent cluster has about 4 children and about 9 neighbours.
%
\section{Critical components of \numcalc{} and Their Effect}\label{Sec:Critical}
While users may not be necessarily interested in the details covered in Sections~\ref{Sec:BriefBEM} and \ref{Sec:Implementation}, %Eqs.~(\ref{Equ:BIE_I}) and (\ref{Equ:SystemI}) and details of the implementation,
they should be aware of some of the consequences of the assumptions used in these sections and the key components of the BEM and the FMM.

\subsection{Singularity of the Green's function}\label{Sec:GreenSgl}
The Green's function $G(\bx,\by)$ and its derivatives $H(\bx,\by), H'(\bx,\by),$ and $E(\bx,\by)$ are essential parts of the BEM. These functions become singular whenever $\bx = \by$, i.e., when we integrate over the element $\Gamma_i$ that contains the collocation point $\bx_i$. %In that case $G$ behaves like $\frac{1}{r}$, $H$ behaves like $\frac{1}{r^2}$ and $E$ behaves like $\frac{1}{r^3}$, where $r = ||\bx - \by||$.
There are algorithms that can deal with these type of singular integrals (cf.~\cite{Duffy82, Krishnasamyetal90}), %This is not a problem per se, because there exist special numerical routines to deal with these singularities \cite{Krishnasami}, however, 
but the singularities also cause \emph{numerical} problems for elements in the neighbourhood of $\Gamma_i$. In \numcalc{}, these almost-singular integrals are calculated by subdividing the element several times (see also Section \ref{Sec:Quadrature}), but these subdivisions slow down calculations. The (almost) singularity especially causes problems if
\begin{itemize}%[leftmargin=*]
\item an external sound source is close to the surface of the object,
\item there are very thin structures in the geometry, thus, if front and backside of the object are close to each other,
\item there are overlapping or twisted elements.
\end{itemize}
%
If \numcalc{} terminates with an error code message ``\texttt{number of subels.\ which are subdivided in a loop must <= 15}'' the reason is most likely the presence of irregular or overlapping elements, causing more then 15 subdivisions of an element. \numcalc{} also displays the element index for which the error occurred. There is a high chance, that there is some problem with the mesh or an external sound source close to this element. So:
% 
\hint{If you get a subdivision loop error, check your mesh in the vicinity of the element for overlaps and irregularities, or if there is a external sound source close to that element, move the source0 further away from the mesh.}
%\mybox{Regulaere elemente schon hier erwaehnen?}

%\subsection{Constant shape functions}
%Constant shape functions mean that, in order to guarantee a good approximation of the acoustic field, the number of elements $N$ depends on the highest frequency used in the calculations.  Also, the convergence rate of the method, i.e., the accuracy of the calculation as a function of  the number of elements $N$, decays slowly compared to other methods like Galerkin with linear or quadratic elements. %(see also Sec.~\ref{Sec:Benchmark} for examples how the error depends on the mesh size $N$). 

%Constant shape functions have the advantage that if the acoustic field is not regular and has discontinuities (e.g., at corners or when boundary conditions change abruptly), the discontinuity provided by constant shape functions may become advantageous in representing the field. Also, constant shape functions can be subdivided without the need to consider adjacent elements, therefore, adaptive numerical BEM methods can be implemented easier. 

\subsection{FMM matrices}\label{Sec:FMM-Matrices}
%
The expansion Eq.~(\ref{Equ:FMMBase}) of the Green's function has several consequences especially in connection with finding the right truncation length, see Section~\ref{Sec:Expansion}.
For small arguments, the spherical Hankel function $h_n(x) = j_n(x) + \I y_n(x)$ of order $n$ behaves like~\cite{AbrSte64}
$$
\lim_{x\rightarrow 0} j_n(x) = \frac{1}{1\cdot3\cdot5\cdots (2n+1)}{x^n},\quad
\lim_{x\rightarrow 0} y_n(x) = -\frac{1\cdot3\cdot5\cdots(2n-1)}{x^{n+1}}.
$$
Thus, numerical problems may arise for large expansion lengths and small arguments because the spherical Hankel function goes to infinity with $O(\frac{1}{x^{n+1}})$. This is a problem, when $k||z_1 - z_2|| \ll 1$, which can happen at low frequencies and clusters being close to each other.
%  For large arguments the spherical Hankel functions behaves like $\approx (-\I)^{l+1}\frac{e^{\I x}}{x}$
The number of quadrature nodes on the sphere is a function of the truncation parameter $L$, which in turn means that the memory required to store the local-to-cluster expansion matrices $\bT$ and $\bS$ (having $2L^2 N$ non-zero entries) depends on $L$. As $L$ depends on the cluster radius (e.g., Eq.~\ref{Equ:Truncation}) choosing the size of the initial bounding box has a huge effect on the memory requirement. %For very large $L$ the memory needed to store the matrices $\bT$ and $\bS$ can become quite large (see also the example in Section~\ref{Sec:MLFMM4}).
%  It will be part of future developments to provide interpolation/filtering options for these matrices.
The spherical Hankel functions are implemented with a simple recursive procedure~\cite[Section 3.2.1]{Giebermann97} with the aim to find a balance between accuracy and computation time. However, for higher orders, the calculations become more involved, and, as the calculation is based on a recursive formula, errors from lower orders propagate and add up at higher orders.

%
Hence, the length of the FMM expansion has a crucial effect on the results of the FMM. Thus:
\hint{If you need an expansion order $L$ beyong 30, check the convergence of the iterative solver and then your results for plausibility.}
If the large expansion length $L$ causes problems, one remedy is to adjust the length of the initial bounding box such that the radii of the clusters decrease.

\section{Benchmarks}\label{Sec:Benchmark}
In order to raise awareness on the limitations of \numcalc{} in terms of accuracy and computer resources, in the following, we present results for three benchmark problems in acoustics. First, we analyze the scattering of a planewave on a sound-hard sphere. Second, we analyze the calculation of the sound pressure on a human head caused by point-sources placed around the head. Finally, we analyze the calculation of the sound field inside a duct. 

\subsection{Sound-Hard Sphere}
We consider the acoustic scattering of a plane wave on a sound-hard sphere with radius $r = 1$\,m. For this example, an analytic solution based on spherical harmonics~\cite[Chapter 7]{Williams99} is available for comparison for both on the surface and on points outside the sphere. To investigate the effect of various discretization methods, the meshes were created using two methods. The first type of meshes (cube-based meshes) was created by discretizing the unit cube with triangles and by projecting the discretized cube onto the unit sphere. The second type of mesh (icosahedron-based meshes) was the often-used construction using an (subdivided) icosahedron that is projected onto the sphere. Both meshes were discretized using triangles, thus, each element had a unique normal vector.

In general, the icosahedron-based meshes had almost equilateral triangular elements of about the same size, but the number of elements were restricted to $N_{\text{ico}} = 20\cdot4^n, n\in \mathbb{N}_0$. The number of elements for the the cube-based meshes were only restricted to $N_{\text{cube}} = 6\cdot (n+1)^2, n\in \mathbb{N}_0$, but the triangles were not as regular as in the icosahedron-based meshes, and the size of the triangles differed over the sphere, especially near the projected edges of the cube. An example for both meshes is depicted in Fig.~\ref{Fig:Spheremesh} where the sphere on the left side is based on a projected cube and has 5292 elements, and the mesh on the right is based on a icosahedron and has 5120 elements. In Section~\ref{Sec:Regularity}, the acoustic field for both spheres was calculated and the effect of the different discretizations are discussed.

In \numcalc{}, the acoustic field on the surface is calculated on the \emph{collocation} nodes, which, in this example, do not lie on the surface of the unit sphere. Thus, the errors depicted here also contain some error introduced by the geometry. For that reason, we also compared the numerical solutions to the analytical solution for a sphere with a radius adjusted to the minimal distance of the collocation nodes to the center of the sphere. This has an effect on the errors reported here, but this effect was tiny because of the accurate discretization of the sphere by the triangles.
%  \begin{figure}[!h]
%    \begin{center}
%    \includegraphics[width=0.8\textwidth]{Sphereexample.pdf}
%    \caption{Example of a mesh of the sphere}
%    \end{center}
%  \end{figure}
%
\begin{figure}[!h]
  \begin{center}
%    \includegraphics[width=0.44\textwidth]{Sphere5000cube}
%    \includegraphics[width=0.44\textwidth]{Sphere5000Ico}
    \includegraphics[width=0.48\textwidth]{SpherePatch}
    \includegraphics[width=0.48\textwidth]{SphereIco}
    \caption{Discretizations of the unit sphere used in our benchmark. On the left side, the mesh was generated by discretizing a cube and projecting it onto a sphere. Note the smaller triangles close to the projected edges of the cube. On the right, the mesh is based on a subdivided and projected icosahedron.}\label{Fig:Spheremesh}
  \end{center}
\end{figure}

To illustrate the combination of incoming and scattered  acoustic field and investigate the errors in the acoustic field outside the sphere, 24874 evaluation points have been placed on a plane around the sphere. Figure~\ref{Fig:BenchmarkField} shows the logarithmic sound pressure level (SPL) in dB as an example of such acoustic field calculated for two frequencies. These evaluation points were also used to compute the errors between the BEM solution and the analytic solution outside the sphere. In the calculations, the MLFMM was used with the default settings.


\begin{figure}[!h]
  \begin{center}
  \includegraphics[width=0.48\textwidth]{NC200Hz.jpg}
  \includegraphics[width=0.48\textwidth]{AcouBEM1000Hz.jpg}
  \caption{Example of the acoustic field calculated for 200\,Hz (left panel) and 1000\,Hz (right panel) caused by a plane wave in the $z$-direction.}\label{Fig:BenchmarkField}
  \end{center}
\end{figure}

%%%%%%%%%%%%%%%%%%%%
%  Error
%%%%%%%%%%%%%%%%%%%%

\subsubsection{Effect of element regularity}\label{Sec:Regularity}
Fig.~\ref{Fig:ErrorSphere} shows the difference between the numerical and the analytical solutions, illustrating the effect of element regularity. The color represents the difference between the SPLs calculated at the midpoint of each element on the sphere. The effect is shown for the frequency of $f = 800$\,Hz and the two different types of discretizations. Note that both meshes fulfil the 6-to-8-elements per wavelength rule for frequencies up to about 950\,Hz. 

Despite having a similar number of elements, the different regularity in the two meshes has a clear effect on the errors. For the very regular icosahedron-based triangularization, the errors between the calculated and the analytic SPLs were between $\pm 0.34$\,dB. For the cube-based triangularization, the errors were larger, i.e., between $-0.7$ and $0.9$\,dB, being considerably larger at the triangles around the projected cube edges. In contrast to the icosahedron-based mesh, for the cube-based mesh, we observed a clearer distinction in the errors between the ``sunny'' and ``shadow'' side, i.e., the side oriented towards the plane wave and the side oriented away from the plane wave. For the icosahedron-based mesh, the error distribution was more smooth across the two sides. 

%On a sidenote,  the clustering used in the FMM varied slightly between the two meshes. For the mesh based on the cube, the number of clusters at levels 1 and 2 were $N_1 = 90$ and $N_2 = 356$, the expansion lengths were $\ell_1 = 13$ and $\ell_2 = 8$. For the icosaeder based mesh $N_1 = 90$, $N_2 = 344$, $\ell_1 =12$, and $\ell_2 = 8$.

%  \begin{figure}[!h]
%      \begin{center}
%        \includegraphics[width=0.3\textwidth]{ErrorPatch5292_1000_2}
%        \includegraphics[width=0.3\textwidth]{ErrorPatch5292_1000Darkside_2}
%      \includegraphics[width=0.3\textwidth]{ErrorIco5120_2}
%    \end{center}
%    \caption{Difference between the dB sound pressures for the BEM calculation and the analytical solution at $f_2 = 1000$\,Hz. Left: ``Sunny'' side, discretization based on a projected cube. Middle: ``Shadow'' side, discretization based on a projected cube. Right: Discretization based on an icosphere, ``sunny'' side.}\label{Fig:ErrorSphere}
%  \end{figure}
\begin{figure}[!h]
  \begin{center}
    \includegraphics[width=0.75\textwidth]{ErrorSunnyDark800Hz.pdf}
    \caption{Sound pressure level differences (in dB) between the BEM calculation and the analytical solution at $f = 800$\,Hz. a) Cube-based mesh, b) Icosahedron-based mesh. As external source a plane wave in the negative $z$ diretion was used.}\label{Fig:ErrorSphere}
  \end{center}
\end{figure}
%
\subsubsection{Error as function of the number of elements $N$}\label{Sec:Errororder}
In this section, we analyzed the effect of the number of elements. To this end, we calculated the relative error
$$
% \varepsilon_r := \frac{|p_{\text{BEM}} - p_{\text{analy}}|}{|p_{\text{analy}}|}
\varepsilon_r(\bx) := \frac{|p(\bx) - p_{0}(\bx)|}{|p_{0}(\bx)|}
$$
between the calculated acoustic field $p$ and the quasi-analytical solution $p_{0}$ given in, e.g., \cite[Chapter 7]{Williams99}. For the calculations, we considered  the error on the surface as well as outside the sphere as a function of the number of elements $N$. Further, we used the cube-based meshes only because the generation of the mesh based on the cube was more flexible in terms of available numbers of elements $N$.

%  In general, $O(1/h)$ is given but there may be jumps in the error, when the number of levels change
%   
%  Should work for sound soft as well as sound hard
%   
%  200 Hz problem at 40000 elemens, bei 100 Hz ein wenig frÃŒher, mit rwfact = 15 kann das behoben werden
%   
%  1000 Hz no problem at all, because the levels are in general higher
%   
%  problem kleine frequenzen feine grids
%   
%  ist es ein problem wenn cluster nur aus einem element bestehen
%   
%   
%  betrachten wir n = 55 thus 34992 elemente so gibts mit traditional bem
%  max(e1)/mean(e1)
%  ans =  0.022003/0.0044199
%  max(e2)
%  ans =  0.0018830/0.00034989
%  max(e3)
%  ans =  0.012982/0.0043212
%  max(e4)
%  ans =  0.012955/0.0043322
%   
%  mit rw 15
%  0.030432/9.6916e-03
%  5.2839e-3/9.1410e-04
%  0.021003/9.4678e-03
%  0.020948/9.4631e-03
%   
%  mit rw 1.8
%  0.033550/0.012497
%  7.6685e-03/1.3163e-03
%  0.024484/0.012291
%  0.024425/0.012286
Figs.~\ref{Fig:Error200} and~\ref{Fig:Error1000} show the maximum, mean, and minumum of the relative errors calculated for the frequency of 200 and 1000\,Hz, respectively. It is known, that the numerical error of the collocation method with constant elements is linearly proportional to the average edge length of the mesh. %To provide a rough estimate on the lower bound given by the discretization expected for collocation with constant elements,
To illustrate this theoretical behaviour, the dotted line represents the functions $\frac{1}{\sqrt{N}}$ which is roughly proportional to the average edge length of the elements. In principle, the errors are in the range of $O(\frac{1}{\sqrt{N}})$, however, for the lower frequency (Fig.~\ref{Fig:Error200}), the errors increased with the number of elements. %This indicates that the BEM has numerical problems with fine grids at low frequencies.
For \numcalc{}, it can be shown that this effect is caused by the following specific details of the implementation:

\begin{figure}[!h]
%  \includegraphics[width=0.48\textwidth]{MaxError200new}
%  \includegraphics[width=0.48\textwidth]{MeanError200new}
  \includegraphics[width=0.48\textwidth]{ErrorOrder200Surface}
  \includegraphics[width=0.48\textwidth]{ErrorOrder200Outside}
  % \caption{a) Maximum and b) mean relative error of the  sound pressure at a frequency of 200\,Hz at the collocation nodes and at the evaluation grid outside the sphere as a function of the square root of the number of elements.}\label{Fig:Error200}
  \caption{Relative errors as a function of the number of elements $N$ for the sphere benchmark at $f = 200$\,Hz. Left side: Error on the surface of the sphere. Right side: Error around the sphere. The continuous (blue) line depicts the mean error over all nodes, the gray area depicts the area between the maximum and the minumum  error (dashed lines). The dotted line depicts the function $\frac{1}{\sqrt{N}}$. }\label{Fig:Error200} 
\end{figure}
%
\begin{figure}[!h]
%  \includegraphics[width=0.48\textwidth]{MaxError1000new}
%  \includegraphics[width=0.48\textwidth]{MeanError1000new}
  \includegraphics[width=0.48\textwidth]{ErrorOrder1000Surface}
  \includegraphics[width=0.48\textwidth]{ErrorOrder1000Outside}
  \caption{Relative errors as a function of the number of elements $N$ for the sphere benchmark at $f = 1000$\,Hz. Left side: Error on the surface of the sphere. Right side: Error around the sphere. The continuous (blue) line depicts the mean error over all nodes, the gray area depicts the area between the maximum and the minimal  error (dashed lines). The dotted line depicts the function $\frac{1}{\sqrt{N}}$.}\label{Fig:Error1000}
\end{figure}
% \mybox{There could be also the reason that the jumps are also caused by adding another level in the MLFMM}

%\mybox{insert condition number plot ???}
\begin{itemize}
\item For the integral over the boundary elements, the maximum order of the Gauss quadrature is six per default. For regular integrals, this is a good compromise between accuracy and efficiency. However, sometimes this maximum order can be too small for quasi-singular integrals, even after the subdivision procedure having applied (see Sec.~\ref{Sec:Quadrature}). %One option to deal with this problem, is to subdivide the element if the maximum Gauss order is reached. 
\item The factor determining the multipole expansion length is 1.8 per default (see Sec.~\ref{Sec:Expansion}). While, in the most cases, this default is sufficient for an accurate but fast calculation, for a better accuracy, it needs to be increased at for the price of a longer calculation.
  % a factor of 12 is about single precision
\end{itemize}

In order to demonstrate the effect of the maximum order and the factor determining the multipole expansion length, we repeated the calculations, however, with two modifications of the parameters. First, the quasi-singular quadrature was modified such that each element was additionally subdivided, if the estimated maximum order was larger than 6. Second, the factor determining the multipole expansion length was set to 15 (see Eq.~(\ref{Equ:Truncation}). This resulted in slightly higher multipole lengths $L$. For example, for a mesh with $N = 74892$ elements, the truncation $L$ calculated for the multipole levels of $\ell = 1, 2$, and $3$  was $L_{1,2,3} = 12, 10, 9$ (in opposite to $L_{1,2,3} = 8, 8, 8$ obtained with the default setting), respectively. Fig.~\ref{Fig:Error200rw15} shows the relative errors obtained with the modified parameter settings. As expected, the errors decreased showing a better accuracy of the calculations. Note that the additional accuracy was paid with the longer computation time. For example, for a mesh with $N = 34992$ elements, the time to set up the linear system increased by factor \emph{two to three} as compared to that obtained with default parameters.
%
%Nevertheless, in all cases, the computation times on a Core i5-4430 computer was still a 2 minutes.
%   \begin{figure}[!h]
%     \includegraphics[width=0.48\textwidth]{MaxError200AcouBEM}
%     \includegraphics[width=0.48\textwidth]{MeanError200AcouBEM}
%     \caption{Mean and maximum relative error at 200Hz with enhanced quasi-singular quadrature}\label{Fig:Error200QuasiSingular}
%   \end{figure}
%
\begin{figure}[!h]
  \includegraphics[width=0.48\textwidth]{ErrorAcouBEM200Surface}
  \includegraphics[width=0.48\textwidth]{ErrorAcouBEM200Outside}
  \caption{Relative errors with a higher accuracy setting as a function of the number of elements $N$ for the sphere benchmark at $f = 200$\,Hz. Left side: Error on the surface of the sphere. Right side: Error around the sphere. The continuous (blue) line depicts the mean error over all nodes, the gray area depicts the area between the maximum and the minimal error (dashed lines). The dotted line depicts the function $\frac{1}{\sqrt{N}}$.}\label{Fig:Error200rw15}
\end{figure}

This observation has several other implications. First, it is possible to achieve a higher accuracy in the calculations. This is however linked with the price of longer calculations, even though the calculation time might be still reasonable. In our example, the calculation time was still within 5 minutes even for the mesh with $N = 74892$ elements. Further, users should ask themselves if the higher accuracy is actually required because the gain can be small (compare Fig.~\ref{Fig:Error200rw15} vs Fig.~\ref{Fig:Error200}). Finally, if optimizing for both speed and accuracy, an promising approach can be replacing the fine mesh with a coarser mesh at low frequencies. %This example should illustrate the importance to know a bit about the the software one uses, to be able to a) ask these questions and b) to answer them. 
%
%   The difference between the behavior of the error at 200 and 1000\;Hz also illustrates the fact that the stability of the BEM suffers when fine meshes are used in combination with low frequencies, which is also illustrated by the condition number illustrated in Fig.~\ref{Fig:CondNr}. As the number of elements of the meshes used in this figure was small enough, it was still possible to use LAPACK \cite{Andersonetal99} and the estimator of the condition number included in this package.
%   \mybox{einfuegen direct solver can be used}
%   \mybox{Unfortunately the error at 200 hz stays that high in NumCalc, so it is not just the C that is changed when adding additional multipole layers. Should we mention that?}
%   %
%
%
% \begin{figure}[!h]
%   \begin{center}
%     \includegraphics[width=0.8\textwidth]{Comparecond}
%     \caption{Condition number of the system as function of frequency for meshes with 4332 and 8112 elements respectively.}\label{Fig:CondNr}
%   \end{center}
% \end{figure}
% \begin{figure}[!h]
%   \includegraphics[width=0.44\textwidth]{Maxerror100Hzrw15}
%   \includegraphics[width=0.44\textwidth]{Meanerror100Hzrw15}
%   \caption{Mean and max. relative error at 100Hz with rwfact = 15}
% \end{figure}
\subsubsection{Effect of the FMM}\label{Sec:NonZero}
In this section, we analyzed the effect of FMM on the accuracy. To this end, we calculated the relative difference
$$
\frac{|p_{\text{FMM}} - p_{\text{w/o FMM}}|}{|p_{\text{w/o FMM}}|}
$$
between the BEM calculations performed with and without the FMM. In the latter condition, we used a direct solver, i.e., \texttt{zgetrf} routine from LAPACK~\cite{Andersonetal99} and limited $N$ to 8748 elements in order to keep the matrices at feasible sizes. The calculations were done for the frequencies 200 and 1000\,Hz.
\begin{figure}[!h]
  \centering
%  \includegraphics[width=0.46\textwidth]{MeanErrorSphereFMM200}
%  \includegraphics[width=0.46\textwidth]{MeanErrorSphereFMM1000}
  \includegraphics[width=0.44\textwidth]{Errors200FMMSurface}
  \includegraphics[width=0.44\textwidth]{Errors1000FMMSurface}
  \caption{Relative error between BEM and analytic solution (lighter colors), and the relative distance of the BEM calculations with and without FMM at 200\,Hz (a) and 1000\,Hz (b) as function of the number of elements $N$. The dashed lines depicts the maximum and minimal relative distances between conventional BEM and FMM solution, the continuous (blue) lines the mean distance between both solutions.}\label{Fig:FMMvsDirect}
\end{figure}
Fig.~\ref{Fig:FMMvsDirect} shows the relative errors obtained in the two conditions, calculated as a function of $N$ for the two frequencies. This figure shows also the overall errors as compared to the analytic solution. The relative errors introduced by the FMM were in the range of $10^{-3}$, thus showing only a minor effect on the overall error. Note that at 1000\,Hz (Fig.~\ref{Fig:FMMvsDirect}b), the wavelength is about $0.34$\,m, which means, that the 6-to-8 elements per wavelength is only fulfilled for meshes with more than $6000$ elements. This explains the large overall errors in Fig.~\ref{Fig:FMMvsDirect}b.
%
%
\subsubsection{Memory requirement: Moderate mesh size}
In this section, we analyzed the memory requirements as en effect of the clustering. To this end, we compared the estimated and actual number of non-zero entries in the FMM matrices (see Sec.~\ref{Sec:FMM-Matrices}). We used the cube-based meshes and considered calculations for the frequency of $200$\,Hz for which we investigated the number of non-zero elements in the SLFMM and the MLFMM. We considered a unit sphere represented by a moderate size of $N = 4332$ elements, for which the minimum, average, and maximum edge lengths was approximately 0.05\;m, 0.09\;m, and 0.15\;m, respectively. 

For the SLFMM, the clustering resulted in $N_C = 90$ clusters, with the number of elements in a single cluster varying between 29 and 72 elements. Fig.~\ref{Fig:Clustering1} shows the mesh and visualizes the obtained clusters. The average number of elements per cluster was $48$ elements, which is significantly different than the the aimed estimation of $\sqrt N \approx 65$. This indicates that the actual clustering yields a different size even for a regular geometry such as the sphere. This difference has an impact on the estimation of the non-zero entries of the FMM matrices and thus the accuracy of the esimation for the memory requirement.%Users should be aware of this fact and that there is a subtle difference between size of the bounding box and the actual size of the cluster inside the box. 
%
\begin{figure}[!h]
  \begin{center}
    \includegraphics[width=0.45\textwidth]{cluster4331a.jpg}
    \caption{Example of default clustering for the unit sphere at 200\,Hz for a mesh with $N = 4332$ elements. The different colors/gray scales depict different clusters.}\label{Fig:Clustering1}
  \end{center}
\end{figure}
%


For the SLFMM, Tab.~\ref{Tab:EntriesSLFMM_I} shows the information about the mesh and the clustering, as well the estimated and the actual numbers of non-zero entries of the FMM matrices. For the midpoint-to-midpoint matrix $\bD$, two types of the estimations are shown: that based on the estimated number of clusters ($\sqrt{N}$) and that based on the actual number of clusters ($N_C$). 
%As there was a difference between the estimated number of clusters and their actual number also the estimates for the non-zero entries were slightly off.
%When looking for example at the entries in  Tab.~\ref{Tab:EntriesSLFMM_I},
It can be seen that the estimated non-zero entries of the midpoint-to-midpoint matrix $\bD$ are more accurate when using the actual number of clusters. Also the estimate gets better, if on average 10 instead of 9 neighbouring clusters (including the cluster itself) are assumed, but this number depends on the individual mesh. Note that in these calculations, the truncation parameter of the SLFMM expansion length (see Eq.~\ref{Equ:FMMTruncation}) was $L = 8$. While at a frequency of $200$\,Hz, Eq.~(\ref{Equ:FMMTruncation}) would result in $L = 4$, $L$ was increased to $L = 8$ because numerical experience showed that $L = 8$ is a good compromise between numerical effort, accuracy, and stability in the low frequency region.



%
%In Table \ref{Tab:EntriesSLFMM_I} the non-zero entries of the different matrices are presented and compared with their respective estimate of non-zero entries. In general, the actual number of non-zeros correspond well with the estimations, except for the matrix $\bD$, in that case, the actual number of clusters used yields a far better correspondence.
%
\begin{table}[!h]
  \begin{center}
    \begin{tabular}{lr}
      \multicolumn{2}{c}{Cluster Info}\\
      \hline
      Nr. Boundary Elems & $N = 4332$\\
      Nr. Clusters  & $N_C = 90$\\
      Truncation & $L = 8$\\
                   &\\
    \end{tabular}\hspace{1cm}
    %
      $\begin{array}{crr}
     \text{Matrix} & \text{Actual} & \multicolumn{1}{c}{\text{Estimate}}\\
    \hline
    \bN & \num{2 700 492} & 10(\sqrt{N})^3 \approx \num{2 851 323}\\
    \bT,\bS & \num{554496} & 2L^2N = \num{554496}\\
    \bD & \num{907520} &  2L^2 \sqrt{N}(\sqrt{N} - 10) \approx \num{470249}\\
                  &        &  2L^2 N_C(N_C-10) = \num{921600}\\
   \end{array}$
 \end{center}
 \caption{Basic information about the mesh and the clustering for the SLFMM example and the estimated and actual number of non-zero entries of the FMM matrices.}\label{Tab:EntriesSLFMM_I}
 % RAM estim 0.0945 G, /usr/bin/time 0.100848 G, time 0.116 for 24884 nodes
\end{table}
% data for 1000Hz
% MLFMM
% n0 = 56, n1 = 252
%   l0 = 17
%   l1 = 10
%    
%   N 931200
%   T0 = 2 503 896
%   T1 = 886 400
%   D0 1 479 680
%   D1 1 865 600
%   0.22G
%    
%   Formulas
%   1083000 (10 neighb) vs 974700 (9 neighb)
%   T0 2503896
%   T1 = 886 400
%   D0 = Number Clusters * (Number Clusters - 9) * 2 L^2 = 1 488928
%   D1 = Number Clusters * 2 L^2 * NumberNeigh * NumberChild =
%        252 * 200 * 9 * 4 = 1814400 vs 2 016 000 (10 neighb)
%  SLFMM
%   
%  l0 = 16
%  Nclust = 90
%   
%  N = 2 700 492  Estim 2 566 109 vs 2 851 232
%  T = 2 217 984
%  D =  3 630 080 Estim 90 * 80 * 16^2 * 2 = 3 685 400
%  0.216G

%\subsubsection{MLFMM: Example 1}
For the MLFMM, the clustering yielded $\ell_{\max{}} = 2$ levels. %the number of clusters at root level was  ${n}_{1,\text{est}} = 0.9\sqrt{N} \approx 59$. .
At the root level, we obtained $N^{[1]}_C = 56$ clusters, with the smallest, average, and maximum number of elements in an individual cluster being 30, 73, and 104 elements, respectively. The estimated number of clusters at the root level was $N^{[1]}_{C,\text{est}} = 0.9\sqrt{N} \approx 59$. At the leaf level, we obtained $N^{[2]}_C = 252$ clusters, with the minimum, average, and maximum number of elements in a cluster being 2, 17, 30 elements, respectively. The low number of elements demonstrates the effect of subdividing all parents on each level. % thus, dynamic clustering will be an important issue in future versions of \numcalc{}.
On both levels, the expansion length was  $L_1 = L_2 = 8$. Tab.~\ref{Tab:EntriesMLFMM_I} shows the information about the clustering and the number of actual and estimated non-zero entries. For the estimation, it was assumed that each cluster has 10 clusters in its nearfield and that each parent has 4 children. For the cluster-to-cluster matrices $\bD_1$ and $\bD_2$, estimations based on both the estimated number of clusters ($\sqrt{N}$) and the actual number of clusters ($N_C$) are shown. The results show that even the estimates based on the number of elements $N$ are well approximating the actual number of non-zero elements. % in this case it is always better to use the actual number of clusters on each level.

%For an estimate of the non-zeros entries of $\bN$ we multiply the square of the number of elements of each leaf cluster, which should be about $25^2$, with the number of neighbouring clusters, which again is assumed to be 9, times the number of clusters on the leaf level: $\text{Entries}(\bN) = 25^2 \cdot 9 \cdot \text{Nclusters}_1$. The non-zero entries of $\bT$ and $\bS$ are again given by $2L^2 \cdot N$, the entries of the midpoint-to-midpoint matrices are estimated by $\text{Entries}(\bD_0) = 2L^2 \cdot \sqrt{N} (\sqrt{N} - 9)$, where $\sqrt{N}$ again is assumed to be the number of clusters on the root level 0 as well as the number of elements in each cluster. For the matrix $D_1$ on the leaf level we assume that each cluster again has 9 neighbours and each parent cluster is split into about 4 children. Thus, $\text{Entries}(\bD_1)  = 2L^2 \cdot \text{Nclusters}_1 \cdot 9 \cdot 4$. 
%x
%T = 2 L^2 N
\begin{table}[!h]
  \begin{minipage}{0.4\textwidth}
    \begin{center}
      \begin{tabular}{ll}
        \multicolumn{2}{c}{Cluster Info}\\
        \hline
        Nr. Boundary Elems & $N = 4332$\\
        Nr. Clusters  & $N^{[1]}_C = 56$\\
                     & $N^{[2]}_C = 252$\\
        Truncation   & $L_1  = 8$\\
                     & $L_2 = 8$\\
                     &
      \end{tabular}
    \end{center}
  \end{minipage}
  \begin{minipage}{0.6\textwidth}
    \begin{center}
      % 
      $\begin{array}{crr}
         
         \text{Matrix} & \text{Actual} & \multicolumn{1}{c}{\text{Estimate}}\\
         \hline
         \bN & \num{931 200} & 10\cdot 25^2 \cdot N / 25 = \num{1 083 000}\\
         % &              & 9 \cdot 19^2 522 \approx \num{907200}\\
         \bT_\ell,\bS_\ell & \num{554496} & 2L^2N = \num{554496}\\
         \bD_1 & \num{327 680} &  2L^2 N^{[1]}_{C,\text{est}}\left(N^{[1]}_{C,\text{est}}-10\right) \approx \num{370 048}\\
                       && 2L^2 N^{[1]}_C (N^{[1]}_C - 10) = \num{329728}\\
         \bD_2 & \num{1 193 984} & 2L^2 \cdot N/25 \cdot 10 \cdot 4 \approx \num{887 194}\\
                       &&                  2L^2\cdot N^{[2]}_C\cdot 10\cdot 4 = \num{1 290 240}
       \end{array}$
     \end{center}
   \end{minipage}
   \caption{Basic information about the mesh and the clustering and estimated and actual number of non-zero entries of the MLFMM matrices.}\label{Tab:EntriesMLFMM_I} %Memory consumption estimated 0.09385 G /usr/sbin/time 0.102 G for 70, time 0.116 for 24884}
 \end{table}

 % \subsubsection{SLFMM: Example 2}
\subsubsection{Memory requirement: Large mesh size}
In this section, we repeated the analysis, however, using a mesh with more elements, i.e., the sphere was represented by $N = 8112$ elements. With such a discretization, the minimum, average, and maximum edge lengths of all boundary elements was about 0.06, 0.04, and 0.118\,m, respectively. 

For the SLFMM, Tab.~\ref{Tab:EntriesSLFMM_II} shows the clustering information and the estimations. The clustering resulted in $N_C = 90$ clusters, with the number of elements in individual clusters ranging from $52$ to $144$ elements. Note that the estimation for the number of clusters $N_C = \sqrt{N}$ is quite accurate. When compared to the previous mesh with only half of the elements, it might appear surprising that despite the higher number of elements, the number of clusters did not increase. Note, however, that for the SLFMM, the number of clusters is per default determined on $\sqrt{N}$ bounding boxes, thus the difference in the number of clusters between the two sizes of the meshes is quite small. And one has to distinguish between the bounding box and the cluster inside this box. The estimation of the non-zero entries of $\bN$ was slightly smaller than the actual number. When assuming that the nearfield for each cluster contains on average 12 clusters (including itself), the estimation was more close to the actual number. 
\begin{table}[!h]
  \begin{center}
    \begin{tabular}{ll}
      \multicolumn{2}{c}{Cluster Info}\\
      \hline
      Nr. Boundary Elems & $N = 8112$\\
      Nr. Clusters  & $N_C = 90$\\
      Truncation & $L = 8$\\
                   &\\
    \end{tabular}\hspace{1cm}
  $\begin{array}{crr}
      \text{Matrix} & \text{Actual} & \multicolumn{1}{c}{\text{Estimate}}\\
     \hline
    \bN & \num{8 678 432} & 10(\sqrt{N})^3 \approx \num{7 306 206}\\
                   &         & 12(\sqrt{N})^3 \approx \num{8767447}\\
      \bT,\bS & \num{1 038 336} &  2L^2  N = \num{1038336}\\
      \bD & \num{911 616} &  2L^2 \sqrt{N}(\sqrt{N} - 10) \approx \num{923050}\\
   \end{array}$
 \end{center}
   \caption{Basic information about the clustering and the comparison between estimated and actual number of non-zero entries of the SLFMM matrices.}\label{Tab:EntriesSLFMM_II} %estim 0.2336, time 0.2548 for 24884 eval nodes, time 0.2402 for 70 eval elems }\label{Tab:EntriesSLFMM_II}
 \end{table}

 
%\mybox{Interessieren uns eigentlich auch die icosphere, bzw. die daten zum mesh? bei beiden meshes? Maximum edge length of boundary elements = 0.082604, Average edge length of boundary elements = 0.0754991, Minimum edge length of boundary elements = 0.0691829. KP: Daten zu Meshes wÀre gut dazuzugeben, damit unsere Claims reproduzierbar sind. Vorschlag: Demofiles auf Sourceforge laden}

%   \subsubsection{SLFMM: Example 3}
%    Icosphers mit 5120 elementen, L = 8,
%    
%    
%   Informations about clusters:
%      Level    Num. of clusters    Maximum radius   Minimum radius
%      0           90             0.335365(80)      0.177335(19)    80  19
%    
%   Nonzeros of the near field matrix ..................... = 3270468
%   Nonzeros of the T-matrix .............................. = 655360
%   Nonzeros of the D-matrix .............................. = 923904
%   Nonzeros of the S-matrix .............................. = 655360
%   Sum of nonzeros of the FMBEM matrices ................. = 5505092   
 
%   
%   
%  $$
%  \begin{array}{ccc}
%    \bN & 8678432  & \approx 10 \cdot 25^2 \cdot 360\\
%    \bT & 554496 & = 128  N\\
%    \bD & 913664 & < 40 \cdot 360 \\\
%  \end{array}
%  $$


%\subsubsection{MLFMM: Example 2}
%In this example, $N = 8112$ elements are used to discretize the sphere, the rest of the setting is like in the previous example.
For the MLFMM, Tab.~\ref{Tab:EntriesMLFMM_II} shows the clustering results and the estimations. At the root level, the clustering yielded $\ell_{\max} = 2$ levels, with $N^{[1]}_C = 90$ clusters, with the number of elements inside the individual clusters ranging from 52 to 144 elements and, on average, each cluster containing 90 elements. On the leaf level, $N^{[2]}_C = 360$ clusters were created, with the number of elements in the individual clusters being between 4 and 54 elements and the average number of elements per cluster being 23 elements. %The  which supports the assumption of about 4 children for each parent cluster.
At both levels, the expansion length of the MLFMM expansion was $L_1 = L_2 = 8$. The estimates for the nearfield matrix was slightly too small, but in overall, the estimates matched well the actual numbers, if the actual number of clusters was considered.

%\mybox{10 neighbours would be better???? also change the N entries to N/25 * 25 * 25 * 9 or 10}

\begin{table}[!h]
  \begin{minipage}{0.4\textwidth}
    \centering
    \begin{tabular}{ll}
      \multicolumn{2}{c}{Cluster Info}\\
      \hline
      Nr. Boundary Elems & $N = 8112$\\
      Nr. Clusters & $N^{[1]}_C = 90$\\
                   & $N^{[2]}_C = 360$\\
      Truncation & $L_1 = L_2 = 8$\\
                        &\\
                        &  % so we are even with the other column
    \end{tabular}
  \end{minipage}
  \begin{minipage}{0.6\textwidth}
    \centering
    $\begin{array}{crr}
       \text{Matrix} & \text{Actual} & \multicolumn{1}{c}{\text{Estimate}}\\
       \hline
       % \bN & \num{2 403 648} & 10 \cdot 25^2 \cdot 360 = \num{2 250 000}\\
       \bN & \num{2 403 648} & 10 \cdot 25^2 \cdot  N/25 = \num{2028000}\\
       %% 11 * 25 * N = 2 230 800
       %% Nclust * 25^2 * 9 = 2025000
       \bT_\ell, \bS_\ell &  \num{1 038 336} & 2L^2 N = \num{1 038 336}\\
       \bD_1 & \num{911 616} & 2L^2 \cdot N^{[1]}_{C,\text{est}} (N^{[1]}_{C,\text{est}} - 10) \approx \num{736 128}\\
                     &                             & 2L^2 \cdot N^{[1]}_C \cdot (N^{[1]}_C - 10) = \num{921600}\\
                     % && 128 \cdot 90 \cdot 80 = 921 600\\
       \bD_2 & \num{1 747 968} & 2L^2 \cdot N/25 \cdot 10 \cdot 4 \approx \num{1 661 337} \\
                     && 2L^2 \cdot N^{[2]}_C \cdot 10 \cdot 4  = \num{1 843 200}\\
                     % && 128 \cdot 360 \cdot 40 = \num{1 843 200}\\
     \end{array}$
   \end{minipage}
  \caption{Basic information about the clustering and the comparison between estimated and actual number of non-zero entries of the MLFMM matrices.}\label{Tab:EntriesMLFMM_II} %estim 0.185024 /usr/sbin 0.195 for 70 nodes, time = 0.210 for 24884}\label{Tab:EntriesMLFMM_II}
\end{table}
 
%For the next example, the default clustering parameters are replaced by setting the length of a root sub-box to 0.7\;m.
\subsubsection{Memory requirement: Bounding box length}
In this section, we investigated the effect of changing the initial length of the bounding box. To this end, the sphere was represented by $N = 8112$ elements as in the previous example, but the the initial length of the bounding box at the root level was set to $0.7$\,m. For the estimation of the non-zero entries of the FMM matrices the actual number of clusters generated by \numcalc{} in the calculation were used.. The calculations were done with the SLFMM only, for the cube-based meshes and the frequency of $200$\,Hz. 
%\mybox{Actual radii of clusters einbauen?, vieleicht auch oben? brauchen wir einheiten, wie m und so dinge, 0.7m fÌr 26 klingt komisch, is' aba so. Denken wir an die orginal Bounding box mit 2m KantenlÀnge, dann ergibt 0.7 drei unterteiungen pro dimension. Jede Seite des WÌrfels wird in 9 quadrate unterteilt, da sind wir mit 26 cluster sowieso noch gut bedient, sollen wir das noch extra anfÌhren? KP: YES!}

Tab.~\ref{Tab:EntriesMLFMM_III} shows the clustering results and the estimations. The clustering generated $\ell = 3$ cluster levels. The root level consisted of $N^{[1]}_C = 26$ clusters, with 312 elements per cluster on average, ranging between 184 and 370 elements per cluster. On the second level, $N^{[2]}_C = 144$ clusters were generated, with 56 elements per cluster on average, ranging between 28 and 77 elements per cluster. On the leaf level, $N^{[3]}_C = 536$ clusters were generated, with 15 elements per cluster on average, ranging between 1 and 32 elements per cluster. On each level, the expansion length was $L_1 = L_2 = L_3 = 8$.

This example illustrates that setting the bounding box length for the root level affects the clustering to a large degree. It affected the number of levels, the number of elements per cluster, and the radii of each cluster. By adjusting the initial length of the bounding boxes at the root level, the radii of each cluster can be controlled, and thus the length of the multipole expansion. This helps in controlling the number of non-zero entries in the matrices $\bT_\ell$ and $\bS_\ell$ to some extent.

%Again, we see the drawback of the non-dynamic clustering where the bounding boxes are subdivided in each level. In general, the clusters on the leaf level will contain a different number of elements, and although small clusters are merged with bigger ones, they can contain very few elements. This also affects the estimators  of the non-zero entries of the different matrices, see Tab.~\ref{Tab:EntriesMLFMM_III}. Also, the estimation on the number of non-zero entries is better if it is assumed that the nearfield of each cluster consists of only 9 elements.


%  144/26 = 5.5
%  664/144 = 3.7
%  radii
%  0.52 0.44
%  0.29 0.12
%  0.16 0.0399
\begin{table}[!h]
  \begin{center}
  \begin{tabular}{lr}
    \multicolumn{2}{c}{Cluster Info}\\
    \hline
    Nr. Boundary Elems & $N = 8112$\\
    Nr. Clusters & $N^{[1]}_C = 26$\\
                      & $N^{[2]}_C = 144$\\
                      & $N^{[3]}_C = 537$\\
    Truncation & $L_1 = L_2 = L_3 = 8$
  \end{tabular}
  \hspace{1cm}
    %
    $\begin{array}{crr}
    \text{Matrix} & \text{Actual} & \multicolumn{1}{c}{\text{Estimate}}\\
    \hline
       \bN & \num{1 525 972} & (N/n_3)^2 \cdot 10 \cdot N^{[3]}_C \approx   \num{1 225 411}\\
    % \bN & \num{1 525 972} &  10\cdot 25\cdot 8112 =  \num{2 028 000}\\
%                  && 536\cdot15^2\cdot 9 = \num{1085400}\\
    \bT_\ell, \bS_\ell & \num{1 038 336} & 2L^2 \cdot N = \num{1 038 336}\\
    \bD_1 & \num{58624} & 2L^2 \cdot N^{[1]}_C \cdot (N^{[1]}_C - 10)  = \num{53248}\\
    \bD_2 & \num{641 024} & 2L^2 \cdot N^{[2]}_C \cdot 4 \cdot 9 = \num{663552}\\
    \bD_3 & \num{2 334 864} &  2L^2 \cdot N^{[3]}_C \cdot 4 \cdot 9 = \num{2 469 888}\\
   \end{array}$
   \end{center}
   \caption{Basic information about the clustering and the comparison between estimated and actual number of non-zero entries of the MLFMM matrices. The initial sub-box length was set to 0.7\;m, additionally it was assumed that each for each cluster the nearfield consists of 9 clusters.}\label{Tab:EntriesMLFMM_III}
\end{table}

%%%%%%%%%%%%%%%%%%%%%%%%%%%%%%%%%%%%%%%%%%%%%%%%%%%%%%
%
%  icosphere kaum neue erkenntnisse
%
%%%%%%%%%%%%%%%%%%%%%%%%%%%%%%%%%%%%%%%%%%%%%%%%%%%%%%%
%

\subsection{MLFMM: Head and Torso}\label{Sec:MLFMM4}
This example was chosen to illustrate some of the restrictions on the current implementation of the FMM in \numcalc. In this example, we look at a model of head and torso of a human consisting of $\num{174 298}$ triangular elements. This mesh is used to calculate the reflection of sound waves at the human head, torso, and especially at the pinna. A point source was placed near the entrance of the pinna. %Because of the high frequency of interest $f \approx 16000$\,Hz, the mesh needs to be very fine.
There is also a large difference in the size of the elements between torso (relatively coarse discretization) and the human pinnae, where the discretization needs to be fine in order to correctly approximate the geometry and the solution. In this case the minimum, average, and maximum edge length of all boundary elements were $0.7, 3.3,$ and $6.7$\,mm, respectively. The default clustering results in 3 levels with $N^{[1]}_C, N^{[2]}_C, N^{[3]}_C = 422, 1558, 6378$ clusters per level, the expansion length on each level is given by $L_1, L_2, L_3  = 31, 19$, and $11$, respectively. 

For this example, the amount of memory additionally used for storing the matrices $\bT_\ell$ and $\bS_\ell$ on each level becomes very large. Although, in theory, the additional memory requirements are only $O(N)$, the length of the expansion and thus the factor $2L_\ell^2$ becomes relatively large. For this example the memory needed by the  FMM matrices alone is about 30.1\,GByte. The non-zero entries of $\bT_\ell$ and $\bS_\ell\; (\ell = 1,2)$ on levels $\ell = 1$ and $\ell = 2$ are about 61\% of all non-zero entries of the FMM matrices, see Tab.~\ref{Tab:EntriesMLFMM_V}. The additional memory needed to explicitly store these matrices could be avoided, e.g, by filtering/interpolation algorithms for the FMM matrices. In this case, the gain %\footnote{We are currently working on interpolation and filtering procedures for next releases and have an alpha version available.}
in computing time by avoiding filtering/interpolation between levels does not outweigh the additional memory consumption. Of course, compared to the $O(N^2)$ memory needs of 486~GByte for the traditional BEM, the FMM is still very effective.

%The assumption of 10 clusters in the near field of each clusters also does not give a good estimate of the memory consumption, at least on the root level. This can be explained by the mesh, especially at the ear, being very fine to catch its geometry and not as smooth. At least at root level, in this case it seems that 18 is a better estimate for the number of near-field clusters. 
%mybox{reformulate}

%The need for such large examples has triggered the implementation of methods like interpolation and filtering of the multipole matrices $\bT$ and $\bS$ between different levels which will be contained in future releases.  There are several options that will be addressed in future versions of \numcalc{}, e.g., interpolation between the local expansions or the fact, that the matrices $\bT$ can be calculated analytically. On as side note: Without the FMM, storing the full system matrix would need about 476G of memory in this example.

\begin{table}[!h]
  \begin{center}
    \begin{tabular}{lr}
      \multicolumn{2}{c}{Cluster Info}\\
      \hline
      Nr. Boundary Elems & $N = \num{170 914}$\\
      Nr. Clusters & $N^{[1]}_C = 403$\\
                & $N^{[2]}_C = 1512$\\
                & $N^{[3]}_C = 6226$\\
      Truncation & $L_1 = 37$\\
                & $L_2 = 20$\\
                & $L_3 = 16$
    \end{tabular}
    \hspace{1cm}
    %
    $\begin{array}{cr}
      \text{Matrix} & \text{Entries}\\
      \hline
      \bN &  \num{74412824}\\
      \bT_1 = \bS_1 &  \num{335000756}\\
      \bT_2 = \bS_2 & \num{125843156}\\
      \bT_3 = \bS_3 & \num{42180116}\\
      \bD_1 & \num{333601540}\\
      \bD_2 & \num{36733916}\\
      \bD_3 & \num{51291416}
     \end{array}$
   \end{center}
   \caption{Basic information about the clustering and number of non-zero entries of the MLFMM matrices for the head-and-torso model.}\label{Tab:EntriesMLFMM_V}
\end{table}


%  Nonzeros of the near field matrix ..................... = 73355520   1.17G
%  Nonzeros of the T-matrix .............................. = 692201700  
%      Nonzeros of T_lev_0 ............................... = 467962532  7.5G
%      Nonzeros of T_lev_1 ............................... = 136731200  2.2G
%      Nonzeros of T_lev_2 ............................... = 87507968   1.4G
%  Nonzeros of the D-matrices ............................ = 581593820
%      Nonzeros of D_lev_0 ............................... = 432729948  6.9G 
%      Nonzeros of D_lev_1 ............................... = 41084800   0.65G
%      Nonzeros of D_lev_2 ............................... = 107779072  1.7G
%  Nonzeros of the S-matrix .............................. = 692201700
%      Nonzeros of S_lev_0 ............................... = 467962532
%      Nonzeros of S_lev_1 ............................... = 136731200
%      Nonzeros of S_lev_2 ............................... = 87507968
%      Sum of nonzeros of the FMBEM matrices ................. = 2039352740

%  Statistic for the job (in second):
%  Input ............................ 25
%  Address computation .............. 0
%  Assembling the equation system ... 529
%  Solving the equation system ...... 1382
%  Post processing .................. 1644
%  The whole job .................... 3580

%% $\begin{array}{crcr}
%      \text{Matrix} & \text{Entries} & \multicolumn{1}{c}{\text{Estimate}} & \multicolumn{1}{c}{\text{Memory}}\\
%      \hline
%       \bN & \num{73355520} &  16 \cdot \left(N/n_2\right)^2 n_2 \approx \num{75 069 953} & 1.2\text{G}\\
%       \bT_0, \bS_0 & \num{467962532} & 2\cdot L_0^2 \cdot N = \num{467962532} & 7.5\text{G}\\
%       \bT_1, \bS_1 & \num{136731200} & 2\cdot L_1^2 \cdot N = \num{136731200} & 2.2\text{G}\\
%       \bT_2, \bS_2 & \num{87507968} & 2\cdot L_2^2 \cdot N = \num{87507968} & 1.4\text{G}\\
%       \bD_0 & \num{432729948} & n_0\cdot(n_0-16)\cdot2\cdot L_0^2  = \num{427 021 218} & 6.9\text{G}\\
%       \bD_1 & \num{41084800} &  n_1\cdot 16\cdot 4 \cdot L_1^2 = \num{48384000} & 0.65\text{G}\\
%       \bD_2 & \num{107779072} & n_2\cdot 16 \cdot 4\cdot L_2^2 = \num{127508480} & 1.7\text{G}\\

\subsection{Duct problem}
This problem is one of the benchmarks described in \cite{Hornikxetal15}. It describes the problem of calculating the acoustic field inside a closed sound-hard duct of length $L = 3.4$\,m that has a quadratic cross section with an edge length of $w = 0.2$\,m. At the closed side at $x = 0$, a velocity boundary condition with $v = 1$\,m/s is assumed, at the closed side at $x = L$ an admittance boundary condition with $Y = \frac{1}{\rho c}$ is assumed, where $\rho = 1.3$\,kg/m$^3$ and $c = 340$\,m/s are the density and the speed of sound in the medium, respectively. The quadratic cross section has the advantage that the geometry of the duct can be modelled accurately. A mesh for this problem including the positions of evaluation nodes inside the duct is partly depicted in Fig.~\ref{Fig:InsideDuct}a. The solution for these boundary conditions is given by a plane wave inside the duct traveling from one end to the other
\begin{equation}\label{Equ:PlaneWaveSol}
p(x) = - \rho c e^{\I kx},
\end{equation}
for the given speed of sound and the given density the magnitude of the solution Eq.~(\ref{Equ:PlaneWaveSol}) is $|\rho c| = 442$.

This problem has been investigated in several publications \cite{Marburg18, Kreuzer22} and it has been shown that calculations with collocation BEM result in a decaying field along the duct, where the rate of decay depends on the frequency and the element size along the duct.
\begin{figure}[!h]
  \begin{minipage}{0.44\textwidth}
    \centering
    \includegraphics[width=0.8\textwidth]{ductsetup}
  \end{minipage}
%     
%
  \begin{minipage}{0.5\textwidth}
    \centering
    \includegraphics[width=0.98\textwidth]{insidefieldadmtube}
    %\caption{Absolut value of the pressure along a line inside the duct.}
  \end{minipage}
   \caption{Discretization of the duct and the evaluation points inside the duct. Only parts of the duct surface are displayed. b)Magnitude in Pa of the pressure along a line inside the duct for different element sizes $h$ at 240\,Hz and at 480\,Hz.}\label{Fig:InsideDuct}
\end{figure}
%
As an example, the absolute value of the BEM solutions inside the duct for two different frequencies $f_1 = 240$\,Hz and $f_2 = 480$\,Hz, and two different discretizations with edge lengths $h = 0.025\,$m and $h = 0.05$\,m are presented in Fig.~\ref{Fig:InsideDuct}b. At these frequencies, a smooth and robust solution is expected. It can be easily seen, that the difference between the plane wave solution and the numerical solution grows along the length of the duct. Overall, the errors between calculated and analytic solutions are smaller than $6\%$. Nevertheless, this benchmark is one of the few problems, where the 6-to-8-elements-per-wavelength rule may not be enough to provide results that are accurate enough.

To better illustrate this fact, the boundary conditions at both ends at $x = 0$ and $x = L$ are changed to sound-hard conditions. For such a duct, the theoretical resonance frequencies\footnote{When neglecting the diameter of the duct.} are at $f = 50 n$\,Hz, with $n \in \mathbb{N}$. In Fig.~\ref{Fig:Resonances} the specific frequency around $250$\,Hz where the acoustic field inside the duct has maximum energy is shown as a function of element size. It can be easily observed that the ``resonance'' frequency of the calculation is only close to the theoretical value if the edge length is about 0.1\,m, which is way shorter than suggested by the 6-to-8-elements-per-wavelength rule. At 250\,Hz, the wavelength is about $\lambda = \frac{340}{250}\approx 1.36$\,m, which would require an edge length of an element of appoximately $h = 0.17$\,m if 8 elements per wavelengths are assumed. In Fig.~\ref{Fig:Resonances}, the frequency with maximum energy for this element size is about 4 to 5\,Hz away from the ``correct'' frequency.

A side note: In Fig.~\ref{Fig:Resonances}a, the frequency step size for the computation was 1\,Hz. In the observed frequency range, the resonance is very sharp and in reality not at an integer frequency. Thus, the calculated acoustic field  inside the duct is bounded, although it may become quite large. The values of 249\,Hz are a bit misleading because of the frequency step size. If a finer frequency grid is used around 250\,Hz (see Fig.~\ref{Fig:Resonances}b) we see that the calculated ``resonances'' are between 249.9\,Hz and 250\,Hz.
%
\begin{figure}[!h]
  \begin{center}
    \includegraphics[width=0.46\textwidth]{DuctResonance}
    \includegraphics[width=0.46\textwidth]{Resonanceszoomed}
    \caption{Frequencies at which the acoustic field inside the duct has maximum energy as a function of the element length $\ell$ in the $x$-direction. The whole duct is assumed to be sound hard. The figure on the right side is a zoomed version of the figure on the left.}\label{Fig:Resonances}
  \end{center}
\end{figure}
%

One explanation for the fact that 8 elements per wavelength are not enough to determine the ``correct'' resonance frequency, may be given by the fact, that, in theory, the field inside the duct is given by the combination of two plane waves travelling in opposite directions along the duct and that the zeros of the resulting standing wave at the resonance frequencies have to be resolved quite accurately.

\section{Conclusion}\label{Sec:Discussion}
In this paper, the open-source BEM program \numcalc{}, which is part of the open-source software project \meshtohrtf, was described. The paper aims at a balance between giving researchers detailed information about certain aspects of the BEM implementation % of the BEM in \numcalc{} 
(Sections~\ref{Sec:Implementation} and \ref{Sec:Benchmark}) and providing users with a general background of the BEM and the methods used in \numcalc{} (Sections~\ref{Sec:BriefBEM}, \ref{Sec:Critical}, and \ref{Sec:Benchmark}). The motivation behind the paper is to give users enough insight to decide if \numcalc{} is a good tool for their problem at hand.

An important aspect of \numcalc{} that users need to consider is that \emph{\numcalc{} does not check the mesh before calculations}. In some cases, e.g., when elements contain double nodes or are twisted, there is a high chance that the program will stop with an error message. However, in most cases, even when the normal vectors point in the wrong direction or the mesh contains holes, \numcalc{} will produce results, and \emph{it is up to the user to check the validity of the results.} Throughout the paper, we provided general hints were added aiming at helping users to see if something goes wrong with the calculation. \emph{"Do not use and trust any code blindly, check your results if they are plausible".}
%The motivation for this paper is twofold:
%\begin{itemize}
%\item Researchers who just want to use the code will probably be more interested in the examples presented here, and maybe skip some details of the implementation. However, we hope that after reading the manuscript they understand the basics, and that they are aware of the possibilities that the code offers, but also of the drawbacks and the pitfalls of \numcalc. \numcalc{} was developed with a certain kind of problem at mind, but it was tried to make it applicable to most common problems in acoustics. 
  
From experience and bugreports/requests, we encountered a few types of errors that are common. If users are ``lucky'', the code stops with an error report, and users immediately see that there is a problem. The most common cause for such errors are badly shaped, loose or overlapping elements in the mesh. In Sec.~\ref{Sec:Quadrature}, it was mentioned that if two elements are close together, one of them will be subdivided based on the distance of their midpoints. If elements are very irregular or even overlap, this distance criterion will never be fulfilled, thus the code stops after a fixed number of subdivisions. A second possible problem is that the iterative solver does not converge. One reason for that can also be a few distorted elements in the mesh, or some problems with the fast multipole method if the expansion length $L$ gets too big, which is the case for a very large number of elements and if small and big elements are close together, i.e. if the mesh is locally not very regular.

In Section \ref{Sec:Expansion}, we described the idea behind the expansion of the Green's function and the problem with the truncation. If users encounter such a problem, one possible solution could be a change in the initial length of the boundary box in the FMM (see also Section \ref{Sec:FMM_I}). A second possible reason for problems with the stability is the direction of the normal vector, which somehow seems to be the most common error when using BEM. A wrong normal vector means that the role of interior problem and exterior problem is switched, which in turn means that in general sound sources are on the wrong side of the scatterer. If users are lucky, a wrong direction triggers some problem with the iterative solver, but in most of the cases, the code will finish without any problem, only the results will be, of course, completely wrong, which leads us the second type of error. In the ``unlucky'' case \numcalc{} runs without a problem, but the results do not make sense at all. Thus, the most important hint we can give is \emph{"Check the plausibility of your results"}. It is easily possible that during a mesh reduction process loose vertices can get inserted, elements can get twisted, normal vectors can be inverted, holes can appear in the surface or external sound sources end up too close to the surface of the scatterer\footnote{Remember the Green's function has a singularity at 0, so does the Hankel function used in the FMM.}. Technically, most of these irregularities are not really problems for the code itself, and such errors can only be discovered by checking the results of the computation.
  

%
%\item For researchers, already familiar with the topic, this manuscript shall provide details of an implementation that most of the time are hardly ever mentioned in the literature, because they are just technicalities of an implementation. However, as \numcalc{} is open-source, people may want to use the code and adapt it to their problem at hand. In that case, we find it very helpful to have a motivation and an idea why things have been implemented like they are. There are certain aspects we did not mention, e.g., parallelization. This is mainly because calculations can/need be done independently for different frequencies, thus in (our) practical work, \numcalc{} is started multiple times in parallel to calculate results for different frequencies, thus already having a simple way to parallelize computations.
%\end{itemize}

In Sec.~\ref{Sec:Benchmark}, a discussion about the accuracy and the memory requirements has been presented. It also provides users with a rough guideline on what to expect when using \numcalc{}, and it also should remind them on their responsibility to know if a numerical method or program is suitable to their problem at hand. In one of the examples it was shown that for low frequencies, it is possible that accuracy can suffer for fine grids. It was also shown, that this problems can be partly avoided by improving the accuracy of the code. But users should ask themselves, if it is really necessary to use such a fine grid for low frequencies in the first place. Sure, a coarser mesh, in general, means less accuracy, but is it necessary to invest computing time to achieve an accuracy of, e.g., $10 ^{-5}$ if other parameters like impedances or geometry have an uncertainty of about $10^{-2}$?

Section~\ref{Sec:Benchmark} also showed that the estimation of the memory requirements of \numcalc{} is not always straightforward. To give a better estimation of the necessary memory, \numcalc{} has an additional option to just do a clustering, but no BEM computation, see \ref{Sec:MemoryEstim}. Based on the information about the size of the clusters a rough estimation of the necessary RAM can be given.

The head-and-torso example in Sec.~\ref{Sec:MLFMM4} also showed some of the limitations of \numcalc{} with respect to the needed RAM. As the length of the multipole expansion was high in the root level, the memory needed for storing the element-to-cluster matrices was relatively high. One way to prevent matrices with that many non-zero entries would be to restrict the size of the initial bounding box on the root level. This in turn results in clusters with smaller radii and therefore smaller expansion lengths $L$. As the number of non-zero entries of $\bT$ and $\bS$ depends on $L$, memory consumption can be reduced this way to some degree.

The discussion and examples provide a starting point for future improvements or adaptations to different computer architectures. If, for example, an architecture is used where memory is a bottleneck, e.g., GPUs, the memory consumption can be reduced by changing the clustering. But as it can be seen from the results for the example with head and torso, the memory consumption for large problems demand more sophisticated approaches than the ones currently used in \numcalc. \numcalc{} was originally aimed at midsize problems where the number of elements are between $\num{60000}$ and $\num{120000}$. Future developments will include dynamic clustering, interpolation and filtering between different multipole levels to avoid the storage of the matrices $\bT$ and $\bS$ at levels different from the leaf level, or different quadrature methods for the integrals over the unit sphere that are needed for the fast multipole method. %It is clear that the future version of \numcalc{} will need to be able to (automatically) adapt to the computer architecture at hand or at least provide users simple rules how to adapt the input file to the architecture at hand.

With respect to accuracy, we also plan on using methods to adapt the mesh size to the frequency used, which has the advantage of shorter computation times and an enhanced stability and accuracy in lower frequency bands. The simple example of plane wave scattering from a sound-hard sphere has already shown the general problem of having very fine grids at low frequencies. It is not a surprising fact that the condition of the scattering problem gets worse for fine meshes and low frequencies. This behavior can be motivated by the fact that the acoustic field at low frequencies changes only slightly, thus, the values on the collocation nodes are also almost the same, which has a negative influence on the stability of the system.

\section*{Acknowledgements}
We want to express our deepest thanks to Z.-S.~Chen, who was the original programmer of the algorithms used in \numcalc{} and H. Ziegelwanger, who cleaned up the original code and created the original version of \meshtohrtf{}.
  
This work was supported by the European Union (EU) within the project SONICOM (grant number: 101017743, RIA action of Horizon 2020).
%\mybox{Brauchen wir hier die Begruenndung? KP: tlws wegen HRTFs? In der Conclusion wÀre wieder ein bigger picture/Rauszoomen angebracht.}
\appendix
\section{The Mesh}\label{Sec:Mesh}
A mesh is a description of the geometry of scattering surfaces using \emph{non-overlapping} triangles or plane quadrilaterals. The vertices of these elements are called BEM-nodes, their coordinates are defined via node lists that contain a unique node-number and the coordinates of each BEM-node in 3D (see Appendix~\ref{Sec:NodeList}). The elements are provided using an element lists containing a unique element number and the node-numbers of the vertices for each element. These lists also provide information if the element lies on the surface of the scatterer (BEM-element) or away from the surface (evaluation elements, that are used, if users are also interested in the field around or inside the object) (see Appendix~\ref{Sec:NodeList}). The unique normal vector of each element always needs to point away from the object, thus the vertices of each element should be ordered counterclockwise. The mesh should not contain any holes, each edge must be part of exactly two elements. The geometry described by the mesh should not contain parts that are very thin in relation to the element size, e.g. very thin plates or cracks, see also Sec. \ref{Sec:GreenSgl} and there should be at least 6 to 8 elements per wavelength, see also Sec. \ref{Sec:Errororder}. The elements of the mesh should be as regular as possible. It improves the stability of the calculations, if elements are (at least locally) about the same size. Non-uniform sampling is possible as long as the change in element size is gradually, e.g., when using graded meshes towards singularities like edges or sudden changes in boundary conditions. In~\cite{Ziegelwangeretal16}, a special kind of grading was developed, that used coarser grids away from regions with much detail, where even the 6-to-8-element rule was violated. This grading was possible because for the specific problem of calculating HRTFs, which was the application at hand, the geometry of the pinna of interest is much more important than the rest of the head. Nevertheless, it is up to the user to have a good look at the result of the computation to judge its quality.

A specific boundary condition needs to be assigned to each element of the mesh: Pressure, particle velocity normal to the element (see Appendix~\ref{Sec:BC}), or (frequency dependent) admittance, see also Appendix~\ref{Sec:Admittance}. If no boundary condition is assigned to an element, it is automatically assumed to be sound hard, thus, the particle velocity is set to 0 and the element is acoustically fully reflecting. Admittance boundary conditions are used to model sound absorbing properties of different materials. %As these properties are, in general, frequency dependent, \numcalc{} offers the possibility to define these conditions based on a piece-wise linear scaling curve dependent on the frequency, see Appendix \ref{Sec:Admittance}.
On each element, it is possible, to combine an admittance boundary condition with a velocity boundary condition. This additional velocity is used to model the effect of a vibrating part of the object, that generates sound.

%  \numcalc{} can be used for internal or external problems, users have the option to define evaluation nodes inside or outside the object.
%  External sound sources can be either point sources or plane waves (see Sec.~\ref{Sec:Incoming} and Appendix \ref{Sec:Sources}). In case of point sources, the source  should not be placed too close to the surface of the scatterer to avoid stability problems, see also Sec.~\ref{Sec:GreenSgl}.
%  Frequencies are defined by using piece-wise linear curves and frequency steps, see Appendix \ref{Sec:FreqCurv}. This approach has the advantage that frequency bands can be defined easily. Also, in combination with command line parameters parallelization is straight forward. Frequencies are treated independently from each other, and \numcalc{} provides users with the option to use the same input file for different ranges of frequencies (see Appendix \ref{Sec:Commandline}). 
%  \end{itemize}

\section{Example Input File}\label{App:Input}
In the following, we present an example input file for the benchmark problem of a plane wave scattering from sound-hard sphere. We want to calculated the acoustic field for 10 frequencies between $100$ an $1000$\,Hz, the plane wave moves along the $z$ axis. The different parts of the input file will be described in the following subsections. A line in the input file that starts with the hash tag '\#'  is ignored by \numcalc{} and can be used for comments. The unit for the coordinates is meters, the speed of sound is given in m/s, and the density is given in kg/m$^3$.
\begin{lstlisting}[basicstyle=\ttfamily,caption=Example input file for \numcalc{},label=Lst:InpFile,captionpos=b]
Mesh2HRTF 1.0.0
Mesh for plane wave reflection on a sound hard sphere
##
## lines starting with '#' are ignored
##
##
## Controlparameters I, reserved for later use
##
0 0 0 0 7 0
##
## Controlparameter II: Frequency Steps
## (dummy, nSteps, stepsize, i0)
# do 10 frequency steps with stepsize 1 and start at s1 = i0 + stepsize
1 10 1.0 0.0
##
## Frequency Curve 
#
# (Curvenr, Nr _of_Nodes)
0 2
# nodes of the piecewise linear curve
0.0 0
10.0 1000
## 1. Main Parameters I 
## (nBGrps, nElems, nNodes, min. ExpLength, nLevelsSym,
##  dummy, dummy, FMM_type, Solver)
## FMM_type = {0,1,4} = {No FMM, SLFMM, MLFMM}
## Solver = {0,4} = {CGS, direct Gauss Elimination}
2 28524 27042 8 0 2 1 4 0
##
## 2. Main Parameters II
# (nplanewaves, npointsrcs, ext. vs. int. probl,
#  default box length, default preconditioning)
1 0 0 0.0e0 0 
##
## 3. Main Parameters III, not used (yet, anymore)
0 0 0 0
##
## 4. Main Parameters IV
## dSoundSpeed, dDensity, harmonic factor
3.400000e+02 1.300000e+00 1.000000e+00
######################################
#
#  Define the geometry of the mesh
#
######################################
NODES
# Name of the file(s) containing the nodal data
TheBEMNodes.txt
TheEvalNodes.txt
ELEMENTS
# Name of the file(s) containing the nodal data
TheBEMElements.txt
TheEvalElems.txt
######################################
#
# define the boundary conditions
# First Elem, Last Elem, Realpart, Curve, Imagpart, Curve
#
######################################
BOUNDARY
# every element is sound hard, thus the default VELO = 0 
# condition can be used, no input necessary
RETURN
#
# external sound sources
# planewave
# (Sourcenr, direction(x,y,z), realpart, curve, imagpart, curve
# pointsource
# (Sourcenr, source position (x,y,z), realpart, curve, imagpart, curve
#
PLANE WAVES
0 0.000000e+00 0.000000e+00 1.000000e+00 1.000000e+00 -1 0.000000e+00 -1
#
# if some frequency dependent admittance, or external sources are used, 
# define the CURVES now
#
#CURVES
# no frequency dependent boundary conditions necessary
POST PROCESS
##
## BEM end of input ----
END
##
## this line seems to be important otherwise you end up in an
## endless loop, or at least add a return after END
\end{lstlisting}
In the current version v1.1.1 of \numcalc{}, the input files needs to be named \texttt{NC.inp}.
\subsection{Frequency-defining curves}\label{Sec:FreqCurv}
\numcalc{} numerically solves the Helmholtz equation in the frequency domain for a given set of frequencies defined by using frequency-defining curves. Frequency-defining curves are piece-wise linear functions that map a frequency step to a frequency or a frequency-dependent value of a boundary condition. The user has to provide the nodes of these curves. An example: If one would want to calculate the sound field for 10 frequencies between 100\,Hz and 1000\,Hz in steps of 100\,Hz, a simple way to describe the curve is given in Lst.~\ref{Lst:HRTFfreqs1}. The first non-comment line consists of a dummy parameter (\texttt{dummy = 1}) followed by the number of frequency steps to be calculated (\texttt{nSteps = 10}) followed by the distance $h_i$ between two frequency steps (\texttt{stepsize = 1.0}). The last entry in this line is given by the start index $i_0$ of the curve (\texttt{i0 = 0.0}).


The next line contains the unique number of the curve (in case of a frequency definition, the curve number always needs to be \texttt{Curvenr = 0}) and the number of nodes defining the piece-wise linear function mapping the frequency step $n$ to the respective frequency $f_n$. In Lst.~\ref{Lst:HRTFfreqs1}, this map is defined as the line between the two nodes
$(S_1, F_1)$ = ($0.0, 0.0$\,Hz) and $(S_2, F_2) = (10.0, 1000.0$\,Hz). The frequency for the $n$-th frequency step is then given as
\begin{equation}\label{Equ:Freqcurve}
f_n = F_1 + \frac{F_2 - F_1}{S_2 - S_1}((i_0 + h_i\cdot n - S_1) = 0\,\text{Hz} + \frac{1000 \,\text{Hz}}{10}(0.0 + 1.0 \cdot n) = 100\,\text{Hz}\cdot n,\, n = 1,\dots, 10.
\end{equation}

%\numcalc{} to calculate results for  \texttt{N\_steps = 10} frequency steps with \texttt{stepsize = 1.0}, starting with \texttt{step1 = step0 + stepsize}. The curve in this easy example is defined below by two (frequency step $\rightarrow$ actual frequency) pairs, which in this case are given by  (0 $\rightarrow$ 0\,Hz) and (11 $\rightarrow$ 10000\,Hz). %, which in term means \texttt{step0} corresponds to a frequency of $f = 0$\,Hz, \texttt{step11} to a frequency of $f = 10000$\,Hz.
%\begin{minipage}{0.5\textwidth}
\begin{lstlisting}[basicstyle=\ttfamily,caption=Definition of 10 uniform frequencies between 100\,Hz and 1000\,Hz,label=Lst:HRTFfreqs1,captionpos=b]
  ## Controlparameter II: Frequency Steps
  ## (dummy, nSteps, stepsize, s0)
  # do 10 frequency steps with stepsize 1 and start at s_1 = s0 + stepsize
  1 10 1.0 0.0
  ##
  ## Frequency Curve 
  #
  # (Curvenr,  Nr_of_Nodes)
  0 2
  # nodes of the piece-wise linear curve
  # (step, frequency)
  0.0 0.0
  10.0 1000.0
\end{lstlisting}

Also note that, for this example, the 6-to-8-elements-per-wavelength rule is \emph{not} fulfilled for the two highest frequencies. Nevertheless, \numcalc{} will do the calculations, however, a warning will be displayed.

A more elaborate example is given in Lst.~\ref{Lst:FreqCurv}. This code snippet contains the definition of 8 third-octave frequency bands (see Fig.~\ref{Fig:FreqCurvFig}), where three frequencies are chosen in each band.\\

%
%  \begin{table}[!h]
%    \begin{center}
%      % \begin{tabular}{cc}
%      \begin{tabular}{lcccccccc}
%        Band: & I & II & III & IV & V & VI & VII & VIII\\
%        Low: & 35.5 & 44.7 & 56.2 & 70.8 & 89.1 & 112 & 141 & 178\\
%        Center: & 40 & 50 & 63 & 80 & 100 & 125 & 160 & 200\\
%        High: & 44.7 & 56.2 & 70.8 & 89.1 & 112 & 141 & 178 & 224\\
%      \end{tabular}
%      \caption{Lowest, center, and highest frequencies in Hz of the third-octave band used in the example.}\label{Tab:FreqBands}
%    \end{center}
%  \end{table}
%
%\begin{minipage}{\textwidth}
%

\begin{figure}[!h]
  \begin{minipage}{0.48\textwidth}
    \begin{lstlisting}[basicstyle=\ttfamily,caption=Definition of a frequency-defining curve for the third octave band example,label=Lst:FreqCurv,captionpos=b]
      #dummy n-steps stepsize 
      1     24  1.0 0.0
      # Curvenr. Nr._of_Nodes
      0      9
      #define the curve
      0.0  35.481
      3.0  44.668
      6.0  56.234
      9.0  70.795
      12.0 89.125
      15.0 112.202
      18.0 141.254
      21.0 177.828
      24.0 223.872
    \end{lstlisting}
  \end{minipage}
  % &
  \qquad
  \begin{minipage}[c]{0.48\textwidth}
    \includegraphics[width=0.92\textwidth]{Freqsteps1}
    \caption{Frequency-defining curve for the third octave band.}\label{Fig:FreqCurvFig}
  \end{minipage}
\end{figure}

% kreiza: change the tab according to fabian
% 
%\begin{minipage}{0.5\textwidth}

%\end{minipage}
\subsection{Nodes and Elements}\label{Sec:NodeList}
The definition of the nodes (keyword \texttt{NODES}) and elements (keyword \texttt{ELEMENTS}) are done by providing the names of the text files containing the data for the nodes and the elements. There needs to be at least one file for the nodes and at least one file for the elements, but it is also possible to use more than one text file to describe different parts of the mesh. This has the advantage that the surface of the scatterer and an evaluation grid can be defined independently, as long as each node and each element have a unique identifying number. In the example input file Lst.~\ref{Lst:InpFile} the information about the BEM nodes is given in the file \texttt{TheBEMNodes.txt}, whereas the coordinates of evaluation nodes not on the surface of the scatterer are given in \texttt{TheEvalNodes.txt}. The name of these files can be chosen arbitrarily by the user.

The structure of the node defining files is as follows: The first line needs to contain the number of nodes in the file, the following lines contain a unique node number, and the $x,y,z$ coordinates of the node in meters. The first few lines in \texttt{TheBEMNodes.txt} looks for example like
\begin{lstlisting}[basicstyle=\ttfamily,caption=First few lines of \texttt{TheBEMNodes.txt}  containing the coordinates of the BEM nodes,label=Lst:TheBEMNodes,captionpos=b]
2168
1 -5.773503e-01 -5.773503e-01 -5.773503e-01
2 -5.975550e-01 -5.975550e-01 -5.346551e-01
3 -6.174169e-01 -6.174169e-01 -4.874348e-01
...
\end{lstlisting}
Jumps in the node numbers are allowed, which makes it easier to mix surface meshes and evaluation points.

The definition of the element files is as follows:
The first line again contains the number of elements in the file. The next lines contain  a unique integer to identify the element and the node numbers of the vertices of the element, followed by 3 integers. % denoting the type of the elements, zero and the group number of the element.
As an example, we show the first few lines of \texttt{TheBEMElements.txt} and \texttt{TheEvalElements.txt}
\begin{lstlisting}[basicstyle=\ttfamily,caption=First few lines of \texttt{TheBEMElements.txt} containing the definition of triangular surface elements,label=Lst:TheBEMElems,captionpos=b]
4332
1 381 382 458 0 0 0
2 381 458 457 0 0 0
3 382 383 459 0 0 0
...
\end{lstlisting}

\begin{lstlisting}[basicstyle=\ttfamily,caption=First few lines of \texttt{TheEvalElements.txt} containing the definition of the quadrilateral evaluation elements,label=Lst:TheEvalElems,captionpos=b]
24192
5000 2169 2170 2351 2350 2 0 1
5001 2170 2171 2352 2351 2 0 1
5002 2171 2172 2353 2352 2 0 1
...
\end{lstlisting}
The last three digits contain a flag for the type of elements, the number 0 and the group number of the element. \numcalc{} automatically determines from the number of the entries in each element line (7 or 8) if the element is triangular or quadrilateral (see Lst.~\ref{Lst:TheBEMElems} and \ref{Lst:TheEvalElems}). An element can either be a boundary element on the surface of an object (\texttt{type = 0}), or an element containing evaluation points (\texttt{type = 2}). Elements can also be assigned to different groups, but this feature is not recommended because the clustering for the FMM is done separately for each group. In general, it is sufficient to assign the BEM elements \texttt{GrpNr = 0} and evaluation elements \texttt{GrpNr = 1}.
%
%  \subsection{Boundary conditions}
%  As each element has its unique identifying number, boundary conditions can be assigned to the element using this number. The definition starts with the keyword \texttt{BOUNDARY} followed by several lines containing the definitions. The definition is finished with the keyword \texttt{RETURN}.
%   
%  The definition lines have the form
%  \begin{verbatim}
%  FROM elem1 TO elem2 bctype realpart curvenr1 imagpart curvenr2
%  \end{verbatim}
%  \texttt{bctype} can be chosen from \texttt{bctype $\in$ \{ADMI,PRES,VELO\}.}
%
\subsection{Boundary conditions}\label{Sec:BC}
Boundary conditions describe the acoustic behavior of the scattering objects, the definition section starts with the keyword \texttt{BOUNDARY}.  In \numcalc{}, three types of boundary conditions \{\texttt{PRES,VELO,ADMI}\} can be defined for each element. The general syntax for the boundary conditions is
\begin{verbatim}
ELEM e1 TO e2 bctype real(v0) curve1 imag(v0) curve2
\end{verbatim}
This line means that all elements with element numbers in the interval [\texttt{e1}, \texttt{e2}] have the boundary condition \texttt{bctype} $\in$ \{\texttt{PRES,VELO, ADMI}\} with value \texttt{v0}. With \texttt{PRES} the sound pressure can be fixed for the element, if the pressure is set to zero, the element is sound-soft. With \texttt{VELO} the particle velocity can be fixed for the given element, if the velocity is set to zero, the element is sound hard. With \texttt{ADMI} an admittance can be assigned to the element to model sound absorbing materials. In general, elements can only have one type of boundary condition, however, it is possible, that elements can have an \texttt{ADMI} as well as a \texttt{VELO} boundary condition. This is done to model (vibrating) sound sources on sound-absorbing surfaces. If no condition is assigned to an element, it is automatically assumed to be sound hard. 

As sound-absorbing behavior can be frequency dependent, \texttt{ADMI} conditions can be combined with a piece-wise linear curve (see Sec. \ref{Sec:FreqCurv}) that maps each frequency step to a scaling factor for the admittance. % and \ref{Sec:ExampleBC} and Lst.~\ref{Lst:ExampleBC}).
Thus, for the \texttt{ADMI} condition, \texttt{curve1} and \texttt{curve2} can contain the identifiers of piece-wise linear curves used to describe the frequency dependence of the real and imaginary part of the admittance. The curves are defined after the external sound source section. If no frequency dependence is needed, \texttt{curve1} and \texttt{curve2} are set to -1. The boundary condition section is closed using the keyword \texttt{RETURN}.
% The frequency steps themselves have already been defined for the frequencies (see Sec.~\ref{Sec:FreqCurv}).
%
\subsection{Sound sources}\label{Sec:Sources}
Besides vibrating sound sources on the scatterer, that are modelled by  \texttt{VELO} boundary conditions, sources away from the surface can be given by a combination of point sources and plane waves.

A point source $p_{\text{point}}(\bx) := P_0 \frac{e^{\I k ||\bx - \bx_0||}}{4\pi ||\bx - \bx_0||}$ at a source point $\bx_0$ is defined by its unique identifying  number, the coordinates of its origin \texttt{(X0,Y0,Z0)} and its source strength \texttt{P0}:
\begin{verbatim}
POINT
Nr X0 Y0 Z0 Real(P0) curve1 Imag(P0) curve2
\end{verbatim}
A plane wave $p_{\text{planewave}}(\bx) := P_0 e^{\I k{\bf d}\cdot \bx}, ||{\bf d}|| = 1$ with direction \texttt{(DX,DY,DZ)} and strength \texttt{P0} is given by
\begin{verbatim}
PLANE
Nr DX DY DZ Real(P0) curve1 Imag(P0) curve2
\end{verbatim}
If the input is not dependent on the frequency, the curves identifiers are \texttt{curve1 = curve2 = -1}.
%
%
\subsection{Frequency definition curves for admittance}\label{Sec:Admittance}
If an admittance boundary condition or the strength of the external sound sources need to be frequency dependent, it is necessary to define additional scaling curves dependent on the frequency steps. The definition of the curves begins with the keyword \texttt{CURVES} followed by a line containing the \texttt{number\_of\_curves} and the largest number of nodes \texttt{nMaxCurves} that define these curves. Each single curve is then defined by one line containing the unique curve number \texttt{icurve} and the number of nodes \texttt{inodes} defining the respective curve. The next \texttt{inodes} lines contain the nodes of the piece-wise linear curve similar to the frequency definition curve. An example:
\begin{verbatim}
CURVES
# number_of_curves max_number_of_points_per_curve
2 4
# curve_number  number_of_nodes_defining_curve
1 4
0.0 0.0 
1.0 1.0 
3.0 4.0
8.0 6.0
# curve_number number_of_nodes_defining_curve
2 3
0.0 0.0 
2.0 2.0
10.0 9.0
\end{verbatim}
defines two curves (one for the realpart, one for the imaginary part) that are used to define a frequency dependent admittance
\texttt{ELEM e0 TO e1 ADMI real(v0) 1 imag(v0) 2}, which assigns all elements between \texttt{e0} and \texttt{e1} a frequency dependent admittance
$
\alpha(s) = c_1(s)\, \text{real}(v0) + \I c_2(s)\, \text{imag}(v0) 
$
with 
$$
c_1(s) = \left\{
  \begin{array}{cl}
    s & \text{for } s\in [0,1],\\[2pt]
    1 + \frac32 (s - 1) & \text{for } s\in [1,3],\\[2pt]
    4 + \frac25 (s - 3) & \text{for } s\in [3,8],
  \end{array}
\right.
$$
and
$$
c_2(s) = \left\{
  \begin{array}{cl}
    s & \text{for } s \in [0,2],\\
    2 + \frac78(s-2) & \text{for } s\in [2,10].\\
  \end{array}
\right. 
$$
The frequency steps $s$ have already been defined via the frequency definition curve, e.g., the curve defined in Section \ref{Sec:FreqCurv}.
% Additionally, it is also possible to define arbitrary elements of the scatterer as a ``vibrating'' source by setting the appropriate \texttt{VELO} boundary condition; in this case the source strength is defined by \texttt{VELO = v0}.

\section{Command-line options}\label{App:Commandline}
Besides the parameter option that can be set using the input file \texttt{NC.inp}, there is the additional option start \numcalc{} with command line parameters.
\subsection{Estimation of memory consumption}\label{Sec:MemoryEstim}
The discussion about the non-zero entries of the multipole matrices in Secs.~\ref{Sec:SLFMMEstim} and \ref{Sec:MLFMMEstim} has also shown that the clustering is very dependent on the mesh and that the numbers of clusters in the nearfield and the interaction list can vary to a large degree.
As it is hard to estimate beforehand how the clustering will look like exactly, \numcalc{} provides users with the option to just compute the clustering and the number of non-zero entries of the FMM matrices \emph{without} doing actual BEM calculations. If users start \numcalc{} with \texttt{NumCalc --estimate\_ram} an estimation of the memory needed based on the entries of the multipole matrices is done. However, auxiliary variables and arrays to store mesh data and results are neglected in this estimation. Thus, the estimate slightly underestimates the RAM needs. For the example of the sphere, \numcalc{} estimates the necessary  RAM to be 0.095 GByte, whereas Gnu/time \cite{Gnutime} listed the maximum memory consumption with 0.116 GByte. This difference can be explained partly by the number $N_e = 24884$ of evaluation nodes used in this example. If $N_e = 70$ the memory consumption is reduced to about 0.1 GByte. Nevertheless, this feature helps to get a rough idea to reserve the right amount of memory for calculations on a computer cluster.

\subsection{Parallelization}\label{Sec:Commandline}
As the BEM systems for different frequencies are independent of each other, parallelization is straight forward. \numcalc{} offers users the option to use the same input file for different ranges of frequency-steps. If, for example, only the first 10 frequencies should be calculated using one single thread, \numcalc{} can be started using \texttt{NumCalc -istart 1 -iend 10}, whereas a second thread can handle the next 10 frequencies using \texttt{NumCalc -istart 11 -iend 20}. However, it is up to the user to distribute the jobs over the different nodes of a computer cluster.

\section{Output}\label{App:Output}
The output of \numcalc{} can be split into two file types. First, there is \texttt{NC.out}\footnote{Or \texttt{NC\{step1\}-\{step2\}.out} if the command line parameters \texttt{-istart} and \texttt{-iend} are used.} containing general information, e.g., frequencies used, information about the mesh, the highest frequency where the 6-to-8-elements-per-wavelength rule is still valid and similar data.  This file also contains information about the FMM, e.g., the number of clusters and the expansion lengths for each level, and information about the number of iterations and the final error of the iterative solver.

Second, the results of the calculations are stored in the folder \texttt{be.out}. This folder contains several sub-folders \texttt{be.}$n$, one for each frequency step $n$. Inside each of these folders, one can find the files
\begin{itemize}
\item \texttt{pBoundary} containing the real and imaginary part of the sound pressure at the midpoints of each element,
\item \texttt{vBoundary} containing the real and imaginary part of the particle velocity normal to the surface at the midpoints of each element,
\item \texttt{pEvalGrid} containing the real and imaginary part of the sound pressure at each evaluation point,
\item \texttt{vEvalGrid} containing the real and imaginary part of the particle velocity in the $x$, $y$, and $z$ direction.
\end{itemize}
The structure of having separate folders for each frequency step has the advantage that the calculation for different frequency steps can be easier distributed on different nodes of a computer cluster. %\numcalc{} can be started with the command-line parameters \texttt{-istart} and \texttt{-iend}, which denote the first and the last frequency step in the current calculation. \texttt{NumCalc -istart 11 -iend 20}, for example, only calculates the 10 frequency steps from 11 to 20.


\bibliography{mesh2hrtf}{}
\bibliographystyle{elsarticle-num}

\end{document}






%The factorization used in the FMM assumes an independence of the Green's functions for $\bx$ and $\by$, which can be approximately assumed for a large  distance between $\bx$ and $\by$. 

%   One drawback of the BEM is the fact that the system matrix is densely populated, which makes numerical computations with even moderately sized meshes prohibitively expensive in terms of memory consumption and computation time. This dense structure is cause by the Green's function $\frac{e^{\I k ||\bx -\by||}}{4\pi ||\bx - \by||}$. Because of the norm $||\bx - \by||$ there is a nonlinear coupling between every element of the mesh and every collocation node. However, if the Green's function $G(\bx,\by)$ could be approximated by two functions, where one is only dependent on $\bx$ and the other is only dependent on $\by$, i.e. $G(\bx,\by)\approx G_1(\bx) G_2(\by)$, then computation time as well memory consumption can be reduced. This is the main idea behind the Fast Multipole Method (FMM).
%     
%   In a nutshell, elements of the mesh that are within a fixed distance to each other are grouped into different clusters $\Cl_i$ and most element-to-element interactions are reduced to interactions between clusters, see for example Fig.~\ref{Fig:FMMScheme}. For two clusters $\Cl_i$ and $\Cl_j$ with midpoints\footnote{A midpoint of a cluster is defined as the average over the coordinates of all vertices of elements in the cluster.}$\bz_i$ and $\bz_j$ that are ``sufficiently'' apart, the interactions of all the elements in both clusters (grey lines) are reduced to (black lines)
%   \begin{itemize}
%   \item local interactions between each elements of a cluster with its midpoint, and
%   \item interactions between clusters or their midpoints, respectively.
%   %$$
%   %\int_{\Gamma_j} G(\bx,\by) d\by \approx \Es(\bx,\bz_1) \D(\bz_1,\bz_2)\int_{\Gamma_j} \T(\by,\bz_2) d\by,
%   %$$
%   \end{itemize}
%   \begin{figure}[!h]
%     \begin{center}
%       \includegraphics[width=0.4\textwidth]{FMMScheme}
%     \end{center}
%     \caption{Scheme of a FMM interaction between two clusters. Instead of element to element interactions between the two clusters (gray lines) the interactions are reduced to an interaction on the cluster level.}\label{Fig:FMMScheme}
%   \end{figure}
%    
%   Note that the local and the cluster-to-cluster expansions can be calculated independently for each cluster.
%   To reduce memory consumption and to speed up the matrix-vector multiplications needed for the iterative solver, \numcalc{} offers the option to combine the BEM with the FMM. In Section~\ref{Sec:FMM_I} it was briefly mentioned, that if clusters are sufficiently apart from each other, the FMM can be applied. However, we have not explained how clusters are defined in the first place, and what ``sufficiently'' apart means.
