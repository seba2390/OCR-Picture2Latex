%\documentclass[aps,prl,epsfig,showpacs,superscriptaddress,amsmath,amssymb,preprint]{revtex4-1}
%\documentclass[aps,prl,epsfig,showpacs,superscriptaddress,amsmath,amssymb,twocolumn]{revtex4-1}
\documentclass[twocolumn,epsfig,a4paper,amsmath,amssymb,showpacs,prl,superscriptaddress]{revtex4-1}
\usepackage{multirow}
\usepackage{amssymb}
\usepackage{amsmath}
\usepackage{amsfonts}
\usepackage{latexsym}
\usepackage{graphicx}
\usepackage{caption}
\usepackage{subcaption}
\usepackage{dcolumn}
\usepackage{bm}
\usepackage{ulem}
\usepackage{color}
\usepackage{soul}
\usepackage{physics}
\usepackage{bbold}
\usepackage{gensymb}
%\usepackage{bibentry}

\graphicspath{ {./figs/} }

\newcommand{\Hamil}{{\cal H}}
\newcommand{\kk}{{\mathbf k}}
\newcommand{\be}{\begin{equation}}
\newcommand{\ee}{\end{equation}}
\newcommand{\ba}{\begin{eqnarray}}
\newcommand{\ea}{\end{eqnarray}}
\newcommand{\baa}{\begin{eqnarray*}}
\newcommand{\eaa}{\end{eqnarray*}}
\newcommand{\Lim}[1]{\raisebox{0.5ex}{\scalebox{0.8}{$\displaystyle \lim_{#1}\;$}}}

\newcommand{\dg}{^{\dagger}}
\newcommand{\mI}{{\mathbb 1}}
\newcommand{\me}{\mathrm{e}}
\newcommand{\bl}{\left (}
\newcommand{\br}{\right )}
\newcommand{\Vbar}{\overline{\overline{V}}}
\newcommand{\Gbar}{\overline{\overline{G}}}
\newcommand{\Tbar}{\overline{\overline{T}}}
\newcommand{\gbar}{\overline{\overline{\gamma}}}
\newcommand{\Abar}{\overline{\overline{A}}}
\newcommand{\Sbar}{\overline{\overline{\Sigma}}}
\newcommand{\Gammabar}{\overline{\overline{\Gamma}}}


\begin{document}

\title{Spin Transport and Accumulation in a 2D Weyl Fermion System}
\author{T.~Tzen Ong}
\affiliation{RIKEN Center for Emergent Matter Science (CEMS), Saitama 351-0198, Japan}
\affiliation{Department of Applied Physics, University of Tokyo, Tokyo 113-8656, Japan}
\author{Naoto Nagaosa}
\affiliation{RIKEN Center for Emergent Matter Science (CEMS), Saitama 351-0198, Japan}
\affiliation{Department of Applied Physics, University of Tokyo, Tokyo 113-8656, Japan}
\date{\today}

\begin{abstract}
In this work, we study the spin Hall effect and Rashba-Edelstein effect of a 2D Weyl fermion system in the clean limit using the Kubo formalism. Spin transport is solely due to the spin-torque current in this strongly spin-orbit coupled (SOC) system, and chiral spin-flip scattering off non-SOC scalar impurities, with potential strength $V$ and size $a$,  gives rise to a skew-scattering mechanism for the spin Hall effect. The key result is that the resultant spin-Hall angle has a fixed sign, with $\theta^{SH} \sim O \left(\tfrac{V^2}{v_F^2/a^2} (k_F a)^4 \right)$ being a strongly-dependent function of $k_F a$, with $k_F$ and $v_F$ being the Fermi wave-vector and Fermi velocity respectively. This, therefore, allows for the possibility of tuning the SHE by adjusting the Fermi energy or impurity size. 
\end{abstract}

\maketitle

The spin Hall effect (SHE) has a long and rich history, starting with the initial proposal of asymmetric Mott scattering by Dyakonov and Perel \cite{DyakonovJETP1971,DyakonovPhysLettA1971}. This extrinsic mechanism was re-introduced in 1999\cite{HirschPRL1999,ZhangPRL2000}, while an intrinsic SHE was first proposed in 2003\cite{MurakamiNagaosaZhangScience2003,SinovaMacDonaldPRL2004}. The proposal of a two-dimensional (2D) $Z_2$-protected Quantum Spin Hall (QSH) state\cite{KaneMelePRL2005}, and its successful prediction in HgTe/CdTe quantum well \cite{BernevigZhangScience2006} quickly followed; thus giving rise to a new field of topological materials\cite{KaneHasanRMP2010, ZhangQiRMP2011}, which now include 2D QSH states~\cite{[{For a review of SHE and QSHE, see }]NagaosaMurakamiCSST2011}, 3D topological insulators (TI)\cite{KaneFuPRL2007, MooreBalentsPRB2007}, topological Kondo insulators\cite{DzeroColemanPRL2010,XuDingNatComm2014} and Weyl semi-metals\cite{VishwanathWanPRB2011}. 

One of the most striking characteristic of 3D TI materials is the existence of spin-momentum locked chiral Weyl fermions on the surfaces, which are expected to provide highly efficient spin-charge conversion\cite{FertNatComms2013, ShiomiSaitohPRL2014}, via the spin Hall effect or spin accumulation in the Rashba-Edelsten effect\cite{EdelsteinSSC1990}. Hence, there is a strong interest in spintronic TI heterostructures, with many theoretical works\cite{ShankarMondalPRB2010, DasSarmaCulcerPRB2010,BurkovHawthornPRL2010,FranzGaratePRL2010,NagaosaZangPRB2010,NagaosaBranislavPRL2012}, discussing a plethora of spin-charge phenomena, including magnetoresistance effects, inverse spin-galvanic effect, and spin-transfer torque, which have stimulated a flurry of experimental efforts\cite{RalphMellnikNature2014, LiJonkerNNano2014, ShiomiSaitohPRL2014, AndoNanoLett2014, KangNatMat2014, TokuraKondouNatPhys2016}.

In heavy-metal/ ferromagnet systems, e.g. FePt/Au, a giant spin Hall angle (SHA) of $\sim 0.1$ has been reported\cite{SekiTakanashiNatMat2008}, which has been interpreted as resonant skew-scattering off the Fe impurities\cite{GuNagaosaPRL2009}. However, recent experiments on TI heterostructures\cite{KangNatMat2014,RalphMellnikNature2014} have reported values of $\tan \theta^{SH} > 100$\%, with combined surface and bulk contributions. In order to disentangle the surface Weyl fermion contribution from the bulk bands, a Cu-layer inserted TI/Cu/ferromagnet heterostructure has recently been engineered, with $\tan \theta^{SH} \sim 50$\% \cite{TokuraKondouNatPhys2016}. 

Similar to the anomalous Hall effect, there are both intrinsic Berry curvature and extrinsic scattering contributions to the SHE. For systems with weak spin-orbit coupling (SOC), it has been shown\cite{HalperinRashbaPRL2005} that the extrinsic skew scattering mechanism dominates in the clean limit; hence, the spin Hall conductivity $\sigma^z_{xy}$ scales with the longitudinal conductivity $\sigma_{yy}$, and the SHA, $\theta^{SH} = \tfrac{\sigma^z_{xy}}{\sigma_{yy}}$ is a well-defined measure of the SHE. The Rasha-Edelstein effect is a closely related transport-driven spin accumulation phenomena, which also scales with $\sigma_{yy}$ in the clean limit; the spin accumulation $\langle S^i \rangle = \sigma^i_{\alpha} E_{\alpha}$ is proportional to the applied electric field $E_{\alpha}$ (along $\alpha$-direction) with a coefficient $\sigma^i_{\alpha}$. For the strongly SOC-coupled Weyl system considered here, the main results are that due to spin-momentum locking, chiral spin-flip scattering off non-magnetic impurities drives an $O(\tfrac{1}{n_i})$ skew-scattering mechanism, and that Rashba-Edelstein is an $O(\tfrac{1}{\gamma_t})$ effect; here, $n_i$ is the impurity concentration and $\gamma_t$ is the transport scattering rate.

We adopt the Kubo formula framework for calculating $\sigma_{yy}$, $\sigma^{z}_{xy}$ and $\sigma^{i}_{y}$, given by the retarded current-current correlation functions, $\sigma_{yy} = - \Lim{\omega \rightarrow 0} \Lim{\vec{k} \rightarrow 0} Im \left[ \frac{\pi_{yy}(\vec{k}, \omega)}{\omega} \right]$, $\sigma^{z}_{xy} = -\Lim{\omega \rightarrow 0} \Lim{\vec{k} \rightarrow 0} Im \left[ \frac{\pi^{z}_{xy}(\vec{k}, \omega)}{\omega} \right]$, and $\sigma^{i}_{y} = -\Lim{\omega \rightarrow 0} \Lim{\vec{k} \rightarrow 0} Im \left[ \frac{\pi^{i}_{y}(\vec{k}, \omega)}{\omega} \right]$; where, $\pi_{yy}(\vec{k}, \omega)$, $\pi^{z}_{xy}(\vec{k}, \omega)$, and $\pi^{i}_{y}(\vec{k}, \omega)$ are the current-current, spin current-current and spin accumulation-current correlation functions respectively.

In spin-orbit coupled systems, the proper definition of the spin current is more subtle as spin is not a conserved quantity. Ref.~\cite{NiuPRL2006} presented a bulk conserved spin current that satisfies a continuity equation, $\tfrac{d S^z}{d t} + \nabla \cdot (\vec{J}_s + \vec{P}_{\tau}) = 0 $, with an additional spin-torque density term, $\nabla \cdot \vec{P}_{\tau}^i = \tfrac{i}{\hbar} [S^i, H^0]$, as well as the conventional spin current $\vec{j}^{z}_s = \psi\dg \tfrac{1}{2} \{\vec{v}, S^z\} \psi$. Hence, the transport spin current is the sum of a spin-polarized and a spin-torque current, $\vec{\mathcal{J}}^i_s = \vec{j}^i_s + \vec{P}^i_{\tau}$, succintly expressed as the time-derivative of a spin-dipole operator, $\hat{\mathcal{J}}_s = \tfrac{d (\hat{\vec{r}} \hat{\vec{S}}) }{d t}$. As pointed out by several groups\cite{MolenkampPRB2003, HalperinMischenkoPRL2004, NagaosaSugimotoPRB2006}, there is no finite conventional spin current for Weyl systems; hence, spin transport for Weyl fermions is solely due to the spin-torque density $P_{\tau}$ coming from quantum-mechanical evolution of the electron spin.  
%
\begin{figure}
\begin{center}
\begin{subfigure}{0.5 \columnwidth}
\makebox[10pt][l]{(a)}
\includegraphics[width=\columnwidth]{Spin_dipole_FS_shift_2.eps}
\end{subfigure}%
\begin{subfigure}{0.45 \columnwidth}
\makebox[10pt][l]{(b)}
\includegraphics[width=\columnwidth]{Fermi_surf.eps}
\end{subfigure}
\end{center}
\captionsetup{justification=RaggedRight, singlelinecheck=false}
\caption{Fig.~(a) shows a colour density plot of the FS contribution to the Rashba-Edelstein effect $\langle \sigma^{y}_y \rangle$ (Eq.~\ref{eqn: sigmaiy}). When the FS is shifted by $\delta k_y = e E_y \tau_t$ due to an external electric field $E_{y}$, the non-equilibrium distribution gives rise to a net $\langle S^y \rangle$.  Fig.~(b) illustrates spin-dependent skew scattering, $\Tbar_{\sigma, \sigma'}(\vec{k}, \vec{k}')$ and $\Tbar_{\sigma', \sigma}(\vec{k}', \vec{k})$ having positive ($\nu$) and negative ($-\nu$) chirality respectively, with the helical Weyl fermions defining positive ($\nu$) chirality.}
\label{fig: Fermi surface}
\end{figure}
 
We consider elastic scattering near the Fermi energy, $E_F$, of 2D Weyl fermions (Dresselhaus-type $v_F \, \vec{k} \cdot \vec{\sigma}$ system) from a dilute ($n_i \ll 1$) random distribution of non-magnetic impurities, with scattering off each impurity given by $H^{imp} = \sum_{\vec{r}} c\dg_{\sigma}(\vec{r}) V \mathrm{e}^{-\frac{|\vec{r}|^2}{a^2}} c_{\sigma}(\vec{r})$, with impurity size $a$. Note that the results can be easily translated into the Rashba-type $v_F \, \hat{z} \times \vec{k} \cdot \vec{\sigma}$ case via rotation of the momentum by $90^{\degree}$. Choosing the chemical potential $\mu$ to lie in the upper helical band, we obtain the following Hamiltonian as,
%
\ba
\label{eqn: Hamiltonian}
H & = & H^0 + H^{imp} \\
H^0 & = & \sum_{\vec{k}, \alpha, \beta} c\dg_{\vec{k}, \alpha}  v_F  \vec{k} \cdot \vec{\sigma}_{\alpha, \beta} c_{\vec{k}, \beta} - \mu \, c\dg_{\vec{k}, \alpha} c_{\vec{k}, \alpha} \cr
H^{imp} & = & \sum_{\vec{k}, \vec{k'}} c\dg_{\vec{k}, \alpha} V_{\vec{k}, \vec{k}'} c_{\vec{k}', \alpha} 
\ea
%
Here, $V_{\vec{k}, \vec{k}'} = \sum_n V_n \me^{i n (\theta_{k} - \theta_{k'})}$, and $V_n \approx \tfrac{V a^2}{2} \frac{(k_F a)^n}{2^n \Gamma(\tfrac{n+1}{2})}$, while $v_F$ and $\sigma^i \in [\mI, \vec{\sigma}]$ are the Fermi velocity and spin Pauli matrices, and $k_F a$ determines $V_n$, which will be shown to control the skew scattering strength. Since the impurity is non-magnetic, the system is invariant under time-reversal symmetry, $\mathcal{T} = \mathcal{K} i \sigma_2$, $H= \mathcal{T} H \mathcal{T}^{-1}$. All the scattering events from an impurity are summed up in the $\Tbar$-matrix, and the spin-dependent skew scattering is captured by the $\sigma^{\pm}$ terms, illustrated in Fig.~\ref{fig: Fermi surface}. The following Dyson equations, in operator formalism, give the effective Green's function, $\hat{\Gbar}_{eff} = \hat{\Gbar}_0 + \hat{\Gbar}_0 \hat{\Tbar} \, \hat{\Gbar}_0$, and $\Tbar$-matrix, $\hat{\Tbar} = \hat{\Vbar} + \hat{\Vbar} \, \hat{\Gbar}_0 \hat{\Tbar}$, with $\hat{\Gbar}_0$ being the bare Green's function, and Fig.~\ref{fig: Geff Feyn diag} shows the Feynman diagram for the effective Green's function.
%
\begin{figure}[bht!]
\begin{center}
\includegraphics[trim=0mm 0mm 0mm 0mm, clip, width=\columnwidth]{Geff_feyn_diag.eps}
\end{center}
\captionsetup{justification=RaggedRight, singlelinecheck=false}
\caption{\label{fig: Geff Feyn diag} 
Feynman diagram for  $\Gbar_{eff}(\vec{k}, \vec{k}', \sigma, \sigma')$ that sums up the infinite set of scattering events from a single impurity. This is captured by the $\Tbar$-matrix, which is represented by the diamond symbol in the second line above.}
\end{figure}
%
\begin{subequations}
\ba
\label{eqn: G0 & T-matrix}
\overline{\overline{G}}_0(\vec{k}, i \omega_n) & = & \frac{1}{i \omega_n + \mu - v_F \vec{k} \cdot \vec{\sigma}} \\
& = & g^0_0(k, i \omega_n) \mI + g^a_0(k, i \omega_n)(\cos \theta \, \sigma^x + \sin \theta \, \sigma^y) \cr
\Tbar(\vec{k}, \vec{k}', i \omega_n) & = & \sum_{nm} T^{i}_{nm}(|\vec{k}|, |\vec{k}'|, i \omega_n) \me^{i n \theta_{k}} \me^{-i m \theta_{k'}} \sigma^i 
\ea
\end{subequations}

Rotational symmetry of the Hamiltonian allows us to carry out a multipole expansion of $\overline{\overline{G}}_0(\vec{k}, i \omega_n)$ and the $\Tbar$-matrix, where $g^0_0(k, i \omega_n) = \tfrac{i \omega_n + \mu}{(i \omega_n + \mu)^2 - v_F^2 k^2}$, and $g^a_0(k, i \omega_n) = \tfrac{v_F k}{(i \omega_n + \mu)^2 - v_F^2 k^2}$. We assume the $\Tbar$-matrix varies slowly near $E_F$, i.e. absence of resonances, thereby simplifying the radial integral and reducing the Dyson equation to a set of coupled algebraic recurrence equations for the retarded $\Tbar$-matrix coefficients, $T^{i}_{nm}(|\vec{k}| = |\vec{k}'| = k_F, \omega = E_F)$.
%
\begin{subequations}
\ba
\label{eqn: T-reccurrence eqns}
T^{z \pm}_{nm} & = &  \delta_{n, m} \Big[V_n \bl 1 - V_{n \pm 1} \langle g_0^0(E_F) \rangle \br \Big] \Big[ \bl 1 - V_n \langle g_0^0(E_F) \rangle \br \cr
& & \times \bl 1 - V_{n \pm 1} \langle g_0^0(E_F) \rangle \br - V_n V_{n \pm 1} \langle g_0^1(E_F) \rangle^2 \Big]^{-1} \\
T^{\pm}_{nm} & = & \frac{\delta_{n \mp 1, m}}{2} \Big[ V_n V_{n \mp 1} \langle g_0^1(E_F) \rangle \Big] \Big[ \bl 1 - V_{n \mp 1} \langle g_0^0(E_F) \rangle \br \cr
 & & \times \bl 1 - V_{n} \langle g_0^0(E_F) \rangle \br - V_n V_{n \mp 1} \langle g_0^1(E_F) \rangle^2 \Big]^{-1}
\ea
\end{subequations}
%
The $\Tbar$-coefficients reduce to two set of coupled equations for $T^{z\pm} = T^0_{nm} \pm T^3_{nm}$ and $T^{\pm} = T^1_{nm} \pm i T^2_{nm}$, given in terms of $V_n$ and the momentum-averaged retarded Green's functions, $\langle g_0^{i, (R)}(\omega) \rangle = \int \tfrac{d k}{2 \pi} k g^{i, (R)}_0(k, \omega)$ (refer to SOM for calculation details). The arguments of the $\Tbar$-matrix coefficients are dropped, understanding that they are evaluated at $k_F$ and $E_F$. Defining the symmetric and asymmetric parts of the spin-flip scattering as $T^{S/A} = T^{+}_{10} \pm T^{-}_{-10}$, $T^3_0 \equiv T^3_{00}$, and $T^3_1 \equiv T^3_{11}$, we can now write down the $s$ and $p$-wave channels of the $\Tbar$-matrix.  
%
\ba
\label{eqn: T-matrix}
\Tbar(\theta_k, \theta_{k'}) & = & T^0 \mI + T^3_0 \sigma^z+ T^3_1 \big( \me^{i (\theta_k - \theta_{k'})} - \me^{-i (\theta_k - \theta_{k'})}\big) \sigma^z \cr
 & + & \frac{T^S + T^A}{2} \me^{i \theta_{k} } \sigma^{-} + \frac{T^S - T^A}{2} \me^{-i \theta_{k}} \sigma^{+}  \cr
 & + & \frac{T^S + T^A}{2} \me^{-i \theta_{k'} } \sigma^{+} + \frac{T^S - T^A}{2} \me^{i \theta_{k'}} \sigma^{-}
\ea
%
with detailed expressions for the $\Tbar$-matrix coefficients shown in the SOM. Charge-transport is dominated by the largest term, $|T^0| \propto V_0$, while spin-flip scatterings are captured by the $T^{S/A} \sigma^{\pm}$ terms. Upon projection into the upper helical band, we obtain a chiral spin-flip scattering term, $T^S \sin(\theta_{k} - \theta_{k'})$, which comes from $3^{rd}$ and higher orders in perturbation; $T^S \propto V_0 V_1^2 N_0(E_F)^2$, in agreement with previous work \cite{NagaosaSugimotoPRB2006}. Hence, the skew scattering strength can be tuned by varying $k_F a$, i.e. either the Fermi level or the impurity size $a$.

It is now straightforward to calculate the effective Green's function in the dilute impurity limit ($n_i \ll 1$)\cite{Rammer2004}, $\Gbar^{(R)}(\vec{k}, \omega) = \Big[ \omega - v_F \vec{k} \cdot \vec{\sigma} - \overline{\overline{\Sigma}}^{(R)}(\vec{k}, \omega)  \Big]^{-1}$, where the retarded self-energy is $\overline{\overline{\Sigma}}^{(R)}(\vec{k}, \omega) = n_i \sum_{\vec{k_1}} \Vbar(\vec{k}, \vec{k}_1) \Gbar^{(R)}_{eff}(\vec{k}_1, \omega) \Tbar^{(R)}(\vec{k}_1, \vec{k}, \omega)$. The appearance of $\Gbar^{(R)}_{eff}(\vec{k}, \omega)$ instead of $\Gbar^{(R)}_{0}(\vec{k}, \omega)$ reflects the presence of multiple impurities. We assume an average quasi-particle scattering rate near the Fermi surface, i.e. $\gbar \equiv Im[\overline{\overline{\Sigma}}^{(R)}(k_F, E_F)]$, and take $v_F$ and $E_F$ to be experimentally determined parameters, thereby dropping the real part of the self-energy.
%
\begin{subequations}
\ba
\label{eqn: qp scattering rate}
\gbar & = & \gamma_0 \mI - \gamma_a \bl \cos \theta \, \sigma^x + \sin \theta \, \sigma^y \br \cr
& - & \, \gamma_b (\sin \theta \, \sigma^x - \cos \theta \, \sigma^y) + i \gamma_3 \, \sigma^z \\
 \gamma_0 & = & n_i N^{(0)}_{eff}(E_F) \Big[ |T^0|^2 + |T^3_0|^2 \cr
 & & - 2\left( |T^3_1|^2 + |T^A|^2 - |T^S|^2 \right) \Big]  \\
 \gamma_a & = & 4 n_i N^{(1)}_{eff}(E_F) \left[ |T^S|^2 - |T^A|^2 \right] 
\ea
\end{subequations}

We have carried out a multipole expansion of $\gbar$, and the main quasi-particle scattering channels relevant to transport are the $s$ and $p$-wave $\gamma_0$ and $\gamma_a$ terms (refer to SOM for complete expressions of all $\gamma$). As we shall show later, the transport scattering rate, $\gamma_t$, will be given in terms of $\gamma_0$ and $\gamma_a$. The angular momentum resolved density of states (DOS) is defined as $N^{(i)}_{eff}(\omega) = \int \frac{k dk}{2 \pi} Im \left[ g^{i}_{eff}(k, \omega) \right]$, and $N^{(0)}_{eff}(E_F)$ and $N^{(1)}_{eff}(E_F)$ correspond to the $s$ and $p$-wave components respectively. Since scattering events that result in a change of angular momentum, i.e involving the $l = 1$ component $N^{(1)}_{eff}(E_F)$, will also cause a spin-flip due to spin-orbit coupling, we see that $\gamma_{0}$ and $\gamma_a$ are due to spin-independent and dependent scattering respectively.

The effective Green's function is therefore given by,
%
\ba
\label{eqn: eff Greens func}
\Gbar^{(R)}_{eff}(\vec{k}, \omega) & = & \left[ \omega + \mu - v_F \vec{k} \cdot \vec{\sigma} - i \gbar(\vec{k}, \omega) \right]^{-1} \\
& = & g^0_{eff}(k ,\omega) \mI + g^a_{eff}(k, \omega) \bl \cos \theta \, \sigma^x + \sin \theta \, \sigma^y \br \cr
 & + & g^b_{eff}(k, \omega) \bl \sin \theta \, \sigma^x - \cos \theta \, \sigma^y \br + g^3_{eff}(k, \omega) \sigma^z \nonumber
\ea
%
where,

\begin{subequations}
\ba
\label{eqn: g0 expr}
 g^0_{eff}(k, \omega) & = & \frac{(\Omega(k) + i \kappa(k)) (\omega + \mu - i \gamma_0 )}{\Omega^2(k) + \kappa^2(k)} \\
 \label{eqn: ga expr}
 g^a_{eff}(k, \omega) & = & \frac{(\Omega(k) + i \kappa(k)) (v_F |\vec{k}| + i \gamma_a )}{\Omega^2(k) + \kappa^2(k)}
\ea
\end{subequations}
%
with $\Omega(k) = (\omega + \mu)^2 - v_F^2 |\vec{k}|^2 - \gamma_0^2 + \gamma_a^2 + \gamma_b^2 - \gamma_3^2$, and $\kappa(k) = 2 \bl (\omega + \mu) \gamma_0 + v_F |\vec{k}| \gamma_a \br$.

A similar multipole expansion of $\Gbar^{(R)}_{eff}(\vec{k}, \omega)$ has been done, and we show here only the main $s$ and $p$-wave terms, $g^0_{eff}(k, \omega)$ and $ g^a_{eff}(k, \omega)$, with complete expressions for the scattering-induced $g^b_{eff}(k, \omega)$ and $g^3_{eff}(k, \omega)$ terms relegated to the SOM for brevity. From Eqs.~(\ref{eqn: g0 expr}) \&~(\ref{eqn: ga expr}), it is clear that Weyl fermions in the $s$ and $p$-wave channels pick up a $\gamma_0$ and $\gamma_a$ scattering rate respectively, and we shall show later that it is chiral scattering between the $s$ and $p$-wave electrons that drive the SHE.

$\Gbar^{(R)}_{eff}(\vec{k}, \omega)$ and $\overline{\overline{\Sigma}}^{(R)}(\vec{k}, \omega)$ are determined self-consistently by solving Eqns.~\ref{eqn: qp scattering rate} \& \ref{eqn: eff Greens func}, i.e. $\overline{\overline{\Sigma}}^{(R)}(\vec{k}, \omega)$ is calculated using the disorder-averaged density of states, $N^{(i)}_{eff}(\omega) = \int \frac{k dk}{2 \pi} Im \left[ g^{i}_{eff}(k, \omega) \right]$. However, in the dilute impurity limit, $N^{(0)/(1)}_{eff}(E_F) = \tfrac{N_{0}(E_F)}{2} (1 + O(\gbar))$ \cite{Rammer2004}; allowing us to drop the $O(n_i)$ corrections.

As stated earlier, the DC longitudinal charge conductivity, spin-Hall conductivity and spin accumulation are given by analytic continuation of the corresponding Matsubara correlation functions, 
%
\begin{subequations}
\ba
\label{eqn: pi_yy}
\pi_{yy}(\vec{k}, i \omega_n) & = & - \int^{\beta}_{0} d \tau \me^{-i \omega_n \tau} \langle T_{\tau } \, j_{y}(\vec{k}, \tau) j_{y}(\vec{k}, 0)  \rangle \\
\label{eqn: pi_iy}
\pi^{i}_{y}(\vec{k}, i \omega_n) & = & - \int^{\beta}_{0} d \tau \me^{-i \omega_n \tau} \langle T_{\tau} \, \sigma^i(\vec{k}, \tau) j_{y}(\vec{k}, 0)  \rangle \\
\label{eqn: pi_zxy}
\pi^{z}_{xy}(\vec{k}, i \omega_n) & = & -\int^{\beta}_{0} d \tau \me^{-i \omega_n \tau} \langle T_{\tau} \, P^{z}_{x}(\vec{k}, \tau) j_{y}(\vec{k}, 0)  \rangle
\ea
\end{subequations}
% 
Note that $\pi_{yy}$ and $\pi^{i}_{y}$ are equal up to a factor of $\tfrac{e v_F}{\hbar}$ for Weyl fermions due to spin-momentum locking, i.e. $\hat{j_y} = e v_F \hat{\sigma}^y$. The spin torque current, $P^{z}_x$, arises from the intrinsic quantum-mechanical evolution of the electron spin, and the $z$-component of the spin-torque current along $\hat{x}$ is,
%
\ba
\label{eqn: spin torque}
P^{z}_x(\vec{k}) & = & \tfrac{i}{k_x} \tfrac{d \hat{S}^z(\vec{k})}{d t} \\
 & = &\frac{2 v_F}{i k_x} \sum_{\vec{p}} c\dg_{\vec{p}, \sigma} \left[  \left( \vec{p} + \frac{\vec{k}}{2} \right)_{y}\sigma^x - \left( \vec{p} + \frac{\vec{k}}{2} \right)_{x} \sigma^y  \right] c_{\vec{p} + \vec{k}, \sigma'} \nonumber
\ea

The Feynman diagrams for these correlation functions are shown in Fig.~\ref{fig: Vertex Feyn diag}, with chiral spin-flip scattering starting to contribute at third-order in perturbation theory. Fig.~\ref{fig: Vertex Feyn diag} shows the infinite subset of Feynman ladder diagrams summed up in the Bethe Salpeter equation for the scattering vertex,
%
\begin{widetext}
\ba
\label{eqn: vertex Dyson eqn}
\Gammabar^{y}(\vec{k} + \vec{p}, \vec{p}, i \Omega_m + i \omega_n, i \omega_n) = \sigma^y + \sum_{\vec{q}} \Tbar(\vec{k} + \vec{p}, \vec{k} + \vec{q}, i \Omega_m + i \omega_n) \Gbar_{eff}(\vec{k}+\vec{q}, i \Omega_m + i \omega_n) \cr
 \times \Gammabar^{y}(\vec{k} + \vec{q}, \vec{q}, i \Omega_m + i \omega_n, i \omega_n) \Gbar_{eff}(\vec{q}, i \omega_n) \Tbar(\vec{q}, \vec{p}, i \omega_n)
\ea
\end{widetext}

\begin{figure}[bht!]
\begin{center}
\includegraphics[trim=0mm 0mm 0mm 0mm, clip, width=\columnwidth]{Vertex_feyn_diag.eps}
\end{center}
\captionsetup{justification=RaggedRight, singlelinecheck=false}
\caption{\label{fig: Vertex Feyn diag} 
Feynman diagram for the effective scattering vertex, $\Gammabar^{y}(\vec{p}, \omega)$, is shown in the second line. This includes an infinite subset of scattering events from the dilute concentration of  impurities. The first line shows all the scattering events from a single impurity, and the second and third diagrams in the first line are the leading-order contributions to skew scattering.}
\end{figure}

Here, $\vec{k}$ and $i \Omega_m$ are the  external momentum and frequency, and the uniform DC limit of the conductivities is obtained by analytic continuation of $i \Omega_m \rightarrow \omega + i \eta$, and taking the limit $\vec{k} \rightarrow 0$ followed by $\omega \rightarrow 0$. Hence, we only need to calculate the on-shell component of the scattering vertex $\Gammabar^{y}(\vec{p}, \omega) = \Gammabar^{y}(\vec{p}, \omega - i \eta, \omega + i \eta)$. The Bethe-Salpeter equation for $\Gammabar^{y}(\vec{p}, \omega)$ is solved self-consistently by expanding $\Gammabar^{y}(\vec{p}, \omega) = \sum_{n} \Gamma^{i}_{n} \me^{i n \theta} \sigma^i $ in multipole terms, assuming that the $\Tbar$-matrix and $\Gammabar^{y}(\vec{p}, \omega)$ vary slowly near $E_F$ (see SOM for details). Keeping only the $s$- and $p$-wave channels, and evaluating $\Gammabar^{y}(|\vec{p}| = k_F, \omega = E_F)$ at the Fermi surface, we obtain,
%
\ba
\label{eqn: vertex func}
\Gammabar^{y}(k_F, E_F) & = & (\Gamma^0_{p_x} \cos \theta + i \Gamma^0_{p_y} \sin \theta) \, \mI \\
 & & + \Gamma^1_0(E_F) \sigma^x + \Gamma^2_0(E_F) \, \sigma^y  \cr
 & & + (\Gamma^3_{p_x}(E_F) \cos \theta + i \Gamma^3_{p_y}(E_F) \sin \theta) \, \sigma^z \nonumber
\ea 
%
where,
%
\be
\Gamma^2_0 = \frac{\gamma_0}{\gamma_{t}}  \text{,}    \hspace{2cm}   \Gamma^3_{p_x} = - \tfrac{\gamma_{s}}{\gamma_t}.
\ee 

After analytic continuation of the current-current correlation functions in Eq.~(\ref{eqn: pi_yy})  - (\ref{eqn: pi_zxy}), we find that the main contributions come from the $\Gamma^2_0$ charge-transport and $\Gamma^3_{p_x}$ spin-transport scattering vertices (refer to SOM for all the $\Gammabar$-coefficients). We can therefore define a transport and chiral spin-flip scattering rate respectively as, 
%
\be
\label{eqn: transport coeffs}
\gamma_t = ( \tfrac{1}{2} \gamma_0 + \gamma_a) \text{,}    \hspace{1cm}  \gamma_{s} = \frac{n_i \pi N_0(E_F)}{2} |T^0| |T^S|.
\ee

The main results of this paper are the charge and spin conductivities, and the Rashba-Edelstein coefficient,
%
\begin{subequations}
\ba
\label{eqn: sigmayy}
 \sigma_{yy} & = & \bl e v_F \br^2 \frac{N_0(E_F)}{2} \frac{1}{\gamma_{t}}  \\
\label{eqn: sigmazxy}
 \sigma^{z}_{xy} & = & - \hbar e v_F^2 \frac{N_0(E_F)}{2} \frac{1}{\gamma_{t}}  \frac{\gamma_{s}}{\gamma_0 + \gamma_a} \\
 \label{eqn: sigmaiy}
\sigma^{y}_{y} & = & \hbar e v_F \frac{N_0(E_F)}{2} \frac{1}{\gamma_{t}} 
\ea
\end{subequations}

Our key finding is Eq.~(\ref{eqn: sigmazxy}), which shows an $O(\tfrac{1}{n_i})$ skew scattering contribution to the SHE. Explicitly writing out the spin and angular-momentum scattering channels for $\sigma^{z}_{xy} = \hbar e v_F^2 Re[\Gamma^3_{p_x} (\eta^{a0}(E_F) - \eta^{0a}(E_F))]$, where $\eta^{ij}(\omega) = \int \tfrac{d p}{2 \pi} p^2 \tfrac{\partial g^{i(R)}_{eff}(p, \omega)}{\partial p} g^{j(A)}_{eff}(p, \omega)$, we see that chiral spin-flip scattering between the $s$ and $p$-wave electrons is the cause of the skew-scattering mechanism, and the strength of which is measured via the spin-Hall angle, 
%
\ba
 \theta^{SH}  & = & -\frac{\hbar}{e} \frac{\gamma_{s}}{\gamma_0 + \gamma_a} 
\ea
%
Here, $e < 0$ is the electron charge, and power counting of  $\gamma_t \sim \gamma_0 \sim n_i V_0^2 N_0(E_F)$ and $\gamma_{s} \sim n_i V_0^2 V_1^2 N_0(E_F)^3$, gives $\theta^{SH} \sim O\left( \tfrac{V^2}{v_F^2/a^2} (k_F a)^4 \right)$. This is our key result: $\theta^{SH}$ has a fixed positive sign, and is a strongly-dependent function of $k_F a$; hence, the SHE can be tuned by $E_F$.

Finally, we briefly discuss the effects of band bending in Weyl systems. The leading $O(\tfrac{1}{m})$ correction comes from including a conventional spin current, $\vec{j}^z_{s} = \psi\dg \tfrac{1}{2} \{\vec{v}, S^z \} \psi$, with $\vec{v} = \tfrac{\hbar \vec{k}}{m}$. However, it has been pointed out\cite{MolenkampPRB2003, HalperinMischenkoPRL2004, NagaosaSugimotoPRB2006} that $\vec{j}^z_{s} \propto \dot{\sigma_y}$ for Rashba-type systems; hence, up to $O(\tfrac{1}{m})$, band bending does not give rise to a spin current for Weyl fermion systems.

In conclusion, we have analysed both the spin Hall and Rashba-Edelstein effects in a 2D Weyl electron system. Our results show that strong spin-orbit coupling in the band-structure is sufficient to cause chiral spin-flip scattering of the helical electrons off non-SOC scalar impurities, resulting in a skew-scattering contribution to the SHE. The strength of this mechanism is measured by the SHA, $\theta^{SH} = -\tfrac{\hbar}{e} \frac{\gamma_{s}}{\gamma_0 + \gamma_a} \sim -\tfrac{\hbar}{e} \; O\left( \tfrac{V^2}{v_F^2/a^2} (k_F a)^4 \right) $, and we highlight the fact that the skew scattering strength can be tuned by varying $k_F a$, thereby providing an experimentally-accessible parameter for controlling the SHE. In addition, we have also found an $O(\tfrac{1}{\gamma_t})$ Rashba-Edelstein effect due to spin-momentum locking of the Weyl fermions. We gratefully acknowledge I. Mertig, K. Kondou and Y. Tokura for helpful discussions, and this work was supported by CREST, Japan Science and Technology Agency (JST).

\bibliography{main.bib}

\end{document}

