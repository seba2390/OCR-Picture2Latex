\documentclass[letter,12pt,preprint,aps]{revtex4-1}
%\documentclass[letter,preprint,aps]{revtex4}
%\documentclass[aps,twocolumn,amsmath,amssymb,showpacs,prl,floatfix]{revtex4}
\usepackage{amsmath}
\usepackage{color}
\usepackage{graphicx}
\usepackage{caption}
\usepackage{subcaption}
\usepackage{epsfig}
\usepackage{SIunits}
\usepackage{physics}
\usepackage{bbold}
\graphicspath{ {./figs/} }

\newcommand{\Hamil}{{\cal H}}
\newcommand{\kk}{{\mathbf k}}
\newcommand{\be}{\begin{equation}}
\newcommand{\ee}{\end{equation}}
\newcommand{\ba}{\begin{eqnarray}}
\newcommand{\ea}{\end{eqnarray}}
\newcommand{\baa}{\begin{eqnarray*}}
\newcommand{\eaa}{\end{eqnarray*}}
\newcommand{\dg}{^{\dagger}}
\newcommand{\mI}{{\mathbb 1}}
\newcommand{\me}{\mathrm{e}}
\newcommand{\bl}{\left (}
\newcommand{\br}{\right )}
\newcommand{\Lim}[1]{\raisebox{0.5ex}{\scalebox{0.8}{$\displaystyle \lim_{#1}\;$}}}

\newcommand{\Vbar}{\overline{\overline{V}}}
\newcommand{\Gbar}{\overline{\overline{G}}}
\newcommand{\Tbar}{\overline{\overline{T}}}
\newcommand{\gbar}{\overline{\overline{\gamma}}}
\newcommand{\Abar}{\overline{\overline{A}}}
\newcommand{\Sbar}{\overline{\overline{\Sigma}}}
\newcommand{\Gammabar}{\overline{\overline{\Gamma}}}

\begin{document}

\centerline{\bf
Supplementary Online Material: Spin Hall Effect on Topological Insulator Surface}

\author{T.~Tzen Ong}
\affiliation{RIKEN Center for Emergent Matter Science (CEMS), Saitama 351-0198, Japan}
\affiliation{Department of Applied Physics, University of Tokyo, Tokyo 113-8656, Japan}
\date{\today}


\tableofcontents

\section{2D Weyl Fermion and Chiral Skew Scattering from Non-magnetic Impurity}

We consider elastic scattering near $E_F$ of 2D Weyl fermions (Dresselhaus-type $v_F \, \vec{k} \cdot \vec{\sigma}$ system) from a dilute ($n_i \ll 1$) random distribution of non-magnetic impurities, at positions $\vec{R}_i$, with impurity scattering $H^{imp} = \sum_{\vec{r}, \vec{R}_i} V \mathrm{e}^{-\frac{|\vec{r}-\vec{R}_i|^2}{a^2}} c\dg_{\sigma}(\vec{r})  \mI_{\sigma \sigma'} c_{\sigma'}(\vec{r})$, and the impurity size $a$ determines the strength of skew scattering. Note that the results can be easily translated into the Rashba-type $v_F \, \hat{z} \times \vec{k} \cdot \vec{\sigma}$ case by rotating the momentum by $90^{\degree}$. The chemical potential $\mu$ is chosen to lie in the upper helical band, with the upper/ lower helical Weyl fermions being $\psi_{\pm, \vec{k}} = \tfrac{1}{\sqrt{2}} (\pm \, c_{\vec{k}, \uparrow} + \me^{i \theta_k} c_{\vec{k}, \downarrow})$,  and the Hamiltonian is,
%
\ba
\label{eqn: Hamiltonian}
H & = & H^0 + H^{imp} \\
H^0 & = & \sum_{\vec{k}, \alpha, \beta} c\dg_{\vec{k}, \alpha}  v_F  \vec{k} \cdot \vec{\sigma}_{\alpha, \beta} c_{\vec{k}, \beta} - \mu \, c\dg_{\vec{k}, \alpha} c_{\vec{k}, \alpha} \cr
H^{imp} & = & \sum_{\vec{k}, \vec{k'}} c\dg_{\vec{k}, \alpha} V_{\vec{k}, \vec{k}', \alpha \beta} c_{\vec{k}', \beta} 
\ea

The non-magnetic impurity is modelled with a scattering potential $V$ and a Gaussian profile, $V \, \me^{-\frac{r^2}{a^2}}$. Hence the scattering matrix element of 2D Weyl fermions off this impurity is,
%
\ba
V_{\vec{k}, \vec{k}', \sigma \sigma'} & = & \bra{\vec{k}, \sigma} V \me^{-\frac{r^2}{a^2}} \ket{\vec{k}', \sigma'} \cr
 & = & \sum_n V_n \me^{i n (\theta_{k} - \theta_{k'})} \mI_{\sigma \sigma'}
 \ea 
%
where $V_n = \tfrac{V a^2}{8} \me^{-\frac{1}{8} k_F^2 a^2} k_F a \Big( I(\tfrac{n-1}{2}, \tfrac{k_F^2 a^2}{8}) - I(\tfrac{n+1}{2}, \tfrac{k_F^2 a^2}{8}) \Big) \approx \tfrac{V a^2}{2} \frac{(k_F a)^n}{2^n \Gamma(\tfrac{n+1}{2})}$. We have assumed that transport involves mainly the quasi-particles near $E_F$, i.e. $|\vec{k}| = |\vec{k}'| \approx k_F$, and have used the result $\int_{0}^{\infty} r dr J_n(k_F r) \me^{-\tfrac{r^2}{a^2}} =  \tfrac{a^2}{8}  k_F a \, \me^{-\frac{1}{8} k_F^2 a^2} \Big( I(\tfrac{n-1}{2}, \tfrac{k_F^2 a^2}{8}) - I(\tfrac{n+1}{2}, \tfrac{k_F^2 a^2}{8}) \Big)$, with $J(n,z)$ and $I(n,z)$ being the Bessel and modified Bessel functions of the first kind respectively, and $\Gamma(n)$ is the Gamma function

All the scattering events from a single impurity are captured in the $\Tbar$-matrix, given by the Dyson equation $\hat{\Tbar} = \hat{\Vbar} + \hat{\Vbar} \, \hat{\Gbar}_{0} \, \hat{\Tbar}$. Making use of the rotational symmetry of the system, we express the Greens function and $\Tbar$-matrix in a multipole-expansion,
%
\ba
\overline{\overline{G}}_0(\vec{k}, i \omega_n) & = & \frac{1}{i \omega_n + \mu - v_F \vec{k} \cdot \vec{\sigma}} \\
& = & g^0_0(k, i \omega_n) \mI + g^1_0(k, i \omega_n)(\cos \theta_k \, \sigma^x + \sin \theta_k \, \sigma^y) \hspace{1cm} \text{, where} \cr
g^{0}_{0}(k, i \omega_n) & = & \frac{i \omega_n + \mu}{(i \omega_n + \mu)^2 - v_F^2 k^2} \cr
g^{1}_{0}(k, i \omega_n) & = & \frac{v_F k}{(i \omega_n + \mu)^2 - v_F^2 k^2} \nonumber
\ea
%
\ba
\label{eqn: T matrix}
\Tbar(\vec{k}, \vec{k}') & \equiv & \sum_{nm} T^{i}_{nm} \me^{i n \theta_k} \me^{-i m \theta_{k'}} \sigma^i \cr 
 & = & \Vbar({\vec{k}, \vec{k}'})  + \sum_{n_1 n_2 n_3}  \int \frac{d \theta_{k_1}}{2 \pi} \int \frac{k_1 d k_1}{2 \pi} V_{n_1} \me^{i n_1 (\theta_{k} - \theta_{k_1})} \cr
 & & \times \left[ g^0_0(k_1, i \omega_n) \mI + g^1_0(k_1, i \omega_n)(\cos \theta_{k_1} \, \sigma^x + \sin \theta_{k_1} \, \sigma^y) \right] \cr
 & & \times T^{j}_{n_2 n_3}(k_1, k') \me^{i n_2 \theta_{k_1}} \me^{-i n_3 \theta_{k'}} \sigma^j
 \ea 
%
The Pauli matrices are defined as $\sigma^i \in [\mI, \vec{\sigma}]$. As discussed in the main paper, we shall assume that there are no resonances, so the $\Tbar$-matrix varies slowly as a function of $\vec{k}$ near $E_F$. Approximating the $\Tbar$-matrix as a constant near $k_F$, the $\int dk_1$-integral is carried out only over the Green's function. This is the momentum-averaged retarded Green's function, $\langle g^{i,(R, A)}_0(i \omega_n) \rangle \equiv \int \frac{k dk}{2 \pi} g^{i,(R, A)}(k, i \omega_n) $, and the results are,
%
\begin{subequations}
\ba
\label{eqn: momentum avg Greens func}
\langle g^{0,(R, A)}_0(E_F) \rangle & = &  \mp \frac{i \pi}{2} N_0(E_F) sgn(E_F)  \\
\langle g^{1,(R, A)}_0(E_F) \rangle & = & \pm \frac{i \pi}{2} N_0(E_F) sgn(E_F)   
\ea
\end{subequations}

Here, $N_0(E_F) = \tfrac{E_F}{2 \pi v_F^2}$ is the bare density of states, and in terms of the momentum-averaged retarded Greens functions, the retarded $\Tbar$-matrix is now given by,

\ba
\label{eqn: T matrix Dyson eqn}
\Tbar(\vec{k}, \vec{k}')  & = & \sum_{n m} V_n \me^{i n (\theta_{k} - \theta_{k'})} \mI \delta_{nm} + V_n \me^{i n \theta_k} \me^{- i m \theta_{k'}} \Big[ \langle g^0_0(E_F) \rangle \big( T^{0}_{n m} \mI + T^{1}_{n m} \sigma^x + T^{2}_{n m} \sigma^y + T^{3}_{n m} \sigma^z \big) \\
 & + & \langle g^1_0(E_F) \rangle \big( T^{-}_{n-1 m} \mI + T^{-}_{n-1 m} \sigma^z + T^{z+}_{n-1 m} \sigma^- \big) + \langle g^1_0(E_F) \rangle \big( T^{+}_{n+1 m} \mI + T^{+}_{n+1 m} \sigma^z + T^{z-}_{n+1 m} \sigma^+ \big) \Big] \nonumber
\ea

The coefficients of the $\Tbar$-matrix are $T^{z\pm}_{nm} \equiv T^{0}_{nm} \pm T^{3}_{nm}$, $T^{\pm}_{nm} \equiv \tfrac{1}{2} \bl T^{1}_{nm} \pm i T^{2}_{nm} \br$, and are now defined by the following set of coupled recurrence equations,
%
\ba
\label{eqn: T-reccurrence eqns}
T^{z+}_{nm} & = & V_n \delta_{nm}  + V_n \langle g^0(E_F) \rangle T^{z+}_{nm} + 2 V_{n} \langle g^1(E_F) \rangle T^{+}_{n+1 m} \cr
T^{+}_{nm} & = & V_n \langle g^0(E_F) \rangle T^{+}_{nm} + \frac{1}{2} V_{n} \langle g^1(E_F) \rangle T^{z+}_{n-1 m} \cr
T^{z-}_{nm} & = & V_n \delta_{nm}  + V_n \langle g^0(E_F) \rangle T^{z-}_{nm} + 2 V_{n} \langle g^1(E_F) \rangle T^{-}_{n-1 m} \cr
T^{-}_{nm} & = & V_n \langle g^0(E_F) \rangle T^{-}_{nm} + \frac{1}{2} V_{n} \langle g^1(E_F) \rangle T^{z-}_{n+1 m}
\ea  

The $\Tbar$-coefficients reduce to two set of coupled equations for $T^{z\pm} = T^0_{nm} \pm T^3_{nm}$ and $T^{\pm} = T^1_{nm} \pm i T^2_{nm}$, given in terms of $V_n$ and the momentum-averaged retarded Green's functions, $\langle g^{i, (R)}(E_F) \rangle$. The arguments of the $\Tbar$-matrix coefficients are dropped, understanding that they are evaluated at $k_F$ and $E_F$. Some straightforward, albeit tedious, algebra allows us to solve Eq.~\ref{eqn: T-reccurrence eqns}.
%
\ba
T^{z+}_{nm} & = & \frac{V_n \bl 1 - V_{n+1} \langle g^0(E_F \rangle) \br \delta_{nm}}{\bl 1 - V_n \langle g^0(E_F) \rangle \br \bl 1 - V_{n+1} \langle g^0(E_F) \rangle \br - V_n V_{n+1} \langle g^1(E_F) \rangle^2} \cr
T^{+}_{nm} & = & \frac{1}{2} \frac{V_n V_{n-1} \langle g^1(E_F \rangle) \delta_{n-1 m} }{\bl 1 - V_{n-1} \langle g^0(E_F) \rangle \br \bl 1 - V_{n} \langle g^0(E_F) \rangle \br - V_n V_{n-1} \langle g^1(E_F) \rangle^2} \cr
T^{z-}_{nm} & = & \frac{V_n \bl 1 - V_{n-1} \langle g^0(E_F \rangle) \br \delta_{nm} }{\bl 1 - V_n \langle g^0(E_F) \rangle \br \bl 1 - V_{n-1} \langle g^0(E_F) \rangle \br - V_n V_{n-1} \langle g^1(E_F) \rangle^2} \cr
T^{-}_{nm} & = & \frac{1}{2} \frac{V_n V_{n+1} \langle g^1(E_F \rangle) \delta_{n+1 m} }{\bl 1 - V_{n+1} \langle g^0(E_F) \rangle \br \bl 1 - V_{n} \langle g^0(E_F) \rangle \br - V_n V_{n+1} \langle g^1(E_F) \rangle^2}
\ea
%
Therefore, the $\Tbar$-matrix coefficients are,
%
\ba
T^{0}_{nm} & = & \frac{1}{2} \frac{V_n \bl 1 - V_{n+1} \langle g^0(E_F \rangle) \br \delta_{nm}}{\bl 1 - V_n \langle g^0(E_F) \rangle \br \bl 1 - V_{n+1} \langle g^0(E_F) \rangle \br - V_n V_{n+1} \langle g^1(E_F) \rangle^2} \cr
 & + & \frac{1}{2} \frac{V_n \bl 1 - V_{n-1} \langle g^0(E_F \rangle) \br \delta_{nm} }{\bl 1 - V_n \langle g^0(E_F) \rangle \br \bl 1 - V_{n-1} \langle g^0(E_F) \rangle \br - V_n V_{n-1} \langle g^1(E_F) \rangle^2}\cr
T^{3}_{nm} & = & \frac{1}{2} \frac{V_n \bl 1 - V_{n+1} \langle g^0(E_F \rangle) \br \delta_{nm}}{\bl 1 - V_n \langle g^0(E_F) \rangle \br \bl 1 - V_{n+1} \langle g^0(E_F) \rangle \br - V_n V_{n+1} \langle g^1(E_F) \rangle^2} \cr
& - &  \frac{1}{2} \frac{V_n \bl 1 - V_{n-1} \langle g^0(E_F \rangle) \br \delta_{nm} }{\bl 1 - V_n \langle g^0(E_F) \rangle \br \bl 1 - V_{n-1} \langle g^0(E_F) \rangle \br - V_n V_{n-1} \langle g^1(E_F) \rangle^2}\cr
T^{1}_{nm} & = & \frac{1}{2} \frac{V_n V_{n-1} \langle g^1(E_F \rangle) \delta_{n-1 m} }{\bl 1 - V_{n-1} \langle g^0(E_F) \rangle \br \bl 1 - V_{n} \langle g^0(E_F) \rangle \br - V_n V_{n-1} \langle g^1(E_F) \rangle^2} \cr
& + & \frac{1}{2} \frac{V_n V_{n+1} \langle g^1(E_F \rangle) \delta_{n+1 m} }{\bl 1 - V_{n+1} \langle g^0(E_F) \rangle \br \bl 1 - V_{n} \langle g^0(E_F) \rangle \br - V_n V_{n+1} \langle g^1(E_F) \rangle^2} \cr
T^{2}_{nm} & = & -\frac{i}{2} \frac{V_n V_{n-1} \langle g^1(E_F \rangle) \delta_{n-1 m} }{\bl 1 - V_{n-1} \langle g^0(E_F) \rangle \br \bl 1 - V_{n} \langle g^0(E_F) \rangle \br - V_n V_{n-1} \langle g^1(E_F) \rangle^2} \cr
& + & \frac{i}{2} \frac{V_n V_{n+1} \langle g^1(E_F \rangle) \delta_{n+1 m} }{\bl 1 - V_{n+1} \langle g^0(E_F) \rangle \br \bl 1 - V_{n} \langle g^0(E_F) \rangle \br - V_n V_{n+1} \langle g^1(E_F) \rangle^2}
\ea
%
We calculate the $\Tbar$-matrix up to order $O(V_0 V_1^2)$, at which skew scattering appears, and keep only the $l = 0$ and $l = 1$ channels. Defining the symmetric and asymmetric parts of the spin-flip scattering as $T^{S/A} = T^{+}_{10} \pm T^{-}_{-10}$, we can now write down the $s$ and $p$-wave channels of the $\Tbar$-matrix.
%
\ba
\label{eqn: T-matrix}
\Tbar(\theta_k, \theta_{k'}) & = & T^0 \mI + T^3_0 \sigma^z+ T^3_1 \big( \me^{i (\theta_k - \theta_{k'})} - \me^{-i (\theta_k - \theta_{k'})}\big) \sigma^z \cr
 & + & \frac{T^S + T^A}{2} \me^{i \theta_{k} } \sigma^{-} + \frac{T^S - T^A}{2} \me^{-i \theta_{k}} \sigma^{+}  \cr
 & + & \frac{T^S + T^A}{2} \me^{-i \theta_{k'} } \sigma^{+} + \frac{T^S - T^A}{2} \me^{i \theta_{k'}} \sigma^{-}
\ea
%
and the coefficients are defined as,
%
\begin{subequations}
\ba
\label{eqn: T-coefficients}
T^0 & = & \frac{1}{2} \frac{V_0 \big( 1- V_1 \langle g^0(E_F) \rangle \big) }{\Big[ \bl 1 - V_0 \langle g^0(E_F) \rangle \br \bl 1 - V_1 \langle g^0(E_F) \rangle \br - V_0 V_{1} \langle g^1(E_F) \rangle^2 \Big]} \cr
 & & + \frac{1}{2} \frac{V_0 \big( 1- V_{-1} \langle g^0(E_F) \rangle \big) }{\Big[ \bl 1 - V_0 \langle g^0(E_F) \rangle \br \bl 1 - V_{-1} \langle g^0(E_F) \rangle \br - V_0 V_{-1} \langle g^1(E_F) \rangle^2 \Big]} \cr
 & = & \frac{V_0}{\Big[ 1 - V_0 \langle g^0(E_F)\rangle \Big]^2} \\
T^3_0 & = & \frac{1}{2} \frac{V_0 \big( 1- V_1 \langle g^0(E_F) \rangle \big) }{\Big[ \bl 1 - V_0 \langle g^0(E_F) \rangle \br \bl 1 - V_1 \langle g^0(E_F) \rangle \br - V_0 V_{1} \langle g^1(E_F) \rangle^2 \Big]} \cr
 & & - \frac{1}{2} \frac{V_0 \big( 1- V_{-1} \langle g^0(E_F) \rangle \big) }{\Big[ \bl 1 - V_0 \langle g^0(E_F) \rangle \br \bl 1 - V_{1} \langle g^0(E_F) \rangle \br - V_0 V_{-1} \langle g^1(E_F) \rangle^2 \Big]} \cr
 & = & \frac{V_0^2 V_1 \langle g^{1}(E_F) \rangle^2}{\Big[ 1 - V_0 \langle g^0(E_F)\rangle  \Big]^2} \\
 T^3_1 & = & \frac{1}{2} \frac{V_1 \big( 1- V_2 \langle g^0(E_F) \rangle \big) }{\Big[ \bl 1 - V_1 \langle g^0(E_F) \rangle \br \bl 1 - V_2 \langle g^0(E_F) \rangle \br - V_1 V_{2} \langle g^1(E_F) \rangle^2 \Big]} \cr
 & & - \frac{1}{2} \frac{V_1 \big( 1- V_{0} \langle g^0(E_F) \rangle \big) }{\Big[ \bl 1 - V_1 \langle g^0(E_F) \rangle \br \bl 1 - V_{0} \langle g^0(E_F) \rangle \br - V_1 V_{0} \langle g^1(E_F) \rangle^2 \Big]} \cr
 & = & -\frac{1}{2} \frac{V_0 V_1^2 \langle g^{1}(E_F) \rangle^2}{\Big[ 1 - V_1 \langle g^0(E_F)\rangle  \Big]^2} \\
T^S & = & \frac{1}{2} \frac{V_0 V_1 \langle g^1(E_F) \rangle}{(1 - V_0 \langle g^0(E_F) \rangle )(1 - V_1 \langle g^0(E_F) \rangle ) - V_0 V_1 \langle g^1(E_F) \rangle^2} \cr
& & + \frac{1}{2} \frac{V_0 V_{-1} \langle g^1(E_F) \rangle}{(1 - V_0 \langle g^0(E_F) \rangle )(1 - V_{-1} \langle g^0(E_F) \rangle ) - V_0 V_{-1} \langle g^1(E_F) \rangle^2} \cr
& = & \frac{V_0 V_1^2 \langle g^0(E_F) \rangle \langle g^1(E_F) \rangle }{\Big[ 1 - V_0 \langle g^0(E_F) \rangle \Big]^2} \\
T^A & = & \frac{1}{2} \frac{V_0 V_1 \langle g^1(E_F) \rangle}{(1 - V_0 \langle g^0(E_F) \rangle )(1 - V_1 \langle g^0(E_F) \rangle ) - V_0 V_1 \langle g^1(E_F) \rangle^2} \cr
& & - \frac{1}{2} \frac{V_0 V_{-1} \langle g^1(E_F) \rangle}{(1 - V_0 \langle g^0(E_F) \rangle )(1 - V_{-1} \langle g^0(E_F) \rangle ) - V_0 V_{-1} \langle g^1(E_F) \rangle^2} \cr
& = & \frac{V_0 V_1 \langle g^1(E_F) \rangle }{\Big[ 1 - V_0 \langle g^0(E_F) \rangle \Big]^2}
\ea
\end{subequations}

We point out that upon projecting into the upper helical band, i.e. calculating the matrix elements $\bra{\vec{k}, +} T^S (\me^{i \theta_{k}} \sigma^- + \me^{-i \theta_{k}'} \sigma^+) \ket{\vec{k}', +} = 2 \, T^S \left(  \cos(\theta_{k} - \theta_{k}') -i \sin(\theta_{k} - \theta_{k}') \right)$, we find that the spin-flip scattering gives rise to a skew-scattering term $2 i T^S \sin(\theta_{k} - \theta_{k}')$ in the chiral band basis, which will drive the SHE. 

\section{Effective Greens Function and Quasi-particle Scattering Rate}

The retarded $\Tbar$-matrix calculated in Eq.~\ref{eqn: T matrix} includes only scattering from a single impurity, and in the dilute impurity limit, the $\Tbar$-matrix for scattering from all impurities can be calculated in the non-crossing approximation NCA)~\cite{Rammer2004}) by including scattering events from other impurities in the bare Greens function leg, i.e. replacing $\Gbar_0$ by $\Gbar_{eff}$, in the calculation of the $\Tbar$-matrix. Hence, this forms an implicit self-consistent solution for the retarded and advanced $\Gbar_{eff}$ function and $\Tbar$-matrix.
%
\ba
\label{eqn: dilute limit T matrix}
\Tbar^{(R)}(\vec{k}, \vec{k}') & = & n_i \Vbar(\vec{k}, \vec{k}') + n_i \sum_{\vec{k}_1} \Vbar(\vec{k}, \vec{k}_1) \Gbar^{(R)}_{eff}(\vec{k}_1, \omega) \Tbar^{(R)}(\vec{k}_1, \vec{k}',\omega) \cr
\Tbar^{(A)}(\vec{k}, \vec{k}') & = & n_i \Vbar(\vec{k}, \vec{k}') + n_i \sum_{\vec{k}_1} \Vbar(\vec{k}, \vec{k}_1) \Gbar^{(A)}_{eff}(\vec{k}_1, \omega) \Tbar^{(A)}(\vec{k}_1, \vec{k}',\omega)
\ea
%
In the non-crossing approximation, the retarded self-energy $\overline{\overline{\Sigma}}^{(R)}(\vec{k}, \omega)$ and quasi-particle scattering rate $\gbar(\vec{k}, \omega) = Im[\overline{\overline{\Sigma}}^{(R)}(\vec{k}, \omega)]$ are given by,
%
\ba
\label{eqn: Self energy}
\overline{\overline{\Sigma}}^{(R)}(\vec{k}, \omega) & = & n_i \sum_{\vec{k}_1} \Vbar(\vec{k}, \vec{k}_1) \Gbar^{(R)}_{eff}(\vec{k}_1, \omega) \Tbar^{(R)}(\vec{k}_1, \vec{k}',\omega) \cr
\gbar(\vec{k}, \omega) = Im[\overline{\overline{\Sigma}}^{(R)}(\vec{k}, \omega)] & = & \sum_{\vec{k}_1} \Tbar^{(A)}(\vec{k}, \vec{k}_1, \omega) \overline{\overline{A}}_{eff}(\vec{k}_1, \omega) \Tbar^{(R)}(\vec{k}_1, \vec{k}, \omega)
\ea
%
The spin-dependent spectral weight is given by $\overline{\overline{A}}_{eff}(\vec{k}, \omega) = 2 Im[\Gbar_{eff}^{(R)}(\vec{k}, \omega)]$. Similar to the calculation of the $\Tbar$-matrix, the $\int dk$-integral for the self-energy is done using the approximation that the $\Tbar$-matrix varies slowly near $k_F$, leaving only the $\int dk$-integral of the spin-dependent spectral weight, which is none other than the density of states,
%
\ba
\label{eqn: DOS}
N_{eff}^{(0)}(\omega) & = & \int \frac{k dk}{2 \pi} Im[g_{eff}^{0}(\vec{k}, \omega)]  \cr
N_{eff}^{(1)}(\omega) & = & \int \frac{k dk}{2 \pi} Im[g_{eff}^{1}(\vec{k}, \omega)]
\ea

As pointed out in the main paper, $N^{(0)/(1)}_{eff}(E_F) = \tfrac{N_0(E_F)}{2} (1 + O(\gbar))$ in the dilute limit; hence, we will approximate $N^{(0)/(1)}_{eff}(E_F) \approx \tfrac{N_0(E_F)}{2} = \tfrac{|E^2_F|}{4 \pi v_F^2}$ and $N^{(1)}_{eff}(E_F) \approx \tfrac{N_0(E_F)}{2} sgn(E_F)$. This finally gives the result for the quasi-particle lifetime near the Fermi surface, i.e. $\gbar = \gbar(k_F, E_F) = Im[\overline{\overline{\Sigma}}^{(R)}(k_F, E_F)]$, which is shown below. The real part of the self-energy that renormalizes $v_F$ and $\mu$ are ignored here, as $v_F$ and $\mu$ are taken to be experimentally determined parameters.
%
\begin{subequations}
\ba
\label{eqn: qp scattering rate}
\gbar & = & \gamma_0 \mI + \gamma_a \bl \cos \theta \, \sigma^x + \sin \theta \, \sigma^y \br \cr
& - & \, \gamma_b (\sin \theta \, \sigma^x - \cos \theta \, \sigma^y) + i \gamma_3 \, \sigma^z \\
\label{eqn: gamma_0}
 \gamma_0 & = & n_i N^{(0)}_{eff}(E_F) \left[ |T^0|^2 -  2\left( |T^3|^2 + |T^S|^2 - |T^A|^2 \right) \right]  \cr
  \label{eqn: gamma_a}
 \gamma_a & = & -4 n_i N^{(1)}_{eff}(E_F) \left[ |T^S|^2 + |T^A|^2 \right]   \cr
 \label{eqn: gamma_b}
 \gamma_b & = & 2 n_i N^{(0)}_{eff}(E_F) \left[ |T^0| |T^A| + |T^3| |T^S| \right]  \cr
 \label{eqn: gamma_3}
 \gamma_3 & = & 4 n_i N^{(1)}_{eff}(E_F)  \left[ |T^0| |T^S| + |T^3| |T^A| \right] 
\ea
\end{subequations}

The effective Greens function in the dilute impurity limit is now given by,
%
\begin{subequations}
\ba
\label{eqn: eff Greens func}
\Gbar^{(R)}_{eff}(\vec{k}, \omega) & = & \left[ \omega + \mu - v_F \vec{k} \cdot \vec{\sigma} - i \gbar(\vec{k}, \omega) \right]^{-1} \cr
& = & g^0_{eff}(k ,\omega) \mI + g^a_{eff}(k, \omega) \bl \cos \theta \, \sigma^x + \sin \theta \, \sigma^y \br \cr
 & + & g^b_{eff}(k, \omega) \bl \sin \theta \, \sigma^x - \cos \theta \, \sigma^y \br + g^3_{eff}(k, \omega) \sigma^z \\
  \label{eqn: G coeffs}
 g^0_{eff}(k, \omega) & = & \frac{(\Omega(k) + i \kappa(k)) (\omega + \mu - i \gamma_0 )}{\Omega^2(k) + \kappa^2(k)} \cr
 g^a_{eff}(k, \omega) & = & \frac{(\Omega(k) + i \kappa(k)) (v_F |\vec{k}| + i \gamma_a )}{\Omega^2(k) + \kappa^2(k)} \cr
 g^b_{eff}(k, \omega) & = & \frac{i \gamma_b (\Omega(k) + i \kappa(k)) }{\Omega^2(k) + \kappa^2(k)} \cr
 g^3_{eff}(k, \omega) & = & -\frac{\gamma_3 (\Omega(k) + i \kappa(k)) }{\Omega^2(k) + \kappa^2(k)}
\ea
\end{subequations}
%
where the denominator terms are $\Omega(k) = (\omega + \mu)^2 - v_F^2 |\vec{k}|^2 - \gamma_0^2 + \gamma_a^2 + \gamma_b^2 - \gamma_3^2$, $\kappa(k) = 2 \bl (\omega + \mu) \gamma_0 + v_F |\vec{k}| \gamma_a \br$. 

\section{SHE \& Rashba Edelstein Effect Correlation Functions}

Within the Kubo formalism, the longitudinal charge conductivity and spin-Hall conductivity, $\sigma_{yy}$ and $\sigma^{z}_{xy}$, are given by the retarded current-current and spin current-current correlation functions respectively,
%
\begin{subequations}
\ba
\pi^{(R)}_{yy}(\vec{k}, \omega) & = & -i \int_{- \infty}^{\infty} dt \, \me^{i \omega t} \theta(t) \, \langle [j_y(\vec{k}, t), j_y(\vec{k},0]) \rangle \\
\pi^{z,(R)}_{xy}(\vec{k}, \omega) & = & -i \int_{- \infty}^{\infty} dt \, \me^{i \omega t} \theta(t) \, \langle [\mathcal{J}^z_x(\vec{k}, t), j_y(\vec{k},0]) \rangle
\ea
\end{subequations}
%
Similarly, it is straightforward to derive a Kubo formula for the spin-accumulation due longitudinal charge transport, i.e. the Rashba-Edelstein effect.
%
\begin{subequations}
\ba
\langle \vec{S} \rangle & = & \Lim{\omega \rightarrow 0} \Lim{\vec{k} \rightarrow 0} \frac{E_{\alpha}}{\omega} \me^{i (\vec{k} \cdot \vec{r} - \omega t)} \int_{-\infty}^{\infty} d t' \theta(t') \langle [\vec{S}(\vec{k}, t'), j_{\alpha}(\vec{k}, 0)] \rangle \\
\pi^{i,(R)}_{\alpha}(\vec{k}, \omega) & = & -i \int_{-\infty}^{\infty} d t' \me^{i \omega t'}  \langle [S^i(\vec{k}, t'), j_{\alpha}(\vec{k}, 0)] \rangle
\ea
\end{subequations}

The spin current $\mathcal{J}^z_x$ has two components, one is the conventional spin current $j^z_x$ due to band-bending effects, and the other is the spin-torque current $P^z_x$, which are defined as follow,
%
\begin{subequations}
\ba
\label{eqn: spin current defns}
j^z_x(\vec{k}, \tau) & = & \sum_{\vec{k}_1} c\dg_{\vec{k}_1, \sigma}(\tau) \frac{(\vec{k} + \vec{k}_1)_x}{m} \sigma^z_{\sigma \sigma'} c_{\vec{k} + \vec{k}_1, \sigma'}(\tau) \\
P^z_x(\vec{k}, \tau) & = & \frac{2 i v_F}{k_x} \sum_{\vec{k}_1} c\dg_{\vec{k}_1, \sigma}(\tau) \Big( (\vec{k}_1 + \frac{\vec{k}}{2})_x \sigma^y - (\vec{k}_1 + \frac{\vec{k}}{2})_y \sigma^x\Big)_{\sigma \sigma'} c_{\vec{k} + \vec{k}_1, \sigma'}(\tau)
\ea
\end{subequations}
%
We will now separate the SHE into two contributions, $\pi^{z(1)}_{xy}$ and $\pi^{z(2)}_{xy}$, coming from the conventional spin current and the spin torque current respectively. All the Matsubara correlation functions, $\pi_{yy}(\vec{k}, i \omega_n)$, $\pi^{i}_{y}(\vec{k}, i \omega_n)$, $\pi^{z(1)}_{xy}(\vec{k}, i \omega_n)$ and $\pi^{z(2)}_{xy}(\vec{k}, i \omega_n)$, are given below, and analytic continuation ($i \omega_n \rightarrow \omega + i \eta$) will give the corresponding retarded correlation functions. 
%
\begin{subequations}
\label{eqn: Matsubara corr funcs}
\ba
\pi_{yy} (\vec{k}, i \omega_n) = - \int_0^{\beta} d \tau \me^{- i \omega_n \tau} \langle T_{\tau} U(\beta, 0) j_y(\vec{k}, \tau) j_y(\vec{k}, 0) \rangle \\
\pi^i_{y} (\vec{k}, i \omega_n) = - \int_0^{\beta} d \tau \me^{- i \omega_n \tau} \langle T_{\tau} U(\beta, 0) S^i(\vec{k}, \tau) j_y(\vec{k}, 0) \rangle \\
\pi^{z, (1)}_{xy} (\vec{k}, i \omega_n) = - \int_0^{\beta} d \tau \me^{- i \omega_n \tau} \langle T_{\tau} U(\beta, 0) j^z_x(\vec{k}, \tau) j_y(\vec{k}, 0) \rangle \\
\pi^{z, (2)}_{xy} (\vec{k}, i \omega_n) = - \int_0^{\beta} d \tau \me^{- i \omega_n \tau} \langle T_{\tau} U(\beta, 0) P^z_x(\vec{k}, \tau) j_y(\vec{k}, 0) \rangle
\ea
\end{subequations}

The correlation functions are written in the interaction representation, and $U(\beta, 0)$ is the S-matrix, which can be formally expanded as an infinite series of interacting terms involving $H^{int}$. Hence, the correlation functions are evaluated by expanding the S-matrix, and we show the expansion for $\pi^{z, (1)}_{xy}(\vec{k}, \tau)$ below.
%
\ba
\label{eqn: S matrix expansion}
\pi^{z,(1)}_{xy} (\vec{k}, \tau) & = & - \sum_{n=0}^{\infty} \frac{(-1)^n}{n!} \int_{0}^{\beta} d \tau_1 \ldots \int_{0}^{\beta} d \tau_n \langle T_{\tau} j^z_x(\vec{k}, \tau) H^{int}(\tau_1) \ldots H^{int}(\tau_n) j_y(\vec{k}, 0) \rangle
\ea

The $n = 0$ term in Eq.~\ref{eqn: S matrix expansion} is just the bare bubble diagram, and the $n = 2$ term will give the first correction to the scattering vertex.
%
\ba
\label{eqn: 1st order vertex correction}
\pi^{z, (1, n=2)}_{xy} (\vec{k}, i \omega_n) & = & - \int_{0}^{\beta} d \tau \int_{0}^{\beta} d \tau_1 \int_{0}^{\beta} d \tau_2 \me^{- i \omega_n \tau} \sum_{\vec{k}_1, \vec{k}_2} \frac{e v_F}{c} \langle T_{\tau} c\dg_{\vec{k}_1, \sigma}(\tau) \frac{(\vec{k} + \vec{k}_{1})_x}{m} \sigma^z_{\sigma \sigma'}  c_{\vec{k} + \vec{k}_1, \sigma'}(\tau) \cr
 & & \times H^{int}(\tau_1) H^{int}(\tau_2) c\dg_{\vec{k}_2, \nu}(0) \sigma^{y}_{\nu \nu'} c_{\vec{k} + \vec{k}_2, \nu'} \rangle \cr
& = & -\frac{e v_F}{m c} \sum_{\vec{p}, \vec{q}} \frac{1}{\beta} \sum_{i \omega_1} \sigma^z_{\sigma \sigma'} G_{\sigma' \mu_1}(\vec{p} + \vec{k}, i \omega_1 + i \omega_n) V_{\mu_1 \mu_2}(\vec{p} + \vec{k}, \vec{p} + \vec{q}) \cr
 & & \times G_{\mu_2 \nu}(\vec{p} + \vec{q}, i \omega_1 + i \omega_n) \sigma^{y}_{\nu \nu'} G_{\nu' \mu_3}(\vec{p} + \vec{q} - \vec{k}, i \omega_1) \cr
 & & \times V_{\mu_3 \mu_4}(\vec{p} + \vec{q} - \vec{k}, \vec{p}) G_{\mu_4 \sigma}(\vec{p}, i \omega_1) (\vec{k}_1 + \vec{k})_x
\ea
%
This corresponds to the Feynman diagram for the vertex correction from a single scattering event. Notice that only elastic scattering is considered here, as each scattering event does not change the energy of the electron; hence, all the Green's functions on the upper (and lower) legs of the bubble diagram have the same energy, e.g. in Eq.~\ref{eqn: 1st order vertex correction}, $G_{\sigma' \mu_1}(\vec{p} + \vec{k}, i \omega_1 + i \omega_n)$ and $G_{\mu_2 \nu}(\vec{p} + \vec{q}, i \omega_1 + i \omega_n)$ undergo a change of momentum and spin upon scattering off $V_{\mu_1 \mu_2}(\vec{p} + \vec{k}, \vec{p} + \vec{q})$, but do not exchange energy with the impurity.

Since energy is conserved in the upper and lower legs of the bubble diagram, we can now include the effect of all the scattering events from a single impurity on the vertex correction by replacing the scattering potential $V_{\mu_1 \mu_2}(\vec{k}, \vec{k}')$ by the full $\Tbar$-matrix to obtain,
%
\ba
\label{eqn: scattering vertex with T matrix}
\pi^{z, (1, T)}_{xy} (\vec{k}, i \omega_n) & = & -\frac{e v_F}{c} \sum_{\vec{p}, \vec{q}} \frac{1}{\beta} \sum_{i \omega_1}  \frac{(\vec{p} + \vec{k})_x}{m}
\Tr \Big[ \sigma^z \Gbar(\vec{p} + \vec{k}, i \omega_1 + i \omega_n) \Tbar(\vec{p} + \vec{k}, \vec{p} + \vec{q}) \cr
 & & \times \Gbar(\vec{p} + \vec{q}, i \omega_1 + i \omega_n) \sigma^{y} \Gbar(\vec{p} + \vec{q} - \vec{k}, i \omega_1) \Tbar(\vec{p} + \vec{q} - \vec{k}, \vec{p}) \Gbar(\vec{p}, i \omega_1) \Big] 
 \ea

Finally, scattering events from all the impurities can be included by defining a scattering vertex $\Gammabar^{y}(\vec{p} + \vec{k}, \vec{k}, i \omega_1 + i \omega_n, i \omega_n)$, whereby an infinite subset of scattering events are included in the Bethe-Salpeter equation,
%
\ba
\label{eqn: vertex Dyson eqn}
\Gammabar^{y}(\vec{p} + \vec{k}, \vec{k}, i \omega_1 + i \omega_n, i \omega_n) & = & \sigma^y + \sum_{\vec{q}} \Tbar(\vec{p} + \vec{k}, \vec{p} + \vec{q}, i \omega_1 + i \omega_n) \Gbar_{eff}(\vec{p}+\vec{q}, i \omega_1 + i \omega_n) \cr
& & \times \Gammabar^{y}(\vec{p} + \vec{q}, \vec{q}, i \omega_1 + i \omega_n, i \omega_n) \Gbar_{eff}(\vec{q}, i \omega_n) \Tbar(\vec{q}, \vec{k}, i \omega_n)
\ea
%
and the full correlation function is therefore,
% 
\ba
\label{eqn: full scattering vertex with Gammabar}
\pi^{z, (1)}_{xy} (\vec{k}, i \omega_n) & = & -\frac{e v_F}{c} \sum_{\vec{p}} \frac{1}{\beta} \sum_{i \omega_1} \frac{(\vec{p} + \vec{k})_x}{m} \cr
 & & \times \Tr \Big[  \Gbar(\vec{p}, i \omega_1) \sigma^z  \Gbar(\vec{p} + \vec{k}, i \omega_1 + i \omega_n) \Gammabar^{y}(\vec{p} + \vec{k}, \vec{p}, i \omega_1 + i \omega_n, i \omega_1) \Big]  
 \ea

This infinite subset of ladder diagrams includes all the scattering corrections to the vertex from all the impurities, but does not include diagrams where scattering events from different impurities cross each other, i.e. this is the non-crossing approximation, which is reasonable in the dilute impurity limit.

Now let us evaluate the uniform limit of the Matsubara correlation function, $\underset{\vec{k} \to 0}{\lim} \, \pi^{z, (1)}_{xy}(\vec{k}, i \omega_n)$, by first doing the sum over the $i  \omega_1$ frequencies using the standard method of integrating over the poles of $n_F(z) = (\me^{\beta z} + 1)^{-1}$ in the complex $z$-plane. The poles of $n_F(z)$ are at $z = i \tfrac{2 \pi (n + 1)}{\beta}$, with residue of $- \tfrac{1}{\beta}$, and the sum $\sum_{i \omega_1}$ is replaced by an integration over the complex plane,
%
\ba
\pi^{z, (1)}_{xy}(\vec{k} = 0, i \omega_n) & = & - \frac{e v_F}{m c} \int \frac{d z}{2 \pi i} \mathcal{P}(z, z + i \omega_n) n_F(z) \cr
\mathcal{P}(z, z + i \omega_n) & =  & \sum_{\vec{p}} \vec{p}_x \Tr \Big[ \Gbar(\vec{p}, z) \sigma^z \Gbar(\vec{p}, z + i \omega_n) \Gammabar^{y}(\vec{p}, \vec{p}, z, z + i \omega_n) \Big]
\ea
%
The integral over the complex $z$-plane will also pick up the branch cuts of the Green's function $\Gbar(\vec{p}, z)$ and $\Gbar(\vec{p}, z + i \omega_n)$, which leads to branch cuts at $z = v_F |\vec{p}| - \mu = \xi(\vec{p})$ and $z + i \omega_n = v_F |\vec{p}| - \mu = \xi(\vec{p})$, and the upper ($\epsilon + i \delta$) and lower ($\epsilon - i \delta$)  paths along the branch cuts will give the following retarded and advanced contributions to the correlation function.

\ba
\pi^{z, (1)}_{xy}(\vec{k} = 0, i \omega_n) & = & - \int \frac{d \epsilon}{2 \pi i} n_F(\epsilon) \Big[ \mathcal{P}(\epsilon + i \delta, \epsilon + i \omega_n) - \mathcal{P}(\epsilon - i \delta, \epsilon + i \omega_n) \cr
 & & + \mathcal{P}(\epsilon - i \omega_n, \epsilon + i \delta) - \mathcal{P}(\epsilon - i \omega_n, \epsilon - i \delta) \Big]
\ea
%
Therefore, the retarded correlation function is obtained by analytic continuation $i \omega_n \rightarrow \omega + i \delta$,
%
\ba
\pi^{z, (1)}_{xy}(\vec{k} = 0, \omega) & = & -\frac{e v_F}{m c} \int \frac{d \epsilon}{2 \pi i} (n_F(\epsilon) - n_F(\epsilon + \omega))  \mathcal{P}(\epsilon - i \delta, \epsilon + \omega + i \delta) \cr
 & & - n_F(\epsilon) \mathcal{P}(\epsilon + i \delta, \epsilon + \omega + i \delta) + n_F(\epsilon + \omega) \mathcal{P}(\epsilon - i \delta, \epsilon + \omega - i \delta)
\ea
%
Following the standard discussion in \cite{Mahan2000}, the most singular contribution comes from $\mathcal{P}(\epsilon - i \delta, \epsilon + \omega + i \delta)$. Since the SHE conductivity is given by $\sigma^{z}_{xy}(\omega = 0) = - \underset{\omega \to 0}{\lim} Im [ \tfrac{\pi^z_{xy}(\vec{k} =0, \omega)}{\omega} ]$, hence we will calculate the following contribution to the retarded SHE correlation function.
%
\ba
\label{eqn: retarded SHE correlation func}
\pi^{z, (1)}_{xy}(\vec{k} = 0, \omega) & = & -\frac{e v_F}{m c} \int \frac{d \epsilon}{2 \pi i} (n_F(\epsilon) - n_F(\epsilon + \omega))  \mathcal{P}(\epsilon - i \delta, \epsilon + \omega + i \delta) \cr
\sigma^{z, (1)}_{xy}(\vec{k} = 0, \omega =0) & = & -Im \Big[\frac{e v_F}{m c} \int \frac{d \epsilon}{2 \pi i} \frac{d n_F(\epsilon)}{d \epsilon}  \mathcal{P}(\epsilon - i \delta, \epsilon + i \delta) \Big] \cr
\mathcal{P}(\epsilon - i \delta, \epsilon + i \delta) & =  & \sum_{\vec{p}} \vec{p}_x \Tr \Big[ \Gbar^{(A)}(\vec{p}, \epsilon) \sigma^z \Gbar^{(R)}(\vec{p}, \epsilon) \Gammabar^{y}(\vec{p}, \vec{p}, \epsilon - i \delta, \epsilon + i \delta) \Big]
\ea

The other correlation functions for the spin-torque current contribution to the SHE ($\pi^{z, (2)}_{xy}(\vec{k}, \omega)$), the Rashba-Edelstein effect ($\pi^{i}_{y}(\vec{k}, \omega)$), and the charge current conductivity ($\pi_{yy}(\vec{k}, \omega)$) are derived in a similar manner, and we obtain,
%
\ba
\label{eqn: pizxy corr function}
\pi_{yy}(\vec{k} = 0, \omega) & = & \underset{\vec{k} \to 0}{\lim} \bl \frac{e v_F}{c} \br^2 \int_{-\infty}^{\infty} \frac{d \epsilon}{2 \pi i} \int \frac{d^2 p}{(2 \pi)^2} \bl n_F(\epsilon) - n_F(\epsilon + \omega) \br   \cr
 & & \times \Tr \left[ \Gbar^{(A)}(\vec{p}, \epsilon) \sigma^y \Gbar^{(R)}(\vec{p} + \vec{k}, \epsilon) \Gammabar^{y}(\vec{p}, \vec{k}_1 + \vec{k}, \epsilon) \right] \\
\pi^{z, (2)}_{xy}(\vec{k} = 0, \omega) & = & \pi^{z, (2a)}_{xy}(\vec{k} = 0, \omega) + \pi^{z, (2b)}_{xy}(\vec{k} = 0, \omega) \\
\pi^{z, (2a)}_{xy}(\vec{k} = 0, \omega) & = & \underset{\vec{k} \to 0}{\lim} \frac{2 i e v_F^2}{c} \int_{-\infty}^{\infty} \frac{d \epsilon}{2 \pi i} \int \frac{d^2 p}{(2 \pi)^2} \frac{p_{y} + \frac{k_y}{2}}{k_x} \bl n_F(\epsilon) - n_F(\epsilon + \omega) \br \cr
& & \times \Tr \left[ \Gbar^{(A)}(\vec{p}, \epsilon) \sigma^x \Gbar^{(R)}(\vec{p} + \vec{k}, \epsilon) \Gammabar^{y}(\vec{p}, \vec{p} + \vec{k}, \epsilon) \right] \cr
\pi^{z, (2b)}_{xy}(\vec{k} = 0, \omega) & = & \underset{\vec{k} \to 0}{\lim} \frac{2 i e v_F^2}{c} \int_{-\infty}^{\infty} \frac{d \epsilon}{2 \pi i} \int \frac{d^2 p}{(2 \pi)^2} \frac{p_x + \frac{k_x}{2}}{k_x} \bl n_F(\epsilon + \omega) - n_F(\epsilon) \br \cr
& & \times \Tr \left[ \Gbar^{(A)}(\vec{p}, \epsilon) \sigma^y \Gbar^{(R)}(\vec{p} + \vec{k}, \epsilon) \Gammabar^{y}(\vec{p}, \vec{p} + \vec{k}, \epsilon) \right] \cr
\pi^{i}_{y}(\vec{k} = 0, \omega) & = & \underset{\vec{k} \to 0}{\lim} \frac{e v_F}{c} \int_{-\infty}^{\infty} \frac{d \epsilon}{2 \pi i} \int \frac{d^2 p}{(2 \pi)^2} \bl n_F(\epsilon + \omega) - n_F(\epsilon) \br \cr
& & \times \Tr \left[ \Gbar^{(A)}(\vec{p}, \epsilon) \sigma^i \Gbar^{(R)}(\vec{p} + \vec{k}, \epsilon) \Gammabar^{y}(\vec{p}, \vec{p} + \vec{k}, \epsilon) \right] 
\ea

\section{Vertex Correction}

For four fermion correlation functions, like the current-current and spin current-current correlation functions, we have to consider the effects of impurity scattering on the scattering vertex\cite{Mahan2000}, in addition to the quasi-particle self-energy corrections. This arises from an infinite subset of Feynman ladder diagrams shown in the main paper, and is summed up in the Bethe Salpeter equation for the scattering vertex $\Gammabar^{y}(\vec{k} + \vec{p}, \vec{p}, i \omega_1 + i \omega_n, i \omega_n)$ (Eq.~\ref{eqn: vertex Dyson eqn}).
%
\ba
\label{eqn: vertex Dyson eqn}
\Gammabar^{y}(\vec{k} + \vec{p}, \vec{p}, i \omega_1 + i \omega_n, i \omega_n) & = & \sigma^y + \sum_{\vec{q}} \Tbar(\vec{k} + \vec{p}, \vec{k} + \vec{q}, i \omega_1 + i \omega_n) \Gbar_{eff}(\vec{k}+\vec{q}, i \omega_1 + i \omega_n) \cr
& & \times \Gammabar^{y}(\vec{k} + \vec{q}, \vec{q}, i \omega_1 + i \omega_n, i \omega_n) \Gbar_{eff}(\vec{q}, i \omega_n) \Tbar(\vec{q}, \vec{p}, i \omega_n)
\ea
%
Here, $\vec{k}$ and $i \omega_1$ are the  external momentum and frequency, and the DC uniform limit of the conductivities are obtained by analytic continuation of $i \omega_1 \rightarrow \omega + i \eta$, setting the limit $\vec{k} \rightarrow 0$, and then setting $\omega \rightarrow 0$, i.e. $\underset{\omega \to 0}{\lim} \; \underset{\vec{k} \to 0}{\lim} $. Hence, we only need to calculate the on-shell component of the scattering vertex $\Gammabar^{y}(\vec{p}, \omega) = \Gammabar^{y}(\vec{p}, \omega - i \eta, \omega + i \eta)$, which is defined by,
%
\ba
\Gammabar^{y}(\vec{p}, \omega) & =  & \sigma^y + \sum_{\vec{q}} \Tbar(\vec{p}, \vec{q}, \omega + i \eta) \Gbar_{eff}(\vec{q}, \omega + i \eta) \cr
& & \times \Gammabar^{y}(\vec{q}, \omega) \Gbar_{eff}(\vec{q}, \omega - i \eta) \Tbar(\vec{q}, \vec{p}, \omega - i \eta) \cr
 & =  & \sigma^y + \sum_{\vec{q}}  \Tbar^{(R)}(\vec{p}, \vec{q}, \omega) \Gbar_{eff}^{(R)}(\vec{q}, \omega) \Gammabar^{y}(\vec{q}, \omega) \Gbar_{eff}^{(A)}(\vec{q}, \omega) \Tbar^{(A)}(\vec{q}, \vec{p}, \omega)
\ea

Note that both the advanced and retarded Green's function and $\Tbar$-matrices, $\Gbar_{eff}^{(R)}(\vec{p}, \omega)$, $\Gbar_{eff}^{(A)}(\vec{p}, \omega)$, $\Tbar^{(R)}(\vec{p}, \vec{q}, \omega)$ and $\Tbar^{(A)}(\vec{p}, \vec{q}, \omega)$ enter into the Bethe-Salpeter equation due to the branch cut in the complex plane, when the integral over the complex plane is carried out. Similar to the assumption for the $\Tbar$-matrix, the scattering vertex is assumed to be momentum-independent near $E_F$, and we will do a similar multipole expansion of $\Gammabar^{y}(|\vec{p}| = k_F, \theta, \omega = E_F) = \sum_{n} \Gamma^i_n \me^{i n \theta} \sigma^i$, keeping only the $l = 0$ and $l = 1$ scattering channels. 
%
\be
\Gammabar^{y}(|\vec{p}| = k_F, \theta, \omega = E_F) =  \Gamma^i_0 \sigma^i + \left[ \Gamma^0_{p_x} \cos \theta + i \Gamma^0_{p_y} \sin \theta \right] \mI + \left[ \Gamma^i_{p_x} \cos \theta + i \Gamma^i_{p_y} \sin \theta \right] \sigma^i 
\ee
%
Hence, the Bethe-Salpeter equation is reduced to,
%
\ba
\Gammabar^{y}(\vec{p}, \omega) & = & \sigma^y + \int \frac{d \theta_q}{2 \pi} \Tbar^{(R)}(|\vec{p}| = |\vec{p} + \vec{q}| = k_F, \theta_p, \theta_{p + q}, \omega) \\
 & & \times \Bigg[ \int \frac{q dq}{2 \pi} \Gbar^{(R)}(\vec{p} + \vec{q}, \omega) \Gammabar^{y}(\vec{p} + \vec{q}, \omega) \Gbar^{(A)}(\vec{p} + \vec{q}, \omega) \Bigg] \cr
 & & \times \Tbar^{(A)}(|\vec{p} + \vec{q}| = |\vec{q}| = k_F, \theta_{p + q}, \theta_q, \omega) \cr
 \sum_{n} \Gamma^i_n \me^{i n \theta} \sigma^i & = & \sigma^y + \sum_{n_1 ... n_7} \int \frac{d \theta_q}{2 \pi} T^{i_1}_{n_1 n_2} \me^{i (n_1 \theta_k - n_2 \theta_{k+q})} T^{i_5}_{n_6 n_7} \me^{i (n_6 \theta_k - n_7 \theta_{k+q})} \sigma^{i_1} \sigma^{i_2} \sigma^{i_3} \sigma^{i_4} \sigma^{i_5}  \cr
 & & \times \Bigg[ \int \frac{q dq}{2 \pi} g_{n_3}^{i_2, (R)}(|\vec{p} + \vec{q}|, \omega) \me^{- i n_3 \theta_{p+q}}  \Gamma^{i_3}_{n_4} \me^{-i n_4 \theta_{p+q}}  g_{n_5}^{i_4, (A)}(|\vec{p} + \vec{q}|, \omega) \me^{- i n_5 \theta_{p+q}} \Bigg] \nonumber
\ea 

Since the $\Gamma^i_n$ coefficients are assumed to be invariant near $k_F$, the $\int dq$-integral is carried out over all the spin and angular momentum resolved Green's function components, $g_{m}^{i, (R)}(|\vec{p} + \vec{q}|, \omega) \, g_{n}^{j, (A)}(|\vec{p} + \vec{q}|, \omega)$. As the Weyl fermions are spin-momentum locked; hence, the spin $i$ and momentum $m$ indices are related, i.e. $m = 0$ for $i = [0,3]$, and $m = \pm 1$ for $i \in [1,2]$. We can now define, 
%
\be
\label{eqn: spectral weight integral}
\xi^{ij}(\epsilon) = \int \frac{k dk}{2 \pi} g^{i, (R)}(|\vec{k}|, \epsilon) g^{j, (A)}(|\vec{k}|, \epsilon)
\ee
%
We have carried out a change of variable from $\omega + \mu \rightarrow \epsilon$ here, thereby absorbing the factors of $\mu$ that appear in the Green's function into $\epsilon$, which is now the energy measured from $E_F$.

Knowing that $G^{(R)}(\vec{k}, \epsilon) G^{(A)}(\vec{k}, \epsilon) = \tfrac{A(\vec{k}, \epsilon)}{Im[\Sigma(\vec{k}, \epsilon)]} \approx \tfrac{A(\vec{k}, \epsilon)}{\gamma}$, this means that $\xi^{ij}(\epsilon)$ is basically the spin-resolved density of states divided by the quasi-particle scattering rate. The dominant terms are the $s$-wave, $p$-wave and $s-p$ spin-flip DOS, $\xi^{00}(\epsilon)$, $\xi^{aa}(\epsilon)$ and $\xi^{0a}(\epsilon) = (\xi^{a0}(\epsilon))^*$ respectively, which are calculated to be,
%
\ba
\xi^{00}(\epsilon) & = & \frac{1}{2 \pi v_F^2} \Big( \frac{\pi \epsilon}{2 (\gamma_0 + \gamma_a)} + \frac{\pi \gamma_0^2}{2(\gamma_0 + \gamma_a) \epsilon} \Big) \cr
\xi^{aa}(\epsilon) & = & \frac{1}{2 \pi v_F^2} \Big( \frac{\pi \epsilon}{2 (\gamma_0 + \gamma_a)} + \frac{\pi \gamma_a^2}{2(\gamma_0 + \gamma_a) \epsilon} \Big) \cr
\xi^{0a}(\epsilon) & = & \frac{1}{2 \pi v_F^2} \frac{\pi}{2 (\gamma_0 + \gamma_a)} (\epsilon - i \gamma_0) (1 - \frac{i \gamma_a}{\epsilon})
\ea

The above set of coupled equations for the $\Gammabar$-coefficients are then solved analytically, and the finite terms are shown below; and the other terms $\Gamma^0_0$, $\Gamma^3_0$, $\Gamma^1_{p_x}$, $\Gamma^1_{p_y}$, $\Gamma^2_{p_x}$ and $\Gamma^2_{p_y}$ are equal to zero. 
%
\begin{subequations}
\ba
\label{eqn: Vertex coeffs}
\Gamma^1_0(E_F) & = &  \Big[ 2 n_i (|T^3_1| |T^A| + |T^0| |T^3_0| - 2 i |T^A| |T^S|) (\xi^{00} + \xi^{aa}) \Big] \cr
 & &  \times \Bigg[ 1 - n_i \Big( |T^0|^2 + |T^3_0|^2 + 2 |T^S|^2 - 2 |T^A|^2 - 2 |T^3_1|^2 \Big) (\xi^{00} + \xi^{aa}) \cr
 &  & - 2 n_i \Big(|T^3_1|^2 - |T^3_0|^2 \Big) (\xi^{0a} + \xi^{a0}) \Bigg]^{-1} \cr
 & = & \frac{\gamma_{asym,1} + \gamma_{30}- i \gamma_{asym,3}}{\gamma_t} + O(\frac{\gamma}{E_F}) \\
 \cr
\Gamma^2_0(E_F) & = & \Bigg[ 1 - n_i \Big( |T^0|^2 + |T^3_0|^2 - 2 |T^3_1|^2 \Big) \xi^{00} - n_i \Big( |T^S|^2 + |T^A|^2\Big) \xi^{aa} + i \, n_i |T^3_1| |T^S| (\xi^{0a} + \xi^{a0}) \Bigg] \cr
 &  & \times \Bigg[ 1 - n_i \Big( |T^0|^2 + |T^3_0|^2 + 2 |T^S|^2 - 2 |T^A|^2 - 2 |T^3_1|^2 \Big) (\xi^{00} + \xi^{aa}) \cr
 &  & - 2 n_i \Big(|T^3_1|^2 - |T^3_0|^2 \Big) (\xi^{0a} + \xi^{a0}) \Bigg]^{-1} \cr
 & = & \frac{\gamma_0 + \gamma_a}{\gamma_t} + i \frac{\gamma_{31,s}}{\gamma_t} + O(\frac{\gamma}{E_F}) \\
 \cr
\Gamma^0_{p_x}(E_F) & = & n_i \Big[ \Big( |T^A|^2| + |T^S|^2 \Big) (\xi^{0a} + \xi^{a0}) + 2 |T^0| |T^3_1| (\xi^{0a} - \xi^{a0}) \Big] \cr
& & \times \Big[ 2 n_i (|T^3_1| |T^A| + |T^0| |T^3_0| - 2 i |T^A| |T^S|) (\xi^{00} + \xi^{aa}) \Big] \cr
 & &  \times \Bigg[ 1 - n_i \Big( |T^0|^2 + |T^3_0|^2 + 2 |T^S|^2 - 2 |T^A|^2 - 2 |T^3_1|^2 \Big) (\xi^{00} + \xi^{aa}) \cr
 &  & - 2 n_i \Big(|T^3_1|^2 - |T^3_0|^2 \Big) (\xi^{0a} + \xi^{a0}) \Bigg]^{-1} \cr
 & & + n_i \Big[ 2 |T^A| |T^3_1| (\xi^{00} + \xi^{aa}) - 2 i |T^A| |T^S| (\xi^{0a} + \xi^{a0}) \Big] \cr
 & & \times \Bigg[ 1 - n_i \Big( |T^0|^2 + |T^3_0|^2 - 2 |T^3_1|^2 \Big) \xi^{00} - n_i \Big( |T^S|^2 + |T^A|^2\Big) \xi^{aa} + i \, n_i |T^3_1| |T^S| (\xi^{0a} + \xi^{a0}) \Bigg] \cr
 &  & \times \Bigg[ 1 - n_i \Big( |T^0|^2 + |T^3_0|^2 + 2 |T^S|^2 - 2 |T^A|^2 - 2 |T^3_1|^2 \Big) (\xi^{00} + \xi^{aa}) \cr
 &  & - 2 n_i \Big(|T^3_1|^2 - |T^3_0|^2 \Big) (\xi^{0a} + \xi^{a0}) \Bigg]^{-1} \cr
& = & \frac{\gamma_{asym,1} - i \gamma_{asym,3}}{\gamma_t}  - \frac{\gamma_a (\gamma_{30} + \gamma_{asym,1} - i \gamma_{asym,3})}{4 \gamma_t (\gamma_0 + \gamma_a)} + O(\frac{\gamma}{E_F}) \\
 \cr
 \Gamma^0_{p_y}(E_F) & = & 2 i \, n_i \Big[ |T^3_1| |T^A| (\xi^{00} + \xi^{aa}) + |T^A| |T^S| (\xi^{0a} + \xi^{a0}) \Big] \cr
 & & \times \Big[ 2 n_i (|T^3_1| |T^A| + |T^0| |T^3_0| - 2 i |T^A| |T^S|) (\xi^{00} + \xi^{aa}) \Big] \cr
 & &  \times \Bigg[ 1 - n_i \Big( |T^0|^2 + |T^3_0|^2 + 2 |T^S|^2 - 2 |T^A|^2 - 2 |T^3_1|^2 \Big) (\xi^{00} + \xi^{aa}) \cr
 &  & - 2 n_i \Big(|T^3_1|^2 - |T^3_0|^2 \Big) (\xi^{0a} + \xi^{a0}) \Bigg]^{-1} \cr
 & & - i \, n_i \Big[ |T^S|^2 + |T^A|^2 \Big] (\xi^{0a} + \xi^{a0}) \cr
 & & \times \Bigg[ 1 - n_i \Big( |T^0|^2 + |T^3_0|^2 - 2 |T^3_1|^2 \Big) \xi^{00} - n_i \Big( |T^S|^2 + |T^A|^2\Big) \xi^{aa} + i \, n_i |T^3_1| |T^S| (\xi^{0a} + \xi^{a0}) \Bigg] \cr
 &  & \times \Bigg[ 1 - n_i \Big( |T^0|^2 + |T^3_0|^2 + 2 |T^S|^2 - 2 |T^A|^2 - 2 |T^3_1|^2 \Big) (\xi^{00} + \xi^{aa}) \cr
 &  & - 2 n_i \Big(|T^3_1|^2 - |T^3_0|^2 \Big) (\xi^{0a} + \xi^{a0}) \Bigg]^{-1} \cr
 & = & \frac{i}{4} \frac{\gamma_a}{\gamma_t} + \frac{i}{2} \frac{(\gamma_{asym,1} - i \gamma_{asym,3})(\gamma_{asym,1} - i \gamma_{asym,3} + \gamma_{30})}{\gamma_t(\gamma_0 + \gamma_a)} + O(\frac{\gamma}{E_F}) \\
 \cr
 \label{eqn: Gamma3px}
\Gamma^3_{p_x}(E_F) & = & 2 n_i \Big[ |T^3_0| |T^S| - i |T^0| |T^A| \Big] \xi^{00} \cr
& & \times  \Big[ 2 n_i (|T^3_1| |T^A| + |T^0| |T^3_0| - 2 i |T^A| |T^S|) (\xi^{00} + \xi^{aa}) \Big] \cr
 & &  \times \Bigg[ 1 - n_i \Big( |T^0|^2 + |T^3_0|^2 + 2 |T^S|^2 - 2 |T^A|^2 - 2 |T^3_1|^2 \Big) (\xi^{00} + \xi^{aa}) \cr
 &  & - 2 n_i \Big(|T^3_1|^2 - |T^3_0|^2 \Big) (\xi^{0a} + \xi^{a0}) \Bigg]^{-1} \cr
& & - 2 n_i \Big[ (|T^0| |T^S| + i |T^A| |T^3_0|) \xi^{00} - i |T^0| |T^3_1| (\xi^{0a} + \xi^{a0}) \Big] \cr
& & \times \Bigg[ 1 - n_i \Big( |T^0|^2 + |T^3_0|^2 - 2 |T^3_1|^2 \Big) \xi^{00} - n_i \Big( |T^S|^2 + |T^A|^2\Big) \xi^{aa} + i \, n_i |T^3_1| |T^S| (\xi^{0a} + \xi^{a0}) \Bigg] \cr
 &  & \times \Bigg[ 1 - n_i \Big( |T^0|^2 + |T^3_0|^2 + 2 |T^S|^2 - 2 |T^A|^2 - 2 |T^3_1|^2 \Big) (\xi^{00} + \xi^{aa}) \cr
 &  & - 2 n_i \Big(|T^3_1|^2 - |T^3_0|^2 \Big) (\xi^{0a} + \xi^{a0}) \Bigg]^{-1} \cr
& = & - \frac{\gamma_{s}}{\gamma_t} - i \frac{\gamma_{31} + \gamma_{asym,2}}{\gamma_t} + \frac{\gamma_{3s} \gamma_{asym,1}}{2 \gamma_t (\gamma_0 + \gamma_a)} \\
\cr
 \Gamma^3_{p_y}(E_F) & = & n_i \Big[ |T^0| |T^A| + i |T^3_0| |T^S| \Big] \xi^{00} \cr
 & & \times \Bigg[ 1 - n_i \Big( |T^0|^2 + |T^3_0|^2 - 2 |T^3_1|^2 \Big) \xi^{00} - n_i \Big( |T^S|^2 + |T^A|^2\Big) \xi^{aa} + i \, n_i |T^3_1| |T^S| (\xi^{0a} + \xi^{a0}) \Bigg] \cr
 &  & \times \Bigg[ 1 - n_i \Big( |T^0|^2 + |T^3_0|^2 + 2 |T^S|^2 - 2 |T^A|^2 - 2 |T^3_1|^2 \Big) (\xi^{00} + \xi^{aa}) \cr
 &  & - 2 n_i \Big(|T^3_1|^2 - |T^3_0|^2 \Big) (\xi^{0a} + \xi^{a0}) \Bigg]^{-1} \cr
& & - n_i \Big[ (|T^3_0| |T^A| + i |T^0| |T^S|) \xi^{00}  + |T^0| |T^3_1| (\xi^{0a} + \xi^{a0}) \Big] \cr
& & \times \Big[ 2 n_i (|T^3_1| |T^A| + |T^0| |T^3_0| - 2 i |T^A| |T^S|) (\xi^{00} + \xi^{aa}) \Big] \cr
 & &  \times \Bigg[ 1 - n_i \Big( |T^0|^2 + |T^3_0|^2 + 2 |T^S|^2 - 2 |T^A|^2 - 2 |T^3_1|^2 \Big) (\xi^{00} + \xi^{aa}) \cr
 &  & - 2 n_i \Big(|T^3_1|^2 - |T^3_0|^2 \Big) (\xi^{0a} + \xi^{a0}) \Bigg]^{-1} \cr
& = & -\frac{\gamma_3}{4 \gamma_t} - i \frac{\gamma_{3s}}{2 \gamma_t} + \frac{1}{2} \frac{(\gamma_{30} + \gamma_{asym,1})(\gamma_{31} + \gamma_{asym,2}) - \gamma_{s} \gamma_{asym,3}}{\gamma_t (\gamma_0 + \gamma_a)} \cr
& & - \frac{i}{2} \frac{\gamma_{s} (\gamma_{30} + \gamma_{asym,1}) + \gamma_{asym,3} (\gamma_{31} + \gamma_{asym,2})}{\gamma_t (\gamma_0 + \gamma_a)}
 \ea
\end{subequations}
%
Hence, using the results of $\xi^{ij}(E_F)$ listed above, the scattering vertex is,
\ba
\Gammabar^{y}(|\vec{k}| = k_F, \theta, E_F) & = & \Gamma^1_0(E_F) \mI + \Gamma^2_0(E_F) \, \sigma^y  + ( \Gamma^0_{p_x}(E_F) \mI + \Gamma^3_{p_x}(E_F) \, \sigma^z) \cos \theta \cr
& & + i \bl \Gamma^0_{p_y}(E_F) \mI + \Gamma^3_{p_y}(E_F) \, \sigma^z \br \sin \theta 
\ea

Since $\Gamma^2_0$ is the scattering vertex channel for longitudinal electrical conductivity, we have defined a transport scattering rate $\gamma_t = ( \tfrac{1}{2} \gamma_0 + \gamma_a - 2 \gamma_{t'})$, in terms of $\gamma_0$, $\gamma_a$, and an additional transport contribution, $\gamma_{t'} = 2 n_i \pi N_0(E_F) (|T^3_1|^2 - |T^3_0|^2)$. Since $\gamma_{t'} \propto V_0^4 V_1^2 N_0(E_F)^5$, it is much weaker than $\gamma_0 \propto V_0^2 N_0(E_F)$ and $\gamma_a \propto V_0^2 V_1^2 N_0(E_F)^3$, and we do not display $\gamma_{t'}$ in the main paper, but instead, display it here for completeness.

In addition, there are spin flip scattering rates arising from $|T^A|$ and $|T^S|$, $\gamma_{s} = \frac{n_i \pi N_0(E_F)}{2} |T^0| |T^S|$, $\gamma_{asym,1} = 2 n_i \pi N_0(E_F) |T^3_1| |T^A|$, $\gamma_{asym,2} = \frac{n_i \pi N_0(E_F)}{2} |T^3_0| |T^A|$, $\gamma_{asym,3} = \frac{n_i \pi N_0(E_F)}{2} |T^S| |T^A|$, $\gamma_{30} = \frac{n_i \pi N_0(E_F)}{2} |T^3_0| |T^0|$, $\gamma_{31} = \frac{n_i \pi N_0(E_F)}{2} |T^3_1| |T^0|$, $\gamma_{3s} = \frac{n_i \pi N_0(E_F)}{2} |T^3_0| |T^S|$ and $\gamma_{31,s} = \frac{n_i \pi N_0(E_F)}{2} |T^3_1| |T^S|$, which are proportional to $T^S$ and $T^A$, the symmetric and asymmetric component of the $\Tbar$-matrix, as well as the $\sigma^z$ components of the $\Tbar$-matrix, $T^3_0$ and $T^3_1$.

\section{Longitudinal Charge Transport and SHE DC Conductivities}

We calculate the longitudinal charge conductivity, the Rashba-Edelstein effect, and the spin torque contribution to the SHE here. The retarded correlation functions for the spin-torque current contribution to the SHE ($\pi^{z, (2)}_{xy}(\vec{k}, \omega)$), the Rashba-Edelstein effect ($\pi^{i}_{y}(\vec{k}, \omega)$), and the charge current conductivity ($\pi_{yy}(\vec{k}, \omega)$) are shown below, and the DC conductivities are all given by first taking the limit of $\lim \vec{k} \to 0$, then taking the DC limit of $\lim \omega \to 0$, $\sigma^{(\text{DC})} = - \underset{\omega \to 0}{\lim} \underset{\vec{k} \to 0}{\lim} \, Im [\tfrac{\pi(\vec{k}, \omega)}{\omega}]$.
%
\ba
\label{eqn: pizxy corr function}
\pi_{yy}(\vec{k} = 0, \omega) & = & \underset{\vec{k} \to 0}{\lim} \bl \frac{e v_F}{c} \br^2 \int_{-\infty}^{\infty} \frac{d \epsilon}{2 \pi i} \int \frac{d^2 p}{(2 \pi)^2} \Tr \left[ \Gbar^{(A)}(\vec{p}, \epsilon) \sigma^y \Gbar^{(R)}(\vec{p} + \vec{k}, \epsilon) \Gammabar^{y}(\vec{p}, \vec{p} + \vec{k}, \epsilon) \right] \cr
 & \times & \bl n_F(\epsilon) - n_F(\epsilon + \omega) \br \\
\pi^{z, (2)}_{xy}(\vec{k} = 0, \omega) & = & \pi^{z, (2a)}_{xy}(\vec{k} = 0, \omega) + \pi^{z, (2b)}_{xy}(\vec{k} = 0, \omega) \\
\pi^{z, (2a)}_{xy}(\vec{k} = 0, \omega) & = & \underset{\vec{k} \to 0}{\lim} \frac{2 i e v_F^2}{c} \int_{-\infty}^{\infty} \frac{d \epsilon}{2 \pi i} \int \frac{d^2 p}{(2 \pi)^2} \Tr \left[ \Gbar^{(A)}(\vec{p}, \epsilon) \sigma^x \Gbar^{(R)}(\vec{p} + \vec{k}, \epsilon) \Gammabar^{y}(\vec{p}, \vec{p} + \vec{k}, \epsilon) \right] \cr
 & \times & \frac{p_y + \frac{k_y}{2}}{p_x} \bl n_F(\epsilon) - n_F(\epsilon + \omega) \br \cr
\pi^{z, (2b)}_{xy}(\vec{k} = 0, \omega) & = & - \underset{\vec{k} \to 0}{\lim} \frac{2 i e v_F^2}{c} \int_{-\infty}^{\infty} \frac{d \epsilon}{2 \pi i} \int \frac{d^2 p}{(2 \pi)^2} \Tr \left[ \Gbar^{(A)}(\vec{p}, \epsilon) \sigma^y \Gbar^{(R)}(\vec{p} + \vec{k}, \epsilon) \Gammabar^{y}(\vec{p}, \vec{p} + \vec{k}, \epsilon) \right] \cr
& \times & \frac{p_x + \frac{k_x}{2}}{p_x} \bl n_F(\epsilon) - n_F(\epsilon  + \omega) \br \cr
\pi^{i}_{y}(\vec{k} = 0, \omega) & = & \underset{\vec{k} \to 0}{\lim} \frac{e v_F}{c} \int_{-\infty}^{\infty} \frac{d \epsilon}{2 \pi i} \int \frac{d^2 p}{(2 \pi)^2} \Tr \left[ \Gbar^{(A)}(\vec{p}, \epsilon) \sigma^i \Gbar^{(R)}(\vec{p} + \vec{k}, \epsilon) \Gammabar^{y}(\vec{p}, \vec{p} + \vec{k}, \epsilon) \right] \cr
& \times & \bl n_F(\epsilon + \omega) - n_F(\epsilon) \br
\ea
%
We have specialized to the case of a charge current along $\hat{y}$ in the expression for the Rashba-Edelstein effect. For the SHE Kubo formula, we have to Taylor expand the Green's function $\Gbar^{(R)}(\vec{p} + \vec{k}, \epsilon) = \Gbar^{(R)}(\vec{p}, \epsilon) + k_i \tfrac{d \Gbar^{(R)}(\vec{p}, \epsilon)}{d p_i}$, which is shown in detail below.

\begin{subequations}
\ba
\label{eqn: Greens func derivative}
\frac{d \Gbar^{(R)}(\vec{p}, \epsilon)}{d p_x} & = & \frac{\partial \Gbar^{(R)}(\vec{p}, \epsilon) }{\partial p} \frac{\partial p}{\partial p_x} + \frac{\partial \Gbar^{(R)}(\vec{p}, \epsilon) }{\partial \theta} \frac{\partial \theta}{\partial p_x} \\
\frac{\partial \Gbar^{(R)}(\vec{p}, \epsilon) }{\partial p} \frac{\partial p}{\partial p_x} & = &  \left[ \frac{d g^0}{d p} \mI + \frac{d g^3}{d p}  \sigma^z + \frac{d g^a}{d p} (\cos \theta_p \sigma^x + \sin \theta_p \sigma^y) + \frac{d g^b}{d p} (\sin \theta_p \sigma^x - \cos \theta_p \sigma^y)\right] \cos \theta_p  \cr
\frac{\partial \Gbar^{(R)}(\vec{p}, \epsilon) }{\partial \theta} \frac{\partial \theta}{\partial p_x} & = &  \left[ g^a (-\sin \theta_p \sigma^x + \cos \theta_p \sigma^y) + g^b (\cos \theta_p \sigma^x + \sin \theta_p \sigma^y) \right] \big( - \frac{\sin \theta_p}{p} \big) \cr
\frac{d \Gbar^{(R)}(\vec{p}, \epsilon)}{d p_y} & = & \frac{\partial \Gbar^{(R)}(\vec{p}, \epsilon) }{\partial p} \frac{\partial p}{\partial p_y} + \frac{\partial \Gbar^{(R)}(\vec{p}, \epsilon) }{\partial \theta} \frac{\partial \theta}{\partial p_y} \\
\frac{\partial \Gbar^{(R)}(\vec{p}, \epsilon) }{\partial p} \frac{\partial p}{\partial p_y} & = &  \left[ \frac{d g^0}{d p} \mI + \frac{d g^3}{d p}  \sigma^z + \frac{d g^a}{d p} (\cos \theta_p \sigma^x + \sin \theta_p \sigma^y) + \frac{d g^b}{d p} (\sin \theta_p \sigma^x - \cos \theta_p \sigma^y)\right] \sin \theta_p \cr
\frac{\partial \Gbar^{(R)}(\vec{p}, \epsilon) }{\partial \theta} \frac{\partial \theta}{\partial p_y} & = &  \left[ g^a (-\sin \theta_p \sigma^x + \cos \theta_p \sigma^y) + g^b (\cos \theta_p \sigma^x + \sin \theta_p \sigma^y) \right] \big( \frac{\cos \theta_p}{p} \big) \nonumber
\ea
\end{subequations}
%
Following the same approximation of an average $\Gammabar$-matrix near $E_F$, the spin current-current correlation function is then given in terms of the $\Gammabar$-coefficients, and the spin-resolved density of states $\xi^{ij}(E_F)$, as well as the quantity involving the integral of $\Gbar^{(A)}(\vec{k}, \epsilon) \tfrac{d \Gbar^{(R)}(\vec{k}, \epsilon)}{d \vec{k}}$, which we term $\eta^{ij} (\epsilon)$,
%
\begin{subequations}
\ba
\label{eqn: Green func integral}
\eta^{ij}(\epsilon) & \equiv & \int_{-\infty}^{\infty} \frac{d p}{2 \pi} \, p^2 \, \frac{d g^{i,(R)}(p, \epsilon)}{d p} g^{j,(A)}_{eff}(p, \epsilon) \\
\cr
\eta^{00}(\epsilon) & = & \int \frac{d p}{2 \pi} v_F p^2 \Bigg[ \frac{2(- v_F p + i \gamma_a) (\epsilon - i \gamma_0) }{ \Omega(p)^2 + \kappa(p)^2} \cr
 & & + \frac{4 (v _F p \Omega(p) - \gamma_a \kappa(p)) (\Omega(p) + i \kappa(p)) (\epsilon - i \gamma_0) }{ (\Omega(p)^2 + \kappa(p)^2)^2 }  \Bigg] \frac{ (\Omega(p) - i \kappa(p)) (\epsilon + i \gamma_0) }{ \Omega(p)^2 + \kappa(p)^2} \cr
  & = & \frac{1}{2 \pi v_F^2} \Big[ \frac{i \pi \epsilon^2}{8 (\gamma_0 + \gamma_a)^2} - \frac{\pi \epsilon}{16 (\gamma_0 + \gamma_a)} \cr
  & & + \frac{i \gamma_0 (\gamma_0^2 + \gamma_a^2) \epsilon}{4 (\gamma_0^2 - \gamma_a^2)^2} + \frac{i \pi (2 \gamma_0^2 - \gamma_0 \gamma_a + \gamma_a^2)}{16 (\gamma_0 + \gamma_a)^2} - \frac{1}{8} + O(\frac{\gamma}{\epsilon}) \Big] \\
  \cr
  \eta^{aa}(\epsilon) & = & \int \frac{d p}{2 \pi} v_F p^2 \Bigg[ \frac{2(- v_F p + i \gamma_a) (v_F p + i \gamma_a) }{ \Omega(p)^2 + \kappa(p)^2} \cr
 & & + \frac{4 (v _F p \Omega(p) - \gamma_a \kappa(p)) (\Omega(p) + i \kappa(p)) (v_F p + i \gamma_a) }{ (\Omega(p)^2 + \kappa(p)^2)^2 }  \Bigg] \frac{ (\Omega(p) - i \kappa(p)) (v_F p - i \gamma_a) }{ \Omega(p)^2 + \kappa(p)^2} \cr
 & = & \frac{1}{2 \pi v_F^2} \Big[ \frac{i \pi \epsilon^2}{8 (\gamma_0 + \gamma_a)^2} - \frac{\pi \epsilon }{16 (\gamma_0 + \gamma_a)} \cr
 & & + \frac{i \gamma_0 (\gamma_0^2 + \gamma_a^2) \epsilon}{4 (\gamma_0^2 - \gamma_a^2)^2} - \frac{i \pi (\gamma_0 -3 \gamma_a) \gamma_a}{16 (\gamma_0 + \gamma_a)^2}  - \frac{\gamma_0^4 + 6 \gamma_0^2 \gamma_a^2 + \gamma_a^4}{8 (\gamma_0^2 - \gamma_a^2)^2} + O(\frac{\gamma}{\epsilon}) \Big] \\
 \cr
\eta^{aa}(\epsilon) - \eta^{00}(\epsilon) & = & \frac{1}{2 \pi v_F^2} \Big[ - \frac{\gamma_0^2 \gamma_a^2}{(\gamma_0^2 - \gamma_a^2)^2} - i \frac{\pi (\gamma_0 - \gamma_a)}{8 (\gamma_0 + \gamma_a)} \Big] \\
\cr
\eta^{0a}(\epsilon) & = & \int \frac{d p}{2 \pi} v_F p^2 \Bigg[ \frac{2(- v_F p + i \gamma_a) (\epsilon - i \gamma_0) }{ \Omega(p)^2 + \kappa(p)^2} \cr
 & & + \frac{4 (v _F p \Omega(p) - \gamma_a \kappa(p)) (\Omega(p) + i \kappa(p)) (\epsilon - i \gamma_0) }{ (\Omega(p)^2 + \kappa(p)^2)^2 }  \Bigg] \frac{ (\Omega(p) - i \kappa(p)) (v_F p - i \gamma_a) }{ \Omega(p)^2 + \kappa(p)^2} \cr
 & = & \frac{1}{2 \pi v_F^2} \Big[ i \frac{\pi  \epsilon^2}{8 (\gamma_0 + \gamma_a)^2} + \frac{\pi  \epsilon}{16 ( \gamma_0 +  \gamma_a)} + i \frac{\gamma_a \epsilon  (\gamma_0^2 + \gamma_a^2)}{4 ( \gamma_0^2 - \gamma_a^2)^2} \cr
 & & - i \frac{\pi (\gamma_0^2 -\gamma_0 \gamma_a + 2 \gamma_a^2) }{16 \left( \gamma_0 + \gamma_a \right)^2} + \frac{ \gamma_0^3 \gamma_a }{2 (\gamma_0^2 - \gamma_a^2)^2} + O(\frac{\gamma}{\epsilon})  \Big] \\
 \cr
 \eta^{a0}(\epsilon) & = & \int \frac{d p}{2 \pi} v_F p^2 \Bigg[ \frac{2(- v_F p + i \gamma_a) (v_F p + i \gamma_a) }{ \Omega(p)^2 + \kappa(p)^2} \cr
 & & + \frac{4 (v _F p \Omega(p) - \gamma_a \kappa(p)) (\Omega(p) + i \kappa(p)) (v_F p + i \gamma_a) }{ (\Omega(p)^2 + \kappa(p)^2)^2 }  \Bigg] \frac{ (\Omega(p) - i \kappa(p)) (\epsilon + i \gamma_0) }{ \Omega(p)^2 + \kappa(p)^2} \cr
 & = & \frac{1}{2 \pi v_F^2} \Big[ i \frac{\pi  \epsilon^2}{8 (\gamma_0 + \gamma_a)^2} + \frac{\gamma_0 \gamma_a \epsilon^2}{2 ( \gamma_0^2 -  \gamma_a^2)^2}  - \frac{3 \pi \epsilon}{16 ( \gamma_0 + \gamma_a)} + \frac{i \gamma_a (5 \gamma_0^2 + \gamma_a^2) \epsilon}{4 (\gamma_0^2 - \gamma_a^2)^2} \cr
 & & - \frac{i \pi \gamma_0 (\gamma_0 + 5 \gamma_a) }{16 \left( \gamma_0 + \gamma_a \right)^2} - \frac{3 \gamma_0^3 \gamma_a  + 2 \gamma_0 \gamma_a^3)}{2 (\gamma_0^2 - \gamma_a^2)^2} + O(\frac{\gamma}{\epsilon}) \Big] \\
 \cr
\eta^{0a}(\epsilon) - \eta^{a0}(\epsilon) & = & \frac{1}{2 \pi v_F^2} \Big[ \frac{\pi \epsilon}{4 (\gamma_0 + \gamma_a)} - \frac{i \gamma_0^2 \gamma_a \epsilon}{(\gamma_0^2 - \gamma_a^2)^2} + \frac{\gamma_0 \gamma_a (2 \gamma_0^2 + \gamma_a^2)}{(\gamma_0^2 - \gamma_a^2)^2} + O(\frac{\gamma}{\epsilon}) \Big]
\ea
\end{subequations}

Note that $\epsilon = \omega + \mu$ is the energy measured from $E_F$; hence, the DC conductivities will depend on $\eta^{ij}(E_F)$. We now re-write the SHE correlation function as a sum of several terms, $\pi^{z,(2)}_{xy}(\vec{k}, \omega) = \pi^{z,(2a)}_{xy}(\vec{k}, \omega)  + \pi^{z,(2b)}_{xy}(\vec{k}, \omega)$, where $\pi^{z,(2a)}_{xy}(\vec{k}, \omega)$ and $\pi^{z,(2b)}_{xy}(\vec{k}, \omega)$ are the $k_y \sigma^x$ and $k_x \sigma^y$ terms respectively. It is then necessary to Taylor expand $\Gbar^{(R)}(\vec{p} + \vec{k}, \epsilon) = \Gbar^{(R)}(\vec{p}, \epsilon) + k_i \tfrac{d \Gbar^{(R)}(\vec{p}, \epsilon)}{d p_i}$, and $\pi^{z, (2a1)}(\vec{k}, \omega)$ is the zeroth-order term, while $\pi^{z, (2a2)}(\vec{k}, \omega)$ and $\pi^{z, (2a3)}(\vec{k}, \omega)$ are the $k_x \tfrac{d \Gbar^{(R)}(\vec{p}, \epsilon)}{d p_x}$ and $k_y \tfrac{d \Gbar^{(R)}(\vec{p}, \epsilon)}{d p_y}$ terms respectively; thus, giving  $\pi^{z,(2a)}_{xy}(\vec{k}, \omega) = \pi^{z,(2a1)}_{xy}(\vec{k} = 0, \omega) + \pi^{z,(2a2)}_{xy}(\vec{k} = 0, \omega) + \pi^{z,(2a3)}_{xy}(\vec{k} = 0, \omega)$ and $\pi^{z,(2b)}_{xy}(\vec{k}, \omega) = \pi^{z,(2b1)}_{xy}(\vec{k} = 0, \omega) + \pi^{z,(2b2)}_{xy}(\vec{k} = 0, \omega) + \pi^{z,(2b3)}_{xy}(\vec{k} = 0, \omega)$. Finally, we make use of the chain rule $\tfrac{d \Gbar^{(R)}(\vec{p}, \epsilon)}{d p_i} = \tfrac{d \Gbar^{(R)}(\vec{p}, \epsilon)}{d p} \tfrac{\partial p}{ \partial p_i} + \tfrac{d \Gbar^{(R)}(\vec{p}, \epsilon)}{\partial \theta} \tfrac{\partial \theta}{ \partial p_i}$, which give $\pi^{z,(2a1)}_{xy}(\vec{k} = 0, \omega) = \pi^{z,(2a1P1)}_{xy}(\vec{k}, \omega) + \pi^{z,(2a1P2)}_{xy}(\vec{k}, \omega)$ respectively, with $\pi^{z,(2a1P1)}_{xy}(\vec{k}, \omega)$ and $\pi^{z,(2a1P2)}_{xy}(\vec{k}, \omega)$ being proportional to the $\tfrac{d \Gbar^{(R)}(\vec{p}, \epsilon)}{d p} \tfrac{\partial p}{ \partial p_i}$ and $\tfrac{d \Gbar^{(R)}(\vec{p}, \epsilon)}{\partial \theta} \tfrac{\partial \theta}{ \partial p_i}$ terms respectively. A similar procedure is carried out for the other terms, and we have symmetrized the expressions for $\pi^{z, (2a)}_{xy}(\vec{k}, \omega)$ and $\pi^{z, (2b)}_{xy}(\vec{k}, \omega)$ by doing a shift of variable $p_y + \tfrac{k_y}{2} \rightarrow p_y$ and $p_x + \tfrac{k_x}{2} \rightarrow p_x$ respectively. The results are shown below.
%
\begin{subequations}
\ba
\pi^{z,(2a)}_{xy}(\vec{k}, \omega) & = & \underset{\vec{k} \to 0}{\lim} \frac{2 i e v_F^2}{c} \int \frac{d \epsilon}{2 \pi i} \sum_{\vec{p}} \bl n_F(\epsilon) - n_F(\epsilon + \omega) \br 
\frac{p_y}{k_x} \cr
 & & \times \Tr \Big[ \Gbar^{(A)}(\vec{p} - \frac{\vec{k}}{2}, \epsilon) \, \sigma^x \, \Gbar^{R)}(\vec{p} + \frac{\vec{k}}{2}, \epsilon) \, \Gammabar^{(y)}(\vec{p}, \epsilon) \Big] \cr
& = & \pi^{z,(2a1)}_{xy}(\vec{k} = 0, \omega) + \pi^{z,(2a2)}_{xy}(\vec{k} = 0, \omega) + \pi^{z,(2a3)}_{xy}(\vec{k} = 0, \omega) \\
\cr
\pi^{z,(2a1)}_{xy}(\vec{k} = 0, \omega) & = &  \underset{\vec{k} \to 0}{\lim} \frac{2 i e v_F^2}{c} \frac{1}{k_x} \int \frac{d \epsilon}{2 \pi i} \sum_{\vec{p}} \bl n_F(\epsilon) - n_F(\epsilon + \omega) \br \cr
 & & \times \Tr \Big[ \Gbar^{(A)}(\vec{p}, \epsilon) \sigma^x \Gbar^{R)}(\vec{p}, \epsilon) \Gammabar^{(y)}(\vec{p}, \epsilon) \Big] p \sin \theta \cr
& = & 0 \\
\cr
\pi^{z,(2a2)}_{xy}(\vec{k} = 0, \omega) & = &  \underset{\vec{k} \to 0}{\lim} \frac{2 i e v_F^2}{c} \frac{k_x}{k_x} \int \frac{d \epsilon}{2 \pi i} \sum_{\vec{p}} \bl n_F(\epsilon) - n_F(\epsilon + \omega) \br \frac{p \sin \theta}{2} \cr
 & & \times \Bigg( \Tr \Big[ \Gbar^{(A)}(\vec{p}, \epsilon) \sigma^x \frac{\partial \Gbar^{(R)}(\vec{p}, \epsilon)}{\partial p_x} \Gammabar^{(y)}(\vec{p}, \epsilon) \Big] - \Tr \Big[ \frac{\partial \Gbar^{(A)}(\vec{p}, \epsilon)}{\partial p_x} \sigma^x \Gbar^{(R)}(\vec{p}, \epsilon) \Gammabar^{(y)}(\vec{p}, \epsilon) \Big] \Bigg) \cr
 & = & \underset{\vec{k} \to 0}{\lim} \frac{2 i e v_F^2}{c} \frac{k_x}{k_x} \int \frac{d \epsilon}{2 \pi i} \sum_{\vec{p}} \bl n_F(\epsilon) - n_F(\epsilon + \omega) \br \frac{p \sin \theta}{2} \cr
 & & \times \Bigg( \Tr \Big[ \Gbar^{(A)}(\vec{p}, \epsilon) \sigma^x \bl \frac{\partial \Gbar^{(R)}(\vec{p}, \epsilon)}{\partial p} \frac{\partial p}{\partial p_x} + \frac{\partial \Gbar^{(R)}(\vec{p}, \epsilon)}{\partial \theta} \frac{\partial \theta}{\partial p_x} \br \Gammabar^{(y)}(\vec{p}, \epsilon) \Big] \cr
 & & - \Tr \Big[ \bl \frac{\partial \Gbar^{(A)}(\vec{p}, \epsilon)}{\partial p} \frac{\partial p}{\partial p_x} + \frac{\partial \Gbar^{(A)}(\vec{p}, \epsilon)}{\partial \theta} \frac{\partial \theta}{\partial p_x} \br\sigma^x \Gbar^{(R)}(\vec{p}, \epsilon) \Gammabar^{(y)}(\vec{p}, \epsilon) \Big] \Bigg) \cr
& = & \pi^{z,(2a2P1)}_{xy}(\vec{k} = 0, \omega) + \pi^{z,(2a2P2)}_{xy}(\vec{k} = 0, \omega) \\
\cr
\pi^{z,(2a2P1)}_{xy}(\vec{k} = 0, \omega) & = &  \underset{\vec{k} \to 0}{\lim} \frac{2 i e v_F^2}{c} \frac{k_x}{k_x} \int \frac{d \epsilon}{2 \pi i} \sum_{\vec{p}} \bl n_F(\epsilon) - n_F(\epsilon + \omega) \br \frac{p \sin \theta}{2} \cr
 & & \times \Bigg( \Tr \Big[ \Gbar^{(A)}(\vec{p}, \epsilon) \sigma^x \bl \frac{\partial \Gbar^{(R)}(\vec{p}, \epsilon)}{\partial p} \frac{\partial p}{\partial p_x} \br \Gammabar^{(y)}(\vec{p}, \epsilon) \Big] \cr
 & & - \Tr \Big[ \bl \frac{\partial \Gbar^{(A)}(\vec{p}, \epsilon)}{\partial p} \frac{\partial p}{\partial p_x} \br \sigma^x \Gbar^{(R)}(\vec{p}, \epsilon) \Gammabar^{(y)}(\vec{p}, \epsilon) \Big] \Bigg) \cr
 & = &  \underset{\vec{k} \to 0}{\lim} \frac{2 i e v_F^2}{c} \frac{k_x}{k_x} \int \frac{d \epsilon}{2 \pi i} \sum_{\vec{p}} \bl n_F(\epsilon) - n_F(\epsilon + \omega) \br \frac{p \sin \theta}{2} \frac{\partial p}{\partial p_x} \cr
 & & \times \Bigg( \Tr \Big[ \Gbar^{(A)}(\vec{p}, \epsilon) \sigma^x \frac{\partial \Gbar^{(R)}(\vec{p}, \epsilon)}{\partial p} \Gammabar^{(y)}(\vec{p}, \epsilon) \Big] - \Tr \Big[ \frac{\partial \Gbar^{(A)}(\vec{p}, \epsilon)}{\partial p} \sigma^x \Gbar^{(R)}(\vec{p}, \epsilon) \Gammabar^{(y)}(\vec{p}, \epsilon) \Big] \Bigg) \cr
& = & \frac{2 i e v_F^2}{c} \int \frac{d \epsilon}{2 \pi i} \bl n_F(\epsilon) - n_F(\epsilon + \omega) \br \frac{1}{2} \cr
& & \times \frac{1}{4} \Big[ \Gamma^2_s(\epsilon) \bl 2 \eta^{aa}(\epsilon) - 2 (\eta^{aa}(\epsilon))^* \br \cr
& & + \Gamma^{0}_{p_y}(\epsilon) \bl \eta^{a0}(\epsilon) + \eta^{0a}(\epsilon) - (\eta^{a0}(\epsilon))^* - (\eta^{0a}(\epsilon))^* \br  \cr
& & + i \Gamma^{3}_{p_x}(\epsilon) \bl \eta^{a0}(\epsilon) - \eta^{0a}(\epsilon) + (\eta^{a0}(\epsilon))^* - (\eta^{0a}(\epsilon))^*  \br \Big] + O(\frac{\gamma}{E_F}) \\
\cr
\pi^{z,(2a2P2)}_{xy}(\vec{k} = 0, \omega) & = &  \underset{\vec{p} \to 0}{\lim} \frac{2 i e v_F^2}{c} \frac{k_x}{k_x} \int \frac{d \epsilon}{2 \pi i} \sum_{\vec{p}}  \bl n_F(\epsilon) - n_F(\epsilon + \omega) \br \frac{p \sin \theta}{2} \frac{\partial \theta}{\partial p_x} \cr
 & & \times \Bigg( \Tr \Big[ \Gbar^{(A)}(\vec{p}, \epsilon) \sigma^x \frac{\partial \Gbar^{(R)}(\vec{p}, \epsilon)}{\partial \theta} \Gammabar^{(y)}(\vec{p}, \epsilon) \Big] - \Tr \Big[ \frac{\partial \Gbar^{(A)}(\vec{p}, \epsilon)}{\partial \theta} \sigma^x \Gbar^{(R)}(\vec{p}, \epsilon) \Gammabar^{(y)}(\vec{p}, \epsilon) \Big] \Bigg) \cr
 & = & \frac{2 i e v_F^2}{c} \int \frac{d \epsilon}{2 \pi i}   \bl n_F(\epsilon) - n_F(\epsilon + \omega) \br \frac{1}{2} \cr
 & & \times \frac{1}{4} \Big[ 3 \Gamma^0_{p_y}(\epsilon) \bl \xi^{0a}(\epsilon) + \xi^{a0}(\epsilon) + i \xi^{3b}(\epsilon) + i \xi^{b3}(\epsilon) \br \cr
 & & + \Gamma^3_{p_x}(\epsilon) \bl  i \xi^{0a}(\epsilon) + i \xi^{a0}(\epsilon) \br \Big] + O(\frac{\gamma}{E_F}) \\
 \cr
 \pi^{z,(2a3)}_{xy}(\vec{k} = 0, \omega) & = & \underset{\vec{k} \to 0}{\lim} \frac{2 i e v_F^2}{c} \frac{k_y}{k_x} \int \frac{d \epsilon}{2 \pi i} \sum_{\vec{p}} \bl n_F(\epsilon) - n_F(\epsilon + \omega) \br \frac{p \sin \theta}{2} \cr
 & & \times \Bigg( \Tr \Big[ \Gbar^{(A)}(\vec{p}, \epsilon) \sigma^x \frac{\partial \Gbar^{(R)}(\vec{p}, \epsilon)}{\partial p_y} \Gammabar^{(y)}(\vec{p}, \epsilon) \Big] - \Tr \Big[ \frac{\partial \Gbar^{(A)}(\vec{p}, \epsilon)}{\partial p_y} \sigma^x \Gbar^{(R)}(\vec{p}, \epsilon) \Gammabar^{(y)}(\vec{p}, \epsilon) \Big] \Bigg) \cr
 & = & \underset{\vec{k} \to 0}{\lim} \frac{2 i e v_F^2}{c} \frac{k_y}{k_x} \int \frac{d \epsilon}{2 \pi i} \sum_{\vec{p}} \bl n_F(\epsilon) - n_F(\epsilon + \omega) \br \frac{p \sin \theta}{2} \cr
 & & \times \Bigg( \Tr \Big[ \Gbar^{(A)}(\vec{p}, \epsilon) \sigma^x \bl \frac{\partial \Gbar^{(R)}(\vec{p}, \epsilon)}{\partial p} \frac{\partial p}{\partial p_y} + \frac{\partial \Gbar^{(R)}(\vec{p}, \epsilon)}{\partial \theta} \frac{\partial \theta}{\partial p_y} \br \Gammabar^{(y)}(\vec{p}, \epsilon) \Big] \cr
 & & - \Tr \Big[ \bl \frac{\partial \Gbar^{(A)}(\vec{p}, \epsilon)}{\partial p} \frac{\partial p}{\partial p_y} + \frac{\partial \Gbar^{(A)}(\vec{p}, \epsilon)}{\partial \theta} \frac{\partial \theta}{\partial p_y} \br\sigma^x \Gbar^{(R)}(\vec{p}, \epsilon) \Gammabar^{(y)}(\vec{p}, \epsilon) \Big] \Bigg) \cr
& = & \pi^{z,(2a3P1)}_{xy}(\vec{k} = 0, \omega) + \pi^{z,(2a3P2)}_{xy}(\vec{k} = 0, \omega) \\
\cr
\pi^{z,(2a3P1)}_{xy}(\vec{k} = 0, \omega) & = &  \underset{\vec{p} \to 0}{\lim} \frac{2 i e v_F^2}{c} \frac{k_y}{k_x} \int \frac{d \epsilon}{2 \pi i} \sum_{\vec{p}} \bl n_F(\epsilon) - n_F(\epsilon + \omega) \br \frac{p \sin \theta}{2} \frac{\partial p}{\partial p_y} \cr
 & & \times \Bigg( \Tr \Big[ \Gbar^{(A)}(\vec{p}, \epsilon) \sigma^x \frac{\partial \Gbar^{(R)}(\vec{p}, \epsilon)}{\partial p} \Gammabar^{(y)}(\vec{p}, \epsilon) \Big] - \Tr \Big[ \frac{\partial \Gbar^{(A)}(\vec{p}, \epsilon)}{\partial p} \sigma^x \Gbar^{(R)}(\vec{p}, \epsilon) \Gammabar^{(y)}(\vec{p}, \epsilon) \Big] \Bigg) \cr
& = & \frac{2 i e v_F^2}{c} \int \frac{d \epsilon}{2 \pi i}   \bl n_F(\epsilon) - n_F(\epsilon + \omega) \br \frac{1}{2} \cr
& & \times \frac{1}{4} \Big[ \Gamma^1_s(\epsilon) \bl 4 \eta^{00}(\epsilon) - 4 (\eta^{00}(\epsilon))^* - 2 \eta^{aa}(\epsilon) +  2 (\eta^{aa}(\epsilon))^* \br \cr
& & + \Gamma^{0}_{p_x}(\epsilon) \bl \eta^{a0}(\epsilon) + \eta^{0a}(\epsilon) - (\eta^{a0}(\epsilon))^* + (\eta^{0a}(\epsilon))^* \br  \cr
& & + \Gamma^{3}_{p_y}(\epsilon) \bl - 3 \eta^{a0}(\epsilon) + 3 \eta^{0a}(\epsilon) - 3 (\eta^{a0}(\epsilon))^* + 3 (\eta^{0a}(\epsilon))^*  \br \Big] + O(\frac{\gamma}{E_F}) \\
\cr
\pi^{z,(2a3P2)}_{xy}(\vec{k} = 0, \omega) & = &  \underset{\vec{p} \to 0}{\lim} \frac{2 i e v_F^2}{c} \frac{k_y}{k_x} \int \frac{d \epsilon}{2 \pi i} \sum_{\vec{p}} \bl n_F(\epsilon) - n_F(\epsilon + \omega) \br \frac{p \sin \theta}{2} \frac{\partial \theta}{\partial p_y} \cr
 & & \times \Bigg( \Tr \Big[ \Gbar^{(A)}(\vec{p}, \epsilon) \sigma^x \frac{\partial \Gbar^{(R)}(\vec{p}, \epsilon)}{\partial \theta} \Gammabar^{(y)}(\vec{p}, \epsilon) \Big] - \Tr \Big[ \frac{\partial \Gbar^{(A)}(\vec{p}, \epsilon)}{\partial \theta} \sigma^x \Gbar^{(R)}(\vec{p}, \epsilon) \Gammabar^{(y)}(\vec{p}, \epsilon) \Big] \Bigg) \cr
 & = & \frac{2 i e v_F^2}{c} \int \frac{d \epsilon}{2 \pi i}   \bl n_F(\epsilon) - n_F(\epsilon + \omega) \br \frac{1}{2} \cr
 & & \times \frac{1}{4} \Big[ \Gamma^0_{p_x}(\epsilon) \bl \xi^{0a}(\epsilon) - \xi^{a0}(\epsilon) \br - \Gamma^3_{p_y}(\epsilon) \bl  \xi^{0a}(\epsilon) + \xi^{a0}(\epsilon) \br \Big] + O(\frac{\gamma}{E_F}) \\
 \cr
\pi^{z,(2b)}_{xy}(\vec{k}, \omega) & = & \underset{\vec{k} \to 0}{\lim} -\frac{2 i e v_F^2}{c} \int \frac{d \epsilon}{2 \pi i} \sum_{\vec{p}} \bl n_F(\epsilon) - n_F(\epsilon + \omega) \br 
\frac{p_x}{k_x} \cr
 & & \times \Tr \Big[ \Gbar^{(A)}(\vec{p} - \frac{\vec{k}}{2}, \epsilon) \, \sigma^y \, \Gbar^{R)}(\vec{p} + \frac{\vec{k}}{2}, \epsilon) \, \Gammabar^{(y)}(\vec{p}, \epsilon) \Big] \cr
& = & \pi^{z,(2b1)}_{xy}(\vec{k} = 0, \omega) + \pi^{z,(2b2)}_{xy}(\vec{k} = 0, \omega) + \pi^{z,(2b3)}_{xy}(\vec{k} = 0, \omega) \\
\cr
\pi^{z,(2b1)}_{xy}(\vec{k} = 0, \omega) & = &  \underset{\vec{k} \to 0}{\lim} -\frac{2 i e v_F^2}{c} \frac{1}{k_x} \int \frac{d \epsilon}{2 \pi i} \sum_{\vec{p}}  \bl n_F(\epsilon) - n_F(\epsilon + \omega) \br \cr
 & & \times \Tr \Big[ \Gbar^{(A)}(\vec{p}, \epsilon) \sigma^y \Gbar^{R)}(\vec{p}, \epsilon) \Gammabar^{(y)}(\vec{p}, \epsilon) \Big] p \cos \theta \cr
& = & 0 \\
\cr
\pi^{z,(2b2)}_{xy}(\vec{k} = 0, \omega) & = &  \underset{\vec{k} \to 0}{\lim} -\frac{2 i e v_F^2}{c} \frac{k_x}{k_x} \int \frac{d \epsilon}{2 \pi i} \sum_{\vec{p}} \bl n_F(\epsilon) - n_F(\epsilon + \omega) \br \frac{p \cos \theta}{2} \cr
 & & \times \Bigg( \Tr \Big[ \Gbar^{(A)}(\vec{p}, \epsilon) \sigma^y \frac{\partial \Gbar^{(R)}(\vec{p}, \epsilon)}{\partial p_x} \Gammabar^{(y)}(\vec{p}, \epsilon) \Big] - \Tr \Big[ \frac{\partial \Gbar^{(A)}(\vec{p}, \epsilon)}{\partial p_x} \sigma^y \Gbar^{(R)}(\vec{p}, \epsilon) \Gammabar^{(y)}(\vec{p}, \epsilon) \Big] \Bigg) \cr
 & = & \underset{\vec{k} \to 0}{\lim} -\frac{2 i e v_F^2}{c} \frac{k_x}{k_x} \int \frac{d \epsilon}{2 \pi i} \sum_{\vec{p}} \bl n_F(\epsilon) - n_F(\epsilon + \omega) \br \frac{p \cos \theta}{2} \cr
 & & \times \Bigg( \Tr \Big[ \Gbar^{(A)}(\vec{p}, \epsilon) \sigma^y \bl \frac{\partial \Gbar^{(R)}(\vec{p}, \epsilon)}{\partial p} \frac{\partial p}{\partial p_x} + \frac{\partial \Gbar^{(R)}(\vec{p}, \epsilon)}{\partial \theta} \frac{\partial \theta}{\partial p_x} \br \Gammabar^{(y)}(\vec{p}, \epsilon) \Big] \cr
 & & - \Tr \Big[ \bl \frac{\partial \Gbar^{(A)}(\vec{p}, \epsilon)}{\partial p} \frac{\partial p}{\partial p_x} + \frac{\partial \Gbar^{(A)}(\vec{p}, \epsilon)}{\partial \theta} \frac{\partial \theta}{\partial p_x} \br\sigma^y \Gbar^{(R)}(\vec{p}, \epsilon) \Gammabar^{(y)}(\vec{p}, \epsilon) \Big] \Bigg) \cr
& = & \pi^{z,(2b2P1)}_{xy}(\vec{k} = 0, \omega) + \pi^{z,(2b2P2)}_{xy}(\vec{k} = 0, \omega) \\
\cr
\pi^{z,(2b2P1)}_{xy}(\vec{k} = 0, \omega) & = &  \underset{\vec{p} \to 0}{\lim} - \frac{2 i e v_F^2}{c} \frac{k_x}{k_x} \int \frac{d \epsilon}{2 \pi i} \sum_{\vec{p}}  \bl n_F(\epsilon) - n_F(\epsilon + \omega) \br \frac{p \cos \theta}{2} \frac{\partial p}{\partial p_x} \cr
 & & \times \Bigg( \Tr \Big[ \Gbar^{(A)}(\vec{p}, \epsilon) \sigma^y \frac{\partial \Gbar^{(R)}(\vec{p}, \epsilon)}{\partial p} \Gammabar^{(y)}(\vec{p}, \epsilon) \Big] - \Tr \Big[ \frac{\partial \Gbar^{(A)}(\vec{p}, \epsilon)}{\partial p} \sigma^y \Gbar^{(R)}(\vec{p}, \epsilon) \Gammabar^{(y)}(\vec{p}, \epsilon) \Big] \Bigg) \cr
& = & - \frac{2 i e v_F^2}{c} \int \frac{d \epsilon}{2 \pi i}   \bl n_F(\epsilon) - n_F(\epsilon + \omega) \br \frac{1}{2} \cr
& & \times \frac{1}{4} \Big[ \Gamma^2_s(\epsilon) \bl 4 \eta^{00}(\epsilon) - 4(\eta^{00}(\epsilon))^* - 2 \eta^{aa}(\epsilon) + 2 (\eta^{aa}(\epsilon))^* \br \cr
& & + \Gamma^{0}_{p_y}(\epsilon) \bl \eta^{a0}(\epsilon) + \eta^{0a}(\epsilon) - (\eta^{a0}(\epsilon))^* - (\eta^{0a}(\epsilon))^* \br  \cr
& & + i \Gamma^{3}_{p_x}(\epsilon) \bl 3\eta^{0a}(\epsilon) -3 \eta^{a0}(\epsilon) + 3 (\eta^{0a}(\epsilon))^* - 3 (\eta^{a0}(\epsilon))^*   \br \Big] + O(\frac{\gamma}{E_F}) \\
\cr
\pi^{z,(2b2P2)}_{xy}(\vec{k} = 0, \omega) & = &  \underset{\vec{p} \to 0}{\lim} - \frac{2 i e v_F^2}{c} \frac{k_x}{k_x} \int \frac{d \epsilon}{2 \pi i} \sum_{\vec{p}}  \bl n_F(\epsilon) - n_F(\epsilon + \omega) \br \frac{p \cos \theta}{2} \frac{\partial \theta}{\partial p_x} \cr
 & & \times \Bigg( \Tr \Big[ \Gbar^{(A)}(\vec{p}, \epsilon) \sigma^y \frac{\partial \Gbar^{(R)}(\vec{p}, \epsilon)}{\partial \theta} \Gammabar^{(y)}(\vec{p}, \epsilon) \Big] - \Tr \Big[ \frac{\partial \Gbar^{(A)}(\vec{p}, \epsilon)}{\partial \theta} \sigma^y \Gbar^{(R)}(\vec{p}, \epsilon) \Gammabar^{(y)}(\vec{p}, \epsilon) \Big] \Bigg) \cr
 & = & - \frac{2 i e v_F^2}{c} \int \frac{d \epsilon}{2 \pi i}   \bl n_F(\epsilon) - n_F(\epsilon + \omega) \br \frac{1}{2} \cr
 & & \times \frac{1}{4} \Big[ \Gamma^0_{p_y}(\epsilon) \bl \xi^{0a}(\epsilon) - \xi^{a0}(\epsilon) \br - \Gamma^3_{p_x}(\epsilon) \bl  i \xi^{0a}(\epsilon) + i \xi^{a0}(\epsilon) \br \Big] + O(\frac{\gamma}{E_F}) \\
 \cr
 \pi^{z,(2b3)}_{xy}(\vec{k} = 0, \omega) & = & \pi^{z,(2b3P1)}_{xy}(\vec{k} = 0, \omega) + \pi^{z,(2b3P2)}_{xy}(\vec{k} = 0, \omega) \\
\cr
\pi^{z,(2b3P1)}_{xy}(\vec{k} = 0, \omega) & = &  \underset{\vec{k} \to 0}{\lim} - \frac{2 i e v_F^2}{c} \frac{k_y}{k_x} \int \frac{d \epsilon}{2 \pi i} \sum_{\vec{p}}  \bl n_F(\epsilon) - n_F(\epsilon + \omega) \br \frac{p \cos \theta}{2} \frac{\partial p}{\partial p_y} \cr
 & & \times \Bigg( \Tr \Big[ \Gbar^{(A)}(\vec{p}, \epsilon) \sigma^y \frac{\partial \Gbar^{(R)}(\vec{p}, \epsilon)}{\partial p} \Gammabar^{(y)}(\vec{p}, \epsilon) \Big] - \Tr \Big[ \frac{\partial \Gbar^{(A)}(\vec{p}, \epsilon)}{\partial p} \sigma^y \Gbar^{(R)}(\vec{p}, \epsilon) \Gammabar^{(y)}(\vec{p}, \epsilon) \Big] \Bigg) \cr
& = & - \frac{2 i e v_F^2}{c} \int \frac{d \epsilon}{2 \pi i}   \bl n_F(\epsilon) - n_F(\epsilon + \omega) \br \frac{1}{2} \cr
& & \times \frac{1}{4} \Big[ \Gamma^1_s(\epsilon) \bl 2 \eta^{aa}(\epsilon) -  2 (\eta^{aa}(\epsilon))^* \br \cr
& & + \Gamma^{0}_{p_x}(\epsilon) \bl \eta^{a0}(\epsilon) + \eta^{0a}(\epsilon) - (\eta^{a0}(\epsilon))^* - (\eta^{0a}(\epsilon))^* \br  \cr
& & + \Gamma^{3}_{p_y}(\epsilon) \bl \eta^{a0}(\epsilon) - \eta^{0a}(\epsilon) + (\eta^{a0}(\epsilon))^*- (\eta^{0a}(\epsilon))^*  \br \Big] + O(\frac{\gamma}{E_F}) \\
\cr
\pi^{z,(2b3P2)}_{xy}(\vec{k} = 0, \omega) & = &  \underset{\vec{k} \to 0}{\lim} - \frac{2 i e v_F^2}{c} \frac{k_y}{k_x} \int \frac{d \epsilon}{2 \pi i} \sum_{\vec{p}}  \bl n_F(\epsilon) - n_F(\epsilon + \omega) \br \frac{p \sin \theta}{2} \frac{\partial \theta}{\partial p_y} \cr
 & & \times \Bigg( \Tr \Big[ \Gbar^{(A)}(\vec{p}, \epsilon) \sigma^x \frac{\partial \Gbar^{(R)}(\vec{p}, \epsilon)}{\partial \theta} \Gammabar^{(y)}(\vec{p}, \epsilon) \Big] - \Tr \Big[ \frac{\partial \Gbar^{(A)}(\vec{p}, \epsilon)}{\partial \theta} \sigma^x \Gbar^{(R)}(\vec{p}, \epsilon) \Gammabar^{(y)}(\vec{p}, \epsilon) \Big] \Bigg) \cr
 & = & - \frac{2 i e v_F^2}{c} \int \frac{d \epsilon}{2 \pi i}   \bl n_F(\epsilon) - n_F(\epsilon + \omega) \br \frac{1}{2} \cr
 & & \times \frac{1}{4} \Big[ 3 \Gamma^0_{p_x}(\epsilon) \bl \xi^{a0}(\epsilon) - \xi^{0a}(\epsilon) \br - \Gamma^3_{p_y}(\epsilon) \bl  \xi^{0a}(\epsilon) + \xi^{a0}(\epsilon) \br \Big] + O(\frac{\gamma}{E_F})
\ea
\end{subequations}

Therefore, summing up all the different contributions, we finally obtain the SHE correlation function,
%
\ba
\label{eqn: Piz2xy result}
\pi^{z, (2)}(\vec{p} =0, \omega) & = & \frac{2 i e v_F^2}{c} \int \frac{d \epsilon}{2 \pi i} (n_F(\epsilon) - n_F(\epsilon + \omega)) \cr
 & & \times \frac{1}{2} \Bigg[ \Gamma^0_{p_x}(\epsilon) \Big( \xi^{0a}(\epsilon) - \xi^{a0}(\epsilon) \Big) + \Gamma^0_{p_y}(\epsilon) \Big( \xi^{a0}(\epsilon) - \xi^{0a}(\epsilon) \Big) \cr
 & & \Gamma^1_s(\epsilon) \Big( \eta^{00}(\epsilon) - \eta^{aa}(\epsilon) - (\eta^{00}(\epsilon))^* + (\eta^{aa}(\epsilon))^* \Big) \cr
 & & + \Gamma^2_s(\epsilon) \Big( \eta^{aa}(\epsilon) - \eta^{00}(\epsilon) - (\eta^{aa}(\epsilon))^* + (\eta^{00}(\epsilon))^*  \Big) \cr
 & & + \Gamma^3_{p_x}(\epsilon) \Big( i \eta^{a0}(\epsilon) - i \eta^{0a}(\epsilon) + i (\eta^{a0}(\epsilon))^* - i (\eta^{0a}(\epsilon))^* \Big) \cr
 & & + \Gamma^3_{p_y}(\epsilon) \Big( \eta^{0a}(\epsilon) - \eta^{a0}(\epsilon) + (\eta^{0a}(\epsilon))^* - (\eta^{a0}(\epsilon))^* \Big) + O(\frac{\gamma}{\epsilon}) \Bigg] 
\ea

Using the results for $\xi^{ij}(\omega)$ and $\eta^{ij}(\omega)$ from above, where $\xi^{0a}(\omega) - \xi^{a0}(\omega) = - \tfrac{i \pi}{2 \pi v_F^2}$, $Im[ \eta^{aa}(\omega)- \eta^{00}(\omega)] = - \tfrac{1}{2 \pi v_F^2} \tfrac{\pi (\gamma_0 - \gamma_a)}{8 (\gamma_0 + \gamma_a)}$, and $Re[\eta^{0a}(\omega) - \eta^{a0}(\omega)] = \tfrac{1}{2 \pi v_F^2} \tfrac{\pi \omega}{4 (\gamma_0 + \gamma_a)} = \tfrac{\pi N_0(\omega)}{4 (\gamma_0 + \gamma_a)}$, we see that the main $O(\tfrac{1}{\gamma})$ contributions come from the $\Gamma^3_{p_x}(\omega)$ scattering channel.

The uniform DC longitudinal charge and spin-Hall conductivity are given by $\sigma_{yy} = - \Lim{\omega \rightarrow 0} \Lim{\vec{k} \rightarrow 0} Im \left[ \frac{\pi_{yy}(\vec{k}, \omega)}{\omega} \right]$, $\sigma^{z}_{xy} = -\Lim{\omega \rightarrow 0} \Lim{\vec{k} \rightarrow 0} Im \left[ \frac{\pi^{z}_{xy}(\vec{k}, \omega)}{\omega} \right]$, and keeping only the $O(\tfrac{1}{\gamma})$ terms, they are,
%
\ba
\label{eqn: spin charge conductivities}
\sigma_{yy} & = & \frac{1}{2 \pi} \bl e v_F \br^2 Re \Big[ 2 \Gamma^2_0(E_F) \, \xi^{00}(E_F) \Big] \cr
 & = & \bl e v_F \br^2 \frac{N_0(E_F)}{2 \gamma_t} + O \bl \frac{\gamma}{E_F} \br \\
\sigma^{z, (2)}_{xy} & = &  \frac{\hbar e v_F^2}{\pi} Im \Big[ i \Gamma^3_{p_x}(E_F) \big[ Re[ \eta^{0a}(E_F) - \eta^{a0}(E_F)] \big] \Big] \cr
 & = & - \hbar e v_F^2 \frac{N_0(E_F)}{2 \gamma_t} \frac{\gamma_{s}}{\gamma_0 + \gamma_a} + O \bl \frac{\gamma}{E_F} \br  \\
\sigma^y_y & = & \frac{\hbar e v_F}{2 \pi} Re \Big[ 2 \Gamma^2_s(E_F) \xi^{00}(E_F) \Big]\cr
 & = & \hbar e v_F \frac{N_0(E_F)}{2 \gamma_t} + O \bl \frac{\gamma}{E_F} \br 
\ea

Hence, we see that the SHE is driven by scattering between the $s$ and $p$-wave electrons due to the symmetric spin-flip $T^S$ term, which occurs at  $3^{rd}$-order in perturbation. Eq.~\ref{eqn: Gamma3px}, $\Gamma^3_{p_x}(E_F) =  - \frac{\gamma_{s}}{\gamma_t} - i \frac{\gamma_{31} + \gamma_{asym,2}}{\gamma_t} + \frac{\gamma_{3s} \gamma_{asym,1}}{2 \gamma_t (\gamma_0 + \gamma_a)} $, shows that the asymmetric spin-flip term $T^A$ also contributes but as a sub-leading term, .

\bibliography{SHE_bibliography}

\end{document}

