
For both random and ranked tie-breaking rules, we present simple single-threshold strategies for the agents that give them high guarantees. 
A single-threshold strategy specifies some threshold $T$, and selects any reward that exceeds $T$.

For $j=1, \ldots, n$, let $y_j$ be the $j$th highest reward. 

Under the random tie-breaking rule, we show a series of thresholds that have the following guarantee:

\begin{theorem*}(Theorem \ref{thm:prophet_threshold_inequality})
	For every $\ell=1, \ldots, n$, let $T^\ell = \frac{1}{k+\ell}\sum_{j=1}^{\ell} \E [y_j]$. 
	Then, for every agent, the single threshold strategy $T^{\ell}$ (i.e., select $v_t$ iff $v_t\geq T^\ell$) guarantees an expected utility of at least $T^{\ell}$.
\end{theorem*}

Two special cases of the last theorem are where $\ell=1$ and $\ell=k$. 
The case of $\ell=1$ implies that every agent can guarantee herself (in expectation) at least $\frac{1}{k+1}$ of the highest reward. 
The case of $\ell=k$ implies that every agent can guarantee herself (in expectation) at least $\frac{1}{2k}$ of the optimal social welfare (i.e., the sum of the highest $k$ rewards), which also implies that the social welfare in equilibrium is at least a half of the optimal social welfare. 

The above result is tight, as shown in Proposition~\ref{pro:lb_random}.

Similarly, for the ranked tie-breaking rule, we show a series of thresholds that have the following guarantee:
\begin{theorem*}(Theorem \ref{thm:prophet_serial_threshold_inequality})
	For every $i\leq n$ and $\ell=0, \ldots, n-i$, let $\hat{T}_i^\ell=\frac{1}{\ell+2}\sum_{j=i}^{i+\ell} \E [y_j]$. 
Then, for the $i$-ranked agent, the single threshold strategy $\hat{T}_i^\ell$ (i.e., select $v_t$ iff $v_t\geq \hat{T}_i^\ell$) guarantees an expected utility of at least $\hat{T}_i^\ell$.
\end{theorem*}

This result implies that for every $i$, the $i$-ranked agent can guarantee herself (in expectation) at least a half of the $i^{th}$ highest reward.
In Proposition~\ref{pro:lb_serial} we show that the last result is also tight.

%(Theorems \ref{thm:prophet_threshold_inequality} and \ref{thm:prophet_serial_threshold_inequality}).
%We also present matching lower bounds (see Propositions \ref{pro:lb_random} and \ref{pro:lb_serial}).

Finally, we show that under the ranked tie-breaking rule, the equilibrium strategies of the (ordered) agents coincide with the decisions of a single decision maker who may select up to $k$ rewards in an online manner and wishes to maximize the sum of selected rewards.
Thus, the fact that every agent is aware of her position in the ranking allows them to coordinate around the socially optimal outcome despite the supposed competition between them.  
\begin{theorem*}(Corollary \ref{cor:welfare_prophet})
	Under the ranked tie-breaking rule, in every equilibrium of the $k$-agent game the expected social welfare is at least $1-O(\frac{1}{\sqrt{k}})$ of the optimal welfare.
\end{theorem*}

A similar phenomenon was observed in a related secretary setting, where the equilibrium strategy profile of a game with several ranked agents, induces an optimal strategy for a single decision maker who is allowed to choose several rewards and wishes to maximize the probability that the highest reward is selected \cite{matsui2016lower}. 
%\michal{In particular, PLEASE COMPLETE}


