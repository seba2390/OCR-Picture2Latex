%\michal{If submitting to AAAI, say a few sentences about multi-agent systems, e-commerce applications, and how competition is an important component. Check for other prophet papers that were published in AAAI/IJCAI. 
%Emphasize the fact that the first connection between prophet inequality and mechanism design was revealed in the 07 paper.} \tomer{fixed?}


In the classical prophet inequality problem a decision maker observes a sequence of $n$ non-negative real-valued rewards $v_1, \ldots, v_n$ that are drawn from known independent distributions $F_1,\ldots ,F_n$. 
At time $t$, the decision maker observes reward $v_t$, and needs to make an immediate and irrevocable decision whether or not to accept it. If she accepts $v_t$, the game terminates with value $v_t$; otherwise, the reward $v_t$ is gone forever and the game continues to the next round.
The goal of the decision maker is to maximize the expected value of the accepted reward.

This family of problems captures many real-life scenarios, such as an employer who interviews potential workers overtime, renters looking for a potential house, a person looking for a potential partner for life, and so on. More recently, starting with the work of \citet{HajiaghayiKS07}, the prophet inequality setting has been studied within the AI community in the context of market and e-commerce scenarios, with applications to pricing schemes for social welfare and revenue maximization. For a survey on a market-based treatment of the prophet inequality problem, see the survey by~\citet{lucier2017economic}.  
%\ronedit{Prophet settings have interesting and important implications to mechanism design and pricing applications for both welfare and revenue maximization \cite{HajiaghayiKS07,DuettingFKL17,kleinberg2012matroid,EzraFN18}.}


An algorithm $\ALG$ has a guarantee $\alpha$ if  the expected value of   $\ALG$ is at least $\alpha$, where the expectation is taken over the coin flips of the algorithm, and the probability distribution of the input. 
\citet{KrengelS77,krengel1978semiamarts} established the existence of an algorithm that gives a tight guarantee of $\frac{1}{2}\E[\max_i v_i]$.  
Later, it has been shown that this guarantee can also be obtained by a single-threshold algorithm--- an algorithm that specifies some threshold from the outset, and accepts a reward if and only if it exceeds the threshold. 
Two such thresholds have been presented by \citet{samuel1984comparison,KleinbergW19}.  
Single-threshold algorithms are simple and easy to explain and implement. 
%\tomer{fix the citations and single threshold}
%\ron{this is the median threshold. where is the max/2 threshold?}
%\michal{the other threshold appears in the Kleinberg-Weinberg paper on matroid prophets.}
%A single-threshold strategy is the strategy where the decision maker selects  rewards if they exceed the threshold.
%These strategies are simple non-adaptive and are easy to implement and explain.

%In the prophet scenario, the best strategy is calculated using backwards induction, which amounts to a vector of $n-1$ thresholds, $t_1, \ldots, t_{n-1}$, where the last award is always accepted, and otherwise award $v_i$ is accepted iff $v_i \geq t_i$. 
%In the worst-case, this strategy achieves a competitive ratio of $2$, which can also be achieved by a simple single threshold strategy.

\paragraph{Competing Agents.}
Most attention in the literature has been given to scenarios with a single decision maker. 
Motivated by the economic aspects of the problem, where competition among multiple agents is a crucial factor, we introduce a multi-agent variant of the prophet model, in which multiple agents compete over the rewards. 

%In this model, a set of $k$ agents compete over a set of awards that arrive online. %Similar scenarios of competing agents has been studied by \cite{immorlica2006secretary} and \cite{karlin2015competitive} in secretary settings.
In our model, a sequence of $n$ non-negative real-valued rewards $v_1, \ldots, v_n$ arrive over time, and a set of $k$ agents make immediate and irrevocable selection decisions. 
The rewards are unknown from the outset, but every reward $v_t$ is drawn independently from a known distribution $F_t$. 
Upon the arrival of reward $v_t$, its value is revealed to all agents, and every agent decides whether or not to select it.

One issue that arises in this setting is how to resolve ties among agents. That is, who gets the reward if more than one agent selects it. We consider two natural tie-breaking rules; namely, {\em random} tie breaking (where ties are broken uniformly at random) and {\em ranked} tie-breaking (where agents are a-priori ranked by some global order, and ties are broken in favor of higher ranked agents). Random tie-breaking fits scenarios with symmetric agents, whereas ranked tie-breaking fits scenarios where some agents are preferred over others, according to some global preference order. For example, it is reasonable to assume that a higher-position/salary job is preferred over lower-position/salary job, or that firms in some industry are globally ordered from most to least desired. 
Random and ranked tie-breaking rules were considered in \citet{immorlica2006secretary} and \citet{karlin2015competitive}, respectively, in secretary settings.

Unlike the classical prophet scenario, which studies the optimization problem of a single decision maker, the setting of competing agents induces a game among multiple agents, were an agent's best strategy depends on the strategies chosen by others. Therefore, we study the equilibria of the induced games. In particular, we study the structure and quality of equilibrium in these settings and devise simple strategies that give agents high guarantees.

When the order of distributions is unknown in advance, calculating the optimal strategy is computationally hard. This motivates the use of simple and efficiently computed strategies that give good guarantees. 

