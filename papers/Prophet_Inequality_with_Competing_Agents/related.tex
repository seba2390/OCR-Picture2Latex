%secretary and prophet surveys
%The prophet model describes situations where a single decision maker selects an award from a series  of online-arriving awards. 
The prophet problem and variants thereof has attracted a vast amount of literature in the last decade. For comprehensive surveys, see, e.g., the survey by %Hill and Kertz 
\citet{hill1992survey} and the survey by %Lucier 
\citet{lucier2017economic} which gives an economic view of the problem. 

A related well-known problem in the optimal stopping theory is the {\em secretary} problem, where the rewards are arbitrary but arrive in a random order. For the secretary problem a tight $1/e$-approximation has been established; for a survey, see, e.g., \cite{Ferguson89whosolved}.

Our work is inspired by a series of studies that consider scenarios where multiple agents compete over the rewards in secretary-like settings, where every agent aims to receive the highest reward.
\citet{karlin2015competitive} and \citet{immorlica2006secretary} considered the ranked- and the random tie-breaking rules, respectively, in secretary settings with competition.
For the ranked tie-breaking rule, \citet{karlin2015competitive} show that the equilibrium strategies take the form of {\em time-threshold strategies}; namely, the agent waits until a specific time $t$, thereafter competes over any reward that is the highest so far. The values of these time-thresholds are given by a recursive formula.
For the random tie-breaking rule, \citet{immorlica2006secretary} characterize the  Nash equilibria of the game and show that for several classes of strategies (such as threshold strategies and adaptive strategies), as the number of competing agents grows, the timing in which the earliest reward is chosen decreases. This confirms the argument that early offers in the job market are the result of competition between employers.
%For the random tie-breaking rule, \citet{immorlica2006secretary} characterize the (pure and mixed) Nash equilibria of the game and show in two models, one in which only time-threshold strategies are allowed, and a second model where agents have adaptive strategies that depend on previous actions taken by other agents. \michal{add a bit more details. For example, what is the key result? are adaptive strategies better than time-threshold strategies?}

Competition among agents in secretary settings has been also studied by \citet{ezra2020competitive}, in a slightly different model. 
Specifically, in their setting, decisions need not be made immediately; rather, any previous reward can be selected as long as it is still available (i.e., has not been taken by a different agent). Thus, the competition is inherent in the model.

Another related work is the dueling framework by \citet{immorlica2011dueling}. 
One of their scenarios considers a 2-agent secretary setting, where one agent aims to maximize the probability of getting the highest reward (as in the classical secretary problem), and the other agent aims to outperform her opponent. 
They show an algorithm for the second agent that guarantees her a winning probability of at least $0.51$. 
They also establish an upper bound of $0.82$ on this probability. 

Other competitive models have been considered in the optimal stopping theory; see \cite{abdelaziz2007optimal} for a survey. 

%\ron{survey on competing agents in optimal stopping: \citet{abdelaziz2007optimal}. in this survey, most relevant (not exactly our model but similar) is \citet{enns1987multi}(couldn't find online). Newer and might be relevant: \cite{cownden2014effects}}.

The work of \citet{KleinbergW19} regarding matroid prophet problems is also related to our work.
They consider a setting where a single decision maker makes online selections under a matroid feasibility constraint, and show an  algorithm that achieve 1/2-approximation to the expected optimum for arbitrary matroids.
For the special case of uniform matroids, namely selecting up to $k$ rewards, earlier works of \citet{alaei2011bayesian} and \citet{HajiaghayiKS07} shows a approximation of $1-O(\frac{1}{\sqrt{k}})$ for the optimal solution. As mentioned above, the same guarantee is obtained in a setting with $k$ ranked competing agents. 


%\paragraph{Online Matching.} \ron{not sure what to say}