We introduce a model of {\em competing} agents in a prophet setting, where rewards arrive online, and decisions are made immediately and irrevocably.
The rewards are unknown from the outset, but they are drawn from a known probability distribution.
In the standard prophet setting, a single agent makes selection decisions in an attempt to maximize her expected reward.
The novelty of our model is the introduction of a competition setting, where multiple agents compete over the arriving rewards, and make online selection decisions simultaneously, as rewards arrive.
If a given reward is selected by more than a single agent, ties are broken either randomly or by a fixed ranking of the agents. 
The consideration of competition turns the prophet setting from an online decision making scenario to a multi-agent game. 

For both random and ranked tie-breaking rules, we present simple threshold strategies for the agents that give them high guarantees, independent of the strategies taken by others. 
In particular, for random tie-breaking, every agent can guarantee herself at least $\frac{1}{k+1}$ of the highest reward, 
and at least $\frac{1}{2k}$ of the optimal social welfare. 
For ranked tie-breaking, the $i$th ranked agent can guarantee herself at least a half of the $i$th highest reward.
We complement these results by matching upper bounds, even with respect to equilibrium profiles.
For ranked tie-breaking rule, we also show a correspondence between the equilibrium of the $k$-agent game and the optimal strategy of a single decision maker who can select up to $k$ rewards.

%We study prophet settings where rewards arrive in an online manner and decisions are made over time. 
%Unlike the standard setting, where a single agent makes online decisions, we consider settings where multiple agents compete over the rewards. 
%The consideration of competition within these scenarios  turns it from an online decision making setting to a multi-agent game.
%If more than a single agent select an reward, ties are broken either randomly or by ranked (based on a known global ranking of the agents). 
%We study the equilibria that arise in the induced games, and devise simple (threshold) strategies for individual agents that provide them with high guarantees. 