\documentclass[aps,preprint,prl,floatfix]{revtex4-1}
%\documentclass[aps,twocolumn,prl,tightenlines,floatfix,longbibliography]{revtex4-1}
%\documentclass[twocolumn,dvips]{wlscirep}

%\usepackage{lineno}
%\usepackage[switch,columnwise]{lineno}
\usepackage{hyperref}
\usepackage{graphicx}
\usepackage[english]{babel}
\usepackage{amsmath}
\usepackage{amsfonts}
\usepackage{amssymb}
\usepackage{times}

%\newcommand{\affiliation}[1]{\affil{#1}}
\newcommand{\D}{\mathrm{D}}
\newcommand{\p}{\partial}
\newcommand{\Tr}{\mathrm{Tr}}
\renewcommand{\d}{\mathrm{d}}
\renewcommand{\thefigure}{{S}\arabic{figure}}


\begin{document}
%\setpagewiselinenumbers
%\modulolinenumbers[5]
%\linenumbers
%\setlength\columnsep{22pt}

\begin{center}{\Large Supplementary Information \\~ \\ Instability of
  Fulde-Ferrell-Larkin-Ovchinnikov states in three and two dimensions}

\vskip 1ex
{Jibiao Wang, Yanming Che, Leifeng Zhang and Qijin Chen$^*$}\\
\textit{Department of Physics and Zhejiang Institute of Modern
  Physics, Zhejiang University, Hangzhou, Zhejiang 310027, China and\\
 Synergetic Innovation Center of Quantum Information and
  Quantum Physics, Hefei, Anhui 230026, China} 
\end{center}

%\date{\today}

\vskip 2ex

Here we provide more data and plots, which serve as supplemental
information to the main text.
%\end{abstract}

%\pacs{03.75.Ss,03.75.Hh,67.85.-d,74.25.Dw}

%\maketitle

\vskip 4ex

%\section{Derivations of the pair dispersion for the GG scheme of T-matrix approximation }


\section{Pair dispersion in the mean-field FFLO phases from near-BCS through near-BEC regimes}

In this section, we will present more results of the pair dispersion at
high population imbalances in the \emph{mean-field} FFLO phase from
near-BCS through near-BEC regimes.

Starting with the unitary case, Fig.~\ref{fig:3D_Unitary} shows the
pair dispersion $\Omega_\mathbf{p}$ in the FFLO phase with a population
imbalance $\eta = 0.75$ and equal masses at temperature
$T/T_F=0.01$. This is just an alternative 3D plot of Fig.~2 in the
main text, treating the angle $\theta$ between pair momentum
$\mathbf{p}$ and the FFLO wavevector $\mathbf{q}$ as a Descartes
coordinate. This makes it easier to see that the minimum value of
$\Omega_{p}$ (as a function of $p$) decreases as $\theta$ varies from
0 to $\pi$, revealing that the point $\mathbf{p=q}$ is indeed merely a
saddle point of $\Omega_\mathbf{p}$.


\begin{figure}
%\centerline{\includegraphics[clip,width=3.4in]
%  {m1-inva0-p0.75-T0.010-alpha-p-omega-CC-full_2_bitmap.eps}}
\centerline{\includegraphics[clip,width=4.in]
  {m1-inva0-p0.75-T0.010-alpha-p-omega_bm.eps}}
\caption{Alternative 3D plot of the pair dispersion
  $\Omega_\mathbf{p}$ in the FFLO phases at unitarity with population
  imbalance $\eta = 0.75$ and temperature $T/T_F=0.01$.  The
  conventions on color coding and units are the same as in
  Fig.~2 of the main text.}
\label{fig:3D_Unitary}
\end{figure}


Now we show the counterpart plot of the near-BCS and near-BEC cases in
Fig.~\ref{fig:3D_BCS_BEC}, as the left and right panel,
respectively. Despite the different radii of the bottom (half) circle,
both confirms that the $\mathbf{p=q}$ point is a saddle point of
$\Omega_\mathbf{p}$.

\begin{figure}
\centerline{\includegraphics[clip,width=3.2in]{m1-inva-0.5-p0.45-T0.010-3D-half_3.4in.eps}\hfill\includegraphics[clip,width=3.2in]{m1-inva0.1-p0.88-T0.010-3D-half_3.4in.eps}}
%\centerline{\includegraphics[clip,width=3.4in]
%{m1-inva0.1-p0.88-T0.010-3D_3.4in.eps}}
\caption{Pair dispersion $\Omega_\mathbf{p}$ in the FFLO phases for
  the near-BCS and near-BEC case, with $(1/k_Fa, \eta, T/T_F) = (-1/2,
  0.45, 0.01)$ and $(0.1, 0.75, 0.01)$ for the left and right panels,
  respectively. The conventions on color coding and units are the
  same as in Fig.~2 of the main text.}
\label{fig:3D_BCS_BEC}
\end{figure}


%\begin{figure}
%\centerline{\includegraphics[clip,width=3.4in]{m1-inva-0.5-p0.45-T0.010-3D-half_3.4in.eps}}
%\centerline{\includegraphics[clip,width=3.4in]
%{m1-inva-0.5-p0.45-T0.010-3D_3.4in.eps}}
%\caption{Typical pair dispersion $\Omega_\mathbf{p}$ in the FFLO
%  phases in Fig.~1. Shown here is the near-BCS case with $1/k_Fa =
%  -1/2$, $\eta = 0.45$ and $T/T_F=0.01$.  The color coding is such
%  that $\Omega_\mathbf{p}$ increases with the wavelength of the
%  light. The units for energy and momentum are $E_F$ and $k_F$,
%  respectively.}
%\label{fig:3D_BCS}
%\end{figure}

\section{Pair dispersion from the $GG$ and $G_0G_0$ approximations of
  pairing fluctuation theories}

Similar to the $G_0G$ scheme of the $T$-matrix approximation, one can
also extract the pair dispersion from the counterpart $T$ matrix in
the $GG$ and $G_0G_0$ schemes. The derivation is straight forward,
using by setting frequency $\Omega$ to zero in the inverse $T$ matrix,
as discussed above Eq. (7) in the main text. With no doubt, the
coefficient $a_0$ is quantitatively different. To make different plots
comparable in numerical values, we use the $a_0$ from the $G_0G$
scheme to plot the pair dispersion here. Note that one could
equivalently plot $-1/U + \chi(0,\mathbf{p})$ rather than
$\Omega_\mathbf{p}$ instead.  The resulting pair dispersion at
unitarity is shown in Fig.~\ref{fig:GG_NSR_unitary} for the (left)
$GG$ and (right) $G_0G_0$ schemes, respectively. Here values of the
chemical potentials $\mu_\sigma$, the gap $\Delta$, and the vector
$\mathbf{q}$ were the same as in Fig.~2 in the main text for the
$G_0G$ case.

Evidently, the pair dispersion for the $GG$ case is similar to that of
the $G_0G$ case, confirming that the $\mathbf{p=q}$ point is a saddle
point of $\Omega_\mathbf{p}$. In contrast, there is an obvious
difference between the $G_0G_0$ case and the other two; the pair
dispersion has no angle dependence. This can be easily understood
since the pair susceptibility $\chi_0(P) = \sum_K G_0(P-K)G_0(K)$ is
isotropic, independent of the gap $\Delta$ and the wavevector
$\mathbf{q}$. This is, of course, the defect of the
approximation. Nevertheless, this angle independence does suggest that
the pair energy minimizes on a finite momentum sphere so that no
spontaneous symmetry breaking or Bose condensation would take
place for the pairs.

Note that the pair dispersion for the $GG$ and $G_0G_0$ schemes does
not vanish at $\mathbf{p=q}$, because these two schemes are
incompatible with the BCS mean-field gap equation.

\begin{figure}
%\centerline{\includegraphics[clip,width=3.4in]
%  {GG_Invt_p-alpha_unitary.eps}}
%\centerline{\includegraphics[clip,width=3.4in]
%  {GG_Omega_theta_p_half_3.4in.eps}}
\centerline{\includegraphics[clip,width=3.2in]
  {GG_Omega_theta_p.eps}\hfill\includegraphics[clip,width=3.2in]
  {NSR_Omega_theta_p_half_3.4in.eps}}
\caption{Typical pair dispersion $\Omega_\mathbf{p}$ in the mean-field
  FFLO phases for the (left) $GG$ and (right) $G_0G_0$ approximations
  of the pairing fluctuation theories. Shown here is the unitary case
  with $\eta = 0.75$ and $T/T_F=0.01$. The conventions on color coding
  and units are the same as in Fig.~2 of the main text.}
\label{fig:GG_NSR_unitary}
\end{figure}

%\begin{figure}
%\centerline{\includegraphics[clip,width=3.4in]
%  {NSR_Omega_q_unitary.eps}}
%\centerline{\includegraphics[clip,width=3.4in]
%  {NSR_Omega_theta_p_half_3.4in.eps}}
%\caption{Typical pair dispersion $\Omega_\mathbf{p}$ in
%  the FFLO phases in . Shown here is the
%  unitary case with $\eta = 0.75$ and $T/T_F=0.01$.  The color coding
%  is such that $\Omega_\mathbf{p}$ increases with the wavelength of
%  the light. The units for energy and momentum are $E_F$ and $k_F$,
%  respectively.}
%\label{fig:NSR_unitary}
%\end{figure}

%\begin{figure}
%\centerline{\includegraphics[clip,width=3.4in]
%  {m1-inva-0.5-p0.45-p-G0G-GG_2.eps}}
%\caption{Typical pair dispersion $\Omega_\mathbf{p}$ in
%  the FFLO phases in . Shown here is the
%  unitary case with $\eta = 0.75$ and $T/T_F=0.01$.  The color coding
%  is such that $\Omega_\mathbf{p}$ increases with the wavelength of
%  the light. The units for energy and momentum are $E_F$ and $k_F$,
%  respectively.}
%\label{fig:Omegap-alpha_G0G+GG}
%\end{figure}

%\begin{figure}
%\centerline{\includegraphics[clip,width=3.4in]
%  {m1-inva-0.5-p0.45-T0.010-p-GG.eps}}
%\caption{Typical pair dispersion $\Omega_\mathbf{p}$ in
%  the FFLO phases in . Shown here is the
%  unitary case with $\eta = 0.75$ and $T/T_F=0.01$.  The color coding
%  is such that $\Omega_\mathbf{p}$ increases with the wavelength of
%  the light. The units for energy and momentum are $E_F$ and $k_F$,
%  respectively.}
%\label{fig:Omegap-p_GG}
%\end{figure}

\section{Evolution of pair dispersion with a forced LOFF wavevector $\mathbf{q}$}

It is illuminating to show how the pairing dispersion evolves if one
forces and continuously tunes the FFLO wavevector $\mathbf{q}$,
starting from the Sarma solution in a mean-field FFLO phase. This can
be done by solving Eqs. (2)-(4) in the main text, without
Eq. (5). Here we work with our own pairing fluctuation theory, i.e.,
the $G_0G$ approximation. The result is shown in
Fig.~\ref{fig:Omegap-p_evolution} for $\eta = 0.6$ at unitarity. For
$q=0$, i.e., the Sarma state, the pair dispersion vanishes at zero
momentum, and is isotropic with no angle dependence.  The solution for
$q/k_F=0.578$ corresponds to the mean-field FF solution. As one can
see, angle dependence develops as $q$ increases from 0. In all finite
$q$ cases, the minimum pair energy along the $\mathbf{q}$ direction is
the highest among all angles. And therefore, these finite $q$ FFLO
states are unstable against pairing fluctuations.

\begin{figure}
\centerline{\includegraphics[clip,width=4.5in]
  {m1-inva0-p0.6-T0.010-p-q-evolution_2.eps}}
\caption{Evolution of the pair dispersion $\Omega_\mathbf{p}$ at
  different angles in the mean-field FFLO phases with increasing
  wavevector $q$, as labeled. Shown here is the unitary case with
  $\eta = 0.6$ at $T/T_F=0.01$.  }
\label{fig:Omegap-p_evolution}
\end{figure}

%\section{Pair dispersion on an optical lattice}
%
%On a 3D optical lattice, the pair dispersion will minimize on a 2D constant
%energy surface. On a 2D optical lattice, it will minimize on a 1D
%constant energy circle. The latter case can be illustrated in
%Fig.~\ref{fig:Q2D}. In 
%
%\begin{figure}
%\centerline{\includegraphics[clip,width=3.4in]
%  {Omegaq2.eps}}
%\caption{Typical pair dispersion $\Omega_\mathbf{p}$ in
%  the FFLO phases in . Shown here is the
%  unitary case with $\eta = 0.75$ and $T/T_F=0.01$.  The color coding
%  is such that $\Omega_\mathbf{p}$ increases with the wavelength of
%  the light. The units for energy and momentum are $E_F$ and $k_F$,
%  respectively.}
%\label{fig:Q2D}
%\end{figure}

%\bibliographystyle{apsrev} 
%\bibliographystyle{naturemag} 
\bibliographystyle{pnas-new} 
%\bibliography{Review3}


\end{document}
