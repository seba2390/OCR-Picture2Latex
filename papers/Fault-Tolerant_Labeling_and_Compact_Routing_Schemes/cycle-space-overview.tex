\section{Overview of the Cycle Space Sampling Technique} \label{sec:cycle_space_overview}

The cycle space sampling technique allows to detect cuts in a graph using a connection between cuts and cycles in a graph.
This beautiful technique was introduced by Pritchard and Thurimella \cite{pritchard2011fast}, that showed its applicability for distributed algorithms identifying small cuts in a graph. We next give a short overview of the technique, for full details see \cite{pritchard2011fast}.

The \emph{cycle space} of a graph is the family of all subsets of edges $F$ that have even degree at each vertex, any such subset $\phi \subseteq E$ is called a \emph{binary circulation}. The \emph{cut space} is the family of all induced edge cuts. It is easy to see that if we take a cycle $C$ in a graph and an induced edge cut, then the number of edges of the cycle that cross the cut is even. The cycle space technique extends this observation and shows that the cycle space and cut space are orthogonal vector spaces. Using this, they show the following (see Propositions 2.2 and 2.5 in \cite{pritchard2011fast}).

\begin{claim} \label{claim_cycle} 
Let $\phi$ be a uniformly random binary circulation and $F \subseteq E$. Then
$$Pr[|F \cap \phi| \ is \ even] = \left\{
                \begin{array}{ll}
                  1,\ if\ F\ is\ an\ induced\ edge\ cut\\
                  1/2,\ otherwise
                \end{array}
              \right. $$ 
\end{claim}   

Hence, sampling a random binary circulation allows to detect if a subset of edges is an induced edge cut with probability $1/2$. To reduce the failure probability to $1/2^b$ we can choose $b$ random binary circulations. To use this technique, the authors provide an efficient way to sample a random binary circulation, we describe next. Let $T$ be a spanning tree of the graph. For any non-tree edge $e$, adding $e$ to the graph creates a cycle. These cycles are the \emph{fundamental cycles}, and it is shown that the \emph{fundamental cycles} are a basis for the cycle space. Based on this, they show that sampling a random binary circulation can be done by choosing each fundamental cycle with probability $1/2$, or equivalently choosing each non-tree edge with probability $1/2$. The binary circulation $\phi$ sampled has all the non-tree edges sampled, and each tree edge that appears in odd number of sampled cycles. Given the sampled non-tree edges in $\phi$, the tree edges in $\phi$ can be identified using a simple scan of the tree, as shown in \cite{pritchard2011fast}. Choosing $b$ random binary circulations, is equivalent to choosing a $b$-bit random string $\phi(e)$ for each non-tree edge. For a tree edge $t$, we define $\phi(t) = \oplus_{e \in C_t} \phi(e)$, where $C_t$ are all non-tree edges $e$ such that $t$ is in the fundamental cycle of $e$. This again can be computed by a simple scan of the tree, and takes $O((n+m)b)$ time if the labels have size $b$. This gives the following.

\cycle*

To see this, let $\phi_1,...,\phi_b$ be the sampled binary circulations. If $F$ is an induced edge cut, then from Claim \ref{claim_cycle}, for every sampled circulation $\phi_i$, we have that $|F \cap \phi_i|$ is even, and hence for all $i$, the $i$'th bit of $\Moplus_{e \in F} \phi(e)$ is equal to 0 as needed. Otherwise, for all $i$, the $i$'th bit $\Moplus_{e \in F} \phi(e)$ equals $0$ with probability $1/2$, hence the probability that the whole vector equals $0$ is $1/2^b$, as needed.