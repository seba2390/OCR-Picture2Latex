\subsection{Our Results}
We provide space-efficient labeling and routing schemes for any $n$-vertex graph. Our schemes are \emph{randomized} and provide a high probability guarantee\footnote{As standard, we use the term high-probability to indicate success guarantee of $1-1/n^c$ for any given constant $c>1$.} for any given triplet $\langle s, t, F\rangle$. In other words, the schemes can faithfully support polynomially many queries\footnote{The same type of guarantee is provided in the centralized sensitivity oracles, e.g., of \cite{DuanConnectivitySODA17}. Providing a high probability guarantee over all possible triplets is possible upon increasing the space bound by a factor of $f$ (largest number of faults supported).}. 

Our first key result presents two independent schemes for FT connectivity labels. These are the first FT connectivity labels for \emph{general graphs}.
%\mtodo{The part that starts here and discusses the techniques may fit better in the technical overview, we have there a short paragraph on the topic, can consider merge them to one paragraph there.} 
%The first construction is based on the cycle-space sampling technique of Pritchard and Thurimella \cite{pritchard2011fast}, which have been applied in the past mainly in the context of computing small cuts in the distributed setting. 
%The second construction is based on the tool of linear sketches by Ahn et al. \cite{ahn2012analyzing}, which have many applications in the context of connectivity computation under various computational settings, e.g., \cite{kapron2013dynamic,kapralov2014spanners,GibbKKT15,DBLP:conf/podc/KingKT15,DBLP:conf/wdag/MashreghiK18,GhaffariP16,DuanConnectivitySODA17}.
% The most related application to our setting is in the computation of connectivity sensitive oracles by Duan and Pettie \cite{DuanConnectivitySODA17}. Whereas their construction employs linear sketches in the form of global and centralized data-structures, our labeling schemes use these sketches in a form of local labels. Altogether, 
These two constructions yield the following theorem, addressing Question \ref{q:label}: 

\begin{theorem}\label{thm:conn-labels}[FT Connectivity Labeling Schemes, Informal]
For any $n$-vertex graph and a bound $f$ on the number of edge faults, there is a \emph{randomized} $f$-FT connectivity labeling scheme with label length of $O(\min\{f+\log n, \log^3 n\})$ bits. The labels are computed in $\widetilde{O}(m)$ time, and the decoding algorithm takes $\poly(f,\log n)$ time. 
\end{theorem}

By the tightness of the label length of fault-free connectivity labels, %\mtodo{is there an easy way to see this lower bound? In any case, would be good to elaborate a bit more and maybe cite a relevant paper.}\mertodo{I could not find a reference so here is the argument I have in mind. Consider an $n$-vertex graph with $\Theta(n)$ connected components. Just to encode the identifier of the component we need $\Theta(\log n)$ bits, thus $\Theta(n \log n)$ bits in total. The collection of all $n$ connectivity labels should encode the connected components of $G$ as well: by making all $n^2$ queries we deduce the components. Therefore, each label should have $\Omega(\log n)$ bits.} 
our scheme is optimal for $f=O(\log n)$. Moreover, the label length is nearly-optimal for any $f$. Our actual scheme provides more information then merely a single bit (connected or not connected). Specifically, we augment the connectivity labels with additional information so that the decoding algorithm, given the labels of $s,t$ and $F$, can also output a succinct description of an $s$-$t$ path in $G\setminus F$ (if such a path exists). This succinct path representation finds applications in the context of our FT routing schemes. 
%\mtodo{I think it's enough to say that we output a description of a path, and it's vital for routing. The sentence that starts here seems too detailed for here, but maybe can be added in the technical overview?} 
%In particular, the scheme is based on some initial spanning forest $T$ \mtodo{we always talk about it as a tree} in the graph, and in the case where $s$ and $t$ are disconnected in $T \setminus F$, but are still connected in $G \setminus F$, the decoding algorithm computes a set of recovery edges that restores the connectivity of $s$ and $t$. This extra information will be vital for our routing schemes.
%
%\textbf{MP: we might want to mention, in the technical sections, or here, the bounds obtained for providing a high probability guarantee for all $n^{\Theta(f)}$ possible queries. I think this increases the length by a $O(f)$ factor.}

We next consider the task of reporting also approximate $s$-$t$ distances in $G\setminus F$ using the labels of $s,t$ and $F$. We employ the reduction of Chechik et al. \cite{chechik2012f} to convert the FT connectivity labels into FT approximate distance labels, providing nearly the same space vs. stretch tradeoff as in the centralized data-structures of \cite{chechik2012f}. Specifically, we show:

\begin{theorem}\label{thm:dist-labels}[FT Approximate Distance Labeling Schemes]
For any $n$-vertex (possibly weighted) graph, a bound $f$ on the number of edge faults, and a stretch parameter $k$, there is a randomized $f$-FT approximate distance labeling scheme with label length of $O(k \cdot n^{1/k}\cdot \log (nW) \cdot \log^3 n)$. Given the labels of $s,t$ and $F$ the scheme returns a distance estimate 
$$\dist_{G\setminus F}(s,t)\leq \delta(s,t,F) \leq (8k-2)(|F|+1)\dist_{G\setminus F}(s,t)~.$$
\end{theorem}

For the purpose of routing, we exploit the extra information provided by our connectivity labels, in order to output, in addition to the distance estimate $\delta(s,t,F)$, also a succinct description of the approximate $s$-$t$ 
shortest path in $G \setminus F$. Our second key result provides FT compact routing schemes, with an almost optimal tradeoff between the space and stretch, for constant number of faults $f$. We answer Question \ref{q:route} by showing:

\begin{theorem}\label{thm:routing}[FT Compact Routing]
For every integers $k,f$, there exists an $f$-sensitive compact routing scheme that given a message $M$ at the source vertex $s$ and the routing label of the destination $t$, in the presence of at most $f$ faulty edges $F$ (unknown to $s$) routes $M$ from $s$ to $t$ in a distributed manner over a path of length at most $32k (|F|+1)^2\cdot \dist_{G \setminus F}(s,t)$. The routing labels have $\widetilde{O}(f)$ bits, the table size of each vertex is $\widetilde{O}(f^3 \cdot n^{1/k} \log(nW))$, the header size (also known as message size) is bounded by $\widetilde{O}(f^3)$ bits. 
\end{theorem}
This improves over the state-of-the-art construction of Chechik \cite{chechik2011fault} that 
obtained routing schemes with stretch of $O(f^2(f+\log^2 n)k)$ and tables of size $O(\deg(v)n^{1/k}\log{(nW)})$ for every vertex $v$. We note that the construction of Chechik \cite{chechik2011fault} has a bounded global space of $\widetilde{O}(n^{1+1/k} \log{(nW)})$, but the individual tables might have even super-linear space (e.g., when $k=O(1)$ and $\deg(v)=O(n)$).  For the special case of $f=2$, Chechik et al. \cite{ChechikLPR10,chechik2012f} provide a stretch bound of $O(k)$, and total space of $\widetilde{O}(n^{1+1/k} \log{(nW)})$, where the space of each table is bounded by $O(\deg(v)n^{1/k})$, thus super-linear in the worst case. Our scheme provides an improved bound on the individual tables, nearly matching the fault-free constructions for $f=O(1)$. We also show an improved scheme if one only aims to optimize for the global space, rather than optimizing for the largest table size for a vertex. For comparison of our results to prior work see Table \ref{table_routing}. %\mtodo{Added this table, I currently mention also \cite{rajan2012space}, see if we also want to mention it in the text here (not sure, as it's already a bit long). Also, from looking at \cite{chechik2012f,chechik2011fault} one of the papers seems to discuss only space per vertex and one discusses global space, so I wrote it like this in the table, but if you prefer to mention the space per vertex and global space for all schemes we can edit the tables accordingly. In our results I mentioned both the scheme with better global space, and the scheme with better space per vertex.}

\begin{table}[h!]
\centering
%\begin{tabular}{ |p{4cm}|p{1.3cm}|p{4.5cm}|p{2.6cm}|p{2.8cm}|  }
\begin{tabular}{ |p{4.2cm}|p{4cm}|p{6cm}|p{0.5cm}|}
 \hline	
 \multicolumn{4}{|c|}{Constructions of Fault-Tolerant Routing Schemes}\\
 \hline
  Reference & Stretch & Table Size & $|F|$ \\
 \hline
  Rajan \cite{rajan2012space} & $O(k^2)$ & $\widetilde{O}(k \deg(v)+ n^{1/k})$ per vertex & 1\\
  Chechik et al. \cite{chechik2012f} & $O(k)$ & $\widetilde{O}(n^{1+1/k} \log(nW))$ total size & 2\\
  Chechik \cite{chechik2011fault}  & $O(|F|^2(|F|+\log^2 n)k)$ & $\widetilde{O}(n^{1+1/k} \log(nW))$ total size & $f$\\
  Chechik \cite{chechik2011fault}  & $O(|F|^2(|F|+\log^2 n)k)$ & $\widetilde{O}(\deg(v)n^{1/k} \log(nW))$ per vertex & $f$\\
  \textbf{Here} & $O(|F|^2 k)$ & $\widetilde{O}(f \cdot n^{1+1/k} \log(nW))$ total size & $f$\\
  \textbf{Here} & $O(|F|^2 k)$ & $\widetilde{O}(f^3 \cdot n^{1/k} \log(nW))$ per vertex & $f$\\
 \hline
 % \multicolumn{4}{|c|}{} \\
 %\hline
 \end{tabular}
 \caption{Comparison between FT routing schemes with a set of failures $F$}
\label{table_routing}
\end{table}

Finally, we provide a lower bound result on the minimal stretch regardless for the \emph{space} of the routing scheme, e.g., even if all vertices store all the graph edges. 

\begin{theorem}[Stretch Lower-Bound for FT Routing]\label{thm:lb-routing}
Any FT routing randomized scheme resilient to $f$ faults induces an expected stretch of $\Omega(f)$ regardless of the size of the routing tables and labels. In particular, this holds even if each routing table contains a complete information on the graph. 
\end{theorem}


\paragraph{Open Problems.} Our work leaves several interesting open ends. One natural direction is to provide labeling and routing schemes resilient to \emph{vertex} faults. The major challenge in handling vertex failure is that even a single faulty vertex might disconnect the graph into $\Omega(n)$ disconnected components. Another interesting direction is to derandomize our constructions. Currently there are no deterministic constructions of FT labeling schemes for general graphs. Finally, it will be also important to provide FT distance approximate labeling schemes whose stretch bound is independent in the number of faults $f$. This problem is also open in the corresponding setting of approximate distance sensitive oracles. 
