\section{Fault-Tolerant (FT) Connectivity Labels}

%\mtodo{decide about FS vs FT.}
%\mtodo{also, for later: vertex - node, $(u,v)$ - $\{u,v\}$} \mertodo{Changed to FT and vertices.}

We next discuss two labeling schemes for connectivity that are based on two different approaches. The first one uses the \emph{cycle space sampling} technique to try to find cuts that disconnect $s$ and $t$. The second one uses \emph{graph sketches} to try to find a path that connects $s$ and $t$. Since the second approach allows to find a path between $s$ and $t$ if exists, it is also useful later for routing. In terms of label size, the first approach gives labels of size $O(f + \log{n})$, which is near-optimal if the number of failures is $f=O(\log{n})$. On the other hand, the second scheme gives labels of size $O(\log^3{n})$, which is better when the number of failures is large.
We next discuss the labeling schemes. During this section, we assume that the input graph $G$ is connected. If not, we can add to the label of each vertex and edge the id of their connected component in $G$, and apply the labeling scheme to each one of the connected components separately. 

\subsection{Connectivity Labels Based on Cycle Space Sampling}

\subsubsection{The Labeling Algorithm}

Our labels are composed of two ingredients, that we review next.

\paragraph{Cycle Space Labels.}
The cycle space sampling technique, introduced in \cite{pritchard2011fast}, allows to give the edges of a graph short labels that allow to detect cuts in the graph. For a set of vertices $S$, $\delta(S)$ is the set of edges with exactly one endpoint in $S$. A subset of edges $F$ is called an \emph{induced edge cut} if $F = \delta(S)$ for some $S$.
The following is shown in \cite{pritchard2011fast} (see Corollary 2.9). 

 \cycle*

\remove{
\begin{restatable}{lemma}{cycle} \label{cycle_space_lemma}
There is an algorithm that assigns the edges of a graph $G=(V,E)$ $b$-bit labels $\phi(e)$ such that given a subset of edges $F \subseteq E$, we have:
$$Pr[\Moplus_{e \in F} \phi(e) = 0] = \left\{
                \begin{array}{ll}
                  1,\ if\ F\ is\ an\ induced\ edge\ cut\\
                  2^{-b},\ otherwise
                \end{array}
              \right. $$ 
Where $0$ is the all-zero vector. The time complexity for assigning the labels is $O((m+n)b)$.
\end{restatable}
}

%The algorithm for assigning the labels is simple. First, we fix a spanning tree. For each non-tree edge $e$, the label $\phi(e)$ is a random $b$-bit string. For any tree edge $t$, the label of $t$ is defined as follows. We say that an edge $e=\{u,v\}$ \emph{covers} $t$ if $t$ in the unique $u-v$ path in the tree. We denote by $C_t$ all the non-tree edges that cover $t$. Then, $\phi(t) = \oplus_{e \in C_t} \phi(e).$ 
For an overview of the technique, see Appendix \ref{sec:cycle_space_overview}. 
In our algorithm, given a subset of edges $F$ of size at most $f$, we want to be able to check for any subset $F' \subseteq F$ if $F'$ is an induced edge cut. To support all these $2^f$ queries w.h.p we choose $b=f+ c \log{n}$ for a constant $c$. This guarantees that the probability of error is at most $\frac{2^f}{2^{f+c\log{n}}}=\frac{1}{n^c}$. This will guarantee that given a query $\langle s,t,F \rangle$, our algorithm answers correctly w.h.p. We remark that if we increase the size of labels to $O(f \log{n})$ we can get an algorithm that is correct for \emph{all} queries w.h.p. 
The reason is that we can then check for any subset of edges $F$ of size at most $f$ if $F$ is an induced edge cut. As the number of subsets of size at most $f$ is bounded by $O(n^f)$, we get that the labels are correct for all such subsets w.h.p.

%In our case, we want to be able to check for any subset of edges $E'$ of size at most $f$ if $E'$ is an induced edge cut. For this, we choose $b=\Theta(f \log{n})$. As the number of subsets of size at most $f$ is bounded by $O(n^f)$, we get that the labels are correct for all such subsets w.h.p.  

\paragraph{Ancestry Labels.} Our second ingredient are ancestry labels for trees.
To use them, we first fix a spanning tree $T$ of the graph rooted at $r$. The goal is to assign vertices short labels, such that given the labels of $u$ and $v$, we can infer if $u$ is an ancestor of $v$ in $T$. A simple labeling scheme based on a DFS scan solves the problem with labels of size $2 \lceil \log{n} \rceil$ per vertex \cite{kannan1992implicat}, the time for assigning the labels is $O(n)$ for the DFS scan of the tree. Labeling schemes with improved label size appear in \cite{abiteboul2006compact,alstrup2002improved,fraigniaud2010compact,fraigniaud2010optimal}.

\begin{lemma} \label{anc_labels}
For every tree $T$, there is an algorithm that assigns the vertices $u$ of the tree labels $\LCALabel_T(u)$ of $O(\log{n})$ bits, such that given the labels of $u$ and $v$ we can infer if $u$ is an ancestor of $v$ in $T$ in $O(1)$ time. The time for assigning the labels is $O(n)$. 
\end{lemma}

\paragraph{The Final Labels.}
Our final labels contain the following ingredients:
\begin{enumerate}
\item The label of the edge $e=(u,v)$ is composed of $(\phi(e),\LCALabel_T(u),\LCALabel_T(v),j)$, where $j$ is a bit indicating if $e$ is a tree edge in $T$. In total, the label size is $O(f + \log{n})$.
\item The label of a vertex $v$ is its ancestry label $\LCALabel_T(v)$ of size $O(\log{n})$ bits.
\end{enumerate}

As discussed, the time for assigning the labels is $O((m+n)b)=\widetilde{O}((m+n)f)$, as $b=f+c\log{n}$.
We next explain how we use these labels to check FT connectivity.


\subsubsection{The Decoding Algorithm}

We next discuss several observations that allow us to check if $s$ and $t$ are disconnected by $F$.

\begin{claim} \label{obs_induced}
The vertices $s$ and $t$ are disconnected by $F$ if an only if there is an induced edge cut $F' \subseteq F$ that disconnects $s$ and $t$.
\end{claim}

\begin{proof}
First, if $F' \subseteq F$ disconnects $s$ and $t$, then clearly $F$ disconnects $s$ and $t$.
On the other hand, if $s$ and $t$ are disconnected by $F$, let $F' \subseteq F$ be a minimal set of edges whose removal disconnects $s$ and $t$. We show that $F'$ is an induced edge cut. Let $V_s$ be the vertices in the connected component of $s$ in $G \setminus F'$. We show that all edges in $F'$ are between $V_s$ and $V \setminus V_s$, implying that $F'$ is an induced edge cut. Assume to the contrary that there is an edge $e \in F'$ with both endpoints in one of the sides, say $V_s$, then $F' \setminus \{e\}$ is still a cut that disconnects $s$ and $t$ (as $V_s$ is still disconnected from the rest of the graph if we add $e$), contradicting the minimality of $F'$. A symmetric argument shows that $e$ cannot have both its endpoints in $V \setminus V_s$.
\end{proof}

We next show that given an induced edge cut $F'$, there is a simple way to determine the two sides of the cut induced by $F'$ (see Figure \ref{cutSidesPic} for illustration). For a vertex $v$ and an induced edge cut $F'$, we denote by $n_v(F')$ the number of edges from $F'$ in the path from the root $r$ to $v$ in the spanning tree $T$. We show the following.

\remove{
\begin{claim}
Let $F'$ be an induced edge cut, and let $T$ be a spanning tree with root $r$. Let $V_0$ be all the vertices $v$ where in the path from $r$ to $v$ there is an even number of edges from $F'$, and let $V_1 = V \setminus V_0$. Then $(V_0,V_1)$ is the induced edge cut defined by $F'$.
\end{claim}
}

\begin{claim} \label{obs_cut_sides}
Let $F'$ be an induced edge cut. Let $$V_0=\{v \in V|\ n_v(F')\ is \ even \},$$ $$V_1=\{v \in V|\ n_v(F')\ is \ odd \}.$$ Then $(V_0,V_1)$ is the induced edge cut defined by $F'$.
\end{claim}

\begin{proof}
Since $F'$ is an induced edge cut, the endpoints of every edge in $F'$ are on different sides of the cut. Hence, if we scan the tree $T$ from the root to the leaves, every time we reach an edge from $F'$ we change the side of the cut. It follows that one side of the cut contains all vertices $v$ such that $n_v(F')$ is even, and the other side has all vertices $v$ such that $n_v(F')$ is odd. Hence $V_0,V_1$ are the two sides of the cut.
\end{proof}

\setlength{\intextsep}{0pt}
\begin{figure}[h]
\centering
\setlength{\abovecaptionskip}{-2pt}
\setlength{\belowcaptionskip}{6pt}
\includegraphics[scale=0.55]{cutSides2.pdf}
 \caption{Here $F'=\{e_1,e_2,e_3,e_4\}$ is an induced edge cut. On the right, you can see the partition into sides in the tree. Every time we reach an edge from $F'$, we change the side of the cut.}
\label{cutSidesPic}
\end{figure}

From Claims \ref{obs_induced} and \ref{obs_cut_sides}, we get the following.

\begin{corollary} \label{cor_ft_connectivity}
The vertices $s$ and $t$ are disconnected by $F$ if an only if there is an induced edge cut $F' \subseteq F$, such that one of the values $n_s(F'),n_t(F')$ is even and the other is odd. 
\end{corollary}

This gives a simple approach to detect if $s$ and $t$ are disconnected by $F$. We go over all subsets $F' \subseteq F$, for each one of them we first check if $F'$ is an induced edge cut using the cycle space labels. Second, if $F'$ is an induced edge cut, we compute the values $n_s(F'),n_t(F')$, if the number is even for one of them and odd for the second, we deduce that $F'$ disconnects $s$ and $t$. Note that we can use the ancestry labels to compute the values $n_s(F'),n_t(F')$. For example, for computing $n_s(F')$ we should check how many edges in $F'$ are in the tree path between $r$ to $s$. For this, for each tree edge $e=(u,v)$ in $F'$, we check if it is above $s$ in the tree, which happens if and only if both $u$ and $v$ are ancestors of $s$.
This simple approach requires time exponential in $|F|$ for going over all subsets of $F$, we next show a faster way to check the same condition.

\subsubsection{Faster Decoding Algorithm}

We next show that checking the condition from Corollary \ref{cor_ft_connectivity} boils down to solving a system of linear equations.
First, note that from Lemma \ref{cycle_space_lemma}, w.h.p, a set of edges $F' \subseteq F$ is an induced edge cut iff $\Moplus_{e \in F'} \phi(e) = 0$. Hence, if we want to check if there is a non-empty subset $F' \subseteq F$ that is an induced edge cut it is equivalent to checking if there exists a binary vector $x=(x_1,...,x_f) \neq 0$ such that $\Moplus_{1 \leq i \leq f} x_i \phi(e_i) = 0$, where $\{e_1,...e_f\}$ are the edges of $F$. Or equivalently checking if the vectors $\{\phi(e)\}_{e \in F}$ are linearly dependant. To check the condition from Corollary \ref{cor_ft_connectivity}, we generalize this idea. 

Let $b = O(f + \log{n})$ be the size of the cycle space labels.
Given a triplet $(s,t,F)$, we assign for each edge $e \in F$, a binary vector $\phi'(e)$ of length $b+2$, as follows.
\begin{enumerate}
\item If $e$ is a tree edge which is in the tree path $r-s$ but not in the path $r-t$, then $\phi'(e)=10\phi(e)$.
\item If $e$ is a tree edge which is in the tree path $r-t$ but not in the path $r-s$, then $\phi'(e)=01\phi(e)$.
\item In all other cases, $\phi'(e)=00\phi(e).$ 
\end{enumerate}

We denote by $w_1,w_2$ binary vectors of length $b+2$ such that $w_1=100..0,w_2=010...0$ (all right entries are equal to 0).
We show that the condition from Corollary \ref{cor_ft_connectivity} holds iff there is a binary vector $x=(x_1,...,x_f)$ and $j \in \{1,2\}$ such that $$\Moplus_{1 \leq i \leq f} x_i \phi'(e_i) = w_j.$$ This holds iff there is a solution to at least one of the linear systems $Ax=w_1,Ax=w_2$, where $A$ is a $(b+2) \times f$ matrix that has the vectors $\{\phi'(e)\}_{e \in F}$ as its column vectors, and $x,w_1,w_2$ are column vectors. All operations are modulo 2.

\begin{lemma}
With high probability, the vertices $s$ and $t$ are disconnected by $F$ if an only if there is a binary vector $x=(x_1,...,x_f)$ and $j \in \{1,2\}$ such that $\Moplus_{1 \leq i \leq f} x_i \phi'(e_i)  = w_j.$
\end{lemma}

\begin{proof}
We assume for the proof that the cycle space labels are correct, i.e., a set of edges $F' \subseteq F$ is an induced edge cut iff $\Moplus_{e \in F'} \phi(e) = 0$. This happens w.h.p from Lemma \ref{cycle_space_lemma} and the choice of $b=O(f + \log{n})$.

First we show that if $s$ and $t$ are disconnected by $F$, the condition of the lemma holds.
From Corollary \ref{cor_ft_connectivity}, $s$ and $t$ are disconnected by $F$ iff there is an induced edge cut $F' \subseteq F$, such that one of the values $n_s(F'),n_t(F')$ is even and the other is odd. Denote by $n'_s(F')$ the number of edges from $F'$ in the $r-s$ tree path that are not in the $r-t$ path, and denote by $n'_t(F')$ the number of edges from $F'$ in the $r-t$ tree path that are not in the $r-s$ path. Note that if one of the values $n_s(F'),n_t(F')$ is even and the other is odd, then also one of $n'_s(F'),n'_t(F')$ is even and the other is odd, as if we denote by $y$ the number of edges from $F'$ that are in both $r-s$ and $r-t$, we get that $n'_s(F') = n_s(F') - y, n'_t(F') = n_t(F') - y$. Assume first that $n'_s(F')$ is even and $n'_t(F')$ is odd. Let $x$ be the characteristic vector of $F'$. We show that $\Moplus_{1 \leq i \leq f} x_i \phi'(e_i)  = w_2$. First, as $F'$ is an induced edge cut, we have that $\Moplus_{e \in F'} \phi(e) = 0$. Hence, the $b$ last bits of $\Moplus_{1 \leq i \leq f} x_i \phi'(e_i)$ are equal to 0 as needed. $F'$ has even number of edges that are in the path $r-s$ and not $r-t$, as the labels $\phi'(e)$ of all these edges start in $10$, the XOR of the first 2 bits of these edges sums to $00$.  $F'$ has odd number of edges that are in the path $r-t$ but not $r-s$. The labels of all these edges start in $01$, as there is an odd number of them, the XOR of the first 2 bits of these edges sums to $01$. All other edges have labels that start in $00$, hence the XOR of their first 2 bits sums to $00$. Overall we get that $\Moplus_{1 \leq i \leq f} x_i \phi'(e_i)=\Moplus_{e \in F'} \phi'(e)=010...0=w_2$. The case that $n'_s(F')$ is odd and $n'_t(F')$ is even is symmetric and results in the equation $\Moplus_{1 \leq i \leq f} x_i \phi'(e_i)=100...0=w_1.$

On the other hand, if we have that $\Moplus_{1 \leq i \leq f} x_i \phi'(e_i)=w_j$ for a binary vector $x=(x_1,...,x_f)$ and $j \in \{1,2\}$, we can build from it $F'$ that satisfies the condition in Corollary \ref{cor_ft_connectivity}, as follows. We define $F'$ to be all edges $e_i \in F$ such that $x_i =1$. Since $\Moplus_{1 \leq i \leq f} x_i \phi'(e_i)=\Moplus_{e \in F'} \phi'(e)=w_j$, we have that $\Moplus_{e \in F'} \phi(e) = 0$, hence $F'$ is an induced edge cut. Additionally if $w_j=w_2$, it implies that the XOR of the first 2 bits of labels $\{\phi'(e)\}_{e \in F'}$ are equal to $01$. By the definition of the labels, this can only happen if $n'_s(F')$ is even and $n'_t(F')$ is odd. Similarly, if $w_j=w_1$, then $n'_s(F')$ is odd and $n'_t(F')$ is even. In both cases we get that one of the values $n_s(F'),n_t(F')$ is even and the other is odd, hence $s$ and $t$ are disconnected by $F$ from Corollary \ref{cor_ft_connectivity}. 
\end{proof}

To conclude, the question if $s$ and $t$ are disconnected by $F$ boils down to checking if there is a solution to at least one of the linear systems $Ax=w_1,Ax=w_2$, where $A$ is a $(b+2) \times f$ matrix, and $b=O(f + \log{n})$. Note that we can construct the labels $\phi'(e)$ and hence the matrix $A$ given the labels of $s,t,F$. For this, we need the labels $\phi(e)$ of edges in $F$, and also to distinguish for each edge in $F$ if it is in the $r-s,r-t$ paths in the tree. The latter can be deduced from the ancestry labels of $s,t,F$ and from the bits indicating which edges in $F$ are tree edges. A tree edge $e=(u,v) \in F$ is in the $r-s$ path iff both $u$ and $v$ are ancestors of $s$, this can be checked in $O(1)$ time using the ancestry labels of $u,v,s$. Hence we can build the matrix $A$ in $O(fb)$ time. To check if the linear systems have a solution we can use Gaussian elimination, that takes $O(MN^2)$ time for $M \times N$ matrix, in our case this is $O((f+\log{n})f^2)$. Alternatively, we can use $O(N^{\omega})$ algorithms for $N \times N$ matrices, where $\omega$ is the exponent of matrix multiplication.
For this, we add zero columns to our matrix $A$ to make it a $(b+2) \times (b+2)$ matrix $A'$ and increase the length of $x$ to $b+2$, the new system $A'x=w_i$ has a solution iff the original system $Ax=w_i$ has a solution. The complexity here is $O((b+2)^{\omega})=O((f+\log{n})^{\omega})$.
This gives the following. 
 
\begin{theorem}
There is a randomized $f$-FT connectivity labeling scheme that assigns the edges and vertices of the graph labels of size $O(\log{n})$ bits per vertex and $O(f + \log{n})$ bits per edge. The decoding time of the scheme is $\min\{O((f+\log{n})f^2),O((f+\log{n})^{\omega})\}$. The time complexity for assigning the labels is $\widetilde{O}((m+n)f).$
\end{theorem}  

\remove{
\begin{theorem}
We can assign the edges and vertices of the graph labels of size $O(\log{n})$ bits per vertex and $O(f + \log{n})$ bits per edge, such that given the labels of $(s,t,F)$ we can check if $s$ and $t$ are disconnected by $F$ in $O(f^3 \log{n})$ \mtodo{check the complexity} time, w.h.p.
\end{theorem} 
} 