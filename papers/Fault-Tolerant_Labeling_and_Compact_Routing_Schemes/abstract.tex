\begin{abstract}
The paper presents fault-tolerant (FT) labeling schemes for general graphs, as well as, improved FT routing schemes. %which significantly improve over the state-of-the-art. 
For a given $n$-vertex graph $G$ and a bound $f$ on the number of faults, an $f$-FT connectivity labeling scheme is a distributed data structure that assigns each of the graph edges and vertices a short label, such that given the labels of a vertex pair $s$ and $t$, and the labels of at most $f$ failing edges $F$, one can determine if $s$ and $t$ are connected in $G \setminus F$. The primary complexity measure is the length of the individual labels. Since their introduction by [Courcelle, Twigg, STACS '07], compact FT labeling schemes have been devised only for a limited collection of graph families. In this work, we fill in this gap by proposing two (independent) FT connectivity labeling schemes for general graphs, with a nearly optimal label length. This serves the basis for providing also FT approximate distance labeling schemes, and ultimately also routing schemes. Our main results for an $n$-vertex graph and a fault bound $f$ are:
\begin{itemize}
\item There is a randomized FT connectivity labeling scheme with a label length of $O(f+\log n)$ bits, hence optimal for $f=O(\log n)$. This scheme is based on the notion of cycle space sampling [Pritchard, Thurimella, TALG '11].

\item There is a randomized FT connectivity labeling scheme with a label length of $O(\log^3 n)$ bits (independent of the number of faults $f$). This scheme is based on the notion of linear sketches of [Ahn et al., SODA '12]. % and inspired by the (centralized) fault-tolerant connectivity oracle of Duan and Pettie (SODA '17).

\item For a given stretch parameter $k\geq 1$, there is a randomized routing scheme that routes a message from $s$ to $t$ in the presence of a set $F$ of faulty edges (unknown to $s$) over a path of length $O(|F|^2 k)\cdot \dist_{G\setminus F}(s,t)$. The routing labels have $\widetilde{O}(f)$ bits, the messages have $\widetilde{O}(f^3)$ bits, and each routing table has only $\widetilde{O}(f^3 n^{1/k})$ bits\footnote{Throughout the paper, we use the notation $\widetilde{O}$ to hide poly-logarithmic in $n$ terms.}. The results also holds for weighted graphs with positive polynomial weights.
\end{itemize}
This significantly improves over the state-of-the-art bounds by [Chechik, ICALP '11], providing the first scheme with sub-linear FT labeling and routing schemes for general graphs. 


%that obtained routing schemes with a stretch of $O(f^2(f+\log n)k)$ and tables of size $O(\deg(v)n^{1/k})$ for every vertex $v$. %(where possibly $\deg(v)=\Theta(n)$). 
%
%that obtained routing schemes with stretch of $O(f^2(f+\log n)k)$ and tables of size $O(\deg(v)n^{1/k})$ for every vertex $v$ (where possibly $\deg(v)=\Theta(n)$). For a constant number of faults, our routing scheme provides a nearly optimal space vs. stretch tradeoff by nearly matching the fault-free solutions. 


%
%In this paper we provide two randomized schemes for FT-connectivity labeling schemes. Our first scheme is based on the notion of cycle-space sampling. It assign labels of size $O(f+\log n)$ bits, thus optimal for $f=O(\log n)$. Our second scheme is based on the notion of linear sketches, and assigns labels of size $O(\log^3 n)$ bits.
%We also extend these labels to the related notion approximate distance labeling scheme. 
%
%Finally, we present fault-tolerant compact routing schemes that for a given stretch parameter $k$, computes routing labels and tables of $\widetilde{O}(f^2)$ and $\widetilde{O}(f^2n^{1/k})$ bits respectively. 
%It then routes a message from a given source $s$ to its destination $t$ in the presence of at most $f$ faulty edges $F$ (unknown to $s$). The length of the $s$-$t$ path is bounded by $O(f^2 k)\cdot \dist_{G\setminus F}(s,t)$. 

\end{abstract}

\newpage