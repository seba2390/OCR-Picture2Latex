\section{Fault-Tolerant Approximate Distance Labels}\label{sec:ft-distance}
Given integer parameters $f,k \geq 1$, an $(f,k)$ \emph{FT approximate distance labeling scheme} assigns labels $\FTDistLabel: V \cup E \to \{0,1\}^{q}$ such that given the labels of $s,t$ and a subset $F \subseteq E$, $|F|\leq f$, there exists a decoding algorithm that outputs a distance estimate $\delta_{G \setminus F}(s,t)$ satisfying:
$$\dist_{G\setminus F}(s,t) \leq \delta_{G \setminus F}(s,t) \leq k\cdot\dist_{G\setminus F}(s,t)~.$$

We next show that there is an efficient transformation from any FT connectivity labeling scheme into an FT approximate distance labeling scheme. This transformation increases the label size by a multiplicative factor of $\widetilde{O}(n^{1/k})$. This technique was first introduced by \cite{chechik2012f} in the context of distance sensitivity oracles, and it is based on the notion of tree covers.

\begin{definition}[Tree Covers]\label{def:tree-cover}
Let $G=(V,E)$ be an undirected graph with edge weights $\omega$, and let $\rho,k$ be two integers. Define $B_{\rho}(v)=\{ u \in V ~\mid~ \dist_G(u,v)\leq \rho\}$. A tree cover $\TreeCover(G, \omega, \rho,k)$ is a collection of rooted trees $\mathcal{T}=\{T_1,\ldots, T_\ell\}$ with root $r(T)$ for every $T \in \mathcal{T}$ such that:
\begin{enumerate}[noitemsep]
\item For every vertex $v$ there exists a tree $T \in \mathcal{T}$ such that $B_{\rho}(v) \subseteq T$.
\item The radius of each tree $T$ is at most $(2k-1)\cdot \rho$.
\item Each vertex participates in $(k \cdot n^{1/k})$ trees.
\end{enumerate}
Let $|\TreeCover(G, \omega, \rho,k)|$ denote the number of trees in the tree cover $\TreeCover(G, \omega, \rho,k)$.
\end{definition}

\begin{proposition}\cite{Peleg:2000}
For any $n$-vertex graph $G=(V,E, \omega)$, and any parameters $\rho,k$, one can compute tree covers $\TreeCover(G, \omega, \rho,k)$ in time $\widetilde{O}(|E(G)| \cdot n^{1/k})$.
\end{proposition}

\begin{lemma}[From Connectivity Labels to Approximate Distance Labels]\label{lem:reduction}
Let $G=(V,E, \omega)$ be a weighted undirected $n$-vertex graph where $\omega(e)\in [1,W]$, and let 
$\FTConnLabel: V \cup E \to \{0,1\}^s$ be an $f$-FT connectivity labeling scheme for $G$ with decoding time $t$. Then for every integer $k\geq 1$, there is an $(f,(8k-2)(|F|+1))$ FT approximate distance labeling scheme $\FTDistLabel: V \cup E \to \{0,1\}^{q}$ for $G$, where $q=O(s \cdot k \cdot n^{1/k}\cdot \log (nW))$, and with decoding time $\widetilde{O}(t \log{(nW)})$. %\mertodo{add complexity of computation time and decoding time.} \mtodo{added a short description. For the computation time we don't get a simple formula for a general connectivity scheme, hence I didn't add it to the statement of the lemma.}
\end{lemma}
%\begin{proof}

\paragraph{The labeling algorithm.}
For every vertex $u$, the label $\FTDistLabel(u)$ consists of $K=\log(nW)$ sub-labels of FT connectivity labels in distinct subgraphs of $G$ defined as follows. The $i^{th}$ sub-label addresses all distances that are at most $2^i$ in $G$. Let $H_i$ be set of heavy edges in $G$ of weight at least $2^i$, and define the $i^{th}$ tree-cover by 
\begin{equation}\label{eq:TC-i}
\TreeCover_i=\TreeCover(G\setminus H_i,\omega, 2^i,k)~.
\end{equation}
For each tree $T_{i,j} \in \TreeCover_i$, the algorithm applies the FT connectivity scheme on the graph $G_{i,j}=G[V(T_{i,j})]$. For every vertex $u$ and $i \in \{1,\ldots, K\}$, let $i^*(u)$ be an index of a tree in $\TreeCover_i$ that covers the $2^i$-ball of $u$. I.e., $B_{2^i}(v) \subseteq T_{i,i^*(u)}$. 
The label of every $u \in V$ is then given by:
$$\FTDistLabel(u)=\{\langle \FTConnLabel_{G_{i,j},T_{i,j}}(u), i,j \rangle ~\mid~ i \in [1,K], j \in \{1,\ldots, |\TreeCover_i|\}, u \in G_{i,j}\} \bigcup \{i^*(u) ~\mid~ i \in [1,K]\}~.$$

Similarly, the label of each edge $e \in G$ contains the FT connectivity label of $e$ in each of the instances $(G_{i,j}, T_{i,j})$:
$$\FTDistLabel(e)=\{\langle \FTConnLabel_{G_{i,j},T_{i,j}}(e), i,j \rangle ~\mid~ i \in [1,K], j \in \{1,\ldots, |\TreeCover_i|\}, e \in G_{i,j}\}.$$ 

The time for assigning the labels is the time for constructing the tree cover and computing the indexes $i^{*}(v)$, and the time for assigning the connectivity labels on each one of the trees. The first part requires polynomial time. The second depends on the connectivity labels. For example, using our scheme from Section \ref{sec:ftconn-sketch} the time complexity of the second part is $\widetilde{O}(mn^{1/k})$, as it is linear in the total number of vertices and edges in the trees. %\mtodo{added a short discussion of the computation time, I'm not sure in how much detail we want to discuss this, as we don't get a simple formula for the complexity for a general connectivity scheme.}

%\textbf{MP: I define now the FT-connectivity labels to be based on $G$ and a spanning tree $T$, since in the distance reduction the tree is already given. This is important as the connectivity labels of tree edges and non-tree edges differ.}


\paragraph{The decoding algorithm.}
Consider the query $\langle s,t,F\rangle$. 
The algorithm has $K$ phases, in each phase $i \in [1,K]$ the decoding algorithm of the FT connectivity labels is applied on the instance $G_{i,i^*(s)}, T_{i,i^*(s)}$ where $G_{i,i^*(s)}$ contains the $2^i$ ball of $s$ in $G$. 
If $t \notin G_{i,i^*(s)}$, the phase $i$ ends and we continue to phase $i+1$. 
Otherwise, the algorithm decides if $s$ and $t$ are connected in $G_{i,i^*(s)} \setminus F$ in the following manner. 
Let $F_i=F \cap G_{i,i^*(s)}$, this subset of edges can be obtained from the labels of the $F$ edges. 
Since the labels of $s,t$ and $F_i$ contain the FT connectivity labels in the subgraph $G_{i,i^*(s)}$ and the tree $T_{i,i^*(s)}$, the algorithm can apply the decoding algorithm of the FT connectivity scheme.
If $s$ and $t$ are indeed connected in $G_{i,i^*(s)}\setminus F_i$, the algorithm returns the estimate $\delta_{G \setminus F}(s,t)= (4k-1) \cdot (|F|+1) \cdot 2^{i}$. Otherwise, it proceeds to the next phase.

Overall, let $i$ be the minimum index in $\{1,\ldots, K\}$ for which $s$ and $t$ are connected in the subgraph $G_{i,i^*(s)} \setminus F$. Then the decoding algorithm returns the distance estimate $\delta_{G \setminus F}(s,t)=(4k-1) \cdot (|F|+1) \cdot 2^{i}$. If no such $i$ exists, the decoding algorithm returns $\delta_{G \setminus F}(s,t)=\infty$, which implies that $s$ and $t$ are not connected in $G \setminus F$. 

The decoding time is $\widetilde{O}(t \log{(nW)})$, where $t$ is the decoding time of the connectivity labels, as we use the decoding algorithm of the connectivity labels $K$ times on the graphs $G_{i,i^*(s)}$. To obtain this, we need to make sure that given the labels of $s,t,F$ we can easily find their connectivity label in the graph $G_{i,i^*(s)}$ if exist. This can be easily done if we store the connectivity labels in a sorted order. %\mtodo{added a short description for the decoding time.}

\paragraph{Analysis.}
We now analyze the construction, and start by bounding the size of the labels. By the properties of the tree-cover in Def. \ref{def:tree-cover}, each vertex appears in $O(K \cdot k \cdot n^{1/k})$ subgraphs. Thus, $\FTDistLabel(u)$consists of $O(K \cdot k n^{1/k})$ FT connectivity labels and the label size is bounded by $O(K \cdot k n^{1/k} \cdot s)$ bits, as desired. Next, we show correctness. By the correctness of the FT connectivity labeling scheme, it is sufficient to show the following. Let $P_{s,t,F}$ be an $s$-$t$ shortest path in $G \setminus F$ of length $(2^{i-1}, 2^{i}]$. By the properties of the tree cover, there is a tree $T_{i,i^*(s)} \in \TreeCover_i$ that contains all the vertices of the path $P_{s,t,F}$. Therefore, we have that $s$ and $t$ are connected in $G_{i,i^*(s)}\setminus F$. Since the labels of $s, t$ and $F_i=F \cap G_{i,i^*(s)}$ contain the FT connectivity labels in $G_{i,i^*(s)}$, we get that the distance estimate returned by the algorithm satisfies that
$$\dist_{G \setminus F}(s,t)\leq \delta_{G \setminus F}(s,t) \leq (4k-1)(|F|+1)  \cdot 2^i \leq (8k-2)(|F|+1) \cdot \dist_{G \setminus F}(s,t)~.$$
To see this, let $j \leq i$ be the first index such that $s$ and $t$ are connected in $G_{j,j^*(s)}\setminus F$. The algorithm returns the estimate $(4k-1)(|F|+1) \cdot 2^j \leq (4k-1)(|F|+1) \cdot 2^i = (8k-2)(|F|+1) \cdot 2^{i-1} \leq (8k-2)(|F|+1) \cdot \dist_{G \setminus F}(s,t)$. To prove the left inequality, we show that if $s$ and $t$ are connected in $G_{j,j^*(s)}\setminus F$, there is indeed a path between them in $G \setminus F$ of length at most $\delta_{G \setminus F}(s,t) = (4k-1)(|F|+1) \cdot 2^j$. First, from the tree cover properties, the radius of $T_{j,j^*(s)}$ is at most $(2k-1)2^j$, implying that any two vertices in $T_{j,j^*(s)}$ are at distance at most $(4k -2) \cdot 2^j$ from each other. Now the graph $T_{j,j^*(s)} \setminus F$ has at most $|F|+1$ connected components. Since $G_{j,j^*(s)}\setminus F$ is connected, it implies that there is a path between $s$ and $t$ in $G_{j,j^*(s)}\setminus F$. This path traverses at most $|F|+1$ different components in $T_{j,j^*(s)} \setminus F$, and at most $|F|$ edges connecting them, each one of weight at most $2^j$. As the diameter of each component is bounded by $(4k - 2) \cdot 2^j$, the length of the path is at most $(4k - 2) \cdot 2^j \cdot (|F|+1) + 2^j \cdot |F| \leq (4k - 1) \cdot 2^j \cdot (|F| + 1)$, as needed.
%The left inequality follows as the algorithm applies the FT-connectivity labeling algorithm on subgraphs of $G$. The claim follows.

\remove{
%\begin{corollary}\label{cor:FTDistance}
%Consider a triplet $s,t,F$ such that $s$ and $t$ are connected in $G \setminus F$. 
%The decoding algorithm can be modified to output a labeled $s$-$t$ path $\widehat{P}$ of length $O(f)$ that provides a succinct description of an $s$-$t$ path in $G \setminus F$, along with indices $i,j$. Each $G$-edge of $\widehat{P}$ is augmented with port information, and each non-tree edge $e=(u,v)$ corresponds to a $u$-$v$ path in $T_{i,j} \setminus F$. In addition, the length of the $s$-$t$ path encoded by $\widehat{P}$ is bounded by $O(k f)\dist_G(s,t)$. 
%\end{corollary}
Let $\mathcal{T}=\bigcup_{i=1}^{K}\TreeCover_i$ be the collection of tree covers with $K=O(\log (nW))$ scales of distances. We call an edge $(u,v)$ a \emph{tree edge} if it appears on at least one of the trees in $\mathcal{T}$. 
Using Lemma \ref{lem:useful-recovery-edges}, we have the following decoding algorithm, which becomes useful in the context of routing schemes, as described in the next section. \mtodo{there could be a problem with treating edges globally as tree edges or non-tree edges, I think for each edge $e \in G_{i,j}$ we need to know its label in this graph (either tree or non-tree label), extended identifiers are also different for different trees because of the ancestry labels.}
\begin{lemma}\label{lem:approx-dist-recovery}
Consider the $(f,(8k-2)(f+1))$ approximate distance labels $\FTDistLabel$ obtained by using Lemma \ref{lem:reduction} with the FT connectivity labeling scheme of Sec. \ref{sec:ftconn-sketch}. For any triplet $s,t,F \subseteq E$ and $|F|\leq f$, let $F_T$ be the tree edges of $F$. Then, given the labels $\{\FTDistLabel(w), w \in \{s,t\} \cup F_T\}$ and the extended edge identifiers of $F \setminus F_T$, the decoding algorithm can be modified to return a labeled $s$-$t$ path $\widehat{P}$ of length $O(f)$ that provides a succinct description of an $s$-$t$ path in $G \setminus F$, along with indices $i,j$. Each $G$-edge $e$ of $\widehat{P}$ is augmented with port information and the extended identifier of $e$; and each non-$G$ edge $e'=(u,v)$ corresponds to a $u$-$v$ path in $T_{i,j} \setminus F$. In addition, the length of the $s$-$t$ path encoded by $\widehat{P}$ is bounded by $(8k-2)(|F|+1)\cdot \dist_{G\setminus F}(s,t)$. 
\end{lemma}
\begin{proof}



\end{proof}
}
