\subsection{Additional Related Work}

\paragraph{Fault-Tolerant Labeling Schemes.} FT labels for connectivity were introduced by \cite{courcelle2007forbidden} under the term \emph{forbidden-set labeling}. Forbidden set refers to a subset $F$ of at most $f$ edges, such that given the labels of $s,t$ and $F$ one should determine if $s$ and $t$ are connected in $G \setminus F$. The forbidden edge set can be treated in this context as faulty edges\footnote{For routing, the forbidden-set scheme is slightly weaker than FT scheme as explained later.}.
Previous works study FT connectivity labels only in restricted graph families. For example, Courcelle et al. \cite{CourcelleT07} presented a labeling scheme with logarithmic label length for the families of $n$-vertex graphs with bounded clique-width, tree-width and planar graphs. For $n$-vertex graphs with doubling dimension at most $\alpha$, Abraham et al. \cite{AbrahamCGP16} designed FT labeling schemes with label length $O((1 + 1/\epsilon)^{2\alpha}\log n)$ that output $(1+\epsilon)$ approximation of the shortest path distances under faults. Recently, \cite{DBLP:journals/corr/abs-2102-07154} studied FT exact distance labels in planar graphs, and show that any directed weighted planar graph admits fault-tolerant distance labels of size $O(n^{2/3})$.

\paragraph{Connectivity and Distance Sensitivity Oracles.} 
Connectivity and distance sensitivity oracles are centralized data structures that support connectivity or distance queries in the presence of failures. 
The first construction of connectivity sensitivity oracles was given by Patrascu and Thorup \cite{patrascu2007planning} providing an $S(n)=\widetilde{O}(fn)$ space oracle that answers $\langle s,t, F \rangle$ connectivity queries in $\widetilde{O}(f)$ time. The state-of-the-art bounds of these oracles are given by Duan and Pettie \cite{DuanConnectivitySODA17}.
Chechik et al. \cite{chechik2012f} presented the first randomized construction of distance sensitivity oracle resilient to $f$ edge faults. 
Specifically, for any $n$-vertex weighted graph, stretch parameter $k$, and a fault bound $f$, they provide a data-structure with $O(f k n^{1+1/k}\log(nW))$ space, query time of $\widetilde{O}(|F|)$, and $O(f k)$ stretch, where $W$ is the weight of the heaviest edge in the graph. Their solution is based on an elegant transformation that converts the FT connectivity oracle of \cite{patrascu2007planning} into an FT approximate distance oracle.

While the main focus of this paper is in approximate distances, sensitivity oracles that report (possibly near) exact distances under faults have been studied also thoroughly in e.g., \cite{demetrescu2002oracles,bernstein2008improved,duan2009dual,WeimannY10,GrandoniW12,ChechikCFK17,van2019sensitive}. Since reporting exact distances requires linear label length already in the fault-free setting \cite{gavoille2004distance}, we focus on the approximate relaxation, where there is still hope to obtain labels of polylogarithmic length.

\paragraph{Fault-Tolerant Routing Schemes.}
The first formalization of FT routing schemes was given by the influential works of Dolev \cite{dolev1984new} and Peleg \cite{peleg1987fault}. These earlier works presented the first non-trivial solutions for general graphs supporting at most $\lambda$ faulty edges, where $\lambda$ is the edge-connectivity of the graph. Their routing labels had linear size, providing $s$-$t$ routes of possibly linear length (even in cases where the surviving $s$-$t$ path is of $O(1)$ length). In competitive FT routing schemes, it is required to provide $s$-$t$ routes of length that competes with the shortest $s$-$t$ path in $G \setminus F$, even in cases where $G \setminus F$ is not connected. Competitive FT routing schemes \cite{peleg2009good} for general graphs were given by Chechik et al. \cite{ChechikLPR10,chechik2012f} for the special case of $f\leq 2$ faults. 
Specifically, for a given stretch parameter $k$, they gave a routing scheme with a total space bound of $\widetilde{O}(n^{1+1/k})$ bits, polylogarithmic-size labels and messages, and a routing \emph{stretch} of $O(k)$. 
This scheme was extended later on for any $f$ by Chechik \cite{chechik2011fault}, at the cost of increasing the routing stretch to $O(f^2(f+\log^2 n)k)$. For a single edge failure, \cite{rajan2012space} showed a routing scheme with routing tables of size $\widetilde{O}(k \deg(v)+ n^{1/k})$ size per vertex, $O(k^2)$ stretch and $O(k+\log{n})$ size header.

\paragraph{Forbidden Set Routing.}
A more relaxed setting of FT routing scheme which has been studied in the literature is given by the \emph{forbidden set routing schemes}, introduced by Courcelle and Twigg \cite{CourcelleT07}. In that setting, it is assumed that the routing protocol knows in advance the set of faulty edges $F$. In contrast, in the FT routing setting, the failing edges are a-priori unknown to the routing algorithm,  and can only be detected upon arriving one of their endpoints. Forbidden set routing schemes have been devised to the same class of restricted graph families as obtained for the forbidden set labeling setting \cite{CourcelleT07,AbrahamCGP16,abraham2012fully}.