The increasingly sophisticated at-home screening systems for obstructive sleep apnea (OSA), integrated with both contactless and contact-based sensing modalities, bring convenience and reliability to remote chronic disease management. However, the device pairing processes between system components are vulnerable to wireless exploitation from a non-compliant user wishing to manipulate the test results. This work presents SIENNA, an insider-resistant context-based pairing protocol. SIENNA leverages JADE-ICA to uniquely identify a user's respiration pattern within a multi-person environment and fuzzy commitment for automatic device pairing, while using friendly jamming technique to prevent an insider with knowledge of respiration patterns from acquiring the pairing key. Our analysis and test results show that SIENNA can achieve reliable ($>$ 90\% success rate) device pairing under a noisy environment and is robust against the attacker with full knowledge of the context information.

% the Multi physiological radar monitoring systems (PRMSs) based on millimeter-wave (mmWave) Continuous-wave (CW) Doppler radar is an emerging technology that promises convenience and unrestricted in-home tests for obstructive sleep apnea (OSA). Despite its proven reliability in the detection of distinct identifiers that distinguish one patient from another, multi-modality OSA screening, which is reliant on unique respiratory pattern of the subject, remains vulnerable to an anomalous user that poses a threat to the patient's privacy.  This work presents SIENNA: a secure and dynamic framework capable of distinguishing a patient's respiratory pattern among multiple subjects and modalities, and preserving the exchange of private user data between selected modalities and mobile OSA screening.

% that adapt to changes between a traditional contact-based compliance tracker and the PRMS; prevents ambiguity during modality transitions through the patient's unique, instantaneous breathing patterns.

% We performed formal security analysis, implemented, and evaluated MASCOT through laboratory and field experiments consisting of 20 subjects and spanning over one month. Results show MASCOT achieves similarly tracking reliability as a contact-based tracker and strong security against both internal and external attacks.