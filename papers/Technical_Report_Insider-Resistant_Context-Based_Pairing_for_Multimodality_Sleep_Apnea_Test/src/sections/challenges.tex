% One of the reasons to employ RF-based patient identification is to minimize design overhead. Existing non-contact sleep monitoring methods already rely on decimeter or mmWave radar technologies to track the patients' physiological patterns \cite{baboli_good_2015,yang_vital_2017}, and some of the measurements, such as cardiovascular and respiration dynamics, are sufficiently distinctive among individuals to support identity verification \cite{rahman_doppler_2018,shi_contactless_2018,islam_identity_2019}. The authentication system can exploit these built-in sensing functions instead of relying on additional credentials. 

%Comparing to exist single-modality identity verification approach, the focus of our design is to enable continuous tracking and verification of the patient undergoing the sleep test, and allow the operation modalities to change between a on-body compliance tracker and a mmWave radar, e.g., a modality-switch compliance tracking system for at-home OSA screening. By doing so, the test can adapt to the patient's physical activities and minimize uncomfortable and inconvenient settings, without creating unsupervised periods that render the test vulnerable to cheating and data forgery. 
%There are several unique challenges to realize such a design.
%targeting the initial pairing phase or the modality switching operations. 
%Finally, since the verification data are stored on the system and later examined by a doctor, their confidentiality and integrity might be compromised by one(s) with hardware hacking experiences \cite{liu_continuous_nodate}. 
A respiratory belt and a PRMS need to pair with the user's mobile phone before an OSA test.  A respiratory belt is often paired with the user's phone by a medical technician during a clinic visit. The PRMS is usually shipped directly to the user's home and paired without supervision. The unsupervised pairing process is subject to exploitation from a non-compliant user. Assuming the respiratory belt has successfully paired with the phone,  we aim to enable automatic device paring between the PRMS and the phone via the shared context information, e.g., the user's respiratory patterns observed by the belt and PRMS. Unlike previous works on context-based zero-effort pairing, our pairing protocol must pair two devices securely in the presence of a co-located adversary who can also observe the context information.

% \begin{figure}[h]
% \centering
% \begin{subfigure}[t]{0.23\textwidth}
% \includegraphics[width=\textwidth]{figures/Attack-Eve.png}
% \end{subfigure}
% \hspace{\fill}
% \begin{subfigure}[t]{0.23\textwidth}
% \includegraphics[width=\textwidth]{figures/Attack-Spoof.png}
% \end{subfigure}
% \caption{(a) Eavesdrop: a non-compliant user seeks to eavesdrop on the initial pairing between the PRMS \& the mobile device; (b) Spoofing: a non-compliant user leverage the eavesdropped key to inject false data to the mobile device.}
% \label{fig:preliminary}
% \end{figure}


%\begin{figure*}[t]
%\begin{subfigure}[t]{0.24\textwidth}
%\includegraphics[width=\textwidth]{figures/ChestbandDoctor.png}
%\end{subfigure}
%\hspace{\fill}
%\begin{subfigure}[t]{0.24\textwidth}
%\includegraphics[width=\textwidth]{figures/OnBed.png}
%\end{subfigure}
%\hspace{\fill}
%\begin{subfigure}[t]{0.24\textwidth}
%\includegraphics[width=\textwidth]{figures/InBed.png}
%\end{subfigure}
%\hspace{\fill}
%\begin{subfigure}[t]{0.24\textwidth}
%\includegraphics[width=\textwidth]{figures/OutofRange.png}
%\end{subfigure}
%\caption{Left to right: (a) Doctor places chestband on patient \& pairs with app; (b) At home: the modality switch \& key evolution made possible by shared observation of patient from two sentinels; (c) mmWave radar handles compliance tracking \& data collection of patient; (d) App alerts user if they are nearing the range limit of the mmWave radar.}
%\label{fig:adversary model}
%\end{figure*}

\subsection{System Model}
We consider a multimodality OSA screening system with three modules: (1) A mobile phone that aggregates the screening data, a PRMS, and a wireless respiratory belt, and assume the following. (1) Wireless interface: The phone, PRMS, and belt are equipped with radio interfaces such as Bluetooth. (2) Computation: The PRMS and belt can perform computational inexpensive cryptographic algorithms, such as SHA-256 hash and AES. (3) Tamper-proof: The phone, PRMS, and belt are tamper-proof. Any attempts to physically modify the circuit would nullify the test. (4) Security: The phone, PRMS, and belt do not have any prior security associations. Secret keys are established between the belt and the user's phone by a medical technician (Fig. \ref{fig:sienna overview}a).
%Revision for System Model below

%\subsection{System Model}
%Consider a typical setting for an in-home OSA screening with a compliance tracking feature. The simplified system model contains the following modules: (1) a chest band tracker: a device attached by a doctor to record the patient's breathing. It also serves as the initial proof of identity for a patient who wishes to undergo the OSA screening. (2) an unobtrusive OSA screening system with a mmWave radar capable of measuring the patient's breathing wirelessly; (3) a computer base system (e.g. smartphone) that records and processes the respiratory data from the two previous modules and transmits the test results to remote healthcare providers. 

%In our design, from the time the patient puts on the chest band tracker till the test concludes, at least one of the modules, e.g., the tracker or the radar, must be able to detect the patient's breathing. If both modules fail to observe the patient's breathing or record unexplained abnormalities in the patient's breathing patterns, the test will later be nullified by the doctor upon review.

%We make the following assumptions in terms of function, security, computation, and communication. (1) The chest band tracker and the OSA screening system contains tamper-resistant storage's with limited capacity. Any attempts to physically modify the content within the storage would nullify the test. (2) Both the chest band tracker and the OSA screening system are capable of performing computational inexpensive cryptographic algorithms, such as XOR, hashing, and modular exponentiation. (3) Both the chest band tracker and the OSA screening system are equipped with radio interfaces. (4) Both devices are in in possession of each others' public keys but do not have any prior key agreement to secure the wireless communication channel.




% \subsection{Modality Switch}
% Ideally, our design should enable the following test procedure. Once being recommended to undergo the OSA screening, the patient first visits his/her doctor to obtain the test authorization. The doctor installs a chest band sensor on the patient. 

% Once the patient is within the view of the mmWave radar, the chest band sensor notifies the patient with the possibility of a modality switch. If the patient gives the consent, the chest band sensor and OSA screening system attempt the modality switch. If the switch is successful, the chest band sensor uploads the recording to the OSA screening system and allows the patient to remove it. The OSA screening system carries on the recording task as the test progresses.

% If the patient wishes to leave the bedroom or be out-of-range of the mmWave radar, the OSA screening system reminds the patient to put on the chest band sensor. Once the sensor is reattached, the OSA screening system and the chest band sensor automatically attempts the modality switch. If the switch is successful, the chest band sensor notifies the patient and resume recording patient's respiratory patterns. 


\subsection{Adversary Model}
A distinguishing feature of our adversary model is that the system's legitimate user could also be an insider attacker (non-compliant user). The attacker's objective is to either eavesdrop on the communication between the system modules or manipulate the system into accepting false data.

\textbf{Eavesdrop.} A non-compliant user may seek to eavesdrop on the pairing communication between the PRMS and the phone, aiming to extract the security context. For instance, the patient may intercept the key exchanged between the PRMS and the phone to decrypt and review all data records before a doctor examines. If patterns related to OSA symptoms were found, the patient would have time to come up with false excuses.

\textbf{Spoofing.} A non-compliant user may leverage the eavesdropped key to transmit false data to the mobile device and manipulate the OSA test outcome. For instance, the patient may use a third device to collect normal patterns prior to the test and replay the normal data records to the phone during the test.

% \textbf{Surrogate.} A non-compliant patient may use a machine or human surrogate to substitute for the test. For instance, the patient may use a mechanical plunger to simulate the muscular movements of human breathing. Or the patient may ask another person to replace him/her entirely. The system needs to detect the deliberate impersonation and ``person swap.''.

% \textbf{Mix-Up.} A patient may follow his/her daily routine and share the bed with his/her spouse or family during the test. For instance, two persons, such as a couple or a mother and a newborn, sharing a bed is common in a real-world setup. The system needs to correctly identify the patient among the two/multiple breathing patterns as well as the background noise.

%  The corresponding security assumption to counter the adversary is that the patient's identity is initially verified by a medical technician. However, the challenge is that the connection would naturally weaken as the test progresses due to the dynamic of human physiological, signal differences between modalities, as well as noise present in the environment and need to be restored without interfering with the unobtrusive OSA screening task.

%Therefore, security context and measurement data need to be exchanged and stored in an authenticated format to prevent spoofing.
