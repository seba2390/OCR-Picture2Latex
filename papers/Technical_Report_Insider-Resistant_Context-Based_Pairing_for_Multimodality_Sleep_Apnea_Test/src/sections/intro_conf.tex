Over twenty-five million adults in the US suffer from obstructive sleep apnea (OSA), an airway muscle-related breathing condition that involuntarily causes respiratory cessations during sleep. Poor treatment can lead to excessive daytime fatigue, high blood pressure, cardio-metabolic conditions, along with a myriad of health problems \cite{baboli_good_2015}. A traditional diagnostic procedure, known as polysomnography (PSG), requires the patient to be in a laboratory overnight with instruments of multiple sensors/electrodes to track various sleep-related physiological parameters. However, PSG is highly obtrusive, expensive, and scarce.

At-home OSA monitoring systems leverage contactless and/or contact-based sensing technologies to monitor respiratory symptoms related to OSA. They allow users to conduct self-administered tests prescribed by their doctors and are considered economical alternatives for PSG. However, At-home OSA tests are subject to test fraud, as several professions within the patient population are deeply concerned that positive OSA test results will jeopardize their careers. As a result, an OSA patient may exploit the unsupervised at-home environment to manipulate the OSA test results. Specifically, the device pairing processes between the OSA screening system components are often the target of eavesdropping and spoofing from a non-compliant user.  

% To reduce the any low-demanding and less expensive diagnostic procedures have been proposed, studied, and commercialized to bridge the gap between effectiveness and accessibility of sleep tests.

% The extensive medical cost for OSA is due to the diagnostic procedure, known as polysomnography (PSG), which places a patient within a laboratory overnight and instrument the subject with multiple sensors/electrodes to track various sleep-related physiological parameters.

% The medical gold standard for diagnosing sleep apnea is by polysomnography (PSG), where the patient is attached to variety of sensors to record sleep activity overnight. However, PSG is also highly obtrusive, expensive, impractical for home-based use, and can inflict discomfort for patients. 

% Non-contact physiological radar monitoring systems (PRMSs), based on Continuous Wave (CW) Doppler radar, is a promising alternative for at-home sleep tests. Nevertheless, single-modality methods are bounded by the patient's vital signals during sleep tests, and recording ceases as soon as the compliance tracking sensor is detached from the patient. 

% Recent studies on multi-modality, such as wireless sensing and vital-based verification, has shown that heartbeat and respiration dynamics extracted from wireless radio signals are relatively distinguishable among individuals. Breathing patterns can depend on tidal volume, airflow profile, breathing depth, and energy \cite{benchetrit_breathing_2000}, all of which are retrieved through an OSA screening data \cite{IslamIdentityAuthenticationOSA2020}. However, patient identity verification remains susceptible to attackers who can eavesdrop and intercept using contactless and contact-based single-modality methods. As a result, it is necessary to introduce an additional system component to secure the multi-modality framework.

To combat these malicious behaviors, we introduce SIENNA: in\textbf{SI}der r\textbf{E}sista\textbf{N}t co\textbf{N}text-based p\textbf{A}iring for unobtrusive at-home OSA screening. SIENNA works with a multi-modality OSA screening system consisting of one data aggregate, e.g., the user's mobile phone, and two sensing modalities, e.g.,  a respiratory belt and a physiological radar monitoring system (PRMS). It leverages the respiration patterns collected by the respiratory belt to allow automatic pairing between the PRMS and the phone. The design of SIENNA uses a novel combination of JADE-ICA \cite{rutledge_independent_2013}, fuzzy commitment, \cite{JuelsFuzzyCommitmentScheme1999}, and friendly jamming \cite{AroraDialogCodesSecure2009,GollakotaPhysicalLayerWireless2011, MelcherIJamChannelRandomization2020}. The JADE-ICA allows the PRMS to identify the unique patterns of a person's breathing from a multi-person environment. The fuzzy commitment leverages the user's breathing patterns to establish a shared secret key between the PRMS and the mobile phone. And the friendly jamming prevents insiders, e.g., a non-compliant and unsupervised user with knowledge of the breathing patterns, from learning the security key.

We formally analyzed the security of SIENNA based on the attacker's knowledge of the context information, and implemented a laboratory prototype consisting of a mmWave PRMS (implemented with SDR and mmWave radio heads), a wireless respiratory belt, and a Android-based OSA app. We conducted an evaluation consisting of 20 subjects spanning over one month. The results show that SIENNA achieves reliable device pairing within a noisy at-home environment with multiple free moving persons in the background. It also prevents unauthorized receivers from retrieving the secret key, regardless of their locations or knowledge of the user's respiration patterns.\footnote{\label{tech_report}More details can found in our technical report:  \href{http://arxiv.org/abs/2105.00314}{arXiv:2105.00314v1}}.

%The rest of the paper is organized as follows. First, in Section \ref{sec:background}, an overview of Doppler radar and non-contact OSA screening, as well as a description of work relevant to compliance monitoring and identity verification are presented. An introduction and an analysis of the system and adversary models of multi-modality compliance tracking are provided  Section \ref{sec:challenges}. Next, the design of MASCOT is described in Section \ref{sec:mascot} and its security characteristics are analyzed in Section \ref{sec:security_analysis}. Implementations and experimentation of MASCOT are documented in Section \ref{sec:evaluation}. And we conclude in Section \ref{sec:conclusion} with a dicussion of future work.

%To the best of our knowledge, our work is the first attempt to achieve secure radio-based multi-subject identity verification for at-home OSA screening, by harnessing the combined power of Doppler radar and Fuzzy key Extractor.