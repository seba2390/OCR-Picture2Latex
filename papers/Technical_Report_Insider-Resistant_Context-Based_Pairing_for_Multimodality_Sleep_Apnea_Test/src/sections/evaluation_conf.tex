We empirically evaluated the performance of SIENNA, which consists of a PRMS implemented with mmWave transceivers/radio heads, one respiratory belt sensor implemented with a piezo-electric respiration transducer, and one Android-based OSA application. We conducted laboratory and field experiments over one month with the SIENNA prototype and 20 subjects selected through a random sample recruitment process. All experiments with human subjects are approved by the Institutional Review Board (IRB) based on the written consent.

We designed the eavesdropping and spoofing attacks with BLE sniffer and spoofer, implemented via Ubertooth and Kismet. During each experiment, the subject was asked to wear a respiratory belt and lie under a PRMS. A third party manually executes the pairing process, and the packets communicated between the mobile OSA app and the PRMS were identified based on their Bluetooth Device Addresses (BDAs) obtained before the experiment. During the eavesdropping attack, the host's codes on the BLE sniffer record the packets containing the fuzzy commitment, which were analyzed offline to deduce the keys. During the spoofing attacks, an attacker-generated compliance tracking data encrypted with the deduced key was transmitted at higher power when the PRMS uploads the data to the mobile OSA app to manipulate the latter into accepting the fraudulent data, which was verified during offline analysis.

SIENNA's performance against eavesdropping and spoofing is evaluated by comparing the aggregated bit error rate (BER) at the receiver versus the aggregated BER at the attacker side. Due to the application of fuzzy commitment, the key establishment protocol allows a maximum of 27\% BER in the breathing fingerprints (when using $(2^8, 255, 201)$ Reed-Solomon codes) to recover the key salt. Compared to Fig. \ref{fig:key evolution}c, such a BER alone prevents any outside attackers who cannot observe and mimick the patient's breathing patterns from stealing the key salt. Our experiment further showed that the jamming signal could suppress the attacker's BER to approximately 50\% within the PRMS's transmission range (Fig. \ref{fig:key evolution}d). The CDF of the accumulated BERs for attackers at any locations within the PRMS's transmission range concentrated between 41\% to 50\% (Fig. \ref{fig:key evolution}d) and is well beyond the correctable range of the selected Reed-Solomon codes.

% \begin{figure}[h]
% \centering
% \includegraphics[width=0.75\linewidth]{figures/smartphone.png}
% \caption{Implementation of MASCOT with, an Android application, a chestband sensor, and mmWave Radar.}
% \label{fig:system implementation}
% \end{figure}

% \begin{figure*}[t]
% \begin{subfigure}[t]{0.24\textwidth}
% \includegraphics[width=\textwidth,height=\textwidth]{figures/Figure_8_A_Test_New.eps}
% \end{subfigure}
% \hspace{\fill}
% \begin{subfigure}[t]{0.24\textwidth}
% \includegraphics[width=\textwidth,height=\textwidth]{figures/Figure_8_B_Test_new.eps}
% \end{subfigure}
% \hspace{\fill}
% \begin{subfigure}[t]{0.24\textwidth}
% \includegraphics[width=\textwidth,height=\textwidth]{figures/Fig-7-journal-2.png}
% \end{subfigure}
% \hspace{\fill}
% \begin{subfigure}[t]{0.24\textwidth}
% \includegraphics[width=\textwidth,height=\textwidth]{figures/Fig-3-journal-2.eps}
% \end{subfigure}
% \caption{Left to right: (a) Raw IQ data received by different RF channels of the the mmWave CW radar in Fig. \ref{fig:experiment set-up}b; (b) Breathing mixture obtained by (linear) demodulating the raw IQ (top), individual breathing patterns after source separation in comparison with the ones collected by the chestband sensor (bottom); (c,d) time and frequency domain analysis show the inhale and exhale characters are distinct between the two subjects, which allows patient tracking during modality switches.}
% \label{fig:breathing separation}
% \end{figure*}

% \subsection{Implementing MASCOT}
% Our MASCOT prototype was developed and tested from 2019 to 2020, with two mmWave radars implemented for the preliminary system. The first design employs a commercial off-the-shelf 4-channel 24GHz monopulse radar transceiver (K-MC4 from RFbeam Microwave GmbH) interfacing a LabVIEW controlled DAQ through four LNA's (SR560 from Stanford Research System). The 3dB beam aperture is $12^{\circ}$ horizontally and $30^{\circ}$ vertically. The angle coverage is between +/-$15^{\circ}$ horizontally. The second design, aiming to address the reflection noise issue identified with the first design, is a customized 14/28GHz harmonic monopulse radar implemented via a mmWave beamforming development kit (BBox one/UD box from TMYTEK). The kit consists of a 16-channel 24-31GHz phased array antenna, and a frequency Up/Down converter, which interfaces with a LabVIEW controlled USRP (USRP-2974 from NI). The 3dB beam aperture is $13^{\circ}$ horizontally and $14^{\circ}$ vertically. The angle coverage is +/-$45^{\circ}$ horizontally and +/-$60^{\circ}$ vertically. The harmonic function is implemented via an analog frequency divider at the TX, and a miniature harmonic transponder (diode based frequency doubler, fabricated at OSH Park) attached to the patient.

% The skin-contacting compliance tracker is implemented with a piezo-electric respiration transducer (Model 1132 Pneumotrace II from UFI), which interfaces with a LabVIEW controlled DAQ. The Android application communicates via Bluetooth Low Energy (BLE) with the host computers that control the mmWave radar(s) and the respiration transducer to execute the modality-switching and data logging functionality. The implementation leverages the Android BLE APIs to connect to the Generic Attribute Profile (GATT) servers set up by the LabVIEW Bluetooth Toolkit on the host computers. The physical layer jammer for the chestband sensor is implemented via the Android application connecting to a Wi-Fi-BT-Bluetooth LE breakout board (ESP32-WROVER-IB from ESPRESSIF). The jammer for the mmWave radar is implemented via with a LabVIEW controlled USRP (USRP-2974 from NI).

% \begin{figure*}[t]
% \begin{subfigure}[t]{0.24\textwidth}
% \includegraphics[width=\textwidth,height=\textwidth]{figures/Fig-4-journal-2.eps}
% \end{subfigure}
% \hspace{\fill}
% \begin{subfigure}[t]{0.24\textwidth}
% \includegraphics[width=\textwidth]{figures/FONTS_PLEASE.eps}
% \end{subfigure}
% \hspace{\fill}
% \begin{subfigure}[t]{0.24\textwidth}
% \includegraphics[width=\textwidth,height=\textwidth]{figures/Fig9_BW.eps}
% \end{subfigure}
% \hspace{\fill}
% \begin{subfigure}[t]{0.24\textwidth}
% \includegraphics[width=\textwidth,height=\textwidth]{figures/slant-ranges-fingerprint-similarity-most-recentt.eps}
% \end{subfigure}
% \caption{Left to right: (a) Multiple vital related dynamic features, including breathing rate, inhale/exhale area, breathing depth, and heart rate, are preserved after level-crossing quantization; (b) The distributions of fingerprint similarity between chestband-based and mmWave-based modalities, captured from the same subject; (b) The distributions of fingerprint similarity between chestband-based and mmWave-based modalities, captured from different subjects (chestband data from the patient, mmWave radar data from the non-patient); (d) The effect on the fingerprint similarity due to the change of slant range between the mmWave radar and the subject.}
% \label{fig:fingerprint extraction}
% \end{figure*}

%\subsection{Experiment Setup}
% We conducted laboratory and field experiments over a one-month period with the MASCOT prototype and 20 subjects selected through a random sample recruitment process. The subjects are between the ages of 16 and 35, weigh between 42 to 85 kilograms, and have no prior history of heart disease. The subjects are asked to participate in multiple trials of sleep studies under laboratory and everyday settings with environmental and adversarial complications detailed below.

% \textbf{Sleep Environment and Data Collection.} We furnished two sleep environments for experimentation. The indoor laboratory environment consists of 2 twin size beds under a quiet and dim ambience. The outdoor field environment consists of two beach mats and umbrellas set up at a local beach park. During each trial, the subjects are attached with chestband sensors and positioned 0.5 meters below the mmWave radar system for data collection. We continuously monitor the respiration of the subjects for 1 hour, during which modality-switches and adversarial activities (if planned) are attempted every 10 minutes. After the experiment, we extract a 60-second data segment around each modality-switch to evaluate MASCOT's performance, with the remaining data to serve as references. Overall, we collected approximately 30,000 breathing cycle samples for each subject.

% \textbf{Mix-Up/Surrogate.} We designed the mix-up and surrogate scenarios via a single-blind experiment set-up. For the mix-up scenario, we randomly paired subjects, asked both of them to wear chestbands and lay down under a mmWave radar, without telling which one should act as the patient. A third-party executed the modality switches via the OSA application paired with the patient's chestband and the mmWave radar. For the surrogate scenario, we selected pairs of subjects with similar physiques, asked both of them to wear chestbands but lay down under two separate mmWave radars, without telling which subject was the patient. The surrogate attack was implemented in the following ways: (1) During modality switch, a third-party executed the switches via the OSA application paired with the patient's chestband and the non-patient's mmWave radar. (2) During chestband modality, a third-party disconnected the patient's chestband from the DAQ and connected the non-patient's chestband to the DAQ. (3) During radar modality, a third-party team swapped the two mmWave radars.

% \textbf{Eavesdrop/Spoof.} We designed the eavesdropping and spoofing attacks via a BLE sniffer and spoofer, implemented via Ubertooth and Kismet. During each experiment, one subject was asked to wear a chestband and lay down under a PRMS. A third-party manually execute the pairing process. The packets communicated between the mobile OSA app and the PRMS  were identified based on their Bluetooth Device Addresses (BDAs) obtained prior of the experiment. To implement the eavesdropping attack, the host's codes recorded the packets containing the fuzzy commitment and hash value of the new key during each modality-switch, which were analyzed offline in attempt to deduce the keys. To implement the spoofing attacks, an attacker-generated compliance tracking data encrypted with the deduced key was transmitted at higher power during data upload toward the OSA app, in attempt to manipulate the latter into accepting the fraudulent data, which was verified during offline analysis.


% \subsection{Performance of Breathing Separation}
% During each experiment, the signal (mixture) captured by the PRMS (Fig. \ref{fig:breathing separation}a) was filtered in real-time by the ICA-JADE-based breathing separation module implemented through a MATLAB script running within LabVIEW. The script filtered the signal with a digital FIR Low pass filter with cut off frequency of 10 Hz to suppress the high-frequency noise while preserving the physiological-related information. The filtered signal was linearly demodulated to compute the phase shifts caused by the displacement(s) of the chest(s) surface(s) during breathing (Fig. \ref{fig:breathing separation}b top). Specifically, the script calculated the covariance matrices of the I and Q channel signals and applied eigenvalue decomposition to the covariance matrices to extract the maximum chest displacement information. The demodulated signal was separated by the ICA-JADE method to isolate individual respiratory signatures (Fig. \ref{fig:breathing separation}b bottom). 

% Finally, the script evaluated the performance of ICA-JADE by computing the cross-correlation between the isolated signatures with the ground truth obtained from the respiration transducer. The empirical result shows that the similarity is above 90\% when we limit the experiment within 60 seconds. As time increases, the subjects tend to move/turn on the bed, which changes the contribution of their breathing to the signal mixture, e.g., the mixing matrix, and the ICA-JADE result deviates from the truth. Thus, we limited the sensing period of MASCOT during the subsequent experiments to within 60 seconds to achieve a stable source separation.


% \subsection{Performance of Fingerprint Extraction}
% An individual breathing signature was quantized in parallel by multiple level-crossing LabVIEW VIs to generate the binary fingerprint after breathing separation. The breathing signature (torso deformation) due to respiratory movements is a complex three-dimensional pattern, and varies greatly with subject parameters and activity context. Based on our preliminary data, the thorax motions due to respiration and heartbeat are limited within +/-0.5cm and +/-0.05cm (Fig. \ref{fig:fingerprint extraction}, and the rates of respiration and heartbeat are below 15 and 60 per minute, when the subject is at rest. Therefore, we employed ten level-crossing quantization branches with a quantization step size of 0.05cm at a sample rate of 10 per second, to preserve the fine-grained respiratory motion when extracting the breathing fingerprint.

% To be compatible with the key evolution protocol, the quality of the binary fingerprints was evaluated based on the Hamming distances between fingerprints of the same subject observed by different modalities and different subjects observed by different modalities. It has been demonstrated in various works that human subjects can be sufficiently differentiated based on the inhales (local maxima), exhales (local minima), and breathing depth (the area between two consecutive maxima and in between one minima point). The similarities of these features directly translate to the Hamming distances between the quantized fingerprints. Therefore, comparing the Hamming distances is equivalent to constructing an equal-weighted linear classifier for patient identification.

% The empirical results (Fig. \ref{fig:fingerprint extraction}b) show that the average Hamming similarity per bit between fingerprints of the same subject observed from different modalities is around 63\% when extracted from 6-second breathing signatures (with 2 breathing cycles). The Hamming similarity increases almost logarithmically towards 100\% as we extend the duration of the breathing signature to 60 seconds (with 20 breathing cycles). The elevated mean and reduced variance in the Hamming similarity is mostly due to the extended time, which allows repeated measurements of the periodic respiratory effort motion unique to each subject. 

% On the contrary, the average Hamming similarity per bit between different subjects' fingerprints remains below 5\% despite the measurement duration (Fig. \ref{fig:fingerprint extraction}b). The position between the mmWave and the subject also poses a significant factor, which attenuates the Hamming similarity. As the subject moves away from the radar's downrange direction, the radar measurements deviate from the ground truth (obtained from the chestband). But their similarity remains sufficiently high compared to the breathing fingerprints between different subjects. Overall, the results show that the Hamming similarity per bit for MASCOT can be set to around 70\% to allow accurate patient tracking during modality switches.


% \subsection{Performance of Key Evolution}
% The binary breathing fingerprints, are chunked into multiple 10-second segments and padded with 0s to meet the codeword length of the $(2^{8}, 255, n)$ Reed-Solomon codes, e.g., $8 \times 255 = 2040$ bits. The particular group of Reed-Solomon codes are chosen to ensure the communicated data can be measured in exact multiple of bytes (8 bits). A key evolution salt of $8 \times n$ bits is randomly selected by the on-duty sentinel and encoded with the $\textsc{RS}(2^{8}, 255, n)$ Reed-Solomon encoder to generate the 2040 bits opening value. The opening value and the (multiple) padded fignerprint segments are XORed together to generate the 2040 bits commitment.

% We evaluate the security of the fuzzy commitment via a randomness test using the NIST statistical test suite. Based on 10 million randomly generated key evolution salts (with entropy per bit equal to one), we measured the randomness of the opening values and the commitments (Fig. \ref{fig:key evolution}a). The empirical test shows that the entropy per bit drops almost by half when the key salt is converted into a commitment. In other words, the entropy of a 2040-bit commitment is approximately 1000 bits. The primary causes of the reduction are due to the redundancy in the Reed-Solomon codes and the human respiratory motion's cyclic character. Two additional factors that affect the randomness of the commitment are the quantization levels and the number of the XOR operation rounds. When the quantization levels increase, the granularity of the binary sequencing improves, slightly elevating the randomness of the breathing fingerprints, resulting in a higher degree of entropy in the commitments. When the commitment is generated with multiple rounds of XOR operations, the entropy decreases due to the cross-correlation between fingerprint segments\footnote{In an XOR operation, the higher correlation between the inputs, the less the entropy in the output.}.

% We further measured the commitment and reconstruction time of the key evolution protocol. As the binary quantization takes negligible time to perform, the Reed-Solomon encoder and decoder's efficiency, besides communication delay,  dominates the protocol's turnaround time. We timed our LabVIEW-based protocol implementation with different parity symbol lengths, $k = 255 - n$, in the $(2^{8}, 255, n)$ Reed-Solomon codes. The results show that the overall commitment time is below 0.3 seconds and grows linearly to the maximum number of correctable symbols, $k/2$, due to the additional Lagrange interpolations needed to compute the parity symbols (Fig. \ref{fig:key evolution}b). The overall reconstruction time is below 0.02 seconds and grows linearly to $k/2$, due to the additional syndromes computed for error corrections (Fig. \ref{fig:key evolution}c). The decoding time is invariance to the number of errors in the codewords, which is a security advantage as the attacker cannot infer the decoded message's correctness based on decoding time.

% \subsection{Performance under Adversarial Settings}
% MASCOT's performance against mix-up and surrogate scenarios is assessed based on the protocol's sensitivity (true positive rate) and specificity (true negative rate) in identifying intentional or unintentional identity changes. MASCOT leverages a support vector machine (SVM) to detect identity changes (as anomalies in breathing patterns) within a single modality recording and rely on the success/failure of the key evolution to detect identity changes during modality switches. The overall sensitivity and specificity are calculated as the average among single modality and modality switching cases. Our implementation ensures a fair comparison between fuzzy commitment and SVM by using a linear SVM, whose decision boundary can be interpreted as a weighted Hamming distance between binary breathing fingerprints. The empirical results show that the MASCOT' scores are close to the single modality SVM approach, lower by 1.1\% and 1.5\% in sensitivity and specificity. The performance of single modality SVM approach is close but not precisely 100\%, due to the variance of human respiratory motions. Not that single modality contact-based methods can achieve 100\% detection accuracy if more rigid rules, such as prohibiting sensor disconnection, are enforced through the hardware. However, the detrimental effect of contact sensor toward the quality of the OSA screening data renders it inferior to MASCOT, as discussed in Sec. \ref{sec:background}.

% SIENNA's performance against eavesdropping and spoofing is evaluated by comparing the aggregated bit error rate (BER) at the receiving sentinel versus the aggregated BER at the attacker. Due to the application of fuzzy commitment, the key evolution protocol allows a maximum of 27\% BER in the breathing fingerprints (when using $(2^8, 255, 201)$ Reed-Solomon codes) to recover the key salt. Such BER alone prevents any outsider attackers who cannot observe and mimick the patient's breathing fingerprint from stealing the key salt. Our experiments further showed that the jamming signal could suppress the attacker's BER to approximately 50\% anywhere within the on-duty sentinel's transmission range, rendering the modality-switching message undecidable, even if an attacker could obtain the patient's breathing fingerprint. Overall, the CDF of the accumulated BERs for any attackers concentrated between 41\% to 50\%, well beyond the correctable range of the selected Reed-Solomon codes. Thus, the combination of both techniques ensures sufficient and complete protection of the security key during modality-switching against both outsider and insider attacks.