The security of SIENNA can be formally analyzed based on the property of a fuzzy commitment, and extended into three cases according to the attacker's knowledge on the user's breathing fingerprint, $f$. In the following, we omit the proof of the first two cases and focus on the third one to conserve space\footnoteref{tech_report}.

{\bf Attacker without Knowledge of $f$.} When the eavesdropper does not have the correct context, SIENNA inherits the security properties of a fuzzy commitment, in terms of concealment and binding \cite{JuelsFuzzyCommitmentScheme1999}.


{\bf Attacker with General Knowledge of $f$.} 
By knowing the distribution of $f$, the attacker can have computationally less expensive strategies to determine $\hat{l}$. One way to enhance SIENNA's security is to commit and decommit $l$ via multiple samples/segments of $f$ as shown in Section \ref{section:respiration fingerprinting}. Yao's XOR lemma \cite{GoldreichYaoXORlemma2011} indicates that the attacker's advantage due to bias in $f$'s distribution diminishes as we increase the number of $f$s in the XOR chain, with the diminishing rate defined in \cite{GoldreichYaoXORlemma2011}.


% \begin{figure*}[ht]
% \begin{subfigure}[t]{0.19\textwidth}
% \includegraphics[width=\textwidth,height=\textwidth]{figures/AlvinAlone.png}
% \end{subfigure}
% \hspace{\fill}
% \begin{subfigure}[t]{0.19\textwidth}
% \includegraphics[width=\textwidth,height=\textwidth]{figures/AlvinThomasClose.png}
% \end{subfigure}
% \hspace{\fill}
% \begin{subfigure}[t]{0.19\textwidth}
% \includegraphics[width=\textwidth,height=\textwidth]{figures/AlvinThomasFar.png}
% \end{subfigure}
% \hspace{\fill}
% \begin{subfigure}[t]{0.19\textwidth}
% \includegraphics[width=\textwidth,height=\textwidth]{figures/Substitute.png}
% \end{subfigure}
% \hspace{\fill}
% \begin{subfigure}[t]{0.19\textwidth}
% \includegraphics[width=\textwidth,height=\textwidth]{figures/Eavesdropper.png}
% \end{subfigure}
% \caption{Left to right: (a) One person scenario for compliance tracking with patient; (b) Mix-up: patient lies together with non-patient in close distance; (c) Mix-up: patient lies separately with non-patient; (d) Surrogate: radar tracks non-patient; (e) Eavesdropper records signal from radar.}
% \label{fig:experiment set-up}
% \end{figure*}

{\bf Attacker with Perfect Knowledge of $f$.}
The XOR-chain trick would not prevent an attacker with perfect knowledge of $f$ to retrieve $s$. For a malicious patient capable of measuring his own breathing patterns. When he is also able to capture the commitment message, $\{c, \mathcal{H}(s)\}$, he can accurately compute $l = c  \ominus f$ and decode to obtain $s$. 

% \begin{figure*}[t]
% \begin{subfigure}[t]{0.24\textwidth}
% \includegraphics[width=\textwidth,height=\textwidth]{figures/Fig-4-journal-2.eps}
% \end{subfigure}
% \hspace{\fill}
% \begin{subfigure}[t]{0.24\textwidth}
% \includegraphics[width=\textwidth]{figures/FONTS_PLEASE.eps}
% \end{subfigure}
% \hspace{\fill}
% \begin{subfigure}[t]{0.24\textwidth}
% \includegraphics[width=\textwidth,height=\textwidth]{figures/Fig9_BW.eps}
% \end{subfigure}
% \hspace{\fill}
% \begin{subfigure}[t]{0.24\textwidth}
% \includegraphics[width=\textwidth,height=\textwidth]{figures/slant-ranges-fingerprint-similarity-most-recentt.eps}
% \end{subfigure}
% \caption{Left to right: (a) Multiple vital related dynamic features, including breathing rate, inhale/exhale area, breathing depth, and heart rate, are preserved after level-crossing quantization; (b) The distributions of fingerprint similarity between chestband-based and mmWave-based modalities, captured from the same subject; (b) The distributions of fingerprint similarity between chestband-based and mmWave-based modalities, captured from different subjects (chestband data from the patient, mmWave radar data from the non-patient); (d) The effect on the fingerprint similarity due to the change of slant range between the mmWave radar and the subject.}
% \label{fig:fingerprint extraction}
% \end{figure*}

\begin{figure*}[t]
\begin{subfigure}[t]{0.24\textwidth}
\includegraphics[width=\textwidth,height=\textwidth]{figures/feature_preservation.eps}
\end{subfigure}
\hspace{\fill}
\begin{subfigure}[t]{0.24\textwidth}
\includegraphics[width=\textwidth]{figures/friend_similarity.eps}
\end{subfigure}
\hspace{\fill}
\begin{subfigure}[t]{0.24\textwidth}
\includegraphics[width=\textwidth,height=\textwidth]{figures/attack_similarity.eps}
\end{subfigure}
\hspace{\fill}
\begin{subfigure}[t]{0.24\textwidth}
\includegraphics[width=\textwidth,height=\textwidth]{figures/friendly_jamming.eps}
\end{subfigure}
\caption{Left to right: (a) Signal reconstructed after 64-level-crossing quantization with vital related dynamic features preserved; (b)  Similarity between belt-based and PRMS-based breathing patterns, measured with the same subject; (c) Similarity between belt-based and PRMS-based breathing patterns, measured with different subjects; (d) Performance of SIENNA against eavesdropper with full knowledge of breathing patterns, measured in aggregated BER.}
\label{fig:key evolution}
\vspace{-15pt}
\end{figure*}

To prevent such an insider attack, SIENNA leverages friendly jamming to obfuscate the commit message for any unintended receivers, and we can analyze the security of it based on a wiretap channel model \cite{LiangCompoundWiretapChannels2009}. Consider a non-compliant patient using an unauthorized receiver to intercept the commit message $X$. We denote the main channel as the wireless channel between $a$ and $b$ and the wiretap channel as the one between the unauthorized receiver and $a$ or $b$. 
Then the frequency-domain representation of the main channel and the wiretap channel is,
\begin{align*}
    & Y_{\text{main}} = X + \frac{P_0}{P_1}\mathcal{N}(0, \sigma^2_0), & Y_{\text{tap}} = X + \frac{P}{P_2}\mathcal{N}(0, \sigma^2)
\end{align*}
respectively, where $P_0$ and $\sigma^2_0$ denote the average power and variance of the intrinsic wireless noise, $P_1$ and $P_2$ denote the average powers of the OFDM signal observed by the receiver and the unauthorized receiver, and $P$ and $\sigma^2$ denote average power and variance of the jamming signal observed by the unauthorized receiver. 

\cite{GollakotaPhysicalLayerWireless2011} has shown that the jamming scheme works at its optimal when the OFDM system operates with high order modulation (at least QPSK), and $1 < P/P_2 < 9$.\footnote{The jamming signal is too weak to degrade the OFDM signal when $P/P_2 \leq 1$, and too strong to be indistinguishable from the OFDM signal when  $P/P_2 \ge 9$.} Therefore, SIENNA prohibits the transmit in BPSK at any SNR. Due to the FFT/IFFT operations in the OFDM system, $X$ is a pseudorandom Gaussian signal according to the central limit theorem. The bit error probability for such an OFDM system, allowing only M-QAM transmission, is 
\[
B_{\text{main}} \simeq \frac{4}{\log_2{M}}\textsc{Q}\left( \sqrt{\frac{3 P_1 \log_2{M}}{P_0(M-1)}} \right)
\]
for the main channel and 
\[
B_{\text{tap}} \simeq \frac{4}{\log_2{M}}\textsc{Q}\left( \sqrt{\frac{3 P_2 \log_2{M}}{P(M-1)}} \right),
\]
for the wiretap channel, where $\textsc{Q}(\cdot)$ denotes the the tail distribution function of the standard normal distribution. The receiver may adjust $P$ to elevate $B_{\text{tap}}$ beyond the error correction capability of the fuzzy commitment, and prevent an insider attack. The issue is that the jamming is only effective when $1 < P/P_2 < 9$, but $P_2$ depends on the location of the unauthorized receiver and is unknown to the receiver. Our solution is to have the transmitter create $L$ commitments, each with one sub-salt, and transmit them one by one, while the receiver jams at $L$ different power levels, $\{P_{\text{max}}, P_{\text{max}}/9, \ldots, P_{\text{max}}/9^{L-1}\}$. Given the fact that $B_{\text{main}}$ is not affected by $P$, the receiver can recover all sub-salts and XOR them together to obtain the key evolution salt. In contrast, the unauthorized receiver will fail to decode at least one sub-salt, therefore, cannot recover the key evolution salt. The number of jamming levels, $L$, can be computed based on the upper bound (the maximum power supported by the hardware, $P_{\text{max}}$) and lower bound (the noise floor, $P_0$) on the OFDM signal power. 