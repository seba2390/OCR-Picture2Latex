Here we provide an overview the fundamental of Continuous Wave (CW) Doppler radar theory, which enables the applications of at-home uuobtrusive sleep apnea screening. Physiological monitoring based on mmWave CW Doppler radar have been used to study sleep related disorders, such as sleep apnea. The symptoms of sleep apnea usually involve stoppages in respiration that wakes patients over night. These temporary stoppages are abnormalities in respiratory rhythm that are visible by observing a patient's chest and abdominal movements. When a sleep apnea event occurs, the patient's normal inhale and exhale patterns are interrupted, and his/her chest and abdomen cease to expand or contract for a short period due to a lack of motions from the diaphragm and intercostal muscles. These stasis can be detected by a mmWave CW Doppler radar installed above the bed with its aperture facing the mattress, as shown in (Figure -Yao).

During the sleep study, the radar detects the tiny movements of the patient's chest surface through the phase shift of the reflected signal \cite{droitcour_microwave_2001,lin_microwave_1992,chen_x-band_1986,lin_noninvasive_1975,greneker_radar_1997,droitcour_non-contact_2009,singh_data-based_2013}. Mathematically, the CW Doppler radar signal can be demodulated into two orthonormal channels, in-phase (\textit{I}) and quadrature phase (\textit{Q}). Let the distance offset due to chest movements be $x(t)$, the two channels can be expressed as \cite{park_arctangent_2007}:
\begin{align*}
  {B}_I(t) &= A_I\cos\left(\theta_0+\frac{4\pi x(t)}{\lambda}+\delta\theta(t)\right)\\
  {B}_Q(t) &= A_Q\sin\left(\theta_0+\frac{4\pi x(t)}{\lambda}+\delta\theta(t)\right),
\end{align*}
Where $\lambda$ is the signal wavelength, $\theta_0$ is the phase delay due to the nominal distance between the radar transmitter and the patient's torso, surface scattering, and radar's RF chains, $\delta\theta(t)$ is the residual phase noise. The phase shift corresponds to the respiratory movement can be computed via arctangent demodulation \cite{park_arctangent_2007}:
\begin{equation*}
  \theta(t) = \theta_0 + \frac{4\pi x(t)}{\lambda} = \arctan \left(\frac{A_I{B}_Q(t)}{A_Q{B}_I(t)}\right).
\end{equation*}
Visually, the respiratory motion, $\theta(t)$, swings around the origin on the I-Q plot with the apnea arcs shorter than the arcs due to normal breathing. Viewed on either the I-t or Q-t plots,  $\theta(t)$ fluctuates for normal breathing and remains relatively constant during apnea intervals.

\begin{figure}[h]
\includegraphics[width=0.8\textwidth]{figures/doppler_radar.eps}
\caption{radar primer}
\label{fig:radar primer}
\end{figure}

% A Doppler radar system can operate in an entirely non-contact fashion or in combination with a passive harmonic tag to monitor a patient's respiration. In the former case, the radar system transmits an RF signal and observes the reflection signal at the same frequency, which is modulated by the motion of the human chest. In the latter case, a planar harmonic tag, consisting of a non-linear diode device, is attached to the patient's chest. When energized, the tag emits reflections at the harmonic frequencies of the transmit signal, which can be observed by the radar's receiver toned to those frequencies (citation -Yao). Since there exist few non-linear devices in a natural environment, the harmonic radar technique is capable of suppressing all other RF interference due to the background features.