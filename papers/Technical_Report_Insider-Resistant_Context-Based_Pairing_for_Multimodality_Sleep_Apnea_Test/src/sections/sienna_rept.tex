% \begin{figure*}[t]
% \begin{subfigure}[t]{0.19\textwidth}
% \includegraphics[width=\textwidth]{figures/ChestbandDoctor.png}
% \end{subfigure}
% \hspace{\fill}
% \begin{subfigure}[t]{0.19\textwidth}
% \includegraphics[width=\textwidth]{figures/OnBed.png}
% \end{subfigure}
% \hspace{\fill}
% \begin{subfigure}[t]{0.19\textwidth}
% \includegraphics[width=\textwidth]{figures/InBed.png}
% \end{subfigure}
% \hspace{\fill}
% \begin{subfigure}[t]{0.19\textwidth}
% \includegraphics[width=\textwidth]{figures/OutofRange.png}
% \end{subfigure}
% \hspace{\fill}
% \begin{subfigure}[t]{0.19\textwidth}
% \includegraphics[width=\textwidth]{figures/Verification.png}
% \end{subfigure}
% \caption{Left to right: (a) Doctor places chestband on patient \& pairs with app; (b) At home: the modality switch \& key evolution made possible by shared observation of patient from two sentinels; (c) mmWave radar handles compliance tracking \& data collection of patient; (d) App alerts user if they are nearing the range limit of the mmWave radar; (e) Doctor verifies compliance tracking \& checks data for any abnormalities.}
% \label{fig:mascot overview}
% \end{figure*}

% \begin{figure*}[t]
% \hspace{\fill}
% \begin{subfigure}[t]{0.28\textwidth}
% \includegraphics[width=\textwidth,height=\textwidth]{figures/doppler_radar_v1.eps}
% \end{subfigure}
% \hspace{\fill}
% \begin{subfigure}[t]{0.28\textwidth}
% \includegraphics[width=\textwidth,height=\textwidth]{figures/system_overview.eps}
% \end{subfigure}
% \hspace{\fill}
% \begin{subfigure}[t]{0.28\textwidth}
% \includegraphics[width=\textwidth,height=\textwidth]{figures/feature_extraction_fuzzy_extractor.png}
% \end{subfigure}
% \hspace{\fill}
% \end{figure*}

\begin{figure*}[t]
\begin{subfigure}[t]{0.24\textwidth}
\includegraphics[width=\textwidth]{figures/clinic_visit.eps}
\end{subfigure}
\hspace{\fill}
\begin{subfigure}[t]{0.24\textwidth}
\includegraphics[width=\textwidth]{figures/pairing.eps}
\end{subfigure}
\hspace{\fill}
\begin{subfigure}[t]{0.24\textwidth}
\includegraphics[width=\textwidth]{figures/mod_change.eps}
\end{subfigure}
\hspace{\fill}
\begin{subfigure}[t]{0.24\textwidth}
\includegraphics[width=\textwidth]{figures/verification.eps}
\end{subfigure}
\caption{Left to right: (a) In hospital: a medical technician places a respiratory belt on a user and pairs it with the user's phone; (b) At home: a user pairs the PRMS with the user's phone; (c) User switches to PRMS for OSA screening; (e) Doctor verifies OSA data for any abnormalities.}
\label{fig:sienna overview}
\end{figure*}

We present SIENNA: in\textbf{SI}der r\textbf{E}sista\textbf{N}t co\textbf{N}text-based p\textbf{A}iring. SIENNA leverages the user's respiratory patterns observed by both the respiratory belt and the PRMS. To defense against an insider attack, SIENNA employs a combination of fuzzy matching and friendly jamming to prevent a non-compliant user from obtaining the pairing key using the same context.

\subsection{Overview}
The pairing procedure of SIENNA is shown in Fig. \ref{fig:sienna overview}. It begins when the user visits a doctor to obtain the test authorization. During the visit, the doctor attaches a respiratory belt to the patient, and pairs it to the user's mobile device OSA app (Fig. \ref{fig:sienna overview}a). Once arriving home, the user lies in bed and the PRMS automatically pairs with the mobile device based on the respiration pattern observed by both the PRMS and the respiratory belt (Fig. \ref{fig:sienna overview}b). Once the pairing completes, both links from the PRMS and respiration belt to the mobile device are secure. The user can freely choose either the respiratory belt or PRMS for OSA screening, and the selected modality communicates encrypted OSA data to the mobile device (Fig. \ref{fig:sienna overview}c). Once testing is completed, the user revisits the doctor and uploads the OSA screening from the mobile device. The doctor runs a compliance check and examines whether there were any significant gaps or inconsistencies with the OSA data. Based on the compliance check report, the doctor decides whether to accept or reject the OSA screening (Fig. \ref{fig:sienna overview}d).

\subsection{Insider Resistant Device Pairing}
\label{section:key evolution}
The core of SIENNA is a context-based secure key establishment protocol (Fig. \ref{fig:key evolution_diag}), which allows two devices, $a$ and $b$, to securely exchange and update a symmetric key in the presence of a nearby eavesdropper by utilizing the context (user's breathing patterns) observed by both devices at the moment of exchange. In the OSA scenario above, $a$ represents the mobile device connected to a respiratory belt, $b$ represents the PRMS.

\begin{figure}[h]
\centering
\includegraphics[width=0.65\linewidth]{figures/protocol.eps}
\caption{Pairing protocol for two devices observing similar data to establish a symmetric key in the presence of an adversary.}
\label{fig:key evolution_diag}
\end{figure}

% \begin{figure}[h]
% \centering
% \includegraphics[width=0.75\linewidth]{figures/Quantization-of-Signal.png}
% \caption{Encoding of a breathing pattern signal \& generation of a key using level-crossing quantization.}
% \label{fig:level-crossing quantization}
% \end{figure}

% \begin{figure}[h]
% \centering
% \includegraphics[width=0.75\linewidth]{figures/Protocol.jpg}
% \caption{Key evolution protocol for two devices that share the same observed data to establish a symmetric key in the presence of an adversary.}
% \label{fig:key evolution}
% \end{figure}

Traditional context-based key establishment protocols are not secure when eavesdroppers are nearby to observe the context. Many of such protocols assume that the adversary can't be near paring devices over extended periods of time \cite{MiettinenContextBasedZeroInteractionPairing2014}. SIENNA addresses this shortcoming through a cross-layer design that employs two security primitives: fuzzy commitment \cite{JuelsFuzzyCommitmentScheme1999} and dialog-codes-based friendly jamming \cite{AroraDialogCodesSecure2009,GollakotaPhysicalLayerWireless2011, MelcherIJamChannelRandomization2020}. The former is a cryptographic scheme that allows secure commit and de-commit of a secrete similar-but-not-identical open value. The latter is a friendly jamming scheme at the receiver by jamming the transmitted signal to flip specific bits in the message. Together, this ensures that an eavesdropper with the right context will not be able to recover the key exchanged between the pairing devices.

First we define the fuzzy commitment scheme \cite{JuelsFuzzyCommitmentScheme1999}. Let $\sigma$ be a secret value. A fuzzy commitment transforms a secret value $\sigma$ into a commitment $\{\chi, \mathcal{H}(\sigma)\}$ using an opening feature, $\phi$, and a hash function, $\mathcal{H}(\cdot)$:
\[
\{\chi, \mathcal{H}(\sigma)\} = \textsc{Commit}(\sigma, \phi),
\]
such that $\chi$ appears random and devoid of any information about $\sigma$. And all open features $\hat{\phi}$ reveals $\sigma$ via 
\[
\sigma=\textsc{Open}(\chi,\hat{\phi}),
\]
if and only if the Hamming distance $\textsc{Ham}(\phi, \hat{\phi}) \leq \tau$, where $\tau$ is a parameter denoting the maximum allowable Hamming distance between $\hat{\phi}$ and $\phi$ to reveal $\sigma$. 

To initiate the key establishment protocol, $a$ broadcasts a message, $\{ \mathcal{H}(k), t_{\text{str}}, t_{\text{end}} \}$, where $k$ denotes the key of the previous iteration. $t_{\text{str}}$ and $t_{\text{end}}$ denotes the starting and ending timestamps of $a$ and $b$'s captures of the respiratory patterns. During the first round,  $\mathcal{H}(k)$ is replaced with a public parameter known to all parties.

From respiratory patterns $r_{a}(t_{\text{str}}, t_{\text{end}})$ and $r_{b}(t_{\text{str}}, t_{\text{end}})$ captured within the specified time interval, the two devices extract breathing fingerprints $f_{a} = \textsc{Ext} \left( r_{a}(t_{\text{str}}, t_{\text{end}}) \right)$ and $f_{b} = \textsc{Ext} \left( r_{b}(t_{\text{str}}, t_{\text{end}}) \right)$, via a fingerprint extraction function $\textsc{Ext} (\cdot): \{1,0\}^{*} \mapsto \{1,0\}^{M*2^K}$ (detailed in Sec. \ref{section:respiration fingerprinting}). In case $a$, e.g., PRMS, observes a breathing mixture of multiple subjects, the mixture is first separated into the breathing patterns of individual subjects (detailed in Sec. \ref{section:breathing separation}), which are processed by $\textsc{Ext} (\cdot)$ to create multiple fingerprints, one for each subject.

Once the breathing fingerprints are generated, $a$ randomly selects a key salt, $ s \in \{1,0\}^{N*2^K}$, and transforms it into a commitment $\{c \in \{1,0\}^{M*2^K}, \mathcal{H}(s)\}$ using the breathing fingerprint, $f_{a}$. Specifically, $a$ encodes $s$ via the Reed-Solomon (RS) encoding function
\[
l = \textsc{RS}(2^K, M, N, s) \in \mathbb{F}^M_{2^K}
\]
and compute:
\[
c = l \ominus f_{a},
\]
with $\ominus$ denoting exclusive OR (XOR).

Henceforth, $a$ and $b$ exchange $c$ through dialog codes to defend against insider attack. First, $a$ converts the commitment,
\[
\{c, \mathcal{H}(s)\},
\]
into OFDM symbols, duplicates each symbol back-to-back,
\[
\textsc{DupSym}\left(\{c, \mathcal{H}(s)\}\right),
\]
and broadcasts all the symbols. In parallel to $a$'s broadcast, $b$ randomly jams either the original symbol or its repetition \cite{GollakotaPhysicalLayerWireless2011,AroraDialogCodesSecure2009}, 
\[
\textsc{PhyJam}\left(\|\textsc{DupSym}\left(\{c, \mathcal{H}(s)\}\right)\|\right),
\]
To jam a symbol, $b$ transmits a signal that is drawn randomly from a zero-mean Gaussian distribution whose variance is the same as the OFDM signal with the same modulation. Since $b$ knows which symbols are jammed, it stitches the unjammed symbols together to create a clean version of the OFDM transmission and decodes the signal to obtain the clear message.

Upon receiving $\{c, \mathcal{H}(s)\}$, $b$ computes 
\[
\hat{l} = f_{b} \ominus c
\]
and decommits the salt by decoding $\hat{l}$ using the Reed-Solomon (RS) decoding function: 
\[
\hat{s} = \overline{RS}(2^K, M, N, \hat{l}).
\]
Due to the error correction capability of Reed–Solomon codes, $s$ equals to $ \hat{s}$ if and only if $l$ and $\hat{l}$ differ in less than $2^{K-1}(M-N)$ bits. Since $l = f_{a} \ominus d$ and $\hat{l} = f_{b} \ominus d$, it is equivalent in saying that $f_{a}$ and $f_{b}$ must differ in less than $T$ bits for $b$ to retrieve $s$.

To confirm whether the retrieval were successful, $b$ computes $\mathcal{H}(\hat{s})$ and compares it with $\mathcal{H}(s)$. Depending on whether they were equal, an $\textsc{ACK}$ or a $\textsc{NAK}$ message is transmitted from $b$ to $a$, with the former initiating the final step of the key establishment protocol and the latter initiating the reattempts.

To conclude the key establishment, both $a$ and $b$ applies a key derivation function 
\[
k' = \textsc{KDF}(k, s),
\]
to obtain the new key.

\subsection{Breathing Separation with JADE-ICA}
\label{section:breathing separation}
Home environments are noisy and unpredictable, with the possibility of irrelevant individuals in close vicinity from the user. To retrieve the correct context in an environment with potentially multiple subjects, SIENNA augments the PRMS modality with a breathing separation module, which reconstructs the breathing signals of multiple co-located individuals to select the correct target. 
%The wireless (mmWave radar) modality in our compliance tracking system is particularly susceptible to such complications, as the RF reflections off the bodies would superimpose over the wireless medium and interfere at the receiver.
% Several works have been proposed to correctly separate and isolate the patient's respiratory signature from the combined signal mixture, all of which fall into the problem domain of blind source separation (BSS) \cite{yue_extracting_2018,petrochilos_blind_2007,cardoso_blind_1998,de_lathauwer_independent_1996,rutledge_independent_2013}. 
The goal of the separation module is to reconstruct a set of source signals from a set of mixtures, without knowing the properties of the sources and the mixing proportion. 
%A common example of such BSS is the cocktail party problem where multiple people's voice signal is mixed into the speaker and need to be separated \cite{yue_extracting_2018,petrochilos_blind_2007,cardoso_blind_1998}. 
Since respiration signals are non-Gaussian and independent from individuals, whilst mixed linearly at the PRMS receiver, one can recover the source signals using independent component analysis (ICA) \cite{yue_extracting_2018}, which is formulated as the following: assume $N$ independent time varying sources $ s_i(t),\ i=1 \ldots N$, and $M$ different observations
$ x_i(t),\ i=1 \ldots M$. For $T$ time units $(t=1 \ldots T)$, we can define the source signal as a $N \times\ T$ matrix,
\[
\mathbf{S}_{N \times T} = \begin{bmatrix} 
    s_{11} & s_{12} & \dots \\
    \vdots & \ddots & \\
    s_{N_1} &        & s_{N_T} 
    \end{bmatrix},
\]
and the observed mixtures as a $M \times\ T$ matrix,
\[
\mathbf{X}_{M \times T} = \begin{bmatrix} 
    x_{11} & x_{12} & \dots \\
    \vdots & \ddots & \\
    x_{M_1} &        & x_{M_T}
    \end{bmatrix}.
\]
The mixtures are produced as the product of the source and a mixing matrix $\mathbf{W}_{M \times N}$, e.g.,
\[
    \mathbf{X}_{M \times T} = \mathbf{W}_{M \times N} \times \mathbf{S}_{N \times T}.
\]
The goal of ICA is to recover $\mathbf{S}_{N \times T}$ and $\mathbf{W}_{M \times N}$ given only $\mathbf{X}_{M \times T}$, assuming
the $s_i(t),\ i=1 \ldots N$ are independent and non-Gaussian. We employ the joint approximate diagonalization of eignematrices (JADE) algorithm \cite{de_lathauwer_independent_1996} to perform ICA, with the details omitted to conserve space. 

The JADE algorithm extracts independent non-Gaussian sources $\mathbf{S}_{N \times T}$, e.g., the breathing pattern of individual targets, from signal mixtures with Gaussian noise $\mathbf{X}_{M \times T}$, e.g., the breathing mixture received by the PRMS, by constructing a fourth-order cumulants array from $\mathbf{X}_{M \times T}$. Specifically, assuming $T > M,N$, the algorithm first applies PCA to whiten $\mathbf{X}_{M \times T}$, resulting
\[
\mathbf{P}^{\text{tr}}_{T \times K} = \mathbf{B}_{K \times M} \times \mathbf{X}_{M \times T},
\]
with a whitening matrix $\mathbf{B}_{K \times M}$.
%has $N$ rows with $N$ being the number of singular values of $\mathbf{X}_{M \times T}$, and $M$ the columns of whitened matrix $\mathbf{P}_{T \times N}$ are orthogonal with equal variance. 
After the whitening, the columns of whitened matrix $\mathbf{P}_{T \times K}$ are orthogonal with equal variance. Any rotation of $\mathbf{P}_{T \times K}$ will not change the independence between it column vectors. The algorithm then tries to find a rotation matrix $\mathbf{V}_{N \times N}$ to maximize the independence between the row vectors of the rotated $\mathbf{P}_{T \times K}$, which is achieved when the fourth-order cross-cumulants between the row vectors are zero and their auto-cumulants maximal. Once $\mathbf{V}_{N \times N}$ has been identified, the demixing matrix can be computed as 
\[
    \mathbf{W}^{+}_{N \times M} = \mathbf{V}_{N \times N} \times \mathbf{B}_{N \times M},
\]
and the source matrix as
\[
    \mathbf{S}_{N \times T} = \mathbf{W}^{+}_{N \times M} \times \mathbf{X}_{M \times T},
\]
More detailed information on JADE and cumulant tensor array can be found at \cite{rutledge_independent_2013} and the Matlab code on the website \cite{CardosoBlindSeparationReal}.
%  {\ \ \ \ \ \ \ a}_{M\times T}=\ \left[\begin{matrix}a_{11}&\cdots&a_{1M}\\\vdots&\ddots&\vdots\\a_{M1}&\cdots&a_{MT}\\\end{matrix}\right]
 
%  The observations are written as \(M\times\ T\) matrix. 
 
%  {\ \ \ \ \ \ \  s}_{N\times T}=\ \left[\begin{matrix}s_{11}&\cdots&s_{1N}\\\vdots&\ddots&\vdots\\s_{N1}&\cdots&s_{NT}\\\end{matrix}\right]
 
%  Basically, the observation is produced by combining source signal \(a_{M\times T}\) via mixing matrix \(W_{N\times M}\), such that we can write  \( s_{N\times T}=W_{N\times M}\times\ a_{M\times T}\). The goal of ICA is to recover pure source signal from their independence properties by extracting demixing matrix which is represented as \(W_{N\times M}^{-1}\): \(a_{M\times T}=W_{N\times M}^{-1}\times\ s_{N\times T}\). In literature, there are different ICA algorithms, among them fast ICA, Infomax and joint approximate diagonalization of eignematrices (JADE) is most popular \cite{de_lathauwer_independent_1996}. JADE is fourth order statistics kurtosis based ICA method \cite{cardoso_blind_1998,de_lathauwer_independent_1996,rutledge_independent_2013}. The main advantages of this technique is that it is matrix diagonalization based approach whereas other algorithm depends on optimization procedure and hence variable results might occur \cite{cardoso_blind_1998,de_lathauwer_independent_1996,rutledge_independent_2013}. In our previous reported results we tested the efficacy of JADE algorithm for separating respiratory signatures from combined mixture of signals \cite{islam_separation_2018,islam_multiple_2019}. 

\subsection{Fingerprinting with Level-Crossing Quantization}
\label{section:respiration fingerprinting}

% \begin{figure}[t]
% \centering
% \begin{subfigure}[t]{0.23\textwidth}
% \includegraphics[width=\textwidth]{figures/Protocol.jpg}
% \end{subfigure}
% \hspace{\fill}
% \begin{subfigure}[t]{0.23\textwidth}
% \includegraphics[width=\textwidth]{figures/Quantization-of-Signal.png}
% \end{subfigure}
% \caption{(a) Key evolution protocol for two devices that share the same observed data to establish a symmetric key in the presence of an adversary; (b) Encoding of a breathing pattern signal \& generation of a key using level-crossing quantization.}
% \label{fig:sienna}
% \end{figure}

Once the devices obtain breathing patterns of individual targets, they apply $\textsc{Ext} (\cdot)$ to extract the binary breathing fingerprints. The binary strings must meet two criteria for $a$ and $b$ to agree on the patient's identity and evolve the shared security key: (1) they should look sufficiently similar in Hamming space if they represent the breathing process of the same person, and (2) they should preserve the uniqueness of the breathing dynamic that distinguishes among individuals.

To achieve these objectives, $\textsc{Ext} (\cdot)$ applies level-crossing quantization to sample the continuous breathing patterns with two predefined thresholds (Fig. \ref{fig:quantization}). Let $q_{+}$, $q_{-}$ be the thresholds values such that $q_{+} > q_{-}$, we define a quantizer $\textsc{Qtz} (\cdot)$:
\begin{equation*}
  \setlength{\arraycolsep}{0pt}
    \textsc{Qtz} (x) = \left\{ \begin{array}{ l l }
    10 &\;\;\;\;\text{if}\; x \geq q_{+} \\
    01 &\;\;\;\;\text{if}\; x \leq q_{-} \\
    00 &\;\;\;\;\text{if}\; q_{-} < x < q_{+}
  \end{array} \right.
\end{equation*}
Let $T$ be the time interval between adjacent sampling instants. The binary sequence obtained by $\textsc{Ext} (\cdot)$ is
\begin{multline*}
f = \textsc{Ext} \left(r(t_{\text{str}}, t_{\text{end}})\right) = [\textsc{Qtz} \left( r(t_{\text{str}}) \right),...,\\
    \;\;\;\;\;\;\;\textsc{Qtz} \left(r(t_{\text{str}} + \lfloor\frac{t_{\text{end}} - t_{\text{str}}}{T}\rfloor T \right)],
\end{multline*}
which can be compared in Hamming space.
If the binary sequence is longer than the length of the commitment,  $\|\lambda\|$, we pad and divide it into multiple subsequences, $f = [ f_1, f_2,...,f_n]$ to commit
\begin{equation}
c = f_1 \ominus f_2 \ominus ... \ominus f_n \ominus l,
\end{equation}
and decommit
\begin{equation}
\hat{l} = \hat{f}_1 \ominus \hat{f}_2 \ominus ... \ominus \hat{f}_n \ominus c,
\end{equation}
The level-crossing quantization described above is an event-based sampling procedure as the crossing events are encoded as changes in the binary patterns. When the signal remains above/below the levels, it produces fixed patterns, which can be exploited to further compress the binary before transmission\footnote{Techniques to produce a compressed the binary string that are satisfiable to the two criteria is beyond the scope of this article and will be investigated in the future.}. 


However, the result of a single level-crossing quantization loses details in the original breath pattern and fails the second objective. To address this issue, we apply multiple passes of level-crossing binary quatizations, each at a distinct pair of levels, $q_{i+}$, $q_{i-}$. Intuitively, it is equivalent to create a pair-wise linear approximation of the original breathing pattern, with quantization error equal to the level density.

If the binary fingerprints after the multi-level quantization is longer than $\|l\|$, we pad and divide it into multiple subsequences, $f = [ f_1, f_2,...,f_n]$ to commit
\[
c = f_1 \ominus f_2 \ominus ... \ominus f_n \ominus l,
\]
and decommit
\[
l = f_1 \ominus f_2 \ominus ... \ominus f_n \ominus c.
\]

\begin{figure}[h]
\centering
\includegraphics[width=0.65\linewidth]{figures/quantization.eps}
\caption{Encoding of a breathing pattern signal \& generation of a key using level-crossing quantization.}
\label{fig:quantization}
\end{figure}
%Since the human breathing signal is bandlimited, we can eliminate alias in the approximation by setting the sampling frequency to above the Nyquist rate.

% \subsection{The MASCOT Scheme}
% The detailed design of MOTIVE is shown in Figure XXX. Once being recommended to undergo the OSA screening, the patient first visits his/her doctor to obtain the test authorization. During the visit, his/her doctor attaches the chest band sensor to the patient and input a signature key to the storage module onboard the sensor. The sensor continuously record the patient's respiratory patterns from the moment it is attached. Any attempts to remove the sensor nullifies the test. For each record, the sensor labels it with a timestamp and store it in the tamper-proof storage module. The OSA screening system broadcasts a \texttt{READY} message, and waits for the signals from the sensor.

% Upon receiving the \texttt{READY} message, the chest band sensor alerts the patient the possibility of modality switch. If the patient wishes to switch the mode, the sensor generates a random session key. It encapsulates the session key and the signature key in a fuzzy extractor built from the patient's most recent respiratory patterns and broadcasts the encapsulation. Upon receiving the message, the OSA screening system attempts to reconstruct both keys with the respiratory patterns collected by the mmWave radar. If there were multiple persons in the room, the OSA screening system tries individual signal traces to find the one that matches the fuzzy extractor. If the key establishments were successful, the OSA screening system acknowledges the transition by replying an \texttt{ACK} message encrypted via the session key. Once the message is decrypted and verified, the chest band sensor transfers its data records securely using the session key and notify the patient of the modality change. The patient may take off the chest band sensor. The OSA screening system signs the data records transferred from the sensor and continue on the recording and signing tasks.

% If the patient is about to move out-of-view from the radar and the radar signals degrade, the OSA screening system notifies the patient to reattach the chest band sensor. Once attached, the chest band sensor reactivates and broadcasts a \texttt{READY} message, and waits for the signal from the by OSA screening system. Upon receiving the \texttt{READY} message, the OSA screening system generates a random session key, encapsulates it in a fuzzy extractor, and initiates the key agreement protocol with the chest band sensor. If the sensor were able to extract and verify the session key, the sensor acknowledges the OSA screening system by replying to an \texttt{ACK} message encrypted by the session key and resume the recording, with the OSA screening system entering the idle mode. The chest band sensor needs to remain attached until the next modality switch. Any attempts to remove the sensor during that period also nullifies the test.

% Modality switching can occur multiple times throughout the test. Once the test completes, the patient's doctor collects the respiratory pattern records from the OSA screening system and run an automatic validity check to examine whether there were significant gaps or inconsistencies between the data entries. Based on the report of the validity check, the doctor decides whether to accept or reject the sleep test.

% arythom it works  methods for 
% Basically, by measuring the phase shift of the reflected radar signal induced by tiny movement of the chest surface due to cardio-respiratory activity breathing rate and heart rate can be extracted remotely \cite{droitcour_microwave_2001,lin_microwave_1992,chen_x-band_1986,lin_noninvasive_1975,greneker_radar_1997,droitcour_non-contact_2009,singh_data-based_2013}. Compared with other proposed conventional biomedical radars based identity authentication system that can only authenticate single subject \cite{lin_cardiac_2017,rahman_doppler_2018,shi_contactless_2018,islam_identity_2019}, the main novelty of the proposed technique is to isolate respiratory pattern from combined mixture by having concurrent measurement of two subjects with the field of view of radar using ICA-JADE algorithm. In our proposed authentication system, if there is multiple subject present in front of the radar then radar gets combined mixture so we integrated ICA-JADE algorithm to isolate independent respiratory signature from combined mixture \cite{islam_separation_2018}. After separating independent respiratory signatures, we extracted respiratory breathing related dynamic features (breathing rate, peak distribution, breathing energy,spectral entropy, inhale area and exhale area). We also integrated machine learning classifiers with the hyper-feature set and combined Fuzzy Extractor with linear coding to transform the hyper-feature sets into strong biometric keys. 

% A typical Doppler radar transceiver transmits electromagnetic signal and demodulates the signal reflected back from human. According to Doppler theory, the phase of the reflected signal is directly proportional to the tiny movement of the chest surface due to cardio-respiratory activity \cite{lin_noninvasive_1975,lin_noninvasive_1975}. A typical direct conversion CW quadrature channel radar evaluates into the following baseband equations \cite{droitcour_microwave_2001,singh_data-based_2013}: 

%  Where, two above equations represent in-phase and quadrature-phase of the demodulated signal. Here, x(t) is the chest displacement, \(\lambda\) is the wavelength,\(A_{BI}\), \(A_{BQ}\) is the amplitude and \(∆\phi(t)\) is the phase noise of the oscillator. Before demodulating the signal, we used to correct channel imbalance of I/Q signal \cite{singh_data-based_2013}. Then we plot the corrected baseband signal \(B_{Ic}(t)\) and \(B_{Qc}(t)\) against each other on complex plane which is I/Q plot \cite{singh_data-based_2013}. A center estimation algorithm is employed to find radius of arc and center of the arc origin \cite{singh_data-based_2013}. Two different type of demodulation technique is generally employed for extracting phase information of the signal \cite{droitcour_microwave_2001}. One is linear demodulation and another one is arc-tangent demodulation \cite{singh_data-based_2013}. After utilizing demodulation technique, we can extract chest wall displacement information which is represented below in equation \cite{droitcour_microwave_2001}. 

% {\ \ \ \ \ \ \ \ \ \ \ \ \theta}\left(t\right)=\frac{4\pi x\left(t\right)}{\lambda}

% Where, \(\theta(t)\) represents the angle formed by the arc in IQ plane. The tiny chest displacement is directly proportional to the phase change of the reflected signal \cite{singh_data-based_2013}. 