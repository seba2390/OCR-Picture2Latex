The following section provides a brief survey on test compliance tracking, radio-based sensing/verification and co-presence authentication.

\subsection{Test Compliance Tracking}

The OSA test compliance tracking works as follows: Prior to the test, the doctor secures a tracker to the identified patient. Once attached, the tracker continuously observes and records the patient's vital signs until the test is finished. During the follow-up visit, the compliance data is compared with the screening data recorded by the OSA test equipment. If there were significant gaps, during which the compliance tracker fails to observe the patient's vital signs or significant inconsistency between the compliance data and the OSA screening data, the test would be nullified. The initial assessment ensures the patient never removes the tracker, and the subsequent data analysis verifies the identification information between the patient wearing the tracker and subject undergoing the test. 

Existing industry solutions utilize skin-contacting sensors to execute test compliance tracking \cite{noauthor_httpspatentsgooglecompatentus8679012b1en_nodate, noauthor_httpwwwsleepreviewmagcom201806tech-fraudulent-sleep-data_nodate}. These approaches are reliable and have minimal chance of false identifications. However, the patient may feel discomfort due to the attached sensors, which hampers the confidence of the test results. Several relevant experiments achieve identity verification by remotely monitoring and verifying the subject's physiological or behavioral traits. Common methods use imaging or audio sensors to perform facial-based or voice-based recognition \cite{hutchison_robust_2005, li_sound-based_2010,abushariah_voice_2012} or  authenticate the subject through behavioral characteristics, such as tactile dynamics and gait patterns\cite{collins_silhouette-based_2002,connor_biometric_2018}. Though essential, these methods are not suitable in addressing our problem due to the low-light and low-sound conditions and the subject's state of consciousness during such sleep studies.

\subsection{Radio-based Identity Verification}
Radio-based identity verification is one of among a few promising directions in addressing the aforementioned issues. It obtains unique physiological traits from radio signals reflected from the subject and does not require the subject's active involvement or ambient conditions unsuitable for the sleeping study. Several research works employ Doppler radar measurement of cardiopulmonary motion at decimeter or mmWave band for continuous user authentication \cite{ChenDopplerSignaturesRadar2008,MolchanovTargetClassificationUsing2011,RahmanNoncontactDopplerRadar2016,VanDorpFeaturebasedHumanMotion2008}. Others utilize the channel measurement protocols inherent to off-the-shelf WiFi devices to extract unique features for individual identification \cite{AbdelnasserUbiBreatheUbiquitousNoninvasive2015, LiuTrackingVitalSigns2015, ZhangWifiidHumanIdentification2016}. Recent development further incorporates advanced signal processing algorithms and machine learning techniques to improve the identification accuracy and reliability \cite{LiuContinuousUserVerificationb,IslamIdentityAuthenticationOSA2020}.

Despite many advances, two fundamental problems associated with our setting remain unsolved. First, %although the identity verification process operates on non-volitional features and require little involvement from the target users, 
the challenges to establish the connection between the radio signature and the subject's identity are largely omitted in existing work. In other words, the initial enrollment, during which the system captures the subject's physiological measurements to be compared with the traits extracted from radio signals, is contingent upon the assurance of the user's identity, which must be verified in a more reliable method. Second, the results in existing work are mostly obtained through single-subject experiments under controlled settings \cite{rahman_doppler_2018,lin_cardiac_2017,shi_contactless_2018,islam_identity_2019}. The challenges to apply radio-based approaches in complex environments subject to disruptive events and multiple targets, e.g., scenarios mostly encountered for in-home sleep arrangements, remain to be addressed.

\subsection{Co-Presence Authentication}
Another group of authentication methods applicable to test compliance tracking is co-presence verification, through which authenticator-certified devices perceive roughly the same ambient context via their on-board sensors. Context-based co-presence verification has been a long-standing challenge in security research. In \cite{ScannellProximitybasedAuthenticationMobile2009, NarayananLocationPrivacyPrivate2011, ZhengLocationBasedHandshake2017}, fluctuations in the radio signal have been used in verifying the immediate proximity between unmet/unassociated users/devices. In \cite{SchurmannSecureCommunicationBased2011, MiettinenContextBasedZeroInteractionPairing2014, HanYouFeelWhat2018, PutzAcousticIntegrityCodes2020}, mutually-observed ambient context--such as sound, luminosity, and the correlation between different sensory modalities-- have been exploited to secure the trust between legitimate parties. Since the context information are usually noisy and differ between observers, co-presence verification predominately incorporates error-tolerant algorithms to match close-to but not identical data. Common techniques include distance-bounding protocols \cite{SingeleeLocationVerificationUsing2005}, fuzzy extractor \cite{NarayananLocationPrivacyPrivate2011, ZhengLocationBasedHandshake2017}, and commitment schemes \cite{MiettinenContextBasedZeroInteractionPairing2014}, and machine learning classifiers \cite{WuLearningDevicePairing2018}. 

There exist two antithetical issues regarding context-based co-presence verification, however. On one hand, momentary snapshots of the ambient context contain little entropy to be robust against forgery or brute-force attacks. In our case, the intrinsic entropy of a person's short-term breathing pattern can be as low as 3 bits (based on our empirical analysis),  which is significantly lower than the entropy level required for security credentials. On the other hand, long-term observations of the ambient context may increase the randomness of the shared experience, but require strict data synchronization and processing techniques to extract usable fingerprints that serve as proof of sustained co-presence among devices.