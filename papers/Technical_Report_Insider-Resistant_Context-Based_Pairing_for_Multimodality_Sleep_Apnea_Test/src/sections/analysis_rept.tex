% MASCOT's security objective can be regarded as a particular case of \textit{chain-of-custody} in telemedicine and telecare, which mandates the proper protection of medical information as they are transferred from one entity to another. Specifically, the chain-of-custody in MASCOT protects the integrity and confidentiality of the patient's compliance tracking data, which are exchanged multiple times between modalities in the presence of environmental/adversarial challenges. The custody chain's security depends on MASCOT's capability to identify the patient's breathing pattern and securely/correctly evolve the key shared between modalities. While the performance of the former can only be assessed via experimentation (Detailed in Sec. \ref{sec:evaluation}), the security of the latter can be formally analyzed in the following.

The security of SIENNA can be formally analyzed based on the property of a fuzzy commitment, and extended into three cases according to the attacker's knowledge on the user's breathing fingerprint, $f$.

\subsection{Concealment and Binding}
Following the analysis conventions of the fuzzy commitment scheme by Juels and Wattenberg \cite{JuelsFuzzyCommitmentScheme1999}, we employ two metrics to analyze SIENNA's key evolution protocol, e.g., \textit{concealment} and \textit{binding}. 
%Note that MASCOT is more complex than a regular fuzzy commitment scheme since it relies on physical-layer jamming to defend against insider attacks, e.g., attacks from a malicious patient. Hence, we start with the standard fuzzy commitment then advances to the combination of fuzzy commitment and jamming to analyze the security achieved by the duo.

\begin{definition}
\label{def:concealment and binding}
A binary fuzzy commitment, $\{\chi, \mathcal{H}(\sigma)\} = \textsc{Commit}(\sigma, \phi)$ where $\sigma \in_{R} \{1,0\}^{\nu}$ and $\mathcal{H}(\cdot) : \{1,0\}^{*} \mapsto \{1,0\}^{\mu}$,%\footnote{Here $\in_{R}$ indicates the value is selected independently and uniformly at random.}
is said to be concealing if it is infeasible for any polynomially-bounded player to guess $\sigma$ with probability greater than $p = 1/2^{\nu}$. It is said to be binding if it is infeasible for any polynomially-bounded player to retrieve $\sigma$ using a incorrect opening feature, e.g., $\hat{\phi},\; \textsc{Ham}(\phi, \hat{\phi}) > \tau$, with probability greater than $p = 1/2^{\mu}$. 
\end{definition}

\subsection{Attacker without Knowledge of $f$}
When the eavesdropper does not have the correct context, SIENNA inherits the security properties of a fuzzy commitment, which we reiterate below, with the proof omitted to conserve space.
\begin{claim}
\label{clm:concealment}
Suppose $s \in_{R} \{1,0\}^{N*{2^K}}$ and $f \in_{R} \{1,0\}^{M*{2^K}}$, the fuzzy commitment construction used by SIENNA, with $l = \textsc{RS}(2^K, M, N, s) \in_{R} \mathbb{F}^M_{2^K}$ and $c = f \ominus l$, is concealing with $p = 1/2^{N*2^K}$ against an attacker with no prior knowledge of $f$.
\end{claim}
% \begin{proof}
% When $s$ (hence $l$) and $f$ are selected independently and uniformly at random, the mutual information between $c$ and $l$ is
% \begin{equation*}
%     I(l:c) = 0,
% \end{equation*}
% since XOR-gate, $\ominus$, is synergistic, meaning $c$ is uncorrelated with the singletons $l$ and $f$ without the knowledge of the other variable. Therefore, knowing only $c$ and not $f$ reveals no information about $l$.

% If there exists a random oracle, which takes the input $\{c, \mathcal{H}(s)\}$ and outputs $\hat{l}$ in time $T$ with probability $p(T)$, such that the hash of $\overline{RS}(2^K, M, N, \hat{l})$ matches $\mathcal{H}(s)$, the attacker may query the oracle with a tuple $\{r, \mathcal{H}(z)\}$ where $r$ and $z$ are independent random values, and obtain $z$ in time $T$ with probability $p(T)$. In other words, for an attacker with no prior knowledge of $f$, determining $l$ (hence $s$) is as hard as inverting $\mathcal{H}(\cdot)$. 

% In particular, when $s \in_{R} \{1,0\}^{N*{2^K}}$, the fuzzy commitment satisfies the concealment definition in Def. \ref{def:concealment and binding} with $p(T) = 1/2^{N*2^K}$.
% \end{proof}

\begin{claim}
\label{clm:binding}
Suppose $s \in_{R} \{1,0\}^{N*{2^K}}$ and $f \in_{R} \{1,0\}^{M*{2^K}}$, the fuzzy commitment construction used by SIENNA, with $l = \textsc{RS}(2^K, M, N, s) \in_{R} \mathbb{F}^M_{2^K}$ and $c = f \ominus l$, is binding with $p = 1/2^{M*2^K}$ and $\tau = 2^{K-1}(M-N)$.
\end{claim}
% \begin{proof}
% Suppose an attacker is to find $\hat{f}$ such that $\textsc{Ham}(f, \hat{f}) > 2^{K-1}(M-N)$ but $\mathcal{H}\left(\textsc{Open}(c,\hat{f})\right)  = \mathcal{H}(s)$, we have $\hat{l} = c \ominus \hat{f}$ and 
% \[
% \textsc{Ham}(l, \hat{l}) > 2^{K-1}(M-N),
% \]
% \[
% \hat{s} = \overline{RS}(2^K, M, N, \hat{l}) \neq s,
% \]
% but
% \[
% \mathcal{H}(s) = \mathcal{H}(\hat{s}).
% \]
% Then the attacker find a collision on $\mathcal{H}(\cdot)$. In other words, the probability to produce an opening feature collision is equal to the probability to produce a collision for the hash function  $\mathcal{H}(\cdot)$.

% In particular, when $\mathcal{H}(\cdot):  \{1,0\}^{*} \mapsto \{1,0\}^{M*{2^K}}$ is a cryptographic hash, the fuzzy commitment satisfies the binding definition in Def. \ref{def:concealment and binding} with $p = 1/2^{M*2^K}$ and $\tau = 2^{K-1}(M-N)$.
% \end{proof}

Overall, Claim \ref{clm:concealment} and \ref{clm:binding} characterize the hardness for an attacker without prior knowledge of $f$ to determine $s$ from $\{c, \mathcal{H}(s)\}$, and identify two security parameters, $\nu$ and $\mu$, which govern the security level for concealment and binding. Assuming that the most effective means of finding a collision for a hash function is a birthday attack, which induces a work factor of $2^{\mu/2}$, we can set $\nu = 128$, and $\mu = 256$ to guarantee strong concealment and binding properties, with the hardness similar to finding a collision in SHA-256.

\subsection{Attacker with General Knowledge of $f$}
While the binding level of Claim \ref{clm:binding} holds regardless of the opening feature $f$'s probability distribution, the concealment level of Claim \ref{clm:concealment} would fall if $f$ is drawn from a non-uniform distribution known to the attacker. Specifically, by knowing the distribution of $f$, the attacker's strategy to determine $\hat{l}$ in the proof for Claim \ref{clm:concealment} can be computationally less expensive than inverting $\mathcal{H}(r)$ for a uniform random value, $r$. 

% One way to improve the concealment level against an attacker with knowledge of $f$'s distribution is to commit and decommit $l$ via multiple samples/segments of $f$ as shown in Section \ref{section:respiration fingerprinting}. The security enhancement is due to Yao's XOR lemma \cite{GoldreichYaoXORlemma2011}, which states that computational weak-unpredictability of Boolean predicates is amplified when the results of several independent instances are XOR together. In other words, the attacker's advantage due to bias in $f$'s distribution diminishes as we increase the number of $f$s in the XOR chain, with the diminishing rate defined in \cite{GoldreichYaoXORlemma2011}.


% \begin{figure*}[ht]
% \begin{subfigure}[t]{0.24\textwidth}
% \includegraphics[width=\textwidth,height=\textwidth]{figures/AlvinAlone.png}
% \end{subfigure}
% \hspace{\fill}
% \begin{subfigure}[t]{0.24\textwidth}
% \includegraphics[width=\textwidth,height=\textwidth]{figures/AlvinThomasClose.png}
% \end{subfigure}
% \hspace{\fill}
% \begin{subfigure}[t]{0.24\textwidth}
% \includegraphics[width=\textwidth,height=\textwidth]{figures/AlvinThomasFar.png}
% \end{subfigure}
% % \hspace{\fill}
% % \begin{subfigure}[t]{0.19\textwidth}
% % \includegraphics[width=\textwidth,height=\textwidth]{figures/Substitute.png}
% % \end{subfigure}
% \hspace{\fill}


\subsection{Attacker with Perfect Knowledge of $f$}
The XOR-chain trick would not prevent an attacker with perfect knowledge of $f$ to retrieve $s$. Specifically, consider a malicious patient capable of measuring his own breathing patterns. Were he also able to capture the commitment message, $\{c, \mathcal{H}(s)\}$, he can accurately compute $l = c  \ominus f$ and decode to obtain $s$. To prevent such an insider attack, SIENNA leverages friendly friendly jamming at the physical layer to obfuscate the commit message for any unintended receivers.

While SIENNA's friendly jamming technique is universally applicable at the physical layer of any digital communication system, it is particularly effective when augmenting an OFDM system, due to pseudorandom character of the signal. During OFDM modulation, a binary sequence is converted into $N$ complex numbersin the frequency-domain, $X_n$, via quadrature amplitude modulation (QAM), then converted into a time-domain sequence, $x_k$ via the inverse fast Fourier transform (IFFT),
\[
x_k = \sum_{n=0}^{N} X_n e^{i2\pi k n / N}.
\]
Each $x_k$ can be regarded as a weighted sum of $N$ pseudorandom variables due to the IFFT, resulting a pseudorandom Gaussian signal according to the central limit theorem (CLT). When the jamming signal is drawn randomly from a zero-mean Gaussian with the same variance of the OFDM signal, a single-antenna attacker cannot distinguish the jammed and clear signal\footnote{The authors also have shown that channel-randomized version of the jamming scheme is robust against eavesdropping from a multi-antenna attacker \cite{SteinmetzerLockpickingPhysicalLayer2015, PanROBinKnownPlaintextAttack2020, MelcherIJamChannelRandomization2020}}, therefore cannot properly reconstruct $\{c, \mathcal{H}(s)\}$.

We can analyze the jamming protection against an insider attack based on a wiretap channel model \cite{LiangCompoundWiretapChannels2009}. Consider a non-compliant patient using an unauthorized receiver to intercept the commit message. We denote the main channel as the wireless channel between the $a$ and $b$ and the wiretap channel as the one between the unauthorized receiver and one of $a$ and $b$. The frequency-domain representation of the main channel is
\[
Y_{\text{main}} = X + \frac{P_0}{P_1}\mathcal{N}(0, \sigma^2_0),
\]
and the frequency-domain representation of the wiretap channel is
\[
Y_{\text{tap}} = X + \frac{P}{P_2}\mathcal{N}(0, \sigma^2),
\]
where $P_0$ and $\sigma^2_0$ denote the average power and variance of the intrinsic wireless noise, $P_1$ and $P_2$ denote the average powers of the OFDM signal observed by the receiver and the unauthorized receiver, and $P$ and $\sigma^2$ denote average power and variance of the jamming signal observed by the unauthorized receiver. The secrecy capacity \cite{WynerWiretapChannel1975} of the wiretap model is
\[
C_s = \left[\log\left(1 + \frac{P_1}{P_0}\right) - \log\left(1 + \frac{P_2}{P}\right)\right]^{+}.
\]

It has been shown in \cite{GollakotaPhysicalLayerWireless2011}, the jamming scheme works at its optimal when the OFDM system operates with high order modulation (at least QPSK), and $1 < P/P_2 < 9$.\footnote{The jamming signal is too weak to degrade the OFDM signal when $P/P_2 \leq 1$, and too strong to be indistinguishable from the OFDM signal when  $P/P_2 \ge 9$.} Therefore, SIENNA prohibits the transmit in BPSK at any SNR. The bit error probability for such an OFDM system, allowing only M-QAM transmission, is 
\[
B_{\text{main}} \simeq \frac{4}{\log_2{M}}\textsc{Q}\left( \sqrt{\frac{3 P_1 \log_2{M}}{P_0(M-1)}} \right)
\]
for the main channel and 
\[
B_{\text{tap}} \simeq \frac{4}{\log_2{M}}\textsc{Q}\left( \sqrt{\frac{3 P_2 \log_2{M}}{P(M-1)}} \right),
\]
for the wiretap channel, where $\textsc{Q}(\cdot)$ denotes the the tail distribution function of the standard normal distribution. The receiver may adjust $P$ to elevate $B_{\text{tap}}$ beyond the error correction capability of the fuzzy commitment, and prevent an insider attack. The issue is that the jamming is only effective when $1 < P/P_2 < 9$, but $P_2$ depends on the location of the unauthorized receiver and is unknown to the receiver. Our solution is to have the transmitter create $L$ commitments, each with one sub-salt, and transmit them one by one, while the receiver jams at $L$ different power levels, $\{P_{\text{max}}, P_{\text{max}}/9, \ldots, P_{\text{max}}/9^{L-1}\}$. Given the fact that $B_{\text{main}}$ is not affected by $P$, the receiver can recover all sub-salts and XOR them together to obtain the key evolution salt. In contrast, the unauthorized receiver will fail to decode at least one sub-salt, therefore cannot recover the key evolution salt. The number of jamming levels, $L$, can be computed based on the upper bound (the maximum power supported by the hardware, $P_{\text{max}}$) and lower bound (the noise floor, $P_0$) on the OFDM signal power. 