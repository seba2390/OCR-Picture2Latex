%As healthcare providers leaning towards at-home sleep test, the task to verify whether patients have faithfully underwent the medical exams becomes increasingly challenging. Members of several professions within the patient population are deeply concerned that positive test results. As a result, some may seek to exploit the unsupervised in-home setting to manipulate the test results. 
Before introducing SIENNA, we briefly review the mechanisms of two common at-home OSA screening modalities: respiratory belt and non-contact PRMS. An external motion respiratory belt sensor utilizes a transducer to generate a substantial linear signal in response to changes in thoracic circumference associated with respiration (Fig. \ref{fig:preliminary}a). The linear signal is first sampled by a analog-to-digital converter (commonly at 100 Hz), then transmitted to a mobile OSA app.

\begin{figure}[t]
\begin{subfigure}[t]{0.23\textwidth}
\includegraphics[width=\textwidth]{figures/resp_belt.eps}
\end{subfigure}
\hspace{\fill}
\begin{subfigure}[t]{0.23\textwidth}
\includegraphics[width=\textwidth]{figures/prms.eps}
\end{subfigure}
\caption{Left to right: (a) The respiratory belt detects chest displacement (breathing) through changes in thoracic or abdominal circumference; (b) The PRMS detects displacement (breathing) in patient's chest through phase offset between the TX and RX signal.}
\label{fig:preliminary}
\vspace{-15pt}
\end{figure}

% Here we provide an overview the fundamental of Continuous Wave (CW) Doppler radar theory, which enables the applications of at-home uuobtrusive sleep apnea screening.
% mmWave CW Doppler radars have been commonly used to monitor sleep related disorders, such as sleep apnea, in a home environment. The symptoms of sleep apnea usually involve stoppages in respiration that wakes patients over night. These temporary stoppages are abnormalities in respiratory rhythm that are visible by observing a patient's chest and abdominal movements. In the event of sleep apnea, the patient's normal inhale and exhale patterns are interrupted, and his/her chest and abdomen cease to expand or contract for a short period due to a lack of motion from the diaphragm and intercostal muscles. These stases can be detected by a mmWave CW Doppler radar installed above the bed with its aperture facing the mattress, as shown in Fig. \ref{fig:radar primer}.

The PRMS (Fig. \ref{fig:preliminary}b) utilizes Continuous Wave (CW) Doppler radar technology to detect the phase shift of reflected signals from the patient's chest movements. Let the distance offset due to chest movements be $x(t)$, and the in-phase (\textit{I}) and quadrature phase (\textit{Q}) can expressed as:
\begin{align*}
  {B}_I(t) &= A_I\cos\left(\theta_0+\frac{4\pi x(t)}{\lambda}+\delta\theta(t)\right)\\
  {B}_Q(t) &= A_Q\sin\left(\theta_0+\frac{4\pi x(t)}{\lambda}+\delta\theta(t)\right)
\end{align*}
where $\lambda$ is the signal wavelength, $\theta_0$ is the phase delay due to the nominal distance between the radar transmitter and the user's torso, surface scattering, and radar's RF chains, and $\delta\theta(t)$ is the residual phase noise. The phase shift corresponds to the respiratory movement and can be computed via arctangent demodulation:
\begin{equation*}
  \theta(t) = \theta_0 + \frac{4\pi x(t)}{\lambda} = \arctan \left(\frac{A_I{B}_Q(t)}{A_Q{B}_I(t)}\right).
\end{equation*}
%and is recorded to on-board storage or wirelessly transmitted to an mobile OSA app.
% Visually, the respiratory motion, $\theta(t)$, swings around the origin on the I-Q plot with the apnea arcs shorter than the arcs due to normal breathing. Viewed on either the I-t or Q-t plots,  $\theta(t)$ fluctuates for normal breathing and remains relatively constant during apnea intervals.


%Compared to the industry approaches, methods that achieve contact-free authentication/identity verification appears advantageous since they address the data fraud issue without diminishing the convenience of in-home medical test. 




%\subsection{Industry Approaches on Test Compliance} 
% Sleep disorder associated social and medical cost is exceeding \$150 billion per year as nearly one in seven people in USA suffer from chronic sleep disorder \cite{noauthor_httpswwwnhlbinihgovhealth-topicssleep-deprivation-and-deficiency_nodate,noauthor_httpspatentsgooglecompatentus8679012b1en_nodate,noauthor_httpwwwsleepreviewmagcom201806tech-fraudulent-sleep-data_nodate}. Sleep disorder has several undesirable impacts on motor vehicle operation, employment, higher earning, job promotion opportunities, education, recreation and personal life also \cite{noauthor_httpspatentsgooglecompatentus8679012b1en_nodate}. One of the most common sleep disorder is OSA which prevalence in society is comparable  to diabetes, asthma and OSA is underdiagnosed with an estimate of 80-90\% persons afflicted due to having expensive and cumbersome clinical screening method called PSG \cite{baboli_good_2015,noauthor_httpswwwnhlbinihgovhealth-topicssleep-deprivation-and-deficiency_nodate,bruyneel_unattended_2014,noauthor_httpspatentsgooglecompatentus8679012b1en_nodate}. PSG sleeping testing in clinical environment doesn’t provide patient’s regular sleeping environment due to travel concern and anxiety and many patients also exhibit a “first night effect” related to change in sleeping environment which also requires another sleep test and associated cost \cite{baboli_good_2015,noauthor_httpspatentsgooglecompatentus8679012b1en_nodate}. To alleviate the sufferings of sleep study, in-home sleep testing is gaining popularity and currently method and devices exist either remotely attended study or remotely unattended study \cite{noauthor_httpspatentsgooglecompatentus8679012b1en_nodate}. In remotely attended study, data from various sensor is transmitted from remote site for analyzing in real time or nearly real time \cite{noauthor_httpswwwnhlbinihgovhealth-topicssleep-deprivation-and-deficiency_nodate,noauthor_httpspatentsgooglecompatentus8679012b1en_nodate}. In addition to that, audio and video data is also transmitted to the remote attendant to visually and/or audibly monitor sleep study, which is quite expensive and cumbersome. Thus, remoted unattended sleep study is much popular because data from different sensor is stored and analyzed by medical practitioner later and no video/audio transmission is sent \cite{baboli_good_2015, noauthor_httpswwwnhlbinihgovhealth-topicssleep-deprivation-and-deficiency_nodate,noauthor_httpspatentsgooglecompatentus8679012b1en_nodate}. 

% One major area of the concern for remote unattended sleep study is that falsifying sleep test data by sleep-critical job holders like transportation workers, truck drivers and airline pilots \cite{noauthor_httpswwwnhlbinihgovhealth-topicssleep-deprivation-and-deficiency_nodate,noauthor_httpwwwsleepreviewmagcom201806tech-fraudulent-sleep-data_nodate}. The most compelling reasons for falsifying sleep test data are fear of losing job or employment, fear of lifestyle change, fear of possible punitive action and fear of prescribed therapy or medication \cite{noauthor_httpswwwnhlbinihgovhealth-topicssleep-deprivation-and-deficiency_nodate,noauthor_httpwwwsleepreviewmagcom201806tech-fraudulent-sleep-data_nodate}. So, integration of simple, secure, non-contact and unobtrusive biometric subject identification during unattended remote sleep study is required to overcome the limitations of the present technology \cite{noauthor_httpswwwnhlbinihgovhealth-topicssleep-deprivation-and-deficiency_nodate,noauthor_httpwwwsleepreviewmagcom201806tech-fraudulent-sleep-data_nodate}.
% Existing industrial solutions utilizes skin-contacting sensors to ensure test compliance. For instance, several companies tether their sleep monitoring devices with bracelet or fingertip based vital sign sensors. Prior of the test, a medical technician identifies the patient and secures the bracelet/fingertip to the patient’s body via a tamper-proof connector. If the bracelet/fingertip are removed during the test, the monitoring device cease to function, which stops the sleep test. The contact-based authenticators are reliable with minimal chance of false detections. However, the patient may feel discomfort due to the attached sensor, which hampers the resluts of the sleep study.

%Industry attempt by Vyaire medical, Inc which introduced custody chain algorithm by incorporating tamper-resistant bracelet for patient verification \cite{noauthor_httpwwwsleepreviewmagcom201806tech-fraudulent-sleep-data_nodate}. Their solution comprises of mechanical technology like bracelet with the combination of biometric verification by correlating the signal from securely fastened pulse oximeter\cite{noauthor_httpwwwsleepreviewmagcom201806tech-fraudulent-sleep-data_nodate}. So, the bracelet was employed for attaching patients with the pulse oximetry system for verification of their identity \cite{noauthor_httpwwwsleepreviewmagcom201806tech-fraudulent-sleep-data_nodate}. One of the potential advantages of the system is it minimizes the risk of false negative slipping through the system. Another company named Itamar medical, Inc also used same bracelet technology which is small plastic band containing electronic circuit that need to be worn around the wrist for identity authentication of patients during unattended remote sleep study \cite{noauthor_httpwwwsleepreviewmagcom201806tech-fraudulent-sleep-data_nodate} . One of the important change they brought was lengthening of the bracelet to accommodate patients with a larger wrist size \cite{noauthor_httpwwwsleepreviewmagcom201806tech-fraudulent-sleep-data_nodate}. Another company named Virtuox’s Verisleep and it also used pulse oximetry for identity authentication \cite{noauthor_httpwwwsleepreviewmagcom201806tech-fraudulent-sleep-data_nodate}. One of the potential advantage of this company product is it worked by interrupting the pulse oximetry ground wire from the home sleep test device and routing through the company’s connector so that If the device is not connected properly there will be no power to the pulse oximetry and heart rate channels. Therefore, it will not allow uploading any fraudulent sleep test data and it was mostly used for federal motor carrier safety administration medical examination. Recently another company named CleveMed, Inc patented their in-home sleep monitoring system by incorporating fingertip sensor for continuous verification of patients \cite{noauthor_httpwwwsleepreviewmagcom201806tech-fraudulent-sleep-data_nodate}. The main invention of the system was less obtrusive sensing system like fingertip sensor, and it was worn under clothing. People also feel uncomfortable wearing traditional mechanical bracelet system \cite{noauthor_httpwwwsleepreviewmagcom201806tech-fraudulent-sleep-data_nodate}. 

% with the sleep monitoring system which might increase the uncomfortableness during sleep test and may hamper the sleep pattern.

% Moreover, none of the work also focused on solving the multi-patient verification scenarios. However, there is a higher probability of presence of multiple occupants as people share their bed with partners. Thus, unobtrusive, wireless and secure identity authentication for multi-subject scenarios during remote unattended sleep study is required to solve the limitations of the recent system. 

