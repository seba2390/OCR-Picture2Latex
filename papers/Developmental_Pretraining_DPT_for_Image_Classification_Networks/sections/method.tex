\section{Proposed DPT Regime}
In our \textsc{Developmental PreTraining} (DPT) approach, we incorporate features from early visual development in humans to create a novel curriculum-based pre-training regime. In this section, we outline, in detail, the components of this regime.

\subsection{Choice of Data}
The DPT approach, in direct contrast with the traditional approach of pre-training on a large dataset like ImageNet \cite{Deng2009ImageNet}, needs to be minimal in the visual representation it learns while also being useful. This allows the network to possess knowledge of basic visual processing that can then be utilised by downstream tasks using means like Transfer Learning. For this reason, we turned to cognitive neuroscience and early visual development in infants (see subsection \ref{subsec_cogneuro_visual}) for inspiration on the data to use in the pre-training regime.

We split DPT into two phases (with plans of extending it to multiple phases in future work) where we exposed the network with a handpicked data curriculum designed to ingrain meaningful priors. Our first phase leveraged the notion that the human visual system comes ingrained with the ability of finding strong continuous lines and contiguous surfaces which are attributed to visual cortical micro-circuitry in our early brains \cite{linsley2020recurrentedge}. This inductive bias for edges in the early human visual system was the focus of \textbf{Phase 1}. DPT then transition from edges to simple structures consisting of edges - basic shapes in \textbf{Phase 2}. This allowed the network to learn representations of shapes formed by a collection of lines. The features chosen to train in DPT were designed to be primitive and low-level such that the knowledge can be transferred to any and all fields. An overview of the phases can be found in \ref{fig:fig1}.

\subsection{Phased Pre-Training}
\subsubsection*{Phase 1}
The first phase of DPT was designed to simulate the bias of early human visual systems to edges. This was achieved with the help of an edge detection task.
\paragraph{Architectural Changes}The network participating in the DPT regime would be pre-pended to a decoder block where the network would function as the encoder. This auto-encoder setup was used to perform an edge-detection task with the dataset detailed below. Once training was completed, the deconvolutional layers in the decoder block are discarded and the convolutional layers are carried forward to the second phase.
\paragraph{Dataset}Phase 1 involves training the participating network on the BIPEDv2 Dataset \cite{soria2023bipedv2}. These outdoor images are originally 1280$\times$720 (resized to 256$\times$256 for DPT) pixels each with ground truths that are annotated by experts in the computer vision field. There are 200 training images and 50 testing images.

\subsubsection*{Phase 2}\label{subsec-p2}
The second phase of DPT was designed to extend the representations learnt by the convolutional layers of the participating network to a shape recognition task. This natural, difficulty-based progression was inspired by the previously-mentioned Curriculum Learning techniques.

\paragraph{Architectural Changes}The convolutional layers with optimised weights for edge detection from Phase 1 are appended with more convolutional layers and a final classification layer. The output layer has 9 neurons denoting the 9 possible classes in the shape dataset from below. 

\paragraph{Dataset}A Geometric 2D shape dataset \cite{el2020shapes2d} was used for Phase 2. There are 9 classes in the data where each class represents a different 2D geometric shape (Triangle, Square, Pentagon, Hexagon, Heptagon, Octagon, Nonagon, Circle and Star). Each class consisted of 100,000 images with dimensions of 256$\times$256.

\begin{figure}
  \centering
  \includegraphics[width=1\textwidth]{figs/DCV_Phases.png}
  \caption{Overview of the DPT phases. Phase 1 teaches the participating network edge detecting representations, Phase 2 teaches the network shape recognising representations. The network is then benchmarked on a dataset containing real-world objects.}
  \label{fig:fig1}
\end{figure}

\subsection{Benchmarking}
Upon the completion of the pre-training phases, we benchmarked our model with a vanilla, control model of the same architecture on a real-world dataset.
\paragraph{Architectural Changes}The classification layer from Phase 2 is discarded and replaced with two new Dense layers. First is a hidden layer for fine tuning and the second is the classification layer.
\paragraph{Dataset}The Imagenette dataset \cite{imagenette} was chosen for the benchmarking. The dataset is a subset of ImageNet with 10 of the easily classified classes (tench, English springer, cassette player, chain saw, church, French horn, garbage truck, gas pump, golf ball, parachute). Imagenette was chosen to evaluate the model's capability of recognising complex features found in everyday objects.

