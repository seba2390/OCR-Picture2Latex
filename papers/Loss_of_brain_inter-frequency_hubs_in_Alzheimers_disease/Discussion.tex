% !TEX root = Guillon2017_arxiv.tex

%% Short Recap

Graph analysis of brain networks have been largely exploited in the study of AD with the aim to extract new predictive diagnostics of disease progression.
Typical approaches in functional neuroimaging, characterized by oscillatory dynamics, analyze brain networks separately at different frequencies thus neglecting the available multivariate spectral information.
Here, we adopted a method to formally take into account the topological information of multi-frequency connectomes obtained from source-reconstructed MEG signals in a group of AD and healthy subjects during EC resting states.

%% Multiplex Results

Main results showed that, while flattening networks of different frequency bands attenuates differences between AD and HC populations, keeping the multiplex nature of MEG connectomes allow to capture higher-order discriminant information.
AD subjects exhibited an aberrant multiplex brain network structure that significantly reduced the global propensity to facilitate information propagation across frequency bands as compared to HC subjects (\autoref{fig:participation}b, inset). This could be in part explained by the higher variability of the individual node degrees across bands (\autoref{fig:coefficient_of_variation}).

% NOTE: High MPCi does not necessarily mean high oi (autoref fig:mpca) but it also seems that having a high number of connections (high oi) and a low MPC is not possible in the case of the brain.

% NOTE: In general, a ROI with a high MPC but with low oi, will have an even higher MPC if its oi increase (for another subject for instance). In clear: for a given i (i.e. a given ROI), the corrcoeff between oi and MPCi is always positive.

% NOTE: I tested different thresholds and with the ImCoh to check if it was not because of the week noisy connections, but the distribution of MPCi values seems to be always the same. Even with an average degree of 1 meaning that the brain always tends to keep connections in multiples frequency bands in the same time. Could it be explained by the fact that coherence is influenced by harmonics?

Such loss of inter-frequency centrality was mostly localized in association areas as well as in the cingulate cortex (\autoref{fig:participation}b; \autoref{tab:local_participation}), which resulted the most important hub promoting interaction across bands in the HC group (\autoref{fig:mpc}a).
Because all these areas are typically affected by AD atrophy \citep{wenk_neuropathologic_2003} we hypothesize that the anatomical withering might have impacted the neural oscillatory mechanisms supporting large-scale brain functional integration. Notably, the significant alteration of the connectivity across bands observed in the cingulate cortex could be ascribed to typical M/EEG connectivity changes observed in AD, such as reduced $alpha$ coherence \citep{stam_magnetoencephalographic_2006,jeong_eeg_2004,dauwels_diagnosis_2010,wang_power_2015} (\autoref{fig:mpc}b).
We also found a significant decrease in the primary motor cortex (right precentral gyrus). While previous studies have identified this specific region as a connector hub in human brain networks \citep{tijms_alzheimers_2013}, its role in AD still needs to be clarified in terms of node centrality's changes with respect to healthy conditions.
%For these affected ROIs the decreased centrality was reflected by fewer interactions with higher sensory rhythms ($>20$ Hz) \citep{basar_review_2013} and more connections to lower attentional ones ($<13$ Hz) \citep{klimesch_EEG_1999} (\autoref{fig:participation}c).

% Single-Layer Results
While flattening network layers represents in general an oversimplification, analyzing single layers can still be a valid approach that is worth of investigation.
Because the $MPC$ is a pure multiplex quantity, we considered the conceptually akin version for single-layer networks, the standard participation coefficient $PC$, which evaluates the tendency of nodes to integrate information from different modules, rather than from different layers \citep{guimera_cartography_2005, battiston_structural_2014}.
AD patients exhibited lower inter-modular connectivity in the \textit{gamma} band with respect to HC subjects (\autoref{fig:participation}a; \autoref{tab:local_participation}) that was localized in association areas including frontal, temporal, and parietal cortices (\autoref{fig:participation}a; \autoref{tab:local_participation}).
%
Damages to these regions can lead to deficits in attention, recognition and planning \citep{purves_neuroscience_2001}. Our results support the hypothesis that AD could include a disconnection syndrome  \citep{pearson_anatomical_1985,arnold_topographical_1991,catani_rises_2005}.
Furthermore, they are in line with previous findings showing $PC$ decrements in AD, although those declines were more evident in lower frequency bands and therefore ascribed to possible long-range low-frequency connectivity alteration \citep{de_haan_disrupted_2012,tijms_alzheimers_2013}.

%% Conclusion
Put together, our findings indicated that AD alters the global brain network organization through connection disruption in several association regions, which play important roles in sensory processing by integrating information from other cortical regions through high-frequency channels \citep{miltner_coherence_1999-1,buschman_top-down_2007, siegel_neuronal_2008, gregoriou_high-frequency_2009, hipp_oscillatory_2011}.
%
Notably, we showed that the global loss of inter-modular interactions in the \textit{gamma} band is paralleled by a diffused decrease of inter-frequency centrality.
Future studies, involving recordings of limbic structures and/or stimulation-based techniques, should elucidate whether these two distinct reorganizational processes are truly independent or linked through possible cross-frequency mechanisms which are known to be essential for normal memory formation \citep{canolty_high_2006,axmacher_cross-frequency_2010, goutagny_alterations_2013}.


%% Classification Results

As a confirmation of the complementary information carried out by the multi-layer approach, we reported an increased classification accuracy when combining the local $PC$ and $MPC$ features.
The observed diagnostic power is in line with previous accuracy values obtained with standard graph theoretic approaches (around $80\%$) but exhibits slightly higher sensitivity ($>90\%$), which is often desired to avoid false negatives \citep{li_discriminant_2012, wang_disrupted_2013, wee_enriched_2011, wee_identification_2012, horwitz_functional_2011}.
Other approaches should determine if and to what extent the use of more sophisticated machine learning algorithms, or the inclusion of basic connectivity features \citep{hutchison_network-based_2011, shao_prediction_2012, zhou_hierarchical_2011} and different imaging modalities \citep{dai_discriminative_2012}, can lead to higher classification performance and better diagnosis \citep{tijms_alzheimers_2013}.

%% Correlation With MMSE

Previous works have documented relationships between brain network properties and neuropsychological measurements in AD, suggesting a potential impact for monitoring disease progression and for the development of new therapies
\citep{de_haan_functional_2009,lo_diffusion_2010,sanz-arigita_loss_2010-1,shu_disrupted_2012,stam_small-world_2007,wang_disrupted_2013}.
This is especially true for the standard $PC$ which has exhibited stronger correlations and larger between-group differences \citep{tijms_alzheimers_2013}.
In line with this prediction, we also reported significant correlations between the MMSE cognitive scores and the $PC$ values of the AD patients in the \textit{gamma} band (\autoref{fig:correlations}a).
%
An even stronger correlation was found, however, for the global $MPC$ values and the TR scores (\autoref{fig:correlations}b, \autoref{tab:local_correlation}).
Recent studies suggest that TR scores could be more specific for AD \citep{grober_free_2010, velayudhan_review_2014} as compared to MMSE scores which could be biased by differences in years of education, lack of sensitivity to progressive changes occurring with AD, as well as fail in detecting impairment caused by focal lesions \citep{tombaugh_mini-mental_1992}.
Locally, the regions whose $MPC$ correlated with TR were part of the default-mode network (DMN) (\autoref{tab:local_correlation}), which is heavily involved in memory formation and retrieval \citep{buckner_brains_2008,sperling_functional_2010}. According to recent hypothesis, these areas are directly affected by atrophy and metabolism disruption, as well as amyloid-$\beta$ deposition \citep{buckner_molecular_2005, greicius_default-mode_2004}.
Put together, our results suggest that AD symptoms related to episodic memory losses could be determined by the lower capacity of strategic DMN association areas to let information flow across different frequency channels.

\subsection*{Methodological considerations}

We estimated brain networks by means of spectral coherence, a connectivity measure widely used in the electrophysiological literature because of its simplicity and relatively intuitive interpretation \citep{srinivasan_eeg_2007}.
While this measure is known to suffer from possible volume conduction effects, recent evidence showed that source reconstruction techniques, like the one we adopted here, could at least mitigate this bias \citep{schoffelen_source_2009} and generate connectivity patterns consistent within and between subjects \citep{colclough_how_2016}.
In a separate analysis, we used the imaginary coherence as a candidate alternative to eliminate volume conduction effects \citep{nolte_identifying_2004}. We demonstrated that while no significant between-group differences could be obtained in terms of $MPC$ (data not shown here), the spatial distribution of the $MPC$ values was very similar to that observed in the brain networks obtained with the spectral coherence, especially for the internal regions along the longitudinal fissure (\autoref{fig:mpc_imcoh}).

Differently from other multiplex network quantities, such as those based on paths and walks \citep{boccaletti_structure_2014}, the $MPC$ has the advantage to not depend on the weights of the inter-layer links which, in general, are difficult to estimate or to assign from empirically obtained biological data. This is especially true in network neuroscience where, so far, the strength of the inter-layer connections is parametric and subject to arbitrariness \citep{de_domenico_mapping_2016} or estimated through measures of cross-frequency coupling \citep{brookes_multi-layer_2016-1} whose biological interpretation remains still to be completely elucidated \citep{jirsa_cross-frequency_2013}.
