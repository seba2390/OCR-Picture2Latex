% !TEX root = Guillon2017_arxiv.tex

% \subsection{Correlation with the MMSE}

% There is no correletation between the MMSE and the \textit{global} multi-participation coefficient (Figure 5.a). Locally some ROIs show either a positive correlation (?, ?, ?) or a negative correlation (...). We can note that the positively correlated ROIs also have a significantly higher multi-participation coefficient in controls (Figure 4.b).
% The small blue region isn't significant in terms of difference of MPC, so its correlation might be "rally linked to the MMSE". Because a correlation associated to a significantly different ROI is easly explained since all the HCs have a hight MPC, and all the ADs a low one... If we plot the correlation curve, it should look like two stairs steps overlaped by a positive linear curve.

\renewcommand{\thefigure}{S\arabic{figure}}
\renewcommand{\thetable}{S\arabic{table}}
\renewcommand{\theequation}{S\arabic{equation}}

\setcounter{figure}{0}
\setcounter{table}{0}
\setcounter{equation}{0}

\subsection*{Supplementary Figures} \label{subsec:FigS}

\begin{figure}[!ht]
	\centering
	\includegraphics[width=10cm]{thresholding.png}
	\caption{Statistical differences between global brain network properties of AD and HC subjects.
	These figures illustrate the $p$-values resulting from the permutation t-tests as a function of the average node degree $k$ used to threshold the layers of the multi-frequency brain networks. In panel a), we show the $p$-values for multi-layer and flattened analysis whereas in panel b) the $p$-values resulting from single-layer analysis.}
	\label{fig:thresholding}
\end{figure}

\newpage
\begin{figure}[!ht]
	\centering
	\includegraphics[width=10cm]{coefficient_of_variation.png}
	\caption{This figure shows the global coefficient of variation ($CV$): first the difference between the populations as an inset plot ($p=0.0521$) and the correlation with the global multi-participation coefficient ($MPC$) as a main plot ($p<10^{-15}$, $R=-0.9742$).}
	\label{fig:coefficient_of_variation}
\end{figure}


\newpage
\begin{figure}[!ht]
	\centering
	\includegraphics[width=14cm]{mpc_distribution_imcoh.png}
	\caption{Inter-frequency hub centrality distribution for brain networks obtained with imaginary coherence. a) The median values of local multi-participation coefficients ($MPC_i$) are shown over the cortical surface for the healthy group. Only the top $25\%$ is illustrated for the sake of visualization. The corresponding list of ROIs is illustrated in the horizontal bar plot. b) Group-median values of the node-degree layer proportion ($NLP_i$) for the right and left cingulate cortex. The grey line corresponds to the expected value if connectivity were equally distributed across frequency bands ($NLP=1/7$).}
	\label{fig:mpc_imcoh}
\end{figure}

%\newpage
%%\begin{figure}[!ht]
%%	\centering
%%	\includegraphics{modularity_indices.eps}
%%	\caption{This figure shows differences between ADs and HCs for multiples single-layer indices related to communities: the modularity ($Q$), the participation coefficient ($PC$), the within-module degree ($Z$) and the number of clusters ($M$) for each frequency band. The bars correspond to the populations means, the whiskers to the standard errors of the mean. Only $PC^{[\gamma]}$ is significantly different between the two population ($p=0.0162$). The number of clusters $M^{[\beta_2]}$ and $M^{[\gamma]}$ respectively have $p$-values equal to $0.0531$ and $0.0667$ and thus can not be considered as significant.}
	% Modularity :
	% p = 0.5037    0.6940    0.8449    0.7272    0.3777    0.4080    0.9817
	% Participation coefficient :
	% p = 0.5453    0.6517    0.5302    0.9729    0.9310    0.2325    0.0162
	% Within-module degree :
	% p = 0.3232    0.8775    0.1032    0.7166    0.8218    0.4239    0.8902
	% Number of clusters :
	% p = 0.1519    0.1221    0.5576    0.5559    0.1376    0.0531    0.0667
%	\label{fig:modularity_indices}
%\end{figure}

%\newpage
%\begin{table}[!ht]
%	\centering
%	\begin{tabular}{r|rr|rr}
%		\hline
%		\multicolumn{1}{c|}{\multirow{2}{*}{Correlation of...}} & \multicolumn{2}{c|}{...with MMSE} & \multicolumn{2}{c}{...with TR} \\
%		\multicolumn{1}{c|}{} & $R$ coefficient & $p$-value       & $R$ coefficient & $p$-value       \\ \hline
%		$PC^{[\delta]}$       & -0.0194         & 0.9268          & \textbf{0.4575} & \textbf{0.0323} \\
%		$PC^{[\theta]}$       & -0.0806         & 0.7018          & 0.0809          & 0.7206          \\
%		$PC^{[\alpha_1]}$     & -0.0190         & 0.9282          & 0.0752          & 0.7394          \\
%		$PC^{[\alpha_2]}$     & 0.3398          & 0.0965          & 0.3630          & 0.0968          \\
%		$PC^{[\beta_1]}$      & \textbf{0.4773} & \textbf{0.0158} & 0.3255          & 0.1526          \\
%		$PC^{[\beta_2]}$      & \textbf{0.4402} & \textbf{0.0277} & \textbf{0.5949} & \textbf{0.0035} \\
%		$PC^{[\gamma]}$       & \textbf{0.4909} & \textbf{0.0127} & 0.0628          & 0.7814          \\ \hline
%		$PC^{A}$              & 0.0337          & 0.8729          & 0.2143          & 0.3382          \\
%		$PC^{O}$              & \textbf{0.5107} & \textbf{0.0091} & -0.0079         & 0.9721          \\
%		$MPC$                 & -0.1128         & 0.5915          & \textbf{0.5547} & \textbf{0.0074} \\ \hline
%	\end{tabular}
%	\caption{Correlations of global indices with neuropsychological tests obtained using a Spearman test.}
%	\label{tab:global_correlation}
%\end{table}
