% !TEX root = Guillon2017_arxiv.tex

%\newpage
\begin{figure}[!ht]
	\centering
	\includegraphics[width=\textwidth]{building_multiplex.png}
	\caption{Multi-frequency brain networks.
	Panel a) shows five representative networks extracted from typical frequency bands.
	b) Procedure to construct a multi-frequency network by virtually connecting the homologous brain nodes among frequency layers.
	c) Inter-frequency node centrality. A two-layer multiplex is considered for the sake of simplicity. The blue node acts as an inter-frequency hub (i.e., multi-participation coefficient $MPC=1$) as it allows for a balanced information transfer between layer $\alpha$ and $\gamma$; the red node, who is disconnected in layer $\alpha$, blocks the information flow and has $MPC=0$.}
	\label{fig:multiplex}
\end{figure}

\newpage
\begin{figure}[!ht]
	\centering
	\includegraphics[width=\textwidth]{psd.png}
	\caption{Spectral analysis of MEG signals.
	a) Power spectrum density (PSD) for a representative occipital sensor before source reconstruction. Each line corresponds to a subject. Bold lines show the group-averaged values in the Alzheimer's disease group (AD) and in the healthy control group (HC).
	b) Statistical PSD group differences. Z-scores are obtained using a non-parametric permutation t-test. Results are represented both as sensor and source space.}
	\label{fig:psd}
\end{figure}

\newpage
\begin{figure}[!ht]
	\centering
	\includegraphics[width=10cm]{participation.png}
	\caption{Network analysis of brain connectivity.
	a) Inter-modular centrality. Statistical brain maps of group differences for local participation coefficients $PC_i$ in the $gamma$ band. Only significant differences are illustrated ($p<0.05$, FDR corrected). The labels same ranks are used as labels. The inset shows the results for the global $PC$; vertical bars stand for group-averaged values while error bars denote standard error means. In both cases, Z-scores are computed using a non-parametric permutation t-test.
	b) Inter-frequency centrality. Statistical brain maps of group differences for local multi-participation coefficients $MPC_i$. The inset shows the results for the global $MPC$; same conventions as in a).}
	% Global participation coefficient 100.000-permutations test for the sevent frequency bands, MATLAB Output :
	%
	%   pValue =
	%
	%     0.5445    0.6551    0.5337    0.9684    0.9302    0.2337    0.0160
	%
	%
	%   zScore =
	%
	%    -0.6061   -0.4494    0.6286   -0.0402   -0.0858   -1.2121   -2.4906
	%
	% Of course results vary accross runs (permutation test)
	\label{fig:participation}
\end{figure}


\newpage
\begin{figure}[!ht]
	\centering
	\includegraphics[width=14cm]{mpc_distribution2.png}
	\caption{Inter-frequency hub centrality distribution. a) The median values of local multi-participation coefficients ($MPC_i$) are shown over the cortical surface for the healthy group. Only the top $25\%$ is illustrated for the sake of visualization. The corresponding list of ROIs is illustrated in the horizontal bar plot. b) Group-median values of the node-degree layer proportion ($NLP_i$) for the right and left cingulate cortex. The grey line corresponds to the expected value if connectivity were equally distributed across frequency bands ($NLP_i=1/7$).}
	\label{fig:mpc}
\end{figure}

\newpage
\begin{figure}[!ht]
	\centering
	\includegraphics[width=14cm]{classification.png}
	\caption{Classification performance of brain network features.
	a) Matrices show the classification rates (accuracy=Acc, specificity=Spec, sensitivity=Sens, area under the curve=AUC) corresponding to the combination of the most significant $PC^{[\gamma]}_i$ and $MPC_i$ network features, respectively on the rows and columns of each matrix. Black squares highlight the highest accuracy rate and the corresponding specificity, sensitivity and AUC.
	b) Scatter plots show the Mahalanobis distance of each subject from the $AD$ and $HC$ classes.  Separation lines ($y=x$: equal distances) are drawn in grey. Red circles stand for Alzheimer's disease (AD) subjects , blue ones for healthy controls (HC).
The bottom right plot shows the ROC curve associated with the best network features configuration. The optimal point is marked by a green circle.}
	\label{fig:classification}
\end{figure}

\newpage
\begin{figure}[!ht]
	\centering
	\includegraphics[width=14cm]{correlations.png}
	\caption{Correlation between brain network properties and cognitive/memory scores.
	a) Scatter plot of the global participation coefficient in the \textit{gamma} band ($PC^{[\gamma]}$) and the mini-mental state examination (MMSE) score of AD subjects (Spearman's correlation $R = 0.4909$, $p = 0.0127$).
	b) Correlation brain maps of the local participation coefficient in the \textit{gamma} band ($PC_i^{[\gamma]}$) and the mini-mental state examination (MMSE) score of AD subjects. Only significant $R$ values are illustrated ($p<0.05$, FDR corrected).
	c) Scatter plot of the global multi-participation coefficient ($PC$) and the total recall (TR) score of AD subjects (Spearman's correlation  $R = 0.5547$, $p = 0.0074$).
	d) Correlation brain maps of the local multi-participation coefficient ($MPC_i$) and the total recall (TR) score of AD subjects. Only significant $R$ values are illustrated ($p<0.05$, FDR corrected).
}
	\label{fig:correlations}
\end{figure}

%%%%%%%%%%%%%%%%%%%%%%%%%%%%%%%%%%%%%%%%%%%%%%%%%%%%%%%%%%%%

\newpage
\begin{table}[!ht]
\footnotesize
	\centering
	\begin{tabular}{lllll}
		\hline
		     & Control (HC) & Alzheimer (AD) & $p$-value  \\ \hline
		% Gender (M/F) & 7/18       & 12/13       & ${0.01}$     \\
		Age  & 70.8 (9.1)   & 73.5 (9.4)     & $0.3142$   \\
		MMSE & 28.2 (1.4)   & 23.2 (3.6)     & $<10^{-5}$ \\
		FR   & 31.5 (6.6)   & 14.9 (6.5)     & $<10^{-5}$ \\
		% CR   & 14.9 (6.4) & 19.0 (7.5)  & $0.0481$   \\
		TR   & 46.3 (1.5)   & 33.9 (10.0)    & $<10^{-5}$ \\
		% React        & 89.6 (8.6) & 61.8 (23.2) & $<10^{-5}$ \\ \hline
	\end{tabular}
	\caption{Characteristics, cognitive and memory scores of experimental subjects. Mean values and standard deviations (between parentheses) are reported. The last column shows the $p$-values returned by a non-parametric permutation t-tests with $10\,000$ realizations. MMSE = mini-mental state examination score; TR = total recall memory test score (/48); FR = free recall memory test (/48).}
	\label{tab:clinical_data}
\end{table}

\newpage
\begin{table}[!ht]
\footnotesize

	\centering
	\begin{tabular}{rrllrr}
		\hline
		Index                              & Rank & ROI label                     & Cortex        & $Z$ score & $p$-value \\ \hline
		\multirow{5}{*}{$PC_i^{[\gamma]}$} & 1    & Lat\_Fis-ant-Horizont L       & Frontal       & -3.6507   & 0.0007    \\
		                                   & 2    & Pole\_temporal R              & Temporal      & -2.8642   & 0.0063    \\
		                                   & 3    & G\_front\_inf-Triangul L      & Frontal       & -2.4562   & 0.0198    \\
		                                   & 4    & \textbf{S\_temporal\_transverse L}     & Temporal      & -2.3887   & 0.0207    \\
		                                   & 5    & \textbf{G\_pariet\_inf-Supramar L}     & Parietal      & -2.3820   & 0.0222    \\ \hline
		\multirow{7}{*}{$MPC_i$}           & 1    & G\_precentral R               & Motor & -3.4735   & 0.0006    \\
		                                   & 2    & G\_front\_inf-Opercular R     &  Motor & -2.5239   & 0.0127    \\
		                                   & 3    & S\_oc\_middle\_and\_Lunatus L & Occipital     & -2.4582   & 0.0138    \\
		                                   & 4    & \textbf{G\_pariet\_inf-Supramar L}     & Parietal      & -2.4860   & 0.0142    \\
		                                   & 5    & \textbf{S\_interm\_prim-Jensen L}      & Parietal      & -2.3708   & 0.0147    \\
		                                   & 6    & \textbf{S\_temporal\_transverse R}     & Temporal      & -2.3996   & 0.0191    \\
		                                   & 7    & \textbf{S\_pericallosal R }            & Limbic        & -2.3041   & 0.0203    \\ \hline
	\end{tabular}
	\caption{Statistical group differences for local brain network properties. ROI labels, abbreviated according to the Destrieux atlas, are ranked according to the resulting $p$-values. The same ranks are used as labels in \autoref{fig:participation}. ROIs highlighted in bold belong to the default mode network (DMN).}
	\label{tab:local_participation}
\end{table}


\newpage
\begin{table}[!ht]
\footnotesize
	\centering
	\begin{tabular}{rrllrr}
		\hline
		Correlation                                 & Rank & ROI label                               & Cortex          & $R$ coeff. & $p$-value \\ \hline
		\multirow{6}{*}{$PC_i^{[\gamma]}$ - MMSE} & 1    & Lat\_Fis-ant-Vertical R                 & Frontal         & 0.5480          & 0.0046    \\
		                                            & 2    & G\_occipital\_sup L                     & Occipital       & 0.5005          & 0.0108    \\
		                                            & 3    & \textbf{S\_interm\_prim-Jensen R}                & Parietal        & 0.4948          & 0.0119    \\
		                                            & 4    & \textbf{G\_and\_S\_cingul-Ant R}        & Limbic         & 0.4864          & 0.0137    \\
		                                            & 5    & \textbf{S\_pericallosal R}                       & Limbic          & 0.4735          & 0.0168    \\
		                                            & 6    & G\_and\_S\_transv\_frontopol R & Frontal         & 0.4585          & 0.0212    \\ \hline

		\multirow{15}{*}{$MPC_i$ - TR}            & 1    & Lat\_Fis-ant-Horizont L        & Frontal         & 0.6915          & 0.0004    \\
		                                            & 2    & S\_collat\_transv\_post L               & Occipital       & 0.6706          & 0.0006    \\
		                                            & 3    & S\_circular\_insula\_ant L              & Frontal         & 0.6214          & 0.0020    \\
		                                            & 4    & \textbf{G\_parietal\_sup R}                      & Parietal        & 0.6061          & 0.0028    \\
		                                            & 5    & S\_orbital\_lateral R          & Frontal         & 0.5920          & 0.0037    \\
		                                            & 6    & Pole\_temporal L               & Temporal        & 0.5739          & 0.0052    \\
		                                            & 7    & S\_orbital\_lateral L          & Frontal         & 0.5462          & 0.0085    \\
		                                            & 8    & \textbf{S\_temporal\_sup R}             & Temporal        & 0.5457          & 0.0086    \\
		                                            & 9    & G\_and\_S\_occipital\_inf L             & Occipital       & 0.5368          & 0.0100    \\
		                                            & 10   & G\_occipital\_sup R                     & Occipital       & 0.5208          & 0.0130    \\
		                                            & 11   & G\_postcentral L                        & Sensory & 0.5191          & 0.0133    \\
		                                            & 12   & \textbf{G\_pariet\_inf-Supramar R}      & Parietal        & 0.5151          & 0.0142    \\
		                                            & 13   & \textbf{S\_subparietal R}               & Parietal        & 0.5066          & 0.0161    \\
		                                            & 14   & \textbf{S\_interm\_prim-Jensen L}                & Parietal        & 0.4915          & 0.0202    \\
		                                            & 15   & \textbf{S\_temporal\_inf L}             & Temporal        & 0.4869          & 0.0216    \\ \hline
	\end{tabular}
	\caption{Correlations of local brain network properties and cognitive/memory scores. ROI labels, abbreviated according to the Destrieux atlas, are ranked according to the resulting $p$-values. ROIs written in bold belong to the default mode network (DMN).}
	\label{tab:local_correlation}
\end{table}
