% !TEX root = Guillon2017_arxiv.tex
%\small{
Alzheimer's disease (AD) causes alterations of brain network structure and function. The latter consists of connectivity changes between oscillatory processes at different frequency channels.
We proposed a multi-layer network approach to analyze multiple-frequency brain networks inferred from magnetoencephalographic recordings during resting-states in AD subjects and age-matched controls. 

Main results showed that brain networks tend to facilitate information propagation across different frequencies, as measured by the multi-participation coefficient ($MPC$).
However, regional connectivity in AD subjects was abnormally distributed across frequency bands as compared to controls, causing significant decreases of $MPC$. This effect was mainly localized in association areas and in the cingulate cortex, which acted, in the healthy group, as a true inter-frequency hub. 

$MPC$ values significantly correlated with memory impairment of AD subjects, as measured by the total recall score. Most predictive regions belonged to components of the default-mode network that are typically affected by atrophy, metabolism disruption and amyloid-$\beta$ deposition. We evaluated the diagnostic power of the $MPC$ and we showed that it led to increased classification accuracy ($78.39\%$) and sensitivity ($91.11\%$).

These findings shed new light on the brain functional alterations underlying AD and provide analytical tools for identifying multi-frequency neural mechanisms of brain diseases.
%}

%diversity of intermodular connections of individual nodes

%inter-modular inter-frequency hubs facilitates the tranistion of information from one frequency to another one
