% !TEX root = Guillon2017_arxiv.tex

Power analysis of source-reconstructed MEG signals confirmed the characteristic changes in the oscillatory activity of AD subjects compared to HC subjects (\autoref{fig:psd}a) \citep{babiloni_mapping_2004,jeong_eeg_2004,dauwels_diagnosis_2010,wang_power_2015}. Significant \textit{alpha} power decreases were more evident in the parietal and occipital regions ($Z < -2.58$), while significant \textit{delta} power increases ($Z > 2.58$) were more localized in the frontal regions of the cortex (\autoref{fig:psd}b).

\subsection{Reduced \textit{gamma} inter-modular connectivity} \label{subsec:decreased_pc_in_gamma}

As expected the value of the connection density threshold had an impact on the network differences between groups. For the sake of simplicity, we selected the first threshold for which we could observe a significant group difference for both single- and multi-layer analysis. The obtained results determined the choice of a representative threshold, common to all the brain networks, corresponding to an average node degree $k=12$ (\autoref{fig:thresholding}).

We first evaluated the results from the single-layer analysis. By inspecting the global participation coefficient $PC$, we reported in the $gamma$ band a significant decrease of inter-modular connectivity in AD as compared to HC ($Z=-2.50$, $p=0.017$; \autoref{fig:participation}a inset).
This behavior was locally identified in association ROIs including temporal and parietal areas ($p < 0.05$, FDR corrected; \autoref{fig:participation}a; \autoref{tab:local_participation}).
No other significant differences were reported in other frequency bands or in flattened brain networks (\autoref{fig:thresholding}).

% Global participation coefficient 100.000-permutations test for the seven frequency bands, MATLAB Output :
%   pValue =   0.5445    0.6551    0.5337    0.9684    0.9302    0.2337    0.0160
%   zScore =  -0.6061   -0.4494    0.6286   -0.0402   -0.0858   -1.2121   -2.4906


\subsection{Disrupted inter-frequency hub centrality}

Then we assessed the results from the multi-layer analysis. Both AD and HC subjects exhibited high global multi-participation coefficients ($MPC>0.9$), suggesting a general propensity of brain regions to promote interactions across frequency bands.
However, such tendency was significantly lower in AD than HC subjects ($Z=-2.24$, $p=0.028$; \autoref{fig:participation}b inset).
This loss of inter-frequency centrality was prevalent in association ROIs including temporal, parietal and cingulate areas, and with a minor extent in motor areas ($p<0.05$, FDR corrected; \autoref{fig:participation}b; \autoref{tab:local_participation}).

Among those regions, the right cingulate cortex was classified as the main inter-frequency hub as revealed by the spatial distribution of the top $25\%$ $MPC$ values in the HC group (\autoref{fig:mpc}a).
%put in this picture the two hemisherepes 25% sectral coherence and the corresponding box plot; a thris panel would be the nlp distribution
% NLP only for cingulate (hub)
In HC subjects the connectivity of this region across bands, as measured by the node degree layer proportion $NLP$, was relatively stable (Kruskal-Wallis test, $\chi^2=10.79$, $p=0.095$), while it was significantly altered in AD subjects (Kruskall-Wallis test, $\chi^2=14.98$, $p=0.020$).
In particular, the AD group exhibited a remarkably reduced $alpha_2$ connectivity and increased $theta$ connectivity (\autoref{fig:mpc}b). Similar results were also reported for the left cingulate cortex (AD: $\chi^2=11.89$, $p=0.064$; HC: $\chi^2=6.98$, $p=0.323$), although it was not significant in terms of $MPC$ differences (\autoref{fig:participation}b; \autoref{tab:local_participation}).

% TODO: NLP for all significant regions
%The connectivity distribution for these significant regions, as measured by the node degree layer proportion $NLP$, was significantly altered in the AD group (Kruskall-Wallis test, $\chi^2=14.35$, $p=0.026$), while it was relatively stable across bands in the HC group (Kruskal-Wallis test, $\chi^2=7.59$, $p=0.270$).
%AD subjects exhibited a decreasing trend with reduced $beta_2$ and $gamma$ connectivity and increased $theta$ and $alpha_1$ connectivity, while a more constant trend was found in HC subjects (\autoref{fig:participation}c).
%In both populations, the contribution of $delta$ connectivity for these ROIs was remarkably low.

% Tijms2013 correspondances:
% ==========================
% Right IFG = G_front_inf-Operocular R (1 AD)
% Right PreCG = G_precentral R (1 AD, 2 HC)
% Right INS = S_temporal_transverse R (1 AD) ???
% Right CING = S_pericallosal R (3 AD, 3 HC) "The cingulate gyrus is limited from the corpus callosum by the pericallosal sulcus [...]" [Destrieux2010]
% Left SMG = G_pariet_inf-Supramar L (1 AD, 1 HC)
% Left MOG = S_oc_middle_and_Lunatus L (2 AD, 5 HC)
% Left ANG = S_interm_prim-Jensen L (2 AD) "The sulcus intermedius primus divides the inferior parietal lobule into supramarginal (anterior) and angular (posterior) gyri." [Destrieux2010]

\subsection{Diagnostic power of brain network features}

We adopted a classification approach to evaluate the power of the most significant local network properties in determining the state (i.e., healthy or diseased) of each individual subject.
The best results were achieved neither when we considered single-layer features (i.e., $PC^{[\gamma]}_i$) nor when we considered multi-layer features ($MPC_i$) (respectively, first column and row of panels in \autoref{fig:classification}a). Instead, a combination of the two most significant features gave the best classification in terms of accuracy ($Acc=78.39\%$) and area under the curve ($AUC=0.8625$)  (\autoref{fig:classification}a,b).
While the corresponding specificity was not particularly high ($Spec=65.68\%$), the sensitivity was remarkably elevated ($Sens=91.11\%$).


% NOTE: The only misclassified AD has the following charecteristics :
% mmseScore: 28
%       age: 81
%       sex: 'F'
%      rimm: 16
%        ri: 21
%        rl: 22
%        rt: 43
%     react: 81
% She has an abnormally high MMSE for an AD patient. Also her total recall and react score are quite similar to those of HC subjects.


\subsection{Relationship with cognitive and memory impairment}

We finally evaluated the ability of the significant brain network changes to predict the cognitive and memory performance of AD subjects.
We first considered the results from single-layer analysis. We found a significant positive correlation between the global participation coefficient $PC$ in the $gamma$ band and the MMSE score ($R=0.4909$, $p=0.0127$; \autoref{fig:correlations}a).
Then we considered the results from multi-layer analysis. We reported a higher significant positive correlation between the global multi-participation coefficient $MPC$ and the TR score ($R = 0.5547$, $p = 0.0074$; \autoref{fig:correlations}c).
These relationships were locally identified in specific ROIs including parietal, temporal and cingulate areas of the default mode network (DMN) \citep{buckner_brains_2008} ($p < 0.05$, FDR corrected; \autoref{fig:correlations}b,d; \autoref{tab:local_correlation}).
