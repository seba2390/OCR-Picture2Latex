% !TEX root = Guillon2017_arxiv.tex

\subsection{Experimental design and data pre-processing}

The study involved 25 Alzheimer's diseased (AD) patients (13 women) and 25 healthy age-matched control (HC) subjects (18 women).
All participants underwent the Mini-Mental State Examination (MMSE) for global cognition \cite{folstein_mini-mental_1975} and the Free and Cued Selective Reminding Test (FCSRT) for verbal episodic memory \citep{buschke_cued_1984, grober_screening_1988, pillon_explicit_1993}. Specifically, we considered the Total Recall (TR) score - given by the sum of the free and cued recall scores - which has been demonstrated to be highly predictive of AD \citep{sarazin_amnestic_2007}.

Inclusion criteria for all participants were: \textit{i)} age between 50 and 90; \textit{ii)} absence of general evolutive pathology; \textit{iii)} no previous history of psychiatric diseases; \textit{iv) }no contraindication to MRI examination; \textit{v)} French as a mother tongue.
Specific criteria for AD patients were: \textit{i)} clinical diagnosis of Alzheimer's disease;\textit{ ii)} Mini-Mental State Examination (MMSE) score greater or equal to $18$.
Magnetic resonance imaging (MRI) acquisitions were obtained using a 3T system (Siemens Trio, 32-channel system, with a 12-channel head coil). The MRI examination included a 3D T1-weighted volumetric magnetization-prepared rapid-gradient echo (MPRAGE) sequence with 1mm isotropic resolution and the following parameters: repetition time (TR)=2300 ms, echo time (TE)=4.18ms, inversion time (TI)=900 ms, matrix=256x256. This sequence provided a high contrast-to-noise ratio and enabled excellent segmentation of high grey/white matter.

The magnetoencephalography (MEG) experimental protocol consisted in a resting-state with eyes-closed (EC). Subjects seated comfortably in a dimly lit electromagnetically and acoustically shielded room and were asked to relax and fix a central point on the screen.
MEG signals were collected using a whole-head MEG system with $102$ magnetometers and $204$ planar gradiometers (Elekta Neuromag TRIUX MEG system) at a sampling rate of $1\,000$ Hz and on-line low-pass filtered at $330$ Hz.
The ground electrode was located on the right shoulder blade. An electrocardiogram (EKG) Ag/AgCl electrodes was placed on the left abdomen for artifacts correction and a vertical electrooculogram (EOG) was simultaneously recorded.
% Horizontal and vertical eye position as well as pupil diameter were monitored using an eye tracker (EyeLink 1000, SR research) and recorded together with MEG and EKG data. % => Useless since eyes are closed.
Four small coils were attached to the participant in order to monitor head position and to provide co-registration with the anatomical MRI. The physical landmarks (the nasion, the left and right pre-auricular points) were digitized using a Polhemus Fastrak digitizer (Polhemus, Colchester, VT).

We recorded three consecutive epochs of approximately $2$ minutes each. All subjects gave written informed consent for participation in the study, which was approved by the local ethics committee of the Pitie-Salpetriere Hospital. Signal space separation was performed using MaxFilter \citep{taulu_spatiotemporal_2006} to remove external noise. We used in-house software to remove  cardiac and ocular blink artifacts from MEG signals by means of principal component analysis.
We visually inspected the preprocessed MEG signals in order to remove epochs that still presented spurious contamination.
At the end of the process, we obtained a coherent dataset consisting of three clean preprocessed epochs for each subject.

\subsection{Source reconstruction, power spectra and brain connectivity} \label{subsec:brain_connectivity}

We reconstructed the MEG activity on the cortical surface by using a source imaging technique \citep{he_brain_1999,baillet_evaluation_2001}.
We used the FreeSurfer 5.3 software (surfer.nmr.mgh.harvard.edu) to perform skull stripping and segment grey/white matter from the 3D T1-weighted images of each single subject \citep{fischl_whole_2002, fischl_sequence-independent_2004}. Cortical surfaces were then modeled with approximately $20000$ equivalent current dipoles (i.e., the vertices of the cortical meshes).
We used the Brainstorm software \citep{tadel_brainstorm:_2011} to solve the linear inverse problem though the wMNE (weighted Minimum Norm Estimate) algorithm with overlapping spheres \citep{lin_assessing_2006}. Both magnetometer and gradiometer, whose position has been registered on the T1 image using the digitized head points, were used to localize the activity over the cortical surface.
The reconstructed time series were then extracted from $148$ regions of interest (ROIs) defined by the Destrieux atlas \citep{destrieux_automatic_2010-1}.
% We chose this Atlas for its number of ROIs that is computationally-wise adapted for this application and close to the number of MEG sensors, because it is provided by the FreeSurfer software, because it is a cortical-only atlas and because the regions have similar size hence a similar number of vertices on the cortical mesh.

We computed the power spectral density (PSD) of the ROI signals by means of the Welch's method; we chose a $2$ seconds sliding Hanning window, with a $25\%$ overlap. The number of FFT points was set to $500$ for a frequency resolution of $0.5 \Hz$.
% NOTE: Finding reference for spectral coherence is not useful.
We estimated functional connectivity by calculating the spectral coherence between each pair of ROI signals \citep{carter_coherence_1987}. 
%For a given frequency $f$, the spectral coherence for the channels pair $(i,j)$ can be computed as follow:
%\begin{equation}
	%Coh_{ij}(f) = \frac{ \abs{S_{ij}(f)} }{ \sqrt{S_{ii}(f)S_{jj}(f)} }
	%\label{eq:coherence}
%\end{equation}
%where $S_{ij}(f)$ is the cross-spectrum of two time series $x_i(t)$ and $x_j(t)$ of ROI $i$ and $j$ respectively. We used the same FFT parameters as for the PSD.
As a result, we obtained for each subject and epoch, a connectivity matrix of size $148 \times 148$ where the $(i,j)$ entry contains the value of the spectral coherence between the signals of the ROI $i$ and $j$ at a frequency $f$.

We then averaged the connectivity matrices within the following characteristic frequency bands \citep{stam_generalized_2002,babiloni_abnormal_2004}: $\delta$ (2-4 Hz); $\theta$ (4-8 Hz); $\alpha=\alpha_{1}$ (8-10.5 Hz) and $\alpha_{2}$ (10.5-13 Hz); $\beta=\beta_{1}$ (13-20 Hz) and $\beta_{2}$ (20-30 Hz); $\gamma$ (30-45 Hz).
We further averaged the resulting connectivity matrices across epochs to obtain our raw individual brain networks whose nodes were the ROIs ($n = 148$) and links, or edges, were the spectral coherence values.

\subsection{Single-layer network analysis} \label{subsec:singlelayer}

In order to cancel the weakest noisy connections, we thresholded the values in the connectivity matrices and retained the same number of links in each brain network at every frequency band, or layer.
We considered six representative connection density thresholds corresponding to an average node degree $k=\{1, 3, 6, 12, 24, 48\}$. These values cover the density range $[0.007, 0.327]$ which contains the typical density values used in complex brain network analysis \citep{bullmore_complex_2009,rubinov_complex_2010,de_vico_fallani_graph_2014}.
The resulting sparse brain networks, or graphs,  were represented by adjacency matrices $A$, where the $a_{ij}$ entry indicates the presence or absence of a link between nodes $i$ and $j$.

\subsubsection{Participation coefficient} \label{subsec:pc}

Given a network partition, the local participation coefficient ($PC_i$) of a node $i$ measures how evenly it is connected to the different clusters, or modules of the network \citep{guimera_cartography_2005}.
Nodes with high participation coefficients are considered central hubs as they allow for the information exchange among different modules.
The global participation coefficient $PC$ of a network at layer $\lambda$  is then given by the average of the $PC_i$ values:
\begin{equation}
	PC^{[\lambda]} 	= \frac{1}{n} \sum_{i=1}^{N} PC_i^{[\lambda]}
	=  \frac{1}{n} \sum_{i=1}^{N} \Bigg[ 1-\sum_{m=1}^{M^{[\lambda]}} \Bigg( \frac{k_{i,m}^{[\lambda]}}{k_i^{[\lambda]}} \Bigg)^2 \Bigg] \text{,}
	\label{eq:pc}
\end{equation}
where $k_{i,m}^{[\lambda]}$ is the number of weighted links from the node $i$ to the nodes of the module $m$ of the layer $\lambda$. By construction, $PC$ ranges from $0$ to $1$.
Here, the partition of the networks into modules was obtained by maximizing the modularity function as defined by \cite{newman_finding_2006}.

\subsubsection{Flattened networks} \label{subsec:flattening}

We also computed the participation coefficients for brain networks obtained by flattening the frequency layers into a single \textit{overlapping}  or \textit{aggregated} network \citep{battiston_structural_2014}.
In an overlapping network, the weight of an edge $o_{ij}$ corresponds to the number of times that the nodes $i$ and $j$ are connected across layers:
\begin{equation}
	o_{ij} = \sum_{\lambda} a_{ij}^{[\lambda]} \text{,}
	\label{eq:overlapping}
\end{equation}

In an aggregated network, the existence of an edge indicates that nodes $i$ and $j$ are connected in at least one layer:
\begin{equation}
	a_{ij} = \left\{
	\begin{array}{ll}
		1 & \text{if } \exists{\lambda} : a_{ij}^{[\lambda]} \ne 0 \\
		0 & \text{otherwise}                                       \\
	\end{array}
	\right. \text{,}
	\label{eq:aggregated}
\end{equation}

Notice that, by construction, flattened networks do not preserve the original connection density of the single layer networks.

\subsection{Multi-layer network analysis} \label{subsec:multiplex_networks}
% TODO: Add reference for multiplex definition
We adopted a multi-layer network approach to integrate the information from brain networks at different frequency bands, while preserving their original structure.
We built for each subject a multiplex network (Fig. \ref{fig:multiplex}a,b) where different layers correspond to different frequency bands and each node in one layer is virtually connected to all its counterparts in all the other layers.

Without loss of generality, if we consider the standard neurophysiological frequency bands, the resulting supra-adjacency matrix $\mathcal{A}$ is given by the following intra-layer of adjacency matrices on the main diagonal:
\begin{equation}
	\mathcal{A} = \{ A^{[\delta]}, A^{[\theta]}, A^{[\alpha]}, A^{[\beta]}, A^{[\gamma]} \}\text{,}
	\label{eq:supraadjacency}
\end{equation}
where $A^{[\lambda]}$ is adjacency matrix of the frequency layer $\lambda$. By construction, the inter-layer adjacency matrices of multiplexes are intrinsically defined as identity matrices.

\subsubsection{Multi-participation coefficient} \label{subsec:mpc}
We considered the multi-layer version of the local participation coefficient $MPC_i$ to measure how evenly a node $i$ is connected to the different layers of the multiplex \citep{battiston_structural_2014}.
This way, nodes with high multi-participation coefficients are considered central hubs as they would allow for a better information exchange among different layers.
The global multi-participation coefficient is then given by the average of the $MPC_i$ values:

\begin{equation}
	MPC = \frac{1}{n} \sum_{i=1}^{N} MPC_i
	= \frac{1}{n} \sum_{i=1}^{N} \frac{M}{M-1} \Bigg[ 1-\sum_{\lambda} \Bigg( NLP_i^{[\lambda]} \Bigg)^2 \Bigg] \text{,}
	\label{eq:mpc}
\end{equation}
where $NLP_i^{[\lambda]} = k_i^{[\lambda]} / o_i$, stands for \textit{node-degree layer proportion}, which measures the percentage of the total number of links (i.e. in all layers) of node $i$ that are in layer $\lambda$.
By construction, if nodes tend to concentrate their connectivity in one layer, the global multi-participation coefficient tends to $0$; on the contrary, if nodes tend to have the same number of connections in every layer, the $MPC$ value tends to  $1$ (Fig. \ref{fig:multiplex}c). Hence, a node with a high $MPC$ has the potential to facilitate communication across layers.
The Matlab code for the computation of the $MPC$ is freely available at \href{https://github.com/devuci/BNT}{https://github.com/devuci/BNT}.

We also used the standard coefficient of variation $CV_i$ to measure the dispersion of the degree of a node $i$ across layers. A global coefficient of variation $CV$ is then obtained by averaging the $CV_i$ values across all the nodes  (\nameref{subsec:TextS}).

\subsection{Statistical analysis}

We first analyzed network features on global topological scales in order to detect statistical differences between AD and HC subjects at the whole system level.
Only for those conditions (e.g., frequency bands) that resulted significantly different on the global scale, we also assessed possible group-differences on the local topological scale of single nodes.
This hierarchical approach allowed us to associate brain network differences on multiple topological scales \citep{de_vico_fallani_interhemispheric_2016}.
For global network features, we used a non-parametric permutation t-test to assess statistical differences between groups, with a significance level of $0.05$. For local network features, we applied a correction for multiple comparisons by computing the rough false discovery rate (FDR) \citep{benjamini_controlling_1995,zar_biostatistical_1999}.
In both cases, surrogate data were generated by randomly exchanging the group labels $10\,000$ times.

%hierarchical analysis
To test the ability of the significant brain network properties to predict the cognitive/memory impairment of AD patients, we used the non-parametric Spearman's correlation coefficient $R$. We set a significance level of $0.05$ for the correlation of global network features, with a FDR correction in the case of multiple comparisons (local features).


\subsection{Classification}

We used a classification approach to evaluate the discriminating power of the local brain network properties which resulted significantly different in the AD and HC group.
Because we did not know in advance which were the most discriminating features, we tested different combinations. In particular, for each local network property, we first ranked the respective ROIs according to the $p$-values returned by the between-group statistical analysis (see previous section).
For each subject $s$, we then tested different feature vectors obtained by concatenating, one-by-one, the values of the network features extracted from the ranked ROIs.
The generic feature vector $c_s$ reads:
\begin{equation}
	c_s =  [g_1, ... ,g_k]
	\label{eq:caractestics}
\end{equation}
where $g_k$ is a generic local network feature and $k$ is a rank that ranges from $1$ (the most significant ROI) to the total number of significant ROIs.
When different network properties were considered (e.g., $PC$ and $MPC$), we concatenated the respective $c_s$ feature vectors allowing for all the possible combinations.
% TODO: Possibly explain better

% Classification algorithm
To quantify the separation between the feature vectors of AD and HC subjects, we used a Mahalanobis distance classifier.
We applied a repeated $5$-fold cross-validation procedure where we randomly split the entire dataset into a training set ($80\%$)  and a testing test ($20\%$). This procedure was eventually iterated $10\,000$ times in order to obtain more accurate classification rates.
To assess the classification performance we computed the sensitivity ($Sens$), specificity ($Spec$) and accuracy ($Acc$), defined respectively as the percentage of AD subjects correctly classified as AD, the percentage of HC subjects classified as HC and the total percentage of subjects (AD and HC) properly classified.
We also computed the receiver operating characteristic (ROC) curve and its area under the curve ($AUC$) \citep{hastie_elements_2009}.
