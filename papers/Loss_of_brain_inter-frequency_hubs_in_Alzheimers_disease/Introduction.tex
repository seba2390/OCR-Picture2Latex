% !TEX root = Guillon2016.tex

% Intro - generic
Recent advances in network science has allowed new insights in the brain organization from a system perspective.
Characterizing brain networks, or connectomes, estimated from neuroimaging data as graphs of connected nodes has not only pointed out important network features of brain functioning - such as smallworldness, modularity, and regional centrality - but it has also led to the development of biomarkers quantifying reorganizational mechanisms of disease \citep{stam_modern_2014}.
% Alzheimer - the clinical context
Among others, Alzheimer's disease (AD), which causes progressive cognitive and functional impairment, has received great attention by the network neuroscience community \citep{stam_modern_2014,tijms_alzheimers_2013,stam_use_2010}.
AD is histopathologically defined by the presence of amyloid-$\beta$ plaques and tau-related neurofibrillary tangles, which cause loss of neurons and synapses in the cerebral cortex and in certain subcortical regions \citep{tijms_alzheimers_2013}. This loss results in gross atrophy of the affected regions, including degeneration in the temporal and parietal lobe, and parts of the frontal cortex and cingulate gyrus \citep{wenk_neuropathologic_2003}.

Structural brain networks, whose connections correspond to inter-regional axonal pathways are therefore directly affected by AD because of connectivity disruption in several areas including cingulate cortices and hippocampus \citep{rose_loss_2000,zhou_abnormal_2008}.
A decreased number of fiber connections eventually lead to a number of network changes on multiple topological scales.
At larger scales, AD brain networks estimated from diffusion tensor imaging (DTI) showed increased characteristic path length as compared to healthy subjects leading to a global loss of network smallworldness \citep{lo_diffusion_2010,tijms_alzheimers_2013}.
Similar topological alterations have been also documented in resting-state brain networks estimated from functional magnetic resonance imaging (fMRI) \citep{sanz-arigita_loss_2010-1}, as well as from magneto/electroencephalographic (M/EEG) signals, the latter ones often reported within the \textit{alpha} frequency range ($8-13$ Hz) which is typically affected in AD \citep{stam_graph_2009,de_haan_functional_2009,miraglia_searching_2017}.
On smaller topological scales, structural brain network studies have demonstrated a loss of connector hubs in temporal and parietal areas that correlates with cognitive decline \citep{bassett_dynamic_2011,tijms_alzheimers_2013,crossley_hubs_2014}.
In addition, higher-order association regions appear to be affected in functional brain networks inferred from fMRI \citep{buckner_cortical_2009,tijms_alzheimers_2013} and MEG signals, the latter showing a characteristic loss of parietal hubs in higher ($>14$ Hz) frequency ranges \citep{de_haan_disrupted_2012,engels_declining_2015}.

% Despite graph analysis of brain networks has advanced our understanding of the organizational mechanisms underlying human cognition and disease, a certain number of issues still remain to be addressed \citep{de_vico_fallani_graph_2014,bullmore_complex_2009}.
Graph analysis of brain networks has advanced our understanding of the organizational mechanisms underlying human cognition and disease, but a certain number of issues still remain to be addressed \citep{de_vico_fallani_graph_2014,bullmore_complex_2009}.
For example,  conventional approaches analyze separately brain networks obtained at different frequency bands, or in some cases, they simply focus on specific frequencies, thus neglecting possible insights of other spectral contents on brain functioning \citep{de_vico_fallani_graph_2014}.
However, several studies have hypothesized and reported signal interaction or modulations between different frequency bands that are supportive of cognitive functions such as memory formation \citep{canolty_functional_2010, jirsa_cross-frequency_2013,brookes_multi-layer_2016-1}.
Moreover, recent evidence shows that neurodegenerative processes in AD do alter functional connectivity in different frequency bands \citep{fraga_characterizing_2013,engels_declining_2015, blinowska_functional_2016}.
How to characterize this multiple information from a network perspective still remains poorly explored.
Here, we proposed a multi-layer network approach to study multi-frequency connectomes as networks of interconnected layers, containing the connectivity maps extracted from different bands.
Multi-layer network theory has been previously used to synthesize MEG connectomes from a whole population \citep{ghanbari_functionally_2014}, characterize temporal changes in dynamic fMRI brain networks \citep{bassett_dynamic_2011}, and integrating structural information from multimodal imaging (fMRI, DTI) \citep{simas_algebraic_2015, battiston_multilayer_2016}.
Its applicability to multi-frequency brain networks has been recently illustrated in fMRI connectomes for which, however, the frequency ranges of interest remains quite limited \citep{de_domenico_mapping_2016}.

We focused on source-reconstructed MEG connectomes, characterized by rich frequency dynamics, that were obtained from a group of AD and control subjects in eyes-closed resting-state condition.
% We focused on functional brain connectivity networks computed from source-reconstructed MEG signals, characterized by rich frequency dynamics, that were obtained from a group of AD and control subjects in eyes-closed resting-state condition.
We hypothesized that the atrophy process in AD would lead to an altered distribution of regional connectivity across different frequency bands and we used the multiplex participation coefficient to quantify this effect both at global and local scale \citep{battiston_structural_2014}.
We evaluated the obtained results, which provide a novel view of the brain reorganization in AD, with respect to standard approaches based on single-layer network analysis and flattening schemes \citep{de_domenico_mathematical_2013}.
Finally, we tested the diagnostic power of the measured brain network features to discriminate AD patients and healthy subjects.
