\section{Conclusion and Future Work}
In many companies, millions of unlabeled documents containing information relevant to many business-related workflows have to be processed to be classified or / and to extract key information. Unfortunately, a large percentage of these documents consists of unstructured formats in the form of images and PDF documents. Examples of these types of documents include: medical bills, attorney letters, contracts, bank statements and personal checks. This process automation it usually refer as "Document intelligence" and relies on the use of models that combine both the image and text information to classify, categorize, and extract the relevant information. However, the labeled data needed by traditional learning approaches maybe too expensive and taxing on business experts and hence not practical in real-world industry settings. Hence, a few-shot learning pipeline is highly desired.

In this work, we proposed a novel method for few-shot document image classification under domain shift for semi-structured business documents, using the canonical correlation block to align extracted text and image feature vectors. We evaluate our work by extensive comparisons with existing  methods on two datasets.
% In this paper we proposed a novel few shot document image classification that uses the canonical correlation block to align extracted text and image feature vectors,  of a document. 
We rigorously benchmarked our method against the state-of-the-art few-shot computer vision models on both an insurance process derived dataset and the miniRVL dataset. We also presented the different ablation studies to show the effectiveness of the proposed method. The results showed our method consistently performed better than existing baselines on few-shot classification tasks. For future work, we would like to further explore more effective document representations including more sophisticated graph representations , or jointly trained layouts \cite{mandivarapu2021efficient} and future directions of implementing the continual learning in document classification \cite{10.3389/frai.2020.00019,blake1,jay2,DJay}.

