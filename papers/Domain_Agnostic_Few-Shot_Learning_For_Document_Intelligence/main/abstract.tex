\begin{abstract}
 Few-shot learning aims to generalize to novel classes with only a few samples with class labels. Research in few-shot learning has borrowed techniques from transfer learning, metric learning, meta-learning, and Bayesian methods. These methods also aim to train models from limited training samples, and while encouraging performance has been achieved, they often fail to generalize to novel domains. Many of the existing meta-learning methods rely on training data for which the base classes are sampled from the same domain as the novel classes used for meta-testing. However, in many applications in the industry, such as document classification, collecting large samples of data for meta-learning is infeasible or impossible. While research in the field of the cross-domain few-shot learning exists, it is mostly limited to computer vision. To our knowledge, no work yet exists that examines the use of few-shot learning for classification of semi-structured documents (scans of paper documents) generated as part of a business workflow (forms, letters, bills, etc.).  Here the domain shift is significant, going from natural images to the semi-structured documents of interest. In this work, we address the problem of few-shot document image classification under domain shift. We evaluate our work by extensive comparisons with existing  methods. Experimental results demonstrate that the proposed method shows consistent improvements on the few-shot classification performance under domain shift.
 
 %[how is this statement of the experimental results different from the previous sentence?] Evaluated on two datasets in cross-domain few-shot learning classification show that the proposed method achieves state of-the-art results and further prove the robustness of the proposed model


%%%% SUBMITTED SO FAR 8/30
%  Few-shot learning aims to classify novel classes with only few labeled images per class. Few shot learning has been benefited from the recent advancements in the methods like transfer learning, Metric learning, Meta-Learning and bayesian methods with the goal of learning from the limited samples. While encouraging performance has been exemplied using these methods, but these methods often fail to generalize to novel or unseen domains due to large disparity before the training and target distribution. As lot of the exisiting methods  relies on training data for meta-learning: base classes sampled from the same domain as the novel classes which will be seen through meta-testing. However, in many applications at enterprise level such as document classification, collecting data consisting of large classes for meta-learning is infeasible or impossible. While works in the field of the cross-domain few-shot scenario exist, these works are limited to natural
% images that still contain a high degree of visual similarity and some works addressing the large domain difference during meta-training and meta-evaluation. No work yet exists that examines few-shot learning across different domains raging from Image domain to NLP domain. In this work, we address the problem of few-shot document image classification under domain shift which is a fundamental NLP task in which a model aims to classify document given only a few training examples per category. We evaluate our work by extensive comparisons with existing  methods on two few-shot classification datasets ;(MiniImageNet and TieredImageNet). Experimental results demonstrate that the proposed onsistent improvements on the few-shot classification performance under domain shift. Evaluated on two datasets in crossdomain few-shot learning classification show that the proposed method achieves state of-the-art results and further prove the robustness of the proposed model
%%%%%%


% Document image classification remains a popular research area because it can be commercialized in many enterprise applications across different industries. Recent advancements in large pre-trained computer vision and language models and graph neural networks has lent document image classification many tools. However using large pre-trained models usually requires substantial computing resources which could defeat the  cost-saving advantages of  automatic document image classification. In the paper we propose an efficient document image classification framework that uses graph convolution neural networks and incorporates textual, visual and layout information of the document.
% %We have rigorously benchmarked our proposed algorithm against several state-of-art vision and language models on both  publicly available dataset and a real-life insurance document classification dataset . 
%  Empirical results on both publicly available and real-world data show that our methods achieve near SOTA performance yet require much less computing resources and time for model training and inference. This results in solutions than offer better cost advantages, especially in scalable deployment for enterprise applications.  
% %The results showed that our algorithm can achieve classification performance quite close to SOTA. We also provide comprehensive comparisons of computing resources, model sizes, train and inference time between our proposed methods and baselines. In addition we delineate the cost per image using our method and other baselines.
\end{abstract}