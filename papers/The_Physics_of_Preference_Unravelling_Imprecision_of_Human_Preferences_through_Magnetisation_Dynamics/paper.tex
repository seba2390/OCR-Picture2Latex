\documentclass[aps,prl,reprint,superscriptaddress]{revtex4-2}
%\documentclass[preprint]{revtex4-1}

\draft % marks overfull lines with a black rule on the right
\usepackage{amssymb,amsmath}
\usepackage{graphicx}% Include figure files
\usepackage{dcolumn}% Align table columns on decimal point
\usepackage{bm}% bold math
\usepackage{color}% bold math
\usepackage{float}
\usepackage{hyperref}
\usepackage[version=4]{mhchem}
\usepackage{relsize}
\usepackage{comment}
\usepackage{media9}
\usepackage{dblfloatfix}    % To enable figures at the bottom of page

\newcommand{\OK}{\color{green}}
\newcommand{\Ivan}{\color{magenta}}
\newcommand{\Rey}{{\mathcal Re}}% Reynolds number

\begin{document}

\title{The Physics of Preference: Unravelling Imprecision of Human Preferences \\through Magnetisation Dynamics}

\author{Ivan S.~Maksymov}
%\email{imaksymov@csu.edu.au}
\affiliation{Artificial Intelligence and Cyber Futures Institute, Charles Sturt University, Bathurst, NSW 2795, Australia\looseness=-1}

\author{Ganna~Pogrebna}
\affiliation{Artificial Intelligence and Cyber Futures Institute, Charles Sturt University, Bathurst, NSW 2795, Australia\looseness=-1}
\affiliation{The Alan Turing Institute, British Library, London, NW1 2DB, United Kingdom}
\affiliation{The University of Sydney Business School, Darlington, NSW 2006, Australia}

%\date{\today}

\begin{abstract}
Paradoxical decision-making behaviours such as preference reversal often arise from imprecise or noisy human preferences. By harnessing the physical principle of magnetisation reversal in ferromagnetic nanostructures driven by electric current, we developed a model that closely reflects human decision-making dynamics. Tested against a spectrum of psychological data, our model adeptly captures the complexities inherent in individual choices. This blend of physics and psychology paves the way for fresh perspectives on understanding human decision-making processes.
\end{abstract}

\maketitle %\maketitle must follow title, authors, abstract and \pacs

%\section{Introduction}
{\it Introduction}---Theoretical and experimental research in economics and psychology posits that human preferences are inherently noisy and imprecise \cite{Lin71} (see \cite{Bha17, loomes2017preference} for a definition of noise and imprecision). For example, such phenomena as \textit{Allais Paradox}, \textit{Ellberg Paradox}, \textit{Endowment Effect} and \textit{Preference Reversal} (PR) demonstrate that human decisions deviate from predictions of traditional rational models \cite{moffatt2009experimental}. These deviations follow certain patterns \cite{loomes2017preference}. While the classical theory suggests that individuals consistently maximise their utility based on stable preferences, experiments present situations where choices are inconsistent or even contradictory \cite{blavatskyy2010models, bardsley2009experimental}.

PR was first observed in the gambling studies involving \cite{Lic71}: (i) a choice task, where people were asked to decide between a relatively safe lottery with a low payoff (Probability- or P-bet) and a relatively risky lottery with a high payoff (Dollar- or \$-bet) as well as (ii) a valuation task, where the same people were asked to indicate lottery prices if they were to sell P-bet and \$-bet. The observed people’s preferences were often contradictory: opting for the P-bet they put a higher price on the \$-bet.

Alongside other behavioural irregularities, PR has reshaped contemporary discourse in both economics and psychology. Traditional models such as Expected Utility Theory (EUT), which held that individual choices were reflective of consistent and stable preferences, have been challenged by empirical findings \cite{Gre79}. The existence of such paradoxes implied that human decision-making was more multifaceted than previously assumed.

Understanding and accurately predicting these anomalies is not merely an academic exercise. As policy-makers strive to devise strategies and interventions that depend on opinion polls, the need for models predicting human behaviour is pressing. Yet, capturing the essence of phenomena like PR within traditional models remains a challenge. For example, transitivity posits that if an individual prefers $A$ to $B$ and $B$ to $C$, then they should prefer $A$ to $C$. However, PR defies the transitivity, suggesting that human preferences might be imprecise.
\begin{figure}
 \includegraphics[width=0.45\textwidth]{Fig1}
 \caption{(a)~Idealised deterministic (the dotted line) and probabilistic (the line with circular markers) PR curves. (b)~Simulated magnetisation reversal behaviour for different values of the static magnetic field $H_0$.\label{Fig1}}
\end{figure}

This instigates debates on the need to (i)~reconsider some axioms of decision theory \cite{butler2018predictably}; (ii)~devise theories relaxing transitivity assumption \cite{loomes1982regret}; (iii)~embed decision theories into models of noise-synthesising stochastic versions of known theories \cite{blavatskyy2010models}; and (iv)~develop models that assume that preferences are inherently imprecise \cite{loomes2017preference}.

The latter approach has demonstrated that individual choices under risk are probabilistic rather than deterministic, i.e.~the same person may make different choices when presented with the same problem multiple times. Experiments show that people's decisions to accept or refuse sure payoff amounts in exchange for a risky lottery varies from consistently selecting bet over cash (when the sure payoff amounts offered against the lottery are not sufficiently attractive) to consistently selecting cash over bet (when the sure payoff amounts increase) with some \textit{imprecision interval} in-between, when individuals make probabilistic choices between cash and the bet \cite{mosteller1951experimental, loomes2014testing, loomes2017preference}. 

While these debates are ongoing, it has been noticed that physical processes can adequately describe many psychological effects and decision-making phenomena. For example, there exist classical physical models of confirmation bias \cite{Gro17}, public opinion formation \cite{Gal05, Cas09, Red19} and backfire effect \cite{Hoh23}. Quantum-mechanical models of decision-making \cite{Atm10, Aer14, Bus12} that originate from the quantum mind hypothesis \cite{Khr06, mindell2012quantum} have also been developed.

However, there remains an imperative to juxtapose the predictions of such models against empirical findings. In this pursuit, our paper not only elucidates the connections between physics and decision theory, but also provides a comprehensive evaluation of a real-life physical model against rich empirical datasets. Through this endeavour, we aim to herald a paradigm shift in understanding and predicting human decision-making.   

{\it Study~1}---In a study of imprecision intervals \cite{mosteller1951experimental, loomes2014testing, loomes2017preference}, an individual is given a number of choices between a lottery $B$ and a series of increasing sure payoff amounts $A_j$,~$j=0,\dots,13$. Since each choice is made multiple times and independently of the previous ones, we model this individual’s preferences as a probability distribution. 

Figure~\ref{Fig1}a shows two theoretically possible distributions. The dotted line denotes the deterministic preferences of an individual who chooses $B$ when $j<6$, never chooses $B$ when $j>6$ and is \textit{exactly indifferent} between $A$ and $B$ when $j=6$. The curve with the circular markers denotes the probabilistic preferences. When the sure thing $ST\leq A_3$, the individual always chooses $B$. When $ST\geq A_{10}$, the individual never chooses $B$. For $A_3<A<A_{10}$ there is an \textit{imprecision interval}.

The probabilistic distribution in Fig.~\ref{Fig1}a is drawn as a sigmoid-like curve since this function robustly fits many intuitions \cite{loomes2017preference} and datasets \cite{mosteller1951experimental, loomes2014testing, loomes2017preference, blavatskyy2010models, butler2018predictably}. Figure~\ref{Fig1_1} shows five examples of experimental curves (median patterns) plotted using the data obtained from the laboratory decision-making experiments with human subjects.

All experiments were conducted in the University of Warwick using subject pools of undergraduate, postgraduate and executive students over the age of 18. In Fig.~\ref{Fig1_1}a, 81~subjects were making a series of binary choices between a lottery which gave \pounds40 with probability 80\% or 0 with probability 20\% and sure payoff amounts between \pounds16 and \pounds36 with a step of \pounds2. In Fig.~\ref{Fig1_1}b, 101\, subjects were offered a lottery which yielded \pounds40 with probability 30\% or 0 with probability 70\% versus sure amounts of money between \pounds4 and \pounds12 with a step of \pounds1. In Fig.~\ref{Fig1_1}c, 101 experimental participants make choices between a lottery providing \pounds15 with probability 70\% and 0 otherwise against sure amounts between \pounds4 and \pounds12 with a step of \pounds1. In Fig.~\ref{Fig1_1}d, 184 subjects face a choice between a chance of winning \pounds12 with probability 80\% and sure amounts from \pounds3 to \pounds11 with a step of \pounds1. In Fig.~\ref{Fig1_1}e, 184 people decide between \pounds15 with probability 25\% and sure monetary amounts between \pounds3 to \pounds11 with a step of \pounds1. In all lotteries, each person made each of the binary choices 4\,times. We can see that all five examples demonstrate robust evidence in favour of the existence of an \textit{imprecision interval}, where people make probabilistic choices between a lottery and sure monetary amounts.
\begin{figure}
 \includegraphics[width=0.49\textwidth]{Fig1_1}
 \caption{Experimental PR data of the five lotteries. The straight dotted lines are the prediction of EUT under the assumption that the participants are risk-neutral.\label{Fig1_1}}
\end{figure}

To model the experimental decision-making behaviour, we employ magnetisation dynamics in ferromagnetic metal (FM) structures. In general, an electric current is unpolarised since it involves electrons with a random polarisation of spins. However, a current that passes through a thin FM film with a fixed magnetisation direction can become spin-polarised since spins become oriented predominantly in the same direction \cite{Ral08}. This physical effect is exploited in a spin transfer torque nano-oscillator (STNO), where a layer with a fixed direction of magnetisation ${\bf M}$ is separated from a thinner FM layer by a non-magnetic metal layer \cite{Rip04}. When a spin-polarised current is directed from the ``fixed'' magnetisation layer into the ``free'' magnetisation layer, the static equilibrium orientation of magnetisation in the ``free'' layer becomes destabilised. Depending on the strength of the electric current, the destabilisation of magnetisation can lead to either stable precession of magnetisation of the ``free'' layer about the direction of the effective magnetic field or to a reversal of the magnetisation direction (Fig.~\ref{Fig2}).
\begin{figure}
 \includegraphics[width=0.4\textwidth]{Fig2}
 \caption{(a)~Sketch of magnetisation vector $M$ precession. Simulated results:~(b)~${\bf M}$ spirals back toward the direction of ${\bf H}_{eff}$ due to the damping. (c)~The spin transfer torque compensates the damping, resulting in a stable precession of ${\bf M}$ around ${\bf H}_{eff}$. (d)~Reversal of the direction of ${\bf M}$. The applied field is $H_0=8$\,kOe. The sphere is used for visualisation only.\label{Fig2}}
\end{figure}

Using the numerical method described below, in Fig.~\ref{Fig1}b we plot the dynamics of the $M_z$ component of the magnetisation vector ${\bf M}$ in the ``free'' layer as a function of the electric current strength for different value of the applied magnetic field $H_0$. The trajectories of ${\bf M}$ for $H_0=8$\,kOe computed at the simulated time of 1\,ns are shown in Fig.~\ref{Fig2}b--c. Figure~\ref{Fig1}b demonstrates that the magnetisation dynamics correctly approximates the sigmoid probabilistic PR curve in Fig.~\ref{Fig1}a but the strength of the driving electric current plays the role of parameter $A$. The experimental curves in Fig.~\ref{Fig1_1} can be regarded as the particular cases of the sigmoid curve and they can be approximated using magnetisation dynamics at different values of $H_0$, which is demonstrated below.

{\it Study~2}---We validate our model using a dataset obtained from a ``Deal or No Deal'' video game based on a popular TV show, where contestants choose boxes with concealed cash amounts and negotiate with a banker to accept an offer or keep opening boxes. The aim is to secure the highest prize, avoiding smaller sums. Once a box is opened, its prize is shown and removed from potential winnings. Essentially, players choose between risky lotteries and certain cash offers. They also may exchange their box for any unopened one in the game. The decision to swap the box or to stick to the original one enables us to study whether the participants behave according to EUT or to Cumulative Prospect Theory (CPT). A previous study \cite{blavatskyy2010endowment} that used TV show data from three countries demonstrated that: (i)~CPT players are more likely to stick to their original choice of the box due to the embedded assumption of loss aversion and (ii)~EUT players are indifferent between swapping and sticking.

We recruited 78 study participants (members of the general public over 18 years of age) who made a total of 1,698 game decisions, including 486 swap or stick decisions. The fact that 63\% of decisions were stick decisions and 37\% were swap decisions demonstrated a higher consistency with EUT than with CPT. However, Cumulative Distribution Function (CDF) of swap or stick decisions revealed an intriguing result: while EUT predicts an approximately equal split between swapping or sticking decisions (the straight dash-dotted lines in Fig.~\ref{Fig3}), the distribution of swap or stick decisions (the dotted curve in Fig.~\ref{Fig3}) cannot be described by EUT.

Since the magnetisation dynamics depends on the static field $H_0$ (Fig.~\ref{Fig1}b), in Fig.~\ref{Fig3} we plot a magnetisation reversal that is recast as $(M_z/M_s+1)/2$ for the sake of comparison with the experimental data. The resulting curve satisfactorily reproduces the experimental decision-making behaviour demonstrated by the players of the game. The values of $H_0$ corresponding to the data points of this curve are, from left to right: 15, 6, 8,75, 9,6, 9,95, 10, 10, 10.75, 11, 11.52, 11.99, 12.1 and 12.5\,kOe.
\begin{figure}
 \includegraphics[width=0.45\textwidth]{Fig3}
 \caption{Top $x$ and right $y$ axes:~Cumulative probability distribution function for sticking decisions in the video game. The dash-dotted straight lines denote the EUT propensity to stick. The dashed curve corresponds to the experimental data. Bottom $x$ and left $y$ axes:~The circular markers denote the simulated magnetisation reversal. The thin solid line is the guide to the eye only.\label{Fig3}}
\end{figure}

{\it Discussion}---The functions of a human brain underpinning the phenomena like PR can be comprehended in detail through neuroscience methods \cite{Sha21}. Nonetheless, understanding human behaviour is pivotal not just for neuroscience but also for fields of finance, politics, religion and sociology. Thus, there is a demand for phenomenological models that can forecast decisions at both individual and societal levels quickly and efficiently.

These models should not necessarily be based on neurobiological insights, mirroring the human cognition (drawing parallels with biological brain processes might even be unhelpful \cite{Tad16}). The primary criteria for evaluating a model of decision-making should be the ability of the model to accurately explain and predict imprecise and noisy decisions. The model we introduce in this paper follows this philosophy, aligning with the ongoing work in decision theory where models are not necessarily capturing human brain functions \cite{wakker2010prospect}.

STNO has been a foundational component in neuromorphic hardware systems that emulate the functions of the human brain \cite{Fur18, Rio19}. The operational principles of STNO obey the laws of quantum mechanics \cite{Bra22}, even though our model's Eq.~(\ref{eq:eq1}) is classical. Quantum-mechanical models have demonstrated superior capacity in depicting decision-making compared with classical models \cite{Bus12}. Therefore, leveraging a realistic STNO structure in our model hints at the potential of STNOs to act as a neuromorphic system for simulating human decision-making processes. The so-designed system can be integrated with a reservoir computing algorithm \cite{Mak23_review}, resulting in a machine learning technique that has an enhanced learning efficacy compared with classical systems \cite{Dud23} and also can simulate the human decision-making. 

Other quantum-mechanical systems can replicate decision-making \cite{Nar14}. However, the magnetisation reversal in STNO naturally extends important models of opinion shifts in social networks based on spin direction reversal \cite{Red19}. Even though those models adopted the spin concept from physics, the processes causing the spin changes were largely synthetic. Thus, integrating a genuine model of magnetisation reversal into the opinion shift model \cite{Red19} will enhance the efficacy of the latter, enabling it to analyse diverse decision-making scenarios.

{\it Conclusions}---We demonstrated that the psychological phenomenon of PR can be modelled using the physical phenomenon of magnetisation reversal. Using a spin transfer torque as the mechanism of magnetisation reversal, we revealed the ability of the magnetisation reversal model to capture complex individual's decisions made in psychological experiments. We have also shown that our model extends beyond PR and encompasses a wide range of decision-making regularities, reaffirming the observation that human preferences are imprecise.

While our work bolsters the exploration of physics as a lens to comprehend human psychology, its tangible implications are profound: the model holds promise for designing neuromorphic hardware systems rooted in nanomagnetic devices. Such interdisciplinary convergence has the potential to reshape the way we interface with both the human mind and advanced computational systems.

The implication of magnetisation reversal as a model for human decision-making not only accentuates the universality of certain principles across ostensibly disparate domains but also underscores the rich tapestry of relationships waiting to be unearthed between the physical and cognitive realms. For example, the integration of these findings can stimulate a paradigm shift in how we design and conceptualise next-generation cognitive devices. Moreover, the ability of physical effects to emulate cognitive processes suggests a future where the boundaries between cognition and artificial intelligence blur, leading to new possibilities for collaborative intelligence.

{\it Numerical method}---We model the dynamics of magnetisation using Landau-Lifshitz-Gilbert equation \cite{Sla09}:
\begin{equation}
  \label{eq:eq1}
  \partial{\bf M}/\partial t = \gamma\left[{\bf H}_{eff}\times\bf M\right] + {\bf T}_G + {\bf T}_{SB}\,, 
\end{equation}
where $\gamma$ is the gyromagnetic ratio. The first term of the right-hand side of Eq.~(\ref{eq:eq1}) governs the precession of $\bf M$ (Fig.~\ref{Fig1}a) about the direction of the effective magnetic field $\textbf{H}_{eff}=H_{0}\textbf{e}_{\rm{z}} + \textbf{H} + \textbf{H}_{ex}$, where $H_{0}$ is the external static magnetic field orientated along the \textit{z}-axis, $\textbf{H}$ is the dynamic field due to currents and magnetic sources such as demagnetising field and eddy current fields and $\textbf{H}_{ex}$ is the exchange field \cite{Mak13}. The dissipative torque is \cite{Sla09}
\begin{equation}
  \label{eq:eq2}
  {\bf T}_G = \alpha_G M_s^{-1}\left[{\bf M}\times{\partial\bf M}/{\partial t}\right]\,, 
\end{equation}
where $M_s$ is the saturation magnetisation of the ``free'' magnetisation layer and $\alpha_G$ is Gilbert damping parameter \cite{Mak13, Sla09}. The Slonczewski-Berger torque is 
\begin{equation}
  \label{eq:eq5}
  {\bf T}_{SB} = \sigma_0IM_s^{-1}\left[{\bf M}\times\left[{\bf M}\times\hat{e}_p\right]\right]\,, 
\end{equation}
where $I$ is the strength and $\hat{e}_p$ is the direction of the spin polarisation of the current. The parameter $\sigma_0$ incorporates fundamental physical constants and the thickness of the ``free'' magnetisation layer \cite{Sla09}. Equation~(\ref{eq:eq1}) is solved consistently with Maxwell's equations using a finite-difference time-domain (FDTD) method and material parameters presented in \cite{Mak13}. We model a point-contact STNO structure experimentally  studied in \cite{Rip04}. Since we employ a one-dimensional FDTD scheme, the strength of the current $I$ required to produce a magnetisation reversal in our model is lower than that in \cite{Rip04}. 

%\acknowledgments{ISM thanks ...}

\bibliography{refs}

\end{document}

%

% ****** End of file aiptemplate.tex ******
