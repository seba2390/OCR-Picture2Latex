
\section{Experimental Results}
\label{sec:expt-rst}

\begin{table}[t]
\centering
\caption{AUC with different $\epsilon$}
\label{tab:AUC}
\scalebox{0.8}{
\begin{tabular}{ccccc}
\toprule
\multirow{2}{*}{Method} & \multicolumn{4}{c}{AUC with different $\epsilon$ (\%)} \\
        & 20      & 15      & 10      & 5       \\ \hline
Vocoder & \textbf{99.94}    & \textbf{99.62}   & \textbf{99.12}   & \textbf{96.52}   \\
GL-lin  & 97.89   & 97.39   & 95.87   & 89.86   \\
GL-mel  & 99.01   & 97.64   & 95.41   & 87.52   \\ \bottomrule
\end{tabular}
\vspace{-5pt}
}
\end{table}




% \subsubsection{Defense performance}
% \label{subsec:Defence performance}
The defense performance is shown in Table~\ref{tab:EER}.
% Vocoder refers to Parallel WaveGAN.
% GL-lin and GL-mel refer to using Griffin-Lim to re-synthesis waveform from linear spectrograms and mel-spectrograms, respectively.
The EER decrease after audio re-synthesis illustrates that all the three methods can slightly alleviate the adversarial noise.
In contrast to \cite{joshi2021adversarial}, which effectively purifies the adversarial noise for speaker identification by vocoders, directly applying vocoders for defending speaker verification does not work.
Yet the re-synthesis process will not affect the genuine EER too much. 
Thus, we adopt the difference of the ASV scores between the original and re-synthesized audio as a good indicator to distinguish among genuine and adversarial samples. 

As mentioned in Sec~\ref{subsec:vocoder based detection}, we use the discrepancy $|s-s'|$ to distinguish between genuine and adversarial samples.
The distributions of the discrepancy are in Fig.~\ref{fig:Hist-ep5}.
A threshold $\tau_{det}$ can be determined using genuine samples by Eq.~\ref{eq:det-threshold} to separate genuine and adversarial samples.
The detection framework based on vocoder in Fig.~\ref{fig:method} is effective, where the discrepancy $|s-s'|$ for genuine samples is small, yet $|s-s'|$ for adversarial samples is large.
Notice that the distribution overlap between the GL-mel genuine samples and adversarial samples is more considerable than that of Vocoder, as shown in Fig.~\ref{fig:Hist-ep5}.
As Vocoder is trained from genuine data, it attains the capacity of making $|s-s'|$ for genuine samples small enough.
Yet Griffin-Lim can only enlarge $|s-s'|$ for adversarial samples.

\begin{figure}[t]
\centering
\includegraphics[width=0.7\linewidth]{figures/Hist_ep10.pdf}
%\caption{Histogram with $\epsilon=10$}
\caption{$|s-s'|$ distribution with $\epsilon=10$}
\label{fig:Hist-ep5}
\end{figure}

\begin{figure}[h]
    \centering
    \begin{minipage}[h]{0.23\textwidth}
        \centering
        \includegraphics[width=1.6in]{figures/roc_curve_epsilon_5.pdf}
        {\par \footnotesize (a)\par}
    \end{minipage}
    \hfill
    \begin{minipage}[h]{0.23\textwidth}
        \centering
        \includegraphics[width=1.6in]{figures/roc_curve_epsilon_10.pdf}
        {\par \footnotesize (b)\par}
    \end{minipage}
    \hfill
    \begin{minipage}[h]{0.23\textwidth}
        \centering
        \includegraphics[width=1.6in]{figures/roc_curve_epsilon_15.pdf}
        {\par \footnotesize (c)\par}
    \end{minipage}
    \hfill
    \begin{minipage}[h]{0.23\textwidth}
        \centering
        \includegraphics[width=1.6in]{figures/roc_curve_epsilon_20.pdf}
        {\par \footnotesize (d)\par}
    \end{minipage}
    \vspace{-5pt}
    \caption{ROC curve under different epsilon ($\epsilon$) }
    \label{fig:roc-curve-epsilon=5}
    \vspace{5pt}
\end{figure}


% \subsubsection{AUC evaluation}
% \label{subsec:AUC evaluation}
Fig.~\ref{fig:roc-curve-epsilon=5} shows the receiver operating characteristic (ROC) curves of different methods.
The curves show the true positive rates for adversarial samples and false positive rates for genuine samples using different thresholds for detection.
The larger the area under the curve (AUC) is, the better the detection performance.
Both GL-lin and GL-mel achieve effective detection performance, and Vocoder performs better than them.
Table~\ref{tab:AUC} shows the AUC of different methods. 
Table~\ref{tab:AUC} indicates that the proposed method is powerful for adversarial samples detection as all AUCs are approaching or greater than 90\%.
Also, Vocoder outperforms GL-lin and GL-mel in all the settings.

% \vspace{-10pt}
% \subsection{Detection performance}
\begin{table}[t!]
\centering
\caption{Detection rate with different $\epsilon$}
\label{tab:detection_rate}
\scalebox{0.7}{
\begin{tabular}{cccccc}
\toprule
\multirow{2}{*}{$FPR_{given}$} & \multirow{2}{*}{Method} & \multicolumn{4}{c}{Detection rate with different $\epsilon$ (\%)} \\
                        &           & 20        & 15        & 10        & 5         \\ \hline
                        & Vocoder   & \textbf{99.76}    & \textbf{98.82}    & \textbf{97.30}    & \textbf{89.33} \\
\multirow{3}{*}{0.05}   & Vocoder-L    & 99.38     & 97.23     & 94.07     & 81.21     \\
                        & GL-lin    & 89.12     & 88.30     & 84.64     & 71.29     \\
                        & GL-mel    & 95.39     & 91.33     & 85.37     & 68.07     \\ 
                        & Gaussian    & 34.54     & 51.29     & 61.56     & 68.57     \\ \hline
\multirow{3}{*}{0.01}   & Vocoder   & \textbf{98.92}    & \textbf{97.56}    & \textbf{94.76}    & \textbf{81.60} \\
                        & Vocoder-L    & 97.96     & 94.37     & 88.77     & 70.15     \\
                        & GL-lin    & 73.62     & 73.63     & 70.62     & 56.37     \\
                        & GL-mel    & 87.98     & 82.27     & 75.04     & 56.07     
                        \\ \hline
\multirow{3}{*}{0.005}  & Vocoder   & \textbf{98.30}    & \textbf{96.78}    & \textbf{93.25}    & \textbf{78.21} \\
                        & Vocoder-L    & 96.78     & 92.58     & 85.81     & 64.65     \\
                        & GL-lin    & 64.76     & 64.97     & 62.85     & 49.32     \\
                        & GL-mel    & 83.94     & 77.71     & 70.47     & 51.42     
                        \\ \hline
\multirow{3}{*}{0.001}  & Vocoder   & \textbf{96.04}    & \textbf{93.89}    & \textbf{88.60}    & \textbf{68.58} \\
                        & Vocoder-L    & 93.36     & 87.34    & 78.24     & 53.18     \\
                        & GL-lin    & 45.10     & 45.27     & 44.72     & 34.28     \\
                        & GL-mel    & 72.53     & 65.98     & 59.66     & 40.98     
                        \\ \bottomrule
\end{tabular}
}
\vspace{-5pt}
\end{table}

Table~\ref{tab:detection_rate} shows the detection results on adversarial samples with different $\epsilon$.
$FPR_{given}$ column lists different false acceptable rates.
The threshold $\tau_{det}$ was determined according to $FPR_{given}$ as shown in Eq.~\ref{eq:det-threshold}.
Gaussian denotes that we use Gaussian filter to replace feature extraction and vocoder (yellow block and gray block in Fig.~\ref{fig:method}).
Gaussian filter \cite{wu2020defense,wu2021adversarialasv} is usually adopted as an attack-agnostic method to counter adversarial samples, so we also set it as our baseline.
The observations and analysis are concluded as follows:
(1) We find that using Vocoder performs the best among all methods.
In most cases, more than 90\% of the adversarial samples could be detected. While with a large $\epsilon$ or $FPR_{given}$, all the detection rates even exceeded 95\%.
Even with a small $\epsilon$ of 5, the detection rates can still approach or exceed 80\%.
The results indicate that the proposed method can effectively detect adversarial samples.
(2) Gaussian based detection performs the worst, and even with $FPR_{given}=0.05$, the detection rates are still lower than vocoder based detection with $FPR_{given}=0.001$.
As it is not a comparable baseline, we do not show its results in other settings due to space limitation.
(3) For Griffin-Lim based methods, we find that they might be good approaches for detection with a large $\epsilon$ or $FPR_{given}$.
However, in stricter cases (smaller $\epsilon$ or $FPR_{given}$), the detection rates of GL-lin and GL-mel decrease drastically.
We argue that Griffin-Lim is a pseudo, nearly lossless transform, so we can, to some extent, adopt it to replace the vocoders in the adversarial detection framework in Fig.~\ref{fig:method}. 
While Griffin-Lim is not a data-driven method and can not model the genuine data manifold well, it results in higher $|s'-s|$ for genuine samples as shown in Fig.~\ref{fig:Hist-ep5}, and thus the detection performance is not comparable to Parallel WaveGAN.
(4) As shown in Table~\ref{tab:detection_rate}, the detection rate for the Vocoder-L is very close to Vocoder, which indicates the vocoder adopted for the proposed detection method is kind of dataset independent.

Also we try to use the vocoder to detect the Gaussian noise. 
Under the $FPR_{given}$ of 1\%, the detection rate for Gaussian noise is 0.97\%.
Due to space limitation, we won't show the details here.