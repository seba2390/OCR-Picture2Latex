% Template for ICASSP-2020 paper; to be used with:
%          spconf.sty  - ICASSP/ICIP LaTeX style file, and
%          IEEEbib.bst - IEEE bibliography style file.
% --------------------------------------------------------------------------
\pdfoutput=1
\documentclass{article}
\usepackage{cite}
\usepackage{spconf,amsmath,graphicx,amsfonts}
\usepackage{algorithm}
\usepackage{algorithmic}
\usepackage{hyperref}
\usepackage[numbers,sort&compress]{natbib}
\usepackage{multirow}
\usepackage{comment}
\usepackage{booktabs}
\usepackage{subfigure}
\usepackage{arabtex}


% \setlength{\textfloatsep}{2pt}
% Example definitions.
% --------------------
\def\x{{\mathbf x}}
\def\L{{\cal L}}

% Title.
% ------
\title{Adversarial sample detection for speaker verification \\ by neural vocoders}
% Defense against adversarial attacks with tera-based auto-encoder for automatic speaker verification

%
% Single address.
% ---------------
% \name{Author(s) Name(s)\thanks{Thanks to XYZ agency for funding.}}
% \address{Author Affiliation(s)}  

\makeatletter
\def\name#1{\gdef\@name{#1\\}}
\makeatother
\name{\em{Haibin Wu$^{1}$,
    Po-chun Hsu$^{1}$,
    Ji Gao$^{2}$,
    Shanshan Zhang$^{2}$,
    Shen Huang$^{2}$,
    Jian Kang$^{2}$,} \\
    \em{Zhiyong Wu$^{3}$,
    Helen Meng$^4$,
    Hung-yi Lee$^{1}$}
% \thanks{This work was done when Haibin Wu was an intern at Tencent Research, Beijing.}
}

\address{
  $^1$ Graduate Institute of Communication Engineering, National Taiwan University \\
  $^4$ Centre for Perceptual and Interactive Intelligence, The Chinese University of Hong Kong \\
  $^3$ Shenzhen International Graduate School, Tsinghua University \\
  $^2$ Tencent Research, Beijing, China 
}
 
% \email{\{f07921092, hungyilee\}@ntu.edu.tw, \{xuli,zywu,hmmeng\}@se.cuhk.edu.hk}

\begin{document}
\ninept

\maketitle
%
% \vspace{-8pt}
\begin{abstract}
Automatic speaker verification (ASV), one of the most important technology for biometric identification, has been widely adopted in security-critical applications.
However, ASV is seriously vulnerable to recently emerged adversarial attacks, yet effective countermeasures against them are limited.
In this paper, we adopt neural vocoders to spot adversarial samples for ASV.
We use the neural vocoder to re-synthesize audio and find that the difference between the ASV scores for the original and re-synthesized audio is a good indicator for discrimination between genuine and adversarial samples.
This effort is, to the best of our knowledge, among the first to pursue such a technical direction for detecting time-domain adversarial samples for ASV, and hence there is a lack of established baselines for comparison.
Consequently, we implement the Griffin-Lim algorithm as the detection baseline. 
The proposed approach achieves effective detection performance that outperforms the baselines in all the settings. 
We also show that the neural vocoder adopted in the detection framework is dataset-independent.
Our codes will be made open-source for future works to do fair comparison \footnote[1]{https://github.com/HaibinWu666/spot-adv-by-vocoder. This work was done while Haibin Wu was an intern at Tencent Research, Beijing.}.
\end{abstract}
%
\begin{keywords}
adversarial attack, speaker verification, vocoder
\end{keywords}

% \vspace{-8pt}

Neural networks are powerful models that excel at a wide range of tasks.
However, they are notoriously difficult to interpret and extracting explanations 
    for their predictions is an open research problem. Linear models, in contrast, are generally considered interpretable, because
    the \emph{contribution} 
    (`the weighted input') of every dimension to the output is explicitly given.
Interestingly, many modern neural networks implicitly model the output as a linear transformation of the input;
    a ReLU-based~\cite{nair2010rectified} neural network, e.g.,
    is piece-wise linear and the output thus a linear transformation of the input, cf.~\cite{montufar2014number}.
    However, due to the highly non-linear manner in which these linear transformations are `chosen', the corresponding contributions per input dimension do not seem to represent the learnt model parameters well, cf.~\cite{adebayo2018sanity}, and a lot of research is being conducted to find better explanations for the decisions of such neural networks, cf.~\cite{simonyan2013deep,springenberg2014striving,zhou2016CAM,selvaraju2017grad,shrikumar2017deeplift,sundararajan2017axiomatic,srinivas2019full,bach2015pixel}.
    
In this work, we introduce a novel network architecture, the \textbf{Convolutional Dynamic Alignment Networks (CoDA-Nets)}, {for which the model-inherent contribution maps are faithful projections of the internal computations and thus good `explanations' of the model prediction.} 
There are two main components to the interpretability of the CoDA-Nets. 
    First, the CoDA-Nets are \textbf{dynamic linear}, i.e., they compute their outputs through a series of input-dependent linear transforms, which are based on our novel \mbox{\textbf{Dynamic Alignment Units (DAUs)}}. 
        As in linear models, the output can thus be decomposed into individual input contributions, see Fig.~\ref{fig:teaser}.
    Second, the DAUs are structurally biased to compute weight vectors that \textbf{align with \mbox{relevant} patterns} in their inputs. 
In combination, the CoDA-Nets thus inherently  
produce contribution maps that are `optimised for interpretability': 
since each linear transformation matrix and thus their combination is optimised to align with discriminative features, the contribution maps reflect the most discriminative features \emph{as used by the model}.

With this work, we present a new direction for building inherently more interpretable neural network architectures with high modelling capacity.
In detail, we would like to highlight the following contributions:
\begin{enumerate}[wide, label={\textbf{(\arabic*)}}, itemsep=-.5em, topsep=0em, labelwidth=0em, labelindent=0pt]
    \item We introduce the Dynamic Alignment Units (DAUs), which 
    improve the interpretability of neural networks and have two key properties:
    they are 
    \emph{dynamic linear} 
    and align their weights with discriminative input patterns.
    \item Further, we show that networks of DAUs \emph{inherit} these two properties. In particular, we introduce Convolutional Dynamic Alignment Networks (CoDA-Nets), which are built out of multiple layers of DAUs. As a result, the \emph{model-inherent contribution maps} of CoDA-Nets highlight discriminative patterns in the input.
    \item We further show that the alignment of the DAUs can be promoted 
    by applying a `temperature scaling' to the final output of the CoDA-Nets. 
    \item We show that the resulting contribution maps 
    perform well under commonly employed \emph{quantitative} criteria for attribution methods. Moreover, under \emph{qualitative} inspection, we note that they exhibit a high degree of detail.
    \item Beyond interpretability, 
    CoDA-Nets are performant classifiers and yield competitive classification accuracies on the CIFAR-10 and TinyImagenet datasets.
\end{enumerate}
% \vspace{-8pt}
\section{Background}
\subsection{Automatic speaker verification}
The objective of ASV is to authenticate the claimed identity of a speaker by a piece of his/her speech and some enrolled speaker records.
The procedure of ASV can be divided into feature engineering, speaker embedding extraction, and similarity scoring. 
Feature engineering aims at transforming a piece of utterance in waveform representation, into acoustic features, such as Mel-frequency cepstral coefficients (MFCCs), filter-banks, and spectrograms. 
The speaker embedding extraction procedure of recently ASV models \cite{dehak2010front,kenny2012small,snyder2018x} usually extracts utterance-level speaker embedding from acoustic features.
Then similarity scoring will measure the similarity between the testing speaker embedding and the enrolled speaker embedding.
% \cite{dehak2010cosine, kenny2013plda}
The higher the score, the more likely that the enrolment utterance and the testing utterance belong to the same speaker, and vice versa. 
Let us denote the testing utterance and the enroll utterance as $x_{t}$ and $x_{e}$ respectively.
For simplicity, we combine the above three procedures and view ASV as an end-to-end function $f$: 
\begin{align}
    &s = f(x_{t}, x_{e}), 
\end{align}
where $s$ is the similarity score between $x_{t}$ and $x_{e}$.


\subsection{Adversarial attack}
Attackers deliberately incorporate a tiny perturbation, which is indistinguishable from human perception, and combine it with the original sample to generate the new sample, which will manipulate the model give wrong prediction.
The new sample and the tiny perturbation are denoted as the adversarial sample and adversarial noise, respectively.
Suppose that the attackers in the wild have access to the internals of the ASV system, including structures, parameters and gradients, and have the access to the testing utterance $x_{t}$.
They aim at crafting such an adversarial utterance by finding an adversarial perturbation.
Different searching strategies for elaborating adversarial noise result in different attack algorithms.
In this work, we adopt a powerful attack method, the basic iterative method (BIM) \cite{kurakin2016adversarial}.
During BIM attack, attackers will start from $x_{t}^{0}=x_{t}$, then iteratively update it to find the adversarial sample:
\begin{equation}
\begin{aligned}
    x_{t}^{k+1}=clip\left(x_{t}^{k} + \alpha \cdot (-1)^{is\_tgt} \cdot sign\left(\nabla_{x_{t}^{k}}f(x_{t}^{k}, x_{e}) \right)\right), 
    \\ for \, k=0,1, \ldots, K-1,
\end{aligned}
\end{equation}
where $clip(.)$ is the clipping function which make sure that $||x_{t}^{k+1} - x_{t}||_{\infty}\leq \epsilon$,
$\epsilon$, denotes the attack budget or intensity predefined by the attackers,
$\epsilon \geq 0 \in \mathbb{R}$,
$\alpha$ is the step size,
$is\_tgt=1$ and $is\_tgt=0$ for the target trial and the non-target trial respectively,
$K$ is the number of iterations and we define $K = \lceil \epsilon / \alpha \rceil$, where $\lceil.\rceil$ denotes the ceiling function.
In target trials, the testing and enrolment utterances are pronounced by the same speaker.  In non-target trials, they belong to different speakers.
% In the target trial, the testing and enroll utterance are pronounced by the same speaker, yet they belong to different speakers in the non-target trial.
Take the non-target trial as an example -- after the BIM attack, the similarity score between the testing and enrolment utterances will be high, which will mislead the ASV system to falsely accept the imposter.
We recommend that our readers listen to the demo of the deliberately crafted adversarial samples \footnote[2]{https://haibinwu666.github.io/adv-audio-demo/index.html}, which tend to be indistinguishable from their genuine counterparts.

\subsection{Vocoder}
Due to the lack of phase information, speech waveforms cannot be restored directly from acoustic features, such as linear spectrograms and mel-spectrograms.
The traditional vocoder, Griffin-Lim, \cite{griffin1984signal} is usually used to reconstruct phase information.
However, it inevitably introduces distortion during reconstruction, resulting in reduced speech quality.
We argue that the introduced distortion may also degrade the effect of the attack.
Another approach, the neural vocoder, takes acoustic features as conditions and uses a neural network to generate speech signals.
Since a neural vocoder is trained to maximize the likelihood of real speech in training data, we expect that when given distorted or attacked acoustic features, the neural vocoder can generate their genuine counterparts. 
In contrast to Griffin-Lim, a neural vocoder is a data-driven method, which can model the manifolds of genuine data, and thus generates waveform with lowered distortion. %without too much distortion given the real acoustic feature.

% Common neural vocoders, such as WaveNet \cite{oord2016wavenet} and WaveRNN \cite{kalchbrenner2018efficient}, can restore high-quality speech but with slow inference speed due to the autoregressive architecture.
Neural vocoders can restore high-quality speech but with slow inference speed due to the autoregressive architecture.
Parallel WaveGAN \cite{yamamoto2020parallel} adopted a model based on dilated CNN, which can generate audio samples in parallel.
They jointly trained the model using the adversarial loss in GAN and the proposed loss on the frequency domain.
Parallel synthesis improves the efficiency of speech generation, while the GAN architecture can make the Parallel WaveGAN effectively model the distribution of real speech.
Thus in this work, we adopt Parallel WaveGAN for spotting adversarial samples.



\section{Neural vocoder is all you need}
\label{sec:method}
\begin{figure}[ht]
  \centering
  \centerline{\includegraphics[width=\linewidth]{figures/defense.png}}
  \vspace{-5pt}
  \caption{Proposed detection framework. $s$ and $s'$ are the ASV scores for $x$ and $x'$. $|s-s'|$ is the absolute value between $s$ and $s'$.}
  \label{fig:method}
  \vspace{-5pt}
\end{figure}

\subsection{The detection procedure}
\label{subsec:vocoder based detection}
We first detail the detection procedure, followed by the reason why it works.
The vocoder\footnote[3]{Unless specified otherwise, the use of “vocoder” refers to the “neural vocoder” in the following sections.}-based detection framework is shown in Fig.~\ref{fig:method}.
For brevity, we omit the enrollment utterance $x_{e}$.
The subscript of $x_{t}$ is also omitted, and we use $x$ to denote the testing utterance.
We use $x'$ to denote the testing utterance after feature extraction and vocoder preprocessing (yellow block and gray block in Fig.~\ref{fig:method}).
We follow the procedure in Fig.~\ref{fig:method}, and get $|s-s'|$ for a piece of testing utterance $x$.
Denote the score variation $d=|s-s'|$.
Denote $\mathbb{T}_{gen} = \{x_{gen}^{1}, x_{gen}^{2},...,x_{gen}^{I}\}$ is the set of genuine testing utterances, and $\vert \mathbb{T}_{gen} \vert $ denotes the number of elements in set $\mathbb{T}_{gen}$.
Then we derive $\{d_{gen}^{1}=|s_{gen}^{1}-{s_{gen}^{1}}'|,d_{gen}^{2}=|s_{gen}^{2}-{s_{gen}^{2}}'|, ...,d_{gen}^{I}=|s_{gen}^{I}-{s_{gen}^{I}}'|\}$ for $\mathbb{T}_{gen}$ as shown in Fig.~\ref{fig:method}, where $s_{gen}^{i}$ and ${s_{gen}^{i}}'$ are the ASV scores for $x_{gen}^{i}$ before and after vocoder preprocessing respectively.
Given a false positive rate for detection ($FPR_{given}$, a real number), such that $FPR_{given} \in [0,1]$, for genuine samples, we derive a detection threshold $\tau_{det}$:
\begin{align}
    &FPR_{det}(\tau) = \frac{\vert \{ d_{gen}^{i} > \tau : x_{gen}^{i} \in \mathbb{T}_{gen} \} \vert}{\vert \mathbb{T}_{gen} \vert} \label{eq:det-far} \\
    &\tau_{det} = \{ \tau \in \mathbb{R} : FPR_{det}(\tau) =FPR_{given} \} \label{eq:det-threshold} 
\end{align}
where $FPR_{det}(\tau)$ is the false positive rate for genuine samples given a threshold $\tau$, $d_{gen}^{i}$ is derived by $x_{gen}^{i}$ as shown in Fig.~\ref{fig:method}.
In realistic conditions, the ASV system designer is unaware of adversarial samples, not to mention which exact adversarial attack algorithm will be adopted.
So the detection threshold $\tau_{det}$ is determined based on genuine samples.
Hence the detection method does not require knowledge of adversarial sample generation.

Given a testing utterance, be it adversarial or genuine, $|s-s'|$ will be derived, and the system will label it adversarial if $|s-s'| > \tau_{det}$, and vice versa.
The detection rate ($DR_{\tau_{det}}$) under $\tau_{det}$, which is determined by Eq.~\ref{eq:det-threshold}, for adversarial data can be derived as:
\begin{align}
    &DR_{\tau_{det}} = \frac{\vert \{ d_{adv}^{i} > \tau_{det} : x_{adv}^{i} \in \mathbb{T}_{adv} \} \vert}{\vert \mathbb{T}_{adv} \vert} \label{eq:det-rate} 
\end{align}
where $\mathbb{T}_{adv}$ denotes the set of adversarial testing utterances, and $d_{adv}^{i}$ is derived by $x_{adv}^{i}$ as the procedure illustrated in Fig.~\ref{fig:method}.

\subsection{Rationale behind the detection framework}
As the vocoder is data-driven and trained with genuine data during training, it models the distribution of genuine data, resulting in less distortion when generating genuine waveforms.
Thus, during inference, the vocoder's preprocessing will not affect the ASV scores of genuine samples too much, as reflected by the EER in the second row and last column of Table~\ref{tab:EER}.
However, suppose the inputs are adversarial samples. In that case, the vocoder will try to pull it back towards the manifold of their genuine counterparts to some extent, resulting in purifying the adversarial noise.

Take a non-target trial as an example, in which an ASV system should give the genuine sample a score below the threshold.
And after the nearly lossless reconstruction procedure (i.e., the yellow block and gray block in Fig.~\ref{fig:method}), the genuine sample will not change much, and the ASV score will remain largely unchanged.
In contrast to the genuine sample, the ASV score for the adversarial one is higher than the threshold.
And the reconstruction procedure will try to counter the adversarial noise, purify the adversarial sample, and decrease the ASV score for the adversarial sample. 
Then we can adopt the discrepancy of the score variations, $d_{adv}$ and $d_{gen}$, to discriminate between them, as shown in Fig.~\ref{fig:Hist-ep5}.
The transform, which makes $d_{gen}$ as small as possible while makes $d_{adv}$ as large as possible, is suitable for adversarial detection.

Also, the Griffin-Lim can be regarded as an imperfect transform as well, and it will also introduce distortion to affect the adversarial noise.
However, for genuine data, the distortion introduced by the Griffin-Lim is more significant than the vocoder, as it is not a data-driven method and can not be customized for a specific dataset, resulting in larger $d_{gen}$ and inferior detection performance.


\section{Experimental setup}

\begin{table}[t]
\centering
\caption{EER with different $\epsilon$}
\scalebox{0.8}{
\begin{tabular}{cccccc}
\toprule
\multirow{2}{*}{Method} & \multicolumn{5}{c}{EER with different $\epsilon$ (\%)} \\
        & 20      & 15      & 10      & 5       & 0 (no attack)     \\ \hline
None    & 99.33   & 95.66   & 90.57   & 74.04   & 2.88     \\
Vocoder & 87.58    & 65.75   & 52.20   & 30.37   & 3.39    \\
GL-lin  & 95.23   & 80.83   & 66.73   & 39.49   & 3.93              \\
GL-mel  & 88.41   &65.39   & 49.76   & 26.67   & 3.81   \\ \bottomrule
\end{tabular}
}
\label{tab:EER}
\end{table}

\subsection{ASV setup}
The adopted system is a variation of X-vector system, and is modified from \cite{chung2020defence}.
We adopt the dev sets of Voxceleb1 \cite{nagrani2017voxceleb} and Voxceleb2 \cite{chung2018voxceleb2} for training.
Spectrograms are extracted with a Hamming window of width 25ms and step 10ms, and 64-dimensional fbanks are extracted as input features.
No further data augmentation and voice activity detection are adopted during training.
% The model is trained for only 50 epochs rather than 500 epochs in \cite{chung2018voxceleb2} because the focus here is to evaluate the performance of detection method rather than the performance of ASV.
Cosine similarity is used for back-end scoring.
We adopt the trials provided in VoxCeleb1 test set for generating adversarial samples, evaluating the ASV performance and detection performance.
% The equal error rate (EER) for genuine samples is 2.88\% as shown in Table~\ref{tab:EER}.

\subsection{Griffin-Lim and Parallel WaveGAN}
We use Griffin-Lim and Parallel WaveGAN in our experiments. The Griffin-Lim method, denoted as ``GL-lin", uses 100 iterations to reconstruct speech from linear spectrograms. ``GL-mel" denotes that linear spectrograms are first estimated from Mel-spectrograms using the pseudo inverse. Our Parallel WaveGAN method, denoted as "Vocoder", is modified from the public implementation\footnote[4]{https://github.com/kan-bayashi/ParallelWaveGAN}.
% \footnote[4]{\href{https://github.com/kan-bayashi/ParallelWaveGAN}{\texttt{Parallel WaveGAN}}}.
We use 80-dimension, band-limited (80-7600 kHz), and normalized log-mel spectrograms as conditional features. The window and hop sizes are set to 50 ms and 12.5 ms.
The architectures of the generator and discriminator follow those in \cite{yamamoto2020parallel}. We trained the model on the dev set of VoxCeleb1 ~\cite{nagrani2017voxceleb} for 1000k iterations, which takes around 5 days. 
Note that there is no overlap between the training data of Voxceleb1 for neural vocoder and the evaluation data of speaker verification.
To further show that the vocoder adopted in the proposed method is dataset independent, we also trained a universal vocoder~\cite{hsu2019towards} with the same structure as Vocoder, but on Lrg dataset \cite{hsu2019towards}, which is a large speech dataset containing 6 languages and more than 600 speakers.
The vocoder trained on Lrg is denoted as "Vocoder-L".

\subsection{ASV performance with genuine and adversarial inputs}
\label{subsec:ASV performance}
To evaluate the performance, we use the trials provide in VoxCeleb1 test set, which contains 37,720 enrollment-testing pairs.
During adversarial samples generation, $\alpha$ is set as $1$, attack budget $\epsilon$ is set as $5,10,15,20$. 
The adversarial attack is conducted in the time domain.
Also, note that it is time-consuming to generate adversarial samples.
We first evaluate the performance of our ASV system on genuine and adversarial samples.
The results are shown in the first row and the last column of Table~\ref{tab:EER}. "None" denotes that utterances are passed directly to the ASV system. 
We find that: 
(1) When testing on genuine samples, the ASV system achieved an EER of 2.88\%, comparable to recent ASV models.
When using generated speech as input, we found that the EER slightly increased.
(2) While introducing the adversarial attack, the EER increased from 2.88\% to over 70\%, which shows the effectiveness of the attack method.
The larger the attack budget $\epsilon$ is, the higher the attack intensity is.
One may question why the EER is over 50\%. The threshold of ASV is fixed, and the attackers try their best to do adversarial attack to make the score over and below the threshold for non-target and target trials respectively, resulting in the decisions for the trials reversed.



\section{Experimental Results}
\label{sec:expt-rst}

\begin{table}[t]
\centering
\caption{AUC with different $\epsilon$}
\label{tab:AUC}
\scalebox{0.8}{
\begin{tabular}{ccccc}
\toprule
\multirow{2}{*}{Method} & \multicolumn{4}{c}{AUC with different $\epsilon$ (\%)} \\
        & 20      & 15      & 10      & 5       \\ \hline
Vocoder & \textbf{99.94}    & \textbf{99.62}   & \textbf{99.12}   & \textbf{96.52}   \\
GL-lin  & 97.89   & 97.39   & 95.87   & 89.86   \\
GL-mel  & 99.01   & 97.64   & 95.41   & 87.52   \\ \bottomrule
\end{tabular}
\vspace{-5pt}
}
\end{table}




% \subsubsection{Defense performance}
% \label{subsec:Defence performance}
The defense performance is shown in Table~\ref{tab:EER}.
% Vocoder refers to Parallel WaveGAN.
% GL-lin and GL-mel refer to using Griffin-Lim to re-synthesis waveform from linear spectrograms and mel-spectrograms, respectively.
The EER decrease after audio re-synthesis illustrates that all the three methods can slightly alleviate the adversarial noise.
In contrast to \cite{joshi2021adversarial}, which effectively purifies the adversarial noise for speaker identification by vocoders, directly applying vocoders for defending speaker verification does not work.
Yet the re-synthesis process will not affect the genuine EER too much. 
Thus, we adopt the difference of the ASV scores between the original and re-synthesized audio as a good indicator to distinguish among genuine and adversarial samples. 

As mentioned in Sec~\ref{subsec:vocoder based detection}, we use the discrepancy $|s-s'|$ to distinguish between genuine and adversarial samples.
The distributions of the discrepancy are in Fig.~\ref{fig:Hist-ep5}.
A threshold $\tau_{det}$ can be determined using genuine samples by Eq.~\ref{eq:det-threshold} to separate genuine and adversarial samples.
The detection framework based on vocoder in Fig.~\ref{fig:method} is effective, where the discrepancy $|s-s'|$ for genuine samples is small, yet $|s-s'|$ for adversarial samples is large.
Notice that the distribution overlap between the GL-mel genuine samples and adversarial samples is more considerable than that of Vocoder, as shown in Fig.~\ref{fig:Hist-ep5}.
As Vocoder is trained from genuine data, it attains the capacity of making $|s-s'|$ for genuine samples small enough.
Yet Griffin-Lim can only enlarge $|s-s'|$ for adversarial samples.

\begin{figure}[t]
\centering
\includegraphics[width=0.7\linewidth]{figures/Hist_ep10.pdf}
%\caption{Histogram with $\epsilon=10$}
\caption{$|s-s'|$ distribution with $\epsilon=10$}
\label{fig:Hist-ep5}
\end{figure}

\begin{figure}[h]
    \centering
    \begin{minipage}[h]{0.23\textwidth}
        \centering
        \includegraphics[width=1.6in]{figures/roc_curve_epsilon_5.pdf}
        {\par \footnotesize (a)\par}
    \end{minipage}
    \hfill
    \begin{minipage}[h]{0.23\textwidth}
        \centering
        \includegraphics[width=1.6in]{figures/roc_curve_epsilon_10.pdf}
        {\par \footnotesize (b)\par}
    \end{minipage}
    \hfill
    \begin{minipage}[h]{0.23\textwidth}
        \centering
        \includegraphics[width=1.6in]{figures/roc_curve_epsilon_15.pdf}
        {\par \footnotesize (c)\par}
    \end{minipage}
    \hfill
    \begin{minipage}[h]{0.23\textwidth}
        \centering
        \includegraphics[width=1.6in]{figures/roc_curve_epsilon_20.pdf}
        {\par \footnotesize (d)\par}
    \end{minipage}
    \vspace{-5pt}
    \caption{ROC curve under different epsilon ($\epsilon$) }
    \label{fig:roc-curve-epsilon=5}
    \vspace{5pt}
\end{figure}


% \subsubsection{AUC evaluation}
% \label{subsec:AUC evaluation}
Fig.~\ref{fig:roc-curve-epsilon=5} shows the receiver operating characteristic (ROC) curves of different methods.
The curves show the true positive rates for adversarial samples and false positive rates for genuine samples using different thresholds for detection.
The larger the area under the curve (AUC) is, the better the detection performance.
Both GL-lin and GL-mel achieve effective detection performance, and Vocoder performs better than them.
Table~\ref{tab:AUC} shows the AUC of different methods. 
Table~\ref{tab:AUC} indicates that the proposed method is powerful for adversarial samples detection as all AUCs are approaching or greater than 90\%.
Also, Vocoder outperforms GL-lin and GL-mel in all the settings.

% \vspace{-10pt}
% \subsection{Detection performance}
\begin{table}[t!]
\centering
\caption{Detection rate with different $\epsilon$}
\label{tab:detection_rate}
\scalebox{0.7}{
\begin{tabular}{cccccc}
\toprule
\multirow{2}{*}{$FPR_{given}$} & \multirow{2}{*}{Method} & \multicolumn{4}{c}{Detection rate with different $\epsilon$ (\%)} \\
                        &           & 20        & 15        & 10        & 5         \\ \hline
                        & Vocoder   & \textbf{99.76}    & \textbf{98.82}    & \textbf{97.30}    & \textbf{89.33} \\
\multirow{3}{*}{0.05}   & Vocoder-L    & 99.38     & 97.23     & 94.07     & 81.21     \\
                        & GL-lin    & 89.12     & 88.30     & 84.64     & 71.29     \\
                        & GL-mel    & 95.39     & 91.33     & 85.37     & 68.07     \\ 
                        & Gaussian    & 34.54     & 51.29     & 61.56     & 68.57     \\ \hline
\multirow{3}{*}{0.01}   & Vocoder   & \textbf{98.92}    & \textbf{97.56}    & \textbf{94.76}    & \textbf{81.60} \\
                        & Vocoder-L    & 97.96     & 94.37     & 88.77     & 70.15     \\
                        & GL-lin    & 73.62     & 73.63     & 70.62     & 56.37     \\
                        & GL-mel    & 87.98     & 82.27     & 75.04     & 56.07     
                        \\ \hline
\multirow{3}{*}{0.005}  & Vocoder   & \textbf{98.30}    & \textbf{96.78}    & \textbf{93.25}    & \textbf{78.21} \\
                        & Vocoder-L    & 96.78     & 92.58     & 85.81     & 64.65     \\
                        & GL-lin    & 64.76     & 64.97     & 62.85     & 49.32     \\
                        & GL-mel    & 83.94     & 77.71     & 70.47     & 51.42     
                        \\ \hline
\multirow{3}{*}{0.001}  & Vocoder   & \textbf{96.04}    & \textbf{93.89}    & \textbf{88.60}    & \textbf{68.58} \\
                        & Vocoder-L    & 93.36     & 87.34    & 78.24     & 53.18     \\
                        & GL-lin    & 45.10     & 45.27     & 44.72     & 34.28     \\
                        & GL-mel    & 72.53     & 65.98     & 59.66     & 40.98     
                        \\ \bottomrule
\end{tabular}
}
\vspace{-5pt}
\end{table}

Table~\ref{tab:detection_rate} shows the detection results on adversarial samples with different $\epsilon$.
$FPR_{given}$ column lists different false acceptable rates.
The threshold $\tau_{det}$ was determined according to $FPR_{given}$ as shown in Eq.~\ref{eq:det-threshold}.
Gaussian denotes that we use Gaussian filter to replace feature extraction and vocoder (yellow block and gray block in Fig.~\ref{fig:method}).
Gaussian filter \cite{wu2020defense,wu2021adversarialasv} is usually adopted as an attack-agnostic method to counter adversarial samples, so we also set it as our baseline.
The observations and analysis are concluded as follows:
(1) We find that using Vocoder performs the best among all methods.
In most cases, more than 90\% of the adversarial samples could be detected. While with a large $\epsilon$ or $FPR_{given}$, all the detection rates even exceeded 95\%.
Even with a small $\epsilon$ of 5, the detection rates can still approach or exceed 80\%.
The results indicate that the proposed method can effectively detect adversarial samples.
(2) Gaussian based detection performs the worst, and even with $FPR_{given}=0.05$, the detection rates are still lower than vocoder based detection with $FPR_{given}=0.001$.
As it is not a comparable baseline, we do not show its results in other settings due to space limitation.
(3) For Griffin-Lim based methods, we find that they might be good approaches for detection with a large $\epsilon$ or $FPR_{given}$.
However, in stricter cases (smaller $\epsilon$ or $FPR_{given}$), the detection rates of GL-lin and GL-mel decrease drastically.
We argue that Griffin-Lim is a pseudo, nearly lossless transform, so we can, to some extent, adopt it to replace the vocoders in the adversarial detection framework in Fig.~\ref{fig:method}. 
While Griffin-Lim is not a data-driven method and can not model the genuine data manifold well, it results in higher $|s'-s|$ for genuine samples as shown in Fig.~\ref{fig:Hist-ep5}, and thus the detection performance is not comparable to Parallel WaveGAN.
(4) As shown in Table~\ref{tab:detection_rate}, the detection rate for the Vocoder-L is very close to Vocoder, which indicates the vocoder adopted for the proposed detection method is kind of dataset independent.

Also we try to use the vocoder to detect the Gaussian noise. 
Under the $FPR_{given}$ of 1\%, the detection rate for Gaussian noise is 0.97\%.
Due to space limitation, we won't show the details here.

\documentclass[../absorber.tex]{subfiles}
\begin{document}

Particle-in-cell simulations are useful for investigating intense laser-plasma interactions in overdense plasmas, but a truncated plasma boundary can produce an unphysically hot return current.  This return current is present with absorbing, reflecting and thermal particle boundary conditions alike, and it can drastically alter simulation results.  We have devised an absorbing particle boundary condition that stops energetic particles over a defined region of the simulation space.  Stopping these particles over a sufficiently large distance allows the background plasma to generate a suitably cool return current that mimics the results of a semi-infinite, causally separated simulation.

Various different schemes were proposed for statistically selecting, stopping and re-emitting hot particles, with the best results given by the linearly varying absorber described in Sec.~\ref{sec:linear} that calculates the local temperature via Eq.~(\ref{Eq:lin-int}).  The appropriate mean free path of the absorber was explored, showing that an absorber with a mean free path of $\lambda \gtrsim \bigO(10\,c/\omega_p)$ gives proper results for our tests.
% Even a very short stopping region performed well for short times, but a longer absorbing region was required for long-term performance.
As simulation behavior can vary greatly depending on the application, care must be taken to ensure that the absorber parameters used for a particular case appropriately mimic the behavior of a semi-infinite boundary.

\end{document}

% \section{acknowledgement}
\label{sec:acknowledge}
This work was done when H. Wu was a visiting student at Human-computer Communications Laboratory, the Chinese University of Hong Kong.
H. Wu is supported by Google PHD Fellowship Scholarship.



\pdfoutput=1
\documentclass{article}
\usepackage[final]{pdfpages}
\begin{document}
\includepdf[pages=1-9]{CVPR18VOlearner.pdf}
\includepdf[pages=1-last]{supp.pdf}
\end{document}
\bibliographystyle{IEEEbib}
% \bibliography{refs}

\end{document}
