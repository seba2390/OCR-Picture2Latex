\documentclass[aspectratio=169, table]{beamer}
\usefonttheme[]{serif}
\usepackage{amsmath,amssymb,graphicx,xypic,multimedia,color,dsfont,arydshln, multirow, rotating}
\usepackage[english]{babel}
\usepackage[utf8]{inputenc}
\usepackage{ragged2e, etoolbox, lipsum, textpos}

\usepackage{cite}
\usepackage{amsmath,amssymb,amsfonts}
\usepackage{algorithmic}
\usepackage{graphicx}
\usepackage{textcomp}
\usepackage{xcolor}
\usepackage{amsmath}
\usepackage{amssymb}
\usepackage{breqn}
\usepackage {hyperref}
\usepackage{import}

\setbeamertemplate{itemize item}[ball]
\apptocmd{\frame}{\justifying}{}{}
\usetheme{Madrid}
%%%%%%%%%%%%%%%%%%%%%%%%%%%%%%%%%%%%%%%%%%%%%%%%%%%%%%%%%%%%%%%%%%%%%%%%%%%%%%%%
%% SET COLORS
\definecolor{USC1}{RGB}{130,0,10}
\definecolor{USC2}{HTML}{FFCC00}
\definecolor{USC3}{HTML}{000000}
\definecolor{USCBG}{HTML}{F5E5E5}
\definecolor{USC0}{RGB}{254,235,251}

\definecolor{MY_BLUE}{RGB}{242,242,255}
\definecolor{MY_BLUE2}{RGB}{242,242,255}
\definecolor{MY_BLUE3}{RGB}{220,220,255}

\definecolor{color1}{RGB}{187,215,255}

\definecolor{color2}{RGB}{0,0,0} %black

\definecolor{color3}{RGB}{242,242,242} %white
\definecolor{color4}{RGB}{255,204,0} %white
\definecolor{color5}{RGB}{130,0,10} %Maroon

\definecolor{color_green}{RGB}{0,217,0} %green
\definecolor{color_blue}{RGB}{90,90,255} %blue
\definecolor{color_red}{RGB}{91,4,72} %red
\definecolor{color_darkgreen}{RGB}{0,128,0} %red

\setbeamertemplate{blocks}[rounded][shadow=true]
\addtobeamertemplate{block begin}{\pgfsetfillopacity{1}}{\pgfsetfillopacity{1}}
\renewcommand{\raggedright}{\leftskip=0pt \rightskip=0pt plus 0cm} %justify text in blocks
\beamertemplatenavigationsymbolsempty
%%%%%%%%%%%%%%%%%%%%%%%%%%%%%%%%%%%%%%%%%%%%%%%%%%%%%%%%%%%%%%%%%%%%%%%%%%%%%%%%
\mode<presentation>
 {  \setbeamercolor*{palette primary}{bg=black, fg=white}
 \setbeamercolor*{palette secondary}{bg=color5, fg=white}
 \setbeamercolor*{palette tertiary}{bg=black, fg=white}
 \setbeamercolor{background canvas}{bg=white}
 \setbeamertemplate{title page}[default][shadow=true,rounded=true]
 \addtobeamertemplate{frametitle}{}{
 \begin{textblock*}{100mm}(0.90\textwidth,-0.85cm)
 \includegraphics[scale=0.04]{nvidia_logo_dark.pdf}
 \end{textblock*}}
 \addtobeamertemplate{frametitle}{}{
 \begin{textblock*}{200mm}(-10mm,-0.3mm)
 \includegraphics[scale=0.22]{temp_LINE_blacky}
 \end{textblock*}}
  \addtobeamertemplate{frametitle}{}{
 \begin{textblock*}{200mm}(-10mm,-8.8mm)
 \includegraphics[scale=0.22]{temp_LINE_blackyp}
 \end{textblock*}}
 \addtobeamertemplate{frametitle}{}{
 \begin{textblock*}{200mm}(-10mm,77.8mm)
 \includegraphics[scale=0.22]{temp_LINE_blackyp}
 \end{textblock*}}
 \addtobeamertemplate{frametitle}{\vskip-0.5ex}{}
 }
%%%%%%%%%%%%%%%%%%%%%%%%%%%%%%%%%%%%%%%%%%%%%%%%%%%%%%%%%%%%%%%%%%%%%%%%%%%%%%%%
\title[\textbf{IEEE MMSP 2021}]{\Large\textbf{Dual Attention Network for Heart Rate and Respiratory Rate Estimation}}
\author[\textbf{Yuzhuo Ren \textit{et al.} (Nvidia)}]{\textbf{Yuzhuo Ren, Braeden Syrnyk, Niranjan Avadhanam}\\\texttt{\{yren, bsyrnyk, navadhanam\}@nvidia.com}}
\institute[]{\normalsize{{Nvidia
\\}}} %Autonomous Vehicle
\date{\textbf{October, 2021}} %\date{\today}
\setbeamercolor*{title}{bg=MY_BLUE,fg=black}
%%%%%%%%%%%%%%%%%%%%%%%%%%%%%%%%%%%%%%%%%%%%%%%%%%%%%%%%%%%%%%%%%%%%%%%%%%%%%%%%

\setbeamertemplate{caption}[numbered]
%\captionsetup[table]{justification=raggedright,singlelinecheck=false}
%%%%%%%%%%%%%%%%%%%%%%%%%%%%%%%%%%%%%%%%%%%%%%%%%%%%%%%%%%%%%%%%%%%%%%%%%%%%%%%%
\begin{document}

% TO SET FOOTER (SLIDE # FORMAT)
\bgroup
\makeatletter
\setbeamertemplate{footline}
{  \leavevmode%
  \hbox{%
  \begin{beamercolorbox}[wd=.333333\paperwidth,ht=2.25ex,dp=1ex,center]{author in head/foot}%
    \usebeamerfont{author in head/foot}\insertshortauthor\expandafter\beamer@ifempty\expandafter{\beamer@shortinstitute}{}{~~(\insertshortinstitute)}
  \end{beamercolorbox}%
  \begin{beamercolorbox}[wd=.333333\paperwidth,ht=2.25ex,dp=1ex,center]{title in head/foot}%
    \usebeamerfont{title in head/foot}\insertshorttitle
  \end{beamercolorbox}%
  \begin{beamercolorbox}[wd=.333333\paperwidth,ht=2.25ex,dp=1ex,right]{date in head/foot}%
    \usebeamerfont{date in head/foot}\insertshortdate{}\hspace*{2em}
    \hspace*{6ex}
  \end{beamercolorbox}}%
  \vskip0pt%
}
\makeatother
%%%%%%%%%%%%%%%%%%%%%%%%%%%%%%%%%%%%%%%%%%%%%%%%%%%%%%%%%%%%%%%%%%%%%%%%%%%%%%%%
\begin{frame}
\frametitle{\textcolor{color1}{}}
  \titlepage
  \vspace*{-5mm}
\begin{figure}[h!]
\centering
\includegraphics[scale=0.10]{nvidia_logo_cover.pdf}
\end{figure}
\end{frame}
\setbeamercolor{background canvas}{bg=white}
\egroup

\setcounter{framenumber}{0}
%%%%%%%%%%%%%%%%%%%%%%%%%%%%%%%%%%%%%%%%%%%%%%%%%%%%%%%%%%%%%%%%%%%%%%%%%%%%%%%%
%%%%%%%%%%%%%%%%%%%%%%%%%%%%%%%%%%%%%%%%%%%%%%%%%%%%%%%%%%%%%%%%%%%%%%%%%%%%%%%%

%%%%%%%%%%%%%%%%%%%%%%%%%%%%%%%%%%%%%%%%%%%%%%%%%%%%%%%%%%%%%%%%%%%%%%%%%%%%%%%%
\begin{frame}
\frametitle{\textbf{Problem Statement}}
\textbf{Video-Based Heart Rate and Respiratory Rate Measurement}:

\begin{figure}
\centering
\includegraphics[height=3.5cm]{presentation_overview.pdf}
\label{flow}
\end{figure}

\end{frame}

%%%%%%%%%%%%%%%%%%%%%%%%%%%%%%%%%%%%%%%%%%%%%%%%%%%%%%%%%%%%%%%%%%%%%%%%%%%%%%%%
\begin{frame}
	\frametitle{\textbf{Related Work}}
	
	\begin{itemize}
		\item \textbf{Convolutional Attention Network (2D-CAN)} [Chen \textit{et al.}, 2018]: 
		\begin{itemize}\normalsize
			\item Two branch (motion map branch and appearance map branch) network with spatial attention was proposed. \\[2mm]
		\end{itemize}
		\item \textbf{Temporal Shift CAN (TS-CAN)} [Liu \textit{et al.}, 2020]:
		\begin{itemize}\normalsize
		 	\item Proposed to add temporal shift [Lin \textit{et al.}, 2019] layer to 2D-CAN. \\[2mm]
		 \end{itemize}
		\item \textbf{Multitask TS-CAN (MT-TS-CAN)} [Liu \textit{et al.}, 2020]:
		\begin{itemize}\normalsize
		 	\item Proposed multitask learning for joint heart rate and respiratory rate estimation.
		 \end{itemize}
	\end{itemize}
	
\end{frame}

%%%%%%%%%%%%%%%%%%%%%%%%%%%%%%%%%%%%%%%%%%%%%%%%%%%%%%%%%%%%%%%%%%%%%%%%%%%%%%%%
\begin{frame}
\frametitle{\textbf{Overview of Our Method}}
\begin{figure}
\begin{center}
\includegraphics[height=5cm]{presentation_overview_2.pdf}
\end{center}
\caption{Overview of our proposed temporal shift dual attention network (TS-DAN) for estimating heart rate and respiratory rate compared to previous proposed temporal shift convolutional attention network (TS-CAN) \cite{liu2020multi}. TS-DAN leverages both spatial attention and channel-wise attention. Each of these models can be applied in a single or multi-task fashion.}
\label{flow}
\end{figure}

\end{frame}

%%%%%%%%%%%%%%%%%%%%%%%%%%%%%%%%%%%%%%%%%%%%%%%%%%%%%%%%%%%%%%%%%%%%%%%%%%%%%%%%
\begin{frame}
\frametitle{\textbf{Temporal Shift Dual Attention Network (TS-DAN)}}
\begin{figure}
\resizebox{\columnwidth}{!}{\begin{center}
\includegraphics[height=5.5cm]{architecture_2.pdf}
\end{center}}
\caption{We present a multi-task temporal shift dual attention convolutional network (MT-TS-DAN) for joint heart rate and breath rate measurement.}
\label{architecture}
\end{figure}
\end{frame}

%%%%%%%%%%%%%%%%%%%%%%%%%%%%%%%%%%%%%%%%%%%%%%%%%%%%%%%%%%%%%%%%%%%%%%%%%%%%%%%%

 \begin{frame}
 \frametitle{\textbf{Spatial Attention}}

\vspace*{7mm}
\begin{figure}[h!]
  		\centering
  		\begin{minipage}[l]{.80\textwidth}
  			    \begin{itemize}
  			        \item $\mathbb{Z}^{k}$: The masked feature map  passed to next layer 
  			        \item $\mathbb{X}_{m}^{k} $: The motion branch feature map
  			        \item $\mathbb{X}_{a}^{k} $: The appearance branch feature map
  			        \item $\sigma(\cdot)$: Sigmoid activation function
  			        \item $\omega^{k}$: The $1\times1$ convolution kernel
  			        \item $b^{k}$ : The bias

  			        \item $\odot$: Element-wise multiplication
  			        \item $k$: The layer index
  			        \item $H_{k}$ and $W_{k}$: Height and width of feature map
  			   \end{itemize}
              %%%%
              \begin{equation*}\boxed{
                 \begin{split}
                  \mathbb{Z}^{k} = \frac{H_{k}W_{k}\cdot \sigma(\omega^{k} \mathbb{X}_{a}^{k} + b^{k})}{2||\sigma(\omega^{k} \mathbb{X}_{a}^{k} + b^{k})||_{1}} \odot \mathbb{X}_{m}^{k}
                       % &s\tilde{\bold{m}} = \bold{K} %\Delta([\bold{R}\:\bold{t}] \tilde{\bold{M}})
                \end{split}}
                \end{equation*}
                %%%%
       \end{minipage}
  		\hfill
\end{figure}
 \end{frame}
 %%%%%%%%%%%%%%%%%%%%%%%%%%%%%%%%%%%%%%%%%%%%%%%%%%%%%%%%%%%%%%%%%%%%%%%%%%%%%%%%
\begin{frame}
\frametitle{\textbf{Channel-wise Attention}}\small

\begin{figure}
\begin{center}
\includegraphics[height=4cm]{presentation_channelattention.pdf}
\end{center}
\caption{Diagram of efficient channel attention (ECA) module. \cite{wang2020eca}}
\label{flow}
\end{figure}

\end{frame}
 
%%%%%%%%%%%%%%%%%%%%%%%%%%%%%%%%%%%%%%%%%%%%%%%%%%%%%%%%%%%%%%%%%%%%%%%%%%%%%%%%
\begin{frame}
\frametitle{\textbf{Multitask Learning Loss}}\small

\vspace*{7mm}
\begin{figure}[h!]
  		\centering
  		\begin{minipage}[l]{.80\textwidth}
  			    \begin{itemize}	
  			        \item $L$: Multitask learning loss
  			        \item $p(t)$ and $r(t)$: Time variant ground truth pulse waveform sequence and respiratory waveform sequence respectively
  			        \item $p(t)^{'}$ and $r(t)^{'}$: Predicted pulse waveform and respiratory waveform
  			        \item $\alpha$ and $\beta$: Empirical parameters to balance pulse waveform loss, respiratory waveform loss and correlation loss. We set $\alpha=\beta=1$.
  			        
  			        \item $T$: The time window
  			   \end{itemize}
              %%%%
              \begin{equation*}\boxed{
                 \begin{split}
                 \textit{L} = \alpha \frac{1}{T} \sum_{t = 1}^{T}(p(t) - p(t)^{'})^{2} + \beta \frac{1}{T} \sum_{t = 1}^{T}(r(t) - r(t)^{'})^{2}
                \end{split}}
                \end{equation*}
                %%%%
       \end{minipage}
  		\hfill
\end{figure}

\end{frame}
%%%%%%%%%%%%%%%%%%%%%%%%%%%%%%%%%%%%%%%%%%%%%%%%%%%%%%%%%%%%%%%%%%%%%%%%%%%%%%%%
\begin{frame}
\frametitle{\textbf{Post Processing}}\small
\textbf{Heart Rate and Respiratory Rate Estimation From Waveform:}
\begin{enumerate}
    \item Butterworth bandpass filter was applied to the heart rate and respiratory rate waveform.
    \begin{itemize}
    \item Cut-off frequencies for heart rate: 0.67 and 4 Hz
    \item Cut-off frequencies for respiratory rate: 0.08 and 0.50 Hz
    \end{itemize}
    \item The filtered signals were then divided into 10-second windows to apply the Fourier transform to get the dominant frequencies as the heart rate and respiratory rate.
\end{enumerate}

\end{frame}
%%%%%%%%%%%%%%%%%%%%%%%%%%%%%%%%%%%%%%%%%%%%%%%%%%%%%%%%%%%%%%%%%%%%%%%%%%%%%%%%
\begin{frame}
\frametitle{\textbf{Experiments}} %small

\begin{enumerate}
  \item \textbf{Benchmark datasets}
  \item \textbf{Evaluation Metric}
  \item \textbf{Ablation on proposed TS-DAN}
   
  
%  \begin{figure}[h!]
%    \centering\hspace*{-10mm}
%    \includegraphics[scale=0.45]{metrics}
%  \end{figure}

%\vspace*{5mm}
  \item \textbf{Single task TS-DAN evaluation}
%\vspace*{5mm}
    \item \textbf{Multitask task TS-DAN evaluation}
%\vspace*{5mm}
    \item \textbf{Cross-dataset generalization evaluation}
\end{enumerate}

\end{frame}
%%%%%%%%%%%%%%%%%%%%%%%%%%%%%%%%%%%%%%%%%%%%%%%%%%%%%%%%%%%%%%%%%%%%%%%%%%%%%%%%
\begin{frame}
	\frametitle{\textbf{Experiments: Benchmark Datasets}}
	\begin{enumerate}
        \item \textbf{COHFACE} \cite{heusch2017reproducible}
            \begin{itemize}
            \item 40 subjects
            \item Sessions for each subject: 2 with clean conditions, 2 with natural conditions (changing lightening variations)
            \item Video frame rate: 20
            \item Recorded ground truth: heart rate & respiratory rate waveform
            \end{itemize}
        \item \textbf{Our dataset}
            \begin{itemize}
            \item 15 subjects
            \item Sessions for each subject: 1 with clean conditions, 1 with natural conditions (changing distance to camera, performing head motions and changing lightening variations)
            \item Video frame rate: 30
            \item Recorded ground truth: heart rate waveform
            \end{itemize}
    \end{enumerate}

\end{frame}

%%%%%%%%%%%%%%%%%%%%%%%%%%%%%%%%%%%%%%%%%%%%%%%%%%%%%%%%%%%%%%%%%%%%%%%%%%%%%%%%
\begin{frame}
	\frametitle{\textbf{Experiments: Evaluation Metric}}
		\begin{enumerate}
            \item \textbf{Mean Absolute Error (MAE)}
            \item \textbf{Signal to Noise Ratio (SNR)}
            \item \textbf{Availability}
            \begin{itemize}
                \item Percentage of evaluation window with SNR $\geq$ 0
            \end{itemize}
\end{enumerate}
\end{frame}

%%%%%%%%%%%%%%%%%%%%%%%%%%%%%%%%%%%%%%%%%%%%%%%%%%%%%%%%%%%%%%%%%%%%%%%%%%%%%%%%
%%%%%%%%%
\begin{frame}
	\frametitle{\textbf{Experiments: Ablation on proposed TS-DAN}}
	%%%%%%%%%%%%%%%%%%%%%%%%%%%%%%%%%%%%%%%%%%%%%%%%%%%%%%%%%%%
\begin{figure}[h!]
\begin{minipage}[c]{.95\textwidth}
\begin{table}[htbp]
\centering
\caption{Ablation on proposed temporal shift dual attention network (TS-DAN) on COHFACE dataset.}
\resizebox{\columnwidth}{!}{\begin{tabular}{lccc|ccc|ccc}
\hline\noalign{\smallskip}
\bf{Method} & \multicolumn{3}{c}{\bf{Full}}  &  \multicolumn{3}{c}{\bf{Clean}} & \multicolumn{3}{c}{\bf{Natural}}\\
%\bf{} & \bf{Mean} \bf{Std} & \bf{Mean} \bf{Std}\\
\noalign{\smallskip}
\hline
\noalign{\smallskip}
Heart Rate & MAE & Availability & SNR & MAE & Availability & SNR & MAE & Availability & SNR\\
2D-CAN\cite{chen2018deepphys}   &3.693&0.848&5.479&2.702&  0.846& 5.998&4.683& 0.850& 4.960\\
TS-CAN\cite{liu2020multi}       &1.751&0.907&6.712&1.771&  \textbf{0.963}& 7.099&1.735& 0.850& 6.325 \\
2D-CAN+1ECA                     &2.959&0.867&5.636&1.681& 0.865 & 6.232& 4.237 & 0.869 & 5.040  \\ %slides #37
TS-DAN(1ECA)                     &\textbf{1.441}&0.923&6.860&1.150&  0.939& 7.395&\textbf{1.731}& 0.906&6.324  \\
TS-DAN(3ECA)  &1.668&\textbf{0.929}&\textbf{6.939}&\textbf{1.111}&  0.939& \textbf{7.437}&2.224& \textbf{0.919}&\textbf{6.441}    \\
\hline
\end{tabular}}
\label{ablation}
\end{table}
\end{minipage}
\end{figure}
%%%%%%%%%%%%%%%%%%%%%%%%%%%%%%%%%%%%%%%%%%%%%%%%%%%%%%%%%%%
\end{frame}

%%%%%%%%%%%%%%%%%%%%%%%%%%%%%%%%%%%%%%%%%%%%%%%%%%%%%%%%%%%%%%%%%%%%%%%%%%%%%%%%

%%%%%%%%%%%%%%%%%%%%%%%%%%%%%%%%%%%%%%%%%%%%%%%%%%%%%%%%%%%%%%%%%%%%%%%%%%%%%%%%
\begin{frame}
	\frametitle{\textbf{Experiments: Ablation on proposed TS-DAN}}

\begin{figure}
\begin{center}
\includegraphics[height=5cm]{attention_map_2.pdf}
\end{center}
\caption{Attention map visualization comparison from 2D-CAN\cite{chen2018deepphys}, TS-CAN\cite{liu2020multi}, TS-DAN (ours). The first attention map is after the second convolution layer and the second attention map is after the fourth convolution layer as shown in \figurename~\ref{architecture} The first and second attention maps from our TS-DAN gives much better face skin localization and gives much lower weights for the background region.}
\label{attention_map}
\end{figure}

\end{frame}

%%%%%%%%%%%%%%%%%%%%%%%%%%%%%%%%%%%%%%%%%%%%%%%%%%%%%%%%%%%%%%%%%%%%%%%%%%%%%%%%
\begin{frame}
	\frametitle{\textbf{Experiments: Single Task TS-DAN}}
%%%%%%%%%%%%%%%%%%%%%%%%10 second evaluation window for HR & RR single task learning%%%%%%%%%%%%%%%%%%%%%%%%%%%%%%%%%%%
%\iffalse

\begin{table}
\centering
\caption{Single task network for heart rate and respiratory rate estimation accuracy comparison using COHFACE dataset.}
\resizebox{\columnwidth}{!}{\begin{tabular}{lccc|ccc|ccc}
\hline\noalign{\smallskip}
\bf{Method} & \multicolumn{3}{c}{\bf{Full}}  &  \multicolumn{3}{c}{\bf{Clean}} & \multicolumn{3}{c}{\bf{Natural}}\\
%\bf{} & \bf{Mean} \bf{Std} & \bf{Mean} \bf{Std}\\
\noalign{\smallskip}
\hline
\noalign{\smallskip}
\textbf{Heart Rate} & MAE & Availability & SNR & MAE & Availability & SNR & MAE & Availability & SNR\\
TS-CAN\cite{liu2020multi} &1.751&0.907&6.712&1.771&  \textbf{0.963}& 7.099&\textbf{1.735}& 0.850& 6.325 \\
TS-DAN(ours) &\textbf{1.668}&\textbf{0.929}&\textbf{6.939}&\textbf{1.111}&0.939&\textbf{7.437}&2.224&\textbf{0.919}&\textbf{6.441} \\\hline
\textbf{Respiratory Rate} &  MAE & Availability & SNR & MAE & Availability & SNR & MAE & Availability & SNR\\
TS-CAN\cite{liu2020multi} &5.845&0.969&11.620&6.107&0.966&\textbf{11.300}&5.583&0.972&11.940 \\
TS-DAN(ours) &\textbf{5.350}&\textbf{0.979}&\textbf{12.010}&\textbf{5.380}&\textbf{0.979}&11.230&\textbf{5.320}&\textbf{0.979}&\textbf{12.790} \\
\hline
\end{tabular}}
%\includegraphics[height=4cm]{sample.pdf}
\label{cohface_single_task}
\end{table}
%\fi
%%%%%%%%%%%%%%%%%%10 second evaluation window for HR & RR%%%%%%%%%%%%%%%%%%%%%%%%%%%%%%%%%%%%%%%%%


\end{frame}
%%%%%%%%%%%%%%%%%%%%%%%%%%%%%%%%%%%%%%%%%%%%%%%%%%%%%%%%%%%%%%%%%%%%%%%%%%%%%%%%

\begin{frame}
	\frametitle{\textbf{Experiments: Multitask TS-DAN}}\small
	%%%%%%%%%%%%%%%%%%%%%%%%10 second evaluation window for HR & RR%%%%%%%%%%%%%%%%%%%%%%%%%%%%%%%%%%%
%\iffalse

\begin{table}
\centering
\caption{Multitask network for heart rate and respiratory rate estimation accuracy comparison using COHFACE dataset.}
\resizebox{\columnwidth}{!}{\begin{tabular}{lccc|ccc|ccc}
\hline\noalign{\smallskip}
\bf{Method} & \multicolumn{3}{c}{\bf{Full}}  &  \multicolumn{3}{c}{\bf{Clean}} & \multicolumn{3}{c}{\bf{Natural}}\\
%\bf{} & \bf{Mean} \bf{Std} & \bf{Mean} \bf{Std}\\
\noalign{\smallskip}
\hline
\noalign{\smallskip}
\textbf{Heart Rate} & MAE & Availability & SNR & MAE & Availability & SNR & MAE & Availability & SNR\\
%2D-CAN\cite{chen2018deepphys}   &3.693&0.848&5.479&2.702&  0.846& 5.998&4.683& 0.850& 4.96\\
%TS-CAN\cite{liu2020multi}       &1.751&0.907&6.712&1.771&  0.963& 7.099& 1.730& 0.850& 6.325 \\
MT-TS-CAN\cite{liu2020multi}  &4.265&0.619&4.454&3.236&0.688&5.190&5.295&0.550&3.718   \\
%TS-DAN(3ECA) &1.668&0.929&6.939&1.111&0.939&7.437&2.224&0.919&6.441 \\
MT-TS-DAN(ours)&\textbf{1.807}&\textbf{0.885}&\textbf{6.651}&\textbf{1.281}&\textbf{0.888}&\textbf{7.312}&\textbf{2.333}&\textbf{0.883}&\textbf{5.990}\\
\hline
\textbf{Respiratory Rate} &  MAE & Availability & SNR & MAE & Availability & SNR & MAE & Availability & SNR\\
%2D-CAN\cite{chen2018deepphys} &   &   &  &  &   &  & &  & \\
%TS-CAN\cite{liu2020multi} &5.845&0.969&11.62&6.107&0.966&11.30&5.583&0.972&11.94 \\
MT-TS-CAN\cite{liu2020multi} &\textbf{5.392}&0.976&\textbf{16.01}&\textbf{5.457}&0.965&\textbf{15.25}&\textbf{5.326}&0.986&\textbf{16.77}  \\
%TS-DAN(3ECA) &5.350&0.979&12.01&5.380&0.9793&11.23&5.32&0.979&12.79 \\
MT-TS-DAN(ours) &5.728&\textbf{0.991}&7.902&6.107&\textbf{0.994}&8.212&5.348&\textbf{0.988}&7.592\\
\hline
\end{tabular}}
\label{cohface}
\end{table}
%\fi
%%%%%%%%%%%%%%%%%%10 second evaluation window for HR & RR%%%%%%%%%%%%%%%%%%%%%%%%%%%%%%%%%%%%%%%%%
	 
	
\end{frame}
%%%%%%%%%%%%%%%%%%%%%%%%%%%%%%%%%%%%%%%%%%%%%%%%%%%%%%%%%%%%%%%%%%%%%%%%%%%%%%%%

\begin{frame}
\frametitle{\textbf{Experiments: Cross-dataset Generalization Study}}
 
%%%%%%%%%%%%%%%%%%%%%%%%%%%%%%%%%%%%%%%%%%%%%%%%%%%%%%%%%%%
\begin{table}[h]
\caption{Model generalization ablation in heart rate estimation. C2N and N2C in parentheses indicate training and testing data. C2N means training data is clean data and testing data is natural data. N2C means training data is natural data and testing data is clean data.}
\begin{center}
\centering
\begin{tabular}{lccc}\hline
  Model & MAE & Availability & SNR\\ \hline
%2D-CAN (C2C)    &0.765 & 0.935 & 4.108 \\  
%TS-CAN (C2C)    &\textbf{0.544} & 0.927 & 4.288 \\  
%TS-DAN (C2C)    &0.655 & \textbf{0.943}& \textbf{4.358}  \\ \hline
2D-CAN \cite{chen2018deepphys} (C2N)    &4.082 & 0.681 & 4.058 \\  
TS-CAN \cite{liu2020multi}(C2N)    &3.230 & \textbf{0.847} & \textbf{5.499}\\  
TS-DAN (C2N)    &\textbf{2.000} & 0.708& 4.816\\ \hline
%2D-CAN (N2N)    &4.422 & 0.528& 3.517\\  
%TS-CAN (N2N)    &\textbf{1.090} & 0.778& 5.391\\  
%TS-DAN (N2N)    &1.456 & \textbf{0.792} & \textbf{6.007}\\ \hline
2D-CAN \cite{chen2018deepphys} (N2C)    &24.797 & 0.236 & 1.822 \\ 
TS-CAN \cite{liu2020multi}(N2C)    &3.617 & 0.699& 3.010 \\ 
TS-DAN (N2C)    &\textbf{1.091} & \textbf{0.805}& \textbf{3.772}\\ \hline
\end{tabular}
\end{center}
\label{generalization}
\end{table}
%%%%%%%%%%%%%%%%%%%%%%%%%%%%%%%%%%%%%%%%%%%%%%%%%%%%%%%%%%%
%%----------------------
%\hfill\scriptsize

\end{frame}

%%%%%%%%%%%%%%%%%%%%%%%%%%%%%%%%%%%%%%%%%%%%%%%%%%%%%%%%%%%%%%%%%%%%%%%%%%%%%%%%
\begin{frame}
	\frametitle{\textbf{Summary and Conclusion}}
	\large
\begin{itemize}
		\item  We apply attention mechanism in both spatial domain and channel-wise domain to improve network accuracy for heart rate and respiratory rate estimation.\\[2mm]
		\item Our proposed  network  is  the  first  one  leveraging  both  spatial attention  and  channel-wise  attention.\\[2mm]
		\item Our proposed network generates attention map with better face skin localization.\\[2mm]
		\item Our proposed network achieves higher accuracy in both single task network and multitask network.\\[2mm]
\end{itemize}

\end{frame}
%%%%%%%%%%%%%%%%%%%%%%%%%%%%%%%%%%%%%%%%%%%%%%%%%%%%%%%%%%%%%%%%%%%%%%%%%%%%%%%%
\section*{}
\begin{frame}
\frametitle{\textbf{References}}

\vspace*{0mm}
{\small
\bibliographystyle{IEEEtran}
\bibliography{bib}
}
\end{frame}

%%%%%%%%%%%%%%%%%%%%%%%%%%%%%%%%%%%%%%%%%%%%%%%%%%%%%%%%%%%%%%%%%%%%%%%%%%%%%%%%
\section*{}
\begin{frame}
\frametitle{\textcolor[rgb]{0.00,0.00,0.00}{}}

\vspace*{0mm}

\begin{center}
\Large{\emph{\textbf{Thank You}}}
\end{center}



\end{frame}
%%%%%%%%%%%%%%%%%%%%%%%%%%%%%%%%%%%%%%%%%%%%%%%%%%%%%%%%%%%%%%%%%%%%%%%%%%%%%%%%

\end{document}
