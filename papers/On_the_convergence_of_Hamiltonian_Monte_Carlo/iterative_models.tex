
In this Section we establish the irreducibility of a Markov kernel associated to a random iterative model.
These results are of independent interest.
Let $\hfunb : \rset^d \times \rset^d \to \rset^d$ and $\alphagen: \rset^d \times \rset^d \to \ccint{0,1}$ be Borel measurable
functions and $\phib : \rset^d \to \ccint{0,\plusinfty}$ be a
probability density with respect to the Lebesgue measure.  Consider the Markov kernel $\Pkerb$ defined for all $x \in \rset^d$ and $\eventA \in \borelSet(\rset^d)$ by
\begin{equation}
\label{eq:def_pkerb}
  \Pkerb(x,\eventA) = \int_{\rset^d} \indi{\eventA}{\hfunb(x,z)} \alpha(x,z) \phib(z) \rmd z + \bar{\alpha}(x) \delta_x(\eventA) \eqsp,
\end{equation}
where $\bar{\alpha}(x)= \int_{\rset^d} \alpha(x,z) \phib(z) \rmd z$. 
Define for all $x \in \rset^d$, $\hfunb_x : \rset^d \to \rset^d$ by $\hfunb_x = \hfunb(x,\cdot)$.

First, we give  a result  from geometric measure
theory together with a proof for the reader's convenience, which will be essential for the proof of the statements of this section.  Let
$\ouvert\subset\rset^d$ be an open set and $\Theta: \ouvert\to \rset^d$ be
a measurable function such that there exist $y_0 , \tilde{y}_0 \in \rset^d$ and
$M, \tilde{M} > 0$ satisfying $\ball{\tilde{y}_0}{\tilde{M}} \subset \ouvert$ and
  \begin{equation}
    \label{eq:condition_sigma_finite}
\ball{y_0}{M} \subset \Theta(\ball{\tilde{y}_0}{\tilde{M}}) \eqsp.
  \end{equation}
 Define the measure $\lambda_{\Theta}$ on $(\rset^d,\borelSet(\rset^d))$
  by setting for any $\eventA \in \borelSet(\rset^d)$
\begin{equation}
  \label{eq:push_forward_mes}
\lambda_\Theta (\eventA) \eqdef \Leb\defEns{\Theta^{-1}(\eventA) \cap \ball{\tilde{y}_0}{\tilde{M}}}  \eqsp.
\end{equation}
Note %by \eqref{eq:condition_sigma_finite}
that $\lambda_\Theta$ is  a finite measure. Therefore by the Lebesgue decomposition theorem (see
\cite[Section 6.10]{rudin:1987}) there exist two  measures
$\lambda_\Theta^{(\text{a})}, \lambda_\Theta^{(\text{s})}$ on
$(\rset^d,\borelSet(\rset^d))$, which are absolutely continuous and
singular with respect to the Lebesgue measure on $\rset^d$
respectively, such that $\lambda_\Theta = \lambda_\Theta^{(\text{a})} +
\lambda_\Theta^{(\text{s})}$.
% Note that if for all compact set $\compact \subset \rset^d$, $\Theta^{-1}(\compact)$ is
% compact (\ie~$\Theta$ is a proper function), then $\lambda_\Theta$ is $\sigma$-finite.
\begin{proposition}\label{le:simple}
  Let $\ouvert\subset\rset^d$ be open and $\Theta: \ouvert\to \rset^d$ be a Lipschitz function
  satisfying \eqref{eq:condition_sigma_finite}.
 For any version  $\phi_\Theta$ of the density of $\lambda_\Theta^{(\text{a})}$ with respect to the Lebesgue
  measure on $\rset^d$, it holds
$$
\phi_\Theta(y)\geq \1_{ \ball{y_0}{M}}(y) \norm{\Theta}_{\Lip}^{-d}\eqsp, \quad \text{$\Leb$-a.e.}
$$
\end{proposition}
\begin{proof}
  Denote by $L = \norm{\Theta}_{\Lip}$. Let $y\in \ball{y_0}{M}$. By  \eqref{eq:condition_sigma_finite}, we may pick
  $z \in \ball{\tilde{y}_0}{\tilde{M}}$ such that $\Theta(z) = y$. Let
  $\delta_0 >0$ be such that $\ball{z}{\delta_0/L} \subset
  \ball{\tilde{y}_0}{\tilde{M}}$.  Since $\Theta$ is
  Lipschitz continuous,  for all
  $\delta \in \rset_+^*$,
  $\Theta(\ball{z}{\delta/L} \cap \open)\subset \ball{y}{\delta}$. Hence, for all
  $\delta \in \ocint{0,\delta_0}$, we have
  $$
\lambda_\Theta(\ball{y}{\delta})\geq
  L^{-d}\Leb(\ball{z}{\delta}) =   L^{-d}\Leb(\ball{y}{\delta}) \eqsp.
$$
 The claim follows from the differentiation theorem
  for measures, see \cite[Theorem 7.14]{rudin:1987}.
\end{proof}

We can now state our main results. Let $\rassG,\MassG \in \rset^*_+$ and $y_0 \in \rset^d$.
Consider the following assumptions.
\begin{assumptionG}
\label{assumG:phi}
$\phib$ and $\alphagen$ are lower semicontinuous and positive on $\rset^d$ and $\rset^{2d}$ respectively.
\end{assumptionG}

% \begin{assumptionG}
% \label{assumG:alpha}
%  is lower semicontinuous and positive on .
% \end{assumptionG}

\begin{assumptionG}[$\rassG,y_0,\MassG$]
  \label{assumG:irred_b}
  \begin{enumerate}[label=(\roman*), wide, labelwidth=!, labelindent=0pt]
  \item \label{assumG:irred_b_item_i} There exists $\constLiphx \in \rset_+$ such that for all $x \in \ball{0}{\rassG}$, $\hfunb_x$ is
    $\constLiphx$-Lipschitz, \ie~for all $z_1,z_2 \in \rset^d$,
    $\norm{\hfunb_x(z_1)-\hfunb_x(z_2)} \leq \constLiphx
    \norm{z_1-z_2}$.
\item \label{assumG:irred_b_item_ii} There exist $\tilde{y}_0 \in \rset^d$ and $\tMassG \in \rset_+^*$,
  such that for all $x \in \ball{0}{\rassG}$, $\ball{y_0}{\MassG} \subset \hfunb_x(\ball{\tilde{y}_0}{\tMassG})$.
  \end{enumerate}
\end{assumptionG}


\begin{theorem}
\label{theo:irred}
Assume \Cref{assumG:phi} and that there exist $y_0 \in \rset^d$, $R > 0$ and $M > 0$ such that \Cref{assumG:irred_b}($\rassG,y_0,\MassG$)  is satisfied. Then $\ball{0}{\rassG}$ is $1$-small for $\Pkerb$: for all $x \in \ball{0}{\rassG}$ and $\eventA \in \borelSet(\rset^d)$,
  \begin{equation}
    \Pkerb(x,\eventA) \geq \constLiphx^{-d} \min_{(x,z) \in \ball{0}{R} \times \ball{\tilde{y}_0}{\tMassG}} \defEns{\alphagen(x,z) \phib(z)} \Leb\defEns{\eventA \cap \ball{y_0}{\MassG}} \eqsp,
  \end{equation}
where $(\tilde{y}_0,\tilde{M}) \in \rset^d \times \rset_+^*$ is defined in \Cref{assumG:irred_b}($\rassG,y_0,\MassG$).
\end{theorem}

\begin{proof}%[Proof of \Cref{theo:irred}]
For all $x  \in \ball{0}{\rassG}$ and $\eventA \in \borelSet(\rset^d)$ we get
\begin{align}
\nonumber
\Pkerb(x,\eventA)  &= \int_{\rset^d}\indi{\eventA}{\hfunb(x,z)} \alphagen(x,z) \phib(z) \rmd z  = \int_{\rset^d}\indi{\hfunb_x^{-1}(\eventA)}{z} \alphagen(x,z) \phib(z) \rmd z \\
&\geq  \min_{(x,z) \in \ball{0}{R} \times \ball{\tilde{y}_0}{\tilde{M}}} \defEns{\alphagen(x,z)\phib(z)} \Leb\defEns{\hfunb_x^{-1}(\eventA) \cap\ball{\tilde{y}_0}{\tilde{M}}} \eqsp.
\label{eq:coro_leb_irred_1}
\end{align}
The proof follows from \Cref{le:simple} and \Cref{assumG:irred_b}$(R,y_0,M)$-\ref{assumG:irred_b_item_i} which imply
$ \Leb\defEns{\hfunb_x^{-1}(\eventA )\cap \ball{\tilde{y}_0}{\tilde{M}}} \geq  \constLiphx^{-d}  \Leb\defEns{\eventA  \cap \ball{y_0}{M}}$.
\end{proof}
The following Corollary is a straightforward consequence of \Cref{theo:irred}.
\begin{corollary}
\label{coro:irred}
Assume \Cref{assumG:phi} and  that there exists $(y_0,M) \in \rset^d \times \rset_+^*$ such that  for all $\rassG \in \rset_+^*$ \Cref{assumG:irred_b}($\rassG,y_0,\MassG$). Then $\Pkerb$ is irreducible with irreducibility measure $\Leb\defEns{\cdot \cap \ball{y_0}{\MassG}} $. In addition, all the compact sets are $1$-small.
\end{corollary}
In the next proposition, we give examples of functions $f$ which satisfy \Cref{assumG:irred_b}.
\begin{proposition}\label{le:degree_application}
Let  $\ga$ a function from $ \rset^d\times \rset^d$ to $\rset^d$ and $\ra \in \rset^{*}_+$. Assume that
 % \begin{enumerate}[label=\roman*)]
%   \item \label{application:irred_b_item_i}
% for all $\ra >0$ there exists $\Lga \geq
%     0$ such that for all $x \in \ball{0}{\ra}$, $z \mapsto \ga(x,z)$ is
%     $\Lga$-Lipschitz, \ie~for all $x \in \ball{0}{\ra}$,  $z_1,z_2 \in \rset^d$,
%     $\abs{\ga(x,z_1)-\ga(x,z_2)} \leq \Lga
%     \abs{z_1-z_2}$.
%\item \label{application:irred_b_item_ii}
\begin{enumerate}[label=(\roman*)]
\item
\label{propo:irred_b_item_i}
 there exists $\lipgr \in \rset_+$  such that for all $z_1,z_2,x \in \rset^d$, $\norm{x} \leq \ra$,
  \begin{equation}
    \label{eq:5}
    \norm{g(x,z_1) - g(x,z_2)} \leq \lipgr\norm{z_1-z_2} \eqsp.
  \end{equation}
\item
\label{propo:irred_b_item_ii}
there exist $ \Cga_{\ra,0} , \Cga_{\ra,1}
\in \rset_+$ such that for all $x,z \in \rset^d$, $\norm{x} \leq \ra$
\begin{equation}
  \label{eq:4}
\norm{g(x,z)} \leq  \Cga_{\ra,0} +   \Cga_{\ra,1} \norm{z}
\end{equation}
\end{enumerate}

% \begin{equation}\label{eq:gr}
% \abs{g(x,z)} \leq \Cga (1+|x|+|z|) \eqsp.
% \end{equation}
%\end{enumerate}
Let $\bg \in \rset$ and define $\hga : \rset^d \times \rset^d$ for all $x,z \in \rset^d$ by
\begin{equation}
  \hga(x,z) =  \bg z + \ga(x,z) \eqsp.
\end{equation}
If $\norm{\bg} > \Cga_{\ra,1} $, then $\hga$ satisfies \Cref{assumG:irred_b}($\ra,0,\MassG$) for all $\MassG \in \rset_+^*$ with $\tilde{y}_0=0$ and 
\begin{equation}
  \label{eq:deftildeM}
\tilde{M} = \{M  + \Cga_{\ra,0} \}/(\norm{\bg}-\Cga_{\ra,1} ) \eqsp.
\end{equation}
% Let $a,b\in\R$ with $|a|> C$ and for all $x \in \rset^d$ define $\Psi_x : \rset^d \to \rset^d$  by
% $$
% \Psi_x(y)=ay+bx+g(x,y) \eqsp, \text{ for all $y \in \rset^d$} \eqsp.
% $$
% Then for all  $R>0$ there exists $\epsilon >0$ satisfying for all $\eventA \in \borelSet(\rset^d)$, $\eventA \subset \ball{0}{R}$,
% \begin{equation}
%   \inf_{x \in \ball{0}{R}} \lambda_{\Psi_x}(\eventA) > \epsilon \Leb(\eventA) \eqsp,
% \end{equation}
% where $\lambda_{\Psi_x}$ is defined in \eqref{eq:push_forward_mes}.
\end{proposition}
We preface the proof by recalling some basic notions of degree theory.
\label{sec:defin-usef-results}
Let $\Dset$ be a bounded open set of $\rset^d$. Let $f:
\Dsetc \to \rset^d$ be a continuous function on
$\Dsetc$ continuously differentiable on $\Dset$. An element $x \in
\Dset$ is said to be a \emph{regular point} of $f$ if the Jacobian matrix of $f$ at $x$, $\Jac_f(x)$, is invertible.
An element $y \in f(\Dset)$ is said to be a \emph{regular value} of $f$ if any $x \in
f^{-1}(\{ y\})$ is a regular point.  %true if $y \not in f(\Dset)$

Let $f : \Dsetc \to \rset^d$ be a continuous function,  $C^{\infty}$-smooth on $\Dset$. Let $y \in \rset^d
  \setminus f(\partial \Dset)$ be a regular value of $f$. It is shown in \cite[Proposition and Definition 1.1]{outerelo:ruiz:2009} that the set $f^{-1}(\{y\})$ is finite. The degree of $f$ at $y$ is defined by
\begin{equation}
  \deg(f,\Dset,y) = \sum_{x \in f^{-1}(\{y \})} \sign\defEns{\det \parenthese{\Jac_f(x)}} \eqsp.
\end{equation}

\begin{proposition}[\protect{\cite[Proposition and Definition 2.1]{outerelo:ruiz:2009}}]
\label{defProp:degree_cont}
  Let $f : \Dsetc \to \rset^d$ be a continuous function and $y \in
  \rset^d \setminus f(\partial \Dset)$.
  \begin{enumerate}[label=(\alph*)]
  \item
\label{defProp:degree_cont_i}
 Then there exists  $g \in C(\Dsetc, \rset^d) \cap C^{\infty}(\Dset, \rset^d)$ such that $y$ is a regular value of $g$
  and $\sup_{x \in \Dsetc} \abs{f(x)-g(x)} < \dist(y,f(\partial
  \Dset))$.
\item For all functions $g_1,g_2:\Dsetc \to \rset^d$ satisfying \ref{defProp:degree_cont_i},
  \begin{equation}
    \deg(g_1,\Dset,y) = \deg(g_2,\Dset,y) \eqsp.
  \end{equation}
  \end{enumerate}
\end{proposition}
Under the assumptions of \Cref{defProp:degree_cont}, the degree of $f$ at $y$ is then defined for any $g:\Dsetc \to \rset^d$ satisfying \ref{defProp:degree_cont_i} by
\begin{equation}
  \deg(f,\Dset,y) =  \deg(g,\Dset,y) \eqsp.
\end{equation}

\begin{proposition}[\protect{\cite[Proposition
  2.4]{outerelo:ruiz:2009}}]
  \label{theo:deg_modif}
  Let $f,g : \Dsetc \to \rset^d$ be  continuous functions. Define
  $\hpy:\ccint{0,1} \times \rset^d \to \rset^d$ for all $t \in
  \ccint{0,1}$ and $x \in \rset^d$ by $\hpy(t,x) = t f(x) +
  (1-t)g(x)$. Let $y \in \rset^d \setminus \hpy(\ccint{0,1} \times \partial \Dset)$. Then
\begin{equation}
  \deg(f,\Dset,y) =  \deg(g,\Dset,y) \eqsp.
\end{equation}
\end{proposition}
We have now all the necessary results to prove \Cref{le:degree_application}.
\begin{proof}[Proof of \Cref{le:degree_application}]
Since $\hga(x,z) =  \bg z + \ga(x,z)$ and $\ga(x,\cdot)$ is Lipschitz with a Lipschitz constant which is uniformly bounded over the ball $\ball{0}{R}$,  $\hga_x$ is Lipschitz with bounded Lipschitz constant over this ball. Hence \Cref{assumG:irred_b}($\ra,0,\MassG$)-\ref{assumG:irred_b_item_i} holds.

  For all $x \in \rset^d$, denote by $\hga_x : z \mapsto \hga(x,z)$ where $\hga(x,z)=bz + g(x,z)$.
  Let $\MassG \in \rset_+^*$. We show that for all $x \in
  \ball{0}{\ra}$, $\ball{0}{\MassG} \subset
  \hga_x(\ball{0}{\tMassG})$, where $\tMassG$ is given by
  \eqref{eq:deftildeM}, which is precisely
  \Cref{assumG:irred_b}($\ra,0,\MassG$)-\ref{assumG:irred_b_item_ii}.

 % Then for all $z \in
%   (\hga_x)^{-1}(\ball{0}{\MassG}) $, by \ref{propo:irred_b_item_ii}
% \begin{equation}
%   \MassG \geq \abs{\hga_x(z)} \geq  \abs{ \bg z}  - \Cga_{\ra,0} -\Cga_{\ra,1} \abs{z} \eqsp.
% \end{equation}
% Therefore since $\abs{\bg} \geq \Cga_{\ra,1} $, $(\hga_x)^{-1}(\ball{0}{\MassG}) \subset \ball{0}{\tMassG}$ where $\tMassG$ is given by
%\eqref{eq:deftildeM}.
% $\tMassG = \{\MassG + \ra \abs{\ag} +
% \Cga(1+\abs{\ra})\}/(\abs{\bg}-\Cga)$.
%Next we show \ref{item:proof:homot_ii}.
%  Let $x \in \ball{0}{\ra}$ and
% $\MassG \geq 0$.
  Let $x \in \ball{0}{\ra}$ and consider the continuous homotopy $\hog : \ccint{0,1}
\times \rset^d$ between the functions $z \mapsto \bg z$ and $\hga_x$ defined for all
$t \in \ccint{0,1}$ and $z \in \rset^d$ by
\begin{equation}
  \hog(t,z) = t \bg z + (1-t)\hga_x(z) = \bg z + (1-t)  \ga(x,z)  \eqsp.
\end{equation}
Then by \ref{propo:irred_b_item_ii}, since $\abs{\bg} \geq \Cga_{\ra,1} $, for all $t\in \ccint{0,1}$ and $z \not \in
\ball{0}{\tMassG}$, where $\tMassG$ is given by \eqref{eq:deftildeM},
\begin{equation}
   \abs{\hog(t,z)} \geq  \abs{\bg z} -(1-t)\defEns{\Cga_{\ra,0} +\Cga_{\ra,1} \abs{z} } \geq \MassG \eqsp.
\end{equation}
In particular, we have $\hog(\ccint{0,1} \times \partial
\ball{0}{\tMassG}) \subset \rset^d \setminus \ball{0}{\MassG}$. Let
$z \in \ball{0}{\MassG}$, then by
%\cite[Proposition 2.4, Proposition-Definition 1.1, Chapter IV]{outerelo:ruiz:2009},
\Cref{theo:deg_modif} we have
\begin{equation}
  \deg(\hga_x,\ball{0}{\tMassG},z) = \deg(\bg \Id, \ball{0}{\tMassG},z) = 1 \eqsp.
\end{equation}
Besides, by \cite[Corollary 2.5, Chapter IV]{outerelo:ruiz:2009},
$\deg(\hga_x,\ball{0}{\tMassG},z) \not = 0$ implies that there exists
$y \in \ball{0}{\tMassG}$ such that $\hga_x(y) = z$. Finally \Cref{assumG:irred_b}($\ra,0,\MassG$)-\ref{assumG:irred_b_item_ii} follows since this
result holds for all $z \in \ball{0}{\MassG}$.
\end{proof}
% which implies that $\hga_x(\rset^d \setminus
% \ball{0}{\tilde{M}}) \subset \rset^d \setminus \ball{0}{M}$
% $(\hga_x)^{-1}(\ball{0}{M}) \subset \ball{0}{\tilde{M}}$.
% % Then \ref{application:irred_b_item_i} implies
% %   that \Cref{assumG:irred_b}($\ra$)-\ref{assumG:irred_b_item_i} holds. We now show that under

% We first show that for all $x \in \rset^d$, $\Psi_x$ is surjective from $\rset^d$ to $\rset^d$. Now a standard perturbation
% theorem from degree theory (see e.g. \alain{I think we can cite
%   Milnor}), applied to the continuous family of maps
% $\{\Phi_x(t,\cdot) \, : \, t \in \ccint{0,1} \}$, defined for all $t
% \in \ccint{0,1}$
% $$
% y \mapsto \Phi_{x}(t,y):=ay+ t(bx+g(x,y)) \eqsp,
% $$
% implies that $\Psi_{x}$ is surjective\alain{expliquer }.
%  Assume that $R>0$ is given. Our  assumptions imply clearly that there exists  $M>R/\abs{a}$ such that for all $x \in \ball{0}{R}$ and $y \not \in \ball{0}{M}$,
% $$
% \abs{ay}-\abs{bx+g(x,y)}\geq R \eqsp.
% $$
% Especially, this holds at the boundary $\{y \in \rset^d \, : \,
% \abs{y} = M \}$. Let $x \in \ball{0}{R}$. Therefore $\ball{0}{R}\subset
% \Psi_x(\ball{0}{M})$ for all $x \in \ball{0}{R}$. Let $L$ be the
% Lipschitz constant of $g$ on $B(0,R)\times B(0,M)$.
%  \Cref{le:simple} yields that for any $x\in B(0,R)$ and $\eventA \in \borelSet(\rset^d)$, $\eventA \subset \ball{0}{R}$
%  \begin{equation}
%    \lambda_{\Psi_x}(\eventA) \geq L^{-d}\Leb(\eventA) \eqsp,
%  \end{equation}
% and the proof follows.
%  the
% push-forward measure of the Lebesgue measure on $B(0,R_0)$ under $y\to
% H(x_0,y)$ has a lower density bound $c> 0$ on the ball $B(0,R)$, that
% is independent of $x_0$. As the Gaussian distribution has lower
% bounded density on $B(0,R_0)$ the claim follows.


%%% Local Variables:
%%% mode: latex
%%% TeX-master: "main"
%%% End:


%  Next we record a
%   well-known fact from geometric measure theory together with a proof
%   for the reader's convenience.  Let $\ouvert\subset\rset^d$ be an open
%   set and $f: \ouvert\to \rset^d$ be a measurable function such that there exist $z_0,y_0 \in \rset^d$ and $M, \tilde{M} \geq 0$ satisfying
%   \begin{equation}
%     \label{eq:condition_sigma_finite}
% f^{-1}(\ball{y_0}{M}) \subset \ball{z_0}{\tilde{M}} \eqsp.
%   \end{equation}
%  Define the measure $\lambda_{f}$ on $(\rset^d,\borelSet(\rset^d))$
%   by setting for any $\eventA \in \borelSet(\rset^d)$
% \begin{equation}
%   \label{eq:push_forward_mes}
% \lambda_f (\eventA):=\Leb(f^{-1}(\eventA \cap \ball{y_0}{M}))  \eqsp.
% \end{equation}
% Note by \eqref{eq:condition_sigma_finite} that $\lambda_f$ is  a finite measure. Therefore by the Lebesgue decomposition theorem (see
% \cite[Section 6.10]{rudin:1987}) there exist two non-negative measures
% $\lambda_f^{(\text{a})}, \lambda_f^{(\text{s})}$ on
% $(\rset^d,\borelSet(\rset^d))$, which are absolutely continuous and
% singular with respect to the Lebesgue measure on $\rset^d$
% respectively, such that $\lambda_f = \lambda_f^{(\text{a})} +
% \lambda_f^{(\text{s})}$.
% % Note that if for all compact set $\compact \subset \rset^d$, $f^{-1}(\compact)$ is
% % compact (\ie~$f$ is a proper function), then $\lambda_f$ is $\sigma$-finite.
% \begin{proposition}\label{le:simple}
%   Let $\ouvert\subset\rset^d$ be open and $f: \ouvert\to \rset^d$ be a Lipschitz function
%   satisfying \eqref{eq:condition_sigma_finite}. Let $\phi_f$ be the
%   density of $\lambda_f^{(\text{a})}$ with respect to the Lebesgue
%   measure on $\rset^d$. Then $\Leb$-almost everywhere, it holds
% $$
% \phi_f(y)\geq \1_{f(\ouvert) \cap \ball{y_0}{M}}(y) \norm{f}_{\Lip}^{-d}\eqsp.
% $$
% \end{proposition}
% \begin{proof}
%   % We only need to prove that for $\Leb$-almost all $y \in f(\ouvert)
%   % \cap \ball{y_0}{M}$, it holds $\phi_f(y)\geq
%   % \norm{f}_{\Lip}^{-d}$.
%  Denote by $L = \norm{f}_{\Lip}$. Let $y\in
%   f(\ouvert) \cap \ball{y_0}{M}$ and $\delta_0 >0$  such that $\ball{y}{\delta_0}
%   \subset \ball{y_0}{M}$. Let $z\in f^{-1}(\{y\})$. Since $f$ is
%   Lipschitz continuous, there exists $\delta_1 >0$ such that for all
%   $\delta \in \ccint{0,\delta_1}$, $\ball{z}{\delta/L}\in \ouvert$ and
%   $f(\ball{z}{\delta/L})\subset \ball{y}{\delta}$. Hence, for all
%   $\delta \in \ocint{0,\min(\delta_0,\delta_1)}$, we have
%   $$
% \lambda_f(\ball{y}{\delta})\geq
%   L^{-d}\Leb(\ball{z}{\delta}) =   L^{-d}\Leb(\ball{y}{\delta}) \eqsp.
% $$
%  The claim follows from the differentiation theorem
%   for measures, see \cite[Theorem 7.14]{rudin:1987}.
% \end{proof}


% \begin{corollary}
%   \label{coro:irred}
%   Let $x \in \rset^d$. Assume that $\hfunb_x$ is Lipschitz and there
%   exist $y_0,z_0 \in \rset^d$, $M,\tilde{M} \geq 0$ such that $\hfunb_x^{-1}(\ball{y_0}{M}) \subset \ball{z_0}{\tilde{M}}$.
% Then for all $\eventA \in \borelSet(\rset^d)$,
% \begin{equation}
%   \Pkerb(x,\eventA) \geq \norm{\hfunb_x}_{\Lip}^{-d} \inf_{z \in \ball{z_0}{\tilde{M}}} \defEns{\phib(z)} \Leb\defEns{\eventA \cap \hfunb_x(\rset^d) \cap \ball{y_0}{M}} \eqsp.
% \end{equation}
% \end{corollary}

% \begin{proof}
%   By definition, we have using that  $\hfunb_x^{-1}(\ball{y_0}{M}) \subset \ball{z_0}{\tilde{M}}$,
%   \begin{align}
% \nonumber
%       \Pkerb(x,\eventA)  &= \int_{\rset^d}\1_{\eventA}(\hfunb(x,z)) \phib(z) \rmd z  = \int_{\rset^d}\1_{\hfunb_x^{-1}(\eventA)}(z) \phib(z) \rmd z \\
% \label{eq:coro_leb_irred_1}
% &\geq  \inf_{z \in \ball{z_0}{\tilde{M}}} \defEns{\phib(z)} \Leb\defEns{\hfunb_x^{-1}(\eventA \cap \ball{y_0}{M})} \eqsp.
%   \end{align}
% Since by assumption $\hfunb_x$ is Lipschitz, we get using \Cref{le:simple} that
% \begin{equation}
% \label{eq:coro_leb_irred_2}
%  \Leb\defEns{\hfunb_x^{-1}(\eventA \cap \ball{y_0}{M})} \geq  \norm{\hfunb_x}_{\Lip}^{-d}  \Leb\defEns{\eventA \cap \hfunb_x(\rset^d) \cap \ball{y_0}{M}} \eqsp.
% \end{equation}
% Combining \eqref{eq:coro_leb_irred_1} and \eqref{eq:coro_leb_irred_2} concludes the proof.
% \end{proof}

%%% Local Variables:
%%% mode: latex
%%% TeX-master: "main"
%%% End:
