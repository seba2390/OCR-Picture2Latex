\subsection{Proofs of \Cref{sec:geom-ergod-hmc}}

\subsubsection{Proof of \Cref{propo:geo_drift_MH}}
\label{sec:proof-crefpr}
By construction   \eqref{eq:def_kenel_MH}, for all $\q \in \rset^d$, we have
\begin{align}
&\Pker \Vgeo (\q) - \Vgeo(\q) = \int_{\rset^{2d}} \defEns{\Vgeo(\projq(z)) - \Vgeo(\q)} \alphagen(\q,z) \Kker(\q, \rmd z )  \\
& \qquad = \Kker\Vgeo(\q)-\Vgeo(\q) +\int_{\rset^{2d}} \defEns{\Vgeo(\projq(z)) - \Vgeo(\q)} \defEns{\alphagen(\q,z)-1} \Kker(\q, \rmd z ) \eqsp.
\end{align}
Using \eqref{eq:assum:geo_ergo_1}, this implies for all $\q \in \rset^d$,
\begin{equation}
\label{eq:proof_geo_drift_MH_1}
\Pker \Vgeo (\q)  \leq \lambdageo \Vgeo(\q) + b
+ \int_{\rset^{2d}} \defEns{\Vgeo(\projq(z)) - \Vgeo(\q)} \defEns{\alphagen(\q,z)-1} \Kker(\q, \rmd z ) \eqsp.
\end{equation}

Note that by definition \eqref{eq:def_rej_ballV} of $\rejectregion(\q)$ and $\ballV(\q)$
\begin{align}
&\int_{\rset^{2d}} \defEns{\Vgeo(\projq(z)) - \Vgeo(\q)} \defEns{\alphagen(\q,z)-1} \Kker(\q, \rmd z )
\\ &  \qquad  \qquad  \qquad  \leq    \int_{\rejectregion(\q) \cap \ballV(\q) } \defEns{ \Vgeo(\q)-\Vgeo(\projq(z))}  \Kker(\q, \rmd z ) \eqsp.
\end{align}
Therefore by \eqref{eq:assum:geo_ergo_2}, we get
\begin{equation}
 \lim_{M\to \plusinfty} \sup_{\set{\q \in \rset^d}{\Vgeo(\q) \geq M}}  \int_{\rset^{2d}} \left\{\Vgeo(\projq(z))/\Vgeo(\q) -1 \right\} \left\{ \alphagen(\q,z)-1 \right\} \Kker(\q, \rmd z ) \leq 0 \eqsp.
\end{equation}
The proof then follows from combining this result and \eqref{eq:proof_geo_drift_MH_1} since they imply
\begin{equation}
   \lim_{M\to \plusinfty} \sup_{\set{\q \in \rset^d}{\Vgeo(\q) \geq M}}  \Pker \Vgeo (\q) / \Vgeo(\q) \leq \lambda  \eqsp.
\end{equation}

\subsubsection{Proof of \Cref{lem:drift_uhmc}}
\label{sec:proof-crefl-2}


%\begin{proof}[Proof of \Cref{lem:drift_uhmc}]
Let $a \in \rset_+^*$. Under \Cref{assum:regOne}$(m-1)$ with $m \in \ocint{1,2}$,   \Cref{lem:bound_first_iterate_leapfrog_a} shows that, for all $\q_0 \in \rset^d$,
$\p \mapsto \Phiverletq[h][T](\q_0,\p)$ is Lipschitz, with a Lipschitz constant $L_{h,T} \in \rset_+$
\begin{equation}
\label{eq:definition-lipshitz}
L_{h,T} \eqdef \defEns{1+h \constzero^{1/2} \vartheta_1(h \constzero^{1/2})}^{T} \eqsp.
\end{equation}
Therefore by the log-Sobolev inequality \cite[Proposition 5.5.1, (5.4.1)]{bakry:gentil:ledoux:2014} and \eqref{eq:def_Pker_proposition_double}, we get for all $\q_0 \in \rset^d$
\begin{equation}
\PkerhmcD[h][T] \Vdrifta[\a](\q_0) \leq \exp\parenthese{(aL_{h,T})^2/2 + a \Eproof[h][T](\q_0)} \eqsp,
\end{equation}
with
\begin{equation}
\Eproof(\q_0) = (2\uppi)^{-d/2} \int_{\rset^d} \norm{\Phiverletq[h][T](\q_0,\p)} \rme^{-\norm{\p}^2/2} \rmd \p \eqsp.
\end{equation}
Set $p_0 \in \rset^d$.
Denote for all $k \in \{0,\ldots,T\}$, $q_k =
\Phiverletq[h][k](\q_0,\p_0)$ and consider the following decomposition given by  \eqref{eq:qk}:
\begin{equation}
\label{eq:drift_uhmc_3}
\norm{\q_T}^2  =  \norm{\q_0}^2 + \operatorname{A}^{(1)}_{h,T}(\q_0,\p_0) -2h^2 \operatorname{A}^{(2)}_{h,T}(\q_0,\p_0) \eqsp,
\end{equation}
where
\begin{align}
\operatorname{A}^{(1)}_{h,T}(\q_0,\p_0) & = 2Th \ps{\q_0}{ \p_0} + \norm{ Th\p_0-(Th^2/2) \nabla \F(\q_0)-h^2 \sum_{i=1}^{T-1}(T-i)\nabla \F (\q_i)}^2 \\
\operatorname{A}^{(2)}_{h,T}(\q_0, \p_0) & = \ps{\q_0}{ (T/2) \nabla \F(\q_0)+ \sum_{i=1}^{T-1}(T-i)\nabla \F (\q_i)} \eqsp.
\end{align}
Jensen's inequality shows that, for all $\q_0 \in \rset^d$,
\[
\Eproof[h][T](\q_0) \leq \left( \norm{\q_0}^2 + \bar{\operatorname{A}}^{(1)}_{h,T}(\q_0) - 2 h^2 \bar{\operatorname{A}}^{(2)}_{h,T}(\q_0) \right)^{1/2} \eqsp,
\]
where we have set $\bar{\operatorname{A}}_{h,T}^{(i)}(\q_0)= (2\uppi)^{-d/2} \int_{\rset^d} \operatorname{A}_{h,T}^{(i)}(\q_0,\p) \rme^{-\norm{\p}^2/2} \rmd \p$, $i=1,2$.
Therefore to conclude the proof, it is sufficient to show that
\begin{equation}
\label{eq:drift_uhmc_minus1}
\limsup_{\norm{\q_0} \to \plusinfty} \defEnsLigne{\Eproof(\q_0) - \norm{\q_0}} = - \infty.
\end{equation}
\begin{enumerate}[label=(\alph*),leftmargin=0cm,itemindent=0.5cm,labelwidth=1.2\itemindent,labelsep=0cm,align=left]
\item Consider the case $m \in \ooint{1,2}$. Using \Cref{assum:regOne}$(m-1)$ and  \Cref{lem:bound_first_iterate_leapfrog_b}-\ref{lem:bound_first_iterate_leapfrog_1}, we get that  there exists a constant $C_0 \geq 0$ such that for all $\p_0,\q_0 \in \rset^d$ and $i \in \{1,\dots,T-1\}$,
  \begin{equation}
    \label{eq:drift_nabla_U_q_i}
    \norm{\nabla \F(\q_i)} \leq C_0 \{1 + \norm{\p_0} + \norm{\q_0}^{m-1}\}
  \end{equation}
  which implies that
\begin{equation}
\label{eq:bound-A-1}
|\bar{A}^{(1)}_{h,T}(\q_0)| \leq C_1 \{1 + \norm{\q_0}^{2(m-1)} \} \eqsp,
\end{equation}
for some constant $C_1 \geq 0$.  On the other hand, note that for any $q_0,p_0 \in \rset^d$,  $\operatorname{A}^{(2)}_{h,T}(\q_0,\p_0) =  \operatorname{A}^{(2,1)}_{h,T}(\q_0,\p_0) +  \operatorname{A}^{(2,2)}_{h,T}(\q_0,\p_0)$ with
\begin{align}
\label{eq:definition-A-2-1}
\operatorname{A}^{(2,1)}_{h,T}(\q_0,\p_0) &= \frac{T}{2} \ps{\q_0}{ \nabla \F(\q_0)}+\sum_{i=1}^{T-1}(T-i)  \ps{\q_i}{\nabla \F (\q_i)}, \\ 
\label{eq:definition-A-2-2}
\operatorname{A}^{(2,2)}_{h,T} &=- \sum_{i=1}^{T-1}(T-i)  \ps{\q_0-\q_i}{\nabla \F (\q_i)} \eqsp.
\end{align}
Under \Cref{assum:potential:c}$(m)$, for any $q_0, p_0 \in \rset^d$, we have that
\begin{equation}
\label{eq:lower-bound-A-2-1}
\operatorname{A}_{h,T}^{(2,1)}(\q_0,\p_0) \geq \constthree \frac{T}{2} \norm{\q_0}^m - \frac{T (T-1)}{2} \constfour \eqsp.
\end{equation}
Further, by \eqref{eq:drift_nabla_U_q_i} and  \Cref{lem:bound_first_iterate_leapfrog_b}-\ref{lem:bound_first_iterate_leapfrog_1},  there exists  $C_2 \geq 0$, such that for all $\p_0,\q_0 \in \rset^d$,
\begin{equation}
\label{eq:bound-A-2-2}
|\operatorname{A}^{(2,2)}_{h,T}(\q_0,\p_0)| \leq C_2 \{1 + \norm{p_0}^2 + \norm{\q_0}^{2(m-1)} \} \eqsp,
\end{equation}
Combining \eqref{eq:lower-bound-A-2-1} and \eqref{eq:bound-A-2-2}, there exists  $C_3 \geq 0$ such that for any $q_0 \in \rset^d$,
\begin{equation}
\label{eq:bound-A-2}
\bar{\operatorname{A}}^{(2)}(\q_0) \geq \frac{T \constthree}{2} \norm{\q_0}^m - C_3 \{1 + \norm{\q_0}^{2(m-1)} \} \eqsp.
\end{equation}
Combining \eqref{eq:bound-A-1} and \eqref{eq:bound-A-2}, and using that $m < 2$, we finally obtain that \eqref{eq:drift_uhmc_minus1} holds.
% , as $\norm{\q_0} \to \infty$,
% \[
% \Eproof(\q_0) - \norm{\q_0} \leq - \constthree h^2 T^2 \norm{\q_0}^{m-1} + o(\norm{\q_0}^{m-1})
% \]
\item  By Cauchy-Schwarz and Hölder inequality and since $\nabla U$ satisfies  \Cref{assum:regOne}$(1)$, we have for any $q_0,p_0 \in \rset^d$,
\begin{align}
&\operatorname{A}^{(1)}_{h,T}(\q_0,\p_0)
\leq 2hT \norm{q_0} \norm{p_0} \\
& +3 \parentheseDeux{ h^2 T^2  \norm[2]{\p_0} +  2 h^4 T^4 \constzeroT^2 (1+ \norm[2]{ \q_0}) +   2 h^4 T^2  \constzero^2  \defEns{ \sum_{i=1}^{T-1} \norm{\q_i - q_0}}^2} \eqsp,
\end{align}
which implies using \Cref{lem:bounded_cum_error}, $\vartheta_1(s) \geq 1$ for any $s \geq 0$, and the dominated convergence theorem that
\begin{align}
\label{eq:bound-A-1-m=2}
&\limsup_{\norm{q_0} \to \plusinfty} |\bar{\operatorname{A}}^{(1)}_{h,T}(\q_0)|/ \norm[2]{q_0}  \\
&\qquad \qquad \leq
6 h^4 T^4 \left(  \constzeroT^2 +  \constzero \vartheta_2^2(h) \left[ \{1 + h \constzero^{1/2} \vartheta_1(\constzero^{1/2} h)\}^T -1 \right]^2 \right)  \eqsp.
\end{align}
% =======
% which implies, using \Cref{lem:bounded_cum_error} that, as $\norm{\q_0}` \to \infty$,
% \begin{equation}
% \label{eq:bound-A-1-m=2}
% |\bar{\operatorname{A}}^{(1)}_{h,T}(\q_0)| \leq
% h^4 T^4 \left( (3/4) \constzeroT^2 + 3 \constzero \vartheta_2^2(h) \left[ \{1 + h \constzero^{1/2} \vartheta_1(\constzero^{1/2} h)\}^T -1 \right] \right) \norm[2]{\q_0} + o(\norm{\q_0}) \eqsp.
% >>>>>>> 42af951f106d06daaead79ba0820f840f21e5191
% \end{equation}
Similarly using in addition  \Cref{assum:potential:c}($2$), we get that for any $q_0,p_0 \in \rset^d$,
\begin{align}
\operatorname{A}^{(2)}_{h,T}(\q_0,\p_0)
&= \ps{\q_0}{ (T^2/2)  \nabla \F(\q_0)+ \sum_{i=1}^{T-1}(T-i)\{\nabla \F (\q_i) - \nabla U(q_0)\}}  \\
&\geq  (T^2/2) \{\constthree  \norm[2]{q_0} - \constfour\} - T \constzero \norm{q_0} \sum_{i=1}^{T-1}\norm{ q_i - q_0 }  \eqsp.
\end{align}
Then, \Cref{lem:bounded_cum_error} and the Fatou Lemma imply that
\begin{align}
& \liminf_{\norm{q_0} \to \plusinfty} h^2 \bar{\operatorname{A}}^{(2)}_{h,T}(\q_0)/\norm{q_0}^2  \\
 & \qquad \qquad  \qquad  \geq
h^2 T^2  \left( \constthree/2 - \constzero^{1/2} \vartheta_2(h) [(1+ h \constzero^{1/2}\vartheta_1(h \constzero^{1/2}))^T-1]\right)  \eqsp.
\end{align}
Therefore, for all  $h > 0$, and $T \in \nset^*$, one obtains
\[
\limsup_{\norm{q_0} \to \plusinfty} \{ \Eproof[h][T](\q_0) \}^2  /\norm[2]{q_0}   \leq 1 -  T^2 h^2 (\constthree -  \Theta(hT))  \eqsp,
\]
where $\Theta$ is defined in \eqref{eq:definition-function-C}. The proof follows.
\end{enumerate}
%\end{proof}


\subsubsection{Proof of \Cref{le:convex}}
\label{sec:proof-crefle:convex}

Note that condition~\ref{le:convex:a}  implies that
\begin{equation}
  \label{eq:le:convex:b:conseq}
  \inf_{\set{\q}{\norm{\q}=\Rexp}} \F_0(\q) >0 \eqsp.
\end{equation}
Condition \Cref{assum:potential}-\ref{assum:potential:a} follows from \ref{le:convex:b} using that, by \ref{le:convex:a},
 for all $\q \in \rset^d$, $\norm{q} \geq \Rexp$
\[
\F_0(q) =  (\norm{q}/\Rexp)^{\m}\F_0 (\Rexp  q / \norm{q} ) 
\]

  In addition, \Cref{assum:potential:c} is also
  easy to check using the Euler's homogeneous function theorem that
$\ps{\nabla \F_0(\q)}{
 \q}= \m \F_0(\q)$ for all $\q \in \rset^d$, $\norm{\q} \geq \Rexp$.
  % Estimates \Cref{assum:potential}-\ref{assum:potential:a} follow for
  % large values of $\norm{\q}$ just by the stipulated $\m$-homogeneity of
  % $F_0$ and the assumed growth of the derivatives of $G$.
  % Let $U=S^{\circ}$ and $0<t<1$. We have $tS+(1−t)S \subset S$ by convexity, so
  % $tU+(1−t)U\subset S$. But $tU$ is open, so $tU+(1−t)U$ as well. Therefore, $tU+(1−t)U\subset S^{\circ}=U$, and

  % hence $U$ is convex.
  We show below that \Cref{assum:potential}-\ref{assum:potential:b}
  holds. First, since $\lim_{\norm{\q} \to \plusinfty} \F_0(\q) = \plusinfty$ and $\F_0$ is continuous  for all $K \geq 0$, $\lset_K=\{ \q \in \rset^d \ ; \ \F_0(\q) \leq K\}$
%     \begin{equation}
% %    \label{eq:def_lset_le:convex}
% \lset_K=\{ \q \in \rset^d \ ; \ \F_0(\q) \leq K\}
%   \end{equation}
 is compact. Besides, using \eqref{eq:le:convex:b:conseq} and that $\F_0$ is continuous,  we can define
  \begin{equation}
    \label{eq:def_M_le:convex}
M =  \sup_{\q \in \ball{0}{\Rexp}} \F_0(\q) +1 \in \ooint{1, \plusinfty}\eqsp,
  \end{equation}
and for all $\q \not \in \lset_M$,
\begin{equation}
  \label{eq:deftq_le:convex_0}
 t_q = \sup \defEns{ t \in \ccint{0,1} \ ; \ \F_0(t \q ) =M } \eqsp,
\end{equation}
%$t_q = \sup \defEns{ t \in \ccint{0,1} \ ; \ \F_0(t \q ) =M }$,
 which satisfies
% $\F_0(q) >M > \sup_{\q \in
%     \ball{0}{\Rexp}} \F_0(\q)$ therefore $\norm{\q} \geq \Rexp$ and
%  by continuity of $\F_0$, there exists $t_{\q} \in \ccint{0,1}$ such that
  \begin{equation}
    \label{eq:deftq_le:convex}
    \F_0(t_{\q} \q) = M > \sup_{\x \in \ball{0}{\Rexp}} \F_0(\x) \eqsp, \eqsp t_{\q} q \in \partial \lset_M \eqsp \text{ and } \eqsp t_{\q}
  \norm{\q} > \Rexp \eqsp.
  \end{equation}
% $\F_0(t_{\q} \q) = M > \sup_{\x \in \ball{0}{\Rexp}} \F_0(\x)$ and $t_{\q}
%   \norm{\q} \geq \Rexp$.
 Finally using \ref{le:convex:a}, we get that the set
  $\lset_M$  is
  convex.
 % In addition, since $\F_0$ is continuous $\lset_M$
 %  contains the origin in its interior and $\partial \lset_M\subset\{\q
 %  \in \rset^d \, ; \, \F_0(\q)=M\}$.

  To show \Cref{assum:potential}-\ref{assum:potential:b}, we check first
  that it is sufficient to prove that
  \begin{equation}
D^2\F_0(\x)\defEnsLigne{\nabla
    \F_0(\x)\otimes \nabla \F_0(\x)}>0 \text{ for any } \x \in \partial
  \lset_M \eqsp.
  \end{equation}
% $D^2\F_0(\x)\defEnsLigne{\nabla
%     \F_0(\x)\otimes \nabla \F_0(\x)}>0$ for any $\x \in \partial
%   \lset_M$.
 Indeed note that if this statement holds, since $\F \in C^2(\rset^d)$
  and $\partial \lset_M$ is compact, we have
  \begin{equation}
    \label{eq:le:convex:1}
\varepsilon =  \inf_{x \in \partial \lset_M} D^2\F_0(\x)\defEnsLigne{\nabla
    \F_0(\x)\otimes \nabla \F_0(\x)}>0 \eqsp.
  \end{equation}
  % Note that for all $\q \not \in \lset_M$, $\F_0(q) >M > \sup_{\q \in
  %   \ball{0}{\Rexp}} \F_0(\q)$ therefore $\norm{\q} \geq \Rexp$.  In
  % addition since $0$ is in the interior of $\lset_M$, $\F_0(0)< M$
  % define for all $\q \not \in \lset_M$, $\phi(\q) \geq 1$ such that
  % $\q = \phi(\q) \x$ for $\x \in \partial \lset_M$.
Let now $\q \not \in \lset_M$ and $t_{\q}$ defined by \eqref{eq:deftq_le:convex_0}.
Since by \ref{le:convex:a}, for all $u
  \geq 1$ and $z \in \rset^d$, $\norm{z} \geq \Rexp$, $\F_0(uz) = u^\m \F_0(z)$,
  differentiating with respect to $z$, we get $\nabla \F_0(uz) = u^{\m-1}
  \nabla \F_0(z)$ and $D^2 \F_0(uz) = u^{\m-2} D^2\F_0(z)$. Therefore by \eqref{eq:deftq_le:convex}, we get
 \begin{equation}
    \label{eq:le:convex:2}
   D^2\F_0(\q)\defEns{\nabla
  \F_0(\q)\otimes \nabla \F_0(\q)} = t_{\q}^{4-3m} D^2\F_0(t_{\q} \q)\defEns{\nabla
  \F_0(t_{\q} \q )\otimes \nabla  \F_0(t_{\q} \q)} \eqsp.
 \end{equation}
Using \eqref{eq:deftq_le:convex} again and since $\partial \lset_M$ is compact, we get that
there exists $R_2 \geq 0$ such that $t_{\q} \norm{q} \in \ccint{\Rexp,R_2}$. Hence by \eqref{eq:le:convex:2}, we have
\begin{equation}
   D^2\F_0(\q)\defEns{\nabla
  \F_0(\q)\otimes \nabla \F_0(\q)} \geq \varepsilon \norm{q}^{3m-4} \min\parentheseDeux{\Rexp^{4-3m},R_2^{4-3m}} \eqsp.
\end{equation}
Thus
 \Cref{assum:potential}-\ref{assum:potential:b} holds for $\F_0$.
Finally \ref{le:convex:b}
 implies that  the function $\F=\F_0+G$ satisfies
 \Cref{assum:potential}-\ref{assum:potential:b} as well.


 Let $ \x \in \partial \lset_M$, we now show that $D^2\F_0(\x)\defEns{\nabla
 \F_0(\x)\otimes \nabla \F_0(\x)}>0$. By Euler's homogeneous function theorem and since $M \geq 1$, we have that
 $\abs{\ps{\nabla \F_0(\x)}{\x}}\geq \m >0$.  Denote by $\Pi$ the tangent hyperplane
 of $\partial\lset_M$ at $\x$, defined by $\Pi = \{ \q \in \rset^d \, :
 \, \ps{\nabla \F_0(x) }{ \x-\q} = 0\}$.  Since $\lset_M$ is convex, for all $\q \in
 \lset_M$ and $t \in \ccint{0,1}$, $ t^{-1}( \F_0(t\q +(1-t)\x)
 -\F_0(\x))\leq 0$. So taking the limit as $t$ goes to $0$, we get that
 $\ps{\nabla \F_0(\x) }{ \q-\x} \leq 0$. Therefore, $\lset_M$ is
 contained in the half-space $\Pi^- = \{ q \in \rset^d \, ; \, \ps{\nabla
 \F_0(\x) }{ \q-\x} \leq 0 \}$.

Define the  $\m$-homogeneous
 function $\tilde{\F} : \rset^d \to \rset_+$  for all $\q \in \rset^d$  by
 \begin{equation}
\label{eq:def:tilde_F}
   \tilde{\F}(\q) = M \abs{\frac{\ps{\q}{ \nabla \F_0(\x)}}{\ps{\x}{ \nabla \F_0(\x)}}}^\m \eqsp.
 \end{equation}
 Since $\F_0(x) = M$, by \eqref{eq:def_M_le:convex}, $\norm{x} > \Rexp$
 and therefore there exists $\epsilon_0 \in \rset_+^*$ such that
 \begin{equation}
\label{eq:inclusion_ball}
   \ball{x}{\epsilon_0} \subset \rset^d \setminus \ball{0}{\Rexp}\eqsp.
 \end{equation}
 We
 now show that $\tilde{\F}(\q) \leq \F_0(\q)$ for all $\q \in
 \ball{\x}{\epsilon}$ with
 \begin{equation}
\epsilon = 2^{-1}\min\parentheseDeux{\epsilon_0, \{\ps{\x}{ \nabla \F_0(\x)}\}/\norm[2]{\nabla \F_0(\x)}} \eqsp.
 \end{equation}
  First consider $\q \in \Pi$. We next argue by contradiction that
 \begin{equation}
\label{eq:bound_Pi}
   \F_0(\q) \geq M = \tilde{\F}(\q)    \eqsp.
 \end{equation}
Indeed assume that  $\F_0(\q) < M$. Then by continuity of $\F_0$, we get that $\q \in \interior{\lset_{M}}$. But since $\lset_{M} \subset \Pi^-$, we get $\q \in \interior{(\Pi^-)}$ which is impossible since $\q \in \Pi = \boundary{\Pi^-} = \clos{\Pi^-} \setminus \interior{(\Pi^-)}$.

Let $\q \in
 \ball{\x}{\epsilon}$. Note that  $  \q= \x + \norm[-2]{\nabla \F_0(\x)}\ps{\q-\x}{\nabla \F_0 (\x)}\nabla \F_0(\x) + z$,
% \begin{equation}
%   \q= \x + t\nabla \F_0(\x) + z \eqsp,
% \end{equation}
where  $z \in \rset^d$ is orthogonal to $\nabla \F_0(\x)$. Define
\begin{equation}
  u = \frac{\ps{\x }{ \nabla \F_0(\x)}}{\ps{\q}{ \nabla \F_0(\x)}}\eqsp.
\end{equation}
Then $u\q \in \Pi$ and by \eqref{eq:bound_Pi}, $\F_0(u \q) \geq M$. If $u \geq 1$, using \ref{le:convex:a} and \eqref{eq:def:tilde_F}, we get
\begin{equation}
\label{eq:6}
  \F_0(\q) \geq  u^{-\m}M = \tilde{\F}(\q) \eqsp.
\end{equation}
In turn, if $u < 1$, since $\norm{\q- \x} \leq \epsilon_0$, by \eqref{eq:inclusion_ball} and \ref{le:convex:a}, $\F_0(\q) = u^{-1} \F_0(u \q)$ and \eqref{eq:6} still holds.

Consider the three times differentiable functions $\phi$ and $\tilde{\phi}$
defined for all $v \in \rset$ by
$$
\phi(v)=\F_0(\x+v \nabla \F_0(\x))\quad\textrm{and}\quad \tilde{\phi}(v)= \tilde{\F}(\x+v \nabla \F_0(\x)) \eqsp.
$$
First, since for all $\q \in \ball{\x}{\epsilon}$, $\F_0(\q) \geq \tilde{\F}(\q)$, we have
\begin{equation}
  \label{eq:bound_f_tildef}
  \phi(v) \geq \tilde{\phi}(v) \eqsp, \text{for all $v \in \ccint{-\epsilon/\norm{\nabla \F_0(\x)},\epsilon/\norm{\nabla \F_0(\x)}}$}\eqsp.
\end{equation}
Moreover, by definition $\F_0(\x) = \tilde{\F}(\x)$ and $\nabla \tilde{\F}(\x)$
is colinear to $ \nabla \F_0(\x)$. Using  Euler's homogeneous function theorem for $\F_0$
and $\tilde{\F}$, we get that $\nabla \tilde{\F}(\x) = \nabla
\F_0(\x)$. Therefore $\phi(0) = \tilde{\phi}(0)$, $\phi'(0) = \tilde{\phi}'(0)$. Combining these equalities, \eqref{eq:bound_f_tildef} and using a Taylor expansion around $0$ of order $2$ with exact remainder for $\phi$ and $\tilde{\phi}$ shows that necessary
\begin{equation}
  D^2\F_0(\x)\defEns{\nabla
  \F_0(\x)\otimes \nabla \F_0(\x)} =   \phi''(0) \geq \tilde{\phi}''(0) >0 \eqsp,
\end{equation}
which concludes the proof.
% Secondly, note that since $\F_0$ is continuous and
%   $\lim_{\norm{\q} \to \plusinfty} \F_0(\q) = \plusinfty$, by
%   \Cref{assum:potential_second}-\ref{le:convex:a}, there exists $M
%   \geq \Rexp+1$ such that the sublevel set $\lset_M:=\{\q \in \rset^d
%   \, ; \, \F_0(\q) \leq M \}$ is a compact convex set and contains the
%   origin in its interior. Moreover, $\partial \lset_M=\{\q \in \rset^d
%   \, ; \, \F_0(\q)=M\}$. To show
%   \Cref{assum:potential}-\ref{assum:potential:b}, we check that it is
%   sufficient to prove that $D^2\F_0(\x)\defEnsLigne{\nabla
%     \F_0(\x)\otimes \nabla \F_0(\x)}>0$ for any $\x \in \partial
%   \lset_M$. Indeed if this statement holds, since $\F \in C^3(\rset^d)$
%   and $\partial \lset_M$ is compact, there exists $\varepsilon >0$
%   such that
%   \begin{equation}
%     \label{eq:le:convex:1}
%  \inf_{x \in \partial \lset_M} D^2\F_0(\x)\defEnsLigne{\nabla
%     \F_0(\x)\otimes \nabla \F_0(\x)}>\varepsilon \eqsp.
%   \end{equation}
%   In addition since $0$ is in the interior of $\lset_M$, $\F_0(0)< M$  define for all
%   $\q \not \in \lset_M$, $\phi(\q) \geq 1$ such that $\q = \phi(\q)
%   \x$ for $\x \in \partial \lset_M$. Since for all $u \geq 0$ and $z
%   \in \rset^d$, $z \geq \Rexp$, $\F(uz) = u^\m \F(z)$, differentiating
%   with respect to $u$, we get $\nabla \F(uz) = u^{\m-1} \nabla \F(z)$
%   and $D^2 \F(uz) = u^{\m-2} D^2\F(z)$. Therefore,
%  \begin{equation}
%    D^2\F_0(\q)\defEns{\nabla
%   \F_0(\q)\otimes \nabla \F_0(\q)}> \phi(\q)^{3\m-4} D^2\F_0(\x)\defEns{\nabla
%   \F_0(\x)\otimes \nabla  \F_0(\x)} \eqsp.
%  \end{equation}
% %phi(\q) = norm{\q}/norm{\x}
%  Since $\lset_M$ is compact, there exists $C \geq 0$ such that for
%  all $\q \not \in \lset_M$, $\phi(\q) \geq C \norm{\q}$ and therefore
%  \Cref{assum:potential}-\ref{assum:potential:b} holds for $\F_0$. \Finally, the assumed growth of the derivatives of $G$ then
%  implies that  the function $\F=\F_0+G$ satisfies
%  \Cref{assum:potential}-\ref{assum:potential:b} as well.

%  Let $ \x \in \partial \lset_M$, we now show that $D^2\F_0(\x)\defEns{\nabla
%  \F_0(\x)\otimes \nabla \F_0(\x)}>0$. By the Euler relation $\ps{\nabla \F(\q)}{
%  \q}= \m \F(\q)$ for all $\q \geq \Rexp$, and since $M \geq 1$, we have that
%  $\norm{\nabla \F(\x)}\geq \m >0$.  Denote by $\Pi$ the tangent hyperplane
%  of $\partial\lset_M$ at $\x$, defined by $\Pi = \{ z \in \rset^d \, :
%  \, \ps{\nabla \F_0 }{ \x-\q} = 0\}$.  By our assumption, for all $\q \in
%  \lset_M$ and $t \in \ccint{0,1}$, $ t^{-1}( \F_0(t\q +(1-t)\x)
%  -\F(\x))\leq 0$ So taking the limit as $t$ goes to $0$, we get that
%  $\ps{\nabla \F_0(\x) }{ \q-\x} \leq 0$. Therefore, $\lset_M$ is
%  contained in the half-space $\Pi^- = \{ z \in \rset^d \, ; \, \ps{\nabla
%  \F_0(\x) }{ \q-\x} \leq 0 \}$. Define the function $\m$-homogeneous
%  function $\tilde{\F} : \rset^d \to \rset_+$ by for all $\q \in \rset^d$
%  \begin{equation}
% \label{eq:def:tilde_\F}
%    \tilde{\F}(\q) = M \norm{\frac{\ps{\q}{ \nabla \F_0(\x)}}{\ps{\x}{ \nabla \F_0(\x)}}}^\m \eqsp.
%  \end{equation}

%  We show that $\tilde{\F}(\q) \leq \F_0(\q)$ for all $\q \in
%  \ball{\x}{\epsilon}$ for
%  \begin{equation}
% \epsilon = 2^{-1}\min\parentheseDeux{1, \{\ps{\x}{ \nabla \F_0(\x)}\}/\norm[2]{\nabla \F_0(\x)}} \eqsp.
%  \end{equation}
%   \First note that for all $\q \in \Pi$, since $\lset_M
%  \subset \Pi^- $, then
%  \begin{equation}
% \label{eq:bound_Pi}
%    \F_0(\q) \geq M = \tilde{\F}(\q)    \eqsp.
%  \end{equation}
%  Now let $\q \in
%  \ball{\x}{1/2}$, note that $\q$ can be written in the form $  \q= \x + t\nabla \F_0(\x) + z$,
% % \begin{equation}
% %   \q= \x + t\nabla \F_0(\x) + z \eqsp,
% % \end{equation}
% where $t \in \ccint{-\epsilon,\epsilon}$ and $z \in \rset^d$ is orthogonal to $\nabla \F_0(\x)$. Define
% \begin{equation}
%   u = \frac{\ps{\q }{ \nabla \F_0(\x)}}{\ps{\x}{ \nabla \F_0(\x)} + t \norm[2]{\nabla \F_0(\x)}}\eqsp.
% \end{equation}
% Then $ux \in \Pi$ and by \eqref{eq:bound_Pi}, $\F_0(\q) \geq M$. Using that $\F_0$ is $\m$-homogeneous,
% \begin{equation}
%   \F_0(\q) \geq  \norm{u}^{-\m}M = \tilde{\F}(\q) \eqsp.
% \end{equation}

% Consider the three times differentiable functions $f$ and $\tilde{f}$
% defined for all $v \in \ccint{-\epsilon,\epsilon}$ by
% $$
% f(v):=\F_0(\x+v \nabla \F_0(\x))\quad\textrm{and}\quad \tilde{f}(v)= \tilde{\F}(\x+v \nabla \F_0(\x)) \eqsp.
% $$
% \First, since for all $\q \in \ball{\x}{\epsilon}$, $\F_0(\q) \geq \tilde{\F}(\q)$, we have for all $v \in \ccint{-\epsilon,\epsilon}$,
% \begin{equation}
%   \label{eq:bound_f_tildef}
%   f(v) \geq \tilde{f}(v) \eqsp.
% \end{equation}
% Moreover, by definition $\F_0(\x) = \tilde{\F}(\x)$, $\nabla \tilde{\F}(\x)$
% is parallel to $ \nabla \F_0(\x)$. Using the Euler identity for $\F_0$
% and $\tilde{\F}$, we get that $\nabla \tilde{\F}(\x) = \nabla
% \F_0(\x)$. Therefore $f(0) = \tilde{f}(0)$, $f'(0) = \tilde{f}'(0)$. Combining these inequalities, \eqref{eq:bound_f_tildef} and using a Taylor expansion around $0$ of order $3$ for $f$ and $\tilde{f}$ shows that necessary
% \begin{equation}
%   D^2\F_0(\x)\defEns{\nabla
%   \F_0(\x)\otimes \nabla \F_0(\x)} =   f''(0) \geq \tilde{f}(0) >0 \eqsp,
% \end{equation}
% which concludes the proof.


\subsubsection{Proof of \Cref  {propo:accept}}
\label{sec:proof-crefth}
We preface the proof by several technical preliminary Lemmas.


%We preface the proof by a useful Lemma.
\begin{lemma}
\label{lem:grad_Lip_F}
Assume \Cref{assum:potential}($\m$)-\ref{assum:potential:a} for some $m\in \ocint{1,2}$.
Then, for all $q,x \in \rset^d$,
$\norm{\nabla \F(\q) - \nabla \F(\x)} \leq \constone  \norm{\q-\x}$
and $\norm{\nabla \F(\q) - \nabla \F(\x)} \leq \constone (m-1)^{-1} \norm{\q-\x}^{m-1}$.
In particular, \Cref{assum:regOne}($m-1$) holds with $\constzero= \constone$ and $\constzeroT= \constone (m-1)^{-1} \vee \norm{\nabla \F(0)}$.
\end{lemma}


\begin{proof}
First by \Cref{assum:potential}($m$)-\ref{assum:potential:a}, we get for all $\q,\x \in \rset^d$,
\begin{align}
\nonumber
  \norm{\nabla \F(\q) - \nabla \F(\x)}
&= \norm{\int_{0}^1 \nabla^2 \F(\x +t(\q-\x)) \defEns{\q-\x} \rmd t } \\
\label{eq:lem_grad_Lip_F_eq_0}
&\leq \constone  \norm{\q-\x} \int_{0}^1 \defEns{1+\norm{\x +t(\q-\x)}}^{\m-2} \rmd t   \eqsp.
\end{align}
Therefore, for all $\q,\x \in \rset^d$, we get $ \normLigne{\nabla \F(\q) - \nabla \F(\x)} \leq \constone \normLigne{\q-\x}$. For all $\q, \x \in \rset^d$, since $m \in \ocint{1,2}$, we have
\begin{align}
&\int_{0}^1 \defEns{1+\norm{\x +t(\q-\x)}}^{\m-2} \rmd t  \leq  \int_{0}^1\defEns{ 1+\abs{\norm{\x} - t \norm{\q-\x}}}^{m-2}  \rmd t  \\
& \leq \int_{0}^{1 \wedge \frac{\norm{\x}}{\norm{\q-\x}}}\defEns{1+  \norm{\x} - t \norm{\q-\x}}^{m-2}  \rmd t
+ \int_{1 \wedge \frac{\norm{\x}}{\norm{\q-\x}}}^1 \defEns{1+ t \norm{\q-\x}-\norm{\x}}^{m-2}  \rmd t \\
&\leq (m-1)^{-1} \norm{\q-\x}^{m-2} \eqsp.
\end{align}
Plugging this result in \eqref{eq:lem_grad_Lip_F_eq_0} concludes the proof.

% \begin{align}
% &   \int_{0}^1 \defEns{1+\norm{\x +t(\q-\x)}}^{\m-2} \rmd t  \leq  \int_{0}^1\defEns{ 1+\abs{\norm{\x} - t \norm{\q-\x}}}^{m-2}  \rmd t  \\
% &\leq \int_{0}^{1 \wedge \frac{\norm{\x}}{\norm{\q-\x}}}\defEns{1+  \norm{\x} - t \norm{\q-\x}}^{m-2}  \rmd t
% \\
% & \phantom{\defEns{1+  \norm{\x} - t \norm{\q-\x}}^{m-2}  \rmd t}+ \int_{1 \wedge \frac{\norm{\x}}{\norm{\q-\x}}}^1 \defEns{1+ t \norm{\q-\x}-\norm{\x}}^{m-2}  \rmd t \eqsp.
% \end{align}
% Then if $\norm{\q-\x} \leq \norm{\x}$, we get
% \begin{multline}
%   \label{eq:lem_grad_Lip_F_eq_1}
%  \int_{0}^1 \defEns{1+\norm{\x +t(\q-\x)}}^{\m-2} \rmd t \leq  \int_{0}^1 \defEns{(1-t)\norm{\q-\x}}^{\m-2} \rmd t \\
% \leq (m-1)^{-1} \norm{\q-\x}^{m-2}\eqsp.
% \end{multline}
% If $\norm{\q-\x} \geq \norm{\x}$, we get
% \begin{align}
% \nonumber
%  &\int_{0}^1 \defEns{1+\norm{\x +t(\q-\x)}}^{\m-2} \rmd t \leq \int_{0}^{1 \wedge \frac{\norm{\x}}{\norm{\q-\x}}}\defEns{1+  \norm{\x} - t \norm{\q-\x}}^{m-2}  \rmd t
% \\
% \nonumber
% & \phantom{\defEns{  \norm{\x} - t \norm{\q-\x}}^{m-2}  \rmd t}+ \int_{1 \wedge \frac{\norm{\x}}{\norm{\q-\x}}}^1 \defEns{ t \norm{\q-\x}-\norm{\x}}^{m-2}  \rmd t \\
%   \label{eq:lem_grad_Lip_F_eq_2}
% & \leq ((m-1)\norm{\q-\x})^{-1} \defEns{\norm{x}^{m-1} + \norm{\q-\x}^{m-1}} \leq (m-1)^{-1} \norm{\q-\x}^{m-2} \eqsp.
% \end{align}
% Combining \eqref{eq:lem_grad_Lip_F_eq_1} and \eqref{eq:lem_grad_Lip_F_eq_2} in \eqref{eq:lem_grad_Lip_F_eq_0} concludes the proof.

% The proof is postponed to \Cref{sec:proof-crefl-1}.
% \end{proof}

%     Let $\q,\x \in \rset^d$ and assume without loss of generality that
%   $\norm{\q} < \norm{\x}$.  First by
%   \Cref{assum:potential}-\ref{assum:potential:a} and Jensen
%   inequality, using that $\m \leq 2$, $t \mapsto t^{\m-2}$ is concave on $\rset_+$,
%   we have
% \begin{align}
% \nonumber
%   \norm{\nabla \F(\q) - \nabla \F(\x)} &= \norm{\int_{0}^1 \nabla^2 \F(\x +t(\q-\x)) \defEns{\q-\x} \rmd t } \\
% \nonumber
% & \leq \constone  \norm{\q-\x} \int_{0}^1 \defEns{1+\norm{\x +t(\q-\x)}}^{\m-2} \rmd t  \\
%   \label{eq:lem_grad_Lip_F_eq_0}
%  & \leq \constone  \norm{\q-\x} \parentheseDeux{ \int_{0}^1 \defEns{1+\norm{\x +t(\q-\x)} }\rmd t }^{\m-2} \eqsp.
% \end{align}
% In addition, we have
% \begin{multline}
% \int_{0}^1 \defEns{1+\norm{\x +t(\q-\x)}} \rmd t \geq \int_{0}^1\defEns{ 1+\abs{\norm{\x} - t \norm{\q-\x}}}  \rmd t  \\
% \geq 1+ \int_{0}^{1 \wedge \frac{\norm{\x}}{\norm{\q-\x}}}\defEns{ \norm{\x} - t \norm{\q-\x}}  \rmd t
% + \int_{1 \wedge \frac{\norm{\x}}{\norm{\q-\x}}}^1 \defEns{ t \norm{\q-\x}-\norm{\x}}  \rmd t \eqsp.
% \end{multline}
% If $\norm{\q-\x} \leq \norm{\x}$, then we get
% \begin{equation}
%   \label{eq:lem_grad_Lip_F_eq_1}
%   \int_{0}^1 \defEns{1+\norm{\x +t(\q-\x)}} \rmd t
% %\geq 1+\norm{\x} - \norm{\q-\x}/2 \\
% \geq 1+\norm{\q-\x}/2 \eqsp.
% \end{equation}
% If $\norm{\q-\x} \geq \norm{\x}$, by the triangle inequality it necessarily  holds that $\norm{\x} \in \ccint{\norm{\q-\x}/2, \norm{\q-\x}}$ since $\norm{\q} \leq \norm{\x}$. Therefore we get
% \begin{equation}
%   \label{eq:lem_grad_Lip_F_eq_2}
%   \int_{0}^1 \defEns{1+\norm{\x +t(\q-\x)}} \rmd t
% %\geq 1+ \int_{0}^{\min(1,\norm{y}/\norm{\q-\x})}\defEns{ \norm{\x} - t \norm{\q-\x}}  \rmd t \\
%  \geq 1+\norm{\x}^2/(2 \norm{\q-\x})
% \geq 1+\norm{\q-\x}/8 \eqsp.
% \end{equation}
% Since $\m \leq 2$, then combining \eqref{eq:lem_grad_Lip_F_eq_1} and \eqref{eq:lem_grad_Lip_F_eq_2} in \eqref{eq:lem_grad_Lip_F_eq_0}, we get
% \begin{equation}
%    \norm{\nabla \F(\q) - \nabla \F(\x)} \leq  \constone  \norm{\q-\x}\defEns{1+ \norm{\q-\x}/8}^{\m-2} \eqsp,
% \end{equation}
% which concludes the proof.
\end{proof}

\begin{lemma}
  \label{lem:prepa_bound_diff_ham}
Assume  \Cref{assum:regOne}$(\beta)$ for $\beta \in \ocint{0,1}$. Let $\gamma \in \ooint{0,\beta}$.
\begin{enumerate}[label=(\roman*)]
\item   \label{lem:prepa_bound_diff_ham_1}
If $ \beta \in \ooint{0,1}$, for any $T \in \nsets$ and  $h_0 \in \rset^*_+$, there exist $\kappa \in \rset^*_+$ and $R \in \rset_+$ such that for all $h \in \ocint{0,h_0}$,  $\q_0,p_0 \in \rset^d$ satisfying $ \norm{p_0} \leq
\norm{\q_0}^{\gamma}$ and $\norm{\q_0} \geq R$, and $i,j,k \in
\{0,\ldots,T\}$,
\begin{align}
\label{eq:bound_iterate_q_2_prood_diff_ham_eq}
%\label{eq:bound_iterate_q_1_prood_diff_ham_1}
&\norm{q_0} \leq \kappa  \norm{\Phiverletq[h][k](q_0,p_0)}\eqsp, \\ &\norm{\Phiverletq[h][i](q_0,p_0)-\Phiverletq[h][j](q_0,p_0)} \leq \kappa h \norm{\Phiverletq[h][k](q_0,p_0)}^{\beta}  \eqsp,
\end{align}
  where $\Phiverletq[h][\ell]$ are defined by \eqref{eq:def_Phiverletq} for $\ell \in \nset^*$.
\item \label{lem:prepa_bound_diff_ham_2}
If $\beta =1$, then there exist $\kappa, \bar{S} \in \rset_+^*$ (depending only on $\constzero$ and $\constzeroT$) such that for any $T \in \nsets$, $h \in \ooint{0,\bar{S}/T}$,  $\q_0,p_0 \in \rset^d$ satisfying $ \norm{p_0} \leq
\norm{\q_0}^{\gamma}$ and $\norm{\q_0} \geq 1$, and $i,j,k \in
\{0,\ldots,T\}$,
\begin{align}
\label{eq:bound_iterate_q_2_prood_diff_ham_eq_2}
&\norm{q_0} \leq 2  \norm{\Phiverletq[h][k](q_0,p_0)} \leq 3 \norm{q_0} \eqsp, \\
&\norm{\Phiverletq[h][i](q_0,p_0)-\Phiverletq[h][j](q_0,p_0)}
\leq \kappa Th
\rme^{(1+\vartheta_1(Th))Th}  \norm{\Phiverletq[h][k](q_0,p_0)} \eqsp,
\end{align}
where $\vartheta_1$ is defined in \eqref{eq:def_vartheta_1}.
\end{enumerate}
\end{lemma}

\begin{proof}
\begin{enumerate}[label=(\roman*),leftmargin=0cm,itemindent=0.5cm,labelwidth=1.2\itemindent,labelsep=0cm,align=left]
\item
Let $T \in \nsets$, $h_0 \in \rset_+^*$ and  $h \in \ocint{0,h_0}$.
Denote for all $k \in \{0,\ldots,T\}$ by $(q_k,p_k) =
  \Phiverlet[h][k](q_0,p_0)$, $q_0, p_0 \in \rset^d$.
  By  \Cref{assum:regOne}$(\beta)$ and  \Cref{lem:bound_first_iterate_leapfrog_b}-\ref{lem:bound_first_iterate_leapfrog_1}, there exist $C \geq 0$ and  $R_1 \geq 0$ such that for all $\q_0,\p_0 \in \rset^d$ satisfying $ \norm{\p_0} \leq \norm{\q_0}^{\gamma}$ and $\norm{\q_0} \geq R_1$, for all $k \in \{0,\ldots,T\}$, we have
\begin{equation}\label{eq:bound_iterate_q_1_prood_diff_ham_0}
 \norm{\q_k-\q_0} \leq C h \norm{\q_0}^{\m-1} \eqsp.
\end{equation}
Then since $m<2$, there exists $R_2 \geq R_1$ and $\omega >0$ such that such that for all $\q_0,p_0 \in \rset^d$ satisfying $ \norm{p_0} \leq \norm{\q_0}^{\gamma}$  and $\norm{\q_0} \geq R_2$, for all $k \in \{0,\ldots,T\}$,
\begin{equation}
  \norm{\q_0} \leq \omega \norm{\q_k} \eqsp.
\end{equation}
In addition, using this inequality and
\eqref{eq:bound_iterate_q_1_prood_diff_ham_0} again, we get that for
all $\q_0,p_0 \in \rset^d$ satisfying $ \norm{p_0} \leq
\norm{\q_0}^{\gamma}$ and $\norm{\q_0} \geq R_2$, for all $i,j,k \in
\{0,\ldots,T\}$,
\begin{equation}
\label{eq:bound_iterate_q_2_prood_diff_ham_2}
\norm{\q_{i}-\q_{j}} \leq 2C h \norm{\q_0}^{\m-1} \leq 2 Ch \omega^{\m-1}\norm{\q_k}^{\m-1}  \eqsp.
\end{equation}
%
\item Let $T \in \nsets$, $h \in \rset_+^*$.
Denote for all $k \in \{0,\ldots,T\}$ by $(q_k,p_k) =
  \Phiverlet[h][k](q_0,p_0)$, $q_0, p_0 \in \rset^d$.
 By  \Cref{assum:regOne}$(1)$ and \Cref{lem:bound_first_iterate_leapfrog_b}-\ref{lem:bound_first_iterate_leapfrog_b_2}, $\vartheta(s) \geq 1$ for any $s \geq 0$, we get that  for all $\q_0, \p_0 \in \rset^{2d}$ satisfying $\norm{\p_0} \leq \norm{\q_0}^\gamma$ and $k \in \{0,\ldots,T\}$,
\begin{equation}\label{eq:bound_iterate_q_1_prood_diff_ham_0_m_2}
 \norm{\q_k-\q_0} \leq \constzero^{-1/2}\left\{(1 + h \constzero^{1/2} \vartheta_1(h \constzero^{1/2}))^{k+1}-1\right\} \left\{ \vartheta_2(h) \vee \vartheta_3(h) \right\} (1 + \norm{\q_0}) \eqsp,
\end{equation}
where $\vartheta_1$, $\vartheta_2$ and $\vartheta_3$ are defined in \eqref{eq:def_vartheta_1} and  \eqref{eq:definition-vartheta-2} respectively.

Therefore,  there exists $\bar{S} >0$ (depending only on $\constzero$ and $\constzeroT$) such that for any $T \in \nsets$ and $h \in \ooint{0,\bar{S}/T}$, for any $q_0,p_0 \in \rset^d$ satisfying $ \norm{\p_0} \leq \norm{\q_0}^{\gamma}$ and $\norm{\q_0} \geq  1 $, $ \normLigne{\Phiverletq[h][k](q_0,p_0)-\q_0} \leq \norm{q_0}/2$ for any $k \in \{0,\ldots,T\}$. As a result, for any $T \in \nsets$ and $h \in \ooint{0,\bar{S}/T}$, for any $q_0,p_0 \in \rset^d$ satisfying $ \norm{\p_0} \leq \norm{\q_0}^{\gamma}$ and $\norm{\q_0} \geq  1 $,  for any $k \in \{0,\ldots,T\}$,
\begin{equation}
  \norm{\q_0} \leq 2 \norm{\Phiverletq[h][k](q_0,p_0)} \leq 3 \norm{q_0} \eqsp.
\end{equation}
In addition, using this inequality and
\eqref{eq:bound_iterate_q_1_prood_diff_ham_0_m_2} again, we get that there exists $C \geq 1$ (depending only on $\constzero$ and $\constzeroT$) such that  for any $T \in \nsets$ and $h \in \ooint{0,\bar{S}/T}$, setting $S= hT$, and  for
all $\q_0,p_0 \in \rset^d$ satisfying $ \norm{p_0} \leq
\norm{\q_0}^{\gamma}$ and $\norm{\q_0} \geq 1$, for all $i,j,k \in
\{0,\ldots,T\}$,
\begin{align}
%\label{eq:bound_iterate_q_2_prood_diff_ham_2}
\norm{\Phiverletq[h][i](q_0,p_0) - \Phiverletq[h][j](q_0,p_0)} 
&\leq 2C S \rme^{(1+\vartheta_1(S))S} \norm{\q_0} \\
&\leq 4 C S \rme^{(1+\vartheta_1(S))S}\norm{\Phiverletq[h][k](q_0,p_0)}  \eqsp.
\end{align}
\end{enumerate}
\end{proof}

\begin{lemma}
\label{lem:variation_assum_hessian}
Assume \Cref{assum:potential}($\m$) for some $m \in \ocint{1,2}$.
Then there exist $\delta \in \ooint{0,1}$, $R_0 \in \rset_+$ and  $\BB_0 \in \rset_+^*$ such that for all $\q,\x,z \in \rset^d$, with
\begin{equation}
\label{eq:hyp_variation_assum_hessian}
\norm{\q} \geq R_0 \eqsp, \qquad \text{ and } \quad \max\parenthese{\norm{\q-\x},\norm{\q-z}} \leq \delta \norm{\q}  \eqsp,
\end{equation}
we have
\begin{equation}
D^2 \F(\q) \defEns{\nabla \F(\x) \otimes \nabla \F(z)} \geq \BB_0 \norm{\q}^{3\m-4}  \eqsp.
\end{equation}
\end{lemma}
\begin{proof}
Under \Cref{assum:potential}($m$), using \Cref{lem:grad_Lip_F}, it can be easily checked that there
exists $C_U \geq 0$ (depending only on $\constone$ and $m$) such that for all  $\q,\x,z \in \rset^d$ satisfying  \eqref{eq:hyp_variation_assum_hessian},
for $\delta \in \ooint{0,1}$ and $R_0 \geq \rhtwo$,
\begin{equation}
D^2 \F(\q) \defEns{\nabla \F(\x) \otimes \nabla \F(z)}  \geq \consttwo  \norm{\q}^{3\m-4} - C_U \{1+ \delta^{m-1} \norm{\q}^{3m-4}\}\eqsp.
\end{equation}
The proof is concluded by taking $\delta$ sufficiently small and $R_0$ sufficiently large.
\end{proof}

% \alain{comment on the Gaussian case the term $\frac{h^4}{8}\norm{ \int_0^1 \nabla ^2 \F( q_{t}) p_{0} \ \rmd t  }^2$ implies that $h$ has to be chosen at least smaller than $C (Ld^{-1/2})^{-1/2}$}
 \begin{lemma}
 \label{lem:diff_hamiltonian_taylor_exp}
 Assume that $\F$ is twice continuously differentiable. Then for all $q_0,p_0 \in \rset^d$ and $h \in \rset^*_+$, the following identity holds
 \begin{align}
%&    \Ham(q_1,p_1) - \Ham(q_0,p_0)   \\
& \Ham \circ \Phiverlet[h][1](q_0,p_0) - \Ham(q_0,p_0) =  h^2\int_0^1D^2\F(q_t)\defEns{p_{0}}^{\otimes 2} (1/2-t) \  \rmd t \\
& + h^3 \int_0^1D^2\F(q_t)\defEns{p_{0}\otimes \nabla \F(q_{0})}(t-1/4) \ \rmd t  \\
&  -\frac{h^4}{4}\int_0^1 D^2\F(q_t)\defEns{\nabla \F(q_{0})}^{\otimes 2} \ t \  \rmd t
+\frac{h^4}{8}\norm{ \int_0^1 \nabla ^2 \F( q_{t}) p_{0} \ \rmd t  }^2\\
&   - \frac{h^5}{8}\ps{\int_0^1\nabla^2\F(q_t)\nabla \F(q_0) \ \rmd t }{\int_0^1 \nabla^2 \F(q_t)p_0 \ \rmd t }
   \\
   &+\frac{h^6}{32}\norm{\int_0^1 \nabla^2 \F( q_{t})\nabla \F (q_0) \ \rmd t }^2
\eqsp,
\end{align}
where  $\Phiverlet[h][1]$ is defined in \eqref{eq:def_Phiverlet}, $(q_1,p_1) = \Phiverlet[h][1](q_0,p_0)$, and $q_t = q_0 +t (q_1-q_0)$ for $t \in \ccint{0,1}$.
\end{lemma}
\begin{proof}
Using the definition of $\Ham(q,p)= \frac{1}{2}\norm{p}^2+ \F(q)$, we get
\begin{equation}
  \Ham(q_1,p_1) - \Ham(q_0,p_0) = (1/2)(\norm{p_1}^2 - \norm{p_0}^2) + \F(q_1) - \F(q_0) \eqsp.
\end{equation}
First, Taylor's formula  with exact remainder  enables us to write
\begin{equation}\label{eq:1}
\F(q_{1})-\F(q_0)= \ps{\nabla \F(q_0)}{ (q_{1}-q_0)}+\int_0^1D^2\F(q_{t})\defEns{q_{1}-q_0 }^{\otimes 2}(1-t) \ \rmd t \eqsp.
\end{equation}
Since  $\nabla \F(q_{1}) = \nabla \F(q_0) + \int_{0}^1 \nabla^2 \F(q_t) \defEns{q_1-q_0} \rmd t $,  we get
 \begin{equation}\label{eq:2}
p_{1}=p_0-\frac{h}{2} \parenthese{\nabla \F(q_{0})+\nabla \F(q_{1})}= p_{0}- h\nabla \F(q_0)-\frac{h}{2}\int_0^1\nabla ^2\F( q_{t}) \defEns{q_{1}-q_{0}} \rmd t \eqsp.
\end{equation}
Using that $q_1 = \Phiverletq[h][1](p_0,q_0)$, with $\Phiverletq[h][1]$ defined by \eqref{eq:def_Phiverletq}, in \eqref{eq:1} and \eqref{eq:2}, we get
\begin{align}
  &\F(q_{1})-\F(q_{0})\\
&=\ps{\nabla \F(q_{0})}{ hp_{0}-(h^2/2)\nabla \F(q_{0})} + \int_0^1D^2\F(q_t)\defEns{q_{1}-q_{0}}^{\otimes 2}(1-t) \rmd t   \eqsp,
\end{align}
and
\begin{align}
&\frac{1}{2}(\norm{p_{1}}^2-\norm{p_{0}}^2)= \frac{h^2}{2}\norm{\nabla \F(q_{0})}^2+ \frac{h^2}{8} \norm{\int_0^1 \nabla^2\F(q_t)\defEns{q_{1}-q_{0} }\rmd t }^2\\
&- h\langle p_{0},\nabla \F(q_{0})\rangle -(h/2)\int_0^1 D^2\F(q_t)\defEns{p_{0}\otimes (q_{1}-q_{0})} \rmd t \\
&+ (h^2/2)\int_0^1D^2\F(q_t) \defEns{\nabla \F(q_{0})\otimes (q_{1}-q_{0})} \rmd t \eqsp.
\end{align}
Summing these equalities up  and observing appropriate cancellations yields
\begin{flalign}
\nonumber
&H(q_{1},p_{1})-H(q_{0},p_{0})=\int_0^1D^2\F(q_t)\defEns{q_{1}-q_{0}}^{\otimes 2}(1-t) \rmd t\\
\nonumber
& - (h/2)\int_0^1D^2\F(q_t)\defEns{p_{0}\otimes (q_{1}-q_{0})} \rmd t +   (h^2/8) \norm{ \int_0^1 \nabla ^2\F(q_t)\defEns{q_{1}-q_{0}} \rmd t}^2\\
\nonumber
&+ (h^2/2)\int_0^1D^2\F(q_t)\defEns{\nabla \F(q_{0})\otimes (q_{1}-q_{0})} \rmd t
\end{flalign}
\vspace{-0.8cm}
\begin{equation}
\label{eq:decomp_lem_hamil}
= I_1+I_2+I_3+I_4 \eqsp.
\end{equation}
By using $q_1 = \Phiverletq[h][1](p_0,q_0)$ again in the definition of each $I_j$ we obtain successively
\begin{align}
I_1&=h^2\int_0^1D^2\F( q_t)\defEns{p_0}^{\otimes 2} (1-t) \rmd t- h^3 \int_0^1D^2\F(q_t)\defEns{p_{0}\otimes \nabla \F(q_{0})}(1-t)\rmd t\\
&\qquad \qquad \qquad +(h^4/4)\int_0^1D^2\F(q_t) \defEns{\nabla \F(q_{0})}^{\otimes 2}(1-t)\rmd t \eqsp, \\
I_2&= - (h^2/2)\int_0^1D^2\F(q_t) \defEns{p_{0}}^{\otimes 2} \rmd t + (h^3/4) \int_0^1D^2\F(q_t)\defEns{p_{0}\otimes \nabla   \F(q_{0})} \rmd t \eqsp,\\
I_3&= (h^4/8) \norm{ \int_0^1\nabla^2 \F(q_t)p_{0} \ \rmd t }^2+(h^6/32) \norm{\int_0^1\nabla^2\F(q_t)\nabla \F(q_{0})\  \rmd t }^2
\\
& \qquad \qquad \qquad  - (h^5/8)\ps{\int_0^1\nabla^2\F(q_t)\nabla \F(q_0) \ \rmd t }{\int_0^1 \nabla^2 \F(q_t)p_0 \ \rmd t } \eqsp.
\end{align}
and
\begin{align}
  I_4& = (h^3/2)\int_0^1D^2\F(q_t)\defEns{\nabla \F(q_{0})\otimes p_{0}}\rmd t \\
  & \qquad \qquad \qquad \qquad  \qquad \qquad - (h^4/4)\int_0^1D^2\F(q_t)\defEns{\nabla \F(q_{0})}^{\otimes 2}\rmd t \eqsp,
\end{align}
Gathering all these equalities  in \eqref{eq:decomp_lem_hamil} concludes the proof.
 \end{proof}


\begin{proof}[Proof of \Cref  {propo:accept}]
Let $\gamma \in \ooint{0,m-1}$, $T \in \nsets$, $h_0 \in \rset_+^*$ and  $h \in \ocint{0,h_0}$.
Denote for all $k \in \{0,\ldots,T\}$ by $(q_k,p_k) =
  \Phiverlet[h][k](q_0,p_0)$, $q_0, p_0 \in \rset^d$.
  For all $q_0,p_0 \in \rset^d$, consider the following decomposition
\begin{equation}
  \label{eq:diff_ham_decompo}
H(p_T,q_T)-H(p_0,q_0)=\sum_{k=0}^{T-1}\defEns{H(p_{k+1},q_{k+1})-H(p_{k},q_{k})} \eqsp.
\end{equation}
We show that each term in the sum in the right hand side of this equation is nonpositive if $\norm{\q_0}$ is large enough and $\norm{p_0} \leq \norm{q_0}^{\gamma}$.
By \Cref{lem:diff_hamiltonian_taylor_exp}, we have
\begin{equation}
\label{eq:diff_ham_k}
H(q_{k+1},p_{k+1})-H(q_{k},p_{k})
= -(h^4/4)A_k+h^2 B_k+ h^3C_k+(h^4/8) D_k \eqsp,
\end{equation}
where, setting $q_{t,k}= q_k + t(q_{k+1}-q_k)$ for $t \in \ccint{0,1}$,
\begin{align}
  A_k &=\int_0^1 D^2\F(q_{t,k})\defEns{\nabla \F(q_{k})}^{\otimes 2} \ t \  \rmd t \\
B_k& = \int_0^1D^2\F(q_{t,k})\defEns{p_{k}}^{\otimes 2} (1/2-t) \  \rmd t \\
C_k & = \int_0^1D^2\F(q_{t,k})\defEns{p_{k}\otimes \nabla \F(q_{k})}(t-1/4) \ \rmd t \\
D_k & = \norm{ \int_0^1 \nabla ^2 \F( q_{t,k}) p_{k} \ \rmd t  }^2 +(h^2/4)\norm{\int_0^1 \nabla^2 \F( q_{t,k})\nabla \F (q_{k}) \ \rmd t }^2 \\
& \qquad \qquad -  h \ps{\int_0^1\nabla^2\F(q_{t,k})\nabla \F(q_{k}) \ \rmd t }{\int_0^1 \nabla^2 \F(q_{t,k})p_{k} \ \rmd t }
\end{align}
Since $q_{t,k}-q_k= -(t h^2/2) \nabla \F(q_k) + th p_k$ and $\int_{0}^1(1/2-t) \, \rmd t = 0$, we have
for all $q_0,p_0 \in \rset^d$,
\begin{align}
B_k &= \int_{0}^1 \int_0^1 D^3 \F(q_{k} +s(q_{t,k} - q_{k})) \defEns{p_{k}^{\otimes 2} \otimes (q_{t,k}-q_{k})} (1/2-t)\rmd s \ \rmd t 
  \\
  \label{eq:definition-B_k}
    &= h B_{k,1} - h^2 B_{k,2}
\end{align}
where
\begin{align}
\label{eq:definition-B_k-1}
B_{k,1} &= \int_{0}^1 \int_0^1 D^3 \F(q_{k} +s(q_{t,k} - q_{k})) \defEns{p_{k}}^{\otimes 3}t (1/2-t)\rmd s \ \rmd t \\
\label{eq:definition-B_k-2}
B_{k,2} &=  -\frac{1}{2} \int_{0}^1 \int_0^1 D^3 \F(q_{k} +s(q_{t,k} - q_{k})) \defEns{p_{k}^{\otimes 2} \otimes \nabla \F(q_k) } t (1/2-t)\rmd s \ \rmd t \eqsp.
\end{align}
Consider now the term $C_k$ in \eqref{eq:diff_ham_k}. Similarly, using again $\int_0^1 (t- 1/2) \rmd t= 0$ and then \eqref{eq:pk}, we get $C_k = C_{k,1} +  C_{k,2} + C_{k,3}$,  where
\begin{align}
\label{eq:definition-C_k-1}
&C_{k,1} = h \int_{0}^1 \int_{0}^t D^3 \F(q_{k} +s(q_{k,t} - q_{k})) \defEns{p_{k}^{\otimes 2} \otimes \nabla \F(q_k) } t (t-1/2)\rmd s \ \rmd t \\
\nonumber
& -(h^2/2)\int_{0}^1 \int_{0}^t  D^3 \F(q_{k} +s(q_{k,t} - q_{k})) \defEns{p_{k} \otimes \parenthese{\nabla \F(q_k)}^{\otimes 2} } t (t-1/2)\rmd s \ \rmd t \\
\label{eq:definition-C_k-2}
&C_{k,2} =   \int_0^1D^2\F(q_{t,k})\defEns{p_{0}\otimes \nabla \F(q_{k})} \ \rmd t \eqsp, \\
\label{eq:definition-C_k-3}
&C_{k,3} = -h \sum_{i=1}^{k-1}\int_0^1D^2\F(q_{t,k})\defEns{\nabla \F(q_i)\otimes \nabla \F(q_{k})} \ \rmd t
\\
&\qquad \qquad  \qquad \qquad - (h/2)\int_0^1D^2\F(\q_{t,k})\defEns{\parenthese{\nabla \F(\q_0)+\nabla \F (\q_k)} \otimes \nabla \F(\q_{k})} \ \rmd t
\end{align}
We will next estimate each of these terms separately.
Let $\delta \in \ooint{0,1}$ and $\BB_0 \in \rset_+^*$ be the constants defined in \Cref{lem:variation_assum_hessian}.


\begin{enumerate}[label=(\alph*),leftmargin=0cm,itemindent=0.5cm,labelwidth=1.2\itemindent,labelsep=0cm,align=left]
\item
We first consider the case $m\in \ooint{1,2}$.
  By \Cref{lem:grad_Lip_F} and \Cref{lem:bound_first_iterate_leapfrog_b}-\ref{lem:bound_first_iterate_leapfrog_1}, there exist $C \geq 0$ and  $R_1 \geq \rhtwo$ such that for all $\q_0,p_0 \in \rset^d$ satisfying $ \norm{p_0} \leq
\norm{\q_0}^{\gamma}$ and $\norm{\q_0} \geq R_1$, for all $i \in
\{0,\ldots,T\}$,
\begin{equation}
\label{eq:bound_iterate_q_3_prood_diff_ham_3_0}
\begin{aligned}
&\norm{\q_{i}-\q_{0}}\leq (\delta/2) \norm{\q_0} \\ 
&\norm{\p_i - \p_0} \leq C(\norm{\p_0} + h \norm{q_0}^{m-1}) \leq C(\norm{\q_0}^\gamma + h \norm{q_0}^{m-1}) \eqsp.
\end{aligned}
\end{equation}
By  \Cref{lem:variation_assum_hessian}, \Cref{lem:prepa_bound_diff_ham}-\ref{lem:prepa_bound_diff_ham_1} and \eqref{eq:bound_iterate_q_3_prood_diff_ham_3_0}, there exists $R_2 \geq R_1$ such that for all $\q_0,\p_0 \in \rset^d$, $\norm{\q_0} \geq R_2$ and $\norm{\p_0} \leq \norm{\q_0}^{\gamma}$,
we get that
\begin{equation}
\label{eq:bound_A_k}
  \inf_{\norm{p_0} \leq \norm{q_0}^{\gamma}} A_k \geq \BB_0 \norm{q_k}^{3\m-4} \geq
  \BB_0 \{ (1-\delta/2)^{3m-4} \wedge (1+\delta/2)^{3m-4} \} \norm{q_0}^{3\m-4} \eqsp.
\end{equation}
Hence, $\limsup_{\norm{q_0} \to \plusinfty} \sup_{\norm{p_0} \leq \norm{q_0}^{\gamma}}\defEns{A_k/\norm{q_0}^{3\m-4}} > 0$.
We now bound $B_k$. Using \Cref{assum:potential}-\ref{assum:potential:a}, \Cref{lem:grad_Lip_F} and \eqref{eq:bound_iterate_q_3_prood_diff_ham_3_0}, we get by \eqref{eq:definition-B_k} that
\begin{equation}
\label{eq:bound_B_k}
\limsup_{\norm{q_0} \to \plusinfty} \sup_{\norm{p_0} \leq \norm{q_0}^{\gamma}}\defEns{\abs{B_k}/\norm{q_0}^{4\m-6}} < \infty \eqsp.
\end{equation}
Combining \Cref{assum:potential}-\ref{assum:potential:a}, \Cref{lem:grad_Lip_F} and \eqref{eq:bound_iterate_q_3_prood_diff_ham_3_0} again, we get by crude estimate that there exists $C \geq 0$ such that
\begin{equation}
  \label{eq:bound_D_k}
  \limsup_{\norm{q_0} \to \plusinfty} \sup_{\norm{p_0} \leq \norm{q_0}^{\gamma}} \defEns{\abs{D_k}/\norm{q_0}^{4\m-6}} \leq C h^2 \eqsp.
\end{equation}
We finally bound the two terms $C_{k,1}$ and $C_{k,2}$. First,
using the same reasoning as for $B_k$, we get that
\begin{equation}
%\label{eq:bound_C_k_1}
\begin{aligned}
  &\limsup_{\norm{q_0} \to \plusinfty} \sup_{\norm{p_0} \leq \norm{q_0}^{\gamma}} \defEns{\abs{C_{k,1}}/\norm{q_0}^{4\m-6}} < \infty \eqsp, \\
  &\limsup_{\norm{q_0} \to \plusinfty} \sup_{\norm{p_0} \leq \norm{q_0}^{\gamma}} \defEns{\abs{C_{k,2}}/\norm{q_0}^{2\m-3+\gamma}} < \infty \eqsp.
\end{aligned}
\end{equation}
Arguing like in \eqref{eq:bound_A_k}, we get that
$\limsup_{\norm{q_0} \to \plusinfty}  \sup_{\norm{p_0} \leq \norm{q_0}^{\gamma}} \defEns{C_{k,3}/\norm{q_0}^{3\m-4}} < 0$.
Gathering all these results and  using that  $3m-4 \geq \max(4m-6,2m-3+\gamma)$ for $m \in \ooint{1,2}$ and $\gamma \in \ooint{0,m-1}$, we get that for all $k\in \{0, \ldots,T-1\}$,
\begin{equation}
%  \label{eq:bound_C_k}
  \limsup_{\norm{q_0} \to \plusinfty}  \sup_{\norm{p_0} \leq \norm{q_0}^{\gamma}} \defEns{\Ham(q_{k+1},p_{k+1})-\Ham(q_{k},p_{k})}/\norm{q_0}^{3m-4} < 0 \eqsp,
\end{equation}
which concludes the proof.
\item
  Consider now the case $\m=2$.
First   by \Cref{lem:grad_Lip_F} and \Cref{lem:bound_first_iterate_leapfrog_b}-\ref{lem:bound_first_iterate_leapfrog_b_2}, there exist $\bar{S}_1 \geq 0$ and  $R_1 \geq \rhtwo$ such that for all $T \in \nsets$ and $h \in \ocint{0,\bar{S}_1/T}$, $\q_0,p_0 \in \rset^d$ such that $ \norm{p_0} \leq
\norm{\q_0}^{\gamma}$ and $\norm{\q_0} \geq R_1$, and $i \in
\{0,\ldots,T\}$,
\begin{equation}\label{eq:bound_iterate_q_3_prood_diff_ham_3_0_2}
\norm{\q_{i}-\q_{0}}\leq (\delta/2) \norm{\q_0}  \eqsp.
\end{equation}
and
\begin{equation}
\label{eq:bound_iterate_q_3_prood_diff_ham_3_0_3}
\begin{aligned}
\norm{\q_{i}-\q_{0}}&\leq \norm{p_0} Th + (1/2) (T+1)^2 h^2 (\constzeroT + \constzero \delta/2) \norm{q_0}\eqsp, \\
 \norm{\p_i - \p_0} &\leq h T \{\constzeroT + (\constzeroT + \constzero \delta/2) \norm{q_0}  \} \eqsp,
\end{aligned}
\end{equation}
where $\constzero$ and $\constzeroT$ are defined in \Cref{lem:grad_Lip_F}. By  \Cref{lem:variation_assum_hessian}, \Cref{lem:prepa_bound_diff_ham}-\ref{lem:prepa_bound_diff_ham_2} and \eqref{eq:bound_iterate_q_3_prood_diff_ham_3_0_2}, there exists $R_2 \geq R_1$ such that for all $\q_0,\p_0 \in \rset^d$, $\norm{\q_0} \geq R_2$ and $\norm{\p_0} \leq \norm{\q_0}^{\gamma}$
\begin{equation}
\label{eq:bound_A_k_2}
  \inf_{\norm{p_0} \leq \norm{q_0}^{\gamma}} A_k \geq \BB_0 \norm{q_k}^{2} \geq
  \BB_0 (1-\delta/2)^{2} \norm{q_0}^{2} \eqsp.
\end{equation}
Hence,
\begin{equation}
\label{eq:bound_A_k_2_2}
\limsup_{\norm{q_0} \to \plusinfty} \sup_{\norm{p_0} \leq \norm{q_0}^{\gamma}}\defEns{A_k/\norm{q_0}^{2}} \geq   \BB_0 (1-\delta/2)^{2}  \eqsp.
\end{equation}
We now bound $B_k$. Using \Cref{assum:potential}-\ref{assum:potential:a}, \Cref{lem:grad_Lip_F} and \eqref{eq:bound_iterate_q_3_prood_diff_ham_3_0_3}, we get by \eqref{eq:definition-B_k} that there exists $\rmD_1 \geq 0$ which does not depend on $T$ and $h$ such that
\begin{equation}
\label{eq:bound_B_k_2}
\limsup_{\norm{q_0} \to \plusinfty} \sup_{\norm{p_0} \leq \norm{q_0}^{\gamma}}\defEns{\abs{B_k}/\norm{q_0}^{2}} \leq \rmD_1 h  \{(hT)^3 + (hT)^4\}  \eqsp.
\end{equation}
Combining \Cref{assum:potential}-\ref{assum:potential:a}, \Cref{lem:grad_Lip_F} and \eqref{eq:bound_iterate_q_3_prood_diff_ham_3_0_3} again, we get by crude estimate that there exists $\rmD_2 \geq 0$ which does not depend on $T$ and $h$ such that
\begin{equation}
  \label{eq:bound_D_k_2}
  \limsup_{\norm{q_0} \to \plusinfty} \sup_{\norm{p_0} \leq \norm{q_0}^{\gamma}} \defEns{\abs{D_k}/\norm{q_0}^{2}} \leq \rmD_2 (hT)^2 \eqsp.
\end{equation}
We finally bound the two terms $C_{k,1}$ and $C_{k,2}$. First,
using the same reasoning as for $B_k$, we get that there exists $\rmD_3 \geq 0$ which does not depend on $T$ and $h$ such that
\begin{equation}
\label{eq:bound_C_k_2}
\begin{aligned}
&\limsup_{\norm{q_0} \to \plusinfty} \sup_{\norm{p_0} \leq \norm{q_0}^{\gamma}} \defEns{\abs{C_{k,1}}/\norm{q_0}^{2}} < \rmD_3 h \{(hT)^4 + (hT)^5\} \eqsp,\\
&\limsup_{\norm{q_0} \to \plusinfty} \sup_{\norm{p_0} \leq \norm{q_0}^{\gamma}} \defEns{\abs{C_{k,2}}/\norm{q_0}^{1+\gamma}} < \infty \eqsp.
\end{aligned}
\end{equation}
Finally, arguing like in \eqref{eq:bound_A_k_2_2}, we get that
\begin{equation}
\label{eq:bound_C_k_2_2}
  \limsup_{\norm{q_0} \to \plusinfty}  \sup_{\norm{p_0} \leq \norm{q_0}^{\gamma}} \defEns{C_{k,3}/\norm{q_0}^{2}} < 0 \eqsp.
\end{equation}
Combining \eqref{eq:bound_A_k_2_2}-\eqref{eq:bound_B_k_2}-\eqref{eq:bound_D_k_2}-\eqref{eq:bound_C_k_2} and \eqref{eq:bound_C_k_2_2} in  \eqref{eq:diff_ham_k}, and using that  $2 \geq 1+ \gamma $ for  $\gamma \in \ooint{0,1}$, we get that for all $k\in \{0, \ldots,T-1\}$,
\begin{align}
  &  \limsup_{\norm{q_0} \to \plusinfty}  \sup_{\norm{p_0} \leq \norm{q_0}^{\gamma}} \defEns{\Ham(q_{k+1},p_{k+1})-\Ham(q_{k},p_{k})}/\norm{q_0}^{2} \\
  & \qquad \qquad\qquad \qquad \leq - \BB_0 (1-\delta/2)^{2} h^4 + \rmD_1\{(hT)^3 + (hT)^4\} h^3 \\
  & \qquad \qquad  \qquad \qquad \qquad + \rmD_2(hT)^2h^4 + \rmD_3\{(hT)^4 + (hT)^5\} h^4\eqsp.
%& \qquad \qquad   < - \BB_0 (1-\delta/2)^{2} h^4 + \rmD_1\{(hT)^3 + (hT)^4\} h^3 + \rmD_2\bar{S}_3^2 h^4 + \rmD_3\{ \bar{S}_3^4 + \bar{S}_3^5\} h^4\eqsp.
\end{align}
Therefore, there exists $\bar{S}_4\leq \bar{S}_3$ such for any $T\in \nsets$, $h \in \ocint{0,\bar{S}_4/T^{3/2}}$,
\begin{equation}
%  \label{eq:9}
    \limsup_{\norm{q_0} \to \plusinfty}  \sup_{\norm{p_0} \leq \norm{q_0}^{\gamma}} \defEns{\Ham(q_{k+1},p_{k+1})-\Ham(q_{k},p_{k})}/\norm{q_0}^{2}< 0 \eqsp,
  \end{equation}
  which completes the proof.
\end{enumerate}
\end{proof}

\subsubsection{Proof of \Cref{propo:accept_pertub}}
\label{sec:proof-crefth_accept_2}


\begin{lemma}
\label{lem:variation_assum_hessian_pertub}
Assume \Cref{ass:pertub}.
Then there exist $\delta \in \ooint{0,1}$, $R_1 \in \rset_+$ $\BB_1 \in \rset_+^*$ such that for all $\q,\x,z \in \rset^d$, with
\begin{equation}
\label{eq:hyp_variation_assum_hessian_pertub}
\norm{q} \geq R_1 \eqsp, \qquad  \max\parenthese{\norm{\q-\x},\norm{\q-z}} \leq \delta \norm{\q}  \eqsp,
\end{equation}
we have
\begin{equation}
  % \ps{\Sigmabf^2 \x}{\Sigmabf z} \geq \BB_1 \norm{\q}^{2} \eqsp,  \quad \ps{\Sigmabf \nabla \F(\x)}{\Sigmabf z} \geq \BB_1 \norm{\q}^{2}  \eqsp.
  \ps{\Sigmabf \nabla \F(\x)}{\Sigmabf z} \geq \BB_1 \norm{\q}^{2}  \eqsp.
\end{equation}
\end{lemma}
\begin{proof}
Under \Cref{ass:pertub},  it can be easily checked that there
exists $\tilde{C}_U \geq 0$ (depending only on $\constfive$ and $\Sigmabf$) such that for all  $\q,\x,z \in \rset^d$ satisfying  \eqref{eq:hyp_variation_assum_hessian} for $\delta \in \ooint{0,1}$ and $R_1 \in \rset_+$,
\begin{equation}
% \ps{\Sigmabf^2 x}{ \Sigmabf z  }  \geq \ps{\Sigmabf^2 q}{ \Sigmabf q} - \tilde{C}_U  \delta \norm{\q}^{2} \eqsp, \quad \ps{\Sigmabf \nabla \F(\x)}{\Sigmabf  z}  \geq \ps{\Sigmabf^2 q}{\Sigmabf q} - \tilde{C}_U  (\delta \norm{\q}^{2} + \norm[\rho]{\q})\eqsp.
 \ps{\Sigmabf \nabla \F(\x)}{\Sigmabf  z}  \geq \ps{\Sigmabf^2 q}{\Sigmabf q} - \tilde{C}_U  (\delta \norm{\q}^{2} + \norm[\rho]{\q})\eqsp, \quad \norm{q} \geq R_1 \eqsp.
\end{equation}
The proof is concluded by using that $\Sigmabf$ is definite positive and  taking $\delta$ sufficiently small and $R_1$ sufficiently large.
\end{proof}

\begin{proof}[Proof of \Cref{propo:accept_pertub}]
  Note that by \Cref{ass:pertub}, \Cref{lem:bound_first_iterate_leapfrog_b}-\ref{lem:bound_first_iterate_leapfrog_b_2}, \Cref{lem:prepa_bound_diff_ham}-\ref{lem:prepa_bound_diff_ham_2} and \Cref{lem:variation_assum_hessian_pertub}, there exists $\BB_1,\bar{S}_1 >0$, $R_1 \geq 0$, such that for any $T \in \nsets$, $h \in \ocint{0,\bar{S}_1/T}$, $q_0,p_0 \in \rset^d$, $\norm{p_0} \leq \norm{q_0}^{\gamma}$, $\norm{q_0} \geq \max(1,R_1)$ and $k,i \in \{0, \ldots,T\}$, $\norm{q_0} \leq 2 \norm{q_k} \leq 3 \norm{q_0}$, $\abs{\tilde{U}(q_k)} \leq C_1 \norm{q_0}^{\rho}$, 
  \begin{equation}
\label{eq:proof_pertub_accept_1}
 \ps{\Sigmabf \nabla U(q_i)}{\Sigma q_k} \geq \BB_1 \norm[2]{q_k}  \eqsp, \quad  \norm{\nabla U(q_i)} \leq C_1 \norm{q_k} \eqsp,
  \end{equation}
  where $q_k = \Phiverletq[T][k](q_0,p_0)$ and  $C_1= \max(4\constfive, 3(\norm{\Sigmabf} + 2 \constfive))$.
  Let now $T \in \nsets$, $h \in \ocint{0,\bar{S}_1/T}$ and denote for any $k \in \{0,\ldots,T\}$, $(q_k,p_k) = \Phiverlet[T][k](q_0,p_0)$ for $q_0,p_0 \in \rset^d$. We consider the following decomposition:
  \begin{equation}
\label{eq:proof_pertub_accept_2}
H(p_T,q_T)-H(p_0,q_0)=\sum_{k=0}^{T-1}\defEns{H(p_{k+1},q_{k+1})-H(p_{k},q_{k})} \eqsp.
\end{equation}
We show below that there exists $\bar{S} < \bar{S}_1$ such that, for all $h \geq 0$ and $T \geq 0$ satisfying $hT \leq \bar{S}$,
\begin{equation}
\label{eq:proof_pertub_accept_3}
\limsup_{\norm{q_0} \to \plusinfty} \sup_{\norm{p_0} \leq \norm{q_0}^{\gamma}} [ \defEns{H(p_{k+1},q_{k+1})-H(p_{k},q_{k})}/\norm[2]{q_0}] <0 \eqsp,
\end{equation}
from which  the proof follows.
First for any $q_0,p_0 \in \rset^d$, $k \in \{0,\ldots,T-1\}$, we have
\begin{equation}
\label{eq:proof_pertub_accept_4}
  H(p_{k+1},q_{k+1})-H(p_{k},q_{k}) = A_k + B_k + C_k \eqsp,
\end{equation}
where $2 A_k =  \ps{\Sigmabf q_{k+1}}{q_{k+1}} - \ps{\Sigmabf q_{k}}{q_{k}}$,
$B_k = \tilde{U}(q_{k+1}) - \tilde{U}(q_k)$, and $2 C_k = \norm[2]{p_{k+1}} - \norm[2]{p_k}$.
By \eqref{eq:proof_pertub_accept_1} and \Cref{ass:pertub}, we have
\begin{equation}
\label{eq:proof_pertub_accept_6}
  \lim_{\norm{q_0} \to \plusinfty} \sup_{\norm{p_0} \leq \norm{q_0}^\gamma} \abs{B_k}/\norm[2]{q_0} =0 \eqsp,
\end{equation}
and
\begin{align}
  \label{eq:proof_pertub_accept_7}
  A_k &= h\ps{\Sigmabf p_k}{q_k} + h^2 \ps{\Sigmabf p_k}{p_k}/2 -h^2 \ps{\Sigmabf q_k }{\Sigmabf q_k }/2 \\
  & \qquad \qquad - h^3 \ps{\Sigmabf p_k}{\Sigmabf q_k }/2  +  h^4 \ps{\Sigmabf^2 q_k}{\Sigmabf q_k } /8  + A_{k,1}  \eqsp,
\end{align}
\begin{align}
    C_k & = -h\ps{\Sigmabf p_k}{q_k} -h^2 \ps{\Sigmabf p_k}{p_k}/2 +h^2 \ps{\Sigmabf q_k }{\Sigmabf q_k }/2\\
  & \qquad \quad  +3h^3 \ps{\Sigmabf p_k}{\Sigmabf q_k}/4  + h^4 \ps{\Sigmabf p_k }{\Sigmabf p_k}/8  
    -h^4 \ps{\Sigmabf^2 q_k }{\Sigmabf q_k} /4 \\
  &\qquad \quad -h^5 \ps{\Sigmabf^2 p_k}{q_k}/8 
    + h^6 \ps{\Sigmabf^2 q_k }{\Sigmabf^2 q_k}/32  + C_{k,1} \eqsp,
    \label{eq:proof_pertub_accept_8}
\end{align}
where
\begin{equation}
\label{eq:proof_pertub_accept_9}
  \lim_{\norm{q_0} \to \plusinfty} \sup_{\norm{p_0} \leq \norm{q_0}^\gamma} \{\abs{A_{k,1}} + \abs{C_{k,1}}\}/\norm[2]{q_0} =0 \eqsp,
\end{equation}
Using \eqref{eq:proof_pertub_accept_4}, \eqref{eq:proof_pertub_accept_7} and \eqref{eq:proof_pertub_accept_8}, we obtain that for any $q_0,p_0 \in \rset^d$,
\begin{equation}
  \label{eq:proof_pertub_accept_10}
  H(q_{k+1},p_{k+1}) - H(q_k,p_k) = D_k + A_{k,1}+  B_k + C_{k,1} \eqsp,
\end{equation}
where
\begin{align}
  D_k &= h^3 \ps{\Sigmabf p_k}{\Sigmabf q_k}/4 + h^4 \ps{\Sigmabf p_k }{\Sigmabf p_k}/8   -h^4 \ps{\Sigmabf^2 q_k }{\Sigmabf q_k} /8 \\
&  \qquad \qquad \qquad -h^5 \ps{\Sigmabf^2 p_k}{q_k}/8 + h^6 \ps{\Sigmabf^2 q_k }{\Sigmabf^2 q_k}/32 \eqsp.
\end{align}
Using that for $k \in \{1,\ldots,T\}$,  $p_k= p_0 -(h/2)\{\nabla U(q_0) + \nabla U(q_k)\} - h \sum_{i=1}^{k-1} \nabla U(q_i)$ and \eqref{eq:proof_pertub_accept_1}, we obtain that for any $k\in \{1,\ldots,T\}$ and $q_0,p_0$, $\norm{q_0} \geq \max(1,R_1)$, $\norm{p_0} \geq \norm[\gamma]{q_0}$,
\begin{align}
  D_k &\leq -h^4 k \BB_1 \norm[2]{q_k}/8 + h^6k^2 \norm{\Sigmabf}^2 C_1 \norm[2]{q_k}  - h^4 \ps{\Sigmabf^2 q_k }{\Sigmabf q_k} /8 \\
      &  \qquad \qquad\qquad \qquad+  h^6 k C_1 \norm{\Sigmabf}^2 \norm[2]{q_k}/8 +  h^6 \norm{\Sigmabf}^4 \norm[2]{q_k}/32     + D_{k,1} \eqsp,
\end{align}
where
\begin{equation}
\label{eq:proof_pertub_accept_lim_D}
    \lim_{\norm{q_0} \to \plusinfty} \sup_{\norm{p_0} \leq \norm{q_0}^\gamma} \abs{D_{k,1}}/\norm[2]{q_0} =0\eqsp.
  \end{equation}
  Define
  \begin{equation}
\label{eq:proof_pertub_accept_def_S_2}
    \bar{S}_2 = \min\defEns{ S \in \ocint{0,\bar{S}_1} \, : \, S^2 ( 2 C_1  \norm{\Sigmabf}^2 + \norm{\Sigmabf}^4)  - \BB_1/8 \geq -\BB_1/16} \eqsp.
  \end{equation}
  Then, if $Th \leq \bar{S_2}$ for any  $q_0,p_0$, $\norm{q_0} \geq \max(1,R_1)$, $\norm{p_0} \geq \norm[\gamma]{q_0}$, we get that
  \begin{equation}
\label{eq:proof_pertub_accept_bound_D}
    D_k \leq -\BB_1h^4 k \norm[2]{q_k}/16 + D_{k,1} \eqsp.
  \end{equation}
  Similarly using that $\Sigmabf$ is definite positive, we obtain that there exist $\BB_2 >0$ and  $\bar{S}_3  \in \ocint{0,\bar{S}_1}$ such that if $hT \leq \bar{S}_3$, for any  $q_0,p_0$, $\norm{q_0} \geq \max(1,R_1)$, $\norm{p_0} \geq \norm[\gamma]{q_0}$, we get that
  \begin{equation}
\label{eq:proof_pertub_accept_bound_D_0}
    D_0 \leq -\BB_2 \norm[2]{q_0} + D_{0,1} \eqsp,
 \quad \text{   where }
    \lim_{\norm{q_0} \to \plusinfty} \sup_{\norm{p_0} \leq \norm{q_0}^\gamma} \abs{D_{0,1}}/\norm[2]{q_0} =0\eqsp.
  \end{equation}
  Combining \eqref{eq:proof_pertub_accept_6}-\eqref{eq:proof_pertub_accept_9}-\eqref{eq:proof_pertub_accept_lim_D}-\eqref{eq:proof_pertub_accept_bound_D} and \eqref{eq:proof_pertub_accept_bound_D_0} in \eqref{eq:proof_pertub_accept_10}, we obtain that \eqref{eq:proof_pertub_accept_3} holds with $\bar{S} = \min(\bar{S}_2,\bar{S}_3)$ since \eqref{eq:proof_pertub_accept_1} implies that  $\norm{q_k} \geq \norm{q_0}/2$.
\end{proof}

%%% Local Variables:
%%% mode: latex
%%% TeX-master: "main"
%%% End:
