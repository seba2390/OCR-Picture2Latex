\documentclass[reqno,11pt]{article}
\newcommand{\argmin}[1]{\underset{#1}{\mathrm{argmin}} \:}
\newcommand{\argmax}[1]{\underset{#1}{\mathrm{argmax}} \:}
\newcommand{\ip}[2]{\left\langle #1, #2 \right\rangle}
\newcommand{\E}{\ensuremath{\mathbb{E}}}
\renewcommand{\P}{\ensuremath{\mathbb{P}}}
\renewcommand{\Pr}[1]{\operatorname{Pr}\left[{#1}\right]}
\newcommand{\Ex}[1]{\ensuremath{\mathbb{E}}\left[#1\right]}
\newcommand{\Proj}[2]{\textrm{Proj}_{#1}\left({#2}\right)}
\newcommand{\norm}[1]{\left\lVert{#1}\right\rVert}
\newcommand{\frob}[1]{\left\lVert{#1}\right\rVert_F}
\newcommand{\Otilde}[1]{\widetilde{O}\left({#1}\right)}
\newcommand{\abs}[1]{\left\lvert{#1}\right\rvert}
\newcommand{\vnorm}[1]{\left\lVert{#1}\right\rVert} % vector norm
\newcommand{\mnorm}[1]{\left\lVert{#1}\right\rVert} % matrix norm
\newcommand{\R}{\mathbb{R}}
\newcommand{\Rnk}{\mathbb{R}^{n\times k}}
\newcommand{\Rn}{\mathbb{R}^{n}}
\newcommand{\Rk}{\mathbb{R}^{k}}
\newcommand{\Rm}{\mathbb{R}^{m}}
\newcommand{\Rnn}{\mathbb{R}^{n\times n}}
\newcommand{\C}{\mathbb{C}}
\newcommand{\Z}{\mathbb{Z}}
\newcommand{\N}{\mathcal{N}}
\newcommand{\ceil}[1]{\lceil #1\rceil}
\newcommand{\floor}[1]{\lfloor #1\rfloor}
\newcommand{\trans}[1]{ {#1}^{\!\top}}
\newcommand{\trace}[1]{\mathrm{Tr}\left(#1\right)}
\newcommand{\Sym}[1]{\mathcal{S}^{#1}}
\newcommand{\Snn}{\Sym{n\times n}}
\renewcommand{\Vec}[1]{\mathrm{Vec}\left(#1\right)}
\newcommand{\mip}[2]{{\left\langle{#1},{#2}\right\rangle}}
\newcommand{\linspan}[1]{\textrm{Span}\left(#1\right)}
\newcommand{\Ps}{\mathscr{P}}

\newcommand{\nnz}{\operatorname{nnz}}
\newcommand{\nulll}{\operatorname{null}}

\newcommand{\praneeth}[1]{{\color{red}#1}}

\newcommand{\zeros}{\mathbf{0}}
\newcommand{\ones}{\mathbf{1}}
\newcommand{\eye}{\mathbf{I}}

\newcommand{\DelU}{\triangle U}
\newcommand{\DelUbar}{\triangle \overline{U}}
\newcommand{\Delu}{\triangle u}

\newcommand{\DelV}{\triangle V}
\newcommand{\DelVbar}{\triangle \overline{V}}
\newcommand{\Delv}{\triangle v}
\newtheorem{theorem}{Theorem}
\newtheorem{lemma}{Lemma}
\newtheorem*{lemma*}{Lemma}
\newtheorem{corollary}{Corollary}
\newtheorem{conj}{Conjecture}
\newtheorem{definition}{Definition}
\newtheorem{proposition}{Proposition}

\newcommand{\defeq}{\triangleq}
\newcommand{\poly}{\operatorname{poly}}
\newcommand{\eps}{\epsilon}
\newcommand{\calA}{\mathcal{A}}
\newcommand{\calC}{\mathcal{C}}
\newcommand{\calN}{\mathcal{N}}
\newcommand{\calM}{\mathcal{M}}
\newcommand{\rank}{\operatorname{rank}}
\newcommand{\im}{\operatorname{im}}

\newcommand{\constkn}{c_{k,n}}
\DeclareMathOperator*{\minimize}{\mathrm{minimize}}
\newcommand{\tLm}{\tilde{L}_{\mu}}
\newcommand{\Fm}{F_{\mu}}
\newcommand{\sG}{\sigma_G}
\newcommand{\tG}{G}
\newcommand{\calS}{\mathcal{S}}
\newcommand{\calE}{\mathcal{E}}
\newcommand{\vr}{\mathbf{r}}
\newcommand{\vb}{\mathbf{b}}
\newcommand{\const}{c_0}
\newcommand{\TODO}[1]{{\color{red}{[#1]}}}
\newcommand{\calG}{\mathcal{G}}
\newcommand{\bz}{b_0}
\newcommand{\tLsm}{\tilde{L}_{\mu,S}}

\newtheorem{assumption}[theorem]{Assumption}
\newtheorem{lemma}{Lemma}
\newtheorem{proposition}{Proposition}
\newtheorem{theorem}{Theorem}
\newtheorem{definition}{Definition}
\newtheorem{corollary}{Corollary}
\newtheorem{remark}{Remark}

\newcommand{\argmin}{\mathop{^\rm argmin}}
\newcommand{\argmax}{\mathop{\rm argmax}}
\newcommand{\rank}{\mathop{\sf rank}}
\newcommand{\iprod}[2]{\langle #1, #2 \rangle}
\newcommand{\lmin}{\lambda_{\min}}
\newcommand{\norm}[1]{\left\|#1\right\|}
\newcommand{\mge}{\succeq}
\newcommand{\mi}{{-1}}
\newcommand{\R}{\mathbb{R}}\newcommand{\B}{\mathbb{B}}\newcommand{\Id}{\mathbf{I}}
\newcommand{\zerovec}{\mathbf{0}}
\newcommand{\onevec}{\mathbf{1}}
\newcommand{\defeq}{:=}
\newcommand{\secref}[1]{Section~\ref{#1}}
\newcommand{\lemref}[1]{Lemma~\ref{#1}}
\newcommand{\proref}[1]{Proposition~\ref{#1}}
\newcommand{\thmref}[1]{Theorem~\ref{#1}}
\newcommand{\assref}[1]{Assumption~\ref{#1}}
\newcommand{\eqnref}[1]{(\ref{#1})}
\newcommand{\algref}[1]{Algorithm~\ref{#1}}

\newcommand{\Prob}{\mathbb{P}}
\newcommand{\E}{\mathbb{E}}
\newcommand{\V}{\mathbb{V}}
\newcommand{\tr}[1]{\textrm{tr}\left(#1\right)}
%\newcommand{\dt}[1]{\textrm{det}\left(#1\right)}

\newcommand{\Evt}{\mathcal{E}}
\newcommand{\EvtG}{\Evt_G}
\newcommand{\EvtD}{\Evt_\Delta}
\newcommand{\EvtX}{\Evt_X}

\newcommand{\lihong}[1]{[[\textbf{LL:} #1]]}
\renewcommand{\lihong}[1]{}

\newcommand{\RN}[1]{%
  \textup{\uppercase\expandafter{\romannumeral#1}}%
}

\newtheorem{assumption}{Assumption}

\title{On the convergence of Hamiltonian Monte Carlo}


\author[1]{Alain Durmus \footnote{Email: alain.durmus@cmla.ens-cachan.fr} }
\author[2]{\'Eric Moulines \footnote{Email: eric.moulines@polytechnique.edu} }
\author[3]{Eero Saksman \footnote{Email: eero.saksman@helsinki.fi} }


\affil[1]{CMLA - \'Ecole normale supérieure Paris-Saclay, CNRS, Université Paris-Saclay, 94235 Cachan, France.}
\affil[2]{Centre de Math\'ematiques Appliqu\'ees,\\ UMR 7641, Ecole Polytechnique}
\affil[3]{University of Helsinki, Department of Mathematics and Statistics}


\begin{document}

\maketitle

\begin{abstract}
This paper discusses the irreducibility and geometric ergodicity of the Hamiltonian Monte Carlo (HMC) algorithm.
We consider cases where the number of steps of the symplectic integrator is either  fixed or random. Under mild conditions on the potential $\F$ associated with target distribution $\pi$, we first show that the Markov kernel associated to the HMC algorithm is irreducible and recurrent.
Under more stringent conditions, we then establish that the Markov kernel is Harris recurrent. Finally, we provide verifiable  conditions on $\F$  under which the HMC sampler is geometrically ergodic. We compare our assumptions with those recently presented in  \cite{livingstone:betancourt:byrne:girolami:2016} and \cite{bou:sanz:2017}.
\end{abstract}


\section{Introduction }
\section{Introduction}  \label{sec:introduction}

\newcommand\inexpIntro[3]{#1?(#2,#3).}
\newcommand\rinexpIntro[3]{*#1?(#2,#3).}
\newcommand\outexpIntro[3]{#1!(#2,#3).}
\newcommand\outatomIntro[3]{#1!(#2,#3)}

We propose a fully automated method for proving termination of \(\pi\)-calculus processes.
Although there have been a lot of studies on termination analysis for the \(\pi\)-calculus
and related calculi~\cite{Deng06IC,Demangeon07,SangiorgiTermination,KobayashiHybrid,Yoshida04IC,DBLP:journals/jlp/DemangeonHS10,Venet98SAS}, most of them have been rather theoretical,
and there have been surprisingly little efforts in developing  fully automated termination
verification methods and tools based on them. To our knowledge,
Kobayashi's \typical{}~\cite{TyPiCal,KobayashiHybrid} is the only exception that
can prove termination of \(\pi\)-calculus processes (extended with natural numbers)
fully automatically, but its termination analysis is quite limited (see Section~\ref{sec:relatedwork}).

Our method is based on a reduction to termination analysis for sequential programs:
we translate a \(\pi\)-calculus process \(P\) to a sequential program \(S_P\), so that
if \(S_P\) is terminating, so is \(P\). The reduction allows us to use
powerful, mature methods and tools
for termination analysis of sequential programs~\cite{heizmann2016ultimate,freqterm,DBLP:conf/lics/PodelskiR04,Kuwahara2014Termination,DBLP:journals/cacm/CookPR11}.

The idea of the translation is to convert a chain of communications on replicated input
channels to a chain of recursive function calls of the target sequential program.
Let us consider the following Fibonacci process:
\begin{align*}
    & \rinexpIntro{\fib}{n}{r}
        \ifexp{n<2}{ \soutatom{r}{1} \\ &\quad}
                   { \nuexp{s_1} \nuexp{s_2} (\outatomIntro{\fib}{n-1}{s_1} \PAR \outatomIntro{\fib}{n-2}{s_2} \PAR \sinexp{s_1}{x}\sinexp{s_2}{y}\soutatom{r}{x+y}) \\}
    & \PAR \outatomIntro{\fib}{m}{r}
\end{align*}
Here, the process
$\rinexpIntro{\fib}{n}{r} \ldots$ is a function server that computes the \(n\)-th Fibonacci number
in parallel and returns the result to \(r\),
and $\outatom{\fib}{m}{r}$ sends a request for computing the \(m\)-th Fibonacci number;
those who are not familiar with the syntax of the \(\pi\)-calculus may wish to consult
Section~\ref{sec:targetlanguage} first.
To prove that the process above is terminating for any integer \(m\),
it suffices to show that there is no infinite chain of communications on $\fib$:
\[
    \fib(m,r) \to \fib(m_1,r_1) \to \fib(m_2,r_2) \to \cdots.
\]
We convert the process above to the following program:\footnote{The actual translation
  given later is a little more complex.}
\begin{verbatim}
 let rec fib(n) = if n<2 then () else (fib(n-1) [] fib(n-2)) in
 fib(m)
\end{verbatim}
Here, \texttt{[]} represents the non-deterministic choice.
Note that, although the calculation of Fibonacci numbers is not preserved,
for each chain of communications on \texttt{fib}, there is a corresponding
sequence of recursive calls:
\[
\mathtt{fib}(m) \to \mathtt{fib}(m_1) \to \mathtt{fib}(m_2) \to \cdots.
\]
Thus, the termination of the sequential program above implies the termination of
the original process.
As shown in the example above, (i) each communication on a replicated input channel
is converted to a function call, (ii) each communication on a non-replicated input
channel is just removed (or, in the actual translation, replaced by a call of
a trivial function defined by \(f(\seq{x})=(\,)\)), and (iii) parallel composition
is replaced by a non-deterministic choice.
We formalize the translation outlined above and prove its correctness.

The basic translation sketched above sometimes loses too much information.
For example, consider the following process:
\begin{align*}
    & \rinexpIntro{\pre}{n}{r} \soutatom{r}{n-1} \\
    & \PAR \rinexpIntro{f}{n}{r} \ifexp{n<0}{ \soutatom{r}{1} }
                                       { \nuexp{s} (\outatomIntro{\pre}{n}{s} \PAR \sinexp{s}{x}\outatomIntro{f}{x}{r}) } \\
    & \PAR \outatomIntro{f}{m}{r}
\end{align*}
The translation sketched above would yield:
\begin{verbatim}
  let pred(n) = n-1 in
  let rec f(n) = if n<0 then () else (pred(n) [] f(*)) in
  f(m)
\end{verbatim}
Here, \texttt{*} represents a non-deterministic integer: since we have removed
the input $\sinatom{s}{x}$, we do not have information about the value of \( x \).
As a result, the sequential program above is non-terminating, although the original
process is terminating.
To remedy this problem, we also refine the basic translation above by using a refinement
type system for the \(\pi\)-calculus. Using the refinement type system,
we can infer that the value of \(x\) in the original process is less than \(n\),
so that we can refine the definition of \texttt{f} to:
\begin{verbatim}
 let rec f(n) = ... else (pred(n) [] let x=* in assume(x<n);f(x))
\end{verbatim}
The target program is now terminating, from which
we can deduce that the original process is also terminating.
We have implemented an automated tool based on the refined translation above.

The contributions of this paper are summarized as follows.
\begin{itemize}
\item The formalization of the basic translation from the \(\pi\)-calculus
  (extended with integers) to sequential programs, and a proof of its correctness.
\item The formalization of a refined translation based on a refinement type system.
\item An implementation of the refined translation, including automated refinement type
  inference based on CHC solving, and experiments to evaluate the effectiveness of
  our method.
\end{itemize}

The rest of this paper is structured as follows.
Section~\ref{sec:targetlanguage} introduces the source and target languages
of our translation.
Section~\ref{sec:approach} 
formalizes the basic translation, and proves its correctness.
Section~\ref{sec:refinement} refines the basic translation by using a refinement type system.
Section~\ref{sec:implementation} reports an implementation and experiments.
Section~\ref{sec:relatedwork} discusses related work,
and Section~\ref{sec:conclusion} concludes the paper.



\subsection*{Notations}
% Bold lowercase: syntax \nb# where # is {a ... z, 0,1}
\def\nba{{\mathbf{a}}}
\def\nbb{{\mathbf{b}}}
\def\nbc{{\mathbf{c}}}
\def\nbd{{\mathbf{d}}}
\def\nbe{{\mathbf{e}}}
\def\nbf{{\mathbf{f}}}
\def\nbg{{\mathbf{g}}}
\def\nbh{{\mathbf{h}}}
\def\nbi{{\mathbf{i}}}
\def\nbj{{\mathbf{j}}}
\def\nbk{{\mathbf{k}}}
\def\nbl{{\mathbf{l}}}
\def\nbm{{\mathbf{m}}}
\def\nbn{{\mathbf{n}}}
\def\nbo{{\mathbf{o}}}
\def\nbp{{\mathbf{p}}}
\def\nbq{{\mathbf{q}}}
\def\nbr{{\mathbf{r}}}
\def\nbs{{\mathbf{s}}}
\def\nbt{{\mathbf{t}}}
\def\nbu{{\mathbf{u}}}
\def\nbv{{\mathbf{v}}}
\def\nbw{{\mathbf{w}}}
\def\nbx{{\mathbf{x}}}
\def\nby{{\mathbf{y}}}
\def\nbz{{\mathbf{z}}}
\def\nb0{{\mathbf{0}}}
\def\nb1{{\mathbf{1}}}


\def\nbPsi{{\mathbf{\Psi}}}
\def\nbpsi{{\mathbf{\psi}}}
\def\nbPhi{{\mathbf{\Phi}}}
\def\nbphi{{\mathbf{\phi}}}

% Bold capital letters: syntax \nb# where # is {A ... Z}
\def\nbA{{\mathbf{A}}}
\def\nbB{{\mathbf{B}}}
\def\nbC{{\mathbf{C}}}
\def\nbD{{\mathbf{D}}}
\def\nbE{{\mathbf{E}}}
\def\nbF{{\mathbf{F}}}
\def\nbG{{\mathbf{G}}}
\def\nbH{{\mathbf{H}}}
\def\nbI{{\mathbf{I}}}
\def\nbJ{{\mathbf{J}}}
\def\nbK{{\mathbf{K}}}
\def\nbL{{\mathbf{L}}}
\def\nbM{{\mathbf{M}}}
\def\nbN{{\mathbf{N}}}
\def\nbO{{\mathbf{O}}}
\def\nbP{{\mathbf{P}}}
\def\nbQ{{\mathbf{Q}}}
\def\nbR{{\mathbf{R}}}
\def\nbS{{\mathbf{S}}}
\def\nbT{{\mathbf{T}}}
\def\nbU{{\mathbf{U}}}
\def\nbV{{\mathbf{V}}}
\def\nbW{{\mathbf{W}}}
\def\nbX{{\mathbf{X}}}
\def\nbY{{\mathbf{Y}}}
\def\nbZ{{\mathbf{Z}}}

% \mathcal: syntax \ncal# where # is {A ... Z}
\def\ncalA{{\mathcal{A}}}
\def\ncalB{{\mathcal{B}}}
\def\ncalC{{\mathcal{C}}}
\def\ncalD{{\mathcal{D}}}
\def\ncalE{{\mathcal{E}}}
\def\ncalF{{\mathcal{F}}}
\def\ncalG{{\mathcal{G}}}
\def\ncalH{{\mathcal{H}}}
\def\ncalI{{\mathcal{I}}}
\def\ncalJ{{\mathcal{J}}}
\def\ncalK{{\mathcal{K}}}
\def\ncalL{{\mathcal{L}}}
\def\ncalM{{\mathcal{M}}}
\def\ncalN{{\mathcal{N}}}
\def\ncalO{{\mathcal{O}}}
\def\ncalP{{\mathcal{P}}}
\def\ncalQ{{\mathcal{Q}}}
\def\ncalR{{\mathcal{R}}}
\def\ncalS{{\mathcal{S}}}
\def\ncalT{{\mathcal{T}}}
\def\ncalU{{\mathcal{U}}}
\def\ncalV{{\mathcal{V}}}
\def\ncalW{{\mathcal{W}}}
\def\ncalX{{\mathcal{X}}}
\def\ncalY{{\mathcal{Y}}}
\def\ncalZ{{\mathcal{Z}}}

% \mathbb: syntax \nbb# where # is {A ... Z}
\def\nbbA{{\mathbb{A}}}
\def\nbbB{{\mathbb{B}}}
\def\nbbC{{\mathbb{C}}}
\def\nbbD{{\mathbb{D}}}
\def\nbbE{{\mathbb{E}}}
\def\nbbF{{\mathbb{F}}}
\def\nbbG{{\mathbb{G}}}
\def\nbbH{{\mathbb{H}}}
\def\nbbI{{\mathbb{I}}}
\def\nbbJ{{\mathbb{J}}}
\def\nbbK{{\mathbb{K}}}
\def\nbbL{{\mathbb{L}}}
\def\nbbM{{\mathbb{M}}}
\def\nbbN{{\mathbb{N}}}
\def\nbbO{{\mathbb{O}}}
\def\nbbP{{\mathbb{P}}}
\def\nbbQ{{\mathbb{Q}}}
\def\nbbR{{\mathbb{R}}}
\def\nbbS{{\mathbb{S}}}
\def\nbbT{{\mathbb{T}}}
\def\nbbU{{\mathbb{U}}}
\def\nbbV{{\mathbb{V}}}
\def\nbbW{{\mathbb{W}}}
\def\nbbX{{\mathbb{X}}}
\def\nbbY{{\mathbb{Y}}}
\def\nbbZ{{\mathbb{Z}}}

% \mathfrak:
\def\nfrakR{{\mathfrak{R}}}

% Roman: {\rm } syntax \nrm# where # is {a ... z}
\def\nrma{{\rm a}}
\def\nrmb{{\rm b}}
\def\nrmc{{\rm c}}
\def\nrmd{{\rm d}}
\def\nrme{{\rm e}}
\def\nrmf{{\rm f}}
\def\nrmg{{\rm g}}
\def\nrmh{{\rm h}}
\def\nrmi{{\rm i}}
\def\nrmj{{\rm j}}
\def\nrmk{{\rm k}}
\def\nrml{{\rm l}}
\def\nrmm{{\rm m}}
\def\nrmn{{\rm n}}
\def\nrmo{{\rm o}}
\def\nrmp{{\rm p}}
\def\nrmq{{\rm q}}
\def\nrmr{{\rm r}}
\def\nrms{{\rm s}}
\def\nrmt{{\rm t}}
\def\nrmu{{\rm u}}
\def\nrmv{{\rm v}}
\def\nrmw{{\rm w}}
\def\nrmx{{\rm x}}
\def\nrmy{{\rm y}}
\def\nrmz{{\rm z}}


% Special symbols
\def\nbydef{:=}
\def\nborel{\ncalB(\nbbR)}
\def\nboreld{\ncalB(\nbbR^d)}
\def\sinc{{\rm sinc}}

% Theorems etc.
\newtheorem{lemma}{Lemma}
\newtheorem{thm}{Theorem}
\newtheorem{definition}{Definition}
\newtheorem{ndef}{Definition}
\newtheorem{nrem}{Remark}
\newtheorem{theorem}{Theorem}
\newtheorem{prop}{Proposition}
\newtheorem{cor}{Corollary}
\newtheorem{example}{Example}
\newtheorem{remark}{Remark}
\newtheorem{assumption}{Assumption}
\newtheorem{approximation}{Approximation}
	
%%%%%%%% Backwards compatibility

\newcommand{\ceil}[1]{\lceil #1\rceil}
\def\argmin{\operatorname{arg~min}}
\def\argmax{\operatorname{arg~max}}
\def\figref#1{Fig.\,\ref{#1}}%
\def\E{\mathbb{E}}
\def\EE{\mathbb{E}^{!o}}
\def\P{\mathbb{P}}
\def\pc{\mathtt{P_c}}
\def\rc{\mathtt{R_c}}   % rate coverage
\def\p{p}
\def\ie{{\em i.e.}}
\def\eg{{\em e.g.}}
\def\V{\operatorname{Var}}
\def\erfc{\operatorname{erfc}}
\def\erf{\operatorname{erf}}
\def\opt{\mathrm{opt}}
\def\R{\mathbb{R}}
\def\Z{\mathbb{Z}}

\def\LL{\mathcal{L}^{!o}}
\def\var{\operatorname{var}}
\def\supp{\operatorname{supp}}

\def\N{\sigma^2}
\def\T{\beta}							% Threshold = \beta_i
\def\sinr{\mathtt{SINR}}			% Signal to interference plus noise ratio
\def\snr{\mathtt{SNR}}
\def\sir{\mathtt{SIR}}
\def\pcf{\mathtt{pcf}}
\def\ase{\mathtt{ASE}}
\def\se{\mathtt{SE}}
\def\cse{\mathtt{CSE}}
\def\csr{\mathtt{CSR}}
\def\ee{\mathtt{EE}}
\def\see{\mathtt{SEE}}
\def\ec{\mathtt{EC}}
\def\sec{\mathtt{SEC}}
\def\rate{\mathtt{Rate}}

\def\calN{\mathcal{N}}
\def\FE{\mathcal{F}}
\def\calA{\mathcal{A}}
\def\calK{\mathcal{K}}
\def\calT{\mathcal{T}}
\def\calI{\mathcal{I}}
\def\calB{\mathcal{B}}
\def\calE{\mathcal{E}}
\def\calP{\mathcal{P}}
\def\calL{\mathcal{L}}
\DeclareMathOperator{\Tr}{Tr}
\DeclareMathOperator{\rank}{rank}
\DeclareMathOperator{\Pois}{Pois}

\DeclareMathOperator{\TC}{\mathtt{TC}}
\DeclareMathOperator{\TCL}{\mathtt{TC_l}}
\DeclareMathOperator{\TCU}{\mathtt{TC_u}}

% Fading
\def\l{\ell}
\newcommand{\fad}[2]{\ensuremath{\mathtt{h}_{#1}[#2]}}
\newcommand{\h}[1]{\ensuremath{\mathtt{h}_{#1}}}

\newcommand{\err}[1]{\ensuremath{\operatorname{Err}(\eta,#1)}}
\newcommand{\FD}[1]{\ensuremath{|\mathcal{F}_{#1}|}}

%% Symbols changed
% \def\i{\mathbf{1}}					% changed to \nb1
% \def\d{\mathrm{d}}					% changed to \nrmd
% \def\L{\mathcal{L}}					% changed to \ncalL
% \begin{definition}					% changed to \begin{ndef}

% \l also gives problems. Use \ell after defining it if needed.


%% D2D def
\def\Bx{{\mathcal{B}}^x}
\def\Bxx{{\mathcal{B}}^{x_0}}
\def\jx{y}
\def\m{(\bar{n}-1)}
\def\mm{\bar{n}-1}
\def\Nx{{\mathcal{N}}^x}
\def\Nxo{{\mathcal{N}}^{x_0}}
\def\wj{w_{jx_0}}
\def\uij{u_{jx}}
% \def\yj{y_{jx}}
\def\yj{y}
\def\yjx{y}
\def\zjx{z_x}
\def \tx {y_0}
\def \htx {h_0}
%% 

\def\rx{z_{1}}
\def\ry{z_{2}}

\def\Rx{Z_{1}}
\def\Ry{Z_{2}}


%% fading
\def \hyxx {h_{y_{x_0}}}
\def \hyx {h_{y_x}}

%% Added by Priyabrata
%\newcommand{\PropMatrix}{\mathbf{G}}
%\newcommand{\SmallScaleGainMat}{\mathbf{H}}
%\newcommand{\LargeScaleGainMat}{\mathbf{D}}
%\newcommand{\myDet}{\text{det}}
%\newcommand{\ComplexNorm}[2]{\mathcal{CN}\left(#1,#2\right)}
%\newcommand{\DetectedSyms}{\myVec{t}}
%\newcommand{\myExpon}{\text{exp}}
%\newcommand{\myVar}{\text{Var}}
%\newcommand{\myCosInt}[1]{\text{Ci}\left(#1\right)}
%\newcommand{\mySinInt}[1]{\text{Si}\left(#1\right)}
%\newcommand{\myProbSym}[1]{\mathbb{P}\left[#1\right]}


\newcommand{\myVec}[1]{\mathbf{#1}}
\newcommand{\myVecRep}[3]{\left[\begin{matrix} #1 & #2 & \ldots & #3 \end{matrix}\right]}
\newcommand{\myVecRepAlt}[3]{\left[\begin{matrix} #1 \\ #2 \\ \vdots \\ #3 \end{matrix}\right]}
\newcommand{\mySinrExp}[2]{#1_{_{_{#2}}}}
% \newcommand{\myGamma}[2]{\mathcal{G}(#1,#2)}
\newcommand{\myGamma}[2]{\text{Gamma}(#1,#2)}
\newcommand{\myNoiseVar}{\sigma_n^2}
\newcommand{\myMin}{\text{min}}
\newcommand{\myMax}{\text{max}}
\newcommand{\myEye}{\mathbf{I}}
\newcommand{\myExp}{\mathbb{E}}
\newcommand{\myInputCovMat}{\myVec{R}_{\myVec{x}\myVec{x}}}
\newcommand{\myTrace}[1]{\text{Tr}\left[#1\right]}
\newcommand{\mySubB}{\text{SB}}
\newcommand{\myUE}{\text{UE}}
\newcommand{\SubBW}{\text{SB}_{_{\text{BW}}}}
\newcommand{\SubCBW}{\text{SC}_{_{\text{BW}}}}
\newcommand{\TTIDuration}{\text{T}_{_{\text{tti}}}}
\newcommand{\SinrNoma}{\text{SINR}}
\newcommand{\RateNoma}{\text{R}}
\newcommand{\IntraCellTax}{T_{_{\text{IntraCell}}}}
\newcommand{\InterCellTax}{T_{_{\text{InterCell}}}}
\newcommand{\cnr}{\text{G}}
\newcommand{\PropMatrix}{\mathbf{G}}
\newcommand{\SmallScaleGainMat}{\mathbf{H}}
\newcommand{\LargeScaleGainMat}{\mathbf{D}}
\newcommand{\myDet}{\text{det}}
\newcommand{\ComplexNorm}[2]{\mathcal{CN}\left(#1,#2\right)}
\newcommand{\DetectedSyms}{\myVec{t}}
\newcommand{\myVar}{\text{Var}}
\newcommand{\myCosInt}[1]{\text{Ci}\left(#1\right)}
\newcommand{\mySinInt}[1]{\text{Si}\left(#1\right)}
\newcommand{\dP}[1]{\mathbb{P}\left[#1\right]}
\newcommand{\dE}[2]{\mathbb{E}_{#2}\left[#1\right]}
\newcommand{\nc}[1]{\mathcal{#1}}



% \section{Description of the Hamiltonian Monte Carlo algorithm}
% \label{sec:descr-hamilt-monte}
% 
%However, it is important to note
%that the invariance of π for this kernel is not a sufficient condition
%for the convergence of algorithm.
% However the invariance of
% $\pi$ for this kernel is not a sufficient condition for its convergence. 
% \alain{put a sentence on the fact that most of HMC versions, the
%   invariance is checked and it is not enough for the convergence of
%   the algorithm} 





%%% Local Variables:
%%% mode: latex
%%% TeX-master: "main"
%%% End:


\section{Ergodicity of the HMC algorithm}
\label{sec:ergodicity-hmc}


For $h >0$ and $T \in \nset^*$, consider the Markov kernel $\Pkerhmc[h][T]$ associated with the Markov chain of the HMC algorithm $(Q_k)_{k \in \nset}$, given for all $\q \in \rset^d$ and $\eventA \in \borelSet(\rset^d)$ by
\begin{align}
  \Pkerhmc[h][T](\q, \eventA) &= \int_{\rset^d} \indi{\eventA}{\Phiverletq[h][T](\q,\tilde{\p})} \ \alphaacc\defEns{(\q,\tilde{\p}),\Phiverlet[h][T](\q,\tilde{\p})}\frac{\rme^{-\norm[2]{\tilde{\p}}/2}}{ (2 \uppi)^{d/2}}  \rmd \tilde{\p}
                                  \nonumber
  \\
&\qquad + \updelta_{\q}(\eventA)  \,   \int_{\rset^d}  \parentheseDeux{1-\alphaacc\defEns{(\q,\tilde{\p}),\Phiverlet[h][T](\q,\tilde{\p})}} \frac{\rme^{-\norm[2]{\tilde{\p}}/2}}{ (2 \uppi)^{d/2}}  \rmd \tilde{\p} \eqsp,
 \label{eq:def_kernel_hmc}
\end{align}
where $\Phiverletq[h][T]$, $\Phiverlet[h][T]$ and $\alphaacc$ are defined by \eqref{eq:def_Phiverlet}-\eqref{eq:def_Phiverletq} and \eqref{eq:def_acc_ratio} respectively.
In this Section, we establish conditions upon which the Markov kernel $ \Pkerhmc[h][T]$ is irreducible or
(Harris) recurrent. % Not surprisingly these conditions imply regularity
%conditions and control of the tails of the target distribution $\pi$.  % Our results are established for
% the marginal chain for which $\pi$ is invariant, but irreducibility
% for the Markov kernel associated with the position and momentum is
% crucial for some variants of HMC, in particular when the process
% associated with the position is no longer Markov, (see
% \cite[Proposition 3.7]{bou:sanz:2017}).
For
$\expozero \in \ccint{0,1}$, we consider the following assumption on
the potential $\F$.

\begin{assumption}[$\expozero$]
  \label{assum:regOne}
  $\F$ is continuously differentiable and
  \begin{enumerate}[label=(\roman*)]
  \item
  \label{assum:regOne_a}
 there exists $\constzero > 0$  such that for all $\q,x \in \rset^d$,
\begin{equation}
\norm{\nabla \F(\q) - \nabla \F(x)} \leq \constzero\norm{\q-x} \eqsp.
  \end{equation}
\item    \label{assum:regOne_b}
there exists $\constzeroT \geq 0$  such that for all $\q \in \rset^d$,
\begin{equation}
%\label{eq:bound_nabla_F_assum_reg_zero}
  \norm{\nabla \F(\q)} \leq \constzeroT\defEns{ 1 + \norm{\q}^{\expozero}} \eqsp.
\end{equation}
  \end{enumerate}
\end{assumption}


% For all $T \in \nset^*$, define $\gpertub[h][T] : \rset^d \times \rset^d \to \rset^d$ for all $(\q,\p) \in \rset^d \times \rset^d$ by
% \begin{equation}
%   \gpertub[h][T](\q,\p) =   \Phiverletq[h][k](\q,\p) - \q \eqsp.
% \end{equation}
% \Cref{lem:bound_first_iterate_leapfrog} shows that there exists $C
% \geq 0$ such that for all $\q \in \rset^d$,
% We first state our main two results regarding the ergodicity of the Markov kernel
% $\Pkerhmc[h][T]$ for $h \in \rset_+^*$ and $T \in \nset$.
Before going further, we need to briefly recall some definitions pertaining to Markov chains.
Let $\Pker$ be a Markov kernel on $(\rset^d,\borelSet(\rset^d))$. Let $n$ be an integer and $\mu$
be a nontrivial measure on $\borelSet(\rset^d)$. A
set $\Csf \in \borelSet(\rset^d)$ is called a $(n,\mu)$-small set for $\Pker$ if
for all $x \in \Csf$ and $\Asf \in \borelSet(\rset^d)$, $\Pker^n(x, \msa) \geq \mu(\msa)$.
A set $\Asf \in \borelSet(\rset^d)$ is said to be accessible for $\Pker$
  if for all $x \in \rset^d$, $\sum_{i=1}^\infty \Pker^i(x,\Asf) > 0$.
  A non-trivial $\sigma$-finite measure $\mu$ is an irreducibility
  measure of $\Pker$ \iff\ any set $\Asf \in \borelSet(\rset^d)$
  satisfying $\mu(\Asf) >0$ is accessible.  The Markov kernel $\Pker$ is said to be
  irreducible if it admits an accessible small set or equivalently an
  irreducibility measure (in \cite{meyn:tweedie:2009}, our notion of irreducibility  is referred to as $\phi$-irreducibility, where $\phi$ is an irreducibility measure; here irreducibility therefore means $\phi$-irreducibility). $\Pker$ is said to be a \Tkernel~is there exists a kernel $\Tker$ on $\rset^d \times \mcb(\rset^d)$ and a sequence of non-negative numbers $(a_i)_{i \in \nsets}$ satisfying $\sum_{i=1}^{\plusinfty} a_i =1$, such that
  \begin{enumerate*}[label=(\roman*)]
  \item for any $x \in \rset^d$, $\Tker(x, \rset^d) >0$;
  \item for any $\msa \in \mcb(\rset^d)$, $x \mapsto \Tker(x,\msa)$ is lower semi-continuous;
\item for any $x \in \rset^d$, $\msa \in \mcb(\rset^d)$, $\sum_{i=1}^{\plusinfty} a_i \Pker^i(x,\msa) \geq \Tker(x,\msa)$.
  \end{enumerate*}
  $\Tker$ is referred to as a continuous component of $\Pker$.


  Let $(X_n)_{n \geq 0}$ be the canonical chain associated with $\Pker$
  defined on the canonical space $(\Omega,\mathcal{F},(\mathbb{P}_x, x \in \rset^d))$. A
  set $\Asf \in \borelSet(\rset^d)$ is said to be recurrent if for all $x \in \msa$, $\PE_x[N_\msa]= \plusinfty$ where $N_\msa = \sum_{i=0}^{\plusinfty} \1_{\msa}(X_i)$ is the number of visits to $\msa$. The set $\msa$ is Harris recurrent  if for any $x \in \msa$, $\mathbb{P}_x(N_\msa = \plusinfty) = 1$. The Markov kernel $\Pker$ is said to be Harris
  recurrent if all accessible sets are Harris recurrent. In this case, for all $x \in \rset^d$, and all accessible sets $\msa$, $\PP_x(N_\msa = \plusinfty)=1$.

  Define $\vartheta_1 : \rset_+ \to \rset_+$, for any $s \in \rset_+$ by
  \begin{equation}
    \label{eq:def_vartheta_1}
    \vartheta_1(s) = 1+  s/2 + s^2/4\eqsp.
  \end{equation}
\begin{theorem}
  \label{theo:irred_harris}
Assume \Cref{assum:regOne}($\expozero$) for some $\expozero \in \ccint{0,1}$ and that $\F$ is twice continuously
differentiable. Then, for all $T \in \nsets$, and  $h > 0$ satisfying
\begin{equation}
\label{eq:condition-h,T-harris}
 \left[ \{1 + h\constzero^{1/2} \vartheta_1(h\constzero^{1/2}) \}^T - 1 \right] < 1 \eqsp,
\end{equation}
and $q \in \rset^d$, there exists a $C^1(\rset^d,\rset^d)$-diffeomorphism $\tilde{q} \mapsto \Phiverletqi[h][T](q,\tilde{q})$ such that for any $p \in \rset^d$,
\begin{equation}
\label{theo:irred_harris_a}
\text{if $q_T =   \Phiverletq[h][T](q,p)$, defined by \eqref{eq:def_Phiverletq}, then $p = \Phiverletqi[h][T](q,q_T)$} \eqsp.
\end{equation}
Moreover,
\begin{enumerate}[label=(\roman*), wide, labelwidth=!, labelindent=0pt]
\item   \label{theo:irred_harris_b}
The Markov kernel $\Pkerhmc[h][T]$, is a \Tkernel; more precisely, for any $\eventB \in \mcb(\rset^d)$,
\begin{align}
\label{eq:def_kernel_hmc_false_density}
&\Pkerhmc[h][T](q, \eventB) =  \Tker_{h,T}(q,\eventB) \\
&\qquad + \updelta_{q}(\eventB)(2 \uppi)^{-d/2} \int_{\rset^d}  \parentheseDeux{1-\alphaacc\defEns{(q,\tilde{p}),\Phiverlet[h][T](q,\tilde{p})}} \rme^{-\norm{\tilde{p}}^2/2} \rmd \tilde{p} \eqsp,
\end{align}
where the kernel $ \Tker_{h,T}$ is a continuous component of $\Pkerhmc[h][T]$  and is given by
\begin{equation}
  \label{eq:def_tker}
\Tker_{h,T}(q,\eventB)
  =   (2 \uppi)^{-d/2} \int_{\eventB}    \balphaacc(q,\bar{q})\rme^{-\norm{\Phiverletqi[h][T](q,\bar{q})}^2/2} \detj_{\Phiverletqi[h][T](q,\cdot)}(\bar{q})  \rmd \bar{q} \eqsp,
\end{equation}
setting for $q,\tilde{q} \in \rset^d$, $\balphaacc(q,\bar{q}) =  \alphaacc\defEns{(q,\Phiverletqi[h][T](q,\bar{q})),\Phiverlet[h][T](q,\Phiverletqi[h][T](q,\bar{q}))}$ and  $\detj_{\Phiverletqi[h][T](q,\cdot)}(\tilde{q}) = \absLigne{\det(\Jac_{\Phiverletqi[h][T](q,\cdot)}(\tilde{q}))}$.
\item \label{theo:irred_harris_c} The Markov kernel $\Pkerhmc[h][T]$ is irreducible and the Lebesgue measure is an irreducibility measure. Moreover,  $\Pkerhmc[h][T]$ is aperiodic, Harris recurrent and all the compact sets are $1$-small. Therefore, for all $\q \in \rset^d$,
\begin{equation}
\label{eq:harris-theorem}
\lim_{n \to \plusinfty}    \tvnorm{\delta_\q \Pkerhmc[h][T]^n - \pi} = 0 \eqsp.
\end{equation}
\end{enumerate}
\end{theorem}

\begin{proof}
The proof is postponed to \Cref{sec:proof-crefth-harris_0}.
\end{proof}
%\erici{à vérifier, mais ça doit être bon}
For all $h > 0$ and $T \in \nsets$, we have
$\{1 + h \constzero^{1/2} \vartheta_1(h \constzero^{1/2} ) \}^T -1 \leq \rme^{h \constzero^{1/2} T \vartheta_1(h \constzero^{1/2} T)} -1$.
using that  $\vartheta_1$ is nondecreasing.
Then, setting $\bar{S} = c \constzero^{-1/2}$ where $c$ is   the unique positive root  of the equation
$c \vartheta_1(c)  = \log(2)$,  all $T \in \nsets$ and $h  \in \ooint{0,\bar{S}/T}$ satisfy \eqref{eq:condition-h,T-harris}\footnote{Note that conversely, if $h >0$ and $T \in \nsets$ satisfies \eqref{eq:condition-h,T-harris}, necessarily $h \in \oointLigne{0,\constzero^{-1/2}}$ because for any $s > 0$, $\vartheta_1(s) \geq 1$. In addition, since $\rme^{\log(2) s} \leq (1+s)$ for all $s \in \oointLigne{0,1}$, $T$ and $h$ satisfy $hT \leq \tilde{S}= \constzero^{-1/2}$.
}.


% Under \Cref{assum:regOne}, for all $h \in \rset_+^*$, $T \in \nset^*$
% and $x \in \rset^d$, $\Pkerhmc[h][T](x , \{x \}) >0$, which ensures
% that $\Pkerhmc[h][T]$ is aperiodic.
% Note that by \cite[Theorem 13.0.1]{meyn:tweedie:2009}, \Cref{theo:irred_harris} implies
% that for all $T \geq 0$, there exists $\hirr>0$ such that for all $h \in \ocintLigne{0,\hirr}$ and all $\q \in \rset^d$
%   \begin{equation}
% \lim_{n \to \plusinfty}    \tvnorm{\delta_\q \Pkerhmc[h][T]^n - \pi} = 0 \eqsp.
%   \end{equation}




%%% Local Variables:
%%% mode: latex
%%% TeX-master: "main"
%%% End:

In our next result, we relax the second order differentiability
condition on $\F$, and in the case $\beta <1$ we even allow for
arbitrary large values of the step size $h$ and the number of iterations $T$.
The result is less quantitative and the proof
is more involved: we use degree theory for continuous
mapping (the main notions  required in the proof are recalled in  \Cref{sec:defin-usef-results}).
\begin{theorem}\label{theo:irred_D}
Let $h > 0$ and $T \in \nsets$ and assume either
\begin{enumerate}[label=(\alph*)]
\item
\label{theo:irred_D_a}
\Cref{assum:regOne} $(\expozero)$ for some $\expozero \in \coint{0,1}$,
%   \begin{equation}
% \lim_{n \to \plusinfty}    \tvnorm{\delta_q \Pkerhmc[h][T]^n - \pi} = 0 \eqsp.
%  \end{equation}`
\item
\label{theo:irred_D_b}
\Cref{assum:regOne} $(1)$ and that  $T \in \nsets$ and $h > 0$ satisfy \eqref{eq:condition-h,T-harris}.
\end{enumerate}
%Moreover, $\Pkerhmc[h][T]$ is .
Then,
\begin{enumerate}[label=(\roman*)]
\item the HMC kernel $\Pkerhmc[h][T]$ defined by \eqref{eq:def_kernel_hmc} is irreducible, aperiodic, the Lebesgue measure is an
  irreducibility measure and any compact set of $\rset^d$ is  small.
\item $\Pkerhmc[h][T]$ is recurrent and for $\pi$-almost every $q \in \rset^d$,
$\lim_{n \to \plusinfty}    \tvnorm{\delta_q \Pkerhmc[h][T]^n - \pi} = 0$.
\end{enumerate}
\end{theorem}

\begin{proof}
The proof is postponed to \Cref{sec:proof-crefth_irred_D}.
\end{proof}


% \alain{put a sentence on the fact that most of HMC versions, the
%   invariance is checked and it is not enough for the convergence of
%   the algorithm}
To the best of the author's knowledge, the first results regarding the
irreducibility of the HMC algorithm are established in
\cite{cances:legoll:stoltz} under the assumption that $\U$ and
$\norm{\nabla \U}$ are bounded above. Note that these assumptions are
in general satisfied only for compact state space. Irreducibility has
also been tackled in
\cite{livingstone:betancourt:byrne:girolami:2016}: in this work
however, the number of leapfrog steps $T$ is assumed to be random and
independent of the current position and momentum. Under this setting
and additional conditions which in particular imply that the number of
leapfrog steps $T$ is equal to $1$ with positive probability,
\cite{livingstone:betancourt:byrne:girolami:2016} shows that the
kernel associated with the HMC algorithm is irreducible. Under this condition,
the proof is  a direct
consequence of the irreducibility of the MALA algorithm - a mixture of
Markov kernels is irreducible as soon as one component of the mixture
is irreducible; the irreducibility of MALA kernel has been established
in \cite{roberts:tweedie:1996}). Finally, \cite[Proposition
3.7]{bou:sanz:2017} shows that RHMC is irreducible under the condition
that $U$ is at least quadratic.  Note that \Cref{theo:irred_D}
establishes irreducibility of HMC of sub-quadratic potential. However,
leap-frog integrator is not numerically stable for lighter than
Gaussian target density, therefore other kind of integrators should be
used instead, see \eg~\cite[Chapter VI]{hairer:wanner:lubish:2002}. %One possibility would be  to consider a taming strategy as
%proposed in

Note that if $\expozero < 1$, then there is no condition in
\Cref{theo:irred_D} on the step-size for HMC to be ergodic. This
conclusion may at first glance be surprising since if $\pi$ is a
$d$-dimensional Gaussian distribution with covariance matrix $\Sigma$,
then the step-size $h$ has to be chosen smaller than
$2/\sqrt{\lambda_{\mathrm{max}}}$, where $\lambda_{\mathrm{max}}$ is
the largest eigenvalues of $\Sigma$, which is also the Lipschitz
constant of the gradient of the associated potential. If a larger
step-size $h$ is used, the leapfrog integrator is unstable, see
\eg~\cite[Example 3.4, Proposition 3.1]{bou-rabee:sanz-serna:2018},
meaning that the iterates of the algorithm diverge. But the Gaussian
distribution satisfies \Cref{assum:regOne}$(\expozero)$ for
$\expozero=1$ strictly.
% This result is however not that
% surprising because of the acceptance/rejection mechanism inherent to
% Metropolis-Hastings schemes: an MCMC algorithm can be (geometrically)
% ergodic whereas the proposal is null or even transient. The most
% straightforward example is the symmetric random walk Metropolis and
% the Metropolis Adjusted Langevin Algorithm (MALA). For the Metropolis
% algorithm, the proposal kernel is a random walk which is null if
% $ d \leq 2$ and transient otherwise. On the other hand, consider a
% target density $\pi$ of the form \eqref{eq:def_density_pi}, where
% $U: \rset^d \to \rset$ is continuously differentiable but with
% $\lim_{\norm{q}} \normLigne{\nabla U(q)}/ \normLigne{q} = \plusinfty$,
% take for example $U(x) = \norm[4]{x}$. Then the Euler Maruyama
% discretization of the Langevin diffusion associated with $U$ is
% unstable (see \cite[Theorem 3.2]{roberts:tweedie:1996}) for all
% discretization step-size but by \cite[Equation
% 9]{roberts:tweedie:1996:biometrika} and \cite[Corollary
% 2]{tierney:1994}, the associated MALA algorithm is Harris recurrent
% for any step-size, hence is ergodic (the $\phi$-irreducibility is all
% what is needed, since the Markov kernel is reversible and therefore
% admits an invariant distribution, which is unique). This former result
% holds for $U(x) = \norm[2]{x}$ for all $x \in \rset^d$ as well. By
% \cite[Theorem 3.1, Theorem 3.2]{roberts:tweedie:1996}, the
% Euler-Maruyama discretization is geometrically ergodic for step-size
% $h < 1/2$ and transient for $h > 1$, whereas MALA in this case is
% Harris recurrent for all discretization step-size $h$ by the same
% arguments as above.
We illustrate on a numerical example that under \Cref{assum:regOne}$(\expozero)$, for $\expozero <1$,
  the \textit{unadjusted} HMC proposal is in fact numerically stable
  and the HMC algorithm does converge for a step-size $h > 2 /
  \sqrt{\constzero}$, where $\constzero$ is the Lipschitz constant of $\nabla U$.
  In this example, we consider the potential $U : \rset \to \rset$ given
  for all $x \in \rset$ by $U(x) = 2 \defEnsLigne{1+\abs{x}^2}^{3/4}$. Then
  $U'(x) = 3(\abs{x}^2 +1)^{-1/4}x$ and is Lipschitz with constant
  $\constzero = 3$. We then run the unadjusted/adjusted HMC algorithm for a
  step-size $h =1.5 > 2/\sqrt{\constzero} \approx 1.15$ and a number of
  leapfrog-step $T =2$.  We can observe in
  \Cref{fig:experiments_convergence} the convergence of the HMC
  algorithm for the test function $f : q \mapsto \abs{q}^2$.
  \Cref{fig:experiments_stability} illustrates that the
  adjusted/unadjusted HMC are numerically stable even if $h =1.5 > 2/\sqrt{\constzero} \approx 1.15$, since the gradient
  is sub-linear.


\begin{figure}[h]
	\begin{center}		
		\includegraphics[width=12.2cm]{convergence_3_4_D}
\end{center}
	\caption{Convergence of the HMC algorithm for $U(x) = 2
  \defEnsLigne{1+\abs{x}^2}^{3/4}$, $h =1.5 > 2/\sqrt{\constzero}$ and $T=2$. The test function is $f : q \mapsto \abs{q}^2$. The red line indicates the real value of $\int_{\rset} f(q) \rmd \pi(q)$ estimated by numerical integration}
	\label{fig:experiments_convergence}
\end{figure}


\begin{figure}[h]
	\begin{center}
	\begin{tabular}{p{0.1cm}cp{0.1cm}c}
&		\includegraphics[width=5.8cm]{trace_plot_HMC_3_4_D}
		& &
		\includegraphics[width=5.8cm]{trace_plot_UHMC_3_4_D}\\
& (a) & & (b)
	\end{tabular}
\end{center}
	\caption{Trace plots for the adjusted (a) / unadjusted (b) HMC algorithm for $U(x) = 2
  \defEnsLigne{1+\abs{x}^2}^{3/4}$, $h =1.5 > 2/\sqrt{\constzero}$ and $T=2$.}
	\label{fig:experiments_stability}
\end{figure}


Finally, note that our results can be easily extended to the case
where the number of steps is random. We briefly describe the main arguments to obtain such
extension.
Let $(\varpi_i)_{i\in \nset^*}$ be a probability distribution on $\nset^*$ and $(h_i)_{i \in \nset^*}$ be a sequence of  positive real
numbers.  Define the randomized Hamiltonian kernel
$\randomkerhmc_{\mathbf{h},\bfvarpi}$ on $(\rset^d, \borelSet(\rset^d))$ associated with $(\varpi_i)_{i\in \nset^*}$ and
$(h_i)_{i \in \nset^*}$ by
\begin{equation}
\label{eq:randomhmc}
\randomkerhmc_{\mathbf{h},\bfvarpi} = \sum_{i\in \nset^*} \varpi_i \Pkerhmc[h_i][i] \eqsp.
\end{equation}
We denote by $\supp(\bfvarpi)= \set{i \in \nset^*}{\omega_i \ne 0}$ the support of the distribution $\bfvarpi$.
\begin{corollary}
  \label{coro:ergod-hmc-algor}
Let $\expozero \in \ccint{0,1}$ and assume \Cref{assum:regOne}($\expozero$).
Let $(\varpi_i)_{i\in \nset^*}$ be a probability distribution on $\nset^*$, $(h_i)_{i \in \nset^*}$ be a sequence of  positive real
numbers, and $\randomkerhmc_{\mathbf{h},\bfvarpi}$ be the randomized Hamiltonian kernel associated with $(\varpi_i)_{i\in \nset^*}$ and
$(h_i)_{i \in \nset^*}$.
\begin{enumerate}[label=(\alph*)]
\item
\label{coro:ergod-hmc-algor_a}
Assume that $\F$ is twice continuously and there exists $i \in \nset^*$ such that   $ [ \{1 + h_i\constzero^{1/2} \vartheta_1( h_i \constzero^{1/2}) \}^i - 1 ] < 1$ and  $\varpi_i > 0$   where $\vartheta_1$ is given by  \eqref{eq:def_vartheta_1}. Then the conclusions  of \Cref{theo:irred_harris}-\ref{theo:irred_harris_c} hold for $\randomkerhmc_{\mathbf{h},\bfvarpi}$.
\item
\label{coro:ergod-hmc-algor_b}
 If $\expozero \in \coint{0,1}$, then the conclusions of \Cref{theo:irred_D}-\ref{theo:irred_D_a} hold for $\randomkerhmc_{\mathbf{h},\bfvarpi}$.
\item
\label{coro:ergod-hmc-algor_c}
 If $\expozero = 1$ and there exists $i \in \supp(\bfvarpi)$ such that  $ [ \{1 + h_i \constzero^{1/2} \vartheta_1(h_i \constzero^{1/2}) \}^i - 1 ] < 1$, then the conclusions of \Cref{theo:irred_D}-\ref{theo:irred_D_b} hold for $\randomkerhmc_{\mathbf{h},\bfvarpi}$.
\end{enumerate}
\end{corollary}

\begin{proof}
  \ref{coro:ergod-hmc-algor_a} follows from \Cref{theo:irred_harris} and \Cref{propo:harris_rec}. \ref{coro:ergod-hmc-algor_b} and \ref{coro:ergod-hmc-algor_c} are straightforward applications of \Cref{theo:irred_D}.
\end{proof}

 % $\mathbb{P}(T=1) >0$ and there exists $s
% >0$ such that for all $q_0 \in
% \rset^d$,$\mathbb{E}[\rme^{s\Phiverletq[h][T](q_0,p_0)}]< \plusinfty$,
% where $\Phiverletq[h][T]$ is defined in \eqref{eq:def_Phiverletq} and
% $p_0$ is a standard Gaussian random variable, .

% Note that by \cite[Theorem 14.0.1]{meyn:tweedie:2009}, under
% \Cref{assum:regOne}($\expozero$) for $\expozero \in \coint{0,1}$,
% \Cref{theo:irred_D} implies that for all $T \in \nset^*$ and $h >0$, for
% $\pi$-almost every $q \in \rset^d$,
%   \begin{equation}
% \lim_{n \to \plusinfty}    \tvnorm{\delta_q \Pkerhmc[h][T]^n - \pi} = 0 \eqsp.
%   \end{equation}

% \begin{corollary}\label{co:irr}
% Assume  \Cref{assum:regOne} $(\expozero)$.
% \begin{enumerate}[label=(\alph*)]
% \item
% \label{item:co:irr_1}
% If $\expozero \in \coint{0,1}$ for all $h>0$ and $T \in
%   \nset^*$, $\Pkerhmc[h][T]$ is irreducible with respect to the Lebesgue
%   measure and therefore is ergodic: for all $x \in \rset^d$,
%   \begin{equation}
% \lim_{n \to \plusinfty}    \tvnorm{\delta_x \Pkerhmc[h][T]^n - \pi} = 0 \eqsp.
%   \end{equation}
% \item
% \label{item:co:irr_2}
% If $\expozero = 1$, there exists $\hirr>0$ such that for all $h \in \ocintLigne{0,\hirr}$ and $T \in \nset^*$, $\Pkerhmc[h][T]$ is irreducible with respect to the Lebesgue  measure and therefore is ergodic.
% \end{enumerate}
% \end{corollary}

% \begin{proof}
%   It is a direct consequence of \Cref{pr:small} and \cite[Corollary 2]{tierney:1994}.
% \end{proof}

%%% Local Variables:
%%% mode: latex
%%% TeX-master: "main"
%%% End:



\section{Geometric ergodicity of HMC}
\label{sec:geom-ergod-hmc}



In this section, we give conditions on the potential $\F$ which imply that
the HMC kernel \eqref{eq:def_Phiverletq} converges geometrically fast to its invariant distribution.
Let $V: \rset^d \to \coint{1,\plusinfty}$ be a measurable function and $\Pker$
be a Markov kernel on $(\rset^d,\borelSet(\rset^d))$. The Markov kernel $\Pker$ is said to
be $V$-uniformly geometrically ergodic if $\Pker$ admits an invariant probability $\pi$
and there exists $\rho \in \coint{0,1}$
and $\varsigma \geq 0$ such that for all $\q \in \rset^d$ and $k \in \nset^*$,
\begin{equation}
  \Vnorm[V]{\Pker^k(\q,\cdot)-\pi} \leq \varsigma \rho^{k} V(\q) \eqsp.
\end{equation}
By \cite[Theorem 16.0.1]{meyn:tweedie:2009}, if $\Pker$ is aperiodic, irreducible and satisfies a Foster-Lyapunov drift condition, \ie~there exists a small set $\Csf$ for $\Pker$, $\lambda \in \coint{0,1}$ and $b < \plusinfty$ such that for all $\q \in \rset^d$,
\begin{equation}
\label{eq:foster-lyapunov}
\Pker V  \leq \lambda V + b \1_{\Csf} \eqsp,
\end{equation}
then $\Pker$ is $V$-uniformly geometrically ergodic. If a function $V : \rset^d \to \coint{1,\infty}$
satisfies \eqref{eq:foster-lyapunov}, then $V$ is said to be a Foster-Lyapunov function for $\Pker$.
We first give an elementary condition to establish the $V$-uniform geometric
ergodicity for a class of generalized Metropolis-Hastings  kernels which includes HMC kernels as a particular example.


Let $\Kker$ be a proposal kernel on $(\rset^d, \borelSet(\rset^{2d}))$ and $\alphagen : \rset^{3 d } \to
\ccint{0,1}$ be an acceptance probability, assumed to be Borel measurable. Consider the Markov kernel $\Pker$ on $(\rset^d,\borelSet(\rset^d))$ defined for all $\q \in \rset^d$ and
$\eventA \in \borelSet(\rset^d)$ by
\begin{equation}
\label{eq:def_kenel_MH}
  \Pker(\q,\eventA) = \int_{\rset^{2d}} \1_{\eventA}(\projq(z)) \alphagen(\q,z) \Kker(\q, \rmd z )
+ \updelta_{\q}(\eventA) \int_{\rset^{2d}} \defEns{1- \alphagen(\q,z) }\Kker(\q, \rmd z)  \eqsp,
\end{equation}
where $\projq : \rset^{d} \times \rset^d \to \rset^d$ is the canonical projection onto the first
$d$ components.
For $h \in \rset^*_+$ and $T \in \nset^*$, $\Pkerhmc[h][T]$ corresponds to $\Pker$ with
$\Kker$ and $\alphagen$  given for all $\q,\p,\x \in \rset^d$ and $\Bsf \in
\borelSet(\rset^{2d})$ respectively by
\begin{align}
\label{eq:def_Pker_proposition_double}
  \PkerhmcD[h][T](\q,\Bsf) &= (2\uppi)^{-d/2}\int_{\rset^{d}} \1_{\Bsf}\parenthese{\Phiverletq[h][T](\q,\tilde{\p}),\tilde{\p}} \rme^{-\norm{\tilde{\p}}^2/2} \rmd \tilde{\p} \eqsp, \\
\label{eq:def_alpha_acc_tilde_hmc}
\tildeAlphaacc(\q,(\tilde{\q},\tilde{\p})) & =
\begin{cases}
\alphaacc\defEns{(\q,\tilde{p}),\Phiverlet[h][T](\q,\tilde{p})} \eqsp, & \text{if}\, \tilde{q}= \Phiverletq[h][T](\q,\tilde{\p}) \eqsp, \\
0 & \text{otherwise} \eqsp,
\end{cases}
\end{align}
where  $\Phiverlet[h][T]$, $\Phiverletq[h][T]$ and $\alphaacc$  are  defined in   \eqref{eq:def_Phiverlet}, \eqref{eq:def_Phiverletq} and \eqref{eq:def_acc_ratio}, respectively. Let $\Vgeo : \rset^d \to \coint{1,\plusinfty}$ be a
\emph{norm-like}  function,  \ie\ a measurable function such that for all $M \in \rset_+$, the level sets $\set{\q \in \rset^d}{\Vgeo(\q) \leq M}$ are compact. Note that if $\Vgeo$ is norm-like, for any $M \in \rset_+$, $\set{\q \in \rset^d}{\Vgeo(\q) \leq M}^{\complementary}$ is non-empty.   The function $\Vgeo$ naturally extends on $\rset^{2d}$ by
setting for all $(\q,\p) \in \rset^{2d}$, $\Vgeo(\q,\p) = \Vgeo(\q)$.
For all $\q \in \rset^d$, define:
\begin{equation}
\label{eq:def_rej_ballV}
%\begin{aligned}
  \rejectregion(\q) = \defEns{z \in \rset^{2d} \, , \, \alphagen(\q,z) < 1  } \eqsp, \,
    \ballV(\q) = \defEns{z \in \rset^{2d} \, , \, \Vgeo(\projq(z)) \leq \Vgeo(\q) } \eqsp.
%\end{aligned}
\end{equation}
The set $\rejectregion(\q)$ is the potential rejection region.
Our next result gives a condition on $\Kker$ and $\alphagen$ which
implies that if $V$ is a Foster-Lyapunov function for $\Kker$ then
$\Pker$ satisfies a Foster-Lyapunov drift condition as well. This
result is inspired by \cite[Theorem~4.1]{roberts:tweedie:1996}, which is used to show the $V$-uniform geometric ergodicity of the MALA algorithm.
\begin{proposition}
\label{propo:geo_drift_MH}
  Let $\Vgeo : \rset^d \to \coint{1,\plusinfty}$ be a norm-like  function.
  Assume moreover that there exist  $\lambdageo \in \coint{0,1}$ and $\bgeo \in \rset_+$ such that
  \begin{equation}
  \label{eq:assum:geo_ergo_1}
  \Kker \Vgeo \leq  \lambdageo \Vgeo + \bgeo \eqsp.
  \end{equation}
and
\begin{equation}
\label{eq:assum:geo_ergo_2}
  \lim_{M \to \plusinfty} \sup_{\set{\q \in \rset^d}{\Vgeo(\q) \geq M}} \Kker(\q,\rejectregion(\q) \cap \ballV(\q)) = 0  \eqsp.
\end{equation}
 Then there exist  $\lambdageotilde \in \coint{0,1}$ and $\bgeotilde \in \rset_+$ such that
 $\Pker \Vgeo \leq  \lambdageotilde \Vgeo + \bgeotilde$ where $\Pker$ is given by \eqref{eq:def_kenel_MH}.
\end{proposition}
\begin{proof}
The proof is postponed to \Cref{sec:proof-crefpr}.
\end{proof}
We show below that under appropriate conditions, the proposal kernel $\PkerhmcD[h][T]$ and
the acceptance probability $\tildeAlphaacc$ given by \eqref{eq:def_Pker_proposition_double} and
\eqref{eq:def_alpha_acc_tilde_hmc} satisfy the conditions of
\Cref{propo:geo_drift_MH} which imply that the HMC kernel
$\Pkerhmc[h][T]$ is $V$-uniformly geometrically ergodic. %We assume in the following conditions.
For $\m \in \ocint{1,2}$, consider the following assumption:



\begin{assumption}[$m$]
  \label{assum:potential:c}
There exist $\constthree \in \rset^*_+$ and $\constfour \in \rset$ such that for all $\q \in \rset^d$,
  \begin{equation}
    \ps{\nabla \F(\q)}{\q} \geq \constthree \norm{\q}^{m} -\constfour \eqsp.
  \end{equation}
\end{assumption}
For all $\a \in \rset_+^*$ and $\q \in \rset^d$, define
\begin{equation}
\label{eq:def_Va}
\Vdrifta[a] (\q) = \exp(\a \norm{\q}) \eqsp.
\end{equation}
\begin{proposition}
\label{lem:drift_uhmc}
%   \begin{equation}
% \label{eq:hyp:drift_uhmc}
%     \liminf_{\q \to \plusinfty} \ps{\nabla \F(\q)}{\q}/ \norm{\q}^{\expozero+1} >0 \eqsp.
%   \end{equation}
 % Let $T \in \nset^*$. %Then the following holds
\begin{enumerate}[label=(\alph*)]
\item   \label{lem:drift_uhmc_1}
 Assume   \Cref{assum:regOne}$(m-1)$ and  \Cref{assum:potential:c}$(\m)$ for some $\m \in \ooint{1,2}$. Then, for all $T \in \nsets$,  $h \in \rset^*_+$, and $\a \in \rsetep$, there exist $\lambda \in \coint{0,1}$ and $\b \in \rsetp$ such that
  \begin{equation}
    \label{eq:drift_lem}
      \PkerhmcD[h][T] \Vdrifta[\a] \leq \lambda  \Vdrifta[\a] + \b \eqsp.
  \end{equation}
\item
\label{lem:drift_uhmc_2}
 Assume   \Cref{assum:regOne}$ (1)$ and  \Cref{assum:potential:c}$ (2)$.  Let $\bar{S} > 0$ be such that $\Theta(S) < \constthree$ for any $S \in \ocint{0,\bar{S}}$, where
\begin{align}
\label{eq:definition-function-C}
  \Theta(s)&= 2 \constzero^{1/2} \vartheta_2(s) \{ \rme^{\constzero^{1/2} s \vartheta_1(\constzero^{1/2} s)} - 1\} \\
  & \qquad  \qquad + 6 s^2 \left(   \constzeroT ^2  +  \constzero \vartheta_2^2(s) \{ \rme^{\constzero^{1/2} s \vartheta_1(\constzero^{1/2} s)} - 1\}^2\right) \eqsp.
\end{align}
Then, for all $a \in \rsetep$,  $T \in \nsets$ and  $h \in \ocint{0,\bar{S}/T}$,  there exist  $\lambda \in \coint{0,1}$ and $\b \in \rsetp$ which satisfy \eqref{eq:drift_lem}.
\end{enumerate}
\end{proposition}
\begin{proof}
  The proof is postponed to \Cref{sec:proof-crefl-2}
\end{proof}
We now derive sufficient conditions under which the condition \eqref{eq:assum:geo_ergo_2} of
\Cref{propo:geo_drift_MH} is satisfied.

\begin{assumption}[$\m$]
\label{assum:potential}
\begin{enumerate}[label = (\roman*)]
\item \label{assum:potential:a}
$\F \in C^3(\rset^d)$  and there exists $\constone \in \rset_+^*$ such that for all $\q \in \rset^d$ and $k=2,3$:
\begin{equation}
%\label{eq:10}
\norm{D^k \F(\q)}\leq \constone \defEns{1+\norm{\q}}^{\m-k} \eqsp.
\end{equation}
\item \label{assum:potential:b}
There exist $\consttwo \in \rset_+^* $ and $\rhtwo \in \rset^+$ such that for all $\q \in \rset^d$, $\norm{q}\geq \rhtwo$,
\begin{equation}
%\label{eq:11}
D^2\F(\q)\defEns{ \nabla \F(\q)\otimes  \nabla \F(\q)}  \geq \consttwo \norm{\q}^{3\m-4} \eqsp.
\end{equation}
  \end{enumerate}
\end{assumption}

It is easily checked that under \Cref{assum:potential}, the results of \Cref{sec:ergodicity-hmc} can be applied, \ie~$\nabla \F$ satisfies \Cref{assum:regOne}($\m-1$); see \Cref{lem:grad_Lip_F}.

Condition \Cref{assum:potential:c}$(m)$ and \Cref{assum:potential}$(m)$ are satisfied by power functions $\q \mapsto c\norm{\q}^\m$. More generally, they are satisfied by $\m$-homogeneously quasiconvex functions with convex level sets  outside a ball and by  perturbations of such functions.

We say that a function $\F_0$ is $m$-homogeneous quasi-convex
outside a ball of radius $\Rexp$ if the following conditions are satisfied:
\begin{enumerate}[(QC-1)]
\item for all $t \geq 1$ and $q \in \rset^d$, $\norm{\q}\geq \Rexp$, $\F_0(t \q)= t^\m \F_0(\q)$.
% $$
% \F_0(t \q)= t^\m \F_0(\q) \eqsp.
% $$
\item for all $\q \in \rset^d$, $\norm{\q} \geq \Rexp$, the level sets $\{ \x\, :\, \F_0(\x) \leq \F_0(\q)\}$ are convex.
\end{enumerate}
\begin{proposition}
\label{le:convex}
Let $m \in \ccint{1,2}$  and $\Rexp \in \rset_+$.  Assume that the potential $\F$ may be decomposed as
$$
\F(\q)=\F_0(\q)+G(\q) \eqsp, \quad \text{$\q\in \rset^d$, $\norm{\q} \geq \Rexp$} \eqsp,
$$
where the functions $\F_0,G \in C^3(\rset^d)$ satisfy the following two conditions:
  \begin{enumerate}[(A)]
  \item $\F_0$ is $\m$-homogeneously quasiconvex outside a ball of radius $\Rexp$ and $\lim_{\norm{\q} \to \plusinfty} \F_0(\q)=\infty$.
\label{le:convex:a}
\item
\label{le:convex:b}
For $k=2,3$, $\lim_{\norm{\q} \to \plusinfty}\normop{D^k G(\q)}/  \norm{\q}^{\m-k}= 0$.
% \begin{equation}%\label{eq:lower}
% \lim_{\norm{\q} \to \plusinfty}\normop{D^k G(\q)}/  \norm{\q}^{\m-k}= 0 \eqsp.
% \end{equation}
  \end{enumerate}
Then $\F$ satisfies  \Cref{assum:potential:c}$(m)$ and  \Cref{assum:potential}$(\m)$.
\end{proposition}
\begin{proof}
The proof is postponed to \Cref{sec:proof-crefle:convex}.
\end{proof}

To show that the condition \eqref{eq:assum:geo_ergo_2} of
\Cref{propo:geo_drift_MH} is satisfied under
\Cref{assum:potential}$(m)$, we rely on the following important result which implies that the probability of accepting a move goes to 1 as $\norm{q} \to \infty$.
\begin{proposition}
  \label{propo:accept} Assume
  \Cref{assum:potential}$(m)$ for some $\m \in \ocint{1,2}$. Let $\gamma \in \ooint{0,\m-1}$.
  \begin{enumerate}[label=(\alph*)]
  \item
  \label{propo:accept_1}
  If $\m\in (1,2)$, for all $T \in \nsets$, $h \in \rset_+^*$, there exists $R_{\Ham} \in \rset_+$ such that for
  all $\q_0,\p_0 \in \rset^d$, $\norm{q_0} \geq R_{\Ham}$ and
  $\norm{p_0} \leq \norm{\q_0}^{\gamma}$, $ \Ham(\Phiverlet[h][T](q_0,p_0)) -
  \Ham(q_0,p_0) \leq 0$.
\item
  \label{propo:accept_2}
  If $\m=2$,   there exists $\bar{S} >0$ such that for any $T \in \nsets$ and $h \in \ocint{0, \bar{S}/T^{3/2}}$,  there exists $R_{\Ham} \in \rset_+$ satisfying for all $\q_0,\p_0 \in \rset^d$, $\norm{q_0} \geq R_{\Ham}$ and
  $\norm{p_0} \leq \norm{\q_0}^{\gamma}$, $ \Ham(\Phiverlet[h][T](q_0,p_0)) -
  \Ham(q_0,p_0) \leq 0$.
  \end{enumerate}
\end{proposition}


\begin{proof}
  The proof is postponed to  \Cref{sec:proof-crefth}.
\end{proof}
%This behavior comes a bit as a surprise. It


This result  means that far in the tail the HMC proposal are "inward".
We illustrate the result of \Cref{propo:accept}-\ref{propo:accept_1}
in \Cref{fig:H_behaviour} for $U$ given by $\q \mapsto
(\norm[2]{\q}+\delta)^{\kappa}$ for $\kappa=3/4$, $h = 0.9$ and
$p_0 \in \rset^d$, $\norm{p_0}=1$. Note that this potential satisfies
the condition of the proposition. We can observe that choosing the different
initial conditions $q_0$ with increasing norm imply that $\tilde{T} =
\max\{k \in \nset ; \Ham(\Phiverlet[h][k](q_0,p_0)) - \Ham(q_0,p_0)
<0\}$ increases as well.

 \begin{figure}[h]
   \centering
   \includegraphics[scale=0.3]{hamiltonian_behaviour}
   \caption{Behaviour of $(\Ham(\Phiverlet[h][k](q_0,p_0)))_{k \in \{0,\ldots,T\}}$ for different initial conditions $q_0$.}
   \label{fig:H_behaviour}
 \end{figure}


 However, in the case $m=2$, \Cref{propo:accept}-\ref{propo:accept_2} only implies that the HMC proposal is inward only if the step size $h$ is sufficiently small with respect to the number of leapfrog step $T$, \ie~is of order $\bigO(T^{-3/2})$. To relax this condition, we strengthen \Cref{assum:potential}($2$) by assuming that $U$ is a smooth perturbation of a quadratic function.
 \begin{assumption}
   \label{ass:pertub}
   There exist $\tilde{U} : \rset^d \to \rset$, continuously differentiable, and  a positive definite matrix $\Sigmabf$ such that
   $U(q) = \ps{\Sigmabf q}{q}/2 + \tilde{U}(q)$ and there exist $\constfive \geq 0$ and  $\varrho \in \coint{1,2}$  such that for any $q,x \in \rset^d$,
   \begin{align}
     \absLigne{\tilde{U}(q)} &\leq \constfive(1+\norm[\varrho]{q}) \eqsp, \quad  \normLigne{\nabla \tilde{U}(q)} \leq \constfive(1+\norm[\varrho-1]{q}) \eqsp,\\  \qquad \qquad \qquad & \normLigne{\nabla \tilde{U}(q) - \nabla \tilde{U}(x)} \leq \constfive \norm{q-x} \eqsp.
   \end{align}
 \end{assumption}
Note that it is straightforward to check that under \Cref{ass:pertub}, the conditions \Cref{assum:regOne}$(1)$ and  \Cref{assum:potential:c}$(2)$ hold.

 \begin{proposition}
  \label{propo:accept_pertub} Assume  \Cref{ass:pertub}  and let $\gamma \in \ooint{0,1}$.
There exists a  constant $\bar{S} >0$ such that for all $T \in \nsets$, $h \in \ocint{0,\bar{S}/T}$, there exists $R_{\Ham} \in \rset_+$ such that for
  all $\q_0,\p_0 \in \rset^d$, $\norm{q_0} \geq R_{\Ham}$ and
  $\norm{p_0} \leq \norm{\q_0}^{\gamma}$, $ \Ham(\Phiverlet[h][T](q_0,p_0)) -
  \Ham(q_0,p_0) \leq 0$.
\end{proposition}
\begin{proof}
  The proof is postponed to  \Cref{sec:proof-crefth_accept_2}.
\end{proof}
We now can establish the geometric ergodicity of the HMC sampler.
\begin{theorem}
  \label{theo:geoErg}
  \begin{enumerate}[label=(\alph*)]
  \item
  \label{item:theo_1}
 If   \Cref{assum:potential:c}$(\m)$ and  \Cref{assum:potential}$(m)$ hold for some $\m\in (1,2)$, then for all $a \in \rset_+^*$,  $T \in \nset^*$ and $h > 0$, the HMC kernel $\Pkerhmc[h][T]$ is $\Vdrifta[\a]$-uniformly geometrically ergodic, where $\Vdrifta[a]$ is defined by \eqref{eq:def_Va}.
\item
  \label{item:theo_2}
  If  \Cref{assum:potential:c}$(2)$ and \Cref{assum:potential}$(2)$ hold, then there exists $\bar{S}>0$ such that for all  $a \in \rset^*_+$, $T \in \nset^*$ and $h \in \ooint{0,\bar{S}/T^{3/2}}$, $\Pkerhmc[h][T]$ is $\Vdrifta[\a]$-uniformly geometrically ergodic.
\item \label{item:theo_3}
  If \Cref{ass:pertub} holds, then there exists $\bar{S}>0$ (depending only on $\Sigmabf$ and $\constfive$) such that for all $a \in \rset^*_+$, $T \in \nset^*$ and $h \in \ooint{0,\bar{S}/T}$, $\Pkerhmc[h][T]$ is $\Vdrifta[\a]$-uniformly geometrically ergodic.
  \end{enumerate}
\end{theorem}
\begin{proof}[Proof of \Cref{theo:geoErg}]
  It is enough to consider \ref{item:theo_1} as the proof
  of  \ref{item:theo_2} and \ref{item:theo_3} follows exactly the same lines
  taking $\bar{S}$ small enough.  \Cref{lem:drift_uhmc} shows that for all $T \in \nsets$,  $h \in \rset^*_+$, and $\a \in \rsetep$, there exist $\lambda \in \coint{0,1}$ and $\b \in \rsetp$ such that the Foster-Lyapunov drift condition
  $\PkerhmcD[h][T] \Vdrifta[\a] \leq \lambda  \Vdrifta[\a] + \b$ is satisfied.
  By \Cref{propo:accept}, there exists $R_{\Ham} \geq 0$ such that for all $\q \in \rset^d$, $\norm{\q} \geq R_{\Ham}$,
\begin{equation}
\label{eq:proofpgeo_erg_0}
  \int_{\rejectregion(q)}   \PkerhmcD[h][T](\q, \rmd z ) \leq (2\uppi)^{-d/2} \int_{\{ \norm{\p} \geq \norm{q}^{\gamma}\} } \rme^{-\norm{p}^2/2} \rmd p \eqsp,
\end{equation}
for $\gamma \in \ooint{0,m-1}$ where $\rejectregion(\q) = \set{z \in \rset^{2d}}{\tildeAlphaacc(\q,z) < 1 }$ (see~\eqref{eq:def_alpha_acc_tilde_hmc}), which implies that
\begin{equation}
\label{eq:proof:geo_erg:1}
  \lim_{M \to \plusinfty} \sup_{\norm{\q} \geq M} \int_{\rejectregion(q)} \PkerhmcD[h][T](\q, \rmd z ) = 0 \eqsp,
\end{equation}


Since $\Vdrifta[a]$ is norm-like,  \Cref{propo:geo_drift_MH} implies that  for all $T > 0$ and $h > 0$, there exists $\lambdageotilde$ and $\bgeotilde$ (depending upon $a$, $h$ and $T$) such that
$\Pkerhmc[h][T] \Vdrifta[a] \leq \lambdageotilde \Vdrifta[a] + \bgeotilde$.
For all $M \geq 0$ the level sets $\{ \Vdrifta[a] \leq M\}$ are compact and hence small by \Cref{theo:irred_D}.
\cite[Corollary~14.1.6]{douc:moulines:priouret:2018} then shows that there exists a small set $\msc$, $\check{\lambda} \in \coint{0,1}$ and $\check{b} \in \coint{0,1}$ such that $\Pkerhmc[h][T] \Vdrifta[a] \leq \check{\lambda} \Vdrifta[a] + \check{b} \1_{\msc}$.  Since $\Pkerhmc[h][T]$ is aperiodic, the result follows from \cite[Theorem~15.2.4]{douc:moulines:priouret:2018}.
\end{proof}

We  finally consider the case where the number of leapfrog steps is a random variable
independent of the current state.
\begin{theorem}
\label{coro:geo_ergod-hmc-algor}
\begin{enumerate}[label=(\alph*)]
  \item
  \label{item:geo_ergod-hmc-algor:theo_1}
 If   \Cref{assum:potential:c}$(\m)$ and  \Cref{assum:potential}$(m)$ hold for  $\m\in (1,2)$,  then for all probability distributions $\bfvarpi=(\omega_i)_{i \in \nset^*}$ on $\nset^*$, all sequences $\mathbf{h}= (h_i)_{i \in \nset*}$ of positive numbers,  and $a \in \rset^*_+$, the randomized kernel  $\randomkerhmc_{\mathbf{h},\bfvarpi}$ \eqref{eq:randomhmc} is $\Vdrifta[\a]$-uniformly geometrically ergodic, where $\Vdrifta[a]$ is defined by \eqref{eq:def_Va}.
\item
  \label{item:geo_ergod-hmc-algor:theo_2}
  If  \Cref{assum:potential:c}$(2)$ and \Cref{assum:potential}($2$) hold, then there exists $\bar{S}>0$ such that for all probability distributions $\bfvarpi=(\omega_i)_{i \in \nset^*}$ on $\nset^*$, all sequences $\mathbf{h}= (h_i)_{i \in \nset^*}$ satisfying $\max_{i \in \supp(\bfvarpi)} i^{3/2} h_i \leq \bar{S}$,  and $a \in \rset^*_+$, $\randomkerhmc_{\mathbf{h},\bfvarpi}$ is $\Vdrifta[\a]$-uniformly geometrically ergodic.
\item   \label{item:geo_ergod-hmc-algor:theo_3}
  If \Cref{ass:pertub} holds, then there exists $\bar{S}>0$ (depending only on $\Sigmabf$ and $\constfive$) such that for all  probability distributions $\bfvarpi=(\omega_i)_{i \in \nset^*}$ on $\nset^*$, all sequences $\mathbf{h}= (h_i)_{i \in \nset^*}$ satisfying $\max_{i\in \supp(\bfvarpi)} i h_i \leq \bar{S}$,  and $a \in \rset^*_+$, $\randomkerhmc_{\mathbf{h},\bfvarpi}$ is $\Vdrifta[\a]$-uniformly geometrically ergodic.
\end{enumerate}
\end{theorem}
\begin{proof}
It is enough to consider \ref{item:geo_ergod-hmc-algor:theo_1} as the proofs
of \ref{item:geo_ergod-hmc-algor:theo_2} and \ref{item:geo_ergod-hmc-algor:theo_3} are along the same lines.
Set $a \in \rset_+^*$. It is established in the proof of \Cref{theo:geoErg}  that  for all $i \in \nset^*$
$\Pkerhmc[i][h_i]$ satisfies a Foster-Lyapunov drift condition:
there exists $\check{\lambda}_i \in \coint{0,1}$ and $\check{b}_i < \infty$ such that
$\Pkerhmc[i][h_i] \Vdrifta[a] \leq \lambda_i \Vdrifta[a] + b_i$,
By \Cref{coro:ergod-hmc-algor},   $\randomkerhmc_{\mathbf{h},\bfvarpi}$ is irreducible and aperiodic and all the compact sets are small. We conclude by applying \cite[Theorem~15.2.4]{douc:moulines:priouret:2018}.
\end{proof}

Compared to \cite{livingstone:betancourt:byrne:girolami:2016}, which
establishes geometric ergodicity of the HMC kernel under an implicit
assumption on the behaviour of the acceptance rate, our conditions are
directly verifiable on the potential $U$.

On the other hand, our conditions are different than the one given by
\cite{bou:sanz:2017} to establish the geometric ergodicity of the
idealized randomized HMC, which assumed to exactly solve the Hamiltonian
ODE \eqref{eq:hamil_ode}. These conditions are the following
1)$\int_{\rset^d} \norm[2]{q} \rmd \pi(q) < \plusinfty$, 2) there
exist $C_1 \in \ooint{0,1}$ and $C_2 >0$ such that for all $q \in \rset^d$
\begin{equation}
\label{eq:hyp_boo_rabee_sanz_serna}
  (1/2) \ps{\nabla U(q)}{q} \geq C_1 U(q) + \frac{(\tau^{-1}C_1/4)^2+\tau^{-2}C_1(1-C_1)/4}{2(1-C_1)}\norm[2]{q} -C_2 \eqsp,
\end{equation}
where $\tau>0$ is the duration parameter of the RHMC algorithm.  Note
that these conditions assumed that the target density is lighter than
Gaussian. In comparison, our results can be applied to sub-quadratic
potentials. In addition, it can be shown that HMC is not geometrically
ergodic under \eqref{eq:hyp_boo_rabee_sanz_serna} on the following example associated with the potential defined by  \eqref{eq:def_U_mixture_gaussian} below.

% The condition \eqref{eq:hyp_boo_rabee_sanz_serna} is satisfied if $\pi$ is a
% mixture of $d$-dimensional Gaussian distributions.  We show
% numerically that there is strong evidences which imply that HMC is not
% geometrically ergodic for such examples.
The main difference with the
setting of \cite{bou:sanz:2017} is that HMC has a acceptance/rejection
step and the integrated acceptance ratio
\[ q \mapsto \int_{\rset^d} \alphaacc\defEnsLigne{(\q,\p),\Phiverlet[h][T](\q,\p)}
\rme^{-\norm[2]{p}/2} (2 \uppi)^{-d/2} \rmd p
\]
must not go to $0$ as
$\norm{q}$ goes to $\plusinfty$. This is essentially the reason why
\Cref{assum:potential} differs from \eqref{eq:hyp_boo_rabee_sanz_serna}. Indeed, to show that an
irreducible Markov kernel $\mathrm{P}$ on $(\rset^d, \mcb(\rset^d))$ is not geometrically
ergodic with respect to an invariant measure $\mu$, \cite[Theorem
5.1]{roberts:tweedie:1996:biometrika} states the following sufficient condition
\begin{equation}
  \label{eq:condition_non_geo_ergodicity}
 \mathrm{ess \, sup}_{q \in \rset^d} \mathrm{P}(q, \{q\}) = 1 \eqsp,
\end{equation}
 where
$\mathrm{ess\, sup}$ is taken with respect to $\mu$. Consider then
the target density $\pi$ with potential $U$ given for all $q =(q_1,q_2) \in \rset^2$ by
\begin{equation}
\label{eq:def_U_mixture_gaussian}
  U(q) = -\log(\rme^{-q_1^2 - 5 q_2^2} + \rme^{-5q_1^2 - q_2^2}) \eqsp.
\end{equation}
Note that $U$ satisfies the condition
\eqref{eq:hyp_boo_rabee_sanz_serna}. On the contrary, we may show that
\eqref{eq:condition_non_geo_ergodicity} holds, and therefore HMC is
not geometrically ergodic for such a potential $U$.  However, the
detailed calculations are very technical and not particularly
informative and we prefer to present a numerical evidence that
\eqref{eq:condition_non_geo_ergodicity} holds. Indeed,
\Cref{fig:accept_mix_gaussian} displays numerical computations of the mean acceptance ratio,
$\int_{\rset^2} \alphaacc\defEnsLigne{(\q,\p),\Phiverlet[h][T](\q,\p)}
\rme^{-\norm[2]{p}/2} (2 \uppi)^{-1} \rmd p= 1 -\Pkerhmc[h][T](q,\{q\})$ for $q_1 \in \{200,250,$
$300,350,400,450,500\}$,
$q_2 \in \ccint{q_1+10^{-4},q_1+2\cdot 10^{-4}}$ and $T=1$  which
corresponds to MALA. We can observe that the larger $q_1$, the smaller $1-\Pkerhmc[h][T](q,\{q\})$, which illustrates that  \eqref{eq:condition_non_geo_ergodicity}  holds for the HMC
kernel.

 \begin{figure}[h]
   \centering
   \includegraphics[scale=0.4]{mix_gaussian_1}
   \caption{}
   \label{fig:accept_mix_gaussian}
 \end{figure}


%For  $T \in \nset^*$, $\alpha \geq 0$ and $h_{\Ham} >0$, consider the following assumption.
% \begin{assumption}[$T,\alpha,h_{\Ham}$]
%   \label{assum:diff_ham}
%  There exists  $R_{\Ham}
%   \geq 0$ and such that for all $h \in
%   \ocint{0,h_{\Ham}}$, if
% \begin{equation}
% \label{eq:100}
% |q_0|\geq R_{\Ham} \quad\textrm{and}\quad |p_0|\leq |q_0|^\alpha,
% \end{equation}
% then
% \begin{equation}
% \label{eq:diff_ham_postitive}
%   \Ham(q_T,p_T) - \Ham(q_0,p_0) \geq 0 \eqsp.
% \end{equation}
% \end{assumption}



%%% Local Variables:
%%% mode: latex
%%% TeX-master: "main"
%%% End:



\section{Irreducibility for a class of iterative models}
\label{sec:irred-class-iter}

In this Section we establish the irreducibility of a Markov kernel associated to a random iterative model.
These results are of independent interest.
Let $\hfunb : \rset^d \times \rset^d \to \rset^d$ and $\alphagen: \rset^d \times \rset^d \to \ccint{0,1}$ be Borel measurable
functions and $\phib : \rset^d \to \ccint{0,\plusinfty}$ be a
probability density with respect to the Lebesgue measure.  Consider the Markov kernel $\Pkerb$ defined for all $x \in \rset^d$ and $\eventA \in \borelSet(\rset^d)$ by
\begin{equation}
\label{eq:def_pkerb}
  \Pkerb(x,\eventA) = \int_{\rset^d} \indi{\eventA}{\hfunb(x,z)} \alpha(x,z) \phib(z) \rmd z + \bar{\alpha}(x) \delta_x(\eventA) \eqsp,
\end{equation}
where $\bar{\alpha}(x)= \int_{\rset^d} \alpha(x,z) \phib(z) \rmd z$. 
Define for all $x \in \rset^d$, $\hfunb_x : \rset^d \to \rset^d$ by $\hfunb_x = \hfunb(x,\cdot)$.

First, we give  a result  from geometric measure
theory together with a proof for the reader's convenience, which will be essential for the proof of the statements of this section.  Let
$\ouvert\subset\rset^d$ be an open set and $\Theta: \ouvert\to \rset^d$ be
a measurable function such that there exist $y_0 , \tilde{y}_0 \in \rset^d$ and
$M, \tilde{M} > 0$ satisfying $\ball{\tilde{y}_0}{\tilde{M}} \subset \ouvert$ and
  \begin{equation}
    \label{eq:condition_sigma_finite}
\ball{y_0}{M} \subset \Theta(\ball{\tilde{y}_0}{\tilde{M}}) \eqsp.
  \end{equation}
 Define the measure $\lambda_{\Theta}$ on $(\rset^d,\borelSet(\rset^d))$
  by setting for any $\eventA \in \borelSet(\rset^d)$
\begin{equation}
  \label{eq:push_forward_mes}
\lambda_\Theta (\eventA) \eqdef \Leb\defEns{\Theta^{-1}(\eventA) \cap \ball{\tilde{y}_0}{\tilde{M}}}  \eqsp.
\end{equation}
Note %by \eqref{eq:condition_sigma_finite}
that $\lambda_\Theta$ is  a finite measure. Therefore by the Lebesgue decomposition theorem (see
\cite[Section 6.10]{rudin:1987}) there exist two  measures
$\lambda_\Theta^{(\text{a})}, \lambda_\Theta^{(\text{s})}$ on
$(\rset^d,\borelSet(\rset^d))$, which are absolutely continuous and
singular with respect to the Lebesgue measure on $\rset^d$
respectively, such that $\lambda_\Theta = \lambda_\Theta^{(\text{a})} +
\lambda_\Theta^{(\text{s})}$.
% Note that if for all compact set $\compact \subset \rset^d$, $\Theta^{-1}(\compact)$ is
% compact (\ie~$\Theta$ is a proper function), then $\lambda_\Theta$ is $\sigma$-finite.
\begin{proposition}\label{le:simple}
  Let $\ouvert\subset\rset^d$ be open and $\Theta: \ouvert\to \rset^d$ be a Lipschitz function
  satisfying \eqref{eq:condition_sigma_finite}.
 For any version  $\phi_\Theta$ of the density of $\lambda_\Theta^{(\text{a})}$ with respect to the Lebesgue
  measure on $\rset^d$, it holds
$$
\phi_\Theta(y)\geq \1_{ \ball{y_0}{M}}(y) \norm{\Theta}_{\Lip}^{-d}\eqsp, \quad \text{$\Leb$-a.e.}
$$
\end{proposition}
\begin{proof}
  Denote by $L = \norm{\Theta}_{\Lip}$. Let $y\in \ball{y_0}{M}$. By  \eqref{eq:condition_sigma_finite}, we may pick
  $z \in \ball{\tilde{y}_0}{\tilde{M}}$ such that $\Theta(z) = y$. Let
  $\delta_0 >0$ be such that $\ball{z}{\delta_0/L} \subset
  \ball{\tilde{y}_0}{\tilde{M}}$.  Since $\Theta$ is
  Lipschitz continuous,  for all
  $\delta \in \rset_+^*$,
  $\Theta(\ball{z}{\delta/L} \cap \open)\subset \ball{y}{\delta}$. Hence, for all
  $\delta \in \ocint{0,\delta_0}$, we have
  $$
\lambda_\Theta(\ball{y}{\delta})\geq
  L^{-d}\Leb(\ball{z}{\delta}) =   L^{-d}\Leb(\ball{y}{\delta}) \eqsp.
$$
 The claim follows from the differentiation theorem
  for measures, see \cite[Theorem 7.14]{rudin:1987}.
\end{proof}

We can now state our main results. Let $\rassG,\MassG \in \rset^*_+$ and $y_0 \in \rset^d$.
Consider the following assumptions.
\begin{assumptionG}
\label{assumG:phi}
$\phib$ and $\alphagen$ are lower semicontinuous and positive on $\rset^d$ and $\rset^{2d}$ respectively.
\end{assumptionG}

% \begin{assumptionG}
% \label{assumG:alpha}
%  is lower semicontinuous and positive on .
% \end{assumptionG}

\begin{assumptionG}[$\rassG,y_0,\MassG$]
  \label{assumG:irred_b}
  \begin{enumerate}[label=(\roman*), wide, labelwidth=!, labelindent=0pt]
  \item \label{assumG:irred_b_item_i} There exists $\constLiphx \in \rset_+$ such that for all $x \in \ball{0}{\rassG}$, $\hfunb_x$ is
    $\constLiphx$-Lipschitz, \ie~for all $z_1,z_2 \in \rset^d$,
    $\norm{\hfunb_x(z_1)-\hfunb_x(z_2)} \leq \constLiphx
    \norm{z_1-z_2}$.
\item \label{assumG:irred_b_item_ii} There exist $\tilde{y}_0 \in \rset^d$ and $\tMassG \in \rset_+^*$,
  such that for all $x \in \ball{0}{\rassG}$, $\ball{y_0}{\MassG} \subset \hfunb_x(\ball{\tilde{y}_0}{\tMassG})$.
  \end{enumerate}
\end{assumptionG}


\begin{theorem}
\label{theo:irred}
Assume \Cref{assumG:phi} and that there exist $y_0 \in \rset^d$, $R > 0$ and $M > 0$ such that \Cref{assumG:irred_b}($\rassG,y_0,\MassG$)  is satisfied. Then $\ball{0}{\rassG}$ is $1$-small for $\Pkerb$: for all $x \in \ball{0}{\rassG}$ and $\eventA \in \borelSet(\rset^d)$,
  \begin{equation}
    \Pkerb(x,\eventA) \geq \constLiphx^{-d} \min_{(x,z) \in \ball{0}{R} \times \ball{\tilde{y}_0}{\tMassG}} \defEns{\alphagen(x,z) \phib(z)} \Leb\defEns{\eventA \cap \ball{y_0}{\MassG}} \eqsp,
  \end{equation}
where $(\tilde{y}_0,\tilde{M}) \in \rset^d \times \rset_+^*$ is defined in \Cref{assumG:irred_b}($\rassG,y_0,\MassG$).
\end{theorem}

\begin{proof}%[Proof of \Cref{theo:irred}]
For all $x  \in \ball{0}{\rassG}$ and $\eventA \in \borelSet(\rset^d)$ we get
\begin{align}
\nonumber
\Pkerb(x,\eventA)  &= \int_{\rset^d}\indi{\eventA}{\hfunb(x,z)} \alphagen(x,z) \phib(z) \rmd z  = \int_{\rset^d}\indi{\hfunb_x^{-1}(\eventA)}{z} \alphagen(x,z) \phib(z) \rmd z \\
&\geq  \min_{(x,z) \in \ball{0}{R} \times \ball{\tilde{y}_0}{\tilde{M}}} \defEns{\alphagen(x,z)\phib(z)} \Leb\defEns{\hfunb_x^{-1}(\eventA) \cap\ball{\tilde{y}_0}{\tilde{M}}} \eqsp.
\label{eq:coro_leb_irred_1}
\end{align}
The proof follows from \Cref{le:simple} and \Cref{assumG:irred_b}$(R,y_0,M)$-\ref{assumG:irred_b_item_i} which imply
$ \Leb\defEns{\hfunb_x^{-1}(\eventA )\cap \ball{\tilde{y}_0}{\tilde{M}}} \geq  \constLiphx^{-d}  \Leb\defEns{\eventA  \cap \ball{y_0}{M}}$.
\end{proof}
The following Corollary is a straightforward consequence of \Cref{theo:irred}.
\begin{corollary}
\label{coro:irred}
Assume \Cref{assumG:phi} and  that there exists $(y_0,M) \in \rset^d \times \rset_+^*$ such that  for all $\rassG \in \rset_+^*$ \Cref{assumG:irred_b}($\rassG,y_0,\MassG$). Then $\Pkerb$ is irreducible with irreducibility measure $\Leb\defEns{\cdot \cap \ball{y_0}{\MassG}} $. In addition, all the compact sets are $1$-small.
\end{corollary}
In the next proposition, we give examples of functions $f$ which satisfy \Cref{assumG:irred_b}.
\begin{proposition}\label{le:degree_application}
Let  $\ga$ a function from $ \rset^d\times \rset^d$ to $\rset^d$ and $\ra \in \rset^{*}_+$. Assume that
 % \begin{enumerate}[label=\roman*)]
%   \item \label{application:irred_b_item_i}
% for all $\ra >0$ there exists $\Lga \geq
%     0$ such that for all $x \in \ball{0}{\ra}$, $z \mapsto \ga(x,z)$ is
%     $\Lga$-Lipschitz, \ie~for all $x \in \ball{0}{\ra}$,  $z_1,z_2 \in \rset^d$,
%     $\abs{\ga(x,z_1)-\ga(x,z_2)} \leq \Lga
%     \abs{z_1-z_2}$.
%\item \label{application:irred_b_item_ii}
\begin{enumerate}[label=(\roman*)]
\item
\label{propo:irred_b_item_i}
 there exists $\lipgr \in \rset_+$  such that for all $z_1,z_2,x \in \rset^d$, $\norm{x} \leq \ra$,
  \begin{equation}
    \label{eq:5}
    \norm{g(x,z_1) - g(x,z_2)} \leq \lipgr\norm{z_1-z_2} \eqsp.
  \end{equation}
\item
\label{propo:irred_b_item_ii}
there exist $ \Cga_{\ra,0} , \Cga_{\ra,1}
\in \rset_+$ such that for all $x,z \in \rset^d$, $\norm{x} \leq \ra$
\begin{equation}
  \label{eq:4}
\norm{g(x,z)} \leq  \Cga_{\ra,0} +   \Cga_{\ra,1} \norm{z}
\end{equation}
\end{enumerate}

% \begin{equation}\label{eq:gr}
% \abs{g(x,z)} \leq \Cga (1+|x|+|z|) \eqsp.
% \end{equation}
%\end{enumerate}
Let $\bg \in \rset$ and define $\hga : \rset^d \times \rset^d$ for all $x,z \in \rset^d$ by
\begin{equation}
  \hga(x,z) =  \bg z + \ga(x,z) \eqsp.
\end{equation}
If $\norm{\bg} > \Cga_{\ra,1} $, then $\hga$ satisfies \Cref{assumG:irred_b}($\ra,0,\MassG$) for all $\MassG \in \rset_+^*$ with $\tilde{y}_0=0$ and 
\begin{equation}
  \label{eq:deftildeM}
\tilde{M} = \{M  + \Cga_{\ra,0} \}/(\norm{\bg}-\Cga_{\ra,1} ) \eqsp.
\end{equation}
% Let $a,b\in\R$ with $|a|> C$ and for all $x \in \rset^d$ define $\Psi_x : \rset^d \to \rset^d$  by
% $$
% \Psi_x(y)=ay+bx+g(x,y) \eqsp, \text{ for all $y \in \rset^d$} \eqsp.
% $$
% Then for all  $R>0$ there exists $\epsilon >0$ satisfying for all $\eventA \in \borelSet(\rset^d)$, $\eventA \subset \ball{0}{R}$,
% \begin{equation}
%   \inf_{x \in \ball{0}{R}} \lambda_{\Psi_x}(\eventA) > \epsilon \Leb(\eventA) \eqsp,
% \end{equation}
% where $\lambda_{\Psi_x}$ is defined in \eqref{eq:push_forward_mes}.
\end{proposition}
We preface the proof by recalling some basic notions of degree theory.
\label{sec:defin-usef-results}
Let $\Dset$ be a bounded open set of $\rset^d$. Let $f:
\Dsetc \to \rset^d$ be a continuous function on
$\Dsetc$ continuously differentiable on $\Dset$. An element $x \in
\Dset$ is said to be a \emph{regular point} of $f$ if the Jacobian matrix of $f$ at $x$, $\Jac_f(x)$, is invertible.
An element $y \in f(\Dset)$ is said to be a \emph{regular value} of $f$ if any $x \in
f^{-1}(\{ y\})$ is a regular point.  %true if $y \not in f(\Dset)$

Let $f : \Dsetc \to \rset^d$ be a continuous function,  $C^{\infty}$-smooth on $\Dset$. Let $y \in \rset^d
  \setminus f(\partial \Dset)$ be a regular value of $f$. It is shown in \cite[Proposition and Definition 1.1]{outerelo:ruiz:2009} that the set $f^{-1}(\{y\})$ is finite. The degree of $f$ at $y$ is defined by
\begin{equation}
  \deg(f,\Dset,y) = \sum_{x \in f^{-1}(\{y \})} \sign\defEns{\det \parenthese{\Jac_f(x)}} \eqsp.
\end{equation}

\begin{proposition}[\protect{\cite[Proposition and Definition 2.1]{outerelo:ruiz:2009}}]
\label{defProp:degree_cont}
  Let $f : \Dsetc \to \rset^d$ be a continuous function and $y \in
  \rset^d \setminus f(\partial \Dset)$.
  \begin{enumerate}[label=(\alph*)]
  \item
\label{defProp:degree_cont_i}
 Then there exists  $g \in C(\Dsetc, \rset^d) \cap C^{\infty}(\Dset, \rset^d)$ such that $y$ is a regular value of $g$
  and $\sup_{x \in \Dsetc} \abs{f(x)-g(x)} < \dist(y,f(\partial
  \Dset))$.
\item For all functions $g_1,g_2:\Dsetc \to \rset^d$ satisfying \ref{defProp:degree_cont_i},
  \begin{equation}
    \deg(g_1,\Dset,y) = \deg(g_2,\Dset,y) \eqsp.
  \end{equation}
  \end{enumerate}
\end{proposition}
Under the assumptions of \Cref{defProp:degree_cont}, the degree of $f$ at $y$ is then defined for any $g:\Dsetc \to \rset^d$ satisfying \ref{defProp:degree_cont_i} by
\begin{equation}
  \deg(f,\Dset,y) =  \deg(g,\Dset,y) \eqsp.
\end{equation}

\begin{proposition}[\protect{\cite[Proposition
  2.4]{outerelo:ruiz:2009}}]
  \label{theo:deg_modif}
  Let $f,g : \Dsetc \to \rset^d$ be  continuous functions. Define
  $\hpy:\ccint{0,1} \times \rset^d \to \rset^d$ for all $t \in
  \ccint{0,1}$ and $x \in \rset^d$ by $\hpy(t,x) = t f(x) +
  (1-t)g(x)$. Let $y \in \rset^d \setminus \hpy(\ccint{0,1} \times \partial \Dset)$. Then
\begin{equation}
  \deg(f,\Dset,y) =  \deg(g,\Dset,y) \eqsp.
\end{equation}
\end{proposition}
We have now all the necessary results to prove \Cref{le:degree_application}.
\begin{proof}[Proof of \Cref{le:degree_application}]
Since $\hga(x,z) =  \bg z + \ga(x,z)$ and $\ga(x,\cdot)$ is Lipschitz with a Lipschitz constant which is uniformly bounded over the ball $\ball{0}{R}$,  $\hga_x$ is Lipschitz with bounded Lipschitz constant over this ball. Hence \Cref{assumG:irred_b}($\ra,0,\MassG$)-\ref{assumG:irred_b_item_i} holds.

  For all $x \in \rset^d$, denote by $\hga_x : z \mapsto \hga(x,z)$ where $\hga(x,z)=bz + g(x,z)$.
  Let $\MassG \in \rset_+^*$. We show that for all $x \in
  \ball{0}{\ra}$, $\ball{0}{\MassG} \subset
  \hga_x(\ball{0}{\tMassG})$, where $\tMassG$ is given by
  \eqref{eq:deftildeM}, which is precisely
  \Cref{assumG:irred_b}($\ra,0,\MassG$)-\ref{assumG:irred_b_item_ii}.

 % Then for all $z \in
%   (\hga_x)^{-1}(\ball{0}{\MassG}) $, by \ref{propo:irred_b_item_ii}
% \begin{equation}
%   \MassG \geq \abs{\hga_x(z)} \geq  \abs{ \bg z}  - \Cga_{\ra,0} -\Cga_{\ra,1} \abs{z} \eqsp.
% \end{equation}
% Therefore since $\abs{\bg} \geq \Cga_{\ra,1} $, $(\hga_x)^{-1}(\ball{0}{\MassG}) \subset \ball{0}{\tMassG}$ where $\tMassG$ is given by
%\eqref{eq:deftildeM}.
% $\tMassG = \{\MassG + \ra \abs{\ag} +
% \Cga(1+\abs{\ra})\}/(\abs{\bg}-\Cga)$.
%Next we show \ref{item:proof:homot_ii}.
%  Let $x \in \ball{0}{\ra}$ and
% $\MassG \geq 0$.
  Let $x \in \ball{0}{\ra}$ and consider the continuous homotopy $\hog : \ccint{0,1}
\times \rset^d$ between the functions $z \mapsto \bg z$ and $\hga_x$ defined for all
$t \in \ccint{0,1}$ and $z \in \rset^d$ by
\begin{equation}
  \hog(t,z) = t \bg z + (1-t)\hga_x(z) = \bg z + (1-t)  \ga(x,z)  \eqsp.
\end{equation}
Then by \ref{propo:irred_b_item_ii}, since $\abs{\bg} \geq \Cga_{\ra,1} $, for all $t\in \ccint{0,1}$ and $z \not \in
\ball{0}{\tMassG}$, where $\tMassG$ is given by \eqref{eq:deftildeM},
\begin{equation}
   \abs{\hog(t,z)} \geq  \abs{\bg z} -(1-t)\defEns{\Cga_{\ra,0} +\Cga_{\ra,1} \abs{z} } \geq \MassG \eqsp.
\end{equation}
In particular, we have $\hog(\ccint{0,1} \times \partial
\ball{0}{\tMassG}) \subset \rset^d \setminus \ball{0}{\MassG}$. Let
$z \in \ball{0}{\MassG}$, then by
%\cite[Proposition 2.4, Proposition-Definition 1.1, Chapter IV]{outerelo:ruiz:2009},
\Cref{theo:deg_modif} we have
\begin{equation}
  \deg(\hga_x,\ball{0}{\tMassG},z) = \deg(\bg \Id, \ball{0}{\tMassG},z) = 1 \eqsp.
\end{equation}
Besides, by \cite[Corollary 2.5, Chapter IV]{outerelo:ruiz:2009},
$\deg(\hga_x,\ball{0}{\tMassG},z) \not = 0$ implies that there exists
$y \in \ball{0}{\tMassG}$ such that $\hga_x(y) = z$. Finally \Cref{assumG:irred_b}($\ra,0,\MassG$)-\ref{assumG:irred_b_item_ii} follows since this
result holds for all $z \in \ball{0}{\MassG}$.
\end{proof}
% which implies that $\hga_x(\rset^d \setminus
% \ball{0}{\tilde{M}}) \subset \rset^d \setminus \ball{0}{M}$
% $(\hga_x)^{-1}(\ball{0}{M}) \subset \ball{0}{\tilde{M}}$.
% % Then \ref{application:irred_b_item_i} implies
% %   that \Cref{assumG:irred_b}($\ra$)-\ref{assumG:irred_b_item_i} holds. We now show that under

% We first show that for all $x \in \rset^d$, $\Psi_x$ is surjective from $\rset^d$ to $\rset^d$. Now a standard perturbation
% theorem from degree theory (see e.g. \alain{I think we can cite
%   Milnor}), applied to the continuous family of maps
% $\{\Phi_x(t,\cdot) \, : \, t \in \ccint{0,1} \}$, defined for all $t
% \in \ccint{0,1}$
% $$
% y \mapsto \Phi_{x}(t,y):=ay+ t(bx+g(x,y)) \eqsp,
% $$
% implies that $\Psi_{x}$ is surjective\alain{expliquer }.
%  Assume that $R>0$ is given. Our  assumptions imply clearly that there exists  $M>R/\abs{a}$ such that for all $x \in \ball{0}{R}$ and $y \not \in \ball{0}{M}$,
% $$
% \abs{ay}-\abs{bx+g(x,y)}\geq R \eqsp.
% $$
% Especially, this holds at the boundary $\{y \in \rset^d \, : \,
% \abs{y} = M \}$. Let $x \in \ball{0}{R}$. Therefore $\ball{0}{R}\subset
% \Psi_x(\ball{0}{M})$ for all $x \in \ball{0}{R}$. Let $L$ be the
% Lipschitz constant of $g$ on $B(0,R)\times B(0,M)$.
%  \Cref{le:simple} yields that for any $x\in B(0,R)$ and $\eventA \in \borelSet(\rset^d)$, $\eventA \subset \ball{0}{R}$
%  \begin{equation}
%    \lambda_{\Psi_x}(\eventA) \geq L^{-d}\Leb(\eventA) \eqsp,
%  \end{equation}
% and the proof follows.
%  the
% push-forward measure of the Lebesgue measure on $B(0,R_0)$ under $y\to
% H(x_0,y)$ has a lower density bound $c> 0$ on the ball $B(0,R)$, that
% is independent of $x_0$. As the Gaussian distribution has lower
% bounded density on $B(0,R_0)$ the claim follows.


%%% Local Variables:
%%% mode: latex
%%% TeX-master: "main"
%%% End:


%  Next we record a
%   well-known fact from geometric measure theory together with a proof
%   for the reader's convenience.  Let $\ouvert\subset\rset^d$ be an open
%   set and $f: \ouvert\to \rset^d$ be a measurable function such that there exist $z_0,y_0 \in \rset^d$ and $M, \tilde{M} \geq 0$ satisfying
%   \begin{equation}
%     \label{eq:condition_sigma_finite}
% f^{-1}(\ball{y_0}{M}) \subset \ball{z_0}{\tilde{M}} \eqsp.
%   \end{equation}
%  Define the measure $\lambda_{f}$ on $(\rset^d,\borelSet(\rset^d))$
%   by setting for any $\eventA \in \borelSet(\rset^d)$
% \begin{equation}
%   \label{eq:push_forward_mes}
% \lambda_f (\eventA):=\Leb(f^{-1}(\eventA \cap \ball{y_0}{M}))  \eqsp.
% \end{equation}
% Note by \eqref{eq:condition_sigma_finite} that $\lambda_f$ is  a finite measure. Therefore by the Lebesgue decomposition theorem (see
% \cite[Section 6.10]{rudin:1987}) there exist two non-negative measures
% $\lambda_f^{(\text{a})}, \lambda_f^{(\text{s})}$ on
% $(\rset^d,\borelSet(\rset^d))$, which are absolutely continuous and
% singular with respect to the Lebesgue measure on $\rset^d$
% respectively, such that $\lambda_f = \lambda_f^{(\text{a})} +
% \lambda_f^{(\text{s})}$.
% % Note that if for all compact set $\compact \subset \rset^d$, $f^{-1}(\compact)$ is
% % compact (\ie~$f$ is a proper function), then $\lambda_f$ is $\sigma$-finite.
% \begin{proposition}\label{le:simple}
%   Let $\ouvert\subset\rset^d$ be open and $f: \ouvert\to \rset^d$ be a Lipschitz function
%   satisfying \eqref{eq:condition_sigma_finite}. Let $\phi_f$ be the
%   density of $\lambda_f^{(\text{a})}$ with respect to the Lebesgue
%   measure on $\rset^d$. Then $\Leb$-almost everywhere, it holds
% $$
% \phi_f(y)\geq \1_{f(\ouvert) \cap \ball{y_0}{M}}(y) \norm{f}_{\Lip}^{-d}\eqsp.
% $$
% \end{proposition}
% \begin{proof}
%   % We only need to prove that for $\Leb$-almost all $y \in f(\ouvert)
%   % \cap \ball{y_0}{M}$, it holds $\phi_f(y)\geq
%   % \norm{f}_{\Lip}^{-d}$.
%  Denote by $L = \norm{f}_{\Lip}$. Let $y\in
%   f(\ouvert) \cap \ball{y_0}{M}$ and $\delta_0 >0$  such that $\ball{y}{\delta_0}
%   \subset \ball{y_0}{M}$. Let $z\in f^{-1}(\{y\})$. Since $f$ is
%   Lipschitz continuous, there exists $\delta_1 >0$ such that for all
%   $\delta \in \ccint{0,\delta_1}$, $\ball{z}{\delta/L}\in \ouvert$ and
%   $f(\ball{z}{\delta/L})\subset \ball{y}{\delta}$. Hence, for all
%   $\delta \in \ocint{0,\min(\delta_0,\delta_1)}$, we have
%   $$
% \lambda_f(\ball{y}{\delta})\geq
%   L^{-d}\Leb(\ball{z}{\delta}) =   L^{-d}\Leb(\ball{y}{\delta}) \eqsp.
% $$
%  The claim follows from the differentiation theorem
%   for measures, see \cite[Theorem 7.14]{rudin:1987}.
% \end{proof}


% \begin{corollary}
%   \label{coro:irred}
%   Let $x \in \rset^d$. Assume that $\hfunb_x$ is Lipschitz and there
%   exist $y_0,z_0 \in \rset^d$, $M,\tilde{M} \geq 0$ such that $\hfunb_x^{-1}(\ball{y_0}{M}) \subset \ball{z_0}{\tilde{M}}$.
% Then for all $\eventA \in \borelSet(\rset^d)$,
% \begin{equation}
%   \Pkerb(x,\eventA) \geq \norm{\hfunb_x}_{\Lip}^{-d} \inf_{z \in \ball{z_0}{\tilde{M}}} \defEns{\phib(z)} \Leb\defEns{\eventA \cap \hfunb_x(\rset^d) \cap \ball{y_0}{M}} \eqsp.
% \end{equation}
% \end{corollary}

% \begin{proof}
%   By definition, we have using that  $\hfunb_x^{-1}(\ball{y_0}{M}) \subset \ball{z_0}{\tilde{M}}$,
%   \begin{align}
% \nonumber
%       \Pkerb(x,\eventA)  &= \int_{\rset^d}\1_{\eventA}(\hfunb(x,z)) \phib(z) \rmd z  = \int_{\rset^d}\1_{\hfunb_x^{-1}(\eventA)}(z) \phib(z) \rmd z \\
% \label{eq:coro_leb_irred_1}
% &\geq  \inf_{z \in \ball{z_0}{\tilde{M}}} \defEns{\phib(z)} \Leb\defEns{\hfunb_x^{-1}(\eventA \cap \ball{y_0}{M})} \eqsp.
%   \end{align}
% Since by assumption $\hfunb_x$ is Lipschitz, we get using \Cref{le:simple} that
% \begin{equation}
% \label{eq:coro_leb_irred_2}
%  \Leb\defEns{\hfunb_x^{-1}(\eventA \cap \ball{y_0}{M})} \geq  \norm{\hfunb_x}_{\Lip}^{-d}  \Leb\defEns{\eventA \cap \hfunb_x(\rset^d) \cap \ball{y_0}{M}} \eqsp.
% \end{equation}
% Combining \eqref{eq:coro_leb_irred_1} and \eqref{eq:coro_leb_irred_2} concludes the proof.
% \end{proof}

%%% Local Variables:
%%% mode: latex
%%% TeX-master: "main"
%%% End:


\section{Proofs}
\label{sec:postponed-proofs}
 In the sequel, $C \geq 0$
  is a constant which can change from line to line but does not depend
  on $h$. Let $h >0$ and $T \in \nset^*$. Note that a simple induction (see \cite[Proposition 4.2]{livingstone:betancourt:byrne:girolami:2016})  implies that for all $(\q_0,\p_0) \in \rset^d
\times \rset^d$ and $k \in \{1,\ldots T\}$, the $k^{\text{th}}$
iteration of the leap-frog integration, $(q_k,p_k) = \Phiverlet[h][k](\q,\p)$, where $\Phiverlet[h][k]$ is defined by \eqref{eq:def_Phiverlet}, takes the form
\begin{align}
\label{eq:qk}
\q_k&=\q_0+kh\p_0-\frac{kh^2}{2} \nabla \F(\q_0)-h^2 \gperthmc[k](\q_0,\p_0)\\
\label{eq:pk}
\p_{k}&= \p_0-\frac{h}{2} \defEns{\nabla \F(\q_0)+\nabla \F \circ \Phiverletq[h][k] (\q_0,\p_0)}-h \sum_{i=1}^{k-1}\nabla \F \circ \Phiverletq[h][i](\q_0,\p_0)  \eqsp,
\end{align}
where  $\gperthmc[k] :\rset^d \times \rset^d \to \rset^d$ is given  for all $(\q,\p) \in \rset^d \times
\rset^d$ by
\begin{equation}
  \label{eq:def_gperthmc}
\gperthmc[k](\q,\p) =    \sum_{i=1}^{k-1}(k-i)\nabla \F \circ \Phiverletq[h][i](\q,\p) \eqsp.
\end{equation}

We prefaces the proofs of our main results by useful bounds  on
  the position and the momentum in the intermediate steps of the leap-frog integration.
% Under the regularity condition
% \Cref{assum:regOne}($\beta$), it is possible to derive  useful bounds  on
%  the position and the momentum in the intermediate steps of the leap-frog integration.


  \begin{lemma}
    \label{lem:bound_first_iterate_leapfrog_a}
  Assume \Cref{assum:regOne}$(\expozero)$-\ref{assum:regOne_a}. Then, for any $k \in \nsets$, $h \geq 0$, $(q_0,p_0) \in \rset^{2d}$ and $(x_0,v_0) \in \rset^{2d}$,
  \begin{align}
   & \norm{q_k-x_k} + \constzero^{-1/2} \norm{p_k-v_k} \\
    & \qquad \qquad \leq \defEns{1+h \constzero^{1/2} \vartheta_1(h \constzero^{1/2})}^{k} \defEns{\norm{q_0-x_0} + \constzero^{-1/2} \norm{p_0-v_0}} \eqsp,
  \end{align}
  where $(q_k,p_k)= \Phiverlet[h][k](q_0,p_0)$, $(x_k,v_k)= \Phiverlet[h][k](x_0,v_0)$ and  $\Phiverlet[h][k]$ and $\vartheta_1$ are defined by \eqref{eq:def_Phiverlet} and \eqref{eq:def_vartheta_1}, respectively.
\end{lemma}
\begin{proof}
  Note that it is sufficient to show the result for $k=1$ and to apply a straightforward induction.
  Let $h >0$, $(q_0,p_0) \in \rset^{2d}$ and $(x_0,v_0) \in \rset^{2d}$. Using \eqref{eq:qk}, the triangle inequality and   \Cref{assum:regOne}$(\expozero)$-\ref{assum:regOne_a}, we first obtain
  \begin{align}
    \nonumber
    \norm{q_1-x_1} & = \norm{q_0 -h^2 \nabla U(q_0) /2 + h p_0 - \defEns{x_0 - h^2/2 \nabla U(x_0) + h v_0}} \\
    \label{eq:bound_q_1_x_1}
         & \leq (1+ h^2 \constzero /2) \norm{q_0 - x_0} + h \norm{p_0 - v_0} \eqsp.
  \end{align}
  Second, similarly using \eqref{eq:pk}, we have that
\begin{align}
\label{eq:bound_p_1_v_1}
&    \norm{p_1-v_1}  \\
&= \norm{p_0-v_0 -(h/2) \defEns{\nabla U(q_1) + \nabla U(q_0)} + (h/2) \defEns{\nabla U(x_1) + \nabla U(x_0)}} \\
\nonumber
&\leq \norm{p_0 - v_0} + (h \constzero/2) \defEns{ \norm{x_1- q_1} + \norm{x_0 - q_0}} \\
\nonumber
& \leq \left(1+h^2 \constzero/2\right) \norm{p_0-v_0} +h\constzero(1+h^2\constzero /4) \norm{q_0-x_0} \eqsp,
\end{align}
  where we have used \eqref{eq:bound_q_1_x_1} for the last inequality. Summing up \eqref{eq:bound_q_1_x_1} and \eqref{eq:bound_p_1_v_1}, we get the desired result for $k=1$.
  \end{proof}
  %\label{lem:bound_first_iterate_leapfrog_1} devient ...
  \begin{lemma}
  \label{lem:bound_first_iterate_leapfrog_b}
Let $\beta \in \ccint{0,1}$ and assume \Cref{assum:regOne}$(\expozero)$-\ref{assum:regOne_b}.
  \begin{enumerate}[label=(\roman*)]
\item
\label{lem:bound_first_iterate_leapfrog_1}
% \alain{ne suppose que \eqref{eq:bound_nabla_F_assum_reg_zero}}
For any $h_0 >0$, $T \in \nsets$, there exists $C < \infty$ (which depends only on $T,h_0$
 and $\constzeroT$) such that for all $h \in \ocint{0,h_0}$,
  $(\q_0,\p_0) \in \rset^d \times \rset^d$ and $k \in \{1,\ldots, T\}$
  \begin{align}
\label{lem:bound_first_iterate_leapfrog_1_q}
    \norm{\q_k-\q_0} &\leq C h\defEns{  \norm{\p_0} +h(1+ \norm{\q_0}^{\expozero})}\\
\label{lem:bound_first_iterate_leapfrog_1_p}
    \norm{\p_k-\p_0}& \leq C h\defEns{  1+ \norm{\p_0}^{\expozero}+ \norm{\q_0}^{\expozero}} \eqsp,
  \end{align}
where $(\q_k,\p_k) = \Phiverlet^{\circ k}_{h}(\q_0,\p_0)$ and  $\Phiverlet^{\circ k}_{h}$ is defined by \eqref{eq:def_Phiverlet}.
\item \label{lem:bound_first_iterate_leapfrog_b_2}
  If in addition \Cref{assum:regOne}$(\expozero)$-\ref{assum:regOne_a} holds, for any $k \in \nsets$, $h >0$, $(q_0,p_0) \in \rset^{2d}$,
  \begin{align}
    &    \norm{q_k-q_0} + \constzero^{-1/2}\norm{p_k - p_0} \leq (\constzero^{1/2}\vartheta_1(h \constzero^{1/2}))^{-1} \\
    & \quad \times \defEns{(1+h \constzero^{1/2}\vartheta_1(h \constzero^{1/2}))^{k+1} - 1} \defEns{\vartheta_2(h) (\norm[\beta]{q_0} +1) + \vartheta_3(h) \norm{p_0}} \eqsp,
  \end{align}
  where $\vartheta_1$ is defined by \eqref{eq:def_vartheta_1} and
  \begin{align}
  \label{eq:definition-vartheta-2}
    \vartheta_2(h) &= \constzeroT/\constzero^{1/2} + \constzeroT h/2 + \constzero^{1/2}\constzeroT h^2/4 \eqsp, \\
  \label{eq:definition-vartheta-3}
    \vartheta_3(h) &= 1+\constzero^{1/2} h /2 \eqsp.
  \end{align}
  \end{enumerate}
\end{lemma}
\begin{proof}
  \begin{enumerate}[label={(\roman*)},wide=0pt, labelindent=\parindent]
% \item We show by induction that for all $k \in \{1,\cdots,T\}$, there
%   exists $C_k$ (which depends only on $T,h_0$ and  $\constzero$) such that for all $ h \in \ocint{0, h_0}$, $(q_0,p_0)
%   \in \rset^d \times \rset^d$ and $(\tilde{q}_0,\tilde{p}_0) \in
%   \rset^d \times \rset^d$,
%   \begin{equation}
%     \norm{q_k-\tilde{q}_k} \leq C_k (\norm{q_0-\tilde{q}_0} + \norm{p_0-\tilde{p}_0}) \eqsp,
%   \end{equation}
%   where $(q_k,p_k) = \Phiverlet^{\circ k}_{h}(q_0,p_0)$,
%   $(\tilde{q}_k,\tilde{p}_k) = \Phiverlet^{\circ
%     k}_{h}(\tilde{q}_0,\tilde{p}_0)$.  Let $ h \in \ocint{0, h_0}$,
%   $(q_0,p_0) \in \rset^d \times \rset^d$ and
%   $(\tilde{q}_0,\tilde{p}_0) \in \rset^d \times \rset^d$.  The case $k=1$ is immediate
%   by \Cref{assum:regOne}($\beta$)-\ref{assum:regOne_a}. Let $k \in \{1,\cdots, T-1\}$ and assume
%   that the inequality holds for all $i \in \{1,\cdots, k\}$. Then by
%   \eqref{eq:qk} and \Cref{assum:regOne}($\beta$)-\ref{assum:regOne_a} we get
% \begin{align}
%   &     \norm{q_{k+1}-\tilde{q}_{k+1}} \leq \norm{q_0- \tilde{q}_0}+(k+1)h\norm{p_0-\tilde{p}_0}+(1/2)(k+1)h^2 \norm{\nabla \F(q_0)-\nabla \F(\tilde{q}_0)}\\
%   & \phantom{\norm{q_{k+1}-\tilde{q}_{k+1}}}\phantom{\leq \norm{q_0- \tilde{q}_0}+(k+1)haaa} +h^2\sum_{i=1}^{k}(k+1-i)\norm{\nabla \F(q_i) - \nabla \F(\tilde{q}_i)} \\
% & \phantom{\norm{q_{k+1}-\tilde{q}_{k+1}}}\leq \norm{q_0- \tilde{q}_0}+(k+1)h\norm{p_0-\tilde{p}_0}+2^{-1}(k+1) h^2  \constzero \norm{q_0-\tilde{q}_0}\\
% & \phantom{\norm{q_{k+1}-\tilde{q}_{k+1}}}\phantom{\leq \norm{q_0- \tilde{q}_0}+(k+1)haaa}+ h^2\constzero \sum_{i=1}^{k}(k+1-i)\norm{q_i -\tilde{q}_i} \eqsp.
%    \end{align}
% An application of the induction hypothesis concludes the proof.
  \item Let $T \in \nsets$ and $h_0 >0$.  We prove by induction that for all $k \in \{1,\ldots,T\}$ there exists $C_k \geq 0$ (which depends only on $T,h_0$
and $\constzeroT$) such that for all $h \in \ocint{0,h_0}$ and
  $(q_0,p_0) \in \rset^d \times \rset^d$
\begin{equation}
\label{lem:bound_first_iterate_leapfrog_1_q}
\begin{aligned}
\norm{q_k-q_0} \leq C_k h\defEns{  \norm{p_0} +h(1+ \norm{q_0}^{\expozero})} \\
\norm{p_k-p_0} \leq C_k h \defEns{ 1 + \norm{p_0}^{\expozero} + \norm{q_0}^{\expozero}} \eqsp.
\end{aligned}
\end{equation}
  where $(q_k,p_k) = \Phiverlet^{\circ k}_{h}(q_0,p_0)$.
 Let $ h \in \ocint{0, h_0}$ and $(q_0,p_0) \in \rset^d \times
    \rset^d$.
 The case $k=1$ is immediate by
\Cref{assum:regOne}($\beta$)-\ref{assum:regOne_b} and \eqref{eq:qk}. Let $k \in \{1,\cdots,
T-1\}$ and assume that the inequalities hold for all $i \in
\{1,\dots, k\}$. Then by \eqref{eq:qk} and
\Cref{assum:regOne}($\beta$)-\ref{assum:regOne_b}, we get
\begin{align}
\label{eq:lem:bound_first_iterate_leapfrog_1}
\norm{q_{k+1}-q_0}
&\leq (k+1)h \norm{p_0}+\frac{k+1}{2}h^2 \constzeroT  \defEns{1+\norm{ q_0}^{\expozero} }\\
&\qquad \qquad +h^2 \constzeroT \sum_{i=1}^{k}(k+1-i)\defEns{1+ \norm{q_i}^{\expozero} }\eqsp.
\end{align}
By the induction hypothesis and using that $t \mapsto t^{\expozero}$ is sub-additive on $\rset^+$ and $t^\beta \leq 1 + t$ for $t \in \rset^+$, we get for all $i \in \{1,\cdots, k\}$,
\begin{equation}
\norm{q_i}^{\expozero} \leq \norm{q_0}^{\expozero} + \norm{q_i-q_0}^\beta \leq 1+ \norm{q_0}^{\expozero} +C_i h\defEns{\norm{p_0}+h(1+ \norm{q_0}^{\expozero})} \eqsp,
\end{equation}
Plugging this inequality in \eqref{eq:lem:bound_first_iterate_leapfrog_1}
conclude the proof of \eqref{lem:bound_first_iterate_leapfrog_1_q}.
Consider now \eqref{lem:bound_first_iterate_leapfrog_1_p}. Since by
definition $p_{k+1} = p_k-(h/2)\defEnsLigne{\nabla \F(q_{k}) + \nabla
  \F(q_{k+1})}$, using the triangle inequality,
\Cref{assum:regOne}($\beta$)-\ref{assum:regOne_b},
\eqref{lem:bound_first_iterate_leapfrog_1_q} to bound $\norm{q_{k}}$ and
$\norm{q_{k+1}}$, and the induction hypothesis, we get that there exist some
constants $C_{k+1,1},C_{k+1,2}$ which only depend on $T,h_0$ and
$\constzeroT$ such that
\begin{align}
&  \norm{p_{k+1} - p_0} \leq \norm{p_k-p_0} + (h/2)\defEnsLigne{\norm{\nabla \F(q_{k})} + \norm{\nabla \F(q_{k+1})}} \\
&\quad \leq C_{k+1,1}h\defEns{1+\norm{p_0}^{\expozero}+\norm{q_0}^{\expozero}} + (\constzeroT h/2)\defEns{2+\norm{q_{k}}^{\expozero}+ \norm{q_{k+1}}^{\expozero}}\\
&\quad \leq C_{k+1,1}h\defEns{1+\norm{p_0}^{\expozero}+\norm{q_0}^{\expozero}} + (C_{k+1,2}h/2)\defEns{1+ \norm{q_0}^{\expozero} + \norm{p_0}^{\expozero}} \eqsp.
\end{align}
Therefore, \eqref{lem:bound_first_iterate_leapfrog_1_q} is satisfied which concludes the induction and the proof.
\item Let $k \in \nset$, $h >0$ and $(q_0, p_0) \in \rset^{2d}$. Using \eqref{eq:qk}, the triangle inequality and \Cref{assum:regOne}($\beta$), we have
  \begin{align}
\label{eq:1_lem_bound_first_iterate_leapfrog_b_2}
&    \norm{q_{k+1} - q_0}  = \norm{q_k - q_0 - (h^2/2) \nabla U(q_k) + h p_k} \\
\nonumber
&                          \leq (1+h^2 \constzero/2) \norm{q_k - q_0} + (h^2\constzeroT/2) (\norm[\beta]{q_0} +1) + h \norm{p_k-p_0} + h \norm{p_0}  \eqsp.
  \end{align}
Second, similarly using \eqref{eq:pk}, we get that
\begin{align}
\nonumber
&    \norm{p_{k+1} - p_0}   \leq  \norm{p_k-p_0 + (h/2) \defEns{\nabla U(q_{k+1}) + \nabla U(q_k)} } \\
    \nonumber
                         & \leq \norm{p_k - p_0} + (h \constzero/2) \defEns{ \norm{q_{k+1}- q_0} + \norm{q_k - q_0}} + h \constzeroT \defEns{\norm[\beta]{q_0} +1} \\
    \nonumber
                         & \leq \norm{p_k - p_0}  +  h \constzeroT \defEns{\norm[\beta]{q_0} +1} +(h \constzero/2) \defEns{  + h \norm{p_k-p_0} + h \norm{p_0}} \\
    \label{eq:2_lem_bound_first_iterate_leapfrog_b_2}
&     +(h \constzero/2) \defEns{ (2+\constzero h^2/2) \norm{q_k-q_0} + (h^2\constzeroT/2) (\norm[\beta]{q_0}+1)}  \eqsp.
  \end{align}
  where we have used \eqref{eq:1_lem_bound_first_iterate_leapfrog_b_2} for the last inequality.
  Summing up \eqref{eq:1_lem_bound_first_iterate_leapfrog_b_2} and \eqref{eq:2_lem_bound_first_iterate_leapfrog_b_2} and using the definition \eqref{eq:def_vartheta_1} of $\vartheta_1(h)$, we get that, setting $A_k= \norm{q_k-q_0}  + \constzero^{-1/2} \norm{p_k-p_0}$,
  \[
  A_{k+1} \leq (1+h\constzero^{1/2}\vartheta_1(h\constzero^{1/2})) A_k    + h \defEns{ \vartheta_2(h) (\norm[\beta]{q_0} +1) +  \vartheta_3(h) \norm{p_0}} \eqsp.
\]
  By a straightforward induction, we obtain that
  \begin{equation}
    A_{k+1} \leq \sum_{i=1}^{k+1} \parentheseDeux{(1+h \constzero^{1/2} \vartheta_1(h \constzero^{1/2} ))^{k+1-i} h \defEns{ \vartheta_2(h) (\norm[\beta]{q_0} +1) +  \vartheta_3(h) \norm{p_0} }} \eqsp,
  \end{equation}
  which completes the proof of \ref{lem:bound_first_iterate_leapfrog_b_2}.
\end{enumerate}
\end{proof}
% \begin{proof}
%   The proof is postponed to \Cref{sec:proof-crefl-lem:bound_first_iterate_leapfrog}.
% \end{proof}

\begin{lemma}
\label{lem:inverse_1}
Let $\beta \in \ccint{0,1}$ and assume \Cref{assum:regOne}($\beta)$. Then for any $T \in \nsets$, $h >0$,
\begin{align}
\label{eq:lem:inverse_1}
  &  \sup_{(\q,\p,v) \in \rset^{3d}} \{ \norm{ \gperthmc[T](q,\p) - \gperthmc[T](q,v)} / \norm{\p-v} \} \\
  & \qquad \qquad \qquad \qquad  \leq   (T/h)  \left\{ (1+\ h \constzero^{1/2} \vartheta_1(h \constzero^{1/2}))^T - 1 \right\} \eqsp,
\end{align}
where  $\gperthmc[T]$ and $\vartheta_1$ are defined by  \eqref{eq:def_gperthmc} and \eqref{eq:def_vartheta_1} respectively.
  In addition, for any $q \in \rset^d$,
  \begin{equation}
    \label{eq:lem:inverse_2}
    \norm{\gperthmc[T](q,0)} \leq  (T/h)  \{ (1+h \constzero^{1/2} \vartheta_1(h \constzero^{1/2} ))^{T}-1\}  \vartheta_2(h) (\norm[\beta]{q} +1)  + T^2 \norm{\nabla U(q)} \eqsp,
  \end{equation}
  where $\vartheta_2$ is defined in \eqref{eq:definition-vartheta-2}.
\end{lemma}
\begin{proof}
By
\Cref{lem:bound_first_iterate_leapfrog_a}, for any $i \in \nsets$, we get
\[
\sup_{(q,\p,v) \in \rset^{3d}} \defEns{ \normLigne{ \Phiverletq[h][i](q,\p) - \Phiverletq[h][i](q,v)} / \norm{\p-v} }
\leq \constzero^{-1/2} A^i
\]
where $A=(1+h \constzero^{1/2} \vartheta_1(h \constzero^{1/2} ))$.
Therefore by definition of $\gperthmc[T]$ \eqref{eq:def_gperthmc} and using  \Cref{assum:regOne}($\beta$), for any $h >0$, $T \in \nsets$, we get that
\begin{align}
&  \sup_{(\q,\p,v) \in \rset^{3d}} \{ \norm{ \gperthmc[T](q,\p) - \gperthmc[T](q,v)} / \norm{\p-v} \} \\
&\qquad  \leq \constzero   \sum_{k=1}^{T-1} (T-i) \sup_{(q,\p,v) \in \rset^{3d}} \defEns{ \normLigne{ \Phiverletq[h][i](q,\p) - \Phiverletq[h][i](q,v)} / \norm{\p-v} }\\
%& \qquad  \leq  \constzero^{1/2} \sum_{k=1}^{T-1} (T-i) \{1+h \constzero^{1/2} \vartheta_1(h \constzero^{1/2})\}^i  \\
& \qquad \leq    T \parentheseDeux{ A^{T} - 1 } /(h \vartheta_1(\constzero^{1/2} h)) \,
\end{align}
showing \eqref{eq:lem:inverse_1} since $\vartheta_1(h\constzero^{1/2}) \geq 1$.

We now consider \eqref{eq:lem:inverse_2}. By \eqref{eq:def_gperthmc}, \Cref{assum:regOne}($\beta$)-\ref{assum:regOne_a} and \Cref{lem:bound_first_iterate_leapfrog_b}-\ref{lem:bound_first_iterate_leapfrog_b_2}, we have that for any $q \in \rset^d$,
\begin{align}
&  \norm{\gperthmc[T](q,0)}  \leq \sum_{i=1}^{T-1} (T-i)  \norm{ \nabla U \circ \Phiverletq[h][i](\q,0) - \nabla U(q)} + T^2 \norm{\nabla U(q)} \\
& \leq  T \constzero \sum_{i=1}^{T-1}  \norm{ \Phiverletq[h][i](\q,0) -q} + T^2 \norm{\nabla U(q)} \\
& \quad \leq T \constzero^{1/2} \vartheta_1^{-1} (h \constzero^{1/2}) \sum_{i=1}^{T-1}\{A^{i+1}-1\} \defEns{\vartheta_2(h) (\norm[\beta]{q} +1)}  + T^2 \norm{\nabla U(q)} \eqsp,
\end{align}
which completes the proof of \eqref{eq:lem:inverse_2} using that $\vartheta_1(h \constzero^{1/2}) \geq 1$.
\end{proof}

\begin{lemma}
  \label{lem:bounded_cum_error}
  Assume \Cref{assum:regOne}$(1)$. Then for any $T \in \nsets$, $h >0$, and $q,p \in \rset^d$,
  \begin{align}
    %\label{eq:3}
    &\sum_{i=1}^{T}  \norm{ \Phiverletq[h][i](\q,p) -q} \\
    &\leq  L^{-1/2}_1 T [\{1 + h  \constzero^{1/2} \vartheta_1(h\constzero^{1/2})\}^T  -1] \defEns{\vartheta_2(h) (1+\norm{q})  + \vartheta_3(h) \norm{p}} \eqsp,
  \end{align}
where  $\vartheta_1$ is defined by  \eqref{eq:def_vartheta_1}.
\end{lemma}
\begin{proof}
For  $T \geq 2$ and $h >0$,  by  \Cref{lem:bound_first_iterate_leapfrog_b}-\ref{lem:bound_first_iterate_leapfrog_b_2},   we have
\begin{align}
\sum_{i=1}^{T-1}  \norm{ \Phiverletq[h][i](\q,p) -q}
&\leq 
%\constzero^{-1/2} \vartheta_1^{-1}(h \constzero^{1/2})  \sum_{i=1}^{T-1}\parentheseDeux{\defEns{(1+h\vartheta_1(h))^{i+1}-1} \defEns{\vartheta_2(h) (\norm{q} +1) + \vartheta_3(h) \norm{p}}} \\
 \constzero^{-1/2} \vartheta_1^{-1}(h \constzero^{1/2}) T \defEns{\{1+h \constzero^{1/2}\vartheta_1(h\constzero^{1/2})\}^{T}-1} \\
 & \qquad \qquad \times \, \defEns{\vartheta_2(h) (\norm{q} +1) + \vartheta_3(h) \norm{p}} \eqsp.
\end{align}
The proof is completed upon using that $\vartheta_1(h \constzero^{1/2}) \geq 1$.
\end{proof}


%%% Local Variables:
%%% mode: latex
%%% TeX-master: "main"
%%% End:

 %(assuming that $h \leq h_0$ for $h_0 >0$). \alain{To discuss}
\section{Influence in Completely Bounded Block-multilinear Forms}
\label{sec:proof}
\newcommand{\blocks}{\mathrm{blocks}}
\newcommand{\lt}{\mathrm{left}}
\newcommand{\rt}{\mathrm{right}}



In this section we prove the non-commutative root-influence inequality (\thmref{thm:bh-intro}),  the special case of the Aaronson-Ambainis conjecture given in \thmref{thm:aa}, and also briefly mention how the simulation result in \corref{cor:sim} follows from \thmref{thm:aa} and the results in \cite{AA14}. We first need some preliminaries from free probability theory. 



\subsection{Low-degree Polynomials of Haar Random Unitaries}

As discussed in the proof overview, we require bounds on the operator norm (as well as normalized trace) of low-degree polynomials of random unitaries and these follow from known results in free probability theory. Here we explain these connections and also prove some auxillary lemmas needed for the proof of \thmref{thm:bh-intro} and \thmref{thm:aa}. 



Let $z_{\ui}$ denote the non-commutative monomial $z_{i_1} z_{i_2} \cdots z_{i_d}$ for a $d$-tuple $\ui  = (i_1, \ldots, i_d) \in [t]^d$ and let $p(z_1, \ldots, z_t)$ be a non-commutative polynomial in the variables $z_1, \ldots, z_t$. We are interested in computing the operator norm $\|\cdot\|_{\op}$ and the normalized trace  $\tr_N$ of the polynomial $p(z_1, \ldots, z_t)$ (or its higher moments) when substituting $N \times N$ Haar random unitaries for the variables $z_i$.

As explained previously, the theory of free probability gives us tools that allow us to compute  the above in the limit $N \to \infty$. In particular, Voiculescu \cite{V98} showed that the  (normalized) trace of polynomials in Haar random unitaries and their conjugates converges to the trace of the same polynomial evaluated on certain infinite-dimensional operators called \emph{Haar unitaries} that satisfy a non-commutative notion of independence called \emph{free independence}. This was strengthened by Collins and Male \cite{CM11} who showed that such convergence also holds for the operator norm. A short primer on free probability is given in \appref{sec:free}, but for now one can think of $\CA$ as a self-adjoint algebra of bounded linear operators on a Hilbert space and $\phi$ as a trace functional for such operators in the statement given below.


\begin{theorem}[\cite{V98, CM11}] \label{thm:voiculescu}
    Let $p(z_1, \ldots, z_{2t})$ be a non-commutative polynomial in $\BR\langle z_1, \ldots, z_{2t}\rangle$. If $U_1, \ldots, U_t$ are $N \times N$ Haar random unitaries, then almost surely,
    \begin{align*}
     \ \tr_N[p(U_1, \ldots, U_t, U^*_1, \ldots, U_t^*)] &~\xrightarrow[N \to \infty]{}~ \phi[p(u_1, \ldots, u_t, u^*_1, \ldots, u^*_t)],\\
    \  \|p(U_1, \ldots, U_t, U^*_1, \ldots, U_t^*)\|_{\op} &~\xrightarrow[N \to \infty]{}~ \| p(u_1, \ldots, u_t, u^*_1, \ldots, u^*_t)\|,
    \end{align*}
    where $u_1, \ldots, u_t$ are free Haar unitaries in a $C^*$-probability space $(\CA, \phi)$ and $\|\cdot\|$ is the norm for the underlying $C^*$-algebra.
\end{theorem}




Using the above result it suffices to consider free Haar unitaries in a $C^*$-probability space to compute the operator norm and trace of polynomials of random unitaries. For a non-commutative polynomial $p(z_1, \ldots, z_t) = \sum_{|\ui|\le d} c_{\ui}z_{\ui}$, denoting by $\|p\|_2 =  \left(\sum_{|\ui| \le d} |c_{\ui}|^2\right)^{1/2}$, one can show the following easily using techniques from free probability. 

\begin{lemma} \label{thm:trace}
    Let $p(z_1, \ldots, z_t) = \sum_{|\ui|\le d} c_{\ui}z_{\ui} $ be a non-commutative degree-$d$ polynomial in $\R\langle z_1, \ldots, z_t\rangle$ and $u_1, \ldots, u_t$ be free Haar unitaries in a $C^*$-probability space $(\CA, \phi)$. Then, 
     \[ \phi[p(u_1, \ldots, u_t) (p(u_1, \ldots, u_t))^*] =  \|p\|_2^2.\]
\end{lemma}

The above implies that $\tr_N[p(U_1, \ldots, U_t) (p(U_1, \ldots, U_t))^*]$ converges to $\|p\|_2^2$ almost surely as $N \to \infty$. We shall defer the proof of \lref{thm:trace} to \appref{sec:app}, but to aid our intuition we note here that since the  $U_i$'s are independent $N \times N$ Haar random unitaries, the expected value

\[ \BE\left[\tr_N[p(U_1, \ldots, U_t) (p(U_1, \ldots, U_t))^*\right] = \|p\|_2^2,\] 
{and from concentration of measure, it is natural to expect that it converges to the above value}. 


Similarly, to compute the operator norm of $p(U_1, \ldots, U_t)$ for Haar random unitaries one can instead study the norm of the polynomial evaluated on free Haar unitaries. Such bounds are easier to prove using the trace method since free independence imposes strong restrictions on the non-commutative moments. For instance, if $U_1$ and $U_2$ are independent $N \times N$ Haar random matrices, then $\BE[\tr_N(U_1U_2U^*_1U_2^*)]$ is non-zero (albeit quite small), while the corresponding trace evaluated on free Haar unitaries $u_1$ and $u_2$ is zero, that is $\phi(u_1u_2u^*_1u_2^*) = 0$. Thus, computing the trace $\phi[p(u_1,\ldots, u_t, u^*_1, \ldots, u_t^*)]$ reduces to handling the combinatorics of the patterns of $u_i$'s and $u_i^*$'s. 

In particular, we will rely on the following result that follows from the work of Kemp and Speicher \cite{KS05}  who consider the operator norm of homogeneous polynomials evaluated on free $R$-diagonal operators, a class that includes free Haar unitaries as well. We also remark that a bound where the right-hand side below is worse by a multiplicative $O(d^{1/2})$ factor also follows from the work of Haagerup\footnote{We note that Haagerup considered the more general case of polynomials in both $u_i$'s and $u^*_i$'s.}\cite{H78} who proved it in another context, predating even the introduction of free probability theory. 


\begin{theorem}[\cite{KS05}]
\label{thm:kemp-speicher}
    Let $p(z_1, \ldots, z_t) = \sum_{|\ui| = d} c_{\ui}z_{\ui} $ be a homogeneous non-commutative degree-$d$ polynomial in $\R\langle z_1, \ldots, z_t\rangle$ and $u_1, \ldots, u_t$ be free Haar unitaries in a $C^*$-probability space. Then, 
    \[ 
    \|p(u_1, \ldots, u_t)\| \le \sqrt{e(d+1)} \cdot \|p\|_2,
    \]
    where the left-hand side denotes the norm in the underlying $C^*$-algebra. 
\end{theorem}

For completeness, we  introduce the necessary free probability background and some combinatorial details in \appref{sec:app}, and we present the fairly short proof of \thmref{thm:kemp-speicher} (from \cite{KS05}) there in a self-contained way. We shall need to extend the above bound to non-homogeneous polynomials. Let $p(z_1, \ldots, z_t) = \sum_{|\ui| \le d} c_{\ui}z_{\ui}$ and  let $p_k(z_1, \ldots, z_t) = \sum_{|\ui| = k} c_{\ui}z_{\ui}$ denote the degree-$k$ homogeneous part of $p$. Writing $p_k = p_k(u_1, \ldots, u_t)$ for $0 \le k  \le d$ and $p = p(u_1, \ldots, u_t)$, it follows from the triangle inequality,  \thmref{thm:kemp-speicher}, and Cauchy-Schwarz, that
    \begin{align*}
        \ \|p\| &\le \sum_{k=0}^d \|p_k\| 
        \le 
        \sum_{k=0}^d\sqrt{e(k+1)}\|p_k\|_2
        \le
       \sqrt{e}\left(\sum_{k=0}^d (k+1)\right)^{1/2} \left(\sum_{k=0}^d  \|p_k\|^2_2\right)^{1/2} \leq \sqrt{e}(d+1)  \cdot\|p\|_2.
    \end{align*}
Thus, we essentially get the same bound as in the homogeneous case, at the expense of an additional $O(d^{1/2})$ factor.



Collecting all the above we have the following as a direct consequence:

\begin{theorem} \label{thm:op-norm}
    Let $p(z_1, \ldots, z_t) = \sum_{|\ui|\le d} c_{\ui}z_{\ui} $ be a non-commutative degree-$d$ polynomial in $\R\langle z_1, \ldots, z_t\rangle$ and $U_1, \ldots, U_t$ be independent $N \times N$ Haar random unitaries. Then, as $N \to \infty$, the following holds almost surely, 
    \[ \tr_N[p(U_1, \ldots, U_t) (p(U_1, \ldots, U_t))^*] =  \|p\|_2^2,\]
    and
    \[ \|p(U_1, \ldots, U_t)\|_{\op} \le \sqrt{e}(d+1)  \cdot \|p\|_2,\]
    Moreover, the factor $(d+1)$ in the operator norm bound can be improved to $\sqrt{d+1}$ if the polynomial is homogeneous.
\end{theorem}

Based on the above theorem, we prove the following key lemma which captures the polar decomposition strategy mentioned in the earlier proof overview (\secref{sec:bh}). This will serve as the key ingredient in the proof of \thmref{thm:aa} and \thmref{thm:bh-intro}. 

\begin{lemma}\label{lem:polar}
    Let $p$ be a non-commutative degree-$d$ polynomial in $\R\langle y_1, \ldots, y_m, z_1, \ldots, z_t\rangle$ given by
    \[ p(y_1, \ldots, y_m, z_1, \ldots, z_t) = \sum_{i=1}^m y_i q_i(z_1, \ldots, z_t) + q_0(z_1, \ldots, z_t).\]
    Then, for every $\delta > 0$, there exist an integer $N$ and $N \times N$ unitaries $V_1,\ldots, V_m, W_1, \ldots, W_t$ such that 
    \[ \|p(V_1, \ldots, V_m, W_1, \ldots, W_t)\|_{\op} \ge \frac{1}{\sqrt{e}(d+1)} \sum_{i=1}^m \|q_i\|_2 - \delta.\]
    Moreover, the factor in front can be improved to $(e(d+1))^{-1/2}$ if $p$ is homogeneous. 
\end{lemma}

\begin{proof}[Proof of \lref{lem:polar}]
     For an arbitrary integer $N$, let us pick independent $N \times N$ Haar random unitaries $W_1, \ldots, W_t$ which we substitute for the variables $z_1,\ldots,z_t$, respectively, and let $M_i = q_i(W_1, \ldots, W_t)$ be the corresponding random matrices. Then, for any tuple of matrices $V_1, \ldots, V_m$ that we could substitute for the variables $y_1, \ldots, y_m$, we have that 
    \[ 
    p(V_1, \ldots, V_m, W_1, \ldots, W_t) = \sum_{i=1}^m V_i M_i + M_0.
    \] 
     \thmref{thm:op-norm} and union bound imply that as $N \to \infty$, with probability $1$ all the following events simultaneously hold: 
    \begin{itemize}
        \item $\|M_i\|_{\op} \le \sqrt{e}(d+1) \cdot \|q_i\|_2$ for each $i$,
        \item $\tr_N(M^*_iM_i) = \|q_i\|_2^2$ for each $i$, where $\tr_N(M)$ is the normalized trace.
    \end{itemize}
   To show that the operator norm must be large, let us fix a sufficiently large $N$ and a choice of $N\times N$ unitaries $W_1, \ldots, W_t$ such that $M_i$ satisfies $\|M_i\|_{\op} \le \sqrt{e}(d+1) \cdot \|q_i\|_2 + \epsilon$ and $\tr_N(M^*_iM_i) \ge \|q_i\|_2^2 - \epsilon$ for each $0\le i\le m$, where $\epsilon$ can be made arbitrarily small by increasing $N$. For $0 \leq i \leq m$, let $M_i = U_i P_i$ be the left polar decomposition of $M_i$, where $U_i$ is a unitary matrix and $P_i$ is a positive semidefinite matrix.
   
   We select the tuple of unitary matrices $V_1, \ldots, V_m$ that we substitute for the variables $y_1, \ldots, y_m$ to be $V_i = U_0U^*_i$ for $i \in [m]$. With this we have that $\|p(V_1, \ldots, V_m, W_1, \ldots, W_t)\|_{\op}$ is at least
    \begin{align*}
         \Big\|M_0 + \sum_{i=1}^m V_iM_i\Big\|_{\op} & = \Big\|U_0 P_0 + \sum_{i=1}^m U_0 U_i^* U_iP_i \Big\|_{\op} \\
        \ & =  \Big\|U_0 P_0 + \sum_{i=1}^m U_0 P_i\Big\|_{\op}  = \Big\| P_0 + \sum_{i=1}^m  P_i\Big\|_{\op}\ge \tr_N\Big(P_0 + \sum_{i=1}^m P_i\Big) \ge \tr_N\Big(\sum_{i=1}^m P_i\Big),
    \end{align*}
    where the last equality follows since the operator norm is unitarily invariant and the last two inequalities follow from the positive semidefiniteness of the $P_i$'s.

    For every positive semidefinite matrix $P$, we have that $\tr_N(P) \ge {\tr_N(P^2)}/{\|P\|_{\op}}$. 
  
    Hence,
     \[ \|p(V_1, \ldots, V_m, W_1, \ldots, W_t)\|_{\op} \ge \sum_{i=1}^m \frac{\tr_N(P_i^2)}{\|P_i\|_{\op}}.\]
     By our choice of $M_i$, we have that $\tr_N(P_i^2) = \tr_N(M_i^* M_i) \ge \|q_i\|_2^2 - \eps$ and $\|P_i\|_{\op} = \|M_i\|_{\op} \le \sqrt{e}(d+1)\|q_i\|_2 + \eps$. Since $\eps$ can be made arbitrarily small by increasing $N$, it follows that 
      \[ \|p(V_1, \ldots, V_m, W_1, \ldots, W_t)\|_{\op} \ge \frac1{\sqrt{e}(d+1)} \sum_{i=1}^m \|q_i\|_2 - \delta ,\]
     for large enough $N$. The improved bound for the homogeneous case follows directly by plugging the bound of \thmref{thm:op-norm} into the above proof.
\end{proof}





\subsection{Non-commutative root-influence inequality}
\label{sec:bh-proof}


For clarity in the proofs below, we remind our  convention that all tuples or blocks are denoted with boldface fonts (e.g. $\BU_1$ or $\BA$), while a single element is denoted without boldface (e.g. $U_1(i)$ or $A_i$ or $A$). Before proceeding with the proof, we restate the statement for convenience.

\bh*





\begin{proof}[Proof of \thmref{thm:bh-intro}] 
Since $f$ is homogeneous, we can write
   \begin{align*}
    f(\x_1,\ldots, \x_d) &= \sum_{i_1, \ldots, i_d \in [n]} \hf_{i_1, \ldots, i_d} ~x_1(i_1)x_2(i_2)\cdots x_d({i_d}) \\
    \ & = \sum_{i=1}^n  x_1(i) \underbrace{\left(\sum_{i_2,\ldots, i_d \in [n]} \hf_{i_1, \ldots, i_d} ~x_2(i_2)\cdots x_d({i_d})\right)}_{\textstyle := f_i(\x_2,\ldots, \x_d)}.
\end{align*}
 In this case, it follows from \eqref{eqn:inf-tensor} that for each $i \in [n]$, we have 
 \begin{equation}\label{eqn:var}
     \ \Var[f_i] = \|f_i\|^2_2 = \inf_{1,i}(f) \text{ and }  \Var[f] = \sum_{i=1}^n \inf_{1,i}(f).
 \end{equation}

  Let us denote the corresponding non-commutative block-multilinear polynomials by $f(\BU_1, \ldots, \BU_d)$ and $f_i(\BU_2, \ldots,\BU_d)$ where $\BU_b = (U_b(1), \ldots, U_b(n))$ denotes the $b^\text{th}$ block of non-commutative variables. To show a lower bound on $\cbnorm{f}$ it suffices to exhibit a collection of square matrices $\{U_b(i)\}_{b\in [d], i \in [n]}$ with operator norm at most~1, such that $\|f(\BU_1, \ldots, \BU_d)\|_{\op}$ is large. 
  
%  

Applying \lref{lem:polar} for the homogeneous case (with $p = f$, $q_i=f_i$ for $i \in [n]$, and $q_0=0)$, it follows that for every $\delta > 0$ there exists an integer $N$ and a choice of tuples of $N \times N$ unitaries $\BU_1, \ldots, \BU_d$ such that  
      \[ \cbnorm{f} \ge \|f(\BU_1, \ldots, \BU_d)\|_{\op} \ge \frac1{\sqrt{e(d+1)}} \sum_{i\in [n]} \|f_i\|_2  -\delta \stackrel{\eqref{eqn:var}}{\ge}  \frac{1}{\sqrt{e(d+1)}} \left(\sum_{i=1}^n \sqrt{\Inf_{1,i}(f)} \right) -\delta.\]
Taking $\delta \to 0$, we get the statement of the lemma. The proof for the inequality when $b=d$ is the last block follows similarly by using the right polar decomposition.
\end{proof}

\subsection{Aaronson-Ambainis Conjecture for non-homogeneous forms}

In this section, we prove \thmref{thm:aa}, which requires handling non-homogeneous forms. The proof will be similar to the proof of \thmref{thm:bh-intro} but we will need to be careful about certain details. 

\begin{proof}[Proof of \thmref{thm:aa}]
Any block-multilinear polynomial $f(x_1, \ldots, x_d)$ can be written as 
\begin{align*}
    f(\x_1,\ldots, \x_d) &= \BE f + \sum_{b\in [d]} f_b(\x_b, \x_{b+1}, \ldots, \x_d),
\end{align*}
where $f_b$ consists of all monomials of $f$ that start with a variable in the $b^\text{th}$ block $\x_b$. Note that $f_b$ depends only on the variables in blocks $\x_b, \x_{b+1},\ldots, \x_d$. Moreover, it follows from \eqref{eqn:inf-tensor} that 
 \begin{equation}\label{eqn:var-general}
     \ \Var[f] = \sum_{b \in [d]} \|f_b\|_2^2 = \sum_{b \in [d]} \Var[f_b],
 \end{equation}
so there exists a block $\beta \in [d]$ such that $\Var[f_{\beta}] \ge \frac{1}{d}\Var[f]$. 

Since $f_{\beta}$ contributes a lot to the variance, it is natural to try to find an influential variable in the block $\x_{\beta}$. Towards this end,  we pull out the variables $x_{\beta}(i)$ and write
\begin{align*}
    f_{\beta}(\x_{\beta},\ldots, \x_d) &= \sum_{i\in [n]} x_{\beta}(i) f_{\beta,i}(\x_{\beta+1}, \ldots, \x_d),
\end{align*}
for block-multilinear polynomials $f_{\beta,i}(\x_{\beta+1}, \ldots, \x_d)$. Note that some of the $f_{\beta,i}$'s could be identically zero, so let us define $S$ to be the set of those $i$ such that $f_{\beta,i}$ is non-zero. We note that
\begin{align} \label{eqn:part-inf}
  \|f_{\beta,i}\|_2^2  =  \Inf_{\beta,i}(f_{\beta}) \le \Inf_{\beta,i}(f)  
\end{align}
which implies that
\begin{align}\label{eqn:var-main}
    \frac{1}{d} \Var[f] \le \Var[f_{\beta}] = \sum_{i \in S}\|f_{\beta,i}\|_2^2 = \sum_{i \in S} \Inf_{\beta,i}(f_{\beta}).
\end{align}
\begin{sloppypar}
Denote the corresponding non-commutative block-multilinear polynomials by $f(\BU_1, \ldots, \BU_d)$,  $f_b(\BU_{b}, \ldots,\BU_d)$, and $f_{\beta}(\BU_{\beta+1}, \ldots,\BU_d)$ where $\BU_b = (U_b(1), \ldots, U_b(n))$ denotes the $b^\text{th}$ block of non-commutative variables. To show a lower bound on $\cbnorm{f}$ it suffices to exhibit a collection of square matrices $\{U_b(i)\}_{b\in [d], i \in [n]}$ with operator norm at most~1 such that $\|f(\BU_1, \ldots, \BU_d)\|_{\op}$ is large.
\end{sloppypar}
  
 We set the matrices in blocks $\BU_1, \ldots, \BU_{\beta-1}$ to be zero (that is, the all-zero matrix $\BZ$). Note that with this choice all polynomials $f_b(\U_b, \ldots, \U_d)$ where $b < \beta$ vanish and the non-commutative polynomial becomes 
 \[ f(\BZ, \ldots, \BZ, \BU_{\beta}, \BU_{\beta+1}, \ldots, \BU_d) = \sum_{i\in S} U_{\beta}(i) f_{\beta,i}(\BU_{\beta+1}, \ldots, \BU_d) + \sum_{b=\beta+1}^d f_b(\BU_b, \BU_{b+1}, \ldots, \BU_d) + \Ef,\]
  which is a non-commutative polynomial of the form considered in \lref{lem:polar} (with $m = |S|$, $q_i = f_{\beta,i}$ and $q_0 = \sum_{b=\beta+1}^d f_b + \Ef$). Thus, by \lref{lem:polar} for every small $\delta>0$ there exists an integer $N$ and a choice of $N \times N$ matrices for the blocks $\BU_{\beta},\ldots, \BU_d$ such that 
        \begin{align*}
             \ \cbnorm{f} & \ge \|f(\BZ, \ldots, \BZ, \BU_{\beta}, \BU_{\beta+1}, \ldots, \BU_d)\|_{\op} & \\
             \  & \ge \frac1{\sqrt{e}(d+1)} \sum_{i\in S} \|f_{\beta,i}\|_2 -\delta  \stackrel{\eqref{eqn:part-inf}}{=}  \frac{1}{\sqrt{e}(d+1)} \left(\sum_{i \in S} \sqrt{\Inf_{\beta,i}(f_{\beta})} \right) -\delta & \\
             \ &\stackrel{\eqref{eqn:var-main}}{\ge}  \frac{1}{\sqrt{e}(d+1)} \left( \frac{\sum_{i \in S} \Inf_{\beta,i}(f_{\beta})}{\sqrt{\maxinf(f)}} \right) -\delta  \stackrel{\eqref{eqn:part-inf}}{\ge}  \frac{1}{\sqrt{e}(d+1)^{2}} \left( \frac{\Var[f]}{ \sqrt{\maxinf(f)}} \right) -\delta
        \end{align*}
        Taking $\delta \to 0$ and using the assumption that $\|f\|_{\cb} \le 1$, we obtain the statement of the theorem:
     \[
     1\geq \cbnorm{f} \ge \frac{1}{\sqrt{e}(d+1)^{2}} \cdot \frac{\Var[f]}{\sqrt{\maxinf(f)}} \implies \maxinf(f) \ge  \frac{(\Var[f])^2}{e(d+1)^4}. \qedhere
     \]
\end{proof}
 

     
     

\subsection{Approximating completely bounded forms with decision trees}



In this section, we briefly mention how to obtain \corref{cor:sim}.
Aaronson and Ambainis \cite[Theorem 3.3]{AA14} showed that querying the most influential variable reduces the variance of the function~$f$, and if that influence is lower bounded by a polynomial in $\Var[f]/d$, then after $\poly(d)$ queries (the exact quantitative dependence can be read off from their proof), the variance of the function becomes small enough so that it can be approximated almost-everywhere by its expectation.  Since the family of degree-$d$ block-multilinear forms with completely bounded norm at most one is closed under restrictions, one can apply \thmref{thm:aa} repeatedly. This gives us \corref{cor:sim}.
\subsection{Proofs of \Cref{sec:geom-ergod-hmc}}

\subsubsection{Proof of \Cref{propo:geo_drift_MH}}
\label{sec:proof-crefpr}
By construction   \eqref{eq:def_kenel_MH}, for all $\q \in \rset^d$, we have
\begin{align}
&\Pker \Vgeo (\q) - \Vgeo(\q) = \int_{\rset^{2d}} \defEns{\Vgeo(\projq(z)) - \Vgeo(\q)} \alphagen(\q,z) \Kker(\q, \rmd z )  \\
& \qquad = \Kker\Vgeo(\q)-\Vgeo(\q) +\int_{\rset^{2d}} \defEns{\Vgeo(\projq(z)) - \Vgeo(\q)} \defEns{\alphagen(\q,z)-1} \Kker(\q, \rmd z ) \eqsp.
\end{align}
Using \eqref{eq:assum:geo_ergo_1}, this implies for all $\q \in \rset^d$,
\begin{equation}
\label{eq:proof_geo_drift_MH_1}
\Pker \Vgeo (\q)  \leq \lambdageo \Vgeo(\q) + b
+ \int_{\rset^{2d}} \defEns{\Vgeo(\projq(z)) - \Vgeo(\q)} \defEns{\alphagen(\q,z)-1} \Kker(\q, \rmd z ) \eqsp.
\end{equation}

Note that by definition \eqref{eq:def_rej_ballV} of $\rejectregion(\q)$ and $\ballV(\q)$
\begin{align}
&\int_{\rset^{2d}} \defEns{\Vgeo(\projq(z)) - \Vgeo(\q)} \defEns{\alphagen(\q,z)-1} \Kker(\q, \rmd z )
\\ &  \qquad  \qquad  \qquad  \leq    \int_{\rejectregion(\q) \cap \ballV(\q) } \defEns{ \Vgeo(\q)-\Vgeo(\projq(z))}  \Kker(\q, \rmd z ) \eqsp.
\end{align}
Therefore by \eqref{eq:assum:geo_ergo_2}, we get
\begin{equation}
 \lim_{M\to \plusinfty} \sup_{\set{\q \in \rset^d}{\Vgeo(\q) \geq M}}  \int_{\rset^{2d}} \left\{\Vgeo(\projq(z))/\Vgeo(\q) -1 \right\} \left\{ \alphagen(\q,z)-1 \right\} \Kker(\q, \rmd z ) \leq 0 \eqsp.
\end{equation}
The proof then follows from combining this result and \eqref{eq:proof_geo_drift_MH_1} since they imply
\begin{equation}
   \lim_{M\to \plusinfty} \sup_{\set{\q \in \rset^d}{\Vgeo(\q) \geq M}}  \Pker \Vgeo (\q) / \Vgeo(\q) \leq \lambda  \eqsp.
\end{equation}

\subsubsection{Proof of \Cref{lem:drift_uhmc}}
\label{sec:proof-crefl-2}


%\begin{proof}[Proof of \Cref{lem:drift_uhmc}]
Let $a \in \rset_+^*$. Under \Cref{assum:regOne}$(m-1)$ with $m \in \ocint{1,2}$,   \Cref{lem:bound_first_iterate_leapfrog_a} shows that, for all $\q_0 \in \rset^d$,
$\p \mapsto \Phiverletq[h][T](\q_0,\p)$ is Lipschitz, with a Lipschitz constant $L_{h,T} \in \rset_+$
\begin{equation}
\label{eq:definition-lipshitz}
L_{h,T} \eqdef \defEns{1+h \constzero^{1/2} \vartheta_1(h \constzero^{1/2})}^{T} \eqsp.
\end{equation}
Therefore by the log-Sobolev inequality \cite[Proposition 5.5.1, (5.4.1)]{bakry:gentil:ledoux:2014} and \eqref{eq:def_Pker_proposition_double}, we get for all $\q_0 \in \rset^d$
\begin{equation}
\PkerhmcD[h][T] \Vdrifta[\a](\q_0) \leq \exp\parenthese{(aL_{h,T})^2/2 + a \Eproof[h][T](\q_0)} \eqsp,
\end{equation}
with
\begin{equation}
\Eproof(\q_0) = (2\uppi)^{-d/2} \int_{\rset^d} \norm{\Phiverletq[h][T](\q_0,\p)} \rme^{-\norm{\p}^2/2} \rmd \p \eqsp.
\end{equation}
Set $p_0 \in \rset^d$.
Denote for all $k \in \{0,\ldots,T\}$, $q_k =
\Phiverletq[h][k](\q_0,\p_0)$ and consider the following decomposition given by  \eqref{eq:qk}:
\begin{equation}
\label{eq:drift_uhmc_3}
\norm{\q_T}^2  =  \norm{\q_0}^2 + \operatorname{A}^{(1)}_{h,T}(\q_0,\p_0) -2h^2 \operatorname{A}^{(2)}_{h,T}(\q_0,\p_0) \eqsp,
\end{equation}
where
\begin{align}
\operatorname{A}^{(1)}_{h,T}(\q_0,\p_0) & = 2Th \ps{\q_0}{ \p_0} + \norm{ Th\p_0-(Th^2/2) \nabla \F(\q_0)-h^2 \sum_{i=1}^{T-1}(T-i)\nabla \F (\q_i)}^2 \\
\operatorname{A}^{(2)}_{h,T}(\q_0, \p_0) & = \ps{\q_0}{ (T/2) \nabla \F(\q_0)+ \sum_{i=1}^{T-1}(T-i)\nabla \F (\q_i)} \eqsp.
\end{align}
Jensen's inequality shows that, for all $\q_0 \in \rset^d$,
\[
\Eproof[h][T](\q_0) \leq \left( \norm{\q_0}^2 + \bar{\operatorname{A}}^{(1)}_{h,T}(\q_0) - 2 h^2 \bar{\operatorname{A}}^{(2)}_{h,T}(\q_0) \right)^{1/2} \eqsp,
\]
where we have set $\bar{\operatorname{A}}_{h,T}^{(i)}(\q_0)= (2\uppi)^{-d/2} \int_{\rset^d} \operatorname{A}_{h,T}^{(i)}(\q_0,\p) \rme^{-\norm{\p}^2/2} \rmd \p$, $i=1,2$.
Therefore to conclude the proof, it is sufficient to show that
\begin{equation}
\label{eq:drift_uhmc_minus1}
\limsup_{\norm{\q_0} \to \plusinfty} \defEnsLigne{\Eproof(\q_0) - \norm{\q_0}} = - \infty.
\end{equation}
\begin{enumerate}[label=(\alph*),leftmargin=0cm,itemindent=0.5cm,labelwidth=1.2\itemindent,labelsep=0cm,align=left]
\item Consider the case $m \in \ooint{1,2}$. Using \Cref{assum:regOne}$(m-1)$ and  \Cref{lem:bound_first_iterate_leapfrog_b}-\ref{lem:bound_first_iterate_leapfrog_1}, we get that  there exists a constant $C_0 \geq 0$ such that for all $\p_0,\q_0 \in \rset^d$ and $i \in \{1,\dots,T-1\}$,
  \begin{equation}
    \label{eq:drift_nabla_U_q_i}
    \norm{\nabla \F(\q_i)} \leq C_0 \{1 + \norm{\p_0} + \norm{\q_0}^{m-1}\}
  \end{equation}
  which implies that
\begin{equation}
\label{eq:bound-A-1}
|\bar{A}^{(1)}_{h,T}(\q_0)| \leq C_1 \{1 + \norm{\q_0}^{2(m-1)} \} \eqsp,
\end{equation}
for some constant $C_1 \geq 0$.  On the other hand, note that for any $q_0,p_0 \in \rset^d$,  $\operatorname{A}^{(2)}_{h,T}(\q_0,\p_0) =  \operatorname{A}^{(2,1)}_{h,T}(\q_0,\p_0) +  \operatorname{A}^{(2,2)}_{h,T}(\q_0,\p_0)$ with
\begin{align}
\label{eq:definition-A-2-1}
\operatorname{A}^{(2,1)}_{h,T}(\q_0,\p_0) &= \frac{T}{2} \ps{\q_0}{ \nabla \F(\q_0)}+\sum_{i=1}^{T-1}(T-i)  \ps{\q_i}{\nabla \F (\q_i)}, \\ 
\label{eq:definition-A-2-2}
\operatorname{A}^{(2,2)}_{h,T} &=- \sum_{i=1}^{T-1}(T-i)  \ps{\q_0-\q_i}{\nabla \F (\q_i)} \eqsp.
\end{align}
Under \Cref{assum:potential:c}$(m)$, for any $q_0, p_0 \in \rset^d$, we have that
\begin{equation}
\label{eq:lower-bound-A-2-1}
\operatorname{A}_{h,T}^{(2,1)}(\q_0,\p_0) \geq \constthree \frac{T}{2} \norm{\q_0}^m - \frac{T (T-1)}{2} \constfour \eqsp.
\end{equation}
Further, by \eqref{eq:drift_nabla_U_q_i} and  \Cref{lem:bound_first_iterate_leapfrog_b}-\ref{lem:bound_first_iterate_leapfrog_1},  there exists  $C_2 \geq 0$, such that for all $\p_0,\q_0 \in \rset^d$,
\begin{equation}
\label{eq:bound-A-2-2}
|\operatorname{A}^{(2,2)}_{h,T}(\q_0,\p_0)| \leq C_2 \{1 + \norm{p_0}^2 + \norm{\q_0}^{2(m-1)} \} \eqsp,
\end{equation}
Combining \eqref{eq:lower-bound-A-2-1} and \eqref{eq:bound-A-2-2}, there exists  $C_3 \geq 0$ such that for any $q_0 \in \rset^d$,
\begin{equation}
\label{eq:bound-A-2}
\bar{\operatorname{A}}^{(2)}(\q_0) \geq \frac{T \constthree}{2} \norm{\q_0}^m - C_3 \{1 + \norm{\q_0}^{2(m-1)} \} \eqsp.
\end{equation}
Combining \eqref{eq:bound-A-1} and \eqref{eq:bound-A-2}, and using that $m < 2$, we finally obtain that \eqref{eq:drift_uhmc_minus1} holds.
% , as $\norm{\q_0} \to \infty$,
% \[
% \Eproof(\q_0) - \norm{\q_0} \leq - \constthree h^2 T^2 \norm{\q_0}^{m-1} + o(\norm{\q_0}^{m-1})
% \]
\item  By Cauchy-Schwarz and Hölder inequality and since $\nabla U$ satisfies  \Cref{assum:regOne}$(1)$, we have for any $q_0,p_0 \in \rset^d$,
\begin{align}
&\operatorname{A}^{(1)}_{h,T}(\q_0,\p_0)
\leq 2hT \norm{q_0} \norm{p_0} \\
& +3 \parentheseDeux{ h^2 T^2  \norm[2]{\p_0} +  2 h^4 T^4 \constzeroT^2 (1+ \norm[2]{ \q_0}) +   2 h^4 T^2  \constzero^2  \defEns{ \sum_{i=1}^{T-1} \norm{\q_i - q_0}}^2} \eqsp,
\end{align}
which implies using \Cref{lem:bounded_cum_error}, $\vartheta_1(s) \geq 1$ for any $s \geq 0$, and the dominated convergence theorem that
\begin{align}
\label{eq:bound-A-1-m=2}
&\limsup_{\norm{q_0} \to \plusinfty} |\bar{\operatorname{A}}^{(1)}_{h,T}(\q_0)|/ \norm[2]{q_0}  \\
&\qquad \qquad \leq
6 h^4 T^4 \left(  \constzeroT^2 +  \constzero \vartheta_2^2(h) \left[ \{1 + h \constzero^{1/2} \vartheta_1(\constzero^{1/2} h)\}^T -1 \right]^2 \right)  \eqsp.
\end{align}
% =======
% which implies, using \Cref{lem:bounded_cum_error} that, as $\norm{\q_0}` \to \infty$,
% \begin{equation}
% \label{eq:bound-A-1-m=2}
% |\bar{\operatorname{A}}^{(1)}_{h,T}(\q_0)| \leq
% h^4 T^4 \left( (3/4) \constzeroT^2 + 3 \constzero \vartheta_2^2(h) \left[ \{1 + h \constzero^{1/2} \vartheta_1(\constzero^{1/2} h)\}^T -1 \right] \right) \norm[2]{\q_0} + o(\norm{\q_0}) \eqsp.
% >>>>>>> 42af951f106d06daaead79ba0820f840f21e5191
% \end{equation}
Similarly using in addition  \Cref{assum:potential:c}($2$), we get that for any $q_0,p_0 \in \rset^d$,
\begin{align}
\operatorname{A}^{(2)}_{h,T}(\q_0,\p_0)
&= \ps{\q_0}{ (T^2/2)  \nabla \F(\q_0)+ \sum_{i=1}^{T-1}(T-i)\{\nabla \F (\q_i) - \nabla U(q_0)\}}  \\
&\geq  (T^2/2) \{\constthree  \norm[2]{q_0} - \constfour\} - T \constzero \norm{q_0} \sum_{i=1}^{T-1}\norm{ q_i - q_0 }  \eqsp.
\end{align}
Then, \Cref{lem:bounded_cum_error} and the Fatou Lemma imply that
\begin{align}
& \liminf_{\norm{q_0} \to \plusinfty} h^2 \bar{\operatorname{A}}^{(2)}_{h,T}(\q_0)/\norm{q_0}^2  \\
 & \qquad \qquad  \qquad  \geq
h^2 T^2  \left( \constthree/2 - \constzero^{1/2} \vartheta_2(h) [(1+ h \constzero^{1/2}\vartheta_1(h \constzero^{1/2}))^T-1]\right)  \eqsp.
\end{align}
Therefore, for all  $h > 0$, and $T \in \nset^*$, one obtains
\[
\limsup_{\norm{q_0} \to \plusinfty} \{ \Eproof[h][T](\q_0) \}^2  /\norm[2]{q_0}   \leq 1 -  T^2 h^2 (\constthree -  \Theta(hT))  \eqsp,
\]
where $\Theta$ is defined in \eqref{eq:definition-function-C}. The proof follows.
\end{enumerate}
%\end{proof}


\subsubsection{Proof of \Cref{le:convex}}
\label{sec:proof-crefle:convex}

Note that condition~\ref{le:convex:a}  implies that
\begin{equation}
  \label{eq:le:convex:b:conseq}
  \inf_{\set{\q}{\norm{\q}=\Rexp}} \F_0(\q) >0 \eqsp.
\end{equation}
Condition \Cref{assum:potential}-\ref{assum:potential:a} follows from \ref{le:convex:b} using that, by \ref{le:convex:a},
 for all $\q \in \rset^d$, $\norm{q} \geq \Rexp$
\[
\F_0(q) =  (\norm{q}/\Rexp)^{\m}\F_0 (\Rexp  q / \norm{q} ) 
\]

  In addition, \Cref{assum:potential:c} is also
  easy to check using the Euler's homogeneous function theorem that
$\ps{\nabla \F_0(\q)}{
 \q}= \m \F_0(\q)$ for all $\q \in \rset^d$, $\norm{\q} \geq \Rexp$.
  % Estimates \Cref{assum:potential}-\ref{assum:potential:a} follow for
  % large values of $\norm{\q}$ just by the stipulated $\m$-homogeneity of
  % $F_0$ and the assumed growth of the derivatives of $G$.
  % Let $U=S^{\circ}$ and $0<t<1$. We have $tS+(1−t)S \subset S$ by convexity, so
  % $tU+(1−t)U\subset S$. But $tU$ is open, so $tU+(1−t)U$ as well. Therefore, $tU+(1−t)U\subset S^{\circ}=U$, and

  % hence $U$ is convex.
  We show below that \Cref{assum:potential}-\ref{assum:potential:b}
  holds. First, since $\lim_{\norm{\q} \to \plusinfty} \F_0(\q) = \plusinfty$ and $\F_0$ is continuous  for all $K \geq 0$, $\lset_K=\{ \q \in \rset^d \ ; \ \F_0(\q) \leq K\}$
%     \begin{equation}
% %    \label{eq:def_lset_le:convex}
% \lset_K=\{ \q \in \rset^d \ ; \ \F_0(\q) \leq K\}
%   \end{equation}
 is compact. Besides, using \eqref{eq:le:convex:b:conseq} and that $\F_0$ is continuous,  we can define
  \begin{equation}
    \label{eq:def_M_le:convex}
M =  \sup_{\q \in \ball{0}{\Rexp}} \F_0(\q) +1 \in \ooint{1, \plusinfty}\eqsp,
  \end{equation}
and for all $\q \not \in \lset_M$,
\begin{equation}
  \label{eq:deftq_le:convex_0}
 t_q = \sup \defEns{ t \in \ccint{0,1} \ ; \ \F_0(t \q ) =M } \eqsp,
\end{equation}
%$t_q = \sup \defEns{ t \in \ccint{0,1} \ ; \ \F_0(t \q ) =M }$,
 which satisfies
% $\F_0(q) >M > \sup_{\q \in
%     \ball{0}{\Rexp}} \F_0(\q)$ therefore $\norm{\q} \geq \Rexp$ and
%  by continuity of $\F_0$, there exists $t_{\q} \in \ccint{0,1}$ such that
  \begin{equation}
    \label{eq:deftq_le:convex}
    \F_0(t_{\q} \q) = M > \sup_{\x \in \ball{0}{\Rexp}} \F_0(\x) \eqsp, \eqsp t_{\q} q \in \partial \lset_M \eqsp \text{ and } \eqsp t_{\q}
  \norm{\q} > \Rexp \eqsp.
  \end{equation}
% $\F_0(t_{\q} \q) = M > \sup_{\x \in \ball{0}{\Rexp}} \F_0(\x)$ and $t_{\q}
%   \norm{\q} \geq \Rexp$.
 Finally using \ref{le:convex:a}, we get that the set
  $\lset_M$  is
  convex.
 % In addition, since $\F_0$ is continuous $\lset_M$
 %  contains the origin in its interior and $\partial \lset_M\subset\{\q
 %  \in \rset^d \, ; \, \F_0(\q)=M\}$.

  To show \Cref{assum:potential}-\ref{assum:potential:b}, we check first
  that it is sufficient to prove that
  \begin{equation}
D^2\F_0(\x)\defEnsLigne{\nabla
    \F_0(\x)\otimes \nabla \F_0(\x)}>0 \text{ for any } \x \in \partial
  \lset_M \eqsp.
  \end{equation}
% $D^2\F_0(\x)\defEnsLigne{\nabla
%     \F_0(\x)\otimes \nabla \F_0(\x)}>0$ for any $\x \in \partial
%   \lset_M$.
 Indeed note that if this statement holds, since $\F \in C^2(\rset^d)$
  and $\partial \lset_M$ is compact, we have
  \begin{equation}
    \label{eq:le:convex:1}
\varepsilon =  \inf_{x \in \partial \lset_M} D^2\F_0(\x)\defEnsLigne{\nabla
    \F_0(\x)\otimes \nabla \F_0(\x)}>0 \eqsp.
  \end{equation}
  % Note that for all $\q \not \in \lset_M$, $\F_0(q) >M > \sup_{\q \in
  %   \ball{0}{\Rexp}} \F_0(\q)$ therefore $\norm{\q} \geq \Rexp$.  In
  % addition since $0$ is in the interior of $\lset_M$, $\F_0(0)< M$
  % define for all $\q \not \in \lset_M$, $\phi(\q) \geq 1$ such that
  % $\q = \phi(\q) \x$ for $\x \in \partial \lset_M$.
Let now $\q \not \in \lset_M$ and $t_{\q}$ defined by \eqref{eq:deftq_le:convex_0}.
Since by \ref{le:convex:a}, for all $u
  \geq 1$ and $z \in \rset^d$, $\norm{z} \geq \Rexp$, $\F_0(uz) = u^\m \F_0(z)$,
  differentiating with respect to $z$, we get $\nabla \F_0(uz) = u^{\m-1}
  \nabla \F_0(z)$ and $D^2 \F_0(uz) = u^{\m-2} D^2\F_0(z)$. Therefore by \eqref{eq:deftq_le:convex}, we get
 \begin{equation}
    \label{eq:le:convex:2}
   D^2\F_0(\q)\defEns{\nabla
  \F_0(\q)\otimes \nabla \F_0(\q)} = t_{\q}^{4-3m} D^2\F_0(t_{\q} \q)\defEns{\nabla
  \F_0(t_{\q} \q )\otimes \nabla  \F_0(t_{\q} \q)} \eqsp.
 \end{equation}
Using \eqref{eq:deftq_le:convex} again and since $\partial \lset_M$ is compact, we get that
there exists $R_2 \geq 0$ such that $t_{\q} \norm{q} \in \ccint{\Rexp,R_2}$. Hence by \eqref{eq:le:convex:2}, we have
\begin{equation}
   D^2\F_0(\q)\defEns{\nabla
  \F_0(\q)\otimes \nabla \F_0(\q)} \geq \varepsilon \norm{q}^{3m-4} \min\parentheseDeux{\Rexp^{4-3m},R_2^{4-3m}} \eqsp.
\end{equation}
Thus
 \Cref{assum:potential}-\ref{assum:potential:b} holds for $\F_0$.
Finally \ref{le:convex:b}
 implies that  the function $\F=\F_0+G$ satisfies
 \Cref{assum:potential}-\ref{assum:potential:b} as well.


 Let $ \x \in \partial \lset_M$, we now show that $D^2\F_0(\x)\defEns{\nabla
 \F_0(\x)\otimes \nabla \F_0(\x)}>0$. By Euler's homogeneous function theorem and since $M \geq 1$, we have that
 $\abs{\ps{\nabla \F_0(\x)}{\x}}\geq \m >0$.  Denote by $\Pi$ the tangent hyperplane
 of $\partial\lset_M$ at $\x$, defined by $\Pi = \{ \q \in \rset^d \, :
 \, \ps{\nabla \F_0(x) }{ \x-\q} = 0\}$.  Since $\lset_M$ is convex, for all $\q \in
 \lset_M$ and $t \in \ccint{0,1}$, $ t^{-1}( \F_0(t\q +(1-t)\x)
 -\F_0(\x))\leq 0$. So taking the limit as $t$ goes to $0$, we get that
 $\ps{\nabla \F_0(\x) }{ \q-\x} \leq 0$. Therefore, $\lset_M$ is
 contained in the half-space $\Pi^- = \{ q \in \rset^d \, ; \, \ps{\nabla
 \F_0(\x) }{ \q-\x} \leq 0 \}$.

Define the  $\m$-homogeneous
 function $\tilde{\F} : \rset^d \to \rset_+$  for all $\q \in \rset^d$  by
 \begin{equation}
\label{eq:def:tilde_F}
   \tilde{\F}(\q) = M \abs{\frac{\ps{\q}{ \nabla \F_0(\x)}}{\ps{\x}{ \nabla \F_0(\x)}}}^\m \eqsp.
 \end{equation}
 Since $\F_0(x) = M$, by \eqref{eq:def_M_le:convex}, $\norm{x} > \Rexp$
 and therefore there exists $\epsilon_0 \in \rset_+^*$ such that
 \begin{equation}
\label{eq:inclusion_ball}
   \ball{x}{\epsilon_0} \subset \rset^d \setminus \ball{0}{\Rexp}\eqsp.
 \end{equation}
 We
 now show that $\tilde{\F}(\q) \leq \F_0(\q)$ for all $\q \in
 \ball{\x}{\epsilon}$ with
 \begin{equation}
\epsilon = 2^{-1}\min\parentheseDeux{\epsilon_0, \{\ps{\x}{ \nabla \F_0(\x)}\}/\norm[2]{\nabla \F_0(\x)}} \eqsp.
 \end{equation}
  First consider $\q \in \Pi$. We next argue by contradiction that
 \begin{equation}
\label{eq:bound_Pi}
   \F_0(\q) \geq M = \tilde{\F}(\q)    \eqsp.
 \end{equation}
Indeed assume that  $\F_0(\q) < M$. Then by continuity of $\F_0$, we get that $\q \in \interior{\lset_{M}}$. But since $\lset_{M} \subset \Pi^-$, we get $\q \in \interior{(\Pi^-)}$ which is impossible since $\q \in \Pi = \boundary{\Pi^-} = \clos{\Pi^-} \setminus \interior{(\Pi^-)}$.

Let $\q \in
 \ball{\x}{\epsilon}$. Note that  $  \q= \x + \norm[-2]{\nabla \F_0(\x)}\ps{\q-\x}{\nabla \F_0 (\x)}\nabla \F_0(\x) + z$,
% \begin{equation}
%   \q= \x + t\nabla \F_0(\x) + z \eqsp,
% \end{equation}
where  $z \in \rset^d$ is orthogonal to $\nabla \F_0(\x)$. Define
\begin{equation}
  u = \frac{\ps{\x }{ \nabla \F_0(\x)}}{\ps{\q}{ \nabla \F_0(\x)}}\eqsp.
\end{equation}
Then $u\q \in \Pi$ and by \eqref{eq:bound_Pi}, $\F_0(u \q) \geq M$. If $u \geq 1$, using \ref{le:convex:a} and \eqref{eq:def:tilde_F}, we get
\begin{equation}
\label{eq:6}
  \F_0(\q) \geq  u^{-\m}M = \tilde{\F}(\q) \eqsp.
\end{equation}
In turn, if $u < 1$, since $\norm{\q- \x} \leq \epsilon_0$, by \eqref{eq:inclusion_ball} and \ref{le:convex:a}, $\F_0(\q) = u^{-1} \F_0(u \q)$ and \eqref{eq:6} still holds.

Consider the three times differentiable functions $\phi$ and $\tilde{\phi}$
defined for all $v \in \rset$ by
$$
\phi(v)=\F_0(\x+v \nabla \F_0(\x))\quad\textrm{and}\quad \tilde{\phi}(v)= \tilde{\F}(\x+v \nabla \F_0(\x)) \eqsp.
$$
First, since for all $\q \in \ball{\x}{\epsilon}$, $\F_0(\q) \geq \tilde{\F}(\q)$, we have
\begin{equation}
  \label{eq:bound_f_tildef}
  \phi(v) \geq \tilde{\phi}(v) \eqsp, \text{for all $v \in \ccint{-\epsilon/\norm{\nabla \F_0(\x)},\epsilon/\norm{\nabla \F_0(\x)}}$}\eqsp.
\end{equation}
Moreover, by definition $\F_0(\x) = \tilde{\F}(\x)$ and $\nabla \tilde{\F}(\x)$
is colinear to $ \nabla \F_0(\x)$. Using  Euler's homogeneous function theorem for $\F_0$
and $\tilde{\F}$, we get that $\nabla \tilde{\F}(\x) = \nabla
\F_0(\x)$. Therefore $\phi(0) = \tilde{\phi}(0)$, $\phi'(0) = \tilde{\phi}'(0)$. Combining these equalities, \eqref{eq:bound_f_tildef} and using a Taylor expansion around $0$ of order $2$ with exact remainder for $\phi$ and $\tilde{\phi}$ shows that necessary
\begin{equation}
  D^2\F_0(\x)\defEns{\nabla
  \F_0(\x)\otimes \nabla \F_0(\x)} =   \phi''(0) \geq \tilde{\phi}''(0) >0 \eqsp,
\end{equation}
which concludes the proof.
% Secondly, note that since $\F_0$ is continuous and
%   $\lim_{\norm{\q} \to \plusinfty} \F_0(\q) = \plusinfty$, by
%   \Cref{assum:potential_second}-\ref{le:convex:a}, there exists $M
%   \geq \Rexp+1$ such that the sublevel set $\lset_M:=\{\q \in \rset^d
%   \, ; \, \F_0(\q) \leq M \}$ is a compact convex set and contains the
%   origin in its interior. Moreover, $\partial \lset_M=\{\q \in \rset^d
%   \, ; \, \F_0(\q)=M\}$. To show
%   \Cref{assum:potential}-\ref{assum:potential:b}, we check that it is
%   sufficient to prove that $D^2\F_0(\x)\defEnsLigne{\nabla
%     \F_0(\x)\otimes \nabla \F_0(\x)}>0$ for any $\x \in \partial
%   \lset_M$. Indeed if this statement holds, since $\F \in C^3(\rset^d)$
%   and $\partial \lset_M$ is compact, there exists $\varepsilon >0$
%   such that
%   \begin{equation}
%     \label{eq:le:convex:1}
%  \inf_{x \in \partial \lset_M} D^2\F_0(\x)\defEnsLigne{\nabla
%     \F_0(\x)\otimes \nabla \F_0(\x)}>\varepsilon \eqsp.
%   \end{equation}
%   In addition since $0$ is in the interior of $\lset_M$, $\F_0(0)< M$  define for all
%   $\q \not \in \lset_M$, $\phi(\q) \geq 1$ such that $\q = \phi(\q)
%   \x$ for $\x \in \partial \lset_M$. Since for all $u \geq 0$ and $z
%   \in \rset^d$, $z \geq \Rexp$, $\F(uz) = u^\m \F(z)$, differentiating
%   with respect to $u$, we get $\nabla \F(uz) = u^{\m-1} \nabla \F(z)$
%   and $D^2 \F(uz) = u^{\m-2} D^2\F(z)$. Therefore,
%  \begin{equation}
%    D^2\F_0(\q)\defEns{\nabla
%   \F_0(\q)\otimes \nabla \F_0(\q)}> \phi(\q)^{3\m-4} D^2\F_0(\x)\defEns{\nabla
%   \F_0(\x)\otimes \nabla  \F_0(\x)} \eqsp.
%  \end{equation}
% %phi(\q) = norm{\q}/norm{\x}
%  Since $\lset_M$ is compact, there exists $C \geq 0$ such that for
%  all $\q \not \in \lset_M$, $\phi(\q) \geq C \norm{\q}$ and therefore
%  \Cref{assum:potential}-\ref{assum:potential:b} holds for $\F_0$. \Finally, the assumed growth of the derivatives of $G$ then
%  implies that  the function $\F=\F_0+G$ satisfies
%  \Cref{assum:potential}-\ref{assum:potential:b} as well.

%  Let $ \x \in \partial \lset_M$, we now show that $D^2\F_0(\x)\defEns{\nabla
%  \F_0(\x)\otimes \nabla \F_0(\x)}>0$. By the Euler relation $\ps{\nabla \F(\q)}{
%  \q}= \m \F(\q)$ for all $\q \geq \Rexp$, and since $M \geq 1$, we have that
%  $\norm{\nabla \F(\x)}\geq \m >0$.  Denote by $\Pi$ the tangent hyperplane
%  of $\partial\lset_M$ at $\x$, defined by $\Pi = \{ z \in \rset^d \, :
%  \, \ps{\nabla \F_0 }{ \x-\q} = 0\}$.  By our assumption, for all $\q \in
%  \lset_M$ and $t \in \ccint{0,1}$, $ t^{-1}( \F_0(t\q +(1-t)\x)
%  -\F(\x))\leq 0$ So taking the limit as $t$ goes to $0$, we get that
%  $\ps{\nabla \F_0(\x) }{ \q-\x} \leq 0$. Therefore, $\lset_M$ is
%  contained in the half-space $\Pi^- = \{ z \in \rset^d \, ; \, \ps{\nabla
%  \F_0(\x) }{ \q-\x} \leq 0 \}$. Define the function $\m$-homogeneous
%  function $\tilde{\F} : \rset^d \to \rset_+$ by for all $\q \in \rset^d$
%  \begin{equation}
% \label{eq:def:tilde_\F}
%    \tilde{\F}(\q) = M \norm{\frac{\ps{\q}{ \nabla \F_0(\x)}}{\ps{\x}{ \nabla \F_0(\x)}}}^\m \eqsp.
%  \end{equation}

%  We show that $\tilde{\F}(\q) \leq \F_0(\q)$ for all $\q \in
%  \ball{\x}{\epsilon}$ for
%  \begin{equation}
% \epsilon = 2^{-1}\min\parentheseDeux{1, \{\ps{\x}{ \nabla \F_0(\x)}\}/\norm[2]{\nabla \F_0(\x)}} \eqsp.
%  \end{equation}
%   \First note that for all $\q \in \Pi$, since $\lset_M
%  \subset \Pi^- $, then
%  \begin{equation}
% \label{eq:bound_Pi}
%    \F_0(\q) \geq M = \tilde{\F}(\q)    \eqsp.
%  \end{equation}
%  Now let $\q \in
%  \ball{\x}{1/2}$, note that $\q$ can be written in the form $  \q= \x + t\nabla \F_0(\x) + z$,
% % \begin{equation}
% %   \q= \x + t\nabla \F_0(\x) + z \eqsp,
% % \end{equation}
% where $t \in \ccint{-\epsilon,\epsilon}$ and $z \in \rset^d$ is orthogonal to $\nabla \F_0(\x)$. Define
% \begin{equation}
%   u = \frac{\ps{\q }{ \nabla \F_0(\x)}}{\ps{\x}{ \nabla \F_0(\x)} + t \norm[2]{\nabla \F_0(\x)}}\eqsp.
% \end{equation}
% Then $ux \in \Pi$ and by \eqref{eq:bound_Pi}, $\F_0(\q) \geq M$. Using that $\F_0$ is $\m$-homogeneous,
% \begin{equation}
%   \F_0(\q) \geq  \norm{u}^{-\m}M = \tilde{\F}(\q) \eqsp.
% \end{equation}

% Consider the three times differentiable functions $f$ and $\tilde{f}$
% defined for all $v \in \ccint{-\epsilon,\epsilon}$ by
% $$
% f(v):=\F_0(\x+v \nabla \F_0(\x))\quad\textrm{and}\quad \tilde{f}(v)= \tilde{\F}(\x+v \nabla \F_0(\x)) \eqsp.
% $$
% \First, since for all $\q \in \ball{\x}{\epsilon}$, $\F_0(\q) \geq \tilde{\F}(\q)$, we have for all $v \in \ccint{-\epsilon,\epsilon}$,
% \begin{equation}
%   \label{eq:bound_f_tildef}
%   f(v) \geq \tilde{f}(v) \eqsp.
% \end{equation}
% Moreover, by definition $\F_0(\x) = \tilde{\F}(\x)$, $\nabla \tilde{\F}(\x)$
% is parallel to $ \nabla \F_0(\x)$. Using the Euler identity for $\F_0$
% and $\tilde{\F}$, we get that $\nabla \tilde{\F}(\x) = \nabla
% \F_0(\x)$. Therefore $f(0) = \tilde{f}(0)$, $f'(0) = \tilde{f}'(0)$. Combining these inequalities, \eqref{eq:bound_f_tildef} and using a Taylor expansion around $0$ of order $3$ for $f$ and $\tilde{f}$ shows that necessary
% \begin{equation}
%   D^2\F_0(\x)\defEns{\nabla
%   \F_0(\x)\otimes \nabla \F_0(\x)} =   f''(0) \geq \tilde{f}(0) >0 \eqsp,
% \end{equation}
% which concludes the proof.


\subsubsection{Proof of \Cref  {propo:accept}}
\label{sec:proof-crefth}
We preface the proof by several technical preliminary Lemmas.


%We preface the proof by a useful Lemma.
\begin{lemma}
\label{lem:grad_Lip_F}
Assume \Cref{assum:potential}($\m$)-\ref{assum:potential:a} for some $m\in \ocint{1,2}$.
Then, for all $q,x \in \rset^d$,
$\norm{\nabla \F(\q) - \nabla \F(\x)} \leq \constone  \norm{\q-\x}$
and $\norm{\nabla \F(\q) - \nabla \F(\x)} \leq \constone (m-1)^{-1} \norm{\q-\x}^{m-1}$.
In particular, \Cref{assum:regOne}($m-1$) holds with $\constzero= \constone$ and $\constzeroT= \constone (m-1)^{-1} \vee \norm{\nabla \F(0)}$.
\end{lemma}


\begin{proof}
First by \Cref{assum:potential}($m$)-\ref{assum:potential:a}, we get for all $\q,\x \in \rset^d$,
\begin{align}
\nonumber
  \norm{\nabla \F(\q) - \nabla \F(\x)}
&= \norm{\int_{0}^1 \nabla^2 \F(\x +t(\q-\x)) \defEns{\q-\x} \rmd t } \\
\label{eq:lem_grad_Lip_F_eq_0}
&\leq \constone  \norm{\q-\x} \int_{0}^1 \defEns{1+\norm{\x +t(\q-\x)}}^{\m-2} \rmd t   \eqsp.
\end{align}
Therefore, for all $\q,\x \in \rset^d$, we get $ \normLigne{\nabla \F(\q) - \nabla \F(\x)} \leq \constone \normLigne{\q-\x}$. For all $\q, \x \in \rset^d$, since $m \in \ocint{1,2}$, we have
\begin{align}
&\int_{0}^1 \defEns{1+\norm{\x +t(\q-\x)}}^{\m-2} \rmd t  \leq  \int_{0}^1\defEns{ 1+\abs{\norm{\x} - t \norm{\q-\x}}}^{m-2}  \rmd t  \\
& \leq \int_{0}^{1 \wedge \frac{\norm{\x}}{\norm{\q-\x}}}\defEns{1+  \norm{\x} - t \norm{\q-\x}}^{m-2}  \rmd t
+ \int_{1 \wedge \frac{\norm{\x}}{\norm{\q-\x}}}^1 \defEns{1+ t \norm{\q-\x}-\norm{\x}}^{m-2}  \rmd t \\
&\leq (m-1)^{-1} \norm{\q-\x}^{m-2} \eqsp.
\end{align}
Plugging this result in \eqref{eq:lem_grad_Lip_F_eq_0} concludes the proof.

% \begin{align}
% &   \int_{0}^1 \defEns{1+\norm{\x +t(\q-\x)}}^{\m-2} \rmd t  \leq  \int_{0}^1\defEns{ 1+\abs{\norm{\x} - t \norm{\q-\x}}}^{m-2}  \rmd t  \\
% &\leq \int_{0}^{1 \wedge \frac{\norm{\x}}{\norm{\q-\x}}}\defEns{1+  \norm{\x} - t \norm{\q-\x}}^{m-2}  \rmd t
% \\
% & \phantom{\defEns{1+  \norm{\x} - t \norm{\q-\x}}^{m-2}  \rmd t}+ \int_{1 \wedge \frac{\norm{\x}}{\norm{\q-\x}}}^1 \defEns{1+ t \norm{\q-\x}-\norm{\x}}^{m-2}  \rmd t \eqsp.
% \end{align}
% Then if $\norm{\q-\x} \leq \norm{\x}$, we get
% \begin{multline}
%   \label{eq:lem_grad_Lip_F_eq_1}
%  \int_{0}^1 \defEns{1+\norm{\x +t(\q-\x)}}^{\m-2} \rmd t \leq  \int_{0}^1 \defEns{(1-t)\norm{\q-\x}}^{\m-2} \rmd t \\
% \leq (m-1)^{-1} \norm{\q-\x}^{m-2}\eqsp.
% \end{multline}
% If $\norm{\q-\x} \geq \norm{\x}$, we get
% \begin{align}
% \nonumber
%  &\int_{0}^1 \defEns{1+\norm{\x +t(\q-\x)}}^{\m-2} \rmd t \leq \int_{0}^{1 \wedge \frac{\norm{\x}}{\norm{\q-\x}}}\defEns{1+  \norm{\x} - t \norm{\q-\x}}^{m-2}  \rmd t
% \\
% \nonumber
% & \phantom{\defEns{  \norm{\x} - t \norm{\q-\x}}^{m-2}  \rmd t}+ \int_{1 \wedge \frac{\norm{\x}}{\norm{\q-\x}}}^1 \defEns{ t \norm{\q-\x}-\norm{\x}}^{m-2}  \rmd t \\
%   \label{eq:lem_grad_Lip_F_eq_2}
% & \leq ((m-1)\norm{\q-\x})^{-1} \defEns{\norm{x}^{m-1} + \norm{\q-\x}^{m-1}} \leq (m-1)^{-1} \norm{\q-\x}^{m-2} \eqsp.
% \end{align}
% Combining \eqref{eq:lem_grad_Lip_F_eq_1} and \eqref{eq:lem_grad_Lip_F_eq_2} in \eqref{eq:lem_grad_Lip_F_eq_0} concludes the proof.

% The proof is postponed to \Cref{sec:proof-crefl-1}.
% \end{proof}

%     Let $\q,\x \in \rset^d$ and assume without loss of generality that
%   $\norm{\q} < \norm{\x}$.  First by
%   \Cref{assum:potential}-\ref{assum:potential:a} and Jensen
%   inequality, using that $\m \leq 2$, $t \mapsto t^{\m-2}$ is concave on $\rset_+$,
%   we have
% \begin{align}
% \nonumber
%   \norm{\nabla \F(\q) - \nabla \F(\x)} &= \norm{\int_{0}^1 \nabla^2 \F(\x +t(\q-\x)) \defEns{\q-\x} \rmd t } \\
% \nonumber
% & \leq \constone  \norm{\q-\x} \int_{0}^1 \defEns{1+\norm{\x +t(\q-\x)}}^{\m-2} \rmd t  \\
%   \label{eq:lem_grad_Lip_F_eq_0}
%  & \leq \constone  \norm{\q-\x} \parentheseDeux{ \int_{0}^1 \defEns{1+\norm{\x +t(\q-\x)} }\rmd t }^{\m-2} \eqsp.
% \end{align}
% In addition, we have
% \begin{multline}
% \int_{0}^1 \defEns{1+\norm{\x +t(\q-\x)}} \rmd t \geq \int_{0}^1\defEns{ 1+\abs{\norm{\x} - t \norm{\q-\x}}}  \rmd t  \\
% \geq 1+ \int_{0}^{1 \wedge \frac{\norm{\x}}{\norm{\q-\x}}}\defEns{ \norm{\x} - t \norm{\q-\x}}  \rmd t
% + \int_{1 \wedge \frac{\norm{\x}}{\norm{\q-\x}}}^1 \defEns{ t \norm{\q-\x}-\norm{\x}}  \rmd t \eqsp.
% \end{multline}
% If $\norm{\q-\x} \leq \norm{\x}$, then we get
% \begin{equation}
%   \label{eq:lem_grad_Lip_F_eq_1}
%   \int_{0}^1 \defEns{1+\norm{\x +t(\q-\x)}} \rmd t
% %\geq 1+\norm{\x} - \norm{\q-\x}/2 \\
% \geq 1+\norm{\q-\x}/2 \eqsp.
% \end{equation}
% If $\norm{\q-\x} \geq \norm{\x}$, by the triangle inequality it necessarily  holds that $\norm{\x} \in \ccint{\norm{\q-\x}/2, \norm{\q-\x}}$ since $\norm{\q} \leq \norm{\x}$. Therefore we get
% \begin{equation}
%   \label{eq:lem_grad_Lip_F_eq_2}
%   \int_{0}^1 \defEns{1+\norm{\x +t(\q-\x)}} \rmd t
% %\geq 1+ \int_{0}^{\min(1,\norm{y}/\norm{\q-\x})}\defEns{ \norm{\x} - t \norm{\q-\x}}  \rmd t \\
%  \geq 1+\norm{\x}^2/(2 \norm{\q-\x})
% \geq 1+\norm{\q-\x}/8 \eqsp.
% \end{equation}
% Since $\m \leq 2$, then combining \eqref{eq:lem_grad_Lip_F_eq_1} and \eqref{eq:lem_grad_Lip_F_eq_2} in \eqref{eq:lem_grad_Lip_F_eq_0}, we get
% \begin{equation}
%    \norm{\nabla \F(\q) - \nabla \F(\x)} \leq  \constone  \norm{\q-\x}\defEns{1+ \norm{\q-\x}/8}^{\m-2} \eqsp,
% \end{equation}
% which concludes the proof.
\end{proof}

\begin{lemma}
  \label{lem:prepa_bound_diff_ham}
Assume  \Cref{assum:regOne}$(\beta)$ for $\beta \in \ocint{0,1}$. Let $\gamma \in \ooint{0,\beta}$.
\begin{enumerate}[label=(\roman*)]
\item   \label{lem:prepa_bound_diff_ham_1}
If $ \beta \in \ooint{0,1}$, for any $T \in \nsets$ and  $h_0 \in \rset^*_+$, there exist $\kappa \in \rset^*_+$ and $R \in \rset_+$ such that for all $h \in \ocint{0,h_0}$,  $\q_0,p_0 \in \rset^d$ satisfying $ \norm{p_0} \leq
\norm{\q_0}^{\gamma}$ and $\norm{\q_0} \geq R$, and $i,j,k \in
\{0,\ldots,T\}$,
\begin{align}
\label{eq:bound_iterate_q_2_prood_diff_ham_eq}
%\label{eq:bound_iterate_q_1_prood_diff_ham_1}
&\norm{q_0} \leq \kappa  \norm{\Phiverletq[h][k](q_0,p_0)}\eqsp, \\ &\norm{\Phiverletq[h][i](q_0,p_0)-\Phiverletq[h][j](q_0,p_0)} \leq \kappa h \norm{\Phiverletq[h][k](q_0,p_0)}^{\beta}  \eqsp,
\end{align}
  where $\Phiverletq[h][\ell]$ are defined by \eqref{eq:def_Phiverletq} for $\ell \in \nset^*$.
\item \label{lem:prepa_bound_diff_ham_2}
If $\beta =1$, then there exist $\kappa, \bar{S} \in \rset_+^*$ (depending only on $\constzero$ and $\constzeroT$) such that for any $T \in \nsets$, $h \in \ooint{0,\bar{S}/T}$,  $\q_0,p_0 \in \rset^d$ satisfying $ \norm{p_0} \leq
\norm{\q_0}^{\gamma}$ and $\norm{\q_0} \geq 1$, and $i,j,k \in
\{0,\ldots,T\}$,
\begin{align}
\label{eq:bound_iterate_q_2_prood_diff_ham_eq_2}
&\norm{q_0} \leq 2  \norm{\Phiverletq[h][k](q_0,p_0)} \leq 3 \norm{q_0} \eqsp, \\
&\norm{\Phiverletq[h][i](q_0,p_0)-\Phiverletq[h][j](q_0,p_0)}
\leq \kappa Th
\rme^{(1+\vartheta_1(Th))Th}  \norm{\Phiverletq[h][k](q_0,p_0)} \eqsp,
\end{align}
where $\vartheta_1$ is defined in \eqref{eq:def_vartheta_1}.
\end{enumerate}
\end{lemma}

\begin{proof}
\begin{enumerate}[label=(\roman*),leftmargin=0cm,itemindent=0.5cm,labelwidth=1.2\itemindent,labelsep=0cm,align=left]
\item
Let $T \in \nsets$, $h_0 \in \rset_+^*$ and  $h \in \ocint{0,h_0}$.
Denote for all $k \in \{0,\ldots,T\}$ by $(q_k,p_k) =
  \Phiverlet[h][k](q_0,p_0)$, $q_0, p_0 \in \rset^d$.
  By  \Cref{assum:regOne}$(\beta)$ and  \Cref{lem:bound_first_iterate_leapfrog_b}-\ref{lem:bound_first_iterate_leapfrog_1}, there exist $C \geq 0$ and  $R_1 \geq 0$ such that for all $\q_0,\p_0 \in \rset^d$ satisfying $ \norm{\p_0} \leq \norm{\q_0}^{\gamma}$ and $\norm{\q_0} \geq R_1$, for all $k \in \{0,\ldots,T\}$, we have
\begin{equation}\label{eq:bound_iterate_q_1_prood_diff_ham_0}
 \norm{\q_k-\q_0} \leq C h \norm{\q_0}^{\m-1} \eqsp.
\end{equation}
Then since $m<2$, there exists $R_2 \geq R_1$ and $\omega >0$ such that such that for all $\q_0,p_0 \in \rset^d$ satisfying $ \norm{p_0} \leq \norm{\q_0}^{\gamma}$  and $\norm{\q_0} \geq R_2$, for all $k \in \{0,\ldots,T\}$,
\begin{equation}
  \norm{\q_0} \leq \omega \norm{\q_k} \eqsp.
\end{equation}
In addition, using this inequality and
\eqref{eq:bound_iterate_q_1_prood_diff_ham_0} again, we get that for
all $\q_0,p_0 \in \rset^d$ satisfying $ \norm{p_0} \leq
\norm{\q_0}^{\gamma}$ and $\norm{\q_0} \geq R_2$, for all $i,j,k \in
\{0,\ldots,T\}$,
\begin{equation}
\label{eq:bound_iterate_q_2_prood_diff_ham_2}
\norm{\q_{i}-\q_{j}} \leq 2C h \norm{\q_0}^{\m-1} \leq 2 Ch \omega^{\m-1}\norm{\q_k}^{\m-1}  \eqsp.
\end{equation}
%
\item Let $T \in \nsets$, $h \in \rset_+^*$.
Denote for all $k \in \{0,\ldots,T\}$ by $(q_k,p_k) =
  \Phiverlet[h][k](q_0,p_0)$, $q_0, p_0 \in \rset^d$.
 By  \Cref{assum:regOne}$(1)$ and \Cref{lem:bound_first_iterate_leapfrog_b}-\ref{lem:bound_first_iterate_leapfrog_b_2}, $\vartheta(s) \geq 1$ for any $s \geq 0$, we get that  for all $\q_0, \p_0 \in \rset^{2d}$ satisfying $\norm{\p_0} \leq \norm{\q_0}^\gamma$ and $k \in \{0,\ldots,T\}$,
\begin{equation}\label{eq:bound_iterate_q_1_prood_diff_ham_0_m_2}
 \norm{\q_k-\q_0} \leq \constzero^{-1/2}\left\{(1 + h \constzero^{1/2} \vartheta_1(h \constzero^{1/2}))^{k+1}-1\right\} \left\{ \vartheta_2(h) \vee \vartheta_3(h) \right\} (1 + \norm{\q_0}) \eqsp,
\end{equation}
where $\vartheta_1$, $\vartheta_2$ and $\vartheta_3$ are defined in \eqref{eq:def_vartheta_1} and  \eqref{eq:definition-vartheta-2} respectively.

Therefore,  there exists $\bar{S} >0$ (depending only on $\constzero$ and $\constzeroT$) such that for any $T \in \nsets$ and $h \in \ooint{0,\bar{S}/T}$, for any $q_0,p_0 \in \rset^d$ satisfying $ \norm{\p_0} \leq \norm{\q_0}^{\gamma}$ and $\norm{\q_0} \geq  1 $, $ \normLigne{\Phiverletq[h][k](q_0,p_0)-\q_0} \leq \norm{q_0}/2$ for any $k \in \{0,\ldots,T\}$. As a result, for any $T \in \nsets$ and $h \in \ooint{0,\bar{S}/T}$, for any $q_0,p_0 \in \rset^d$ satisfying $ \norm{\p_0} \leq \norm{\q_0}^{\gamma}$ and $\norm{\q_0} \geq  1 $,  for any $k \in \{0,\ldots,T\}$,
\begin{equation}
  \norm{\q_0} \leq 2 \norm{\Phiverletq[h][k](q_0,p_0)} \leq 3 \norm{q_0} \eqsp.
\end{equation}
In addition, using this inequality and
\eqref{eq:bound_iterate_q_1_prood_diff_ham_0_m_2} again, we get that there exists $C \geq 1$ (depending only on $\constzero$ and $\constzeroT$) such that  for any $T \in \nsets$ and $h \in \ooint{0,\bar{S}/T}$, setting $S= hT$, and  for
all $\q_0,p_0 \in \rset^d$ satisfying $ \norm{p_0} \leq
\norm{\q_0}^{\gamma}$ and $\norm{\q_0} \geq 1$, for all $i,j,k \in
\{0,\ldots,T\}$,
\begin{align}
%\label{eq:bound_iterate_q_2_prood_diff_ham_2}
\norm{\Phiverletq[h][i](q_0,p_0) - \Phiverletq[h][j](q_0,p_0)} 
&\leq 2C S \rme^{(1+\vartheta_1(S))S} \norm{\q_0} \\
&\leq 4 C S \rme^{(1+\vartheta_1(S))S}\norm{\Phiverletq[h][k](q_0,p_0)}  \eqsp.
\end{align}
\end{enumerate}
\end{proof}

\begin{lemma}
\label{lem:variation_assum_hessian}
Assume \Cref{assum:potential}($\m$) for some $m \in \ocint{1,2}$.
Then there exist $\delta \in \ooint{0,1}$, $R_0 \in \rset_+$ and  $\BB_0 \in \rset_+^*$ such that for all $\q,\x,z \in \rset^d$, with
\begin{equation}
\label{eq:hyp_variation_assum_hessian}
\norm{\q} \geq R_0 \eqsp, \qquad \text{ and } \quad \max\parenthese{\norm{\q-\x},\norm{\q-z}} \leq \delta \norm{\q}  \eqsp,
\end{equation}
we have
\begin{equation}
D^2 \F(\q) \defEns{\nabla \F(\x) \otimes \nabla \F(z)} \geq \BB_0 \norm{\q}^{3\m-4}  \eqsp.
\end{equation}
\end{lemma}
\begin{proof}
Under \Cref{assum:potential}($m$), using \Cref{lem:grad_Lip_F}, it can be easily checked that there
exists $C_U \geq 0$ (depending only on $\constone$ and $m$) such that for all  $\q,\x,z \in \rset^d$ satisfying  \eqref{eq:hyp_variation_assum_hessian},
for $\delta \in \ooint{0,1}$ and $R_0 \geq \rhtwo$,
\begin{equation}
D^2 \F(\q) \defEns{\nabla \F(\x) \otimes \nabla \F(z)}  \geq \consttwo  \norm{\q}^{3\m-4} - C_U \{1+ \delta^{m-1} \norm{\q}^{3m-4}\}\eqsp.
\end{equation}
The proof is concluded by taking $\delta$ sufficiently small and $R_0$ sufficiently large.
\end{proof}

% \alain{comment on the Gaussian case the term $\frac{h^4}{8}\norm{ \int_0^1 \nabla ^2 \F( q_{t}) p_{0} \ \rmd t  }^2$ implies that $h$ has to be chosen at least smaller than $C (Ld^{-1/2})^{-1/2}$}
 \begin{lemma}
 \label{lem:diff_hamiltonian_taylor_exp}
 Assume that $\F$ is twice continuously differentiable. Then for all $q_0,p_0 \in \rset^d$ and $h \in \rset^*_+$, the following identity holds
 \begin{align}
%&    \Ham(q_1,p_1) - \Ham(q_0,p_0)   \\
& \Ham \circ \Phiverlet[h][1](q_0,p_0) - \Ham(q_0,p_0) =  h^2\int_0^1D^2\F(q_t)\defEns{p_{0}}^{\otimes 2} (1/2-t) \  \rmd t \\
& + h^3 \int_0^1D^2\F(q_t)\defEns{p_{0}\otimes \nabla \F(q_{0})}(t-1/4) \ \rmd t  \\
&  -\frac{h^4}{4}\int_0^1 D^2\F(q_t)\defEns{\nabla \F(q_{0})}^{\otimes 2} \ t \  \rmd t
+\frac{h^4}{8}\norm{ \int_0^1 \nabla ^2 \F( q_{t}) p_{0} \ \rmd t  }^2\\
&   - \frac{h^5}{8}\ps{\int_0^1\nabla^2\F(q_t)\nabla \F(q_0) \ \rmd t }{\int_0^1 \nabla^2 \F(q_t)p_0 \ \rmd t }
   \\
   &+\frac{h^6}{32}\norm{\int_0^1 \nabla^2 \F( q_{t})\nabla \F (q_0) \ \rmd t }^2
\eqsp,
\end{align}
where  $\Phiverlet[h][1]$ is defined in \eqref{eq:def_Phiverlet}, $(q_1,p_1) = \Phiverlet[h][1](q_0,p_0)$, and $q_t = q_0 +t (q_1-q_0)$ for $t \in \ccint{0,1}$.
\end{lemma}
\begin{proof}
Using the definition of $\Ham(q,p)= \frac{1}{2}\norm{p}^2+ \F(q)$, we get
\begin{equation}
  \Ham(q_1,p_1) - \Ham(q_0,p_0) = (1/2)(\norm{p_1}^2 - \norm{p_0}^2) + \F(q_1) - \F(q_0) \eqsp.
\end{equation}
First, Taylor's formula  with exact remainder  enables us to write
\begin{equation}\label{eq:1}
\F(q_{1})-\F(q_0)= \ps{\nabla \F(q_0)}{ (q_{1}-q_0)}+\int_0^1D^2\F(q_{t})\defEns{q_{1}-q_0 }^{\otimes 2}(1-t) \ \rmd t \eqsp.
\end{equation}
Since  $\nabla \F(q_{1}) = \nabla \F(q_0) + \int_{0}^1 \nabla^2 \F(q_t) \defEns{q_1-q_0} \rmd t $,  we get
 \begin{equation}\label{eq:2}
p_{1}=p_0-\frac{h}{2} \parenthese{\nabla \F(q_{0})+\nabla \F(q_{1})}= p_{0}- h\nabla \F(q_0)-\frac{h}{2}\int_0^1\nabla ^2\F( q_{t}) \defEns{q_{1}-q_{0}} \rmd t \eqsp.
\end{equation}
Using that $q_1 = \Phiverletq[h][1](p_0,q_0)$, with $\Phiverletq[h][1]$ defined by \eqref{eq:def_Phiverletq}, in \eqref{eq:1} and \eqref{eq:2}, we get
\begin{align}
  &\F(q_{1})-\F(q_{0})\\
&=\ps{\nabla \F(q_{0})}{ hp_{0}-(h^2/2)\nabla \F(q_{0})} + \int_0^1D^2\F(q_t)\defEns{q_{1}-q_{0}}^{\otimes 2}(1-t) \rmd t   \eqsp,
\end{align}
and
\begin{align}
&\frac{1}{2}(\norm{p_{1}}^2-\norm{p_{0}}^2)= \frac{h^2}{2}\norm{\nabla \F(q_{0})}^2+ \frac{h^2}{8} \norm{\int_0^1 \nabla^2\F(q_t)\defEns{q_{1}-q_{0} }\rmd t }^2\\
&- h\langle p_{0},\nabla \F(q_{0})\rangle -(h/2)\int_0^1 D^2\F(q_t)\defEns{p_{0}\otimes (q_{1}-q_{0})} \rmd t \\
&+ (h^2/2)\int_0^1D^2\F(q_t) \defEns{\nabla \F(q_{0})\otimes (q_{1}-q_{0})} \rmd t \eqsp.
\end{align}
Summing these equalities up  and observing appropriate cancellations yields
\begin{flalign}
\nonumber
&H(q_{1},p_{1})-H(q_{0},p_{0})=\int_0^1D^2\F(q_t)\defEns{q_{1}-q_{0}}^{\otimes 2}(1-t) \rmd t\\
\nonumber
& - (h/2)\int_0^1D^2\F(q_t)\defEns{p_{0}\otimes (q_{1}-q_{0})} \rmd t +   (h^2/8) \norm{ \int_0^1 \nabla ^2\F(q_t)\defEns{q_{1}-q_{0}} \rmd t}^2\\
\nonumber
&+ (h^2/2)\int_0^1D^2\F(q_t)\defEns{\nabla \F(q_{0})\otimes (q_{1}-q_{0})} \rmd t
\end{flalign}
\vspace{-0.8cm}
\begin{equation}
\label{eq:decomp_lem_hamil}
= I_1+I_2+I_3+I_4 \eqsp.
\end{equation}
By using $q_1 = \Phiverletq[h][1](p_0,q_0)$ again in the definition of each $I_j$ we obtain successively
\begin{align}
I_1&=h^2\int_0^1D^2\F( q_t)\defEns{p_0}^{\otimes 2} (1-t) \rmd t- h^3 \int_0^1D^2\F(q_t)\defEns{p_{0}\otimes \nabla \F(q_{0})}(1-t)\rmd t\\
&\qquad \qquad \qquad +(h^4/4)\int_0^1D^2\F(q_t) \defEns{\nabla \F(q_{0})}^{\otimes 2}(1-t)\rmd t \eqsp, \\
I_2&= - (h^2/2)\int_0^1D^2\F(q_t) \defEns{p_{0}}^{\otimes 2} \rmd t + (h^3/4) \int_0^1D^2\F(q_t)\defEns{p_{0}\otimes \nabla   \F(q_{0})} \rmd t \eqsp,\\
I_3&= (h^4/8) \norm{ \int_0^1\nabla^2 \F(q_t)p_{0} \ \rmd t }^2+(h^6/32) \norm{\int_0^1\nabla^2\F(q_t)\nabla \F(q_{0})\  \rmd t }^2
\\
& \qquad \qquad \qquad  - (h^5/8)\ps{\int_0^1\nabla^2\F(q_t)\nabla \F(q_0) \ \rmd t }{\int_0^1 \nabla^2 \F(q_t)p_0 \ \rmd t } \eqsp.
\end{align}
and
\begin{align}
  I_4& = (h^3/2)\int_0^1D^2\F(q_t)\defEns{\nabla \F(q_{0})\otimes p_{0}}\rmd t \\
  & \qquad \qquad \qquad \qquad  \qquad \qquad - (h^4/4)\int_0^1D^2\F(q_t)\defEns{\nabla \F(q_{0})}^{\otimes 2}\rmd t \eqsp,
\end{align}
Gathering all these equalities  in \eqref{eq:decomp_lem_hamil} concludes the proof.
 \end{proof}


\begin{proof}[Proof of \Cref  {propo:accept}]
Let $\gamma \in \ooint{0,m-1}$, $T \in \nsets$, $h_0 \in \rset_+^*$ and  $h \in \ocint{0,h_0}$.
Denote for all $k \in \{0,\ldots,T\}$ by $(q_k,p_k) =
  \Phiverlet[h][k](q_0,p_0)$, $q_0, p_0 \in \rset^d$.
  For all $q_0,p_0 \in \rset^d$, consider the following decomposition
\begin{equation}
  \label{eq:diff_ham_decompo}
H(p_T,q_T)-H(p_0,q_0)=\sum_{k=0}^{T-1}\defEns{H(p_{k+1},q_{k+1})-H(p_{k},q_{k})} \eqsp.
\end{equation}
We show that each term in the sum in the right hand side of this equation is nonpositive if $\norm{\q_0}$ is large enough and $\norm{p_0} \leq \norm{q_0}^{\gamma}$.
By \Cref{lem:diff_hamiltonian_taylor_exp}, we have
\begin{equation}
\label{eq:diff_ham_k}
H(q_{k+1},p_{k+1})-H(q_{k},p_{k})
= -(h^4/4)A_k+h^2 B_k+ h^3C_k+(h^4/8) D_k \eqsp,
\end{equation}
where, setting $q_{t,k}= q_k + t(q_{k+1}-q_k)$ for $t \in \ccint{0,1}$,
\begin{align}
  A_k &=\int_0^1 D^2\F(q_{t,k})\defEns{\nabla \F(q_{k})}^{\otimes 2} \ t \  \rmd t \\
B_k& = \int_0^1D^2\F(q_{t,k})\defEns{p_{k}}^{\otimes 2} (1/2-t) \  \rmd t \\
C_k & = \int_0^1D^2\F(q_{t,k})\defEns{p_{k}\otimes \nabla \F(q_{k})}(t-1/4) \ \rmd t \\
D_k & = \norm{ \int_0^1 \nabla ^2 \F( q_{t,k}) p_{k} \ \rmd t  }^2 +(h^2/4)\norm{\int_0^1 \nabla^2 \F( q_{t,k})\nabla \F (q_{k}) \ \rmd t }^2 \\
& \qquad \qquad -  h \ps{\int_0^1\nabla^2\F(q_{t,k})\nabla \F(q_{k}) \ \rmd t }{\int_0^1 \nabla^2 \F(q_{t,k})p_{k} \ \rmd t }
\end{align}
Since $q_{t,k}-q_k= -(t h^2/2) \nabla \F(q_k) + th p_k$ and $\int_{0}^1(1/2-t) \, \rmd t = 0$, we have
for all $q_0,p_0 \in \rset^d$,
\begin{align}
B_k &= \int_{0}^1 \int_0^1 D^3 \F(q_{k} +s(q_{t,k} - q_{k})) \defEns{p_{k}^{\otimes 2} \otimes (q_{t,k}-q_{k})} (1/2-t)\rmd s \ \rmd t 
  \\
  \label{eq:definition-B_k}
    &= h B_{k,1} - h^2 B_{k,2}
\end{align}
where
\begin{align}
\label{eq:definition-B_k-1}
B_{k,1} &= \int_{0}^1 \int_0^1 D^3 \F(q_{k} +s(q_{t,k} - q_{k})) \defEns{p_{k}}^{\otimes 3}t (1/2-t)\rmd s \ \rmd t \\
\label{eq:definition-B_k-2}
B_{k,2} &=  -\frac{1}{2} \int_{0}^1 \int_0^1 D^3 \F(q_{k} +s(q_{t,k} - q_{k})) \defEns{p_{k}^{\otimes 2} \otimes \nabla \F(q_k) } t (1/2-t)\rmd s \ \rmd t \eqsp.
\end{align}
Consider now the term $C_k$ in \eqref{eq:diff_ham_k}. Similarly, using again $\int_0^1 (t- 1/2) \rmd t= 0$ and then \eqref{eq:pk}, we get $C_k = C_{k,1} +  C_{k,2} + C_{k,3}$,  where
\begin{align}
\label{eq:definition-C_k-1}
&C_{k,1} = h \int_{0}^1 \int_{0}^t D^3 \F(q_{k} +s(q_{k,t} - q_{k})) \defEns{p_{k}^{\otimes 2} \otimes \nabla \F(q_k) } t (t-1/2)\rmd s \ \rmd t \\
\nonumber
& -(h^2/2)\int_{0}^1 \int_{0}^t  D^3 \F(q_{k} +s(q_{k,t} - q_{k})) \defEns{p_{k} \otimes \parenthese{\nabla \F(q_k)}^{\otimes 2} } t (t-1/2)\rmd s \ \rmd t \\
\label{eq:definition-C_k-2}
&C_{k,2} =   \int_0^1D^2\F(q_{t,k})\defEns{p_{0}\otimes \nabla \F(q_{k})} \ \rmd t \eqsp, \\
\label{eq:definition-C_k-3}
&C_{k,3} = -h \sum_{i=1}^{k-1}\int_0^1D^2\F(q_{t,k})\defEns{\nabla \F(q_i)\otimes \nabla \F(q_{k})} \ \rmd t
\\
&\qquad \qquad  \qquad \qquad - (h/2)\int_0^1D^2\F(\q_{t,k})\defEns{\parenthese{\nabla \F(\q_0)+\nabla \F (\q_k)} \otimes \nabla \F(\q_{k})} \ \rmd t
\end{align}
We will next estimate each of these terms separately.
Let $\delta \in \ooint{0,1}$ and $\BB_0 \in \rset_+^*$ be the constants defined in \Cref{lem:variation_assum_hessian}.


\begin{enumerate}[label=(\alph*),leftmargin=0cm,itemindent=0.5cm,labelwidth=1.2\itemindent,labelsep=0cm,align=left]
\item
We first consider the case $m\in \ooint{1,2}$.
  By \Cref{lem:grad_Lip_F} and \Cref{lem:bound_first_iterate_leapfrog_b}-\ref{lem:bound_first_iterate_leapfrog_1}, there exist $C \geq 0$ and  $R_1 \geq \rhtwo$ such that for all $\q_0,p_0 \in \rset^d$ satisfying $ \norm{p_0} \leq
\norm{\q_0}^{\gamma}$ and $\norm{\q_0} \geq R_1$, for all $i \in
\{0,\ldots,T\}$,
\begin{equation}
\label{eq:bound_iterate_q_3_prood_diff_ham_3_0}
\begin{aligned}
&\norm{\q_{i}-\q_{0}}\leq (\delta/2) \norm{\q_0} \\ 
&\norm{\p_i - \p_0} \leq C(\norm{\p_0} + h \norm{q_0}^{m-1}) \leq C(\norm{\q_0}^\gamma + h \norm{q_0}^{m-1}) \eqsp.
\end{aligned}
\end{equation}
By  \Cref{lem:variation_assum_hessian}, \Cref{lem:prepa_bound_diff_ham}-\ref{lem:prepa_bound_diff_ham_1} and \eqref{eq:bound_iterate_q_3_prood_diff_ham_3_0}, there exists $R_2 \geq R_1$ such that for all $\q_0,\p_0 \in \rset^d$, $\norm{\q_0} \geq R_2$ and $\norm{\p_0} \leq \norm{\q_0}^{\gamma}$,
we get that
\begin{equation}
\label{eq:bound_A_k}
  \inf_{\norm{p_0} \leq \norm{q_0}^{\gamma}} A_k \geq \BB_0 \norm{q_k}^{3\m-4} \geq
  \BB_0 \{ (1-\delta/2)^{3m-4} \wedge (1+\delta/2)^{3m-4} \} \norm{q_0}^{3\m-4} \eqsp.
\end{equation}
Hence, $\limsup_{\norm{q_0} \to \plusinfty} \sup_{\norm{p_0} \leq \norm{q_0}^{\gamma}}\defEns{A_k/\norm{q_0}^{3\m-4}} > 0$.
We now bound $B_k$. Using \Cref{assum:potential}-\ref{assum:potential:a}, \Cref{lem:grad_Lip_F} and \eqref{eq:bound_iterate_q_3_prood_diff_ham_3_0}, we get by \eqref{eq:definition-B_k} that
\begin{equation}
\label{eq:bound_B_k}
\limsup_{\norm{q_0} \to \plusinfty} \sup_{\norm{p_0} \leq \norm{q_0}^{\gamma}}\defEns{\abs{B_k}/\norm{q_0}^{4\m-6}} < \infty \eqsp.
\end{equation}
Combining \Cref{assum:potential}-\ref{assum:potential:a}, \Cref{lem:grad_Lip_F} and \eqref{eq:bound_iterate_q_3_prood_diff_ham_3_0} again, we get by crude estimate that there exists $C \geq 0$ such that
\begin{equation}
  \label{eq:bound_D_k}
  \limsup_{\norm{q_0} \to \plusinfty} \sup_{\norm{p_0} \leq \norm{q_0}^{\gamma}} \defEns{\abs{D_k}/\norm{q_0}^{4\m-6}} \leq C h^2 \eqsp.
\end{equation}
We finally bound the two terms $C_{k,1}$ and $C_{k,2}$. First,
using the same reasoning as for $B_k$, we get that
\begin{equation}
%\label{eq:bound_C_k_1}
\begin{aligned}
  &\limsup_{\norm{q_0} \to \plusinfty} \sup_{\norm{p_0} \leq \norm{q_0}^{\gamma}} \defEns{\abs{C_{k,1}}/\norm{q_0}^{4\m-6}} < \infty \eqsp, \\
  &\limsup_{\norm{q_0} \to \plusinfty} \sup_{\norm{p_0} \leq \norm{q_0}^{\gamma}} \defEns{\abs{C_{k,2}}/\norm{q_0}^{2\m-3+\gamma}} < \infty \eqsp.
\end{aligned}
\end{equation}
Arguing like in \eqref{eq:bound_A_k}, we get that
$\limsup_{\norm{q_0} \to \plusinfty}  \sup_{\norm{p_0} \leq \norm{q_0}^{\gamma}} \defEns{C_{k,3}/\norm{q_0}^{3\m-4}} < 0$.
Gathering all these results and  using that  $3m-4 \geq \max(4m-6,2m-3+\gamma)$ for $m \in \ooint{1,2}$ and $\gamma \in \ooint{0,m-1}$, we get that for all $k\in \{0, \ldots,T-1\}$,
\begin{equation}
%  \label{eq:bound_C_k}
  \limsup_{\norm{q_0} \to \plusinfty}  \sup_{\norm{p_0} \leq \norm{q_0}^{\gamma}} \defEns{\Ham(q_{k+1},p_{k+1})-\Ham(q_{k},p_{k})}/\norm{q_0}^{3m-4} < 0 \eqsp,
\end{equation}
which concludes the proof.
\item
  Consider now the case $\m=2$.
First   by \Cref{lem:grad_Lip_F} and \Cref{lem:bound_first_iterate_leapfrog_b}-\ref{lem:bound_first_iterate_leapfrog_b_2}, there exist $\bar{S}_1 \geq 0$ and  $R_1 \geq \rhtwo$ such that for all $T \in \nsets$ and $h \in \ocint{0,\bar{S}_1/T}$, $\q_0,p_0 \in \rset^d$ such that $ \norm{p_0} \leq
\norm{\q_0}^{\gamma}$ and $\norm{\q_0} \geq R_1$, and $i \in
\{0,\ldots,T\}$,
\begin{equation}\label{eq:bound_iterate_q_3_prood_diff_ham_3_0_2}
\norm{\q_{i}-\q_{0}}\leq (\delta/2) \norm{\q_0}  \eqsp.
\end{equation}
and
\begin{equation}
\label{eq:bound_iterate_q_3_prood_diff_ham_3_0_3}
\begin{aligned}
\norm{\q_{i}-\q_{0}}&\leq \norm{p_0} Th + (1/2) (T+1)^2 h^2 (\constzeroT + \constzero \delta/2) \norm{q_0}\eqsp, \\
 \norm{\p_i - \p_0} &\leq h T \{\constzeroT + (\constzeroT + \constzero \delta/2) \norm{q_0}  \} \eqsp,
\end{aligned}
\end{equation}
where $\constzero$ and $\constzeroT$ are defined in \Cref{lem:grad_Lip_F}. By  \Cref{lem:variation_assum_hessian}, \Cref{lem:prepa_bound_diff_ham}-\ref{lem:prepa_bound_diff_ham_2} and \eqref{eq:bound_iterate_q_3_prood_diff_ham_3_0_2}, there exists $R_2 \geq R_1$ such that for all $\q_0,\p_0 \in \rset^d$, $\norm{\q_0} \geq R_2$ and $\norm{\p_0} \leq \norm{\q_0}^{\gamma}$
\begin{equation}
\label{eq:bound_A_k_2}
  \inf_{\norm{p_0} \leq \norm{q_0}^{\gamma}} A_k \geq \BB_0 \norm{q_k}^{2} \geq
  \BB_0 (1-\delta/2)^{2} \norm{q_0}^{2} \eqsp.
\end{equation}
Hence,
\begin{equation}
\label{eq:bound_A_k_2_2}
\limsup_{\norm{q_0} \to \plusinfty} \sup_{\norm{p_0} \leq \norm{q_0}^{\gamma}}\defEns{A_k/\norm{q_0}^{2}} \geq   \BB_0 (1-\delta/2)^{2}  \eqsp.
\end{equation}
We now bound $B_k$. Using \Cref{assum:potential}-\ref{assum:potential:a}, \Cref{lem:grad_Lip_F} and \eqref{eq:bound_iterate_q_3_prood_diff_ham_3_0_3}, we get by \eqref{eq:definition-B_k} that there exists $\rmD_1 \geq 0$ which does not depend on $T$ and $h$ such that
\begin{equation}
\label{eq:bound_B_k_2}
\limsup_{\norm{q_0} \to \plusinfty} \sup_{\norm{p_0} \leq \norm{q_0}^{\gamma}}\defEns{\abs{B_k}/\norm{q_0}^{2}} \leq \rmD_1 h  \{(hT)^3 + (hT)^4\}  \eqsp.
\end{equation}
Combining \Cref{assum:potential}-\ref{assum:potential:a}, \Cref{lem:grad_Lip_F} and \eqref{eq:bound_iterate_q_3_prood_diff_ham_3_0_3} again, we get by crude estimate that there exists $\rmD_2 \geq 0$ which does not depend on $T$ and $h$ such that
\begin{equation}
  \label{eq:bound_D_k_2}
  \limsup_{\norm{q_0} \to \plusinfty} \sup_{\norm{p_0} \leq \norm{q_0}^{\gamma}} \defEns{\abs{D_k}/\norm{q_0}^{2}} \leq \rmD_2 (hT)^2 \eqsp.
\end{equation}
We finally bound the two terms $C_{k,1}$ and $C_{k,2}$. First,
using the same reasoning as for $B_k$, we get that there exists $\rmD_3 \geq 0$ which does not depend on $T$ and $h$ such that
\begin{equation}
\label{eq:bound_C_k_2}
\begin{aligned}
&\limsup_{\norm{q_0} \to \plusinfty} \sup_{\norm{p_0} \leq \norm{q_0}^{\gamma}} \defEns{\abs{C_{k,1}}/\norm{q_0}^{2}} < \rmD_3 h \{(hT)^4 + (hT)^5\} \eqsp,\\
&\limsup_{\norm{q_0} \to \plusinfty} \sup_{\norm{p_0} \leq \norm{q_0}^{\gamma}} \defEns{\abs{C_{k,2}}/\norm{q_0}^{1+\gamma}} < \infty \eqsp.
\end{aligned}
\end{equation}
Finally, arguing like in \eqref{eq:bound_A_k_2_2}, we get that
\begin{equation}
\label{eq:bound_C_k_2_2}
  \limsup_{\norm{q_0} \to \plusinfty}  \sup_{\norm{p_0} \leq \norm{q_0}^{\gamma}} \defEns{C_{k,3}/\norm{q_0}^{2}} < 0 \eqsp.
\end{equation}
Combining \eqref{eq:bound_A_k_2_2}-\eqref{eq:bound_B_k_2}-\eqref{eq:bound_D_k_2}-\eqref{eq:bound_C_k_2} and \eqref{eq:bound_C_k_2_2} in  \eqref{eq:diff_ham_k}, and using that  $2 \geq 1+ \gamma $ for  $\gamma \in \ooint{0,1}$, we get that for all $k\in \{0, \ldots,T-1\}$,
\begin{align}
  &  \limsup_{\norm{q_0} \to \plusinfty}  \sup_{\norm{p_0} \leq \norm{q_0}^{\gamma}} \defEns{\Ham(q_{k+1},p_{k+1})-\Ham(q_{k},p_{k})}/\norm{q_0}^{2} \\
  & \qquad \qquad\qquad \qquad \leq - \BB_0 (1-\delta/2)^{2} h^4 + \rmD_1\{(hT)^3 + (hT)^4\} h^3 \\
  & \qquad \qquad  \qquad \qquad \qquad + \rmD_2(hT)^2h^4 + \rmD_3\{(hT)^4 + (hT)^5\} h^4\eqsp.
%& \qquad \qquad   < - \BB_0 (1-\delta/2)^{2} h^4 + \rmD_1\{(hT)^3 + (hT)^4\} h^3 + \rmD_2\bar{S}_3^2 h^4 + \rmD_3\{ \bar{S}_3^4 + \bar{S}_3^5\} h^4\eqsp.
\end{align}
Therefore, there exists $\bar{S}_4\leq \bar{S}_3$ such for any $T\in \nsets$, $h \in \ocint{0,\bar{S}_4/T^{3/2}}$,
\begin{equation}
%  \label{eq:9}
    \limsup_{\norm{q_0} \to \plusinfty}  \sup_{\norm{p_0} \leq \norm{q_0}^{\gamma}} \defEns{\Ham(q_{k+1},p_{k+1})-\Ham(q_{k},p_{k})}/\norm{q_0}^{2}< 0 \eqsp,
  \end{equation}
  which completes the proof.
\end{enumerate}
\end{proof}

\subsubsection{Proof of \Cref{propo:accept_pertub}}
\label{sec:proof-crefth_accept_2}


\begin{lemma}
\label{lem:variation_assum_hessian_pertub}
Assume \Cref{ass:pertub}.
Then there exist $\delta \in \ooint{0,1}$, $R_1 \in \rset_+$ $\BB_1 \in \rset_+^*$ such that for all $\q,\x,z \in \rset^d$, with
\begin{equation}
\label{eq:hyp_variation_assum_hessian_pertub}
\norm{q} \geq R_1 \eqsp, \qquad  \max\parenthese{\norm{\q-\x},\norm{\q-z}} \leq \delta \norm{\q}  \eqsp,
\end{equation}
we have
\begin{equation}
  % \ps{\Sigmabf^2 \x}{\Sigmabf z} \geq \BB_1 \norm{\q}^{2} \eqsp,  \quad \ps{\Sigmabf \nabla \F(\x)}{\Sigmabf z} \geq \BB_1 \norm{\q}^{2}  \eqsp.
  \ps{\Sigmabf \nabla \F(\x)}{\Sigmabf z} \geq \BB_1 \norm{\q}^{2}  \eqsp.
\end{equation}
\end{lemma}
\begin{proof}
Under \Cref{ass:pertub},  it can be easily checked that there
exists $\tilde{C}_U \geq 0$ (depending only on $\constfive$ and $\Sigmabf$) such that for all  $\q,\x,z \in \rset^d$ satisfying  \eqref{eq:hyp_variation_assum_hessian} for $\delta \in \ooint{0,1}$ and $R_1 \in \rset_+$,
\begin{equation}
% \ps{\Sigmabf^2 x}{ \Sigmabf z  }  \geq \ps{\Sigmabf^2 q}{ \Sigmabf q} - \tilde{C}_U  \delta \norm{\q}^{2} \eqsp, \quad \ps{\Sigmabf \nabla \F(\x)}{\Sigmabf  z}  \geq \ps{\Sigmabf^2 q}{\Sigmabf q} - \tilde{C}_U  (\delta \norm{\q}^{2} + \norm[\rho]{\q})\eqsp.
 \ps{\Sigmabf \nabla \F(\x)}{\Sigmabf  z}  \geq \ps{\Sigmabf^2 q}{\Sigmabf q} - \tilde{C}_U  (\delta \norm{\q}^{2} + \norm[\rho]{\q})\eqsp, \quad \norm{q} \geq R_1 \eqsp.
\end{equation}
The proof is concluded by using that $\Sigmabf$ is definite positive and  taking $\delta$ sufficiently small and $R_1$ sufficiently large.
\end{proof}

\begin{proof}[Proof of \Cref{propo:accept_pertub}]
  Note that by \Cref{ass:pertub}, \Cref{lem:bound_first_iterate_leapfrog_b}-\ref{lem:bound_first_iterate_leapfrog_b_2}, \Cref{lem:prepa_bound_diff_ham}-\ref{lem:prepa_bound_diff_ham_2} and \Cref{lem:variation_assum_hessian_pertub}, there exists $\BB_1,\bar{S}_1 >0$, $R_1 \geq 0$, such that for any $T \in \nsets$, $h \in \ocint{0,\bar{S}_1/T}$, $q_0,p_0 \in \rset^d$, $\norm{p_0} \leq \norm{q_0}^{\gamma}$, $\norm{q_0} \geq \max(1,R_1)$ and $k,i \in \{0, \ldots,T\}$, $\norm{q_0} \leq 2 \norm{q_k} \leq 3 \norm{q_0}$, $\abs{\tilde{U}(q_k)} \leq C_1 \norm{q_0}^{\rho}$, 
  \begin{equation}
\label{eq:proof_pertub_accept_1}
 \ps{\Sigmabf \nabla U(q_i)}{\Sigma q_k} \geq \BB_1 \norm[2]{q_k}  \eqsp, \quad  \norm{\nabla U(q_i)} \leq C_1 \norm{q_k} \eqsp,
  \end{equation}
  where $q_k = \Phiverletq[T][k](q_0,p_0)$ and  $C_1= \max(4\constfive, 3(\norm{\Sigmabf} + 2 \constfive))$.
  Let now $T \in \nsets$, $h \in \ocint{0,\bar{S}_1/T}$ and denote for any $k \in \{0,\ldots,T\}$, $(q_k,p_k) = \Phiverlet[T][k](q_0,p_0)$ for $q_0,p_0 \in \rset^d$. We consider the following decomposition:
  \begin{equation}
\label{eq:proof_pertub_accept_2}
H(p_T,q_T)-H(p_0,q_0)=\sum_{k=0}^{T-1}\defEns{H(p_{k+1},q_{k+1})-H(p_{k},q_{k})} \eqsp.
\end{equation}
We show below that there exists $\bar{S} < \bar{S}_1$ such that, for all $h \geq 0$ and $T \geq 0$ satisfying $hT \leq \bar{S}$,
\begin{equation}
\label{eq:proof_pertub_accept_3}
\limsup_{\norm{q_0} \to \plusinfty} \sup_{\norm{p_0} \leq \norm{q_0}^{\gamma}} [ \defEns{H(p_{k+1},q_{k+1})-H(p_{k},q_{k})}/\norm[2]{q_0}] <0 \eqsp,
\end{equation}
from which  the proof follows.
First for any $q_0,p_0 \in \rset^d$, $k \in \{0,\ldots,T-1\}$, we have
\begin{equation}
\label{eq:proof_pertub_accept_4}
  H(p_{k+1},q_{k+1})-H(p_{k},q_{k}) = A_k + B_k + C_k \eqsp,
\end{equation}
where $2 A_k =  \ps{\Sigmabf q_{k+1}}{q_{k+1}} - \ps{\Sigmabf q_{k}}{q_{k}}$,
$B_k = \tilde{U}(q_{k+1}) - \tilde{U}(q_k)$, and $2 C_k = \norm[2]{p_{k+1}} - \norm[2]{p_k}$.
By \eqref{eq:proof_pertub_accept_1} and \Cref{ass:pertub}, we have
\begin{equation}
\label{eq:proof_pertub_accept_6}
  \lim_{\norm{q_0} \to \plusinfty} \sup_{\norm{p_0} \leq \norm{q_0}^\gamma} \abs{B_k}/\norm[2]{q_0} =0 \eqsp,
\end{equation}
and
\begin{align}
  \label{eq:proof_pertub_accept_7}
  A_k &= h\ps{\Sigmabf p_k}{q_k} + h^2 \ps{\Sigmabf p_k}{p_k}/2 -h^2 \ps{\Sigmabf q_k }{\Sigmabf q_k }/2 \\
  & \qquad \qquad - h^3 \ps{\Sigmabf p_k}{\Sigmabf q_k }/2  +  h^4 \ps{\Sigmabf^2 q_k}{\Sigmabf q_k } /8  + A_{k,1}  \eqsp,
\end{align}
\begin{align}
    C_k & = -h\ps{\Sigmabf p_k}{q_k} -h^2 \ps{\Sigmabf p_k}{p_k}/2 +h^2 \ps{\Sigmabf q_k }{\Sigmabf q_k }/2\\
  & \qquad \quad  +3h^3 \ps{\Sigmabf p_k}{\Sigmabf q_k}/4  + h^4 \ps{\Sigmabf p_k }{\Sigmabf p_k}/8  
    -h^4 \ps{\Sigmabf^2 q_k }{\Sigmabf q_k} /4 \\
  &\qquad \quad -h^5 \ps{\Sigmabf^2 p_k}{q_k}/8 
    + h^6 \ps{\Sigmabf^2 q_k }{\Sigmabf^2 q_k}/32  + C_{k,1} \eqsp,
    \label{eq:proof_pertub_accept_8}
\end{align}
where
\begin{equation}
\label{eq:proof_pertub_accept_9}
  \lim_{\norm{q_0} \to \plusinfty} \sup_{\norm{p_0} \leq \norm{q_0}^\gamma} \{\abs{A_{k,1}} + \abs{C_{k,1}}\}/\norm[2]{q_0} =0 \eqsp,
\end{equation}
Using \eqref{eq:proof_pertub_accept_4}, \eqref{eq:proof_pertub_accept_7} and \eqref{eq:proof_pertub_accept_8}, we obtain that for any $q_0,p_0 \in \rset^d$,
\begin{equation}
  \label{eq:proof_pertub_accept_10}
  H(q_{k+1},p_{k+1}) - H(q_k,p_k) = D_k + A_{k,1}+  B_k + C_{k,1} \eqsp,
\end{equation}
where
\begin{align}
  D_k &= h^3 \ps{\Sigmabf p_k}{\Sigmabf q_k}/4 + h^4 \ps{\Sigmabf p_k }{\Sigmabf p_k}/8   -h^4 \ps{\Sigmabf^2 q_k }{\Sigmabf q_k} /8 \\
&  \qquad \qquad \qquad -h^5 \ps{\Sigmabf^2 p_k}{q_k}/8 + h^6 \ps{\Sigmabf^2 q_k }{\Sigmabf^2 q_k}/32 \eqsp.
\end{align}
Using that for $k \in \{1,\ldots,T\}$,  $p_k= p_0 -(h/2)\{\nabla U(q_0) + \nabla U(q_k)\} - h \sum_{i=1}^{k-1} \nabla U(q_i)$ and \eqref{eq:proof_pertub_accept_1}, we obtain that for any $k\in \{1,\ldots,T\}$ and $q_0,p_0$, $\norm{q_0} \geq \max(1,R_1)$, $\norm{p_0} \geq \norm[\gamma]{q_0}$,
\begin{align}
  D_k &\leq -h^4 k \BB_1 \norm[2]{q_k}/8 + h^6k^2 \norm{\Sigmabf}^2 C_1 \norm[2]{q_k}  - h^4 \ps{\Sigmabf^2 q_k }{\Sigmabf q_k} /8 \\
      &  \qquad \qquad\qquad \qquad+  h^6 k C_1 \norm{\Sigmabf}^2 \norm[2]{q_k}/8 +  h^6 \norm{\Sigmabf}^4 \norm[2]{q_k}/32     + D_{k,1} \eqsp,
\end{align}
where
\begin{equation}
\label{eq:proof_pertub_accept_lim_D}
    \lim_{\norm{q_0} \to \plusinfty} \sup_{\norm{p_0} \leq \norm{q_0}^\gamma} \abs{D_{k,1}}/\norm[2]{q_0} =0\eqsp.
  \end{equation}
  Define
  \begin{equation}
\label{eq:proof_pertub_accept_def_S_2}
    \bar{S}_2 = \min\defEns{ S \in \ocint{0,\bar{S}_1} \, : \, S^2 ( 2 C_1  \norm{\Sigmabf}^2 + \norm{\Sigmabf}^4)  - \BB_1/8 \geq -\BB_1/16} \eqsp.
  \end{equation}
  Then, if $Th \leq \bar{S_2}$ for any  $q_0,p_0$, $\norm{q_0} \geq \max(1,R_1)$, $\norm{p_0} \geq \norm[\gamma]{q_0}$, we get that
  \begin{equation}
\label{eq:proof_pertub_accept_bound_D}
    D_k \leq -\BB_1h^4 k \norm[2]{q_k}/16 + D_{k,1} \eqsp.
  \end{equation}
  Similarly using that $\Sigmabf$ is definite positive, we obtain that there exist $\BB_2 >0$ and  $\bar{S}_3  \in \ocint{0,\bar{S}_1}$ such that if $hT \leq \bar{S}_3$, for any  $q_0,p_0$, $\norm{q_0} \geq \max(1,R_1)$, $\norm{p_0} \geq \norm[\gamma]{q_0}$, we get that
  \begin{equation}
\label{eq:proof_pertub_accept_bound_D_0}
    D_0 \leq -\BB_2 \norm[2]{q_0} + D_{0,1} \eqsp,
 \quad \text{   where }
    \lim_{\norm{q_0} \to \plusinfty} \sup_{\norm{p_0} \leq \norm{q_0}^\gamma} \abs{D_{0,1}}/\norm[2]{q_0} =0\eqsp.
  \end{equation}
  Combining \eqref{eq:proof_pertub_accept_6}-\eqref{eq:proof_pertub_accept_9}-\eqref{eq:proof_pertub_accept_lim_D}-\eqref{eq:proof_pertub_accept_bound_D} and \eqref{eq:proof_pertub_accept_bound_D_0} in \eqref{eq:proof_pertub_accept_10}, we obtain that \eqref{eq:proof_pertub_accept_3} holds with $\bar{S} = \min(\bar{S}_2,\bar{S}_3)$ since \eqref{eq:proof_pertub_accept_1} implies that  $\norm{q_k} \geq \norm{q_0}/2$.
\end{proof}

%%% Local Variables:
%%% mode: latex
%%% TeX-master: "main"
%%% End:



\appendix
\section{Harris recurrence for mixture of Metropolis-Hastings type Markov kernels}
\label{sec:harr-recurr-metr}
Let $(\Xset,\Xtribu)$ be a measurable space and $\lambda$ be a
$\sigma$-finite measure on $\Xtribu$.  For all $i \in \nset^*$,
let $\alpha_i: \Xset \times \Xset \to \ccint{0,1}$ be a measurable function and   $\qker_i : \Xset
\times \Xset \to \ccint{0,\plusinfty}$ be a Markov transition density \wrt\ $\lambda$. Consider the Markov kernel
$\kernel_i$ on $\Xset \times \Xtribu$ defined by
\begin{equation}
  \label{eq:form_MH_gene}
  \kernel_i(x,\eventA) = \int_{\eventA} \alpha_i(x,y) \qker_i(x,y) \lambda(\rmd y ) + \updelta_x(\eventA) r_i(x) \eqsp, \quad  \text{$x \in \Xset$  and $\eventA \in \Xtribu$,}
\end{equation}
where for all $x
\in \Xset$
\begin{equation}
  \label{eq:def_r_i_harris_tierney}
  r_i(x) = 1 - \int_{\Xset} \alpha_i(x,y) \qker_i(x,y) \lambda(\rmd y ) \eqsp.
\end{equation}
For instance,  $\kernel_i$ may be a Markov kernel associated to the Metropolis-Hastings
algorithm, \ie
\begin{equation}
\label{eq:definition-MH-ratio}
  \alpha_i(x,y) =
  \begin{cases}
\min\parentheseDeux{1, \frac{\pi(y) \qker_i(y,x)}{\pi(x) \qker_i(x,y)}} \eqsp, & \text{ if } \pi(x) \qker_i(x,y) >0 \eqsp,\\
1\eqsp, & \text{otherwise} \eqsp,
  \end{cases}
\end{equation}
for some probability density $\pi: \Xset \to \coint{0,\plusinfty}$
with respect to $\lambda$.
We use the results below in the case
where for any $i \in \nsets$, $\kernel_i$ is a Markov kernel associated to the HMC algorithm.
\cite[Corollary 2]{tierney:1994}
considers Metropolis-Hastings kernels $\kernel_i$ with $\alpha_i$ defined by \eqref{eq:definition-MH-ratio} and shows that that if $\kernel_i$ is irreducible, then $\kernel_i$ is Harris recurrent. We extend this
result to kernels $\kernel_i$ of the form \eqref{eq:form_MH_gene} (but that do not satisfy \eqref{eq:definition-MH-ratio}) and  mixture of Markov kernels $\kernel_{\bfvarpi}$ defined on
$(\Xset, \Xtribu)$ by
\begin{equation}
\label{eq:mixture_kernel_Harris}
  \kernel_{\bfvarpi} = \sum_{i \in \nset^*} \varpi_i \kernel_i
\end{equation}
where $(\varpi_i)_{i\in
  \nset^*}$ is a sequence of non-negative numbers satisfying $\sum_{i
  \in \nset^*} \varpi_i = 1$.

%The proof of this result can be extended  to $\kernel$ of the form \eqref{eq:form_MH_gene}.
\begin{proposition}
  \label{propo:harris_rec}
  Let $\kernel_{\bfvarpi}$ be the Markov kernel given by
  \eqref{eq:mixture_kernel_Harris} and associated with the sequence of
  Markov kernel $(\kernel_i)_{i \in \nset^*}$ given by
  \eqref{eq:form_MH_gene}.  Let $\pi$ be a probability measure on
  $(\Xset,\Xtribu)$. Assume that $\pi$ and $\lambda$ are mutually
  absolutely continuous and for all $i \in \nset^*$, $\pi$ is invariant for $\kernel_i$. If $\kernel_{\bfvarpi}$ is irreducible and there exists $i \in \nset^*$ such that $\varpi_i >0$ and for all $x \in \Xset$ $r_i(x) <1$, with $r_i$ defined by \eqref{eq:def_r_i_harris_tierney},  then $\kernel$ is Harris
  recurrent.
\end{proposition}

\begin{proof}
A bounded measurable function is said to be harmonic if  $\kernel_{\bfvarpi}\harmonic = \harmonic$.
By \cite[Theorem 17.1.4, Theorem
17.1.7]{meyn:tweedie:2009} a Markov kernel $\kernel_{\bfvarpi}$ is Harris recurrent if $\kernel_{\bfvarpi}$ is recurrent and any
bounded harmonic function $\harmonic : \rset^d \to \rset$ is constant.
By \cite[Theorem 10.1.1]{meyn:tweedie:2009}, since $\kernel_{\bfvarpi}$ is irreducible and admits $\pi$ as an invariant probability measure, then $\kernel_{\bfvarpi}$ is recurrent.
On the other hand, any bounded harmonic function $\phi$ is $\pi$-almost surely equal to $\pi(\phi)$ by \cite[Theorem 17.1.1, Lemma 17.1.1]{meyn:tweedie:2009}.
Using that $\pi$ and $\lambda$ are mutually
absolutely continuous, and $\pi$ is an invariant probability measure for $\kernel_i$ for all $i \in \nset^*$,  we get by \eqref{eq:form_MH_gene} that for all $x \in \Xset$
\begin{equation}
   \kernel_{\bfvarpi} \harmonic(x) =  \sum_{i\in \nset^*} \varpi_i \defEns{\pi(\harmonic) (1-r_i(x)) + \harmonic(x)r_i(x)} \eqsp.
\end{equation}
Combining this result with $\kernel_{\bfvarpi} \phi = \phi$, we get for all $x \in \Xset$
\begin{equation}
\{\harmonic(x)-\pi(\harmonic)\} \sum_{i\in \nset^*} \varpi_i \{1-r_i(x)\}= 0\eqsp.
\end{equation}
The condition that there exists $i \in \nset^*$ such that $\varpi_i >0$ and for all $x \in \Xset$ $r_i(x) <1$, implies  that for all $x \in \Xset$, $\harmonic(x) = \pi(\harmonic)$.
\end{proof}
%%% Local Variables:
%%% mode: latex
%%% TeX-master: "main"
%%% End:



\bibliographystyle{plain}
\bibliography{../Bibliography/bibliography}
\end{document}

%%% Local Variables:
%%% mode: latex
%%% TeX-master: t
%%% End:
