\documentclass[reqno,11pt]{article}
\usepackage{lmodern}
% \arxiv{arXiv:0000.0000}

% \documentclass[reqno]{article}
% \startlocaldefs
% \usepackage{fancyhdr}
% \usepackage[dvips]{epsfig}
\usepackage{amsmath}
\usepackage{color}
\usepackage{mathtools}
\mathtoolsset{showonlyrefs}
\usepackage{amsfonts,amstext, amssymb,amsthm, amsmath, rotating, latexsym,wasysym}
\usepackage{graphicx,bm,bbm,bigints}% ,
\usepackage{adjustbox}
% \usepackage[russian]{babel}

\usepackage[colorinlistoftodos]{todonotes}

% \usepackage{floatrow}
% Table float box with bottom caption, box width adjusted to content
% \newfloatcommand{capbtabbox}{table}[][\FBwidth]

% \usepackage{blindtext}

% \usepackage[authoryear]{natbib}
% \usepackage{authblk}
\usepackage{footnote}
\makesavenoteenv{tabular}
\makesavenoteenv{table}
\usepackage{nameref,xr}
% \usepackage{zref-xr}
% \zxrsetup{toltxlabel}
\usepackage[caption=false]{subfig}
\usepackage[subfigure]{tocloft}


% \usepackage[longnamesfirst,sort,comma]{natbib}% Citation support using natbib.sty
% \bibpunct[, ]{(}{)}{;}{a}{,}{,}% Citation support using natbib.sty
% \renewcommand\bibfont{\fontsize{10}{12}\selectfont}% To set the list of references in 10 point font using natbib.sty

\usepackage{multirow}
\usepackage{booktabs}
%\usepackage{breakcites}
\usepackage{epstopdf}
%\usepackage{euler}
% \usepackage[]{geometry}
%\usepackage{nath}
%\usepackage{stmaryrd}
\usepackage{bigstrut}
% \usepackage{mathrsfs}
% \usepackage{setspace}
%\MakeRobust{\eqref}
\usepackage{enumitem}
%\usepackage{breakcites}
% \usepackage{etex}
% \usepackage[thinlines,thiklines]{easybmat}
% 
% \newenvironment{definition}[1][Definition:]{\begin{trivlist}
% \item[\hskip \labelsep {\bfseries #1}]}{\end{trivlist}}

% \pdfstringdefDisableCommands{\def\eqref#1{(\ref{#1})}}
\DeclareRobustCommand{\rchi}{{\mathpalette\irchi\relax}}
\newcommand{\irchi}[2]{\raisebox{\depth}{$#1\chi$}}
%  \usepackage[ampersand]{mathtools}
% \mathtoolsset{showonlyrefs}

% filecolor=red
% \usepackage[noabbrev,capitalize]{cleveref}

% \usepackage[pagewise]{lineno}
% \usepackage[toc,page]{appendix}
% \usepackage[nottoc]{tocbibind}
% \usepackage[english]{babel}
% \usepackage{relsize}
% \usepackage{cellspace}
% \usepackage[hyphenbreaks]{breakurl}
% \usepackage[authoryear]{natbib}

%[colorlinks,citecolor=blue,urlcolor=blue,bookmarks=false]

%  \textwidth = 6.0 in \textheight = 8.5 in \oddsidemargin = 0.3 in
%  \evensidemargin = 0.3 in \topmargin = 0.0 in \headheight = 0.0 in
% \headsep = 0.2 in \parskip = 0.0 in \parindent = 0.2 in
%\def\baselinestretch{1.2}f

% \usepackage[countmax]{subfloat}
% \startlocaldefs

\theoremstyle{plain}
\newtheorem{thm}{Theorem}[section]
\newtheorem{lemma}[thm]{Lemma}
\newtheorem{cor}[thm]{Corollary}
\newtheorem{prop}[thm]{Proposition}
\newtheorem{remark}[thm]{Remark}
\newtheorem{example}[thm]{Example}
\newtheorem{con}[thm]{Conjecture}

\makeatletter
\def\namedlabel#1#2{\begingroup
    #2%
    \def\@currentlabel{#2}%
    \phantomsection\label{#1}\endgroup
}
\makeatother

\makeatletter
\newcommand{\rome}[1]{\romannumeral #1}
\newcommand{\Rome}[1]{\expandafter\@slowromancap\romannumeral #1@}
\makeatother

\newcommand{\be}{\begin{equation}}
\newcommand{\ee}{\end{equation}}

\newcommand{\esssup}{\operatornamewithlimits{ess\,sup}}
\newcommand{\essinf}{\operatornamewithlimits{ess\,inf}}
\DeclareMathOperator*{\argmin}{arg\,min}
\newcommand{\argmax}{\operatornamewithlimits{argmax}}

\DeclareMathOperator{\sgn}{sgn}
\def \law {\,{\buildrel d \over \rightarrow}\,}
\newcommand{\R}{\mathbb{R}}
\newcommand{\F}{\mathbb{F}}
\newcommand{\X}{\mathbb{X}}
\newcommand{\Q}{{\mathbb Q}}
\newcommand{\D}{\mathcal{D}}
\newcommand{\p}{\mathbb{P}}
\newcommand{\pc}{\mathcal{P}}
\newcommand{\w}{\mathcal{W}}
\newcommand{\g}{\mathbb{G}}
\newcommand{\f}{\mathcal{F}}
\newcommand{\E}{\mathbb{E}}
\newcommand{\M}{\mathcal{M}}
\newcommand{\G}{\mathcal{G}}
\newcommand{\h}{\mathcal{H}}
\newcommand{\C}{\mathfrak{C}}
 \newcommand{\Ss}{\mathcal{S}}

\newcommand{\A}{\mathfrak{A}}
\newcommand{\B}{\mathfrak{B}}
\newcommand{\Var}{\mathrm{Var}}
\definecolor{lgray}{gray}{0.70}
% \captionsetup{belowskip=7pt,aboveskip=3pt}


% \newcommand{\rpm}{\sbox0{$1$}\sbox2{$\scriptstyle\pm$}
  % \raise\dimexpr(\ht0-\ht2)/2\relax\box2 }

% \newcount\colveccount
% \newcommand*\colvec[1]{
%         \global\colveccount#1
%         \begin{pmatrix}
%         \colvecnext
% }
% \def\colvecnext#1{
%         #1
%         \global\advance\colveccount-1
%         \ifnum\colveccount>0
%                 \\
%                 \expandafter\colvecnext
%         \else
%                 \end{pmatrix}
%         \fi
% }
% \DeclareMathOperator{\Span}{span}
\newcommand\independent{\protect\mathpalette{\protect\independenT}{\perp}}
% \def\independenT#1#2{\mathrel{\rlap{$#1#2$}\mkern2mu{#1#2}}}
\usepackage{tikz}

\newcommand{\clg}{\color{lgray}}
\newcommand{\clr}{\color{red}}
\newcommand{\cln}{\color{blue}}

%%% Spacing
%%%
% \newcommand{\singlespace}{\renewcommand{\baselinestretch}{1.15} \small \normalsize}
% \newcommand{\oneandhalfspace}{\renewcommand{\baselinestretch}{1.3} \small \normalsize}
% \newcommand{\doublespace}{\renewcommand{\baselinestretch}{1.5} \small \normalsize}
% \newcommand{\normalspace}{\doublespace}
% \footnotesep=1\baselineskip

%%%
%%% Counters depth
%%%
% \setcounter{secnumdepth}{3}
% \setcounter{tocdepth}{3}

%%%
%%% Title page.
%%%
% \newcommand{\thesistitlepage}{
%     \normalspace
%     \thispagestyle{empty}
%     \begin{center}
%         \textbf{\LARGE \thesistitle} \\[1cm]
%         \textbf{\LARGE \thesisauthor} \\[8cm]
%         Submitted in partial fulfillment of the \\
%         requirements for the degree \\
%         of Doctor of Philosophy \\
%         in the Graduate School of Arts and Sciences \\[4cm]
%         \textbf{\Large COLUMBIA UNIVERSITY} \\[5mm]
%         \thesisyear
%     \end{center}
%     \clearpage
% }

%%%
%%% Copyright page.
%%%
% \newcommand{\thesiscopyrightpage}{
%     \thispagestyle{empty}
%     \strut \vfill
%     \begin{center}
%       \copyright \thesisyear \\
%       \thesisauthor \\
%       All Rights Reserved
%     \end{center}
%     \cleardoublepage
% }

%%%
%%% Abstract page.
%%%
% \newcommand{\thesisabstract}{
%     \thispagestyle{empty}
%     \begin{center}
%     \textbf{\LARGE ABSTRACT} \\[1cm]
%      \textbf{\LARGE \thesistitle} \\[1cm]
%      \textbf{\LARGE \thesisauthor} \\[1cm]
%     \end{center}
%       In this paper, we explore the connection between secret key agreement and secure omniscience within the setting of the multiterminal source model with a wiretapper who has side information. While the secret key agreement problem considers the generation of a maximum-rate secret key through public discussion, the secure omniscience problem is concerned with communication protocols for omniscience that minimize the rate of information leakage to the wiretapper. The starting point of our work is a lower bound on the minimum leakage rate for omniscience, $\rl$, in terms of the wiretap secret key capacity, $\wskc$. Our interest is in identifying broad classes of sources for which this lower bound is met with equality, in which case we say that there is a duality between secure omniscience and secret key agreement. We show that this duality holds in the case of certain finite linear source (FLS) models, such as two-terminal FLS models and pairwise independent network models on trees with a linear wiretapper. Duality also holds for any FLS model in which $\wskc$ is achieved by a perfect linear secret key agreement scheme. We conjecture that the duality in fact holds unconditionally for any FLS model. On the negative side, we give an example of a (non-FLS) source model for which duality does not hold if we limit ourselves to communication-for-omniscience protocols with at most two (interactive) communications.  We also address the secure function computation problem and explore the connection between the minimum leakage rate for computing a function and the wiretap secret key capacity.
  
%   Finally, we demonstrate the usefulness of our lower bound on $\rl$ by using it to derive equivalent conditions for the positivity of $\wskc$ in the multiterminal model. This extends a recent result of Gohari, G\"{u}nl\"{u} and Kramer (2020) obtained for the two-user setting.
  
   
%   In this paper, we study the problem of secret key generation through an omniscience achieving communication that minimizes the 
%   leakage rate to a wiretapper who has side information in the setting of multiterminal source model.  We explore this problem by deriving a lower bound on the wiretap secret key capacity $\wskc$ in terms of the minimum leakage rate for omniscience, $\rl$. 
%   %The former quantity is defined to be the maximum secret key rate achievable, and the latter one is defined as the minimum possible leakage rate about the source through an omniscience scheme to a wiretapper. 
%   The main focus of our work is the characterization of the sources for which the lower bound holds with equality \textemdash it is referred to as a duality between secure omniscience and wiretap secret key agreement. For general source models, we show that duality need not hold if we limit to the communication protocols with at most two (interactive) communications. In the case when there is no restriction on the number of communications, whether the duality holds or not is still unknown. However, we resolve this question affirmatively for two-user finite linear sources (FLS) and pairwise independent networks (PIN) defined on trees, a subclass of FLS. Moreover, for these sources, we give a single-letter expression for $\wskc$. Furthermore, in the direction of proving the conjecture that duality holds for all FLS, we show that if $\wskc$ is achieved by a \emph{perfect} secret key agreement scheme for FLS then the duality must hold. All these results mount up the evidence in favor of the conjecture on FLS. Moreover, we demonstrate the usefulness of our lower bound on $\wskc$ in terms of $\rl$ by deriving some equivalent conditions on the positivity of secret key capacity for multiterminal source model. Our result indeed extends the work of Gohari, G\"{u}nl\"{u} and Kramer in two-user case.
%     \cleardoublepage
% }

% %%%
% %%% Miscellaneous
% %%%
% \newcommand{\draft}{
%     \renewcommand{\normalspace}{\singlespace}
%     \normalspace
%     \chapter*{Draft. Version \today}
% \clearpage }


\providecommand{\norm}[1]{\left\lVert#1\right\rVert}

\usepackage[linesnumbered,ruled,vlined]{algorithm2e}

% \makeatletter
% \def\subsection{\@startsection{subsection}{2}%
%   \z@{.5\linespacing\@plus.7\linespacing}{-.5em}%
%   {\normalfont\bfseries}}
% \def\subsubsection{\@startsection{subsubsection}{3}%
%   \z@{.5\linespacing\@plus.7\linespacing}{-.5em}%
%   {\normalfont\bfseries}}
%   \makeatother
%\input{defD}
\def\mrl\mathrm{L}
\def\F{{\mathfrak{F}}}
\def\MK{{\rm Q}}
\def\MKK{{\rm K}}
\def\bb{\tilde{b}}
\def\PE{\mathrm{E}}
\def\msa{\mathsf{A}}
\def\Xset{\mathsf{X}}
\def\Xsigma{\mathcal{X}}
\def\Yset{\mathsf{Y}}
\def\Ysigma{\mathcal{Y}}

\def\maxind{\varkappa}
\def\distance{\mathsf{d}}
\newcommand{\ensemble}[2]{\left\{#1\,:\eqsp #2\right\}}
\newcommand{\set}[2]{\ensemble{#1}{#2}}
\newcommand\diagonal{\Delta} %%%% diagonale de l'espace produit $\Xset\times\Xset$; notation à  changer!!!
\DeclareMathAlphabet{\mathpzc}{OT1}{pzc}{m}{it}
\def\lyapW{\mathpzc{W}}
\def\cost{\metricc}
\def\rplus{\rset_+}
\def\covop{\mathcal{C}_{\msh}}

\def\vectpr{h}
\def\sigmaconst{\tilde{\sigma}}
\def\rootwas{\zeta}
\def\Xcoupling{\check{X}}
\def\complement{\mathrm{c}}
\newcommand{\1}{\mathbbm{1}}
\def\rset{\mathbb{R}}
\def\setProba{\mathcal{P}}
\def\probaSet{\mathcal{P}}
\def\PP{\mathrm{P}}
\def\zset{\mathbb{Z}}
\def\nset{\mathbb{N}}
\def\Constmainros{\mathsf{C}_0}
\def\Constroslog{\mathsf{C}_1}
\def\Constroslogmom{\mathsf{C}_2}
\def\Constwasspoly{\mathsf{C}_1}
\def\Constroslogwas{\mathsf{C}_2}
\newcommand{\absLigne}[1]{ \vert #1  \vert}
\def\eqsp{\;}
\newcommand{\Vnorm}[2][1=V]{\| #2 \|_{#1}}
\newcommand{\Nnorm}[2][1=V]{[ #2]_{#1}}
\def\mrl{\mathrm{L}}
\newcommand{\eqdef}{\ensuremath{:=}}
\newcommand{\eqspp}{\ \ }
\def\Gammabf{\mathbf{\Gamma}}
\newcommand{\momentGq}[1][1=q]{\mathrm{m}_{\mathrm{G},#1}}
\def\rme{\mathrm{e}}
\newcommandx{\PVar}[1][1=]{\ensuremath{\operatorname{Var}_{#1}}}
\newcommandx{\PCov}[1][1=]{\ensuremath{\operatorname{Cov}_{#1}}}
\def\metric{\mathsf{d}}
\def\metricc{\mathsf{c}}
\def\Lip{\operatorname{Lip}}
\def\gapindex{\mathsf{g}}
\def\maxgap{\operatorname{gap}}
\def\diam{\operatorname{diam}}
\newcommand{\card}[1]{\operatorname{card}(#1)}
\newcommand{\lr}[1]{\left( #1 \right)}
\newcommand{\lrb}[1]{\left[ #1 \right]}
\newcommand{\lrcb}[1]{\left\{ #1 \right\}}
\newcommand{\lrav}[1]{\left| #1 \right|}
\def\Lclass{\mathcal L}
\newcommand{\indi}[1]{\1_{#1}}
\newcommand{\indiacc}[1]{\1_{\{#1\}}}
\newcommand{\indin}[1]{\1\left\{#1\right\}}

\def\rmi{\mathrm{i}}
\newcommand{\expeLigne}[1]{\PE [ #1 ]}
\newcommand{\ps}[2]{\langle#1,#2 \rangle}
\def\bnu{\boldsymbol{\nu}}
\newcommand{\PEC}{\overline{\PE}}
\def\bk{{\boldsymbol k}}
\newcommand{\abs}[1]{\left\vert #1 \right\vert}
\newcommand{\ConstB}{\operatorname{B}}
\newcommand{\ConstC}{\operatorname{C}}
\newcommand{\ConstDd}{\operatorname{D}}
\newcommandx{\ConstD}[1][1={q,\lyapW}]{\operatorname{D}_{#1}}
\newcommandx{\ConstDW}[1][1={q,\lyapW}]{\mathfrak{D}_{#1}}
\newcommandx{\ConstJ}[1][1={n,W^\gamma}]{\operatorname{J}_{#1}}
\newcommandx{\ConstJW}[1][1={n,W^\gamma}]{\mathfrak{J}_{#1}}

\newcommandx{\arate}[1][1={q,\lyapW}]{{\alpha_{#1}}}
\newcommandx{\wrate}[1][1={q,\lyapW}]{{\beta_{#1}}}

\newcommand{\ConstK}{\operatorname{K}}
\newcommand{\ConstE}{\operatorname{E}}
\newcommand{\ConstL}{\operatorname{L}}
\newcommandx{\ConstG}[1][1={n,\lyapW}]{\operatorname{G}_{#1}}
\newcommandx{\ConstGW}[1][1={n,\lyapW}]{\mathfrak{G}_{#1}}

\newcommandx{\ConstM}[1][1={n,\lyapW}]{\operatorname{M}_{#1}}
\newcommandx{\ConstMW}[1][1={n,\lyapW}]{\mathfrak{M}_{#1}}

\newcommandx{\boldb}[1][1={q}]{\mathsf{b}_{#1}}
\newcommand{\ConstGgamma}{\bar{\operatorname{G}}}
\newcommand{\ConstMgamma}{\bar{\operatorname{M}}}
\newcommand{\CKset}{\bar{\mathsf{C}}}

\newcommand{\Const}[1]{\operatorname{C}_{{#1}}}
\def\Lclass{\mathcal L}
\newcommand{\myrho}[3]{\rho_{#3}}
\def\rmd{\mathrm{d}}
\newcommand{\tcr}[1]{\textcolor{blue}{#1}}
%\newcommand{\PCov}[1][1=]{\ensuremath{\operatorname{Cov}_{#1}}}
\def\ba{\boldsymbol{a}}
\newcommand{\pp}{\tilde{p}}
\newcommand{\tvnorm}[1]{\left\Vert #1 \right\Vert_{\mathrm{TV}}}
\newcommandx{\CPE}[3][1=]{\PE_{#1}\bigl[\bigl. #2 \, \bigr| #3 \bigr]}
\newcommand{\CPELine}[3][1=]{\PE_{#1}[ #2 \, | #3 ]}

\newcommand{\coint}[1]{\left[#1\right)}
\newcommand{\ocint}[1]{\left(#1\right]}
\newcommand{\ooint}[1]{\left(#1\right)}
\newcommand{\ccint}[1]{\left[#1\right]}
\renewcommand{\iint}[2]{\{#1,\ldots,#2\}}

\newcommandx\dsequence[4][3=,4=]{\ensuremath{\{ (#1_{#3}, #2_{#3})~:~\#3 \in #4 \}}}
\newcommandx\sequence[3][2=,3=]
{\ifthenelse{\equal{#3}{}}{\ensuremath{\{ #1_{#2}\}}}{\ensuremath{\{ #1_{#2}: #2 \in #3 \}}}}
\newcommandx\sequenceDouble[4][3=,4=]
{\ifthenelse{\equal{#4}{}}{\ensuremath{\{ (#1_{#3},#2_{#3})\}}}{\ensuremath{\{ (#1_{#3},#2_{#3})~:~#3 \in #4 \}}}}
\newcommand{\restric}[2]{\left(#1\right)_{#2}}
\newcommandx{\sequencen}[2][2=n\in\nset]{\ensuremath{\{ #1, \eqsp #2 \}}}

\def\constlemqnpi{\zeta}
\def\minorwas{\varepsilon}
\def\mcf{\mathcal{F}}
\def\mcg{\mathcal{G}}
\newcommand{\QQ}[1][]{\ifthenelse{\equal{#1}{}}{\ensuremath{\mathbf{Q}}}{\ensuremath{\mathbf{Q}\left( #1 \right)}}}
\newcommand{\QQbf}[1][]{\ifthenelse{\equal{#1}{}}{\ensuremath{\mathbf{Q}}}{\ensuremath{\mathbf{Q}\left( #1 \right)}}}
\def\shift{\theta}
\newcommandx{\PPcoupling}[1][1={\QQ,\QQ'}]{\mathbb{P}_{#1}}
\newcommandx{\PEcoupling}[1][1={\QQ,\QQ'}]{\mathbb{E}_{#1}}
\newcommandx{\as}[1][1=\PP]{\ensuremath{#1\, -\mathrm{a.s.}}}
\newcommand{\x}{\ensuremath{x}}
\newcommandx{\rate}[1][1={q,\lyapW}]{\rho_{#1}}
\def\ratev{\rho}
\def\ratewas{\varrho}
\def\cmconstv{c}
\def\deltawas{\delta_*}
\def\vartconstwas{c_{\MKK}}
\def\ie{i.e.}
\def\boundmetric{\kappa_{\MKK}}
\def\setfunction{\mathsf{F}}
\def\normlike{\Psi}
%\newcommandx{\arate}[1][1={q,\lyapW}]{\alpha_{#1}}

\newtheorem{theorem}{Theorem}
\crefname{theorem}{theorem}{Theorems}
\Crefname{Theorem}{Theorem}{Theorems}

\def\iid{i.i.d.}

%\newaliascnt{lemma}{theorem}
\newtheorem{lemma}{Lemma}
%\aliascntresetthe{lemma}
\crefname{lemma}{lemma}{lemmas}
\Crefname{Lemma}{Lemma}{Lemmas}

%\newaliascnt{corollary}{theorem}
\newtheorem{corollary}{Corollary}
%\aliascntresetthe{corollary}
\crefname{corollary}{corollary}{corollaries}
\Crefname{Corollary}{Corollary}{Corollaries}

%\newaliascnt{proposition}{theorem}
\newtheorem{proposition}{Proposition}
%\aliascntresetthe{proposition}
\crefname{proposition}{proposition}{propositions}
\Crefname{Proposition}{Proposition}{Propositions}

\newaliascnt{definition}{theorem}
\newtheorem{definition}[definition]{Definition}
\aliascntresetthe{definition}
\crefname{definition}{definition}{definitions}
\Crefname{Definition}{Definition}{Definitions}


\newaliascnt{definition-proposition}{theorem}
\newtheorem{definitionProp}[definition-proposition]{Definition-Proposition}
\aliascntresetthe{definition-proposition}
\crefname{definition-proposition}{definition-proposition}{definitions-propositions}
\Crefname{Definition-Proposition}{Definition-Proposition}{Definitions-Propositions}

%\newaliascnt{remark}{theorem}
\newtheorem{remark}{Remark}
%\aliascntresetthe{remark}
\crefname{remark}{remark}{remarks}
\Crefname{Remark}{Remark}{Remarks}


%\newtheorem{example}[theorem]{Example}
\crefname{example}{example}{examples}
\Crefname{Example}{Example}{Examples}


\crefname{figure}{figure}{figures}
\Crefname{Figure}{Figure}{Figures}

\crefname{table}{table}{tables}
\Crefname{Table}{Table}{Tables}


\newtheorem{assumptionH}{\textbf{H}\hspace{-1pt}}
\Crefname{assumption}{\textbf{H}\hspace{-1pt}}{\textbf{H}\hspace{-1pt}}
\crefname{assumption}{\textbf{H}}{\textbf{H}}

\newtheorem{assumptionSGD}{\textbf{A-SGD}\hspace{-1pt}}
\Crefname{assumptionSGD}{\textbf{A-SGD}\hspace{-1pt}}{\textbf{A-SGD}\hspace{-1pt}}
\crefname{assumptionSGD}{\textbf{A-SGD}}{\textbf{A-SGD}}

\newtheorem{assumptionpCN}{\textbf{A-pCN}\hspace{-1pt}}
\Crefname{assumptionpCN}{\textbf{A-pCN}\hspace{-1pt}}{\textbf{A-pCN}\hspace{-1pt}}
\crefname{assumptionpCN}{\textbf{A-pCN}}{\textbf{A-pCN}}


\newtheorem{assumptionC}{\textbf{C}\hspace{-1pt}}
\Crefname{assumptionC}{\textbf{C}\hspace{-1pt}}{\textbf{C}\hspace{-1pt}}
\crefname{assumptionC}{\textbf{C}}{\textbf{C}}


\newtheorem{assumption}{\textbf{A}\hspace{-2pt}}
\Crefname{assumption}{\textbf{A}\hspace{-2pt}}{\textbf{A}\hspace{-2pt}}
\crefname{assumption}{\textbf{A}\hspace{-2pt}}{\textbf{A}\hspace{-2pt}}

\newtheorem{assumptionW}{\textbf{W}\hspace{-3pt}}
\Crefname{assumptionW}{\textbf{W}\hspace{-3pt}}{\textbf{W}\hspace{-3pt}}
\crefname{assumptionW}{\textbf{W}}{\textbf{W}}


\newtheorem{probleme}{\textbf{Problem}}
\Crefname{probleme}{\textbf{Problem}\hspace{-3pt}}{\textbf{Problem}\hspace{-3pt}}
\crefname{probleme}{\textbf{Problem}}{\textbf{Problem}}


\newtheorem{assumptionG}{\textbf{G}\hspace{-4pt}}
\Crefname{assumptionG}{\textbf{G}\hspace{-4pt}}{\textbf{G}\hspace{-4pt}}
\crefname{assumptionG}{\textbf{G}}{\textbf{G}}
\def\couplingmeasure{\mathcal{C}}
\newcommandx{\wassersym}[1][1=\distance]{\mathbf{W}_{#1}}
\newcommandx{\wasser}[4][1=\distance,4=]{\mathbf{W}_{#1}^{#4}\left(#2,#3\right)}

\def\plusinfty{+\infty}
\newcommand{\txts}[1]{\textstyle #1}
\def\mff{\mathfrak{F}}
\def\scrE{\mathscr{E}}
\def\nsets{\nset^*}
\def\scrA{\mathscr{A}}
\newcommand{\parenthese}[1]{\left( #1\right)}
\newcommand{\defEns}[1]{\left\{ #1\right\}}
\newcommand{\parentheseDeux}[1]{\left[ #1\right]}
\def\Sigmabf{\boldsymbol{\Sigma}}
\def\pcost{p_{\cost}}
\def\eg{e.g.}
\def\msh{\mathsf{H}}
\def\msk{\mathsf{K}}
\def\msu{\mathsf{U}}
\def\mch{\mathcal{H}}


\def\potU{\Phi_{\msh}}
\def\alphaH{\alpha_{\msh}}
\def\rhoH{\rho_{\msh}}
\def\muH{\mu_{\msh}}


\newcommandx{\normH}[2][1=]{\ifthenelse{\equal{#1}{}}{\left\Vert #2 \right\Vert_{\msh}}{\left\Vert #2 \right\Vert^{#1}_{\msh}}}
\newcommandx{\normHLigne}[2][1=]{\ifthenelse{\equal{#1}{}}{\Vert #2 \Vert_{\msh}}{\Vert #2\Vert^{#1}_{\msh}}}
\newcommandx{\normHLine}[2][1=]{\ifthenelse{\equal{#1}{}}{\Vert #2 \Vert_{\msh}}{\Vert #2\Vert^{#1}_{\msh}}}

\newcommandx{\norm}[2][1=]{\ifthenelse{\equal{#1}{}}{\left\Vert #2 \right\Vert}{\left\Vert #2 \right\Vert^{#1}}}
\newcommandx{\normLigne}[2][1=]{\ifthenelse{\equal{#1}{}}{\Vert #2 \Vert}{\Vert #2\Vert^{#1}}}
\newcommandx{\normLine}[2][1=]{\ifthenelse{\equal{#1}{}}{\Vert #2 \Vert}{\Vert #2\Vert^{#1}}}

\newcommand{\ball}[2]{\operatorname{B}(#1,#2)}
\newcommand{\ballH}[2]{\operatorname{B}_{\msh}(#1,#2)}

\def\varepsilonH{\varepsilon_{\msh}}
\def\vH{\upsilon_{\msh}}
\def\upsilonH{\upsilon_{\msh}}
\def\objf{\mathrm{f}}

\def\Rsgd{R}
\def\Lip{\operatorname{Lip}}
\def\lbprobpcn{p_1}
\def\Lippcn{\mathsf{L}}
\def\pcngamma{\gamma}
\def\Filtr{\mathcal{F}}
\def\RandSpace{\Omega}
\def\Rassumapcn{R}
\def\tstar{t^{\star}}
\def\constdriftpcnfirst{\mathsf{b}_1}
\def\constdriftpcnsecond{\mathsf{b}_2}
\def\constKone{K_1}
\def\Constexpmoment{\mathsf{D}}
\def\smallconst{\zeta}
\def\prop{z}
\def\eventA{A}
\def\eventB{B}
\def\eventC{C}
\def\gausc{c_\tau}
\def\betagaus{\beta}
\def\alphagaus{\alpha_{\tau}}
\def\fieldH{H}
\def\muY{\mu_{\Yset}}
\def\YSGD{\mathrm{Y}}
\def\MKSGD{\MK}
\def\MKKSGD{\MKK}
\def\Ccoco{C_S}
\def\thetas{\theta^{\star}}

\def\constprfirst{c_1}
\def\constprsecond{c_2}
\def\Lf{L_{\objf}}
\def\muf{\mu_{\objf}}
\def\kapf{\kappa_{\objf}}
\def\sgvarfac{\sigma_{\objf}^2}
\def\tsgvarfac{\tilde{\sigma}_{\objf}^2}
\def\tsgstdfac{\tilde{\sigma}_{\objf}}
\def\pCN{\mathrm{pCN}}
\def\RpCN{\mathsf{R}}
\def\alphalpCN{\bar{\alpha}_{\msh}}
\def\rpCN{r}
\def\rpCNconst{\bar{r}}

\def\LipHessianf{L_{\nabla \objf}}
\def\msx{\mathsf{X}}
\def\dims{\mathrm{d}}
\title{On the convergence of Hamiltonian Monte Carlo}


\author[1]{Alain Durmus \footnote{Email: alain.durmus@cmla.ens-cachan.fr} }
\author[2]{\'Eric Moulines \footnote{Email: eric.moulines@polytechnique.edu} }
\author[3]{Eero Saksman \footnote{Email: eero.saksman@helsinki.fi} }


\affil[1]{CMLA - \'Ecole normale supérieure Paris-Saclay, CNRS, Université Paris-Saclay, 94235 Cachan, France.}
\affil[2]{Centre de Math\'ematiques Appliqu\'ees,\\ UMR 7641, Ecole Polytechnique}
\affil[3]{University of Helsinki, Department of Mathematics and Statistics}


\begin{document}

\maketitle

\begin{abstract}
This paper discusses the irreducibility and geometric ergodicity of the Hamiltonian Monte Carlo (HMC) algorithm.
We consider cases where the number of steps of the symplectic integrator is either  fixed or random. Under mild conditions on the potential $\F$ associated with target distribution $\pi$, we first show that the Markov kernel associated to the HMC algorithm is irreducible and recurrent.
Under more stringent conditions, we then establish that the Markov kernel is Harris recurrent. Finally, we provide verifiable  conditions on $\F$  under which the HMC sampler is geometrically ergodic. We compare our assumptions with those recently presented in  \cite{livingstone:betancourt:byrne:girolami:2016} and \cite{bou:sanz:2017}.
\end{abstract}


\section{Introduction }
% \leavevmode
% \\
% \\
% \\
% \\
% \\
\section{Introduction}
\label{introduction}

AutoML is the process by which machine learning models are built automatically for a new dataset. Given a dataset, AutoML systems perform a search over valid data transformations and learners, along with hyper-parameter optimization for each learner~\cite{VolcanoML}. Choosing the transformations and learners over which to search is our focus.
A significant number of systems mine from prior runs of pipelines over a set of datasets to choose transformers and learners that are effective with different types of datasets (e.g. \cite{NEURIPS2018_b59a51a3}, \cite{10.14778/3415478.3415542}, \cite{autosklearn}). Thus, they build a database by actually running different pipelines with a diverse set of datasets to estimate the accuracy of potential pipelines. Hence, they can be used to effectively reduce the search space. A new dataset, based on a set of features (meta-features) is then matched to this database to find the most plausible candidates for both learner selection and hyper-parameter tuning. This process of choosing starting points in the search space is called meta-learning for the cold start problem.  

Other meta-learning approaches include mining existing data science code and their associated datasets to learn from human expertise. The AL~\cite{al} system mined existing Kaggle notebooks using dynamic analysis, i.e., actually running the scripts, and showed that such a system has promise.  However, this meta-learning approach does not scale because it is onerous to execute a large number of pipeline scripts on datasets, preprocessing datasets is never trivial, and older scripts cease to run at all as software evolves. It is not surprising that AL therefore performed dynamic analysis on just nine datasets.

Our system, {\sysname}, provides a scalable meta-learning approach to leverage human expertise, using static analysis to mine pipelines from large repositories of scripts. Static analysis has the advantage of scaling to thousands or millions of scripts \cite{graph4code} easily, but lacks the performance data gathered by dynamic analysis. The {\sysname} meta-learning approach guides the learning process by a scalable dataset similarity search, based on dataset embeddings, to find the most similar datasets and the semantics of ML pipelines applied on them.  Many existing systems, such as Auto-Sklearn \cite{autosklearn} and AL \cite{al}, compute a set of meta-features for each dataset. We developed a deep neural network model to generate embeddings at the granularity of a dataset, e.g., a table or CSV file, to capture similarity at the level of an entire dataset rather than relying on a set of meta-features.
 
Because we use static analysis to capture the semantics of the meta-learning process, we have no mechanism to choose the \textbf{best} pipeline from many seen pipelines, unlike the dynamic execution case where one can rely on runtime to choose the best performing pipeline.  Observing that pipelines are basically workflow graphs, we use graph generator neural models to succinctly capture the statically-observed pipelines for a single dataset. In {\sysname}, we formulate learner selection as a graph generation problem to predict optimized pipelines based on pipelines seen in actual notebooks.

%. This formulation enables {\sysname} for effective pruning of the AutoML search space to predict optimized pipelines based on pipelines seen in actual notebooks.}
%We note that increasingly, state-of-the-art performance in AutoML systems is being generated by more complex pipelines such as Directed Acyclic Graphs (DAGs) \cite{piper} rather than the linear pipelines used in earlier systems.  
 
{\sysname} does learner and transformation selection, and hence is a component of an AutoML systems. To evaluate this component, we integrated it into two existing AutoML systems, FLAML \cite{flaml} and Auto-Sklearn \cite{autosklearn}.  
% We evaluate each system with and without {\sysname}.  
We chose FLAML because it does not yet have any meta-learning component for the cold start problem and instead allows user selection of learners and transformers. The authors of FLAML explicitly pointed to the fact that FLAML might benefit from a meta-learning component and pointed to it as a possibility for future work. For FLAML, if mining historical pipelines provides an advantage, we should improve its performance. We also picked Auto-Sklearn as it does have a learner selection component based on meta-features, as described earlier~\cite{autosklearn2}. For Auto-Sklearn, we should at least match performance if our static mining of pipelines can match their extensive database. For context, we also compared {\sysname} with the recent VolcanoML~\cite{VolcanoML}, which provides an efficient decomposition and execution strategy for the AutoML search space. In contrast, {\sysname} prunes the search space using our meta-learning model to perform hyperparameter optimization only for the most promising candidates. 

The contributions of this paper are the following:
\begin{itemize}
    \item Section ~\ref{sec:mining} defines a scalable meta-learning approach based on representation learning of mined ML pipeline semantics and datasets for over 100 datasets and ~11K Python scripts.  
    \newline
    \item Sections~\ref{sec:kgpipGen} formulates AutoML pipeline generation as a graph generation problem. {\sysname} predicts efficiently an optimized ML pipeline for an unseen dataset based on our meta-learning model.  To the best of our knowledge, {\sysname} is the first approach to formulate  AutoML pipeline generation in such a way.
    \newline
    \item Section~\ref{sec:eval} presents a comprehensive evaluation using a large collection of 121 datasets from major AutoML benchmarks and Kaggle. Our experimental results show that {\sysname} outperforms all existing AutoML systems and achieves state-of-the-art results on the majority of these datasets. {\sysname} significantly improves the performance of both FLAML and Auto-Sklearn in classification and regression tasks. We also outperformed AL in 75 out of 77 datasets and VolcanoML in 75  out of 121 datasets, including 44 datasets used only by VolcanoML~\cite{VolcanoML}.  On average, {\sysname} achieves scores that are statistically better than the means of all other systems. 
\end{itemize}


%This approach does not need to apply cleaning or transformation methods to handle different variances among datasets. Moreover, we do not need to deal with complex analysis, such as dynamic code analysis. Thus, our approach proved to be scalable, as discussed in Sections~\ref{sec:mining}.


\subsection*{Notations}
\section{Notation}
\label{sec:notation}

Let bold functions $\mathbf{f}$ represent the row-wise vectorized versions of their scalar counterparts $f$: $\mathbf{f}(\mathbf{x}) = [f(x_1), f(x_2) \dots f(x_n)]^T$. 

Let $I_a(z)$ be the indicator function for $z=a$.

Let $P(Z^*)$ be a distribution that places probability $\frac{1}{n}$ on each element of  a set of samples $\{z_1, z_2 \dots z_n\}$. Let $z^*$ denote a sample from $Z^*$ and $\bm{z}^*$ denote a set of $n$ such samples (a bootstrap sample).

\subsection{Notation for Observational data}

Observational datasets are represented by a set of $n$ tuples of observations $(x_i,w_i,y_i)$. For each subject $i$, $x_i \in \mathcal{R}^{p}$ is a vector of observed pre-treatment covariates, $w_i \in \{0,1\}^n$ is a binary indicator of treatment status, and $y_i \in \mathcal{R}^n$ is a real-valued outcome. Let capital letters $X$, $W$, and $Y$ represent the corresponding random variables. Let $\mathcal{S}_0$ be the set of indices of untreated subjects, and $\mathcal{S}_1$ the set of indices of treated subjects: $\mathcal{S}_w = \{i | w_i = w\}$.  Denote a series of I.I.D. realizations of a random variable $z \sim P(Z)$ as $\bm{z}$ so that the full observational dataset can be written as $d = (\mathbf{x}, \mathbf{w}, \mathbf{y})$ where $(x_i,w_i,y_i) \overset{\text{I.I.D.}}{\sim} P(X,W,Y)$. 

The average treatment effect is defined as 

\begin{equation}
\tau = E_{X,Y}[Y|X,W=1] - E_{X,Y}[Y|X,W=0]
\label{eq:effect}
\end{equation}

which is the expected difference between what the outcome would have been had a subject received the treatment and what the outcome would have been had a subject not received the treatment, averaged over all subjects in the population. %The outcomes under each of these conditions are referred to as potential outcomes. Let those be denoted as $f_w(x,u) = f(x,u,w) = E[Y_w | X, U]$. %Note that $f_w(\mathbf{x}, \mathbf{u}) + \bm{\eta} \in \mathcal{R}^n$, but $\mathbf{y}_w \in \mathcal{R}^{n_w}$. The latter does not include potential outcomes under treatment $w$ that were not observed.

% \section{Description of the Hamiltonian Monte Carlo algorithm}
% \label{sec:descr-hamilt-monte}
% 
%However, it is important to note
%that the invariance of π for this kernel is not a sufficient condition
%for the convergence of algorithm.
% However the invariance of
% $\pi$ for this kernel is not a sufficient condition for its convergence. 
% \alain{put a sentence on the fact that most of HMC versions, the
%   invariance is checked and it is not enough for the convergence of
%   the algorithm} 





%%% Local Variables:
%%% mode: latex
%%% TeX-master: "main"
%%% End:


\section{Ergodicity of the HMC algorithm}
\label{sec:ergodicity-hmc}


For $h >0$ and $T \in \nset^*$, consider the Markov kernel $\Pkerhmc[h][T]$ associated with the Markov chain of the HMC algorithm $(Q_k)_{k \in \nset}$, given for all $\q \in \rset^d$ and $\eventA \in \borelSet(\rset^d)$ by
\begin{align}
  \Pkerhmc[h][T](\q, \eventA) &= \int_{\rset^d} \indi{\eventA}{\Phiverletq[h][T](\q,\tilde{\p})} \ \alphaacc\defEns{(\q,\tilde{\p}),\Phiverlet[h][T](\q,\tilde{\p})}\frac{\rme^{-\norm[2]{\tilde{\p}}/2}}{ (2 \uppi)^{d/2}}  \rmd \tilde{\p}
                                  \nonumber
  \\
&\qquad + \updelta_{\q}(\eventA)  \,   \int_{\rset^d}  \parentheseDeux{1-\alphaacc\defEns{(\q,\tilde{\p}),\Phiverlet[h][T](\q,\tilde{\p})}} \frac{\rme^{-\norm[2]{\tilde{\p}}/2}}{ (2 \uppi)^{d/2}}  \rmd \tilde{\p} \eqsp,
 \label{eq:def_kernel_hmc}
\end{align}
where $\Phiverletq[h][T]$, $\Phiverlet[h][T]$ and $\alphaacc$ are defined by \eqref{eq:def_Phiverlet}-\eqref{eq:def_Phiverletq} and \eqref{eq:def_acc_ratio} respectively.
In this Section, we establish conditions upon which the Markov kernel $ \Pkerhmc[h][T]$ is irreducible or
(Harris) recurrent. % Not surprisingly these conditions imply regularity
%conditions and control of the tails of the target distribution $\pi$.  % Our results are established for
% the marginal chain for which $\pi$ is invariant, but irreducibility
% for the Markov kernel associated with the position and momentum is
% crucial for some variants of HMC, in particular when the process
% associated with the position is no longer Markov, (see
% \cite[Proposition 3.7]{bou:sanz:2017}).
For
$\expozero \in \ccint{0,1}$, we consider the following assumption on
the potential $\F$.

\begin{assumption}[$\expozero$]
  \label{assum:regOne}
  $\F$ is continuously differentiable and
  \begin{enumerate}[label=(\roman*)]
  \item
  \label{assum:regOne_a}
 there exists $\constzero > 0$  such that for all $\q,x \in \rset^d$,
\begin{equation}
\norm{\nabla \F(\q) - \nabla \F(x)} \leq \constzero\norm{\q-x} \eqsp.
  \end{equation}
\item    \label{assum:regOne_b}
there exists $\constzeroT \geq 0$  such that for all $\q \in \rset^d$,
\begin{equation}
%\label{eq:bound_nabla_F_assum_reg_zero}
  \norm{\nabla \F(\q)} \leq \constzeroT\defEns{ 1 + \norm{\q}^{\expozero}} \eqsp.
\end{equation}
  \end{enumerate}
\end{assumption}


% For all $T \in \nset^*$, define $\gpertub[h][T] : \rset^d \times \rset^d \to \rset^d$ for all $(\q,\p) \in \rset^d \times \rset^d$ by
% \begin{equation}
%   \gpertub[h][T](\q,\p) =   \Phiverletq[h][k](\q,\p) - \q \eqsp.
% \end{equation}
% \Cref{lem:bound_first_iterate_leapfrog} shows that there exists $C
% \geq 0$ such that for all $\q \in \rset^d$,
% We first state our main two results regarding the ergodicity of the Markov kernel
% $\Pkerhmc[h][T]$ for $h \in \rset_+^*$ and $T \in \nset$.
Before going further, we need to briefly recall some definitions pertaining to Markov chains.
Let $\Pker$ be a Markov kernel on $(\rset^d,\borelSet(\rset^d))$. Let $n$ be an integer and $\mu$
be a nontrivial measure on $\borelSet(\rset^d)$. A
set $\Csf \in \borelSet(\rset^d)$ is called a $(n,\mu)$-small set for $\Pker$ if
for all $x \in \Csf$ and $\Asf \in \borelSet(\rset^d)$, $\Pker^n(x, \msa) \geq \mu(\msa)$.
A set $\Asf \in \borelSet(\rset^d)$ is said to be accessible for $\Pker$
  if for all $x \in \rset^d$, $\sum_{i=1}^\infty \Pker^i(x,\Asf) > 0$.
  A non-trivial $\sigma$-finite measure $\mu$ is an irreducibility
  measure of $\Pker$ \iff\ any set $\Asf \in \borelSet(\rset^d)$
  satisfying $\mu(\Asf) >0$ is accessible.  The Markov kernel $\Pker$ is said to be
  irreducible if it admits an accessible small set or equivalently an
  irreducibility measure (in \cite{meyn:tweedie:2009}, our notion of irreducibility  is referred to as $\phi$-irreducibility, where $\phi$ is an irreducibility measure; here irreducibility therefore means $\phi$-irreducibility). $\Pker$ is said to be a \Tkernel~is there exists a kernel $\Tker$ on $\rset^d \times \mcb(\rset^d)$ and a sequence of non-negative numbers $(a_i)_{i \in \nsets}$ satisfying $\sum_{i=1}^{\plusinfty} a_i =1$, such that
  \begin{enumerate*}[label=(\roman*)]
  \item for any $x \in \rset^d$, $\Tker(x, \rset^d) >0$;
  \item for any $\msa \in \mcb(\rset^d)$, $x \mapsto \Tker(x,\msa)$ is lower semi-continuous;
\item for any $x \in \rset^d$, $\msa \in \mcb(\rset^d)$, $\sum_{i=1}^{\plusinfty} a_i \Pker^i(x,\msa) \geq \Tker(x,\msa)$.
  \end{enumerate*}
  $\Tker$ is referred to as a continuous component of $\Pker$.


  Let $(X_n)_{n \geq 0}$ be the canonical chain associated with $\Pker$
  defined on the canonical space $(\Omega,\mathcal{F},(\mathbb{P}_x, x \in \rset^d))$. A
  set $\Asf \in \borelSet(\rset^d)$ is said to be recurrent if for all $x \in \msa$, $\PE_x[N_\msa]= \plusinfty$ where $N_\msa = \sum_{i=0}^{\plusinfty} \1_{\msa}(X_i)$ is the number of visits to $\msa$. The set $\msa$ is Harris recurrent  if for any $x \in \msa$, $\mathbb{P}_x(N_\msa = \plusinfty) = 1$. The Markov kernel $\Pker$ is said to be Harris
  recurrent if all accessible sets are Harris recurrent. In this case, for all $x \in \rset^d$, and all accessible sets $\msa$, $\PP_x(N_\msa = \plusinfty)=1$.

  Define $\vartheta_1 : \rset_+ \to \rset_+$, for any $s \in \rset_+$ by
  \begin{equation}
    \label{eq:def_vartheta_1}
    \vartheta_1(s) = 1+  s/2 + s^2/4\eqsp.
  \end{equation}
\begin{theorem}
  \label{theo:irred_harris}
Assume \Cref{assum:regOne}($\expozero$) for some $\expozero \in \ccint{0,1}$ and that $\F$ is twice continuously
differentiable. Then, for all $T \in \nsets$, and  $h > 0$ satisfying
\begin{equation}
\label{eq:condition-h,T-harris}
 \left[ \{1 + h\constzero^{1/2} \vartheta_1(h\constzero^{1/2}) \}^T - 1 \right] < 1 \eqsp,
\end{equation}
and $q \in \rset^d$, there exists a $C^1(\rset^d,\rset^d)$-diffeomorphism $\tilde{q} \mapsto \Phiverletqi[h][T](q,\tilde{q})$ such that for any $p \in \rset^d$,
\begin{equation}
\label{theo:irred_harris_a}
\text{if $q_T =   \Phiverletq[h][T](q,p)$, defined by \eqref{eq:def_Phiverletq}, then $p = \Phiverletqi[h][T](q,q_T)$} \eqsp.
\end{equation}
Moreover,
\begin{enumerate}[label=(\roman*), wide, labelwidth=!, labelindent=0pt]
\item   \label{theo:irred_harris_b}
The Markov kernel $\Pkerhmc[h][T]$, is a \Tkernel; more precisely, for any $\eventB \in \mcb(\rset^d)$,
\begin{align}
\label{eq:def_kernel_hmc_false_density}
&\Pkerhmc[h][T](q, \eventB) =  \Tker_{h,T}(q,\eventB) \\
&\qquad + \updelta_{q}(\eventB)(2 \uppi)^{-d/2} \int_{\rset^d}  \parentheseDeux{1-\alphaacc\defEns{(q,\tilde{p}),\Phiverlet[h][T](q,\tilde{p})}} \rme^{-\norm{\tilde{p}}^2/2} \rmd \tilde{p} \eqsp,
\end{align}
where the kernel $ \Tker_{h,T}$ is a continuous component of $\Pkerhmc[h][T]$  and is given by
\begin{equation}
  \label{eq:def_tker}
\Tker_{h,T}(q,\eventB)
  =   (2 \uppi)^{-d/2} \int_{\eventB}    \balphaacc(q,\bar{q})\rme^{-\norm{\Phiverletqi[h][T](q,\bar{q})}^2/2} \detj_{\Phiverletqi[h][T](q,\cdot)}(\bar{q})  \rmd \bar{q} \eqsp,
\end{equation}
setting for $q,\tilde{q} \in \rset^d$, $\balphaacc(q,\bar{q}) =  \alphaacc\defEns{(q,\Phiverletqi[h][T](q,\bar{q})),\Phiverlet[h][T](q,\Phiverletqi[h][T](q,\bar{q}))}$ and  $\detj_{\Phiverletqi[h][T](q,\cdot)}(\tilde{q}) = \absLigne{\det(\Jac_{\Phiverletqi[h][T](q,\cdot)}(\tilde{q}))}$.
\item \label{theo:irred_harris_c} The Markov kernel $\Pkerhmc[h][T]$ is irreducible and the Lebesgue measure is an irreducibility measure. Moreover,  $\Pkerhmc[h][T]$ is aperiodic, Harris recurrent and all the compact sets are $1$-small. Therefore, for all $\q \in \rset^d$,
\begin{equation}
\label{eq:harris-theorem}
\lim_{n \to \plusinfty}    \tvnorm{\delta_\q \Pkerhmc[h][T]^n - \pi} = 0 \eqsp.
\end{equation}
\end{enumerate}
\end{theorem}

\begin{proof}
The proof is postponed to \Cref{sec:proof-crefth-harris_0}.
\end{proof}
%\erici{à vérifier, mais ça doit être bon}
For all $h > 0$ and $T \in \nsets$, we have
$\{1 + h \constzero^{1/2} \vartheta_1(h \constzero^{1/2} ) \}^T -1 \leq \rme^{h \constzero^{1/2} T \vartheta_1(h \constzero^{1/2} T)} -1$.
using that  $\vartheta_1$ is nondecreasing.
Then, setting $\bar{S} = c \constzero^{-1/2}$ where $c$ is   the unique positive root  of the equation
$c \vartheta_1(c)  = \log(2)$,  all $T \in \nsets$ and $h  \in \ooint{0,\bar{S}/T}$ satisfy \eqref{eq:condition-h,T-harris}\footnote{Note that conversely, if $h >0$ and $T \in \nsets$ satisfies \eqref{eq:condition-h,T-harris}, necessarily $h \in \oointLigne{0,\constzero^{-1/2}}$ because for any $s > 0$, $\vartheta_1(s) \geq 1$. In addition, since $\rme^{\log(2) s} \leq (1+s)$ for all $s \in \oointLigne{0,1}$, $T$ and $h$ satisfy $hT \leq \tilde{S}= \constzero^{-1/2}$.
}.


% Under \Cref{assum:regOne}, for all $h \in \rset_+^*$, $T \in \nset^*$
% and $x \in \rset^d$, $\Pkerhmc[h][T](x , \{x \}) >0$, which ensures
% that $\Pkerhmc[h][T]$ is aperiodic.
% Note that by \cite[Theorem 13.0.1]{meyn:tweedie:2009}, \Cref{theo:irred_harris} implies
% that for all $T \geq 0$, there exists $\hirr>0$ such that for all $h \in \ocintLigne{0,\hirr}$ and all $\q \in \rset^d$
%   \begin{equation}
% \lim_{n \to \plusinfty}    \tvnorm{\delta_\q \Pkerhmc[h][T]^n - \pi} = 0 \eqsp.
%   \end{equation}




%%% Local Variables:
%%% mode: latex
%%% TeX-master: "main"
%%% End:

In our next result, we relax the second order differentiability
condition on $\F$, and in the case $\beta <1$ we even allow for
arbitrary large values of the step size $h$ and the number of iterations $T$.
The result is less quantitative and the proof
is more involved: we use degree theory for continuous
mapping (the main notions  required in the proof are recalled in  \Cref{sec:defin-usef-results}).
\begin{theorem}\label{theo:irred_D}
Let $h > 0$ and $T \in \nsets$ and assume either
\begin{enumerate}[label=(\alph*)]
\item
\label{theo:irred_D_a}
\Cref{assum:regOne} $(\expozero)$ for some $\expozero \in \coint{0,1}$,
%   \begin{equation}
% \lim_{n \to \plusinfty}    \tvnorm{\delta_q \Pkerhmc[h][T]^n - \pi} = 0 \eqsp.
%  \end{equation}`
\item
\label{theo:irred_D_b}
\Cref{assum:regOne} $(1)$ and that  $T \in \nsets$ and $h > 0$ satisfy \eqref{eq:condition-h,T-harris}.
\end{enumerate}
%Moreover, $\Pkerhmc[h][T]$ is .
Then,
\begin{enumerate}[label=(\roman*)]
\item the HMC kernel $\Pkerhmc[h][T]$ defined by \eqref{eq:def_kernel_hmc} is irreducible, aperiodic, the Lebesgue measure is an
  irreducibility measure and any compact set of $\rset^d$ is  small.
\item $\Pkerhmc[h][T]$ is recurrent and for $\pi$-almost every $q \in \rset^d$,
$\lim_{n \to \plusinfty}    \tvnorm{\delta_q \Pkerhmc[h][T]^n - \pi} = 0$.
\end{enumerate}
\end{theorem}

\begin{proof}
The proof is postponed to \Cref{sec:proof-crefth_irred_D}.
\end{proof}


% \alain{put a sentence on the fact that most of HMC versions, the
%   invariance is checked and it is not enough for the convergence of
%   the algorithm}
To the best of the author's knowledge, the first results regarding the
irreducibility of the HMC algorithm are established in
\cite{cances:legoll:stoltz} under the assumption that $\U$ and
$\norm{\nabla \U}$ are bounded above. Note that these assumptions are
in general satisfied only for compact state space. Irreducibility has
also been tackled in
\cite{livingstone:betancourt:byrne:girolami:2016}: in this work
however, the number of leapfrog steps $T$ is assumed to be random and
independent of the current position and momentum. Under this setting
and additional conditions which in particular imply that the number of
leapfrog steps $T$ is equal to $1$ with positive probability,
\cite{livingstone:betancourt:byrne:girolami:2016} shows that the
kernel associated with the HMC algorithm is irreducible. Under this condition,
the proof is  a direct
consequence of the irreducibility of the MALA algorithm - a mixture of
Markov kernels is irreducible as soon as one component of the mixture
is irreducible; the irreducibility of MALA kernel has been established
in \cite{roberts:tweedie:1996}). Finally, \cite[Proposition
3.7]{bou:sanz:2017} shows that RHMC is irreducible under the condition
that $U$ is at least quadratic.  Note that \Cref{theo:irred_D}
establishes irreducibility of HMC of sub-quadratic potential. However,
leap-frog integrator is not numerically stable for lighter than
Gaussian target density, therefore other kind of integrators should be
used instead, see \eg~\cite[Chapter VI]{hairer:wanner:lubish:2002}. %One possibility would be  to consider a taming strategy as
%proposed in

Note that if $\expozero < 1$, then there is no condition in
\Cref{theo:irred_D} on the step-size for HMC to be ergodic. This
conclusion may at first glance be surprising since if $\pi$ is a
$d$-dimensional Gaussian distribution with covariance matrix $\Sigma$,
then the step-size $h$ has to be chosen smaller than
$2/\sqrt{\lambda_{\mathrm{max}}}$, where $\lambda_{\mathrm{max}}$ is
the largest eigenvalues of $\Sigma$, which is also the Lipschitz
constant of the gradient of the associated potential. If a larger
step-size $h$ is used, the leapfrog integrator is unstable, see
\eg~\cite[Example 3.4, Proposition 3.1]{bou-rabee:sanz-serna:2018},
meaning that the iterates of the algorithm diverge. But the Gaussian
distribution satisfies \Cref{assum:regOne}$(\expozero)$ for
$\expozero=1$ strictly.
% This result is however not that
% surprising because of the acceptance/rejection mechanism inherent to
% Metropolis-Hastings schemes: an MCMC algorithm can be (geometrically)
% ergodic whereas the proposal is null or even transient. The most
% straightforward example is the symmetric random walk Metropolis and
% the Metropolis Adjusted Langevin Algorithm (MALA). For the Metropolis
% algorithm, the proposal kernel is a random walk which is null if
% $ d \leq 2$ and transient otherwise. On the other hand, consider a
% target density $\pi$ of the form \eqref{eq:def_density_pi}, where
% $U: \rset^d \to \rset$ is continuously differentiable but with
% $\lim_{\norm{q}} \normLigne{\nabla U(q)}/ \normLigne{q} = \plusinfty$,
% take for example $U(x) = \norm[4]{x}$. Then the Euler Maruyama
% discretization of the Langevin diffusion associated with $U$ is
% unstable (see \cite[Theorem 3.2]{roberts:tweedie:1996}) for all
% discretization step-size but by \cite[Equation
% 9]{roberts:tweedie:1996:biometrika} and \cite[Corollary
% 2]{tierney:1994}, the associated MALA algorithm is Harris recurrent
% for any step-size, hence is ergodic (the $\phi$-irreducibility is all
% what is needed, since the Markov kernel is reversible and therefore
% admits an invariant distribution, which is unique). This former result
% holds for $U(x) = \norm[2]{x}$ for all $x \in \rset^d$ as well. By
% \cite[Theorem 3.1, Theorem 3.2]{roberts:tweedie:1996}, the
% Euler-Maruyama discretization is geometrically ergodic for step-size
% $h < 1/2$ and transient for $h > 1$, whereas MALA in this case is
% Harris recurrent for all discretization step-size $h$ by the same
% arguments as above.
We illustrate on a numerical example that under \Cref{assum:regOne}$(\expozero)$, for $\expozero <1$,
  the \textit{unadjusted} HMC proposal is in fact numerically stable
  and the HMC algorithm does converge for a step-size $h > 2 /
  \sqrt{\constzero}$, where $\constzero$ is the Lipschitz constant of $\nabla U$.
  In this example, we consider the potential $U : \rset \to \rset$ given
  for all $x \in \rset$ by $U(x) = 2 \defEnsLigne{1+\abs{x}^2}^{3/4}$. Then
  $U'(x) = 3(\abs{x}^2 +1)^{-1/4}x$ and is Lipschitz with constant
  $\constzero = 3$. We then run the unadjusted/adjusted HMC algorithm for a
  step-size $h =1.5 > 2/\sqrt{\constzero} \approx 1.15$ and a number of
  leapfrog-step $T =2$.  We can observe in
  \Cref{fig:experiments_convergence} the convergence of the HMC
  algorithm for the test function $f : q \mapsto \abs{q}^2$.
  \Cref{fig:experiments_stability} illustrates that the
  adjusted/unadjusted HMC are numerically stable even if $h =1.5 > 2/\sqrt{\constzero} \approx 1.15$, since the gradient
  is sub-linear.


\begin{figure}[h]
	\begin{center}		
		\includegraphics[width=12.2cm]{convergence_3_4_D}
\end{center}
	\caption{Convergence of the HMC algorithm for $U(x) = 2
  \defEnsLigne{1+\abs{x}^2}^{3/4}$, $h =1.5 > 2/\sqrt{\constzero}$ and $T=2$. The test function is $f : q \mapsto \abs{q}^2$. The red line indicates the real value of $\int_{\rset} f(q) \rmd \pi(q)$ estimated by numerical integration}
	\label{fig:experiments_convergence}
\end{figure}


\begin{figure}[h]
	\begin{center}
	\begin{tabular}{p{0.1cm}cp{0.1cm}c}
&		\includegraphics[width=5.8cm]{trace_plot_HMC_3_4_D}
		& &
		\includegraphics[width=5.8cm]{trace_plot_UHMC_3_4_D}\\
& (a) & & (b)
	\end{tabular}
\end{center}
	\caption{Trace plots for the adjusted (a) / unadjusted (b) HMC algorithm for $U(x) = 2
  \defEnsLigne{1+\abs{x}^2}^{3/4}$, $h =1.5 > 2/\sqrt{\constzero}$ and $T=2$.}
	\label{fig:experiments_stability}
\end{figure}


Finally, note that our results can be easily extended to the case
where the number of steps is random. We briefly describe the main arguments to obtain such
extension.
Let $(\varpi_i)_{i\in \nset^*}$ be a probability distribution on $\nset^*$ and $(h_i)_{i \in \nset^*}$ be a sequence of  positive real
numbers.  Define the randomized Hamiltonian kernel
$\randomkerhmc_{\mathbf{h},\bfvarpi}$ on $(\rset^d, \borelSet(\rset^d))$ associated with $(\varpi_i)_{i\in \nset^*}$ and
$(h_i)_{i \in \nset^*}$ by
\begin{equation}
\label{eq:randomhmc}
\randomkerhmc_{\mathbf{h},\bfvarpi} = \sum_{i\in \nset^*} \varpi_i \Pkerhmc[h_i][i] \eqsp.
\end{equation}
We denote by $\supp(\bfvarpi)= \set{i \in \nset^*}{\omega_i \ne 0}$ the support of the distribution $\bfvarpi$.
\begin{corollary}
  \label{coro:ergod-hmc-algor}
Let $\expozero \in \ccint{0,1}$ and assume \Cref{assum:regOne}($\expozero$).
Let $(\varpi_i)_{i\in \nset^*}$ be a probability distribution on $\nset^*$, $(h_i)_{i \in \nset^*}$ be a sequence of  positive real
numbers, and $\randomkerhmc_{\mathbf{h},\bfvarpi}$ be the randomized Hamiltonian kernel associated with $(\varpi_i)_{i\in \nset^*}$ and
$(h_i)_{i \in \nset^*}$.
\begin{enumerate}[label=(\alph*)]
\item
\label{coro:ergod-hmc-algor_a}
Assume that $\F$ is twice continuously and there exists $i \in \nset^*$ such that   $ [ \{1 + h_i\constzero^{1/2} \vartheta_1( h_i \constzero^{1/2}) \}^i - 1 ] < 1$ and  $\varpi_i > 0$   where $\vartheta_1$ is given by  \eqref{eq:def_vartheta_1}. Then the conclusions  of \Cref{theo:irred_harris}-\ref{theo:irred_harris_c} hold for $\randomkerhmc_{\mathbf{h},\bfvarpi}$.
\item
\label{coro:ergod-hmc-algor_b}
 If $\expozero \in \coint{0,1}$, then the conclusions of \Cref{theo:irred_D}-\ref{theo:irred_D_a} hold for $\randomkerhmc_{\mathbf{h},\bfvarpi}$.
\item
\label{coro:ergod-hmc-algor_c}
 If $\expozero = 1$ and there exists $i \in \supp(\bfvarpi)$ such that  $ [ \{1 + h_i \constzero^{1/2} \vartheta_1(h_i \constzero^{1/2}) \}^i - 1 ] < 1$, then the conclusions of \Cref{theo:irred_D}-\ref{theo:irred_D_b} hold for $\randomkerhmc_{\mathbf{h},\bfvarpi}$.
\end{enumerate}
\end{corollary}

\begin{proof}
  \ref{coro:ergod-hmc-algor_a} follows from \Cref{theo:irred_harris} and \Cref{propo:harris_rec}. \ref{coro:ergod-hmc-algor_b} and \ref{coro:ergod-hmc-algor_c} are straightforward applications of \Cref{theo:irred_D}.
\end{proof}

 % $\mathbb{P}(T=1) >0$ and there exists $s
% >0$ such that for all $q_0 \in
% \rset^d$,$\mathbb{E}[\rme^{s\Phiverletq[h][T](q_0,p_0)}]< \plusinfty$,
% where $\Phiverletq[h][T]$ is defined in \eqref{eq:def_Phiverletq} and
% $p_0$ is a standard Gaussian random variable, .

% Note that by \cite[Theorem 14.0.1]{meyn:tweedie:2009}, under
% \Cref{assum:regOne}($\expozero$) for $\expozero \in \coint{0,1}$,
% \Cref{theo:irred_D} implies that for all $T \in \nset^*$ and $h >0$, for
% $\pi$-almost every $q \in \rset^d$,
%   \begin{equation}
% \lim_{n \to \plusinfty}    \tvnorm{\delta_q \Pkerhmc[h][T]^n - \pi} = 0 \eqsp.
%   \end{equation}

% \begin{corollary}\label{co:irr}
% Assume  \Cref{assum:regOne} $(\expozero)$.
% \begin{enumerate}[label=(\alph*)]
% \item
% \label{item:co:irr_1}
% If $\expozero \in \coint{0,1}$ for all $h>0$ and $T \in
%   \nset^*$, $\Pkerhmc[h][T]$ is irreducible with respect to the Lebesgue
%   measure and therefore is ergodic: for all $x \in \rset^d$,
%   \begin{equation}
% \lim_{n \to \plusinfty}    \tvnorm{\delta_x \Pkerhmc[h][T]^n - \pi} = 0 \eqsp.
%   \end{equation}
% \item
% \label{item:co:irr_2}
% If $\expozero = 1$, there exists $\hirr>0$ such that for all $h \in \ocintLigne{0,\hirr}$ and $T \in \nset^*$, $\Pkerhmc[h][T]$ is irreducible with respect to the Lebesgue  measure and therefore is ergodic.
% \end{enumerate}
% \end{corollary}

% \begin{proof}
%   It is a direct consequence of \Cref{pr:small} and \cite[Corollary 2]{tierney:1994}.
% \end{proof}

%%% Local Variables:
%%% mode: latex
%%% TeX-master: "main"
%%% End:



\section{Geometric ergodicity of HMC}
\label{sec:geom-ergod-hmc}



In this section, we give conditions on the potential $\F$ which imply that
the HMC kernel \eqref{eq:def_Phiverletq} converges geometrically fast to its invariant distribution.
Let $V: \rset^d \to \coint{1,\plusinfty}$ be a measurable function and $\Pker$
be a Markov kernel on $(\rset^d,\borelSet(\rset^d))$. The Markov kernel $\Pker$ is said to
be $V$-uniformly geometrically ergodic if $\Pker$ admits an invariant probability $\pi$
and there exists $\rho \in \coint{0,1}$
and $\varsigma \geq 0$ such that for all $\q \in \rset^d$ and $k \in \nset^*$,
\begin{equation}
  \Vnorm[V]{\Pker^k(\q,\cdot)-\pi} \leq \varsigma \rho^{k} V(\q) \eqsp.
\end{equation}
By \cite[Theorem 16.0.1]{meyn:tweedie:2009}, if $\Pker$ is aperiodic, irreducible and satisfies a Foster-Lyapunov drift condition, \ie~there exists a small set $\Csf$ for $\Pker$, $\lambda \in \coint{0,1}$ and $b < \plusinfty$ such that for all $\q \in \rset^d$,
\begin{equation}
\label{eq:foster-lyapunov}
\Pker V  \leq \lambda V + b \1_{\Csf} \eqsp,
\end{equation}
then $\Pker$ is $V$-uniformly geometrically ergodic. If a function $V : \rset^d \to \coint{1,\infty}$
satisfies \eqref{eq:foster-lyapunov}, then $V$ is said to be a Foster-Lyapunov function for $\Pker$.
We first give an elementary condition to establish the $V$-uniform geometric
ergodicity for a class of generalized Metropolis-Hastings  kernels which includes HMC kernels as a particular example.


Let $\Kker$ be a proposal kernel on $(\rset^d, \borelSet(\rset^{2d}))$ and $\alphagen : \rset^{3 d } \to
\ccint{0,1}$ be an acceptance probability, assumed to be Borel measurable. Consider the Markov kernel $\Pker$ on $(\rset^d,\borelSet(\rset^d))$ defined for all $\q \in \rset^d$ and
$\eventA \in \borelSet(\rset^d)$ by
\begin{equation}
\label{eq:def_kenel_MH}
  \Pker(\q,\eventA) = \int_{\rset^{2d}} \1_{\eventA}(\projq(z)) \alphagen(\q,z) \Kker(\q, \rmd z )
+ \updelta_{\q}(\eventA) \int_{\rset^{2d}} \defEns{1- \alphagen(\q,z) }\Kker(\q, \rmd z)  \eqsp,
\end{equation}
where $\projq : \rset^{d} \times \rset^d \to \rset^d$ is the canonical projection onto the first
$d$ components.
For $h \in \rset^*_+$ and $T \in \nset^*$, $\Pkerhmc[h][T]$ corresponds to $\Pker$ with
$\Kker$ and $\alphagen$  given for all $\q,\p,\x \in \rset^d$ and $\Bsf \in
\borelSet(\rset^{2d})$ respectively by
\begin{align}
\label{eq:def_Pker_proposition_double}
  \PkerhmcD[h][T](\q,\Bsf) &= (2\uppi)^{-d/2}\int_{\rset^{d}} \1_{\Bsf}\parenthese{\Phiverletq[h][T](\q,\tilde{\p}),\tilde{\p}} \rme^{-\norm{\tilde{\p}}^2/2} \rmd \tilde{\p} \eqsp, \\
\label{eq:def_alpha_acc_tilde_hmc}
\tildeAlphaacc(\q,(\tilde{\q},\tilde{\p})) & =
\begin{cases}
\alphaacc\defEns{(\q,\tilde{p}),\Phiverlet[h][T](\q,\tilde{p})} \eqsp, & \text{if}\, \tilde{q}= \Phiverletq[h][T](\q,\tilde{\p}) \eqsp, \\
0 & \text{otherwise} \eqsp,
\end{cases}
\end{align}
where  $\Phiverlet[h][T]$, $\Phiverletq[h][T]$ and $\alphaacc$  are  defined in   \eqref{eq:def_Phiverlet}, \eqref{eq:def_Phiverletq} and \eqref{eq:def_acc_ratio}, respectively. Let $\Vgeo : \rset^d \to \coint{1,\plusinfty}$ be a
\emph{norm-like}  function,  \ie\ a measurable function such that for all $M \in \rset_+$, the level sets $\set{\q \in \rset^d}{\Vgeo(\q) \leq M}$ are compact. Note that if $\Vgeo$ is norm-like, for any $M \in \rset_+$, $\set{\q \in \rset^d}{\Vgeo(\q) \leq M}^{\complementary}$ is non-empty.   The function $\Vgeo$ naturally extends on $\rset^{2d}$ by
setting for all $(\q,\p) \in \rset^{2d}$, $\Vgeo(\q,\p) = \Vgeo(\q)$.
For all $\q \in \rset^d$, define:
\begin{equation}
\label{eq:def_rej_ballV}
%\begin{aligned}
  \rejectregion(\q) = \defEns{z \in \rset^{2d} \, , \, \alphagen(\q,z) < 1  } \eqsp, \,
    \ballV(\q) = \defEns{z \in \rset^{2d} \, , \, \Vgeo(\projq(z)) \leq \Vgeo(\q) } \eqsp.
%\end{aligned}
\end{equation}
The set $\rejectregion(\q)$ is the potential rejection region.
Our next result gives a condition on $\Kker$ and $\alphagen$ which
implies that if $V$ is a Foster-Lyapunov function for $\Kker$ then
$\Pker$ satisfies a Foster-Lyapunov drift condition as well. This
result is inspired by \cite[Theorem~4.1]{roberts:tweedie:1996}, which is used to show the $V$-uniform geometric ergodicity of the MALA algorithm.
\begin{proposition}
\label{propo:geo_drift_MH}
  Let $\Vgeo : \rset^d \to \coint{1,\plusinfty}$ be a norm-like  function.
  Assume moreover that there exist  $\lambdageo \in \coint{0,1}$ and $\bgeo \in \rset_+$ such that
  \begin{equation}
  \label{eq:assum:geo_ergo_1}
  \Kker \Vgeo \leq  \lambdageo \Vgeo + \bgeo \eqsp.
  \end{equation}
and
\begin{equation}
\label{eq:assum:geo_ergo_2}
  \lim_{M \to \plusinfty} \sup_{\set{\q \in \rset^d}{\Vgeo(\q) \geq M}} \Kker(\q,\rejectregion(\q) \cap \ballV(\q)) = 0  \eqsp.
\end{equation}
 Then there exist  $\lambdageotilde \in \coint{0,1}$ and $\bgeotilde \in \rset_+$ such that
 $\Pker \Vgeo \leq  \lambdageotilde \Vgeo + \bgeotilde$ where $\Pker$ is given by \eqref{eq:def_kenel_MH}.
\end{proposition}
\begin{proof}
The proof is postponed to \Cref{sec:proof-crefpr}.
\end{proof}
We show below that under appropriate conditions, the proposal kernel $\PkerhmcD[h][T]$ and
the acceptance probability $\tildeAlphaacc$ given by \eqref{eq:def_Pker_proposition_double} and
\eqref{eq:def_alpha_acc_tilde_hmc} satisfy the conditions of
\Cref{propo:geo_drift_MH} which imply that the HMC kernel
$\Pkerhmc[h][T]$ is $V$-uniformly geometrically ergodic. %We assume in the following conditions.
For $\m \in \ocint{1,2}$, consider the following assumption:



\begin{assumption}[$m$]
  \label{assum:potential:c}
There exist $\constthree \in \rset^*_+$ and $\constfour \in \rset$ such that for all $\q \in \rset^d$,
  \begin{equation}
    \ps{\nabla \F(\q)}{\q} \geq \constthree \norm{\q}^{m} -\constfour \eqsp.
  \end{equation}
\end{assumption}
For all $\a \in \rset_+^*$ and $\q \in \rset^d$, define
\begin{equation}
\label{eq:def_Va}
\Vdrifta[a] (\q) = \exp(\a \norm{\q}) \eqsp.
\end{equation}
\begin{proposition}
\label{lem:drift_uhmc}
%   \begin{equation}
% \label{eq:hyp:drift_uhmc}
%     \liminf_{\q \to \plusinfty} \ps{\nabla \F(\q)}{\q}/ \norm{\q}^{\expozero+1} >0 \eqsp.
%   \end{equation}
 % Let $T \in \nset^*$. %Then the following holds
\begin{enumerate}[label=(\alph*)]
\item   \label{lem:drift_uhmc_1}
 Assume   \Cref{assum:regOne}$(m-1)$ and  \Cref{assum:potential:c}$(\m)$ for some $\m \in \ooint{1,2}$. Then, for all $T \in \nsets$,  $h \in \rset^*_+$, and $\a \in \rsetep$, there exist $\lambda \in \coint{0,1}$ and $\b \in \rsetp$ such that
  \begin{equation}
    \label{eq:drift_lem}
      \PkerhmcD[h][T] \Vdrifta[\a] \leq \lambda  \Vdrifta[\a] + \b \eqsp.
  \end{equation}
\item
\label{lem:drift_uhmc_2}
 Assume   \Cref{assum:regOne}$ (1)$ and  \Cref{assum:potential:c}$ (2)$.  Let $\bar{S} > 0$ be such that $\Theta(S) < \constthree$ for any $S \in \ocint{0,\bar{S}}$, where
\begin{align}
\label{eq:definition-function-C}
  \Theta(s)&= 2 \constzero^{1/2} \vartheta_2(s) \{ \rme^{\constzero^{1/2} s \vartheta_1(\constzero^{1/2} s)} - 1\} \\
  & \qquad  \qquad + 6 s^2 \left(   \constzeroT ^2  +  \constzero \vartheta_2^2(s) \{ \rme^{\constzero^{1/2} s \vartheta_1(\constzero^{1/2} s)} - 1\}^2\right) \eqsp.
\end{align}
Then, for all $a \in \rsetep$,  $T \in \nsets$ and  $h \in \ocint{0,\bar{S}/T}$,  there exist  $\lambda \in \coint{0,1}$ and $\b \in \rsetp$ which satisfy \eqref{eq:drift_lem}.
\end{enumerate}
\end{proposition}
\begin{proof}
  The proof is postponed to \Cref{sec:proof-crefl-2}
\end{proof}
We now derive sufficient conditions under which the condition \eqref{eq:assum:geo_ergo_2} of
\Cref{propo:geo_drift_MH} is satisfied.

\begin{assumption}[$\m$]
\label{assum:potential}
\begin{enumerate}[label = (\roman*)]
\item \label{assum:potential:a}
$\F \in C^3(\rset^d)$  and there exists $\constone \in \rset_+^*$ such that for all $\q \in \rset^d$ and $k=2,3$:
\begin{equation}
%\label{eq:10}
\norm{D^k \F(\q)}\leq \constone \defEns{1+\norm{\q}}^{\m-k} \eqsp.
\end{equation}
\item \label{assum:potential:b}
There exist $\consttwo \in \rset_+^* $ and $\rhtwo \in \rset^+$ such that for all $\q \in \rset^d$, $\norm{q}\geq \rhtwo$,
\begin{equation}
%\label{eq:11}
D^2\F(\q)\defEns{ \nabla \F(\q)\otimes  \nabla \F(\q)}  \geq \consttwo \norm{\q}^{3\m-4} \eqsp.
\end{equation}
  \end{enumerate}
\end{assumption}

It is easily checked that under \Cref{assum:potential}, the results of \Cref{sec:ergodicity-hmc} can be applied, \ie~$\nabla \F$ satisfies \Cref{assum:regOne}($\m-1$); see \Cref{lem:grad_Lip_F}.

Condition \Cref{assum:potential:c}$(m)$ and \Cref{assum:potential}$(m)$ are satisfied by power functions $\q \mapsto c\norm{\q}^\m$. More generally, they are satisfied by $\m$-homogeneously quasiconvex functions with convex level sets  outside a ball and by  perturbations of such functions.

We say that a function $\F_0$ is $m$-homogeneous quasi-convex
outside a ball of radius $\Rexp$ if the following conditions are satisfied:
\begin{enumerate}[(QC-1)]
\item for all $t \geq 1$ and $q \in \rset^d$, $\norm{\q}\geq \Rexp$, $\F_0(t \q)= t^\m \F_0(\q)$.
% $$
% \F_0(t \q)= t^\m \F_0(\q) \eqsp.
% $$
\item for all $\q \in \rset^d$, $\norm{\q} \geq \Rexp$, the level sets $\{ \x\, :\, \F_0(\x) \leq \F_0(\q)\}$ are convex.
\end{enumerate}
\begin{proposition}
\label{le:convex}
Let $m \in \ccint{1,2}$  and $\Rexp \in \rset_+$.  Assume that the potential $\F$ may be decomposed as
$$
\F(\q)=\F_0(\q)+G(\q) \eqsp, \quad \text{$\q\in \rset^d$, $\norm{\q} \geq \Rexp$} \eqsp,
$$
where the functions $\F_0,G \in C^3(\rset^d)$ satisfy the following two conditions:
  \begin{enumerate}[(A)]
  \item $\F_0$ is $\m$-homogeneously quasiconvex outside a ball of radius $\Rexp$ and $\lim_{\norm{\q} \to \plusinfty} \F_0(\q)=\infty$.
\label{le:convex:a}
\item
\label{le:convex:b}
For $k=2,3$, $\lim_{\norm{\q} \to \plusinfty}\normop{D^k G(\q)}/  \norm{\q}^{\m-k}= 0$.
% \begin{equation}%\label{eq:lower}
% \lim_{\norm{\q} \to \plusinfty}\normop{D^k G(\q)}/  \norm{\q}^{\m-k}= 0 \eqsp.
% \end{equation}
  \end{enumerate}
Then $\F$ satisfies  \Cref{assum:potential:c}$(m)$ and  \Cref{assum:potential}$(\m)$.
\end{proposition}
\begin{proof}
The proof is postponed to \Cref{sec:proof-crefle:convex}.
\end{proof}

To show that the condition \eqref{eq:assum:geo_ergo_2} of
\Cref{propo:geo_drift_MH} is satisfied under
\Cref{assum:potential}$(m)$, we rely on the following important result which implies that the probability of accepting a move goes to 1 as $\norm{q} \to \infty$.
\begin{proposition}
  \label{propo:accept} Assume
  \Cref{assum:potential}$(m)$ for some $\m \in \ocint{1,2}$. Let $\gamma \in \ooint{0,\m-1}$.
  \begin{enumerate}[label=(\alph*)]
  \item
  \label{propo:accept_1}
  If $\m\in (1,2)$, for all $T \in \nsets$, $h \in \rset_+^*$, there exists $R_{\Ham} \in \rset_+$ such that for
  all $\q_0,\p_0 \in \rset^d$, $\norm{q_0} \geq R_{\Ham}$ and
  $\norm{p_0} \leq \norm{\q_0}^{\gamma}$, $ \Ham(\Phiverlet[h][T](q_0,p_0)) -
  \Ham(q_0,p_0) \leq 0$.
\item
  \label{propo:accept_2}
  If $\m=2$,   there exists $\bar{S} >0$ such that for any $T \in \nsets$ and $h \in \ocint{0, \bar{S}/T^{3/2}}$,  there exists $R_{\Ham} \in \rset_+$ satisfying for all $\q_0,\p_0 \in \rset^d$, $\norm{q_0} \geq R_{\Ham}$ and
  $\norm{p_0} \leq \norm{\q_0}^{\gamma}$, $ \Ham(\Phiverlet[h][T](q_0,p_0)) -
  \Ham(q_0,p_0) \leq 0$.
  \end{enumerate}
\end{proposition}


\begin{proof}
  The proof is postponed to  \Cref{sec:proof-crefth}.
\end{proof}
%This behavior comes a bit as a surprise. It


This result  means that far in the tail the HMC proposal are "inward".
We illustrate the result of \Cref{propo:accept}-\ref{propo:accept_1}
in \Cref{fig:H_behaviour} for $U$ given by $\q \mapsto
(\norm[2]{\q}+\delta)^{\kappa}$ for $\kappa=3/4$, $h = 0.9$ and
$p_0 \in \rset^d$, $\norm{p_0}=1$. Note that this potential satisfies
the condition of the proposition. We can observe that choosing the different
initial conditions $q_0$ with increasing norm imply that $\tilde{T} =
\max\{k \in \nset ; \Ham(\Phiverlet[h][k](q_0,p_0)) - \Ham(q_0,p_0)
<0\}$ increases as well.

 \begin{figure}[h]
   \centering
   \includegraphics[scale=0.3]{hamiltonian_behaviour}
   \caption{Behaviour of $(\Ham(\Phiverlet[h][k](q_0,p_0)))_{k \in \{0,\ldots,T\}}$ for different initial conditions $q_0$.}
   \label{fig:H_behaviour}
 \end{figure}


 However, in the case $m=2$, \Cref{propo:accept}-\ref{propo:accept_2} only implies that the HMC proposal is inward only if the step size $h$ is sufficiently small with respect to the number of leapfrog step $T$, \ie~is of order $\bigO(T^{-3/2})$. To relax this condition, we strengthen \Cref{assum:potential}($2$) by assuming that $U$ is a smooth perturbation of a quadratic function.
 \begin{assumption}
   \label{ass:pertub}
   There exist $\tilde{U} : \rset^d \to \rset$, continuously differentiable, and  a positive definite matrix $\Sigmabf$ such that
   $U(q) = \ps{\Sigmabf q}{q}/2 + \tilde{U}(q)$ and there exist $\constfive \geq 0$ and  $\varrho \in \coint{1,2}$  such that for any $q,x \in \rset^d$,
   \begin{align}
     \absLigne{\tilde{U}(q)} &\leq \constfive(1+\norm[\varrho]{q}) \eqsp, \quad  \normLigne{\nabla \tilde{U}(q)} \leq \constfive(1+\norm[\varrho-1]{q}) \eqsp,\\  \qquad \qquad \qquad & \normLigne{\nabla \tilde{U}(q) - \nabla \tilde{U}(x)} \leq \constfive \norm{q-x} \eqsp.
   \end{align}
 \end{assumption}
Note that it is straightforward to check that under \Cref{ass:pertub}, the conditions \Cref{assum:regOne}$(1)$ and  \Cref{assum:potential:c}$(2)$ hold.

 \begin{proposition}
  \label{propo:accept_pertub} Assume  \Cref{ass:pertub}  and let $\gamma \in \ooint{0,1}$.
There exists a  constant $\bar{S} >0$ such that for all $T \in \nsets$, $h \in \ocint{0,\bar{S}/T}$, there exists $R_{\Ham} \in \rset_+$ such that for
  all $\q_0,\p_0 \in \rset^d$, $\norm{q_0} \geq R_{\Ham}$ and
  $\norm{p_0} \leq \norm{\q_0}^{\gamma}$, $ \Ham(\Phiverlet[h][T](q_0,p_0)) -
  \Ham(q_0,p_0) \leq 0$.
\end{proposition}
\begin{proof}
  The proof is postponed to  \Cref{sec:proof-crefth_accept_2}.
\end{proof}
We now can establish the geometric ergodicity of the HMC sampler.
\begin{theorem}
  \label{theo:geoErg}
  \begin{enumerate}[label=(\alph*)]
  \item
  \label{item:theo_1}
 If   \Cref{assum:potential:c}$(\m)$ and  \Cref{assum:potential}$(m)$ hold for some $\m\in (1,2)$, then for all $a \in \rset_+^*$,  $T \in \nset^*$ and $h > 0$, the HMC kernel $\Pkerhmc[h][T]$ is $\Vdrifta[\a]$-uniformly geometrically ergodic, where $\Vdrifta[a]$ is defined by \eqref{eq:def_Va}.
\item
  \label{item:theo_2}
  If  \Cref{assum:potential:c}$(2)$ and \Cref{assum:potential}$(2)$ hold, then there exists $\bar{S}>0$ such that for all  $a \in \rset^*_+$, $T \in \nset^*$ and $h \in \ooint{0,\bar{S}/T^{3/2}}$, $\Pkerhmc[h][T]$ is $\Vdrifta[\a]$-uniformly geometrically ergodic.
\item \label{item:theo_3}
  If \Cref{ass:pertub} holds, then there exists $\bar{S}>0$ (depending only on $\Sigmabf$ and $\constfive$) such that for all $a \in \rset^*_+$, $T \in \nset^*$ and $h \in \ooint{0,\bar{S}/T}$, $\Pkerhmc[h][T]$ is $\Vdrifta[\a]$-uniformly geometrically ergodic.
  \end{enumerate}
\end{theorem}
\begin{proof}[Proof of \Cref{theo:geoErg}]
  It is enough to consider \ref{item:theo_1} as the proof
  of  \ref{item:theo_2} and \ref{item:theo_3} follows exactly the same lines
  taking $\bar{S}$ small enough.  \Cref{lem:drift_uhmc} shows that for all $T \in \nsets$,  $h \in \rset^*_+$, and $\a \in \rsetep$, there exist $\lambda \in \coint{0,1}$ and $\b \in \rsetp$ such that the Foster-Lyapunov drift condition
  $\PkerhmcD[h][T] \Vdrifta[\a] \leq \lambda  \Vdrifta[\a] + \b$ is satisfied.
  By \Cref{propo:accept}, there exists $R_{\Ham} \geq 0$ such that for all $\q \in \rset^d$, $\norm{\q} \geq R_{\Ham}$,
\begin{equation}
\label{eq:proofpgeo_erg_0}
  \int_{\rejectregion(q)}   \PkerhmcD[h][T](\q, \rmd z ) \leq (2\uppi)^{-d/2} \int_{\{ \norm{\p} \geq \norm{q}^{\gamma}\} } \rme^{-\norm{p}^2/2} \rmd p \eqsp,
\end{equation}
for $\gamma \in \ooint{0,m-1}$ where $\rejectregion(\q) = \set{z \in \rset^{2d}}{\tildeAlphaacc(\q,z) < 1 }$ (see~\eqref{eq:def_alpha_acc_tilde_hmc}), which implies that
\begin{equation}
\label{eq:proof:geo_erg:1}
  \lim_{M \to \plusinfty} \sup_{\norm{\q} \geq M} \int_{\rejectregion(q)} \PkerhmcD[h][T](\q, \rmd z ) = 0 \eqsp,
\end{equation}


Since $\Vdrifta[a]$ is norm-like,  \Cref{propo:geo_drift_MH} implies that  for all $T > 0$ and $h > 0$, there exists $\lambdageotilde$ and $\bgeotilde$ (depending upon $a$, $h$ and $T$) such that
$\Pkerhmc[h][T] \Vdrifta[a] \leq \lambdageotilde \Vdrifta[a] + \bgeotilde$.
For all $M \geq 0$ the level sets $\{ \Vdrifta[a] \leq M\}$ are compact and hence small by \Cref{theo:irred_D}.
\cite[Corollary~14.1.6]{douc:moulines:priouret:2018} then shows that there exists a small set $\msc$, $\check{\lambda} \in \coint{0,1}$ and $\check{b} \in \coint{0,1}$ such that $\Pkerhmc[h][T] \Vdrifta[a] \leq \check{\lambda} \Vdrifta[a] + \check{b} \1_{\msc}$.  Since $\Pkerhmc[h][T]$ is aperiodic, the result follows from \cite[Theorem~15.2.4]{douc:moulines:priouret:2018}.
\end{proof}

We  finally consider the case where the number of leapfrog steps is a random variable
independent of the current state.
\begin{theorem}
\label{coro:geo_ergod-hmc-algor}
\begin{enumerate}[label=(\alph*)]
  \item
  \label{item:geo_ergod-hmc-algor:theo_1}
 If   \Cref{assum:potential:c}$(\m)$ and  \Cref{assum:potential}$(m)$ hold for  $\m\in (1,2)$,  then for all probability distributions $\bfvarpi=(\omega_i)_{i \in \nset^*}$ on $\nset^*$, all sequences $\mathbf{h}= (h_i)_{i \in \nset*}$ of positive numbers,  and $a \in \rset^*_+$, the randomized kernel  $\randomkerhmc_{\mathbf{h},\bfvarpi}$ \eqref{eq:randomhmc} is $\Vdrifta[\a]$-uniformly geometrically ergodic, where $\Vdrifta[a]$ is defined by \eqref{eq:def_Va}.
\item
  \label{item:geo_ergod-hmc-algor:theo_2}
  If  \Cref{assum:potential:c}$(2)$ and \Cref{assum:potential}($2$) hold, then there exists $\bar{S}>0$ such that for all probability distributions $\bfvarpi=(\omega_i)_{i \in \nset^*}$ on $\nset^*$, all sequences $\mathbf{h}= (h_i)_{i \in \nset^*}$ satisfying $\max_{i \in \supp(\bfvarpi)} i^{3/2} h_i \leq \bar{S}$,  and $a \in \rset^*_+$, $\randomkerhmc_{\mathbf{h},\bfvarpi}$ is $\Vdrifta[\a]$-uniformly geometrically ergodic.
\item   \label{item:geo_ergod-hmc-algor:theo_3}
  If \Cref{ass:pertub} holds, then there exists $\bar{S}>0$ (depending only on $\Sigmabf$ and $\constfive$) such that for all  probability distributions $\bfvarpi=(\omega_i)_{i \in \nset^*}$ on $\nset^*$, all sequences $\mathbf{h}= (h_i)_{i \in \nset^*}$ satisfying $\max_{i\in \supp(\bfvarpi)} i h_i \leq \bar{S}$,  and $a \in \rset^*_+$, $\randomkerhmc_{\mathbf{h},\bfvarpi}$ is $\Vdrifta[\a]$-uniformly geometrically ergodic.
\end{enumerate}
\end{theorem}
\begin{proof}
It is enough to consider \ref{item:geo_ergod-hmc-algor:theo_1} as the proofs
of \ref{item:geo_ergod-hmc-algor:theo_2} and \ref{item:geo_ergod-hmc-algor:theo_3} are along the same lines.
Set $a \in \rset_+^*$. It is established in the proof of \Cref{theo:geoErg}  that  for all $i \in \nset^*$
$\Pkerhmc[i][h_i]$ satisfies a Foster-Lyapunov drift condition:
there exists $\check{\lambda}_i \in \coint{0,1}$ and $\check{b}_i < \infty$ such that
$\Pkerhmc[i][h_i] \Vdrifta[a] \leq \lambda_i \Vdrifta[a] + b_i$,
By \Cref{coro:ergod-hmc-algor},   $\randomkerhmc_{\mathbf{h},\bfvarpi}$ is irreducible and aperiodic and all the compact sets are small. We conclude by applying \cite[Theorem~15.2.4]{douc:moulines:priouret:2018}.
\end{proof}

Compared to \cite{livingstone:betancourt:byrne:girolami:2016}, which
establishes geometric ergodicity of the HMC kernel under an implicit
assumption on the behaviour of the acceptance rate, our conditions are
directly verifiable on the potential $U$.

On the other hand, our conditions are different than the one given by
\cite{bou:sanz:2017} to establish the geometric ergodicity of the
idealized randomized HMC, which assumed to exactly solve the Hamiltonian
ODE \eqref{eq:hamil_ode}. These conditions are the following
1)$\int_{\rset^d} \norm[2]{q} \rmd \pi(q) < \plusinfty$, 2) there
exist $C_1 \in \ooint{0,1}$ and $C_2 >0$ such that for all $q \in \rset^d$
\begin{equation}
\label{eq:hyp_boo_rabee_sanz_serna}
  (1/2) \ps{\nabla U(q)}{q} \geq C_1 U(q) + \frac{(\tau^{-1}C_1/4)^2+\tau^{-2}C_1(1-C_1)/4}{2(1-C_1)}\norm[2]{q} -C_2 \eqsp,
\end{equation}
where $\tau>0$ is the duration parameter of the RHMC algorithm.  Note
that these conditions assumed that the target density is lighter than
Gaussian. In comparison, our results can be applied to sub-quadratic
potentials. In addition, it can be shown that HMC is not geometrically
ergodic under \eqref{eq:hyp_boo_rabee_sanz_serna} on the following example associated with the potential defined by  \eqref{eq:def_U_mixture_gaussian} below.

% The condition \eqref{eq:hyp_boo_rabee_sanz_serna} is satisfied if $\pi$ is a
% mixture of $d$-dimensional Gaussian distributions.  We show
% numerically that there is strong evidences which imply that HMC is not
% geometrically ergodic for such examples.
The main difference with the
setting of \cite{bou:sanz:2017} is that HMC has a acceptance/rejection
step and the integrated acceptance ratio
\[ q \mapsto \int_{\rset^d} \alphaacc\defEnsLigne{(\q,\p),\Phiverlet[h][T](\q,\p)}
\rme^{-\norm[2]{p}/2} (2 \uppi)^{-d/2} \rmd p
\]
must not go to $0$ as
$\norm{q}$ goes to $\plusinfty$. This is essentially the reason why
\Cref{assum:potential} differs from \eqref{eq:hyp_boo_rabee_sanz_serna}. Indeed, to show that an
irreducible Markov kernel $\mathrm{P}$ on $(\rset^d, \mcb(\rset^d))$ is not geometrically
ergodic with respect to an invariant measure $\mu$, \cite[Theorem
5.1]{roberts:tweedie:1996:biometrika} states the following sufficient condition
\begin{equation}
  \label{eq:condition_non_geo_ergodicity}
 \mathrm{ess \, sup}_{q \in \rset^d} \mathrm{P}(q, \{q\}) = 1 \eqsp,
\end{equation}
 where
$\mathrm{ess\, sup}$ is taken with respect to $\mu$. Consider then
the target density $\pi$ with potential $U$ given for all $q =(q_1,q_2) \in \rset^2$ by
\begin{equation}
\label{eq:def_U_mixture_gaussian}
  U(q) = -\log(\rme^{-q_1^2 - 5 q_2^2} + \rme^{-5q_1^2 - q_2^2}) \eqsp.
\end{equation}
Note that $U$ satisfies the condition
\eqref{eq:hyp_boo_rabee_sanz_serna}. On the contrary, we may show that
\eqref{eq:condition_non_geo_ergodicity} holds, and therefore HMC is
not geometrically ergodic for such a potential $U$.  However, the
detailed calculations are very technical and not particularly
informative and we prefer to present a numerical evidence that
\eqref{eq:condition_non_geo_ergodicity} holds. Indeed,
\Cref{fig:accept_mix_gaussian} displays numerical computations of the mean acceptance ratio,
$\int_{\rset^2} \alphaacc\defEnsLigne{(\q,\p),\Phiverlet[h][T](\q,\p)}
\rme^{-\norm[2]{p}/2} (2 \uppi)^{-1} \rmd p= 1 -\Pkerhmc[h][T](q,\{q\})$ for $q_1 \in \{200,250,$
$300,350,400,450,500\}$,
$q_2 \in \ccint{q_1+10^{-4},q_1+2\cdot 10^{-4}}$ and $T=1$  which
corresponds to MALA. We can observe that the larger $q_1$, the smaller $1-\Pkerhmc[h][T](q,\{q\})$, which illustrates that  \eqref{eq:condition_non_geo_ergodicity}  holds for the HMC
kernel.

 \begin{figure}[h]
   \centering
   \includegraphics[scale=0.4]{mix_gaussian_1}
   \caption{}
   \label{fig:accept_mix_gaussian}
 \end{figure}


%For  $T \in \nset^*$, $\alpha \geq 0$ and $h_{\Ham} >0$, consider the following assumption.
% \begin{assumption}[$T,\alpha,h_{\Ham}$]
%   \label{assum:diff_ham}
%  There exists  $R_{\Ham}
%   \geq 0$ and such that for all $h \in
%   \ocint{0,h_{\Ham}}$, if
% \begin{equation}
% \label{eq:100}
% |q_0|\geq R_{\Ham} \quad\textrm{and}\quad |p_0|\leq |q_0|^\alpha,
% \end{equation}
% then
% \begin{equation}
% \label{eq:diff_ham_postitive}
%   \Ham(q_T,p_T) - \Ham(q_0,p_0) \geq 0 \eqsp.
% \end{equation}
% \end{assumption}



%%% Local Variables:
%%% mode: latex
%%% TeX-master: "main"
%%% End:



\section{Irreducibility for a class of iterative models}
\label{sec:irred-class-iter}

In this Section we establish the irreducibility of a Markov kernel associated to a random iterative model.
These results are of independent interest.
Let $\hfunb : \rset^d \times \rset^d \to \rset^d$ and $\alphagen: \rset^d \times \rset^d \to \ccint{0,1}$ be Borel measurable
functions and $\phib : \rset^d \to \ccint{0,\plusinfty}$ be a
probability density with respect to the Lebesgue measure.  Consider the Markov kernel $\Pkerb$ defined for all $x \in \rset^d$ and $\eventA \in \borelSet(\rset^d)$ by
\begin{equation}
\label{eq:def_pkerb}
  \Pkerb(x,\eventA) = \int_{\rset^d} \indi{\eventA}{\hfunb(x,z)} \alpha(x,z) \phib(z) \rmd z + \bar{\alpha}(x) \delta_x(\eventA) \eqsp,
\end{equation}
where $\bar{\alpha}(x)= \int_{\rset^d} \alpha(x,z) \phib(z) \rmd z$. 
Define for all $x \in \rset^d$, $\hfunb_x : \rset^d \to \rset^d$ by $\hfunb_x = \hfunb(x,\cdot)$.

First, we give  a result  from geometric measure
theory together with a proof for the reader's convenience, which will be essential for the proof of the statements of this section.  Let
$\ouvert\subset\rset^d$ be an open set and $\Theta: \ouvert\to \rset^d$ be
a measurable function such that there exist $y_0 , \tilde{y}_0 \in \rset^d$ and
$M, \tilde{M} > 0$ satisfying $\ball{\tilde{y}_0}{\tilde{M}} \subset \ouvert$ and
  \begin{equation}
    \label{eq:condition_sigma_finite}
\ball{y_0}{M} \subset \Theta(\ball{\tilde{y}_0}{\tilde{M}}) \eqsp.
  \end{equation}
 Define the measure $\lambda_{\Theta}$ on $(\rset^d,\borelSet(\rset^d))$
  by setting for any $\eventA \in \borelSet(\rset^d)$
\begin{equation}
  \label{eq:push_forward_mes}
\lambda_\Theta (\eventA) \eqdef \Leb\defEns{\Theta^{-1}(\eventA) \cap \ball{\tilde{y}_0}{\tilde{M}}}  \eqsp.
\end{equation}
Note %by \eqref{eq:condition_sigma_finite}
that $\lambda_\Theta$ is  a finite measure. Therefore by the Lebesgue decomposition theorem (see
\cite[Section 6.10]{rudin:1987}) there exist two  measures
$\lambda_\Theta^{(\text{a})}, \lambda_\Theta^{(\text{s})}$ on
$(\rset^d,\borelSet(\rset^d))$, which are absolutely continuous and
singular with respect to the Lebesgue measure on $\rset^d$
respectively, such that $\lambda_\Theta = \lambda_\Theta^{(\text{a})} +
\lambda_\Theta^{(\text{s})}$.
% Note that if for all compact set $\compact \subset \rset^d$, $\Theta^{-1}(\compact)$ is
% compact (\ie~$\Theta$ is a proper function), then $\lambda_\Theta$ is $\sigma$-finite.
\begin{proposition}\label{le:simple}
  Let $\ouvert\subset\rset^d$ be open and $\Theta: \ouvert\to \rset^d$ be a Lipschitz function
  satisfying \eqref{eq:condition_sigma_finite}.
 For any version  $\phi_\Theta$ of the density of $\lambda_\Theta^{(\text{a})}$ with respect to the Lebesgue
  measure on $\rset^d$, it holds
$$
\phi_\Theta(y)\geq \1_{ \ball{y_0}{M}}(y) \norm{\Theta}_{\Lip}^{-d}\eqsp, \quad \text{$\Leb$-a.e.}
$$
\end{proposition}
\begin{proof}
  Denote by $L = \norm{\Theta}_{\Lip}$. Let $y\in \ball{y_0}{M}$. By  \eqref{eq:condition_sigma_finite}, we may pick
  $z \in \ball{\tilde{y}_0}{\tilde{M}}$ such that $\Theta(z) = y$. Let
  $\delta_0 >0$ be such that $\ball{z}{\delta_0/L} \subset
  \ball{\tilde{y}_0}{\tilde{M}}$.  Since $\Theta$ is
  Lipschitz continuous,  for all
  $\delta \in \rset_+^*$,
  $\Theta(\ball{z}{\delta/L} \cap \open)\subset \ball{y}{\delta}$. Hence, for all
  $\delta \in \ocint{0,\delta_0}$, we have
  $$
\lambda_\Theta(\ball{y}{\delta})\geq
  L^{-d}\Leb(\ball{z}{\delta}) =   L^{-d}\Leb(\ball{y}{\delta}) \eqsp.
$$
 The claim follows from the differentiation theorem
  for measures, see \cite[Theorem 7.14]{rudin:1987}.
\end{proof}

We can now state our main results. Let $\rassG,\MassG \in \rset^*_+$ and $y_0 \in \rset^d$.
Consider the following assumptions.
\begin{assumptionG}
\label{assumG:phi}
$\phib$ and $\alphagen$ are lower semicontinuous and positive on $\rset^d$ and $\rset^{2d}$ respectively.
\end{assumptionG}

% \begin{assumptionG}
% \label{assumG:alpha}
%  is lower semicontinuous and positive on .
% \end{assumptionG}

\begin{assumptionG}[$\rassG,y_0,\MassG$]
  \label{assumG:irred_b}
  \begin{enumerate}[label=(\roman*), wide, labelwidth=!, labelindent=0pt]
  \item \label{assumG:irred_b_item_i} There exists $\constLiphx \in \rset_+$ such that for all $x \in \ball{0}{\rassG}$, $\hfunb_x$ is
    $\constLiphx$-Lipschitz, \ie~for all $z_1,z_2 \in \rset^d$,
    $\norm{\hfunb_x(z_1)-\hfunb_x(z_2)} \leq \constLiphx
    \norm{z_1-z_2}$.
\item \label{assumG:irred_b_item_ii} There exist $\tilde{y}_0 \in \rset^d$ and $\tMassG \in \rset_+^*$,
  such that for all $x \in \ball{0}{\rassG}$, $\ball{y_0}{\MassG} \subset \hfunb_x(\ball{\tilde{y}_0}{\tMassG})$.
  \end{enumerate}
\end{assumptionG}


\begin{theorem}
\label{theo:irred}
Assume \Cref{assumG:phi} and that there exist $y_0 \in \rset^d$, $R > 0$ and $M > 0$ such that \Cref{assumG:irred_b}($\rassG,y_0,\MassG$)  is satisfied. Then $\ball{0}{\rassG}$ is $1$-small for $\Pkerb$: for all $x \in \ball{0}{\rassG}$ and $\eventA \in \borelSet(\rset^d)$,
  \begin{equation}
    \Pkerb(x,\eventA) \geq \constLiphx^{-d} \min_{(x,z) \in \ball{0}{R} \times \ball{\tilde{y}_0}{\tMassG}} \defEns{\alphagen(x,z) \phib(z)} \Leb\defEns{\eventA \cap \ball{y_0}{\MassG}} \eqsp,
  \end{equation}
where $(\tilde{y}_0,\tilde{M}) \in \rset^d \times \rset_+^*$ is defined in \Cref{assumG:irred_b}($\rassG,y_0,\MassG$).
\end{theorem}

\begin{proof}%[Proof of \Cref{theo:irred}]
For all $x  \in \ball{0}{\rassG}$ and $\eventA \in \borelSet(\rset^d)$ we get
\begin{align}
\nonumber
\Pkerb(x,\eventA)  &= \int_{\rset^d}\indi{\eventA}{\hfunb(x,z)} \alphagen(x,z) \phib(z) \rmd z  = \int_{\rset^d}\indi{\hfunb_x^{-1}(\eventA)}{z} \alphagen(x,z) \phib(z) \rmd z \\
&\geq  \min_{(x,z) \in \ball{0}{R} \times \ball{\tilde{y}_0}{\tilde{M}}} \defEns{\alphagen(x,z)\phib(z)} \Leb\defEns{\hfunb_x^{-1}(\eventA) \cap\ball{\tilde{y}_0}{\tilde{M}}} \eqsp.
\label{eq:coro_leb_irred_1}
\end{align}
The proof follows from \Cref{le:simple} and \Cref{assumG:irred_b}$(R,y_0,M)$-\ref{assumG:irred_b_item_i} which imply
$ \Leb\defEns{\hfunb_x^{-1}(\eventA )\cap \ball{\tilde{y}_0}{\tilde{M}}} \geq  \constLiphx^{-d}  \Leb\defEns{\eventA  \cap \ball{y_0}{M}}$.
\end{proof}
The following Corollary is a straightforward consequence of \Cref{theo:irred}.
\begin{corollary}
\label{coro:irred}
Assume \Cref{assumG:phi} and  that there exists $(y_0,M) \in \rset^d \times \rset_+^*$ such that  for all $\rassG \in \rset_+^*$ \Cref{assumG:irred_b}($\rassG,y_0,\MassG$). Then $\Pkerb$ is irreducible with irreducibility measure $\Leb\defEns{\cdot \cap \ball{y_0}{\MassG}} $. In addition, all the compact sets are $1$-small.
\end{corollary}
In the next proposition, we give examples of functions $f$ which satisfy \Cref{assumG:irred_b}.
\begin{proposition}\label{le:degree_application}
Let  $\ga$ a function from $ \rset^d\times \rset^d$ to $\rset^d$ and $\ra \in \rset^{*}_+$. Assume that
 % \begin{enumerate}[label=\roman*)]
%   \item \label{application:irred_b_item_i}
% for all $\ra >0$ there exists $\Lga \geq
%     0$ such that for all $x \in \ball{0}{\ra}$, $z \mapsto \ga(x,z)$ is
%     $\Lga$-Lipschitz, \ie~for all $x \in \ball{0}{\ra}$,  $z_1,z_2 \in \rset^d$,
%     $\abs{\ga(x,z_1)-\ga(x,z_2)} \leq \Lga
%     \abs{z_1-z_2}$.
%\item \label{application:irred_b_item_ii}
\begin{enumerate}[label=(\roman*)]
\item
\label{propo:irred_b_item_i}
 there exists $\lipgr \in \rset_+$  such that for all $z_1,z_2,x \in \rset^d$, $\norm{x} \leq \ra$,
  \begin{equation}
    \label{eq:5}
    \norm{g(x,z_1) - g(x,z_2)} \leq \lipgr\norm{z_1-z_2} \eqsp.
  \end{equation}
\item
\label{propo:irred_b_item_ii}
there exist $ \Cga_{\ra,0} , \Cga_{\ra,1}
\in \rset_+$ such that for all $x,z \in \rset^d$, $\norm{x} \leq \ra$
\begin{equation}
  \label{eq:4}
\norm{g(x,z)} \leq  \Cga_{\ra,0} +   \Cga_{\ra,1} \norm{z}
\end{equation}
\end{enumerate}

% \begin{equation}\label{eq:gr}
% \abs{g(x,z)} \leq \Cga (1+|x|+|z|) \eqsp.
% \end{equation}
%\end{enumerate}
Let $\bg \in \rset$ and define $\hga : \rset^d \times \rset^d$ for all $x,z \in \rset^d$ by
\begin{equation}
  \hga(x,z) =  \bg z + \ga(x,z) \eqsp.
\end{equation}
If $\norm{\bg} > \Cga_{\ra,1} $, then $\hga$ satisfies \Cref{assumG:irred_b}($\ra,0,\MassG$) for all $\MassG \in \rset_+^*$ with $\tilde{y}_0=0$ and 
\begin{equation}
  \label{eq:deftildeM}
\tilde{M} = \{M  + \Cga_{\ra,0} \}/(\norm{\bg}-\Cga_{\ra,1} ) \eqsp.
\end{equation}
% Let $a,b\in\R$ with $|a|> C$ and for all $x \in \rset^d$ define $\Psi_x : \rset^d \to \rset^d$  by
% $$
% \Psi_x(y)=ay+bx+g(x,y) \eqsp, \text{ for all $y \in \rset^d$} \eqsp.
% $$
% Then for all  $R>0$ there exists $\epsilon >0$ satisfying for all $\eventA \in \borelSet(\rset^d)$, $\eventA \subset \ball{0}{R}$,
% \begin{equation}
%   \inf_{x \in \ball{0}{R}} \lambda_{\Psi_x}(\eventA) > \epsilon \Leb(\eventA) \eqsp,
% \end{equation}
% where $\lambda_{\Psi_x}$ is defined in \eqref{eq:push_forward_mes}.
\end{proposition}
We preface the proof by recalling some basic notions of degree theory.
\label{sec:defin-usef-results}
Let $\Dset$ be a bounded open set of $\rset^d$. Let $f:
\Dsetc \to \rset^d$ be a continuous function on
$\Dsetc$ continuously differentiable on $\Dset$. An element $x \in
\Dset$ is said to be a \emph{regular point} of $f$ if the Jacobian matrix of $f$ at $x$, $\Jac_f(x)$, is invertible.
An element $y \in f(\Dset)$ is said to be a \emph{regular value} of $f$ if any $x \in
f^{-1}(\{ y\})$ is a regular point.  %true if $y \not in f(\Dset)$

Let $f : \Dsetc \to \rset^d$ be a continuous function,  $C^{\infty}$-smooth on $\Dset$. Let $y \in \rset^d
  \setminus f(\partial \Dset)$ be a regular value of $f$. It is shown in \cite[Proposition and Definition 1.1]{outerelo:ruiz:2009} that the set $f^{-1}(\{y\})$ is finite. The degree of $f$ at $y$ is defined by
\begin{equation}
  \deg(f,\Dset,y) = \sum_{x \in f^{-1}(\{y \})} \sign\defEns{\det \parenthese{\Jac_f(x)}} \eqsp.
\end{equation}

\begin{proposition}[\protect{\cite[Proposition and Definition 2.1]{outerelo:ruiz:2009}}]
\label{defProp:degree_cont}
  Let $f : \Dsetc \to \rset^d$ be a continuous function and $y \in
  \rset^d \setminus f(\partial \Dset)$.
  \begin{enumerate}[label=(\alph*)]
  \item
\label{defProp:degree_cont_i}
 Then there exists  $g \in C(\Dsetc, \rset^d) \cap C^{\infty}(\Dset, \rset^d)$ such that $y$ is a regular value of $g$
  and $\sup_{x \in \Dsetc} \abs{f(x)-g(x)} < \dist(y,f(\partial
  \Dset))$.
\item For all functions $g_1,g_2:\Dsetc \to \rset^d$ satisfying \ref{defProp:degree_cont_i},
  \begin{equation}
    \deg(g_1,\Dset,y) = \deg(g_2,\Dset,y) \eqsp.
  \end{equation}
  \end{enumerate}
\end{proposition}
Under the assumptions of \Cref{defProp:degree_cont}, the degree of $f$ at $y$ is then defined for any $g:\Dsetc \to \rset^d$ satisfying \ref{defProp:degree_cont_i} by
\begin{equation}
  \deg(f,\Dset,y) =  \deg(g,\Dset,y) \eqsp.
\end{equation}

\begin{proposition}[\protect{\cite[Proposition
  2.4]{outerelo:ruiz:2009}}]
  \label{theo:deg_modif}
  Let $f,g : \Dsetc \to \rset^d$ be  continuous functions. Define
  $\hpy:\ccint{0,1} \times \rset^d \to \rset^d$ for all $t \in
  \ccint{0,1}$ and $x \in \rset^d$ by $\hpy(t,x) = t f(x) +
  (1-t)g(x)$. Let $y \in \rset^d \setminus \hpy(\ccint{0,1} \times \partial \Dset)$. Then
\begin{equation}
  \deg(f,\Dset,y) =  \deg(g,\Dset,y) \eqsp.
\end{equation}
\end{proposition}
We have now all the necessary results to prove \Cref{le:degree_application}.
\begin{proof}[Proof of \Cref{le:degree_application}]
Since $\hga(x,z) =  \bg z + \ga(x,z)$ and $\ga(x,\cdot)$ is Lipschitz with a Lipschitz constant which is uniformly bounded over the ball $\ball{0}{R}$,  $\hga_x$ is Lipschitz with bounded Lipschitz constant over this ball. Hence \Cref{assumG:irred_b}($\ra,0,\MassG$)-\ref{assumG:irred_b_item_i} holds.

  For all $x \in \rset^d$, denote by $\hga_x : z \mapsto \hga(x,z)$ where $\hga(x,z)=bz + g(x,z)$.
  Let $\MassG \in \rset_+^*$. We show that for all $x \in
  \ball{0}{\ra}$, $\ball{0}{\MassG} \subset
  \hga_x(\ball{0}{\tMassG})$, where $\tMassG$ is given by
  \eqref{eq:deftildeM}, which is precisely
  \Cref{assumG:irred_b}($\ra,0,\MassG$)-\ref{assumG:irred_b_item_ii}.

 % Then for all $z \in
%   (\hga_x)^{-1}(\ball{0}{\MassG}) $, by \ref{propo:irred_b_item_ii}
% \begin{equation}
%   \MassG \geq \abs{\hga_x(z)} \geq  \abs{ \bg z}  - \Cga_{\ra,0} -\Cga_{\ra,1} \abs{z} \eqsp.
% \end{equation}
% Therefore since $\abs{\bg} \geq \Cga_{\ra,1} $, $(\hga_x)^{-1}(\ball{0}{\MassG}) \subset \ball{0}{\tMassG}$ where $\tMassG$ is given by
%\eqref{eq:deftildeM}.
% $\tMassG = \{\MassG + \ra \abs{\ag} +
% \Cga(1+\abs{\ra})\}/(\abs{\bg}-\Cga)$.
%Next we show \ref{item:proof:homot_ii}.
%  Let $x \in \ball{0}{\ra}$ and
% $\MassG \geq 0$.
  Let $x \in \ball{0}{\ra}$ and consider the continuous homotopy $\hog : \ccint{0,1}
\times \rset^d$ between the functions $z \mapsto \bg z$ and $\hga_x$ defined for all
$t \in \ccint{0,1}$ and $z \in \rset^d$ by
\begin{equation}
  \hog(t,z) = t \bg z + (1-t)\hga_x(z) = \bg z + (1-t)  \ga(x,z)  \eqsp.
\end{equation}
Then by \ref{propo:irred_b_item_ii}, since $\abs{\bg} \geq \Cga_{\ra,1} $, for all $t\in \ccint{0,1}$ and $z \not \in
\ball{0}{\tMassG}$, where $\tMassG$ is given by \eqref{eq:deftildeM},
\begin{equation}
   \abs{\hog(t,z)} \geq  \abs{\bg z} -(1-t)\defEns{\Cga_{\ra,0} +\Cga_{\ra,1} \abs{z} } \geq \MassG \eqsp.
\end{equation}
In particular, we have $\hog(\ccint{0,1} \times \partial
\ball{0}{\tMassG}) \subset \rset^d \setminus \ball{0}{\MassG}$. Let
$z \in \ball{0}{\MassG}$, then by
%\cite[Proposition 2.4, Proposition-Definition 1.1, Chapter IV]{outerelo:ruiz:2009},
\Cref{theo:deg_modif} we have
\begin{equation}
  \deg(\hga_x,\ball{0}{\tMassG},z) = \deg(\bg \Id, \ball{0}{\tMassG},z) = 1 \eqsp.
\end{equation}
Besides, by \cite[Corollary 2.5, Chapter IV]{outerelo:ruiz:2009},
$\deg(\hga_x,\ball{0}{\tMassG},z) \not = 0$ implies that there exists
$y \in \ball{0}{\tMassG}$ such that $\hga_x(y) = z$. Finally \Cref{assumG:irred_b}($\ra,0,\MassG$)-\ref{assumG:irred_b_item_ii} follows since this
result holds for all $z \in \ball{0}{\MassG}$.
\end{proof}
% which implies that $\hga_x(\rset^d \setminus
% \ball{0}{\tilde{M}}) \subset \rset^d \setminus \ball{0}{M}$
% $(\hga_x)^{-1}(\ball{0}{M}) \subset \ball{0}{\tilde{M}}$.
% % Then \ref{application:irred_b_item_i} implies
% %   that \Cref{assumG:irred_b}($\ra$)-\ref{assumG:irred_b_item_i} holds. We now show that under

% We first show that for all $x \in \rset^d$, $\Psi_x$ is surjective from $\rset^d$ to $\rset^d$. Now a standard perturbation
% theorem from degree theory (see e.g. \alain{I think we can cite
%   Milnor}), applied to the continuous family of maps
% $\{\Phi_x(t,\cdot) \, : \, t \in \ccint{0,1} \}$, defined for all $t
% \in \ccint{0,1}$
% $$
% y \mapsto \Phi_{x}(t,y):=ay+ t(bx+g(x,y)) \eqsp,
% $$
% implies that $\Psi_{x}$ is surjective\alain{expliquer }.
%  Assume that $R>0$ is given. Our  assumptions imply clearly that there exists  $M>R/\abs{a}$ such that for all $x \in \ball{0}{R}$ and $y \not \in \ball{0}{M}$,
% $$
% \abs{ay}-\abs{bx+g(x,y)}\geq R \eqsp.
% $$
% Especially, this holds at the boundary $\{y \in \rset^d \, : \,
% \abs{y} = M \}$. Let $x \in \ball{0}{R}$. Therefore $\ball{0}{R}\subset
% \Psi_x(\ball{0}{M})$ for all $x \in \ball{0}{R}$. Let $L$ be the
% Lipschitz constant of $g$ on $B(0,R)\times B(0,M)$.
%  \Cref{le:simple} yields that for any $x\in B(0,R)$ and $\eventA \in \borelSet(\rset^d)$, $\eventA \subset \ball{0}{R}$
%  \begin{equation}
%    \lambda_{\Psi_x}(\eventA) \geq L^{-d}\Leb(\eventA) \eqsp,
%  \end{equation}
% and the proof follows.
%  the
% push-forward measure of the Lebesgue measure on $B(0,R_0)$ under $y\to
% H(x_0,y)$ has a lower density bound $c> 0$ on the ball $B(0,R)$, that
% is independent of $x_0$. As the Gaussian distribution has lower
% bounded density on $B(0,R_0)$ the claim follows.


%%% Local Variables:
%%% mode: latex
%%% TeX-master: "main"
%%% End:


%  Next we record a
%   well-known fact from geometric measure theory together with a proof
%   for the reader's convenience.  Let $\ouvert\subset\rset^d$ be an open
%   set and $f: \ouvert\to \rset^d$ be a measurable function such that there exist $z_0,y_0 \in \rset^d$ and $M, \tilde{M} \geq 0$ satisfying
%   \begin{equation}
%     \label{eq:condition_sigma_finite}
% f^{-1}(\ball{y_0}{M}) \subset \ball{z_0}{\tilde{M}} \eqsp.
%   \end{equation}
%  Define the measure $\lambda_{f}$ on $(\rset^d,\borelSet(\rset^d))$
%   by setting for any $\eventA \in \borelSet(\rset^d)$
% \begin{equation}
%   \label{eq:push_forward_mes}
% \lambda_f (\eventA):=\Leb(f^{-1}(\eventA \cap \ball{y_0}{M}))  \eqsp.
% \end{equation}
% Note by \eqref{eq:condition_sigma_finite} that $\lambda_f$ is  a finite measure. Therefore by the Lebesgue decomposition theorem (see
% \cite[Section 6.10]{rudin:1987}) there exist two non-negative measures
% $\lambda_f^{(\text{a})}, \lambda_f^{(\text{s})}$ on
% $(\rset^d,\borelSet(\rset^d))$, which are absolutely continuous and
% singular with respect to the Lebesgue measure on $\rset^d$
% respectively, such that $\lambda_f = \lambda_f^{(\text{a})} +
% \lambda_f^{(\text{s})}$.
% % Note that if for all compact set $\compact \subset \rset^d$, $f^{-1}(\compact)$ is
% % compact (\ie~$f$ is a proper function), then $\lambda_f$ is $\sigma$-finite.
% \begin{proposition}\label{le:simple}
%   Let $\ouvert\subset\rset^d$ be open and $f: \ouvert\to \rset^d$ be a Lipschitz function
%   satisfying \eqref{eq:condition_sigma_finite}. Let $\phi_f$ be the
%   density of $\lambda_f^{(\text{a})}$ with respect to the Lebesgue
%   measure on $\rset^d$. Then $\Leb$-almost everywhere, it holds
% $$
% \phi_f(y)\geq \1_{f(\ouvert) \cap \ball{y_0}{M}}(y) \norm{f}_{\Lip}^{-d}\eqsp.
% $$
% \end{proposition}
% \begin{proof}
%   % We only need to prove that for $\Leb$-almost all $y \in f(\ouvert)
%   % \cap \ball{y_0}{M}$, it holds $\phi_f(y)\geq
%   % \norm{f}_{\Lip}^{-d}$.
%  Denote by $L = \norm{f}_{\Lip}$. Let $y\in
%   f(\ouvert) \cap \ball{y_0}{M}$ and $\delta_0 >0$  such that $\ball{y}{\delta_0}
%   \subset \ball{y_0}{M}$. Let $z\in f^{-1}(\{y\})$. Since $f$ is
%   Lipschitz continuous, there exists $\delta_1 >0$ such that for all
%   $\delta \in \ccint{0,\delta_1}$, $\ball{z}{\delta/L}\in \ouvert$ and
%   $f(\ball{z}{\delta/L})\subset \ball{y}{\delta}$. Hence, for all
%   $\delta \in \ocint{0,\min(\delta_0,\delta_1)}$, we have
%   $$
% \lambda_f(\ball{y}{\delta})\geq
%   L^{-d}\Leb(\ball{z}{\delta}) =   L^{-d}\Leb(\ball{y}{\delta}) \eqsp.
% $$
%  The claim follows from the differentiation theorem
%   for measures, see \cite[Theorem 7.14]{rudin:1987}.
% \end{proof}


% \begin{corollary}
%   \label{coro:irred}
%   Let $x \in \rset^d$. Assume that $\hfunb_x$ is Lipschitz and there
%   exist $y_0,z_0 \in \rset^d$, $M,\tilde{M} \geq 0$ such that $\hfunb_x^{-1}(\ball{y_0}{M}) \subset \ball{z_0}{\tilde{M}}$.
% Then for all $\eventA \in \borelSet(\rset^d)$,
% \begin{equation}
%   \Pkerb(x,\eventA) \geq \norm{\hfunb_x}_{\Lip}^{-d} \inf_{z \in \ball{z_0}{\tilde{M}}} \defEns{\phib(z)} \Leb\defEns{\eventA \cap \hfunb_x(\rset^d) \cap \ball{y_0}{M}} \eqsp.
% \end{equation}
% \end{corollary}

% \begin{proof}
%   By definition, we have using that  $\hfunb_x^{-1}(\ball{y_0}{M}) \subset \ball{z_0}{\tilde{M}}$,
%   \begin{align}
% \nonumber
%       \Pkerb(x,\eventA)  &= \int_{\rset^d}\1_{\eventA}(\hfunb(x,z)) \phib(z) \rmd z  = \int_{\rset^d}\1_{\hfunb_x^{-1}(\eventA)}(z) \phib(z) \rmd z \\
% \label{eq:coro_leb_irred_1}
% &\geq  \inf_{z \in \ball{z_0}{\tilde{M}}} \defEns{\phib(z)} \Leb\defEns{\hfunb_x^{-1}(\eventA \cap \ball{y_0}{M})} \eqsp.
%   \end{align}
% Since by assumption $\hfunb_x$ is Lipschitz, we get using \Cref{le:simple} that
% \begin{equation}
% \label{eq:coro_leb_irred_2}
%  \Leb\defEns{\hfunb_x^{-1}(\eventA \cap \ball{y_0}{M})} \geq  \norm{\hfunb_x}_{\Lip}^{-d}  \Leb\defEns{\eventA \cap \hfunb_x(\rset^d) \cap \ball{y_0}{M}} \eqsp.
% \end{equation}
% Combining \eqref{eq:coro_leb_irred_1} and \eqref{eq:coro_leb_irred_2} concludes the proof.
% \end{proof}

%%% Local Variables:
%%% mode: latex
%%% TeX-master: "main"
%%% End:


\section{Proofs}
\label{sec:postponed-proofs}
 In the sequel, $C \geq 0$
  is a constant which can change from line to line but does not depend
  on $h$. Let $h >0$ and $T \in \nset^*$. Note that a simple induction (see \cite[Proposition 4.2]{livingstone:betancourt:byrne:girolami:2016})  implies that for all $(\q_0,\p_0) \in \rset^d
\times \rset^d$ and $k \in \{1,\ldots T\}$, the $k^{\text{th}}$
iteration of the leap-frog integration, $(q_k,p_k) = \Phiverlet[h][k](\q,\p)$, where $\Phiverlet[h][k]$ is defined by \eqref{eq:def_Phiverlet}, takes the form
\begin{align}
\label{eq:qk}
\q_k&=\q_0+kh\p_0-\frac{kh^2}{2} \nabla \F(\q_0)-h^2 \gperthmc[k](\q_0,\p_0)\\
\label{eq:pk}
\p_{k}&= \p_0-\frac{h}{2} \defEns{\nabla \F(\q_0)+\nabla \F \circ \Phiverletq[h][k] (\q_0,\p_0)}-h \sum_{i=1}^{k-1}\nabla \F \circ \Phiverletq[h][i](\q_0,\p_0)  \eqsp,
\end{align}
where  $\gperthmc[k] :\rset^d \times \rset^d \to \rset^d$ is given  for all $(\q,\p) \in \rset^d \times
\rset^d$ by
\begin{equation}
  \label{eq:def_gperthmc}
\gperthmc[k](\q,\p) =    \sum_{i=1}^{k-1}(k-i)\nabla \F \circ \Phiverletq[h][i](\q,\p) \eqsp.
\end{equation}

We prefaces the proofs of our main results by useful bounds  on
  the position and the momentum in the intermediate steps of the leap-frog integration.
% Under the regularity condition
% \Cref{assum:regOne}($\beta$), it is possible to derive  useful bounds  on
%  the position and the momentum in the intermediate steps of the leap-frog integration.


  \begin{lemma}
    \label{lem:bound_first_iterate_leapfrog_a}
  Assume \Cref{assum:regOne}$(\expozero)$-\ref{assum:regOne_a}. Then, for any $k \in \nsets$, $h \geq 0$, $(q_0,p_0) \in \rset^{2d}$ and $(x_0,v_0) \in \rset^{2d}$,
  \begin{align}
   & \norm{q_k-x_k} + \constzero^{-1/2} \norm{p_k-v_k} \\
    & \qquad \qquad \leq \defEns{1+h \constzero^{1/2} \vartheta_1(h \constzero^{1/2})}^{k} \defEns{\norm{q_0-x_0} + \constzero^{-1/2} \norm{p_0-v_0}} \eqsp,
  \end{align}
  where $(q_k,p_k)= \Phiverlet[h][k](q_0,p_0)$, $(x_k,v_k)= \Phiverlet[h][k](x_0,v_0)$ and  $\Phiverlet[h][k]$ and $\vartheta_1$ are defined by \eqref{eq:def_Phiverlet} and \eqref{eq:def_vartheta_1}, respectively.
\end{lemma}
\begin{proof}
  Note that it is sufficient to show the result for $k=1$ and to apply a straightforward induction.
  Let $h >0$, $(q_0,p_0) \in \rset^{2d}$ and $(x_0,v_0) \in \rset^{2d}$. Using \eqref{eq:qk}, the triangle inequality and   \Cref{assum:regOne}$(\expozero)$-\ref{assum:regOne_a}, we first obtain
  \begin{align}
    \nonumber
    \norm{q_1-x_1} & = \norm{q_0 -h^2 \nabla U(q_0) /2 + h p_0 - \defEns{x_0 - h^2/2 \nabla U(x_0) + h v_0}} \\
    \label{eq:bound_q_1_x_1}
         & \leq (1+ h^2 \constzero /2) \norm{q_0 - x_0} + h \norm{p_0 - v_0} \eqsp.
  \end{align}
  Second, similarly using \eqref{eq:pk}, we have that
\begin{align}
\label{eq:bound_p_1_v_1}
&    \norm{p_1-v_1}  \\
&= \norm{p_0-v_0 -(h/2) \defEns{\nabla U(q_1) + \nabla U(q_0)} + (h/2) \defEns{\nabla U(x_1) + \nabla U(x_0)}} \\
\nonumber
&\leq \norm{p_0 - v_0} + (h \constzero/2) \defEns{ \norm{x_1- q_1} + \norm{x_0 - q_0}} \\
\nonumber
& \leq \left(1+h^2 \constzero/2\right) \norm{p_0-v_0} +h\constzero(1+h^2\constzero /4) \norm{q_0-x_0} \eqsp,
\end{align}
  where we have used \eqref{eq:bound_q_1_x_1} for the last inequality. Summing up \eqref{eq:bound_q_1_x_1} and \eqref{eq:bound_p_1_v_1}, we get the desired result for $k=1$.
  \end{proof}
  %\label{lem:bound_first_iterate_leapfrog_1} devient ...
  \begin{lemma}
  \label{lem:bound_first_iterate_leapfrog_b}
Let $\beta \in \ccint{0,1}$ and assume \Cref{assum:regOne}$(\expozero)$-\ref{assum:regOne_b}.
  \begin{enumerate}[label=(\roman*)]
\item
\label{lem:bound_first_iterate_leapfrog_1}
% \alain{ne suppose que \eqref{eq:bound_nabla_F_assum_reg_zero}}
For any $h_0 >0$, $T \in \nsets$, there exists $C < \infty$ (which depends only on $T,h_0$
 and $\constzeroT$) such that for all $h \in \ocint{0,h_0}$,
  $(\q_0,\p_0) \in \rset^d \times \rset^d$ and $k \in \{1,\ldots, T\}$
  \begin{align}
\label{lem:bound_first_iterate_leapfrog_1_q}
    \norm{\q_k-\q_0} &\leq C h\defEns{  \norm{\p_0} +h(1+ \norm{\q_0}^{\expozero})}\\
\label{lem:bound_first_iterate_leapfrog_1_p}
    \norm{\p_k-\p_0}& \leq C h\defEns{  1+ \norm{\p_0}^{\expozero}+ \norm{\q_0}^{\expozero}} \eqsp,
  \end{align}
where $(\q_k,\p_k) = \Phiverlet^{\circ k}_{h}(\q_0,\p_0)$ and  $\Phiverlet^{\circ k}_{h}$ is defined by \eqref{eq:def_Phiverlet}.
\item \label{lem:bound_first_iterate_leapfrog_b_2}
  If in addition \Cref{assum:regOne}$(\expozero)$-\ref{assum:regOne_a} holds, for any $k \in \nsets$, $h >0$, $(q_0,p_0) \in \rset^{2d}$,
  \begin{align}
    &    \norm{q_k-q_0} + \constzero^{-1/2}\norm{p_k - p_0} \leq (\constzero^{1/2}\vartheta_1(h \constzero^{1/2}))^{-1} \\
    & \quad \times \defEns{(1+h \constzero^{1/2}\vartheta_1(h \constzero^{1/2}))^{k+1} - 1} \defEns{\vartheta_2(h) (\norm[\beta]{q_0} +1) + \vartheta_3(h) \norm{p_0}} \eqsp,
  \end{align}
  where $\vartheta_1$ is defined by \eqref{eq:def_vartheta_1} and
  \begin{align}
  \label{eq:definition-vartheta-2}
    \vartheta_2(h) &= \constzeroT/\constzero^{1/2} + \constzeroT h/2 + \constzero^{1/2}\constzeroT h^2/4 \eqsp, \\
  \label{eq:definition-vartheta-3}
    \vartheta_3(h) &= 1+\constzero^{1/2} h /2 \eqsp.
  \end{align}
  \end{enumerate}
\end{lemma}
\begin{proof}
  \begin{enumerate}[label={(\roman*)},wide=0pt, labelindent=\parindent]
% \item We show by induction that for all $k \in \{1,\cdots,T\}$, there
%   exists $C_k$ (which depends only on $T,h_0$ and  $\constzero$) such that for all $ h \in \ocint{0, h_0}$, $(q_0,p_0)
%   \in \rset^d \times \rset^d$ and $(\tilde{q}_0,\tilde{p}_0) \in
%   \rset^d \times \rset^d$,
%   \begin{equation}
%     \norm{q_k-\tilde{q}_k} \leq C_k (\norm{q_0-\tilde{q}_0} + \norm{p_0-\tilde{p}_0}) \eqsp,
%   \end{equation}
%   where $(q_k,p_k) = \Phiverlet^{\circ k}_{h}(q_0,p_0)$,
%   $(\tilde{q}_k,\tilde{p}_k) = \Phiverlet^{\circ
%     k}_{h}(\tilde{q}_0,\tilde{p}_0)$.  Let $ h \in \ocint{0, h_0}$,
%   $(q_0,p_0) \in \rset^d \times \rset^d$ and
%   $(\tilde{q}_0,\tilde{p}_0) \in \rset^d \times \rset^d$.  The case $k=1$ is immediate
%   by \Cref{assum:regOne}($\beta$)-\ref{assum:regOne_a}. Let $k \in \{1,\cdots, T-1\}$ and assume
%   that the inequality holds for all $i \in \{1,\cdots, k\}$. Then by
%   \eqref{eq:qk} and \Cref{assum:regOne}($\beta$)-\ref{assum:regOne_a} we get
% \begin{align}
%   &     \norm{q_{k+1}-\tilde{q}_{k+1}} \leq \norm{q_0- \tilde{q}_0}+(k+1)h\norm{p_0-\tilde{p}_0}+(1/2)(k+1)h^2 \norm{\nabla \F(q_0)-\nabla \F(\tilde{q}_0)}\\
%   & \phantom{\norm{q_{k+1}-\tilde{q}_{k+1}}}\phantom{\leq \norm{q_0- \tilde{q}_0}+(k+1)haaa} +h^2\sum_{i=1}^{k}(k+1-i)\norm{\nabla \F(q_i) - \nabla \F(\tilde{q}_i)} \\
% & \phantom{\norm{q_{k+1}-\tilde{q}_{k+1}}}\leq \norm{q_0- \tilde{q}_0}+(k+1)h\norm{p_0-\tilde{p}_0}+2^{-1}(k+1) h^2  \constzero \norm{q_0-\tilde{q}_0}\\
% & \phantom{\norm{q_{k+1}-\tilde{q}_{k+1}}}\phantom{\leq \norm{q_0- \tilde{q}_0}+(k+1)haaa}+ h^2\constzero \sum_{i=1}^{k}(k+1-i)\norm{q_i -\tilde{q}_i} \eqsp.
%    \end{align}
% An application of the induction hypothesis concludes the proof.
  \item Let $T \in \nsets$ and $h_0 >0$.  We prove by induction that for all $k \in \{1,\ldots,T\}$ there exists $C_k \geq 0$ (which depends only on $T,h_0$
and $\constzeroT$) such that for all $h \in \ocint{0,h_0}$ and
  $(q_0,p_0) \in \rset^d \times \rset^d$
\begin{equation}
\label{lem:bound_first_iterate_leapfrog_1_q}
\begin{aligned}
\norm{q_k-q_0} \leq C_k h\defEns{  \norm{p_0} +h(1+ \norm{q_0}^{\expozero})} \\
\norm{p_k-p_0} \leq C_k h \defEns{ 1 + \norm{p_0}^{\expozero} + \norm{q_0}^{\expozero}} \eqsp.
\end{aligned}
\end{equation}
  where $(q_k,p_k) = \Phiverlet^{\circ k}_{h}(q_0,p_0)$.
 Let $ h \in \ocint{0, h_0}$ and $(q_0,p_0) \in \rset^d \times
    \rset^d$.
 The case $k=1$ is immediate by
\Cref{assum:regOne}($\beta$)-\ref{assum:regOne_b} and \eqref{eq:qk}. Let $k \in \{1,\cdots,
T-1\}$ and assume that the inequalities hold for all $i \in
\{1,\dots, k\}$. Then by \eqref{eq:qk} and
\Cref{assum:regOne}($\beta$)-\ref{assum:regOne_b}, we get
\begin{align}
\label{eq:lem:bound_first_iterate_leapfrog_1}
\norm{q_{k+1}-q_0}
&\leq (k+1)h \norm{p_0}+\frac{k+1}{2}h^2 \constzeroT  \defEns{1+\norm{ q_0}^{\expozero} }\\
&\qquad \qquad +h^2 \constzeroT \sum_{i=1}^{k}(k+1-i)\defEns{1+ \norm{q_i}^{\expozero} }\eqsp.
\end{align}
By the induction hypothesis and using that $t \mapsto t^{\expozero}$ is sub-additive on $\rset^+$ and $t^\beta \leq 1 + t$ for $t \in \rset^+$, we get for all $i \in \{1,\cdots, k\}$,
\begin{equation}
\norm{q_i}^{\expozero} \leq \norm{q_0}^{\expozero} + \norm{q_i-q_0}^\beta \leq 1+ \norm{q_0}^{\expozero} +C_i h\defEns{\norm{p_0}+h(1+ \norm{q_0}^{\expozero})} \eqsp,
\end{equation}
Plugging this inequality in \eqref{eq:lem:bound_first_iterate_leapfrog_1}
conclude the proof of \eqref{lem:bound_first_iterate_leapfrog_1_q}.
Consider now \eqref{lem:bound_first_iterate_leapfrog_1_p}. Since by
definition $p_{k+1} = p_k-(h/2)\defEnsLigne{\nabla \F(q_{k}) + \nabla
  \F(q_{k+1})}$, using the triangle inequality,
\Cref{assum:regOne}($\beta$)-\ref{assum:regOne_b},
\eqref{lem:bound_first_iterate_leapfrog_1_q} to bound $\norm{q_{k}}$ and
$\norm{q_{k+1}}$, and the induction hypothesis, we get that there exist some
constants $C_{k+1,1},C_{k+1,2}$ which only depend on $T,h_0$ and
$\constzeroT$ such that
\begin{align}
&  \norm{p_{k+1} - p_0} \leq \norm{p_k-p_0} + (h/2)\defEnsLigne{\norm{\nabla \F(q_{k})} + \norm{\nabla \F(q_{k+1})}} \\
&\quad \leq C_{k+1,1}h\defEns{1+\norm{p_0}^{\expozero}+\norm{q_0}^{\expozero}} + (\constzeroT h/2)\defEns{2+\norm{q_{k}}^{\expozero}+ \norm{q_{k+1}}^{\expozero}}\\
&\quad \leq C_{k+1,1}h\defEns{1+\norm{p_0}^{\expozero}+\norm{q_0}^{\expozero}} + (C_{k+1,2}h/2)\defEns{1+ \norm{q_0}^{\expozero} + \norm{p_0}^{\expozero}} \eqsp.
\end{align}
Therefore, \eqref{lem:bound_first_iterate_leapfrog_1_q} is satisfied which concludes the induction and the proof.
\item Let $k \in \nset$, $h >0$ and $(q_0, p_0) \in \rset^{2d}$. Using \eqref{eq:qk}, the triangle inequality and \Cref{assum:regOne}($\beta$), we have
  \begin{align}
\label{eq:1_lem_bound_first_iterate_leapfrog_b_2}
&    \norm{q_{k+1} - q_0}  = \norm{q_k - q_0 - (h^2/2) \nabla U(q_k) + h p_k} \\
\nonumber
&                          \leq (1+h^2 \constzero/2) \norm{q_k - q_0} + (h^2\constzeroT/2) (\norm[\beta]{q_0} +1) + h \norm{p_k-p_0} + h \norm{p_0}  \eqsp.
  \end{align}
Second, similarly using \eqref{eq:pk}, we get that
\begin{align}
\nonumber
&    \norm{p_{k+1} - p_0}   \leq  \norm{p_k-p_0 + (h/2) \defEns{\nabla U(q_{k+1}) + \nabla U(q_k)} } \\
    \nonumber
                         & \leq \norm{p_k - p_0} + (h \constzero/2) \defEns{ \norm{q_{k+1}- q_0} + \norm{q_k - q_0}} + h \constzeroT \defEns{\norm[\beta]{q_0} +1} \\
    \nonumber
                         & \leq \norm{p_k - p_0}  +  h \constzeroT \defEns{\norm[\beta]{q_0} +1} +(h \constzero/2) \defEns{  + h \norm{p_k-p_0} + h \norm{p_0}} \\
    \label{eq:2_lem_bound_first_iterate_leapfrog_b_2}
&     +(h \constzero/2) \defEns{ (2+\constzero h^2/2) \norm{q_k-q_0} + (h^2\constzeroT/2) (\norm[\beta]{q_0}+1)}  \eqsp.
  \end{align}
  where we have used \eqref{eq:1_lem_bound_first_iterate_leapfrog_b_2} for the last inequality.
  Summing up \eqref{eq:1_lem_bound_first_iterate_leapfrog_b_2} and \eqref{eq:2_lem_bound_first_iterate_leapfrog_b_2} and using the definition \eqref{eq:def_vartheta_1} of $\vartheta_1(h)$, we get that, setting $A_k= \norm{q_k-q_0}  + \constzero^{-1/2} \norm{p_k-p_0}$,
  \[
  A_{k+1} \leq (1+h\constzero^{1/2}\vartheta_1(h\constzero^{1/2})) A_k    + h \defEns{ \vartheta_2(h) (\norm[\beta]{q_0} +1) +  \vartheta_3(h) \norm{p_0}} \eqsp.
\]
  By a straightforward induction, we obtain that
  \begin{equation}
    A_{k+1} \leq \sum_{i=1}^{k+1} \parentheseDeux{(1+h \constzero^{1/2} \vartheta_1(h \constzero^{1/2} ))^{k+1-i} h \defEns{ \vartheta_2(h) (\norm[\beta]{q_0} +1) +  \vartheta_3(h) \norm{p_0} }} \eqsp,
  \end{equation}
  which completes the proof of \ref{lem:bound_first_iterate_leapfrog_b_2}.
\end{enumerate}
\end{proof}
% \begin{proof}
%   The proof is postponed to \Cref{sec:proof-crefl-lem:bound_first_iterate_leapfrog}.
% \end{proof}

\begin{lemma}
\label{lem:inverse_1}
Let $\beta \in \ccint{0,1}$ and assume \Cref{assum:regOne}($\beta)$. Then for any $T \in \nsets$, $h >0$,
\begin{align}
\label{eq:lem:inverse_1}
  &  \sup_{(\q,\p,v) \in \rset^{3d}} \{ \norm{ \gperthmc[T](q,\p) - \gperthmc[T](q,v)} / \norm{\p-v} \} \\
  & \qquad \qquad \qquad \qquad  \leq   (T/h)  \left\{ (1+\ h \constzero^{1/2} \vartheta_1(h \constzero^{1/2}))^T - 1 \right\} \eqsp,
\end{align}
where  $\gperthmc[T]$ and $\vartheta_1$ are defined by  \eqref{eq:def_gperthmc} and \eqref{eq:def_vartheta_1} respectively.
  In addition, for any $q \in \rset^d$,
  \begin{equation}
    \label{eq:lem:inverse_2}
    \norm{\gperthmc[T](q,0)} \leq  (T/h)  \{ (1+h \constzero^{1/2} \vartheta_1(h \constzero^{1/2} ))^{T}-1\}  \vartheta_2(h) (\norm[\beta]{q} +1)  + T^2 \norm{\nabla U(q)} \eqsp,
  \end{equation}
  where $\vartheta_2$ is defined in \eqref{eq:definition-vartheta-2}.
\end{lemma}
\begin{proof}
By
\Cref{lem:bound_first_iterate_leapfrog_a}, for any $i \in \nsets$, we get
\[
\sup_{(q,\p,v) \in \rset^{3d}} \defEns{ \normLigne{ \Phiverletq[h][i](q,\p) - \Phiverletq[h][i](q,v)} / \norm{\p-v} }
\leq \constzero^{-1/2} A^i
\]
where $A=(1+h \constzero^{1/2} \vartheta_1(h \constzero^{1/2} ))$.
Therefore by definition of $\gperthmc[T]$ \eqref{eq:def_gperthmc} and using  \Cref{assum:regOne}($\beta$), for any $h >0$, $T \in \nsets$, we get that
\begin{align}
&  \sup_{(\q,\p,v) \in \rset^{3d}} \{ \norm{ \gperthmc[T](q,\p) - \gperthmc[T](q,v)} / \norm{\p-v} \} \\
&\qquad  \leq \constzero   \sum_{k=1}^{T-1} (T-i) \sup_{(q,\p,v) \in \rset^{3d}} \defEns{ \normLigne{ \Phiverletq[h][i](q,\p) - \Phiverletq[h][i](q,v)} / \norm{\p-v} }\\
%& \qquad  \leq  \constzero^{1/2} \sum_{k=1}^{T-1} (T-i) \{1+h \constzero^{1/2} \vartheta_1(h \constzero^{1/2})\}^i  \\
& \qquad \leq    T \parentheseDeux{ A^{T} - 1 } /(h \vartheta_1(\constzero^{1/2} h)) \,
\end{align}
showing \eqref{eq:lem:inverse_1} since $\vartheta_1(h\constzero^{1/2}) \geq 1$.

We now consider \eqref{eq:lem:inverse_2}. By \eqref{eq:def_gperthmc}, \Cref{assum:regOne}($\beta$)-\ref{assum:regOne_a} and \Cref{lem:bound_first_iterate_leapfrog_b}-\ref{lem:bound_first_iterate_leapfrog_b_2}, we have that for any $q \in \rset^d$,
\begin{align}
&  \norm{\gperthmc[T](q,0)}  \leq \sum_{i=1}^{T-1} (T-i)  \norm{ \nabla U \circ \Phiverletq[h][i](\q,0) - \nabla U(q)} + T^2 \norm{\nabla U(q)} \\
& \leq  T \constzero \sum_{i=1}^{T-1}  \norm{ \Phiverletq[h][i](\q,0) -q} + T^2 \norm{\nabla U(q)} \\
& \quad \leq T \constzero^{1/2} \vartheta_1^{-1} (h \constzero^{1/2}) \sum_{i=1}^{T-1}\{A^{i+1}-1\} \defEns{\vartheta_2(h) (\norm[\beta]{q} +1)}  + T^2 \norm{\nabla U(q)} \eqsp,
\end{align}
which completes the proof of \eqref{eq:lem:inverse_2} using that $\vartheta_1(h \constzero^{1/2}) \geq 1$.
\end{proof}

\begin{lemma}
  \label{lem:bounded_cum_error}
  Assume \Cref{assum:regOne}$(1)$. Then for any $T \in \nsets$, $h >0$, and $q,p \in \rset^d$,
  \begin{align}
    %\label{eq:3}
    &\sum_{i=1}^{T}  \norm{ \Phiverletq[h][i](\q,p) -q} \\
    &\leq  L^{-1/2}_1 T [\{1 + h  \constzero^{1/2} \vartheta_1(h\constzero^{1/2})\}^T  -1] \defEns{\vartheta_2(h) (1+\norm{q})  + \vartheta_3(h) \norm{p}} \eqsp,
  \end{align}
where  $\vartheta_1$ is defined by  \eqref{eq:def_vartheta_1}.
\end{lemma}
\begin{proof}
For  $T \geq 2$ and $h >0$,  by  \Cref{lem:bound_first_iterate_leapfrog_b}-\ref{lem:bound_first_iterate_leapfrog_b_2},   we have
\begin{align}
\sum_{i=1}^{T-1}  \norm{ \Phiverletq[h][i](\q,p) -q}
&\leq 
%\constzero^{-1/2} \vartheta_1^{-1}(h \constzero^{1/2})  \sum_{i=1}^{T-1}\parentheseDeux{\defEns{(1+h\vartheta_1(h))^{i+1}-1} \defEns{\vartheta_2(h) (\norm{q} +1) + \vartheta_3(h) \norm{p}}} \\
 \constzero^{-1/2} \vartheta_1^{-1}(h \constzero^{1/2}) T \defEns{\{1+h \constzero^{1/2}\vartheta_1(h\constzero^{1/2})\}^{T}-1} \\
 & \qquad \qquad \times \, \defEns{\vartheta_2(h) (\norm{q} +1) + \vartheta_3(h) \norm{p}} \eqsp.
\end{align}
The proof is completed upon using that $\vartheta_1(h \constzero^{1/2}) \geq 1$.
\end{proof}


%%% Local Variables:
%%% mode: latex
%%% TeX-master: "main"
%%% End:

 %(assuming that $h \leq h_0$ for $h_0 >0$). \alain{To discuss}
\subsection{Proofs of \Cref{sec:ergodicity-hmc}}


% \begin{proof}
%\end{proof}

\subsubsection{Proof of \Cref{theo:irred_harris} }
\label{sec:proof-crefth-harris_0}
We first prove  \eqref{theo:irred_harris_a}.  Under the assumption that $\F$ is twice continuously
  differentiable, it follows by a straightforward induction, that for
  all $h >0$ and $q \in \rset^d$, $p \mapsto
  \Phiverletq[h][k](q,p)$, defined by  \eqref{eq:def_Phiverletq}, and $p \mapsto \gperthmc[k](q,p)$, defined by \eqref{eq:def_gperthmc}, are
  continuously differentiable and for all $(q,p) \in \rset^d \times
  \rset^d$,
\begin{equation}
  \Jac_{p,\gperthmc[T]}(q,p) =  \sum_{i=1}^{T-1}(T-i)\defEns{\nabla^2 \F \circ \Phiverletq[h][i](q,p)} \Jac_{p,\Phiverletq[h][i]}(q,p) \eqsp,
\end{equation}
where for all $q \in \rset^d$, $\Jac_{p,\gperthmc[k]}(q,p)$ ($\Jac_{p,\Phiverletq[i][h]}(q,p)$ respectively) is the Jacobian of the function $\tilde{p} \mapsto
\gperthmc[k](q,\tilde{p})$ ($\tilde{p} \mapsto
\Phiverletq[i][h](q,\tilde{p})$ respectively) at $p \in \rset^{d}$.


%  Under \Cref{assum:regOne}, $\sup_{x \in \rset^d} \normLigne{\nabla^2 \F(x)}
% \leq \constzero$ and by
% \Cref{lem:bound_first_iterate_leapfrog_a},
%  $ \sup_{(q,p) \in \rset^d \times \rset^d} \normLigne{\nabla_p \Phiverletq[h][i](q,p)} \leq (1+h \vartheta_1(h))^i$ for any $i \in \nsets$.
% Therefore for any $h >0$, $T \in \nsets$, setting $S = hT$ and using that $\tilde{h} \mapsto \vartheta_1(\tilde{h})$ is nondecreasing and greater than $1$ on $\rset_+^*$ and for any $u,s \geq 0$, $u \geq 1$, $(1+s/u)^{u-1} \leq \log(s+1) \rme^s$, we get that
Under \Cref{assum:regOne}, $\sup_{x \in \rset^d} \normLigne{\nabla^2 \F(x)}
 \leq \constzero$, therefore by \Cref{lem:inverse_1}, we have that for any $T \in \nsets$ and $h >0$,
\begin{equation}
  \label{eq:inverse_1}
 \sup_{(\q,\p) \in \rset^d \times \rset^d} \norm{\Jac_{p,\gperthmc[T]}(q,p)}
 \leq  T (\{1 + h \constzero^{1/2} \vartheta_1(h \constzero^{1/2})\}^T  -1) /h  \eqsp.
\end{equation}
%$\sup_{p \in \rset^d } \nabla_p \gperthmc[k](q,p) \leq C$.
% \begin{equation}
% \label{lem:inverse_1}
% \sup_{p \in \rset^d } \nabla_p \gperthmc[k](q,p) \leq C \eqsp.
% \end{equation}
% Then for all $q, p_1,p_2 \in \rset^d$,
% \begin{equation}
% \label{lem:inverse_1}
% \norm{\gperthmc[k](q,p_1) -\gperthmc[k](q,p_2)} \leq C \norm{p_1 - p_2} \eqsp.
% \end{equation}
For any $q \in \rset^d$, $T\in \nsets$ and $h >0$, define $\phia_{q,T,h}(p)$  for all  $p \in \rset^d$ by
\begin{equation}
  \phia_{q,T,h} (p) = p-(h/T) \gperthmc[T](q,p) \eqsp.
\end{equation}
It is a well known fact (see for example
\cite[Exercise 3.26]{duistermaat:kolk:2004}) that if
\begin{equation}
  \label{eq:inverse_1_2}
  \sup_{(q,p) \in \rset^d \times \rset^d} (h/T)\norm{ \Jac_{p,\gperthmc[T]}(q,p)} < 1 \eqsp,
\end{equation}
then for any $q \in \rset^d$, $\phia_{q,T,h}$ is a
diffeomorphism and  therefore by \eqref{eq:qk}, the same conclusion holds
for $p \mapsto \Phiverletq[h][T](q,p)$. Using \eqref{eq:inverse_1}, if $T \in \nsets$ and $h > 0$  satisfies \eqref{eq:condition-h,T-harris},
then the condition \eqref{eq:inverse_1_2} is verified and as a result \eqref{theo:irred_harris_a}.

Denoting for any $q \in \rset^d$ by $\Phiverletqi[h][T](q,\cdot) : \rset^d \to \rset$ the
continuously differentiable inverse of $p \mapsto
\Phiverletq[h][T](q,p)$ and using a change of variable with $\Phiverletqi[h][T](q,\cdot)$ in \eqref{eq:def_kernel_hmc} concludes the proof of \eqref{eq:def_kernel_hmc_false_density}.

We now show that $\Tker_{h,T}$ satisfies the condition which implies that $\Pkerhmc[h][T]$ is a \Tkernel. We first establish some estimates on the function $(q,p) \mapsto \Phiverletqi[h][T](q,p)$. By
\eqref{eq:inverse_1_2} and \eqref{eq:qk}, for any $q,p,v \in \rset^d$, there exists $\varepsilon \in \ooint{0,1}$ such that $  \normLigne{\Phiverletq[h][T](q,p)-\Phiverletq[h][T](q,v)} \geq (hT) \normLigne{\phi_{q,T,h}(p)-\phi_{q,T,h}(v)} \geq (hT) (1-\varepsilon)\norm{p-v}$ which implies that that there exists $C \geq 0$ satisfying
\begin{equation}
  \label{eq:regularity_phinverse1}
  \begin{aligned}
    \norm{\Phiverletqi[h][T](q,p)-\Phiverletqi[h][T](q,v)} &\leq (1-\varepsilon)^{-1} \norm{v-p}\eqsp, \\
    \norm{  \Phiverletqi[h][T](q,p)} &\leq C\defEns{\norm{\p} + \norm{\Phiverletq[h][T](q,0)}} \eqsp.
  \end{aligned}
\end{equation}
In addition, for $q,x,p \in \rset^d$, we have setting $\tilde{q} = \Phiverletqi[h][T](q,p)$ that
\begin{align}
  \nonumber
  \normLigne{\Phiverletqi[h][T](q,p) - \Phiverletqi[h][T](x,p)} &= \normLigne{\tilde{q} - \Phiverletqi[h][T](x, \Phiverletq[h][T](q,\tilde{q}))} \\
  \nonumber
                                                                &= \normLigne{\Phiverletqi[h][T](x, \Phiverletq[h][T](x,\tilde{q})) - \Phiverletqi[h][T](x, \Phiverletq[h][T](q,\tilde{q}))} \eqsp,
\end{align}
which implies by \eqref{eq:regularity_phinverse1} and \Cref{lem:bound_first_iterate_leapfrog_a}
that there exists $C \geq 0$ satisfying
\begin{equation}
  \label{eq:regularity_phinverse2}
  \norm{\Phiverletqi[h][T](q,p) - \Phiverletqi[h][T](x,p)} \leq C \norm{q-x} \eqsp.
\end{equation}


We now can prove that $\Tker_{h,T}$ is the continuous component of $\Pkerhmc[h][T]$. First by \eqref{eq:def_tker}, for all $\eventB \in \borelSet(\rset^d)$,
\begin{equation}
\label{eq:minoration_pseudo_density_P}
    \Tker_{h,T}(q, \eventB) \geq (2 \uppi)^{-d/2} \Leb(\eventB)
 \times \inf_{\bar{q} \in \eventB} \defEns{ \balphaacc(q,\bar{q}) \rme^{-\norm{\Phiverletqi_q(\bar{q})}^2/2}\detj_{\Phiverletqi[h][T](q,\cdot)}(\bar{q})} \eqsp,
\end{equation}
with the convention $0 \times \plusinfty = 0$ and
\begin{equation}
%  \label{eq:7}
  \balphaacc(q,\bar{q}) =  \alphaacc\defEns{(q,\Phiverletqi[h][T](q,\bar{q})),\Phiverlet[h][T](q,\Phiverletqi[h][T](q,\bar{q}))}\eqsp. 
\end{equation}
Since the function $  (q,p) \mapsto (\Phiverletq[h][T](q,p),\Phiverletqi[h][T](q,p), \detj_{\Phiverletqi[h][T](q,\cdot)}(p)) $
is continuous on $\rset^d\times \rset^d$ by \Cref{lem:bound_first_iterate_leapfrog_a}, \eqref{eq:regularity_phinverse1} and \eqref{eq:regularity_phinverse2}, and for any $q,p \in \rset^d$, $\Jac_{\Phiverletq[h][T](\q,\cdot)}(\Phiverletqi[h][T](q,p))
\Jac_{\Phiverletqi[h][t](q,\cdot)}(\p) = \operatorname{I}_n$, we get that  $\Tker_{h,T}(q,\eventB) >0$ for all $q \in \rset^d$ and all compact set $\eventB$ satisfying $\Leb(\eventB) > 0$. Therefore, using that the Lebesgue measure is regular which implies that for any $\msa \in \mcb(\rset^d)$ with $\Leb(\msa) >0$, there exists a compact set $\msb \subset\msa$, $\Leb(\msb)>0$, we can conclude that $\Pkerhmc[h][T]$ is irreducible with respect to the Lebesgue measure. In addition, we get  $\Tker_{h,T}(q,\rset^d) >0$, and therefore we obtain that $\Pkerhmc[h][T]$ is aperiodic.  Similarly we get that any compact set is $(1,\Leb)$-small.

It remains to show that for any $\eventB \in\mcb(\rset^d)$, $q \mapsto \Tker_{h,T}(q,\eventB)$ is lower semi-continuous which is a straightforward consequence of Fatou's Lemma and that for any $p \in \rset^d$, $q \mapsto (\Phiverlet[h][T](q,p), \Phiverletqi[h][T](q,p),\detj_{\Phiverletqi[h][T](q,\cdot)}(p))$ is continuous.

% We now show that all the compact sets are $(1,\Leb)$-small. Let $\eventB \subset \rset^d$ be compact.  Using
% \eqref{eq:inverse_1_2} there exists $C \geq 0$ such that for all
% $q,p,v \in \rset^d$, $ \norm{p-v} \leq C \normLigne{
%   \Phiverletq[h][T](q,p)- \Phiverletq[h][T](q,v)}$. It follows
% that for all $p \in \rset^d$, $\sup_{q \in \eventB} \normLigne{
%   \Phiverletqi[h][T](q,p)} \leq C\defEnsLigne{\norm{\p} + \sup_{q \in
%     \eventB} \normLigne{\Phiverletq[h][T](q,0)}}$. Using this upper
% bound and $\Jac_{\Phiverletq(\q,\cdot)}(\Phiverletqi_q(\p))
% \Jac_{\Phiverletqi_q}(\tilde{\p}) = \operatorname{I}_n$ in
% \eqref{eq:minoration_pseudo_density_P}, where $\operatorname{I}_n$ is
% the identity matrix, we deduce that there exists $\varepsilon >0$ such
% that for all $\eventA\in \borelSet(\rset^d)$, $\eventA \subset \eventB$,
% \begin{equation}
%   \inf_{q \in \eventB} \Pkerhmc[h][T](q, \eventA)  \geq \varepsilon \Leb(\eventA) \eqsp,
% \end{equation}
% and therefore $\eventB$ is small for $\Pkerhmc[h][T]$.
% \begin{equation}
% \norm{  \Phiverletqi_q(p_1)-  \Phiverletqi_q(p_2)} \leq C \norm{p_1-p_2} \eqsp,
% \end{equation}
% This result, \eqref{eq:def_acc_ratio}, \Cref{lem:bound_first_iterate_leapfrog} and \eqref{eq:minoration_pseudo_density_P} imply that $
% \Pkerhmc[h][T]$ is irreducible with respect to the Lebesgue measure
% and aperiodic.
% and any ball on $\rset^d$ is small.

% A straightforward adaptation of the proof of \cite[Corollary
% 2]{tierney:1994} shows that $ \Pkerhmc[h][T]$ is Harris recurrent, see \Cref{propo:harris_rec} in \Cref{sec:harr-recurr-metr}. The desired conclusion then follows from \cite[Theorem 13.0.1]{meyn:tweedie:2009}.
 % \Cref{theo:irred_harris} implies
% that for all $T \geq 0$, there exists $\hirr>0$ such that for all $h \in \ocintLigne{0,\hirr}$ and all $\q \in \rset^d$
%   \begin{equation}
% \lim_{n \to \plusinfty}    \tvnorm{\delta_\q \Pkerhmc[h][T]^n - \pi} = 0 \eqsp.
%   \end{equation}


% By \cite[Theorem 17.1.4, Theorem
% 17.1.7]{meyn:tweedie:2009}, it suffices the to prove that for all
% bounded harmonic function $\harmonic : \rset^d \to \rset$ satisfying
% \begin{equation}
%   \label{eq:def_harm}
%   \Pkerhmc[h][T]\harmonic = \harmonic \eqsp,
% \end{equation}
% %$\Pkerhmc[h][T]\harmonic = \harmonic$,
% are constant. First since $\Pkerhmc[h][T]$ is irreducible with respect
% to the Lebesgue measure and aperiodic, by \cite[Theorem
% 14.0.1]{meyn:tweedie:2009} for $\Leb$-almost all $q$ we get $\lim_{n
%   \to \plusinfty} \Pkerhmc[h][T]^n \harmonic(q) = \pi(\harmonic)$ and therefore by
% \eqref{eq:def_harm} $\harmonic(q) = \pi(\harmonic)$. Therefore we get that for all $q \in \rset^d$ by \eqref{eq:def_kernel_hmc_false_density},
% \begin{multline}
%    \Pkerhmc[h][T]^n \harmonic(q) = \pi(\harmonic)  \int_{\rset^d}  \alphaacc\defEns{(q,\tilde{p}),\Phiverlet[h][T](q,\tilde{p})} \rme^{-\norm{\tilde{p}}^2/2} \rmd \tilde{p} \\
% +   \harmonic(x) \int_{\rset^d}  \parentheseDeux{1-\alphaacc\defEns{(q,\tilde{p}),\Phiverlet[h][T](q,\tilde{p})}} \rme^{-\norm{\tilde{p}}^2/2} \rmd \tilde{p} \eqsp.
% \end{multline}
% Combining this result with \eqref{eq:def_harm}, we get for all $q \in \rset^d$
% \begin{equation}
% (\harmonic(q)-\pi(\harmonic)) \int_{\rset^d} \alphaacc\defEns{(q,\tilde{p}),\Phiverlet[h][T](q,\tilde{p})} \rme^{-\norm{\tilde{p}}^2/2} \rmd \tilde{p} = 0\eqsp.
% \end{equation}
% It follows from \Cref{lem:bound_first_iterate_leapfrog} and \eqref{eq:def_acc_ratio} that for all $q \in \rset^d$, $\harmonic(q) = \pi(\harmonic)$
% which concludes the proof.
% =======
% The proof of \ref{theo:irred_harris_b} using a change of variable with $\Phiverletqi[h][T](q,\cdot)$.

% We now show that $\Tker_{h,T}$ satisfies the condition which implies that $\Pkerhmc[h][T]$ is a \Tkernel.
% First, for all $\eventB \in \borelSet(\rset^d)$,
% \begin{equation}
% \label{eq:minoration_pseudo_density_P}
% \Tker_{h,T}(q, \eventB) \geq (2 \uppi)^{-d/2} \Leb(\eventB)
%  \times \inf_{\bar{q} \in \eventB} \defEns{\alphaacc\defEns{(q,\Phiverletqi[h][T](q,\bar{q})),\Phiverlet[h][T](q,\Phiverletqi[h][T](q,\bar{q}))} \rme^{-\norm{\Phiverletqi[h][T](q,\bar{q})}^2/2} \detj_{\Phiverletqi[h][T]}(q,\bar{q})} \eqsp,
% \end{equation}
% with the convention $0 \times \plusinfty = 0$. Since $\Phiverletqi[h][T](q,\cdot)$
% is a diffeomorphism on $\rset^d$, we get that  $
% \Tker_{h,T}(q,\eventB) >0$ for all $q \in \rset^d$ and all compact set $\eventB$ satisfying $\Leb(\eventB) > 0$. Since the Lebesgue measure is regular, this implies that $\Pkerhmc[h][T]$ is irreducible with respect to the Lebesgue measure and aperiodic.

% By Fatou's Lemma, for any $\eventB \in\mcb(\rset^d)$, $q \mapsto \Tker_{h,T}(q,\eventB)$ is lower semi-continuous.
% We now show that all the compact sets are small. Let $\eventB \subset \rset^d$ be compact.  Using
% \eqref{eq:inverse_1_2} there exists $C \geq 0$ such that for all
% $q,p,v \in \rset^d$, $ \norm{p-v} \leq C \normLigne{
%   \Phiverletq[h][T](q,p)- \Phiverletq[h][T](q,v)}$. It follows
% that for all $p \in \rset^d$, $\sup_{q \in \eventB} \normLigne{
%   \Phiverletqi[h][T](q,\p)} \leq C\defEnsLigne{\norm{\p} + \sup_{q \in
%     \eventB} \normLigne{\Phiverletq[h][T](q,0)}}$. Using this upper
% bound and $\Jac_{\Phiverletq(\q,\cdot)}(\Phiverletqi[h][T](q,\p))
% \Jac_{\Phiverletqi[h][T]}(q,\tilde{\p}) = \operatorname{I}_n$ in
% \eqref{eq:minoration_pseudo_density_P}, where $\operatorname{I}_n$ is
% the identity matrix, we deduce that there exists $\varepsilon >0$ such
% that for all $\eventA\in \borelSet(\rset^d)$, $\eventA \subset \eventB$,
% \begin{equation}
%   \inf_{q \in \eventB} \Pkerhmc[h][T](q, \eventA)  \geq \varepsilon \Leb(\eventA) \eqsp,
% \end{equation}
% and therefore $\eventB$ is small for $\Pkerhmc[h][T]$.
% >>>>>>> f8207bad5c0353bdfe37210ffc64a715e92e53ed

Finally, the last statements of \ref{theo:irred_harris_c} follows from \Cref{propo:harris_rec} in \Cref{sec:harr-recurr-metr} which implies that  $ \Pkerhmc[h][T]$ is Harris recurrent and  \cite[Theorem 13.0.1]{meyn:tweedie:2009} which implies  \eqref{eq:harris-theorem}.

\subsubsection{Proof of \Cref{theo:irred_D}}
\label{sec:proof-crefth_irred_D}
We use \Cref{coro:irred}. Indeed $\Pkerhmc[h][T]$ is
of form \eqref{eq:def_pkerb} and it is straightforward to check that it
satisfies \Cref{assumG:phi} (note that \Cref{lem:bound_first_iterate_leapfrog_a}
shows that $\Phiverlet[h][T]$ is a Lipshitz function on $\rset^{2d}$).

We now check that $\Pkerhmc[h][T]$ satisfies \Cref{assumG:irred_b}($\rassG,0,\MassG$) for all $\rassG,\MassG \in
\rset_+^*$ using \Cref{le:degree_application}.  By \eqref{eq:qk}, for all $T \in \nsets$, $h >0$, $q,p \in \rset^d$,
\begin{equation}
  \label{eq:phiverlet_gqth}
  \Phiverletq[h][T](q,p) = T
h p + g_{q,T,h}(p)
\end{equation}
where $g_{q,T,h}(p) = q - (Th^2/2) \nabla \F(q) -
h^2 \gperthmc[T](q,p)$ where $\gperthmc[T]$ is defined by \eqref{eq:def_gperthmc}. \Cref{lem:inverse_1} shows that for any $T \in \nsets$ and $h >0$, it holds that
\begin{equation}
    \label{eq:2:theo:irred_D}
\sup_{p,v,q \in  \rset^d} \frac{\norm{g_{q,T,h}(p)-g_{q,T,h}(v)}}{\norm{p - v}} \leq T h [\{1 + h \constzero^{1/2} \vartheta_1(h \constzero^{1/2} )\}^T-1] \eqsp,
\end{equation}
which implies that the condition
\Cref{le:degree_application}-\ref{propo:irred_b_item_i} is satisfied. To check that
condition  \Cref{le:degree_application}-\ref{propo:irred_b_item_ii} holds, we consider separately the two cases: $\beta <1$ and $\beta =1$.

\begin{enumerate}[label=$\bullet$, wide, labelwidth=!, labelindent=0pt]
\item Consider first the case $\beta <1$. By \Cref{assum:regOne}-\ref{assum:regOne_b},
for any $T \in \nsets$ and $h >0$, we get
\begin{equation}
\norm{\gperthmc[T](\q,\p)} \leq  T \sum_{i=1}^{T-1} \norm{\nabla \F \circ \Phiverletq[h][i](\q,\p)} \leq
\constzeroT T \sum_{i=1}^{T-1} \defEns{ 1 + \norm{\Phiverletq[h][i](\q,\p)}^{\expozero}}
 \eqsp.
\end{equation}
Hence, by \Cref{lem:bound_first_iterate_leapfrog_b}-\ref{lem:bound_first_iterate_leapfrog_1}
there exists $C \geq 0$ such that for all $R\in \rset_+^*$ and
$q,p \in \rset^d$, $\norm{q} \leq R$,
\begin{equation}
\label{eq:3:theo:irred_D}
\norm{g_{q,T,h}(p)} \leq C \defEns{1+R^{\beta} +\norm{p}^{\expozero}} \eqsp,
\end{equation}
which implies that condition \ref{propo:irred_b_item_ii} of \Cref{le:degree_application} holds for any $T \in \nsets$ and $h >0$.

\item Consider now the case $\beta =1$.  For any $T \in \nsets$, $h >0$,  $q,p \in \rset^d$ we get using \Cref{assum:regOne}-\ref{assum:regOne_a}
\begin{align}
  \norm{g_{q,T,h}(p)} &\leq \norm{q} + Th^2 \constzero  \norm{q} /2 + Th^2 \norm{\nabla U(0)} /2\\
  & \qquad \qquad +h^2 \norm{\gperthmc[T](q,p) - \gperthmc[T](q,0)} + h^2 \norm{ \gperthmc[T](q,0)} \eqsp.
\end{align}
Therefore using \Cref{lem:inverse_1}, for any $q,p \in \rset^d$, $\norm{q} \leq R$ for $R \geq 0$, for any $T \in \nsets$ and $h >0$ satisfying \eqref{eq:condition-h,T-harris}, there exists $C \geq 0$ such that
\begin{equation}
  \norm{g_{q,T,h}(p)} \leq C + h T  [ \{1+ h \constzero^{1/2} \vartheta_1(h\constzero^{1/2})\}^T-1]  \norm{p} \eqsp,
\end{equation}
showing that condition \ref{propo:irred_b_item_ii} of \Cref{le:degree_application} is satisfied.
\end{enumerate}

Therefore,  \Cref{le:degree_application} can be applied and for any $T \in \nsets$ and $h >0$ if $\beta <1$ and for any $h > 0$ and $T \in \nsets$ satisfying \eqref{eq:condition-h,T-harris} if $\beta =1$, $\Pkerhmc[h][T]$ satisfies \Cref{assumG:irred_b}($\rassG,0,\MassG$) for all $\rassG,\MassG \in
\rset_+^*$.  \Cref{coro:irred} concludes the proof of \ref{theo:irred_D_a} and \ref{theo:irred_D_b}.
The last statement then follows from   \cite[Theorem 14.0.1]{meyn:tweedie:2009}.

% Using this result and \Cref{theo:irred}, we get that for all $\rassG,\MassG  \in \rset_+^*$ there exists $\varepsilon >0$ such that
% for all $\q \in \ball{0}{\rassG}$ and $\eventA \in \borelSet(\rset^d)$,
% \begin{equation}
%   \Pkerhmc[h][T](q, \eventA) \geq \varepsilon \Leb(\eventA \cap \ball{0}{M}) \eqsp.
% \end{equation}
% \Cref{coro:irred} Combining this result and \eqref{eq:1:theo:irred_D} concludes the proof of \ref{theo:irred_D_a} and \ref{theo:irred_D_b}.

% The proof is a consequence of \Cref{lem:bound_first_iterate_leapfrog},
% \Cref{le:degree_application} and \Cref{theo:irred}.  \alain{give some
%   details}

%%% Local Variables:
%%% mode: latex
%%% TeX-master: "main"
%%% End:

\subsection{Proofs of \Cref{sec:geom-ergod-hmc}}

\subsubsection{Proof of \Cref{propo:geo_drift_MH}}
\label{sec:proof-crefpr}
By construction   \eqref{eq:def_kenel_MH}, for all $\q \in \rset^d$, we have
\begin{align}
&\Pker \Vgeo (\q) - \Vgeo(\q) = \int_{\rset^{2d}} \defEns{\Vgeo(\projq(z)) - \Vgeo(\q)} \alphagen(\q,z) \Kker(\q, \rmd z )  \\
& \qquad = \Kker\Vgeo(\q)-\Vgeo(\q) +\int_{\rset^{2d}} \defEns{\Vgeo(\projq(z)) - \Vgeo(\q)} \defEns{\alphagen(\q,z)-1} \Kker(\q, \rmd z ) \eqsp.
\end{align}
Using \eqref{eq:assum:geo_ergo_1}, this implies for all $\q \in \rset^d$,
\begin{equation}
\label{eq:proof_geo_drift_MH_1}
\Pker \Vgeo (\q)  \leq \lambdageo \Vgeo(\q) + b
+ \int_{\rset^{2d}} \defEns{\Vgeo(\projq(z)) - \Vgeo(\q)} \defEns{\alphagen(\q,z)-1} \Kker(\q, \rmd z ) \eqsp.
\end{equation}

Note that by definition \eqref{eq:def_rej_ballV} of $\rejectregion(\q)$ and $\ballV(\q)$
\begin{align}
&\int_{\rset^{2d}} \defEns{\Vgeo(\projq(z)) - \Vgeo(\q)} \defEns{\alphagen(\q,z)-1} \Kker(\q, \rmd z )
\\ &  \qquad  \qquad  \qquad  \leq    \int_{\rejectregion(\q) \cap \ballV(\q) } \defEns{ \Vgeo(\q)-\Vgeo(\projq(z))}  \Kker(\q, \rmd z ) \eqsp.
\end{align}
Therefore by \eqref{eq:assum:geo_ergo_2}, we get
\begin{equation}
 \lim_{M\to \plusinfty} \sup_{\set{\q \in \rset^d}{\Vgeo(\q) \geq M}}  \int_{\rset^{2d}} \left\{\Vgeo(\projq(z))/\Vgeo(\q) -1 \right\} \left\{ \alphagen(\q,z)-1 \right\} \Kker(\q, \rmd z ) \leq 0 \eqsp.
\end{equation}
The proof then follows from combining this result and \eqref{eq:proof_geo_drift_MH_1} since they imply
\begin{equation}
   \lim_{M\to \plusinfty} \sup_{\set{\q \in \rset^d}{\Vgeo(\q) \geq M}}  \Pker \Vgeo (\q) / \Vgeo(\q) \leq \lambda  \eqsp.
\end{equation}

\subsubsection{Proof of \Cref{lem:drift_uhmc}}
\label{sec:proof-crefl-2}


%\begin{proof}[Proof of \Cref{lem:drift_uhmc}]
Let $a \in \rset_+^*$. Under \Cref{assum:regOne}$(m-1)$ with $m \in \ocint{1,2}$,   \Cref{lem:bound_first_iterate_leapfrog_a} shows that, for all $\q_0 \in \rset^d$,
$\p \mapsto \Phiverletq[h][T](\q_0,\p)$ is Lipschitz, with a Lipschitz constant $L_{h,T} \in \rset_+$
\begin{equation}
\label{eq:definition-lipshitz}
L_{h,T} \eqdef \defEns{1+h \constzero^{1/2} \vartheta_1(h \constzero^{1/2})}^{T} \eqsp.
\end{equation}
Therefore by the log-Sobolev inequality \cite[Proposition 5.5.1, (5.4.1)]{bakry:gentil:ledoux:2014} and \eqref{eq:def_Pker_proposition_double}, we get for all $\q_0 \in \rset^d$
\begin{equation}
\PkerhmcD[h][T] \Vdrifta[\a](\q_0) \leq \exp\parenthese{(aL_{h,T})^2/2 + a \Eproof[h][T](\q_0)} \eqsp,
\end{equation}
with
\begin{equation}
\Eproof(\q_0) = (2\uppi)^{-d/2} \int_{\rset^d} \norm{\Phiverletq[h][T](\q_0,\p)} \rme^{-\norm{\p}^2/2} \rmd \p \eqsp.
\end{equation}
Set $p_0 \in \rset^d$.
Denote for all $k \in \{0,\ldots,T\}$, $q_k =
\Phiverletq[h][k](\q_0,\p_0)$ and consider the following decomposition given by  \eqref{eq:qk}:
\begin{equation}
\label{eq:drift_uhmc_3}
\norm{\q_T}^2  =  \norm{\q_0}^2 + \operatorname{A}^{(1)}_{h,T}(\q_0,\p_0) -2h^2 \operatorname{A}^{(2)}_{h,T}(\q_0,\p_0) \eqsp,
\end{equation}
where
\begin{align}
\operatorname{A}^{(1)}_{h,T}(\q_0,\p_0) & = 2Th \ps{\q_0}{ \p_0} + \norm{ Th\p_0-(Th^2/2) \nabla \F(\q_0)-h^2 \sum_{i=1}^{T-1}(T-i)\nabla \F (\q_i)}^2 \\
\operatorname{A}^{(2)}_{h,T}(\q_0, \p_0) & = \ps{\q_0}{ (T/2) \nabla \F(\q_0)+ \sum_{i=1}^{T-1}(T-i)\nabla \F (\q_i)} \eqsp.
\end{align}
Jensen's inequality shows that, for all $\q_0 \in \rset^d$,
\[
\Eproof[h][T](\q_0) \leq \left( \norm{\q_0}^2 + \bar{\operatorname{A}}^{(1)}_{h,T}(\q_0) - 2 h^2 \bar{\operatorname{A}}^{(2)}_{h,T}(\q_0) \right)^{1/2} \eqsp,
\]
where we have set $\bar{\operatorname{A}}_{h,T}^{(i)}(\q_0)= (2\uppi)^{-d/2} \int_{\rset^d} \operatorname{A}_{h,T}^{(i)}(\q_0,\p) \rme^{-\norm{\p}^2/2} \rmd \p$, $i=1,2$.
Therefore to conclude the proof, it is sufficient to show that
\begin{equation}
\label{eq:drift_uhmc_minus1}
\limsup_{\norm{\q_0} \to \plusinfty} \defEnsLigne{\Eproof(\q_0) - \norm{\q_0}} = - \infty.
\end{equation}
\begin{enumerate}[label=(\alph*),leftmargin=0cm,itemindent=0.5cm,labelwidth=1.2\itemindent,labelsep=0cm,align=left]
\item Consider the case $m \in \ooint{1,2}$. Using \Cref{assum:regOne}$(m-1)$ and  \Cref{lem:bound_first_iterate_leapfrog_b}-\ref{lem:bound_first_iterate_leapfrog_1}, we get that  there exists a constant $C_0 \geq 0$ such that for all $\p_0,\q_0 \in \rset^d$ and $i \in \{1,\dots,T-1\}$,
  \begin{equation}
    \label{eq:drift_nabla_U_q_i}
    \norm{\nabla \F(\q_i)} \leq C_0 \{1 + \norm{\p_0} + \norm{\q_0}^{m-1}\}
  \end{equation}
  which implies that
\begin{equation}
\label{eq:bound-A-1}
|\bar{A}^{(1)}_{h,T}(\q_0)| \leq C_1 \{1 + \norm{\q_0}^{2(m-1)} \} \eqsp,
\end{equation}
for some constant $C_1 \geq 0$.  On the other hand, note that for any $q_0,p_0 \in \rset^d$,  $\operatorname{A}^{(2)}_{h,T}(\q_0,\p_0) =  \operatorname{A}^{(2,1)}_{h,T}(\q_0,\p_0) +  \operatorname{A}^{(2,2)}_{h,T}(\q_0,\p_0)$ with
\begin{align}
\label{eq:definition-A-2-1}
\operatorname{A}^{(2,1)}_{h,T}(\q_0,\p_0) &= \frac{T}{2} \ps{\q_0}{ \nabla \F(\q_0)}+\sum_{i=1}^{T-1}(T-i)  \ps{\q_i}{\nabla \F (\q_i)}, \\ 
\label{eq:definition-A-2-2}
\operatorname{A}^{(2,2)}_{h,T} &=- \sum_{i=1}^{T-1}(T-i)  \ps{\q_0-\q_i}{\nabla \F (\q_i)} \eqsp.
\end{align}
Under \Cref{assum:potential:c}$(m)$, for any $q_0, p_0 \in \rset^d$, we have that
\begin{equation}
\label{eq:lower-bound-A-2-1}
\operatorname{A}_{h,T}^{(2,1)}(\q_0,\p_0) \geq \constthree \frac{T}{2} \norm{\q_0}^m - \frac{T (T-1)}{2} \constfour \eqsp.
\end{equation}
Further, by \eqref{eq:drift_nabla_U_q_i} and  \Cref{lem:bound_first_iterate_leapfrog_b}-\ref{lem:bound_first_iterate_leapfrog_1},  there exists  $C_2 \geq 0$, such that for all $\p_0,\q_0 \in \rset^d$,
\begin{equation}
\label{eq:bound-A-2-2}
|\operatorname{A}^{(2,2)}_{h,T}(\q_0,\p_0)| \leq C_2 \{1 + \norm{p_0}^2 + \norm{\q_0}^{2(m-1)} \} \eqsp,
\end{equation}
Combining \eqref{eq:lower-bound-A-2-1} and \eqref{eq:bound-A-2-2}, there exists  $C_3 \geq 0$ such that for any $q_0 \in \rset^d$,
\begin{equation}
\label{eq:bound-A-2}
\bar{\operatorname{A}}^{(2)}(\q_0) \geq \frac{T \constthree}{2} \norm{\q_0}^m - C_3 \{1 + \norm{\q_0}^{2(m-1)} \} \eqsp.
\end{equation}
Combining \eqref{eq:bound-A-1} and \eqref{eq:bound-A-2}, and using that $m < 2$, we finally obtain that \eqref{eq:drift_uhmc_minus1} holds.
% , as $\norm{\q_0} \to \infty$,
% \[
% \Eproof(\q_0) - \norm{\q_0} \leq - \constthree h^2 T^2 \norm{\q_0}^{m-1} + o(\norm{\q_0}^{m-1})
% \]
\item  By Cauchy-Schwarz and Hölder inequality and since $\nabla U$ satisfies  \Cref{assum:regOne}$(1)$, we have for any $q_0,p_0 \in \rset^d$,
\begin{align}
&\operatorname{A}^{(1)}_{h,T}(\q_0,\p_0)
\leq 2hT \norm{q_0} \norm{p_0} \\
& +3 \parentheseDeux{ h^2 T^2  \norm[2]{\p_0} +  2 h^4 T^4 \constzeroT^2 (1+ \norm[2]{ \q_0}) +   2 h^4 T^2  \constzero^2  \defEns{ \sum_{i=1}^{T-1} \norm{\q_i - q_0}}^2} \eqsp,
\end{align}
which implies using \Cref{lem:bounded_cum_error}, $\vartheta_1(s) \geq 1$ for any $s \geq 0$, and the dominated convergence theorem that
\begin{align}
\label{eq:bound-A-1-m=2}
&\limsup_{\norm{q_0} \to \plusinfty} |\bar{\operatorname{A}}^{(1)}_{h,T}(\q_0)|/ \norm[2]{q_0}  \\
&\qquad \qquad \leq
6 h^4 T^4 \left(  \constzeroT^2 +  \constzero \vartheta_2^2(h) \left[ \{1 + h \constzero^{1/2} \vartheta_1(\constzero^{1/2} h)\}^T -1 \right]^2 \right)  \eqsp.
\end{align}
% =======
% which implies, using \Cref{lem:bounded_cum_error} that, as $\norm{\q_0}` \to \infty$,
% \begin{equation}
% \label{eq:bound-A-1-m=2}
% |\bar{\operatorname{A}}^{(1)}_{h,T}(\q_0)| \leq
% h^4 T^4 \left( (3/4) \constzeroT^2 + 3 \constzero \vartheta_2^2(h) \left[ \{1 + h \constzero^{1/2} \vartheta_1(\constzero^{1/2} h)\}^T -1 \right] \right) \norm[2]{\q_0} + o(\norm{\q_0}) \eqsp.
% >>>>>>> 42af951f106d06daaead79ba0820f840f21e5191
% \end{equation}
Similarly using in addition  \Cref{assum:potential:c}($2$), we get that for any $q_0,p_0 \in \rset^d$,
\begin{align}
\operatorname{A}^{(2)}_{h,T}(\q_0,\p_0)
&= \ps{\q_0}{ (T^2/2)  \nabla \F(\q_0)+ \sum_{i=1}^{T-1}(T-i)\{\nabla \F (\q_i) - \nabla U(q_0)\}}  \\
&\geq  (T^2/2) \{\constthree  \norm[2]{q_0} - \constfour\} - T \constzero \norm{q_0} \sum_{i=1}^{T-1}\norm{ q_i - q_0 }  \eqsp.
\end{align}
Then, \Cref{lem:bounded_cum_error} and the Fatou Lemma imply that
\begin{align}
& \liminf_{\norm{q_0} \to \plusinfty} h^2 \bar{\operatorname{A}}^{(2)}_{h,T}(\q_0)/\norm{q_0}^2  \\
 & \qquad \qquad  \qquad  \geq
h^2 T^2  \left( \constthree/2 - \constzero^{1/2} \vartheta_2(h) [(1+ h \constzero^{1/2}\vartheta_1(h \constzero^{1/2}))^T-1]\right)  \eqsp.
\end{align}
Therefore, for all  $h > 0$, and $T \in \nset^*$, one obtains
\[
\limsup_{\norm{q_0} \to \plusinfty} \{ \Eproof[h][T](\q_0) \}^2  /\norm[2]{q_0}   \leq 1 -  T^2 h^2 (\constthree -  \Theta(hT))  \eqsp,
\]
where $\Theta$ is defined in \eqref{eq:definition-function-C}. The proof follows.
\end{enumerate}
%\end{proof}


\subsubsection{Proof of \Cref{le:convex}}
\label{sec:proof-crefle:convex}

Note that condition~\ref{le:convex:a}  implies that
\begin{equation}
  \label{eq:le:convex:b:conseq}
  \inf_{\set{\q}{\norm{\q}=\Rexp}} \F_0(\q) >0 \eqsp.
\end{equation}
Condition \Cref{assum:potential}-\ref{assum:potential:a} follows from \ref{le:convex:b} using that, by \ref{le:convex:a},
 for all $\q \in \rset^d$, $\norm{q} \geq \Rexp$
\[
\F_0(q) =  (\norm{q}/\Rexp)^{\m}\F_0 (\Rexp  q / \norm{q} ) 
\]

  In addition, \Cref{assum:potential:c} is also
  easy to check using the Euler's homogeneous function theorem that
$\ps{\nabla \F_0(\q)}{
 \q}= \m \F_0(\q)$ for all $\q \in \rset^d$, $\norm{\q} \geq \Rexp$.
  % Estimates \Cref{assum:potential}-\ref{assum:potential:a} follow for
  % large values of $\norm{\q}$ just by the stipulated $\m$-homogeneity of
  % $F_0$ and the assumed growth of the derivatives of $G$.
  % Let $U=S^{\circ}$ and $0<t<1$. We have $tS+(1−t)S \subset S$ by convexity, so
  % $tU+(1−t)U\subset S$. But $tU$ is open, so $tU+(1−t)U$ as well. Therefore, $tU+(1−t)U\subset S^{\circ}=U$, and

  % hence $U$ is convex.
  We show below that \Cref{assum:potential}-\ref{assum:potential:b}
  holds. First, since $\lim_{\norm{\q} \to \plusinfty} \F_0(\q) = \plusinfty$ and $\F_0$ is continuous  for all $K \geq 0$, $\lset_K=\{ \q \in \rset^d \ ; \ \F_0(\q) \leq K\}$
%     \begin{equation}
% %    \label{eq:def_lset_le:convex}
% \lset_K=\{ \q \in \rset^d \ ; \ \F_0(\q) \leq K\}
%   \end{equation}
 is compact. Besides, using \eqref{eq:le:convex:b:conseq} and that $\F_0$ is continuous,  we can define
  \begin{equation}
    \label{eq:def_M_le:convex}
M =  \sup_{\q \in \ball{0}{\Rexp}} \F_0(\q) +1 \in \ooint{1, \plusinfty}\eqsp,
  \end{equation}
and for all $\q \not \in \lset_M$,
\begin{equation}
  \label{eq:deftq_le:convex_0}
 t_q = \sup \defEns{ t \in \ccint{0,1} \ ; \ \F_0(t \q ) =M } \eqsp,
\end{equation}
%$t_q = \sup \defEns{ t \in \ccint{0,1} \ ; \ \F_0(t \q ) =M }$,
 which satisfies
% $\F_0(q) >M > \sup_{\q \in
%     \ball{0}{\Rexp}} \F_0(\q)$ therefore $\norm{\q} \geq \Rexp$ and
%  by continuity of $\F_0$, there exists $t_{\q} \in \ccint{0,1}$ such that
  \begin{equation}
    \label{eq:deftq_le:convex}
    \F_0(t_{\q} \q) = M > \sup_{\x \in \ball{0}{\Rexp}} \F_0(\x) \eqsp, \eqsp t_{\q} q \in \partial \lset_M \eqsp \text{ and } \eqsp t_{\q}
  \norm{\q} > \Rexp \eqsp.
  \end{equation}
% $\F_0(t_{\q} \q) = M > \sup_{\x \in \ball{0}{\Rexp}} \F_0(\x)$ and $t_{\q}
%   \norm{\q} \geq \Rexp$.
 Finally using \ref{le:convex:a}, we get that the set
  $\lset_M$  is
  convex.
 % In addition, since $\F_0$ is continuous $\lset_M$
 %  contains the origin in its interior and $\partial \lset_M\subset\{\q
 %  \in \rset^d \, ; \, \F_0(\q)=M\}$.

  To show \Cref{assum:potential}-\ref{assum:potential:b}, we check first
  that it is sufficient to prove that
  \begin{equation}
D^2\F_0(\x)\defEnsLigne{\nabla
    \F_0(\x)\otimes \nabla \F_0(\x)}>0 \text{ for any } \x \in \partial
  \lset_M \eqsp.
  \end{equation}
% $D^2\F_0(\x)\defEnsLigne{\nabla
%     \F_0(\x)\otimes \nabla \F_0(\x)}>0$ for any $\x \in \partial
%   \lset_M$.
 Indeed note that if this statement holds, since $\F \in C^2(\rset^d)$
  and $\partial \lset_M$ is compact, we have
  \begin{equation}
    \label{eq:le:convex:1}
\varepsilon =  \inf_{x \in \partial \lset_M} D^2\F_0(\x)\defEnsLigne{\nabla
    \F_0(\x)\otimes \nabla \F_0(\x)}>0 \eqsp.
  \end{equation}
  % Note that for all $\q \not \in \lset_M$, $\F_0(q) >M > \sup_{\q \in
  %   \ball{0}{\Rexp}} \F_0(\q)$ therefore $\norm{\q} \geq \Rexp$.  In
  % addition since $0$ is in the interior of $\lset_M$, $\F_0(0)< M$
  % define for all $\q \not \in \lset_M$, $\phi(\q) \geq 1$ such that
  % $\q = \phi(\q) \x$ for $\x \in \partial \lset_M$.
Let now $\q \not \in \lset_M$ and $t_{\q}$ defined by \eqref{eq:deftq_le:convex_0}.
Since by \ref{le:convex:a}, for all $u
  \geq 1$ and $z \in \rset^d$, $\norm{z} \geq \Rexp$, $\F_0(uz) = u^\m \F_0(z)$,
  differentiating with respect to $z$, we get $\nabla \F_0(uz) = u^{\m-1}
  \nabla \F_0(z)$ and $D^2 \F_0(uz) = u^{\m-2} D^2\F_0(z)$. Therefore by \eqref{eq:deftq_le:convex}, we get
 \begin{equation}
    \label{eq:le:convex:2}
   D^2\F_0(\q)\defEns{\nabla
  \F_0(\q)\otimes \nabla \F_0(\q)} = t_{\q}^{4-3m} D^2\F_0(t_{\q} \q)\defEns{\nabla
  \F_0(t_{\q} \q )\otimes \nabla  \F_0(t_{\q} \q)} \eqsp.
 \end{equation}
Using \eqref{eq:deftq_le:convex} again and since $\partial \lset_M$ is compact, we get that
there exists $R_2 \geq 0$ such that $t_{\q} \norm{q} \in \ccint{\Rexp,R_2}$. Hence by \eqref{eq:le:convex:2}, we have
\begin{equation}
   D^2\F_0(\q)\defEns{\nabla
  \F_0(\q)\otimes \nabla \F_0(\q)} \geq \varepsilon \norm{q}^{3m-4} \min\parentheseDeux{\Rexp^{4-3m},R_2^{4-3m}} \eqsp.
\end{equation}
Thus
 \Cref{assum:potential}-\ref{assum:potential:b} holds for $\F_0$.
Finally \ref{le:convex:b}
 implies that  the function $\F=\F_0+G$ satisfies
 \Cref{assum:potential}-\ref{assum:potential:b} as well.


 Let $ \x \in \partial \lset_M$, we now show that $D^2\F_0(\x)\defEns{\nabla
 \F_0(\x)\otimes \nabla \F_0(\x)}>0$. By Euler's homogeneous function theorem and since $M \geq 1$, we have that
 $\abs{\ps{\nabla \F_0(\x)}{\x}}\geq \m >0$.  Denote by $\Pi$ the tangent hyperplane
 of $\partial\lset_M$ at $\x$, defined by $\Pi = \{ \q \in \rset^d \, :
 \, \ps{\nabla \F_0(x) }{ \x-\q} = 0\}$.  Since $\lset_M$ is convex, for all $\q \in
 \lset_M$ and $t \in \ccint{0,1}$, $ t^{-1}( \F_0(t\q +(1-t)\x)
 -\F_0(\x))\leq 0$. So taking the limit as $t$ goes to $0$, we get that
 $\ps{\nabla \F_0(\x) }{ \q-\x} \leq 0$. Therefore, $\lset_M$ is
 contained in the half-space $\Pi^- = \{ q \in \rset^d \, ; \, \ps{\nabla
 \F_0(\x) }{ \q-\x} \leq 0 \}$.

Define the  $\m$-homogeneous
 function $\tilde{\F} : \rset^d \to \rset_+$  for all $\q \in \rset^d$  by
 \begin{equation}
\label{eq:def:tilde_F}
   \tilde{\F}(\q) = M \abs{\frac{\ps{\q}{ \nabla \F_0(\x)}}{\ps{\x}{ \nabla \F_0(\x)}}}^\m \eqsp.
 \end{equation}
 Since $\F_0(x) = M$, by \eqref{eq:def_M_le:convex}, $\norm{x} > \Rexp$
 and therefore there exists $\epsilon_0 \in \rset_+^*$ such that
 \begin{equation}
\label{eq:inclusion_ball}
   \ball{x}{\epsilon_0} \subset \rset^d \setminus \ball{0}{\Rexp}\eqsp.
 \end{equation}
 We
 now show that $\tilde{\F}(\q) \leq \F_0(\q)$ for all $\q \in
 \ball{\x}{\epsilon}$ with
 \begin{equation}
\epsilon = 2^{-1}\min\parentheseDeux{\epsilon_0, \{\ps{\x}{ \nabla \F_0(\x)}\}/\norm[2]{\nabla \F_0(\x)}} \eqsp.
 \end{equation}
  First consider $\q \in \Pi$. We next argue by contradiction that
 \begin{equation}
\label{eq:bound_Pi}
   \F_0(\q) \geq M = \tilde{\F}(\q)    \eqsp.
 \end{equation}
Indeed assume that  $\F_0(\q) < M$. Then by continuity of $\F_0$, we get that $\q \in \interior{\lset_{M}}$. But since $\lset_{M} \subset \Pi^-$, we get $\q \in \interior{(\Pi^-)}$ which is impossible since $\q \in \Pi = \boundary{\Pi^-} = \clos{\Pi^-} \setminus \interior{(\Pi^-)}$.

Let $\q \in
 \ball{\x}{\epsilon}$. Note that  $  \q= \x + \norm[-2]{\nabla \F_0(\x)}\ps{\q-\x}{\nabla \F_0 (\x)}\nabla \F_0(\x) + z$,
% \begin{equation}
%   \q= \x + t\nabla \F_0(\x) + z \eqsp,
% \end{equation}
where  $z \in \rset^d$ is orthogonal to $\nabla \F_0(\x)$. Define
\begin{equation}
  u = \frac{\ps{\x }{ \nabla \F_0(\x)}}{\ps{\q}{ \nabla \F_0(\x)}}\eqsp.
\end{equation}
Then $u\q \in \Pi$ and by \eqref{eq:bound_Pi}, $\F_0(u \q) \geq M$. If $u \geq 1$, using \ref{le:convex:a} and \eqref{eq:def:tilde_F}, we get
\begin{equation}
\label{eq:6}
  \F_0(\q) \geq  u^{-\m}M = \tilde{\F}(\q) \eqsp.
\end{equation}
In turn, if $u < 1$, since $\norm{\q- \x} \leq \epsilon_0$, by \eqref{eq:inclusion_ball} and \ref{le:convex:a}, $\F_0(\q) = u^{-1} \F_0(u \q)$ and \eqref{eq:6} still holds.

Consider the three times differentiable functions $\phi$ and $\tilde{\phi}$
defined for all $v \in \rset$ by
$$
\phi(v)=\F_0(\x+v \nabla \F_0(\x))\quad\textrm{and}\quad \tilde{\phi}(v)= \tilde{\F}(\x+v \nabla \F_0(\x)) \eqsp.
$$
First, since for all $\q \in \ball{\x}{\epsilon}$, $\F_0(\q) \geq \tilde{\F}(\q)$, we have
\begin{equation}
  \label{eq:bound_f_tildef}
  \phi(v) \geq \tilde{\phi}(v) \eqsp, \text{for all $v \in \ccint{-\epsilon/\norm{\nabla \F_0(\x)},\epsilon/\norm{\nabla \F_0(\x)}}$}\eqsp.
\end{equation}
Moreover, by definition $\F_0(\x) = \tilde{\F}(\x)$ and $\nabla \tilde{\F}(\x)$
is colinear to $ \nabla \F_0(\x)$. Using  Euler's homogeneous function theorem for $\F_0$
and $\tilde{\F}$, we get that $\nabla \tilde{\F}(\x) = \nabla
\F_0(\x)$. Therefore $\phi(0) = \tilde{\phi}(0)$, $\phi'(0) = \tilde{\phi}'(0)$. Combining these equalities, \eqref{eq:bound_f_tildef} and using a Taylor expansion around $0$ of order $2$ with exact remainder for $\phi$ and $\tilde{\phi}$ shows that necessary
\begin{equation}
  D^2\F_0(\x)\defEns{\nabla
  \F_0(\x)\otimes \nabla \F_0(\x)} =   \phi''(0) \geq \tilde{\phi}''(0) >0 \eqsp,
\end{equation}
which concludes the proof.
% Secondly, note that since $\F_0$ is continuous and
%   $\lim_{\norm{\q} \to \plusinfty} \F_0(\q) = \plusinfty$, by
%   \Cref{assum:potential_second}-\ref{le:convex:a}, there exists $M
%   \geq \Rexp+1$ such that the sublevel set $\lset_M:=\{\q \in \rset^d
%   \, ; \, \F_0(\q) \leq M \}$ is a compact convex set and contains the
%   origin in its interior. Moreover, $\partial \lset_M=\{\q \in \rset^d
%   \, ; \, \F_0(\q)=M\}$. To show
%   \Cref{assum:potential}-\ref{assum:potential:b}, we check that it is
%   sufficient to prove that $D^2\F_0(\x)\defEnsLigne{\nabla
%     \F_0(\x)\otimes \nabla \F_0(\x)}>0$ for any $\x \in \partial
%   \lset_M$. Indeed if this statement holds, since $\F \in C^3(\rset^d)$
%   and $\partial \lset_M$ is compact, there exists $\varepsilon >0$
%   such that
%   \begin{equation}
%     \label{eq:le:convex:1}
%  \inf_{x \in \partial \lset_M} D^2\F_0(\x)\defEnsLigne{\nabla
%     \F_0(\x)\otimes \nabla \F_0(\x)}>\varepsilon \eqsp.
%   \end{equation}
%   In addition since $0$ is in the interior of $\lset_M$, $\F_0(0)< M$  define for all
%   $\q \not \in \lset_M$, $\phi(\q) \geq 1$ such that $\q = \phi(\q)
%   \x$ for $\x \in \partial \lset_M$. Since for all $u \geq 0$ and $z
%   \in \rset^d$, $z \geq \Rexp$, $\F(uz) = u^\m \F(z)$, differentiating
%   with respect to $u$, we get $\nabla \F(uz) = u^{\m-1} \nabla \F(z)$
%   and $D^2 \F(uz) = u^{\m-2} D^2\F(z)$. Therefore,
%  \begin{equation}
%    D^2\F_0(\q)\defEns{\nabla
%   \F_0(\q)\otimes \nabla \F_0(\q)}> \phi(\q)^{3\m-4} D^2\F_0(\x)\defEns{\nabla
%   \F_0(\x)\otimes \nabla  \F_0(\x)} \eqsp.
%  \end{equation}
% %phi(\q) = norm{\q}/norm{\x}
%  Since $\lset_M$ is compact, there exists $C \geq 0$ such that for
%  all $\q \not \in \lset_M$, $\phi(\q) \geq C \norm{\q}$ and therefore
%  \Cref{assum:potential}-\ref{assum:potential:b} holds for $\F_0$. \Finally, the assumed growth of the derivatives of $G$ then
%  implies that  the function $\F=\F_0+G$ satisfies
%  \Cref{assum:potential}-\ref{assum:potential:b} as well.

%  Let $ \x \in \partial \lset_M$, we now show that $D^2\F_0(\x)\defEns{\nabla
%  \F_0(\x)\otimes \nabla \F_0(\x)}>0$. By the Euler relation $\ps{\nabla \F(\q)}{
%  \q}= \m \F(\q)$ for all $\q \geq \Rexp$, and since $M \geq 1$, we have that
%  $\norm{\nabla \F(\x)}\geq \m >0$.  Denote by $\Pi$ the tangent hyperplane
%  of $\partial\lset_M$ at $\x$, defined by $\Pi = \{ z \in \rset^d \, :
%  \, \ps{\nabla \F_0 }{ \x-\q} = 0\}$.  By our assumption, for all $\q \in
%  \lset_M$ and $t \in \ccint{0,1}$, $ t^{-1}( \F_0(t\q +(1-t)\x)
%  -\F(\x))\leq 0$ So taking the limit as $t$ goes to $0$, we get that
%  $\ps{\nabla \F_0(\x) }{ \q-\x} \leq 0$. Therefore, $\lset_M$ is
%  contained in the half-space $\Pi^- = \{ z \in \rset^d \, ; \, \ps{\nabla
%  \F_0(\x) }{ \q-\x} \leq 0 \}$. Define the function $\m$-homogeneous
%  function $\tilde{\F} : \rset^d \to \rset_+$ by for all $\q \in \rset^d$
%  \begin{equation}
% \label{eq:def:tilde_\F}
%    \tilde{\F}(\q) = M \norm{\frac{\ps{\q}{ \nabla \F_0(\x)}}{\ps{\x}{ \nabla \F_0(\x)}}}^\m \eqsp.
%  \end{equation}

%  We show that $\tilde{\F}(\q) \leq \F_0(\q)$ for all $\q \in
%  \ball{\x}{\epsilon}$ for
%  \begin{equation}
% \epsilon = 2^{-1}\min\parentheseDeux{1, \{\ps{\x}{ \nabla \F_0(\x)}\}/\norm[2]{\nabla \F_0(\x)}} \eqsp.
%  \end{equation}
%   \First note that for all $\q \in \Pi$, since $\lset_M
%  \subset \Pi^- $, then
%  \begin{equation}
% \label{eq:bound_Pi}
%    \F_0(\q) \geq M = \tilde{\F}(\q)    \eqsp.
%  \end{equation}
%  Now let $\q \in
%  \ball{\x}{1/2}$, note that $\q$ can be written in the form $  \q= \x + t\nabla \F_0(\x) + z$,
% % \begin{equation}
% %   \q= \x + t\nabla \F_0(\x) + z \eqsp,
% % \end{equation}
% where $t \in \ccint{-\epsilon,\epsilon}$ and $z \in \rset^d$ is orthogonal to $\nabla \F_0(\x)$. Define
% \begin{equation}
%   u = \frac{\ps{\q }{ \nabla \F_0(\x)}}{\ps{\x}{ \nabla \F_0(\x)} + t \norm[2]{\nabla \F_0(\x)}}\eqsp.
% \end{equation}
% Then $ux \in \Pi$ and by \eqref{eq:bound_Pi}, $\F_0(\q) \geq M$. Using that $\F_0$ is $\m$-homogeneous,
% \begin{equation}
%   \F_0(\q) \geq  \norm{u}^{-\m}M = \tilde{\F}(\q) \eqsp.
% \end{equation}

% Consider the three times differentiable functions $f$ and $\tilde{f}$
% defined for all $v \in \ccint{-\epsilon,\epsilon}$ by
% $$
% f(v):=\F_0(\x+v \nabla \F_0(\x))\quad\textrm{and}\quad \tilde{f}(v)= \tilde{\F}(\x+v \nabla \F_0(\x)) \eqsp.
% $$
% \First, since for all $\q \in \ball{\x}{\epsilon}$, $\F_0(\q) \geq \tilde{\F}(\q)$, we have for all $v \in \ccint{-\epsilon,\epsilon}$,
% \begin{equation}
%   \label{eq:bound_f_tildef}
%   f(v) \geq \tilde{f}(v) \eqsp.
% \end{equation}
% Moreover, by definition $\F_0(\x) = \tilde{\F}(\x)$, $\nabla \tilde{\F}(\x)$
% is parallel to $ \nabla \F_0(\x)$. Using the Euler identity for $\F_0$
% and $\tilde{\F}$, we get that $\nabla \tilde{\F}(\x) = \nabla
% \F_0(\x)$. Therefore $f(0) = \tilde{f}(0)$, $f'(0) = \tilde{f}'(0)$. Combining these inequalities, \eqref{eq:bound_f_tildef} and using a Taylor expansion around $0$ of order $3$ for $f$ and $\tilde{f}$ shows that necessary
% \begin{equation}
%   D^2\F_0(\x)\defEns{\nabla
%   \F_0(\x)\otimes \nabla \F_0(\x)} =   f''(0) \geq \tilde{f}(0) >0 \eqsp,
% \end{equation}
% which concludes the proof.


\subsubsection{Proof of \Cref  {propo:accept}}
\label{sec:proof-crefth}
We preface the proof by several technical preliminary Lemmas.


%We preface the proof by a useful Lemma.
\begin{lemma}
\label{lem:grad_Lip_F}
Assume \Cref{assum:potential}($\m$)-\ref{assum:potential:a} for some $m\in \ocint{1,2}$.
Then, for all $q,x \in \rset^d$,
$\norm{\nabla \F(\q) - \nabla \F(\x)} \leq \constone  \norm{\q-\x}$
and $\norm{\nabla \F(\q) - \nabla \F(\x)} \leq \constone (m-1)^{-1} \norm{\q-\x}^{m-1}$.
In particular, \Cref{assum:regOne}($m-1$) holds with $\constzero= \constone$ and $\constzeroT= \constone (m-1)^{-1} \vee \norm{\nabla \F(0)}$.
\end{lemma}


\begin{proof}
First by \Cref{assum:potential}($m$)-\ref{assum:potential:a}, we get for all $\q,\x \in \rset^d$,
\begin{align}
\nonumber
  \norm{\nabla \F(\q) - \nabla \F(\x)}
&= \norm{\int_{0}^1 \nabla^2 \F(\x +t(\q-\x)) \defEns{\q-\x} \rmd t } \\
\label{eq:lem_grad_Lip_F_eq_0}
&\leq \constone  \norm{\q-\x} \int_{0}^1 \defEns{1+\norm{\x +t(\q-\x)}}^{\m-2} \rmd t   \eqsp.
\end{align}
Therefore, for all $\q,\x \in \rset^d$, we get $ \normLigne{\nabla \F(\q) - \nabla \F(\x)} \leq \constone \normLigne{\q-\x}$. For all $\q, \x \in \rset^d$, since $m \in \ocint{1,2}$, we have
\begin{align}
&\int_{0}^1 \defEns{1+\norm{\x +t(\q-\x)}}^{\m-2} \rmd t  \leq  \int_{0}^1\defEns{ 1+\abs{\norm{\x} - t \norm{\q-\x}}}^{m-2}  \rmd t  \\
& \leq \int_{0}^{1 \wedge \frac{\norm{\x}}{\norm{\q-\x}}}\defEns{1+  \norm{\x} - t \norm{\q-\x}}^{m-2}  \rmd t
+ \int_{1 \wedge \frac{\norm{\x}}{\norm{\q-\x}}}^1 \defEns{1+ t \norm{\q-\x}-\norm{\x}}^{m-2}  \rmd t \\
&\leq (m-1)^{-1} \norm{\q-\x}^{m-2} \eqsp.
\end{align}
Plugging this result in \eqref{eq:lem_grad_Lip_F_eq_0} concludes the proof.

% \begin{align}
% &   \int_{0}^1 \defEns{1+\norm{\x +t(\q-\x)}}^{\m-2} \rmd t  \leq  \int_{0}^1\defEns{ 1+\abs{\norm{\x} - t \norm{\q-\x}}}^{m-2}  \rmd t  \\
% &\leq \int_{0}^{1 \wedge \frac{\norm{\x}}{\norm{\q-\x}}}\defEns{1+  \norm{\x} - t \norm{\q-\x}}^{m-2}  \rmd t
% \\
% & \phantom{\defEns{1+  \norm{\x} - t \norm{\q-\x}}^{m-2}  \rmd t}+ \int_{1 \wedge \frac{\norm{\x}}{\norm{\q-\x}}}^1 \defEns{1+ t \norm{\q-\x}-\norm{\x}}^{m-2}  \rmd t \eqsp.
% \end{align}
% Then if $\norm{\q-\x} \leq \norm{\x}$, we get
% \begin{multline}
%   \label{eq:lem_grad_Lip_F_eq_1}
%  \int_{0}^1 \defEns{1+\norm{\x +t(\q-\x)}}^{\m-2} \rmd t \leq  \int_{0}^1 \defEns{(1-t)\norm{\q-\x}}^{\m-2} \rmd t \\
% \leq (m-1)^{-1} \norm{\q-\x}^{m-2}\eqsp.
% \end{multline}
% If $\norm{\q-\x} \geq \norm{\x}$, we get
% \begin{align}
% \nonumber
%  &\int_{0}^1 \defEns{1+\norm{\x +t(\q-\x)}}^{\m-2} \rmd t \leq \int_{0}^{1 \wedge \frac{\norm{\x}}{\norm{\q-\x}}}\defEns{1+  \norm{\x} - t \norm{\q-\x}}^{m-2}  \rmd t
% \\
% \nonumber
% & \phantom{\defEns{  \norm{\x} - t \norm{\q-\x}}^{m-2}  \rmd t}+ \int_{1 \wedge \frac{\norm{\x}}{\norm{\q-\x}}}^1 \defEns{ t \norm{\q-\x}-\norm{\x}}^{m-2}  \rmd t \\
%   \label{eq:lem_grad_Lip_F_eq_2}
% & \leq ((m-1)\norm{\q-\x})^{-1} \defEns{\norm{x}^{m-1} + \norm{\q-\x}^{m-1}} \leq (m-1)^{-1} \norm{\q-\x}^{m-2} \eqsp.
% \end{align}
% Combining \eqref{eq:lem_grad_Lip_F_eq_1} and \eqref{eq:lem_grad_Lip_F_eq_2} in \eqref{eq:lem_grad_Lip_F_eq_0} concludes the proof.

% The proof is postponed to \Cref{sec:proof-crefl-1}.
% \end{proof}

%     Let $\q,\x \in \rset^d$ and assume without loss of generality that
%   $\norm{\q} < \norm{\x}$.  First by
%   \Cref{assum:potential}-\ref{assum:potential:a} and Jensen
%   inequality, using that $\m \leq 2$, $t \mapsto t^{\m-2}$ is concave on $\rset_+$,
%   we have
% \begin{align}
% \nonumber
%   \norm{\nabla \F(\q) - \nabla \F(\x)} &= \norm{\int_{0}^1 \nabla^2 \F(\x +t(\q-\x)) \defEns{\q-\x} \rmd t } \\
% \nonumber
% & \leq \constone  \norm{\q-\x} \int_{0}^1 \defEns{1+\norm{\x +t(\q-\x)}}^{\m-2} \rmd t  \\
%   \label{eq:lem_grad_Lip_F_eq_0}
%  & \leq \constone  \norm{\q-\x} \parentheseDeux{ \int_{0}^1 \defEns{1+\norm{\x +t(\q-\x)} }\rmd t }^{\m-2} \eqsp.
% \end{align}
% In addition, we have
% \begin{multline}
% \int_{0}^1 \defEns{1+\norm{\x +t(\q-\x)}} \rmd t \geq \int_{0}^1\defEns{ 1+\abs{\norm{\x} - t \norm{\q-\x}}}  \rmd t  \\
% \geq 1+ \int_{0}^{1 \wedge \frac{\norm{\x}}{\norm{\q-\x}}}\defEns{ \norm{\x} - t \norm{\q-\x}}  \rmd t
% + \int_{1 \wedge \frac{\norm{\x}}{\norm{\q-\x}}}^1 \defEns{ t \norm{\q-\x}-\norm{\x}}  \rmd t \eqsp.
% \end{multline}
% If $\norm{\q-\x} \leq \norm{\x}$, then we get
% \begin{equation}
%   \label{eq:lem_grad_Lip_F_eq_1}
%   \int_{0}^1 \defEns{1+\norm{\x +t(\q-\x)}} \rmd t
% %\geq 1+\norm{\x} - \norm{\q-\x}/2 \\
% \geq 1+\norm{\q-\x}/2 \eqsp.
% \end{equation}
% If $\norm{\q-\x} \geq \norm{\x}$, by the triangle inequality it necessarily  holds that $\norm{\x} \in \ccint{\norm{\q-\x}/2, \norm{\q-\x}}$ since $\norm{\q} \leq \norm{\x}$. Therefore we get
% \begin{equation}
%   \label{eq:lem_grad_Lip_F_eq_2}
%   \int_{0}^1 \defEns{1+\norm{\x +t(\q-\x)}} \rmd t
% %\geq 1+ \int_{0}^{\min(1,\norm{y}/\norm{\q-\x})}\defEns{ \norm{\x} - t \norm{\q-\x}}  \rmd t \\
%  \geq 1+\norm{\x}^2/(2 \norm{\q-\x})
% \geq 1+\norm{\q-\x}/8 \eqsp.
% \end{equation}
% Since $\m \leq 2$, then combining \eqref{eq:lem_grad_Lip_F_eq_1} and \eqref{eq:lem_grad_Lip_F_eq_2} in \eqref{eq:lem_grad_Lip_F_eq_0}, we get
% \begin{equation}
%    \norm{\nabla \F(\q) - \nabla \F(\x)} \leq  \constone  \norm{\q-\x}\defEns{1+ \norm{\q-\x}/8}^{\m-2} \eqsp,
% \end{equation}
% which concludes the proof.
\end{proof}

\begin{lemma}
  \label{lem:prepa_bound_diff_ham}
Assume  \Cref{assum:regOne}$(\beta)$ for $\beta \in \ocint{0,1}$. Let $\gamma \in \ooint{0,\beta}$.
\begin{enumerate}[label=(\roman*)]
\item   \label{lem:prepa_bound_diff_ham_1}
If $ \beta \in \ooint{0,1}$, for any $T \in \nsets$ and  $h_0 \in \rset^*_+$, there exist $\kappa \in \rset^*_+$ and $R \in \rset_+$ such that for all $h \in \ocint{0,h_0}$,  $\q_0,p_0 \in \rset^d$ satisfying $ \norm{p_0} \leq
\norm{\q_0}^{\gamma}$ and $\norm{\q_0} \geq R$, and $i,j,k \in
\{0,\ldots,T\}$,
\begin{align}
\label{eq:bound_iterate_q_2_prood_diff_ham_eq}
%\label{eq:bound_iterate_q_1_prood_diff_ham_1}
&\norm{q_0} \leq \kappa  \norm{\Phiverletq[h][k](q_0,p_0)}\eqsp, \\ &\norm{\Phiverletq[h][i](q_0,p_0)-\Phiverletq[h][j](q_0,p_0)} \leq \kappa h \norm{\Phiverletq[h][k](q_0,p_0)}^{\beta}  \eqsp,
\end{align}
  where $\Phiverletq[h][\ell]$ are defined by \eqref{eq:def_Phiverletq} for $\ell \in \nset^*$.
\item \label{lem:prepa_bound_diff_ham_2}
If $\beta =1$, then there exist $\kappa, \bar{S} \in \rset_+^*$ (depending only on $\constzero$ and $\constzeroT$) such that for any $T \in \nsets$, $h \in \ooint{0,\bar{S}/T}$,  $\q_0,p_0 \in \rset^d$ satisfying $ \norm{p_0} \leq
\norm{\q_0}^{\gamma}$ and $\norm{\q_0} \geq 1$, and $i,j,k \in
\{0,\ldots,T\}$,
\begin{align}
\label{eq:bound_iterate_q_2_prood_diff_ham_eq_2}
&\norm{q_0} \leq 2  \norm{\Phiverletq[h][k](q_0,p_0)} \leq 3 \norm{q_0} \eqsp, \\
&\norm{\Phiverletq[h][i](q_0,p_0)-\Phiverletq[h][j](q_0,p_0)}
\leq \kappa Th
\rme^{(1+\vartheta_1(Th))Th}  \norm{\Phiverletq[h][k](q_0,p_0)} \eqsp,
\end{align}
where $\vartheta_1$ is defined in \eqref{eq:def_vartheta_1}.
\end{enumerate}
\end{lemma}

\begin{proof}
\begin{enumerate}[label=(\roman*),leftmargin=0cm,itemindent=0.5cm,labelwidth=1.2\itemindent,labelsep=0cm,align=left]
\item
Let $T \in \nsets$, $h_0 \in \rset_+^*$ and  $h \in \ocint{0,h_0}$.
Denote for all $k \in \{0,\ldots,T\}$ by $(q_k,p_k) =
  \Phiverlet[h][k](q_0,p_0)$, $q_0, p_0 \in \rset^d$.
  By  \Cref{assum:regOne}$(\beta)$ and  \Cref{lem:bound_first_iterate_leapfrog_b}-\ref{lem:bound_first_iterate_leapfrog_1}, there exist $C \geq 0$ and  $R_1 \geq 0$ such that for all $\q_0,\p_0 \in \rset^d$ satisfying $ \norm{\p_0} \leq \norm{\q_0}^{\gamma}$ and $\norm{\q_0} \geq R_1$, for all $k \in \{0,\ldots,T\}$, we have
\begin{equation}\label{eq:bound_iterate_q_1_prood_diff_ham_0}
 \norm{\q_k-\q_0} \leq C h \norm{\q_0}^{\m-1} \eqsp.
\end{equation}
Then since $m<2$, there exists $R_2 \geq R_1$ and $\omega >0$ such that such that for all $\q_0,p_0 \in \rset^d$ satisfying $ \norm{p_0} \leq \norm{\q_0}^{\gamma}$  and $\norm{\q_0} \geq R_2$, for all $k \in \{0,\ldots,T\}$,
\begin{equation}
  \norm{\q_0} \leq \omega \norm{\q_k} \eqsp.
\end{equation}
In addition, using this inequality and
\eqref{eq:bound_iterate_q_1_prood_diff_ham_0} again, we get that for
all $\q_0,p_0 \in \rset^d$ satisfying $ \norm{p_0} \leq
\norm{\q_0}^{\gamma}$ and $\norm{\q_0} \geq R_2$, for all $i,j,k \in
\{0,\ldots,T\}$,
\begin{equation}
\label{eq:bound_iterate_q_2_prood_diff_ham_2}
\norm{\q_{i}-\q_{j}} \leq 2C h \norm{\q_0}^{\m-1} \leq 2 Ch \omega^{\m-1}\norm{\q_k}^{\m-1}  \eqsp.
\end{equation}
%
\item Let $T \in \nsets$, $h \in \rset_+^*$.
Denote for all $k \in \{0,\ldots,T\}$ by $(q_k,p_k) =
  \Phiverlet[h][k](q_0,p_0)$, $q_0, p_0 \in \rset^d$.
 By  \Cref{assum:regOne}$(1)$ and \Cref{lem:bound_first_iterate_leapfrog_b}-\ref{lem:bound_first_iterate_leapfrog_b_2}, $\vartheta(s) \geq 1$ for any $s \geq 0$, we get that  for all $\q_0, \p_0 \in \rset^{2d}$ satisfying $\norm{\p_0} \leq \norm{\q_0}^\gamma$ and $k \in \{0,\ldots,T\}$,
\begin{equation}\label{eq:bound_iterate_q_1_prood_diff_ham_0_m_2}
 \norm{\q_k-\q_0} \leq \constzero^{-1/2}\left\{(1 + h \constzero^{1/2} \vartheta_1(h \constzero^{1/2}))^{k+1}-1\right\} \left\{ \vartheta_2(h) \vee \vartheta_3(h) \right\} (1 + \norm{\q_0}) \eqsp,
\end{equation}
where $\vartheta_1$, $\vartheta_2$ and $\vartheta_3$ are defined in \eqref{eq:def_vartheta_1} and  \eqref{eq:definition-vartheta-2} respectively.

Therefore,  there exists $\bar{S} >0$ (depending only on $\constzero$ and $\constzeroT$) such that for any $T \in \nsets$ and $h \in \ooint{0,\bar{S}/T}$, for any $q_0,p_0 \in \rset^d$ satisfying $ \norm{\p_0} \leq \norm{\q_0}^{\gamma}$ and $\norm{\q_0} \geq  1 $, $ \normLigne{\Phiverletq[h][k](q_0,p_0)-\q_0} \leq \norm{q_0}/2$ for any $k \in \{0,\ldots,T\}$. As a result, for any $T \in \nsets$ and $h \in \ooint{0,\bar{S}/T}$, for any $q_0,p_0 \in \rset^d$ satisfying $ \norm{\p_0} \leq \norm{\q_0}^{\gamma}$ and $\norm{\q_0} \geq  1 $,  for any $k \in \{0,\ldots,T\}$,
\begin{equation}
  \norm{\q_0} \leq 2 \norm{\Phiverletq[h][k](q_0,p_0)} \leq 3 \norm{q_0} \eqsp.
\end{equation}
In addition, using this inequality and
\eqref{eq:bound_iterate_q_1_prood_diff_ham_0_m_2} again, we get that there exists $C \geq 1$ (depending only on $\constzero$ and $\constzeroT$) such that  for any $T \in \nsets$ and $h \in \ooint{0,\bar{S}/T}$, setting $S= hT$, and  for
all $\q_0,p_0 \in \rset^d$ satisfying $ \norm{p_0} \leq
\norm{\q_0}^{\gamma}$ and $\norm{\q_0} \geq 1$, for all $i,j,k \in
\{0,\ldots,T\}$,
\begin{align}
%\label{eq:bound_iterate_q_2_prood_diff_ham_2}
\norm{\Phiverletq[h][i](q_0,p_0) - \Phiverletq[h][j](q_0,p_0)} 
&\leq 2C S \rme^{(1+\vartheta_1(S))S} \norm{\q_0} \\
&\leq 4 C S \rme^{(1+\vartheta_1(S))S}\norm{\Phiverletq[h][k](q_0,p_0)}  \eqsp.
\end{align}
\end{enumerate}
\end{proof}

\begin{lemma}
\label{lem:variation_assum_hessian}
Assume \Cref{assum:potential}($\m$) for some $m \in \ocint{1,2}$.
Then there exist $\delta \in \ooint{0,1}$, $R_0 \in \rset_+$ and  $\BB_0 \in \rset_+^*$ such that for all $\q,\x,z \in \rset^d$, with
\begin{equation}
\label{eq:hyp_variation_assum_hessian}
\norm{\q} \geq R_0 \eqsp, \qquad \text{ and } \quad \max\parenthese{\norm{\q-\x},\norm{\q-z}} \leq \delta \norm{\q}  \eqsp,
\end{equation}
we have
\begin{equation}
D^2 \F(\q) \defEns{\nabla \F(\x) \otimes \nabla \F(z)} \geq \BB_0 \norm{\q}^{3\m-4}  \eqsp.
\end{equation}
\end{lemma}
\begin{proof}
Under \Cref{assum:potential}($m$), using \Cref{lem:grad_Lip_F}, it can be easily checked that there
exists $C_U \geq 0$ (depending only on $\constone$ and $m$) such that for all  $\q,\x,z \in \rset^d$ satisfying  \eqref{eq:hyp_variation_assum_hessian},
for $\delta \in \ooint{0,1}$ and $R_0 \geq \rhtwo$,
\begin{equation}
D^2 \F(\q) \defEns{\nabla \F(\x) \otimes \nabla \F(z)}  \geq \consttwo  \norm{\q}^{3\m-4} - C_U \{1+ \delta^{m-1} \norm{\q}^{3m-4}\}\eqsp.
\end{equation}
The proof is concluded by taking $\delta$ sufficiently small and $R_0$ sufficiently large.
\end{proof}

% \alain{comment on the Gaussian case the term $\frac{h^4}{8}\norm{ \int_0^1 \nabla ^2 \F( q_{t}) p_{0} \ \rmd t  }^2$ implies that $h$ has to be chosen at least smaller than $C (Ld^{-1/2})^{-1/2}$}
 \begin{lemma}
 \label{lem:diff_hamiltonian_taylor_exp}
 Assume that $\F$ is twice continuously differentiable. Then for all $q_0,p_0 \in \rset^d$ and $h \in \rset^*_+$, the following identity holds
 \begin{align}
%&    \Ham(q_1,p_1) - \Ham(q_0,p_0)   \\
& \Ham \circ \Phiverlet[h][1](q_0,p_0) - \Ham(q_0,p_0) =  h^2\int_0^1D^2\F(q_t)\defEns{p_{0}}^{\otimes 2} (1/2-t) \  \rmd t \\
& + h^3 \int_0^1D^2\F(q_t)\defEns{p_{0}\otimes \nabla \F(q_{0})}(t-1/4) \ \rmd t  \\
&  -\frac{h^4}{4}\int_0^1 D^2\F(q_t)\defEns{\nabla \F(q_{0})}^{\otimes 2} \ t \  \rmd t
+\frac{h^4}{8}\norm{ \int_0^1 \nabla ^2 \F( q_{t}) p_{0} \ \rmd t  }^2\\
&   - \frac{h^5}{8}\ps{\int_0^1\nabla^2\F(q_t)\nabla \F(q_0) \ \rmd t }{\int_0^1 \nabla^2 \F(q_t)p_0 \ \rmd t }
   \\
   &+\frac{h^6}{32}\norm{\int_0^1 \nabla^2 \F( q_{t})\nabla \F (q_0) \ \rmd t }^2
\eqsp,
\end{align}
where  $\Phiverlet[h][1]$ is defined in \eqref{eq:def_Phiverlet}, $(q_1,p_1) = \Phiverlet[h][1](q_0,p_0)$, and $q_t = q_0 +t (q_1-q_0)$ for $t \in \ccint{0,1}$.
\end{lemma}
\begin{proof}
Using the definition of $\Ham(q,p)= \frac{1}{2}\norm{p}^2+ \F(q)$, we get
\begin{equation}
  \Ham(q_1,p_1) - \Ham(q_0,p_0) = (1/2)(\norm{p_1}^2 - \norm{p_0}^2) + \F(q_1) - \F(q_0) \eqsp.
\end{equation}
First, Taylor's formula  with exact remainder  enables us to write
\begin{equation}\label{eq:1}
\F(q_{1})-\F(q_0)= \ps{\nabla \F(q_0)}{ (q_{1}-q_0)}+\int_0^1D^2\F(q_{t})\defEns{q_{1}-q_0 }^{\otimes 2}(1-t) \ \rmd t \eqsp.
\end{equation}
Since  $\nabla \F(q_{1}) = \nabla \F(q_0) + \int_{0}^1 \nabla^2 \F(q_t) \defEns{q_1-q_0} \rmd t $,  we get
 \begin{equation}\label{eq:2}
p_{1}=p_0-\frac{h}{2} \parenthese{\nabla \F(q_{0})+\nabla \F(q_{1})}= p_{0}- h\nabla \F(q_0)-\frac{h}{2}\int_0^1\nabla ^2\F( q_{t}) \defEns{q_{1}-q_{0}} \rmd t \eqsp.
\end{equation}
Using that $q_1 = \Phiverletq[h][1](p_0,q_0)$, with $\Phiverletq[h][1]$ defined by \eqref{eq:def_Phiverletq}, in \eqref{eq:1} and \eqref{eq:2}, we get
\begin{align}
  &\F(q_{1})-\F(q_{0})\\
&=\ps{\nabla \F(q_{0})}{ hp_{0}-(h^2/2)\nabla \F(q_{0})} + \int_0^1D^2\F(q_t)\defEns{q_{1}-q_{0}}^{\otimes 2}(1-t) \rmd t   \eqsp,
\end{align}
and
\begin{align}
&\frac{1}{2}(\norm{p_{1}}^2-\norm{p_{0}}^2)= \frac{h^2}{2}\norm{\nabla \F(q_{0})}^2+ \frac{h^2}{8} \norm{\int_0^1 \nabla^2\F(q_t)\defEns{q_{1}-q_{0} }\rmd t }^2\\
&- h\langle p_{0},\nabla \F(q_{0})\rangle -(h/2)\int_0^1 D^2\F(q_t)\defEns{p_{0}\otimes (q_{1}-q_{0})} \rmd t \\
&+ (h^2/2)\int_0^1D^2\F(q_t) \defEns{\nabla \F(q_{0})\otimes (q_{1}-q_{0})} \rmd t \eqsp.
\end{align}
Summing these equalities up  and observing appropriate cancellations yields
\begin{flalign}
\nonumber
&H(q_{1},p_{1})-H(q_{0},p_{0})=\int_0^1D^2\F(q_t)\defEns{q_{1}-q_{0}}^{\otimes 2}(1-t) \rmd t\\
\nonumber
& - (h/2)\int_0^1D^2\F(q_t)\defEns{p_{0}\otimes (q_{1}-q_{0})} \rmd t +   (h^2/8) \norm{ \int_0^1 \nabla ^2\F(q_t)\defEns{q_{1}-q_{0}} \rmd t}^2\\
\nonumber
&+ (h^2/2)\int_0^1D^2\F(q_t)\defEns{\nabla \F(q_{0})\otimes (q_{1}-q_{0})} \rmd t
\end{flalign}
\vspace{-0.8cm}
\begin{equation}
\label{eq:decomp_lem_hamil}
= I_1+I_2+I_3+I_4 \eqsp.
\end{equation}
By using $q_1 = \Phiverletq[h][1](p_0,q_0)$ again in the definition of each $I_j$ we obtain successively
\begin{align}
I_1&=h^2\int_0^1D^2\F( q_t)\defEns{p_0}^{\otimes 2} (1-t) \rmd t- h^3 \int_0^1D^2\F(q_t)\defEns{p_{0}\otimes \nabla \F(q_{0})}(1-t)\rmd t\\
&\qquad \qquad \qquad +(h^4/4)\int_0^1D^2\F(q_t) \defEns{\nabla \F(q_{0})}^{\otimes 2}(1-t)\rmd t \eqsp, \\
I_2&= - (h^2/2)\int_0^1D^2\F(q_t) \defEns{p_{0}}^{\otimes 2} \rmd t + (h^3/4) \int_0^1D^2\F(q_t)\defEns{p_{0}\otimes \nabla   \F(q_{0})} \rmd t \eqsp,\\
I_3&= (h^4/8) \norm{ \int_0^1\nabla^2 \F(q_t)p_{0} \ \rmd t }^2+(h^6/32) \norm{\int_0^1\nabla^2\F(q_t)\nabla \F(q_{0})\  \rmd t }^2
\\
& \qquad \qquad \qquad  - (h^5/8)\ps{\int_0^1\nabla^2\F(q_t)\nabla \F(q_0) \ \rmd t }{\int_0^1 \nabla^2 \F(q_t)p_0 \ \rmd t } \eqsp.
\end{align}
and
\begin{align}
  I_4& = (h^3/2)\int_0^1D^2\F(q_t)\defEns{\nabla \F(q_{0})\otimes p_{0}}\rmd t \\
  & \qquad \qquad \qquad \qquad  \qquad \qquad - (h^4/4)\int_0^1D^2\F(q_t)\defEns{\nabla \F(q_{0})}^{\otimes 2}\rmd t \eqsp,
\end{align}
Gathering all these equalities  in \eqref{eq:decomp_lem_hamil} concludes the proof.
 \end{proof}


\begin{proof}[Proof of \Cref  {propo:accept}]
Let $\gamma \in \ooint{0,m-1}$, $T \in \nsets$, $h_0 \in \rset_+^*$ and  $h \in \ocint{0,h_0}$.
Denote for all $k \in \{0,\ldots,T\}$ by $(q_k,p_k) =
  \Phiverlet[h][k](q_0,p_0)$, $q_0, p_0 \in \rset^d$.
  For all $q_0,p_0 \in \rset^d$, consider the following decomposition
\begin{equation}
  \label{eq:diff_ham_decompo}
H(p_T,q_T)-H(p_0,q_0)=\sum_{k=0}^{T-1}\defEns{H(p_{k+1},q_{k+1})-H(p_{k},q_{k})} \eqsp.
\end{equation}
We show that each term in the sum in the right hand side of this equation is nonpositive if $\norm{\q_0}$ is large enough and $\norm{p_0} \leq \norm{q_0}^{\gamma}$.
By \Cref{lem:diff_hamiltonian_taylor_exp}, we have
\begin{equation}
\label{eq:diff_ham_k}
H(q_{k+1},p_{k+1})-H(q_{k},p_{k})
= -(h^4/4)A_k+h^2 B_k+ h^3C_k+(h^4/8) D_k \eqsp,
\end{equation}
where, setting $q_{t,k}= q_k + t(q_{k+1}-q_k)$ for $t \in \ccint{0,1}$,
\begin{align}
  A_k &=\int_0^1 D^2\F(q_{t,k})\defEns{\nabla \F(q_{k})}^{\otimes 2} \ t \  \rmd t \\
B_k& = \int_0^1D^2\F(q_{t,k})\defEns{p_{k}}^{\otimes 2} (1/2-t) \  \rmd t \\
C_k & = \int_0^1D^2\F(q_{t,k})\defEns{p_{k}\otimes \nabla \F(q_{k})}(t-1/4) \ \rmd t \\
D_k & = \norm{ \int_0^1 \nabla ^2 \F( q_{t,k}) p_{k} \ \rmd t  }^2 +(h^2/4)\norm{\int_0^1 \nabla^2 \F( q_{t,k})\nabla \F (q_{k}) \ \rmd t }^2 \\
& \qquad \qquad -  h \ps{\int_0^1\nabla^2\F(q_{t,k})\nabla \F(q_{k}) \ \rmd t }{\int_0^1 \nabla^2 \F(q_{t,k})p_{k} \ \rmd t }
\end{align}
Since $q_{t,k}-q_k= -(t h^2/2) \nabla \F(q_k) + th p_k$ and $\int_{0}^1(1/2-t) \, \rmd t = 0$, we have
for all $q_0,p_0 \in \rset^d$,
\begin{align}
B_k &= \int_{0}^1 \int_0^1 D^3 \F(q_{k} +s(q_{t,k} - q_{k})) \defEns{p_{k}^{\otimes 2} \otimes (q_{t,k}-q_{k})} (1/2-t)\rmd s \ \rmd t 
  \\
  \label{eq:definition-B_k}
    &= h B_{k,1} - h^2 B_{k,2}
\end{align}
where
\begin{align}
\label{eq:definition-B_k-1}
B_{k,1} &= \int_{0}^1 \int_0^1 D^3 \F(q_{k} +s(q_{t,k} - q_{k})) \defEns{p_{k}}^{\otimes 3}t (1/2-t)\rmd s \ \rmd t \\
\label{eq:definition-B_k-2}
B_{k,2} &=  -\frac{1}{2} \int_{0}^1 \int_0^1 D^3 \F(q_{k} +s(q_{t,k} - q_{k})) \defEns{p_{k}^{\otimes 2} \otimes \nabla \F(q_k) } t (1/2-t)\rmd s \ \rmd t \eqsp.
\end{align}
Consider now the term $C_k$ in \eqref{eq:diff_ham_k}. Similarly, using again $\int_0^1 (t- 1/2) \rmd t= 0$ and then \eqref{eq:pk}, we get $C_k = C_{k,1} +  C_{k,2} + C_{k,3}$,  where
\begin{align}
\label{eq:definition-C_k-1}
&C_{k,1} = h \int_{0}^1 \int_{0}^t D^3 \F(q_{k} +s(q_{k,t} - q_{k})) \defEns{p_{k}^{\otimes 2} \otimes \nabla \F(q_k) } t (t-1/2)\rmd s \ \rmd t \\
\nonumber
& -(h^2/2)\int_{0}^1 \int_{0}^t  D^3 \F(q_{k} +s(q_{k,t} - q_{k})) \defEns{p_{k} \otimes \parenthese{\nabla \F(q_k)}^{\otimes 2} } t (t-1/2)\rmd s \ \rmd t \\
\label{eq:definition-C_k-2}
&C_{k,2} =   \int_0^1D^2\F(q_{t,k})\defEns{p_{0}\otimes \nabla \F(q_{k})} \ \rmd t \eqsp, \\
\label{eq:definition-C_k-3}
&C_{k,3} = -h \sum_{i=1}^{k-1}\int_0^1D^2\F(q_{t,k})\defEns{\nabla \F(q_i)\otimes \nabla \F(q_{k})} \ \rmd t
\\
&\qquad \qquad  \qquad \qquad - (h/2)\int_0^1D^2\F(\q_{t,k})\defEns{\parenthese{\nabla \F(\q_0)+\nabla \F (\q_k)} \otimes \nabla \F(\q_{k})} \ \rmd t
\end{align}
We will next estimate each of these terms separately.
Let $\delta \in \ooint{0,1}$ and $\BB_0 \in \rset_+^*$ be the constants defined in \Cref{lem:variation_assum_hessian}.


\begin{enumerate}[label=(\alph*),leftmargin=0cm,itemindent=0.5cm,labelwidth=1.2\itemindent,labelsep=0cm,align=left]
\item
We first consider the case $m\in \ooint{1,2}$.
  By \Cref{lem:grad_Lip_F} and \Cref{lem:bound_first_iterate_leapfrog_b}-\ref{lem:bound_first_iterate_leapfrog_1}, there exist $C \geq 0$ and  $R_1 \geq \rhtwo$ such that for all $\q_0,p_0 \in \rset^d$ satisfying $ \norm{p_0} \leq
\norm{\q_0}^{\gamma}$ and $\norm{\q_0} \geq R_1$, for all $i \in
\{0,\ldots,T\}$,
\begin{equation}
\label{eq:bound_iterate_q_3_prood_diff_ham_3_0}
\begin{aligned}
&\norm{\q_{i}-\q_{0}}\leq (\delta/2) \norm{\q_0} \\ 
&\norm{\p_i - \p_0} \leq C(\norm{\p_0} + h \norm{q_0}^{m-1}) \leq C(\norm{\q_0}^\gamma + h \norm{q_0}^{m-1}) \eqsp.
\end{aligned}
\end{equation}
By  \Cref{lem:variation_assum_hessian}, \Cref{lem:prepa_bound_diff_ham}-\ref{lem:prepa_bound_diff_ham_1} and \eqref{eq:bound_iterate_q_3_prood_diff_ham_3_0}, there exists $R_2 \geq R_1$ such that for all $\q_0,\p_0 \in \rset^d$, $\norm{\q_0} \geq R_2$ and $\norm{\p_0} \leq \norm{\q_0}^{\gamma}$,
we get that
\begin{equation}
\label{eq:bound_A_k}
  \inf_{\norm{p_0} \leq \norm{q_0}^{\gamma}} A_k \geq \BB_0 \norm{q_k}^{3\m-4} \geq
  \BB_0 \{ (1-\delta/2)^{3m-4} \wedge (1+\delta/2)^{3m-4} \} \norm{q_0}^{3\m-4} \eqsp.
\end{equation}
Hence, $\limsup_{\norm{q_0} \to \plusinfty} \sup_{\norm{p_0} \leq \norm{q_0}^{\gamma}}\defEns{A_k/\norm{q_0}^{3\m-4}} > 0$.
We now bound $B_k$. Using \Cref{assum:potential}-\ref{assum:potential:a}, \Cref{lem:grad_Lip_F} and \eqref{eq:bound_iterate_q_3_prood_diff_ham_3_0}, we get by \eqref{eq:definition-B_k} that
\begin{equation}
\label{eq:bound_B_k}
\limsup_{\norm{q_0} \to \plusinfty} \sup_{\norm{p_0} \leq \norm{q_0}^{\gamma}}\defEns{\abs{B_k}/\norm{q_0}^{4\m-6}} < \infty \eqsp.
\end{equation}
Combining \Cref{assum:potential}-\ref{assum:potential:a}, \Cref{lem:grad_Lip_F} and \eqref{eq:bound_iterate_q_3_prood_diff_ham_3_0} again, we get by crude estimate that there exists $C \geq 0$ such that
\begin{equation}
  \label{eq:bound_D_k}
  \limsup_{\norm{q_0} \to \plusinfty} \sup_{\norm{p_0} \leq \norm{q_0}^{\gamma}} \defEns{\abs{D_k}/\norm{q_0}^{4\m-6}} \leq C h^2 \eqsp.
\end{equation}
We finally bound the two terms $C_{k,1}$ and $C_{k,2}$. First,
using the same reasoning as for $B_k$, we get that
\begin{equation}
%\label{eq:bound_C_k_1}
\begin{aligned}
  &\limsup_{\norm{q_0} \to \plusinfty} \sup_{\norm{p_0} \leq \norm{q_0}^{\gamma}} \defEns{\abs{C_{k,1}}/\norm{q_0}^{4\m-6}} < \infty \eqsp, \\
  &\limsup_{\norm{q_0} \to \plusinfty} \sup_{\norm{p_0} \leq \norm{q_0}^{\gamma}} \defEns{\abs{C_{k,2}}/\norm{q_0}^{2\m-3+\gamma}} < \infty \eqsp.
\end{aligned}
\end{equation}
Arguing like in \eqref{eq:bound_A_k}, we get that
$\limsup_{\norm{q_0} \to \plusinfty}  \sup_{\norm{p_0} \leq \norm{q_0}^{\gamma}} \defEns{C_{k,3}/\norm{q_0}^{3\m-4}} < 0$.
Gathering all these results and  using that  $3m-4 \geq \max(4m-6,2m-3+\gamma)$ for $m \in \ooint{1,2}$ and $\gamma \in \ooint{0,m-1}$, we get that for all $k\in \{0, \ldots,T-1\}$,
\begin{equation}
%  \label{eq:bound_C_k}
  \limsup_{\norm{q_0} \to \plusinfty}  \sup_{\norm{p_0} \leq \norm{q_0}^{\gamma}} \defEns{\Ham(q_{k+1},p_{k+1})-\Ham(q_{k},p_{k})}/\norm{q_0}^{3m-4} < 0 \eqsp,
\end{equation}
which concludes the proof.
\item
  Consider now the case $\m=2$.
First   by \Cref{lem:grad_Lip_F} and \Cref{lem:bound_first_iterate_leapfrog_b}-\ref{lem:bound_first_iterate_leapfrog_b_2}, there exist $\bar{S}_1 \geq 0$ and  $R_1 \geq \rhtwo$ such that for all $T \in \nsets$ and $h \in \ocint{0,\bar{S}_1/T}$, $\q_0,p_0 \in \rset^d$ such that $ \norm{p_0} \leq
\norm{\q_0}^{\gamma}$ and $\norm{\q_0} \geq R_1$, and $i \in
\{0,\ldots,T\}$,
\begin{equation}\label{eq:bound_iterate_q_3_prood_diff_ham_3_0_2}
\norm{\q_{i}-\q_{0}}\leq (\delta/2) \norm{\q_0}  \eqsp.
\end{equation}
and
\begin{equation}
\label{eq:bound_iterate_q_3_prood_diff_ham_3_0_3}
\begin{aligned}
\norm{\q_{i}-\q_{0}}&\leq \norm{p_0} Th + (1/2) (T+1)^2 h^2 (\constzeroT + \constzero \delta/2) \norm{q_0}\eqsp, \\
 \norm{\p_i - \p_0} &\leq h T \{\constzeroT + (\constzeroT + \constzero \delta/2) \norm{q_0}  \} \eqsp,
\end{aligned}
\end{equation}
where $\constzero$ and $\constzeroT$ are defined in \Cref{lem:grad_Lip_F}. By  \Cref{lem:variation_assum_hessian}, \Cref{lem:prepa_bound_diff_ham}-\ref{lem:prepa_bound_diff_ham_2} and \eqref{eq:bound_iterate_q_3_prood_diff_ham_3_0_2}, there exists $R_2 \geq R_1$ such that for all $\q_0,\p_0 \in \rset^d$, $\norm{\q_0} \geq R_2$ and $\norm{\p_0} \leq \norm{\q_0}^{\gamma}$
\begin{equation}
\label{eq:bound_A_k_2}
  \inf_{\norm{p_0} \leq \norm{q_0}^{\gamma}} A_k \geq \BB_0 \norm{q_k}^{2} \geq
  \BB_0 (1-\delta/2)^{2} \norm{q_0}^{2} \eqsp.
\end{equation}
Hence,
\begin{equation}
\label{eq:bound_A_k_2_2}
\limsup_{\norm{q_0} \to \plusinfty} \sup_{\norm{p_0} \leq \norm{q_0}^{\gamma}}\defEns{A_k/\norm{q_0}^{2}} \geq   \BB_0 (1-\delta/2)^{2}  \eqsp.
\end{equation}
We now bound $B_k$. Using \Cref{assum:potential}-\ref{assum:potential:a}, \Cref{lem:grad_Lip_F} and \eqref{eq:bound_iterate_q_3_prood_diff_ham_3_0_3}, we get by \eqref{eq:definition-B_k} that there exists $\rmD_1 \geq 0$ which does not depend on $T$ and $h$ such that
\begin{equation}
\label{eq:bound_B_k_2}
\limsup_{\norm{q_0} \to \plusinfty} \sup_{\norm{p_0} \leq \norm{q_0}^{\gamma}}\defEns{\abs{B_k}/\norm{q_0}^{2}} \leq \rmD_1 h  \{(hT)^3 + (hT)^4\}  \eqsp.
\end{equation}
Combining \Cref{assum:potential}-\ref{assum:potential:a}, \Cref{lem:grad_Lip_F} and \eqref{eq:bound_iterate_q_3_prood_diff_ham_3_0_3} again, we get by crude estimate that there exists $\rmD_2 \geq 0$ which does not depend on $T$ and $h$ such that
\begin{equation}
  \label{eq:bound_D_k_2}
  \limsup_{\norm{q_0} \to \plusinfty} \sup_{\norm{p_0} \leq \norm{q_0}^{\gamma}} \defEns{\abs{D_k}/\norm{q_0}^{2}} \leq \rmD_2 (hT)^2 \eqsp.
\end{equation}
We finally bound the two terms $C_{k,1}$ and $C_{k,2}$. First,
using the same reasoning as for $B_k$, we get that there exists $\rmD_3 \geq 0$ which does not depend on $T$ and $h$ such that
\begin{equation}
\label{eq:bound_C_k_2}
\begin{aligned}
&\limsup_{\norm{q_0} \to \plusinfty} \sup_{\norm{p_0} \leq \norm{q_0}^{\gamma}} \defEns{\abs{C_{k,1}}/\norm{q_0}^{2}} < \rmD_3 h \{(hT)^4 + (hT)^5\} \eqsp,\\
&\limsup_{\norm{q_0} \to \plusinfty} \sup_{\norm{p_0} \leq \norm{q_0}^{\gamma}} \defEns{\abs{C_{k,2}}/\norm{q_0}^{1+\gamma}} < \infty \eqsp.
\end{aligned}
\end{equation}
Finally, arguing like in \eqref{eq:bound_A_k_2_2}, we get that
\begin{equation}
\label{eq:bound_C_k_2_2}
  \limsup_{\norm{q_0} \to \plusinfty}  \sup_{\norm{p_0} \leq \norm{q_0}^{\gamma}} \defEns{C_{k,3}/\norm{q_0}^{2}} < 0 \eqsp.
\end{equation}
Combining \eqref{eq:bound_A_k_2_2}-\eqref{eq:bound_B_k_2}-\eqref{eq:bound_D_k_2}-\eqref{eq:bound_C_k_2} and \eqref{eq:bound_C_k_2_2} in  \eqref{eq:diff_ham_k}, and using that  $2 \geq 1+ \gamma $ for  $\gamma \in \ooint{0,1}$, we get that for all $k\in \{0, \ldots,T-1\}$,
\begin{align}
  &  \limsup_{\norm{q_0} \to \plusinfty}  \sup_{\norm{p_0} \leq \norm{q_0}^{\gamma}} \defEns{\Ham(q_{k+1},p_{k+1})-\Ham(q_{k},p_{k})}/\norm{q_0}^{2} \\
  & \qquad \qquad\qquad \qquad \leq - \BB_0 (1-\delta/2)^{2} h^4 + \rmD_1\{(hT)^3 + (hT)^4\} h^3 \\
  & \qquad \qquad  \qquad \qquad \qquad + \rmD_2(hT)^2h^4 + \rmD_3\{(hT)^4 + (hT)^5\} h^4\eqsp.
%& \qquad \qquad   < - \BB_0 (1-\delta/2)^{2} h^4 + \rmD_1\{(hT)^3 + (hT)^4\} h^3 + \rmD_2\bar{S}_3^2 h^4 + \rmD_3\{ \bar{S}_3^4 + \bar{S}_3^5\} h^4\eqsp.
\end{align}
Therefore, there exists $\bar{S}_4\leq \bar{S}_3$ such for any $T\in \nsets$, $h \in \ocint{0,\bar{S}_4/T^{3/2}}$,
\begin{equation}
%  \label{eq:9}
    \limsup_{\norm{q_0} \to \plusinfty}  \sup_{\norm{p_0} \leq \norm{q_0}^{\gamma}} \defEns{\Ham(q_{k+1},p_{k+1})-\Ham(q_{k},p_{k})}/\norm{q_0}^{2}< 0 \eqsp,
  \end{equation}
  which completes the proof.
\end{enumerate}
\end{proof}

\subsubsection{Proof of \Cref{propo:accept_pertub}}
\label{sec:proof-crefth_accept_2}


\begin{lemma}
\label{lem:variation_assum_hessian_pertub}
Assume \Cref{ass:pertub}.
Then there exist $\delta \in \ooint{0,1}$, $R_1 \in \rset_+$ $\BB_1 \in \rset_+^*$ such that for all $\q,\x,z \in \rset^d$, with
\begin{equation}
\label{eq:hyp_variation_assum_hessian_pertub}
\norm{q} \geq R_1 \eqsp, \qquad  \max\parenthese{\norm{\q-\x},\norm{\q-z}} \leq \delta \norm{\q}  \eqsp,
\end{equation}
we have
\begin{equation}
  % \ps{\Sigmabf^2 \x}{\Sigmabf z} \geq \BB_1 \norm{\q}^{2} \eqsp,  \quad \ps{\Sigmabf \nabla \F(\x)}{\Sigmabf z} \geq \BB_1 \norm{\q}^{2}  \eqsp.
  \ps{\Sigmabf \nabla \F(\x)}{\Sigmabf z} \geq \BB_1 \norm{\q}^{2}  \eqsp.
\end{equation}
\end{lemma}
\begin{proof}
Under \Cref{ass:pertub},  it can be easily checked that there
exists $\tilde{C}_U \geq 0$ (depending only on $\constfive$ and $\Sigmabf$) such that for all  $\q,\x,z \in \rset^d$ satisfying  \eqref{eq:hyp_variation_assum_hessian} for $\delta \in \ooint{0,1}$ and $R_1 \in \rset_+$,
\begin{equation}
% \ps{\Sigmabf^2 x}{ \Sigmabf z  }  \geq \ps{\Sigmabf^2 q}{ \Sigmabf q} - \tilde{C}_U  \delta \norm{\q}^{2} \eqsp, \quad \ps{\Sigmabf \nabla \F(\x)}{\Sigmabf  z}  \geq \ps{\Sigmabf^2 q}{\Sigmabf q} - \tilde{C}_U  (\delta \norm{\q}^{2} + \norm[\rho]{\q})\eqsp.
 \ps{\Sigmabf \nabla \F(\x)}{\Sigmabf  z}  \geq \ps{\Sigmabf^2 q}{\Sigmabf q} - \tilde{C}_U  (\delta \norm{\q}^{2} + \norm[\rho]{\q})\eqsp, \quad \norm{q} \geq R_1 \eqsp.
\end{equation}
The proof is concluded by using that $\Sigmabf$ is definite positive and  taking $\delta$ sufficiently small and $R_1$ sufficiently large.
\end{proof}

\begin{proof}[Proof of \Cref{propo:accept_pertub}]
  Note that by \Cref{ass:pertub}, \Cref{lem:bound_first_iterate_leapfrog_b}-\ref{lem:bound_first_iterate_leapfrog_b_2}, \Cref{lem:prepa_bound_diff_ham}-\ref{lem:prepa_bound_diff_ham_2} and \Cref{lem:variation_assum_hessian_pertub}, there exists $\BB_1,\bar{S}_1 >0$, $R_1 \geq 0$, such that for any $T \in \nsets$, $h \in \ocint{0,\bar{S}_1/T}$, $q_0,p_0 \in \rset^d$, $\norm{p_0} \leq \norm{q_0}^{\gamma}$, $\norm{q_0} \geq \max(1,R_1)$ and $k,i \in \{0, \ldots,T\}$, $\norm{q_0} \leq 2 \norm{q_k} \leq 3 \norm{q_0}$, $\abs{\tilde{U}(q_k)} \leq C_1 \norm{q_0}^{\rho}$, 
  \begin{equation}
\label{eq:proof_pertub_accept_1}
 \ps{\Sigmabf \nabla U(q_i)}{\Sigma q_k} \geq \BB_1 \norm[2]{q_k}  \eqsp, \quad  \norm{\nabla U(q_i)} \leq C_1 \norm{q_k} \eqsp,
  \end{equation}
  where $q_k = \Phiverletq[T][k](q_0,p_0)$ and  $C_1= \max(4\constfive, 3(\norm{\Sigmabf} + 2 \constfive))$.
  Let now $T \in \nsets$, $h \in \ocint{0,\bar{S}_1/T}$ and denote for any $k \in \{0,\ldots,T\}$, $(q_k,p_k) = \Phiverlet[T][k](q_0,p_0)$ for $q_0,p_0 \in \rset^d$. We consider the following decomposition:
  \begin{equation}
\label{eq:proof_pertub_accept_2}
H(p_T,q_T)-H(p_0,q_0)=\sum_{k=0}^{T-1}\defEns{H(p_{k+1},q_{k+1})-H(p_{k},q_{k})} \eqsp.
\end{equation}
We show below that there exists $\bar{S} < \bar{S}_1$ such that, for all $h \geq 0$ and $T \geq 0$ satisfying $hT \leq \bar{S}$,
\begin{equation}
\label{eq:proof_pertub_accept_3}
\limsup_{\norm{q_0} \to \plusinfty} \sup_{\norm{p_0} \leq \norm{q_0}^{\gamma}} [ \defEns{H(p_{k+1},q_{k+1})-H(p_{k},q_{k})}/\norm[2]{q_0}] <0 \eqsp,
\end{equation}
from which  the proof follows.
First for any $q_0,p_0 \in \rset^d$, $k \in \{0,\ldots,T-1\}$, we have
\begin{equation}
\label{eq:proof_pertub_accept_4}
  H(p_{k+1},q_{k+1})-H(p_{k},q_{k}) = A_k + B_k + C_k \eqsp,
\end{equation}
where $2 A_k =  \ps{\Sigmabf q_{k+1}}{q_{k+1}} - \ps{\Sigmabf q_{k}}{q_{k}}$,
$B_k = \tilde{U}(q_{k+1}) - \tilde{U}(q_k)$, and $2 C_k = \norm[2]{p_{k+1}} - \norm[2]{p_k}$.
By \eqref{eq:proof_pertub_accept_1} and \Cref{ass:pertub}, we have
\begin{equation}
\label{eq:proof_pertub_accept_6}
  \lim_{\norm{q_0} \to \plusinfty} \sup_{\norm{p_0} \leq \norm{q_0}^\gamma} \abs{B_k}/\norm[2]{q_0} =0 \eqsp,
\end{equation}
and
\begin{align}
  \label{eq:proof_pertub_accept_7}
  A_k &= h\ps{\Sigmabf p_k}{q_k} + h^2 \ps{\Sigmabf p_k}{p_k}/2 -h^2 \ps{\Sigmabf q_k }{\Sigmabf q_k }/2 \\
  & \qquad \qquad - h^3 \ps{\Sigmabf p_k}{\Sigmabf q_k }/2  +  h^4 \ps{\Sigmabf^2 q_k}{\Sigmabf q_k } /8  + A_{k,1}  \eqsp,
\end{align}
\begin{align}
    C_k & = -h\ps{\Sigmabf p_k}{q_k} -h^2 \ps{\Sigmabf p_k}{p_k}/2 +h^2 \ps{\Sigmabf q_k }{\Sigmabf q_k }/2\\
  & \qquad \quad  +3h^3 \ps{\Sigmabf p_k}{\Sigmabf q_k}/4  + h^4 \ps{\Sigmabf p_k }{\Sigmabf p_k}/8  
    -h^4 \ps{\Sigmabf^2 q_k }{\Sigmabf q_k} /4 \\
  &\qquad \quad -h^5 \ps{\Sigmabf^2 p_k}{q_k}/8 
    + h^6 \ps{\Sigmabf^2 q_k }{\Sigmabf^2 q_k}/32  + C_{k,1} \eqsp,
    \label{eq:proof_pertub_accept_8}
\end{align}
where
\begin{equation}
\label{eq:proof_pertub_accept_9}
  \lim_{\norm{q_0} \to \plusinfty} \sup_{\norm{p_0} \leq \norm{q_0}^\gamma} \{\abs{A_{k,1}} + \abs{C_{k,1}}\}/\norm[2]{q_0} =0 \eqsp,
\end{equation}
Using \eqref{eq:proof_pertub_accept_4}, \eqref{eq:proof_pertub_accept_7} and \eqref{eq:proof_pertub_accept_8}, we obtain that for any $q_0,p_0 \in \rset^d$,
\begin{equation}
  \label{eq:proof_pertub_accept_10}
  H(q_{k+1},p_{k+1}) - H(q_k,p_k) = D_k + A_{k,1}+  B_k + C_{k,1} \eqsp,
\end{equation}
where
\begin{align}
  D_k &= h^3 \ps{\Sigmabf p_k}{\Sigmabf q_k}/4 + h^4 \ps{\Sigmabf p_k }{\Sigmabf p_k}/8   -h^4 \ps{\Sigmabf^2 q_k }{\Sigmabf q_k} /8 \\
&  \qquad \qquad \qquad -h^5 \ps{\Sigmabf^2 p_k}{q_k}/8 + h^6 \ps{\Sigmabf^2 q_k }{\Sigmabf^2 q_k}/32 \eqsp.
\end{align}
Using that for $k \in \{1,\ldots,T\}$,  $p_k= p_0 -(h/2)\{\nabla U(q_0) + \nabla U(q_k)\} - h \sum_{i=1}^{k-1} \nabla U(q_i)$ and \eqref{eq:proof_pertub_accept_1}, we obtain that for any $k\in \{1,\ldots,T\}$ and $q_0,p_0$, $\norm{q_0} \geq \max(1,R_1)$, $\norm{p_0} \geq \norm[\gamma]{q_0}$,
\begin{align}
  D_k &\leq -h^4 k \BB_1 \norm[2]{q_k}/8 + h^6k^2 \norm{\Sigmabf}^2 C_1 \norm[2]{q_k}  - h^4 \ps{\Sigmabf^2 q_k }{\Sigmabf q_k} /8 \\
      &  \qquad \qquad\qquad \qquad+  h^6 k C_1 \norm{\Sigmabf}^2 \norm[2]{q_k}/8 +  h^6 \norm{\Sigmabf}^4 \norm[2]{q_k}/32     + D_{k,1} \eqsp,
\end{align}
where
\begin{equation}
\label{eq:proof_pertub_accept_lim_D}
    \lim_{\norm{q_0} \to \plusinfty} \sup_{\norm{p_0} \leq \norm{q_0}^\gamma} \abs{D_{k,1}}/\norm[2]{q_0} =0\eqsp.
  \end{equation}
  Define
  \begin{equation}
\label{eq:proof_pertub_accept_def_S_2}
    \bar{S}_2 = \min\defEns{ S \in \ocint{0,\bar{S}_1} \, : \, S^2 ( 2 C_1  \norm{\Sigmabf}^2 + \norm{\Sigmabf}^4)  - \BB_1/8 \geq -\BB_1/16} \eqsp.
  \end{equation}
  Then, if $Th \leq \bar{S_2}$ for any  $q_0,p_0$, $\norm{q_0} \geq \max(1,R_1)$, $\norm{p_0} \geq \norm[\gamma]{q_0}$, we get that
  \begin{equation}
\label{eq:proof_pertub_accept_bound_D}
    D_k \leq -\BB_1h^4 k \norm[2]{q_k}/16 + D_{k,1} \eqsp.
  \end{equation}
  Similarly using that $\Sigmabf$ is definite positive, we obtain that there exist $\BB_2 >0$ and  $\bar{S}_3  \in \ocint{0,\bar{S}_1}$ such that if $hT \leq \bar{S}_3$, for any  $q_0,p_0$, $\norm{q_0} \geq \max(1,R_1)$, $\norm{p_0} \geq \norm[\gamma]{q_0}$, we get that
  \begin{equation}
\label{eq:proof_pertub_accept_bound_D_0}
    D_0 \leq -\BB_2 \norm[2]{q_0} + D_{0,1} \eqsp,
 \quad \text{   where }
    \lim_{\norm{q_0} \to \plusinfty} \sup_{\norm{p_0} \leq \norm{q_0}^\gamma} \abs{D_{0,1}}/\norm[2]{q_0} =0\eqsp.
  \end{equation}
  Combining \eqref{eq:proof_pertub_accept_6}-\eqref{eq:proof_pertub_accept_9}-\eqref{eq:proof_pertub_accept_lim_D}-\eqref{eq:proof_pertub_accept_bound_D} and \eqref{eq:proof_pertub_accept_bound_D_0} in \eqref{eq:proof_pertub_accept_10}, we obtain that \eqref{eq:proof_pertub_accept_3} holds with $\bar{S} = \min(\bar{S}_2,\bar{S}_3)$ since \eqref{eq:proof_pertub_accept_1} implies that  $\norm{q_k} \geq \norm{q_0}/2$.
\end{proof}

%%% Local Variables:
%%% mode: latex
%%% TeX-master: "main"
%%% End:



\appendix
\section{Harris recurrence for mixture of Metropolis-Hastings type Markov kernels}
\label{sec:harr-recurr-metr}
Let $(\Xset,\Xtribu)$ be a measurable space and $\lambda$ be a
$\sigma$-finite measure on $\Xtribu$.  For all $i \in \nset^*$,
let $\alpha_i: \Xset \times \Xset \to \ccint{0,1}$ be a measurable function and   $\qker_i : \Xset
\times \Xset \to \ccint{0,\plusinfty}$ be a Markov transition density \wrt\ $\lambda$. Consider the Markov kernel
$\kernel_i$ on $\Xset \times \Xtribu$ defined by
\begin{equation}
  \label{eq:form_MH_gene}
  \kernel_i(x,\eventA) = \int_{\eventA} \alpha_i(x,y) \qker_i(x,y) \lambda(\rmd y ) + \updelta_x(\eventA) r_i(x) \eqsp, \quad  \text{$x \in \Xset$  and $\eventA \in \Xtribu$,}
\end{equation}
where for all $x
\in \Xset$
\begin{equation}
  \label{eq:def_r_i_harris_tierney}
  r_i(x) = 1 - \int_{\Xset} \alpha_i(x,y) \qker_i(x,y) \lambda(\rmd y ) \eqsp.
\end{equation}
For instance,  $\kernel_i$ may be a Markov kernel associated to the Metropolis-Hastings
algorithm, \ie
\begin{equation}
\label{eq:definition-MH-ratio}
  \alpha_i(x,y) =
  \begin{cases}
\min\parentheseDeux{1, \frac{\pi(y) \qker_i(y,x)}{\pi(x) \qker_i(x,y)}} \eqsp, & \text{ if } \pi(x) \qker_i(x,y) >0 \eqsp,\\
1\eqsp, & \text{otherwise} \eqsp,
  \end{cases}
\end{equation}
for some probability density $\pi: \Xset \to \coint{0,\plusinfty}$
with respect to $\lambda$.
We use the results below in the case
where for any $i \in \nsets$, $\kernel_i$ is a Markov kernel associated to the HMC algorithm.
\cite[Corollary 2]{tierney:1994}
considers Metropolis-Hastings kernels $\kernel_i$ with $\alpha_i$ defined by \eqref{eq:definition-MH-ratio} and shows that that if $\kernel_i$ is irreducible, then $\kernel_i$ is Harris recurrent. We extend this
result to kernels $\kernel_i$ of the form \eqref{eq:form_MH_gene} (but that do not satisfy \eqref{eq:definition-MH-ratio}) and  mixture of Markov kernels $\kernel_{\bfvarpi}$ defined on
$(\Xset, \Xtribu)$ by
\begin{equation}
\label{eq:mixture_kernel_Harris}
  \kernel_{\bfvarpi} = \sum_{i \in \nset^*} \varpi_i \kernel_i
\end{equation}
where $(\varpi_i)_{i\in
  \nset^*}$ is a sequence of non-negative numbers satisfying $\sum_{i
  \in \nset^*} \varpi_i = 1$.

%The proof of this result can be extended  to $\kernel$ of the form \eqref{eq:form_MH_gene}.
\begin{proposition}
  \label{propo:harris_rec}
  Let $\kernel_{\bfvarpi}$ be the Markov kernel given by
  \eqref{eq:mixture_kernel_Harris} and associated with the sequence of
  Markov kernel $(\kernel_i)_{i \in \nset^*}$ given by
  \eqref{eq:form_MH_gene}.  Let $\pi$ be a probability measure on
  $(\Xset,\Xtribu)$. Assume that $\pi$ and $\lambda$ are mutually
  absolutely continuous and for all $i \in \nset^*$, $\pi$ is invariant for $\kernel_i$. If $\kernel_{\bfvarpi}$ is irreducible and there exists $i \in \nset^*$ such that $\varpi_i >0$ and for all $x \in \Xset$ $r_i(x) <1$, with $r_i$ defined by \eqref{eq:def_r_i_harris_tierney},  then $\kernel$ is Harris
  recurrent.
\end{proposition}

\begin{proof}
A bounded measurable function is said to be harmonic if  $\kernel_{\bfvarpi}\harmonic = \harmonic$.
By \cite[Theorem 17.1.4, Theorem
17.1.7]{meyn:tweedie:2009} a Markov kernel $\kernel_{\bfvarpi}$ is Harris recurrent if $\kernel_{\bfvarpi}$ is recurrent and any
bounded harmonic function $\harmonic : \rset^d \to \rset$ is constant.
By \cite[Theorem 10.1.1]{meyn:tweedie:2009}, since $\kernel_{\bfvarpi}$ is irreducible and admits $\pi$ as an invariant probability measure, then $\kernel_{\bfvarpi}$ is recurrent.
On the other hand, any bounded harmonic function $\phi$ is $\pi$-almost surely equal to $\pi(\phi)$ by \cite[Theorem 17.1.1, Lemma 17.1.1]{meyn:tweedie:2009}.
Using that $\pi$ and $\lambda$ are mutually
absolutely continuous, and $\pi$ is an invariant probability measure for $\kernel_i$ for all $i \in \nset^*$,  we get by \eqref{eq:form_MH_gene} that for all $x \in \Xset$
\begin{equation}
   \kernel_{\bfvarpi} \harmonic(x) =  \sum_{i\in \nset^*} \varpi_i \defEns{\pi(\harmonic) (1-r_i(x)) + \harmonic(x)r_i(x)} \eqsp.
\end{equation}
Combining this result with $\kernel_{\bfvarpi} \phi = \phi$, we get for all $x \in \Xset$
\begin{equation}
\{\harmonic(x)-\pi(\harmonic)\} \sum_{i\in \nset^*} \varpi_i \{1-r_i(x)\}= 0\eqsp.
\end{equation}
The condition that there exists $i \in \nset^*$ such that $\varpi_i >0$ and for all $x \in \Xset$ $r_i(x) <1$, implies  that for all $x \in \Xset$, $\harmonic(x) = \pi(\harmonic)$.
\end{proof}
%%% Local Variables:
%%% mode: latex
%%% TeX-master: "main"
%%% End:



\bibliographystyle{plain}
\bibliography{../Bibliography/bibliography}
\end{document}

%%% Local Variables:
%%% mode: latex
%%% TeX-master: t
%%% End:
