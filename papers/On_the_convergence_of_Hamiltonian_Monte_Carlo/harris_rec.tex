\section{Harris recurrence for mixture of Metropolis-Hastings type Markov kernels}
\label{sec:harr-recurr-metr}
Let $(\Xset,\Xtribu)$ be a measurable space and $\lambda$ be a
$\sigma$-finite measure on $\Xtribu$.  For all $i \in \nset^*$,
let $\alpha_i: \Xset \times \Xset \to \ccint{0,1}$ be a measurable function and   $\qker_i : \Xset
\times \Xset \to \ccint{0,\plusinfty}$ be a Markov transition density \wrt\ $\lambda$. Consider the Markov kernel
$\kernel_i$ on $\Xset \times \Xtribu$ defined by
\begin{equation}
  \label{eq:form_MH_gene}
  \kernel_i(x,\eventA) = \int_{\eventA} \alpha_i(x,y) \qker_i(x,y) \lambda(\rmd y ) + \updelta_x(\eventA) r_i(x) \eqsp, \quad  \text{$x \in \Xset$  and $\eventA \in \Xtribu$,}
\end{equation}
where for all $x
\in \Xset$
\begin{equation}
  \label{eq:def_r_i_harris_tierney}
  r_i(x) = 1 - \int_{\Xset} \alpha_i(x,y) \qker_i(x,y) \lambda(\rmd y ) \eqsp.
\end{equation}
For instance,  $\kernel_i$ may be a Markov kernel associated to the Metropolis-Hastings
algorithm, \ie
\begin{equation}
\label{eq:definition-MH-ratio}
  \alpha_i(x,y) =
  \begin{cases}
\min\parentheseDeux{1, \frac{\pi(y) \qker_i(y,x)}{\pi(x) \qker_i(x,y)}} \eqsp, & \text{ if } \pi(x) \qker_i(x,y) >0 \eqsp,\\
1\eqsp, & \text{otherwise} \eqsp,
  \end{cases}
\end{equation}
for some probability density $\pi: \Xset \to \coint{0,\plusinfty}$
with respect to $\lambda$.
We use the results below in the case
where for any $i \in \nsets$, $\kernel_i$ is a Markov kernel associated to the HMC algorithm.
\cite[Corollary 2]{tierney:1994}
considers Metropolis-Hastings kernels $\kernel_i$ with $\alpha_i$ defined by \eqref{eq:definition-MH-ratio} and shows that that if $\kernel_i$ is irreducible, then $\kernel_i$ is Harris recurrent. We extend this
result to kernels $\kernel_i$ of the form \eqref{eq:form_MH_gene} (but that do not satisfy \eqref{eq:definition-MH-ratio}) and  mixture of Markov kernels $\kernel_{\bfvarpi}$ defined on
$(\Xset, \Xtribu)$ by
\begin{equation}
\label{eq:mixture_kernel_Harris}
  \kernel_{\bfvarpi} = \sum_{i \in \nset^*} \varpi_i \kernel_i
\end{equation}
where $(\varpi_i)_{i\in
  \nset^*}$ is a sequence of non-negative numbers satisfying $\sum_{i
  \in \nset^*} \varpi_i = 1$.

%The proof of this result can be extended  to $\kernel$ of the form \eqref{eq:form_MH_gene}.
\begin{proposition}
  \label{propo:harris_rec}
  Let $\kernel_{\bfvarpi}$ be the Markov kernel given by
  \eqref{eq:mixture_kernel_Harris} and associated with the sequence of
  Markov kernel $(\kernel_i)_{i \in \nset^*}$ given by
  \eqref{eq:form_MH_gene}.  Let $\pi$ be a probability measure on
  $(\Xset,\Xtribu)$. Assume that $\pi$ and $\lambda$ are mutually
  absolutely continuous and for all $i \in \nset^*$, $\pi$ is invariant for $\kernel_i$. If $\kernel_{\bfvarpi}$ is irreducible and there exists $i \in \nset^*$ such that $\varpi_i >0$ and for all $x \in \Xset$ $r_i(x) <1$, with $r_i$ defined by \eqref{eq:def_r_i_harris_tierney},  then $\kernel$ is Harris
  recurrent.
\end{proposition}

\begin{proof}
A bounded measurable function is said to be harmonic if  $\kernel_{\bfvarpi}\harmonic = \harmonic$.
By \cite[Theorem 17.1.4, Theorem
17.1.7]{meyn:tweedie:2009} a Markov kernel $\kernel_{\bfvarpi}$ is Harris recurrent if $\kernel_{\bfvarpi}$ is recurrent and any
bounded harmonic function $\harmonic : \rset^d \to \rset$ is constant.
By \cite[Theorem 10.1.1]{meyn:tweedie:2009}, since $\kernel_{\bfvarpi}$ is irreducible and admits $\pi$ as an invariant probability measure, then $\kernel_{\bfvarpi}$ is recurrent.
On the other hand, any bounded harmonic function $\phi$ is $\pi$-almost surely equal to $\pi(\phi)$ by \cite[Theorem 17.1.1, Lemma 17.1.1]{meyn:tweedie:2009}.
Using that $\pi$ and $\lambda$ are mutually
absolutely continuous, and $\pi$ is an invariant probability measure for $\kernel_i$ for all $i \in \nset^*$,  we get by \eqref{eq:form_MH_gene} that for all $x \in \Xset$
\begin{equation}
   \kernel_{\bfvarpi} \harmonic(x) =  \sum_{i\in \nset^*} \varpi_i \defEns{\pi(\harmonic) (1-r_i(x)) + \harmonic(x)r_i(x)} \eqsp.
\end{equation}
Combining this result with $\kernel_{\bfvarpi} \phi = \phi$, we get for all $x \in \Xset$
\begin{equation}
\{\harmonic(x)-\pi(\harmonic)\} \sum_{i\in \nset^*} \varpi_i \{1-r_i(x)\}= 0\eqsp.
\end{equation}
The condition that there exists $i \in \nset^*$ such that $\varpi_i >0$ and for all $x \in \Xset$ $r_i(x) <1$, implies  that for all $x \in \Xset$, $\harmonic(x) = \pi(\harmonic)$.
\end{proof}
%%% Local Variables:
%%% mode: latex
%%% TeX-master: "main"
%%% End:
