


In this section, we give conditions on the potential $\F$ which imply that
the HMC kernel \eqref{eq:def_Phiverletq} converges geometrically fast to its invariant distribution.
Let $V: \rset^d \to \coint{1,\plusinfty}$ be a measurable function and $\Pker$
be a Markov kernel on $(\rset^d,\borelSet(\rset^d))$. The Markov kernel $\Pker$ is said to
be $V$-uniformly geometrically ergodic if $\Pker$ admits an invariant probability $\pi$
and there exists $\rho \in \coint{0,1}$
and $\varsigma \geq 0$ such that for all $\q \in \rset^d$ and $k \in \nset^*$,
\begin{equation}
  \Vnorm[V]{\Pker^k(\q,\cdot)-\pi} \leq \varsigma \rho^{k} V(\q) \eqsp.
\end{equation}
By \cite[Theorem 16.0.1]{meyn:tweedie:2009}, if $\Pker$ is aperiodic, irreducible and satisfies a Foster-Lyapunov drift condition, \ie~there exists a small set $\Csf$ for $\Pker$, $\lambda \in \coint{0,1}$ and $b < \plusinfty$ such that for all $\q \in \rset^d$,
\begin{equation}
\label{eq:foster-lyapunov}
\Pker V  \leq \lambda V + b \1_{\Csf} \eqsp,
\end{equation}
then $\Pker$ is $V$-uniformly geometrically ergodic. If a function $V : \rset^d \to \coint{1,\infty}$
satisfies \eqref{eq:foster-lyapunov}, then $V$ is said to be a Foster-Lyapunov function for $\Pker$.
We first give an elementary condition to establish the $V$-uniform geometric
ergodicity for a class of generalized Metropolis-Hastings  kernels which includes HMC kernels as a particular example.


Let $\Kker$ be a proposal kernel on $(\rset^d, \borelSet(\rset^{2d}))$ and $\alphagen : \rset^{3 d } \to
\ccint{0,1}$ be an acceptance probability, assumed to be Borel measurable. Consider the Markov kernel $\Pker$ on $(\rset^d,\borelSet(\rset^d))$ defined for all $\q \in \rset^d$ and
$\eventA \in \borelSet(\rset^d)$ by
\begin{equation}
\label{eq:def_kenel_MH}
  \Pker(\q,\eventA) = \int_{\rset^{2d}} \1_{\eventA}(\projq(z)) \alphagen(\q,z) \Kker(\q, \rmd z )
+ \updelta_{\q}(\eventA) \int_{\rset^{2d}} \defEns{1- \alphagen(\q,z) }\Kker(\q, \rmd z)  \eqsp,
\end{equation}
where $\projq : \rset^{d} \times \rset^d \to \rset^d$ is the canonical projection onto the first
$d$ components.
For $h \in \rset^*_+$ and $T \in \nset^*$, $\Pkerhmc[h][T]$ corresponds to $\Pker$ with
$\Kker$ and $\alphagen$  given for all $\q,\p,\x \in \rset^d$ and $\Bsf \in
\borelSet(\rset^{2d})$ respectively by
\begin{align}
\label{eq:def_Pker_proposition_double}
  \PkerhmcD[h][T](\q,\Bsf) &= (2\uppi)^{-d/2}\int_{\rset^{d}} \1_{\Bsf}\parenthese{\Phiverletq[h][T](\q,\tilde{\p}),\tilde{\p}} \rme^{-\norm{\tilde{\p}}^2/2} \rmd \tilde{\p} \eqsp, \\
\label{eq:def_alpha_acc_tilde_hmc}
\tildeAlphaacc(\q,(\tilde{\q},\tilde{\p})) & =
\begin{cases}
\alphaacc\defEns{(\q,\tilde{p}),\Phiverlet[h][T](\q,\tilde{p})} \eqsp, & \text{if}\, \tilde{q}= \Phiverletq[h][T](\q,\tilde{\p}) \eqsp, \\
0 & \text{otherwise} \eqsp,
\end{cases}
\end{align}
where  $\Phiverlet[h][T]$, $\Phiverletq[h][T]$ and $\alphaacc$  are  defined in   \eqref{eq:def_Phiverlet}, \eqref{eq:def_Phiverletq} and \eqref{eq:def_acc_ratio}, respectively. Let $\Vgeo : \rset^d \to \coint{1,\plusinfty}$ be a
\emph{norm-like}  function,  \ie\ a measurable function such that for all $M \in \rset_+$, the level sets $\set{\q \in \rset^d}{\Vgeo(\q) \leq M}$ are compact. Note that if $\Vgeo$ is norm-like, for any $M \in \rset_+$, $\set{\q \in \rset^d}{\Vgeo(\q) \leq M}^{\complementary}$ is non-empty.   The function $\Vgeo$ naturally extends on $\rset^{2d}$ by
setting for all $(\q,\p) \in \rset^{2d}$, $\Vgeo(\q,\p) = \Vgeo(\q)$.
For all $\q \in \rset^d$, define:
\begin{equation}
\label{eq:def_rej_ballV}
%\begin{aligned}
  \rejectregion(\q) = \defEns{z \in \rset^{2d} \, , \, \alphagen(\q,z) < 1  } \eqsp, \,
    \ballV(\q) = \defEns{z \in \rset^{2d} \, , \, \Vgeo(\projq(z)) \leq \Vgeo(\q) } \eqsp.
%\end{aligned}
\end{equation}
The set $\rejectregion(\q)$ is the potential rejection region.
Our next result gives a condition on $\Kker$ and $\alphagen$ which
implies that if $V$ is a Foster-Lyapunov function for $\Kker$ then
$\Pker$ satisfies a Foster-Lyapunov drift condition as well. This
result is inspired by \cite[Theorem~4.1]{roberts:tweedie:1996}, which is used to show the $V$-uniform geometric ergodicity of the MALA algorithm.
\begin{proposition}
\label{propo:geo_drift_MH}
  Let $\Vgeo : \rset^d \to \coint{1,\plusinfty}$ be a norm-like  function.
  Assume moreover that there exist  $\lambdageo \in \coint{0,1}$ and $\bgeo \in \rset_+$ such that
  \begin{equation}
  \label{eq:assum:geo_ergo_1}
  \Kker \Vgeo \leq  \lambdageo \Vgeo + \bgeo \eqsp.
  \end{equation}
and
\begin{equation}
\label{eq:assum:geo_ergo_2}
  \lim_{M \to \plusinfty} \sup_{\set{\q \in \rset^d}{\Vgeo(\q) \geq M}} \Kker(\q,\rejectregion(\q) \cap \ballV(\q)) = 0  \eqsp.
\end{equation}
 Then there exist  $\lambdageotilde \in \coint{0,1}$ and $\bgeotilde \in \rset_+$ such that
 $\Pker \Vgeo \leq  \lambdageotilde \Vgeo + \bgeotilde$ where $\Pker$ is given by \eqref{eq:def_kenel_MH}.
\end{proposition}
\begin{proof}
The proof is postponed to \Cref{sec:proof-crefpr}.
\end{proof}
We show below that under appropriate conditions, the proposal kernel $\PkerhmcD[h][T]$ and
the acceptance probability $\tildeAlphaacc$ given by \eqref{eq:def_Pker_proposition_double} and
\eqref{eq:def_alpha_acc_tilde_hmc} satisfy the conditions of
\Cref{propo:geo_drift_MH} which imply that the HMC kernel
$\Pkerhmc[h][T]$ is $V$-uniformly geometrically ergodic. %We assume in the following conditions.
For $\m \in \ocint{1,2}$, consider the following assumption:



\begin{assumption}[$m$]
  \label{assum:potential:c}
There exist $\constthree \in \rset^*_+$ and $\constfour \in \rset$ such that for all $\q \in \rset^d$,
  \begin{equation}
    \ps{\nabla \F(\q)}{\q} \geq \constthree \norm{\q}^{m} -\constfour \eqsp.
  \end{equation}
\end{assumption}
For all $\a \in \rset_+^*$ and $\q \in \rset^d$, define
\begin{equation}
\label{eq:def_Va}
\Vdrifta[a] (\q) = \exp(\a \norm{\q}) \eqsp.
\end{equation}
\begin{proposition}
\label{lem:drift_uhmc}
%   \begin{equation}
% \label{eq:hyp:drift_uhmc}
%     \liminf_{\q \to \plusinfty} \ps{\nabla \F(\q)}{\q}/ \norm{\q}^{\expozero+1} >0 \eqsp.
%   \end{equation}
 % Let $T \in \nset^*$. %Then the following holds
\begin{enumerate}[label=(\alph*)]
\item   \label{lem:drift_uhmc_1}
 Assume   \Cref{assum:regOne}$(m-1)$ and  \Cref{assum:potential:c}$(\m)$ for some $\m \in \ooint{1,2}$. Then, for all $T \in \nsets$,  $h \in \rset^*_+$, and $\a \in \rsetep$, there exist $\lambda \in \coint{0,1}$ and $\b \in \rsetp$ such that
  \begin{equation}
    \label{eq:drift_lem}
      \PkerhmcD[h][T] \Vdrifta[\a] \leq \lambda  \Vdrifta[\a] + \b \eqsp.
  \end{equation}
\item
\label{lem:drift_uhmc_2}
 Assume   \Cref{assum:regOne}$ (1)$ and  \Cref{assum:potential:c}$ (2)$.  Let $\bar{S} > 0$ be such that $\Theta(S) < \constthree$ for any $S \in \ocint{0,\bar{S}}$, where
\begin{align}
\label{eq:definition-function-C}
  \Theta(s)&= 2 \constzero^{1/2} \vartheta_2(s) \{ \rme^{\constzero^{1/2} s \vartheta_1(\constzero^{1/2} s)} - 1\} \\
  & \qquad  \qquad + 6 s^2 \left(   \constzeroT ^2  +  \constzero \vartheta_2^2(s) \{ \rme^{\constzero^{1/2} s \vartheta_1(\constzero^{1/2} s)} - 1\}^2\right) \eqsp.
\end{align}
Then, for all $a \in \rsetep$,  $T \in \nsets$ and  $h \in \ocint{0,\bar{S}/T}$,  there exist  $\lambda \in \coint{0,1}$ and $\b \in \rsetp$ which satisfy \eqref{eq:drift_lem}.
\end{enumerate}
\end{proposition}
\begin{proof}
  The proof is postponed to \Cref{sec:proof-crefl-2}
\end{proof}
We now derive sufficient conditions under which the condition \eqref{eq:assum:geo_ergo_2} of
\Cref{propo:geo_drift_MH} is satisfied.

\begin{assumption}[$\m$]
\label{assum:potential}
\begin{enumerate}[label = (\roman*)]
\item \label{assum:potential:a}
$\F \in C^3(\rset^d)$  and there exists $\constone \in \rset_+^*$ such that for all $\q \in \rset^d$ and $k=2,3$:
\begin{equation}
%\label{eq:10}
\norm{D^k \F(\q)}\leq \constone \defEns{1+\norm{\q}}^{\m-k} \eqsp.
\end{equation}
\item \label{assum:potential:b}
There exist $\consttwo \in \rset_+^* $ and $\rhtwo \in \rset^+$ such that for all $\q \in \rset^d$, $\norm{q}\geq \rhtwo$,
\begin{equation}
%\label{eq:11}
D^2\F(\q)\defEns{ \nabla \F(\q)\otimes  \nabla \F(\q)}  \geq \consttwo \norm{\q}^{3\m-4} \eqsp.
\end{equation}
  \end{enumerate}
\end{assumption}

It is easily checked that under \Cref{assum:potential}, the results of \Cref{sec:ergodicity-hmc} can be applied, \ie~$\nabla \F$ satisfies \Cref{assum:regOne}($\m-1$); see \Cref{lem:grad_Lip_F}.

Condition \Cref{assum:potential:c}$(m)$ and \Cref{assum:potential}$(m)$ are satisfied by power functions $\q \mapsto c\norm{\q}^\m$. More generally, they are satisfied by $\m$-homogeneously quasiconvex functions with convex level sets  outside a ball and by  perturbations of such functions.

We say that a function $\F_0$ is $m$-homogeneous quasi-convex
outside a ball of radius $\Rexp$ if the following conditions are satisfied:
\begin{enumerate}[(QC-1)]
\item for all $t \geq 1$ and $q \in \rset^d$, $\norm{\q}\geq \Rexp$, $\F_0(t \q)= t^\m \F_0(\q)$.
% $$
% \F_0(t \q)= t^\m \F_0(\q) \eqsp.
% $$
\item for all $\q \in \rset^d$, $\norm{\q} \geq \Rexp$, the level sets $\{ \x\, :\, \F_0(\x) \leq \F_0(\q)\}$ are convex.
\end{enumerate}
\begin{proposition}
\label{le:convex}
Let $m \in \ccint{1,2}$  and $\Rexp \in \rset_+$.  Assume that the potential $\F$ may be decomposed as
$$
\F(\q)=\F_0(\q)+G(\q) \eqsp, \quad \text{$\q\in \rset^d$, $\norm{\q} \geq \Rexp$} \eqsp,
$$
where the functions $\F_0,G \in C^3(\rset^d)$ satisfy the following two conditions:
  \begin{enumerate}[(A)]
  \item $\F_0$ is $\m$-homogeneously quasiconvex outside a ball of radius $\Rexp$ and $\lim_{\norm{\q} \to \plusinfty} \F_0(\q)=\infty$.
\label{le:convex:a}
\item
\label{le:convex:b}
For $k=2,3$, $\lim_{\norm{\q} \to \plusinfty}\normop{D^k G(\q)}/  \norm{\q}^{\m-k}= 0$.
% \begin{equation}%\label{eq:lower}
% \lim_{\norm{\q} \to \plusinfty}\normop{D^k G(\q)}/  \norm{\q}^{\m-k}= 0 \eqsp.
% \end{equation}
  \end{enumerate}
Then $\F$ satisfies  \Cref{assum:potential:c}$(m)$ and  \Cref{assum:potential}$(\m)$.
\end{proposition}
\begin{proof}
The proof is postponed to \Cref{sec:proof-crefle:convex}.
\end{proof}

To show that the condition \eqref{eq:assum:geo_ergo_2} of
\Cref{propo:geo_drift_MH} is satisfied under
\Cref{assum:potential}$(m)$, we rely on the following important result which implies that the probability of accepting a move goes to 1 as $\norm{q} \to \infty$.
\begin{proposition}
  \label{propo:accept} Assume
  \Cref{assum:potential}$(m)$ for some $\m \in \ocint{1,2}$. Let $\gamma \in \ooint{0,\m-1}$.
  \begin{enumerate}[label=(\alph*)]
  \item
  \label{propo:accept_1}
  If $\m\in (1,2)$, for all $T \in \nsets$, $h \in \rset_+^*$, there exists $R_{\Ham} \in \rset_+$ such that for
  all $\q_0,\p_0 \in \rset^d$, $\norm{q_0} \geq R_{\Ham}$ and
  $\norm{p_0} \leq \norm{\q_0}^{\gamma}$, $ \Ham(\Phiverlet[h][T](q_0,p_0)) -
  \Ham(q_0,p_0) \leq 0$.
\item
  \label{propo:accept_2}
  If $\m=2$,   there exists $\bar{S} >0$ such that for any $T \in \nsets$ and $h \in \ocint{0, \bar{S}/T^{3/2}}$,  there exists $R_{\Ham} \in \rset_+$ satisfying for all $\q_0,\p_0 \in \rset^d$, $\norm{q_0} \geq R_{\Ham}$ and
  $\norm{p_0} \leq \norm{\q_0}^{\gamma}$, $ \Ham(\Phiverlet[h][T](q_0,p_0)) -
  \Ham(q_0,p_0) \leq 0$.
  \end{enumerate}
\end{proposition}


\begin{proof}
  The proof is postponed to  \Cref{sec:proof-crefth}.
\end{proof}
%This behavior comes a bit as a surprise. It


This result  means that far in the tail the HMC proposal are "inward".
We illustrate the result of \Cref{propo:accept}-\ref{propo:accept_1}
in \Cref{fig:H_behaviour} for $U$ given by $\q \mapsto
(\norm[2]{\q}+\delta)^{\kappa}$ for $\kappa=3/4$, $h = 0.9$ and
$p_0 \in \rset^d$, $\norm{p_0}=1$. Note that this potential satisfies
the condition of the proposition. We can observe that choosing the different
initial conditions $q_0$ with increasing norm imply that $\tilde{T} =
\max\{k \in \nset ; \Ham(\Phiverlet[h][k](q_0,p_0)) - \Ham(q_0,p_0)
<0\}$ increases as well.

 \begin{figure}[h]
   \centering
   \includegraphics[scale=0.3]{hamiltonian_behaviour}
   \caption{Behaviour of $(\Ham(\Phiverlet[h][k](q_0,p_0)))_{k \in \{0,\ldots,T\}}$ for different initial conditions $q_0$.}
   \label{fig:H_behaviour}
 \end{figure}


 However, in the case $m=2$, \Cref{propo:accept}-\ref{propo:accept_2} only implies that the HMC proposal is inward only if the step size $h$ is sufficiently small with respect to the number of leapfrog step $T$, \ie~is of order $\bigO(T^{-3/2})$. To relax this condition, we strengthen \Cref{assum:potential}($2$) by assuming that $U$ is a smooth perturbation of a quadratic function.
 \begin{assumption}
   \label{ass:pertub}
   There exist $\tilde{U} : \rset^d \to \rset$, continuously differentiable, and  a positive definite matrix $\Sigmabf$ such that
   $U(q) = \ps{\Sigmabf q}{q}/2 + \tilde{U}(q)$ and there exist $\constfive \geq 0$ and  $\varrho \in \coint{1,2}$  such that for any $q,x \in \rset^d$,
   \begin{align}
     \absLigne{\tilde{U}(q)} &\leq \constfive(1+\norm[\varrho]{q}) \eqsp, \quad  \normLigne{\nabla \tilde{U}(q)} \leq \constfive(1+\norm[\varrho-1]{q}) \eqsp,\\  \qquad \qquad \qquad & \normLigne{\nabla \tilde{U}(q) - \nabla \tilde{U}(x)} \leq \constfive \norm{q-x} \eqsp.
   \end{align}
 \end{assumption}
Note that it is straightforward to check that under \Cref{ass:pertub}, the conditions \Cref{assum:regOne}$(1)$ and  \Cref{assum:potential:c}$(2)$ hold.

 \begin{proposition}
  \label{propo:accept_pertub} Assume  \Cref{ass:pertub}  and let $\gamma \in \ooint{0,1}$.
There exists a  constant $\bar{S} >0$ such that for all $T \in \nsets$, $h \in \ocint{0,\bar{S}/T}$, there exists $R_{\Ham} \in \rset_+$ such that for
  all $\q_0,\p_0 \in \rset^d$, $\norm{q_0} \geq R_{\Ham}$ and
  $\norm{p_0} \leq \norm{\q_0}^{\gamma}$, $ \Ham(\Phiverlet[h][T](q_0,p_0)) -
  \Ham(q_0,p_0) \leq 0$.
\end{proposition}
\begin{proof}
  The proof is postponed to  \Cref{sec:proof-crefth_accept_2}.
\end{proof}
We now can establish the geometric ergodicity of the HMC sampler.
\begin{theorem}
  \label{theo:geoErg}
  \begin{enumerate}[label=(\alph*)]
  \item
  \label{item:theo_1}
 If   \Cref{assum:potential:c}$(\m)$ and  \Cref{assum:potential}$(m)$ hold for some $\m\in (1,2)$, then for all $a \in \rset_+^*$,  $T \in \nset^*$ and $h > 0$, the HMC kernel $\Pkerhmc[h][T]$ is $\Vdrifta[\a]$-uniformly geometrically ergodic, where $\Vdrifta[a]$ is defined by \eqref{eq:def_Va}.
\item
  \label{item:theo_2}
  If  \Cref{assum:potential:c}$(2)$ and \Cref{assum:potential}$(2)$ hold, then there exists $\bar{S}>0$ such that for all  $a \in \rset^*_+$, $T \in \nset^*$ and $h \in \ooint{0,\bar{S}/T^{3/2}}$, $\Pkerhmc[h][T]$ is $\Vdrifta[\a]$-uniformly geometrically ergodic.
\item \label{item:theo_3}
  If \Cref{ass:pertub} holds, then there exists $\bar{S}>0$ (depending only on $\Sigmabf$ and $\constfive$) such that for all $a \in \rset^*_+$, $T \in \nset^*$ and $h \in \ooint{0,\bar{S}/T}$, $\Pkerhmc[h][T]$ is $\Vdrifta[\a]$-uniformly geometrically ergodic.
  \end{enumerate}
\end{theorem}
\begin{proof}[Proof of \Cref{theo:geoErg}]
  It is enough to consider \ref{item:theo_1} as the proof
  of  \ref{item:theo_2} and \ref{item:theo_3} follows exactly the same lines
  taking $\bar{S}$ small enough.  \Cref{lem:drift_uhmc} shows that for all $T \in \nsets$,  $h \in \rset^*_+$, and $\a \in \rsetep$, there exist $\lambda \in \coint{0,1}$ and $\b \in \rsetp$ such that the Foster-Lyapunov drift condition
  $\PkerhmcD[h][T] \Vdrifta[\a] \leq \lambda  \Vdrifta[\a] + \b$ is satisfied.
  By \Cref{propo:accept}, there exists $R_{\Ham} \geq 0$ such that for all $\q \in \rset^d$, $\norm{\q} \geq R_{\Ham}$,
\begin{equation}
\label{eq:proofpgeo_erg_0}
  \int_{\rejectregion(q)}   \PkerhmcD[h][T](\q, \rmd z ) \leq (2\uppi)^{-d/2} \int_{\{ \norm{\p} \geq \norm{q}^{\gamma}\} } \rme^{-\norm{p}^2/2} \rmd p \eqsp,
\end{equation}
for $\gamma \in \ooint{0,m-1}$ where $\rejectregion(\q) = \set{z \in \rset^{2d}}{\tildeAlphaacc(\q,z) < 1 }$ (see~\eqref{eq:def_alpha_acc_tilde_hmc}), which implies that
\begin{equation}
\label{eq:proof:geo_erg:1}
  \lim_{M \to \plusinfty} \sup_{\norm{\q} \geq M} \int_{\rejectregion(q)} \PkerhmcD[h][T](\q, \rmd z ) = 0 \eqsp,
\end{equation}


Since $\Vdrifta[a]$ is norm-like,  \Cref{propo:geo_drift_MH} implies that  for all $T > 0$ and $h > 0$, there exists $\lambdageotilde$ and $\bgeotilde$ (depending upon $a$, $h$ and $T$) such that
$\Pkerhmc[h][T] \Vdrifta[a] \leq \lambdageotilde \Vdrifta[a] + \bgeotilde$.
For all $M \geq 0$ the level sets $\{ \Vdrifta[a] \leq M\}$ are compact and hence small by \Cref{theo:irred_D}.
\cite[Corollary~14.1.6]{douc:moulines:priouret:2018} then shows that there exists a small set $\msc$, $\check{\lambda} \in \coint{0,1}$ and $\check{b} \in \coint{0,1}$ such that $\Pkerhmc[h][T] \Vdrifta[a] \leq \check{\lambda} \Vdrifta[a] + \check{b} \1_{\msc}$.  Since $\Pkerhmc[h][T]$ is aperiodic, the result follows from \cite[Theorem~15.2.4]{douc:moulines:priouret:2018}.
\end{proof}

We  finally consider the case where the number of leapfrog steps is a random variable
independent of the current state.
\begin{theorem}
\label{coro:geo_ergod-hmc-algor}
\begin{enumerate}[label=(\alph*)]
  \item
  \label{item:geo_ergod-hmc-algor:theo_1}
 If   \Cref{assum:potential:c}$(\m)$ and  \Cref{assum:potential}$(m)$ hold for  $\m\in (1,2)$,  then for all probability distributions $\bfvarpi=(\omega_i)_{i \in \nset^*}$ on $\nset^*$, all sequences $\mathbf{h}= (h_i)_{i \in \nset*}$ of positive numbers,  and $a \in \rset^*_+$, the randomized kernel  $\randomkerhmc_{\mathbf{h},\bfvarpi}$ \eqref{eq:randomhmc} is $\Vdrifta[\a]$-uniformly geometrically ergodic, where $\Vdrifta[a]$ is defined by \eqref{eq:def_Va}.
\item
  \label{item:geo_ergod-hmc-algor:theo_2}
  If  \Cref{assum:potential:c}$(2)$ and \Cref{assum:potential}($2$) hold, then there exists $\bar{S}>0$ such that for all probability distributions $\bfvarpi=(\omega_i)_{i \in \nset^*}$ on $\nset^*$, all sequences $\mathbf{h}= (h_i)_{i \in \nset^*}$ satisfying $\max_{i \in \supp(\bfvarpi)} i^{3/2} h_i \leq \bar{S}$,  and $a \in \rset^*_+$, $\randomkerhmc_{\mathbf{h},\bfvarpi}$ is $\Vdrifta[\a]$-uniformly geometrically ergodic.
\item   \label{item:geo_ergod-hmc-algor:theo_3}
  If \Cref{ass:pertub} holds, then there exists $\bar{S}>0$ (depending only on $\Sigmabf$ and $\constfive$) such that for all  probability distributions $\bfvarpi=(\omega_i)_{i \in \nset^*}$ on $\nset^*$, all sequences $\mathbf{h}= (h_i)_{i \in \nset^*}$ satisfying $\max_{i\in \supp(\bfvarpi)} i h_i \leq \bar{S}$,  and $a \in \rset^*_+$, $\randomkerhmc_{\mathbf{h},\bfvarpi}$ is $\Vdrifta[\a]$-uniformly geometrically ergodic.
\end{enumerate}
\end{theorem}
\begin{proof}
It is enough to consider \ref{item:geo_ergod-hmc-algor:theo_1} as the proofs
of \ref{item:geo_ergod-hmc-algor:theo_2} and \ref{item:geo_ergod-hmc-algor:theo_3} are along the same lines.
Set $a \in \rset_+^*$. It is established in the proof of \Cref{theo:geoErg}  that  for all $i \in \nset^*$
$\Pkerhmc[i][h_i]$ satisfies a Foster-Lyapunov drift condition:
there exists $\check{\lambda}_i \in \coint{0,1}$ and $\check{b}_i < \infty$ such that
$\Pkerhmc[i][h_i] \Vdrifta[a] \leq \lambda_i \Vdrifta[a] + b_i$,
By \Cref{coro:ergod-hmc-algor},   $\randomkerhmc_{\mathbf{h},\bfvarpi}$ is irreducible and aperiodic and all the compact sets are small. We conclude by applying \cite[Theorem~15.2.4]{douc:moulines:priouret:2018}.
\end{proof}

Compared to \cite{livingstone:betancourt:byrne:girolami:2016}, which
establishes geometric ergodicity of the HMC kernel under an implicit
assumption on the behaviour of the acceptance rate, our conditions are
directly verifiable on the potential $U$.

On the other hand, our conditions are different than the one given by
\cite{bou:sanz:2017} to establish the geometric ergodicity of the
idealized randomized HMC, which assumed to exactly solve the Hamiltonian
ODE \eqref{eq:hamil_ode}. These conditions are the following
1)$\int_{\rset^d} \norm[2]{q} \rmd \pi(q) < \plusinfty$, 2) there
exist $C_1 \in \ooint{0,1}$ and $C_2 >0$ such that for all $q \in \rset^d$
\begin{equation}
\label{eq:hyp_boo_rabee_sanz_serna}
  (1/2) \ps{\nabla U(q)}{q} \geq C_1 U(q) + \frac{(\tau^{-1}C_1/4)^2+\tau^{-2}C_1(1-C_1)/4}{2(1-C_1)}\norm[2]{q} -C_2 \eqsp,
\end{equation}
where $\tau>0$ is the duration parameter of the RHMC algorithm.  Note
that these conditions assumed that the target density is lighter than
Gaussian. In comparison, our results can be applied to sub-quadratic
potentials. In addition, it can be shown that HMC is not geometrically
ergodic under \eqref{eq:hyp_boo_rabee_sanz_serna} on the following example associated with the potential defined by  \eqref{eq:def_U_mixture_gaussian} below.

% The condition \eqref{eq:hyp_boo_rabee_sanz_serna} is satisfied if $\pi$ is a
% mixture of $d$-dimensional Gaussian distributions.  We show
% numerically that there is strong evidences which imply that HMC is not
% geometrically ergodic for such examples.
The main difference with the
setting of \cite{bou:sanz:2017} is that HMC has a acceptance/rejection
step and the integrated acceptance ratio
\[ q \mapsto \int_{\rset^d} \alphaacc\defEnsLigne{(\q,\p),\Phiverlet[h][T](\q,\p)}
\rme^{-\norm[2]{p}/2} (2 \uppi)^{-d/2} \rmd p
\]
must not go to $0$ as
$\norm{q}$ goes to $\plusinfty$. This is essentially the reason why
\Cref{assum:potential} differs from \eqref{eq:hyp_boo_rabee_sanz_serna}. Indeed, to show that an
irreducible Markov kernel $\mathrm{P}$ on $(\rset^d, \mcb(\rset^d))$ is not geometrically
ergodic with respect to an invariant measure $\mu$, \cite[Theorem
5.1]{roberts:tweedie:1996:biometrika} states the following sufficient condition
\begin{equation}
  \label{eq:condition_non_geo_ergodicity}
 \mathrm{ess \, sup}_{q \in \rset^d} \mathrm{P}(q, \{q\}) = 1 \eqsp,
\end{equation}
 where
$\mathrm{ess\, sup}$ is taken with respect to $\mu$. Consider then
the target density $\pi$ with potential $U$ given for all $q =(q_1,q_2) \in \rset^2$ by
\begin{equation}
\label{eq:def_U_mixture_gaussian}
  U(q) = -\log(\rme^{-q_1^2 - 5 q_2^2} + \rme^{-5q_1^2 - q_2^2}) \eqsp.
\end{equation}
Note that $U$ satisfies the condition
\eqref{eq:hyp_boo_rabee_sanz_serna}. On the contrary, we may show that
\eqref{eq:condition_non_geo_ergodicity} holds, and therefore HMC is
not geometrically ergodic for such a potential $U$.  However, the
detailed calculations are very technical and not particularly
informative and we prefer to present a numerical evidence that
\eqref{eq:condition_non_geo_ergodicity} holds. Indeed,
\Cref{fig:accept_mix_gaussian} displays numerical computations of the mean acceptance ratio,
$\int_{\rset^2} \alphaacc\defEnsLigne{(\q,\p),\Phiverlet[h][T](\q,\p)}
\rme^{-\norm[2]{p}/2} (2 \uppi)^{-1} \rmd p= 1 -\Pkerhmc[h][T](q,\{q\})$ for $q_1 \in \{200,250,$
$300,350,400,450,500\}$,
$q_2 \in \ccint{q_1+10^{-4},q_1+2\cdot 10^{-4}}$ and $T=1$  which
corresponds to MALA. We can observe that the larger $q_1$, the smaller $1-\Pkerhmc[h][T](q,\{q\})$, which illustrates that  \eqref{eq:condition_non_geo_ergodicity}  holds for the HMC
kernel.

 \begin{figure}[h]
   \centering
   \includegraphics[scale=0.4]{mix_gaussian_1}
   \caption{}
   \label{fig:accept_mix_gaussian}
 \end{figure}


%For  $T \in \nset^*$, $\alpha \geq 0$ and $h_{\Ham} >0$, consider the following assumption.
% \begin{assumption}[$T,\alpha,h_{\Ham}$]
%   \label{assum:diff_ham}
%  There exists  $R_{\Ham}
%   \geq 0$ and such that for all $h \in
%   \ocint{0,h_{\Ham}}$, if
% \begin{equation}
% \label{eq:100}
% |q_0|\geq R_{\Ham} \quad\textrm{and}\quad |p_0|\leq |q_0|^\alpha,
% \end{equation}
% then
% \begin{equation}
% \label{eq:diff_ham_postitive}
%   \Ham(q_T,p_T) - \Ham(q_0,p_0) \geq 0 \eqsp.
% \end{equation}
% \end{assumption}



%%% Local Variables:
%%% mode: latex
%%% TeX-master: "main"
%%% End:
