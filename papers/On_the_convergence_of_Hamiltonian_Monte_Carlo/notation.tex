%\section{Notations}
Denote by $\rset_+$ and $\rset_+^*$, the set of non-negative and
  positive real numbers respectively. Denote by $\operatorname{I}_n$ the identity matrix. Denote by $\norm{\cdot}$ the
Euclidean norm on $\rset^d$.  Denote by $\borelSet(\rset^d)$ the
Borel $\sigma$-field of $\rset^d$, $\functionspace[]{\rset^d}$ the set
of all Borel measurable functions on $\rset^d$ and for $f \in
\functionspace[]{\rset^d}$, $\Vnorm[\infty]{f}= \sup_{x \in \rset^d}
\abs{f(x)}$. Denote by $\Leb$ the Lebesgue-measure on $\rset^d$. For
$\mu$ a probability measure on $(\rset^d, \borelSet(\rset^d))$ and
$f \in \functionspace[]{\rset^d}$ a $\mu$-integrable function, denote
by $\mu(f)$ the integral of $f$ \wrt~$\mu$.  For $f \in
\functionspace[]{\rset^d}$, set $\Vnorm[\infty]{f}= \sup_{x \in
  \rset^d} \abs{f(x)}$.  Let $V: \rset^d \to \coint{1,\infty}$ be a
measurable function. For $f \in \functionspace[]{\rset^d}$, the
$V$-norm of $f$ is given by $\Vnorm[V]{f}= \Vnorm[\infty]{f/V}$. For
two probability measures $\mu$ and $\nu$ on $(\rset^d,
\borelSet(\rset^d))$, the $V$-total variation distance of $\mu$ and
$\nu$ is defined as
\begin{equation}
%\label{eq:definition_TV}
\Vnorm[V]{\mu-\nu} = \sup_{f \in \functionspace[]{\rset^d}, \Vnorm[V]{f} \leq 1}  \abs{\int_{\rset^d } f(x) \rmd \mu (x) - \int_{\rset^d}  f(x) \rmd \nu (x)} \eqsp
\end{equation}
If $V \equiv 1$, then $\Vnorm[V]{\cdot}$ is the total variation  denoted by $\tvnorm{\cdot}$.
%  For two
% probability measures $\mu$ and $\nu$ on $\rset^d$, the total variation
% distance distance between $\mu$ and $\nu$ is defined by
% \begin{equation}
% %\label{eq:definition_TV}
% \tvnorm{\mu-\nu} = \sup_{\eventA \in \borelSet(\rset^d)}\abs{\mu(\eventA) - \nu(\eventA)} = (1/2) \sup_{\substack{f \in \functionspace[b]{\rset^d} \\
% \Vnorm[\infty]{f} \leq 1}}\abs{\mu(f) - \nu(f)}\eqsp.
% \end{equation}
For all $x \in \rset^d$ and $M >0$, we denote by $\boule{x}{M}$, the
ball centered at $x$ of radius $M$.  Let $\matrix$ be a $d \times
m$-matrix, then denote by $\matrix^{\Tr}$ and $\det(\matrix)$ (in the case $m=d$) the
transpose and the determinant of $\matrix$ respectively.  Let $k \geq
1$. Denote by $(\rset^{d})^{\otimes k}$ the $k^{\text{th}}$ tensor
power of $\rset^d$, for all $x \in \rset^d, y \in \rset^{\ell}$, $x
\otimes y \in (\rset^d)^{\otimes 2}$ the tensor product of $x$ and
$y$, and $x^{\otimes k} \in (\rset^{d})^{\otimes k}$ the
$k^{\text{th}}$ tensor power of $x$.  For all $x_1,\ldots,x_k \in
\rset^d$, set $\norm{x_1\otimes \cdots \otimes x_k} = \sup_{i \in
  \{1,\ldots,k\}} \norm{x_i}$. We let $\mathcal{L}((\rset^{d})^{\otimes
  k} , \rset^{\ell})$ stand for the set of linear maps from
$(\rset^{n})^{\otimes k} $ to $\rset^{\ell}$ and for $\linearmap \in
\mathcal{L}((\rset^{d})^{\otimes k} , \rset^{\ell})$, we denote by
$\normop{\linearmap}$ the operator norm of $\linearmap$. Let $f :
\rset^d \to \rset^{\ell}$ be a Lipschitz function, namely there exists
$C \geq 0$ such that for all $x,y \in \rset^d$, $ \norm{f(x) - f(y)}
\leq C \norm{x-y}$. Then we denote $\norm{f}_{\Lip} = \inf \{
\norm{f(x) - f(y)}/\norm{x-y} \ | \ x,y \in \rset^d , x \not = y
\}$.  Let $k \geq 0$ and $\open$ be an open subset of
$\rset^d$. Denote by $C^k(\open,\rset^{\ell})$ the set of all $k$
times continuously differentiable funtions from $\open$ to
$\rset^{\ell}$. Let $\Phi \in C^k(\open, \rset^{\ell})$. Write
$\Jac_{\Phi}$ for the Jacobian matrix of $\Phi \in C^{1}(\rset^d,
\rset^{\ell})$, and $D^k \Phi : \open \to
\mathcal{L}((\rset^{d})^{\otimes k} , \rset^{\ell})$ for the
$k^{\text{th}}$ differential of $\Phi \in C^{k}(\rset^d,
\rset^{\ell})$. For smooth enough functions $f
: \rset^d \to \rset$, denote by $\nabla f$ and $\nabla^2 f$ the gradient
and the Hessian of $f$ respectively. Let $\eventA \subset
\rset^d$. We write  $\clos{\eventA},\interior{\eventA}$ and
$\boundary{\eventA}$ for the closure, the interior and the boundary of
$\eventA$, respectively. For any $n_1,n_2 \in \nset$, $n_1 > n_2$, we take the convention that $\sum_{k=n_2}^{n_1} = 0$. 
% \begin{multline}
% \mathrm{log}\!\left(\mathrm{e}^{A - B} + 1\right)\\ - \mathrm{log}\!\left(\mathrm{e}^{- 3\, q_{1}\, q_{2}} + 1\right) + {\left(q_{1} - q_{2}\right)}^2 - {\left(C\right)}^2 \\- \frac{{\left(\frac{h\, \left(4\, \mathrm{e}^{3\, q_{1}\, q_{2}} + 1\right)\, \left(q_{1} - q_{2}\right)}{\mathrm{e}^{3\, q_{1}\, q_{2}} + 1} - 2\, p_{2} + \frac{h\, \left(4\, \mathrm{e}^{B - A} + 1\right)\, \left(q_{1} - q_{2} + 2\, h\, p_{2} + q_{1}\, \mathrm{e}^{- 3\, q_{1}\, q_{2}} - q_{2}\, \mathrm{e}^{- 3\, q_{1}\, q_{2}} - 4\, h^2\, q_{1} + 4\, h^2\, q_{2} - h^2\, q_{1}\, \mathrm{e}^{- 3\, q_{1}\, q_{2}} + h^2\, q_{2}\, \mathrm{e}^{- 3\, q_{1}\, q_{2}} + 2\, h\, p_{2}\, \mathrm{e}^{- 3\, q_{1}\, q_{2}}\right)}{\left(\mathrm{e}^{- 3\, q_{1}\, q_{2}} + 1\right)\, \left(\mathrm{e}^{B - A} + 1\right)}\right)}^2}{8} + \frac{{\left(q_{1} + q_{2}\right)}^2}{4} - \frac{{\left(q_{1} + q_{2} - \frac{h\, \left(4\, h\, q_{1} - 2\, p_{1} + 4\, h\, q_{2} - 2\, p_{1}\, \mathrm{e}^{3\, q_{1}\, q_{2}} + h\, q_{1}\, \mathrm{e}^{3\, q_{1}\, q_{2}} + h\, q_{2}\, \mathrm{e}^{3\, q_{1}\, q_{2}}\right)}{\mathrm{e}^{3\, q_{1}\, q_{2}} + 1}\right)}^2}{4} - \frac{{\left(\frac{h\, \left(\mathrm{e}^{3\, q_{1}\, q_{2}} + 4\right)\, \left(q_{1} + q_{2}\right)}{\mathrm{e}^{3\, q_{1}\, q_{2}} + 1} - 2\, p_{1} + \frac{h\, \left(\mathrm{e}^{B - A} + 4\right)\, \left(q_{1} + q_{2} + 2\, h\, p_{1} + q_{1}\, \mathrm{e}^{- 3\, q_{1}\, q_{2}} + q_{2}\, \mathrm{e}^{- 3\, q_{1}\, q_{2}} - h^2\, q_{1} - h^2\, q_{2} - 4\, h^2\, q_{1}\, \mathrm{e}^{- 3\, q_{1}\, q_{2}} - 4\, h^2\, q_{2}\, \mathrm{e}^{- 3\, q_{1}\, q_{2}} + 2\, h\, p_{1}\, \mathrm{e}^{- 3\, q_{1}\, q_{2}}\right)}{\left(\mathrm{e}^{- 3\, q_{1}\, q_{2}} + 1\right)\, \left(\mathrm{e}^{B - A} + 1\right)}\right)}^2}{8} + \frac{{p_{1}}^2}{2} + \frac{{p_{2}}^2}{2} 
% \end{multline}

% \begin{align}
%   A & = \frac{3\, {\left(q_{1} - q_{2} + \frac{h\, \left(2\, p_{2} - h\, q_{1} + h\, q_{2} + 2\, p_{2}\, \mathrm{e}^{3\, q_{1}\, q_{2}} - 4\, h\, q_{1}\, \mathrm{e}^{3\, q_{1}\, q_{2}} + 4\, h\, q_{2}\, \mathrm{e}^{3\, q_{1}\, q_{2}}\right)}{\mathrm{e}^{3\, q_{1}\, q_{2}} + 1}\right)}^2}{4} \\
%   B &= \frac{3\, {\left(q_{1} + q_{2} - \frac{h\, \left(4\, h\, q_{1} - 2\, p_{1} + 4\, h\, q_{2} - 2\, p_{1}\, \mathrm{e}^{3\, q_{1}\, q_{2}} + h\, q_{1}\, \mathrm{e}^{3\, q_{1}\, q_{2}} + h\, q_{2}\, \mathrm{e}^{3\, q_{1}\, q_{2}}\right)}{\mathrm{e}^{3\, q_{1}\, q_{2}} + 1}\right)}^2}{4}\\
%   C &= q_{1} - q_{2} + \frac{h\, \left(2\, p_{2} - h\, q_{1} + h\, q_{2} + 2\, p_{2}\, \mathrm{e}^{3\, q_{1}\, q_{2}} - 4\, h\, q_{1}\, \mathrm{e}^{3\, q_{1}\, q_{2}} + 4\, h\, q_{2}\, \mathrm{e}^{3\, q_{1}\, q_{2}}\right)}{\mathrm{e}^{3\, q_{1}\, q_{2}} + 1}
% \end{align}

%%% Local Variables:
%%% mode: latex
%%% TeX-master: "main"
%%% End:
