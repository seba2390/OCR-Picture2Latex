 In the sequel, $C \geq 0$
  is a constant which can change from line to line but does not depend
  on $h$. Let $h >0$ and $T \in \nset^*$. Note that a simple induction (see \cite[Proposition 4.2]{livingstone:betancourt:byrne:girolami:2016})  implies that for all $(\q_0,\p_0) \in \rset^d
\times \rset^d$ and $k \in \{1,\ldots T\}$, the $k^{\text{th}}$
iteration of the leap-frog integration, $(q_k,p_k) = \Phiverlet[h][k](\q,\p)$, where $\Phiverlet[h][k]$ is defined by \eqref{eq:def_Phiverlet}, takes the form
\begin{align}
\label{eq:qk}
\q_k&=\q_0+kh\p_0-\frac{kh^2}{2} \nabla \F(\q_0)-h^2 \gperthmc[k](\q_0,\p_0)\\
\label{eq:pk}
\p_{k}&= \p_0-\frac{h}{2} \defEns{\nabla \F(\q_0)+\nabla \F \circ \Phiverletq[h][k] (\q_0,\p_0)}-h \sum_{i=1}^{k-1}\nabla \F \circ \Phiverletq[h][i](\q_0,\p_0)  \eqsp,
\end{align}
where  $\gperthmc[k] :\rset^d \times \rset^d \to \rset^d$ is given  for all $(\q,\p) \in \rset^d \times
\rset^d$ by
\begin{equation}
  \label{eq:def_gperthmc}
\gperthmc[k](\q,\p) =    \sum_{i=1}^{k-1}(k-i)\nabla \F \circ \Phiverletq[h][i](\q,\p) \eqsp.
\end{equation}

We prefaces the proofs of our main results by useful bounds  on
  the position and the momentum in the intermediate steps of the leap-frog integration.
% Under the regularity condition
% \Cref{assum:regOne}($\beta$), it is possible to derive  useful bounds  on
%  the position and the momentum in the intermediate steps of the leap-frog integration.


  \begin{lemma}
    \label{lem:bound_first_iterate_leapfrog_a}
  Assume \Cref{assum:regOne}$(\expozero)$-\ref{assum:regOne_a}. Then, for any $k \in \nsets$, $h \geq 0$, $(q_0,p_0) \in \rset^{2d}$ and $(x_0,v_0) \in \rset^{2d}$,
  \begin{align}
   & \norm{q_k-x_k} + \constzero^{-1/2} \norm{p_k-v_k} \\
    & \qquad \qquad \leq \defEns{1+h \constzero^{1/2} \vartheta_1(h \constzero^{1/2})}^{k} \defEns{\norm{q_0-x_0} + \constzero^{-1/2} \norm{p_0-v_0}} \eqsp,
  \end{align}
  where $(q_k,p_k)= \Phiverlet[h][k](q_0,p_0)$, $(x_k,v_k)= \Phiverlet[h][k](x_0,v_0)$ and  $\Phiverlet[h][k]$ and $\vartheta_1$ are defined by \eqref{eq:def_Phiverlet} and \eqref{eq:def_vartheta_1}, respectively.
\end{lemma}
\begin{proof}
  Note that it is sufficient to show the result for $k=1$ and to apply a straightforward induction.
  Let $h >0$, $(q_0,p_0) \in \rset^{2d}$ and $(x_0,v_0) \in \rset^{2d}$. Using \eqref{eq:qk}, the triangle inequality and   \Cref{assum:regOne}$(\expozero)$-\ref{assum:regOne_a}, we first obtain
  \begin{align}
    \nonumber
    \norm{q_1-x_1} & = \norm{q_0 -h^2 \nabla U(q_0) /2 + h p_0 - \defEns{x_0 - h^2/2 \nabla U(x_0) + h v_0}} \\
    \label{eq:bound_q_1_x_1}
         & \leq (1+ h^2 \constzero /2) \norm{q_0 - x_0} + h \norm{p_0 - v_0} \eqsp.
  \end{align}
  Second, similarly using \eqref{eq:pk}, we have that
\begin{align}
\label{eq:bound_p_1_v_1}
&    \norm{p_1-v_1}  \\
&= \norm{p_0-v_0 -(h/2) \defEns{\nabla U(q_1) + \nabla U(q_0)} + (h/2) \defEns{\nabla U(x_1) + \nabla U(x_0)}} \\
\nonumber
&\leq \norm{p_0 - v_0} + (h \constzero/2) \defEns{ \norm{x_1- q_1} + \norm{x_0 - q_0}} \\
\nonumber
& \leq \left(1+h^2 \constzero/2\right) \norm{p_0-v_0} +h\constzero(1+h^2\constzero /4) \norm{q_0-x_0} \eqsp,
\end{align}
  where we have used \eqref{eq:bound_q_1_x_1} for the last inequality. Summing up \eqref{eq:bound_q_1_x_1} and \eqref{eq:bound_p_1_v_1}, we get the desired result for $k=1$.
  \end{proof}
  %\label{lem:bound_first_iterate_leapfrog_1} devient ...
  \begin{lemma}
  \label{lem:bound_first_iterate_leapfrog_b}
Let $\beta \in \ccint{0,1}$ and assume \Cref{assum:regOne}$(\expozero)$-\ref{assum:regOne_b}.
  \begin{enumerate}[label=(\roman*)]
\item
\label{lem:bound_first_iterate_leapfrog_1}
% \alain{ne suppose que \eqref{eq:bound_nabla_F_assum_reg_zero}}
For any $h_0 >0$, $T \in \nsets$, there exists $C < \infty$ (which depends only on $T,h_0$
 and $\constzeroT$) such that for all $h \in \ocint{0,h_0}$,
  $(\q_0,\p_0) \in \rset^d \times \rset^d$ and $k \in \{1,\ldots, T\}$
  \begin{align}
\label{lem:bound_first_iterate_leapfrog_1_q}
    \norm{\q_k-\q_0} &\leq C h\defEns{  \norm{\p_0} +h(1+ \norm{\q_0}^{\expozero})}\\
\label{lem:bound_first_iterate_leapfrog_1_p}
    \norm{\p_k-\p_0}& \leq C h\defEns{  1+ \norm{\p_0}^{\expozero}+ \norm{\q_0}^{\expozero}} \eqsp,
  \end{align}
where $(\q_k,\p_k) = \Phiverlet^{\circ k}_{h}(\q_0,\p_0)$ and  $\Phiverlet^{\circ k}_{h}$ is defined by \eqref{eq:def_Phiverlet}.
\item \label{lem:bound_first_iterate_leapfrog_b_2}
  If in addition \Cref{assum:regOne}$(\expozero)$-\ref{assum:regOne_a} holds, for any $k \in \nsets$, $h >0$, $(q_0,p_0) \in \rset^{2d}$,
  \begin{align}
    &    \norm{q_k-q_0} + \constzero^{-1/2}\norm{p_k - p_0} \leq (\constzero^{1/2}\vartheta_1(h \constzero^{1/2}))^{-1} \\
    & \quad \times \defEns{(1+h \constzero^{1/2}\vartheta_1(h \constzero^{1/2}))^{k+1} - 1} \defEns{\vartheta_2(h) (\norm[\beta]{q_0} +1) + \vartheta_3(h) \norm{p_0}} \eqsp,
  \end{align}
  where $\vartheta_1$ is defined by \eqref{eq:def_vartheta_1} and
  \begin{align}
  \label{eq:definition-vartheta-2}
    \vartheta_2(h) &= \constzeroT/\constzero^{1/2} + \constzeroT h/2 + \constzero^{1/2}\constzeroT h^2/4 \eqsp, \\
  \label{eq:definition-vartheta-3}
    \vartheta_3(h) &= 1+\constzero^{1/2} h /2 \eqsp.
  \end{align}
  \end{enumerate}
\end{lemma}
\begin{proof}
  \begin{enumerate}[label={(\roman*)},wide=0pt, labelindent=\parindent]
% \item We show by induction that for all $k \in \{1,\cdots,T\}$, there
%   exists $C_k$ (which depends only on $T,h_0$ and  $\constzero$) such that for all $ h \in \ocint{0, h_0}$, $(q_0,p_0)
%   \in \rset^d \times \rset^d$ and $(\tilde{q}_0,\tilde{p}_0) \in
%   \rset^d \times \rset^d$,
%   \begin{equation}
%     \norm{q_k-\tilde{q}_k} \leq C_k (\norm{q_0-\tilde{q}_0} + \norm{p_0-\tilde{p}_0}) \eqsp,
%   \end{equation}
%   where $(q_k,p_k) = \Phiverlet^{\circ k}_{h}(q_0,p_0)$,
%   $(\tilde{q}_k,\tilde{p}_k) = \Phiverlet^{\circ
%     k}_{h}(\tilde{q}_0,\tilde{p}_0)$.  Let $ h \in \ocint{0, h_0}$,
%   $(q_0,p_0) \in \rset^d \times \rset^d$ and
%   $(\tilde{q}_0,\tilde{p}_0) \in \rset^d \times \rset^d$.  The case $k=1$ is immediate
%   by \Cref{assum:regOne}($\beta$)-\ref{assum:regOne_a}. Let $k \in \{1,\cdots, T-1\}$ and assume
%   that the inequality holds for all $i \in \{1,\cdots, k\}$. Then by
%   \eqref{eq:qk} and \Cref{assum:regOne}($\beta$)-\ref{assum:regOne_a} we get
% \begin{align}
%   &     \norm{q_{k+1}-\tilde{q}_{k+1}} \leq \norm{q_0- \tilde{q}_0}+(k+1)h\norm{p_0-\tilde{p}_0}+(1/2)(k+1)h^2 \norm{\nabla \F(q_0)-\nabla \F(\tilde{q}_0)}\\
%   & \phantom{\norm{q_{k+1}-\tilde{q}_{k+1}}}\phantom{\leq \norm{q_0- \tilde{q}_0}+(k+1)haaa} +h^2\sum_{i=1}^{k}(k+1-i)\norm{\nabla \F(q_i) - \nabla \F(\tilde{q}_i)} \\
% & \phantom{\norm{q_{k+1}-\tilde{q}_{k+1}}}\leq \norm{q_0- \tilde{q}_0}+(k+1)h\norm{p_0-\tilde{p}_0}+2^{-1}(k+1) h^2  \constzero \norm{q_0-\tilde{q}_0}\\
% & \phantom{\norm{q_{k+1}-\tilde{q}_{k+1}}}\phantom{\leq \norm{q_0- \tilde{q}_0}+(k+1)haaa}+ h^2\constzero \sum_{i=1}^{k}(k+1-i)\norm{q_i -\tilde{q}_i} \eqsp.
%    \end{align}
% An application of the induction hypothesis concludes the proof.
  \item Let $T \in \nsets$ and $h_0 >0$.  We prove by induction that for all $k \in \{1,\ldots,T\}$ there exists $C_k \geq 0$ (which depends only on $T,h_0$
and $\constzeroT$) such that for all $h \in \ocint{0,h_0}$ and
  $(q_0,p_0) \in \rset^d \times \rset^d$
\begin{equation}
\label{lem:bound_first_iterate_leapfrog_1_q}
\begin{aligned}
\norm{q_k-q_0} \leq C_k h\defEns{  \norm{p_0} +h(1+ \norm{q_0}^{\expozero})} \\
\norm{p_k-p_0} \leq C_k h \defEns{ 1 + \norm{p_0}^{\expozero} + \norm{q_0}^{\expozero}} \eqsp.
\end{aligned}
\end{equation}
  where $(q_k,p_k) = \Phiverlet^{\circ k}_{h}(q_0,p_0)$.
 Let $ h \in \ocint{0, h_0}$ and $(q_0,p_0) \in \rset^d \times
    \rset^d$.
 The case $k=1$ is immediate by
\Cref{assum:regOne}($\beta$)-\ref{assum:regOne_b} and \eqref{eq:qk}. Let $k \in \{1,\cdots,
T-1\}$ and assume that the inequalities hold for all $i \in
\{1,\dots, k\}$. Then by \eqref{eq:qk} and
\Cref{assum:regOne}($\beta$)-\ref{assum:regOne_b}, we get
\begin{align}
\label{eq:lem:bound_first_iterate_leapfrog_1}
\norm{q_{k+1}-q_0}
&\leq (k+1)h \norm{p_0}+\frac{k+1}{2}h^2 \constzeroT  \defEns{1+\norm{ q_0}^{\expozero} }\\
&\qquad \qquad +h^2 \constzeroT \sum_{i=1}^{k}(k+1-i)\defEns{1+ \norm{q_i}^{\expozero} }\eqsp.
\end{align}
By the induction hypothesis and using that $t \mapsto t^{\expozero}$ is sub-additive on $\rset^+$ and $t^\beta \leq 1 + t$ for $t \in \rset^+$, we get for all $i \in \{1,\cdots, k\}$,
\begin{equation}
\norm{q_i}^{\expozero} \leq \norm{q_0}^{\expozero} + \norm{q_i-q_0}^\beta \leq 1+ \norm{q_0}^{\expozero} +C_i h\defEns{\norm{p_0}+h(1+ \norm{q_0}^{\expozero})} \eqsp,
\end{equation}
Plugging this inequality in \eqref{eq:lem:bound_first_iterate_leapfrog_1}
conclude the proof of \eqref{lem:bound_first_iterate_leapfrog_1_q}.
Consider now \eqref{lem:bound_first_iterate_leapfrog_1_p}. Since by
definition $p_{k+1} = p_k-(h/2)\defEnsLigne{\nabla \F(q_{k}) + \nabla
  \F(q_{k+1})}$, using the triangle inequality,
\Cref{assum:regOne}($\beta$)-\ref{assum:regOne_b},
\eqref{lem:bound_first_iterate_leapfrog_1_q} to bound $\norm{q_{k}}$ and
$\norm{q_{k+1}}$, and the induction hypothesis, we get that there exist some
constants $C_{k+1,1},C_{k+1,2}$ which only depend on $T,h_0$ and
$\constzeroT$ such that
\begin{align}
&  \norm{p_{k+1} - p_0} \leq \norm{p_k-p_0} + (h/2)\defEnsLigne{\norm{\nabla \F(q_{k})} + \norm{\nabla \F(q_{k+1})}} \\
&\quad \leq C_{k+1,1}h\defEns{1+\norm{p_0}^{\expozero}+\norm{q_0}^{\expozero}} + (\constzeroT h/2)\defEns{2+\norm{q_{k}}^{\expozero}+ \norm{q_{k+1}}^{\expozero}}\\
&\quad \leq C_{k+1,1}h\defEns{1+\norm{p_0}^{\expozero}+\norm{q_0}^{\expozero}} + (C_{k+1,2}h/2)\defEns{1+ \norm{q_0}^{\expozero} + \norm{p_0}^{\expozero}} \eqsp.
\end{align}
Therefore, \eqref{lem:bound_first_iterate_leapfrog_1_q} is satisfied which concludes the induction and the proof.
\item Let $k \in \nset$, $h >0$ and $(q_0, p_0) \in \rset^{2d}$. Using \eqref{eq:qk}, the triangle inequality and \Cref{assum:regOne}($\beta$), we have
  \begin{align}
\label{eq:1_lem_bound_first_iterate_leapfrog_b_2}
&    \norm{q_{k+1} - q_0}  = \norm{q_k - q_0 - (h^2/2) \nabla U(q_k) + h p_k} \\
\nonumber
&                          \leq (1+h^2 \constzero/2) \norm{q_k - q_0} + (h^2\constzeroT/2) (\norm[\beta]{q_0} +1) + h \norm{p_k-p_0} + h \norm{p_0}  \eqsp.
  \end{align}
Second, similarly using \eqref{eq:pk}, we get that
\begin{align}
\nonumber
&    \norm{p_{k+1} - p_0}   \leq  \norm{p_k-p_0 + (h/2) \defEns{\nabla U(q_{k+1}) + \nabla U(q_k)} } \\
    \nonumber
                         & \leq \norm{p_k - p_0} + (h \constzero/2) \defEns{ \norm{q_{k+1}- q_0} + \norm{q_k - q_0}} + h \constzeroT \defEns{\norm[\beta]{q_0} +1} \\
    \nonumber
                         & \leq \norm{p_k - p_0}  +  h \constzeroT \defEns{\norm[\beta]{q_0} +1} +(h \constzero/2) \defEns{  + h \norm{p_k-p_0} + h \norm{p_0}} \\
    \label{eq:2_lem_bound_first_iterate_leapfrog_b_2}
&     +(h \constzero/2) \defEns{ (2+\constzero h^2/2) \norm{q_k-q_0} + (h^2\constzeroT/2) (\norm[\beta]{q_0}+1)}  \eqsp.
  \end{align}
  where we have used \eqref{eq:1_lem_bound_first_iterate_leapfrog_b_2} for the last inequality.
  Summing up \eqref{eq:1_lem_bound_first_iterate_leapfrog_b_2} and \eqref{eq:2_lem_bound_first_iterate_leapfrog_b_2} and using the definition \eqref{eq:def_vartheta_1} of $\vartheta_1(h)$, we get that, setting $A_k= \norm{q_k-q_0}  + \constzero^{-1/2} \norm{p_k-p_0}$,
  \[
  A_{k+1} \leq (1+h\constzero^{1/2}\vartheta_1(h\constzero^{1/2})) A_k    + h \defEns{ \vartheta_2(h) (\norm[\beta]{q_0} +1) +  \vartheta_3(h) \norm{p_0}} \eqsp.
\]
  By a straightforward induction, we obtain that
  \begin{equation}
    A_{k+1} \leq \sum_{i=1}^{k+1} \parentheseDeux{(1+h \constzero^{1/2} \vartheta_1(h \constzero^{1/2} ))^{k+1-i} h \defEns{ \vartheta_2(h) (\norm[\beta]{q_0} +1) +  \vartheta_3(h) \norm{p_0} }} \eqsp,
  \end{equation}
  which completes the proof of \ref{lem:bound_first_iterate_leapfrog_b_2}.
\end{enumerate}
\end{proof}
% \begin{proof}
%   The proof is postponed to \Cref{sec:proof-crefl-lem:bound_first_iterate_leapfrog}.
% \end{proof}

\begin{lemma}
\label{lem:inverse_1}
Let $\beta \in \ccint{0,1}$ and assume \Cref{assum:regOne}($\beta)$. Then for any $T \in \nsets$, $h >0$,
\begin{align}
\label{eq:lem:inverse_1}
  &  \sup_{(\q,\p,v) \in \rset^{3d}} \{ \norm{ \gperthmc[T](q,\p) - \gperthmc[T](q,v)} / \norm{\p-v} \} \\
  & \qquad \qquad \qquad \qquad  \leq   (T/h)  \left\{ (1+\ h \constzero^{1/2} \vartheta_1(h \constzero^{1/2}))^T - 1 \right\} \eqsp,
\end{align}
where  $\gperthmc[T]$ and $\vartheta_1$ are defined by  \eqref{eq:def_gperthmc} and \eqref{eq:def_vartheta_1} respectively.
  In addition, for any $q \in \rset^d$,
  \begin{equation}
    \label{eq:lem:inverse_2}
    \norm{\gperthmc[T](q,0)} \leq  (T/h)  \{ (1+h \constzero^{1/2} \vartheta_1(h \constzero^{1/2} ))^{T}-1\}  \vartheta_2(h) (\norm[\beta]{q} +1)  + T^2 \norm{\nabla U(q)} \eqsp,
  \end{equation}
  where $\vartheta_2$ is defined in \eqref{eq:definition-vartheta-2}.
\end{lemma}
\begin{proof}
By
\Cref{lem:bound_first_iterate_leapfrog_a}, for any $i \in \nsets$, we get
\[
\sup_{(q,\p,v) \in \rset^{3d}} \defEns{ \normLigne{ \Phiverletq[h][i](q,\p) - \Phiverletq[h][i](q,v)} / \norm{\p-v} }
\leq \constzero^{-1/2} A^i
\]
where $A=(1+h \constzero^{1/2} \vartheta_1(h \constzero^{1/2} ))$.
Therefore by definition of $\gperthmc[T]$ \eqref{eq:def_gperthmc} and using  \Cref{assum:regOne}($\beta$), for any $h >0$, $T \in \nsets$, we get that
\begin{align}
&  \sup_{(\q,\p,v) \in \rset^{3d}} \{ \norm{ \gperthmc[T](q,\p) - \gperthmc[T](q,v)} / \norm{\p-v} \} \\
&\qquad  \leq \constzero   \sum_{k=1}^{T-1} (T-i) \sup_{(q,\p,v) \in \rset^{3d}} \defEns{ \normLigne{ \Phiverletq[h][i](q,\p) - \Phiverletq[h][i](q,v)} / \norm{\p-v} }\\
%& \qquad  \leq  \constzero^{1/2} \sum_{k=1}^{T-1} (T-i) \{1+h \constzero^{1/2} \vartheta_1(h \constzero^{1/2})\}^i  \\
& \qquad \leq    T \parentheseDeux{ A^{T} - 1 } /(h \vartheta_1(\constzero^{1/2} h)) \,
\end{align}
showing \eqref{eq:lem:inverse_1} since $\vartheta_1(h\constzero^{1/2}) \geq 1$.

We now consider \eqref{eq:lem:inverse_2}. By \eqref{eq:def_gperthmc}, \Cref{assum:regOne}($\beta$)-\ref{assum:regOne_a} and \Cref{lem:bound_first_iterate_leapfrog_b}-\ref{lem:bound_first_iterate_leapfrog_b_2}, we have that for any $q \in \rset^d$,
\begin{align}
&  \norm{\gperthmc[T](q,0)}  \leq \sum_{i=1}^{T-1} (T-i)  \norm{ \nabla U \circ \Phiverletq[h][i](\q,0) - \nabla U(q)} + T^2 \norm{\nabla U(q)} \\
& \leq  T \constzero \sum_{i=1}^{T-1}  \norm{ \Phiverletq[h][i](\q,0) -q} + T^2 \norm{\nabla U(q)} \\
& \quad \leq T \constzero^{1/2} \vartheta_1^{-1} (h \constzero^{1/2}) \sum_{i=1}^{T-1}\{A^{i+1}-1\} \defEns{\vartheta_2(h) (\norm[\beta]{q} +1)}  + T^2 \norm{\nabla U(q)} \eqsp,
\end{align}
which completes the proof of \eqref{eq:lem:inverse_2} using that $\vartheta_1(h \constzero^{1/2}) \geq 1$.
\end{proof}

\begin{lemma}
  \label{lem:bounded_cum_error}
  Assume \Cref{assum:regOne}$(1)$. Then for any $T \in \nsets$, $h >0$, and $q,p \in \rset^d$,
  \begin{align}
    %\label{eq:3}
    &\sum_{i=1}^{T}  \norm{ \Phiverletq[h][i](\q,p) -q} \\
    &\leq  L^{-1/2}_1 T [\{1 + h  \constzero^{1/2} \vartheta_1(h\constzero^{1/2})\}^T  -1] \defEns{\vartheta_2(h) (1+\norm{q})  + \vartheta_3(h) \norm{p}} \eqsp,
  \end{align}
where  $\vartheta_1$ is defined by  \eqref{eq:def_vartheta_1}.
\end{lemma}
\begin{proof}
For  $T \geq 2$ and $h >0$,  by  \Cref{lem:bound_first_iterate_leapfrog_b}-\ref{lem:bound_first_iterate_leapfrog_b_2},   we have
\begin{align}
\sum_{i=1}^{T-1}  \norm{ \Phiverletq[h][i](\q,p) -q}
&\leq 
%\constzero^{-1/2} \vartheta_1^{-1}(h \constzero^{1/2})  \sum_{i=1}^{T-1}\parentheseDeux{\defEns{(1+h\vartheta_1(h))^{i+1}-1} \defEns{\vartheta_2(h) (\norm{q} +1) + \vartheta_3(h) \norm{p}}} \\
 \constzero^{-1/2} \vartheta_1^{-1}(h \constzero^{1/2}) T \defEns{\{1+h \constzero^{1/2}\vartheta_1(h\constzero^{1/2})\}^{T}-1} \\
 & \qquad \qquad \times \, \defEns{\vartheta_2(h) (\norm{q} +1) + \vartheta_3(h) \norm{p}} \eqsp.
\end{align}
The proof is completed upon using that $\vartheta_1(h \constzero^{1/2}) \geq 1$.
\end{proof}


%%% Local Variables:
%%% mode: latex
%%% TeX-master: "main"
%%% End:
