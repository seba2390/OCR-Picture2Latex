We preface the proof by recalling some basic notions of degree theory.
\label{sec:defin-usef-results}
Let $\Dset$ be a bounded open set of $\rset^d$. Let $f:
\Dsetc \to \rset^d$ be a continuous function on
$\Dsetc$ continuously differentiable on $\Dset$. An element $x \in
\Dset$ is said to be a \emph{regular point} of $f$ if the Jacobian matrix of $f$ at $x$, $\Jac_f(x)$, is invertible.
An element $y \in f(\Dset)$ is said to be a \emph{regular value} of $f$ if any $x \in
f^{-1}(\{ y\})$ is a regular point.  %true if $y \not in f(\Dset)$

Let $f : \Dsetc \to \rset^d$ be a continuous function,  $C^{\infty}$-smooth on $\Dset$. Let $y \in \rset^d
  \setminus f(\partial \Dset)$ be a regular value of $f$. It is shown in \cite[Proposition and Definition 1.1]{outerelo:ruiz:2009} that the set $f^{-1}(\{y\})$ is finite. The degree of $f$ at $y$ is defined by
\begin{equation}
  \deg(f,\Dset,y) = \sum_{x \in f^{-1}(\{y \})} \sign\defEns{\det \parenthese{\Jac_f(x)}} \eqsp.
\end{equation}

\begin{proposition}[\protect{\cite[Proposition and Definition 2.1]{outerelo:ruiz:2009}}]
\label{defProp:degree_cont}
  Let $f : \Dsetc \to \rset^d$ be a continuous function and $y \in
  \rset^d \setminus f(\partial \Dset)$.
  \begin{enumerate}[label=(\alph*)]
  \item
\label{defProp:degree_cont_i}
 Then there exists  $g \in C(\Dsetc, \rset^d) \cap C^{\infty}(\Dset, \rset^d)$ such that $y$ is a regular value of $g$
  and $\sup_{x \in \Dsetc} \abs{f(x)-g(x)} < \dist(y,f(\partial
  \Dset))$.
\item For all functions $g_1,g_2:\Dsetc \to \rset^d$ satisfying \ref{defProp:degree_cont_i},
  \begin{equation}
    \deg(g_1,\Dset,y) = \deg(g_2,\Dset,y) \eqsp.
  \end{equation}
  \end{enumerate}
\end{proposition}
Under the assumptions of \Cref{defProp:degree_cont}, the degree of $f$ at $y$ is then defined for any $g:\Dsetc \to \rset^d$ satisfying \ref{defProp:degree_cont_i} by
\begin{equation}
  \deg(f,\Dset,y) =  \deg(g,\Dset,y) \eqsp.
\end{equation}

\begin{proposition}[\protect{\cite[Proposition
  2.4]{outerelo:ruiz:2009}}]
  \label{theo:deg_modif}
  Let $f,g : \Dsetc \to \rset^d$ be  continuous functions. Define
  $\hpy:\ccint{0,1} \times \rset^d \to \rset^d$ for all $t \in
  \ccint{0,1}$ and $x \in \rset^d$ by $\hpy(t,x) = t f(x) +
  (1-t)g(x)$. Let $y \in \rset^d \setminus \hpy(\ccint{0,1} \times \partial \Dset)$. Then
\begin{equation}
  \deg(f,\Dset,y) =  \deg(g,\Dset,y) \eqsp.
\end{equation}
\end{proposition}
We have now all the necessary results to prove \Cref{le:degree_application}.
\begin{proof}[Proof of \Cref{le:degree_application}]
Since $\hga(x,z) =  \bg z + \ga(x,z)$ and $\ga(x,\cdot)$ is Lipschitz with a Lipschitz constant which is uniformly bounded over the ball $\ball{0}{R}$,  $\hga_x$ is Lipschitz with bounded Lipschitz constant over this ball. Hence \Cref{assumG:irred_b}($\ra,0,\MassG$)-\ref{assumG:irred_b_item_i} holds.

  For all $x \in \rset^d$, denote by $\hga_x : z \mapsto \hga(x,z)$ where $\hga(x,z)=bz + g(x,z)$.
  Let $\MassG \in \rset_+^*$. We show that for all $x \in
  \ball{0}{\ra}$, $\ball{0}{\MassG} \subset
  \hga_x(\ball{0}{\tMassG})$, where $\tMassG$ is given by
  \eqref{eq:deftildeM}, which is precisely
  \Cref{assumG:irred_b}($\ra,0,\MassG$)-\ref{assumG:irred_b_item_ii}.

 % Then for all $z \in
%   (\hga_x)^{-1}(\ball{0}{\MassG}) $, by \ref{propo:irred_b_item_ii}
% \begin{equation}
%   \MassG \geq \abs{\hga_x(z)} \geq  \abs{ \bg z}  - \Cga_{\ra,0} -\Cga_{\ra,1} \abs{z} \eqsp.
% \end{equation}
% Therefore since $\abs{\bg} \geq \Cga_{\ra,1} $, $(\hga_x)^{-1}(\ball{0}{\MassG}) \subset \ball{0}{\tMassG}$ where $\tMassG$ is given by
%\eqref{eq:deftildeM}.
% $\tMassG = \{\MassG + \ra \abs{\ag} +
% \Cga(1+\abs{\ra})\}/(\abs{\bg}-\Cga)$.
%Next we show \ref{item:proof:homot_ii}.
%  Let $x \in \ball{0}{\ra}$ and
% $\MassG \geq 0$.
  Let $x \in \ball{0}{\ra}$ and consider the continuous homotopy $\hog : \ccint{0,1}
\times \rset^d$ between the functions $z \mapsto \bg z$ and $\hga_x$ defined for all
$t \in \ccint{0,1}$ and $z \in \rset^d$ by
\begin{equation}
  \hog(t,z) = t \bg z + (1-t)\hga_x(z) = \bg z + (1-t)  \ga(x,z)  \eqsp.
\end{equation}
Then by \ref{propo:irred_b_item_ii}, since $\abs{\bg} \geq \Cga_{\ra,1} $, for all $t\in \ccint{0,1}$ and $z \not \in
\ball{0}{\tMassG}$, where $\tMassG$ is given by \eqref{eq:deftildeM},
\begin{equation}
   \abs{\hog(t,z)} \geq  \abs{\bg z} -(1-t)\defEns{\Cga_{\ra,0} +\Cga_{\ra,1} \abs{z} } \geq \MassG \eqsp.
\end{equation}
In particular, we have $\hog(\ccint{0,1} \times \partial
\ball{0}{\tMassG}) \subset \rset^d \setminus \ball{0}{\MassG}$. Let
$z \in \ball{0}{\MassG}$, then by
%\cite[Proposition 2.4, Proposition-Definition 1.1, Chapter IV]{outerelo:ruiz:2009},
\Cref{theo:deg_modif} we have
\begin{equation}
  \deg(\hga_x,\ball{0}{\tMassG},z) = \deg(\bg \Id, \ball{0}{\tMassG},z) = 1 \eqsp.
\end{equation}
Besides, by \cite[Corollary 2.5, Chapter IV]{outerelo:ruiz:2009},
$\deg(\hga_x,\ball{0}{\tMassG},z) \not = 0$ implies that there exists
$y \in \ball{0}{\tMassG}$ such that $\hga_x(y) = z$. Finally \Cref{assumG:irred_b}($\ra,0,\MassG$)-\ref{assumG:irred_b_item_ii} follows since this
result holds for all $z \in \ball{0}{\MassG}$.
\end{proof}
% which implies that $\hga_x(\rset^d \setminus
% \ball{0}{\tilde{M}}) \subset \rset^d \setminus \ball{0}{M}$
% $(\hga_x)^{-1}(\ball{0}{M}) \subset \ball{0}{\tilde{M}}$.
% % Then \ref{application:irred_b_item_i} implies
% %   that \Cref{assumG:irred_b}($\ra$)-\ref{assumG:irred_b_item_i} holds. We now show that under

% We first show that for all $x \in \rset^d$, $\Psi_x$ is surjective from $\rset^d$ to $\rset^d$. Now a standard perturbation
% theorem from degree theory (see e.g. \alain{I think we can cite
%   Milnor}), applied to the continuous family of maps
% $\{\Phi_x(t,\cdot) \, : \, t \in \ccint{0,1} \}$, defined for all $t
% \in \ccint{0,1}$
% $$
% y \mapsto \Phi_{x}(t,y):=ay+ t(bx+g(x,y)) \eqsp,
% $$
% implies that $\Psi_{x}$ is surjective\alain{expliquer }.
%  Assume that $R>0$ is given. Our  assumptions imply clearly that there exists  $M>R/\abs{a}$ such that for all $x \in \ball{0}{R}$ and $y \not \in \ball{0}{M}$,
% $$
% \abs{ay}-\abs{bx+g(x,y)}\geq R \eqsp.
% $$
% Especially, this holds at the boundary $\{y \in \rset^d \, : \,
% \abs{y} = M \}$. Let $x \in \ball{0}{R}$. Therefore $\ball{0}{R}\subset
% \Psi_x(\ball{0}{M})$ for all $x \in \ball{0}{R}$. Let $L$ be the
% Lipschitz constant of $g$ on $B(0,R)\times B(0,M)$.
%  \Cref{le:simple} yields that for any $x\in B(0,R)$ and $\eventA \in \borelSet(\rset^d)$, $\eventA \subset \ball{0}{R}$
%  \begin{equation}
%    \lambda_{\Psi_x}(\eventA) \geq L^{-d}\Leb(\eventA) \eqsp,
%  \end{equation}
% and the proof follows.
%  the
% push-forward measure of the Lebesgue measure on $B(0,R_0)$ under $y\to
% H(x_0,y)$ has a lower density bound $c> 0$ on the ball $B(0,R)$, that
% is independent of $x_0$. As the Gaussian distribution has lower
% bounded density on $B(0,R_0)$ the claim follows.


%%% Local Variables:
%%% mode: latex
%%% TeX-master: "main"
%%% End:
