

For $h >0$ and $T \in \nset^*$, consider the Markov kernel $\Pkerhmc[h][T]$ associated with the Markov chain of the HMC algorithm $(Q_k)_{k \in \nset}$, given for all $\q \in \rset^d$ and $\eventA \in \borelSet(\rset^d)$ by
\begin{align}
  \Pkerhmc[h][T](\q, \eventA) &= \int_{\rset^d} \indi{\eventA}{\Phiverletq[h][T](\q,\tilde{\p})} \ \alphaacc\defEns{(\q,\tilde{\p}),\Phiverlet[h][T](\q,\tilde{\p})}\frac{\rme^{-\norm[2]{\tilde{\p}}/2}}{ (2 \uppi)^{d/2}}  \rmd \tilde{\p}
                                  \nonumber
  \\
&\qquad + \updelta_{\q}(\eventA)  \,   \int_{\rset^d}  \parentheseDeux{1-\alphaacc\defEns{(\q,\tilde{\p}),\Phiverlet[h][T](\q,\tilde{\p})}} \frac{\rme^{-\norm[2]{\tilde{\p}}/2}}{ (2 \uppi)^{d/2}}  \rmd \tilde{\p} \eqsp,
 \label{eq:def_kernel_hmc}
\end{align}
where $\Phiverletq[h][T]$, $\Phiverlet[h][T]$ and $\alphaacc$ are defined by \eqref{eq:def_Phiverlet}-\eqref{eq:def_Phiverletq} and \eqref{eq:def_acc_ratio} respectively.
In this Section, we establish conditions upon which the Markov kernel $ \Pkerhmc[h][T]$ is irreducible or
(Harris) recurrent. % Not surprisingly these conditions imply regularity
%conditions and control of the tails of the target distribution $\pi$.  % Our results are established for
% the marginal chain for which $\pi$ is invariant, but irreducibility
% for the Markov kernel associated with the position and momentum is
% crucial for some variants of HMC, in particular when the process
% associated with the position is no longer Markov, (see
% \cite[Proposition 3.7]{bou:sanz:2017}).
For
$\expozero \in \ccint{0,1}$, we consider the following assumption on
the potential $\F$.

\begin{assumption}[$\expozero$]
  \label{assum:regOne}
  $\F$ is continuously differentiable and
  \begin{enumerate}[label=(\roman*)]
  \item
  \label{assum:regOne_a}
 there exists $\constzero > 0$  such that for all $\q,x \in \rset^d$,
\begin{equation}
\norm{\nabla \F(\q) - \nabla \F(x)} \leq \constzero\norm{\q-x} \eqsp.
  \end{equation}
\item    \label{assum:regOne_b}
there exists $\constzeroT \geq 0$  such that for all $\q \in \rset^d$,
\begin{equation}
%\label{eq:bound_nabla_F_assum_reg_zero}
  \norm{\nabla \F(\q)} \leq \constzeroT\defEns{ 1 + \norm{\q}^{\expozero}} \eqsp.
\end{equation}
  \end{enumerate}
\end{assumption}


% For all $T \in \nset^*$, define $\gpertub[h][T] : \rset^d \times \rset^d \to \rset^d$ for all $(\q,\p) \in \rset^d \times \rset^d$ by
% \begin{equation}
%   \gpertub[h][T](\q,\p) =   \Phiverletq[h][k](\q,\p) - \q \eqsp.
% \end{equation}
% \Cref{lem:bound_first_iterate_leapfrog} shows that there exists $C
% \geq 0$ such that for all $\q \in \rset^d$,
% We first state our main two results regarding the ergodicity of the Markov kernel
% $\Pkerhmc[h][T]$ for $h \in \rset_+^*$ and $T \in \nset$.
Before going further, we need to briefly recall some definitions pertaining to Markov chains.
Let $\Pker$ be a Markov kernel on $(\rset^d,\borelSet(\rset^d))$. Let $n$ be an integer and $\mu$
be a nontrivial measure on $\borelSet(\rset^d)$. A
set $\Csf \in \borelSet(\rset^d)$ is called a $(n,\mu)$-small set for $\Pker$ if
for all $x \in \Csf$ and $\Asf \in \borelSet(\rset^d)$, $\Pker^n(x, \msa) \geq \mu(\msa)$.
A set $\Asf \in \borelSet(\rset^d)$ is said to be accessible for $\Pker$
  if for all $x \in \rset^d$, $\sum_{i=1}^\infty \Pker^i(x,\Asf) > 0$.
  A non-trivial $\sigma$-finite measure $\mu$ is an irreducibility
  measure of $\Pker$ \iff\ any set $\Asf \in \borelSet(\rset^d)$
  satisfying $\mu(\Asf) >0$ is accessible.  The Markov kernel $\Pker$ is said to be
  irreducible if it admits an accessible small set or equivalently an
  irreducibility measure (in \cite{meyn:tweedie:2009}, our notion of irreducibility  is referred to as $\phi$-irreducibility, where $\phi$ is an irreducibility measure; here irreducibility therefore means $\phi$-irreducibility). $\Pker$ is said to be a \Tkernel~is there exists a kernel $\Tker$ on $\rset^d \times \mcb(\rset^d)$ and a sequence of non-negative numbers $(a_i)_{i \in \nsets}$ satisfying $\sum_{i=1}^{\plusinfty} a_i =1$, such that
  \begin{enumerate*}[label=(\roman*)]
  \item for any $x \in \rset^d$, $\Tker(x, \rset^d) >0$;
  \item for any $\msa \in \mcb(\rset^d)$, $x \mapsto \Tker(x,\msa)$ is lower semi-continuous;
\item for any $x \in \rset^d$, $\msa \in \mcb(\rset^d)$, $\sum_{i=1}^{\plusinfty} a_i \Pker^i(x,\msa) \geq \Tker(x,\msa)$.
  \end{enumerate*}
  $\Tker$ is referred to as a continuous component of $\Pker$.


  Let $(X_n)_{n \geq 0}$ be the canonical chain associated with $\Pker$
  defined on the canonical space $(\Omega,\mathcal{F},(\mathbb{P}_x, x \in \rset^d))$. A
  set $\Asf \in \borelSet(\rset^d)$ is said to be recurrent if for all $x \in \msa$, $\PE_x[N_\msa]= \plusinfty$ where $N_\msa = \sum_{i=0}^{\plusinfty} \1_{\msa}(X_i)$ is the number of visits to $\msa$. The set $\msa$ is Harris recurrent  if for any $x \in \msa$, $\mathbb{P}_x(N_\msa = \plusinfty) = 1$. The Markov kernel $\Pker$ is said to be Harris
  recurrent if all accessible sets are Harris recurrent. In this case, for all $x \in \rset^d$, and all accessible sets $\msa$, $\PP_x(N_\msa = \plusinfty)=1$.

  Define $\vartheta_1 : \rset_+ \to \rset_+$, for any $s \in \rset_+$ by
  \begin{equation}
    \label{eq:def_vartheta_1}
    \vartheta_1(s) = 1+  s/2 + s^2/4\eqsp.
  \end{equation}
\begin{theorem}
  \label{theo:irred_harris}
Assume \Cref{assum:regOne}($\expozero$) for some $\expozero \in \ccint{0,1}$ and that $\F$ is twice continuously
differentiable. Then, for all $T \in \nsets$, and  $h > 0$ satisfying
\begin{equation}
\label{eq:condition-h,T-harris}
 \left[ \{1 + h\constzero^{1/2} \vartheta_1(h\constzero^{1/2}) \}^T - 1 \right] < 1 \eqsp,
\end{equation}
and $q \in \rset^d$, there exists a $C^1(\rset^d,\rset^d)$-diffeomorphism $\tilde{q} \mapsto \Phiverletqi[h][T](q,\tilde{q})$ such that for any $p \in \rset^d$,
\begin{equation}
\label{theo:irred_harris_a}
\text{if $q_T =   \Phiverletq[h][T](q,p)$, defined by \eqref{eq:def_Phiverletq}, then $p = \Phiverletqi[h][T](q,q_T)$} \eqsp.
\end{equation}
Moreover,
\begin{enumerate}[label=(\roman*), wide, labelwidth=!, labelindent=0pt]
\item   \label{theo:irred_harris_b}
The Markov kernel $\Pkerhmc[h][T]$, is a \Tkernel; more precisely, for any $\eventB \in \mcb(\rset^d)$,
\begin{align}
\label{eq:def_kernel_hmc_false_density}
&\Pkerhmc[h][T](q, \eventB) =  \Tker_{h,T}(q,\eventB) \\
&\qquad + \updelta_{q}(\eventB)(2 \uppi)^{-d/2} \int_{\rset^d}  \parentheseDeux{1-\alphaacc\defEns{(q,\tilde{p}),\Phiverlet[h][T](q,\tilde{p})}} \rme^{-\norm{\tilde{p}}^2/2} \rmd \tilde{p} \eqsp,
\end{align}
where the kernel $ \Tker_{h,T}$ is a continuous component of $\Pkerhmc[h][T]$  and is given by
\begin{equation}
  \label{eq:def_tker}
\Tker_{h,T}(q,\eventB)
  =   (2 \uppi)^{-d/2} \int_{\eventB}    \balphaacc(q,\bar{q})\rme^{-\norm{\Phiverletqi[h][T](q,\bar{q})}^2/2} \detj_{\Phiverletqi[h][T](q,\cdot)}(\bar{q})  \rmd \bar{q} \eqsp,
\end{equation}
setting for $q,\tilde{q} \in \rset^d$, $\balphaacc(q,\bar{q}) =  \alphaacc\defEns{(q,\Phiverletqi[h][T](q,\bar{q})),\Phiverlet[h][T](q,\Phiverletqi[h][T](q,\bar{q}))}$ and  $\detj_{\Phiverletqi[h][T](q,\cdot)}(\tilde{q}) = \absLigne{\det(\Jac_{\Phiverletqi[h][T](q,\cdot)}(\tilde{q}))}$.
\item \label{theo:irred_harris_c} The Markov kernel $\Pkerhmc[h][T]$ is irreducible and the Lebesgue measure is an irreducibility measure. Moreover,  $\Pkerhmc[h][T]$ is aperiodic, Harris recurrent and all the compact sets are $1$-small. Therefore, for all $\q \in \rset^d$,
\begin{equation}
\label{eq:harris-theorem}
\lim_{n \to \plusinfty}    \tvnorm{\delta_\q \Pkerhmc[h][T]^n - \pi} = 0 \eqsp.
\end{equation}
\end{enumerate}
\end{theorem}

\begin{proof}
The proof is postponed to \Cref{sec:proof-crefth-harris_0}.
\end{proof}
%\erici{à vérifier, mais ça doit être bon}
For all $h > 0$ and $T \in \nsets$, we have
$\{1 + h \constzero^{1/2} \vartheta_1(h \constzero^{1/2} ) \}^T -1 \leq \rme^{h \constzero^{1/2} T \vartheta_1(h \constzero^{1/2} T)} -1$.
using that  $\vartheta_1$ is nondecreasing.
Then, setting $\bar{S} = c \constzero^{-1/2}$ where $c$ is   the unique positive root  of the equation
$c \vartheta_1(c)  = \log(2)$,  all $T \in \nsets$ and $h  \in \ooint{0,\bar{S}/T}$ satisfy \eqref{eq:condition-h,T-harris}\footnote{Note that conversely, if $h >0$ and $T \in \nsets$ satisfies \eqref{eq:condition-h,T-harris}, necessarily $h \in \oointLigne{0,\constzero^{-1/2}}$ because for any $s > 0$, $\vartheta_1(s) \geq 1$. In addition, since $\rme^{\log(2) s} \leq (1+s)$ for all $s \in \oointLigne{0,1}$, $T$ and $h$ satisfy $hT \leq \tilde{S}= \constzero^{-1/2}$.
}.


% Under \Cref{assum:regOne}, for all $h \in \rset_+^*$, $T \in \nset^*$
% and $x \in \rset^d$, $\Pkerhmc[h][T](x , \{x \}) >0$, which ensures
% that $\Pkerhmc[h][T]$ is aperiodic.
% Note that by \cite[Theorem 13.0.1]{meyn:tweedie:2009}, \Cref{theo:irred_harris} implies
% that for all $T \geq 0$, there exists $\hirr>0$ such that for all $h \in \ocintLigne{0,\hirr}$ and all $\q \in \rset^d$
%   \begin{equation}
% \lim_{n \to \plusinfty}    \tvnorm{\delta_\q \Pkerhmc[h][T]^n - \pi} = 0 \eqsp.
%   \end{equation}




%%% Local Variables:
%%% mode: latex
%%% TeX-master: "main"
%%% End:
