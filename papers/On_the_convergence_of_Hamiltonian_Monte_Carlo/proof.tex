\subsection{Proofs of \Cref{sec:ergodicity-hmc}}


% \begin{proof}
%\end{proof}

\subsubsection{Proof of \Cref{theo:irred_harris} }
\label{sec:proof-crefth-harris_0}
We first prove  \eqref{theo:irred_harris_a}.  Under the assumption that $\F$ is twice continuously
  differentiable, it follows by a straightforward induction, that for
  all $h >0$ and $q \in \rset^d$, $p \mapsto
  \Phiverletq[h][k](q,p)$, defined by  \eqref{eq:def_Phiverletq}, and $p \mapsto \gperthmc[k](q,p)$, defined by \eqref{eq:def_gperthmc}, are
  continuously differentiable and for all $(q,p) \in \rset^d \times
  \rset^d$,
\begin{equation}
  \Jac_{p,\gperthmc[T]}(q,p) =  \sum_{i=1}^{T-1}(T-i)\defEns{\nabla^2 \F \circ \Phiverletq[h][i](q,p)} \Jac_{p,\Phiverletq[h][i]}(q,p) \eqsp,
\end{equation}
where for all $q \in \rset^d$, $\Jac_{p,\gperthmc[k]}(q,p)$ ($\Jac_{p,\Phiverletq[i][h]}(q,p)$ respectively) is the Jacobian of the function $\tilde{p} \mapsto
\gperthmc[k](q,\tilde{p})$ ($\tilde{p} \mapsto
\Phiverletq[i][h](q,\tilde{p})$ respectively) at $p \in \rset^{d}$.


%  Under \Cref{assum:regOne}, $\sup_{x \in \rset^d} \normLigne{\nabla^2 \F(x)}
% \leq \constzero$ and by
% \Cref{lem:bound_first_iterate_leapfrog_a},
%  $ \sup_{(q,p) \in \rset^d \times \rset^d} \normLigne{\nabla_p \Phiverletq[h][i](q,p)} \leq (1+h \vartheta_1(h))^i$ for any $i \in \nsets$.
% Therefore for any $h >0$, $T \in \nsets$, setting $S = hT$ and using that $\tilde{h} \mapsto \vartheta_1(\tilde{h})$ is nondecreasing and greater than $1$ on $\rset_+^*$ and for any $u,s \geq 0$, $u \geq 1$, $(1+s/u)^{u-1} \leq \log(s+1) \rme^s$, we get that
Under \Cref{assum:regOne}, $\sup_{x \in \rset^d} \normLigne{\nabla^2 \F(x)}
 \leq \constzero$, therefore by \Cref{lem:inverse_1}, we have that for any $T \in \nsets$ and $h >0$,
\begin{equation}
  \label{eq:inverse_1}
 \sup_{(\q,\p) \in \rset^d \times \rset^d} \norm{\Jac_{p,\gperthmc[T]}(q,p)}
 \leq  T (\{1 + h \constzero^{1/2} \vartheta_1(h \constzero^{1/2})\}^T  -1) /h  \eqsp.
\end{equation}
%$\sup_{p \in \rset^d } \nabla_p \gperthmc[k](q,p) \leq C$.
% \begin{equation}
% \label{lem:inverse_1}
% \sup_{p \in \rset^d } \nabla_p \gperthmc[k](q,p) \leq C \eqsp.
% \end{equation}
% Then for all $q, p_1,p_2 \in \rset^d$,
% \begin{equation}
% \label{lem:inverse_1}
% \norm{\gperthmc[k](q,p_1) -\gperthmc[k](q,p_2)} \leq C \norm{p_1 - p_2} \eqsp.
% \end{equation}
For any $q \in \rset^d$, $T\in \nsets$ and $h >0$, define $\phia_{q,T,h}(p)$  for all  $p \in \rset^d$ by
\begin{equation}
  \phia_{q,T,h} (p) = p-(h/T) \gperthmc[T](q,p) \eqsp.
\end{equation}
It is a well known fact (see for example
\cite[Exercise 3.26]{duistermaat:kolk:2004}) that if
\begin{equation}
  \label{eq:inverse_1_2}
  \sup_{(q,p) \in \rset^d \times \rset^d} (h/T)\norm{ \Jac_{p,\gperthmc[T]}(q,p)} < 1 \eqsp,
\end{equation}
then for any $q \in \rset^d$, $\phia_{q,T,h}$ is a
diffeomorphism and  therefore by \eqref{eq:qk}, the same conclusion holds
for $p \mapsto \Phiverletq[h][T](q,p)$. Using \eqref{eq:inverse_1}, if $T \in \nsets$ and $h > 0$  satisfies \eqref{eq:condition-h,T-harris},
then the condition \eqref{eq:inverse_1_2} is verified and as a result \eqref{theo:irred_harris_a}.

Denoting for any $q \in \rset^d$ by $\Phiverletqi[h][T](q,\cdot) : \rset^d \to \rset$ the
continuously differentiable inverse of $p \mapsto
\Phiverletq[h][T](q,p)$ and using a change of variable with $\Phiverletqi[h][T](q,\cdot)$ in \eqref{eq:def_kernel_hmc} concludes the proof of \eqref{eq:def_kernel_hmc_false_density}.

We now show that $\Tker_{h,T}$ satisfies the condition which implies that $\Pkerhmc[h][T]$ is a \Tkernel. We first establish some estimates on the function $(q,p) \mapsto \Phiverletqi[h][T](q,p)$. By
\eqref{eq:inverse_1_2} and \eqref{eq:qk}, for any $q,p,v \in \rset^d$, there exists $\varepsilon \in \ooint{0,1}$ such that $  \normLigne{\Phiverletq[h][T](q,p)-\Phiverletq[h][T](q,v)} \geq (hT) \normLigne{\phi_{q,T,h}(p)-\phi_{q,T,h}(v)} \geq (hT) (1-\varepsilon)\norm{p-v}$ which implies that that there exists $C \geq 0$ satisfying
\begin{equation}
  \label{eq:regularity_phinverse1}
  \begin{aligned}
    \norm{\Phiverletqi[h][T](q,p)-\Phiverletqi[h][T](q,v)} &\leq (1-\varepsilon)^{-1} \norm{v-p}\eqsp, \\
    \norm{  \Phiverletqi[h][T](q,p)} &\leq C\defEns{\norm{\p} + \norm{\Phiverletq[h][T](q,0)}} \eqsp.
  \end{aligned}
\end{equation}
In addition, for $q,x,p \in \rset^d$, we have setting $\tilde{q} = \Phiverletqi[h][T](q,p)$ that
\begin{align}
  \nonumber
  \normLigne{\Phiverletqi[h][T](q,p) - \Phiverletqi[h][T](x,p)} &= \normLigne{\tilde{q} - \Phiverletqi[h][T](x, \Phiverletq[h][T](q,\tilde{q}))} \\
  \nonumber
                                                                &= \normLigne{\Phiverletqi[h][T](x, \Phiverletq[h][T](x,\tilde{q})) - \Phiverletqi[h][T](x, \Phiverletq[h][T](q,\tilde{q}))} \eqsp,
\end{align}
which implies by \eqref{eq:regularity_phinverse1} and \Cref{lem:bound_first_iterate_leapfrog_a}
that there exists $C \geq 0$ satisfying
\begin{equation}
  \label{eq:regularity_phinverse2}
  \norm{\Phiverletqi[h][T](q,p) - \Phiverletqi[h][T](x,p)} \leq C \norm{q-x} \eqsp.
\end{equation}


We now can prove that $\Tker_{h,T}$ is the continuous component of $\Pkerhmc[h][T]$. First by \eqref{eq:def_tker}, for all $\eventB \in \borelSet(\rset^d)$,
\begin{equation}
\label{eq:minoration_pseudo_density_P}
    \Tker_{h,T}(q, \eventB) \geq (2 \uppi)^{-d/2} \Leb(\eventB)
 \times \inf_{\bar{q} \in \eventB} \defEns{ \balphaacc(q,\bar{q}) \rme^{-\norm{\Phiverletqi_q(\bar{q})}^2/2}\detj_{\Phiverletqi[h][T](q,\cdot)}(\bar{q})} \eqsp,
\end{equation}
with the convention $0 \times \plusinfty = 0$ and
\begin{equation}
%  \label{eq:7}
  \balphaacc(q,\bar{q}) =  \alphaacc\defEns{(q,\Phiverletqi[h][T](q,\bar{q})),\Phiverlet[h][T](q,\Phiverletqi[h][T](q,\bar{q}))}\eqsp. 
\end{equation}
Since the function $  (q,p) \mapsto (\Phiverletq[h][T](q,p),\Phiverletqi[h][T](q,p), \detj_{\Phiverletqi[h][T](q,\cdot)}(p)) $
is continuous on $\rset^d\times \rset^d$ by \Cref{lem:bound_first_iterate_leapfrog_a}, \eqref{eq:regularity_phinverse1} and \eqref{eq:regularity_phinverse2}, and for any $q,p \in \rset^d$, $\Jac_{\Phiverletq[h][T](\q,\cdot)}(\Phiverletqi[h][T](q,p))
\Jac_{\Phiverletqi[h][t](q,\cdot)}(\p) = \operatorname{I}_n$, we get that  $\Tker_{h,T}(q,\eventB) >0$ for all $q \in \rset^d$ and all compact set $\eventB$ satisfying $\Leb(\eventB) > 0$. Therefore, using that the Lebesgue measure is regular which implies that for any $\msa \in \mcb(\rset^d)$ with $\Leb(\msa) >0$, there exists a compact set $\msb \subset\msa$, $\Leb(\msb)>0$, we can conclude that $\Pkerhmc[h][T]$ is irreducible with respect to the Lebesgue measure. In addition, we get  $\Tker_{h,T}(q,\rset^d) >0$, and therefore we obtain that $\Pkerhmc[h][T]$ is aperiodic.  Similarly we get that any compact set is $(1,\Leb)$-small.

It remains to show that for any $\eventB \in\mcb(\rset^d)$, $q \mapsto \Tker_{h,T}(q,\eventB)$ is lower semi-continuous which is a straightforward consequence of Fatou's Lemma and that for any $p \in \rset^d$, $q \mapsto (\Phiverlet[h][T](q,p), \Phiverletqi[h][T](q,p),\detj_{\Phiverletqi[h][T](q,\cdot)}(p))$ is continuous.

% We now show that all the compact sets are $(1,\Leb)$-small. Let $\eventB \subset \rset^d$ be compact.  Using
% \eqref{eq:inverse_1_2} there exists $C \geq 0$ such that for all
% $q,p,v \in \rset^d$, $ \norm{p-v} \leq C \normLigne{
%   \Phiverletq[h][T](q,p)- \Phiverletq[h][T](q,v)}$. It follows
% that for all $p \in \rset^d$, $\sup_{q \in \eventB} \normLigne{
%   \Phiverletqi[h][T](q,p)} \leq C\defEnsLigne{\norm{\p} + \sup_{q \in
%     \eventB} \normLigne{\Phiverletq[h][T](q,0)}}$. Using this upper
% bound and $\Jac_{\Phiverletq(\q,\cdot)}(\Phiverletqi_q(\p))
% \Jac_{\Phiverletqi_q}(\tilde{\p}) = \operatorname{I}_n$ in
% \eqref{eq:minoration_pseudo_density_P}, where $\operatorname{I}_n$ is
% the identity matrix, we deduce that there exists $\varepsilon >0$ such
% that for all $\eventA\in \borelSet(\rset^d)$, $\eventA \subset \eventB$,
% \begin{equation}
%   \inf_{q \in \eventB} \Pkerhmc[h][T](q, \eventA)  \geq \varepsilon \Leb(\eventA) \eqsp,
% \end{equation}
% and therefore $\eventB$ is small for $\Pkerhmc[h][T]$.
% \begin{equation}
% \norm{  \Phiverletqi_q(p_1)-  \Phiverletqi_q(p_2)} \leq C \norm{p_1-p_2} \eqsp,
% \end{equation}
% This result, \eqref{eq:def_acc_ratio}, \Cref{lem:bound_first_iterate_leapfrog} and \eqref{eq:minoration_pseudo_density_P} imply that $
% \Pkerhmc[h][T]$ is irreducible with respect to the Lebesgue measure
% and aperiodic.
% and any ball on $\rset^d$ is small.

% A straightforward adaptation of the proof of \cite[Corollary
% 2]{tierney:1994} shows that $ \Pkerhmc[h][T]$ is Harris recurrent, see \Cref{propo:harris_rec} in \Cref{sec:harr-recurr-metr}. The desired conclusion then follows from \cite[Theorem 13.0.1]{meyn:tweedie:2009}.
 % \Cref{theo:irred_harris} implies
% that for all $T \geq 0$, there exists $\hirr>0$ such that for all $h \in \ocintLigne{0,\hirr}$ and all $\q \in \rset^d$
%   \begin{equation}
% \lim_{n \to \plusinfty}    \tvnorm{\delta_\q \Pkerhmc[h][T]^n - \pi} = 0 \eqsp.
%   \end{equation}


% By \cite[Theorem 17.1.4, Theorem
% 17.1.7]{meyn:tweedie:2009}, it suffices the to prove that for all
% bounded harmonic function $\harmonic : \rset^d \to \rset$ satisfying
% \begin{equation}
%   \label{eq:def_harm}
%   \Pkerhmc[h][T]\harmonic = \harmonic \eqsp,
% \end{equation}
% %$\Pkerhmc[h][T]\harmonic = \harmonic$,
% are constant. First since $\Pkerhmc[h][T]$ is irreducible with respect
% to the Lebesgue measure and aperiodic, by \cite[Theorem
% 14.0.1]{meyn:tweedie:2009} for $\Leb$-almost all $q$ we get $\lim_{n
%   \to \plusinfty} \Pkerhmc[h][T]^n \harmonic(q) = \pi(\harmonic)$ and therefore by
% \eqref{eq:def_harm} $\harmonic(q) = \pi(\harmonic)$. Therefore we get that for all $q \in \rset^d$ by \eqref{eq:def_kernel_hmc_false_density},
% \begin{multline}
%    \Pkerhmc[h][T]^n \harmonic(q) = \pi(\harmonic)  \int_{\rset^d}  \alphaacc\defEns{(q,\tilde{p}),\Phiverlet[h][T](q,\tilde{p})} \rme^{-\norm{\tilde{p}}^2/2} \rmd \tilde{p} \\
% +   \harmonic(x) \int_{\rset^d}  \parentheseDeux{1-\alphaacc\defEns{(q,\tilde{p}),\Phiverlet[h][T](q,\tilde{p})}} \rme^{-\norm{\tilde{p}}^2/2} \rmd \tilde{p} \eqsp.
% \end{multline}
% Combining this result with \eqref{eq:def_harm}, we get for all $q \in \rset^d$
% \begin{equation}
% (\harmonic(q)-\pi(\harmonic)) \int_{\rset^d} \alphaacc\defEns{(q,\tilde{p}),\Phiverlet[h][T](q,\tilde{p})} \rme^{-\norm{\tilde{p}}^2/2} \rmd \tilde{p} = 0\eqsp.
% \end{equation}
% It follows from \Cref{lem:bound_first_iterate_leapfrog} and \eqref{eq:def_acc_ratio} that for all $q \in \rset^d$, $\harmonic(q) = \pi(\harmonic)$
% which concludes the proof.
% =======
% The proof of \ref{theo:irred_harris_b} using a change of variable with $\Phiverletqi[h][T](q,\cdot)$.

% We now show that $\Tker_{h,T}$ satisfies the condition which implies that $\Pkerhmc[h][T]$ is a \Tkernel.
% First, for all $\eventB \in \borelSet(\rset^d)$,
% \begin{equation}
% \label{eq:minoration_pseudo_density_P}
% \Tker_{h,T}(q, \eventB) \geq (2 \uppi)^{-d/2} \Leb(\eventB)
%  \times \inf_{\bar{q} \in \eventB} \defEns{\alphaacc\defEns{(q,\Phiverletqi[h][T](q,\bar{q})),\Phiverlet[h][T](q,\Phiverletqi[h][T](q,\bar{q}))} \rme^{-\norm{\Phiverletqi[h][T](q,\bar{q})}^2/2} \detj_{\Phiverletqi[h][T]}(q,\bar{q})} \eqsp,
% \end{equation}
% with the convention $0 \times \plusinfty = 0$. Since $\Phiverletqi[h][T](q,\cdot)$
% is a diffeomorphism on $\rset^d$, we get that  $
% \Tker_{h,T}(q,\eventB) >0$ for all $q \in \rset^d$ and all compact set $\eventB$ satisfying $\Leb(\eventB) > 0$. Since the Lebesgue measure is regular, this implies that $\Pkerhmc[h][T]$ is irreducible with respect to the Lebesgue measure and aperiodic.

% By Fatou's Lemma, for any $\eventB \in\mcb(\rset^d)$, $q \mapsto \Tker_{h,T}(q,\eventB)$ is lower semi-continuous.
% We now show that all the compact sets are small. Let $\eventB \subset \rset^d$ be compact.  Using
% \eqref{eq:inverse_1_2} there exists $C \geq 0$ such that for all
% $q,p,v \in \rset^d$, $ \norm{p-v} \leq C \normLigne{
%   \Phiverletq[h][T](q,p)- \Phiverletq[h][T](q,v)}$. It follows
% that for all $p \in \rset^d$, $\sup_{q \in \eventB} \normLigne{
%   \Phiverletqi[h][T](q,\p)} \leq C\defEnsLigne{\norm{\p} + \sup_{q \in
%     \eventB} \normLigne{\Phiverletq[h][T](q,0)}}$. Using this upper
% bound and $\Jac_{\Phiverletq(\q,\cdot)}(\Phiverletqi[h][T](q,\p))
% \Jac_{\Phiverletqi[h][T]}(q,\tilde{\p}) = \operatorname{I}_n$ in
% \eqref{eq:minoration_pseudo_density_P}, where $\operatorname{I}_n$ is
% the identity matrix, we deduce that there exists $\varepsilon >0$ such
% that for all $\eventA\in \borelSet(\rset^d)$, $\eventA \subset \eventB$,
% \begin{equation}
%   \inf_{q \in \eventB} \Pkerhmc[h][T](q, \eventA)  \geq \varepsilon \Leb(\eventA) \eqsp,
% \end{equation}
% and therefore $\eventB$ is small for $\Pkerhmc[h][T]$.
% >>>>>>> f8207bad5c0353bdfe37210ffc64a715e92e53ed

Finally, the last statements of \ref{theo:irred_harris_c} follows from \Cref{propo:harris_rec} in \Cref{sec:harr-recurr-metr} which implies that  $ \Pkerhmc[h][T]$ is Harris recurrent and  \cite[Theorem 13.0.1]{meyn:tweedie:2009} which implies  \eqref{eq:harris-theorem}.

\subsubsection{Proof of \Cref{theo:irred_D}}
\label{sec:proof-crefth_irred_D}
We use \Cref{coro:irred}. Indeed $\Pkerhmc[h][T]$ is
of form \eqref{eq:def_pkerb} and it is straightforward to check that it
satisfies \Cref{assumG:phi} (note that \Cref{lem:bound_first_iterate_leapfrog_a}
shows that $\Phiverlet[h][T]$ is a Lipshitz function on $\rset^{2d}$).

We now check that $\Pkerhmc[h][T]$ satisfies \Cref{assumG:irred_b}($\rassG,0,\MassG$) for all $\rassG,\MassG \in
\rset_+^*$ using \Cref{le:degree_application}.  By \eqref{eq:qk}, for all $T \in \nsets$, $h >0$, $q,p \in \rset^d$,
\begin{equation}
  \label{eq:phiverlet_gqth}
  \Phiverletq[h][T](q,p) = T
h p + g_{q,T,h}(p)
\end{equation}
where $g_{q,T,h}(p) = q - (Th^2/2) \nabla \F(q) -
h^2 \gperthmc[T](q,p)$ where $\gperthmc[T]$ is defined by \eqref{eq:def_gperthmc}. \Cref{lem:inverse_1} shows that for any $T \in \nsets$ and $h >0$, it holds that
\begin{equation}
    \label{eq:2:theo:irred_D}
\sup_{p,v,q \in  \rset^d} \frac{\norm{g_{q,T,h}(p)-g_{q,T,h}(v)}}{\norm{p - v}} \leq T h [\{1 + h \constzero^{1/2} \vartheta_1(h \constzero^{1/2} )\}^T-1] \eqsp,
\end{equation}
which implies that the condition
\Cref{le:degree_application}-\ref{propo:irred_b_item_i} is satisfied. To check that
condition  \Cref{le:degree_application}-\ref{propo:irred_b_item_ii} holds, we consider separately the two cases: $\beta <1$ and $\beta =1$.

\begin{enumerate}[label=$\bullet$, wide, labelwidth=!, labelindent=0pt]
\item Consider first the case $\beta <1$. By \Cref{assum:regOne}-\ref{assum:regOne_b},
for any $T \in \nsets$ and $h >0$, we get
\begin{equation}
\norm{\gperthmc[T](\q,\p)} \leq  T \sum_{i=1}^{T-1} \norm{\nabla \F \circ \Phiverletq[h][i](\q,\p)} \leq
\constzeroT T \sum_{i=1}^{T-1} \defEns{ 1 + \norm{\Phiverletq[h][i](\q,\p)}^{\expozero}}
 \eqsp.
\end{equation}
Hence, by \Cref{lem:bound_first_iterate_leapfrog_b}-\ref{lem:bound_first_iterate_leapfrog_1}
there exists $C \geq 0$ such that for all $R\in \rset_+^*$ and
$q,p \in \rset^d$, $\norm{q} \leq R$,
\begin{equation}
\label{eq:3:theo:irred_D}
\norm{g_{q,T,h}(p)} \leq C \defEns{1+R^{\beta} +\norm{p}^{\expozero}} \eqsp,
\end{equation}
which implies that condition \ref{propo:irred_b_item_ii} of \Cref{le:degree_application} holds for any $T \in \nsets$ and $h >0$.

\item Consider now the case $\beta =1$.  For any $T \in \nsets$, $h >0$,  $q,p \in \rset^d$ we get using \Cref{assum:regOne}-\ref{assum:regOne_a}
\begin{align}
  \norm{g_{q,T,h}(p)} &\leq \norm{q} + Th^2 \constzero  \norm{q} /2 + Th^2 \norm{\nabla U(0)} /2\\
  & \qquad \qquad +h^2 \norm{\gperthmc[T](q,p) - \gperthmc[T](q,0)} + h^2 \norm{ \gperthmc[T](q,0)} \eqsp.
\end{align}
Therefore using \Cref{lem:inverse_1}, for any $q,p \in \rset^d$, $\norm{q} \leq R$ for $R \geq 0$, for any $T \in \nsets$ and $h >0$ satisfying \eqref{eq:condition-h,T-harris}, there exists $C \geq 0$ such that
\begin{equation}
  \norm{g_{q,T,h}(p)} \leq C + h T  [ \{1+ h \constzero^{1/2} \vartheta_1(h\constzero^{1/2})\}^T-1]  \norm{p} \eqsp,
\end{equation}
showing that condition \ref{propo:irred_b_item_ii} of \Cref{le:degree_application} is satisfied.
\end{enumerate}

Therefore,  \Cref{le:degree_application} can be applied and for any $T \in \nsets$ and $h >0$ if $\beta <1$ and for any $h > 0$ and $T \in \nsets$ satisfying \eqref{eq:condition-h,T-harris} if $\beta =1$, $\Pkerhmc[h][T]$ satisfies \Cref{assumG:irred_b}($\rassG,0,\MassG$) for all $\rassG,\MassG \in
\rset_+^*$.  \Cref{coro:irred} concludes the proof of \ref{theo:irred_D_a} and \ref{theo:irred_D_b}.
The last statement then follows from   \cite[Theorem 14.0.1]{meyn:tweedie:2009}.

% Using this result and \Cref{theo:irred}, we get that for all $\rassG,\MassG  \in \rset_+^*$ there exists $\varepsilon >0$ such that
% for all $\q \in \ball{0}{\rassG}$ and $\eventA \in \borelSet(\rset^d)$,
% \begin{equation}
%   \Pkerhmc[h][T](q, \eventA) \geq \varepsilon \Leb(\eventA \cap \ball{0}{M}) \eqsp.
% \end{equation}
% \Cref{coro:irred} Combining this result and \eqref{eq:1:theo:irred_D} concludes the proof of \ref{theo:irred_D_a} and \ref{theo:irred_D_b}.

% The proof is a consequence of \Cref{lem:bound_first_iterate_leapfrog},
% \Cref{le:degree_application} and \Cref{theo:irred}.  \alain{give some
%   details}

%%% Local Variables:
%%% mode: latex
%%% TeX-master: "main"
%%% End:
