\documentclass[todo]{myarticle}
\raggedbottom

\addbibresource{continuum_outline/bibliography.bib}

\DeclareMathOperator{\sgn}{sgn}


\usepackage{authblk}

\title{Gauge Symmetry Breaking Lattice Regularizations\\and their Continuum Limit}

\author[1]{Thorsten Lang\thanks{\texttt{thorsten.lang@fau.de}}}
\affil[1]{Institute for Quantum Gravity, FAU Erlangen--Nürnberg, Staudtstraße 7/B2, 91058 Erlangen, Germany}
\author[2]{Susanne Schander\thanks{\texttt{sschander@perimeterinstitute.ca}}}
\affil[2]{Perimeter Institute, 31 Caroline St N, Waterloo, ON N2L 2Y5, Canada}
\date{\today}

\begin{document}
\maketitle


\begin{abstract}
    Lattice regularizations are pivotal in the non--perturbative quantization of gauge field theories.
    Wilson's proposal to employ group-valued link fields simplifies the regularization of gauge fields in principal fiber bundles, preserving gauge symmetry within the discretized lattice theory.
    Maintaining gauge symmetry is desirable as its violation can introduce unwanted degrees of freedom.
    However, not all theories with gauge symmetries admit gauge--invariant lattice regularizations, as observed in general relativity where the diffeomorphism group serves as the gauge symmetry.
    In such cases, gauge symmetry--breaking regularizations become necessary.
    In this paper, we argue that a broken lattice gauge symmetry is acceptable as long as gauge symmetry is restored in the continuum limit.
    We propose a method to construct the continuum limit for a class of lattice--regularized Hamiltonian field theories, where the regularization breaks the Lie algebra of first--class constraints.
    Additionally, we offer an approach to represent the exact gauge group on the Hilbert space of the continuum theory. 
    The considered class of theories is limited to those with first--class constraints linear in momenta, excluding the entire gauge group of general relativity but encompassing its subgroup of spatial diffeomorphisms.
    We discuss potential techniques for extending this quantization to the full gauge group.
\end{abstract}
\tableofcontents*

\section{Introduction}  \label{sec:introduction}

\newcommand\inexpIntro[3]{#1?(#2,#3).}
\newcommand\rinexpIntro[3]{*#1?(#2,#3).}
\newcommand\outexpIntro[3]{#1!(#2,#3).}
\newcommand\outatomIntro[3]{#1!(#2,#3)}

We propose a fully automated method for proving termination of \(\pi\)-calculus processes.
Although there have been a lot of studies on termination analysis for the \(\pi\)-calculus
and related calculi~\cite{Deng06IC,Demangeon07,SangiorgiTermination,KobayashiHybrid,Yoshida04IC,DBLP:journals/jlp/DemangeonHS10,Venet98SAS}, most of them have been rather theoretical,
and there have been surprisingly little efforts in developing  fully automated termination
verification methods and tools based on them. To our knowledge,
Kobayashi's \typical{}~\cite{TyPiCal,KobayashiHybrid} is the only exception that
can prove termination of \(\pi\)-calculus processes (extended with natural numbers)
fully automatically, but its termination analysis is quite limited (see Section~\ref{sec:relatedwork}).

Our method is based on a reduction to termination analysis for sequential programs:
we translate a \(\pi\)-calculus process \(P\) to a sequential program \(S_P\), so that
if \(S_P\) is terminating, so is \(P\). The reduction allows us to use
powerful, mature methods and tools
for termination analysis of sequential programs~\cite{heizmann2016ultimate,freqterm,DBLP:conf/lics/PodelskiR04,Kuwahara2014Termination,DBLP:journals/cacm/CookPR11}.

The idea of the translation is to convert a chain of communications on replicated input
channels to a chain of recursive function calls of the target sequential program.
Let us consider the following Fibonacci process:
\begin{align*}
    & \rinexpIntro{\fib}{n}{r}
        \ifexp{n<2}{ \soutatom{r}{1} \\ &\quad}
                   { \nuexp{s_1} \nuexp{s_2} (\outatomIntro{\fib}{n-1}{s_1} \PAR \outatomIntro{\fib}{n-2}{s_2} \PAR \sinexp{s_1}{x}\sinexp{s_2}{y}\soutatom{r}{x+y}) \\}
    & \PAR \outatomIntro{\fib}{m}{r}
\end{align*}
Here, the process
$\rinexpIntro{\fib}{n}{r} \ldots$ is a function server that computes the \(n\)-th Fibonacci number
in parallel and returns the result to \(r\),
and $\outatom{\fib}{m}{r}$ sends a request for computing the \(m\)-th Fibonacci number;
those who are not familiar with the syntax of the \(\pi\)-calculus may wish to consult
Section~\ref{sec:targetlanguage} first.
To prove that the process above is terminating for any integer \(m\),
it suffices to show that there is no infinite chain of communications on $\fib$:
\[
    \fib(m,r) \to \fib(m_1,r_1) \to \fib(m_2,r_2) \to \cdots.
\]
We convert the process above to the following program:\footnote{The actual translation
  given later is a little more complex.}
\begin{verbatim}
 let rec fib(n) = if n<2 then () else (fib(n-1) [] fib(n-2)) in
 fib(m)
\end{verbatim}
Here, \texttt{[]} represents the non-deterministic choice.
Note that, although the calculation of Fibonacci numbers is not preserved,
for each chain of communications on \texttt{fib}, there is a corresponding
sequence of recursive calls:
\[
\mathtt{fib}(m) \to \mathtt{fib}(m_1) \to \mathtt{fib}(m_2) \to \cdots.
\]
Thus, the termination of the sequential program above implies the termination of
the original process.
As shown in the example above, (i) each communication on a replicated input channel
is converted to a function call, (ii) each communication on a non-replicated input
channel is just removed (or, in the actual translation, replaced by a call of
a trivial function defined by \(f(\seq{x})=(\,)\)), and (iii) parallel composition
is replaced by a non-deterministic choice.
We formalize the translation outlined above and prove its correctness.

The basic translation sketched above sometimes loses too much information.
For example, consider the following process:
\begin{align*}
    & \rinexpIntro{\pre}{n}{r} \soutatom{r}{n-1} \\
    & \PAR \rinexpIntro{f}{n}{r} \ifexp{n<0}{ \soutatom{r}{1} }
                                       { \nuexp{s} (\outatomIntro{\pre}{n}{s} \PAR \sinexp{s}{x}\outatomIntro{f}{x}{r}) } \\
    & \PAR \outatomIntro{f}{m}{r}
\end{align*}
The translation sketched above would yield:
\begin{verbatim}
  let pred(n) = n-1 in
  let rec f(n) = if n<0 then () else (pred(n) [] f(*)) in
  f(m)
\end{verbatim}
Here, \texttt{*} represents a non-deterministic integer: since we have removed
the input $\sinatom{s}{x}$, we do not have information about the value of \( x \).
As a result, the sequential program above is non-terminating, although the original
process is terminating.
To remedy this problem, we also refine the basic translation above by using a refinement
type system for the \(\pi\)-calculus. Using the refinement type system,
we can infer that the value of \(x\) in the original process is less than \(n\),
so that we can refine the definition of \texttt{f} to:
\begin{verbatim}
 let rec f(n) = ... else (pred(n) [] let x=* in assume(x<n);f(x))
\end{verbatim}
The target program is now terminating, from which
we can deduce that the original process is also terminating.
We have implemented an automated tool based on the refined translation above.

The contributions of this paper are summarized as follows.
\begin{itemize}
\item The formalization of the basic translation from the \(\pi\)-calculus
  (extended with integers) to sequential programs, and a proof of its correctness.
\item The formalization of a refined translation based on a refinement type system.
\item An implementation of the refined translation, including automated refinement type
  inference based on CHC solving, and experiments to evaluate the effectiveness of
  our method.
\end{itemize}

The rest of this paper is structured as follows.
Section~\ref{sec:targetlanguage} introduces the source and target languages
of our translation.
Section~\ref{sec:approach} 
formalizes the basic translation, and proves its correctness.
Section~\ref{sec:refinement} refines the basic translation by using a refinement type system.
Section~\ref{sec:implementation} reports an implementation and experiments.
Section~\ref{sec:relatedwork} discusses related work,
and Section~\ref{sec:conclusion} concludes the paper.

\vspace{-0.1in}
\section{Neural Program Synthesis from Input-Output Examples}
\vspace{-0.1in}
In programming by example tasks, the program specification is a set of input-output examples~\cite{devlin2017robustfill,bunel2018leveraging}. Specifically, we provide the synthesizer with a set of $K$ input-output pairs $\{(I^{(k)}, O^{(k)})\}_{k=1}^K$ ($\{IO\}^K$ in short). These input-output pairs are annotated with a ground truth program $P^\star$, so that $P^\star(I^{(k)})=O^{(k)}$ for any $k \in \{1, 2, ..., K\}$. To measure the program correctness, we include another set of held-out test cases $\{IO\}_{test}^{K_{test}}$ that differs from $\{IO\}^K$. The goal of the program synthesizer is to predict a program $P$ from $\{IO\}^K$, so that $P(I)=P^\star(I)=O$ for any $(I, O) \in \{IO\}^K + \{IO\}_{test}^{K_{test}}$.

%\label{sec:c-data}
\textbf{C Program Synthesis}. In this work, we make the first attempt of synthesizing C code in a restricted domain from input-output examples only, and we focus on programs for list processing. List processing tasks have been studied in some prior works on input-output program synthesis, but they synthesize programs in restricted domain-specific languages instead of full-fledged popular programming languages~\cite{balog2016deepcoder,odena2020learning,odena2020bustle}. 

Our C code synthesis problem brings new challenges for programming by example. Compared to domain-specific languages, the syntax and semantics of C are much more complicated, which significantly enlarges the program search space. Meanwhile, learning good representations for partially decoded programs also becomes more difficult. In particular, prior neural program synthesizers that utilize per-line interpreters for the programming language to guide the synthesis and representation learning~\cite{chen2018execution,shin2018improving,nye2020representing,Ellis2019WriteEAExtendExecution,odena2020bustle} are not directly applicable to C. Although it is possible to dump some intermediate variable states during C code execution~\cite{campbell2012executable}, since partial C programs are not executable, we are able to obtain all the execution states only until a full C code is generated, which is too late to include them in the program decoding process. In particular, the intermediate execution state is not available when the partial program is syntactically invalid, and this happens more frequently for C due to its syntax design.
\begin{figure}
    \centering
    \includegraphics[width=\textwidth]{fig/c-program-synthesis-crop.pdf}
\caption{\small Illustration of the C program synthesis pipeline. For dataset construction, we develop a random program generator to sample random C programs, then execute the program over randomly generated inputs and obtain the outputs. The input-output pairs are fed into the neural program synthesizer to predict the programs. Note that the synthesized program can be more concise than the original random program.}
\label{fig:ex-c}
\end{figure}


\section{The Continuum Limit}
\label{sec:The Continuum Limit}
Now that we have obtained quantized versions of our lattice theories, the logical next step is to study their continuum limit.
In the following, we will outline a method to define a continuum version of these theories.

First, we note that on every lattice Hilbert space $\mathcal H_n$, there exists a $C*$--algebra $W_n$ of Weyl elements, which is given by
\begin{equation}
    W_n = \overline{\mathrm{span}\Set{\e^{\im\hat\phi_n[f_n] + \im\hat\pi_n[g_n]} | (f_{nk})_k, (g_{nk})_k \in \mathbb R^{N_n}}} .
\end{equation}
We display the continuum Weyl algebra as the projective limit $W = \varprojlim W_n$, where the identifications
\begin{equation}
    \hat\phi_{n+1,2k}f_{n+1,2k} + \hat\phi_{n+1,2k+1}f_{n+1,2k+1} \equiv \hat\phi_{nk}(f_{n+1,2k} + f_{n+1,2k+1}),
\end{equation}
and similarly for $\hat\pi_{nk}$, are made.
We note that the Schrödinger representations on the lattice are irreducible representations of $W_n$, so every vector $\psi_n\in\mathcal H_n$ is a cyclic vector with respect to $W_n$.

In order to find a continuum limit, we need to find a sequence $(\psi_n)_n$ of normalized vectors $\psi_n\in\mathcal H_n$ such that the algebraic states
\begin{equation}
    \omega_n\left(\e^{\im\hat\phi_n[f_n] + \im\hat\pi_n[g_n]}\right) \coloneq \Braket{\psi_n, \e^{\im\hat\phi_n[f_n] + \im\hat\pi_n[g_n]} \psi_n} \label{eq:algebraic-states}
\end{equation}
converge to a state $\omega$ on $W$.
In order to check that, we need to show that the $\omega_n$ are a Cauchy sequence in the following sense:
Let $f$ be a measurable function on $\mathbb T$ and $(f_n)_n$ be a sequence of piecewise constant functions on the $n$--th lattice that converges to $f$.
Analogously, choose a sequence $g_n\to g$.
We say that the $\omega_n$ form a Cauchy sequence if for every $\epsilon>0$, there is $N>0$ such that for all $f_n\to f$, $g_n\to g$ and all numbers $n,m>N$, we have
\begin{equation}
    \abs*{\omega_n\left(\e^{\im\hat\phi_n[f_n] + \im\hat\pi_n[g_n]}\right) - \omega_m\left(\e^{\im\hat\phi_m[f_m] + \im\hat\pi_m[g_m]}\right)} < \epsilon .
\end{equation}
In that case, we define
\begin{equation}
    \omega\left(\lim_{n\to\infty} \e^{\im\hat\phi_n[f_n] + \im\hat\pi_n[g_n]}\right) \coloneq \lim_{n\to\infty} \omega_n\left(\e^{\im\hat\phi_n[f_n] + \im\hat\pi_n[g_n]}\right) .
\end{equation}
The continuum Hilbert space $\mathcal H$ and a cyclic representation of the continuum Weyl algebra $W$ with cyclic vector $\psi$ can then be reconstructed from $\omega$ using the GNS--construction.

We note that this way of obtaining a continuum theory is much simpler than the application of a renormalization group method \parencite[e.g.][]{Lang:2017beo}.
While a renormalization group flow does produce a converging sequence $(\omega_n)_n$ at the fixed point of the flow, the resulting sequence is a very complex object that is hard to obtain.
The fixed point sequence does not only converge, but it satisfies the additional property of cylindrical consistency, which means that the expectation value of a coarse Weyl element is the same on both a coarse lattice and a fine lattice.
This additional property is much stronger than mere convergence, but not strictly required for the definition of a theory in the continuum.
On the other hand, generating a convergent sequence by means of a renormalization group flow has the advantage of being quite systematic.

\subsection{Defining Convergent Sequences}
It can be difficult to choose a sequence $(\psi_n)_n$ of states that has a continuum limit.
In general, one may need to make use of the freedom to be able to choose the lattice approximations $f_n$ of the continuum test functions $f$.
Different choices may lead to different limits or may develop divergences.

We want to illustrate one particularly simple approach to obtaining such a sequence.
Let us choose an arbitrary state $\psi_{m} \in \mathcal H_{m}$.
We iteratively define a sequence of fine states $(\psi_n)_{n\geq m}$ from $\psi_{m}$ as follows:
\begin{equation}
    \psi_{n+1}((\phi_{n+1,k})_k) \coloneq 2^{-2^{n-1}} \psi_n((\phi_{nk}^+)_k) \prod_{k=1}^{N_n} \rho_{n}(\phi_{nk}^-),
\end{equation}
where
\begin{equation}
    \phi_{nk}^+ = \frac{\phi_{n+1,2k} + \phi_{n+1,2k-1}}{2}, \quad \phi_{nk}^- = \frac{\phi_{n+1,2k} - \phi_{n+1,2k-1}}{2}. \label{eq:change-of-variables}
\end{equation}
Analogously, we define $\pi_{nk}^\pm$.
The functions $\rho_n$ are arbitrary, yet to be chosen functions of one variable.
They quantify the distribution of the local fluctuations on each scale.

By expanding the canonical variables in $\hat\phi_{nk}^\pm$ and $\hat\pi_{nk}^\pm$, where $\commutator{\hat\phi_{nk}^\pm}{\hat\pi_{nk'}^\mp} = 0$, we can evaluate the algebraic states $\omega_n$ (as defined in \cref{eq:algebraic-states}) on the corresponding Weyl elements.
Using the shorthand $f_{nk}^{n+1} = f_{n+1,2k-1}+f_{n+1,2k}$, we find:
\begin{align}
    \MoveEqLeft \phi_{n+1}[f_{n+1}]
         = \sum_{k=1}^{N_{n+1}} \phi_{n+1,k}f_{n+1,k} \nonumber \\
        &= \sum_{k=1}^{N_n}\phi_{nk}^+(f_{n+1,2k-1}+f_{n+1,2k}) + \sum_{k=1}^{N_n} \phi_{nk}^-(f_{n+1,2k}-f_{n+1,2k-1}) \nonumber \\
        &= \sum_{k=1}^{N_n}\phi_{nk}^+ f_{nk}^{n+1} + \eta_{n+1} \sum_{k=1}^{N_n} \phi_{nk}^- (\Delta f)^{n+1}_{n k} \nonumber \\
        &= \phi_n^+[f_n^{n+1}] + \eta_{n+1} \,\phi_{n}^- \left[ (\Delta f)^{n+1}_{n} \right],
\end{align}
where we defined
\begin{align}
    (\Delta f)^{n+1}_{n k} \coloneq \frac{1}{\eta_{n+1}} \left( f_{n+1, 2k} - f_{n+1,2k-1}\right).
\end{align}
A similar expression follows for $\pi_{n+1}[f_{n+1}]$.
Consequently, we find the following relation for the Weyl elements:
\begin{align}
    \e^{\im\hat\phi_{n+1}[f_{n+1}] + \im\hat\pi_{n+1}[g_{n+1}]}
        = \e^{\im\hat\phi_n^+[f_n^{n+1}] + \im\hat\pi_n^+[g_n^{n+1}]} \prod_{k=1}^{N_n} \e^{\im\eta_{n+1}\left(\hat\phi_{n}^- [(\Delta f)^{n+1}_{n }] + \hat\pi_{n}^- [(\Delta g)^{n+1}_{n }]\right)},
\end{align}
We note that all factors on the right hand side commute.
This allows us to evaluate the $\omega_{n+1}$ on the Weyl elements by a simple change of variables according to \cref{eq:change-of-variables}:
\begin{align}
    \omega_{n+1}(\e^{\im\hat\phi_{n+1}[f_{n+1}] + \im\hat\pi_{n+1}[g_{n+1}]})
        = \omega_n(\e^{\im\hat\phi_n[f_n^{n+1}] + \im\hat\pi_n[g_n^{n+1}]}) \prod_{k=1}^{N_n} \zeta_{n}((\Delta f)^{n+1}_{nk}, (\Delta g)^{n+1}_{nk}) .
\end{align}
Here, we defined
\begin{equation} \label{def:zeta_n}
    \zeta_{n}((\Delta f)^{n+1}_{nk}, (\Delta g)^{n+1}_{nk}) \coloneq \Braket{\rho_n, \e^{\im \eta_{n+1} (\hat\phi_{nk}^- (\Delta f)^{n+1}_{nk} + \hat\pi_{nk}^-(\Delta g)^{n+1}_{nk})} \rho_n}.
\end{equation}
It follows that the corresponding states $\omega_n$ for any $n>m$, evaluated on elements of the Weyl algebra, read
\begin{equation}
    \omega_n\left(\e^{\im\hat\phi_n[f_n] + \im\hat\pi_n[g_n]}\right) = \omega_{m}\left(\e^{\im\hat\phi_{m}[f^n_{m}] + \im\hat\pi_{m}[g^n_{m}]}\right) \prod_{l=m}^{n-1} \prod_{k=1}^{N_l} \zeta_l ((\Delta f)^n_{m k}, (\Delta g)^n_{m k}),
\end{equation}
where we use that the components of the vectors $f^n_{m}$ and $(\Delta f)^n_m$ are respectively given by
\begin{align}
    f^n_{m k} &= \sum_{j=1}^{N_n N_{m}^{-1} } f_{n, (k-1) N_n N_{m}^{-1} +j}, \\
    (\Delta f)^n_{m k} &= \sum_{j=1}^{N_n N_{m}^{-1}} f_{n, (k-1) N_n N_{m}^{-1} +j} \cdot \sgn (2j -1- N_n N_{m}^{-1}),
\end{align}
where $k =1,\dots, N_{m}$. Similar expressions hold for $g^n_{m}$ and $(\Delta g)^n_m$.
If the $\rho_l$ are chosen appropriately, the product of the $\zeta_l$ converges and a continuum limit exists.

Possible examples that lead to convergence are given by Dirac sequences with suitable speed of convergence.
A Dirac sequence of Gaussian distributions will lead to a Gaussian state in the continuum and thus to a Fock space representation.
Dirac sequences with compact support can be used to constrain the admissible states to those with support on configurations of bounded variation.
Such states are interesting, because functions of bounded variation are differentiable almost everywhere, which can aid the convergence of finite differences $\Delta^{\eta_n} \hat\phi_{nk}$ to derivatives $\partial\hat\phi(x)$ of field operators.

\subsection{Representation of the Group of Gauge Transformations}
Now that we have a continuum Hilbert space, we would like to implement gauge transformations on it.
Unfortunately, the operators $U_{D_n[f_n]}(s) W_n U_{D_n[f_n]}(s)^\dagger$ are not necessarily Weyl elements again.
Thus, implementing the gauge transformations on the continuum Hilbert space is not as easy as taking the limit of the transformed Weyl elements in the projective limit algebra $W$.
Nevertheless, we can describe a plausible procedure:

Suppose we want to apply a gauge transformation $U_{D[f]}(s)$ on a state $\psi \in \mathcal H$ in the continuum Hilbert space.
We have to choose a sequence $(\psi_n)_n$ in the lattice Hilbert spaces $\mathcal H_n$ that converges to $\psi$ in the sense of algebraic states.
Moreover, we choose a sequence $f_n$ of piecewise constant functions with $f_n\to f$.
We can then apply the approximate gauge transformations $U_{D_n[f_n]}(s)$ to $\psi_n$.
Since every state in $\mathcal H_n$ is cyclic for $W_n$, we can expand the transformed state in every lattice Hilbert space again in terms of Weyl elements:
\begin{equation}
    U_{D_n[f_n]}(s)\psi_n = \sum_k c_{nk} \e^{\im\hat\phi_n[f_{nk}] + \im\hat\pi_n[g_{nk}]} \psi_n
\end{equation}
We can then use this sequence of Weyl elements to define the continuum gauge transformation as follows:
\begin{equation}
    U_{D[f]}(s)\psi = \lim_{n\to\infty}\sum_k c_{nk} \e^{\im\hat\phi_n[f_{nk}] + \im\hat\pi_n[g_{nk}]} \psi.
\end{equation}


\begin{comment}
\begin{figure}
\includegraphics[width=\linewidth]{figs/beyond_tss_lesion.pdf}
\caption[]{End-to-End runtime lesion study of the entire MNIST dataset and the FMA featurized music dataset. Each of DROP's contributions provides a runtime improvement.}
\label{fig:beyond_lesion}
\end{figure}
\end{comment}



\section{Conclusion}
\label{sec:conclusion}

Advanced data analytics techniques must scale to rising data volumes. 
DR techniques offer a powerful toolkit when processing these datasets, with PCA frequently outperforming popular techniques in exchange for high computational cost. 
In response, we propose DROP, a new dimensionality reduction optimizer. 
DROP combines progressive sampling, progress estimation, and online aggregation to identify high quality low dimensional bases via PCA without processing the entire dataset by balancing the runtime of downstream tasks and achieved dimensionality. 
Thus, DROP provides a first step in bridging the gap between quality and efficiency in end-to-end DR for downstream \red{analytics}. 

%We revisit canonical operators for time series dimensionality reduction and the measurement study of~\cite{keogh-study}, and show that PCA is more effective than popular alternatives in the data mining literature often by a margin of over $2\times$ on average on gold-standard time series benchmark data sets with respect to output data dimension. More surprisingly, we empirically demonstrate that a small number of samples are sufficient to accurately characterize directions of maximum variance and obtain a high-quality low-dimensional transformation.




\printbibliography

\end{document}
 
