\documentclass[todo]{myarticle}
\raggedbottom

\addbibresource{continuum_outline/bibliography.bib}

\DeclareMathOperator{\sgn}{sgn}


\usepackage{authblk}

\title{Gauge Symmetry Breaking Lattice Regularizations\\and their Continuum Limit}

\author[1]{Thorsten Lang\thanks{\texttt{thorsten.lang@fau.de}}}
\affil[1]{Institute for Quantum Gravity, FAU Erlangen--Nürnberg, Staudtstraße 7/B2, 91058 Erlangen, Germany}
\author[2]{Susanne Schander\thanks{\texttt{sschander@perimeterinstitute.ca}}}
\affil[2]{Perimeter Institute, 31 Caroline St N, Waterloo, ON N2L 2Y5, Canada}
\date{\today}

\begin{document}
\maketitle


\begin{abstract}
    Lattice regularizations are pivotal in the non--perturbative quantization of gauge field theories.
    Wilson's proposal to employ group-valued link fields simplifies the regularization of gauge fields in principal fiber bundles, preserving gauge symmetry within the discretized lattice theory.
    Maintaining gauge symmetry is desirable as its violation can introduce unwanted degrees of freedom.
    However, not all theories with gauge symmetries admit gauge--invariant lattice regularizations, as observed in general relativity where the diffeomorphism group serves as the gauge symmetry.
    In such cases, gauge symmetry--breaking regularizations become necessary.
    In this paper, we argue that a broken lattice gauge symmetry is acceptable as long as gauge symmetry is restored in the continuum limit.
    We propose a method to construct the continuum limit for a class of lattice--regularized Hamiltonian field theories, where the regularization breaks the Lie algebra of first--class constraints.
    Additionally, we offer an approach to represent the exact gauge group on the Hilbert space of the continuum theory. 
    The considered class of theories is limited to those with first--class constraints linear in momenta, excluding the entire gauge group of general relativity but encompassing its subgroup of spatial diffeomorphisms.
    We discuss potential techniques for extending this quantization to the full gauge group.
\end{abstract}
\tableofcontents*

% \leavevmode
% \\
% \\
% \\
% \\
% \\
\section{Introduction}
\label{introduction}

AutoML is the process by which machine learning models are built automatically for a new dataset. Given a dataset, AutoML systems perform a search over valid data transformations and learners, along with hyper-parameter optimization for each learner~\cite{VolcanoML}. Choosing the transformations and learners over which to search is our focus.
A significant number of systems mine from prior runs of pipelines over a set of datasets to choose transformers and learners that are effective with different types of datasets (e.g. \cite{NEURIPS2018_b59a51a3}, \cite{10.14778/3415478.3415542}, \cite{autosklearn}). Thus, they build a database by actually running different pipelines with a diverse set of datasets to estimate the accuracy of potential pipelines. Hence, they can be used to effectively reduce the search space. A new dataset, based on a set of features (meta-features) is then matched to this database to find the most plausible candidates for both learner selection and hyper-parameter tuning. This process of choosing starting points in the search space is called meta-learning for the cold start problem.  

Other meta-learning approaches include mining existing data science code and their associated datasets to learn from human expertise. The AL~\cite{al} system mined existing Kaggle notebooks using dynamic analysis, i.e., actually running the scripts, and showed that such a system has promise.  However, this meta-learning approach does not scale because it is onerous to execute a large number of pipeline scripts on datasets, preprocessing datasets is never trivial, and older scripts cease to run at all as software evolves. It is not surprising that AL therefore performed dynamic analysis on just nine datasets.

Our system, {\sysname}, provides a scalable meta-learning approach to leverage human expertise, using static analysis to mine pipelines from large repositories of scripts. Static analysis has the advantage of scaling to thousands or millions of scripts \cite{graph4code} easily, but lacks the performance data gathered by dynamic analysis. The {\sysname} meta-learning approach guides the learning process by a scalable dataset similarity search, based on dataset embeddings, to find the most similar datasets and the semantics of ML pipelines applied on them.  Many existing systems, such as Auto-Sklearn \cite{autosklearn} and AL \cite{al}, compute a set of meta-features for each dataset. We developed a deep neural network model to generate embeddings at the granularity of a dataset, e.g., a table or CSV file, to capture similarity at the level of an entire dataset rather than relying on a set of meta-features.
 
Because we use static analysis to capture the semantics of the meta-learning process, we have no mechanism to choose the \textbf{best} pipeline from many seen pipelines, unlike the dynamic execution case where one can rely on runtime to choose the best performing pipeline.  Observing that pipelines are basically workflow graphs, we use graph generator neural models to succinctly capture the statically-observed pipelines for a single dataset. In {\sysname}, we formulate learner selection as a graph generation problem to predict optimized pipelines based on pipelines seen in actual notebooks.

%. This formulation enables {\sysname} for effective pruning of the AutoML search space to predict optimized pipelines based on pipelines seen in actual notebooks.}
%We note that increasingly, state-of-the-art performance in AutoML systems is being generated by more complex pipelines such as Directed Acyclic Graphs (DAGs) \cite{piper} rather than the linear pipelines used in earlier systems.  
 
{\sysname} does learner and transformation selection, and hence is a component of an AutoML systems. To evaluate this component, we integrated it into two existing AutoML systems, FLAML \cite{flaml} and Auto-Sklearn \cite{autosklearn}.  
% We evaluate each system with and without {\sysname}.  
We chose FLAML because it does not yet have any meta-learning component for the cold start problem and instead allows user selection of learners and transformers. The authors of FLAML explicitly pointed to the fact that FLAML might benefit from a meta-learning component and pointed to it as a possibility for future work. For FLAML, if mining historical pipelines provides an advantage, we should improve its performance. We also picked Auto-Sklearn as it does have a learner selection component based on meta-features, as described earlier~\cite{autosklearn2}. For Auto-Sklearn, we should at least match performance if our static mining of pipelines can match their extensive database. For context, we also compared {\sysname} with the recent VolcanoML~\cite{VolcanoML}, which provides an efficient decomposition and execution strategy for the AutoML search space. In contrast, {\sysname} prunes the search space using our meta-learning model to perform hyperparameter optimization only for the most promising candidates. 

The contributions of this paper are the following:
\begin{itemize}
    \item Section ~\ref{sec:mining} defines a scalable meta-learning approach based on representation learning of mined ML pipeline semantics and datasets for over 100 datasets and ~11K Python scripts.  
    \newline
    \item Sections~\ref{sec:kgpipGen} formulates AutoML pipeline generation as a graph generation problem. {\sysname} predicts efficiently an optimized ML pipeline for an unseen dataset based on our meta-learning model.  To the best of our knowledge, {\sysname} is the first approach to formulate  AutoML pipeline generation in such a way.
    \newline
    \item Section~\ref{sec:eval} presents a comprehensive evaluation using a large collection of 121 datasets from major AutoML benchmarks and Kaggle. Our experimental results show that {\sysname} outperforms all existing AutoML systems and achieves state-of-the-art results on the majority of these datasets. {\sysname} significantly improves the performance of both FLAML and Auto-Sklearn in classification and regression tasks. We also outperformed AL in 75 out of 77 datasets and VolcanoML in 75  out of 121 datasets, including 44 datasets used only by VolcanoML~\cite{VolcanoML}.  On average, {\sysname} achieves scores that are statistically better than the means of all other systems. 
\end{itemize}


%This approach does not need to apply cleaning or transformation methods to handle different variances among datasets. Moreover, we do not need to deal with complex analysis, such as dynamic code analysis. Thus, our approach proved to be scalable, as discussed in Sections~\ref{sec:mining}.
%!TEX ROOT = ../centralized_vs_distributed.tex

\section{Problem Setup}\label{sec:setup}

We consider {an undirected network} with $ N $ agents 
{in which the state of the $ i $th agent at time $ t $ is given by $ \xbar{i}{t}\in\Real{} $ with the control input $ \u{i}{t}\in\Real{} $.}
For {notational} convenience,
we {introduce} the aggregate {state of the} system  $ \xbar{}{t} $ and {the}
aggregate control input $ \u{}{t} $ {by stacking states and control inputs of each subsystem} $ \xbar{i}{t} $ and $ \u{i}{t} $, 
respectively.

\iffalse
\begin{rem}[Network topology]
	While in the first part of the paper we focus on circular formations for the sake of analysis and ease of presentation,
	the control design can be readily extended to generic undirected topologies.
	We discuss theoretical guarantees in~\autoref{sec:generic-topology} and observe with computational experiments in~\autoref{sec:numerical-results}
	that the \tradeoff holds regardless of the specific topology. % at hand.
\end{rem}
\fi

\myParagraph{Problem Statement}
The agents aim to reach consensus towards a common state trajectory. 
The $i$th component of the vector $ \x{}{t} \doteq \Omega\xbar{}{t} $ represents 
the mismatch between the state of agent $ i $ and the average network state at time $ t $\revision{\cite{bamjovmitpat12}},
where
\begin{equation}\label{eq:error-matrix}
	\Omega \doteq I_{N}-\consMatrix
\end{equation}
and $ \mathds{1}_N \in\Real{N} $ is the vector of all ones,
such that $ \Omega\mathds{1}_N=0 $.
%\red{The target consensus vector is defined as $ \x{m}{t} \doteq \xbar{}{t} - \x{}{t} $.}

\done{
	\tcb{\myParagraph{Ring Topology}
	We focus on ring topology to obtain analytical insights about 
	optimal control design and fundamental performance trade-offs in the presence of communication delays. 
	While some of our notation is tailored to such topology (\eg see equations~\eqref{eq:meas} and~\eqref{eq:feedback-matrix}), 
	in~\autoref{sec:generic-topology} we discuss extension of the optimal control design to generic undirected networks 
	and complement these developments with computational experiments in~\autoref{sec:numerical-results}.}
}

%\myParagraph{\titlecap{Communication model}}
%The agents communicate through a {shared wireless channel}.
%Data are exchanged through a {shared wireless channel} in a symmetric fashion.
%Agent $ i $ communicates with
%\red{$ n $ pairs of agents,
%both agents in each such pair being at equal distance from $ i $}
%\tcb{its} $ 2n $ closest neighbors \tcb{in ring topology.}
%Also, we make the following assumption
%to address channel constraints.

\begin{ass}[Communication model]\label{ass:hypothesis}
	Data are exchanged through a shared wireless channel in a symmetric fashion.
%	\tcb{its} $ 2n $ closest neighbors \tcb{in ring topology.}
	\revisiontwo{Agent $ i $ receives state measurements from
	all agents within $ n $ communication hops.}
	All measurements are received with delay $ \taun \doteq f(n) $
	where $ f(\cdot) $ is a positive increasing sequence.
	\revisiontwo{In particular,
		in ring topology,
		agent $ i $ receives state measurements from the
		$ 2n $ closest agents,
		that is,
		from the $ n $ pairs of agents at distance $ \ell = 1,\dots,n $,
		with $ 1\le n<\nicefrac{N}{2} $.}\footnote{\revision{
%	where both agents in each such pair are at equal distance $ \ell $ from $ i $. % in the ring topology.
%%	located $ \ell $ positions ahead and behind in the formation,.
%	In what follows,
%	without loss of generality,
%	we assume that such $ n $ agent pairs coincide with the
%	$ 2n $ closest agents in ring topology,
%	and that each pair is at distance $ \ell = 1,\dots,n<\nicefrac{N}{2} $.
	\revisiontwo{For example,
	$ n = 1 $ corresponds to nearest-neighbor interaction in ring topology
	and $ n = \floor{\nicefrac{(N-1)}{2}} $ to all-to-all communication topology.}}}
%	Also, each agent measures its own state instantaneously.
\end{ass}

\revision{\begin{rem}[Architecture parametrization]\label{rem:architecture-param}
	Parameter $ n $ will play a crucial role throughout our discussion. 
	In particular,
	we will use it to (i) evaluate the optimal performance %that can be attained 
	for a given
	budget of links
	\revisiontwo{(see~\cref{prob:variance-minimization})};
	and to (ii) compare optimal performance of different control architectures.
	In the first part of the paper, 
	we examine circular formations and
	$ n $ represents how many neighbor pairs communicate with each agent.
	For \linebreak general undirected networks,
	$ n $ determines the number of communication hops for each agent.
	In general,
	$ n $ characterizes sparsity of a controller architecture:
	sparse controllers correspond to small $n$ while highly connected ones to 
	large $ n $.
\end{rem}}

%\begin{rem}
%	The time $ \delayn $ embeds both the communication delay,
%	due to channel constraints,
%	and the computation delay,
%%	Even though the rate $ f(n) = n $ may seem natural,
%%	other rates are possible, 
%	arising if the agents preprocess the acquired measurements.
%	In practice, $ f(n) $ is to be estimated or learned from data.
%\end{rem}

\myParagraph{\titlecap{Feedback control}}
Agent $ i $ uses the received information to compute the
state mismatches $ \meas{i}{\ell^\pm}{t} $ {relative to its} neighbors,
\begin{equation}\label{eq:meas}
	\meas{i}{\ell^\pm}{t} = 
	\begin{cases}
		\xbar{i}{t} - \xbar{i\pm\ell}{t}, & 0<i\pm\ell\le N\\
		\xbar{i}{t} - \xbar{i\pm\ell\mp N}{t}, & \mbox{otherwise},
	\end{cases}
\end{equation}
%Such mismatches are exploited to compute the 
{and} the proportional control input is {given by}
\begin{equation}\label{eq:prop-control}
	\u{P,i}{t} = -\sum_{\ell=1}^{n}k_\ell\left(\meas{i}{\ell^+}{t-\taun}+\meas{i}{\ell^-}{t-\taun}\right),
\end{equation}
where measurements are delayed according to~\cref{ass:hypothesis}.

For networks with double integrator agents,
the control input $u_i(t)$ may also include a derivative term,
%Depending on the agent dynamics, the control input $ \u{i}{t} $
%may be purely proportional or include a derivative term, such as
\begin{equation}\label{eq:control-input-PD}
	\u{i}{t} = \gvel\u{P,i}{t} - \gvel\dfrac{d\xbar{i}{t}}{dt} = \gvel\u{P,i}{t} -\gvel\dfrac{d\x{i}{t}}{dt}.
\end{equation}
The derivative term in~\eqref{eq:control-input-PD} is delay free
because it only requires measurements coming from the agent itself,
which we assume {to be} available instantaneously. 
%The latter will be defined in due time.
The proportional input can be compactly written as $ \u{P}{t} = -K\xbar{}{t-\taun}=-K\x{}{t-\taun} $.
\revision{With ring topology, the feedback gain matrix is}
\begin{equation}\label{eq:feedback-matrix}
%	\begin{array}{c}
%		K \doteq K_f + K_f^\top \\
		K = \mathrm{circ}
		\left(\sum_{\ell=1}^nk_\ell, -k_1, \dots, -k_n, 0,  \dots, 0, -k_n, \dots, -k_1\right),
%	\end{array}
\end{equation}
where $ \mathrm{circ}\left(a_1,\dots,a_n\right) $ denotes the circulant matrix in $ \Real{n\times n} $
with elements $ a_1,\dots,a_n $ in the first row.

\revisiontwo{For agents with additive stochastic disturbances
	(see Sections \ref{sec:cont-time} and~\ref{sec:disc-time}),
	we consider the following problem for each $ n $.}

\begin{prob}\label{prob:variance-minimization}
	Design the feedback gains in order to minimize the steady-state variance of the consensus error,
%	\marginpar{\tiny Added both problems to highlight the optimization variables in the two cases.}
	\blue{\begin{subequations}\label{eq:problem}
		\begin{equation}\label{eq:variance-minimization-P}
		\mbox{P control:} \qquad \argmin_{K} \; \var(K),
		\end{equation}
		\begin{equation}\label{eq:variance-minimization-PD}
		\mbox{PD control:} \qquad \argmin_{\gvel,K} \; \var(\gvel,K),
		\end{equation}
	\end{subequations}}
	where
	\begin{equation}\label{}
	\var \doteq \lim_{t\rightarrow+\infty} \mathbb{E}\left[\lVert\x{}{t}\rVert^2\right]
	\end{equation}
	and w.l.o.g. we assume $ \mathbb{E}\left[\x{}{\cdot}\right] \equiv \mathbb{E}\left[\x{}{0}\right] = 0 $.
	%	and $ \varx{x} \stackrel{!}{=} $ if the system is mean-square unstable.
\end{prob}
\section{The Continuum Limit}
\label{sec:The Continuum Limit}
Now that we have obtained quantized versions of our lattice theories, the logical next step is to study their continuum limit.
In the following, we will outline a method to define a continuum version of these theories.

First, we note that on every lattice Hilbert space $\mathcal H_n$, there exists a $C*$--algebra $W_n$ of Weyl elements, which is given by
\begin{equation}
    W_n = \overline{\mathrm{span}\Set{\e^{\im\hat\phi_n[f_n] + \im\hat\pi_n[g_n]} | (f_{nk})_k, (g_{nk})_k \in \mathbb R^{N_n}}} .
\end{equation}
We display the continuum Weyl algebra as the projective limit $W = \varprojlim W_n$, where the identifications
\begin{equation}
    \hat\phi_{n+1,2k}f_{n+1,2k} + \hat\phi_{n+1,2k+1}f_{n+1,2k+1} \equiv \hat\phi_{nk}(f_{n+1,2k} + f_{n+1,2k+1}),
\end{equation}
and similarly for $\hat\pi_{nk}$, are made.
We note that the Schrödinger representations on the lattice are irreducible representations of $W_n$, so every vector $\psi_n\in\mathcal H_n$ is a cyclic vector with respect to $W_n$.

In order to find a continuum limit, we need to find a sequence $(\psi_n)_n$ of normalized vectors $\psi_n\in\mathcal H_n$ such that the algebraic states
\begin{equation}
    \omega_n\left(\e^{\im\hat\phi_n[f_n] + \im\hat\pi_n[g_n]}\right) \coloneq \Braket{\psi_n, \e^{\im\hat\phi_n[f_n] + \im\hat\pi_n[g_n]} \psi_n} \label{eq:algebraic-states}
\end{equation}
converge to a state $\omega$ on $W$.
In order to check that, we need to show that the $\omega_n$ are a Cauchy sequence in the following sense:
Let $f$ be a measurable function on $\mathbb T$ and $(f_n)_n$ be a sequence of piecewise constant functions on the $n$--th lattice that converges to $f$.
Analogously, choose a sequence $g_n\to g$.
We say that the $\omega_n$ form a Cauchy sequence if for every $\epsilon>0$, there is $N>0$ such that for all $f_n\to f$, $g_n\to g$ and all numbers $n,m>N$, we have
\begin{equation}
    \abs*{\omega_n\left(\e^{\im\hat\phi_n[f_n] + \im\hat\pi_n[g_n]}\right) - \omega_m\left(\e^{\im\hat\phi_m[f_m] + \im\hat\pi_m[g_m]}\right)} < \epsilon .
\end{equation}
In that case, we define
\begin{equation}
    \omega\left(\lim_{n\to\infty} \e^{\im\hat\phi_n[f_n] + \im\hat\pi_n[g_n]}\right) \coloneq \lim_{n\to\infty} \omega_n\left(\e^{\im\hat\phi_n[f_n] + \im\hat\pi_n[g_n]}\right) .
\end{equation}
The continuum Hilbert space $\mathcal H$ and a cyclic representation of the continuum Weyl algebra $W$ with cyclic vector $\psi$ can then be reconstructed from $\omega$ using the GNS--construction.

We note that this way of obtaining a continuum theory is much simpler than the application of a renormalization group method \parencite[e.g.][]{Lang:2017beo}.
While a renormalization group flow does produce a converging sequence $(\omega_n)_n$ at the fixed point of the flow, the resulting sequence is a very complex object that is hard to obtain.
The fixed point sequence does not only converge, but it satisfies the additional property of cylindrical consistency, which means that the expectation value of a coarse Weyl element is the same on both a coarse lattice and a fine lattice.
This additional property is much stronger than mere convergence, but not strictly required for the definition of a theory in the continuum.
On the other hand, generating a convergent sequence by means of a renormalization group flow has the advantage of being quite systematic.

\subsection{Defining Convergent Sequences}
It can be difficult to choose a sequence $(\psi_n)_n$ of states that has a continuum limit.
In general, one may need to make use of the freedom to be able to choose the lattice approximations $f_n$ of the continuum test functions $f$.
Different choices may lead to different limits or may develop divergences.

We want to illustrate one particularly simple approach to obtaining such a sequence.
Let us choose an arbitrary state $\psi_{m} \in \mathcal H_{m}$.
We iteratively define a sequence of fine states $(\psi_n)_{n\geq m}$ from $\psi_{m}$ as follows:
\begin{equation}
    \psi_{n+1}((\phi_{n+1,k})_k) \coloneq 2^{-2^{n-1}} \psi_n((\phi_{nk}^+)_k) \prod_{k=1}^{N_n} \rho_{n}(\phi_{nk}^-),
\end{equation}
where
\begin{equation}
    \phi_{nk}^+ = \frac{\phi_{n+1,2k} + \phi_{n+1,2k-1}}{2}, \quad \phi_{nk}^- = \frac{\phi_{n+1,2k} - \phi_{n+1,2k-1}}{2}. \label{eq:change-of-variables}
\end{equation}
Analogously, we define $\pi_{nk}^\pm$.
The functions $\rho_n$ are arbitrary, yet to be chosen functions of one variable.
They quantify the distribution of the local fluctuations on each scale.

By expanding the canonical variables in $\hat\phi_{nk}^\pm$ and $\hat\pi_{nk}^\pm$, where $\commutator{\hat\phi_{nk}^\pm}{\hat\pi_{nk'}^\mp} = 0$, we can evaluate the algebraic states $\omega_n$ (as defined in \cref{eq:algebraic-states}) on the corresponding Weyl elements.
Using the shorthand $f_{nk}^{n+1} = f_{n+1,2k-1}+f_{n+1,2k}$, we find:
\begin{align}
    \MoveEqLeft \phi_{n+1}[f_{n+1}]
         = \sum_{k=1}^{N_{n+1}} \phi_{n+1,k}f_{n+1,k} \nonumber \\
        &= \sum_{k=1}^{N_n}\phi_{nk}^+(f_{n+1,2k-1}+f_{n+1,2k}) + \sum_{k=1}^{N_n} \phi_{nk}^-(f_{n+1,2k}-f_{n+1,2k-1}) \nonumber \\
        &= \sum_{k=1}^{N_n}\phi_{nk}^+ f_{nk}^{n+1} + \eta_{n+1} \sum_{k=1}^{N_n} \phi_{nk}^- (\Delta f)^{n+1}_{n k} \nonumber \\
        &= \phi_n^+[f_n^{n+1}] + \eta_{n+1} \,\phi_{n}^- \left[ (\Delta f)^{n+1}_{n} \right],
\end{align}
where we defined
\begin{align}
    (\Delta f)^{n+1}_{n k} \coloneq \frac{1}{\eta_{n+1}} \left( f_{n+1, 2k} - f_{n+1,2k-1}\right).
\end{align}
A similar expression follows for $\pi_{n+1}[f_{n+1}]$.
Consequently, we find the following relation for the Weyl elements:
\begin{align}
    \e^{\im\hat\phi_{n+1}[f_{n+1}] + \im\hat\pi_{n+1}[g_{n+1}]}
        = \e^{\im\hat\phi_n^+[f_n^{n+1}] + \im\hat\pi_n^+[g_n^{n+1}]} \prod_{k=1}^{N_n} \e^{\im\eta_{n+1}\left(\hat\phi_{n}^- [(\Delta f)^{n+1}_{n }] + \hat\pi_{n}^- [(\Delta g)^{n+1}_{n }]\right)},
\end{align}
We note that all factors on the right hand side commute.
This allows us to evaluate the $\omega_{n+1}$ on the Weyl elements by a simple change of variables according to \cref{eq:change-of-variables}:
\begin{align}
    \omega_{n+1}(\e^{\im\hat\phi_{n+1}[f_{n+1}] + \im\hat\pi_{n+1}[g_{n+1}]})
        = \omega_n(\e^{\im\hat\phi_n[f_n^{n+1}] + \im\hat\pi_n[g_n^{n+1}]}) \prod_{k=1}^{N_n} \zeta_{n}((\Delta f)^{n+1}_{nk}, (\Delta g)^{n+1}_{nk}) .
\end{align}
Here, we defined
\begin{equation} \label{def:zeta_n}
    \zeta_{n}((\Delta f)^{n+1}_{nk}, (\Delta g)^{n+1}_{nk}) \coloneq \Braket{\rho_n, \e^{\im \eta_{n+1} (\hat\phi_{nk}^- (\Delta f)^{n+1}_{nk} + \hat\pi_{nk}^-(\Delta g)^{n+1}_{nk})} \rho_n}.
\end{equation}
It follows that the corresponding states $\omega_n$ for any $n>m$, evaluated on elements of the Weyl algebra, read
\begin{equation}
    \omega_n\left(\e^{\im\hat\phi_n[f_n] + \im\hat\pi_n[g_n]}\right) = \omega_{m}\left(\e^{\im\hat\phi_{m}[f^n_{m}] + \im\hat\pi_{m}[g^n_{m}]}\right) \prod_{l=m}^{n-1} \prod_{k=1}^{N_l} \zeta_l ((\Delta f)^n_{m k}, (\Delta g)^n_{m k}),
\end{equation}
where we use that the components of the vectors $f^n_{m}$ and $(\Delta f)^n_m$ are respectively given by
\begin{align}
    f^n_{m k} &= \sum_{j=1}^{N_n N_{m}^{-1} } f_{n, (k-1) N_n N_{m}^{-1} +j}, \\
    (\Delta f)^n_{m k} &= \sum_{j=1}^{N_n N_{m}^{-1}} f_{n, (k-1) N_n N_{m}^{-1} +j} \cdot \sgn (2j -1- N_n N_{m}^{-1}),
\end{align}
where $k =1,\dots, N_{m}$. Similar expressions hold for $g^n_{m}$ and $(\Delta g)^n_m$.
If the $\rho_l$ are chosen appropriately, the product of the $\zeta_l$ converges and a continuum limit exists.

Possible examples that lead to convergence are given by Dirac sequences with suitable speed of convergence.
A Dirac sequence of Gaussian distributions will lead to a Gaussian state in the continuum and thus to a Fock space representation.
Dirac sequences with compact support can be used to constrain the admissible states to those with support on configurations of bounded variation.
Such states are interesting, because functions of bounded variation are differentiable almost everywhere, which can aid the convergence of finite differences $\Delta^{\eta_n} \hat\phi_{nk}$ to derivatives $\partial\hat\phi(x)$ of field operators.

\subsection{Representation of the Group of Gauge Transformations}
Now that we have a continuum Hilbert space, we would like to implement gauge transformations on it.
Unfortunately, the operators $U_{D_n[f_n]}(s) W_n U_{D_n[f_n]}(s)^\dagger$ are not necessarily Weyl elements again.
Thus, implementing the gauge transformations on the continuum Hilbert space is not as easy as taking the limit of the transformed Weyl elements in the projective limit algebra $W$.
Nevertheless, we can describe a plausible procedure:

Suppose we want to apply a gauge transformation $U_{D[f]}(s)$ on a state $\psi \in \mathcal H$ in the continuum Hilbert space.
We have to choose a sequence $(\psi_n)_n$ in the lattice Hilbert spaces $\mathcal H_n$ that converges to $\psi$ in the sense of algebraic states.
Moreover, we choose a sequence $f_n$ of piecewise constant functions with $f_n\to f$.
We can then apply the approximate gauge transformations $U_{D_n[f_n]}(s)$ to $\psi_n$.
Since every state in $\mathcal H_n$ is cyclic for $W_n$, we can expand the transformed state in every lattice Hilbert space again in terms of Weyl elements:
\begin{equation}
    U_{D_n[f_n]}(s)\psi_n = \sum_k c_{nk} \e^{\im\hat\phi_n[f_{nk}] + \im\hat\pi_n[g_{nk}]} \psi_n
\end{equation}
We can then use this sequence of Weyl elements to define the continuum gauge transformation as follows:
\begin{equation}
    U_{D[f]}(s)\psi = \lim_{n\to\infty}\sum_k c_{nk} \e^{\im\hat\phi_n[f_{nk}] + \im\hat\pi_n[g_{nk}]} \psi.
\end{equation}

% \vspace{-0.5em}
\section{Conclusion}
% \vspace{-0.5em}
Recent advances in multimodal single-cell technology have enabled the simultaneous profiling of the transcriptome alongside other cellular modalities, leading to an increase in the availability of multimodal single-cell data. In this paper, we present \method{}, a multimodal transformer model for single-cell surface protein abundance from gene expression measurements. We combined the data with prior biological interaction knowledge from the STRING database into a richly connected heterogeneous graph and leveraged the transformer architectures to learn an accurate mapping between gene expression and surface protein abundance. Remarkably, \method{} achieves superior and more stable performance than other baselines on both 2021 and 2022 NeurIPS single-cell datasets.

\noindent\textbf{Future Work.}
% Our work is an extension of the model we implemented in the NeurIPS 2022 competition. 
Our framework of multimodal transformers with the cross-modality heterogeneous graph goes far beyond the specific downstream task of modality prediction, and there are lots of potentials to be further explored. Our graph contains three types of nodes. While the cell embeddings are used for predictions, the remaining protein embeddings and gene embeddings may be further interpreted for other tasks. The similarities between proteins may show data-specific protein-protein relationships, while the attention matrix of the gene transformer may help to identify marker genes of each cell type. Additionally, we may achieve gene interaction prediction using the attention mechanism.
% under adequate regulations. 
% We expect \method{} to be capable of much more than just modality prediction. Note that currently, we fuse information from different transformers with message-passing GNNs. 
To extend more on transformers, a potential next step is implementing cross-attention cross-modalities. Ideally, all three types of nodes, namely genes, proteins, and cells, would be jointly modeled using a large transformer that includes specific regulations for each modality. 

% insight of protein and gene embedding (diff task)

% all in one transformer

% \noindent\textbf{Limitations and future work}
% Despite the noticeable performance improvement by utilizing transformers with the cross-modality heterogeneous graph, there are still bottlenecks in the current settings. To begin with, we noticed that the performance variations of all methods are consistently higher in the ``CITE'' dataset compared to the ``GEX2ADT'' dataset. We hypothesized that the increased variability in ``CITE'' was due to both less number of training samples (43k vs. 66k cells) and a significantly more number of testing samples used (28k vs. 1k cells). One straightforward solution to alleviate the high variation issue is to include more training samples, which is not always possible given the training data availability. Nevertheless, publicly available single-cell datasets have been accumulated over the past decades and are still being collected on an ever-increasing scale. Taking advantage of these large-scale atlases is the key to a more stable and well-performing model, as some of the intra-cell variations could be common across different datasets. For example, reference-based methods are commonly used to identify the cell identity of a single cell, or cell-type compositions of a mixture of cells. (other examples for pretrained, e.g., scbert)


%\noindent\textbf{Future work.}
% Our work is an extension of the model we implemented in the NeurIPS 2022 competition. Now our framework of multimodal transformers with the cross-modality heterogeneous graph goes far beyond the specific downstream task of modality prediction, and there are lots of potentials to be further explored. Our graph contains three types of nodes. while the cell embeddings are used for predictions, the remaining protein embeddings and gene embeddings may be further interpreted for other tasks. The similarities between proteins may show data-specific protein-protein relationships, while the attention matrix of the gene transformer may help to identify marker genes of each cell type. Additionally, we may achieve gene interaction prediction using the attention mechanism under adequate regulations. We expect \method{} to be capable of much more than just modality prediction. Note that currently, we fuse information from different transformers with message-passing GNNs. To extend more on transformers, a potential next step is implementing cross-attention cross-modalities. Ideally, all three types of nodes, namely genes, proteins, and cells, would be jointly modeled using a large transformer that includes specific regulations for each modality. The self-attention within each modality would reconstruct the prior interaction network, while the cross-attention between modalities would be supervised by the data observations. Then, The attention matrix will provide insights into all the internal interactions and cross-relationships. With the linearized transformer, this idea would be both practical and versatile.

% \begin{acks}
% This research is supported by the National Science Foundation (NSF) and Johnson \& Johnson.
% \end{acks}

\printbibliography

\end{document}
 
