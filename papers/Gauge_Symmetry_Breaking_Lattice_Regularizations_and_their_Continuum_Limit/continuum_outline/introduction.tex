\section{Introduction}
Establishing rigorous definitions of quantum field theories in the continuum presents a formi\-dable challenge.
It arises from the presence of infinities within naively constructed interaction terms and is rooted in the generally ill--defined process of multiplying distributions. 
Consequently, it is not feasible to directly formulate quantum field theories in the continuum. 
Instead, a common approach is to first establish regularized versions of the theory as an intermediate step. 
These regularized theories encompass a distinct set of parameters to govern the regularization process, supplementing the inherent free parameters of the theory. 
Subsequently, the continuum theory emerges as a limit reached through a carefully tuned trajectory in the theory space. Along this path, divergent terms are compensated by appropriately chosen counterterms, ensuring the existence of the limit \parencite{Montvay:1994cy}.

Facilitating the existence of the limit is greatly enhanced by preserving key attributes of the continuum theory within the regularized theories.
Although, in principle, certain features may and sometimes have to manifest in the limit despite their absence at finite (lattice) orders, it proves significantly more straightforward to seamlessly inherit these features from the regularized theories. 

An illustrative instance demonstrating the challenge of preserving a feature after the regularization process pertains to the problem of fermion doubling in lattice field theory. 
In fact, due to a theorem by \textcite{Nielsen:1981hk}, lattice field theories cannot accommodate chiral fermions --- an essential component of the standard model in particle physics. 
To circumvent this issue, a widely adopted approach involves acknowledging the existence of the spurious doubler degrees of freedom on the lattice while progressively suppressing them as the continuum limit is approached. 

A similar phenomenon arises when a gauge symmetry is broken on a lattice. 
In the Hamiltonian formalism, the existence of gauge symmetry is characterized by the presence of a system of first--class constraints. 
The lattice regularization process breaks the first--class property, leading to a situation where the constraints are no longer conserved during time evolution. 
Consequently, physical states eventually acquire unphysical degrees of freedom. 
While the origin of these unphysical degrees of freedom differs from the fermion doubling phenomenon, the effect is similar and poses comparable challenges. 

Unlike the issue of fermion doubling, breaking gauge symmetries on a lattice can often be circumvented. 
A method introduced by \textcite{Wilson:1974sk} allows to regularize a gauge field in terms of group--valued link variables on a lattice, representing the holonomies of the gauge field. 
From these link variables, one can construct gauge--invariant entities around closed loops known as Wilson loops, which are then utilized to obtain regularized and invariant representations of gauge theory actions.
A notable example of this approach is the Wilson action, which serves as a gauge--invariant lattice regularization of the Yang--Mills action. 

We would like to emphasize that this lattice regularization scheme only applies to gauge fields defined in terms of connections on principal $G$--bundles and when gauge transformations align with principal $G$--automorphisms. 
While this framework is well--suited for many theories of interest, including Yang--Mills theory and the standard model, it is inapplicable in the case of general relativity.
In principle, it's feasible to reformulate the theory, using the frame bundle, in the language of principal $GL(n)$--bundles and connections thereon. 
However, a crucial distinction arises in the group of gauge transformations, given by the group of spacetime diffeomorphisms. 
In general, these diffeomorphisms do not correspond to principal $GL(n)$--automorphisms on the frame bundle as they don't preserve the base manifold \parencite{Isham:1999rh}.
Consequently this hinders the direct implementation of Wilson's techniques for obtaining a lattice regularization of general relativity while maintaining gauge invariance.

The enduring prevalence of the Wilson action (and its extensions) as the primary choice for a lattice formulation of Yang--Mills theory is no mere coincidence. 
The inherent gauge invariance of this action serves to effectively prevent the emergence of unwanted physical degrees of freedom and greatly contributes to the theory's renormalizability. 
However, as appealing as the preservation of gauge invariance may be, there exists room for exploration into alternative regularization schemes that do break gauge invariance. 
Indeed, the literature contains various attempts in this direction \parencite[see, for instance,][]{Rivasseau-YM4}. 
Similar to the previously mentioned case of fermion doubling, alternative approaches may initially introduce unphysical degrees of freedom at the regularization level, which must subsequently be carefully controlled and suppressed. 
While the introduction of these additional technical challenges should not be underestimated, it does not diminish the general viability of such alternative methods. 

In fact, our need for an alternative approach arises from the inability to construct a gauge--invariant lattice regularization for general relativity, prompting us to explore a different path. 
In our previous work \parencite{lattice}, we introduced a lattice regularization for the Hamiltonian formulation of classical general relativity employing spatial metric variables. 
See also \textcite{Regge,spinfoams,CDT} for other approaches of discretizing gravity.
The natural progression toward a quantum gravity theory first necessitates the quantization of the regularized expression, followed by the challenging task of taking the continuum limit. 
This paper aims to provide an overview of our proposed approach for achieving this limit, specifically addressing the representation of the diffeomorphism group, derived from the diffeomorphism constraints, on the resulting Hilbert space.
To enhance the readability of this outline paper, we opt not to delve directly into the intricate details arising from our prior work \parencite{lattice}. 
Instead, we explore a simplified scenario where the metric tensor is replaced by a scalar field, and a general set of constraints is considered. 
While these restrictions may not directly apply to many intriguing theories, their extension to general relativity should become evident.

A significant limitation of this method lies in the necessity for the regularized constraints to be linear in the momenta.
Although this requirement might initially appear stringent, it finds compliance within a wide range of symmetry generators in physics.
This criterion ensures that the generated symmetries are defined by their action on the configuration variables, with their action on the momenta already prescribed.
In the context of gravity, this condition holds true for the diffeomorphism constraints that generate spatial diffeomorphisms, but not for the Hamiltonian constraint. 

Our current approach addresses only a portion of the challenge at hand. 
While the Hamiltonian constraint can be quantized on the lattice utilizing the technique introduced in \textcite{cholesky}, bridging the gap to its continuum limit necessitates a distinct undertaking. 
While the regularized algebra of spatial diffeomorphism constraints encounters gauge--breaking anomalies solely proportional to the lattice spacing, the complete regularized Dirac algebra incorporates terms proportional to $\hbar$.
These would persist in a naive limit of an infinitesimal lattice spacing.
Consequently, we anticipate the need for the application of renormalization group methods (e.g., \textcite{Lang:2017beo,Thiemann-Renormalization}, see also \textcite{Asante:2022dnj,Ambjorn:2020rcn,Saueressig:2023irs} for overviews of renormalization schemes in path integral approaches to quantum gravity\footnote{A more detailed overview of renormalization in quantum gravity will be provided in \textcite{gravity-continuum} along with an extensive number of references.}), to produce suitable counterterms.
Nevertheless, we envision our proposal as providing an ideal foundation for such analyses. 
As astute readers may discern, our method seemingly holds the potential to produce non--trivial continuum Hilbert spaces with corresponding representations of the gauge group. 

Lastly, it is worth emphasizing that this paper outlines the general framework of our method without delving into the technical for every statement. 
We have chosen to reserve these details for an upcoming paper \parencite{gravity-continuum}, where we will provide a more concrete and comprehensive treatment. 

With this, let us come to the structure of this paper:
In \cref{sec:General Setup}, we present a comprehensive review of the regularization and the quantization of the Hamiltonian field theory under examination.
This deliberately aligns with the findings of \textcite{lattice}.
In contrast to the findings of \textcite{cholesky}, we use a standard Schrödinger representation for the purpose of illustration.
Subsequently, \Cref{sec:The Continuum Limit} examines methods for obtaining non--trival continuum limits of the discretized and quantized theories.
Finally, in \cref{sec:Conclusion}, we offer a discussion of our results, summarizing our key findings.
