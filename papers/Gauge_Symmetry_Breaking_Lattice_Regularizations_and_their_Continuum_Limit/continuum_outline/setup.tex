\section{General Setup}
\label{sec:General Setup}

\subsection{Classical Theory}
For simplicity, let us consider a classical one--dimensional Hamiltonian field theory of a scalar field $\phi(x)$ and its canonically conjugate momentum $\pi(x)$ on a torus $\mathbb T = [0,1]$, where $0$ and $1$ are identified.
We assume that there exists an algebra of constraints $D[f]$ with
\begin{equation}
    D[f] = \int_{\mathbb T} \mathcal D(\phi(x), \partial \phi(x), \pi(x), \partial \pi(x)) f(x) \d x 
\end{equation}
such that the first class relations
\begin{equation}
    \poisson{D[f]}{D[g]} = D[F(f,\partial f,g,\partial g)] \label{eq:classical-continuum-algebra}
\end{equation}
hold for some $F$ that does not depend on the fields $\phi$ or $\pi$.

The aim is to derive a lattice discretized version of this theory on a regular lattice with lattice spacing $\eta$.
In order to achieve, this, we proceed in two steps.
First, we replace all occurring spatial derivatives in $D[f]$, such as $\partial f(x)$, by finite differences
\begin{equation}
    \Delta^\eta f(x) \coloneq \frac{f(x+\eta)-f(x)}\eta,
\end{equation}
and obtain approximate expressions $D_\eta[f]$.
In a second step, we evaluate the resulting expression on a restricted set of phase space functions, parameterized only by a finite number of degrees of freedom.
On a lattice with lattice spacing $\eta_n = 2^{-n}$, we define a piecewise constant version of the field given by
\begin{equation}
    \phi_n(x) \coloneq \sum_{k=1}^{N_n} \phi_{nk} \, \chi_{X_k}(x),
\end{equation}
where $N_n = 2^n$, $X_k = [(k-1) 2^{-n}, k 2^{-n}]$ with $k = 1, \dots, 2^n$, and $\chi_{X_k}$ is the characteristic function on $X_k$.
Similarly, we define piecewise constant versions of $\pi$ and $f$.

Strictly speaking, these functions do not really arise as a restriction of the original phase space, because the latter consists of differentiable functions.
However, our choice of the phase space is a useful way of understanding the process.
Of course, it is possible to use piecewise polynomial, everywhere differentiable functions instead.
This would constitute a higher--order approximation and a true restriction of phase space.
It would even allow us to skip the first step of replacing derivatives by finite differences.
The zeroth order approximation using piecewise constant functions, however, turns out to be most convenient and sufficient in order to obtain a lattice approximation of the continuum theory.

By evaluating the approximate constraint $D_{\eta_n}[f]$ on the piecewise constant functions $\phi_n$, $\pi_n$ and $f_n$, we obtain a version of the constraints that only depends on the finitely many degrees of freedom $(\phi_{nk})_k$ and $(\pi_{nk})_k$:
\begin{align}
    \MoveEqLeft D_{\eta_n}[f_n]
         = \int_{\mathbb T} \mathcal D(\phi_n(x), \Delta^{\eta_n} \phi_n(x), \pi_n(x), \Delta^{\eta_n} \pi_n(x)) f_n(x) \d x \\
        &= \sum_{k=1}^{N_n} \int_{X_k} \mathcal D(\phi_n(x), \Delta^{\eta_n} \phi_n(x), \pi_n(x), \Delta^{\eta_n} \pi_n(x)) f_n(x) \d x \\
        &= \sum_{k=1}^{N_n} \mathcal D(\phi_{nk}, \Delta^{\eta_n} \phi_{nk}, \pi_{nk}, \Delta^{\eta_n} \pi_{nk}) f_{nk} \eta_n . \label{eq:constraint-lattice}
\end{align}
By abuse of notation, we defined
\begin{equation}
    \Delta^{\eta_n} \phi_{nk} \coloneq \frac{\phi_{n,k+1} - \phi_{n,k}}{\eta_n} .
\end{equation}
\Cref{eq:constraint-lattice} represents a well defined expression on a finite--dimensional phase space with canonical variables $(\phi_{nk})_k$ and $(\pi_{nk})_k$.
Since we choose the underlying manifold to be a torus $\mathbb T$, we will assume periodic boundary conditions such as $\phi_{n,N_n+1} = \phi_{n,1}$.
Moreover, we will restrict ourselves to the lattice spacings $\eta_n = 2^{-n}$ as introduced above in the remainder of the paper.
Moreover, we will assume that $D_n$ is sufficiently well behaved such that $D_n[f_n]\to D[f]$ as $n\to\infty$ if an appropriate sequence $f_n(x)\to f(x)$ is chosen.

The Poisson bracket on the phase space of piecewise constant functions must be chosen in order to be consistent with the continuum Poisson bracket
\begin{equation}
    \poisson{\phi[f]}{\pi[g]} = \int_{\mathbb T} f(x)\, g(x) \d x .
\end{equation}
By evaluating this expression on piecewise constant functions, we find \parencite[cf.][]{lattice}
\begin{equation}
    \poisson{\phi_{nk}}{\pi_{nk'}} = \eta_n^{-1} \delta_{kk'} . \label{eq:classical-lattice-ccr}
\end{equation}
This allows us to compute the Poisson brackets of the lattice constraints $D_n[f_n]$.
We will assume that they obey the relation
\begin{equation}
    \poisson{D_n[f_n]}{D_n[g_n]} = D_n[F_n(f_n,\Delta^n f_n,g_n,\Delta^n g_n)] + \eta_n G_n(f_n,\Delta^n f_n,g_n,\Delta^n g_n) \label{eq:classical-lattice-algebra}
\end{equation}
with functions $F_n$ such that $D_n[F_n]\to D[F]$, and $G_n$ is bounded as $n\to\infty$.
This assumption holds, for instance, in the case of the diffeomorphism constraints, as was shown in \textcite{lattice}.

Let us interpret this relation.
Due to our assumptions, the $\eta_n G_n$ term will approach zero as $n$ increases.
Thus, for sufficiently fine lattices, the resulting algebra is dominated by the $D_n$ terms.
As $D_n[f_n]\to D[f]$, the algebra more and more resembles the original continuum algebra \labelcref{eq:classical-continuum-algebra}.
The continuum constraint $D[f]$ generates gauge transformations by the evolution according to Hamilton's equations of motion
\begin{equation}
    \diff{\phi[g]}{s} = \poisson{\phi[g]}{D[f]}, \quad \diff{\pi[g]}{s} = \poisson{\pi[g]}{D[f]}.
\end{equation}
\Cref{eq:classical-lattice-algebra} implies that we can interpret the Hamiltonian evolution
\begin{equation}
    \diff{\phi_n[g_n]}{s} = \poisson{\phi_n[g_n]}{D_n[f_n]}, \quad \diff{\pi_n[g_n]}{s} = \poisson{\pi_n[g_n]}{D_n[f_n]} \label{eq:hamiltons-equations}
\end{equation}
of the lattice degrees of freedom with respect to $D_n[f_n]$ as lattice approximations of continuum gauge transformations.

It is important to contemplate for a moment the presence of the $G_n$ terms in the lattice algebra \labelcref{eq:classical-lattice-algebra}.
Strictly speaking, their existence makes the lattice constraints $D_n[f_n]$ into second class constraints, since their Poisson bracket ceases to be a linear combination of constraints.
This means that the evolution of a point in phase space with respect to an approximate gauge transformation will move that point out of the constraint hypersurface.
Thus, states will acquire unphysical degrees of freedom, even if they initially satisfied the constraints.
Moreover, the composition of two approximate gauge transformations can not again be written as an exact approximate gauge transformation with respect to a constraint $D_n[h_n]$ for some $h_n$ and thus, they will not form a group.

However, due to the smallness of $\eta_n G_n$, the composition of two approximate gauge transformations will again be very close to an approximate gauge transformation of the same class, at least for sufficiently short evolution periods.
Also, and again for short evolution periods, the condition $D_n[f_n]\to D[f]$ ensures that approximate gauge transformations will resemble actual continuum gauge transformations quite well as the lattice approaches the continuum.
In order to control the error bounds, one may have to let the lattice spacing $\eta_n$ approach zero more quickly than one needs to let the sequence $f_n$ of lattice test functions approach the continuum test function $f$, or in other words, in order to obtain a good approximation to a continuum gauge transformation, one needs to work on a sufficiently fine lattice.

For the remainder of the paper, we will assume that $D_n[f_n]$ depends only linearly on the momenta $\pi_{nk}$.
As a consequence, Hamilton's equation of motion \labelcref{eq:hamiltons-equations} for $\phi_n[g_n]$ is a first order system of differential equations that only depends on the configuration variables $\phi_{nk}$ and not on the momenta.
Thus, its solution only requires initial data for the configuration variables.

The map $\varphi_s^{D_n[f_n]} \colon \mathbb R^n\to\mathbb R^n$ that sends an initial configuration to its evolved version with regards to the evolution with respect to $D_n[f_n]$ and with evolution parameter $s$, is called the Hamiltonian flow.
It satisfies $\varphi_0^{D_n[f_n]} = \id$, is invertible with inverse given by $\cramped{\left(\varphi_s^{D_n[f_n]}\right)^{\mathrlap{-1}} = \varphi_{-s}^{D_n[f_n]}}$ and respects the composition rule $\varphi_s^{D_n[f_n]} \circ \varphi_t^{D_n[f_n]} = \varphi_{s+t}^{D_n[f_n]}$.
The Hamiltonian flow of $D_n[f_n]$ is a canonical transformation on phase space and represents the approximate gauge transformation generated by it.

\subsection{Quantum Theory}
We will now proceed with the quantization of the above--introduced classical lattice field theories, focusing on the Schrödinger representation as an illustrative case. 
Nonetheless, it is worth noting that alternative representations may prove advantageous depending on the circumstances \parencite[cf.]{cholesky}.
Within the Schrödinger representation, the Hilbert space associated with a theory featuring a lattice spacing $\eta_n$ is defined as $\mathcal H_n = L^2(\mathbb R^n)$.
To quantize the field degrees of freedom $\phi_{nk}$, we employ multiplication operators
\begin{equation}
    (\hat\phi_{nk}\psi_n)(\phi_{n1},\ldots,\phi_{n N^n}) = \phi_{nk}\psi_n(\phi_{n1},\ldots,\phi_{n N^n}),
\end{equation}
while the momentum degrees of freedom $\pi_{nk}$ act as differential operators according to
\begin{equation}
    (\hat\pi_{nk}\psi_n)(\phi_{n1},\ldots,\phi_{n N^n}) = -\im\eta_n^{-1}\diffp{\psi_n}{\phi_{nk}}(\phi_{n1},\ldots,\phi_{n N^n}) .
\end{equation}
We emphasize the factor $\eta_n^{-1}$, which ensures the correct quantization of the classical Poisson bracket relation \labelcref{eq:classical-lattice-ccr}
\begin{equation}
    \commutator{\hat\phi_{nk}}{\hat\pi_{nk'}} = \im\eta_n^{-1} \delta_{kk'} .
\end{equation}

We then proceed to define a representation of the approximate gauge transformations $\varphi_s^{D_n[f_n]}$ on the lattice Hilbert space $\mathcal H_n$ as follows\footnote{A similar construction has been used by \textcite{Thiemann-U1} for the quantization of Euclidean $U(1)$ gravity in a Narnhofer--Thirring type representation.}:
\begin{equation}
    \left(U\,\left(\varphi_s^{D_n[f_n]}\right)\psi_n\right)((\phi_{nk})_k) = \sqrt{\det\left(J_{\varphi_s^{D_n[f_n]}}((\phi_{nk})_k)\right)}\,\psi_n(\varphi_s^{D_n[f_n]}((\phi_{nk})_k)). \label{eq:quantum-gauge-transformation}
\end{equation}
For simplicity, let us denote this expression by $U_{D_n[f_n]}(s)$ from now on.
Here, $\cramped{J_{\varphi_s^{D_n[f_n]}}}$ is the Jacobian matrix of the Hamiltonian flow.

This representation inherits the properties of the Hamiltonian flow $\varphi_s^{D_n[f_n]}$:
Evidently, it evaluates to the identity transformation for $s=0$, and a simple application of the multivariable chain rule immediately yields
\begin{equation}
    U_{D_n[f_n]}(s)\,U_{D_n[f_n]}(t)=U_{D_n[f_n]}(s+t) ,
\end{equation}
which also implies that $(U_{D_n[f_n]}(s))^{-1} = U_{D_n[f_n]}(-s)$.
The square root in \cref{eq:quantum-gauge-transformation} ensures that $\norm{U_{D_n[f_n]}(s)} = 1$.
Taken together, we find that $U_{D_n[f_n]}(s)$ is a unitary operator.
Moreover, the continuity of $\varphi_s^{D_n[f_n]}$ in $s$ implies that $U_{D_n[f_n]}(s)$ is strongly continuous, as is evident from \cref{eq:quantum-gauge-transformation}.
Therefore, we have obtained a strongly continuous one--parameter group of transformations, which, according to Stone's theorem, also possesses a generator, which can be taken to be the representation of the lattice constraint $D_n[f_n]$ on the lattice Hilbert space:
\begin{equation}
    \hat D_n[f_n] \psi_n \coloneq -\im\diff{U_{D_n[f_n]}(s)}s[\mathrlap{s=0}]\; \psi_n .
\end{equation}

We emphasize that for this construction to work, it is essential that $D_n[f_n]$ depends only linearly on the momenta $\pi_{nk}$.
Otherwise, the Hamiltonian flow of the configuration variables would depend not only on the configuration variables, but also on the momenta as initial conditions.
Since the $\psi_n\in\mathcal H_n$ are functions of the configuration variables only, we can not construct the representation of the approximate gauge transformations as in \cref{eq:quantum-gauge-transformation}.
In the more general case, one may, for example, apply Weyl quantization techniques.
However, this may lead to other difficulties later on.
We will address these issues in a future publication.

As in the classical case, the totality of approximate lattice gauge transformations don't form a group and their generators don't form a first class system.
However, one may use the classical relation
\begin{equation}
    \diff{}{s}[\mathrlap{s=0}]\;\, \varphi_{-\sqrt{s}}^{D_n[f_n]}\circ\varphi_{-\sqrt{s}}^{D_n[g_n]}\circ\varphi_{\sqrt{s}}^{D_n[f_n]}\circ\varphi_{\sqrt{s}}^{D_n[g_n]} = \diff{}{s}[\mathrlap{s=0}]\;\, \varphi_{s}^{\poisson{D_n[f_n]}{D_n[g_n]}},
\end{equation}
and the fact that $\eta_n G_n\to 0$ to show that the corresponding relations between the quantum representations of the approximate gauge transformations will hold better and better as $n\to\infty$.
