\section{Conclusion} \label{sec:Conclusion}
In this paper, we presented a novel approach for taking the continuum limit of a Hamiltonian lattice regularized quantum field theory with first--class constraints.
Specifically, we considered a one--dimensional Hamiltonian theory of a scalar field that we discretized by introducing a phase space consisting of piecewise constant functions.
We assumed the first--class character of the continuous field theory to be broken by discretization in such a way that the additional anomalies scale with the lattice spacing of the regular spatial lattice.
Thereby, the evolution leading out of the constraint surface due to the second--class anomalies can be made arbitrarily small by passing to finer lattices.
We then proceeded to quantize these lattice theories using a standard Schrödinger representation.
By assuming that the constraints depend only linearly on the momenta, we obtain unitary representations of the approximate gauge transformations associated with the regularized constraints.
Since, on the level of the lattice, they are also strongly continuous, they admit generators which are the appropriate quantizations of the regularized classical constraints.
Finally, we presented a way of defining states on the algebra of continuum field operators, making use of the Cauchy sequence criterion.
Our method provides the user with considerable freedom of tuning the limit and thereby fixing desirable properties in the continuum theory.
We then provide a prescription for defining a representation of the group of gauge symmetries on the resulting continuum Hilbert space.
We expect that one can utilize the aforementioned freedom to have the continuum limit inherit the property of strong continuity of the approximate gauge transformations on the lattices.
Given the broad generality of the framework introduced in this paper, our discussion has been more of an outline than a detailed construction.
To establish concrete convergence and continuity, case--specific proofs will be necessary in more specific contexts. 
This is because the precise tuning of the continuum limit must be adapted to the particular expression for which we seek to determine the limit. 

In a next step, we intend to apply our techniques to the theory of general relativity.
In this case, the set of constraints consists of the diffeomorphism constraints and the Hamiltonian constraint.
Since the diffeomorphism constraints are linear in the momenta, we aim to utilize the methods introduced in this paper to construct a strongly continuous representation of the associated group of spatial diffeomorphisms.
This will be the topic of a future publication \parencite{gravity-continuum}, wherein missing proofs and hard estimates will be provided.
For the Hamiltonian constraint, we expect the need of more sophisticated methods as it is quadratic in the momenta and consequently, the gauge invariance breaking terms no longer tend to vanish in the limit of infinitesimal lattice spacings.
In order to produce counterterms that compensate these extra terms, renormalization group methods will become relevant.

The central theme of our proposal is to sacrifice gauge invariance on the lattice at the expense of introducing unphysical degrees of freedom at the level of the regularized theories.
Instead of fixing gauge invariance on the level of the regularized theory, our ansatz is to retain strong continuity of the approximate gauge transformations on the lattice.
On the one hand, this makes it harder to recover gauge invariance in the continuum limit.
On the other hand, we expect this to facilitate the inheritance of the strong continuity property in the continuum limit.
Hence, we find it plausible to assume that our method may, in principle, facilitate the attainment of an unprecedented achievement: a strongly continuous representation of the group of spatial diffeomorphisms on a continuum Hilbert space. Such a milestone would mark significant progress in addressing the challenging aspects of quantum gravity. 


\subsection*{Acknowledgements}
We would like to express our gratitude to Bianca Dittrich for providing helpful references.
In the end, we decided to defer these references to \textcite{gravity-continuum} where they thematically fit better.
Besides, we thank Thomas Thiemann for discussions and the Insitute for Quantum Gravity at FAU Erlangen--Nürnberg for their hospitality during the final stage of this work.

Research at Perimeter Institute is supported in part by the Government of Canada through the Department of Innovation, Science and Economic Development and by the Province of Ontario through the Ministry of Colleges and Universities.
