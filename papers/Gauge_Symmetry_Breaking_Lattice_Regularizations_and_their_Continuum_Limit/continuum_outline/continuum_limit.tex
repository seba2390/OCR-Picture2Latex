\section{The Continuum Limit}
\label{sec:The Continuum Limit}
Now that we have obtained quantized versions of our lattice theories, the logical next step is to study their continuum limit.
In the following, we will outline a method to define a continuum version of these theories.

First, we note that on every lattice Hilbert space $\mathcal H_n$, there exists a $C*$--algebra $W_n$ of Weyl elements, which is given by
\begin{equation}
    W_n = \overline{\mathrm{span}\Set{\e^{\im\hat\phi_n[f_n] + \im\hat\pi_n[g_n]} | (f_{nk})_k, (g_{nk})_k \in \mathbb R^{N_n}}} .
\end{equation}
We display the continuum Weyl algebra as the projective limit $W = \varprojlim W_n$, where the identifications
\begin{equation}
    \hat\phi_{n+1,2k}f_{n+1,2k} + \hat\phi_{n+1,2k+1}f_{n+1,2k+1} \equiv \hat\phi_{nk}(f_{n+1,2k} + f_{n+1,2k+1}),
\end{equation}
and similarly for $\hat\pi_{nk}$, are made.
We note that the Schrödinger representations on the lattice are irreducible representations of $W_n$, so every vector $\psi_n\in\mathcal H_n$ is a cyclic vector with respect to $W_n$.

In order to find a continuum limit, we need to find a sequence $(\psi_n)_n$ of normalized vectors $\psi_n\in\mathcal H_n$ such that the algebraic states
\begin{equation}
    \omega_n\left(\e^{\im\hat\phi_n[f_n] + \im\hat\pi_n[g_n]}\right) \coloneq \Braket{\psi_n, \e^{\im\hat\phi_n[f_n] + \im\hat\pi_n[g_n]} \psi_n} \label{eq:algebraic-states}
\end{equation}
converge to a state $\omega$ on $W$.
In order to check that, we need to show that the $\omega_n$ are a Cauchy sequence in the following sense:
Let $f$ be a measurable function on $\mathbb T$ and $(f_n)_n$ be a sequence of piecewise constant functions on the $n$--th lattice that converges to $f$.
Analogously, choose a sequence $g_n\to g$.
We say that the $\omega_n$ form a Cauchy sequence if for every $\epsilon>0$, there is $N>0$ such that for all $f_n\to f$, $g_n\to g$ and all numbers $n,m>N$, we have
\begin{equation}
    \abs*{\omega_n\left(\e^{\im\hat\phi_n[f_n] + \im\hat\pi_n[g_n]}\right) - \omega_m\left(\e^{\im\hat\phi_m[f_m] + \im\hat\pi_m[g_m]}\right)} < \epsilon .
\end{equation}
In that case, we define
\begin{equation}
    \omega\left(\lim_{n\to\infty} \e^{\im\hat\phi_n[f_n] + \im\hat\pi_n[g_n]}\right) \coloneq \lim_{n\to\infty} \omega_n\left(\e^{\im\hat\phi_n[f_n] + \im\hat\pi_n[g_n]}\right) .
\end{equation}
The continuum Hilbert space $\mathcal H$ and a cyclic representation of the continuum Weyl algebra $W$ with cyclic vector $\psi$ can then be reconstructed from $\omega$ using the GNS--construction.

We note that this way of obtaining a continuum theory is much simpler than the application of a renormalization group method \parencite[e.g.][]{Lang:2017beo}.
While a renormalization group flow does produce a converging sequence $(\omega_n)_n$ at the fixed point of the flow, the resulting sequence is a very complex object that is hard to obtain.
The fixed point sequence does not only converge, but it satisfies the additional property of cylindrical consistency, which means that the expectation value of a coarse Weyl element is the same on both a coarse lattice and a fine lattice.
This additional property is much stronger than mere convergence, but not strictly required for the definition of a theory in the continuum.
On the other hand, generating a convergent sequence by means of a renormalization group flow has the advantage of being quite systematic.

\subsection{Defining Convergent Sequences}
It can be difficult to choose a sequence $(\psi_n)_n$ of states that has a continuum limit.
In general, one may need to make use of the freedom to be able to choose the lattice approximations $f_n$ of the continuum test functions $f$.
Different choices may lead to different limits or may develop divergences.

We want to illustrate one particularly simple approach to obtaining such a sequence.
Let us choose an arbitrary state $\psi_{m} \in \mathcal H_{m}$.
We iteratively define a sequence of fine states $(\psi_n)_{n\geq m}$ from $\psi_{m}$ as follows:
\begin{equation}
    \psi_{n+1}((\phi_{n+1,k})_k) \coloneq 2^{-2^{n-1}} \psi_n((\phi_{nk}^+)_k) \prod_{k=1}^{N_n} \rho_{n}(\phi_{nk}^-),
\end{equation}
where
\begin{equation}
    \phi_{nk}^+ = \frac{\phi_{n+1,2k} + \phi_{n+1,2k-1}}{2}, \quad \phi_{nk}^- = \frac{\phi_{n+1,2k} - \phi_{n+1,2k-1}}{2}. \label{eq:change-of-variables}
\end{equation}
Analogously, we define $\pi_{nk}^\pm$.
The functions $\rho_n$ are arbitrary, yet to be chosen functions of one variable.
They quantify the distribution of the local fluctuations on each scale.

By expanding the canonical variables in $\hat\phi_{nk}^\pm$ and $\hat\pi_{nk}^\pm$, where $\commutator{\hat\phi_{nk}^\pm}{\hat\pi_{nk'}^\mp} = 0$, we can evaluate the algebraic states $\omega_n$ (as defined in \cref{eq:algebraic-states}) on the corresponding Weyl elements.
Using the shorthand $f_{nk}^{n+1} = f_{n+1,2k-1}+f_{n+1,2k}$, we find:
\begin{align}
    \MoveEqLeft \phi_{n+1}[f_{n+1}]
         = \sum_{k=1}^{N_{n+1}} \phi_{n+1,k}f_{n+1,k} \nonumber \\
        &= \sum_{k=1}^{N_n}\phi_{nk}^+(f_{n+1,2k-1}+f_{n+1,2k}) + \sum_{k=1}^{N_n} \phi_{nk}^-(f_{n+1,2k}-f_{n+1,2k-1}) \nonumber \\
        &= \sum_{k=1}^{N_n}\phi_{nk}^+ f_{nk}^{n+1} + \eta_{n+1} \sum_{k=1}^{N_n} \phi_{nk}^- (\Delta f)^{n+1}_{n k} \nonumber \\
        &= \phi_n^+[f_n^{n+1}] + \eta_{n+1} \,\phi_{n}^- \left[ (\Delta f)^{n+1}_{n} \right],
\end{align}
where we defined
\begin{align}
    (\Delta f)^{n+1}_{n k} \coloneq \frac{1}{\eta_{n+1}} \left( f_{n+1, 2k} - f_{n+1,2k-1}\right).
\end{align}
A similar expression follows for $\pi_{n+1}[f_{n+1}]$.
Consequently, we find the following relation for the Weyl elements:
\begin{align}
    \e^{\im\hat\phi_{n+1}[f_{n+1}] + \im\hat\pi_{n+1}[g_{n+1}]}
        = \e^{\im\hat\phi_n^+[f_n^{n+1}] + \im\hat\pi_n^+[g_n^{n+1}]} \prod_{k=1}^{N_n} \e^{\im\eta_{n+1}\left(\hat\phi_{n}^- [(\Delta f)^{n+1}_{n }] + \hat\pi_{n}^- [(\Delta g)^{n+1}_{n }]\right)},
\end{align}
We note that all factors on the right hand side commute.
This allows us to evaluate the $\omega_{n+1}$ on the Weyl elements by a simple change of variables according to \cref{eq:change-of-variables}:
\begin{align}
    \omega_{n+1}(\e^{\im\hat\phi_{n+1}[f_{n+1}] + \im\hat\pi_{n+1}[g_{n+1}]})
        = \omega_n(\e^{\im\hat\phi_n[f_n^{n+1}] + \im\hat\pi_n[g_n^{n+1}]}) \prod_{k=1}^{N_n} \zeta_{n}((\Delta f)^{n+1}_{nk}, (\Delta g)^{n+1}_{nk}) .
\end{align}
Here, we defined
\begin{equation} \label{def:zeta_n}
    \zeta_{n}((\Delta f)^{n+1}_{nk}, (\Delta g)^{n+1}_{nk}) \coloneq \Braket{\rho_n, \e^{\im \eta_{n+1} (\hat\phi_{nk}^- (\Delta f)^{n+1}_{nk} + \hat\pi_{nk}^-(\Delta g)^{n+1}_{nk})} \rho_n}.
\end{equation}
It follows that the corresponding states $\omega_n$ for any $n>m$, evaluated on elements of the Weyl algebra, read
\begin{equation}
    \omega_n\left(\e^{\im\hat\phi_n[f_n] + \im\hat\pi_n[g_n]}\right) = \omega_{m}\left(\e^{\im\hat\phi_{m}[f^n_{m}] + \im\hat\pi_{m}[g^n_{m}]}\right) \prod_{l=m}^{n-1} \prod_{k=1}^{N_l} \zeta_l ((\Delta f)^n_{m k}, (\Delta g)^n_{m k}),
\end{equation}
where we use that the components of the vectors $f^n_{m}$ and $(\Delta f)^n_m$ are respectively given by
\begin{align}
    f^n_{m k} &= \sum_{j=1}^{N_n N_{m}^{-1} } f_{n, (k-1) N_n N_{m}^{-1} +j}, \\
    (\Delta f)^n_{m k} &= \sum_{j=1}^{N_n N_{m}^{-1}} f_{n, (k-1) N_n N_{m}^{-1} +j} \cdot \sgn (2j -1- N_n N_{m}^{-1}),
\end{align}
where $k =1,\dots, N_{m}$. Similar expressions hold for $g^n_{m}$ and $(\Delta g)^n_m$.
If the $\rho_l$ are chosen appropriately, the product of the $\zeta_l$ converges and a continuum limit exists.

Possible examples that lead to convergence are given by Dirac sequences with suitable speed of convergence.
A Dirac sequence of Gaussian distributions will lead to a Gaussian state in the continuum and thus to a Fock space representation.
Dirac sequences with compact support can be used to constrain the admissible states to those with support on configurations of bounded variation.
Such states are interesting, because functions of bounded variation are differentiable almost everywhere, which can aid the convergence of finite differences $\Delta^{\eta_n} \hat\phi_{nk}$ to derivatives $\partial\hat\phi(x)$ of field operators.

\subsection{Representation of the Group of Gauge Transformations}
Now that we have a continuum Hilbert space, we would like to implement gauge transformations on it.
Unfortunately, the operators $U_{D_n[f_n]}(s) W_n U_{D_n[f_n]}(s)^\dagger$ are not necessarily Weyl elements again.
Thus, implementing the gauge transformations on the continuum Hilbert space is not as easy as taking the limit of the transformed Weyl elements in the projective limit algebra $W$.
Nevertheless, we can describe a plausible procedure:

Suppose we want to apply a gauge transformation $U_{D[f]}(s)$ on a state $\psi \in \mathcal H$ in the continuum Hilbert space.
We have to choose a sequence $(\psi_n)_n$ in the lattice Hilbert spaces $\mathcal H_n$ that converges to $\psi$ in the sense of algebraic states.
Moreover, we choose a sequence $f_n$ of piecewise constant functions with $f_n\to f$.
We can then apply the approximate gauge transformations $U_{D_n[f_n]}(s)$ to $\psi_n$.
Since every state in $\mathcal H_n$ is cyclic for $W_n$, we can expand the transformed state in every lattice Hilbert space again in terms of Weyl elements:
\begin{equation}
    U_{D_n[f_n]}(s)\psi_n = \sum_k c_{nk} \e^{\im\hat\phi_n[f_{nk}] + \im\hat\pi_n[g_{nk}]} \psi_n
\end{equation}
We can then use this sequence of Weyl elements to define the continuum gauge transformation as follows:
\begin{equation}
    U_{D[f]}(s)\psi = \lim_{n\to\infty}\sum_k c_{nk} \e^{\im\hat\phi_n[f_{nk}] + \im\hat\pi_n[g_{nk}]} \psi.
\end{equation}
