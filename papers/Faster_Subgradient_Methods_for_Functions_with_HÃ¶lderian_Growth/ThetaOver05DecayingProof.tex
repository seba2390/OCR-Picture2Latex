The recursion describing the subgradient method is, for $k\geq 1$,
\begin{eqnarray}\label{recurting}
e_{k+1}\leq e_k - 2\alpha_k c e_k^{\gamma}+\alpha_k^2 G^2,
\end{eqnarray}
where $e_k=d(x_k,\calX)^2$ and $\gamma=\frac{1}{2\theta}$.
Let $\alpha_k = \alpha_1 k^{-p}$. We wish to prove that if 
$$
p=\frac{\gamma}{2\gamma-1}
$$
and the constant $\alpha_1$ is chosen as in (\ref{alph1Choice}), then
\begin{eqnarray}\label{inductiveHypothesis}
e_k \leq C_e k^{-b}
\end{eqnarray}
 where 
$$b\triangleq\frac{p}{\gamma}=\frac{1}{2\gamma-1},$$ 
for all $k\geq k_0 \triangleq \lceil2 b\rceil$, and $C_e$ is given by $C_e=(\kappa^2 b)^b$.
%Note that $p=\gamma b$. 

We will prove this result by induction. The initial condition is
\begin{eqnarray*}
e_{k_0}\leq C_e k_0^{-b}
\end{eqnarray*}
which is implied by
\begin{eqnarray}\label{C_initCond}
\Omega_{\calC}
\leq
C_e k_0^{-b}
\iff
C_e = (\kappa^2 b)^b \geq \Omega_{\calC} k_0^b.
\end{eqnarray}
Since $k_0=\lceil 2b \rceil\leq 2b+1\leq 3b$, this is implied by
\begin{eqnarray*}
\left(
b \kappa^2\right)^{b}
\geq 
\Omega_{\calC}(3b)^b.
\end{eqnarray*}
Dividing by $b^b$ and taking the $b$th root yields 
\begin{eqnarray*}
\kappa^2 \geq 3\Omega_{\calC}^{\frac{1}{b}},
\end{eqnarray*}
which is (\ref{kappaCond}).

Next, assume (\ref{inductiveHypothesis}) is true for some $k\geq k_0$. That is, assume $e_k = a C_e k^{-b}$, where $0\leq a \leq 1$. We will show that this implies $e_{k+1}\leq C_e (k+1)^{-b}$. Substituting $e_k = a C_e k^{-b}$ and $\alpha_k=\alpha_1 k^{-p}$ into the right hand side of (\ref{recurting}) yields
\begin{eqnarray}
e_{k+1}
&\leq&
a C_e k^{-b}-2 \alpha_1 c a^\gamma C_e^\gamma k^{-(p+\gamma b)}+\alpha_1^2 G^2 k^{-2p} 
\nonumber\\\nonumber
&=&
a C_e k^{-b}
+
\left(
\alpha_1^2 G^2
-
2 \alpha_1 c a^\gamma C_e^\gamma
\right)
k^{-2p}
\end{eqnarray}
using the fact that $p+\gamma b=2 p$. Thus we wish to enforce the inequality:
\begin{eqnarray}
\label{TheIn}
a C_e k^{-b}
+
\left(
\alpha_1^2 G^2
-
2 \alpha_1 c a^\gamma C_e^\gamma
\right)
k^{-2p}
\leq
C_e (k+1)^{-b}.
\end{eqnarray}
We need (\ref{TheIn}) to hold for all $a\in[0,1]$. Since $\frac{1}{2}\leq\theta<1$, $\frac{1}{2}<\gamma\leq 1$, therefore the L.H.S. is a convex function of $a$ for $a\geq 0$. Therefore if the inequality holds for $a=0$ and $a=1$, then it holds for all $a\in[0,1]$. 

Consider first, $a=0$. The condition is
\begin{eqnarray*}
\label{alphaCondFirst}
\alpha_1^2 G^2
k^{-2\gamma b}
\leq
C_e (k+1)^{-b}.
\end{eqnarray*}
This is equivalent to
\begin{eqnarray}\label{alphaCond}
\alpha_1 
\leq
G^{-1}C_e^{\frac{1}{2}} k^{\gamma b} (k+1)^{-\frac{b}{2}}.
\end{eqnarray}
Note that $\alpha_1$, given in (\ref{alph1Choice}), can be rewritten as
\begin{eqnarray*}
\alpha_1 = \frac{c C_e^\gamma}{G^2}.
\end{eqnarray*}
 Substituting $\alpha_1$ into (\ref{alphaCond}) yields
\begin{eqnarray*}
\frac{c}{G^2}C_e^\gamma \leq 
G^{-1}C_e^{\frac{1}{2}}k^{\gamma b} (k+1)^{-\frac{b}{2}}
\end{eqnarray*}
which can be rearranged to
\begin{eqnarray}
G\geq
c
C_e^{\gamma-\frac{1}{2}}k^{-\gamma b} (k+1)^{\frac{b}{2}}.\label{love}
\end{eqnarray}
Now 
\begin{eqnarray*}
C_e^{\frac{2\gamma-1}{2}} 
=
\kappa\sqrt{b}.
\end{eqnarray*}
Substituting this into (\ref{love}) yields
\begin{eqnarray}
k^{\gamma b} (k+1)^{-\frac{b}{2}}\geq 
\sqrt{b}.\label{sxx}
\end{eqnarray}
Now
\begin{eqnarray*}
(k+1)^{-\frac{b}{2}}
&=&
k^{-\frac{b}{2}}(1+k^{-1})^{-\frac{b}{2}}
\\
&\geq&
k^{-\frac{b}{2}}
\left(1
-\frac{b}{2}k^{-1}
\right)
\\
&=&
k^{-\frac{b}{2}}
-\frac{b}{2}k^{-\frac{b}{2}-1}.
\end{eqnarray*}
Therefore (\ref{sxx}) is implied by
\begin{eqnarray*}
k^{b(\gamma-\frac{1}{2})}
-
\frac{b}{2}
k^{b(\gamma-\frac{1}{2})-1}
\geq
\sqrt{b}.
\end{eqnarray*}
Now substituting $b=(2\gamma-1)^{-1}$ into the two exponents yields
\begin{eqnarray*}
k^{\frac{1}{2}}-\frac{b}{2}k^{-\frac{1}{2}}\geq\sqrt{b}
\end{eqnarray*}
which is equivalent to
\begin{eqnarray*}
t^2 - \sqrt{b} t - \frac{b}{2}\geq 0 
\end{eqnarray*}
with the substitution $t=\sqrt{k}$. Thus we require
\begin{eqnarray*}
t
\geq
\frac{1+\sqrt{3}}{2}\sqrt{b}
\end{eqnarray*}
which is implied by $k\geq 2 b$. Thus $k\geq \lceil2 b\rceil$ implies (\ref{TheIn}) holds with $a=0$.


Now consider $a=1$ in (\ref{TheIn}). We again simplify (\ref{TheIn}) using
\begin{eqnarray*}
C_e(k+1)^{-b}
=
C_e k^{-b}(1+k^{-1})^{-b}
\geq
C_e k^{-b} - b C_e k^{-(b+1)}.
\end{eqnarray*}
Therefore in the case $a=1$, (\ref{TheIn}) is implied by 
\begin{eqnarray}
\left(
\alpha_1^2 G^2
-
2 \alpha_1 c C_e^\gamma
\right)
k^{-2p}
\leq
 - b C_e k^{-(b+1)}\label{TheIn2}.
\end{eqnarray}
Now $2p=b+1$, therefore (\ref{TheIn2}) is equivalent to
\begin{eqnarray*}
\alpha_1^2 G^2
-
2 \alpha_1 c C_e^\gamma+b C_e
\leq
0
\end{eqnarray*}
for all $k\geq 1$. The L.H.S. is a positive-definite quadratic in $\alpha_1$. Solving it yields the two solutions
\begin{eqnarray*}
\frac{
2c C_e^\gamma \pm \sqrt{4c^2C_e^{2\gamma}-4 G^2 b C_e}
}{2 G^2}.
\end{eqnarray*}
The quadratic has a real solution if
\begin{eqnarray}
4c^2C_e^{2\gamma}-4 G^2 b C_e
\geq 
0
\iff
C_e
\geq
\left(
\kappa^2 b\right)^{b}.
\label{CeLowerBound}
\end{eqnarray}
Thus since $C_e=(\kappa^2 b)^b$, the only valid choice for $\alpha_1$ is
\begin{eqnarray*}\label{AlphaVal}
\alpha_1 = \frac{c C_e^\gamma}{G^2}
\end{eqnarray*}
which corresponds to (\ref{alph1Choice}).
This completes the proof.