\section{Program Transformations}
\label{sec:appendix-transformations}

In this section, we supplement the discussion in
\autoref{sec:little-transformations} of transformations currently implemented in
\deuce{}.

\subsection{General-Purpose Transformations}

\begin{figure*}[b]
\small

\newcommand{\moreName}[1]{\textit{#1}}

\beginToolTable
\toolTableHeaders
\toolCategoryReuse
\rowTool
  {Create Function from Definition}
  {0}{1}{1}{0}
  {all constants or only named constants}
\rowTool
  {Create Function from Arguments}
  {1+}{0}{0}{0}
  {}
%% \rowTool
%%   {Abstract}
%%   {0}{1}{0}{0}
%%   {all constants or only named constants}
%% \rowTool
%%   {Abstract}
%%   {0}{0}{1}{0}
%%   {all constants or only named constants}
\rowTool
  {Merge}
  {2+}{0}{0}{0}
  {}
\hline
\toolCategoryStyle
\rowTool
  {Move Definitions}
  {0}{1+}{1+}{1}
  {renaming; dep. lifting; dep. inversion}
\rowTool
  {Swap Definitions}
  {0}{2}{2}{0}
  {}
%% \rowTool
%%   {Move Definitions}
%%   {0}{1+}{0}{1}
%%   {renaming; dep. lifting; dep. inversion}
%% \rowTool
%%   {Move Definitions}
%%   {0}{0}{1+}{1}
%%   {renaming; dep. lifting; dep. inversion}
\rowTool
  {Introduce Variable}
  {1+}{0}{0}{1}
  {}
\rowTool
  {Add Arguments}
  {1+}{0}{0}{1 in arg list}
  {}
\rowTool
  {Remove Arguments \moreName{at function definition}}
  {0}{1+ arg}{0}{0}
  {}
\rowTool
  {Remove Arguments \moreName{at function call}}
  {1+ arg}{0}{0}{0}
  {}
\rowTool
  {Reorder Arguments \moreName{at function definition}}
  {0}{1+}{0}{1 in arg list}
  {}
\rowTool
  {Reorder Arguments \moreName{at function call}}
  {1+}{0}{0}{1 in arg list}
  {}
\rowTool
  {Reorder List Items}
  {1+}{0}{0}{1 in parent list}
  {}
%% \rowTool
%%   {Eliminate Common Subexpression}
%%   {1+}{0}{0}{0}
%%   {}
\rowTool
  {Rename Variable \moreName{at definition}}
  {0}{1}{0}{0}
  {new name}
\rowTool
  {Rename Variable \moreName{at use}}
  {1 variable}{0}{0}{0}
  {new name}
\rowTool
  {Swap Variable Names and Usages}
  {0}{2}{0}{0}
  {}
\rowTool
  {Inline Definition}
  {0}{1+}{0}{0}
  {}
\rowTool
  {Duplicate Definition}
  {0}{1+}{0}{1}
  {}
\rowTool
  {Clean Up}
  {0}{0}{0}{0}
  {}
\rowTool
  {Make Single Line}
  {1}{0}{0}{0}
  {}
\rowTool
  {Make Multi-Line}
  {1}{0}{0}{0}
  {}
\rowTool
  {Align Expressions}
  {2+}{0}{0}{0}
  {}
\hline
\toolCategoryChange
\rowTool
  {Make Equal with Single Variable}
  {2+ constants}{0}{0}{0}
  {choose suggested name}
\rowTool
  {Make Equal by Copying}
  {2+}{0}{0}{0}
  {choose expression to copy}
\rowTool
  {Reorder Expressions}
  {1+}{0}{0}{1}
  {}
\rowToolNoBreak
  {Swap Variable Usages}
  {0}{2}{0}{0}
  {}
\end{tabular}

\caption{General-purpose transformations in \deuce{}.}
\label{fig:transformations-1}
\end{figure*}


\autoref{fig:transformations-1} shows a list of the general-purpose code tools
in our implementation, along with the number of selected code items and target
positions required to make the tools \verb+Active+.

\parahead{\codeTool{Move Definitions}}

A common, mundane text-editing task is to rearrange definitions. While
conceptually simple, there are several aspects that are subject to
error, such as making sure not to break dependencies and making sure not to break
scoping. Furthermore, particularly in functional languages where
local-bindings can be arbitrarily nested and where tuples can be used to
simultaneously define multiple bindings, there are many stylistic
reasons for arranging definitions in a certain way. These choices are
often in flux during the prototyping and repairing process, where
definitions may be reordered to more clearly explain program
dependencies and to ensure that the concrete layout fits nicely within
the screen width.

The \codeTool{Move Definitions} transformation takes a set of selected patterns
and a single target position, and attempts to move the pattern and its
definition to the target position. If the target position is
an expression, a new let-binding is added to surround the target.
Whitespace is added or removed to match the indentation of the target
scope. If the target position already defines a
list pattern, then the selected definitions are inserted into the list.
If the target position defines a single variable,
then a list pattern is created.

There are three cases in which \deuce{} provides the user options for
how to correct an otherwise invalid transformation.
In all cases, the user may choose the original
invalid option since breaking the code temporarily may be the
intention during the course of prototyping and repairing.
%
Note that the screenshots in \autoref{fig:move-def} show a preliminary
implementation of \deuce{}, with cosmetic differences in the user interface.

\begin{figure*}[t]

\newcommand{\scaleABD}{0.29}
\newcommand{\scaleC}{0.41}
\newcommand{\vsepFigure}{\vspace{0.08in}}

\begin{subfigure}{\textwidth}
  \centering
  \includegraphics[scale=\scaleABD]{move-def-capture-before-after.png}
  \caption{Simply moving the definition of \texttt{x} on line 4 to line 2 would
change the binding structure of the program. In addition to this
option, so the second and third options use renaming to
preserve the binding structure of the original program.}
  \label{fig:move-def-a}
\end{subfigure}

\vsepFigure

\begin{subfigure}{\textwidth}
  \centering
  \includegraphics[scale=\scaleABD]{move-def-lift-before-after.png}
  \caption{Simply moving the definition of \texttt{a3} above \texttt{b} would
result in a use of \texttt{a2} before it is defined, so the second
option additionally lifts the definition of \texttt{a2}.}
  \label{fig:move-def-b}
\end{subfigure}

\vsepFigure

\begin{subfigure}{\textwidth}
  \centering
  \includegraphics[scale=\scaleC]{move-def-invert-before-after-2.png}
  \caption{For arithmetic constraints among expressions---such as the
top-left corner, width, height, and bot-right corner of a rectangle---
\codeTool{Move Definitions} can move definitions above dependencies by rewriting
the expressions.}
  \label{fig:move-def-c}
\end{subfigure}

\vsepFigure

\begin{subfigure}{\textwidth}
  \centering
  \includegraphics[scale=\scaleABD]{move-pat-def-before-after.png}
  \caption{When selecting the patterns \texttt{c} and \texttt{d} and
moving them before \texttt{b}, they are put into a single definition.
When selecting the definitions \texttt{c} and \texttt{d}, the separate
definitions are preserved in the new location.}
  \label{fig:move-def-d}
\end{subfigure}

\caption{Examples demonstrating four configuration options for
\codeTool{Move Definition} transformation.}
\label{fig:move-def}
\end{figure*}



\paragraph{Option: Renaming to Preserve Binding Structure}

One issue is that the binding structure may change. For example, in
\autoref{fig:move-def-a}, the uses of \verb+x+ in the expression
\verb&(+ x 1)& on lines 2 and 5 resolve to different definitions of
\verb+x+, on lines 1 and 4, respectively. Moving the definition of
\verb+x+ from line 4 before \verb+y1+ will result in a program that
evaluates safely but with different binding structure.  In this case,
the \codeTool{Move Definition} transformation provides several options:
the unsafe option that performs the transformation without renaming
which allows a binding to be captured, and two safe options that
rename either of the definitions to avoid capture.


\paragraph{Option: Lifting Dependencies}

A second potential issue is that the definitions would be moved before
its dependencies. In the example in \autoref{fig:move-def-b},
\verb+a2+ is defined to be \verb+a1+, and \verb+a3+ is defined to be
\verb+a2+. Trying to move \verb+a3+ above \verb+b+ is not safe,
because \verb+a2+ is not in scope. The transformation provides two
options: the unsafe transformation, and
a version where the dependency, \verb+a2+, is automatically
moved as well to make the original transformation safe.


\paragraph{Option: Inverting Dependencies}

A third situation is when the user may want to rewrite definitions so
that a dependency violation can be avoided. In the top-half of
\autoref{fig:move-def-c}, the program defines the top-left point
(abbreviated to \verb+tl+ for space) to be \verb+(100,100)+ and the
\verb+width+ and \verb+height+ to be \verb+100+ and \verb+200+,
respectively; the location of the bottom-right point (abbreviated to
\verb+br+) is derived in terms of these parameters. Dragging the
third definition above the second results in \verb+brx+ and \verb+bry+
being rewritten to constants that the previous expressions evaluated
to (\ie{} \verb+200+ and \verb+300+),
and then \verb+width+ and \verb+height+ are rewritten to arithmetic
expressions that preserve the original relationship
(\ie{} \verb&(- brx tlx)& and \verb&(- bry tly)&). In situations
where there are constraints among expressions---particularly common in
programming domains that generate visual output, such as
a web application or data visualization---this transformation
allows the programmer to vary the choices about which parameters are
defined first in the program with constants and which are derived
in terms of them. The bottom half
of \autoref{fig:move-def-c} shows how the new definitions
can also be inverted, returning the program to the original.


\paragraph{Option: Flattening or Preserving Definition Structure}

As listed in \autoref{fig:transformations-1}, \codeTool{Move Definition} can be
initiated by selecting either one or more patterns (\eg{} just the
variables \verb+c+ and \verb+d+ in the top half of \autoref{fig:move-def-d})
or one or more
definitions (\eg{} the definitions \verb+(def c "c")+ and \verb+(def d "d")+
in the bottom half of \autoref{fig:move-def-d}).
In the former case, the selected patterns are moved into the
same list pattern in the target position; in the latter case, separate
patterns are kept separate. This is useful for preserving stylistic
choices of the definitions (\eg{} the line length of each definition)
as well as semantic properties (\eg{} dependencies between the
selected definitions).


\parahead{\codeTool{Make Equal by Copying}}

When selecting multiple expressions, this transformation copies one of the
expressions in place of all others.  Unlike the case where all selected
expressions are constant, the key choice here is which expression to use, so the
user is asked to choose which one to replace the others with.


\parahead{\codeTool{Create Function by Merging Definitions}}

While prototyping, it is often convenient to copy-paste code and then
make changes to the different clones as needed. Afterwards, it may be
desirable to pull out the common code between the clones into a single function.
The \codeTool{Create Function by Merging Definitions}
tool takes multiple selected expressions and attempts to abstract them over their syntactic
differences: any differing subtrees become arguments to a new function inserted into the program.
The selected expressions are rewritten as calls to this function.
To avoid suggesting unhelpful small abstractions, the \codeTool{Create Function
by Merging Definitions} tool is displayed only if
the resulting function is larger than a threshold (more precisely, if the number of AST nodes in the function
body is at least double the number of arguments to the function).


\parahead{\codeTool{Create Function from Definition}}

The \codeTool{Create Function from Definition}
tool transforms a selected expression into a lambda abstracted
over constants within the expression body. The transformation provides
two choices: (1) abstracting all constants, or (2) abstracting
constants that are immediately let-bound to variables.
The latter is a (simplified version of a) heuristic proposed by
\citet{sns-uist}. We could add configuration options to ask the user about
all potential constants to abstract; to keep this process lightweight,
however, we propose only the two parameterizations and then allow the user to
modify the result with tools for arguments (below).
After the expression is rewritten, the variable it is
bound to is tracked so that its uses can be rewritten to calls to the
new lambda, with the constants that had been pulled out of the
definition.


\parahead{\codeTool{Add, Remove, or Reorder Arguments}}

The tools to \codeTool{Add, Remove, or Reorder Arguments} allow the interface of a function to be
changed. Arguments may be added to a function by selecting expressions within a lambda and
a target position in the argument list. The \codeTool{Remove} and \codeTool{Reorder Argument} tools allow
the modification to be specified either at the argument list of the function definition or
at a call-site of the function. All three transformations require call-sites to be updated in sync.
Currently, we use a simple
static analysis to track when a lambda is let-bound to a variable.
If the lambda ever escapes this simple syntactic discipline, then we
cannot guarantee that all function calls are rewritten appropriately.
In which case, the transformation is marked as potentially unsafe
(\ie{} yellow). As in other
cases with unsafe transformations, the user must rely on other
mechanisms to ensure correctness (\eg{} types, tests, viewing the output,
or code review).


\parahead{Reorder Items}

This transformation allows one or more (potentially non-consecutive)
items to be removed and inserted elsewhere in the same list. The
whitespace between each pair of consecutive elements is preserved in
the transformed list, a detail that can often be tedious to ensure
with manual text-edits.


\parahead{\codeTool{Rename Variable}}

A transformation commonly found in IDEs
is to \codeTool{Rename} a variable and all its uses. In \deuce{}, the variable to be renamed may
be selected either at
its definition or at one of its usage sites. In either case, as the user
types the new name it is checked to ensure the name will not introduce
collisions; if it would, the transformation is marked as
unsafe (\ie{} yellow) to indicate that a different name may be desired.


\parahead{\codeTool{Swap} Tools}

\codeTool{Swap Variable Names and Usages} can be used to correct the names
chosen for two variables. The alternative is to perform the sequence
of \codeTool{Renaming} the first variable to a temporary, \codeTool{Renaming} the second to
the first, and \codeTool{Renaming} the temporary to the second.
A related transformation, \codeTool{Swap Variables Usages}, is handy for when the
definitions were correct,
but their usages are the opposite of what is intended (\eg{} mixing up
\verb+width+ and \verb+height+ values).


\parahead{\codeTool{Duplicate Definition}}

Text-based copy-paste works especially well when entire, adjacent
lines are copied. For expressions with smaller delimitations (within a
single line) or larger, ``jagged'' ones (different positions across multiple
lines), text-based selection may be more cumbersome.
The \codeTool{Duplicate Definition} transformation is a mouse-based alternative
for copy-pasting an expression to a different target position.

\parahead{\codeTool{Inline Definition}}

This transformation replaces all uses of a selected variable with its
definition. For convenience, multiple different variables can be selected and
inlined simultaneously. As any definition being inlined may itself have
variables, if any such variables are accidentally captured at their new
locations then capture avoiding renamings are offered.


\parahead{\codeTool{Clean Up; Make Single Line; Make Multi-Line; Align Expressions}}

All transformations in \deuce{} attempt to handle whitespace reasonably. Even so,
occasionally a transformation or series of transformations
will result in a program with, for example, a long line of code.
We supplement the whole-program \codeTool{Clean Up} tool of \cite{sns-uist}
with whitespace reformatting rules to break long definitions into
multiple lines and ensure that any multi-line definition is comfortably padded
by a blank line before and after. This is currently the only transformation that
requires no selected items---it applies to the entire program.
We also implement several transformations to help format selected
expressions, by adding or removing line breaks and indentation.

\subsection{Domain-Specific Transformations}
\label{sec:little-transformations-sns}

To demonstrate how our approach can be instantiated with custom
structured transformations, we have implemented several
transformations (in addition to the general-purpose ones)
that are specific to features in \sns{} and/or are particularly
useful for the domain of SVG graphics;
these are summarized in \autoref{fig:transformations-2}.

\parahead{\codeTool{Thaw/Freeze Number; Add/Remove Range; Show/Hide Slider; Flip Boolean}}


Numeric constants in \little{} can be annotated with an optional
range---written \verb+15{1-30}+---to instruct the \sns{}
editor to display a slider in the
output to make it easy to change the number without text-editing.
This feature is an example of ``scrubbing'' constants, a live
programming feature described by Bret Victor
(\url{http://worrydream.com/#!/DrawingDynamicVisualizationsTalk})
and Sean McDirmid
(\url{https://www.youtube.com/watch?v=YLrdhFEAiqo}).
We made one minor addition to the \little{} language: the option to
mark a range annotation as hidden---written
\verb@15{1-30,"hidden"}@---to keep the range information in the
program while suppressing the slider widget in the output.

In \deuce{}, the \codeTool{Add/Remove Range} tool operates on constant literals to attach
or remove these range annotations. New minimum and maximum range
values are determined based on the current value.
The \codeTool{Show/Hide Sliders} tool annotates a
range to be \verb+15{1-30,"hidden"}+, which keeps the range in the
text but suppresses the slider from the output. This tool makes it
quicker to toggle the sliders on and off (and, furthermore, preserves
the possibly-edited min- and max-values to remain in the text).
\sns{} allows numbers to be frozen with the annotation \verb+15!+,
which tells \sns{} not to change this value in response to changes to
the output. (Compared to the discussion of the \codeTool{Create Function from
Definition} tool,
the heuristic for arguments is to choose named \emph{and unfrozen}
constants.)
As with sliders, it can be tedious to use text-editing to
toggle this annotation on and off, so the \codeTool{Thaw/Freeze Constants} tool
provides an alternative.
%%
The \codeTool{Flip Boolean} tool is a quick way to ``scrub'' boolean literals.


\begin{figure*}[b]
\small

\beginToolTable
\toolTableHeaders
%% \rowTool
%%   {Invert Definitions}
%%   {0}{2}{0}{0}
%%   {}
%% \rowTool
%%   {Invert Definitions}
%%   {0}{0}{2}{0}
%%   {}
%% \rowTool
%%   {\rkc{Move Definition}}
%%   {0}{0}{1}{0}
%%   {...; Inversion}
\toolCategoryStyle
\rowTool
  {Thaw/Freeze Number}
  {1+}{0}{0}{0}
  {}
\rowTool
  {Add/Remove Range}
  {1+}{0}{0}{0}
  {}
\rowTool
  {Show/Hide Slider}
  {1+}{0}{0}{0}
  {}
\rowTool
  {Rewrite as Offset}
  {1+}{1}{0}{0}
  {}
\rowTool
  {Convert Color String}
  {1+ strings}{0}{0}{0}
  {RGB or Hue}
\hline
\toolCategoryChange
\rowToolNoBreak
  {Flip Boolean}
  {1+ booleans}{0}{0}{0}
  {}
\end{tabular}

\caption{Domain-specific transformations in \deuce{}.}
\label{fig:transformations-2}
\end{figure*}



\parahead{\codeTool{Rewrite as Offset}}

\citet{sns-uist} point out that there are often (at least) three
different common ways to programmatically describe a positional
attribute of a shape: with constants, with constant offsets from an
anchor point, or with constant relative percentages with respect to a
bounding box. When prototyping, it is often easiest to start by
using constants and then later switching to one of the relative versions.
The \codeTool{Rewrite as Offset} tool converts one or more selected constants
into offsets from a selected value. For example, in the expression
\verb+(let x 10 15)+, rewriting \verb+15+ as an offset from \verb+x+
transforms the program to \verb&(let x 10 (+ x 5))&. As with several
other structured transformations we have discussed, this transformation is
conceptually simple but can become tedious and error-prone to perform
with manual text-editing when
multiple numbers scattered across the program need to be offset and when
the base values are not easy-to-remember numbers.


\parahead{\codeTool{Convert Color String}}

In \sns{}, as in HTML and SVG, colors can be specified in a variety of ways: RGBA
codes, HSL codes, HEX, and color strings. The \sns{} editor provides
special support for color attributes defined as ``color numbers''
which are are (essentially) just the hue component of an HSL
triple; in particular, the editor displays a slider next to the shape to
control this color value. The \codeTool{Convert Color String} tool converts
color names (often useful for prototyping) into corresponding numbers, to enable the direct
manipulation support for color numbers that \sns{} provides. This
transformation is an example of how a custom program transformation can
be used to complement other features provided by the IDE.

