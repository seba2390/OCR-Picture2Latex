\parahead{Implementation in \sns{}}

We have chosen to implement our design within \sns{}~\citep{sns-pldi,sns-uist},
an interactive programming system for generating SVG images.
%
Whereas \sns{} provides capabilities for
directly manipulating the \emph{output} of a program, \deuce{}
provides capabilities for directly manipulating the
\emph{code} itself.

Direct code manipulation is particularly useful for a system like \sns{} for a couple reasons.
First, while the existing output-directed synthesis features in \sns{}
attempt to generate program updates that
are readable and which maintain stylistic choices in the existing code,
the generated code often requires subsequent edits, \eg{} to
choose more meaningful names, to rearrange definitions, and to
override choices made automatically by heuristics; \deuce{} aims to provide
an intuitive and efficient interface for performing such tasks.
Furthermore, by allowing users to interactively manipulate both code
and output, we provide another step towards the goal of \emph{direct
manipulation programming systems} identified by \citet{sns-pldi}.
These two capabilities---direct manipulation of code and
output---are complementary.

\sns{} is written in Elm (\url{http://elm-lang.org/}),
a language in which programs are
compiled to JavaScript and run in the browser.
The project uses the Ace text editor (\url{https://ace.c9.io/})
for manipulating \little{} programs.
(The second reason for the name \deuce{}
is that it extends Ace.)
We extended \sns{} to implement \deuce{};
our changes constitute approximately \locImplementation{} lines of
Elm and JavaScript code.
The new version (\version{0.6.2})
is available at \url{http://ravichugh.github.io/sketch-n-sketch/}.

