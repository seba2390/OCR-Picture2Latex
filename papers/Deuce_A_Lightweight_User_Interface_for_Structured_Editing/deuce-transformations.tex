\subsection{Program Transformations}
\label{sec:little-transformations}


\begin{wrapfigure}{r}{0pt}
\includegraphics[scale=\lambdaScreenshotScale]{all-code-tools-no-flip-bool.png}
\end{wrapfigure}
We have implemented a variety of program transformations, shown on the right.
While we believe these transformations form a
useful set of basic tools for common programming tasks, we
do not argue that these constitute a necessary or sufficient set.
One benefit of our design is that different sets of
transformations---such as
refactorings for class-based languages~\citep{Fowler1999},
refactorings for functional languages~\citep{Thompson2013},
transformations that selectively change program behavior~\citep{Reichenbach:2009}, and
task-specific transformations that do not have common, recognizable
names~\citep{Steimann:2012}---can be incorporated and
displayed to the user within our interface.

We limit our discussion below to transformations that
are not implemented in existing refactoring tools.
The \suppMaterials{} %% \autoref{sec:appendix-transformations}
describe
other transformations, but these details are not necessary to understand the rest of the
paper.

\parahead{\codeTool{Make Equal with Single Variable}}

When multiple constants and an optional target position are selected,
the \codeTool{Make Equal with Single Variable} transformation introduces a new
variable, bound to one of the constants, and replaces all the
constants with the new variable. This has the effect of changing the
program to make each of these values equal.
The transformation attempts to suggest meaningful names, based on how the
selected expressions appear in the program.
For Example 1 from \autoref{sec:overview}, because the numbers \verb+120+ and
\verb+80+ are passed as the fourth and fifth arguments, respectively, to the
function \verb+(def rect (\(fill x y w h) ...))+, the suggested names include
\verb+w+ and \verb+h+.
The user is asked to choose a name.
%
The value itself (in this case, \verb+120+ or \verb+80+) is not as
important---the intention is that the values vary at once by a single
change---so, to keep the number of \verb+Result+s small,
the transformation does not ask the user to choose
which value to use for the variable.


\parahead{\codeTool{Move Definitions}}

Because of nested scopes and simultaneous bindings (\ie{} tuples), there are
many stylistic choices about variable definitions when programming in
functional languages.
%
The \codeTool{Move Definitions} transformation takes a set of selected definitions %% patterns
and a single target position, and attempts to move the
definitions to the target position. If the target position is before
an expression, a new \verb+let+-binding is added to surround the target.
Whitespace is added or removed to match the indentation of the target
scope. If the target position already defines a
list pattern, then the selected definitions are inserted into the list.
If the target position defines a single variable,
then a list pattern is created.
In cases where the intended transformation would capture variable uses or move
definitions above their dependencies (errors that are easy to make when using
text-edits alone), the transformation makes secondary edits (alpha-renaming
variables and moving dependencies) to the program to avoid these issues.
Our implementation of \codeTool{Move Definitions} also provides options for whether or not to
collapse multiple definitions into a single tuple, and also provides support for
rewriting arithmetic expression definitions as an alternative way to deal with dependency
inversion issues.
