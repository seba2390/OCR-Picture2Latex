\section{User Study}
\label{sec:appendix-user-study}

We configured a pared down version of the system that turned off all \sns{}
features unrelated to the interactions being studied. This version also
contained a panel in the bottom-right corner with instructions about the current
task. To improve readability, the screenshots below show a lighter color theme
than used in the user study)
%
The user study version of the system, as well as a video that
demonstrates how to complete the following tasks, is available
at \url{http://ravichugh.github.io/sketch-n-sketch/blog/}.

\subsection{Tasks}

\newcommand{\taskScreenshotScale}{0.33}

\parahead{Head-to-Head Task: One Rectangle}

This task required swapping two arguments to a function (either with
Swap Expressions or Swap Variable Usages), and combining and reordering
five separate definitions into a single tuple definitions (requiring at least
two calls to Move Definitions).

\begin{figure}[h]
\center{\includegraphics[scale=\taskScreenshotScale]{task-one-rectangle-with-goals.png}}
\end{figure}

\parahead{Head-to-Head Task: Two Circles}

This task (described as Example 1 in \autoref{sec:overview}) required turning
a definition into a function (Create Function from Definition or Create Function
from Arguments) and then rearranging arguments (Reorder Arguments).

\begin{figure}[h]
\center{\includegraphics[scale=\taskScreenshotScale]{task-two-circles-with-goals.png}}
\end{figure}

\clearpage

\parahead{Head-to-Head Task: Three Rectangles}

This task required factoring three nearly-identical definitions into a helper
function (Create Function by Merging Definitions) and renaming the resulting
function (Rename Variable).

\begin{figure}[h]
\center{\includegraphics[scale=\taskScreenshotScale]{task-three-rectangles-with-goals.png}}
\end{figure}

\parahead{Head-to-Head Task: Four Circles (a.k.a. Target Icon)}

This task required removing a function argument (Remove Argument), renaming a
function argument (Rename Variable), moving a function definition inside another
(Move Definition), and adding several additional arguments to an existing
function (Add Arguments).

\begin{figure}[h]
\center{\includegraphics[scale=\taskScreenshotScale]{task-target-with-goals.png}}
\end{figure}

\clearpage

\parahead{Open-Ended Task: Four Squares}

This task required factoring four similar function calls into a helper function
(Create Function by Merging Definitions), 
creating a function (Create Function from Definition followed by Add Arguments,
or Create Function from Arguments), and renaming five variables (five uses of
Rename Variable).

\begin{figure}[h]
\center{\includegraphics[scale=\taskScreenshotScale]{task-four-squares-with-goals.png}}
\end{figure}


\parahead{Open-Ended Task: Lambda Icon}

This task is the same as Example 3 in \autoref{sec:overview}, not including the
last step to Move Definitions. That is, there are seven variables to define
(one using Introduce Variable and six with Make Equal with Single Variable).
These seven variables must be defined in two tuples, which can be accomplished
by Move Definitions but also without it if the previous tools were used with the
appropriate optional target positions.

\begin{figure}[h]
\center{\includegraphics[scale=\taskScreenshotScale]{task-lambda-with-goals.png}}
\end{figure}



\clearpage

\subsection{Exit Survey}

\setlength{\parindent}{0em}
\setlength{\parskip}{1em}

%% There are no penalties for skipping questions you do not want to answer.

\subsection*{Background Questions}

How many years of programming experience do you have?

\begin{itemize}
  \item Less than 1
  \item 1-2
  \item 3-5
  \item 6-10
  \item 11-20
  \item More than 20
\end{itemize}

How many years of functional programming experience do you have
(in languages such as Racket, Haskell, OCaml, Standard ML, or Elm)?

\begin{itemize}
  \item Less than 1
  \item 1-2
  \item 3-5
  \item 6-10
  \item 11-20
  \item More than 20
\end{itemize}


Do you use languages and tools that provide automated support
for refactorings or other program transformations?
(Examples include Java and Eclipse, Idris and its editor, etc.)
If so, please describe which tools and how often you use them.


Did have knowledge or hands-on experience with Sketch-n-Sketch
before this study?

\begin{itemize}
  \item Yes
  \item No
\end{itemize}

Did have knowledge or hands-on experience with the
Code Tools in Sketch-n-Sketch before this study?

\begin{itemize}
  \item Yes
  \item No
\end{itemize}

If you answered Yes to either question above,
please explain briefly.

\clearpage

\subsection*{Head-to-Head Comparisons on First Four Tasks}

\newcommand{\interactionA}{Text-Select Mode}
\newcommand{\interactionB}{Box-Select Mode}

\newcommand{\interactionSummary}{
\textbf{\interactionA{}:}
Text-selection + Code Tools menu or right-click Code Tools menu.\\
\textbf{\interactionB{}:}
Box-selection + pop-up Code Tools menu.
}

\interactionSummary{}

\newcommand{\sidebysidetaskquestions}[1]{

Which interaction worked better for the #1 task?

\begin{itemize}
  \item \interactionA{} worked much better
  \item \interactionA{} worked a little better
  \item They are about the same
  \item \interactionB{} worked a little better
  \item \interactionB{} worked much better
\end{itemize}

}
\sidebysidetaskquestions{One Rectangle}

\sidebysidetaskquestions{Two Circles}


\sidebysidetaskquestions{Three Rectangles}

\sidebysidetaskquestions{Target Icon}


Explain your answers to the four previous questions.
When did \interactionA{} work better and why?
When did \interactionB{} work better and why?

\clearpage

\subsection*{Questions about Final Two Tasks}

For the last two tasks, you were allowed to use both
\interactionA{} and \interactionB{}. The following
questions ask about this combination of features.
\\

\newcommand{\likertBox}[1]{\framebox[0.50in][c]{#1}}
\newcommand{\likertOptions}
  {\likertBox{1}\likertBox{2}\likertBox{3}\likertBox{4}\likertBox{5}}
\newcommand{\susQuestion}[1]{#1 & \likertOptions \\\\}

\begin{tabular}{p{3.0in}c}
\renewcommand{\arraystretch}{3.0}
& \textbf{Strongly} \hfill \textbf{Strongly} \\
& \textbf{Disagree} \hfill \textbf{Agree} \\[20pt]
\susQuestion{1. I think that I would like to use this system frequently.}
\susQuestion{2. I found the system unnecessarily complex.} \\
\susQuestion{3. I thought the system was easy to use.} \\
\susQuestion{4. I think that I would need the support of a technical person to be able to use this system.}
\susQuestion{5. I found the various functions in this system were well integrated.}
\susQuestion{6. I thought there was too much inconsistency in this system.}
\susQuestion{7. I would imagine that most people would learn to use this system very quickly.}
\susQuestion{8. I found the system very cumbersome to use.}
\susQuestion{9. I felt very confident using the system.} \\
\susQuestion{10. I needed to learn a lot of things before I could get going with this system.}
\end{tabular}

%% \clearpage


\newcommand{\fullmodequestion}[1]{

Did the Code Tools (either with Text-Select Mode
or Box-Select Mode) work well for the {#1} task?
If so, how? If not, why not?

}

\fullmodequestion{Four Squares}


\fullmodequestion{Lambda Icon}


%% \clearpage

\subsection*{Additional Questions}

What computer did you use?

\begin{itemize}
  \item My own personal laptop
  \item The laptop provided by the user study administrator
\end{itemize}

What improvements or new features would make Code
Tools in Sketch-n-Sketch better?

%% \clearpage

Are there any other comments about Code Tools in
Sketch-n-Sketch that you would like to share?



Are there other languages, application domains, or
settings where you would like to see the Code Tools
features?


\subsection{Additional Discussion}

\parahead{Within-Subjects Experimental Design}

One of our experiments measured which mode participants
preferred when given the ability to mix modes. This was
possible only because of the within-subjects design.
%
Our within-subjects design also enabled us to control for
each subject's measured skill level (via the random effect
for each participant in the mixed model). A between-subjects
design would require more participants to avoid an imbalance
of participant skill between the treatment groups.
%
On the other hand, a between-subjects design would be
simpler to interpret and would mitigate concerns about
learning effects between modes. Instead, we relied on the
mixed model to control for learning effects.

\parahead{System Usability Results}

In \autoref{sec:user-study-results}, we discussed the responses to
all survey questions except one (due to space constraints).
%
Our survey asked users to rate Combined Mode using the System
Usability Scale~\cite{SUS}. The score (mean: 63.9) was in the second
quartile (describable as between ``OK'' and ``Good'') compared to a corpus of
SUS evaluations~\cite{Bangor2009}. There was moderate correlation between
completion rate and SUS score (Pearson's r=0.54); the score among participants
who completed all tasks (extrapolated mean: 75.0) was in the third
quartile (describable as between ``Good'' and ``Excellent'').

