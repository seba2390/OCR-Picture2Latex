\parahead{\little{}}

\noindent
To make the discussion of our design concrete, we choose to work with
a small functional language called \little{}, defined in \autoref{fig:syntax}.
A \little{} program is a sequence of top-level definitions, the last of which is
called \verb+main+. Notice that all (sub)expressions, (sub)patterns, definitions
(both at the top-level and locally via \verb+let+), and branches of \verb+case+
expressions are surrounded in the abstract syntax by orange boxes; these denote
\emph{code items} that will be exposed for selection and deselection in the user
interface. In addition, there are \emph{target positions}, denoted by blue
dots, before and after every definition, expression, and pattern in the program. Target
positions are ``abstract whitespace'' between items in the abstract syntax tree,
which will also be exposed for selection.

%% \begin{figure}[h]
\begin{figure}[t]

\small

\centering

%% \judgementHeadNameOnly{Syntax of Expressions}

\vsepRule

\figSyntaxBegin

\figSyntaxRowLabel{Programs}{\varProgram}
  %% {\expTopDef{\varVar}{\varExp_\varVar}}^* \hspace{0.04in}
  %% \expTopDef{\mathit{main}}{\varExp}
  %% \targetBefore{\expDef{\varVar}{\varExp}{\varProgram}}
  %% \figSyntaxSpaceItem
  %% \targetBefore{\expTopDef{\mathtt{main}}{\varExp}}
  \targetBefore{\expDef{\varVar_0}{\varExp_0}{\targetsAround{\ \cdots\ }}}\
  \expTopDef{\mathtt{main}}{\varExp}
\figSyntaxSpaceNextCategory
\figSyntaxRowLabel{Expressions}{\varExpBare}
  \varConst
  \figSyntaxSpaceItem
  \varVar
  \figSyntaxSpaceItem
  \expFun{\varPat}{\varExp}
  \figSyntaxSpaceItem
  \expApp{\varExp_1}{\varExp_2}
  \figSyntaxSpaceItem
  \expCons{\varExp_1}{\varExp_2}
\figSyntaxLineBreak
\figSyntaxRow
  \expLet{\varPat}{\varExp_1}{\varExp_2}
  \figSyntaxSpaceItem
  \expCaseTwo{\varExp}{\expBranchWithId{\varExpId}{1}{\varPat_1}{\varExp_1}}{\cdots}
\figSyntaxSpaceNextCategory
\figSyntaxRowLabel{Patterns}{\varPatBare}
  \varConst
  \figSyntaxSpaceItem
  \varVar
  \figSyntaxSpaceItem
  \expNil
  \figSyntaxSpaceItem
  \expCons{\varPat_1}{\varPat_2}
\\[5pt]
\multicolumn{3}{c}{
  \mathrm{Expressions}\hspace{0.13in}\varExp\hspace{0.13in}{::=}\hspace{0.13in}\targetsAround{\wrapInBox{$\varExpBare$}}
  \hspace{0.30in}
  \mathrm{Patterns}\hspace{0.13in}\varPat\hspace{0.13in}{::=}\hspace{0.13in}\targetsAround{\wrapInBox{$\varPatBare$}}
}
%% \targetsAround{\wrapInBox{$\varExpBare$}_{\color{orange}\ \varExpId}}
%% \targetsAround{\wrapInBox{$\varPatBare$}_{\color{orange}\ \varExpId.\varPathId}}

\figSyntaxEnd

%% \vsepRule

\caption{Syntax of \little{}.
The orange boxes and blue dots \mbox{identify} features for structural selection.}
%% \caption{Syntax of \little{}. Expressions, patterns, definitions, and
%% let-bindings are all surrounded by orange boxes, to denote that they
%% can be selected in the user interface. The target positions, denoted
%% by orange circles, before and after expressions and patterns
%% can also be selected in the user interface.}
\label{fig:syntax}
\end{figure}


\parahead{Code Tool Interface}

\begin{figure}[t]

\small

%% [commandchars=\\\{\}]

\begin{Verbatim}
  EditorState = { code: Program, selections: Set Selection }
  ActiveState = Active | NotYetActive | Inactive
  Options     = NoOptions | StringOption String
  Result      = { description: String, code: Program }

  CodeTool =
    { name : String
    , requirements : String
    , active : EditorState -> ActiveState
    , run : (EditorState, Options) -> List Result }
\end{Verbatim}
%
\caption{Code tool interface.}
\label{fig:code-tool-interface}
\end{figure}


\begin{figure*} %% [t]
%% \begin{wrapfigure}{r}{0pt}
\includegraphics[scale=0.40]{target-positions-1.png}
\hspace{2em}
\includegraphics[scale=0.40]{target-positions-2.png}
\hspace{2em}
\includegraphics[scale=0.40]{target-positions-3.png}
%% \end{wrapfigure}
\caption{Example target positions.}
\label{fig:whitespace-polygons}
\end{figure*}
 %% HACK: here instead of deuce-ui.tex
                                %% to get this on page 4

Each code tool must implement the interface in \autoref{fig:code-tool-interface}.
A tool has access to the \verb+EditorState+, which contains a
\verb+Program+ and the \verb+Set+ of structural \verb+Selection+s within it.
Based on the \verb+EditorState+, the \verb+active+ predicate specifies
whether the tool is
\verb+Active+ (ready to run and produce \verb+Result+ options),
\verb+NotYetActive+ (could be \verb+Active+ if given more valid selections), or
\verb+Inactive+ (invalid based on the selections).
For example, \codeTool{Move Definitions} is \verb+NotYetActive+ if the user has
selected one or more definitions but no target position.
%
When invoked via \verb+run+, a tool has access to the \verb+EditorState+ and
configuration \verb+Options+, namely, an optional \verb+String+. This strategy
supports the ubiquitous \codeTool{Rename} tool. A more full-featured interface
may allow a more general set of configuration parameters; the challenge would be
to expose them using a lightweight user interface. In our implementation, all
transformations besides \codeTool{Rename} require \verb+NoOptions+. Each
\verb+Result+ is a new \verb+Program+ and a \verb+description+ of the changes.

This API between the user interface and code tool implementations is
shallow, in the sense that a code tool implementation can do whatever it
wants with the selection information. A framework for defining notions of
transformation correctness would be a useful line of work, but is beyond the scope of
this paper. Currently, code tools must be implemented inside the
\deuce{} implementation. Designing a domain-specific language for writing
transformations would be useful, but is also beyond the scope of this paper.
