\parahead{\little{}}

\noindent
To make the discussion of our design concrete, we choose to work with
a small functional language called \little{}, defined in \autoref{fig:syntax}.
A \little{} program is a sequence of top-level definitions, the last of which is
called \verb+main+. Notice that all (sub)expressions, (sub)patterns, definitions
(both at the top-level and locally via \verb+let+), and branches of \verb+case+
expressions are surrounded in the abstract syntax by orange boxes; these denote
\emph{code items} that will be exposed for selection and deselection in the user
interface. In addition, there are \emph{target positions}, denoted by blue
dots, before and after every definition, expression, and pattern in the program. Target
positions are ``abstract whitespace'' between items in the abstract syntax tree,
which will also be exposed for selection.

% !TEX root = hazelnut-dynamics.tex
\begin{figure}[t]
$\arraycolsep=4pt\begin{array}{rllllll}
\mathsf{HTyp} & \htau & ::= &
  b ~\vert~
  \tarr{\htau}{\htau} ~\vert~
  % \tprod{\htau}{\htau} ~\vert~
  % \tsum{\htau}{\htau} ~\vert~
  \tehole\\
\mathsf{HExp} & \hexp & ::= &
  c ~\vert~
  x ~\vert~
  \halam{x}{\htau}{\hexp} ~\vert~
      {\hlam{x}{\hexp}} ~\vert~
  \hap{\hexp}{\hexp} ~\vert~
  % \hpair{\hexp}{\hexp} ~\vert~
  % \hprj{i}{\hexp} ~\vert~
  % \hinj{i}{\hexp} ~\vert~
  % \hcase{\hexp}{x}{\hexp}{x}{\hexp} ~\vert~
  % \hadd{\hexp}{\hexp} ~\vert~
  \hehole{u} ~\vert~
  \hhole{\hexp}{u} ~\vert~
  \hexp : \htau\\
% \mathsf{Mark} & \markname{} & ::= &
%   \evaled{} ~\vert~  \unevaled{}\\
 \mathsf{IHExp} & \dexp  & ::= &
  c ~\vert~
  x ~\vert~
  {\halam{x}{\htau}{\dexp}} ~\vert~
  \hap{\dexp}{\dexp} ~\vert~
  % \hpair{\dexp}{\dexp} ~\vert~
  % \hprj{i}{\dexp} ~\vert~
  % \hinj{i}{\dexp} ~\vert~
  % \hcase{\dexp}{x}{\dexp}{x}{\dexp} ~\vert~
  % \hadd{\dexp}{\dexp} ~\vert~
  \dehole{\mvar}{\subst}{} ~\vert~
  \dhole{\dexp}{\mvar}{\subst}{} ~\vert~
  \dcasttwo{\dexp}{\htau}{\htau} ~\vert~
  \dcastfail{\dexp}{\htau}{\htau}\\
\end{array}$
$$
\dcastthree{\dexp}{\htau_1}{\htau_2}{\htau_3} \defeq
  \dcasttwo{\dcasttwo{\dexp}{\htau_1}{\htau_2}}{\htau_2}{\htau_3}
$$
\vspace{-12px}
\CaptionLabel{Syntax of types, $\htau$, external expressions, $\hexp$, and internal expressions, $\dexp$.
We write $x$ to range over variables,
$u$ over hole names, and
$\sigma$ over finite substitutions (i.e., environments) 
which map variables to internal expressions, written $d_1/x_1, ~\cdots, d_n/x_n$ for $n \geq 0$.}{fig:hazelnut-live-syntax}
\label{fig:HTyp}
\label{fig:HExp}
\end{figure}


\parahead{Code Tool Interface}

\begin{figure}[t]

\small

%% [commandchars=\\\{\}]

\begin{Verbatim}
  EditorState = { code: Program, selections: Set Selection }
  ActiveState = Active | NotYetActive | Inactive
  Options     = NoOptions | StringOption String
  Result      = { description: String, code: Program }

  CodeTool =
    { name : String
    , requirements : String
    , active : EditorState -> ActiveState
    , run : (EditorState, Options) -> List Result }
\end{Verbatim}
%
\caption{Code tool interface.}
\label{fig:code-tool-interface}
\end{figure}


\begin{figure*} %% [t]
%% \begin{wrapfigure}{r}{0pt}
\includegraphics[scale=0.40]{target-positions-1.png}
\hspace{2em}
\includegraphics[scale=0.40]{target-positions-2.png}
\hspace{2em}
\includegraphics[scale=0.40]{target-positions-3.png}
%% \end{wrapfigure}
\caption{Example target positions.}
\label{fig:whitespace-polygons}
\end{figure*}
 %% HACK: here instead of deuce-ui.tex
                                %% to get this on page 4

Each code tool must implement the interface in \autoref{fig:code-tool-interface}.
A tool has access to the \verb+EditorState+, which contains a
\verb+Program+ and the \verb+Set+ of structural \verb+Selection+s within it.
Based on the \verb+EditorState+, the \verb+active+ predicate specifies
whether the tool is
\verb+Active+ (ready to run and produce \verb+Result+ options),
\verb+NotYetActive+ (could be \verb+Active+ if given more valid selections), or
\verb+Inactive+ (invalid based on the selections).
For example, \codeTool{Move Definitions} is \verb+NotYetActive+ if the user has
selected one or more definitions but no target position.
%
When invoked via \verb+run+, a tool has access to the \verb+EditorState+ and
configuration \verb+Options+, namely, an optional \verb+String+. This strategy
supports the ubiquitous \codeTool{Rename} tool. A more full-featured interface
may allow a more general set of configuration parameters; the challenge would be
to expose them using a lightweight user interface. In our implementation, all
transformations besides \codeTool{Rename} require \verb+NoOptions+. Each
\verb+Result+ is a new \verb+Program+ and a \verb+description+ of the changes.

This API between the user interface and code tool implementations is
shallow, in the sense that a code tool implementation can do whatever it
wants with the selection information. A framework for defining notions of
transformation correctness would be a useful line of work, but is beyond the scope of
this paper. Currently, code tools must be implemented inside the
\deuce{} implementation. Designing a domain-specific language for writing
transformations would be useful, but is also beyond the scope of this paper.
