\subsection{Domain-Specific Transformations}
\label{sec:little-transformations-sns}

To demonstrate how our approach can be instantiated with custom
structured transformations, we have implemented several
transformations (in addition to the general-purpose ones)
that are specific to features in \sns{} and/or are particularly
useful for the domain of SVG graphics;
these are summarized in \autoref{fig:transformations-2}.

\parahead{\codeTool{Thaw/Freeze Number; Add/Remove Range; Show/Hide Slider; Flip Boolean}}


Numeric constants in \little{} can be annotated with an optional
range---written \verb+15{1-30}+---to instruct the \sns{}
editor to display a slider in the
output to make it easy to change the number without text-editing.
This feature is an example of ``scrubbing'' constants, a live
programming feature described by Bret Victor
(\url{http://worrydream.com/#!/DrawingDynamicVisualizationsTalk})
and Sean McDirmid
(\url{https://www.youtube.com/watch?v=YLrdhFEAiqo}).
We made one minor addition to the \little{} language: the option to
mark a range annotation as hidden---written
\verb@15{1-30,"hidden"}@---to keep the range information in the
program while suppressing the slider widget in the output.

In \deuce{}, the \codeTool{Add/Remove Range} tool operates on constant literals to attach
or remove these range annotations. New minimum and maximum range
values are determined based on the current value.
The \codeTool{Show/Hide Sliders} tool annotates a
range to be \verb+15{1-30,"hidden"}+, which keeps the range in the
text but suppresses the slider from the output. This tool makes it
quicker to toggle the sliders on and off (and, furthermore, preserves
the possibly-edited min- and max-values to remain in the text).
\sns{} allows numbers to be frozen with the annotation \verb+15!+,
which tells \sns{} not to change this value in response to changes to
the output. (Compared to the discussion of the \codeTool{Create Function from
Definition} tool,
the heuristic for arguments is to choose named \emph{and unfrozen}
constants.)
As with sliders, it can be tedious to use text-editing to
toggle this annotation on and off, so the \codeTool{Thaw/Freeze Constants} tool
provides an alternative.
%%
The \codeTool{Flip Boolean} tool is a quick way to ``scrub'' boolean literals.


\begin{figure*}[b]
\small

\beginToolTable
\toolTableHeaders
%% \rowTool
%%   {Invert Definitions}
%%   {0}{2}{0}{0}
%%   {}
%% \rowTool
%%   {Invert Definitions}
%%   {0}{0}{2}{0}
%%   {}
%% \rowTool
%%   {\rkc{Move Definition}}
%%   {0}{0}{1}{0}
%%   {...; Inversion}
\toolCategoryStyle
\rowTool
  {Thaw/Freeze Number}
  {1+}{0}{0}{0}
  {}
\rowTool
  {Add/Remove Range}
  {1+}{0}{0}{0}
  {}
\rowTool
  {Show/Hide Slider}
  {1+}{0}{0}{0}
  {}
\rowTool
  {Rewrite as Offset}
  {1+}{1}{0}{0}
  {}
\rowTool
  {Convert Color String}
  {1+ strings}{0}{0}{0}
  {RGB or Hue}
\hline
\toolCategoryChange
\rowToolNoBreak
  {Flip Boolean}
  {1+ booleans}{0}{0}{0}
  {}
\end{tabular}

\caption{Domain-specific transformations in \deuce{}.}
\label{fig:transformations-2}
\end{figure*}



\parahead{\codeTool{Rewrite as Offset}}

\citet{sns-uist} point out that there are often (at least) three
different common ways to programmatically describe a positional
attribute of a shape: with constants, with constant offsets from an
anchor point, or with constant relative percentages with respect to a
bounding box. When prototyping, it is often easiest to start by
using constants and then later switching to one of the relative versions.
The \codeTool{Rewrite as Offset} tool converts one or more selected constants
into offsets from a selected value. For example, in the expression
\verb+(let x 10 15)+, rewriting \verb+15+ as an offset from \verb+x+
transforms the program to \verb&(let x 10 (+ x 5))&. As with several
other structured transformations we have discussed, this transformation is
conceptually simple but can become tedious and error-prone to perform
with manual text-editing when
multiple numbers scattered across the program need to be offset and when
the base values are not easy-to-remember numbers.


\parahead{\codeTool{Convert Color String}}

In \sns{}, as in HTML and SVG, colors can be specified in a variety of ways: RGBA
codes, HSL codes, HEX, and color strings. The \sns{} editor provides
special support for color attributes defined as ``color numbers''
which are are (essentially) just the hue component of an HSL
triple; in particular, the editor displays a slider next to the shape to
control this color value. The \codeTool{Convert Color String} tool converts
color names (often useful for prototyping) into corresponding numbers, to enable the direct
manipulation support for color numbers that \sns{} provides. This
transformation is an example of how a custom program transformation can
be used to complement other features provided by the IDE.
