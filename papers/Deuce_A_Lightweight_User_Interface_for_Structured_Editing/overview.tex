\section{Overview Examples}
\label{sec:overview}


%% TODO should probably reset these after each
\setlength{\intextsep}{6pt}%
\setlength{\columnsep}{10pt}%

\newcommand{\overviewExample}[2]{\paragraph{Example {#1}}}


\overviewExample{1}{Red Square}


\begin{wrapfigure}{r}{0pt}
\includegraphics[scale=\lambdaScreenshotScale]{red-square-make-equal-and-output.png}
\end{wrapfigure}
Despite the intention of the following program,
the \verb+redSquare+ definition uses different values
for the width and height of the rectangle
(the fourth and fifth arguments, respectively, to the
\verb+rect+ function).
The user chooses \deuce{} code tools---rather than
text-edits---to correct this mistake.

The user presses the Shift key to enter structured editing mode, and then hovers
over and clicks the two constants \verb+120+ and \verb+80+ to select them; the
selected code items are colored orange in the screenshot above.
Based on these selections, \deuce{} shows a pop-up Code Tools menu with several
potential transformations.
The \codeTool{Make Equal by Copying} tool would replace one of the constants with the
other, thus generating a square. However, such a program would require two
constants to be changed whenever a different size is desired.
Instead, the user wishes to invoke \codeTool{Make Equal with Single Variable} to introduce a
new variable that will be used for both arguments. Hovering over this menu item
displays a second-level menu (shown above) with tool-specific options, in this
case, the names of four suggested new variable names.

\begin{wrapfigure}{r}{0pt}
\includegraphics[scale=\lambdaScreenshotScaleBigger]{red-square-final.png}
\end{wrapfigure}
The user hovers over the second option, which shows a preview of
the transformed code (shown on the right).
The user clicks to choose the second option.
Notice that the number \verb+80+ (rather than \verb+120+) was
chosen to be the value of the new variable \verb+w+.
Whereas the tool provided configuration options for the
variable name, it did not provide options for which value to use;
this choice was made by the implementor of the \codeTool{Make Equal} code tool,
not by the \deuce{} user interface.


\overviewExample{2}{Two Circles}

Consider the following program that draws two circles connected by a line.
All design parameters and shapes have been organized within
a single top-level \verb+connectedCircles+ definition.
%
To make the design more reusable, the user wants
\verb+connectedCircles+ to be a function that is abstracted over the positions
of the two circles.
The user hovers over and clicks the \verb+def+ keyword, and selects the \codeTool{Create Function
from Definition} tool
(shown in the screenshot).

\begin{center}
\includegraphics[scale=\lambdaScreenshotScale]{two-circles-and-output.png}
\end{center}

\begin{wrapfigure}{r}{0pt}
\includegraphics[scale=\lambdaScreenshotScaleBigger]{two-circles-reorder-args.png}
\end{wrapfigure}
%
\noindent
In response, \codeTool{Create Function}
rewrites the definition
to be a function (shown on the right), and
previous uses of \verb+connectedCircles+ are rewritten to
appropriate function calls (not shown).

The order of arguments to the function match the order of definitions
in the previous program, but that order was unintuitive---the coordinates of the
\verb+start+ and \verb+end+ points were interleaved. To fix this, as shown above, the
user selects the last two arguments and the \emph{target position} (\ie{} the
space enclosed by a blue rectangular selection widget) between the first two, and
selects the \codeTool{Reorder Arguments} tool so that the order of arguments becomes
\verb+startX+, \verb+startY+, \verb+endX+, and \verb+endY+ (not shown).
Calls to \verb+connectedCircles+ are, again, rewritten to match the new order
(not shown).


\overviewExample{3}{Lambda Icon}

In the program below, the user would like to organize all design parameters and
shapes within the single \verb+logo+ definition. The user hovers over and
selects the five definitions on lines 2 through 9, as well as the space
on line 13, and selects the \codeTool{Move Definitions} tool to move the definitions inside
\verb+logo+. The transformation manipulates indentation and
delimiters appropriately in the final code (not shown).

\vspace{2pt} %% HACK

\begin{center}
\includegraphics[scale=\lambdaScreenshotScale]{lambda-move-inside-logo-and-output.png}
\end{center}
