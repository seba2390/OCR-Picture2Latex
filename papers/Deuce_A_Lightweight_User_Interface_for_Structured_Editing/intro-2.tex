\section{Introduction}
\label{sec:intro}

Plain text continues to dominate as the universal format for programs
in most languages. Although the simplicity and generality of text are
extremely useful, the benefits come at some costs. For novice
programmers, the unrestricted nature of text leaves room for syntax errors
that make learning how to program more difficult~\cite{GreenfootCost}.
For expert programmers, many editing tasks---perhaps even the vast
majority~\cite{Ko2005}---fall within specific patterns that could be
performed more easily and safely by automated tools. Broadly speaking,
two lines of work have, respectively, sought to address these
limitations.


\paragraph{Structured Editing}

Structured editors---such as
the Cornell Program Synthesizer~\cite{Teitelbaum1981},
Scratch~\cite{Scratch:2009,Scratch:2010}, and
TouchDevelop~\cite{TouchDevelop}---reduce the
amount of unstructured text used to represent programs, relying on
blocks and other visual elements to demarcate structural components of
a program (\eg{} a conditional with two branches, and a function with an
argument and a body). Operations that create and manipulate structural
components avoid classes of errors that may otherwise arise in plain
text, and text-editing is limited to within well-formed structures.
Structured editing has not yet, however, become popular among expert
programmers, in part due to their cumbersome interfaces compared to
plain text editors~\citep{Monig:2015}, as well as their restrictions
that even transitory, evolving programs always be well-formed.

\paragraph{Text Selection-Based Refactoring}

An alternative approach in integrated development environments (IDEs),
such as Eclipse, is to augment unrestricted plain text with support
for a variety of
refactorings~\cite{GriswoldThesis,Fowler1999,SmalltalkRefactoring}. In such
systems, the user text-selects something of interest in the
program---an expression, statement, type, or class---and then selects
a transformation either from a menu at the top of the IDE or
in a right-click pop-up menu. This approach provides experts both the full
flexibility of text as well as mechanisms to perform common tasks more
efficiently and with fewer errors than with manual, low-level text-edits.
Although useful, this workflow suffers limitations:

\begin{enumerate}

\item The text-selection mechanism is error-prone when the item to select is
long, spanning a non-rectangular region or requiring
scrolling~\citep{Murphy-Hill-ICSE2008}.

\item All transformations must require a single ``primary'' selection argument,
and any additional arguments are relegated to a separate Configuration Wizard
window.

\item The list of tools is typically very long---even in the right-click menu
where tools that are not applicable to the primary selection are filtered
out---making it hard to \emph{identify}, \emph{invoke}, and \emph{configure} a
desired refactoring~\citep{Murphy-Hill:2009,Vakilian:2012,Mealy:2007}.

\item Even when a transformation has no configuration options or when the
defaults are acceptable---as is often the case~\citep{Murphy-Hill:2009}---the
user must go through a separate Configuration Wizard to make the change. The
user must, furthermore, navigate to another pane within the Configuration Wizard
to preview the changes before confirming them.

\end{enumerate}

\parahead{Our Approach}

Our goal is to enable a workflow that enjoys the benefits
of both approaches. Specifically, programs ought to be represented in
plain text for familiar and flexible editing by expert programmers,
and the editing environment ought to provide automated support for a
variety of code transformations without deviating
from the text-editing workflow.

In this paper, we present a structure-aware editor, called \deuce{}, that
achieves these goals by augmenting a text editor with (i) clickable
widgets directly atop the program text that allow the user to \emph{structurally
select} the unstructured text for subexpressions and other relevant features of
the program structure, and (ii) a \emph{context-sensitive tool menu with
previews} based on the current selections.

\paragraph{Structural \maybeCode{} Selection}

Rather than relying on keyboard-based text-edits for selection, our editor draws
direct manipulation widgets to \emph{structurally select} items in the code with
a single mouse-click. In particular, holding down the Shift key transitions the
editor into structural selection mode. In this mode, the editor draws a box
(which resembles a text-selection highlight) around the code item below the
current mouse position. Clicking the box selects the entire text for that code
item, eliminating any possibility for error and reducing the time needed to
select long, non-rectangular sequences of lines. Furthermore, this interface naturally
allows multiple selection, even when items are far apart in the code. Structural
text selection helps address concerns (1) and (2) above.

\paragraph{Context-Sensitive Menu with Previews}

Because structural selection naturally supports multiple selection, we
address concern (3) by showing only tools for which \emph{all} necessary
arguments have been selected, reducing the number of tools shown to the user
compared to a typical right-click menu.
Hovering over a result description previews the changes, and
clicking a result chooses it. For tools with few
configuration options, we believe the preview menu provides a lightweight way to
consider multiple options while staying within the normal editing workflow,
helping to address concern (4).
\\

\noindent The resulting workflow in \deuce{} is largely
text-driven, but augmented with automated support for code transformations
(\eg{} to introduce local variables, rearrange definitions, and introduce
function abstractions) that are tedious and error-prone (\eg{} because of typos,
name collisions, and mismatched delimiters), allowing the user to spend
keystrokes on more creative and difficult tasks that are harder to automate.
The name \deuce{} reflects this streamlined combination of text- and mouse-based
editing.

\parahead{Contributions}

This paper makes the following contributions:

\begin{itemize}

\item We present the design of \deuce{}, a code editor equipped with
\emph{structural \maybecode{} selection}, a lightweight direct manipulation mechanism
that helps to identify and invoke program transformations while
retaining the freedom and familiarity of traditional text-based editing. Our
design can be instantiated for different programming languages and with
different sets of program transformations.
(\autoref{sec:user-interface})

\item We implement \deuce{} within \sns{}, a programming
environment for creating Scalable Vector Graphics (SVG) images.
Most of the functional program transformations in our implementation
are common to existing refactoring tools, but two
transformations---\codeTool{Move Definitions} and \codeTool{Make Equal}---are,
to the best of our knowledge, novel.
(\autoref{sec:little-transformations})

\item To evaluate the utility of our user interface, we
performed a controlled user study with \numUsers{} participants. The results
show that, compared to a more traditional text selection-based refactoring
interface, structural \maybecode{} selection is
preferred and may be faster for invoking
transformations, particularly as users gain experience with the tool.
(\autoref{sec:user-study})

\end{itemize}

\noindent
Our implementation, videos of examples, and
user study materials are available at
\url{http://ravichugh.github.io/sketch-n-sketch/}.
%
In the next section, we introduce \deuce{} with a few short examples.
