\section{Examples in \sns{}}
\label{sec:appendix-examples}

We describe five examples we authored in \sns{}, using a
combination of text- and mouse-based code edits.
Together, the \numExamples{} examples required a
total of \timeRawVideos{} minutes of development time in \deuce{} (the videos
are a few minutes longer because of pauses during the narration), resulting in
approximately \locExamples{} lines of \little{} code for the final programs.
%
We encourage the reader to watch the accompanying videos, available on the web.
At several points in the videos, we use the existing \sns{} ability to directly
manipulate the size, position, and color of shapes in the output, as a way to
indirectly update constant literals in the program. We use this feature for
simplicity; the constant literals could, of course, be changed with ordinary
text-edits.

\parahead{Preliminary Version of \deuce{}}

Note that the screenshots below, as well as the videos, show a preliminary
version of \deuce{} (implemented in \sns{}~\version{0.6.0}),
with cosmetic differences compared to the bounding boxes
and target positions shown in this paper. In the preliminary version,
\codeTool{Create Definition from Definition} was called \codeTool{Abstract},
\codeTool{Create Definition by Merging Expressions} was called \codeTool{Merge
Expressions}, and the two \codeTool{Make Equal} tools were combined into a
single tool.

%%%%%%%%%%%%%%%%%%%%%%%%%%%%%%%%%%%%%%%%%%%%%%%%%%%%%%%%%%%%%%%%%%%%%%%%%%%%%%%%
%%%%%%%%%%%%%%%%%%%%%%%%%%%%%%%%%%%%%%%%%%%%%%%%%%%%%%%%%%%%%%%%%%%%%%%%%%%%%%%%

\subsection{Example: \exampleZero{}}

We use an example in this section to motivate and summarize the
workflow enabled by \deuce{}. The task, to write a program that
generates an SVG design, is borrowed from \citet{sns-uist}.
Throughout the discussion, notice how \deuce{} automates tasks
that would otherwise be tedious and error-prone, and the user
performs manual text-edits for tasks that require additional human
insight and choice.


%
\setlength{\intextsep}{6pt}%
\setlength{\columnsep}{10pt}%
\begin{wrapfigure}{r}{0pt}
\includegraphics[scale=0.20]{sns-logo.png}
\end{wrapfigure}
%
Consider the task of writing a program that generates the \sns{} logo
(shown in the adjacent screenshot), where two white lines
atop a black rectangle reveal a lambda symbol between three triangles.
We would like the program to be structured so that it can be reused
to easily generate configurations of the logo with different design parameters,
namely, the size, background color, and color and width of the
lines. We will describe a sequence of text- and mouse-based code
edits in \deuce{} that allows the user to
prototype, repair, and refactor the code until it achieves
the desired goal. The reader is encouraged to follow along with
the example in our online demo, or to watch the
video.

\parahead{Phase I: Prototyping}

The user writes definitions for the background rectangle and the first line,
between the top-left and bottom-right corners of the logo.
To start, these shapes are both
positioned at the origin \verb+(0,0)+ and stretch to
\verb+(200,200)+. They are combined in
\verb+(svg [rectangle line1])+ to generate the first version to
render; this code can be seen in the screenshot in the left half
of \autoref{fig:overview-1}.
The user confirms that the pieces look roughly as intended, and
now returns to the program to replace the hard-coded \verb+(0,0)+
constants with variables so that the positions of both shapes can be
modified together.

Rather than text-editing, however, the
user can use the structured editing capabilities of \deuce{} to
perform this task in a quick, and safe, way.
The user holds down the Shift key, which causes \deuce{} to display
clickable widgets when hovering over different parts of the program
text. The user clicks on the two \verb+0+ constants that correspond to
the x-positions of the top-left corners of the shapes, and \deuce{}
displays a menu of potential transformations underneath the selected
widgets. When the user hovers over the \codeTool{Make Equal} tool in the menu,
\deuce{} displays a list of candidate new variable names to add to the
program. When hovering over each choice, \deuce{} previews the result
of the transformation; in each case, the selected constants are replaced by
uses of the new variable, which binds \verb+0+ and is defined at the
innermost scope relative to the constants.
The user chooses the new variable name to be \verb+x1+, and
the code is transformed as shown in the right half of
\autoref{fig:overview-1}.

\begin{figure}[h]
\centering
%
\includegraphics[scale=\snsScreenshotScale]{sns-logo-1-before-after.png}
%
\caption{When the user holds the Shift key and selects the two constants
(highlighted in orange), a context-sensitive menu identifies the program
transformations that may be applied. When the user invokes the ``Make
Equal, New variable: x1'' tool, the program is transformed to the
version on the right.}
\label{fig:overview-1}
\end{figure}


\begin{wrapfigure}{r}{0pt}
\includegraphics[scale=\snsScreenshotScale]{sns-logo-2-before-2.png}
\end{wrapfigure}

Next, the user employs the \codeTool{Make Equal} tool three more times, selecting the
remaining pairs of constants that define the x- and y-positions of the
top-left and bottom-right corners. Although happy with the four
names \deuce{} has suggested (\verb+x1+, \verb+y1+, \verb+w+, and
\verb+h+), the user wants to combine the
definitions into a single line, because each of the names and values is
short.
%
The user selects three variables to move---\verb+y1+,
\verb+w+, and \verb+h+---and the \emph{target position} after the
\verb+x1+ variable, to indicate that the selected variables should
appear inline after the \verb+x1+ (shown on the right).
The user hovers over the \codeTool{Move
Definitions} tool to preview the transformation where all four
variables are defined in a tuple (\ie{} 4-element list), and then
selects this transformation.

Now, the user uses normal copy-paste to duplicate the \verb+line1+ definition,
renames it \verb+line2+, and adds \verb+line2+ to the list of shapes.
This second line will eventually connect the bottom-left corner of the
logo to its center, which will form the lambda symbol.
%% To start testing the second line,
To start testing, however,
the user edits the start and end points to be \verb+(x1,h)+ and
\verb+(w,y1)+, respectively, which generates the symbol ``X'' when run.

\begin{wrapfigure}{r}{0pt}
\includegraphics[scale=\snsScreenshotScale]{sns-logo-3-before-3.png}
\end{wrapfigure}

The user invokes
\codeTool{Make Equal} twice to relate the stroke color and width of the two
lines with two new variables called \verb+stroke+ and \verb+w1+,
respectively.
%%
The user is not happy with the name chosen for the latter, so
she clicks on the variable definition and uses the \codeTool{Rename} tool to rename
\verb+w1+ to \verb+strokeWidth+ (the interaction is shown on the
right), which automatically replaces all uses with the new name.

\begin{wrapfigure}{r}{0pt}
\includegraphics[scale=\snsScreenshotScale]{sns-logo-5-before-2.png}
\end{wrapfigure}

Then the user invokes the \codeTool{Move Definition} tool to combine these two
variables into a tuple (not shown). The user invokes \codeTool{Move Definition}
once again
to move \verb+rectangle+ below \verb+stroke+ and \verb+strokeWidth+ (not shown).
%%
Having taken care to organize the code in a readable fashion, the
user would like to define a variable to clearly identify that the
constant \verb+'black'+ defines the color of the rectangle. The user
selects this constant and the target position before \verb+stroke+,
and invokes the \codeTool{Introduce Variable} tool (shown on the right) to add a
new variable called \verb+fill+ in place of the string literal.


\parahead{Phase II: Repairing}

At this stage, the user has become comfortable with the basics of the
design, but is aware of two issues that must be addressed.
The first issue is that even though the position of the top-left
corner has been factored into variables \verb+x1+ and \verb+y1+,
the relationships for the other endpoints depend on the values of these
variables both being \verb+0+. To verify this,
the user text-edits them to be \verb+50+ and \verb+50+, re-runs the
program, and confirms that the lines do not ``move'' with the rectangle.
Knowing what the intended relationships ought to be, the user
text-edits the second endpoint of \verb+line1+ to be \verb@(+ x1 w)@ and
\verb@(+ y1 h)@.

\begin{wrapfigure}{r}{0pt}
\includegraphics[scale=\snsScreenshotScale]{sns-logo-6-before-3.png}
\end{wrapfigure}

Now, to snap the other line of the ``X'' to the
top-right corner, the user must use these same subexpressions in the
definition of \verb+line2+. The user selects \verb@(+ y1 h)@ of \verb+line1+
and \verb+h+ in \verb+line2+, and invokes the \codeTool{Make Equal} tool (shown
on the right). Because these
two subexpressions differ, the \deuce{} menu asks the user to select which
of these expressions should be used in both places; the user chooses
to use the sum expression. Similarly, the user invokes \codeTool{Make Equal} (not
shown) to replace the \verb+w+ in \verb+line2+ with the \verb@(+ x1 w)@.

The remaining issue is that the second line needs to be half the
length, so that it reveals the lambda symbol instead of the letter
``X.'' To do this, the user text-edits the coordinates of the
endpoint to be \verb@(+ x1 (/ w 2))@ and \verb@(+ y1 (/ h 2))@,
respectively. Viewing the output now reveals the lambda. Either with
text-edits or the existing output-directed manipulation features of
\sns{}, the user
varies the values of the four positional variables, and visually
confirms that the output continues to exhibit the intended lambda
symbol.

\parahead{Phase III: Refactoring}

At this point, the user has finished encoding all the desired
relationships in the program. Now is the time to refactor the program
so that it can be reused to generate multiple variations. First, the
user selects the list of shapes at the end of the program
and invokes the \codeTool{Introduce Variable} tool
(shown below left) to give it a name (\verb+shapes+) outside the \verb+svg+
expression. Next, the user selects the definitions that contribute to
\verb+shapes+, and invokes the \codeTool{Move Definitions} tool to place them inside
the \verb+shapes+ definition (shown below right).

\begin{figure}[h]
\centering
%
\begin{minipage}{0.49\textwidth}
\centering
\includegraphics[scale=\snsScreenshotScale]{sns-logo-7-before-2.png}
\end{minipage}
%
\hspace{0.0\textwidth}
%
\begin{minipage}{0.49\textwidth}
\centering
\includegraphics[scale=\snsScreenshotScale]{sns-logo-8-before-2.png}
\end{minipage}
%
\end{figure}


\noindent
The top-level definitions are turned into local
let-bindings, taking care of indentation and parenthesis delimiters
that would otherwise require tedious, manual text-edits. The user uses
\codeTool{Rename Variable} (not shown) to change \verb+shapes+ to \verb+logo+.

The final step
is to turn \verb+logo+ into a function that is parameterized over the
design constants inside the definition. Selecting the definition (shown below),
\deuce{} shows a \codeTool{Abstract} tool to turn several of the constants into
function arguments.
In \autoref{fig:overview-variations},
notice how the use of \verb+logo+ has been rewritten to a call, with
the appropriate arguments selected from their values within the
definition. Again, this would be a tedious and error-prone manual
transformation, as the connection between formal parameters and
actual arguments are
not syntactically adjacent in the program. However, this
transformation is easy to automate with structure information.
To create other versions, the user copies and pastes the
function call and changes the arguments to each (shown in
\autoref{fig:overview-variations}).


\begin{wrapfigure}{r}{0pt}
\includegraphics[scale=\snsScreenshotScale]{sns-logo-9-before-2.png}
\end{wrapfigure}

To recap, the process of prototyping, repairing, and refactoring the
program is a text-driven process, as usual, but with support for
automating low-level, structural edits that are tedious and
error-prone (\eg{} because of typos, accidental shadowing, and mismatched
delimiters). Furthermore, the tools can suggest useful information,
such as the different variable names offered by Make Equal and
Introduce Variable. As a result, the user spends keystrokes on the more
interesting tasks that are harder to automate---arithmetic
relationships in the design,
the choice of names, and
final decisions about whitespace and formatting.

\begin{figure*}[h] %% [t]
\includegraphics[scale=0.50]{sns-logo-variations.png}
\caption{Program to generate \sns{} logo,
developed with a combination of text-edits and \deuce{} code tools.}
\label{fig:overview-variations}
\end{figure*}





%%%%%%%%%%%%%%%%%%%%%%%%%%%%%%%%%%%%%%%%%%%%%%%%%%%%%%%%%%%%%%%%%%%%%%%%%%%%%%%%
%%%%%%%%%%%%%%%%%%%%%%%%%%%%%%%%%%%%%%%%%%%%%%%%%%%%%%%%%%%%%%%%%%%%%%%%%%%%%%%%


\subsection{Example: \exampleOne{}}

\begin{wrapfigure}{r}{0pt}
\includegraphics[scale=0.10]{target-output.png}
\end{wrapfigure}

In this example, our goal is to generate a target comprising
concentric rings of alternating color. We start by writing a single
red circle, using the expression \verb+(* 1 50)+ for the radius to
anticipate that it will scale linearly for ring \verb+2+, \verb+3+, and
so on. We use \codeTool{Abstract} to extract a function,
\verb+ring+, that is parameterized over this index \verb+i+, and we use
\codeTool{Remove Arguments} so that \verb+ring+ is parameterized over
only \verb+i+. We use \codeTool{Introduce Variable} to give a name to the ring color,
and then use text-editing to choose red or gray depending on the
parity of index \verb+i+.
%
Next, we update the main expression with text-editing to \verb+map+
the function's \verb+i+ over the list of indices \verb+(reverse (range 1 4))+.
We use the \codeTool{Abstract} tool to extract a \verb+target+ function
for drawing these concentric circles. We would like the
\verb+target+ function to take \verb+cx+, \verb+cy+, \verb+ringWidth+, and
\verb+ringCount+, but the first three of these are currently constants
inside the \verb+ring+ function. To turn these constants into function
arguments, we move the entire \verb+ring+ definition (and thus the constants of interest)
inside of \verb+target+, which then allows us to use the \deuce{} tools to
introduce and rename the desired arguments.


\subsection{Example: \exampleTwo{}}

\begin{wrapfigure}{r}{0pt}
\includegraphics[scale=0.30]{placeholder-battery.png}
\end{wrapfigure}

In this example, our goal is to build a program that generates a
battery icon, akin to those often found in operating system task bars.
The design comprises three shapes: a polygon with rounded stroke
for the body of the battery, a rectangle for the cap, and a
rectangle for the battery juice inside. In addition to setting up the
appropriate positional relationships, we want the color of the battery
juice to change based on the amount that remains.
Our development process from start to finish, which takes
approximately \timeRawVideoBattery{} minutes (without narration),
mixes text-edits and \deuce{} transformations throughout. Our
general workflow is to incrementally add new shapes and features,
often starting with hard-coded or copy-pasted expressions, and then
iteratively repairing the program by adding new variables and
relationships.

The first shape we add to the program is the polygon for the body outline. We use the
\codeTool{Introduce Variable} tool to give names to the top-left and top-right
corners of the body, which are needed by subsequent expressions.
We use the \codeTool{Make Equal} tool to equate certain offsets among edges in
the design, and we use the domain-specific \codeTool{Freeze} tool to ensure that
some of these offsets are always the constant zero. If we accidentally
swap the usages of \verb+width+ and \verb+height+ variables,
we can use the \codeTool{Swap Variable Names and Usages}
tool to correct the bug. The second shape we add to the program
is the cap. Again, we use a combination of text-edits and structured
editing tools, such as \codeTool{Introduce Variable} and \codeTool{Move Definitions},
during this process.

The last shape we add is the colored rectangle for the battery juice.
After we add it to our list of shapes, we see that the juice appears
on top of the body rather than behind it. We use the \codeTool{Reorder Items}
tool to move this rectangle earlier in this list, which results in the
desired z-ordering.
When prototyping, it is
natural to define the width of this rectangle directly as a constant
\verb+w+, with the intention that it remains less than the \verb+width+ of the body.
Later in the development process,
we use text-editing to introduce a conditional
expression that determines the color (\ie{}, green, black, orange, or
red) based on the value of the percentage \verb+(/ w width)+. This
expression appears in several guards of a multi-way conditional, so
we use the \codeTool{Make Equal with Single Variable} tool to give it a name,
which we \codeTool{Rename} to \verb+juicePct+. Now that the relationships are set
up, we realize that it would be better to first define \verb+juicePct+
(the percentage denoted by a number between \verb+0.0+ and \verb+1.0+)
and then derive \verb+w+ in terms of it. We use the \codeTool{Move Definition}
tool to drag the former above the latter, and \deuce{} proposes an
option where the definitions are inverted, specifically,
\verb+juicePct+ is redefined to be a constant percentage and \verb+w+
is defined in terms of it. We use the \codeTool{Add Range} tool twice twice, once
on \verb+w+ before it was rewritten and once on \verb+juicePct+
afterwards. In each case, the automatically chosen range was useful
for allowing the slider to quickly manipulate reasonable values for
each quantity.

At this point, our program generates the entire icon in terms of the design
parameters. To finish, we turn the definition into a function
using a similar series of \codeTool{Move Definitions} and \codeTool{Create
Function from Definition} transformations as described earlier.
This time, we realize
that the \codeTool{Abstract} tool did not make all of our desired constants into
parameters. So, we use the \codeTool{Add Argument} and \codeTool{Rename} tools to reach our
desired parameterization of \verb+topLeft+, \verb+bodyWidth+, \verb+bodyHeight+,
\verb+capWidth+, \verb+capHeight+, \verb+strokeWidth+, and \verb+juicePct+.


\subsection{Example: \exampleThree{}}

In this example, our goal is to build a program that generates a
coffee mug, in a way that it is easy to reposition and resize the logo.
When developing the body of the mug and two concentric ellipses for
the handle, the \codeTool{Introduce Variable} and \codeTool{Move Definition} tools help turn
initially hard-coded shapes into the desired relationships.

\begin{wrapfigure}{r}{0pt}
\includegraphics[scale=0.22]{mug-1.png}
\end{wrapfigure}

When designing the steam, we use tools not already exercised in the previous
examples. When we add the first steam puff, we use hard-coded
constants for all of the points and control points of the path. This
helps us get the initial design for the curvy puff, but makes it hard
to move to a different position; all 12 constants must be updated by
the same offset to translate it. We use the \codeTool{Rewrite as Offset} tool
several times to make the steam puff rigid.
Then we use the \codeTool{Duplicate Definition} tool to copy-paste (via \deuce{}
rather than text-editing) the first steam definition twice. After
changing just the initial position of each puff, our copy-pasted
definitions contain nearly identical code. We use the \codeTool{Merge} tool to abstract the
three steam puffs over their differences (\ie{} the position).
We use \codeTool{Rewrite as Offset} several more times to position the
second and third steam puffs in terms of the first, and then again to
position the first steam puff in terms of the mug; the effect is that
the steam remains rigid and correctly positioned as the mug is
translated to different positions. During these last steps, we move
several definitions from the bottom of the program up to the top so
that the related expressions are closer together; the \codeTool{Move Definition}
tool allows us to make such transformations without fear of breaking
dependencies in or changing the binding structure of the program.


\subsection{Example: \exampleFour{}}

\begin{wrapfigure}{r}{0pt}
\includegraphics[scale=0.17]{arch.png}
\end{wrapfigure}

Inspired by the Mondrian programming-by-demonstration graphics editor~\cite{Mondrian},
our final example is an arch, where two upright
rectangles support a third horizontal rectangle, all of which are of
equal width. As in the previous examples, we use tools like \codeTool{Introduce
Variable} often to help reorganize the code and text-edits to fill in
arithmetic relationships. Unlike the previous examples, we use tools
that manipulate concrete whitespace---\codeTool{Make Multi-Line} to facilitate
the step of going from one call to \verb+rect+ to multiple ones, and
\codeTool{Make Multi-Line} to make the arguments to these adjacent calls line up
vertically, making it easier to distinguish the differences between
all calls. These tools eliminate some of the tedious text-edits that
arise when making such stylistic changes to the code.

