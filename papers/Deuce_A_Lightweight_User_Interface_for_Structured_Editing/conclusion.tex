%% \section{Conclusion and Future Work}
%% \section{Future Work}
%% \label{sec:discussion-future}
\section{Conclusion}

%% As described in \autoref{sec:intro}, one primary motivation for our
%% work is to simplify the steps to identify, invoke, and configure
%% transformations provided by refactoring tools. Another motivation for
%% our work is to demonstrate how a variety of structured transformations
%% that are not refactorings (\ie{} because they change program behavior)
%% ought to be provided with automated tooling.

Based on our experience and the results of our user study,
we believe \deuce{} represents a proof-of-concept for how to achieve a
lightweight, integrated combination of text- and structured editing. In future
work, our design may be adapted and implemented for full-featured programming
languages and development environments, incorporating additional well-known
transformations (\eg{}~\citet{Fowler1999,Thompson2013}). Additional direct code
manipulation gestures, as well as incremental parsing (\eg{}~the algorithm
of \citet{Wagner:1998} used by Barista~\citep{Barista}),
could further help streamline, and augment,
support for structured editing within an unrestricted text-editing workflow.
