%% \begin{figure}[h]
\begin{figure}[t]

\small

\centering

%% \judgementHeadNameOnly{Syntax of Expressions}

\vsepRule

\figSyntaxBegin

\figSyntaxRowLabel{Programs}{\varProgram}
  %% {\expTopDef{\varVar}{\varExp_\varVar}}^* \hspace{0.04in}
  %% \expTopDef{\mathit{main}}{\varExp}
  %% \targetBefore{\expDef{\varVar}{\varExp}{\varProgram}}
  %% \figSyntaxSpaceItem
  %% \targetBefore{\expTopDef{\mathtt{main}}{\varExp}}
  \targetBefore{\expDef{\varVar_0}{\varExp_0}{\targetsAround{\ \cdots\ }}}\
  \expTopDef{\mathtt{main}}{\varExp}
\figSyntaxSpaceNextCategory
\figSyntaxRowLabel{Expressions}{\varExpBare}
  \varConst
  \figSyntaxSpaceItem
  \varVar
  \figSyntaxSpaceItem
  \expFun{\varPat}{\varExp}
  \figSyntaxSpaceItem
  \expApp{\varExp_1}{\varExp_2}
  \figSyntaxSpaceItem
  \expCons{\varExp_1}{\varExp_2}
\figSyntaxLineBreak
\figSyntaxRow
  \expLet{\varPat}{\varExp_1}{\varExp_2}
  \figSyntaxSpaceItem
  \expCaseTwo{\varExp}{\expBranchWithId{\varExpId}{1}{\varPat_1}{\varExp_1}}{\cdots}
\figSyntaxSpaceNextCategory
\figSyntaxRowLabel{Patterns}{\varPatBare}
  \varConst
  \figSyntaxSpaceItem
  \varVar
  \figSyntaxSpaceItem
  \expNil
  \figSyntaxSpaceItem
  \expCons{\varPat_1}{\varPat_2}
\\[5pt]
\multicolumn{3}{c}{
  \mathrm{Expressions}\hspace{0.13in}\varExp\hspace{0.13in}{::=}\hspace{0.13in}\targetsAround{\wrapInBox{$\varExpBare$}}
  \hspace{0.30in}
  \mathrm{Patterns}\hspace{0.13in}\varPat\hspace{0.13in}{::=}\hspace{0.13in}\targetsAround{\wrapInBox{$\varPatBare$}}
}
%% \targetsAround{\wrapInBox{$\varExpBare$}_{\color{orange}\ \varExpId}}
%% \targetsAround{\wrapInBox{$\varPatBare$}_{\color{orange}\ \varExpId.\varPathId}}

\figSyntaxEnd

%% \vsepRule

\caption{Syntax of \little{}.
The orange boxes and blue dots \mbox{identify} features for structural selection.}
%% \caption{Syntax of \little{}. Expressions, patterns, definitions, and
%% let-bindings are all surrounded by orange boxes, to denote that they
%% can be selected in the user interface. The target positions, denoted
%% by orange circles, before and after expressions and patterns
%% can also be selected in the user interface.}
\label{fig:syntax}
\end{figure}
