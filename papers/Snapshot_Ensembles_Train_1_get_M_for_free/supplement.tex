%!TEX root=main.tex

\section*{Supplementary}
\textbf{A. Single model and Snapshot Ensemble performance over time}

In Figures~\ref{fig:all-resuls-part1}-\ref{fig:all-resuls-part3}, we compare the test error of Snapshot Ensembles with the error of individual model snapshots. The blue curve shows the test error of a single model snapshot using a cyclic cosine learning rate. The green curve shows the test error when ensembling model snapshots over time. (Note that, unlike Figure~\ref{fig:free-ensemble-barplot}, we construct these ensembles beginning with the earliest snapshots.) As a reference, the red dashed line in each panel represents the test error of single model trained for 300 epochs using a standard learning rate schedule. Without Snapshot Ensembles, in about half of the cases, the test error of final model using a cyclic learning rate---the right most point in the blue curve---is no better than using a standard learning rate schedule.

One can observe that under almost all settings, complete Snapshot Ensembles---the right most points of the green curves---outperform the single model baselines. In many cases, ensembles of just 2 or 3 model snapshots are able to match the performance of the single model trained with a standard learning rate. Not surprisingly, the ensembles of model snapshots consistently outperform any of its members, yielding a smooth curve of test error over time.

\begin{figure}[!h]
    \includegraphics[width=1\textwidth]{figures/all_results_5x2_part1.pdf}
    \caption{\small Single model and Snapshot Ensemble performance over time (part 1).}
    \label{fig:all-resuls-part1}
\end{figure}

\begin{figure}
    \includegraphics[width=1\textwidth]{figures/all_results_5x2_part2.pdf}
    \caption{\small Single model and Snapshot Ensemble performance over time (part 2).}
    \label{fig:all-resuls-part2}
\end{figure}

\begin{figure}
    \includegraphics[width=1\textwidth]{figures/all_results_5x2_part3.pdf}
    \caption{\small Single model and Snapshot Ensemble performance over time (part 3).}
    \label{fig:all-resuls-part3}
\end{figure}

