%% ****** Start of file apstemplate.tex ****** %
%%
%%
%%   This file is part of the APS files in the REVTeX 4.2 distribution.
%%   Version 4.2a of REVTeX, January, 2015
%%
%%
%%   Copyright (c) 2015 The American Physical Society.
%%
%%   See the REVTeX 4 README file for restrictions and more information.
%%
%
% This is a template for producing manuscripts for use with REVTEX 4.2
% Copy this file to another name and then work on that file.
% That way, you always have this original template file to use.
%
% Group addresses by affiliation; use superscriptaddress for long
% author lists, or if there are many overlapping affiliations.
% For Phys. Rev. appearance, change preprint to twocolumn.
% Choose pra, prb, prc, prd, pre, prl, prstab, prstper, or rmp for journal
%  Add 'draft' option to mark overfull boxes with black boxes
%  Add 'showkeys' option to make keywords appear
\documentclass[aps,pre,groupedaddress]{revtex4-2}
%\documentclass[aps,pre,preprint,groupedaddress]{revtex4-2}
%\documentclass[aps,prl,preprint,superscriptaddress]{revtex4-2}
%\documentclass[aps,prl,reprint,groupedaddress]{revtex4-2}
\usepackage{xcolor}
\usepackage{graphicx}
\usepackage[ruled,vlined]{algorithm2e}
\usepackage{amsmath}
% You should use BibTeX and apsrev.bst for references
% Choosing a journal automatically selects the correct APS
% BibTeX style file (bst file), so only uncomment the line
% below if necessary.
%\bibliographystyle{apsrev4-2}
\newcommand{\sjg}[1]{\textcolor{orange}{#1}}
\newcommand{\nes}[1]{\textcolor{red}{#1}}
\begin{document}

% Use the \preprint command to place your local institutional report
% number in the upper righthand corner of the title page in preprint mode.
% Multiple \preprint commands are allowed.
% Use the 'preprintnumbers' class option to override journal defaults
% to display numbers if necessary
%\preprint{}

%Title of paper
\title{An Entropy Law Governing Retail Shopper Decisions: Supplemental Material}

% repeat the \author .. \affiliation  etc. as needed
% \email, \thanks, \homepage, \altaffiliation all apply to the current
% author. Explanatory text should go in the []'s, actual e-mail
% address or url should go in the {}'s for \email and \homepage.
% Please use the appropriate macro foreach each type of information

% \affiliation command applies to all authors since the last
% \affiliation command. The \affiliation command should follow the
% other information
% \affiliation can be followed by \email, \homepage, \thanks as well.
\author{Nicholas Sohre$^{1}$}
\email[]{sohre007@umn.edu}
\author{Alisdair Wallis$^{2}$}
\author{Stephen J. Guy$^{1}$}
\email[]{sjguy@umn.edu}
\affiliation{University of Minnesota$^{1}$, Computer Science \& Engineering, Tesco PLC$^{2}$}
%\homepage[]{Your web page}
%\thanks{}
%\altaffiliation{}

%Collaboration name if desired (requires use of superscriptaddress
%option in \documentclass). \noaffiliation is required (may also be
%used with the \author command).
%\collaboration can be followed by \email, \homepage, \thanks as well.
%\collaboration{}
%\noaffiliation
\maketitle
% Specify following sections are appendices. Use \appendix* if there
% only one appendix.
\appendix
\section{Proofs \label{app:proof}}
\paragraph{\textbf{Lemma:} Simulated Inversion rate of two items}
Given two items having true geodesic distances $b$ and $c$ from the agent, their estimated distances $\hat{b}$ and $\hat{c}$ under our simulation model are
\begin{equation}
\begin{split}
\hat{b} = b + w_{b}, \epsilon_{b} \sim \mathcal{N}(0,\alpha*b)\\
\hat{c} = c + w_{c}, \epsilon_{c} \sim \mathcal{N}(0,\alpha*c) 
\end{split}
\end{equation}
Which produce Gaussian random variables of the estimated distances $B \sim \mathcal{N}(b,\alpha*b) $ and $C \sim \mathcal{N}(c,\alpha*c)$.  Suppose $b < c$ (that is, item $b$ is closer to the agent than item $c$). Then, the probability of an inversion is the probability that the estimated distances swap in magnitude: $C < B \rightarrow C - B < 0$. Let $Y = C - B$ be a new Gaussian random variable, then 
\begin{equation}
\begin{split}
Y \sim \mathcal{N}(c-b, \sqrt{(\alpha*c)^2 + (\alpha*b)^2})\\
= \mathcal{N}(c-b, \alpha*\sqrt{c^2+b^2})
\end{split}
\end{equation}
and the likelihood of inversion is $P(Y < 0)$. For a given $b$, $c$, and $\alpha$, this quantity can be computed analytically from the CDF of $Y$ evaluated at $0$:
\begin{equation}
P(Y < 0) = \frac{1}{2} \left[ 1 + \text{erf}\left( \frac{0 - (c - b)}{\alpha*\sqrt{2(c^2+b^2)}} \right) \right] 
\label{eq:gauss_inv}
\end{equation}

\paragraph{\textbf{Theorem:} Simulated inversion chance increases with distance to the closer item}
First, we note that $1 + \text{erf}(x)$ is a positive, increasing function of $x$. Then, given equation~\ref{eq:gauss_inv}, it suffices to show that for any $b$ (the distance to the closer item), the input to \textit{erf} is increasing:
\begin{equation}
\forall b \left[ \frac{\delta}{\delta b}\, \frac{b - c }{\alpha*\sqrt{2(c^2+b^2)}} \geq 0\right] ,\;  0 < b \leq c
\label{eq:erf_inner}
\end{equation}

\textbf{Proof:} evaluating the partial derivative in equation~\ref{eq:erf_inner} with respect to $b$, we have

\begin{align}
& \frac{\delta}{\delta b}\, \frac{b - c }{\alpha*\sqrt{2(c^2+b^2)}}& \nonumber\\
& = \frac{1}{\alpha\sqrt{2}} \frac{\delta}{\delta b}\, \frac{b - c }{\sqrt{c^2+b^2}}& \nonumber\\
& = \frac{1}{\alpha\sqrt{2}} \frac{\sqrt{b^2+c^2} - \frac{(b-c)b}{\sqrt{b^2+c^2}}}{(b^2+c^2)}\;\; & \text{(quotient rule)} \nonumber\\
& = \frac{1}{\alpha\sqrt{2}}\frac{c(c+b)}{(b^2+c^2)^{\frac{3}{2}}}\;\; & \text{(simplify)}
\label{eq:inv_derivative}
\end{align}
Since equation~\ref{eq:inv_derivative} is positive whenever $b, c$ and $\alpha$ are positive, the derivative is positive with respect to $b$ and constrains equation~\ref{eq:gauss_inv} to be increasing with increasing $b$. 

\paragraph{\textbf{Theorem:} Simulated inversion chance decreases with increasing distance between items}
Another important property for maintaining the relationship between difficulty and inversion chance is that the inversion chance must decrease with increasing $W = c-b$.\\
\textbf{Proof:} We wish to show that the derivative with respect to $W$ is always negative:
\begin{equation}
\forall W = c-b, \left[ \frac{\delta}{\delta W}\, \frac{b - c }{\alpha*\sqrt{2(c^2+b^2)}} \leq 0\right] ,\;  0 < b \leq c
\label{eq:erf_inner_W}
\end{equation}

Noting that $(b^2 + c^2) = W^2 + 2cb = W^2 + 2b(b+W)$, we can substitute into equation~\ref{eq:erf_inner_W} and get
\begin{equation}
    \frac{\delta}{\delta W}\, \frac{-W}{\alpha*\sqrt{2(W^2 + 2b(b+W))}}
\label{eq:inv_derivative_subW}
\end{equation}

While we cannot write the derivative in terms of only $W$, we can treat $b$ as a positive constant and take the partial with respect to $W$. If the result is negative for any value of $b$, then the derivative with respect to $W$ is negative regardless of $b$ and the property is satisfied:
\begin{align}
& \frac{\delta}{\delta W}\, \left[-\frac{W}{\alpha*\sqrt{2(W^2 + 2b(b+W))}}\right]& \nonumber\\
& = \frac{1}{\alpha\sqrt{2}} \frac{\delta}{\delta W}\, \left[-\frac{W}{\sqrt{W^2 + 2b(b+W)}}\right]& \nonumber\\
& = \frac{1}{\alpha\sqrt{2}} \left[-\frac{\sqrt{W^2 + 2b(b+W)} - \frac{W(W+b)}{\sqrt{W^2 + 2b(b+W)}} }{W^2 + 2b(b+W)}\right]\;\; & \text{(quotient rule)} \nonumber\\
& = \frac{1}{\alpha\sqrt{2}} \left[ \frac{W(W+b)}{(W^2 + 2b(b+W))^\frac{3}{2}} - \frac{W^2+2b(b+W)}{(W^2 + 2b(b+W))^\frac{3}{2}}\right]\;\; & \text{(simplify)}\nonumber\\
& = \frac{1}{\alpha\sqrt{2}} \frac{-b(2b+W)}{(W^2 + 2b(b+W))^\frac{3}{2}}\;\; & \text{(simplify)}
\label{eq:inv_derivative_W}
\end{align}

Since $W$ and $b$ are both positive non-zero quantities, the resulting derivative is always negative as desired.  

\paragraph{\textbf{Theorem:} Simulation inversion chance increases monotonically with difficulty}
To show this property, it is sufficient to show that difficulty also increases monotonically with decreasing $W$ and increasing $b$, since both these two values fully specify both the inversion rate (given an $\alpha$) and the difficulty.\\
\textbf{Proof:} First, we note that $W$ and $b$ are sufficient to fully describe difficulty:
\begin{equation}
\text{\textit{difficulty}} = log_2\left(\frac{A}{W}\right) = log_2\left(\frac{ b+W}{W}\right)  
\end{equation}
Then we can construct the partial derivatives with respect to both $W$ and $b$ for difficulty and see that they are always positive and negative respectively:
\begin{equation}
\frac{\delta}{\delta W} \frac{A}{W} =\frac{\delta}{\delta W} \frac{b+W}{W} = \frac{-b}{W^2}
\end{equation}
\begin{equation}
\frac{\delta}{\delta b} \frac{A}{W} =\frac{\delta}{\delta b} \frac{b+W}{W} = \frac{b+W}{W^2}
\end{equation}
Thus, both inversion rate and difficulty monotonically increase with increasing $b$ and decrease with increasing $W$. Since $W$ and $b$ are both sufficient to fully specify both inversion rate and difficulty, we cannot make one smaller or larger (by adjusting $W$ or $b$) without having the same effect on the other. Therefore, both these quantities will have a monotonic relationship with each other.

\section{Simulation Details \label{app:sim}}
\begin{algorithm}[H]
\SetAlgoLined
\textbf{Input:} itemsToRetrieve\;
\textbf{Output:} itemOrder\;
 itemOrder = []\;
 alpha = 0.30\;
 \While{While itemsToRetrieve.length $<$ 1}{
  distances = []\;
  \For{i in 0 : itemsToRetrieve.length-1 }
  {
    trueDistance = getGeodesicDistance(itemsToRetrieve[i])\;
    noisyDistance = trueDistance + sampleNormal(0, alpha  * trueDistance)\;
    distances.push(noisyDistance)\;
  }
  itemsByEstimatedDist = sort(itemsToRetrieve, by = distances)\;
  itemOrder.push(itemsByEstimatedDist[0])\;
  itemsToRetrieve.remove(itemsByEstimatedDist[0])\;
 }
 itemOrder.push(itemsToRetrieve[0])\;
 return itemOrder;
 \caption{Basket Simulation \label{alg:simulation}}
\end{algorithm}

\end{document}
%
% ****** End of file apstemplate.tex ******

