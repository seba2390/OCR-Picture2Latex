%% bare_jrnl.tex
%% V1snr.4a
%% 2014/09/17
%% by Michael Shell
%% see http://www.michaelshell.org/
%% for current contact information.
%%
%% This is a skeleton file demonstrating the use of IEEEtran.cls
%% (requires IEEEtran.cls version 1.8a or later) with an IEEE
%% journal paper.
%%
%% Support sites:
%% http://www.michaelshell.org/tex/ieeetran/
%% http://www.ctan.org/tex-archive/macros/latex/contrib/IEEEtran/
%% and
%% http://www.ieee.org/

%%*************************************************************************
%% Legal Notice:
%% This code is offered as-is without any warranty either expressed or
%% implied; without even the implied warranty of MERCHANTABILITY or
%% FITNESS FOR A PARTICULAR PURPOSE! 
%% User assumes all risk.
%% In no event shall IEEE or any contributor to this code be liable for
%% any damages or losses, including, but not limited to, incidental,
%% consequential, or any other damages, resulting from the use or misuse
%% of any information contained here.
%%
%% All comments are the opinions of their respective authors and are not
%% necessarily endorsed by the IEEE.
%%
%% This work is distributed under the LaTeX Project Public License (LPPL)
%% ( http://www.latex-project.org/ ) version 1.3, and may be freely used,
%% distributed and modified. A copy of the LPPL, version 1.3, is included
%% in the base LaTeX documentation of all distributions of LaTeX released
%% 2003/12/01 or later.
%% Retain all contribution notices and credits.
%% ** Modified files should be clearly indicated as such, including  **
%% ** renaming them and changing author support contact information. **
%%
%% File list of work: IEEEtran.cls, IEEEtran_HOWTO.pdf, bare_adv.tex,
%%                    bare_conf.tex, bare_jrnl.tex, bare_conf_compsoc.tex,
%%                    bare_jrnl_compsoc.tex, bare_jrnl_transmag.tex
%%*************************************************************************


% *** Authors should verify (and, if needed, correct) their LaTeX system  ***
% *** with the testflow diagnostic prior to trusting their LaTeX platform ***
% *** with production work. IEEE's font choices and paper sizes can       ***
% *** trigger bugs that do not appear when using other class files.       ***                          ***
% The testflow support page is at:
% http://www.michaelshell.org/tex/testflow/



\documentclass[conference]{IEEEtran}
%
% If IEEEtran.cls has not been installed into the LaTeX system files,
% manually specify the path to it like:
% \documentclass[journal]{../sty/IEEEtran}





% Some very useful LaTeX packages include:
% (uncomment the ones you want to load)


% *** MISC UTILITY PACKAGES ***
%
%\usepackage{ifpdf}
% Heiko Oberdiek's ifpdf.sty is very useful if you need conditional
% compilation based on whether the output is pdf or dvi.
% usage:
% \ifpdf
%   % pdf code
% \else
%   % dvi code
% \fi
% The latest version of ifpdf.sty can be obtained from:
% http://www.ctan.org/tex-archive/macros/latex/contrib/oberdiek/
% Also, note that IEEEtran.cls V1.7 and later provides a builtin
% \ifCLASSINFOpdf conditional that works the same way.
% When switching from latex to pdflatex and vice-versa, the compiler may
% have to be run twice to clear warning/error messages.






% *** CITATION PACKAGES ***
%
\usepackage{cite}
\usepackage{verbatim}
% cite.sty was written by Donald Arseneau
% V1.6 and later of IEEEtran pre-defines the format of the cite.sty package
% \cite{} output to follow that of IEEE. Loading the cite package will
% result in citation numbers being automatically sorted and properly
% "compressed/ranged". e.g., [1], [9], [2], [7], [5], [6] without using
% cite.sty will become [1], [2], [5]--[7], [9] using cite.sty. cite.sty's
% \cite will automatically add leading space, if needed. Use cite.sty's
% noadjust option (cite.sty V3.8 and later) if you want to turn this off
% such as if a citation ever needs to be enclosed in parenthesis.
% cite.sty is already installed on most LaTeX systems. Be sure and use
% version 5.0 (2009-03-20) and later if using hyperref.sty.
% The latest version can be obtained at:
% http://www.ctan.org/tex-archive/macros/latex/contrib/cite/
% The documentation is contained in the cite.sty file itself.






% *** GRAPHICS RELATED PACKAGES ***
%
  \usepackage[pdftex]{graphicx}
  \usepackage{caption}
  \usepackage{subcaption}
  \usepackage[font=footnotesize,labelfont=footnotesize]{subcaption}
  \usepackage[font=footnotesize,labelfont=footnotesize]{caption}
  \usepackage{epstopdf}

  %\usepackage{subfig}
  
  % declare the path(s) where your graphic files are
  % \graphicspath{{../pdf/}{../jpeg/}}
  % and their extensions so you won't have to specify these with
  % every instance of \includegraphics
  % \DeclareGraphicsExtensions{.pdf,.jpeg,.png}
  % or other class option (dvipsone, dvipdf, if not using dvips). graphicx
  % will default to the driver specified in the system graphics.cfg if no
  % driver is specified.
  % \usepackage[dvips]{graphicx}
  % declare the path(s) where your graphic files are
  % \graphicspath{{../eps/}}
  % and their extensions so you won't have to specify these with
  % every instance of \includegraphics
  % \DeclareGraphicsExtensions{.eps}
% graphicx was written by David Carlisle and Sebastian Rahtz. It is
% required if you want graphics, photos, etc. graphicx.sty is already
% installed on most LaTeX systems. The latest version and documentation
% can be obtained at: 
% http://www.ctan.org/tex-archive/macros/latex/required/graphics/
% Another good source of documentation is "Using Imported Graphics in
% LaTeX2e" by Keith Reckdahl which can be found at:
% http://www.ctan.org/tex-archive/info/epslatex/
%
% latex, and pdflatex in dvi mode, support graphics in encapsulated
% postscript (.eps) format. pdflatex in pdf mode supports graphics
% in .pdf, .jpeg, .png and .mps (metapost) formats. Users should ensure
% that all non-photo figures use a vector format (.eps, .pdf, .mps) and
% not a bitmapped formats (.jpeg, .png). IEEE frowns on bitmapped formats
% which can result in "jaggedy"/blurry rendering of lines and letters as
% well as large increases in file sizes.
%
% You can find documentation about the pdfTeX application at:
% http://www.tug.org/applications/pdftex


% *** MATH PACKAGES ***
\usepackage[cmex10]{amsmath}
\usepackage{amssymb}
\usepackage{amsfonts}
\usepackage{relsize}
\usepackage[capitalise]{cleveref}
\usepackage{cleveref}
%\usepackage{hyperref}
\newcommand{\crefrangeconjunction}{--}
\newcommand{\crefconjunction}{--}
\usepackage{amsthm}
\newtheorem{theorem}{Theorem}[section]
\newtheorem{corollary}{Corollary}[theorem]
\newtheorem{lemma}{Lemma}
\Crefname{figure}{Fig.}{Figs.}

% A popular package from the American Mathematical Society that provides
% many useful and powerful commands for dealing with mathematics. If using
% it, be sure to load this package with the cmex10 option to ensure that
% only type 1 fonts will utilized at all point sizes. Without this option,
% it is possible that some math symbols, particularly those within
% footnotes, will be rendered in bitmap form which will result in a
% document that can not be IEEE Xplore compliant!
%
% Also, note that the amsmath package sets \interdisplaylinepenalty to 10000
% thus preventing page breaks from occurring within multiline equations. Use:
%\interdisplaylinepenalty=2500
% after loading amsmath to restore such page breaks as IEEEtran.cls normally
% does. amsmath.sty is already installed on most LaTeX systems. The latest
% version and documentation can be obtained at:
% http://www.ctan.org/tex-archive/macros/latex/required/amslatex/math/





% *** SPECIALIZED LIST PACKAGES ***
%
%\usepackage{algorithmic}
%\usepackage[plain]{algorithm}
\usepackage{algpseudocode}
% algorithmic.sty was written by Peter Williams and Rogerio Brito.
% This package provides an algorithmic environment fo describing algorithms.
% You can use the algorithmic environment in-text or within a figure
% environment to provide for a floating algorithm. Do NOT use the algorithm
% floating environment provided by algorithm.sty (by the same authors) or
% algorithm2e.sty (by Christophe Fiorio) as IEEE does not use dedicated
% algorithm float types and packages that provide these will not provide
% correct IEEE style captions. The latest version and documentation of
% algorithmic.sty can be obtained at:
% http://www.ctan.org/tex-archive/macros/latex/contrib/algorithms/
% There is also a support site at:
% http://algorithms.berlios.de/index.html
% Also of interest may be the (relatively newer and more customizable)
% algorithmicx.sty package by Szasz Janos:
% http://www.ctan.org/tex-archive/macros/latex/contrib/algorithmicx/


% *** ALIGNMENT PACKAGES ***
%
\usepackage{array}
% Frank Mittelbach's and David Carlisle's array.sty patches and improves
% the standard LaTeX2e array and tabular environments to provide better
% appearance and additional user controls. As the default LaTeX2e table
% generation code is lacking to the point of almost being broken with
% respect to the quality of the end results, all users are strongly
% advised to use an enhanced (at the very least that provided by array.sty)
% set of table tools. array.sty is already installed on most systems. The
% latest version and documentation can be obtained at:
% http://www.ctan.org/tex-archive/macros/latex/required/tools/


% IEEEtran contains the IEEEeqnarray family of commands that can be used to
% generate multiline equations as well as matrices, tables, etc., of high
% quality.

\usepackage{mdwmath}
\usepackage{mdwtab}


% *** SUBFIGURE PACKAGES ***
%\ifCLASSOPTIONcompsoc
%  \usepackage[caption=false,font=normalsize,labelfont=sf,textfont=sf]{subfig}
%\else
%  \usepackage[caption=false,font=footnotesize]{subfig}
%\fi
% subfig.sty, written by Steven Douglas Cochran, is the modern replacement
% for subfigure.sty, the latter of which is no longer maintained and is
% incompatible with some LaTeX packages including fixltx2e. However,
% subfig.sty requires and automatically loads Axel Sommerfeldt's caption.sty
% which will override IEEEtran.cls' handling of captions and this will result
% in non-IEEE style figure/table captions. To prevent this problem, be sure
% and invoke subfig.sty's "caption=false" package option (available since
% subfig.sty version 1.3, 2005/06/28) as this is will preserve IEEEtran.cls
% handling of captions.
% Note that the Computer Society format requires a larger sans serif font
% than the serif footnote size font used in traditional IEEE formatting
% and thus the need to invoke different subfig.sty package options depending
% on whether compsoc mode has been enabled.
%
% The latest version and documentation of subfig.sty can be obtained at:
% http://www.ctan.org/tex-archive/macros/latex/contrib/subfig/




% *** FLOAT PACKAGES ***
%
%\usepackage{fixltx2e}
% fixltx2e, the successor to the earlier fix2col.sty, was written by
% Frank Mittelbach and David Carlisle. This package corrects a few problems
% in the LaTeX2e kernel, the most notable of which is that in current
% LaTeX2e releases, the ordering of single and double column floats is not
% guaranteed to be preserved. Thus, an unpatched LaTeX2e can allow a
% single column figure to be placed prior to an earlier double column
% figure. The latest version and documentation can be found at:
% http://www.ctan.org/tex-archive/macros/latex/base/


\usepackage{stfloats}
% stfloats.sty was written by Sigitas Tolusis. This package gives LaTeX2e
% the ability to do double column floats at the bottom of the page as well
% as the top. (e.g., "\begin{figure*}[!b]" is not normally possible in
% LaTeX2e). It also provides a command:
%\fnbelowfloat
% to enable the placement of footnotes below bottom floats (the standard
% LaTeX2e kernel puts them above bottom floats). This is an invasive package
% which rewrites many portions of the LaTeX2e float routines. It may not work
% with other packages that modify the LaTeX2e float routines. The latest
% version and documentation can be obtained at:
% http://www.ctan.org/tex-archive/macros/latex/contrib/sttools/
% Do not use the stfloats baselinefloat ability as IEEE does not allow
% \baselineskip to stretch. Authors submitting work to the IEEE should note
% that IEEE rarely uses double column equations and that authors should try
% to avoid such use. Do not be tempted to use the cuted.sty or midfloat.sty
% packages (also by Sigitas Tolusis) as IEEE does not format its papers in
% such ways.
% Do not attempt to use stfloats with fixltx2e as they are incompatible.
% Instead, use Morten Hogholm'a dblfloatfix which combines the features
% of both fixltx2e and stfloats:
%
% \usepackage{dblfloatfix}
% The latest version can be found at:
% http://www.ctan.org/tex-archive/macros/latex/contrib/dblfloatfix/




%\ifCLASSOPTIONcaptionsoff
%  \usepackage[nomarkers]{endfloat}
% \let\MYoriglatexcaption\caption
% \renewcommand{\caption}[2][\relax]{\MYoriglatexcaption[#2]{#2}}
%\fi
% endfloat.sty was written by James Darrell McCauley, Jeff Goldberg and 
% Axel Sommerfeldt. This package may be useful when used in conjunction with 
% IEEEtran.cls'  captionsoff option. Some IEEE journals/societies require that
% submissions have lists of figures/tables at the end of the paper and that
% figures/tables without any captions are placed on a page by themselves at
% the end of the document. If needed, the draftcls IEEEtran class option or
% \CLASSINPUTbaselinestretch interface can be used to increase the line
% spacing as well. Be sure and use the nomarkers option of endfloat to
% prevent endfloat from "marking" where the figures would have been placed
% in the text. The two hack lines of code above are a slight modification of
% that suggested by in the endfloat docs (section 8.4.1) to ensure that
% the full captions always appear in the list of figures/tables - even if
% the user used the short optional argument of \caption[]{}.
% IEEE papers do not typically make use of \caption[]'s optional argument,
% so this should not be an issue. A similar trick can be used to disable
% captions of packages such as subfig.sty that lack options to turn off
% the subcaptions:
% For subfig.sty:
% \let\MYorigsubfloat\subfloat
% \renewcommand{\subfloat}[2][\relax]{\MYorigsubfloat[]{#2}}
% However, the above trick will not work if both optional arguments of
% the \subfloat command are used. Furthermore, there needs to be a
% description of each subfigure *somewhere* and endfloat does not add
% subfigure captions to its list of figures. Thus, the best approach is to
% avoid the use of subfigure captions (many IEEE journals avoid them anyway)
% and instead reference/explain all the subfigures within the main caption.
% The latest version of endfloat.sty and its documentation can obtained at:
% http://www.ctan.org/tex-archive/macros/latex/contrib/endfloat/
%
% The IEEEtran \ifCLASSOPTIONcaptionsoff conditional can also be used
% later in the document, say, to conditionally put the References on a 
% page by themselves.




% *** PDF, URL AND HYPERLINK PACKAGES ***
%
%\usepackage{url}
% url.sty was written by Donald Arseneau. It provides better support for
% handling and breaking URLs. url.sty is already installed on most LaTeX
% systems. The latest version and documentation can be obtained at:
% http://www.ctan.org/tex-archive/macros/latex/contrib/url/
% Basically, \url{my_url_here}.




% *** Do not adjust lengths that control margins, column widths, etc. ***
% *** Do not use packages that alter fonts (such as pslatex).         ***
% There should be no need to do such things with IEEEtran.cls V1.6 and later.
% (Unless specifically asked to do so by the journal or conference you plan
% to submit to, of course. )

\usepackage{amsfonts}
\usepackage[fleqn,tbtags]{mathtools}


\usepackage{multirow}

% correct bad hyphenation here
%\newcommand{\subparagraph}{}
%\usepackage{titlesec}

\def\dashfill{\cleaders\hbox{-~-}\hfill}

\usepackage[ruled]{algorithm2e}
%\usepackage{algorithmicx}
\newcommand{\blue}{\color{blue}}
\usepackage{color,setspace}
\definecolor{blue}{rgb}{0.2,0.3,0.8}
\newcommand{\red}{\color{red}}


\def\tr{\mbox{${\mbox{tr}}$}}
\def\dashfill{\cleaders\hbox{-~-}\hfill}
\makeatletter
\newcommand*{\rom}[1]{\expandafter\@slowromancap\romannumeral #1@}
\makeatother
% % % % % % % % % % % % % % % % % % % % % % % % % % % % % % % % % % % %

\hyphenation{op-tical net-works semi-conduc-tor}

\begin{document}
\title{Sensor Selection and Power Allocation via Maximizing Bayesian Fisher Information for Distributed Vector Estimation}

\author{\IEEEauthorblockN{Mojtaba Shirazi, Alireza Sani, and Azadeh Vosoughi}
\IEEEauthorblockA{Department of Electrical Engineering and Computer Science\\
University of Central Florida\\
Email: {\tt \small mojsh@knights.ucf.edu, Alireza@knights.ucf.edu, azadeh@ucf.edu}}}
%Point of Contact: Mojtaba Shirazi, Tel:407-968-8548, Address:2314 River park Circle, Apt 2125, Orlando, Florida\\
%Review Topics: D.Signal Processing and Adaptive Systems}
\maketitle
\begin{abstract}
In this paper we study the problem of distributed estimation of a Gaussian vector with linear observation model in a wireless sensor network (WSN) consisting of $K$ sensors that transmit their modulated quantized observations over orthogonal erroneous wireless channels (subject to fading and noise) to a fusion center, which estimates the unknown vector. Due to limited network transmit power, only a subset of sensors can be active at each task period. Here, we formulate the problem of sensor selection and transmit power allocation that maximizes the trace of Bayesian Fisher Information Matrix (FIM) under network transmit power constraint, and propose three algorithms to solve it. Simulation results demonstarte the superiority of these algorithms compared to the algorithm that uniformly allocates power among all sensors.
%and propose three algorithms USU, greedy, and MCKP to solve it. Simulation results compare the performance of these algorithms. We observe that MCKP could reach the optimal solution in a very short time compared to other algorithms, and USU, which is very simple to implement, has a close performance to othaer algorithms.  
\end{abstract}
% Note that keywords are not normally used for peerreview papers.
\begin{IEEEkeywords}
Distributed estimation, linear observation model, Bayesian Fisher Information Matrix, sensor selection, power allocation, multiple-choice Knapsack problem.
\end{IEEEkeywords}
% For peer review papers, you can put extra information on the cover
% page as needed:
% \ifCLASSOPTIONpeerreview
% \begin{center} \bfseries EDICS Category: 3-BBND \end{center}
% \fi
%
% For peerreview papers, this IEEEtran command inserts a page break and
% creates the second title. It will be ignored for other modes.
\IEEEpeerreviewmaketitle
\section{Introduction} \label{Introduction}
Researchers have made tremendous contributions to the study of distributed estimation problem for energy constrained sensor networks \cite{Vosoughi_Sani_2016,Goldsmith_2006,AlRegib_2009,Giannakis_2008,Vandendorpe_2012}. Furthermore, significant progress has been made toward data selection approaches and specifically designing distributed sensor selection/activation algorithms to optimize the performance of resource constrained sensor networks \cite{Matin_2,Matin_3,Mahleqa_M3,IEEE_nights_saeid_comm_let,Boyd_sensor_selection_tsp2009,Leus_tsp_2015,Varshney_sensor_sel_and_colab_tsp_2015,Varshney_sensor_selection_tsp2016,Mitra_tsp_2008,IEEE_nights_Parsa2}. The authors in \cite{Boyd_sensor_selection_tsp2009,Leus_tsp_2015} studied the sensor selection problem that minimizes parameter estimation error, and described a heuristic algorithm, based on convex optimization, that approximately solves this problem. \cite{Leus_tsp_2015} focused on a general nonlinear observation model and formulated the sensor selection problem as the design of a sparse vector, considering several functions of the Cram\'{e}r-Rao Bound (CRB) performance measures. \cite{Varshney_sensor_sel_and_colab_tsp_2015} introduced a unified framework to jointly design the optimal sensor selection and collaboration schemes subject to a certain information or energy constraint. The authors in \cite{Varshney_sensor_selection_tsp2016,Mitra_tsp_2008} considered the problem of sensor selection for parameter estimation under a network transmit power constraint. In particular, \cite{Varshney_sensor_selection_tsp2016} considered a linear observation model with correlated measurement noises and sought optimal sensor activation algorithm by formulating an optimization problem, in which the trace of Fisher Information Matrix (FIM) is maximized subject to energy constraints. \cite{Mitra_tsp_2008} considered estimation of a function of a random parameter, and analyzed the effect of measurement noise variance on the optimal power allocation, for different network topologies. The authors in \cite{IEEE_nights_Parsa2} developed a multi-tier distributed computing infrastructure for sensor selection and data allocation using mobile device cloud, where all tasks are executed in parallel by sharing the workload among multiple nearby sensors.
%In a recent work, \cite{Leus_ICASSP_2017} studied the sensor selection problem for field estimation, and proposed a decentralized architecture where sensor selection can be carried out in a distributed way and by the sensors themselves. 
%
\begin{figure*}[t]
	\centering
	\includegraphics[width=5.6in]{system_model-eps-converted-to.pdf}
	\vspace{-0.1cm}
	\caption{Our system model consists of $K$ sensors and a FC, that is tasked with estimating a Gaussian vector {\boldmath$\theta$}, via fusing collective received signals.}
	\label{system-model}
	\vspace{-.5cm}
\end{figure*}
%
%{\blue
In our previous works \cite{Shirazi_PIMRC2014}\cite{Shirazi_asilomar2014}, we presented our preliminary results on deriving Bayesian CRB matrix and studied the behavior of its trace, with respect to the system parameters. In our recent work \cite{Shirazi_Vosoughi_journal_2017}, considering the problem of distributed estimation of a Gaussian random vector with linear observation model, we derived the Bayesian FIM. We studied two transmit power optimization problems that maximize the trace of FIM and log-determinant of FIM under network transmit power constraint, and demonstrated that the estimation performance significantly enhances, using the transmit power allocation solutions corresponding to these two problems, compared with uniform power allocation among all sensors. Furthermore, a comparison was made between the estimation performance of coherent and noncoherent reception at the fusion center (FC).

{\it Adopting the same system model as in \cite{Shirazi_Vosoughi_journal_2017}, in this paper we study the sensor selection and power allocation problem that maximizes the trace of Bayesian FIM under network transmit power constraint.} We propose three algorithms to address this problem, and compare the performances of these algorithms in simulation results. 
%}
%{\bf Notations}: Throughout this paper, matrices are denoted by bold uppercase letters, vectors by bold lowercase letters, and scalars by normal letters. Note that $\mathbb{E}$ denotes the mathematical expectation operator, $||.||$ and $[.]^T$ represent the $L^2$ norm of a vector and the matrix-vector transpose operation, respectively. tr(.) and $|.|$ indicate trace and determinant of a matrix, respectively, and $|\cal A|$ is the cardinality of set $\cal A$.
%=======================================
%=======================================
%=======================================
\section{System Model and Problem Statement} \label{System Model}
We consider a network of $K$ sensors,
%with system model shown in Fig.~\ref{system-model}, 
where each sensor observes a common zero-mean Gaussian vector {\boldmath$\theta$}= $[\theta_1, \theta_2,..., \theta_q]^T\!\in\!\mathbb{R}^q$ with covariance matrix $\boldsymbol{\mathcal C}_{\boldsymbol{\theta}}=\mathbb{E}\{\boldsymbol{\theta}\boldsymbol{\theta}^T\}$. 
%We make two core assumptions for our system model:\\
%$\underline{assumption} 1)$ $n_k$ denotes zero-mean Gaussian observation noise with variance $\sigma_{n_k}^2$ at sensor $k$. We assume that $n_k$'s are uncorrelated across the sensors and also independent of {\boldmath$\theta$}.\\
%$\underline{assumption} 2)$ Modulated symbols are sent to the FC over orthogonal channels, subject to independent flat fading $h_k$ and zero-mean Gaussian receiver noise $w_k$ with variance $2\sigma_{w_k}^2$, that are independent across the sensors, and also independent of communication channels' inputs.
We assume a linear observation model for sensor $k$ as:
%
\begin{equation} \label{obs_model}
x_k=\mathbf{a}_k^T \boldsymbol{\theta}+n_k, \ \ \ \ \ \ k=1,..., K
\end{equation}
%
where $\mathbf{a}_k\!=\![a_{k_1}, a_{k_2},..., a_{k_q}]^T\!\in\!\mathbb{R}^q$ is known observation gain vector and $n_k$ denotes zero-mean Gaussian observation noise with variance $\sigma_{n_k}^2$. We assume that $n_k$'s are uncorrelated observation noises across sensors and also are uncorrelated with {\boldmath$\theta$}. Sensor $k$ employs a uniform scalar quantizer with $M_k\!=\!2^{L_k}$ quantization levels $m_{k,l}\!=\!\frac{(2l-1-M_k)\Delta_k}{2}$ for $l\!=\!1,...,M_k$, where $\Delta_k$ denotes quantization step size and index $l$ indicates the quantization level $m_{k,l}$. We assume $p(|x_k|\!\geq\!\tau_k)\! \approx \! 0$ for some $\tau_k$ value. Hence, we choose ${\Delta_k} \!= \! \frac{2\tau_k}{(2^{L_k}-1)}$  \cite{Vandendorpe_2012,Vosoughi_Sani_2016}. The quantizer maps $x_k$ to one of the quantization levels $m_k \in \{m_{k,1},...,m_{k,M_k}\}$. 
%as the following: 
%
%\begin{equation*} 
%m_k=m_{k,l},\ \ \ \text{for}\ \ x_k\in [u_{k,l},u_{k,l+1}],\ \ l=1,...,M_k\\
%\end{equation*}
%
%where $u_{k,l}\!=\!\frac{(2l-2-M_k)\Delta_k}{2},\ l\!=\!2,...,M_k$, are quantization boundaries with $u_{k,1}$ and $u_{k,M_k+1}$ denoting the largest lower bound and the smallest upper bound on $x_k$, respectively. We let $u_{k,1}\!=\!-\infty$ and $u_{k,M_k+1}\!=\!+\infty$. 
Following quantization, sensor $k$ employs a fixed length encoder, which encodes the index $l$ corresponding to the quantization level $m_{k,l}$ to a binary sequence of length $L_k=\log_2 M_k$ according to natural binary encoding \cite{Vandendorpe_2012,Vosoughi_Sani_2016}, and finally modulates these $L_k$ bits into $L_k$ binary symbols. Let $P_k$ denote transmit power corresponding to $L_k$ symbols from sensor $k$, which is equally distributed among $L_k$ symbols. Each sensor employs a Binary Phase Shift Keying (BPSK) modulator, which maps each bit of $L_k$-bit sequence into one symbol with transmit power $P_k/L_k$.

Sensors send their modulated symbols to the FC over orthogonal flat fading channels, with complex-valued fading coefficient $h_k$. 
%and zero-mean complex Gaussian additive receiver noise $w_k$ with variance $2\sigma_{w_k}^2$, that are uncorrelated across channels. Also, $\phi_k$ is the $k$-th channel phase, $|h_k|$ is the $k$-th channel envelope, which is assumed to be Rayleigh distributed with scale parameter $\sigma_{h_k}$.
%\begin{equation*} 
%f(|h_k|;\sigma_{h_k})=\frac{|h_k|}{\sigma_{h_k}^2}e^{-|h_k|^2/(2\sigma_{h_k}^2)}
%\end{equation*}
%where $\sigma_h$ is the scale parameter of the distribution.
Suppose channel $h_k$ remains constant during the transmission of $L_k$ symbols. 
Denote $\nu_{k,i}$ as communication channel noise during the transmission of $i$-th symbol of $L_k$ symbols corresponding to sensor $k$. We assume $\nu_{k,i}$'s are independent and identically distributed across $L_k$ transmitted symbols and $K$ channels, $\nu_{k,i}\sim \mathcal{CN}\left(0,2\sigma_{\nu_k}^2\right)$. Furthermore, there is a constraint on the network transmit power, i.e., $\sum_{k=1}^{K}P_k\leq P_{tot}$, and thus, only a subset of sensors might be active at each task period.

Let $\hat{m}_{k}$ denote the recovered quantization level corresponding to sensor $k$, where in general, $\hat{m}_{k}\neq m_{k}$ due to communication channel errors. The FC processes channel output corresponding to sensor $k$ to recover transmitted quantization level $\hat{m}_k \in \{\hat{m}_{k,1},...,\hat{m}_{k,M_k}\}$. Having $\{\hat{m}_1,...,\hat{m}_K\}$, the FC applies a Bayesian estimator to form the estimate $\hat{\boldsymbol{\theta}}$. 
%
%\begin{figure*}[t]
%	\centering
%	\includegraphics[width=6in]{system_model.eps}
%	%\vspace{-0.1cm}
%	\caption{Our system model consists of $K$ sensors and a FC, that is tasked with estimating a Gaussian vector {\boldmath$\theta$}, via fusing collective received signals.}
	%\label{system-model}
	%\vspace{-.5cm}
%\end{figure*}
%
%Define vector $\boldsymbol{\hat{m}}=[\hat{m}_1,...,\hat{m}_K]^T$ that includes recovered quantization levels at the FC.
% % % % % % % % % % % % % % % % % % % % % % % 
\begin{comment}
Under certain regularity conditions that are satidfied by Gaussian vectors, the $q\times q$ FIM, denoted as $\boldsymbol{J}$, is defined as \cite{Van_Trees_estimation_book,Vosoughi2006sp2,Vosoughi2006sp1}:
%
%\begin{equation} \label{FIM_Van_Trees}
%\boldsymbol{J}=\mathbb{E}\left\{\left(\frac{\partial \ln p(\boldsymbol{\hat{m}},\boldsymbol{\theta})}{\partial \boldsymbol{\theta}}\right)\left(\frac{\partial \ln p(\boldsymbol{\hat{m}},\boldsymbol{\theta})}{\partial \boldsymbol{\theta}}\right)^T\right\},
%\end{equation}
%
%where the expectation is taken over $p(\boldsymbol{\hat{m}},\boldsymbol{\theta})$.
%
\begin{equation} \label{FIM_Van_Trees}
\boldsymbol{J}\!=\!\mathbb{E}\!\left\{\!\!\left(\!\frac{\partial \ln p(\hat{m}_1,...,\hat{m}_K,\boldsymbol{\theta})}{\partial \boldsymbol{\theta}}\!\right)\!\!\left(\!\frac{\partial \ln p(\hat{m}_1,...,\hat{m}_K,\boldsymbol{\theta})}{\partial \boldsymbol{\theta}}\!\right)^T\!\right\},
\end{equation}
%
where the expectation is taken over $p(\hat{m}_1,...,\hat{m}_K,\boldsymbol{\theta})$.  
We refer the interested readers to section IV of \cite{Shirazi_Vosoughi_journal_2017} for the derivation of matrix $\boldsymbol{J}$. Our goal is to study sensor selection and transmit power allocation that maximizes tr($\boldsymbol{J}$) \cite{Varshney_sensor_selection_tsp2016}, subject to network transmit power constraint. In other words, we are interested in solving the following constrained optimization problem:
%
\begin{align*} \tag{P1}\label{power aloc sen selection problem}
\mathop{\text{maximize}}_{P_k, w_k, \forall k}\ \ \ \ &\sum_{k=1}^{K}w_kt_k(P_k)\nonumber\\
\text{s.t.}\ \ \ \ &\sum_{k=1}^{K}P_k\leq P_{tot},\ P_k\in \mathbb{R}^{+},\ \forall k\nonumber\\
&w_k\in \{0,1\},\ \forall k
\end{align*}
%
\end{comment}  
Under certain regularity conditions that are satidfied by Gaussian vectors, one can derive the $q\times q$ FIM, denoted here as $\boldsymbol{J}$. We refer the interested readers to section IV of \cite{Shirazi_Vosoughi_journal_2017} for the derivation of matrix $\boldsymbol{J}$. From (30) in \cite{Shirazi_Vosoughi_journal_2017}, we obtain:
%
\begin{eqnarray} \label{trace of J}
\text{tr}(\boldsymbol{J})&=&\text{tr}\left({\boldsymbol{\cal C}_{\boldsymbol{\theta}}}^{-1}\right)+\sum_{k=1}^{K}t_k(P_k),\\
t_k(P_k)&=&\frac{\mathbf{a}_k^T\mathbf{a}_k}{2\pi\sigma_{n_k}^2}\mathbb{E}_{\boldsymbol{\theta}}\{G_k(\boldsymbol{\theta},P_k)\},\nonumber\\
G_k(\boldsymbol{\theta},P_k)&=&\sum_{t=1}^{M_k}\frac{\left(\sum_{l=1}^{M_k}\alpha_{k,t,l}(P_k)\dot{\beta}_{k,l}
(\boldsymbol{\theta})\right)^2}{\sum_{l=1}^{M_k}\alpha_{k,t,l}(P_k)\beta_{k,l}(\boldsymbol{\theta})}.\nonumber 
\end{eqnarray}
%
Note that $t_k$ is a function of $P_k$. Due to the cap on the network transmit power,
only a subset of the sensors might be active at each task
period. So we introduce a sensor selection parameter $w_k\in\{0,1\}$, to indicate whether or not sensor $k$ is selected to participate in the distributed estimation task and transmit to the FC. {\it Our goal is to study sensor selection and transmit power allocation that maximizes tr($\boldsymbol{J}$), subject to network transmit power constraint. In other words, we are interested in solving the following constrained optimization problem}:
%
\begin{subequations}
\begin{align*} \tag{P1}\label{power aloc sen selection problem}
\mathop{\text{maximize}}_{P_k, w_k, \forall k}\ \ \ \ &\sum_{k=1}^{K}w_kt_k(P_k)\nonumber\\
\text{s.t.}\ \ \ \ &\sum_{k=1}^{K}P_k\leq P_{tot},\ P_k\in \mathbb{R}^{+},\ \forall k\\
&w_k\in \{0,1\},\ \forall k
\end{align*}
\end{subequations}
%
\begin{comment}
%
\begin{align*} \tag{P0}\label{maximization problem of tr(J)}
\mathop{\text{maximize}}_{P_k,\forall k}\ \ \ \ &\text{tr}\left(\boldsymbol{J}\left(\{P_k\}_{k=1}^K\right)\right)\nonumber\\
\text{s.t.}\ \ \ \ &\sum_{k=1}^{K}P_k\leq P_{tot},\ P_k\in \mathbb{R}^{+},\ \forall k
%\text{P_k is a positive real number}
\end{align*}
% 
In subsection VI-A of \cite{Shirazi_Vosoughi_journal_2017}, we argued that this problem is concave, and therefore, its global solution can be found using Newton's method. In the following section, we propose different methods to address this problem, and finally compare their results.
\end{comment}  
%=======================================
%=======================================
%=======================================
\section{Constrained Maximization of Bayesian Fisher Information} \label{Power Allocation} 
%From (30) in \cite{Shirazi_Vosoughi_journal_2017}, we obtain:
%
%\begin{equation} \label{trace of J}
%\text{tr}(\boldsymbol{J})=\text{tr}\left({\boldsymbol{\cal C}_{\boldsymbol{\theta}}}^{-1}\right)+\sum_{k=1}^{K}t_k,
%\end{equation}
%
%where we defined $t_k\!=\!\frac{\mathbf{a}_k^T\mathbf{a}_k}{2\pi\sigma_{n_k}^2}\mathbb{E}\{G_k(\boldsymbol{\theta})\}$. Note that $t_k$ is a function of $P_k$.
%, which we denote this subset as $S_{\cal A}$. We insert a sensor selection parameter $w_k$ to \eqref{trace of J}, and therefore, the maximization problem \eqref{maximization problem of tr(J)} becomes:
The constrained optimization problem in \eqref{power aloc sen selection problem} is a nonconvex mixed integer non-linear programming problem which is NP-hard combinatorial problem and its computational complexity
increases exponentially with the problem size. 

In the following, we propose three different algorithms to tackle \eqref{power aloc sen selection problem}. In the first two algorithms, we use two different relaxations of \eqref{power aloc sen selection problem} and in the third algorithm, we propose a reformulation of \eqref{power aloc sen selection problem} and solve it. As part of the first algorithm, relaxation \eqref{Boolean relaxed sensor selection} is obtained based on uniform transmit power allocation among all sensors and relaxation of the Boolean constraints in \eqref{power aloc sen selection problem}. Problem \eqref{Boolean relaxed sensor selection} becomes a linear programing problem and we solve it using simplex method. We refer to the first algorithm as uniform-select-uniform (USU) algorithm. In the second algorithm, we iteratively select a new sensor in a greedy manner until a stopping criteria is met. As part of the second algorithm, relaxation \eqref{max tr(J) for k in U_j} is obtained when we assume $\{w_k\}_{k=1}^{K}$ is given. Problem \eqref{max tr(J) for k in U_j} becomes a convex problem and we find the solution using Newton's method. We refer to the second algorithm as greedy algorithm. The third formulation \eqref{MCKP} is obtained based on discretizing transmit power and is called in the literature a multiple-choice knapsack problem (MCKP). Although \eqref{MCKP} is an NP-hard problem, we can find the solution in pseudo-polynomial time using dynamic programming. We refer to the third algorithm as MCKP algorithm. Define $\boldsymbol{t}\!=\![t_1,..., t_K]^T$, and let $\boldsymbol{P}\!=\![P_1,..., P_K]^T$ and $\boldsymbol{w}\!=\![w_1,..., w_K]^T$ be the vectors of sensors' powers and sensor selection parameters, respectively. Suppose $\boldsymbol{P}^{*}$ and $\boldsymbol{w}^{*}$ are final solutions for the algorithms.\\
%\subsection{USU Algorithm}
$\bullet$ \underline{USU Algorithm}:
Initialize with $i=1$. Emplying uniform power allocation among all sensors, $\boldsymbol{P}$ and $\boldsymbol{t}$ can be obtained. Then, we find the best set of active sensors $S_{i+1}$ with cardinality $|S_{i+1}|=i$ via solving the following Boolean relaxed sensor selection problem:
%
\begin{align} \tag{P2}\label{Boolean relaxed sensor selection}
\mathop{\text{maximize}}_{\boldsymbol{w}}\ \ \ \ &\boldsymbol{w}^T\boldsymbol{t}\nonumber\\
\text{s.t.}\ \ \ \ &\boldsymbol{1}^T\boldsymbol{w}\leq i,\nonumber\\
&\boldsymbol{w}\in [0,1]^K.\nonumber
\end{align}
%
Given the set $S_{i+1}$ we uniformly allocate the power among the selected $i$ sensors, i.e., $P_k=P_{tot}/i,\ k\in S_{i+1}$ and repeat this procedure while incrementing $i$ until a stopping criteria is met. A summary is described in Algorithm \ref{USU algorithm}.\\
\begin{comment}
which is given as:
%
\begin{align} \tag{P2}\label{Boolean relaxed problem}
\mathop{\text{maximize}}_{P_k, w_k\forall k}\ \ \ \ &\sum_{k=1}^{K}w_kt_k(P_k)\nonumber\\
\text{s.t.}\ \ \ \ &\sum_{k=1}^{K}P_k\leq P_{tot},\ P_k\in \mathbb{R}^{+},\ \forall k\nonumber\\
&w_k\in [0,1],\ \forall k\nonumber
\end{align}
%
We propose two algorithms to solve \eqref{Boolean relaxed problem}: 1) USU, 2) greedy. 
%1) {\bf USU}: We first consider a uniform power allocation among all sensors, i.e., $P_k=P_{tot}/K,\ \forall k$. Having $\{P_k, t_k(P_k)\}_{k=1}^K$, simplex method can be used to efficiently solve the following linear programming problem:
%
%\begin{align} \tag{P3}\label{Boolean relaxed sensor selection}
%\mathop{\text{maximize}}_{w_k, \forall k}\ \ \ \ &\sum_{k=1}^{K}w_kt_k(P_k)\nonumber\\
%\text{s.t.}\ \ \ \ &w_k\in [0,1],\ \forall k\nonumber
%\end{align}
%
%Then to obtain the active set $S_{\cal A}$,

1) {\bf USU}: It is described in Algorithm 1.
%\vspace{-0.3cm}
\end{comment}
%\subsection{Greedy Algorithm}
$\bullet$ \underline{Greedy Algorithm}:
Define ${\cal A}$ and ${\cal I}$ as sets of active and inactive sensors, respectively. Initialize with $i=1$, ${\cal A}\!=\!\{\}$, ${\cal I}\!=\!\{1, \dots, K\}$. For each sensor in the inactive set ${\cal I}$, say ${\cal I}_j,\ j=1, ..., |{\cal I}|$, use Newton's method to find the optimal solution $\{P^{'}_k\}_{k\in {\cal U}_j}$ for the following power allocation problem:
 %
 \begin{align*} \tag{P3}\label{max tr(J) for k in U_j}
 \mathop{\text{maximize}}_{P_k,k\in {\cal U}_j}\ \ \ \ &\boldsymbol{1}^T\boldsymbol{t}\nonumber\\
 \text{s.t.}\ \ \ \ &\sum_{k\in {\cal U}_j}P_k\leq P_{tot},\ P_k\in \mathbb{R}^{+},\ \forall k\in {\cal U}_j,
 %\text{P_k is a positive real number}
 \end{align*}
 %
where ${\cal U}_j\!=\!{\cal A}\cup {\cal I}_j$, and save $Y_{j}=\text{tr}\left(\boldsymbol{J}\left(\{P^{'}_k\}_{k\in {\cal U}_j}\right)\right)$. In \cite{Shirazi_Vosoughi_journal_2017}, we argue that \eqref{max tr(J) for k in U_j} is a convex programming problem. Then we select the sensor ${\cal I}_j$ which gives the largest performance improvement, i.e. the one corresponding to $max\{Y_{1}, ..., Y_{|{\cal I}|}\}$. We repeat this procedure while updating ${\cal A}$ and ${\cal I}$ and incrementing $i$ until a stopping criteria is met. 
A summary is described in Algorithm \ref{greedy algorithm}.\\
%
\begin{algorithm}[t]
\caption{USU algorithm}
\label{USU algorithm}
 \KwData{System parameters defined in Section \ref{System Model}}
 \KwResult{Solution for vectors $\boldsymbol{P}^{*}, \boldsymbol{w}^{*}$}
 \vspace{-0.1cm}
 \hbox to \hsize{\dashfill\hfil}
  % % % % % % % % % %
\vspace{-0.1cm}
initialization\;  
$i=1$, $S_1=\{\}$, $T_1=0$, $\boldsymbol{P}^{*}=\boldsymbol{0}$, $\boldsymbol{w}^{*}=\boldsymbol{0}$,\\

\While{$1$}{

1: $P_k=P_{tot}/K,\ \forall k$. \\ 
2: Having $\boldsymbol{P}$ and $\boldsymbol{t}$, use simplex method to efficiently solve \eqref{Boolean relaxed sensor selection}.\\
%\vspace{-0.1cm}
%
%\begin{align*} %\label{Boolean relaxed sensor selection}
%\mathop{\text{maximize}}_{\boldsymbol{w}}\ \ \ \ &\boldsymbol{w}^T\boldsymbol{t}\nonumber\\
%\text{s.t.}\ \ \ \ &\boldsymbol{1}^T\boldsymbol{w}\leq i,\nonumber\\
%&\boldsymbol{w}\in [0,1]^K.
%\end{align*}\\
%
3: $S_{i+1}$ is the set of indices corresponding to $i$ largest values of $\boldsymbol{w}$.\\
4: $P_k=P_{tot}/i,\ k\in S_{i+1}$. \\
5: $T_{i+1}=\text{tr}\left(\boldsymbol{J}\left(\{P_k\}_{k\in S_{i+1}}\right)\right)$.\\
\If{$T_{i+1}\leq T_{i}$ ~$\vee$ ~$i\geq K$,}
{$w_k^{*}=1,\ k\in S_{i}$,\\
$P_k^{*}=P_{tot}/i,\ k\in S_{i}$.\\
Return $T_i$ as maximum value of objective function.\\
break
}
$i=i+1$.
% % % % % % % % % % % %
}
\vspace{-0.1cm}
\end{algorithm}
%
\begin{algorithm}[t]
\caption{greedy algorithm}
\label{greedy algorithm}
 \KwData{System parameters defined in Section \ref{System Model}}
  \KwResult{Solution for vectors $\boldsymbol{P}^{*}, \boldsymbol{w}^{*}$}
  \vspace{-0.1cm}
  \hbox to \hsize{\dashfill\hfil}
   % % % % % % % % % %
 \vspace{-0.1cm}
 initialization\;  
 $i\!=\!1$, ${\cal A}\!=\!\{\}$, ${\cal I}\!=\!\{1, \dots, K\}$, $T_1\!=\!0^{+}$, $\boldsymbol{P}^{*}\!=\!\boldsymbol{0}$, $\boldsymbol{w}^{*}\!=\!\boldsymbol{0}$,\\
 
 \While{$1$}{
 
 1: $\boldsymbol{Y}\!=\!\boldsymbol{0}$.\\
 2: \For{$j=1, ..., |{\cal I}|$}
 {2-1: ${\cal U}_j\!=\!{\cal A}\cup {\cal I}_j$.\\
 %2-2: Use Newton's method to solve the following power allocation problem:
 %
 %\begin{align*} \tag{P3}\label{max tr(J) for k in U_j}
 %\mathop{\text{maximize}}_{P_k,k\in {\cal U}_j}\ \ \ \ &\text{tr}\left(\boldsymbol{J}\left(P_k\right)\right)\nonumber\\
 %\text{s.t.}\ \ \ \ &\sum_{k\in {\cal U}_j}P_k\leq P_{tot},\ P_k\in \mathbb{R}^{+},\ \forall k\in {\cal U}_j.
 %\text{P_k is a positive real number}
 %\end{align*}
 %
 2-2: Solve \eqref{max tr(J) for k in U_j} to obtain the optimal solution $\{P^{'}_k\}_{k\in {\cal U}_j}$.\\
 2-3: $Y_{j}=\text{tr}\left(\boldsymbol{J}\left(\{P^{'}_k\}_{k\in {\cal U}_j}\right)\right)$.\\
 }
 3: $T_{i+1}=max\ \boldsymbol{Y}$.\\
 \If{$\frac{T_{i+1}-T_{i}}{T_{i}}\leq\epsilon_0$, (i.e., solution converged) ~$\vee$ ~$i\geq K$,}
 {$w_k^{*}=1,\ k\in {\cal A}$,\\
 $\{P_k^{*}\}_{k\in {\cal A}}=\{P^{'}_k\}_{k\in {\cal A}}$.\\
 Return $T_i$ as maximum value of objective function.\\
 break
 }
 4: Update ${\cal A}$ by setting ${\cal A}={\cal U}_{\{\underset{j} {\mbox{argmax}}\ \boldsymbol{Y}\}}$.\\
 5: Remove ${\cal I}_{\{\underset{j} {\mbox{argmax}}\ \boldsymbol{Y}\}}$ from ${\cal I}$.\\
 $i=i+1$.
 % % % % % % % % % % % %
 }
 \vspace{-0.1cm}
 \end{algorithm}
%
$\bullet$ \underline{MCKP Algorithm}:
%\subsection{MCKP Algorithm}
%We discretize $P_{tot}$ into $N$ samples $\boldsymbol{P}_d\!=\![P_{d_1},..., P_{d_N}]^T$ and obtain the matrix $\boldsymbol{T}$ as follows:
%
%\begin{align*} %\label{matrix T}
%\boldsymbol{T}=&\begin{bmatrix}
%t_1(P_{d_1}) & \cdots & t_1(P_{d_N}) \\
%\vdots & \ddots & \vdots \\
%t_K(P_{d_1}) & \cdots & t_K(P_{d_N}) 
%\end{bmatrix}.
%\end{align*}  
%
We discretize $P_{tot}$ into $N$ samples $\boldsymbol{P}_d\!=\![P_{d_1},..., P_{d_N}]^T$ and obtain matrix $\boldsymbol{T}$ such that its $(k,j)$th entry $[\boldsymbol{T}]_{k,j}\!=\!t_k(P_{d_j}),\ k\!=\!1, \dots, K,\ j\!=\!1, \dots, N$. Let $\boldsymbol{W}$ be a $K\!\times\!N$ matrix whose entries can be zeros or ones. Then \eqref{power aloc sen selection problem} becomes:
%
\begin{align} \tag{P4}\label{MCKP}
\mathop{\text{maximize}}_{\boldsymbol{W}}\ \ \ \ &\boldsymbol{1}^T\underbrace{(\boldsymbol{W}\circ\boldsymbol{T})}_{\mathclap{\text{Hadamard product}}}\boldsymbol{1}\nonumber\\
\text{s.t.}\ \ \ \ &\boldsymbol{1}^T(\boldsymbol{W}\circ(\boldsymbol{1}\boldsymbol{P}_d^T))\boldsymbol{1}\leq P_{tot},\nonumber\\
&[\boldsymbol{W}]_k\boldsymbol{1}=1,\ k\!=\!1, \dots, K,\nonumber\\ 
&\boldsymbol{W}\in \{0,1\}^{K\!\times\!N},\nonumber
\end{align}
%
in which $[\boldsymbol{W}]_k$ is $k$-th row of $\boldsymbol{W}$. The second constraint states that only one value from vector $\boldsymbol{P}_d$ should be chosen for each sensor, which in fact is the associated power of that sensor. Problem \eqref{MCKP} is a MCKP that is NP-hard, but it can be solved in pseudo-polynomial time through dynamic programming \cite{PISINGER_MCKP}. 
%We use {\it Cplex}, which is an optimization software, to solve the binary integer linear programming problem \eqref{MCKP}.
%=======================================
%=======================================
%=======================================
\section{Numerical and Simulation Results} \label{simulation}
We assume a zero mean Gaussian vector
$\boldsymbol{\theta}=\left[\theta_1,\theta_2\right]^T$ with $\boldsymbol{\cal C}_{\boldsymbol{\theta}}=[4,0.5;0.5,0.25]$, and $\sigma_{n_k}\!=\!1$, $\sigma_{\nu_k}\!=\!1$, $|h_k|\!=\!0.7$, $L_k=3$ for all $k$. We choose a proper $\tau_k$ based on the observation model and joint pdf of the unknown vector such that $p(|x_k|\geq\tau_k)\approx0$. To this end, we choose $\tau_k=3\sqrt{\sigma_{n_k}^2+\mathbf{a}_k^T\boldsymbol{\mathcal C}_{\boldsymbol{\theta}}\mathbf{a}_k}$. 
%=============================================
%\subsection{Homogeneous Sensor Network}

First, we consider a homogeneous sensor network with $\mathbf{a}_k=[0.6,0.8]^T$ for all $k$. Fig.~\ref{homog_max_tr_J} shows that an increase in $P_{tot}$ results in increase in $\text{tr}(\boldsymbol{J})$. Moreover, we are interested to know the optimal number of sensors that maximizes $\text{tr}(\boldsymbol{J})$. From Fig.~\ref{homog_num_of_sen} we observe that for a given $P_{tot}$, it is always beneficial to uniformly allocate $P_{tot}$ among all of the sensors. 
%\vspace{.5 cm}
%\subsection{Heterogeneous Sensor Network}

Next, we consider $K\!=\!20$ sensors that are randomly deployed in a $2m\times2m$ field. Assuming Cartesian coordinate system with origin at center of the field, the goal is to estimate two signal sources $\boldsymbol{\theta}=\left[\theta_1,\theta_2\right]^T$. The distance between $\theta_i$ located at $(x_{t_i},y_{t_i})$ and sensor $k$ located at $(x_{s_k},y_{s_k})$ is: 
%
\begin{equation*}
d_{ki}=\sqrt{(x_{s_k}-x_{t_i})^2+(y_{s_k}-y_{t_i})^2}, \ \ \ k=1,..., 20, \ \ i=1,2
\end{equation*}
Let $d_{0i}$ be distance of $\theta_i$ from origin, $d_{01}=d_{02}=1m$, and 
%To characterize sensing channels between signal sources and sensors we adopt an isotropic signal intensity attenuation model. In particular, we model $x_k$ as:
%
%\begin{equation*}
%x_k=\mathbf{a}_k^T \boldsymbol{\theta}+n_k, \ \ \ \ \ \ k=1,..., K
%\end{equation*}
% 
$\mathbf{a}_k^T=[(\frac{d_{01}}{d_{k1}})^n, (\frac{d_{02}}{d_{k2}})^n]$ where $n$ is signal decay exponent which is approximately 2 for distances $\leq\!1km$ \cite{Li_Eurasip_2003}. Fig.~\ref{heterog} compares the performance of 3 algorithms USU, greedy, and MCKP. We also consider uniform power allocation among all sensors, i.e., $\{P_k=P_{tot}/K\}_{k=1}^K$ where no sensor selection is carried out. We refer to this algorithm as UFA algorithm. For MCKP, our simulations suggest that $N=100$ samples are sufficient to reach the optimal solution. Fig.~\ref{heterog_max_tr_J} illustrates that greedy and MCKP algorithms perform similarly, and they both outperform USU algorithm, which is intuitive. However, the performance of USU algorithm is not far from other algorithms, which highlights the advantage of sensor selection even when power is uniformly allocated among sensors. Moreover, all three algorithms significantly outperform UFA algorithm. Fig.~\ref{heterog_num_of_sen} reveals that USU algorithm always selects less number of sensors compared to other algorithms, which can save energy in battery-powered sensor networks. 
%{\bf Remark 1}: MCKP algorithm could reach the optimal solution in a very short time compared to other algorithms, since we solve a binary integer linear programming problem which can be solved efficiently using optimization solvers. Furthermore, Fig.~\ref{heterog_num_of_sen} suggests that USU always selects less number of sensors compared to other algorithms, which can save energy in battery-powered sensor networks. 
%
\begin{figure}[h]
	\centering
	\subcaptionbox{\label{homog_max_tr_J}}{\vspace{-.25 cm}\includegraphics[width=3.5in]{homog_max_tr_J-eps-converted-to.pdf}}
	%\vspace{0cm}
	\subcaptionbox{\label{homog_num_of_sen}}{\vspace{-.25 cm}\includegraphics[width=3.5in]{homog_num_of_sen-eps-converted-to.pdf}}
	%\vspace{0cm}     
	\caption{Homogeneous sensor network: (a) tr($\boldsymbol{J}$), and (b) number of selected sensors, versus $P_{tot}$.}  
	%\vspace{-.5cm}
	\label{homog} 
\end{figure}
%
\begin{figure}[h]
	\centering
	\subcaptionbox{\label{heterog_max_tr_J}}{\vspace{-.25 cm}\includegraphics[ width=3.5in]{heterog_max_tr_J-eps-converted-to.pdf}}
	%\vspace{0cm}
	\subcaptionbox{\label{heterog_num_of_sen}}{\vspace{-.25 cm}\includegraphics[width=3.5in]{heterog_num_of_sen-eps-converted-to.pdf}}
	%\vspace{0cm}     
	\caption{Heterogeneous sensor network: comparison of (a) tr($\boldsymbol{J}$), and (b) number of selected sensors, versus $P_{tot}$ for different algorithms.}  
	%\vspace{-.5cm}
	\label{heterog} 
\end{figure}
% 
%=======================================
\section{Conclusions} \label{conclusions}
We considered the problem of sensor selection and power allocation to maximize the trace of Bayesian Fisher information matrix $\boldsymbol{J}$ under network transmit power constraint for distributed estimation of a zero mean Gaussian vector in wireless sensor networks. Three algorithms named as USU, greedy, and MCKP algorithms were proposed and their performances were compared numerically. Simulation results demonstarted the superiority of these algorithms compared to the algorithm that allocates power equally among all sensors. Moreover, the advantage of USU algorithm is that its performance is close to those of other algorithms and also, less number of sensors get activated compared to other algorithms, which contributes to energy saving in battery-powered sensor networks.  
%=======================================
%=======================================
%=======================================
\section*{Acknowledgment} \label{acknowledgment} 
This research is supported by NSF under grants CCF-1341966 and CCF-1319770.
%\begin{thebibliography}
\bibliographystyle{IEEETran}
\bibliography{myref}
%
%\begin{figure*}[b]
%	\centering
%	\includegraphics[width=6in]{system_model.eps}
%	\vspace{-0.1cm}
%	\caption{Our system model consists of $K$ sensors and a FC, that is tasked with estimating a Gaussian vector {\boldmath$\theta$}, via fusing c %ollective received signals.}
%	\label{system-model}
%	%\vspace{-.5cm}
%\end{figure*}
%
%=======================================
%=======================================
%=======================================
%\end{thebibliography}

% biography section
% 
% If you have an EPS/PDF photo (graphicx package needed) extra braces are
% needed around the contents of the optional argument to biography to prevent
% the LaTeX parser from getting confused when it sees the complicated
% \includegraphics command within an optional argument. (You could create
% your own custom macro containing the \includegraphics command to make things
% simpler here.)
%\begin{IEEEbiography}[{\includegraphics[width=1in,height=1.25in,clip,keepaspectratio]{mshell}}]{Michael Shell}
% or if you just want to reserve a space for a photo:

% You can push biographies down or up by placing
% a \vfill before or after them. The appropriate
% use of \vfill depends on what kind of text is
% on the last page and whether or not the columns
% are being equalized.

%\vfill

% Can be used to pull up biographies so that the bottom of the last one
% is flush with the other column.
%\enlargethispage{-5in}



% that's all folks
\end{document}
%=======================================
%=======================================
%=======================================
%=======================================
%=======================================
%=======================================
%=======================================
%=======================================
%=======================================

% An example of a floating figure using the graphicx package.
% Note that \label must occur AFTER (or within) \caption.
% For figures, \caption should occur after the \includegraphics.
% Note that IEEEtran v1.7 and later has special internal code that
% is designed to preserve the operation of \label within \caption
% even when the captionsoff option is in effect. However, because
% of issues like this, it may be the safest practice to put all your
% \label just after \caption rather than within \caption{}.
%
% Reminder: the "draftcls" or "draftclsnofoot", not "draft", class
% option should be used if it is desired that the figures are to be
% displayed while in draft mode.
%
%\begin{figure}[!t]
%\centering
%\includegraphics[width=2.5in]{myfigure}
% where an .eps filename suffix will be assumed under latex, 
% and a .pdf suffix will be assumed for pdflatex; or what has been declared
% via \DeclareGraphicsExtensions.
%\caption{Simulation results for the network.}
%\label{fig_sim}
%\end{figure}

% Note that IEEE typically puts floats only at the top, even when this
% results in a large percentage of a column being occupied by floats.


% An example of a double column floating figure using two subfigures.
% (The subfig.sty package must be loaded for this to work.)
% The subfigure \label commands are set within each subfloat command,
% and the \label for the overall figure must come after \caption.
% \hfil is used as a separator to get equal spacing.
% Watch out that the combined width of all the subfigures on a 
% line do not exceed the text width or a line break will occur.
%
%\begin{figure*}[!t]
%\centering
%\subfloat[Case I]{\includegraphics[width=2.5in]{box}%
%\label{fig_first_case}}
%\hfil
%\subfloat[Case II]{\includegraphics[width=2.5in]{box}%
%\label{fig_second_case}}
%\caption{Simulation results for the network.}
%\label{fig_sim}
%\end{figure*}
%
% Note that often IEEE papers with subfigures do not employ subfigure
% captions (using the optional argument to \subfloat[]), but instead will
% reference/describe all of them (a), (b), etc., within the main caption.
% Be aware that for subfig.sty to generate the (a), (b), etc., subfigure
% labels, the optional argument to \subfloat must be present. If a
% subcaption is not desired, just leave its contents blank,
% e.g., \subfloat[].


% An example of a floating table. Note that, for IEEE style tables, the
% \caption command should come BEFORE the table and, given that table
% captions serve much like titles, are usually capitalized except for words
% such as a, an, and, as, at, but, by, for, in, nor, of, on, or, the, to
% and up, which are usually not capitalized unless they are the first or
% last word of the caption. Table text will default to \footnotesize as
% IEEE normally uses this smaller font for tables.
% The \label must come after \caption as always.
%
%\begin{table}[!t]
%% increase table row spacing, adjust to taste
%\renewcommand{\arraystretch}{1.3}
% if using array.sty, it might be a good idea to tweak the value of
% \extrarowheight as needed to properly center the text within the cells
%\caption{An Example of a Table}
%\label{table_example}
%\centering
%% Some packages, such as MDW tools, offer better commands for making tables
%% than the plain LaTeX2e tabular which is used here.
%\begin{tabular}{|c||c|}
%\hline
%One & Two\\
%\hline
%Three & Four\\
%\hline
%\end{tabular}
%\end{table}


% Note that the IEEE does not put floats in the very first column
% - or typically anywhere on the first page for that matter. Also,
% in-text middle ("here") positioning is typically not used, but it
% is allowed and encouraged for Computer Society conferences (but
% not Computer Society journals). Most IEEE journals/conferences use
% top floats exclusively. 
% Note that, LaTeX2e, unlike IEEE journals/conferences, places
% footnotes above bottom floats. This can be corrected via the
% \fnbelowfloat command of the stfloats package.




% if have a single appendix:
%\appendix[Proof of the Zonklar Equations]
% or
%\appendix  % for no appendix heading
% do not use \section anymore after \appendix, only \section*
% is possibly needed

% use appendices with more than one appendix
% then use \section to start each appendix
% you must declare a \section before using any
% \subsection or using \label (\appendices by itself
% starts a section numbered zero.)
%



% Can use something like this to put references on a page
% by themselves when using endfloat and the captionsoff option.
\ifCLASSOPTIONcaptionsoff
  \newpage
\fi



% trigger a \newpage just before the given reference
% number - used to balance the columns on the last page
% adjust value as needed - may need to be readjusted if
% the document is modified later
%\IEEEtriggeratref{8}
% The "triggered" command can be changed if desired:
%\IEEEtriggercmd{\enlargethispage{-5in}}

% references section

% can use a bibliography generated by BibTeX as a .bbl file
% BibTeX documentation can be easily obtained at:
% http://www.ctan.org/tex-archive/biblio/bibtex/contrib/doc/
% The IEEEtran BibTeX style support page is at:
% http://www.michaelshell.org/tex/ieeetran/bibtex/
%\bibliographystyle{IEEEtran}
% argument is your BibTeX string definitions and bibliography database(s)
%\bibliography{IEEEabrv,../bib/paper}
%
% <OR> manually copy in the resultant .bbl file
% set second argument of \begin to the number of references
% (used to reserve space for the reference number labels box)

