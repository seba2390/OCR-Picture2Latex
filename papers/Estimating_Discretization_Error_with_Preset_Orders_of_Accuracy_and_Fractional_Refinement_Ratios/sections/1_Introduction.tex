\section{Introduction} \label{sec:intro}
As numerical simulation becomes an essential part of decision-making in science and engineering, verification and validation \citep{Roache1998VerificationEngineering,Stern2001ComprehensiveProcedures,OberkampfRoy2010} are widely used to establish the reliability of the approximate solutions given by simulations. Verification intends to show that one is ``solving the equations right'', whereas validation intends to show that one is ``solving the right equations'' \citep{Roache1998article}. In particular, solution verification is the process of estimating the numerical errors in an approximate solution for which the exact solution is unknown. Among all sources of numerical errors, the discretization error (DE) is usually the largest and most difficult to estimate \citep{Roy2005,Eca2014AStudies}. The estimation of DE in solution verification is the focus of this paper.

The estimation of DE by grid refinement methods \citep{Roy2010} has received substantial interest in computational physics \citep{Roache1994Perspective:Studies,Eca2014AStudies,Phillips2016ADynamics,Rider2016RobustAnalysis}. The main advantage of these methods is that they can be applied to any system response quantity, on any discretization method, and in a post-processing manner \citep{Stern2001ComprehensiveProcedures,Roy2010}. In principle, these methods use approximate solutions on a sequence of grids to obtain an estimate of the exact solution, followed by comparing it with the approximate solutions to obtain an estimate of the DE. This estimation is reliable if the approximate solutions lie in the so-called \textit{asymptotic range} \citep{Salas2006,Xing2010,Roy2010,Phillips2016ADynamics,Eca2018}, which is defined as the sequence of grids over which the DE reduces in the formal (theoretical) order of accuracy \citep{Roy2010}.

Most of the current grid refinement methods assess the reliability of the estimated DE by comparing the formal order of accuracy with the observed order of accuracy \citep{Roy2010,Orozco2010VerificationSolutions}; however, the use of the observed order has two problems. First, a small variation of the observed order may cause both the magnitude and the uncertainty of the estimated DE to change irregularly \citep{Roy2003,Eca2009,Xing2010,Hodis2012GridAneurysms}. Second, in certain cases the observed order is not well defined \citep{Roy2003,Phillips2014RichardsonDynamics,Phillips2016ADynamics}. These problems are usually overcome by placing a limit on the observed order, but then the estimated DE may contain unpredictable errors \citep{Phillips2014RichardsonDynamics,Phillips2016ADynamics}.

On the other hand, grid refinement methods generally suffer from a high computational cost due to the large number of grid points required to achieve the asymptotic range \citep{Eca2018}. One possible way to lower the cost is using refinement ratios greater than 0.5 \citep{Roy2010}. This has been applied to estimate local errors in critical locations of the domain \citep{Roache1994Perspective:Studies,Trivedi2019}, but there has been less evidence for such applications throughout the domain to estimate global errors. A central challenge is that the number of shared grid points across refinement levels decreases as the refinement ratio increases, which in turn increases the statistical uncertainty of the global estimates.

In this paper, we propose an alternative grid refinement method, called the Preset Orders Expansion Method (POEM). This method avoids the above problems with the observed order by using constant orders given by the user. In addition, it can be used to assess the asymptotic convergence of approximate solutions. While the idea of using constant orders is not new \citep{Eca2002}, no study to date has applied it to estimate the magnitude of DE. We also study the use of fractional refinement ratios greater than 0.5 to reduce the computational demand of POEM. The reduction in the number of shared grid points due to fractional refinement is overcome by a technique called Method of Interpolating Differences between Approximate Solutions (MIDAS).

%------------------------------------------------------------------------------