\section{Discretization Error} \label{sec:DiscErr}
Physical and engineering systems are often described by a mathematical model, for example, a set of partial differential equations with initial and boundary conditions. Such a model is discretized when being solved on digital computers. The numerical error associated with the discretization process is known as \textit{discretization error} (DE).

DE, denoted by $\varepsilon$, is defined as the difference between the exact solution to the discretized model $\phi$ and the exact solution to the mathematical model $\phi_{e}$ \citep{Roy2010}. This can be formulated as follows.
\begin{equation} \label{eq:DE}
    \varepsilon \equiv \phi - \phi_{e}.
\end{equation}
In solution verification, $\phi_{e}$ is not available, so the DE can only be estimated. An estimate of DE, denoted by $\Tilde{\varepsilon}$, is defined as the difference between $\phi$ and an estimate of $\phi_{e}$, denoted by $\Tilde{\phi}_{e}$. This can be formulated as follows.
\begin{equation} \label{eq:est_DE}
    \Tilde{\varepsilon} \equiv \phi - \Tilde{\phi}_{e}.
\end{equation}
The solution $\phi$ is termed \textit{approximate solution}: the exact solution to the discretized model can be understood as an approximation to the mathematical model.

One of the assumptions of grid refinement methods is that DE constitutes the majority of numerical errors in an approximate solution \citep{Roy2010}. This is valid for most practical applications, since other sources of numerical errors can be easily eliminated or quantified \citep{Roy2005,Eca2014AStudies}. The calculations in this study are based on this assumption.

%------------------------------------------------------------------------------