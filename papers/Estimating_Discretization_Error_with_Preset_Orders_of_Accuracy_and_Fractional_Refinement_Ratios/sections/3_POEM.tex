\section{Preset Orders Expansion Method (POEM)} \label{sec:POEM}
In this section, we explain the principles of POEM and demonstrate its applications on a few typical problems. We first present the principal equation and the procedures for estimating DE in this method. Then, we address how to choose the preset orders and how to adapt the principal equation to different refinement paths. Lastly, we discuss the computational demand of this method.

%------------------------------------------------------------------------------

\subsection{The Principal Equation}
When $\phi_e$ is smooth over the domain of the mathematical model, it can be related to $\phi$ by a series expansion in orders of a grid spacing parameter $h$ \cite{Roy2010}:
\begin{equation} \label{eq:model_infty}
    \phi = \phi_{e} + \sum_{m=1}^{\infty} C_{q_m} h^{q_m}.
\end{equation}
Here, the coefficients $\{C_{q}\}$ are functions of space and time but independent of $h$. The orders of the coefficients $\{q_m: q_1 < q_2 < \dots \}$ are integers determined by the numerical scheme being used. In particular, $q_1$ is known as \textit{formal order of accuracy} \citep{Roy2010}.

In POEM, this infinite series is modeled by the following principal equation.
\begin{equation} \label{eq:model_orig}
    \phi = \Tilde{\phi}_{e} + \sum_{m=1}^{k} C_{p_m} h^{p_m},
\end{equation}
where $k \ge 2$ and $\{p_m: p_1 < p_2 < \dots \}$ are constant values given by the user. Preset orders $\{p_m\}$ do not necessarily match the correct ones in $\{q_m\}$. In this equation the unknowns are $\Tilde{\phi}_e$ and $\{C_{p}\}$, whereas in many other grid refinement methods the unknowns include an exponent of $h$, called the \textit{observed order of accuracy} \cite{Roy2010,Orozco2010VerificationSolutions}. Because of this, the problems with the observed orders mentioned in Section \ref{sec:intro} are avoided.

%------------------------------------------------------------------------------

\subsection{Estimation of Discretization Error} \label{subsec:est_DE}
Solving for $\Tilde{\phi}_e$ and $\{C_{p}\}$ uniquely requires $k + 1$ approximate solutions orthogonal to each other. Such approximate solutions can be obtained by performing the same simulation on a set of systematically refined grids \citep{Roy2010}. This set of approximate solutions, denoted by $\{\phi_l\}$, forms a system of equations.
\begin{equation} \label{eq:model_system}
    \phi_l = \Tilde{\phi}_{e} + \sum_{m=1}^{k} C_{p_m} (r^{l-1} h)^{p_m},
\end{equation}
where $l \in [1,k+1]$ denotes the $l$th refinement level, $r \in [0.5,1)$ is the refinement ratio between two adjacent grid levels. The solutions to this system are $\Tilde{\phi}_e$ and $\{C_{p}h^p\}$.

For example, when $k = 2$, the principal equation (Equation \ref{eq:model_orig}) reads
\begin{equation} \label{eq:model_k2}
    \phi = \Tilde{\phi}_{e} + C_{p_1} h^{p_1} + C_{p_2} h^{p_2}.
\end{equation}
The system of equations (Equation \ref{eq:model_system}) can be written as
\begin{align} \label{eq:sys3eq_t}
    \begin{bmatrix}
        1 & 1 & 1 \\
        1 & r^{p_1} & r^{p_2} \\
        1 & r^{2p_1} & r^{2p_2}
    \end{bmatrix}
    \begin{bmatrix}
        \Tilde{\phi}_{e} \\
        C_{p_1}h^{p_1} \\
        C_{p_2}h^{p_2}
    \end{bmatrix}
    &=
    \begin{bmatrix}
        \phi_1 \\
        \phi_2 \\
        \phi_3
    \end{bmatrix}.
\end{align}
The analytical solutions of the unknowns are as follows.
\begin{align}
    \Tilde{\phi}_{e} &= \phi_1 - C_{p_1} h^{p_1} - C_{p_2} h^{p_2}, \\
    C_{p_1} h^{p_1} &= \frac{r^{p_2} e_{21} - e_{32}}{r^{p_1} (1 - r^{p_1}) (1 - r)}, \label{eq:C_p1} \\
    C_{p_2} h^{p_2} &= \frac{e_{32} - r^{p_1} e_{21}}{r^{p_1} (1 - r^{p_2}) (1 - r)} \label{eq:C_p2},
\end{align}
where $e_{ij} \equiv \phi_i - \phi_j$ are the \textit{differences between approximate solutions}.

By comparing $\phi_l$ with $\Tilde{\phi}_e$ at the same location, we can obtain a local estimate of the DE, $\Tilde{\varepsilon}$ (see Equation \ref{eq:est_DE}). To further quantify $\Tilde{\varepsilon}$ over the entire simulation domain, we evaluate its $L_1$-, $L_2$-, and $L_{\infty}$-norms. These norms are defined respectively as
\begin{align}
    ||f||_1 &\equiv \frac{1}{N}\sum_{i=1}^{N}|f(x_i)|, \label{eq:L1_norm} \\
    ||f||_2 &\equiv \sqrt{\frac{1}{N}\sum_{i=1}^{N}|f(x_i)|^2}, \label{eq:L2_norm} \\
    ||f||_{\infty} &\equiv \max_{1 \leq i \leq N} |f(x_i)|, \label{eq:L8_norm}
\end{align}
where $f:\{x_1,\ \dots \ ,x_N\} \rightarrow \mathbb{R}$ is a discrete function on a domain of $N$ grid points.

%------------------------------------------------------------------------------

\subsection{Assessment of Asymptotic Convergence of Numerical Solutions}
An advantage of POEM is that it can be used to examine the asymptotic convergence of approximate solutions. From the definition of an asymptotic range \cite{Roy2010}, one can deduce that the approximate solutions lie in the asymptotic range if the higher-order coefficient terms dominate the lower-order coefficient terms. We quantify this by measuring the $L_2$-norms (see Equation \ref{eq:L2_norm}) of the coefficient terms.

In the vicinity of the asymptotic range, it suffices to compare only the first two terms ($k = 1,2$), where their ratio is given by
\begin{equation} \label{eq:beta_tilde}
    \Tilde{\beta} \equiv \frac{||C_{p_2} h^{p_2}||_2}{||C_{p_1} h^{p_1}||_2}.
\end{equation}
If $\Tilde{\beta}$ is smaller than a certain threshold $\beta \in [0,1]$, approximate solutions are considered to be in the asymptotic range and vice versa. To regularize such a criterion in simulation problems, we use a fixed value of $\beta = 0.01$ throughout this study.

%------------------------------------------------------------------------------

\subsection{Choice of Preset Orders} \label{subsec:choice_orders}
The estimated DE is accurate if the estimated exact solution $\Tilde{\phi}_e$ converges to the exact solution $\phi_e$ faster than the approximate solution $\phi$ as the grid is refined. This amounts to eliminating the first coefficient terms of $\phi$ in Equation \ref{eq:model_infty}. In POEM, this is achieved by prescribing their actual orders $q_1,\ \dots \ , q_k$ in the preset orders. In principle, actual orders are determined solely by the numerical scheme; therefore, they can be identified during code verification \citep{Knupp2002,Salari2000}, a process recommended to take place before solution verification \citep{Roy2005}. However, it is sometimes tricky to interpret the orders observed in this process due to their high sensitivity to errors \citep{Love2013}. In the following, we present a straightforward way to obtain actual orders using POEM.

%------------------------------------------------------------------------------

\subsubsection{Guaranteed for the Actual Orders of Coefficient Terms} \label{subsubsec:guarantee}
By substituting Equation \ref{eq:model_infty} into the analytical solution of Equation \ref{eq:model_system}, such as Equation \ref{eq:C_p1} and \ref{eq:C_p2}, the following results can be obtained. If all actual orders are prescribed, that is, $p_m \equiv q_m \ \forall \ m \in [1,k]$, then
\begin{align} \label{eq:correctPreset}
    C_{p_m} h^{p_m} = C_{q_m} h^{q_m} + \mathcal{O}(h^{q_{k+1}}) \hspace{2em} \forall \ m \in [1,k].
\end{align}
In other words, every extracted coefficient term takes the desired coefficient term. If some of the actual orders are not prescribed, then
\begin{alignat}{2}
    C_{p_m} h^{p_m} &= \mathcal{O}(h^{q_m}) &&\hspace{2em} \forall \ p_m < \mu, \label{eq:wrongPreset_small} \\ 
    C_{p_m} h^{p_m} &= \mathcal{O}(h^{\mu}) &&\hspace{2em} \forall \ p_m > \mu, \label{eq:wrongPreset_large}
\end{alignat}
where $\mu \in \{q_1,\ \dots \ , q_k\}$ is the smallest missed order. We note that $q_m$ does not necessarily equal to $p_m$ since we have defined $\{p_m\}$ and $\{q_m\}$ to be ordered sets.

These results suggest that if the entire set of preset orders is correct, then every extracted coefficient term will converge at the rate of its preset order; on the contrary, if there is a missed actual order $\mu$, then some of the extracted terms will converge at a different rate than its preset order. Therefore, by replacing the wrong preset orders with $\mu$ in repeated applications of POEM, the user is guaranteed to find all the actual orders $q_1,\ \dots \ , q_k$. With these orders substituted into the preset orders, the estimated exact solution $\Tilde{\phi}_e$ will possess the highest possible order of accuracy.

%------------------------------------------------------------------------------

\subsubsection{Realizing Correct Orders from Wrong Orders} \label{subsubsec:realize_correct}
Here we show the differences in the extracted coefficient terms when the correct or a wrong set of preset orders is used. We consider the following initial-boundary value problem: solving the linear advection equation of $\phi(x,t)$,
\begin{equation}
    \frac{\partial \phi}{\partial t} + a \frac{\partial \phi}{\partial x} = 0,
\end{equation}
in the periodic domain $x \in [0,1]$ at time $t=2$ with the initial condition $\phi(x,0) = 2 + \cos(2 \pi x)$ and the speed of advection $a = 0.5$. In this demonstration, two discretization schemes are used. The first is the Beam-Warming (BW) scheme, which is first-order-accurate in time ($q_1 = 1$) when only the time dimension is refined \citep{Love2013}. Another is the second-order Runge-Kutta, second-order upwind (RK2U2) scheme, which is second-order-accurate in time ($q_1 = 2$).

We adopt the principal equation in Equation \ref{eq:model_k2} with $h \equiv \Delta t$. To obtain a set of systematically-refined grids, we refine the coarsest grid of size $\Delta x = \Delta t = 0.01$ in the time dimension with $r = 0.5$. We solve the system of equations in Equation \ref{eq:sys3eq_t} numerically.

Let us first consider the case where the preset orders are correct, that is, $p_1 = q_1$ and $p_2 = q_2$. For the case of the BW scheme, we set $p_1 = 1$ and $p_2 = 2$. The $L_2$-norm of the extracted coefficient terms and their convergence rates are plotted in Figure \ref{fig:cNorm-1D-BW-t-correct}, where $\Delta t$ corresponds to the coarsest level in each set of three refinement levels (see Equation \ref{eq:model_system}). In the right figure, we observe that the coefficient terms converge at the rate of their respective preset order. We find similar results (not shown) in the case of the RK2U2 scheme, where $p_1 = 2$ and $p_2 = 3$ are set. These results are consistent with Equation \ref{eq:correctPreset}.

\begin{figure}[!htb]
\centering
\begin{tikzpicture}
\pgfplotstableread[skip first n=2]{figures/correct_orders/cNorm-1D-BW-t-correct.dat}{\table}
\begin{axis}[
    xtick distance = 0.5,
    ytick distance = 1,
    xlabel = $\log \Delta t$,
    ylabel = $\log L_2$,
    legend pos = south east
]
\addplot[color=blue!80, mark=*, mark size=2] table [x index=0, y index=1] {\table};
\addplot[color=red!80, mark=triangle*, mark size=3] table [x index=0, y index=2] {\table};
\legend{$C_{1} \Delta t$, $C_{2} \Delta t^{2}$}
\end{axis}
\end{tikzpicture}
\hskip 20pt
\begin{tikzpicture}
\pgfplotstableread{figures/correct_orders/cNorm_slope-1D-BW-t-correct.dat}{\table}
\begin{axis}[
    xtick distance = 0.5,
    ytick distance = 0.2,
    xlabel = $\log \Delta t$,
    ylabel = $\Delta (\log L_2) / \Delta (\log \Delta t)$
]
\addplot[color=blue!80, mark=*, mark size=2] table [x index=0, y index=1] {\table};
\addplot[color=red!80, mark=triangle*, mark size=3] table [x index=0, y index=2] {\table};
\end{axis}
\end{tikzpicture}
\caption{The $L_2$-norm of coefficient terms and their convergence rates obtained by using the correct orders, 1 and 2, in the principal equation when the $t$ dimension is refined. Approximate solutions are obtained by solving the one-dimensional advection equation using the BW scheme.}
\label{fig:cNorm-1D-BW-t-correct}
\end{figure}

Next, we consider the case in which a wrong set of preset orders is chosen. For the case of the BW scheme, we set $p_1 = 2$ and $p_2 = 3$. The results are shown in Figure \ref{fig:cNorm-1D-BW-t-wrong}. In this case, the coefficient terms do not converge at the rate of their respective preset order but a rate of 1. For the case of the RK2U2 scheme, we set $p_1 = 1$ and $p_2 = 2$. The convergence rates of $C_{1} \Delta t$ and $C_{2} \Delta t^2$ are found to be 3 and 2 respectively (not shown). These results agree with Equation \ref{eq:wrongPreset_small} and \ref{eq:wrongPreset_large}.

\begin{figure}[!htb]
\centering
\begin{tikzpicture}
\pgfplotstableread[skip first n=2]{figures/wrong_orders/cNorm-1D-BW-t-wrong.dat}{\table}
\begin{axis}[
    xtick distance = 0.5,
    ytick distance = 0.5,
    xlabel = $\log \Delta t$,
    ylabel = $\log L_2$,
    legend pos = south east
]
\addplot[color=blue!80, mark=*, mark size=2] table [x index=0, y index=1] {\table};
\addplot[color=red!80, mark=triangle*, mark size=3] table [x index=0, y index=2] {\table};
\legend{$C_{2} \Delta t^{2}$, $C_{3} \Delta t^{3}$}
\end{axis}
\end{tikzpicture}
\hskip 20pt
\begin{tikzpicture}
\pgfplotstableread{figures/wrong_orders/cNorm_slope-1D-BW-t-wrong.dat}{\table}
\begin{axis}[
    xtick distance = 0.5,
    ytick distance = 0.02,
    xlabel = $\log \Delta t$,
    ylabel = $\Delta (\log L_2) / \Delta (\log \Delta t)$
]
\addplot[color=blue!80, mark=*, mark size=2] table [x index=0, y index=1] {\table};
\addplot[color=red!80, mark=triangle*, mark size=3] table [x index=0, y index=2] {\table};
\end{axis}
\end{tikzpicture}
\caption{The $L_2$-norm of coefficient terms and their convergence rates obtained by using the wrong orders 2 and 3 in the principal equation when the $t$ dimension is refined. The approximate solutions are the same as those used in Figure \ref{fig:cNorm-1D-BW-t-correct}.}
\label{fig:cNorm-1D-BW-t-wrong}
\end{figure}

%------------------------------------------------------------------------------

\FloatBarrier
\subsection{Adaptations to Different Refinement Paths} \label{subsec:adapt_paths}
When multiple dimensions are refined, the coefficient terms $\{C_p h^p\}$ in Equation \ref{eq:model_infty} can be expressed as a sum of multiple terms using the Taylor expansion. It is not obvious whether the form of Equation \ref{eq:model_orig} can be recovered and applied. This motivates us to do the following derivation.

To begin with, we write the explicit form of the coefficient terms in Equation \ref{eq:model_infty}. For the moment, we focus on refinement in the time dimension and a space dimension. Suppose the formal order of accuracy of the numerical scheme is $q_x$ when the $x$ dimension is refined and $q_t$ when the $t$ dimension is refined. Then, the equation can be written as follows.
\begin{equation} \label{eq:model_xt}
    \phi = \phi_{e} + \sum_{m=q_x}^{\infty} C_{m,0}\Delta x^{m} + \sum_{m=q_t}^{\infty} C_{0,m}\Delta t^{m} + \sum_{m=q_x+q_t}^{\infty} \sum_{n=q_t}^{m-q_x} C_{m-n,n}\Delta x^{m-n}\Delta t^{n}.
\end{equation}
As an example, the equation for the RK2U2 scheme ($q_x=q_t=2$) is
\begin{align} \label{eq:model_RK2U2_infty}
\begin{split}
    \phi &= \phi_{e} \\
    &+ C_{2,0}\Delta x^{2} + C_{3,0}\Delta x^{3} + \dots \\
    &+ C_{0,2}\Delta t^{2} + C_{0,3}\Delta t^{3} + \dots \\
    &+ C_{2,2}\Delta t^{2}\Delta x^{2} + C_{2,3}\Delta x^{2}\Delta t^{3} + \dots.
\end{split}
\end{align}

If we now truncate the series and try to form a system of equations like Equation \ref{eq:model_system}, the resulting system will be singular. The reason is that the refinement ratios for all dimensions are interdependent in a systematic grid refinement \citep{Roy2010}, which in turn lowers the degrees of freedom of that system. Such a dependency corresponds to the so-called \textit{refinement path}, which is of the form $r_t = (r_x)^s$ with $s \in \mathbb{R}$ in this case.

To obtain a non-singular system from Equation \ref{eq:model_xt}, we need to group the coefficient terms that depend on each other. After that, we obtain
\begin{align} \label{eq:model_path_infty}
    \phi = \phi_{e} + \sum_{m=1}^{\infty} D_{q_m} \Delta x^{q_m},
\end{align}
where $\{D_q\}$ are the effective coefficients given by
\begin{align}
    D_{q} &\equiv \sum_{n=0}^{\lfloor q/s \rfloor} C_{q-ns,n} \Big( \frac{\Delta t}{\Delta x^s} \Big)^{n}.
\end{align}
To ensure that $D_q$ does not contribute to the convergence rate of $D_q \Delta x^q$, we check whether $D_q$ is constant on all refinement levels. Since $\{C_q\}$ are independent of grid spacing, we only need to check the term $\Delta t / \Delta x^s$. Indeed, on the $l$th refinement level,
\begin{equation}
    \frac{(r_t)^l \Delta t}{[(r_x)^{l} \Delta x]^s} = \bigg[ \frac{r_t}{(r_x)^s} \bigg]^l \frac{\Delta t}{\Delta x^s} = \frac{\Delta t}{\Delta x^s}.
\end{equation}

We find that the expansion series in this case, Equation \ref{eq:model_path_infty}, is essentially the same as the previous one, Equation \ref{eq:model_infty}. The only difference is that while $q_m$ must previously be an integer, it can be a decimal in this case. In addition, the details of $D_q$ are less important as long as they do not affect the convergence rate of $D_q \Delta x^q$. These results can be generalized to refinement in multiple dimensions.

Therefore, the principles and methods presented so far are generally applicable. The general form of the principal equation is as follows.
\begin{equation} \label{eq:model_path}
    \phi = \Tilde{\phi}_{e} + \sum_{m=1}^{k} D_{p_m} \Delta x^{p_m}
\end{equation}
with the unknowns $\Tilde{\phi}_{e}$ and $\{D_p\}$. The unknowns can be obtained by solving the corresponding system of equations
\begin{align} \label{eq:model_path_system}
    \phi_l = \Tilde{\phi}_{e} + \sum_{m=1}^{k} D_{p_m} (r_x^{l-1} \Delta x)^{p_m}
\end{align}
formed by approximate solutions $\{\phi_l\}$ on systematically-refined grids. The knowledge in Section \ref{subsec:est_DE} -- \ref{subsubsec:realize_correct} applies accordingly.

%------------------------------------------------------------------------------

\subsubsection{Refinement with Constant CFL Number} \label{subsubsec:const_cfl}
Here we show the applicability of POEM when the Courant-Friedrichs-
Lewy (CFL) number \citep{CFL1928} is kept constant ($r_t = r_x$) during refinement. We use the advection problem in Section \ref{subsubsec:realize_correct} as a test case, where the CFL number is given by $a \Delta t / \Delta x$. The partial differential equation is discretized with the RK2U2 scheme, and the refinement ratios $r_t = r_x = 0.5$ are used. Starting with Equation \ref{eq:model_RK2U2_infty}, we can show that the RK2U2 scheme is formally second-order-accurate in this case. Therefore, we use the principal equation in Equation \ref{eq:model_path_infty} with $k = 2, p_1 = 2, p_2 = 3$:
\begin{equation} \label{eq:model_path_k2}
    \phi = \Tilde{\phi}_{e} + D_{2}\Delta x^{2} + D_{3}\Delta x^{3}.
\end{equation}
By solving the corresponding system of equations (Equation \ref{eq:model_path_system}), we obtain $\{\Tilde{\phi}_{e}, D_{2}\Delta x^{2}, D_{3}\Delta x^{3}\}$. After that, we calculate the estimated DE, $\Tilde{\varepsilon}$, and the actual DE, $\varepsilon$, using Equations \ref{eq:est_DE} and \ref{eq:DE}, respectively.

\begin{figure}[!htb]
\centering
\begin{tikzpicture}
\pgfplotstableread[skip first n=2]{figures/const_cfl/cNorm-1D-RK2U2-c-cfl.dat}{\table}
\begin{axis}[
    xtick distance = 0.5,
    ytick distance = 1,
    xlabel = $\log \Delta x$,
    ylabel = $\log L_2$,
    legend pos = south east
]
\addplot[color=blue!80, mark=*, mark size=2] table [x index=0, y index=1] {\table};
\addplot[color=red!80, mark=triangle*, mark size=3] table [x index=0, y index=2] {\table};
\legend{$D_{2} \Delta x^{2}$, $D_{3} \Delta x^{3}$}
\end{axis}
\end{tikzpicture}
\hskip 20pt
\begin{tikzpicture}
\pgfplotstableread{figures/const_cfl/cNorm_slope-1D-RK2U2-c-cfl.dat}{\table}
\begin{axis}[
    xtick distance = 0.5,
    ytick distance = 0.2,
    xlabel = $\log \Delta x$,
    ylabel = $\Delta (\log L_2) / \Delta (\log \Delta x)$
]
\addplot[color=blue!80, mark=*, mark size=2] table [x index=0, y index=1] {\table};
\addplot[color=red!80, mark=triangle*, mark size=3] table [x index=0, y index=2] {\table};
\end{axis}
\end{tikzpicture}
\caption{The $L_2$-norm of effective coefficient terms and their convergence rates obtained by using the correct orders, 2 and 3, in the principal equation when the dimensions $x$ and $t$ are refined along the path of a constant CFL number. Approximate solutions are obtained by solving the one-dimensional advection equation using the RK2U2 scheme.}
\label{fig:cNorm-1D-RK2U2-c-cfl}
\end{figure}

\begin{figure}[!htb]
\centering
\begin{tikzpicture}
\pgfplotstableread[skip first n=2]{figures/const_cfl/discErr-1D-RK2U2-c-cfl.dat}{\table}
\begin{axis}[
    xtick distance = 0.5,
    ytick distance = 1,
    xlabel = $\log \Delta x$,
    ylabel = $\log L$,
    legend pos = south east
]
\addplot[color=blue!80, mark=*, mark size=1] table [x index=0, y index=1] {\table};
\addplot[color=red!80, mark=triangle, mark size=4] table [x index=0, y index=2] {\table};
\addplot[color=teal, mark=square, mark size=4] table [x index=0, y index=2] {\table};
\legend{$L_1$, $L_2$, $L_{\infty}$}
\end{axis}
\end{tikzpicture}
\hskip 20pt
\begin{tikzpicture}
\pgfplotstableread{figures/const_cfl/discErr_slope-1D-RK2U2-c-cfl.dat}{\table}
\begin{axis}[
    minor tick num = 3,
    xtick distance = 0.5,
    ytick distance = 0.08,
    xlabel = $\log \Delta x$,
    ylabel = $\Delta (\log L) / \Delta (\log \Delta x)$
]
\addplot[color=blue!80, mark=*, mark size=1] table [x index=0, y index=1] {\table};
\addplot[color=red!80, mark=triangle, mark size=4] table [x index=0, y index=2] {\table};
\addplot[color=teal, mark=square, mark size=4] table [x index=0, y index=2] {\table};
\end{axis}
\end{tikzpicture}
\caption{Different norms of $\varepsilon$ and their convergence rates in addition to the results in Figure \ref{fig:cNorm-1D-RK2U2-c-cfl}.}
\label{fig:discErr-1D-RK2U2-c-cfl}
\end{figure}

We first examine the results of the coefficient terms. The right figure in Figure \ref{fig:cNorm-1D-RK2U2-c-cfl} suggests that the preset orders 2 and 3 are correct, and the left figure suggests that the approximate solutions lie in the asymptotic range when $\log \Delta x < -2.7$, following the criterion in Equation \ref{eq:beta_tilde}. These implications can be confirmed by the graph of $\varepsilon$ in Figure \ref{fig:discErr-1D-RK2U2-c-cfl}: the convergence rate of $\varepsilon$ deviates from the formal order of 2 by less than 1\% when $\log \Delta x < -1.5$.

\begin{figure}[!htb]
\centering
\begin{tikzpicture}
\pgfplotstableread[skip first n=2]{figures/const_cfl/es_ex-1D-RK2U2-c-cfl.dat}{\table}
\begin{axis}[
    xtick distance = 0.5,
    ytick distance = 2,
    xlabel = $\log \Delta x$,
    ylabel = $\log L$,
    legend pos = south east
]
\addplot[color=blue!80, mark=*, mark size=1] table [x index=0, y index=1] {\table};
\addplot[color=red!80, mark=triangle, mark size=4] table [x index=0, y index=2] {\table};
\addplot[color=teal, mark=square, mark size=4] table [x index=0, y index=2] {\table};
\legend{$L_1$, $L_2$, $L_{\infty}$}
\end{axis}
\end{tikzpicture}
\hskip 20pt
\begin{tikzpicture}
\pgfplotstableread{figures/const_cfl/es_ex_slope-1D-RK2U2-c-cfl.dat}{\table}
\begin{axis}[
    xtick distance = 0.5,
    ytick distance = 0.04,
    xlabel = $\log \Delta x$,
    ylabel = $\Delta (\log L) / \Delta (\log \Delta x)$
]
\addplot[color=blue!80, mark=*, mark size=1] table [x index=0, y index=1] {\table};
\addplot[color=red!80, mark=triangle, mark size=4] table [x index=0, y index=2] {\table};
\addplot[color=teal, mark=square, mark size=4] table [x index=0, y index=2] {\table};
\end{axis}
\end{tikzpicture}
\caption{Different norms of $\Tilde{\phi}_{e} - \phi_{e}$ and their convergence rates in addition to the results in Figure \ref{fig:cNorm-1D-RK2U2-c-cfl}.}
\label{fig:es_ex-1D-RK2U2-c-cfl}
\end{figure}

\begin{figure}[!htb]
\centering
\begin{tikzpicture}
\pgfplotstableread[skip first n=2]{figures/const_cfl/esErr-1D-RK2U2-c-cfl.dat}{\table}
\begin{axis}[
    xtick distance = 0.5,
    ytick distance = 1,
    xlabel = $\log \Delta x$,
    ylabel = $\log L$,
    legend pos = south east
]
\addplot[color=blue!80, mark=*, mark size=1] table [x index=0, y index=1] {\table};
\addplot[color=red!80, mark=triangle, mark size=4] table [x index=0, y index=2] {\table};
\addplot[color=teal, mark=square, mark size=4] table [x index=0, y index=2] {\table};
\legend{$L_1$, $L_2$, $L_{\infty}$}
\end{axis}
\end{tikzpicture}
\hskip 20pt
\begin{tikzpicture}
\pgfplotstableread{figures/const_cfl/esErr_slope-1D-RK2U2-c-cfl.dat}{\table}
\begin{axis}[
    minor tick num = 3,
    xtick distance = 0.5,
    ytick distance = 0.08,
    xlabel = $\log \Delta x$,
    ylabel = $\Delta (\log L) / \Delta (\log \Delta x)$
]
\addplot[color=blue!80, mark=*, mark size=1] table [x index=0, y index=1] {\table};
\addplot[color=red!80, mark=triangle, mark size=4] table [x index=0, y index=2] {\table};
\addplot[color=teal, mark=square, mark size=4] table [x index=0, y index=2] {\table};
\end{axis}
\end{tikzpicture}
\caption{Different norms of $\Tilde{\varepsilon}$ and their convergence rates in addition to the results in Figure \ref{fig:cNorm-1D-RK2U2-c-cfl}.}
\label{fig:app_es-1D-RK2U2-c-cfl}
\end{figure}

We also compare the estimated exact solution $\Tilde{\phi}_e$ with the exact solution $\phi_e$ to examine the accuracy of $\Tilde{\phi}_e$. Figure \ref{fig:es_ex-1D-RK2U2-c-cfl} shows that the error norms of $\Tilde{\phi}_e$ converge at a rate of 4, which is expected since the preset orders have proved correct. Furthermore, these error norms are between $10^{-2}$ and $10^{-10}$, which are much lower than those of $\phi_l$ shown in Figure \ref{fig:discErr-1D-RK2U2-c-cfl}. This superior accuracy of $\Tilde{\phi}_e$ suggests that the estimated DE, $\Tilde{\varepsilon}$, is close to the actual DE, $\varepsilon$. Indeed, we find excellent agreement between them in Figure \ref{fig:app_es-1D-RK2U2-c-cfl} and \ref{fig:discErr-1D-RK2U2-c-cfl}.

%------------------------------------------------------------------------------

\FloatBarrier % too many figures
\subsubsection{Refinement with Constant Diffusion Number}
Here we show the applicability of POEM when the diffusion number is kept constant ($r_t = r_x^2$) during refinement. The test problem is solving the advection-diffusion equation with a source
\begin{equation}
    \frac{\partial \phi}{\partial t} + a \frac{\partial \phi}{\partial x} = \nu \frac{\partial \phi}{\partial x^2} + 4\pi^2 \nu^2 \cos[2\pi (x - at)]
\end{equation}
over the periodic domain $x \in [0,1]$ with the initial condition $2 + \cos(2\pi x)$. This problem has an analytical solution $\phi(x,t) = 2 + \cos[2\pi (x - at)]$. We are concerned with the solution at $t=2.5$ with $a=0.4$ and $\nu=0.01$. The time derivative and the advection term are discretized using the RK2U2 scheme, whereas the diffusion term is discretized using the fourth-order-centered approximation. When the diffusion number $\nu \Delta t / \Delta x^2$ is kept constant during refinement, such a discretized model is formally second-order-accurate. Therefore, we use the principal equation in Equation \ref{eq:model_path_k2} and the refinement ratios $r_t = r_x^2 = 0.5$.

\begin{figure}[!htb]
\centering
\begin{tikzpicture}
\pgfplotstableread[skip first n=2]{figures/const_dif/cNorm-1D-RK2U2-d-dif.dat}{\table}
\begin{axis}[
    xtick distance = 0.5,
    ytick distance = 1,
    xlabel = $\log \Delta x$,
    ylabel = $\log L_2$,
    legend pos = south east
]
\addplot[color=blue!80, mark=*, mark size=2] table [x index=0, y index=1] {\table};
\addplot[color=red!80, mark=triangle*, mark size=3] table [x index=0, y index=2] {\table};
\legend{$D_{2} \Delta x^{2}$, $D_{3} \Delta x^{3}$}
\end{axis}
\end{tikzpicture}
\hskip 20pt
\begin{tikzpicture}
\pgfplotstableread{figures/const_dif/cNorm_slope-1D-RK2U2-d-dif.dat}{\table}
\begin{axis}[
    xtick distance = 0.5,
    ytick distance = 0.2,
    xlabel = $\log \Delta x$,
    ylabel = $\Delta (\log L_2) / \Delta (\log \Delta x)$
]
\addplot[color=blue!80, mark=*, mark size=2] table [x index=0, y index=1] {\table};
\addplot[color=red!80, mark=triangle*, mark size=3] table [x index=0, y index=2] {\table};
\end{axis}
\end{tikzpicture}
\caption{The $L_2$-norm of effective coefficient terms and their convergence rates obtained by using the correct orders, 2 and 3, in the principal equation when the dimensions $x$ and $t$ are refined along the path of a constant diffusion number. Approximate solutions are obtained by solving the one-dimensional advection-diffusion equation using the RK2U2 scheme for the time derivative and the advection term and the fourth-order-centered approximation for the diffusion term. }
\label{fig:cNorm-1D-RK2U2-d-dif}
\end{figure}

\begin{figure}[!htb]
\centering
\begin{tikzpicture}
\pgfplotstableread[skip first n=2]{figures/const_dif/es_ex-1D-RK2U2-d-dif.dat}{\table}
\begin{axis}[
    xtick distance = 0.5,
    ytick distance = 2,
    xlabel = $\log \Delta x$,
    ylabel = $\log L$,
    legend pos = south east
]
\addplot[color=blue!80, mark=*, mark size=1] table [x index=0, y index=1] {\table};
\addplot[color=red!80, mark=triangle, mark size=4] table [x index=0, y index=2] {\table};
\addplot[color=teal, mark=square, mark size=4] table [x index=0, y index=2] {\table};
\legend{$L_1$, $L_2$, $L_{\infty}$}
\end{axis}
\end{tikzpicture}
\hskip 20pt
\begin{tikzpicture}
\pgfplotstableread{figures/const_dif/es_ex_slope-1D-RK2U2-d-dif.dat}{\table}
\begin{axis}[
    xtick distance = 0.5,
    ytick distance = 0.04,
    xlabel = $\log \Delta x$,
    ylabel = $\Delta (\log L) / \Delta (\log \Delta x)$
]
\addplot[color=blue!80, mark=*, mark size=1] table [x index=0, y index=1] {\table};
\addplot[color=red!80, mark=triangle, mark size=4] table [x index=0, y index=2] {\table};
\addplot[color=teal, mark=square, mark size=4] table [x index=0, y index=2] {\table};
\end{axis}
\end{tikzpicture}
\caption{Different norms of $\Tilde{\phi}_{e} - \phi_{e}$ and their convergence rates in addition to the results in Figure \ref{fig:cNorm-1D-RK2U2-d-dif}.}
\label{fig:es_ex-1D-RK2U2-d-dif}
\end{figure}

The $L_2$-norms of the effective coefficient terms and their convergence rates are shown in Figure \ref{fig:cNorm-1D-RK2U2-d-dif}. With Equations \ref{eq:beta_tilde} and \ref{eq:correctPreset}, we can conclude that the preset orders 2 and 3 are correct and that the asymptotic range begins around $\log \Delta x = -2.7$. The error norms of the estimated exact solution $\Tilde{\phi}_e$ are plotted in Figure \ref{fig:es_ex-1D-RK2U2-d-dif}. We observe the expected convergence rate of 4 and the superior precision of $\Tilde{\phi}_e$.

%------------------------------------------------------------------------------

\FloatBarrier
\subsubsection{Refinement in 2 + 1 Dimensions} \label{subsec:2+1_dim}
In this last example, we show that POEM is applicable when two space dimensions and a time dimension are refined. We demonstrate with the following problem: solving the two-dimensional linear advection equation of $\phi(x,y,t)$,
\begin{equation}
    \frac{\partial \phi}{\partial t} + a_x \frac{\partial \phi}{\partial x} + a_y \frac{\partial \phi}{\partial y} = 0,
\end{equation}
at $t = 2$ over the periodic domain $(x,y) \in [0,1] \cross [0,1]$ with the initial condition $\phi(x,y,0) = 2 + \cos[2\pi (x + y)]$ and advection speeds $a_x = a_y = 0.25$. This equation is discretized using the RK2U2 scheme. We consider the refinement path in which the two CFL numbers $a_x \Delta t / \Delta x$ and $a_y \Delta t / \Delta y$ are constant and use a refinement ratio of 0.5 in all dimensions. Hence, we have $r_t = r_x = r_y = 0.5$. We apply the principal equation in Equation \ref{eq:model_path_k2}. In fact, the principal equation in Equation \ref{eq:model_path} can be recovered with $q_1 = 2$ in this case.

\begin{figure}[!htb]
\centering
\begin{tikzpicture}
\pgfplotstableread[skip first n=2]{figures/extend_2D/cNorm-2D-RK2U2-c.dat}{\table}
\begin{axis}[
    xtick distance = 0.5,
    ytick distance = 1,
    xlabel = $\log \Delta x$,
    ylabel = $\log L_2$,
    legend pos = south east
]
\addplot[color=blue!80, mark=*, mark size=2] table [x index=0, y index=1] {\table};
\addplot[color=red!80, mark=triangle*, mark size=3] table [x index=0, y index=2] {\table};
\legend{$D_{2} \Delta x^{2}$, $D_{3} \Delta x^{3}$}
\end{axis}
\end{tikzpicture}
\hskip 20pt
\begin{tikzpicture}
\pgfplotstableread{figures/extend_2D/cNorm_slope-2D-RK2U2-c.dat}{\table}
\begin{axis}[
    xtick distance = 0.5,
    ytick distance = 0.2,
    xlabel = $\log \Delta x$,
    ylabel = $\Delta (\log L_2) / \Delta (\log \Delta x)$
]
\addplot[color=blue!80, mark=*, mark size=2] table [x index=0, y index=1] {\table};
\addplot[color=red!80, mark=triangle*, mark size=3] table [x index=0, y index=2] {\table};
\end{axis}
\end{tikzpicture}
\caption{The $L_2$-norm of effective coefficient terms and their convergence rates obtained by using the correct orders, 2 and 3, in the principal equation when the dimensions $x$, $y$, and $t$ are refined along the path of constant CFL numbers. The approximate solutions are obtained by solving the two-dimensional advection equation using the RK2U2 scheme.}
\label{fig:cNorm-2D-RK2U2-c}
\end{figure}

\begin{figure}[!htb]
\centering
\begin{tikzpicture}
\pgfplotstableread[skip first n=2]{figures/extend_2D/es_ex_DE-2D-RK2U2-c.dat}{\table}
\begin{axis}[
    xtick distance = 0.5,
    ytick distance = 2,
    xlabel = $\log \Delta x$,
    ylabel = $\log L$,
    legend pos = south east
]
\addplot[color=blue!80, mark=*, mark size=1] table [x index=0, y index=1] {\table};
\addplot[color=red!80, mark=triangle, mark size=4] table [x index=0, y index=2] {\table};
\addplot[color=teal, mark=square, mark size=4] table [x index=0, y index=2] {\table};
\legend{$L_1$, $L_2$, $L_{\infty}$}
\end{axis}
\end{tikzpicture}
\hskip 20pt
\begin{tikzpicture}
\pgfplotstableread{figures/extend_2D/es_ex_DE_slope-2D-RK2U2-c.dat}{\table}
\begin{axis}[
    xtick distance = 0.5,
    ytick distance = 0.04,
    xlabel = $\log \Delta x$,
    ylabel = $\Delta (\log L) / \Delta (\log \Delta x)$
]
\addplot[color=blue!80, mark=*, mark size=1] table [x index=0, y index=1] {\table};
\addplot[color=red!80, mark=triangle, mark size=4] table [x index=0, y index=2] {\table};
\addplot[color=teal, mark=square, mark size=4] table [x index=0, y index=2] {\table};
\end{axis}
\end{tikzpicture}
\caption{Different norms of $\Tilde{\phi}_{e} - \phi_{e}$ and their convergence rates in addition to the results in Figure \ref{fig:cNorm-2D-RK2U2-c}.}
\label{fig:es_ex-2D-RK2U2-c}
\end{figure}

The effective coefficient terms and the error of the estimated exact solution are plotted in Figure \ref{fig:cNorm-2D-RK2U2-c} and \ref{fig:es_ex-2D-RK2U2-c} respectively. In particular, the estimated exact solution shows a higher order of accuracy and a superior accuracy.

%------------------------------------------------------------------------------

\subsection{Computational Demand}
Like many other grid refinement methods, POEM itself has a relatively low computational cost compared with the simulations calculating the approximate solutions required. However, the requirement of an approximate solution per refinement level may lead to a significant computational cost. Suppose the computational grid is refined along $d$ dimensions with a refinement ratio $r$. Then the number of grid points increases by $1/r^d$ times with $r$. Thus, the commonly used grid doubling ($r = 0.5$) strategy corresponds to an increase rate of $2^d$. The resulting large number of grid points would pose a huge computational burden to the simulations.

A less expensive strategy is to use refinement ratios greater than 0.5, which is compatible with POEM. However, such a strategy involves complications in finding the common locations of the grid points on all refinement levels. This would restrict us to estimating local errors at critical locations of the domain \citep{Roache1994Perspective:Studies,Trivedi2019}. In the next section, we try to resolve those complications so that POEM can be applied to the entire domain.

%------------------------------------------------------------------------------

\FloatBarrier