\section{Review of the Generalized Richardson Extrapolation} \label{sec:review_GRE}
When $\phi_{e}$ is smooth over the domain of the mathematical model, we can expand $\phi$ about $\phi_{e}$ in a power series at every grid point to obtain
\begin{equation} \label{eq:model_orig}
    \phi = \phi_{e} + \sum_{m=1}^{\infty} C_{q_m} h^{q_m},
\end{equation}
where the coefficients $\{C_{q}\}$ are functions of space and time but not the grid spacing $h$, and the orders $\{q_m: q_1 < q_2 < \dots \}$ are integers determined by the choice of the numerical scheme. In particular, $q_1$ is the formal (theoretical) order of accuracy of the numerical scheme. Here $C_{q} h^{q}$ represents the sum of all the coefficient terms of order $q$. Its specific form will be clarified in Section \ref{subsec:1+1_dim}.

In the generalized Richardson extrapolation \citep{Roache1994Perspective:Studies,Roy2010}, the series of coefficient terms is modeled by a single coefficient term:
\begin{equation} \label{eq:coef_term_GRE}
    C_{\Tilde{q}} h^{\Tilde{q}} = \sum_{m=1}^{\infty} C_{q_m} h^{q_m},
\end{equation}
where $C_{\Tilde{q}}$ and ${\Tilde{q}}$ are unknowns to be determined. Moreover, as this work is concerned with solution verification, ${\phi}_{e}$ is replaced by $\Tilde{\phi}_{e}$. Therefore,
\begin{equation}
    \phi = \Tilde{\phi}_{e} + C_{\Tilde{q}} h^{\Tilde{q}}.
\end{equation}
The exponent $\Tilde{q}$ is known as the observed order of accuracy. Unlike $\{q_m\}$, it is generally not an integer. In fact, it is because of this non-integer nature that $C_{\Tilde{q}} h^{\Tilde{q}}$ can map to any real number and therefore model the series of coefficient terms accurately.

To solve for $\{\Tilde{\phi}_{e}, C_{\Tilde{q}}, {\Tilde{q}}\}$ uniquely, solutions on three systematically-refined grids are required \citep{Roy2010}. Consider such a set of grids, where the refinement ratio $r \in [0.5,1)$ is uniform over the domain and constant across refinement levels. Let $\phi_1, \phi_2, \phi_3$ be the solution on the coarse, medium, and fine grid respectively. Then, we have
\begin{align}
    \phi_1 &= \Tilde{\phi}_{e} + C_{\Tilde{q}} h^{\Tilde{q}}, \\
    \phi_2 &= \Tilde{\phi}_{e} + C_{\Tilde{q}} (r h)^{\Tilde{q}}, \\
    \phi_3 &= \Tilde{\phi}_{e} + C_{\Tilde{q}} (r^2 h)^{\Tilde{q}}.
\end{align}
This system of equations has the solution
\begin{align}
    \Tilde{\phi}_{e} &= \phi_1 - C_{\Tilde{q}} h^{\Tilde{q}}, \\
    \Tilde{q} &= \ln \Big( \frac{\phi_3 - \phi_2}{\phi_2 - \phi_1} \Big) / \ln r, \label{eq:q_tilde} \\
    C_{\Tilde{q}} h^{\Tilde{q}} &= \frac{\phi_2 - \phi_1}{r^{\Tilde{q}} - 1}.
\end{align}
As a result, the estimated DE is given by
\begin{equation}
    \Tilde{\varepsilon} = \phi_1 - \Tilde{\phi_{e}} = \frac{\phi_2 - \phi_1}{r^{\Tilde{q}} - 1}.
\end{equation}

The reliability of DE estimation in grid refinement methods is associated with the concept of asymptotic range \citep{Roy2010}. The asymptotic range is defined as the sequence of grids over which the DE reduces at the formal order of accuracy. This is achieved when the grid spacing, $h$, is sufficiently small that the leading order term ($m=1$) dominates the series of coefficient terms in Equation \ref{eq:model_orig}. In this case, Equation \ref{eq:coef_term_GRE} reduces to linear maps, i.e. $C_{\Tilde{q}} \mapsto C_{q_1}$ and $\Tilde{q} \mapsto q_1$. Because of this, the DE estimation is considered reliable.

However, using the observed order of accuracy for DE estimation has two potential problems. First, since the observed order, $\Tilde{q}$, is the exponent of the coefficient term, $C_{\Tilde{q}} h^{\Tilde{q}}$, the estimated DE behaves non-linearly as the observed order changes. A small variation of the value can lead to irregular alteration of the estimated DE \citep{Roy2003,Eca2009,Hodis2012GridAneurysms}. Second, when the sign of $\phi_3 - \phi_2$ and $\phi_2 - \phi_1$ are different, the value of $\Tilde{q}$ is undefined according to Equation \ref{eq:q_tilde}. This can happen in the vicinity of an abrupt change, for example, a shock wave \citep{Roy2003}. A remedy of these problems is to limit $\Tilde{q}$ in a range \citep{OberkampfRoy2010}; however, the estimated DE may then contain unpredictable errors.

The observed order of accuracy is also the basis of many approaches to evaluating the uncertainty of the estimated DE \citep{Orozco2010VerificationSolutions,Phillips2014RichardsonDynamics}. Due to the aforementioned problems, the uncertainty estimate may also contain unpredictable errors. From a more fundamental viewpoint, the problem originates from the failure of asymptotic convergence of the approximate solutions \citep{Roy2010}. The reliability of the DE estimation is highly questionable in this case. Therefore, we focus on assessing the reliability rather than evaluating the uncertainty of DE estimates.

%------------------------------------------------------------------------------