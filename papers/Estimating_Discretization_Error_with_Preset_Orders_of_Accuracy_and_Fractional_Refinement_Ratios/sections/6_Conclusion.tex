\section{Conclusion}
POEM exploits the fact that the orders of coefficient terms depend solely on the numerical scheme being used, whereas the coefficients of these terms depend additionally on the numerical problem being solved. Therefore, this method calculates the coefficients instead of the orders of the coefficient terms. With the procedure described in this paper, the user is guaranteed to obtain the correct orders by repeated use of this method. Consequently, the estimated exact solution achieves a high order of accuracy and produces an accurate estimate of the discretization error. In addition, this method allows for a direct comparison of the coefficient terms to assess the asymptotic convergence of the approximate solutions, which is fundamental to the reliability of the estimated discretization error.

POEM requires a lower computational cost when the refinement ratio is higher. However, the estimated error suffers from higher uncertainty due to the reduced number of shared grid points. We show that the proportion of shared grid points attains maximum if the refinement ratios are in a specific form of fractions. Furthermore, we introduce additional shared grid points using MIDAS, which exploits the linearity of coefficient terms to interpolate them at once. As a result, POEM becomes applicable to interior grid points not shared with all refinement levels. Therefore, we can obtain a global estimate of the discretization error of lower uncertainty at a reduced computational cost. Although we focus on the implementations of POEM and MIDAS for Cartesian grids, these methods are directly applicable to other grids which can be transformed into a Cartesian grid.

%------------------------------------------------------------------------------
