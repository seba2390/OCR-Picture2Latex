\section{POEM in Combination with MIDAS} \label{sec:POEM-MIDAS}
We are prepared to incorporate MIDAS into POEM. First, we outline a general procedure for implementing POEM together with MIDAS. Then, we revisit the test case in Section \ref{subsec:2+1_dim} with the additional use of MIDAS.

\subsection{General Procedure} \label{subsec:general_proc}
Suppose we have obtained approximate solutions on a set of systematically-refined grids of uniform grid spacing. Then, we can implement POEM in combination with MIDAS by following this general procedure:
\begin{enumerate}
    \item Identify the irreducible unit based on Proposition \ref{prop:IU_u&e}.
    \item Iterate over the irreducible units in the simulation domain. Perform the following steps in each iteration.
    \item Calculate $e_{ij} = \phi_{i} - \phi_{j}$ at the grids points shared by at least two refinement levels. (See Figure \ref{fig:grids-1D}.)
    \item Apply interpolations on $e_{ij}$ at the objective locations. (See Equation \ref{eq:eij_extend}.)
    \item Construct a model for $e_{ij}$ of the form
    \begin{equation}  \label{eq:eij_path}
        e_{ij} = \sum_{m=1}^{k} D_{p_m} h^{p_m} \big[ r^{(i-j)p_m} - 1 \big] r^{(j-1)p_m}
    \end{equation}
    with preset orders $\{p_m\}$. (See Equation \ref{eq:model_path_system} and \ref{eq:eij_infty}.)
    \item Form a system of equations of $\{e_{ij}\}$ at the objective locations and solve for $\{D_{p_m} h^{p_m}\}$. (See Equation \ref{eq:sys2eq_x}.)
    \item Check whether each $D_{p_m} h^{p_m}$ converges at the rate of $p_m$. If this is true, proceed to the next step; otherwise, go back to the previous step with the wrong orders replaced by $\mu$. (See Section \ref{subsubsec:guarantee}.)
    \item Obtain $\Tilde{\phi}_e$ by subtracting the sum of $\{D_{p_m} h^{p_m}\}$ from $\phi$. (See Equation \ref{eq:model_path}.)
    \item Estimate the DE using Equation \ref{eq:est_DE}. Assess the reliability of the estimate using Equation \ref{eq:beta_tilde}.
\end{enumerate}

\subsection{Revisiting the Refinement in 2+1 Dimensions} \label{subsec:revisit}
Here we revisit the test case in Section \ref{subsec:2+1_dim} with the additional use of MIDAS. We examine if the two results are consistent and independent of refinement ratios. We also evaluate the reduced computational cost compared with the grid doubling approach.

\begin{figure}[!htb]
\centering
\begin{tikzpicture}[
    scale=0.7, font=\small,
    nd/.style={circle, inner sep=0pt, minimum size=5pt},
    color12/.style={fill=orange, color=orange},
    color23/.style={fill=blue!50, color=blue!50},
    colorInt/.style={fill=green!80!black, color=green!80!black}
]

    %%% Draws grids
    \begin{scope}
        \myGlobalTransformation{0}{9};
        \fill[black!10, opacity=0.3] (0,0) rectangle (9,9);
        \draw [step=2.25cm, black!50] grid (9,9);
    \end{scope}
    \begin{scope}
        \myGlobalTransformation{0}{4.5};
        \fill[black!10, opacity=0.3] (0,0) rectangle (9,9);
        \draw [step=1.5cm, black!50] grid (9,9);
    \end{scope}
    \begin{scope}
        \myGlobalTransformation{0}{0};
        \fill[black!10, opacity=0.3] (0,0) rectangle (9,9);
        \draw [step=1cm, black!50] grid (9,9);
    \end{scope}
    
    %%% Draw vertical lines
    %% shared grid points
    % level 2 to level 1
    \begin{scope}
        \myGlobalTransformation{0}{4.5};
        \foreach \x in {0,4.5,9} {
            \foreach \y in {0,4.5,9} {
                % name the node from which the line starts
                \node[below] (thisNode) at (\x,\y) {};
                % reset transformation to identity
                \pgftransformreset
                \draw[color12, thick] (thisNode) -- ++(0,3.3);
            }
        }
    \end{scope}
    % level 3 to level 2
    \begin{scope}
        \myGlobalTransformation{0}{0};
        \foreach \x in {0,3,6,9} {
            \foreach \y in {0,3,6,9} {
                \node[below] (thisNode) at (\x,\y) {};
                \pgftransformreset
                \draw[color23, thick] (thisNode) -- ++(0,3.3);
            }
        }
    \end{scope}
    
    %% interpolated points
    \begin{scope}
        \myGlobalTransformation{0}{4.5};
        \foreach \x in {0,3,6,9} {
            \foreach \y in {0,3,6,9} {
                \ifthenelse {\(0=\x \AND 0=\y\) \OR 
                            \(9=\x \AND 0=\y\) \OR 
                            \(0=\x \AND 9=\y\) \OR 
                            \(9=\x \AND 9=\y\)}{}{
                    \node[below] (thisNode) at (\x,\y) {};
                    \pgftransformreset
                    \draw[colorInt, very thick, dotted] (thisNode) -- ++(0,3.3);
                }
            }
        }
    \end{scope}
    
    %%% Draw nodes
    %% shared grid points
    % which connect level 1 and level 2
    \begin{scope}
        \myGlobalTransformation{0}{9};
        \foreach \x in {0,4.5,9} {
            \foreach \y in {0,4.5,9} {
                \node[nd, color12] at (\x,\y) {};
            }
        }
    \end{scope}
    \begin{scope}
        \myGlobalTransformation{0}{4.5};
        \foreach \x in {0,4.5,9} {
            \foreach \y in {0,4.5,9} {
                \ifthenelse {\(0=\x \AND 0=\y\) \OR 
                            \(9=\x \AND 0=\y\) \OR 
                            \(0=\x \AND 9=\y\) \OR 
                            \(9=\x \AND 9=\y\)}{
                    \node[nd, circle split, circle split part fill={color12,color23}, draw=black!0] at (\x,\y) {};
                }{
                    \node[nd, color12] at (\x,\y) {};
                }
            }
        }
    \end{scope}
    % which connect level 2 and level 3
    \begin{scope}
        \myGlobalTransformation{0}{4.5};
        \foreach \x in {0,3,6,9} {
            \foreach \y in {0,3,6,9} {
                \ifthenelse {\(0=\x \AND 0=\y\) \OR 
                            \(9=\x \AND 0=\y\) \OR 
                            \(0=\x \AND 9=\y\) \OR 
                            \(9=\x \AND 9=\y\)}{}{
                    \node[nd, color23] at (\x,\y) {};
                }
            }
        }
    \end{scope}
    \begin{scope}
        \myGlobalTransformation{0}{0};
        \foreach \x in {0,3,6,9} {
            \foreach \y in {0,3,6,9} {
                \node[nd, color23] at (\x,\y) {};
            }
        }
    \end{scope}
    
    %% interpolated points
    \begin{scope}
        \myGlobalTransformation{0}{9};
        \foreach \x in {0,3,6,9} {
            \foreach \y in {0,3,6,9} {
                \ifthenelse {\(0=\x \AND 0=\y\) \OR 
                            \(9=\x \AND 0=\y\) \OR 
                            \(0=\x \AND 9=\y\) \OR 
                            \(9=\x \AND 9=\y\)}{}{
                    \node[nd, colorInt, style=cross out, draw, thick] at (\x,\y) {};
                }
            }
        }
    \end{scope}
    
    %%% Mark orientations
    \begin{scope}
        \myGlobalTransformation{13}{0.5};
        \draw[->, very thick] (0,0) -- (1.5,0) node[anchor=west] {$x$};
        \draw[->, very thick] (0,0) -- (0,1.5) node[anchor=south west] {$y$};
    \end{scope}
    
    %%% Label coordinates
    \begin{scope}
        \myGlobalTransformation{0}{9};
        \node[left] at (0,0) {(0,0)};
        \node[above left] at (0,9) {(0,4)};
        \node[right] at (9,9) {(4,4)};
    \end{scope}
    \begin{scope}
        \myGlobalTransformation{0}{4.5};
        \node[left] at (0,0) {(0,0)};
        \node[right] at (9,9) {(6,6)};
    \end{scope}
    \begin{scope}
        \myGlobalTransformation{0}{0};
        \node[left] at (0,0) {(0,0)};
        \node[below right] at (9,0) {(9,0)};
        \node[right] at (9,9) {(9,9)};
    \end{scope}
    
    % Label levels
    \node at (-1, 9.0cm + 20pt) {level 1};
    \node at (-1, 4.5cm + 20pt) {level 2};
    \node at (-1, 0.0cm + 20pt) {level 3};
    
    %%% Make legend
    \matrix[draw, semithick, column sep=10pt, below] at (current bounding box.south) {
        \node[color12, label=right:shared by level 1 and 2] {}; &
        \node[color23, label=right:shared by level 2 and 3] {}; &
        \node[colorInt, label=right:interpolation] {}; \\
    };
    
\end{tikzpicture}
\caption{The irreducible unit of $G$ that consists of three grids with $r_x=r_y=\frac{2}{3}$. Coordinates $(i,j)$ refer to the indices of the grid points on the respective level. While the orange and blue circles connected with solid lines represent existing shared grid points, the green crosses attached to a dotted line represent the objective locations of MIDAS being implemented here.}
\label{fig:grids-2D}
\end{figure}

We solve the test problem using the refinement ratios $\{\frac{2}{3}, \frac{3}{4}, \frac{4}{5}\}$. As an illustration of our applications, we consider a set of three systematically refined grids characterized by $r = r_x = r_y = \frac{2}{3}$. The irreducible unit contains grid segments $4 \cross 4$, $6 \cross 6$, and $9 \cross 9$ in the coarse, medium, and fine grid, respectively, as shown in Figure \ref{fig:grids-2D}.

\begin{figure}[!htb]
\centering
\begin{tikzpicture}
\pgfplotsset{
    width=0.6\textwidth,
    %height=0.5\textwidth
}
\begin{axis}[
    xtick distance = 0.1,
    ytick distance = 0.5,
    xlabel = $\log \Delta x$,
    ylabel = $\log ||D_{p} \Delta x^{p}||_2$,
    legend pos = outer north east,
    legend entries = {
        $p=2, r=\frac{1}{2}$\\$p=2, r=\frac{2}{3}$\\$p=2, r=\frac{3}{4}$\\$p=2, r=\frac{4}{5}$\\
        $p=3, r=\frac{1}{2}$\\$p=3, r=\frac{2}{3}$\\$p=3, r=\frac{3}{4}$\\$p=3, r=\frac{4}{5}$\\
    }
]
\addplot[color=blue!80, mark=*, mark size=1.5, line width=0.4pt] table [skip first n=2, x index=0, y index=1] {figures/revisit/cNorm-2D-RK2U2-c0.5.dat};
\addplot[color=blue!80, mark=+, mark size=4, line width=0.4pt] table [skip first n=2, x index=0, y index=1] {figures/revisit/cNorm-2D-RK2U2-c0.67.dat};
\addplot[color=blue!80, mark=o, mark size=4, line width=0.4pt] table [skip first n=2, x index=0, y index=1] {figures/revisit/cNorm-2D-RK2U2-c0.75.dat};
\addplot[color=blue!80, mark=diamond, mark size=4, line width=0.4pt] table [skip first n=2, x index=0, y index=1] {figures/revisit/cNorm-2D-RK2U2-c0.8.dat};

\addplot[color=red!80, mark=triangle*, mark size=2, line width=0.4pt] table [skip first n=2, x index=0, y index=2] {figures/revisit/cNorm-2D-RK2U2-c0.5.dat};
\addplot[color=red!80, mark=x, mark size=4, line width=0.4pt] table [skip first n=2, x index=0, y index=2] {figures/revisit/cNorm-2D-RK2U2-c0.67.dat};
\addplot[color=red!80, mark=triangle, mark size=4, line width=0.4pt] table [skip first n=2, x index=0, y index=2] {figures/revisit/cNorm-2D-RK2U2-c0.75.dat};
\addplot[color=red!80, mark=square, mark size=4, line width=0.4pt] table [skip first n=2, x index=0, y index=2] {figures/revisit/cNorm-2D-RK2U2-c0.8.dat};
\end{axis}
\end{tikzpicture}
\caption{The results of a real application of POEM and MIDAS in solution verification using different refinement ratios. The problem setting is that described in Figure \ref{fig:cNorm-2D-RK2U2-c}; however, here the refinements start with a much finer grid.}
\label{fig:compare_ratios}
\end{figure}


We choose the objective locations of MIDAS to be where the medium and fine grids have shared grid points. To this end, we need to extend the definition of $e_{ij}$ to the places marked by a green cross in Figure \ref{fig:grids-2D}. For the objective locations on an edge, we apply the two-point interpolation given in Equation \ref{eq:C_xo}. For those in the interior, we apply a four-point linear interpolation with the four nearest shared grid points and suitable weights $\{\Gamma_i\}$:
\begin{align}
    e_{ij}(x_o,y_o) &\equiv \Gamma_{a}e_{ij}(x_a,y_a) + \Gamma_{b}e_{ij}(x_b,y_b) + \Gamma_{c}e_{ij}(x_c,y_c) + \Gamma_{d}e_{ij}(x_d,y_d).
\end{align}
To describe $\{e_{ij}\}$ in terms of coefficient terms, we use Equation \ref{eq:eij_path} with $k=2$.

We use a single refinement ratio, in contrast to a global and a local refinement ratio in Section \ref{subsec:consistencies}, to mimic real applications of fractional refinement. The number of grid segments in the coarsest grid is 243 for $r \in \{\frac{1}{2}, \frac{3}{4}\}$ and 256 for $r \in \{\frac{2}{3}, \frac{4}{5}\}$ in both $x$ and $y$ dimensions.

The resulting $L_2$-norms of the effective coefficient terms are plotted in Figure \ref{fig:compare_ratios}. The results from grid doubling are also plotted in this figure for comparisons. We find excellent agreement among these results. This implies that the additional use of MIDAS with different refinement ratios does not affect the results of POEM.

Last but not least, we compare the computational cost of applying fractional refinement and MIDAS with that of applying solely grid doubling. We conducted the experiment serially on an Intel Core i5-7200U (2.5GHz, 3MB cache) with 4GB of memory using double precision. We repeat the simulation of $r=0.5$ and $r=0.75$ 20 times each to collect run-time statistics. The average run-time is $(1474 \pm 9)$ s for $r=0.5$ and $(324 \pm 1)$ s for $r=0.75$ in the form of a 95\% confidence interval. Hence, we have achieved a speed-up of 4.55 times in this problem.

%------------------------------------------------------------------------------

\FloatBarrier