\usepackage{graphicx}
\usepackage{subcaption}
\usepackage{placeins}
\usepackage{physics, amsmath, amssymb}

\usepackage{amsthm}
\newtheorem{proposition}{Proposition}
\newtheorem{definition}{Definition}

\usepackage{pgfplots, pgfplotstable}
\pgfplotsset{
    compat = 1.9, % package version
    xlabel style = {font=\footnotesize, yshift = 1ex},
    ylabel style = {font=\footnotesize, yshift = -1ex},
    xticklabel style = {font=\scriptsize},
    yticklabel style = {font=\scriptsize},
    legend style = {font=\scriptsize},
    title style = {font=\normalsize, yshift = -1ex},
    minor tick num = 1,
    grid = both,
    major grid style = {lightgray},
    minor grid style = {lightgray!40},
    width=0.45\textwidth,
    %height=0.4\textwidth
}
\usepackage{tikz}
\usepackage{xifthen} % for using conditionals with tikz
%%%%%% For drawing 2D grids in 3D (start) %%%%%%
% Linear transformation about the input coordinates (used within a scope!)
\newcommand{\myGlobalTransformation}[2]
{
    \pgftransformcm{1}{0}{0.8}{0.4}{\pgfpoint{#1cm}{#2cm}}
}

% Define "circle split part fill" for circle split
\usetikzlibrary{shapes,backgrounds,calc}
\makeatletter
\tikzset{circle split part fill/.style  args={#1,#2}{%
    alias=tmp@name,%
    postaction={%
     insert path={\pgfextra{%
      \pgfpointdiff{\pgfpointanchor{\pgf@node@name}{center}}%
      {\pgfpointanchor{\pgf@node@name}{east}}%            
      \pgfmathsetmacro\insiderad{\pgf@x}%
      \fill[#1] (\pgf@node@name.base) ([xshift=-\pgflinewidth]\pgf@node@name.east) arc (0:180:\insiderad-\pgflinewidth)--cycle;
      \fill[#2] (\pgf@node@name.base) ([xshift=\pgflinewidth]\pgf@node@name.west) arc (180:360:\insiderad-\pgflinewidth)--cycle;
}}}}}
\makeatother
%%%%%% For drawing 2D grids in 3D (end) %%%%%%

\usepackage{hyperref}
\hypersetup{
    colorlinks=true,
    linkcolor=blue,
    filecolor=magenta,      
    urlcolor=cyan,
}

\usepackage{comment} % need removing at the end

% \usepackage{color}

%Prevent name appearing after email address (for single author)
\usepackage{etoolbox}
% \patchcmd{<cmd>}{<search>}{<replace>}{<success>}{<failure>}
\patchcmd{\emailauthor}{(#2)}{}{}{}
\patchcmd{\urlauthor}{(#2)}{}{}{}