% \documentclass[journal=nalefd,manuscript=letter]{achemso}
\documentclass[reprint,onecolumn]{revtex4-1}
\usepackage{amsmath}
\usepackage{physics}
\usepackage{hyperref}
\usepackage{xfrac}
\usepackage[flushleft]{threeparttable}
\usepackage{array}
\usepackage[T1]{fontenc}
\usepackage[latin9]{luainputenc}
\usepackage{booktabs}
\usepackage{units}
\usepackage{mathrsfs}
\usepackage{multirow}
\usepackage{graphicx}
\usepackage{setspace}
\graphicspath{ {Figures/} }
% \newcommand*\mycommand[1]{\texttt{\emph{#1}}}
\renewcommand\thefigure{S\arabic{figure}}
\renewcommand\thetable{S\arabic{table}}
\renewcommand\theequation{S\arabic{equation}}   
\renewcommand{\thesection}{\Roman{section}}
\renewcommand{\thesubsection}{\Roman{section}.\Alph {subsection}}
\setcounter{section}{0}
\setcounter{equation}{0}
\setcounter{table}{0}
% \usepackage{lineno}
% \linenumbers
% \renewcommand\linenumberfont{\normalfont\small}
% \setlength\linenumbersep{1cm}
\makeatletter

% %%%%%%%%%%%%%%%%%%%%%%%%%%%%%% LyX specific LaTeX commands.
% %% Because html converters don't know tabularnewline
% \providecommand{\tabularnewline}{\\}

% \makeatother

% \usepackage[english]{babel}
%%%%%%%%%%%%%%%%%%%%%%%%%%%%%%%%%%%%%%%%%%%%%%%%%%%%%%%%%%%%%%%%%%%%%%
%%                        TITLE Block    							%%
%%%%%%%%%%%%%%%%%%%%%%%%%%%%%%%%%%%%%%%%%%%%%%%%%%%%%%%%%%%%%%%%%%%%%%
\begin{document}
\title{{\bf Supplementary Information} \protect\\ Purification of single photons by temporal heralding of quantum dot sources}

\author{Hamza Abudayyeh}%
\author{Boaz Lubotzky}
\author{Ronen Rapaport}
 \email{ronenr@phys.huji.ac.il}
\affiliation{%
Racah Institute for Physics and Center for Nanoscience and Nanotechnology, The Hebrew University of Jerusalem, Jerusalem 9190401, Israel
}
\author{Somak Majumder}
\author{Jennifer A. Hollingsworth}
\affiliation{Materials Physics \& Applications Division: Center for Integrated Nanotechnologies, Los Alamos National Laboratory, Los Alamos, New Mexico 87545, United States}





\maketitle

\section{Theoretical derivation of heralding parameters}

The quantum yield of an emission process $i$ is given by the ratio
of the radiative decay rate $\Gamma_{i}^{r}$ to total decay rate
$\Gamma_{i}$:

\[
QY_{i}=\frac{\Gamma_{i}^{r}}{\Gamma_{i}}
\]

For a biexciton-exciton emission cascade simple statistical scaling
implies that the radiative decay rate of the biexciton should be 4
times that of the exciton. In reality other factors play a role such
as the degree of confinement of the electrons and holes to the core
of a core/shell quantum dot for example. Therefore in general this
scaling factor will be called $\beta$, i.e. $\Gamma_{BX}^{r}=\beta\Gamma_{X}^{r}$
and thus:

\[
\begin{array}{ccc}
\frac{QY_{BX}}{QY_{X}} & = & \frac{\Gamma_{BX}^{r}\Gamma_{X}}{\Gamma_{BX}\Gamma_{X}^{r}}\\
 & = & \frac{\beta\Gamma_{X}^{r}\Gamma_{X}}{\Gamma_{BX}\Gamma_{X}^{r}}\\
 & = & \beta\frac{\Gamma_{X}}{\Gamma_{BX}}
\end{array}
\]

Therefore if the $\beta$-scaling factor is known a simple lifetime
measurement can be used to find the ratio of the biexciton to exciton
quantum yield. 

We will assume that our nanocrystal quantum dot can emit at most two
photons based on the biexciton-exciton cascade and that it is being
pumped well above saturation. The probability that a biexciton photon
is emitted $T_{BX}$ after the laser pulse is given by:

\begin{equation*}
p(T_{BX})=\alpha QY_{BX}\Gamma_{BX}e^{-\Gamma_{BX}T_{BX}}
\end{equation*}

whereas the exciton emission time ($T_{X}$) is conditioned on the
emission time of the biexciton as follows:

\begin{equation*}
p\left(T_{x}\right|\left.T_{BX}\right)=\alpha QY_{X}\Gamma_{X}e^{-\Gamma_{X}\left(T_{X}-T_{BX}\right)}
\end{equation*}

\begin{tabular}{llp{13 cm}l}
\multicolumn{2}{l}{where:} & \tabularnewline
 & $\alpha$: & overall detection efficiency of the system\tabularnewline
 & $QY_{i}$: & quantum yield of the $i^{\text{th}}$ emission process\tabularnewline
 & $p(T_{BX})$: & probability of biexciton event at local time $T_{BX}$\tabularnewline
 & $p\left(T_{x}\right|\left.T_{BX}\right):$ &{conditional probability of exciton event at local time $T_{X}$ given
that a biexciton event was detected at $T_{BX}$}\tabularnewline
\end{tabular}

Therefore for a standalone NQD the probability to emit one and two photons ($P_1$ and $P_2$ respectively) are:
\begin{equation*}
P_1=\alpha QY_X + \alpha QY_{BX} - 2 \alpha^2 QY_X QY_{BX} 
\end{equation*}
\begin{equation*}
P_2=\alpha^2 QY_X QY_{BX} 
\end{equation*}
\subsection{Fixed gate techniques}

If we assume that a fixed time gate is applied as in the case of TGF
and TIMED than we can ask what is the probability that a photon will
take a certain route. For example what is the probability that one
photon will arrive within $T$ from the laser pulse and the other
photon will arrive after $T$. This probability is given by the following
integral:

\begin{eqnarray*}
\mathscr{\mathcal{\mathscr{P}}}_{1}(T) & = & \alpha^{2}QY_{X}QY_{BX}\int_{0}^{T}dT_{BX}\int_{T}^{\infty}dT_{X}\,p(T_{BX})\,p\left(T_{x}\right|\left.T_{BX}\right)\nonumber \\
 & = & \alpha^{2}QY_{X}QY_{BX}\frac{\Gamma_{BX}}{\Gamma_{X}-\Gamma_{BX}}\exp\left(-\Gamma_{X}T\right)\left\{ 1-\exp\left(-\left(\Gamma_{BX}-\Gamma_{X}\right)T\right)\right\} 
\end{eqnarray*}

On the other hand the probability that both will arrive prior to $T$
$\left(\mathscr{\mathcal{\mathscr{P}}}_{2}(T)\right)$ or after $T$
$\left(\mathscr{\mathcal{\mathscr{P}}}_{3}(T)\right)$ are given by
the following integrals:

\[
\mathscr{\mathcal{\mathscr{P}}}_{2}(T)=\alpha^{2}QY_{X}QY_{BX}\int_{0}^{T}dT_{BX}\int_{T_{BX}}^{T}dT_{X}\,p(T_{BX})\,p\left(T_{x}\right|\left.T_{BX}\right)
\]

\[
\mathscr{\mathcal{\mathscr{P}}}_{3}(T)=\alpha^{2}QY_{X}QY_{BX}\int_{T}^{\infty}dT_{BX}\int_{T_{BX}}^{\infty}dT_{X}\,p(T_{BX})\,p\left(T_{x}\right|\left.T_{BX}\right)
\]

These events in addition to all other possibilities  are summarized in
table \ref{table: emission_paths}.
\begin{table*}
\caption{Possible events resulting from an emitter that has a maximum of a
two photon cascade by applying a fixed time gate\label{table: emission_paths}}
% \begin{onehalfspace}
\begin{centering}
\resizebox{\textwidth}{!}{%
\begin{tabular}{|c|ccl|}
\cline{2-4} 
\multicolumn{1}{c|}{\#} & Condition on BX & Condition on X & Prob. of Process\tabularnewline
\hline 
1. & $T_{BX}\le T$ & $T_{X}\ge T$ & $\mathscr{\mathcal{\mathscr{P}}}_{1}(T)=\alpha^{2}QY_{X}QY_{BX}\frac{\Gamma_{BX}}{\Gamma_{BX}-\Gamma_{X}}\left\{ \exp\left(-\Gamma_{X}T\right)-\exp\left(-\Gamma_{BX}T\right)\right\} $\tabularnewline
\hline 
2. & $T_{BX}\le T$ & $T_{X}\le T$ & $\mathscr{\mathcal{\mathscr{P}}}_{2}(T)=\alpha^{2}QY_{BX}QY_{x}\left\{ 1-\frac{\Gamma_{BX}}{\Gamma_{BX}-\Gamma_{X}}\exp\left(-\Gamma_{X}T\right)+\frac{\Gamma_{X}}{\Gamma_{BX}-\Gamma_{X}}\exp\left(-\Gamma_{BX}T\right)\right\} $\tabularnewline
\hline 
3. & $T_{BX}\ge T$ & $T_{X}\ge T$ & $\mathscr{\mathcal{\mathscr{P}}}_{3}(T)=\alpha^{2}QY_{BX}QY_{x}\exp\left(-\Gamma_{BX}T\right)$\tabularnewline
\hline 
4. & $T_{BX}\le T$ & \textemdash{} & $\mathscr{\mathcal{\mathscr{P}}}_{4}(T)=\alpha QY_{BX}(1-\alpha QY_{X})\left\{ 1-\exp\left(-\Gamma_{BX}T\right)\right\} $\tabularnewline
\hline 
5. & $T_{BX}\ge T$ & \textemdash{} & $\mathscr{\mathcal{\mathscr{P}}}_{5}(T)=\alpha QY_{BX}(1-\alpha QY_{X})\exp\left(-\Gamma_{BX}T\right)$\tabularnewline
\hline 
6. & \textemdash{} & $T_{X}\le T$ & $\mathscr{\mathcal{\mathscr{P}}}_{6}(T)=\alpha QY_{X}(1-\alpha QY_{BX})\left\{ 1-\exp\left(-\Gamma_{X}T\right)\right\} $\tabularnewline
\hline 
7. & \textemdash{} & $T_{X}\ge T$ & $\mathscr{\mathcal{\mathscr{P}}}_{7}(T)=\alpha QY_{X}(1-\alpha QY_{BX})\exp\left(-\Gamma_{X}T\right)$\tabularnewline
\hline 
8. & \textemdash{} & \textemdash{} & $\mathscr{\mathcal{\mathscr{P}}}_{8}=\left(1-\alpha QY_{BX}\right)\left(1-\alpha QY_{X}\right)$\tabularnewline
\hline 
\end{tabular}} 
\par\end{centering}
% \end{onehalfspace}

\end{table*}


In Time Gated Filtering (TGF) shown in figure \ref{fig: TGF}a the switch is open to an optical dump from the beginning of each pulse up to a filtering time $T_F$ after which the photons are routed to the signal port.
In such a manner the short lifetime components of an optical signal can be filtered out.
This can be used to filter out the biexciton emission where the  the efficiency of the source
would be the probability of only one photon events after a time gate
of $T_{F}$ namely:

\begin{eqnarray*}
\eta_{TGF} & = & \mathscr{\mathcal{\mathscr{P}}}_{1}(T_{F})+\mathscr{\mathcal{\mathscr{P}}}_{5}(T_{F})+\mathscr{\mathcal{\mathscr{P}}}_{7}(T_{F})
\end{eqnarray*}

where as the purity is the same quantity normalized by the probability
of obtaining one or two photons after $T_{F}$ which is given by:

\begin{eqnarray*}
S_{TGF} & = & \frac{\mathscr{\mathcal{\mathscr{P}}}_{1}(T_{F})+\mathscr{\mathcal{\mathscr{P}}}_{5}(T_{F})+\mathscr{\mathcal{\mathscr{P}}}_{7}(T_{F})}{\mathscr{\mathcal{\mathscr{P}}}_{1}(T_{F})+\mathscr{\mathcal{\mathscr{P}}}_{3}(T_{F})+\mathscr{\mathcal{\mathscr{P}}}_{5}(T_{F})+\mathscr{\mathcal{\mathscr{P}}}_{7}(T_{F})}
\label{eq: S_TGF}
\end{eqnarray*}
This case of TGF with QY$_X$=0.6 and QY$_{BX}$=0.71 \cite{Matsuzaki2017StrongAntenna} is plotted in figure \ref{fig: TGF}b where the trade-off between purity and efficiency in terms of $T_F$ is clear. 

For the special case of unity collection efficiency and unity quantum
yields the expressions reduce to:

\begin{eqnarray*}
\eta_{TGF}^{ideal} & = & \frac{\Gamma_{BX}}{\Gamma_{BX}-\Gamma_{X}}\left\{ \exp\left(-\Gamma_{X}T_{F}\right)-\exp\left(-\Gamma_{BX}T_{F}\right)\right\} 
\end{eqnarray*}

\begin{eqnarray*}
S_{TGF}^{ideal} & = & \frac{\Gamma_{BX}\left\{ \exp\left(-\Gamma_{X}T_{F}\right)-\exp\left(-\Gamma_{BX}T_{F}\right)\right\} }{\Gamma_{BX}\exp\left(-\Gamma_{X}T_{F}\right)-\Gamma_{X}\exp\left(-\Gamma_{BX}T_{F}\right)}
\end{eqnarray*}

% Note that as in standalone NQDs the efficiency of the source also
% determines its determinicity (how well one can predict the desired
% event).

On the other hand for the TIMe resolved heraldED (TIMED) scheme only
the first process constitutes a successful heralding event namely:

\begin{eqnarray*}
\eta_{TIMED} & = & \mathscr{\mathcal{\mathscr{P}}}_{1}(T_{C})=\alpha^{2}QY_{X}QY_{BX}\frac{\Gamma_{BX}}{\Gamma_{BX}-\Gamma_{X}}\left\{ \exp\left(-\Gamma_{X}T_{C}\right)-\exp\left(-\Gamma_{BX}T_{C}\right)\right\} 
\end{eqnarray*}

By using $\tau_{i}=\Gamma_{i}^{-1}$ for $i=X,BX$ we arrive at the
equation stated in the main text.

We can choose the optimum cutoff time $T_{c}^{opt}$ that will maximize
$\eta$. This is given in units of the exciton lifetime as:

\begin{equation*}
\Gamma_{x}T_{c}^{opt}=\left(\frac{\Gamma_{BX}}{\Gamma_{X}}-1\right)^{-1}\ln\left(\frac{\Gamma_{BX}}{\Gamma_{X}}\right)\label{eq:tcopt}
\end{equation*}

Using this the optimum efficiency $\eta^{opt}$ is :

\begin{equation*}
\eta_{TIMED}^{opt}=\alpha^{2}QY_{X}QY_{BX}\frac{\nicefrac{\Gamma_{BX}}{\Gamma_{X}}}{\nicefrac{\Gamma_{BX}}{\Gamma_{X}}-1}\left\{ \left(\frac{\Gamma_{BX}}{\Gamma_{X}}\right)^{\frac{-1}{\nicefrac{\Gamma_{BX}}{\Gamma_{X}}-1}}-\left(\frac{\Gamma_{BX}}{\Gamma_{X}}\right)^{\frac{-\nicefrac{\Gamma_{BX}}{\Gamma_{X}}}{\nicefrac{\Gamma_{BX}}{\Gamma_{X}}-1}}\right\} \label{eq:etaopt}
\end{equation*}

An important parameter for the case
of multiplexed sources is the determinicity defined as the ratio of
true heralded events to overall trigger events. This parameter defines
the reliability of the trigger signal in heralding a signal photon.
In statistics this often called the Positive Predictive Value (PPV)
defined as the ratio of true postives to overall postives, but we
will stick to the term determinicity for comparasion with non-heralded
sources. For the TIMED scheme the determinicity $D$ is given by:

\begin{equation*}
D_{TIMED}=\frac{\mathscr{\mathcal{\mathscr{P}}}_{1}(T_{C})}{\mathscr{\mathcal{\mathscr{P}}}_{1}(T_{C})+\mathscr{\mathcal{\mathscr{P}}}_{2}(T_{C})+\mathscr{\mathcal{\mathscr{P}}}_{4}(T_{C})+\mathscr{\mathcal{\mathscr{P}}}_{6}(T_{C})}
\end{equation*}

Again for the idealistic case of unity collection efficiency and quantum
yields this reduces to:

\begin{equation*}
D_{TIMED}^{ideal}=\frac{\Gamma_{BX}}{\Gamma_{BX}-\Gamma_{X}}\frac{\exp\left(-\Gamma_{X}T_{C}\right)-\exp\left(-\Gamma_{BX}T_{C}\right)}{1-\exp\left(-\Gamma_{BX}T\right)}
\end{equation*}

\subsection{Active Switching Heralded (ASH) scheme}

Due to the active switching the only parameter of importance for this
scheme is the resolution time $T_{R}$ of the system. It is the difference
in the arrival time between the two photons that needs to be taken
into consideration. There are five possible outcomes:
\begin{itemize}
\item Two photons with temporal seperation more than $T_{R}$ (successful
heralding event) : 
\begin{eqnarray*}
\mathscr{\mathcal{\mathscr{P}}}_{2S} & = & \alpha^{2}QY_{X}QY_{BX}\int_{0}^{\infty}dT_{BX}\int_{T_{BX}+T_{R}}^{\infty}dT_{X}\,p(T_{BX})\,p\left(T_{x}\right|\left.T_{BX}\right)\\
 & = & \alpha^{2}QY_{X}QY_{BX}\exp\left(-\Gamma_{X}T_{R}\right)
\end{eqnarray*}
\item Two photons with temporal seperation arriving within $T_{R}$ from
each other:
\begin{eqnarray*}
\mathscr{\mathcal{\mathscr{P}}}_{2F} & = & \alpha^{2}QY_{X}QY_{BX}\int_{0}^{\infty}dT_{BX}\int_{T_{BX}}^{T_{BX}+T_{R}}dT_{X}\,p(T_{BX})\,p\left(T_{x}\right|\left.T_{BX}\right)\\
 & = & \alpha^{2}QY_{X}QY_{BX}\left\{ 1-\exp\left(-\Gamma_{X}T_{R}\right)\right\} 
\end{eqnarray*}
\item Exciton photon emission only:
\begin{eqnarray*}
\mathscr{\mathcal{\mathscr{P}}}_{1X} & = & \alpha QY_{X}\left(1-\alpha QY_{BX}\right)
\end{eqnarray*}
\item Biexciton photon emission only: 
\begin{eqnarray*}
\mathscr{\mathcal{\mathscr{P}}}_{1BX} & = & \alpha QY_{BX}\left(1-\alpha QY_{X}\right)
\end{eqnarray*}
\item No photon emission: 
\begin{eqnarray*}
\mathscr{\mathcal{\mathscr{P}}}_{0} & = & \left(1-\alpha QY_{X}\right)\left(1-\alpha QY_{BX}\right)
\end{eqnarray*}
\end{itemize}
Only the first of these events is a successful heralding event therefore
the efficiency is given by:

\begin{equation*}
\eta_{ASH}=\alpha^{2}QY_{X}QY_{BX}\exp\left(-\Gamma_{X}T_{R}\right)
\end{equation*}

On the other hand all but the last of these outcomes still will consitute
a trigger event therefore the determinicity can be written as: 

\begin{eqnarray*}
D_{ASH} & = & \frac{\mathscr{\mathcal{\mathscr{P}}}_{2S}}{\mathscr{\mathcal{\mathscr{P}}}_{2S}+\mathscr{\mathcal{\mathscr{P}}}_{2F}+\mathscr{\mathcal{\mathscr{P}}}_{1X}+\mathscr{\mathcal{\mathscr{P}}}_{1BX}}\nonumber \\
 & = & \frac{\alpha^{2}QY_{X}QY_{BX}\exp\left(-\Gamma_{X}T_{R}\right)}{\alpha QY_{X}+\alpha QY_{BX}+\alpha^{2}QY_{X}QY_{BX}}
\end{eqnarray*}

\begin{figure}
\begin{centering}
\includegraphics{supplementary_fig1}
\par\end{centering}
\caption{(a) Schematic of Time-Gated Filtering technique (TGF). (b) The purity (solid line) and efficiency (dashed lines) of the TGF technique for the quantum yields corresponding to a gQD coupled to a nanocone \cite{Matsuzaki2017StrongAntenna}. This is compared to the efficiency of the ASH and TIMED technique \label{fig: TGF}}
\end{figure}

\begin{figure}
\begin{centering}
\includegraphics{determinicity}
\par\end{centering}
\caption{Determinicity as a function of the quantum yield for ASH ($T_{R}<<\tau_X)$
and TIMED ($T_{C}^{opt})$ as compared to an ideal SPS \label{fig:Determinicity}}
\end{figure}

To establish the relevance of these techniques we compare the efficiency of ASH and TIMED to the simpler TGF technique for the case of $QY_X=0.61$ and $QY_{BX}=0.7$ (see main text) \cite{Matsuzaki2017StrongAntenna} in figure \ref{fig: TGF}b. 
For TGF $T_F>1.8$ (indicated by the dotted line) is needed to produce the same purity (0.995)  measured in our heralded experiments.
For this value of $T_F$ it is clear that the efficiency of both ASH and TIMED are superior to TGF.

Figure \ref{fig:Determinicity} displays the determinicity as a function
of the quantum yield for equal quantum yields (solid line) and for
the case where $QY_{BX}=0.5QY_{X}$ (dashed line). As can be clearly
seen the ASH scheme has a determinicity of unity for unity quantum
yields but TIMED reaches to a maximum of around 75\% when the optimum
cutoff time is used. It should be noted however that this cutoff time
was optimized for efficiency not determinicity and in principle one
can operate at higher determinicities at the cost of lower efficiency.
In the non-ideal case $\left(QY_{BX}=0.5QY_{X}\right)$ the TIMED
scheme overcomes the ASH scheme since in this case false positives
start to appear. ASH is more susceptible to false triggers since it
is always open to the idler port until a trigger photon has arrived
, whereas the TIMED scheme switches after $T_{C}$ regardless of the
presence of a trigger photon or not.

Determinicity is an important parameter for a multiplexed heralded source.
This parameter is important because it defines the upper limit for the overall efficiency of the multiplexed source since the optical router will rely on the trigger signal to switch between different components.
The schemes proposed in this paper have maximum theoretical determinicities that approach unity for optimized parameters. 


\section{Experimental Details}

\subsection{Setup and data format }

The measurement is conducted by spin coating a sample of core/thick
shell CdSe/CdS nanocrystal quantum dots dispersed in PMMA onto a silicon
wafer. The input laser light is a femtosecond 2 MHz 405 nm pulsed
laser generated by second harmonic generation of a femtosecond Ti:Sapphire
laser operating at 810 nm. The sample is scanned using a periscopic
system of scanning stages and the laser light is focused onto the
sample using a 0.9 NA objective (Olympus MPLFLN100xBD). The emission from a single NQD
is collected using the same objective and directed to the collection
arm using a 600nm shortpass dichroic mirror. The sample is then imaged
using a CMOS camera to verify the excitation of a single NQD. After that the emission is redirected to a set of beam-splitters and single
photon avalanche photodiodes (Excelitas SPCM-AQRH-14-FC) as shown in figure 3a in the
main text. 

The signal from each detector is routed to a different channel on
the timetagger device (Swabian TimeTagger 2.0). The output from the timetagger to the computer
is a vector containing the arrival times of all the photons within
the set exposure time with respect to the beginning of the measurement
in addition to a tag labeling which channel this count came from.
These arrival times are commonly referred to as global times. Another
channel on the time-tagger records the excitation pulse times. By
comparing the global time of each count with the nearest preceding
laser pulse one can find the local time for the count. A histogram
of these local times is what constitutes a lifetime measurement. Therefore
at the end of this step we have information about which pulse and
what channel the count came from, and its global and local times.
This constitutes all the information needed for subsequent analysis
steps.

\subsection{Photon number correction}

Due to the deadtime of each detector, counts from the same detector
within the same pulse must be neglected. If this was not conducted
then a bias would be introduced since two photons arriving with a
time separation less than the dead time would not be detected whereas
photons separated by more than the deadtime would be. However this
does introduce an error in estimating the two and three photon counts
due to the finite probability that two or more photons would
arrive at the same detector. To correct for this error we must the
calculate the probability for the photons to arrive at different detectors
and correct the photon counts by dividing by this probability. We
have a two beamsplitter scheme the first with a reflectivity $R_{1}$ and
the second with a reflectivity $R_{2}$. Assuming two photons were
emitted the probability that they would arrive at different detectors
is a simple exercise in probabilities and is given by:

\[
P_{diff}^{(2)}=2R_{1}\left(1-R_{1}\right)+\left(1-R_{1}\right)^{2}\left[2R_{2}\left(1-R_{2}\right)\right]
\]
If the pulse contains three photons the probability of them each arriving
at a different detector is given by:

\[
P_{diff}^{(3)}=3R_{1}\left(1-R_{1}\right)^{2}\left[2R_{2}\left(1-R_{2}\right)\right]=6R_{1}R_{2}\left(1-R_{1}\right)^{2}\left(1-R_{2}\right)
\]
Therefore the ``true'' photon numbers ($N_{2}$ and $N_{3}$) are
related to the measured ones ($N_{2m}$ and $N_{3m}$) by:

\begin{eqnarray*}
N_{2} & = & \frac{N_{2m}}{P_{diff}^{(2)}}\\
N_{3} & = & \frac{N_{3m}}{P_{diff}^{(3)}}
\end{eqnarray*}
In our case $R_{1}=0.4$ , $R_{2}=0.5$ and therefore:

\begin{eqnarray*}
N_{2} & = & 1.52\,N_{2m}\\
N_{3} & = & 4.63\,N_{3m}
\end{eqnarray*}
From now on it will be assumed that this correction is conducted whenever
there is a two photon or three photon event. 

\subsection{Analysis Method}

% \subsubsection*{TGF}

% Using the data as set before we throw out all counts with local times
% less than $T_{F}$ and with the remaining data we do the following:
% \begin{itemize}
% \item A correlation measurement on the remaining global data to obtain the
% second order correlation function.
% \item Calculate the number of two photon $\left(N_{2}\right)$ and one photon
% $\left(N_{1}\right)$ events and calculate the purity using $S=1-\nicefrac{N_{2}}{\alpha N_{1}}$.
% The factor of $\alpha$ is present because the probability of collecting
% two photons emitted by the NQD is $\alpha^{2}$ where the probability
% to collect a single photon is $\alpha$. Since we are interested in
% the purity of the photons emitted from the source rather than the
% photons seen in the detectors we must normalize by this factor. Furthermore
% we make two more justifiable assumptions in this formula. First we
% assume that the 3 photon probability is negligible and indeed the three photon probability
% is three orders of magnitude lower than the two photon probability.
% Also we assume that $N_{2}\ll N_{1}$ which is also the case. 
% \end{itemize}

% \subsubsection*{Heralded Schemes}
Prior to purification the purity of the QD was calculated by comparing the number of multiphoton ($N_{\ge2}$) events to single photon events ($N_{1}$) using this formula:  
$$S= \frac{N_1}{N_1+\nicefrac{N_{2}}{\alpha}+\nicefrac{N_{3}}{\alpha^2}} $$
Higher order terms were neglected due to the much lower probability of four or more emission events. This also is consistent with the ability to resolve up to three photons in our setup.   
The factor of $\alpha$ in the denominator is to account for the extra optical loss that a two photon and three photon states encounter. 

In the heralded schemes we first filter out all counts having local
times less than $T_{F}$=300 ps to reduce the effect of correlated noise.
Then all pulses containing two or more photons were chosen for further
analysis. 

For the TIMED scheme the condition implemented afterwards is that
one (or more) photons had a local time less than $T_{C}$ and one
(or more) photons had a local time more than $T_{C}$. The number
of these events constituted the number of successful heralding events
and divided by the overall number of excitation pulses yields the
heralding efficiency. This is what is
plotted in figure 3d in the main text. From these figures we chose
the $T_{C}$ that yields the maximum efficiency and for the successful
heralding events we count the number of photons in each case that
arrive after $T_{C}$ which will be called the signal photons. The
number of cases in which we have two signal photons per heralding
event $N_{2}$ compared to the number of cases where there is only
one $N_{1}$ gives the purity: $S=1-\nicefrac{N_{2}}{\alpha N_{1}}$
which can be plotted as a function of the original filtering time
$T_{F}$ as in figure 3e in the main text. The factor of $\alpha$ in the denominator is to account for the extra optical loss that a two photon state encounters. 

In the ASH technique the time differences between the two (or more)
photons is calculated. If this time difference is more than $T_{R}$
then this is counted as a successful heralding event and the efficiency
is calculated as in the previous
case. Again in these heralding events the number of photons following
the trigger photon is counted ($N_{2}$ if there is 2 and $N_{1}$
if there is 1) and the purity can be calculated using the same formula
as above as a function of the filtering time. 

\subsection{Collection efficiency estimation}

To estimate the collection efficiency in our setup we take into account the efficiencies of the various optical components using three techniques:
\begin{itemize}
\item Simulation (sim) of collection efficiency into objective by using
a commercial FDTD software (Lumerical)
\item Transmission/Reflection measurement using a 655 nm diode laser (meas).
\item Factory data (fact)
\end{itemize}
The efficiency of our system is estimated in table \ref{tab: Coll_eff}. 

\begin{table}
\caption{System collection efficiency estimation}\label{tab: Coll_eff}
\begin{centering}
\begin{tabular}{|c|c|c|}
\hline 
Component & Method & Efficiency\tabularnewline
\hline 
\hline 
Collection into objective & sim & 39.0 \%\tabularnewline
\hline 
Objective transmission & meas & 90.0 \%\tabularnewline
\hline 
600 nm SP dichroic & meas & 96.6 \%\tabularnewline
\hline 
700 nm SP filter & meas & 88.5 \%\tabularnewline
\hline 
550 nm LP filter & meas & 95.2 \%\tabularnewline
\hline 
600 nm LP filter & meas & 84.3 \%\tabularnewline
\hline 
Beam splitters (2) & meas & 86.3 \%\tabularnewline
\hline 
Mirrors (6) & meas & 75.9 \%\tabularnewline
\hline 
Fiber coupling & meas & 80.0 \%\tabularnewline
\hline 
Detector efficiency & fact & 70.0 \%\tabularnewline
\hline 
\multicolumn{2}{|r|}{\textbf{Total}} & \textbf{8.83 \%}\tabularnewline
\hline 
\end{tabular}
\par\end{centering}


\end{table}

\subsection{Effect of noise on single photon purity }
Experimentally we expect that the sole source reducing $S$ in the signal port is from noise. 
To check this point we show the purity, $S$, of both schemes as a function of the filtering time $T_F$ after the excitation pulse. 
We can see from figure 3e in the main text that the ASH purity improves dramatically with even 1 ns of filtering which fits well with the extracted correlated noise lifetime. 
The TIMED scheme, on the other hand, is hardly affected by the filtering. 
This difference in sensitivity can be understood since in the TIMED scheme we apply a passive gate, so all short-lifetime counts will by default be directed to the idler port without affecting the fraction of photons arriving to the signal port. 
On the other hand, in the ASH scheme a short lifetime count will lead to premature "switching" to the signal port causing both bi-exciton and exciton photons to be directed to the signal port. 
To confirm that that the short-lifetime counts are indeed from correlated noise counts and not from higher order multiexcitons we estimate their rate based on the measured ASH purity at $T_F=0$.
This is equivalent to finding the ratio of three photon to two photon events in the raw data. 
If indeed correlated noise is the source for the reduced $T_F=0$ purity, this ratio  should give the probability per pulse to get a correlated noise count which turns out to be $1.4 \times 10^{-3}$ or correspondingly the rate is $\sim 2800$ cps. 
This agrees with the correlated noise level measured at the same excitation power on the same substrate in a region with no quantum dots ($\sim 2500-3500$ cps). 
For $T_F>1$ ns, $S$ reaches a maximum of around 0.995, after filtering out nearly all correlated noise.
Uncorrelated noise is what limits this value from reaching unity. 
Using the procedure as above we find that this noise rate is around 880 cps which is in agreement with our uncorrelated noise rate $\sim$ 600-800 cps.

To confirm these statements we will attempt to reconstruct the measured purities based on the error rates stated above.
In general the single photon purity of a signal can be defined as:
\[
S=\frac{P_{1}}{P_{\ge1}}=1-\frac{P_{\ge2}}{P_{\ge1}}
\]
where $P_{1}$, $P_{\ge1}$ and $P_{\ge2}$ are the probabilities
per relevant event of obtaining a single photon count, at least one
photon , and at least two photons respectively.

Due to the low quantum yields in our case we can assume that $P_{\ge2}\approx P_{2}$ and
therefore:

\begin{equation*}
S=1-\frac{P_{2}}{P_{\ge1}}
\end{equation*}
Note the absence of the $\alpha$ factor here since we are theoretically
considering the events obtained directly from the emitter. Now assume
we have a certain probability per pulse $\eta_{cn}=1.4\times10^{-3}/\alpha$ and $\eta_{un}=4.4\times10^{-4}/\alpha$
of obtaining correlated and uncorrelated noise event respectively. 
We will now study the effect of these noise terms on our measured purity:
% \subsubsection*{TGF}

% Lets take two examples:
% \begin{itemize}
% \item $T_{F}=0$ (i.e. the purity of the original source) for simplicity.
% In the absence of noise we can see from eq. \ref{eq: S_TGF} and table \ref{table: emission_paths} (and neglecting
% the collection efficiency) that the purity can be given by:
% \[
% S_{no\,noise}=1-\frac{QY_{X}QY_{BX}}{QY_{X}+QY_{BX}}\approx0.962
% \]
% where we have used the quantum yields in our measurement $QY_{X}=0.1729$
% and $QY_{BX}=0.0465$. In the presence of noise we have to take into
% account two photon events arising from correlations between the signal
% and noise counts yielding a purity given by:
% \[
% S_{with\,noise}=1-\frac{QY_{X}QY_{BX}+\left(\eta_{cn}+\eta_{un}\right)\left(QY_{X}+QY_{BX}\right)}{QY_{X}+QY_{BX}+\eta_{cn}+\eta_{un}}\approx0.948
% \]
%  where we have kept only the lowest non-vanishing orders in the numerator
% and denominator and we used $\eta_{cn}=1.6\times10^{-2}$, and $\eta_{un}=5\times10^{-3}$
% as described in the main text (and taking into acount the losses in
% the optical setup). This displays that the noise has a minimal effect
% due to the high SNR. 
% \item $T_{F}\gg\tau_{BX},\tau_{cn}$: In this case we can neglect biexciton
% emission and correlated noise all together and the only source of
% two photon counts would be the correlation of exciton emission and uncorrelated noise . Thus
% the purity can be estimated as:
% \[
% S_{with\,noise}=1-\frac{\eta_{un}QY_{X}\exp\left(-\frac{T_{F}}{\tau_{X}}\right)}{QY_{X}\exp\left(-\frac{T_{F}}{\tau_{X}}\right)+\eta_{un}}
% \]
%  for $T_{F}=\tau_{X}$ and using the values above we find that $S_{with\,noise}\approx0.9954$
% and therefore we expect the purity to converge to unity as $T_{F}$
% increases as can be seen in figure 3d in the main text. 
% \end{itemize}

\subsubsection*{Heralded schemes}

For the heralded schemes in principle the purity should be unity unless
there is noise. $P_{2}$ and $P_{\ge1}$
are the probabilities of getting two photons and at least one photon
\textbf{conditioned on the presence of another trigger photon.} For
simplicity lets consider the ASH technique with $T_{R}=0$ in two
regimes:
\begin{itemize}
\item In the presence of correlated noise ($T_{F}=0$) in this case $P_{\ge1}\approx QY_{X}QY_{BX}+\left(\eta_{cn}+\eta_{un}\right)\left(QY_{X}+QY_{BX}\right)$
notice that due to the heralding requirement this is effectively the
probability of two photons in a non heralding scheme. On the other
hand the probability of obtaining two photons in the signal port of
a heralding technique is just the correlation between the noise counts
and the heralded counts given by $P_{2}=QY_{X}QY_{BX}\left(\eta_{cn}+\eta_{un}\right)$
. Therefore by using the values obtained from our experiment we find
that the purity is given by:
\[
S_{with\,noise}=1-\frac{QY_{X}QY_{BX}\left(\eta_{cn}+\eta_{un}\right)}{QY_{X}QY_{BX}+\left(\eta_{cn}+\eta_{un}\right)\left(QY_{X}+QY_{BX}\right)}=0.9867
\]
\item In the absence of correlated noise ($T_{F}\gg\tau_{cn}$) and assuming
that $T_{F}$ is still much shorter than the biexciton lifetime then
we can effectively set $\eta_{cn}=0$ in the previous equation to
obtain:
\[
S_{with\,noise}=1-\frac{QY_{X}QY_{BX}\eta_{un}}{QY_{X}QY_{BX}+\eta_{un}\left(QY_{X}+QY_{BX}\right)}=0.996
\]
This is in good agreement with the results shown in figure 3e of the
main text. 
\end{itemize}

\begin{figure}[ht!]
\begin{centering}
\includegraphics{ASH_implementationscheme.pdf}
\par\end{centering}
\caption{Schemes for the (a) ASH scheme and (b) S/R latch. (c) The heralding efficiency and (d) single photon rate of ASH under different response times as compared to the TIMED technique  \label{fig:Implementation}}
\end{figure}

\section{Practical Details}



\begin{table}[h!]
(a)

\begin{tabular}{|c|c|c|c|}
\hline 
Circuit Component & Type & Description & Value\tabularnewline
\hline 
\hline 
\multirow{4}{*}{Detector} & \multirow{2}{*}{SNSPD} & Latency & $\sim$ 50 ps \cite{Allmaras2018IntrinsicDetectors}\tabularnewline
\cline{3-4} 
 &  & Jitter & 15 ps \cite{SingleQuantumSingleEos}\tabularnewline
\cline{2-4} 
 & \multirow{2}{*}{SPAD} & Latency & $\sim$ 2 ns \cite{Prochazka2011MeasurementTransfer}\tabularnewline
\cline{3-4} 
 &  & Jitter & 50 ps \cite{PicoQuantGmbHPDMSeries}\tabularnewline
\hline 
Logic Circuit & S/R Latch & Propagation delay & 185 ps \cite{ONSemiconductor2.5V/3.3VSmartGate} \tabularnewline
\hline 
Optical Switch & 25 GHz Modulator & Rise time & 15 ps \cite{iXblueNIR-MX800-LN-10}\tabularnewline
\hline 
\multirow{2}{*}{Optical/Electronic Propagation} & On Chip &  & Negligible\tabularnewline
\cline{2-4} 
 & Free Space &  & 500 ps\tabularnewline
\hline 
\end{tabular}

(b)

\begin{tabular}{c|c|c|}
 & Free Space & On-Chip\tabularnewline
\hline 
\hline 
SNSPD & 765 ps & 265 ps\tabularnewline
\hline 
SPAD & 2.75 ns & 2.25 ns\tabularnewline
\hline 
\end{tabular}

\caption{(a) Contributions of various components to $T_R$. (b) The total response time under different configurations. \label{Tab:TR}.}
\end{table}


In this section we will discuss in detail the effect of response time on the performance of ASH.
The diagram of the proposed optical and electronic circuit needed to implement this technique is displayed \ref{fig:Implementation}a and b. We  conduct a detailed analysis of the different components that will contribute to $T_R$ in table \ref{Tab:TR} where the propagation delay of the S/R latch is the composite delay of two NOR gates \cite{ONSemiconductor2.5V/3.3VSmartGate}. 
We consider different situations with either single photon avalanche detectors (SPAD) or semiconductor nanowire single photon detectors (SNSPD) combined with the other components either on-chip or free space. We compare the efficiency and single photon rate of the resulting 4 scenarios in Fig. \ref{fig:Implementation}c and d with a comparison with TIMED.
Depending on the specific implementation the ASH technique will be better than TIMED for $\tau_X$ more than a certain value. 

% \section{Quantum key distribution}
% In this section we will discuss the expected performance of our schemes in QKD protocols as compared to standard decoy state protocols. \cite{Lo2005DecoyDistribution} 
% The figure of merit that is usually chosen for such comparasion is the secure bit rate as a function of the transmission of the quantum channel. 
% Specifically we will consider the BB84 protocol \cite{Bennett1984QuantumTossing} with polarization based encoding in two orthogonal bases leading to a sifting probability of $p_{sift}=1/2$.
% \subsection{Single photon source (SPS)}
% For a single photon source the secure bit rate is \cite{Waks2002SecurityLight}:
% \begin{equation}
% R_{SPS}=\frac{1}{2}P_c \left[\beta\tau(e) - f(e)h(e)\right] \label{eq:SPS_beg}
% \end{equation}
% where the $1/2$ is to take into account the sifting probability. 

% $P_c$ is the probability per pulse to get a detection event. 
% This can be written in terms of the source efficiency $\eta$, purity ($S$), the total optical loss from the quantum channel and Bob's apparatus ($T$) and the dark count probability ($d$). 
% The probability of one photon detection is just $P_1^{det}=\eta S T$, whereas the probability for two photon detection is just $P_2^{det}=(1-S)\eta \left[1-(1-T)^2\right]$. 
% Therefore:
% \begin{equation}
% P_c=P_1^{det}+P_2^{det}+d= \eta S T + \eta(1-S) \left[1-(1-T)^2\right] + d
% \end{equation}

% The parameter $\beta$ is the purity of the photons arriving at the detection port. Taking into account only one and two photon events this can be given as:
% \begin{equation}
% \beta \approx 1-\frac{P_2^{det}}{P_1^{det}}=1-\left(\frac{1-S}{S}\right)\frac{1-(1-T)^2}{T}
% \end{equation}

% To account for Eve's attacks error correction and privacy amplification are carried out and these steps are included using $f(e)$ and $\tau(e)$ respectively, where $e$ is the error rate. $f(e)$ characterizes the error correction algorithm and is usually taken to be 1.22 \cite{Leifgen2014EvaluationDistributionb}. The compression function $\tau(e)$, on the other hand is given by:
% \begin{equation}
% \tau(e)= -\log_2\left[\frac{1}{2}+2\frac{e}{\beta}-2\left(\frac{e}{\beta}\right)^2\right]
% \end{equation}

% The function $h(e)$ is the Shannon entropy of a single bit given by:
% \begin{equation}
% h(e)=-e\log_2(e)-\left(1-e\right)\log_2\left(1-e\right)
% \end{equation}

% Two components contribute to the error rate $e$. 
% The first comes from the signal due to imperfect preparation, imperfect optics or decoherence in the quantum channel. 
% This can be accounted for using a baseline error rate $e_0$ which we set to 2\% in our case.
% The other source of error is the dark counts at Bob's unit characterized by the dark count probability $d$. Each dark count is completely uncorrelated with the signal and therefore causes a 50\% error rate. 
% Using this we can calculate what the probability of an innocent error is as:
% \begin{equation}
% e=\frac{e_0\left(P_1^{det}+P_2^{det}\right)+d/2}{P_c}\label{eq:SPS_end}
% \end{equation}
% \begin{table}
% \renewcommand{\arraystretch}{2}
% \begin{tabular}{|c|c|c|c|c|}
% \cline{2-5} 
% \multicolumn{1}{c|}{} & TIMED & ASH & Ideal SPS & Infinite Decoy\tabularnewline
% \hline 
% $\eta$ & 0.15 & 0.22 & 1 & ---\tabularnewline
% \hline 
% $S$ & 0.995 & 0.995 & 1 & \textemdash{}\tabularnewline
% \hline 
% $\mu$ & \textemdash{} & \textemdash{} & \textemdash{} & 0.48 \cite{Ma2005PracticalDistribution}\tabularnewline
% \hline 
% Eqns. & \multicolumn{3}{c|}{\ref{eq:SPS_beg}-\ref{eq:SPS_end}} & \ref{eq:IDS_beg}-\ref{eq:IDS_end}\tabularnewline
% \hline 
% $e_{0}$ & \multicolumn{4}{c|}{0.02 \cite{Waks2002SecurityLight}}\tabularnewline
% \hline 
% $d$ & \multicolumn{4}{c|}{$\frac{4\times20\,\text{cps}}{200\text{MHz}}=4\times10^{-8}$}\tabularnewline
% \hline 
% $f(e)$ & \multicolumn{4}{c|}{1.22 \cite{Leifgen2014EvaluationDistributionb}}\tabularnewline
% \hline 
% \end{tabular}
% \caption{Parameter and details used for the calculations shown in figure \ref{fig:keyrate} \label{table: parameters}}
% \end{table}

% \subsection{Infinite Decoy State (IDS)}
% Most QKD systems are based on weak coherent states (WCS) due to the relative ease of implementation of such sources. 
% These sources however occasionally  produce more than one photon opening up the possibility to the photon number splitting (PNS) attack causing the secure bit rate to drop quadratically as a function of the quantum channel transmission $\left(R_{WCS}\propto T^2\right)$.
% This is in comparasion  to a single photon source whose rate scales linearly with $T$.
% Decoy states were later proposed \cite{Hwang2003QuantumCommunication} to restore the linear scaling and improve the performance of WCS for QKD applications. 
% The idea of decoy states is that Alice randomly prepares a certain number of intensity states (with average photon numbers $\nu_i$) that are not going to be used for encoding. 
% These decoy states are sent in tandem with the usual WCS state with average photon number $\mu$.
% Eve has no way of determining which pulses contain the true bits and which contain the decoy states since they are sent in random order. 
% Later the information about which pulses contain the encoded bits is transferred over the classical channel. 
% Any interference by Eve can be detected by performing an analysis on the decoy state bits giving Alice and Bob information about the number of bits Eve may have intercepted. 
% This information can be used in the error correction and privacy amplification steps to increase the security of the encoded bits. 

% The best performance is obtained for the infinite decoy state protocol where an infinite number of weak decoy states are used. However it has been shown that two decoy states (one vacuum and one weak coherent state) along with the encoded WCS are enough to obtain security approaching that of the infinite decoy method \cite{Ma2005PracticalDistribution}.
% We will however compare our results to that of the infinite decoy method since it forms an upper limit on what can be achieved using the decoy state protocols. 

% The secure bit rate of an infinite decoy method is given by \cite{Lo2005DecoyDistribution}:
% \begin{equation}
% R_{IDS}\ge \frac{1}{2}\left[-Q_\mu f(e) h(e) + Q_1 \left(1-h(e_1)\right)\right] \label{eq:IDS_beg}
% \end{equation}
% where $Q_\mu$ is the gain of the signal states, $Q_1$ and $e_1$ are the gain and the error rate of the single photon state respectively. These quantities are given by \cite{Ma2005PracticalDistribution}:
% \begin{equation}
% Q_\mu = d + 1- \exp(-T\mu)
% \end{equation}
% \begin{equation}
% e Q_\mu=\frac{1}{2}d+e_0\left( 1- \exp(-T\mu)\right)
% \end{equation}
% \begin{equation}
% Q_1=\left(d+T\right)\mu \exp(-\mu)
% \end{equation}
% \begin{equation}
% e_1=\frac{\frac{1}{2}d+e_0 T }{d+T} \label{eq:IDS_end}
% \end{equation}
% \subsection{Comparison}
% \begin{figure}
% \begin{centering}
% \includegraphics{key_vs_loss}
% \par\end{centering}
% \caption{Secure key rate as a function of the channel loss for the various methods.\label{fig:keyrate}}
% \end{figure}


% We compare the results of the ASH and TIMED scheme based on the model system discussed in the main text with an ideal SPS and infinite decoy state in figure \ref{fig:keyrate}. 
% The parameters and experimental details used in the calculation are summarized in table \ref{table: parameters} where the value of $\mu$ was chosen to optimize the rate for the infinite decoy state \cite{Ma2005PracticalDistribution}.
% The factor of 4 in the dark count probability is due to the four detectors used in Bob's apparatus. 
% As can be clearly seen from the figure our protocols, even under suboptimal parameters, can operate at least as well as any decoy state can. 
% This points to the relevance of our methods for QKD applications. 
\bibliography{Mendeley}
\bibliographystyle{ieeetr}
\end{document}
