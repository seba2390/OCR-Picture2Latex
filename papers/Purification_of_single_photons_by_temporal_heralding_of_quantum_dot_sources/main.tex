%%%%%%%%%%%%%%%%%%%%%%%%%%%%%%%%%%%%%%%%%%%%%%%%%%%%%%%%%%%%%%%%%%%%%%
%%                           PREAMBLE    							%%
%%%%%%%%%%%%%%%%%%%%%%%%%%%%%%%%%%%%%%%%%%%%%%%%%%%%%%%%%%%%%%%%%%%%%%
% \documentclass[reprint,aps,pra]{revtex4-1}
\documentclass[journal=nalefd,manuscript=letter]{achemso}

% \usepackage{authblk}
\usepackage{amsmath}
% \usepackage{physics}
\usepackage{hyperref}
\usepackage{xfrac}
% \usepackage[flushleft]{threeparttable}
% \usepackage[top=30pt,bottom=50pt,left=60pt,right=60pt]{geometry}
\usepackage{array}
% \usepackage{lineno}
% \linenumbers
% \renewcommand\linenumberfont{\normalfont\small}
% \setlength\linenumbersep{1cm}
\usepackage{notes2bib}
\graphicspath{ {Figures/} }
% \newcommand*\mycommand[1]{\texttt{\emph{#1}}}



%%%%%%%%%%%%%%%%%%%%%%%%%%%%%%%%%%%%%%%%%%%%%%%%%%%%%%%%%%%%%%%%%%%%%%
%%                           Title    							    %%
%%%%%%%%%%%%%%%%%%%%%%%%%%%%%%%%%%%%%%%%%%%%%%%%%%%%%%%%%%%%%%%%%%%%%%

\title{Purification of single photons by temporal heralding of quantum dot sources}

\author{Hamza Abudayyeh}%
\author{Boaz Lubotzky}
 \affiliation{%
Racah Institute for Physics, the Applied Physics Department and the Center for Nanoscience and Nanotechnology, The Hebrew University of Jerusalem, Jerusalem 9190401, Israel}
\author{Somak Majumder}
\author{Jennifer A. Hollingsworth}
\affiliation{Materials Physics \& Applications Division: Center for Integrated Nanotechnologies, Los Alamos National Laboratory, Los Alamos, New Mexico 87545, United States}

\author{Ronen Rapaport}
 \email{ronenr@phys.huji.ac.il}
 \affiliation{%
Racah Institute for Physics and Center for Nanoscience and Nanotechnology, The Hebrew University of Jerusalem, Jerusalem 9190401, Israel}



\begin{document}

\maketitle
% ABSTRACT
\begin{abstract}
Efficient, high rate photon sources with high single photon purity are essential ingredients for quantum technologies. Single photon sources based on solid state emitters such as quantum dots are very advantageous for integrated photonic circuits, but they can suffer from a high two-photon emission probability, which in cases of non-cryogenic environment cannot be spectrally filtered.  
Here we propose two \textit{temporal} purification-by-heralding methods for using a two photon emission process to yield highly pure and efficient single photon emission, bypassing the inherent problem of spectrally overlapping bi-photon emission at elevated temperatures.
We experimentally emulate their feasibility on the emission from a single nanocrystal quantum dot at room temperature, exhibiting single photon purities exceeding 99.5\%, without a significant loss of single photon efficiency. 
By utilizing these methods with commercially available components, nanocrystal quantum dots  can be made realistic high-quality room-temperature sources for single photons. These approaches can also be applied for any indeterministic source of spectrally broadband photon pairs.



\end{abstract}


%%%%%%%%%%%%%%%%%%%%%%%%%%%%%%%%%%%%%%%%%%%%%%%%%%%%%%%%%%%%%%%%%%%%%%
%%                           MAIN TEXT    							%%
%%%%%%%%%%%%%%%%%%%%%%%%%%%%%%%%%%%%%%%%%%%%%%%%%%%%%%%%%%%%%%%%%%%%%%
%% INTRODUCTION
\section{Introduction:}
A single photon source (SPS) is proposed as a crucial resource for several major applications in quantum technologies \cite{Lounis2005,Eisaman2011InvitedDetectors,Aharonovich2016Solid-stateEmitters}.
The required performance of such a source depends on the application it would be used for \cite{Aharonovich2016Solid-stateEmitters}, however generally most applications require stable, compact, high-rate and high-single-photon-purity SPSs. 
%A few types of other light sources, which do not strictly emit single photons were developed and are being used frequently. These include faint lasers or 
The most widely used resource for single photons comes from low efficiency down-converted photon-pairs from non-linear processes in solids \cite{PaulG.Kwiat1995NewPairs,Yan2015GenerationWaveguide,Ngah2015Ultra-fastTechnology,Joshi2018FrequencySources}. 
Recently however, research has been focused on developing more efficient solid state SPSs such as semiconductor self assembled quantum dots (QDs) \cite{Ding2016,Somaschi2016NearState,Schlehahn2016,Daveau2017EfficientWaveguide,Liu2017HighPhotons}, chemically synthesized semiconductor nanocrystal quantum dots (NQDs) \cite{Lounis2000PhotonFluorescence,Michler2000QuantumTemperature,Chen2008GiantBlinking,Bidault2016PicosecondTemperature,Hoang2016UltrafastNanocavities,Matsuzaki2017StrongAntenna},  color centers in crystals \cite{Aharonovich2014DiamondNanophotonics,Schroder2016QuantumInvited,Castelletto2014ASource}, carbon nanotubes \cite{He2017TunableNanotubes,Ishii2017Room-TemperatureNanotubes}, and defect states in two-dimensional materials \cite{Chakraborty2015Voltage-controlledSemiconductor,Grosso2017TunableNitride} among others. 

One crucial requirement from an SPS is a very high single photon purity, i.e., a high probability that an emission event contains a single photon. 
Unfortunately most solid-state SPSs, especially room-temperature sources, do not inherently have sufficiently high purities. 
QD-based SPSs which operate at cryogenic temperatures, for example, have resolvable multi-exciton spectral lines.
Therefore multi-exciton emission can be spectrally filtered leaving only single photon emission from the single exciton state (X). 
However, the necessity of a  cryogenic work environment is limiting technologies, and semiconductor QD sources working at elevated temperatures have only been achieved with either wide bandgap QDs \cite{Arita2017UltracleanDot}, or using the smaller NQDs \cite{Lounis2000PhotonFluorescence,Michler2000QuantumTemperature}.  
Yet, operating at elevated temperatures introduces a significant limitation in maintaining both high purity together with a high source efficiency due to the spectral overlap between the single exciton (X) and bi-exciton (BX) emission.
The severity of BX emission increases significantly when the Auger recombination processes \cite{Efros2016OriginDots} are suppressed to increase QD stability or when NQDs are radiatively enhanced to achieve higher rates \cite{Park2013Single-NanocrystalTemperature,Wang2015CorrelatedPathways,Dey2016PlasmonicDots}. 
This detrimental problem of reduced purity is rather inherent for such elevated temperature SPSs, and no solution to increase purity without a significant sacrifice in terms of brightness has been successfully proposed or implemented to the best of our knowledge. 

Here we propose and demonstrate two \textit{temporal} purification-by-heralding methods for using a two photon emission process to yield highly pure and efficient single photon emission, thus bypassing the inherent problem of spectrally overlapping bi-photon emission.
We then experimentally emulate these temporal purification methods on the emission from a single NQD that is particularly prone to efficient BX emission \cite{Mangum2014InfluenceNanocrystals}, and show that employing such techniques enable reaching single photon purities exceeding 99.5\%, without a significant loss of single photon efficiency.
%These methods can be applied for any indeterministic source of photon pairs.

%% THEORETICAL FRAMEWORK 
\section{Purification by Temporal Heralding:}
Chemically synthesized NQDs (see Fig. \ref{fig:QDandTGF}a) are an excellent exemplary system to demonstrate and apply the temporal photon purification and heralding concepts.
NQDs are very promising for integrated and efficient SPSs at room temperature \cite{Brokmann2004ColloidalSources,Livneh2015EfficientNanoantenna,Livneh2016,Lin2017Electrically-drivenTemperature,Chandrasekaran2017NearlyDots}.
Unfortunately, it turns out that the photostability (in terms of photo-blinking) of an NQD can be intimately related to its BX quantum yield (QY$_{BX}$) defined as the probability that the BX would radiatively decay to the X state.
A key mechanism responsible for photo-blinking involves Auger-mediated non-radiative recombination of charged excitons \cite{Efros2016OriginDots}.
Thus, as the stability of the source against photo-blinking increases,  QY$_{BX}$ also usually increases due to the suppression of Auger recombination processes \cite{Efros2016OriginDots}, leading to a dramatic drop of both the source efficiency ($\eta=P_1$) and purity ($S=\sfrac{P_1}{(P_1+P_2)}$), where $P_1$ and $P_2$ are the probability of emitting one and two photons respectively (see Fig. \ref{fig:QDandTGF}b). 
Higher order multi-exciton emission is usually negligible in these quantum dots due their strong excitation power dependence and due to the cubic (or nearly cubic) scaling of the Auger recombination rate with exciton multiplicity whereas the radiative decay rate only has a quadratic scaling \cite{Htoon2010HighlyNanocrystals}. 
Furthermore the emission of these multi-excitons can be spectrally separated from the exciton line at room temperature \cite{Htoon2010HighlyNanocrystals}.
Thus NQD typically emits at most two photons per excitation cycle from the BX and X state with probabilities given by the BX and X quantum yields (QY$_{BX}$ and QY$_X$ respectively). 
Notably a similar increase in QY$_{BX}$ has been reported when NQDs were coupled to plasmonic emission rate enhancers \cite{Park2013Single-NanocrystalTemperature,Wang2015CorrelatedPathways,Dey2016PlasmonicDots}, showing that trying to increase the inherent emission rate of an NQD via the Purcell effect  has a detrimental effect on the purity of the source. 
Therefore it seems necessary to develop some photon purification methods that will enable utilizing NQDs as efficient high-quality room temperature SPSs. 
   

\begin{figure}[ht!]
\centering{\includegraphics{standalone_QD}}
\caption{(a) Illustration of a spherical nanocrystal quantum dot displaying the radiative recombination pathways available for the X and BX state. (b) Single photon purity (solid line) and efficiency (dashed line) for a quantum dot as a function of the BX quantum yield.  }
\label{fig:QDandTGF}
\end{figure}




Since neither spectral nor polarization based filtering of the BX emission is possible, a different approach is essential. 
One such approach, referred to here as time-gated filtering (TGF), is available in the time domain by using the difference in the lifetimes between the X and BX  to temporally filter out (and discard!) the BX emission in a time-resolved measurement (see Fig. \ref{fig:QDandTGF}a) \cite{Mangum2013DisentanglingExperiments.,Feng2017PurificationDots}.
In this approach however, due to the overlap between the BX and X decays, a very unfavorable  trade-off between the achievable purity and efficiency appears for high BX and X QYs \bibnote[supp]{see supplementary information for further details.}.
For example, in the theoretical limit when $QY_X=QY_{BX}=1$ obtaining a purity of 99.5\% would entail operating at an efficiency of about 25\% even if the collection efficiency of the system, $\alpha$, is unity  \cite{supp}. For currently available high-quality high-rate NQD emitters having $QY_X=0.61,QY_{BX}=0.7$ \cite{Matsuzaki2017StrongAntenna}, the efficiency would only be 8\% for $\alpha=1$.    
Therefore such an approach is not so useful for a practical high-quality SPS which requires both $\eta$ and $S$ to approach unity. 

An alternative approach would be to use the two photon events in a heralded scheme to produce highly pure single photon states.
The simplest possible heralding approach, referred to here as the beam splitter heralded scheme, would be to use a 50:50 beam splitter that would randomly direct the emission either to an idler port or a signal port. 
In a post processing step only signal photons heralded by an idler photon would be chosen, assuring $S=1$ in the signal channel.
This technique however has a maximum theoretical efficiency of $\eta=0.5$ in the ideal case where $QY_{X}=QY_{BX}=1$ and $\alpha=1$, due to the random routing of the beam splitter. 

\begin{figure*}[t]
\centering{\includegraphics[width=\textwidth]{theoretical_calculations}}
\caption{Schematic representations of the TIMe-resolved heraldED (TIMED), and (b) Active Switching Heralded (ASH) techniques proposed in this work. (c) Theoretical calculations of the heralding efficiency as a function of the cutoff time $T_C$ (for TIMED) or the response time $T_R$ (for ASH) normalized by the collection efficiency and the exciton quantum yield. Solid lines correspond to unity BX quantum yield whereas dashed lines correspond to $QY_{BX}=0.5$.  (d) Theoretical normalized heralding efficiency  using optimal parameters as a function of the BX quantum yield assuming unity X quantum yield. }
\label{fig:theoretical}
\end{figure*}

In order to overcome all these limitations we propose two main purification heralded schemes that are illustrated in Fig. \ref{fig:theoretical}a and b. 
\subsection{Passive Heralding:}
The first scheme, called the TIMe-resolved heraldED (TIMED) technique, utilizes the large difference between the BX and X spontaneous radiative lifetimes, and their cascaded emission nature to implement a time gate on an optical switch (see Fig. \ref{fig:theoretical}a).
In this scheme, an optical switch with a fixed time gate will replace the 50:50 beam splitter. 
The switch will be open to the idler port from the beginning of the excitation cycle  up to a fixed cutoff time ($T_C$). 
For $t>T_C$ the switch will direct the photons to the signal port until the the next excitation cycle.\bibnote{In this analysis we assume that the NQD is pumped well above saturation with a short pulse so as to prepare the NQD in the BX state at the beginning of each excitation cycle.}
Here again, only signal photons that were preceded by an idler photon detection would be chosen in a post-processing step.  The single photon efficiency in this scheme is given by \cite{supp}:
\begin{equation}
\begin{split}
\eta_{TIMED}=\alpha^2 & QY_X QY_{BX}\frac{\tau_X}{\tau_X-\tau_{BX}}\times \\ \times&\left[ \exp\left(\sfrac{-T_C}{\tau_X}\right)-\exp\left(\sfrac{-T_C}{\tau_{BX}}\right)\right],
\label{eq: etaTIMED}
\end{split}
\end{equation}
where $\tau_X$ and $\tau_{BX}$ are the X and BX lifetimes respectively.
As seen in Fig. \ref{fig:theoretical}c there is an optimal gate time $T_C^{opt}$  given by:
\begin{equation*}
\frac{T_C^{opt}}{\tau_X}= \left(\frac{\tau_X}{\tau_{BX}}-1\right)^{-1} \ln\left(\frac{\tau_X}{\tau_{BX}}\right). 
\label{eq: Tcopt}
\end{equation*}
The X and BX quantum yields are related to their radiative lifetimes by:
\begin{equation}
\frac{QY_{BX}}{QY_X}=\beta\frac{\tau_{BX}}{\tau_{X}},
\label{eq: QYratio}
\end{equation}
where $\beta$ is a scaling factor which in this case is determined merely by the simple statistical scaling of the optical recombination pathways as a function of the number of excitons \cite{Park2014AugerDots,Matsuzaki2017StrongAntenna}, i.e. $\beta$ = 4. This is shown schematically in Fig. \ref{fig:QDandTGF}a for the X and BX states.
As compared to a standalone NQD (Fig. \ref{fig:QDandTGF}b) the efficiency of TIMED increases with QY$_{BX}$ (Fig. \ref{fig:theoretical}d) illustrating the viability of this technique for high two-photon emission probabilities. 
Furthermore TIMED is a passive scheme requiring no active feedback from the idler port to the switch, which is advantageous due to the simple architecture of the optical circuit. 
Using the above value of $\beta$ and the optimized $T_C^{opt}$ yields a pure source with $S=1$ with a maximum theoretical efficiency $\eta\simeq 0.63$ (again taking $QY_{X}=QY_{BX}=1$, see Fig. \ref{fig:theoretical}c), already better than the simple beam-splitter technique (which can only reach $\eta=0.5$ even in an ideal system).
For currently realized QY values ($QY_X=0.61,QY_{BX}=0.7$) \cite{Matsuzaki2017StrongAntenna} we expect a theoretical efficiency of 26\%, compared to 20\% for the beam-splitter technique. 
This is a significant improvement, yet even better results can be achieved , with an active heralding technique, as is shown next.

\subsection{Active Heralding:}
This maximum efficiency of TIMED is a limitation of passive heralding techniques. To overcome this limitation in $\eta$ without sacrificing purity a more sophisticated solution, the Active Switching Heralded (ASH) technique is suggested (see Fig.\ref{fig:theoretical}b).
In this active switching scheme the switch is open to the idler port from the beginning of the excitation cycle until a photon detection has occurred. 
This detection signal would be forwarded to the switch causing it to route the photons to the signal port until the end of the cycle. 
Realistically, there is a finite time between the arrival of the idler photon to the detector and the switching event which we call the response time ($T_R$) of the system. 
This response time takes into account the rise times of the detector and switch and any optical or electronic delays in the system. 
The heralding efficiency as a function of the response time can be expressed as\cite{supp}:
\begin{equation}
\eta_{ASH}= \alpha^2 QY_X QY_{BX} \exp\left(\sfrac{-T_R}{\tau_X}\right)
\label{eq: etaASH}.
\end{equation}
Notably, it can be seen (Fig.\ref{fig:theoretical}c) that in the ASH scheme, the maximal $\eta$ for the theoretically ideal case $QY_{X}=QY_{BX}=1$ exponentially approaches 100\% (or 43\% for the case $QY_X=0.61,QY_{BX}=0.7$ \cite{Matsuzaki2017StrongAntenna}), as $T_R$ becomes smaller than $\tau_X$. Therefore implementing ASH with very short response gates should approach the ideal SPS scenario.




%% EXPERIMENYAL RESULTS
\section{Experimental Emulation:}
\begin{figure}[t]
\centering{\includegraphics[width=\textwidth]{experimental1}}
\caption{(a) Illustration of the experimental setup used to emulate the TIMED and ASH schemes. (b) Lifetime measurement and (c) power saturation curve for the QD studied in this work.(d) Experimental results (dotted lines) for the efficiency (in units of $\alpha^2QY_X$ in the TIMED and ASH schemes compared to the predicted values (solid lines). (e) The single photon purity as a function of the filtering time which is used to filter out short-time correlated noise illustrated in the inset. }
\label{fig:exp1}
\end{figure}
Rather than physically implementing the setups for the TIMED and ASH protocols, which require building a dedicated setup for each, we emulate the switching on the time-tagged data obtained from unified experimental setup shown in Fig. \ref{fig:exp1}a  
(for details regarding the experimental technique and analysis methods refer to the supplementary material). 
% A core/thick-shell CdSe/CdS NQD \cite{Park2011Near-UnitySpectroscopy} was placed under a 0.9 NA objective and pumped using a 2MHz 405 nm pulsed laser. The NQD emission is directed through two 50/50 beam splitters with three single photon avalanche detectors (Excelitas SPCM-AQRH-14-FC) at their outputs. 
% The counts from the detectors are recorded using a time correlated single photon counting module (Swabian TimeTagger 2.0) operated in time-tagged mode.
% The three detectors are used to measure the purity of the heralded schemes, i.e. once one detector triggers an event we conduct a coincidence measurement on the other two detectors \cite{Note1}.
In such a way we can compare the performance of multiple schemes for the same NQD. 
Fig. \ref{fig:exp1}b displays the lifetime measurement for the NQD under investigation, which was fitted to a three exponential fit yielding the X and BX lifetimes, and a significant short lifetime component which we attribute to a correlated laser-induced noise (hereinafter referred to as correlated noise). 
Fig. \ref{fig:exp1}c displays the saturation curve for the same NQD indicating that at the maximum power used in this experiment we are pumping well above saturation. This is done to ensure a maximal occupation of the BX states.
Using the measured lifetimes and Eq. \ref{eq: QYratio} we find that the ratio between the quantum yields is: $QY_{BX}/QY_X$=0.269. 
With this ratio, the one photon probability, and our calibrated system collection efficiency ($\alpha = 0.088$) we find the value of each of the quantum yields separately to be: $QY_X=0.1729$ and $QY_{BX}=0.0465$ \cite{supp}. 
Prior to any purification protocol we measured the photon purity to be only $S\approx0.9$ for this NQD \cite{supp}.
 
Fig. \ref{fig:exp1}d displays the experimental efficiency for both TIMED and ASH compared to the theoretically expected values (Eqs. \ref{eq: etaTIMED} and \ref{eq: etaASH}) as a function of $T_C$ or $T_R$ respectively. 
% The first 300 ps of data after the laser pulse was ignored to minimize the effects of laser-induced correlated noise and this was also taken into account in calculating the theoretical curves. 
We find a good agreement between the experimental results and the expected values for both schemes,
and the small shift in the position of maximum of the experimental $\eta_{TIMED}$ to a lower $T_C$ is attributed to the effect of the correlated noise. 
We find experimentally that the maximal efficiency for the ASH scheme is $\eta_{ASH}=0.0468 \ \alpha^2 QY_X$, which is in agreement with the experimentally extracted value $QY_{BX} = 0.0465$. Comparing this to Eq. \ref{eq: etaASH} with $T_R=0$ indicates that indeed this efficiency is only limited by the quantum yields of the X and BX, as is theoretically predicted. 

Going to the source purity, theoretically, if the NQD cannot emit more than two-photons per excitation pulse, we expect a unity purity, i.e. $S=1$. Experimentally we expect that in this case the sole source reducing $S$ in the signal port is from noise. 
To check this point we show the purity, $S$, of both schemes as a function of the filtering time $T_F$ after the excitation pulse, 
where all data with $t<T_F$ is ignored. This is done to minimize the contribution of short lifetime laser-induced correlated noise.
For $T_F>1$ns, $S$ reaches a maximum of around 0.995, after filtering out nearly all correlated noise.
We attribute this limit to uncorrelated noise rather to inherent multi-photon ($>2$) emission events from the NQD \cite{supp}.

%% DISCUSSION
\section{Discussion and Outlook:}
\begin{figure*}
\centering{\includegraphics[width=0.5\textwidth]{comparasion}}
\caption{ (a) Difference between the efficiency of the ASH and TGF scheme displaying the regions where each technique is superior. The overlaying points are measured values of the QY$_X$ and QY$_{BX}$ for: (1) Standalone CdSe/CdZnS NQD \cite{Nair2011BiexcitonStatistics} , (2) CdSe/ 16 monolayer CdS gQD coupled to plasmonic nanocone \cite{Matsuzaki2017StrongAntenna}, and (3) CdSe/ 19 monolayer CdS standalone gQD \cite{Park2011Near-UnitySpectroscopy}. (b) The single photon rate of ASH, TIMED and TGF as a function of the exciton lifetime for $QY_X$=0.61, $QY_{BX}$=0.7, $S$=0.995, $\alpha$=0.72 and assuming an excitation rate equal to 3/$\tau_X$. }
\label{fig:comparasion}
\end{figure*}
%\begin{table}
% \begin{threeparttable}
%\caption{Comparison between a projected NQD-on-antenna system operated with TIMED and ASH photon purification protocols and other state-of-the-art SPS sources}
%\label{tab: comparasion}
%\begin{tabular}{|p{ 4.6 cm}|m{ 1.2 cm}|m{ 1.2 cm}|}
%\cline{2-3} 
%\multicolumn{1}{c|}{} & $\eta$ & S\tabularnewline
%\hline 
%NQD on Nanoantenna+TIMED, room Temperature $\left(T_{C}^{opt}\right)$&  0.15 & $>0.995$
%\tabularnewline
%\hline 
%NQD on Nanoantenna+ASH, room Temperature $\left(T_{R}=0.05 \ \tau_X \right)$ & 0.21 & $>0.995$\tabularnewline
%\hline 
%NQD on Nanoantenna+TGF, room Temperature  & 0.067 & $>0.995$\tabularnewline
%\hline 
%Parametric downconversion - nonlinear crystal\cite{Ngah2015Ultra-fastTechnology} &  $2.1\times10^{-4}$ & 0.989\tabularnewline
%\hline 
%Self-assembled QD in a microcavity, cryogenic temperature\cite{Ding2016}& 0.33 & 0.991\tabularnewline
%\hline 
%\end{tabular}
% \begin{tablenotes}
% \small
% \item $^a$ Projected values based on parameters defined in text.
% \item $^b$ Operated at 4.5-10 K.
% \item $^c$ Objective coll. eff. (NA=0.68) and polarizer (50\%) to suppress laser scattering.
% \end{tablenotes}
% \end{threeparttable}
%\end{table}
As can be seen from the model and the emulation experiments, the heralded purification of single photons from BX emission in NQDs can significantly increase their single photon purity without compromising their efficiencies, which can make such NQD sources (or other sources that have high quantum yield for two-photon emission), good candidates for high-quality room-temperature SPSs.
To demonstrate the significant superiority of the heralded techniques over TGF we plot in Fig. \ref{fig:comparasion}a the difference in the efficiency between the ASH and TGF techniques for a specific chosen purity (S=0.995). 
This displays that it is not necessary to have very high QYs for the ASH technique to outperform the TGF technique. 
In fact, currently available NQD systems already are in the regime where ASH is better. 
Importantly, as the quality of these QDs improve, and as larger and larger Purcell factors are achieved to increase their single photon brightness, it is expected that the values of the $QY$s would increase making these techniques even more advantageous. 

In order to assert the applicability of our techniques and to make a quantitative comparison with other state-of-the-art devices we calculate the expected performance of our techniques based on a model system that was previously introduced, namely an NQD coupled to the near field of a plasmonic nanocone. 
In \citet{Matsuzaki2017StrongAntenna} it has been shown experimentally that this coupling induces a very large Purcell effect, leading to a significant reduction in the radiative lifetimes of both the X and the BX, and in a very large increase of $QY_{BX}$ yielding: $QY_X=0.61$, $QY_{BX}=0.7$, $\tau_X=1.6$ ns, and $\tau_{BX}=0.5$ ns. This makes the coupled NQD a rather good bi-photon emitter.  
In addition we assume a realistic photon collection efficiency of 0.8 \cite{Abudayyeh2017QuantumSources} and a detection efficiency of 90\% \cite{Ma2015SimulationDiode,Zang2017SiliconTrapping}(i.e. $\alpha = 0.72$).

One particularly important point to consider is that for ASH the response time should be fast compared to the X lifetime. 
In the supplementary information \cite{supp} we show that an on-chip implementation of ASH could realistically have response times of about 265 ps. 
In the model system suggested in the previous paragraph this would correspond to $T_R/\tau_X \approx 0.17$ and $\eta_{ASH}=0.19$ compared to $\eta_{TIMED}=0.15$ and $\eta_{TGF}=0.089$.
Under such efficiencies the single photon rate can be estimated to be $R_{ASH}= 38$ MHz, $R_{TIMED}=30$ MHz, and $R_{TGF}=18$ MHz under pulsed excitation at a rate of 200 MHz ($1/3\tau_X$).
This is another confirmation that under realistic conditions both heralded techniques can outperform TGF.
Nonetheless it is clear that $T_R$ imposes a limitation on the maximum X decay rate that can be achieved before TIMED and TGF outperform ASH.
We analyze this situation in Fig.\ref{fig:comparasion}b where we vary $\tau_X$ while holding all other parameters constant as in the suggested model system above.
As expected at low $\tau_X$ ASH suffers however TIMED (TGF) only has better performance for $\tau_X <$0.6 (0.3) ns. 
These values of single photon purity and rate also compare well with state-of-the-art parametric heralded sources \cite{Ngah2015Ultra-fastTechnology,Wang2016ExperimentalEntanglement,Joshi2018FrequencySources}, and cryogenic-operated self-assembled QD \cite{Ding2016,Somaschi2016NearState} sources.
We note that in contrast to cryogenic QD sources, photons emitted from room-temperature sources are distinguishable in nature. However several recent reports have addressed this issue by proposing cavity enhanced techniques to increase the indistinguishablity of room-temperature emitters \cite{Grange2015,Wein2018FeasibilityCavities,Choi2018HighlyRegime}. 

One interesting application, even under limited indistinguishability, is quantum key distribution (QKD) for which true room-temperature SPSs are desired \cite{Takemoto2015QuantumDetectors}.
%We show in the supplementary information \cite{supp} that using the parameters in table \ref{tab: comparasion} the projected NQD based SPSs will be able to compete with an ideal decoy state protocol \cite{Hwang2003QuantumCommunication,Lo2005DecoyDistribution}.
In the quantum metrology realm one may devise using these highly intensity-squeezed sources to conduct weak absorption measurements on highly sensitive samples or for calibrating photodetectors with high precision well beyond the shot-noise limit \cite{Lounis2005}. 
Eventually such sources may become the ultimate standard for intensity measurements in what is known as the quantum candela \cite{Cheung2007TheRadiation}.
Furthermore the heralding nature of the schemes proposed in this work open up the possibility of multiplexing a few NQDs together to overcome any inefficiencies that may be caused due to inadequate quantum yields or collection efficiencies, thus increasing the attainable single photon rates \cite{supp}.

In summary, we show both theoretically and by experimental emulation the ability to break the trade-off between stability, efficiency and single photon purity in statistical two-photon emitters by temporal heralding schemes, allowing to achieve near-unity efficiency and single photon purity simultaneously.

\begin{acknowledgement}
This work was performed, in parts under the financial support from: The Einstein Foundation Berlin; The U.S. Department of Energy: Office of Basic Energy Sciences - Division of Materials Sciences and Engineering; User Project 2017BU0062 at the Center for Integrated Nanotechnologies, an U.S. Department of Energy (DOE) Office of Science User Facility operated for the DOE by Los Alamos National Laboratory (LANL) (Contract DE-AC52-06NA25396) and Sandia National Laboratories (Contract DE-NA-0003525). J.H. and S.M. acknowledge partial support from the LANL Laboratory Directed Research and Development funds.

\end{acknowledgement}

%%%%%%%%%%%%%%%%%%%%%%%%%%%%%%%%%%%%%%%%%%%%%%%%%%%%%%%%%%%%%%%%%%%%%
%% The same is true for Supporting Information, which should use the
%% suppinfo environment.
%%%%%%%%%%%%%%%%%%%%%%%%%%%%%%%%%%%%%%%%%%%%%%%%%%%%%%%%%%%%%%%%%%%%%
\begin{suppinfo}

% % A listing of the contents of each file supplied as Supporting Information
% % should be included. For instructions on what should be included in the
% % Supporting Information as well as how to prepare this material for
% % publications, refer to the journal's Instructions for Authors.

The supplementary information contains the theoretical derivations for the equations shown in the text in addition to experimental and analysis details.% and an assessment of the utility of the sources proposed in this work to QKD applications. 
% % \begin{itemize}
% %   \item Filename: brief description
% %   \item Filename: brief description
% % \end{itemize}

\end{suppinfo}
% \nolinenumbers
%%%%%%%%%%%%%%%%%%%%%%%%%%%%%%%%%%%%%%%%%%%%%%%%%%%%%%%%%%%%%%%%%%%%%%
%%                           REFERENCES    							%%
%%%%%%%%%%%%%%%%%%%%%%%%%%%%%%%%%%%%%%%%%%%%%%%%%%%%%%%%%%%%%%%%%%%%%%
\bibliography{Mendeley}

\end{document}